% This is samplepaper.tex, a sample chapter demonstrating the
% LLNCS macro package for Springer Computer Science proceedings;
% Version 2.21 of 2022/01/12
%-----------------LNCS version---------------
 \documentclass[runningheads]{llncs}  %% Change it to TCS
% %
%  \usepackage[T1]{fontenc}

%  \documentclass[11pt,a4paper]{article}
  \usepackage[margin=1in]{geometry}
% T1 fonts will be used to generate the final print and online PDFs,
% so please use T1 fonts in your manuscript whenever possible.
% Other font encondings may result in incorrect characters.
%
\usepackage{blindtext}
\usepackage{hyperref}
\usepackage{graphicx}
\usepackage{amsfonts} 
\usepackage{pifont}
% Used for displaying a sample figure. If possible, figure files should
% be included in EPS format.
%
% If you use the hyperref package, please uncomment the following two lines
% to display URLs in blue roman font according to Springer's eBook style:
%\usepackage{color}
%\renewcommand\UrlFont{\color{blue}\rmfamily}
%\def\A{{\cal A}}
\newcommand{\IR}{\mathbb{R}}
\def\s{{\mathcal S}}
\def\OO{{\cal O}} % \O is a capital oh with a slash
\def\Q{{\cal Q}}
\def\A{{\cal A}}
\def\H{{\mathcal H}}
% \newcommand{\mathbb{R}}{\mathbb{R}}
\def\I{{\mathcal I}}
\def\K{{\mathcal K}}
\def\P1{{\mathcal P_1}}
\def\P2{{\mathcal P_2}}
\def\P{{\mathcal P}}
\def\B{{\mathcal B}}
\def\C{{\mathcal C}}
%\def\Z{{\mathbb{Z}}
\def\QH{{\mathcal Q(\sigma)}}
\def\QB{{\mathcal Q(B)}}
 \usepackage{mathtools}
 \usepackage{pdfpages}
\usepackage{caption}
\usepackage{subcaption}
\DeclarePairedDelimiter\abs{\lvert}{\rvert}%
\DeclarePairedDelimiter\norm{\lVert}{\rVert}%


% \newtheorem{property}{Property}
 \newtheorem{clm}{Claim}
 % \newtheorem{problem}{Problem}
% \newtheorem{remark}{Remark}
% \newtheorem{definition}{Definition}
% \newtheorem{theorem}{Theorem}
% \newtheorem{lemma}{Lemma}
% \newtheorem{problem}{Problem}
% \newtheorem{corollary}{Corollary}
 \newtheorem{observation}{Observation}
% \newtheorem{question}{Question}
\DeclareMathOperator{\interior}{int}
\DeclareMathOperator{\opt}{OPT}
\DeclareMathOperator{\alg}{ALG}
\DeclareMathOperator*{\argmin}{argmin}
\DeclareMathOperator{\GS}{Game-Of-Same-Side}
\DeclareMathOperator{\BPA}{\textsf{BestPoint-Algorithm}}
\newcommand{\cmark}{(\ding{51})}
\newcommand{\xmark}{(\ding{55})}%
\DeclareMathOperator{\As}{Aspect}
\newcommand{\Sep}{\ensuremath\mathcal{X}}
% \newcommand{\mathbb{R}}{\mathbb{R}}
\newcommand{\mb}[1]{{#1}}
%\renewcommand{\todo}{}
\usepackage{color}
\newcommand{\red}{\textcolor{red}}
\newcommand{\blue}{\textcolor{blue}}
\newcommand{\green}{\textcolor{Green}}
\newcommand{\teal}{\textcolor{Teal}}
\newcommand{\magenta}{\textcolor{magenta}}
\newcommand{\cyan}{\textcolor{cyan}}
\newcommand{\navy}{\textcolor{Navy}}
\newcommand{\gray}{\textcolor{Gray}}
\newcommand{\darkmagenta}{\textcolor{DarkMagenta}}
\newcommand{\mediumpurple}{\textcolor{MediumPurple}}
\newcommand{\gold}{\textcolor{Gold}}
\newcommand{\darkorange}{\textcolor{DarkOrange}}
\newcommand{\myqed}{\hfill $\Box$}
\begin{document}
%
% \title{Hitting Geometric Objects Online  via  Points in~$\mathbb{Z}^d$\thanks{
%       Work on this paper by the first author has been partially supported 
%       by  SERB MATRICS Grant MTR/2021/000584, and 
%       work by the second author has been supported 
%       by CSIR (File Number-09/086(1429)/2019-EMR\_I).
%    }}


% \title{Online hitting of $d$-dimensional unit balls and hypercubes using integer points}
% \title{Online hitting of $d$-dimensional unit balls and hypercubes}
% \title{Online Geometric hitting set and set cover for $d$-dimensional unit balls and hypercubes}
\title{Online Hitting of Unit Balls and  Hypercubes in $\IR^d$ using  Points from~$\mathbb{Z}^d$\thanks{Preliminary version of this paper appeared in the 28th international computing and combinatorics conference (COCOON), 2022~\cite{DeS22}.
      Work on this paper by the first author has been partially supported 
      by  SERB MATRICS Grant MTR/2021/000584, and 
      work by the second author has been supported 
      by CSIR, India (File Number-09/086(1429)/2019-EMR\_I).
   }}
%
%\titlerunning{Abbreviated paper title}
% If the paper title is too long for the running head, you can set
% an abbreviated paper title here
%

 
\author{Minati De\inst{1}\orcidID{0000-0002-1859-8800} \and
Satyam Singh\inst{1}}
%
\authorrunning{M. De and S. Singh}
% First names are abbreviated in the running head.
% If there are more than two authors, 'et al.' is used.
%
\institute{Dept. of Mathematics, \\ 
Indian Institute of Technology Delhi,\\
New Delhi,
India\\
\email{\{minati,satyam.singh\}@maths.iitd.ac.in}}
%\url{http://web.iitd.ac.in/~minati}}
%
\maketitle              % typeset the header of the contribution
%



% \\\\\\\\\\\---------------Abstract old: Start------------\\\\\\\\\\
% \begin{abstract}
% In this paper, we consider the online version of the minimum hitting set problem where geometric objects arrive one by one.  The online algorithm must maintain a hitting set for the arrived objects by making irrevocable decisions. Here the centers of objects and hitting point are in $\mathbb{R}^d$ and $\mathbb{Z}^d$, respectively. First, we show that for hitting unit intervals using points from $\mathbb{Z}$, we achieve a tight bound of $2$. After that, we consider the case of $d$-dimensional unit hypercubes.  For hitting unit hypercubes in $\IR^d$, we propose a deterministic online algorithm with a competitive ratio at most~$4$ when $d=2$. For $d>2$, we propose a  randomized algorithm having a competitive ratio of $O(d^2)$. Then we prove that every deterministic online algorithm for hitting hypercubes using integer points has a competitive ratio of at least~$d+1$ when $d\in\mathbb{N}$. Finally, we consider studying unit balls in $\IR^d$. For hitting unit balls in $\IR^d$ using integer points, at first, we propose a deterministic algorithm having a competitive ratio of at most $O(d^4)$ and $O(d^3)$, for $d\geq4$ and $d=3$, respectively. Then, for unit disk in $\IR^2$, we propose a simple  deterministic algorithm that achieves a competitive ratio of~$4$. At last, we show that every deterministic online algorithm for hitting balls using integer points has a competitive ratio of at least~$d+1$ when $d<4$.



% \keywords{Hitting set \and Online algorithm \and Competitive ratio \and Unit covering \and Geometric objects.}
% \end{abstract}
%
%
%--------------------------Abstract old- End=------------------

% \\\\\\\\\\\\\\\\\\\-----------Abstract for full version-----------\\\\\\\\\\\\\\\\

\begin{abstract}
We consider the online hitting set problem for the range space $\Sigma=(\cal X,\cal R)$, where the point set $\cal X$ is known beforehand, but the set $\cal R$ of geometric objects is not known in advance.
Here, geometric objects arrive one by one, the objective is to maintain a hitting set of minimum cardinality by taking irrevocable decisions. 
In this paper, we have considered the problem when the objects are unit balls or unit hypercubes in $\IR^d$, and the  points from $\mathbb{Z}^d$ are used  for hitting them.
First, we consider the  problem for objects (unit balls and unit hypercubes) in lower dimensions. We obtain $4$ and $8$-competitive deterministic online algorithms for hitting unit hypercubes in $\IR^2$ and $\IR^3$, respectively. On the other hand,  we present $4$ and $14$-competitive deterministic online algorithms for hitting unit balls in $\IR^2$ and $\IR^3$, respectively.
Next, we consider the problem for objects (unit balls and unit hypercubes) in the higher dimension. For hitting unit hypercubes in $\IR^d$, we present a $O(d^2)$-competitive randomized online algorithm for $d\geq 3$ and prove the competitive ratio of any deterministic algorithm for the problem is at least $d+1$ for any  $d\in\mathbb{N}$. Then, for hitting unit balls in $\IR^d$, we propose a $O(d^4)$-competitive deterministic algorithm, and  for $d<4$, we establish that the competitive ratio of any deterministic algorithm is at least $d+1$.



%  All the above-mentioned results also hold for an equivalent unit covering problem.  Here points in $\IR^d$ come one by one. Upon the arrival of an uncovered point, the online algorithm needs to take irrevocable decisions to cover the arrived point using a unit object centered at some point in $\mathbb{Z}^d$. The aim of the problem is to minimize the number of unit objects to cover all the presented points.

\keywords{Hitting set \and Online algorithm \and Competitive ratio \and Unit covering \and Geometric objects.}
\end{abstract}



\section{Introduction}
The hitting set problem and the set cover problem are one of the most fundamental problems in combinatorial optimization~\cite{Alon09,EvenS14,Feige98,amitkumar,HochbaumM85}. 
% They have several applications in wireless networks, VLSI design, resource allocation, databases etc.~\cite{}.
Let  $\Sigma=(\cal X,\cal R)$  be a \emph{range space} where $\cal X$ is a set of \emph{elements} and $\cal R$ is a family of subsets of $\cal X$ called \emph{ranges}. A subset $\cal H \subseteq \cal X$ is called a \emph{hitting set} of the range space $\Sigma$ if the set $\cal H$ intersects every range $r$ in $\cal R$ and a subset $\cal C \subseteq \cal R$ is called a \emph{set cover} of the range space $\Sigma$ if the union of ranges in $\cal C$ covers all elements of $\cal X$. The aim of  the hitting set (resp., set cover) problem is to find a hitting set $\cal H$ (resp., set cover $\cal C$)  of the minimum cardinality. 
It is well known that a set cover of $\Sigma =({\cal X}, {\cal R})$ is a hitting set of the dual range space $\Sigma^{\perp}=({\cal X}^{\perp}, {\cal R}^{\perp})$. Here, for each range $r\in{\cal R}$ there is an element in ${\cal X}^{\perp}$ and for each element $x\in{\cal X}$ there is a range $r_x$, namely, $r_x=\{r\in{\cal R}\ |\ x\in r\}$, in ${\cal R}^{\perp}$~\cite{AgarwalP20}.

% , where ${\cal X}^{\perp}$ is the set of elements and ${\cal R}^{\perp}$} is the family of subsets of ${\cal X}^{\perp}$.
 
 Due to numerous applications in wireless sensor networks, VLSI design, resource allocation and databases,   researchers have  considered the set $\cal X$ to be a collection of points from $\IR^d$ and $\cal R$ to be a finite family of geometric objects chosen from some infinite class (hypercubes, balls, etc.)~\cite{AgarwalP20,ChanH20,Ganjugunte11,MegiddoS84,MustafaR10}. In this case, ranges are ${\cal X} \cap r$ for any object $r \in\cal R$. With a slight misuse of the notation, we will use $\cal R$ to signify both the set of ranges as well as the set of objects that define these ranges. A \emph{geometric range space} $\Sigma=(\cal X, \cal R)$ consists of a point set $\cal X$ containing points and a family $\cal R$ of geometric objects. The \emph{geometric hitting set} problem is to find the minimum number of points from $\cal X$ to hit all the objects in $\cal R$. The \emph{geometric set cover} problem is to find the minimum number of objects in $\cal R$ that covers all the points in $\cal X$.

% The geometric set cover and the geometric hitting set are the special variant of the set cover and the hitting set, respectively, in the geometric setup. 

In this paper, we consider the geometric hitting set problem in an online setting, where ${\cal X}=\mathbb{Z}^d$ and the set $\cal R$ is a finite family of translates of a  unit object $\sigma^*$ in $\IR^d$. Note that the set $\cal R$ is not known in advance, and objects of $\cal R$ arrive one by one. 
An online algorithm needs to maintain a feasible hitting set $\cal H$ for the already-arrived  objects. Upon the arrival of a new  object $\sigma\in\mathcal{R}$, if $\sigma$ does not contain any point from the existing hitting set $\cal H$, the algorithm needs to add a point $p\in\cal X$ to $\cal H$ to hit $\sigma$.
The decision to add a point to the solution set is  irrevocable, i.e., the online algorithm can not remove any point from the existing hitting set in future. 
% The  aim of the problem is to obtain a hitting set of minimum cardinality.
For simplicity, we will use the term `` online hitting set problem" (resp., ``online covering problem") instead of ``geometric hitting set problem in the online setup'' (resp., ``geometric set cover problem in the online setup'').

For any instance $({\cal X},\cal R)$ of the online hitting set problem considered in this paper, one can transform it into an equivalent instance $({\cal X}^{\perp},{\cal R}^{\perp})$ of the online covering problem as follows. 
The points in the set $\cal X^{\perp}$ are obtained by considering the centers of the unit object in $\cal R$, and unit objects in ${\cal R}^{\perp}$ are obtained by placing unit balls (or unit hypercubes) centered at the points of $\cal X$.
One application of this  problem is as follows. Let us consider a planned city where one can install base stations at specific locations from a rectilinear grid. Each base station can cover clients who are within its covering radius. Clients can appear from any location in the city.
The objective is to minimize the number of base stations covering all clients.  Since installing a base station is expensive, the decision is considered to be irrevocable. The  need is to place a base station that serves new uncovered clients.




We use competitive analysis to analyze the quality of our online algorithm~\cite{BorodinE}. Let $\A$ be an online algorithm for a minimization problem. The algorithm $\cal A$ is said to be \emph{$c$-competitive}, if 
$c=\sup_{\beta}\frac{\A_{\beta}}{\OO_{\beta}}$,  where $\A_{\beta}$ and $\OO_{\beta}$ are  the costs of the solution produced by the online algorithm $\A$ and  an optimal offline algorithm, respectively, with respect to an input sequence $\beta$. If $\A$ is a  randomized algorithm, then $\A_{\beta}$ is replaced by the expectation ${\mathbb{E}[\A_{\beta}]}$, and the 
competitive ratio of  $\A$ is  $\sup_{\beta}\frac{\mathbb{E}[\A_{\beta}]}{\OO_{\beta}}$~\cite{BorodinE}.

% For a given  set $\mathcal{S}$ of geometric objects and  a set $\P$ of points, a subset $\cal H \subseteq{\P}$ is said to be a \emph{hitting set} for the set $\cal S$ if each object of $\cal S$ contains at least one point of $\cal H$. The objective of the \emph{hitting set problem} is to find a hitting set $\cal H$ of minimum cardinality. This problem has applications in wireless networks, VLSI design, resource allocation, databases etc. 

% % \subset \mathbb{R}^d$
% In this paper, we consider the hitting set problem in an online setup, where the point set $\P=\mathbb{Z}^d$  and the set $\mathcal{S}$ is a family of unit objects (unit balls or unit hypercubes). Note that the set $\cal S$ of objects is not known in advance. Objects of $\cal S$ arrive one by one. An online algorithm must maintain a feasible hitting set $\cal A$ for the already-arrived  objects. Upon the arrival of a new  object $\sigma\in\mathcal{S}$, if $\sigma$ does not contain any point from the existing hitting set $\cal A$, the algorithm needs to add a point $p\in\P$ to $\cal A$ to hit $\sigma$.
% The decision to add a point to the solution is  irrevocable, i.e., the online algorithm can not remove  any point in the future from the existing hitting set. The  aim of the problem is to obtain a hitting set of minimum cardinality. 

% For any instance $(\P,\cal S)$ of the online hitting set problem considered in this paper, one can transform it into an equivalent instance $(\P',{\cal S}')$ of the online unit covering problem as follows. 
% The points in the set $\P'$ are obtained by considering the centers of the objects in $\cal S$, and the objects in $\cal S'$ are obtained by placing unit balls (or unit hypercubes) centered at the points of $\P$.
% % If we consider the points of $\P$ of the hitting set as the centers of unit balls (or unit hypercube) for covering and centers of objects in $\mathcal{S}$ of the hitting set as the points of covering. Then, one can formulate an equivalent  \emph{online unit covering problem}.
% % Here, the points from $\mathbb{R}^d$ arrive one by one, and we need to cover each point with a unit ball (or unit hypercube) whose center belongs to $\mathbb{Z}^d$. The aim is to minimize the number of unit objects to cover all the presented points.
% % The above reduction shows that the online hitting set problem and online covering problem are equivalent for unit balls (or unit hypercubes).
% One application of this  problem is as follows. Let us consider a planned city where one can install base stations at specific locations from a rectilinear grid. Each base station can cover clients who are within its covering radius. Clients can appear from any location in the city.
% The objective is to minimize the number of base stations covering all clients.  Since installing a base station is expensive, the decision is considered to be irrevocable. The  need is to place a base station that serves new uncovered clients.




% We use competitive analysis to analyze the quality of our online algorithm~\cite{BorodinE}. Let $\A$ be an online algorithm for a minimization problem. The algorithm $\A$ is said to be \emph{$c$-competitive}, if 
% $c=\sup_{\beta}\frac{\A_{\beta}}{\OO_{\beta}}$,  where $\A_{\beta}$ and $\OO_{\beta}$ are  the costs of the solution produced by the online algorithm $\A$ and  an optimal offline algorithm, respectively, with respect to an input sequence $\beta$. If $\A$ is a  randomized algorithm, then $\A_{\beta}$ is replaced by the expectation ${\mathbb{E}[\A_{\beta}]}$, and the 
% competitive ratio of  $\A$ is  $\sup_{\beta}\frac{\mathbb{E}[\A_{\beta}]}{\OO_{\beta}}$~\cite{BorodinE}.


\subsection{Related Work}
% Let $\P$ be a set of $n$ elements, and let ${\cal S}$ be a family of $m$ subsets of $\P$. The hitting set problem is to find the smallest cardinality subset $\H\subseteq\P$ such that each set in $\cal S$ contains at least one element from $\H$. A related (dual) problem is the \emph{set cover} problem, where    the aim is to find the minimum-sized collection $\cal S^{*}\subseteq\cal S$ such that their union covers the whole set $\P$. For any instance $(\P,\cal S)$ of the set cover problem, one can transform it into an equivalent instance $(\P',{\cal S}')$ of the hitting set problem by duality~\cite{aronov}. 
The hitting set and set cover problems are classical NP-hard problems~\cite{Karp}.
In the offline setup, if the set $\cal X$ contains points on the real line and $\cal R$ consists of intervals in $\mathbb{R}$, the set cover problem can be solved in polynomial time using a greedy algorithm{~\cite{Trotter83}}. However, these problems remain  NP-hard, even when $\cal R$ consists of simple geometric objects like unit disks in $\mathbb{R}^2$~\cite{FOWLER1981133} and ${\cal X}$ is a set of points in ${\IR}^2$. 
%  Later, Feige~\cite{Feige98} provided a sharp logarithmic threshold below which the set cover problem cannot be approximated efficiently.
Alon et al.~\cite{Alon09} initiated the study of the set cover problem in the online setup. They considered the model where the sets $\cal X$ and $\cal R$ are already known, but the order of arrivals of points in $\cal X$ is unknown. Upon the arrival of an uncovered point in $\cal X$, the online algorithm must
choose a range $r\in \cal R$ that covers the point.
The algorithm presented by Alon et al.~\cite{Alon09} has a competitive ratio of $O(\log n \log m)$.
% \red{The results obtained in~\cite{Alon09} also hold for the online hitting set problem.} 
Later, Even and Smorodinsky~\cite{EvenS14} studied the online hitting set problem, where the set $\cal X$ and $\cal R$ are known in advance, but the order of arrival of the input objects in $\cal R$ is unknown. They proposed online algorithms having a competitive ratio of $O(\log n)$ when $\cal R$ consists of half-planes and unit disks in $\IR^2$. They gave matching lower bounds of the competitive ratio for these cases.  They also proposed an online algorithm that achieves an optimal bound of $\Theta(\log n)$ when $\cal R$ consists of intervals in the range $[1,n]$ and  $\cal X$ consists of all  integers in the range $[1,n]$. 
In this paper, we 
consider online hitting set problem where  ${\cal X}=\mathbb{Z}^d$ and objects in $\cal R$ consists of unit balls (and hypercubes) in $\IR^d$.
 We consider the model in which $\cal X$ is known in advance, but  objects in $\cal R$ are not known beforehand.

 A variant of the set cover problem  is known as the \emph{unit covering problem} where $\cal X$ is a set of points in $\IR^d$ and the set $\cal R$ consists of all (infinite) possible translated copies of a  given unit object $\sigma^*$ in $\IR^d$. In the online version of the unit covering problem, the set $\cal X$ is not known in advance.
 Charikar et al.~\cite{CharikarCFM04} studied the online version of the unit covering problem where $\sigma^*$ is a unit ball in $\IR^d$.  
 They proposed an online algorithm having a competitive ratio of~$O(2^dd\log d)$. They also proved $\Omega(\log d/\log\log \log d)$ as the lower bound for this problem. Dumitrescu et al.~\cite{DumitrescuGT20} improved both the upper and lower bound of the competitive ratio to $O({1.321}^d)$ and $\Omega(d+1)$, respectively.  When $\sigma^*$ is a centrally symmetric convex object in $\mathbb{R}^d$,   they  proved  the competitive ratio of every deterministic online algorithm  is at least $I(\sigma^*)$, where $I(\sigma^*)$ is the illumination number (for definition, see~\cite{DumitrescuGT20}) of the convex object $\sigma^*$. 
 When $\sigma^*\subset \IR^d$ is any object having  aspect$_\infty$ ratio (for definition see~\cite{DeJKS22}) as $\alpha$, 
  a deterministic online algorithm  is known as having a competitive ratio of at most~$\left(\frac{2}{\alpha}\right)^d\left((1+{\alpha})^d-1\right)$ $\log_{(1+\alpha)}(\frac{2}{\alpha}) +1$~\cite{DeJKS22}. Note that the aspect$_\infty$ ratio of any object is in the range $(0,1]$.
Dumitrescu and T{\'{o}}th~\cite{DumitrescuT22} studied another variant of the online unit covering problem where $\cal X$ is a set of points in $\mathbb{Z}^d$. They consider the case when $\sigma^*$  is a hypercube of side length one unit  in $\mathbb{R}^d$.  They~\cite{DumitrescuT22} proved that the competitive ratio of every deterministic online algorithm for this problem is at least $d + 1$.
 They also proposed a randomized online algorithm with a competitive ratio of $O(d^2)$ for this problem. 
% \red{We observe that  the results obtained for covering integer hypercubes in~\cite{DumitrescuT22} are still  valid if we consider  hypercubes of a side length  one instead of  integer hypercubes.}
For this problem, an equivalent version of the online hitting set problem is as follows:  ${\cal X}=\IR^d$ and the center of the objects in $\cal R$ are from $\mathbb{Z}^d$. To complement their result, in this paper, we consider the online hitting set problem when the ${\cal X}=\mathbb{Z}^d$ and the center of objects in $\cal R$ are from $\IR^d$.

% \blue {Another relevant problem is the piercing set problem. A collection of points $\P\subset \IR^d$ is defined as a \emph{piercing set} if each object in a given set ${\cal S}$ of geometric objects in $\IR^d$ contains at least one point from ${\P}$. The
% objective is to find a set $\P$ with the minimum number of points.
% The problem is known to be NP-complete even for unit squares in $\IR^2$~\cite{GareyJ79}.
% De et al.~\cite{SSingh} studied the problem in the online setting, where the objects (in $\IR^d$) arrive one by
% one to the online algorithm. Upon the arrival of an unstabbed object, the objective of the online algorithm is to assign a point from $\IR^d$ inside the newly arrived object. An online algorithm may add points to the piercing set, 
% but it cannot remove points from it. They proposed an online algorithm for the bounded scaled $\alpha$-aspect$_\infty$ objects having a competitive ratio of at most~$\left(\frac{2}{\alpha}\right)^d\left((1+{\alpha})^d-1\right)$ $\left(\lceil \log_{(1+\alpha)}(\frac{2k}{\alpha})\rceil \right)+1$ (for definition of bounded scaled $\alpha$-aspect$_\infty$ object, see~\cite[Section 2]{SSingh}).}

% Let us imagine that we have a power that helps us know what will happen in the future, and we can predict how today's decisions/actions will affect things in the long run. With the help of this, we can make appropriate decisions about what to do and what not to do. Due to this power, we can optimize all the decisions that will benefit our life. Nevertheless, in real life, things are almost the opposite. It is hard to predict the future; a lot can happen in the next three to five years. Some situations in our life are outside our control, and one terrible mistake in those situations might lead to much worse nightmares. We prefer to make strategies for the things that are primarily within our control so that we may reduce risk and increase productivity. Since all the events in real life are dynamic, we never know what is coming next (we do not know about the future in advance). Therefore, in this paper, instead of the offline hitting set problem, where we know the whole set of points and objects in advance, we will study the \emph{online version} of the hitting set problem in which geometric objects arrive by one. The online algorithm must maintain a valid hitting set $H\subset{\mathbb{Z}^d}$ for the already arrived objects. Once a new geometric object $\sigma \subset \mathbb{R}^d$ arrives, the online algorithm needs to hit it using some point $h\in H$ if the existing hitting set has not already hit it. An online algorithm may add points to the hitting set, but it can not remove the already added/chosen points (the online algorithm needs to take irrevocable decisions). The hitting set problem aims to minimize the cardinality of the hitting set.
\subsection{Notation and Preliminaries}\label{1.2}
 We use $[n]$ to denote the set $\{1,2,\ldots,n\}$. 
By an \emph{object}, we refer to  a simply connected compact set in $\mathbb{R}^d$ having a nonempty interior. {For any point $p\in\IR^d$, we use $p(x_i)$ to denote the $i$th coordinate of $p$, where $i\in[d]$.}
 An \emph{integer point} is a point $p \in  \mathbb{R}^d$  such that for each $i\in[d]$ the coordinate $p(x_i)$ is an integer. Any two integer points $p$ and $q$ are said to be \emph{consecutive integer points} if there exists an index $j\in[d]$ such that $|p(x_j)-q(x_j)|=1$ and  $p(x_i)=q(x_i)$ for all $i\in[d]\setminus\{j\}$.
  We use $\Q(\sigma)$ to denote the set of integer points contained in an object $\sigma$. For any $\chi\subset\mathbb{Z}^d$, the term $\chi(\sigma)$ denotes the intersection of $\chi$ and $\Q(\sigma)$.
 % Let $p=(p_1,p_2,\ldots,p_d)$ and $q=(q_1,q_2,\ldots,q_d)$ are two integer points, then $p$ and $q$ are said to be \emph{adjacent integer point}  such that for some $i\in[d]$ 
 % % if they differ exactly at one coordinate, say $i$th coordinate, such that 
 % $|p_i-q_i|=1$, and $p_j=q_j$, for $j\in[d]$ and $j\neq i$.
The term \emph{integer hypercube} refers to a 
 hypercube $H\subset\mathbb{R}^d$ of side length one having all corners as integer points. {We use $dist(x,y)$ (respectively, $d_{\infty}(x,y)$) to represent the distance between two points $x$ and $y$ under $L_2$-norm (respectively, $L_{\infty}$-norm).} Let $c$ be a point  in $\IR^d$.
We use $H_d(c,r)$ to denote an $L_{\infty}$ ball of radius $r$ centered at $c$. In other words,  $H_d(c,r)=\{x\in\mathbb{R}^d:dist_{\infty}(x,c) \leq r\}$.
 A \emph{unit hypercube}  $H_d(c,1)\subset \mathbb{R}^d$ centered at $c$, is defined as $H_d(c,1)=\{x\in\mathbb{R}^d:dist_{\infty}(x,c) \leq 1\}$. Note that, according to our definition, an integer hypercube is not a unit hypercube.  A \emph{unit ball}  $B_d(c,1)\subset \mathbb{R}^d$ centered at $c$, is defined as $B_d(c,1)=\{x\in\mathbb{R}^d:dist(x,c)\leq 1\}$.
 Throughout the paper, if not stated otherwise, the term ``hypercube'' is used to refer to an axis-aligned unit hypercube and the term ``ball'' is used to refer to a unit ball.
% Let $C$ be a convex object in $\mathbb{R}^d$. Let $p$ and $q$ are two integer points, then $p$ and $q$ are said to be \emph{adjacent integer point} if they differ exactly at one coordinate, say $j$th coordinate, such that $|p(x_j)-q(x_j)|=1$, and all the other $(d-1)$ coordinate of $p$ and $q$ are same.
\subsection{Our Contributions}

% -------------------------Old Contribution : Starts---------------------------


% {need to change this and abstract.}\\
% First,  for hitting unit intervals using points in $\mathbb{Z}$, we propose a deterministic online algorithm that achieves a tight bound of two (Theorem~\ref{thm:int}, Section~\ref{sec:hyp}). Next, we consider the case of $d$-dimensional unit hypercubes. For this, we propose a simple deterministic algorithm that achieves a competitive ratio of at most~$14$ when $d=3$. For $d>2$, we propose a randomized algorithm having a competitive ratio of at most $O(d^2)$. We also prove that for hitting hypercubes using integer points every deterministic online algorithm has a competitive ratio of at least~$d+1$ when $d\in\mathbb{N}$. Finally, we consider studying unit balls in $\IR^d$. For hitting unit balls in $\IR^d$ using integer points from $\mathbb{Z}^d$, at first, we propose a deterministic algorithm having a competitive ratio of at most $O(d^4)$ and $O(d^3)$, for $d\geq4$ and $d=3$, respectively. For unit disks in $\IR^2$, we propose a simple  deterministic algorithm that achieves a competitive ratio of~$4$. At last, we prove that for hitting $d$-dimensional unit balls using integer points in $\mathbb{Z}^d$ every deterministic algorithm has a competitive ratio of at least~$d+1$ when $d<4$. All the above-mentioned results are also valid for the equivalent unit covering problem where the points $p\in\mathbb{R}^d$ are coming one by one, and we need to cover each point with a unit object centered at some integer point $c\in\mathbb{Z}^d$.


% -----------------------------Old Contribution: Ends-----------------------------


% -----------------------------New Contribution -----------------------------
We consider the online hitting set problem where ${\cal X}={\mathbb Z}^d$, and $\cal R$ consists of translated copies of a unit geometric object in $\IR^d$.
First, we consider the  problem for the lower-dimensional geometric objects.
As a warm-up, first, we consider the case when the set $\cal{R}$ consists of  one-dimensional unit hypercubes, i.e., unit intervals. For this case, we propose a deterministic online algorithm having an optimal competitive ratio of $2$ (Theorem~\ref{thm:int}). When $\cal R$ consists of unit hypercubes in $\IR^2$ and $\IR^3$, we propose a deterministic online algorithm having a competitive ratio of at most $4$ and $8$, respectively (Theorem~\ref{square_ub} and Theorem~\ref{cube_ub}). When $\cal R$ consists of unit balls in $\IR^2$ and $\IR^3$, we present a deterministic online algorithm with a competitive ratio of at most~$4$ and $14$, respectively (Theorem~\ref{2d-balls} and Theorem~\ref{3d-ball}).
Next, we consider the online hitting set problem for geometric objects in $\IR^d$. When $\cal R$ consists of unit hypercubes in $\IR^d$, first, we prove that every deterministic online algorithm has a competitive ratio of at least~$d+1$ when $d\in\mathbb{N}$ (Theorem~\ref{hyp_lb}). Afterwards, we propose a randomized iterative reweighting algorithm that is similar in nature to an algorithm proposed by Dumitrescu and T{\'{o}}th in~\cite{DumitrescuT22}. We use some  structural properties  to analyze that this randomized algorithm has a  competitive ratio of at most~$O(d^2)$, for $d\geq3$ (Theorem~\ref{hyp_ub}). 
 When $\cal R$ consists of unit balls in $\IR^d$, first, we show that every deterministic online algorithm has a competitive ratio of at least~$d+1$ when $d<4$  (Theorem~\ref{ball_lb}). Then, we propose a deterministic online algorithm having a competitive ratio of at most $O(d^4)$ for $d\in\mathbb{N}$ (Theorem~\ref{ball_ub}). All the above-mentioned results also hold for the equivalent  geometric set cover problem in the online setup. 
 % All the results obtained in this paper are summarized  in Table~\ref{table:1}.
%%% Ommitted for the time being
%
% \begin{table}[htbp]
%     \centering
%     \begin{tabular}{|p{1.8 cm}|p{2.5 cm}|p{5.5 cm}|p{2.5 cm}|p{2.5 cm}|}
%     \hline
%   \multicolumn{1}{|p{1.8 cm}|}{Problem} &  \multicolumn{2}{p{6.5 cm}|}{Model of Computation} & \multicolumn{1}{p{3.1 cm}|}{Lower Bound of Competitive Ratio} & \multicolumn{1}{p{2.75 cm}|}{Upper Bound of Competitive Ratio}\\
% \hline
%    &Point Set ($\cal X$)  &Ranges ({$\cal R$})   & &\\
% \hline Hitting Set & \cmark;$\IR^d$ &\xmark; $I_d$& $d+1$~\cite{DumitrescuT22} & $O(d^2)$~\cite{DumitrescuT22}\\
% \cline{2-5}  & \cmark;$\mathbb{Z}^d$ $(d=2,3)$&\xmark; $B_d(p,1)$, where $p\in\IR^d$\&$(d=2,3)$& 3, 4 & 4, 14\\
% % &  Known & Not Known &  & \\
% % \cline { 2 - 5 } & $\mathbb{Z}^3$ & $B_3(p,1)$, where $p\in\IR^3$ & $4$ & $14$\\
% % &  Known & Not Known &  & \\
% \cline { 2 - 5 } &\cmark; $\mathbb{Z}^d$ & \xmark;$B_d(p,1)$, where $p\in\IR^d$ & $d+1$ ($d<4$) & $O(d^4)$\\
% % &  Known & Not Known &  & \\
% \cline { 2 - 5 } &\cmark; $\mathbb{Z}^d$ $(d=2,3)$ & \xmark;$H_d(p,1)$, where $p\in\IR^d$ \& $(d=2,3)$ & $3$, 4 & $4$, 8\\
% % &  Known & Not Known &  & \\
% % \cline { 2 - 5 } & Known & Not Known & $\mathbb{Z}^3$ & $H_3(p,1)$, where $p\in\IR^3$ & $4$ & $8$\\
% \cline { 2 - 5 } & \cmark;$\mathbb{Z}^d$ & \xmark;$H_d(p,1)$, where $p\in\mathbb{R}^d$& $d+1$ & $O(d^2)$\\
% % &  Known & Not Known &  & \\
% \hline  Set Cover & \xmark;$\mathbb{Z}^d$ & \cmark; $I_d$ & $d+1$~\cite{DumitrescuT22} & $O(d^2)$~\cite{DumitrescuT22}\\
% \cline{2-5} & \xmark;$\IR^d$ $(d=2,3)$ & \cmark; $B_d(p,1)$, where $p\in\mathbb{Z}^2$ \&$(d=2,3)$& $3$, 4 & $4$, 14\\
% % & Not Known & Known &  & \\
% % \cline { 2 - 5 } &\xmark; $\IR^3$ &\cmark; $B_3(p,1)$, where $p\in\mathbb{Z}^3$ & $4$ & $14$\\
% % & Not Known & Known &  & \\
% \cline { 2 - 5 } &\xmark; $\IR^d$ &\cmark; $B_d(p,1)$, where $p\in\mathbb{Z}^d$ & $d+1$ ($d<4$) & $O(d^4)$\\
% % & Not Known & Known &  & \\
% \cline { 2 - 5 } & \xmark;$\IR^d$ $(d=2,3)$ & \cmark;$H_d(p,1)$, where $p\in\mathbb{Z}^d$\&$(d=2,3)$ & $3$, 4 & $4$,8\\
% % & Not Known & Known &  & \\
% % \cline { 2 - 5 }  &\xmark; $\IR^3$ & \cmark;$H_3(p,1)$, where $p\in\mathbb{Z}^3$ & $4$ & $8$\\
% % & Not Known & Known &  & \\
% \cline { 2 - 5 }  &\xmark; $\IR^d$ &\cmark; $H_d(p,1)$, where $p\in\mathbb{Z}^d$ & $d+1$ & $O(d^2)$\\
% % & Not Known & Known &  & \\
% \hline
% \end{tabular}
%     \caption{Summary of results obtained in this paper and in~\cite{DumitrescuT22} for geometric hitting set and set cover problem in the online setup. Here, \cmark denotes that the set is already known beforehand, whereas \xmark denotes that the set is not known in advance. The notation $B_d(p,1)$ denotes the unit ball in $\IR^d$ centered at $p$, $H_d(p,1)$ denotes the unit hypercube in $\IR^d$ centered at $p$ and $I_d$ denotes the integer hypercube of side-length one.}
%     \label{table:1}
% \end{table}

%%% Ommitted for the time being
%%%%%%%%%%%%%%%%%%%%%----------------Other Table: Start -------------%%%%%%%%%%%%%%%%%%%%%%
% \begin{table}[htbp]
%     \centering
%     \begin{tabular}{|c|c|c|c|c|}
% \hline Ranges/Objects  & \multicolumn{2}{|c|}{Lower Bound of Competitive Ratio } &  \multicolumn{2}{|c|}{Upper Bound of Competitive Ratio }  \\
% \cline { 2 - 5 } & Online Hitting Set & Online Set Cover &Online Hitting Set & Online Set Cover  \\
% \hline Unit Hypercubes $(d=2,3)$ & 3,4 & 3,4 & 4,8 & 4,8 \\
% \hline Unit Hypercubes $(d\in\IR^d), d\geq 3$ & $d+1$ & $d+1$ & $O(d^2)$ & $O(d^2)$ \\
% \hline Unit Balls $(d=2,3)$ & 3,4 &3,4  & 4,14 &4,14 \\
% \hline Unit Balls $(d\in\IR^d)$ & 4 & 4 & $O(d^4)$ & $O(d^4)$\\
% \hline
% \end{tabular}
%     \caption{Summary of results obtained in this paper.}
%     \label{tab:1}
% \end{table}

%%%%%%%%%%%%%%%%%%%%%----------------Other Table: End -------------%%%%%%%%%%%%%%%%%%%%%%
%%
% Note that the randomized iterative reweighting algorithm (Theorem~\ref{hyp_ub}) is only valid for hypercubes in $\IR^d$, where $d\geq 3$. For hitting unit hypercubes in $\IR^3$, the iterative reweighting algorithm has a competitive ratio of at most~40.657. For the case of  unit balls in $\IR^2$ and $\IR^3$, after a careful analysis of the deterministic algorithm (Theorem~\ref{ball_ub}), one can have a competitive ratio of at most~13~and~33, respectively. 
% All the above-mentioned results also hold for the equivalent  geometric set cover problem in the online setup.
% % , where the points in $\IR^d$ are coming one by one. Upon the arrival of an uncovered point, the online algorithm needs to cover it using a unit object centered at some point in $\mathbb{Z}^d$. The online algorithm cannot remove any unit object in the future from the existing covering set. The  aim of the unit covering problem is to minimize the number of unit objects to cover all the presented points. 
% All the results obtained in this paper are summarized below in Table~\ref{table:1}.
% % We summarize all the obtained results in~Section~\ref{sec:cover}.

 
  
   
 %   For the case of the unit hypercube in $\IR^3$, the randomized online algorithm achieves a competitive ratio of at most~40.657, and the deterministic algorithm for unit balls in $\IR^2$ and $\IR^3$ achieve a competitive ratio of at most~13~and~33, respectively. 
 % Eventually, to close the gap between upper and lower bounds for lower dimensional unit balls and unit hypercubes, we present deterministic online algorithms achieving  better competitive ratios as follows.
   
   
 %   Then, for unit disk in $\IR^2$, we propose a simple  deterministic algorithm that achieves a competitive ratio of~$4$. At last, we show that every deterministic online algorithm for hitting balls using integer points has a competitive ratio of at least~$d+1$ when $d<4$.







% \begin{itemize}
    % \item [(I)] As a warm-up,  we consider the case when objects are one-dimensional unit hypercubes, i.e., unit intervals. For this case, we propose a deterministic online algorithm with an optimal competitive ratio of $2$ (Theorem~\ref{thm:int}, Section~\ref{sec:hyp}).
    % \item [(II)] Next, we consider the case of axis-aligned $d$-dimensional unit hypercubes. Initially, we prove that every deterministic online algorithm for hitting unit hypercubes in $\IR^d$ achieves a competitive ratio of at least~$d+1$ when $d\in\mathbb{N}$ (Theorem~\ref{hyp_lb}, Section~\ref{sec:hyp}). Later, first, we discuss some useful structural properties and then, we propose a  randomized online algorithm for  hitting unit hypercube in $\IR^d$ that has a competitive ratio of at most~$O(d^2)$, where $d\geq3$ (Theorem~\ref{hyp_ub}, Section~\ref{sec:hyp}).
    % \item[(III)] Then, we consider studying unit balls in $\IR^d$. At first, we show that every deterministic online algorithm for hitting unit balls in $\IR^d$ has a competitive ratio of at least~$d+1$ when $d<4$  (Theorem~\ref{ball_lb}, Section~\ref{sec:balls}). Later, for hitting unit balls in $\IR^d$ using integer points, we propose a deterministic online algorithm having a competitive ratio of at most $O(d^4)$ for $d\in\mathbb{N}$ (Theorem~\ref{ball_ub}, Section~\ref{sec:balls}).
   
    % \end{itemize}
% \noindent
%     The result obtained in  Theorem~\ref{hyp_ub} is valid only for hypercubes in $\IR^d$, where $d\geq 3$. For $d=3$, the randomized online algorithm mentioned in Section~\ref{hypercube_upperb} achieves a competitive ratio of at most~40.657. On the other hand, after careful analysis of Theorem~\ref{ball_ub}, one can achieve a competitive ratio of at most~13~and~33 for balls in $\IR^2$ and $\IR^3$, respectively. Eventually, 
%         to close the gap between upper and lower bounds for lower dimensional unit balls and unit hypercubes, we present  deterministic online algorithms achieving  better competitive ratios as follows.

%     \begin{itemize}
%         % \item[(IV)] 
%     % \begin{itemize}
%         \item [(IV)] We propose a deterministic online algorithm for unit hypercubes in $\IR^2$ and $\IR^3$ having a competitive ratio of at most $4$ and $8$, respectively (Theorem~\ref{square_ub} and Theorem~\ref{cube_ub}, Section~\ref{sec:lower_dimensional}).
%         \item [(V)] We present a deterministic online algorithm for unit balls in $\IR^2$ and $\IR^3$ with a competitive ratio of at most~$4$ and $14$, respectively (Theorem~\ref{2d-balls} and Theorem~\ref{3d-ball}, Section~\ref{sec:lower_dimensional}).
%     % \end{itemize} 
% \end{itemize}

\subsection{Organization}
In Section~\ref{sec:hyp}, we present the lower and upper bound of the competitive ratio for hitting axis-aligned $d$-dimensional unit hypercubes.
Next, in Section~\ref{sec:balls}, for hitting unit balls in $\IR^d$, we give the lower and upper bound of the competitive ratio. In Section~\ref{sec:lower_dimensional}, we present deterministic online algorithms for unit balls and unit hypercubes in $\IR^2$ and $\IR^3$. Later, in Section~\ref{sec:cover}, we summarize the results obtained for the unit covering problem.
 Eventually, in Section~\ref{Conclusion}, we conclude.



% \section{Unit Covering and Unit Hitting }\label{Unit_cover}
% Let $C$ be an object, and let $-C$ be the reflection of $C$ through a point $c\in  C$.
% \begin{observation}\label{obs-1}
% Let $x$ and $y$ be any two points in $\mathbb{R}^d$, then $d_C(x,y)=d_{-C}(y,x)$. 
% \end{observation}

% Using the above observation, we have the following.

% \begin{lemma}\label{lemma_d_C}
% Let $x$ and $y$ be any two points in $\mathbb{R}^d$. Then $x$ lies in $y+C$ if and only if $y$ lies in $x+(-C)$.
% \end{lemma}
% \begin{proof}
% If  $x$ lies in $y+C$, then $d_C(y,x) \leq 1$. Due to Observation~\ref{obs-1}, $d_{-C}(x,y) \leq 1$. Therefore, $y$ lies in $x+(-C)$. Similarly, one can prove the converse. Hence the lemma follows.
% \myqed
% \end{proof}


% \begin{definition}[Unit Hitting Problem]
% Given a family $\mathcal{S}$  of translated copies of an object $C$ in $\mathbb{R}^d$, and  a set $\P$ of points in $\mathbb{Z}^d$, in the {unit hitting problem}, we need to hit each object by placing the minimum number of points in $\mathbb{Z}^d$.
% \end{definition}

% \begin{definition}[Unit Covering Problem]
% Given a set of points $X \subseteq \mathbb{R}^d$, in the unit covering problem, we need to place the minimum number of translated copies of an object $-C$ whose center $-c\in\mathbb{Z}^d$ to cover all the points in $X$.
% \end{definition}



% The following theorem connects the above two problems.

% \begin{theorem}\label{equivalence}
% The unit hitting problem is equivalent to the unit covering problem.
% \end{theorem}
% The proof of the above theorem is inspired by Durocher and Fraser~\cite{DurocherF15}; they proved that the geometric set cover problem is dual to the geometric hitting set problem for translated copies of a fixed object. 
% \begin{proof}
% For the unit hitting problem, let us assume that each unit object in $\cal S$ is a  translated copy of $C\subset \mathbb{R}^d$. Due to Lemma~\ref{lemma_d_C}, we know  that  some point $x\in \mathbb{R}^d$ pierce an object $y+C$ if and only if the object $x+(-C)$ covers the point $y$. Thus,  we can convert this problem to an equivalent unit covering problem, where the set $X$ of points are the center of all the objects in $\cal S$, and 
%  we need to cover them by translates of $(-C)$. 
% In a similar fashion, one can prove the other side of the equivalence.
% \myqed
% \end{proof}

% As a consequence of Theorem~\ref{equivalence}, all the results in this paper are carried out for the online covering problem, when objects are translates of each other.  



%%%%%%%%%
%%%%.   Lower Bounds
%%%%%%%%%




% \subsubsection{Lower Bound for Squares}
% \begin{lemma}
% The  competitive ratio of every deterministic online algorithm  for hitting unit squares using points in $\mathbb{Z}^2$ is at least~$3$.
% \end{lemma}
% \begin{figure}[htbp]
%   \centering
%   \includegraphics[width=50mm]{square_a.pdf}
% \caption{Lower bound illustration for unit squares.}
% \label{fig:sq_lb}
% \end{figure}
% \begin{proof}
% To prove the lower bound on the competitive ratio for hitting unit squares using points in $\mathbb{Z}^2$, think of a game between Alice and Bob. Here, Alice plays the role of adversary and Bob plays the role of the online algorithm, i.e., Alice will present the unit squares, and Bob needs to hit it by a hitting point from $\mathbb{Z}^2$, if the presented square is not hit by a previously placed hitting point.
% Note that Alice needs to present squares to force Bob to place as many new hitting points as possible, whereas Bob needs to put as few hitting points as possible for the win.

% Initially Alice presented the square $\sigma_1$ centered at integer point (the square $\sigma_1$ contains maximum number of integer points inside it, see Fig.\ref{fig:sq_lb}). To hit the input square $\sigma_1$ Bob needs places the integer point $h_1$ such that $h_1\in \sigma_1$. Let Bob places the center of the input square $\sigma_1$ as the piercing point (if Bob is choosing any point other than the center of $\sigma_1$ to hit the input square $\sigma_1$, it will allow Alice to present the next square containing more than three integer points). Depending on the position of the piercing point placed by Bob, Alice will present the second square $\epsilon$ distance away in 1st coordinate from the hitting point $h_1$ such that $\sigma_2$ contains at least three integer points (see Fig.\ref{fig:sq_lb}). Clearly the input square $\sigma_2$ cannot hit by the point $h_1$ forcing Bob to place a new hitting point $h_2$ such that $h_2\in \sigma_2$. Depending on the position of $h_2$, Alice will present the third square $\sigma_3$ at $\epsilon$ distance away in the 2nd coordinate from the hitting point $h_2$ such that $\sigma_3$ contains at least one integer point.

% (use translation vectors for more concise proof).
% \end{proof}
% \subsubsection{Upper Bound  for Squares}
% \begin{lemma}\label{lemma:sq_ub}
% For hitting unit squares using points in $\mathbb{Z}^2$, there exists a deterministic online algorithm whose competitive ratio is at most $4$.
% \end{lemma}

% \begin{figure}[htbp]
% \begin{subfigure}{.45\textwidth}
%   \centering
%   \includegraphics[width=63 mm]{square_algo.pdf}
%   \label{fig:sq_algo}
% \end{subfigure}
% \begin{subfigure}{.45\textwidth}
%   \centering
%   \includegraphics[width=.7\linewidth]{square_ub.pdf}
%   \label{fig:sq_ub}
% \end{subfigure}
% \caption{Figure on left is illustration of the algorithm for square lattice, where $i,j\in \mathbb{Z}^2$; while figure on right Illustrate about the neighbourhood of $p$ (denoted as $N(p)$, red colored square), which includes all the centres of unit squares containing the point $p$.}
% \label{fig:fig_1}
% \end{figure}
% \begin{proof}
% The square lattice $\Lambda=\{q \bf{e_1}$ $+r\bf{e_2}$ $|\ q,r\in \mathbb{Z}^2\}$ is generated by standard unit vector $\bf{e_1}$ and $\bf{e_2}$. Partition the whole square lattice using unit squares centered at $\varXi=\{q \bf{u}$ $+r \bf{v}$ $|\ q,r\in \mathbb{Z}^2\}$, where the vector $\bf{u}$ $=2$ $\bf{e_1}$ and $\bf{v}={e_1}$ $+2$ $\bf{e_2}$.

% \noindent
% \textbf{Algorithm:}
% On receiving a new input square $\sigma$ centered at $c$, if it has not been hit by the existing hitting set, then our online algorithm adds the point $h\in\varXi$ as the hitting point such that the center $c$ lies in the partitioned square having center $h$.

% \noindent
% \textbf{Analysis:}  Let $\OO$ be the optimal hitting set and $\A$ be the hitting set returned by our online algorithm.  Let $p \in \OO$ be any point in the optimal hitting set (see Fig.\ref{fig:int_ub}).  
% Let $N(p)$ denote the neighbourhood of $p$, which includes all the centers of unit squares containing the point $p$. From Fig.\ref{fig:int_ub} observe that $N(p)$ can intersect with at most three squares, therefore, for all the squares whose centre lies in the interior of $S_2, S_3$ and $S_4$, our algorithm will choose the point $h_2,h_3$ and $h_4$, respectively to hit them. For all the squares whose center lies on the boundary of $S_2$, our algorithm will choose the point $h_1$ or $h_2$ depending on the position of the center. Therefore,
% our online algorithm needs four hitting points $h_1,h_2,h_3\text{ and} h_4$; whereas the offline optimum needs
% just one point $p$. Hence the lemma follows.
% \end{proof}

% \begin{theorem}\label{thm:sq}
% For hitting unit squares using points from $\mathbb{Z}^2$, there exists a deterministic online algorithm having competitive ratio at most $4$. While the competitive ratio of any
% deterministic online algorithm for this problem is at least $3$.
% \end{theorem}




















% \textbf{Upper Bounds for Balls in $\IR^2$ and $\IR^3$}\\
% For the lower dimensions, one can give an improved analysis of the nearest point algorithm mentioned in the above theorem.
% Observe that for $d=2$ and $3$, the value of $i$ is at most 2 and 3, respectively. As a result, for these special cases, there will be at most~13 and 33 integer points, respectively, satisfying Equation~(\ref{eqn_z}). Hence, for $d=2$ and 3, the nearest point algorithm (in the above theorem) has a competitive ratio of at most~13 and 33, respectively.

\section{Lower Dimensional Unit Balls and Hypercubes}\label{sec:lower_dimensional}

In  this section, we consider the online hitting set problem for lower-dimensional unit balls and hypercubes. For this purpose, we propose a deterministic online ``$\BPA$".  Before we give a general overview of the algorithm, 
 let us define  a `relation' $\prec$ among distinct points in $\IR^d$ as follows.
Note that for any pair of distinct points $p$ and $q$ in $\IR^d$, there exists a unique index $i\in [d]$ such that  $p(x_{i})\neq q(x_{i})$ and  $p(x_j)=q(x_j)$ for each $j\in \{i+1,\ldots,d\}$.
 If $p(x_{i})< q(x_{i})$,  we say that $p \prec q$; otherwise  $q \prec p$. Note that this gives a strict total ordering for any set $P\subset \IR^d$ of distinct elements.
 For a set $P$ of distinct points, a point $p^*\in P$ is defined as the \emph{best-point}   if $q \prec p^{*}$, for all $q(\neq p^{*})\in P$. 
Depending upon the objects and dimensions, we consider a \emph{filter-set}: a subset $\chi$ of integer points such that 
any input object must contain at least one point of $\chi$.
Our algorithm maintains a hitting set $\A$ consisting of points from $\chi$. The algorithm works as follows.

{{$\BPA$:}}
Initially $\A=\emptyset$. 
 On receiving a new input  object $\sigma$, if it is not hit by any of the points from $\A$, our online algorithm adds the best-point from $\chi$ lying inside $\sigma$ to the set $\A$. 
 
 The correctness and efficiency (competitive ratio)  of the algorithm follow from judiciously choosing the filter-set $\chi$.
As a warm-up, first, we consider when objects are one-dimensional, i.e., intervals.
%\subsection{Unit Intervals}

\begin{theorem}\label{thm:int}
For hitting unit intervals using points from $\mathbb{Z}$, there exists a deterministic online algorithm having a competitive ratio at most $2$. This result is tight: the competitive ratio of any
deterministic online algorithm for this problem is at least $2$.
\end{theorem}
% \subsection{Proof of Theorem~\ref{thm:int}}\label{app_thm:int}

\begin{proof}
First, we  prove the upper bound of the competitive ratio.
Let $\Lambda=\{q{\bf e}_1\ |\ q\in \mathbb{Z}\}$ be the integer lattice generated by standard unit vector ${\bf e}_1$. Partition the whole integer lattice using integer point from $\chi=\{2q {\bf e_1}\ |\ q\in \mathbb{Z}\}$. Note that any unit interval can contain at least one and at most two integer points from $\chi$.
Our algorithm maintains a hitting set $\A$. Initially $\A=\emptyset$.
On receiving a new input interval $\sigma$, if it is not hit by any of the points from $\A$, our algorithm adds one integer point from $\chi$ contained in the interval $\sigma$ to the set $\A$.
 
Let $\I$ be the set of intervals  presented to the algorithm.
Let $\OO$ be an optimal offline hitting set for $\I$.  
 Let $\A'=\A\setminus \{\A\cap \OO\}$ and  $\OO'=\OO\setminus \{\A\cap \OO\}$. 
 Let $p \in \OO'$ be an integer point and let $\I_p\subseteq \I$ be the set of input intervals that are hit by the point $p$.
 Let $\A_{p}\subseteq \A'$ be the set of points used by our algorithm to hit all the intervals in $\I_p$.   If $p\in\chi$, then $\A_{p}$  contains either $\{p,p+2\}$ or $\{p-2,p\}$ from $\chi$, since $p\notin \A\cap \OO$, thus $|\A_p|\leq 1$; otherwise, $\A_{p}$  contains at most two integer points: $p-1$ and $p+1$ from $\chi$. 
Therefore, $|\A_p|\leq 2$. Since $\A'=\cup_{p\in \OO'}\A_p$, we have $|\A'|\leq\sum_{p\in \OO'}|\A_p|\leq 2\times |\OO'|$. Note that $\frac{|\A'| }{|\OO'|}\leq 2$ implies $\frac{|\A|}{|\OO|}\leq 2$.
Thus, the competitive ratio of our algorithm is at most~2.


To prove the lower bound of the competitive ratio, we construct a sequence of intervals  $\sigma_1,\sigma_2$ adaptively such that any online algorithm needs to place two integer points; while an offline optimum needs just one point.
Initially, we present a unit interval $\sigma_1=[x,x+2]$, where $x\in \mathbb{Z}$. Any online algorithm places an integer point $h_1=x+i$, where $i\in\{0,1,2\}$,  to hit the interval $\sigma_1$. For any choice of $i\in\{0,1,2\}$ for the hitting point $h_1$, it is always possible to present another interval $\sigma_2$ that does not contain the point $h_1=x+i$ but contains the  point $x'=x+((i+1)\mod 3)\in\{x,x+1,x+2\}$. Hence, the theorem follows. \myqed
\end{proof}




% Note that the result obtained in  Theorem~\ref{hyp_ub} is valid only for hypercubes in $\IR^d$, where $d\geq 3$. For $d=3$, the randomized online algorithm mentioned in Section~\ref{hypercube_upperb} achieves a competitive ratio of at most~40.657. On the other hand, after careful analysis of Theorem~\ref{ball_ub}, one can achieve a competitive ratio of at most~13~and~33 for balls in $\IR^2$ and $\IR^3$, respectively. 




\subsection{Unit Hypercubes in $\IR^2$ and $\IR^3$}
Let $\Lambda_d=\{\alpha_1 {\bf e}_1+\alpha_2 {\bf e}_2+\ldots+\alpha_d {\bf e}_d\ |\ \alpha_i\in \mathbb{Z}\ \forall i\in[d]\} $  be the integer lattice in $\IR^d$ generated by standard unit vectors ${\bf e}_1$, ${\bf e}_2\ldots,{\bf e}_d$. Consider a subset $\chi_d\subset \Lambda_d$ defined as follows:\\
$\chi_d=\{\alpha_1 {\bf u}_1+\alpha_2 {\bf u}_2+\ldots+\alpha_d {\bf u}_d\ |\ \alpha_i\in \mathbb{Z}\text{ and } \forall\ i\in[d]\}$, where ${\bf u}_i=-{\bf e}_{i-1}+2{\bf e}_i$ for all $i\in[d]\setminus\{1\}$ and  ${\bf u}_1=2{\bf e}_1$.

\begin{lemma}\label{lem:correct_hyp}
For {$d\in\mathbb{N}$}, each unit hypercube $H_d(r,1)$ centered at any point $r\in\IR^d$ contains at least one point of $\chi_d$.
\end{lemma}
\begin{proof}
Let $\Lambda_d'=\{\alpha_1 {\bf u}_1+\alpha_2 {\bf u}_2+\ldots+\alpha_d {\bf u}_d\ |\ \alpha_1,\alpha_2,\ldots,\alpha_d\in \mathbb{Z}\}$  be a lattice in $\IR^d$ generated by vectors ${\bf u}_1, {\bf u}_2,\ldots,{\bf u}_d$ such that ${\bf u}_i={\bf e}_i+{\bf v}$ , where ${\bf v}=\left(\frac{1}{2},\frac{1}{2},\ldots,\frac{1}{2}\right)\in\IR^d$, for all $i\in[d]$ (for description of $\Lambda_2'$, see Fig.~\ref{fig:lambda'}). Let us consider that $H_d(r,1)$ be a unit hypercube centered at any point $r\in\IR^d$. We need to show that $H_d(r,1)$ contains at least one point from $\chi_d$.
Note that there exists a point $s'\in\Lambda_d'$ such that $r$ belongs to the integer hypercube $H_d\left(s',\frac{1}{2}\right)$. Since the $dist_{\infty}(r,x)\leq 1$ for any $x\in H_d\left(s',\frac{1}{2}\right)$, thus, the $H_d\left(s',\frac{1}{2}\right)$ is totally contained in $H_d\left(r,1\right)$.
Due to Claim~\ref{clm:cont}, each integer hypercube $H_d\left(s,\frac{1}{2}\right)$  contains at least one integer point of $\chi_d$. As a result, the hypercube $H_d(r,1)$  contains at least one point of $\chi_d$.
% Let $H_d\left(s,\frac{1}{2}\right)$ be a hypercube centered at a point $s\in\Lambda_d'$.
%  Recall that a hypercube is said to be an integer hypercube if all of its corners are integer points. Notice that $H_d\left(s,\frac{1}{2}\right)$ is an integer hypercube. 
\begin{clm}\label{clm:cont}
    For $d\in\mathbb{N}$, each integer hypercube $H_d\left(s,\frac{1}{2}\right)$ contains at least one integer point of $\chi_d$, where $s\in\Lambda_d'$.
\end{clm}
 \begin{proof}
 We  prove this claim using induction on $d$.
 % Let $S(d)$ be the statement that each integer hypercube $H_d\left(s,\frac{1}{2}\right)\subset\IR^d$ contains exactly one integer point of $\chi_d$, where $s\in\Lambda'_{d}$.
In the base case of the induction, for $d=1$ and 2, it is easy to observe from Fig.~\ref{fig:l1} that each integer hypercube $H_i\left(s,\frac{1}{2}\right)$ contains exactly one integer point of $\chi_d$.
For $d\in\{3,4,\ldots,i\}$, let us assume that the induction hypothesis holds.
% for $S(i)$,i.e., under this hypothesis for $i=3,4,\ldots,d-1$ each integer hypercube $H_{i}\left(s,\frac{1}{2}\right)$ contains exactly one integer point of $\chi_i$, where $s\in\Lambda_{i}'$. 
Now, to complete the proof, we need to show that the induction is also true for $d=i+1$. Now, for any $k\in \mathbb{Z}$, we define the hyperplane $P_k=\{x\in\IR^{i+1}\ |\ x_{i+1}=k\}$.
Notice that the integer hypercube $H_{i+1}\left(s,\frac{1}{2}\right)$ contains integer points from two consecutive hyperplanes $P_k$ and $P_{k+1}$ for some $k\in\mathbb{Z}$. As per the definition of $\chi_{i+1}$, exactly one of the hyperplanes $P_k$ and $P_{k+1}$ contains points from $\chi_{i+1}$, and the other one does not contain any point from $\chi_{i+1}$. Without loss of generality, let us assume that the hyperplane $P_k$ contains points of $\chi_{i+1}$. Notice that $\chi_{i+1}\cap P_k=\chi_i$ and $H_{i+1}\left(s,\frac{1}{2}\right)\cap P_k= H_{i}\left(s',\frac{1}{2}\right)$, for some $s'\in\Lambda_i'$.
% the intersection of hypercube $H_{i+1}\left(s,\frac{1}{2}\right)$ with the hyperplane $P_k$ is a integer hypercube in $\IR^{i}$. 
Due to the induction hypothesis, for any point $p\in\Lambda_{i}'$, the integer hypercube $H_{i}\left(p ,\frac{1}{2}\right)$ contains at least one point from $\chi_{i}$. Thus, $H_{i+1}\left(s,\frac{1}{2}\right)$ contains at least one point of $\chi_{i+1}$ from the hyperplane $P_k$.
\myqed
\end{proof}

 \myqed\end{proof}
 
\begin{theorem}\label{cube_ub}
For hitting unit hypercubes in $\IR^3$ using points from $\mathbb{Z}^3$, there exists a deterministic online algorithm that achieves a competitive ratio of at most~8.
\end{theorem}
\begin{proof}
% Let $\Lambda=\{q {\bf e}_1+r {\bf e}_2+s {\bf e}_3 | q,r,s\in \mathbb{Z}\}$  be the integer lattice in $\IR^3$ generated by standard unit vectors ${\bf e}_1$, ${\bf e}_2$ and ${\bf e}_3$. Consider a subset $\chi\subset \Lambda$ defined as follows: $\chi=\{q {\bf u}+r {\bf v}+s {\bf w}| q,r,s\in \mathbb{Z}\}$, where ${\bf u}=2{\bf e}_1$, ${\bf v}=-{\bf e}_1+2{\bf e}_2$ and ${\bf w}=-{\bf e}_2+2{\bf e}_3$. 
For the sake of simplicity, throughout the proof, we use $\chi$ instead of $\chi_3$. For any $k\in  \mathbb{Z}$, we use $P_k$  to denote the plane parallel to $xy$-plane with $z$-coordinate value $k$. 
The projections of  planes  $P_{2k}, P_{2k+1},  \ldots, P_{2k+2}$  over a rectangular region are depicted in Fig.~\ref{fig:l1}, Fig.~\ref{fig:l2} and  Fig.~\ref{fig:l3}, respectively. 
Note that $P_{2k+1}\cap\chi=\phi$. Also, observe that $P_{2k}\cap\chi$ and $P_{2k+2}\cap\chi$ are translated copy of each other by 1 unit in $y$-coordinate.
$\BPA$ maintains a hitting set $\A$ consisting of points from $\chi$. On receiving a new input unit cube $\sigma$, if it is not hit by any of the points from $\A$ then the algorithm adds the best-point from $\chi$ lying inside $\sigma$ to the set $\A$. Correctness of the algorithm follows from Lemma~\ref{lem:correct_hyp}

Let  $\I$ be the set of input unit cubes  presented to the algorithm.
Let $\OO$ be an optimal offline hitting set for $\I$.  
 Let $\A'=\A\setminus \{\A\cap \OO\}$ and  $\OO'=\OO\setminus \{\A\cap \OO\}$. 
Let $p \in \OO'$ be an integer point and let $\I_p\subseteq \I$ be the set of input unit cubes  containing the point $p$.
 Let $\A_{p}\subseteq \A'$ be the set of points our algorithm will place to hit explicitly when some unit cube in $\I_p$ arrives. In the following lemma, we prove that the cardinality of $\A_p$ is bounded by~8.
 Since $\A'=\cup_{p\in \OO'}\A_p$, we have $|\A'|\leq\sum_{p\in \OO'}|\A_p|\leq 8\times |\OO'|$. Note that $\frac{|\A'| }{|\OO'|}\leq 8$ implies $\frac{|\A|}{|\OO|}\leq 8$.
Thus, the competitive ratio of our algorithm is at most~8.
\myqed
\end{proof}
\begin{lemma}\label{cube}
$|\A_p|\leq 8$.
\end{lemma}
\begin{proof}
 % Let $H_3(p,1)$ be a unit cube centered at $p$.
Observe that the center of each $\sigma \in \I_p$ lies in the region $H_3(p,1)$, and to hit unit cubes of $\I_p$, our algorithm places integer points from $\chi(H_3(p,2))$.
 %where $H_3(p,2)$ is a cube of radius two-unit centered at $p$. 
 Therefore, $\A_p$ contains points from $\chi(H_3(p,2))$.
Let $p(z)$ be the z-coordinate value of the point $p$.
 Note that the cube $H_3(p,2)$ contains integer points only from five planes, 
 namely, $P_{p(z)+2},P_{p(z)+1},P_{p(z)},P_{p(z)-1}$ and $P_{p(z)-2}$. 
 As per the definition of $\chi$, if $p(z)$ is odd then $P_{p(z)+2}\cap\chi,\ P_{p(z)}\cap\chi,\ P_{p(z)-2}\cap\chi$ are empty, otherwise $P_{p(z)+1}\cap\chi$, $P_{p(z)-1}\cap\chi$ are empty.
 % If the cube $H_3(p,2)$ have three planes ($P_{p(z)+2},P_{p(z)}$ and $P_{p(z)-2}$) containing integer points from $\chi$, then layers $P_{p(z)-2}$ and $P_{p(z)+2}$ are {symmetric}.
 
% { Note that the cube $H_3(p,1)$ contains integer points only from three  planes $P_{p(z)+1},P_{p(z)}$ and $P_{p(z)-1}$. It is easy to observe that the unit cube $H_3(p,1)$ contains at least  one and at most five points of $\chi$. As a result, we have the following five cases.}\\
%  As per definition of $\chi$, consecutive three plane have either only one plane containing points of $\chi or two plane s are containing points of $\chi$. As a result, any unit cube contains at least one and  at most five points of $\chi$.





\begin{figure}[htbp]
          \centering
          \hfill
     \begin{subfigure}[b]{0.24\textwidth}
         \centering
         \includegraphics[width=28 mm]{Figures/lambda.pdf}
         \caption{}
         \label{fig:lambda'}
     \end{subfigure}
     \hfill
          \begin{subfigure}[b]{0.24\textwidth}
         \centering
    \includegraphics[width=28 mm]{Figures/Layer1.pdf}
    \caption{}
    \label{fig:l1}
     \end{subfigure}
     \hfill
     \begin{subfigure}[b]{0.24\textwidth}
         \centering
         \includegraphics[width=28 mm]{Figures/Layer2.pdf}
         \caption{}
         \label{fig:l2}
     \end{subfigure}
       \begin{subfigure}[b]{0.24\textwidth}
         \centering
         \includegraphics[width=28 mm]{Figures/Layer3.pdf}
         \caption{}
         \label{fig:l3}
     \end{subfigure}
     \caption{(a) The points of $\Lambda$ and $\Lambda'$ are represented in black and red color, respectively. {The projections of  planes  over a rectangular region (b) $P_{2k}$, (c) $P_{2k+1}$ and (d) $P_{2k+2}$.}}
   \label{fig:plane_part}
       \end{figure}
    
      \begin{figure}[htbp]
     \centering
     \hfill
     \begin{subfigure}[b]{0.33\textwidth}
         \centering
    \includegraphics[width=32 mm]{Figures/Hyp_case_1.pdf}
    \caption{}
    \label{case-1_1}
     \end{subfigure}
     \hfill
     \begin{subfigure}[b]{0.33\textwidth}
         \centering
         \includegraphics[width=32 mm]{Figures/Hyp_case_2.pdf}
         \caption{}
         \label{case-2_1}
     \end{subfigure}
     \hfill
       \begin{subfigure}[b]{0.32\textwidth}
         \centering
         \includegraphics[width=32 mm]{Figures/Hyp_case_3.pdf}
         \caption{}
         \label{case-3_1}
     \end{subfigure}
     
    
         \caption{Illustration of Case~1: Here boundaries of $H_3(p,1)$ and $H_3(p,2)$ are marked in red and blue colors, respectively, (a) Case~1.1, (b) Case~1.2 and (c) Case~1.3.
         }
   \label{fig:hq_ub_main}
\end{figure}



      \begin{figure}[htbp]
     \centering
    \hfill
     \begin{subfigure}[b]{0.49\textwidth}
         \centering
    \includegraphics[width=32 mm]{Figures/Hyp_case_4.pdf}
    \caption{}
   \label{case-4_2}
     \end{subfigure}
     \hfill
     \begin{subfigure}[b]{0.5\textwidth}
         \centering
         \includegraphics[width=32 mm]{Figures/Hyp_Case_5.pdf}
         \caption{}
         \label{case-5_2}
     \end{subfigure}
         \caption{Illustration of Case~2: Here boundaries of $H_3(p,1)$ and $H_3(p,2)$ are marked in red and blue color, respectively, (a) Case~2.1 and (b) Case~2.2.}
   \label{fig:hq_ub_main_b}
\end{figure}
\noindent
\textbf{Case 1 :} $p(z)$ is even. In this case, $|\chi(H_3(p,1))|$ is either 1,2 or 3. Depending on the value of $|\chi(H_3(p,1))|$, we have the following three subcases.\\ 
\textbf{Case 1.1 :} $|\chi(H_3(p,1))|=1$. Observe that, in this case, $p\in\chi$.
% Representative figures of five planes $P_{p(z)-2}$, $P_{p(z)-1}$, $P_{p(z)}$, $P_{p(z)+1}$ and $P_{p(z)+2}$ intersecting the cube $H_3(p,2)$ are  shown in Fig.~\ref{case-1_1}, Fig.~\ref{fig:l2}, Fig.~\ref{case-1_3},Fig.~\ref{fig:l2} and Fig.~\ref{case-1_5}, respectively.
Representative figures of planes  $P_{p(z)-2}$, $P_{p(z)}$ and $P_{p(z)+2}$ intersecting the cube $H_3(p,2)$ are  shown in Fig.~\ref{case-1_1}.
Observe that  $H_3(p,2)$ contains 17 integer points of $\chi$ including $p$. As per the definition of $\A_p$, we know $p\notin \A_p$. 
Remember that any unit cube $\sigma\in\I_p$  contains  the point $p$. For  any point $p'\in P_{p(z)-2}$, we have $p'\prec p$.  Thus, our algorithm does not add $p'$ to $\A_p$. As a result, none of the points of $P_{p(z)-2}$ are in $\A_p$.
Similarly, it is easy to see $p_4,p_5,p_6\prec p$.  Thus, points $p_4,p_5, p_6 \notin\A_p$. 
Now, consider  any unit cube $\sigma_1\in\I_p$ that contains the point $p_7$. Note that $\sigma_1$ must also contain $p_8$, and $p_7\prec p_8$. Therefore, our algorithm does not add $p_7$ to $\A_p$ upon the arrival of $\sigma_1$.  Putting all these together, all five points from $P_{p(z)-2}$, $\{p,p_4,p_5,p_6\}$ from $P_{p(z)}$ and $p_{7}$ from $P_{p(z)+2}$ are not in $\A_p$.  Therefore, we have $|\A_p|\leq 7$.
\\
\textbf{Case 1.2 :} $|\chi(H_3(p,1))|=2$. In this case, it is easy to observe that the plane $P_{p(z)}$  contains both the  points of $\chi(H_3(p,1))$.
Representative figures of planes $P_{p(z)-2}$, $P_{p(z)}$ and $P_{p(z)+2}$ intersecting the cube $H_3(p,2)$ are  depicted in Fig.~\ref{case-2_1}. It is easy to see that the cube  $H_3(p,2)$ contains 18 integer points of $\chi$.
 Notice that any unit cube $\sigma\in\I_p$ that contains $p$ must also contain either $p_4,p_5$ or both from the plane $P_{p(z)}$. Observe that for  any point $p'\in P_{p(z)-2}$, we have $p'\prec p_4$ and $p'\prec p_5$.  As a result, none of the points from the plane $P_{p(z)-2}$ are in $\A_p$.
Similarly, since $p_6,p_7\prec p_4,p_5$, we know that $p_6, p_7 \notin\A_p$.
 Observe that any unit cube  that contains $p$ and $p_1$ (resp., $p_8$) must also contain the point $p_2$ (resp. $p_4$). Since $p_1\prec p_2$ and $p_8\prec p_4$, thus, $p_1, p_8 \notin\A_p$. 
 Similarly, any unit cube that contains the point $p$ and $p_{13}$, must also contain $p_{12}$, and we have $p_{13}\prec p_{12}$. Thus, $p_{13}\notin\A_p$. 
Combining all of these, we know that  all five points from $P_{p(z)-2}$, $\{p_1,p_6,p_7,p_8\}$ from $P_{p(z)}$ and $p_{13}$ from $P_{p(z)+2}$ are not in $\A_p$. Hence, we have $|\A_p|\leq 8$. \\
\noindent
%  \textbf{(2.2):} 
\textbf{Case 1.3:} $|\chi(H_3(p,1))|=3$. 
Planes $P_{p(z)-2}$, $P_{p(z)}$ and $P_{p(z)+2}$  intersecting the cube $H_3(p,2)$ are depicted in Fig.~\ref{case-3_1}. 
Observe that the cube $H_3(p,2)$ contains only 19 integer points of $\chi$.
Notice that any unit cube that contains $p$ must contain either $p_1,p_2$ or $p_4$ from $p_{p(z)}$. On the other hand, for any point $p'\in P_{p(z)-2}$,  we know that $p'\prec p_1,p_2,p_4$. Thus, our algorithm does not add any of the points from the plane $P_{p(z)-2}$ to $\A_p$. Any unit cube that contains $p$ and $p_5$ must also contain $p_4$, and we know that $p_5\prec p_4$.  Therefore, $p_5 \notin \A_p$. Now, observe that any unit cube that contains $p$ and  some point from $P_{p(z)+2}$ must also contain the point $p_9$. Since $p_{10},p_{11}, p_{12} \prec p_9$, therefore, $p_{10},p_{11}, p_{12} \notin \A_p$. 
After putting all of these together, we know that all seven points from $P_{p(z)-2}$, $p_5$ from $P_{p(z)}$ and $\{p_{10},p_{11},p_{12}\}$ from $P_{p(z)+2}$ are not in $\A_p$. Thus, we have $|\A_p|\leq 8$. \\
%  \textbf{(2.3):}

\noindent
\textbf{Case 2 :} $p(z)$ is odd. In this case  $|\chi(H_3(p,1))|$ is either 4 or 5. Depending on the value of $|\chi(H_3(p,1))|$, we have the following two subcases.\\
\textbf{Case 2.1:} $|\chi(H_3(p,1))|=4$. 
Representative figures of planes  $P_{p(z)-1}$ and $P_{p(z)+1}$ intersecting the cube $H_3(p,2)$ are  depicted in Fig.~\ref{case-4_2}.
Observe that the cube $H_3(p,2)$ contains only 12 integer points of $\chi$. 
Observe that any unit cube that contains $p$ and 
any of the points in $\{p_5,p_6,p_7\}$ must also contain  $p_4$, and we know that  $p_5,p_6,p_7\prec p_4$. Similarly, any unit cube that contains $p$ and $p_8$ must also contain $p_9$, and we have $p_8\prec p_9$. As a result,  $p_5,p_6,p_7$ and $p_{8}$  are not in $\A_p$. Hence, $|\A_p|\leq 8$. \\
%%%%%%%%------------Full Proof----------%%%%%%%%%%%%
% At first, we show that none of the points $p_5,p_6,p_7$ are in $\A_p$.
% Observe that any unit cube $\sigma\in\I_p$ that contains some point of $P_{p(z)-1}$ must also contain $p_4$. Since $p_5,p_6,p_7\prec p_4$, our algorithm does not add $p_5$ to $\A_p$ upon the arrival of $\sigma$. Hence, $|\A_p|\leq 9$. 
% Now, we show that the point $p_7$ from $P_{p(z)+1}$ is not in $\A_p$. Note that any unit cube $\sigma'\in\I_p$ that contains the point $p_7$ must also contain $p_8$, and $p_7\prec p_8$, therefore, our algorithm does not add $p_4$ to $\A_p$ upon the arrival of $\sigma'$.
% Hence, $|\A_p|\leq 8$.\\
\textbf{Case 2.2:} $|\chi(H_3(p,1))|=5$. 
% Note that the cube $H_3(p,2)$ comprises integer points from the five layers $P_{p(z)-2}$, $P_{p(z)-1}$, $P_{p(z)}$, $P_{p(z)+1}$ and $P_{p(z)+2}$ as shown in Fig.~\ref{fig:l2}, Fig.~\ref{case-5_2},Fig.~\ref{fig:l2}, Fig.~\ref{case-5_4} and Fig.~\ref{fig:l2}, respectively.
Planes $P_{p(z)-1}$ and $P_{p(z)+1}$ intersecting the cube $H_3(p,2)$ are shown in Fig.~\ref{case-5_2}. 
The cube $H_3(p,2)$ contains 13 integer points of $\chi$.
Similar to Case 1.2, one can observe that none of the points $p_1,p_6,p_7,p_8$ from $P_{p(z)-1}$ and $p_{13}$ from $P_{p(z)+1}$ are in $\A_p$. As a result, $|\A_p|\leq 8$. 
\myqed\end{proof}






\begin{theorem}\label{square_ub}
For hitting unit squares using points from $\mathbb{Z}^2$, there exists a deterministic online algorithm that achieves a competitive ratio of at most~4.
\end{theorem}
\begin{proof}

The proof is similar to Theorem~\ref{cube_ub}. Here,
the integer lattice $\Lambda=\{q {\bf e}_1+r {\bf e}_2| q,r\in \mathbb{Z}\}$ in $\IR^2$ is  generated by standard unit vectors ${\bf e}_1$ and ${\bf e}_2$.
Consider a subset $\chi\subset\Lambda$ defined as follows: $\chi=\{q {\bf u}+r {\bf v} |  q,r\in \mathbb{Z}\}$, where ${\bf u}=2{\bf e}_1$ and ${\bf v}=-{\bf e}_1+2{\bf e}_2$.  Next, we prove that $|\A_p|\leq 4$ in this case. Hence, the competitive ratio of $\BPA$ is at most 4.

Note that a unit square centered at an integer point contains at least one and at most 3 points of $\chi$ (see plane $P_{p(z)}$ in Fig.~\ref{case-1_1},~\ref{case-2_1} and~\ref{case-3_1}). Hence, we have the following three cases \\
\textbf{Case 1:} $|\chi(H_2(p,1))|=1$. Note that $H_2(p,2)$ contains 7 integer points of $\chi$ including $p$ (see plane $P_{p(z)}$ in Fig.~\ref{case-1_1}). With the similar argument of Case 1.1 of Lemma~\ref{cube},   one can easily notice that none of the points $\{p_4,p_5,p_6\}$ are in $\A_p$.
As a result $|\A_p|\leq 4$.\\
\textbf{Case 2:} $|{\cal Q}(H_2(p,1))\cap\chi|=2$. Observe that $H_2(p,2)$ contains only eight integer points of $\chi$ (see plane $P_{p(z)}$ in Fig.~\ref{case-2_1}). With the similar argument of Case 1.2 of Lemma~\ref{cube}, it is easy to observe that none of the points $\{p_1,p_6,p_7,p_8\}$ are in $\A_p$. As a result, $|\A_p|\leq 4$.\\
\textbf{Case 3:} $|{\cal Q}(H_2(p,1))\cap\chi|=3$. Note that $H_2(p,2)$ contains only five integer points of $\chi$ (see plane $P_{p(z)}$ in Fig.~\ref{case-3_1}). With the similar argument of Case 1.3 of Lemma~\ref{cube}, one can see that $p_5$ is not in $\A_p$.
% Note that a unit square $\sigma\in\I_p$ that contains the point $p_5$ will also contain the point $p_4$, therefore, $p_5$ is not the best-point and will not be added to $\A_p$ by our algorithm.
Hence, $|\A_p|\leq4$.
\myqed
% Thus, we have the following theorem:
\end{proof}







\subsection{Unit Balls in $\IR^2$ and $\IR^3$}
Let $\Lambda_d=\{\alpha_1 {\bf e}_1+\alpha_2 {\bf e}_2+\ldots+\alpha_d {\bf e}_d\ |\ \alpha_i\in \mathbb{Z},\ \forall\ i\in[d]\}$   be the integer lattice in $\IR^d$ generated by standard unit vectors ${\bf e}_1$, ${\bf e}_2\ldots,{\bf e}_d$. Consider a subset $\chi_d\subset \Lambda_d$ defined as follows:\\ $\chi_d=\{\alpha_1 {\bf u}_1+\alpha_2 {\bf u}_2+\ldots+\alpha_d {\bf u}_d\ |\ \alpha_i\in \mathbb{Z},\ \forall\ i\in[d]\}$. Here, for $d\leq 4$ we have
\[
 {\bf u}_i=
\begin{cases}
 2{\bf e}_1, & \text{for } i=1\\
   -{\bf e}_{i-1}+{\bf e}_i,& \text{for } i\in[d]\setminus\{1\}.
  
\end{cases}
\]



\begin{lemma}\label{lem:correct}
    For $d\leq 4$, each unit ball  $B_d(r,1)$ centered at any point $r \in \IR^d$ contains at least one point of $\chi_d$.
\end{lemma}
\begin{proof}
As per the definition of $\chi_d$, precisely one among every two consecutive integer points belongs to the set $\chi_d$. To prove the lemma, it is sufficient to prove that, a unit ball $B_d(r,1)\subset \IR^d$ centered at any point $r\in \IR^d$ contains at least  two consecutive integer points. Note that any real number $x\in\IR$ can be expressed as $x=y+z$, where $z\in\mathbb{Z}$ and $y\in\left(\frac{-1}{2},\frac{1}{2}\right]$.
Let $r=(z(x_1)+y(x_1),z(x_2)+y(x_2),\ldots,z(x_d)+y(x_d))$ and $z=(z(x_1),z(x_2),\ldots,z(x_d))$. To show that a point $p\in\IR^d$ belongs to the unit ball $B_d(r,1)$, we need to show that the $dist(r,p)\leq 1$. Now, consider the square of the distance between $r$ and $z$ as follows 

\small
\begin{align*}
dist^2(r,z)=&\sum_{i=1}^d\left(z(x_i)-r(x_i)\right)^2\\
=&(z(x_1)-(z(x_1)+y(x_1)))^2+(z(x_2)-(z(x_2)+y(x_2)))^2+\ldots+(z(x_d)-(z(x_d)+y(x_d)))^2\\
=& y(x_1)^2+y(x_2)^2+\ldots+y(x_d)^2\\
 \leq&\ d\left(\frac{1}{2}\right)^2\\
 \leq&1.
\end{align*}  
% Since $x_i\leq\frac{1}{2}$,  we have
\normalsize
The last inequality follows due to the fact that $d\leq 4$.
    %   Let $x'=\max\{|x_i|:i\in[3]\}$. 
Let $t\in[d]$ be an index such that $|y(x_t)|=\max\{|y(x_i)|:i\in[d]\text{ and } d\leq4 \}$.
       Let $z' \in \mathbb{Z}^d$ be an integer point such that
      \[
      z'(x_i)=
      \begin{cases}
     z(x_i)+1,& \text{ if $i=t$} \\
      z(x_i),& \text{ otherwise}. 
      \end{cases}
      \]
       % To prove that the ball $B_d(r,1)$ contains the point $z'$, again we need to show that $\sum_{i=1}^d\left(z_i'-r_i\right)^2\leq1$, for $d=2$ and 3.
     % Since $|x_t|=\max\{|x_i|:i\in[d]\}$, $d\in[3]\setminus\{1\}$  and $x_t\leq\frac{1}{2}$, we have $x_1^2+x_2^2+\ldots+x_d^2\leq\ dx_t^2\leq \frac{d}{2}x_t \leq 2x_t$. 
      Now, consider the square of the distance between $z'$ and $r$ as follows
\small
\begin{align*}
dist^2(r,z')=&\sum_{i=1}^d\left(z'(x_i)-r(x_i)\right)^2\\
=&((z(x_t)+1)-(z(x_t)+y(x_t)))^2\ +\sum_{i\in[d]\setminus\{t\}}\left(z(x_i)-(z(x_i)+y(x_i))\right)^2\\
=&\ 1-2y(x_t)+y(x_1)^2+y(x_2)^2+\ldots+y(x_d)^2\\
\leq&\ 1-2y(x_t)+2y(x_t)=1.
\end{align*} 
\normalsize
 Here, the last inequality follows due to the following: since $|y(x_t)|=\max\{|y(x_i)|:i\in[d]\}$, $d\in[4]$  and $y(x_t)\leq\frac{1}{2}$, we have $y(x_1)^2+y(x_2)^2+\ldots+y(x_d)^2\leq\ d \left(y(x_t)^2\right)\leq \frac{d}{2}y(x_t) \leq 2y(x_t)$. 
Note that the distance of $r$ from both integer points $z$ and $z'$ is less than or equal to 1. Since $z$ and $z'$ are consecutive integer points, one of them must belong to $\chi_d$ and the ball $B_d(r,1)$ contains at least one integer point of $\chi_d$.
\myqed
\end{proof}
\begin{theorem}\label{3d-ball}
    For hitting unit balls using points in $\mathbb{Z}^3$, there exists a deterministic online algorithm that achieves a competitive ratio of at most~14.
\end{theorem}


\begin{figure}[htbp]
	\centering
\hfill
\begin{subfigure}[b]{0.45\textwidth}
	\centering
	\includegraphics[width=30 mm]{Figures/Ball_case_1.pdf}
	\caption{}
	\label{fig:ib_(-2)}
\end{subfigure}
\begin{subfigure}[b]{0.45\textwidth}
	\centering
	\includegraphics[width=23.7 mm]{Figures/Ball_case_2.pdf}
	\caption{}
	\label{fig:nb_(-2)}
\end{subfigure}
 \caption{Illustration of Theorem~\ref{3d-ball}. Here the boundary of balls $B_3(p,1)$ and $B_3(p,2)$ are represented with red and blue color, respectively, (a) Case~1.1 and (b) Case~1.2.}
   \label{fig:ball}
\end{figure}





\begin{proof}
For the sake of simplicity, throughout the proof, we use $\chi$ to represent $\chi_3$.
For any $k\in  \mathbb{Z}$, we use $P_k$  to denote the plane parallel to $xy$-plane with $z$-coordinate value $k$. The projections of  planes  $P_{k}, P_{k+1},  \ldots, P_{k+4}$  over a rectangular region are depicted in Fig.~\ref{fig:ib_(-2)}. 
Observe that $P_{k}\cap\chi$ and $P_{k+1}\cap\chi$ are translated copies of each other by 1 unit of $y$-coordinate.
$\BPA$ maintains a hitting set $\A$ consisting of points of $\chi$. On receiving a new input unit ball $\sigma$, if it is not hit by any of the points from $\A$ then the algorithm adds the best-point of $\chi$ lying inside $\sigma$ to the set $\A$.
{Correctness of the algorithm follows from Lemma~\ref{lem:correct}}.

Let $\I$ be the set of input balls presented to the algorithm.
Let $\OO$ be an optimal offline hitting set for $\I$.  
 Let {$\A'=\A\setminus \{\A\cap \OO\}$ and  $\OO'=\OO\setminus \{\A\cap \OO\}$}. 
Let $p \in \OO'$, and let $\I_p\subseteq \I$ be the set of input balls containing the point $p$.
 Let $\A_{p}\subseteq \A'$ be the set of hitting points placed by our algorithm to hit explicitly when some ball in $\I_p$ arrives.
  In the following lemma, we prove that the cardinality of
$\A_p$ is bounded by 14. 
Since $\A'=\cup_{p\in \OO'}\A_p$, we have $|\A'|\leq\sum_{p\in \OO'}|\A_p|\leq 14\times |\OO'|$. Note that $\frac{|\A'| }{|\OO'|}\leq 14$ implies $\frac{|\A|}{|\OO|}\leq 14$.
Thus, the competitive ratio of our algorithm is at most~14.
\myqed
\end{proof}
 \begin{lemma}\label{ball_3d}
 $|\A_p|\leq 14$.
 \end{lemma}
 \begin{proof}
 Observe that the center of each $\sigma \in \I_p$ lies in the region $B_3(p,1)$, and to hit balls of $\I_p$, our algorithm places integer points from $\chi(B_3(p,2))$.
 Therefore, $\A_p$ contains points from $\chi(B_3(p,2))$.
Let $p(z)$ be the z-coordinate value of the point $p$.
 Note that the ball $B_3(p,2)$ contains integer points only from five planes, 
 namely, $P_{p(z)+2},P_{p(z)+1},P_{p(z)},P_{p(z)-1}$ and $P_{p(z)-2}$. 
 As per the definition of $\chi$, we know that $P_{p(z)+2}\cap\chi$,  $P_{p(z)}\cap\chi$ and $P_{p(z)-2}\cap\chi$ are same. Similarly,  $P_{p(z)+1}\cap\chi$ is same as $P_{p(z)-1}\cap\chi$. 
%  Any ball $B(p,1)$ contains integer points only from three planes $P_{p(z)-1},P_{p(z)}$ and $P_{p(z)+1}$.
Observe that, if the center of the unit ball $B_3(p,1)$ coincides with some point of $\chi$, then the ball contains only one point of $\chi$ (see planes $P_{p(z)-1}, P_{p(z)}$ and $P_{p(z)+1}$ in Fig.~\ref{fig:ib_(-2)}); otherwise, it contains six points of $\chi$ (see planes $P_{p(z)-1}, P_{p(z)}$ and $P_{p(z)+1}$ in Fig.~\ref{fig:nb_(-2)}). As a result, we have the following two cases.



\noindent
\textbf{Case 1:} 
$|\chi(B_3(p,1))|=1$. In this case, $p\in\chi$.
Representative figures of five planes $P_{p(z)-2}$, $P_{p(z)-1}$, $P_{p(z)}$, $P_{p(z)+1}$ and $P_{p(z)+2}$ intersecting the ball $B_3(p,2)$ are  shown in Fig.~\ref{fig:ib_(-2)}.
Observe that  $B_3(p,2)$ contains 19 integer points of $\chi$ including $p$. As per the definition of $\A_p$, we know $p\notin \A_p$.
 Now, we show that none of the points from plane $P_{p(z)-1}$ and $P_{p(z)-2}$ are in $\A_p$. Here,  we want to remind the reader  that any unit ball $\sigma\in\I_p$  contains  the point $p$. For any point $p'\in\chi\cap P_{p(z)-2}$, we have $p'\prec p$, thus our algorithm does not add $p'$ to $\A_p$. Hence, $|\A_p|\leq 13$.
\\
\textbf{Case 2:}  $|\chi(B_3(p,1))|=6$. Notice that, in this case, $p\notin\chi$. Representative figures of five planes $P_{p(z)-2}$, $P_{p(z)-1}$, $P_{p(z)}$, $P_{p(z)+1}$ and $P_{p(z)+2}$ intersecting the ball $B_3(p,2)$ are  shown in Fig.~\ref{fig:nb_(-2)}.
Observe that the ball $B_3(p,2)$ contains only fourteen integer points of $\chi$. Hence, $|\A_p|\leq 14$.\myqed
\end{proof}



\begin{theorem}\label{2d-balls}
    For hitting unit disks using points in $\mathbb{Z}^2$, there exists a deterministic online algorithm that achieves a competitive ratio of at most~4.
\end{theorem}
\begin{proof}

The proof is similar to Theorem~\ref{3d-ball}. Here,
the integer lattice $\Lambda=\{q {\bf e}_1+r {\bf e}_2 | q,r\in \mathbb{Z}\}$ in $\IR^2$ is  generated by standard unit vectors ${\bf e}_1$ and ${\bf e}_2$ and the $\chi\subset \Lambda$ is defined as $\chi=\{q {\bf u}+r {\bf v} |  q,r\in \mathbb{Z}\}$, where ${\bf u}=2{\bf e}_1$ and ${\bf v}=-{\bf e}_1+{\bf e}_2$. Here, we prove that $|\A_p|\leq 4$. Thus, the competitive ratio $\BPA$ is at most 4.
 Observe that, if the center of the ball $B_3(p,1)$ coincides with some point in $\chi$, then $B_3(p,1)$ contains only 1 point of $\chi$ (see plane $P_{p(z)}$ in Fig.~\ref{fig:ib_(-2)}); otherwise, it contains 4 points of $\chi$ (see plane $P_{p(z)}$ in Fig.~\ref{fig:nb_(-2)}). Similar to Lemma~\ref{ball_3d}, we have two cases.\\
\textbf{Case 1:} $p\in\chi$. Note that $B_2(p,2)$ contains 9 integer points of $\chi$ (see plane $P_{p(z)}$ in Fig.\ref{fig:ib_(-2)}). As per the definition of $\A_p$, we know $p\notin \A_p$. Let us consider a unit disk $\sigma\in\I_p$ that contains the point $p_1$. Since $p_1\prec p$, our algorithm does not add $p_1$ to $A_p$ upon the arrival of $\sigma$. In a similar way, one can observe that none of the points $\{p_2,p_6,p_7,p_8\}$ are in $\A_p$. As a result $|\A_p|\leq 3$.\\
\textbf{Case 2:} $p\notin \chi$. Observe that $B_2(p,2)$ contains only four integer points of $\chi$ (see plane $P_{p(z)}$ in Fig.\ref{fig:nb_(-2)}). Hence, $|\A_p|\leq4.$

% Thus, we have the following theorem 
\myqed
\end{proof}








\section{Unit Hypercubes in $\mathbb{R}^d$}\label{sec:hyp}

%\subsection{Intervals}\label{unit}





% %%%%%%% Will Keep in the full version%%%%%%



% %%%%%%% Will Keep in the full version%%%%%%





In this section, we present the lower and upper bound of the competitive ratio for hitting unit hypercubes in $\IR^d$ using points from $\mathbb{Z}^d$. 
\subsection{Lower Bound}
In this subsection, we present the lower bound of the competitive ratio for hitting unit hypercubes in $\IR^d$.

\begin{theorem}\label{hyp_lb}
The  competitive ratio of every deterministic online algorithm  for hitting hypercubes in $\mathbb{R}^d$ using points in $\mathbb{Z}^d$ is at least~$d+1$, where $d\in\mathbb{N}$.
\end{theorem}
\begin{proof}
 Let us consider a game between  two players: Alice and Bob. Here, Alice plays the role of an adversary, and Bob plays the role of an online algorithm. In each round of the game, Alice presents a new unit hypercube $\sigma\subset\mathbb{R}^d$ such that Bob needs to hit it by a new hitting point $h\in \mathbb{Z}^d$.
To prove the lower bound of the competitive ratio, we show by induction that Alice can  present a sequence of unit hypercubes  $\sigma_1,\sigma_2,\ldots,\sigma_{d+1}\subset \mathbb{R}^d$ adaptively, depending on the position of hitting points placed by Bob such that Bob needs to place $d+1$ integer points $\{h_1,h_2,\ldots,h_{d+1}\}$; whereas the offline optimum needs just one integer point.
Let $\sigma_{1}$ be a hypercube presented by Alice in the first round of the game.
For the sake of simplicity, we assume that the center of $\sigma_1$ is the origin.
 For $i=1,\ldots, {d+1}$, we maintain the following two invariants:
%  when Alice presents  hypercubes  $\sigma_1,\ldots, \sigma_i$ and Bob presents hitting points $h_1, \ldots, h_{i-1}$, 
\begin{itemize}
    \item The hypercube $\sigma_i\subset \mathbb{R}^d$ does not contain any of the previously placed hitting point $h_j\in \mathbb{Z}^d$, for 
$j<i$.
    \item The common intersection region $Q_i=\cap_{j=1}^{i}\sigma_j$  contains  $3^{(d-i+1)}$ integer points. 
    % All the hypercubes presented can be hit by one integer point i.e., $\cap_{i=1}{d+1} \sigma_i$ contains at least one integer point.
\end{itemize}

%  Let $x_1,\ldots,x_d$ be the $d$ coordinate axes in $\mathbb{R}^d$; 
%  and $x_{d+1}$ be the new axis in $\mathbb{R}^{d+1}$.
  For $i=1$, the first invariant trivially holds. Since the unit hypercube $\sigma_1$ is centered at the origin, each coordinate of any integer point $p\in \sigma_1$ has three possible values from $\{-1,0,1\}$. As a result, the unit hypercube $\sigma_1$ contains $3^d$ integer points. Thus, the second invariant also holds.
  
  At the beginning of the round $i$ (for $i=2,\ldots,d$), let us assume that both invariants hold. Now, we only need to show that the induction is true for $i=d+1$. Let us define a translation vector ${\bf v}_i\in \mathbb{R}^{d}$ as follows:
${\bf v}_i=(s(1)(1+\epsilon),s(2)(1+\epsilon),\ldots,s(i-1)(1+\epsilon),0,\ldots,0)$, where $0<\epsilon<\frac{1}{2}$ is an arbitrary constant, and for any $j< i$, we have
  \[
  s(j)= 
\begin{cases}
    +1,& \text{if}\ h_{j}(x_j)\leq 0,\text{ where}\ h_{j}(x_j)\text{ is $jth$ coordinate of $h_{j}$}, \\
    -1,              & \text{otherwise.}
\end{cases}
\]
We define the hypercube $\sigma_i=\sigma_1 +{\bf v}_i$. For any  $j<i$, due to the definition of the $j$th component of the translation vector ${\bf v}_i$, the hypercube $\sigma_i$ does not contain the point $h_j$. Hence, the first invariant is maintained. 
 Let us count the number of integer points contained in $\sigma_1\cap\sigma_i$. Consider any integer point $p\in \sigma_1\cap\sigma_i$. 
Since $\sigma_i=\sigma_1 +{\bf v}_i$ and $\sigma_1$ is centered at the origin, thus, $\sigma_i$ is centered at ${\bf v}_i$.
As a result, for any $j\in[i-1]$,  the $j$th coordinate of  $p$ is fixed at $s(j)$. The value of each of the remaining $(d-i+1)$ coordinates of $p$ has three possibilities from $\{-1,0,1\}$. Therefore, $\sigma_1\cap\sigma_i$  contains $3^{(d-i+1)}$ integer points. Because of the above argument, observe that all the integer points that belong to $\sigma_1\cap\sigma_i$ are also contained in  $\sigma_1\cap\sigma_j$, where $j<i$. Hence, $Q_i$ contains $3^{(d-i+1)}$ integer points.
\myqed
\end{proof}




\subsection{Upper Bound for $d\geq 3$} \label{hypercube_upperb}

% \paragraph{\textbf{Structural Properties:}}
We first present some concepts that we utilize to analyze the algorithm we propose for unit hypercubes in $\mathbb{R}^d$, where $d\geq 3$. Let $\cal F$ be the family of all possible unit hypercubes in $\IR^d$.
Any pair of unit hypercubes $\sigma_i$ and $\sigma_j$ in  $\cal F$ are said to be \emph{related} if $\Q(\sigma_i)=\Q(\sigma_j)$, in other words, each of them contains the same set of integer points. So, we have an equivalence relation on $\cal F$ where each \emph{equivalence class} corresponds to a set $S$ of unit hypercubes such that each $\sigma\in S$ contains the same set of integer points. 



Let $\sigma$ be a unit hypercube centered at a point $c\in\IR^d$. Partition $[d]$ into two sets $\K_1$ and $\K_2$ such that 
 for each $i\in\K_1$, the value of $c(x_i)$ is non-integer and for each $i\in\K_2$, the value of $c(x_i)$ is integer. Let $r\in{\cal Q}(\sigma)$ be  an integer point. For any $i\in\K_1$, the value of $r(x_i)$ can be one from the two possible values: $\{\lfloor c(x_i) \rfloor,\lceil c(x_i) \rceil\}$, and for any $i\in\K_2$, the value of $r(x_i)$ can be one from the three possible values $\{c(x_i)-1,c(x_i),c(x_i)+1\}$ when $i\in\K_2$.  Hence, ${\cal Q}(\sigma)$ contains exactly $2^{|\K_1|}3^{|\K_2|}$ integer points. 
The following lemma is an important ingredient for classifying the equivalence classes.
\begin{lemma}\label{Lemma_1}
Let $\sigma_1$ and $\sigma_2$ be two  unit hypercubes centered at $c_1$ and $c_2$ in $\IR^d$, respectively. Both $\sigma_1$ and $\sigma_2$ contain the same set of integer points if and only if 
$[d]$ can be partitioned into two sets $\K_1$ and $\K_2$
 such that 
\begin{itemize}
\item  
for each $i\in \K_1$, the value of the $i$th coordinate of $c_1$ and $c_2$ is non-integer 
and $\lfloor c_1(x_i)\rfloor=\lfloor c_2(x_i)\rfloor$,

\item for each $i\in\K_2$, the value of 
 the $i$th coordinate  of  $c_1$ and $c_2$ is same,  i.e., $c_1(x_i)=c_2(x_i)$.
\end{itemize}
\end{lemma}
% \subsection{Proof of Lemma~\ref{Lemma_1}}\label{app_Lemma_1}
\begin{proof}
For the forward direction, we prove the contrapositive statement:  ``If there exists some $i\in[d]$ such that $\lfloor c_1(x_i)\rfloor\neq\lfloor c_2(x_i)\rfloor$, then ${\cal Q}(\sigma_1)\neq{\cal Q}(\sigma_2)$". 
Since  $\lfloor c_1(x_i) \rfloor\neq\lfloor c_2(x_i)\rfloor$,   without loss of generality, let us assume that $c_1(x_i)>c_2(x_i)$.
 Let us consider a point $r$ whose $j$th coordinate 
 is defined as $r(x_j)=\lfloor c_1(x_j) \rfloor +1$, for all $j\in [d]$.
  It is easy to note that the point $r\in {\cal Q}(\sigma_1)$.
Since the difference between $c_2(x_i)$ and $r(x_i)$ is more than one, thus, the point $r\notin {\cal Q}(\sigma_2)$. Hence, ${\cal Q}(\sigma_1)\neq{\cal Q}(\sigma_2)$.

 
Now, we consider the converse part. Assume that for each $i\in\K_1$, $\lfloor c_1(x_i)\rfloor=\lfloor c_2(x_i)\rfloor$, and for each $i\in\K_2$, $c_1(x_i)=c_2(x_i)$. We need to prove that ${\cal Q}(\sigma_1)={\cal Q}(\sigma_2)$. First, we prove that $ {\cal Q}(\sigma_1)\subseteq {\cal Q}(\sigma_2)$. The other case ($ {\cal Q}(\sigma_2)\subseteq {\cal Q}(\sigma_1)$) is symmetric in nature. Let $r_1\in{\cal Q}(\sigma_1)$. 
 For each $i\in\K_1$, $r_1(x_i)$ has only two possibilities from $\{\lfloor c_1(x_i) \rfloor,\lceil c_1(x_i)\rceil\}$. Since  $\lfloor c_1(x_i)\rfloor=\lfloor c_2(x_i)\rfloor$, therefore, the difference between $c_2(x_i)$ and $r_1(x_i)$ is at most one. For each $i\in\K_2$, $r_1(x_i)$ has three possibilities from $\{c_1(x_i)-1,c_1(x_i),c_1(x_i)+1\}$. Since $c_1(x_i)=c_2(x_i)$,  the difference between $c_2(x_i)$ and $r_1(x_i)$ is at most one. As a result, $dist_{\infty}(c_2,r_1)\leq 1$. Hence, $r_1\in{\cal Q}(\sigma_2)$.
 % The other part ${\cal Q}(\sigma_2)\subseteq{\cal Q}(\sigma_1)$ is also symmetric in nature.
% {As a result, any hypercube having a center at the interior of the $k$-hypercube simplex contains the same set of $2 3^{d-1}$ integer points.} 
\myqed
\end{proof}
Using the above lemma, we prove the next two lemmas that play an important role in analysing our algorithm.
We have $d+1$ types of equivalence classes depending on the number of integer points they cover.
We refer to an equivalence class that contains exactly $2^k3^{d-k}$ integer points as \emph{an equivalence class of  Type-($k$)}, where $k\in[d]\cup\{0\}$.
By careful observation, one can note the following.
\begin{lemma}\label{claim:subset}
Let $\sigma$ be a unit hypercube in $\mathbb{R}^d$, centered at a point $c\in\IR^d$, belonging to some equivalence class of  Type-($k$), where $k\in[d-1]\cup\{0\}$. There exists a set ${\mathbb S}_{\sigma}$ of distinct $2^{(d-k)}$ equivalence classes of  Type-($d$) such that ${\cal Q}(\sigma)=\cup_{\sigma'\in{\mathbb S}_{\sigma}}{\cal Q}\left(\sigma'\right)$.
\end{lemma}
% \subsection{Proof of Lemma~\ref{claim:subset}}\label{app_claim:subset}
\begin{proof}
Let  $\K_1$ and $\K_2$ be the partition of $[d]$ depending upon the noninteger coordinates of the center $c$ such that $|\K_1|=k$.  For the sake of simplicity, let us assume that  the first $d-k$ indices of  $[d]$  belong to $\K_2$, and  the remaining $k$ indices belong to $\K_1$.
Now we construct $2^{d-k}$ many  hypercubes.
Let $0\leq t < 2^{d-k}$ be an integer, and  
let $t_1t_2\ldots t_{d-k}$ be the  binary representation of $t$.
Let $\sigma_t$ be a unit hypercube, centered at $c_t$,  such that,
for each $i\in\K_1$, $c_t(x_i)$ is equal to $c(x_i)$.
% , i.e., $c_t(x_i)=c(x_i)$.
For each $i\in\K_2$, the $i$th coordinate of $c_t$ is defined as follows.
\begin{equation*}\label{c_p}
c_t(x_i)=
    \begin{cases}
   c(x_i)+\epsilon,& \text{if $t_i=0$}\\
     c(x_i)-\epsilon, & \text{otherwise (i.e., if $t_i=1$)},
    \end{cases}      
\end{equation*}
 where $c$ is the center of the hypercube $\sigma$ belonging to an equivalence class of  Type-($k$) and  $0<\epsilon<1$ is a fixed arbitrary constant. 
Let ${\mathbb S}_{\sigma}=\{\sigma_0,\sigma_1,\ldots,\sigma_{2^{d-k}-1}\}$.
 Due to Lemma~\ref{Lemma_1}, it is easy to observe that each hypercube in ${\mathbb S}_{\sigma}$ belongs to  a distinct equivalence class. Therefore, ${\mathbb S}_{\sigma}$ consists of $2^{d-k}$ distinct equivalence classes of  Type-($d$). 
 
 Now, we show that  ${\cal Q}(\sigma')\subseteq{\cal Q}(\sigma)$, for each $\sigma'\in{\mathbb S}_{\sigma}$ centered at $c'$.
Let $r'\in {\cal Q}(\sigma')$ be an integer point. For each $i\in\K_1$, $r'(x_i)$ has only two possibilities from $\{\lfloor c(x_i) \rfloor,\lceil c(x_i)\rceil\}$. Therefore, the difference between $c(x_i)$ and $r'(x_i)$ is at most one. On the other hand, for each $i\in\K_2$, $r'(x_i) $ has two possibilities from $\{\lfloor c'(x_i)\rfloor,\lceil c'(x_i)\rceil\}$. 
Since the value of $c'(x_i)$ is either $c(x_i)+\epsilon$ or $c(x_i)-\epsilon$, it is easy to observe that 
the difference between $c(x_i)$ and $r'(x_i)$ is at most one. As a result, $r'$ belongs to ${\cal Q}(\sigma)$. Hence, for any $\sigma'\in{\mathbb S}_{\sigma}$, ${\cal Q}(\sigma')\subseteq{\cal Q}(\sigma)$. Therefore, $\cup_{\sigma'\in{\mathbb S}_{\sigma}} {\cal Q}(\sigma')\subseteq {\cal Q}(\sigma)$.

Now we prove that $ {\cal Q}(\sigma)\subseteq\cup_{\sigma'\in{\mathbb S}_{\sigma}}{\cal Q}(\sigma')$. Let $r\in{\cal Q}(\sigma)$  be an integer point. Now we need to construct some $c'$ such that $\sigma'\in{\mathbb S}_{\sigma}$, centered at $c'$, contains the point $r$.
 Let us define, for each $i\in\K_1$, $c'(x_i)$ to be equal to $c(x_i)$. Note that, for each $i\in\K_2$, $r(x_i)$ has three possibilities from $\{c(x_j)-1,c(x_j),c(x_j)+1\}$. For each $i\in\K_2$, the $i$th coordinate of $c'$ is constructed as follows.
\begin{equation*}\label{c_q}
c'(x_i)=
    \begin{cases}
    c(x_i)-\epsilon,& \text{if $r(x_i)=c(x_i)-1$}\\
     c(x_i)+\epsilon, & \text{otherwise}.
    \end{cases}      
\end{equation*}
Observe that the hypercube $\sigma'$ defined above belongs to ${\mathbb S}_{\sigma}$.
Since the distance between $r$ and $c'$ is at most one under $L_{\infty}$ norm, the hypercube $\sigma'$ contains the point $r$.
Hence $r\in\cup_{\sigma'\in\s}{\cal Q}(\sigma')$. As a result, we have ${\cal Q}(\sigma)\subseteq\cup_{\sigma'\in{\mathbb S}_{\sigma}}{\cal Q}(\sigma')$.
 Therefore, ${\cal Q}(\sigma)=\cup_{\sigma'\in{\mathbb S}_{\sigma}}{\cal Q}(\sigma')$.
\myqed
\end{proof}

 \begin{lemma}\label{hit_hyp}
Each integer point $p\in\mathbb{Z}^d$ is contained in exactly $2^d$ distinct equivalence classes of  Type-($d$), where $d\in\mathbb{N}$.
\end{lemma}

% \subsection{Proof of Lemma~\ref{hit_hyp}}\label{app_hit_hyp}
\begin{proof}
Let $\sigma$ be a hypercube, centered at $c$, belonging to an equivalence class of  Type-($d$).
Due to Lemma~\ref{Lemma_1}, any hypercube $\sigma'$ belongs to the same equivalence class of $\sigma$ if and only if for each $i\in[d]$, $c'(x_i)$ lies in the open interval $\left(\lfloor c(x_i)\rfloor,\lceil c(x_i)\rceil\right)$. In other words, the center of each of these hypercubes $\sigma'$ lies in the interior of an integer hypercube that contains the point $c$.
Therefore, the interior of each integer hypercube represents centers of hypercubes belonging to an equivalence class of  Type-($d$).
\noindent
Let $H_p$ be a $d$-dimensional unit hypercube centered at an integer point $p$. Note that all the centers of unit hypercubes containing the point $p$ must lie in $H_p$. Observe that the  unit hypercube $H_p$ contains  exactly $2^d$ many integer hypercubes. This implies that the integer point $p$ is contained in exactly $2^d$ many equivalence classes of  Type-($d$).
\myqed 
\end{proof}
% So any point $p \in \mathbb{Z}^d$ can be contained in $2^d$ equivalence classes of hypercubes in $\mathbb{R}^d$. Hence, the lemma follows.

To obtain the upper bound,  for $d\geq 3$,   we propose an $O(d^2)$ competitive randomized iterative reweighting algorithm that is similar in nature to an algorithm from~\cite{DumitrescuT22}. It was presented for covering integer points using integer hypercubes in the online setup.


\paragraph{\textbf{Algorithm:}} Let $\I$ be the set of hypercubes presented to the algorithm and $\A$ be the set of points chosen by our algorithm such that each hypercube in $\I$ contains at least one point from $\A$. The algorithm maintains two disjoint sets $\A_1$ and $\A_2$ such that $\A= \A_1 \cup \A_2$. The algorithm also maintains another set $\B$ of points for bookkeeping purposes; initially, each of the set $\I,\A\text{ and }\B$ are empty. A weight function  $w$ over all integer points is also maintained by the algorithm; initially, $w(p)=3^{-(d+1)}$, for all points $p \in \mathbb{Z}^d$. One iteration of the algorithm is described below.

Let $\sigma$ be a new hypercube; update $\I = \I \cup\{\sigma\}$. Note that $|\Q(\sigma)|$ is at least $2^d$ and at most $3^d$.
\begin{itemize}
\item[1.] If the hypercube $\sigma$ contains any point from $\A$, then do nothing.
\item[2.] Else if the hypercube $\sigma$ contains any point from $\B$, then let $p\in \B\cap\QH$ be an arbitrary point, and update $\A_1 = \A_1 \cup\{p\}$.
\item[3.] Else if $\sum_{p\in \QH} w(p)\geq 1$, then let $p$ be an arbitrary point in $\QH$, and update $\A_2 = \A_2 \cup\{p\}$.
\item[4.] Else, the weights give a probability distribution on $\QH$. Successively choose points from $\QH$ at random with this distribution in $\lceil\frac{5d}{2}\rceil$ independent trails and add them to $\B$. Let $p\in \B\cap\QH$ be an arbitrary point, and update $\A_1 = \A_1 \cup\{p\}$. Triple the weight of every point in $\QH$.
\end{itemize}




Now, we analyze the performance  of the above algorithm.
\begin{theorem}\label{hyp_ub}
For hitting unit hypercubes using points in $\mathbb{Z}^d$, there exists a randomized algorithm whose competitive ratio is at most~$O(d^2)$, where $d\geq3$.
\end{theorem}
\begin{proof}
Let  $\I$ be the set of $n$ hypercubes presented to our algorithm. Let $\OO$ be an offline optimum hitting set for $\I$.  Note that our algorithm creates two disjoint sets $\A_1\text{ and }\A_2$ such that $\A=\A_1\cup\A_2$ is a hitting set for $\I$.
From the description of the algorithm, it is easy to follow that $\A_1\subseteq \B$. We prove that $\mathbb{E}[|\B|]=O(d^2|\OO|)$, and $\mathbb{E}[|\A_2|]=O(|\OO|)$. This immediately implies that $\mathbb{E}[|\A|]\leq\mathbb{E}[|\A_1|] +\mathbb{E}[|\A_2|]\leq \mathbb{E}[|\B|]+\mathbb{E}[|\A_2|]=O(d^2|\OO|)$.

First, consider $\mathbb{E}[|\B|]$. Note that in the set $\B$, new points are added only in step 4 of the algorithm. In this case, the algorithm adds at most $\lceil\frac{5d}{2}\rceil$ points (independently) in $\B$ and triples the weight of every point in $\QH$. Let $\OO$ denote the offline optimum set of integer points. Each hypercube $\sigma\in \I$ contains some point $p\in \OO$. Initially, the weight of $p$ is $3^{-(d+1)}$, and it will never exceed 3. Since $p\in \QH$, its weight before the last tripling must have been at most 1 in step 4 of the algorithm; thus, its weight is tripled in at most $d+2$ iterations. Consequently, the algorithm invokes step 4  of the algorithm in at most $(d+2)|\OO|$ iterations. In each such iteration, the algorithm adds at most $\lceil\frac{5d}{2}\rceil$ points (independently) in the set $\B$. Therefore, we have $|\B|\leq \lceil\frac{5d}{2}\rceil(d+2)|\OO|=O(d^2|\OO|)$.

Next, we consider $\mathbb{E}[|\A_2|]$. Note that in the set $\A_2$, new points are added only in step 3 of the algorithm. In this case,  when a hypercube $\sigma$ arrives,  none of the points of $\QH$ is in $\B$ and $\sum_{p\in \QH} w(p)\geq 1$, and the algorithm increments the cardinality of the set $\A_2$ by one.
% Let $W=\left(\sum_{p\in \QH} w(p)\right)$ be the sum of weights of all point $p\in \QH$.
At the beginning of the algorithm, we have $W_{initially}=\sum_{p\in \QH} w(p)=\sum_{p\in\QH} 3^{-(d+1)}\leq 3^d  3^{-(d+1)}= \frac{1}{3}$. Suppose that the weights of the points in $\QH$ are increased in $k$ iterations (starting from the beginning of the algorithm), and the sum of weights of points in $\QH$ is increased by $\delta_1,\delta_2,\ldots,\delta_k>0$. When $\sigma$ arrives, the sum of the weights of all the points in $\Q(\sigma)$ is $W_{now}=W_{initially}+\sum_{i=1}^k\delta_i\geq 1$  and we know $\ W_{initially}\leq \frac{1}{3}$. This
implies that $\sum_{i=1}^k\delta_i\geq \frac{2}{3}$. For every $i\in[k]$, the sum of weights of some points in $\QH$, say $Q_i\subset \QH$ is increased by $\delta_i$ in step 4 of the algorithm. Since the weights are tripled, therefore, the sum of the weights of these points was $\frac{\delta_i}{2}$ at the beginning of that iteration. The algorithm added a point from $Q_i$ to $\B$ with probability at least $\frac{\delta_i}{2}$ in one random draw, which was repeated $\lceil\frac{5d}{2}\rceil$ times independently. As a result, the probability that the algorithm does not add any point from $Q_i$ to the set $\B$ is at most $\left(1-\frac{\delta_i}{2}\right)^{\lceil\frac{5d}{2}\rceil}$. The probability that none of the points of $\QH$ are added to $\B$ before the arrival of $\sigma$ is at most $ \prod_{i=1}^k \left(1-\frac{\delta_i}{2}\right)^{\lceil\frac{5d}{2}\rceil}\leq e^{-\lceil\frac{5d}{2}\rceil\sum_{i=1}^k\frac{\delta_i}{2}}\leq e^{-\frac{5d}{4}\sum_{i=1}^k\delta_i}\leq e^{-\frac{5d}{6}}$.
Since $\I$ is the set of hypercubes presented to the algorithm, therefore, step 3 of the algorithm can be invoked at most $|\I|$ times $e^{-\frac{5d}{6}}$. As a result $\mathbb{E}[|\A_2|]\leq |\I|  e^{-\frac{5d}{6}}$.
Note that this is a very loose upper bound. 
Let $N$ be the set of distinct  equivalence classes containing all the hypercubes in $\cal I$.
Observe that if the algorithm hits  one hypercube from an equivalence class, then the algorithm executes only step 1 for all subsequent hypercubes coming from the same equivalence class. 
Therefore, step 3 of the algorithm can be invoked at most $|N|e^{-\frac{5d}{6}}$ times. We can further improve this bound  as follows.

Let $\sigma\in \I$.   
According to Lemma~\ref{claim:subset}, 
we have a set ${\mathbb S}_{\sigma}$ of equivalence classes of type-($d$) such that ${\cal Q}(\sigma)=\cup_{\sigma'\in{{\mathbb S}_{\sigma}}}{\cal Q}\left(\sigma'\right)$. 
Observe that  if  some hypercube $\sigma$ arrives, and our algorithm needs to place a hitting point $p$ for it, then it implies that none of the hypercubes belonging to ${\mathbb S}_{\sigma}$ arrived before $\sigma$ to the algorithm.
Let $p\in {\cal Q}(\sigma')$ for some $\sigma' \in {\mathbb S}_{\sigma}$. Note that the point $p$ acts as a  hitting point for any  hypercube in $\I$ belonging to the same class of $\sigma'$. Not only that but $p$ also acts as a hitting point for all hypercubes $\sigma''\in \I$ such that $\sigma'\in {\mathbb S}_{\sigma''}$. Therefore, step 3 of the algorithm is invoked at most $|N_d|e^{-\frac{5d}{6}}$ times, where $N_d=\cup_{\sigma\in\I} {\mathbb S}_{\sigma}$.
 Hence, $\mathbb{E}[|\A_2|]\leq |N_d|e^{-\frac{5d}{6}}$.
Now we  give an upper bound of $|N_d|$ in terms of $|\OO|$. Due to Lemma~\ref{hit_hyp}, we know that any arbitrary integer point $p\in\OO$ can be contained in at most  $2^d$ equivalence classes of Type-($d$) hypercubes. Thus we have $|N_d|\leq 2^d  |\OO|)$.
 Since we have, $\mathbb{E}[|\A_2|]\leq |N_d|  e^{-\frac{5d}{6}}$, and $|N_d|\leq 2^d |\OO|$, therefore $\mathbb{E}[|\A_2|]\leq O\left(\left(\frac{2}{e^\frac{5}{6}}\right)^d |\OO| \right)\leq |\OO|$. Hence the theorem follows.
 \myqed
\end{proof}

% \textbf{Upper Bounds for Unit Hypercubes in $\IR^3$}\\
% Observe that for $d=3$, in the above theorem,  the value of $\mathbb{E}[|\B|]\leq 8\times5|\OO|=40|\OO|)$ and $\mathbb{E}[|\A_2|]\leq \left(\frac{2}{e^\frac{5}{6}}\right)^3 |\OO| \leq 0.657|\OO|$. Hence, when $d=3$, the randomized iterative reweighting algorithm achieves a competitive ratio of at most~40.657. 








% -*-*-**-*-*-**-*-*-*-*-*- Balls *-*-**-*-*-*-*-*-*-**-*-*-**--*

\section{Unit Balls in $\mathbb{R}^d$}\label{sec:balls}

In this section, we present the lower and upper bounds of the competitive ratio for hitting unit balls in $\IR^d$ using points from $\mathbb{Z}^d$. 
\subsection{Lower Bound for $d<4$}

%%%%%%%%%%%%%%%%%%%%%------- May add this in Full version --------%%%%%%%%%%%%%%%%%%%%%

% \begin{figure}[ht]
%   \centering
%   \includegraphics[width=33 mm]{disk_a.pdf}
%   \caption{ The figure illustrates the case for $d=2$. In each round of the game the ball $\sigma_i$ is presented, and our online algorithm places hitting points $h_i$ to hit the presented ball $\sigma_i$.}
% \label{fig:disk_lb}
% \end{figure}


%%%%%%%%%%%%------------------------%%%%%%%%%%%%%%%%%


To obtain a lower bound of the competitive ratio, we think of a game between two players: Alice and Bob. Here, Alice plays the role of the adversary, and Bob plays the role of the online algorithm. In each round of the game, Alice presents a unit ball such that Bob needs to place a new hitting point.
%to hit the presented ball if it is already not hit by any of the previously placed hitting points. Alice needs to present balls to force Bob to place as many new hitting points as possible, while Bob needs to place as few hitting points as possible to win the game. Note that 
We show that Alice can present an input sequence of balls  $\sigma_1,\sigma_2,\ldots,\sigma_{d+1}\subset \mathbb{R}^d$, centered at $c_1,c_2,\ldots,c_{d+1}$, respectively, depending on the position of hitting points placed by Bob, for which Bob needs to place $d+1$ integer points; while the offline optimum needs just one point.
For the sake of simplicity, let us assume that the center $c_1$ of the first ball $\sigma_1$ coincides with the origin. Note that the ball $\sigma_1$ contains exactly $2d$ integer points $\P=\{p_1,p_2,\ldots,p_{2d}\}$ apart from the origin. The coordinates of these points are given below:
\small
\begin{equation}\label{eqn_p}
p_k(x_j) =
    \begin{cases}
     \ \ 1,\quad \text{ if $k=j$,} & \text{for $k,j\in[d]$}\\
      -1,\quad \text{ if $k=d+j$,} & \text{for $k\in[2d]\setminus[d]$ \& $j\in[d]$}\\ 
     \ \ 0,\quad \text{ otherwise}.&
    \end{cases}      
\end{equation}
\normalsize
Let $\P_1=\{p_{1},p_{2},\ldots,p_{d}\}$ and  $\P_2=\{p_{d+1},p_{d+2},\ldots,p_{2d}\}$. To hit the input ball $\sigma_1$, Bob needs to choose a point $h_1\in \P_1\cup\P_2\cup \{c_1\}$. 
Depending on the position of $h_1$, Alice presents a ball $\sigma_2$ centered at a point $c_2$ that satisfies the following:
\small
\begin{equation}\label{eq:c_2}
c_2 =
    \begin{cases}
    \left(\frac{1}{2}+\epsilon_d,\frac{1}{2}+\epsilon_d,\ldots,\frac{1}{2}+\epsilon_d\right),& \text{if $h_1\in \P_2\cup\{c_1\}$}\\
      \left(-(\frac{1}{2}+\epsilon_d),-(\frac{1}{2}+\epsilon_d),\ldots,-(\frac{1}{2}+\epsilon_d)\right), & \text{otherwise (i.e., if $h_1\in \P_1$)},
    \end{cases}      
\end{equation}
\normalsize
where 
 the value of $\epsilon_d$ is $0.5$ and $0.15$ for $d=2\text{ and } 3$, respectively.
Note that $\sigma_2$ does not contain the point $c_1$.
 
 
\begin{lemma}\label{claim_1}
\begin{itemize}
 \item[(i)]  If $h_1\in \P_2\cup\{c_1\}$, then ${\cal Q}(\sigma_2)$ contains all the points of $\P_1$ and it does not contain any point of~$\P_2\cup \{c_1\}$.
\item[(ii)]  If $h_1\in \P_1$, then ${\cal Q}(\sigma_2)$ contains all the points of $\P_2$ and it does not contain any point of~$\P_1$
\end{itemize}
\end{lemma}
% \subsection{Proof of Lemma~\ref{claim_1}}\label{app_claim_1}
\begin{proof}
We prove part(i) of the lemma statement. The proof of part(ii) would be similar in nature.    Assume that  $h_1\in\P_2\cup\{c_1\}$. According to~(\ref{eq:c_2}), Alice presents $\sigma_2$ centered at $c_2=\left(\frac{1}{2}+\epsilon_d,\frac{1}{2}+\epsilon_d,\ldots,\frac{1}{2}+\epsilon_d\right)$.
% \noindent
%  Let $dist(c_2,p_k)$ be the distance between the center  $c_2$ and the point $p_k$, 
Note that  the ball $\sigma_2$ does not contain the point $c_1$.
To see that $\sigma_2$ does not contain any point from $\P_2$, observe that 
for each $p_k\in\P_2$, we have  
\small
\begin{align*}
    dist(c_2,p_k)^2=&\left(\frac{3}{2}+\epsilon_d\right)^2+\sum_{j\in[d ]\setminus\{k-d\}} \left(\frac{1}{2}+\epsilon_d\right)^2>1.
\end{align*}
\normalsize
% Since the ball $\sigma_2$ does not contain any point from $\P_2$, so $\sigma_2$ does not contain the hitting point $h_1\in\P_2$. 
Finally, we prove that the ball $\sigma_2$ contains all the points of $\P_1$.
For each $p_k\in\P_1$, we have
\small\[
dist(c_2,p_k)^2=\left(\frac{1}{2}-\epsilon_d\right)^2+\sum_{j\in[d]\setminus \{k\}} \left(\frac{1}{2}+\epsilon_d\right)^2=\left(-\frac{1}{2}+\epsilon_d\right)^2+(d-1)\left(\frac{1}{2}+\epsilon_d\right)^2\leq1,
\]
\normalsize
The last inequality follows by placing the specific values of $\epsilon_d$, i.e., 0.5 and 0.15 for $d=2$ and $3$, respectively.
\noindent
% Therefore, if the hitting point $h_1\in \P_2\cup\{c_1\}$, then the ball $\sigma_2$ centered at $c_2$ contains all the $d$ points of $\P_1$ and it does not contain any point from $\P_2\cup\{c_1\}$.
% In the other case, when $h_1\in\P_1$, the proof is similar in nature. 
Hence, the lemma follows.
\myqed
\end{proof}


From now onwards, we assume that Bob chooses $h_1\in\P_2\cup\{c_1\}$. The other case is similar in nature. Now we  show by induction that Alice and Bob can play the game for  the next $d+1$ rounds maintaining the following two invariants:  For $i=2,\ldots, {d+1}$, when Alice presents  balls  $\sigma_2,\ldots, \sigma_i$ and Bob presents piercing points $p_{\pi(2)}, p_{\pi(3)}\ldots, p_{\pi(i-1)}\in\P_1$, 
\begin{itemize}
    \item[(I)] The ball $\sigma_i\subset \mathbb{R}^d$ does not contain any previously placed hitting point $h_j\in \mathbb{Z}^d$, for 
$j<i$.
   \item[(II)] The ball $\sigma_{i}$ contains all the points from $\P_1\setminus\{p_{\pi(2)}, p_{\pi(3)}\ldots, p_{\pi(i-1)}\}$. 
   % where $Q_i=\cap_{j=1}^{i} \sigma_j=\sigma_1\cap\sigma_i$.
   % \item The common intersection region $Q_i=\cap_{i=1}^{i} \sigma_i$ contains at least $(d-i+2)$ integer points.
\end{itemize}
Invariant (I) ensures that Bob needs a new point to hit $\sigma_i$. On the other hand, Invariant (II) ensures that $\cap\sigma_i$ contains a point from $\P_1$ that is not used by Bob.
% ---------------------Proof starts-------------------
 For $i=2$, due to Lemma~\ref{claim_1}, both the invariants are maintained. 
%  Since $h_1\notin\sigma_2$, therefore, Bob is forced to place a new hitting point $h_2\in\mathbb{Z}^d$ such that $\sigma_2$ contains $h_2$. Hence invariant (I) is also maintained.
At the beginning of the round $i$ (for $i=2,\ldots,d$), assume that both invariants hold.
Let $\Pi=\{\pi(2),\pi(3),\ldots,\pi(i)\}$ be the set of indices of integer points chosen from $\P_1$ to hit the previously arrived balls.
Depending on the position of the hitting point $p_{\pi(i)}$, Alice presents a ball $\sigma_{i+1}$, in the $(i+1)$th round of the game, centering at $c_{i+1}$ that satisfies the following.
\small
\begin{equation}\label{eqn}
c_{i+1}(x_j)=
\begin{cases}
     \left(\frac{3}{2}\right)^{(i-1)}c_2(x_j),& \text{for all $j\in[d]\setminus \Pi$, and}\\
      0, & \text{for $j\in \Pi$}.
    \end{cases}      
\end{equation}
\normalsize
\noindent
${\bullet}$ First, we prove that $\sigma_{i+1}$ does not contain the first hitting point $h_1$. Observe that $dist(c_{i+1},h_1)^2=\sum_{j\in [d]} \left(c_{i+1}(x_j)-h_1(x_j)\right)^2$.
% \small
% \begin{align*}
%  dist(c_{i+1},h_1)^2=&\sum_{j\in [d]} \left(c_{i+1}(x_j)-h_1(x_j)\right)^2,
%  \end{align*}
%  \normalsize
 Note that for $j\in \Pi$, the value of $c_{i+1}(x_j)$ is zero. So we have 
 \begin{align*}
 dist(c_{i+1},h_1)^2=&\sum_{j\in \Pi} \left(0-h_1(x_j)\right)^2+\sum_{j\in[d]\setminus \Pi} \left(\left(\frac{3}{2}\right)^{(i-1)}\left(\frac{1}{2}+\epsilon_d\right)-h_1(x_j)\right)^2.
 \end{align*}
 \normalsize
 If $h_1=c_1$, then we have $dist(c_{i+1},h_1)^2= \quad0 + (d-i+1)\left(\frac{3}{2}\right)^{2(i-1)}\left(\frac{1}{2}+\epsilon_d\right)^2>1$.
%   \small
%   \begin{align*}
%   dist(c_{i+1},h_1)^2= &\quad0 + (d-i+1)\left(\frac{3}{2}\right)^{2(i-1)}\left(\frac{1}{2}+\epsilon_d\right)^2>1.
% \end{align*}
% \normalsize
% \noindent
 If $h_1=p_k\in\P_2$, then we have the following two sub-cases. 
If $(k-d)\in \Pi$, we have $dist(c_{i+1},h_1)^2= 1 + (d-i+1)\left(\frac{3}{2}\right)^{2(i-1)}\left(\frac{1}{2}+\epsilon_d\right)^2>1$,
 otherwise ($(k-d)\in[d]\setminus \Pi$), we have $dist(c_{i+1},h_1)^2= (d-i+1)\left(\left(\frac{3}{2}\right)^{(i-1)}\left(\frac{1}{2}+\epsilon_d\right)+1\right)^2>1$.
%   \small
%   \begin{align*}
%   dist(c_{i+1},h_1)^2= & (d-i+1)\left(\left(\frac{3}{2}\right)^{(i-1)}\left(\frac{1}{2}+\epsilon_d\right)+1\right)^2>1.  
% \end{align*}
% \normalsize
Now, we show that  $\sigma_{i+1}$ does not contain any of the previously placed hitting points of $\P_1$. Here, for any $p_{\pi(k)}\in\{p_{\pi(2)}, p_{\pi(3)}\ldots, p_{\pi(i)}\}$, we have
 \small
\begin{align*}
dist(c_{i+1},p_{\pi(k)})^2=&\sum_{j\in [d]} \left(c_{i+1}(x_j)-p_{\pi(k)}(x_j)\right)^2.
\end{align*}
\normalsize
\noindent
Note that for $j\in \Pi$, $c_{i+1}(x_j)=0$, and $p_{\pi(k)} \in\P_1$ has only one nonzero coordinate that is the $\pi(k)$th coordinate with value 1 and $\pi(k)\in\Pi$. Therefore, we have
% for $\in[i]\setminus\{1\}$ and $\pi(k)\in \Pi$.
\small
\begin{align*}
dist(c_{i+1},p_{\pi(k)})^2=&\sum_{j\in \Pi} \left(0-p_{\pi(k)}(x_j)\right)^2+\sum_{j\in[d]\setminus \Pi} \left(\left(\frac{3}{2}\right)^{(i-1)}\left(\frac{1}{2}+\epsilon_d\right)\right)^2\\
=& 1+(d-i+1)\left(\frac{3}{2}\right)^{2(i-1)}\left(\frac{1}{2}+\epsilon_d\right)^2>1. 
\end{align*}
\normalsize
Therefore, the distance between the center $c_{i+1}$ and previously placed hitting points $\{p_{\pi(2)}, p_{\pi(3)}\ldots, p_{\pi(i)}\}$ is greater than one.
Hence the invariant (I) holds.\\

\noindent
${\bullet}$ Now, we show that $\sigma_{i+1}$ contains all $(d-i+1)$ integer points from $\P_1\setminus\{p_{\pi(2)}, p_{\pi(3)}\ldots, p_{\pi(i)}\}$.  Here, for any $p_k\in\P_1\setminus\{p_{\pi(2)}, p_{\pi(3)}\ldots, p_{\pi(i)}\}$, we have
 \small
    \begin{align*}
        dist(c_{i+1},p_k)^2=&\sum_{j\in [d]} \left(c_{i+1}(x_j)-p_k(x_j)\right)^2\\
        =&\sum_{j\in \Pi} \left(c_{i+1}(x_j)-p_k(x_j)\right)^2+\sum_{j\in[d]\setminus \Pi} \left(\left(\frac{3}{2}\right)^{(i-1)}\left(\frac{1}{2}+\epsilon_d\right)-p_k(x_j)\right)^2.
        \end{align*}
        \normalsize
Note that for $j \in  \Pi$, both $c_{i+1}(x_j)$ and $p_k(x_j)$ are zero. Here, $p_k$ has only one nonzero coordinate, which is the $k$th coordinate with value one and $k\notin\Pi$. Therefore, we have
       \small
        \begin{align}\label{eqn:why}
        \begin{split}
      dist(c_{i+1},p_k)^2=&\ 0
       +\left(\left(\frac{3}{2}\right)^{(i-1)}\left(\frac{1}{2}+\epsilon_d\right)-1\right)^2+\sum_{j\in [d]\setminus \{\Pi\cup\{k\}\}}\left(\left(\frac{3}{2}\right)^{(i-1)}\left(\frac{1}{2}+\epsilon_d\right)\right)^2\\
      =& \left(\left(\frac{3}{2}\right)^{(i-1)}\left(\frac{1}{2}+\epsilon_d\right)-1\right)^2+(d-i) \left(\left(\frac{3}{2}\right)^{(i-1)}\left(\frac{1}{2}+\epsilon_d\right)\right)^2\leq1.
      \end{split}
    \end{align} 
    \normalsize
    The last inequality follows by placing specific values of $\epsilon_d$, i.e., $0.5$ and $0.15$ for $d=2$ and $3$, respectively.
   Hence invariant (II) is maintained.

As a result, any online algorithm needs $d+1$ hitting points $\{p_{\pi(2)},p_{\pi(3)}\ldots p_{\pi(d+1)}\}$ and $h_1$; whereas offline optimum needs just one point $p_{\pi(d+1)}$.
Thus, we have the following theorem:
\begin{theorem}\label{ball_lb}
The  competitive ratio of every deterministic online algorithm  for hitting unit balls in $\mathbb{R}^d$ using points in $\mathbb{Z}^d$ is at least~$d+1$, for $d<4$.
\end{theorem}

\noindent
\textbf{Remark:} In equation~\ref{eqn:why}, for any $\epsilon_d>0$ and $d\geq 4$, $dist(c_{i+1},p_k)^2>1$. As a result, Theorem~\ref{ball_ub} is only valid for $d< 4$.





\subsection{Upper Bound}
In this subsection, we present the upper bound on the competitive ratio for hitting unit balls in $\IR^d$.

\begin{theorem}\label{ball_ub}
For hitting unit balls using points in $\mathbb{Z}^d$, there exists a deterministic online algorithm whose competitive ratio is at most~$O(d^4)$, for any $d\in\mathbb{N}$.
\end{theorem}

\begin{proof}
\textbf{Algorithm:}
On receiving a new input ball $\sigma\subset\mathbb{R}^d$ centered at $c$, 
if it has not been hit by the existing hitting set, then our online algorithm adds the nearest integer point from the center $c$ as the hitting point. If the center is equidistant from several integer points, our algorithm arbitrarily chooses one of the nearest points as the hitting point.
% if the existing hitting set has not hit it, then our online algorithm adds the nearest integer point from the center $c$ as the hitting point. If the center is equidistant from several integer points, our algorithm arbitrarily chooses one of the nearest points as the hitting point.

\noindent
\textbf{Analysis:}  Let $\A$ and $\OO$ be the hitting set returned by our online algorithm and the optimal hitting set, respectively.  Let $p \in \OO$ be any point.
Note that a unit ball $B_d(p,1)$ centered at $p$ contains all the centers of unit balls that can be hit by the point $p$. For simplicity, throughout the proof, let us assume that the point $p$ coincides with the origin. Let $\A_p$ be the set of hitting points placed by our online algorithm to pierce the ball having a center in $B_d(p,1)$.
It is easy to see that $\A=\cup_{p\in \OO}\A_p$.
Therefore, the competitive ratio of our algorithm is  upper bounded by  $\max_{p\in\OO}|\A_p|$. For any point, $r\in B_d(p,1)$, the maximum distance from $r$ to its nearest integer point can be at most one (the maximum distance from the center $c$ of the unit ball to any point $r\in B_d(c,1)$ is at most one). Therefore, a ball $B_d(p,2)$ centered  at $p$ having radius $2$ will contain all nearest integer points for all the centers $r$ lying in the ball $B_d(p,1)$. To complete the proof, we only need to calculate the cardinality of the set $\{z\in \mathbb{Z}^d: \sum_{i=1}^d |z|^2\leq 4\}$. In other words, we need to count the number of $z=(z(x_1),z(x_2),\ldots,z(x_d))\in\mathbb{Z}^d$ that satisfies:
\begin{equation}\label{eqn_z}
z(x_1)^2+z(x_2)^2+\ldots+z(x_d)^2\leq 4.
\end{equation}
\noindent
Note that to satisfy Equation~(\ref{eqn_z}), the coordinates of $z$ cannot be other than $\{-2,-1,0,1,2\}$
\begin{itemize}
    \item When all the $d$ coordinates are $0$. There is only one possibility for this.
    \item When exactly one coordinate is nonzero. There will be ${d \choose 1}$ many choices for the position of the nonzero coordinate. Now observe that for each nonzero coordinate, we have four choices $\{-2,-1,1,2\}$. 
   So for this case, there will be a total of $4d$ integer points satisfying Equation~(\ref{eqn_z}).
   \item  Note that any integer point having more than four nonzero coordinates will not satisfy Equation~(\ref{eqn_z}). Now consider exactly $i$ nonzero coordinates for $i=2,3,4$. 
    There will be ${d \choose i}$ many choices for the position of the nonzero coordinates. Now observe that if any of the nonzero coordinates is $\{-2,2\}$, then the integer point will not satisfy Equation~(\ref{eqn_z}). Therefore, for each nonzero coordinate, we have just two choices $\{-1,1\}$. Thus, there will be a total of $2^i{d \choose i}$ integer points satisfying Equation~(\ref{eqn_z}).

 \end{itemize}
Now from above cases, there will be at most $1+4d+\sum_{i=2}^4 2^i{d \choose i}=O(d^4)$ integer points satisfying Equation~(\ref{eqn_z}). Hence $|\A|\leq O(d^4)|\OO|$.
\myqed
\end{proof}








\section{Unit Covering Problem}\label{sec:cover}
Recall that by interchanging the role of unit objects and points, one can formulate an equivalent  \emph{online unit covering problem}.
Here, the points belonging to $\mathbb{R}^d$ arrive one by one. Upon the arrival of an uncovered point, we need to cover it using a unit object having center in $\mathbb{Z}^d$. 
Similar to the online hitting set problem, the decision to add a unit object is  irrevocable, i.e., the online algorithm can not remove  any unit object in the future from the existing cover. 
The aim of the online unit covering problem is to minimize the number of unit objects to cover all the presented points.
Since the above-mentioned unit covering problem is an equivalent version of the online hitting set problem studied in this paper, thus, all the results obtained for the online hitting set problem are also valid for the equivalent online unit covering problem.
We summarize the results obtained for the online unit covering problem as follows. First, we present the lower bound of the online covering problem.
Due to Theorem~\ref{hyp_lb} and~\ref{ball_lb}, we obtain the following.
\begin{corollary} The  competitive ratio of every deterministic online algorithm  for covering points in $\mathbb{R}^d$ using
    \begin{itemize}
    % \item [(i)] Unit intervals whose center are from $\mathbb{Z}^d$ is at least $2$, where $d=1$.
    \item [(i)] unit hypercubes whose centers are from $\mathbb{Z}^d$ is at least~$d+1$, where $d\in\mathbb{N}$.
    % \item [(iii)] Unit hypercubes centered at $\mathbb{Z}^d$, there exists a randomized algorithm whose competitive ratio is at most~$O(d^2)$, where $d\geq3$.
    \item [(ii)] unit balls centered at $\mathbb{Z}^d$ is at least~$d+1$, where $d<4$.
    \end{itemize}
\end{corollary}
\noindent
Now, we present obtained upper bounds.
Due to Theorem~\ref{thm:int}-~\ref{2d-balls}, we have the following.
% ~\ref{cube_ub},~\ref{square_ub},~\ref{3d-ball} and ~\ref{2d-balls}
\begin{corollary} 
    \begin{itemize}
    \item [(i)] For covering points in $\IR^d$ using unit hypercubes centered at $\mathbb{Z}^d$, where $d=1,2$ and 3, respectively, there exist  deterministic online algorithms having  competitive ratios at most 2,4 and 8, respectively.
   
     \item [(ii)] For covering points in $\IR^d$ using unit balls centered at $\mathbb{Z}^d$, where $d=2$ and 3, respectively, there exists a deterministic online algorithm that achieves a competitive ratio of at most~4 and 14, respectively.
    % \item [(iii)] For covering points in $\IR^3$ using unit balls centered at $\mathbb{Z}^3$, there exists a deterministic online algorithm that achieves a competitive ratio of at most~14 (due to Theorem~\ref{3d-ball}).
    \end{itemize}
\end{corollary}
% \textbf{Upper Bound of the Online Algorithm}
Due to Theorem~\ref{hyp_ub} and~\ref{ball_ub}, we have the following.
\begin{corollary} For covering points in $\IR^d$ using
    \begin{itemize}
    % \item [(i)] Unit intervals whose center are from $\mathbb{Z}^d$, there exists a deterministic online algorithm having a competitive ratio at most $2$, where $d=1$.
    \item [(i)] unit hypercubes centered at $\mathbb{Z}^d$, there exists a randomized algorithm whose competitive ratio is~$O(d^2)$, where $d\geq3$.
    \item [(ii)] unit balls centered at $\mathbb{Z}^d$, there exists a deterministic online algorithm whose competitive ratio is~$O(d^4)$, where $d\in\mathbb{N}$.
   
    \end{itemize}
\end{corollary}


\section{Conclusion}\label{Conclusion}
Due to the result of Even and Smorodinsky\cite{EvenS14}, we know that no online algorithm can obtain a competitive ratio better than $\Omega(\log n)$ for hitting $n$ intervals in the range $[1,n]$ using points $\P=\{1,2,\ldots,n\}$. Due to this pessimistic result, we restricted our attention to unit objects- unit balls and hypercubes in $\mathbb{R}^d$ using integer points in $\mathbb{Z}^d$. 
On the one hand, we obtain almost tight bounds on the competitive ratio in the lower dimension. On the other hand,  there is a significant gap between the lower and upper bound of the competitive ratio in higher-dimensional cases. We propose the following open problems.
\begin{enumerate}
    \item Can  the lower bound result of unit balls be extended to any $d\in \mathbb{N}$?  
    \item Is there a lower bound on the competitive ratio for hitting unit hypercubes that match the upper bound of the problem? Is there any algorithm for hitting unit hypercubes with a competitive ratio of at most $O(d)$?
    \item There are small gaps between the lower and the upper bounds for unit balls and unit hypercubes in $\IR^2$ and $\IR^3$. We propose bridging these gaps as a future direction of research.
\end{enumerate}

\bibliographystyle{plainurl}% the mandatory bibstyle


\bibliography{biblio,references}
  % \appendix
% \section{Missing Proofs}
















% \section{Related Work}
%  \subsection{covering}




% \subsection{Piercing problem}


\end{document}
























\section{Unit Ball}
\textbf{\textit{Remark:}} After first round of the game, if Bob places any hitting point $h_1\in\P_2\cup\{c_1\}$ for the presented ball $\sigma_1$, then for $d\geq 5$, Alice cannot place a ball $\sigma_2$ such that $\sigma_1\cap \sigma_2$ contains $d$ integer points and $\sigma_2$ do not contain $h_1$.
\begin{proof}
If Bob places origin $o$ as the hitting point $h_1$ for the presented disk $\sigma_1$. Let us assume that Alice can place a ball $\sigma_2$ centered at $c_2=\left(c_2(x_1),c_2(x_2),\ldots,c_2(x_d)\right)$ containing the points $p_1,p_2,\ldots,p_d$ such that $dist(c_2,o)>1$ and $dist(c_2,p_i)\leq1$, for all $i\in[d]$.

\noindent
 Let $dist(c_2,p_i)$ be the distance between the center $c_2$ and the integer point $p_i$, for each $p_i$, $i\in[d]$. Then, we have
\begin{equation}\label{eqn_1}
\begin{split}
 dist(c_2,p_i)^2\leq 1
&\implies
\left(1-c_2(x_i)\right)^2+\sum_{j=1,j\neq i}^{j=d} c_2(x_j)^2 \leq 1\\
&\implies \sum_{j=1}^{j=d} c_2(x_j)^2 \leq 2\left(c_2(x_i)\right).
\end{split}
\end{equation}

\noindent
Since $\sigma_2$ do not contain origin, therefore, the square of distance between the center of $\sigma_2$ ($c_2$) and origin $o$ is greater than one, hence we have the following
\begin{equation}\label{eqn_2}
  dist(c_2,o)^2>1\text{ i.e., }\sum_{j=1}^{j=d} c_2(x_j)^2 >1.
\end{equation}

\noindent
From equation~\ref{eqn_1} and equation~\ref{eqn_2}, we get the following
\begin{equation}\label{eqn_3}
\begin{aligned}
  1<\sum_{j=1}^{j=d} c_2(x_j)^2 \leq 2\left(c_2(x_i)\right)\implies 2\left(c_2(x_i)\right)\leq 1+\epsilon_d'\implies c_2(x_i)\leq\frac{1}{2}+\epsilon_d,\ \forall\ i\in[d],
\end{aligned}
\end{equation}
where $\epsilon_d=\frac{\epsilon_d'}{2}$ and $\epsilon_d'>0$ is a arbitrary constant close to zero.

\noindent
From equation~\ref{eqn_3}, the coordinates of the center $c_2$ will be at least $\frac{1}{2}+\epsilon_d$. For sake of simplicity, let us assume that $c_2=(\frac{1}{2}+\epsilon_d,\frac{1}{2}+\epsilon_d,\ldots,\frac{1}{2}+\epsilon_d)$. Let $ dist(c_2,p_i)$ be the distance between the centre $c_2$ and any arbitrary point $p_i$, for $i\in[d]$. Then, we have
\begin{align*}
  dist(c_2,p_i)^2=&\left(\frac{1}{2}-\epsilon_d\right)^2+\sum_{j=1,j\neq i}^{j=d} \left(\frac{1}{2}+\epsilon_d\right)^2\\
  =&\left(\frac{1}{2}-\epsilon_d\right)^2+(d-1)\left(\frac{1}{2}+\epsilon_d\right)^2>1,\text{ for $d\geq5$}.  
\end{align*}
Hence the lemma follows.
\myqed
\end{proof}

\end{document}





