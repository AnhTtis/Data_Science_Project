
\section{Related Work}
    The concept of knowledge distillation was first proposed by Hinton et al. ~\cite{hinton2015distilling}, with the goal of transferring dark knowledge from a cumbersome teacher model to a smaller student model to improve the student's performance. Based on the types of dark knowledge, mainstream knowledge distillation methods can be divided into two categories: Logits-based knowledge distillation and feature-based knowledge distillation.
\subsection{Logits-based knowledge distillation}
    Classical logits-based knowledge distillation methods~\cite{hinton2015distilling} minimize the KL divergence between the output logits of teacher and student models. One recent line of research focuses on refining the vanilla knowledge distillation loss function to better leverage the logits information. WSLD~\cite{zhou2020wsld} rethinks the knowledge distillation process from a bias-variance trade-off perspective and proposes weighted soft labels for knowledge distillation. DKD~\cite{zhao2022dkd}, reformulates the classical knowledge distillation loss into the target and non-target part and calculates the distillation loss separately. While these works have improved the performance of logits-based knowledge distillation methods on classification tasks, they have often not achieved significant results on other tasks, such as dense prediction tasks.
%Head-related:   
    %However, this distillation manner is related to the classification head, for more sophisticated tasks such as object detection which involves the classification task and the regression task, Logits-based methods often fail to achieve ideal performance due to the task-specific property such as the imbalanced distribution of pixels between fore-groud and background.
%Trial on different tasks: 
%Recently, there are some works that try to use the logit-based knowledge distillation methods on other tasks, which often consider a reformulation of traditional Knowledge distillation loss.
    Another line of work involves modeling other tasks into a classification task, and adopting the logits-based knowledge distillation on other tasks.
        LD\cite{zheng2022ld}, reformulates the output form of the regression head to a probability distribution and applies classical knowledge distillation to the regression task. However, it is only for object detection tasks and requires changes to the detection head. RMKD\cite{li2022rm} reformulates the ordering between anchors into the form of the probability distribution for knowledge transfer and applies classical knowledge distillation to the regression task. However, it is only limited to anchor-based detectors. 
\subsection{Feature-based knowledge distillation}
% \paragraph{Feature-based knowledge distillation for classification}



\paragraph{For classification.}
The feature of the teacher model is another kind of dark knowledge and was first used in~\cite{romero2014fitnets}. Subsequent works have primarily focused on finding more effective ways to utilize this type of dark knowledge.
    AT\cite{kzagoruyko2016at} extracts attention maps to help the student to pay attention to the import regions. However, it squeezes the channel dimension of the feature map and fails to utilize the channel information, resulting in limited improvements to the student models
    OFD\cite{heo2019overhaul} designs a new loss function and used marginal ReLU to extract the major information in the network. 
    CRD\cite{tian2019crd} incorporates the idea of contrastive learning into knowledge distillation, and although it has achieved relatively good performance, However, its training cost is high due to the use of memory banks for a large amount of negative samples.
    KR\cite{reviewkd} proposes conducting knowledge distillation on multi-level features in a review manner,resulting in good performance, especially on classification tasks.
    TaT~\cite{lin2022tat} propose a novel one-to-all spatial matching approach for knowledge distillation based on similarity generated from a target-aware transformer.
% \paragraph{Feature-based knowledge distillation for dense prediction}


\paragraph{For objection detection.}
% For objection task:
    Object detection is a significantly more complex task than image classification. The extreme imbalance between foreground and background pixels poses a major challenge for object detection. To address this issue, many knowledge distillation methods attempt to have the student model imitate the key regions of the teacher model.
    FGFI\cite{wang2019fgfi} leverages fine-grained masks to force students to focus on foreground regions.
    GID\cite{dai2021GID} identifies regions where the student and teacher models perform differently as the key regions for distillation, without relying on anchors.
    Defeat\cite{guo2021defeat} finds that the background region also contains valuable information and proposes distilling foreground and background regions separately.
    Recent methods have also discovered that the relationships between different pixels are important knowledge for distillation and propose various global modules to address this problem. 
    FKD\cite{zhang2020fkd} employs attention masks to direct the student model's focus on key regions and non-local modules to capture the relationships between different pixels, resulting in improved knowledge distillation.
    FGD\cite{yang2022fgd} proposes focal and global distillation mechanisms, forcing the student model to learn the teacher's crucial region and global information through a global context block~\cite{cao2019gcnet}


\vspace{-4mm}
\paragraph{For semantic segmentation.}
\section{Network Architecture}
\label{sec:arch}

Throughout this section we work with general qudits of local dimension $d$. We first describe the general structure of the class of NoRA networks we consider, then we specialize to a particular network structure inspired by scaling and renormalization group (RG) considerations. We analyze both the entanglement and complexity of the scaling-adapted ground state network and discuss an extension to describe excited states. In particular, we show that a natural choice of energy scales in a toy model Hamiltonian can give rise to a power-law temperature dependence of the thermodynamic entropy and heat capacity.

\subsection{General Structure}

The NoRA network is defined by $L$ layers as in Fig.~\ref{fig:arch}, where we refer to the bottom qudits as \emph{ground state} qudits and the other qudits as \emph{excited state or thermal} qudits. When we set the thermal qudits to some fixed product state, $\ket{0}$, we obtain the \emph{ground state network} as in Fig.~\ref{fig:arch}. This nomenclature is chosen because we can view the network as a variational ansatz for the ground space of a mean-field model. From this point of view, the ground state qudits parameterize a space of states that would be identified with the degenerate ground space of the concrete model of interest.

\begin{figure}[htb]
    \centering
    \scalebox{0.8}{\tikzfig{figures/syk_circuit}}
    \caption{Basic architecture of the proposed NoRA tensor network ansatz. A code word $\ket{\psi_{\textrm{code}}}$ consisting of $n_0 \equiv k$ (logical) \enquote{ground-state} qudits is embedded  as $\ket{\Psi_{\textrm{phys}}}$ in the (physical) ground space of the $d^N$-dimensional many-body Hilbert space by the means of $L$ layers of some given depth $D$ quantum circuits. For each layer $1 \leq \ell \leq L$ the circuit $D_{\ell}$ acts on the $n_{\ell - 1}$ qudit output from the previous layer and an additional $\Delta n_{\ell}$ new ancillary \enquote{thermal} qudits initialized in state $\ket{0}$. We stress that the layer circuits $D_\ell$ do not have to respect locality structure depicted by the 1d arrangement of qudit lines.}
    \label{fig:arch}
\end{figure}

One way to think about the network is as a ``fine-graining'' circuit moving upwards from the bottom ground state qudits. This is the inverse of a conventional RG transformation since we are adding degrees of freedom. We start with $k$ of these ground state qudits. Then at each layer $\ell$ we add $\Delta n_{\ell}$ thermal qudits in the fixed state $\ket{0}^{\otimes \Delta n_{\ell}}$ and apply a depth $D$ quantum circuit to all the qudits in that layer. This circuit could also be generalized to be time evolution with a suitably normalized all-to-all Hamiltonian for a constant time (proportional to $D$). The next layer takes all the qudits from the previous layer and adds more thermal qudits to generate the hierarchical structure in Fig.~\ref{fig:arch}. The total number of qudits at layer $\ell$ is denoted $n_{\ell}$ and given by
\begin{equation}
    n_{\ell} = k + \sum_{\ell'=1}^{\ell} \Delta n_{\ell'}.
\end{equation}
The total number of qudits is therefore
\begin{equation}
    N \equiv n_L = k + \sum_{\ell=1}^L \Delta n_{\ell}.
\end{equation}


\subsection{Scaling Specialization}

As is, we have described a fairly general architecture. Motivated by scaling and renormalization group considerations, we will primarily consider the special case where $n_{\ell} \sim k + r^{\ell}$, so that the number of thermal qudits is increasing exponentially with each layer up from the bottom. Viewing the top layer as the UV or microscopic degrees of freedom and the bottom layer as the IR or emergent degrees of freedom, moving from the UV to IR (top to bottom) mimics a renormalization group transformation where we remove some fraction of the thermal degrees of freedom at each step. Indeed, borrowing the language of MERA and DMERA and viewing the circuit from top to bottom, the individual layers are like disentanglers that leave behind some decoupled degrees of freedom, the thermal qudits added at that layer. In this scheme, we choose the number of qudits at layer $\ell$ to be
\begin{equation}
    k + r^{\ell} \stackrel{!}{=} n_{\ell} = k + \sum_{\ell'=1}^{\ell} \Delta n_{\ell},
\end{equation}
implying that the number of new thermal qudits for each layer must be
\begin{align}
\begin{split}
    \Delta n_{\ell > 1} &= r^{\ell} - r^{\ell - 1}, \\
    \Delta n_{1} &= r.
\end{split}
\end{align}
For the case of $r=2$, which we primarily consider in this work, this simplifies to approximately $\Delta n_{\ell} = 2^{\ell - 1}$ for all layers $\ell$.

\subsection{Entanglement and Complexity}

We next discuss the entanglement and complexity of the RG-inspired network. There are $\text{O}(N)$ non-trivial bonds in the circuit, of which $N$ bonds connect to the same constant-depth circuit in the last layer. It is therefore straightforward to establish that the network has the potential to encode volume law entanglement for sub-regions of a \enquote{typical} UV state. We also explicitly demonstrate that this is achievable within the Clifford model discussed below in Sections~\ref{sec:clifford}, \ref{sec:numerical}, and \ref{sec:code_structure}.

Turning to the complexity, we take the number of gates in the network as an estimate of the circuit complexity of the UV state, although in general this is only an upper bound. For a layer $\ell$ with $n_{\ell}$ total qubits in it, we apply $D$ rounds of $\lfloor n_{\ell}/q \rfloor$ $q$-qudit gates, so the number of gates of layer $\ell$ is
\begin{equation}
   \text{gates at layer } \ell = D \cdot \lfloor n_{\ell}/q \rfloor.
\end{equation}
Summing this result over all layers and assuming that $q$ divides $n_{\ell}$ without remainder gives a total number of gates equal to
\begin{equation}
\label{eq:circuit_complexity}
    \text{total gates} = \frac{D}{q} \sum_{\ell=1}^L n_{\ell} = \frac{D}{q} \left(L \cdot k + \frac{r^{L+1} - r}{r-1} \right).
\end{equation}
In sections \ref{sec:numerical} and \ref{sec:code_structure} we will cast this result into simpler leading-order expressions that correspond to the respective types of ground space scaling being considered.

\subsection{Extension to Excited States}

Let us conclude this section by extending the ground state network we have so far discussed to the case of excited states. As we have repeatedly emphasized, the discussion so far is general and does not consider a particular physical Hamiltonian. We are simply trying to match certain qualitative features of the entanglement and complexity expected for mean-field models. A structure similar to what we will consider here was recently studied for non-interacting fermions and advocated for as a general approach to approximating thermal states~\cite{sewell_thermal_multi_scale_2022}.

The idea is to introduce a toy Hamiltonian for which the above network is an exact ground state for any choice of state on the $k$ ground state qudits. In other words, the toy Hamiltonian has an exactly degenerate ground space. The Hamiltonian is constructed in a standard way by introducing projectors $P = |0\rangle \langle 0|$ for each thermal qudit and defining corresponding projectors acting on the UV qudits by conjugating these elementary projectors with the network circuit. Let $\tilde{P}_i$ denote the projector for thermal qudit $i$ conjugated by the network circuit. The toy Hamiltonian is
\begin{equation}
\label{eq:stab_hamiltonian}
    H = - \sum_i J_i \tilde{P}_i,
\end{equation}
where $J_i$ are a set of free parameters that determine the energy scale associated with each thermal qudit. Note that -- just like the circuit it encodes -- this Hamiltonian is highly non-local and not necessarily few-body, thus limiting the potential for physical interpretation. The setup is described in more detail in appendix \ref{sec:entropy_scaling}. 

Again motivated by RG considerations, in which the energy scale of excitations decreases by a fixed factor after every RG step (top to bottom), we take the $J_i$ to be equal within a layer and to depend on the layer index $\ell$ as
\begin{equation}
\label{eq:energy_scaling}
    J_{\ell} = \Lambda \cdot e^{-\gamma (L-\ell)}.
\end{equation}
In this way, the UV energy scale is $\Lambda$ and the energy of excitations decreases exponentially with the layer index decreasing towards the IR. The free parameter $\gamma$ controls the rate of decrease.

As computed in appendix \ref{sec:entropy_scaling}, the entropy for the Gibbs ensemble associated to said toy Hamiltonian describing our tensor network ansatz (and for general scaling of $J_{\ell}$) is
\begin{equation}
\label{eq:stab_entropy}
    S = \log\left(d^{k} \cdot (d-1)^{\braket{N-k}}\right) + \sum_{\ell} \Delta n_{\ell} \cdot S(p_{\ell}),
\end{equation}
where we defined a probability, 
\begin{equation}
    p_{\ell} = \frac{d - 1}{e^{\beta J_{\ell}} + d - 1},
\end{equation}
$S(p_{\ell})$ is the classical binary entropy function,
\begin{equation}
    S(p_{\ell}) = - p_{\ell} \cdot \log(p_{\ell}) - (1-p_{\ell}) \cdot \log(1-p_{\ell}),
\end{equation}
and 
\begin{equation}
    \braket{N-k} = \sum_i p_i = \sum_{\ell} \Delta n_{\ell} \, p_{\ell}.
\end{equation}
Note that in the case of qubits ($d=2$), $p_{\ell}$ coincides with the ordinary Fermi-Dirac distribution, in which case $\braket{N-k}$ is analogous to a sum of occupation numbers.

Plugging in \eqref{eq:energy_scaling} and going to the low-temperature regime (relative to the energy scale $\Lambda$), \eqref{eq:stab_entropy} can be approximated in the continuum limit as
\begin{align}
\label{eq:stab_entropy_approx}
\begin{split}
    & S - k \cdot \log(d) \\ \lessapprox{}& (d-1) (N-k) \cdot \frac{\alpha}{\gamma} \cdot \Gamma\left(\frac{\alpha}{\gamma} + 1\right) (\beta \Lambda)^{-\alpha/\gamma} \\
    \propto{}& (T/\Lambda)^{\alpha/\gamma},
\end{split}
\end{align}
with $N = k + r^L$ and $\alpha = \log(r)$. This together with the specific example depicted in figure \ref{fig:entropy_scaling} confirms that in this limit the entropy does obey a power law. By choosing the parameters $\alpha$ and $\gamma$ suitable, one could even match the precise low-temperature behavior of the SYK heat capacity $C_V$ (which is proportional to $T$) due to $dS = \frac{C_V}{T} dT$:
\begin{equation}
    C_V = T \left( \frac{dS}{dT} \right) \propto (T/\Lambda)^{\alpha/\gamma}.
\end{equation}

\begin{figure}[htb]
    \centering
    \includegraphics{figures/entropy_scaling.pdf}
    \caption{Logarithmic scaling of the exact Gibbs entropy $S_{\textrm{stab}}$ associated to $H$, and the low-temperature approximation $S_{\textrm{approx}}$ for $L=20$, $r=2$, $k=1$, $d=2$, $\Lambda=1$ and $\gamma = 0.4$. Both match almost exactly for our choice of parameters and small $T/\Lambda$, confirming the existence of a scaling law. The same is also true for other choices of $\gamma$ (the only significant free parameter), as seen in figure \ref{fig:entropy_scaling_appendix}.}
    \label{fig:entropy_scaling}
\end{figure}

\subsection{Summary}

Starting from the general architecture in Figure~\ref{fig:arch}, we introduced the RG-inspired network in which the number of qudits at layer $\ell$ is $k + r^\ell$. In the special case where $k=0$, i.e. a non-degenerate ground space, the number of qudits decreases by a factor from one layer to the next into the IR. This decrease is analogous to a block decimation RG procedure applied to a quantum state. The case of $k\neq 0$ describes a generalization of such an RG procedure. The entanglement entropy of the physical states produced by the RG-inspired network can be volume-law, as expected for mean-field models. We also showed that the ground state network can be extended to provide a model of thermal excitations in which the thermodynamic heat capacity has a power-law temperature dependence at low temperature. These general features are all chosen to match characteristics of the SYK model, which also features a nearly degenerate space of highly entangled ground states and a power-law heat capacity at low temperature.
% For semantic segmentation:
    Semantic segmentation is a per-pixel prediction problem, and strictly aligning the feature maps between the student and teacher models may impose overly strict constraints and lead to sub-optimal results.~\cite{shu2021cwd}. Recent works~\cite{liu2019skds,wang2020ifvd} try to force the student to learn the correlations among different spatial regions.
    IFVD\cite{wang2020ifvd} focuses on the intra-class feature variation among pixels with the same label and designs an IFV module to transfer the structural knowledge. 
    SKDS~\cite{liu2019skds} combines pixel-wise distillation, pair-wise distillation, and holistic distillation using a GAN-based approach to align the output maps of teacher and student models.
    CIRKD\cite{yang2022cirkd} aims to model the pixel-to-pixel and pixel-to-region relationships as supervisory signals for knowledge distillation in the semantic segmentation task.
    CWD~\cite{shu2021cwd} derives probability maps by normalizing the activation maps of each channel of intermediate features, and minimizes the KL divergence between these probability maps, applying the method to dense prediction tasks, including object detection and semantic segmentation. 
\vspace{-4mm}
\paragraph{For general tasks.}
% General:
MGD~\cite{yang2022mgd} employs a generative approach that involves the use of  random masks that randomly to erases a portion of the student's feature map and then forces it to generate features similar to the teacher's through an adversarial generator and applies it to classification, detection, and segmentation tasks.  

In this paper, we focus on channel-wise transformations, and propose a simple and generic method for feature-based knowledge distillation. 
\label{sec:related}


