\documentclass{article}
\usepackage{amsthm}
\usepackage{graphicx} % Required for inserting images
\usepackage{authblk}
\usepackage{longtable}
\usepackage[a4paper,margin=3cm]{geometry}
\usepackage{booktabs,multirow,tabularx} % for better table rules
\usepackage{multicol} % for multiple columns
\usepackage{float}
\usepackage{amsmath, amssymb, mathtools}
\usepackage{supertabular,booktabs}
\usepackage{makecell}
\usepackage{caption}
\usepackage{setspace}
\usepackage{tikz}
\usepackage{qtree}
\usepackage{tikz-qtree}
\usepackage{subfigure}
\usepackage{algpseudocode}
\usepackage{algorithm}
\usepackage{multirow}



\newtheorem*{prop}{Proposition}
\newtheorem*{proofoutline}{Proof (outline)}
\newtheorem*{Cor}{Corollary}

\algnewcommand\algorithmicforeach{\textbf{for each}}
\algdef{S}[FOR]{ForEach}[1]{\algorithmicforeach\ #1\ \algorithmicdo}

\renewcommand\Authands{ and }


\title{The Largest Condorcet Domains on 8 Alternatives}

\author[1]{Charles Leedham-Green}
\author[2]{Klas Markstr{\"o}m}
\author[3]{S\o ren Riis\thanks{Corresponding author}}
\affil[2]{University of Umeå}
\affil[1,3]{Queen Mary University of London}
\date{}
\begin{document}
\maketitle

\begin{abstract}
In this note, we report on a record-breaking Condorcet domain (CD) for n=8 alternatives. We show that there exists a CD of size 224, which is optimal and essentially unique (up to isomorphism). If we consider the underlying permutations and focus on Condorcet domains containing the identity permutation, 56 isomorphic such Condorcet domains exist. Our work sheds light on the structure of CDs and UCDs and has potential applications in voting theory and social choice.

\end{abstract}

\section{Introduction}
\label{sec:intro}
Condorcet domains (CD), which are sets of linear orders characterized by an acyclic pairwise majority relation, has been studied by mathematicians, economists, and mathematical social scientists since the 1950s \cite{danilov2013maximal,fishburn2002acyclic}. Condorcet domains find use in Arrovian aggregation and social choice theory \cite{bruner2014likelihood}. In social choice theory, a Condorcet winner is a candidate who would win over every other candidate in a pairwise comparison by securing the majority of votes \cite{monjardet2005social}. However, the existence of such a candidate is not always guaranteed, leading to the relevance of Condorcet Domains. 
A central question in this field has revolved around identifying "large" Condorcet domains 
Fishburn, Gamlambos \& Reiner, Monjardet, Danilov \& Karzanov, Karpov, Slinko, \& Arkadii
\cite{fishburn1997acyclic,galambos2008acyclic, monjardet2009acyclic, danilov2012condorcet, puppe2022maximal,karpov2022structured}.

 A significant category of Condorcet domains is rooted in Fishburn's alternating scheme, which alternates between two restriction rules on a subset of candidates and has been employed to construct numerous maximum size Condorcet domains. We refer to such domains based on the alternating scheme as Fishburn domains.
 
 Fishburn introduced a function $f(n)$ in \cite{fishburn1997acyclic}, defined to be the maximum size of a Condorcet domain on a set of $n$ alternatives, posed the problem of determining the growth rate for $f(n)$. This was followed by further research and bounds on $f(n)$ by Gamlambos \& Reiner, Danilov \& Karzanov, and Monjardet \cite{galambos2008acyclic,danilov2012condorcet,monjardet2009acyclic}. Karpov \& Slinko extended and refined this work in \cite{karpov2022symmetric} and Zhou \& Riis \cite{zhou2023new}.

Although extensive research has been carried out, all known maximum-sized Condorcet domains have been built using components based on either Fishburn's alternating schemes or his replacement scheme. For instance, Karpov, Slinko, \& Arkadii \cite{karpov2022constructing} introduced a novel construction that enabled the creation of new Condorcet domains with unprecedented sizes. This allowed the authors to construct a Condorcet Domain, superseding the size of Fishburn's domain for 13 alternatives. Recently, Zhou \& Riis \cite{zhou2023new} constructed Condorcet domains on 10 and 11 alternatives, superseding the size of the corresponding Fishburn domains. 

In this paper, we show that even for as few as 8 alternatives, the Fishburn domain (size 222) is not the largest and that there is a Condorcet Domain of size 224. Furthermore, relying on extensive computer calculation on the super-computer Abisko at Umeå, we also established 224 as an upper bound and that there, up to isomorphism, is only one such Condorcet Domain.

\section{Preliminaries}
\label{sec:preliminaries}

There are many equivalent definitions of Condorcet Domains.
In this paper, we adopted the definition proposed by Ward in~\cite{ward65}. According to this definition, a Condorcet Domain of degree $n\geq 3$ is a set of orderings of $X_n$ that satisfies certain local conditions. 

Specifically, a Condorcet Domain of degree $n=3$ is defined as a set of orderings of $X_3$ that satisfies one of nine laws, denoted by $x$N$i$, where $x$ is an element of $X$, and $i$ is an integer between 1 and 3. The law $x$N$i$ requires that $x$ does not come in the $i$-th position in any order in the Condorcet Domain. For example, $x$N$1$ means that $x$ may never come first, while $x$N$3$ means that $x$ may never come last.

A Condorcet Domain of degree $n>3$ is a set $A$ of orderings of $X_n$ that satisfies the following property: the restriction of $A$ to every subset of $A$ of size 3 is a Condorcet Domain. In other words, for every triple ${a,b,c}$ of elements of $X$, one of the nine laws $x$N$i$ must be satisfied, where $x\in{a,b,c}$. For example, $c$N$2$ would mean that $c$ may not come between $a$ and $b$ in any orderings in $A$.

A Maximal Condorcet Domain of degree $n$ is a Condorcet Domain of degree $n$ that is maximal under inclusion among the set of all Condorcet Domains of degree $n$. A Unitary Condorcet Domain contains the identity order, and a Maximal Unitary Condorcet Domain is a Condorcet Domain that is maximal among all Unitary Condorcet Domains. Since the class of Condorcet domains of degree $n$ is closed under taking subsets, a Maximal Unitary Condorcet Domain is also a Maximal Condorcet Domain.

To avoid repetition, we will use the acronyms CD, MCD, UCD, and MUCD to refer to Condorcet Domain, Maximal Condorcet Domain, Unitary Condorcet Domain, and Maximal Unitary Condorcet Domain, respectively.

For the case of degree 3, there are nine Maximal Condorcet Domains, each corresponding to one of the nine different laws $x$N$i$. It is easy to verify that these nine Maximal Condorcet Domains contain exactly four elements: odd transpositions and two even permutations (either the identity or a 3-cycle). Among the 9 maximal Condorcet Domains of order 3, precisely six are Unitary Condorcet Domains since the laws 1N1, 2N2, and 3N3 each rule out a CD of degree 3.

\subsection{Transformations and isomorphism of Condorcet domains}

Let $g$ be a permutation and $i$ an integer. We define $ig$ as $g(i)$ and $Ag$ as the set obtained by applying $g$ to each element of a set $A$ of integers. If $A$ is a CD and $g$ is any permutation in $S_n$, then $Ag$ is also a CD. Specifically, if $A$ satisfies the law $x$N$i$ on a triple ${a,b,c}$ for some $x\in{a,b,c}$, then $Ag$ satisfies the law $xg$N$i$ on the triple ${ag,bg,cg}$. We call CDs $A$ and $Ag$ \emph{isomorphic}. Therefore, two isomorphic CDs differ only by a relabeling the elements of $X_n$.

The \emph{core} of a CD $A$ is the set of permutations $g\in A$ such that $Ag=A$. The core of $A$ is a group. Every CD is isomorphic to a Unitary Condorcet Domain (UCD). Two UCDs $A$ and $B$ are isomorphic if $Ag^{-1}=B$ for some $g$ in $A$. When we refer to an isomorphism class of UCDs, we mean the set of UCDs in an isomorphism class of CDs. Thus, if $A$ is a UCD of size $m$ with a core of size $k$, then $k$ divides $m$, and the isomorphism class of $A$ as a UCD has size $m/k$.



\begin{table}[H]
    \renewcommand{\arraystretch}{1.3}
    \centering
    \begin{tabular}{cccc}
    \toprule
   \textbf{Triplet} & \textbf{Rule assigned} & \textbf{Condorcet domains} & \\

    \midrule
    \multirow{6}{*}{(i, j, k)}
    & 1N3 &
    \multirow{2}{*}{
    $\begin{rcases*}
        \begin{tabular}{cccc}
             ijk & jki & jik & kij   \\
             ijk & jik & jki & kji \\
        \end{tabular}  
     \end{rcases*}$ 
    } & \multirow{2}{*}{equivalent}\\
    & 2N3 & \\

    \cmidrule(l){2-4}

    & 3N1 &
    \multirow{2}{*}{
    $\begin{rcases*}
        \begin{tabular}{cccc}
             ijk & ikj & jik & jki  \\
              ijk & ikj & kij & kji\\
        \end{tabular}  
     \end{rcases*}$ 
    } & \multirow{2}{*}{equivalent}\\
    & 2N1 & \\

    \cmidrule(l){2-4}

    & 1N2 &
    \multirow{2}{*}{
    $\begin{rcases*}
        \begin{tabular}{cccc}
             ijk & ikj & jki & kji \\
             ijk & jik & kij & kji\\
        \end{tabular}  
     \end{rcases*}$ 
    } & \multirow{2}{*}{equivalent}\\
    & 3N2 & \\
    
    \bottomrule
    
\end{tabular}
    \caption{The Condorcet domains for 3 alternatives. Each rule assigned to the triplet (i, j, k) with i$<$j$<$k is associated with a CD. The CDs displayed fall into 3 equivalence classes. }
    \label{tab:cd_3}
\end{table}

Table \ref{tab:cd_3} is essentially from Zhou et al.'s \cite{zhou2023new}. It can be readily shown that for any Condorcet Domain, the total number of 1N3 and 2N3 rules remains invariant under isomorphism. Likewise, this holds for the total number of 2N1 and 3N1 rules and the total number of 1N2 and 3N2 rules.
%  A Condorcet domain's size is defined by the number of permutations within it. Our work aims to identify large Condorcet domains, a combinatorial optimization problem.
 


% The 5 alternatives have 10 triplets, as demonstrated in Table \ref{table:triplets_rules_5}. Their rules are assigned by applying the alternating scheme. A CD created by the alternating scheme is called the alternating scheme domain. The alternating scheme domain for 5 alternatives is size 20, as shown in Table \ref{tab:cd_5}, which is the largest CD for 5 alternatives. 

% \begin{table}[H]
% \captionsetup{width=.8\textwidth}
% \centering
% \begin{tabular}{lc}
% \toprule
% \textbf{Triplets} & \textbf{Rules}\\
% \midrule
% (1, 2, 3) & 2N3 \\
% (1, 2, 4) & 2N3 \\
% (1, 2, 5) & 2N3 \\
% (1, 3, 4) & 2N1 \\
% (1, 3, 5) & 2N1 \\
% \bottomrule
% \end{tabular}
% \hspace{1em}
% \begin{tabular}{lc}
% \toprule
% \textbf{Triplets} & \textbf{Rules}\\
% \midrule
% (1, 4, 5) & 2N3 \\
% (2, 3, 4) & 2N1 \\
% (2, 3, 5) & 2N1 \\
% (2, 4, 5) & 2N3 \\
% (3, 4, 5) & 2N3 \\
% \bottomrule
% \end{tabular}
% \caption{The list of triplets for 5 alternatives and their rules are assigned as per the alternating scheme.}
% \label{table:triplets_rules_5}
% \end{table}

% \begin{table}[h]
% \captionsetup{width=.8\textwidth}
% \centering
% \begin{tabular}{cccc}
% \toprule
% \multicolumn{4}{c}{$n=5$ alternating scheme domain} \\
% \midrule
% 12453 & 12435 & 12345 & 54321 \\
% 45321 & 54231 & 45231 & 42531 \\
% 24531 & 54213 & 45213 & 42513 \\
% 42153 & 42135 & 24513 & 24153 \\
% 24135 & 21453 & 21435 & 21345 \\

% \bottomrule
% \end{tabular}
% \caption{The resulting Condorcet domain consisting of 20 permutations built from the triplet and rules in Table \ref{table:triplets_rules_5} for 5 alternatives. }
% \label{tab:cd_5}
% \end{table}


% A Condorcet domain can be restricted to $k\in[3, n-1]$ element subset CDs by removing the predetermined  $(n-k)$ elements in the range of 1 to $n$ from permutations and retaining only one of the duplicated resulting permutations. The resulting subset CD contains the permutations that only  contain $k$ elements in the range of 1 to $n$.

% Analogously, a list of triplets for $n$ alternatives can be restricted to $k\in[3, n-1]$ subset triplets by retaining the triplets that only contain $k$ elements in the range of 1 to $n$, which is equivalent to removing the triplets that contain least one of $(n-k)$ elements in that range. The rules are transferred from the original triplets to the subset triplets. A set of triplets for $n$ alternative, when restricted to $k$ elements subset, produces $\frac{n!}{m!(n-m)!}$ sets of subset triplets. For example, a set of 11 alternatives produces 10 sets of subset triplets when restricted to 10 element subsets and 45 sets of subsets when restricted to 9 element subsets. The Condorcet domain constructed from the triplets and rules in Table \ref{table:triplets_rules_5}, when restricted to 4 elements subset, yields 5 subset CDs for 4 alternatives, whose sizes are 9, 9, 8, 9, 9 as demonstrated in Table \ref{tab:subset_sizes_5}. Each resulting subset CD has one element removed from the original CD. For example, the CD that appears by removing the alternative one from all the permutations contains 9 permutations. 

% \begin{table}[H]
%     \centering
%     \begin{tabular}{clcccc}
%     \toprule
%     \textbf{Removed element} & \multicolumn{5}{c}{\textbf{Subset CD}}\\

%     \midrule
%     \multirow{2}{*}{1} & 4253 & 4523 & 5432 & 2345 & 5423 \\
%     & 2453 & 4532 & 4235 & 2435 \\

%     \midrule
%     \multirow{2}{*}{2} & 4531 & 5413 & 4513 & 5431 & 4135 \\ 
%     & 1453 & 4153 & 1345 & 1435\\ 
    
%     \midrule
%     \multirow{2}{*}{3} &4521 & 4215 & 2145 & 2415 & 1245 \\ 
%     & 4251 & 5421 & 2451\\ 

%  \midrule
%     \multirow{2}{*}{4} & 1235 & 2513 & 2153 & 2135 & 2531 \\ 
%     & 1253 & 5213 & 5231 & 5321\\

%      \midrule
%     \multirow{2}{*}{5} & 2143 & 2134 & 1234 & 2431 & 4213 \\
%     & 4231 & 2413 & 1243 & 4321 \\

    
%     \bottomrule
% \end{tabular}
%     \caption{The subset CD sizes for 4 alternatives for the 5 alternatives CD.}
%     \label{tab:subset_sizes_5}
% \end{table}

% As the size of CDs is of interest, we will use the lattice structure in this paper to illustrate the size of the original CD and the sizes of their subset CDs, as shown in Figure \ref{fig:n_5_subsets}. The numbers sitting aside the arrows indicate that the subset CD is built by eliminating the permutations that contain that number from the original CD. 

% \begin{figure}[H]
%     \centering
%     \begin{tikzpicture}[scale=1.5]
    
%     \centering
%     \node[] (0) at (0,0) {\textbf{20}};
    
%     \node[] (1) at (-2,-1) {\textbf{9}} ;
%     \node[] (2) at (-1,-1) {\textbf{9}};
%     \node[] (3) at (0,-1) {\textbf{8}};
%     \node[] (4) at (1,-1) {\textbf{9}};
%     \node[] (5) at (2,-1) {\textbf{9}};

%     \draw[->] (0.south) -- (1.north) node[near end, above]{1};
%     \draw[->] (0.south) -- (2.north) node[near end, above]{2};
%     \draw[->] (0.south) -- (3.north) node[near end, left]{3};
%     \draw[->] (0.south) -- (4.north) node[near end, left]{4};
%     \draw[->] (0.south) -- (5.north) node[near end, above]{5};
  

%     \end{tikzpicture}
%     \caption{The 5 subset CD sizes for the 5 alternative CD of size 20.}
%     \label{fig:n_5_subsets}
% \end{figure}

% The distance between two permutations is the minimal number of neighbour inversions from one permutation to the other. An edge connects two permutations with minimal distance (\textit{i.e.} that can be made identical by swapping two neighbour elements in one of the two permutations).
% A median graph \cite{puppe2019condorcet} is a bidirectional graph formed by all the permutations in a Condorcet domain as nodes and their connections as edges.
% The width of the Condorcet domain is the longest possible distance between two permutations in the median graph. Figure \ref{fig:n_4_width} shows a CD for 4 alternatives. Each connected permutation only differs at two neighbour positions and will be the same if the elements at these positions are exchanged. For instance, 1243 and 1234 will be equivalent if elements 3 and 4 are swapped in one of them. The width of this graph is, by definition 6 as there are 6 edges between permutation 1234 and 4321.  

% \begin{figure}[H]
%     \centering
%     \begin{tikzpicture}[scale=1.5]
    
%     \centering
%     \node[] (2413) at (0,0) {\textbf{2413}};
    
%     \node[] (1234) at (-3,0) {\textbf{1234}};
%     \node[] (1243) at (-2,1) {\textbf{1243}};
%     \node[] (2134) at (-2,-1) {\textbf{2134}};
%     \node[] (2143) at (-1,0) {\textbf{2143}};

%     \node[] (2431) at (1,1) {\textbf{2431}};
%     \node[] (4213) at (1,-1) {\textbf{4213}};
%     \node[] (4231) at (2,0) {\textbf{4231}};
%     \node[] (4321) at (3,0) {\textbf{4321}};

%     \draw[] (2413.west) -- (2143.east);
%     \draw[] (1234.east) -- (1243.south);
%     \draw[] (1234.east) -- (2134.north);
%     \draw[] (1243.south) -- (2143.west);
%     \draw[] (2134.north) -- (2143.west);
  
%     \draw[] (2413.east) -- (2431.south);
%     \draw[] (2413.east) -- (4213.north);
%     \draw[] (2431.south) -- (4231.west);
%     \draw[] (4213.north) -- (4231.west);
%     \draw[] (4231.east) -- (4321.west);

%     \end{tikzpicture}
%     \caption{The median graph for 4 alternatives. The distance between permutations 1234 and 4321 is the longest among all pairs of permutations. The shortest path between them has a length of 6, so the width of this graph is 6.}
%     \label{fig:n_4_width}
% \end{figure}

\section{Search methodology}
We developed an algorithm to generate all maximal MUCDs of a given degree $n$ and size at least equal to a user specified  cutoff value (e.g. size $\geq 217$ for $n=8$).   We implemented this algorithm in C in a serial version sufficient for $n\leq 6$ and a parallelized version that we used for $n=7$ and $8$.

It is important to stress that this algorithm - unlike the one used by Zhou \& Riis \cite{zhou2023new} - aims to establish optimal upper bounds. 

Very briefly, and avoiding various technical details which can be found in \cite{n7paper}, our algorithm starts by arranging the ${n\choose 3}$ triples of integers in $\{1,2,\ldots,n\}$ in some fixed order. We then define the full Condorcet tree as a homogeneous rooted tree of depth ${n \choose 3}$, where every non-leaf has six descendants. Each vertex is associated with a closed permutation set, and every edge joining a vertex of depth $t$ to a vertex of depth $t+1$ is associated with one of the six laws that may be applied to the $t$-th triple.
To make the computation feasible, we define a reduced Condorcet tree obtained from the whole tree by only expanding vertices that are not already a maximal unitary CD (MUCD) for some smaller triple. We also define $t$-UCDs and $t$-MUCDs as permutation sets that satisfy some Condorcet law for every triple $s<t$, with $t$-MUCDs being the maximal $t$-UCD permutation sets. Using this, we prove that every $t$-MUCD is associated with a vertex in the reduced Condorcet tree whose parent is associated with a $(t-1)$-MUCD. Therefore, we only need to search the subtree of the reduced Condorcet tree whose associated permutation sets are $t$-MUCDs for the appropriate $t$.
To check if a permutation set associated with a vertex and triple satisfies the $t$-MUCD condition, we compute the set of laws that the permutation set satisfies on all triples $s<t$, the corresponding principal closed permutation sets, and the intersection of those sets. If the permutation set equals the intersection, it is a $t$-MUCD.
Regarding technical details, we represent subsets of $S_n$ as bit-strings of length $n!$, and all computations with sets of permutations are carried out using bit operations.


\section{Condorcet domains on 8 alternatives with size 224}
\label{sec:10}
Even for as few as 8 alternatives, the Fishburn domain (size 222) is not the largest, and there is a Condorcet Domain of size 224. Furthermore, relying on extensive computer calculation on the super-computer Abisko at Umeå, we have  established that: 

\begin{prop}
The maximum size of a CD on 8 alternatives is 224. Up to isomorphism, there is only one such CD, but as unitary CDs (UCD), there are 56 isomorphic UCDs. Each of these UCDs has a core of size $4$.
There are no MUCDs of size 223.

The largest Condorcet Domain containing the identity permutation and its reverse is the Fishburn domain on 8 alternatives, which has a size of 222. 
\end{prop}

A longer paper with a more detailed explanation of our search procedure and more precise counts and analysis of other large Condorcet Domains on 8 alternatives is in preparation. 

There are 56 (isomorphic) Unitary Condorcet Domains of size 224. Here is one special MUCD we will refer to as D224, where each never-rule - except for the two triplets (123) and (678) - is 1N3 or 3N1.

\begin{table}[H]
\centering
\begin{tabular}{lc}
\toprule
\textbf{Triplets} & \textbf{Rules}\\
\midrule
(1, 2, 3) & 2N3 \\
(1, 2, 4) & 1N3 \\
(1, 2, 5) & 3N1 \\
(1, 2, 6) & 3N1 \\
(1, 2, 7) & 3N1 \\
(1, 2, 8) & 3N1 \\
(1, 3, 4) & 1N3 \\
(1, 3, 5) & 3N1 \\
(1, 3, 6) & 3N1 \\
(1, 3, 7) & 3N1 \\
(1, 3, 8) & 3N1 \\
(1, 4, 5) & 1N3 \\
(1, 4, 6) & 1N3 \\
(1, 4, 7) & 1N3 \\


\bottomrule
\end{tabular}
\hspace{1em}
\begin{tabular}{lc}
\toprule
\textbf{Triplets} & \textbf{Rules}\\
\midrule
(1, 4, 8) & 1N3 \\
(1, 5, 6) & 1N3 \\
(1, 5, 7) & 1N3 \\
(1, 5, 8) & 3N1 \\
(1, 6, 7) & 3N1 \\
(1, 6, 8) & 1N3 \\
(1, 7, 8) & 1N3 \\
(2, 3, 4) & 1N3 \\
(2, 3, 5) & 1N3 \\
(2, 3, 6) & 1N3 \\
(2, 3, 7) & 1N3 \\
(2, 3, 8) & 1N3 \\
(2, 4, 5) & 3N1 \\
(2, 4, 6) & 3N1 \\


\bottomrule
\end{tabular}
\hspace{1em}
\begin{tabular}{lc}
\toprule
\textbf{Triplets} & \textbf{Rules}\\
\midrule
(2, 4, 7) & 3N1 \\
(2, 4, 8) & 3N1 \\
(2, 5, 6) & 1N3 \\
(2, 5, 7) & 1N3 \\
(2, 5, 8) & 3N1 \\
(2, 6, 7) & 3N1 \\
(2, 6, 8) & 1N3 \\
(2, 7, 8) & 1N3 \\
(3, 4, 5) & 3N1 \\
(3, 4, 6) & 3N1 \\
(3, 4, 7) & 3N1 \\
(3, 4, 8) & 3N1 \\
(3, 5, 6) & 1N3 \\
(3, 5, 7) & 1N3 \\
\bottomrule
\end{tabular}
\hspace{1em}
\begin{tabular}{lc}
\toprule
\textbf{Triplets} & \textbf{Rules}\\
\midrule


(3, 5, 8) & 3N1 \\
(3, 6, 7) & 3N1 \\
(3, 6, 8) & 1N3 \\
(3, 7, 8) & 1N3 \\
(4, 5, 6) & 1N3 \\
(4, 5, 7) & 1N3 \\
(4, 5, 8) & 3N1 \\
(4, 6, 7) & 3N1 \\
(4, 6, 8) & 1N3 \\
(4, 7, 8) & 1N3 \\
(5, 6, 7) & 3N1 \\
(5, 6, 8) & 3N1 \\
(5, 7, 8) & 3N1 \\
(6, 7, 8) & 2N1 \\
\bottomrule
\end{tabular}

\caption{Table of triplets and rules that produces the Condorcet domain D224 of size 224 for 8 alternatives. This specific CD is invariant under
the action by the permutations group $G =\{{\rm id}, (12)(34), (56)(78), (12)(34)(56)(78)\}$}
\label{table:triplets_rules_8}
\end{table}

\begin{table}[H]
  \hrule
  \vspace{1ex}
  \begin{center}
    \textbf{Condorcet Domain with 224 Permutations for 8 Alternatives}    
  \end{center}
  \vspace{-2ex}
  \hrule 
  \vspace{2ex}
\begin{spacing}{1}
% \begin{table}[H]
% \centering
% \begin{spacing}{1}
% \toprule
% \textbf{Condorcet Domain} \\
% \midrule
\noindent 
\underline{12345678}
12345687 
12345867 
12345876 
12346578 
\underline{12346587} 
12346758 
12346785 
12354678 
12354687 
12354867 
12354876  
12358467  
12358476 
12364578 
12364587 
12364758 
12364785 
12367458 
12367485 
12435678 
12435687 
12435867 
12435876 
12436578 
12436587 
12436758 
12436785 
12453678 
12453687 
12453867 
12453876 
12458367 
12458376 
12463578 
12463587 
12463758 
12463785 
12467358 
12467385 
14235678
14235687
14235867
14235876
14236578
14236587
14236758
14236785
14253678
14253687
14253867
14253876
14258367
14258376
14263578
14263587
14263758
14263785
14267358
14267385
14523678
14523687
14523867
14523876
14528367
14528376
14582367
14582376
14623578
14623587
14623758
14623785
14627358
14627385
14672358
14672385
21345678
21345687
21345867
21345876
21346578
21346587
21346758
21346785
21354678
21354687
21354867
21354876
21358467
21358476
21364578
21364587
21364758
21364785
21367458
21367485
\underline{21435678}
21435687
21435867
21435876
21436578
\underline{21436587}
21436758
21436785
21453678
21453687
21453867
21453876
21458367
21458376
21463578
21463587
21463758
21463785
21467358
21467385
23145678
23145687
23145867
23145876
23146578
23146587
23146758
23146785
23154678
23154687
23154867
23154876
23158467
23158476
23164578
23164587
23164758
23164785
23167458
23167485
23514678
23514687
23514867
23514876
23518467
23518476
23581467
23581476
23614578
23614587
23614758
23614785
23617458
23617485
23671458
23671485
32145678
32145687
32145867
32145876
32146578
32146587
32146758
32146785
32154678
32154687
32154867
32154876
32158467
32158476
32164578
32164587
32164758
32164785
32167458
32167485
32514678
32514687
32514867
32514876
32518467
32518476
32581467
32581476
32614578
32614587
32614758
32614785
32617458
32617485
32671458
32671485
41235678
41235687
41235867
41235876
41236578
41236587
41236758
41236785
41253678
41253687
41253867
41253876
41258367
41258376
41263578
41263587
41263758
41263785
41267358
41267385
41523678
41523687
41523867
41523876
41528367
41528376
41582367
41582376
41623578
41623587
41623758
41623785
41627358
41627385
41672358
41672385
\end{spacing}
\caption{Permutation in Condorcet domain corresponding to the rules in table \ref{table:triplets_rules_8}} The CD's core consists of the underlined permutations 12345678, 12346587,21435678 and 21436587.
\label{table:triplets_rules_8B}
\end{table}

For a given Condorcet domain (CD) of order n on alternatives $\{1,2,\ldots,n\}$, we can consider the induced CD on each $k$-element subset $A \subseteq \{1,2,\ldots,n\}$. We have compiled tables that list the size of these induced Condorcet domains for different values of $k$. Specifically, Table \ref{tab:subset_sizes_8} presents the count of the sizes of the induced CDs for $k=4,5,\ldots,7$. In particular, we note that the maximum size CD for $n=8$ is not an extension of a maximum size CD for $n=7$.

\begin{table}[H]
    \centering
    \begin{tabular}{clccccccccccc}
    \toprule
    \textbf{Subset n} & \multicolumn{12}{c}{\textbf{Statistics of the subset CD sizes}}\\

    \midrule
    \multirow{2}{*}{3} & size & 4 \\
    & count & 56\\ 

    \midrule
    \multirow{2}{*}{4} & size & 8 & 9 \\
    & count & 34 & 36 \\ 
    
    \midrule
    \multirow{2}{*}{5} & size & 18 & 19 & 20 \\
    & count & 14 & 16 & 18 \\ 

 \midrule
    \multirow{2}{*}{6} & size & 40 & 42 & 44   \\
& count & 8 & 16 & 4  \\

     \midrule
    \multirow{2}{*}{7} & size & 96  \\
&count& 8 \\
    
    \bottomrule
\end{tabular}
    \caption{Restricted CD sizes}
    \label{tab:subset_sizes_8}
\end{table}

Our analysis of 7-element subsets of alternatives revealed that all 8 subsets have a Condorcet domain (CD) size of 96, which is just below the maximum possible size of 100. These findings are presented in Table \ref{tab:subset_sizes_7}, highlighting the relationship between the MUCD D224 data and those for the Fishburn domain.

\begin{table}[H]
\centering
\begin{tabular}{lcc}
\toprule
\textbf{7-element} & \textbf{D224}  & \textbf{AS222}\\
\textbf{subset} & \textbf{CD size} & \textbf{CD size} \\
\midrule

\{1, 2, 3, 4, 5, 6, 7\} & 96  & 100 \\
\{1, 2, 3, 4, 5, 6, 8\} & 96  & 100  \\
\{1, 2, 3, 4, 5, 7, 8\} & 96  &  96 \\
\{1, 2, 3, 4, 6, 7, 8\} & 96  &  97 \\
\{1, 2, 3, 5, 6, 7, 8\} & 96  & 97 \\
\{1, 2, 4, 5, 6, 7, 8\} & 96  & 96 \\
\{1, 3, 4, 5, 6, 7, 8\} & 96  & 100 \\
\{2, 3, 4, 5, 6, 7, 8\} & 96  & 100\\ 
\bottomrule
\end{tabular}
 \caption{Restrictions of the specific size 224 MUCD, D224 to 7 alternatives. For comparison, we include the size of the restrictions for Fishbun's domain (we denote AS222) based on the Alternating Scheme of size 222}
 \label{tab:subset_sizes_7}
\end{table}

When considering 6-element subsets of alternatives, we have found that the CD sizes fall into three possible values: 40, 42, or 44. As a comparison, the maximum CD on 6 alternatives has size 45. Further details and information on the specific subsets of alternatives that yield each CD size can be found in Table \ref{table:triplets_rules_8C}.
Similarly, restrictions to CD on 5-element subsets of alternatives have sizes 18,19 or 20. 
More details and information on the specific subsets of alternatives that yield each CD size can be found in Table \ref{table:triplets_rules_8D}. For comparison, we also list the size of the restrictions of the Fishburn domains for n=8. 
The restrictions to 4-element subsets have sizes 8 or 9, as seen in Table \ref{table:triplets_rules_8E}.

Finally, it turns out that the restrictions to 3-element subsets all have size 4. This is equivalent to the D224 having each triple satisfy exactly one never rule. Notice that the property is invariant under isomorphism, so the property is automatically satisfied by all 56 equivalent MUCS of size 224.  

\begin{table}[H]
% \centering
\setlength{\tabcolsep}{2pt} 
\begin{tabular}{lcc}
\toprule
\textbf{6-element} & \textbf{D224} & \textbf{AS222}\\
\textbf{subset} & \textbf{CD size} & \textbf{CD size} \\
\midrule

\{1, 2, 3, 4, 5, 6\} & 40 & 45 \\
\{1, 2, 3, 4, 5, 7\} & 40 & 45 \\
\{1, 2, 3, 4, 5, 8\} & 44 & 45 \\
\{1, 2, 3, 4, 6, 7\} & 44 & 42 \\
\{1, 2, 3, 4, 6, 8\} & 40 & 42 \\
\{1, 2, 3, 4, 7, 8\} & 40 & 45 \\
\{1, 2, 3, 5, 6, 7\} & 42 & 44 \\ 
\{1, 2, 3, 5, 6, 8\} & 42 & 44 \\ 
\{1, 2, 3, 5, 7, 8\} & 42 & 39 \\ 
\{1, 2, 3, 6, 7, 8\} & 42 & 45 \\ 
\{1, 2, 4, 5, 6, 7\} & 42 & 42 \\ 
\{1, 2, 4, 5, 6, 8\} & 42 & 42 \\ 
\{1, 2, 4, 5, 7, 8\} & 42 & 42\\
\{1, 2, 4, 6, 7, 8\} & 42 & 39\\
\bottomrule
\end{tabular}
\hspace{1em}
\setlength{\tabcolsep}{2pt} 
\begin{tabular}{lcc}
\toprule
\textbf{6-element} & \textbf{D224} & \textbf{AS222}\\
\textbf{subset}  & \textbf{CD size} & \textbf{CD size} \\
\midrule

\{1, 2, 5, 6, 7, 8\} & 40 & 45\\
\{1, 3, 4, 5, 6, 7\} & 42 & 45\\
\{1, 3, 4, 5, 6, 8\} & 42 & 45\\
\{1, 3, 4, 5, 7, 8\} & 42 & 42\\
\{1, 3, 4, 6, 7, 8\} & 42 & 44\\
\{1, 3, 5, 6, 7, 8\} & 40 & 42\\
\{1, 4, 5, 6, 7, 8\} & 44 & 45\\
\{2, 3, 4, 5, 6, 7\} & 42 & 45\\
\{2, 3, 4, 5, 6, 8\} & 42 & 45\\
\{2, 3, 4, 5, 7, 8\} & 42 & 42\\
\{2, 3, 4, 6, 7, 8\} & 42 & 44\\
\{2, 3, 5, 6, 7, 8\} & 44 & 42\\
\{2, 4, 5, 6, 7, 8\} & 40 & 45\\
\{3, 4, 5, 6, 7, 8\} & 40 & 45\\
\bottomrule
\end{tabular}

\caption{Restrictions to 6 alternatives}
\label{tab:subset_sizes_6}
\label{table:triplets_rules_8C}
\end{table}


\begin{table}[H]
\centering
\setlength{\tabcolsep}{2pt} 
\begin{tabular}{lcc}
\toprule
\textbf{5-element} & \textbf{D224} & \textbf{AS222} \\
\textbf{subset}  & \textbf{CD size} & \textbf{CD size} \\
\midrule
\{1, 2, 3, 4, 5\} & 20 & 20 \\
\{1, 2, 3, 4, 6\} & 20 & 20 \\
\{1, 2, 3, 4, 7\} & 20 & 20 \\
\{1, 2, 3, 4, 8\} & 20 & 20 \\
\{1, 2, 3, 5, 6\} & 18 & 19 \\
\{1, 2, 3, 5, 7\} & 18 & 20 \\
\{1, 2, 3, 5, 8\} & 19 & 20 \\
\{1, 2, 3, 6, 7\} & 19 & 20 \\
\{1, 2, 3, 6, 8\} & 18 & 20 \\
\{1, 2, 3, 7, 8\} & 18 & 19 \\
\{1, 2, 4, 5, 6\} & 18 & 19 \\
\{1, 2, 4, 5, 7\} & 18 & 19 \\
\{1, 2, 4, 5, 8\} & 19 & 19 \\
\{1, 2, 4, 6, 7\} & 19 & 16 \\
\{1, 2, 4, 6, 8\} & 18 & 16 \\
\{1, 2, 4, 7, 8\} & 18 & 19 \\
\{1, 2, 5, 6, 7\} & 18 & 20 \\
\{1, 2, 5, 6, 8\} & 18 & 20 \\
\{1, 2, 5, 7, 8\} & 18 & 19 \\
\{1, 2, 6, 7, 8\} & 18 & 19 \\
\{1, 3, 4, 5, 6\} & 18 & 20 \\
\{1, 3, 4, 5, 7\} & 18 & 20 \\
\{1, 3, 4, 5, 8\} & 19 & 20 \\
\{1, 3, 4, 6, 7\} & 19 & 19 \\
\{1, 3, 4, 6, 8\} & 18 & 19 \\
\{1, 3, 4, 7, 8\} & 18 & 20 \\
\{1, 3, 5, 6, 7\} & 18 & 19 \\
\{1, 3, 5, 6, 8\} & 18 & 19 \\



\bottomrule
\end{tabular}
\hspace{1em}
\setlength{\tabcolsep}{2pt} 
\begin{tabular}{lcc}
\toprule
\textbf{5-element} & \textbf{D224} & \textbf{AS222} \\
\textbf{subset}  & \textbf{CD size} & \textbf{CD size} \\
\midrule
\{1, 3, 5, 7, 8\} & 18 & 16 \\
\{1, 3, 6, 7, 8\} & 18 & 20 \\
\{1, 4, 5, 6, 7\} & 19 & 20 \\
\{1, 4, 5, 6, 8\} & 19 & 20 \\
\{1, 4, 5, 7, 8\} & 19 & 19 \\
\{1, 4, 6, 7, 8\} & 19 & 19 \\
\{1, 5, 6, 7, 8\} & 20 & 20 \\
\{2, 3, 4, 5, 6\} & 18 & 20 \\
\{2, 3, 4, 5, 7\} & 18 & 20 \\
\{2, 3, 4, 5, 8\} & 19 & 20 \\
\{2, 3, 4, 6, 7\} & 19 & 19 \\
\{2, 3, 4, 6, 8\} & 18 & 19 \\
\{2, 3, 4, 7, 8\} & 18 & 20 \\
\{2, 3, 5, 6, 7\} & 19 & 19 \\
\{2, 3, 5, 6, 8\} & 19 & 19 \\
\{2, 3, 5, 7, 8\} & 19 & 16 \\
\{2, 3, 6, 7, 8\} & 19 & 20 \\
\{2, 4, 5, 6, 7\} & 18 & 20 \\
\{2, 4, 5, 6, 8\} & 18 & 20 \\
\{2, 4, 5, 7, 8\} & 18 & 19 \\
\{2, 4, 6, 7, 8\} & 18 & 19 \\
\{2, 5, 6, 7, 8\} & 20 & 20 \\
\{3, 4, 5, 6, 7\} & 18 & 20 \\
\{3, 4, 5, 6, 8\} & 18 & 20 \\
\{3, 4, 5, 7, 8\} & 18 & 19 \\
\{3, 4, 6, 7, 8\} & 18 & 19 \\
\{3, 5, 6, 7, 8\} & 20 & 20 \\
\{4, 5, 6, 7, 8\} & 20 & 20 \\

\bottomrule
\end{tabular}

\caption{Restrictions to 5 alternatives}
\label{table:triplets_rules_8D}
\end{table}


\begin{table}[H]
\centering
\setlength{\tabcolsep}{2pt} 
\begin{tabular}{lcc}
\toprule
\textbf{4-element} & \textbf{D224} & \textbf{AS222} \\
\textbf{subset}  & \textbf{CD size} & \textbf{CD size} \\
\midrule
\{1, 2, 3, 4\} & 8 & 9 \\
\{1, 2, 3, 5\} & 9 & 9 \\
\{1, 2, 3, 6\} & 9 & 9 \\
\{1, 2, 3, 7\} & 9 & 9 \\
\{1, 2, 3, 8\} & 9 & 9 \\
\{1, 2, 4, 5\} & 9 & 8 \\
\{1, 2, 4, 6\} & 9 & 8 \\
\{1, 2, 4, 7\} & 9 & 8 \\
\{1, 2, 4, 8\} & 9 & 8 \\
\{1, 2, 5, 6\} & 8 & 9 \\
\{1, 2, 5, 7\} & 8 & 9 \\
\{1, 2, 5, 8\} & 8 & 9 \\
\{1, 2, 6, 7\} & 8 & 8 \\
\{1, 2, 6, 8\} & 8 & 8 \\
\{1, 2, 7, 8\} & 8 & 9 \\
\{1, 3, 4, 5\} & 9 & 9 \\
\{1, 3, 4, 6\} & 9 & 9 \\
\{1, 3, 4, 7\} & 9 & 9 \\

\bottomrule
\end{tabular}
\hspace{1em}
\setlength{\tabcolsep}{2pt} 
\begin{tabular}{lcc}
\toprule
\textbf{4-element} & \textbf{D224} & \textbf{AS222} \\
\textbf{subset}  & \textbf{CD size} & \textbf{CD size} \\
\midrule

\{1, 3, 4, 8\} & 9 & 9 \\
\{1, 3, 5, 6\} & 8 & 8 \\
\{1, 3, 5, 7\} & 8 & 8 \\
\{1, 3, 5, 8\} & 8 & 8 \\
\{1, 3, 6, 7\} & 8 & 9 \\
\{1, 3, 6, 8\} & 8 & 9 \\
\{1, 3, 7, 8\} & 8 & 8 \\
\{1, 4, 5, 6\} & 8 & 9 \\
\{1, 4, 5, 7\} & 8 & 9 \\
\{1, 4, 5, 8\} & 9 & 9 \\
\{1, 4, 6, 7\} & 9 & 8 \\
\{1, 4, 6, 8\} & 8 & 8 \\
\{1, 4, 7, 8\} & 8 & 9 \\
\{1, 5, 6, 7\} & 9 & 9 \\
\{1, 5, 6, 8\} & 9 & 9 \\
\{1, 5, 7, 8\} & 9 & 8 \\
\{1, 6, 7, 8\} & 9 & 9 \\
\bottomrule
\end{tabular}
\hspace{1em}
\setlength{\tabcolsep}{2pt} 
\begin{tabular}{lcc}
\toprule
\textbf{4-element} & \textbf{D224} & \textbf{AS222} \\
\textbf{subset}  & \textbf{CD size} & \textbf{CD size} \\
\midrule
\{2, 3, 4, 5\} & 9 & 9 \\
\{2, 3, 4, 6\} & 9 & 9 \\
\{2, 3, 4, 7\} & 9 & 9 \\
\{2, 3, 4, 8\} & 9 & 9 \\
\{2, 3, 5, 6\} & 8 & 8 \\
\{2, 3, 5, 7\} & 8 & 8 \\
\{2, 3, 5, 8\} & 9 & 8 \\
\{2, 3, 6, 7\} & 9 & 9 \\
\{2, 3, 6, 8\} & 8 & 9 \\
\{2, 3, 7, 8\} & 8 & 8 \\
\{2, 4, 5, 6\} & 8 & 9 \\
\{2, 4, 5, 7\} & 8 & 9 \\
\{2, 4, 5, 8\} & 8 & 9 \\
\{2, 4, 6, 7\} & 8 & 8 \\
\{2, 4, 6, 8\} & 8 & 8 \\
\{2, 4, 7, 8\} & 8 & 9 \\
\{2, 5, 6, 7\} & 9 & 9 \\

\bottomrule
\end{tabular}
\hspace{1em}
\setlength{\tabcolsep}{2pt} 
\begin{tabular}{lcc}
\toprule
\textbf{4-element} & \textbf{D224} & \textbf{AS222} \\
\textbf{subset}  & \textbf{CD size} & \textbf{CD size} \\
\midrule
\{2, 5, 6, 8\} & 9 & 9 \\
\{2, 5, 7, 8\} & 9 & 8 \\
\{2, 6, 7, 8\} & 9 & 9 \\
\{3, 4, 5, 6\} & 8 & 9 \\
\{3, 4, 5, 7\} & 8 & 9 \\
\{3, 4, 5, 8\} & 8 & 9 \\
\{3, 4, 6, 7\} & 8 & 8 \\
\{3, 4, 6, 8\} & 8 & 8 \\
\{3, 4, 7, 8\} & 8 & 9 \\
\{3, 5, 6, 7\} & 9 & 9\\
\{3, 5, 6, 8\} & 9 & 9 \\
\{3, 5, 7, 8\} & 9 & 8 \\
\{3, 6, 7, 8\} & 9 & 9 \\
\{4, 5, 6, 7\} & 9 & 9 \\
\{4, 5, 6, 8\} & 9 & 9\\
\{4, 5, 7, 8\} & 9 & 8 \\
\{4, 6, 7, 8\} & 9 & 9 \\
\{5, 6, 7, 8\} & 8 & 9 \\
\bottomrule
\end{tabular}
\caption{Restrictions to 4 alternatives. Comparison between D224 and AS222}
\label{table:triplets_rules_8E}
\end{table}

\section{Comparison of UCD Representations for Size 224}
There are $56$ equivalent UCDs of size 224. In our analysis, we focused on the D224 representation, which maximizes the number of triples assigned 1N3 or 3N1. However, since D224 (and all its equivalent UCDs) lacks the reverse identity permutation (87654321), none of the representations contain rules for each triple being either 2N1 or 2N3.

Interestingly, there is another representation, denoted as E224, that stands out because it has the highest number of triplets (36 out of 56) assigned as either 2N1 or 2N3 rules. Table \ref{table:triplets_rules_8Special} shows the rules for this UCD.


\begin{table}[H]
\centering
\begin{tabular}{lc}
\toprule
\textbf{Triplets} & \textbf{Rules}\\
\midrule
(1, 2, 3) & 2N3 \\
(1, 2, 4) & 2N3 \\
(1, 2, 5) & 2N3 \\
(1, 2, 6) & 2N3 \\
(1, 2, 7) & 2N3 \\
(1, 2, 8) & 2N3 \\
(1, 3, 4) & 3N1 \\
(1, 3, 5) & 2N1 \\
(1, 3, 6) & 2N1 \\
(1, 3, 7) & 1N3 \\
(1, 3, 8) & 1N3 \\
(1, 4, 5) & 2N1 \\
(1, 4, 6) & 2N1 \\
(1, 4, 7) & 1N3 \\


\bottomrule
\end{tabular}
\hspace{1em}
\begin{tabular}{lc}
\toprule
\textbf{Triplets} & \textbf{Rules}\\
\midrule
(1, 4, 8) & 1N3 \\
(1, 5, 6) & 2N3 \\
(1, 5, 7) & 3N1 \\
(1, 5, 8) & 3N1 \\
(1, 6, 7) & 3N1 \\
(1, 6, 8) & 3N1 \\
(1, 7, 8) & 2N1 \\
(2, 3, 4) & 3N1 \\
(2, 3, 5) & 2N1 \\
(2, 3, 6) & 2N1 \\
(2, 3, 7) & 1N3 \\
(2, 3, 8) & 1N3 \\
(2, 4, 5) & 2N1 \\
(2, 4, 6) & 2N1 \\


\bottomrule
\end{tabular}
\hspace{1em}
\begin{tabular}{lc}
\toprule
\textbf{Triplets} & \textbf{Rules}\\
\midrule
(2, 4, 7) & 1N3 \\
(2, 4, 8) & 1N3 \\
(2, 5, 6) & 2N3 \\
(2, 5, 7) & 3N1 \\
(2, 5, 8) & 3N1 \\
(2, 6, 7) & 3N1 \\
(2, 6, 8) & 3N1 \\
(2, 7, 8) & 2N1 \\
(3, 4, 5) & 2N1 \\
(3, 4, 6) & 2N1 \\
(3, 4, 7) & 2N1 \\
(3, 4, 8) & 2N1 \\
(3, 5, 6) & 2N3 \\
(3, 5, 7) & 2N3 \\
\bottomrule
\end{tabular}
\hspace{1em}
\begin{tabular}{lc}
\toprule
\textbf{Triplets} & \textbf{Rules}\\
\midrule


(3, 5, 8) & 2N3 \\
(3, 6, 7) & 2N3 \\
(3, 6, 8) & 2N3 \\
(3, 7, 8) & 2N1 \\
(4, 5, 6) & 2N3 \\
(4, 5, 7) & 2N3 \\
(4, 5, 8) & 2N3 \\
(4, 6, 7) & 2N3 \\
(4, 6, 8) & 2N3 \\
(4, 7, 8) & 2N1 \\
(5, 6, 7) & 1N3 \\
(5, 6, 8) & 1N3 \\
(5, 7, 8) & 2N1 \\
(6, 7, 8) & 2N1 \\
\bottomrule
\end{tabular}

\caption{Table of triplets and rules that produces the Condorcet domain E224 of size 224 for 8 alternatives. This specific UCD is invariant under
the action by the permutations group $G =\{{\rm id}, (38)(47), (16)(25), (38)(47)(16)(25)\}$}
\label{table:triplets_rules_8Special}
\end{table}




\begin{table}[H]
\centering
\setlength{\tabcolsep}{2pt} 
\begin{tabular}{lccccc}
\toprule
\textbf{Id} & \textbf{Core} & \textbf{1N3} & \textbf{2N1} & \textbf{2N3} & \textbf{3N1}\\
\midrule
1 & 65872143 & 10 & 18 & 18 & 10 \\
2 & 36178245 & 22 & 7 & 6 & 21 \\
3 & 35182764 & 24 & 10 & 4 & 18 \\
4 & 45612387 & 21 & 5 & 7 & 23 \\
5 & 54721836 & 15 & 6 & 13 & 22 \\
6 & 54621387 & 15 & 5 & 13 & 23 \\
7 & 43216587 & 20 & 1 & 8 & 27 \\
8 & 54821763 & 15 & 12 & 13 & 16 \\
9 & 21437856 & 27 & 2 & 1 & 26 \\
10 & 21687354 & 23 & 13 & 5 & 15 \\
11 & 63287154 & 16 & 13 & 12 & 15 \\
12 & 21563487 & 25 & 3 & 3 & 25 \\
13 & 21436587 & 27 & 1 & 1 & 27 \\
14 & 21573846 & 25 & 4 & 3 & 24 \\
15 & 46718235 & 19 & 9 & 9 & 19 \\
16 & 35162487 & 24 & 3 & 4 & 25 \\
17 & 56871243 & 16 & 18 & 12 & 10 \\
18 & 21436587 & 27 & 1 & 1 & 27 \\
19 & 21436587 & 27 & 1 & 1 & 27 \\
20 & 35162487 & 24 & 3 & 4 & 25 \\
21 & 45812763 & 21 & 12,& 7 & 16 \\
22 & 34128765 & 26 & 8 & 2 & 20 \\
23 & 64728135 & 13 & 9 & 15 & 19 \\
24 & 21678345 & 23 & 7 & 5 & 21 \\
25 & 53271846 & 18 & 4 & 10 & 24 \\
26 & 21583764 & 25 & 10 & 3 & 18 \\
27 & 64827153 & 13 & 15 & 15 & 13 \\
28 & 63278145 & 16 & 7 & 12 & 21 \\

\bottomrule
\end{tabular}
\hspace{1em}
\setlength{\tabcolsep}{2pt} 
\begin{tabular}{lccccc}
\toprule
\textbf{Id} & \textbf{Core} & \textbf{1N3} & \textbf{2N1} & \textbf{2N3} & \textbf{3N1}\\
\midrule

29 & 53261487 & 18 & 3 & 10 & 25 \\
30 & 43217856 & 20 & 2 & 8 & 26 \\
31 & 34126587 & 26 & 1 & 2 & 27 \\
32 & 43216587 & 20 & 1 & 8 & 27 \\
33 & 35172846 & 24 & 4 & 4 & 24 \\
34 & 21573846 & 25 & 4 & 3 & 24 \\
35 & 53261487 & 18 & 3 & 10 & 25 \\
36 & 21563487 & 25 & 3 & 3 & 25 \\
37 & 21437856 & 27 & 2 & 1 & 26 \\
38 & 21678345 & 23 & 7 & 5 & 21 \\
39 & 45612387 & 21 & 5 & 7 & 23 \\
40 & 21438765 & 27 & 8 & 1 & 20 \\
41 & 53281764 & 18 & 10 & 10 & 18 \\
42 & 21438765 & 27 & 8 & 1 & 20 \\
43 & 34126587 & 26 & 1 & 2 & 27 \\
44 & 56781234 & 16 & 12 & 12 & 16 \\
45 & 36187254 & 22 & 13 & 6 & 15 \\
46 & 21563487 & 25 & 3 & 3 & 25 \\
47 & 21563487 & 25 & 3 & 3 & 25 \\
48 & 43218765 & 20 & 8 & 8 & 20 \\
49 & 21687354 & 23 & 13 & 5 & 15 \\
50 & 46817253 & 19 & 15 & 9 & 13 \\
51 & 54621387 & 15 & 5 & 13 & 23 \\
52 & 21583764 & 25 & 10 & 3 & 18 \\
53 & 21436587 & 27 & 1 & 1 & 27 \\
54 & 34127856 & 26 & 2 & 2 & 26 \\
55 & 45712836 & 21 & 6 & 7 & 22 \\
56 & 65782134 & 10 & 12 & 18 & 16 \\
\bottomrule
\end{tabular}
\caption{Presenting 56 isomorphic representations of the MUCD of size 224, with the sum of 1N3 and 2N3 rules and the sum of 3N1 and 2N1 rules being invariants at 28.
% The CD D224 has Id=19, while E1224 has Id=1}
}
\label{table:56variants}
\end{table}

Table \ref{table:56variants} lists the 56 equivalent versions of the CD, with D224 having Id=19 and E224 having Id=1. These versions can be organized according to their Core and the number of triples assigned to each of the four rules: 1N3, 2N1, 2N3, and 3N1. The core is represented by the core element that consists of four inversions. 


\section{Conclusion} In conclusion, our work has demonstrated a record-breaking Condorcet domain (CD) for $n=8$ alternatives, which is optimal and essentially unique (up to isomorphism). Our findings contribute to understanding the structure of CDs and Unitary Condorcet Domains (UCDs) and have potential applications in voting theory and social choice. 

Overall, our work highlights the importance of understanding the properties and structures of CDs in order to construct larger examples  and might pave the way for future research in this area.

We also observe that some record-breaking CDs for $n=8$ alternatives exhibit almost all rules of the form 1N3 and 3N1. These rules can be interpreted as a form of seeded voting. In such a system, for each set of three alternatives, a seeding is implemented to restrict the lowest-seeded alternative from being the highest-ranked preference or the highest-seeded alternative from being the lowest-ranked preference. A better understanding  of the global effects of this type of local seeding could serve as a foundation for future research, potentially offering insights into algorithmic fairness and impartiality in computer-supported decision-making.

\section*{Acknowledgements}
This research was conducted using the resources of High Performance Computing Center North (HPC2N).

%-----------------------------------
\bibliographystyle{plain}
\bibliography{references}

% \section*{Appendix}
\end{document}