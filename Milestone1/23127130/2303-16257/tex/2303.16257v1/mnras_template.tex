% mnras_template.tex 
%
% LaTeX template for creating an MNRAS paper
%
% v3.0 released 14 May 2015
% (version numbers match those of mnras.cls)
%
% Copyright (C) Royal Astronomical Society 2015
% Authors:
% Keith T. Smith (Royal Astronomical Society)

% Change log
%
% v3.0 May 2015
%    Renamed to match the new package name
%    Version number matches mnras.cls
%    A few minor tweaks to wording
% v1.0 September 2013
%    Beta testing only - never publicly released
%    First version: a simple (ish) template for creating an MNRAS paper

%%%%%%%%%%%%%%%%%%%%%%%%%%%%%%%%%%%%%%%%%%%%%%%%%%
% Basic setup. Most papers should leave these options alone.
\documentclass[fleqn,usenatbib]{mnras}

% MNRAS is set in Times font. If you don't have this installed (most LaTeX
% installations will be fine) or prefer the old Computer Modern fonts, comment
% out the following line
\usepackage{newtxtext,newtxmath}
%\usepackage{rotating, graphicx}
%\usepackage{lscape}
%\usepackage{txfonts}
%\usepackage{rotating,threeparttable,booktabs,caption,dcolumn} 
%\usepackage[normalem]{ulem}
%\usepackage{xcolor}
%\usepackage[]{hyperref}
%\usepackage{float}
%\usepackage{amssymb}
%\usepackage{multirow}
%\usepackage{float}
%\usepackage{gensymb}
%\usepackage{verbatim} 
%\usepackage{sidecap}
% Depending on your LaTeX fonts installation, you might get better results with one of these:
%\usepackage{mathptmx}
%\usepackage{txfonts}

% Use vector fonts, so it zooms properly in on-screen viewing software
% Don't change these lines unless you know what you are doing
\usepackage[T1]{fontenc}

% Allow "Thomas van Noord" and "Simon de Laguarde" and alike to be sorted by "N" and "L" etc. in the bibliography.
% Write the name in the bibliography as "\VAN{Noord}{Van}{van} Noord, Thomas"
\DeclareRobustCommand{\VAN}[3]{#2}
\let\VANthebibliography\thebibliography
\def\thebibliography{\DeclareRobustCommand{\VAN}[3]{##3}\VANthebibliography}


%%%%% AUTHORS - PLACE YOUR OWN PACKAGES HERE %%%%%

% Only include extra packages if you really need them. Common packages are:
\usepackage{graphicx}	% Including figure files
\usepackage{amsmath}	% Advanced maths commands
% \usepackage{amssymb}	% Extra maths symbols
\usepackage{multirow}
%%%%%%%%%%%%%%%%%%%%%%%%%%%%%%%%%%%%%%%%%%%%%%%%%%

%%%%% AUTHORS - PLACE YOUR OWN COMMANDS HERE %%%%%

% Please keep new commands to a minimum, and use \newcommand not \def to avoid
% overwriting existing commands. Example:
%\newcommand{\pcm}{\,cm$^{-2}$}	% per cm-squared

%%%%%%%%%%%%%%%%%%%%%%%%%%%%%%%%%%%%%%%%%%%%%%%%%%

%%%%%%%%%%%%%%%%%%% TITLE PAGE %%%%%%%%%%%%%%%%%%%

% Title of the paper, and the short title which is used in the headers.
% Keep the title short and informative.
%\title[Short title, max. 45 characters]{FAUST VII. Gas kinematics around the VLA1623--2417 W protostar: the disk and the molecular streamers}
\title[The disk and the streamers of VLA 1623W]{FAUST VIII. The protostellar disk of VLA 1623--2417 W and its streamers imaged by ALMA}

% The list of authors, and the short list which is used in the headers.
% If you need two or more lines of authors, add an extra line using \newauthor
\author[S. Mercimek et al.]{
S. Mercimek,$^{1,2}$\thanks{E-mail: seyma.mercimek@inaf.it}
L. Podio,$^{1}$
C. Codella,$^{1,3}$
L. Chahine,$^{4,5}$
A. L\'{o}pez-Sepulcre,$^{3,4}$
S. Ohashi,$^{6}$
L. Loinard,$^{7,8}$
\newauthor
D. Johnstone,$^{9,10}$
F. Menard,$^{3}$
N. Cuello,$^{3}$
P. Caselli,$^{11}$
J. Zamponi,$^{11}$
Y. Aikawa,$^{12}$ 
E. Bianchi,$^{13,1}$
\newauthor
G. Busquet,$^{14,15,16}$
J. E. Pineda,$^{11}$,
M. Bouvier,$^{17}$ 
M. De Simone,$^{18,1}$
Y. Zhang,$^{6,19}$
N. Sakai,$^{6}$
C. J. Chandler,$^{20}$
\newauthor
C. Ceccarelli,$^{3}$
F. Alves,$^{11}$
A. Dur\'an,$^{7}$
D. Fedele,$^{1}$
N. Murillo,$^{6}$
I. Jim\'{e}nez-Serra,$^{21}$
S. Yamamoto$^{22,23}$
\\
\\
% List of institutions
$^{1}$INAF, Osservatorio Astrofisico di Arcetri, Largo E. Fermi 5, I-50125, Firenze, Italy\\
$^{2}$Universi\`a degli Studi di Firenze, Dipartimento di Fisica e Astronomia, via G. Sansone 1, 50019 Sesto Fiorentino, Italy\\
$^{3}$Univ. Grenoble Alpes, CNRS, IPAG, 38000 Grenoble, France\\
$^{4}$Institut de Radioastronomie Millim\'{e}trique, 38406 Saint-Martin d’H\`{e}res, France\\
$^{5}$\'Ecole doctorale de Physique, Universit\'e Grenoble Alpes, 110 Rue de la Chimie, 38400 Saint-Martin-d'H\`eres, France\\ 
$^{6}$RIKEN Cluster for Pioneering Research, 2-1, Hirosawa, Wako-shi, Saitama 351-0198, Japan\\
$^{7}$Instituto de Radioastronomía y Astrofísica , Universidad Nacional Autónoma de México, A.P. 3-72 (Xangari), 8701, Morelia, Mexico\\
$^{8}$Instituto de Astronomía, Universidad Nacional Autónoma de México, Ciudad Universitaria, A.P. 70-264, Ciudad de México 04510, Mexico\\
$^{9}$NRC Herzberg Astronomy and Astrophysics, 5071 West Saanich Road, Victoria, BC V9E 2E7, Canada\\
$^{10}$Department of Physics and Astronomy, University of Victoria, Elliott Building, 3800 Finnerty Road, Victoria, BC V8P 5C2, Canada\\
$^{11}$Max-Planck-Institut für extraterrestrische Physik (MPE), Gießenbachstr. 1, D-85741 Garching, Germany\\
$^{12}$Department of Astronomy, The University of Tokyo, 7-3-1 Hongo, Bunkyo-ku, Tokyo 113-0033, Japan\\
$^{13}$ ORIGINS, Excellence Cluster Origins, Boltzmannstrasse 2, D-85748 Garching bei München, Germany\\
$^{14}$Departament de Física Quàntica i Astrofísica, Universitat de Barcelona (UB), c/ Martí i Franquès 1, 08028 Barcelona, Spain\\
$^{15}$Institut de Ciències del Cosmos (ICCUB), Universitat de Barcelona, c. Martí i Franquès 1, 08028, Barcelona, Spain\\
$^{16}$Institut d'Estudis Espacials de Catalunya (IEEC), c. Gran Capità 2-4, 08034, Barcelona, Spain\\
$^{17}$Leiden Observatory, Leiden University, PO Box 9513, 2300 RA Leiden, The Netherlands\\
$^{18}$European Southern Observatory, Karl-Schwarzschild-Strasse 2, 85748, Garching bei München, Germany \\
$^{19}$Department of Astronomy, University of Virginia, Charlottesville, VA 22904, USA\\
$^{20}$National Radio Astronomy Observatory, PO Box O, Socorro, NM 87801, USA\\
$^{21}$Centro de Astrobiología (CSIC/INTA), Ctra.de Torrejón a Ajalvir km 4, 28806, Torrejón de Ardoz, Spain\\
$^{22}$Department of Astronomy, The University of Tokyo, 7-3-1 Hongo, Bunkyo-ku, Tokyo 113-0033, Japan\\
$^{23}$Research Center for the Early Universe, The University of Tokyo, 7-3-1 Hongo, Bunkyo-ku, Tokyo 113-0033, Japan\\
%Affiliations are listed at the end of the paper.
}
%FOR THE OTHERS:
%Univ. Grenoble Alpes, CNRS, Institut de Plan\'etologie et d'AstrOphiuchusysique de Grenoble (IPAG), 38000 Grenoble, France
%\ \'Ecole doctorale de Physique, Universit\'e Grenoble Alpes, 110 Rue de la Chimie, 38400 Saint-Martin-d'H\`eres, France 
%Institut de Radioastronomie Millimétrique, 38406 Saint-Martin d’Hères, France


% These dates will be filled out by the publisher
\date{Accepted XXX. Received YYY; in original form ZZZ}

% Enter the current year, for the copyright statements etc.
\pubyear{2022}

% Don't change these lines
\begin{document}
\label{firstpage}
\pagerange{\pageref{firstpage}--\pageref{lastpage}}
\maketitle

% Abstract of the paper
\begin{abstract}
More than 50\% of solar-mass stars form in multiple systems. It is therefore crucial to investigate how multiplicity affects the star and planet formation processes at the protostellar stage. We report continuum and C$^{18}$O (2--1) observations of the VLA 1623-2417 protostellar system at 50 au angular resolution as part of the ALMA Large Program FAUST. The 1.3 mm continuum probes the disks of VLA 1623A, B, and W, and the circumbinary disk of the A1+A2 binary. The C$^{18}$O emission reveals, for the first time, the gas in the disk-envelope of VLA 1623W. We estimate the dynamical mass of VLA 1623W, $M_{\rm dyn}=0.45\pm0.08$ M$_{\rm \sun}$, and the mass of its disk, $M_{\rm disk}\sim6\times10^{-3}$ M$_{\rm \sun}$. C$^{18}$O also reveals streamers that extend up to 1000 au, spatially and kinematically connecting the envelope and outflow cavities of the A1+A2+B system with the disk of VLA 1623W. The presence of the streamers, as well as the spatial ($\sim$1300 au) and velocity ($\sim$2.2 km/s) offset of VLA 1623W suggest that either sources W and A+B formed in different cores, interacting between them, or that source W has been ejected from the VLA 1623 multiple system during its formation. In the latter case, the streamers may funnel material from the envelope and cavities of VLA 1623AB onto VLA 1623W, thus concurring to set its final mass and chemical content.
\end{abstract}

% Select between one and six entries from the list of approved keywords.
% Don't make up new ones.
\begin{keywords}
ISM: kinematics and dynamics -- ISM: molecules -- stars: formation -- ISM: individual objects: VLA 1623–2417
\end{keywords}

%%%%%%%%%%%%%%%%%%%%%%%%%%%%%%%%%%%%%%%%%%%%%%%%%%

%%%%%%%%%%%%%%%%% BODY OF PAPER %%%%%%%%%%%%%%%%%%

\section{Introduction}
%%%%%%%%%%%%%%%%%%%%%%%%%
%%  Put info about ejection mech. related to star formation then at the end of pass. make a link between VLA1623 and then source W
%%%%%%%%%%%%%%%%%%%%%%%%%
%Protostellar accretion flow as a binary system \citep[e.g.,][]{Pineda2012, Jorgensen2016}.

%Three body interaction \citep{Reipurth2000} with ejection scenario.

%multiple system how they move \citep{Sadavoy2017}

%Explanation how Class O protostars which consist of binary systems to understand their formation route related to fragmentation of the core, explaining them with some steps, including VLA1623 \citep{Chen2013}.

Observational studies of solar-mass star forming regions indicate a high fraction of multiplicity, $\sim30\% - 50\%$ \citep[e.g.,][]{Duchene2007, Chen2013, Tobin2016, Tobin2022, Offner2022}. 
It is, therefore, crucial to investigate what are the processes that lead
%concur 
to the formation of low-mass stars and of their disks in multiple systems, in terms of dynamical interactions between protostars, ejection phenomena, and streamers stripping away or feeding gas and dust to the individual protostellar disks. 
%within the system.
%Understanding the formation of multiple systems during the low mass star-forming process is essential since the protostellar (the so called Class 0 and Class I objects; age $\leq$ 10$^{5}$ yr) phase seems to have higher binary possibility than the later stage \citep[e.g.,][]{Reipurth2000, Haisch2004}. 
%Several observational studies of protostellar regions indeed revealed high fraction of multiplicity \citep[e.g][]{Duchene2007, Kounkel2016, Tobin2016, Tobin2022}. 
Protostellar surveys \citep[e.g.,][]{Reipurth2000,Ward-Thompson2007,Chen2013, Tobin2016}
indicate that the multiplicity fraction (MF) is higher during the Class 0 stage (sources with age $ \sim 10^{4}$ yr, MF up to 0.5-0.8), with respect to later Class I and Class II sources (age $> 10^{5}$ yr, MF of 0.2-0.3). 
This lowering MF with time may be due to dynamical interactions that cause the ejection of one component from the protostellar system \citep{Reipurth2000,Sadavoy2017}. %\citep{Reipurth2000,Sadavoy2017} show that 
%three-body interactions may play a role for the formation of multiple system. Namely, 
%In this context, it is mandatory to improve our knowledge of the embedded protostellar systems in terms of accretion mechanisms. 
Moreover, recent interferometric observations reveal the presence of accretion streamers in protostellar systems spanning sizes from 1000 au \citep[e.g.,][]{Takakuwa2017,Alves2019, Hull2020,Pineda2022} up to $10^4$ au \citep[e.g.,][]{Pineda2020, Murillo2022}. 
%One of the major implications is that infalling process could be bursted by
Such accretion streamers are observed also at more evolved Class I/II stages and may play a crucial role in determining the final disk mass and chemical composition \citep{Garufi2022,Valvidia2022}.

%\citet{Takakuwa2017} mapped a tail-like structure associated with the disk with spiral arms; \citet{Tokuda2018} proposed that turbulent fragmentation causes those structures; \citet{Akiyama2019} suggested  interactions with the external cloud; \citet{Pineda2020} reported accelerating accretion streamers. 

%%%%%
% REFEREE: This sentence should be justified using proper citations for the presence of the different phenomena within VLA 1623. Otherwise, it might seem like an arbitrary sentence.
%%%
The VLA 1623--2417 region (VLA 1623 hereafter) is an archetypical laboratory to investigate low-mass star formation within a multiple protostellar system and the associated phenomena: ejection, accretion, and dynamical interactions between sources \citep{Murillo2013,Harris2018,Hara2021,Ohashi2022}.
%how the parental envelope lead the star forming process as well as it is in turn sculpted by the protostellar activity. More specifically, 
%VLA 1623 is a protostellar system, associated with four objects, 
VLA1623 is located in Ophiuchus A at a distance, $d$, of 131 pc \citep{Gagne2018}, and consists of four protostellar sources: VLA 1623A, a close Class 0 binary, with separation between the components (A1 and A2) of $\sim 30$ au, surrounded by a circumbinary disk; VLA 1623B, a Class 0 protostar located $\sim130$ au West of VLA 1623A and associated with an edge-on disk; and VLA 1623W, classified as a Class I protostar and located $\sim1300$ au West of the A binary, also associated with an edge-on disk \citep[e.g.,][]{Bontemps1997,Murillo2013SED, Harris2018, Kawabe2018}. 

The A1, A2, and B protostellar sources drive high velocity outflows along the NW-SE direction detected in CO and H$_2$ \citep[e.g., ][]{Andre1990,Caratti2006,Santangelo2015,Hara2021}. The outflows open low velocity wide angle cavities observed in CCH and CS \citep[e.g.,][]{Ohashi2022}.
Emission in SO and C$^{18}$O also probes accretion flows towards the circumbinary disk of VLA 1623A \citep{Hsieh2020}. 
The outflow cavities rotate coherently with the dense parental envelope \citep{Ohashi2022}.
On the other hand, the disk of VLA 1623B counter-rotates with respect to the envelope and outflow cavities, and the A1, A2, and B disks are misaligned, suggesting
that the system is dynamically unstable \citep{Hara2021,Ohashi2022,Codella2022}. 
%Finally, there is another source in the protostellar system, called source W, at the distance $\sim$ 10$\arcsec$ ($\sim$1300 au) from system A+B in the Western direction \citep[e.g.,][]{Bontemps1997,Harris2018}. 

The nature of the more distant component of the cluster, VLA 1623W, located $\sim$1300 au away from VLA 1623A and B, and its associated phenomena are only poorly characterized. \citet{Maury2012} suggested that W could be a shocked cloudlet produced by the outflow driven by VLA 1623A. ALMA continuum images, however, revealed that W is a protostar with an edge-on disk
\citep{Harris2018,Sadavoy2019,Michel2022}. 
Based on the analysis of the spectral energy distribution (SED), \citet{Murillo2018} classified VLA 1623W as a Class I, due to the 4 times lower luminosity and 20 times lower envelope mass compared to VLA 1623A and B.
%It is unclear if VLA 1623W is a Class 0 as proposed by \citet{Hsieh2013, Sadavoy2019}, or if it is a Class I \citep{Murillo2013SED,Murillo2018}.
%{\bf According to the ALMA 3mm data, which can recover emission around $\sim$ 3000 au, \citep{Kirk2017} reported comparable mass for source W (0.1 M$_{\rm \sun}$) and measured 0.7 M$_{\rm \sun}$ for system A+B. All sources are associated with the same dense core \citep{pat2015}. Thus, three sources could be considered Class 0 sources. On the other hand, source W has no obvious outflow structure, and \citet{Murillo2013SED} with SMA observations reported that Class W is Class I while source A and source B are Class 0 protostars studying SED analysis. \citet{Murillo2018} reported envelope masses 0.8 M$_{\rm \sun}$, 0.2 M$_{\rm \sun}$, and 0.04 M$_{\rm \sun}$ for sources A, B, and W respectively, revising SED analysis using combination of the ALMA and the {\it Sptizer} data.}
%{\bf Recently, the disk of source W modeled proposing that it may represent a flared disk \citep{Michel2022}.} 
Observations of C$^{18}$O ($2-1$) line emission by \citet{Murillo2013} suggest that the systemic velocity of VLA 1623W is between 0 and 1 km\,s$^{-1}$, which differs from VLA 1623A’s systemic velocity \citep[+3.8 km\,s$^{-1}$,][]{Ohashi2022}.
\citet{Murillo2013} and \citet{Harris2018} suggest that VLA 1623W may have been ejected from the system composed by A1, A2, and B.
In this paper, we report ALMA observations of C$^{18}$O (2--1) and continuum emission at 1.3~mm used to investigate the gas in the disk of VLA 1623W, and the source dynamical interaction  with the other protostellar sources in the VLA 1623 multiple protostellar system.



%%%%%%%%
%VLA system and source W \citep{Ohashi2022, Hsieh2020}

%source W as a class I from SED \citep{Murillo2013}

%C18O emission towards source W \citep{Murillo2013}

%\citet{Sadavoy2019} check page 41, there are good references about source W.
%%%%%%%%%%%
\section{Observations and Data Reduction}
\label{observations}

The VLA 1623 protostellar system was observed between October 2018 and March 2020 as part of the ALMA Large Program, FAUST (Fifty AU STudy of the chemistry in the disk/envelope system of
Solar-like protostars; 2018.1.01205.L, PI: S. Yamamoto, \citealt{Codella2022}). We observed in the Band 6 frequency range 216--234\,GHz, using the 12-m array (C43-4 and C43-1) and 7-m array of the Atacama Compact Array (ACA). %We merged the visibility of the 12-m data and the 7-m data in the UV plane to obtain the extended emission.
%The maximum recoverable scale is $\theta_{\rm MRS}\sim 24\arcsec$ {\bf ($\sim$ 3150 au)}. 
%(corresponding to $\sim 3150$ au). 
%For this, $\theta_{\rm MRS}$ $\sim$ 0.6$\lambda$/$L_{\rm min}$ applies, where $L_{\rm min}$ is the minimum baseline and $\lambda$ is the wavelength. 
%Thus, can safely cover source W with its extended emission, which is $\sim$10$\arcsec$ further from the center of the observational field. 
The observations were centered at $\alpha_{\rm 2000}$ = 16${^h}$26${^m}$26${^s}$.392, $\delta_{\rm 2000}$ = --24$^\circ$24$'$30$''$.69. 
% We then reprojected source W according to its coordinates from the continuum fit analysis (see section \ref{Continuum}). 
The C$^{ 18}$O(2--1) line at 219560.3 MHz \citep[$E_{\rm up}$ = 16 K,][]{Muller2005}, is covered by a narrow spectral window with bandwidth of 62.5 MHz (87 km s$^{-1}$) and a channel width of 122 kHz (0.17 km s$^{-1}$).  
A spectral window with bandwidth of 1825 MHz and channel width of 977 kHz (1.25 km s$^{-1}$) has been used to image the continuum emission.

We used the Common Astronomy Software Applications package (CASA) 5.6.1-8 version \citep{McMullin2007} to obtain the calibrated visibilities, and 6.2.1 version to obtain clean, and to image the data. The multiscale deconvolver was used \citep{Cornwell2008, rau2011}.
In addition to the standard pipelines, an additional calibration routine (\url{http://www.aoc.nrao.edu/~gmoellen/}; Moellenbrock et al. in preparation) has been used to correct for the $T_\mathrm{sys}$ and for spectral data normalization. We used line-free frequencies to recover the continuum emission for each configuration and perform self-calibration. The correction of the complex gain has been derived from the self-calibration and spontaneously carried out to line visibilities of the data. Following that, to produce continuum-subtracted line data we subtracted the continuum model, derived from the self-calibration. Also the phase self-calibration technique along with long solution interval amplitudes have been used to align positions across all configurations. 
%and amplitudes.
%\footnote{\url{http://www.aoc.nrao.edu/~gmoellen/}; Moellenbrock et al. (in preparation)}. 
The task $tclean$ was used to obtain the image of the continuum and the datacube of the molecular emission. We adopted  Briggs weighting with a robustness parameter of $-2.0$ (uniform weighting) for the continuum to obtain the highest angular resolution (beam: 0$\farcs$42 $\times$ 0$\farcs$32, PA=$-65^\circ$). On the other hand, a robustness parameter of $0.5$ was employed for the molecular emission to optimize the signal-to-noise, consistently with the previous FAUST paper on VLA 1623 \citep{Ohashi2022}.
%The robustness parameter of --2.0 uniform weighting was chosen for the dust continuum emission as it gives all UV grid points equally with the highest angular resolution we can reach (beam: 0$\farcs$42 $\times$ 0$\farcs$32, PA=$-65^\circ$). A robustness parameter of 0.5 was chosen for the Briggs weighting of the molecular emission to be consistent with the previous FAUST work, which has used combined data (12-m+7-m) \citep{Ohashi2022}.
%Robustness parameters of 0.5 and --2.0 were chosen for the Briggs weighting of the molecular and dust continuum emission, respectively. 
Finally, we applied primary beam corrections.

The data analysis was carried out using the IRAM-GILDAS\footnote{\label{note5}\url{http://www.iram.fr/IRAMFR/GILDAS}} software package. 
We produced two continuum-subtracted C$^{ 18}$O (2--1) datacubes: (1) combining only the data taken with the 12-m array to sample small scales structures (beam: 0$\farcs$48 $\times$ 0$\farcs$40, PA=$-82^\circ$; $\theta_{MRS} \sim 14\arcsec$, corresponding to $\sim 1800$ au) (see Figs. \ref{channel_small} and \ref{appendix}, and panels c) and d) of Fig. \ref{supermoments}); and (2) combining the data taken with the 12-m and the 7-m arrays to recover emission extending up to $\sim 3000$ au (beam: 0$\farcs$53 $\times$ 0$\farcs$44, PA=$-74^\circ$, $\theta_{MRS} \sim 24\arcsec$, corresponding to $\sim 3150$ au) (used in Figs. \ref{Con-mom0}, \ref{channel_large}, and panels a) and b) of Fig. \ref{supermoments}). The noise root mean square (r.m.s) is 1.8 mJy beam$^{-1}$ and 1.4 mJy beam$^{-1}$ per channel for C$^{18}$O datacubes (1) and (2), respectively. The r.m.s of the continuum map is 0.26 mJy\,beam$^{-1}$.

% (Tab. \ref{Lines}).
%the synthesized beams and the noise r.m.s., as well as observed line transitions, are reported in Table \ref{Lines}. 
%To be precise, we used 12-m array observation to trace inner envelope emission to avoid contamination of the large scale emission. On the other hand, we performed the merged of 12-m and 7-m array observation to trace the extended emission. Table \ref{Lines} shows analyzed properties of these two configuration. For the clarification, we used the merged observation of C$^{18}$O in Fig. \ref{Con-mom0}, a) and b) of Fig. \ref{supermoments}, and \ref{channel_large}; 12-m array observation of C$^{18}$O in Fig. c) and d) of Fig. \ref{supermoments} and \ref{channel_small}.

%\begin{table}
%    \caption{The C$^{18}$O(2--1) transition observed towards VLA1623--2417.}
%    \label{Lines}
%    \begin{tabular}{ccllcc}
%    \hline
%    Arrays & $\nu$$^{a}$ & $E_{\rm up}$$^{a}$ & $S\mu^2$ $^{a}$ & Beam & r.m.s. \\
%    & (MHz) & (K) & (D$^{2}$) & ($\arcsec$ $\times$ $\arcsec$, $^{\circ}$) & (mJy/beam) \\
%    \hline
%    {\smallskip}
%12m+7m  & 219560.3 & 16& 0.05 & 0.53$\times$0.44 (--74) & 1.4 \\
%12m & -- & -- & -- & 0.48$\times$0.40 (--82) & 1.8 \\
   % p-H$_{2}$CO & 218222.1 &21 &16 &0$\farcs$53 $\times$ 0$\farcs$44 &0.9 \\
    %(3$_{0,3}$ -- 2$_{0,2}$)& & & & P.A --76$^{\circ}$ & \\
%    \hline
%    {\smallskip}
%    \end{tabular}
%    $(^a)$ Frequencies and spectroscopic parameters have been taken from the Cologne Database for Molecular Spectroscopy \citep{Muller2005}. The channel width is smoothed to 0.2 km s$^{-1}$
%   $(^c)$ The merged observation consists of 12-m and 7-m array. 
%    $(^d)$ The 12-m array observation. For both observation, the channel width is smoothed to 0.2 km s$^{-1}$. 
%\end{table}

\begin{figure*}
	% To include a figure from a file named example.*
	% Allowable file formats are eps or ps if compiling using latex
	% or pdf, png, jpg if compiling using pdflatex
	\vspace{-1cm}
%	\includegraphics[width=\columnwidth]{ConSetup1_AND_MOM0_VLA.pdf}
\centering 
\includegraphics[width=13.0cm, angle =90]{FIG_1.pdf}
	\vspace{-1cm}
    \caption{{\it Left:} The VLA1623--2417 system: integrated intensity map (moment 0) of C$^{18}$O (2--1) (color scale) with overlaid the dust continuum emission at 1.3 mm (black contour). The 12m+7m dataset has been used.
    The position of the A, B, and W protostars are labelled. 
    The C$^{18}$O emission is integrated from --6.0 to +10.0 km s$^{-1}$. The black continuum contours start from 3$\sigma$ (0.78 mJy beam$^{-1}$) with intervals of 40$\sigma$. The green contour is for the 3$\sigma$ level (18 mJy km s$^{-1}$ beam$^{-1}$) of the C$^{18}$O moment 0 map. 
    %I did bold for moment 0 since Laurent thought that i used sigma of C18O line, where i wrote it in the text as 1.4 mjy/beam. But i used sigma for moment 0, which is 6 mjy/beam, meaning 3 sigma is 18 mjy....
    The magenta thick contour is the CS(5--4) emission (25$\sigma$) which traces the outflow cavity walls associated with VLA1623 A \citep[from][]{Ohashi2022}.
    %, which level is 20$\sigma$ (1$\sigma$ is 1.2 mJy beam$^{-1}$).
    The synthesized beams 
    (upper left corner) are drawn in green, magenta, and black for the C$^{18}$O, CS, and the dust continuum emission, respectively
    (Sect. \ref{observations}).
 {\it Right:} Zoom in of the moment 0 map towards sources A and B.}
    
    \label{Con-mom0}
\end{figure*}

\section{Results}
\label{results}
\subsection{Continuum emission at 1.3~mm} 
\label{Continuum}

Figure \ref{Con-mom0} shows the map of the 1.3 mm continuum emission (black contours), obtained combining the 12-m and the 7-m data. The protostellar sources A, B, and W are detected, as well as the circumbinary disk, but the angular resolution is too low to disentangle the close binary components, A1 and A2, resolved by \citet{Harris2018} at 0.9~mm (separation $\sim 30$ au). 
%The dust continuum map is in agreement with that previously reported by \citet{Harris2018}. The analysis of the continuum is beyond the scope of this paper; however, the spatial distribution of the protostars, and more specifically, that of source W are used to determine the origin of the detected molecular features reported below.
We fit the continuum emission towards VLA 1623A, B, and W with the CASA task \textit{imfit}, which perform a two-dimensional elliptical Gaussian fit. 
%is an application that fits one or more two-dimensional Gaussians to sources in an image. Fitting is only possible for a single polarization; however, it can be done over several adjacent spectral channels.} 
%We obtained the coordinates of the continuum peak ($\alpha$ $_{\rm J2000}$ = 16${^h}$26${^m}$25${^s}$.63, $\delta$ $_{\rm J2000}$ =  --24$^\circ$24$'$29$''$.69), the disk major and minor axis (0$\farcs$71$\pm$0.01 $\times$ 0$\farcs$12$\pm$0.02), position angle (PA$_{\rm disk}\sim 10^\circ\pm0.80^\circ$), the integrated flux at 1.3~mm ($F_{1.3~mm} = 59 \pm 1$ mJy) and the peak intensity ($23.7 \pm 0.3$ mJy beam$^{-1}$). The disk is almost edge-on (inclination $i \simeq$ 80$^{\circ}$) and has a diameter of $\sim 93$ au, in agreement with the results obtained by \citet{Harris2018} at 0.9~mm. 
 The obtained coordinates of the continuum peak (R.A.$_{\rm J2000}$, Dec.$_{\rm J2000}$), the ellipse size, the position angle, the integrated intensity at 1.3~mm ($F_{1.3~mm}$), and the peak intensity are reported in Table \ref{Confit}. 
The results of the fit are in agreement with those obtained by \citet{Harris2018} at 0.9 mm.
For source W, we find that the disk is almost edge-on (inclination $i \simeq$ 80$^{\circ}$), and has a diameter of $\sim 93$ au. %in agreement with the results obtained by \citet{Harris2018} at 0.9~mm. 

From the integrated intensity at 1.3~mm we estimate the mass of dust in the disks, $M_{\rm dust}$, as \citep{Hildebrand1983, Beckwith1990}:
%
\begin{equation}
M_{dust} = \frac{F_{1.3~mm} d^{2}}{\kappa_{\nu}B_{\nu}\left(T_{dust}\right)}.
\label{eq:quadratic}
\end{equation}
%
We assume isothermal conditions and optically thin emission, dust opacity ($\kappa_{\nu}$) at 1.3 mm of 2.17 cm$^{2}$g$^{-1}$ (Zamponi et al. submitted), and dust temperature, $T_{\rm dust}$, between $20$ K, the typical value assumed for Class II disks \citep[e.g.,][]{Beckwith1990}, and 50 K, to account for possible warmer dust in Class 0 and I disks \citep[e.g., ][]{Zamponi2021}. %After all it turns out : %After all it turns out,
%$M_{\rm dust} = F_{\rm 1.3~mm} d^{2} \times \left( \kappa_{\nu} B_{\nu} T_{\rm dust}\right)^{-1}$.
%\footnote{$M_{\rm dust}$ = $F_{1.3~mm}$ $\times$ $d^{2}$ $\times$ ($\kappa_{\nu} \times B_{\nu}$($T_{dust}$))$^{-1}$} .
%We assume optically thin emission, isothermal conditions, dust opacity ($\kappa_{\nu}$) at 1.3 mm of 2.3 cm$^{2}$g$^{-1}$ \citep{Beckwith1990}, and dust temperature $T_{\rm dust} = 20$ K.
Table \ref{Confit} reports the derived $M_{\rm dust}$ values towards source A, source B, and source W. 

For source W, assuming a gas-to-dust ratio of 100, we obtain a total disk mass of $6\pm3 \times 10^{-3}$ M$_{\odot}$. %\citet{Sadavoy2019} reported a disk mass of $1 \times 10^{-2}$ M$_{\odot}$ in the same band and using the same gas-to-dust ratio of 100. 
The difference with the estimate obtained by \citet{Sadavoy2019} in the same ALMA band ($1 \times 10^{-2}$ M$_{\odot}$) is due to the 10\% uncertainty on the flux calibration and the different assumed distance, dust opacity and temperature.
%: (1) 10\% difference in the measured flux at 1.3mm (65 mJy in \citet{Sadavoy2019}); (2) slightly different parameters assumed to estimate the dust mass using Eg. \ref{eq:quadratic}. }
  Moreover, the derived disk mass is affected by large uncertainty due to: (i) the assumption on the gas-to-dust ratio. If this is lower than 100 \citep[e.g., ][]{Ansdell2016}, the estimated disk mass should be regarded as an upper limit; (ii) the assumption that the dust continuum emission is optically thin. If the emission is optically thick, the estimated disk mass is a lower limit.
%\textcolor{red}{ The disk mass of W is also in agreement with those of disks around Class 0 and I protostars, as discussed e.g., by \citet{Sheehan2020,Sheehan2022}.}

\begin{table*}
\centering
\begin{tabular}{lccccccc}
\hline
Source & R.A.$_{\rm J2000}$ & Dec.$_{\rm J2000}$  & Ellipse size & Position angle & $F_{\rm 1.3~mm}$ & Peak I &  M$_{\rm dust}^{*}$\\
& ($^h$ $^m$ $^s$) & ($^{\circ}$ $\arcmin$ $\arcsec$) & ($\arcsec \times \arcsec$) &  ($^{\circ}$) & (mJy) & (mJy/beam) &   ($10^{-4}$ M$_{\rm \odot}$)\\
\hline
A1+A2 &16:26:26.3907 $\pm$ 0.0013 & --24.24.30.934 $\pm$ 0.017 & 0.49 ($\pm$0.01) $\times$ 0.33 ($\pm$0.01) 
 & 74 $\pm$ 20 & 158 $\pm$ 14 & 71 $\pm$ 4   & 1.6$\pm$0.8 \\
%\hline
B & 16:26:26.3063 $\pm$ 0.0001 & --24.24.30.787 $\pm$ 0.002 & 0.32 ($\pm$0.04) $\times$ 0.16 ($\pm$0.03) & 43 $\pm$ 5 & 121 $\pm$ 2 & 80 $\pm$ 1 & 1.2$\pm$0.6 \\
%\hline
W &16:26:25.6312 $\pm$ 0.0002 & --24:24:29.669 $\pm$ 0.005 & 0.71 ($\pm$0.01) $\times$ 0.12 ($\pm$0.02) & 10.0 $\pm$ 0.8 & 59 $\pm$ 1 & 23.7 $\pm$ 0.3  & 0.6$\pm$0.3  \\
\hline
\end{tabular}
\caption{ Peak coordinates, ellipse size, position angle, integrated and  peak intensity of the continuum emission at 1.3~mm towards sources A1+A2, B, and W. ($^*$) The dust mass, M$_{\rm dust}$, is derived from the integrated intensity assuming a dust temperature of $20-50$ K.}
\label{Confit}
\end{table*}

\subsection{C$^{18}$O (2--1) emission}

\begin{figure*}
\centering
	% To include a figure from a file named example.*
	% Allowable file formats are eps or ps if compiling using latex
	% or pdf, png, jpg if compiling using pdflatex
	%\includegraphics[width=18cm]{SourceW_C18O_channel_largescale_PAPER.pdf}
%	\includegraphics[width=12cm, angle =90]{largescale_channels.pdf}
%\includegraphics[width=12cm, angle =90]{largescale2.pdf}
\includegraphics[width=12cm, angle =90]{fig2.pdf}
    \caption{Channel maps of the C$^{18}$O (2--1) emission on the low velocity range ([+0.2, +3.0] km s$^{-1}$). The first contour is at 3$\sigma$ 
    (4.2 mJy beam$^{-1}$) and the step is 10$\sigma$.  The synthesized beam is shown by the black ellipse in the bottom-left corner of the first channel (beam: 0$\farcs$53 $\times$ 0$\farcs$44). The positions of VLA 1623A, B, and W are indicated by the white stars and are labelled in the first channel. The magenta contours in the channels from +1.0 km s$^{-1}$ to +2.2 km s$^{-1}$ indicate the outflow cavity walls probed by CS (5--4) emission  \citep[25$\sigma$ contour, from][]{Ohashi2022}.The northern, N, and southern, S1 and S2, streamers are labelled.}
    \label{channel_large}
\end{figure*}

\begin{figure*}
\centering
\vspace{-2cm}
	% To include a figure from a file named example.*
	% Allowable file formats are eps or ps if compiling using latex
	% or pdf, png, jpg if compiling using pdflatex
	\includegraphics[width=10.5cm, angle =90]{DISK_SourceW_C18O_channel.pdf}
	\vspace{-2cm}
    \caption{Channel maps of the C$^{18}$O (2--1) compact emission around the VLA 1623W protostar (green star) at high red- and blue-shifted velocities, i.e. from $\pm2$ km s$^{-1}$ to $\pm4.2$ km s$^{-1}$ with respect to the VLA 1623W systemic velocity (+1.6 km s$^{-1}$).  First contours and steps are 3$\sigma$ (5.4 mJy beam$^{-1}$). The velocity offset with respect to $V_{\rm sys}$(W) is reported in the top right corner of each panel. The black contour is the 3$\sigma$ level of the 1.3~mm continuum emission, which is also shown by the gray scale background. The synthesized beam (0$\farcs$48 $\times$ 0$\farcs$40) is shown in the bottom left corner of the first channel.}
    \label{channel_small}
\end{figure*}

Figure \ref{Con-mom0} shows the velocity-integrated intensity map (moment 0) of C$^{18}$O (2--1) towards the VLA1623 protostellar system (color scale) with overlaid the dust continuum emission at 1.3 mm (black contour) to pinpoint the positions of the protostellar sources A, B, and W, and of the circumbinary disk around the A1+A2 binary system \citep{Harris2018}. The magenta contour indicates the $25\sigma$ level of the CS(5--4) emission integrated on the velocity interval 3.4 -- 4.2 km s$^{-1}$, which probes the outflow cavity walls associated with VLA 1623A \citep[from][]{Ohashi2022}.
% COMMENT: \textcolor{green}{\it I took the $25\sigma$ value and the velocity interval for the integration of CS (5-4) from the paper of Ohashi. Please, check if these values are correct.}.
The C$^{18}$O emission probes the circumbinary disk and the envelope around the
A1+A2 binary system and the outflow cavity walls first identified through CS (5--4) emission \citep{Ohashi2022}. 
C$^{18}$O (2--1) also traces a bright elongated structure south of the circumbinary disk. 
Moreover, C$^{18}$O shows emission towards the VLA 1623W disk continuum, as well as an elongated structure to the North and to the South of W, roughly along the {disk position angle}. This elongated emission cannot be due to a jet or an outflow as it is not perpendicular to the disk PA.
%In the following, we describe in details small- and large-scale structures probed by C$^{18}$O (2-1) towards source W.

To analyse the spatial distribution and kinematics of the different emitting components, we examine the channel maps of C$^{18}$O (see Figs. \ref{channel_large}-\ref{channel_small}). 
At low velocities, i.e. between $+0.2$  km\,s$^{-1}$ and $+3.0$ km\,s$^{-1}$,  C$^{18}$O emission extends on large scales ($>$\,$2\arcsec$ from  the continuum peak of VLA 1623W) and shows arc-like structures (hereafter called streamers) which elongate up to $>$\,$1000$ au distances from VLA 1623W connecting with the emission detected towards VLA 1623A and B (Fig. \ref{channel_large}). The spatio-kinematical properties and the origin of the C$^{18}$O  low-velocity emission is discussed in Sect. \ref{bridges}. 
In contrast, the emission at high velocities, i.e. between $-2.6$ and $-0.4$ km\,s$^{-1}$, and $+3.6$ and $+5.8$ km\,s$^{-1}$, is compact ($<1$\,$\arcsec$ from the VLA 1623W continuum peak) and shows a velocity gradient along the PA of the dusty disk (Fig. \ref{channel_small}). 
This compact high-velocity emission
probes the molecular gas in the disk of VLA 1623W and is discussed next, in Sect. \ref{inner envelope}.
\subsubsection{The gas towards the disk of VLA 1623W}
\label{inner envelope}

Figure \ref{supermoments} (panels c \& d) shows the moment 0 and moment 1 maps of C$^{18}$O (2--1) emission towards VLA 1623W integrated up to high-velocities (from --2.6 km s$^{-1}$ to +5.8 km s$^{-1}$). In this case, only the combined 12-m array is used to minimise contamination from the large scale emission. %The moment 1 map is obtained applying a 3$\sigma$ threshold {\bf on the emission intensity.} 
The continuum emission at 1.3~mm is shown by black contours. A velocity gradient along the disk PA (as derived from the continuum fit, PA$_{\rm disk} \sim 10^\circ$) is observed at a $\sim$ 50 au scale.
The moment 1 map indicates that the systemic velocity of VLA 1623W is V$_{\rm sys}$(W) $\simeq$ +1.6 km s$^{-1}$, as it corresponds to the central velocity of the range where compact blue-shifted and red-shifted emission is detected and to the mean velocity at the peak continuum emission.
In agreement with \citet{Murillo2013}, our C$^{18}$O map indicates that the systemic velocity of VLA 1623W is 
different from that of VLA 1623A and B \citep[+3.8 km s$^{-1}$,][]{Ohashi2022}.
The channel maps in Fig. \ref{channel_small} complement the information on the gas kinematics towards the disk of VLA 1623W, showing the high velocity emission  at symmetric blue-shifted and red-shifted velocities with respect to V$_{\rm sys}$(W) [from ($V_{\rm LSR} - V_{\rm sys}$) $\simeq \pm 2$ km\,s$^{-1}$ to ($V_{\rm LSR} - V_{\rm sys}$) $\simeq \pm 4.2$ km\,s$^{-1}$].  The channels maps of the emission at lower velocity, i.e. from  $V_{\rm sys}$ to $\pm 1.8$ km\,s$^{-1}$ with respect to systemic, are shown in the Appendix (Fig. \ref{appendix}), and shows that at low velocities the kinematics of the gas in the disk is contaminated by emission from the streamers mapped on larger scales in Fig. \ref{channel_large} and panels a) and b) of Fig. \ref{supermoments}. In the channel maps at high velocities, instead, the peaks of the blue- and red-shifted emission are located along the disk major axis, and the emission is more compact and peaks at smaller distances for increasing velocities with respect to V$_{\rm sys}$(W) as expected in a Keplerian rotating disk. 

The previous molecular line study towards VLA 1623W \citep{Murillo2013} showed only C$^{18}$O(2--1) blueshifted emission towards the northern disk side, plausibly due to lower sensitivity. Indeed, we find that the blue-shifted disk side is brighter than the red-shifted one up to velocities of $\pm 2.6$ km s$^{-1}$ with respect to systemic, likely due to contamination from the extended streamers observed at larger scales (see Sect. \ref{bridges}).
At higher velocities the emission from each disk side is  symmetric. We therefore use the emission in the channels at radial velocities of $\pm$ 2.8 and $\pm$ 3.0 km s$^{-1}$ to derive an estimate of the VLA 1623W dynamical mass, $M_{\rm dyn}$. The emission in these channels peak at a radial distance of 0$\farcs$37 and 0$\farcs$33. By assuming Keplerian motion we estimate a dynamical mass,
%
\begin{equation}
  M_{dyn} = r \frac{V^{2}}{G} 
\end{equation}
%
%\footnote{$M_{dyn}$ = $r$ $\times$ $V^{2}$/ $G$, 
where $r$ and $V$ are the distance and velocity of the blue- and red-shifted peaks, deprojected for the disk inclination of 80$\degr$. The estimated dynamical mass is $0.45 \pm 0.08$ M$_{\odot}$. 

 The discovery of molecular emission towards the edge-on source W makes it a good candidate to investigate the gas vertical structure on scales $< 50$ au, as recently performed for highly inclined protoplanetary disks by \cite{Louvet2018, Teague2020, Podio2020}. 
 This is key to investigate the chemical composition of the disk in the region where planets are expected to form.
 
 %{\bf These studies showed that it is important to investigate molecular distribution in highly inclined low-mass star forming disk to understand molecular composition of Solar-like protostars. Since molecules are easier to detect in warm molecular layer with respect to the low inclined disk according to chemical models \citep[e.g.,][]{dutrey2014}.   }
  

%Finally, higher red-shifted and blue-shifted velocity channels peak closer to the protostar. The spatial resolution is not high enough to infer between an inner envelope ($V$ $\propto$ $R^{-1}$) and Keplerian disk rotation ($V$ $\propto$ $R^{-0.5}$), but the C$^{18}$O data indicate a gravitationally bound system. 
%The blue- and red-shifted peaks from the continuum peak are 0$\farcs$37 and 0$\farcs$33 in the channel maps of $\pm$ 2.8 and $\pm$ 3.0 km s$^{-1}$ from the systemic velocity (Fig. \ref{channel_small}).  If we assume Keplerian motion, then the dynamical mass ($M_{\rm dyn}$)\footnote{$M_{dyn}$ = $r$ $\times$ $V_{rot}^{2}$ $\times$ $G^{-1}$} is 0.51 $\pm$ 0.04 M$_{\odot}$. 

\subsubsection{The streamers connecting VLA 1623W with the A+B system}
\label{bridges}
%%\textcolor{green}{COMMENTS for THIS SECTION:}

%%%\begin{itemize}

%%%\item[-] \textcolor{green}{In the result section we present the results, i.e. the detection of two arc-like structures connecting source W with A+B but we do not discuss their origin. The suggestion that the bridges may be accretion streamers should be in the discussion. Hence, we removed the word accretion from the title and the text in this sect.} 

%%%\item[-] \textcolor{green}{We should always use appropriate words. With Claudio we agreed to remove the word "filamentary structures" because filaments refer to star formation sites on large scales detected in molecular clouds. We can refer to "arcs", "arc-like structures", "bridges". }
%%%%\end{itemize}

Figure \ref{channel_large} shows the channel maps of C$^{18}$O (2--1) emission at low velocities, i.e., between +0.2 km s$^{-1}$ and +3.0 km s$^{-1}$. Three streamers connecting VLA 1623A and B with VLA 1623W are observed: one in the northern VLA1623 region, labeled as N, the other two in the southern region labeled as S1 and S2 and detected on velocities of [$+1.2$, $+1.8$] km s$^{-1}$, and [$+2.0$, $+2.8$] km s$^{-1}$, respectively.
Figure \ref{supermoments} shows the moment 0 (panel a) and moment 1 (panel b) maps obtained for the low velocity range ([+0.2, +3.0] km s$^{-1}$). Both figures use the 12-m + 7-m dataset.
The northern streamer partially overlaps with the North-West blueshifted cavity wall opened by the
outflow(s) driven by VLA 1623A \citep[see the yellow/white contours, from][]{Ohashi2022}. 
At distances from VLA 1623A larger than $\sim$ 1000 au, the molecular emission bends towards the south until it connects to the northern side of the VLA 1623W disk.
In the southern region, the streamer S1 overlaps with the South-West cavity wall opened by the outflow(s) driven by VLA 1623A. The streamer S2, on the other hand, connects the envelope surrounding VLA 1623A and B with the southern side of the VLA 1623W disk extending towards the South. In summary, the S1 and S2 streamers have both different spatial distribution and different velocities, therefore they are labelled as different streamers.

The velocities of the observed streamers are consistent with those of the outflow cavities and the envelope probed by CS(5--4) and H$^{13}$CO$^{+}$ emission by \citet[][see their Fig. 12, 13, and 16]{Ohashi2022}, and with the velocities of the VLA 1623W disk: the blueshifted outflow cavity/streamer north of VLA1623A connects with the northern blueshifted side of the VLA 1623W edge-on disk, while the redshifted outflow cavity/streamer south of VLA 1623A connects with the southern redshifted side of the VLA 1623W disk. This indicates that the protostellar sources A and W are kinematically linked. 
%showing that the observed morphology is not due to projection effects of different molecular structures. 
%In addition, the moment 1 map indicates that the bridges are associated with velocity gradient, more specifically,  in the southern one, where the emission close to source W is definitely more red-shifted (by $\sim$ 1 km s$^{-1}$) with respect to the portion of the structure close to the envelope hosting the A+B protostar. These findings suggest accretion motion towards the W protostar.
In addition, the moment 1 map of  C$^{18}$O shows a velocity gradient along the southern streamer having larger redshifted velocities  (by $\sim$ 1 km s$^{-1}$) at the connection with the VLA 1623W disk  with respect to the portion of streamer connected with the VLA 1623A+B envelope. 
%This could indicate that the arc streams material from the A+B envelope towards the disk of VLA 1623W.
On the contrary, no clear velocity gradient is observed along the northern streamer. The lack of a velocity gradient along the northern streamer may indicate that the gas motion occurs in the plane of the sky \citep{Alves2020}.

\begin{figure*}
\centering
\vspace{-2cm}
	\includegraphics[width=12.5cm, angle =90]{Fig4.pdf}
	\vspace{-1cm}
    \caption{Integrated intensity (moment 0) and intensity-weighted mean velocity (moment 1) maps of C$^{18}$O (2--1) towards VLA 1623--2417. The RA and Dec offsets are with respect to the position of VLA 1623W. \textit{Panels a) and b)}: Moment 0 and 1 maps over the low velocity range ([+0.2, +3.0] km s$^{-1}$). 
    %(see Fig. \ref{channel_small}). 
    The white stars indicate the A, B, and W protostars, the white contour in Panel a) indicates the 3$\sigma$ emission (9 mJy km s$^{-1}$ beam$^{-1}$), the ellipse shows
    the synthesized beam (0$\farcs$53 $\times$ 0$\farcs$44), and the yellow in Panel a) and the white contour in Panel b) reveal the 25$\sigma$ CS (5--4) emission probing the cavity outflow walls associated with VLA 1623A \citep[from][]{Ohashi2022}. \textit{Panels c) and d)}: Moment 0 and moment 1 maps of VLA 1623W over the high velocity range ([--2.6, +5.8] km s$^{-1}$). Continuum emission at 1.3~mm is shown by the white contours. The black and white stars indicate the position of VLA 1623W, and the white ellipse the synthesized beam (0$\farcs$48 $\times$ 0$\farcs$40).}
    \label{supermoments}
\end{figure*}

\section{Discussion: On the origin of VLA 1623W}
\label{Discussion}

The FAUST ALMA observations of the multiple system VLA 1623 reveal for the first time the gas kinematics towards the more distant component, VLA 1623W. Specifically, C$^{18}$O (2-1) emission probes the following structures: (i) the molecular gas in the disk of VLA 1623W with Keplerian motion on $50-100$ au scales, constraining the protostellar mass (M$_{dyn}\sim0.45$ M$_{\odot}$); (ii) streamers which extend on scales $> 1000$ au and connect spatially and kinematically the two sides of the edge-on disk of VLA 1623W (PA$_{\rm disk} \sim 10^\circ$, $i\sim 80^\circ$, $M_{\rm disk} \sim 6\times10^{-3}$ M$_{\odot}$) with the envelope associated with VLA 1623A and B.

Molecular streamers have been observed in other multiple systems, e.g., between the components of IRAS 16293-2422 separated by a distance of $\simeq$ 400 au \citep{Pineda2012,Jacobsen2018,vanderWiel2019,Murillo2022}.
In the case of VLA 1623, however, the large spatial ($\sim 1300$ au) and kinematic ($\sim 2.2$ km\,s$^{-1}$) offset between VLA 1623AB and VLA 1623W suggest that VLA 1623W does not belong to the same molecular core as VLA 1623AB. On the other hand, the molecular streamers connecting the disk of VLA 1623W with the envelope/cavities of VLA 1623A and VLA1623B suggest that VLA 1623W is not a background or foreground object.
In this context, there are two possible scenarios: (1) VLA 1623W  has formed in a core which is close by and gravitationally interacting with the envelope of VLA 1623AB; (2) VLA 1623W was ejected from the multiple system composed by A1, A2, and B due to dynamical interactions during the system’s formation, as first proposed by \citet{Murillo2013}. 

\subsection{Hypothesis 1: formation in a separate core}


In the first scenario the observed streamers are produced by the interaction of multiple cores in the Ophiuchus A star-forming region hosting the VLA 1623A, B and W protostars. 
\citet{Chen2018} mapped Ophiuchus A at spatial resolution $\geq$ 5$\arcsec$ to investigate the physical and chemical properties of the region in continuum and molecular lines by combining interferometric (SMA) and single-dish (IRAM-30m) data. The maps show that Ophiuchus A consists of three ridges aligned along the north-south direction and that VLA 1623A+B and VLA 1623W are located in adjacent ridges \citep[see Figs. 1 and 4 by][]{Chen2018}.
Therefore, the streamers probed by C$^{18}$O (2--1) which connect VLA 1623A+B with VLA 1623W could be the signature of the gravitational pull across adjacent ridges.
However, the maps by \citet{Chen2018} indicate that the systemic velocity of the two adjacent ridges is similar (both being between +3 and +4 km s$^{-1}$), while our observations indicate that VLA 1623W has a systemic velocity of $+1.6$ km\,s$^{-1}$, which differs from the systemic velocity of VLA 1623AB ($+3.8$ km\,s$^{-1}$) and the two ridges. Thus, this first scenario is unlikely.

\subsection{Hypothesis 2: ejection or flyby}
\label{ejection}

In the second scenario the protostellar source VLA 1623W formed in the same envelope as VLA 1623A and B, and has been later ejected from the multiple system due to a close interaction 
%with the rest of the system  during its formation 
\citep{Murillo2013}.
According to \citet{Harris2018}, the later evolutionary stage of VLA 1623W (Class I) could be due to the loss of most of its original parental envelope during the ejection.
The ejection scenario is further supported by the kinematics of the system.
%This scenario is supported by observational evidence that the system composed by VLA 16293A1, A2, and B is unstable. 
The ratio between the kinematical and the gravitational energy of the system composed by A, B, and W (protostellar masses of 0.4  M$_{\odot}$, 1.7  M$_{\odot}$, and 0.45 M$_{\odot}$, envelope masses of 0.8  M$_{\odot}$, 0.2  M$_{\odot}$, and 0.04  M$_{\odot}$) is $\simeq 1$, which indicates that the system is likely unstable \citep{Pineda2015}.
%{\textcolor{red}{About unstability}}{\bf Considering the all kinematics, the ratio between velocity and mass distribution is close to 1, which means the system is likely unstable \citep{Pineda2015}. {\textcolor{red}{We did not mention all mass values, so maybe in the introduction, i should?} 
%This is also supported by the fact that source B has a counter rotation with respect to the rotation of the all system. 
%{\bf The following properties of the system suggest that the system was disrupted by close encounter of the multiples.} {\textcolor{red}{these next sentences up to black bold sentences can be ruled out maybe, since its detailed information about the system, so we could gain some lines:}}}
Other signatures of instability are: (i) the axis of the circumbinary disk around the A1+A2 binary is misaligned by 12$\degr$ with respect to both the large-scale outflow and the rotation axis of the molecular envelope \citep{Ohashi2022}; (ii) 
the edge-on disk of VLA 1623B counter rotates with respect to the outflow driven by VLA 1623A and the surrounding envelope; and (iii) the circumstellar disks of A1 and A2 have inclinations which may differ by $\sim 70\degr$ based on the orientation of the high-velocity outflows \citep{Harris2018,Murillo2018,Ohashi2022,Codella2022}. 
Given the high dynamic instability, all of the protostars in the system might have been  bound at the time of their formation, with one or more components later ejected due to a close encounter \citep[e.g.,][]{Reipurth2012, Pineda2015}.

%As a consequence, the protostars should be pulled towards different directions in space.

%To further test the ejection scenario we estimate the velocities on the plane of the sky of VLA 1623W with respect to A and B by using VLA data at 3.6~cm taken in 1991 \citep{Andre1993} and our ALMA data at 1.3~mm (see online Appendix A). In agreement with \citet{Harris2018}, our measurements show no significant difference between the separation of W and A (or B) across a time range of 27.7 years. Therefore, their plane of the sky velocities are the same at the 3$\sigma$ level ($\sigma \sim 1.3-1.5$ km\,s$^{-1}$). 

In the ejection scenario, the velocity gradient of 1 km\,s$^{-1}$ detected along the southern streamer indicates either material falling on VLA 1623W if the streamer is located in front (i.e. between W and the observer) or  alternatively gas moving away from VLA 1623W towards A+B if the streamer is located on the other side of W.
%More specifically, Fig. \ref{supermoments} indicates a spatial and kinematical continuity between the NW cavity and the northern bridge.
%The southern bridge, on the other hand, connect W with the molecular envelope hosting the A+B system.
%Further measurements of the proper motions of the VLA1623 protostars are needed to definitely assess if VLA 1623W is ejected from the multiple system formed by the A and B protostellar objects. 
Note that the close encounter that occurs during the ejection of one of the members of a multiple stellar system has a similar dynamical effect as that of a stellar flyby \citep{Cuello2023}. In the case of a flyby, if the outer perturber, VLA 1623W, follows a prograde orbit near the disk-envelope of A+B this could lead to the formation of streamers like the ones observed in C$^{18}$O \citep[e.g., UX Tau,][]{Menard2020, Zapata2020}. It is, however, puzzling that in VLA 1623 both streamers appear to point towards VLA 1623W since tidal perturbations of the disk typically trigger the formation of two diametrically-opposed streamers, pointing in opposite directions \citep{Clarke1993,pfalzner2003,Cuello2020}.
Interestingly, in the moment 0 maps in Figs. \ref{Con-mom0} and \ref{supermoments} we tentatively detect a spiral arm southern to VLA 1623AB which is diametrically-opposed with respect to the northern spiral arm, N, which connect AB with W. If so, this would indicate that W and AB  interacted recently forming the diametrically opposed spiral arms during the encounter. 
In this scenario the two southern arcs detected in the moment 1 map (S1 and S2) would not be associated to the encounter between W and the AB system.
%I think the scenario we propose makes more sense. I would re-emphasise that this system is a gigantic mess and that kinematic structures are hard to interpret in the presence of accretion and several stars presumably perturbing each other. With this explanations, at least we do not need to explain two spirals pointing towards W, which is extremely puzzling for a flyby.
%: Referee says This seems to contradict the fly-by or ejection scenario. Is there a way to produce streamers with the same orientation in an ejection scenario? This is a huge caveat of the interpretation and needs further discussion.}
%\textcolor{red}{As I understand Nicolas said that we can make a point to Mom 0 map (panel a) mentioning spiral arms between system A+B and source W and another direction from system A+B towards south-west direction.If so, please see revised text in blue, starting from the original text "In the case of a flyby, if the outer perturber, VLA 1623W,...."} \textcolor{blue}{In the case of a flyby, if the outer perturber, source W and system A+B with disks before the encounter have interacted recently and following, the spiral arms have formed during the encounter. Moment 0 in Fig. \ref{supermoments} shows that in the circumbinary disk around system A+B, the two spiral arms are diametrically opposed, and the northern arm of W is likely part of the "bridge" that forms in between the two stars \citep[e.g., UX Tau,][]{Menard2020, Zapata2020}.
%}\textcolor{red}{Then we can quit the part: 
%It is, however, puzzling that in VLA 1623 both streamers appear to point towards VLA 1623W since tidal perturbations of the disk typically trigger the formation of two diametrically-opposed streamers, pointing in opposite directions \citep{Clarke1993,pfalzner2003,Cuello2020} As I understand from the sum-up from Nicolas.} 
%pfalzner2003,
More in general, assuming that at least one of the streamers was produced by the dynamical interaction during the ejection of VLA 1623W, then the streamers' misalignment with respect to the disks' planes can be due to the fact that misaligned stellar flybys are more likely than coplanar ones \citep{Bate2018, Cuello2019}. 

We stress that both the ejection of a member in young multiple systems as well as stellar flybys are very common processes. In both cases the expected velocities and eccentricities may cover a broad range of values \citep[e.g., ][]{Cuello2023}, therefore, to first order, flybys and ejections leave similar signatures. To distinguish between the two scenarios would require observations at different epochs and accurate astrometry and radial velocity measurements in order to reconstruct the orbit.

\begin{table*}
    \centering
    \begin{tabular}{lllll}
    \hline
     & \multicolumn{2}{c}{Offset between A and W} & \multicolumn{2}{c}{Offset between B and W}  \\
     \hline
   & VLA-X  & ALMA-B6 & VLA-X  & ALMA-B6 \\
   & 1991.8 & 2019.5  & 1991.8 & 2019.5 \\
   \hline
    $\delta$R.A. & $-10\farcs56 \pm 0\farcs03$ &$-10\farcs383 \pm 0\farcs003$ &$-9$\farcs$37 \pm 0\farcs02$ &$-9$\farcs$218 \pm 0\farcs001$ \\
    $\delta$Dec. & $+1\farcs39 \pm 0\farcs03$&$+1\farcs260 \pm 0\farcs003$ &$+1\farcs39 \pm 0\farcs03$ &$+1\farcs120 \pm 0\farcs002$ \\
    $\rho$ &$10\farcs65 \pm 0\farcs06$ &$10\farcs46 \pm 0\farcs02$ & 9$\farcs44 \pm 0\farcs05$& 9$\farcs29 \pm 0\farcs01$ \\
    \hline
    \end{tabular}
    \caption{ The offset (in arcsec) in right ascension ($\delta$R.A.) and declination ($\delta$Dec.), and the  resulting separation ($\rho$) between source A and W and between source B and W at the epochs of the VLA-X observations (1991.8) and the ALMA-Band 6 observations (2019.5).}
     \label{propoermotion}
\end{table*}
\subsection{Proper motions between 1991.8 and 2019.5}
\label{proper motion}
 
In order to {\bf test} the ejection scenario, we estimate the proper motions of VLA 1623W with respect to VLA 1623A and B by combining our FAUST ALMA Band 6 observations of the continuum at 1.3~mm, taken in 2019, with the data at 3.6~cm, taken with the VLA in the X-band in 1991 (\cite{Andre1993}, project AB817).
We measured the separation between A and W, and between B and W, in the VLA-X (1991.8) and ALMA-B6 (2019.5) images, which are separated by 27.7 years. 
Table \ref{propoermotion} reports the offset (in arcsec) in right ascension ($\delta$R.A.) and declination ($\delta$Dec.), and the resulting separation ($\rho$) between source A and source W and between source B and source W  at the two epochs.
%%%%%%%%%%

From these values, the velocities on the plane of the sky between the sources can be derived. Namely, the velocity on the plane of the sky of A with respect to W is ($-4.4 \pm 1.47$) km s$^{-1}$, while that of B with respect to W is ($-3.5 \pm 1.26$) km s$^{-1}$.
These measurements are in agreement with the proper motions estimated by \cite{Harris2018} based on ALMA observations taken in 2013 \citep{Murillo2013} and 2016 but are affected by a lower uncertainty given the larger distance between the two epochs.
Based on the above estimates, there is no significant difference in tangential velocity between VLA 1623W and A (or B) at the three sigma level. 
However, in the ejection scenario the velocity needed for VLA 1623W to move away from A+B up to a distance of $\sim10\arcsec$ in 10$^4$ years (i.e., the typical Class 0 age) is only 0.65 km\,s$^{-1}$, i.e. below the uncertainty associated with our proper motion measurements ($\sim 1.3-1.5$ km\,s$^{-1}$). 
%However, if the system is 10$^{4}$ years old (i.e. the typical age of a Class 0 system), then VLA 1623W would need a velocity of $\ge 0.65$ km\,s$^{-1}$ (i.e. below the uncertainty on the estimated velocity) to reach its present position 10$\arcsec$ away from A and B.
Thus, the ejection scenario can be neither ruled out nor confirmed by the available proper motion.


\section{Conclusions}
\label{Conclusions}

We report observations of the continuum at 1.3~mm and C$^{18}$O (2--1) line emission towards the multiple protostellar system VLA 1623--2417.
We reveal for the first time the gas associated with the edge-on disk of VLA 1623W. From the gas kinematics, assuming Keplerian rotation, we estimate the source dynamical mass (M$_* = 0.45 \pm 0.08$ M$_{\odot}$).
Moreover, we reveal three streamers connecting VLA 1623W with the VLA 1623A+B system. 
The spatial ($\sim 1300$ au) and velocity ($\sim 2.2$ km\,s$^{-1}$) offset of VLA 1623W with respect to A+B, and its later evolutionary stage (Class I), suggest that either sources W and A+B formed in different cores, or source W has been ejected from the multiple system during its formation, due to the interaction with one of its member.
The available data on proper motions cannot confirm or rule out neither of the two scenarios. Additional kinematical constraints are required in order to test the stellar encounter scenario for VLA 1623.
In addition, observations of shock tracers, such as SO and SiO, at a spatial resolution $\sim 10$ au will allow for verification that the observed streamers are feeding the disk of VLA 1623W, or that vice versa they funnel material from W to A and B. Such observations are required to assess the importance of streamers for disk formation and evolution.

%%% PROPOSED BY SEYMA: CLAUDIO: I DO NOT FULLY UNDERSTAND. LET'S SEE LINDA.
%On the other side, the southern bridge reaching the source W along the N-S direction could interact with another outflow driven by source B that has been showed with CO emission by \citet{Hsieh2020}. 
%They proposed that the colliding outflow meets rotating-infall material creating SO shock region in the circumstellar disk, where we starts seeing in south of system A+B in channel map with +1.4 km s$^{-1}$ and finishes at +2.4 km s$^{-1}$ in Fig. \ref{channel_large}. Interestingly, the southern accretion streamer towards source A+B (between +2.0 and +2.4 km s$^{-1}$) looks located outflow cavity originating from \citet{Hsieh2020}. This feature could affect the interaction between accretion streamer from system A+B to the south streamer of source W. (\textcolor{red}{Here you can check the overlaid map, showing Hshieh map (CO outflow driven by source B) and one channel map of C18O at +2.2 km/s ))see please example in fig. \ref{screenshot}}, that we discussed a bit in the last meeting with Tomoyuki.) 


\section*{Acknowledgements}

This project has received funding from the EC H2020 research and innovation
programme for: (i) the project "Astro-Chemical Origins” (ACO, No 811312),  (ii) the European Research Council (ERC) project “The Dawn of Organic
Chemistry” (DOC, No 741002), (iii) the ERC project "Stellar-MADE" (No. 101042275). This study is also supported by grants-in-aid from the Ministry of Education, Culture, Sports, Science, and Technology of Japan (18H05222, 19H05069, 19K14753, and 21K13954), by the Spanish Ministry of Science and Innovation/State Agency of Research MCIN/AEI/10.13039/501100011033 (PID2019-105552RB-C41), “ERDF A way of making Europe”, by the DGAPA PAPIIT grants IN112417 and IN112820, CONACYT-AEM grant 275201, and CONACYT-CF grant 263356, and by
the German Research Foundation (DFG) as part of the Excellence Strategy of the federal and state governments - EXC 2094 - 390783311. 
We are grateful to R.\ Neri for fruitful discussions. D.J.\ is supported by NRC Canada and by an NSERC Discovery Grant. G.B. acknowledges funding from the State Agency for Research (AEI) of the Spanish MCIU trough the PID2020-117710GB-I00 grant funded by MCIN/AEI/10.13039/501100011033. We thank the anonymous referee for the very constructive comments and suggestions.


%%%%%%%%%%%%%%%%%%%%%%%%%%%%%%%%%%%%%%%%%%%%%%%%%%
\section*{Data Availability}
The raw data will be available on the ALMA archive at the end of the proprietary period (ADS/JAO.ALMA\#2018.1.01205.L).


%%%%%%%%%%%%%%%%%%%% REFERENCES %%%%%%%%%%%%%%%%%%

% The best way to enter references is to use BibTeX:

\bibliographystyle{mnras}
\bibliography{example} % if your bibtex file is called example.bib
%\section*{Affiliations}
\noindent


% Alternatively you could enter them by hand, like this:
% This method is tedious and prone to error if you have lots of references
%\begin{thebibliography}{99}
%\bibitem[\protect\citeauthoryear{Author}{2012}]{Author2012}
%Author A.~N., 2013, Journal of Improbable Astronomy, 1, 1
%\bibitem[\protect\citeauthoryear{Others}{2013}]{Others2013}
%Others S., 2012, Journal of Interesting Stuff, 17, 198
%\end{thebibliography}

%%%%%%%%%%%%%%%%%%%%%%%%%%%%%%%%%%%%%%%%%%%%%%%%%%

%%%%%%%%%%%%%%%%% APPENDICES %%%%%%%%%%%%%%%%%%%%%

%\newpage
%
\appendix

\section{Channel maps of C$^{18}$O ($2-1$) towards source W at low velocities}

Figure \ref{appendix} shows the channel maps of C$^{18}$O (2--1) emission towards source W at low-velocities, i.e. up to  $\pm1.8$ km s$^{-1}$  with respect to the systemic velocity of W ($V_{\rm sys}$(W) $\sim +1.6$ km s$^{-1}$). 
The maps show that the kinematics of the gas in the disk, which is well detected at high-velocities (see Fig. \ref{channel_small}), is affected by the extended emission from the streamers and/or the residual envelope at low velocities.


\begin{figure*}
\centering
\vspace{-2cm}
\includegraphics[width=12.5cm, angle =90]{low_vel_appendix.pdf}
\vspace{-2cm}
    \caption{Channel maps of C$^{18}$O (2--1) emission towards the VLA 1623W protostar (green star) at low red- and blue-shifted velocities, i.e. up to $\pm1.8$ km s$^{-1}$ with respect to the VLA 1623W systemic velocity ($V_{\rm sys}$(W) $\sim +1.6$ km s$^{-1}$).  First contours and steps are 3$\sigma$ (7.7 mJy beam$^{-1}$). The velocity offset with respect to $V_{\rm sys}$(W) is reported in the bottom left corner of each panel. The black contour is the 3$\sigma$ level of the 1.3~mm continuum emission, which is also shown by the gray scale background. The synthesized beam (0$\farcs$48 $\times$ 0$\farcs$40) is shown in the top left corner of the first channel. }
    \label{appendix}
\end{figure*}

%%%%%%%%%%%%%%%%%%%%%%%%%%%%%%%%%%%%%%%%%%%%%%%%%%


% Don't change these lines


% Don't change these lines
\bsp	% typesetting comment
\label{lastpage}
\end{document}

% End of mnras_template.tex
