% \section{Appendix}
\section{Interaction between Functional Neurons and Gates inside GRU}
After realizing the validity of four functional neurons, we go deep into the formulation of GRU, including each gate and its corresponding weight.
% After the model encoded the word information with storing neuron and the position information with counting neuron, it raised another question to  solve: what is the interaction between storing and counting neuron to trigger model output the right word at the right time-step ? 
In this section, we combine the mathematical and the empirical perspective to investigate the relationship between functional neurons and gates. In Fig. X, we visualize the neuron interaction  to clarify our argument. 

\begin{figure}[h]
    \centering
    \includegraphics[width=1.0\textwidth]{figures/appendix_trig_vis.png}
    \caption{Visualization of neurons interaction. }
    \label{gruv-1}
\end{figure}

%\newpage

Recall the mathematical formulation of GRU cell:

% \begin{small}
\begin{equation*}
% \begin{aligned}
GRU\begin{cases}
    &z_t = \sigma(W_{z}x_t + U_{z}h_{t-1} + b_z)  \\
    &r_t = \sigma(W_{r}x_t + U_{r}h_{t-1} + b_r) \\
    &\widetilde{h_t} = \text{tanh}(W_{h}x_t + U_{h}(r_t \odot h_{t-1}) + b_h)  \\ 
    &h_t = z_t \odot h_{t-1} + (1 - z_t) \odot \widetilde{h}_t\\
    \end{cases} \quad
GRU^{'}\begin{cases}
    &z_t = \sigma(U_{z}h_{t-1} + b_z)  \\
    &r_t = \sigma(U_{r}h_{t-1} + b_r) \\
    &\widetilde{h_t} = \text{tanh}(U_{h}(r_t \odot h_{t-1}) + b_h)  \\ 
    &h_t = z_t \odot h_{t-1} + (1 - z_t) \odot \widetilde{h}_t
    \end{cases}
% \begin{aligned}
%     &z_t = \sigma(W_{z}x_t + U_{z}h_{t-1} + b_z)  \\
%     &r_t = \sigma(W_{r}x_t + U_{r}h_{t-1} + b_r) \\
%     &\widetilde{h_t} = \text{tanh}(W_{h}x_t + U_{h}(r_t \odot h_{t-1}) + b_h)  \\ 
%     &h_t = z_t \odot h_{t-1} + (1 - z_t) \odot \widetilde{h}_t\\
% \end{aligned}
\label{GRU_Formula}
\end{equation*}
% \end{small}
% \caption{The original formulation of GRU and its simplified version}


with $z_t, r_t, h_t, \widetilde{h}_t$ as the \emph{update gate}, \emph{reset gate}, \emph{activation}, and \emph{candidate activation}. Since the random replacement experiment shows that $x_t$ has very few influence on $z_t, r_t, h_t$ that can be neglected, we simplify the formulation of GRU cell and noted as $GRU^{'}$.

% \begin{small}
% \begin{equation}
% \begin{aligned}
%     &z_t = \sigma(U_{z}h_{t-1} + b_z)  \\
%     &r_t = \sigma(U_{r}h_{t-1} + b_r) \\
%     &\widetilde{h_t} = \text{tanh}(U_{h}(r_t \odot h_{t-1}) + b_h)  \\ 
%     &h_t = z_t \odot h_{t-1} + (1 - z_t) \odot \widetilde{h}_t
% \end{aligned}
% \label{GRU}
% \end{equation}
% \end{small}

Then we can simply view the operation inside the $\sigma(\cdot)$ function as a linear transformation of $h_{t-1}$ to $z_t, r_t$ or $h_t$. Therefore, the weight $U_z, U_r, U_h$ are indeed the feature importance. In order to verify our hypothesis that triggering neurons trigger the storing neurons to affect the outputting neurons, we take the transition weight in $U_z, U_r, U_h$ as the measure for the triggering capability to the specific storing neuron and outputting neurons. Following the formulation above, for \emph{z gate}, the transition weight from triggering neuron to outputting neurons should be low in order to update the neuron value in $h_{t-1}$ (in Fig. X(a)), and for the \emph{r gate}, the transition weight from trigger neuron to storing neuron should be high in order to make $h_{t-1}$ effective to $\widetilde{h_t}$  or it will be masked.(in Fig. X(b))  After the value in each gate are determined, the transition weight in $U_h$ should be extreme value in order to trigger the outputting neurons to reach extreme value as well.(in Fig. X(c)) Additionally, we apply integrated gradient to check the impact from triggering neuron to outputting neurons, which shows the same result as directly analyzing the inner operation of GRU cell. Statistics are listed in Figure X and Table X.





% \begin{table}[h]
% 	\centering
% 	\begin{tabular}{ccccc}
% 		\toprule
% 		Neurons set & Z gate & R gate & H gate \\
% 		\midrule
% 		$h_{T-1, store}$ & 7.9E-3 & 7.3E-3 & 6.7E-3\\
% 		$h_{T-1, trig}$ & 1.3E-2 & 9.8E-3 & 1.6E-3 \\
% 		Whole $h_{T-1}$ & 4.0E-3 & 5.1E-3  & 2.4E-3\\
% 		\bottomrule
		
% 	\end{tabular}
	
% 	\caption{Different $S_{Imp}$ importance through different gates by calculating integrated gradient score. The first and second columns show that $h_{T-1, trig}$ is the most important through z and r gate, whereas the third column shows that $h_{T-1, store}$ is the most important through h gate.}
% 	\label{IG_score}
% \end{table}


% \begin{figure}[h]
%   \centering
%   \includegraphics[width=0.48\textwidth]{figures/appendix_z_impact.png}
%   \caption{Impact of triggering neurons to decrease \emph{z gate} value of outputting neurons. X-axis are sixteen $S_{fc,tok}$ neurons. Y-axis represents the impact from $h_{T-1, trig}$ (blue bars) and other neurons (red bars). Most blue bars have obvious negative values and it implies that it is $h_{T-1, trig}$ to increase $z_{T, fc}$.}
%   \label{z_impact}
% \end{figure}




\begin{figure}[h]
    \begin{center}
    
    \begin{minipage}[b]{0.43\textwidth}
        \centering
        	\begin{tabular}{ccccc}
    		\toprule
    		Neurons set & Z gate & R gate & H gate \\
    		\midrule
    		$h_{T-1, store}$ & 7.9E-3 & 7.3E-3 & 6.7E-3\\
    		$h_{T-1, trig}$ & 1.3E-2 & 9.8E-3 & 1.6E-3 \\
    		Whole $h_{T-1}$ & 4.0E-3 & 5.1E-3  & 2.4E-3\\
    		\bottomrule
        		
        	\end{tabular}
        	
        	\caption{Different $S_{Imp}$ importance through different gates by calculating integrated gradient score. The first and second columns show that $h_{T-1, trig}$ is the most important through z and r gate, whereas the third column shows that $h_{T-1, store}$ is the most important through h gate.}
    \end{minipage}
    \hfill
    \begin{minipage}[b]{0.53\textwidth}
         \centering
         \includegraphics[width=0.6\textwidth]{figures/appendix_z_impact.png}
          \caption{Impact of triggering neurons to decrease \emph{z gate} value of outputting neurons. X-axis are sixteen $S_{fc,tok}$ neurons. Y-axis represents the impact from $h_{T-1, trig}$ (blue bars) and other neurons (red bars). Most blue bars have obvious negative values and it implies that it is $h_{T-1, trig}$ to increase $z_{T, fc}$.}
          \label{z_impact}
        
    \end{minipage}%
    \end{center}
    
    \label{POS_10}
\end{figure}

