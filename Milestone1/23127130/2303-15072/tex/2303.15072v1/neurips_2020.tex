\documentclass{article}

% if you need to pass options to natbib, use, e.g.:
%     \PassOptionsToPackage{numbers, compress}{natbib}
% before loading neurips_2020

% ready for submission
% \usepackage{neurips_2020}

% to compile a preprint version, e.g., for submission to arXiv, add add the
% [preprint] option:
\usepackage[preprint,nonatbib]{neurips_2020}

% to compile a camera-ready version, add the [final] option, e.g.:
%     \usepackage[final]{neurips_2020}

% to avoid loading the natbib package, add option nonatbib:
%     \usepackage[nonatbib]{neurips_2020}

\usepackage[utf8]{inputenc} % allow utf-8 input
\usepackage[T1]{fontenc}    % use 8-bit T1 fonts
\usepackage{hyperref}       % hyperlinks
\usepackage{url}            % simple URL typesetting
\usepackage{booktabs}       % professional-quality tables
\usepackage{amsfonts}       % blackboard math symbols
\usepackage{nicefrac}       % compact symbols for 1/2, etc.
\usepackage{microtype}      % microtypography

% Customized import packages
\usepackage{amsmath}        % For aligned equations
\usepackage{graphicx}       % figure
%%\usepackage{verbatim}       % comment

%\usepackage{algorithm}
%\usepackage[]{algorithm2e}
\usepackage{import}         % split each section into different files
%%\usepackage{caption}        % subfigure
%%\usepackage{subcaption}     % subfigure
%%\usepackage{multicol}       % multicol in table
%%\usepackage{wrapfig}        % place figure and text in the same row
%%\usepackage{subfig}         % subfig
%%\usepackage{makecell}
%%\usepackage{mathtools}
%%\usepackage[shortlabels]{enumitem}
\usepackage[dvipsnames]{xcolor}   % For customized colors.
\usepackage{comment}         % For comment
\usepackage{cite} % For citation
\usepackage[utf8]{inputenc}
\usepackage{amsmath}



%\title{Explore the unit functions of gated recurrent neural network by explaining token-positioning task}
\title{ Exposing the Functionalities of Neurons for Gated Recurrent Unit Based Sequence-to-Sequence Model}

% \title{Why a Vanilla GRU-based Seq2Seq model can perform token positioning?}

% The \author macro works with any number of authors. There are two commands
% used to separate the names and addresses of multiple authors: \And and \AND.
%
% Using \And between authors leaves it to LaTeX to determine where to break the
% lines. Using \AND forces a line break at that point. So, if LaTeX puts 3 of 4
% authors names on the first line, and the last on the second line, try using
% \AND instead of \And before the third author name.

\author{
Yi-Ting Lee, Da-Yi Wu, Chih-Chun Yang, Shou-De Lin\\
 Department of Computer Science \\and Information Engineering, \\
National Taiwan University  \\
}
 %\texttt{tiffany70072@gmail.com, r07922119@csie.ntu.edu.tw, }\\
  %\And
  
\begin{document}

\maketitle

\begin{abstract}
  The goal of this paper is to report certain scientific discoveries about a Seq2Seq model. It is known that analyzing the behavior of RNN-based models at the neuron level is considered a more challenging task than analyzing a DNN or CNN models due to their recursive mechanism in nature. This paper aims to provide neuron-level analysis to explain why a vanilla GRU-based Seq2Seq model without attention can achieve token-positioning. We found four different types of neurons: storing, counting, triggering, and outputting and further uncover the mechanism for these neurons to work together in order to produce the right token in the right position. %We also propose a 3-stage strategy to identify, filter, and verify the functionality of neurons. % propose a series of strategies that allow us to perform neuron-level analysis for a Seq2Seq model. Our goal is to understand and explain why and how a Seq2Seq model can perform token-positioning with high accuracy. We have identified various kinds of neurons that perform different functions, and explained the mechanism about how they can work jointly to achieve the task. 
\end{abstract}



\import{sections/}{section1-introduction.tex}
\import{sections/}{section2-experiment-setting.tex}
\import{sections/}{section3-methods.tex}
\import{sections/}{section4-results.tex}
\import{sections/}{section5-interaction.tex}
%\import{sections/}{ck20jimmy_Finding_Summary.tex}
%\import{sections/}{ck20jimmy_Gate_Interaction.tex}
\import{sections/}{section6-related-works.tex}
\import{sections/}{section7-conclusion.tex}

%%%%%%%%%%%%%%%%%%%%%%%%%%%%%%%%%%%%%%%%
\import{sections/}{Impact_statement}

\bibliography{neurips_2020.bib}
\bibliographystyle{plain}

% \usepackage[
% backend=biber,
% sorting=unsrt]{biblatex}
% \addbibresource{neurips_2020.bib}

% As-is
%\printbibliography


\end{document}