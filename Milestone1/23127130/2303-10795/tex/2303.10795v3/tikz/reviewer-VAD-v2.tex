\documentclass[tikz]{standalone} 
\input{ncsuColors}
\usepackage{pgfplots} % package used to implement the plot  
\pgfplotsset{width=6.5cm}  
\pgfplotsset{compat=newest}  
\pgfplotsset{every axis plot/.append code= {#1}}  

\begin{document}  
\begin{tikzpicture}  
\begin{axis}  
[   axis y line*=none,
    axis x line*=bottom,
    xbar,
    enlargelimits=0.22,% these limits are used to shrink or expand the graph. The lesser the limit, the higher the graph will expand or grow. The greater the limit, the more graph will shrink.
    legend style={
        at={(0.34,-0.1)},
        % at={(0.75,0.67)},
        anchor=north,
        legend columns=-1
        },     
    xmin = 0.1,
    xmax = 0.75,
    % y axis line style = {opacity = 0},
    % axis x line = none,
    % x tick label style={/pgf/number format/1000 sep=},
    tickwidth = 0pt,
    % xlabel={},
    % xtick distance= 0.1,
    % xlabel style={font=\footnotesize\sffamily},
    symbolic y coords={Valence,Arousal,Dominance},
    % ytick={0,500,...,10000},
    nodes near coords,  
    nodes near coords style={font=\footnotesize\sffamily},
    % nodes near coords align={vertical},
    style={font=\footnotesize\sffamily},
    ]  
\addplot[draw=none,fill=ncsuBlue,font=\footnotesize\sffamily,label={gffd}] coordinates {(0.56,Valence) (0.51,Arousal) (0.52,Dominance)}; % these are the measures of a particular bar graph. The tick marks of the y-axis will be adjusted automatically according to the data values entered in the coordinates. 

\addplot[draw=none,fill=ncsuDarkRed,font=\footnotesize\sffamily] coordinates {(0.73,Valence) (0.50,Arousal) (0.61,Dominance)};  
% \addplot coordinates {(Valence,61) (Arousal,55) (Dominance, 59)};  

\addplot[draw=none,fill=ncsuGray,font=\footnotesize\sffamily] coordinates {(0.54,Valence) (0.46,Arousal) (0.50,Dominance)}; % these are the measures of a particular bar graph. The tick marks of the y-axis will be adjusted automatically according to the data values entered in the coordinates.  

\legend{Victim, Abuser, Third person}  

\end{axis}  
\end{tikzpicture}  
\end{document}