\documentclass{IEEEtran}%screen,manuscript,review
% \setcopyright{acmcopyright}
% \copyrightyear{2022}
% \acmYear{2022}
% \acmDOI{10.1145/1122445.1122456}

% \acmPrice{15.00}
% \acmISBN{978-1-4503-XXXX-X/18/06}

% \acmJournal{TOIT} 
% \acmVolume{20}
% \acmNumber{1}
% \acmArticle{55}
% \acmMonth{12}


% \usepackage{cite}
\usepackage{amsmath,amsfonts}
\usepackage{algorithmic}
\usepackage{graphicx}
\usepackage{textcomp}
% \usepackage{xcolor}

\usepackage{tabularx} 
% \usepackage{graphicx}
% \usepackage{textcomp}

\newcolumntype{P}[1]{>{\centering\arraybackslash}p{#1}}
\usepackage{latexsym}
% \usepackage{mathrsfs}
% \usepackage{soul}
\usepackage{float}

\usepackage{url}

\usepackage{xcolor}
\definecolor{mpsNavy}{rgb}{0.01,0.01,0.5}
\definecolor{mpsGray}{rgb}{0.85,0.84,0.84}
\definecolor{mpsMyrtleGreen}{rgb}{0.19,0.47,0.45}
\definecolor{huiGray}{rgb}{0.97,0.97,0.97}
\definecolor{chromeyellow}{rgb}{1.0, 0.65, 0.0}
\definecolor{VG}{rgb}{0.6,0.6,0.6}

\usepackage{algorithmic}
\usepackage[linesnumbered,ruled,vlined]{algorithm2e}
\usepackage{subfigure}

\usepackage{adjustbox}
\usepackage{tikz}
\usepackage{standalone}
\usetikzlibrary{arrows}
\usetikzlibrary{calc}
\usetikzlibrary{fit}%
\usetikzlibrary{matrix}
\usetikzlibrary{positioning}
\usetikzlibrary{shapes.geometric}
\usetikzlibrary{shapes.symbols}
\usetikzlibrary{shapes.misc}

\usepackage{diagbox}

%\usepackage{tikz}
%\usetikzlibrary{positioning,calc,backgrounds,arrows.meta,shapes,fit,decorations.pathmorphing}
%\usepackage{aurical}
\usepackage{pgfplots}
\pgfplotsset{compat=1.16}
\usepgfplotslibrary{statistics}

\usepackage{afterpage}
\usepackage{url} 
\usepackage{booktabs}
% \usepackage{enumitem}
\usepackage[inline]{enumitem}
\usepackage{centernot}
\usepackage{stmaryrd}
\usepackage{mathtools}
\usepackage{fancybox}
% \usepackage{multicol}
\usepackage{multirow}
\usepackage{tabularx}



\usepackage{natbib}
\usepackage[np]{numprint}
\npthousandsep{,}
\npdecimalsign{.}
\usepackage{balance}
\usepackage{pifont}
\newcommand{\starsOne}{\ding{72}\ding{73}\ding{73}\ding{73}\ding{73}}
\newcommand{\starsTwo}{\ding{72}\ding{72}\ding{73}\ding{73}\ding{73}}
\newcommand{\starsThree}{\ding{72}\ding{72}\ding{72}\ding{73}\ding{73}}
\newcommand{\starsFour}{\ding{72}\ding{72}\ding{72}\ding{72}\ding{73}}
\newcommand{\starsFive}{\ding{72}\ding{72}\ding{72}\ding{72}\ding{72}}

\newcommand{\mypara}[1]{\vspace*{0.75ex}\noindent\textbf{#1}~~}
 
% \definecolor{codegreen}{rgb}{0,0.6,0}
% \definecolor{codegray}{rgb}{0.5,0.5,0.5}
% \definecolor{codepurple}{rgb}{0.58,0,0.82}
% \definecolor{backcolour}{rgb}{0.95,0.95,0.92}
% \definecolor{colorEntityBack}{rgb}{0.01, 0.01, 0.4}
% \definecolor{colorPolicyBackDarkDark}{rgb}{0.81, 0.81, 0.86}
% \definecolor{ashgrey}{rgb}{0.7, 0.75, 0.71}
% \definecolor{beaublue}{rgb}{0.74, 0.83, 0.9}
% \definecolor{hotpink}{rgb}{1.0, 0.41, 0.71}
% \definecolor{darkgreen}{rgb}{0.09, 0.45, 0.27}
\definecolor{cerulean}{rgb}{0.105, 0.672, 0.836}
% \definecolor{candyapplered}{rgb}{1.0, 0.03, 0.0}

\usepackage{soul}
\usepackage{tabu}

\usepackage{xargs}
\newcommand{\hg}[1]{\textcolor{cerulean}{\textsf{HG:~~#1}}}
\newcommand{\mps}[1]{\textcolor{blue!50!black}{\textsf{}{MPS:~~#1}}}
\newcommand{\vg}[1]{\textcolor{red!50!black}{\textsf{}{VG:~~#1}}}
\newcommand{\nsa}[1]{\textcolor{green!50!black}{NSA:~~#1}}


\newcommand{\nsaremove}[1]{\textcolor{brown!50!red}{NSA r: \st{#1}}}
\newcommand{\nsaadd}[1]{\textcolor{brown!50!green}{NSA a:~~#1}}
\newcommand{\nsareplace}[2]{\textcolor{brown!50!red}{~~\st{#1}}\textcolor{green!40!black}{~~#2}}

\usepackage{xspace}
\newcommand{\etal}{et al.\xspace}

\usepackage{siunitx}

\newcommand{\todo}[1]{\hl{#1}} 
\newcommand{\ifsubmit}[1]{}
\newcommand{\ifnotsubmit}[1]{#1}
\newcommand{\optindent}{}


\newcommand{\frm}{\textrm}
\newcommand{\fbf}{\textbf}
\newcommand{\fit}{\textit}
\newcommand{\fsf}[1]{\normalsize\textsf{#1}}
\newcommand{\fsc}{\textsc}
\newcommand{\fsl}{\textsl}
\newcommand{\ftt}{\texttt}
\newcommand{\msf}{\mathsf}
\newcommand{\N}{\msf{N}}
\newcommand{\A}{\msf{A}}
\newcommand{\Pro}{\msf{P}}
\DeclareMathAlphabet{\mathsl}{OT1}{ptm}{m}{sl}
\newcommand{\msl}{\mathsl}

\newcommand{\fsub}{\textsubscript}
\newcommand{\fsup}{\textsuperscript}


\newcommand{\approach}{i\fsc{Rogue}\xspace}


\newcounter{textboxno}
\usepackage[many]{tcolorbox}
\newtcolorbox{textbox}[1][]{
tikznode boxed title,
enhanced,
float,
interior style={white},
boxsep=3pt,left=2pt,right=2pt,bottom=5pt,
width=\columnwidth,
boxrule=1pt,
attach boxed title to top center= {yshift=-\tcboxedtitleheight/2},
colbacktitle=white,coltitle=black,
boxed title style={size=normal,colframe=white,boxrule=0pt},
title={\refstepcounter{textboxno}\label{#1}
Example \arabic{textboxno}
\def\@currentlabel{\p@textboxno\thetextboxno}},
}

\newtcolorbox{mybox}[2][]{%
boxsep=3pt,left=2pt,right=2pt,bottom=5pt,
width=\columnwidth,
boxrule=1pt,
attach boxed title to top center = {yshift=-\tcboxedtitleheight/2},
colbacktitle=white,coltitle=black,
boxed title style={size=normal,colframe=white,boxrule=0pt}, 
interior style={white},
title={\refstepcounter{textboxno}\label{#1}
Example \arabic{textboxno}: {#2}
\def\@currentlabel{\p@textboxno\thetextboxno}},
enhanced,
float,
}





%%\acmSubmissionID{123-A56-BU3}

\begin{document}

\title{\approach: Identifying Rogue Behavior from App Reviews}


\author{Vaibhav Garg, 
Hui Guo, 
Nirav Ajmeri, 
Saikath Bhattarcharya, 
Munindar P. Singh \\
vgarg3@ncsu.edu,
hguo@quora.com,
nirav.ajmeri.bristol.ac.uk,
saikath.bhattacharya@gmail.com,
mpsingh@ncsu.edu}




%%
%% The "author" command and its associated commands are used to define
%% the authors and their affiliations.
%% Of note is the shared affiliation of the first two authors, and the
%% "authornote" and "authornotemark" commands
%% used to denote shared contribution to the research.
% \author{Ben Trovato}
% \authornote{Both authors contributed equally to this research.}
% \email{trovato@corporation.com}
% \orcid{1234-5678-9012}
% \author{G.K.M. Tobin}
% \authornotemark[1]
% \email{webmaster@marysville-ohio.com}
% \affiliation{%
%   \institution{Institute for Clarity in Documentation}
%   \streetaddress{P.O. Box 1212}
%   \city{Dublin}
%   \state{Ohio}
%   \country{USA}
%   \postcode{43017-6221}
% }

% \author{Lars Th{\o}rv{\"a}ld}
% \affiliation{%
%   \institution{The Th{\o}rv{\"a}ld Group}
%   \streetaddress{1 Th{\o}rv{\"a}ld Circle}
%   \city{Hekla}
%   \country{Iceland}}
% \email{larst@affiliation.org}

% \author{Valerie B\'eranger}
% \affiliation{%
%   \institution{Inria Paris-Rocquencourt}
%   \city{Rocquencourt}
%   \country{France}
% }

% \author{Aparna Patel}
% \affiliation{%
%  \institution{Rajiv Gandhi University}
%  \streetaddress{Rono-Hills}
%  \city{Doimukh}
%  \state{Arunachal Pradesh}
%  \country{India}}

% \author{Huifen Chan}
% \affiliation{%
%   \institution{Tsinghua University}
%   \streetaddress{30 Shuangqing Rd}
%   \city{Haidian Qu}
%   \state{Beijing Shi}
%   \country{China}}

% \author{Charles Palmer}
% \affiliation{%
%   \institution{Palmer Research Laboratories}
%   \streetaddress{8600 Datapoint Drive}
%   \city{San Antonio}
%   \state{Texas}
%   \country{USA}
%   \postcode{78229}}
% \email{cpalmer@prl.com}

% \author{John Smith}
% \affiliation{%
%   \institution{The Th{\o}rv{\"a}ld Group}
%   \streetaddress{1 Th{\o}rv{\"a}ld Circle}
%   \city{Hekla}
%   \country{Iceland}}
% \email{jsmith@affiliation.org}

% \author{Julius P. Kumquat}
% \affiliation{%
%   \institution{The Kumquat Consortium}
%   \city{New York}
%   \country{USA}}
% \email{jpkumquat@consortium.net}

%%


% \renewcommand{\shortauthors}{}



%% The code below is generated by the tool at http://dl.acm.org/ccs.cfm.
%% Please copy and paste the code instead of the example below.



\maketitle

\begin{abstract}
An app user can access information of other users or third parties.
We define rogue mobile apps as those that enable a user (abuser) to access information of another user or third party (victim), in a way that violates the victim's privacy expectations. Such apps are dual-use and their identification is nontrivial. We propose 
\approach, an approach for identifying rogue apps based on their reviews, posted by victims, abusers, and others.
\approach involves training on deep learning features extracted from their \np{1884} manually labeled reviews.
\approach first identifies how alarming a review is with respect to rogue behavior and, second, generates a rogue score for an app. \approach predicts \np{100} rogue apps from a seed dataset curated following a previous study. Also, \approach examines apps in other datasets of scraped reviews, and predicts an additional \np{139} rogue apps. On labeled ground truth, \approach achieves the highest recall, and outperforms baseline approaches that leverage app descriptions and reviews. A qualitative analysis of alarming reviews reveals rogue functionalities. App users, platforms, and developers should be aware of such apps and their functionalities and take measures to curb privacy risk.
\end{abstract}


\section{Introduction}
\label{sec:intro}
%%%%%%%%%%%%%%%%%%%%%%%%%%%%%%%%%%%%%%%%%%%%%%%%%

With the expansion of mobile technologies, privacy threats arise not only from malicious or careless app developers, but also from app users. The privacy expectations of an app user or a third party (i.e., a \emph{victim}) are violated when the victim (1) doesn't know about another user (i.e., an \emph{abuser}) accessing the victim's information (spying) or (2) may know about the access, but is uncomfortable with it. The latter case includes incidents of forced consent or when public information on apps (such as a dating platforms) is accessed beyond the victim's level of comfort, such as profile stalking. We use term \emph{rogue behavior} to mean these two types of information access, and use \emph{rogue apps} to mean the apps that enable rogue behavior. 



Research \citep{IPVspyware2018, Phonehacked2019, Clinicalsecurity2019} shows that rogue apps may cause discomfort, fear, and potential harm to the victim. Possible ways to prevent this risk include highlighting these apps and their rogue functionalities to users, warning app distribution platforms, and informing app developers. 
All these actions rely upon identifying rogue apps and their functionalities.
Our proposed approach, \approach, shows how to do so.


Previous studies \citep{IPVspyware2018,Creepware2020} focus only on the access performed without the victim's knowledge, but do not consider cases when the victim is uncomfortable of the access (of public information) even if aware of it. Chatterjee \etal \cite{IPVspyware2018} use app descriptions to identify intimate partner surveillance (IPS) apps (subset of rogue apps), the apps that someone can use to spy on his or her intimate partner (spouse, boyfriend, or girlfriend). Such IPS apps are dual-use apps, that have a legitimate purpose but are misused for spying.
This concept of dual-use apps also applies to the general setting of rogue apps. Since app descriptions of rogue apps indicate only intended legitimate behavior, app descriptions may not be suitable for identify all misuses. We leverage app reviews to identify all misuses of such apps. 
We observe that the reviews of an app describe rogue functionalities, its misuse (potential and actual), and the privacy expectations of users. Such reviews are evidence of rogue behavior and should be brought to the attention of users, developers, and app platforms.

Example~\ref{box:rogue-behavior-relevant} shows three reviews (edited for grammar), taken from the Apple's App Store \cite{Appstore}, and are relevant to the rogue behavior. Although our study is based on Apple App Store's reviews, \approach can be applied on reviews from other sources, including Google's Play Store \cite{Playstore}.

\begin{mybox}[box:rogue-behavior-relevant]{Cases Relevant to Rogue Behavior}

\noindent\textbf{Fly on the wall!}\\
(for the AirBeam Video Surveillance app \cite{AirBeam})\\
\fsl{``with this app, i can spy on my family without them knowing it! it's such an awesome app!''}\bigskip

\noindent\textbf{This app basically ruined my family to an extent}\\
(for the Life360 app \cite{Life360})\\
\fsl{``My mother made everyone in the family get this app. She freaks out when the app doesn't do its job because of random obstacles that mess with the location accuracy. \ul{Drains the battery and makes my parents paranoid to know where I am at all times. I don't even do any bad stuff, yet years of trust building are being swept away by the ability to spy on the children of a household.} If you're a parent I highly recommend you don't get this app because it is extremely uncomfortable to have and it makes parents trust their children less.''}\bigskip

\noindent\textbf{Honest}\\
(for the 3Fun: Threesome \& Swingers app \cite{Threefun})\\
\fsl{``\ldots A lot of the local people I’ve talked to
(Male half of a couple) have been guys who are saying they’re part of a couple, and \ul{in all reality are single guys just looking to collect pictures}. There is no way to report that that is why you are reporting them. It’s just a boilerplate report feature. I feel there should be a way for the 3Fun community to point out people for bad behavior like this.''}
\end{mybox}

In Example~\ref{box:rogue-behavior-relevant}, the first review for AirBeam Video \cite{AirBeam}, addresses the scenario where the app assists a user to access a victim's information without the victim's knowledge. AirBeam Video is a surveillance app to be installed on the abuser's device.
Hence, the victim may not be an app user but a third party. 
The second review, for Life360 \cite{Life360}, complains about the problem of inappropriate access of user's location by the user's mother.
Due to the unequal power dynamics between the victim (reviewer in this case) and the abuser (mother in this case), the victim is forced to install apps that violate privacy.
The third review, from 3Fun \cite{Threefun}, describes the story of improper access of profile pictures. 
Even though the profile pictures are public, the victim
is uncomfortable with the access. 
It is common for users to upload such information (pictures in this case) on an app. 
When doing so, they hold expectations of how other users would access it. 
Information access, as shown in these three, cases may lead to discomfort, fear, or potential harm \citep{Phonehacked2019,Clinicalsecurity2019}. Thus, despite such cases of information access being common, they should be brought in front of app developers and platforms. However, app descriptions don't reveal possibility of a user (victim) to be uncomfortable of such access.


To address victims' privacy expectations, we propose the following research questions:
\begin{description}[leftmargin=1em]
\item[RQ\fsub{identify}.] How can we identify rogue apps from reviews?
\item[RQ\fsub{functionality}.] How can we uncover rogue functionalities? 
\end{description}

Section~\ref{sec:approach}
To address \textbf{RQ\fsub{identify}},
we propose \approach, an approach that is trained on the deep learning features extracted from \np{1884} app reviews. \approach includes three phases (described in Section~\ref{sec:approach}). 
First, it assigns an \emph{alarmingness} score to each review. The alarmingness score is used to rate and rank each review according to the claims and severity of rogue behavior.
Second, \approach identifies rogue apps, based on a \emph{rogue score}, computed by aggregating the alarmingness scores of an app's reviews. The rogue score ranks each identified app, according to the rogue behavior reported in app reviews. Such ranking can be useful for app distribution platforms, such as Apple App Store \cite{Appstore} and Google Play Store \cite{Playstore}, to prioritize the scrutiny of identified apps.
Third, \approach identifies additional rogue apps, by examining apps in the other datasets. To evaluate the performance of \approach, we report its precision, recall, and F1 score in identifying rogue apps. 

To address \textbf{RQ\fsub{functionality}}, we leverage reviews with the top 10 alarmingness scores and manually analyze them to find their rogue functionalities. We further installed a few rogue apps on an iOS device to verify reported rogue functionalities and include our findings in Section~\ref{sec:rogue-capabilities}. We contacted Apple App Store \cite{Appstore}. We shared with them the list of identified rogue apps, along with reviews containing evidence against each app. They told us they will investigate the rogue apps and reach out to developers to rectify apps. 

\mypara{Contributions.}
Our work's novelty lies in leveraging app reviews to identify rogue apps and their rogue functionalities. We introduce assigning alarmingness scores to reviews and rogue scores to apps, based on the reported rogue behavior. We contribute to mobile app security by providing:
\begin{itemize}
\item \approach, an app reviews based approach for identifying rogue apps and their functionalities.
\item A ranked list of rogue apps along with their alarming reviews revealing rogue behavior.
\end{itemize}

\mypara{Organization.}
The rest of this paper is organized as follows. 
Section~\ref{sec:motivation} describes our preliminary investigation that shows that app reviews contain evidence of rogue behavior. 
Section~\ref{sec:approach} describes our proposed \approach approach to identify rogue apps, along with its evaluation. Section~\ref{sec:rogue-capabilities} shows the procedure to uncover rogue functionalities of rogue apps. Section~\ref{sec:background} lists related work on information access in mobile apps. Section~\ref{sec:conclusion} concludes this paper.
%%%%%%%%%%%%%%%%%%%%%%%%%%%%%%%%%%%%%%%%%%%%%%%%%
\section{App Reviews Reveal Rogue Behavior}
\label{sec:motivation}


We now describe rogue behavior reported in app reviews. 


\subsection{Seed Dataset}
\label{sec:IPSdataset}

Chatterjee \etal \cite{IPVspyware2018} identify \np{2707} iOS apps as potentially IPS. Out of these apps, they confirm \np{414} apps to be IPS, using semi-supervised pruning. 

When we collected our data, \np{724} of Chatterjee \etal's \np{2707} apps (including 125 IPS) were already removed from the Apple App Store \cite{Appstore}, meaning only \np{1983} were available.
Of these \np{1983} apps, \np{1687} received at least one review from 2008-07-10 to 2020-01-30, yielding \np{11.57} million reviews in all. Only \np{210} of these \np{1687} apps were on Chatterjee \etal's IPS list.
Table~\ref{tab:prelim} describes our \emph{seed dataset}, which comprises these \np{1687} apps and their \np{11.57} million reviews. 
\begin{table}[!htb]
\caption{Details of our seed dataset.}
\label{tab:prelim}
\centering
\begin{tabular}{@{~~}l@{~~}rrr@{~~}}
\toprule
% \rowcolor{LightCyan}
App Type & Apps & Apps w/ Reviews & Reviews \\
\midrule
    % \rowcolorlightgray!50!}
    Removed & \np{724} & --  & --\\
    
    IPS apps & \np{289} & \np{210} & \np{190584} \\
    
    % \rowcolor{lightgray!50!}
    Other apps & \np{1694} & \np{1477} & \np{11381377} \\
    \midrule
    Total & \np{2707} & \np{1687} & \np{11571961} \\
\bottomrule
\end{tabular}
\end{table}

\subsection{Investigating Reviews}
\label{sec:preliminvestigation}

Since the seed dataset contains \np{11.57} million reviews, it is impractical to manually check each review for rogue behavior. Hence, we sampled app reviews containing at least one keyword related to rogue behavior. To form a set of such keywords, we initialized a set
with the words: \fsl{spy}, \fsl{stalk}, and \fsl{stealth}. We queried WordNet \cite{Wordnet1995} for synonyms of these words. We performed the query operation until we didn't find any new word in the set. The resulting set contained keywords: \fsl{spy}, \fsl{stalk}, \fsl{stealth},  \fsl{descry}, \fsl{chaff}, and \fsl{haunt}. However, we didn't consider \fsl{chaff} and \fsl{haunt} to be relevant for describing rogue behavior in reviews. Also, \fsl{descry} is present in only two reviews, both of which are irrelevant for rogue behavior. To expand the set of keywords, we explored other corpora such as PyDictionary \cite{pydictionary} and Thesaurus \cite{Pythesaurus} but did not find synonyms that are widely used in app reviews. Moreover, keywords used in the previous study \cite{IPVspyware2018}, such as \fsl{track} and \fsl{control} bring many false positive reviews. For example, ``I like \ul{tracking} my distance when I walk with my dog.'' and ``\ldots you can also \ul{control} the audio of your mac through the app \ldots I can control music \ul{tracks} without having to touch the computer.'' are not relevant. Thus, our relevant set of keywords reverts to \fsl{spy}, \fsl{stalk}, and \fsl{stealth}. Table~\ref{tab:KeywordsOccurrence} shows the occurrence of each keyword, in reviews of the seed dataset. We refer to this set as \emph{our keywords}.

There are \np{5287} reviews containing at least one of our keywords. From these \np{5287} reviews, we randomly sampled \np{995} reviews for manual scrutiny. 
This sample involves \np{179} apps with between 1 and 237 reviews each. 

\begin{table}[!htb]
    \centering
    \caption{Occurrence of each keyword.
    }
    \label{tab:KeywordsOccurrence}
    \begin{tabular}{ln{9}{0}}
    \toprule
    {Keyword} & {Review Count} \\
    \midrule
    Spy & 2479\\ 
    Stalk & 2605\\
    Stealth & 218\\
     \midrule
    Total Unique &  5287\\
    \bottomrule
    \end{tabular}
\end{table}
The first author manually checked 995 reviews for rogue behavior. Out of 995 reviews, we found \np{402} reviews (of \np{83} apps in this sample) reporting rogue behavior. 
Our manual analysis categorize these 402 reviews along the dimensions of \fsl{story} and \fsl{reviewer}. Based on rogue story, we observe reviews of following two types:

\begin{description}[leftmargin=1em]

\item[\emph{Rogue Act}:] Reviews describing someone performing a rogue behavior. In such reviews, the reviewer is sure about the app's rogue functionality.

\item[\emph{Rogue Potential}:] Reviews express the possibility of rogue behavior. The reviewer may not be sure of rogue functionality, but identifies risks with the app. 

\end{description}

Example~\ref{box:roguestories} shows a review for each type of rogue story. 
\begin{mybox}[box:roguestories]{Types of Rogue Stories}


\noindent\textbf{Rogue Act}\\
 \fsl{``This is a really good app if u want to \ul{spy} on your spouse I found out my boyfriend was cheating on me great app I recommend this app''}
\bigskip

\noindent\textbf{Rogue Potential}\\
\fsl{``\ldots May work well to \ul{spy} on the kids by `accidentally' leaving iPhone in secret place.''}

\end{mybox}






We also found three types of reviewers writing rogue stories.
First, reviewers who are \fsl{victims}: they state their concerns and grievances, including frustration at the loss of privacy. Second, reviewers who are \fsl{abusers}: they admit to the rogue behavior and sometimes express their delight in it. Third, reviewers are \fsl{third persons}: they report on others misusing the app or the potential to misuse. Example~\ref{box:reviewer-identity} shows a review for each type of reviewer.

\begin{mybox}[box:reviewer-identity]{Categories Based on Reviewer}


\noindent\textbf{Victim}\\
\fsl{``I hate this app so much! My mother is always questioning me and if I delete it she will ground me \ldots No one want their parents to \ul{stalk} them!!''}
\bigskip

\noindent\textbf{Abuser}\\
\fsl{``I can \ul{spy} on my child whenever i want its amazing he cant go anywhere without me knowing look.''}
\bigskip

\noindent\textbf{Third Person}\\
\fsl{``\ldots I don't feel like parents should track their kids AT ALL. everyone needs a little something called trust and if you don't have it then your kids will act out and have to become sneaky. This app is designed to track families and see everything just like the parent is with you at all times. I do have this app but only with my fiends and we don't \ul{stalk} each-other we just use it to see where everyone's at. And Bc we are so close and we all wanted it we all got it.''}

\end{mybox}

Table~\ref{tab:storyreviewer} shows the count of stories for each type of reviewer. The third person writes most of the potential reviews (44 out of 47) because such cases are only possibilities and not acts, meaning the reviewer is neither a victim nor an abuser. Whereas, abusers write other three potential cases. Such reviews (by abusers) indicate possible threats with the reviewed app, but suggest other apps for better rogue functionalities. For rogue act reviews, we found abusers (219) and victims (120) writing a majority of stories, followed by third person (16).


\begin{table}[!htb]
\centering
\caption{Count of stories for each reviewer type.}
\label{tab:storyreviewer}
\begin{tabular}{lrr}
\toprule
Reviewer & Rogue Act & Rogue Potential \\\midrule
Victim& 120 & 0 \\
Abuser & 219 & 3 \\
Third Person & 16 & 44 \\
\bottomrule
\end{tabular}
\end{table}


To sum up, reviews describe apps' rogue behavior and show how victims such as children, parents, and friends are abused.
 
%%%%%%%%%%%%%%%%%%%%%%%%%%%%%%%%%%%%%%%%%%%%%%%%%
\section{The \approach Approach}
\label{sec:approach}

\approach consists of three phases. 
First, \approach predicts the alarmingness score of each app review (Section~\ref{sec:alarming}).
Second, \approach generates a rogue score for each app based on the alarmingness of its reviews. 
We selected a threshold on rogue score, above which apps are predicted as rogue (Section~\ref{sec:rogue-score}). 
Third, \approach finds additional rogue apps by examining apps in other datasets of scraped reviews (Section~\ref{sec:candidateapps}).
Figure~\ref{fig:flowdiagram} shows an overview of the \approach approach. We envision \approach to be incrementally updated by adding alarming reviews, of newly found rogue apps. Moreover, apps that don't have reviews yet, will be identified rogue, when their new reviews arrive.  

\begin{figure}[!htb]
    \noindent\includegraphics[width=0.9\columnwidth]{figs/RogueOverviewNew.pdf}
    \caption{Overview of \approach approach.}
    \label{fig:flowdiagram}
\end{figure}  


\subsection{Computing Alarmingness of Reviews}
\label{sec:alarming}
Section~\ref{sec:motivation} shows that app reviews reveal evidence of rogue behavior. However, identifying evidence in reviews is nontrivial, especially when an app receives a large number of reviews. Instead of binary classification of reviews, we introduce the alarmingness score that not only identifies relevant reviews but also ranks them based on the rogue behavior.
To assess alarmingness of a review, we consider two factors: (i) the review's \emph{convincingness} about rogue behavior and (ii) the
\emph{severity} of the reported rogue behavior. The alarmingness score of a review is the geometric mean of its convincingness and severity scores.

Reviews can vary in their claims about the rogue behavior. Some reviews report detailed rogue behavior, whereas some others are merely suspicion. The convincingness score measures how convincing the app review is in describing the rogue behavior. In Example~\ref{box:convincingexample}, the first review is unrelated to rogue behavior and hence is not convincing. The second review describes the reviewer's suspicion on the app, which may or may not be true (slightly convincing). The third review (by an abuser) confirms the rogue behavior but lacks details of the rogue functionalities or victims. On the contrary, other reviews (in Example~\ref{box:convincingexample}) are extremely convincing because they confirm rogue behavior along with mentioning the location feature, or how to set up devices, or the victims being stalked. Extremely convincing reviews include cases when the app is used for positive purposes (tracking family members or pets for safety) but has the potential to be misused in future. The reviews that are slightly, moderately, or extremely convincing are relevant to identifying rogue behavior. Assigning a convincingness score helps in ranking these reviews according to the strength of their claims.



% Sometimes, app reviews just express suspicion (about rogue behavior), which may not be true in real app usage. Thus, we analyze how convincing the app review is, about rogue nature of the app. To access convincingness, we consider the reviewer's certainty about claim (i.e., firm claim or just suspicion). Moreover, previous works on convincingness (also called persuasion in other setting) discuss the role of detailed text for it to be convincing \cite{Opinionholder2020,Winningarguments2016}. Thus, we analyze both reviewer's certainty and details of rogue behavior (mentioned in review), to measure convincingness of reviews. In other words, convincingness of a review depends on its details about rogue behavior and the reviewer's certainty while making claims about that behavior.  Example~\ref{box:convincingexample} shows reviews with contrasting convincingness. 
% The first review includes details of rogue behavior, making it more convincing than the second, which lacks such details. 
% The third review in Example~\ref{box:convincingexample} describes the author's suspicion, which may not be true, so it is the least convincing of the three. All three cases are relevant to rogue behavior, but differ in their convincingness.


The severity score measures the effect of rogue behavior on the victim. Example~\ref{box:severityexamples} shows range of reviews varying in severity. The first review is unrelated to rogue behavior. Thus, it is not severe. The second review shows that the rogue act is performed with consent, making this review a slightly severe case. The third review is written by the abuser and lacks the victim's perspective to analyze rogue effect. We assume such acts are performed without consent and consider them moderately severe. The fourth review describes the victim's misery. The victim even says ``This app has truthfully ruined my teenage years'' in the review, which gives solid evidence to be an extremely severe case. Moreover, in the fifth review, the victim complains that others can see when he was last active (also known as last seen information). This is the public information on each profile, but still the victim is uncomfortable with the access. App developers should be aware of such users' privacy expectations. However, such cases are still missed by the existing studies \citep{IPVspyware2018,Creepware2020}. Since app reviews discuss privacy expectations, we are able to identify such cases and rate them extremely severe.


% To decide whether an app enables rogue behavior, we need to rely on reviews that are both sufficiently convincing and reveal severe rogue behavior. 
% Thus, it is important to compute the convincingness and severity scores for each review of an app and discard low scoring reviews. 


\begin{mybox}[box:convincingexample]{Varying Degree of Convincingness}
% \label{box:convincingexample}
\noindent\textbf{1: Not Convincing}\\
\fsl{``It is such a great game, love it so much!''}\bigskip

\noindent\textbf{2: Slightly Convincing}\\
\fsl{``Setup was a breeze. Quicktime 7 pro found it easily. Unfortunately, resolution seems much, much, lower than hoped.  Video size can not be adjusted live. Hate to be a hater. \ul{May work well to spy on the kids by `accidentally' leaving iPhone in secret place}.''}\bigskip


\noindent\textbf{3: Moderately Convincing}\\
\fsl{``This app is perfect for stalking people\ldots''}\bigskip

\noindent\textbf{4: Extremely Convincing}\\
\fsl{``\ul{This app is awesome for our family to keep track of where everyone is at all times!} (You can turn the location off too in case you want to be in stealth mode when buying Christmas presents too.) \ldots Even our dog knows that the alert sound when a family member arrives home means \ldots''}\bigskip

\fsl{``\ldots \ul{I use it to spy on my dogs while I'm at work}; so I use it for fun, nothing fancy. My iPad is my camera, and my iPhone is my viewer. \ldots''}\bigskip


\fsl{``\ul{bro this app is high key creepy. when i'm with my dad on his days my mom even mentions how she knew everything} i was doing and it even made my dad creeped out. if you need this app then ngl yo wack. i don't want my mom stalking me.''}

\end{mybox}

\begin{mybox}[box:severityexamples]{Varying Degree of Severity}
% \begin{center}\textbf{\ul{Reviews with Varying Degree of Severity}}\end{center}

\noindent\textbf{1: Not Severe}\\
\fsl{``Love the graphics so far it is a great game''}\bigskip


\noindent\textbf{2: Slightly Severe}\\
\fsl{``I love this app, just great because you can time your day accordingly, I like my girlfriend knowing where I am and I love stalking her, we have fun with it.\ldots''}\bigskip

\noindent\textbf{3: Moderately Severe}\\
\fsl{``This app is perfect for stalking people\ldots''}\bigskip

\noindent\textbf{4: Extremely Severe}\\
\fsl{``honestly if you want your kid to rebel against you even more, this is the app for you! This app has truthfully ruined my teenage years all because my mother now has a way of tracking me down 24/7. I couldn't do the normal teenage things because I was being stalked all day \ldots''}\bigskip

\fsl{``\ldots i want to share my last seen just to my family and my girlfriend not others. please add new feature in privacy that i can share my last seen to no body except my family and girl friend 
thanks soo much !''}
\end{mybox}
  

We first rate the convincingness and severity of rogue behavior reported in the app reviews (Section~\ref{sec:crowdsourcing}). 
We then extract deep learning features from the reviews (Section~\ref{sec:sentenceencoder}). 
Leveraging the annotated set and extracted features, we evaluate various regression models and choose the best one (Section~\ref{sec:regressionmodel}).
Finally, we calculate the alarmingness score of an app review as the geometric mean of its predicted convincingness and severity scores. 
% We now describe our process of obtaining a training set for the regression models via crowdsourcing. 


\subsubsection{Review Annotation}
\label{sec:crowdsourcing}
We selected \np{1884} reviews from the seed dataset: \np{952} (set s\fsub{1}) that contain at least one of our keywords and \np{932} reviews (set s\fsub{2}) that do not contain any of those keywords. While preparing this annotation data, we exclude the reviewers' identifiers such as their usernames. Each selected review is rated for convincingness and severity on a four-point Likert scale (1: not, 2: slightly, 3: moderately, 4: extremely).   
For quality of annotations, we measure Inter Rater Reliability (IRR) via Intraclass Correlation Coefficient (ICC) \cite{Hallgren-TQMP12-IRR}. ICC is suitable for Likert scale ratings. Unlike other IRR measures such as Cohen's \cite{Cohen-88-Statistics} kappa, which are based on (all or nothing) agreement, ICC takes into account the magnitude of agreement (or disagreement) to compute IRR. 



The annotation was conducted in three steps. First, two authors of this paper rated \np{599} of \np{1884} selected reviews according to the initial set of annotation instructions. The initial instructions included definitions (of convincingness and severity scores) and examples corresponding to each point on the likert scale.
In this step, for each annotator, we calculated the alarmingness scores of reviews using the convincingness and severity ratings. If the alarmingness scores computed for both the annotators were not at least three (median value on Likert scale), or both are not less than three, annotators resolved such cases via discussions. After discussing, the annotators produced the final set of annotation instructions. In the second step, the annotators followed the final instructions and rated 900 reviews. In this step of the annotation process, we achieved ICC of 0.9195 for convincingness and 0.9190 for severity. An ICC score in the range 0.75--1 indicates excellent agreement \cite{Hallgren-TQMP12-IRR}. In the third step, the remaining reviews were divided among the two annotators so that only one annotator rates each review.



For reviews that were rated by the two annotators, we computed the average convincingness and severity scores. This annotation study possessed minimal risk and was approved by the Institutional Review Board (IRB) of our university. 



\subsubsection{Extracting Deep Learning Features from App Review}
\label{sec:sentenceencoder}

We obtained the feature vector of each app review as follows:

\begin{description}[leftmargin=1em]

\item[Combine Sentences:] Remove periods in each app review and combine all its sentences to form single sentence.

\item[Text Preprocessing:] Remove all punctuation marks, stop words \cite{Textpreprocessing2014}, and our keywords, the latter because those keywords may correlate with reviews with higher scores and could create bias in the model.

\item[Sentence Embedding:] We leverage the Universal Sentence Encoder (USE) \cite{UniversalSentenceEncoder2018} to extract embeddings for each app review.
USE uses Deep Averaging Network (DAN) to
provide a 512-dimension embedding for a long text. USE is trained on a large variety of natural language tasks with the aim of capturing the context. In our case, 
USE directly provides sentence level embeddings of an app review, by keeping the context intact. However, alternatives such as GLoVe \cite{Glove2014} and Word2Vec \cite{Word2Vec} lose such context. We leverage the pretrained USE network by using Tensorflow Hub \cite{Tensorflowhub}.
% Hence, we leverage USE instead of word embeddings.

\end{description}

\subsubsection{Training Regression Model}
\label{sec:regressionmodel}

We treat score prediction as a multi-target regression problem \cite{Mutioutputregression2015}. 
Here, the \np{1884} annotated reviews form the training set, and convincingness and severity are target variables, predicted using the extracted deep learning features.

We evaluated the performance of three regression models: support vector regression \cite{Supportvector2007}, random forest \cite{Randomforest2013}, and decision tree \cite{Decisiontree2005}, by ten-fold cross validation on our dataset. To mitigate bias of our keywords, we remove such keywords in the preprocessing step, so that the regression model learns from the the context of the review and not from specific keywords.
Table~\ref{tab:perfmodel} shows average and standard deviation of mean squared error (MSE) \cite{Meansquared2010} in ten folds. The reported MSE is the combined MSE for two targets. The Support Vector Regressor (SVR) yields the smallest MSE, so we choose it for the subsequent phases of our approach.


\begin{table}[!htb]
\caption{Performance of three regression models on ten-fold cross validation.}
\label{tab:perfmodel}
\centering
\begin{tabular}{lrr}
\toprule
Regression Model & Average MSE & Standard Deviation \\
\midrule
Decision Tree  & 1.344 & 0.402\\
Random Forest & 0.712& 0.417\\
Support Vector & \textbf{0.625}&0.458\\
\bottomrule
\end{tabular}
\end{table}


We use this trained model to predict convincingness and severity scores of all \np{11.57} million reviews in the seed dataset.
The alarmingness of each review is calculated by taking the geometric mean of its predicted convincingness and severity. We use geometric mean because it ensures high alarmingness value only if both the convincingness and severity scores are high.


\subsection{Identifying Rogue Apps}
\label{sec:rogue-score}

We produce an app's rogue score by aggregating the alarmingness scores of its reviews as follows.

\begin{description}[leftmargin=1em]
\item[Weighted Mean of Alarmingness:] In general, for a rogue app, a small proportion of reviews report rogue behavior. Thus, we need to catch rogue apps using their few reviews that have high values on the alarmingness scale. Thus, we assign weights to reviews based on their alarmigness, as follows:

\emph{Defining score buckets:} While annotating reviews, we defined levels of convincing reviews (not convincing to extremely convincing) and severe reviews (not severe to extremely severe) on a Likert scale. We also follow same levels on the alarmingness scale (1: not alarming to 4: extremely alarming). We define a score bucket between every consecutive level of alarmingness (not alarming to slightly, slightly alarming to moderately alarming, moderately alarming to extremely alarming). Table~\ref{tab:scorebuckets1} shows how score buckets are formed using levels of alarmingness. 


\emph{Assigning weights to score buckets:}
We have 11.57 million reviews in the seed dataset. Based on the alarmingness computation (Section~\ref{sec:alarming}), we calculated the probability of a review falling in a score bucket. Since, the reviews reporting rogue behavior are less, probabilities in buckets~2 and 3 are less than that in bucket~1. We take inverse of these probabilities to get the weights for each score bucket. As a result, we assign higher weights to buckets~2 and 3 than to the bucket~1. Table~\ref{tab:scorebuckets1} also shows the weight assigned to each score bucket. 

\begin{table}[!htb]
\caption{Score buckets for alarmingness.}
\label{tab:scorebuckets1}
\centering
% \begin{tabular}{P{1cm}P{2cm}P{2cm}P{2cm}}
% \begin{tabular}{@{}P{2cm}@{~}P{2cm}@{~}P{0.8cm}@{~}S@{}}
\begin{tabular}{P{2cm}P{3cm}P{0.8cm} n{1}{3}}
\toprule
Alarmingness Score Range & Alarmingness Level Range & Bucket & {Bucket Weight}  \\
\midrule

$[1,2)$& Not alarming to Slightly & 1 & 2.29e-3 \\
$[2,3)$ & Slightly to Moderately & 2 & 6.08e-2\\
$[3,4]$ & Moderately to Extremely & 3 & 9.36e-1\\
\bottomrule
\end{tabular}
\end{table}

% To compute the weighted mean of alarmingness, we assign weight to each review, according to the the score bucket it falls in. For example, a review `r' with alarmingness score between three and four, falls in bucket 3, and has 0.00666 as weight (Table~\ref{tab:scorebuckets1}).

If $a\fsub{1}$, $a\fsub{2}$, \ldots, $a\fsub{n}$ are the alarmingness scores of an app's reviews, and $w\fsub{1}$, $w\fsub{2}$, \ldots, $w\fsub{n}$ are their respective weights (according to Table~\ref{tab:scorebuckets1}), then, $W\fsub{alarmingness}$, the weighted mean of alarmingness is given by:
\begin{equation*}
W\fsub{alarmingness}=\frac{a\fsub{1}*w\fsub{1}+ a\fsub{2}*w\fsub{2}+ \ldots a\fsub{n}*w\fsub{n}}{w\fsub{1}+ w\fsub{2}+ \ldots w\fsub{n}}    
\end{equation*}




The weighted mean of alarmingess ranges from 1 to 4. 


\item[Normalized Count:] The weighted mean of alarmingness does not account for the count of reviews that report rogue behavior against an app. Suppose, \emph{app A} has 15 reviews reporting rogue behavior and \emph{app B} has 25 reviews reporting rogue behavior. If all reviews reporting rogue behavior have the same alarmingness score, the weighted means of the two apps would be the same. But, \emph{app B} shows more evidence of rogue behavior and should have a higher rogue score than \emph{app A}. Thus, we also consider the count of reviews. For each app, we calculate the number of reviews in bucket~3. We tried incorporating counts of other buckets, but it led to worse performance of the approach. 

The minimum possible value of the count is zero. However, in some cases, counts can be high, leading to no definite upper limit. Thus, we normalize the counts of all the apps between one and four.
\end{description}

We want to assign high rogue score to apps that have high scores in both (i) weighted mean of alarmingness and (ii) normalized count. Thus, rogue score is computed as the geometric mean of these two values. 

			
\subsubsection{Selecting Threshold for Prediction}
\label{sec:verification}

For each app in the seed dataset, \approach computes the rogue score. All apps are ranked in decreasing order of rogue scores. The apps with a score greater than a threshold are predicted rogue. To decide the correct threshold, we follow two steps: (i) label the ground truth of rogue apps and (ii) vary a threshold between certain values and choose the threshold which gives the best performance of \approach.


We create our ground truth by manually scrutinizing reviews. However, scrutinizing reviews of all \np{1687} apps (seed dataset) is not feasible. Thus, first, we scrutinize the \np{50} most alarming reviews (with minimum score of two---at least slightly alarming) for apps with the highest \np{100} rogue scores. Second, we scrutinize reviews containing our keywords for these \np{100} apps. The first step is aligned with our approach since it checks top alarming reviews. However, the second step is neutral because it searches for evidence for the apps that \approach failed to identify through top alarming reviews. This way we mitigate the threat of bias, while curating ground truth for apps with the highest 100 rogue scores.


We label an app as rogue provided any of the scrutinized reviews report a rogue behavior. Table~\ref{tab:instruction} shows the types of reviews we consider indicative or otherwise of a rogue evidence. If the reviews of an app describe information access performed without the victim's knowledge or when the victim shows discomfort (first three reviews in Table~\ref{tab:instruction}), we consider the app as rogue. Also, some rogue apps are used for positive purposes such tracking family members for safety (fourth review in Table~\ref{tab:instruction}) but still possess a potential for future misuse. Similarly, apps used for tracking pets or other objects are considered rogue (fifth review). Reviews of some apps don't possess any misuse in present or in future, leading to their final label of not rogue (sixth review). Through manual inspection, we determine that of the \np{100} apps, \np{73} are rogue and \np{27} are not.





\begin{table*}[!htb]
    \centering
    \caption{Instructions followed for labeling rogue apps.}
    \label{tab:instruction}    
    \begin{tabular}{p{3cm}p{3.3cm}p{5.0cm}c}
    \toprule
        Type of case & Subtype of case & Example & \parbox{1.5cm}{Evidence of rogue behavior}  \\
        \midrule
          Tracking people's information& Without the victim's knowledge & \emph{``Now that I can spy on my wife I will always know when she is cheating''}  & Yes
          \\
        
         Tracking people's information & With the victim's knowledge but with discomfort& \emph{``Ok my mom got this for me and \ldots it's kinda creepy that this app was made so parent could basically stalk their kids.''} & Yes \\
         Tracking people's information & Public information but the victim is uncomfortable &   \emph{``I had someone cyberstalking and harassing me. Multiple attempts in every way shape and form were made to contact app-name to block and ban the stalker's account due to a concern for my well- being.''} & Yes \\
% \rowcolor{lightgray!50!}
% \multirow{-4}{*}[3em]{
%             \parbox{3cm}{People's information}
%             }
          Tracking people's information& Positive purpose & \emph{``I love finding my family members. Wife was in bad car wreak and I was able to find her location using this app. Thank you!''} & Yes\\
        Tracking pets or other objects& & \emph{``Wow! Day one and I'm stalking my puppet like a soccer mom that ran out of adderall! I'm very excited to use this to interact with my puppet while I'm at work and to check in on the dog walker!''} & Yes \\

            
         Not related to information accessing & & \emph{``an absolutely amazing and very helpful app.  i don't know how i would keep track of prayer times without it.  love the app.  thank u!!!''} & No \\
    \bottomrule
    \end{tabular}
\end{table*}


For the 100 apps, the rogue score varies between 1.74 and 3.60. We vary the threshold from 1.73 to 3.59 in steps of 0.01. Apps with a rogue score above the threshold are predicted rogue but not rogue otherwise. At each value of threshold, we report the recall, precision, and F1 score.

Table~\ref{tab:thresholding} shows the performance achieved at specific thresholds. As we increase the threshold, the precision increases at the cost of recall. For rogue apps, a false negative costs more than a false positive because a false negative leaves a rogue app undetected, which can harm many victims, whereas a false positive causes only wasted effort in manual scrutiny. Hence, achieving high recall is more important than achieving high precision. Thus, from Table~\ref{tab:thresholding}, we choose \np{1.73} threshold that gives the best recall of \np{100}\% at \np{73}\% precision. Since we fine-tune the threshold on the same seed dataset, we also check \approach's performance (using the chosen threshold) on the other dataset (Section~\ref{sec:candidateapps}).  

\begin{table}[!htb]
\caption{Choosing an appropriate threshold according to the recall scores.}
\label{tab:thresholding}
\centering
% \begin{tabular}{P{1cm}P{2cm}P{2cm}P{2cm}}
% \begin{tabular}{@{}P{2cm}@{~}P{2cm}@{~}P{0.8cm}@{~}S@{}}
\begin{tabular}{rrrr}
\toprule
Threshold & Precision (\%) & Recall (\%)& F1 Score (\%)\\
\midrule
%
\textbf{1.73}&	\textbf{73.00}&	\textbf{100.00}&	\textbf{84.39}\\
1.74&	73.40&	94.52&	82.63\\
1.75&	76.13&	91.78&	83.22\\
1.76&	76.82&	86.30&	81.29\\
1.77&	76.92&	82.19&	79.47\\
1.78&	78.37&	79.45&	78.91\\
1.79&	79.71&	75.34&	77.46\\
1.80&	80.95&	69.86&	75.00\\
1.81&	80.95&	69.86&	75.00\\
1.82&	81.96&	68.49&	74.62\\
1.83&	81.03&	64.38&	71.75\\
1.84&	80.35&	61.64&	69.76\\
1.85&	82.69&	58.90&	68.80\\
1.86&	83.33&	54.79&	66.11\\
1.87&	82.97&	53.42&	65.00\\
\bottomrule
\end{tabular}
\end{table}


% Considering, 1.73 as the final threshold, \approach predicts \np{88} apps as rogue with 91.78\% recall at 83.22\% F1 score. Example~\ref{box:seed-rogue-behavior1} and Example~\ref{box:seed-rogue-behavior2} show alarming reviews of Find My Family \& Friends App \cite{Life360} and OurPact Jr. Child App \cite{OurPact}, which \approach correctly identifies as rogue. For Find My Family \& Friends App, alarming reviews report parents misusing the tracking functionality on children, to which children are uncomfortable. Whereas, alarming reviews of the OurPact Jr. Child App report that parents can install this app on the child's device to monitor their texts and visited websites. In Section~\ref{sec:rogue-capabilities}, we will be discussing about these rogue functionalities in detail.

% \begin{mybox}[box:dualuseexample2]{Rogue App from Seed Dataset}
% % \begin{center}\textbf{\ul{App with Evidence of Rogue Behavior }}\end{center}
% \textbf{App: Kik \cite{Kik}} \\
% \textbf{Description:} 
% \fsl{``Get connected. Kik is way more than just messaging. It's the easiest way to connect with your friends, stay in the loop, and explore – all through chat. No phone numbers, just pick a username. Choose who to chat with one-on-one and in groups. Share pics, videos, gifs, games, and more. Meet new friends with similar interests. Get on Kik now. Start chatting!''}
% \bigskip

% \textbf{Evidence of Rogue Behavior in App Reviews} \\
% Date of Review: 2015-06-24\\
% \fsl{``MAKE THIS SAFE PEOPLE WANT TO USE IT BUT DONT WANT TO BE STALKED OR ABUSED IN THE APP. PROTECT THE APP OR PEOPLE WONT USE IT \ldots''}
% \end{mybox}

\begin{mybox}[box:seed-rogue-behavior1]{Rogue App from Seed Dataset}
% \begin{center}\textbf{\ul{Rogue App from Seed Dataset}}\end{center}

\textbf{App: Find My Family \& Friends \cite{Life360}} \\ 
\textbf{Rogue Score: 3.60/4.00}\\

\textbf{Alarming Review 1} \\
\textbf{Alarmingness: 4.00/4.00}\\
Date of Review: 2019-11-28\\
\fsl{``\ldots Such a terrible thing for unaware parents to use. \ul{Most parents think teens don't need privacy and they constantly need to know where they are and what they're doing and who they're with at all times.} This may make the parent feel at peace but what about the child? It's selfish of parents to not take into consideration of how the teen may feel about always having this app and the parent giving them a very stalkish feeling, it's very uncomfortable.''}\bigskip

% \textbf{Alarming Review 2} \\
% \textbf{Alarmingness: 3.92/4.00} \\
% Date of Review: 2019-11-18\\
% \fsl{``This app has ruined so many families trust wise and so many relationships trust wise as well. My personal opinion is if you have to put an app on your child or your significant others cell phone there is a huge trust issue and it will create so many unnecessary issues. If it was up to me this app should be taken down. This is ridiculously `stalkerish', if you will, and I think it helps nothing. If you are a mother or father and you put this app on your child's phone thinking that it will save them from being in harm because you know where they are, you are just about as ignorant as the makers of this app. your child could be at a friends house, or so you think, they leave their phone there and go off and do something else that they think you will not approve of and then if they get into a situation where they are in trouble and need help and they don't have a phone to get in touch with someone to help, then that is a situation that you have created and it is so much worse. Please think, and really think before putting this app on your child phones.''}

% \bigskip
% \textbf{Alarming Review 3} \\
% \textbf{Alarmingness: 4.332/5}

% \fsl{``This app only goes to show how little trust parents put in their children. Being able to see their every move and read their messages among other things is a complete invasion of privacy. Children feel trapped and untrusted and it creates an unhealthy relationship between a child and their parent. Overall a horrible idea for an app.''}
\end{mybox}

\begin{mybox}[box:seed-rogue-behavior2]{Rogue App from Seed Dataset}
% \begin{center}\textbf{\ul{Apps with No Rogue Behavior}}\end{center}

\textbf{App: OurPact Jr. Child App \cite{OurPact}} \\
\textbf{Rogue Score: 2.47/4}\\

\textbf{Alarming Review 1} \\
\textbf{Alarmingness: 4.00/4.00}\\
Date of Review: 2018-06-19\\
\fsl{``\ldots \ul{however this app shuts down almost everything and can see every text and website you've visited}. now, i haven't done anything bad online (recently), but \ul{i find that a little creepy and honestly an invasion of privacy. no wonder this app has such crappy reviews}. also, i used to have way more apps than i do now. because my parents now have the ability to restrict apps that may be ``inappropriate''. i already have to ask permission to download apps, so if they were inappropriate my parents wouldn't let me download them. there's too many apps like this and i think kids need a break from all this crap on their devices.''}


\bigskip

\textbf{Alarming Review 2} \\
\textbf{Alarmingness: 4.00/4.00} \\
Date of Review: 2018-08-16\\
\fsl{``This is a useless app that no parent need to install I pray for every child who has this app installed on their electronics some parents don't understand the modern society but that's okay (but not really) I'm only given 2 hours and writing this review is using up time WHICH IS NOT FRIKEN OK!!! \ul{I hate this hate this app and I hope every child that has had their device attacked by this installment hates this app as much as me.} This app should never be okay to use its inappropriate and everybody's children who have this app installed are making there children ANTI-SOCIAL AND VERY NOT COOL. I have many reasons why this app is SOOOOOOO scaring and dreadful so if your reading and thinking about installing this on ur child's device DONT INSTALL IT because that will ruin their future.''}

\end{mybox}

\subsubsection{Performance of Baseline Methods}

On the seed dataset, we also check the performance of baseline methods described below.
\begin{description}[leftmargin=1em]
\item[Our Keywords on App Description.] We search for the presence of one of our keywords (\fsl{spy}, \fsl{stalk}, and \fsl{stealth}) in app descriptions. Apps whose descriptions contain any of these keywords are predicted rogue, whereas other apps are predicted as not rogue.

\item[Extended Keywords on App Description.] We identify additional relevant keywords by extracting verbs through Part-Of-Speech (POS) tagging \cite{POS2011}. POS tagging marks every word in a sentence to an appropriate part of speech (verb, noun, adjective, and so on). Applying this process on descriptions of 73 rogue apps (from the ground truth) produced 145 verbs, out of which six (\fsl{track}, \fsl{monitor}, \fsl{locate}, \fsl{control}, \fsl{stolen}, \fsl{lost}) we selected as relevant to rogue behavior. The verbs: \fsl{stolen} and \fsl{lost} are relevant because they describe the apps that are used to find a misplaced phone, which indicates an ability to track another device. We extend our keywords by adding these six verbs. Apps whose description contain these keywords are predicted rogue.

% \item[Reading App Description.] Keyword based filtering may not work effectively in identifying rogue apps. Thus, we manually inspect descriptions of all 100 apps, to see if they indicate any rogue functionality such as tracking, monitoring web history, and so on. If they do, they are predicted rogue. Otherwise, they are predicted not rogue.

\item[T\% Keyword Reviews.] For each app, we compute the percentage of reviews containing our keywords. We set a threshold, \fsl{T}, on this percentage, above which apps are predicted rogue. In our evaluation, \fsl{T} takes the values of 0.3, 0.2, and 0.1, respectively.  

\end{description}

Table~\ref{tab:performancecomparison} summarizes the precision, recall, and F1 scores of all baselines and our approach. Our keywords on the description predict only one rogue app, leading to 100\% precision (highest among all). However, our keywords miss 72 rogue apps, which leads to the worst recall of 1.36\%. Among all the baselines, keyword search on reviews with 0.1\% threshold achieves the highest recall of \np{65.07}\%, which is much lower than \approach's recall value. \approach's better performance may be due to fine tuning \approach's threshold on the same seed dataset. Thus, we also compare \approach's performance with these baselines on the other dataset (Section~\ref{sec:candidateapps}).


\begin{table}[!htb]
\caption{Performance (in \%) of baseline methods and \approach on the seed dataset. Bold value for a metric indicates the highest score among all approaches.}
\label{tab:performancecomparison}
\centering
\begin{tabular}{p{5.25cm}ccc}
\toprule
Method & Recall & Precision & F1\\
\midrule
Our keywords on app descriptions & 01.36& \textbf{100.00} & 02.68 \\
Extended keywords on app descriptions & 61.64& 80.35& 69.76 \\
% Reading App Description& 73.97 & 87.09 & 80.00 \\
\midrule
0.3\% keyword reviews & 46.03& 96.66& 62.36 \\
0.2\% keyword reviews& 50.79& 96.96 & 66.66 \\
0.1\% keyword reviews &65.07 & 95.34& 77.35 \\

\midrule
\approach & \textbf{100.00} & 73.00& \textbf{84.39} \\
\bottomrule
\end{tabular}
\end{table}


Examples~\ref{box:seed-rogue-behavior1} and~\ref{box:seed-rogue-behavior2} show alarming reviews of Find My Family \& Friends App \cite{Life360} and OurPact Jr. Child App \cite{OurPact}, which \approach correctly identifies as rogue. Both of them are dual-use apps. Find My Family \& Friends is a safety app, but alarming reviews report parents misusing the tracking functionality on children, to which children are uncomfortable. Moreover, the alarming reviews of the OurPact Jr. Child App report that parents can monitor children's texts and visited websites by installing the app on the child's device. In Section~\ref{sec:rogue-capabilities}, we discuss these rogue functionalities in detail.


% \subsubsection{Identified Dual-Use Apps}
% We observed that many rogue apps have legitimate purposes (according to their descriptions) but are misused in practice. Their misuses are found via app reviews. Example~\ref{box:dualuseexample3} shows a few lines of the description of Find My iPhone app \cite{Findmyiphone}, which states that it is developed to track lost phones. However, \approach identifies alarming reviews that show how tracking functionality is misused for stalking children. Also, apps such as Snapchat \cite{Snapchat} and Kik \cite{Kik} connect users across globe, but are misused for stalking, as shown in reviews of Examples~\ref{box:dualuseexample1} and~\ref{box:dualuseexample2}. 



% \begin{mybox}[box:dualuseexample3]{Find My iPhone: Dual-Use App}
% % \begin{center}\textbf{\ul{App with Evidence of Rogue Behavior }}\end{center}
% \textbf{App: Find My iPhone \cite{Findmyiphone}} \\
% \textbf{Description:}
% \fsl{``If you misplace your iPhone, iPad, iPod touch, or Mac, the Find My iPhone app will let you use any iOS device to find it and protect your data. Simply install this free app, open it, and sign in with the Apple ID you use for iCloud. Find My iPhone will help you locate your missing device on a map, remotely lock it, play a sound, display a message, or erase all the data on it.\ldots''}
% \bigskip

% \textbf{Evidence of Rogue Behavior in App Reviews} \\
% Date of Review: 2014-03-14\\
% \fsl{``This app could be used for tracking a stolen or lost device, but is mainly used as a violation of privacy for children, by parents. This app confiscates trust and OCD in tracking adolescents, unpreparing them for the real world. Highly disagree with the disability of trust this app gives guardians over their children. More privacy please.''}
% \bigskip

% Date of Review: 2014-02-13\\
% \fsl{``This app is highly misused by soccer moms everywhere. It's supposed to be used to recover a lost phone, not to religiously stalk your children. Reading the reviews made me disgusted. The fact that a mom actually installed this app onto her son's phone without his knowledge is flat out wrong. Teens are supposed to do things their parents don't want them to do. If you're constantly monitoring your child 24/7, just imagine what your child will do when they go off to college. They'll go nuts with freedom. Parents are whats wrong with this generation, it's not the kids. - A mother of 2''}



% \end{mybox}



% \begin{mybox}[box:dualuseexample1]{Snapchat: Dual-Use App}
% % \begin{center}\textbf{\ul{App with Evidence of Rogue Behavior }}\end{center}
% \textbf{App: Snapchat \cite{Snapchat}} \\
% \textbf{Description:}
% \fsl{``Snapchat is a fast and fun way to share the moment with your friends and family. Snapchat opens right to the Camera — just tap to take a photo, or press and hold for video. Express yourself with Lenses, Filters, Bitmoji and more! Try out new Lenses daily created by the Snapchat community! \ldots''}
% \bigskip

% \textbf{Evidence of Rogue Behavior in App Reviews} \\
% Date of Review: 2019-10-30\\
% \fsl{``It's really dangerous bc people can track u and it makes people insecure because u know u didn't get invited to something bc u see everyone with out u and it allows kids to swear and be inappropriate with no supervision.''}
% \bigskip

% Date of Review: 2017-06-27\\
% \fsl{``You really need to remove this Snap Maps. It so dangerous for kids. \ul{Sadly not all parents monitor their kids and this is a perfect way for predators to find them.} I can't believe anyone would think this was a good idea. \ul{It is PREFECT for rapist, stalkers, thieves, and pedofiles.} Worst part is that most people don't even know it is turned on.''}



% \end{mybox}


% Further, from Chatterjee \etal's IPS list, we find \np{182} apps to be not rogue. Just because location tracking and remote data storage are functionalities of these apps, they may not be rogue because they obtain consent prior to exercising those functionalities. Example~\ref{box:Ipsexamples} shows the descriptions of three such apps, myTracks - The GPS Logger app \cite{Mytracks}, Where Are We Track \& Location app \cite{Wherearewe} and Familo: Find My Phone Locator app \cite{Familo}. These descriptions stress on taking consent, and we do not see any evidence of forced consent in the reviews. Thus, both descriptions and reviews do not reveal any evidence against these apps. However,
% Chatterjee \etal assume that a spy can physically access the victim's phone to install such apps. Moreover, they assume that victims ignore notifications about being tracked. Therefore, Chatterjee \etal include many tracking apps (Where Are We Track \& Location app \cite{Wherearewe} and Familo: Find My Phone Locator app \cite{Familo}) and remote data storage apps (myTracks - The GPS Logger \cite{Mytracks}) as IPS apps. Chatterjee \etal's assumptions and claims might not be true. 

% For the remaining \np{1450} apps in the seed dataset, \approach agrees with Chatterjee \etal, modulo the change in conception from IPS to rogue apps. Table~\ref{tab:labeled-dataset} summarizes this comparison.

\subsection{Identifying Additional Rogue Apps}
\label{sec:candidateapps}
The scoring part of our approach is not dependent on the choice of candidate apps and could be applied on any dataset of apps.
To identify additional rogue apps, we applied \approach's first two phases on two datasets: (i) dataset of similar apps and (ii) dataset of \np{100} popular apps in the utilities category.

\subsubsection{Similar Apps}
We retrieved \np{975} apps (similar to \np{100} predicted apps from the seed dataset), using the Apple App Store's recommendations (``You May Also Like'' section). Our motivation in using Apple's recommendations is that these apps should offer functionalities similar to those in 100 predicted apps. Out of the \np{975} apps, reviews of \np{896} apps were present on the Apple App Store, over the period 2008-08-13 to 2022-08-24. We obtained \np{2652678} reviews.
These \np{896} apps along with their reviews form our \emph{snowball dataset}, as shown in Table~\ref{tab:snowballdataset}.


\begin{table}[!htb]
\caption{Apps and reviews in the snowball dataset.}
\label{tab:snowballdataset}
\centering
% \begin{tabular}{p{6cm} l}
\begin{tabular}{n{4}{0}n{7}{0}}
\toprule
{Similar Apps} & {Reviews} \\
\midrule
896 & 2652678 \\

\bottomrule
\end{tabular}
\end{table}

We apply \approach's first two phases (described in Sections~\ref{sec:alarming} and~\ref{sec:rogue-score}) on the snowball dataset. \approach predicts \np{138} rogue apps. Examples~\ref{box:alarmingreview1} and~\ref{box:alarmingreview2} show alarming reviews of two such apps from the snowball dataset, Smart Family Companion App \cite{Smartfamily} and Bark - Parental Controls app \cite{Barkpc}.

In the snowball dataset, to curate the ground truth, we follow the same labeling process as described in Section~\ref{sec:verification}, for the apps with the highest \np{200} rogue scores. That's how we label \np{132} apps as rogue. 

Table~\ref{tab:performancecomparisonsnowball} shows the performance of all baseline methods and \approach on the snowball dataset. Our keywords when used on descriptions predicts only one app as rogue, leading to the lowest recall. This is because, on Apple App Store \cite{Appstore}, dual-use apps are not advertised using keywords: \fsl{spy}, \fsl{stalk}, and \fsl{stealth}. The same approach achieves \np{100}\% precision but high recall is desirable in the context of rogue apps.

On app descriptions, extended keywords perform better (68.18\% recall at 85.71\% precision) than our keywords due to commonly used words (such as \fsl{track}, \fsl{locate}) in app descriptions. Our keywords are applied on reviews (rows 3--5 in Table~\ref{tab:performancecomparisonsnowball}), and discover evidence of rogue behavior. However, among all approaches, \approach yields the best recall of 77.27\%. As we discussed, high recall is desirable than high precision, we conclude that \approach outperforms all other methods.



\begin{table}[!htb]
\caption{Performance (in \%) of baseline methods and \approach on the snowball dataset. Bold value for a metric indicates the highest score among all approaches.}
\label{tab:performancecomparisonsnowball}
\centering
\begin{tabular}{p{5.25cm}ccc}
\toprule
Method & Recall & Precision & F1 \\
\midrule
Our keywords on app descriptions & 00.75& \textbf{100.00} & 01.48 \\
Extended keywords on app descriptions & 68.18& 85.71& \textbf{76.26} \\
% Reading App Description& 73.97 & 87.09 & 80.00 \\
\midrule
0.3\% keyword reviews & 41.66& 91.66& 57.29 \\
0.2\% keyword reviews& 44.69& 92.18 & 60.20 \\
0.1\% keyword reviews &51.51 & 88.31& 65.07 \\

\midrule
\approach & \textbf{77.27} & 73.91& 75.55 \\
\bottomrule
\end{tabular}
\end{table}



\begin{mybox}[box:alarmingreview1]{Rogue from Snowball Dataset}
% \begin{center}\textbf{\ul{Rogue App from Snowball Dataset}}\end{center}

\textbf{App: Smart Family Companion \cite{Smartfamily}} \\
\textbf{Rogue Score: 2.35/4.00}\\


\textbf{Alarming Review 1} \\
\textbf{Alarmingness: 4.00/4.00}\\
Date of Review: 2020-03-15 \\
\fsl{``\ul{How is this even ethical? To put out an app in which you can completely control what's going on on someone else's phone? It's a huge privacy concern.} To be honest, apps like this shouldn't exist. It's one thing to put control on a YOUNG CHILD'S phone (which can be done in settings easily) put to put this on an older kids phone is going to destroy trust. No parent should be able to see what their child is doing on their phone 24/7. It's borderline abusive.''}

\bigskip

\textbf{Alarming Review 2} \\
\textbf{Alarmingness: 4.00/4.00}\\
Date of Review: 2019-09-03 \\
\fsl{``\ldots \ul{This app tracks every last thing your child does on their phone. As you can imagine, no 16 year old wants to have their own private life constantly exposed to you.} Just because you are their parent, and you live together, doesn't mean that they have to share everything with you. \ul{Location, data usage, browsing history, etc. frankly aren't your business. That's their own private information that you don't need to know.} Coming from a family with control issues, there is no better way to destroy your relationship with your children. I doubt anyone would want to be around someone who is constantly monitoring and controlling them \ldots''}


\end{mybox}
\begin{mybox}[box:alarmingreview2]{Rogue App from Snowball Dataset}
% \begin{center}\textbf{\ul{Rogue App from Snowball Dataset}}\end{center}

\textbf{App: Bark - Parental Controls  \cite{Barkpc}} \\
\textbf{Rogue Score: 2.38/4.00}\\


\textbf{Alarming Review 1} \\
\textbf{Alarmingness: 4.00/4.00}\\
Date of Review: 2020-05-28 \\
\fsl{``This app is unfair and invasion of privacy! Kids shouldn't be watched like this all kids cuss and this app tracks that and then snitches on you for it. I don't understand why someone can have so much doubt in their kids. Yes, I understand some kids do some really bad things that shouldn't be done, but if u raise your kids right and teach them right from wrong then you'd be able to trust them. One of my best friends has this app and she literally tells me how much she hates her parents. My friend has never even done anything and she has no reason for this app to be on her phone. I know the internet is dangerous but telling your kids it's dangerous honestly has a bigger effect. Maybe try other methods until it gets to the point of this app abolishing all their freedom and happiness.''}\bigskip


\textbf{Alarming Review 2} \\
\textbf{Alarmingness: 4.00/4.00}\\
Date of Review: 2021-03-12 \\
% 
\fsl{``If I could give this zero stars I would. This app is a total invasion of privacy and if you want to ruin your chances of having a relationship with your child, then get this. But if you are one of those parents who don't give your child/teen privacy you are not only hurting them but you are also hurting that bond and relationship with them. As a teen we don't want privacy because we are trying to hide something we just want privacy to be able to feel like our own person \ldots''}\bigskip

% \textbf{Alarming Review 3} \\
% \textbf{Alarmingness: 3.198/5}\\
% Date of Review: 2017-08-05 \\
% % 
% \fsl{``i cant unadd people!! this one creepy guy who wont stop stalking me wont leave me alone and i cant unadd him!! help''}

\end{mybox}


\subsubsection{100 Popular Utility Apps}
Surveillance apps that can be misused for spying fall under the ``Utilities'' category, making utilities an important category to scrutinize. We consider \np{100} popular utility apps that are mentioned on Apple App Store page \cite{Utilitiesappstore}. Out of \np{100} apps, nine are already scrutinized either in the seed or snowball dataset. For the rest \np{91} apps, we retrieved \np{392928} reviews, over the duration of 2008-10-18 to 2022-08-04. 

\approach predicts only one app as rogue, which after reviews' scrutiny by us, comes out to be non-rogue. We also scrutinize \np{10} apps with the highest rogue scores, by reading their top 50 alarming reviews and reviews containing our keywords. But, none of them are actually rogue. Since the Apple App Store \cite{Appstore} contains a wide variety of utility apps, the selected \np{91} apps contain subcategories such as payment, calculator, and television remote. As a result, no video surveillance or parental control app, which have high potential to be misused, are part of 91 apps. Thus, popular utility apps don't form a good candidate set for rogue apps. This problem can arise for any generalized set of apps under any category. Thus, an iterative process of checking similar apps (through \approach), to the already identified rogue apps, can accelerate identifying more and more rogue apps. 


\section{Uncovering Rogue Functionalities}
\label{sec:rogue-capabilities}

We now uncover rogue functionalities that are found via app reviews.
Identifying such functionalities can help both users and developers. App users can understand the risk associated with the app and developers can rectify apps to reduce such risks.

For this study, we considered apps in the seed dataset with the \np{40} highest rogue scores. 
For each app, we manually analyzed its description and its \np{10} most alarming reviews discovered by \approach. An app's description provides basic knowledge about the app's functionalities and alarming reviews report misuse of such functionalities. Through this exercise, we discovered the following types of rogue functionalities:

\begin{description}[leftmargin=1em]
\item[Monitoring phone activities.] Some apps monitor a victim's phone activities, such as browsing history and text messages. Such apps are installed on the victim's device and activities can be monitored on another synced device.

\item[Audio or video surveillance.] Some apps enable audio or video surveillance without the victim's knowledge. These apps listen, view, or record a victim's voice or actions and some of them need not be installed on the victim's phone.

\item[Tracking location.] Some Global Positioning System (GPS) apps enable tracking a victim's phone, with (forced consent) or without their knowledge.

\item[Profile stalking.] Some apps are misused for stalking of user profile or user content (such as images), making the victim uncomfortable of the information access.

\end{description}

Table~\ref{tab:rogue-capability-types} shows these four types of rogue functionalities, and alarming reviews reporting them. Some reviews in Table~\ref{tab:rogue-capability-types} are old (2014 or 2012), but we confirmed that similar concerns are being raised in the recent reviews of the same apps. For example, the Find My iPhone \cite{Findmyiphone} app still lets its users see the location of the connected devices. Due to unequal power dynamics, the victim can be forced to connect to such apps and allow their device to be located.% \nsa{Good; think in terms of how to convince readers including reviewers.}\vg{addressed above}


We also verified the rogue behavior of the SaferKid Text Monitoring App by installing it on two devices: a parent's device (iOS version 14.4.1) and a child's device (Android version 11.0). Figure~\ref{fig:Appusage} shows the rogue features present in this app.  Activities on the child's device can be monitored on the synced parent's device. Figure~\ref{fig:Safercapabilities} shows SaferKid rogue functionalities such as monitoring text messages, web history, and call history. We verified each of these functionalities. Figure~\ref{fig:Saferchats} shows the screen displaying all chats of the victim. Apps such as SaferKid are advertised as safety apps for children, but can be secretly or forcefully installed on another device to monitor user's activity. Not only parents, but any individual can misuse such apps by installing them on the victim's phone. 

\begin{figure}[t]
\centering
\subfigure[The SaferKid \cite{Saferkid} app provides multiple ways to monitor victim's phone activities.]{
\includegraphics[width=0.45\columnwidth]{figs/SaferCapabilities.pdf}\label{fig:Safercapabilities}
}
\subfigure[Using the SaferKid \cite{Saferkid} app, victim's chats can be seen on the abuser's phone.]{
\includegraphics[width=0.45\columnwidth]{figs/SaferChats.pdf}\label{fig:Saferchats}
}

\caption{Rogue functionalities in  SaferKid Text Monitoring \cite{Saferkid}.}
\label{fig:Appusage}
\end{figure}

%  idk1727272727 , 
% 02/06/2021
% This is sickening

%     This app is appalling like other apps like this it ruins my teenage years I have to have the burdens of being spied on every day of my life for all I know this app could be spying on this review this app has kept me from making new friends because if I add a new contact it will send an alert and I am forced to delete it I also am punished if I talk bad about a teacher In the slightest bit teenage years are supposed to be the best years of your life and I just can't wait for the day that I turn 18 and go to college cause the. I will get the freedom I need but it will be to late and I will already have adult responsibilities my parents got to live in a care free teenage years environment and it's not right that I have to suffer. I just will never make the same mistake with my kids because I want them to have fun I want them to not hate there life and I want them to have a good relationship with me because I have seen what these apps do and I just am counting down the days till I'm free my teen age years should not be a prison I never did anything wrong . I hope this wasn't the intent of the app but you trying to make things better has grown into a terrible problem and I hate my life and am counting down the days I look forward to school because that's when I get to live my life and I go to a very strict school so that's saying something



% \begin{figure}[!htb]
%     \noindent\includegraphics[width=\columnwidth,height=10cm]{figs/Safercapabilities.pdf}
%     \caption{The rogue capabilities mentioned in reviews and confirmed while app usage.}
%     \label{fig:safercapability}
% \end{figure}  



% \begin{figure}[!htb]
%     \noindent\includegraphics[width=\columnwidth,height=10cm]{figs/SaferChats.pdf}
%     \caption{The perpetrator can use the SaferKid Text Monitoring app to see victim's chats.}
%     \label{fig:saferchats}
% \end{figure}  


\begin{table*}[!htb]
    \centering
    \caption{Types of rogue functionalities.
    % \nsa{But it would not be appropriate to flag an app now for a 10 year old review. A reviewer may point out that. If  we have a list of inappropriate apps and their inappropriate reviews, we may consider submitting that as supplementary material (if that is allowed).}\vg{I have new alarming reviews for these apps, but they are not in top 10. For functionality section, we focused more on top 10, that's why including them in this table. But if we can submit supplementary material, we can add all reviews. I am pretty sure that these apps still have mentioned functionalities. In text, I can write about similar concerns also raised by new reviews.}
    % \nsa{Good; think in terms of how to convince readers including reviewers.}\vg{I hv addressed in the text describing table 11}
    } \label{tab:rogue-capability-types}
    % \begin{tabular}{p{2.9cm} p{2.1cm} p{7.7cm}}
    \begin{tabu}{X[2.9,L] X[2.1,L] X[7.7,L]}
    \toprule
    Rogue Functionality & App Example & Alarming Review\\
    
    \midrule
    
    Monitoring phone activities& SaferKid Text Monitoring App \cite{Saferkid}  &\fsl{\ldots Tracking things like social media, texts, and search history is just a complete disregard of privacy. You have to have trust in your kids \ldots Apps like these shouldn't be allowed. IF YOU TRUST YOUR KID, DONT DOWNLOAD.} (Date: 2019-12-07)\\

    Audio or video surveillance & Find My Kids: Parental control \cite{Findkids} & \fsl{This app proves to have a invasion of privacy. Due to the fact if your kids was at a friends house and talking to his friends parents, this app records what is going on and is a invasion of privacy. If your child left their phone downstairs or anywhere and they are playing it can record private conversation between adult and is a unsafe \ldots} (Date: 2019-01-16)\\
    

    Tracking location& Find My iPhone \cite{Findmyiphone} &\fsl{\ldots It's supposed to be used to recover a lost phone, not to religiously stalk your children.\ldots The fact that a mom actually installed this app onto her son's phone without his knowledge is flat out wrong. \ldots If you're constantly monitoring your child 24/7, just imagine what your child will do when they go off to college. \ldots} (Date: 2014-02-13)\\
    
    Profile stalking& WhatsApp Messenger \cite{Whatsapp}  &\fsl{ \ldots However there is one negative about the App! The stalker look at the time stamp to monitor other people not nice please improve on that we need a sense of privacy from theses stalker"} (Date: 2012-11-21)\\
    
    \bottomrule
    \end{tabu}
    % \end{tabular}
\end{table*}




\section{Related Work}
\label{sec:background}
We describe previous works focusing on (i) spying through mobile apps, (ii) user privacy on social media, and (iii) NLP techniques to find apps' privacy issues.

\subsection{Spying through Mobile Apps}
Prior studies \citep{IPVspyware2018,Creepware2020,IPSTools2020,Phonehacked2019,Stalkerparadise2018,IPVcutomersupport2021,Safetydilemma2021} investigate how technology is abused for spying. A major segment of this research deals only with IPS. Chatterjee \etal \cite{IPVspyware2018} identify IPS apps with carefully designed search queries and manual verification based on app information. They leverage information such as app descriptions and permissions. 
However, for dual-use apps, the actual usage deviates from the intended purpose shown in app descriptions.
To identify such misuse, we focus on the evidence provided in app reviews. Moreover, the scope of rogue apps is broader than IPS apps. 


Roundy \etal \cite{Creepware2020} focus on identifying apps used for phone number spoofing and message bombing, which lie outside the scope of rogue apps. Conversely, rogue apps include those that enable stalking public information, which are outside their scope. 
Roundy \etal use metadata such as installation data, to uncover spying apps that are installed on infected devices. However, we focus on evidence of rogue behavior present in app reviews, to uncover rogue apps. Roundy \etal rely upon Norton's security app \cite{NortonApp} to determine which devices are infected. Thus, their approach would miss apps that a general user can leverage to spy.

Some prior studies focus on analyzing spyware apps or victims' experiences. Freed \etal \cite{Phonehacked2019} present a qualitative analysis of victims' experiences, including their technology-related concerns. They report that security vulnerabilities were present in the phones of 14 out of 31 victims in their sample. Tseng \etal \cite{IPSTools2020} study the IPS problem from the attackers' perspective. They analyze online forums in which attackers participate, propose a taxonomy of IPS tools and attacks. Tools may require physical device access, e.g., to install GPS trackers; or, they may rely on virtual access, e.g., through shared accounts of intimate partners. Attacks may include coercion or subterfuge, or may involve hiring another person to spy on someone. Havron \etal \cite{Clinicalsecurity2019} propose a consultation method, called clinical security, to help victims by discovering and removing spyware, and advising victims about security vulnerabilities in their phones. Moreover, \citet{Careinfrastructure2022} develop sociotechnical systems with feminist notions to help IPV survivors.
Freed \etal \cite{Stalkerparadise2018} survey spyware apps for intimate partners. They mention covert apps (also known as dual-use apps) that are capable of spying on victims but are not advertised as such. \citet{Antistalkerware2022}, through app reviews, study  users' expectations from anti-stalkerware apps. They perform thematic analysis on \np{518} reviews of two apps and find a huge gap between users' perception and the actual abilities of such apps. All these studies along with others \citep{Belline+PACMHCI20-IPVNarratives,IPVcutomersupport2021,Safetydilemma2021} are limited to IPS apps and not the broader set of rogue apps. Moreover, they do not consider cases when the victim is uncomfortable of the access (of public information) even if aware of it. 

% \citet{TikTokadvice2022} study TicTok videos that describe how to abuse technology for surveillance on others. They investigate the types of techniques, assets being targeted, and motivation to abuse.
\subsection{User Privacy}
Prior works study risk of losing users' private information on online social media platforms and propose methods to mitigate such risk. Georgiou \etal \cite{Cyborg2017} protect users' privacy by giving warnings whenever a user may reveal sensitive attributes such as location or race present in social media posts. Mahmood and Desmedt \etal \cite{Cloakedspy2012} claim that the Facebook friends of a user can access the user's private information in a cloaked manner. The study shows that it is possible to stalk and target victims on Facebook. Mahmood and Desmedt \etal provide strategies to avoid such attacks. 
Reichel \etal \cite{Facebookprivacy2020} study the privacy perspective of users in developing countries. They interview \np{52} social media users in South Africa to understand their privacy beliefs. Reichel \etal conclude that many participants are concerned about other users being able to see their online posts and messages, instead of the private data collected by the app platform itself. Many participants admitted that unknown people (on WhatsApp and Facebook) stalked or harassed them. To combat these challenges, Reichel \etal provide recommendations to fulfill users' security needs in resource-constrained situations. Some studies contribute to uncovering privacy risks associated with shared images \citep{Snapme2013,Privacytag2014,Bystanders2017}, which can contain bystanders (persons who are not prime subject of image) and are shared widely without bystanders' consent. Hasan \etal \cite{Bystanders2020} leverage visual features to detect bystanders in images present in the Google open image dataset \cite{Imagedata2020}. 

These studies are applicable to specific social media platforms and not to all rogue apps. Moreover, they do not provide a framework to identify rogue apps and their functionalities.  


\subsection{Using Natural Language Processing}
We present prior studies that apply NLP techniques for security and privacy of mobile apps. 

Some previous works leverage app reviews and privacy policies to identify user's security and privacy issues. 
Nguyen \etal \cite{Userreviews2019} train a classifier to predict if an app review pertains to security and privacy concerns. Using regression analysis, they show that security and privacy related reviews play an important factor in predicting privacy related app updates. Besmer \etal \cite{Userperception2020} leverage app reviews to understand how users' perception of privacy is reflected in their sentiments about the app. They train a machine learning classifier to determine whether a review is privacy related. Further, they analyze the sentiments of reviews predicted as privacy related. Harkous \etal \cite{Polisis2018} propose a privacy-centric language model to extract useful information from long privacy policies. The extracted information helps users understand how apps collect and manage users' personal information. To train the language model, they leverage \np{130000} privacy policies. The trained model extracts both high-level and fine-grained details from policies. However, these studies focus on how an app which can steal a user's information. In contrast, we focus on the privacy of a victim (user or third party) with respect to another user.


% Khalid \etal \cite{Usercomplain2015} leverage app reviews to uncover most common user complains related to mobile apps. Khalid \etal focus on one and two star rated user reviews to read through the complains. Khalid \etal find that complains related to privacy and ethical issues have one of the most negative impact on app score. 

Some prior works distinguish between the actual and the expected behavior (from user's perspective) of an app, by using textual sources such as descriptions and privacy policies.
Gorla \etal \cite{Appdescription2014} identify which apps deviate from their descriptions, by extracting topics from app descriptions, using Latent Dirichlet Allocation (LDA), and clustering apps based on those topics. For each cluster, Gorla \etal find outliers with respect to apps' APIs usage. Qu \etal \cite{Autocog2014} and Pandita \etal \cite{Whyper2013} use NLP techniques on app descriptions and find disparities between app descriptions and functionalities. Zimmeck \etal \cite{Privacypolicy2017}
propose an automated system to find Android apps' compliance with their privacy policies. They combine static code analysis and machine learning to uncover inconsistencies between privacy policies and app source code. Out of \np{9050} apps, Zimmeck \etal find that \np{17}\% of apps collect sensitive information such as location, but do not mention it in their privacy policies. All these works address expectation violation when the app developer has malicious intentions. However, in our work, we address expectation violation when an app user has malicious intentions to spy or stalk. To the best of our knowledge, \approach is the first automated system to identify rogue apps.


% App Stores provide platform to their users to submit feedback about the apps. App users provide this feedback in the form of text reviews. Previous studies show that app reviews include valuable information to which app developers should pay close attention \citep{appstore2013,SentimentAppReviews2014}. This information can be about bugs, suggestions for new features, or appreciation of existing features in the app \cite{bugorfeature}. 
% Chen \etal \cite{ARMiner2014} extract informative reviews that developers can investigate to gather important feedback. Sorbo \etal \cite{Userschange2016} summarize the relevant information present in app reviews so that the developer can identify the parts of apps that need to be improved. Nguyen \etal \cite{Userreviews2019} analyze app reviews to discover users' security and privacy concerns about Android apps. Our work contributes to the security community by showing app reviews include evidence of unexpected information gathering.

\section{Discussion}
\label{sec:conclusion}
%%%%%%%%%%%%%%%%%%%%%%%%%%%%%%%%%%%%%%%%%%%%%%%%%
We proposed \approach, an approach to automatically analyze app reviews for detection of rogue apps and rogue functionalities. \approach, first, predicts alarmingness of reviews, followed by rogue score for each app. In total, \approach predicts 239 rogue apps (100 and 139) from multiple sources, leading to the best recall, as compared to other baseline methods. We have also shared the identified rogue apps along with their reviews, to the Apple App Store. The platform will investigate these apps and will reaching out to the developers for correcting functionalities in their apps. 


Below, we describe our data availability, threats to validity, and promising future directions.

\subsection{Data Availability}


Upon acceptance of this paper, we will release the complete list of all identified rogue apps. We will also release the scraped reviews as an open dataset.
Our work is reproducible; we used Tensorflow Hub \cite{Tensorflowhub} for extracting features and Scikit-learn library \cite{Scikitlearn2011} for training the regression model, which will make available with the dataset.

\subsection{Threats to Validity}

We now discuss the threats we identify in our work. The identified threats are of two types: (i) the threats that we mitigate; and (ii) the threats that still remain.

\subsubsection{Threats Mitigated}
\label{sec:threats-mitigated}
We mitigated the following threats to validity. First, reviews in the set s\fsub{1} contained our keywords. This may create a bias in the model to predict high scores for only reviews having our keywords. To mitigate this threat, we removed our keywords before training the model. This helped the model to learn from the context and not from specific keywords (described in Section~\ref{sec:sentenceencoder}). Second, review annotation by crowd workers could yield incorrectly rated reviews, because of their inability to understand the problem well. Thus, two authors of this paper annotate the whole training data. Third, the ground truth (of rogue apps) could be biased if it was formed only using top alarming reviews. We mitigated this bias by scrutinizing reviews containing our keywords, which can contain evidence missed by alarming reviews

\subsubsection{Threats Remaining}
Now, we describe the threats that still remain in our work.
First, we investigate only a few thousand apps, which may not be representative of all apps on the Apple App Store \cite{Appstore}. The performance of \approach may vary while testing it on all apps of the Apple App Store. 
Second, we target apps and their reviews only on Apple's App Store. Upon deployment on other app stores, the performance of our approach can differ. 
Third, if an app distribution platform does not have a similarity recommendation, \approach may have to be applied on all apps---a computationally expensive task. However, in such cases, \approach can be prioritized for the apps that are flagged by app users (victims). Fourth, some negative reviews (about rogue behavior) may be written by the app's competitors. Identifying such fake reviews is out of the scope of our study.


\subsection{Limitations and Future Directions}

We identify following limitations of this work. Each limitation gives rise to possible future work. First, \approach may miss some rogue apps if they do not have alarming reviews at the time of analysis, possibly because they are new apps. However, such rogue apps can be identified as soon as alarming reviews begin to arrive. By leveraging the current evidence, \approach helps protect future users and third parties. 
A possible extension for \approach would be to include other information sources such as privacy policies, to identify rogue apps ahead of time. 

Second, uncovering rogue functionalities involves manual effort of inspecting top alarming reviews. A possible future direction is to automate this process.




 
\bibliographystyle{IEEEtranSN}
\bibliography{Hui,Nirav,Vaibhav}

\end{document}
