\section{Experiments}
\label{sec:experiments}
We evaluate the proposed models on the arrival trajectories to KJFK.
We trained the models on the flight tracks and procedures for all the runways except for 13L. Then, the performance of the models \textcolor{blue}{is} validated using 13L data.
This makes an approximately 90--10 training-test split of our data. 
To construct the input vectors as described in Section \ref{subsec:input vector construction}, we set the input length to 150 for the final approach segment and 350 for the radar vector segment.

\subsection{Model Selection}
\label{subsec:model_selection}

% \soyeonstrike{Prior to training, we determine the numbers of Gaussian components and the rank for each Gaussian covariance matrix. To evaluate the cluster performance with different numbers of Gaussian components, we use the silhouette score, a measure of how well a data point is matched to its own cluster (cohesion) compared to the other clusters (separation). The silhouette coefficient for one data point $i$ is defined as}


The number of clusters is an important hyper-parameter for clustering algorithms including the GMM.
To determine the number of clusters, i.e. Gaussian components in GMMs, we evaluate the cluster performance of our models with different numbers of Gaussian components using the silhouette method.
The silhouette method selects the number of clusters by finding the number that maximizes the silhouette score, i.e. the average of silhouette coefficients over all data points in the entire dataset.
A silhouette coefficient measures how well a data point is matched to its own cluster (cohesion) versus to the other clusters (separation) \cite{rousseeuw1987silhouettes}.
The silhouette coefficient for the $i$th data point is
\begin{align}
s(i) = \frac{b(i)-a(i)}{\max \{ a(i), b(i) \} }
\end{align}
where $a(i)$ is the mean intra-cluster distance (i.e., mean Euclidean distance to the other points in the same cluster) and $b(i)$ is the mean nearest-cluster distance (i.e., mean Euclidean distance to the points in the closest cluster).
The value of $s(i)$ ranges from $-1$ to $+1$, where a higher value indicates that $i$ is well matched to its cluster while poorly to its neighboring cluster. 
If most points have high values, the data points are appropriately clustered.


The silhouette method has several advantages for optimizing the number of clusters. 
It provides an exact solution whereas other methods including the elbow method have to rely on heuristics. 
The silhouette method also quantifies the clustering performance better, as it evaluates both intra-cluster and inter-cluster distances. A good clustering can be characterized by high inter-cluster distance and low intra-cluster distance. \deleted{There are other possible heuristics, which include the Akaike Information Criterion (AIC) and Bayesian Information Criterion (BIC); however, for the aforementioned reasons we use the silhouette method in this work \cite{assess_components2000}.} \added{The silhouette score is illustrated here primarily due to its simplicity compared to heuristic ``elbow'' methods. There are many other possible ways to choose the number of clusters, such as the Akaike Information Criterion (AIC) and the Bayesian Information Criterion (BIC) \cite{assess_components2000}. AIC and BIC tend to give similar results as the silhouette score.}

We select the number of Gaussian components that maximize the silhouette score.
% \soyeonstrike{(i.e., the mean silhouette coefficient over all data points of the entire dataset.)}
Fig. \ref{fig:silhouette_iap_vector} shows the silhouette scores with different numbers of Gaussian components for the final approach model and radar vector model.
The final approach model has the highest score of 0.5432 with two Gaussian components, and the radar vector model has the highest score of 0.4937 with six Gaussian components.
\begin{figure}[bt!]
\centering
\setlength\figureheight{5.5cm}
\setlength\figurewidth{9cm}
% This file was created by matplotlib2tikz v0.6.18.
\begin{tikzpicture}

\definecolor{color0}{rgb}{0.392156862745098,0.584313725490196,0.929411764705882}

\begin{axis}[
height=\figureheight,
legend cell align={left},
legend entries={{final approach},{radar vectoring}},
legend style={at={(0.97,0.03)}, anchor=south east, draw=black},
tick align=outside,
tick pos=left,
width=\figurewidth,
x grid style={white!69.01960784313725!black},
xlabel={Number of Gaussian components},
xmin=1.15, xmax=19.85,
y grid style={black},
ylabel={Silhouette scores},
ymin=0.26, ymax=0.57,
ytick={0.30, 0.35, 0.40, 0.45, 0.50, 0.55},
tick label style={font=\tiny},
label style={font=\scriptsize},
legend style={font=\scriptsize}
]
\addlegendimage{mark=square*, line width=1pt, color0}
\addlegendimage{mark=square*, line width=1pt, blue, dashed}
\addplot [line width=1pt, color0, mark=square*, mark size=1, mark options={solid}]
table [row sep=\\]{%
2	0.543160999467294 \\
3	0.447885381746271 \\
4	0.44150655956062 \\
5	0.314570032084669 \\
6	0.332451939072078 \\
7	0.441186079661681 \\
8	0.476373440832541 \\
9	0.490450337400935 \\
10	0.530992943464481 \\
11	0.529126425038986 \\
12	0.53115630163428 \\
13	0.529013952826381 \\
14	0.526929042322887 \\
15	0.523626300441002 \\
16	0.518805112545959 \\
17	0.521322903581658 \\
18	0.515069268838517 \\
19	0.518925910263206 \\
};
\addplot [line width=1pt, blue, dashed, mark=square*, mark size=1, mark options={solid}]
table [row sep=\\]{%
2	0.285397566028243 \\
3	0.323408402354702 \\
4	0.475909060085525 \\
5	0.493234614872401 \\
6	0.493661614016874 \\
7	0.484684414694431 \\
8	0.473991067650981 \\
9	0.465647769098023 \\
10	0.464150055667017 \\
11	0.452642576444028 \\
12	0.442960394770543 \\
13	0.438190935025079 \\
14	0.432229102142355 \\
15	0.422802327630602 \\
16	0.423581095476773 \\
17	0.412475495558834 \\
18	0.399606936627248 \\
19	0.392875310312757 \\
};
\end{axis}

\end{tikzpicture}
\caption{Number of Gaussian components vs. silhouette scores.}
\label{fig:silhouette_iap_vector}
\end{figure}

% \soyeonstrike{Using the resulting numbers of Gaussian components, 
% we determine the optimal rank for each Gaussian covariance matrix (i.e., the rank that maximizes the log-likelihod) as described in Section} \ref{subsec:low-rank approximation}.

Once we determine the numbers of Gaussian components, we optimize the ranks for the low-rank approximations of the covariance matrices 
as described in Section \ref{subsec:low-rank approximation}.
Fig. \ref{fig:low_rank_approximation_iap_vector} shows the log-likelihood with different numbers of principal components (PCs) for the final approach model and the radar vector model.
As a result, the 452-dimensional input data of the final approach model can be best represented with 82 PCs, and the 1052-dimensional input data of the radar vector model can be best represented with 241 PCs.
\begin{figure}[bt!]
    \centering
    \subfigure[Final approach model]{
    \setlength\figureheight{5cm}
    \setlength\figurewidth{8cm}
    % This file was created by tikzplotlib v0.9.2.
\begin{tikzpicture}

\definecolor{color0}{rgb}{0.392156862745098,0.584313725490196,0.929411764705882}

\begin{axis}[
legend cell align={left},
legend style={fill opacity=0.8, draw opacity=1, text opacity=1, at={(0.03,0.03)}, anchor=south west, draw=white!80!black}, font=\scriptsize,
tick align=outside,
tick pos=left,
x grid style={white!69.0196078431373!black},
xlabel={Number of principal components},
xmajorgrids,
xmin=-4.85, xmax=123.85,
xtick style={color=black},
y grid style={white!69.0196078431373!black},
ylabel={Log-likelihood},
ymajorgrids,
ymin=-3673.16171610547, ymax=-1540.86963896701,
ytick style={color=black},
height=\figureheight,
width=\figurewidth,
tick label style={font=\tiny},
label style={font=\scriptsize}
]
\addplot [line width=1pt, color0, forget plot]
table {%
1 -3576.23934896281
4 -3166.46342875311
7 -2676.37524161896
10 -2532.93526479994
13 -2401.83329424859
16 -2312.24628242538
19 -2237.40096413937
22 -2161.20964576721
25 -2098.32510007955
28 -2061.22182697219
31 -2014.65900535917
34 -1973.7597839886
37 -1935.44639568868
40 -1902.37030692099
43 -1869.55544064166
46 -1844.09827742081
49 -1812.5556854179
52 -1787.23316980128
55 -1762.68212411381
58 -1743.63158220817
61 -1724.7368759245
64 -1703.38068384118
67 -1687.07140172762
70 -1674.45362868706
73 -1660.10416529263
76 -1652.32328136812
79 -1646.26829681805
82 -1637.79200610967
85 -1644.88461469593
88 -1663.75225907883
91 -1693.11227943591
94 -1738.31469100778
97 -1788.36248019015
100 -1866.17047215528
103 -1959.40653891655
106 -2082.63064946482
109 -2233.56695964214
112 -2423.42309932243
115 -2640.96195551629
118 -2921.71241034859
};
\addplot [line width=1pt, color0, dashed]
table {%
82 -3673.16171610547
82 -1540.86963896701
};
\addlegendentry{argmax(ll) = 82}
\end{axis}

\end{tikzpicture}
}
    \subfigure[Radar vector model]{
    \setlength\figureheight{5cm}
    \setlength\figurewidth{8cm}
    % This file was created by tikzplotlib v0.9.2.
\begin{tikzpicture}

\begin{axis}[
legend cell align={left},
legend style={fill opacity=0.8, draw opacity=1, text opacity=1, at={(0.97,0.03)}, anchor=south east, draw=white!80!black, font=\scriptsize},
tick align=outside,
tick pos=left,
x grid style={white!69.0196078431373!black},
xlabel={Number of principal components},
xmajorgrids,
xmin=-24, xmax=526,
xtick style={color=black},
y grid style={white!69.0196078431373!black},
ylabel={Log-likelihood},
ymajorgrids,
ymin=-11051.7608679236, ymax=-3727.05901271434,
ytick style={color=black},
height=\figureheight,
width=\figurewidth,
tick label style={font=\tiny},
label style={font=\scriptsize}
]
\addplot [line width=1pt, blue, forget plot]
table {%
1 -10351.091372581
6 -9390.29359640526
11 -8527.74752744774
16 -7903.40206105794
21 -7490.5928993267
26 -7120.96882745967
31 -6831.28167651557
36 -6571.9952268888
41 -6359.88297586737
46 -6170.96145818454
51 -6014.11290402957
56 -5872.2950960039
61 -5739.01784809046
66 -5615.47991276329
71 -5511.89210434147
76 -5407.96787815711
81 -5316.5264021807
86 -5229.85766572322
91 -5151.52607292256
96 -5078.44127039378
101 -5011.42107834493
106 -4950.74432311115
111 -4889.44458859846
116 -4831.94946353567
121 -4778.78372448935
126 -4726.38447108067
131 -4678.45004290727
136 -4628.80688123572
141 -4585.33708008752
146 -4541.31427657975
151 -4498.77372454233
156 -4459.26989206967
161 -4417.11993132059
166 -4376.1896841801
171 -4337.37939533848
176 -4302.4885183807
181 -4266.23227598754
186 -4233.9499301315
191 -4204.62757723826
196 -4178.05280795301
201 -4152.24345407912
206 -4129.76973772367
211 -4109.34721272493
216 -4093.00207578259
221 -4080.01033512945
226 -4068.53320608522
231 -4063.447341123
236 -4061.19573093753
241 -4060.00000613294
246 -4065.05278745489
251 -4075.08725256417
256 -4090.41227087023
261 -4109.35525721331
266 -4136.15960424546
271 -4170.77513539739
276 -4203.78637926737
281 -4251.85993686046
286 -4299.111034555
291 -4355.41587096141
296 -4418.19416609366
301 -4484.392697973
306 -4561.92931089684
311 -4643.84898202899
316 -4738.03267096697
321 -4843.10810509342
326 -4957.71811848518
331 -5085.18928061224
336 -5224.23827479698
341 -5365.80307883986
346 -5510.2139115369
351 -5671.6354297764
356 -5838.8209801691
361 -6008.67048258964
366 -6184.53722794659
371 -6375.57497777825
376 -6557.8284285908
381 -6758.27707151676
386 -6953.46289810854
391 -7147.98109913903
396 -7358.01794421731
401 -7543.44595770761
406 -7743.54925260866
411 -7926.39769325349
416 -8100.14000500198
421 -8257.18107727013
426 -8445.39302748818
431 -8627.46085570133
436 -8801.36137819993
441 -8964.28339598023
446 -9130.44110746846
451 -9272.72648759467
456 -9440.08745745432
461 -9542.04406503531
466 -9728.61914326514
471 -9840.02121999606
476 -9993.8857247293
481 -10145.146107708
486 -10273.3291432573
491 -10426.6355094752
496 -10553.6656116963
501 -10718.819874505
};
\addplot [line width=1pt, blue, dashed]
table {%
241 -11051.7608679236
241 -3727.05901271433
};
\addlegendentry{argmax(ll) = 241}
\end{axis}

\end{tikzpicture}
}
    \caption{Number of principal components (PC) vs. log-likelihood.}
    \label{fig:low_rank_approximation_iap_vector}
\end{figure}


\subsection{Model Validation and Generalization}
\label{subsec:experiment_single_trajectory_model}
With the optimized parameters, we train the single trajectory model and generate synthetic trajectories from the trained model
as described in Section \ref{sec:single trajectory model}.
Fig. \ref{fig:jfk_13L_hist_org} and Fig. \ref{fig:jfk_13L_hist_syn} show 2D log-histograms of 1,200 actual trajectories and 1,000 synthetic trajectories arriving at KJFK 13L.
As in Fig. \ref{fig:jfk_04R_orginal_paths}, the radar vector procedures and the IAP are indicated in blue dotted lines and a white dashed line with orange edges, respectively.
From the generated trajectories, we observe that the model can learn the general behavior of aircraft with respect to the procedure, but not the distinctive patterns of each individual procedure.
This is because after our model obtains the deviation trajectories with different procedures, it merges all of them into a single dataset for each segment.
\begin{figure}[bt!]
    \centering
    \subfigure[Actual trajectories]{
    \setlength\figureheight{6.7cm}
    \setlength\figurewidth{6.7cm}
    % This file was created by matplotlib2tikz v0.6.18.
\begin{tikzpicture}
\definecolor{color0}{rgb}{0.392156862745098,0.584313725490196,0.929411764705882}
\definecolor{color1}{RGB}{255,165,0}

\begin{axis}[
colorbar,
colorbar style={width=7,
ytick={0,1,2,3},
yticklabel={$10^{\pgfmathprintnumber{\tick}}$}},
colormap = {whiteblack}{color(0cm)=(white); color(1cm)=(black)},
height=\figureheight,
point meta max=3.2,
point meta min=0,
tick align=outside,
tick pos=left,
width=\figurewidth,
x grid style={black},
xlabel={East (NM)},
xmin=-20, xmax=25,
y grid style={black},
ylabel={North (NM)},
ymin=-25, ymax=20,
tick label style={font=\tiny},
label style={font=\scriptsize}]

\addplot [draw=none, draw=black, fill=black, colormap/viridis]
table [row sep=\\]{%
x                      y\\ 
+0.000000000000000e+00 -2.500000000000000e-01\\ 
+6.630077500000001e-02 -2.500000000000000e-01\\ 
+1.298949676962134e-01 -2.236584228970604e-01\\ 
+1.767766952966369e-01 -1.767766952966369e-01\\ 
+2.236584228970604e-01 -1.298949676962134e-01\\ 
+2.500000000000000e-01 -6.630077500000001e-02\\ 
+2.500000000000000e-01 +0.000000000000000e+00\\ 
+2.500000000000000e-01 +6.630077500000001e-02\\ 
+2.236584228970604e-01 +1.298949676962134e-01\\ 
+1.767766952966369e-01 +1.767766952966369e-01\\ 
+1.298949676962134e-01 +2.236584228970604e-01\\ 
+6.630077500000001e-02 +2.500000000000000e-01\\ 
+0.000000000000000e+00 +2.500000000000000e-01\\ 
-6.630077500000001e-02 +2.500000000000000e-01\\ 
-1.298949676962134e-01 +2.236584228970604e-01\\ 
-1.767766952966369e-01 +1.767766952966369e-01\\ 
-2.236584228970604e-01 +1.298949676962134e-01\\ 
-2.500000000000000e-01 +6.630077500000001e-02\\ 
-2.500000000000000e-01 +0.000000000000000e+00\\ 
-2.500000000000000e-01 -6.630077500000001e-02\\ 
-2.236584228970604e-01 -1.298949676962134e-01\\ 
-1.767766952966369e-01 -1.767766952966369e-01\\ 
-1.298949676962134e-01 -2.236584228970604e-01\\ 
-6.630077500000001e-02 -2.500000000000000e-01\\ 
+0.000000000000000e+00 -2.500000000000000e-01\\ 
+0.000000000000000e+00 -2.500000000000000e-01\\
};

\addplot graphics [includegraphics cmd=\pgfimage,xmin=-20, xmax=25, ymin=-25, ymax=20] {figures/jfk_13L_original_hist_1200.png};

\addplot[black, line width=0.75pt, forget plot] 
table[row sep=\\] {%
-0.314790350956774	-1.07388212440464 \\
0.701926179643114	0.634446106243 \\
};
\addplot [black, line width=0.75pt, forget plot]
table [row sep=\\]{%
0.381018579296101	-0.869410987002108 \\
1.0883680175494	0.318538875436125 \\
};
\addplot [black, line width=0.75pt, forget plot]
table [row sep=\\]{%
-0.527871725321972	1.06913560548559 \\
0.886517107609026	0.227348929251132 \\
};
\addplot [black, line width=0.75pt, forget plot]
table [row sep=\\]{%
-1.73637529142706	0.505844290966593 \\
0.315676001958449	-0.715917614289667 \\
};


\addplot [line width=1.2pt, blue, dotted, forget plot]
table [row sep=\\]{%
1.3121685420855	-28.3393554046402 \\
1.34665246291117	-28.2566449884263 \\
1.37899398638968	-28.1742007639849 \\
1.40925102931435	-28.0920242308766 \\
1.43748150847849	-28.0101168886619 \\
1.46374334067543	-27.9284802369017 \\
1.48809444269847	-27.8471157751563 \\
1.51059273134093	-27.7660250029866 \\
1.53129612339614	-27.685209419953 \\
1.55026253565741	-27.6046705256163 \\
1.56754988491805	-27.524409819537 \\
1.58321608797139	-27.4444288012759 \\
1.59731906161074	-27.3647289703934 \\
1.60991672262942	-27.2853118264503 \\
1.62106698782075	-27.2061788690071 \\
1.63082777397803	-27.1273315976246 \\
1.6392569978946	-27.0487715118632 \\
1.64641257636376	-26.9705001112837 \\
1.65235242617884	-26.8925188954466 \\
1.65713446413315	-26.8148293639127 \\
1.66081660702001	-26.7374330162424 \\
1.66345677163274	-26.6603313519965 \\
1.66511287476465	-26.5835258707356 \\
1.66584283320906	-26.5070180720203 \\
1.66509838912	-26.4306869630722 \\
1.66102412762519	-26.3541668601724 \\
1.65364670199167	-26.2774941834582 \\
1.64314657858279	-26.2007372128249 \\
1.62970422376189	-26.1239642281677 \\
1.61350010389232	-26.0472435093818 \\
1.59471468533742	-25.9706433363623 \\
1.57352843446056	-25.8942319890044 \\
1.55012181762506	-25.8180777472034 \\
1.52467530119428	-25.7422488908543 \\
1.49736935153157	-25.6668136998525 \\
1.46838443500028	-25.5918404540929 \\
1.43790101796374	-25.5173974334709 \\
1.40609956678532	-25.4435529178817 \\
1.37316054782835	-25.3703751872203 \\
1.33926442745618	-25.297932521382 \\
1.30459167203216	-25.2262932002619 \\
1.26932274791965	-25.1555255037554 \\
1.23363812148197	-25.0856977117574 \\
1.19771825908249	-25.0168781041632 \\
1.16174362708455	-24.9491349608681 \\
1.12589469185149	-24.8825365617671 \\
1.09035191974667	-24.8171511867555 \\
1.05508677207476	-24.7530173549948 \\
1.01756459924012	-24.6897891859991 \\
0.977135940507521	-24.6273306953381 \\
0.93398784282729	-24.5656181238047 \\
0.888307353149743	-24.5046277121915 \\
0.840281518425196	-24.4443357012915 \\
0.790097385603965	-24.3847183318973 \\
0.737942001636368	-24.3257518448017 \\
0.684002413472721	-24.2674124807975 \\
0.62846566806334	-24.2096764806775 \\
0.571518812358544	-24.1525200852344 \\
0.513348893308648	-24.095919535261 \\
0.454142957863969	-24.03985107155 \\
0.394088052974825	-23.9842909348943 \\
0.33337122559153	-23.9292153660866 \\
0.272179522664404	-23.8746006059197 \\
0.210699991143761	-23.8204228951863 \\
0.14911967797992	-23.7666584746792 \\
0.0876256301231961	-23.7132835851912 \\
0.0264048945239065	-23.660274467515 \\
-0.0343554818676316	-23.6076073624435 \\
-0.0944684521011018	-23.5552585107693 \\
-0.153746969226187	-23.5032041532853 \\
-0.212003986292571	-23.4514205307842 \\
-0.269985081068615	-23.400091357425 \\
-0.328531796845562	-23.3493899749813 \\
-0.387601034957418	-23.2992770656681 \\
-0.447149696738189	-23.2497133117002 \\
-0.507134683521882	-23.2006593952925 \\
-0.567512896642503	-23.1520759986601 \\
-0.628241237434058	-23.1039238040176 \\
-0.689276607230553	-23.0561634935802 \\
-0.750575907365995	-23.0087557495626 \\
-0.81209603917439	-22.9616612541798 \\
-0.873793903989743	-22.9148406896467 \\
-0.935626403146062	-22.8682547381782 \\
-0.997550437977353	-22.8218640819892 \\
-1.05952290981762	-22.7756294032945 \\
-1.12150072000087	-22.7295113843092 \\
-1.18344076986112	-22.6834707072481 \\
-1.24529996073236	-22.6374680543261 \\
-1.3070351939486	-22.5914641077581 \\
-1.36860337084385	-22.5454195497591 \\
-1.42996139275212	-22.4992950625438 \\
-1.49106616100741	-22.4530513283273 \\
-1.55187457694372	-22.4066490293245 \\
-1.61234354189507	-22.3600488477502 \\
-1.67249794115532	-22.3132997999578 \\
-1.73253804661121	-22.2666626817992 \\
-1.79248529167504	-22.2201478777196 \\
-1.85234501067647	-22.1737442097373 \\
-1.91212253794516	-22.1274404998703 \\
-1.97182320781078	-22.0812255701368 \\
-2.03145235460299	-22.0350882425551 \\
-2.09101531265145	-21.9890173391432 \\
-2.15051741628583	-21.9430016819194 \\
-2.2099639998358	-21.8970300929018 \\
-2.269360397631	-21.8510913941086 \\
-2.32871194400112	-21.805174407558 \\
-2.38802397327581	-21.7592679552681 \\
-2.44730181978473	-21.713360859257 \\
-2.50655081785755	-21.667441941543 \\
-2.56577630182393	-21.6215000241443 \\
-2.62498360601354	-21.5755239290789 \\
-2.68417806475603	-21.5295024783651 \\
-2.74336501238108	-21.483424494021 \\
-2.80254978321834	-21.4372787980648 \\
-2.86173771159748	-21.3910542125146 \\
-2.92093413184816	-21.3447395593887 \\
-2.98014437830004	-21.2983236607052 \\
-3.03937863704518	-21.2518130112638 \\
-3.09869873861381	-21.2054205837542 \\
-3.15810887216512	-21.1591884054218 \\
-3.21759217563114	-21.1130864135139 \\
-3.27713178694392	-21.0670845452777 \\
-3.33671084403551	-21.0211527379605 \\
-3.39631248483794	-20.9752609288097 \\
-3.45591984728327	-20.9293790550726 \\
-3.51551606930352	-20.8834770539964 \\
-3.57508428883075	-20.8375248628285 \\
-3.63460764379701	-20.7914924188162 \\
-3.69406927213432	-20.7453496592069 \\
-3.75345231177474	-20.6990665212477 \\
-3.81273990065031	-20.6526129421861 \\
-3.87191517669308	-20.6059588592692 \\
-3.93096127783507	-20.5590742097445 \\
-3.98986134200835	-20.5119289308593 \\
-4.04859850714495	-20.4644929598608 \\
-4.10715591117692	-20.4167362339964 \\
-4.1655166920363	-20.3686286905133 \\
-4.22366398765512	-20.320140266659 \\
-4.28158093596545	-20.2712408996806 \\
-4.33925067489931	-20.2219005268255 \\
-4.39665634238876	-20.172089085341 \\
-4.45396571993826	-20.1218240622015 \\
-4.51133777952682	-20.0711528991493 \\
-4.56874244479825	-20.0200904655832 \\
-4.62614963939638	-19.9686516309021 \\
-4.68352928696504	-19.9168512645049 \\
-4.74085131114804	-19.8647042357906 \\
-4.7980856355892	-19.812225414158 \\
-4.85520218393236	-19.7594296690061 \\
-4.91217087982133	-19.7063318697338 \\
-4.96896164689993	-19.6529468857399 \\
-5.025544408812	-19.5992895864235 \\
-5.08188908920134	-19.5453748411833 \\
-5.13796561171179	-19.4912175194184 \\
-5.19374389998716	-19.4368324905276 \\
-5.24919387767129	-19.3822346239099 \\
-5.30428546840798	-19.3274387889641 \\
-5.35898859584107	-19.2724598550891 \\
-5.41327318361438	-19.2173126916839 \\
-5.46710915537173	-19.1620121681474 \\
-5.52046643475694	-19.1065731538785 \\
-5.57331494541383	-19.0510105182761 \\
-5.62562461098624	-18.9953391307391 \\
-5.67736535511797	-18.9395738606664 \\
-5.72856842975324	-18.8836849338758 \\
-5.77941818551084	-18.8275392640926 \\
-5.82993191130702	-18.7711344880655 \\
-5.8801121683829	-18.7144792888773 \\
-5.92996151797959	-18.6575823496105 \\
-5.97948252133821	-18.6004523533479 \\
-6.02867773969986	-18.5430979831722 \\
-6.07754973430567	-18.4855279221661 \\
-6.12610106639674	-18.4277508534122 \\
-6.17433429721419	-18.3697754599932 \\
-6.22225198799912	-18.3116104249919 \\
-6.26985669999265	-18.2532644314908 \\
-6.3171509944359	-18.1947461625727 \\
-6.36413743256998	-18.1360643013202 \\
-6.410818575636	-18.0772275308161 \\
-6.45719698487506	-18.0182445341431 \\
-6.5032752215283	-17.9591239943837 \\
-6.54905584683681	-17.8998745946207 \\
-6.59454142204172	-17.8405050179368 \\
-6.63973450838413	-17.7810239474146 \\
-6.68463766710515	-17.7214400661369 \\
-6.72925345944591	-17.6617620571863 \\
-6.7735844466475	-17.6019986036455 \\
-6.81765646924161	-17.5421565227272 \\
-6.86175858101172	-17.4822160135552 \\
-6.9059360283774	-17.422165084331 \\
-6.95013690603731	-17.3619981498237 \\
-6.99430930869011	-17.3017096248028 \\
-7.03840133103448	-17.2412939240375 \\
-7.08236106776906	-17.180745462297 \\
-7.12613661359253	-17.1200586543507 \\
-7.16967606320354	-17.059227914968 \\
-7.21292751130075	-16.998247658918 \\
-7.25583905258283	-16.9371123009701 \\
-7.29835878174844	-16.8758162558936 \\
-7.34043479349624	-16.8143539384577 \\
-7.38201518252489	-16.7527197634319 \\
-7.42304804353306	-16.6909081455853 \\
-7.4634814712194	-16.6289134996873 \\
-7.50326356028258	-16.5667302405072 \\
-7.54234240542126	-16.5043527828143 \\
-7.5806661013341	-16.4417755413779 \\
-7.61818274271976	-16.3789929309672 \\
-7.65484042427691	-16.3159993663516 \\
-7.69058724070421	-16.2527892623004 \\
-7.72537128670032	-16.1893570335829 \\
-7.7591406569639	-16.1256970949683 \\
-7.79223691781884	-16.0617575116361 \\
-7.8250223175974	-15.9975011614777 \\
-7.85747597401081	-15.9329446783307 \\
-7.88957700477028	-15.8681046960323 \\
-7.92130452758704	-15.8029978484202 \\
-7.95263766017229	-15.7376407693318 \\
-7.98355552023726	-15.6720500926044 \\
-8.01403722549317	-15.6062424520756 \\
-8.04406189365123	-15.5402344815828 \\
-8.07360864242267	-15.4740428149635 \\
-8.1026565895187	-15.407684086055 \\
-8.13118485265055	-15.341174928695 \\
-8.15917254952942	-15.2745319767207 \\
-8.18659879786654	-15.2077718639697 \\
-8.21344271537313	-15.1409112242795 \\
-8.23968341976041	-15.0739666914874 \\
-8.2653000287396	-15.0069548994309 \\
-8.2902716600219	-14.9398924819475 \\
-8.31457743131855	-14.8727960728746 \\
-8.33819646034077	-14.8056823060497 \\
-8.36110786479976	-14.7385678153102 \\
-8.38329076240675	-14.6714692344936 \\
-8.40472427087296	-14.6044031974373 \\
-8.42551359864758	-14.5373347747467 \\
-8.44604817451377	-14.4701112608182 \\
-8.46633973315513	-14.4027326697359 \\
-8.48636850721769	-14.3352118360343 \\
-8.5061147293475	-14.267561594248 \\
-8.52555863219058	-14.1997947789114 \\
-8.54468044839299	-14.1319242245591 \\
-8.56346041060076	-14.0639627657258 \\
-8.58187875145993	-13.9959232369459 \\
-8.59991570361653	-13.9278184727539 \\
-8.61755149971661	-13.8596613076845 \\
-8.6347663724062	-13.7914645762722 \\
-8.65154055433135	-13.7232411130515 \\
-8.66785427813809	-13.655003752557 \\
-8.68368777647246	-13.5867653293232 \\
-8.6990212819805	-13.5185386778847 \\
-8.71383502730825	-13.450336632776 \\
-8.72810924510174	-13.3821720285317 \\
-8.74182416800702	-13.3140576996863 \\
-8.75496002867013	-13.2460064807744 \\
-8.7674970597371	-13.1780312063305 \\
-8.77941549385398	-13.1101447108891 \\
-8.79069556366679	-13.0423598289849 \\
-8.80136449783292	-12.9746837512008 \\
-8.81200336234838	-12.9070521439463 \\
-8.8227487651629	-12.8394434010893 \\
-8.83354680148152	-12.7718575431147 \\
-8.84434356650931	-12.7042945905077 \\
-8.85508515545132	-12.6367545637531 \\
-8.86571766351259	-12.5692374833361 \\
-8.8761871858982	-12.5017433697416 \\
-8.88643981781319	-12.4342722434548 \\
-8.89642165446262	-12.3668241249606 \\
-8.90607879105154	-12.299399034744 \\
-8.91535732278501	-12.2319969932901 \\
-8.92420334486809	-12.1646180210839 \\
-8.93256295250583	-12.0972621386105 \\
-8.94038224090328	-12.0299293663548 \\
-8.9476073052655	-11.9626197248018 \\
-8.95418424079755	-11.8953332344367 \\
-8.96005914270448	-11.8280699157445 \\
-8.96517810619135	-11.7608297892101 \\
-8.9694872264632	-11.6936128753185 \\
-8.97293259872511	-11.6264191945549 \\
-8.97546031818211	-11.5592487674043 \\
-8.97701648003928	-11.4921016143516 \\
-8.97754717950165	-11.4249777558819 \\
-8.97753100109496	-11.357806133597 \\
-8.97748268103005	-11.2905283745476 \\
-8.97740254203969	-11.223163527392 \\
-8.97729090685665	-11.1557306407886 \\
-8.97714809821369	-11.0882487633957 \\
-8.97697443884358	-11.0207369438718 \\
-8.97677025147907	-10.9532142308751 \\
-8.97653585885295	-10.8856996730641 \\
-8.97627158369797	-10.8182123190972 \\
-8.97597774874689	-10.7507712176325 \\
-8.97565467673248	-10.6833954173287 \\
-8.97530269038751	-10.6161039668439 \\
-8.97492211244474	-10.5489159148366 \\
-8.97451326563694	-10.4818503099651 \\
-8.97407647269688	-10.4149262008878 \\
-8.9736120563573	-10.348162636263 \\
-8.97312033935099	-10.2815786647492 \\
-8.97260164441071	-10.2151933350047 \\
-8.97205629426921	-10.1490256956877 \\
-8.97148461165928	-10.0830947954568 \\
-8.97088691931366	-10.0174196829703 \\
-8.97026353996513	-9.95201940688646 \\
-8.96961479634646	-9.88691301586374 \\
-8.96872984300487	-9.82208881393346 \\
-8.96694385364532	-9.75745564249083 \\
-8.96424919181348	-9.69300305511979 \\
-8.96069185900442	-9.62872788007195 \\
-8.95631785671321	-9.56462694559898 \\
-8.9511731864349	-9.50069707995252 \\
-8.94530384966457	-9.43693511138421 \\
-8.93875584789729	-9.3733378681457 \\
-8.93157518262811	-9.30990217848863 \\
-8.92380785535211	-9.24662487066466 \\
-8.91549986756436	-9.18350277292542 \\
-8.90669722075991	-9.12053271352256 \\
-8.89744591643383	-9.05771152070773 \\
-8.88779195608119	-8.99503602273257 \\
-8.87778134119706	-8.93250304784873 \\
-8.86746007327651	-8.87010942430785 \\
-8.85687415381459	-8.80785198036159 \\
-8.84606958430637	-8.74572754426158 \\
-8.83509236624693	-8.68373294425947 \\
-8.82398850113132	-8.6218650086069 \\
-8.81280399045462	-8.56012056555553 \\
-8.80158483571189	-8.498496443357 \\
-8.79037703839819	-8.43698947026295 \\
-8.77917736548895	-8.37559730584816 \\
-8.76738481678902	-8.31432865367086 \\
-8.75482672779342	-8.25318700510613 \\
-8.741525633842	-8.192172793263 \\
-8.72750407027465	-8.13128645125048 \\
-8.71278457243123	-8.0705284121776 \\
-8.69738967565161	-8.00989910915338 \\
-8.68134191527564	-7.94939897528685 \\
-8.66466382664321	-7.88902844368703 \\
-8.64737794509418	-7.82878794746294 \\
-8.62950680596841	-7.76867791972362 \\
-8.61107294460578	-7.70869879357808 \\
-8.59209889634614	-7.64885100213534 \\
-8.57260719652937	-7.58913497850443 \\
-8.55262038049534	-7.52955115579438 \\
-8.53216098358391	-7.47009996711421 \\
-8.51125154113495	-7.41078184557294 \\
-8.48991458848833	-7.3515972242796 \\
-8.46817266098391	-7.29254653634321 \\
-8.44604829396156	-7.23363021487279 \\
-8.42356402276115	-7.17484869297736 \\
-8.40074238272255	-7.11620240376597 \\
-8.37760590918563	-7.05769178034761 \\
};
\addplot [line width=1.2pt, blue, dotted, forget plot]
table [row sep=\\]{%
29.1627359289825	0.0678998390251722 \\
29.0372550293976	-0.0182133345106331 \\
28.9119040171828	-0.104432436909519 \\
28.7866831035281	-0.190757282483846 \\
28.6615924996238	-0.277187685545976 \\
28.5366324166601	-0.36372346040827 \\
28.4118030658272	-0.450364421383089 \\
28.2871046583153	-0.537110382782795 \\
28.1625374053146	-0.623961158919749 \\
28.0381015180153	-0.710916564106313 \\
27.9137972076077	-0.797976412654846 \\
27.7896246852819	-0.885140518877712 \\
27.6655841622282	-0.97240869708727 \\
27.5416758496367	-1.05978076159588 \\
27.4178999586977	-1.14725652671591 \\
27.2942567006014	-1.23483580675972 \\
27.170746286538	-1.32251841603966 \\
27.0473689276977	-1.4103041688681 \\
26.9241248352706	-1.49819287955741 \\
26.8010142204472	-1.58618436241993 \\
26.6780372944174	-1.67427843176804 \\
26.5551942683716	-1.7624749019141 \\
26.4324853534999	-1.85077358717046 \\
26.3099107609926	-1.93917430184948 \\
26.1874839124543	-2.02771863731816 \\
26.0652439077491	-2.11653101788763 \\
25.9431856849037	-2.20559803142499 \\
25.8213007175339	-2.29489537491651 \\
25.6995804792558	-2.38439874534847 \\
25.5780164436855	-2.47408383970717 \\
25.4566000844389	-2.56392635497888 \\
25.335322875132	-2.65390198814988 \\
25.2141762893809	-2.74398643620646 \\
25.0931518008015	-2.8341553961349 \\
24.9722408830099	-2.92438456492148 \\
24.8514350096221	-3.01464963955248 \\
24.7307256542541	-3.10492631701419 \\
24.6101042905219	-3.19519029429289 \\
24.4895623920415	-3.28541726837485 \\
24.369091432429	-3.37558293624637 \\
24.2486828853003	-3.46566299489373 \\
24.1283282242714	-3.5556331413032 \\
24.0080189229584	-3.64546907246107 \\
23.8877464549773	-3.73514648535363 \\
23.7675022939441	-3.82464107696714 \\
23.6472779134748	-3.91392854428791 \\
23.5270647871854	-4.0029845843022 \\
23.4068640568114	-4.09180451258722 \\
23.2867912520231	-4.18062009895263 \\
23.1668714267321	-4.26948873706547 \\
23.0470905549977	-4.35838960401449 \\
22.9274346108791	-4.44730187688841 \\
22.8078895684355	-4.536204732776 \\
22.6884414017262	-4.62507734876599 \\
22.5690760848103	-4.71389890194712 \\
22.4497795917472	-4.80264856940814 \\
22.330537896596	-4.89130552823779 \\
22.211336973416	-4.97984895552481 \\
22.0921627962663	-5.06825802835795 \\
21.9730013392063	-5.15651192382594 \\
21.8538385762952	-5.24458981901754 \\
21.7346604815921	-5.33247089102149 \\
21.6154530291564	-5.42013431692652 \\
21.4962021930472	-5.50755927382138 \\
21.3768939473238	-5.59472493879481 \\
21.2575142660455	-5.68161048893556 \\
21.1380491232713	-5.76819510133237 \\
21.0184844930607	-5.85445795307398 \\
20.8988063494727	-5.94037822124914 \\
20.7790006665667	-6.02593508294658 \\
20.6590534184019	-6.11110771525506 \\
20.5390307929379	-6.19605538823187 \\
20.4190041121013	-6.28093548578476 \\
20.2989670509017	-6.36572435577811 \\
20.1789132843488	-6.4503983460763 \\
20.0588364874521	-6.53493380454373 \\
19.9387303352212	-6.61930707904479 \\
19.8185885026659	-6.70349451744384 \\
19.6984046647956	-6.78747246760529 \\
19.57817249662	-6.87121727739353 \\
19.4578856731487	-6.95470529467293 \\
19.3375378693914	-7.03791286730788 \\
19.2171227603576	-7.12081634316278 \\
19.0966340210569	-7.203392070102 \\
18.976065326499	-7.28561639598994 \\
18.8554103516935	-7.36746566869097 \\
18.73466277165	-7.4489162360695 \\
18.6138162613781	-7.5299444459899 \\
18.4928644958874	-7.61052664631656 \\
18.3718011501876	-7.69063918491387 \\
18.2506198992882	-7.77025840964622 \\
18.1293144181989	-7.84936066837798 \\
18.0078783819292	-7.92792230897356 \\
17.8863054654889	-8.00591967929733 \\
17.764603227678	-8.08335225745426 \\
17.6428125557116	-8.16029051462297 \\
17.5209350261056	-8.23675675529655 \\
17.3989688280236	-8.31276825940925 \\
17.2769121506293	-8.3883423068953 \\
17.1547631830863	-8.46349617768894 \\
17.0325201145583	-8.5382471517244 \\
16.910181134209	-8.61261250893595 \\
16.7877444312021	-8.68660952925781 \\
16.6652081947011	-8.76025549262421 \\
16.5425706138697	-8.83356767896941 \\
16.4198298778717	-8.90656336822765 \\
16.2969841758706	-8.97925984033316 \\
16.1740316970302	-9.05167437522018 \\
16.050970630514	-9.12382425282295 \\
15.9277991654857	-9.19572675307572 \\
15.8045154911091	-9.26739915591272 \\
15.6811177965477	-9.3388587412682 \\
15.5576042709653	-9.41012278907639 \\
15.4339731035254	-9.48120857927153 \\
15.3102224833918	-9.55213339178787 \\
15.1863505997281	-9.62291450655965 \\
15.0623556416979	-9.6935692035211 \\
14.9382368054141	-9.76411776645616 \\
14.8140056654396	-9.83460468657133 \\
14.6896653607963	-9.90502226303158 \\
14.5652150764746	-9.97534781990961 \\
14.4406539974648	-10.0455586812781 \\
14.3159813087572	-10.1156321712099 \\
14.1911961953421	-10.1855456137776 \\
14.0662978422099	-10.255276333054 \\
13.9412854343507	-10.3248016531117 \\
13.816158156755	-10.3940988980236 \\
13.690915194413	-10.4631453918622 \\
13.5655557323151	-10.5319184587004 \\
13.4400789554515	-10.6003954226109 \\
13.3144840488125	-10.6685536076663 \\
13.1887701973885	-10.7363703379394 \\
13.0629365861697	-10.803822937503 \\
12.9369824001465	-10.8708887304297 \\
12.8109068243092	-10.9375450407922 \\
12.684709043648	-11.0037691926633 \\
12.5583882431533	-11.0695385101157 \\
12.4319436078154	-11.1348303172221 \\
12.3053743226246	-11.1996219380553 \\
12.1786795725712	-11.2638906966878 \\
12.0518585426454	-11.3276139171926 \\
11.9249150623875	-11.3911372603397 \\
11.7978530192223	-11.4547764245914 \\
11.670671684968	-11.5184637905582 \\
11.5433703314425	-11.5821317388508 \\
11.4159482304636	-11.6457126500799 \\
11.2884046538492	-11.7091389048561 \\
11.1607388734174	-11.7723428837901 \\
11.032950160986	-11.8352569674925 \\
10.905037788373	-11.8978135365739 \\
10.7770010273962	-11.959944971645 \\
10.6488391498736	-12.0215836533164 \\
10.5205514276232	-12.0826619621988 \\
10.3921371324628	-12.1431122789029 \\
10.2635955362103	-12.2028669840392 \\
10.1349259106838	-12.2618584582184 \\
10.0061275277011	-12.3200190820512 \\
9.87719965908013	-12.3772812361482 \\
9.74814157663882	-12.4335773011201 \\
9.61895255219512	-12.4888396575774 \\
9.48963185756692	-12.5430006861309 \\
9.36017876457217	-12.5959927673912 \\
9.23059254502878	-12.6477482819689 \\
9.10087247075467	-12.6981996104747 \\
8.97100494512446	-12.7474845434842 \\
8.84095147179198	-12.7962263538505 \\
8.71071670053099	-12.8444581915718 \\
8.58030866039716	-12.8921627865491 \\
8.44973538044614	-12.9393228686836 \\
8.3190048897336	-12.9859211678764 \\
8.18812521731519	-13.0319404140285 \\
8.05710439224658	-13.0773633370411 \\
7.92595044358342	-13.1221726668153 \\
7.79467140038137	-13.1663511332522 \\
7.6632752916961	-13.2098814662528 \\
7.53177014658326	-13.2527463957184 \\
7.40016399409852	-13.2949286515499 \\
7.26846486329753	-13.3364109636485 \\
7.13668078323596	-13.3771760619152 \\
7.00481978296946	-13.4172066762513 \\
6.8728898915537	-13.4564855365578 \\
6.74089913804433	-13.4949953727358 \\
6.60885555149702	-13.5327189146864 \\
6.47676716096742	-13.5696388923106 \\
6.3446419955112	-13.6057380355097 \\
6.21248808418401	-13.6409990741847 \\
6.08031345604151	-13.6754047382367 \\
5.94811694447869	-13.7089932000695 \\
5.81578857139419	-13.7424497204137 \\
5.68330438455059	-13.7759218563541 \\
5.55067759846118	-13.8093308395385 \\
5.41792142763925	-13.8425979016151 \\
5.2850490865981	-13.8756442742319 \\
5.15207378985101	-13.9083911890367 \\
5.01900875191127	-13.9407598776777 \\
4.88586718729217	-13.9726715718028 \\
4.752662310507	-14.00404750306 \\
4.61940733606905	-14.0348089030972 \\
4.48611547849161	-14.0648770035626 \\
4.35279995228798	-14.0941730361041 \\
4.21947397197143	-14.1226182323697 \\
4.08615075205526	-14.1501338240074 \\
3.95284350705276	-14.1766410426651 \\
3.81956545147722	-14.202061119991 \\
3.68632979984193	-14.2263152876329 \\
3.55314976666018	-14.2493247772389 \\
3.42003856644526	-14.2710108204569 \\
3.28700941371045	-14.291294648935 \\
3.15407552296906	-14.3100974943212 \\
3.02125010873436	-14.3273405882635 \\
2.88854638551965	-14.3429451624098 \\
2.75586141806927	-14.3577235549759 \\
2.62309464962629	-14.3725011340632 \\
2.49026286245212	-14.3872186761482 \\
2.35738283880815	-14.4018169577075 \\
2.22447136095579	-14.4162367552175 \\
2.09154521115643	-14.4304188451547 \\
1.95862117167148	-14.4443040039956 \\
1.82571602476233	-14.4578330082168 \\
1.69284655269039	-14.4709466342947 \\
1.56002953771705	-14.4835856587058 \\
1.42728176210371	-14.4956908579267 \\
1.29462000811178	-14.5072030084338 \\
1.16206105800265	-14.5180628867037 \\
1.02962169403772	-14.5282112692128 \\
0.897318698478393	-14.5375889324376 \\
0.765168853586068	-14.5461366528547 \\
0.633188941622146	-14.5537952069406 \\
0.501395744848025	-14.5605053711717 \\
0.369806045525106	-14.5662079220246 \\
0.238436625914788	-14.5708436359757 \\
0.10730426827847	-14.5743532895016 \\
-0.0235742451224469	-14.5766776590788 \\
-0.154182132026564	-14.5777575211837 \\
-0.284634580420856	-14.5775331522587 \\
-0.415323599978307	-14.5759934525672 \\
-0.5462227648892	-14.5731845412411 \\
-0.677271852326812	-14.5691546523769 \\
-0.808410639464418	-14.5639520200712 \\
-0.939578903475293	-14.5576248784205 \\
-1.07071642153272	-14.5502214615211 \\
-1.20176297080996	-14.5417900034696 \\
-1.3326583284803	-14.5323787383625 \\
-1.46334227171702	-14.5220359002963 \\
-1.59375457769338	-14.5108097233675 \\
-1.72383502358267	-14.4987484416725 \\
-1.85352338655816	-14.4859002893079 \\
-1.98275944379313	-14.4723135003701 \\
-2.11148297246085	-14.4580363089556 \\
-2.2396337497346	-14.4431169491609 \\
-2.36715155278766	-14.4276036550825 \\
-2.49397615879329	-14.4115446608168 \\
-2.62004734492479	-14.3949882004605 \\
-2.74530488835541	-14.3779825081099 \\
-2.86968856625845	-14.3605758178615 \\
-2.99313815580717	-14.3428163638119 \\
-3.11559343417485	-14.3247523800574 \\
-3.23708580885719	-14.3062198900188 \\
-3.35872131710877	-14.2845530185266 \\
-3.4807283382982	-14.2590906242261 \\
-3.60296249697681	-14.2300315658461 \\
-3.72527941769596	-14.1975747021157 \\
-3.84753472500701	-14.1619188917638 \\
-3.96958404346131	-14.1232629935193 \\
-4.09128299761019	-14.0818058661114 \\
-4.21248721200503	-14.0377463682688 \\
-4.33305231119716	-13.9912833587207 \\
-4.45283391973795	-13.9426156961959 \\
-4.57168766217873	-13.8919422394234 \\
-4.68946916307086	-13.8394618471322 \\
-4.8060340469657	-13.7853733780512 \\
-4.92123793841459	-13.7298756909094 \\
-5.03493646196889	-13.6731676444359 \\
-5.14698524217994	-13.6154480973594 \\
-5.2572399035991	-13.5569159084091 \\
-5.36555607077772	-13.4977699363139 \\
-5.47178936826715	-13.4382090398027 \\
-5.57579542061874	-13.3784320776045 \\
-5.67742985238384	-13.3186379084483 \\
-5.77654828811381	-13.2590253910631 \\
-5.87300635235999	-13.1997933841777 \\
-5.96862158140325	-13.1396717346192 \\
-6.06510524586727	-13.0773226724802 \\
-6.16215357181764	-13.0128436313883 \\
-6.25946278531993	-12.946332044971 \\
-6.35672911243976	-12.877885346856 \\
-6.45364877924269	-12.8076009706709 \\
-6.54991801179433	-12.7355763500432 \\
-6.64523303616024	-12.6619089186004 \\
-6.73929007840603	-12.5866961099703 \\
-6.83178536459729	-12.5100353577803 \\
-6.92241512079959	-12.4320240956581 \\
-7.01087557307853	-12.3527597572311 \\
-7.09686294749969	-12.2723397761271 \\
-7.18007347012866	-12.1908615859736 \\
-7.26020336703103	-12.1084226203981 \\
-7.33694886427239	-12.0251203130283 \\
-7.41000618791832	-11.9410520974918 \\
-7.47907156403441	-11.856315407416 \\
-7.54384121868626	-11.7710076764287 \\
-7.60401137793944	-11.6852263381573 \\
-7.65927826785954	-11.5990688262294 \\
-7.70933811451216	-11.5126325742728 \\
-7.75388714396288	-11.4260150159148 \\
-7.79393361774861	-11.3388851257531 \\
-7.83351282430557	-11.2499593118333 \\
-7.87280788639547	-11.1592522015092 \\
-7.91167691579575	-11.0668859955428 \\
-7.94997802428384	-10.972982894696 \\
-7.98756932363718	-10.8776650997308 \\
-8.0243089256332	-10.781054811409 \\
-8.06005494204934	-10.6832742304927 \\
-8.09466548466303	-10.5844455577438 \\
-8.12799866525171	-10.4846909939242 \\
-8.15991259559281	-10.3841327397959 \\
-8.19026538746377	-10.2828929961207 \\
-8.21891515264202	-10.1810939636608 \\
-8.245720002905	-10.0788578431779 \\
-8.27053805003015	-9.976306835434 \\
-8.29322740579489	-9.8735631411911 \\
-8.31364618197666	-9.77074896121112 \\
-8.3316524903529	-9.66798649625599 \\
-8.34710444270105	-9.56539794708766 \\
-8.35986015079853	-9.46310551446807 \\
-8.36977772642278	-9.36123139915915 \\
-8.37671528135125	-9.25989780192286 \\
-8.38053092736135	-9.15922692352114 \\
-8.38125238635882	-9.0592872677433 \\
-8.38096007107669	-8.95946679919252 \\
-8.38024481283058	-8.85955988225643 \\
-8.37902334474498	-8.75956662533435 \\
-8.37721239994441	-8.65948713682567 \\
-8.37472871155337	-8.55932152512972 \\
-8.37148901269635	-8.45906989864587 \\
-8.36741003649788	-8.35873236577347 \\
-8.36240851608244	-8.25830903491187 \\
-8.35640118457453	-8.15780001446043 \\
-8.34930477509868	-8.0572054128185 \\
-8.34103602077937	-7.95652533838544 \\
-8.33151165474112	-7.85575989956059 \\
-8.32064841010842	-7.75490920474333 \\
-8.30836302000578	-7.653973362333 \\
-8.2945722175577	-7.55295248072894 \\
-8.27919273588869	-7.45184666833053 \\
-8.26214130812325	-7.35065603353712 \\
-8.24333466738589	-7.24938068474805 \\
-8.2226895468011	-7.14802073036268 \\
-8.20012267949339	-7.04657627878037 \\
-8.17555079858727	-6.94504743840048 \\
-8.14889063720723	-6.84343431762234 \\
};
\addplot [line width=1.2pt, blue, dotted, forget plot]
table [row sep=\\]{%
9.55281788482596	-4.72836718000976 \\
9.5462922431973	-4.82925976119607 \\
9.53915194110472	-4.92996877146037 \\
9.53141683256985	-5.03049366637246 \\
9.52310677161429	-5.13083390150216 \\
9.51424161225967	-5.23098893241926 \\
9.5048412085276	-5.33095821469356 \\
9.49492541443968	-5.43074120389487 \\
9.48451408401755	-5.530337355593 \\
9.4736270712828	-5.62974612535774 \\
9.46228423025704	-5.72896696875889 \\
9.45050541496191	-5.82799934136627 \\
9.438310479419	-5.92684269874967 \\
9.42571927764994	-6.0254964964789 \\
9.41275166367634	-6.12396019012376 \\
9.3994274915198	-6.22223323525405 \\
9.38576661520195	-6.32031508743957 \\
9.3717888887444	-6.41820520225014 \\
9.35751416616876	-6.51590303525554 \\
9.34296230149665	-6.61340804202559 \\
9.32815314874967	-6.71071967813009 \\
9.31310656194945	-6.80783739913883 \\
9.2978423951176	-6.90476066062163 \\
9.28238050227572	-7.00148891814829 \\
9.2665647244015	-7.09803139416169 \\
9.24985355805173	-7.19441763135466 \\
9.23224550603446	-7.29064601796055 \\
9.21378350181582	-7.38671243428586 \\
9.19451047886196	-7.48261276063709 \\
9.17446937063902	-7.57834287732074 \\
9.15370311061314	-7.67389866464331 \\
9.13225463225047	-7.76927600291129 \\
9.11016686901716	-7.86447077243118 \\
9.08748275437933	-7.95947885350949 \\
9.06424522180314	-8.0542961264527 \\
9.04049720475474	-8.14891847156732 \\
9.01628163670025	-8.24334176915986 \\
8.99164145110583	-8.3375618995368 \\
8.96661958143762	-8.43157474300464 \\
8.94125896116176	-8.52537617986989 \\
8.91560252374439	-8.61896209043904 \\
8.88969320265166	-8.71232835501859 \\
8.86357393134972	-8.80547085391504 \\
8.83728764330469	-8.89838546743489 \\
8.81087727198274	-8.99106807588464 \\
8.78438575084999	-9.08351455957078 \\
8.75785601337259	-9.17572079879981 \\
8.73131065584512	-9.26770506740452 \\
8.70451238872539	-9.35974169038839 \\
8.67737329748747	-9.4518765248365 \\
8.64988285634724	-9.54406268594266 \\
8.62203053952056	-9.6362532889007 \\
8.59380582122334	-9.72840144890448 \\
8.56519817567143	-9.82046028114782 \\
8.53619707708074	-9.91238290082456 \\
8.50679199966713	-10.0041224231285 \\
8.4769724176465	-10.0956319632536 \\
8.44672780523472	-10.1868646363935 \\
8.41604763664767	-10.2777735577422 \\
8.38492138610123	-10.3683118424935 \\
8.35333852781129	-10.4584326058411 \\
8.32128853599373	-10.5480889629791 \\
8.28876088486443	-10.6372340291011 \\
8.25574504863926	-10.725820919401 \\
8.22223050153411	-10.8138027490727 \\
8.18820671776487	-10.9011326333099 \\
8.15366317154741	-10.9877636873066 \\
8.11858933709761	-11.0736490262566 \\
8.08297468863136	-11.1587417653536 \\
8.04680870036453	-11.2429950197916 \\
8.01008084651301	-11.3263619047643 \\
7.97264938413752	-11.4091685989897 \\
7.93439106944188	-11.491746800181 \\
7.8953231253567	-11.5740579020251 \\
7.85546277481256	-11.656063298209 \\
7.81482724074006	-11.7377243824196 \\
7.77343374606978	-11.8190025483439 \\
7.73129951373232	-11.8998591896687 \\
7.68844176665825	-11.9802557000809 \\
7.64487772777819	-12.0601534732676 \\
7.6006246200227	-12.1395139029156 \\
7.55569966632238	-12.2182983827119 \\
7.51012008960783	-12.2964683063435 \\
7.46390311280962	-12.3739850674971 \\
7.41706595885836	-12.4508100598598 \\
7.36962585068463	-12.5269046771185 \\
7.32160001121902	-12.6022303129601 \\
7.27300566339211	-12.6767483610715 \\
7.22386003013451	-12.7504202151398 \\
7.17418033437679	-12.8232072688517 \\
7.12398379904956	-12.8950709158943 \\
7.07328764708339	-12.9659725499544 \\
7.02210910140888	-13.035873564719 \\
6.97046538495661	-13.1047353538751 \\
6.91829183396697	-13.1727488191346 \\
6.865333398662	-13.2406051071886 \\
6.81157381145319	-13.3082887548062 \\
6.75701703078596	-13.3757262758275 \\
6.70166701510576	-13.4428441840921 \\
6.64552772285801	-13.5095689934399 \\
6.58860311248814	-13.5758272177108 \\
6.53089714244158	-13.6415453707445 \\
6.47241377116376	-13.706649966381 \\
6.41315695710011	-13.7710675184601 \\
6.35313065869606	-13.8347245408216 \\
6.29233883439705	-13.8975475473053 \\
6.23078544264851	-13.9594630517511 \\
6.16847444189585	-14.0203975679989 \\
6.10540979058452	-14.0802776098884 \\
6.04159544715995	-14.1390296912595 \\
5.97703537006756	-14.1965803259521 \\
5.91173351775278	-14.252856027806 \\
5.84569384866105	-14.307783310661 \\
5.7789203212378	-14.3612886883569 \\
5.71141689392845	-14.4132986747336 \\
5.64318752517844	-14.463739783631 \\
5.57423617343319	-14.5125385288889 \\
5.5045182319819	-14.5597059589447 \\
5.43342813120073	-14.6062701133677 \\
5.36084375316652	-14.6525036942612 \\
5.28684269532164	-14.6983419093277 \\
5.21150255510845	-14.7437199662699 \\
5.13490092996931	-14.7885730727901 \\
5.05711541734658	-14.8328364365909 \\
4.97822361468261	-14.8764452653749 \\
4.89830311941978	-14.9193347668445 \\
4.81743152900044	-14.9614401487023 \\
4.73568644086695	-15.0026966186508 \\
4.65314545246168	-15.0430393843925 \\
4.56988616122699	-15.0824036536299 \\
4.48598616460523	-15.1207246340656 \\
4.40152306003877	-15.1579375334021 \\
4.31657444496997	-15.1939775593419 \\
4.23121791684119	-15.2287799195874 \\
4.14553107309479	-15.2622798218413 \\
4.05959151117314	-15.2944124738061 \\
3.97347682851858	-15.3251130831843 \\
3.8872646225735	-15.3543168576783 \\
3.80103249078024	-15.3819590049907 \\
3.71485803058116	-15.4079747328241 \\
3.62881883941864	-15.432299248881 \\
3.54248133018063	-15.4559062853192 \\
3.45539122437712	-15.4797360185261 \\
3.36759508238351	-15.5036733225479 \\
3.27913946457518	-15.5276030714313 \\
3.19007093132753	-15.5514101392224 \\
3.10043604301595	-15.5749793999679 \\
3.01028136001583	-15.598195727714 \\
2.91965344270256	-15.6209439965072 \\
2.82859885145154	-15.643109080394 \\
2.73716414663815	-15.6645758534206 \\
2.64539588863778	-15.6852291896336 \\
2.55334063782583	-15.7049539630793 \\
2.46104495457769	-15.7236350478042 \\
2.36855539926874	-15.7411573178547 \\
2.27591853227439	-15.7574056472772 \\
2.18318091397002	-15.7722649101181 \\
2.09038910473103	-15.7856199804238 \\
1.9975896649328	-15.7973557322407 \\
1.90482915495072	-15.8073570396152 \\
1.8121541351602	-15.8155087765939 \\
1.71961116593661	-15.821695817223 \\
1.62724680765535	-15.825803035549 \\
1.53510762069182	-15.8277153056182 \\
1.44312008508233	-15.8276668589769 \\
1.3509291421302	-15.8267464515174 \\
1.25852725322945	-15.8250714530014 \\
1.16593644554006	-15.8226746326145 \\
1.07317874622199	-15.8195887595425 \\
0.980276182435225	-15.815846602971 \\
0.887250781339743	-15.8114809320857 \\
0.79412457009552	-15.8065245160723 \\
0.70091957586253	-15.8010101241164 \\
0.60765782580075	-15.7949705254038 \\
0.514361347070156	-15.7884384891202 \\
0.421052166830724	-15.7814467844511 \\
0.327752312242431	-15.7740281805823 \\
0.234483810465253	-15.7662154466995 \\
0.141268688659165	-15.7580413519883 \\
0.048128973984145	-15.7495386656345 \\
-0.0449133063998324	-15.7407401568236 \\
-0.13783612533279	-15.7316785947414 \\
-0.230617455654752	-15.7223867485736 \\
-0.323235270205743	-15.7128973875058 \\
-0.415667541825785	-15.7032432807237 \\
-0.507892243354904	-15.6934571974131 \\
-0.599887347633122	-15.6835719067594 \\
-0.691653402094792	-15.6735301226016 \\
-0.78345947418067	-15.6622096449056 \\
-0.875366412840808	-15.6493322411294 \\
-0.967344174866564	-15.6349838376918 \\
-1.05936271704929	-15.6192503610115 \\
-1.15139199618035	-15.6022177375073 \\
-1.2434019690511	-15.583971893598 \\
-1.3353625924529	-15.5645987557024 \\
-1.42724382317709	-15.5441842502391 \\
-1.51901561801505	-15.522814303627 \\
-1.61064793375811	-15.5005748422849 \\
-1.70211072719766	-15.4775517926315 \\
-1.79337395512503	-15.4538310810855 \\
-1.88440757433159	-15.4294986340658 \\
-1.97518154160869	-15.4046403779911 \\
-2.06566581374769	-15.3793422392802 \\
-2.15583034753995	-15.3536901443518 \\
-2.24564509977682	-15.3277700196247 \\
-2.33508002724966	-15.3016677915177 \\
-2.42410508674983	-15.2754693864495 \\
-2.51269023506869	-15.2492607308389 \\
-2.60080542899759	-15.2231277511047 \\
-2.68842062532789	-15.1971563736657 \\
-2.77550578085094	-15.1714325249405 \\
-2.86223292943586	-15.1455858131203 \\
-2.94878212103344	-15.1191946821505 \\
-3.03513533655751	-15.0922683120812 \\
-3.12127455692191	-15.0648158829626 \\
-3.20718176304049	-15.0368465748449 \\
-3.2928389358271	-15.0083695677783 \\
-3.37822805619557	-14.9793940418129 \\
-3.46333110505974	-14.9499291769989 \\
-3.54813006333347	-14.9199841533866 \\
-3.63260691193059	-14.889568151026 \\
-3.71674363176495	-14.8586903499674 \\
-3.80052220375039	-14.827359930261 \\
-3.88392460880075	-14.7955860719568 \\
-3.96693282782988	-14.7633779551052 \\
-4.04952884175161	-14.7307447597563 \\
-4.1316946314798	-14.6976956659603 \\
-4.21341217792829	-14.6642398537673 \\
-4.29466346201091	-14.6303865032276 \\
-4.37543046464152	-14.5961447943912 \\
-4.45569516673395	-14.5615239073085 \\
-4.53543954920204	-14.5265330220295 \\
-4.61464559295965	-14.4911813186045 \\
-4.69329527892061	-14.4554779770837 \\
-4.77144266613143	-14.4193253315915 \\
-4.84930442294032	-14.3823968863607 \\
-4.92688146882105	-14.3446996456547 \\
-5.0041567210544	-14.3062677881561 \\
-5.0811130969211	-14.2671354925475 \\
-5.15773351370192	-14.2273369375113 \\
-5.2340008886776	-14.1869063017302 \\
-5.3098981391289	-14.1458777638866 \\
-5.38540818233657	-14.1042855026631 \\
-5.46051393558137	-14.0621636967422 \\
-5.53519831614405	-14.0195465248065 \\
-5.60944424130536	-13.9764681655386 \\
-5.68323462834605	-13.9329627976208 \\
-5.75655239454688	-13.8890645997359 \\
-5.82938045718861	-13.8448077505663 \\
-5.90170173355197	-13.8002264287946 \\
-5.97349914091774	-13.7553548131032 \\
-6.04475559656665	-13.7102270821748 \\
-6.11545401777947	-13.6648774146919 \\
-6.18557732183695	-13.619339989337 \\
-6.25510842601984	-13.5736489847927 \\
-6.32403024760888	-13.5278385797415 \\
-6.39232570388485	-13.481942952866 \\
-6.4600388222289	-13.4359898958465 \\
-6.52791779145875	-13.3899124567156 \\
-6.59609884498048	-13.3436627048227 \\
-6.6644664858564	-13.2972137326196 \\
-6.73290521714884	-13.250538632558 \\
-6.80129954192013	-13.2036104970898 \\
-6.86953396323257	-13.1564024186666 \\
-6.93749298414849	-13.1088874897402 \\
-7.00506110773021	-13.0610388027624 \\
-7.07212283704005	-13.0128294501848 \\
-7.13856267514033	-12.9642325244593 \\
-7.20426512509337	-12.9152211180376 \\
-7.26911468996149	-12.8657683233713 \\
-7.332995872807	-12.8158472329124 \\
-7.39579317669224	-12.7654309391125 \\
-7.45739110467951	-12.7144925344233 \\
-7.51767415983115	-12.6630051112966 \\
-7.57652684520946	-12.6109417621842 \\
-7.63383366387677	-12.5582755795378 \\
-7.6894791188954	-12.5049796558092 \\
-7.74334771332767	-12.45102708345 \\
-7.7953239502359	-12.3963909549121 \\
-7.84529233268241	-12.3410443626472 \\
-7.89313736372951	-12.284960399107 \\
-7.94016314281733	-12.2280151726835 \\
-7.98762853096421	-12.1701248530882 \\
-8.03535017359987	-12.1113226247523 \\
-8.08314471615402	-12.0516416721069 \\
-8.13082880405635	-11.9911151795832 \\
-8.17821908273658	-11.9297763316123 \\
-8.22513219762441	-11.8676583126253 \\
-8.27138479414955	-11.8047943070534 \\
-8.3167935177417	-11.7412174993277 \\
-8.36117501383057	-11.6769610738793 \\
-8.40434592784587	-11.6120582151394 \\
-8.4461229052173	-11.5465421075392 \\
-8.48632259137457	-11.4804459355097 \\
-8.52476163174739	-11.4138028834821 \\
-8.56125667176545	-11.3466461358875 \\
-8.59562435685847	-11.2790088771571 \\
-8.62768133245616	-11.210924291722 \\
-8.65724424398821	-11.1424255640134 \\
-8.68412973688434	-11.0735458784624 \\
-8.70815445657425	-11.0043184195001 \\
-8.72913504848765	-10.9347763715577 \\
-8.74688815805424	-10.8649529190662 \\
-8.76123043070373	-10.7948812464569 \\
-8.77274873548934	-10.7243841828287 \\
-8.78383742932838	-10.6528324972197 \\
-8.79466362114429	-10.5802518128333 \\
-8.80520484334307	-10.5067211912836 \\
-8.81543862833072	-10.4323196941848 \\
-8.82534250851323	-10.3571263831507 \\
-8.83489401629661	-10.2812203197954 \\
-8.84407068408685	-10.2046805657331 \\
-8.85285004428995	-10.1275861825777 \\
-8.86120962931191	-10.0500162319432 \\
-8.86912697155873	-9.97204977544376 \\
-8.87657960343641	-9.89376587469338 \\
-8.88354505735094	-9.8152435913061 \\
-8.89000086570833	-9.73656198689595 \\
-8.89592456091457	-9.65780012307698 \\
-8.90129367537566	-9.57903706146324 \\
-8.9060857414976	-9.50035186366876 \\
-8.91027829168639	-9.42182359130758 \\
-8.91384885834803	-9.34353130599376 \\
-8.91677497388852	-9.26555406934132 \\
-8.91903417071385	-9.18797094296431 \\
-8.92060398123003	-9.11086098847678 \\
-8.92146193784304	-9.03430326749276 \\
-8.92151776790133	-8.9583420719002 \\
-8.91989202907166	-8.88258105860152 \\
-8.91636756793364	-8.80688709070552 \\
-8.91100457404947	-8.73126026065732 \\
-8.90386323698135	-8.65570066090205 \\
-8.89500374629151	-8.58020838388484 \\
-8.88448629154213	-8.50478352205081 \\
-8.87237106229543	-8.42942616784509 \\
-8.85871824811361	-8.35413641371281 \\
-8.84358803855887	-8.2789143520991 \\
-8.82704062319344	-8.20376007544909 \\
-8.8091361915795	-8.1286736762079 \\
-8.78993493327928	-8.05365524682065 \\
-8.76949703785497	-7.97870487973249 \\
-8.74788269486878	-7.90382266738853 \\
-8.72515209388292	-7.8290087022339 \\
-8.70136542445959	-7.75426307671374 \\
-8.676582876161	-7.67958588327316 \\
-8.65086463854936	-7.6049772143573 \\
-8.62427090118686	-7.53043716241128 \\
-8.59686185363573	-7.45596581988023 \\
-8.56869768545817	-7.38156327920929 \\
-8.53983858621637	-7.30722963284356 \\
};
\addplot [line width=1.2pt, blue, dotted, forget plot]
table [row sep=\\]{%
4.63457073622982	-4.78537613812698 \\
4.64906636307585	-4.89600373864589 \\
4.66291405890851	-5.00640430905794 \\
4.67611502264962	-5.11657708123744 \\
4.68867045322098	-5.2265212870587 \\
4.70058154954443	-5.33623615839605 \\
4.71184951054177	-5.4457209271238 \\
4.72247553513481	-5.55497482511626 \\
4.73246082224539	-5.66399708424776 \\
4.74180657079531	-5.77278693639261 \\
4.75051397970638	-5.88134361342511 \\
4.75858424790043	-5.9896663472196 \\
4.76601857429926	-6.09775436965039 \\
4.77281815782471	-6.20560691259179 \\
4.77898419739857	-6.31322320791812 \\
4.78451789194268	-6.4206024875037 \\
4.78942044037884	-6.52774398322284 \\
4.79369304162886	-6.63464692694986 \\
4.79733689461458	-6.74131055055907 \\
4.80035319825779	-6.8477340859248 \\
4.80274315148033	-6.95391676492135 \\
4.80450795320399	-7.05985781942305 \\
4.80564880235061	-7.16555648130421 \\
4.80616689784199	-7.27101198243915 \\
4.80599330817813	-7.37623517567685 \\
4.80491836031602	-7.48126132103833 \\
4.80296187061307	-7.58608886679707 \\
4.80016187010839	-7.69071328638206 \\
4.79655638984108	-7.79513005322228 \\
4.79218346085026	-7.89933464074671 \\
4.78708111417502	-8.00332252238433 \\
4.78128738085448	-8.10708917156413 \\
4.77484029192774	-8.21063006171507 \\
4.76777787843391	-8.31394066626615 \\
4.7601381714121	-8.41701645864634 \\
4.7519592019014	-8.51985291228464 \\
4.74327900094094	-8.62244550061001 \\
4.73413559956982	-8.72478969705143 \\
4.72456702882713	-8.8268809750379 \\
4.714611319752	-8.92871480799839 \\
4.70430650338352	-9.03028666936188 \\
4.69369061076081	-9.13159203255735 \\
4.68280167292297	-9.23262637101379 \\
4.6716777209091	-9.33338515816018 \\
4.66035678575832	-9.43386386742549 \\
4.64887689850973	-9.53405797223871 \\
4.63727609020244	-9.63396294602882 \\
4.62549689595795	-9.73358781461074 \\
4.61234684527029	-9.83309810943368 \\
4.59751618983222	-9.93252437910521 \\
4.5810799013528	-10.0318414319399 \\
4.56311295154109	-10.1310240762523 \\
4.54369031210615	-10.2300471203571 \\
4.52288695475705	-10.3288853725689 \\
4.50077785120286	-10.4275136412021 \\
4.47743797315264	-10.5259067345715 \\
4.45294229231545	-10.6240394609917 \\
4.42736578040037	-10.7218866287771 \\
4.40078340911644	-10.8194230462424 \\
4.37327015017274	-10.9166235217023 \\
4.34490097527834	-11.0134628634712 \\
4.31575085614229	-11.1099158798638 \\
4.28589476447366	-11.2059573791946 \\
4.25540767198152	-11.3015621697784 \\
4.22436455037493	-11.3967050599296 \\
4.19284037136296	-11.4913608579628 \\
4.16091010665466	-11.5855043721927 \\
4.12864872795911	-11.6791104109339 \\
4.09613120698537	-11.7721537825008 \\
4.0634325154425	-11.8646092952082 \\
4.03062762503957	-11.9564517573706 \\
3.99703492254986	-12.0479573208913 \\
3.96196123211463	-12.1393872723111 \\
3.9254641814949	-12.2306941219014 \\
3.88760139845169	-12.3218303799333 \\
3.84843051074602	-12.412748556678 \\
3.80800914613892	-12.5034011624067 \\
3.76639493239139	-12.5937407073907 \\
3.72364549726447	-12.6837197019011 \\
3.67981846851918	-12.7732906562091 \\
3.63497147391654	-12.8624060805861 \\
3.58916214121756	-12.951018485303 \\
3.54244809818327	-13.0390803806312 \\
3.49488697257469	-13.1265442768419 \\
3.44653639215283	-13.2133626842063 \\
3.39745398467873	-13.2994881129955 \\
3.34769737791341	-13.3848730734809 \\
3.29732419961787	-13.4694700759335 \\
3.24639207755315	-13.5532316306245 \\
3.19495863948026	-13.6361102478253 \\
3.14308151316023	-13.718058437807 \\
3.09081832635408	-13.7990287108408 \\
3.03822670682282	-13.8789735771979 \\
2.98536428232748	-13.9578455471496 \\
2.93209463715405	-14.0359443420207 \\
2.8778278429028	-14.1143092079046 \\
2.82253715694822	-14.1928910536181 \\
2.76624367093887	-14.2715523172104 \\
2.70896847652332	-14.3501554367307 \\
2.65073266535014	-14.4285628502284 \\
2.59155732906788	-14.5066369957525 \\
2.53146355932512	-14.5842403113524 \\
2.47047244777042	-14.6612352350773 \\
2.40860508605234	-14.7374842049764 \\
2.34588256581945	-14.812849659099 \\
2.28232597872032	-14.8871940354942 \\
2.21795641640351	-14.9603797722115 \\
2.15279497051759	-15.0322693072998 \\
2.08686273271112	-15.1027250788086 \\
2.02018079463267	-15.1716095247871 \\
1.95277024793081	-15.2387850832844 \\
1.88465218425409	-15.3041141923498 \\
1.81584769525108	-15.3674592900326 \\
1.74637787257036	-15.428682814382 \\
1.67626380786048	-15.4876472034472 \\
1.60552659277001	-15.5442148952774 \\
1.53418731894751	-15.598248327922 \\
1.46220269512731	-15.6497458073967 \\
1.38878017681896	-15.7003528420318 \\
1.31375626697124	-15.7505278002263 \\
1.23723083562649	-15.800189775191 \\
1.15930375282706	-15.8492578601371 \\
1.08007488861529	-15.8976511482756 \\
0.999644113033541	-15.9452887328174 \\
0.918111296124153	-15.9920897069737 \\
0.835576307929476	-16.0379731639553 \\
0.75213901849186	-16.0828581969735 \\
0.66789929785365	-16.1266638992391 \\
0.582957016057197	-16.1693093639632 \\
0.497412043144848	-16.2107136843567 \\
0.411364249158952	-16.2507959536308 \\
0.324913504141856	-16.2894752649965 \\
0.23815967813591	-16.3266707116647 \\
0.151202641183461	-16.3623013868465 \\
0.0641422633268572	-16.3962863837529 \\
-0.0229215853915525	-16.4285447955949 \\
-0.10988903492942	-16.4589957155836 \\
-0.196660215244398	-16.4875582369299 \\
-0.283135256294137	-16.5141514528449 \\
-0.369214288036289	-16.5386944565396 \\
-0.454797440428506	-16.561106341225 \\
-0.54039421119232	-16.5824561312735 \\
-0.626544924484217	-16.6038094621936 \\
-0.713195949914276	-16.6250803439008 \\
-0.80029365709258	-16.6461827863107 \\
-0.887784415629209	-16.6670307993387 \\
-0.975614595134243	-16.6875383929003 \\
-1.06373056521776	-16.707619576911 \\
-1.15207869548985	-16.7271883612863 \\
-1.24060535556059	-16.7461587559416 \\
-1.32925691504006	-16.7644447707925 \\
-1.41797974353834	-16.7819604157545 \\
-1.50672021066551	-16.7986197007429 \\
-1.59542468603166	-16.8143366356734 \\
-1.68403953924686	-16.8290252304614 \\
-1.77251113992119	-16.8425994950224 \\
-1.86078585766474	-16.8549734392719 \\
-1.94881006208759	-16.8660610731253 \\
-2.03653012279981	-16.8757764064982 \\
-2.1238924094115	-16.8840334493061 \\
-2.21084329153272	-16.8907462114644 \\
-2.29732913877357	-16.8958287028886 \\
-2.38329632074411	-16.8991949334942 \\
-2.46869120705444	-16.9007589131967 \\
-2.55357330156898	-16.9006100886924 \\
-2.63827739814948	-16.8992962081935 \\
-2.72282211437074	-16.8969099065792 \\
-2.80719861376097	-16.8935025022839 \\
-2.89139805984839	-16.8891253137421 \\
-2.9754116161612	-16.8838296593883 \\
-3.05923044622762	-16.877666857657 \\
-3.14284571357587	-16.8706882269825 \\
-3.22624858173415	-16.8629450857993 \\
-3.30943021423067	-16.854488752542 \\
-3.39238177459366	-16.8453705456449 \\
-3.47509442635131	-16.8356417835425 \\
-3.55755933303185	-16.8253537846692 \\
-3.63976765816348	-16.8145578674596 \\
-3.72171056527442	-16.803305350348 \\
-3.80337921789289	-16.791647551769 \\
-3.88476477954708	-16.7796357901569 \\
-3.96585841376523	-16.7673213839462 \\
-4.04665128407553	-16.7547556515715 \\
-4.1271345540062	-16.7419899114671 \\
-4.20729938708545	-16.7290754820674 \\
-4.2871369468415	-16.716063681807 \\
-4.36663839680256	-16.7030058291203 \\
-4.44581697088637	-16.6898705394536 \\
-4.52494154430626	-16.6756356226961 \\
-4.60406280158993	-16.6600266800226 \\
-4.68314084486226	-16.643100532136 \\
-4.76213577624813	-16.6249139997392 \\
-4.84100769787243	-16.6055239035353 \\
-4.91971671186004	-16.5849870642271 \\
-4.99822292033585	-16.5633603025177 \\
-5.07648642542473	-16.5407004391099 \\
-5.15446732925157	-16.5170642947069 \\
-5.23212573394125	-16.4925086900114 \\
-5.30942174161865	-16.4670904457265 \\
-5.38631545440866	-16.4408663825552 \\
-5.46276697443617	-16.4138933212003 \\
-5.53873640382605	-16.3862280823649 \\
-5.61418384470318	-16.3579274867519 \\
-5.68906939919246	-16.3290483550643 \\
-5.76335316941876	-16.299647508005 \\
-5.83699525750696	-16.269781766277 \\
-5.90995576558196	-16.2395079505833 \\
-5.98219479576863	-16.2088828816267 \\
-6.05367245019185	-16.1779633801104 \\
-6.12434883097651	-16.1468062667371 \\
-6.1941840402475	-16.11546836221 \\
-6.26358383653422	-16.0833935441674 \\
-6.33293796738639	-16.050019559134 \\
-6.40218251845934	-16.0153940311419 \\
-6.47125357540838	-15.979564584223 \\
-6.54008722388885	-15.9425788424096 \\
-6.60861954955608	-15.9044844297336 \\
-6.67678663806537	-15.8653289702272 \\
-6.74452457507207	-15.8251600879224 \\
-6.8117694462315	-15.7840254068514 \\
-6.87845733719897	-15.7419725510462 \\
-6.94452433362983	-15.6990491445388 \\
-7.00990652117938	-15.6553028113614 \\
-7.07453998550296	-15.6107811755461 \\
-7.1383608122559	-15.5655318611249 \\
-7.20130508709351	-15.5196024921299 \\
-7.26330889567112	-15.4730406925932 \\
-7.32430832364407	-15.4258940865469 \\
-7.38423945666766	-15.378210298023 \\
-7.44303838039724	-15.3300369510537 \\
-7.50064118048812	-15.2814216696709 \\
-7.55698394259562	-15.2324120779069 \\
-7.61200275237508	-15.1830557997937 \\
-7.66563369548182	-15.1334004593633 \\
-7.71815648655937	-15.0832563120683 \\
-7.77060828501565	-15.031904898351 \\
-7.82297775637989	-14.979345872401 \\
-7.8751670301159	-14.9256384451245 \\
-7.92707823568748	-14.8708418274277 \\
-7.97861350255843	-14.8150152302169 \\
-8.02967496019257	-14.7582178643981 \\
-8.0801647380537	-14.7005089408777 \\
-8.12998496560562	-14.6419476705619 \\
-8.17903777231215	-14.5825932643569 \\
-8.22722528763709	-14.5225049331688 \\
-8.27444964104424	-14.4617418879039 \\
-8.32061296199742	-14.4003633394685 \\
-8.36561737996043	-14.3384284987687 \\
-8.40936502439707	-14.2759965767107 \\
-8.45175802477115	-14.2131267842008 \\
-8.49269851054648	-14.1498783321451 \\
-8.53208861118687	-14.08631043145 \\
-8.56983045615612	-14.0224822930215 \\
-8.60582617491803	-13.958453127766 \\
-8.63997789693642	-13.8942821465895 \\
-8.67218775167509	-13.8300285603984 \\
-8.70235786859785	-13.7657515800989 \\
-8.73052190109139	-13.7014616989707 \\
-8.75828970190309	-13.6365806702517 \\
-8.78608297208149	-13.5709599327015 \\
-8.813796795055	-13.5046447634539 \\
-8.84132625425204	-13.4376804396428 \\
-8.868566433101	-13.370112238402 \\
-8.89541241503028	-13.3019854368655 \\
-8.92175928346829	-13.233345312167 \\
-8.94750212184345	-13.1642371414405 \\
-8.97253601358415	-13.0947062018197 \\
-8.99675604211881	-13.0247977704387 \\
-9.02005729087582	-12.9545571244312 \\
-9.04233484328358	-12.8840295409311 \\
-9.06348378277052	-12.8132602970723 \\
-9.08339919276503	-12.7422946699886 \\
-9.10197615669551	-12.6711779368139 \\
-9.11910975799038	-12.5999553746821 \\
-9.13469508007804	-12.528672260727 \\
-9.14862720638689	-12.4573738720825 \\
-9.16080122034534	-12.3861054858825 \\
-9.1711122053818	-12.3149123792608 \\
-9.17945524492466	-12.2438398293513 \\
-9.18572542240234	-12.1729331132879 \\
-9.18981782124324	-12.1022375082044 \\
-9.19273754401742	-12.0315224616446 \\
-9.19555789685758	-11.9605434704216 \\
-9.19827626853611	-11.8893248855728 \\
-9.20089004782541	-11.8178910581356 \\
-9.20339662349787	-11.7462663391476 \\
-9.20579338432587	-11.6744750796461 \\
-9.20807771908181	-11.6025416306687 \\
-9.21024701653808	-11.5304903432528 \\
-9.21229866546707	-11.458345568436 \\
-9.21423005464117	-11.3861316572556 \\
-9.21603857283278	-11.3138729607492 \\
-9.21772160881427	-11.2415938299542 \\
-9.21927655135805	-11.1693186159081 \\
-9.22070078923651	-11.0970716696483 \\
-9.22199171122202	-11.0248773422124 \\
-9.223146706087	-10.9527599846377 \\
-9.22416316260382	-10.8807439479618 \\
-9.22503846954488	-10.8088535832222 \\
-9.22577001568257	-10.7371132414563 \\
-9.22635518978928	-10.6655472737016 \\
-9.2267913806374	-10.5941800309955 \\
-9.22707597699932	-10.5230358643755 \\
-9.22720636764743	-10.4521391248792 \\
-9.22692841101759	-10.3814623636378 \\
-9.22545402511809	-10.3108524937667 \\
-9.22278556213613	-10.2403016492848 \\
-9.21898953695621	-10.1698145506582 \\
-9.21413246446285	-10.0993959183534 \\
-9.20828085954052	-10.0290504728365 \\
-9.20150123707376	-9.95878293457387 \\
-9.19386011194706	-9.88859802403173 \\
-9.18542399904492	-9.81850046167636 \\
-9.17625941325185	-9.74849496797404 \\
-9.16643286945236	-9.67858626339103 \\
-9.15601088253096	-9.60877906839363 \\
-9.14505996737214	-9.53907810344808 \\
-9.13364663886041	-9.46948808902067 \\
-9.12183741188028	-9.40001374557766 \\
-9.10969880131625	-9.33065979358533 \\
-9.09729732205283	-9.26143095350996 \\
-9.08469948897453	-9.19233194581781 \\
-9.07197181696584	-9.12336749097515 \\
-9.05918082091128	-9.05454230944826 \\
-9.04639301569534	-8.98586112170342 \\
-9.03367491620254	-8.91732864820688 \\
-9.02109303731738	-8.84894960942492 \\
-9.00866132365563	-8.78072623450993 \\
-8.99574854694792	-8.71262955521698 \\
-8.98216616081542	-8.64465025916107 \\
-8.96793231753914	-8.57678867380509 \\
-8.95306516940013	-8.50904512661193 \\
-8.93758286867941	-8.44141994504449 \\
-8.92150356765802	-8.37391345656566 \\
-8.90484541861698	-8.30652598863834 \\
-8.88762657383734	-8.23925786872541 \\
-8.86986518560011	-8.17210942428978 \\
-8.85157940618633	-8.10508098279433 \\
-8.83278738787703	-8.03817287170196 \\
-8.81350728295325	-7.97138541847557 \\
-8.793757243696	-7.90471895057804 \\
-8.77355542238634	-7.83817379547227 \\
-8.75291997130527	-7.77175028062116 \\
-8.73186904273385	-7.7054487334876 \\
-8.71042078895309	-7.63926948153448 \\
-8.68859336224403	-7.57321285222469 \\
-8.6664049148877	-7.50727917302114 \\
-8.64387359916513	-7.44146877138671 \\
-8.62101756735735	-7.3757819747843 \\
-8.5978549717454	-7.31021911067679 \\
};

\addplot [color1, line width=2.0pt, dashed]
table [row sep=\\]{%
-8.47956495366373	-6.73014424774022 \\
-8.43813866930791	-6.66249999015269 \\
-8.39671244907378	-6.59485567119804 \\
-8.3552862933861	-6.52721129046985 \\
-8.31386019894849	-6.45956685112124 \\
-8.27243416507789	-6.39192235380544 \\
-8.23100818900704	-6.32427780116944 \\
-8.18958226961191	-6.25663319428832 \\
-8.14815640496863	-6.18898853500225 \\
-8.10673059294376	-6.12134382535192 \\
-8.06530483255243	-6.05369906627927 \\
-8.02387912096031	-5.98605426049545 \\
-7.98245345747764	-5.91840940866011 \\
-7.94102783938074	-5.85076451337853 \\
-7.89960226547013	-5.78311957579797 \\
-7.85817673392425	-5.71547459766077 \\
-7.81675124251237	-5.64782958110063 \\
-7.77532579031002	-5.58018452700176 \\
-7.73390037445934	-5.51253943809777 \\
-7.69247499425722	-5.44489431506113 \\
-7.65104964703024	-5.37724916044901 \\
-7.60962433150013	-5.30960397548404 \\
-7.56819904594426	-5.24195876181395 \\
-7.52677378804133	-5.17431352165903 \\
-7.48534855692056	-5.1066682558522 \\
-7.44392334970731	-5.03902296714306 \\
-7.40249816567834	-4.97137765622337 \\
-7.36107300221725	-4.90373232559573 \\
-7.31964785796361	-4.8360869765614 \\
-7.27822273129035	-4.76844161067665 \\
-7.23679761979234	-4.700796230242 \\
-7.19537252264577	-4.63315083604543 \\
-7.1539474369659	-4.56550543084619 \\
-7.11252236200271	-4.49786001536168 \\
-7.07109729520347	-4.43021459203357 \\
-7.02967223512226	-4.3625691622449 \\
-6.98824718022423	-4.29492372746374 \\
-6.94682212802712	-4.22727829006439 \\
-6.90539707774735	-4.15963285079633 \\
-6.86397202649684	-4.09198741242201 \\
-6.82254697349202	-4.02434197569092 \\
-6.78112191625061	-3.95669654297737 \\
-6.73969685323779	-3.8890511157494 \\
-6.69827178300761	-3.82140569539007 \\
-6.65684670300736	-3.75376028434105 \\
-6.61542161248701	-3.68611488331974 \\
-6.57399650856183	-3.61846949508538 \\
-6.532571390408	-3.55082412042596 \\
-6.49114625562038	-3.48317876164199 \\
-6.44972110257192	-3.41553342028976 \\
-6.40829592990219	-3.34788809767051 \\
-6.36687073499478	-3.28024279628685 \\
-6.32544551712645	-3.21259751683052 \\
-6.28402027342257	-3.14495226205116 \\
-6.24259500301234	-3.07730703278165 \\
-6.20116970357448	-3.00966183124231 \\
-6.15974437338635	-2.94201665908084 \\
-6.11831901116969	-2.8743715175199 \\
-6.07689361425107	-2.80672640911664 \\
-6.03546818192738	-2.73908133454355 \\
-5.99404271134076	-2.67143629653421 \\
-5.95261720156675	-2.60379129597285 \\
-5.91119165037461	-2.53614633499317 \\
-5.86976605594279	-2.46850141533751 \\
-5.8283404170718	-2.40085653815312 \\
-5.78691473103795	-2.33321170604529 \\
-5.74548899715152	-2.26556691967367 \\
-5.70406321257812	-2.1979221817494 \\
-5.66263737633294	-2.13027749321443 \\
-5.62121148628249	-2.06263285610944 \\
-5.57978554050293	-1.99498827227461 \\
-5.53835953787024	-1.92734374278502 \\
-5.49693347561712	-1.85969927028766 \\
-5.45550735306055	-1.79205485543575 \\
-5.41408116739635	-1.72441050091158 \\
-5.37265491757261	-1.65676620772124 \\
-5.33122860155982	-1.58912197780597 \\
-5.2898022173285	-1.52147781310702 \\
-5.2483757638267	-1.45383371463046 \\
-5.20694923657938	-1.38618968662267 \\
-5.16552263716412	-1.31854572762074 \\
-5.12409594745858	-1.25090185464477 \\
-5.08266927600936	-1.18325796611252 \\
-5.04124273417977	-1.11561395781935 \\
-4.99981658205179	-1.04796958636344 \\
-4.95839092738537	-0.980324750885173 \\
-4.91696608302646	-0.912679158618176 \\
-4.87554212501023	-0.845032738382948 \\
-4.83411935369316	-0.777385209104908 \\
-4.79269790138513	-0.709736446956309 \\
-4.75127797956758	-0.642086254364424 \\
-4.70985985302877	-0.574434384241008 \\
-4.66844378186997	-0.506780598589802 \\
-4.62702843896589	-0.439126082242612 \\
-4.58561310942601	-0.371471476691176 \\
-4.5441946375265	-0.303819614932557 \\
-4.50277150069113	-0.236171864475958 \\
-4.46134167477074	-0.168530041330788 \\
-4.41990243767561	-0.10089659154131 \\
-4.3784527650517	-0.033272431113538 \\
-4.33698726947216	0.0343378153456178 \\
-4.29550906219039	0.101936520231401 \\
-4.25400611952714	0.169513954806513 \\
-4.21257318600338	0.237144332782636 \\
-4.17136367238424	0.304944051780433 \\
-4.13025479273707	0.372838808212557 \\
-4.08930975692192	0.440884769496837 \\
-4.04615126969165	0.507308128270454 \\
-4.00047321033762	0.571936236798669 \\
-3.95135423372468	0.633943040725652 \\
-3.89906045841412	0.693504420883291 \\
-3.84369489948365	0.750648352541727 \\
-3.78550804850776	0.805319854056822 \\
-3.72445173069645	0.857614586162729 \\
-3.66084247020391	0.907436898475953 \\
-3.5946392784701	0.954884554853098 \\
-3.52608547190208	0.999898943793181 \\
-3.45525758132844	1.042502674138 \\
-3.38225256640832	1.08274793177036 \\
-3.30728090804851	1.12054240240135 \\
-3.2303556544862	1.15602350316877 \\
-3.15172211208202	1.18905263219321 \\
-3.07140458256091	1.21976145913672 \\
-2.98959589558014	1.24807648769521 \\
-2.90640092323269	1.27400588849502 \\
-2.82191437826662	1.29765436119647 \\
-2.7363299166944	1.31884520271552 \\
-2.64968813461763	1.33782080769698 \\
-2.5622025457239	1.35433305861814 \\
-2.47392695379314	1.36859589600366 \\
-2.38503052929996	1.38052343320928 \\
-2.2956326572057	1.39001234229848 \\
-2.20582453704006	1.39749402580129 \\
-2.11579537822207	1.40215548341569 \\
-2.02559480373332	1.40508039161931 \\
-1.93542503344344	1.40489340005478 \\
-1.84535167837823	1.40331675648135 \\
-1.75553203590408	1.3992677070414 \\
-1.66609625444215	1.3927349850756 \\
-1.57712671386871	1.38385101744279 \\
-1.48882002198003	1.37290172821211 \\
-1.40121985960123	1.3594942579927 \\
-1.31453235675915	1.34399958318646 \\
-1.22882333399858	1.32615458326952 \\
-1.1442433503075	1.30610071121763 \\
-1.06093582882194	1.28383289537142 \\
-0.978963934215403	1.2592601947644 \\
-0.898547532973915	1.23254761900685 \\
-0.819706629345612	1.20352838892884 \\
-0.742710031351454	1.17238672005497 \\
};


\addplot [white, line width=1.0pt, dashed]
table [row sep=\\]{%
-8.47956495366373	-6.73014424774022 \\
-8.43813866930791	-6.66249999015269 \\
-8.39671244907378	-6.59485567119804 \\
-8.3552862933861	-6.52721129046985 \\
-8.31386019894849	-6.45956685112124 \\
-8.27243416507789	-6.39192235380544 \\
-8.23100818900704	-6.32427780116944 \\
-8.18958226961191	-6.25663319428832 \\
-8.14815640496863	-6.18898853500225 \\
-8.10673059294376	-6.12134382535192 \\
-8.06530483255243	-6.05369906627927 \\
-8.02387912096031	-5.98605426049545 \\
-7.98245345747764	-5.91840940866011 \\
-7.94102783938074	-5.85076451337853 \\
-7.89960226547013	-5.78311957579797 \\
-7.85817673392425	-5.71547459766077 \\
-7.81675124251237	-5.64782958110063 \\
-7.77532579031002	-5.58018452700176 \\
-7.73390037445934	-5.51253943809777 \\
-7.69247499425722	-5.44489431506113 \\
-7.65104964703024	-5.37724916044901 \\
-7.60962433150013	-5.30960397548404 \\
-7.56819904594426	-5.24195876181395 \\
-7.52677378804133	-5.17431352165903 \\
-7.48534855692056	-5.1066682558522 \\
-7.44392334970731	-5.03902296714306 \\
-7.40249816567834	-4.97137765622337 \\
-7.36107300221725	-4.90373232559573 \\
-7.31964785796361	-4.8360869765614 \\
-7.27822273129035	-4.76844161067665 \\
-7.23679761979234	-4.700796230242 \\
-7.19537252264577	-4.63315083604543 \\
-7.1539474369659	-4.56550543084619 \\
-7.11252236200271	-4.49786001536168 \\
-7.07109729520347	-4.43021459203357 \\
-7.02967223512226	-4.3625691622449 \\
-6.98824718022423	-4.29492372746374 \\
-6.94682212802712	-4.22727829006439 \\
-6.90539707774735	-4.15963285079633 \\
-6.86397202649684	-4.09198741242201 \\
-6.82254697349202	-4.02434197569092 \\
-6.78112191625061	-3.95669654297737 \\
-6.73969685323779	-3.8890511157494 \\
-6.69827178300761	-3.82140569539007 \\
-6.65684670300736	-3.75376028434105 \\
-6.61542161248701	-3.68611488331974 \\
-6.57399650856183	-3.61846949508538 \\
-6.532571390408	-3.55082412042596 \\
-6.49114625562038	-3.48317876164199 \\
-6.44972110257192	-3.41553342028976 \\
-6.40829592990219	-3.34788809767051 \\
-6.36687073499478	-3.28024279628685 \\
-6.32544551712645	-3.21259751683052 \\
-6.28402027342257	-3.14495226205116 \\
-6.24259500301234	-3.07730703278165 \\
-6.20116970357448	-3.00966183124231 \\
-6.15974437338635	-2.94201665908084 \\
-6.11831901116969	-2.8743715175199 \\
-6.07689361425107	-2.80672640911664 \\
-6.03546818192738	-2.73908133454355 \\
-5.99404271134076	-2.67143629653421 \\
-5.95261720156675	-2.60379129597285 \\
-5.91119165037461	-2.53614633499317 \\
-5.86976605594279	-2.46850141533751 \\
-5.8283404170718	-2.40085653815312 \\
-5.78691473103795	-2.33321170604529 \\
-5.74548899715152	-2.26556691967367 \\
-5.70406321257812	-2.1979221817494 \\
-5.66263737633294	-2.13027749321443 \\
-5.62121148628249	-2.06263285610944 \\
-5.57978554050293	-1.99498827227461 \\
-5.53835953787024	-1.92734374278502 \\
-5.49693347561712	-1.85969927028766 \\
-5.45550735306055	-1.79205485543575 \\
-5.41408116739635	-1.72441050091158 \\
-5.37265491757261	-1.65676620772124 \\
-5.33122860155982	-1.58912197780597 \\
-5.2898022173285	-1.52147781310702 \\
-5.2483757638267	-1.45383371463046 \\
-5.20694923657938	-1.38618968662267 \\
-5.16552263716412	-1.31854572762074 \\
-5.12409594745858	-1.25090185464477 \\
-5.08266927600936	-1.18325796611252 \\
-5.04124273417977	-1.11561395781935 \\
-4.99981658205179	-1.04796958636344 \\
-4.95839092738537	-0.980324750885173 \\
-4.91696608302646	-0.912679158618176 \\
-4.87554212501023	-0.845032738382948 \\
-4.83411935369316	-0.777385209104908 \\
-4.79269790138513	-0.709736446956309 \\
-4.75127797956758	-0.642086254364424 \\
-4.70985985302877	-0.574434384241008 \\
-4.66844378186997	-0.506780598589802 \\
-4.62702843896589	-0.439126082242612 \\
-4.58561310942601	-0.371471476691176 \\
-4.5441946375265	-0.303819614932557 \\
-4.50277150069113	-0.236171864475958 \\
-4.46134167477074	-0.168530041330788 \\
-4.41990243767561	-0.10089659154131 \\
-4.3784527650517	-0.033272431113538 \\
-4.33698726947216	0.0343378153456178 \\
-4.29550906219039	0.101936520231401 \\
-4.25400611952714	0.169513954806513 \\
-4.21257318600338	0.237144332782636 \\
-4.17136367238424	0.304944051780433 \\
-4.13025479273707	0.372838808212557 \\
-4.08930975692192	0.440884769496837 \\
-4.04615126969165	0.507308128270454 \\
-4.00047321033762	0.571936236798669 \\
-3.95135423372468	0.633943040725652 \\
-3.89906045841412	0.693504420883291 \\
-3.84369489948365	0.750648352541727 \\
-3.78550804850776	0.805319854056822 \\
-3.72445173069645	0.857614586162729 \\
-3.66084247020391	0.907436898475953 \\
-3.5946392784701	0.954884554853098 \\
-3.52608547190208	0.999898943793181 \\
-3.45525758132844	1.042502674138 \\
-3.38225256640832	1.08274793177036 \\
-3.30728090804851	1.12054240240135 \\
-3.2303556544862	1.15602350316877 \\
-3.15172211208202	1.18905263219321 \\
-3.07140458256091	1.21976145913672 \\
-2.98959589558014	1.24807648769521 \\
-2.90640092323269	1.27400588849502 \\
-2.82191437826662	1.29765436119647 \\
-2.7363299166944	1.31884520271552 \\
-2.64968813461763	1.33782080769698 \\
-2.5622025457239	1.35433305861814 \\
-2.47392695379314	1.36859589600366 \\
-2.38503052929996	1.38052343320928 \\
-2.2956326572057	1.39001234229848 \\
-2.20582453704006	1.39749402580129 \\
-2.11579537822207	1.40215548341569 \\
-2.02559480373332	1.40508039161931 \\
-1.93542503344344	1.40489340005478 \\
-1.84535167837823	1.40331675648135 \\
-1.75553203590408	1.3992677070414 \\
-1.66609625444215	1.3927349850756 \\
-1.57712671386871	1.38385101744279 \\
-1.48882002198003	1.37290172821211 \\
-1.40121985960123	1.3594942579927 \\
-1.31453235675915	1.34399958318646 \\
-1.22882333399858	1.32615458326952 \\
-1.1442433503075	1.30610071121763 \\
-1.06093582882194	1.28383289537142 \\
-0.978963934215403	1.2592601947644 \\
-0.898547532973915	1.23254761900685 \\
-0.819706629345612	1.20352838892884 \\
-0.742710031351454	1.17238672005497 \\
};
\end{axis}
\end{tikzpicture}
    \label{fig:jfk_13L_hist_org}}
    \subfigure[Synthetic trajectories]{
    \setlength\figureheight{6.7cm}
    \setlength\figurewidth{6.7cm}
    \input{figures/jfk_13L_synthetic_hist_1000.tex}
    \label{fig:jfk_13L_hist_syn}}
    \caption{Log-histograms of arrival tracks to KJFK 13L and associated flight procedures.}
    \label{fig:jfk_13L_org_syn}
\end{figure}

To investigate how well the proposed model generalizes to a new environment, we generated synthetic trajectories given Charlotte Douglas Airport (KCLT) runway 36C arrival procedures using the model trained with KJFK data.
Fig. \ref{fig:clt_36C_hist_org} shows a density plot of one month of actual arrivals to KCLT 36C in 2016.
Fig. \ref{fig:clt_36C_hist_syn} is a density plot of 1,000 synthetic trajectories generated by sampling the deviation trajectories from our model and re-constructing the position trajectories with the given flight procedures.
Again, the radar vector procedures and the IAP are indicated in blue dotted lines and a white dashed line with orange edges, respectively.

From the two histograms, we see that some traffic patterns in the actual trajectory set do not appear in the synthetic trajectory set.
Part of the reason is because some trajectories do not appear to follow any of the procedures. 
For example, in Fig. \ref{fig:clt_36C_hist_org}, there is no standard procedure for those trajectories approaching from the northwest, passing north of the airport, and joining the downwind leg on the east side.
Another reason is that the two environments are extremely different.
The radar vector procedures and the actual trajectories of KCLT are mostly straight with only some parts curved, while
those of KJFK do not have straight segments. 
Also, the KJFK trajectories have greater variability in their early arrival stages, while the variability of KCLT trajectories depends on the procedure.
Thus, the synthetic trajectories in Fig. \ref{fig:clt_36C_hist_syn} are more representative of KJFK traffic than KCLT traffic.
\begin{figure}[tb]
    \centering
    \subfigure[Actual trajectories]{
        \setlength\figureheight{7cm}
        \setlength\figurewidth{7cm}
        % This file was created by matplotlib2tikz v0.6.18.
\begin{tikzpicture}
% \usetikzlibrary{
%     pgfplots.fillbetween,
% }
    
\definecolor{color0}{rgb}{0.392156862745098,0.584313725490196,0.929411764705882}
\definecolor{color1}{RGB}{255,165,0}

\begin{axis}[
colorbar,
colorbar style={width=7,
ytick={-7,-6,-5,-4,-3},
yticklabel={$10^{\pgfmathprintnumber{\tick}}$}},
colormap = {whiteblack}{color(0cm)=(white); color(1cm)=(black)},
height=\figureheight,
point meta max=-2.8,
point meta min=-7,
tick align=outside,
tick pos=left,
width=\figurewidth,
x grid style={black},
xlabel={East (NM)},
xmin=-30, xmax=30,
y grid style={white!69.01960784313725!black},
ylabel={North (NM)},
ymin=-30, ymax=30,
xtick={-30, -20, -10, 0, 10, 20, 30},
ytick={-30, -20, -10, 0, 10, 20, 30},
tick label style={font=\tiny},
label style={font=\scriptsize}
]

\addplot graphics [includegraphics cmd=\pgfimage,xmin=-30.7, xmax=29.3, ymin=-30, ymax=30] {figures/clt_36C_original_hist.png};

\addplot [draw=none, draw=black, fill=black, colormap/viridis]
table [row sep=\\]{%
x                      y\\ 
+0.000000000000000e+00 -2.500000000000000e-01\\ 
+6.630077500000001e-02 -2.500000000000000e-01\\ 
+1.298949676962134e-01 -2.236584228970604e-01\\ 
+1.767766952966369e-01 -1.767766952966369e-01\\ 
+2.236584228970604e-01 -1.298949676962134e-01\\ 
+2.500000000000000e-01 -6.630077500000001e-02\\ 
+2.500000000000000e-01 +0.000000000000000e+00\\ 
+2.500000000000000e-01 +6.630077500000001e-02\\ 
+2.236584228970604e-01 +1.298949676962134e-01\\ 
+1.767766952966369e-01 +1.767766952966369e-01\\ 
+1.298949676962134e-01 +2.236584228970604e-01\\ 
+6.630077500000001e-02 +2.500000000000000e-01\\ 
+0.000000000000000e+00 +2.500000000000000e-01\\ 
-6.630077500000001e-02 +2.500000000000000e-01\\ 
-1.298949676962134e-01 +2.236584228970604e-01\\ 
-1.767766952966369e-01 +1.767766952966369e-01\\ 
-2.236584228970604e-01 +1.298949676962134e-01\\ 
-2.500000000000000e-01 +6.630077500000001e-02\\ 
-2.500000000000000e-01 +0.000000000000000e+00\\ 
-2.500000000000000e-01 -6.630077500000001e-02\\ 
-2.236584228970604e-01 -1.298949676962134e-01\\ 
-1.767766952966369e-01 -1.767766952966369e-01\\ 
-1.298949676962134e-01 -2.236584228970604e-01\\ 
-6.630077500000001e-02 -2.500000000000000e-01\\ 
+0.000000000000000e+00 -2.500000000000000e-01\\ 
+0.000000000000000e+00 -2.500000000000000e-01\\ 
};

\addplot [black, line width=0.75pt, forget plot]
table [row sep=\\]{%
-0.0437825821575746	-0.287458925473235 \\
0.880549086394463	0.531262238591788 \\
};
\addplot [black, line width=0.75pt, forget plot]
table [row sep=\\]{%
-0.202217584560991	0.817568405721303 \\
-0.086374858888876	-0.824271773162329 \\
};
\addplot [black, line width=0.75pt, forget plot]
table [row sep=\\]{%
0.634124780369762	0.65817841566564 \\
0.734630033450532	-0.766402068204335 \\
};
\addplot [black, line width=0.75pt, forget plot]
table [row sep=\\]{%
-0.902780813753093	0.691052682018094 \\
-0.791814774184891	-0.779693639257527 \\
};

\addplot [blue, line width=1.2pt, dotted, forget plot]
table [row sep=\\]{%
-8.51672772854829	-29.2506600204454 \\
-8.47036478290207	-29.2357365433276 \\
-8.42394217375767	-29.2212906741963 \\
-8.37746002528326	-29.2073314429095 \\
-8.33091846164698	-29.1938678793252 \\
-8.28431760701701	-29.1809090133014 \\
-8.23765758556151	-29.1684638746959 \\
-8.19093852144864	-29.1565414933669 \\
-8.14416053884656	-29.1451508991722 \\
-8.09732376192343	-29.1343011219699 \\
-8.05042831484741	-29.1240011916179 \\
-8.00347432178668	-29.1142601379743 \\
-7.95646190690938	-29.1050869908969 \\
-7.90935138832272	-29.0963458127502 \\
-7.86210599573572	-29.0878879289377 \\
-7.81473022071928	-29.0797032638758 \\
-7.76722855484426	-29.0717817419809 \\
-7.71960548968155	-29.0641132876696 \\
-7.67186551680204	-29.0566878253583 \\
-7.62401312777661	-29.0494952794634 \\
-7.57605281417615	-29.0425255744015 \\
-7.52798906757154	-29.0357686345889 \\
-7.47982637953366	-29.0292143844422 \\
-7.43156924163339	-29.0228527483779 \\
-7.38322214544163	-29.0166736508123 \\
-7.33478958252925	-29.010667016162 \\
-7.28627604446715	-29.0048227688434 \\
-7.23768602282619	-28.999130833273 \\
-7.18902400917727	-28.9935811338672 \\
-7.14029449509127	-28.9881635950426 \\
-7.09150197213908	-28.9828681412155 \\
-7.04265093189157	-28.9776846968025 \\
-6.99374586591963	-28.97260318622 \\
-6.94479126579416	-28.9676135338845 \\
-6.89579162308602	-28.9627056642125 \\
-6.8467514293661	-28.9578695016203 \\
-6.7976751762053	-28.9530949705245 \\
-6.74856735517448	-28.9483719953416 \\
-6.69943245784454	-28.943690500488 \\
-6.65027497578636	-28.9390404103802 \\
-6.60109940057083	-28.9344116494346 \\
-6.55191022376882	-28.9297941420677 \\
-6.50271193695122	-28.925177812696 \\
-6.45350903168892	-28.9205525857359 \\
-6.40430599955279	-28.9159083856039 \\
-6.35510733211373	-28.9112351367165 \\
-6.30591752094262	-28.9065227634901 \\
-6.25674105761034	-28.9017611903412 \\
-6.20758243368777	-28.8969403416863 \\
-6.1584461407458	-28.8920501419418 \\
-6.10933667035531	-28.8870805155243 \\
-6.06025851408719	-28.88202138685 \\
-6.01121616351232	-28.8768626803356 \\
-5.96221411020158	-28.8715943203975 \\
-5.91325684572587	-28.8662062314522 \\
-5.86434886165605	-28.8606883379161 \\
-5.81549464956302	-28.8550305642057 \\
-5.76669870101766	-28.8492228347374 \\
-5.71796550759086	-28.8432550739278 \\
-5.66929956085349	-28.8371172061932 \\
-5.62070535237645	-28.8307991559502 \\
-5.5721873737306	-28.8242908476152 \\
-5.52375011648685	-28.8175822056047 \\
-5.47514610915991	-28.8108119455373 \\
-5.42614233322395	-28.804119559415 \\
-5.37676495968905	-28.7974961362297 \\
-5.32704015956528	-28.7909327649731 \\
-5.27699410386271	-28.7844205346369 \\
-5.22665296359142	-28.7779505342129 \\
-5.17604290976147	-28.771513852693 \\
-5.12519011338294	-28.7651015790687 \\
-5.0741207454659	-28.758704802332 \\
-5.02286097702043	-28.7523146114745 \\
-4.97143697905659	-28.745922095488 \\
-4.91987492258445	-28.7395183433643 \\
-4.8682009786141	-28.7330944440951 \\
-4.8164413181556	-28.7266414866722 \\
-4.76462211221903	-28.7201505600873 \\
-4.71276953181445	-28.7136127533323 \\
-4.66090974795194	-28.7070191553988 \\
-4.60906893164157	-28.7003608552787 \\
-4.55727325389342	-28.6936289419637 \\
-4.50554888571755	-28.6868145044455 \\
-4.45392199812404	-28.6799086317159 \\
-4.40241876212297	-28.6729024127667 \\
-4.35106534872439	-28.6657869365896 \\
-4.29988792893839	-28.6585532921764 \\
-4.24891267377504	-28.6511925685188 \\
-4.19816575424441	-28.6436958546087 \\
-4.14767334135656	-28.6360542394377 \\
-4.09746160612159	-28.6282588119976 \\
-4.04755671954955	-28.6203006612802 \\
-3.99798485265052	-28.6121708762773 \\
-3.94877217643457	-28.6038605459806 \\
-3.89994486191177	-28.5953607593818 \\
-3.8515290800922	-28.5866626054728 \\
-3.80355100198593	-28.5777571732452 \\
-3.75603679860302	-28.5686355516909 \\
-3.70901264095356	-28.5592888298016 \\
-3.66250470004762	-28.549708096569 \\
-3.61653914689526	-28.539884440985 \\
-3.57114215250656	-28.5298089520412 \\
-3.52633988789159	-28.5194727187295 \\
-3.48215852406043	-28.5088668300416 \\
-3.43862423202314	-28.4979823749692 \\
-3.3957631827898	-28.4868104425042 \\
-3.35360154737049	-28.4753421216382 \\
-3.31216549677526	-28.4635685013631 \\
-3.2714812020142	-28.4514806706706 \\
-3.23157483409738	-28.4390697185524 \\
-3.19247256403487	-28.4263267340003 \\
-3.15420056283674	-28.4132428060062 \\
-3.11678500151306	-28.3998090235616 \\
-3.07998453103585	-28.385336925255 \\
-3.04353717508638	-28.3691765388785 \\
-3.00743816373858	-28.35137668723 \\
-2.97168272706642	-28.3319861931073 \\
-2.93626609514385	-28.311053879308 \\
-2.90118349804482	-28.28862856863 \\
-2.86643016584327	-28.2647590838709 \\
-2.83200132861316	-28.2394942478287 \\
-2.79789221642845	-28.212882883301 \\
-2.76409805936308	-28.1849738130856 \\
-2.730614087491	-28.1558158599803 \\
-2.69743553088616	-28.1254578467828 \\
-2.66455761962253	-28.0939485962909 \\
-2.63197558377404	-28.0613369313025 \\
-2.59968465341464	-28.0276716746151 \\
-2.5676800586183	-27.9930016490266 \\
-2.53595702945896	-27.9573756773348 \\
-2.50451079601058	-27.9208425823374 \\
-2.47333658834709	-27.8834511868323 \\
-2.44242963654247	-27.845250313617 \\
-2.41178517067065	-27.8062887854895 \\
-2.38139842080558	-27.7666154252475 \\
-2.35126461702123	-27.7262790556887 \\
-2.32137898939154	-27.685328499611 \\
-2.29173676799045	-27.643812579812 \\
-2.26233318289194	-27.6017801190896 \\
-2.23316346416993	-27.5592799402415 \\
-2.20422284189839	-27.5163608660655 \\
-2.17550654615127	-27.4730717193593 \\
-2.14700980700252	-27.4294613229207 \\
-2.11872785452608	-27.3855784995475 \\
-2.09065591879592	-27.3414720720374 \\
-2.06278922988598	-27.2971908631883 \\
-2.03512301787021	-27.2527836957978 \\
-2.00765251282256	-27.2082993926637 \\
-1.98037294481699	-27.1637867765838 \\
-1.95327954392745	-27.1192946703559 \\
-1.92636754022789	-27.0748718967777 \\
-1.89963216379226	-27.030567278647 \\
-1.87306864469451	-26.9864296387616 \\
-1.84667221300859	-26.9425077999191 \\
-1.82043809880845	-26.8988505849175 \\
-1.79436153216805	-26.8555068165544 \\
-1.76843774316133	-26.8125253176276 \\
-1.74266196186225	-26.769954910935 \\
-1.71702941834476	-26.7278444192741 \\
-1.69153534268282	-26.6862426654429 \\
-1.66617496495036	-26.645198472239 \\
-1.64094351522134	-26.6047606624603 \\
-1.61583622356972	-26.5649780589045 \\
-1.59093014489368	-26.5254109188262 \\
-1.5662998371531	-26.4855922337116 \\
-1.54193678501445	-26.4455299277061 \\
-1.5178324731442	-26.4052319249547 \\
-1.49397838620883	-26.3647061496026 \\
-1.47036600887481	-26.323960525795 \\
-1.44698682580862	-26.2830029776772 \\
-1.42383232167673	-26.2418414293942 \\
-1.40089398114561	-26.2004838050914 \\
-1.37816328888175	-26.1589380289138 \\
-1.35563172955161	-26.1172120250068 \\
-1.33329078782166	-26.0753137175153 \\
-1.31113194835839	-26.0332510305848 \\
-1.28914669582826	-25.9910318883602 \\
-1.26732651489776	-25.948664214987 \\
-1.24566289023335	-25.9061559346101 \\
-1.2241473065015	-25.8635149713748 \\
-1.2027712483687	-25.8207492494264 \\
-1.18152620050142	-25.7778666929099 \\
-1.16040364756613	-25.7348752259706 \\
-1.13939507422931	-25.6917827727537 \\
-1.11849196515742	-25.6485972574043 \\
-1.09768580501695	-25.6053266040677 \\
-1.07696807847437	-25.561978736889 \\
-1.05633027019614	-25.5185615800134 \\
-1.03576386484876	-25.4750830575862 \\
-1.01526034709869	-25.4315510937524 \\
-0.994811201612397	-25.3879736126574 \\
-0.974407913056368	-25.3443585384462 \\
-0.954041966097072	-25.3007137952642 \\
-0.933704845400983	-25.2570473072563 \\
-0.913388035634576	-25.213366998568 \\
-0.893083021464326	-25.1696807933442 \\
-0.872781287556705	-25.1259966157303 \\
-0.852474318578189	-25.0823223898714 \\
-0.832153599195252	-25.0386660399128 \\
-0.811810614074367	-24.9950354899995 \\
-0.79143684788201	-24.9514386642768 \\
-0.771023785284653	-24.9078834868899 \\
-0.750562910948772	-24.864377881984 \\
-0.730045709540841	-24.8209297737042 \\
-0.709463665727334	-24.7775470861958 \\
-0.688808264174725	-24.7342377436039 \\
-0.668070989549488	-24.6910096700737 \\
-0.647243326518097	-24.6478707897505 \\
-0.626316759747027	-24.6048290267793 \\
-0.605282773902752	-24.5618923053055 \\
-0.584132853651746	-24.5190685494741 \\
-0.562858483660483	-24.4763656834304 \\
-0.541451148595438	-24.4337916313196 \\
-0.519893773540127	-24.3913053286624 \\
-0.498181567788796	-24.3488589278603 \\
-0.476324442324006	-24.3064501783792 \\
-0.45433230812832	-24.2640768296852 \\
-0.432215076184303	-24.2217366312443 \\
-0.409982657474517	-24.1794273325225 \\
-0.387644962981525	-24.1371466829859 \\
-0.365211903687891	-24.0948924321004 \\
-0.342693390576177	-24.0526623293321 \\
-0.320099334628946	-24.010454124147 \\
-0.297439646828762	-23.9682655660112 \\
-0.274724238158188	-23.9260944043906 \\
-0.251963019599786	-23.8839383887514 \\
-0.229165902136121	-23.8417952685594 \\
-0.206342796749755	-23.7996627932808 \\
-0.183503614423251	-23.7575387123815 \\
-0.160658266139172	-23.7154207753276 \\
-0.137816662880082	-23.6733067315851 \\
-0.114988715628543	-23.63119433062 \\
-0.0921843353671195	-23.5890813218984 \\
-0.0694134330783736	-23.5469654548863 \\
-0.0466859197448685	-23.5048444790496 \\
-0.0240117063491675	-23.4627161438545 \\
-0.00140070387383374	-23.4205781987669 \\
0.0211371766985697	-23.3784283932529 \\
0.0435920243854798	-23.3362644767785 \\
0.0659539282043333	-23.2940841988097 \\
0.0882129771725673	-23.2518853088125 \\
0.110359260307618	-23.209665556253 \\
0.132382866626923	-23.1674226905972 \\
0.15427388514792	-23.125154461311 \\
0.176022404888044	-23.0828586178607 \\
0.197618514864733	-23.040532909712 \\
0.219052304095423	-22.9981750863312 \\
0.240313861597552	-22.9557828971841 \\
0.261393276388556	-22.9133540917369 \\
0.282280637485873	-22.8708864194555 \\
0.302966033906939	-22.828377629806 \\
0.323439554669191	-22.7858254722544 \\
0.343691288790066	-22.7432276962667 \\
0.363711325287	-22.700582051309 \\
0.383489753177432	-22.6578862868472 \\
0.403016661478796	-22.6151381523474 \\
0.422282139208532	-22.5723353972756 \\
0.441276275384074	-22.5294757710979 \\
0.459989159022861	-22.4865570232802 \\
0.478410879142329	-22.4435769032886 \\
0.496531524759915	-22.4005331605891 \\
0.514341184893056	-22.3574235446478 \\
0.531829948559189	-22.3142458049305 \\
0.549112053307254	-22.2710158525392 \\
0.566307272727909	-22.2277511249155 \\
0.583413999901165	-22.1844516610353 \\
0.600430627907035	-22.1411174998746 \\
0.617355549825527	-22.0977486804094 \\
0.634187158736655	-22.0543452416156 \\
0.65092384772043	-22.0109072224692 \\
0.667564009856863	-21.9674346619462 \\
0.684106038225966	-21.9239275990225 \\
0.700548325907749	-21.8803860726741 \\
0.716889265982224	-21.836810121877 \\
0.733127251529403	-21.7931997856071 \\
0.749260675629297	-21.7495551028403 \\
0.765287931361917	-21.7058761125527 \\
0.781207411807275	-21.6621628537202 \\
0.797017510045382	-21.6184153653188 \\
0.812716619156249	-21.5746336863244 \\
0.828303132219889	-21.530817855713 \\
0.843775442316312	-21.4869679124605 \\
0.859131942525529	-21.443083895543 \\
0.874371025927552	-21.3991658439364 \\
0.889491085602393	-21.3552137966166 \\
0.904490514630062	-21.3112277925596 \\
0.919367706090571	-21.2672078707414 \\
0.934121053063932	-21.223154070138 \\
0.948748948630155	-21.1790664297252 \\
0.963249785869253	-21.1349449884792 \\
0.977621957861237	-21.0907897853757 \\
0.991863857686117	-21.0466008593908 \\
1.00597387842391	-21.0023782495005 \\
1.01995041315461	-20.9581219946807 \\
1.03379185495825	-20.9138321339074 \\
1.04749659691484	-20.8695087061566 \\
1.06106303210437	-20.8251517504041 \\
1.07448955360687	-20.780761305626 \\
1.08777455450235	-20.7363374107983 \\
1.10091642787082	-20.6918801048969 \\
1.11391356679228	-20.6473894268977 \\
1.12676436434676	-20.6028654157767 \\
1.13946721361426	-20.55830811051 \\
1.15202050767479	-20.5137175500734 \\
1.16442263960836	-20.4690937734429 \\
1.176672002495	-20.4244368195945 \\
1.1887669894147	-20.3797467275041 \\
1.20070599344748	-20.3350235361477 \\
1.21248740767335	-20.2902672845013 \\
1.22410962517232	-20.2454780115409 \\
1.2355710390244	-20.2006557562423 \\
1.24687004230961	-20.1558005575816 \\
};
\addplot [blue, line width=1.2pt, dotted, forget plot]
table [row sep=\\]{%
17.600797784798	23.9797993375339 \\
17.521604019041	23.8514475147994 \\
17.4424731370187	23.7231633168305 \\
17.3634052190942	23.5949467995841 \\
17.2844003456309	23.4667980190171 \\
17.2054585969918	23.3387170310866 \\
17.1265800535403	23.2107038917494 \\
17.0477647956395	23.0827586569626 \\
16.9690129036526	22.9548813826831 \\
16.8903244579428	22.8270721248679 \\
16.8116995388734	22.699330939474 \\
16.7331382268076	22.5716578824583 \\
16.6546406021085	22.4440530097779 \\
16.5762067451394	22.3165163773896 \\
16.4978367362635	22.1890480412504 \\
16.4195306558441	22.0616480573174 \\
16.3412885842442	21.9343164815475 \\
16.2631106018271	21.8070533698976 \\
16.1849967889561	21.6798587783247 \\
16.1069472259943	21.5527327627859 \\
16.028961993305	21.425675379238 \\
15.9510411712514	21.298686683638 \\
15.8731848401966	21.171766731943 \\
15.7953930805039	21.0449155801098 \\
15.7176659725365	20.9181332840954 \\
15.6400035966576	20.7914198998569 \\
15.5624060332304	20.6647754833512 \\
15.4848733626181	20.5382000905352 \\
15.407405665184	20.4116937773659 \\
15.3300030212913	20.2852565998003 \\
15.2526655113031	20.1588886137954 \\
15.1753932155826	20.0325898753081 \\
15.0981862144931	19.9063604402955 \\
15.0210445883979	19.7802003647143 \\
14.94396841766	19.6541097045218 \\
14.8669577826427	19.5280885156747 \\
14.7900127637093	19.4021368541301 \\
14.7131334412228	19.276254775845 \\
14.6363198955467	19.1504423367763 \\
14.5595722070439	19.024699592881 \\
14.4828904560779	18.899026600116 \\
14.4062747230117	18.7734234144384 \\
14.3297250882085	18.647890091805 \\
14.2532416320317	18.5224266881729 \\
14.1768244348444	18.3970332594991 \\
14.1004735770098	18.2717098617405 \\
14.0241891388911	18.146456550854 \\
13.9479712008515	18.0212733827967 \\
13.8718313032994	17.8961619328228 \\
13.7957798324536	17.7711234998985 \\
13.7198151373936	17.6461577255485 \\
13.6439355671984	17.5212642512975 \\
13.5681394709476	17.3964427186705 \\
13.4924251977202	17.2716927691921 \\
13.4167910965957	17.1470140443871 \\
13.3412355166533	17.0224061857802 \\
13.2657568069722	16.8978688348964 \\
13.1903533166318	16.7734016332602 \\
13.1150233947113	16.6490042223966 \\
13.0397653902901	16.5246762438302 \\
12.9645776524474	16.4004173390858 \\
12.8894585302624	16.2762271496882 \\
12.8144063728146	16.1521053171623 \\
12.739419529183	16.0280514830326 \\
12.6644963484472	15.9040652888241 \\
12.5896351796862	15.7801463760615 \\
12.5148343719795	15.6562943862695 \\
12.4400922744062	15.532508960973 \\
12.3654072360457	15.4087897416967 \\
12.2907776059773	15.2851363699653 \\
12.2162017332802	15.1615484873037 \\
12.1416779670337	15.0380257352366 \\
12.0672046563171	14.9145677552888 \\
11.9927801502097	14.7911741889851 \\
11.9184027977908	14.6678446778502 \\
11.8440709481396	14.5445788634089 \\
11.7697829503355	14.4213763871859 \\
11.6955371534577	14.2982368907061 \\
11.6213319065854	14.1751600154942 \\
11.5471655587981	14.052145403075 \\
11.4730364591749	13.9291926949733 \\
11.3989429567952	13.8063015327137 \\
11.3248834007382	13.6834715578212 \\
11.2508561400832	13.5607024118204 \\
11.1768595239095	13.4379937362362 \\
11.1028919012963	13.3153451725933 \\
11.0289516213231	13.1927563624165 \\
10.955037033069	13.0702269472305 \\
10.8811464856133	12.9477565685601 \\
10.8072783280353	12.8253448679302 \\
10.7334309094143	12.7029914868654 \\
10.6596025788296	12.5806960668905 \\
10.5857916853605	12.4584582495303 \\
10.5119965780862	12.3362776763096 \\
10.438215606086	12.2141539887532 \\
10.3644471184392	12.0920868283858 \\
10.2906894642251	11.9700758367322 \\
10.216940992523	11.8481206553171 \\
10.142293342944	11.7262353340649 \\
10.0659143851464	11.6044331728214 \\
9.98791681328044	11.4827126879939 \\
9.90841332149606	11.3610723959897 \\
9.82751660394341	11.2395108132162 \\
9.74533935477258	11.1180264560806 \\
9.66199426813365	10.9966178409902 \\
9.57759403817673	10.8752834843522 \\
9.4922513590519	10.7540219025741 \\
9.40607892490925	10.6328316120631 \\
9.31918942989888	10.5117111292264 \\
9.23169556817087	10.3906589704714 \\
9.14371003387532	10.2696736522054 \\
9.05534552116232	10.1487536908357 \\
8.96671472418196	10.0278976027694 \\
8.87793033708433	9.90710390441408 \\
8.78910505401953	9.78637111217685 \\
8.70035156913765	9.66569774246506 \\
8.61178257658877	9.54508231168598 \\
8.52351077052299	9.42452333624693 \\
8.4356488450904	9.30401933255518 \\
8.34830949444109	9.18356881701804 \\
8.26160541272516	9.0631703060428 \\
8.17564929409268	8.94282231603675 \\
8.09055383269377	8.82252336340718 \\
8.0064317226785	8.70227196456138 \\
7.92339565819698	8.58206663590666 \\
7.84155833339928	8.4619058938503 \\
7.76103244243551	8.3417882547996 \\
7.68193067945575	8.22171223516184 \\
7.6043657386101	8.10167635134433 \\
7.52845031404864	7.98167911975435 \\
7.45429709992147	7.86171905679921 \\
7.38201879037868	7.74179467888618 \\
7.31172807957036	7.62190450242258 \\
7.24353766164661	7.50204704381567 \\
7.17756023075751	7.38222081947278 \\
7.11390848105316	7.26242434580117 \\
7.05269510668364	7.14265613920816 \\
6.99403280179906	7.02291471610103 \\
6.93803426054949	6.90319859288707 \\
6.88481217708504	6.78350628597358 \\
6.83447924555579	6.66383631176785 \\
6.78714816011184	6.54418718667717 \\
6.74293161490327	6.42455742710884 \\
6.70194230408018	6.30494554947016 \\
6.66429292179266	6.1853500701684 \\
6.6300961621908	6.06576950561088 \\
6.5994647194247	5.94620237220487 \\
6.57251128764444	5.82664718635768 \\
6.54753097861973	5.70707525698988 \\
6.52275211069066	5.58746211170141 \\
6.49818718786288	5.46781259491858 \\
6.47384871414205	5.34813155106772 \\
6.44974919353383	5.22842382457514 \\
6.42590113004388	5.10869425986717 \\
6.40231702767784	4.98894770137012 \\
6.37900939044139	4.86918899351031 \\
6.35599072234018	4.74942298071407 \\
6.33327352737985	4.6296545074077 \\
6.31087030956609	4.50988841801755 \\
6.28879357290453	4.39012955696991 \\
6.26705582140084	4.27038276869112 \\
6.24566955906067	4.15065289760749 \\
6.22464728988969	4.03094478814533 \\
6.20400151789354	3.91126328473099 \\
6.18374474707789	3.79161323179076 \\
6.1638894814484	3.67199947375097 \\
6.14444822501071	3.55242685503795 \\
6.12543348177049	3.432900220078 \\
6.10685775573341	3.31342441329745 \\
6.0887335509051	3.19400427912263 \\
6.07107337129123	3.07464466197984 \\
6.05388972089746	2.95535040629541 \\
6.03719510372945	2.83612635649566 \\
6.02100202379285	2.71697735700691 \\
6.00532298509332	2.59790825225548 \\
5.99017049163652	2.47892388666768 \\
5.9755570474281	2.36002910466985 \\
5.96149515647373	2.24122875068829 \\
5.94799732277905	2.12252766914933 \\
5.93507605034974	2.00393070447928 \\
5.92274384319143	1.88544270110447 \\
5.9110132053098	1.76706850345122 \\
5.8998966407105	1.64881295594584 \\
5.88940665339918	1.53068090301466 \\
5.87955574738151	1.412677189084 \\
5.87035642666314	1.29480665858016 \\
5.86182119524973	1.17707415592949 \\
5.85396255714693	1.05948452555829 \\
5.84679301636041	0.942042611892883 \\
5.84032507689582	0.82475325935959 \\
5.83457124275881	0.707621312384731 \\
5.82954401795505	0.590651615394626 \\
5.8252559064902	0.473849012815594 \\
5.8217194123699	0.357218349073955 \\
5.81894703959982	0.240764468596029 \\
5.81695129218562	0.124492215808135 \\
5.81574467413295	0.008406435136593 \\
5.81533968944747	-0.107488028992277 \\
5.81578813801881	-0.223688052623814 \\
5.81710924326952	-0.340656369726643 \\
5.81926664450463	-0.458336922759377 \\
5.82222398102915	-0.576673654180626 \\
5.82594489214812	-0.695610506449003 \\
5.83039301716657	-0.815091422023118 \\
5.83553199538951	-0.935060343361585 \\
5.84132546612198	-1.05546121292301 \\
5.84773706866901	-1.17623797316602 \\
5.85473044233561	-1.2973345665492 \\
5.86226922642681	-1.41869493553119 \\
5.87031706024765	-1.54026302257058 \\
5.87883758310315	-1.661982770126 \\
5.88779443429833	-1.78379812065605 \\
5.89715125313822	-1.90565301661934 \\
5.90687167892784	-2.02749140047448 \\
5.91691935097223	-2.14925721468009 \\
5.92725790857641	-2.27089440169479 \\
5.93785099104541	-2.39234690397717 \\
5.94866223768424	-2.51355866398585 \\
5.95965528779795	-2.63447362417945 \\
5.97079378069155	-2.75503572701657 \\
5.98204135567007	-2.87518891495583 \\
5.99336165203854	-2.99487713045583 \\
6.00471830910198	-3.1140443159752 \\
6.01607496616542	-3.23263441397254 \\
6.02739526253389	-3.35059136690646 \\
6.03864283751241	-3.46785911723558 \\
6.04978133040601	-3.5843816074185 \\
6.06077438051971	-3.70010277991384 \\
6.07158562715855	-3.81496657718021 \\
6.08217870962754	-3.92891694167622 \\
6.09251726723172	-4.04189781586049 \\
6.10256493927611	-4.15385314219162 \\
6.11228536506574	-4.26472686312822 \\
6.12164218390563	-4.37446292112891 \\
6.13059903510081	-4.48300525865231 \\
6.1391195579563	-4.59029781815701 \\
6.14716739177714	-4.69628454210164 \\
6.15470617586835	-4.8009093729448 \\
6.16169954953495	-4.90411625314511 \\
6.16811115208197	-5.00584912516117 \\
6.17390462281444	-5.1060519314516 \\
6.17904360103739	-5.20466861447502 \\
6.18349172605583	-5.30164311669003 \\
6.1872126371748	-5.39691938055523 \\
6.19016997369933	-5.49044134852926 \\
6.19232737493443	-5.58215296307072 \\
6.19364848018514	-5.67199816663821 \\
6.19409692875649	-5.75992090169035 \\
6.19273097901602	-5.84760219891126 \\
6.18867861224175	-5.9366735466024 \\
6.18200805210435	-6.02700457364839 \\
6.17278752227451	-6.11846490893384 \\
6.1610852464229	-6.21092418134337 \\
6.14696944822021	-6.3042520197616 \\
6.13050835133712	-6.39831805307314 \\
6.11177017944431	-6.49299191016262 \\
6.09082315621246	-6.58814321991465 \\
6.06773550531225	-6.68364161121385 \\
6.04257545041437	-6.77935671294483 \\
6.01541121518949	-6.87515815399222 \\
5.9863110233083	-6.97091556324063 \\
5.95534309844148	-7.06649856957469 \\
5.92257566425971	-7.161776801879 \\
5.88807694443367	-7.25661988903818 \\
5.85191516263404	-7.35089745993686 \\
5.81415854253151	-7.44447914345966 \\
5.77487530779675	-7.53723456849118 \\
5.73413368210045	-7.62903336391605 \\
5.69200188911328	-7.71974515861888 \\
5.64854815250593	-7.80923958148429 \\
5.60384069594909	-7.89738626139691 \\
5.55794774311342	-7.98405482724134 \\
5.51093751766962	-8.06911490790221 \\
5.46287824328836	-8.15243613226412 \\
5.41383814364033	-8.23388812921172 \\
5.3638854423962	-8.3133405276296 \\
5.31308836322667	-8.39066295640238 \\
5.2615151298024	-8.4657250444147 \\
5.20923396579408	-8.53839642055115 \\
5.1563130948724	-8.60854671369636 \\
5.10282074070803	-8.67604555273495 \\
5.04882512697166	-8.74076256655153 \\
4.99439447733396	-8.80256738403073 \\
4.93959701546562	-8.86132963405716 \\
4.88450096503732	-8.91691894551544 \\
4.82917454971974	-8.96920494729018 \\
4.77368599318355	-9.01805726826601 \\
4.71810351909946	-9.06334553732754 \\
4.66249535113813	-9.10493938335939 \\
4.60692971297024	-9.14270843524617 \\
4.55147482826648	-9.17652232187251 \\
4.49619892069753	-9.20625067212302 \\
4.44117021393407	-9.23176311488233 \\
4.38645693164678	-9.25292927903504 \\
4.33212729750634	-9.26961879346578 \\
4.27824953518344	-9.28170128705916 \\
4.22489186834875	-9.2890463886998 \\
4.17212252067296	-9.29152372727232 \\
4.11912724704858	-9.29151476377853 \\
4.06505444254054	-9.29145201932197 \\
4.00991629107814	-9.2912817129399 \\
3.95372497659066	-9.29095006366954 \\
3.89649268300739	-9.29040329054813 \\
3.83823159425763	-9.28958761261292 \\
3.77895389427065	-9.28844924890115 \\
3.71867176697576	-9.28693441845005 \\
3.65739739630224	-9.28498934029686 \\
3.59514296617939	-9.28256023347883 \\
3.53192066053648	-9.27959331703318 \\
3.46774266330282	-9.27603480999716 \\
3.40262115840769	-9.27183093140801 \\
3.33656832978038	-9.26692790030297 \\
3.26959636135019	-9.26127193571927 \\
3.2017174370464	-9.25480925669416 \\
3.1329437407983	-9.24748608226487 \\
3.06328745653519	-9.23924863146865 \\
2.99276076818635	-9.23004312334273 \\
2.92137585968107	-9.21981577692435 \\
2.84914491494865	-9.20851281125075 \\
2.77608011791836	-9.19608044535917 \\
2.70219365251952	-9.18246489828685 \\
2.62749770268139	-9.16761238907102 \\
2.55200445233328	-9.15146913674893 \\
2.47572608540447	-9.13398136035782 \\
2.39867478582426	-9.11509527893493 \\
2.32086273752193	-9.09475711151748 \\
2.24230212442677	-9.07291307714273 \\
2.16300513046808	-9.04950939484791 \\
2.08298393957514	-9.02449228367026 \\
2.00225073567725	-8.99780796264702 \\
1.92081770270369	-8.96940265081543 \\
1.83869702458375	-8.93922256721272 \\
1.75590088524673	-8.90721393087615 \\
1.67244146862191	-8.87332296084294 \\
1.58833095863859	-8.83749587615033 \\
1.50358153922605	-8.79967889583557 \\
1.41820539431359	-8.75981823893589 \\
1.33221470783049	-8.71786012448853 \\
1.24562166370605	-8.67375077153073 \\
1.15843844586955	-8.62743639909973 \\
1.07067723825028	-8.57886322623277 \\
0.982350224777542	-8.52797747196709 \\
0.893469589380617	-8.47472535533993 \\
0.804047515988798	-8.41905309538852 \\
0.714096188531374	-8.36090691115011 \\
0.623627790937635	-8.30023302166193 \\
0.532654507136871	-8.23697764596122 \\
};
\addplot [blue, line width=1.2pt, dotted, forget plot]
table [row sep=\\]{%
-28.2192111248953	12.2563817269873 \\
-28.0795056950692	12.1891021836207 \\
-27.9398140711411	12.1218170754679 \\
-27.8001362551771	12.054526403214 \\
-27.6604722492434	11.9872301675438 \\
-27.5208220554064	11.9199283691423 \\
-27.3811856757322	11.8526210086944 \\
-27.2415631122871	11.7853080868852 \\
-27.1019543671371	11.7179896043995 \\
-26.9623594423487	11.6506655619223 \\
-26.822778339988	11.5833359601385 \\
-26.6832110621212	11.5160007997332 \\
-26.5436576108144	11.4486600813911 \\
-26.4041179881341	11.3813138057974 \\
-26.2645921961463	11.3139619736369 \\
-26.1250802369173	11.2466045855946 \\
-25.9855821125133	11.1792416423554 \\
-25.8460978250005	11.1118731446043 \\
-25.7066273764452	11.0444990930262 \\
-25.5671707689135	10.9771194883061 \\
-25.4277280044717	10.909734331129 \\
-25.288299085186	10.8423436221797 \\
-25.1488840131226	10.7749473621432 \\
-25.0094827903477	10.7075455517045 \\
-24.8700954189276	10.6401381915485 \\
-24.7307219009284	10.5727252823601 \\
-24.5913622384164	10.5053068248244 \\
-24.4520164334579	10.4378828196262 \\
-24.3126844881189	10.3704532674506 \\
-24.1733664044658	10.3030181689824 \\
-24.0340621845647	10.2355775249066 \\
-23.8947718304819	10.1681313359081 \\
-23.7554953442837	10.100679602672 \\
-23.6162327280361	10.0332223258831 \\
-23.4769839838054	9.96575950622639 \\
-23.3377491136579	9.89829114438684 \\
-23.1985281196598	9.83081724104937 \\
-23.0593210038772	9.76333779689895 \\
-22.9201277683765	9.69585281262051 \\
-22.7809484152238	9.62836228889901 \\
-22.6417829464853	9.5608662264194 \\
-22.5026313642272	9.49336462586661 \\
-22.3634936705159	9.42585748792561 \\
-22.2243698674174	9.35834481328134 \\
-22.0852599569981	9.29082660261874 \\
-21.9461639413241	9.22330285662276 \\
-21.8070818224616	9.15577357597836 \\
-21.6680136024769	9.08823876137048 \\
-21.5289592834362	9.02069841348406 \\
-21.3899188674057	8.95315253300406 \\
-21.2508923564516	8.88560112061543 \\
-21.1118334366836	8.81803351403494 \\
-20.9726995694837	8.75043991956193 \\
-20.8334964198261	8.68282164075523 \\
-20.694229652685	8.61517998117366 \\
-20.5549049330348	8.54751624437601 \\
-20.4155279258496	8.47983173392112 \\
-20.2761042961039	8.41212775336779 \\
-20.1366397087719	8.34440560627483 \\
-19.9971398288278	8.27666659620107 \\
-19.857610321246	8.2089120267053 \\
-19.7180568510006	8.14114320134636 \\
-19.5784850830661	8.07336142368304 \\
-19.4389006824167	8.00556799727418 \\
-19.2993093140266	7.93776422567857 \\
-19.1597166428702	7.86995141245504 \\
-19.0201283339217	7.8021308611624 \\
-18.8805500521554	7.73430387535946 \\
-18.7409874625456	7.66647175860504 \\
-18.6014462300666	7.59863581445794 \\
-18.4619320196926	7.530797346477 \\
-18.322450496398	7.46295765822101 \\
-18.183007325157	7.39511805324879 \\
-18.0436081709438	7.32727983511916 \\
-17.9042586987329	7.25944430739093 \\
-17.7649645734984	7.19161277362291 \\
-17.6257314602146	7.12378653737392 \\
-17.4865650238558	7.05596690220278 \\
-17.3474709293964	6.98815517166829 \\
-17.2084548418105	6.92035264932928 \\
-17.0695224260725	6.85256063874455 \\
-16.9306793471566	6.78478044347291 \\
-16.7919312700372	6.71701336707319 \\
-16.6532838596884	6.6492607131042 \\
-16.5147427810847	6.58152378512475 \\
-16.3763136992001	6.51380388669365 \\
-16.2380022790092	6.44610232136973 \\
-16.0998141854861	6.37842039271178 \\
-15.961755083605	6.31075940427863 \\
-15.8238306383404	6.2431206596291 \\
-15.6860465146664	6.17550546232199 \\
-15.5484083775574	6.10791511591612 \\
-15.4109218919876	6.0403509239703 \\
-15.2735927229314	5.97281419004335 \\
-15.1364265353629	5.90530621769408 \\
-14.9994289942565	5.8378283104813 \\
-14.8626057645864	5.77038177196384 \\
-14.725962511327	5.7029679057005 \\
-14.5895048994525	5.63558801525009 \\
-14.4532385939372	5.56824340417144 \\
-14.3171692597554	5.50093537602335 \\
-14.1797796163911	5.43407467159751 \\
-14.0396688289051	5.36803682686182 \\
-13.8970172306361	5.30277033813941 \\
-13.752005154923	5.23822370175344 \\
-13.6048129351047	5.17434541402703 \\
-13.4556209045202	5.11108397128332 \\
-13.3046093965082	5.04838786984547 \\
-13.1519587444077	4.98620560603661 \\
-12.9978492815576	4.92448567617988 \\
-12.8424613412968	4.86317657659842 \\
-12.6859752569641	4.80222680361536 \\
-12.5285713618985	4.74158485355386 \\
-12.3704299894387	4.68119922273706 \\
-12.2117314729238	4.62101840748808 \\
-12.0526561456926	4.56099090413008 \\
-11.893384341084	4.50106520898619 \\
-11.7340963924368	4.44118981837956 \\
-11.5749726330901	4.38131322863332 \\
-11.4161933963825	4.32138393607061 \\
-11.2579390156532	4.26135043701458 \\
-11.1003898242408	4.20116122778837 \\
-10.9437261554844	4.14076480471512 \\
-10.7881283427228	4.08010966411796 \\
-10.6337767192948	4.01914430232004 \\
-10.4808516185395	3.95781721564449 \\
-10.3295333737956	3.89607690041447 \\
-10.1800023184021	3.8338718529531 \\
-10.0324387856979	3.77115056958354 \\
-9.88702310902175	3.70786154662891 \\
-9.74393562171266	3.64395328041237 \\
-9.60335665710948	3.57937426725704 \\
-9.46546654855109	3.51407300348608 \\
-9.3304456293764	3.44799798542262 \\
-9.19847423292429	3.3810977093898 \\
-9.06973269253364	3.31332067171077 \\
-8.94440134154335	3.24461536870866 \\
-8.8226605132923	3.17493029670661 \\
-8.70469054111938	3.10421395202777 \\
-8.59067175836348	3.03241483099527 \\
-8.48078449836348	2.95948142993226 \\
-8.37520909445829	2.88536224516187 \\
-8.27412587998677	2.81000577300725 \\
-8.17771518828783	2.73336050979154 \\
-8.08615735270035	2.65537495183788 \\
-7.99963270656321	2.5759975954694 \\
-7.91832158321531	2.49517693700925 \\
-7.84240431599554	2.41286147278058 \\
-7.77206123824278	2.32899969910651 \\
-7.70747268329592	2.24354011231019 \\
-7.64881898449385	2.15643120871476 \\
-7.59365495581358	2.0668564522924 \\
-7.53941090800453	1.97410022906793 \\
-7.4860804195642	1.87826241844341 \\
-7.43365706899012	1.77944289982093 \\
-7.3821344347798	1.67774155260255 \\
-7.33150609543073	1.57325825619036 \\
-7.28176562944043	1.46609288998644 \\
-7.23290661530642	1.35634533339285 \\
-7.18492263152621	1.24411546581168 \\
-7.13780725659729	1.129503166645 \\
-7.0915540690172	1.0126083152949 \\
-7.04615664728342	0.893530791163434 \\
-7.00160856989349	0.772370473652695 \\
-6.9579034153449	0.649227242164755 \\
-6.91503476213516	0.524200976101691 \\
-6.87299618876179	0.397391554865579 \\
-6.83178127372231	0.268898857858497 \\
-6.79138359551421	0.138822764482521 \\
-6.75179673263501	0.00726315413972619 \\
-6.71301426358221	-0.125680093767809 \\
-6.67502976685334	-0.259907099838009 \\
-6.6378368209459	-0.395317984668797 \\
-6.6014290043574	-0.531812868858096 \\
-6.56579989558535	-0.669291873003829 \\
-6.53094307312726	-0.807655117703921 \\
-6.49685211548065	-0.946802723556294 \\
-6.46352060114301	-1.08663481115887 \\
-6.43094210861188	-1.22705150110958 \\
-6.39911021638474	-1.36795291400634 \\
-6.36801850295912	-1.50923917044707 \\
-6.33766054683252	-1.6508103910297 \\
-6.30802992650246	-1.79256669635215 \\
-6.27912022046645	-1.93440820701235 \\
-6.25092500722199	-2.07623504360822 \\
-6.2234378652666	-2.21794732673768 \\
-6.19665237309779	-2.35944517699865 \\
-6.17056210921306	-2.50062871498906 \\
-6.14516065210993	-2.64139806130684 \\
-6.12044158028592	-2.7816533365499 \\
-6.09639847223852	-2.92129466131617 \\
-6.07302490646525	-3.06022215620357 \\
-6.05031446146362	-3.19833594181003 \\
-6.02826071573115	-3.33553613873346 \\
-6.00685724776533	-3.4717228675718 \\
-5.98609763606369	-3.60679624892297 \\
-5.96597545912373	-3.74065640338488 \\
-5.94648429544296	-3.87320345155547 \\
-5.92761772351889	-4.00433751403265 \\
-5.90936932184904	-4.13395871141435 \\
-5.89173266893092	-4.2619671642985 \\
-5.87479747062155	-4.39155878267256 \\
-5.85863476443925	-4.52579635907598 \\
-5.84321012637347	-4.66438033793461 \\
-5.82848913241363	-4.80701116367431 \\
-5.81443735854916	-4.95338928072094 \\
-5.80102038076951	-5.10321513350037 \\
-5.7882037750641	-5.25618916643844 \\
-5.77595311742237	-5.41201182396102 \\
-5.76423398383376	-5.57038355049397 \\
-5.7530119502877	-5.73100479046314 \\
-5.74225259277362	-5.8935759882944 \\
-5.73192148728095	-6.05779758841361 \\
-5.72198420979914	-6.22337003524662 \\
-5.71240633631762	-6.38999377321929 \\
-5.70315344282582	-6.55736924675749 \\
-5.69419110531317	-6.72519690028706 \\
-5.68548489976911	-6.89317717823389 \\
-5.67700040218308	-7.0610105250238 \\
-5.66870318854451	-7.22839738508268 \\
-5.66055883484283	-7.39503820283638 \\
-5.65253291706748	-7.56063342271075 \\
-5.64459101120789	-7.72488348913166 \\
-5.6366986932535	-7.88748884652497 \\
-5.62882153919373	-8.04814993931652 \\
-5.62092512501803	-8.2065672119322 \\
-5.61297502671583	-8.36244110879784 \\
-5.60493682027657	-8.51547207433932 \\
-5.59677608168967	-8.6653605529825 \\
-5.58845838694458	-8.81180698915322 \\
-5.57994931203072	-8.95451182727735 \\
-5.57121443293754	-9.09317551178075 \\
-5.56221932565446	-9.22749848708928 \\
-5.55292956617093	-9.35718119762879 \\
-5.54331073047636	-9.48192408782516 \\
-5.53332839456021	-9.60142760210423 \\
-5.5229481344119	-9.71539218489186 \\
-5.51213552602087	-9.82351828061391 \\
-5.50085614537655	-9.92550633369626 \\
-5.48907556846837	-10.0210567885647 \\
-5.47675937128578	-10.1098700896452 \\
-5.46387312981821	-10.1916466813636 \\
-5.45038242005508	-10.2660870081456 \\
-5.43625281798584	-10.3328915144173 \\
-5.42144989959992	-10.3917606446044 \\
-5.40593924088675	-10.4423948431327 \\
-5.38968641783577	-10.4844945544283 \\
-5.37265700643641	-10.5177602229168 \\
-5.3548165826781	-10.5418922930242 \\
-5.33613072255029	-10.5565912091764 \\
-5.31656500204241	-10.5615574157991 \\
-5.29364607710647	-10.559943916489 \\
-5.2650673651573	-10.5551549445166 \\
-5.23104458437856	-10.547267788819 \\
-5.19179345295387	-10.5363597383334 \\
-5.14752968906688	-10.5225080819968 \\
-5.09846901090123	-10.5057901087463 \\
-5.04482713664056	-10.486283107519 \\
-4.98681978446852	-10.464064367252 \\
-4.92466267256874	-10.4392111768825 \\
-4.85857151912487	-10.4118008253474 \\
-4.78876204232054	-10.3819106015839 \\
-4.7154499603394	-10.3496177945291 \\
-4.63885099136509	-10.3149996931201 \\
-4.55918085358125	-10.2781335862939 \\
-4.47665526517152	-10.2390967629877 \\
-4.39148994431955	-10.1979665121385 \\
-4.30390060920897	-10.1548201226835 \\
-4.21410297802342	-10.1097348835597 \\
-4.12231276894655	-10.0627880837042 \\
-4.028745700162	-10.0140570120542 \\
-3.93361748985341	-9.96361895754668 \\
-3.83714385620442	-9.91155120911876 \\
-3.73954051739867	-9.85793105570755 \\
-3.64102319161981	-9.80283578625013 \\
-3.54180759705147	-9.7463426896836 \\
-3.44210945187729	-9.68852905494505 \\
-3.34214447428092	-9.62947217097158 \\
-3.242128382446	-9.56924932670027 \\
-3.14227689455617	-9.50793781106822 \\
-3.04280572879506	-9.44561491301252 \\
-2.94393060334633	-9.38235792147027 \\
-2.84586723639362	-9.31824412537855 \\
-2.74883134612055	-9.25335081367447 \\
-2.65303865071079	-9.18775527529511 \\
-2.55870486834796	-9.12153479917757 \\
-2.4660457172157	-9.05476667425894 \\
-2.37527691549767	-8.98752818947631 \\
-2.2866141813775	-8.91989663376678 \\
-2.20027323303883	-8.85194929606744 \\
-2.1164697886653	-8.78376346531538 \\
-2.03541956644055	-8.7154164304477 \\
-1.95733828454823	-8.64698548040149 \\
-1.88244166117198	-8.57854790411383 \\
-1.81094541449544	-8.51018099052184 \\
-1.74306526270224	-8.44196202856258 \\
-1.67901692397604	-8.37396830717317 \\
-1.61901611650047	-8.30627711529069 \\
-1.56327855845917	-8.23896574185224 \\
-1.51201996803578	-8.17211147579491 \\
-1.46545606341395	-8.10579160605579 \\
-1.42138275536189	-8.03889715673572 \\
-1.37746787951477	-7.97028183561159 \\
-1.33373504135666	-7.89996695713458 \\
-1.29020784637163	-7.82797383575588 \\
-1.24690990004376	-7.75432378592665 \\
-1.20386480785712	-7.67903812209809 \\
-1.16109617529578	-7.60213815872137 \\
-1.11862760784382	-7.52364521024767 \\
-1.07648271098531	-7.44358059112817 \\
-1.03468509020431	-7.36196561581405 \\
-0.993258350984915	-7.27882159875649 \\
-0.952226098811183	-7.19416985440667 \\
-0.911611939167191	-7.10803169721577 \\
-0.871439477537011	-7.02042844163497 \\
-0.831732319404717	-6.93138140211546 \\
-0.792514070254382	-6.8409118931084 \\
-0.753808335570078	-6.74904122906498 \\
-0.715638720835878	-6.65579072443639 \\
-0.678028831535856	-6.56118169367379 \\
-0.641002273154083	-6.46523545122837 \\
-0.604582651174634	-6.36797331155131 \\
-0.56879357108158	-6.26941658909379 \\
-0.533658638358994	-6.16958659830699 \\
-0.49920145849095	-6.06850465364209 \\
-0.465445636961521	-5.96619206955026 \\
-0.432414779254779	-5.8626701604827 \\
-0.400132490854797	-5.75796024089057 \\
-0.368622377245648	-5.65208362522506 \\
-0.337908043911405	-5.54506162793735 \\
-0.30801309633614	-5.43691556347862 \\
-0.278961140003928	-5.32766674630005 \\
-0.25077578039884	-5.21733649085281 \\
-0.223480623004949	-5.1059461115881 \\
-0.197099273306328	-4.99351692295707 \\
-0.171655336787051	-4.88007023941093 \\
-0.14717241893119	-4.76562737540085 \\
-0.123674125222818	-4.650209645378 \\
-0.101184061146007	-4.53383836379357 \\
-0.0797258321848316	-4.41653484509873 \\
-0.0593230438233634	-4.29832040374468 \\
-0.0399993015456757	-4.17921635418258 \\
-0.0217782108358413	-4.05924401086361 \\
-0.00468337717793302	-3.93842468823897 \\
0.0112615939439761	-3.81677970075981 \\
0.0260330970458134	-3.69433036287734 \\
0.0396075266435057	-3.57109798904272 \\
0.0519612772529803	-3.44710389370714 \\
0.0630707433901644	-3.32236939132177 \\
0.072912319570985	-3.1969157963378 \\
};
\addplot [blue, line width=1.2pt, dotted, forget plot]
table [row sep=\\]{%
-21.6560009941107	-21.95057367142 \\
-21.5696456563005	-21.9120105519812 \\
-21.483240186937	-21.8735188640832 \\
-21.3967846282021	-21.8350988319073 \\
-21.3102790222774	-21.7967506796346 \\
-21.2237234113449	-21.7584746314466 \\
-21.1371178375864	-21.7202709115243 \\
-21.0504623431835	-21.6821397440491 \\
-20.9637569703182	-21.6440813532023 \\
-20.8770017611722	-21.606095963165 \\
-20.7901967579273	-21.5681837981185 \\
-20.7033420027654	-21.5303450822442 \\
-20.6164375378682	-21.4925800397231 \\
-20.5294834054175	-21.4548888947367 \\
-20.4424796475951	-21.4172718714661 \\
-20.3554263065829	-21.3797291940926 \\
-20.2683234245626	-21.3422610867975 \\
-20.181171043716	-21.304867773762 \\
-20.093969206225	-21.2675494791673 \\
-20.0067179542713	-21.2303064271947 \\
-19.9194173300367	-21.1931388420256 \\
-19.8320673757031	-21.156046947841 \\
-19.7446681334522	-21.1190309688223 \\
-19.6572196454659	-21.0820911291507 \\
-19.5697219539259	-21.0452276530075 \\
-19.482175101014	-21.008440764574 \\
-19.3945791289121	-20.9717306880313 \\
-19.3069340798019	-20.9350976475608 \\
-19.2192399958653	-20.8985418673437 \\
-19.131496919284	-20.8620635715613 \\
-19.0437048922398	-20.8256629843947 \\
-18.9558639569146	-20.7893403300253 \\
-18.8679741554902	-20.7530958326343 \\
-18.7800355301483	-20.716929716403 \\
-18.6920481230707	-20.6808422055126 \\
-18.6040119764393	-20.6448335241443 \\
-18.5159271324359	-20.6089038964795 \\
-18.4277936332422	-20.5730535466994 \\
-18.33961152104	-20.5372826989852 \\
-18.2513808380112	-20.5015915775182 \\
-18.1631016263376	-20.4659804064796 \\
-18.0747739282009	-20.4304494100507 \\
-17.9863732606453	-20.3950110449586 \\
-17.8978770279864	-20.359676159338 \\
-17.8092881033137	-20.324442564482 \\
-17.7206093597163	-20.2893080716836 \\
-17.6318436702835	-20.2542704922357 \\
-17.5429939081046	-20.2193276374313 \\
-17.4540629462689	-20.1844773185634 \\
-17.3650536578656	-20.1497173469249 \\
-17.2759689159839	-20.1150455338089 \\
-17.1868115937133	-20.0804596905084 \\
-17.0975845641429	-20.0459576283162 \\
-17.008290700362	-20.0115371585254 \\
-16.9189328754599	-19.977196092429 \\
-16.8295139625258	-19.9429322413199 \\
-16.7400368346491	-19.9087434164911 \\
-16.6505043649189	-19.8746274292357 \\
-16.5609194264246	-19.8405820908465 \\
-16.4712848922554	-19.8066052126166 \\
-16.3816036355007	-19.7726946058389 \\
-16.2918785292496	-19.7388480818064 \\
-16.2021124465915	-19.7050634518122 \\
-16.1123082606156	-19.6713385271491 \\
-16.0224688444111	-19.6376711191102 \\
-15.9325970710674	-19.6040590389884 \\
-15.8426958136738	-19.5705000980767 \\
-15.7527679453194	-19.5369921076681 \\
-15.6628163390936	-19.5035328790556 \\
-15.5728438680857	-19.4701202235321 \\
-15.4828534053849	-19.4367519523907 \\
-15.3928478240804	-19.4034258769242 \\
-15.3028299972616	-19.3701398084258 \\
-15.2128027980177	-19.3368915581883 \\
-15.122769099438	-19.3036789375048 \\
-15.0327317746117	-19.2704997576682 \\
-14.9426936966282	-19.2373518299715 \\
-14.8526577385767	-19.2042329657077 \\
-14.7626267735465	-19.1711409761698 \\
-14.6726036746268	-19.1380736726507 \\
-14.5825913149069	-19.1050288664435 \\
-14.4925925674762	-19.072004368841 \\
-14.4026103054237	-19.0389979911363 \\
-14.3126474018389	-19.0060075446224 \\
-14.222706729811	-18.9730308405922 \\
-14.1327911624293	-18.9400656903388 \\
-14.042903572783	-18.907109905155 \\
-13.9530468339614	-18.8741612963339 \\
-13.8632238190537	-18.8412176751685 \\
-13.7734374011494	-18.8082768529517 \\
-13.6836904533375	-18.7753366409765 \\
-13.5939858487074	-18.742394850536 \\
-13.5042275900578	-18.7094879191268 \\
-13.4143238421042	-18.676652045767 \\
-13.324283720811	-18.6438846840309 \\
-13.2341163421428	-18.6111832874933 \\
-13.1438308220642	-18.5785453097285 \\
-13.0534362765398	-18.5459682043112 \\
-12.9629418215339	-18.5134494248158 \\
-12.8723565730113	-18.480986424817 \\
-12.7816896469364	-18.4485766578892 \\
-12.6909501592738	-18.416217577607 \\
-12.600147225988	-18.3839066375449 \\
-12.5092899630436	-18.3516412912775 \\
-12.4183874864051	-18.3194189923793 \\
-12.327448912037	-18.2872371944248 \\
-12.2364833559039	-18.2550933509886 \\
-12.1454999339703	-18.2229849156452 \\
-12.0545077622008	-18.1909093419692 \\
-11.9635159565599	-18.158864083535 \\
-11.8725336330122	-18.1268465939173 \\
-11.7815699075221	-18.0948543266905 \\
-11.6906338960543	-18.0628847354292 \\
-11.5997347145733	-18.0309352737079 \\
-11.5088814790435	-17.9990033951012 \\
-11.4180833054297	-17.9670865531836 \\
-11.3273493096962	-17.9351822015297 \\
-11.2366886078077	-17.9032877937139 \\
-11.1461103157286	-17.8714007833108 \\
-11.0556235494236	-17.839518623895 \\
-10.9652374248572	-17.807638769041 \\
-10.8749610579938	-17.7757586723233 \\
-10.7848035647981	-17.7438757873165 \\
-10.6947740612346	-17.7119875675951 \\
-10.6048816632678	-17.6800914667336 \\
-10.5151354868623	-17.6481849383066 \\
-10.4255446479827	-17.6162654358886 \\
-10.3361182625933	-17.5843304130541 \\
-10.2468654466589	-17.5523773233778 \\
-10.1577953161439	-17.520403620434 \\
-10.0689169870129	-17.4884067577974 \\
-9.98023957523041	-17.4563841890426 \\
-9.89177219676099	-17.4243333677439 \\
-9.80352396756917	-17.392251747476 \\
-9.7155040036195	-17.3601367818134 \\
-9.62772142087652	-17.3279859243307 \\
-9.54018533530476	-17.2957966286024 \\
-9.45290486286878	-17.2635663482029 \\
-9.36588911953311	-17.231292536707 \\
-9.27914722126229	-17.198972647689 \\
-9.19268828402087	-17.1666041347235 \\
-9.10652142377339	-17.1341844513852 \\
-9.0204180897554	-17.101756075693 \\
-8.93415624352203	-17.0693612665864 \\
-8.84775476951716	-17.0369971490202 \\
-8.76123255218466	-17.004660847949 \\
-8.67460847596842	-16.9723494883277 \\
-8.58790142531232	-16.9400601951111 \\
-8.50113028466023	-16.9077900932538 \\
-8.41431393845604	-16.8755363077108 \\
-8.32747127114363	-16.8432959634367 \\
-8.24062116716688	-16.8110661853863 \\
-8.15378251096966	-16.7788440985145 \\
-8.06697418699586	-16.7466268277758 \\
-7.98021507968937	-16.7144114981253 \\
-7.89352407349405	-16.6821952345175 \\
-7.80692005285378	-16.6499751619073 \\
-7.72042190221246	-16.6177484052495 \\
-7.63404850601396	-16.5855120894987 \\
-7.54781874870216	-16.5532633396099 \\
-7.46175151472093	-16.5209992805377 \\
-7.37586568851417	-16.488717037237 \\
-7.29018015452575	-16.4564137346624 \\
-7.20471379719955	-16.4240864977689 \\
-7.11948550097945	-16.391732451511 \\
-7.03451415030933	-16.3593487208437 \\
-6.94981862963307	-16.3269324307217 \\
-6.86541782339455	-16.2944807060997 \\
-6.78133061603766	-16.2619906719325 \\
-6.69757589200626	-16.2294594531749 \\
-6.61417253574425	-16.1968841747817 \\
-6.5311394316955	-16.1642619617076 \\
-6.4484954643039	-16.1315899389075 \\
-6.36625951801331	-16.098865231336 \\
-6.28445047726763	-16.0660849639479 \\
-6.20308722651074	-16.0332462616981 \\
-6.12218865018651	-16.0003462495412 \\
-6.04177363273882	-15.9673820524322 \\
-5.96186105861155	-15.9343507953256 \\
-5.88246981224859	-15.9012496031763 \\
-5.80361877809381	-15.8680756009391 \\
-5.7253268405911	-15.8348259135688 \\
-5.64761288418434	-15.8014976660201 \\
-5.57049579331739	-15.7680879832477 \\
-5.49399445243416	-15.7345939902065 \\
-5.41812774597851	-15.7010128118512 \\
-5.34291455839432	-15.6673415731366 \\
-5.26837377412548	-15.6335773990175 \\
-5.19452427761587	-15.5997174144486 \\
-5.12138495330937	-15.5657587443847 \\
-5.04897468564985	-15.5316985137806 \\
-4.9773123590812	-15.497533847591 \\
-4.90581515015511	-15.4634553092043 \\
-4.83391921867002	-15.429641352358 \\
-4.76167192393995	-15.396070937531 \\
-4.6891206252789	-15.3627230252023 \\
-4.61631268200086	-15.3295765758506 \\
-4.54329545341985	-15.2966105499548 \\
-4.47011629884988	-15.2638039079937 \\
-4.39682257760494	-15.2311356104463 \\
-4.32346164899904	-15.1985846177914 \\
-4.25008087234618	-15.1661298905079 \\
-4.17672760696038	-15.1337503890745 \\
-4.10344921215563	-15.1014250739703 \\
-4.03029304724595	-15.0691329056739 \\
-3.95730647154533	-15.0368528446644 \\
-3.88453684436778	-15.0045638514204 \\
-3.8120315250273	-14.972244886421 \\
-3.73983787283791	-14.939874910145 \\
-3.6680032471136	-14.9074328830712 \\
-3.59657500716838	-14.8748977656784 \\
-3.52560051231626	-14.8422485184456 \\
-3.45512712187124	-14.8094641018516 \\
-3.38520219514732	-14.7765234763753 \\
-3.31587309145851	-14.7434056024955 \\
-3.24718717011882	-14.7100894406911 \\
-3.17919179044225	-14.6765539514409 \\
-3.1119343117428	-14.6427780952238 \\
-3.04546209333448	-14.6087408325187 \\
-2.9798224945313	-14.5744211238044 \\
-2.91506287464725	-14.5397979295597 \\
-2.85123059299635	-14.5048502102636 \\
-2.7883730088926	-14.4695569263949 \\
-2.72653748165001	-14.4338970384325 \\
-2.66577137058257	-14.3978495068552 \\
-2.6061220350043	-14.3613932921418 \\
-2.5476368342292	-14.3245073547713 \\
-2.49036312757127	-14.2871706552225 \\
-2.43434827434452	-14.2493621539742 \\
-2.37963963386295	-14.2110608115053 \\
-2.32628456544057	-14.1722455882947 \\
-2.27433042839139	-14.1328954448213 \\
-2.2238245820294	-14.0929893415638 \\
-2.17481438566862	-14.0525062390012 \\
-2.12734719862305	-14.0114250976123 \\
-2.08147038020669	-13.9697248778759 \\
-2.03723128973355	-13.927384540271 \\
-1.99467728651763	-13.8843830452764 \\
-1.95385572987294	-13.8406993533709 \\
-1.91481397911349	-13.7963124250334 \\
-1.87759939355327	-13.7512012207428 \\
-1.84225933250629	-13.7053447009779 \\
-1.80824162211771	-13.6583512633953 \\
-1.77494983388879	-13.6098743246491 \\
-1.74236494516774	-13.5599603737146 \\
-1.71046793330277	-13.5086558995672 \\
-1.67923977564208	-13.4560073911821 \\
-1.64866144953387	-13.4020613375348 \\
-1.61871393232635	-13.3468642276006 \\
-1.58937820136773	-13.2904625503547 \\
-1.5606352340062	-13.2329027947726 \\
-1.53246600758996	-13.1742314498294 \\
-1.50485149946724	-13.1144950045007 \\
-1.47777268698622	-13.0537399477616 \\
-1.45121054749511	-12.9920127685876 \\
-1.42514605834212	-12.9293599559539 \\
-1.39956019687545	-12.8658279988359 \\
-1.37443394044331	-12.8014633862089 \\
-1.34974826639389	-12.7363126070483 \\
-1.32548415207541	-12.6704221503293 \\
-1.30162257483606	-12.6038385050273 \\
-1.27814451202405	-12.5366081601176 \\
-1.25503094098759	-12.4687776045756 \\
-1.23226283907488	-12.4003933273766 \\
-1.20982118363412	-12.3315018174959 \\
-1.18768695201352	-12.2621495639088 \\
-1.16584112156128	-12.1923830555906 \\
-1.1442646696256	-12.1222487815168 \\
-1.12293857355469	-12.0517932306626 \\
-1.10184381069676	-11.9810628920033 \\
-1.0809613584	-11.9101042545143 \\
-1.06027219401263	-11.838963807171 \\
-1.03975729488284	-11.7676880389485 \\
-1.01939763835884	-11.6963234388223 \\
-0.999174201788833	-11.6249164957677 \\
-0.979067962521024	-11.5535136987601 \\
-0.959059897903617	-11.4821615367747 \\
-0.939130985284816	-11.4109064987868 \\
-0.919262202012825	-11.3397950737719 \\
-0.899434525435848	-11.2688737507052 \\
-0.879628932902088	-11.198189018562 \\
-0.859826401759751	-11.1277873663178 \\
-0.84000790935704	-11.0577152829478 \\
-0.820154433042159	-10.9880192574273 \\
-0.800246950163313	-10.9187457787317 \\
-0.780266438068704	-10.8499413358363 \\
-0.760193874106539	-10.7816524177164 \\
-0.740010235625019	-10.7139255133474 \\
-0.719696499972351	-10.6468071117046 \\
-0.699233644496736	-10.5803437017633 \\
-0.678602646546381	-10.5145817724989 \\
-0.657784483469488	-10.4495678128866 \\
-0.636904525269031	-10.3849529365497 \\
-0.616101288142305	-10.3203572703422 \\
-0.595375468728993	-10.2557808230852 \\
-0.574727763668781	-10.1912236036002 \\
-0.554158869601351	-10.1266856207082 \\
-0.53366948316639	-10.0621668832307 \\
-0.513260301003581	-9.99766739998887 \\
-0.492932019752608	-9.93318717980404 \\
-0.472685336053156	-9.86872623149748 \\
-0.45252094654491	-9.80428456389046 \\
-0.432439547867554	-9.73986218580427 \\
-0.412441836660771	-9.67545910606019 \\
-0.392528509564248	-9.61107533347949 \\
-0.372700263217667	-9.54671087688346 \\
-0.352957794260714	-9.48236574509338 \\
-0.333301799333073	-9.41803994693052 \\
-0.313732975074427	-9.35373349121617 \\
-0.294252018124463	-9.28944638677161 \\
-0.274859625122863	-9.22517864241812 \\
-0.255556492709312	-9.16093026697697 \\
-0.236343317523496	-9.09670126926945 \\
-0.217220796205098	-9.03249165811684 \\
-0.198189625393802	-8.96830144234041 \\
-0.179250501729293	-8.90413063076145 \\
-0.160404121851255	-8.83997923220124 \\
-0.141651182399374	-8.77584725548106 \\
-0.122992380013332	-8.71173470942218 \\
-0.104428411332815	-8.6476416028459 \\
-0.0859599729975071	-8.58356794457348 \\
-0.0675877616470925	-8.51951374342621 \\
-0.0493124739212555	-8.45547900822537 \\
-0.0311348064596807	-8.39146374779223 \\
-0.0130554559020525	-8.32746797094809 \\
0.00492488111194484	-8.26349168651421 \\
0.0228055079426267	-8.19953490331188 \\
0.0405857279503089	-8.13559763016238 \\
0.058264844495307	-8.07167987588699 \\
0.0758421609379366	-8.00778164930699 \\
0.0933169806385132	-7.94390295924366 \\
0.110688606957352	-7.88004381451827 \\
0.12795634325477	-7.81620422395212 \\
0.145119492891081	-7.75238419636647 \\
0.162177359226602	-7.68858374058262 \\
0.179129245621648	-7.62480286542183 \\
0.195974455436535	-7.56104157970539 \\
0.212712292031578	-7.49729989225458 \\
0.229342058767093	-7.43357781189069 \\
0.245863059003395	-7.36987534743498 \\
0.2622745961008	-7.30619250770874 \\
};
\addplot [blue, line width=1.2pt, dotted, forget plot]
table [row sep=\\]{%
-15.779763530623	25.849615460525 \\
-15.698363747378	25.7074990257949 \\
-15.6169060527878	25.5655122976994 \\
-15.5353905047273	25.4236554712973 \\
-15.4538171610717	25.281928741647 \\
-15.3721860796957	25.1403323038072 \\
-15.2904973184746	24.9988663528364 \\
-15.2087509352832	24.8575310837932 \\
-15.1269469879966	24.7163266917361 \\
-15.0450855344898	24.5752533717239 \\
-14.9631666326378	24.4343113188149 \\
-14.8811903403156	24.2935007280678 \\
-14.7991567153981	24.1528217945412 \\
-14.7170658157605	24.0122747132937 \\
-14.6349176992776	23.8718596793838 \\
-14.5527124238246	23.73157688787 \\
-14.4704500472763	23.591426533811 \\
-14.3881306275078	23.4514088122654 \\
-14.3057542223942	23.3115239182917 \\
-14.2233208898103	23.1717720469485 \\
-14.1408306876313	23.0321533932943 \\
-14.0582836737321	22.8926681523878 \\
-13.9756799059877	22.7533165192875 \\
-13.8930194422731	22.614098689052 \\
-13.8103023404634	22.4750148567399 \\
-13.7275286584334	22.3360652174097 \\
-13.6446984540583	22.19724996612 \\
-13.5618117852131	22.0585692979294 \\
-13.4788687097726	21.9200234078965 \\
-13.395869285612	21.7816124910798 \\
-13.3128135706063	21.6433367425379 \\
-13.2297016226304	21.5051963573294 \\
-13.1465334995593	21.3671915305128 \\
-13.0633092592681	21.2293224571468 \\
-12.9800289596317	21.0915893322899 \\
-12.8966926585252	20.9539923510007 \\
-12.8133004138236	20.8165317083378 \\
-12.7298522834018	20.6792075993596 \\
-12.6463483251349	20.5420202191249 \\
-12.5627885968978	20.4049697626922 \\
-12.4791731565656	20.26805642512 \\
-12.3955020620133	20.131280401467 \\
-12.3117753711158	19.9946418867917 \\
-12.2279931417483	19.8581410761526 \\
-12.1441554317856	19.7217781646084 \\
-12.0602622991028	19.5855533472176 \\
-11.9763138015748	19.4494668190389 \\
-11.8923099970768	19.3135187751307 \\
-11.8082509434837	19.1777094105516 \\
-11.7241366986704	19.0420389203603 \\
-11.6399673205121	18.9065074996153 \\
-11.5547293004266	18.7712724354009 \\
-11.4674917124679	18.636477660191 \\
-11.3783784884649	18.5021033371284 \\
-11.2875135602465	18.368129629356 \\
-11.1950208596417	18.2345367000168 \\
-11.1010243184793	18.1013047122534 \\
-11.0056478685882	17.9684138292089 \\
-10.9090154417974	17.8358442140261 \\
-10.8112509699358	17.7035760298479 \\
-10.7124783848323	17.5715894398172 \\
-10.6128216183158	17.4398646070767 \\
-10.5124046022152	17.3083816947695 \\
-10.4113512683594	17.1771208660383 \\
-10.3097855485774	17.0460622840261 \\
-10.2078313746981	16.9151861118757 \\
-10.1056126785503	16.78447251273 \\
-10.003253391963	16.6539016497319 \\
-9.90087744676506	16.5234536860242 \\
-9.79860877478546	16.3931087847499 \\
-9.69657130785308	16.2628471090517 \\
-9.59488897779683	16.1326488220726 \\
-9.49368571644563	16.0024940869554 \\
-9.39308545562839	15.872363066843 \\
-9.29321212717403	15.7422359248783 \\
-9.19418966291145	15.6120928242042 \\
-9.09614199466958	15.4819139279635 \\
-8.99919305427731	15.3516793992991 \\
-8.90346677356357	15.2213694013539 \\
-8.80908708435727	15.0909640972707 \\
-8.71617791848733	14.9604436501924 \\
-8.62486320778264	14.829788223262 \\
-8.53526688407214	14.6989779796222 \\
-8.44751287918472	14.567993082416 \\
-8.36172512494931	14.4368136947862 \\
-8.27802755319482	14.3054199798756 \\
-8.19654409575015	14.1737921008273 \\
-8.11739868444423	14.041910220784 \\
-8.04071525110596	13.9097545028886 \\
-7.96661772756426	13.7773051102839 \\
-7.89523004564804	13.644542206113 \\
-7.82667613718622	13.5114459535186 \\
-7.7610799340077	13.3779965156435 \\
-7.69856536794141	13.2441740556308 \\
-7.63925637081624	13.1099587366232 \\
-7.58327687446113	12.9753307217637 \\
-7.53075081070497	12.840270174195 \\
-7.48180211137669	12.7047572570601 \\
-7.43655470830518	12.5687721335019 \\
-7.39513253331938	12.4322949666632 \\
-7.35765951824819	12.2953059196869 \\
-7.32254275994515	12.1576950087313 \\
-7.28810033821927	12.0193859489329 \\
-7.2543226593327	11.8804094519022 \\
-7.22120012954757	11.7407962292492 \\
-7.18872315512603	11.6005769925845 \\
-7.1568821423302	11.4597824535181 \\
-7.12566749742224	11.3184433236604 \\
-7.09506962666427	11.1765903146217 \\
-7.06507893631845	11.0342541380122 \\
-7.0356858326469	10.8914655054423 \\
-7.00688072191177	10.7482551285223 \\
-6.97865401037519	10.6046537188623 \\
-6.9509961042993	10.4606919880728 \\
-6.92389740994625	10.3164006477639 \\
-6.89734833357817	10.1718104095459 \\
-6.87133928145719	10.0269519850293 \\
-6.84586065984547	9.88185608582409 \\
-6.82090287500513	9.73655342354075 \\
-6.79645633319832	9.59107470978949 \\
-6.77251144068718	9.44545065618061 \\
-6.74905860373383	9.29971197432439 \\
-6.72608822860044	9.15388937583111 \\
-6.70359072154912	9.00801357231106 \\
-6.68155648884202	8.86211527537451 \\
-6.65997593674128	8.71622519663175 \\
-6.63883947150904	8.57037404769307 \\
-6.61813749940743	8.42459254016874 \\
-6.5978604266986	8.27891138566905 \\
-6.57799865964468	8.13336129580429 \\
-6.55854260450782	7.98797298218473 \\
-6.53948266755015	7.84277715642066 \\
-6.52080925503381	7.69780453012236 \\
-6.50251277322094	7.55308581490012 \\
-6.48458362837367	7.40865172236422 \\
-6.46701222675416	7.26453296412493 \\
-6.44978897462453	7.12076025179255 \\
-6.43290427824692	6.97736429697736 \\
-6.41634854388348	6.83437581128964 \\
-6.40011217779634	6.69182550633967 \\
-6.38418558624764	6.54974409373774 \\
-6.36855917549952	6.40816228509413 \\
-6.35322335181412	6.26711079201912 \\
-6.33816852145357	6.12662032612299 \\
-6.32338509068002	5.98672159901604 \\
-6.30886346575561	5.84744532230853 \\
-6.29459405294247	5.70882220761076 \\
-6.28056725850274	5.57088296653301 \\
-6.26677348869857	5.43365831068556 \\
-6.25320314979208	5.29717895167869 \\
-6.23984664804542	5.16147560112269 \\
-6.22673375409789	5.02634036963349 \\
-6.21389547619257	4.8915418784455 \\
-6.20132407699731	4.75707560509665 \\
-6.18901181917996	4.62293702712491 \\
-6.17695096540835	4.48912162206823 \\
-6.16513377835034	4.35562486746455 \\
-6.15355252067376	4.22244224085183 \\
-6.14219945504647	4.08956921976803 \\
-6.13106684413629	3.95700128175109 \\
-6.12014695061108	3.82473390433898 \\
-6.10943203713868	3.69276256506964 \\
-6.09891436638693	3.56108274148102 \\
-6.08858620102368	3.42968991111109 \\
-6.07843980371677	3.29857955149778 \\
-6.06846743713405	3.16774714017906 \\
-6.05866136394335	3.03718815469288 \\
-6.04901384681252	2.90689807257718 \\
-6.03951714840941	2.77687237136993 \\
-6.03016353140186	2.64710652860908 \\
-6.02094525845771	2.51759602183257 \\
-6.01185459224481	2.38833632857837 \\
-6.00288379543099	2.25932292638442 \\
-5.99402513068411	2.13055129278868 \\
-5.985270860672	2.0020169053291 \\
-5.97661324806252	1.87371524154364 \\
-5.9680445555235	1.74564177897024 \\
-5.95955704572278	1.61779199514685 \\
-5.95114298132822	1.49016136761144 \\
-5.94279462500765	1.36274537390196 \\
-5.93450423942892	1.23553949155635 \\
-5.92626408725988	1.10853919811258 \\
-5.91806643116835	0.981739971108582 \\
-5.9099035338222	0.855137288082325 \\
-5.90176765788926	0.728726626571758 \\
-5.89365106603737	0.602503464114833 \\
-5.88554602093439	0.476463278249503 \\
-5.87744478524815	0.35060154651372 \\
-5.8693396216465	0.224913746445438 \\
-5.86122279279727	0.0993953555826099 \\
-5.85308656136832	-0.0259581485368118 \\
-5.84492319002749	-0.151151288374873 \\
-5.83672494144262	-0.276188586393624 \\
-5.82848407828156	-0.401074565055109 \\
-5.82019286321214	-0.525813746821376 \\
-5.81184355890222	-0.650410654154472 \\
-5.80342842801963	-0.774869809516444 \\
-5.79493973323223	-0.899195735369339 \\
-5.78636973720784	-1.0233929541752 \\
-5.77771070261432	-1.14746598839609 \\
-5.76895489211952	-1.27141936049404 \\
-5.76025879849358	-1.39526760077846 \\
-5.75177591337778	-1.51901882476286 \\
-5.74349399774663	-1.64266793271547 \\
-5.73540081257464	-1.7662098249045 \\
-5.72748411883632	-1.88963940159816 \\
-5.7197316775062	-2.01295156306466 \\
-5.71213124955877	-2.13614120957223 \\
-5.70467059596855	-2.25920324138908 \\
-5.69733747771006	-2.38213255878342 \\
-5.69011965575781	-2.50492406202347 \\
-5.68300489108629	-2.62757265137744 \\
-5.67598094467004	-2.75007322711355 \\
-5.66903557748357	-2.87242068950001 \\
-5.66215655050137	-2.99460993880504 \\
-5.65533162469797	-3.11663587529685 \\
-5.64854856104788	-3.23849339924366 \\
-5.64179512052561	-3.36017741091369 \\
-5.63505906410567	-3.48168281057514 \\
-5.62832815276258	-3.60300449849624 \\
-5.62159014747085	-3.72413737494519 \\
-5.61483280920498	-3.84507634019022 \\
-5.60804389893949	-3.96581629449953 \\
-5.6012111776489	-4.08635213814135 \\
-5.59432240630771	-4.20667877138389 \\
-5.58736534589044	-4.32679109449536 \\
-5.5803277573716	-4.44668400774398 \\
-5.5731974017257	-4.56635241139796 \\
-5.56596203992726	-4.68579120572552 \\
-5.55860943295078	-4.80499529099488 \\
-5.55112734177078	-4.92395956747424 \\
-5.54350352736176	-5.04267893543183 \\
-5.53572575069826	-5.16114829513585 \\
-5.52778177275476	-5.27936254685453 \\
-5.5196593545058	-5.39731659085608 \\
-5.51134625692587	-5.51500532740871 \\
-5.50283024098949	-5.63242365678064 \\
-5.49409906767117	-5.74956647924009 \\
-5.48514049794544	-5.86642869505526 \\
-5.47594229278678	-5.98300520449438 \\
-5.46649221316973	-6.09929090782565 \\
-5.45677802006879	-6.2152807053173 \\
-5.44678747445847	-6.33096949723754 \\
-5.43650833731329	-6.44635218385458 \\
-5.42592836960776	-6.56142366543665 \\
-5.41503533231638	-6.67617884225194 \\
-5.40381698641368	-6.79061261456869 \\
-5.39226109287417	-6.9047198826551 \\
-5.38035541267235	-7.01849554677939 \\
-5.36808770678274	-7.13193450720977 \\
-5.35544573617985	-7.24503166421446 \\
-5.34228016795151	-7.3598221975267 \\
-5.32845117304739	-7.47818278272874 \\
-5.31396076823474	-7.59987056386237 \\
-5.29881097028081	-7.72464268496936 \\
-5.28300379595284	-7.85225629009147 \\
-5.26654126201809	-7.98246852327049 \\
-5.24942538524381	-8.11503652854819 \\
-5.23165818239724	-8.24971744996634 \\
-5.21324167024565	-8.38626843156672 \\
-5.19417786555627	-8.5244466173911 \\
-5.17446878509636	-8.66400915148125 \\
-5.15411644563317	-8.80471317787894 \\
-5.13312286393396	-8.94631584062595 \\
-5.11149005676596	-9.08857428376406 \\
-5.08922004089643	-9.23124565133503 \\
-5.06631483309262	-9.37408708738064 \\
-5.04277645012179	-9.51685573594267 \\
-5.01860690875117	-9.65930874106288 \\
-4.99380822574803	-9.80120324678306 \\
-4.96838241787961	-9.94229639714496 \\
-4.94233150191316	-10.0823453361904 \\
-4.91565749461594	-10.2211072079611 \\
-4.88836241275518	-10.3583391564988 \\
-4.86044827309815	-10.4937983258454 \\
-4.8319170924121	-10.6272418600426 \\
-4.80277088746427	-10.7584269031321 \\
-4.77301167502191	-10.8871105991558 \\
-4.74264147185228	-11.0130500921555 \\
-4.71166229472262	-11.1360025261728 \\
-4.68007616040019	-11.2557250452496 \\
-4.64788508565224	-11.3719747934276 \\
-4.61509108724601	-11.4845089147486 \\
-4.58169618194875	-11.5930845532545 \\
-4.54770238652773	-11.6974588529869 \\
-4.51311171775018	-11.7973889579876 \\
-4.47792619238336	-11.8926320122985 \\
-4.44214782719451	-11.9829451599612 \\
-4.40577863895089	-12.0680855450176 \\
-4.36882064441976	-12.1478103115094 \\
-4.33127586036835	-12.2218766034784 \\
-4.29314630356392	-12.2900415649664 \\
-4.25443399077372	-12.3520623400152 \\
-4.21514093876499	-12.4076960726664 \\
-4.175269164305	-12.456699906962 \\
-4.13482068416099	-12.4988309869437 \\
-4.09379751510021	-12.5338464566532 \\
-4.05220167388991	-12.5615034601323 \\
-4.01003517729735	-12.5815591414228 \\
-3.96730004208976	-12.5937706445664 \\
-3.9239982850344	-12.597895113605 \\
-3.87913986870525	-12.5978818680773 \\
-3.83175958329018	-12.5977891493833 \\
-3.78189668697731	-12.5975374843566 \\
-3.72959043795477	-12.5970473998309 \\
-3.67488009441066	-12.59623942264 \\
-3.61780491453312	-12.5950340796174 \\
-3.55840415651027	-12.5933518975969 \\
-3.49671707853022	-12.5911134034122 \\
-3.43278293878109	-12.5882391238968 \\
-3.36664099545101	-12.5846495858846 \\
-3.2983305067281	-12.5802653162091 \\
-3.22789073080046	-12.5750068417041 \\
-3.15536092585624	-12.5687946892032 \\
-3.08078035008354	-12.5615493855401 \\
-3.00418826167049	-12.5531914575484 \\
-2.9256239188052	-12.543641432062 \\
-2.8451265796758	-12.5328198359143 \\
-2.76273550247041	-12.5206471959392 \\
-2.67848994537714	-12.5070440389703 \\
-2.59242916658413	-12.4919308918412 \\
-2.50459242427948	-12.4752282813857 \\
-2.41501897665132	-12.4568567344374 \\
-2.32374808188776	-12.43673677783 \\
-2.23081899817694	-12.4147889383972 \\
-2.13627098370696	-12.3909337429726 \\
-2.04014329666596	-12.36509171839 \\
-1.94247519524204	-12.3371833914829 \\
-1.84330593762334	-12.3071292890852 \\
-1.74267478199796	-12.2748499380304 \\
-1.64062098655404	-12.2402658651522 \\
-1.53718380947969	-12.2032975972844 \\
-1.43240250896303	-12.1638656612605 \\
-1.32631634319217	-12.1218905839143 \\
-1.21896457035526	-12.0772928920795 \\
-1.11038644864039	-12.0299931125897 \\
-1.0006212362357	-11.9799117722786 \\
-0.889708191329295	-11.9269693979798 \\
-0.777686572109307	-11.8710865165272 \\
-0.664595636763852	-11.8121836547542 \\
-0.550474643481051	-11.7501813394947 \\
-0.435362850449024	-11.6850000975822 \\
-0.31929951585589	-11.6165604558505 \\
-0.202323897889771	-11.5447829411333 \\
-0.0844752547387871	-11.4695880802641 \\
0.0342071554089419	-11.3908964000768 \\
0.153684074365296	-11.3086284274049 \\
0.273916243942153	-11.2227046890822 \\
0.394864405951395	-11.1330457119423 \\
0.516489302204901	-11.0395720228189 \\
};
\addplot [blue, line width=1.2pt, dotted, forget plot]
table [row sep=\\]{%
19.2086480503449	-22.6238513560195 \\
19.0972356425957	-22.4526591393327 \\
18.9853228437932	-22.2837211311975 \\
18.872708919344	-22.1180463169255 \\
18.7591931346547	-21.9566436818285 \\
18.6445747551319	-21.8005222112183 \\
18.5286530461824	-21.6506908904066 \\
18.4112272732127	-21.5081587047052 \\
18.2920967016294	-21.3739346394258 \\
18.1710605968391	-21.2490276798802 \\
18.0479182242486	-21.1344468113802 \\
17.9224688492644	-21.0312010192374 \\
17.7945117372931	-20.9402992887638 \\
17.6638461537414	-20.8627506052709 \\
17.530271364016	-20.7995639540705 \\
17.3932846162323	-20.7483833433804 \\
17.2519403220574	-20.6988700697969 \\
17.1064640517834	-20.6503194664596 \\
16.9571663481232	-20.6028516502555 \\
16.8043577537895	-20.5565867380713 \\
16.6483488114955	-20.5116448467938 \\
16.489450063954	-20.46814609331 \\
16.3279720538779	-20.4262105945067 \\
16.1642253239801	-20.3859584672707 \\
15.9985204169737	-20.3475098284888 \\
15.8311678755714	-20.310984795048 \\
15.6624782424863	-20.276503483835 \\
15.4927620604313	-20.2441860117366 \\
15.3223298721192	-20.2141524956398 \\
15.1514922202631	-20.1865230524313 \\
14.9805596475758	-20.161417798998 \\
14.8098426967703	-20.1389568522268 \\
14.6396519105594	-20.1192603290045 \\
14.4702978316562	-20.1024483462179 \\
14.3020910027735	-20.0886410207538 \\
14.1353419666243	-20.0779584694992 \\
13.9703612659215	-20.0705208093408 \\
13.807459443378	-20.0664481571655 \\
13.6468186748534	-20.0657587417289 \\
13.4869831602849	-20.0671790629313 \\
13.3274450572134	-20.0702058373822 \\
13.1681855971721	-20.0747260663957 \\
13.0091860116945	-20.0806267512857 \\
12.8504275323139	-20.0877948933665 \\
12.6918913905637	-20.096117493952 \\
12.5335588179773	-20.1054815543565 \\
12.3754110460881	-20.1157740758938 \\
12.2174293064293	-20.1268820598783 \\
12.0595948305345	-20.1386925076238 \\
11.9018888499369	-20.1510924204445 \\
11.74429259617	-20.1639687996545 \\
11.586787300767	-20.1772086465679 \\
11.4293541952614	-20.1906989624988 \\
11.2719745111866	-20.2043267487611 \\
11.1146294800759	-20.2179790066691 \\
10.9573003334626	-20.2315427375368 \\
10.7999683028803	-20.2449049426783 \\
10.6426146198621	-20.2579526234076 \\
10.4852205159416	-20.2705727810388 \\
10.327767222652	-20.2826524168861 \\
10.1702359715268	-20.2940785322635 \\
10.0126079940992	-20.3047381284851 \\
9.85373295322235	-20.3153958633273 \\
9.69268931363643	-20.3268166135522 \\
9.52980150970589	-20.3388877089526 \\
9.3653939757952	-20.3514964793214 \\
9.19979114626879	-20.3645302544514 \\
9.03331745549109	-20.3778763641353 \\
8.86629733782657	-20.391422138166 \\
8.69905522763964	-20.4050549063363 \\
8.53191555929476	-20.418661998439 \\
8.36520276715638	-20.4321307442669 \\
8.19924128558892	-20.4453484736127 \\
8.03435554895684	-20.4582025162694 \\
7.87086999162457	-20.4705802020298 \\
7.70910904795655	-20.4823688606865 \\
7.54939715231724	-20.4934558220325 \\
7.39205873907107	-20.5037284158606 \\
7.23741824258247	-20.5130739719635 \\
7.08580009721591	-20.521379820134 \\
6.93752873733581	-20.5285332901651 \\
6.79292859730661	-20.5344217118494 \\
6.65232411149277	-20.5389324149798 \\
6.51603971425871	-20.5419527293491 \\
6.38439983996889	-20.5433699847501 \\
6.2571479210386	-20.5405381718197 \\
6.13252127199154	-20.5255283611233 \\
6.01027748607209	-20.4984787848861 \\
5.89031145247168	-20.4601776938437 \\
5.77251806038177	-20.4114133387317 \\
5.65679219899377	-20.3529739702856 \\
5.54302875749914	-20.2856478392408 \\
5.43112262508929	-20.210223196333 \\
5.32096869095568	-20.1274882922976 \\
5.21246184428973	-20.0382313778702 \\
5.10549697428288	-19.9432407037862 \\
4.99996897012656	-19.8433045207812 \\
4.89577272101222	-19.7392110795908 \\
4.79280311613128	-19.6317486309504 \\
4.69095504467519	-19.5217054255956 \\
4.59012339583537	-19.4098697142619 \\
4.49020305880327	-19.2970297476847 \\
4.39108892277032	-19.1839737765998 \\
4.29267587692795	-19.0714900517425 \\
4.19485881046761	-18.9603668238484 \\
4.09753261258072	-18.851392343653 \\
4.00059217245872	-18.7453548618918 \\
3.90393237929306	-18.6430426293004 \\
3.80738800433926	-18.5448243225295 \\
3.71017090483294	-18.4458366959837 \\
3.61224331450565	-18.3445739752193 \\
3.51383256413518	-18.2411594238246 \\
3.41516598449935	-18.1357163053876 \\
3.31647090637597	-18.0283678834967 \\
3.21797466054283	-17.9192374217398 \\
3.11990457777775	-17.8084481837054 \\
3.02248798885853	-17.6961234329814 \\
2.92595222456299	-17.5823864331562 \\
2.83052461566892	-17.4673604478178 \\
2.73643249295414	-17.3511687405546 \\
2.64390318719645	-17.2339345749545 \\
2.55316402917365	-17.115781214606 \\
2.46444234966357	-16.996831923097 \\
2.37796547944399	-16.8772099640159 \\
2.29396074929273	-16.7570386009507 \\
2.2126554899876	-16.6364410974898 \\
2.1342770323064	-16.5155407172212 \\
2.05905270702695	-16.3944607237331 \\
1.98720984492703	-16.2733243806137 \\
1.91897577678448	-16.1522549514513 \\
1.85457783337708	-16.0313756998339 \\
1.79424334548265	-15.9108098893499 \\
1.73607374009458	-15.789804806343 \\
1.67812095595047	-15.6675876550968 \\
1.62053809144446	-15.5442515761084 \\
1.56347824497067	-15.419889709875 \\
1.50709451492325	-15.2945951968939 \\
1.45153999969632	-15.1684611776623 \\
1.39696779768401	-15.0415807926773 \\
1.34353100728046	-14.9140471824363 \\
1.2913827268798	-14.7859534874364 \\
1.24067605487615	-14.6573928481749 \\
1.19156408966366	-14.5284584051489 \\
1.14419992963646	-14.3992432988557 \\
1.09873667318866	-14.2698406697926 \\
1.05532741871442	-14.1403436584566 \\
1.01412526460786	-14.0108454053452 \\
0.975283309263109	-13.8814390509554 \\
0.938954651074302	-13.7522177357844 \\
0.905292388435573	-13.6232746003296 \\
0.874449619741051	-13.4947027850882 \\
0.846579443384869	-13.3665954305572 \\
0.821834957761159	-13.2390456772341 \\
0.800369261264054	-13.1121466656159 \\
0.782335452287685	-12.9859915361999 \\
0.767277459491121	-12.8603244440172 \\
0.753272085303509	-12.7341115476974 \\
0.740115567714028	-12.6073740496737 \\
0.727752577372579	-12.480220687582 \\
0.716127784929066	-12.3527601990581 \\
0.705185861033392	-12.2251013217378 \\
0.694871476335459	-12.0973527932568 \\
0.68512930148517	-11.969623351251 \\
0.675904007132427	-11.8420217333561 \\
0.667140263927135	-11.7146566772079 \\
0.658782742519195	-11.5876369204422 \\
0.65077611355851	-11.4610712006949 \\
0.643065047694983	-11.3350682556017 \\
0.635594215578516	-11.2097368227983 \\
0.628308287859013	-11.0851856399206 \\
0.621151935186377	-10.9615234446045 \\
0.614069828210509	-10.8388589744855 \\
0.607006637581313	-10.7173009671996 \\
0.599907033948692	-10.5969581603826 \\
0.592715687962548	-10.4779392916701 \\
0.585377270272784	-10.3603530986981 \\
0.577836451529303	-10.2443083191024 \\
0.570037902382008	-10.1299136905186 \\
0.5619511628027	-10.0172475821311 \\
0.553859586830749	-9.90597322329046 \\
0.545855391793604	-9.79591938277826 \\
0.537935163282315	-9.68702362357812 \\
0.530095486887933	-9.5792235086737 \\
0.52233294820151	-9.47245660104868 \\
0.514644132814094	-9.36666046368673 \\
0.507025626316738	-9.26177265957152 \\
0.49947401430049	-9.15773075168671 \\
0.491985882356403	-9.05447230301598 \\
0.484557816075527	-8.951934876543 \\
0.477186401048911	-8.85005603525142 \\
0.469868222867607	-8.74877334212494 \\
0.462599867122665	-8.64802436014721 \\
0.455377919405136	-8.5477466523019 \\
0.448198965306071	-8.44787778157268 \\
0.441059590416519	-8.34835531094323 \\
0.433956380327532	-8.24911680339721 \\
0.42688592063016	-8.15009982191829 \\
0.419844796915454	-8.05124192949014 \\
0.412829594774464	-7.95248068909643 \\
0.405836899798241	-7.85375366372083 \\
0.398863297577835	-7.75499841634701 \\
0.391905373704297	-7.65615250995864 \\
0.384991869843333	-7.55753649975505 \\
0.378150123965257	-7.45948715626457 \\
0.371374616118033	-7.36196636546439 \\
0.364659826349624	-7.26493601333171 \\
0.358000234707992	-7.16835798584371 \\
0.351390321241101	-7.07219416897758 \\
0.344824565996914	-6.9764064487105 \\
0.338297449023394	-6.88095671101967 \\
0.331803450368505	-6.78580684188227 \\
0.325337050080209	-6.6909187272755 \\
0.318892728206469	-6.59625425317654 \\
0.312464964795249	-6.50177530556257 \\
0.306048239894512	-6.40744377041079 \\
0.299637033552221	-6.31322153369839 \\
0.293225825816339	-6.21907048140256 \\
0.28680909673483	-6.12495249950047 \\
0.280381326355655	-6.03082947396933 \\
0.27393699472678	-5.93666329078632 \\
0.267470581896166	-5.84241583592862 \\
0.260976567911776	-5.74804899537343 \\
0.254449432821575	-5.65352465509794 \\
0.247883656673524	-5.55880470107932 \\
0.241273719515588	-5.46385101929478 \\
0.234619890328479	-5.36869562030519 \\
0.227939112023013	-5.27354528576784 \\
0.221235047402937	-5.17841545326602 \\
0.214510057992068	-5.0833047088684 \\
0.207766505314223	-4.98821163864366 \\
0.20100675089322	-4.8931348286605 \\
0.194233156252874	-4.79807286498759 \\
0.187448082917003	-4.7030243336936 \\
0.180653892409424	-4.60798782084723 \\
0.173852946253953	-4.51296191251715 \\
0.167047605974408	-4.41794519477205 \\
0.160240233094606	-4.32293625368061 \\
0.153433189138362	-4.22793367531151 \\
0.146628835629496	-4.13293604573344 \\
0.139829534091822	-4.03794195101506 \\
0.133037646049158	-3.94294997722507 \\
0.126255533025322	-3.84795871043215 \\
0.119485556544129	-3.75296673670498 \\
0.112730078129397	-3.65797264211224 \\
0.105991459304943	-3.56297501272261 \\
0.099272061594583	-3.46797243460477 \\
0.0925742465221349	-3.37296349382741 \\
0.0859003756114152	-3.27794677645921 \\
0.0792517498890644	-3.18292148823959 \\
0.0726162505632184	-3.08789468990304 \\
0.0659898288802846	-2.9928687554078 \\
0.0593725046122802	-2.89784368476565 \\
0.0527642975312224	-2.80281947798837 \\
0.0461652274091285	-2.70779613508776 \\
0.0395753140180158	-2.61277365607558 \\
0.0329945771299015	-2.51775204096363 \\
0.0264230365168028	-2.42273128976369 \\
0.0198607119507371	-2.32771140248754 \\
0.0133076232037216	-2.23269237914697 \\
0.00676379004777349	-2.13767421975376 \\
0.000229232254910097	-2.04265692431969 \\
-0.00629603040285135	-1.94764049285655 \\
-0.0128119781534936	-1.85262492537613 \\
-0.0193185912249993	-1.75761022189019 \\
-0.0258158498453514	-1.66259638241054 \\
-0.0323037342425324	-1.56758340694894 \\
-0.0387822246445252	-1.47257129551719 \\
-0.0452513012793125	-1.37756004812707 \\
-0.0517109443748771	-1.28254966479037 \\
-0.0581611341592016	-1.18754014551885 \\
-0.0646018508602689	-1.09253149032432 \\
};
\addplot [blue, line width=1.2pt, dotted, forget plot]
table [row sep=\\]{%
22.1662881365753	-19.5498847346191 \\
21.9600803919088	-19.5552141754009 \\
21.7524732391489	-19.561322880049 \\
21.5436293188644	-19.5681523351325 \\
21.3337112716243	-19.5756440272207 \\
21.1228817379976	-19.5837394428826 \\
20.9113033585533	-19.5923800686874 \\
20.6991387738603	-19.6015073912042 \\
20.4865506244875	-19.6110628970021 \\
20.2737015510039	-19.6209880726504 \\
20.0607541939785	-19.6312244047182 \\
19.8478711939803	-19.6417133797745 \\
19.6352151915782	-19.6523964843885 \\
19.4229488273411	-19.6632152051294 \\
19.2112347418381	-19.6741110285664 \\
19.000235575638	-19.6850254412684 \\
18.7901139693099	-19.6958999298048 \\
18.5810325634227	-19.7066759807446 \\
18.3731539985453	-19.7172950806569 \\
18.166104403557	-19.7279669896813 \\
17.9582955062363	-19.7394999326579 \\
17.749740449658	-19.751891544784 \\
17.540586214827	-19.7650721868309 \\
17.3309797827485	-19.7789722195701 \\
17.1210681344275	-19.7935220037728 \\
16.910998250869	-19.8086519002104 \\
16.7009171130782	-19.8242922696544 \\
16.4909717020601	-19.840373472876 \\
16.2813089988199	-19.8568258706466 \\
16.0720759843624	-19.8735798237376 \\
15.863419639693	-19.8905656929203 \\
15.6554869458165	-19.9077138389661 \\
15.4484248837381	-19.9249546226465 \\
15.2423804344629	-19.9422184047326 \\
15.0375005789958	-19.9594355459959 \\
14.8339322983421	-19.9765364072078 \\
14.6318225735068	-19.9934513491396 \\
14.4313183854948	-20.0101107325627 \\
14.2325667153114	-20.0264449182484 \\
14.0357145439616	-20.0423842669681 \\
13.8409088524505	-20.0578591394932 \\
13.648296621783	-20.072799896595 \\
13.4579929161033	-20.0871663569863 \\
13.2696567644356	-20.101293841674 \\
13.0830792528807	-20.1152993150624 \\
12.8981608767867	-20.1291874115398 \\
12.7148021315016	-20.1429627654947 \\
12.5329035123737	-20.1566300113152 \\
12.3523655147509	-20.1701937833896 \\
12.1730886339814	-20.1836587161063 \\
11.9949733654133	-20.1970294438536 \\
11.8179202043948	-20.2103106010196 \\
11.6418296462738	-20.2235068219928 \\
11.4666021863986	-20.2366227411615 \\
11.2921383201173	-20.2496629929138 \\
11.1183385427779	-20.2626322116382 \\
10.9451033497285	-20.2755350317228 \\
10.7723332363173	-20.2883760875561 \\
10.5999286978924	-20.3011600135263 \\
10.4277902298018	-20.3138914440217 \\
10.2558183273938	-20.3265750134306 \\
10.0839134860163	-20.3392153561413 \\
9.91197620101753	-20.351817106542 \\
9.73990696774556	-20.3643848990212 \\
9.5676062815485	-20.376923367967 \\
9.39497463777447	-20.3894371477678 \\
9.22170839422824	-20.402379006693 \\
9.04767772107115	-20.4161154704903 \\
8.87304383218609	-20.4305218097654 \\
8.69796794145597	-20.4454732951238 \\
8.52261126276371	-20.4608451971713 \\
8.34713500999221	-20.4765127865135 \\
8.17170039702438	-20.492351333756 \\
7.99646863774313	-20.5082361095044 \\
7.82160094603137	-20.5240423843644 \\
7.647258535772	-20.5396454289417 \\
7.47360262084794	-20.5549205138419 \\
7.30079441514209	-20.5697429096705 \\
7.12899513253736	-20.5839878870334 \\
6.95836598691665	-20.5975307165361 \\
6.78906819216289	-20.6102466687842 \\
6.62126296215897	-20.6220110143834 \\
6.45511151078781	-20.6326990239394 \\
6.29077505193231	-20.6421859680577 \\
6.12841479947537	-20.6503471173441 \\
5.96819196729992	-20.6570577424041 \\
5.81026776928886	-20.6621931138434 \\
5.65480341932509	-20.6656285022677 \\
5.50196013129153	-20.6672391782826 \\
5.3506413972015	-20.6669338184937 \\
5.19709487948163	-20.6647772588782 \\
5.04153633614147	-20.6608248327889 \\
4.88450291744275	-20.6551267619138 \\
4.72653177364723	-20.6477332679406 \\
4.56816005501664	-20.6386945725573 \\
4.40992491181273	-20.6280608974515 \\
4.25236349429724	-20.6158824643112 \\
4.0960129527319	-20.6022094948241 \\
3.94141043737847	-20.5870922106782 \\
3.78909309849868	-20.5705808335611 \\
3.63959808635427	-20.5527255851608 \\
3.49346255120699	-20.5335766871651 \\
3.35122364331858	-20.5131843612619 \\
3.21341851295078	-20.4915988291388 \\
3.08058431036533	-20.4688703124838 \\
2.95325818582397	-20.4450490329847 \\
2.83197728958845	-20.4201852123294 \\
2.71727877192051	-20.3943290722056 \\
2.60969978308188	-20.3675308343011 \\
2.50977747333432	-20.3398407203039 \\
2.41804899293956	-20.3113089519017 \\
2.33505149215934	-20.2819857507824 \\
2.26085418827746	-20.251188521006 \\
2.18977937067682	-20.2096817344563 \\
2.12008079469036	-20.1553370902989 \\
2.05185101452147	-20.0890177780492 \\
1.98518258437352	-20.0115869872228 \\
1.92016805844992	-19.9239079073354 \\
1.85689999095403	-19.8268437279025 \\
1.79547093608925	-19.7212576384396 \\
1.73597344805897	-19.6080128284623 \\
1.67850008106657	-19.4879724874861 \\
1.62314338931543	-19.3619998050267 \\
1.56999592700895	-19.2309579705996 \\
1.5191502483505	-19.0957101737203 \\
1.47069890754348	-18.9571196039044 \\
1.42473445879127	-18.8160494506674 \\
1.38134945629725	-18.673362903525 \\
1.34063645426481	-18.5299231519927 \\
1.30268800689735	-18.386593385586 \\
1.26759666839823	-18.2442367938205 \\
1.23545499297086	-18.1037165662118 \\
1.20635553481861	-17.9658958922755 \\
1.18039084814488	-17.831637961527 \\
1.15765348715304	-17.701805963482 \\
1.13823600604648	-17.577263087656 \\
1.12109317583134	-17.4549997747936 \\
1.10506655533471	-17.331491570498 \\
1.09007888049144	-17.2068293215728 \\
1.07605288723639	-17.0811038748215 \\
1.06291131150441	-16.9544060770476 \\
1.05057688923034	-16.8268267750548 \\
1.03897235634905	-16.6984568156465 \\
1.02802044879537	-16.5693870456264 \\
1.01764390250417	-16.4397083117979 \\
1.00776545341028	-16.3095114609646 \\
0.99830783744857	-16.17888733993 \\
0.989193790553881	-16.0479267954978 \\
0.980346048661066	-15.9167206744714 \\
0.971687347704976	-15.7853598236544 \\
0.963140423620462	-15.6539350898503 \\
0.954628012342374	-15.5225373198628 \\
0.946072849805564	-15.3912573604953 \\
0.937397671944883	-15.2601860585513 \\
0.928525214695181	-15.1294142608345 \\
0.919378213991309	-14.9990328141484 \\
0.909879405768119	-14.8691325652965 \\
0.89995152596046	-14.7398043610824 \\
0.889517310503184	-14.6111390483096 \\
0.878586456740733	-14.4830439854665 \\
0.867413982360907	-14.354977619066 \\
0.856049636855138	-14.226908469708 \\
0.84452341564933	-14.0988494875408 \\
0.832865314169387	-13.9708136227125 \\
0.821105327841213	-13.8428138253712 \\
0.809273452090713	-13.714863045665 \\
0.797399682343791	-13.5869742337422 \\
0.785514014026352	-13.4591603397509 \\
0.773646442564298	-13.3314343138392 \\
0.761826963383536	-13.2038091061552 \\
0.750085571909969	-13.0762976668472 \\
0.738452263569501	-12.9489129460632 \\
0.726957033788037	-12.8216678939515 \\
0.71562987799148	-12.6945754606601 \\
0.704500791605736	-12.5676485963372 \\
0.693599770056708	-12.4409002511309 \\
0.682956808770302	-12.3143433751895 \\
0.67260190317242	-12.1879909186611 \\
0.662565048688967	-12.0618558316937 \\
0.652876240745849	-11.9359510644356 \\
0.643565474768968	-11.8102895670349 \\
0.634662746184229	-11.6848842896397 \\
0.626192311168087	-11.5597484981563 \\
0.618092174257011	-11.434891023722 \\
0.610315059358507	-11.3103003799382 \\
0.60283230221623	-11.1859617614078 \\
0.595615238573835	-11.0618603627338 \\
0.588635204174979	-10.937981378519 \\
0.581863534763317	-10.8143100033664 \\
0.575271566082504	-10.6908314318789 \\
0.568830633876196	-10.5675308586593 \\
0.562512073888049	-10.4443934783107 \\
0.556287221861718	-10.3214044854359 \\
0.550127413540859	-10.1985490746379 \\
0.544003984669127	-10.0758124405195 \\
0.537888270990178	-9.95317977768369 \\
0.531751608247667	-9.83063628073337 \\
0.525565332185251	-9.70816714427145 \\
0.519300778546584	-9.58575756290085 \\
0.512929283075322	-9.46339273122449 \\
0.506422181515122	-9.34105784384528 \\
0.499750809609637	-9.21873809536613 \\
0.492886503102525	-9.09641868038996 \\
0.48580059773744	-8.97408479351969 \\
0.478464429258038	-8.85172162935822 \\
0.470849333407975	-8.72931438250849 \\
0.462919224162833	-8.60651723406788 \\
0.454676688792083	-8.48306422211275 \\
0.446151072980214	-8.35906032671436 \\
0.437371722411717	-8.23461052794401 \\
0.42836798277108	-8.109819805873 \\
0.419169199742793	-7.98479314057261 \\
0.409804719011346	-7.85963551211414 \\
0.400303886261227	-7.73445190056887 \\
0.390696047176928	-7.60934728600809 \\
0.381010547442936	-7.4844266485031 \\
0.371276732743742	-7.35979496812519 \\
0.361523948763835	-7.23555722494563 \\
0.351781541187704	-7.11181839903574 \\
0.34207885569984	-6.98868347046678 \\
0.332445237984731	-6.86625741931007 \\
0.322910033726866	-6.74464522563688 \\
0.313502588610737	-6.6239518695185 \\
0.304252248320831	-6.50428233102623 \\
0.295188358541639	-6.38574159023136 \\
0.286340264957649	-6.26843462720518 \\
0.277737313253352	-6.15246642201897 \\
0.269408849113237	-6.03794195474403 \\
0.261384218221793	-5.92496620545165 \\
0.253654305768819	-5.81354290556789 \\
0.24610429016505	-5.70336759031938 \\
0.238713260159363	-5.594349507711 \\
0.231469152797669	-5.48642000243903 \\
0.224359905125879	-5.37951041919976 \\
0.217373454189903	-5.27355210268946 \\
0.210497737035651	-5.16847639760441 \\
0.203720690709034	-5.0642146486409 \\
0.197030252255963	-4.96069820049521 \\
0.190414358722347	-4.85785839786362 \\
0.183860947154099	-4.75562658544242 \\
0.177357954597127	-4.65393410792787 \\
0.170893318097342	-4.55271231001627 \\
0.164454974700656	-4.45189253640389 \\
0.158030861452978	-4.35140613178702 \\
0.151608915400219	-4.25118444086194 \\
0.145177073588289	-4.15115880832492 \\
0.138723273063099	-4.05126057887226 \\
0.132235450870559	-3.95142109720023 \\
0.125701544056581	-3.85157170800511 \\
0.119109489667073	-3.75164375598318 \\
0.112447224747948	-3.65156858583073 \\
0.105702686345115	-3.55127754224404 \\
0.0988690770725386	-3.45073129770004 \\
0.0920063654572274	-3.35026297533525 \\
0.0851347354935453	-3.24998534750488 \\
0.0782542036777391	-3.14989900476443 \\
0.0713647865060558	-3.05000453766938 \\
0.0644665004747423	-2.95030253677522 \\
0.0575593620800455	-2.85079359263744 \\
0.0506433878182124	-2.75147829581153 \\
0.0437185941854899	-2.65235723685296 \\
0.036784997678125	-2.55343100631724 \\
0.0298426147923646	-2.45470019475984 \\
0.0228914620244556	-2.35616539273625 \\
0.015931555870645	-2.25782719080197 \\
0.00896291282717962	-2.15968617951248 \\
0.0019855493903065	-2.06174294942326 \\
-0.00500051794372747	-1.9639980910898 \\
-0.0119952726786754	-1.8664521950676 \\
-0.0189986983182902	-1.76910585191213 \\
-0.0260107783663251	-1.67195965217889 \\
-0.0330314963265331	-1.57501418642336 \\
-0.0400608357026672	-1.47827004520102 \\
-0.0470987799984806	-1.38172781906738 \\
-0.0541453127177263	-1.28538809857791 \\
};

\addplot [color1, line width=2.0pt, dashed]
table [row sep=\\]{%
1.5180163655554	-23.5470553861719 \\
1.50747554550815	-23.3977502408164 \\
1.4969346788029	-23.2484450987542 \\
1.48639376541872	-23.0991399599866 \\
1.47585280651154	-22.9498348244322 \\
1.46531180252859	-22.8005296920594 \\
1.45477075417296	-22.6512245628184 \\
1.44422966229716	-22.5019194366493 \\
1.43368852727881	-22.3526143135252 \\
1.42314735011752	-22.2033091933757 \\
1.4126061313072	-22.0540040761659 \\
1.4020648714813	-21.9046989618511 \\
1.39152357155182	-21.7553938503668 \\
1.38098223188015	-21.6060887416877 \\
1.37044085343886	-21.456783635745 \\
1.35989943677527	-21.3074785325002 \\
1.34935798245996	-21.1581734319129 \\
1.33881649145088	-21.0088683339155 \\
1.32827496410675	-20.8595632384827 \\
1.31773340135956	-20.7102581455488 \\
1.30719180381679	-20.5609530550708 \\
1.29665017199293	-20.4116479670125 \\
1.28610850687837	-20.2623428813039 \\
1.27556680884264	-20.113037797919 \\
1.26502507876365	-19.9637327167958 \\
1.25448331731581	-19.8144276378866 \\
1.24394152496426	-19.6651225611587 \\
1.2333997027183	-19.5158174865409 \\
1.22285785097183	-19.3665124140052 \\
1.21231597053514	-19.2172073434946 \\
1.20177406215632	-19.0679022749562 \\
1.19123212625794	-18.9185972083602 \\
1.1806901638547	-18.769292143635 \\
1.17014817537712	-18.6199870807501 \\
1.15960616155805	-18.470682019654 \\
1.14906412322236	-18.3213769602882 \\
1.13852206075801	-18.1720719026256 \\
1.12797997517157	-18.0227668465949 \\
1.11743786693754	-17.8734617921628 \\
1.1068957367164	-17.7241567392825 \\
1.09635358539793	-17.5748516878914 \\
1.08581141334847	-17.4255466379635 \\
1.07526922155295	-17.2762415894294 \\
1.06472701053655	-17.1269365422519 \\
1.05418478089421	-16.977631496389 \\
1.04364253356707	-16.8283264517744 \\
1.03310026891335	-16.6790214083827 \\
1.02255798788284	-16.529716366147 \\
1.01201569105818	-16.3804113250261 \\
1.0014733789755	-16.2311062849821 \\
0.990931052613776	-16.0818012459459 \\
0.980388712336662	-15.9324962078919 \\
0.969846359045269	-15.7831911707565 \\
0.959303993386442	-15.6338861344941 \\
0.94876161584428	-15.4845810990704 \\
0.938219227422082	-15.3352760644146 \\
0.927676828502429	-15.1859710304997 \\
0.917134419924251	-15.0366659972666 \\
0.90659200240536	-14.8873609646646 \\
0.896049576384573	-14.7380559326626 \\
0.885507142876	-14.5887509011891 \\
0.874964702294532	-14.4394458702149 \\
0.864422255403806	-14.2901408396859 \\
0.853879802998951	-14.1408358095462 \\
0.843337345480398	-13.9915307797674 \\
0.83279488385955	-13.8422257502782 \\
0.822252418592529	-13.6929207210463 \\
0.81170995036795	-13.5436156920232 \\
0.801167480051232	-13.3943106631477 \\
0.790625008015771	-13.2450056343936 \\
0.78008253525675	-13.0957006056905 \\
0.769540062278277	-12.9463955770029 \\
0.758997589700704	-12.797090548287 \\
0.748455118446222	-12.6477855194778 \\
0.737912648874712	-12.4984804905497 \\
0.727370181951618	-12.3491754614347 \\
0.716827718235791	-12.1998704320933 \\
0.706285258286084	-12.0505654024861 \\
0.69574280306794	-11.901260372545 \\
0.685200352941239	-11.7519553422445 \\
0.674657908828172	-11.6026503115195 \\
0.664115471349094	-11.4533452803263 \\
0.653573041008109	-11.3040402486292 \\
0.643030618800403	-11.1547352163581 \\
0.63248820509937	-11.0054301834866 \\
0.62194580077043	-10.8561251499535 \\
0.611403406502201	-10.7068201157104 \\
0.600861022750797	-10.557515080725 \\
0.59031865052763	-10.4082100449258 \\
0.579776290233124	-10.2589050082847 \\
0.569233942662411	-10.1095999707455 \\
0.558691608579131	-9.9602949322544 \\
0.548149288398167	-9.81098989278199 \\
0.537606983133639	-9.66168485225674 \\
0.527064693224357	-9.51237981064769 \\
0.516522419388133	-9.36307476790416 \\
0.505980162463903	-9.21376972396694 \\
0.495437922834237	-9.06446467880903 \\
0.484895701502448	-8.91515963235961 \\
0.474353498952626	-8.76585458458451 \\
0.463811315831614	-8.61654953543808 \\
0.453269153040535	-8.46724448485672 \\
0.442727010943026	-8.31793943281475 \\
0.432184890518075	-8.16863437924308 \\
0.421642792301814	-8.01932932410386 \\
0.411100716876872	-7.87002426735598 \\
0.400558665193044	-7.7207192089324 \\
0.390016637608552	-7.57141414880783 \\
0.379474635064539	-7.42210908691584 \\
0.368932658155941	-7.27280402321443 \\
0.358390707407926	-7.12349895766656 \\
0.34784878380545	-6.97419389020268 \\
0.337306887714834	-6.82488882079695 \\
0.326765020025858	-6.67558374938656 \\
0.316223181399023	-6.5262786759249 \\
0.305681372308794	-6.37697360037846 \\
0.295139593761769	-6.22766852267622 \\
0.28459784614589	-6.07836344279078 \\
0.274056130286049	-5.92905836066393 \\
0.263514446915079	-5.77975327624393 \\
0.252972796463495	-5.6304481895004 \\
0.242431179946021	-5.48114310036172 \\
0.2318895977852	-5.33183800879806 \\
0.221348050729132	-5.18253291475663 \\
0.210806539588127	-5.03322781818022 \\
0.200265064756047	-4.88392271904104 \\
0.189723627242224	-4.73461761726785 \\
0.179182227511778	-4.58531251282782 \\
0.168640866239125	-4.43600740567333 \\
0.158099544302185	-4.28670229574244 \\
0.147558262070469	-4.13739718300906 \\
0.137017020534395	-3.98809206740329 \\
0.126475820208427	-3.8387869488888 \\
0.115934661700046	-3.68948182742273 \\
0.105393545941272	-3.54017670293929 \\
0.0948524732907801	-3.39087157541315 \\
0.0843114447065533	-3.24156644477672 \\
0.0737704607591645	-3.09226131098971 \\
0.0632295219959136	-2.94295617401349 \\
0.0526886293894049	-2.79365103377942 \\
0.0421477833009992	-2.64434589026198 \\
0.0316069846427167	-2.49504074339682 \\
0.0210662340480081	-2.34573559313921 \\
0.0105255320107463	-2.1964304394543 \\
-1.51204693978756e-05	-2.04712528227153 \\
-0.0105557230148547	-1.89782012156424 \\
-0.0210962747730965	-1.74851495727227 \\
-0.0316367750410241	-1.599209789346 \\
-0.042177223371437	-1.44990461775384 \\
-0.0527176186083618	-1.30059944241422 \\
};

\addplot [white, line width=1.0pt, dashed]
table [row sep=\\]{%
1.5180163655554	-23.5470553861719 \\
1.50747554550815	-23.3977502408164 \\
1.4969346788029	-23.2484450987542 \\
1.48639376541872	-23.0991399599866 \\
1.47585280651154	-22.9498348244322 \\
1.46531180252859	-22.8005296920594 \\
1.45477075417296	-22.6512245628184 \\
1.44422966229716	-22.5019194366493 \\
1.43368852727881	-22.3526143135252 \\
1.42314735011752	-22.2033091933757 \\
1.4126061313072	-22.0540040761659 \\
1.4020648714813	-21.9046989618511 \\
1.39152357155182	-21.7553938503668 \\
1.38098223188015	-21.6060887416877 \\
1.37044085343886	-21.456783635745 \\
1.35989943677527	-21.3074785325002 \\
1.34935798245996	-21.1581734319129 \\
1.33881649145088	-21.0088683339155 \\
1.32827496410675	-20.8595632384827 \\
1.31773340135956	-20.7102581455488 \\
1.30719180381679	-20.5609530550708 \\
1.29665017199293	-20.4116479670125 \\
1.28610850687837	-20.2623428813039 \\
1.27556680884264	-20.113037797919 \\
1.26502507876365	-19.9637327167958 \\
1.25448331731581	-19.8144276378866 \\
1.24394152496426	-19.6651225611587 \\
1.2333997027183	-19.5158174865409 \\
1.22285785097183	-19.3665124140052 \\
1.21231597053514	-19.2172073434946 \\
1.20177406215632	-19.0679022749562 \\
1.19123212625794	-18.9185972083602 \\
1.1806901638547	-18.769292143635 \\
1.17014817537712	-18.6199870807501 \\
1.15960616155805	-18.470682019654 \\
1.14906412322236	-18.3213769602882 \\
1.13852206075801	-18.1720719026256 \\
1.12797997517157	-18.0227668465949 \\
1.11743786693754	-17.8734617921628 \\
1.1068957367164	-17.7241567392825 \\
1.09635358539793	-17.5748516878914 \\
1.08581141334847	-17.4255466379635 \\
1.07526922155295	-17.2762415894294 \\
1.06472701053655	-17.1269365422519 \\
1.05418478089421	-16.977631496389 \\
1.04364253356707	-16.8283264517744 \\
1.03310026891335	-16.6790214083827 \\
1.02255798788284	-16.529716366147 \\
1.01201569105818	-16.3804113250261 \\
1.0014733789755	-16.2311062849821 \\
0.990931052613776	-16.0818012459459 \\
0.980388712336662	-15.9324962078919 \\
0.969846359045269	-15.7831911707565 \\
0.959303993386442	-15.6338861344941 \\
0.94876161584428	-15.4845810990704 \\
0.938219227422082	-15.3352760644146 \\
0.927676828502429	-15.1859710304997 \\
0.917134419924251	-15.0366659972666 \\
0.90659200240536	-14.8873609646646 \\
0.896049576384573	-14.7380559326626 \\
0.885507142876	-14.5887509011891 \\
0.874964702294532	-14.4394458702149 \\
0.864422255403806	-14.2901408396859 \\
0.853879802998951	-14.1408358095462 \\
0.843337345480398	-13.9915307797674 \\
0.83279488385955	-13.8422257502782 \\
0.822252418592529	-13.6929207210463 \\
0.81170995036795	-13.5436156920232 \\
0.801167480051232	-13.3943106631477 \\
0.790625008015771	-13.2450056343936 \\
0.78008253525675	-13.0957006056905 \\
0.769540062278277	-12.9463955770029 \\
0.758997589700704	-12.797090548287 \\
0.748455118446222	-12.6477855194778 \\
0.737912648874712	-12.4984804905497 \\
0.727370181951618	-12.3491754614347 \\
0.716827718235791	-12.1998704320933 \\
0.706285258286084	-12.0505654024861 \\
0.69574280306794	-11.901260372545 \\
0.685200352941239	-11.7519553422445 \\
0.674657908828172	-11.6026503115195 \\
0.664115471349094	-11.4533452803263 \\
0.653573041008109	-11.3040402486292 \\
0.643030618800403	-11.1547352163581 \\
0.63248820509937	-11.0054301834866 \\
0.62194580077043	-10.8561251499535 \\
0.611403406502201	-10.7068201157104 \\
0.600861022750797	-10.557515080725 \\
0.59031865052763	-10.4082100449258 \\
0.579776290233124	-10.2589050082847 \\
0.569233942662411	-10.1095999707455 \\
0.558691608579131	-9.9602949322544 \\
0.548149288398167	-9.81098989278199 \\
0.537606983133639	-9.66168485225674 \\
0.527064693224357	-9.51237981064769 \\
0.516522419388133	-9.36307476790416 \\
0.505980162463903	-9.21376972396694 \\
0.495437922834237	-9.06446467880903 \\
0.484895701502448	-8.91515963235961 \\
0.474353498952626	-8.76585458458451 \\
0.463811315831614	-8.61654953543808 \\
0.453269153040535	-8.46724448485672 \\
0.442727010943026	-8.31793943281475 \\
0.432184890518075	-8.16863437924308 \\
0.421642792301814	-8.01932932410386 \\
0.411100716876872	-7.87002426735598 \\
0.400558665193044	-7.7207192089324 \\
0.390016637608552	-7.57141414880783 \\
0.379474635064539	-7.42210908691584 \\
0.368932658155941	-7.27280402321443 \\
0.358390707407926	-7.12349895766656 \\
0.34784878380545	-6.97419389020268 \\
0.337306887714834	-6.82488882079695 \\
0.326765020025858	-6.67558374938656 \\
0.316223181399023	-6.5262786759249 \\
0.305681372308794	-6.37697360037846 \\
0.295139593761769	-6.22766852267622 \\
0.28459784614589	-6.07836344279078 \\
0.274056130286049	-5.92905836066393 \\
0.263514446915079	-5.77975327624393 \\
0.252972796463495	-5.6304481895004 \\
0.242431179946021	-5.48114310036172 \\
0.2318895977852	-5.33183800879806 \\
0.221348050729132	-5.18253291475663 \\
0.210806539588127	-5.03322781818022 \\
0.200265064756047	-4.88392271904104 \\
0.189723627242224	-4.73461761726785 \\
0.179182227511778	-4.58531251282782 \\
0.168640866239125	-4.43600740567333 \\
0.158099544302185	-4.28670229574244 \\
0.147558262070469	-4.13739718300906 \\
0.137017020534395	-3.98809206740329 \\
0.126475820208427	-3.8387869488888 \\
0.115934661700046	-3.68948182742273 \\
0.105393545941272	-3.54017670293929 \\
0.0948524732907801	-3.39087157541315 \\
0.0843114447065533	-3.24156644477672 \\
0.0737704607591645	-3.09226131098971 \\
0.0632295219959136	-2.94295617401349 \\
0.0526886293894049	-2.79365103377942 \\
0.0421477833009992	-2.64434589026198 \\
0.0316069846427167	-2.49504074339682 \\
0.0210662340480081	-2.34573559313921 \\
0.0105255320107463	-2.1964304394543 \\
-1.51204693978756e-05	-2.04712528227153 \\
-0.0105557230148547	-1.89782012156424 \\
-0.0210962747730965	-1.74851495727227 \\
-0.0316367750410241	-1.599209789346 \\
-0.042177223371437	-1.44990461775384 \\
-0.0527176186083618	-1.30059944241422 \\
};
\end{axis}

\end{tikzpicture}
        \label{fig:clt_36C_hist_org}}
    \subfigure[Synthetic trajectories]{
        \setlength\figureheight{7cm}
        \setlength\figurewidth{7cm}
        \input{figures/clt_36C_synthetic_hist_500.tex}
        \label{fig:clt_36C_hist_syn}}
  \caption{Log-histograms of arrival tracks to KCLT 36C and RNAV standard arrival procedures.
  }
\end{figure}

We also provide quantitative analyses on the single trajectory model in the next section, along with the multiple trajectory model.


\subsection{Analyses on Multiple Trajectory Scenarios}
To evaluate the performance of the single and multiple trajectory models, we experiment with both models on traffic scenarios where three aircraft arrive at KJFK 13L with inter-arrival times of less than 180 seconds.
As described in Section \ref{sec:multiple trajectory model}, we train a pairwise trajectory model for each combination of two procedures and generate synthetic trajectory sets of three aircraft.
We also generate synthetic trajectory sets using the single trajectory model. 
This is done by generating each trajectory independently and then connecting the trajectory data points through the inter-arrival time variables.

\begin{figure*}[tb!]
    \centering
    \subfigure[xEast. $D_{JS}$: 0.0119 (multiple), 0.0245 (single)]{
        \label{fig:hist_x_coords}
        % This file was created by tikzplotlib v0.9.2.
\begin{tikzpicture}

\begin{axis}[
legend cell align={left},
legend style={fill opacity=0.8, draw opacity=1, text opacity=1, draw=white!80!black},
width=3.4in,
height=2.2in,
tick align=outside,
tick pos=left,
x grid style={white!69.0196078431373!black},
xlabel={x coordinate (NM)},
xmin=-17.25, xmax=27.25,
xtick style={color=black},
y grid style={white!69.0196078431373!black},
ytick={0.00, 0.02, 0.04, 0.06, 0.08, 0.10, 0.12, 0.14},
ylabel={density},
ymin=0, ymax=0.149110572440798,
ytick style={color=black},
yticklabel style={/pgf/number format/.cd,fixed,precision=2},
tick label style={font=\tiny},
label style={font=\scriptsize},
legend entries={{actual},{synthetic--multiple},{synthetic--single}},
legend style={at={(0.97,0.95)}, anchor=north east, draw=black},
legend style={font=\scriptsize}
]

\addlegendimage{area legend, draw=black, fill=black, opacity=0.6}
\addlegendimage{area legend, draw=black, fill=white!50.1960784313725!black, opacity=0.7}
\addlegendimage{area legend, draw=black, fill=white, opacity=0.5}

\draw[draw=black,fill=black,opacity=0.7] (axis cs:-15,0) rectangle (axis cs:-14.4285714285714,0);
\draw[draw=black,fill=black,opacity=0.7] (axis cs:-14.4285714285714,0) rectangle (axis cs:-13.8571428571429,0);
\draw[draw=black,fill=black,opacity=0.7] (axis cs:-13.8571428571429,0) rectangle (axis cs:-13.2857142857143,0.000130523960469886);
\draw[draw=black,fill=black,opacity=0.7] (axis cs:-13.2857142857143,0) rectangle (axis cs:-12.7142857142857,0.00296942010068991);
\draw[draw=black,fill=black,opacity=0.7] (axis cs:-12.7142857142857,0) rectangle (axis cs:-12.1428571428571,0.00623251911243705);
\draw[draw=black,fill=black,opacity=0.7] (axis cs:-12.1428571428571,0) rectangle (axis cs:-11.5714285714286,0.00959351109453664);
\draw[draw=black,fill=black,opacity=0.7] (axis cs:-11.5714285714286,0) rectangle (axis cs:-11,0.0100829759462987);
\draw[draw=black,fill=black,opacity=0.7] (axis cs:-11,0) rectangle (axis cs:-10.4285714285714,0.0155976132761514);
\draw[draw=black,fill=black,opacity=0.7] (axis cs:-10.4285714285714,0) rectangle (axis cs:-9.85714285714286,0.0326962520977065);
\draw[draw=black,fill=black,opacity=0.7] (axis cs:-9.85714285714286,0) rectangle (axis cs:-9.28571428571429,0.0502517247809062);
\draw[draw=black,fill=black,opacity=0.7] (axis cs:-9.28571428571428,0) rectangle (axis cs:-8.71428571428571,0.086700540742122);
\draw[draw=black,fill=black,opacity=0.7] (axis cs:-8.71428571428572,0) rectangle (axis cs:-8.14285714285714,0.142010068991236);
\draw[draw=black,fill=black,opacity=0.7] (axis cs:-8.14285714285714,0) rectangle (axis cs:-7.57142857142857,0.109607495804587);
\draw[draw=black,fill=black,opacity=0.7] (axis cs:-7.57142857142857,0) rectangle (axis cs:-7,0.0877447324258809);
\draw[draw=black,fill=black,opacity=0.7] (axis cs:-7,0) rectangle (axis cs:-6.42857142857143,0.083796382621667);
\draw[draw=black,fill=black,opacity=0.7] (axis cs:-6.42857142857143,0) rectangle (axis cs:-5.85714285714286,0.0873205295543539);
\draw[draw=black,fill=black,opacity=0.7] (axis cs:-5.85714285714286,0) rectangle (axis cs:-5.28571428571429,0.0858521349990677);
\draw[draw=black,fill=black,opacity=0.7] (axis cs:-5.28571428571429,0) rectangle (axis cs:-4.71428571428572,0.089539436882342);
\draw[draw=black,fill=black,opacity=0.7] (axis cs:-4.71428571428572,0) rectangle (axis cs:-4.14285714285714,0.0951845981726646);
\draw[draw=black,fill=black,opacity=0.7] (axis cs:-4.14285714285714,0) rectangle (axis cs:-3.57142857142857,0.0881689352974082);
\draw[draw=black,fill=black,opacity=0.7] (axis cs:-3.57142857142857,0) rectangle (axis cs:-3,0.0695040089502142);
\draw[draw=black,fill=black,opacity=0.7] (axis cs:-3,0) rectangle (axis cs:-2.42857142857143,0.0648704083535335);
\draw[draw=black,fill=black,opacity=0.7] (axis cs:-2.42857142857143,0) rectangle (axis cs:-1.85714285714286,0.0644788364721238);
\draw[draw=black,fill=black,opacity=0.7] (axis cs:-1.85714285714286,0) rectangle (axis cs:-1.28571428571429,0.0651314562744733);
\draw[draw=black,fill=black,opacity=0.7] (axis cs:-1.28571428571429,0) rectangle (axis cs:-0.714285714285715,0.0633041208278949);
\draw[draw=black,fill=black,opacity=0.7] (axis cs:-0.714285714285715,0) rectangle (axis cs:-0.142857142857144,0.026365840014917);
\draw[draw=black,fill=black,opacity=0.7] (axis cs:-0.142857142857144,0) rectangle (axis cs:0.428571428571427,0.0220259183292933);
\draw[draw=black,fill=black,opacity=0.7] (axis cs:0.428571428571427,0) rectangle (axis cs:1,0.0270184598172664);
\draw[draw=black,fill=black,opacity=0.7] (axis cs:1,0) rectangle (axis cs:1.57142857142857,0.0239185157561067);
\draw[draw=black,fill=black,opacity=0.7] (axis cs:1.57142857142857,0) rectangle (axis cs:2.14285714285714,0.0218301323885884);
\draw[draw=black,fill=black,opacity=0.7] (axis cs:2.14285714285714,0) rectangle (axis cs:2.71428571428571,0.0182080924855492);
\draw[draw=black,fill=black,opacity=0.7] (axis cs:2.71428571428571,0) rectangle (axis cs:3.28571428571428,0.0184038784262539);
\draw[draw=black,fill=black,opacity=0.7] (axis cs:3.28571428571428,0) rectangle (axis cs:3.85714285714286,0.0192849151594256);
\draw[draw=black,fill=black,opacity=0.7] (axis cs:3.85714285714286,0) rectangle (axis cs:4.42857142857143,0.0177512586239046);
\draw[draw=black,fill=black,opacity=0.7] (axis cs:4.42857142857143,0) rectangle (axis cs:5,0.0173270557523774);
\draw[draw=black,fill=black,opacity=0.7] (axis cs:5,0) rectangle (axis cs:5.57142857142857,0.0152386723848593);
\draw[draw=black,fill=black,opacity=0.7] (axis cs:5.57142857142857,0) rectangle (axis cs:6.14285714285714,0.0147165765429796);
\draw[draw=black,fill=black,opacity=0.7] (axis cs:6.14285714285714,0) rectangle (axis cs:6.71428571428571,0.0138029088196905);
\draw[draw=black,fill=black,opacity=0.7] (axis cs:6.71428571428571,0) rectangle (axis cs:7.28571428571428,0.0111271676300578);
\draw[draw=black,fill=black,opacity=0.7] (axis cs:7.28571428571428,0) rectangle (axis cs:7.85714285714285,0.00894089129218724);
\draw[draw=black,fill=black,opacity=0.7] (axis cs:7.85714285714285,0) rectangle (axis cs:8.42857142857143,0.010311392877121);
\draw[draw=black,fill=black,opacity=0.7] (axis cs:8.42857142857143,0) rectangle (axis cs:9,0.00773354465784074);
\draw[draw=black,fill=black,opacity=0.7] (axis cs:9,0) rectangle (axis cs:9.57142857142857,0.00802722356889803);
\draw[draw=black,fill=black,opacity=0.7] (axis cs:9.57142857142857,0) rectangle (axis cs:10.1428571428571,0.00867984337124741);
\draw[draw=black,fill=black,opacity=0.7] (axis cs:10.1428571428571,0) rectangle (axis cs:10.7142857142857,0.00433992168562373);
\draw[draw=black,fill=black,opacity=0.7] (axis cs:10.7142857142857,0) rectangle (axis cs:11.2857142857143,0.00404624277456646);
\draw[draw=black,fill=black,opacity=0.7] (axis cs:11.2857142857143,0) rectangle (axis cs:11.8571428571429,0.00388308782397913);
\draw[draw=black,fill=black,opacity=0.7] (axis cs:11.8571428571429,0) rectangle (axis cs:12.4285714285714,0.00306731307104232);
\draw[draw=black,fill=black,opacity=0.7] (axis cs:12.4285714285714,0) rectangle (axis cs:13,0.00166418049599105);
\draw[draw=black,fill=black,opacity=0.7] (axis cs:13,0) rectangle (axis cs:13.5714285714286,0.000848405743054264);
\draw[draw=black,fill=black,opacity=0.7] (axis cs:13.5714285714286,0) rectangle (axis cs:14.1428571428571,0.00088103673317173);
\draw[draw=black,fill=black,opacity=0.7] (axis cs:14.1428571428571,0) rectangle (axis cs:14.7142857142857,0.000913667723289207);
\draw[draw=black,fill=black,opacity=0.7] (axis cs:14.7142857142857,0) rectangle (axis cs:15.2857142857143,0.000913667723289201);
\draw[draw=black,fill=black,opacity=0.7] (axis cs:15.2857142857143,0) rectangle (axis cs:15.8571428571429,0.000946298713406679);
\draw[draw=black,fill=black,opacity=0.7] (axis cs:15.8571428571429,0) rectangle (axis cs:16.4285714285714,0.000848405743054258);
\draw[draw=black,fill=black,opacity=0.7] (axis cs:16.4285714285714,0) rectangle (axis cs:17,0.000587357822114487);
\draw[draw=black,fill=black,opacity=0.7] (axis cs:17,0) rectangle (axis cs:17.5714285714286,0.000261047920939773);
\draw[draw=black,fill=black,opacity=0.7] (axis cs:17.5714285714286,0) rectangle (axis cs:18.1428571428571,0.000261047920939773);
\draw[draw=black,fill=black,opacity=0.7] (axis cs:18.1428571428571,0) rectangle (axis cs:18.7142857142857,0.000228416930822299);
\draw[draw=black,fill=black,opacity=0.7] (axis cs:18.7142857142857,0) rectangle (axis cs:19.2857142857143,0.000261047920939773);
\draw[draw=black,fill=black,opacity=0.7] (axis cs:19.2857142857143,0) rectangle (axis cs:19.8571428571429,0.000358940891292188);
\draw[draw=black,fill=black,opacity=0.7] (axis cs:19.8571428571429,0) rectangle (axis cs:20.4285714285714,0.000848405743054264);
\draw[draw=black,fill=black,opacity=0.7] (axis cs:20.4285714285714,0) rectangle (axis cs:21,0.00336099198209954);
\draw[draw=black,fill=black,opacity=0.7] (axis cs:21,0) rectangle (axis cs:21.5714285714286,0);
\draw[draw=black,fill=black,opacity=0.7] (axis cs:21.5714285714286,0) rectangle (axis cs:22.1428571428571,0);
\draw[draw=black,fill=black,opacity=0.7] (axis cs:22.1428571428571,0) rectangle (axis cs:22.7142857142857,0);
\draw[draw=black,fill=black,opacity=0.7] (axis cs:22.7142857142857,0) rectangle (axis cs:23.2857142857143,0);
\draw[draw=black,fill=black,opacity=0.7] (axis cs:23.2857142857143,0) rectangle (axis cs:23.8571428571429,0);
\draw[draw=black,fill=black,opacity=0.7] (axis cs:23.8571428571429,0) rectangle (axis cs:24.4285714285714,0);
\draw[draw=black,fill=black,opacity=0.7] (axis cs:24.4285714285714,0) rectangle (axis cs:25,0);
\draw[draw=black,fill=white!50.1960784313725!black,opacity=0.6] (axis cs:-15,0) rectangle (axis cs:-14.4285714285714,0.000135501896527005);
\draw[draw=black,fill=white!50.1960784313725!black,opacity=0.6] (axis cs:-14.4285714285714,0) rectangle (axis cs:-13.8571428571429,0.000290236320303005);
\draw[draw=black,fill=white!50.1960784313725!black,opacity=0.6] (axis cs:-13.8571428571429,0) rectangle (axis cs:-13.2857142857143,0.000654780132249851);
\draw[draw=black,fill=white!50.1960784313725!black,opacity=0.6] (axis cs:-13.2857142857143,0) rectangle (axis cs:-12.7142857142857,0.00147915618660447);
\draw[draw=black,fill=white!50.1960784313725!black,opacity=0.6] (axis cs:-12.7142857142857,0) rectangle (axis cs:-12.1428571428571,0.0024818702209043);
\draw[draw=black,fill=white!50.1960784313725!black,opacity=0.6] (axis cs:-12.1428571428571,0) rectangle (axis cs:-11.5714285714286,0.00530642910915434);
\draw[draw=black,fill=white!50.1960784313725!black,opacity=0.6] (axis cs:-11.5714285714286,0) rectangle (axis cs:-11,0.0103165024575174);
\draw[draw=black,fill=white!50.1960784313725!black,opacity=0.6] (axis cs:-11,0) rectangle (axis cs:-10.4285714285714,0.0168852847188329);
\draw[draw=black,fill=white!50.1960784313725!black,opacity=0.6] (axis cs:-10.4285714285714,0) rectangle (axis cs:-9.85714285714286,0.0273145597224921);
\draw[draw=black,fill=white!50.1960784313725!black,opacity=0.6] (axis cs:-9.85714285714286,0) rectangle (axis cs:-9.28571428571429,0.041218802717731);
\draw[draw=black,fill=white!50.1960784313725!black,opacity=0.6] (axis cs:-9.28571428571428,0) rectangle (axis cs:-8.71428571428571,0.0590316197734359);
\draw[draw=black,fill=white!50.1960784313725!black,opacity=0.6] (axis cs:-8.71428571428572,0) rectangle (axis cs:-8.14285714285714,0.0764204470238786);
\draw[draw=black,fill=white!50.1960784313725!black,opacity=0.6] (axis cs:-8.14285714285714,0) rectangle (axis cs:-7.57142857142857,0.0919515869832259);
\draw[draw=black,fill=white!50.1960784313725!black,opacity=0.6] (axis cs:-7.57142857142857,0) rectangle (axis cs:-7,0.10349285174419);
\draw[draw=black,fill=white!50.1960784313725!black,opacity=0.6] (axis cs:-7,0) rectangle (axis cs:-6.42857142857143,0.112663270418825);
\draw[draw=black,fill=white!50.1960784313725!black,opacity=0.6] (axis cs:-6.42857142857143,0) rectangle (axis cs:-5.85714285714286,0.112004993463439);
\draw[draw=black,fill=white!50.1960784313725!black,opacity=0.6] (axis cs:-5.85714285714286,0) rectangle (axis cs:-5.28571428571429,0.107521191997071);
\draw[draw=black,fill=white!50.1960784313725!black,opacity=0.6] (axis cs:-5.28571428571429,0) rectangle (axis cs:-4.71428571428572,0.100177863411091);
\draw[draw=black,fill=white!50.1960784313725!black,opacity=0.6] (axis cs:-4.71428571428572,0) rectangle (axis cs:-4.14285714285714,0.089344705335203);
\draw[draw=black,fill=white!50.1960784313725!black,opacity=0.6] (axis cs:-4.14285714285714,0) rectangle (axis cs:-3.57142857142857,0.0795841977563384);
\draw[draw=black,fill=white!50.1960784313725!black,opacity=0.6] (axis cs:-3.57142857142857,0) rectangle (axis cs:-3,0.0708211589770691);
\draw[draw=black,fill=white!50.1960784313725!black,opacity=0.6] (axis cs:-3,0) rectangle (axis cs:-2.42857142857143,0.065920361351712);
\draw[draw=black,fill=white!50.1960784313725!black,opacity=0.6] (axis cs:-2.42857142857143,0) rectangle (axis cs:-1.85714285714286,0.0635294085323484);
\draw[draw=black,fill=white!50.1960784313725!black,opacity=0.6] (axis cs:-1.85714285714286,0) rectangle (axis cs:-1.28571428571429,0.0622967783768447);
\draw[draw=black,fill=white!50.1960784313725!black,opacity=0.6] (axis cs:-1.28571428571429,0) rectangle (axis cs:-0.714285714285715,0.0606751266474408);
\draw[draw=black,fill=white!50.1960784313725!black,opacity=0.6] (axis cs:-0.714285714285715,0) rectangle (axis cs:-0.142857142857144,0.0402694152362579);
\draw[draw=black,fill=white!50.1960784313725!black,opacity=0.6] (axis cs:-0.142857142857144,0) rectangle (axis cs:0.428571428571427,0.0338842161895918);
\draw[draw=black,fill=white!50.1960784313725!black,opacity=0.6] (axis cs:0.428571428571427,0) rectangle (axis cs:1,0.0339690141506441);
\draw[draw=black,fill=white!50.1960784313725!black,opacity=0.6] (axis cs:1,0) rectangle (axis cs:1.57142857142857,0.033752211116201);
\draw[draw=black,fill=white!50.1960784313725!black,opacity=0.6] (axis cs:1.57142857142857,0) rectangle (axis cs:2.14285714285714,0.033561634255279);
\draw[draw=black,fill=white!50.1960784313725!black,opacity=0.6] (axis cs:2.14285714285714,0) rectangle (axis cs:2.71428571428571,0.0315299800131582);
\draw[draw=black,fill=white!50.1960784313725!black,opacity=0.6] (axis cs:2.71428571428571,0) rectangle (axis cs:3.28571428571428,0.0289475761270885);
\draw[draw=black,fill=white!50.1960784313725!black,opacity=0.6] (axis cs:3.28571428571428,0) rectangle (axis cs:3.85714285714286,0.0269753678782825);
\draw[draw=black,fill=white!50.1960784313725!black,opacity=0.6] (axis cs:3.85714285714286,0) rectangle (axis cs:4.42857142857143,0.0224635918268254);
\draw[draw=black,fill=white!50.1960784313725!black,opacity=0.6] (axis cs:4.42857142857143,0) rectangle (axis cs:5,0.0185139300945091);
\draw[draw=black,fill=white!50.1960784313725!black,opacity=0.6] (axis cs:5,0) rectangle (axis cs:5.57142857142857,0.0165705706365767);
\draw[draw=black,fill=white!50.1960784313725!black,opacity=0.6] (axis cs:5.57142857142857,0) rectangle (axis cs:6.14285714285714,0.013870148969661);
\draw[draw=black,fill=white!50.1960784313725!black,opacity=0.6] (axis cs:6.14285714285714,0) rectangle (axis cs:6.71428571428571,0.011546509995669);
\draw[draw=black,fill=white!50.1960784313725!black,opacity=0.6] (axis cs:6.71428571428571,0) rectangle (axis cs:7.28571428571428,0.0100962025999379);
\draw[draw=black,fill=white!50.1960784313725!black,opacity=0.6] (axis cs:7.28571428571428,0) rectangle (axis cs:7.85714285714285,0.00784250008866947);
\draw[draw=black,fill=white!50.1960784313725!black,opacity=0.6] (axis cs:7.85714285714285,0) rectangle (axis cs:8.42857142857143,0.00567097292110116);
\draw[draw=black,fill=white!50.1960784313725!black,opacity=0.6] (axis cs:8.42857142857143,0) rectangle (axis cs:9,0.00465252318268851);
\draw[draw=black,fill=white!50.1960784313725!black,opacity=0.6] (axis cs:9,0) rectangle (axis cs:9.57142857142857,0.00306496547886246);
\draw[draw=black,fill=white!50.1960784313725!black,opacity=0.6] (axis cs:9.57142857142857,0) rectangle (axis cs:10.1428571428571,0.00226768980381323);
\draw[draw=black,fill=white!50.1960784313725!black,opacity=0.6] (axis cs:10.1428571428571,0) rectangle (axis cs:10.7142857142857,0.00175890203749894);
\draw[draw=black,fill=white!50.1960784313725!black,opacity=0.6] (axis cs:10.7142857142857,0) rectangle (axis cs:11.2857142857143,0.00147565936346829);
\draw[draw=black,fill=white!50.1960784313725!black,opacity=0.6] (axis cs:11.2857142857143,0) rectangle (axis cs:11.8571428571429,0.00124574324226441);
\draw[draw=black,fill=white!50.1960784313725!black,opacity=0.6] (axis cs:11.8571428571429,0) rectangle (axis cs:12.4285714285714,0.000868960549340921);
\draw[draw=black,fill=white!50.1960784313725!black,opacity=0.6] (axis cs:12.4285714285714,0) rectangle (axis cs:13,0.000679257894203115);
\draw[draw=black,fill=white!50.1960784313725!black,opacity=0.6] (axis cs:13,0) rectangle (axis cs:13.5714285714286,0.000509661972098351);
\draw[draw=black,fill=white!50.1960784313725!black,opacity=0.6] (axis cs:13.5714285714286,0) rectangle (axis cs:14.1428571428571,0.000397763631740563);
\draw[draw=black,fill=white!50.1960784313725!black,opacity=0.6] (axis cs:14.1428571428571,0) rectangle (axis cs:14.7142857142857,0.000335695021073356);
\draw[draw=black,fill=white!50.1960784313725!black,opacity=0.6] (axis cs:14.7142857142857,0) rectangle (axis cs:15.2857142857143,0.000230790326987931);
\draw[draw=black,fill=white!50.1960784313725!black,opacity=0.6] (axis cs:15.2857142857143,0) rectangle (axis cs:15.8571428571429,0.000187080037785672);
\draw[draw=black,fill=white!50.1960784313725!black,opacity=0.6] (axis cs:15.8571428571429,0) rectangle (axis cs:16.4285714285714,0.000145992365935547);
\draw[draw=black,fill=white!50.1960784313725!black,opacity=0.6] (axis cs:16.4285714285714,0) rectangle (axis cs:17,9.26658131087904e-05);
\draw[draw=black,fill=white!50.1960784313725!black,opacity=0.6] (axis cs:17,0) rectangle (axis cs:17.5714285714286,9.2665813108791e-05);
\draw[draw=black,fill=white!50.1960784313725!black,opacity=0.6] (axis cs:17.5714285714286,0) rectangle (axis cs:18.1428571428571,8.56721668364294e-05);
\draw[draw=black,fill=white!50.1960784313725!black,opacity=0.6] (axis cs:18.1428571428571,0) rectangle (axis cs:18.7142857142857,8.65463726204735e-05);
\draw[draw=black,fill=white!50.1960784313725!black,opacity=0.6] (axis cs:18.7142857142857,0) rectangle (axis cs:19.2857142857143,8.56721668364294e-05);
\draw[draw=black,fill=white!50.1960784313725!black,opacity=0.6] (axis cs:19.2857142857143,0) rectangle (axis cs:19.8571428571429,9.79110478130622e-05);
\draw[draw=black,fill=white!50.1960784313725!black,opacity=0.6] (axis cs:19.8571428571429,0) rectangle (axis cs:20.4285714285714,7.16848742917062e-05);
\draw[draw=black,fill=white!50.1960784313725!black,opacity=0.6] (axis cs:20.4285714285714,0) rectangle (axis cs:21,6.90622569395698e-05);
\draw[draw=black,fill=white!50.1960784313725!black,opacity=0.6] (axis cs:21,0) rectangle (axis cs:21.5714285714286,7.25590800757514e-05);
\draw[draw=black,fill=white!50.1960784313725!black,opacity=0.6] (axis cs:21.5714285714286,0) rectangle (axis cs:22.1428571428571,7.16848742917062e-05);
\draw[draw=black,fill=white!50.1960784313725!black,opacity=0.6] (axis cs:22.1428571428571,0) rectangle (axis cs:22.7142857142857,7.51816974278861e-05);
\draw[draw=black,fill=white!50.1960784313725!black,opacity=0.6] (axis cs:22.7142857142857,0) rectangle (axis cs:23.2857142857143,7.5181697427887e-05);
\draw[draw=black,fill=white!50.1960784313725!black,opacity=0.6] (axis cs:23.2857142857143,0) rectangle (axis cs:23.8571428571429,7.78043147800226e-05);
\draw[draw=black,fill=white!50.1960784313725!black,opacity=0.6] (axis cs:23.8571428571429,0) rectangle (axis cs:24.4285714285714,8.13011379162034e-05);
\draw[draw=black,fill=white!50.1960784313725!black,opacity=0.6] (axis cs:24.4285714285714,0) rectangle (axis cs:25,6.11944048831631e-05);
\draw[draw=black,fill=white,opacity=0.5] (axis cs:-15,0) rectangle (axis cs:-14.4285714285714,0.00190133249248095);
\draw[draw=black,fill=white,opacity=0.5] (axis cs:-14.4285714285714,0) rectangle (axis cs:-13.8571428571429,0.00357741659345743);
\draw[draw=black,fill=white,opacity=0.5] (axis cs:-13.8571428571429,0) rectangle (axis cs:-13.2857142857143,0.00797123565393978);
\draw[draw=black,fill=white,opacity=0.5] (axis cs:-13.2857142857143,0) rectangle (axis cs:-12.7142857142857,0.0129945698697185);
\draw[draw=black,fill=white,opacity=0.5] (axis cs:-12.7142857142857,0) rectangle (axis cs:-12.1428571428571,0.0208812144246652);
\draw[draw=black,fill=white,opacity=0.5] (axis cs:-12.1428571428571,0) rectangle (axis cs:-11.5714285714286,0.0323324885464776);
\draw[draw=black,fill=white,opacity=0.5] (axis cs:-11.5714285714286,0) rectangle (axis cs:-11,0.0443060635236999);
\draw[draw=black,fill=white,opacity=0.5] (axis cs:-11,0) rectangle (axis cs:-10.4285714285714,0.0607836227135813);
\draw[draw=black,fill=white,opacity=0.5] (axis cs:-10.4285714285714,0) rectangle (axis cs:-9.85714285714286,0.0734506479790597);
\draw[draw=black,fill=white,opacity=0.5] (axis cs:-9.85714285714286,0) rectangle (axis cs:-9.28571428571429,0.081902872556343);
\draw[draw=black,fill=white,opacity=0.5] (axis cs:-9.28571428571428,0) rectangle (axis cs:-8.71428571428571,0.0898288618084959);
\draw[draw=black,fill=white,opacity=0.5] (axis cs:-8.71428571428572,0) rectangle (axis cs:-8.14285714285714,0.0941105484819408);
\draw[draw=black,fill=white,opacity=0.5] (axis cs:-8.14285714285714,0) rectangle (axis cs:-7.57142857142857,0.0903167360538507);
\draw[draw=black,fill=white,opacity=0.5] (axis cs:-7.57142857142857,0) rectangle (axis cs:-7,0.0861304602711298);
\draw[draw=black,fill=white,opacity=0.5] (axis cs:-7,0) rectangle (axis cs:-6.42857142857143,0.0833527246483849);
\draw[draw=black,fill=white,opacity=0.5] (axis cs:-6.42857142857143,0) rectangle (axis cs:-5.85714285714286,0.08358190750961);
\draw[draw=black,fill=white,opacity=0.5] (axis cs:-5.85714285714286,0) rectangle (axis cs:-5.28571428571429,0.0835789566573195);
\draw[draw=black,fill=white,opacity=0.5] (axis cs:-5.28571428571429,0) rectangle (axis cs:-4.71428571428572,0.0786470988625448);
\draw[draw=black,fill=white,opacity=0.5] (axis cs:-4.71428571428572,0) rectangle (axis cs:-4.14285714285714,0.0702676619750926);
\draw[draw=black,fill=white,opacity=0.5] (axis cs:-4.14285714285714,0) rectangle (axis cs:-3.57142857142857,0.0633951269906309);
\draw[draw=black,fill=white,opacity=0.5] (axis cs:-3.57142857142857,0) rectangle (axis cs:-3,0.0574678483565437);
\draw[draw=black,fill=white,opacity=0.5] (axis cs:-3,0) rectangle (axis cs:-2.42857142857143,0.0542681408562643);
\draw[draw=black,fill=white,opacity=0.5] (axis cs:-2.42857142857143,0) rectangle (axis cs:-1.85714285714286,0.0532018995619811);
\draw[draw=black,fill=white,opacity=0.5] (axis cs:-1.85714285714286,0) rectangle (axis cs:-1.28571428571429,0.0502461291843788);
\draw[draw=black,fill=white,opacity=0.5] (axis cs:-1.28571428571429,0) rectangle (axis cs:-0.714285714285715,0.0488543105207159);
\draw[draw=black,fill=white,opacity=0.5] (axis cs:-0.714285714285715,0) rectangle (axis cs:-0.142857142857144,0.0297180334171376);
\draw[draw=black,fill=white,opacity=0.5] (axis cs:-0.142857142857144,0) rectangle (axis cs:0.428571428571427,0.0290904854967015);
\draw[draw=black,fill=white,opacity=0.5] (axis cs:0.428571428571427,0) rectangle (axis cs:1,0.0284570358716845);
\draw[draw=black,fill=white,opacity=0.5] (axis cs:1,0) rectangle (axis cs:1.57142857142857,0.0278452258301311);
\draw[draw=black,fill=white,opacity=0.5] (axis cs:1.57142857142857,0) rectangle (axis cs:2.14285714285714,0.0261671744942941);
\draw[draw=black,fill=white,opacity=0.5] (axis cs:2.14285714285714,0) rectangle (axis cs:2.71428571428571,0.0244645327227037);
\draw[draw=black,fill=white,opacity=0.5] (axis cs:2.71428571428571,0) rectangle (axis cs:3.28571428571428,0.0224943470101121);
\draw[draw=black,fill=white,opacity=0.5] (axis cs:3.28571428571428,0) rectangle (axis cs:3.85714285714286,0.0207907216210914);
\draw[draw=black,fill=white,opacity=0.5] (axis cs:3.85714285714286,0) rectangle (axis cs:4.42857142857143,0.0186513537105141);
\draw[draw=black,fill=white,opacity=0.5] (axis cs:4.42857142857143,0) rectangle (axis cs:5,0.0157841089016253);
\draw[draw=black,fill=white,opacity=0.5] (axis cs:5,0) rectangle (axis cs:5.57142857142857,0.0138237593633354);
\draw[draw=black,fill=white,opacity=0.5] (axis cs:5.57142857142857,0) rectangle (axis cs:6.14285714285714,0.0120001326478363);
\draw[draw=black,fill=white,opacity=0.5] (axis cs:6.14285714285714,0) rectangle (axis cs:6.71428571428571,0.0100889639810539);
\draw[draw=black,fill=white,opacity=0.5] (axis cs:6.71428571428571,0) rectangle (axis cs:7.28571428571428,0.00879058897325514);
\draw[draw=black,fill=white,opacity=0.5] (axis cs:7.28571428571428,0) rectangle (axis cs:7.85714285714285,0.00671613981306775);
\draw[draw=black,fill=white,opacity=0.5] (axis cs:7.85714285714285,0) rectangle (axis cs:8.42857142857143,0.00581612986447999);
\draw[draw=black,fill=white,opacity=0.5] (axis cs:8.42857142857143,0) rectangle (axis cs:9,0.00495153014337768);
\draw[draw=black,fill=white,opacity=0.5] (axis cs:9,0) rectangle (axis cs:9.57142857142857,0.00354298998340217);
\draw[draw=black,fill=white,opacity=0.5] (axis cs:9.57142857142857,0) rectangle (axis cs:10.1428571428571,0.0028721628960395);
\draw[draw=black,fill=white,opacity=0.5] (axis cs:10.1428571428571,0) rectangle (axis cs:10.7142857142857,0.00242461696532103);
\draw[draw=black,fill=white,opacity=0.5] (axis cs:10.7142857142857,0) rectangle (axis cs:11.2857142857143,0.00169083836242873);
\draw[draw=black,fill=white,opacity=0.5] (axis cs:11.2857142857143,0) rectangle (axis cs:11.8571428571429,0.00137214631505997);
\draw[draw=black,fill=white,opacity=0.5] (axis cs:11.8571428571429,0) rectangle (axis cs:12.4285714285714,0.00107017576400376);
\draw[draw=black,fill=white,opacity=0.5] (axis cs:12.4285714285714,0) rectangle (axis cs:13,0.000835091198197786);
\draw[draw=black,fill=white,opacity=0.5] (axis cs:13,0) rectangle (axis cs:13.5714285714286,0.000636400477307386);
\draw[draw=black,fill=white,opacity=0.5] (axis cs:13.5714285714286,0) rectangle (axis cs:14.1428571428571,0.000489841480214956);
\draw[draw=black,fill=white,opacity=0.5] (axis cs:14.1428571428571,0) rectangle (axis cs:14.7142857142857,0.000353118657424037);
\draw[draw=black,fill=white,opacity=0.5] (axis cs:14.7142857142857,0) rectangle (axis cs:15.2857142857143,0.000231150096085371);
\draw[draw=black,fill=white,opacity=0.5] (axis cs:15.2857142857143,0) rectangle (axis cs:15.8571428571429,0.000197707103460254);
\draw[draw=black,fill=white,opacity=0.5] (axis cs:15.8571428571429,0) rectangle (axis cs:16.4285714285714,0.000150493466813029);
\draw[draw=black,fill=white,opacity=0.5] (axis cs:16.4285714285714,0) rectangle (axis cs:17,0.000119017709048212);
\draw[draw=black,fill=white,opacity=0.5] (axis cs:17,0) rectangle (axis cs:17.5714285714286,9.54108907246004e-05);
\draw[draw=black,fill=white,opacity=0.5] (axis cs:17.5714285714286,0) rectangle (axis cs:18.1428571428571,7.7705776981891e-05);
\draw[draw=black,fill=white,opacity=0.5] (axis cs:18.1428571428571,0) rectangle (axis cs:18.7142857142857,7.08204549708365e-05);
\draw[draw=black,fill=white,opacity=0.5] (axis cs:18.7142857142857,0) rectangle (axis cs:19.2857142857143,6.59023678200848e-05);
\draw[draw=black,fill=white,opacity=0.5] (axis cs:19.2857142857143,0) rectangle (axis cs:19.8571428571429,6.49187503899343e-05);
\draw[draw=black,fill=white,opacity=0.5] (axis cs:19.8571428571429,0) rectangle (axis cs:20.4285714285714,7.27876898311384e-05);
\draw[draw=black,fill=white,opacity=0.5] (axis cs:20.4285714285714,0) rectangle (axis cs:21,7.96730118421911e-05);
\draw[draw=black,fill=white,opacity=0.5] (axis cs:21,0) rectangle (axis cs:21.5714285714286,8.55747164230952e-05);
\draw[draw=black,fill=white,opacity=0.5] (axis cs:21.5714285714286,0) rectangle (axis cs:22.1428571428571,6.39351329597838e-05);
\draw[draw=black,fill=white,opacity=0.5] (axis cs:22.1428571428571,0) rectangle (axis cs:22.7142857142857,6.49187503899335e-05);
\draw[draw=black,fill=white,opacity=0.5] (axis cs:22.7142857142857,0) rectangle (axis cs:23.2857142857143,6.39351329597838e-05);
\draw[draw=black,fill=white,opacity=0.5] (axis cs:23.2857142857143,0) rectangle (axis cs:23.8571428571429,6.59023678200848e-05);
\draw[draw=black,fill=white,opacity=0.5] (axis cs:23.8571428571429,0) rectangle (axis cs:24.4285714285714,7.573854212159e-05);
\draw[draw=black,fill=white,opacity=0.5] (axis cs:24.4285714285714,0) rectangle (axis cs:25,6.68859852502345e-05);
\end{axis}

\end{tikzpicture}}
    \hspace*{1em}%
    \subfigure[yNorth. $D_{JS}$: 0.0446 (multiple), 0.0166 (single)]{
        \label{fig:hist_y_coords}
        % This file was created by tikzplotlib v0.9.2.
\begin{tikzpicture}

\begin{axis}[
legend cell align={left},
legend style={fill opacity=0.8, draw opacity=1, text opacity=1, draw=white!80!black},
width=3.4in,
height=2.2in,
tick align=outside,
tick pos=left,
x grid style={white!69.0196078431373!black},
xlabel={y coordinate (NM)},
xmin=-37, xmax=5,
xtick style={color=black},
y grid style={white!69.0196078431373!black},
ytick={0.00, 0.02, 0.04, 0.06, 0.08, 0.10, 0.12, 0.14},
ylabel={density},
ymin=0, ymax=0.148014171172852,
ytick style={color=black},
yticklabel style={/pgf/number format/.cd,fixed,precision=2},
tick label style={font=\tiny},
label style={font=\scriptsize},
legend entries={{actual},{synthetic--multiple}, {synthetic--single}},
legend style={at={(0.03,0.95)}, anchor=north west, draw=black},
legend style={font=\scriptsize}
]

\addlegendimage{area legend, draw=black, fill=black, opacity=0.6}
\addlegendimage{area legend, draw=black, fill=white!50.1960784313725!black, opacity=0.7}
\addlegendimage{area legend, draw=black, fill=white, opacity=0.5}

\draw[draw=black,fill=black,opacity=0.7] (axis cs:-35,0) rectangle (axis cs:-34.4285714285714,0);
\draw[draw=black,fill=black,opacity=0.7] (axis cs:-34.4285714285714,0) rectangle (axis cs:-33.8571428571429,0);
\draw[draw=black,fill=black,opacity=0.7] (axis cs:-33.8571428571429,0) rectangle (axis cs:-33.2857142857143,0);
\draw[draw=black,fill=black,opacity=0.7] (axis cs:-33.2857142857143,0) rectangle (axis cs:-32.7142857142857,0);
\draw[draw=black,fill=black,opacity=0.7] (axis cs:-32.7142857142857,0) rectangle (axis cs:-32.1428571428571,0);
\draw[draw=black,fill=black,opacity=0.7] (axis cs:-32.1428571428572,0) rectangle (axis cs:-31.5714285714286,0);
\draw[draw=black,fill=black,opacity=0.7] (axis cs:-31.5714285714286,0) rectangle (axis cs:-31,0);
\draw[draw=black,fill=black,opacity=0.7] (axis cs:-31,0) rectangle (axis cs:-30.4285714285714,0);
\draw[draw=black,fill=black,opacity=0.7] (axis cs:-30.4285714285714,0) rectangle (axis cs:-29.8571428571429,0.00088103673317173);
\draw[draw=black,fill=black,opacity=0.7] (axis cs:-29.8571428571429,0) rectangle (axis cs:-29.2857142857143,0.00411150475480141);
\draw[draw=black,fill=black,opacity=0.7] (axis cs:-29.2857142857143,0) rectangle (axis cs:-28.7142857142857,0.00443781465597615);
\draw[draw=black,fill=black,opacity=0.7] (axis cs:-28.7142857142857,0) rectangle (axis cs:-28.1428571428571,0.00479675554726831);
\draw[draw=black,fill=black,opacity=0.7] (axis cs:-28.1428571428571,0) rectangle (axis cs:-27.5714285714286,0.005188327428678);
\draw[draw=black,fill=black,opacity=0.7] (axis cs:-27.5714285714286,0) rectangle (axis cs:-27,0.00597147119149728);
\draw[draw=black,fill=black,opacity=0.7] (axis cs:-27,0) rectangle (axis cs:-26.4285714285714,0.00636304307290698);
\draw[draw=black,fill=black,opacity=0.7] (axis cs:-26.4285714285714,0) rectangle (axis cs:-25.8571428571429,0.00695040089502142);
\draw[draw=black,fill=black,opacity=0.7] (axis cs:-25.8571428571429,0) rectangle (axis cs:-25.2857142857143,0.0081251165392504);
\draw[draw=black,fill=black,opacity=0.7] (axis cs:-25.2857142857143,0) rectangle (axis cs:-24.7142857142857,0.0108008577288831);
\draw[draw=black,fill=black,opacity=0.7] (axis cs:-24.7142857142857,0) rectangle (axis cs:-24.1428571428571,0.0122366212940519);
\draw[draw=black,fill=black,opacity=0.7] (axis cs:-24.1428571428571,0) rectangle (axis cs:-23.5714285714286,0.0130523960469886);
\draw[draw=black,fill=black,opacity=0.7] (axis cs:-23.5714285714286,0) rectangle (axis cs:-23,0.0144228976319224);
\draw[draw=black,fill=black,opacity=0.7] (axis cs:-23,0) rectangle (axis cs:-22.4285714285714,0.0171965317919076);
\draw[draw=black,fill=black,opacity=0.7] (axis cs:-22.4285714285714,0) rectangle (axis cs:-21.8571428571429,0.0166744359500279);
\draw[draw=black,fill=black,opacity=0.7] (axis cs:-21.8571428571429,0) rectangle (axis cs:-21.2857142857143,0.0162176020883833);
\draw[draw=black,fill=black,opacity=0.7] (axis cs:-21.2857142857143,0) rectangle (axis cs:-20.7142857142857,0.0155976132761515);
\draw[draw=black,fill=black,opacity=0.7] (axis cs:-20.7142857142857,0) rectangle (axis cs:-20.1428571428571,0.0217322394182361);
\draw[draw=black,fill=black,opacity=0.7] (axis cs:-20.1428571428571,0) rectangle (axis cs:-19.5714285714286,0.0288784262539623);
\draw[draw=black,fill=black,opacity=0.7] (axis cs:-19.5714285714286,0) rectangle (axis cs:-19,0.0267574118963266);
\draw[draw=black,fill=black,opacity=0.7] (axis cs:-19,0) rectangle (axis cs:-18.4285714285714,0.0256479582323327);
\draw[draw=black,fill=black,opacity=0.7] (axis cs:-18.4285714285714,0) rectangle (axis cs:-17.8571428571429,0.026920566846914);
\draw[draw=black,fill=black,opacity=0.7] (axis cs:-17.8571428571429,0) rectangle (axis cs:-17.2857142857143,0.02878053328361);
\draw[draw=black,fill=black,opacity=0.7] (axis cs:-17.2857142857143,0) rectangle (axis cs:-16.7142857142857,0.0332509789297034);
\draw[draw=black,fill=black,opacity=0.7] (axis cs:-16.7142857142857,0) rectangle (axis cs:-16.1428571428571,0.0348172664553421);
\draw[draw=black,fill=black,opacity=0.7] (axis cs:-16.1428571428571,0) rectangle (axis cs:-15.5714285714286,0.0400055938840203);
\draw[draw=black,fill=black,opacity=0.7] (axis cs:-15.5714285714286,0) rectangle (axis cs:-15,0.0478044005220957);
\draw[draw=black,fill=black,opacity=0.7] (axis cs:-15,0) rectangle (axis cs:-14.4285714285714,0.0447044564609362);
\draw[draw=black,fill=black,opacity=0.7] (axis cs:-14.4285714285714,0) rectangle (axis cs:-13.8571428571429,0.0515569643856049);
\draw[draw=black,fill=black,opacity=0.7] (axis cs:-13.8571428571429,0) rectangle (axis cs:-13.2857142857143,0.0447697184411711);
\draw[draw=black,fill=black,opacity=0.7] (axis cs:-13.2857142857143,0) rectangle (axis cs:-12.7142857142857,0.0401034868543724);
\draw[draw=black,fill=black,opacity=0.7] (axis cs:-12.7142857142857,0) rectangle (axis cs:-12.1428571428571,0.0402013798247251);
\draw[draw=black,fill=black,opacity=0.7] (axis cs:-12.1428571428571,0) rectangle (axis cs:-11.5714285714286,0.0375909006153271);
\draw[draw=black,fill=black,opacity=0.7] (axis cs:-11.5714285714286,0) rectangle (axis cs:-11,0.0402666418049598);
\draw[draw=black,fill=black,opacity=0.7] (axis cs:-11,0) rectangle (axis cs:-10.4285714285714,0.0369709118030954);
\draw[draw=black,fill=black,opacity=0.7] (axis cs:-10.4285714285714,0) rectangle (axis cs:-9.85714285714286,0.0346214805146372);
\draw[draw=black,fill=black,opacity=0.7] (axis cs:-9.85714285714286,0) rectangle (axis cs:-9.28571428571429,0.0346867424948724);
\draw[draw=black,fill=black,opacity=0.7] (axis cs:-9.28571428571429,0) rectangle (axis cs:-8.71428571428571,0.0355351482379264);
\draw[draw=black,fill=black,opacity=0.7] (axis cs:-8.71428571428572,0) rectangle (axis cs:-8.14285714285715,0.0352088383367519);
\draw[draw=black,fill=black,opacity=0.7] (axis cs:-8.14285714285715,0) rectangle (axis cs:-7.57142857142857,0.0347193734849897);
\draw[draw=black,fill=black,opacity=0.7] (axis cs:-7.57142857142857,0) rectangle (axis cs:-7,0.034882528435577);
\draw[draw=black,fill=black,opacity=0.7] (axis cs:-7,0) rectangle (axis cs:-6.42857142857143,0.0349151594256947);
\draw[draw=black,fill=black,opacity=0.7] (axis cs:-6.42857142857143,0) rectangle (axis cs:-5.85714285714286,0.0365793399216855);
\draw[draw=black,fill=black,opacity=0.7] (axis cs:-5.85714285714286,0) rectangle (axis cs:-5.28571428571429,0.0371340667536828);
\draw[draw=black,fill=black,opacity=0.7] (axis cs:-5.28571428571429,0) rectangle (axis cs:-4.71428571428571,0.0370035427932127);
\draw[draw=black,fill=black,opacity=0.7] (axis cs:-4.71428571428572,0) rectangle (axis cs:-4.14285714285715,0.0367751258623906);
\draw[draw=black,fill=black,opacity=0.7] (axis cs:-4.14285714285715,0) rectangle (axis cs:-3.57142857142857,0.0371340667536825);
\draw[draw=black,fill=black,opacity=0.7] (axis cs:-3.57142857142857,0) rectangle (axis cs:-3,0.0365140779414506);
\draw[draw=black,fill=black,opacity=0.7] (axis cs:-3,0) rectangle (axis cs:-2.42857142857143,0.0362856610106285);
\draw[draw=black,fill=black,opacity=0.7] (axis cs:-2.42857142857143,0) rectangle (axis cs:-1.85714285714286,0.0367424948722731);
\draw[draw=black,fill=black,opacity=0.7] (axis cs:-1.85714285714286,0) rectangle (axis cs:-1.28571428571428,0.03690564982286);
\draw[draw=black,fill=black,opacity=0.7] (axis cs:-1.28571428571428,0) rectangle (axis cs:-0.714285714285715,0.0363509229908634);
\draw[draw=black,fill=black,opacity=0.7] (axis cs:-0.714285714285715,0) rectangle (axis cs:-0.142857142857146,0.0380803654670894);
\draw[draw=black,fill=black,opacity=0.7] (axis cs:-0.142857142857146,0) rectangle (axis cs:0.428571428571423,0.039222450121201);
\draw[draw=black,fill=black,opacity=0.7] (axis cs:0.428571428571423,0) rectangle (axis cs:1,0.0440192056684688);
\draw[draw=black,fill=black,opacity=0.7] (axis cs:1,0) rectangle (axis cs:1.57142857142857,0.140965877307478);
\draw[draw=black,fill=black,opacity=0.7] (axis cs:1.57142857142857,0) rectangle (axis cs:2.14285714285714,0.0619336192429613);
\draw[draw=black,fill=black,opacity=0.7] (axis cs:2.14285714285714,0) rectangle (axis cs:2.71428571428572,0);
\draw[draw=black,fill=black,opacity=0.7] (axis cs:2.71428571428572,0) rectangle (axis cs:3.28571428571428,0);
\draw[draw=black,fill=black,opacity=0.7] (axis cs:3.28571428571428,0) rectangle (axis cs:3.85714285714285,0);
\draw[draw=black,fill=black,opacity=0.7] (axis cs:3.85714285714285,0) rectangle (axis cs:4.42857142857142,0);
\draw[draw=black,fill=black,opacity=0.7] (axis cs:4.42857142857142,0) rectangle (axis cs:5,0);
\draw[draw=black,fill=white!50.1960784313725!black,opacity=0.6] (axis cs:-35,0) rectangle (axis cs:-34.4285714285714,0.000716522537277172);
\draw[draw=black,fill=white!50.1960784313725!black,opacity=0.6] (axis cs:-34.4285714285714,0) rectangle (axis cs:-33.8571428571429,0.00106034837018305);
\draw[draw=black,fill=white!50.1960784313725!black,opacity=0.6] (axis cs:-33.8571428571429,0) rectangle (axis cs:-33.2857142857143,0.00128694096744166);
\draw[draw=black,fill=white!50.1960784313725!black,opacity=0.6] (axis cs:-33.2857142857143,0) rectangle (axis cs:-32.7142857142857,0.00164826429820536);
\draw[draw=black,fill=white!50.1960784313725!black,opacity=0.6] (axis cs:-32.7142857142857,0) rectangle (axis cs:-32.1428571428571,0.00205858062297092);
\draw[draw=black,fill=white!50.1960784313725!black,opacity=0.6] (axis cs:-32.1428571428572,0) rectangle (axis cs:-31.5714285714286,0.00258613018338377);
\draw[draw=black,fill=white!50.1960784313725!black,opacity=0.6] (axis cs:-31.5714285714286,0) rectangle (axis cs:-31,0.0031889189845852);
\draw[draw=black,fill=white!50.1960784313725!black,opacity=0.6] (axis cs:-31,0) rectangle (axis cs:-30.4285714285714,0.00394656064182823);
\draw[draw=black,fill=white!50.1960784313725!black,opacity=0.6] (axis cs:-30.4285714285714,0) rectangle (axis cs:-29.8571428571429,0.00487042852872038);
\draw[draw=black,fill=white!50.1960784313725!black,opacity=0.6] (axis cs:-29.8571428571429,0) rectangle (axis cs:-29.2857142857143,0.00581354366325614);
\draw[draw=black,fill=white!50.1960784313725!black,opacity=0.6] (axis cs:-29.2857142857143,0) rectangle (axis cs:-28.7142857142857,0.00674528542418436);
\draw[draw=black,fill=white!50.1960784313725!black,opacity=0.6] (axis cs:-28.7142857142857,0) rectangle (axis cs:-28.1428571428571,0.0082194496187045);
\draw[draw=black,fill=white!50.1960784313725!black,opacity=0.6] (axis cs:-28.1428571428571,0) rectangle (axis cs:-27.5714285714286,0.00997532352873542);
\draw[draw=black,fill=white!50.1960784313725!black,opacity=0.6] (axis cs:-27.5714285714286,0) rectangle (axis cs:-27,0.0117548190608743);
\draw[draw=black,fill=white!50.1960784313725!black,opacity=0.6] (axis cs:-27,0) rectangle (axis cs:-26.4285714285714,0.0128492875518801);
\draw[draw=black,fill=white!50.1960784313725!black,opacity=0.6] (axis cs:-26.4285714285714,0) rectangle (axis cs:-25.8571428571429,0.0147407670703089);
\draw[draw=black,fill=white!50.1960784313725!black,opacity=0.6] (axis cs:-25.8571428571429,0) rectangle (axis cs:-25.2857142857143,0.0164240263642299);
\draw[draw=black,fill=white!50.1960784313725!black,opacity=0.6] (axis cs:-25.2857142857143,0) rectangle (axis cs:-24.7142857142857,0.018433613993199);
\draw[draw=black,fill=white!50.1960784313725!black,opacity=0.6] (axis cs:-24.7142857142857,0) rectangle (axis cs:-24.1428571428571,0.0210109954276539);
\draw[draw=black,fill=white!50.1960784313725!black,opacity=0.6] (axis cs:-24.1428571428571,0) rectangle (axis cs:-23.5714285714286,0.0243127732734218);
\draw[draw=black,fill=white!50.1960784313725!black,opacity=0.6] (axis cs:-23.5714285714286,0) rectangle (axis cs:-23,0.0259365410746263);
\draw[draw=black,fill=white!50.1960784313725!black,opacity=0.6] (axis cs:-23,0) rectangle (axis cs:-22.4285714285714,0.0278516422151633);
\draw[draw=black,fill=white!50.1960784313725!black,opacity=0.6] (axis cs:-22.4285714285714,0) rectangle (axis cs:-21.8571428571429,0.0287956322245918);
\draw[draw=black,fill=white!50.1960784313725!black,opacity=0.6] (axis cs:-21.8571428571429,0) rectangle (axis cs:-21.2857142857143,0.0310046913291398);
\draw[draw=black,fill=white!50.1960784313725!black,opacity=0.6] (axis cs:-21.2857142857143,0) rectangle (axis cs:-20.7142857142857,0.0327176963694193);
\draw[draw=black,fill=white!50.1960784313725!black,opacity=0.6] (axis cs:-20.7142857142857,0) rectangle (axis cs:-20.1428571428571,0.0344709456547715);
\draw[draw=black,fill=white!50.1960784313725!black,opacity=0.6] (axis cs:-20.1428571428571,0) rectangle (axis cs:-19.5714285714286,0.0366870037584624);
\draw[draw=black,fill=white!50.1960784313725!black,opacity=0.6] (axis cs:-19.5714285714286,0) rectangle (axis cs:-19,0.037982693474833);
\draw[draw=black,fill=white!50.1960784313725!black,opacity=0.6] (axis cs:-19,0) rectangle (axis cs:-18.4285714285714,0.0395250979109989);
\draw[draw=black,fill=white!50.1960784313725!black,opacity=0.6] (axis cs:-18.4285714285714,0) rectangle (axis cs:-17.8571428571429,0.0413675844354256);
\draw[draw=black,fill=white!50.1960784313725!black,opacity=0.6] (axis cs:-17.8571428571429,0) rectangle (axis cs:-17.2857142857143,0.0424051860583938);
\draw[draw=black,fill=white!50.1960784313725!black,opacity=0.6] (axis cs:-17.2857142857143,0) rectangle (axis cs:-16.7142857142857,0.0413947055571052);
\draw[draw=black,fill=white!50.1960784313725!black,opacity=0.6] (axis cs:-16.7142857142857,0) rectangle (axis cs:-16.1428571428571,0.0425565394148636);
\draw[draw=black,fill=white!50.1960784313725!black,opacity=0.6] (axis cs:-16.1428571428571,0) rectangle (axis cs:-15.5714285714286,0.0425206695442553);
\draw[draw=black,fill=white!50.1960784313725!black,opacity=0.6] (axis cs:-15.5714285714286,0) rectangle (axis cs:-15,0.0393361249341343);
\draw[draw=black,fill=white!50.1960784313725!black,opacity=0.6] (axis cs:-15,0) rectangle (axis cs:-14.4285714285714,0.0388278226213653);
\draw[draw=black,fill=white!50.1960784313725!black,opacity=0.6] (axis cs:-14.4285714285714,0) rectangle (axis cs:-13.8571428571429,0.0371909316967673);
\draw[draw=black,fill=white!50.1960784313725!black,opacity=0.6] (axis cs:-13.8571428571429,0) rectangle (axis cs:-13.2857142857143,0.0355067975279536);
\draw[draw=black,fill=white!50.1960784313725!black,opacity=0.6] (axis cs:-13.2857142857143,0) rectangle (axis cs:-12.7142857142857,0.0342146073111544);
\draw[draw=black,fill=white!50.1960784313725!black,opacity=0.6] (axis cs:-12.7142857142857,0) rectangle (axis cs:-12.1428571428571,0.032687950623061);
\draw[draw=black,fill=white!50.1960784313725!black,opacity=0.6] (axis cs:-12.1428571428571,0) rectangle (axis cs:-11.5714285714286,0.0321743990609342);
\draw[draw=black,fill=white!50.1960784313725!black,opacity=0.6] (axis cs:-11.5714285714286,0) rectangle (axis cs:-11,0.0314220066530485);
\draw[draw=black,fill=white!50.1960784313725!black,opacity=0.6] (axis cs:-11,0) rectangle (axis cs:-10.4285714285714,0.0311026773171438);
\draw[draw=black,fill=white!50.1960784313725!black,opacity=0.6] (axis cs:-10.4285714285714,0) rectangle (axis cs:-9.85714285714286,0.0300825731920335);
\draw[draw=black,fill=white!50.1960784313725!black,opacity=0.6] (axis cs:-9.85714285714286,0) rectangle (axis cs:-9.28571428571429,0.0298026132263087);
\draw[draw=black,fill=white!50.1960784313725!black,opacity=0.6] (axis cs:-9.28571428571429,0) rectangle (axis cs:-8.71428571428571,0.029471910516796);
\draw[draw=black,fill=white!50.1960784313725!black,opacity=0.6] (axis cs:-8.71428571428572,0) rectangle (axis cs:-8.14285714285715,0.0288481247181654);
\draw[draw=black,fill=white!50.1960784313725!black,opacity=0.6] (axis cs:-8.14285714285715,0) rectangle (axis cs:-7.57142857142857,0.0284964250112233);
\draw[draw=black,fill=white!50.1960784313725!black,opacity=0.6] (axis cs:-7.57142857142857,0) rectangle (axis cs:-7,0.0280983569349582);
\draw[draw=black,fill=white!50.1960784313725!black,opacity=0.6] (axis cs:-7,0) rectangle (axis cs:-6.42857142857143,0.0280624870643499);
\draw[draw=black,fill=white!50.1960784313725!black,opacity=0.6] (axis cs:-6.42857142857143,0) rectangle (axis cs:-5.85714285714286,0.0282505851663211);
\draw[draw=black,fill=white!50.1960784313725!black,opacity=0.6] (axis cs:-5.85714285714286,0) rectangle (axis cs:-5.28571428571429,0.0286460286179077);
\draw[draw=black,fill=white!50.1960784313725!black,opacity=0.6] (axis cs:-5.28571428571429,0) rectangle (axis cs:-4.71428571428571,0.0291088374362465);
\draw[draw=black,fill=white!50.1960784313725!black,opacity=0.6] (axis cs:-4.71428571428572,0) rectangle (axis cs:-4.14285714285715,0.0292969355382181);
\draw[draw=black,fill=white!50.1960784313725!black,opacity=0.6] (axis cs:-4.14285714285715,0) rectangle (axis cs:-3.57142857142857,0.0293494280317914);
\draw[draw=black,fill=white!50.1960784313725!black,opacity=0.6] (axis cs:-3.57142857142857,0) rectangle (axis cs:-3,0.0291088374362465);
\draw[draw=black,fill=white!50.1960784313725!black,opacity=0.6] (axis cs:-3,0) rectangle (axis cs:-2.42857142857143,0.0286818984885162);
\draw[draw=black,fill=white!50.1960784313725!black,opacity=0.6] (axis cs:-2.42857142857143,0) rectangle (axis cs:-1.85714285714286,0.0285042988852595);
\draw[draw=black,fill=white!50.1960784313725!black,opacity=0.6] (axis cs:-1.85714285714286,0) rectangle (axis cs:-1.28571428571428,0.0291377083077117);
\draw[draw=black,fill=white!50.1960784313725!black,opacity=0.6] (axis cs:-1.28571428571428,0) rectangle (axis cs:-0.714285714285715,0.0317352118647035);
\draw[draw=black,fill=white!50.1960784313725!black,opacity=0.6] (axis cs:-0.714285714285715,0) rectangle (axis cs:-0.142857142857146,0.0387333361329331);
\draw[draw=black,fill=white!50.1960784313725!black,opacity=0.6] (axis cs:-0.142857142857146,0) rectangle (axis cs:0.428571428571423,0.0555003134551761);
\draw[draw=black,fill=white!50.1960784313725!black,opacity=0.6] (axis cs:0.428571428571423,0) rectangle (axis cs:1,0.100321029092839);
\draw[draw=black,fill=white!50.1960784313725!black,opacity=0.6] (axis cs:1,0) rectangle (axis cs:1.57142857142857,0.0708149984552212);
\draw[draw=black,fill=white!50.1960784313725!black,opacity=0.6] (axis cs:1.57142857142857,0) rectangle (axis cs:2.14285714285714,0.000251963969152412);
\draw[draw=black,fill=white!50.1960784313725!black,opacity=0.6] (axis cs:2.14285714285714,0) rectangle (axis cs:2.71428571428572,0.000133855858612217);
\draw[draw=black,fill=white!50.1960784313725!black,opacity=0.6] (axis cs:2.71428571428572,0) rectangle (axis cs:3.28571428571428,8.83623641819223e-05);
\draw[draw=black,fill=white!50.1960784313725!black,opacity=0.6] (axis cs:3.28571428571428,0) rectangle (axis cs:3.85714285714285,6.21161173952127e-05);
\draw[draw=black,fill=white!50.1960784313725!black,opacity=0.6] (axis cs:3.85714285714285,0) rectangle (axis cs:4.42857142857142,5.07427437876385e-05);
\draw[draw=black,fill=white!50.1960784313725!black,opacity=0.6] (axis cs:4.42857142857142,0) rectangle (axis cs:5,3.84944952871736e-05);
\draw[draw=black,fill=white,opacity=0.5] (axis cs:-35,0) rectangle (axis cs:-34.4285714285714,0.000785316907566544);
\draw[draw=black,fill=white,opacity=0.5] (axis cs:-34.4285714285714,0) rectangle (axis cs:-33.8571428571429,0.00102347065647769);
\draw[draw=black,fill=white,opacity=0.5] (axis cs:-33.8571428571429,0) rectangle (axis cs:-33.2857142857143,0.00142006553586281);
\draw[draw=black,fill=white,opacity=0.5] (axis cs:-33.2857142857143,0) rectangle (axis cs:-32.7142857142857,0.00180681935124333);
\draw[draw=black,fill=white,opacity=0.5] (axis cs:-32.7142857142857,0) rectangle (axis cs:-32.1428571428571,0.00226738114665829);
\draw[draw=black,fill=white,opacity=0.5] (axis cs:-32.1428571428572,0) rectangle (axis cs:-31.5714285714286,0.00270334028206176);
\draw[draw=black,fill=white,opacity=0.5] (axis cs:-31.5714285714286,0) rectangle (axis cs:-31,0.00319539348229142);
\draw[draw=black,fill=white,opacity=0.5] (axis cs:-31,0) rectangle (axis cs:-30.4285714285714,0.00389115670741618);
\draw[draw=black,fill=white,opacity=0.5] (axis cs:-30.4285714285714,0) rectangle (axis cs:-29.8571428571429,0.00455444442132574);
\draw[draw=black,fill=white,opacity=0.5] (axis cs:-29.8571428571429,0) rectangle (axis cs:-29.2857142857143,0.00530334939207529);
\draw[draw=black,fill=white,opacity=0.5] (axis cs:-29.2857142857143,0) rectangle (axis cs:-28.7142857142857,0.00636027966616864);
\draw[draw=black,fill=white,opacity=0.5] (axis cs:-28.7142857142857,0) rectangle (axis cs:-28.1428571428571,0.00748117685629177);
\draw[draw=black,fill=white,opacity=0.5] (axis cs:-28.1428571428571,0) rectangle (axis cs:-27.5714285714286,0.00835703155270061);
\draw[draw=black,fill=white,opacity=0.5] (axis cs:-27.5714285714286,0) rectangle (axis cs:-27,0.00946316714681684);
\draw[draw=black,fill=white,opacity=0.5] (axis cs:-27,0) rectangle (axis cs:-26.4285714285714,0.0107336485098099);
\draw[draw=black,fill=white,opacity=0.5] (axis cs:-26.4285714285714,0) rectangle (axis cs:-25.8571428571429,0.0118447046359284);
\draw[draw=black,fill=white,opacity=0.5] (axis cs:-25.8571428571429,0) rectangle (axis cs:-25.2857142857143,0.0131427409781342);
\draw[draw=black,fill=white,opacity=0.5] (axis cs:-25.2857142857143,0) rectangle (axis cs:-24.7142857142857,0.0141150381017881);
\draw[draw=black,fill=white,opacity=0.5] (axis cs:-24.7142857142857,0) rectangle (axis cs:-24.1428571428571,0.0149574331805813);
\draw[draw=black,fill=white,opacity=0.5] (axis cs:-24.1428571428571,0) rectangle (axis cs:-23.5714285714286,0.0158903660482167);
\draw[draw=black,fill=white,opacity=0.5] (axis cs:-23.5714285714286,0) rectangle (axis cs:-23,0.0168006644686415);
\draw[draw=black,fill=white,opacity=0.5] (axis cs:-23,0) rectangle (axis cs:-22.4285714285714,0.0186950692895259);
\draw[draw=black,fill=white,opacity=0.5] (axis cs:-22.4285714285714,0) rectangle (axis cs:-21.8571428571429,0.0198799333956788);
\draw[draw=black,fill=white,opacity=0.5] (axis cs:-21.8571428571429,0) rectangle (axis cs:-21.2857142857143,0.0210775908850378);
\draw[draw=black,fill=white,opacity=0.5] (axis cs:-21.2857142857143,0) rectangle (axis cs:-20.7142857142857,0.0226708591473815);
\draw[draw=black,fill=white,opacity=0.5] (axis cs:-20.7142857142857,0) rectangle (axis cs:-20.1428571428571,0.024025973660814);
\draw[draw=black,fill=white,opacity=0.5] (axis cs:-20.1428571428571,0) rectangle (axis cs:-19.5714285714286,0.0253564855142349);
\draw[draw=black,fill=white,opacity=0.5] (axis cs:-19.5714285714286,0) rectangle (axis cs:-19,0.0275293924464491);
\draw[draw=black,fill=white,opacity=0.5] (axis cs:-19,0) rectangle (axis cs:-18.4285714285714,0.0283166775668167);
\draw[draw=black,fill=white,opacity=0.5] (axis cs:-18.4285714285714,0) rectangle (axis cs:-17.8571428571429,0.0299689922131877);
\draw[draw=black,fill=white,opacity=0.5] (axis cs:-17.8571428571429,0) rectangle (axis cs:-17.2857142857143,0.0337066283221324);
\draw[draw=black,fill=white,opacity=0.5] (axis cs:-17.2857142857143,0) rectangle (axis cs:-16.7142857142857,0.0373369968334267);
\draw[draw=black,fill=white,opacity=0.5] (axis cs:-16.7142857142857,0) rectangle (axis cs:-16.1428571428571,0.0400147503490765);
\draw[draw=black,fill=white,opacity=0.5] (axis cs:-16.1428571428571,0) rectangle (axis cs:-15.5714285714286,0.041761539209892);
\draw[draw=black,fill=white,opacity=0.5] (axis cs:-15.5714285714286,0) rectangle (axis cs:-15,0.0448289988601235);
\draw[draw=black,fill=white,opacity=0.5] (axis cs:-15,0) rectangle (axis cs:-14.4285714285714,0.0461614789263457);
\draw[draw=black,fill=white,opacity=0.5] (axis cs:-14.4285714285714,0) rectangle (axis cs:-13.8571428571429,0.0471229508795942);
\draw[draw=black,fill=white,opacity=0.5] (axis cs:-13.8571428571429,0) rectangle (axis cs:-13.2857142857143,0.049095100106115);
\draw[draw=black,fill=white,opacity=0.5] (axis cs:-13.2857142857143,0) rectangle (axis cs:-12.7142857142857,0.0487742814195649);
\draw[draw=black,fill=white,opacity=0.5] (axis cs:-12.7142857142857,0) rectangle (axis cs:-12.1428571428571,0.0490537676372957);
\draw[draw=black,fill=white,opacity=0.5] (axis cs:-12.1428571428571,0) rectangle (axis cs:-11.5714285714286,0.0474860861413637);
\draw[draw=black,fill=white,opacity=0.5] (axis cs:-11.5714285714286,0) rectangle (axis cs:-11,0.0462756352687987);
\draw[draw=black,fill=white,opacity=0.5] (axis cs:-11,0) rectangle (axis cs:-10.4285714285714,0.0434551863250826);
\draw[draw=black,fill=white,opacity=0.5] (axis cs:-10.4285714285714,0) rectangle (axis cs:-9.85714285714286,0.0406996884037962);
\draw[draw=black,fill=white,opacity=0.5] (axis cs:-9.85714285714286,0) rectangle (axis cs:-9.28571428571429,0.0394301911472039);
\draw[draw=black,fill=white,opacity=0.5] (axis cs:-9.28571428571429,0) rectangle (axis cs:-8.71428571428571,0.0381055839321854);
\draw[draw=black,fill=white,opacity=0.5] (axis cs:-8.71428571428572,0) rectangle (axis cs:-8.14285714285715,0.0365999011394829);
\draw[draw=black,fill=white,opacity=0.5] (axis cs:-8.14285714285715,0) rectangle (axis cs:-7.57142857142857,0.0354327509485379);
\draw[draw=black,fill=white,opacity=0.5] (axis cs:-7.57142857142857,0) rectangle (axis cs:-7,0.034608069784953);
\draw[draw=black,fill=white,opacity=0.5] (axis cs:-7,0) rectangle (axis cs:-6.42857142857143,0.0339319886878376);
\draw[draw=black,fill=white,opacity=0.5] (axis cs:-6.42857142857143,0) rectangle (axis cs:-5.85714285714286,0.0333179062939508);
\draw[draw=black,fill=white,opacity=0.5] (axis cs:-5.85714285714286,0) rectangle (axis cs:-5.28571428571429,0.0328002663273094);
\draw[draw=black,fill=white,opacity=0.5] (axis cs:-5.28571428571429,0) rectangle (axis cs:-4.71428571428571,0.0324469721295443);
\draw[draw=black,fill=white,opacity=0.5] (axis cs:-4.71428571428572,0) rectangle (axis cs:-4.14285714285715,0.0321773269758187);
\draw[draw=black,fill=white,opacity=0.5] (axis cs:-4.14285714285715,0) rectangle (axis cs:-3.57142857142857,0.0317561294364219);
\draw[draw=black,fill=white,opacity=0.5] (axis cs:-3.57142857142857,0) rectangle (axis cs:-3,0.0314559769842818);
\draw[draw=black,fill=white,opacity=0.5] (axis cs:-3,0) rectangle (axis cs:-2.42857142857143,0.0316734644987835);
\draw[draw=black,fill=white,opacity=0.5] (axis cs:-2.42857142857143,0) rectangle (axis cs:-1.85714285714286,0.0325709695360024);
\draw[draw=black,fill=white,opacity=0.5] (axis cs:-1.85714285714286,0) rectangle (axis cs:-1.28571428571428,0.0334783156372255);
\draw[draw=black,fill=white,opacity=0.5] (axis cs:-1.28571428571428,0) rectangle (axis cs:-0.714285714285715,0.0325709695360024);
\draw[draw=black,fill=white,opacity=0.5] (axis cs:-0.714285714285715,0) rectangle (axis cs:-0.142857142857146,0.0291797388800195);
\draw[draw=black,fill=white,opacity=0.5] (axis cs:-0.142857142857146,0) rectangle (axis cs:0.428571428571423,0.0291728501352163);
\draw[draw=black,fill=white,opacity=0.5] (axis cs:0.428571428571423,0) rectangle (axis cs:1,0.0579028523902252);
\draw[draw=black,fill=white,opacity=0.5] (axis cs:1,0) rectangle (axis cs:1.57142857142857,0.0995738538112757);
\draw[draw=black,fill=white,opacity=0.5] (axis cs:1.57142857142857,0) rectangle (axis cs:2.14285714285714,0.00111105612611858);
\draw[draw=black,fill=white,opacity=0.5] (axis cs:2.14285714285714,0) rectangle (axis cs:2.71428571428572,0.000395610772984645);
\draw[draw=black,fill=white,opacity=0.5] (axis cs:2.71428571428572,0) rectangle (axis cs:3.28571428571428,0.000316882260947904);
\draw[draw=black,fill=white,opacity=0.5] (axis cs:3.28571428571428,0) rectangle (axis cs:3.85714285714285,0.000259804089721263);
\draw[draw=black,fill=white,opacity=0.5] (axis cs:3.85714285714285,0) rectangle (axis cs:4.42857142857142,0.000185012003286354);
\draw[draw=black,fill=white,opacity=0.5] (axis cs:4.42857142857142,0) rectangle (axis cs:5,0.000154504704872113);
\end{axis}

\end{tikzpicture}}
    \vspace*{1em}%
    \hspace*{0.8em}%
    \subfigure[horizontal speed. $D_{JS}$: 0.0147 (multiple), 0.0650 (single)]{
        \label{fig:hist_speed}
        % This file was created by tikzplotlib v0.9.2.
\begin{tikzpicture}

\begin{axis}[
legend cell align={left},
legend style={fill opacity=0.8, draw opacity=1, text opacity=1, draw=white!80!black},
width=3.4in,
height=2.2in,
tick align=outside,
tick pos=left,
x grid style={white!69.0196078431373!black},
xlabel={horizontal speed (knots)},
xmin=-35, xmax=675,
xtick style={color=black},
y grid style={white!69.0196078431373!black},
ytick={0.0, 0.001, 0.002, 0.003, 0.004, 0.005, 0.006},
ylabel={density},
ymin=0, ymax=0.00666249626531222,
ytick style={color=black},
% yticklabel style={/pgf/number format/.cd,fixed,precision=3},
tick label style={font=\tiny},
label style={font=\scriptsize},
legend entries={{actual},{synthetic--multiple}, {synthetic--single}},
legend style={at={(0.97,0.95)}, anchor=north east, draw=black},
legend style={font=\scriptsize}
]

\addlegendimage{area legend, draw=black, fill=black, opacity=0.5}
\addlegendimage{area legend, draw=black, fill=white!50.1960784313725!black, opacity=0.5}
\addlegendimage{area legend, draw=black, fill=white, opacity=0.5}

\draw[draw=black,fill=black,opacity=0.7] (axis cs:0,0) rectangle (axis cs:10,0);
\draw[draw=black,fill=black,opacity=0.7] (axis cs:10,0) rectangle (axis cs:20,0);
\draw[draw=black,fill=black,opacity=0.7] (axis cs:20,0) rectangle (axis cs:30,0);
\draw[draw=black,fill=black,opacity=0.7] (axis cs:30,0) rectangle (axis cs:40,1.86734389005079e-06);
\draw[draw=black,fill=black,opacity=0.7] (axis cs:40,0) rectangle (axis cs:50,3.73468778010158e-06);
\draw[draw=black,fill=black,opacity=0.7] (axis cs:50,0) rectangle (axis cs:60,3.73468778010158e-06);
\draw[draw=black,fill=black,opacity=0.7] (axis cs:60,0) rectangle (axis cs:70,2.98775022408127e-05);
\draw[draw=black,fill=black,opacity=0.7] (axis cs:70,0) rectangle (axis cs:80,7.65610994920825e-05);
\draw[draw=black,fill=black,opacity=0.7] (axis cs:80,0) rectangle (axis cs:90,0.000153122198984165);
\draw[draw=black,fill=black,opacity=0.7] (axis cs:90,0) rectangle (axis cs:100,0.00042388706304153);
\draw[draw=black,fill=black,opacity=0.7] (axis cs:100,0) rectangle (axis cs:110,0.000956080071706005);
\draw[draw=black,fill=black,opacity=0.7] (axis cs:110,0) rectangle (axis cs:120,0.00178891544666866);
\draw[draw=black,fill=black,opacity=0.7] (axis cs:120,0) rectangle (axis cs:130,0.00301202569465193);
\draw[draw=black,fill=black,opacity=0.7] (axis cs:130,0) rectangle (axis cs:140,0.00416417687481327);
\draw[draw=black,fill=black,opacity=0.7] (axis cs:140,0) rectangle (axis cs:150,0.00493912458918434);
\draw[draw=black,fill=black,opacity=0.7] (axis cs:150,0) rectangle (axis cs:160,0.00549932775619958);
\draw[draw=black,fill=black,opacity=0.7] (axis cs:160,0) rectangle (axis cs:170,0.00571780699133552);
\draw[draw=black,fill=black,opacity=0.7] (axis cs:170,0) rectangle (axis cs:180,0.00571593964744547);
\draw[draw=black,fill=black,opacity=0.7] (axis cs:180,0) rectangle (axis cs:190,0.00623506124887959);
\draw[draw=black,fill=black,opacity=0.7] (axis cs:190,0) rectangle (axis cs:200,0.00634523453839259);
\draw[draw=black,fill=black,opacity=0.7] (axis cs:200,0) rectangle (axis cs:210,0.00572154167911563);
\draw[draw=black,fill=black,opacity=0.7] (axis cs:210,0) rectangle (axis cs:220,0.0054152972811473);
\draw[draw=black,fill=black,opacity=0.7] (axis cs:220,0) rectangle (axis cs:230,0.00501008365700627);
\draw[draw=black,fill=black,opacity=0.7] (axis cs:230,0) rectangle (axis cs:240,0.0049447266208545);
\draw[draw=black,fill=black,opacity=0.7] (axis cs:240,0) rectangle (axis cs:250,0.00488123692859277);
\draw[draw=black,fill=black,opacity=0.7] (axis cs:250,0) rectangle (axis cs:260,0.00483268598745145);
\draw[draw=black,fill=black,opacity=0.7] (axis cs:260,0) rectangle (axis cs:270,0.00452830893337317);
\draw[draw=black,fill=black,opacity=0.7] (axis cs:270,0) rectangle (axis cs:280,0.00434904391992829);
\draw[draw=black,fill=black,opacity=0.7] (axis cs:280,0) rectangle (axis cs:290,0.00345832088437407);
\draw[draw=black,fill=black,opacity=0.7] (axis cs:290,0) rectangle (axis cs:300,0.00280848521063639);
\draw[draw=black,fill=black,opacity=0.7] (axis cs:300,0) rectangle (axis cs:310,0.00220533313414998);
\draw[draw=black,fill=black,opacity=0.7] (axis cs:310,0) rectangle (axis cs:320,0.00166380340603526);
\draw[draw=black,fill=black,opacity=0.7] (axis cs:320,0) rectangle (axis cs:330,0.00124925306244398);
\draw[draw=black,fill=black,opacity=0.7] (axis cs:330,0) rectangle (axis cs:340,0.00096728413504631);
\draw[draw=black,fill=black,opacity=0.7] (axis cs:340,0) rectangle (axis cs:350,0.000733866148789961);
\draw[draw=black,fill=black,opacity=0.7] (axis cs:350,0) rectangle (axis cs:360,0.000535927696444577);
\draw[draw=black,fill=black,opacity=0.7] (axis cs:360,0) rectangle (axis cs:370,0.000427621750821631);
\draw[draw=black,fill=black,opacity=0.7] (axis cs:370,0) rectangle (axis cs:380,0.000298775022408127);
\draw[draw=black,fill=black,opacity=0.7] (axis cs:380,0) rectangle (axis cs:390,0.000222213922916044);
\draw[draw=black,fill=black,opacity=0.7] (axis cs:390,0) rectangle (axis cs:400,0.000162458918434419);
\draw[draw=black,fill=black,opacity=0.7] (axis cs:400,0) rectangle (axis cs:410,0.000143785479533911);
\draw[draw=black,fill=black,opacity=0.7] (axis cs:410,0) rectangle (axis cs:420,0.000100836570062743);
\draw[draw=black,fill=black,opacity=0.7] (axis cs:420,0) rectangle (axis cs:430,7.09590678219301e-05);
\draw[draw=black,fill=black,opacity=0.7] (axis cs:430,0) rectangle (axis cs:440,7.46937556020317e-05);
\draw[draw=black,fill=black,opacity=0.7] (axis cs:440,0) rectangle (axis cs:450,3.73468778010158e-05);
\draw[draw=black,fill=black,opacity=0.7] (axis cs:450,0) rectangle (axis cs:460,2.61428144607111e-05);
\draw[draw=black,fill=black,opacity=0.7] (axis cs:460,0) rectangle (axis cs:470,1.68060950104571e-05);
\draw[draw=black,fill=black,opacity=0.7] (axis cs:470,0) rectangle (axis cs:480,1.86734389005079e-05);
\draw[draw=black,fill=black,opacity=0.7] (axis cs:480,0) rectangle (axis cs:490,1.12040633403048e-05);
\draw[draw=black,fill=black,opacity=0.7] (axis cs:490,0) rectangle (axis cs:500,5.60203167015238e-06);
\draw[draw=black,fill=black,opacity=0.7] (axis cs:500,0) rectangle (axis cs:510,3.73468778010158e-06);
\draw[draw=black,fill=black,opacity=0.7] (axis cs:510,0) rectangle (axis cs:520,1.86734389005079e-06);
\draw[draw=black,fill=black,opacity=0.7] (axis cs:520,0) rectangle (axis cs:530,1.86734389005079e-06);
\draw[draw=black,fill=black,opacity=0.7] (axis cs:530,0) rectangle (axis cs:540,1.86734389005079e-06);
\draw[draw=black,fill=black,opacity=0.7] (axis cs:540,0) rectangle (axis cs:550,0);
\draw[draw=black,fill=black,opacity=0.7] (axis cs:550,0) rectangle (axis cs:560,1.86734389005079e-06);
\draw[draw=black,fill=black,opacity=0.7] (axis cs:560,0) rectangle (axis cs:570,0);
\draw[draw=black,fill=black,opacity=0.7] (axis cs:570,0) rectangle (axis cs:580,0);
\draw[draw=black,fill=black,opacity=0.7] (axis cs:580,0) rectangle (axis cs:590,0);
\draw[draw=black,fill=black,opacity=0.7] (axis cs:590,0) rectangle (axis cs:600,0);
\draw[draw=black,fill=black,opacity=0.7] (axis cs:600,0) rectangle (axis cs:610,0);
\draw[draw=black,fill=black,opacity=0.7] (axis cs:610,0) rectangle (axis cs:620,0);
\draw[draw=black,fill=black,opacity=0.7] (axis cs:620,0) rectangle (axis cs:630,0);
\draw[draw=black,fill=black,opacity=0.7] (axis cs:630,0) rectangle (axis cs:640,0);
\draw[draw=black,fill=black,opacity=0.7] (axis cs:640,0) rectangle (axis cs:650,0);
\draw[draw=black,fill=black,opacity=0.7] (axis cs:650,0) rectangle (axis cs:660,0);
\draw[draw=black,fill=black,opacity=0.7] (axis cs:660,0) rectangle (axis cs:670,0);
\draw[draw=black,fill=black,opacity=0.7] (axis cs:670,0) rectangle (axis cs:680,0);
\draw[draw=black,fill=black,opacity=0.7] (axis cs:680,0) rectangle (axis cs:690,0);
\draw[draw=black,fill=black,opacity=0.7] (axis cs:690,0) rectangle (axis cs:700,0);
\draw[draw=black,fill=white!50.1960784313725!black,opacity=0.6] (axis cs:0,0) rectangle (axis cs:10,6.23251481972976e-05);
\draw[draw=black,fill=white!50.1960784313725!black,opacity=0.6] (axis cs:10,0) rectangle (axis cs:20,8.36837663997423e-05);
\draw[draw=black,fill=white!50.1960784313725!black,opacity=0.6] (axis cs:20,0) rectangle (axis cs:30,0.000123849973464293);
\draw[draw=black,fill=white!50.1960784313725!black,opacity=0.6] (axis cs:30,0) rectangle (axis cs:40,0.000149510327417113);
\draw[draw=black,fill=white!50.1960784313725!black,opacity=0.6] (axis cs:40,0) rectangle (axis cs:50,0.000220488967298305);
\draw[draw=black,fill=white!50.1960784313725!black,opacity=0.6] (axis cs:50,0) rectangle (axis cs:60,0.000300671320877974);
\draw[draw=black,fill=white!50.1960784313725!black,opacity=0.6] (axis cs:60,0) rectangle (axis cs:70,0.000402162272476944);
\draw[draw=black,fill=white!50.1960784313725!black,opacity=0.6] (axis cs:70,0) rectangle (axis cs:80,0.000480693960012865);
\draw[draw=black,fill=white!50.1960784313725!black,opacity=0.6] (axis cs:80,0) rectangle (axis cs:90,0.000672271261453997);
\draw[draw=black,fill=white!50.1960784313725!black,opacity=0.6] (axis cs:90,0) rectangle (axis cs:100,0.000973592844712842);
\draw[draw=black,fill=white!50.1960784313725!black,opacity=0.6] (axis cs:100,0) rectangle (axis cs:110,0.00134119117063735);
\draw[draw=black,fill=white!50.1960784313725!black,opacity=0.6] (axis cs:110,0) rectangle (axis cs:120,0.00171564226165258);
\draw[draw=black,fill=white!50.1960784313725!black,opacity=0.6] (axis cs:120,0) rectangle (axis cs:130,0.00224935761579797);
\draw[draw=black,fill=white!50.1960784313725!black,opacity=0.6] (axis cs:130,0) rectangle (axis cs:140,0.00289631866458116);
\draw[draw=black,fill=white!50.1960784313725!black,opacity=0.6] (axis cs:140,0) rectangle (axis cs:150,0.00359620106713059);
\draw[draw=black,fill=white!50.1960784313725!black,opacity=0.6] (axis cs:150,0) rectangle (axis cs:160,0.00436786243249151);
\draw[draw=black,fill=white!50.1960784313725!black,opacity=0.6] (axis cs:160,0) rectangle (axis cs:170,0.00501942533812393);
\draw[draw=black,fill=white!50.1960784313725!black,opacity=0.6] (axis cs:170,0) rectangle (axis cs:180,0.00548851461564741);
\draw[draw=black,fill=white!50.1960784313725!black,opacity=0.6] (axis cs:180,0) rectangle (axis cs:190,0.00583365387934031);
\draw[draw=black,fill=white!50.1960784313725!black,opacity=0.6] (axis cs:190,0) rectangle (axis cs:200,0.00602488103949944);
\draw[draw=black,fill=white!50.1960784313725!black,opacity=0.6] (axis cs:200,0) rectangle (axis cs:210,0.00597821220862618);
\draw[draw=black,fill=white!50.1960784313725!black,opacity=0.6] (axis cs:210,0) rectangle (axis cs:220,0.00575912380645591);
\draw[draw=black,fill=white!50.1960784313725!black,opacity=0.6] (axis cs:220,0) rectangle (axis cs:230,0.00547896076066693);
\draw[draw=black,fill=white!50.1960784313725!black,opacity=0.6] (axis cs:230,0) rectangle (axis cs:240,0.00511481382737935);
\draw[draw=black,fill=white!50.1960784313725!black,opacity=0.6] (axis cs:240,0) rectangle (axis cs:250,0.00463096859582842);
\draw[draw=black,fill=white!50.1960784313725!black,opacity=0.6] (axis cs:250,0) rectangle (axis cs:260,0.00423390838203215);
\draw[draw=black,fill=white!50.1960784313725!black,opacity=0.6] (axis cs:260,0) rectangle (axis cs:270,0.00385620597911257);
\draw[draw=black,fill=white!50.1960784313725!black,opacity=0.6] (axis cs:270,0) rectangle (axis cs:280,0.00340027200975594);
\draw[draw=black,fill=white!50.1960784313725!black,opacity=0.6] (axis cs:280,0) rectangle (axis cs:290,0.00293583460926484);
\draw[draw=black,fill=white!50.1960784313725!black,opacity=0.6] (axis cs:290,0) rectangle (axis cs:300,0.00258774415476645);
\draw[draw=black,fill=white!50.1960784313725!black,opacity=0.6] (axis cs:300,0) rectangle (axis cs:310,0.00217637816859103);
\draw[draw=black,fill=white!50.1960784313725!black,opacity=0.6] (axis cs:310,0) rectangle (axis cs:320,0.00187030466793351);
\draw[draw=black,fill=white!50.1960784313725!black,opacity=0.6] (axis cs:320,0) rectangle (axis cs:330,0.00161240060364357);
\draw[draw=black,fill=white!50.1960784313725!black,opacity=0.6] (axis cs:330,0) rectangle (axis cs:340,0.00142472487648766);
\draw[draw=black,fill=white!50.1960784313725!black,opacity=0.6] (axis cs:340,0) rectangle (axis cs:350,0.00118102654421059);
\draw[draw=black,fill=white!50.1960784313725!black,opacity=0.6] (axis cs:350,0) rectangle (axis cs:360,0.000977344358448634);
\draw[draw=black,fill=white!50.1960784313725!black,opacity=0.6] (axis cs:360,0) rectangle (axis cs:370,0.000817079691655583);
\draw[draw=black,fill=white!50.1960784313725!black,opacity=0.6] (axis cs:370,0) rectangle (axis cs:380,0.000674672230244904);
\draw[draw=black,fill=white!50.1960784313725!black,opacity=0.6] (axis cs:380,0) rectangle (axis cs:390,0.000577783135495172);
\draw[draw=black,fill=white!50.1960784313725!black,opacity=0.6] (axis cs:390,0) rectangle (axis cs:400,0.00047434139675359);
\draw[draw=black,fill=white!50.1960784313725!black,opacity=0.6] (axis cs:400,0) rectangle (axis cs:410,0.0003767019992567);
\draw[draw=black,fill=white!50.1960784313725!black,opacity=0.6] (axis cs:410,0) rectangle (axis cs:420,0.000303072289668881);
\draw[draw=black,fill=white!50.1960784313725!black,opacity=0.6] (axis cs:420,0) rectangle (axis cs:430,0.000253152146891271);
\draw[draw=black,fill=white!50.1960784313725!black,opacity=0.6] (axis cs:430,0) rectangle (axis cs:440,0.000219538583818571);
\draw[draw=black,fill=white!50.1960784313725!black,opacity=0.6] (axis cs:440,0) rectangle (axis cs:450,0.000192277584005146);
\draw[draw=black,fill=white!50.1960784313725!black,opacity=0.6] (axis cs:450,0) rectangle (axis cs:460,0.000160564787891914);
\draw[draw=black,fill=white!50.1960784313725!black,opacity=0.6] (axis cs:460,0) rectangle (axis cs:470,0.000128902011961827);
\draw[draw=black,fill=white!50.1960784313725!black,opacity=0.6] (axis cs:470,0) rectangle (axis cs:480,0.000110494584564872);
\draw[draw=black,fill=white!50.1960784313725!black,opacity=0.6] (axis cs:480,0) rectangle (axis cs:490,8.01823535796694e-05);
\draw[draw=black,fill=white!50.1960784313725!black,opacity=0.6] (axis cs:490,0) rectangle (axis cs:500,6.94280142037312e-05);
\draw[draw=black,fill=white!50.1960784313725!black,opacity=0.6] (axis cs:500,0) rectangle (axis cs:510,5.71730693334761e-05);
\draw[draw=black,fill=white!50.1960784313725!black,opacity=0.6] (axis cs:510,0) rectangle (axis cs:520,5.46720601762811e-05);
\draw[draw=black,fill=white!50.1960784313725!black,opacity=0.6] (axis cs:520,0) rectangle (axis cs:530,4.39677409834868e-05);
\draw[draw=black,fill=white!50.1960784313725!black,opacity=0.6] (axis cs:530,0) rectangle (axis cs:540,3.93158639511043e-05);
\draw[draw=black,fill=white!50.1960784313725!black,opacity=0.6] (axis cs:540,0) rectangle (axis cs:550,3.04122713514903e-05);
\draw[draw=black,fill=white!50.1960784313725!black,opacity=0.6] (axis cs:550,0) rectangle (axis cs:560,2.31093246124811e-05);
\draw[draw=black,fill=white!50.1960784313725!black,opacity=0.6] (axis cs:560,0) rectangle (axis cs:570,2.10084769204374e-05);
\draw[draw=black,fill=white!50.1960784313725!black,opacity=0.6] (axis cs:570,0) rectangle (axis cs:580,1.42057320128672e-05);
\draw[draw=black,fill=white!50.1960784313725!black,opacity=0.6] (axis cs:580,0) rectangle (axis cs:590,1.11044806579455e-05);
\draw[draw=black,fill=white!50.1960784313725!black,opacity=0.6] (axis cs:590,0) rectangle (axis cs:600,9.15369351533344e-06);
\draw[draw=black,fill=white!50.1960784313725!black,opacity=0.6] (axis cs:600,0) rectangle (axis cs:610,7.70310820416038e-06);
\draw[draw=black,fill=white!50.1960784313725!black,opacity=0.6] (axis cs:610,0) rectangle (axis cs:620,5.25211923010935e-06);
\draw[draw=black,fill=white!50.1960784313725!black,opacity=0.6] (axis cs:620,0) rectangle (axis cs:630,4.80193758181426e-06);
\draw[draw=black,fill=white!50.1960784313725!black,opacity=0.6] (axis cs:630,0) rectangle (axis cs:640,3.95159446836799e-06);
\draw[draw=black,fill=white!50.1960784313725!black,opacity=0.6] (axis cs:640,0) rectangle (axis cs:650,3.25131190435341e-06);
\draw[draw=black,fill=white!50.1960784313725!black,opacity=0.6] (axis cs:650,0) rectangle (axis cs:660,3.4013724537851e-06);
\draw[draw=black,fill=white!50.1960784313725!black,opacity=0.6] (axis cs:660,0) rectangle (axis cs:670,2.90117062234612e-06);
\draw[draw=black,fill=white!50.1960784313725!black,opacity=0.6] (axis cs:670,0) rectangle (axis cs:680,2.45098897405103e-06);
\draw[draw=black,fill=white!50.1960784313725!black,opacity=0.6] (axis cs:680,0) rectangle (axis cs:690,1.70068622689255e-06);
\draw[draw=black,fill=white!50.1960784313725!black,opacity=0.6] (axis cs:690,0) rectangle (axis cs:700,1.75070641003645e-06);
\draw[draw=black,fill=white,opacity=0.5] (axis cs:0,0) rectangle (axis cs:10,7.18621237649225e-05);
\draw[draw=black,fill=white,opacity=0.5] (axis cs:10,0) rectangle (axis cs:20,0.000115530884956449);
\draw[draw=black,fill=white,opacity=0.5] (axis cs:20,0) rectangle (axis cs:30,0.000189531427439514);
\draw[draw=black,fill=white,opacity=0.5] (axis cs:30,0) rectangle (axis cs:40,0.000254696924166045);
\draw[draw=black,fill=white,opacity=0.5] (axis cs:40,0) rectangle (axis cs:50,0.000339051915179059);
\draw[draw=black,fill=white,opacity=0.5] (axis cs:50,0) rectangle (axis cs:60,0.000427008453506838);
\draw[draw=black,fill=white,opacity=0.5] (axis cs:60,0) rectangle (axis cs:70,0.000532184889933338);
\draw[draw=black,fill=white,opacity=0.5] (axis cs:70,0) rectangle (axis cs:80,0.00063916210001722);
\draw[draw=black,fill=white,opacity=0.5] (axis cs:80,0) rectangle (axis cs:90,0.000793184521900221);
\draw[draw=black,fill=white,opacity=0.5] (axis cs:90,0) rectangle (axis cs:100,0.000972530323340165);
\draw[draw=black,fill=white,opacity=0.5] (axis cs:100,0) rectangle (axis cs:110,0.00121540967037964);
\draw[draw=black,fill=white,opacity=0.5] (axis cs:110,0) rectangle (axis cs:120,0.00144061892590604);
\draw[draw=black,fill=white,opacity=0.5] (axis cs:120,0) rectangle (axis cs:130,0.00170150600951934);
\draw[draw=black,fill=white,opacity=0.5] (axis cs:130,0) rectangle (axis cs:140,0.00204624161655451);
\draw[draw=black,fill=white,opacity=0.5] (axis cs:140,0) rectangle (axis cs:150,0.00233577225615555);
\draw[draw=black,fill=white,opacity=0.5] (axis cs:150,0) rectangle (axis cs:160,0.00263380029645236);
\draw[draw=black,fill=white,opacity=0.5] (axis cs:160,0) rectangle (axis cs:170,0.00304690902829128);
\draw[draw=black,fill=white,opacity=0.5] (axis cs:170,0) rectangle (axis cs:180,0.00332676050948389);
\draw[draw=black,fill=white,opacity=0.5] (axis cs:180,0) rectangle (axis cs:190,0.00360762492585877);
\draw[draw=black,fill=white,opacity=0.5] (axis cs:190,0) rectangle (axis cs:200,0.00380210848085609);
\draw[draw=black,fill=white,opacity=0.5] (axis cs:200,0) rectangle (axis cs:210,0.00393052615229818);
\draw[draw=black,fill=white,opacity=0.5] (axis cs:210,0) rectangle (axis cs:220,0.00404138628058079);
\draw[draw=black,fill=white,opacity=0.5] (axis cs:220,0) rectangle (axis cs:230,0.00414442429828915);
\draw[draw=black,fill=white,opacity=0.5] (axis cs:230,0) rectangle (axis cs:240,0.00419338283209924);
\draw[draw=black,fill=white,opacity=0.5] (axis cs:240,0) rectangle (axis cs:250,0.00430176689660295);
\draw[draw=black,fill=white,opacity=0.5] (axis cs:250,0) rectangle (axis cs:260,0.00436226163665565);
\draw[draw=black,fill=white,opacity=0.5] (axis cs:260,0) rectangle (axis cs:270,0.00420953352083889);
\draw[draw=black,fill=white,opacity=0.5] (axis cs:270,0) rectangle (axis cs:280,0.00399614184243906);
\draw[draw=black,fill=white,opacity=0.5] (axis cs:280,0) rectangle (axis cs:290,0.00382152307184974);
\draw[draw=black,fill=white,opacity=0.5] (axis cs:290,0) rectangle (axis cs:300,0.00354392255772886);
\draw[draw=black,fill=white,opacity=0.5] (axis cs:300,0) rectangle (axis cs:310,0.00329732911502104);
\draw[draw=black,fill=white,opacity=0.5] (axis cs:310,0) rectangle (axis cs:320,0.00300475966987317);
\draw[draw=black,fill=white,opacity=0.5] (axis cs:320,0) rectangle (axis cs:330,0.00273166108989575);
\draw[draw=black,fill=white,opacity=0.5] (axis cs:330,0) rectangle (axis cs:340,0.00251236062293263);
\draw[draw=black,fill=white,opacity=0.5] (axis cs:340,0) rectangle (axis cs:350,0.00226818696982691);
\draw[draw=black,fill=white,opacity=0.5] (axis cs:350,0) rectangle (axis cs:360,0.0020455100522562);
\draw[draw=black,fill=white,opacity=0.5] (axis cs:360,0) rectangle (axis cs:370,0.00177021677938384);
\draw[draw=black,fill=white,opacity=0.5] (axis cs:370,0) rectangle (axis cs:380,0.00158479336685023);
\draw[draw=black,fill=white,opacity=0.5] (axis cs:380,0) rectangle (axis cs:390,0.00144754064965161);
\draw[draw=black,fill=white,opacity=0.5] (axis cs:390,0) rectangle (axis cs:400,0.0012596411733391);
\draw[draw=black,fill=white,opacity=0.5] (axis cs:400,0) rectangle (axis cs:410,0.00109982251124639);
\draw[draw=black,fill=white,opacity=0.5] (axis cs:410,0) rectangle (axis cs:420,0.000958349230788277);
\draw[draw=black,fill=white,opacity=0.5] (axis cs:420,0) rectangle (axis cs:430,0.000842987168362208);
\draw[draw=black,fill=white,opacity=0.5] (axis cs:430,0) rectangle (axis cs:440,0.000718452415118846);
\draw[draw=black,fill=white,opacity=0.5] (axis cs:440,0) rectangle (axis cs:450,0.000624812184934953);
\draw[draw=black,fill=white,opacity=0.5] (axis cs:450,0) rectangle (axis cs:460,0.000550080078153577);
\draw[draw=black,fill=white,opacity=0.5] (axis cs:460,0) rectangle (axis cs:470,0.000472984455946886);
\draw[draw=black,fill=white,opacity=0.5] (axis cs:470,0) rectangle (axis cs:480,0.000394875898557918);
\draw[draw=black,fill=white,opacity=0.5] (axis cs:480,0) rectangle (axis cs:490,0.000343947768560068);
\draw[draw=black,fill=white,opacity=0.5] (axis cs:490,0) rectangle (axis cs:500,0.000308157392119589);
\draw[draw=black,fill=white,opacity=0.5] (axis cs:500,0) rectangle (axis cs:510,0.000257848278066464);
\draw[draw=black,fill=white,opacity=0.5] (axis cs:510,0) rectangle (axis cs:520,0.000220819869736536);
\draw[draw=black,fill=white,opacity=0.5] (axis cs:520,0) rectangle (axis cs:530,0.000182891074577916);
\draw[draw=black,fill=white,opacity=0.5] (axis cs:530,0) rectangle (axis cs:540,0.000161619435750084);
\draw[draw=black,fill=white,opacity=0.5] (axis cs:540,0) rectangle (axis cs:550,0.000145581295364021);
\draw[draw=black,fill=white,opacity=0.5] (axis cs:550,0) rectangle (axis cs:560,0.000120370464160665);
\draw[draw=black,fill=white,opacity=0.5] (axis cs:560,0) rectangle (axis cs:570,0.000101124695697389);
\draw[draw=black,fill=white,opacity=0.5] (axis cs:570,0) rectangle (axis cs:580,9.13892631121646e-05);
\draw[draw=black,fill=white,opacity=0.5] (axis cs:580,0) rectangle (axis cs:590,8.08659920518353e-05);
\draw[draw=black,fill=white,opacity=0.5] (axis cs:590,0) rectangle (axis cs:600,6.98925275771603e-05);
\draw[draw=black,fill=white,opacity=0.5] (axis cs:600,0) rectangle (axis cs:610,5.64429993235844e-05);
\draw[draw=black,fill=white,opacity=0.5] (axis cs:610,0) rectangle (axis cs:620,4.19805358877308e-05);
\draw[draw=black,fill=white,opacity=0.5] (axis cs:620,0) rectangle (axis cs:630,3.72535050371016e-05);
\draw[draw=black,fill=white,opacity=0.5] (axis cs:630,0) rectangle (axis cs:640,3.33143126615772e-05);
\draw[draw=black,fill=white,opacity=0.5] (axis cs:640,0) rectangle (axis cs:650,3.0781974705883e-05);
\draw[draw=black,fill=white,opacity=0.5] (axis cs:650,0) rectangle (axis cs:660,2.82496367501888e-05);
\draw[draw=black,fill=white,opacity=0.5] (axis cs:660,0) rectangle (axis cs:670,2.13841871814178e-05);
\draw[draw=black,fill=white,opacity=0.5] (axis cs:670,0) rectangle (axis cs:680,1.82328332809983e-05);
\draw[draw=black,fill=white,opacity=0.5] (axis cs:680,0) rectangle (axis cs:690,1.457501178944e-05);
\draw[draw=black,fill=white,opacity=0.5] (axis cs:690,0) rectangle (axis cs:700,1.25491414248847e-05);
\end{axis}

\end{tikzpicture}
}
    \hspace*{1em}%
    \subfigure[distance to the closest aircraft. $D_{JS}$: 0.0307 (multiple), 0.0261 (single)]{
        \label{fig:hist_encounter}
        % This file was created by tikzplotlib v0.9.2.
\begin{tikzpicture}

\begin{axis}[
legend cell align={left},
legend style={fill opacity=0.8, draw opacity=1, text opacity=1, draw=white!80!black},
width=3.4in,
height=2.2in,
tick align=outside,
tick pos=left,
x grid style={white!69.0196078431373!black},
xlabel={distance to closest aircraft (NM)},
xmin=-2.25, xmax=35.25,
xtick style={color=black},
y grid style={white!69.0196078431373!black},
ylabel={density},
ytick={0, 0.02, 0.04, 0.06, 0.08, 0.10},
yticklabel style={/pgf/number format/.cd,fixed,precision=2},
ymin=0, ymax=0.102280215023725,
ytick style={color=black},
tick label style={font=\tiny},
label style={font=\scriptsize},
legend cell align={left},
legend entries={{actual},{synthetic--multiple}, {synthetic--single}},
legend style={at={(0.97,0.95)}, anchor=north east, draw=black},
legend style={font=\scriptsize}
]

\addlegendimage{area legend, draw=black, fill=black, opacity=0.6}
\addlegendimage{area legend, draw=black, fill=white!50.1960784313725!black, opacity=0.7}
\addlegendimage{area legend, draw=black, fill=white, opacity=0.5}

\draw[draw=black,fill=black,opacity=0.7] (axis cs:0,0) rectangle (axis cs:0.5,1.65058265567745e-05);
\draw[draw=black,fill=black,opacity=0.7] (axis cs:0.5,0) rectangle (axis cs:1,9.90349593406472e-05);
\draw[draw=black,fill=black,opacity=0.7] (axis cs:1,0) rectangle (axis cs:1.5,0.000148552439010971);
\draw[draw=black,fill=black,opacity=0.7] (axis cs:1.5,0) rectangle (axis cs:2,0.00054469227637356);
\draw[draw=black,fill=black,opacity=0.7] (axis cs:2,0) rectangle (axis cs:2.5,0.00382935176117169);
\draw[draw=black,fill=black,opacity=0.7] (axis cs:2.5,0) rectangle (axis cs:3,0.0124949107034783);
\draw[draw=black,fill=black,opacity=0.7] (axis cs:3,0) rectangle (axis cs:3.5,0.0336828900601912);
\draw[draw=black,fill=black,opacity=0.7] (axis cs:3.5,0) rectangle (axis cs:4,0.0567305258756341);
\draw[draw=black,fill=black,opacity=0.7] (axis cs:4,0) rectangle (axis cs:4.5,0.0844438086644586);
\draw[draw=black,fill=black,opacity=0.7] (axis cs:4.5,0) rectangle (axis cs:5,0.0817368531091475);
\draw[draw=black,fill=black,opacity=0.7] (axis cs:5,0) rectangle (axis cs:5.5,0.0786832751961442);
\draw[draw=black,fill=black,opacity=0.7] (axis cs:5.5,0) rectangle (axis cs:6,0.0733628971026773);
\draw[draw=black,fill=black,opacity=0.7] (axis cs:6,0) rectangle (axis cs:6.5,0.0720699406890632);
\draw[draw=black,fill=black,opacity=0.7] (axis cs:6.5,0) rectangle (axis cs:7,0.0641361400574403);
\draw[draw=black,fill=black,opacity=0.7] (axis cs:7,0) rectangle (axis cs:7.5,0.0635254244748396);
\draw[draw=black,fill=black,opacity=0.7] (axis cs:7.5,0) rectangle (axis cs:8,0.0632338215390033);
\draw[draw=black,fill=black,opacity=0.7] (axis cs:8,0) rectangle (axis cs:8.5,0.0666120140409565);
\draw[draw=black,fill=black,opacity=0.7] (axis cs:8.5,0) rectangle (axis cs:9,0.064328708033936);
\draw[draw=black,fill=black,opacity=0.7] (axis cs:9,0) rectangle (axis cs:9.5,0.0659682868052422);
\draw[draw=black,fill=black,opacity=0.7] (axis cs:9.5,0) rectangle (axis cs:10,0.0635584361279532);
\draw[draw=black,fill=black,opacity=0.7] (axis cs:10,0) rectangle (axis cs:10.5,0.0620454020269155);
\draw[draw=black,fill=black,opacity=0.7] (axis cs:10.5,0) rectangle (axis cs:11,0.0575833269144008);
\draw[draw=black,fill=black,opacity=0.7] (axis cs:11,0) rectangle (axis cs:11.5,0.0588322677905301);
\draw[draw=black,fill=black,opacity=0.7] (axis cs:11.5,0) rectangle (axis cs:12,0.057176183192667);
\draw[draw=black,fill=black,opacity=0.7] (axis cs:12,0) rectangle (axis cs:12.5,0.0549148849543889);
\draw[draw=black,fill=black,opacity=0.7] (axis cs:12.5,0) rectangle (axis cs:13,0.0586617075827767);
\draw[draw=black,fill=black,opacity=0.7] (axis cs:13,0) rectangle (axis cs:13.5,0.0582655677454141);
\draw[draw=black,fill=black,opacity=0.7] (axis cs:13.5,0) rectangle (axis cs:14,0.0569781132739857);
\draw[draw=black,fill=black,opacity=0.7] (axis cs:14,0) rectangle (axis cs:14.5,0.05339084696898);
\draw[draw=black,fill=black,opacity=0.7] (axis cs:14.5,0) rectangle (axis cs:15,0.0540620839156222);
\draw[draw=black,fill=black,opacity=0.7] (axis cs:15,0) rectangle (axis cs:15.5,0.0457761589841214);
\draw[draw=black,fill=black,opacity=0.7] (axis cs:15.5,0) rectangle (axis cs:16,0.0434708452083586);
\draw[draw=black,fill=black,opacity=0.7] (axis cs:16,0) rectangle (axis cs:16.5,0.0416166906918142);
\draw[draw=black,fill=black,opacity=0.7] (axis cs:16.5,0) rectangle (axis cs:17,0.0385796186053677);
\draw[draw=black,fill=black,opacity=0.7] (axis cs:17,0) rectangle (axis cs:17.5,0.039047283691143);
\draw[draw=black,fill=black,opacity=0.7] (axis cs:17.5,0) rectangle (axis cs:18,0.0345852085786283);
\draw[draw=black,fill=black,opacity=0.7] (axis cs:18,0) rectangle (axis cs:18.5,0.0288191731681283);
\draw[draw=black,fill=black,opacity=0.7] (axis cs:18.5,0) rectangle (axis cs:19,0.0258481243879089);
\draw[draw=black,fill=black,opacity=0.7] (axis cs:19,0) rectangle (axis cs:19.5,0.022183830892305);
\draw[draw=black,fill=black,opacity=0.7] (axis cs:19.5,0) rectangle (axis cs:20,0.0194713733948084);
\draw[draw=black,fill=black,opacity=0.7] (axis cs:20,0) rectangle (axis cs:20.5,0.0158235857257612);
\draw[draw=black,fill=black,opacity=0.7] (axis cs:20.5,0) rectangle (axis cs:21,0.0161867139100102);
\draw[draw=black,fill=black,opacity=0.7] (axis cs:21,0) rectangle (axis cs:21.5,0.0133972292219153);
\draw[draw=black,fill=black,opacity=0.7] (axis cs:21.5,0) rectangle (axis cs:22,0.0108498299899865);
\draw[draw=black,fill=black,opacity=0.7] (axis cs:22,0) rectangle (axis cs:22.5,0.0100190367199621);
\draw[draw=black,fill=black,opacity=0.7] (axis cs:22.5,0) rectangle (axis cs:23,0.00895165993595739);
\draw[draw=black,fill=black,opacity=0.7] (axis cs:23,0) rectangle (axis cs:23.5,0.0095293638654445);
\draw[draw=black,fill=black,opacity=0.7] (axis cs:23.5,0) rectangle (axis cs:24,0.00819239191434576);
\draw[draw=black,fill=black,opacity=0.7] (axis cs:24,0) rectangle (axis cs:24.5,0.00676738888827756);
\draw[draw=black,fill=black,opacity=0.7] (axis cs:24.5,0) rectangle (axis cs:25,0.00561198102930334);
\draw[draw=black,fill=black,opacity=0.7] (axis cs:25,0) rectangle (axis cs:25.5,0.00464914114682483);
\draw[draw=black,fill=black,opacity=0.7] (axis cs:25.5,0) rectangle (axis cs:26,0.00452259647655622);
\draw[draw=black,fill=black,opacity=0.7] (axis cs:26,0) rectangle (axis cs:26.5,0.00418147606104955);
\draw[draw=black,fill=black,opacity=0.7] (axis cs:26.5,0) rectangle (axis cs:27,0.00363678378467599);
\draw[draw=black,fill=black,opacity=0.7] (axis cs:27,0) rectangle (axis cs:27.5,0.00264643419126952);
\draw[draw=black,fill=black,opacity=0.7] (axis cs:27.5,0) rectangle (axis cs:28,0.00232732154450521);
\draw[draw=black,fill=black,opacity=0.7] (axis cs:28,0) rectangle (axis cs:28.5,0.00223929046953575);
\draw[draw=black,fill=black,opacity=0.7] (axis cs:28.5,0) rectangle (axis cs:29,0.00251438757881532);
\draw[draw=black,fill=black,opacity=0.7] (axis cs:29,0) rectangle (axis cs:29.5,0.00151303410103767);
\draw[draw=black,fill=black,opacity=0.7] (axis cs:29.5,0) rectangle (axis cs:30,0.00124894087612927);
\draw[draw=black,fill=black,opacity=0.7] (axis cs:30,0) rectangle (axis cs:30.5,0.00113890203241744);
\draw[draw=black,fill=black,opacity=0.7] (axis cs:30.5,0) rectangle (axis cs:31,0.00083629521220991);
\draw[draw=black,fill=black,opacity=0.7] (axis cs:31,0) rectangle (axis cs:31.5,0.000792279674725178);
\draw[draw=black,fill=black,opacity=0.7] (axis cs:31.5,0) rectangle (axis cs:32,0.000698746657570122);
\draw[draw=black,fill=black,opacity=0.7] (axis cs:32,0) rectangle (axis cs:32.5,0.000269595167093984);
\draw[draw=black,fill=black,opacity=0.7] (axis cs:32.5,0) rectangle (axis cs:33,0.000242085456166027);
\draw[draw=black,fill=black,opacity=0.7] (axis cs:33,0) rectangle (axis cs:33.5,0.000214575745238069);
\draw[draw=black,fill=black,opacity=0.7] (axis cs:33.5,0) rectangle (axis cs:34,0.000198069918681294);
\draw[draw=black,fill=black,opacity=0.7] (axis cs:34,0) rectangle (axis cs:34.5,0.000148552439010971);
\draw[draw=black,fill=black,opacity=0.7] (axis cs:34.5,0) rectangle (axis cs:35,0.000137548554639788);
\draw[draw=black,fill=white!50.1960784313725!black,opacity=0.6] (axis cs:0,0) rectangle (axis cs:0.5,0.00186518945891941);
\draw[draw=black,fill=white!50.1960784313725!black,opacity=0.6] (axis cs:0.5,0) rectangle (axis cs:1,0.00625131211478192);
\draw[draw=black,fill=white!50.1960784313725!black,opacity=0.6] (axis cs:1,0) rectangle (axis cs:1.5,0.0114872251250221);
\draw[draw=black,fill=white!50.1960784313725!black,opacity=0.6] (axis cs:1.5,0) rectangle (axis cs:2,0.018770664500928);
\draw[draw=black,fill=white!50.1960784313725!black,opacity=0.6] (axis cs:2,0) rectangle (axis cs:2.5,0.0271247058980971);
\draw[draw=black,fill=white!50.1960784313725!black,opacity=0.6] (axis cs:2.5,0) rectangle (axis cs:3,0.032770458744602);
\draw[draw=black,fill=white!50.1960784313725!black,opacity=0.6] (axis cs:3,0) rectangle (axis cs:3.5,0.0426501080220711);
\draw[draw=black,fill=white!50.1960784313725!black,opacity=0.6] (axis cs:3.5,0) rectangle (axis cs:4,0.0501710872214448);
\draw[draw=black,fill=white!50.1960784313725!black,opacity=0.6] (axis cs:4,0) rectangle (axis cs:4.5,0.0608854715123675);
\draw[draw=black,fill=white!50.1960784313725!black,opacity=0.6] (axis cs:4.5,0) rectangle (axis cs:5,0.0693264993237643);
\draw[draw=black,fill=white!50.1960784313725!black,opacity=0.6] (axis cs:5,0) rectangle (axis cs:5.5,0.077976629092439);
\draw[draw=black,fill=white!50.1960784313725!black,opacity=0.6] (axis cs:5.5,0) rectangle (axis cs:6,0.0844002412200179);
\draw[draw=black,fill=white!50.1960784313725!black,opacity=0.6] (axis cs:6,0) rectangle (axis cs:6.5,0.0908539640294449);
\draw[draw=black,fill=white!50.1960784313725!black,opacity=0.6] (axis cs:6.5,0) rectangle (axis cs:7,0.0933314040192742);
\draw[draw=black,fill=white!50.1960784313725!black,opacity=0.6] (axis cs:7,0) rectangle (axis cs:7.5,0.0926304942584785);
\draw[draw=black,fill=white!50.1960784313725!black,opacity=0.6] (axis cs:7.5,0) rectangle (axis cs:8,0.0968493353485186);
\draw[draw=black,fill=white!50.1960784313725!black,opacity=0.6] (axis cs:8,0) rectangle (axis cs:8.5,0.0974097285940235);
\draw[draw=black,fill=white!50.1960784313725!black,opacity=0.6] (axis cs:8.5,0) rectangle (axis cs:9,0.0936709855978936);
\draw[draw=black,fill=white!50.1960784313725!black,opacity=0.6] (axis cs:9,0) rectangle (axis cs:9.5,0.0880770900367936);
\draw[draw=black,fill=white!50.1960784313725!black,opacity=0.6] (axis cs:9.5,0) rectangle (axis cs:10,0.0863741636967219);
\draw[draw=black,fill=white!50.1960784313725!black,opacity=0.6] (axis cs:10,0) rectangle (axis cs:10.5,0.0822924934906561);
\draw[draw=black,fill=white!50.1960784313725!black,opacity=0.6] (axis cs:10.5,0) rectangle (axis cs:11,0.0753252162741544);
\draw[draw=black,fill=white!50.1960784313725!black,opacity=0.6] (axis cs:11,0) rectangle (axis cs:11.5,0.0688781847273603);
\draw[draw=black,fill=white!50.1960784313725!black,opacity=0.6] (axis cs:11.5,0) rectangle (axis cs:12,0.0618289395436057);
\draw[draw=black,fill=white!50.1960784313725!black,opacity=0.6] (axis cs:12,0) rectangle (axis cs:12.5,0.0572638256123133);
\draw[draw=black,fill=white!50.1960784313725!black,opacity=0.6] (axis cs:12.5,0) rectangle (axis cs:13,0.0528040990674889);
\draw[draw=black,fill=white!50.1960784313725!black,opacity=0.6] (axis cs:13,0) rectangle (axis cs:13.5,0.0464959112203274);
\draw[draw=black,fill=white!50.1960784313725!black,opacity=0.6] (axis cs:13.5,0) rectangle (axis cs:14,0.0417250409630734);
\draw[draw=black,fill=white!50.1960784313725!black,opacity=0.6] (axis cs:14,0) rectangle (axis cs:14.5,0.0365192386346813);
\draw[draw=black,fill=white!50.1960784313725!black,opacity=0.6] (axis cs:14.5,0) rectangle (axis cs:15,0.0334546403488155);
\draw[draw=black,fill=white!50.1960784313725!black,opacity=0.6] (axis cs:15,0) rectangle (axis cs:15.5,0.0293913711149902);
\draw[draw=black,fill=white!50.1960784313725!black,opacity=0.6] (axis cs:15.5,0) rectangle (axis cs:16,0.0262799339906941);
\draw[draw=black,fill=white!50.1960784313725!black,opacity=0.6] (axis cs:16,0) rectangle (axis cs:16.5,0.0245335144435086);
\draw[draw=black,fill=white!50.1960784313725!black,opacity=0.6] (axis cs:16.5,0) rectangle (axis cs:17,0.021440478291453);
\draw[draw=black,fill=white!50.1960784313725!black,opacity=0.6] (axis cs:17,0) rectangle (axis cs:17.5,0.0189864577208388);
\draw[draw=black,fill=white!50.1960784313725!black,opacity=0.6] (axis cs:17.5,0) rectangle (axis cs:18,0.0155521671745056);
\draw[draw=black,fill=white!50.1960784313725!black,opacity=0.6] (axis cs:18,0) rectangle (axis cs:18.5,0.0135665349881941);
\draw[draw=black,fill=white!50.1960784313725!black,opacity=0.6] (axis cs:18.5,0) rectangle (axis cs:19,0.0116478154282115);
\draw[draw=black,fill=white!50.1960784313725!black,opacity=0.6] (axis cs:19,0) rectangle (axis cs:19.5,0.0101389357044938);
\draw[draw=black,fill=white!50.1960784313725!black,opacity=0.6] (axis cs:19.5,0) rectangle (axis cs:20,0.00801278700289146);
\draw[draw=black,fill=white!50.1960784313725!black,opacity=0.6] (axis cs:20,0) rectangle (axis cs:20.5,0.00697564129479278);
\draw[draw=black,fill=white!50.1960784313725!black,opacity=0.6] (axis cs:20.5,0) rectangle (axis cs:21,0.00582808975325133);
\draw[draw=black,fill=white!50.1960784313725!black,opacity=0.6] (axis cs:21,0) rectangle (axis cs:21.5,0.00439783236547007);
\draw[draw=black,fill=white!50.1960784313725!black,opacity=0.6] (axis cs:21.5,0) rectangle (axis cs:22,0.00413687312278718);
\draw[draw=black,fill=white!50.1960784313725!black,opacity=0.6] (axis cs:22,0) rectangle (axis cs:22.5,0.00347276530647237);
\draw[draw=black,fill=white!50.1960784313725!black,opacity=0.6] (axis cs:22.5,0) rectangle (axis cs:23,0.00263468466170231);
\draw[draw=black,fill=white!50.1960784313725!black,opacity=0.6] (axis cs:23,0) rectangle (axis cs:23.5,0.00227837492650066);
\draw[draw=black,fill=white!50.1960784313725!black,opacity=0.6] (axis cs:23.5,0) rectangle (axis cs:24,0.00197224966104573);
\draw[draw=black,fill=white!50.1960784313725!black,opacity=0.6] (axis cs:24,0) rectangle (axis cs:24.5,0.00154568166819869);
\draw[draw=black,fill=white!50.1960784313725!black,opacity=0.6] (axis cs:24.5,0) rectangle (axis cs:25,0.00117431659207303);
\draw[draw=black,fill=white!50.1960784313725!black,opacity=0.6] (axis cs:25,0) rectangle (axis cs:25.5,0.000936776768605265);
\draw[draw=black,fill=white!50.1960784313725!black,opacity=0.6] (axis cs:25.5,0) rectangle (axis cs:26,0.000841426276086515);
\draw[draw=black,fill=white!50.1960784313725!black,opacity=0.6] (axis cs:26,0) rectangle (axis cs:26.5,0.000685854419871712);
\draw[draw=black,fill=white!50.1960784313725!black,opacity=0.6] (axis cs:26.5,0) rectangle (axis cs:27,0.00063065150315033);
\draw[draw=black,fill=white!50.1960784313725!black,opacity=0.6] (axis cs:27,0) rectangle (axis cs:27.5,0.000628978687492106);
\draw[draw=black,fill=white!50.1960784313725!black,opacity=0.6] (axis cs:27.5,0) rectangle (axis cs:28,0.000396457310999014);
\draw[draw=black,fill=white!50.1960784313725!black,opacity=0.6] (axis cs:28,0) rectangle (axis cs:28.5,0.00037638352310033);
\draw[draw=black,fill=white!50.1960784313725!black,opacity=0.6] (axis cs:28.5,0) rectangle (axis cs:29,0.000368019444809211);
\draw[draw=black,fill=white!50.1960784313725!black,opacity=0.6] (axis cs:29,0) rectangle (axis cs:29.5,0.000292742740189145);
\draw[draw=black,fill=white!50.1960784313725!black,opacity=0.6] (axis cs:29.5,0) rectangle (axis cs:30,0.000225830113860198);
\draw[draw=black,fill=white!50.1960784313725!black,opacity=0.6] (axis cs:30,0) rectangle (axis cs:30.5,0.000210774772936185);
\draw[draw=black,fill=white!50.1960784313725!black,opacity=0.6] (axis cs:30.5,0) rectangle (axis cs:31,0.000192373800695724);
\draw[draw=black,fill=white!50.1960784313725!black,opacity=0.6] (axis cs:31,0) rectangle (axis cs:31.5,0.000127133990025);
\draw[draw=black,fill=white!50.1960784313725!black,opacity=0.6] (axis cs:31.5,0) rectangle (axis cs:32,0.000117097096075658);
\draw[draw=black,fill=white!50.1960784313725!black,opacity=0.6] (axis cs:32,0) rectangle (axis cs:32.5,0.000113751464759211);
\draw[draw=black,fill=white!50.1960784313725!black,opacity=0.6] (axis cs:32.5,0) rectangle (axis cs:33,9.86961238351975e-05);
\draw[draw=black,fill=white!50.1960784313725!black,opacity=0.6] (axis cs:33,0) rectangle (axis cs:33.5,8.69864142276317e-05);
\draw[draw=black,fill=white!50.1960784313725!black,opacity=0.6] (axis cs:33.5,0) rectangle (axis cs:34,7.69495202782896e-05);
\draw[draw=black,fill=white!50.1960784313725!black,opacity=0.6] (axis cs:34,0) rectangle (axis cs:34.5,8.36407829111844e-05);
\draw[draw=black,fill=white!50.1960784313725!black,opacity=0.6] (axis cs:34.5,0) rectangle (axis cs:35,7.86223359365133e-05);
\draw[draw=black,fill=white,opacity=0.5] (axis cs:0,0) rectangle (axis cs:0.5,0.00447782186497453);
\draw[draw=black,fill=white,opacity=0.5] (axis cs:0.5,0) rectangle (axis cs:1,0.0128077186932797);
\draw[draw=black,fill=white,opacity=0.5] (axis cs:1,0) rectangle (axis cs:1.5,0.020611375772496);
\draw[draw=black,fill=white,opacity=0.5] (axis cs:1.5,0) rectangle (axis cs:2,0.0285700343778076);
\draw[draw=black,fill=white,opacity=0.5] (axis cs:2,0) rectangle (axis cs:2.5,0.0348944793623122);
\draw[draw=black,fill=white,opacity=0.5] (axis cs:2.5,0) rectangle (axis cs:3,0.0394966851679805);
\draw[draw=black,fill=white,opacity=0.5] (axis cs:3,0) rectangle (axis cs:3.5,0.0482207834083305);
\draw[draw=black,fill=white,opacity=0.5] (axis cs:3.5,0) rectangle (axis cs:4,0.0523866886220269);
\draw[draw=black,fill=white,opacity=0.5] (axis cs:4,0) rectangle (axis cs:4.5,0.0579476075705807);
\draw[draw=black,fill=white,opacity=0.5] (axis cs:4.5,0) rectangle (axis cs:5,0.0608467101882886);
\draw[draw=black,fill=white,opacity=0.5] (axis cs:5,0) rectangle (axis cs:5.5,0.0659579333512574);
\draw[draw=black,fill=white,opacity=0.5] (axis cs:5.5,0) rectangle (axis cs:6,0.0663693571550905);
\draw[draw=black,fill=white,opacity=0.5] (axis cs:6,0) rectangle (axis cs:6.5,0.0716183594523662);
\draw[draw=black,fill=white,opacity=0.5] (axis cs:6.5,0) rectangle (axis cs:7,0.0749556762611336);
\draw[draw=black,fill=white,opacity=0.5] (axis cs:7,0) rectangle (axis cs:7.5,0.0760732181282896);
\draw[draw=black,fill=white,opacity=0.5] (axis cs:7.5,0) rectangle (axis cs:8,0.0748102427304763);
\draw[draw=black,fill=white,opacity=0.5] (axis cs:8,0) rectangle (axis cs:8.5,0.0782394122954482);
\draw[draw=black,fill=white,opacity=0.5] (axis cs:8.5,0) rectangle (axis cs:9,0.079282323798188);
\draw[draw=black,fill=white,opacity=0.5] (axis cs:9,0) rectangle (axis cs:9.5,0.0781839179219079);
\draw[draw=black,fill=white,opacity=0.5] (axis cs:9.5,0) rectangle (axis cs:10,0.0790507783085889);
\draw[draw=black,fill=white,opacity=0.5] (axis cs:10,0) rectangle (axis cs:10.5,0.0753441368759155);
\draw[draw=black,fill=white,opacity=0.5] (axis cs:10.5,0) rectangle (axis cs:11,0.0747757979468995);
\draw[draw=black,fill=white,opacity=0.5] (axis cs:11,0) rectangle (axis cs:11.5,0.0718097193611258);
\draw[draw=black,fill=white,opacity=0.5] (axis cs:11.5,0) rectangle (axis cs:12,0.0689278391352063);
\draw[draw=black,fill=white,opacity=0.5] (axis cs:12,0) rectangle (axis cs:12.5,0.0636597008470546);
\draw[draw=black,fill=white,opacity=0.5] (axis cs:12.5,0) rectangle (axis cs:13,0.0586958248138307);
\draw[draw=black,fill=white,opacity=0.5] (axis cs:13,0) rectangle (axis cs:13.5,0.0558713525605391);
\draw[draw=black,fill=white,opacity=0.5] (axis cs:13.5,0) rectangle (axis cs:14,0.0519714376200185);
\draw[draw=black,fill=white,opacity=0.5] (axis cs:14,0) rectangle (axis cs:14.5,0.0471281183293132);
\draw[draw=black,fill=white,opacity=0.5] (axis cs:14.5,0) rectangle (axis cs:15,0.0409567612718162);
\draw[draw=black,fill=white,opacity=0.5] (axis cs:15,0) rectangle (axis cs:15.5,0.0357364629608545);
\draw[draw=black,fill=white,opacity=0.5] (axis cs:15.5,0) rectangle (axis cs:16,0.0319800679519036);
\draw[draw=black,fill=white,opacity=0.5] (axis cs:16,0) rectangle (axis cs:16.5,0.0284437368380263);
\draw[draw=black,fill=white,opacity=0.5] (axis cs:16.5,0) rectangle (axis cs:17,0.0251351240155729);
\draw[draw=black,fill=white,opacity=0.5] (axis cs:17,0) rectangle (axis cs:17.5,0.0237688142670294);
\draw[draw=black,fill=white,opacity=0.5] (axis cs:17.5,0) rectangle (axis cs:18,0.0211050843370958);
\draw[draw=black,fill=white,opacity=0.5] (axis cs:18,0) rectangle (axis cs:18.5,0.0183341928582568);
\draw[draw=black,fill=white,opacity=0.5] (axis cs:18.5,0) rectangle (axis cs:19,0.0161718258892734);
\draw[draw=black,fill=white,opacity=0.5] (axis cs:19,0) rectangle (axis cs:19.5,0.0133530944332446);
\draw[draw=black,fill=white,opacity=0.5] (axis cs:19.5,0) rectangle (axis cs:20,0.0109189963938225);
\draw[draw=black,fill=white,opacity=0.5] (axis cs:20,0) rectangle (axis cs:20.5,0.00935749953834422);
\draw[draw=black,fill=white,opacity=0.5] (axis cs:20.5,0) rectangle (axis cs:21,0.0084676759626121);
\draw[draw=black,fill=white,opacity=0.5] (axis cs:21,0) rectangle (axis cs:21.5,0.00620006104381089);
\draw[draw=black,fill=white,opacity=0.5] (axis cs:21.5,0) rectangle (axis cs:22,0.00530449667081599);
\draw[draw=black,fill=white,opacity=0.5] (axis cs:22,0) rectangle (axis cs:22.5,0.0043534379242808);
\draw[draw=black,fill=white,opacity=0.5] (axis cs:22.5,0) rectangle (axis cs:23,0.00381188938249114);
\draw[draw=black,fill=white,opacity=0.5] (axis cs:23,0) rectangle (axis cs:23.5,0.00336793439416888);
\draw[draw=black,fill=white,opacity=0.5] (axis cs:23.5,0) rectangle (axis cs:24,0.00315169769727054);
\draw[draw=black,fill=white,opacity=0.5] (axis cs:24,0) rectangle (axis cs:24.5,0.00287805302774432);
\draw[draw=black,fill=white,opacity=0.5] (axis cs:24.5,0) rectangle (axis cs:25,0.00240539405310812);
\draw[draw=black,fill=white,opacity=0.5] (axis cs:25,0) rectangle (axis cs:25.5,0.00178538794872703);
\draw[draw=black,fill=white,opacity=0.5] (axis cs:25.5,0) rectangle (axis cs:26,0.00147347129744889);
\draw[draw=black,fill=white,opacity=0.5] (axis cs:26,0) rectangle (axis cs:26.5,0.00116920904252113);
\draw[draw=black,fill=white,opacity=0.5] (axis cs:26.5,0) rectangle (axis cs:27,0.00102951630912662);
\draw[draw=black,fill=white,opacity=0.5] (axis cs:27,0) rectangle (axis cs:27.5,0.000876428382118947);
\draw[draw=black,fill=white,opacity=0.5] (axis cs:27.5,0) rectangle (axis cs:28,0.000748217243250018);
\draw[draw=black,fill=white,opacity=0.5] (axis cs:28,0) rectangle (axis cs:28.5,0.000650623689782625);
\draw[draw=black,fill=white,opacity=0.5] (axis cs:28.5,0) rectangle (axis cs:29,0.000593215717154746);
\draw[draw=black,fill=white,opacity=0.5] (axis cs:29,0) rectangle (axis cs:29.5,0.000620006104381089);
\draw[draw=black,fill=white,opacity=0.5] (axis cs:29.5,0) rectangle (axis cs:30,0.000466918177373413);
\draw[draw=black,fill=white,opacity=0.5] (axis cs:30,0) rectangle (axis cs:30.5,0.000470745375548605);
\draw[draw=black,fill=white,opacity=0.5] (axis cs:30.5,0) rectangle (axis cs:31,0.000313830250365737);
\draw[draw=black,fill=white,opacity=0.5] (axis cs:31,0) rectangle (axis cs:31.5,0.00026981747135103);
\draw[draw=black,fill=white,opacity=0.5] (axis cs:31.5,0) rectangle (axis cs:32,0.000246854282299878);
\draw[draw=black,fill=white,opacity=0.5] (axis cs:32,0) rectangle (axis cs:32.5,0.000237286286861898);
\draw[draw=black,fill=white,opacity=0.5] (axis cs:32.5,0) rectangle (axis cs:33,0.000216236696898343);
\draw[draw=black,fill=white,opacity=0.5] (axis cs:33,0) rectangle (axis cs:33.5,0.000199014305109979);
\draw[draw=black,fill=white,opacity=0.5] (axis cs:33.5,0) rectangle (axis cs:34,0.000176051116058828);
\draw[draw=black,fill=white,opacity=0.5] (axis cs:34,0) rectangle (axis cs:34.5,0.000137779134306909);
\draw[draw=black,fill=white,opacity=0.5] (axis cs:34.5,0) rectangle (axis cs:35,0.000132038337044121);
\end{axis}

\end{tikzpicture}
}
    \caption{Distributions of position coordinates, horizontal speed, and distance to the closest aircraft (actual vs. synthetic--multiple vs. synthetic--single trajectories of KJFK 13L).}
    \label{fig:result-hist}
\end{figure*}

Fig. \ref{fig:result-hist} shows the empirical distributions of the position coordinates, horizontal speed, and distance to the closest aircraft, for the actual dataset and the synthetic datasets from each model.
From Fig. \ref{fig:hist_speed} and \ref{fig:hist_encounter}, we observe for both variables that the synthetic distribution from the multiple trajectory model (synthetic--multiple) is more concentrated near its mode than the one from the single trajectory model (synthetic--single).
This result is probable because the multiple model not only learns individual trajectories but also the pairwise relations between them. 
%Thus, the degree of freedom of individual trajectory is not high compared to a trajectory generated from the single model.

We observe from the figures that the synthetic trajectory sets have much higher densities than the actual dataset at lower speeds and ranges.
In practice, most aircraft cannot maintain $100$ knots, which is why the actual dataset has minimal density in this range. 
Also, it is specified in FAA Order JO 7110.65Y \cite{FAAorderATC} that the separation minimum between two IFR aircraft flying below FL$290$ within $40$ miles from the radar antenna is $3$ miles horizontally and $1,000$ feet vertically.
The invalid outputs may occur from the fact that our models do not directly learn the aircraft trajectories, but learn the deviations and then re-construct the trajectories.

To measure the similarity between the actual distribution and each of the synthetic distributions,
we use the Jensen--Shannon divergence $(D_{JS})$, which is a symmetric and smoothed variation of the Kullback--Leibler divergence $(D_{KL})$ \cite{lin1991divergence}.
The $D_{JS}$ between two empirical distributions $P$ and $Q$ is defined as
\begin{align}
    \begin{split}
        D_{JS}(P, Q) &= \frac{1}{2} \Big( D_{KL}(P, M) + D_{KL}(Q, M) \Big)\\
        &= \frac{1}{2}\sum_{x_i} \Big( P(x_i) \log \frac{P(x_i)}{M(x_i)} + Q(x_i) \log \frac{Q(x_i)}{M(x_i)} \Big)
    \end{split}
\end{align}
where ${M = (P+Q)/2}$. The output is bounded by $0$ and $1$.

\begin{table}[ht]
\centering
\caption{$D_{JS}$ between actual distribution, distribution of synthetic trajectories (single model) and distribution of synthetic trajectories (multiple model)\label{tab:djs}}
\begin{tabular}{@{}lrr@{}} \hline
\textbf{Variable} & \textbf{$D_{JS}$ Single} & \textbf{$D_{JS}$ Multiple} \\ \hline
X-coordinate (NM) & 0.0245 &  0.0119 \\ 
Y-coordinate (NM) & 0.0166 & 0.0446 \\ 
Horizontal speed (Knots) &  0.0650 &  0.0147 \\ 
Distance to closest aircraft (NM) & 0.0261 & 0.0307 \\ \hline
\end{tabular}
\end{table}

Table \ref{tab:djs} presents the $D_{JS}$ between the actual distribution and each of the synthetic distributions for different variables (See Fig. \ref{fig:result-hist} for a visualization of the distributions).
We find that the multiple model outperforms the single model in terms of horizontal speed, not only because the multiple model has lower $D_{JS}$ but the single model outputs speeds that are too high or too low.
For the closest distance, the single model has \textcolor{blue}{a }slightly lower JS divergence.
What is more important, however, is the loss of separation that occurs whenever the separation minima are breached.
We counted the number of times when the distance to the closest aircraft goes below $3$ miles within the 1,000 trajectory sets generated using each model.
It turns out that loss of separation occurred $781$ times when using the single model, and $482$ times when using the multiple model.
These results indicate that the multiple model performs better, or at least comparably than the single model in terms of both similarity and safety. We note that an explicit collision avoidance model can be added on top of this nominal trajectory model to prevent actual collisions in simulation.

Fig. \ref{fig:multi_trajs_generation_sets} shows two sample three-trajectory sets generated using the multiple model\textcolor{blue}{s.}
Along each trajectory, the dots indicate the current positions and the ticks indicate previous positions at each minute.
For example, the ${-1}$ ticks indicate the locations of the three aircraft one minute ago.
For the first set shown in Fig. \ref{fig:multi_trajs_generation_set1}, the model successfully generated the trajectories, satisfying the horizontal and vertical separation requirements.
After the second AC gets close to the first AC, it makes a detour and reduces speed to maintain the horizontal and vertical separation from the first AC.
For the second set shown in Fig. \ref{fig:multi_trajs_generation_set2}, on the other hand, loss of separation occurred between the first AC and the second AC at ${t=-2}$.
Besides, at the current time step, the second AC and the third AC overlap in the top view.
Although the two ACs are maintaining the vertical separation minimum, in practice, the air traffic controllers do not want to see overlapping tracks from their radar control screen.
A plausible reason for this undesirable output is the configuration of the procedures.
As with all the other experiments, each trajectory follows one of the procedures for KFJK 13L shown in Fig. \ref{fig:jfk_13L_org_syn}.
We can see that the procedures themselves overlap with each other.
Thus, the synthetic trajectories for KJFK 13L arrivals are likely to overlap or fail in separation even if our models separate them based on their inter-arrival times. %\todo{would it be helpful to have stronger takeaways about results here?}
We could use some heuristics to solve this problem or experiment on another environment such as KCLT 36C where the procedures do not overlap. For now, we will leave these for future work.

\begin{figure*}[bt!]
    \centering
    \subfigure[First example set.]{
    \includegraphics[width=7.1in]{figures/gen_three_trajs_set1_t=560.pdf}
    \label{fig:multi_trajs_generation_set1}}
    \subfigure[Second example set.]{
    \includegraphics[width=7.1in]{figures/gen_three_trajs_set2_t=240.pdf}
    \label{fig:multi_trajs_generation_set2}}
    \caption{Example sets of three trajectories generated based on the multiple trajectory model. Top view (left column) and side views from north and east (right column).}
    \label{fig:multi_trajs_generation_sets}
\end{figure*}

To visualize the behavior of multiple trajectories over time, we developed a dynamic simulation tool embedded with Google Maps.
The source code for the experiments and a simulation demo are publicly available at \\
https://github.com/sisl/terminal\_airspace\_modeling.
