\documentclass[journal]{new-aiaa}
%\documentclass[conf]{new-aiaa} for conference papers
\usepackage[utf8]{inputenc}
\usepackage{textcomp}

% \usepackage{amsmath, amsfonts, amssymb, amscd}
\usepackage[usenames,dvipsnames]{color} 

\usepackage[capitalize]{cleveref}
\usepackage{longtable}
\usepackage{multicol}
\usepackage{pdfpages}
\usepackage[utf8]{inputenc}
\usepackage[table]{xcolor}
\usepackage[final]{changes}

\usepackage{graphics}
\usepackage{subfigure}
\usepackage{tikz}

\usepackage{pgfplots}
\usetikzlibrary{pgfplots.fillbetween}
\pgfplotsset{scaled x ticks=false}
\pgfplotsset{compat=newest}
\pgfplotsset{every axis/.append style={axis on top}}
\setlength{\unitlength}{1in} 
\usepgfplotslibrary{groupplots}

\usepackage{nomencl}
% \usepackage{enumerate}
% \usepackage[normalem]{ulem}
\usepackage{cancel}
% \usepackage{subcaption}
% \usepackage{graphicx}

\usepackage[none]{hyphenat}

\usepackage[style=ieee, url=false, doi=true]{biblatex}
\renewcommand*{\bibfont}{\footnotesize}
\addbibresource{jais.bib}

%\usepackage{lineno}
%\linenumbers

% \usepackage[version=4]{mhchem}
\usepackage{siunitx}
\usepackage{longtable,tabularx}
\usepackage{todonotes}
\setlength\LTleft{0pt} 

\title{Inferring Traffic Models in Terminal Airspace \\ from Flight Tracks and Procedures}

\author{Soyeon Jung,\footnote{Ph.D. candidate, Aeronautics and Astronautics.} Amelia Hardy,\footnote{M.S. student, Computer Science.} and Mykel J. Kochnederfer\footnote{Associate Professor, Aeronautics and Astronautics. AIAA Associate Fellow.  Corresponding author: mykel@stanford.edu.}}
\affil{Stanford University, Stanford, CA 94304 USA}

\begin{document}

\maketitle
\vspace{-1.2cm}
\begin{center}
    DOI: \href{https://doi.org/10.48550/arXiv.2303.09981}{10.48550/arXiv.2303.09981}
\end{center}

\begin{abstract}
Realistic aircraft trajectory models are useful in the design and validation of air traffic management (ATM) systems. 
Models of aircraft operated under instrument flight rules (IFR) require capturing the variability inherent in how aircraft follow standard flight procedures. The variability in aircraft behavior differs among flight stages. In this paper, we propose a simple probabilistic model that can learn this variability from procedural data and flight tracks collected from radar surveillance data. For each segment, we use a Gaussian mixture model to learn the deviations of aircraft trajectories from their procedures. Given new procedures, we generate synthetic trajectories by sampling a series of deviations from the Gaussian mixture model and reconstructing the aircraft trajectory using the deviations and the procedures. We extend this method to capture pairwise correlations between aircraft and show how a pairwise model can be used to generate traffic involving an arbitrary number of aircraft. We demonstrate the proposed models on the arrival tracks and procedures of the John F. Kennedy International Airport. Distributional similarity between the original and the synthetic trajectory dataset was evaluated using the Jensen-Shannon divergence between the empirical distributions of different variables and we provide qualitative analyses of the synthetic trajectories generated. 
\end{abstract}

% \section*{Nomenclature}

% \noindent(Nomenclature entries should have the units identified)

% {\renewcommand\arraystretch{1.0}
% \noindent\begin{longtable*}{@{}l @{\quad=\quad} l@{}}
% $A$  & amplitude of oscillation \\
% $a$ &    cylinder diameter \\
% $C_p$& pressure coefficient \\
% $Cx$ & force coefficient in the \textit{x} direction \\
% $Cy$ & force coefficient in the \textit{y} direction \\
% c   & chord \\
% d$t$ & time step \\
% $Fx$ & $X$ component of the resultant pressure force acting on the vehicle \\
% $Fy$ & $Y$ component of the resultant pressure force acting on the vehicle \\
% $f, g$   & generic functions \\
% $h$  & height \\
% $i$  & time index during navigation \\
% $j$  & waypoint index \\
% $K$  & trailing-edge (TE) nondimensional angular deflection rate\\
% $\Theta$ & boundary-layer momentum thickness\\
% $\rho$ & density\\
% \multicolumn{2}{@{}l}{Subscripts}\\
% cg & center of gravity\\
% $G$ & generator body\\
% iso	& waypoint index
% \end{longtable*}}




% \section{Introduction}
% \lettrine{T}{his} document is a \LaTeX{} template for preparation of papers for AIAA Technical Journals. If you are reading a hard-copy or .pdf version of this document, download the electronic file, new-aiaa.cls, and use it to prepare your manuscript.

% Authors using \url{https://www.overleaf.com} may simply open the AIAA template from the Overleaf gallery to work online; no local installation of any files is required. Authors using a local \LaTeX{} installation will need to open the template in Overleaf and use the ``Download as zip'' option from the project menu to download a local copy of the template files. To create your formatted manuscript, type your own text over sections of the template, or cut and paste from another document and then use the available markup styles. Note that special formatting such as subscripts, superscripts, and italics may be lost when you copy your text into the template from a Word Processing program such as Microsoft Word. See Sec. IV for more detailed formatting guidelines.

\newlength\figureheight
\newlength\figurewidth

\section{Introduction}

The increasing complexity of source code poses a key challenge to the reliability of large-scale software systems. Software bugs in these systems can lead to safety issues~\cite{bug_safety} for users around the world as well as cause non-negligible financial losses~\cite{bug_loss}. As such, developers have to spend a large amount of time and effort on bug fixing. Consequently, \aprfull (\apr), designed to automatically generate patches to fix software bugs, has attracted wide attention from both academia and industry~\cite{long2016prophet, legoues2012genprog, long2015spr, lou2020can, tufano2018empstudy}. 


To achieve \apr, one popular approach is known as Generate-and-Validate (G\&V)~\cite{qi2015gv, ghanbari2019prapr, lou2020can, le2016hdrepair, legoues2012genprog, wen2018capgen, hua2018sketchfix, martinez2016astor, koyuncu2020fixminder, liu2019tbar, liu2019avatar}, which is typically based on the following pipeline: First, fault localization techniques~\cite{wong2016fl, abreu2007ochiai, zhang2013injecting, papadakis2015metallaxis, li2019deepfl, li2017transforming} are applied to determine the suspicious locations in programs where bugs are likely to exist. Then, the buggy locations are used by the \apr tools to generate a list of patches that replace buggy lines with correct lines. Afterward, each patch is validated against the original test suite to identify any \emph{plausible patches} (i.e., passing all tests in the test suite). Finally, to determine the \emph{correct patches}, developers examine the list of plausible patches to see if any of them can correctly fix the bug. 

Traditional \apr tools can mainly be categorized into heuristic-based~\cite{legoues2012genprog, le2016hdrepair, wen2018capgen}, constraint-based~\cite{mechtaev2016angelix, le2017s3, demacro2014nopol, long2015spr} and \template~\cite{ghanbari2019prapr, hua2018sketchfix, martinez2016astor, liu2019tbar, liu2019avatar}. Among these traditional tools, \template \apr tools~\cite{ghanbari2019prapr, liu2019tbar, benton2020effectiveness} have been able to achieve state-of-the-art results. \Template \apr tools typically leverage pre-defined templates (e.g., adding a nullness check) for bug fixing. However, since these fix templates are typically handcrafted, the number and types of bugs they are able to fix can be limited. 



To address the limitations of traditional \apr, researchers have proposed various \learning \apr tools~\cite{li2020dlfix, chen2018sequencer, jiang2021cure, lutellier2020coconut, zhu2021recoder, ye2022rewardrepair} based on the \nmtfull (\nmt) architecture~\cite{sutskever2014mt} where the input is the buggy code snippets and the goal is to translate the buggy code snippets into a fixed version. To accomplish this, \learning \apr tools require supervised training datasets with pairs of both buggy and fixed code snippets in order to learn how to perform this translation step. These training data are usually obtained by mining historical bug fixes using heuristics/keywords~\cite{dallmeier2007benchmark}, which can be imprecise for identifying bug-fixing commits; even the actual bug-fixing commits can include irrelevant code changes, leading to further pollution in the dataset~\cite{xia2022alpharepair}.
% 
Moreover, it can be hard for such \apr tools to generalize and fix bug types unseen during training. 



To better leverage recent advances in \plmfull{s} (\plm{s}), researchers~\cite{xia2022alpharepair, xia2023repairstudy, kolak2022patch, prenner2021codexws} have directly applied \plm{s} to generate patches without bug-fixing datasets. These \llm-based \apr tools work by either directly generating a complete code function~\cite{prenner2021codexws, xia2023repairstudy} or predict/infill the correct code snippet given its surrounding context~\cite{xia2022alpharepair, xia2023repairstudy}. By directly using \llm{s} that are pre-trained on billions of open-source code snippets, \llm-based \apr tools can achieve state-of-the-art performance on many repair datasets~\cite{xia2022alpharepair}. 


% 
%
%

Traditional \apr tools have long used the insight of the \emph{plastic surgery hypothesis}~\cite{barr2014plastic} where it states that the code ingredients to fix a bug already exist within the same project. Traditional \apr tools have manually designed pattern-~\cite{ghanbari2019prapr, saha2017elixir} or heuristic-based~\cite{jiang2018simfix, legoues2012genprog} approaches to finding and using such relevant code ingredients to generate fixes for bugs. However, the plastic surgery hypothesis has been largely ignored in \llm-based \apr. In fact, \llm provides a unique opportunity to fully automate the plastic surgery hypothesis idea via fine-tuning (learning project-specific information via model updates from the buggy project) and prompting (directly providing relevant code ingredients to the model), and make it directly applicable to different languages (since the \llm{s} are typically multi-lingual).%
Moreover, despite the intensive manual efforts involved, traditional \apr tools still cannot fully leverage project-specific information due to large search space for leveraging/composing existing code ingredients. In contrast, the project-specific information can effectively leveraged by \llm{s} due to their power in code understanding/vectorization, e.g., even partial/imprecise information may still guide \llm{s} in correct patch generation!
 To this end, we ask the question: \emph{How useful is the plastic surgery hypothesis in the era of \plm{s}}?








\mypara{Our Work.} To answer the question, we present \ourtech{\xspace} -- a \llm-based approach that automatically utilizes the plastic surgery hypothesis by systematically combining multiple fine-tuning and prompting strategies for \apr. \ourtech fine-tunes \plm{s} using two novel domain-specific training strategies: \textbf{\epfinetune} -- we fine-tune using the original buggy project by aggressively masking out a high percentage of tokens, which allows \plm to learn project-specific code tokens and programming styles; and \textbf{\rofinetune} -- which only masks out a single continuous code sequence per training sample, allowing the model to get used to the final \csapr task of predicting a single continuous code sequence. Furthermore, we directly leverage the ability for \plm{s} to understand natural language instructions and introduce a novel prompting strategy, \textbf{\idprompting}, which uses information retrieval and static analysis to obtain a list of relevant identifiers for the buggy lines. While such relevant identifiers are critical for fixing some difficult bugs, they may not be seen by the \llm during inference due to limited context window size. Through the use of prompting, we directly tell the model to use these extracted identifiers (relevant code ingredients) to generate the correct code. Finally, to perform repair, we combine all four model variants (including the base model, both fine-tuned models and the base model with prompting) for the final repair.





While our insight of leveraging the plastic surgery hypothesis for \llm-based \apr is generalizable across different types of \plm{s}, to implement \ourtech, we choose a recent \plm{\xspace}, \ctfive~\cite{wang2021codet5}, which is pre-trained on millions of open-source code snippets. \ctfive is an encoder-decoder model trained using \mspfull (\msp) objective where a percentage of tokens are masked out and each continuous masked token sequence is referred to as a masked span. Also, although we only extract relevant identifiers from the current buggy project (since this paper focuses on the plastic surgery hypothesis), our work can be easily extended to obtain other code information (such as relevant statements or functions) from other sources, such as  the massive pre-training corpora~\cite{husain2020codesearchnet} or historical bug-fixing datasets~\cite{jiang2019infer}, which can provide more coding knowledge for \llm{s}. Besides, although we mainly focus on using traditional string comparison algorithms for information retrieval in this paper, these techniques can be easily replaced by other frequency-based retrieval~\cite{robertson2009probabilistic} and neural search (or embedding-based search)~\cite{reimers2019sentence}.
  In summary, this paper makes the following contributions:


%


\begin{itemize}[noitemsep, leftmargin=*, topsep=0pt]
    \item \textbf{Dimension.} This paper is the first to revisit the important plastic surgery hypothesis in the era of \llm{s}. It opens up a new dimension for \llm-based \apr to incorporate previously neglected information from the buggy project itself to boost \apr performance. Furthermore, it demonstrates the promising future of retrieval-based prompting for modern \llm-based \apr.
    \item \textbf{Implementation.} We implement \ourtech based on the recent \ctfive model. We augment the model using two novel fine-tuning strategies: \epfinetune and \rofinetune, along with a novel prompting strategy based on information retrieval and static analysis: \idprompting. We combine the patches generated by all four models together and perform patch ranking to speed up \apr.% 
    \item \textbf{Evaluation Study.} We conduct an extensive evaluation against state-of-the-art \apr tools. On the widely studied \dfj 1.2 and 2.0 datasets~\cite{just2014dfj}, \ourtech is able to achieve the new state-of-the-art results of 89 and 44 correct bug fixes (15 and 8 more than best baseline) respectively.  Furthermore, we perform a broad ablation study to justify our design. \ourtech demonstrates for the first time that the plastic surgery hypothesis can substantially boost \llm-based \apr and advance state-of-the-art \apr, while being fully automated and general. Moreover, even partial/imprecise code ingredients may still effectively guide \llm{s} for \apr!
\end{itemize}


% PTMTorrrent
\newcommand{\numberOfModelHub}{5\xspace}

\newcommand{\TotalNumberOfPackages}{{15,913}\xspace}
% 12401 from Hugging Face
% 185 from ONNX
% 33 from Model Hub
% 3245 from Model Zoo
% 49 from PyTorch Hub
% SUM (by Nick): 15,913

\newcommand{\HFNumberOfPackages}{{12,401}\xspace}
\newcommand{\HFNumberOfPackagesMetadata}{{124,427}\xspace}
\newcommand{\MZNumberOfPackages}{3,245\xspace}
\newcommand{\PHNumberOfPackages}{{49}\xspace}
\newcommand{\MHNumberOfPackages}{{33}\xspace}
\newcommand{\ONNXNumberOfPackages}{{185}\xspace}

\newcommand{\TotalDataSize}{\textasciitilde{61TB}\xspace}
\newcommand{\HFDataSize}{{61TB}\xspace}
\newcommand{\MZDataSize}{{115GB}\xspace}
\newcommand{\PHDataSize}{{1.5GB}\xspace}
\newcommand{\MHDataSize}{{721MB}\xspace}
\newcommand{\ONNXDataSize}{{441MB}\xspace}
%%%



% ICSE submission - HFTorrent v1

\newcommand{\PTMDatasetNPackages}{63,182\xspace}
\newcommand{\PTMDatasetPercentage}{{99.7\%}\xspace}
\newcommand{\PTMDatasetFailedPackages}{{186}\xspace}
\newcommand{\PTMDatasetFailedPercentage}{{0.3\%}\xspace}

\newcommand{\PTMDatasetNReposWithSignedCommits}{{132}\xspace}
\newcommand{\PTMDatasetPercentOfSignedCommits}{{0.208\%}\xspace}


\newcommand{\PercentOfVerifiedOrgs}{{3.188\%}\xspace}
\newcommand{\NOrganizations}{{6,243}\xspace}
\newcommand{\NVerifedOrgs}{{199}\xspace}

\newcommand{\NOfRepositoriesWithMalware}{{1}\xspace}
\newcommand{\PercentageOfRepositoriesWithMalware}{{0.002\%}\xspace}
\newcommand{\TotalRepositoriesForMalwareScanning}{{63,366}\xspace}
\section{Single Trajectory Model}
\label{sec:single trajectory model}
This section outlines our method to learn the distribution of aircraft trajectories relative to their flight procedures and to generate synthetic traffic scenarios. 
The interactions between multiple aircraft are not considered in this section. Our approach follows the following steps, which are described in detail in this section. First, we segment our aircraft trajectories to \textbf{construct input vectors} for our Gaussian mixture models. Next, we use these input vectors to \textbf{train a Gaussian Mixture Model} for each flight stage. To avoid overfitting and manage noise within our GMMs, we \textbf{use low-rank approximations of the GMM covariance matrices}. Finally, we use these approximated matrices to \textbf{generate synthetic trajectories}. \Cref{fig:flowchart} illustrates this process. %We note that our use of Gaussian Mixture Models assumes that our data points are generated from multiple Gaussian distributions.

\begin{figure}
    \centering
    \tikzstyle{process} = [rectangle, rounded corners, minimum width=3cm, minimum height=1cm, minimum width=5.5cm, text centered, draw=black]
    \tikzstyle{arrow} = [thick, ->, >=stealth]
    \begin{tikzpicture}[node distance=2cm]
    
        \node (step1) [process] {Segment aircraft trajectories};
        \node (step2) [process, below of=step1] {Fit GMMs};
        \node (step3) [process, below of=step2] {Compute low-rank approximation};
        \node (step4) [process, below of=step3] {Generate synthetic trajectories};
    
        \draw [arrow] (step1.south) -- (step2.north) node[midway,right] {Trajectory segments};
        \draw [arrow] (step2.south) -- (step3.north) node[midway,right] {GMMs};
        \draw [arrow] (step3.south) -- (step4.north) node[midway,right] {Low-rank covariance matrices};
    
    \end{tikzpicture}
        \caption{Flowchart overview of single trajectory model\label{fig:flowchart}}
\end{figure}

\subsection{Construction of Input Vector}
\label{subsec:input vector construction}
Before training a separate Gaussian mixture model for each flight stage, we need to construct the input vector sets in the proper format. 
First, each aircraft trajectory is divided into radar vector and final approach segments based on the distance to its IAP.
Then, the radar vector trajectory is assigned to one of the radar vector procedures using dynamic time warping (DTW) \cite{Muller2007}.

DTW is an algorithm for measuring similarity between two temporal sequences, which may vary in length.
It uses a dynamic programming approach to find the shortest distance between these sequences.
Given a pair of vectors ${x_1=[x_1^{(1)},\ldots, x_1^{(m)}] \in \mathbb{R}^m}$ and ${x_2=[x_2^{(1)}, \ldots, x_2^{(n)}] \in \mathbb{R}^n}$, the DTW distance between them is computed as
\begin{equation}
    DTW(x_1, x_2) = D(m, n)
\end{equation}
where for all ${i \in \{1,\ldots,m\},\ j \in \{1,\ldots,n\}}$,
\begin{equation}
    D(i, j) = \lVert x_1^{(i)} - x_2^{(j)} \rVert_2 + \min \begin{cases}
        D(i, j-1) \\ D(i-1, j) \\ D(i-1, j-1).
    \end{cases}
\end{equation}
After measuring the DTW distances between the radar vector trajectory and each of the radar vector procedures, we label the trajectory as the procedure to which it is closest, as measured by DTW distance.

Now with the two sets of aircraft trajectory-procedure pairs, one for each flight stage, we would like to train a GMM that takes as input the sequence of deviations between aircraft positions and the procedure points.
One challenge for training a GMM is that all the input vectors are required to be of the same length.
To deal with the issue of varying lengths of trajectories, we separately interpolate each dimension in an aircraft trajectory as a polynomial function of time and then re-sample a fixed number of points.
We also generate the procedural trajectories by interpolating the waypoints of the procedures and re-sampling the same number of points.
To integrate time into the spatial procedural trajectories of IAPs, we extract aircraft trajectories that pass very close to all the waypoints and take their mean. This allows us to estimate how a trajectory projected on a nominal path would proceed over time. %\todo{why do we do this?} 
The radar vector procedures already involve temporal factors because the procedures are defined as the nominal paths extracted from aircraft trajectories.
% \soyeonstrike{For the interpolation, we use piecewise cubic Hermite interpolating polynomials (PCHIP).
% PCHIP constructs a piecewise function where each piece ${p_i(x)}$ is a cubic polynomial for the observed data points with the specified values and first derivatives (slopes) at the interpolation points.}

\added{As done in prior work \cite{kochenderfer2010airspace}}, we use the piecewise cubic Hermite interpolation method. \added{We select this method because it has a number of desirable characteristics. First, it ensures that the interpolated function is continuous by fitting a cubic polynomial for each piece of the function and requiring first derivative continuity between pieces. Second, it preserves derivative information such as monotonicity; i.e., where the data is monotonic, the interpolated function will be monotonic. Finally, due to its use of low-degree polynomials, it generally avoids oscillation (Runge's phenomenon) that can be common in interpolation methods using high-degree polynomials. As a result of these properties, it is suitable for interpolating trajectory data and in particular procedural trajectories that should not oscillate between sample points.} \deleted{This method fits a cubic polynomial for each piece of the given function and imposes the continuity of the first derivative.
It preserves monotonicity and avoids oscillation in the intervals where the data is monotonic.
This property makes the piecewise cubic Hermite interpolation method appropriate for interpolating trajectory data, especially the procedural trajectories, which should not oscillate between the sample points.}

% For each sub-interval $x_i \leq x \leq x_{i+1}$,
% \begin{align}
%     p_i(x) = a_ix^3 + b_ix^2 + c_ix + d_i
% \end{align}
% with constraints:
% \begin{align}
%     \begin{split}
%         p_i(x_i) &= y_i\\
%         p_i(x_{i+1}) &= y_{i+1}\\
%         p'_i(x_i) &= y'_i = \frac{y_{i+1}-y_{i-1}}{x_{i+1}-x_{i-1}}\\
%         p'_i(x_{i+1}) &= y'_{i+1} = \frac{y_{i+2}-y_i}{x_{i+2}-x_i}
%     \end{split}
% \end{align} 

Finally, we have two sets of training input vectors. An individual input ${\tau \in \mathbb{R}^{3T+2}}$ is defined as
\begin{align}
\begin{split}
\tau = &[t, d, (x_1-x_1^p), (y_1-y_1^p), (z_1-z_1^p),\\
&\ldots,(x_T-x_T^p), (y_T-y_T^p), (z_T-z_T^p)]
\end{split}
\label{eq:single_model_training_data}
\end{align}
where $t$ and $d$ are the transit time and the total distance of the trajectory,
$[x_{1:T}, y_{1:T}, z_{1:T}]$ are the ENU coordinates of the aircraft trajectory at each timestep from 1 to $T$, and $[x_{1:T}^p, y_{1:T}^p, z_{1:T}^p]$ are the ENU coordinates of the corresponding procedural trajectory. 
We keep the transit time and total distance information of each trajectory so that we can later generate synthetic trajectories with reasonable airspeed.

%%%%%%%%%%%%%%%%%%%%%%%%%%%%%%%%%%%%%%%%%%%%%%%%%%%%%%%%%%%%%%%%%%
\subsection{Gaussian Mixture Model}
\label{subsec:GMM}

The Gaussian mixture model (GMM) is a probabilistic generative model that assumes the data points are generated from a mixture of Gaussian distributions.
% \soyeonstrike{
% It provides a convenient analytical form for representing data distributions that are potentially multimodal.
% Due to its flexibility of capturing an arbitrary level of complexity, GMM is widely used for many tasks including density estimation, statistical inference, and clustering.}
This model is capable of representing the multimodality and the uncertainty of complex data distributions \cite{kochenderfer2015decision}.
Additionally, GMMs are commonly used across broad domains to generate samples that capture representative characteristics of the training data \cite{liu2019improving, chokwitthaya2020applying, li2021gaussian}. %\todo{do  we need to cite these statements about GMMs, or okay to treat as common sense?}
These advantages align with our objectives of modeling the aircraft behavior and generating realistic trajectories. %\todo{not sure what this is missing that the reviewer was hoping for regarding justifying GMM}

For each flight stage, we construct a GMM with a set of training input vectors defined as (\ref{eq:single_model_training_data}).
If a single vector $\tau$ is sampled from $K$ mixture components, the marginal probability distribution of $\tau$ is 
% Consider a training set ${\mathbf{\tau}=[\tau^{(1)}, \ldots, \tau^{(m)}] \in \mathbb{R}^{m \times n}}$ where each data point ${\tau^{(i)} \in \mathbb{R}^{n}}$ is generated from a mixture of $K$ Gaussian distributions. The density function of $\tau^{(i)}$ is
\begin{align}
    \label{eq:gmm}
    p(\tau) = \sum_{j=1}^K \pi_j \mathcal{N}(\tau \mid \mu_j, \Sigma_j)
\end{align}
where ${\pi_j}$ are mixing coefficients that must satisfy ${ {\textstyle\sum}_{j=1}^{K} \pi_j=1}$ and ${\pi_j \geq 0}$ for all ${j \in \{1,\ldots,K\}}$.
Each Gaussian density ${\mathcal{N}(\tau  \mid \mu_j, \Sigma_j)}$ is called a component of the mixture.
The maximum likelihood estimates of the parameters ${\{\pi_j, \mu_j, \Sigma_j\}}$ for all $j$ given a dataset of the observations are obtained using the expectation-maximization (EM) algorithm \cite{dempster1977maximum}.

Our model learns the sequence of deviations (i.e., relative positions) of an aircraft from the corresponding points of the procedure. Fig. \ref{fig:gmm_deviations} shows an example sequence of deviations. 
The aircraft positions and the procedure points are indicated by black and blue crosses respectively, and the deviations are indicated by red dotted lines.
\begin{figure}[tb!]
\centering
\setlength\figureheight{4.8cm}
\setlength\figurewidth{6.3cm}
% This file was created by matplotlib2tikz v0.6.18.
\begin{tikzpicture}
\node[inner sep=0pt] (plot) at (0,0)
    {\includegraphics[height=4.8cm,width=6.3cm]{figures/gmm_dev.png}};

\node[text=blue, font=\fontsize{7pt}{9}\selectfont] at (-3.65,0.14) {\begin{math} ( x^p_6, y^p_6 ) \end{math}};
\node[text=blue, font=\fontsize{7pt}{9}\selectfont] at (-3.45,-0.7) {\begin{math} ( x^p_5, y^p_5 ) \end{math}};
\node[text=blue, font=\fontsize{7pt}{9}\selectfont] at (-1.55,-0.68) {\begin{math} ( x^p_4, y^p_4 ) \end{math}};
\node[text=blue, font=\fontsize{7pt}{9}\selectfont] at (-0.45,-0.68) {\begin{math} ( x^p_3, y^p_3 ) \end{math}};
\node[text=blue, font=\fontsize{7pt}{9}\selectfont] at (0.8,-0.6) {\begin{math} ( x^p_2, y^p_2 ) \end{math}};
\node[text=blue, font=\fontsize{7pt}{9}\selectfont] at (2.1,-0.28) {\begin{math} ( x^p_1, y^p_1 ) \end{math}};

\node[text=black, font=\fontsize{7pt}{9}\selectfont] at (-2.1,0.15) {\begin{math} ( x_6, y_6 ) \end{math}};
\node[text=black, font=\fontsize{7pt}{9}\selectfont] at (-1.96,-0.3) {\begin{math} ( x_5, y_5 ) \end{math}};
\node[text=black, font=\fontsize{7pt}{9}\selectfont] at (-2.46,-1.35) {\begin{math} ( x_4, y_4 ) \end{math}};
\node[text=black, font=\fontsize{7pt}{9}\selectfont] at (-1.6,-2.1) {\begin{math} ( x_3, y_3 ) \end{math}};
\node[text=black, font=\fontsize{8pt}{9}\selectfont] at (-0.19,-2.4) {\begin{math} ( x_2, y_2 ) \end{math}};
\node[text=black, font=\fontsize{8pt}{9}\selectfont] at (1.3,-2.5) {\begin{math} ( x_1, y_1 ) \end{math}};

\end{tikzpicture}
\caption{Example sequence of deviations between aircraft and procedural trajectory.}
\label{fig:gmm_deviations}
\end{figure}


%%%%%%%%%%%%%%%%%%%%%%%%%%%%%%%%%%%%%%%%%%%%%%%%%%%%%%%%%%%%%%%%%%%%
\subsection{Low-rank Approximation of Covariance Matrices}
\label{subsec:low-rank approximation}
Derived from the aircraft and procedural trajectories as in (\ref{eq:single_model_training_data}), the input data matrix of our model is likely to be high-dimensional and contain redundant features.
Also, the model can overfit the noise in the training set. %sparsity
To eliminate redundant features and reduce overfitting, we perform a low-rank approximation for each covariance matrix of our GMM in Section \ref{subsec:GMM} using eigenvalue decomposition. 

Consider the covariance of the $j$th Gaussian component ${\Sigma_j \in \mathbb{R}^{n \times n}}$.
The eigenvalue decomposition of ${\Sigma_j}$ is
\begin{align}
    \begin{split}
    \Sigma_j = Q\Lambda Q^{-1} &= Q\Lambda Q^T
    = \sum_{i=1}^n \lambda_i q_i q_i^T
    \end{split}
\end{align}
where ${\Lambda = \text{diag}(\lambda_1, \ldots, \lambda_n) \in \mathbb{R}^{n \times n}}$ is a diagonal matrix with eigenvalues in decreasing order, and ${Q \in \mathbb{R}^{n \times n}}$ is a matrix of the associated eigenvectors.

The best rank-${k}$ (${k \leq n}$) approximation of ${\Sigma_j}$ is obtained by 
\begin{align}
    \widehat{\Sigma}_j = Q_k \Lambda_k Q_k^{-1}
\end{align}
where ${\Lambda_k \in \mathbb{R}^{k \times k}}$ is a diagonal matrix with the largest $k$ eigenvalues, and the columns of ${Q_k \in \mathbb{R}^{n \times k}}$ are the first $k$ eigenvectors.
This is closely related to principal component analysis (PCA) where the $k$ principal axes, a set of orthonormal axes onto which the projection of the original data maximizes variance, are given by the first $k$ eigenvectors.
We can obtain equivalent results from performing a singular value decomposition of $\tau_j$, a set of data vectors assigned to the $j$th Gaussian component.

While the eigenvalue decomposition or PCA provides an analytical solution, both require the rank $k$ to be specified. 
To determine the optimal rank $k^*$ from the observed data rather than setting a specific value for $k$, we adopt the probabilistic principal component analysis (PPCA), a probabilistic formulation of PCA based on a latent variable model \cite{tipping1999mixtures}.

Consider a dataset ${x = \{x_i\}_{i=1}^m \in \mathbb{R}^{m \times n}}$ of $m$ observations.
PPCA assumes that each observation ${x_i \in \mathbb{R}^n}$ is generated from a low-dimensional latent variable ${z_i \in \mathbb{R}^k}$ (${k<n}$) via the following model. For ${i \in \{1,\ldots ,m\}}$, 
\begin{align}
    \label{eq:ppca}
    x_i &= W z_i+\mu+\varepsilon_i
\end{align}
where ${z_i \sim \mathcal{N}(0,I)}$ is a Gaussian latent variable with unit variance, 
${W \in \mathbb{R}^{n \times k}}$ is the weight matrix explaining the dependencies between latent and observed variables,
${\mu \in \mathbb{R}^n}$ is the location parameter that shifts the data,
and ${\varepsilon_i \sim \mathcal{N}(0, \sigma^2 I)}$ is an isotropic Gaussian noise unique to each observed variable.
From (\ref{eq:ppca}), we can compute the following conditional and marginal distributions:
\begin{align}
    \begin{split}
    x_i \mid z_i &\sim \mathcal{N}(W z_i+\mu, \sigma^2 I)\\
    x_i &\sim \mathcal{N}(\mu, WW^T + \sigma^2 I).
    \end{split}
\end{align}

The maximum likelihood estimate (MLE) of the parameters ${\{W, \sigma^2 \}}$ can be solved in closed form \cite{tipping1999mixtures} or using the EM algorithm \cite{roweis1998algorithms}.
The columns of the estimated $W$ define the principal subspace of standard PCA.

The optimal rank $k^*$ can be determined by choosing the latent dimension ${\mathbb{R}^{k}}$ that maximizes the marginal likelihood of the model.

% \begin{align}
%      \max_\theta{\log p(x;\theta)} &= \max_\theta{\sum_{i=1}^m \log p(x_i;\theta)}.
%     %  &= \max_\theta{\sum_{i=1}^m \log  {\int_{z_i} p(x_i \mid z_i;\theta) p(z_i;\theta) dz_i}}
% \end{align}
% \begin{align}
%     z_i \mid x_i &\sim \mathcal{N}(M^{-1} W^T (x_i-\mu), \sigma^2 M^{-1})\\
%     &\text{ where } M=W^T W + \sigma^2 I
% \end{align}

%%%%%%%%%%%%%%%%%%%%%%%%%%%%%%%%%%%%%%%%%%%%%%%%%%%%%%%%%%%%%%%%%%%%%%%%%%%%%%
\subsection{Trajectory Generation}
\label{subsec:trajs_generation}
Once we have the GMM parameters for each segment, we can generate synthetic trajectories of aircraft positions based on the trained models and test procedures.
To test the model on a set of procedures not used for the training, we need the procedural trajectories with the relative frequencies of procedures in each segment.

We start generating an aircraft trajectory with its radar vector segment.
First, we randomly select one of the test radar vector procedures with probability proportional to their relative frequencies.
Then, we sample a sequence of deviations ${\tau^v}$ from the radar vector GMM. Next, we construct a trajectory of aircraft positions using the sampled deviations and the test procedural trajectory.
While the total distance an aircraft travels varies with the procedure it follows, the reconstructed trajectory always has the same number of points.
To generate a trajectory with reasonable airspeed, we first compute the adjusted total transit time as ${t'^v = (\tau^v_1 / \tau^v_2) \times d'^v}$
% \begin{align}
%     t'^v = \frac{\tau^v_1}{\tau^v_2} \times d'^v
% \end{align}
where ${\tau^v_1, \tau^v_2}$ are the transit time and total distance of the sample, and ${d'^v}$ is the total distance of the given test procedure.
Then we align the trajectory with a vector of evenly spaced numbers over the interval ${[0, t'^v]}$.

For a smooth transition between two separately modeled segments of our generated trajectory,
we take the final $n$ positions of the reconstructed radar vector trajectory to compute the first $n$ deviations from the test final approach procedure.
Then, we form a conditional distribution of the final approach GMM to sample the remainder of the final approach segment given the first $n$ measurements.
% https://stats.stackexchange.com/questions/348941/general-conditional-distributions-for-multivariate-gaussian-mixtures

Suppose the input vector and the Gaussian components in (\ref{eq:gmm}) are partitioned as
\begin{align}
    % p(\tau) &= \sum_{j=1}^K \pi_j \mathcal{N}(\tau \mid \mu_j, \Sigma_j)\\
    p\left(
    \begin{bmatrix} 
        \tau_a \\ \tau_b 
    \end{bmatrix}\right) 
    &= \sum_{j=1}^K \pi_j \mathcal{N}\left(
    \begin{bmatrix} 
        \tau_a \\ \tau_b 
    \end{bmatrix} \;\middle|\;
    \begin{bmatrix}
        \mu_{j,a} \\ \mu_{j,b} 
    \end{bmatrix}, 
    \begin{bmatrix}
        \Sigma_{j,aa} \ \Sigma_{j,ab} \\
        \Sigma_{j,ba} \ \Sigma_{j,bb} 
    \end{bmatrix}\right).
\end{align}

Then, the conditional distribution of ${\tau_b}$ given ${\tau_a}$ is
\begin{align}
    \begin{split}  
    p(\tau_b \mid \tau_a) &=
    \sum_{j=1}^K \pi_{j,b \mid a}
    \mathcal{N}\left(\tau_b \mid \mu_{j, b \mid a}, \Sigma_{j, b \mid a}
    \right)\\
    \text{where } \ \pi_{j,b \mid a} &= \frac{\pi_j \mathcal{N}\left( \tau_a \mid \mu_{j,a}, \Sigma_{j,aa} \right)}{\sum_{k=1}^K \pi_k \mathcal{N}\left( \tau_a \mid \mu_{k,a}, \Sigma_{k,aa} \right)} \\
    \mu_{j, b \mid a} &= \mu_{j,b} + \Sigma_{j,ba}\Sigma_{j,aa}^{-1}(\tau_a - \mu_{j,a}) \\
    \Sigma_{j,b \mid a} &= \Sigma_{j,bb} - \Sigma_{j,ba} \Sigma_{j,aa}^{-1} \Sigma_{j,ba}^T.
    \end{split}
    \label{eq:conditional_density_GMM}
\end{align}

We form this conditional distribution for the final approach GMM by partitioning each vector $\tau^f$ as illustrated in Fig. \ref{fig:traj_generate_conditional}.
In our case, $\tau_a$ is defined as the first $n$ 3D coordinates of deviations and $\tau_b$ corresponds to the remainder of the vector.
In the figure, the blue crosses indicate the procedural points along the IAP. 
The set of dotted lines are the sequence of deviations $\tau^f$ partitioned into $\tau_a$ and $\tau_b$, and
the dots are the reconstructed trajectory of aircraft positions. 
Those marked red correspond to the conditioned part.

\begin{figure}[tb!]
\centering
% This file was created by matplotlib2tikz v0.6.18.
\begin{tikzpicture}

\node[inner sep=0pt] (plot) at (0,0)
    {\includegraphics[width=5.5cm, height=5.2cm]{figures/traj_generate_background.pdf}};
    
\node[mark size=3pt,color=gray] at (-1.1,-1.43) {\pgfuseplotmark{*}};
\node[mark size=3pt,color=gray] at (-0.72,-1.85) {\pgfuseplotmark{*}};

\node[mark size=3pt,color=red] at (-1.33,-0.96) {\pgfuseplotmark{*}};
\node[mark size=3pt,color=red] at (-1.42,-0.48) {\pgfuseplotmark{*}};

\node[mark size=3pt,color=black] at (-1.45,-0.02) {\pgfuseplotmark{*}};
\node[mark size=3pt,color=black] at (-1.35,0.4) {\pgfuseplotmark{*}};
\node[mark size=3pt,color=black] at (-1.2,0.75) {\pgfuseplotmark{*}};

\node[text=black, font=\fontsize{14pt}{9}\selectfont] at (-0.4,0.12) {\begin{math} \tau_b = \end{math}};

\node[text=red, font=\fontsize{14pt}{9}\selectfont] at (-0.4,-0.62) {\begin{math} \tau_a = \end{math}};

\node[text=black, font=\fontsize{12pt}{9}\selectfont] at (0.9,0.22) {\begin{math} \tau^f_{1:2,3n+3:} \end{math}};

\node[text=red, font=\fontsize{12pt}{9}\selectfont] at (0.75,-0.52) {\begin{math}  \tau^f_{3:3n+2} \end{math}};

\node[text=black, font=\fontsize{14pt}{9}\selectfont] at (2.6,-0.15) {\begin{math} \tau^f \end{math}};

\node[text=black, font=\fontsize{12pt}{9}\selectfont] at (-2.7,0.1) {\begin{math} n \end{math}};

\node[text=gray, align=right, font=\fontsize{9pt}{9}\selectfont, ] at (-2.35,-1.7) {\textbf{radar vector} \\ \textbf{segment}};

\node[text=blue, font=\fontsize{9pt}{9}\selectfont, ] at (-1.6,1.6) {\textbf{IAP}};

\end{tikzpicture}
\caption{Partition of the sequence of deviations for the final approach segment to form a conditional distribution.}
\label{fig:traj_generate_conditional}
\end{figure}

After we sample the remaining final approach segment of deviations from the conditional, we reconstruct an aircraft position trajectory as we need for the radar vector segment.
We also repeat the process for integrating time into the trajectory using $\tau^f_1$ and $\tau^f_2$.
Finally, the whole synthetic trajectory is obtained by combining the aircraft trajectories of both segments.

% For a smooth transition between two separately modeled segments of our generated trajectory, 
% we condition each Gaussian distribution component of the final approach GMM on partial observations computed with the final $n$ positions of the reconstructed radar vector trajectory.
% A sequence of deviations for the final approach segment $\tau^f$ and its associated Gaussian distribution can be partitioned as 
% \begin{align}
%     \begin{bmatrix} \tau_{3:3n+2}^f \\ \tau_{3n+3:}^f \end{bmatrix} =
%     \begin{bmatrix} \tau_{a} \\ \tau_{b} \end{bmatrix} &\sim
%     \mathcal{N}\left(\begin{bmatrix}
%                 \mu_a \\ \mu_b 
%             \end{bmatrix}, \begin{bmatrix}
%                     \Sigma_{aa} & \Sigma_{ab} \\
%                     \Sigma_{ba} & \Sigma_{bb} 
%                 \end{bmatrix}\right)
% \end{align}
% where $\tau_a$ corresponds to the first $n$ coordinates of deviations and $\tau_b$ is the remainder, as illustrated in Fig. \ref{fig:traj_generate_conditional}.
% Then, we can form the posterior distribution 
% \begin{align}
%     \begin{split}
%     p(\tau_{3n+3:}^f \mid \tau_{3:3n+2}^f) &= \mathcal{N}\left(\mu_{b \mid a}, \Sigma_{b \mid a}
%     \right)\\
%     \text{where } \ \mu_{b \mid a} &= \mu_b + \Sigma_{ba}\Sigma_{aa}^{-1}(\tau_a - \mu_a) \\
%     \Sigma_{b \mid a} &= \Sigma_{bb} - \Sigma_{ba} \Sigma_{aa}^{-1} \Sigma_{ba}^T.
%     \end{split}
%     \label{eq:conditional_density_GMM}
% \end{align}
% to sample the remainder of the final approach segment given the first $n$ measurements, obtained by computing the deviations between the last $n$ positions of the reconstructed radar vector trajectory and the first $n$ points of the final approach procedure.
% Then, the aircraft position trajectory is constructed using the sample and the remainder of the final approach procedure.

% Finally, the whole synthetic trajectory is obtained by combining the aircraft trajectories of both segments.


\section{Multiple Trajectory Model}
\label{sec:multiple trajectory model}
When multiple aircraft are within close proximity, their behavior will likely influence each other.
To capture correlations in behavior, we extend the single trajectory GMM described in Section \ref{sec:single trajectory model} to a pairwise trajectory GMM.
Then, we introduce a method to generate multiple trajectories based on the pairwise GMM.
This pairwise approach can scale to large sizes of trajectory set\textcolor{blue}{s} and efficiently generate an arbitrary number of trajectories from a single model, whereas fitting a GMM for the whole set of trajectories may lead to failure in EM due to singularities and quadratic increase in the covariance parameters.
To illustrate, we provide the steps to generate a set of three arrival trajectories.
To define the range of influence, we use the inter-arrival time between each pair of successive arrivals. 

\subsection{Pairwise Trajectory GMM}
For each combination of two procedures, a pairwise trajectory GMM is trained using the trajectory data of two aircraft following their corresponding procedures.
Each training input has the form
\begin{align}
\tau^{pair} = [\tau^{(1)}, \delta^{12}, \tau^{(2)}]
\end{align}
where $\delta^{12}$ is the inter-arrival time between the pair of aircraft, and
\begin{align*}
\tau^{(1)} = \ &[t^1, d^1, (x_1^1-x_1^{1p}), (y_1^1-y_1^{1p}), (z_1^1-z_1^{1p}),\\
& \ \ldots,(x_T^1-x_T^{1p}), (y_T^1-y_T^{1p}), (z_T^1-z_T^{1p})] \in \mathbb{R}^{3T + 2},\\
\tau^{(2)} = \ &[t^2, d^2, (x_1^2-x_1^{2p}), (y_1^2-y_1^{2p}), (z_1^2-z_1^{2p}),\\
& \ \ldots,(x_T^2-x_T^{2p}), (y_T^2-y_T^{2p}), (z_T^2-z_T^{2p})] \in \mathbb{R}^{3T + 2}
\end{align*}
are the sequences of deviations for the first and the second aircraft,
% \soyeonstrike{, $\tau^{(1)}$ and $\tau^{(2)}$, as defined in} 
each of which are in the form of
(\ref{eq:single_model_training_data}).


\subsection{Multiple Trajectory Generation}
\label{subsec:construction_of_cov_matrix}

% \soyeonstrike{To generate the trajectories of three aircraft within the range of influence, we construct a mean vector and a covariance matrix of }
% \begin{align}
%     \label{eq:input_multiple}
%     \tau^{set} = [\tau^{(1)}, \delta^{12}, \tau^{(2)}, \delta^{23}, \tau^{(3)}]
% \end{align}
% \soyeonstrike{to sample the sequence of deviations for three aircraft and their inter-arrival times from.
% \\
% The construction of $\mu^{set}$ and $\Sigma^{set}$, which represent the mean vector and the covariance matrix of $\tau^{set}$, is done according to the following steps:}
% \begin{enumerate}[1)]
% \setlength{\itemindent}{1em}
% \item \soyeonstrike{Construct $\mu^{1\&2}$ and $\Sigma_{1\&2}$ by sampling a Gaussian component from the trained pairwise trajectory model.}
% \item \soyeonstrike{Construct $\mu^{2\&3}$ and $\Sigma_{2\&3}$ by selecting the Gaussian component from the pairwise model that has the closest $\Sigma_{22}$ sub-block from the sampled one in step 1.}
% \item \soyeonstrike{Construct $\Sigma_{1\&3}$ by selecting the Gaussian component from the pairwise model that has the closest $\Sigma_{11}$ and $\Sigma_{33}$ sub-blocks from the sampled ones in the previous steps.}
% \end{enumerate}
% \soyeonstrike{The matrix $\Sigma^{set}$ is constructed from the sub-blocks, $\Sigma_{11}, \Sigma_{22}, \Sigma_{33}, \Sigma_{1\&2}, \Sigma_{2\&3}, \text{ and } \Sigma_{1\&3}$, as shown in Fig.} \ref{fig:cov_set_subblocks}.

\begin{figure}[tb!]
    \centering
    \includegraphics[width=3.6in]{figures/cov_matrix_3trajs.pdf}
    \caption{
        Sub-blocks of the covariance matrix $\Sigma_{set}$.
        % \soyeonstrike{Covariance matrix for $\tau^{set}$ of} (\ref{eq:input_multiple}).
        }
    \label{fig:cov_set_subblocks}
\end{figure}

\begin{figure*}[tb!]
    \centering
    \subfigure[Step 1. Construct $\Sigma_{1\&2}$ by sampling a Gaussian component from the trained pairwise trajectory model.]{
        \label{fig:cov_set_step1}
        \includegraphics[width=2in]{figures/cov_set_step1.pdf}}
    \hspace*{2em}%
    \subfigure[Step 2. Construct $\Sigma_{2\&3}$ by selecting the Gaussian component from the pairwise model that has the closest $\Sigma_{22}$ sub-block from the sampled one in step 1.]{
        \label{fig:cov_set_step2}
        \includegraphics[width=2in]{figures/cov_set_step2.pdf}}
    \hspace*{2em}%
    \subfigure[Step 3. Construct $\Sigma_{1\&3}$ by selecting the Gaussian component from the pairwise model that has the closest $\Sigma_{11}$ and $\Sigma_{33}$ sub-blocks from the sampled ones in the previous steps.]{
        \label{fig:cov_set_step3}
        \includegraphics[width=2in]{figures/cov_set_step3.pdf}}
    \caption{The process of constructing $\Sigma^{set}$, the covariance matrix of $\tau^{set}$.}
    \label{fig:cov_set_construction}
\end{figure*}

Using the mean vectors and the covariance matrices of the trained pairwise trajectory GMM, we can generate a set of three arrival trajectories that are within the range of influence.
The output vector is
\begin{align}
    \label{eq:input_multiple}
    \tau^{set} = [\tau^{(1)}, \delta^{12}, \tau^{(2)}, \delta^{23}, \tau^{(3)}]
\end{align}
where $\tau^{(1)}$, $\tau^{(2)}$, and $\tau^{(3)}$ are the sequences of deviations for the three aircraft indexed by the order of arrival time, 
and $\delta^{12}$ and $\delta^{23}$ are the inter-arrival times between the aircraft.


We sample $\tau^{set}$ from a Gaussian distribution parameterized by a mean vector $\mu^{set}$ and covariance matrix $\Sigma^{set}$.
To construct $\Sigma^{set}$, we partition the matrix and use its sub-blocks, $\Sigma_{11}, \Sigma_{22}, \Sigma_{33}, \Sigma_{1\&2}, \Sigma_{2\&3}, \text{ and } \Sigma_{1\&3}$. 
These are indicated in Fig. \ref{fig:cov_set_subblocks}.
Then, $\mu^{set}$ and $\Sigma^{set}$ are constructed as follows:
\begin{enumerate}
    \item Construct $\mu^{1\&2}$ and $\Sigma_{1\&2}$ by sampling a Gaussian component from the trained pairwise trajectory model.
    \item Construct $\mu^{2\&3}$ and $\Sigma_{2\&3}$ by selecting the Gaussian component from the pairwise model that has the closest $\Sigma_{22}$ sub-block from the sampled one in step 1.
    \item Construct $\Sigma_{1\&3}$ by selecting the Gaussian component from the pairwise model that has the closest $\Sigma_{11}$ and $\Sigma_{33}$ sub-blocks from the sampled ones in the previous steps.
\end{enumerate}
The corresponding process of constructing $\Sigma^{set}$ is also illustrated in Fig. \ref{fig:cov_set_construction}.


We can scale up to larger number of trajectories by extending this process.
To construct a diagonal pairwise sub-block $\Sigma_{k\&k+1}$, we repeat step 2) to select the Gaussian component that has the closest $\Sigma_{k-1\&k}$ sub-block that is already chosen.
For the other pairwise sub-blocks $\Sigma_{j\&k}$, we repeat step 3) to select the Gaussian component that has the closest $\Sigma_{jj}$ and $\Sigma_{kk}$ sub-blocks that are already constructed in the previous steps.


We present in section~\ref{ssec:faces} an application of PnP-HVAE on face images, using a pretrained state-of-the-art hierarchical VAE. 
Next, we study the application of our framework to natural images. To that end, we introduce  in section~\ref{ssec:patchVDVAE}  a patch hierachical VAE architecture, that is able to model natural images of different resolutions. In section~\ref{ssec:app_nat}, we provide deblurring, super-resolution and inpainting experiments to demonstrate the relevance of the proposed method.

Additional results are presented in Appendix~\ref{app:add}. All experiments can be reproduced using the code available at \url{https://github.com/jprost76/PnP-HVAE}.



\subsection{Face Image restoration (FFHQ)}\label{ssec:faces}
We first demonstrate the effectiveness of PnP-HVAE on highly structured data, by performing face image restoration.
Latent variable generative models can accurately model structured images such as face images \cite{karras2019style,vahdat2020nvae,child2021very,kingma2018glow}, and then be used to produce high quality restoration of such data. 
In our experiments, we use the VDVAE model of~\cite{child2021very}, pre-trained on the FFHQ dataset~\cite{karras2019style}, as our hierarchical VAE prior.
VDVAE has $L=66$ latent variable groups in its hierarchy and generates images at resolution $256\times256$.

We compare PnP-HVAE with the intermediate layer optimization algorithm (ILO)~\cite{daras2021intermediate} that is based on a different class of generative models than HVAE. ILO is a GAN inversion method which optimizes the image latent code along with the intermediate layer representation of a StyleGAN to generate an image consistent with a degraded observation.
We use the official implementation of ILO, along with a StyleGAN2 model~\cite{karras2020analyzing, stylegan2pytorch}, that was trained for 550k iterations on images of resolution $256\times256$ from FFHQ.  
As VDVAE and StyleGAN models are not trained on the same train-test split of FFHQ, we chose to evaluate the methods on a subset of 100 images from the CelebA dataset~\cite{liu2018large}. 
For super-resolution, the degradation model corresponds to the application of a gaussian low-pass filter followed by a $\times 4$ sub-sampling, and the addition of a gaussian white noise with $\sigma=3$.
For the deblurring, we considered motion blur and  gaussian kernels, both with a noise level $\sigma=8$. %

We provide quantitative comparisons in table~\ref{table:comp_ILO}, along with a visual comparison of the results in figure~\ref{fig:face_restoration}.
PnP-HVAE has the best  PSNR and SSIM results for all the considered restoration tasks, while ILO provides better results  for the perceptual distance.
By jointly optimizing the image and its latent variable, PnP-HVAE provides  results that are both realistic and consistent with the degraded observation.
On the other hand,  ILO  only optimizes on an extended latent space. This method generates  sharp and realistic images with better LPIPS scores,   
but the results lack  of consistency with respect to the observation, which explains the overall lower PSNR performance. 






\subsection{PatchVDVAE: a HVAE for natural images}\label{ssec:patchVDVAE}
Available generative models in the literature operate on images of  fixed resolutions and
are either restrained to datasets of limited diversity, or even to registered face images~\cite{kingma2018glow,child2021very, vahdat2020nvae, karras2019style}, or requiring additional class information~\cite{brock2018large, dhariwal2021diffusion, song2020score, luhman2022optimizing}.
Fitting an unconditional model on natural images appears to be a more difficult task, as their resolution can change, and their content is highly diverse.
The complexity of the problem can be reduced by learning a prior model on patches of reduced dimension. 
For image restoration problems, the patch model can be reused on images of higher dimensions~\cite{zoran2011learning,prost2021learning,altekruger2022patchnr}. When the model is a full CNN, the prior on the set of the  patches can  be computed efficiently by applying the network on the full image~\cite{prost2021learning}.

We thus introduce  patchVDVAE, a fully convolutional hierarchical VAE.
Contrary to existing HVAE models whose resolution is constrained by the constant tensor at the input of the top-down block, patchVDVAE can generate images of different resolutions by controlling the dimension of the input latent. 
This amounts to defining a prior on patches whose dimension corresponds to the receptive field of the VAE. A similar model is used for image denoising in~\cite{prakash2021interpretable}.

 
For PatchVDVAE architecture, we use the same bottom-up and top-down blocks as VDVAE~\cite{child2021very}, and replace the constant trainable input in the first top-down block by a latent variable, to make the model fully convolutional (details on the  architecture are given in Appendix~\ref{app:details}). 
The training dataset is composed of $128\times 128$ patches extracted from a combination of DIV2K~\cite{agustsson2017ntire} and Flickr2K~\cite{Lim_2017_CVPR_workshops} datasets.
We perform data augmentation by extracting  patches at $3$ resolutions: HR-images and $\times 2$ and $\times 4$ downscaled images. 
The model is trained for $7.10^5$ iterations with a batch size of $64$. Following the recommendation of~\cite{hazami2022efficient}, we use Adamax optimizer with an exponential moving average and gradient smoothing of the variance.
We set the decoder model to be a gaussian with diagonal covariance, as in~\cite{luhman2022optimizing}.
PatchVDVAE is fully convolutional and can generate images of dimension that are multiples of $64$ as illustrated by
figure~\ref{fig:vdvae}.

\newlength{\patchwidth}
\setlength{\patchwidth}{0.135\columnwidth}
\begin{figure}[!ht]
    \centering
    \begin{subfigure}[t]{.34\columnwidth}\hspace{0.1cm}
        \setlength{\tabcolsep}{0.02pt}
\renewcommand{\arraystretch}{0}
        \begin{tabular}{*{2}{p{1.03\patchwidth}}}
            \includegraphics[width=\patchwidth]{figures_arxiv/patchVDVAE/samples/generated/64x64/setup-5-image-0018.png} &
            \includegraphics[width=\patchwidth]{figures_arxiv/patchVDVAE/samples/generated/64x64/setup-5-image-0016.png} \\
            \includegraphics[width=\patchwidth]{figures_arxiv/patchVDVAE/samples/generated/64x64/setup-5-image-0008.png} &
            \includegraphics[width=\patchwidth]{figures_arxiv/patchVDVAE/samples/generated/64x64/setup-5-image-0019.png}   
        \end{tabular}
    \end{subfigure}\hspace{-0.15cm}
    \begin{subfigure}[t]{.64\columnwidth}
\begin{tabular}{cc}\vspace{-0.1cm}
\includegraphics[width=2\patchwidth]{figures_arxiv/patchVDVAE/samples/generated/256x256/setup-2-image-0009.png}&
        \includegraphics[width=2\patchwidth]{figures_arxiv/patchVDVAE/samples/generated/256x256/setup-2-image-0002.png}\end{tabular}

    \end{subfigure}
    \caption{\label{fig:vdvae} Left: $64\times64$ patches samples from our patchVDVAE model trained on patches from natural images.
    Right: PatchVDVAE is fully convolutional and it can generate images of higher resolution (here: $128\times128$).\vspace{-0.2cm}}
\end{figure}

\subsection{Natural images restoration}\label{ssec:app_nat}
We  evaluate PnP-HVAE on natural image restoration.
For each task, we report the average value of the PSNR, the SSIM, and the LPIPS metrics on $20$ images from the test set of the BSD dataset~\cite{MartinFTM01}.\\


\noindent
{\bf Image deblurring.}
In the experiments, we consider $2$ gaussian kernels and $2$ motion blur kernels from~\cite{levin2009understanding}, with $3$ different noise levels 
$\sigma \in \{2.55, 7.65, 12.75\}$.
As a baseline we consider  EPLL~\cite{zoran2011learning}, which learns a prior on image patches with a gaussian mixture model.
We also compare PnP-HVAE  with PnP-MMO and GS-PnP, $2$ competing convergent Plug-and-Play methods based on CNN denoisers.
PnP-MMO~\cite{pesquet2021learning} restricts the denoiser to be contraction in order to guarantee the convergence of the PnP forward-backard algorithm. GS-PnP~\cite{hurault2022gradient} considers a gradient step denoiser and reaches state-of-the-art performances of non converging methods~\cite{zhang2021plug}.
We set the temperature $\tau$  in our method as $0.95$, $0.8$ and $0.6$ for noise levels $2.55$, $7.65$ and $12.75$ respectively, and we let it run for a maximum of $50$ iterations. 
For the three compared methods we use the official implementations and pre-trained models provided by the respective authors. 
Details on the choice of hyperparameters for the concurrent methods are provided in the Appendix~\ref{app:details}
Figure~\ref{fig:deblurring_bsd} illustrates that our method provides correct deblurring results. 

According to table~\ref{tab:deb}, the performance of PnP-HVAE is between those of EPLL and GS-PnP and it outperforms PnP-MMO for large noise levels.\\

\begin{table}
\begin{center}\footnotesize
    \begin{tabular}{>{\centering}m{.3cm}*{5}{c}}
    $\sigma$ &Method & PSNR$\uparrow$ & SSIM$\uparrow$ & LPIPS$\downarrow$  \\ 
    \hline
    \multirow{4}{*}{\vcell{$2.55$}}
    & PnP-HVAE & $27.75$ & $0.79$ & $0.31$\\
    & GS-PNP \cite{hurault2022gradient} & $\mathbf{29.59}$ & $\mathbf{0.84}$ & $\mathbf{0.22}$\\
    & EPLL \cite{zoran2011learning} & $26.49$ & $0.71$ & $0.36$\\ 
    & PnP-MMO \cite{pesquet2021learning} & $\underbar{29.50}$ & $\underbar{0.83}$ & $\underbar{0.20}$ \\ \hline
    \multirow{4}{*}{\vcell{$7.65$}}
    & PnP-HVAE & $\underbar{26.36}$ & $\underbar{0.72}$ & $\underbar{0.40}$\\
    & GS-PNP \cite{hurault2022gradient} & $\mathbf{27.33}$ & $\mathbf{0.77}$ & $\mathbf{0.31}$\\
    & EPLL \cite{zoran2011learning} & $24.04$ & $0.66$ & $0.45$ \\ 
    & PnP-MMO \cite{pesquet2021learning} & $25.34$ & $0.69$ & $0.34$\\
    \hline
    \multirow{4}{*}{\vcell{$12.75$}}
    & PnP-HVAE & $\underbar{25.12}$ & $\mathbf{0.73}$ & $\underbar{0.47}$\\
    & GS-PNP \cite{hurault2022gradient} & $\mathbf{26.32}$ & $\mathbf{0.73}$ & $\mathbf{0.37}$\\
    & EPLL \cite{zoran2011learning} & $23.28$ & $0.61$ & $0.51$ \\ 
    & PnP-MMO \cite{pesquet2021learning} & $22.42$ & $0.53$& $0.54$ \\
    \hline
    &\vspace*{-.3cm}\\
            \multicolumn{2}{c}{Blur and motion kernels}& \multicolumn{3}{c}{
        \includegraphics*[scale=1]{figures_arxiv/kernels/4.png}\;\includegraphics*[scale=1]{figures_arxiv/kernels/7.png}\;\includegraphics*[scale=1]{figures_arxiv/kernels/9.png}\;\includegraphics*[scale=1]{figures_arxiv/kernels/11.png}} 
    \end{tabular}
        \caption{\label{tab:deb}Comparison  of PnP-HVAE  and other restoration methods on deblurring. Results are averaged on $4$ kernels.\vspace{-0.2cm}}% on image deblurring.}
    \end{center}
\end{table}

\begin{figure}
    
    \begin{subfigure}[h]{\linewidth}
        \centering
        \includegraphics*[width=\columnwidth]{figures_arxiv/deb_s255_k7.pdf}\vspace{-0.1cm}
        \caption{Gaussian blur, $\sigma=2.55$}
    \end{subfigure}
    \begin{subfigure}[h]{\linewidth}
        \centering
        \includegraphics*[width=\columnwidth]{figures_arxiv/deb_s765_k11.pdf}\vspace{-0.1cm}
        \caption{Motion blur, $\sigma=7.65$}
    \end{subfigure}\vspace*{-0.1cm}
    \caption{\label{fig:deblurring_bsd} Natural image deblurring\vspace{-0.1cm}}
\end{figure}

\noindent {\bf Effect of the temperature.}
PnP-HVAE gives control on the temperature of the prior over the latent space.
In figure~\ref{fig:temp_effect}, we illustrate that reducing the temperature increases the strength of the regularization prior. In this example the tuning $\tau=0.7$ produces the best performance.\\
\begin{figure}[!ht]
   
    \includegraphics[width=\columnwidth]{figures_arxiv/demo_temp.pdf}\vspace{-0.15cm}
    \caption{ \label{fig:temp_effect} Effect of the temperature in PnP-VAE on a deblurring problem, with $\sigma=7.65$.\vspace{-0.15cm}}
\end{figure}


\noindent
{\bf Image inpainting.}
Next we consider the task of noisy image inpainting. 
We compose a test-set of 10 images from the validation set of BSD~\cite{MartinFTM01} and we create masks
  by occluding diverse objects of small size in the images. 
A gaussian white noise with $\sigma=3$ is added to the images.
As a comparaison, we still consider GS-PnP and EPLL.
For PnP-HVAE, the temperature is set to $\tau=0.6$, and the algorithm is run for a maximum of $200$ iterations, unless the residual $||\x_{k+1}-\x_k||$ is on a plateau.
We provide on Table~\ref{tab:inpainting_bsd} the distortion metrics with the ground truth, as well as a visual
\begin{table}



\begin{center}
    \begin{tabular}{cccc}
        & PSNR$\uparrow$ & SSIM$\uparrow$ &LPIPS$\downarrow$ \\\hline
        PnP-HVAE  & $\mathbf{29.54}$ & $\mathbf{0.93}$ & $\mathbf{0.06}$\\
        GS-PNP & $28.52$ & $\mathbf{0.93}$ & $0.09$\\
        EPLL & $\underline{29.16}$ & $\mathbf{0.93}$ & $\mathbf{0.06}$\\
    \end{tabular}
    \caption{\label{tab:inpainting_bsd}Quantitative evaluation for inpainting on BSD.}
    \end{center}
\end{table}
comparison on figure~\ref{fig:inpainting_bsd}. 
With its hierarchical structure,  PnP-HVAE outperforms the compared methods. \vspace{0.05cm}



\begin{figure}[!h]
    \includegraphics[width=\columnwidth]{figures_arxiv/demo_inp_bsd2.pdf}\vspace{-0.1cm}
    \caption{\label{fig:inpainting_bsd}Natural image inpainting\vspace{-0.3cm}}
\end{figure}











\section{Conclusions}
We consider the phase-extraction problem, and we showed that, given a unitary $U = e^{i\pi H}$ and its inverse $U^{\dag}$, we could implement a block-encoding of $\phi(H)$ for some smooth function $\phi(x)$. The word `smooth' here means existence and continuity of the derivatives: the higher the number of continuous derivatives that a function has, the faster its Fourier sum (and thus the Laurent polynomial on the eigenphases) uniformly converges to that function. We are confident this can have many more applications beyond what is shown in this work. It is also worth remarking that Jackson showed that the convergence rate of a Fourier series is almost-optimal, in the sense that no trigonometric (or, equivalently, complex exponential) series can approximate the desired function faster, up to that $\log d$ factor~\cite[p.\ 21]{jacksonTheoryApproximation1930a}. Also remember that `smoothing' a function, i.e., replacing its derivative with a continuous function, does not give faster convergence for free in general, as its derivative will become steep in the points where we smooth out discontinuities, and this translates to a high Lipschitz constant: a~clear example is given by Eq.~\ref{eq:lipschitz-constant-recurrence-solution}, but in that case, fortunately, nothing depends on the size of the input $N$, and thus does not influence the asymptotic query complexity of Algorithm~\ref{alg:prop-sampling-qsp}, although the constant factor can become large even for $p = 20$. From a theoretical point of view, this work shows that, for any $\eta > 0$, there is an algorithm with query complexity 
$$\Tilde{\bigO}\left(\frac{1}{\bar{c}^{\frac{1}{2} + \eta}} \frac{1}{\epsilon^\eta} \right)$$
solving the proportional-sampling problem. This statement seems to suggest there exists an algorithm which directly solves the problem with $\eta = 0$, and an open question would be to find such algorithm.


It is also interesting to remark that Theorems~\ref{thm:haah-construction},~\ref{thm:haah-completion} indeed allow the construction for any $\phi$, even complex-valued, provided that its absolute value is reciprocal.

One could think that, in Section~\ref{sec:prop-sampling}, instead of using the linear function in the phase-extraction subroutine, we could approximate the square root and then apply the transformation directly on $e^{i \pi c(x)}$. However, in the case of proportional sampling this would be inconvenient, as the derivative of the square root function has a discontinuity with an infinite jump around 0, and we could not choose a constant $\delta$ if we had values of the oracle that are too close to $0$.

\section*{Acknowledgments}
This material is based upon work supported by the Department of the Air Force under Air Force Contract FA8702-15-D-0001.
Opinions, findings, conclusions and recommendations are those of the authors and do not necessarily reflect the views of the Department of the Air Force.
The authors acknowledge the FAA and the MITRE Corporation for sharing the surveillance data.
They thank Evan Maki, Randal Guendel, and Mikhail Krichman from MIT Lincoln Laboratory for their support and assistance.
This article benefited from the work of Shane Barratt, and conversations with Rachael Tompa and Kyle Julian.

\printbibliography

\end{document}
