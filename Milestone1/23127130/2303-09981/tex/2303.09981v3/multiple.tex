\section{Multiple Trajectory Model}
\label{sec:multiple trajectory model}
When multiple aircraft are within close proximity, their behavior will likely influence each other.
To capture correlations in behavior, we extend the single trajectory GMM described in Section \ref{sec:single trajectory model} to a pairwise trajectory GMM.
Then, we introduce a method to generate multiple trajectories based on the pairwise GMM.
This pairwise approach can scale to large sizes of trajectory set\textcolor{blue}{s} and efficiently generate an arbitrary number of trajectories from a single model, whereas fitting a GMM for the whole set of trajectories may lead to failure in EM due to singularities and quadratic increase in the covariance parameters.
To illustrate, we provide the steps to generate a set of three arrival trajectories.
To define the range of influence, we use the inter-arrival time between each pair of successive arrivals. 

\subsection{Pairwise Trajectory GMM}
For each combination of two procedures, a pairwise trajectory GMM is trained using the trajectory data of two aircraft following their corresponding procedures.
Each training input has the form
\begin{align}
\tau^{pair} = [\tau^{(1)}, \delta^{12}, \tau^{(2)}]
\end{align}
where $\delta^{12}$ is the inter-arrival time between the pair of aircraft, and
\begin{align*}
\tau^{(1)} = \ &[t^1, d^1, (x_1^1-x_1^{1p}), (y_1^1-y_1^{1p}), (z_1^1-z_1^{1p}),\\
& \ \ldots,(x_T^1-x_T^{1p}), (y_T^1-y_T^{1p}), (z_T^1-z_T^{1p})] \in \mathbb{R}^{3T + 2},\\
\tau^{(2)} = \ &[t^2, d^2, (x_1^2-x_1^{2p}), (y_1^2-y_1^{2p}), (z_1^2-z_1^{2p}),\\
& \ \ldots,(x_T^2-x_T^{2p}), (y_T^2-y_T^{2p}), (z_T^2-z_T^{2p})] \in \mathbb{R}^{3T + 2}
\end{align*}
are the sequences of deviations for the first and the second aircraft,
% \soyeonstrike{, $\tau^{(1)}$ and $\tau^{(2)}$, as defined in} 
each of which are in the form of
(\ref{eq:single_model_training_data}).


\subsection{Multiple Trajectory Generation}
\label{subsec:construction_of_cov_matrix}

% \soyeonstrike{To generate the trajectories of three aircraft within the range of influence, we construct a mean vector and a covariance matrix of }
% \begin{align}
%     \label{eq:input_multiple}
%     \tau^{set} = [\tau^{(1)}, \delta^{12}, \tau^{(2)}, \delta^{23}, \tau^{(3)}]
% \end{align}
% \soyeonstrike{to sample the sequence of deviations for three aircraft and their inter-arrival times from.
% \\
% The construction of $\mu^{set}$ and $\Sigma^{set}$, which represent the mean vector and the covariance matrix of $\tau^{set}$, is done according to the following steps:}
% \begin{enumerate}[1)]
% \setlength{\itemindent}{1em}
% \item \soyeonstrike{Construct $\mu^{1\&2}$ and $\Sigma_{1\&2}$ by sampling a Gaussian component from the trained pairwise trajectory model.}
% \item \soyeonstrike{Construct $\mu^{2\&3}$ and $\Sigma_{2\&3}$ by selecting the Gaussian component from the pairwise model that has the closest $\Sigma_{22}$ sub-block from the sampled one in step 1.}
% \item \soyeonstrike{Construct $\Sigma_{1\&3}$ by selecting the Gaussian component from the pairwise model that has the closest $\Sigma_{11}$ and $\Sigma_{33}$ sub-blocks from the sampled ones in the previous steps.}
% \end{enumerate}
% \soyeonstrike{The matrix $\Sigma^{set}$ is constructed from the sub-blocks, $\Sigma_{11}, \Sigma_{22}, \Sigma_{33}, \Sigma_{1\&2}, \Sigma_{2\&3}, \text{ and } \Sigma_{1\&3}$, as shown in Fig.} \ref{fig:cov_set_subblocks}.

\begin{figure}[tb!]
    \centering
    \includegraphics[width=3.6in]{figures/cov_matrix_3trajs.pdf}
    \caption{
        Sub-blocks of the covariance matrix $\Sigma_{set}$.
        % \soyeonstrike{Covariance matrix for $\tau^{set}$ of} (\ref{eq:input_multiple}).
        }
    \label{fig:cov_set_subblocks}
\end{figure}

\begin{figure*}[tb!]
    \centering
    \subfigure[Step 1. Construct $\Sigma_{1\&2}$ by sampling a Gaussian component from the trained pairwise trajectory model.]{
        \label{fig:cov_set_step1}
        \includegraphics[width=2in]{figures/cov_set_step1.pdf}}
    \hspace*{2em}%
    \subfigure[Step 2. Construct $\Sigma_{2\&3}$ by selecting the Gaussian component from the pairwise model that has the closest $\Sigma_{22}$ sub-block from the sampled one in step 1.]{
        \label{fig:cov_set_step2}
        \includegraphics[width=2in]{figures/cov_set_step2.pdf}}
    \hspace*{2em}%
    \subfigure[Step 3. Construct $\Sigma_{1\&3}$ by selecting the Gaussian component from the pairwise model that has the closest $\Sigma_{11}$ and $\Sigma_{33}$ sub-blocks from the sampled ones in the previous steps.]{
        \label{fig:cov_set_step3}
        \includegraphics[width=2in]{figures/cov_set_step3.pdf}}
    \caption{The process of constructing $\Sigma^{set}$, the covariance matrix of $\tau^{set}$.}
    \label{fig:cov_set_construction}
\end{figure*}

Using the mean vectors and the covariance matrices of the trained pairwise trajectory GMM, we can generate a set of three arrival trajectories that are within the range of influence.
The output vector is
\begin{align}
    \label{eq:input_multiple}
    \tau^{set} = [\tau^{(1)}, \delta^{12}, \tau^{(2)}, \delta^{23}, \tau^{(3)}]
\end{align}
where $\tau^{(1)}$, $\tau^{(2)}$, and $\tau^{(3)}$ are the sequences of deviations for the three aircraft indexed by the order of arrival time, 
and $\delta^{12}$ and $\delta^{23}$ are the inter-arrival times between the aircraft.


We sample $\tau^{set}$ from a Gaussian distribution parameterized by a mean vector $\mu^{set}$ and covariance matrix $\Sigma^{set}$.
To construct $\Sigma^{set}$, we partition the matrix and use its sub-blocks, $\Sigma_{11}, \Sigma_{22}, \Sigma_{33}, \Sigma_{1\&2}, \Sigma_{2\&3}, \text{ and } \Sigma_{1\&3}$. 
These are indicated in Fig. \ref{fig:cov_set_subblocks}.
Then, $\mu^{set}$ and $\Sigma^{set}$ are constructed as follows:
\begin{enumerate}
    \item Construct $\mu^{1\&2}$ and $\Sigma_{1\&2}$ by sampling a Gaussian component from the trained pairwise trajectory model.
    \item Construct $\mu^{2\&3}$ and $\Sigma_{2\&3}$ by selecting the Gaussian component from the pairwise model that has the closest $\Sigma_{22}$ sub-block from the sampled one in step 1.
    \item Construct $\Sigma_{1\&3}$ by selecting the Gaussian component from the pairwise model that has the closest $\Sigma_{11}$ and $\Sigma_{33}$ sub-blocks from the sampled ones in the previous steps.
\end{enumerate}
The corresponding process of constructing $\Sigma^{set}$ is also illustrated in Fig. \ref{fig:cov_set_construction}.


We can scale up to larger number of trajectories by extending this process.
To construct a diagonal pairwise sub-block $\Sigma_{k\&k+1}$, we repeat step 2) to select the Gaussian component that has the closest $\Sigma_{k-1\&k}$ sub-block that is already chosen.
For the other pairwise sub-blocks $\Sigma_{j\&k}$, we repeat step 3) to select the Gaussian component that has the closest $\Sigma_{jj}$ and $\Sigma_{kk}$ sub-blocks that are already constructed in the previous steps.

