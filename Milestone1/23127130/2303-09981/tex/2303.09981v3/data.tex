\section{Trajectory and Procedure Data}
\label{sec:dataset}
\subsection{Trajectory Data}
In this paper, we use a trajectory dataset that is generated using the Federal Aviation Administration (FAA) multisensor fusion tracker \cite{jagodnik2008fusion}.
This tracker collects radar detections from multiple sensors (primary radar, mode A/C and S transponders, wide area multilateration systems, and ADS-B receivers) and fuses them into a set of flight tracks.
Each track contains timestamped entries of the target address (used to group entries into flights), aircraft ID (if available), latitude and longitude (in WGS84), pressure altitude, geometric altitude (if available), and horizontal and vertical velocities.
The data were collected for six months starting March 2012 from three locations: Central Florida, New York City, and Southern California.
We train our models using data from %\todo{is JFK the only data used?} 
the vicinity of John F. Kennedy International Airport (KJFK), located within the New York Class B airspace that includes Newark Liberty International Airport (KEWR) and LaGuardia Airport (KLGA).

We process the track data as follows. First, the latitude-longitude-altitude coordinates of each position measurement are transformed to the local east-north-up (ENU) coordinates centered at the airport.
Next, we extract trajectories that enter into the airspace defined as 25 NM from the airport center. Based on the change in distance to the airport and altitude over time, the trajectories are sorted into arrivals, departures, and overflights
Finally, we separate the trajectories based on the runways: 04L/22R, 04R/22L, 13L/31R, and 13R/31L.
Fig. \ref{fig:jfk_arrivals_hist} shows the lateral view of a log-histogram of all arrival flight tracks to KJFK.
The origin indicates the center of the airport and the blue lines indicate the runways in use.
\begin{figure}[bt]
\centering
\setlength\figureheight{7cm}
\setlength\figurewidth{7cm}
% This file was created by matplotlib2tikz v0.6.18.
\begin{tikzpicture}

\begin{axis}[
colorbar,
colorbar style={width=7,
ytick={0,1,2,3,4},
yticklabel={\normalsize $10^{\pgfmathprintnumber{\tick}}$}},
colormap = {whiteblack}{color(0cm)=(white); color(1cm)=(black)},
height=\figureheight,
point meta max=4.5,
point meta min=0,
tick align=outside,
tick pos=left,
width=\figurewidth,
x grid style={black},
xlabel={East (NM)},
xmin=-20, xmax=25,
y grid style={black},
ylabel={North (NM)},
ymin=-25, ymax=20,
tick label style={font=\tiny},
label style={font=\scriptsize}
]

\addplot graphics [includegraphics cmd=\pgfimage,xmin=-30, xmax=30, ymin=-30, ymax=30] {figures/jfk_all_original_hist_fig.png};

\addplot [draw=none, draw=black, fill=black, colormap/viridis]
table [row sep=\\]{%
x                      y\\ 
+0.000000000000000e+00 -2.500000000000000e-01\\ 
+6.630077500000001e-02 -2.500000000000000e-01\\ 
+1.298949676962134e-01 -2.236584228970604e-01\\ 
+1.767766952966369e-01 -1.767766952966369e-01\\ 
+2.236584228970604e-01 -1.298949676962134e-01\\ 
+2.500000000000000e-01 -6.630077500000001e-02\\ 
+2.500000000000000e-01 +0.000000000000000e+00\\ 
+2.500000000000000e-01 +6.630077500000001e-02\\ 
+2.236584228970604e-01 +1.298949676962134e-01\\ 
+1.767766952966369e-01 +1.767766952966369e-01\\ 
+1.298949676962134e-01 +2.236584228970604e-01\\ 
+6.630077500000001e-02 +2.500000000000000e-01\\ 
+0.000000000000000e+00 +2.500000000000000e-01\\ 
-6.630077500000001e-02 +2.500000000000000e-01\\ 
-1.298949676962134e-01 +2.236584228970604e-01\\ 
-1.767766952966369e-01 +1.767766952966369e-01\\ 
-2.236584228970604e-01 +1.298949676962134e-01\\ 
-2.500000000000000e-01 +6.630077500000001e-02\\ 
-2.500000000000000e-01 +0.000000000000000e+00\\ 
-2.500000000000000e-01 -6.630077500000001e-02\\ 
-2.236584228970604e-01 -1.298949676962134e-01\\ 
-1.767766952966369e-01 -1.767766952966369e-01\\ 
-1.298949676962134e-01 -2.236584228970604e-01\\ 
-6.630077500000001e-02 -2.500000000000000e-01\\ 
+0.000000000000000e+00 -2.500000000000000e-01\\ 
+0.000000000000000e+00 -2.500000000000000e-01\\
};

\addplot [blue, line width=0.75pt, forget plot]
table [row sep=\\]{%
-0.314790350956774	-1.07388212440464 \\
0.701926179643114	0.634446106243 \\
};
\addplot [blue, line width=0.75pt, forget plot]
table [row sep=\\]{%
0.381018579296101	-0.869410987002108 \\
1.0883680175494	0.318538875436125 \\
};
\addplot [blue, line width=0.75pt, forget plot]
table [row sep=\\]{%
-0.527871725321972	1.06913560548559 \\
0.886517107609026	0.227348929251132 \\
};
\addplot [blue, line width=0.75pt, forget plot]
table [row sep=\\]{%
-1.73637529142706	0.505844290966593 \\
0.315676001958449	-0.715917614289667 \\
};
\end{axis}

\end{tikzpicture}
\caption{Log-histogram of all actual arrival tracks of KJFK.}
\label{fig:jfk_arrivals_hist}
\end{figure}

\subsection{Flight Procedure Data}
To model flight procedures, we use the standard flight procedures published by the air navigation service providers (e.g., the FAA). 
The procedures are defined as a set of waypoints, which are fixed points in 2D space (latitude and longitude). Each aircraft is instructed to fly from one waypoint to the next along the procedure.
Arrival traffic involves two types of standard flight procedures: standard terminal arrival routes (STARs) and instrument approach procedures (IAPs).
A STAR connects the end of an airway from the en-route airspace to the vicinity of the airport.
An IAP is a sequence of predetermined maneuvers used for the final approach. It allows the pilots to align the aircraft with the runway, make their final descent, and land safely in low visibility conditions. 
For the integration of multiple traffic approaches and the smooth transition of each aircraft from a STAR to an IAP, air traffic controllers (ATC) often provide radar vectoring (i.e., guide aircraft by assigning headings, altitudes, and speeds).

For KJFK, only the final approach and radar vectoring stages fall in the range of 25 NM from the airport.
Since there are no standard procedures published for radar vectoring, we take the nominal paths extracted from the actual trajectory set as the radar vector procedures.
To extract the nominal paths, we cluster the trajectory set using $k$-means and then select the ones representative of the flight patterns.
Fig. \ref{fig:jfk_04R_orginal_paths} shows the arrival traffic to runway 04R of KJFK and the associated procedures overlaid on them.
The radar vector procedures are indicated with blue dotted lines, while the IAP is indicated with a white dashed line with orange edges.
\begin{figure}[bt]
\centering
\setlength\figureheight{7cm}
\setlength\figurewidth{7cm}
% This file was created by matplotlib2tikz v0.6.18.
\begin{tikzpicture}
\definecolor{color0}{rgb}{0.392156862745098,0.584313725490196,0.929411764705882}
\definecolor{color1}{RGB}{255,165,0}

\begin{axis}[
colorbar,
colorbar style={width=7,
ytick={0,1,2,3,4},
yticklabel={$10^{\pgfmathprintnumber{\tick}}$}},
colormap = {whiteblack}{color(0cm)=(white); color(1cm)=(black)},
height=\figureheight,
point meta max=4.25,
point meta min=0,
tick align=outside,
tick pos=left,
width=\figurewidth,
x grid style={black},
xlabel={East (NM)},
xmin=-20, xmax=25,
y grid style={black},
ylabel={North (NM)},
ymin=-25, ymax=20,
tick label style={font=\tiny},
label style={font=\scriptsize}
]

\addplot graphics [includegraphics cmd=\pgfimage,xmin=-30, xmax=30, ymin=-30, ymax=30] {figures/jfk_04R_original_hist.png};

\addplot [draw=none, draw=black, fill=black, colormap/viridis]
table [row sep=\\]{%
x                      y\\ 
+0.000000000000000e+00 -2.500000000000000e-01\\ 
+6.630077500000001e-02 -2.500000000000000e-01\\ 
+1.298949676962134e-01 -2.236584228970604e-01\\ 
+1.767766952966369e-01 -1.767766952966369e-01\\ 
+2.236584228970604e-01 -1.298949676962134e-01\\ 
+2.500000000000000e-01 -6.630077500000001e-02\\ 
+2.500000000000000e-01 +0.000000000000000e+00\\ 
+2.500000000000000e-01 +6.630077500000001e-02\\ 
+2.236584228970604e-01 +1.298949676962134e-01\\ 
+1.767766952966369e-01 +1.767766952966369e-01\\ 
+1.298949676962134e-01 +2.236584228970604e-01\\ 
+6.630077500000001e-02 +2.500000000000000e-01\\ 
+0.000000000000000e+00 +2.500000000000000e-01\\ 
-6.630077500000001e-02 +2.500000000000000e-01\\ 
-1.298949676962134e-01 +2.236584228970604e-01\\ 
-1.767766952966369e-01 +1.767766952966369e-01\\ 
-2.236584228970604e-01 +1.298949676962134e-01\\ 
-2.500000000000000e-01 +6.630077500000001e-02\\ 
-2.500000000000000e-01 +0.000000000000000e+00\\ 
-2.500000000000000e-01 -6.630077500000001e-02\\ 
-2.236584228970604e-01 -1.298949676962134e-01\\ 
-1.767766952966369e-01 -1.767766952966369e-01\\ 
-1.298949676962134e-01 -2.236584228970604e-01\\ 
-6.630077500000001e-02 -2.500000000000000e-01\\ 
+0.000000000000000e+00 -2.500000000000000e-01\\ 
+0.000000000000000e+00 -2.500000000000000e-01\\
};

\addplot [black, line width=0.75pt, forget plot]
table [row sep=\\]{%
-0.314790350956774	-1.07388212440464 \\
0.701926179643114	0.634446106243 \\
};
\addplot [black, line width=0.75pt, forget plot]
table [row sep=\\]{%
0.381018579296101	-0.869410987002108 \\
1.0883680175494	0.318538875436125 \\
};
\addplot [black, line width=0.75pt, forget plot]
table [row sep=\\]{%
-0.527871725321972	1.06913560548559 \\
0.886517107609026	0.227348929251132 \\
};
\addplot [black, line width=0.75pt, forget plot]
table [row sep=\\]{%
-1.73637529142706	0.505844290966593 \\
0.315676001958449	-0.715917614289667 \\
};

\node[text=black, text width=10cm, anchor=west, right,  font=\fontsize{8pt}{9}\selectfont] at (-18,-3) {\textbf{instrument} \\ \textbf{approach} \\ \textbf{procedure}};

\node[text=black, font=\fontsize{8pt}{9}\selectfont, text width=10cm, anchor=west] at (6,-20) {\textbf{radar vector} \\ \textbf{procedures}};

\addplot [line width=1.2pt, blue, dotted, forget plot]
table [row sep=\\]{%
-0.69592398683708	-28.7389082836494 \\
-0.696280793942828	-28.6746878257532 \\
-0.697331127675969	-28.6106074159044 \\
-0.699044856660347	-28.5466675608247 \\
-0.701391849519806	-28.4828687672361 \\
-0.704341974878191	-28.4192115418603 \\
-0.707865101359345	-28.3556963914193 \\
-0.711931097587113	-28.2923238226348 \\
-0.71650983218534	-28.2290943422287 \\
-0.721571173777869	-28.1660084569229 \\
-0.727084990988545	-28.1030666734393 \\
-0.733021152441212	-28.0402694984996 \\
-0.739349526759715	-27.9776174388258 \\
-0.746039982567897	-27.9151110011396 \\
-0.753062388489604	-27.852750692163 \\
-0.760386613148679	-27.7905370186177 \\
-0.767982525168966	-27.7284704872257 \\
-0.77581999317431	-27.6665516047088 \\
-0.783868885788556	-27.6047808777888 \\
-0.792099071635546	-27.5431588131876 \\
-0.800480419339127	-27.481685917627 \\
-0.808982797523142	-27.420362697829 \\
-0.817576074811434	-27.3591896605152 \\
-0.82623011982785	-27.2981673124077 \\
-0.835200556863182	-27.2372815786656 \\
-0.845369876465388	-27.1764886306106 \\
-0.856745288108885	-27.1157918054922 \\
-0.86926191899161	-27.0551982081376 \\
-0.882854896311503	-26.9947149433742 \\
-0.897459347266505	-26.9343491160292 \\
-0.913010399054552	-26.8741078309299 \\
-0.929443178873586	-26.8139981929035 \\
-0.946692813921546	-26.7540273067773 \\
-0.96469443139637	-26.6942022773785 \\
-0.983383158495997	-26.6345302095346 \\
-1.00269412241837	-26.5750182080726 \\
-1.02256245036142	-26.51567337782 \\
-1.0429232695231	-26.4565028236039 \\
-1.06371170710133	-26.3975136502517 \\
-1.08486289029407	-26.3387129625905 \\
-1.10631194629924	-26.2801078654478 \\
-1.1279940023148	-26.2217054636506 \\
-1.14984418553867	-26.1635128620264 \\
-1.1717976231688	-26.1055371654024 \\
-1.19378944240312	-26.0477854786058 \\
-1.21575477043958	-25.990264906464 \\
-1.23762873447612	-25.9329825538042 \\
-1.25940109423679	-25.8759446655842 \\
-1.28172395558797	-25.8191429391627 \\
-1.30477208779715	-25.7625678237443 \\
-1.32850341898455	-25.7062121507194 \\
-1.35287587727037	-25.6500687514784 \\
-1.37784739077484	-25.5941304574117 \\
-1.40337588761816	-25.5383900999099 \\
-1.42941929592054	-25.4828405103634 \\
-1.4559355438022	-25.4274745201626 \\
-1.48288255938336	-25.372284960698 \\
-1.51021827078422	-25.31726466336 \\
-1.53790060612499	-25.2624064595391 \\
-1.56588749352589	-25.2077031806258 \\
-1.59413686110714	-25.1531476580104 \\
-1.62260663698893	-25.0987327230835 \\
-1.65125474929149	-25.0444512072354 \\
-1.68003912613503	-24.9902959418567 \\
-1.70891769563976	-24.9362597583378 \\
-1.73784838592589	-24.8823354880691 \\
-1.76678912511364	-24.8285159624411 \\
-1.79569784132321	-24.7747940128443 \\
-1.82453246267483	-24.7211624706691 \\
-1.85325091728869	-24.6676141673059 \\
-1.88181113328503	-24.6141419341452 \\
-1.91041323841592	-24.5607995912495 \\
-1.93927390859137	-24.5076394451815 \\
-1.96837589416848	-24.4546520570818 \\
-1.99770194550431	-24.4018279880914 \\
-2.02723481295596	-24.3491577993508 \\
-2.05695724688051	-24.2966320520009 \\
-2.08685199763505	-24.2442413071823 \\
-2.11690181557665	-24.1919761260358 \\
-2.14708945106241	-24.1398270697022 \\
-2.1773976544494	-24.0877846993223 \\
-2.2078091760947	-24.0358395760366 \\
-2.23830676635542	-23.983982260986 \\
-2.26887317558861	-23.9322033153113 \\
-2.29949115415138	-23.8804933001531 \\
-2.33014345240081	-23.8288427766522 \\
-2.36081282069397	-23.7772423059494 \\
-2.39148200938795	-23.7256824491854 \\
-2.42213376883985	-23.6741537675009 \\
-2.45275084940673	-23.6226468220366 \\
-2.48331600144568	-23.5711521739334 \\
-2.51381197531379	-23.5196603843319 \\
-2.54422152136814	-23.468162014373 \\
-2.57452738996582	-23.4166476251972 \\
-2.60475788940071	-23.3651289791374 \\
-2.63504771571502	-23.313668637429 \\
-2.66539923941799	-23.2622690254981 \\
-2.69580359683025	-23.2109273923218 \\
-2.72625192427247	-23.1596409868774 \\
-2.75673535806527	-23.1084070581419 \\
-2.78724503452932	-23.0572228550924 \\
-2.81777208998525	-23.0060856267061 \\
-2.84830766075371	-22.9549926219601 \\
-2.87884288315536	-22.9039410898313 \\
-2.90936889351083	-22.8529282792971 \\
-2.93987682814077	-22.8019514393345 \\
-2.97035782336584	-22.7510078189206 \\
-3.00080301550667	-22.7000946670326 \\
-3.03120354088392	-22.6492092326475 \\
-3.06155053581822	-22.5983487647424 \\
-3.09183513663024	-22.5475105122945 \\
-3.1220484796406	-22.4966917242809 \\
-3.15218170116997	-22.4458896496787 \\
-3.18222593753899	-22.395101537465 \\
-3.2121723250683	-22.3443246366169 \\
-3.24201200007855	-22.2935561961116 \\
-3.27173609889039	-22.2427934649262 \\
-3.30134242578957	-22.1920372837392 \\
-3.33090960039232	-22.1413307400173 \\
-3.36045793009597	-22.0906827819469 \\
-3.38998125037473	-22.0400877775004 \\
-3.41947339670281	-21.9895400946502 \\
-3.44892820455442	-21.9390341013688 \\
-3.47833950940377	-21.8885641656287 \\
-3.50770114672507	-21.8381246554024 \\
-3.53700695199253	-21.7877099386621 \\
-3.56625076068035	-21.7373143833805 \\
-3.59542640826275	-21.6869323575299 \\
-3.62452773021394	-21.6365582290829 \\
-3.65354856200812	-21.5861863660117 \\
-3.68248273911951	-21.535811136289 \\
-3.71132409702231	-21.4854269078871 \\
-3.74006647119073	-21.4350280487785 \\
-3.76870369709899	-21.3846089269357 \\
-3.79722961022129	-21.334163910331 \\
-3.82563804603184	-21.283687366937 \\
-3.85392284000485	-21.2331736647261 \\
-3.88207782761453	-21.1826171716706 \\
-3.91009684433509	-21.1320122557432 \\
-3.93797372564073	-21.0813532849162 \\
-3.96570230700568	-21.0306346271621 \\
-3.99335619450802	-20.9798835903417 \\
-4.02100408101605	-20.9291294089968 \\
-4.04863243218925	-20.8783693411214 \\
-4.07622771368713	-20.8276006447096 \\
-4.10377639116916	-20.7768205777554 \\
-4.13126493029484	-20.7260263982526 \\
-4.15867979672366	-20.6752153641953 \\
-4.18600745611511	-20.6243847335774 \\
-4.21323437412869	-20.5735317643931 \\
-4.24034701642389	-20.5226537146362 \\
-4.2673318486602	-20.4717478423007 \\
-4.2941753364971	-20.4208114053807 \\
-4.3208639455941	-20.3698416618701 \\
-4.34738414161068	-20.3188358697629 \\
-4.37372239020633	-20.2677912870531 \\
-4.39986515704055	-20.2167051717347 \\
-4.42579890777282	-20.1655747818016 \\
-4.45151010806265	-20.1143973752479 \\
-4.47698522356951	-20.0631702100676 \\
-4.50221071995291	-20.0118905442545 \\
-4.52717306287233	-19.9605556358028 \\
-4.55185871798727	-19.9091627427065 \\
-4.57625415095721	-19.8577091229594 \\
-4.60040873322445	-19.8062034385281 \\
-4.62451096935941	-19.7546794269247 \\
-4.64855795311505	-19.7031359603433 \\
-4.6725309556033	-19.6515690366496 \\
-4.69641124793609	-19.599974653709 \\
-4.72018010122535	-19.5483488093869 \\
-4.74381878658301	-19.4966875015488 \\
-4.767308575121	-19.4449867280603 \\
-4.79063073795126	-19.3932424867867 \\
-4.8137665461857	-19.3414507755936 \\
-4.83669727093627	-19.2896075923464 \\
-4.85940418331489	-19.2377089349106 \\
-4.88186855443349	-19.1857508011517 \\
-4.904071655404	-19.1337291889352 \\
-4.92599475733836	-19.0816400961264 \\
-4.94761913134849	-19.029479520591 \\
-4.96892604854631	-18.9772434601944 \\
-4.98989678004378	-18.924927912802 \\
-5.0105125969528	-18.8725288762793 \\
-5.03075477038532	-18.8200423484919 \\
-5.05060457145327	-18.7674643273051 \\
-5.07004327126856	-18.7147908105845 \\
-5.08905214094314	-18.6620177961955 \\
-5.10763396736542	-18.609140395263 \\
-5.12604948930278	-18.5561460016066 \\
-5.14436529897691	-18.5030362023219 \\
-5.16256186614387	-18.4498168623625 \\
-5.18061966055973	-18.3964938466818 \\
-5.19851915198056	-18.3430730202335 \\
-5.21624081016243	-18.2895602479709 \\
-5.2337651048614	-18.2359613948476 \\
-5.25107250583355	-18.1822823258171 \\
-5.26814348283493	-18.1285289058329 \\
-5.28495850562163	-18.0747069998484 \\
-5.3014980439497	-18.0208224728173 \\
-5.31774256757522	-17.966881189693 \\
-5.33367254625424	-17.9128890154289 \\
-5.34926844974285	-17.8588518149787 \\
-5.3645107477971	-17.8047754532958 \\
-5.37937991017307	-17.7506657953338 \\
-5.39385640662683	-17.696528706046 \\
-5.40792070691443	-17.6423700503861 \\
-5.42155328079195	-17.5881956933075 \\
-5.43473459801546	-17.5340114997637 \\
-5.44744512834103	-17.4798233347083 \\
-5.45966534152471	-17.4256370630947 \\
-5.47137570732259	-17.3714585498764 \\
-5.48280546107768	-17.3172537605184 \\
-5.49418118756087	-17.2629875965459 \\
-5.50547938067008	-17.2086643270034 \\
-5.51667653430318	-17.154288220935 \\
-5.52774914235809	-17.0998635473852 \\
-5.53867369873267	-17.0453945753981 \\
-5.54942669732485	-16.9908855740182 \\
-5.5599846320325	-16.9363408122897 \\
-5.57032399675353	-16.8817645592569 \\
-5.58042128538582	-16.8271610839641 \\
-5.59025299182728	-16.7725346554557 \\
-5.59979560997579	-16.7178895427759 \\
-5.60902563372925	-16.663230014969 \\
-5.61791955698555	-16.6085603410794 \\
-5.6264538736426	-16.5538847901514 \\
-5.63460507759827	-16.4992076312292 \\
-5.64234966275048	-16.4445331333571 \\
-5.6496641229971	-16.3898655655796 \\
-5.65652495223605	-16.3352091969407 \\
-5.6629086443652	-16.280568296485 \\
-5.66879169328246	-16.2259471332566 \\
-5.67415059288571	-16.1713499763 \\
-5.67896183707286	-16.1167810946593 \\
-5.68332926408424	-16.0622202607763 \\
-5.68764705067319	-16.0075948152217 \\
-5.69193161195732	-15.952908596116 \\
-5.69616767951836	-15.8981716731499 \\
-5.70033998493804	-15.8433941160142 \\
-5.70443325979806	-15.7885859943996 \\
-5.70843223568015	-15.7337573779968 \\
-5.71232164416602	-15.6789183364965 \\
-5.7160862168374	-15.6240789395893 \\
-5.71971068527601	-15.569249256966 \\
-5.72317978106356	-15.5144393583174 \\
-5.72647823578177	-15.459659313334 \\
-5.72959078101236	-15.4049191917066 \\
-5.73250214833706	-15.3502290631259 \\
-5.73519706933757	-15.2955989972827 \\
-5.73766027559562	-15.2410390638675 \\
-5.73987649869293	-15.1865593325712 \\
-5.74183047021122	-15.1321698730844 \\
-5.7435069217322	-15.0778807550978 \\
-5.7448905848376	-15.0237020483021 \\
-5.74596619110913	-14.9696438223881 \\
-5.74671847212851	-14.9157161470464 \\
-5.74713215947746	-14.8619290919677 \\
-5.74719786506154	-14.8082867698523 \\
-5.74698454657787	-14.7547197981032 \\
-5.74652477535839	-14.7012081327886 \\
-5.74582954691612	-14.6477550550856 \\
-5.74490985676405	-14.5943638461716 \\
-5.7437767004152	-14.5410377872237 \\
-5.74244107338257	-14.4877801594191 \\
-5.74091397117916	-14.4345942439352 \\
-5.73920638931798	-14.381483321949 \\
-5.73732932331204	-14.3284506746379 \\
-5.73529376867434	-14.2754995831791 \\
-5.73311072091789	-14.2226333287498 \\
-5.73079117555569	-14.1698551925272 \\
-5.72834612810074	-14.1171684556885 \\
-5.72578657406606	-14.0645763994111 \\
-5.72312350896465	-14.012082304872 \\
-5.72036792830952	-13.9596894532486 \\
-5.71753082761367	-13.9074011257181 \\
-5.7146232023901	-13.8552206034576 \\
-5.71165604815182	-13.8031511676445 \\
-5.70864036041184	-13.751196099456 \\
-5.70558713468317	-13.6993586800692 \\
-5.7025073664788	-13.6476421906614 \\
-5.69941205131175	-13.5960499124099 \\
-5.6961113867024	-13.5445618481182 \\
-5.69241922597108	-13.4931571960069 \\
-5.68835204833065	-13.4418380303781 \\
-5.68392633299401	-13.3906064255343 \\
-5.67915855917402	-13.3394644557776 \\
-5.67406520608357	-13.2884141954103 \\
-5.66866275293554	-13.2374577187348 \\
-5.66296767894281	-13.1865971000533 \\
-5.65699646331826	-13.1358344136682 \\
-5.65076558527476	-13.0851717338816 \\
-5.6442915240252	-13.034611134996 \\
-5.63759075878246	-12.9841546913136 \\
-5.63067976875941	-12.9338044771366 \\
-5.62357503316894	-12.8835625667674 \\
-5.61629303122392	-12.8334310345083 \\
-5.60885024213724	-12.7834119546615 \\
-5.60126314512177	-12.7335074015294 \\
-5.5935482193904	-12.6837194494142 \\
-5.585721944156	-12.6340501726182 \\
-5.57780079863146	-12.5845016454438 \\
-5.56980126202965	-12.5350759421931 \\
-5.56173981356345	-12.4857751371685 \\
-5.55363293244574	-12.4366013046723 \\
-5.54541608711783	-12.3875527873962 \\
-5.53684194756207	-12.3386182269141 \\
-5.52790562894592	-12.2897959980371 \\
-5.5186224781608	-12.2410853562797 \\
-5.50900784209815	-12.1924855571559 \\
-5.49907706764939	-12.1439958561801 \\
-5.48884550170596	-12.0956155088664 \\
-5.47832849115928	-12.0473437707291 \\
-5.4675413829008	-11.9991798972824 \\
-5.45649952382192	-11.9511231440406 \\
-5.4452182608141	-11.9031727665177 \\
-5.43371294076874	-11.8553280202282 \\
-5.4219989105773	-11.8075881606861 \\
-5.41009151713119	-11.7599524434058 \\
-5.39800610732185	-11.7124201239014 \\
-5.38575802804071	-11.6649904576872 \\
-5.37336262617919	-11.6176627002774 \\
-5.36083524862873	-11.5704361071863 \\
-5.34819124228076	-11.5233099339279 \\
-5.33544595402671	-11.4762834360167 \\
-5.322614730758	-11.4293558689667 \\
-5.30971291936607	-11.3825264882923 \\
-5.29675586674236	-11.3357945495077 \\
-5.28374768163567	-11.2891592235824 \\
-5.27055410411349	-11.2426189818937 \\
-5.25713455671543	-11.1961737259714 \\
-5.24349253680899	-11.1498237742591 \\
-5.22963154176169	-11.1035694452003 \\
-5.21555506894106	-11.0574110572386 \\
-5.20126661571461	-11.0113489288176 \\
-5.18676967944984	-10.9653833783809 \\
-5.17206775751429	-10.9195147243721 \\
-5.15716434727546	-10.8737432852347 \\
-5.14206294610087	-10.8280693794122 \\
-5.12676705135803	-10.7824933253484 \\
-5.11128016041448	-10.7370154414866 \\
-5.09560577063771	-10.6916360462706 \\
-5.07974737939525	-10.6463554581439 \\
-5.06370848405461	-10.6011739955501 \\
-5.04749258198331	-10.5560919769327 \\
-5.03110317054886	-10.5111097207353 \\
-5.01454374711878	-10.4662275454016 \\
-4.99781780906059	-10.421445769375 \\
-4.98092885374181	-10.3767647110991 \\
-4.96388037852994	-10.3321846890175 \\
-4.94667588079251	-10.2877060215739 \\
};
\addplot [line width=1.2pt, blue, dotted, forget plot]
table [row sep=\\]{%
29.7355471766366	0.36622258092544 \\
29.6263105105992	0.28280652898206 \\
29.5171544848586	0.199328232815621 \\
29.4080791916673	0.115787759933134 \\
29.2990847232776	0.0321851778416096 \\
29.1901711719421	-0.0514794459519428 \\
29.0813386299131	-0.135206043940513 \\
28.9725871894432	-0.21899454861709 \\
28.8639169427847	-0.302844892474664 \\
28.7553279821901	-0.386757008006224 \\
28.6468203999119	-0.470730827704762 \\
28.5383942882025	-0.554766284063266 \\
28.4300497393143	-0.638863309574725 \\
28.3217868454998	-0.723021836732131 \\
28.2136056990115	-0.807241798028472 \\
28.1055063921018	-0.891523125956739 \\
27.997489017023	-0.975865753009921 \\
27.8895536660278	-1.06026961168101 \\
27.7817004313685	-1.14473463446299 \\
27.6739294052976	-1.22926075384885 \\
27.5662406800675	-1.3138479023316 \\
27.4586343479306	-1.3984960124042 \\
27.3511105011395	-1.48320501655966 \\
27.2436692319465	-1.56797484729096 \\
27.1363246771416	-1.65283690206765 \\
27.0291185252872	-1.73788482724177 \\
26.9220457970832	-1.82310830124324 \\
26.8150978401577	-1.90848879436522 \\
26.7082660021387	-1.99400777690089 \\
26.6015416306545	-2.07964671914342 \\
26.4949160733331	-2.16538709138598 \\
26.3883806778025	-2.25121036392176 \\
26.281926791691	-2.33709800704391 \\
26.1755457626265	-2.42303149104563 \\
26.0692289382373	-2.50899228622007 \\
25.9629676661513	-2.59496186286042 \\
25.8567532939968	-2.68092169125984 \\
25.7505771694017	-2.76685324171152 \\
25.6444306399942	-2.85273798450862 \\
25.5383050534025	-2.93855738994431 \\
25.4321917572545	-3.02429292831179 \\
25.3260820991784	-3.1099260699042 \\
25.2199674268024	-3.19543828501474 \\
25.1138390877544	-3.28081104393657 \\
25.0076884296626	-3.36602581696287 \\
24.9015068001551	-3.45106407438681 \\
24.7952855468601	-3.53590728650157 \\
24.6890234727331	-3.6205495806277 \\
24.5828088585015	-3.70513937300084 \\
24.4766629163541	-3.78971627635623 \\
24.3705770406291	-3.87426987646614 \\
24.2645426256647	-3.95878975910283 \\
24.1585510657991	-4.0432655100386 \\
24.0525937553706	-4.12768671504572 \\
23.9466620887174	-4.21204295989645 \\
23.8407474601778	-4.29632383036309 \\
23.73484126409	-4.38051891221789 \\
23.6289348947922	-4.46461779123314 \\
23.5230197466227	-4.54861005318112 \\
23.4170872139197	-4.6324852838341 \\
23.3111286910214	-4.71623306896435 \\
23.2051355722661	-4.79984299434416 \\
23.099099251992	-4.88330464574579 \\
22.9930111245374	-4.96660760894153 \\
22.8868625842405	-5.04974146970364 \\
22.7806450254395	-5.13269581380441 \\
22.6743498424727	-5.21546022701611 \\
22.5679684296782	-5.29802429511102 \\
22.4614921813944	-5.38037760386141 \\
22.3549124919595	-5.46250973903955 \\
22.2482207557117	-5.54441028641773 \\
22.1414663843218	-5.62616258331302 \\
22.0347003291252	-5.70784831632968 \\
21.9279162939569	-5.78945459517072 \\
21.8211079826517	-5.87096852953917 \\
21.7142690990445	-5.95237722913807 \\
21.6073933469705	-6.03366780367043 \\
21.5004744302644	-6.11482736283929 \\
21.3935060527612	-6.19584301634766 \\
21.286481918296	-6.27670187389859 \\
21.1793957307035	-6.35739104519509 \\
21.0722411938188	-6.43789763994019 \\
20.9650120114768	-6.51820876783693 \\
20.8577018875124	-6.59831153858831 \\
20.7503045257606	-6.67819306189738 \\
20.6428136300564	-6.75784044746715 \\
20.5352229042346	-6.83724080500066 \\
20.4275260521303	-6.91638124420093 \\
20.3197167775783	-6.99524887477099 \\
20.2117887844136	-7.07383080641386 \\
20.1037357764711	-7.15211414883258 \\
19.9955514575858	-7.23008601173016 \\
19.8872295315927	-7.30773350480964 \\
19.7787637023266	-7.38504373777404 \\
19.6701614618485	-7.46205043466849 \\
19.5614633813086	-7.53889274874438 \\
19.4526709694308	-7.61557060279575 \\
19.3437823701056	-7.69207228790856 \\
19.2347957272235	-7.7683860951688 \\
19.125709184675	-7.84450031566242 \\
19.0165208863507	-7.92040324047539 \\
18.9072289761412	-7.9960831606937 \\
18.797831597937	-8.0715283674033 \\
18.6883268956285	-8.14672715169017 \\
18.5787130131064	-8.22166780464027 \\
18.4689880942612	-8.29633861733958 \\
18.3591502829833	-8.37072788087407 \\
18.2491977231634	-8.4448238863297 \\
18.1391285586921	-8.51861492479245 \\
18.0289409334597	-8.59208928734829 \\
17.9186329913569	-8.66523526508318 \\
17.8082028762741	-8.7380411490831 \\
17.6976487321021	-8.81049523043401 \\
17.5869687027311	-8.88258580022189 \\
17.4761609320519	-8.95430114953271 \\
17.3652235639549	-9.02562956945243 \\
17.2541547423307	-9.09655935106703 \\
17.1429544495156	-9.1670958431067 \\
17.0316444461442	-9.23744549457432 \\
16.9202293453573	-9.30765360382657 \\
16.8087063357499	-9.37769634077874 \\
16.6970726059169	-9.44754987534611 \\
16.5853253444535	-9.51719037744399 \\
16.4734617399545	-9.58659401698765 \\
16.3614789810151	-9.65573696389238 \\
16.2493742562302	-9.72459538807347 \\
16.1371447541949	-9.79314545944621 \\
16.0247876635041	-9.86136334792589 \\
15.9123001727529	-9.92922522342779 \\
15.7996794705363	-9.99670725586722 \\
15.6869227454494	-10.0637856151594 \\
15.5740271860871	-10.1304364712198 \\
15.4609899810444	-10.1966359939634 \\
15.3478083189164	-10.2623603533058 \\
15.2344793882981	-10.3275857191621 \\
15.1210003777845	-10.3922882614477 \\
15.0073684759706	-10.4564441500778 \\
14.8935808714515	-10.5200295549677 \\
14.7796347528221	-10.5830206460327 \\
14.6655273086774	-10.6453935931882 \\
14.5512557276126	-10.7071245663493 \\
14.4368084502364	-10.7683820736761 \\
14.3221790039988	-10.829340013653 \\
14.2073722077559	-10.8899826416816 \\
14.092392880364	-10.9502942131636 \\
13.9772458406795	-11.0102589835009 \\
13.8619359075585	-11.069861208095 \\
13.7464678998573	-11.1290851423478 \\
13.6308466364322	-11.1879150416608 \\
13.5150769361395	-11.2463351614359 \\
13.3991636178353	-11.3043297570748 \\
13.2831115003761	-11.3618830839791 \\
13.166925402618	-11.4189793975506 \\
13.0506101434174	-11.475602953191 \\
12.9341705416304	-11.531738006302 \\
12.8176114161134	-11.5873688122854 \\
12.7009375857226	-11.6424796265428 \\
12.5841538693143	-11.6970547044759 \\
12.4672650857447	-11.7510783014866 \\
12.3502760538701	-11.8045346729764 \\
12.2331915925468	-11.8574080743472 \\
12.1160165206311	-11.9096827610005 \\
11.9987556569792	-11.9613429883382 \\
11.8814138204473	-12.012373011762 \\
11.7639806830463	-12.062859130122 \\
11.6464112634542	-12.1131104717605 \\
11.5287053792044	-12.1631224542152 \\
11.410866611624	-12.212864713431 \\
11.2928985420404	-12.2623068853527 \\
11.1748047517809	-12.3114186059254 \\
11.0565888221729	-12.3601695110938 \\
10.9382543345436	-12.4085292368028 \\
10.8198048702204	-12.4564674189975 \\
10.7012440105305	-12.5039536936226 \\
10.5825753368014	-12.550957696623 \\
10.4638024303604	-12.5974490639437 \\
10.3449288725346	-12.6433974315295 \\
10.2259582446516	-12.6887724353254 \\
10.1068941280385	-12.7335437112762 \\
9.98774010402275	-12.7776808953269 \\
9.86849975393162	-12.8211536234223 \\
9.74917665909244	-12.8639315315073 \\
9.62977440083252	-12.9059842555268 \\
9.51029656047919	-12.9472814314258 \\
9.39074671935976	-12.9877926951491 \\
9.27112845880156	-13.0274876826416 \\
9.1514453601319	-13.0663360298482 \\
9.03169139352804	-13.1043493934037 \\
8.91174967075016	-13.1420401018575 \\
8.79159931409908	-13.1795346977724 \\
8.67125900932179	-13.2167907468599 \\
8.55074744216525	-13.2537658148319 \\
8.43008329837646	-13.2904174674 \\
8.30928526370237	-13.3267032702757 \\
8.18837202388997	-13.3625807891708 \\
8.06736226468623	-13.3980075897968 \\
7.94627467183812	-13.4329412378655 \\
7.82512793109263	-13.4673392990885 \\
7.70394072819673	-13.5011593391774 \\
7.58273174889739	-13.5343589238439 \\
7.46151967894158	-13.5668956187996 \\
7.34032320407629	-13.5987269897562 \\
7.21916101004849	-13.6298106024254 \\
7.09805178260515	-13.6601040225187 \\
6.97701420749325	-13.6895648157478 \\
6.85606697045977	-13.7181505478245 \\
6.73522875725167	-13.7458187844603 \\
6.61451825361594	-13.7725270913668 \\
6.49395414529955	-13.7982330342558 \\
6.37355511804948	-13.8228941788388 \\
6.2533398576127	-13.8464680908276 \\
6.13318395151659	-13.8697848648092 \\
6.01296253079132	-13.8936094935684 \\
5.89269359643802	-13.9178018901134 \\
5.77239514945783	-13.9422219674523 \\
5.6520851908519	-13.9667296385935 \\
5.53178172162136	-13.991184816545 \\
5.41150274276735	-14.0154474143151 \\
5.29126625529102	-14.0393773449119 \\
5.1710902601935	-14.0628345213436 \\
5.05099275847593	-14.0856788566185 \\
4.93099175113945	-14.1077702637447 \\
4.8111052391852	-14.1289686557303 \\
4.69135122361433	-14.1491339455837 \\
4.57174770542797	-14.1681260463129 \\
4.45231268562725	-14.1858048709262 \\
4.33306416521333	-14.2020303324317 \\
4.21402014518734	-14.2166623438376 \\
4.09519862655042	-14.2295608181522 \\
3.97661761030371	-14.2405856683836 \\
3.85829509744834	-14.2495968075401 \\
3.74024908898547	-14.2564541486297 \\
3.62249758591622	-14.2610176046606 \\
3.50505858924175	-14.2631470886412 \\
3.38785819638976	-14.2632772455778 \\
3.27062345155939	-14.2632009436546 \\
3.15336576705907	-14.2630597403846 \\
3.03611984028305	-14.2628541133284 \\
2.91892036862556	-14.2625845400468 \\
2.80180204948085	-14.2622514981003 \\
2.68479958024317	-14.2618554650496 \\
2.56794765830677	-14.2613969184554 \\
2.45128098106588	-14.2608763358784 \\
2.33483424591475	-14.2602941948792 \\
2.21864215024763	-14.2596509730185 \\
2.10273939145876	-14.2589471478569 \\
1.98716066694239	-14.2581831969551 \\
1.87194067409276	-14.2573595978737 \\
1.75711411030412	-14.2564768281734 \\
1.6427156729707	-14.255535365415 \\
1.52878005948677	-14.2545356871589 \\
1.41534196724655	-14.2534782709659 \\
1.3024360936443	-14.2523635943967 \\
1.19009713607426	-14.2511921350119 \\
1.07835979193068	-14.2499643703722 \\
0.967258758607797	-14.2486807780382 \\
0.85682873349986	-14.2473418355706 \\
0.747040207150973	-14.2457848131972 \\
0.637115415259288	-14.2419310880136 \\
0.526901688305167	-14.2352614171934 \\
0.416508741008466	-14.2259200359571 \\
0.306046288089045	-14.2140511795253 \\
0.19562404426676	-14.1997990831186 \\
0.0853517242614687	-14.1833079819576 \\
-0.0246609572069701	-14.1647221112627 \\
-0.1343042854187	-14.1441857062545 \\
-0.243468545653862	-14.1218430021537 \\
-0.352044023192599	-14.0978382341808 \\
-0.459921003315053	-14.0723156375562 \\
-0.566989771301367	-14.0454194475007 \\
-0.673140612431682	-14.0172938992348 \\
-0.778263811986141	-13.9880832279789 \\
-0.882249655244887	-13.9579316689537 \\
-0.984988427488062	-13.9269834573798 \\
-1.08637041399581	-13.8953828284777 \\
-1.18628590004827	-13.863274017468 \\
-1.28462517092558	-13.8308012595711 \\
-1.38127851190789	-13.7981087900078 \\
-1.47613620827534	-13.7653408439985 \\
-1.56908854530808	-13.7326416567638 \\
-1.66002580828624	-13.7001554635243 \\
-1.75035773256955	-13.6665840084294 \\
-1.84140300272164	-13.6306021679677 \\
-1.93291532782849	-13.5923138758691 \\
-2.02464841697609	-13.5518230658631 \\
-2.11635597925042	-13.5092336716793 \\
-2.20779172373746	-13.4646496270475 \\
-2.29870935952321	-13.4181748656972 \\
-2.38886259569365	-13.3699133213581 \\
-2.47800514133476	-13.3199689277598 \\
-2.56589070553253	-13.268445618632 \\
-2.65227299737293	-13.2154473277044 \\
-2.73690572594197	-13.1610779887065 \\
-2.81954260032562	-13.105441535368 \\
-2.89993732960987	-13.0486419014186 \\
-2.97784362288069	-12.9907830205878 \\
-3.05301518922409	-12.9319688266054 \\
-3.12520573772604	-12.872303253201 \\
-3.19416897747253	-12.8118902341043 \\
-3.25965861754954	-12.7508337030448 \\
-3.32142836704305	-12.6892375937522 \\
-3.37923193503907	-12.6272058399561 \\
-3.43282303062355	-12.5648423753863 \\
-3.48195536288251	-12.5022511337723 \\
-3.52753880064663	-12.4390170504017 \\
-3.57310556811929	-12.3735737740439 \\
-3.61874292573046	-12.3059303839683 \\
-3.66424993381178	-12.2362263674538 \\
-3.70942565269489	-12.1646012117795 \\
-3.75406914271146	-12.0911944042242 \\
-3.79797946419314	-12.0161454320671 \\
-3.84095567747156	-11.939593782587 \\
-3.88279684287839	-11.8616789430629 \\
-3.92330202074528	-11.7825404007738 \\
-3.96227027140387	-11.7023176429987 \\
-3.99950065518583	-11.6211501570166 \\
-4.03479223242279	-11.5391774301064 \\
-4.06794406344641	-11.4565389495472 \\
-4.09875520858835	-11.3733742026178 \\
-4.12702472818025	-11.2898226765973 \\
-4.15255168255376	-11.2060238587646 \\
-4.17513513204054	-11.1221172363988 \\
-4.19457413697224	-11.0382422967788 \\
-4.2106677576805	-10.9545385271835 \\
-4.22321505449698	-10.871145414892 \\
-4.23201508775334	-10.7882024471832 \\
-4.23686691778121	-10.7058491113361 \\
-4.23779476144033	-10.6241619880155 \\
-4.23755276673549	-10.5424218562171 \\
-4.23692828513628	-10.4603886945788 \\
-4.23581505520207	-10.3780638776207 \\
-4.23410681549219	-10.2954487798628 \\
-4.23169730456602	-10.2125447758253 \\
-4.22848026098289	-10.1293532400282 \\
-4.22434942330217	-10.0458755469915 \\
-4.21919853008321	-9.96211307123519 \\
-4.21292131988537	-9.87806718727946 \\
-4.20541153126799	-9.7937392696443 \\
-4.19656290279044	-9.70913069284976 \\
-4.18626917301207	-9.6242428314159 \\
-4.17442408049223	-9.53907705986277 \\
-4.16092136379027	-9.45363475271041 \\
-4.14565476146556	-9.36791728447889 \\
-4.12851801207744	-9.28192602968824 \\
-4.10940485418527	-9.19566236285853 \\
-4.08820902634841	-9.1091276585098 \\
-4.0648242671262	-9.02232329116211 \\
-4.03914431507801	-8.9352506353355 \\
-4.01106290876319	-8.84791106555003 \\
-3.98047378674109	-8.76030595632575 \\
};
\addplot [line width=1.2pt, blue, dotted, forget plot]
table [row sep=\\]{%
29.2119725329878	4.27649516409521 \\
29.0828969607111	4.23820368941495 \\
28.9537541360625	4.19984930677697 \\
28.8245441104512	4.16143216105627 \\
28.6952669352867	4.12295239712784 \\
28.565922661978	4.0844101598667 \\
28.4365113419345	4.04580559414784 \\
28.3070330265654	4.00713884484625 \\
28.1774877672799	3.96841005683695 \\
28.0478756154874	3.92961937499492 \\
27.9181966225971	3.89076694419516 \\
27.7884508400182	3.85185290931269 \\
27.65863831916	3.81287741522249 \\
27.5287591114318	3.77384060679956 \\
27.3988132682427	3.73474262891892 \\
27.2688008410021	3.69558362645554 \\
27.1387218811192	3.65636374428444 \\
27.0085764400033	3.61708312728062 \\
26.8783645690636	3.57774192031907 \\
26.7480863197094	3.53834026827479 \\
26.6177417433499	3.49887831602279 \\
26.4873308913944	3.45935620843806 \\
26.3568538152521	3.4197740903956 \\
26.2263105663324	3.38013210677041 \\
26.0956567754196	3.3405214004216 \\
25.9647602772242	3.30121249213876 \\
25.8336357767022	3.26217449712138 \\
25.7023095699793	3.22335276637158 \\
25.5708079531812	3.18469265089148 \\
25.4391572224333	3.14613950168319 \\
25.3073836738614	3.10763866974883 \\
25.175513603591	3.0691355060905 \\
25.0435733077478	3.03057536171034 \\
24.9115890824573	2.99190358761044 \\
24.7795872238452	2.95306553479293 \\
24.6475940280371	2.91400655425993 \\
24.5156357911585	2.87467199701354 \\
24.3837388093352	2.83500721405589 \\
24.2519293786927	2.79495755638909 \\
24.1202337953565	2.75446837501526 \\
23.9886783554524	2.7134850209365 \\
23.857289355106	2.67195284515494 \\
23.7260930904428	2.62981719867269 \\
23.5951158575884	2.58702343249188 \\
23.4643839526686	2.5435168976146 \\
23.3339236718088	2.49924294504298 \\
23.2037613111347	2.45414692577913 \\
23.073892783589	2.4082022400776 \\
22.9439546036333	2.36172603513836 \\
22.8138677837673	2.31479822121201 \\
22.6836761691506	2.26738830051857 \\
22.5534236049427	2.21946577527808 \\
22.4231539363033	2.17100014771056 \\
22.2929110083921	2.12196092003602 \\
22.1627386663686	2.07231759447451 \\
22.0326807553925	2.02203967324603 \\
21.9027811206233	1.97109665857061 \\
21.7730836072208	1.91945805266829 \\
21.6436320603445	1.86709335775908 \\
21.5144703251541	1.81397207606301 \\
21.3856422468092	1.76006370980011 \\
21.2571916704693	1.70533776119039 \\
21.1291624412942	1.64976373245388 \\
21.0015984044434	1.59331112581062 \\
20.8745434050766	1.53594944348061 \\
20.7480412883534	1.47764818768389 \\
20.6221358994334	1.41837686064049 \\
20.4968710834762	1.35810496457042 \\
20.3722906856415	1.29680200169371 \\
20.2484385510889	1.23443747423039 \\
20.125358524978	1.17098088440047 \\
20.0025938379107	1.10619023996849 \\
19.8797127813139	1.03986613345261 \\
19.7567967915844	0.972041944203367 \\
19.633927305119	0.902751051571294 \\
19.5111857583147	0.832026834906916 \\
19.3886535875681	0.759902673560764 \\
19.2664122292762	0.686411946883366 \\
19.1445431198358	0.611588034225251 \\
19.0231276956438	0.535464314936948 \\
18.9022473930969	0.458074168368987 \\
18.7819836485921	0.379450973871895 \\
18.6624178985262	0.299628110796203 \\
18.5436315792959	0.218638958492439 \\
18.4257061272983	0.136516896311133 \\
18.30872297893	0.0532953036028122 \\
18.1927635705879	-0.0309924402819928 \\
18.077909338669	-0.116312955992754 \\
17.9642417195699	-0.202632864178941 \\
17.8518421496876	-0.289918785490027 \\
17.7407920654189	-0.37813734057548 \\
17.6311729031607	-0.467255150084774 \\
17.5230660993098	-0.557238834667378 \\
17.4165530902629	-0.648055014972764 \\
17.3115344024323	-0.739986668848195 \\
17.2074632151961	-0.833996567827521 \\
17.1042903023709	-0.930062560641761 \\
17.0020100532867	-1.02808212632215 \\
16.9006168572734	-1.12795274389991 \\
16.8001051036612	-1.22957189240627 \\
16.7004691817799	-1.33283705087248 \\
16.6017034809596	-1.43764569832975 \\
16.5038023905303	-1.54389531380933 \\
16.406760299822	-1.65148337634244 \\
16.3105715981646	-1.76030736496031 \\
16.2152306748882	-1.87026475869418 \\
16.1207319193229	-1.98125303657527 \\
16.0270697207985	-2.09316967763482 \\
15.934238468645	-2.20591216090406 \\
15.8422325521926	-2.31937796541422 \\
15.7510463607711	-2.43346457019652 \\
15.6606742837107	-2.54806945428221 \\
15.5711107103412	-2.66309009670252 \\
15.4823500299927	-2.77842397648867 \\
15.3943866319951	-2.89396857267189 \\
15.3072149056786	-3.00962136428343 \\
15.220829240373	-3.1252798303545 \\
15.1352504975436	-3.24092956410858 \\
15.0507963610619	-3.35761744176717 \\
14.9675019737506	-3.47560562629008 \\
14.8852913790878	-3.59480480204092 \\
14.8040886205519	-3.71512565338329 \\
14.723817741621	-3.83647886468079 \\
14.6444027857734	-3.95877512029704 \\
14.5657677964873	-4.08192510459563 \\
14.4878368172409	-4.20583950194018 \\
14.4105338915125	-4.33042899669428 \\
14.3337830627802	-4.45560427322155 \\
14.2575083745224	-4.58127601588559 \\
14.1816338702172	-4.70735490905 \\
14.1060835933428	-4.83375163707839 \\
14.0307815873775	-4.96037688433436 \\
13.9556518957995	-5.08714133518153 \\
13.880618562087	-5.21395567398349 \\
13.8056056297183	-5.34073058510385 \\
13.7305371421715	-5.46737675290621 \\
13.655337142925	-5.59380486175419 \\
13.5799296754569	-5.71992559601138 \\
13.5042387832454	-5.8456496400414 \\
13.4281885097688	-5.97088767820784 \\
13.3517028985054	-6.09555039487431 \\
13.275245749562	-6.22020462565548 \\
13.1992898323294	-6.34542412158131 \\
13.1237726442113	-6.47112994013832 \\
13.0486316826115	-6.59724313881305 \\
12.9738044449338	-6.72368477509205 \\
12.8992284285818	-6.85037590646186 \\
12.8248411309594	-6.977237590409 \\
12.7505800494703	-7.10419088442001 \\
12.6763826815182	-7.23115684598145 \\
12.602186524507	-7.35805653257983 \\
12.5279290758403	-7.48481100170171 \\
12.4535478329219	-7.61134131083361 \\
12.3789802931556	-7.73756851746208 \\
12.3041639539452	-7.86341367907365 \\
12.2290363126943	-7.98879785315486 \\
12.1535348668067	-8.11364209719226 \\
12.0775971136863	-8.23786746867236 \\
12.0011605507366	-8.36139502508173 \\
11.9241626753616	-8.48414582390688 \\
11.8465409849649	-8.60604092263437 \\
11.7682329769504	-8.72700137875072 \\
11.6891761487217	-8.84694824974247 \\
11.6093079976826	-8.96580259309617 \\
11.5287479712487	-9.08385152436885 \\
11.4480309511365	-9.2021838046143 \\
11.3671527414149	-9.32076539729361 \\
11.2860639005205	-9.43946987194373 \\
11.2047149868899	-9.55817079810165 \\
11.12305655896	-9.67674174530434 \\
11.0410391751673	-9.79505628308877 \\
10.9586133939487	-9.9129879809919 \\
10.8757297737407	-10.0304104085507 \\
10.7923388729802	-10.1471971353022 \\
10.7083912501037	-10.2632217307833 \\
10.6238374635481	-10.378357764531 \\
10.53862807175	-10.4924788060823 \\
10.4527136331462	-10.6054584249741 \\
10.3660447061732	-10.7171701907434 \\
10.2785718492679	-10.8274876729272 \\
10.190245620867	-10.9362844410625 \\
10.101016579407	-11.0434340646862 \\
10.0108352833249	-11.1488101133353 \\
9.91965229105711	-11.2522861565467 \\
9.82741816104052	-11.3537357638576 \\
9.73408345171179	-11.4530325048047 \\
9.63959872150761	-11.5500499489251 \\
9.54390207839135	-11.6447975544366 \\
9.44681209070583	-11.7389127446558 \\
9.34832284643104	-11.8328042092459 \\
9.24848570195534	-11.9263364083794 \\
9.14735201366706	-12.0193738022285 \\
9.04497313795455	-12.1117808509657 \\
8.94140043120616	-12.2034220147632 \\
8.83668524981022	-12.2941617537935 \\
8.73087895015508	-12.3838645282287 \\
8.62403288862909	-12.4723947982414 \\
8.51619842162059	-12.5596170240037 \\
8.40742690551792	-12.6453956656882 \\
8.29776969670944	-12.729595183467 \\
8.18727815158347	-12.8120800375126 \\
8.07600362652838	-12.8927146879973 \\
7.96399747793249	-12.9713635950934 \\
7.85131106218417	-13.0478912189733 \\
7.73799573567173	-13.1221620198093 \\
7.62410285478355	-13.1940404577738 \\
7.50968377590795	-13.263390993039 \\
7.39478985543329	-13.3300780857774 \\
7.2794724497479	-13.3939661961613 \\
7.16378291524014	-13.454919784363 \\
7.04777260829834	-13.5128033105548 \\
6.93090766686579	-13.5691043636034 \\
6.81267820578743	-13.6252994449727 \\
6.69317154680402	-13.681236383731 \\
6.57247501165636	-13.7367630089468 \\
6.45067592208523	-13.7917271496884 \\
6.3278615998314	-13.8459766350244 \\
6.20411936663566	-13.899359294023 \\
6.0795365442388	-13.9517229557526 \\
5.9542004543816	-14.0029154492818 \\
5.82819841880483	-14.0527846036788 \\
5.70161775924929	-14.1011782480121 \\
5.57454579745575	-14.14794421135 \\
5.447069855165	-14.1929303227611 \\
5.31927725411782	-14.2359844113136 \\
5.19125531605499	-14.2769543060759 \\
5.06309136271731	-14.3156878361166 \\
4.93487271584554	-14.3520328305039 \\
4.80668669718047	-14.3858371183063 \\
4.67862062846289	-14.4169485285922 \\
4.55076183143357	-14.4452148904299 \\
4.42319762783331	-14.4704840328879 \\
4.29601533940288	-14.4926037850346 \\
4.16930228788306	-14.5114219759384 \\
4.04285903873939	-14.5275225330816 \\
3.91583669239364	-14.543197525667 \\
3.78824184305793	-14.5585888783798 \\
3.66015260347623	-14.573656568113 \\
3.5316470863925	-14.5883605717596 \\
3.40280340455071	-14.6026608662129 \\
3.27369967069483	-14.6165174283658 \\
3.14441399756882	-14.6298902351116 \\
3.01502449791664	-14.6427392633433 \\
2.88560928448227	-14.6550244899539 \\
2.75624647000966	-14.6667058918367 \\
2.62701416724279	-14.6777434458847 \\
2.49799048892561	-14.6880971289909 \\
2.3692535478021	-14.6977269180486 \\
2.24088145661621	-14.7065927899508 \\
2.11295232811192	-14.7146547215905 \\
1.98554427503319	-14.721872689861 \\
1.85873541012399	-14.7282066716553 \\
1.73260384612827	-14.7336166438665 \\
1.60722769579001	-14.7380625833877 \\
1.48268507185317	-14.741504467112 \\
1.35905408706172	-14.7439022719325 \\
1.23641285415962	-14.7452159747423 \\
1.11473768703184	-14.7452751435296 \\
0.992805394445639	-14.7424063663393 \\
0.870355897428538	-14.7362415060519 \\
0.747541832741595	-14.7269495428962 \\
0.624515837145872	-14.7146994571007 \\
0.50143054740243	-14.6996602288942 \\
0.378438600272332	-14.6820008385053 \\
0.25569263251664	-14.6618902661625 \\
0.133345280896415	-14.6394974920945 \\
0.0115491821727194	-14.61499149653 \\
-0.109543026893385	-14.5885412596976 \\
-0.229778709540836	-14.5603157618259 \\
-0.349005229008572	-14.5304839831436 \\
-0.467069948535531	-14.4992149038793 \\
-0.583820231360651	-14.4666775042616 \\
-0.699103440722871	-14.4330407645191 \\
-0.812766939861128	-14.3984736648806 \\
-0.92465809201436	-14.3631451855745 \\
-1.03462426042151	-14.3272243068297 \\
-1.1425128083215	-14.2908800088746 \\
-1.24817109895329	-14.2542812719379 \\
-1.35144649555581	-14.2175970762484 \\
-1.45218636136799	-14.1809964020345 \\
-1.55023805962877	-14.1446482295249 \\
-1.6471969944682	-14.1068602292912 \\
-1.74460036572892	-14.0659294199324 \\
-1.84220823021883	-14.0220088037186 \\
-1.9397806447458	-13.9752513829198 \\
-2.03707766611771	-13.9258101598059 \\
-2.13385935114243	-13.8738381366471 \\
-2.22988575662785	-13.8194883157134 \\
-2.32491693938185	-13.7629136992747 \\
-2.4187129562123	-13.7042672896011 \\
-2.51103386392708	-13.6437020889626 \\
-2.60163971933408	-13.5813710996293 \\
-2.69029057924116	-13.5174273238711 \\
-2.77674650045622	-13.4520237639582 \\
-2.86076753978712	-13.3853134221604 \\
-2.94211375404175	-13.3174493007479 \\
-3.02054520002799	-13.2485844019906 \\
-3.09582193455371	-13.1788717281586 \\
-3.16770401442679	-13.108464281522 \\
-3.23595149645512	-13.0375150643506 \\
-3.30032443744656	-12.9661770789146 \\
-3.36058289420901	-12.894603327484 \\
-3.41648692355033	-12.8229468123288 \\
-3.46779658227841	-12.751360535719 \\
-3.51545493670337	-12.6793977369243 \\
-3.56308149342534	-12.6052577714509 \\
-3.6107631597169	-12.5289343367509 \\
-3.65829167723148	-12.4505708943463 \\
-3.70545878762248	-12.3703109057588 \\
-3.75205623254334	-12.2882978325104 \\
-3.79787575364747	-12.2046751361229 \\
-3.84270909258828	-12.1195862781182 \\
-3.88634799101921	-12.0331747200181 \\
-3.92858419059366	-11.9455839233445 \\
-3.96920943296506	-11.8569573496192 \\
-4.00801545978683	-11.767438460364 \\
-4.04479401271239	-11.6771707171009 \\
-4.07933683339516	-11.5862975813517 \\
-4.11143566348855	-11.4949625146383 \\
-4.14088224464598	-11.4033089784824 \\
-4.16746831852089	-11.311480434406 \\
-4.19098562676668	-11.2196203439309 \\
-4.21122591103677	-11.127872168579 \\
-4.22798091298459	-11.0363793698721 \\
-4.24104237426355	-10.945285409332 \\
-4.25020203652707	-10.8547337484807 \\
-4.25525164142858	-10.76486784884 \\
-4.25620003014996	-10.6757674611588 \\
-4.25569404659818	-10.5867054917149 \\
-4.25449635673863	-10.4974383239531 \\
-4.25250943051865	-10.407966605611 \\
-4.24963573788558	-10.3182909844263 \\
-4.24577774878677	-10.2284121081365 \\
-4.24083793316955	-10.1383306244794 \\
-4.23471876098127	-10.0480471811925 \\
-4.22732270216926	-9.95756242601365 \\
-4.21855222668088	-9.86687700668034 \\
-4.20830980446347	-9.77599157093029 \\
-4.19649790546436	-9.68490676650113 \\
-4.1830189996309	-9.59362324113051 \\
-4.16777555691043	-9.5021416425561 \\
-4.15067004725029	-9.41046261851553 \\
-4.13160494059783	-9.31858681674646 \\
-4.11048270690038	-9.22651488498655 \\
-4.08720581610529	-9.13424747097344 \\
-4.0616767381599	-9.04178522244478 \\
-4.03379794301156	-8.94912878713823 \\
-4.00347190060759	-8.85627881279143 \\
-3.97060108089536	-8.76323594714204 \\
-3.93508795382219	-8.67000083792771 \\
};
\addplot [line width=1.2pt, blue, dotted, forget plot]
table [row sep=\\]{%
14.1815398562066	-0.355135946184831 \\
14.1812403290875	-0.45859000060182 \\
14.1803537235026	-0.562090634459289 \\
14.1788980031105	-0.665637816896845 \\
14.1768911315696	-0.769231517054097 \\
14.1743510725387	-0.872871704070653 \\
14.1712957896763	-0.976558347086123 \\
14.167743246641	-1.08029141524011 \\
14.1637114070913	-1.18407087767224 \\
14.1592182346858	-1.28789670352209 \\
14.1542816930832	-1.3917688619293 \\
14.1489197459419	-1.49568732203346 \\
14.1431503569206	-1.59965205297419 \\
14.1369914896779	-1.70366302389109 \\
14.1304611078722	-1.80772020392377 \\
14.1235771751623	-1.91182356221184 \\
14.1163576552067	-2.0159730678949 \\
14.1088205116639	-2.12016869011258 \\
14.1009837081925	-2.22441039800446 \\
14.0928652084512	-2.32869816071017 \\
14.0844829760984	-2.43303194736932 \\
14.0758549747929	-2.5374117271215 \\
14.0669991681931	-2.64183746910633 \\
14.0579335199576	-2.74630914246342 \\
14.048001691297	-2.85083447556228 \\
14.0351059771269	-2.9554365743759 \\
14.0192564916243	-3.06011293684406 \\
14.0006349571472	-3.16485903787253 \\
13.9794230960533	-3.26967035236708 \\
13.9558026307004	-3.37454235523345 \\
13.9299552834462	-3.47947052137742 \\
13.9020627766485	-3.58445032570475 \\
13.8723068326652	-3.6894772431212 \\
13.8408691738539	-3.79454674853253 \\
13.8079315225724	-3.8996543168445 \\
13.7736756011787	-4.00479542296288 \\
13.7382831320303	-4.10996554179343 \\
13.7019358374851	-4.2151601482419 \\
13.6648154399009	-4.32037471721407 \\
13.6271036616354	-4.42560472361569 \\
13.5889822250464	-4.53084564235253 \\
13.5506328524917	-4.63609294833034 \\
13.5122372663291	-4.74134211645489 \\
13.4739771889164	-4.84658862163195 \\
13.4360343426113	-4.95182793876727 \\
13.3985904497716	-5.05705554276662 \\
13.361827232755	-5.16226690853575 \\
13.3258169577169	-5.26746199252783 \\
13.2892788328383	-5.37269413368664 \\
13.2518330570945	-5.47797549536362 \\
13.2135277583849	-5.58330021570107 \\
13.1744110646092	-5.68866243284135 \\
13.1345311036669	-5.79405628492677 \\
13.0939360034575	-5.89947591009968 \\
13.0526738918806	-6.0049154465024 \\
13.0107928968359	-6.11036903227727 \\
12.9683411462228	-6.21583080556661 \\
12.9253667679409	-6.32129490451277 \\
12.8819178898899	-6.42675546725806 \\
12.8380426399691	-6.53220663194483 \\
12.7937891460784	-6.63764253671541 \\
12.7492055361171	-6.74305731971213 \\
12.7043399379848	-6.84844511907731 \\
12.6592404795812	-6.9538000729533 \\
12.6139552888058	-7.05911631948243 \\
12.5685324935581	-7.16438799680702 \\
12.5230202217378	-7.2696092430694 \\
12.4774666012443	-7.37477419641192 \\
12.4319197599773	-7.4798769949769 \\
12.3864278258363	-7.58491177690668 \\
12.3410389267209	-7.68987268034358 \\
12.295593151949	-7.79487039268225 \\
12.2498982134097	-7.90000291564432 \\
12.2039536728594	-8.00524535886262 \\
12.1577590920542	-8.11057283197001 \\
12.1113140327506	-8.2159604445993 \\
12.0646180567047	-8.32138330638336 \\
12.0176707256729	-8.426816526955 \\
11.9704716014115	-8.53223521594706 \\
11.9230202456768	-8.6376144829924 \\
11.875316220225	-8.74292943772385 \\
11.8273590868125	-8.84815518977423 \\
11.7791484071956	-8.95326684877639 \\
11.7306837431306	-9.05823952436317 \\
11.6819646563736	-9.16304832616741 \\
11.6329907086812	-9.26766836382194 \\
11.5837614618095	-9.3720747469596 \\
11.5342764775148	-9.47624258521323 \\
11.4845353175534	-9.58014698821567 \\
11.4345375436817	-9.68376306559975 \\
11.3842827176559	-9.78706592699832 \\
11.3337704012324	-9.8900306820442 \\
11.2830001561673	-9.99263244037025 \\
11.2319715442171	-10.0948463116093 \\
11.1807155348331	-10.1967775870383 \\
11.1293234604141	-10.2988129857362 \\
11.0777808893819	-10.4009358232439 \\
11.0260651260596	-10.5030964104553 \\
10.9741534747706	-10.6052450582644 \\
10.9220232398381	-10.7073320775652 \\
10.8696517255855	-10.8093077792517 \\
10.8170162363361	-10.9111224742179 \\
10.764094076413	-11.0127264733578 \\
10.7108625501397	-11.1140700875653 \\
10.6572989618393	-11.2151036277344 \\
10.6033806158352	-11.3157774047592 \\
10.5490848164507	-11.4160417295336 \\
10.4943888680089	-11.5158469129515 \\
10.4392700748334	-11.615143265907 \\
10.3837057412472	-11.7138810992941 \\
10.3276731715737	-11.8120107240067 \\
10.2711496701361	-11.9094824509388 \\
10.2141125412579	-12.0062465909845 \\
10.1565390892621	-12.1022534550376 \\
10.0984066184722	-12.1974533539922 \\
10.0396924332114	-12.2917965987423 \\
9.98037383780303	-12.3852335001817 \\
9.92043804688644	-12.4777779089267 \\
9.85999397243531	-12.5701949018901 \\
9.79906722735214	-12.6626537914569 \\
9.73764447666001	-12.755066918532 \\
9.67571238538206	-12.8473466240204 \\
9.61325761854139	-12.9394052488271 \\
9.55026684116112	-13.0311551338571 \\
9.48672671826437	-13.1225086200153 \\
9.42262391487425	-13.2133780482068 \\
9.35794509601388	-13.3036757593365 \\
9.29267692670637	-13.3933140943094 \\
9.22680607197484	-13.4822053940305 \\
9.1603191968424	-13.5702619994048 \\
9.09320296633217	-13.6573962513372 \\
9.02544404546726	-13.7435204907328 \\
8.95702909927078	-13.8285470584965 \\
8.88794479276586	-13.9123882955334 \\
8.81817779097561	-13.9949565427484 \\
8.74771475892313	-14.0761641410464 \\
8.67654236163156	-14.1559234313326 \\
8.604647264124	-14.2341467545118 \\
8.53201613142357	-14.310746451489 \\
8.45863562855338	-14.3856348631693 \\
8.38449242053655	-14.4587243304576 \\
8.30936096993993	-14.5307024435877 \\
8.23305626992209	-14.6022703030346 \\
8.1556257813937	-14.6733605150644 \\
8.0771169652654	-14.7439056859434 \\
7.99757728244788	-14.8138384219379 \\
7.91705419385179	-14.8830913293142 \\
7.8355951603878	-14.9515970143385 \\
7.75324764296658	-15.0192880832772 \\
7.67005910249878	-15.0860971423965 \\
7.58607699989509	-15.1519567979627 \\
7.50134879606615	-15.2167996562421 \\
7.41592195192264	-15.2805583235009 \\
7.32984392837522	-15.3431654060054 \\
7.24316218633456	-15.404553510022 \\
7.15592418671132	-15.4646552418168 \\
7.06817739041616	-15.5234032076563 \\
6.97996925835975	-15.5807300138066 \\
6.89134725145277	-15.636568266534 \\
6.80235883060586	-15.6908505721048 \\
6.7130514567297	-15.7435095367853 \\
6.62347259073495	-15.7944777668418 \\
6.53366969353228	-15.8436878685405 \\
6.44369022603234	-15.8910724481478 \\
6.35339365556363	-15.9369885365998 \\
6.26221884169668	-15.9827306801601 \\
6.1701712878869	-16.0283006180808 \\
6.07730359109728	-16.0735932858393 \\
5.98366834829086	-16.1185036189132 \\
5.88931815643065	-16.1629265527798 \\
5.79430561247965	-16.2067570229167 \\
5.6986833134009	-16.2498899648013 \\
5.6025038561574	-16.2922203139109 \\
5.50581983771218	-16.3336430057232 \\
5.40868385502824	-16.3740529757155 \\
5.31114850506861	-16.4133451593654 \\
5.2132663847963	-16.4514144921501 \\
5.11509009117432	-16.4881559095473 \\
5.0166722211657	-16.5234643470343 \\
4.91806537173345	-16.5572347400886 \\
4.81932213984058	-16.5893620241877 \\
4.72049512245012	-16.619741134809 \\
4.62163691652508	-16.64826700743 \\
4.52280011902847	-16.6748345775281 \\
4.42403732692331	-16.6993387805808 \\
4.32540113717262	-16.7216745520655 \\
4.22694414673941	-16.7417368274597 \\
4.1286781435216	-16.7595542524226 \\
4.03012178297374	-16.7767735060637 \\
3.93115372685449	-16.7938665873041 \\
3.83181501726741	-16.8107740321549 \\
3.73214669631608	-16.8274363766273 \\
3.63218980610405	-16.8437941567326 \\
3.53198538873491	-16.8597879084819 \\
3.43157448631222	-16.8753581678864 \\
3.33099814093954	-16.8904454709574 \\
3.23029739472045	-16.904990353706 \\
3.12951328975851	-16.9189333521435 \\
3.02868686815728	-16.9322150022811 \\
2.92785917202035	-16.9447758401299 \\
2.82707124345127	-16.9565564017012 \\
2.72636412455362	-16.9674972230062 \\
2.62577885743096	-16.977538840056 \\
2.52535648418685	-16.986621788862 \\
2.42513804692488	-16.9946866054352 \\
2.3251645877486	-17.001673825787 \\
2.22547714876159	-17.0075239859284 \\
2.12611677206741	-17.0121776218708 \\
2.02712449976963	-17.0155752696253 \\
1.92854137397182	-17.0176574652031 \\
1.83040843677754	-17.0183647446154 \\
1.7324850745484	-17.0177636325143 \\
1.63452333966952	-17.015994827463 \\
1.53655327327972	-17.0131101263393 \\
1.43860491651783	-17.009161326021 \\
1.34070831052268	-17.0042002233861 \\
1.24289349643311	-16.9982786153123 \\
1.14519051538794	-16.9914482986775 \\
1.04762940852601	-16.9837610703597 \\
0.950240216986149	-16.9752687272366 \\
0.853052981907183	-16.9660230661861 \\
0.756097744427945	-16.956075884086 \\
0.659404545687265	-16.9454789778142 \\
0.563003426823975	-16.9342841442487 \\
0.466924428976905	-16.9225431802671 \\
0.371197593284886	-16.9103078827474 \\
0.275852960886749	-16.8976300485675 \\
0.180920572921323	-16.8845614746051 \\
0.0864304705274413	-16.8711539577382 \\
-0.00758730515606719	-16.8574592948446 \\
-0.101102712990371	-16.8435292828022 \\
-0.19408571183664	-16.8294157184887 \\
-0.286506260556043	-16.8151703987822 \\
-0.378334318009749	-16.8008451205604 \\
-0.469703822396294	-16.786078587269 \\
-0.561103345430582	-16.7695924985397 \\
-0.65251348242359	-16.7513617854777 \\
-0.743873261052247	-16.7314651508027 \\
-0.835121708993481	-16.7099812972343 \\
-0.92619785392422	-16.686988927492 \\
-1.01704072352139	-16.6625667442956 \\
-1.10758934546193	-16.6367934503645 \\
-1.19778274742275	-16.6097477484185 \\
-1.2875599570808	-16.5815083411772 \\
-1.37686000211299	-16.5521539313602 \\
-1.46562191019625	-16.521763221687 \\
-1.55378470900752	-16.4904149148774 \\
-1.64128742622372	-16.4581877136509 \\
-1.72806908952178	-16.4251603207272 \\
-1.81406872657863	-16.3914114388258 \\
-1.8992253650712	-16.3570197706664 \\
-1.98347803267641	-16.3220640189687 \\
-2.06676575707119	-16.2866228864521 \\
-2.14902756593248	-16.2507750758364 \\
-2.23020248693719	-16.2145992898412 \\
-2.31022954776226	-16.178174231186 \\
-2.38904777608462	-16.1415786025906 \\
-2.46669676380418	-16.1048100557249 \\
-2.54439608240174	-16.0668848907238 \\
-2.62240329427432	-16.0275302969175 \\
-2.70056700715325	-15.9867954830988 \\
-2.77873582876987	-15.9447296580607 \\
-2.85675836685549	-15.9013820305959 \\
-2.93448322914145	-15.8568018094975 \\
-3.01175902335907	-15.8110382035583 \\
-3.08843435723969	-15.7641404215712 \\
-3.16435783851462	-15.7161576723291 \\
-3.23937807491519	-15.6671391646249 \\
-3.31334367417275	-15.6171341072515 \\
-3.3861032440186	-15.5661917090017 \\
-3.45750539218408	-15.5143611786685 \\
-3.52739872640051	-15.4616917250447 \\
-3.59563185439923	-15.4082325569233 \\
-3.66205338391156	-15.3540328830972 \\
-3.72651192266883	-15.2991419123591 \\
-3.78885607840236	-15.2436088535021 \\
-3.84893445884349	-15.1874829153191 \\
-3.90659567172354	-15.1308133066028 \\
-3.96168832477383	-15.0736492361462 \\
-4.01406102572571	-15.0160399127422 \\
-4.06356238231048	-14.9580345451838 \\
-4.11178054437989	-14.8989970097219 \\
-4.16028111992662	-14.8383183599093 \\
-4.20887890389341	-14.7760885704908 \\
-4.25738869122301	-14.7123976162113 \\
-4.30562527685818	-14.6473354718156 \\
-4.35340345574165	-14.5809921120487 \\
-4.40053802281618	-14.5134575116555 \\
-4.44684377302451	-14.4448216453809 \\
-4.4921355013094	-14.3751744879697 \\
-4.53622800261359	-14.3046060141668 \\
-4.57893607187983	-14.2332061987172 \\
-4.62007450405087	-14.1610650163657 \\
-4.65945809406946	-14.0882724418572 \\
-4.69690163687834	-14.0149184499366 \\
-4.73221992742027	-13.9410930153489 \\
-4.76522776063799	-13.8668861128389 \\
-4.79573993147424	-13.7923877171514 \\
-4.82357123487179	-13.7176878030314 \\
-4.84853646577337	-13.6428763452239 \\
-4.87045041912174	-13.5680433184735 \\
-4.88912788985965	-13.4932786975254 \\
-4.90438367292983	-13.4186724571243 \\
-4.91603256327505	-13.3443145720152 \\
-4.92470517572796	-13.2700221076442 \\
-4.93294776360096	-13.19498520632 \\
-4.94094876951141	-13.1191983157665 \\
-4.94869588285328	-13.0427235959437 \\
-4.95617679302053	-12.9656232068115 \\
-4.96337918940715	-12.8879593083297 \\
-4.97029076140709	-12.8097940604584 \\
-4.97689919841433	-12.7311896231574 \\
-4.98319218982284	-12.6522081563867 \\
-4.9891574250266	-12.5729118201062 \\
-4.99478259341956	-12.4933627742759 \\
-5.00005538439572	-12.4136231788557 \\
-5.00496348734902	-12.3337551938054 \\
-5.00949459167345	-12.2538209790851 \\
-5.01363638676298	-12.1738826946547 \\
-5.01737656201158	-12.094002500474 \\
-5.02070280681321	-12.0142425565031 \\
-5.02360281056185	-11.9346650227018 \\
-5.02606426265147	-11.8553320590301 \\
-5.02807485247605	-11.776305825448 \\
-5.02962226942954	-11.6976484819152 \\
-5.03069420290593	-11.6194221883919 \\
-5.03127834229918	-11.5416891048378 \\
-5.03126597198688	-11.4644838364404 \\
-5.02941854373446	-11.3874919849647 \\
-5.02542258807158	-11.3106081595611 \\
-5.01935701248955	-11.2338326019331 \\
-5.01130072447967	-11.157165553784 \\
-5.00133263153324	-11.0806072568174 \\
-4.98953164114156	-11.0041579527365 \\
-4.97597666079593	-10.9278178832449 \\
-4.96074659798765	-10.8515872900459 \\
-4.94392036020802	-10.7754664148431 \\
-4.92557685494834	-10.6994554993398 \\
-4.90579498969991	-10.6235547852394 \\
-4.88465367195403	-10.5477645142454 \\
-4.862231809202	-10.4720849280612 \\
-4.83860830893513	-10.3965162683903 \\
-4.8138620786447	-10.321058776936 \\
-4.78807202582203	-10.2457126954018 \\
-4.7613170579584	-10.1704782654911 \\
-4.73367608254513	-10.0953557289074 \\
-4.70522800707351	-10.0203453273541 \\
-4.67605173903484	-9.94544730253452 \\
-4.64622618592042	-9.8706618961522 \\
-4.61583025522156	-9.79598934991053 \\
};
\addplot [line width=1.2pt, blue, dotted, forget plot]
table [row sep=\\]{%
4.23658623601368	-0.876356480612449 \\
4.26465892432579	-0.961374002758918 \\
4.29246800835488	-1.04637978334453 \\
4.32000659770427	-1.13137381992228 \\
4.34726780197728	-1.21635611004517 \\
4.37424473077721	-1.3013266512662 \\
4.4009304937074	-1.38628544113837 \\
4.42731820037114	-1.47123247721466 \\
4.45340096037176	-1.5561677570481 \\
4.47917188331258	-1.64109127819167 \\
4.50462407879689	-1.72600303819836 \\
4.52975065642803	-1.81090303462119 \\
4.55454472580931	-1.89579126501315 \\
4.57899939654403	-1.98066772692723 \\
4.60310777823553	-2.06553241791644 \\
4.6268629804871	-2.15038533553378 \\
4.65025811290207	-2.23522647733224 \\
4.67328628508375	-2.32005584086482 \\
4.69594060663546	-2.40487342368452 \\
4.71821418716051	-2.48967922334434 \\
4.74010013626222	-2.57447323739728 \\
4.7615915635439	-2.65925546339634 \\
4.78268157860887	-2.74402589889452 \\
4.80336329106044	-2.8287845414448 \\
4.82374042552355	-2.91354577453557 \\
4.84414822524408	-2.9983523879644 \\
4.86457063940398	-3.08319957623197 \\
4.88496326882267	-3.16807877887332 \\
4.90528171431959	-3.25298143542347 \\
4.92548157671418	-3.33789898541745 \\
4.94551845682585	-3.42282286839029 \\
4.96534795547405	-3.50774452387703 \\
4.9849256734782	-3.59265539141267 \\
5.00420721165774	-3.67754691053227 \\
5.0231481708321	-3.76241052077084 \\
5.04170415182071	-3.84723766166341 \\
5.05983075544301	-3.93201977274502 \\
5.07748358251842	-4.01674829355069 \\
5.09461823386637	-4.10141466361544 \\
5.1111903103063	-4.18601032247432 \\
5.12715541265765	-4.27052670966234 \\
5.14246914173983	-4.35495526471454 \\
5.15708709837229	-4.43928742716594 \\
5.17096488337445	-4.52351463655157 \\
5.18405809756574	-4.60762833240647 \\
5.19632234176561	-4.69161995426566 \\
5.20771321679348	-4.77548094166417 \\
5.21824557384012	-4.85920914828742 \\
5.22864415740716	-4.94288014667305 \\
5.23909809747037	-5.02651337373115 \\
5.24955924473384	-5.11010279703207 \\
5.25997944990163	-5.19364238414619 \\
5.27031056367783	-5.27712610264387 \\
5.2805044367665	-5.3605479200955 \\
5.29051291987173	-5.44390180407145 \\
5.30028786369758	-5.52718172214208 \\
5.30978111894815	-5.61038164187776 \\
5.3189445363275	-5.69349553084888 \\
5.3277299665397	-5.77651735662579 \\
5.33608926028884	-5.85944108677888 \\
5.34397426827899	-5.94226068887851 \\
5.35133684121422	-6.02497013049506 \\
5.35812882979862	-6.10756337919889 \\
5.36430208473625	-6.19003440256038 \\
5.3698084567312	-6.2723771681499 \\
5.37459979648753	-6.35458564353782 \\
5.37862795470934	-6.43665379629452 \\
5.38184478210068	-6.51857559399036 \\
5.38420212936564	-6.60034500419572 \\
5.38565184720829	-6.68195599448096 \\
5.38614578633272	-6.76340253241646 \\
5.3857284911172	-6.84474041933641 \\
5.38450069116811	-6.92602316396285 \\
5.38249851503168	-7.00724129494794 \\
5.3797580912541	-7.08838534094385 \\
5.3763155483816	-7.16944583060274 \\
5.37220701496039	-7.25041329257679 \\
5.36746861953667	-7.33127825551815 \\
5.36213649065666	-7.41203124807899 \\
5.35624675686658	-7.49266279891148 \\
5.34983554671264	-7.57316343666778 \\
5.34293898874104	-7.65352369000006 \\
5.335593211498	-7.73373408756048 \\
5.32783434352974	-7.81378515800122 \\
5.31969851338247	-7.89366742997442 \\
5.31122184960239	-7.97337143213227 \\
5.30244048073573	-8.05288769312692 \\
5.29339053532869	-8.13220674161053 \\
5.28410814192749	-8.21131910623529 \\
5.27462942907834	-8.29021531565335 \\
5.26499052532745	-8.36888589851687 \\
5.25522755922103	-8.44732138347803 \\
5.24537665930531	-8.52551229918898 \\
5.23547395412648	-8.6034491743019 \\
5.22526794788357	-8.68117498939491 \\
5.21386952926453	-8.75884595446939 \\
5.20126314398145	-8.83645556184945 \\
5.187505525404	-8.91398399938132 \\
5.17265340690188	-8.9914114549112 \\
5.15676352184477	-9.06871811628532 \\
5.13989260360235	-9.14588417134988 \\
5.12209738554432	-9.22288980795111 \\
5.10343460104035	-9.29971521393522 \\
5.08396098346013	-9.37634057714842 \\
5.06373326617335	-9.45274608543694 \\
5.0428081825497	-9.52891192664699 \\
5.02124246595885	-9.60481828862478 \\
4.99909284977049	-9.68044535921653 \\
4.97641606735432	-9.75577332626845 \\
4.95326885208001	-9.83078237762677 \\
4.92970793731725	-9.90545270113769 \\
4.90579005643572	-9.97976448464744 \\
4.88157194280512	-10.0536979160022 \\
4.85711032979512	-10.1272331830483 \\
4.83246195077542	-10.2003504736318 \\
4.80768353911569	-10.273029975599 \\
4.78283182818563	-10.3452518767961 \\
4.75791261845123	-10.4170243542493 \\
4.73231128283623	-10.4886851794378 \\
4.70585197091422	-10.5603068853464 \\
4.67856025569898	-10.6318484793133 \\
4.65046171020433	-10.7032689686771 \\
4.62158190744406	-10.7745273607759 \\
4.59194642043198	-10.8455826629483 \\
4.56158082218189	-10.9163938825326 \\
4.5305106857076	-10.9869200268671 \\
4.4987615840229	-11.0571201032902 \\
4.4663590901416	-11.1269531191404 \\
4.4333287770775	-11.196378081756 \\
4.3996962178444	-11.2653539984753 \\
4.36548698545611	-11.3338398766367 \\
4.33072665292643	-11.4017947235786 \\
4.29544079326916	-11.4691775466394 \\
4.25965497949811	-11.5359473531575 \\
4.22339478462707	-11.6020631504712 \\
4.18668578166985	-11.6674839459189 \\
4.14955354364025	-11.732168746839 \\
4.11202364355208	-11.7960765605698 \\
4.07412165441914	-11.8591663944497 \\
4.03587314925522	-11.9213972558172 \\
3.99730370107414	-11.9827281520105 \\
3.95812138181359	-12.0435836519992 \\
3.91803672929992	-12.1043727731502 \\
3.87707901537988	-12.165031195398 \\
3.8352775119002	-12.2254945986773 \\
3.79266149070764	-12.2856986629223 \\
3.74926022364894	-12.3455790680678 \\
3.70510298257085	-12.4050714940482 \\
3.66021903932011	-12.464111620798 \\
3.61463766574348	-12.5226351282517 \\
3.56838813368769	-12.5805776963439 \\
3.52149971499951	-12.637875005009 \\
3.47400168152566	-12.6944627341816 \\
3.4259233051129	-12.7502765637962 \\
3.37729385760798	-12.8052521737874 \\
3.32814261085764	-12.8593252440895 \\
3.27849883670862	-12.9124314546372 \\
3.22839180700769	-12.964506485365 \\
3.17785079360157	-13.0154860162073 \\
3.12690506833702	-13.0653057270988 \\
3.07558390306079	-13.1139012979738 \\
3.02391656961962	-13.161208408767 \\
2.97193233986026	-13.2071627394128 \\
2.91966048562945	-13.2516999698457 \\
2.86697148248325	-13.2950658981968 \\
2.81338656162047	-13.338201430738 \\
2.75891235608444	-13.3811141424141 \\
2.70359551988303	-13.4237339723089 \\
2.64748270702413	-13.4659908595063 \\
2.59062057151561	-13.5078147430899 \\
2.53305576736536	-13.5491355621435 \\
2.47483494858124	-13.5898832557509 \\
2.41600476917114	-13.6299877629959 \\
2.35661188314294	-13.6693790229622 \\
2.29670294450451	-13.7079869747336 \\
2.23632460726373	-13.7457415573938 \\
2.17552352542848	-13.7825727100267 \\
2.11434635300664	-13.818410371716 \\
2.05283974400609	-13.8531844815454 \\
1.9910503524347	-13.8868249785988 \\
1.92902483230035	-13.9192618019598 \\
1.86680983761092	-13.9504248907123 \\
1.80445202237429	-13.9802441839401 \\
1.74199804059834	-14.0086496207268 \\
1.67949454629094	-14.0355711401563 \\
1.61698819345997	-14.0609386813123 \\
1.55452563611332	-14.0846821832786 \\
1.49211582318799	-14.1068443788514 \\
1.42931249914508	-14.1288111456606 \\
1.36600292387891	-14.1509311610275 \\
1.30222449166081	-14.1730993324942 \\
1.23801459676212	-14.1952105676031 \\
1.17341063345417	-14.2171597738965 \\
1.1084499960083	-14.2388418589167 \\
1.04317007869585	-14.2601517302059 \\
0.97760827578815	-14.2809842953065 \\
0.911801981556537	-14.3012344617608 \\
0.84578859027235	-14.320797137111 \\
0.779605496206921	-14.3395672288995 \\
0.713290093631591	-14.3574396446685 \\
0.646879776817693	-14.3743092919604 \\
0.580411940036563	-14.3900710783173 \\
0.513923977559538	-14.4046199112817 \\
0.447453283657955	-14.4178506983958 \\
0.381037252603148	-14.4296583472019 \\
0.314713278666455	-14.4399377652424 \\
0.248518756119211	-14.4485838600594 \\
0.182491079232753	-14.4554915391953 \\
0.116667642278416	-14.4605557101924 \\
0.0510858395275365	-14.463671280593 \\
-0.0142169347485489	-14.4647331579393 \\
-0.0794879362249597	-14.4644369965285 \\
-0.144975550812267	-14.4635617309642 \\
-0.21064407959231	-14.4621271892491 \\
-0.276457823646931	-14.4601531993857 \\
-0.342381084057969	-14.4576595893764 \\
-0.408378161907267	-14.4546661872238 \\
-0.474413358276664	-14.4511928209304 \\
-0.540450974248001	-14.4472593184986 \\
-0.606455310903118	-14.442885507931 \\
-0.672390669323858	-14.4380912172302 \\
-0.73822135059206	-14.4328962743985 \\
-0.803911655789564	-14.4273205074386 \\
-0.869425885998213	-14.4213837443529 \\
-0.934728342299845	-14.4151058131439 \\
-0.999783325776303	-14.4085065418142 \\
-1.06455513750943	-14.4016057583662 \\
-1.12900807858106	-14.3944232908025 \\
-1.19310645007303	-14.3869789671256 \\
-1.2568145530672	-14.3792926153379 \\
-1.32009668864539	-14.371384063442 \\
-1.38291715788946	-14.3632731394405 \\
-1.44524026188123	-14.3549796713357 \\
-1.50703030170256	-14.3465234871303 \\
-1.56842949227229	-14.3374611984408 \\
-1.62996724364613	-14.3263489468415 \\
-1.69162454005266	-14.3131793019089 \\
-1.75333736282138	-14.298062307989 \\
-1.81504169328176	-14.2811080094279 \\
-1.87667351276329	-14.2624264505716 \\
-1.93816880259546	-14.2421276757661 \\
-1.99946354410774	-14.2203217293575 \\
-2.06049371862962	-14.1971186556918 \\
-2.12119530749059	-14.1726284991151 \\
-2.18150429202012	-14.1469613039733 \\
-2.24135665354771	-14.1202271146126 \\
-2.30068837340283	-14.092535975379 \\
-2.35943543291496	-14.0639979306184 \\
-2.4175338134136	-14.034723024677 \\
-2.47491949622823	-14.0048213019007 \\
-2.53152846268832	-13.9744028066356 \\
-2.58729669412337	-13.9435775832278 \\
-2.64216017186286	-13.9124556760233 \\
-2.69605487723626	-13.8811471293681 \\
-2.74891679157307	-13.8497619876082 \\
-2.80068189620277	-13.8184102950898 \\
-2.85128617245484	-13.7872020961587 \\
-2.90074064594965	-13.7561551978576 \\
-2.94995111232011	-13.7241649737168 \\
-2.99913572544033	-13.6909238130253 \\
-3.04821102165138	-13.6564891578188 \\
-3.09709353729432	-13.620918450133 \\
-3.14569980871019	-13.5842691320036 \\
-3.19394637224008	-13.5465986454663 \\
-3.24174976422503	-13.5079644325568 \\
-3.2890265210061	-13.4684239353107 \\
-3.33569317892437	-13.4280345957639 \\
-3.38166627432088	-13.3868538559518 \\
-3.4268623435367	-13.3449391579104 \\
-3.4711979229129	-13.3023479436752 \\
-3.51458954879052	-13.2591376552819 \\
-3.55695375751064	-13.2153657347663 \\
-3.59820708541431	-13.171089624164 \\
-3.63826606884259	-13.1263667655108 \\
-3.67704724413655	-13.0812546008422 \\
-3.71446714763724	-13.0358105721941 \\
-3.75044231568572	-12.9900921216021 \\
-3.78488928462307	-12.9441566911019 \\
-3.81772459079033	-12.8980617227293 \\
-3.84886477052857	-12.8518646585198 \\
-3.87822636017884	-12.8056229405092 \\
-3.90674344373362	-12.7589002789039 \\
-3.93533829673124	-12.7112578987182 \\
-3.96390974283159	-12.6627495141999 \\
-3.99235660569456	-12.6134288395965 \\
-4.02057770898003	-12.5633495891557 \\
-4.04847187634789	-12.5125654771251 \\
-4.07593793145803	-12.4611302177524 \\
-4.10287469797034	-12.4090975252853 \\
-4.1291809995447	-12.3565211139714 \\
-4.15475565984099	-12.3034546980583 \\
-4.17949750251912	-12.2499519917937 \\
-4.20330535123895	-12.1960667094252 \\
-4.22607802966039	-12.1418525652006 \\
-4.24771436144332	-12.0873632733673 \\
-4.26811317024761	-12.0326525481731 \\
-4.28717327973317	-11.9777741038657 \\
-4.30479351355988	-11.9227816546927 \\
-4.32087269538763	-11.8677289149016 \\
-4.33530964887629	-11.8126695987403 \\
-4.34800319768576	-11.7576574204562 \\
-4.35885216547593	-11.7027460942972 \\
-4.36775537590669	-11.6479893345107 \\
-4.37461165263791	-11.5934408553445 \\
-4.37978051538528	-11.5390082146135 \\
-4.38470139504916	-11.4842586887286 \\
-4.38948018082878	-11.4291852224373 \\
-4.39410932171179	-11.3738168424183 \\
-4.39858126668585	-11.3181825753502 \\
-4.40288846473861	-11.2623114479114 \\
-4.40702336485772	-11.2062324867806 \\
-4.41097841603083	-11.1499747186363 \\
-4.4147460672456	-11.093567170157 \\
-4.41831876748968	-11.0370388680215 \\
-4.42168896575072	-10.9804188389081 \\
-4.42484911101638	-10.9237361094956 \\
-4.4277916522743	-10.8670197064624 \\
-4.43050903851215	-10.8102986564871 \\
-4.43299371871756	-10.7536019862483 \\
-4.43523814187821	-10.6969587224246 \\
-4.43723475698173	-10.6403978916945 \\
-4.43897601301578	-10.5839485207365 \\
-4.44045435896802	-10.5276396362293 \\
-4.44166224382609	-10.4715002648515 \\
-4.44259211657765	-10.4155594332815 \\
-4.44323642621035	-10.359846168198 \\
-4.44358762171185	-10.3043894962795 \\
-4.44357919634798	-10.2492055095209 \\
-4.44245371218211	-10.1941465321616 \\
-4.44001947487512	-10.1391631285443 \\
-4.43632474733997	-10.0842554638585 \\
-4.43141779248964	-10.0294237032935 \\
-4.42534687323709	-9.97466801203859 \\
-4.4181602524953	-9.91998855528305 \\
-4.40990619317724	-9.86538549821624 \\
-4.40063295819588	-9.81085900602745 \\
-4.39038881046419	-9.75640924390602 \\
-4.37922201289516	-9.70203637704126 \\
-4.36718082840173	-9.64774057062248 \\
-4.3543135198969	-9.59352198983901 \\
-4.34066835029363	-9.53938079988016 \\
-4.32629358250489	-9.48531716593525 \\
-4.31123747944365	-9.43133125319359 \\
-4.2955483040229	-9.37742322684451 \\
-4.27927431915558	-9.32359325207731 \\
-4.26246378775469	-9.26984149408133 \\
-4.24516497273319	-9.21616811804587 \\
-4.22742613700406	-9.16257328916025 \\
-4.20929554348025	-9.10905717261379 \\
-4.19082145507476	-9.05561993359581 \\
};

\addplot [color1, line width=2.0pt, dashed, forget plot]
table [row sep=\\]{%
-4.80556194313644	-9.58679303186979 \\
-4.77158329167161	-9.5297037824758 \\
-4.73760489207213	-9.47261438318655 \\
-4.70362674536103	-9.41552483339157 \\
-4.66964884598998	-9.3584351363947 \\
-4.63567119296355	-9.30134529278699 \\
-4.60169378122457	-9.24425530557997 \\
-4.56771660985194	-9.18716517532046 \\
-4.53373967411042	-9.13007490482812 \\
-4.49976297236071	-9.07298449507776 \\
-4.4657865009821	-9.01589394822489 \\
-4.43181025695521	-8.95880326606665 \\
-4.39783423817414	-8.901712449856 \\
-4.36385844022851	-8.84462150221885 \\
-4.32988286202107	-8.78753042380721 \\
-4.29590749852212	-8.73043921761594 \\
-4.26193234885741	-8.67334788416425 \\
-4.2279574081706	-8.61625642634367 \\
-4.19398267501769	-8.55916484501286 \\
-4.16000814550847	-8.50207314248778 \\
-4.12603381691384	-8.44498132039264 \\
-4.09205968687026	-8.387889380134 \\
-4.05808575117149	-8.33079732421595 \\
-4.02411200859051	-8.27370515336807 \\
-3.99013845415304	-8.21661287055192 \\
-3.95616508700373	-8.15952047627573 \\
-3.92219190219303	-8.10242797348629 \\
-3.88821889844444	-8.0453353629427 \\
-3.854246071626	-7.9882426471048 \\
-3.82027341928509	-7.93114982743215 \\
-3.78630093878646	-7.87405690549318 \\
-3.75232862615598	-7.81696388365363 \\
-3.71835648000894	-7.75987076273717 \\
-3.68438449545437	-7.70277754565553 \\
-3.65041267162795	-7.64568423292251 \\
-3.61644100351479	-7.5885908275236 \\
-3.582469489978	-7.53149733013493 \\
-3.54849812667175	-7.47440374334356 \\
-3.514526911406	-7.41731006845278 \\
-3.48055584125978	-7.36021630720117 \\
-3.44658491251905	-7.30312246179949 \\
-3.4126141236134	-7.24602853318214 \\
-3.37864346976318	-7.18893452419439 \\
-3.34467295006712	-7.13184043537222 \\
-3.31070255947292	-7.07474626972321 \\
-3.27673229695543	-7.01765202785714 \\
-3.24276215798278	-6.9605577124717 \\
-3.20879214061364	-6.90346332472216 \\
-3.17482224162729	-6.84636886652558 \\
-3.14085245759774	-6.78927433992116 \\
-3.10688278674096	-6.73217974597059 \\
-3.07291322441652	-6.67508508743611 \\
-3.03894376965829	-6.61799036489247 \\
-3.00497441740448	-6.56089558135275 \\
-2.97100516671379	-6.50380073737695 \\
-2.93703601289616	-6.44670583575681 \\
-2.90306695424196	-6.38961087750976 \\
-2.86909798722605	-6.33251586473403 \\
-2.83512910872912	-6.27542079928626 \\
-2.80116031672867	-6.21832568237015 \\
-2.76719160675145	-6.16123051664832 \\
-2.73322297773852	-6.10413530275089 \\
-2.69925442464655	-6.04704004367983 \\
-2.66528594659011	-5.98994473996199 \\
-2.63131753874893	-5.93284939446653 \\
-2.59734919961794	-5.87575400808918 \\
-2.56338092539311	-5.81865858309415 \\
-2.52941271325024	-5.7615631211624 \\
-2.49544456091359	-5.70446762364843 \\
-2.46147646410502	-5.64737209309863 \\
-2.42750842164488	-5.59027653021509 \\
-2.39354042853616	-5.53318093797208 \\
-2.35957248392147	-5.47608531687987 \\
-2.32560458287816	-5.41898966986848 \\
-2.29163672407787	-5.3618939977285 \\
-2.25766890346553	-5.30479830287355 \\
-2.22370111849816	-5.24770258681721 \\
-2.18973336663278	-5.19060685107304 \\
-2.1557656438143	-5.13351109805467 \\
-2.1217979487144	-5.07641532855269 \\
-2.08783027641039	-5.01931954549711 \\
-2.05386262604491	-4.96222374939821 \\
-2.01989499262091	-4.90512794323026 \\
-1.98592737495879	-4.84803212769534 \\
-1.95195976878037	-4.79093630533985 \\
-1.91799217180993	-4.7338404775183 \\
-1.88402458122329	-4.67674464591163 \\
-1.85005699321635	-4.61964881278405 \\
-1.81608940628416	-4.56255297903131 \\
-1.78212181560628	-4.50545714752254 \\
-1.74815422029748	-4.44836131878463 \\
-1.71418661531423	-4.39126549581951 \\
-1.68021899959777	-4.33416967925733 \\
-1.6462513686747	-4.27707387176068 \\
-1.61228372052261	-4.21997807453329 \\
-1.57831605202219	-4.16288228943179 \\
-1.54434835964821	-4.10578651855439 \\
-1.51038064169121	-4.04869076291857 \\
-1.4764128934609	-3.991595025316 \\
-1.44244511401626	-3.93449930630675 \\
-1.40847729829518	-3.87740360890383 \\
-1.37450944533184	-3.82030793368207 \\
-1.34054155048556	-3.76321228340364 \\
-1.30657361197247	-3.70611665913026 \\
-1.27260562636653	-3.64902106290115 \\
-1.23863759044695	-3.59192549663331 \\
-1.2046695022726	-3.53482996148208 \\
-1.1707013573113	-3.47773446014509 \\
-1.13673315453828	-3.42063899323219 \\
-1.10276488890082	-3.36354356375086 \\
-1.06879655949808	-3.30644817223719 \\
-1.03482816155001	-3.24935282153636 \\
-1.00085969348651	-3.19225751258283 \\
-0.966891151593368	-3.13516224758732 \\
-0.932922532949607	-3.07806702828844 \\
-0.898953835365402	-3.02097185598948 \\
-0.864985054494545	-2.96387673327744 \\
-0.831016189200622	-2.90678166082856 \\
-0.797047234468214	-2.84968664162821 \\
-0.763078189433538	-2.79259167619031 \\
-0.729109049205119	-2.73549676742646 \\
-0.695139812398663	-2.67840191616045 \\
-0.661170475039766	-2.62130712475794 \\
-0.627201034493251	-2.56421239478737 \\
-0.593231488306666	-2.50711772770837 \\
-0.559261832347682	-2.45002312598068 \\
-0.525292065340327	-2.39292859036349 \\
-0.491322182334409	-2.33583412380349 \\
-0.457352182475266	-2.27873972680904 \\
-0.423382060787941	-2.22164540234156 \\
-0.389411816045999	-2.16455115113074 \\
-0.355441444042785	-2.10745697568061 \\
-0.32147094241493	-2.05036287739775 \\
-0.287500308433452	-1.99326885790638 \\
-0.25352953820777	-1.93617491952241 \\
-0.219558630294459	-1.8790810631046 \\
-0.185587579836467	-1.82198729154434 \\
-0.151616385960324	-1.764893605361 \\
-0.117645043635533	-1.70780000754928 \\
-0.08367355176557	-1.65070649876133 \\
-0.0497019059394786	-1.59361308162296 \\
-0.0157301040516963	-1.53651975738718 \\
0.0182418569171809	-1.47942652785112 \\
0.0522139805881392	-1.42233339517025 \\
0.0861862685999254	-1.36524036031949 \\
0.120158725688557	-1.30814742611835 \\
0.154131352774229	-1.25105459311382 \\
0.188104155579844	-1.193961864713 \\
0.222077133847292	-1.1368692407614 \\
0.2560502872635	-1.07977672107354 \\
};

\addplot [white, line width=1.0pt, dashed, forget plot]
table [row sep=\\]{%
-4.80556194313644	-9.58679303186979 \\
-4.77158329167161	-9.5297037824758 \\
-4.73760489207213	-9.47261438318655 \\
-4.70362674536103	-9.41552483339157 \\
-4.66964884598998	-9.3584351363947 \\
-4.63567119296355	-9.30134529278699 \\
-4.60169378122457	-9.24425530557997 \\
-4.56771660985194	-9.18716517532046 \\
-4.53373967411042	-9.13007490482812 \\
-4.49976297236071	-9.07298449507776 \\
-4.4657865009821	-9.01589394822489 \\
-4.43181025695521	-8.95880326606665 \\
-4.39783423817414	-8.901712449856 \\
-4.36385844022851	-8.84462150221885 \\
-4.32988286202107	-8.78753042380721 \\
-4.29590749852212	-8.73043921761594 \\
-4.26193234885741	-8.67334788416425 \\
-4.2279574081706	-8.61625642634367 \\
-4.19398267501769	-8.55916484501286 \\
-4.16000814550847	-8.50207314248778 \\
-4.12603381691384	-8.44498132039264 \\
-4.09205968687026	-8.387889380134 \\
-4.05808575117149	-8.33079732421595 \\
-4.02411200859051	-8.27370515336807 \\
-3.99013845415304	-8.21661287055192 \\
-3.95616508700373	-8.15952047627573 \\
-3.92219190219303	-8.10242797348629 \\
-3.88821889844444	-8.0453353629427 \\
-3.854246071626	-7.9882426471048 \\
-3.82027341928509	-7.93114982743215 \\
-3.78630093878646	-7.87405690549318 \\
-3.75232862615598	-7.81696388365363 \\
-3.71835648000894	-7.75987076273717 \\
-3.68438449545437	-7.70277754565553 \\
-3.65041267162795	-7.64568423292251 \\
-3.61644100351479	-7.5885908275236 \\
-3.582469489978	-7.53149733013493 \\
-3.54849812667175	-7.47440374334356 \\
-3.514526911406	-7.41731006845278 \\
-3.48055584125978	-7.36021630720117 \\
-3.44658491251905	-7.30312246179949 \\
-3.4126141236134	-7.24602853318214 \\
-3.37864346976318	-7.18893452419439 \\
-3.34467295006712	-7.13184043537222 \\
-3.31070255947292	-7.07474626972321 \\
-3.27673229695543	-7.01765202785714 \\
-3.24276215798278	-6.9605577124717 \\
-3.20879214061364	-6.90346332472216 \\
-3.17482224162729	-6.84636886652558 \\
-3.14085245759774	-6.78927433992116 \\
-3.10688278674096	-6.73217974597059 \\
-3.07291322441652	-6.67508508743611 \\
-3.03894376965829	-6.61799036489247 \\
-3.00497441740448	-6.56089558135275 \\
-2.97100516671379	-6.50380073737695 \\
-2.93703601289616	-6.44670583575681 \\
-2.90306695424196	-6.38961087750976 \\
-2.86909798722605	-6.33251586473403 \\
-2.83512910872912	-6.27542079928626 \\
-2.80116031672867	-6.21832568237015 \\
-2.76719160675145	-6.16123051664832 \\
-2.73322297773852	-6.10413530275089 \\
-2.69925442464655	-6.04704004367983 \\
-2.66528594659011	-5.98994473996199 \\
-2.63131753874893	-5.93284939446653 \\
-2.59734919961794	-5.87575400808918 \\
-2.56338092539311	-5.81865858309415 \\
-2.52941271325024	-5.7615631211624 \\
-2.49544456091359	-5.70446762364843 \\
-2.46147646410502	-5.64737209309863 \\
-2.42750842164488	-5.59027653021509 \\
-2.39354042853616	-5.53318093797208 \\
-2.35957248392147	-5.47608531687987 \\
-2.32560458287816	-5.41898966986848 \\
-2.29163672407787	-5.3618939977285 \\
-2.25766890346553	-5.30479830287355 \\
-2.22370111849816	-5.24770258681721 \\
-2.18973336663278	-5.19060685107304 \\
-2.1557656438143	-5.13351109805467 \\
-2.1217979487144	-5.07641532855269 \\
-2.08783027641039	-5.01931954549711 \\
-2.05386262604491	-4.96222374939821 \\
-2.01989499262091	-4.90512794323026 \\
-1.98592737495879	-4.84803212769534 \\
-1.95195976878037	-4.79093630533985 \\
-1.91799217180993	-4.7338404775183 \\
-1.88402458122329	-4.67674464591163 \\
-1.85005699321635	-4.61964881278405 \\
-1.81608940628416	-4.56255297903131 \\
-1.78212181560628	-4.50545714752254 \\
-1.74815422029748	-4.44836131878463 \\
-1.71418661531423	-4.39126549581951 \\
-1.68021899959777	-4.33416967925733 \\
-1.6462513686747	-4.27707387176068 \\
-1.61228372052261	-4.21997807453329 \\
-1.57831605202219	-4.16288228943179 \\
-1.54434835964821	-4.10578651855439 \\
-1.51038064169121	-4.04869076291857 \\
-1.4764128934609	-3.991595025316 \\
-1.44244511401626	-3.93449930630675 \\
-1.40847729829518	-3.87740360890383 \\
-1.37450944533184	-3.82030793368207 \\
-1.34054155048556	-3.76321228340364 \\
-1.30657361197247	-3.70611665913026 \\
-1.27260562636653	-3.64902106290115 \\
-1.23863759044695	-3.59192549663331 \\
-1.2046695022726	-3.53482996148208 \\
-1.1707013573113	-3.47773446014509 \\
-1.13673315453828	-3.42063899323219 \\
-1.10276488890082	-3.36354356375086 \\
-1.06879655949808	-3.30644817223719 \\
-1.03482816155001	-3.24935282153636 \\
-1.00085969348651	-3.19225751258283 \\
-0.966891151593368	-3.13516224758732 \\
-0.932922532949607	-3.07806702828844 \\
-0.898953835365402	-3.02097185598948 \\
-0.864985054494545	-2.96387673327744 \\
-0.831016189200622	-2.90678166082856 \\
-0.797047234468214	-2.84968664162821 \\
-0.763078189433538	-2.79259167619031 \\
-0.729109049205119	-2.73549676742646 \\
-0.695139812398663	-2.67840191616045 \\
-0.661170475039766	-2.62130712475794 \\
-0.627201034493251	-2.56421239478737 \\
-0.593231488306666	-2.50711772770837 \\
-0.559261832347682	-2.45002312598068 \\
-0.525292065340327	-2.39292859036349 \\
-0.491322182334409	-2.33583412380349 \\
-0.457352182475266	-2.27873972680904 \\
-0.423382060787941	-2.22164540234156 \\
-0.389411816045999	-2.16455115113074 \\
-0.355441444042785	-2.10745697568061 \\
-0.32147094241493	-2.05036287739775 \\
-0.287500308433452	-1.99326885790638 \\
-0.25352953820777	-1.93617491952241 \\
-0.219558630294459	-1.8790810631046 \\
-0.185587579836467	-1.82198729154434 \\
-0.151616385960324	-1.764893605361 \\
-0.117645043635533	-1.70780000754928 \\
-0.08367355176557	-1.65070649876133 \\
-0.0497019059394786	-1.59361308162296 \\
-0.0157301040516963	-1.53651975738718 \\
0.0182418569171809	-1.47942652785112 \\
0.0522139805881392	-1.42233339517025 \\
0.0861862685999254	-1.36524036031949 \\
0.120158725688557	-1.30814742611835 \\
0.154131352774229	-1.25105459311382 \\
0.188104155579844	-1.193961864713 \\
0.222077133847292	-1.1368692407614 \\
0.2560502872635	-1.07977672107354 \\
};
\end{axis}

\end{tikzpicture}
\caption{Log-histogram of actual arrival tracks to KJFK 04R and associated flight procedures.}
\label{fig:jfk_04R_orginal_paths}
\end{figure}
