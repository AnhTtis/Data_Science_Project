\section{Conclusions}
\label{sec:conclusions}

In this paper, we presented a method for modeling aircraft behavior in terminal airspace based on radar flight tracks and procedure data. 
We first partitioned the aircraft trajectories into segments based on the structures of flight procedures.
Then, we investigated Gaussian mixture model (GMM) and conditional Gaussian distributions 
for modeling the deviations of aircraft trajectories from their intended flight procedures.
A low-rank approximation was performed on the covariance matrices in GMMs to remove 
feature redundancy and improve the robustness of our model. 
Furthermore, we proposed an idea of modeling pairwise correlations between aircraft which can efficiently generate traffic scenes involving an arbitrary number of aircraft.
The results showed that the proposed models are able to capture the major patterns as well as uncertainties in aircraft behavior relative to the flight procedures. Furthermore, since the models used are simple, they are more robust, faster to train, and more reliable than their complex counterparts.%\todo {add a statement about the advantages which result from these models being simple?}

% \soyeonstrike{
% The proposed method has potential limitations that can be addressed in future work.
% First, it is unable to incorporate different characteristics of each procedure.
% Thus, the trained model is not generalizable to test procedures that have distinctively different characteristics than the training data, as discussed in Section V-B. % \ref{subsec:experiment_single_trajectory_model}.
% While the significance of this work lies in its ability to generate trajectories given any procedural data, 
% further study should investigate how the learning can be transferred across different traffic environments.
% Second, the multiple trajectory model does not assess whether the trajectories in a given scenario are correlated.
% This paper constructs the set of correlated trajectory data by extracting three aircraft with certain inter-arrival times.
% Future study should analyze how differently the pairwise model and single trajectory model behave for larger numbers of aircraft.
% Finally, this paper defines the deviations between trajectory points from the procedure points as the difference in east-north-up positions.
% Future work could reduce the size of state space and improve the model by using other definitions for the deviation, such as distance and relative bearing.
% }


There are many potential directions for extending this work. First, while the significance of this work lies in its ability to generate trajectories given any procedural data, the proposed model is not generalizable to flight procedures that have distinctively different characteristics from the ones in the training data. Future study should investigate how the knowledge can be transferred across different traffic environments incorporating minimal procedural information into the model. 
Second, further research could be done to reduce the model complexity without sacrificing performance. Using an autoencoder instead of PPCA would further reduce the feature space dimension as it can encode non-linear relationships between variables. Although they are difficult to interpret, transformer-based architectures also demonstrate promising potential for capturing complex patterns. %\todo{should this be refreshed, e.g. with mentions of transformer models?}  
Using distances and relative bearings instead of the ENU coordinates to define the deviations could better represent the aircraft behavior, and thus comprise a more explainable feature set. 
Finally, this work distinguished modeling single trajectories and their pairwise correlations in two separate problems. Future work could evaluate these two models on various scenarios with longer durations and different numbers of aircraft involved to explore how they can be integrated.
