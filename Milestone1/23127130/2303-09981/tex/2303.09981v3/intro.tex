\section{Introduction}
\lettrine{D}{eveloping} and assessing air traffic management (ATM) systems is a key step towards increasing the safety of air transport that requires probabilistically modeling future aircraft trajectories based on past data.
The Federal Aviation Administration's (FAA) Next Generation Air Transportation System (NextGen) \cite{FAAnextgen} and EUROCONTROL's Single European Sky ATM Research Program (SESAR) \cite{Sesar}, which have led the modernization of global air traffic management (ATM) systems, are examples of such model-dependent ATM programs.
The concept of trajectory-based operations (TBO) is a cornerstone of these systems, managing air traffic based on four-dimensional trajectory (4DT) information which consists of a series of three-dimensional coordinates (latitude, longitude, and altitude) with an added time variable. However, open-source, real-world flight data is limited or inaccessible. This necessitates
% TBO will provide increased levels of automation and safety by assessing the aircraft's current and future positions.
constructing robust aircraft trajectory models, which are important for the successful implementation of modern ATM systems. 
These trajectory models can be useful for developing new concepts of operation, flight planning, aircraft scheduling, and conflict detection. All of these tasks are critical for the safety, efficiency, and predictability of air transport and are dependent on effective trajectory models.

Trajectory models should be able to represent the behavior of aircraft following real operations.
Aircraft operated under instrument flight rules (IFR) are instructed to follow standard flight procedures, defined as a series of virtual waypoints.
However, the actual paths flown by each aircraft vary due to various factors such as pilot behavior, weather, and preferences of air traffic controllers. 
Aircraft trajectory models should capture both this variability and the general tendency of aircraft to follow procedures.

An additional factor these models must account for is that the variability of aircraft behavior differs between flight stages.
For example, during the initial approach, aircraft tend to have high variability in following procedures because they are converging from multiple directions to the landing runway, and they can be radar vectored to the final approach course. By contrast, aircraft on the final approach have low variability because they need to stay aligned with the runway to land safely.


%%%%%%%%%%%%%%%%%%%%%%%%%%%%%%%%%%%%%%%%%%%%%%%%%%%%%%%%%%%%%%%%%
Various approaches have been proposed for modeling aircraft trajectories.
Traditional methods explicitly model aircraft behavior based on kinematic equations of motion.
Several studies predicted the nominal trajectory by propagating the state estimates into the future \cite{chatterji1996route, chatterji1999short} or to the next flight segments \cite{slattery1997trajectory}.
Aircraft performance models, such as Base of Aircraft Data (BADA) \cite{nuic2010user}, were also developed based on aircraft dynamics models for trajectory simulation and prediction.
These physics-based approaches do not account for the uncertainties that lie in aircraft trajectory prediction \cite{yepes2007new}. Furthermore, they are unable to capture the multimodality that is seen in real-world data.

Other efforts have developed probabilistic methods for modeling aircraft behavior.
Some studies involved learning aircraft dynamics or navigational intent based on dynamic Bayesian networks to account for the uncertainty in the future states of aircraft \cite{liu2011probabilistic,kochenderfer2008uncorrelated,weinert2013uncorrelated,  lowe2015learning, mahboubi2017learning}.
Researchers also applied supervised learning algorithms such as spline functions \cite{kun20084}, neural networks \cite{le1999using, hamed2013statistical}, and generalized linear models (GLM) \cite{de2013machine} to predict future trajectories or the estimated times of arrival. Although these models consider uncertainty, they do not capture how such uncertainty varies depending on flight stage.%\todo{should we add a statement here regarding the limitations of these models?}

The previously applied approaches offer interpretability and efficiency; however, they present relatively simple models that do not take advantage of the full amount of historical data available. Work that addresses these limitations includes recent studies that present recurrent neural networks (RNNs) to learn spatiotemporal patterns of aircraft trajectories \cite{liu2018predicting, pang2019aircraft}.
Another study applied generative adversarial imitation learning (GAIL) to learn the optimal policy given historical trajectories from expert demonstrations \cite{bastas2020data}.
Other researchers proposed unsupervised clustering algorithms such as $k$-means clustering \cite{gariel2011trajectory}, hierarchical clustering \cite{hong2015trajectory}, and Gaussian mixture models (GMMs) \cite{mahboubi2017learning, Barratt2019}. 
Some of them clustered turning points extracted from trajectories represented as a sequence of these clusters \cite{gariel2011trajectory} or transitions from one another \cite{mahboubi2017learning}.
Others clustered the entire trajectories directly using position measurements \cite{hong2015trajectory, conde2016trajectory, Barratt2019}. An advantage of such clustering-based approaches is that they can be used to generate synthetic trajectories based on historical data. However, clustering approaches define normative behavior as that which is most similar to previous observation and thus do not account for situations when deviations are more or less appropriate.

Recent work has leveraged long short-term memory (LSTM) networks for the trajectory prediction task 
 \cite{schimpf2023generalized, shi20204, yang2023aircraft}. To address instances when Automatic Dependent Surveillance Broadcast (ADS-B) technology is unavailable, such as in the event of an onboard equipment failure, researchers train a bidirectional LSTM on historical ADS-B data \cite{yang2023aircraft}. Flights may also deviate from expected patterns due to inclement weather, a challenge that researchers have approached by including weather data in their LSTM model \cite{schimpf2023generalized}. Others segment trajectories into climbing, cruising, and descending/approaching phases, training a unique LSTM network for each phase to allow optimizing for that phase's distinct characteristics \cite{shi20204}. Like the method we propose, such LSTM-based approaches can be used to produce synthetic trajectories. In our work, we also segment trajectories; however, the stages selected are different, as described in Section \ref{sec:dataset}. 

 Encoder-decoder-based architectures, such as variational autoencoders (VAE) and transformers have also been applied to trajectory prediction \cite{dong2023tcn, krauth2023deep, guo2022flightbert, pang2022bayesian}. When applying VAE networks to the trajectory prediction task, some researchers focus on the Terminal Manoeuvre area near Zurich airport, where aircraft are frequently observed deviating from nominal approach procedures \cite{krauth2023deep}. Their work shows these models to be highly effective in handling this uncertainty. To address trajectory prediction for multiple aircraft, other researchers use a Bayesian spatiotemporal graph transformer model \cite{pang2022bayesian}. This model combines a spatial transformer with a temporal transformer, adding Bayesian linear layers to the decoder to model trajectory uncertainty. Another hybrid approach combines a temporal convolutional network (TCN) with a transformer-based Informer model \cite{dong2023tcn}. Rather than formulating trajectory prediction as a time sequence prediction problem, other researchers define the task as a Multi Binary Classification (MBC) problem by transforming input and output trajectories into Binary Encoding (BE) representations \cite{guo2022flightbert}. Each scalar value of a flight trajectory attribute is converted into this format. Then, a Transformer embeds these binary representations and a predictor network makes a series of binary classifications to produce the output trajectory. These large models are powerful; however, such approaches are also highly complex and require tuning many hyper-parameters. 
 
 Complex, neural network models such as the RNN, LSTM, VAE, and Transformer-based models discussed all demonstrate promising results, but introduce potential for overfitting and decrease the potential for interpretability. Such complex models are best used when a simpler model is unable to capture the complexity seen in the real-world data modeled. In our work, we propose leveraging known flight procedures as the basis for a simpler, GMM-based model. We find that despite its lesser complexity, this model remains effective in modeling the flight trajectory data studied. 
 
Much of the prior work on trajectory models has not focused on the relationship between aircraft trajectories and flight procedures. The procedures, however, are the basis of air traffic control that facilitates control over many aircraft.
In our recent work \cite{Jung2019}, we proposed a probabilistic model that learns aircraft behavior in relation to procedures directly from recorded radar flight tracks and standard procedural data. 
We fit a GMM to learn a sequence of deviations between aircraft trajectory points and corresponding points on the flight procedure.
To accommodate the varying relationship between aircraft trajectories and procedures over multiple flight stages, we segmented aircraft trajectories for each flight stage and fit a separate model for each segment. This model is simple, trains quickly, and does not require hyperparameter tuning. %\todo{sufficient claim regarding why our model is good?}

This paper extends the trajectory model from our prior work to incorporate multiple aircraft into full traffic scenes. Full traffic scenes are especially important when validating the safety and operational efficiency of new air traffic control operations and concepts. Developing models that explicitly address these cases is crucial, as aircraft within close proximity influence each other's behavior. Thus, repeatedly applying a single trajectory model is not sufficient. %\todo{May want to be sure we sufficiently answer the question as to why this is a significant enough addition to the previous work}
We fit GMMs for pairs of trajectories and generate multiple trajectories by combining the mean vectors and covariance matrices of the pairwise GMMs.
This approach allows us to efficiently generate an arbitrary number of trajectories using a single model. Being able to predict and simulate these trajectories has many practical real-world applications. For example, a model like the one our work proposes can be used to forecast trajectories in advance or to support scheduling decisions by predicting landing time. To help avoid feature redundancy within and improve the robustness of our model, we further develop the trajectory model by performing a low-rank approximation for the covariance matrices in GMMs.

The remainder of this paper is structured as follows. Section \ref{sec:dataset} describes the trajectory and procedure data.
Section \ref{sec:single trajectory model} outlines the single trajectory model with discussions on the approximation method and synthetic trajectory generation method.
Section \ref{sec:multiple trajectory model} then extends the single trajectory model to incorporate multiple aircraft.
Section \ref{sec:experiments} presents experimental results, and section \ref{sec:conclusions} closes with a summary and future works. Our main contributions are as follows:
\begin{itemize}
    \item For a single trajectory scenario, a separate Gaussian mixture model is trained for each flight stage, to model the differences in variance across stages.
    \item We extend these single trajectory GMMs to pairwise GMMs, which we use to model trajectories of multiple aircraft in proximity. Then, we show how this model can be used to generate multiple trajectories within a single scenario.
    \item We train our proposed models using KJFK arrival trajectories. We evaluate these models on a test split of this data. Additionally, we study how well these models generalize by evaluating them on Charlotte Douglas Airport (KCLT) data. 
    \item Finally, we present both quantitative and qualitative analyses of our models' performance. Our results suggest that the proposed models are effective \textcolor{blue}{in }predicting both major patterns and uncertainties.
\end{itemize}
%\todo{a couple of the recent papers I read ended the intro section with bulleted contributions. I think this may be stronger than an outline - will send examples via slack}
