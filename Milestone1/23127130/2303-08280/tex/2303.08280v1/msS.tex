%%
%% Beginning of file 'sample61.tex'
%%
%% Modified 2016 September
%%
%% This is a sample manuscript marked up using the
%% AASTeX v6.1 LaTeX 2e macros.
%%
%% AASTeX is now based on Alexey Vikhlinin's emulateapj.cls 
%% (Copyright 2000-2015).  See the classfile for details.

%% AASTeX requires revtex4-1.cls (http://publish.aps.org/revtex4/) and
%% other external packages (latexsym, graphicx, amssymb, longtable, and epsf).
%% All of these external packages should already be present in the modern TeX 
%% distributions.  If not they can also be obtained at www.ctan.org.

%% The first piece of markup in an AASTeX v6.x document is the \documentclass
%% command. LaTeX will ignore any data that comes before this command. The 
%% documentclass can take an optional argument to modify the output style.
%% The command below calls the preprint style  which will produce a tightly 
%% typeset, one-column, single-spaced document.  It is the default and thus
%% does not need to be explicitly stated.
%%
%%
%% using aastex version 6.1
%\documentclass{aastex61}
%\documentclass[modern]{aastex61}
%\documentclass[twocolumn,dvipdfmx]{aastex61}
\documentclass[twocolumn]{aastex61}
%\documentclass[linenumbers]{aastex61}
\usepackage{comment}
\usepackage{amsmath,amssymb}
\usepackage{graphicx}
\usepackage{textcomp}
\usepackage{booktabs}
\usepackage{bm}
%\usepackage{pdfpages}
%\usepackage{mathtools}
\def\VEV#1{\left\langle #1\right\rangle}

%% The default is a single spaced, 10 point font, single spaced article.
%% There are 5 other style options available via an optional argument. They
%% can be envoked like this:
%%
%% \documentclass[argument]{aastex61}
%% 
%% where the arguement options are:
%%
%%  twocolumn   : two text columns, 10 point font, single spaced article.
%%                This is the most compact and represent the final published
%%                derived PDF copy of the accepted manuscript from the publisher
%%  manuscript  : one text column, 12 point font, double spaced article.
%%  preprint    : one text column, 12 point font, single spaced article.  
%%  preprint2   : two text columns, 12 point font, single spaced article.
%%  modern      : a stylish, single text column, 12 point font, article with
%% 		  wider left and right margins. This uses the Daniel
%% 		  Foreman-Mackey and David Hogg design.
%%
%% Note that you can submit to the AAS Journals in any of these 6 styles.
%%
%% There are other optional arguments one can envoke to allow other stylistic
%% actions. The available options are:
%%
%%  astrosymb    : Loads Astrosymb font and define \astrocommands. 
%%  tighten      : Makes baselineskip slightly smaller, only works with 
%%                 the twocolumn substyle.
%%  times        : uses times font instead of the default
%%  linenumbers  : turn on lineno package.
%%  trackchanges : required to see the revision mark up and print its output
%%  longauthor   : Do not use the more compressed footnote style (default) for 
%%                 the author/collaboration/affiliations. Instead print all
%%                 affiliation information after each name. Creates a much
%%                 long author list but may be desirable for short author papers
%%
%% these can be used in any combination, e.g.
%%
%% \documentclass[twocolumn,linenumbers,trackchanges]{aastex61}

%% AASTeX v6.* now includes \hyperref support. While we have built in specific
%% defaults into the classfile you can manually override them with the
%% \hypersetup command. For example,
%%
%%\hypersetup{linkcolor=red,citecolor=green,filecolor=cyan,urlcolor=magenta}
%%
%% will change the color of the internal links to red, the links to the
%% bibliography to green, the file links to cyan, and the external links to
%% magenta. Additional information on \hyperref options can be found here:
%% https://www.tug.org/applications/hyperref/manual.html#x1-40003

%% If you want to create your own macros, you can do so
%% using \newcommand. Your macros should appear before
%% the \begin{document} command.
%%
\newcommand{\vdag}{(v)^\dagger}
\newcommand\aastex{AAS\TeX}
\newcommand\latex{La\TeX}
\newcommand{\add}[1]{\textcolor{blue}{#1}}
\newcommand{\addb}[1]{\textcolor{red}{#1}}
\newcommand{\del}[1]{\textcolor{red}{\sout{#1}}}
\newcommand\ltsima{$\; \buildrel <\over\sim \;$}
\newcommand\simlt{\lower.5ex\hbox{\ltsima}}
\newcommand\gtsima{$\; \buildrel >\over\sim \;$}
\newcommand\simgt{\lower.5ex\hbox{\gtsima}}



%% Reintroduced the \received and \accepted commands from AASTeX v5.2
%\received{July 1, 2016}
%\revised{September 27, 2016}
%\accepted{\today}
%% Command to document which AAS Journal the manuscript was submitted to.
%% Adds "Submitted to " the arguement.
%\submitjournal{ApJ}

%% Mark up commands to limit the number of authors on the front page.
%% Note that in AASTeX v6.1 a \collaboration call (see below) counts as
%% an author in this case.
%
%\AuthorCollaborationLimit=3
%
%% Will only show Schwarz, Muench and "the AAS Journals Data Scientist 
%% collaboration" on the front page of this example manuscript.
%%
%% Note that all of the author will be shown in the published article.
%% This feature is meant to be used prior to acceptance to make the
%% front end of a long author article more manageable. Please do not use
%% this functionality for manuscripts with less than 20 authors. Conversely,
%% please do use this when the number of authors exceeds 40.
%%
%% Use \allauthors at the manuscript end to show the full author list.
%% This command should only be used with \AuthorCollaborationLimit is used.

%% The following command can be used to set the latex table counters.  It
%% is needed in this document because it uses a mix of latex tabular and
%% AASTeX deluxetables.  In general it should not be needed.
%\setcounter{table}{1}

%%%%%%%%%%%%%%%%%%%%%%%%%%%%%%%%%%%%%%%%%%%%%%%%%%%%%%%%%%%%%%%%%%%%%%%%%%%%%%%%
%%
%% The following section outlines numerous optional output that
%% can be displayed in the front matter or as running meta-data.
%%
%% If you wish, you may supply running head information, although
%% this information may be modified by the editorial offices.
\shorttitle{\aastex\ MOA-II FFP mass function}
\shortauthors{Sumi et al.}
%%
%% You can add a light gray and diagonal water-mark to the first page 
%% with this command:
% \watermark{text}
%% where "text", e.g. DRAFT, is the text to appear.  If the text is 
%% long you can control the water-mark size with:
%  \setwatermarkfontsize{dimension}
%% where dimension is any recognized LaTeX dimension, e.g. pt, in, etc.
%%
%%%%%%%%%%%%%%%%%%%%%%%%%%%%%%%%%%%%%%%%%%%%%%%%%%%%%%%%%%%%%%%%%%%%%%%%%%%%%%%%

%% This is the end of the preamble.  Indicate the beginning of the
%% manuscript itself with \begin{document}.

\defcitealias{Koshimoto2023}{K23}
\defcitealias{Koshimoto2021}{Koshimoto+21a}
\defcitealias{Gould2022}{Gould+22}

%  \usepackage{setspace}
% \doublespacing
\begin{document}

%\title{Free-Floating planets mass function from MOA-II 9-year survey toward the Galactic Bulge}
\title{Free-Floating planet Mass Function from MOA-II 9-year survey towards the Galactic Bulge}

%% LaTeX will automatically break titles if they run longer than
%% one line. However, you may use \\ to force a line break if
%% you desire. In v6.1 you can include a footnote in the title.

%% A significant change from earlier AASTEX versions is in the structure for 
%% calling author and affilations. The change was necessary to implement 
%% autoindexing of affilations which prior was a manual process that could 
%% easily be tedious in large author manuscripts.
%%
%% The \author command is the same as before except it now takes an optional
%% arguement which is the 16 digit ORCID. The syntax is:
%% \author[xxxx-xxxx-xxxx-xxxx]{Author Name}
%%
%% This will hyperlink the author name to the author's ORCID page. Note that
%% during compilation, LaTeX will do some limited checking of the format of
%% the ID to make sure it is valid.
%%
%% Use \affiliation for affiliation information. The old \affil is now aliased
%% to \affiliation. AASTeX v6.1 will automatically index these in the header.
%% When a duplicate is found its index will be the same as its previous entry.
%%
%% Note that \altaffilmark and \altaffiltext have been removed and thus 
%% can not be used to document secondary affiliations. If they are used latex
%% will issue a specific error message and quit. Please use multiple 
%% \affiliation calls for to document more than one affiliation.
%%
%% The new \altaffiliation can be used to indicate some secondary information
%% such as fellowships. This command produces a non-numeric footnote that is
%% set away from the numeric \affiliation footnotes.  NOTE that if an
%% \altaffiliation command is used it must come BEFORE the \affiliation call,
%% right after the \author command, in order to place the footnotes in
%% the proper location.
%%
%% Use \email to set provide email addresses. Each \email will appear on its
%% own line so you can put multiple email address in one \email call. A new
%% \correspondingauthor command is available in V6.1 to identify the
%% corresponding author of the manuscript. It is the author's responsibility
%% to make sure this name is also in the author list.
%%
%% While authors can be grouped inside the same \author and \affiliation
%% commands it is better to have a single author for each. This allows for
%% one to exploit all the new benefits and should make book-keeping easier.
%%
%% If done correctly the peer review system will be able to
%% automatically put the author and affiliation information from the manuscript
%% and save the corresponding author the trouble of entering it by hand.


%\correspondingauthor{Takahiro Sumi}
%\email{sumi@ess.sci.osaka-u.ac.jp}

\author[0000-0002-4035-5012]{Takahiro Sumi}
\affil{Department of Earth and Space Science, Graduate School of Science, Osaka University, Toyonaka, Osaka 560-0043, Japan. e-mail: {\tt sumi@ess.sci.osaka-u.ac.jp}}
\author[0000-0003-2302-9562]{Naoki koshimoto}
\affiliation{Code 667, NASA Goddard Space Flight Center, Greenbelt, MD 20771, USA}
\affiliation{Department of Astronomy, University of Maryland, College Park, MD 20742, USA}
\author{David P.~Bennett}
\affiliation{Code 667, NASA Goddard Space Flight Center, Greenbelt, MD 20771, USA}
\affiliation{Department of Astronomy, University of Maryland, College Park, MD 20742, USA}
\author{Nicholas J. Rattenbury}
\affiliation{Department of Physics, University of Auckland, Private Bag 92019, Auckland, New Zealand}


%\nocollaboration

\author{Fumio Abe}
\affiliation{Institute for Space-Earth Environmental Research, Nagoya University, Nagoya 464-8601, Japan}
\author{Richard Barry}
\affiliation{Code 667, NASA Goddard Space Flight Center, Greenbelt, MD 20771, USA}
%\author{David P.~Bennett}
%\affiliation{Code 667, NASA Goddard Space Flight Center, Greenbelt, MD 20771, USA}
%\affiliation{Department of Astronomy, University of Maryland, College Park, MD 20742, USA}
\author{Aparna Bhattacharya}
\affiliation{Code 667, NASA Goddard Space Flight Center, Greenbelt, MD 20771, USA}
\affiliation{Department of Astronomy, University of Maryland, College Park, MD 20742, USA}
\author{Ian A. Bond}
\affiliation{Institute of Natural and Mathematical Sciences, Massey University, Auckland 0745, New Zealand}
\author{Hirosane Fujii}
\affiliation{Institute for Space-Earth Environmental Research, Nagoya University, Nagoya 464-8601, Japan}
\author{Akihiko Fukui}
\affiliation{Department of Earth and Planetary Science, Graduate School of Science, The University of Tokyo, 7-3-1 Hongo, Bunkyo-ku, Tokyo 113-0033, Japan}
\affiliation{Instituto de Astrof\'isica de Canarias, V\'ia L\'actea s/n, E-38205 La Laguna, Tenerife, Spain}
\author{Ryusei Hamada}
\affiliation{Department of Earth and Space Science, Graduate School of Science, Osaka University, Toyonaka, Osaka 560-0043, Japan}
\author{Yuki Hirao}
\affiliation{Department of Earth and Space Science, Graduate School of Science, Osaka University, Toyonaka, Osaka 560-0043, Japan}
\author{Stela Ishitani Silva}
\affiliation{Department of Physics, The Catholic University of America, Washington, DC 20064, USA}
\affiliation{Code 667, NASA Goddard Space Flight Center, Greenbelt, MD 20771, USA}
\author{Yoshitaka Itow}
\affiliation{Institute for Space-Earth Environmental Research, Nagoya University, Nagoya 464-8601, Japan}
\author{Rintaro Kirikawa}
\affiliation{Department of Earth and Space Science, Graduate School of Science, Osaka University, Toyonaka, Osaka 560-0043, Japan}
\author{Iona Kondo}
\affiliation{Department of Earth and Space Science, Graduate School of Science, Osaka University, Toyonaka, Osaka 560-0043, Japan}
%\author{Naoki Koshimoto}
%\affiliation{Department of Astronomy, Graduate School of Science, The University of Tokyo, 7-3-1 Hongo, Bunkyo-ku, Tokyo 113-0033, Japan}
\author{Yutaka Matsubara}
\affiliation{Institute for Space-Earth Environmental Research, Nagoya University, Nagoya 464-8601, Japan}
\author{Shota Miyazaki}
\affiliation{Department of Earth and Space Science, Graduate School of Science, Osaka University, Toyonaka, Osaka 560-0043, Japan}
\author{Yasushi Muraki}
\affiliation{Institute for Space-Earth Environmental Research, Nagoya University, Nagoya 464-8601, Japan}
\author{Greg Olmschenk}
\affiliation{Code 667, NASA Goddard Space Flight Center, Greenbelt, MD 20771, USA}
\author{Cl\'ement Ranc}
\affiliation{Sorbonne Universit\'e, CNRS, UMR 7095, Institut d'Astrophysique de Paris, 98 bis bd Arago, 75014 Paris, France}
%\author{Nicholas J. Rattenbury}
%\affiliation{Department of Physics, University of Auckland, Private Bag 92019, Auckland, New Zealand}
\author{Yuki Satoh}
\affiliation{Department of Earth and Space Science, Graduate School of Science, Osaka University, Toyonaka, Osaka 560-0043, Japan}
%\author{Takahiro Sumi}
%\affiliation{Department of Earth and Space Science, Graduate School of Science, Osaka University, Toyonaka, Osaka 560-0043, Japan}
\author{Daisuke Suzuki}
\affiliation{Department of Earth and Space Science, Graduate School of Science, Osaka University, Toyonaka, Osaka 560-0043, Japan}
\author{Mio Tomoyoshi}
\affiliation{Department of Earth and Space Science, Graduate School of Science, Osaka University, Toyonaka, Osaka 560-0043, Japan}
\author{Paul . J. Tristram}
\affiliation{University of Canterbury Mt.\ John Observatory, P.O. Box 56, Lake Tekapo 8770, New Zealand}
\author{Aikaterini Vandorou}
\affiliation{Code 667, NASA Goddard Space Flight Center, Greenbelt, MD 20771, USA}
\affiliation{Department of Astronomy, University of Maryland, College Park, MD 20742, USA}
\author{Hibiki Yama}
\affiliation{Department of Earth and Space Science, Graduate School of Science, Osaka University, Toyonaka, Osaka 560-0043, Japan}
\author{Kansuke Yamashita}
\affiliation{Department of Earth and Space Science, Graduate School of Science, Osaka University, Toyonaka, Osaka 560-0043, Japan}
\collaboration{(MOA collaboration)}

%% Note that the \and command from previous versions of AASTeX is now
%% depreciated in this version as it is no longer necessary. AASTeX 
%% automatically takes care of all commas and "and"s between authors names.

%% AASTeX 6.1 has the new \collaboration and \nocollaboration commands to
%% provide the collaboration status of a group of authors. These commands 
%% can be used either before or after the list of corresponding authors. The
%% argument for \collaboration is the collaboration identifier. Authors are
%% encouraged to surround collaboration identifiers with ()s. The 
%% \nocollaboration command takes no argument and exists to indicate that
%% the nearby authors are not part of surrounding collaborations.
%% Mark off the abstract in the ``abstract'' environment. 
\begin{abstract}
We present the first measurement of the mass function of free-floating planets (FFP) or very wide orbit planets down to
an Earth mass, based on microlensing data from the MOA-II survey in 2006-2014.
The shortest duration event has an Einstein radius crossing time of $t_{\rm E} = 0.057\pm 0.016\,$days and
an angular Einstein radius of $\theta_{\rm E} = 0.90\pm 0.14\,\mu$as. 
There are seven short events with $t_{\rm E}<0.5$ day, which are likely to be due to planets.
The detection efficiency for short events depends on both 
$t_{\rm E}$ and $\theta_{\rm E}$, and we measure this with image-level simulations for the first time. 
These short events 
can be well modeled by a power-law mass function, 
$dN_4/d\log M = (2.18^{+0.52}_{-1.40})\times (M/8\,M_\earth)^{-\alpha_4}$ dex$^{-1}$star$^{-1}$ with
 $\alpha_4 = 0.96^{+0.47}_{-0.27}$ for $M/M_\odot < 0.02$.
This implies a total of $f=  21^{+23}_{-13}$ FFP or wide orbit planets 
in the mass range $0.33<M/M_\earth < 6660$ per star, with a total FFP mass of
$m =  80^{+73}_{-47} M_\earth$ per star.
The number of FFP is $19_{-13}^{+23}$ times the number of planets 
in wide orbits (beyond the snow line), while the total masses are of the same order.
This suggests that the FFPs have been ejected from bound planetary systems that
may have had an initial mass function with a power-law index of $\alpha\sim 0.9$, which would imply
a total number of $22_{-13}^{+23}$ planets star$^{-1}$ and 
a total mass of $171_{-52}^{+80} M_\earth$ star$^{-1}$.
\end{abstract}

%% Keywords should appear after the \end{abstract} command. 
%% See the online documentation for the full list of available subject
%% keywords and the rules for their use.
\keywords{gravitational microlensing; exoplanet; Free floating planets}

%% From the front matter, we move on to the body of the paper.
%% Sections are demarcated by \section and \subsection, respectively.
%% Observe the use of the LaTeX \label
%% command after the \subsection to give a symbolic KEY to the
%% subsection for cross-referencing in a \ref command.
%% You can use LaTeX's \ref and \label commands to keep track of
%% cross-references to sections, equations, tables, and figures.
%% That way, if you change the order of any elements, LaTeX will
%% automatically renumber them.

%% We recommend that authors also use the natbib \citep
%% and \citet commands to identify citations.  The citations are
%% tied to the reference list via symbolic KEYs. The KEY corresponds
%% to the KEY in the \bibitem in the reference list below. 




\section{Introduction} \label{sec:intro}



Gravitational microlensing observations toward the Galactic bulge (Galactic Bulge) enable exoplanet searches \citep{mao1991,gaudi-ogle109,bennett-ogle109, suzuki2016, kos21b}, 
and the measurement of the stellar and sub-stellar mass functions (MFs) \citep{pac91,sumi2011, Mroz17,Mroz19, Mroz20a}.

\cite{sumi2011}  first interpreted the detection of short timescale ($0.5<t_{\rm E}/{\rm day}<2$) 
microlensing events as evidence for the existence of a population of free-floating planets (FFP) and/or wide orbit planets.
While that analysis was limited by the small number of events found in a 2 year subset of the survey by the Microlensing Observation in Astrophysics (MOA) group \citep{sumi03} in collaboration 
with Optical Gravitational Lensing Experiment (OGLE) \citep{uda94},
it opened up the field of FFP studies using microlensing.

\cite{Mroz17} extended the work by using a larger sample from 5 years of the OGLE survey.
They discovered 6 events with timescales shorter ($t_{\rm E}\sim0.2$ day) than those in the previous work.
These events are separated from the longer events by a gap around $t_{\rm E}\sim0.5$ day 
which implying the possibility of a several Earth-mass FFP population.

These studies are based on distribution of $t_{\rm E}$, in which
$t_{\rm E}$ is proportional to the square root of the lens mass $M$ as follows,


% \begin{align}
\begin{equation}
  \label{eq:tE}
   \begin{split}
t_{\rm E}&=
% \frac{\theta_{\rm E}}{\mu_{\rm rel}}  =  
\frac{ \sqrt{\kappa M \pi_{\rm rel}}}{\mu_{\rm rel}}\\
&= 0.1\, {\rm day}  \left(  \frac{M}{5 M_\earth} \right)^{1/2} 
 \left(  \frac{\pi_{\rm rel}}{18\mu \rm as} \right)^{1/2} 
 \left(  \frac{\mu_{\rm rel}}{5 \rm mas\,yr^{-1}} \right)^{-1}.
   \end{split}
   \end{equation}
%   \end{align}
% \begin{equation}
%  \label{eq:M}
%M=\frac{t_{\rm E}^2\mu_{\rm rel}^2}{\kappa \pi_{\rm rel}}= 
%5M_\earth  \left(  \frac{t_{\rm E}}{0.1{\rm d}} \right)^2 
% \left(  \frac{18\mu \rm as}{\pi_{\rm rel}} \right) 
% \left(  \frac{\mu_{\rm rel}}{5 \rm mas\,yr^{-1}} \right)^2 
%\end{equation}
Here, $\kappa =4G/(c^2 {\rm au})= 8.144 {\rm mas}/M_\odot$  and we expect $t_{\rm E}\sim0.1$ day assuming typical value of the lens-source relative parallax: 
$\pi_{\rm rel}=\pi_{\rm l}^{-1}- \pi_{\rm s}^{-1}=1$ au$(D_{\rm l}^{-1}-D_{\rm s}^{-1})=18\mu$as for the bulge lens and a typical value of the lens-source proper motion in the direction of the Galactic center of $\mu_{\rm rel}= 5$ mas\,yr$^{-1} $.
The lens mass $M$, the distance $D_{\rm l}$ to the lens and the relative proper motion $\mu_{\rm rel}$ are degenerate in the observable  $t_{\rm E}$.
This means that the mass function of the lens population has to be determined
statistically, assuming a model of the star population density and velocities in the Galaxy.

\cite{Mroz18} found the first short ($t_{\rm E}=0.32$ day) event showing the Finite Source (FS) effect,
i.e., a finite source and a single point lens (FSPL), 
 in which one can measure a FS parameter $\rho=\theta_*/\theta_{\rm E}$.
 Here $\theta_*$ is the angler source radius which can be estimated 
 from an empirical relation with the source magnitude and color.
 The $\theta_{\rm E}$ is an angular Einstein radius given by
 %  $\theta_{\rm E}=\mu_{\rm rel}/t_{\rm E}=9.2 \mu$
 \begin{equation}
  \label{eq:thetaE}
\theta_{\rm E} = \frac{\mu_{\rm rel}}{t_{\rm E}} = \sqrt{\kappa M \pi_{\rm rel}}.
\end{equation}


This value of $\theta_{\rm E}$ can give us an inferred mass of the lens with better accuracy as we can eliminate one of the three-fold degenerate terms
which affecte $t_{\rm E}$, namely, $\mu_{\rm rel}$:

 \begin{equation}
  \label{eq:M_thetaE_pirel}
M=\frac{\theta_{\rm E}^2}{\kappa \pi_{\rm rel}}  =  
5 M_\earth  \left(  \frac{\theta_{\rm E}} {1.5\mu \rm as} \right)^2 
\left(  \frac{\pi_{\rm rel}}{18\mu \rm as} \right)^{-1}.
\end{equation}

So far, seven short FSPL events have been discovered 
\citep{Mroz18,Mroz19b,Mroz20b,Mroz20c,Kim2021,Ryu2021}.
All of these have $\theta_{\rm E}<10\,\mu$as, implying that their lenses
 are most likely of planetary mass.
All of these sources are red giants because their angular radii, i.e., cross-section,  
are significantly larger than main sequence stars.

\cite{Mroz20b} found the short FSPL event, OGLE-2016-BLG-1928, 
with the smallest value of $\theta_{\rm E}=  0.842 \pm 0.064 \mu$as to date. 
Its lens is the first terrestrial mass FFP candidate and
the first evidence of such population.

\cite{Kim2021} began a new approach to probing the FFP population 
by focusing on analyzing the $\theta_{\rm E}$ distribution in events
with giant sources.
\cite{Ryu2021} found a gap at $10<\theta_{\rm E}/\mu {\rm as} <30$ 
in the cumulative $\theta_{\rm E}$ distribution, which
suggests a separation between the planetary mass population and
 other known populations, like brown dwarfs.

\cite{Gould2022} completed the analysis of 29 FSPL giant-source events 
found in the 2016-2019 KMTNet survey. 
 They presented the $\theta_{\rm E}$  distributions
down to  $\theta_{\rm E}=4.35\,\mu$as and  confirmed that there is a clear gap 
in the distribution of $\theta_{\rm E}$ at $9<\theta_{\rm E}/\mu{\rm as}<26$. 
They note that it is consistent with the gap in the $t_{\rm E}$ distribution
shown by \cite{Mroz17}, indicating 
the existence of the low mass FFP population.
They modeled the $\theta_{\rm E}$ distribution with a power law MF
for the FFP and found 
$dN_{\rm FFP}/d\log M=(0.4\pm 0.2)(M/38M_\earth)^{-p}$ dex$^{-1}$\,star$^{-1}$, 
with $0.9<p<1.2$. This implies that the number of FFPs is at least 
an order of magnitude larger than that for known bound planets.
%This power-law index is larger than $\alpha_{\rm PL}=0.3_{-0.4}^{+0.3}$ of the power-law model 
%driven by \cite{sumi2011}.

% Mroz18 OB161540                        tE=0.32  thetaE=9.2 uas
% Mroz19b  OGLE-2012-BLG-1323 tE = 0.155, thetaE = 2.37 uas
% Mroz19b  OGLE-2017-BLG-0560 tE = 0.905, thetaE = 38.7 uas  
% Mroz20b OGLE-2016-BLG-1928, tE = 0.0288  thetaE = 0.842 +- 0.064 uas
% Mroz20c OGLE-2019-BLG-0551 tE = 0.381  thetaE = 4.35 uas
% Kim2021  KMT-2019-BLG-2073    tE=0.267   thetaE= 4.8 uas
% Ryu2021  KMT-2017-BLG-2820, tE=0.288, thetaE=5.94 uas,
%
% [sumi@moao1 art]$ plot_artlistFS2.pl gb5 5 0
%#----------------------
%#Eq.(2) of Kim et al. (2021)
%M_Earth_M_sun: 3.0404e-06
%kappa  [mass/M_sun]: 8.144
%pi_rel [uas]       : 16
%thetaE [uas]       : 9.2
%M      [M_sun]     : 0.000649557956777996
%M      [M_Earth]   : 213.642269694118
%#----------------------
% Gould2022, Eq.(1),
%#Eq.(1) and bellow of Gould et al. (2022)
%Ds     [pc]        : 8000
%Dl     [pc]        : 7700
%pi_rel [uas]       : 4.87012987012987
%thetaE [uas]       : 30
%M      [M_sun]     : 0.0226915520628684
%M      [M_Earth]   : 7463.34431748072

%#----------------------
%Ds     [pc]        : 8000
%Dl     [pc]        : 7000
%pi_rel [uas]       : 17.8571428571429
%thetaE [uas]       : 5
%M      [M_sun]     : 0.000171905697445972
%M      [M_Earth]   : 56.5404872536418
%#----------------------

%Ds     [pc]        : 8000
%Dl     [pc]        : 4000
%pi_rel [uas]       : 125
%thetaE [uas]       : 5
%M      [M_sun]     : 2.45579567779961e-05
%M      [M_Earth]   : 8.07721246480597
%#----------------------



In this paper, we present the distributions  $\theta_{\rm E}$ and $t_{\rm E}$
values for the microlensing events 
toward the Galactic Bulge from 9 years of the MOA-II survey.
We also present the MF of the planetary mass objects 
using the $t_{\rm E}$ distribution.
We describe the data in section \S\,\ref{sec:Data}.
We show the $\theta_{\rm E}$ distribution
in section \S\,\ref{sec:FSPL}.
We present the $t_{\rm E}$ distribution and the best-fit MF in \S\,\ref{sec:timescale}. 
The discussion and conclusions are given in section \S\,\ref{sec:discussionAndSummary}.






%\begin{enumerate}
%\item improved citations for third party data repositories and software,
%\item easier construction of matrix figures consisting of multiple 
%encapsulated postscript (EPS) or portable document format (PDF) files,
%\item figure set mark up for large collections of similar figures,
%\end{enumerate}


\section{Data} \label{sec:Data}
We use the microlensing sample selected from the MOA-II high cadence photometric survey 
toward the Galactic Bulge in the 2006-2014 seasons \citep[][hereafter \citetalias{Koshimoto2023}]{Koshimoto2023}.
MOA-II uses the 1.8-m MOA-II telescope which has 
a 2.18 deg$^2$ field of view (FOV) and which is
located at the Mt.\ John University
Observatory, New Zealand\footnote{\url{https://www.massey.ac.nz/~iabond/moa/alerts/}}.
%We briefly summarize here.

%\subsection{Database} \label{sec:Database}


\citetalias{Koshimoto2023} used an analysis method similar to what was used by \cite{sumi2011,sumi2013},
but includes a correction of systematic errors and takes into 
account the finite source effect.
They applied a de-trending code to all light curves to remove the systematic errors that correlate with seeing and airmass due to
differential refraction, differential extinction and relative proper motion of stars
in the same way as in \cite{bennett2012} and \cite{sumi2016a}. 
These corrections are important as they result in higher confidence
in the light curve fitting parameters.




%=================================
%\subsection{Microlensing event selection}
%\label{sec:select}


 \citetalias{Koshimoto2023} selected light curves with a single instantaneous 
brightening episode and a flat constant baseline, which can be well 
fit with a point-source point-lens (PSPL) microlensing model \citep{pac86}.
In addition to PSPL, they modeled the events with a FSPL model \citep{Bozza2018}, which is especially important for short events.
These are the major improvements compared to the previous analysis in \cite{sumi2011,sumi2013}
in addition to the extension of the survey duration.


Although they identified 6,111 microlensing candidates, 
they selected only 3,554 and 3,535 objects as the statistical sample using the two relatively strict criteria CR1 and CR2, 
respectively.
Here, CR2 was defined as the stricter criteria compared to 
their nominal criteria CR1 to check the effect of the choice of 
the criteria on a statistical study.
These strict criteria ensure that $t_{\rm E}$ is well constrained
for each event and reject any contamination. 




\cite{sumi2011} reported 10 short events with $t_{\rm E}<2$ days in the 2006-2007 dataset.
Only 5 and 4 events survived following the application of CR1 and CR2, respectively. 
This is because the fitting results changed due to the re-reduction of the dataset.
On the other hand, two events are newly found resulting 7 and 6 events following the application of CR1 and CR2, respectively.
As a result, the excess at $t_{\rm E}=0.5-2$ day in the $t_{\rm E}$ distribution is not significant anymore, however, an even shorter event MOA-9y-6057 ($t_{\rm E}=0.22 \pm 0.06$ day) is added.









%==================================================================================

%===================
%------------------------Table 1.---------------------------------
\begin{deluxetable*}{lrrrrrc}
\tabletypesize{\scriptsize}
%\rotate
\tablecaption{Comparison of parameters of short FS events with known FFP candidates.
\label{tbl:candlistFFP}
}
\tablewidth{0pt}
\tablehead{
\colhead{field-chip-sub-ID} &
\colhead{$t_{\rm E}$} & 
\colhead{$\rho$} &
\colhead{$I_{\rm s,0}$} &
\colhead{$\theta_*$} &
\colhead{$\theta_{\rm E}$} & 
\colhead{reference} \\
\colhead{} & 
\colhead{$(\rm day)$} & 
\colhead{} &
\colhead{${(\rm mag)}$} & 
\colhead{($\mu$as)} & 
\colhead{($\mu$as)} &
\colhead{} 
}
\startdata
MOA-9y-5919     &       0.057 $\pm$ 0.016   &  1.40 $\pm$ 0.46   &      17.23   &   1.26 $\pm$   0.48 &  0.90 $\pm$   0.14 & \citetalias{Koshimoto2023} \\ % gb19-7-7-39836 
MOA-9y-770      &       0.315 $\pm$ 0.017    &  1.08 $\pm$ 0.07    &      14.71   &   5.13 $\pm$   0.86  & 4.73 $\pm$   0.75 & \citetalias{Koshimoto2023} \\ % gb3-7-6-65303 
\hline
OGLE-2016-BLG-1928 & 0.0288 $_{-0.0016}^{+0.0024}$  & 3.39$_{-0.11}^{+0.10}$  & 15.78 &  2.85 $\pm$ 0.20   & 0.842 $\pm$ 0.064  & \cite{Mroz20b} \\
KMT-2019-BLG-2073    & 0.272 $\pm$ 0.007      & 1.138 $\pm$ 0.012          & 14.45    &     5.43 $\pm$ 0.17   &  4.77 $\pm$ 0.19   & \cite{Kim2021} \\
KMT-2017-BLG-2820    & 0.288 $\pm$ 0.015    & 1.096 $\pm$ 0.079           & 14.31     &     7.05 $\pm$ 0.44 &  5.94 $\pm$ 0.37  & \cite{Ryu2021}\\
OGLE-2012-BLG-1323  & 0.155 $\pm$ 0.005     & 5.03 $\pm$ 0.07               &    14.09   &    11.9 $\pm$ 0.5      & 2.37 $\pm$ 0.10    & \cite{Mroz19b}\\
OGLE-2016-BLG-1540  &  0.320 $\pm$ 0.003    &  $1.65 \pm 0.01$             &    13.51   &   15.1 $\pm$ 0.8     &  9.2 $\pm$ 0.5     &   \cite{Mroz18}\\
OGLE-2019-BLG-0551 & 0.381 $\pm$ 0.017     & 4.49 $\pm$ 0.15              &  12.61    &    19.5 $\pm$ 1.6     &  4.35 $\pm$ 0.34   & \cite{Mroz20c} \\
\hline
MOA-9y-1944\tablenotemark{$a$}     &       1.594 $\pm$ 0.136   &  0.00928 $\pm$ 0.00032   &      20.14   &   0.43 $\pm$   0.10 &  46.1 $\pm$   10.5 & \citetalias{Koshimoto2023} \\ %  
OGLE-2017-BLG-0560\tablenotemark{$a$}  & 0.905 $\pm$ 0.005     & 0.901 $\pm$ 0.005           &    12.47   &    34.9 $\pm$ 1.5      & 38.7 $\pm$ 1.6      & \cite{Mroz19b}\\
 \enddata
 \tablenotetext{a}{Likely Brown dwarf lens.}
%\tablecomments{
%}
\end{deluxetable*}




%----------------------------- FIG. 1 -------------------------------------
\begin{figure}
\begin{center}
\includegraphics[scale=0.35,keepaspectratio]{figs/plot_thetaE_MCMC}
\caption{
  \label{fig:thetaE}
Observed cumulative distribution of $\theta_{\rm E}$ for 13 FSPL events from MOA (red line)
and 29 FSPL events from KMTNet (black line) \citep{Gould2022}. The blue line indicates $\theta_{\rm E}=  0.842 \pm 0.064$ of 
terrestrial mass FFP candidate, OGLE-2016-BLG-1928 \citep{Mroz20b}.
}
\end{center}
\end{figure}
%--------------------------------------------------------------------------


%----------------------------- FIG. 2 -------------------------------------
\begin{figure}
\begin{center}
\includegraphics[angle=0,width=8.5cm,keepaspectratio]{figs/getVI_HST_Gould_FSVu_MCMC}
\caption{
  \label{fig:CMDLDallGould}
Extinction free CMD of gb3-7-6. The orange curve is the isochrone matched to this subfield.
The cyan square is the RCG centroid. The red circles with error bars are sources of the 13 FSPL events in this work.
The blue filled circles indicates the 2 FFP candidates in this work.
The black open and filled circles are FSPL events and FFP events from \cite{Gould2022}, respectively.
The purple triangle indicates the source of terrestrial FFP,  OGLE-2016-BLG-1928S \citep{Mroz20b}.
}
\end{center}
\end{figure}
%--------------------------------------------------------------------------



%\section{FSPL events}
\section{Angular Einstein radius distribution}
\label{sec:FSPL}
There are 13 FSPL events with $\theta_{\rm E}$ measurements in the sample, including two FFP candidates, MOA-9y-5919 and MOA-9y-770, that have terrestrial and Neptune masses, respectively.
See \citetalias{Koshimoto2023} for the light curves and detailed parameters of the 13 events.


%\subsection{$\theta_{\rm E}$ Distribution}
The red line in Figure \ref{fig:thetaE} indicates the cumulative distribution of 
$\theta_{\rm E}$ from Table 7 of \citetalias{Koshimoto2023}.
The black line indicates the distribution of 29 FFPs by \cite{Gould2022} normalized to 13 events as a comparison.
Although these can not be directly compared because these are not corrected for detection efficiencies, the general trends seen Figure \ref{fig:thetaE} 
may give us some insights. 

The distributions are consistent for $\theta_{\rm E}>30$\,$\mu$as, where the effect of the detection efficiencies are likely small.
There is a gap around $5<\theta_{\rm E}/{\rm \mu as}<70$ which is roughly consistent with 
the gap at  $10<\theta_{\rm E}/{\rm \mu as}<30$ found by  \cite{Ryu2021} and \cite{Gould2022}.
This gap confirmed the existence of the planetary mass population 
as distinct and separated from the stellar/brown dwarf population as 
indicated by \cite{Gould2022}.

The MOA cumulative distribution shows fewer events over $30<\theta_{\rm E}/{\rm \mu as}<70$ compared to \cite{Gould2022}.
This may be just due to the small number of statistics.
But note that \citetalias{Koshimoto2023} found a brown dwarf candidate MOA-9y-1944 with $\theta_{\rm E}= 46.1 \pm 10.5\,{\rm \mu as}$ although this is not
in the final sample for statistical analysis because the source magnitude of $I_{\rm s}=21.91$ mag is fainter than the threshold of $I_{\rm s}<21.4$ mag.

In our sample, there is one event with a very small value of 
$\theta_{\rm E}$ of $0.90 \pm 0.14$\,$\mu$as.
This confirms the existence of the terrestrial mass population 
which gives rise to events such as 
OGLE-2016-BLG-1928 which has $\theta_{\rm E}=  0.842 \pm 0.064$ \citep{Mroz20b}.
These values are significantly smaller than the lower edge of $\theta_{\rm E}\sim  4.35$\,$\mu$as as reported in  \cite{Gould2022}.
This is likely a result of selection bias given that 
\cite{Gould2022} focused on the sample with super-giant sources,
see Figure \ref{fig:CMDLDallGould}.


We compare the parameters of these events to seven known FFP candidates with  $\theta_{\rm E}$ measurements in Table \ref{tbl:candlistFFP}.
The sources of all known FFP candidates except OGLE-2016-BLG-1928 are 
red clump giants (RCGs) or red super-giants which have large $\theta_{\rm *}=5.4, 7.1, 11.9, 15.1, 19.5$ and $34.9$\,$\mu$as.
The magnification tend to be suppressed by large $\theta_{\rm *}$ with small $\theta_{\rm E}$, i.e., large $\rho$ as
$A_{\rm FS,max}= \sqrt{1+4/\rho^2}$ $(\rho>1)$ \citep{Maeder1973,Agol2003,Riffeser2006}.
For example, in case of the terrestrial mass lens 
with $\theta_{\rm E}\sim 1\mu$as, 
the maximum magnification will be only $A_{\rm FS, max}= 1.066, 1.039, 1.014, 1.009, 1.005$ and $1.002$ for the above values of $\theta_{\rm *}$, respectively.
Note that the source of the terrestrial FFP candidate event, 
OGLE-2016-BLG-1928S is a sub-giant with  $\theta_{\rm *}=2.37$\,$\mu$as.
It is important to search for short FSPL with sub-giants and 
dwarf sources to find low mass FFP.
There is no FSPL event with a red super-giant source 
in our sample because these are saturated in MOA image data.


\citetalias{Koshimoto2023} provides the detection efficiency for our FSPL event sample
to compare to those in \citet{Gould2022} who tried to constrain the FFP MF from only $\theta_{\rm E}$ distribution for their FSPL event sample.
However, we perform a likelihood analysis using not only $\theta_{\rm E}$ of the FSPL events, but also include $t_{\rm E}$ of PSPL events which are more informative than $\theta_{\rm E}$ distribution alone. See more details in Section \ref{sec:timescale}.





%==================================

%\clearpage
%----------------------------- FIG. 3 -------------------------------------
\begin{figure}
\begin{center}
\includegraphics[scale=0.43,keepaspectratio]{figs/plot_tE_best_CR6_addt0}
\caption{
  \label{fig:tE}
The observed timescale $t_{\rm E}$ distribution passing criteria CR2 from the 9 year MOA-II survey..
The red line indicates the best fit single lens model for all population. The blue dotted line represents the known
populations of stars, brown dwarfs, and stellar remnants, and the 
green dashed line represents the planetary mass population.
}
\end{center}
\end{figure}
%--------------------------------------------------------------------------


%\section{Timescale distribution}
\section{Likelihood analysis of mass function}
\label{sec:timescale}
In the final sample of \citetalias{Koshimoto2023}, there are 10 (12) short timescale events with $t_{\rm E} <1$~day after applying CR2 (CR1).
Figure \ref{fig:tE} shows the $t_{\rm E}$ distribution of the CR2 sample.
The distribution is roughly symmetric in $\log t_{\rm E}$, 
with a tail at $t_{\rm E}<0.5$.
This confirmed the existence of such short timescale events 
with $t_{\rm E}<0.5$ day as reported by \cite{Mroz17}.
In this section, we 
perform a likelihood analysis on each of the 3554 (CR1) and 3535 (CR2) events using a Galaxy model to 
constrain the mass function of lens objects.

We define the likelihood, ${\cal L}$, in Section \ref{sec:efficiency}. In Sections \ref{sec:MFknown} and \ref{sec:MFFFP}, we determine the mass function without and with 
 a planetary mass population, respectively, by minimizing $\chi^2 \equiv -2 \ln {\cal L}$. Although the absolute value of $\chi^2$ is not meaningful due to its dependence on an arbitrary normalization associated with our likelihood calculation, the fitting procedure is still statistically valid as the relative likelihood between two models, represented by $\Delta \chi^2$, is independent of the normalization.

Note that results of the likelihood analysis for sample CR1 and CR2 are very similar.
In the following sections, we show only the results for CR2 as our final results 
except in the tables.



\subsection{Likelihood}
\label{sec:efficiency}
Although our sample contains more than 3500 events, the mass function of planetary-mass objects is largely determined by the events with $t_{\rm E} < 1$ day, which account for about 0.3\% of them.
We separately define two likelihoods; ${\cal L}_{\rm short}$ for short timescale events with the best-fit $t_{\rm E} < 1$ day, and ${\cal L}_{\rm long}$ for events with the best-fit $t_{\rm E} \geq 1~{\rm day}$. 
In our likelihood analysis, we use the combined likelihood of ${\cal L} = {\cal L}_{\rm short} {\cal L}_{\rm long}$.


For ${\cal L}_{\rm long}$, we simply use the best-fit $t_{\rm E}$ values provided by \citetalias{Koshimoto2023}, which is similar to the approach by previous studies  \citep{sumi2011, Mroz17}.
This is because (i) the relatively smaller errors of $t_{\rm E}$, (ii) the effect of individual $t_{\rm E}$ error is statistically marginalized by the large number of events, (iii) the limited sensitivity to $\theta_{\rm E}$, and (iv) the minimal impact on our primary goal of measuring the mass function of planetary mass objects.


On the other hand, the situation is the opposite for the short events, ${\cal L}_{\rm short}$, i.e., (i) the $t_{\rm E}$ errors are relatively large due to their shorter magnification period, although smaller than the event selection threshold (see Table 2 of \citetalias{Koshimoto2023}),
(ii) the number of events is very limited (12 for CR1 and 10 for CR2), and $t_{\rm E} <$ 1 day is only sparsely covered in Figure \ref{fig:tE}. Thus, these may not be sufficient to statistically marginalize the effect of $t_{\rm E}$ errors of individual events in the likelihood analysis,
(iii) because $\rho=\theta_*/\theta_{\rm E}$ are expected to be larger than those of longer timescale events, one may get beneficial constraints on $\theta_{\rm E}$ even if $\theta_{\rm E}$ are not well determined, and (iv) they play a crucial role in determining the mass function of planetary mass objects.
Therefore, we use the joint posterior probability distribution of $(t_{\rm E}, \theta_{\rm E})$ for each event derived by \citetalias{Koshimoto2023} using the Markov Chain Monte Carlo (MCMC) method for ${\cal L}_{\rm short}$, to take into account both $t_{\rm E}$ and $\theta_{\rm E}$ information and their uncertainties.


We first describe the simpler one, ${\cal L}_{\rm long}$, in Section \ref{sec:L_longtE}, then describe ${\cal L}_{\rm short}$ in Section \ref{sec:L_shorttE}.







\subsubsection{Likelihood for events with $t_{\rm E} \geq 1$ day} \label{sec:L_longtE}

We define the likelihood for events with $t_{\rm E} \geq 1$ day by
\begin{align}
    {\cal L}_{\rm long} \propto \prod_{i = 1}^{N_{\rm long}} {\cal G} (t_{{\rm E}, i} ; \Gamma), \label{eq:Llong}
\end{align}
where $i$ runs over all the $N_{\rm long}$ events that have the best-fit $t_{\rm E} \geq 1$ day in our sample ($N_{\rm long} = 3542$  for CR1 and $N_{\rm long} = 3525$ for CR2), 
and $t_{{\rm E}, i}$ is the best-fit $t_{\rm E}$ value for $i$th event given by \citetalias{Koshimoto2023}.

The function ${\cal G} (t_{\rm E} ; \Gamma)$ is the model's detectable event rate as a function 
of $t_{\rm E}$ with given model event rate $\Gamma$, combined for the 20 survey fields, given by
\begin{align}
    {\cal G} (t_{\rm E} ; \Gamma) = \sum_j w_j \, g_j (t_{\rm E} ; \Gamma_j) \label{eq:G_tE}.
\end{align}
Here, $j$ takes field index values
gb1 to gb21, except for gb6. See Table 1 of \citetalias{Koshimoto2023} for the location and properties of each field.
The weight $w_j$ for the $j$th field is given by 
\begin{align}
    w_j = \sum_{k \in j} n_{{\rm RC}, k}^2 f_{{\rm LF}, k} \label{eq:w_j},
\end{align}
where $k$ indicates a 1024 pixel $\times$ 1024 pixel subframe in the $j$th field ($k = 1, 2, ..., 80$), $n_{{\rm RC}, k}$ is the number density of RCGs in the $k$th subfield, $f_{{\rm LF}, k}$ is the fraction of stars with magnitude $I < 21.4$\,mag in the $k$th subfield, and $w_j$ is thus proportional to the expected event rate in the $j$th field. To calculate $f_{{\rm LF}, k}$,  
we used a combined luminosity function that uses the OGLE-III photometry map \citep{uda11} for bright stars and the {\it Hubble Space Telescope} data by \citep{hol98} for faint stars.


The function $g_j$ is the model's detectable event rate as a function of $t_{\rm E}$ for field $j$ as given by
\begin{align}
    g_j (t_{\rm E} ; \Gamma_j) &= \tilde{\epsilon}_j (t_{\rm E} ; \Gamma_j) \, \Gamma_j (t_{\rm E}), \label{eq:f_tE}
% \notag\\
%    &= \int_{\theta_{\rm E}} \epsilon_{0, j} (t_{\rm E}, \theta_{\rm E}) \Gamma_j (t_{\rm E}, \theta_{\rm E})   d\theta_{\rm E}. \label{eq:f_tE}
\end{align}
where $\tilde{\epsilon} (t_{\rm E} ; \Gamma)$ is the integrated detection efficiency of the survey as a function of $t_{\rm E}$. \citetalias{Koshimoto2023} demonstrated that when finite source effects are important, the detection efficiency,
$\epsilon (t_{\rm E}, \theta_{\rm E})$ is a function of two variables, $t_{\rm E}$ and $\theta_{\rm E}$.
Therefore, we must integrate over $\theta_{\rm E}$ to obtain the integrated detection efficiency, 
$\tilde{\epsilon}(t_{\rm E} ; \Gamma)$, which now depends upon the event rate and the mass function of the lens objects.
This gives
\begin{equation}
\tilde{\epsilon}_j (t_{\rm E} ; \Gamma) =  \int_{\theta_{\rm E}} \epsilon_{j} (t_{\rm E}, \theta_{\rm E}) \, \Gamma_j (\theta_{\rm E} | t_{\rm E})   d\theta_{\rm E}  \label{eq:eps_tE},
\end{equation}
where $\epsilon_{j} (t_{\rm E}, \theta_{\rm E})$ is the detection efficiency for events with $t_{\rm E}$ and $\theta_{\rm E}$ for $i$th field.
We use the detection efficiency $\epsilon_{j} (t_{\rm E}, \theta_{\rm E})$ estimated by the image level simulations in \citetalias{Koshimoto2023} for the 20 fields of the MOA-II 9-yr survey.

We consider the model event rate as a function of $t_{\rm E}$ and $(t_{\rm E},\theta_{\rm E})$, $\Gamma (t_{\rm E})$ and $\Gamma (t_{\rm E},\theta_{\rm E})$, respectively, as a normalized function so that its integration gives one, i.e., these are probability density functions of $t_{\rm E}$ and $(t_{\rm E},\theta_{\rm E})$, respectively. $\Gamma (\theta_{\rm E} | t_{\rm E}) = \Gamma (t_{\rm E}, \theta_{\rm E})/\Gamma (t_{\rm E})$ is the probability density of events with $\theta_{\rm E}$ given $t_{\rm E}$.
Thus, the calculation of $\tilde{\epsilon}(t_{\rm E} ; \Gamma)$ in Eq. (\ref{eq:eps_tE}) has to be done for every proposed MF
during the fitting procedure because $\Gamma (\theta_{\rm E} | t_{\rm E})$ depends on the MF. 


The function $\Gamma_j (t_{\rm E}, \theta_{\rm E})$ for $j$th field can be separated from the MF \citep{han96},
\begin{align}
 \Gamma_j (t_{\rm E}, \theta_{\rm E}) = \int \gamma_j (t_{\rm E} M^{-1/2}, \theta_{\rm E} M^{-1/2}) \Phi (M) \sqrt{M} dM, \label{eq:G_tEthE}
\end{align}
where $\gamma_j (t_{\rm E}, \theta_{\rm E})$ is the event rate for lenses with
mass $1\,M_\sun$ and $\Phi (M)$ is the present-day MF (expressed as $dN/dM$).
Although substituting Eqs. (\ref{eq:eps_tE}) and (\ref{eq:G_tEthE}) makes the calculation of $g_j (t_{\rm E} ; \Gamma_j)$ in Eq. (\ref{eq:f_tE}) a double integral over $M$ and $\theta_{\rm E}$, \citetalias{Koshimoto2023} showed that the integration over $\theta_{\rm E}$ is largely avoidable during a fitting procedure by switching the order of the integrals and calculating the integral over $\theta_{\rm E}$ before the fitting.


We calculate $\gamma_j (t_{\rm E}, \theta_{\rm E})$ for each field using the density and velocity distribution of stars from the latest parametric Galactic model toward 
the Galactic Bulge based on Gaia and microlensing data \citep{Koshimoto2021}.



Figure \ref{fig:EFF} shows the integrated detection efficiencies $\tilde{\epsilon} (t_{\rm E} ; \Gamma)$ for the event rate calculated with the best fit MF model with the criteria CR2.
The curve for CR1 is similar.




%----------------------------- FIG. 4 -------------------------------------
\begin{figure}
\begin{center}
\includegraphics[width=8.5cm]{figs/DE_thetaEwt_CR6_addt0}
%\vspace{3mm}
\caption{
  \label{fig:EFF}
Integrated detection efficiencies, $\tilde{\epsilon} (t_{\rm E} ; \Gamma)$, as a function of the timescale $t_{\rm E}$ down to 
the source magnitude of $I_{\rm s} <21.4$ mag for the criteria CR2.
Red, black, green and blue lines indicate the efficiencies of fields with the highest, high, medium and low cadence, respectively.
}
\end{center}
\end{figure}
%--------------------------------------------------------------------------


\subsubsection{Likelihood for short timescale ($t_{\rm E} < 1$ day) events} \label{sec:L_shorttE}
We follow \citet{hog10} to utilize the posterior probability distribution of each event from MCMC to calculate the likelihood for the short timescale events, ${\cal L}_{\rm short}$. Given the output MCMC samples of posterior distributions for individual events by \citetalias{Koshimoto2023}, the likelihood is presented by 
\begin{align}
    {\cal L}_{\rm short} \propto \prod_{i = 1}^{N_{\rm short}} \sum_{k = 1}^{K_i} \frac{{\cal G} (\log t_{{\rm E}, ik}, \log \theta_{{\rm E}, ik} ; \Gamma)}{p_0 (\log t_{{\rm E}, ik}, \log \theta_{{\rm E}, ik})}, \label{eq:Lshort}
\end{align}
where $i$ runs over all the $N_{\rm short}$ events that have the best-fit $t_{\rm E} < 1$ day ($N_{\rm short} = 12$ for CR1 and $N_{\rm short} = 10$ for CR2), $k$ runs over all the $K_i$ samples in the MCMC sample of posterior distribution for $i$th event, and $p_0 (\log t_{\rm E}, \log \theta_{\rm E})$ is the prior distribution multiplied in the posterior.
The model's detectable event rate as a function of $(\log t_{\rm E}, \log \theta_{\rm E})$ is given by
\begin{align}
    {\cal G} (\log t_{\rm E}, \log \theta_{\rm E} ; \Gamma) = \sum_j w_j \, g_j (\log t_{\rm E}, \log \theta_{\rm E} ; \Gamma_j)
\end{align}
with
\begin{align}
    g_j (\log t_{\rm E}, \log \theta_{\rm E} ; \Gamma_j) = \epsilon_{j} (\log t_{\rm E}, \log \theta_{\rm E}) \, \Gamma_j (\log t_{\rm E}, \log \theta_{\rm E}), \label{eq:g_tEthE}
\end{align}
where we represented it as a function of $(\log t_{\rm E}, \log \theta_{\rm E})$ rather than $(t_{\rm E}, \theta_{\rm E})$ because \citetalias{Koshimoto2023} provide the posterior distributions with the uniform prior in $(\log t_{\rm E}, \log \theta_{\rm E})$, i.e.,
$p_0 (\log t_{\rm E}, \log \theta_{\rm E}) = {\rm const.}$.

Eq. (\ref{eq:Lshort}) calculates the likelihood by summing the ratio of ${\cal G} (\log t_{\rm E}, \log \theta_{\rm E} ; \Gamma)$ to $p_0 (\log t_{\rm E}, \log \theta_{\rm E})$ to replace the used prior (i.e., $p_0$) with the new prior (i.e., ${\cal G}$) over the MCMC samples. This method, which uses all the MCMC samples, allows ${\cal L}_{\rm short}$ to account for the uncertainty of the parameters, unlike ${\cal L}_{\rm long}$ given in Eq. (\ref{eq:Llong}).

Despite the significant computational cost of Eq. (\ref{eq:Lshort}) associated with performing a summation over $K_i$ (typically $\sim 5 \times 10^5$) samples for each proposed mass function during the fitting process, we addressed this by implementing a binning strategy for the MCMC sample using grids of $(\log t_{\rm E}, \log \theta_{\rm E})$ with a size of (0.05 dex $\times$ 0.05 dex), which significantly increased the computational efficiency.

\subsection{Mass function of known population} \label{sec:MFknown}

Firstly, we perform the likelihood analysis without the short events with $t_{\rm E}< 1$ day using the Galactic model 
with the MF of known population, i.e., stellar remnants (black holes (BH), neutron stars (NS) and white dwarfs(WD)), 
main sequence stars (MS) and brown dwarfs (BD).
We use a broken power-law MF given by
\begin{equation}
 \frac{dN}{d\log M} \propto
  \left\{
  %\begin{cases}
   \begin{array}{ll}
     {M^{-\alpha_1}} & (M_1 < M/M_\sun < 120)\\
     {M^{-\alpha_2}} & (0.08 < M/M_\sun < M_1) \\
     {M^{-\alpha_3}} & (3 \times 10^{-4} < M/M_\sun < 0.08). \\
    \end{array}
    %\end{cases}
    \right.
  \label{eq:MF}
\end{equation}

We adopt the values of parameters $\alpha_1=1.32$ and   $\alpha_2=0.13$,  $\alpha_3=-0.82$ and $ M_1=0.86$
from the E+E$_{\rm X}$ model of \citet{Koshimoto2021} by default unless specified as fitting parameters in the following three models.
The minimum mass $3 \times 10^{-4} \, M_\sun$ is taken to be smaller than 
the theoretical minimum mass of the gas cloud, $\sim$Jupiter-mass, 
that collapses to form a brown dwarf \citep{bos03}.
During our fitting procedure, a proposed initial mass function (IMF) is converted into a present-day mass function following the procedure used by \citet{Koshimoto2021} that combines their stellar age distribution and the initial-final mass relation by \citet{lam20} to evolve stars into stellar remnants.


We consider three models here: BD1, BD2, and BD3. In BD1, we fit only $\alpha_3$ as a fitting parameter, while fixing $\alpha_1$, $\alpha_2$, and $M_1$. Similarly, in BD2, we fit $\alpha_3$ and $\alpha_2$, and in BD3, we fit $\alpha_1$, $\alpha_2$, $\alpha_3$, and $M_1$. To perform the fitting, we use the Markov Chain Monte Carlo (MCMC) method \citep{metrop}, and assign uniform distributions as priors for all the parameters.

The best fit models BD1, BD2 and BD3 are almost indistinguishable from the blue dotted line in Figure \ref{fig:tE}.
One can see that the models fit the data with $t_{\rm E} > 1$ day very well.
The best fit parameters and $\chi^2$ values are listed in Table \ref{tbl:fitparamBD}.
There is no significant difference in the resultant parameters between different selection criteria or among the BD1, BD2, and BD3 models.

 All of the parameters are consistent with those of \citet{Koshimoto2021} within $1\sigma$.
This indicates that our dataset confirmed the Galactic model and MF of known objects by \citet{Koshimoto2021}. This also indicates that our dataset is consistent with the OGLE-IV $t_{\rm E}$ distribution for $t_{\rm E} > 1$ day \citep{Mroz17, Mroz19} that is fitted by \citet{Koshimoto2021}.


In the following analysis, we fit only $\alpha_3$ and 
fix all other parameters for the known populations.
Note, in  \citet{Koshimoto2021}, the Galactic model and MF are constrained to satisfy 
the microlensing $t_{\rm E}$ distribution, stellar number counts and the Galactic Bulge mass 
from other observations, simultaneously. 
In principle, the MF should not be changed alone because it is related to other parameters of the Galactic model.
However, the contribution of objects with $M/M_\sun < 0.08$ are negligible in stellar number counts and as a fraction of the Galactic Bulge mass.
Thus, we assume that a model with a different slope at lower masses with $M/M_\sun < 0.08$ is still valid.

%------------------------Table 2.---------------------------------
%\startlongtable
\begin{deluxetable*}{lrrrrrrr}
%\rotate
%\tabletypesize{\footnotesize}
\tabletypesize{\scriptsize}
\tablecaption{Best fit parameters of the mass function for known population.
 \label{tbl:fitparamBD}}
\tablewidth{0pt}
\tablehead{
\colhead{model}      &      \multicolumn{2}{c} {BD1}               &          \multicolumn{2}{c} {BD2}                 &           \multicolumn{2}{c} {BD3}                &   \citetalias{Koshimoto2021}\tablenotemark{a} \\
\colhead{} & \colhead{CR1}     &  \colhead{CR2}          &         \colhead{CR1}   &  \colhead{CR2}          & \colhead{CR1}           &  \colhead{CR2}          &
}                                                                                                                                                
\startdata
$M_1$           & ($0.86)$               &   ($0.86)$              &      ($0.86)$           &       ($0.86)$          & $0.97^{-0.04}_{-0.34}$  & $0.99^{-0.06}_{-0.37}$  &   $ 0.86^{+0.09}_{-0.10}$     \\
$\alpha_1$      & ($1.32)$               &   ($1.32)$              &      ($1.32)$           &       ($1.32)$          & $1.33^{+0.21}_{-0.17}$  & $1.34^{+0.18}_{-0.18}$  &   $ 1.32^{+0.14}_{-0.10}$     \\
$\alpha_2$      & ($0.13)$               &   ($0.13)$              & $0.20^{+0.07}_{-0.05}$  & $0.20^{+0.07}_{-0.05}$  & $0.23^{+0.04}_{-0.19}$  & $0.24^{+0.04}_{-0.21}$  &   $ 0.13^{+0.11}_{-0.12}$     \\
$\alpha_3$      &$-0.60^{+0.08}_{-0.13}$ & $-0.62^{+0.09}_{-0.14}$ & $-0.74^{+0.13}_{-0.30}$ & $-0.76^{+0.14}_{-0.30}$ & $-0.76^{+0.19}_{-0.26}$ & $-0.79^{+0.22}_{-0.25}$ &   $-0.82^{+0.24}_{-0.51}$     \\
$\chi^2$        &  35919.4               &        35722.6          &           35918.2       &       35721.5           &      35918.0            &   35721.3               &              \\
\enddata
\tablenotetext{a}{Results of fitting to various bulge data including the OGLE-IV $t_{\rm E}$ distribution of $t_{\rm E} > 1$ day \citep{Mroz17,Mroz19}. The representative values are shifted to the ones for the E+E$_{\rm X}$ model from their original ones for the G+G$_{\rm X}$ model.}
\tablecomments{Some of the upper errors of $M_1$ is negative because the best fit value is outside of the 68\% range. This is because $M_1$ is restricted to be less than 1 $M_\sun$.}
\end{deluxetable*}
%-------------------------------------------------------------------


\subsection{Mass function of planetary mass population} \label{sec:MFFFP}

If the candidates with $t_{\rm E}<0.5$ day are really due to microlensing, 
they can not be explained by known populations, i.e., stellar remnants, MS or BD.
To explain the tail for short values of $t_{\rm E}$, 
we defined a new model ``PL" which introduces a planetary mass population 
by the following power law in addition to known populations (Eq. \ref{eq:MF}),

\begin{equation}
  \frac{dN_4}{d\log M}  =Z  \left(\frac{M}{M_{\rm norm}}\right)^{-\alpha_4},  (M_{\rm min} < M/M_\sun<0.02).
  \label{eq:MF_PL}
\end{equation}
Here $Z$ is a normalization factor and
$M_{\rm norm}$ is a reference mass whose inclusion allows
$Z$ to have a unit of (dex)$^{-1}$. 
Although $M_{\rm norm}$ can be an arbitrary zero point, 
we found that the uncertainty in $Z$ is minimized when we adopt 
$M_{\rm norm}=8\,M_\earth$ which is recognized as a pivot point.


In the model PL, we use $\alpha_3$, $\alpha_4$ and $Z$, as fitting parameters and fix parameters
$\alpha_1=1.32$,  $\alpha_2=0.13$ and $ M_1=0.86$ \citep{Koshimoto2021}. We assign
uniform distributions as priors for $\alpha_3$, $\alpha_4$ and $\log Z$ in our MCMC run.
We found that the fitting result does not depend on $M_{\rm min}$ at all when $M_{\rm min} < 3 \times 10^{-7} \, M_\sun$, which indicates our data sensitivity is down to $\sim 3 \times 10^{-7} M_\sun$. Thus, we decided to use $M_{\rm min} = 10^{-7}$.

The red solid line in Figure \ref{fig:tE} represents the best fit model for all populations with the CR2 sample.
This figure indicates that the model represents the observed $t_{\rm E}$ distribution well.
Note that although the observed $t_{\rm E}$ distribution shown in black in Figure \ref{fig:tE} does not include error bars along the $t_{\rm E}$ axis, the best-fit line is derived from our likelihood analysis that takes into account the $t_{\rm E}$ errors as well as the $\theta_{\rm E}$ constraints for the short events with $t_{\rm E} < 1$ day.
Figure \ref{fig:MCMC_PL} shows the posterior distributions of the parameters of PL model.
The best fit parameters and $\chi^2$ are listed in Table \ref{tbl:fitparamPL}.


The best fit power index for BD is $\alpha_3= -0.58^{+0.12}_{-0.16}$ which 
is consistent with the model without the planetary mass population.

The best fit MF of the planetary mass populations 
with the normalization $Z$ relative to stars (MS+BD+WD)  (integrated IMF over $3\times 10^{-4}<M/M_\sun < 8$)
can be expressed as  

\begin{equation}
  \frac{dN_4}{d\log M}  = \frac{2.18^{+0.52}_{-1.40}}{\rm dex\times star}   \left( \frac{M}{8\,M_\earth} \right)^{-\alpha_4}, 
  \label{eq:MF_PL2}
\end{equation}
where  $\alpha_4= 0.96^{+0.47}_{-0.27}$.
Figure \ref{fig:IMF} shows the IMF of the best fit PL model.
This $\alpha_4$ is consistent with the corresponding power law index 
of $0.9\lesssim p\lesssim1.2$ reported by \cite{Gould2022}.


This can be translated to the normalization per stellar mass
of stars, $Z^{M_\sun}$, as,

\begin{equation}
  \frac{dN_4}{d\log M} =\frac{5.48^{+1.18}_{-3.50}}{ {\rm dex} \times M_\sun}  \left( \frac{M}{8\,M_\earth} \right)^{-\alpha_4}.
  \label{eq:ZM_MS_BD}
\end{equation}
This implies that the number of FFPs per stars is $f= 21^{+23}_{-13}$\,star$^{-1}$
over the mass range $10^{-6}<M/M_\sun < 0.02$.
Note taht this value is vary depending on the minimum mass.
The total mass of FFPs per star is $m =  80^{+73}_{-47} M_\earth (0.25_{-0.15}^{+0.23} M_{\rm J})$\,star$^{-1}$.
This is less dependent from the minimum mass.
The total mass of FFPs per $M_\sun$ is  
$m^{M_\sun}=202^{+166}_{-114}$ $M_\earth (0.64_{-0.11}^{+0.19} M_{\rm J}) M_\sun^{-1}$. 
This is more robust values less dependent on uncertainty in the abundances 
of the low mass objects for both FFP and BD.

The normalization, number and total mass of FFP relative to MS+BD ($3\times 10^{-4}<M/M_\sun < 1.1$) are 
also shown in Table \ref{tbl:fitparamPL}.
These normalizations can be translated to 
$Z_{\rm MS+BD}=0.53^{+0.19}_{-0.40}$ dex$^{-1}$star$^{-1}$ and 
$Z_{\rm MS+BD}^{M_\sun}=2.44^{+0.71}_{-1.82}$ dex$^{-1} M_\sun ^{-1}$
with $M_{\rm norm}=38M_\earth$.
These are almost same as
$Z_{\rm MS+BD}=0.39\pm0.18$ dex$^{-1}$star$^{-1}$ and
$Z_{\rm MS+BD}=1.96\pm0.98$ dex$^{-1}M_{\sun}^{-1}$
with $M_{\rm norm}=38M_\earth$
by \cite{Gould2022}.





Note that the lenses for these short events could be either FFP or planets 
with very wide separations of more than about ten astronomical units (AU) from their host stars, 
for which we cannot detect the host star in the light curves.
%------------------------Table 3.---------------------------------
%\startlongtable
\begin{deluxetable}{lccccc}
%\rotate
%\tabletypesize{\footnotesize}
\tabletypesize{\scriptsize}
\tablecaption{Best fit parameters of the mass function for the planetary mass population.
 \label{tbl:fitparamPL}}
\tablewidth{0pt}
\tablehead{
                           &   \colhead{CR1}          &          \multicolumn{2}{c}{CR2}                     &     \citetalias{Gould2022}         \\
\colhead{($M_{\rm norm}$)} & \colhead{($8~M_\earth$)} & \colhead{($8~M_\earth$)} & \colhead{($38~M_\earth$)} & \colhead{($38~M_\earth$)}
}
 \startdata
 $M_1$                           & ($0.86)$                    &       ($0.86)$          &                         &                   \\
 $\alpha_1$                      & ($1.32)$                    &       ($1.32)$          &                         &                   \\
 $\alpha_2$                      & ($0.13)$                    &       ($0.13)$          &                         &                   \\
 $\alpha_3$                      & $-0.55^{+0.13}_{-0.17}$     & $-0.58^{+0.12}_{-0.16}$ &                         &                   \\
 $\alpha_4$                      &  $0.90^{+0.48}_{-0.27}$     &  $0.96^{+0.47}_{-0.27}$ &                         &  fixed at $0.9$ or $1.2$       \\
 $Z$                       & $2.08^{+0.54}_{-1.33}$ & $2.18^{+0.52}_{-1.40}$  & $0.49^{+0.17}_{-0.37}$  &                   \\
 $Z_{\rm MS+BD}$                 &  $2.27^{+0.60}_{-1.46}$     &  $2.38^{+0.58}_{-1.53}$ & $0.53^{+0.19}_{-0.40}$  &  $0.39 \pm 0.20 \pm ?$  \\
 $Z^{M_\sun}$              & $5.33^{+1.26}_{-3.40}$  & $5.48^{+1.18}_{-3.50}$ & $1.22^{+0.35}_{-0.91}$  &                   \\
 $Z^{M_\sun}_{\rm MS+BD}$        & $10.63^{+2.52}_{-6.78}$     & $10.95^{+2.36}_{-6.97}$ & $2.44^{+0.71}_{-1.82}$  &  $1.96 \pm 0.98 \pm ?$  \\
 $f$\tablenotemark{a}             &  $17^{+20}_{-11}$       & $21^{+23}_{-13}$       &                   &                   \\
 $f_{\rm MS+BD}$\tablenotemark{a} & $19^{+22}_{-12}$              &  $23^{+25}_{-15}$            &                   &                   \\
 $f^{M_\sun}$\tablenotemark{a} &  $45^{+54}_{-30}$    & $53^{+59}_{-34}$       &                   &                   \\
 $f^{M_\sun}_{\rm MS+BD}$\tablenotemark{a} & $89^{+107}_{-59}$    &  $106^{+117}_{-68}$          &                   &                   \\
 $m$\tablenotemark{b}             & $89^{+96}_{-56} M_\earth$  & $80^{+73}_{-47} M_\earth$  &                   &                   \\
 $m_{\rm MS+BD}$\tablenotemark{b} &  $98^{+107}_{-61} M_\earth$   &  $88^{+81}_{-51} M_\earth$   &                   &                   \\
 $m^{M_\sun}$\tablenotemark{b} & $229^{+219}_{-140} M_\earth$ & $202^{+166}_{-114} M_\earth$      &           &                   \\
 $m^{M_\sun}_{\rm MS+BD}$\tablenotemark{b} & $457^{+439}_{-279} M_\earth$ & $404^{+333}_{-228} M_\earth$ &           &                   \\
 $\chi^2$                       &   36273.0                       &  36024.1                &                        &                   \\
 \enddata
\tablenotetext{a}{Number of planetary mass objects per BD+MS+WD ($f$), per MS+BD ($f_{\rm MS+BD}$), per solar mass of BD+MS+WD ($f^{M_\sun}$) or per solar mass of MS+BD ($f^{M_\sun}_{\rm MS+BD}$) when MF down to $10^{-6} M_\sun$ are integrated. These are vary depending on the minimum mass.}
\tablenotetext{b}{Total mass of planetary mass objects per BD+MS+WD ($m$), per MS+BD ($m_{\rm MS+BD}$), per solar mass of BD+MS+WD ($m^{M_\sun}$) or per solar mass of MS+BD ($m^{M_\sun}_{\rm MS+BD}$) when MF down to $10^{-6} M_\sun$ are integrated.}
\tablecomments{We adopt the model for CR2 as the final result.}
\end{deluxetable}
%-------------------------------------------------------------------






%\begin{figure}
%\plotone{figs/plot_tE_a34_M3_CR4}
%\caption{The Swift/XRT X-ray light curve for the first year after
%the component figures are available in the Figure Set. \label{fig:fig4}}
%\end{figure}



\begin{comment}
%----------------------------- FIG. 5 -------------------------------------
\begin{figure}
\begin{center}
\includegraphics[angle=-90,scale=0.35,keepaspectratio]{figs/MCMCchain_a3_tEgt0.5_CR4}
\includegraphics[angle=-90,scale=0.35,keepaspectratio]{Koshimoto/figs/MCMCchai_CR4_a23_tEgt0.5}
\includegraphics[angle=-90,scale=0.35,keepaspectratio]{figs/MCMCchain_a123_M1_tEgt0.5_CR4}
\caption{
  \label{fig:tEBD}
The posterior distributions of the parameters by MCMC for known populations with sample CR2.
Top panel shows $\alpha_3$ in the model BD1.
Middle panel shows $\alpha_2$ and $\alpha_3$ in the model BD2.
Bottom panel shows $\alpha_1$,  $\alpha_2$, $\alpha_3$ and $M_1$ in the model BD3.
}
\end{center}
\end{figure}
%--------------------------------------------------------------------------
\end{comment}


%----------------------------- FIG. 5 -------------------------------------
\begin{figure}
\begin{center}
\includegraphics[scale=0.35,keepaspectratio]{figs/MCMCchain_a34logZS22_CR6}
\caption{
  \label{fig:MCMC_PL}
Posterior distributions of the parameters of the PL model for sample CR2.
The vertical red dotted lines indicate the median and $\pm 1\sigma$.
The vertical orange line indicates the best fit.
}
\end{center}
\end{figure}
%--------------------------------------------------------------------------

%----------------------------- FIG. 6 -------------------------------------
\begin{figure}
\begin{center}
\includegraphics[scale=0.49,keepaspectratio]{figs/IMFerror}
\caption{
  \label{fig:IMF}
Initial mass function (IMF) of the best fit PL model for CR2. 
The red line indicates the best fit for all population.
The blue dotted line and green dashed line show the IMFs for the stellar and brown dwarf population and for the planetary mass population, respectively.
The shaded areas indicate $1\sigma$ error. 
The gray dashed-line and the shaded area indicate the best-fit and $1~\sigma$ range of the bound planet MF
by \citet{suzuki2016} via microlens.
}
\end{center}
\end{figure}
%--------------------------------------------------------------------------


%==================================
\section{Discussion and conclusions}
\label{sec:discussionAndSummary}

We derived the MF of lens objects from the 9-year MOA-II survey towards the Galactic Bulge.
The 3,535 high quality single lens light curves used in our statistical analysis include 
10 very short ($t_{\rm E}<1$ day) events, and  13 events with strong finite source
effects that allow the determination of the angular Einstein radius, $\theta_{\rm E}$.

The cumulative $\theta_{\rm E}$ histogram for these 13 events reveals an
``Einstein gap" at $5<\theta_{\rm E}/{\rm \mu as}<70$ which is roughly consistent with 
the gap at  $10<\theta_{\rm E}/\mu {\rm as} <30$ found by the KMTNet group
 \citep{Ryu2021,Gould2022}. 
This gap indicates that there is a distinct planetary mass population
separated from the known populations of brown dwarfs, stars and stellar remnants.


We constructed the $t_{\rm E}$ distribution of all selected samples including both PSPL and FSPL.
We calculated the integrated detection efficiency $\tilde{\epsilon} (t_{\rm E} ; \Gamma)$ of the survey 
by integrating the two dimensional detection efficiency, $\epsilon (t_{\rm E}, \theta_{\rm E})$,
measured from image level simulations that included the FS effect, and convolving this with
the event rate $\Gamma (t_{\rm E}, \theta_{\rm E})$ given by a Galactic model and MF.
We found that the $t_{\rm E}$ distribution has an excess at short $t_{\rm E}$ values
which can not be explained by known populations.

We then adopted the power law MF for the planetary mass population.
We found that these short events can be well modeled by
 $dN_4/d\log M = (2.18^{+0.52}_{-1.40})\times   (M/8\,M_\earth)^{-\alpha_4}$ dex$^{-1}$star$^{-1}$ with 
 $\alpha_4 = 0.96^{+0.47}_{-0.27}$ at $10^{-7}<M/M_\odot < 0.02$ (or $ 0.033 < M/M_\earth < 6660$).

This can also be expressed by the MF per stellar mass as,
$dN_4/d\log M =5.48^{+1.18}_{-3.50}\times   (M/8\,M_\earth)^{-\alpha_4}$ dex$^{-1} M_\sun ^{-1}$.
We showed the number of FFP or distant planets is $f= 21^{+23}_{-13}$ per stars.

It is well known that planet-planet scattering during the planet formation process 
is likely to produce a population of
unbound or wide orbit planetary mass objects \citep{rasio96,weiden96,lin97}
The probability of planet scattering likely increases with declining mass 
because planets usually require more massive planets to scatter.
 So, we expect the power law index of MF of bound planets $\alpha_{\rm b}$ is smaller 
 than that of $\alpha_4$ for unbound or large orbit planets, i.e., $\alpha_4 > \alpha_{\rm b}$.
 
One can compare our FFP result to the MF of known bound planets. At present, 
microlensing surveys have only measured the mass ratio function, rather than the 
mass function, of the bound planets. Currently, the most sensitive study of the bound planet
mass ratio function
\cite{suzuki2016} found that the mass ratio function can be well explained by the broken power law with
$\alpha_{\rm b}=0.93\pm0.13$ for $q> q_{\rm br} = 1.7\times 10^{-4}$,
$\alpha_{\rm b}=-0.6_{-0.5}^{+0.4}$ for $q< q_{\rm br} = 1.7\times 10^{-4}$. While the
\cite{suzuki2016} data could establish the existence of the power-law break
with reasonably high confidence (a Bayes factor of 21), there was a large, correlated
uncertainty in the mass ratio of the break and slope of the mass ratio function below the
break. So, we chose to fix the mass ratio of
break at $q_{\rm br} = 1.7\times 10^{-4}$ in order to estimate the
power law below the break. 

More recently, several papers have attempted to
improve upon this estimate by including a heterogeneous set of lower mass ratio planets
found by a number of groups without a calculation of the detection efficiency. 
These efforts included attempts to estimate the effect of a ``publication bias" 
that might cause planets deemed to be of greater interest to be published much more quickly, 
leading to biased, inhomogeneous sample of planets. This ``publication bias"  is caused by
the decision to publish some planet discoveries at a higher priority than others.
With such an analysis
\cite{Udalski2018} reported 
$\alpha_{\rm b}=-1.05_{-0.78}^{+0.68}$ with their sample and 
$\alpha_{\rm b}=-0.73_{-0.34}^{+0.42}$ when combined with the \cite{suzuki2016} result
for $q< 1\times 10^{-4} < q_{\rm br}$. A similar analysis by \citet{Jung2019}, attempted a
new measurement of the location of the break and found
$\alpha_{\rm b}=-4.5$ for $q< q_{\rm br} = 0.55\times 10^{-4}$ which is consistent
with the \cite{suzuki2016}  result when $q_{\rm br}$ is not fixed. However, a more recent
paper \citep{zang22} by many of the same authors, reported a number of planetary
microlensing events that were missed by the analyses described in \cite{Udalski2018} and 
\citet{Jung2019}. This casts some doubt on the validity of some of the assumptions
in these papers. This later paper also suggests that planets with mass ratios of $q< q_{\rm br} = 1.7\times 10^{-4}$
may be more common than previously thought, although a more definitive claim awaits a
detection efficiency calculation. Also, the \cite{suzuki2016} analysis does not imply that there is a peak
in the mass ratio. Instead it concludes that the slope does not rise as steeply toward low mass ratios
as is does for $q > 1.7\times 10^{-4}$.


The broken power-law model of  \cite{suzuki2016} is consistent with the hypothesis that these unbound or wide 
orbit planetary mass objects are the result of scattering from bound planetary systems.
It is the lower mass planets that are preferentially removed by planet-planet scattering interactions, 
so the initial planetary mass function may have been closer to a single power-law with $\alpha_{\rm b} \sim 0.9$, but 
planet-planet scattering has likely depleted the numbers of low-mass planets at separations beyond
the snow line where microlensing is most sensitive. Thus, planet-planet scattering may be responsible for the
mass ratio function ``break" observed in the \cite{suzuki2016} sample

This idea that planet-planet scattering is responsible for a FFP mass function slope
that is steeper than the slope of the mass ratio function for low-mass bound planets is also
consistent with the single power-law models that were found in smaller data sets \citep{sumi2010}.
The best fit single power-law model for the \cite{suzuki2016} sample gives 
$dN_{\rm bound}/d\log q = 0.068^{+ 0.016}_{-0.014} {\rm dex^{-2}star^{-1}}\times (q/0.001)^{-\alpha_{\rm b}}$ with
$\alpha_{\rm b}=0.58\pm 0.08$ for $3\times 10^{-6} < q< 3\times 10^{-2}$, but the broken power-law
is a significantly better fit to the  \cite{suzuki2016} data.
Note, this single power law model with $\alpha_{\rm b}=0.58$ satisfies $\alpha_4 > \alpha_{\rm b}$, for
our value of $\alpha_4 = 0.96^{+0.47}_{-0.27}$, implying that unbound (or very wide orbit) planets
increase more rapidly than bound planets at low masses.
Thus our main conclusion discussed bellow with the broken power law model, 
which the lower mass planets are increasingly scattered, is not specific to the \citet{suzuki2016} broken
power-law model.


As a comparison, we transformed the bound planet's mass ``ratio" function
of \cite{suzuki2016} to a mass function by using 
the estimated average mass of their hosts of $\sim 0.56M_\sun$ 
as shown\footnote{The 1$\sigma$ range indicated by the gray shaded area in Figure \ref{fig:IMF} does not match the one provided in \citet{suzuki2016}. This was due to an error in the \citet{suzuki2016} figure, but there is no error in the other results in that paper.}
in Figure \ref{fig:IMF}.
We estimate the abundance of the wide-orbit bound planets to be
$f_{\rm wide}=1.1_{-0.3}^{+0.6}$ planets star$^{-1}$
 in the mass range $10^{-6} < M/M_\sun < 0.02$ ($0.33<M/M_\earth < 6660$) and 
 separation range $0.3 < s < 5$, 
 which corresponds to a semi-major axis of roughly $0.7<a/{\rm au}<12$.
 This indicates that the abundance of FFP, $f=21^{+23}_{-13}$
planets star$^{-1}$, is  $19_{-13}^{+23}$
times more than wide-orbit bound planets in this mass range.

This is because the number of wide-orbit bound planets decreases at lower masses than the break
at  $M_{\rm break}\approx 1.0\times10^{-4} M_\sun$, 
while the number of high-mass bound planets is larger than that for FFP.
Again, this is consistent with the hypothesis that the low-mass planets 
are more likely to be scattered.
Note that there is still large uncertainty in the MF 
at low masses for both bound and unbound planets. It is very important to constrain the 
these MFs at low masses.

We can also compare our number for the FFP abundance with
the abundance of the bound planets with short period orbits of
$P=0.5-256$ days and planetary radii of $R_{\rm p}=0.5-4 R_\earth$
found by Kepler. \cite{Hsu2019} find
$f_{\rm FGK}= 3.5_{-0.6}^{+0.7}$ for FGK dwarfs  and \cite{Hsu2020} find
$f_{\rm M}  = 4.2_{-0.6}^{+0.6}$ for M dwarfs.
Because the typical spectral types of their samples are 
G2 ($M=1M_\sun$) and M2.5 ($M=0.4M_\sun$),
their typical semi-major axis are $0.012\lesssim a/{\rm au} \lesssim 0.79$ and
$0.009 \lesssim a/{\rm au} \lesssim 0.58$, respectively.
The fraction of FGK and M dwarfs relative to all population except BH and NS 
are  $0.157 : 0.465$ in our best fit MF.
By weighting with these stellar type fractions, the abundance of the known close-orbit bound planets is about  
$f_{\rm close}= 2.5_{-0.3}^{+0.3}$ per star. (This ignores the relatively small number of gas
giant planets in short period orbits \citep{bryant23}).


The total abundance of the wide-orbit and known close-orbit bound planets is about  
$f_{\rm bound}= 3.6_{-0.4}^{+0.7}$ per star. 
This indicates that the abundance of FFP, $f=21^{+23}_{-13}$
planets star$^{-1}$, is 
$5.8_{-3.8}^{+6.4}$ 
times more than known bound planets in this mass range.




We found the total mass of FFPs or distant planets per star 
is $m =  80^{+73}_{-47} M_\earth (0.25_{-0.15}^{+0.23} M_{\rm J})$ star$^{-1}$ in this
$10^{-6} < M/M_\sun < 0.02$ ($0.33<M/M_\earth < 6660$) mass range.
This is comparable to the value of
$91_{-22}^{+33}$ $M_\earth$ star$^{-1}$
for wide-orbit bound planets with separations of $0.3 < s < 5$ in the same mass range.
It is not straight forward to estimate the total mass of inner planet found by Kepler because only a small,
and somewhat biased, sample of Kepler planets have mass measurements.
The total masses of FFP and bound planets are less dependent on the uncertainty of the 
number of low mass planets than the total numbers of FFP and bound planets are.

These comparisons indicate that 
$19_{-13}^{+23}$
times more
planets than the ones currently in wide orbits 
have been ejected to unbound or very wide orbits.
These comparisons also suggest that the total mass of 
scattered planets is of the same order as those remaining bound
in wide orbits (beyond the snow line) in their planetary systems.
The low mass bound planets in wide orbits are much less abundant than those orbiting
closer to their host stars.
This may be explained by that planets in wide orbits are more easily ejected than those in 
close orbit.

The power-law index of the IMF of planets formed in wide orbits in 
protoplanetary disks is likely to be $\alpha_4\sim 0.9$ 
with an abundance of 
$22_{-13}^{+23}$ planets star$^{-1}$
or 
$171_{-52}^{+80} M_\earth(0.54_{-0.16}^{+0.25} M_{\rm J})$ star$^{-1}$.


Another, rather speculative, possibility is that most of the low-mass objects
found by microlensing are primordial black holes (PBH).
\cite{Hashino2022} predicted PBH generated at a first order electroweak phase transition
have masses of about $10^{-5} M_\sun$. They found that depending on parameters 
of the phase transition a sufficient number of PBH can be 
produced to be observed by current and future microlensing surveys.
The mass of such PBH is a function of the time of their generation, i.e., 
the electroweak phase transition, and is expected to be a  delta-function distribution.
To differentiate PBH from FFP, we need to measure the shape of 
the MF accurately.
This can be done by the current (MOA, OGLE, KMTNet) surveys, 
the near future (PRIME) ground telescope and the Roman Space telescope.

For the first time, we have determined the detection efficiency as a function of both the Einstein
radius crossing time and the angular Einstein radius, because finite source effects
have a large influence on the detectability of microlensing events due to low-mass planets.
This method is necessary for reliable results for low-mass FFPs, and it should be very useful 
for the analysis of these future surveys which will detect many short events.

A precise measurement of the free floating planet mass function will require a microlensing survey
that can obtain precise photometry of main sequence stars with relatively low magnification, because
the small angular Einstein radii, $\theta_{\rm E}$, of low-mass planetary lenses prevent high magnification.
The exoplanet microlensing survey of the Roman Space Telescope is such a survey, and it should
provide the definitive measurement of the free floating planet mass function.

%----------------------------------------------------------------------------------------
\acknowledgments
We are grateful to ** for helpful comments.
The MOA project is supported by JSPS KAKENHI Grant Number JSPS24253004, JSPS26247023, JSPS23340064, JSPS15H00781, JP16H06287, JP17H02871 and JP22H00153.
NK was supported by the JSPS overseas research fellowship.
DPB acknowledges support from NASA grants 80NSSC20K0886 and 80NSSC18K0793.


%----------------------------------------------------------------------------------------
%\appendix

%\section{Appendix information}



%\begin{longrotatetable}
%\begin{deluxetable*}{lllrrrrrrll}
%\tablecaption{Observable Characteristics of 
%Galactic/Magellanic Cloud novae with X-ray observations\label{chartable}}
%\tablewidth{700pt}
%\tabletypesize{\scriptsize}
%\tablehead{
%\colhead{Name} & \colhead{V$_{max}$} & 
%\colhead{Date} & \colhead{t$_2$} & 
%\colhead{FWHM} & \colhead{E(B-V)} & 
%\colhead{N$_H$} & \colhead{Period} & 
%\colhead{D} & \colhead{Dust?} & \colhead{RN?} \\ 
%\colhead{} & \colhead{(mag)} & \colhead{(JD)} & \colhead{(d)} & 
%\colhead{(km s$^{-1}$)} & \colhead{(mag)} & \colhead{(cm$^{-2}$)} &
%\colhead{(d)} & \colhead{(kpc)} & \colhead{} & \colhead{}
%} 
%\startdata
%\enddata
%\end{deluxetable*}
%\end{longrotatetable}



\begin{thebibliography}{}
\bibitem[Agol(2003)]{Agol2003}Agol, E. 2003, \apj, 594, 449
%\bibitem[Afonso et al.(2003)]{afo03}Afonso, C. et al. 2003, A\&A, 404, 145
%\bibitem[Alard(2000)]{ala00}Alard C., 2000, A\&AS, 144, 363
%\bibitem[Alard \& Lupton(1998)]{ala98}Alard C., Lupton R. H., 1998, ApJ, 503, 325
%\bibitem[Alcock et al.(1995)]{alc95}Alcock, C. et al. 1995, ApJ, 445, 133
%\bibitem[Alcock et al.(1997)]{alc97}Alcock, C. et al. 1997, ApJ, 486, 697
%\bibitem[Alcock et al.(2000a)]{alc00a}Alcock C. et al., 2000a, ApJ, 541, 270 
%\bibitem[Alcock et al.(2000b)]{alc00b}Alcock C. et al., 2000b, ApJ, 541, 734
%\bibitem[Bennett(2008)]{bennett_rev} Bennett, D.P, 2008, in Exoplanets, ed. J. Mason (Berlin: Springer), 47
%\bibitem[Bennett \& Rhie(2002)]{ben02}Bennett, D.P. \& Rhie, S.H.\ 2002, \apj, 574, 985
\bibitem[Bennett et al.(2010)]{bennett-ogle109} Bennett, D.~P., Rhie, S.~H., Nikolaev, S., et~al.\ 2010, \apj, 713, 837
\bibitem[Bennett et al.(2012)]{bennett2012} Bennett, D.~P., Sumi, T., Bond, I. A., et~al.\ 2012, \apj, 757, 119
%\bibitem[Bensby et al.(2011)]{Bensby2011}Bensby, T.,  Ad\'{e}n, D., Mel\'{e}ndez, J., et al. 2011, A\&A, 533, A134
%\bibitem[Bensby et al.(2013)]{Bensby2013}Bensby, T., Yee, J. C., Feltzing, S., et al. 2013, A\&A, 549, A147
%\bibitem[Binney et al.(2000)]{bin00}Binney, J. et al. 2000, ApJ, 537, L99
%\bibitem[Binney, Gerhard \& Spergel(1997)]{bin97}Binney, J. Gerhard, O. \& Spergel, D.  1997, MNRAS, 288, 365
%\bibitem[Bissantz \& Gerhard(2002)]{bis02}Bissantz, N. \& Gerhard, O. 2002, MNRAS, 330, 591
%\bibitem[Bissantz, Debattista \& Gerhard(2004)]{bis04}Bissantz, N., Debattista, V. P. \& Gerhard, O. 2004, ApJ, 601, L155 
%\bibitem[Bond et al.(2001)]{bon01}Bond I. A. et al., 2001, MNRAS, 327, 868
%\bibitem[Boyajian et al.(2014)]{Boyajian2014}Boyajian, T. S., van Belle, G., \& von Braun, K. 2014, AJ, 147, 47
\bibitem[Boss et al.(2003)]{bos03} Boss, A.~P., Basri, G., Kumar, S.~S., et al.\ 2003, Brown Dwarfs, 211, 529
\bibitem[Bozza et al.(2018)]{Bozza2018}Bozza V., Bachelet, E., \& Bartoli\'{c}, F., et al., 2018, MNRAS, 479, 5157
\bibitem[Bryant et al.(2023)]{bryant23} Bryant, E.~M., Bayliss, D., \& Van Eylen, V.\ 2023, arXiv:2303.00659. doi:10.48550/arXiv.2303.00659
%\bibitem[Cassan et al.(2012)]{Cassan2012}Cassan, A., Kubas, D., Beaulieu, J. P., et al. 2012, Nature, 481, 167, doi: 10.1038/nature10684
%\bibitem[CERN Lib.(1998)]{CERN98}CERN Lib., 1998, http://wwwasdoc.web.cern.ch/wwwasdoc/minuit/minmain.html
%\bibitem[Evans \& Belokurov(2002)]{eva02} Evans N.W., \& Belokurov, 2002, ApJ, 567, 119 
%\bibitem[Dwek et al.(1995)]{dwe95} Dwek, E. et al. 1995, ApJ, 445, 716
%\bibitem[Fukui et al.(2015)]{Fukui2015} Fukui, A., Gould, A., Sumi, T.,  et al. 2015, \apj, 809, 74
%\bibitem[Gaudi(2012)]{gaudi_rev} Gaudi, B.~S.\ 2012, \araa, 50, 411 
\bibitem[Gaudi et al.(2008)]{gaudi-ogle109} Gaudi, B.~S., Bennett, D.~P., Udalski, A., et al.\ 2008, Science, 319, 927
%\bibitem[Gould et al.(2010)]{Gould2010}Gould, A., et al. 2010, \apj, 720 1073
%\bibitem[Gould(1992)]{gou92}Gould, A. 1992, ApJ, 392, 442
%\bibitem[Gould(1994a)]{gou94a}Gould, A. 1994a, ApJ, 421, L71 
%\bibitem[Gould \& An(2002)]{gou02}Gould, A. \& An, J. H. 2002, ApJ, 565, 1381
\bibitem[Gould et al. (2022)]{Gould2022}Gould, A. et al. 2022, Journal of the Korean Astronomical Society, 55, 173
%\bibitem[Green et al.(2012)]{green2012}Green, J. et al., 2012, preprint, astro-ph/1208.4012
%\bibitem[Griest et al.(1991)]{gri91}Griest, K., et al. 1991, ApJ, 372, L79
%\bibitem[Gyuk(1999)]{gyu99}Gyuk, G. 1999, ApJ, 510, 205 
%\bibitem[Hamadache et al.(2006)]{ham06} Hamadache, C., Le Guillou, L., Tisserand, P., et al.\ 2006, \aap, 454, 185
%\bibitem[Han \& Gould(1995)]{han95}Han, C. \& Gould, A. 1995, ApJ, 449, 521
\bibitem[Han \& Gould(1996)]{han96} Han, C. \& Gould, A.\ 1996, \apj, 467, 540. doi:10.1086/177631
%\bibitem[Han \& Gould(2003)]{han03}Han, C. \& Gould, A. 2002, ApJ, 592, 172
%\bibitem[Han(1999)]{han99}Han, C. 1999, MNRAS, 309, 373
%\bibitem[Han \& Gould(2003)]{han03}Han, C. \& Gould, A. 2003, ApJ, 592, 172
\bibitem[Hashino et al.(2022)]{Hashino2022} Hashino, K., Kanemura, S., \& Takahashi, T.\ 2022, Physics Letters B, 833, 137261. doi:10.1016/j.physletb.2022.137261
%\bibitem[Hirao et al. (2020)]{Hirao2020}Hirao, Y., Bennett, D. P. \& Ryu, Y-H. 2020, \aj, 160, 74
\bibitem[Hogg et al.(2010)]{hog10} Hogg, D.~W., Myers, A.~D., \& Bovy, J.\ 2010, \apj, 725, 2166. doi:10.1088/0004-637X/725/2/2166
\bibitem[Holtzman et al.(1998)]{hol98}Holtzman, J.~A., Watson, A.~M., Baum, W.~A., et al.\ 1998, \aj, 115, 1946
%\bibitem[James(1994)]{James1994}James F., 1994, MINUIT reference Manual, https://root.cern.ch/download/minuit.pdf
%\bibitem[Jaroszy\'{n}ski(2002)]{jar02}Jaroszi\'{n}ski, M., 2002, Acta Astronomica, 52, 39
%\bibitem[Jaroszy\'{n}ski et al.(2004)]{jar04}Jaroszy\'{n}ski, M. et al., 2004, AcA, 54, 103
%\bibitem[Juri{\'c} \& Tremaine(2008)]{juric08} Juri{\'c}, M. \& Tremaine, S.\ 2008, \apj, 686, 603. doi:10.1086/590047
%\bibitem[Kerins, Robin \& Marshal(2009)]{kerins2009} Kerins, E., Robin, A. C., \& Marshal, D. J. 2009, MNRAS, 396, 1202
%\bibitem[Kim et al.(2010)]{kmtnet2010} Kim, S.-L., Park, B.-G., Lee, C.-U., et al.\ 2010, \procspie, 7733, 77333F
\bibitem[Hsu, Ford, Ragozzine \& Ashby(2019)]{Hsu2019} Hsu, D. C., Ford, E. B., Ragozzine, D.\& Ashby, K. \ 2019, \aj, 158, 109
\bibitem[Hsu, Ford \& Terrien(2020)]{Hsu2020} Hsu, D. C., Ford, E. B., Terrien, R. \ 2020, \mnras, 498, 2249
\bibitem[Jung et al. (2019)]{Jung2019}Jung, Y. K., Gould, A., \& Zang, W., et al. 2019, \aj, 157, 72
\bibitem[Kim et al.(2021)]{Kim2021} Kim, H.-W., Hwang, K.-H., Gould, A., et al.\ 2021, \aj, 162, 15
%\bibitem[Kiraga \& Paczy\'{n}ski(1994)]{kir94}Kiraga, M., \& Paczy\'{n}ski, B. 1994, ApJ, 430, L101
%\bibitem[Kiraga, Paczy\'{n}ski \& Stanek(1997)]{kir97}Kiraga, M., Paczy\'{n}ski, B.  \& Stanek, K. Z.,  1997, ApJ, 485, 611
%\bibitem[Kondo et al. (2019)]{Kondo2019}Kondo, I., Sumi, T. \& Bennett, D. P. 2019, \aj, 158, 224
\bibitem[Koshimoto et al. (2021a)]{Koshimoto2021} Koshimoto, N., Baba, J., \& Bennett, D.~P.\ 2021a, \apj, 917, 78. doi:10.3847/1538-4357/ac07a8
\bibitem[Koshimoto et al.(2021b)]{kos21b} Koshimoto, N., Bennett, D.~P., Suzuki, D., et al.\ 2021b, \apjl, 918, L8. doi:10.3847/2041-8213/ac17ec
\bibitem[Koshimoto et al.(2023)]{Koshimoto2023} Koshimoto, N., Sumi, T. and Bennett, D.~P., et al.\ 2023, \apj, submitted \citepalias{Koshimoto2023}
\bibitem[Lam et al.(2020)]{lam20} Lam, C.~Y., Lu, J.~R., Hosek, M.~W., et al.\ 2020, \apj, 889, 31. doi:10.3847/1538-4357/ab5fd3
\bibitem[Lin \& Ida(1997)]{lin97} Lin, D.~N.~C. \& Ida, S.\ 1997, \apj, 477, 781. doi:10.1086/303738
\bibitem[Maeder(1973)]{Maeder1973}Maeder, A. 1973, A\&A, 26, 215
\bibitem[Mao \& Paczy\'{n}ski(1991)]{mao1991}Mao, S., \& Paczy\'{n}ski, B. 1991, ApJ, 374, L37
\bibitem[Metropolis et al.(1953)]{metrop} Metropolis, N., Rosenbluth, A.~W., Rosenbluth, M.~N., Teller, A.~H.,
   \& Teller, E.\ 1953, \jcp, 21, 1087
%\bibitem[Moniez(2010)]{moniez10} Moniez, M.\ 2010, General Relativity and Gravitation, 42, 2047 
\bibitem[Mr\'{o}z et al.(2017)]{Mroz17} Mr\'{o}z, P. Udalski A., Skowron, J.,  et al. 2017, Nature, 548, 183
\bibitem[Mr\'{o}z et al.(2018)]{Mroz18} Mr\'{o}z, P. Y.-H.Ryu, Skowron, J.,  et al. 2018, \apj, 155, 121
\bibitem[Mr\'{o}z et al.(2019)]{Mroz19} Mr\'{o}z, P. Udalski A., Skowron, J.,  et al. 2019, \apjs, 244, 29
\bibitem[Mr\'{o}z et al.(2019b)]{Mroz19b} Mr\'{o}z, P. Udalski A., Bennett, D. P.  et al. 2019, \aa, 622, A201
\bibitem[Mr\'{o}z et al.(2020a)]{Mroz20a} Mr\'{o}z, P. Udalski A., Szyma\'{n}ski, M.,  et al. 2020, \apjs, 249, 16
\bibitem[Mr\'{o}z et al.(2020b)]{Mroz20b} Mr\'{o}z, P., Poleski, R. \& Gould, A. 2020, \apj, 903, 11
\bibitem[Mr\'{o}z et al.(2020c)]{Mroz20c} Mr\'{o}z, P., Poleski, R., Han, C. 2020, \aj, 159, 262
%\bibitem[Nataf et al.(2012)]{Nataf2012}Nataf, D. M., et al. 2012, \apj, 769, 88
%\bibitem[Nataf et al.(2016)]{Nataf2016}Nataf, D. M., et al. 2016, \mnras, 456, 2692
%\bibitem[Nemiroff \& Wickramasinghe(1994)]{nem94}Nemiroff, R. J. \& Wickramasinghe, W. A. D. T. 1994, ApJ, 424, 21
%\bibitem[Niikura et al.(2019a)]{Niikura2019a}Niikura, H., Takada, M. \& Yasuda, N. et al. 2019, Nature Astronomy. 3, 524
%\bibitem[Niikura et al.(2019b)]{Niikura2019b}Niikura, H., Takada, M. \& Yokoyama, S. et al. 2019, PhRvD. 99, 3503
%\bibitem[Novati et al.(2008)]{novati2008}Novati S.C., Luca, F. De., Jetzer, Ph.,  Mancini, L., \& Scarpetta, G.  2008, A\&A, 480, 723
%\bibitem[Oasa et al(1999)]{Oasa1999}Oasa,Y., et al,  1999, ApJ, 526, 336 
\bibitem[Paczy\'{n}ski(1986)]{pac86}Paczy\'{n}ski, B.  1986, ApJ, 304, 1 
\bibitem[Paczy\'{n}ski(1991)]{pac91}Paczy\'{n}ski, B.  1991, ApJ, 371, L63
%\bibitem[Paczy\'{n}ski et al.(1994)]{pac94}Paczy\'{n}ski, B. et al. 1994, ApJ., 435, L113
%\bibitem[Paczy\'{n}ski(1996)]{pac96}Paczy\'{n}ski, B. 1996, ARA\&A, 34, 419
%\bibitem[Palanque-Delabrouille et al.(1998)]{pal98} Palanque-Delabrouille, N., et al. 1998, \aa, 332, 1
%\bibitem[Park et al.(2004)]{par04}Park, B. -G. et al. 2004, ApJ, 609, 166 
%\bibitem[Peale(1998)]{pea98}Peale, S. J. 1998, ApJ, 509, 177
%\bibitem[Penny et al.(2013)]{penny13} Penny, M.~T., Kerins, E., Rattenbury, N., et al.\ 2013,  \mnras, 434, 2
%\bibitem[Poindexter et al.(2005)]{poi05} Poindexter, S., et al.\ 2005, \apj, 633, 914
%\bibitem[Popowski et al.(2001)]{pop01}Popowski, P. et al. 2001, in ASP Conference Series: Microlensing 2000: A New Era of Microlensing Astrophysics, eds. J.W. Menzies \& P.D. Sackett (San Francisco: Astronomical Society of the Pacific), Vol. 239, p. 244, (astro-ph/0005466)
%\bibitem[Popowski et al.(2005)]{pop05}Popowski, P. et al., \apj, 631, 879
\bibitem[Rasio \& Ford(1996)]{rasio96} Rasio, F.~A. \& Ford, E.~B.\ 1996, Science, 274, 954. doi:10.1126/science.274.5289.954
\bibitem[Riffeser et al.(2006)]{Riffeser2006}Riffeser, A., Fliri, J., Seitz, S., \& Bender, R. 2006, \apjs, 163, 225
\bibitem[Ryu et al.(2021)]{Ryu2021} Ryu, Y-H, Mr\'{o}z, P., Gould, A. 2021, \aj, 161, 126
%\bibitem[Sako et al.(2008)]{sako2008}Sako, T., et al. 2008, Experimental Astronomy, 22, 51
%\bibitem[Schechter, Mateo \& Saha(1993)]{sch93}Schechter, L., Mateo, M., \& Saha, A., 1993, PASP, 105, 1342S
%\bibitem[Shvartzvald \& Maoz(2012)]{wise_survey} Shvartzvald, Y., \& Maoz, D.\ 2012, \mnras, 419, 3631 
%\bibitem[Shvartzvald et al.(2016)]{Shvartzvald2016} Shvartzvald, Y.,  Maoz, D. \& A. Udalski\ 2016, \mnras, 457, 4089 
%\bibitem[Smith, Mao \& Wo\'{z}niak(2002)]{smi02}Smith, M. C., Mao, S. \& Wo\'{z}niak, P. R., 2002, MNRAS, 332, 962
%\bibitem[Smith, Wo\'{z}niak, Mao \& Sumi(2007)]{smith2007}Smith, M. C., Wo\'{z}niak, P. R., Mao, S. \& Sumi, T., 2007, MNRAS, 380, 805
%\bibitem[Soszy\'{n}ski et al.(2001)]{sos01}Soszy\'{n}ski, I. et al. 2001, ApJ, 552, 731
%\bibitem[Stanek et al.(2000)]{sta00}Stanek, K. Z. et al. 2000, Acta Astronomica, 50, 191
%\bibitem[Stetson(1987)]{daophot}Stetson, P.~B. DAOPHOT - A computer program for crowded-field stellar photometry. \pasp {\ \bf 99}, 191-222 (1987)
%\bibitem[Sumi(2004)]{sumEX04}Sumi, T., 2004, MNRAS, 349, 193
%\bibitem[Sumi, Eyer \& Wo\'{z}niak(2003)]{sumEW03}Sumi T., Eyer, L. \& Wo\'{z}niak, P. R., 2003, MNRAS, 340, 1346
\bibitem[Sumi et al.(2003)]{sumi03}Sumi, T. et al., 2003, ApJ, 591, 204  
%\bibitem[Sumi et al.(2004a)]{sume04}Sumi, T. et al., 2004a, MNRAS, 348, 1439
%\bibitem[Sumi et al.(2004b)]{sumQSO04}Sumi, T. et al., 2004b, MNRAS, 356, 331
%\bibitem[Sumi et al.(2006)]{sumi2006}Sumi, T. et al., 2006, ApJ, 636, 240
\bibitem[Sumi et al.(2010)]{sumi2010}Sumi, T. et al., 2010, ApJ, 710, 1641
\bibitem[Sumi et al.(2011)]{sumi2011}Sumi, T. et al., 2011, Nature, 473, 349
\bibitem[Sumi et al.(2013)]{sumi2013}Sumi, T. Bennett, D. P. \& Bond I. A. et al., 2013, ApJ, 778, 150
\bibitem[Sumi et al.(2016a)]{sumi2016a}Sumi, T., Udalski, A., \& Bennett, D. P.  et al., 2016, ApJ, 825, 112
%\bibitem[Sumi \& Penny (2016)]{sumi2016}Sumi, T. \& Penny, M. T. , 2016, ApJ, 827, 139
\bibitem[Suzuki et al.(2016)]{suzuki2016}Suzuki, D. Bennett, D. P. \& Sumi T. et al., 2016, ApJ, 833, 145
%\bibitem[Tomany \& Crotts(1996)]{tom96}Tomany, A. B. \& Crotts, A. P., 1996, AJ, 112, 2872
%\bibitem[Thomas et al.(2004)]{tho04}Thomas, C.L.. et al., 2004, astro-ph/0410341
%\bibitem[Udalski et al.(2000)]{uda00}Udalski A., Zebru\'{n} K., Szyma\'{n}ski, M., Kubiak M., Pietrzy\'{n}ski G., Soszy\'{n}ski I., Wo\'{z}niak P. R. 2000, Acta Astronomica, 50, 1
\bibitem[Udalski et al.(1994)]{uda94}Udalski, A. et al. 1994, Acta Astronomica, 44, 165
%original EWS paper
%\bibitem[Udalski et al.(1994)]{uda94}Udalski, A., Szymanski, M., Kaluzny, J., Kubiak, M., Mateo, M., Krzeminski, W.,
%Paczynski, B. 1994a, Acta Astronomica, 44, 227
%\bibitem[Udalski et al.(2002)]{uda02}Udalski A. et al. 2002, Acta Astronomica, 52, 217
%\bibitem[Udalski, Kubiak \& Szyma\'{n}ski(1997)]{uda97}Udalski A.,  Kubiak M., Szyma\'{n}ski M. K., 1997, Acta Astronomica, 47, 319 
%\bibitem[Udalski(2003)]{uda03}Udalski, A. 2003, Acta Astronomica, 53, 291
\bibitem[Szyma\'{n}ski(2011)]{uda11}Szyma\'{n}ski, M., Udalski, A. \&  Soszy\'{n}ski I. et al. 2011, Acta Astronomica, 61, 83
%\bibitem[Udalski, Szyma\'{n}ski and Szyma\'{n}ski(2015)]{UdalskiSzymanski2015}Udalski,A., Szyma\'{n}ski, M.K. and Szyma\'{n}ski, G. 2015, Acta Astronomica, 65, 1
\bibitem[Udalski et al.(2018)]{Udalski2018}Udalski,A., Ryu, Y.-H., Sajadian, S., et al. 2018, AcA, 68, 1
%\bibitem[Youn Kil et al.(2020)]{YounKil2020}Youn Kil, J. Gould, A. \& Udalski, A. 2020, AJ, 160, 148
\bibitem[Weidenschilling \& Marzari(1996)]{weiden96} Weidenschilling, S.~J. \& Marzari, F.\ 1996, \nat, 384, 619. doi:10.1038/384619a0
%\bibitem[Witt \& Mao(1994)]{wit94}Witt, H. J. \& Mao, S. 1994, ApJ, 430, 505
%\bibitem[Wood \& Mao(2005)]{wood05} Wood, A., \& Mao, S. 2005,  \mnras, 362, 945 
%\bibitem[Wo\'{z}niak \& Paczy\'{n}ski(1997)]{woz97}Wo\'{z}niak P. R., \& Paczy\'{n}ski, B. 1997, ApJ, 487, 55
%\bibitem[Wo\'{z}niak(2000)]{wozniak2000}Wo\'{z}niak P. R., 2000, Acta Astronomica, 50, 421
%\bibitem[Wo\'{z}niak et al.(2001)]{woz01}Wo\'{z}niak, P. R., et al. 2001, Acta Astronomica, 51, 175
%\bibitem[Zhao \& Mao(1996)]{zha96}Zhao, H. \& Mao, S. 1996, MNRAS, 283, 1197
\bibitem[Zang et al.(2022)]{zang22} Zang, W., Yang, H., Han, C., et al.\ 2022, \mnras, 515, 928. doi:10.1093/mnras/stac1883
%\bibitem[Zapatero Osorio(2000)]{Zapatero2000}Zapatero Osorio, M. R. et al.  2000, Science 290, 103
%\bibitem[Zhao, Spergel \& Rich(1995)]{zha95}Zhao, H., Spergel, D. N. \& Rich, R. 1995, ApJ, 440, L13


\end{thebibliography}















\end{document}

% End of file `sample61.tex'.
