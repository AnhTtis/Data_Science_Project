\chapter{Introduction}\label{chap:intro}


Given a non-negative function \(f:\mathbb{R}^n\rightarrow\mathbb{R}\), it is often useful to decompose \(f\) into a finite sum of squares of real-valued functions and write
\begin{equation}\label{eq:sos}
    f=\sum_{j=1}^m g_j^2.
\end{equation}
Usually, though, it is not straightforward to see what properties the functions \(g_1,\dots,g_m\) can inherit from \(f\). In this thesis we assume that \(f\) belongs to the H\"older space \(C^{k,\alpha}(\mathbb{R}^n)\) for a non-negative integer \(k\) and \(0<\alpha\leq 1\), and we investigate sufficient conditions for there to exist functions \(g_1,\dots,g_m\) which satisfy \eqref{eq:sos} and which belong to the `half-regular' H\"older space
\[
    C^\frac{k+\alpha}{2}(\mathbb{R}^n)=\begin{cases}\hfil C^{\frac{k}{2},\frac{\alpha}{2}}(\mathbb{R}^n) & k\textrm{ even}, \\ C^{\frac{k-1}{2},\frac{1+\alpha}{2}}(\mathbb{R}^n) & k\textrm{ odd}.
    \end{cases}
\]
In other words, we explore when \(f\) can be decomposed into a finite sum of squares of functions that are essentially half as regular as \(f\). Remarkably, we find that \eqref{eq:sos} always holds for some half-regular functions \(g_1,\dots,g_m\) whenever \(k\leq 3\) or \(n=1\). When \(k\geq 4\) and \(n\geq 2\) we are able to construct examples of non-decomposable functions, except in the special case of \(C^{4,\alpha}(\mathbb{R}^2)\).


The problem of finding sum of squares decompositions that preserve regularity is interesting in the setting of classical analysis, and it has important consequences elsewhere in mathematics, especially in the study of partial differential equations. It is shown in \cite{SOS_III} that if the symbol of a second-order differential operator \(L\) can be written as sums of squares with enough regularity, then one can find sufficient and sometimes necessary conditions for \(L\) to be hypoelliptic. This means that a distribution \(u\) must be smooth if \(Lu\) is smooth. 


For another application, Guan shows in \cite{Guan} that if a non-negative function \(f\) defined on a domain \(\Omega\subset\mathbb{R}^n\) can be written as a sum of \(C^{1,1}(\Omega)\) squares, then solutions to the Dirichlet problem for the Monge-Amp\`ere equation
\begin{align*}
    D^2u=f & \;\;\textrm{ in }\Omega,\\
    \hfil u=0 & \;\;\textrm{ on }\partial\Omega,
\end{align*}
also belong to \(C^{1,1}(\Omega)\). A similar regularity preserving decomposition is employed by Tataru to study pseudodifferential operators in \cite{Tataru}, and these examples highlight the wide applicability of decompositions like \eqref{eq:sos}. In this thesis we do not present any novel applications in partial differential equations, but we emphasize that our decomposition results may be useful in addressing several open problems in the field; see \Cref{sec:applics} for further discussion.


Regularity preserving sum of squares decompositions have already been studied in several settings. Fefferman \& Phong showed in \cite{Fefferman-Phong} that if \(f\in C^{3,1}(\mathbb{R}^n)\) and \(f\) is non-negative, then there exist functions \(g_1,\dots,g_m\in C^{1,1}(\mathbb{R}^n)\) for which \eqref{eq:sos} holds. This result has been extended to various other H\"older spaces by Bony et al. in \cite{Bony}, Tataru in \cite{Tataru}, and Sawyer \& Korobenko in \cite{SOS_I}. In this work, we generalize the decomposition results from each of these works by studying non-negative \(C^{k,\alpha}(\mathbb{R}^n)\) functions for arbitrary \(k\geq 0\) and \(0<\alpha\leq 1\).


Finding a regularity preserving decomposition is often difficult. At a glance, one might suspect that taking a square root of a non-negative function \(f\in C^{k,\alpha}(\mathbb{R}^n)\) affords the desired decomposition with one function. However, this approach can fail spectacularly -- in \cite{Bony}, Bony gives an example of a non-negative \(C^\infty(\mathbb{R})\) function whose square root is not in \(C^{1,\alpha}(\mathbb{R})\) for any positive \(\alpha\). This example is given by taking \(f(0)=0\) and for \(x\neq0\) defining
\[
    f(x)=e^{-\frac{1}{|x|}}\bigg(\sin^2\bigg(\frac{\pi}{|x|}\bigg)+e^{-\frac{1}{x^2}}\bigg).
\]
Thus, if the functions \(g_1,\dots,g_m\) in \eqref{eq:sos} are to inherit half of the regularity of \(f\), then in general it is necessary that \(m> 1\) even when working with functions on the line. The exception is when \(f\in C^{\alpha}(\mathbb{R}^n)\) or \(C^{1,\alpha}(\mathbb{R}^n)\), in which case \(\sqrt{f}\) is indeed half as regular as \(f\) as we show in \Cref{chap:holderchap}.


There do not always exist decompositions of the form \eqref{eq:sos} which inherit half-regularity. Bony shows in \cite{Bony} that there are \(C^4(\mathbb{R}^4)\) functions that are not sums of squares in \(C^2(\mathbb{R}^4)\), suggesting that additional hypotheses are needed to guarantee the existence of regularity preserving decompositions. We introduce such conditions in \Cref{chap:higher}, to try and understand when functions in \(C^{k,\alpha}(\mathbb{R}^n)\) can be written as sums of half-regular squares when \(k\geq 4\). Though we make progress on this problem, a characterization of the decomposable functions remains elusive.


In \Cref{sec:gen} we also develop some techniques for constructing non-decomposable functions in \(C^{k,\alpha}(\mathbb{R}^n)\) for \(k\geq 4\). Our examples turn out to be polynomials which cannot be written as sums of squares of polynomials. The existence of such polynomials was proved by Hilbert in \cite{Hilbert}, but an explicit example did not appear in the literature until Motzkin found one in \cite{Motzkin} nearly 80 years later. Several more examples have since been found, many in \cite{reznik_extpsd}. Motivated by the question of decomposability of functions in \(C^{k,\alpha}(\mathbb{R}^n)\), we devise a method to construct non-negative polynomials which are not sums of squares in all possible cases.

Our findings concerning the possibility of forming a regularity preserving sum of squares decompositions in \(C^{k,\alpha}(\mathbb{R}^n)\), for every \(k\geq 0\) and \(n\geq 1\), are summarized in the following table.


\medskip
\renewcommand{\arraystretch}{1.3}
\newcommand\highbrace[2]{%
  \left.\rule{0pt}{#1}\right\}\text{#2}}
\setlength{\tabcolsep}{3.5pt}
\begin{center}
Is every non-negative \(C^{k,\alpha}(\mathbb{R}^n)\) function a sum of squares in \(C^\frac{k+\alpha}{2}(\mathbb{R}^n)\) (\(0<\alpha\leq 1\))?\\
\medskip
\begin{tabular}{|c|cc|cc|cc|cc|cc|cc|}
\hline
\diagbox[height=2em,width=3em]{\(n\)}{\(k\)} & \multicolumn{2}{c|}{0} & \multicolumn{2}{c|}{1} & \multicolumn{2}{c|}{2} & \multicolumn{2}{c|}{3} & \multicolumn{2}{c|}{4} & \multicolumn{2}{c|}{\(\geq 4\)}\\
\hline
1 & Yes &\multirow{4}{*}{\kern-0.4em\(\highbrace{7ex}{\makecell{Cor.\\\ref{cor:easycase}}}\)}& Yes &\multirow{4}{*}{\kern-0.4em\(\highbrace{7ex}{\makecell{Thm.\\\ref{thm:rootsin}}}\)}& Yes&\multirow{4}{*}{\kern-0.4em\(\highbrace{7ex}{\makecell{Thm.\\\ref{thm:c3main}}}\)} & Yes&\multirow{4}{*}{\kern-0.4em\(\highbrace{7ex}{\makecell{Thm.\\\ref{thm:c3main}}}\)} & \multicolumn{2}{c|}{Yes (\cite{Bony2})} & \multicolumn{2}{c|}{Yes (\cite{Bony2})} \\
2 & Yes && Yes && Yes && Yes && \multicolumn{2}{c|}{Unknown (\S \ref{sec:maybe})} & No & \multirow{3}{*}{\kern-2em\(\highbrace{5.3ex}{\makecell{Cor.\\\ref{cor:existence}}}\)} \\
3 & Yes && Yes && Yes && Yes && No &\multirow{2}{*}{\quad\(\highbrace{3.5ex}{\makecell{Cor.\\\ref{cor:existence}}}\)}& No &\\
\(\geq 3\) & Yes && Yes && Yes && Yes && No && No &\\
\hline
\end{tabular}
\end{center}
\bigskip


The plan for the remainder of this thesis is as follows. In \Cref{chap:holderchap} we describe the basic properties of H\"older spaces, before specializing to non-negative H\"older functions and studying their pointwise behaviour. Using these preliminary results, we show that the roots of \(C^\alpha(\mathbb{R}^n)\) and \(C^{1,\alpha}(\mathbb{R}^n)\) functions are half-regular. Then  in \Cref{chap:decomps}, we adapt the arguments employed in \cite{Tataru} and \cite{SOS_I} to prove our main result \Cref{thm:c3main}, which shows that non-negative \(C^{2,\alpha}(\mathbb{R}^n)\) and \(C^{3,\alpha}(\mathbb{R}^n)\) functions can be decomposed into sums of half-regular squares for every \(0<\alpha\leq 1\). Following \Cref{chap:poly}, in which we construct non-decomposable polynomials, we seek conditions in \Cref{chap:higher} for functions in \(C^{k,\alpha}(\mathbb{R}^n)\) to be partially decomposable when \(k\geq 4\). 


In \Cref{chap:alg} we establish supplementary results which are only tangentially related to our central study of sums of squares. Similarly, we include \Cref{chap:appendix} to collect various standard results in one place for the convenience of the reader. We conclude the thesis with \Cref{chap:last}, in which we summarize our findings and discuss potential applications of our work along with possible directions for future research. In closing this summary, we include a road map of the thesis to help clarify the hierarchy of our main results, and for the reader to refer to later on.


\medskip
{\hspace{-1em}
\begin{tikzpicture}[block/.style={rounded corners, minimum width=3cm, minimum height=2cm, draw}]
\node[block,align=center] (1) {\Cref{thm:IFT}\\[0.5em]Implicit Functions and\\ Their Derivatives};
\node[block,very thick,above right=-1.2 and 1.15 of 1,align=center] (3) {\Cref{sec:reg}\\[0.5em]Localized Roots of\\ \(C^{k,\alpha}(\mathbb{R}^n)\) Functions.};
\node[block,double,very thick,below right=-0.3 and 0.85 of 3,align=center] (5) {\Cref{thm:locdecomp}\\[0.5em] Localized Decompositions of\\ \(C^{k,\alpha}_{\mathrm{loc}}(\Omega)\) Functions for \(k\leq 3\).};
\node[block,double,very thick,below=0.7of 5,align=center] (6) {\Cref{thm:c3main} (Main Result)\\[0.5em]SOS Decompositions\\ in \(C^{k,\alpha}(\mathbb{R}^n)\) for \(k\leq 3\).};
\node[block, below=0.5of 1,align=center] (2) {\Cref{thm:cauchylike}\\[0.5em]Derivative Estimates for\\ \(C^{k,\alpha}(\mathbb{R}^n)\) Functions.};
\node[block,very thick, below=1of 3,align=center] (4) {\Cref{thm:party}\\[0.5em]Partitions of Unity and\\ Slow Variation};
\node[block,double,very thick,below=0.7of 6,align=center] (7) {\Cref{thm:seconddmain}\\[0.5em] Conditional Decompositions\\ in \(C^{k,\alpha}(\mathbb{R}^n)\) for \(k\geq 4 \).}; 
\node[block, above=0.5of 1,align=center] (8) {\Cref{lem:extbyzero}\\[0.5em]Extension of \(C^{k,\alpha}_{\mathrm{loc}}(\Omega)\)\\ Functions by Zero to \(\mathbb{R}^n\).}; 
\node[block, below=0.5of 2,align=center] (11) {\Cref{lem:cptmb}\\[0.5em]Compact Embeddings of\\Bounded H\"older Spaces}; 
\node[block, below=0.5of 11,align=center] (9) {\Cref{cor:hullcond}\\[0.5em]Criterion for Polynomials\\to be Non-Decomposable.}; 
\node[block,below=0.5of 9,align=center] (10) {\Cref{thm:posweights}\\[0.5em]Computational Method \\for Polytope Points}; 
\node[block,very thick, below=1of 4,align=center] (13) {\Cref{cor:existence}\\[0.5em]Non-Decomposable\\\(C^{k,\alpha}(\mathbb{R}^n)\) Functions.};
\node[block,very thick, below=1of 13,align=center] (12) {\Cref{sec:gen}\\[0.5em] Examples of Polynomial\\Non-Sums of Squares}; 
\begin{scope}[->, shorten >=1mm, shorten <=1mm]
\draw (2) to[out=0,in=180] (4);
\draw (4) to[out=350,in=180] (6);
\draw (1) to[out=0,in=180] (3);
\draw (3) to[out=330,in=155] (6);
\draw (2) to[out=20,in=200] (3);
\draw (10) to[out=0,in=190] (12);
\draw (9) to[out=0,in=175] (12);
\draw (6) -- (5);
\draw (6) -- (7);
\draw (8) to[out=0,in=120] (5);
\draw (11) to[out=0,in=180] (13);
\draw (12) -- (13);
\draw (13) to[out=350,in=180] (7);
\end{scope}
\end{tikzpicture}
}