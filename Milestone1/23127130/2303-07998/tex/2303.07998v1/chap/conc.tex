\chapter{Conclusion and
Discussion }\label{chap:last}


\section{Summary of Findings}


This thesis studied non-negative H\"older continuous functions, to determine when they can be decomposed as sums of squares of functions that are `half-regular' in an appropriate sense. Building upon the work of Fefferman \& Phong in \cite{Fefferman-Phong}, Tataru in \cite{Tataru} and Sawyer \& Korobenko in \cite{SOS_I}, and taking inspiration from the work of many others, we have demonstrated that if \(f\) is a non-negative \(C^{k,\alpha}(\mathbb{R}^n)\) function for \(k\leq 3\) and \(0<\alpha\leq 1\), then there exists a decomposition of the form
\begin{equation}\label{eq:sosos}
    f=\sum_{n=1}^{m_n}g_j^2,
\end{equation}
where \(g_j\in C^\frac{k+\alpha}{2}(\mathbb{R}^n)\) for each \(j\), and the constant \(m_n\) depends only on \(n\). Formerly, the version of this result in \(C^{3,1}(\mathbb{R}^n)\) was well-known thanks to Fefferman \& Phong, and in \(C^\alpha(\mathbb{R}^n)\) for \(0<\alpha\leq 1\) it is a straightforward consequence of concavity; see \Cref{lem:smallpow}. The remaining cases do not seem to have been dealt with prior to our work; accordingly, their resolution in \Cref{thm:c3main} comprises the main contribution of this thesis.


In addition to this novel result, we also presented some progress toward an adaptation of our decomposition result to large values of \(k\) in \Cref{chap:higher}, building upon the work of Sawyer \& Korobenko in \cite{SOS_I}. While the characterization of decomposable \(C^{k,\alpha}(\mathbb{R}^n)\) functions remains an open problem for \(k\geq 4\) in every dimension \(n\geq 2\), we have shown in \Cref{thm:seconddmain} that \eqref{eq:sosos} can hold under some restrictive pointwise conditions on \(f\). The conditions we impose come at the expense of some regularity on H\"older scale, meaning that the decomposition we form is not truly `half-regular' in three or more dimensions.


While it is unclear whether the conditions of \Cref{thm:seconddmain} can be weakened, we showed in \Cref{chap:poly} that some additional restrictions on \(f\) are necessary if an analog to \Cref{thm:c3main} is to hold for large values of \(k\). This is because \eqref{eq:sosos} fails for some functions in \(C^{k,\alpha}(\mathbb{R}^n)\) when \(k\geq 4\) and \(n\geq 2\), and indeed our counterexamples in this setting are simply polynomials. By means of a careful construction motivated by the work of Reznik in \cite{reznik_extpsd}, we were able to provide a technique for finding explicit examples of such polynomials in all settings in which they exist, few of which could be found in the literature. It is hoped that our examples contribute to a greater understanding of sums of squares of polynomials and H\"older continuous functions.


Finally, throughout the thesis we presented several supplementary results which enjoy various degrees of novelty. \Cref{lem:genchain}, viewed as an expression of the structure of high order derivatives, is very well-known -- our contribution is to explicitly compute the constants involved. This might be useful in streamlining the computation of such derivatives. Likewise, the proof of \Cref{thm:specialodds} uses a well-known technique to derive the determinant of a Vandermonde-like matrix, but our application to extracting weighted summation identities seems to be new. It is hoped that such results, whether they are new or rediscovered, might be useful in other applications.


\section{Potential Applications}\label{sec:hypoelliptchap}


The starting point for this research was the work of Sawyer \& Korobenko in \cite{SOS_I}, in which they studied decompositions of \(C^{4,\alpha}(\mathbb{R}^n)\) functions and matrices. Their motivation in this, realized in the third paper of the series \cite{SOS_III}, was to study hypoellipticity of certain differential operators in divergence form. An operator \(L\) is said to hypoelliptic if \(Lu\in C^\infty(\mathbb{R}^n)\) for a distribution \(u\) implies that \(u\in C^\infty(\mathbb{R}^n)\). Put another way, \(L\) is hypoelliptic if a solution \(u\) to the differential equation \(Lu=f\) must be smooth when the data \(f\) is smooth. This is a significantly weakened form of the frequently exploited property of ellipticity, and it is poorly understood. In \cite{Christ_IDR} Christ suggests that a characterization of hypoelliptic operators is unlikely to exist, and warrants this suspicion with several counter-intuitive examples. 


Nevertheless, for certain operators like those considered in \cite{SOS_III}, sufficient conditions for hypoellipticity have been obtained, and many of these rely in some way on a sum of squares decompositions or related concepts. H\"ormander's famous `bracket condition' for hypoellipticity deals with operators which can be decomposed as sums of squares of vector fields \(X_1,\dots,X_m\),
\[
    L=-\sum_{j=1}^m X_j^*X_j.   
\]
Here, \(X_j^*\) is the \(L^2(\mathbb{R}^n)\) adjoint of \(X_j\), and the right-hand side of the equation above becomes an algebraic sum of squares of dyads under a Fourier transform. 


H\"ormander proves that \(L\) as above is hypoelliptic if the iterated commutators of the \(X_j\)'s span \(\mathbb{R}^n\) at every point in \cite{hormander}. Likewise, Christ proves some sufficient and sometimes necessary conditions for hypoellipticity of \(L\) in \cite{Christ_IDR}, even in situations where H\"ormander's result fails. Building upon the latter work, Sawyer \& Korobenko use a \(C^{4,\alpha}(\mathbb{R}^n)\) sum of squares decomposition to generalize Christ's result to situations in which the vector fields enjoy limited regularity. Even the work of Kohn in \cite{Kohn}, which does not explicitly involve sums of squares, nevertheless employs a variant of the Malgrange inequality \eqref{eq:malg}, which is central to showing that non-negative \(C^{1,\alpha}(\mathbb{R}^n)\) functions have half-regular roots; see \cite[Eq.(2.18)]{Kohn}.


It is not clear whether there is any fundamental connection between hypoellipticity and sums of squares, or if sums of squares are merely a convenient technique with which to study hypoelliptic operators. In any case, equipped with the expanded decomposition results of this thesis, it may be possible to revisit and build upon several of the results mentioned above, or at least to show that they apply to broader classes of differential operators. In particular, our higher-order decomposition theorems might be suitable for studying the hypoellipticity of equations of higher order, whereas many of the results above restrict to second-order operators.


In passing, we also note that in \cite{Guan}, Guan uses the \(C^{3,1}(\mathbb{R}^n)\) decomposition from Fefferman and Phong \cite{Fefferman-Phong} to study behaviour of the Monge-Amp\`ere equation, and similar decomposition result to those obtained in our work may be useful in expanding upon Guan's work. In particular, Guan shows that if the data function \(f\) of the Monge-Amp\`ere equation \(D^2u=f\) is a sum of squares of \(C^{1,1}(\Omega)\) functions on a suitable domain \(\Omega\), then the solution \(u\) is also in \(C^{1,1}(\Omega)\). In particular, it follows from \cite{Fefferman-Phong} that it is enough to have \(f\in C^{3,1}(\Omega)\). By generalizing to higher values of \(k\) and working with decompositions into other integer powers (e.g. cubes and fourth powers), our techniques may be suitable for adapting Guan's work to the study of higher-order regularity of solutions to the Monge-Amp\`ere equations.





%Bibliographic note: I want the amsalpha style without the funky keys. Hence the custom bst file. Once all the references are in, label it with a key that corresponds to its position in the alphabetical ordering. Compile with empty keys first to determine the proper order before doing this. See the first Bibliography entry for an example of how this is done.


%\textbf{Later update} Rather than this janky and laborious fix, it could be better to do as scott did in the OrliczLimits paper, adding specific bibitems without using a .bib file to control tags. In any case, this should be left to the end. The current system is fine in the drafting stage.

\section{Some Open Questions}\label{sec:applics}


Briefly, we discuss some questions which are left open by this thesis, either due to constraints of time and length of this document, or due to difficulty of the problems themselves.


First, what is the smallest number \(m\) for which every sum of half-regular squares in \(C^{k,\alpha}(\mathbb{R}^n)\) is a sum of at most \(m\) squares? Borrowing (and perhaps adulterating) terminology from algebra, we might call this number the Pythagoras number of \(C^{k,\alpha}(\mathbb{R}^n)\), and we denote it here by \(p_{k,n}\). 

In \Cref{thm:malgrangeineq} we show that \(p_{1,n}=p_{2,n}=1\) for every \(n\), and Bony shows in \cite{Bony} that \(p_{k,1}=2\) for every \(k\geq 2\) (see \Cref{thm:onedim}). The work of \Cref{sec:sizebounds} also shows that
\[
    p_{2,n}\leq 2\cdot15^{n^2-n}+\frac{15^{n^2}-15^n}{15^n-1},
\]
but this bound is likely to be far from optimal. The same crude bound is obtained for \(p_{3,n}\), and it is reasonable to suspect from the similarity of the cases \(k=0,1\) and \(k=2,3\) that in general, \(p_{k,n}=p_{k+1,n}\) for even \(k\in\mathbb{N}\cup\{0\}\). Otherwise, we know little about \(p_{k,n}\) in general. 

In particular, no lower bound was found in the course of this research, though it is suspected that one might be found by using the Pythagoras number of polynomials owing to the counterexamples we obtained in \Cref{chap:poly}. These numbers have been studied in a more general setting by Choi, Lam \& Reznik in \cite{CLR}. It is unclear if these numbers are useful in any way, but it is apparent that finding them will require the development of some interesting mathematics.


A second open question is the following: does there exist a non-negative smooth function on \(\mathbb{R}^n\) which cannot be written as a sum of squares of smooth functions? As Sawyer \& Korobenko remark in \cite{SOS_I}, such an example is said to have been found by the late Paul Cohen, but a search of the literature afforded no such example and Pieroni affirms in \cite[Remark 5.1]{pieroni} that the problem remains open.

There are several directions which we can identify that may, with some work, yield an answer to this open question. First, by honing in on necessary conditions for \(C^{k,\alpha}(\mathbb{R}^n)\) functions to be decomposed into a sum of half-regular squares for arbitrarily large \(k\) (as we attempted in \Cref{thm:seconddmain}), it may be possible to construct a non-negative smooth function violating said conditions. Such a function would necessarily be non-decomposable as a sum of squares of smooth functions, affording the desired counterexample.

As a second approach, one could employ the non-decomposable polynomials found in \Cref{chap:poly} as candidates for such functions. The argument used in \Cref{sec:nd} to show that such polynomials are not sums of half-regular squares fails to show that these polynomials are not sums of squares of smooth functions, since there is no longer a highest-order term in the Taylor polynomial to consider. It may also be the case that a decomposition function is smooth and nowhere analytic, in which case the Taylor polynomial has little relation to the function. As such, an altogether different argument would be needed to approach the problem from this angle.

Of course, it may be the case that all non-negative smooth functions can in fact be decomposed as sums of squares of smooth functions. This seems feasible at least for functions of one dimension, thanks to the results of Bony in \cite{Bony}. However, owing to the counterexamples to our regularity-preserving decomposition results in higher dimensions, the behaviour of \(C^{\infty}(\mathbb{R}^n)\) functions for \(n\geq 2\) is less predictable. 


\section{Closing Remark}


The study of regularity-preserving decompositions undertaken in this thesis was motivated by their potential for applications to the study of hypoelliptic operators -- in that sense, this work can be seen as a means to an end. However, in undertaking that study, it was quickly discovered that the problem harbours considerable underlying beauty, and the technical difficulties  the decompositions presented often pointed the way to rich mathematical structures. Working on this problem therefore quickly became an end unto itself.


Most grade school students encounter sums of squares for the first time upon learning of the Pythagorean theorem, and thereafter it is easy to think of them as fairly unremarkable, given their commonality in many areas of mathematics. Yet by placing sums of squares within new contexts, the simple `certificate of non-negativity' can take on a new life involving a great deal of novel properties and fresh insights. Such was the case with sums of squares in H\"older spaces.


The author is grateful to have been directed to this problem by his supervisor, Dr. Eric T. Sawyer. It was an approachable and straightforward problem to understand, yet sufficiently challenging that it induced the author to strike out and learn new mathematics with the aspiration of solving it. It is hoped that the contributions of this work on sums of squares and regularity preserving decompositions are found to be meaningful by some mathematicians, and that this work can be built upon to advance our understanding of the subjects it touches.