\chapter{Sum of Squares Decompositions}\label{chap:decomps}


Given a non-negative function \(f\in C^{k,\alpha}(\mathbb{R}^n)\) for \(k\leq 3\) and \(0<\alpha\leq 1\), we now begin working to show that \(f\) can be decomposed as a sum of squares of functions in the H\"older space of functions with `half' the regularity of \(f\). By this, we mean that we can write \(f=g_1^2+\cdots+g_m^2\) for functions \(g_1,\dots,g_m\) which belong to the H\"older space
\begin{equation}\label{eq:halfreg}
    C^\frac{k+\alpha}{2}(\mathbb{R}^n)=
    \begin{cases}
    \hfil C^{\frac{k}{2},\frac{\alpha}{2}}(\mathbb{R}^n) & k\;\textrm{even,}\\
    C^{\frac{k-1}{2},\frac{1+\alpha}{2}}(\mathbb{R}^n) & k\;\textrm{odd}.
    \end{cases}    
\end{equation}
Earlier, we showed that if \(f\in C^{k,\alpha}(\mathbb{R}^n)\) for \(k=0\) or \(k=1\) and \(0<\alpha\leq 1\) then \(\sqrt{f}\in C^\frac{k+\alpha}{2}(\mathbb{R}^n)\). Thus, when \(k\leq 1\) the desired decomposition holds with just one square given by \(g_1=\sqrt{f}\). 


Moving on to \(k\geq 2\), the techniques required to prove the existence of a sum of squares decomposition are altogether different and considerably more elaborate. This is because taking a single root does not suffice to preserve regularity for \(k\geq 2\), as we showed with \eqref{eq:badroot}. Our main result in this thesis, which we prove using these more advanced techniques, is the following.


\begin{thm}\label{thm:c3main}
If \(0<\alpha\leq1\) and \(f\in C^{2,\alpha}(\mathbb{R}^n)\) is non-negative, then there exist functions \(g_1,\dots, g_{m_n}\in C^{1,\frac{\alpha}{2}}(\mathbb{R}^n)\) which satisfy
\begin{equation}\label{eq:decompident}
    f=\sum_{j=1}^{m_n}g_j^2.
\end{equation}
Similarly, if \(f\in C^{3,\alpha}(\mathbb{R}^n)\) then \eqref{eq:decompident} holds for functions \(g_1,\dots, g_m\in C^{1,\frac{1+\alpha}{2}}(\mathbb{R}^n)\). Moreover, the number of squares \(m_n\) depends only on the dimension \(n\). 
\end{thm}


Our proof of this result actually goes further, to show that the functions \(g_1,\dots,g_m\) and their derivatives are controlled pointwise in the same way that the derivatives of \(f\) are by \Cref{thm:cauchylike}. Specifically, we show that when \(f\in C^{2,\alpha}(\mathbb{R}^n)\), each \(g_j\) in \eqref{eq:decompident} satisfies the pointwise estimates
\[
    |g_j(x)|\leq C\max\bigg\{\sqrt{f(x)},\sup_{|\xi|=1}[\partial^2_\xi f(x)]_+^\frac{2+\alpha}{2\alpha}\bigg\}\textrm{ and }|\nabla g_j(x)|\leq C\max\bigg\{f(x)^\frac{\alpha}{4+2\alpha},\sup_{|\xi|=1}[\partial^2_\xi f(x)]_+^\frac{1}{2}\bigg\}.
\]
Similar estimates hold when \(f\in C^{3,\alpha}(\mathbb{R}^n)\), and these bounds show that the functions \(g_1,\dots,g_m\) inherit some of the pointwise structure of \(f\). In \Cref{sec:ext} we extend \Cref{thm:c3main} to the local spaces \(C^{2,\alpha}_\mathrm{loc}(\Omega)\) and \(C^{3,\alpha}_\mathrm{loc}(\Omega)\) for any open set \(\Omega\subseteq\mathbb{R}^n\), see \Cref{thm:locdecomp}.


The classical result of Fefferman \& Phong appearing in \cite{Fefferman-Phong}, which we record here for completeness, follows as a special case of \Cref{thm:c3main}.


\begin{cor}[Fefferman \& Phong]\label{cor:FP}
If \(f\in C^{3,1}(\mathbb{R}^n)\) is non-negative, then it can be written as a finite sum of squares of \(C^{1,1}(\mathbb{R}^n)\) functions.
\end{cor}


There is a self-contained proof of this result in \cite{Guan} which Guan attributes to Fefferman. The key idea underlying the argument communicated by Guan is that the domain of \(f\) can be partitioned into regions on which \(f\) is well-behaved, and others where \(f\) does not have a half-regular root. The same approach is used to prove the generalizations of \Cref{cor:FP} in \cite[Proposition 1.1]{Tataru} and \cite[Theorem 4.5]{SOS_I}.  In each partition region, separate techniques are employed to recover an appropriate local decomposition of \(f\). Then, using an inductive argument, these local estimates are combined to obtain a global sum of squares decomposition.


Following suit, we begin by using the properties of non-negative \(C^{k,\alpha}(\mathbb{R}^n)\) functions established in \Cref{chap:holderchap} to partition the domain of \(f\) and argue locally. Our main instrument to this end is a partition of unity that we construct in \Cref{sec:pou}, which takes the form
\begin{equation}\label{eq:pou}
    \sum_{j=1}^\infty\psi_j^2=\begin{cases}
    1 & \textrm{if }f\textrm{ or one of its derivatives is nonzero},\\
    0 & \textrm{where }f\textrm{ and all of its derivatives  vanish}.
    \end{cases}
\end{equation}
The functions \(\psi_j\) in \eqref{eq:pou} are smooth, and they have special properties relating to \(f\) which we summarize in \Cref{thm:party}. Combining this partition with the localized results that we establish in the following section, we go on to prove \Cref{thm:c3main} in \Cref{sec:c3proof}.


\section{Local Regularity of Roots}\label{sec:reg}


Now we show that the localization of a non-negative function \(f\in C^{k,\alpha}(\mathbb{R}^n)\) to a sufficiently small ball either has a square root that belongs to the half-regular H\"older space \eqref{eq:halfreg}, or it can be decomposed into a sum of two functions; one which has a half-regular root, and another which depends on fewer variables than \(f\). These local decompositions are performed in the supports of the partition functions \(\psi_j\) written in \eqref{eq:pou}, and they always exist when \(k=2\) and \(k=3\). If \(k\geq4\) then \(f\) can be decomposed locally in the same way, but doing so requires additional hypotheses which are discussed in \Cref{chap:higher} and in \cite{SOS_I}.


The results of this section are established in full generality, and we do not restrict to the cases \(k\leq 3\), since we also use these results to study \(C^{k,\alpha}(\mathbb{R}^n)\) for \(k\geq 4\) later in the work. This generality sometimes comes at the expense of cumbersome notation, but our efforts are rewarded by a much clearer picture of the behaviour of \(C^{k,\alpha}(\mathbb{R}^n)\) functions. Unless stated  otherwise, throughout this section it is understood that \(k\) is any non-negative integer and \(0<\alpha\leq 1\).


Our primary instrument in this chapter is a function \(r\) that controls the derivatives of a function \(f\) pointwise. It was originally identified by Fefferman \& Phong in \cite{Fefferman-Phong} in the case \(k=3\) and \(\alpha=1\), and modified by Sawyer \& Korobenko in \cite{SOS_I} to work for \(k=4\) and \(0<\alpha\leq1\). For our purposes, we require a generalized form of this control function which is suitable for any pairing of \(k\) and \(\alpha\). This object lets us identify balls on which \(f\) is well-behaved in a manner that permits decomposition. Fixing a non-negative function \(f\in C^{k,\alpha}(\mathbb{R}^n)\), we define
\begin{equation}\label{eq:controlfunc}
    r(x)=\max_{\substack{0\leq j\leq k,\\j\;\mathrm{even}}}\bigg\{\sup_{|\xi|=1}[\partial^j_\xi f(x)]_+^\frac{1}{k-j+\alpha}\bigg\}.
\end{equation}
From this definition and \Cref{thm:cauchylike}, we see that \(r\) controls \(f\) pointwise in the following way.
\begin{cor}\label{cor:controlbound} 
For \(r\)  as in \eqref{eq:controlfunc} and for \(\ell\leq k\), the following pointwise derivative bounds hold:
\[
    |\nabla^\ell f(x)|\leq Cr(x)^{k-\ell+\alpha}.
\]
\end{cor}
Following \cite{Tataru} and \cite{SOS_I}, we show that \(r\) is essentially constant on sufficiently small balls. This property, which we call slow variation, is critical to our ensuing arguments.


\begin{lem}\label{lem:slowvar}
There exists a constant \(\nu>0\) such that \(|r(x)-r(y)|\leq \frac{1}{4}r(x)\) when \(|x-y|\leq \nu r(x)\).
\end{lem}

\textit{Remark}: Throughout this chapter we will continue to put size restrictions on the parameter \(\nu\). Our conditions ensure that in the end, \(\nu\) is a very small positive constant. The actual value of \(\nu\) turns out to be inconsequential for our final construction, thanks to \Cref{thm:party}, so we encounter no issue in occasionally asking that \(\nu\) be smaller than previously assumed.


\begin{proof}
Fix a non-negative function \(f\in C^{k,\alpha}(\mathbb{R}^n)\) and let \(r\) be as in \eqref{eq:controlfunc}. Given any two points \(x,y\in\mathbb{R}^n\) for which \(|x-y|\leq \nu r(x)\), we employ Lemmas \ref{lem:maxproplem} and \ref{lem:supprop} to first make the estimate
\begin{equation}\label{eq:intermed4}
    |r(x)-r(y)|\leq \max_{\substack{0\leq j\leq k,\\j\;\mathrm{even}}}\big\{\sup_{|\xi|=1}\big|[\partial^j_\xi f(x)]_+^\frac{1}{k-j+\alpha}-[\partial^j_\xi f(y)]_+^\frac{1}{k-j+\alpha}\big|\big\}.
\end{equation}
To proceed, first assume that \(j<k\), so that \(k-j+\alpha> 1\). Then \Cref{lem:smallpow} applies, and we get
\[
    \big|[\partial^j_\xi f(x)]_+^\frac{1}{k-j+\alpha}-[\partial^j_\xi f(y)]_+^\frac{1}{k-j+\alpha}\big|\leq |\partial^j_\xi f(x)-\partial^j_\xi f(y)|^\frac{1}{k-j+\alpha}.
\]
To estimate the right-hand side above, we expand the directional operator \(\partial_\xi^j\) from \eqref{eq:directionalop} to write
\[
    |\partial^j_\xi f(x)-\partial^j_\xi f(y)|=\bigg|\sum_{|\beta|=j}\frac{\beta!}{j!}\xi^\beta(\partial^\beta f(x)-\partial^\beta f(y))\bigg|\leq \sum_{|\beta|=j}\frac{\beta!}{j!}|\partial^\beta f(x)-\partial^\beta f(y)|.
\]
Then, \Cref{lem:Taylorest} together with the bounds \(|\nabla^\ell f(x)|\leq Cr(x)^{k-\ell+\alpha}\) and \(|x-y|\leq \nu r(x)\) gives us
\[
    |\partial^\beta f(x)-\partial^\beta f(y)|\leq C\sum_{0\leq |\gamma|\leq k-|\beta|}\nu^{|\gamma|}r(x)^{k-|\beta|+\alpha}+C\nu^{k-|\beta|+\alpha}r(x)^{k-|\beta|+\alpha}.
\]
Every power of \(\nu\) above is larger than one, since \(|\beta|=j<k\). Thus, assuming that \(\nu\leq 1\), we find there exists a constant \(C\) which depends only on \(k\), \(\alpha\) and \(|\beta|\) for which
\begin{equation}\label{eq:omegaeqn}
    |\partial^\beta f(x)-\partial^\beta f(y)|\leq C\nu r(x)^{k-|\beta|+\alpha}.
\end{equation}
Therefore \(|\partial^j_\xi f(x)-\partial^j_\xi f(y)|\leq C\nu r(x)^{k-|\beta|+\alpha}\), and since this bound is independent of the direction \(\xi\) we can choose a small number \(\nu\) which is independent of \(x\) such that the following holds,
\[
    \sup_{|\xi|=1}\big|[\partial^j_\xi f(x)]_+^\frac{1}{k-j+\alpha}-[\partial^j_\xi f(y)]_+^\frac{1}{k-j+\alpha}\big|\leq  C\nu^\frac{1}{k-j+\alpha}r(x)\leq \frac{1}{4}r(x).
\]


It remains to consider the case \(j=k\), which arises when \(k\) is even. This time, we can employ \Cref{lem:almostLipschitz}, taking \(\beta=\frac{1}{\alpha}\) and \(\varepsilon=\nu^\alpha\). Combining with the inequality \(|x-y|\leq \nu r(x)\), we get
\[
    \sup_{|\xi|=1}\big|[\partial^k_\xi f(x)]_+^\frac{1}{\alpha}-[\partial^k_\xi f(y)]_+^\frac{1}{\alpha}\big|\leq\frac{C\nu (1+\nu^\alpha)r(x)}{((1+\nu^\alpha)^{\frac{\alpha}{1-\alpha}}-1)^\frac{1-\alpha}{\alpha}}+\nu^\alpha\max\{r(x),r(y)\}.
\]
If \(\nu\leq 1\) the quotient above is bounded by \(C\nu^\alpha r(x)\), for a constant \(C\) that depends only on \(\alpha\) and the H\"older semi-norms of \(f\). Since \(\max\{r(x),r(y)\}\leq r(x)+|r(x)-r(y)|\) by non-negativity of \(r\), when \(\nu\) is small enough we  have
\[
    \sup_{|\xi|=1}\big|[\partial^k_\xi f(x)]_+^\frac{1}{\alpha}-[\partial^k_\xi f(y)]_+^\frac{1}{\alpha}\big|\leq C\nu^\alpha r(x)+\nu^\alpha|r(x)-r(y)|\leq \frac{1}{8}r(x)+\frac{1}{2}|r(x)-r(y)|.
\]
It follows from these bounds and \eqref{eq:intermed4} that \(|r(x)-r(y)|\leq \frac{1}{4}r(x)\), as we wished to show.
\end{proof}


\textit{Remark}: The choice of the constant \(\frac{1}{4}\) in the preceding lemma is largely arbitrary; any constant between \(0\) and \(1\) will suffice, and we only choose \(\frac{1}{4}\) to simplify some later calculations and avoid introducing a new parameter unnecessarily.


Our main analysis takes place in the balls \(B(x,\nu r(x))\) for \(\nu\) as above, and the preceding lemma shows that \(r\) is essentially constant in these sets. Explicitly, for every \(y\in B(x,\nu r(x))\) the preceding lemma shows that \(\frac{3}{4}r(x)\leq r(y)\leq \frac{5}{4}r(x)\). The following result goes further, ensuring that \(f\) and its second derivatives also vary slowly on suitably small balls.


\begin{cor}\label{cor:supplementary}
Let \(f\in C^{k,\alpha}(\mathbb{R}^n)\) be non-negative for \(k\geq 2\), and let \(r\) be as in \eqref{eq:controlfunc}. If \(\nu\) is a sufficiently small positive constant, then there exists another constant \(\omega\) for which
\begin{itemize}
    \item[(1)] \(|f(x)-f(y)|\leq \frac{1}{2}\omega \nu r(x)^{k+\alpha}\) whenever \(|x-y|\leq \nu r(x)\), and
    \item[(2)] \(|\partial^\beta f(x)-\partial^\beta f(y)|\leq \frac{1}{2}r(x)^{k-2+\alpha}\) whenever \(|x-y|\leq \sqrt{\nu^2+3\nu\omega}r(x)\) and \(|\beta|=2\).
\end{itemize}
\end{cor}


\begin{proof}
Estimate \textit{(1)} follows by taking \(\omega=2C\) for \(C\) as in \eqref{eq:omegaeqn} when \(|\beta|=0\), while \textit{(2)} is given by replacing \(\nu\) with \(\sqrt{\nu^2+3\nu\omega}\) in \eqref{eq:omegaeqn} when \(|\beta|=2\), and once again choosing \(\nu\) small enough.
\end{proof}


For the remainder of this section, we fix \(x\in\mathbb{R}^n\) and the corresponding ball \(B=B(x,\nu r(x))\). We study the behaviour of \(f\) on \(B\) in two exhaustive cases: when \(f(x)\geq \omega\nu r(x)^{k+\alpha}\), and when \(f(x)<\omega\nu r(x)^{k+\alpha}\). In the first case, \(f\) has a half-regular square root on \(B\), as we now show.


\begin{lem}\label{lem:local1}
Let \(f\), \(r\), and \(\nu\) be as above and assume that \(f(x)\geq \omega\nu r(x)^{k+\alpha}\). Then for any multi-index \(\beta\) of order \(|\beta|\leq \frac{k+\alpha}{2}\), and for any \(y\in B\), the following pointwise estimates hold:
\begin{itemize}
    \item[(1)] \(|\partial^\beta\sqrt{f(y)}|\leq Cr(y)^{\frac{k+\alpha}{2}-|\beta|}\),
    \item[(2)] \([\partial^\beta\sqrt{f}]_{\frac{\alpha}{2}}(y)\leq Cr(y)^{\frac{k}{2}-|\beta|}\) if \(k\) is even,
    \item[(3)] \([\partial^\beta\sqrt{f}]_{\frac{1+\alpha}{2}}(y)\leq Cr(y)^{\frac{k-1}{2}-|\beta|}\) if \(k\) is odd.
\end{itemize}
Recall that \([f]_\alpha(y)\) denotes the pointwise variant of the \(\alpha\)-H\"older semi-norm, defined in \eqref{eq:pwop}. 
\end{lem}


\begin{proof}
Using \Cref{cor:rootscor} we bound derivatives of \(f\) at \(y\in B\), estimating pointwise to first get
\[
    |\partial^\beta\sqrt{f(y)}|=\bigg|\sum_{\Gamma\in P(\beta)}C_{\beta,\Gamma}f(y)^{\frac{1}{2}-|\Gamma|}\prod_{\gamma\in\Gamma}\partial^\gamma f(y)\bigg|\leq C\sum_{\Gamma\in P(\beta)}f(y)^{\frac{1}{2}-|\Gamma|}\prod_{\gamma\in\Gamma}|\partial^\gamma f(y)|.
\]
Observe that \(f(y)\geq \frac{1}{2}\omega\nu r(x)^{k+\alpha}\) for any \(y\in B\) by estimate \textit{(1)} of \Cref{cor:supplementary}. Additionally, \Cref{cor:controlbound} and slow variation of \(r\) on \(B\) together show that
\[
    |\partial^\beta\sqrt{f(y)}|\leq C\sum_{\Gamma\in P(\beta)}r(y)^{(k+\alpha)(\frac{1}{2}-|\Gamma|)}\prod_{\gamma\in\Gamma}r(y)^{k+\alpha-|\gamma|}= Cr(y)^{\frac{k+\alpha}{2}-|\beta|}.
\]
Since this holds pointwise on \(B\) we conclude that item \textit{(1)} holds.


For the semi-norm estimates \textit{(2)} and \textit{(3)}, we treat the odd and even cases simultaneously by letting \(\Lambda\) denote the integer part of \(\frac{k}{2}\) and setting \(\lambda=\frac{k+\alpha}{2}-\Lambda\). To prove the claimed results it suffices to show that \([\partial^\beta\sqrt{f}]_{\lambda}(y)\leq Cr(y)^{\Lambda-|\beta|}\) whenever \(|\beta|\leq \Lambda\) and \(y\in B\). Let \(\beta\) be given and observe that by \Cref{cor:rootscor} together with the triangle inequality, we have for \(w,z\in B\) that
\begin{equation}\label{eq:ctrl}
\begin{split}
    |\partial^\beta\sqrt{f(w)}-\partial^\beta\sqrt{f(z)}|&\leq C\sum_{\Gamma\in P(\beta)} f(w)^{\frac{1}{2}-|\Gamma|}\bigg|\prod_{\gamma\in\Gamma}\partial^\gamma f(w)-\prod_{\gamma\in\Gamma}\partial^\gamma f(z)\bigg|\\
    &\qquad+C\sum_{\Gamma\in P(\beta)} |f(w)^{\frac{1}{2}-|\Gamma|}-f(z)^{\frac{1}{2}-|\Gamma|}|\prod_{\gamma\in\Gamma}|\partial^\gamma f(z)|.    
\end{split}
\end{equation}
We treat the terms on the right-hand side above separately, first using the Mean Value Theorem and slow variation of \(r\) on \(B\) to obtain the bound
\[
    |f(w)^{\frac{1}{2}-|\Gamma|}-f(z)^{\frac{1}{2}-|\Gamma|}|\leq Cr(y)^{\frac{k+\alpha}{2}-1-|\Gamma|(k+\alpha)}|w-z|\leq Cr(y)^{\frac{k+\alpha}{2}-|\Gamma|(k+\alpha)-\lambda}|w-z|^\lambda.
\]
Explicitly, we have used here that \(f\) is bounded below by a multiple of \(r(y)^{k+\alpha}\) on \(B\), followed by the fact that \(|w-z|\leq Cr(y)^{1-\lambda}|w-z|^\lambda\) for \(w,z\in B\). Bounding the other derivatives of \(f\) on \(B\) in the same fashion, we see that
\[
    |f(w)^{\frac{1}{2}-|\Gamma|}-f(z)^{\frac{1}{2}-|\Gamma|}|\prod_{\gamma\in\Gamma}|\partial^\gamma f(z)|\leq Cr(y)^{\frac{k+\alpha}{2}-|\Gamma|(k+\alpha)-\lambda}|w-z|^\lambda\prod_{\gamma\in\Gamma}r(y)^{k+\alpha-|\gamma|}.
\]
Since \(\Gamma\in P(\beta)\), we can simplify the right-hand side above to \(Cr(y)^{\Lambda-|\beta|}|w-z|^\lambda\). To estimate the remaining term of \eqref{eq:ctrl}, we use \Cref{lem:proddiff} to produce the bound
\begin{equation}\label{eq:sub}
    \bigg|\prod_{\gamma\in\Gamma}\partial^\gamma f(w)-\prod_{\gamma\in\Gamma}\partial^\gamma f(z)\bigg|\leq \sum_{\gamma\in\Gamma}\bigg(\prod_{\mu\in\Gamma\setminus\{\gamma\}}\sup_{B}|\partial^\mu f|\bigg)|\partial^\gamma f(w)-\partial^\gamma f(z)|.    
\end{equation}


Arguing as above, we can control the product in \eqref{eq:sub} by \(Cr(y)^{(k+\alpha)(|\Gamma|-1)-|\beta|+|\gamma|}\). Further, since \(|\beta|\leq\Lambda< \frac{k+\alpha}{2}\), whenever \(\gamma\in P(\beta)\) we must have \(|\gamma|<k\), meaning that we can use the Mean Value Theorem and \Cref{cor:controlbound} to get for \(w,z\in B\) that
\[
    |\partial^\gamma f(w)-\partial^\gamma f(z)|\leq Cr(y)^{k+\alpha-|\gamma|-\lambda}|w-z|^\lambda.
\]
Therefore the left-hand side of \eqref{eq:sub} is bounded by \(Cr(y)^{(k+\alpha)|\Gamma|-|\beta|-\lambda}|w-z|^\lambda\). Consequently,
\[
    f(w)^{\frac{1}{2}-|\Gamma|}\bigg|\prod_{\gamma\in\Gamma}\partial^\gamma f(w)-\prod_{\gamma\in\Gamma}\partial^\gamma f(z)\bigg|\leq Cr(y)^{\frac{k+\alpha}{2}-|\beta|-\lambda}|y-z|^\lambda=Cr(y)^{\Lambda-|\beta|}|y-z|^\lambda.
\]
Now from \eqref{eq:ctrl} and definition \eqref{eq:pwop} of the pointwise semi-norm, we observe for any \(y\in B\) that
\[
    [\partial^\beta\sqrt{f}]_\lambda(y)=\limsup_{w,z\rightarrow y}\frac{|\partial^\beta\sqrt{f(w)}-\partial^\beta\sqrt{f(z)}|}{|w-z|^\lambda}\leq Cr(y)^{\Lambda-|\beta|},
\]
where critically, the constant \(C\) is independent of \(B\). Hence items \textit{(2)} and \textit{(3)} hold as claimed.
\end{proof}


In the proof above we constructed pointwise \(\alpha\)-H\"older semi-norm estimates, rather than global semi-norm estimates, since the former are needed to prove \Cref{thm:c3main}. These pointwise estimates actually imply local variants, since \textit{(2)} and \textit{(3)} together with slow variation of \(r\) on the ball \(B=B(x,\nu r(x))\) give \([\partial^\beta\sqrt{f}]_{\frac{\alpha}{2}}(y)\leq Cr(x)^{\frac{k}{2}-|\beta|}\) and \([\partial^\beta\sqrt{f}]_{\frac{1+\alpha}{2}}(y)\leq Cr(x)^{\frac{k-1}{2}-|\beta|}\), respectively when \(k\) is even and odd. Taking a supremum over \(y\in B\) yields the local bounds
\[
    [\partial^\beta\sqrt{f}]_{\frac{\alpha}{2},B}\leq Cr(x)^{\frac{k}{2}-|\beta|}\qquad\mathrm{and}\qquad[\partial^\beta\sqrt{f}]_{\frac{1+\alpha}{2},B}\leq Cr(x)^{\frac{k-1}{2}-|\beta|},
\]
again when \(k\) is even and odd respectively. Since \(x\) and \(B\) are fixed, we can conclude the following.


\begin{cor}
Let \(f\in C^{k,\alpha}(\mathbb{R}^n)\) be non-negative and let \(r\) be as in \eqref{eq:controlfunc}. If \(\nu\) is a sufficiently small positive constant and \(f(x)\geq \omega\nu r(x)^{k+\alpha}\), then \(\sqrt{f}\in C^\frac{k+\alpha}{2}(B(x,\nu r(x)))\).
\end{cor}


On the other hand, if \(f(x)<\omega\nu r(x)^{k+\alpha}\) then we can no longer conclude that the localization of \(f\) to \(B\) has a half-regular square root. Rather, the best we can do is form a local decomposition of \(f\) on \(B\) that takes the form
\begin{equation}\label{eq:loooooocal}
    f=g^2+F,
\end{equation}
where \(g\) is half as regular as \(f\) and the remainder \(F\) can be handled using other techniques. Specifically, \(F\) can be made to depend on only \(n-1\) variables and it inherits the regularity of \(f\).


Constructing the local decomposition described above is a laborious task, which comprises the remainder of this section. Lemmas \ref{lem:minia}, \ref{lem:minia2}, \ref{lem:nice}, and \ref{lem:implicitest} are devoted to constructing \(F\) and verifying that it has the desired properties -- our sequence of results is patterned after the decomposition arguments in \cite{Tataru} and \cite{SOS_I}, but we go to greater lengths to obtain estimates for every \(k\geq 0\). As such, the pages which follow are the most technically difficult of this thesis.


In what follows, we partition a variable \(x\in\mathbb{R}^n\) by writing \(x=(x',x_n)\) for \(x'\in\mathbb{R}^{n-1}\) and \(x_n\in\mathbb{R}\). Given a ball \(B\) we also write \(B'=\{x':x\in B\}\) to denote the projection of \(B\) onto \(\mathbb{R}^{n-1}\). Equipped with notation we have the following result.


\begin{lem}\label{lem:minia}
Assume that a non-negative function \(f\in C^{k,\alpha}(\mathbb{R}^n)\) satisfies \(f(x)<\omega \nu r(x)^{k+\alpha}\), and let \(r\) satisfy the following pointwise bound on \(B=B(x,\nu r(x))\),
\vspace{-0.25em}
\begin{equation}\label{eq:needed}
    r(y)\leq \max\bigg\{f(y)^\frac{1}{k+\alpha},\sup_{|\xi|=1}[\partial^2_\xi f(y)]_+^\frac{1}{k-2+\alpha}\bigg\}.
\end{equation}
Then after a suitable coordinate rotation centred at \(x\), there exists a function \(X\in C^{k-1}(B')\) which enjoys the following properties for \(y\in B\),
\begin{itemize}
    \item[(1)] \(|x_n-X(y')|\leq r(x)\),
    \item[(2)] \(f(y',X(y'))\leq f(y)\),
    \item[(3)] \(\partial_{x_n}f(y',X(y'))=0\).
\end{itemize}
\end{lem}

\medskip

\textit{Remark}: Put simply, this result states that \(f\) has a unique minimum along each vertical ray in a cylinder of height \(2r(x)\) that contains \(B\). Moreover, the collection of these minima, viewed as a function on the base \(B'\), actually defines a function belonging to \(C^{k-1}(B')\). The required pointwise bound \eqref{eq:needed} holds automatically when \(k=2\) and \(k=3\), while if \(k\geq 4\) then additional conditions must be placed on \(f\) for \eqref{eq:needed} to hold; see \Cref{chap:higher}.


\begin{proof}
Let \(y'\) satisfy \(|y'-x'|\leq \nu r(x)\) so that \(y'\in B'\) and set \(h=\sqrt{3\omega\nu}\), where \(\nu\) is chosen small enough that \(\omega\nu\leq \frac{1}{3}\). Our aim is to show that the mapping \(t\mapsto f(y',t)\) has a unique minimum in the open interval \((x_n-h,x_n+h)\), so that items \textit{(1)-(3)} above follow as direct consequences. Observe that if \(f(x)<\omega\nu r(x)^{k+\alpha}< r(x)^{k+\alpha}\) then \(r(x)\) is strictly positive, and by \eqref{eq:needed} we have
\[
    r(x)=\sup_{|\xi|=1}[\partial^2_\xi f(x)]_+^\frac{1}{k-2+\alpha}.
\]
After a rotation of coordinates centred at \(x\), we can assume without loss of generality that the supremum above is attained by the \(x_n\) directional derivative of \(f\), giving \(r(x)^{k-2+\alpha}=\partial^2_{x_n}f(x)\). We note that non-negativity of \(f\) as well as its inclusion in \(C^{k,\alpha}(\mathbb{R}^n)\) are preserved by such coordinate changes. 


For each \(t\in [x_n-h,x_n+h]\) we have \(\partial^2_{x_n}f(y',t)\geq\frac{1}{2}r(x)^{k-2+\alpha}\) by item \textit{(2)} of \Cref{cor:supplementary}, meaning that \(t\mapsto f(x',t)\) has a unique minimum on the compact interval \([x_n-h ,x_n+h]\).  Assume toward a contradiction that the minimum on this interval occurs at the left endpoint \(x_n-h\), so that \(\partial_{x_n}f(x',x_n-h)\geq 0\). Additionally, note that \(f(y',x_n)<\frac{3}{2}\omega\nu r(x)^{k+\alpha}\) by item \textit{(1)} of \Cref{cor:supplementary}. So for some \(\xi\in (x_n-h,x_n)\) we have from our choice of \(h\) that
\[
     \frac{3}{2}\omega\nu r(x)^{k+\alpha}>f(y',x_n)=f(y',x_n-h)+h\partial_{x_n}f(x',x_n-h)+\frac{1}{2}h^2\partial^2_{x_n}f(y',\xi)\geq \frac{3}{2}\omega\nu r(x)^{k+\alpha}.
\]
This is a contradiction since \(r(x)>0\), and an identical contradiction arises if we assume that the minimum occurs at \(x_n+h\). It follows that \(t\mapsto f(y',t)\) has a unique minimum in \((x_n-h,x_n+h)\) which we call \(X(y')\). Properties \textit{(1)-(3)} follow from our construction, and as \(y'\) was any point satisfying \(|y'-x'|\leq \nu r(x)\), we see that \(X\) is well-defined as a function on all of \( B'\).


To see that \(X\in C^{k-1}(B')\), we observe that for each \(y'\) we have \(\partial_{x_n}f(y',X(y'))=0\) and that \(\partial^2_{x_n}f(y',X(y'))>0\) by item \textit{(2)} of \Cref{cor:supplementary}. Applying the Implicit Function Theorem (\Cref{thm:IFT}) to \(\partial_{x_n}f\) at the point \((y',X(y'))\), we see that \(X\) is \(C^{k-1}\) in some neighbourhood of \(y'\) since \(\partial_{x_n}f\in C^{k-1}(\mathbb{R}^n)\). Further, as \(y'\) was any point in \(B'\) we can find such a neighbourhood around every point in the domain of \(X\). By the uniqueness statement of \Cref{thm:IFT}, these \(C^{k-1}\) functions must agree wherever their neighbourhoods overlap, meaning that \(X\in C^{k-1}(B')\).
\end{proof}


\begin{lem}\label{lem:minia2}
The function \(X\) defined above belongs to \(C^{k-1,\alpha}(B')\). In particular, it satisfies the following pointwise estimates for every \(y\in B\) and for all derivatives of order \(|\beta|\leq k-1\),
\begin{itemize}
    \item[(1)] \(|\partial^\beta X(y')|\leq Cr(y)^{1-|\beta|}\),
    \item[(2)] \([\partial^\beta X]_{\alpha}(y')\leq Cr(y)^{1-\alpha-|\beta|}\).
\end{itemize}
\end{lem}

\textit{Remark}: Our proof of this result is challenging, as it employs the multivariate calculus identities established in the previous chapter to their full extent. The reader may wish to pass over this argument on their first read, and return later when the significance of \Cref{lem:minia2} in the proof of \Cref{thm:c3main} is understood.

\begin{proof}
To establish derivative estimates of \(X\) for item \textit{(1)}, we first use \Cref{thm:IFT} to write
\[
    |\partial^\beta X(y')|=\bigg|\frac{1}{\partial^2_{x_n}f(y',X(y'))}\sum_{0\leq\eta\leq\beta}\sum_{ \substack{\Gamma\in P(\eta),\\\Gamma\neq \{\beta\}}}C_{\beta,\Gamma}(\partial^{\beta-\eta}\partial^{|\Gamma|+1}_{x_n}f(y',X(y')))\prod_{\gamma\in\Gamma}\partial^\gamma X(y')\bigg|.
\]
We simplify this by observing that \(\partial^2_{x_n}f(y',X(y'))\geq Cr(y)^{k-2+\alpha}\) on \(B'\) by item \textit{(2)} of \Cref{cor:supplementary} and slow variation of \(r\). Using this fact with the estimates of \Cref{cor:controlbound}, we get
\[
    |\partial^\beta X(y')|\leq Cr(y)^{2-k-\alpha}\sum_{0\leq\eta\leq\beta}\sum_{ \substack{\Gamma\in P(\eta),\\\Gamma\neq \{\beta\}}}r(y',X(y'))^{k+\alpha+|\eta|-|\beta|-|\Gamma|-1}\prod_{\gamma\in\Gamma}|\partial^\gamma X(y')|.
\]
Taking \(\nu\) small enough that \Cref{lem:slowvar} holds when \(\nu\) is replaced by \(\sqrt{\nu^2+3\omega\nu}\), it follows from the definition of \(X\) that \(r(y',X(y'))\leq Cr(y)\) whenever \(y'\in B'\).


Now we argue by strong induction that \(|\partial^\beta X(y')|\leq Cr(y)^{1-|\beta|}\), observing for a base case that if \(|\beta|=1\) then for some \(j=1,\dots,n-1\), and \(y'\) in the domain of \(X\) we can write 
\[
    |\partial^\beta X(y')|=|\partial_{x_j}X(y')|=\bigg|\frac{\partial_{x_j}\partial_{x_n}f(y',X(y'))}{\partial^2_{x_n}f(y',X(y'))}\bigg|\leq \frac{Cr(y)^{k+\alpha-2}}{r(y)^{k+\alpha-2}}=C.
\]
For an inductive hypothesis we assume that \(|\partial^\gamma X(y')|\leq Cr(y)^{1-|\gamma|}\) whenever \(|\gamma|<|\beta|\), so that
\[
    |\partial^\beta X(y')|\leq C\sum_{0\leq\eta\leq\beta}\sum_{ \substack{\Gamma\in P(\eta),\\\Gamma\neq \{\beta\}}}r(y)^{1-|\beta|+|\eta|-|\Gamma|}\prod_{\gamma\in\Gamma}r(y)^{1-|\gamma|}=Cr(y)^{1-|\beta|}.
\]
It follows that the claimed estimate holds for every \(\beta\) with \(|\beta|\leq k-1\), giving item \textit{(1)}.

It remains to demonstrate that estimate \textit{(2)} holds, and once again we prove this using strong induction. For a base case we observe that \(|X(w')-X(z')|\leq C|w'-z'|\) by the Mean Value Theorem and the derivative bound on \(X\) shown above. Further, if \(w',z'\in B'\) then we have that \(|w'-z'|\leq Cr(y)^{1-\alpha}|w'-z'|^\alpha\). Therefore 
\[
    |X(w')-X(z')|\leq Cr(y)^{1-\alpha}|w'-z'|^\alpha,
\]
from which it follows by taking a limit supremum as \(w',z'\rightarrow y'\) that \([X]_{\alpha}(y')\leq Cr(y)^{1-\alpha}\) for \(y\in B\). Thus the required semi-norm estimate holds in the base case.


For the inductive step we begin by using \Cref{thm:IFT} to expand the pointwise difference \(\partial^\beta X(w')-\partial^\beta X(z')\) as follows,
\[
    \sum_{\eta\leq\beta}\sum_{ \substack{\Gamma\in P(\eta),\\\Gamma\neq \{\beta\}}}C_{\beta,\Gamma}\bigg(\frac{\partial^{\beta-\eta}\partial^{|\Gamma|+1}_{x_n}f(w',X(w'))}{\partial^2_{x_n}f(w',X(w'))}\prod_{\gamma\in\Gamma}\partial^\gamma X(w')-\frac{\partial^{\beta-\eta}\partial^{|\Gamma|+1}_{x_n}f(z',X(z'))}{\partial^2_{x_n}f(z',X(z'))}\prod_{\gamma\in\Gamma}\partial^\gamma X(z')\bigg).
\]
For brevity, we will henceforth write \(\Tilde{w}=(w',X(w'))\) and note that \(|\tilde{w}-\Tilde{z}|\leq C|w'-z'|\). Using the triangle inequality we can bound each of the terms in the sum above by the quantity
\[
    \frac{|\partial^{\beta-\eta}\partial^{|\Gamma|+1}_{x_n}f(\tilde{w})|}{\partial^2_{x_n}f(\tilde{w})}\bigg|\prod_{\gamma\in\Gamma}\partial^\gamma X(w)-\prod_{\gamma\in\Gamma}\partial^\gamma X(z)\bigg|+\bigg|\frac{\partial^{\beta-\eta}\partial^{|\Gamma|+1}_{x_n}f(\tilde{w})}{\partial^2_{x_n}f(\tilde{w})}-\frac{\partial^{\beta-\eta}\partial^{|\Gamma|+1}_{x_n}f(\Tilde{z})}{\partial^2_{x_n}f(\Tilde{z})}\bigg|\prod_{\gamma\in\Gamma}|\partial^\gamma X(w)|
\]
meaning that the required semi-norm estimate on \(\partial^\beta X\) will follow if we can bound each of the four factors above in an appropriate fashion.


To this end we first observe that by lower control of \(\partial^2_{x_n}f\) on \(B\), together with slow variation of \(r\) and the derivative estimates of \Cref{cor:controlbound}, we have
\[
    \frac{|\partial^{\beta-\eta}\partial^{|\Gamma|+1}_{x_n}f(\tilde{w})|}{\partial^2_{x_n}f(\tilde{w})}\leq \frac{Cr(y)^{k+\alpha-|\beta|+|\eta|-|\Gamma|-1}}{r(y)^{k+\alpha-2}}=Cr(y)^{1-|\beta|+|\eta|-|\Gamma|}.
\]
Next, we employ the derivative bounds on \(X\) proved above for item \textit{(1)}. Note that since \(\Gamma\in P(\eta)\) for \(\eta\) satisfying \(|\eta|\leq |\beta|\leq k\), and since \(\Gamma\neq\{\beta\}\), for each \(\gamma\in \Gamma \) we must have that \(|\gamma|\leq k-1\). Therefore \(|\partial^\gamma X(w')|\leq Cr(y)^{1-|\gamma|}\) uniformly on \(B'\) by \textit{(1)}, and it follows that
\[
    \bigg|\prod_{\gamma\in\Gamma}\partial^\gamma X(w')\bigg|\leq C\prod_{\gamma\in\Gamma}r(y)^{1-|\gamma|}=Cr(y)^{|\Gamma|-|\eta|}.
\]
Using lower control of \(\partial^2_{x_n}f\) by \(r(y)^{k-2+\alpha}\) once again, we can also make the the following estimate,
\[
    \bigg|\frac{\partial^{\beta-\eta}\partial^{|\Gamma|+1}_{x_n}f(\tilde{w})}{\partial^2_{x_n}f(\tilde{w})}-\frac{\partial^{\beta-\eta}\partial^{|\Gamma|+1}_{x_n}f(\Tilde{z})}{\partial^2_{x_n}f(\Tilde{z})}\bigg|\leq \frac{C|\partial^2_{x_n}f(\Tilde{z})\partial^{\beta-\eta}\partial^{|\Gamma|+1}_{x_n}f(\tilde{w})-\partial^2_{x_n}f(\tilde{w})\partial^{\beta-\eta}\partial^{|\Gamma|+1}_{x_n}f(\Tilde{z})|}{r(y)^{2k+2\alpha-4}}.
\]
Employing the triangle inequality and our local control of derivatives of \(f\), we can bound the numerator on the right-hand side above by a constant multiple of the following expression,
\[
    r(y)^{k-2+\alpha}|\partial^{\beta-\eta}\partial^{|\Gamma|+1}_{x_n}f(\tilde{w})-\partial^{\beta-\eta}\partial^{|\Gamma|+1}_{x_n}f(\Tilde{z})|+r(y)^{k+\alpha-|\beta|+|\eta|-|\Gamma|-1}|\partial^2_{x_n}f(\Tilde{z})-\partial^2_{x_n}f(\tilde{w})|.
\]
If \(|\beta|-|\eta|+|\Gamma|+1<k\) then the Mean Value Theorem, together with our pointwise derivative estimates on \(f\), and the fact that \(|\tilde{w}-\Tilde{z}|\leq Cr(y)\) for \(w',z'\in B'\), all give
\[
    |\partial^{\beta-\eta}\partial^{|\Gamma|+1}_{x_n}f(\tilde{w})-\partial^{\beta-\eta}\partial^{|\Gamma|+1}_{x_n}f(\Tilde{z})|\leq Cr(y)^{k-|\beta|+|\eta|-|\Gamma|-1}|w'-z'|^\alpha.
\]
On the other hand, if \(|\beta|-|\eta|+|\Gamma|+1=k\) this estimate holds since \(\partial^{\beta-\eta}\partial^{|\Gamma|+1}_{x_n}f\in C^\alpha(\mathbb{R}^n)\) and 
\[
    |\partial^{\beta-\eta}\partial^{|\Gamma|}_{x_n}f(\tilde{w})-\partial^{\beta-\eta}\partial^{|\Gamma|}_{x_n}f(\Tilde{z})|\leq C|\tilde{w}-\Tilde{z}|^\alpha\leq C|w'-z'|^\alpha=Cr(y)^{k-|\beta|+|\eta|-|\Gamma|-1}|w'-z'|^\alpha.
\]
An identical argument shows that \(|\partial^2_{x_n}f(z',X(z'))-\partial^2_{x_n}f(w',X(w'))|\leq Cr(y)^{k-2}|w'-z'|^\alpha\) when \(k\geq 2\), and altogether we find now that
\[
    \bigg|\frac{\partial^{\beta-\eta}\partial^{|\Gamma|+1}_{x_n}f(\tilde{w})}{\partial^2_{x_n}f(\tilde{w})}-\frac{\partial^{\beta-\eta}\partial^{|\Gamma|+1}_{x_n}f(\Tilde{z})}{\partial^2_{x_n}f(\Tilde{z})}\bigg|\leq Cr(x)^{1-\alpha-|\beta|+|\eta|-|\Gamma|}|w'-z'|^\alpha.
\]


One more term estimate remains before we can invoke an inductive hypothesis on the H\"older semi-norms of \(X\) to simplify our bounds. Iterating the triangle inequality using \Cref{lem:proddiff} gives
\[
    \bigg|\prod_{\gamma\in\Gamma}\partial^\gamma X(w')-\prod_{\gamma\in\Gamma}\partial^\gamma X(z')\bigg|\leq \sum_{\gamma\in\Gamma}\bigg(\prod_{\mu\in\Gamma\setminus\{\gamma\}}\sup_{B'}|\partial^\mu X|\bigg)|\partial^\gamma X(w')-\partial^\gamma X(z')|,
\]
and using the uniform estimates established above for item \textit{(1)}, we can bound this further by
\[
    \bigg|\prod_{\gamma\in\Gamma}\partial^\gamma X(w')-\prod_{\gamma\in\Gamma}\partial^\gamma X(z')\bigg|\leq C\sum_{\gamma\in\Gamma}r(y)^{|\Gamma|-1-|\eta|+|\gamma|}|\partial^\gamma X(w')-\partial^\gamma X(z')|. 
\]
Consequently, we can bound each term in our earlier expansion of \(\partial^\beta X(w')-\partial^\beta X(z')\) to get
\[
    |\partial^\beta X(w')-\partial^\beta X(z')|\leq C\sum_{\eta\leq\beta}\sum_{ \substack{\Gamma\in P(\eta),\\\Gamma\neq \{\beta\}}}\sum_{\gamma\in\Gamma}(r(y)^{-|\beta|+|\gamma|}|\partial^\gamma X(w')-\partial^\gamma X(z')|+r(y)^{1-\alpha-|\beta|}|w'-z'|^\alpha).
\]


Dividing the expression above by \(|w'-z'|^\alpha\) and taking a limit supremum as \(w',z'\rightarrow y'\), it follows from definition \eqref{eq:pwop} of the pointwise H\"older semi-norm that
\[
    [\partial^\beta X]_\alpha(y')\leq C\sum_{\eta\leq\beta}\sum_{ \substack{\Gamma\in P(\eta),\\\Gamma\neq \{\beta\}}}\sum_{\gamma\in\Gamma}r(y)^{-|\beta|+|\gamma|}[\partial^\gamma X]_\alpha(y')+Cr(y)^{1-\alpha-|\beta|}.
\]
Finally, assume for a strong inductive hypothesis that \([\partial^\gamma X]_{\alpha}(y')\leq Cr(y)^{1-\alpha-|\gamma|}\) for each \(y\in B\) whenever \(|\gamma|<|\beta|\). It follows immediately that
\[
    [\partial^\beta X]_\alpha(y')\leq C\sum_{\eta\leq\beta}\sum_{ \substack{\Gamma\in P(\eta),\\\Gamma\neq \{\beta\}}}\sum_{\gamma\in\Gamma}r(y)^{-|\beta|+|\gamma|}r(y)^{1-\alpha-|\gamma|}+Cr(y)^{1-\alpha-|\beta|}=Cr(y)^{1-\alpha-|\beta|}.
\]
By induction this holds for every \(\beta\) with \(|\beta|\leq k-1\), showing that \textit{(2)} holds on \(B'\).
\end{proof}


Henceforth, we assume that \(\nu\) has been chosen small enough that the \(\nu\)-dependant estimates of Lemmas \ref{lem:slowvar}, \ref{cor:supplementary}, \ref{lem:minia} and \ref{lem:minia2} all hold. Moreover, we assume that \eqref{eq:needed} also holds and we recall that this is automatic when \(k=2\) and \(k=3\). Our reward for undergoing the hard work of defining \(X\) on \(B\) and proving its various properties is the ability to unambiguously define
\[
    F(y)=f(y',X(y')).
\]
It follows from the construction above that \(f-F\geq 0\) on \(B\), with equality at local minima of \(f\). 


Additionally, we will show that \(f-F\) has a root which is half as regular as \(f\) in the sense of \eqref{eq:rootspaces}, meaning that we can form the decomposition \(f=(\sqrt{f-F})^2+F\) on \(B\). This was the key insight underlying Fefferman \& Phong's result from \cite{Fefferman-Phong} in the setting of \(C^{3,1}(\mathbb{R}^n)\). To prove that this decomposition has the desired properties, we begin by establishing some estimates.


\begin{lem}\label{lem:nice}
Assume that the hypotheses of \Cref{lem:minia} hold, let \(X\) and be as above, and define \(F(y)=f(y',X(y'))\). Then for \(\beta\) with \(|\beta|<\frac{k+\alpha}{2}\) and every \(y\in B\) the following estimates hold,
\begin{itemize}
    \item[(1)] \(|\partial^\beta\sqrt{f(y)-F(y)}|\leq Cr(y)^{\frac{k+\alpha}{2}-|\beta|}\),
    \item[(2)] \([\partial^\beta\sqrt{f-F}]_{\frac{\alpha}{2}}(y)\leq Cr(y)^{\frac{k}{2}-|\beta|}\) if \(k\) is even,
    \item[(3)] \([\partial^\beta\sqrt{f-F}]_{\frac{1+\alpha}{2}}(y)\leq Cr(y)^{\frac{k-1}{2}-|\beta|}\) if \(k\) is odd.
\end{itemize}
\end{lem}


\begin{proof}
By the fundamental theorem of calculus we have \(f(y)-F(y)=(y_n-X(y'))^2H(y)\), where
\[
    H(y)=\int_0^1 (1-t)\partial^2_{x_n}f(y',ty_n+(1-t)X(y'))dt.
\]
Arguing as in the proof of \Cref{lem:minia} we have \(\frac{1}{2}r(x)^{k-2+\alpha}\leq \partial^2_{x_n}f(y',ty_n+(1-t)X(y'))\) for every \(t\in[0,1]\), meaning that \(H(y)\geq Cr(y)^{k-2+\alpha}\) on \(B\). Additionally, we can bound the derivatives of \(H\) on \(B\). To this end we first observe that
\[
    \partial^\beta H(y)=\int_0^1 (1-t)\partial^\beta[\partial^2_{x_n}f(y',ty_n+(1-t)X(y'))]dt.
\]
To evaluate the derivative inside the integral, we use the shorthand \(L(y,t)=ty_n+(1-t)X(y')\) and write \(\beta'=(\beta_1,\dots,\beta_{n-1})\), so that an application of \Cref{lem:genchain} gives
\[
    \partial^\beta[\partial^2_{x_n}f(y',L(y,t))]=t^{\beta_n}\sum_{\mu\leq\beta'}\sum_{\Gamma\in P(\mu)}C_{\beta',\Gamma}(1-t)^{|\Gamma|}\partial^{\beta'-\mu}\partial_{x_n}^{2+\beta_n+|\Gamma|}f(y',L(y,t))\prod_{\gamma\in\Gamma}\partial^\gamma X(y'),
\]
where the constants \(C_{\beta',\Gamma}\) are given by \eqref{eq:cm}. It follows from the triangle inequality that
\[
    |\partial^\beta H(y)|\leq  C\sum_{\mu\leq\beta'}\sum_{\Gamma\in P(\mu)}\int_0^1t^{\beta_n}(1-t)^{|\Gamma|+1}|\partial^{\beta'-\mu}\partial_{x_n}^{2+\beta_n+|\Gamma|}f(y',L(y,t))|dt\prod_{\gamma\in\Gamma}|\partial^\gamma X(y')|.
\]
Using estimate \textit{(1)} of \Cref{lem:minia2} we get \(|\partial^\gamma X(y')|\leq Cr(y)^{1-|\gamma|}\). Additionally, \Cref{cor:controlbound} and \Cref{lem:slowvar} give \(|\partial^{\beta'-\mu}\partial_{x_n}^{2+\beta_n+|\Gamma|}f(y',L(y,t))|\leq Cr(y)^{k+\alpha-2-|\Gamma|-|\beta|+|\mu|}\), so
\[
    |\partial^\beta H(y)|\leq  C\sum_{\mu\leq\beta'}\sum_{\Gamma\in P(\mu)}r(y)^{k+\alpha-2-|\Gamma|-|\beta|+|\mu|}\prod_{\gamma\in\Gamma}r(y)^{1-|\gamma|}=Cr(y)^{k+\alpha-2-|\beta|}.
\]


Next, using these bounds we can estimate derivatives of \(\sqrt{H}\) on \(B\). To this end we employ \Cref{cor:rootscor}, the lower bound \(H(y)\geq Cr(y)^{k-2+\alpha}\), and the derivative bound above to estimate
\[
    |\partial^\beta \sqrt{H(y)}| =\bigg|\sum_{\Gamma\in P(\beta)}C_{\beta,\Gamma}H(y)^{\frac{1}{2}-|\Gamma|}\prod_{\gamma\in\Gamma}\partial^\gamma H(y)\bigg|\leq Cr(y)^{\frac{k+\alpha}{2}-1-|\beta|}.
\]
Further, estimate \textit{(1)} of \Cref{lem:minia} and estimate \textit{(1)} of \Cref{lem:minia2} together show us that \(|\partial^\beta(y_n-X(y'))|\leq Cr(y)^{1-|\beta|}\) on \(B\), respectively when \(|\beta|=0\) and when \(|\beta|>0\). Using the Leibniz rule we can thus bound derivatives of \(\sqrt{f(y)-F(y)}=(y_n-X(y'))\sqrt{H(y)}\) by writing
\[
    \partial^\beta\sqrt{f(y)-F(y)}=\sum_{\gamma\leq\beta}\binom{\beta}{\gamma}\partial^{\beta-\gamma}(y_n-X(y'))\partial^\gamma\sqrt{H(y)}.
\]
Employing the triangle inequality and the derivative bounds computed above, we find now that
\[
    |\partial^\beta\sqrt{f(y)-F(y)}|\leq C\sum_{\gamma\leq\beta}|\partial^{\beta-\gamma}(y_n-X(y'))||\partial^\gamma\sqrt{H(y)}|\leq Cr(y)^{\frac{k+\alpha}{2}-|\beta|}.
\]
Since \(y\in B\) was arbitrary, estimate \textit{(1)} of \Cref{lem:nice} follows.


For estimates \textit{(2)} and \textit{(3)} we once again denote by \(\Lambda\) the integer part of \(\frac{k}{2}\), and set \(\lambda=\frac{k+\alpha}{2}-\Lambda\) so that the claimed estimates will follow if we can show that \([\partial^\beta\sqrt{f-F}]_{\lambda}(y)\leq Cr(y)^{\Lambda-|\beta|}\). To this end we observe that if \(w,z\in B\), then the Mean Value Theorem and the derivative estimate above together give
\[
    |\partial^\beta\sqrt{f(w)-F(w)}-\partial^\beta\sqrt{f(z)-F(z)}|\leq Cr(y)^{\frac{k+\alpha}{2}-|\beta|-1}|w-z|\leq Cr(y)^{\Lambda-|\beta|}|w-z|^\lambda.
\]
It follows by taking a limit supremum as \(w,z\rightarrow y\) that \([\partial^\beta\sqrt{f-F}]_{\lambda}(y)\leq Cr(y)^{\Lambda-|\beta|}\).
\end{proof}


There remains one more critical property of the remainder \(F\) to establish before we can proceed to extend the local results of this section to global ones. These are derivative and local semi-norm estimates, which follow from the properties of \(X\) established in \Cref{lem:minia2}.

\begin{lem}\label{lem:implicitest}
Assume that the hypotheses of \Cref{lem:minia} hold on \(B\) for some fixed \(x\), and define \(X\) as above on \(B'=\{x':x\in B\}\). If \(F(y')=f(y',X(y'))\) then for every \(y\in B\) the following estimates hold, 
\begin{itemize}
    \item[(1)] \(|\partial^\beta F(y)|\leq Cr(y)^{k+\alpha-|\beta|}\),
    \item[(2)] \([\partial^\beta F]_{\alpha}(y)\leq Cr(y)^{k-|\beta|}\).
\end{itemize}
\end{lem}

\begin{proof}
Given a multi-index \(\beta\) of order \(|\beta|\leq k\), we compute \(\partial^\beta F\) by first writing \(\beta=\mu+\rho\), for a multi-index \(\rho\leq \beta\) with \(|\rho|=1\), and with \(\mu=\beta-\rho\). It follows from the definition of \(X\) that
\[
    \partial^\beta F(y')=\partial^\mu[\partial^\rho f(y',X(y'))+\partial_{x_n}f(y',X(y'))\partial^\rho X(y')]=\partial^\mu[\partial^\rho f(y',X(y'))].
\]
To compute the remaining \(\mu\)-derivative on the right-hand side, we employ \Cref{lem:genchain} to get
\[
    \partial^\beta F(y')=\partial^\mu[\partial^\rho f(y',X(y'))]=\sum_{\eta\leq\mu }\sum_{\Gamma\in P(\eta)}C_{\mu,\Gamma}(\partial^{\beta-\eta}\partial^{|\Gamma|}_{x_n}f(y',X(y')))\prod_{\gamma\in\Gamma}\partial^\gamma X(y').
\]


Although the right-hand side of the formula above for \(\partial^\beta F\) appears to depend on our choice of \(\mu\), it turns out that this formula yields the same derivative regardless of which \(\mu\) is chosen. This is due to commutation relations which follow directly from the identity \(\partial_{x_n}f(y',X(y'))=0\). For instance, with second-order derivatives we have everywhere on the domain of \(F\) that
\[
    \partial_{x_i}\partial_{x_n}f(y',X(y'))\partial_{x_j}X(y')=\partial_{x_j}\partial_{x_n}f(y',X(y'))\partial_{x_i}X(y').
\]
%Independence of \(\partial^\beta F\) from our choice of \(\mu\) also follows as a straightforward consequence of Schwarz's Theorem for mixed partials. 
Thus no generality is lost by fixing \(\rho\) and \(\mu\) as above to compute \(\partial^\beta F\). With this selection we observe by the triangle inequality that
\[
    |\partial^\beta F(y')|\leq C\sum_{\eta\leq\mu }\sum_{\Gamma\in P(\eta)}|\partial^{\beta-\eta}\partial^{|\Gamma|}_{x_n}f(y',X(y'))|\prod_{\gamma\in\Gamma}|\partial^\gamma X(y')|.
\]
Since \(|\mu|=|\beta|-1\leq k-1\), for each \(\gamma\) in the product above, we have that \(|\partial^\gamma X(y')|\leq Cr(y)^{1-|\gamma|}\) on \(B'\) by estimate \textit{(1)} of \Cref{lem:minia2}. Similarly, \Cref{cor:controlbound} and slow variation of \(r\) allow us to bound the derivatives of \(F\) above uniformly by powers of \(r(y)\) to get item \textit{(1)},
\[
    |\partial^\beta F(y')|\leq C\sum_{\eta\leq\mu }\sum_{\Gamma\in P(\eta)}r(y)^{k+\alpha-|\beta|+|\eta|-|\Gamma|}\prod_{\gamma\in\Gamma}r(y)^{1-|\gamma|}=Cr(y)^{k+\alpha-|\beta|}.
\]


For the estimates of \textit{(2)}, we take \(w',z'\in B'\) and for brevity we write \(\tilde{w}=(w',X(w'))\). As above, for some multi-index \(\mu< \beta\) of order \(|\beta|-1\) we can expand \(\partial^\beta F\) as follows:
\[
    \partial^\beta F(w')-\partial^\beta F(z')= \sum_{0\leq\eta\leq\mu}\sum_{\Gamma\in P(\eta)}C_{\beta,\Gamma}\bigg(\partial^{\beta-\eta}\partial^{|\Gamma|}_{x_n}f(\tilde{w})\prod_{\gamma\in\Gamma}\partial^\gamma X(w)-\partial^{\beta-\eta}\partial^{|\Gamma|}_{x_n}f(\Tilde{z})\prod_{\gamma\in\Gamma}\partial^\gamma X(z)\bigg).
\]
From here, an argument identical to that used in our proof of item \textit{(2)} of \Cref{lem:minia} shows that \([\partial^\beta F]_{\alpha}(y')\leq Cr(y)^{k-|\beta|}\) for every derivative of order \(|\beta|\leq k\) and every \(y'\in B'\), meaning that item \textit{(2)} holds as claimed.
\end{proof}


For later convenience, we now replace \(\nu\) with \(\frac{1}{2}\nu\) to ensure that the functions \(X\) and \(F\) are well-defined and satisfy the estimates established above on the dilated ball \(2B'\). This substitution comes only at the expense of making several constant coefficients in the estimates above larger. Extension of \(F\) to \(2B'\) is critical, as it allows us to show in the following lemma that there exists a \(C^{k,\alpha}(\mathbb{R}^{n-1})\) function which agrees with \(F\) on \(B'\). This function is required in the inductive argument that we employ to prove \Cref{thm:c3main}.


\begin{lem}\label{lem:crudeext}
There exists a non-negative function in \( C^{k,\alpha}(\mathbb{R}^{n-1})\) which agrees with \(F\) on \(B'\).
\end{lem}


\begin{proof}
Let \(\psi\) be a smooth non-negative function with compact support contained in the ball \(B(2,0)\subset\mathbb{R}^{n-1}\), such that \(\psi=1\) on the ball of half radius \(B(1,0)\subset\mathbb{R}^{n-1}\). Then set
\[
    \varphi(y')=\psi\bigg(\frac{y'-x'}{\nu r(x)}\bigg),
\]
so that \(\varphi\) is supported in \(2B'\) and \(\varphi=1\) on \(B'\). It remains to show that \(\varphi F\in C^{k,\alpha}(\mathbb{R}^n)\). To this end we first construct derivative estimates on \(\varphi\), observing that repeat differentiation gives
\[
    |\partial^\beta\varphi(y')|=\frac{1}{\nu^{|\beta|}r(x)^{|\beta|}}\bigg|\partial^\beta\psi\bigg(\frac{y'-x'}{\nu r(x)}\bigg)\bigg|\leq Cr(x)^{-|\beta|},
\]
since \(\psi\) and all of its derivatives are uniformly bounded independent of \(r\) on \(\mathbb{R}^{n-1}\). Additionally, we can observe for any multi-index \(\beta\) and \(0<\alpha\leq 1\) that \([\partial^\beta\varphi]_{\alpha,\mathbb{R}^n}=[\partial^\beta\varphi]_{\alpha,B'}\). Therefore by the Mean Value Theorem and the estimate on \(\partial^\beta\psi\) above, it follows that \([\partial^\beta\varphi]_{\alpha,\mathbb{R}^n}\leq Cr(x)^{-\alpha-|\beta|}\).


Combining these estimates on \(\varphi\) with the inequalities of \Cref{lem:implicitest}, a calculation using the Leibniz rule shows that for any multi-index \(\beta\) and \(y'\in B'\),
\[
    |\partial^\beta\varphi(y') F(y')|\leq C\sum_{\gamma\leq\beta}|\partial^\gamma\varphi (y')||\partial^{\beta-\gamma}F(y')|\leq C\sum_{\gamma\leq\beta}r(x)^{-|\gamma|}r(x)^{k+\alpha-|\beta|+|\gamma|}=Cr(x)^{k+\alpha-|\beta|}.
\]
Taking a supremum over \(y'\in 2B'\) yields \(\sup_{2B'}|\partial^\beta(\varphi F)|\leq Cr(x)^{k+\alpha-\beta}\), and since \(\varphi F\) is identically zero outside of \(2B'\) it follows that this bound holds over \(\mathbb{R}^n\) and \(\varphi F\in C^{k}(\mathbb{R}^{n-1})\). 

It remains to verify that \(\partial^\beta(\varphi F)\in C^{\alpha}(\mathbb{R}^{n-1})\) whenever \(|\beta|=k\). To this end we first use the sub-product rule of \Cref{lem:subprod} to write
\[
    [\partial^\beta(\varphi F)]_{\alpha,\mathbb{R}^{n-1}}=[\partial^\beta(\varphi F)]_{\alpha,2B'}\leq\sum_{\gamma\leq\beta}\binom{\beta}{\gamma}([\partial^{\beta-\gamma}\varphi]_{\alpha}\sup_{2B'}|\partial^\gamma F|+[\partial^{\beta-\gamma}F]_{\alpha,B'}\sup_{\mathbb{R}^{n-1}}|\partial^\gamma \varphi|).
\]
Employing the estimates from above, we see that
\[
    [\partial^\beta(\varphi F)]_{\alpha,\mathbb{R}^{n-1}}\leq C\sum_{\gamma\leq\beta}(r(x)^{-\alpha -|\beta|+|\gamma|}r(x)^{k+\alpha-|\gamma|}+r(x)^{k-|\beta|+|\gamma|}r(x)^{-|\gamma|})=Cr(x)^{k-|\beta|}
\]
In particular, we see that if \(|\beta|=k\) then \([\partial^\beta(\varphi F)]_{\alpha,\mathbb{R}^{n-1}}\leq C\), giving \(\varphi F\in C^{k,\alpha}(\mathbb{R}^{n-1})\).
\end{proof}


\textit{Remark}: In passing, we observe that by iterating the preceding argument for \(\varphi\) as above, we can show that in fact \(\varphi^mF\) is a non-negative \(C^{k,\alpha}(\mathbb{R}^{n-1})\) function for any \(m\in\mathbb{N}\). The most important consequence of the preceding result is that the restriction of \(F\) to a small enough ball can be extended to a function defined on all of \(\mathbb{R}^{n-1}\), and this extension can be achieved without sacrificing regularity.


To summarize the results of this section, if \(f\in C^{k,\alpha}(\mathbb{R}^n)\) is non-negative and bounded below by a constant multiple of \(r(x)^{k+\alpha}\) at the center of the ball \(B=B(x,\nu r(x))\), then \(\sqrt{f}\) is half as regular as \(f\) on \(B\). On the other hand, if \(f\) is bounded above at the center of \(B\) then we can construct a local decomposition of the form
\[
    f=g^2+F,
\]
where \(g=\sqrt{f-F}\in C^\frac{k+\alpha}{2}(B)\) and \(F\in C^{k,\alpha}(B)\) depends on \(n-1\) variables. Moreover, there exists a non-negative function in \(C^{k,\alpha}(\mathbb{R}^n)\) whose restriction to \(B'\) agrees with \(F\). In any case, we see that \(f\) can be locally decomposed.


\section{Partitions of Unity}\label{sec:pou}


To extend our local decompositions to global ones, we construct functions which sum to a constant on the set where \(f\) and its derivatives are nonzero, and which are compactly supported on the balls identified in the previous section. Our main instrument for extending local estimates is the following theorem.


\begin{thm}\label{thm:party}
Let \(r\) be slowly-varying on balls of the form \(B(x,\nu r(x))\). Then there exists a countable collection of balls \(\{B(x_j,\nu r_j)\}\), where \(r_j=r(x_j)\), and a partition of unity
\begin{equation}\label{eq:pou2}
    \sum_{j=1}^\infty\psi_j(x)^2=
    \begin{cases}
    1 & \textrm{if }\;r(x)>0,\\
    0 & \textrm{if }\;r(x)=0,
    \end{cases}
\end{equation}
which has the following properties:
\begin{itemize}
    \item[(1)] The support of each \(\psi_j\) is contained in the ball \(B(x_j,\nu r_j)\),
    \item[(2)] Every \(x\in\mathbb{R}^n\) belongs to at most \(N_n\) of the balls for a constant \(N_n\) that depends only on \(n\),
    \item[(3)] For every multi-index \(\beta\), \(\psi_j\) satisfies the estimate \(\sup_{\mathbb{R}^n}|\partial^\beta\psi_j|\leq C r_j^{-|\beta|}\),
    \item[(4)] For every multi-index \(\beta\), \(x\in\mathbb{R}^n\) and \(\alpha\in(0,1]\), \(\psi_j\) satisfies \([\partial^\beta\psi_j]_{\alpha}(x)\leq Cr_j^{-\alpha-|\beta|}\),
    \item[(5)] There exist \(N_n\) sub-collections of \(\{\psi_j\}\) that are each comprised of functions which have pairwise disjoint support (i.e. no two functions in any of the \(N_n\) sub-collections are nonzero simultaneously at any given point in \(\mathbb{R}^n\)). 
\end{itemize}
\end{thm}

\medskip

\textit{Remark}: Such a partition can be constructed for any function with the slow-variation property, and it is not exclusive to \(r\) as given by \eqref{eq:controlfunc}. A similar result to \Cref{thm:party} is proved by De Guzman in \cite[Lemma 1.2]{guzman}, however the version we require differs in several respects. As such, we include a complete and independent proof here. Our construction is motivated in spirit by the argument in \cite{Tataru}, but it differs considerably at several points.


\begin{proof}
We first set \(S=\{x\in\mathbb{R}^n:r(x)>0\}\) and recursively construct a cover of \(S\) by balls of the form \(B(x_j,\frac{1}{2}\nu r_j)\). Having chosen the first \(\ell\) balls, we add to that collection the ball \(B(x_{\ell+1},\frac{1}{2}\nu r_{\ell+1})\) centred at any point \(x_{\ell+1}\in S\setminus(\bigcup_{j=1}^\ell B(x_j,\frac{1}{2}\nu r_j))\). If \(x\in B(x_j,\frac{1}{2}\nu r_j)\) for some \(j\) then \(2r(x)>r_j>0\) by slow variation, meaning \(x\in S\), and from this construction we thus get
\begin{equation}\label{eq:seteq}
    S=\bigcup_{j=1}^\infty B\bigg(x_j,\frac{1}{2}\nu r_j\bigg).
\end{equation}


A useful property of these balls is that their dilates \(B(x_j,\frac{1}{8}\nu r_j)\) are pairwise disjoint. To see this, assume toward a contradiction that \(x\in B(x_i,\frac{1}{8}\nu r_i)\cap B(x_j,\frac{1}{8}\nu r_j)\) for some \(i>j\). Then \(r_i\leq \frac{4}{3}r(z)\leq \frac{5}{3}r_j\) by slow variation and
\[
    |x_i-x_j|\leq |x_i-x|+|x_j-x|<\frac{1}{8}\nu(r_i+r_j)\leq \frac{1}{8}\nu\bigg(\frac{5}{3}r_j+r_j\bigg)=\frac{1}{3}\nu r_j<\frac{1}{2}\nu r_j.
\]
Thus \(x_i\in B(x_j,\frac{1}{2}\nu r_j)\), contrary to our construction. Using the pairwise disjoint property we can show that any given point belongs to only finitely many balls of the form \(B(x_j,\nu r_j)\). Let \(m\) denote the number of balls which contain a fixed point \(x\in S\). If \(B(x_j,\nu r_j)\) is one of these balls then \(\frac{4}{5}r(x)\leq r_j \leq \frac{4}{3}r(x)\) by slow variation, meaning that \(B(x_j,\frac{1}{10}r(x))\subseteq B(x_j,\frac{1}{8}r_j)\). Additionally, if \(y\in B(x_j,\frac{1}{8}\nu r_j)\) then \(|x-y|\leq |x-x_j|+|y-x_j|< \frac{9}{8}\nu r_j\leq \frac{3}{2}r(x)\), meaning that \(B(x_j,\frac{1}{8}\nu r_j)\subseteq B(x,\frac{3}{2}r(x))\). Taking a union of the \(m\) disjoint balls centred at \(x_{j_1},\dots,x_{j_m}\) whose dilates contain \(x\), we see from these estimates that
\[
    \bigcup_{k=1}^m B\bigg(x_{j_k},\frac{1}{10}\nu r(x)\bigg)\subseteq \bigcup_{k=1}^m B\bigg(x_{j_k},\frac{1}{8}\nu r_j\bigg)\subseteq B\bigg(x,\frac{3}{2}\nu r(x)\bigg).
\]
Since the \(m\) balls on the left are pairwise disjoint and they all have the same size, it follows that
\[
     \frac{mV \nu^nr(x)^n}{10^n}\leq \frac{3^nV \nu^nr(x)^n}{2^n},
\]
where \(V\) denotes the volume of the unit ball in \(\mathbb{R}^n\). This simplifies to give \(m \leq 15^n\), meaning that each \(x\in S\) belongs to at most \(N_n=15^n\) balls of the form \(B(x_j,\nu r_j)\) from our cover of \(S\).


Next let \(\psi\) be a non-negative smooth function that is supported in the unit ball in \(\mathbb{R}^n\), which satisfies \(\psi\leq 1\) everywhere and \(\psi=1\) on \(B(0,\frac{1}{2})\). For \(x\in S\) we then define
\begin{equation}\label{eq:partfunc}
    \psi_j(x)=\psi\bigg(\frac{x-x_j}{\nu r_j}\bigg)\bigg(\sum_{k=1}^\infty\psi\bigg(\frac{x-x_k}{\nu r_k})\bigg)^2\bigg)^{-\frac{1}{2}},
\end{equation}
so that \(\sup|\psi_j|\leq 1\) and \(\psi_j\) is supported in \(B(x_j,\nu r_j)\). Each \(x\) belongs to some ball of the form \(B(x_j,\frac{1}{2}\nu r_j)\) and at most \(N_n\) balls of the form \(B(x_j,\nu r_j)\), meaning that 
\begin{equation}\label{eq:lowercontrol}
    1\leq \sum_{k=1}^\infty\psi\bigg(\frac{x-x_k}{\nu r_k}\bigg)^2\leq N_n.
\end{equation}
Consequently the sum in \eqref{eq:partfunc} is nonzero and finite, so there is no issue of convergence.

Now we establish derivative and semi-norm estimates for \(\psi_j\). Since \(\psi\) is a smooth function with compact support, \(\sup_{\mathbb{R}^n}|\partial^\beta\psi|\leq C \) for a constant that depends on \(\beta\). By the general Leibniz rule \Cref{lem:Leib}, we may write
\begin{equation}\label{eq:leib}
    \partial^\beta\psi_j(x)=\sum_{\gamma\leq\beta}\binom{\beta}{\gamma}\bigg(\partial^\gamma\psi\bigg(\frac{x-x_j}{\nu r_j}\bigg)\bigg)\bigg(\partial^{\beta-\gamma}\bigg(\sum_{k=1}^\infty\psi\bigg(\frac{x-x_{k}}{\nu r_k}\bigg)^2\bigg)^{-\frac{1}{2}}\bigg).
\end{equation}
The derivatives of \(\psi_j\) are bounded, so \(\sup|\partial^\gamma\psi(\frac{x-x_j}{\nu r_j})|\leq C r_j^{-|\gamma|}\). Additionally, we can use the first inequality of \eqref{eq:lowercontrol} along with the fact that \(cr_j\leq r_k\leq Cr_j\) when \(x\in B(x_j,\nu r_j)\cap B(x_k,\nu r_k) \), to get from the chain rule \Cref{cor:chain2} that 
\[
    \bigg|\partial^{\beta-\gamma}\bigg(\sum_{k=1}^\infty\psi\bigg(\frac{x-x_{k}}{\nu r_k}\bigg)^2\bigg)^{-\frac{1}{2}}\bigg|\leq Cr_j^{|\gamma|-|\beta|}.
\]
Combining these estimates with \eqref{eq:leib} we get \(\sup|\partial^\beta \psi_j|\leq Cr_j^{-|\beta|}\). For semi-norm estimates we let \(y,z\in B(x_j,\nu r_j)\) and use the Mean Value Theorem together with derivative bound \textit{(3)} to get
\[
    |\partial^\beta\psi_j(y)-\partial^\beta\psi_j(z)|\leq Cr_j^{-|\beta|-1}\leq  C\nu^{1-\alpha} r_j^{-\alpha-|\beta|}|y-z|^\alpha.
\]
The semi-norm estimate \textit{(4)} then follows from taking a limit supremum as \(y,z\rightarrow x\).


Finally, to prove item \textit{(5)} we construct an infinite graph \(G\) as follows: to each ball in the collection \(\{B(x_j,\nu r_j)\}\) we assign a vertex, and we add an edge between two given vertices if their corresponding balls intersect. Since our collection has bounded overlap, each vertex of \(G\) has degree at most \(N_n\), and by \cite[Theorem 3]{behzad} it follows that the chromatic number of \(G\) is bounded above by \(N_n\). Consequently we can find \(N_n\) sub-collections of \(\{B(x_j,\nu r_j)\}\), each comprised of pairwise disjoint balls. The functions \(\psi_j\) corresponding to the balls in any given sub-collection of \(\{B(x_j,\nu r_j)\}\) form the desired subset of functions having pairwise disjoint supports, giving item \textit{(5)} and completing the proof.
\end{proof}
 

\textit{Remark}: Our use of squares in \eqref{eq:pou2} is helpful in the proof of \Cref{thm:c3main}, however from a general standpoint it is insignificant. Given a countable collection of integers \(\{m_j\}\) we can replace \eqref{eq:partfunc} with the alternative definition
\[
    \psi_j(x)=\psi\bigg(\frac{x-x_j}{\nu r_j}\bigg)\bigg(\sum_{k=1}^\infty\psi\bigg(\frac{x-x_k}{\nu r_k})\bigg)^{m_k}\bigg)^{-\frac{1}{m_j}}.
\]
This allows us to replace the sum of squares in \eqref{eq:pou2} with \(\sum_{j=1}^\infty\psi_j^{m_j}\), and it is easy to check that \(\psi_j\) defined in this way satisfies similar differential inequalities to those found above. For the purposes of this work, our continued use of squares is sufficient. However, this observation may be useful if one wishes to decompose into sums of arbitrary integer powers instead.


Combining \Cref{thm:party} with the results of the preceding section, we are now able to extend our local decompositions of \(f\) to a global one. Our first result to this end essentially states that the local roots identified in the previous section extend to \(C^{\frac{k+\alpha}{2}}(\mathbb{R}^n)\) functions when multiplied by the partition functions constructed in Theorem \ref{thm:party}.



\begin{lem}\label{lem:rootscors}
Let \(f\) be a non-negative \(C^{k,\alpha}(\mathbb{R}^n)\) function, and let \(\psi_j\) be one of the partition functions in \eqref{eq:pou2} supported in \(B(x_j,\nu r_j)\). If \(f(x_j)\geq \omega r_j^{k+\alpha}\) then the following estimates hold pointwise in \(\mathbb{R}^n\),
\begin{itemize}
    \item[(1)] \(|\partial^\beta[\psi_j\sqrt{f}](x)|\leq Cr(x)^{\frac{k+\alpha}{2}-|\beta|}\),
    \item[(2)] \([\partial^\beta(\psi_j\sqrt{f})]_{\frac{\alpha}{2}}(x)\leq Cr(x)^{\frac{k}{2}-|\beta|}\) if \(k\) is even,
    \item[(3)] \([\partial^\beta(\psi_j\sqrt{f})]_{\frac{1+\alpha}{2}}(x)\leq Cr(x)^{\frac{k-1}{2}-|\beta|}\) if \(k\) is odd.
\end{itemize}
Similarly, if \(f(x_j)<\omega r_j^{k+\alpha}\) and the other hypotheses of \Cref{lem:minia} are satisfied, then for \(F\) as in \Cref{lem:nice} we also have
\begin{itemize}
    \item[(4)] \(|\partial^\beta[\psi_j\sqrt{f-F}](x)|\leq Cr(x)^{\frac{k+\alpha}{2}-|\beta|}\),
    \item[(5)] \([\partial^\beta(\psi_j\sqrt{f-F})]_{\frac{\alpha}{2}}(x)\leq Cr(x)^{\frac{k}{2}-|\beta|}\) if \(k\) is even,
    \item[(6)] \([\partial^\beta(\psi_j\sqrt{f-F})]_{\frac{1+\alpha}{2}}(x)\leq Cr(x)^{\frac{k-1}{2}-|\beta|}\) if \(k\) is odd.
\end{itemize}
\end{lem}


\begin{proof}
First we bound the derivatives of \(\psi_j\sqrt{f}\) pointwise on the ball \(B=B(x_j,\nu r_j)\), employing \Cref{lem:Leib} and the triangle inequality to achieve the following pointwise bound
\[
    |\partial^\beta[\psi_j\sqrt{f}](x)|\leq \sum_{\gamma\leq\beta}\binom{\beta}{\gamma}|\partial
    ^\gamma \sqrt{f(x)}||\partial^{\beta-\gamma}\psi_j(x)|.
\]
Using the derivative bounds on \(\psi_j\) from \Cref{thm:party}, the local derivative bounds on \(\sqrt{f}\) from \Cref{lem:local1}, and the fact that \(r(x)\leq Cr_j\) for \(x\in B\) by slow variation of \(r\), it follows that
\[
    |\partial^\beta[\psi_j\sqrt{f}](x)|\leq C\sum_{\gamma\leq\beta}r(x)^{\frac{k+\alpha}{2}-|\gamma|}r_j^{|\gamma|-|\beta|}\leq Cr(x)^{\frac{k+\alpha}{2}-|\beta|}.
\]
If \(x\not\in B\) then \(\psi_j\) and all of its derivatives are identically zero, so \textit{(1)} holds trivially there and we conclude that \textit{(1)} holds on all of \(\mathbb{R}^n\).


For \textit{(2)} and \textit{(3)} we let \(\Lambda\) denote the integer part of \(\frac{k}{2}\) and set \(\lambda=\frac{k+\alpha}{2}-\Lambda\) so that it suffices to show that \([\partial^\beta(\psi_j\sqrt{f})]_\lambda(x)\leq Cr(x)^{\Lambda-|\beta|}\) on \(\mathbb{R}^n\). To this end we employ the sub-product rule of \Cref{lem:subprod} to see for \(x\in B\) that
\[
    [\partial^\beta(\psi_j\sqrt{f})]_\lambda(x)\leq \sum_{\gamma\leq\beta}\binom{\beta}{\gamma}([\partial^{\beta-\gamma}\sqrt{f}]_{\lambda}(x)\sup_{B}|\partial^\gamma \psi_j|+[\partial^{\beta-\gamma}\psi_j]_\lambda(x)\sup_{B}|\partial^\gamma \sqrt{f}|).
\]
Employing the derivative estimates on \(\psi_j\) and \(\sqrt{f}\) on \(B\) from \Cref{thm:party} and \Cref{lem:local1} as above, and using slow variation of \(r\) on \(B\) to see that \(cr_j\leq r(x)\leq Cr_j\), we get
\[
    [\partial^\beta(\psi_j\sqrt{f})]_\lambda(x)\leq C\sum_{\gamma\leq\beta}(r(x)^{\Lambda-|\beta|+|\gamma|}r_j^{-|\gamma|}+r_j^{|\gamma|-|\beta|-\lambda}r(x)^{\frac{k+\alpha}{2}-|\gamma|})\leq Cr(x)^{\Lambda-|\beta|}.
\]
On the other hand, if \(x\not\in B\) then \([\partial^\beta(\psi_j\sqrt{f})]_\lambda(x)\) is identically zero, and the estimate holds trivially. Therefore \textit{(2)} and \textit{(3)} hold throughout \(\mathbb{R}^n\). The proofs of items \textit{(4)} through \textit{(6)} are virtually identical, using the estimates from \Cref{lem:nice} in place of those from \Cref{lem:local1}.
\end{proof}


The pointwise semi-norm estimates given in \Cref{lem:rootscors} imply some useful inclusions.

\begin{cor}\label{cor:theniceone}
The functions \(\psi_j\sqrt{f}\) and \(\psi_j\sqrt{f-F}\) defined above belong to \(C^\frac{k+\alpha}{2}(\mathbb{R}^n)\).
\end{cor}


Finally, with the preceding constructions in place, we are equipped to prove the main result.


\section{Proof of \texorpdfstring{\Cref{thm:c3main}}{}}\label{sec:c3proof}


Arguing by induction on \(n\), we show that if \(k=2\) or \(k=3\) then identity \eqref{eq:decompident} holds for every non-negative \(f\in C^{k,\alpha}(\mathbb{R}^{n})\). Indeed, we also show that the functions \(g_1,\dots,g_{m_n}\) satisfy the following set of pointwise estimates:
\begin{itemize}
    \item[\textit{(1)}] \(|\partial^\beta g_j(x)|\leq Cr_f(x)^{\frac{k+\alpha}{2}-|\beta|}\),
    \item[\textit{(2)}] \([\partial^\beta g_j]_{\frac{\alpha}{2},\mathbb{R}^{n-1}}(x)\leq Cr_f(x)^{\frac{k}{2}-|\beta|}\) if \(k\) is even,
    \item[\textit{(3)}] \([\partial^\beta g_j]_{\frac{1+\alpha}{2},\mathbb{R}^{n-1}}(x)\leq Cr_f(x)^{\frac{k-1}{2}-|\beta|}\) if \(k\) is odd.
\end{itemize}
The function \(r_f\) is as in \eqref{eq:controlfunc}, and we now include the subscript to emphasize dependence on \(f\). 


Like we did in \Cref{sec:reg}, we prove all estimates for arbitrary integer-valued \(k\geq 0\) and we do not restrict to \(k=2\) and \(k=3\), since the argument that follows is valid whenever inequality \eqref{eq:needed} holds pointwise. The generality of this argument saves us from repeating a virtually identical proof for the separate but similar cases when \(k=2\) and \(k=3\). 


\subsection*{\normalsize Base Case in One Dimension}


For a base case of \Cref{thm:c3main} in one dimension, we show that every non-negative \(C^{k,\alpha}(\mathbb{R})\) function can be decomposed as a sum of half-regular squares for every \(k\geq 0\) and \(0<\alpha\leq 1\). Such a result is already proved in \cite{Bony2}, and for completeness we include Bony's result as \Cref{thm:onedim}. However, the claim appearing in \cite{Bony2} omits the derivative and semi-norm estimates that we require for our high dimensional inductive argument, so we furnish a complete proof together with the required estimates. We emphasize that this one-dimensional result is not new, and we do not show that \(m_1=2\) as Bony does, however our proof avoids some of the difficult technical algebraic arguments in \cite{Bony2} and thus it is considerably shorter.


Fixing a non-negative function \(f\in C^{k,\alpha}(\mathbb{R})\) and using \Cref{thm:party}, we construct a partition of unity using the control function \(r_f\) define from \eqref{eq:controlfunc}, so that we may write \(f\) as follows,
\[
    f=\sum_{j=1}^\infty\psi_j^2f.
\]
For a sufficiently small parameter \(\nu\), each function \(\psi_j\) in the sum above is supported on the an interval of the form \(B(x_j,\nu r_j)=(x_j-\nu r_j,x_j+\nu r_j)\). Fixing \(j\in\mathbb{N}\), we observe that at the center of this interval, either \(f(x_j)\geq \omega\nu r_j^{k+\alpha}\) or the converse \(f(x_j)<\omega\nu r_j^{k+\alpha}\) is true.


In the first case, \Cref{lem:rootscors} shows that \(\psi_j\sqrt{f}\) is a \(C^\frac{k+\alpha}{2}(\mathbb{R})\) function that satisfies the pointwise inequalities \textit{(1)} through \textit{(3)} above. On the other hand, if \(f(x_j)<\omega\nu r_j^{k\alpha}\) then \Cref{lem:minia} gives a unique local minimum \(X_j\) of \(f\) near \(x_j\) for which the number \(F_j=f(X_j)\) satisfies 
\[
    \psi_j\sqrt{f-F_j}\in C^\frac{k+\alpha}{2}(\mathbb{R}),
\] 
once again by \Cref{lem:rootscors}. Moreover, \Cref{lem:rootscors} shows that \(\psi_j\sqrt{f-F_j}\) satisfies the required pointwise inequalities \textit{(1)} through \textit{(3)}. In the case that \(f(x_j)<\omega\nu r_j^{k\alpha}\) then, we see that \(f\) can be decomposed into a sum of two squares on the interval \(B(x_j,\nu r_j)\),
\[
    \psi_j^2f=(\psi_j\sqrt{f-F_j})^2+(\psi_j\sqrt{F_j})^2.
\]


To proceed with our decomposition in one dimension, we must show that \(\psi_j\sqrt{F_j}\) is a \(C^\frac{k+\alpha}{2}(\mathbb{R})\) function which satisfies the required pointwise estimates. This turns out to be rather straightforward, since \(F_j\) is a non-negative constant. By slow variation of \(r_f\) we have \(F_j\leq Cr_j^{k+\alpha}\), and it follows by item \textit{(3)} of \Cref{thm:party} that for \(x\in B(x_j,\nu r(x_j))\) the following estimate holds,
\[
    |\partial^\beta[\psi_j\sqrt{F_j}](x)|=\sqrt{F_j}|\partial^\beta\psi_j(x)|\leq Cr_j^{\frac{k+\alpha}{2}-|\beta|}\leq Cr(x)^{\frac{k+\alpha}{2}-|\beta|}.
\]
This bound trivially holds outside the support of \(\psi_j\), meaning it holds everywhere as required. 


Next we prove that \(\psi_j\sqrt{F_j}\) satisfies the semi-norm estimates \textit{(2)} and \textit{(3)}. To deal with the cases in which \(k\) is odd and even simultaneously, we employ the usual tactic defining \(\Lambda\) as integer part of \(\frac{k}{2}\) and setting \(\lambda=\frac{k+\alpha}{2}-\Lambda\), so that it suffices to prove that \([\partial^\beta(\psi_j\sqrt{F})]_\lambda(x)\leq Cr(x)^{\Lambda-|\beta|}\). This follows at once from item \textit{(4)} of \Cref{thm:party} together with slow variation,
\[
    [\partial^\beta(\psi_j\sqrt{F})]_\lambda(x)=\sqrt{F}[\partial^\beta\psi_j]_\lambda(x)\leq Cr_j^\frac{k+\alpha}{2}r_j^{-\lambda-|\beta|}=Cr_j^{\Lambda-|\beta|}\leq Cr(x)^{\Lambda-|\lambda|}.
\]
Consequently, \(\psi_j\sqrt{F}\) satisfies the estimates \textit{(1)} through \textit{(3)} required for our induction. 


Regardless of the behaviour of \(f\) at \(x_j\), we see now that we can write \(\psi_j^2f\) as a sum of at most two squares of functions which satisfy the pointwise estimates \textit{(1)} through \textit{(3)}. After relabelling, we can therefore write
\begin{equation}\label{eq:infsum}
    f=\sum_{j=1}^\infty g_j^2
\end{equation}
for functions \(g_j\in C^\frac{k+\alpha}{2}(\mathbb{R})\). By estimate \textit{(5)} of \Cref{thm:party}, we can identify \(m_1=2N_{1}=30\) sub-collections of functions in the sum above which enjoy pairwise disjoint support. Fix any of these sub-collections, and let its functions be indexed by \(S\subseteq\mathbb{N}\). Observe that the function 
\begin{equation}\label{eq:recombsum}
    g_S=\sum_{j\in S}g_j
\end{equation}
belongs to \(C^\frac{k+\alpha}{2}(\mathbb{R}^n)\) by \Cref{lem:recomb}. Moreover, \(g_S\) satisfies the pointwise estimates \textit{(1)} through \textit{(3)} above owing to the argument employed in proving \Cref{lem:recomb}, and we can also observe that
\[
    g_S^2=\bigg(\sum_{j\in S}g_j\bigg)^2=\sum_{j\in S}g_j^2,
\]
since the functions in the sum have pairwise disjoint supports. Repeating this argument for each of the \(m_1\) sub-collections identified from \eqref{eq:infsum}, and relabelling the recombined functions as \(g_1,\dots,g_{m_1}\), we see that we can write
\[
    f=\sum_{j=1}^{m_1}g_j^2.
\]
Each \(g_j\) above satisfies estimates \textit{(1)} through \textit{(3)} on \(\mathbb{R}^n\), and we conclude that the base case of our induction holds. Specifically, this follows from taking \(k=2\) or \(k=3\).  


Further, since the derivative estimates which require \eqref{eq:needed} are bypassed in one dimension since \(F_j\) is constant, we see that the preceding argument is valid for any \(k\geq 0\), meaning that our argument above also proves most of \Cref{thm:onedim}. The only shortcoming is that our argument above does not recover the optimal constant \(m_1=2\) found in \cite{Bony}.

\medskip

\subsection*{\normalsize Inductive Step in Higher Dimensions}


Now we proceed with the inductive stage of our argument, assuming for our inductive hypothesis that for every non-negative function \(f\in C^{k,\alpha}(\mathbb{R}^{n-1})\) with \(k=2\) or \(k=3\), there exist \(g_1,\dots,g_{m_{n-1}}\) satisfying estimates \textit{(1)}, \textit{(2)} and \textit{(3)} from the beginning of this section for which
\[
    f=\sum_{j=1}^{m_{n-1}}g_j^2.
\]
To this end we fix a non-negative function \(f\in C^{k,\alpha}(\mathbb{R}^n)\). Once again using \Cref{thm:party}, we are able to form the partition of unity induced by \(r_f\) as in \eqref{eq:controlfunc} to write
\[
    f=\sum_{j=1}^\infty\psi_j^2f.
\]


An identical argument to that employed for the base case shows that either \(\psi_j\sqrt{f}\) satisfies the required derivative estimates and belongs to the half-regular H\"older space, or we can write 
\[
    \psi_j^2f=(\psi_j\sqrt{f-F_j})^2+\psi_j^2F_j.
\]
From \Cref{lem:rootscors}, the first function on the right-hand side above satisfies the required estimates \textit{(1)} through \textit{(3)} and belongs to the appropriate H\"older space. 


On the other hand, the remainder function \(F_j\) is non-negative and by \Cref{lem:crudeext} it can be extended from the support of \(\psi_j\) to a function in \(C^{k,\alpha}(\mathbb{R}^{n-1})\) which agrees with \(F_j\) on this support set. For the latter reason, we identify \(F_j\) with its extension to \(\mathbb{R}^{n-1}\), and from our inductive hypothesis, it follows that can decompose \(F_j\) as follows,
\begin{equation}\label{eq:inductivedecomps}
    F_j=\sum_{\ell=1}^{m_{n-1}}g_\ell^2.
\end{equation}
By hypothesis, the functions \(g_1,\dots,g_{m_{n-1}}\) above each satisfy the pointwise estimates \textit{(1)}, \textit{(2)} and \textit{(3)} with \(r_f\) replaced by \(r_{F_j}\). Further, the differential inequalities satisfied by \(F_j\) which we proved in \Cref{lem:implicitest} continue to hold for the extension of \(F_j\) on the support of \(\psi_j\). So we have for \(x\) in the support of \(\psi_j\) that
\[
    r_F(x')=\max\{F(x')^\frac{1}{k+\alpha},\sup_{|\xi|=1}[\partial^2_\xi F(x')]_+^\frac{1}{k-2+\alpha}\}\leq Cr_f(x),
\]
meaning that the estimates satisfied by \(g_1,\dots,g_{m_{n-1}}\) in \eqref{eq:inductivedecomps} actually hold when \(r_F\) is replaced by \(r_f\). The argument employed to prove \Cref{lem:rootscors} shows now that \textit{(1)} through \textit{(3)} are also satisfied by \(\psi_jg_\ell\) for each \(\ell=1,\dots,m_{n-1}\) and \(x\in\mathbb{R}^n\). That is,
\begin{itemize}
    \item[\textit{(1)}] \(|\partial^\beta[\psi_j g_\ell](x)|\leq Cr_f(x)^{\frac{k+\alpha}{2}-|\beta|}\),
    \item[\textit{(2)}] \([\partial^\beta(\psi_j g_\ell)]_{\frac{\alpha}{2},\mathbb{R}^{n-1}}(x)\leq Cr_f(x)^{\frac{k}{2}-|\beta|}\) if \(k\) is even,
    \item[\textit{(3)}] \([\partial^\beta (\psi_jg_\ell)]_{\frac{1+\alpha}{2},\mathbb{R}^{n-1}}(x)\leq Cr_f(x)^{\frac{k-1}{2}-|\beta|}\) if \(k\) is odd.
\end{itemize}
It follows from these estimates that \(\psi_j g_\ell\in C^\frac{k+\alpha}{2}(\mathbb{R}^n)\), and thus we can write \(\psi_j^2F_j\) as a sum of at most \(m_{n-1}\) half-regular squares. 


In summary, for each \(j\in\mathbb{N}\) we can write \(\psi_j^2f\) as a sum of \(m_{n-1}+1\) squares in \(C^\frac{k+\alpha}{2}(\mathbb{R}^n)\), and indeed if \(f\) is locally bounded below then only one square is needed. Combining the squares obtained for each \(j\) and relabelling, we see now that we may write
\[
    f=\sum_{j=1}^\infty g_j^2,
\]
for \(g_j\) satisfying \textit{(1)} through \textit{(3)} everywhere. By item \textit{(5)} of \Cref{thm:party}, we may now partition the sum above into \(N_n(m_{n-1}+1)\) sub-collections of functions which enjoy pairwise disjoint supports. Recombining and relabelling these functions exactly as we did in the one-dimensional setting, we see that we can set \(m_n=N_n(m_{n-1}+1)\) to write
\[
    f=\sum_{j=1}^{m_n}g_j^2
\]
for functions \(g_1,\dots,g_{m_n}\) which inherit the required differential inequalities. Finally, since \(f\) was any non-negative \(C^{k,\alpha}(\mathbb{R}^n)\) function for \(k=2\) or \(k=3\) this completes our inductive step, and it follows that \Cref{thm:c3main} holds for every \(n\).\hfill\qedsymbol



\section{Decompositions Over Open Sets}\label{sec:ext}


In the study of partial differential equations, and in many other settings, it is often useful to restrict attention to functions defined on an open set \(\Omega\subset\mathbb{R}^n\). As Bony points out in \cite{Bony2}, the decomposition theorem of Fefferman \& Phong in \cite{Fefferman-Phong} for \(C^{3,1}(\mathbb{R}^n)\) functions can be extended to \(C^{3,1}_{\mathrm{loc}}(\Omega)\), whenever \(\Omega\) is an open subset of \(\mathbb{R}^n\). Recall that \(C^{k,\alpha}_{\mathrm{loc}}(\Omega)\) is the set of functions whose \(k^\mathrm{th}\)-order derivatives are \(\alpha\)-H\"older continuous on compact subsets of \(\Omega\). 


Bony does not give an explicit proof of this claim, but points to a helpful lemma in \cite{Bony} which can be used to obtain the desired extension of Fefferman \& Phong's result. Motivated by Bony's remark, in this section we prove the following generalization of \Cref{thm:c3main}.


\begin{thm}\label{thm:locdecomp}
Let \(f\in C_{\mathrm{loc}}^{k,\alpha}(\Omega)\) be non-negative on an open set \(\Omega\), for \(k\leq 3\) and \(\alpha\leq 1\). Then
\[
    f=\sum_{j=1}^{m_n}g_j^2
\]
for functions \(g_1,\dots, g_{m_n}\in C^{\frac{k+\alpha}{2}}_{\mathrm{loc}}(\Omega)\). Moreover, \(m_n\) depends only on the dimension \(n\).
\end{thm}


To prove this generalization, we require the following modification of \cite[Lemma 2.1]{Bony}, which enables us to extend \(C^{k,\alpha}_{\mathrm{loc}}(\Omega)\) functions by zero to all of \(\mathbb{R}^n\). Since our version differs somewhat from that appearing in \cite{Bony}, and since the original result is stated and proved in French, we furnish a translation in the form of our modified result, along with a proof.


\begin{lem}[Bony]\label{lem:extbyzero}
Let \(p\) be a positive continuous function defined on an open set \(\Omega\subset\mathbb{R}^n\). There exists a function \(\varphi\in C^\infty(\Omega)\) such that \(\varphi>0\) on \(\Omega\) and for every multi-index \(\beta\),
\begin{equation}\label{eq:boundarydecay}
    \lim_{x\rightarrow\partial\Omega}\frac{\partial^\beta \varphi(x)}{p(x)}=0.
\end{equation}
That is, all derivatives of the function \(\varphi\) decay faster than \(p\) approaching the boundary of \(\Omega\).
\end{lem}


\begin{proof}
We define \(\varphi\) in such a way that derivatives of \(\varphi\) decay exponentially approaching the boundary of \(\Omega\), and to do this we require some component functions. For \(x\in\Omega\) define \(\delta(x)\) as the distance from \(x\) to \(\partial\Omega\). Explicitly, 
\[
    \delta(x)=\inf_{y\in\partial\Omega}|x-y|.
\]
Also let \(\psi\) be a non-negative, smooth function compactly supported in the ball of radius \(\frac{1}{3}\) about the origin, and for \(x\in\Omega\) define a function \(q\) as a local minimum of \(p\),
\[
    q(x)=\inf_{y\in B(x,\frac{1}{2}\delta(x))}p(y).
\]
Equipped with these functions, we claim that the following function has the desired properties,
\[
    \varphi(x)=\int_{\Omega}\frac{q(z)}{\delta(z)^n}\psi\bigg(\frac{x-z}{\delta(z)}\bigg)e^{-\frac{1}{\delta(z)}}dz.
\]


First, observe that the integrand may only be positive if \(x\in B(z,\frac{1}{3}\delta(z))\). If this inclusion holds then \(\delta(x)>\frac{2}{3}\delta(z)\), and the integrand above is nonzero only when \(|x-z|<\frac{1}{3}\delta(z)<\frac{1}{2}\delta(x)\), meaning that the integrand is supported on the ball \(B(x,\frac{1}{2}\delta(x))\subset\Omega\). Thus for \(x\in\Omega\),
\[
    \varphi(x)=\int_{B(x,\frac{1}{2}\delta(x))}\frac{q(z)}{\delta(z)^n}\psi\bigg(\frac{x-z}{\delta(z)}\bigg)e^{-\frac{1}{\delta(z)}}dz.
\]
Additionally, if \(x\not\in B(z,\frac{1}{2}\delta(z))\) then \(|x-z|\geq \frac{1}{2}\delta(z)\) and \(x\not\in B(z,\frac{1}{3}\delta(z))\), meaning that the integrand vanishes. Consequently on the domain of integration we have \(q(z)\leq p(x)\) and
\[
    |\partial^\beta\varphi(x)|=\bigg|\int_{\Omega}\frac{q(z)}{\delta(z)^n}\partial^\beta_x\psi\bigg(\frac{x-z}{\delta(z)}\bigg)e^{-\frac{1}{\delta(z)}}dz\bigg|=\bigg|\int_{B(x,\frac{1}{2}\delta(x))}\frac{q(z)}{\delta(z)^n}\partial^\beta_x\psi\bigg(\frac{x-z}{\delta(z)}\bigg)e^{-\frac{1}{\delta(z)}}dz\bigg|.
\]
Since \(\psi\in C_0^\infty(\mathbb{R}^n)\) by assumption, for each \(\beta\) we have \(|\partial^\beta\psi|\leq C_\beta\) uniformly, and it follows that
\[
    \bigg|\partial^\beta_x\psi\bigg(\frac{x-z}{\delta(z)}\bigg)\bigg|\leq \frac{C_\beta}{\delta(z)^{|\beta|}}.
\]
Employing this estimate together with our bound on \(q\), and using that \(\frac{1}{2}\delta(x)\leq \delta(z)\leq \frac{3}{2}\delta(x)\) for \(z\in B(x,\frac{1}{2}\delta(x))\), we see that
\[
    |\partial^\beta\varphi(x)|\leq C_\beta p(x)\int_{B(x,\frac{1}{2}\delta(x))}\frac{e^{-\frac{1}{\delta(z)}}}{\delta(z)^{n+|\beta|}}dz\leq  2^{n+|\beta|}C_\beta p(x)\frac{e^{-\frac{2}{3\delta(x)}}}{\delta(x)^{n+|\beta|}}\int_{B(x,\frac{1}{2}\delta(x))}dz.
\]
The last integral is the volume of the ball \(B(x,\frac{1}{2}\delta(x))\), which is proportional to \(\delta(x)^n\). It follows that for \(x\in\Omega\) we have
\begin{equation}\label{eq:nicebound}
    |\partial^\beta\varphi(x)|\leq  C p(x)\frac{e^{-\frac{2}{3\delta(x)}}}{\delta(x)^{|\beta|}}
\end{equation}
for \(C\) depending only on \(n\), \(\beta\), and \(\psi\). We see now that the claimed decay result at the boundary holds, since for any multi-index \(\beta\) we have
\[
    \lim_{\delta(x)\rightarrow0^+}\frac{|\partial^\beta\varphi(x)|}{p(x)}\leq C\lim_{\delta(x)\rightarrow0^+}\frac{e^{-\frac{2}{3\delta(x)}}}{\delta(x)^{|\beta|}}=0.
\]
Finally, we note that \(q\) is strictly positive on \(B(x,\frac{1}{2}\delta(x))\), as are the remaining terms of the integrand when \(\delta(x)>0\), meaning that \(\varphi>0\) in \(\Omega\).
\end{proof}


It is a straightforward consequence of the Mean Value Theorem that for \(\varphi\) as above, the following limit also holds for any positive \(\alpha\) with \(0<\alpha\leq 1\),
\begin{equation}\label{eq:alphadecay}
    \lim_{x\rightarrow\partial\Omega}\frac{[\partial^\beta \varphi]_\alpha(x)}{p(x)}=0.
\end{equation}
Additionally, we note that for any multi-index \(\beta\) the derivative \(\partial^\beta\varphi/p\) is bounded on \(\Omega\) by \eqref{eq:nicebound}. Equipped with these results, we now establish our sum of squares theorem for \(C_{\mathrm{loc}}^{k,\alpha}(\Omega)\) functions.


\begin{proof}[Proof of \Cref{thm:locdecomp}]
Given a non-negative function \(f\in C^{k,\alpha}_{\mathrm{loc}}(\Omega)\) defined on an open set \(\Omega\) in \(\mathbb{R}^n\), we can now show that \(f\) is decomposable as a sum of half-regular squares. To do this we make a selection for \(p\) that is similar to that made by Bony in \cite{Bony},
\[
    p(x)=\bigg(1+\sum_{|\beta|\leq k}|\partial^\beta f(x)|+\sum_{|\beta|\leq k}[\partial^\beta f]_\alpha(x)\bigg)^{-1}.
\]
From this choice of \(p\), it follows that \(|\partial^\beta f(x)|\leq p(x)^{-1}\) and \([\partial^\beta f]_\alpha(x)\leq p(x)^{-1}\) for every \(x\in\Omega\) and \(\beta\) such that \(|\beta|\leq k\). Note that \(p\) is both positive and continuous in \(\Omega\) since \(f\in C^{k,\alpha}_{\mathrm{loc}}(\Omega)\) by assumption, so it satisfies the hypotheses of \Cref{lem:extbyzero} and we can find a function \(\varphi\in C^\infty(\Omega)\) that is positive on \(\Omega\) and which satisfies \eqref{eq:boundarydecay}. 


Next we show that \(\varphi f\in C^{k,\alpha}(\mathbb{R}^n)\). By the form of the sub-product rule given in \Cref{lem:sharpsubprod}, we have the pointwise estimate
\[
    [\partial^\beta(\varphi f)]_\alpha(x)\leq \sum_{\gamma\leq\beta }\binom{\beta}{\gamma}([\partial^{\beta-\gamma} f]_\alpha(x)|\partial^\gamma\varphi(x)|+[\partial^\gamma \varphi]_\alpha(x)|\partial^{\beta-\gamma}f(x)|).
\]
The norms and semi-norms containing \(f\) above are all bounded above by \(p(x)^{-1}\), so we have
\[
    [\partial^\beta(\varphi f)]_\alpha(x)\leq \sum_{\gamma\leq\beta}\binom{\beta}{\gamma}\bigg(\frac{|\partial^\gamma\varphi(x)|}{p(x)}+\frac{[\partial^\gamma \varphi]_\alpha(x)}{p(x)}\bigg).
\]
It follows that \([\partial^\beta\varphi f]_\alpha(x)\) is bounded on \(\Omega\) and decays to zero at the boundary whenever \(|\beta|=k\) thanks to \eqref{eq:nicebound} and \eqref{eq:alphadecay}, meaning that \([\partial^\beta(\varphi f)]_{\alpha,\Omega}<\infty\). Extending \(\varphi f\) by zero outside of \(\Omega\), we obtain a function (which we also call \(\varphi f\)) that belongs to \(C^{k,\alpha}(\mathbb{R}^n)\). By \Cref{thm:c3main} it follows that we can write
\[
    \varphi f=\sum_{j=1}^{m_n} g_j^2
\] 
for \(g_1,\dots,g_{m_n}\in C^\frac{k+\alpha}{2}(\mathbb{R}^n)\). Since \(\varphi\) is smooth and bounded below on any compact subset \(K\) of \(\Omega\), we find that on such sets we have for each \(j=1,\dots,m_n\) that
\[
    \tilde{g}_j=\frac{g_j}{\sqrt{\varphi}}\in C^\frac{k+\alpha}{2}(K),
\]
and since \(K\) is any compact subset we see in turn that \(\tilde{g}_j\in C^\frac{k+\alpha}{2}_{\mathrm{loc}}(\Omega)\). Since we are able to write
\[
    f=\sum_{j=1}^m \tilde{g_j}^2,
\]
we see that \(f\) is a sum of half-regular squares on compact subsets of \(\Omega\), as claimed.
\end{proof}


\section{Bounds on Decomposition Size}\label{sec:sizebounds}


The number of squares needed in a sum of squares decomposition, which we denote by \(m_n\), is finite for non-negative \(C^{k,\alpha}(\mathbb{R}^n)\) functions when \(k\leq 3\) thanks to Theorem \ref{thm:c3main}. Indeed, we can find an upper bound for \(m_n\) that depends only on the dimension \(n\). If \(n=1\), Bony shows in \cite{Bony2} that \(m_1=2\) is optimal. In higher dimensions, our methods require the use of more squares, owing to the inductive nature of our argument and the growth of the bounded overlap constant \(N_n\) in \Cref{thm:party} as \(n\) becomes large. In this section, we give upper bounds on \(m_n\).


Recall that in the proof of \Cref{thm:c3main}, we could locally decompose \(\psi_j^2 f\in C^{k,\alpha}(\mathbb{R}^n)\) into a sum of squares of at most \(m_{n-1}+1\) functions, and using item \textit{(5)} of \Cref{thm:party} we were able to recombine to write the infinite sum
\[
    \sum_{j=1}^\infty\psi_j^2
\]
as a finite sum of squares of functions. As such, we can write \(f\) as a sum of squares of at most \(m_n\leq N_n(1+m_{n-1})\) functions, where \(m_1=2\) thanks to the work of Bony in \cite{Bony2}. Solving this recursion is a straightforward task which yields the sequence of estimates
\[
    m_n\leq 2N_n^{n-1}+\frac{N_n^n-N_n}{N_n-1}.
\]
Using the bound \(N_n\leq 15^n\) obtained in the proof of \Cref{thm:party}, which seems to be far from optimal since our construction uses balls instead of dyadic cubes, we obtain the crude estimate
\[
    m_n\leq 2\cdot15^{n^2-n}+\frac{15^{n^2}-15^n}{15^n-1}.
\]


This grows very fast with \(n\), which is unsurprising since there is little evidence to suggest our construction is optimal. A separate approach may require many fewer squares, and little is known of examples requiring a maximal number of squares in general. While two squares are required in one dimension, the number of squares required for general \(n\) is unknown. 


It may be fruitful to investigate the minimal number of squares required to decompose an arbitrary function, rather than focusing exclusively on upper bounds. One possible way to do this could be finding polynomials which can be decomposed into no fewer than a given number of squares; such a technique is employed in the next chapter to produce counterexamples to our main theorem when \(k\) becomes too large, so it is plausible that the behaviour of polynomial sums of squares can afford more information about the sizes of our decompositions.


Though we do not pursue them in this work, there are also  two potential approaches which we can identify that may be able to improve our upper bounds for \(m_n\). First, by using better sets in \Cref{thm:party} (e.g. cubes instead of balls) it is possible to reduce the constant \(N_n\), though in general this number should still increase exponentially with \(n\) since the bounded overlap constant for cubes in \(\mathbb{R}^n\) is roughly \(2^n\).

Second, we note that the coefficient \(m_{n-1}+1\) only enters the recursion for \(m_n\) in the case that \(f\) is very small near a nonzero local minimum; otherwise a single square suffices locally. For functions which only have a small number of nonzero local minima, it seems that we can reduce the need for \(m_{n-1}+1\) functions in a decomposition to a relatively small number of cases thereby improving our bound on \(m_n\).

In any case, there is substantial room for improvement concerning the number of squares need in a regularity preserving decomposition, and the problem of finding an optimal number seems likely to involve some interesting mathematics; our upper bound arguments involve notions from Euclidean geometry and measure theory, while the lower bounds (at least conceptually) use techniques from algebra and the convex geometry induced by polynomial sums of squares which we explore in the next chapter. See \Cref{sec:applics} for further discussion of this problem.
