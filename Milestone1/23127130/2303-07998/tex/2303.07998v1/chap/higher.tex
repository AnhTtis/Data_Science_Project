\chapter{Higher Order Decompositions}\label{chap:higher}


In \Cref{chap:decomps} we studied decompositions in \(C^{k,\alpha}(\mathbb{R}^n)\) for \(k\leq3\), and now we move on to higher-order H\"older spaces. If \(k\geq 4\) then the function \(r\) given by \eqref{eq:controlfunc} contains at least three terms instead of two, and the proof of \Cref{thm:c3main} fails since the local behaviour of \(f\) can no longer be restricted to two cases -- in particular, when \(f\) is not bounded below, we can no longer use the Implicit Function Theorem to extract the local minimizer function \(F\). As Korobenko \& Sawyer point out in \cite{SOS_I}, this causes the inductive technique introduced in \cite{Fefferman-Phong} to fail, and indeed if \(k\geq 4\) then there exist non-negative functions in \(C^{k,\alpha}(\mathbb{R}^n)\) which cannot be written as sums of half-regular squares -- we found such functions in the last chapter. As such, to form regularity preserving sum of squares decompositions of \(C^{k,\alpha}(\mathbb{R}^n)\) for \(k\geq 4\), it is necessary to impose additional hypotheses on \(f\). 


This chapter presents some results concerning partial decompositions; the first shows that a non-negative \(C^{k,\alpha}(\mathbb{R}^n)\) function \(f\) can be decomposed as sums of squares plus a remainder which can be made arbitrarily small (at least in magnitude). Another result (\Cref{thm:seconddmain}) allows us to decompose \(f\) as a sum of squares which retain slightly less than half the regularity of \(f\), but this result is exclusive to functions which satisfy certain differential inequalities. It turns out that in two dimensions at least, we can assume less restrictive inequalities yet still recover half-regular decompositions -- see \Cref{thm:nooooo}.


Little is known about the decompositions of higher order functions in more than one dimension, and our work expands upon results in the known literature, in particular \cite{SOS_I}. However, we fall short of a total characterization of the decomposable functions. We hope that these results may provide deeper insight into the decompositions problem, and perhaps be a starting point for further investigation into a characterization.



\section{Partial Decompositions}


We begin our exploration of higher order decompositions with a qualitative remark; for each fixed \(n\), we should expect the necessary conditions for existence of a half-regular decomposition to become more restrictive as \(n\) and \(k\) become large, owing to the abundance of polynomial counter-examples noted by Blekherman in \cite{Blekherman} and by our constructions in the previous chapter. Similarly, we expect that \(C^{k,\alpha}(\mathbb{R}^n)\) functions that are not sums of half-regular squares to become increasingly common for large \(k\), and this will be reflected by a stronger condition needed to ensure that decompositions exist. 


Before examining some sufficient conditions, we point out that the proof of \Cref{thm:c3main} already affords an interesting and potentially useful decomposition result that requires no extra hypotheses on \(f\).

\newpage


\begin{thm}\label{thm:almostthere}
Let \(f\) be a non-negative \(C^{k,\alpha}_b(\mathbb{R}^n)\) function, and let \(\varepsilon>0\) be given. There exists a non-negative function \(h\in C_b^{k,\alpha}(\mathbb{R}^n)\) such that \(\sup_{\mathbb{R}^n}|h|\leq\varepsilon\) and 
\[
    f=\sum_{j=1}^{m_n}g_j^2+h
\]
for functions \(g_1,\dots,g_{m_n}\in C_b^\frac{k+\alpha}{2}(\mathbb{R}^n)\). Moreover, the constant \(m_n\) depends only on \(n\).
\end{thm}


An equivalent way of rephrasing this result is as follows: if \(f\in C^{k,\alpha}(\mathbb{R}^n)\) is non-negative and has  bounded derivatives, there exists a function \(\Tilde{f}\) which is also in \(C^{k,\alpha}(\mathbb{R}^n)\), \(\Tilde{f}\) is a sum of squares of half regular functions, and \(0\leq \Tilde{f}\leq f\leq \Tilde{f}+\varepsilon\) everywhere. This is a slightly stronger result than one obtains by simply taking a square root \(f+\varepsilon\), since the function \(\Tilde{f}\) is controlled by \(f\) from above. 

%It is also worth noting that an application of \Cref{lem:extbyzero} allows us to extend a given function \(f\) by zero outside of compact sets, and thereby we can drop the assumption of  that \(f\) and its derivatives be uniformly bounded on \(\mathbb{R}^n\). The price we pay for such a generalization is that the functions \(g_1,\dots,g_m\) are only well-behaved on compact subsets of \(\mathbb{R}^n\).


\begin{proof}
Let \(B(x_j,\nu r_j)\) be as in \Cref{thm:party} for small \(\nu\), and recall from the proof of \Cref{thm:c3main} that if \(f(x_j)\geq \omega\nu r_j^{k+\alpha}\) for some \(j\), then 
\[
    \psi_j\sqrt{f}\in C^\frac{k+\alpha}{2}(\mathbb{R}^n).
\]
Combining the functions for which this inclusion holds as in \eqref{eq:recombsum}, we obtain a sum of at most \(m_n=N_n=15^n\) squares \(g_1,\dots,g_{m_n}\) which retain half the regularity of \(f\). Now we examine the remainder \(f-(g_1^2+\cdots+g_{m_n}^2)\). If
\[
    f(x_j)<\omega\nu r_j^{k+\alpha}, 
\]
then from \Cref{cor:supplementary} it follows that \(f(x)\leq \frac{3}{2}\omega\nu r_j^{k+\alpha}\) for \(x\in B(x_j,\nu r_j)\). Since \(r_j=r(x_j)\) is bounded whenever \(f\) and its derivatives are uniformly bounded, we can choose \(\nu\) small enough to ensure that \(f\) is as small as we like on \(B(x_j,\nu r_j)\). Since \(\psi_j\leq 1\) by construction, we see that we may choose \(\nu\) sufficiently small that
\begin{equation}\label{eq:tinnnnny}
    \sup_{\mathbb{R}^n}\psi_j^2f\leq \frac{\varepsilon}{N_n}.
\end{equation}
Now, let \(S\) be the collection of \(j\in\mathbb{N}\) for which \(f(x_j)<\omega\nu r_j^{k+\alpha}\) and define
\[
    h=\sum_{j\in S}\psi_j^2 f.
\]
Owing to our construction of the half-regular functions \(g_1,\dots,g_{m_n}\) above, we are able to write
\[
    f=\sum_{j=1}^{m_n}g_j^2+h.
\]
Since each \(x\in\mathbb{R}^n\) is in at most \(N_n\) of the balls on which the functions \(\psi_j\) are supported, it follows from \eqref{eq:tinnnnny} that \(h\leq \varepsilon\) everywhere. Moreover, since each \(\psi_j\) is smooth and \(f\in C_b^{k,\alpha}(\mathbb{R}^n)\), we conclude by item \textit{(5)} of \Cref{thm:party} together with the recombination \Cref{lem:recomb} that \(h\in C_b^{k,\alpha}(\mathbb{R}^n)\), as claimed.
\end{proof}


This result has two major shortcomings when compared to \Cref{thm:c3main}. First, it requires the assumption that \(\|f\|_{C_b^{k,\alpha}}<\infty\), as opposed to the weaker condition required for \Cref{thm:c3main},
\[
    \sum_{|\beta|=k}[\partial^\beta f]_{\alpha,\mathbb{R}^n}<\infty.
\]
Second, and more importantly, it only shows that \(f\in C_b^{k,\alpha}(\mathbb{R}^n)\) is `almost' decomposable, in the sense that we can find a decomposition that uniformly approximates \(f\). However, \Cref{thm:almostthere} gives no control of the derivatives of the small remainder \(h\), and the examples of the previous chapter show that in general we cannot expect \(\sqrt{h}\) to be half as regular as \(f\). %As such, we would like to find sufficient conditions which give us a result analogous to \Cref{thm:c3main} for large \(k\).


The following useful result of Bony et al. from \cite{Bony} indicates that in one dimension, no additional hypotheses are needed for a regularity preserving decomposition to exist. Recall that we proved this result in \Cref{sec:c3proof}.


\begin{thm}[Bony]\label{thm:onedim}
If \(f\in C^{k,\alpha}(\mathbb{R})\) is non-negative, it is a sum of two squares in \(C^\frac{k+\alpha}{2}(\mathbb{R})\).
\end{thm}


In \cite[Theorem 4.5]{SOS_I}, Sawyer \& Korobenko show that if the even-order derivatives of a non-negative function \(f\in C^{4,\alpha}(\mathbb{R}^n)\) satisfy pointwise inequalities that depend on \(\alpha\), and which resemble the Malgrange inequalities we proved in \Cref{thm:malgrangeineq}, then \(f\) can be decomposed as a sum of squares. However, the decomposition functions do lose some regularity in H\"older scale, meaning that they are not all half as regular as \(f\). Our next result, \Cref{thm:seconddmain}, is a generalization of the result from \cite{SOS_I}. We prove it using an inductive argument, and by assuming some similarly restrictive pointwise properties of \(f\). 


Plainly, this theorem states that we can decompose \(f\) as a sum of squares of H\"older continuous functions, but our decomposition loses regularity on H\"older scale in high dimensions, and moreover the theorem imposes a lot of conditions on \(f\). This is unsurprising, since even the roots of non-negative smooth functions need not inherit more than one derivative -- recall the example from the introduction of this thesis. 


\begin{thm}\label{thm:seconddmain}
Let \(f\) be a non-negative function in \(C^{k,\alpha}(\mathbb{R}^n)\) for \(n\geq2\), \(k\geq 4\), and \(0<\alpha\leq 1\). Assume that the following inequalities hold pointwise on \(\mathbb{R}^n\) for a constant \(\eta<\frac{k-2+\alpha}{k+\alpha}\) and for every even \(\ell\) such that \(2<\ell\leq k\),
\begin{equation}\label{eq:induc}
    |\nabla^2 f(x)|\leq Cf(x)^\eta\qquad \textrm{and}\qquad|\nabla^\ell f(x)|\leq Cf(x)^\frac{k-\ell+\alpha}{k+\alpha}.
\end{equation}
Then there exists a positive number \(\alpha_n<\alpha\) and functions \(g_1,\dots,g_{m_n}\in C^\frac{k+\alpha_n}{2}(\mathbb{R}^n)\) such that
\[
    f=\sum_{j=1}^{m_n} g_j^2.
\]
Moreover, \(\alpha_n\) is given by taking \(\alpha_2=\alpha\) and recursively defining \(\alpha_{j+1}=\frac{k\eta}{k-2+\alpha_j-\eta}\).
\end{thm}

\textit{Remark}: If \eqref{eq:induc} holds for \(\eta\geq \frac{k-2+\alpha}{k+\alpha}\) then \(f(x)\geq cr(x)^{k+\alpha}\) everywhere and \(\sqrt{f}\in C^\frac{k+\alpha}{2}(\mathbb{R}^n)\) by \Cref{lem:local1}. Thus, this result only yields new information if \(\eta\) is small enough. It is unclear whether the inequalities in \eqref{eq:induc} are sharp in general, in the sense that any weaker set of inequalities can be satisfied by a function which is non-decomposable in the sense described above. It is shown in \cite{SOS_I} that when \(k=4\) the inequalities of \eqref{eq:induc} are sharp for a class of functions which Sawyer \& Korobenko call `H\"older monotone.' In any case, it is not clear if \eqref{eq:induc} can be strengthened to ensure that there exists a decomposition that sacrifices no regularity on H\"older scale, and which is not simply comprised of \(\sqrt{f}\) on its own. 


\begin{proof}
As in the proof of \Cref{thm:c3main}, we proceed using induction on \(n\). The base case \(n=1\) is covered by \Cref{thm:onedim}, in which the differential inequalities \eqref{eq:induc} are not needed. This leaves us to deal with the inductive step. 


Thus, we assume now that every non-negative function in \(C^{k,\alpha}(\mathbb{R}^{n-1})\) which satisfies \eqref{eq:induc} can be decomposed as a sum of squares in the space \(C^\frac{k+\alpha_{n-1}}{2}(\mathbb{R}^{n-1})\), and we let \(f\in C^{k,\alpha}(\mathbb{R}^n)\) satisfy \eqref{eq:induc}. It follows that
\begin{equation}\label{eq:equiv27}
    r_f(x)=\max_{\substack{0\leq j\leq k,\\j\;\mathrm{even}}}\bigg\{\sup_{|\xi|=1}[\partial^j_\xi f(x)]_+^\frac{1}{k-j+\alpha}\bigg\}\leq C\max\bigg\{f(y)^\frac{1}{k+\alpha},\sup_{|\xi|=1}[\partial^2_\xi f(y)]_+^\frac{1}{k-2+\alpha}\bigg\}.
\end{equation}
Via \Cref{thm:party} the slowly-varying function \(r_f\) induces a partition of unity; as usual, we fix one of the balls \(B(x_j,\nu r_j)\) on which a partition function \(\psi_j\) is supported, and we consider the behaviour of \(f\) at its center. If \(f(x_j)\geq \omega\nu r_j^{k+\alpha}\), where \(r_j=r_f(x_j)\), then \(f\) is bounded below and \Cref{lem:rootscors} shows that 
\[
    \psi_j\sqrt{f}\in C^\frac{k+\alpha}{2}(\mathbb{R}^n).
\]
On the other hand, if \(f(x_j)< \omega\nu r_j^{k+\alpha}\) then \eqref{eq:equiv27} implies that
\[
    f(x_j)< C\omega\nu \max\bigg\{f(x_j),\sup_{|\xi|=1}[\partial^2_\xi f(x_j)]_+^\frac{k+\alpha}{k-2+\alpha}\bigg\},
\]
and for \(\nu\) small enough (recall that \(\nu\) may be chosen as small as we like) this yields a contradiction if the maximum above equals \(f(x_j)\). It follows that 
\[
    \sup_{|\xi|=1}[\partial^2_\xi f(x_j)]_+^\frac{1}{k-2+\alpha}\leq r_j\leq C\sup_{|\xi|=1}[\partial^2_\xi f(x_j)]_+^\frac{1}{k-2+\alpha},
\]
and moreover since \(r_j>0\) it follows that \(\sup_{|\xi|=1}\partial^2_\xi f(x_j)>0\). Thus the argument employed to prove \Cref{thm:c3main} allows us to locally define a non-negative function \(F_j\) for which 
\[
    \psi_j\sqrt{f-F_j}\in C^\frac{k+\alpha}{2}(\mathbb{R}^n).
\]

To apply the inductive hypothesis and complete our decomposition of \(f\), we must show that the remainder function \(F_j\) satisfies the differential inequalities \eqref{eq:induc} in \(n-1\) dimensions. If we consider the control function for \(F_j\) defined by
\[
    r_{F_j}(x')=\max_{\substack{0\leq j\leq k,\\j\;\mathrm{even}}}\bigg\{\sup_{|\xi|=1}[\partial^j_\xi F_j(x')]_+^\frac{1}{k-j+\alpha}\bigg\},
\]
then the estimates of \Cref{lem:implicitest} show that \(r_{F_j}(x')\leq Cr_f(x)\), and since \(r_f\) is slowly varying on \(B(x_j,\nu r_j)\), we find that \(r_f(x)\leq Cr_f(x',X(x'))\). Combining these estimates with the first inequality of \eqref{eq:induc}, we find altogether that
\[
    r_F(x')\leq Cr_f(x',X(x'))\leq C\sup_{|\xi|=1}[\partial^2_\xi f(x',X(x'))]_+^\frac{1}{k-2+\alpha}\leq Cf(x',X(x'))^\frac{\eta}{k-2+\alpha}=CF_j(x')^\frac{\eta}{k-2+\alpha}.
\]
In particular, this implies that \(|\nabla^2 F_j(x')|\leq C r_{F_j}(x')^{k-2+\alpha}\leq CF_j(x')^\eta\), showing that \(F_j\) inherits the first differential inequality of \eqref{eq:induc} from \(f\). Additionally, for even \(\ell>2\) we see that
\[
    |\nabla ^\ell F_j(x')|\leq r_{F_j}(x)^{k-\ell+\alpha}\leq CF_j(x')^\frac{\eta(k-\ell+\alpha)}{k-2+\alpha}.
\]


Now, observe that if \(f(x_j)< \omega\nu r_j^{k+\alpha}\), as we are assuming, then \(f(x_j)\leq Cf(x_j)^\eta\) and \(f(x_j)^{1-\eta}\leq C\). After rescaling if necessary, we can assume without loss of generality that \(F_j(x')\leq f(x)\leq 1\) where \(F_j\) is defined. Defining \(\tilde{\alpha}=\frac{k\eta}{k-2+\alpha-\eta}\) so that \(\frac{\eta}{k-2+\alpha}=\frac{\tilde{\alpha}}{k+\tilde{\alpha}}\), we see that \(\tilde{\alpha}<\alpha\) when \(\eta<\frac{k-2+\alpha}{k+\alpha}\) and
\[
    \frac{\eta(k-\ell+\alpha)}{k-2+\alpha}=\frac{k-\ell+\alpha}{k+\tilde{\alpha}}\geq \frac{k-\ell+\tilde{{\alpha}} }{k+\tilde{\alpha}}.
\]
Thus for every even \(\ell \) with \(2<\ell\leq k\) we have that \(|\nabla ^\ell F_j(x')|\leq CF_j(x')^\frac{k-\ell+\tilde{\alpha} }{k+\tilde{\alpha}}\). Moreover an identical set of inequalities is satisfied by the extension of \(F_j\) to all of \(\mathbb{R}^{n-1}\) via \Cref{lem:crudeext}, and since \(F_j\in C^{k,\alpha}(\mathbb{R}^{n-1})\) and it is compactly supported, it also belongs to \(C^{k,\tilde{\alpha}}(\mathbb{R}^{n-1})\). It follows by our inductive hypothesis we can write \(F_j\) as a sum of squares,
\[
    F_j=\sum_{\ell=1}^{m_{n-1}} g_\ell^2,
\]
where each \(g_\ell\) belongs to \(C^{\frac{k+\tilde{\alpha}_{n-1}}{2}}(\mathbb{R}^{n-1})=C^{\frac{k+\alpha_n}{2}}(\mathbb{R}^{n-1})\). Since \(\psi_j\) is smooth and compactly supported, \(\psi_jg_\ell\) also has compact support and belongs to the same H\"older space. Using item \textit{(5)} of \Cref{thm:party}, we see now that we can combine the locally decomposed functions as in the proof of \Cref{thm:c3main} to write
\[
    f=\sum_{j=1}^{m_n}g_j^2,
\]
where \(g_j\in C^{\frac{k+\alpha_n}{2}}(\mathbb{R}^n)\) for each \(j\) and \(m_n\leq N_n(1+m_{n-1})\), as we wished to show.
\end{proof}


The first inequality in \eqref{eq:induc} is only used to show that \(F_j\) inherits the remaining differential inequalities from \(f\), and since decomposition of functions over \(\mathbb{R}\) require no differential inequalities by \Cref{thm:onedim}, we can omit the first differential inequality in the case \(n=2\). The proof above thus yields the following special case in two dimensions which involves no parameter \(\eta\), and which generalizes \cite[Theorem 4.7]{SOS_I}.


\begin{thm}\label{thm:nooooo}
Let \(f\) be a non-negative function in \(C^{k,\alpha}(\mathbb{R}^2)\) for \(k\geq 4\) and \(0<\alpha\leq 1\). Assume that the following inequalities hold pointwise on \(\mathbb{R}^2\)
\begin{equation}\label{eq:induc2}
    |\nabla^\ell f(x)|\leq Cf(x)^\frac{k-\ell+\alpha}{k+\alpha}.
\end{equation}
Then there exist  functions \(g_1,\dots,g_{m_n}\in C^\frac{k+\alpha}{2}(\mathbb{R}^2)\) such that \( f=g_1^2+\cdots+g_{m_n}^2.\)

\end{thm}

\newpage

\section{The Special Case of \texorpdfstring{\(C^{4,\alpha}(\mathbb{R}^2)\)}{}}\label{sec:maybe}


The result of Hilbert in \cite{Hilbert} shows that every non-negative polynomial of degree four over \(\mathbb{R}^2\) can be written as a sum of squares of polynomials, meaning that the techniques of \Cref{chap:poly} give no example of a function in \(C^{4,\alpha}(\mathbb{R}^2)\) which is not a sum of half-regular squares. While we do not claim that every function in this space is not a sum of half-regular squares, there is evidence to suggest that \Cref{thm:nooooo} holds with considerably weaker inequalities than \eqref{eq:induc2} in this setting. Indeed, in proving that a non-negative function \(f\in C^{4}(\mathbb{R}^2)\) is a sum of squares of \(C^{2}(\mathbb{R}^2)\) functions in \cite{Bony}, Bony requires only that the fourth derivatives of \(f\) vanish whenever \(f\) and its second derivatives simultaneously vanish. 

Thus, it may be the case that \(C^{4,\alpha}(\mathbb{R}^n)\) functions can be decomposed under weaker assumptions than those we used in the previous section. Our workings in this thesis do not indicate what such conditions look like, and this special case awaits further examination.