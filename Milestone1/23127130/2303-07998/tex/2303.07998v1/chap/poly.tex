\chapter{Non-Decomposable Functions}\label{chap:poly}


In the previous chapter, we found that non-negative functions in \(C^{k,\alpha}(\mathbb{R}^n)\) can be decomposed into sums of half-regular squares when \(k\leq 3\). Now we show that if \(k\geq 4\) and \(\alpha>0\), then there exist non-negative functions in \(C^{k,\alpha}(\mathbb{R}^n)\) which cannot be decomposed into sums of half-regular squares. In doing so we show that \Cref{thm:c3main} is essentially optimal, since the decomposition it provides can fail if we do not restrict to the H\"older spaces \(C^{2,\alpha}(\mathbb{R}^n)\) and \(C^{3,\alpha}(\mathbb{R}^n)\). 


Bony gives an example of a non-decomposable function in \cite{Bony}, and several more appear in \cite{SOS_I}. Each of these is constructed using non-negative polynomials which cannot be written as sums of squares of polynomials. The first example of such a polynomial to appear in the literature is the Motzkin polynomial from \cite{Motzkin}, which we examine in detail in \Cref{sec:nd},
\[
    M(x,y)=x^4y^2+x^2y^4-3x^2y^2+1.
\]

Non-negative polynomials of even degree \(d\) over \(n\) variables which are not sums of squares are known to exist in many settings, thanks to the work of Hilbert in \cite{Hilbert}, but Hilbert's existence result is not constructive. Some examples of non-decomposable polynomials are exhibited by Reznik \cite{reznik_extpsd}, Robinson \cite{robinson}, and Choi \& Lam \cite{CL2}, but cumulatively these examples only cover a handful of cases in which \(n\) and \(d\) are both small. Our contribution is to devise a procedure for constructing examples in the cases not covered in the literature; see \Cref{sec:gen}.


This chapter serves two purposes. First, we clarify the connection between polynomials which are not sums of squares and \(C^{k,\alpha}(\mathbb{R}^n)\) functions which cannot be decomposed into half regular squares; indeed, we show that the former functions are examples of the latter type. Second, we present new examples of polynomials which are not sums of squares, and a technique for constructing them. This gives many functions which violate \Cref{thm:c3main} when \(k\geq 4\).


\section{Non-Decomposable Polynomials}\label{sec:nd}


Hilbert showed in \cite{Hilbert} and \cite{Hilbert2} that for every even degree \(d\geq 4\) and in every dimension \(n\geq 2\), there exist non-negative polynomials of degree \(d\) over \(n\) variables which cannot be written as sums of squares of polynomials, except when \(d=4\) and \(n=2\), in which case non-negative polynomials can be written as sums of three squares.   


\medskip
\begin{center}
Summary of Hilbert's Results in \cite{Hilbert} and \cite{Hilbert2}.\\ Is every non-negative polynomial of degree \(d\) over \(n\) variables a sum of squares of polynomials?\\
\medskip
\begin{tabular}{|c|c|c|c|c|}
\hline
\diagbox[height=2em,width=3em]{\(n\)}{\(d\)} & \parbox[c]{2cm}{\hfil 2} & \parbox[c]{2cm}{\hfil 4}& \parbox[c]{2cm}{\hfil \(\geq 6\)}\\
\hline
1 & Yes&Yes&Yes\\
2 & Yes&Yes&No\\
\(\geq 3\) & Yes&No&No\\
\hline
\end{tabular}
\end{center}

 
Later, Artin confirmed in \cite{Artin} that all non-negative polynomials can be decomposed into sums of squares of rational functions, resolving Hibert's \(17^{\mathrm{th}}\) problem. Non-decomposable polynomials are hardly an anomaly; Blekherman proved in \cite{Blekherman} that for each fixed \(n\), degree \(d\) polynomials that are not sums of squares become increasingly common as \(d\) becomes large. Indeed, this frequency is quantified in \cite[Theorems 2.1 \& 2.2]{Blekherman}. In \Cref{sec:gen} we find that it is easier to produce examples for which \(d\) is large, affirming the observation that non-decomposable polynomials are abundant at large degrees.


Our interest in these polynomials is rooted in the following useful fact, which we formalize in \Cref{lem:contpoly} below: if a degree \(d\) polynomial \(P\) is not a sum of squares of polynomials, then it cannot be written as a sum of squares of functions in \(C^\frac{d+\alpha}{2}(\mathbb{R}^n)\) for any \(\alpha>0\). Thus, by finding such a polynomial \(P\) we immediately obtain a \(C^{d,\alpha}(\mathbb{R}^n)\) function which is not a sum of half-regular squares. We prove this using the following technical result, which is modelled after \cite[Lemma 1.3]{Bony} and \cite[Lemma 5.2]{SOS_I}.


\begin{lem}\label{lem:contpoly}
Let \(P\) be a non-negative polynomial of even degree \(k\) which is not a sum of squares of polynomials, let \(\alpha>0\), and let \(\Omega\subset\mathbb{R}^n\) be compact. For every \(m\in\mathbb{N}\) and \(N>0\), there exists a number \(\delta>0\) such that 
\[
    \qquad\sup_{\Omega}\bigg|\sum_{j=1}^mg_j^2-P\bigg|<\delta\qquad\implies\qquad\sum_{j=1}^m\|g_j\|_{C_b^\frac{k+\alpha}{2}(\Omega)}>N.
\]
\end{lem}


\begin{proof}
Fix \(N\) and assume to the contrary that for every \(\delta>0\) there exist \(g_{1,\delta},\dots,g_{m,\delta}\) such that 
\[
    \sum_{j=1}^m\|g_{j,\delta}\|_{C_b^\frac{k+\alpha}{2}(\Omega)}\leq N\qquad\mathrm{and}\qquad\sup_{\Omega}\bigg|\sum_{j=1}^mg_{j,\delta}^2-P\bigg|<\delta.
\]
For \(\ell\in\mathbb{N}\) choose functions \(g_{1,\ell},\dots,g_{m,\ell}\) that satisfy the inequalities above with \(\delta=\frac{1}{\ell}\), and for fixed \(j\) consider the sequence \(\{g_{j,\ell}\}_{\ell}\). This sequence is uniformly bounded in the norm of the half-regular H\"older space since 
\[
    \|g_{j,\ell}\|_{C_b^\frac{k+\alpha}{2}(\Omega )}\leq N
\]
for every \(\ell\). Recall that the embedding of \(C_b^\frac{k+\alpha}{2}(\Omega)\) into \( C_b^\frac{k}{2}(\Omega)\) is compact by \Cref{lem:cptmb}, so there exists a subsequence of \(\{g_{j,\ell}\}_{\ell}\) that converges uniformly to a function \(g_j\) in the target space. Passing to this subsequence, we see that
\[
    \sup_\Omega\bigg|\sum_{j=1}^mg_j^2-P\bigg|=\lim_{\ell\rightarrow\infty}\sup_\Omega\bigg|\sum_{j=1}^mg_{j,\ell}^2-P\bigg|\leq\lim_{\ell\rightarrow\infty}\frac{1}{\ell}=0,
\]
meaning that \(P\) is a sum of squares in \(C_b^\frac{k}{2}(\Omega)\). 


Now we argue that each \(g_j\) is in fact a polynomial. If the highest-order term in the Taylor expansion of any \(g_j\) has degree exceeding \(\frac{k}{2}\), then the sum of squares \(P=g_1^2+\cdots+g_m^2\) would include a monomial term whose degree exceeds \(k\), since the coefficient on the highest order term is a sum of squares of real numbers, hence strictly positive. This contradicts the fact that \(P\) has degree \(k\), meaning that each \(g_j\) is a polynomial of degree at most \(\frac{k}{2}\) and \(P\) is a sum of squares of polynomials. This contradicts our initial assumption on \(P\).
\end{proof}


Equipped with this result, we draw a useful conclusion about non-decomposable polynomials.


\begin{cor}\label{cor:notsos}
Let \(P\) be a non-negative polynomial of even degree \(k\) which is not a sum of squares of polynomials. Then \(P\) is not a sum of squares of functions in \(C^\frac{k+\alpha}{2}(\mathbb{R}^n)\) for \(\alpha>0\).
\end{cor}

\begin{proof}
Assume that there exist functions \(g_1,\dots,g_m\in C^\frac{k+\alpha}{2}(\mathbb{R}^n)\) such that
\[
    P=\sum_{j=1}^m g_j^2.
\]
Then restricting to any compact set \(\Omega\subset\mathbb{R}^n\) we necessarily have for each \(j\) that \(g_j\in C_b^{\frac{k+\alpha}{2}}(\Omega)\). On the other hand, \Cref{lem:contpoly} implies that for any \(N\in\mathbb{R}\),
\[
    \sum_{j=1}^m\|g_j\|_{C_b^\frac{k+\alpha}{2}(\Omega)}>N.
\]
It follows that for some \(j\) one of the norms above is infinite. Since \(g_j\) and its derivatives are bounded, we conclude that for some \(\beta\) whose order is the integer part of \(\frac{k+\alpha}{2}\), we have
\[
    [\partial^\beta g_j]_{\frac{k+\alpha}{2}-|\beta|,\Omega}=\infty.
\]
It follows that a derivative of \(g_j\) is not \(\alpha\)-H\"older continuous on \(\Omega\), meaning that \(g_j\not\in C^\frac{k+\alpha}{2}(\mathbb{R}^n)\). This contradicts our initial assumption, and the claim follows.
\end{proof}


Since the Motzkin polynomial is not a sum of squares of polynomials, as we verify in the next section, we find now that \(C^{6,\alpha}(\mathbb{R}^2)\) contains a non-decomposable function.


\begin{cor}
For any \(\alpha>0\), the Motzkin polynomial \(M(x,y)=x^4y^2+x^2y^2-3x^2y^2+1\) is a non-negative \(C^{6,\alpha}(\mathbb{R}^2)\) function which is not a sum of squares in \(C^{3,\frac{\alpha}{2}}(\mathbb{R}^2)\).  
\end{cor}


It is important to observe that the only properties of the Motzkin polynomial we used for this corollary were non-negativity, and the fact that \(M\) is not a sum of squares. Polynomials on \(\mathbb{R}^n\) with these properties abound at high degrees when \(n\geq 2\), thanks to the results of Hilbert \cite{Hilbert} and Blekherman \cite{Blekherman}. Despite providing no examples, Hilbert's argument is nevertheless sufficient for us to draw a striking conclusion from the sequence of results obtained above. The following sweeping result seems to have been overlooked in many earlier works on sums of squares, save for the cases \(k=4\) and \(k=6\) which are discussed in \cite{Bony} and \cite{SOS_I}.


\begin{cor}\label{cor:existence}
Let \(k\geq 4\) and \(n\geq 2\) and assume that one of these inequalities is strict. For any \(\alpha>0\) there exist non-negative functions in \(C^{k,\alpha}(\mathbb{R}^n)\) which are not sums of half-regular squares.
\end{cor}


\begin{proof}
If \(k\) is even this is immediate from \Cref{cor:notsos} and \cite{Hilbert}. If \(k\) is odd, it suffices to repeat the argument above with a non-negative and non-decomposable polynomial \(P\) of degree \(k-1\), recalling our definition of half-regular H\"older spaces in \eqref{eq:halfreg}. 
\end{proof}


This existence result does the job of verifying that \Cref{thm:c3main} does not hold for \(k\geq 4\), but it is unsatisfying since it gives no concrete examples. For the remainder of this chapter, we work to obtain explicit polynomials which are not sums of squares for each admissible pairing of \(n\) and \(k\) in \Cref{cor:existence}. Through \Cref{cor:notsos}, we can readily conclude that such examples are not decomposable as sums of half-regular squares.


\section{Explicit Polynomial Examples}\label{sec:examples}


To prove that the Motzkin polynomial is not a sum of squares of polynomials, and to construct explicit examples, we introduce some basic concepts and techniques from convex geometry. 


\begin{defn}
Let \(P(x)=c_1 x^{q_1}+\cdots+c_mx^{q_m}\) be a polynomial on \(\mathbb{R}^n\) for constants \(c_j\neq0\) and \(q_j\in\mathbb{N}_0^n\). The Newton polytope \(C_P\) of \(P\) is the convex hull in \(\mathbb{R}^n\) generated by the points \(q_1,\dots,q_m\). That is, \(C_P\) is the smallest convex set in \(\mathbb{R}^n\) which contains each \(\alpha_j\). 
\end{defn}


For example, the Newton polytope of the Motzkin polynomial is the closed triangle \(C_M\) in \(\mathbb{R}^2\) with corners at \((0,0)\), \((2,4)\) and \((4,2)\). Frequently, we employ a more useful realization of \(C_P\) as the set of all convex combinations of the lattice points \(q_1,\dots,q_m\). That is,
\[
    C_P=\bigg\{\sum_{j=1}^m\lambda_jq_j\;:\;\sum_{j=1}^m\lambda_j=1\;\mathrm{and}\;0\leq\lambda_j\leq 1\bigg\}.
\]
Observe that if \(P(0)\neq 0\) then \(C_P\) has a vertex fixed at the origin, and if \(m=n+1\), then the weights \(\lambda_1,\dots,\lambda_{n+1}\) turn out to be uniquely determined. We verify this fact momentarily in \Cref{thm:posweights} before employing it to construct polynomials which are not sums of squares.


Our next result will also be useful in our construction, and it follows as a straightforward consequence of the weighted arithmetic-mean geometric-mean inequality (Theorem \ref{thm:amgm}). Simply put, the following lemma states that if a multi-index \(p\) can be realized as a weighted average of multi-indices \(q_1,\dots,q_m\), each having even entries, then we can construct a non-negative polynomial which contains monomials associated to each \(q_j\) and \(p\). This is a generalized variant of the result used by Motzkin in \cite{Motzkin} to verify non-negativity of the polynomial \(M\).

\begin{lem}\label{lem:amgm}
Let \(q_1,\dots,q_m\in 2\mathbb{N}_0^n\), let \(\lambda_1,\dots,\lambda_m\) be such that \(0\leq \lambda_j\leq 1\) for each \(j\) and \(\lambda_1+\cdots+\lambda_m=1\), and assume that \(p=\lambda_1q_1+\cdots+q_m\alpha_m\in \mathbb{N}_0^n\). Then for every \(x\in\mathbb{R}^n\),
\[
    x^p\leq \sum_{j=1}^m \lambda_jx^{q_j}.
\]
Moreover, equality holds above when \(x^{q_1}=\cdots=x^{q_m}\) and in particular when \(x=(1,\dots,1)\).
\end{lem}


\begin{proof}
Let \(p\) be as above and note that \(x^\beta=x^{\lambda_1q_1}\cdots x^{\lambda_mq_m}\) for any \(x\in\mathbb{R}^n\). Given a point \(x=(x_1,\dots,x_n)\in\mathbb{R}^n\), we set \(\Tilde{x}=(|x_1|,\dots,|x_n|)\) so that by Theorem \ref{thm:amgm},
\[
    x^p= \prod_{j=1}^mx^{\lambda_jq_j}\leq \prod_{j=1}^m(\Tilde{x}^{q_j})^{\lambda_j}\leq \sum_{j=1}^m\lambda_j\Tilde{x}^{q_j}=\sum_{j=1}^m\lambda_jx^{q_j}.
\]
From the equality case of the AMGM inequality, we see that equality holds above when \(x^p=\Tilde{x}^p\) and \(x^{q_1}=\cdots=x^{q_m}\). The first inequality and the last identity above hold since \(\Tilde{x}^{q_j}=x^{q_j}\geq0\) whenever \(q_j\in 2\mathbb{N}_0^n\), and since \(x\in\mathbb{R}^n\) was arbitrary we are finished.
\end{proof}


Using this result, it is straightforward to verify non-negativity of the Motzkin polynomial. The monomial terms of \(M\) correspond to the multi-indices \(q_1=(0,0)\), \(q_2=(4,2)\), \(q_3=(2,4)\), and \(p=(2,2)\). Since we can write \(p=\frac{1}{3}(q_1+q_2+q_3)\), it follows from \Cref{lem:amgm} that for every \(x\in\mathbb{R}^2\) we have \(3x^p\leq x^{q_1}+x^{q_2}+x^{q_3}\). In standard notation this reads \(3x^2y^2\leq 1+x^4y^2+x^2y^4\) or equivalently, \(M(x,y)\geq0\).


In the course of constructing new counterexamples like the Motzkin polynomial, the preceding result allows us to select coefficients which ensure that the polynomial we extract from a well-chosen Newton polytope is non-negative everywhere. In \cite{reznik_extpsd}, Reznik makes an important connection between Newton polytopes and polynomial sums of squares which we summarize in the following lemma. The proof of this result can be found in \cite[Theorem 1]{reznik_extpsd}, and it employs some basic convex geometry which we avoid repeating. 


\begin{lem}[Reznik]\label{lem:rezlem}
Let \(P\) be a polynomial with Newton polytope \(C_P\). If there exist polynomials \(g_1,\dots,g_m\) such that \(P=g_1^2+\cdots+g_m^2\) then 
\[
    \bigcup_{j=1}^m C_{g_j}\subseteq \frac{1}{2}C_P.
\]
Here, \(\frac{1}{2}C_{P}\) denotes the dilated polytope \(\{\frac{1}{2}x:x\in C_{P}\}\).
\end{lem}


The use of this connection between Newton polytopes and polynomial sums of squares is leveraged by Reznik in \cite{reznik_extpsd}, and later the connection is also used extensively by Choi, Lam, and Reznik in \cite{CLR} to study the Pythagoras number of a ring. This is the smallest number of squares needed in the sum of squares decomposition of any decomposable element, and the analog in our study of decompositions of H\"older functions is the number \(m_n\) estimated in \Cref{sec:sizebounds}. 


Employing the useful lemma above, we can argue in a similar fashion to Reznik to show that the Motzkin polynomial is not a sum of squares. This argument is not new, and variants appear in both \cite{reznik_extpsd} and \cite{Powers}, however it is important to understand since it is a foundation of our more general construction.


\begin{cor}
The polynomial \(M(x,y)=x^2y^4+x^4y^2-3x^2y^2+1\) is not a sum of squares.
\end{cor}

\begin{proof}
Assume toward a contradiction that \(M=g_1^2+\cdots+g_m^2\) and note that \(C_P\) is the triangular region in \(\mathbb{R}^2\) with vertices at \((0,0)\), \((4,2)\) and \((2,4)\). By \Cref{lem:rezlem}, for each \(j\) the polytope \(C_{g_j}\) must be a subset of the triangle with vertices at \((0,0)\), \((2,1)\) and \((1,2)\). Since the only lattice points which belong to to this closed triangle are the vertices themselves and \((1,1)\), the square of \(g_j\) must take the form \(g_j^2=(c_1+c_2xy+c_3x^2y+c_4xy^2)^2\), and expanding gives
\[
    g_j^2=c_1^2+c_2^2x^2y^2+c_3^2x^4y^2+c_4^2x^2y^4+2xy(c_1c_2+c_1c_3x+c_1c_4y+c_2c_3x^2y+c_2c_4xy^2+c_3c_4x^2y^2).
\]
The monomial \(x^2y^2\) only appears above in the term \(c_2^2x^2y^2\). It follows that the coefficient of \(x^2y^2\) in \(g_1^2+\cdots+g_m^2\) is a sum of squares, hence non-negative. This contradicts the fact that the coefficient of \(x^2y^2\) in \(M(x,y)\) is negative, and we conclude that \(M\) is not a sum of squares.
\end{proof}




Generalizing the idea of the preceding proof, we establish the following criterion which allows us to determine whether a given Newton polytope may correspond to a polynomial which is not a sum of squares.


\begin{cor}\label{cor:hullcond}
Let \(P\) be a polynomial which contains a monomial \(kx^p\) for which \(k<0\)
and such that \(p\) is not the sum of any two distinct points in \(\frac{1}{2}C_P\). Then \(P\) is not a sum of squares. 
\end{cor}


The conditions above are not necessary for a polynomial not to be a sum of squares. Observe that if a polynomial is not a sum of squares then neither are any of its translations, so
\[
    P(x,y)=M(x+1,y+1)=x^2 y^4+4x^2 y^3+3x^2 y^2+x^4 y^2+2x^4 y+x^4-2x^2 y-2x^2+1
\]
is not a sum of squares. However it is easily verified by direct calculation that every integer lattice point belonging to \(C_P\) can be written as the average of two distinct lattice points in \(C_P\). It follows that the polynomials which satisfy \Cref{cor:hullcond} do not exhaust the collection of non-decomposable polynomials. 


Nevertheless, equipped with \Cref{cor:hullcond} we can construct many polynomials which are not sums of squares. Using \Cref{lem:amgm}, we ensure that these are non-negative and only vanish at a single point. To illustrate how the latter properties are achieved, we observe by \Cref{lem:amgm} that \(M(x,y)\) is non-negative and \(M(x,y)=0\) when \(x^2=y^2=1\). Further, the lemma indicates that \(x^2y^4+1-2xy^2\geq 0\) with equality when \(x=1\) and \(y^2=1\), while \(x^4y^2+1-2x^2y\geq0\) with equality when \(x^2=1\) and \(y=1\). Adding these polynomials to \(M(x,y)\), we see that the following is a non-negative polynomial with a single zero at \((1,1)\),
\[
    P(x,y)=2x^2 y^4+2x^4 y^2-3x^2 y^2-2x y^2-2x^2 y+3
\]
Moreover, \(C_P=C_M\) and \(P\) has a negative coefficient on the critical monomial \(x^2y^2\), so \(P\) is not a sum of squares of polynomials. A translation of \(P\) allows us to place the zero at the origin.


Convex hulls in \(\mathbb{R}^n\) with the special property outlined in \Cref{cor:hullcond} are the Newton polytopes of polynomials which are not sums of squares. In the next section we show how such polytopes can be constructed systematically, and using Lemma \ref{lem:amgm} we extract non-negative polynomials from these sets. Most often, we work with polytopes in \(\mathbb{R}^n\) which have \(n+1\) vertices, since doing so involves especially simple calculations as we outline in the following theorem.


\begin{thm}\label{thm:posweights}
Let \(C\) be a polytope in \(\mathbb{R}^n\) with \(n+1\) lattice point vertices, \(q_1,\dots,q_{n+1}\). If \(p\) is a lattice point in the interior of \(C\), then there exist \(\lambda_1,\dots,\lambda_{n+1}>0\) such that
\begin{equation}\label{eq:conv}
    p=\sum_{j=1}^{n+1}\lambda_jq_j\qquad\textrm{and}\qquad\sum_{j=1}^{n+1}\lambda_j=1.
\end{equation}
Moreover, if \(q_{n+1}\) is fixed at the origin and the vectors \(q_1,\dots,q_n\) are linearly independent then \(\lambda_1,\dots,\lambda_{n+1}\) are uniquely determined and can be found by setting \(Q=[q_1 \cdots q_n]\) and computing
\begin{equation}\label{eq:solving}
    [\lambda_1\cdots\lambda_n]=(Q^{-1}p)^T\qquad\textrm{and}\qquad\lambda_{n+1}=1-\sum_{j=1}^n\lambda_j.
\end{equation}
\end{thm}


\begin{proof}
Since \(C\) is a convex hull which contains \(p\), the identities of \eqref{eq:conv} hold for some set of weights \(\lambda_1,\dots,\lambda_{n+1}\geq 0\). If \(\lambda_j=0\) for some \(j\) then \(p\) must be a convex combination of at most \(n\) points, meaning it belongs to one of the faces of \(C\) and not the interior; this contradicts our assumption that \(p\) is an interior point. 

If \(q_{n+1}=0\) and \(q_{1},\dots,q_n\) are linearly independent then \(C\) has non-empty interior; simply consider the point
\[
    p=\frac{1}{n}\sum_{j=1}^n q_j,
\]
or indeed any weighted average of the vertices \(q_1,\dots,q_n\). The identities of \eqref{eq:solving} are thus obtained by solving the linear system in \eqref{eq:conv}.
\end{proof}

The uniqueness of solutions to \eqref{eq:solving} gives us a simple criterion for determining whether an integer lattice point \(r\in\mathbb{R}^n\) belongs to a given polytope \(C\) with \(n\) linearly independent vertices. This result allows us to efficiently search for the polytopes outlined in \Cref{cor:hullcond}.



\begin{cor}
Let \(C\) be a convex hull in \(\mathbb{R}^n\) with one vertex at the origin and the remaining vertices at linearly independent points \(q_1,\dots,q_n\). Set \(Q=[q_1\;\cdots\;q_n]\), and given \(r\in\mathbb{R}^n\) define \([\lambda_1\;\cdots\;\lambda_n]^T=Q^{-1}r\) and \(\lambda_{n+1}=1-(\lambda_1+\cdots+\lambda_n)\). Then exactly one of the following holds,
\begin{itemize}
    \item[(1)] If \(\lambda_1,\dots,\lambda_{n+1}>0\) then \(r\) is interior to \(C\),
    \item[(2)] If \(\lambda_1,\dots\lambda_{n+1}\geq0\) and \(\lambda_j=0\) for some \(j\) then \(r\) is a boundary point of \(C\),
    \item[(3)] If \(\lambda_j<0\) for some \(j\) then \(r\) is exterior to \(C\).
\end{itemize}
\end{cor}


The result above reduces determination of whether a point lies in \(C\) to a simple matrix-vector multiplication. This makes it straightforward to check whether \(C\) has the property outlined in \Cref{cor:hullcond}, by searching for a lattice point \(p\in C\) which is not the sum of two distinct lattice points in \(\frac{1}{2}C\). If no such point is found, we simply try another polytope with \(n+1\) vertices. 


%First, one can search over the lattice points in a box containing \(C\) and check whether a given point \(p\) is interior to \(C\). If \(p\in C\) then one can perform an identical search over a box containing \(\frac{1}{2}C\), stopping if a point \(t\) belongs to \(\frac{1}{2}C\) and \(t\neq \frac{1}{2}p\). If \(t\in \frac{1}{2}C\) and \(p-t\in\frac{1}{2}C\) then \(p\) can be written as the sum of two distinct lattice points in \(\frac{1}{2}C\), and the search can proceed. If the search of \(\frac{1}{2}C\) completes and no such point \(t\) is found, then \(C\) has the property outlined in \Cref{cor:hullcond}, with \(\alpha=p\).


Finally, we arrive at the critical construction. Given a polytope \(C\) of \(n+1\) vertices, one fixed at the origin and the rest linearly independent, and an interior point \(p\in C\) which is not the sum of two distinct lattice points in \(\frac{1}{2}C\), we use \Cref{thm:posweights} to write \(p=\lambda_1q_1+\cdots+\lambda_nq_n\), where \(\lambda_j>0\) for each \(j\) and \(\lambda_1+\cdots+\lambda_n<1\). It follows from \Cref{lem:amgm} that the following polynomial is non-negative and not a sum of squares,
\begin{equation}\label{eq:shortone}
    P(x)=\sum_{j=1}^n\lambda_jx^{2q_j}+1-\sum_{j=1}^n\lambda_j-x^{2p}.
\end{equation}
To recover integer coefficients, it suffices to multiply this polynomial by \(|\mathrm{det}\;Q|\). Additionally, as we modified the Motzkin polynomial to have a single zero, we can add lower-order terms as follows to construct a polynomial that only has one zero,
\begin{equation}\label{eq:poly}
    Q(x)=\sum_{j=1}^n\lambda_jx^{2q_j}+1-\sum_{j=1}^n\lambda_j-x^{2p}+c\sum_{j=1}^n(x^{2q_j}+1-2x^{q_j}).
\end{equation}
It is easy to see that \(C_Q=C_P\) for any \(c>0\), so \(Q\) is not a sum of squares. Moreover, we have \(Q(1,\dots,1)=0\) and \(Q\) is strictly positive elsewhere by \Cref{lem:amgm}.


For example, one can verify that the Newton polytope \(C\) with vertices at \((0,0)\), \((6,2)\), \((2,4)\) contains the point \((2,2)\) which is not the sum of two distinct points in \(\frac{1}{2}C\), and undergoing the construction above gives the non-negative polynomial
\[
    P(x,y)=x^6y^2+2x^2y^4-5x^2y^2+2,
\]
which is not a sum of squares of polynomials. Moreover, the following modified form is also not a sum of squares, and it is only zero at at the point \((1,1)\),
\[
    Q(x,y)=2 x^6 y^2+3 x^2 y^4-5 x^2 y^2-2 x^3 y-2 x y^2+4.
\]
In summary, to obtain a non-negative polynomial which is not a sum of squares, it suffices to find a polytope \(C\) in \(\mathbb{R}^n\) with \(n+1\) vertices at non-negative integer lattice points in \(\mathbb{R}^n\), one fixed at the origin, which satisfies the hypothesis of \Cref{cor:hullcond}. Given such a polytope, the desired polynomial is provided by \eqref{eq:poly}.


Momentarily we develop a systemic way of constructing these polytopes. First, we present the following algorithm for finding polynomials of even degree \(d\) on \(\mathbb{R}^n\) that are not sums of squares. For shorthand, we let \(v\) denote the column vector in \(\mathbb{R}^n\) in which each entry is one. For a vector \(w\) we write \(w\geq0\) to indicate that \(w\) has non-negative entries, and if \(z\) is another vector we write \(w\geq z\) if each entry in \(w\) exceeds the corresponding entry in \(z\).

\medskip

\noindent\textit{Algorithm For Finding Non-Negative, Non-SOS Polynomials (Direct Search)}
\begin{itemize}
    \item[(1)] Choose \(n\) linearly independent non-negative lattice points in \(q_1,\dots,q_n\in\mathbb{R}^n\) such that \(q_{j,1}+\cdots+q_{j,n}\leq \frac{1}{2}d\), with equality for some \(j\). Set \(Q=[q_1\cdots q_n]\) and compute \(Q^{-1}\).
    \item[(2)] Fix a lattice point \(p\) in the box with corners at the origin and \(2(\max_jq_{j,1},\dots,\max_{j}q_{j,n})\).
    \item[(3)] If \(0\leq Q^{-1}p\) and \(1<v^TQ^{-1}p\leq 2\) then \(p\) is in the candidate hull -- proceed to step (3). If either inequality fails, return to step (2) and choose a new point.
    \item[(4)] Fix a lattice point \(t\) in the box with corners at the origin and \((\max_jq_{j,1},\dots,\max_{j}q_{j,n})\). 
    \begin{itemize}
        \item[(4a)] If \(t\neq p\), \(0\leq Q^{-1}t\leq Q^{-1}p\) and \(v^TQ^{-1}p-1\leq v^TQ^{-1}t\leq 1\), then \(p\) is the average of two distinct points in the candidate hull. Return to step (2) and choose a new point.
        \item[(4b)] If no such \(t\) is found after all points are checked, stop; \(p\) has the desired property.
    \end{itemize}
    \item[(7)] If no viable point \(p\) is found after all possible choices in step (2), return to step (1).
\end{itemize}

\medskip

A successful search gives points \(q_1,\dots,q_n\) and \(p\) which can be used to compute \(\lambda_1,\dots,\lambda_{n+1}\) from \eqref{eq:solving}, and in turn these give the polynomial \eqref{eq:poly} which is non-negative, not a sum of squares, and it is only zero at a single point. Re-scaling by \(|\mathrm{det}\;Q|\) also gives integer coefficients. Using this algorithm, we are able to find many non-negative polynomials which cannot be written as sums of squares. We summarize these examples in the table on the next page, noting that most could not be found in existing literature (we indicate those that were). For brevity, we do not include the terms which enforce a single zero, since these make the expressions quite long.

\newpage
\renewcommand{\arraystretch}{1.3}
\begin{center}
Non-Negative Polynomials Which Are Not Sums of Squares (\(n\leq 4\), \(d\leq 20\)).

\begin{tabular}{|c|c|c|}
    \hline
    Dimension \(n\)  & Degree \(d\) & Polynomial\\
    \hline
    2 & 6 & \(x^4y^2+x^2y^4-3x^2y^2+1\)\;\;(\cite{Motzkin})\\
    2 & 8 & \(x^6y^2+2x^2y^4-5x^2y^2+2\)\\
    2 & 10 & \(x^4y^6+x^2y^6-3x^2y^4+1\)\\
    2 & 12 & \(2x^8y^4+13y^8-16xy^7+1\)\\
    2 & 14 & \(x^4y^{10}+x^2y^2-3x^2y^4+1\)\\
    2 & 16 & \(x^6y^{10}+y^2-3x^2y^4+1\)\\
    2 & 18 & \(x^{14}y^4+x^4y^2-3x^6y^2+1\)\\
    2 & 20 & \(3x^{10}y^{10}+20y^6-30xy^5+7\)\\
    \hline
    3 & 4 & \(x^2y^2+y^2z^2+x^2z^2-4xyz+1\)\;\;(\cite{CL2})\\
    3 & 6 & \(x^4z^2+4x^2z^4+3y^4z^2-12xyz^2+4\)\\
    3 & 8 & \(x^2y^2z^4+x^2y^4z^2+x^4y^2z^2-4x^2y^2z^2+1\)\\
    3 & 10 & \(x^2y^2z^6+x^4y^2z^2+x^2y^4-4x^2y^2z^2+1\)\\
    3 & 12 & \(x^4y^2z^6+x^4y^4z^2+y^2-4x^2y^2z^2+1\)\\
    3 & 14 & \(x^4y^4z^6+x^6y^2+x^6y^2z^2-4x^4y^2z^2+1\)\\
    3 & 16 & \(x^8y^6z^2+y^2+z^6-4x^2y^2z^2+1\)\\
    3 & 18 & \(x^6y^8z^4+z^6+x^2z^6-4x^2y^2z^4+1\)\\
    3 & 20 & \(x^{14}y^2z^4+x^2z^2+y^6z^2-4x^4y^2z^2+1\)\\
    %3 & 22 & \(y^6z^{16}+x^6y^2+x^2-4x^2y^2z^4+1\)\\
    %3 & 26 & \(x^{14}y^6z^6+x^2y^2+z^2-4x^4y^2z^2+1\)\\
    %3 & 28 & \(x^{24}y^6z^4+x^{10}y^2z^{10}+x^6z^{10}-4x^10y^2z^6+1\)\\
    %3 & 72 & \(x^{38}y^{10}z^{24}+y^{22}z^6+x^{10}y^{16}z^{10}-4x^{12}y^{12}z^{10}+1\)\\
    \hline
    4 & 4 & No example provided by our algorithm -- see \Cref{sec:gen}\\
    4 & 6 & \(3x^2z^2w^2+y^2z^4+2y^2w^2+2x^2y^2-10wxyz+2\)\\
    4 & 8 & \(x^2y^2z^4+y^4z^4+x^2y^2w^4+2x^2z^4w^2-8xyz^2w+3\)\\
    4 & 10 & \(x^6y^2z^2+x^2w^8+y^8w^2+x^2z^8-5x^2y^2z^2w^2+1\)\\
    4 & 12 & \(x^6y^4w^2+x^6y^4+z^6w^6+y^4z^4+z^2w^4-5x^2y^2z^2w^2+1\)\\
    4 & 14 & \(x^2y^6z^4w^2+x^4z^4w^4+x^4y^4w^2+z^2w^2-5x^2y^2z^2w^2+1\)\\
    4 & 16 & \(x^8z^4w^4+x^8y^2z^2+x^4y^6x^2+y^2z^2w^6-5x^4y^2z^2w^2+1\)\\
    4 & 18 & \(x^8y^4+x^2y^4z^2w^2+x^4z^8w^6+x^6w^2+x^4y^4z^4w^2-6x^4y^2z^2w^2+1\)\\
    4 & 20 & \(x^{10}y^6w^2+y^6z^{10}+y^6z^6w^8+y^6z^4-x^2y^4z^4w^2+1\)\\
    \hline
\end{tabular}
\end{center}
\bigskip
\setlength\parindent{15pt}

Unfortunately, the algorithm written above slows considerably as \(n\) and \(d\) become large. Moreover, we quickly learned that this crude exhaustive search actually fails to find examples in all cases; the table of examples above does not contain an example when \(n=4\) and \(d=4\). To fill in these `gaps,' and to find examples for any degree and dimension, we now turn to the the construction of classes of general counterexamples.



\section{General Constructions}\label{sec:gen}


Our brute force search from the last section yields many examples of non-decomposable polynomials, but it is not suitable for finding examples when \(n\) and \(d\) are large. Thankfully, it is often easy to find examples of arbitrarily large degree for fixed \(n\). To illustrate, observe that for each \(m\geq1\) the triangle with vertices at \((0,1)\), \((1,m+2)\), and \((2,2m-1)\) has no boundary lattice points except the vertices. Moreover, the only interior lattice point is \((1,m)\), which is the average of the three vertex points. Since no two vertices sum to \((2,2m)\), the doubling of this triangle has the property outlined in \Cref{cor:hullcond}. Thus the following non-negative generalization of the Motzkin polynomial is not a sum of squares,
\[
    P_m(x,y)=x^4y^{4m-2}+x^2y^{2m+1}-3x^2y^{2m}+1.
\]


An identical argument works in three dimensions if one considers the tetrahedron with vertices at \((1,1,m+1)\), \((2,1,2m)\), \((1,2,m-1)\), and \((0,0,0)\), which contains the interior average \((1,1,m)\). In fact, the projection of this tetrahedron onto the first two variables is exactly the polytope for the Motzkin polynomial, meaning that polynomial
\[
    Q_m(x,y,z)=x^2y^2z^{2m+2}+x^4y^2z^{4m}+x^2y^4z^{2m-2}-4x^2y^2z^{2m}+1
\]
is not a sum of squares for any \(m\in\mathbb{N}\). For clarity, we draw \(C_{Q_1}\) and \(C_{Q_2}\) below with relevant points and point out that \(Q_1(x,y,z)\) resembles the example of Choi \& Lam from \cite{CL2}. Several more classes of examples like these are found by Reznik in \cite[\S 7]{reznik_extpsd} using a different technique.
\smallskip
\begin{center}
Newton Polytopes for \(Q_1\) (left) and \(Q_2\) (right).\\[0.5em] 
\framebox{
\tdplotsetmaincoords{50}{160}
\begin{tikzpicture}[tdplot_main_coords,line join=round,scale=0.75, every node/.style={scale=0.75}]
\path (2,2,4) coordinate (V)
(0,0,0)  coordinate (A)
(4,2,4)  coordinate (C)
(2,4,0)  coordinate (B)
(4,2,0)  coordinate (D)
(2,4,0)  coordinate (E)
(2,2,4)  coordinate (F)
(2,2,0)  coordinate (G)
(2,2,2)  coordinate (H);
%\draw (A) -- (C) -- (V) -- cycle;
%\draw (A) -- (B)  -- (C) -- cycle;
\draw (B) -- (C)  -- (V) -- cycle;
\foreach \p in {A,B,C,V,D,E,F,G,H}
\draw[fill=black] (\p) circle (1.3pt);
\draw[dashed] (B) -- (E);
\draw[dashed] (C) -- (D);
\draw (A) -- (V);
\draw (A) -- (B);
\draw[dashed] (C) -- (A);

\draw[thick,->] (A) -- (5,0,0) node[left]{$x$};
\draw[thick,->] (A) -- (0,5,0) node[right]{$y$};
\draw[thick,->] (A) -- (0,0,6) node[above]{$z$};
\path[fill=gray, fill opacity = 0.2] (A) -- (C) -- (V) -- cycle;
\path[fill=gray, fill opacity = 0.2] (A) -- (B)  -- (C) -- cycle;
\path[fill=gray, fill opacity = 0.2] (B) -- (C)  -- (V) -- cycle;
\path[fill=gray, fill opacity = 0.2] (A) -- (B)  -- (V) -- cycle;
\path[fill=black, fill opacity = 0.2] (A) -- (D)  -- (E) -- cycle;
\path (A)+(0:3mm) node{$0$};
\path (E)+(0:6.5mm) node{$(2,4,0)$};
\path (D)+(180:6.5mm) node{$(4,2,0)$};
\path (C)+(180:6.5mm) node{\((4,2,4)\)};
\path (V)+(45:6.5mm) node{\((2,2,4)\)};
\path (G)+(200:7mm) node{\((2,2,0)\)};
\path (V)+(170:9mm) node{\(C_{Q_1}\)};
\path (E)+(180:12mm) node{\(C_M\)};
\draw (H)+(160:6mm) node{\((2,2,2)\)};
\end{tikzpicture}
\begin{tikzpicture}[tdplot_main_coords,line join=round,scale=0.75, every node/.style={scale=0.75}]
\path (2,2,6) coordinate (V)
(0,0,0)  coordinate (A)
(4,2,8)  coordinate (C)
(2,4,2)  coordinate (B)
(4,2,0)  coordinate (D)
(2,4,0)  coordinate (E)
(2,2,4)  coordinate (F)
(2,2,0)  coordinate (G);
%\draw (A) -- (C) -- (V) -- cycle;
%\draw (A) -- (B)  -- (C) -- cycle;
\draw (B) -- (C)  -- (V) -- cycle;
\foreach \p in {A,B,C,V,D,E,F,G}
\draw[fill=black] (\p) circle (1.3pt);
\draw[dashed] (B) -- (E);
\draw[dashed] (C) -- (D);
\draw (A) -- (V);
\draw (A) -- (B);
\draw[dashed] (C) -- (A);
%\draw[dashed] (F) -- (G);
\draw[thick,->] (A) -- (5,0,0) node[left]{$x$};
\draw[thick,->] (A) -- (0,5,0) node[right]{$y$};
\draw[thick,->] (A) -- (0,0,6) node[above]{$z$};
\path[fill=gray, fill opacity = 0.2] (A) -- (C) -- (V) -- cycle;
\path[fill=gray, fill opacity = 0.2] (A) -- (B)  -- (C) -- cycle;
\path[fill=gray, fill opacity = 0.2] (B) -- (C)  -- (V) -- cycle;
\path[fill=gray, fill opacity = 0.2] (A) -- (B)  -- (V) -- cycle;
\path[fill=black, fill opacity = 0.2] (A) -- (D)  -- (E) -- cycle;
\path (A)+(0:3mm) node{$0$};
\path (B)+(0:6.5mm) node{$(2,4,2)$};
\path (E)+(0:6.5mm) node{$(2,4,0)$};
\path (D)+(180:6.5mm) node{$(4,2,0)$};
\path (C)+(180:6.5mm) node{\((4,2,8)\)};
\path (V)+(45:6.5mm) node{\((2,2,6)\)};
\path (G)+(225:4mm) node{\((2,2,0)\)};
\path (C)+(0:11mm) node{\(C_{Q_2}\)};
\path (E)+(180:12mm) node{\(C_M\)};
\draw[dashed] (F)+(200:12mm) node{\((2,2,4)\)};
\end{tikzpicture}
}
\medskip
\end{center}


These examples highlight a critical idea: the projection of an \(n\)-dimensional Newton polytope \(C\) into \(\mathbb{R}^{n-1}\) along a coordinate axis is also a polytope with lattice point vertices. Moreover, if the projection has the property outlined in \Cref{cor:hullcond} (i.e. it has an interior point which is not the average of two distinct points in the hull), then so too does \(C\). To verify that a polytope satisfies the condition of \eqref{cor:hullcond} then, it suffices to check its projections along the coordinate axes. 


On the other hand, given a polytope \(C\) in \(\mathbb{R}^n\) with one vertex fixed at the origin, the rest at linearly independent points \(q_1,\dots,q_n\), and a point \(p\in C\) which is not the sum of two distinct lattice points in \(\frac{1}{2}C\), we can easily construct a polytope in \(\mathbb{R}^{n+1}\) corresponding to a non-decomposable polynomial. We do this by choosing non-negative even integers \(r_0,\dots,r_n\) and \(s\) so that \((p,s)\) is an interior point of the convex hull \(C'\) generated by the origin and the linearly independent points \((0,r_0),(q_1,r_1),\dots,(q_n,r_n)\in \mathbb{R}^{n+1}\). There are many ways to do this, and using these points we easily obtain a non-negative and non-decomposable polynomial.
%Effectively, this gives us a way to construct a new class of non-decomposable polynomials out of any known example. 


To illustrate, we found earlier that \(x^4y^6+x^2y^6-3x^2y^4+1\) is not a sum of squares, since the point \(p=(2,4)\) is not the sum of two distinct even lattice points in the contraction by \(\frac{1}{2}\) of the closed planar triangle with vertices at \(q_1=(4,6)\), \(q_2=(2,6)\) and \(q_3=(0,0)\). Replacing \(p\) with \(\Tilde{p}=(2,4,2)\), and the other points respectively with \(\Tilde{q}_1=(4,6,2)\), \(\Tilde{q}_2=(2,6,2)\), \(\Tilde{q}_3=(0,0,4)\), we can set \(\Tilde{q}_4=(0,0,0)\) and it is easy to check using \Cref{thm:posweights} that \(\Tilde{p}\) belongs to the convex hull \(C\) induced by \(\Tilde{q}_1,\dots,\Tilde{q}_4\). Indeed, we can write
\[
    \Tilde{p}=\frac{1}{3}\Tilde{q}_1+\frac{1}{3}\Tilde{q}_2+\frac{1}{6}\Tilde{q}_3+\frac{1}{6}\Tilde{q}_4.
\]
There is nothing special about the third coordinates selected here; they were only chosen so that \(\Tilde{p}\in C'\), and our choice for \(C'\) and \(\Tilde{p}\) was far from unique. Since \(\Tilde{p}\) is not the sum of two distinct lattice points in \(\frac{1}{2}C'\) (this property is inherited from \(p\) and \(C\)), we find using \eqref{eq:shortone} that the following non-negative polynomial is not a sum of squares,
\[
    P(x,y,z)=2x^4y^6z^2+2x^2y^2z^2+z^4-6x^2y^4z^2+1.
\]


The preceding construction always increases the dimension of our polynomials by one, while the amount by which the degree increases will depend on the magnitudes of the constants \(r_0,\dots,r_n\) one chooses. Henceforth we focus on the most challenging case of constructing polynomials of minimal degree (i.e. degree \(d=4\) over \(n\geq 3\) variables), from which higher-degree examples can be easily constructed via the process described above. To do this we use a technique called homogenization.


\begin{defn}
Let \(P\) be a non-negative polynomial of even degree \(d\) on \(\mathbb{R}^n\). Then the homogenization of \(P\) is the non-negative polynomial on \(\mathbb{R}^{n+1}\) defined by
\[
    hP(x_1,\cdots,x_{n+1})=x_{n+1}^dP\bigg(\frac{x_1}{x_{n+1}},\cdots,\frac{x_n}{x_{n+1}}\bigg).
\]
\end{defn}


Using homogenization, we now devise a transformation which increases \(n\) by one while leaving \(d\) fixed. The idea is that homogenizing a degree \(d\) polynomial over \(n\) variables effectively `lifts' it's Newton polytope up to the hyperplane \(x_1+\cdots+x_{n+1}=d\) in \(\mathbb{R}^{n+1}\). By adding some well-chosen lower-order terms, we can `fill out' this homogenized polytope to give it volume.

Let \(C_P\subset\mathbb{R}^n\) denote a Newton polytope of a polynomial \(P\), and assume that \(C_P\) contains a point \(p\) as in \Cref{cor:hullcond}. That is, \(p\) is not the sum of two distinct lattice points in \(\frac{1}{2}C_P\). The polytope \(C_{hP}\) is a segment of a hyperplane in \(\mathbb{R}^{n+1}\) which can be obtained by replacing each vertex \(q=(q_1,\dots,q_n)\) of \(C_P\) with a new point \(q'=(q_1,\dots,q_n,q_{n+1})\), where \(q_{n+1}\) is chosen so that \(q_1+\cdots+q_{n+1}=d\). For instance, if \(0\) is a vertex of \(C_P\) then it is mapped to \((0,\dots,0,d)\), while extremal points for which \(q_1+\cdots+q_n=d\) are unchanged. 


Next we modify the polytope \(C_{hP}\) by appending the origin to it -- this is equivalent to adding a constant to the homogenized polynomial. We do this by adding \(1+x^{2q'}-x^{q'}\) for some vertex \(q'\) of \(C_{hP}\). Denoting by \(C_Q\) the convex hull generated by the points \(0\) and \(q_1',\dots,q_n'\), we argue that this polytope and \(p'\) have the property outlined in \Cref{cor:hullcond}.

Observe that the critical point \(p\in C_P\) is mapped by homogenization to a point \(p'\) on the hyperplane of \(C_{hP}\), meaning that \(p'\in C_Q\). Hence \(p'\) can be written as a convex combination of the vertex points \(q_1',\dots,q_n'\), and it cannot be realized as the sum of any two distinct points in \(\frac{1}{2}C_Q\) since this is not possible in the first \(n\) variables by assumption. It follows that the following polynomial is non-negative and not a sum of squares
\[
    Q(x)=hP(x)+\sum_{j=1}^{n+1}c_j(x^{2q_j'}+1-x^{q_j'})=\sum_{j=1}^{n+1}\lambda_jx^{2q_j'}-x^{p'}+\sum_{j=1}^{n+1}c_j(x^{2q_j'}+1-2x^{q_j'}).
\]
Here, \(c_1,\dots,c_{n+1}\) are non-negative constants and not all zero. To illustrate, if we begin with the minimal degree four counter example in three dimensions \(P(x,y,z)=x^2y^2+y^2z^2+x^2z^2-4xyz+1\) of Choi \& Lam \cite{CL2}, then one of the many examples this procedure gives  is
\[
    Q(w,x,y,z)=x^2y^2+y^2z^2+x^2z^2-4wxyz+2w^4-2w^2+1.
\]


It is easy to check that \(Q(1,1,1,1)=0\), that \(Q\) is positive elsewhere, and \(Q\) cannot be written as a sum of squares. The latter fact can be verified directly by analyzing the Newton polytope of \(Q\), but it is much easier to observe that if \(Q=g_1^2+\cdots+g_m^2\) for polynomials \(g_1,\dots,g_m\) then
\[
    Q(1,x,y,z)=x^2y^2+y^2z^2+x^2z^2-4xyz+1=g_1(x,y,z,1)^2+\cdots+g_m(x,y,z,1)^2.
\]
This contradicts the fact that \(P\) is not a sum of squares. The construction above gives us a polynomial of degree four over four variables which is not a sum of squares, providing the missing entry in the table of polynomials above. 


Iterating this construction affords examples of polynomials of degree \(d=4\) over \(n\geq 3\) variables which are not sums of squares of polynomials. Repeating the argument employed in \(n=2\) and \(n=3\) dimensions at the beginning of the section also gives transformations which increase the degree, rather than the dimension. Using these techniques, it is possible to construct polynomials which are not sums of squares in all possible cases.


\subsection{Discussion}


We leave much of the theory of polynomial sums of squares untouched, however insight into our main application can be gleaned from some known results. Lasserre \cite{lasserre} shows that polynomial sums of squares are dense in the space of polynomials equipped with the norm
\[
    \bigg\|\sum_\alpha c_\alpha x^\alpha\bigg\|=\sum_{\alpha}|c_\alpha|,
\]
and in particular this implies that on any compact set a given polynomial can be uniformly approximated by a sum of squares. Later, we find that an analogous result holds in the setting of H\"older continuous functions; see \Cref{thm:almostthere}.


It is interesting that non-negative polynomials of degree \(d=4\) over \(n=2\) variables can always be decomposed, for in the two dimensional setting we later find that higher-order decompositions are more straightforward, in the sense that fewer restrictions need to be imposed on a function for a decomposition to be possible. Moreover, the non-existence of a polynomial counter example in the setting of \(C^{4,\alpha}(\mathbb{R}^2)\) makes it unclear whether or not the functions in that space are all sums of half-regular squares; see the brief discussion of \Cref{sec:maybe}.


To summarize this chapter on non-decomposable functions, we have found many new examples of polynomials which are non-negative and not sums of squares, and ways to construct yet more. Moreover, we have demonstrated each of these correspond to a function which cannot be written as a sum of squares of half-regular functions. Equipped with these examples, we can return to our study of H\"older continuous functions.