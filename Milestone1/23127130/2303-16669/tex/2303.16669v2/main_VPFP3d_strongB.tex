\documentclass[11pt, a4paper]{article}

\usepackage{amsmath}
\usepackage{amsfonts}
\usepackage{amssymb}
\usepackage[english]{babel}
%\usepackage{refcheck}
\usepackage{xcolor}
\usepackage{euscript}
\usepackage{eufrak}
%\usepackage{hyperref}
\usepackage{graphicx}

\usepackage{dsfont}

\newcommand{\red}{\textcolor{red!95!black}}
\newcommand{\redm}{\textcolor{red!85!black}}
\newcommand{\blue}{\textcolor{blue!60!black}}
\newcommand{\green}{\textcolor{green!70!black}}
\newcommand{\black}{\textcolor{black}}
\newcommand{\orange}{\textcolor{red!50!yellow}}
\newcommand{\cyan}{\textcolor{green!50!blue}}
\newcommand{\yellow}{\textcolor{yellow}}




\setlength{\topmargin}{-0.5cm} \setlength{\textheight}{24cm}
\setlength{\textwidth}{15.5cm} \setlength{\oddsidemargin}{0.5cm}

\def\francais{\selectlanguage{francais}\hfuzz2pt }
\def\english{\selectlanguage{english}}


% general defs
\providecommand\mathbb{\bf}
\newcommand\R{{\mathbb R}}
\newcommand\N{{\mathbb N}}
\newcommand\Z{{\mathbb Z}}
\newcommand\T{{\mathbb T}}
\newcommand\Sp{{\mathbb S}}


\newtheorem{thm}{Theorem}[section]
\newtheorem{lemma}{Lemma}[section]
\newtheorem{pro}{Proposition}[section]
\newtheorem{defi}{Definition}[section]
\newtheorem{coro}{Corollary}[section]
\newtheorem{remark}{Remark}[section]
\newtheorem{exa}{Example}[section]



\newcounter{Remark}
\newenvironment{Remark}
{\par\refstepcounter{Remark}
\medbreak\noindent\textbf{Remark~\theRemark.}} {\medbreak}
\renewcommand\theRemark{\arabic{Remark}}


\newcounter{steps}
\newenvironment{proof}[1][]{%
\par\medbreak\setcounter{steps}{0}
{\noindent\bfseries Proof#1. }} {\hfill\fbox{\ }\medbreak}
\newcommand\step{\par\smallbreak
\refstepcounter{steps} \noindent{\bf Step
\arabic{steps}.\unskip\quad}}
\newcounter{substeps}[steps]
\newcommand\substep{\par\smallbreak
\refstepcounter{substeps} \noindent{\bf Substep
\arabic{steps}.\alph{substeps}.\unskip\quad}}



%%%%%%%%%%%%%%%%%%%%
\newcommand{\tb}[1]{\textcolor{blue}{#1}}
\newcommand{\tr}[1]{\textcolor{red}{#1}}
\newcommand{\tp}[1]{\textcolor{orange}{#1}}

\newcommand{\bgk}[0]{
Q_{\mathrm{BGK}}}

\newcommand{\calLbgk}[0]{
\mathcal{L}_{\mathrm{BGK}}}

\newcommand{\calLbgkf}[0]{
{\mathcal{L}_{\mathrm{BGK}}}_f}

\newcommand{\rot}[0]{
\mathrm{rot}}

\newcommand{\marmd}[0]{\mathrm{d}}

\newcommand{\calE}[0]{
\mathcal{E}}

\newcommand{\calN}[0]{
\mathcal{N}}

\newcommand{\calR}[0]{
\mathcal{R}}

\newcommand{\calV}[0]{
\mathcal{V}}

\newcommand{\calA}[0]{
\mathcal{A}}

\newcommand{\calB}[0]{
\mathcal{B}}

\newcommand{\calF}[0]{
\mathcal{F}}

\newcommand{\intct}[1]{
\int _{0}^{2\pi} \!#1 \;\mathrm{d}s}

\newcommand{\intst}[1]{
\int _{0}^{S} \!#1 \;\mathrm{d}s}

\newcommand{\intxt}[1]{
\int _{\R ^3} \!#1 \;\mathrm{d}x}

\newcommand{\intxlt}[1]{
\int _{\R ^2 \times \mathbb{T}^1} \!#1 \;\mathrm{d}x}

\newcommand{\intylt}[1]{
\int _{\R ^2 \times \mathbb{T}^1} \!#1 \;\mathrm{d}y}

\newcommand{\intyt}[1]{
\int _{\R ^m} \!#1 \;\mathrm{d}y}


\newcommand{\intyyt}[1]{
\int _{\R ^2} \!#1 \;\mathrm{d}\bar{y}}

\newcommand{\intyzt}[1]{
\int_{-\pi}^{\pi}\int _{\R ^2} \!#1 \;\mathrm{d}\bar{y} \mathrm{d}y_3}

\newcommand{\intvt}[1]{
\int _{\R ^3} \!#1 \;\mathrm{d}v}

\newcommand{\intvxt}[1]{
\int _{\R ^3} \int _{\R ^3} \!#1 \;\mathrm{d}v \mathrm{d}x}

\newcommand{\inttvxt}[1]{
\int _{0}^{t} \int _{\R ^3} \int _{\R ^3} \!#1 \;\mathrm{d}v \mathrm{d}x \mathrm{d}s}

\newcommand{\intTvxt}[1]{
\int _{0}^{T} \int _{\R ^3} \int _{\R ^3} \!#1 \;\mathrm{d}v \mathrm{d}x \mathrm{d}t}

\newcommand{\intw}[1]{
\int _{\R ^d} \!#1 \;\mathrm{d}w}

\newcommand{\intvp}[1]{
\int _{\R ^d} \!#1 \;\mathrm{d}v^\prime}

\newcommand{\vp}[0]{
v^{\prime}}

\newcommand{\fe}[0]{
f ^\varepsilon}

\newcommand{\fin}[0]{
f ^{\mathrm{in}}}

\newcommand{\Divx}[0]{
\mathrm{div}_x}

\newcommand{\Divv}[0]{
\mathrm{div}_v}

\newcommand{\fz}[0]{
f}

\newcommand{\fo}[0]{
f _{1}}

\newcommand{\ftw}[0]{
f _{2}}

\newcommand{\lime}[0]{
\lim _{\varepsilon \searrow 0}}

\newcommand{\muf}[0]{
M_{u [\fz]}}

\newcommand{\calmu}[0]{
{\mathcal M}_{u}}

\newcommand{\calM}[0]{
\mathcal{M}}

\newcommand{\zu}[0]{
Z_u}

\newcommand{\Phiu}{
\Phi _u}

\newcommand{\Phiuf}{
\Phi _{u[\fz]}}

\newcommand{\sphere}[0]{
\mathbb{S}^{d-1}}


\newcommand{\tfe}[0]{
\tilde{f}^\varepsilon}

\newcommand{\tue}[0]{
\tilde{u}^\varepsilon}

\newcommand{\trhoe}[0]{
\tilde{\rho}^\varepsilon}

\newcommand{\tje}[0]{
\tilde{j}^\varepsilon}

\newcommand{\tomegae}[0]{
\tilde{\Omega}^\varepsilon}

\newcommand{\ltmu}{
L^2 _{M_u}}

%\newcommand{\bltmu}{
%{\bf L}^2 _{M_u}}

\newcommand{\homu}{
H^1 _{M_u}}

%\newcommand{\bhomu}{
%{\bf H}^1 _{M_u}}

%\newcommand{\thomu}{
%\tilde{H}^1 _{M_u}}

%\newcommand{\bthomu}{
%{\bf \tilde{H}}^1 _{M_u}}

\newcommand{\intth}[1]{
\int _0 ^{\pi} #1 \;\mathrm{d}\theta}

\newcommand{\calL}[0]{
\mathcal{L}}

\newcommand{\pf}[0]{
P_f}

\newcommand{\sumi}[0]{
\sum _{i = 1}^{d-1}}

\newcommand{\sumj}[0]{
\sum _{j = 1}^{d-1}}

\newcommand{\eps}[0]{
\varepsilon}

\newcommand{\calTu}[0]{
\mathcal{T}_u}

\newcommand{\calTz}[0]{
\mathcal{T}_0}

\newcommand{\calO}[0]{
\mathcal{O}}

%\newcommand{\vb}[0]{
%\overline{v}}

\newcommand{\Exp}[1]{
\exp \left (#1 \right )}

\newcommand{\md}[0]{
\mathrm{d}}

\newcommand{\rhoe}[0]{
\rho ^\varepsilon}

\newcommand{\je}[0]{
j^\varepsilon}

\newcommand{\rhote}[0]{
\tilde{\rho} ^\varepsilon}

\newcommand{\jte}[0]{
\tilde{j}^\varepsilon}



%\newcommand{\intcr}[2]{
%\int _{\R_+}#1\int _{-1} ^{+1} #2 \;\mathrm{d}c\mathrm{d}r}

\newcommand{\vab}{
V_{\alpha, \beta} (|v|)}

%\newcommand{\lims}[0]{
%\lim _{\sigma \searrow 0}}

% ----- SPACING -------
%\baselinestretch\renewcommand{\baselinestretch}{1.5}



%%%%%%%%%%%%%%%%%%%%%%%%%%%


\begin{document}
\english


\title{Long time behavior for collisional strongly magnetized plasma in three space dimensions}



\author{Miha\"i BOSTAN \thanks{Aix Marseille Universit\'e, CNRS, Centrale Marseille, Institut de Math\'ematiques de Marseille, UMR 7373, Ch\^ateau Gombert 39 rue F. Joliot Curie, 13453 Marseille FRANCE. E-mail : {\tt mihai.bostan@univ-amu.fr}}, 
Anh-Tuan VU \thanks{Aix Marseille Universit\'e, CNRS, Centrale Marseille, Institut de Math\'ematiques de Marseille, UMR 7373, Ch\^ateau Gombert 39 rue F. Joliot Curie, 13453 Marseille FRANCE. E-mail : {\tt anh-tuan.vu@univ-amu.fr}}
}


\date{(\today)}

\maketitle


\begin{abstract}
We consider the long time evolution of a population of charged particles, under strong magnetic fields and collision mechanisms. We derive a fluid model and justify the asymptotic behavior toward smooth solutions of this regime. In three space dimensions, a constraint ocurs along the parallel direction. For eliminating the corresponding Lagrange multiplier, we average along the magnetic lines.

\end{abstract}

\paragraph{Keywords:} Long time behavior, Strongly magnetized plasmas, Relative entropy.

\paragraph{AMS classification:} 35Q75, 78A35, 82D10.
\\
\\

\section{Introduction}
\label{Intro}
We consider a population of charged particles of charge $q$, mass $m$, whose density in the phase space $(x,v)\in\R^3\times\R^3$, at time $\tilde{t}\in \R_+$, is denoted by $\tilde{f} = \tilde{f}(\tilde{t},x,v)$. We concentrate on the long time behavior, that is 
\[
\tilde{f}(\tilde{t},x,v) = f^\eps(t,x,v),\,\, t = \eps \tilde{t}.
\]
Here $\eps >0$ is a small parameter, related to the ratio between the cyclotronic period $T^\eps _c$ and the observation time $T_{\mathrm{obs}}$. The notation $\textbf{B}^\eps = B^\eps e$, $B^\eps >0$, $|e| =1$ stands for the magnetic field, assumed to be divergence free. We know that $\frac{qB^\eps}{m} \sim \omega_c^\eps = \frac{2\pi}{T^\eps _c}$ and therefore we consider strong magnetic fields
\[
B^\eps = \dfrac{B}{T^\eps_c / T_{\mathrm{obs}}} = \dfrac{B}{\eps},
\]
where $B$ is a reference magnetic field, corresponding to $T_{\mathrm{obs}}$, $\textit{i.e.}$, $\frac{qB}{m}= \omega_c = \frac{2\pi}{T_{\mathrm{obs}}}$. The collision mechanism accounts for friction and diffusion effects and is described by Fokker-Planck operator
\[
Q(f) = \dfrac{1}{\tau}\Divv{\{\sigma \nabla_v f + vf\}},
\]
where $\tau$ is the relaxation time and $\sigma$ is the velocity diffusion, see \cite{Cha1949} for the introduction of this operator, based on the principle of Brownian motion.\\
The self consistent electric field writes
\[
E[f] = - \nabla_x \Phi[f],\,\, \Phi[f] = \dfrac{q}{4\pi\epsilon_0}\int_{\R^3}{\dfrac{n[f(x',\cdot)]}{|x- x'|}}\mathrm{d}x' = \dfrac{q}{4\pi\epsilon_0}\int_{\R^3}{\int_{\R^3}{\dfrac{f(x',v')}{|x-x'|}}}\mathrm{d}v'\mathrm{d}x' ,
\]
where the potential $\Phi[f]$ satisfies the Poisson equation
\[
-\epsilon_0 \Delta_x \Phi[f]  =  q\, n[f]=q \intvt{f},
\]
and where $n[f]$ stands for the particle density. Here $\epsilon_0$ is the electric permittivity of the vacuum. We obtain the Vlasov-Poisson-Fokker-Planck (VPFP) system, with external magnetic field
\begin{equation}
\label{equ:VPFP-Scale}
\eps \partial_t f^\eps + v\cdot \nabla_x f^\eps + \dfrac{q}{m}\left(E[f^\eps]+ v\wedge \dfrac{Be}{\eps}  \right)\cdot\nabla_v f^\eps = \dfrac{1}{\tau}\Divv(\sigma \nabla_v f^\eps + v f^\eps),\,\,(t,x,v)\in \R_+ \times\R^3\times\R^3,
\end{equation}
\begin{equation}
\label{equ:PoissonEpsi}
E[f^\eps] = - \nabla_x \Phi[f^\eps],\,\, -\epsilon_0\Delta_x\Phi[f^\eps] =q \, n[f^\eps]= q \intvt{f^\eps}.
\end{equation}
We complete the above system by the initial condition
\begin{equation}
\label{equ:Initial}
f^\eps(0,x,v) = f^\eps_{\mathrm{in}}(x,v),\,\, (x,v)\in \R^3\times\R^3.
\end{equation}
There are many works dealing with the existence and uniqueness of solutions to
the VPFP system, in the three dimensional setting. For the existence of weak solutions for the VPFP problem \eqref{equ:VPFP-Scale}, \eqref{equ:PoissonEpsi} and \eqref{equ:Initial} we refer to \cite{CarSol1995, Vic1991}. Existence and uniqueness results for strong solutions of the VPFP problem can be found in \cite{Bou1993, Bou1995, Deg1986, ODwVic1990, ReinWeckler1990}.

The VPFP system \eqref{equ:VPFP-Scale}, \eqref{equ:PoissonEpsi} and \eqref{equ:Initial} describes the dynamic of charged particles under the action of strong magnetic field $|\textbf{B}^\eps| \to +\infty$, as $\eps \searrow 0$, and also accounts for collisions between particles. The mathematical literature in this field, we refer interested readers to the works \cite{AbdaHajj08, BosCollis2010, BosGam2012}. Other asymptotic regimes for strongly magnetized plasmas, incorporating collision effects, are discussed in \cite{BosQueBolt2014, BosQueFPL2014, BosLarmor2016}.

We are interested in the asymptotic behavior of the problem  \eqref{equ:VPFP-Scale}, \eqref{equ:PoissonEpsi} and \eqref{equ:Initial} as $\eps\searrow 0$. This study is motivated by the description of tokamak plasma \cite{BerDel2000}. In the large magnetic field regime, charged particles get trapped along the magnetic field lines and they rotate around these lines with small radius. This gyration radius of the particles, called the Larmor radius, is inversely proportional to the strength of the magnetic field. Therefore, charged particles are well-confined within the tokamak. However, numerically solving the kinetic equation in the presence of such large magnetic fields requires the resolution of small time steps (typycally smaller than $\eps^2$) due to high oscillations in time of the particles around the magnetic lines, leading to a huge time computations cost. Hence, the question of deriving asymptotic model to reduce the cost of numerical simulation is of great importance. Many kinetic models with strong magnetic field have been studied, usually leading to the so-called guiding-center or gyro-kinetic models. We refer to \cite{LeeGyro1983, LittHam1981} for a physical references and \cite{BosAsyAna, Bre2000, GolSaintMag1999, GolSaintQuas2003, Miot, Saint2002, Saint2003} for  mathematical results on this topic.

We derive a new asymptotic model as $\eps \searrow 0$. Let us now analyze the Vlasov-Fokker-Planck equation \eqref{equ:VPFP-Scale}. The dynamics of the charged particles are dominated by the transport in velocity  along the magnetic force $\frac{1}{\eps}(v\wedge Be)\cdot\nabla_v$, while the transport $v\cdot \nabla_x + \frac{q}{m}E[f^\eps]\cdot\nabla_v$ and the collision operator $Q(f^\eps)$ are of the same order, leading to the guiding-center approximation as $\eps$ goes to $0$. The limit distribution function is constant along the characteristic flow associated with the dominant advection field $v\wedge Be$. It depends only on space, time and two components of the velocity, corresponding to the parallel component along the magnetic field line and the magnitude of the perpendicular velocity. Moreover, for collision plasma, the charged particles seem to reach a thermal equilibrium. By performing the balance of the free energy functional associated with the VPFP system
\[
\calE[f^\eps] = \intvxt{\left( \sigma f^\eps \ln f^\eps + f^\eps \dfrac{|v|^2}{2} \right)} + \dfrac{\epsilon_0}{2m}\intxt{|E[f^\eps]|^2},
\]
then the analysis of the dissipation term
\[
\mathcal{D} [f^\eps] = \intvxt{\dfrac{|\sigma \nabla_v f^\eps + v f^\eps|^2}{f^\eps}},
\]
which allows us to conclude that the limit distribution function $f$ of the family $(f^\eps)_{\eps >0}$, as $\eps \searrow 0$, is an equilibrium of the form of local Maxwellian distribution in velocity, parametrized by macroscopic quantities (particle concentration), for any $(t,x)\in \R_+ \times \R^3$, $\mathit{i.e.}$,
\[
f(t,x,v) = n(t,x)M(v)= n(t,x)\dfrac{e^{-|v|^2/2\sigma}}{(2\pi\sigma)^{3/2}},\,\,(t,x,v)\in \R_+ \times \R^3 \times \R^3.
\]
The concentration $n(t,x)$ satisfies the following transport equation with a constraint
\begin{equation}
\label{equ:gyro-kinetic}
\partial_t n + \Divx \left[ n \left( \dfrac{E[n]\wedge e}{B} -\sigma \dfrac{\nabla_x \omega_c \wedge e}{\omega_c ^2} - \sigma \dfrac{\partial_x e e\wedge e}{\omega_c} \right) \right] + B e\cdot \nabla_x p =0,\,\, (t,x)\in \R_+ \times \R^3,
\end{equation}
\begin{equation}
\label{equ:constraint}
B e\cdot \nabla_x k[n] = 0,\,\, k[n] = \sigma (1+\ln n) + \dfrac{q}{m}\Phi[n],
\end{equation}
coupled to the Poisson equation
\begin{equation}
\label{equ:PoissonLimit}
E[n] = -\nabla_x \Phi[n],\,\, -\epsilon_0 \Delta_x \Phi[n] = q n,
\end{equation}
with initial condition 
\[
n(0,x) = n_{\mathrm{in}} (x) =  \intvt{f(0,x,v)},
\]
where $p$ is thought as a Lagrange multiplier associated to the constraint \eqref{equ:constraint}. At the limit, the concentration $n$ is advected along the electric cross-field drift, magnetic gradient drift, and magnetic curvature drift. The model obtained in
the three dimensional framework is much more complex in the two-dimensional one (see \cite{BosTuan}), since in this case, we need to handle extra constraints. The constraint \eqref{equ:constraint} arises from the perturbation of the limit particle densities $f$ as $\eps \searrow 0$, $i.e.$, $f^\eps \simeq f + \eps f_1$ leading to the following equation
\begin{equation}
\label{equ:EquaConF}
v\cdot\nabla_x f - \dfrac{q}{m}\nabla_x \Phi[f]\cdot\nabla_v f + \dfrac{q}{m}(v\wedge Be)\cdot \nabla_v f_1 =0.
\end{equation}
We want to find a closure for the dominant term $f$ or the concentration $n$, so we need to eliminate the magnetic term of $f_1$ enters \eqref{equ:EquaConF} as a Lagrange multiplier. In the absence of magnetic fields, equation \eqref{equ:EquaConF} becomes
\[
v\cdot\nabla_x f - \dfrac{q}{m}\nabla_x \Phi[f]\cdot\nabla_v f =0.
\]
Substituting $f(t,x,v) = n(t,x)M(v)$ in the previous equality, and by direct computations yield the following relation
\[
\nabla_x k[n] = 0,\,\,\, k[n] =  \sigma (1+\ln n) + \dfrac{q}{m} \Phi[n].
\]
This constraint implies that the concentration $n(t,x)$ has the form
\begin{equation}
\label{equ:Boltz-Gibb}
n(t,x) = Z(t) e^{-\frac{q}{m\sigma}\Phi[n(t)](x)},
\end{equation}
which is the so-called Boltzmann-Gibbs relation, relating the electron density to the electric potential, cf. \cite{BarGolToanSen16}. In the general case of the magnetic field $B(x)e(x)$, we apply the average along the characteristic flow with respect to the operator $(v\wedge e(x))\cdot\nabla_v$. Employing this method, we rigorously derive the constraint \eqref{equ:constraint} for the concentration $n(t,x)$. Moreover, when the magnetic field is uniform $\mathit{i.e.}$, $Be = (0,0,1)$, the constraint \eqref{equ:constraint} becomes
\[
\partial_{x_3} k[n] =0,\,\,\, k[n] = \sigma (1+\ln n) + \dfrac{q}{m}\Phi[n], 
\]
which leads to the concentration $n(t,x)$ can be written as
\begin{equation}
\label{equ:ReduBoltz-Gibb}
n(t,x) = N(t,x_{\perp})\dfrac{e^{-\frac{q}{m\sigma}\Phi[n(t)](x)}}{\int_{\R}e^{-\frac{q}{m\sigma}\Phi[n(t)](x_\perp, x_3)}\mathrm{d}x_3},
\end{equation}
where $ x= (x_\perp, x_3)\in \R^2\times \R$, cf. \cite{HerRod2019, Negu}. It is worth noting that our limit model \eqref{equ:gyro-kinetic} is consistent with the limit model of the electron distribution function obtained in \cite{HerRod2019}. Indeed, in the case of a uniform magnetic field, the limit equation \eqref{equ:gyro-kinetic} becomes
\[
\partial_t n + \Divx \left( n E \wedge e \right) + \partial_{x_3}p =0.
\]
Integrating in $x_3$ to eliminate the Lagrange multiplier $p$ and using \eqref{equ:ReduBoltz-Gibb} we obtain 
\[
\partial_t N(t,x_{\perp}) + \mathrm{div}_{x_\perp} \left( N(t,x_\perp){^\perp} \nabla_{x_\perp}\tilde{\Phi}  \right)  =0,
\]
where $\tilde{\Phi}: \R_+ \times \R^2 \to \R$ is an $x_3$ averaged of $\Phi[n]$
\[
\tilde{\Phi}(t,x_\perp) = \dfrac{m\sigma}{q} \ln \left( \int_{\R} e^{-\frac{q}{m\sigma}\Phi[n(t)](x_\perp, x_3)}\mathrm{d}x_3 \right),
\]
which is exactly the limit model introduced in \cite{HerRod2019}.

The asymptotic regime will be investigated by appealing to the relative entropy or modulated energy method, as introduced in \cite{Yau1991}. By this technique one gets strong converges, provided that the solution of the limit system is smooth as well as the convergence of the initial data. Many asymptotic regimes were obtained using this technique, see \cite{Bre2000, BreMauPue2003, GolSaintQuas2003, PueSaint2004} for quasineutral regimes in collisionless plasma physics, \cite{Saint2003, BerVas2005} for hydrodynamic limits in gaz dynamics, \cite{GouJabVas2004} for fluid-particle interaction, \cite{BosGou08, Bos2007} for high electric or magnetic field limits in plasma physics. 

Before writing our main result, we define the modulated energy $\calE[n^\eps(t)|n(t)]$ by
\[
\calE[n^\eps(t)|n(t)] = \sigma \intxt{n(t) h\left( \dfrac{n^\eps(t)}{n(t)}\right)} + \dfrac{\epsilon_0}{2m}\intxt{|\nabla_x \Phi[n^\eps] - \nabla_x \Phi[n]|^2},
\]
where $h: \R_+ \to \R_+$ is the convex function defined by $h(s) = s\ln s -s +1$, $s\in \R_+$. This quantity splits into the standard $L^2$ norm of the electric field plus the relative entropy between the particle density $n^\eps$ of \eqref{equ:VPFP-Scale}, \eqref{equ:PoissonEpsi} and \eqref{equ:Initial} and the particle concentration $n$ of the limit model \eqref{equ:gyro-kinetic}, \eqref{equ:constraint} and \eqref{equ:PoissonLimit}. For any nonnegative integer $k$ and $p \in [1, \infty]$, $W^{ k,p} = W^{ k,p} (\R^d )$ stands for the $k$-th order $L^p$ Sobolev space. $C_b^k$ stands for $k$ times continuously differentiable functions, whose partial derivatives, up to order $k$, are all bounded and $C^k ([0, T ]; E)$ is the set of $k$-times continuously differentiable functions from an interval $[0, T] \subset \R$ into a Banach space $E$. $L^p(0, T ; E)$ is the set of measurable functions from an interval $(0, T )$ to a Banach space $E$, whose $p$-th power of the $E$-norm is Lebesgue measurable. The main result of this paper is the following
%%
\begin{thm}$\;$\\
\label{MainThm}
Let $\textbf{B}\in C^1_b(\R^3)$ be a smooth magnetic field, such that $\inf_{x\in\R^3}|\textbf{B}(x)|=B_0 >0$. 
Assume that the initial particle densities $(f^\eps_{\mathrm{in}})_{\eps>0}$ satisfy $f^\eps_{\mathrm{in}}\geq 0$, $M_{\mathrm{in}}:=\sup_{\eps>0}M^\eps_{\mathrm{in}}<+\infty$, $U_{\mathrm{in}}:=\sup_{\eps>0}U^\eps_{\mathrm{in}}<+\infty$ where
\[
M^\eps _{\mathrm{in}} := \intvxt{f^\eps _{\mathrm{in}} (x,v)},\,\, U^\eps _{\mathrm{in}} := \intvxt{\dfrac{|v|^2}{2}f^\eps _{\mathrm{in}} (x,v)} + \dfrac{\epsilon_0}{2m}\intxt{|\nabla_x \Phi[f^\eps _{\mathrm{in}}]|^2}.
\]
Let $T>0$. We denote by $(f^\eps)_{\eps>0}$ the solutions of \eqref{equ:VPFP-Scale}, \eqref{equ:PoissonEpsi} and \eqref{equ:Initial} in the sense of Definition \ref{Weaksol3D} below on $[0,T]$. We assume that $n$ is a non-negative smooth solution of \eqref{equ:gyro-kinetic}, \eqref{equ:constraint} and \eqref{equ:PoissonLimit} on $[0,T]$ such that $W[n] = \frac{e}{\omega_c}\wedge \nabla_x k[n] + \frac{pBe}{n}$ belongs to $W^{1,\infty}((0,T)\times\R^3)$, $n_{\mathrm{in}}\geq 0$, $n_{\mathrm{in}} \in L^{1}(\R^3)$, $k[n_{\mathrm{in}}]\in \mathrm{ker}(Be\cdot\nabla_x)$. We suppose that
\[
\lim_{\eps\searrow 0}\sigma \intvxt{n^\eps_{\mathrm{in}}M(v) h\left(\dfrac{f^\eps_{\mathrm{in}}}{n^\eps_{\mathrm{in}}M} \right)} =0,\,\, \lim_{\eps\searrow 0}\calE[n^\eps_{\mathrm{in}}|n_{\mathrm{in}}] =0 ,
\]
where $n^\eps_{\mathrm{in}} = \intvt{f^\eps_{\mathrm{in}}},\eps>0$. Then we have
\[
\lim_{\eps\searrow 0} \sup_{0 \leq t \leq T} \sigma \intvxt{n^\eps(t)M(v)h\left(\dfrac{f^\eps}{n^\eps M} \right)} = 0,\,\,\lim_{\eps\searrow 0} \sup_{0 \leq t \leq T}\calE[n^\eps(t)|n(t)] =0 ,
\]
\[
\lim_{\eps\searrow 0}\dfrac{1}{\eps \tau}\intTvxt{\dfrac{|\sigma \nabla_v f^\eps + f^\eps v|^2}{f^\eps}} =0.
\]
In particular we have the convergences $\lim_{\eps\searrow 0} f^\eps = nM$ in $L^\infty (0,T;L^1(\R^3\times \R^3))$ and $\lim_{\eps\searrow 0} \nabla_x \Phi[f^\eps] = \nabla_x\Phi[n]$ in $L^\infty(0,T;L^2(\R^3))$.
\end{thm}
%%
Our paper is organized as follows. In Section $2$, we establish some a priori estimates 
on the three dimensional VPFP system. In the next section, using Hilbert expansion, we derive the asymptotic model. The limit model is a transport equation that involves a Lagrange multiplier with a constraint in the direction of the magnetic field lines. Section $4$ is devoted to finding an equivalent model by eliminating the Lagrange multiplier. The idea is to apply the average along the characteristic flow associated with the magnetic field. The new limit model, after averaging, needs analysis of the commutation property between the average operator and $\rot_x$. We establish a result for this commutation property for the special class of vector fields which present angle variables in Section $5$. In particular, we apply this formula to tokamak magnetic fields in the next section. The convergence towards the asymptotic model is rigorously proved in Section $7$ under the assumption that the solution of the limit problem is smooth. In the last section we investigate the well-posedness of the limit model obtained from Section $6$.


%%
\section{Preliminaires}
\label{Prelim}
We start by introducing the concept of weak solution to the VPFP system \eqref{equ:VPFP-Scale}, \eqref{equ:PoissonEpsi} and \eqref{equ:Initial} for any fixed $\eps>0$. 
\begin{defi}
\label{Weaksol3D}
Let $T>0$. Given $f^\eps_{\mathrm{in}}\in L^1(\R^3\times\R^3)$ we will say that $f^\eps$ is a weak solution of \eqref{equ:VPFP-Scale}-\eqref{equ:Initial} on the time interval $[0,T]$ if\\
(i) $f^\eps \geq 0,\,\, f^\eps\in L^\infty(0,T; L^1\cap L^\infty(\R^3\times\R^3))$,\\
(ii) for any $\psi\in C^\infty([0,T[\times\R^3\times\R^3)$
\begin{align*}
\int_{0}^{T}\int_{\R^3}\int_{\R^3} f^\eps\left[ \eps\dfrac{\partial \psi}{\partial t} + v\cdot\nabla_x \psi + \dfrac{q}{m}\left( E[f^\eps] + v \wedge \dfrac{Be}{\eps} \right)\cdot\nabla_v \psi\right]\mathrm{d}v\mathrm{d}x\mathrm{d}t \\
+ \int_{0}^{T}\int_{\R^3}\int_{\R^3}\dfrac{1}{\tau} f^\eps(\sigma \Delta_v \psi - v\cdot\nabla_v \psi) \mathrm{d}v\mathrm{d}x\mathrm{d}t + \int_{\R^3}\int_{\R^3}\eps f^\eps_{\mathrm{in}}(x,v)\psi(0,x,v)\mathrm{d}v\mathrm{d}x =0.
\end{align*}
\end{defi}
%%
The global-in-time of weak solution for the nonlinear VPFP system  \eqref{equ:VPFP-Scale}-\eqref{equ:Initial} comes from almost the same argument as presented in \cite{CarChoiJung2021, ChoiJeong2023}, we merely state the existence theorem for the solution and do not give any details on that here.
%%
\begin{thm}
\label{Thm:Weaksol}
Let $B\in L^\infty(\R^3)$. Suppose that the initial data $f^\eps_\mathrm{in}$ satisfies
\[
f^\eps _\mathrm{in} \geq 0,\,\, f^\eps _\mathrm{in} \in L^1\cap L^\infty (\R^3\times\R^3),\,\, (|x|^2 + |v|^2 + \Phi[f^\eps _\mathrm{in}])f^\eps _\mathrm{in} \in L^1(\R^3\times\R^3).
\]
Then, for any $T>0$, there exists a global weak solution of the system \eqref{equ:VPFP-Scale}-\eqref{equ:Initial} in the sense of Definition \ref{Weaksol3D} satisfying:
\[
f^\eps \in L^\infty(0,T;L^1\cap L^\infty (\R^3\times\R^3))\,\,\text{and}\,\, (|x|^2 + |v|^2 + \Phi[f^\eps])f^\eps \in L^\infty(0,T;L^1(\R^3\times\R^3)).
\]
\end{thm}
%%

The asymptotic behavior of the Vlasov-Fokker-Planck-Poisson equation \eqref{equ:VPFP-Scale} when $\eps$ becomes small comes from the balance of the free energy functional
\[
\calE[f^\eps] = \intvxt{\left( \sigma f^\eps \ln f^\eps + f^\eps \dfrac{|v|^2}{2} \right)} + \dfrac{\epsilon_0}{2m}\intxt{|E[f^\eps]|^2}.
\]
Multiplying the left hand side of \eqref{equ:VPFP-Scale} by $\sigma(1+ \ln f^\eps) + \frac{|v|^2}{2}$ and integrating with respect to $(x,v)\in \R^3\times\R^3$ yield
\begin{align}
\label{equ:BalanceEner}
&\intvxt{\left[ \eps \partial_t f^\eps + v\cdot \nabla_x f^\eps + \dfrac{q}{m}\left( E[f^\eps]+ v\wedge \dfrac{Be}{\eps} \right)\cdot \nabla_v f^\eps \right]\left[\sigma(1+ \ln f^\eps) + \dfrac{|v|^2}{2} \right]}\nonumber\\
&= \intvxt{\left[ \eps \partial_t  + v\cdot \nabla_x  + \dfrac{q}{m}\left( E[f^\eps]+ v\wedge \dfrac{Be}{\eps} \right)\cdot \nabla_v  \right]\left[\sigma f^\eps \ln f^\eps + f^\eps \dfrac{|v|^2}{2} \right]}\nonumber\\
&- \intvxt{\dfrac{q}{m}E[f^\eps]\cdot v f^\eps}\nonumber\\
&= \eps \dfrac{\mathrm{d}}{\mathrm{d}t} \intvxt{\left( \sigma f^\eps \ln f^\eps + f^\eps \dfrac{|v|^2}{2} \right)} + \dfrac{q}{m} \intxt{\nabla_x\Phi[f^\eps]\cdot \left( \intvt{v f^\eps}\right)}.
\end{align} 
Thanks to the continuty equation
\[
\eps \partial_t n[f^\eps] + \Divx \intvt{v f^\eps} =0,
\]
we write
\begin{align}
\label{equ:EvoluElect}
\dfrac{q}{m} \intxt{\nabla_x\Phi[f^\eps]\cdot \left( \intvt{v f^\eps}\right)} &=\eps \dfrac{q}{m}\intxt{\Phi[f^\eps]\partial_t n[f^\eps]}\\
&= -\dfrac{\epsilon _0 \eps}{m} \intxt{\Phi[f^\eps]\partial_t \Delta_x \Phi[f^\eps]}\nonumber\\
&=\dfrac{\epsilon _0 \eps}{2m}\dfrac{\mathrm{d}}{\mathrm{d}t}\intxt{|\nabla_x \Phi[f^\eps]|^2}\nonumber.
\end{align}
Multiplying the right hand side of \eqref{equ:VPFP-Scale} by $\sigma(1+ \ln f^\eps) + \frac{|v|^2}{2}$ and then integrating with respect to $(x,v)\in \R^3\times\R^3$ imply
\begin{align}
\label{equ:Dissipation}
\intvxt{Q(f^\eps)\left[ \sigma(1+ \ln f^\eps) + \dfrac{|v|^2}{2} \right]} &= -\dfrac{1}{\tau} \intvxt{\dfrac{|\sigma\nabla_v f^\eps + v f^\eps|^2}{f^\eps}}\\
&= -\dfrac{1}{\tau} \intvxt{\dfrac{|\sigma M\nabla_v (f^\eps/M) |^2}{f^\eps}},\nonumber
\end{align}
where $M$ stands for the Maxwellian equilibrium $M(v) = (2\pi\sigma)^{-3/2}\exp\left(-\frac{|v|^2}{2\sigma}\right)$, $v\in\R^3$. Combining \eqref{equ:BalanceEner}, \eqref{equ:EvoluElect} and \eqref{equ:Dissipation} leads to the balance
\begin{align}
\label{equ:EquFreeEne}
&\eps \dfrac{\mathrm{d}}{\mathrm{d}t} \left[ \intvxt{\left( \sigma f^\eps \ln f^\eps + f^\eps \dfrac{|v|^2}{2} \right)} + \dfrac{\epsilon_0}{2m}\intxt{|\nabla_x \Phi[f^\eps]|^2}\right]\\
&+ \dfrac{1}{\tau} \intvxt{\dfrac{|\sigma M\nabla_v (f^\eps/M) |^2}{f^\eps}} =0, \nonumber
\end{align}
or equivalently
\[
\eps \calE[f^\eps(t)] + \dfrac{1}{\tau} \inttvxt{\dfrac{|\sigma M\nabla_v (f^\eps/M) |^2}{f^\eps}} = \eps \calE[f^\eps(0)].
\]
Notice that weak solutions may only satisfy an inequality in the above relation that is enough for our purposes. At least formally, we deduce that $f^\eps = f + \calO(\eps)$, as $\eps \searrow 0$, where the leading order density f satisfies
\[
\dfrac{1}{\tau}\intvxt{\dfrac{|\sigma M\nabla_v (f/M) |^2}{f}} =0,\,\, t\in \R_+ .
\]
Therefore we have $f(t,x,v) = n(t,x)M(v), (t,x,v)\in \R_+ \times \R^3\times\R^3$ and it remains to determine the time evolution of the concentration $n = \intvt{f}$.

We establish uniform bounds for the kinetic energy.
%%
\begin{lemma}$\;$\\
\label{KinEne}
Let $T>0$. Assume that the initial particle densities $(f^\eps_{\mathrm{in}})$ satisfy $f^\eps _{\mathrm{in}} \geq 0$, $M_{\mathrm{in}}:= \sup_{\eps >0} M^\eps _{\mathrm{in}} < +\infty$, $U_{\mathrm{in}} := \sup_{\eps >0} U^\eps _{\mathrm{in}} < +\infty$, where for any $\eps >0$
\[
M^\eps _{\mathrm{in}} := \intvxt{f^\eps _{\mathrm{in}} (x,v)},\,\, U^\eps _{\mathrm{in}} := \intvxt{\dfrac{|v|^2}{2}f^\eps _{\mathrm{in}} (x,v)} + \dfrac{\epsilon_0}{2m}\intxt{|\nabla_x \Phi[f^\eps _{\mathrm{in}}]|^2}.
\]
We assume that $(f^\eps)_{\eps>0}$ are weak solutions of \eqref{equ:VPFP-Scale}, \eqref{equ:PoissonEpsi} and \eqref{equ:Initial}. Then we have 
\[
\eps \sup_{0\leq t\leq T} \left\{ \intvxt{\dfrac{|v|^2}{2}f^\eps(t,x,v)}+ \dfrac{\epsilon_0}{2m}\intxt{|\nabla_x \Phi[f^\eps]|^2} \right\} \leq \eps U_{\mathrm{in}} + \dfrac{3\sigma}{\tau}TM_{\mathrm{in}}
\]
and
\[
\dfrac{1}{\tau} \intTvxt{|v|^2 f^\eps(t,x,v)} \leq \eps U_{\mathrm{in}} + \dfrac{3\sigma}{\tau}TM_{\mathrm{in}}.
\]
\end{lemma}
%%
\begin{proof}$\;$\\
We will establish the results  for smooth solutions, and we observe that the same conclusions hold true in the framework of weak solutions by combining the formal arguments to be exposed here with the choice of an appropriate sequence of test functions in Definition \ref{Weaksol3D} for every studied property (cf. \cite{BoniCarSoler1997, BouDol1995}).
Multiplying \eqref{equ:VPFP-Scale} by $\frac{|v|^2}{2}$ and integrating with respect to $(x,v)\in\R^3\times\R^3$ yield
\[
\eps \dfrac{\mathrm{d}}{\mathrm{d}t} \left\{ \intvxt{\dfrac{|v|^2}{2}f^\eps(t,x,v)}+ \dfrac{\epsilon_0}{2m}\intxt{|\nabla_x \Phi[f^\eps]|^2} \right\} = \dfrac{3\sigma}{\tau}M^\eps _{\mathrm{in}} - \dfrac{1}{\tau} \intvxt{|v|^2 f^\eps}
\]
and therefore we obtain
\begin{align*}
&\eps  \left\{ \intvxt{\dfrac{|v|^2}{2}f^\eps(t,x,v)}+ \dfrac{\epsilon_0}{2m}\intxt{|\nabla_x \Phi[f^\eps]|^2} \right\} + \dfrac{1}{\tau} \inttvxt{|v|^2 f^\eps} \\
&= \eps U^\eps _{\mathrm{in}}+  \dfrac{3\sigma}{\tau} t M^\eps _{\mathrm{in}},
\end{align*}
which yields the results.
\end{proof}
%%




%%
\section{Formal derivation of the limit model}
\label{ForDerLimMod}
This section is devoted to deriving the limit model for \eqref{equ:VPFP-Scale}, \eqref{equ:PoissonEpsi} and \eqref{equ:Initial} when $\eps$ becomes
very small, using the properties of the average dominant operator transport. At the formal level, we initiate our analysis with a Hilbert expansion 
\[f^\eps = f + \eps f_1 + \eps^2 f_2 + ....\]
 Plugging the above ansatz into the kinetic equation \eqref{equ:VPFP-Scale} yields
\begin{align*}
&\eps \partial_t (f + \eps f_1 + \eps^2 f_2 + ...)+ v\cdot\nabla_x (f + \eps f_1 + \eps^2 f_2 + ...)\\
&+ \dfrac{q}{m}\left( E[f + \eps f_1 + \eps^2 f_2 + ...]+ v\wedge \dfrac{Be}{\eps} \right)\cdot\nabla_v (f + \eps f_1 + \eps^2 f_2 + ...) = Q(f + \eps f_1 + \eps^2 f_2 + ...).
\end{align*}
Identifying the contributions to any power of $\eps$ leads to
\begin{equation}
\label{equ:Order0}
\dfrac{q}{m}(v\wedge Be)\cdot\nabla_v f =0.
\end{equation}
\begin{equation}
\label{equ:Order1}
v\cdot \nabla_x f + \dfrac{q}{m}E[f]\cdot \nabla_v f + \dfrac{q}{m}(v\wedge Be)\cdot \nabla_v f_1 = Q(f).
\end{equation}
\begin{equation}
\label{equ:Order2}
\partial_t f + v\cdot \nabla_x f_1 + \dfrac{q}{m}E[f_1]\cdot\nabla_v f + \dfrac{q}{m}(v\wedge Be)\cdot \nabla_v f_2 = Q(f_1).
\end{equation}
Multiplying \eqref{equ:Order1} by $\sigma(1+ \ln f) + \frac{|v|^2}{2}$ and integrating with respect to $(x,v)\in \R^3\times\R^3$ yields
\begin{align}
\label{equ:EquBalancef1}
&\intvxt{\left( v\cdot\nabla_x + \dfrac{q}{m}E[f]\cdot\nabla_v \right)\left(\sigma f \ln f + f\dfrac{|v|^2}{2} \right)} + \dfrac{1}{\tau}\intvxt{\dfrac{|\sigma M \nabla_v(f/M)|^2}{f}}\nonumber\\
&= \intvxt{\dfrac{q}{m}E[f]\cdot vf} + \intvxt{f_1 \dfrac{q}{m}(v\wedge Be)\cdot\dfrac{\sigma \nabla_v f}{f}}.
\end{align}
Integrating \eqref{equ:Order1} with respect to $v\in\R^3$ we deduce that $\Divx \intvt{vf} =0$ and therefore we have
\[
\intvxt{\dfrac{q}{m}E[f]\cdot vf} = - \dfrac{q}{m}\intxt{\nabla_x \Phi[f]\cdot \left( \intvt{v f}\right)} = 0.
\]
Using also \eqref{equ:Order0}, the last contribution in the right hand side of 
\eqref{equ:EquBalancef1} cancels, and therefore we obtain
\[
\dfrac{1}{\tau} \intvxt{\dfrac{|\sigma M\nabla_v (f/M)|^2}{f}} =0,\,\, t\in \R_+ ,
\]
saying that $f =nM$, for some function $n=n(t,x)$ to be determined. In that case, the constraint \eqref{equ:Order0} is satisfied and \eqref{equ:Order1} becomes
\begin{equation*}
v\cdot\nabla_x f + \dfrac{q}{m}E[f]\cdot\nabla_v f \in \mathrm{Range}((v\wedge e(x))\cdot\nabla_v),\,\, x\in \R^3.
\end{equation*}
For any $e \in \mathbb{S}^2 $, we denote by $\mathcal{R} (\theta,e)$ the rotation of angle $\theta$ around the axis $e$
\begin{equation*}
\mathcal{R}(\theta, e ) = \cos\theta (I_3 -e\otimes e)v - \sin\theta (v\wedge e) + (v\cdot e)e,\,\, v\in\R^3.
\end{equation*}
The characteristic flow of the field $(v\wedge e)\cdot\nabla_v$
\[
\dfrac{\mathrm{d}\calV}{\mathrm{d}\theta} = \calV(\theta;v)\wedge e, \,\,\calV(0;v)= v,
\]
is given by
\[
\calV(\theta;v) = \mathcal{R}(-\theta,e)v = \cos\theta (I_3 -e\otimes e)v + \sin\theta (v\wedge e) + (v\cdot e)e,\,\, (\theta,v)\in \R\times\R^3.
\]
For any function $g(v) = (v\wedge e)\cdot\nabla_v h$ in the range of the operator $(v\wedge e)\cdot\nabla_v$, we have
\[
g(\calV(\theta;v)) = \dfrac{\mathrm{d}}{\mathrm{d}\theta}h(\calV(\theta;v)),\,\, (\theta,v)\in \R\times\R^3,
\]
and by the periodicity of the flow we obtain
\[
\dfrac{1}{2\pi}\int_{0}^{2\pi} g(\calV(\theta;v)) \mathrm{d}\theta =0, \,\, v\in\R^3.
\]
Therefore, for any $x\in\R^3$, the average along the characteristic flow with respect to $(v\wedge e(x))\cdot\nabla_v$ of the function $v\cdot\nabla_x f + \frac{q}{m}E[f]\cdot\nabla_v f$ vanishes. But
\[
v\cdot\nabla_x f + \frac{q}{m}E[f]\cdot\nabla_v f = (v \cdot \nabla_x n)M - \dfrac{q}{m}(E[f]\cdot v)n\dfrac{M}{\sigma} = \dfrac{n}{\sigma}Mv\cdot\nabla_x(\sigma\ln n + \dfrac{q}{m}\Phi[f]),
\]
and since
\[
\dfrac{1}{2\pi} \int_{0}^{2\pi} M(\calV(\theta;v))\calV(\theta;v)\mathrm{d}\theta = M(v)(v\cdot e)e,
\]
finally we obtain the constraint
\[
e\cdot \nabla_x k[n] = 0,\,\, k[n] = \sigma(1+\ln n) + \dfrac{q}{m}\Phi[n],\,\, x\in\R^3.
\]
Here the potential $\Phi = \Phi[n]$ writes
\[
\Phi[n(t)](x) = \dfrac{q}{4\pi\epsilon_0}\int_{\R^3}\dfrac{n(t,x')}{|x-x'|}\mathrm{d}x',\,\, (t,x)\in \R_+ \times\R^3.
\]
The time evolution for the concentration $n$ comes by integrating \eqref{equ:Order2} with respect to $v\in\R^3$
\begin{equation}
\label{equ:EquContinum}
\partial_t n + \Divx\intvt{v f_1} =0.
\end{equation}
Multiplying \eqref{equ:Order1} by $v$ and integrating with respect to $v\in\R^3$ we obtain
\[
\Divx \intvt{v\otimes v f} - n\dfrac{q}{m}E[f] - \dfrac{qB}{m}\intvt{v f_1\wedge e} =0.
\]
Since $f$ is a Maxwellian equilibrium, we have $\intvt{v\otimes v f} = \sigma n I_3$ and the previous equality becomes
\[
\omega_c \intvt{v f_1}\wedge e = \sigma \nabla_x n - n\dfrac{q}{m}E[f],
\]
or equivalently
\begin{align*}
\omega_c (I_3 - e\otimes e)\intvt{v f_1} &= ne\wedge \left( \sigma \dfrac{\nabla_x n}{n} - \dfrac{q}{m}E[f] \right)\\
& = ne\wedge \nabla_x (\sigma\ln n + \dfrac{q}{m}\Phi[n])\\
&= ne\wedge \nabla_x k[n].
\end{align*}
The divergence with respect to $x$ of $\intvt{vf_1}$ writes
\begin{align*}
\Divx \intvt{v f_1} &= \Divx \left[ (I_3 - e\otimes e)\intvt{v f_1} \right] + \Divx \left[ e\otimes e\intvt{v f_1} \right]\\
&= \Divx\left( \dfrac{n e}{\omega_c}\wedge \nabla_x k[n]\right) +  Be\cdot\nabla_x \intvt{\dfrac{(v\cdot e)f_1}{B}}.
\end{align*}
Coming back in \eqref{equ:EquContinum} we obtain the limit model
\begin{equation}
\label{equ:EquLimitKn}
\partial_t n + \Divx\left( \dfrac{ne}{\omega_c}\wedge \nabla_x k[n] \right) + Be\cdot\nabla_x p =0,
\end{equation}
for some function $p$ such that the following constraint holds true
\begin{equation}
\label{equ:EquConsSec3}
Be\cdot\nabla_x k[n] =0,\,\, k[n] = \sigma(1+\ln n) + \dfrac{q}{m}\Phi[n].
\end{equation}
The limit model involves a Lagrange multiplier $p$, associated to the constraint \eqref{equ:EquConsSec3}. One of the main difficulty is that the unknown is the concentration $n$, whereas the constraint relies on $k[n]$. Formally, we have the balances
%%
\begin{pro}$\;$\\
\label{BalLiMod}
Any non-negative smooth solution of the limit model \eqref{equ:EquLimitKn}, \eqref{equ:EquConsSec3} verifies the mass and free energy conservations
\[
\dfrac{\mathrm{d}}{\mathrm{d}t}\intxt{n(t,x)} =0,\,\, \dfrac{\mathrm{d}}{\mathrm{d}t}\intxt{\left\{ \sigma n\ln n + \dfrac{\epsilon_0}{2m}|\nabla_x\Phi[n]|^2\right\}} =0.
\]
\end{pro}
\begin{proof}$\;$\\
Clearly we have the total mass conservation. For the energy conservation, we multiply \eqref{equ:EquLimitKn} by $k[n]$ and integrate with respect to $x\in\R^3$, observing that
\[
\intxt{\partial_t n k[n]} = \dfrac{\mathrm{d}}{\mathrm{d}t}\intxt{\left\{ \sigma n\ln n + \dfrac{\epsilon_0}{2m}|\nabla_x\Phi[n]|^2\right\}},
\]
\[
\intxt{\Divx\left(\dfrac{ne}{\omega_c}\wedge \nabla_x k[n] \right)k[n]} = - \intxt{\left(\dfrac{ne}{\omega_c}\wedge \nabla_x k[n] \right)\cdot\nabla_x k[n]} =0,
\]
\[
\intxt{Be\cdot\nabla_xp k[n]} = - \intxt{pBe\cdot \nabla_x k[n]} =0.
\]
\end{proof}
%%
Recall the usual drift velocities when dealing with magnetic confinement: the electric field drift, the magnetic gradient drift, and the magnetic curvature drift
\[
\dfrac{E\wedge e}{B}, \,\, - \dfrac{m|v\wedge e|^2}{2qB}\dfrac{\nabla_x B\wedge e}{B} = - \dfrac{|v\wedge e|^2}{2}\dfrac{\nabla_x \omega_c \wedge e}{\omega_c ^2},\,\, -\dfrac{m|v\wedge e|^2}{qB}\partial_x e e\wedge e = -\dfrac{(v\cdot e)^2}{\omega_c}\partial_x e e\wedge e.
\]
When working at the fluid level, the averages with respect to $v\in\R^3$ of the above drift velocities become
\[
v_{\wedge D} = \intvt{\dfrac{E\wedge e}{B}M} = \dfrac{E\wedge e}{B},
\]
\[
v_{GD} = - \intvt{\dfrac{|v\wedge e|^2}{2}\dfrac{\nabla_x \omega_c \wedge e}{\omega_c ^2}M} = -\sigma \dfrac{\nabla_x \omega_c \wedge e}{\omega_c ^2},
\]
\[
v_{CD} = - \intvt{\dfrac{(v\cdot e)^2}{\omega_c}\partial_x e e\wedge e M} = - \sigma \dfrac{\partial_x e e\wedge e}{\omega_c}.
\]
The flux in the limit model \eqref{equ:EquLimitKn} also writes $n\calV[n]$, where $\calV[n] = v_{\wedge D} + v_{GD} + v_{CD}$.
%%
\begin{pro}$\;$\\
Any non-negative smooth function $n$ satisfying
\[
\partial_t n + \Divx\left( \dfrac{ne}{\omega_c}\wedge \nabla_x k[n] \right) + Be\cdot\nabla_x p =0,\,\, k[n] = \sigma(1+\ln n) + \dfrac{q}{m}\Phi[n],
\]
also verifies
\[
\partial_t n + \Divx(n\calV[n]) + Be\cdot\nabla_x\tilde{p} =0,\,\, \calV[n] = \dfrac{E\wedge e}{B} -\sigma \dfrac{\nabla_x \omega_c \wedge e}{\omega_c ^2}- \sigma \dfrac{\partial_x e e\wedge e}{\omega_c},
\]
and $\tilde{p} = p+ \frac{\sigma n}{B\omega_c}(e\cdot \rot_x e)$.
\end{pro}
%%
\begin{proof}$\;$\\
Recall the formula $\Divx{(\xi \wedge \eta)}= \eta\cdot \rot_x \xi - \xi\cdot \rot_x \eta$, for any smooth vector fields $\xi$ and $\eta$. Therefore we can write
\begin{align*}
\Divx\left( \dfrac{ne}{\omega_c}\wedge \nabla_x k[n] \right) &= \Divx\left[\dfrac{ne}{\omega_c}\wedge\left( \sigma\dfrac{\nabla_x n}{n} -\dfrac{q}{m}E \right) \right]\\
&= \Divx\left( n\dfrac{E\wedge e}{B}\right) + \Divx\left(\sigma \dfrac{e}{\omega_c}\wedge \nabla_x n \right)\\
&=\Divx\left( n\dfrac{E\wedge e}{B}\right) + \sigma\, \rot_x\left(\dfrac{e}{\omega_c} \right)\cdot\nabla_x n\\
&= \Divx\left( n\dfrac{E\wedge e}{B}\right) + \sigma\, \Divx\left(n \,\rot_x \left(\dfrac{e}{\omega_c} \right) \right)\\
&= \Divx\left( n\dfrac{E\wedge e}{B}\right) - \sigma\, \Divx\left( n\dfrac{\nabla_x \omega_c \wedge e}{\omega_c ^2} - \dfrac{n}{\omega_c}\rot_x e \right)\\
&= \Divx\left( n v_{\wedge D} + n v_{GD} + \dfrac{\sigma n}{\omega_c}(I_3 -e\otimes e)\rot_x e\right) + \Divx\left(\dfrac{\sigma n}{\omega_c}(e\cdot\rot_x e)e \right).
\end{align*}
Notice that we can write 
\[
(I_3 -e\otimes e)\rot_x e = e\wedge (\rot_x e \wedge e) = e\wedge[(\partial_x e - {^t}\partial_x e)e] = e \wedge \partial_x e e,
\]
implying that
\[
\sigma \dfrac{n}{\omega_c} (I_3 - e\otimes e)\rot_x e = - \sigma \dfrac{n}{\omega_c} \partial_x e e \wedge e = n v_{CD}.
\]
Finally we obtain
\[
\Divx\left( \dfrac{ne}{\omega_c}\wedge \nabla_x k[n] \right) = \Divx(n\calV[n]) + Be\cdot \nabla_x\left[ \dfrac{\sigma n}{B\omega_c}(e\cdot \rot_x e) \right],
\]
and our conclusion follows.
\end{proof}
%%



\section{Reformulation of the limit model}
\label{RefLiMod}
We intend to find an equivalent formulation for \eqref{equ:EquLimitKn}, \eqref{equ:EquConsSec3} by eliminating the Lagrange multiplier $p$ which appears in \eqref{equ:EquLimitKn}. For doing that, we will average along the characteristic flow of the magnetic field cf. \cite{BogMit61, BosTraEquSin, BosAsyAna, BosGuiCen3D, BosFinHauCRAS, BosFin16, BosSIAM09}. Let us recall briefly the definition of the average operators along a characteristic flow for functions and vector fields cf. \cite{BosSIAM16}. Consider a smooth, divergence free vector field $b=b(y):\R^m \to \R^m$
\begin{equation}
\label{equ:EquLipDiv}
b\in W^{1,\infty}_{\mathrm{loc}}(\R^m),\,\,\mathrm{div}_y b =0,
\end{equation}
with at most linear growth at infinity
\begin{equation}
\label{equ:EquGrowth}
\exists C>0\,\,\mathrm{such\,that}\,\, |b(y)|\leq C(1+|y|),\,\,y\in\R^m .
\end{equation}
We denote by $Y(s;y)$ the characteristic flow associated to $b$
\[
\dfrac{\mathrm{d}Y}{\mathrm{d}s} = b(Y(s;y)),\,\, Y(0;y) = y,\,\, s\in\R,\,\, y\in\R^m.
\]
Under the above hypothese, this flow has the regularity $Y\in W^{1,\infty}_{\mathrm{loc}}(\R\times\R^m)$ and is measure preserving. We concentrate on periodic characteristic flows (the tokamak characteristic flows are periodic, with uniform period) that is:
\[
\exists S>0\,\,\mathrm{such\,that}\,\, Y(S;y)=y,\,\,y\in\R^m.
\]
For any function $u=u(y):\R^m\to \R$ we define the average $\left< u\right>$ along the flow of $b\cdot\nabla_y$ by
\[
\left< u\right> (y) = \dfrac{1}{S} \intst{u(Y(s;y))},\,\, y\in\R^m.
\]
When applied to $L^2(\R^m)$ functions, the above operator coincides with the orthogonal projection in $L^2(\R^m)$, over the subspace of constant functions along the flow of $b\cdot\nabla_y$, cf. \cite{BosTraEquSin}. Indeed, it is easily seen that for any $y\in\R^m$, $h\in\R$
\[
\left< u\right> (Y(h;y)) = \dfrac{1}{S} \int_{0}^{S}u(Y(s;Y(h;y)))\mathrm{d}s = \dfrac{1}{S} \int_{0}^{S}u(Y(s+h;y))\mathrm{d}s = \left< u\right>(y),
\]
and for any $\psi\in L^2(\R^m)$ which is constant along the flow $Y$ we have
\begin{align*}
\intyt{u(y)\psi(y)} &= \intyt{u(y)\psi(Y(-s;y))} \\
&= \intyt{u(Y(s;y))\psi(y)}\\
&= \intyt{\dfrac{1}{S}\intst{ u(Y(s;y))}\, \psi(y)}\\
&= \intyt{\left< u\right>(y)\psi(y)}.
\end{align*}
For any vector field $c = c(y):\R^m \to \R^m$, we define the average $\left< c\right>$ along the flow of $b\cdot\nabla_y$ by
\[
\left< c\right> = \dfrac{1}{S}\intst{\partial Y(-s;Y(s;\cdot))c(Y(s;\cdot))}.
\]
Notice that the family of transformations $c\to \partial Y(-s;Y(s;\cdot))c(Y(s;\cdot)) $, $s\in\R$, is a one parameter group. The average operators for functions and vector fields are related by the following formulas:
\begin{equation}
\label{equ:AveConsFlowY}
\left< c\cdot \nabla \psi\right> = \left< c\right>\cdot\nabla \psi,
\end{equation}
for any function $\psi$ which is constant along the flow $Y$ and 
\begin{equation}
\label{equ:AveInvoFieldb}
\left< a\cdot\nabla\theta\right> = a\cdot \nabla \left< \theta\right>,
\end{equation}
for any vector field $a$ which is in involution with respect to $b$, that is, their Poisson bracket vanishes
\[
[a,b] : = (a\cdot\nabla_y)b - (b \cdot \nabla_y )a  =0.
\]
 Indeed, as $\psi(Y(s;\cdot)) =\psi, s\in\R$, we have $^t \partial Y(s;y)(\nabla\psi)(Y(s;y)) = \nabla\psi(y), s\in\R$ and therefore
\begin{align*}
\left< c\right>\cdot\nabla \psi &= \dfrac{1}{S} \intst{ \partial Y(-s;Y(s;\cdot))c(Y(s;\cdot))} \cdot \nabla \psi \\
&= \dfrac{1}{S} \intst{ \partial Y(-s;Y(s;\cdot))c(Y(s;\cdot)) \cdot {^t} \partial Y(s;\cdot)(\nabla\psi)(Y(s;\cdot)) }\\
&= \dfrac{1}{S} \intst{ (c\cdot \nabla \psi)(Y(s;\cdot))} \\
&= \left< c\cdot\nabla\psi\right>.
\end{align*}
In the previous computations, we utilized the equality
$
 Y(-s;Y(s;y)) = y,\,\,y\in\R^m
$
which, upon differentiation with respect to $y$, implies
\[
\partial_y Y(-s;Y(s;\cdot)\partial_y Y(s;\cdot) = I_m.
\]
Similarly, the condition $[a,b] =0$ expresses the commutation between the flows associated to the vector fields $a$ and $b$
\begin{equation}
\label{equ:EquComFlow}
Z(h;Y(s;y)) = Y(s;Z(h;y)),\,\, h,s \in\R, \,\, y\in\R^m,
\end{equation}
where $Z(h;y)$ denotes the characteristic flow associated to $a$
\[
\dfrac{\mathrm{d}}{\mathrm{d}h}Z(h;y) = a(Z(h;y)),\,\,(h,y)\in\R\times\R^m.
\]
Taking the derivative of \eqref{equ:EquComFlow} with respect to $h$ at $h=0$ we obtain
\[
a(Y(s;y)) = \dfrac{\mathrm{d}}{\mathrm{d}h}|_{h=0}Z(h;Y(s;y)) =  \dfrac{\mathrm{d}}{\mathrm{d}h}|_{h=0}Y(s;Z(h;y)) = \partial_y Y(s;y)a(y),\,\,(s,y)\in (\R\times\R^m).
\]
Hence we have
\begin{align*}
\left< a\cdot\nabla\theta \right> &= \dfrac{1}{S}\intst{a(Y(s;\cdot))\cdot(\nabla\theta)(Y(s;\cdot))}\\
&=  \dfrac{1}{S}\intst{ a\cdot {^t}\partial_y Y(s;\cdot)(\nabla\theta)(Y(s;\cdot))}\\
&= \dfrac{1}{S}\intst{a\cdot \nabla(\theta(Y(s;\cdot)))}\\
&= a\cdot\nabla \left<\theta\right>.
\end{align*}
We come back to the limit model \eqref{equ:EquLimitKn}, \eqref{equ:EquConsSec3} and we consider a smooth magnetic field $Be\cdot \nabla_x$, whose characteristic flow is periodic, with a uniform period $S$. The properties of the average along the magnetic field lines are investigated in the mathematical literature, cf. \cite{Negu}. If we denote by $X=X(s;x)$ the flow of the magnetic field, we have by $S$ periodicity
\[
\left< Be\cdot \nabla_x p \right> = \dfrac{1}{S}\intst{(Be\cdot \nabla_x p)(X(s;\cdot))}= \dfrac{1}{S}\intst{\dfrac{\mathrm{d}}{\mathrm{d}s}\left\{ p(X(s;\cdot)) \right\}} =0.
\]
Therefore the  Lagrange multiplier $p$ can be eliminated, by taking the average in \eqref{equ:EquLimitKn} 
\begin{equation}
\label{equ:EquAveLim}
\partial_t \left< n \right> + \left< \Divx\left( \dfrac{ne}{\omega_c}\wedge \nabla_x k[n]\right) \right> =0.
\end{equation}
The difficulty task is how to express the average of the divergence term, with respect to $\left<n\right>$, such that we get a model for the new unknown $\left<n\right>$.
%%
\begin{pro}$\;$\\
\label{ZeroAve}
For any zero average function $\alpha$, and constant along the flow $X$ function $\psi$, we have
\[
\left< \Divx \left( \dfrac{\alpha e}{B}\wedge \nabla\psi \right)\right>=0.
\]
\end{pro}
%%
\begin{proof}$\;$\\
We are done if we prove that for any constant along the flow function $\theta$ we have
\begin{equation}
\label{equ:EquZeroWeak}
\intxt{\Divx\left( \dfrac{\alpha e}{B}\wedge \nabla\psi \right)\theta(x)} =0.
\end{equation}
As $e\cdot\nabla\psi =0$, $e\cdot \nabla\theta =0$, therefore we have $(I_3 -e\otimes e)(\nabla\theta \wedge \nabla\psi) =0$. The vector field $\nabla\theta \wedge \nabla\psi$ is divergence free
\[
\Divx(\nabla\theta \wedge \nabla\psi) = \nabla \psi \cdot \rot_x(\nabla\theta)-\nabla\theta \cdot \rot_x(\nabla\psi) =0,
\]
and therefore there is a constant function $\lambda$ along the flow $X$ such that $\nabla\theta \wedge \nabla\psi = \lambda Be$. We deduce that
\begin{align*}
\intxt{\Divx\left( \dfrac{\alpha e}{B}\wedge \nabla\psi \right)\theta(x)} &= - \intxt{\left( \dfrac{\alpha e}{B}\wedge \nabla\psi \right)\cdot\nabla\theta}\\
&= \intxt{(\nabla\theta\wedge\nabla\psi)\cdot\dfrac{\alpha e}{B}}\\
&= \intxt{\lambda Be \cdot \dfrac{\alpha e}{B}}\\
&= \intxt{\lambda \alpha} = \intxt{\lambda\left<\alpha\right>} =0,
\end{align*}
and therefore \eqref{equ:EquZeroWeak} holds true.
\end{proof}
%%
Applying Proposition \ref{ZeroAve} with the function $k[n]$, which is constant along the flow of $Be\cdot\nabla_x$, we obtain
\[
\left< \Divx\left( \dfrac{ne}{\omega_c}\wedge \nabla_x k[n]\right) \right> = \left< \Divx\left( \dfrac{\left<n\right>e}{\omega_c}\wedge \nabla_x k[n]\right) \right>.
\]
We also need to express $k[n] = \sigma(1+\ln n)+ \frac{q}{m}\Phi[n]$, with respect to $\left< n\right>$, where the concentration $n$ is such that the constraint \eqref{equ:EquConsSec3} holds true.
%%
\begin{lemma}$\;$\\
\label{FirstVar}
The first variation of the free energy 
\[
\calE [n] = \intxt{\sigma n\ln n + \dfrac{\epsilon_0}{2m}|\nabla_x \Phi[n]|^2}
\]
 is $k[n]=\sigma(1+\ln n)+ \dfrac{q}{m}\Phi[n]$. For any concentration $n,n_0 \geq 0$ we have
\begin{align*}
\calE [n] - \calE [n_0] - \intxt{k[n_0](n-n_0)} &= \sigma \intxt{n_0 \left(\dfrac{n}{n_0}\ln \dfrac{n}{n_0} -\dfrac{n}{n_0} +1 \right)}\\
&+\dfrac{\epsilon_0}{2m}\intxt{|\nabla_x \Phi[n] - \nabla_x \Phi[n_0]|^2} \geq 0,
\end{align*}
with equality iff $n=n_0$.
\end{lemma}
%%
\begin{proof}$\;$\\
By direct computations one gets
\begin{align*}
\calE [n] &-\calE [n_0] - \intxt{k[n_0](n-n_0)} \\
&= \sigma \intxt{\left\{ n\ln n -n_0 \ln n_0 - (1+ \ln n_0)(n-n_0) \right\}}\\
&+ \intxt{\left\{ \dfrac{\epsilon_0}{2m}|\nabla_x \Phi[n]|^2 - \dfrac{\epsilon_0}{2m}|\nabla_x \Phi[n_0]|^2 - \dfrac{q}{m}\Phi[n_0](n-n_0)\right\}}\\
&= \sigma \intxt{\left\{n\ln n -n +n_0 - n\ln n_0 \right\}}\\
&+ \intxt{\left\{ \dfrac{\epsilon_0}{2m}|\nabla_x \Phi[n]|^2 - \dfrac{\epsilon_0}{2m}|\nabla_x \Phi[n_0]|^2 - \dfrac{\epsilon_0}{m}\nabla_x\Phi[n_0]\cdot(\nabla_x\Phi[n]-\nabla_x\Phi[n_0])\right\}}\\ 
&= \sigma \intxt{n_0 \left( \dfrac{n}{n_0}\ln \dfrac{n}{n_0} -\dfrac{n}{n_0} +1 \right)} + \dfrac{\epsilon_0}{2m}\intxt{|\nabla_x \Phi[n] - \nabla_x \Phi[n_0]|^2} \geq 0
\end{align*}
which equality iff $n=n_0$. Obviously we have
\begin{align*}
&\lim_{h\to 0} \dfrac{\calE [n_0 + hz]-\calE[n_0]-h\intxt{k[n_0]z}}{h}\\
&= \lim_{h\to 0} \dfrac{\sigma}{h}\intxt{n_0\left( \dfrac{n_0 + hz}{n_0}\ln \dfrac{n_0 +hz}{n_0} -\dfrac{n_0 +hz}{n_0} +1 \right)} + \lim_{h\to 0} \dfrac{\epsilon_0}{2mh}\intxt{h^2|\nabla_x\Phi[z]|^2}=0,
\end{align*}
saying that $\lim_{h\to 0} h^{-1}(\calE [n_0 + hz]- \calE[n_0]) = \intxt{k[n_0]z}$.
\end{proof}
%%
Thanks to the previous lemma we deduce that there is at most one concentration $n$ with a given average, such that $Be\cdot \nabla_x k[n] =0$.
%%
\begin{lemma}$\;$\\
\label{Uniqueness}
Let $n_1, n_2$ be two concentrations such that $\left< n_1\right> = \left< n_2 \right> $  and $Be\cdot\nabla_x k[n_1] = Be\cdot \nabla_x k[n_2]$. Therefore we have $n_1 = n_2$. In particular, for a given average, there is at most one concentration $n$ such that $Be\cdot \nabla_x k[n]=0$.
\end{lemma}
%%
\begin{proof}$\;$\\
We have by Lemma \ref{FirstVar}
\begin{align*}
\calE [n_1] - \calE [n_2] - \intxt{k[n_2](n_1 - n_2)} &= \sigma \intxt{n_2 \left(\dfrac{n_1}{n_2}\ln \dfrac{n_1}{n_2} -\dfrac{n_1}{n_2} +1 \right)}\\
&+\dfrac{\epsilon_0}{2m}\intxt{|\nabla_x \Phi[n_1] - \nabla_x \Phi[n_2]|^2},
\end{align*}
and
\begin{align*}
\calE [n_2] - \calE [n_1] - \intxt{k[n_1](n_2 - n_1)} &= \sigma \intxt{n_1 \left(\dfrac{n_2}{n_1}\ln \dfrac{n_2}{n_1} -\dfrac{n_2}{n_1} +1 \right)}\\
&+\dfrac{\epsilon_0}{2m}\intxt{|\nabla_x \Phi[n_2] - \nabla_x \Phi[n_1]|^2},
\end{align*}
implying that 
\begin{align*}
\intxt{(k[n_1]-k[n_2])(n_1 - n_2)} = \sigma \intxt{(n_1 -n_2)\ln \left(\dfrac{n_1}{n_2} \right)} +\dfrac{\epsilon_0}{m}\intxt{|\nabla_x \Phi[n_2] - \nabla_x \Phi[n_1]|^2}.
\end{align*}
Since $Be\cdot \nabla_x (k[n_1]-k[n_2]) =0$, $\left< n_1 - n_2 \right> =0$, we deduce
\[
\intxt{(k[n_1]-k[n_2])(n_1 - n_2)} =0,
\]
and thus $n_1 = n_2$.
\end{proof}
%%
If $n$ is such that $Be\cdot \nabla_x k[n] = 0$, then for any concentration $\bar{n}$ having the same average as $n$ we have
\[
\calE [\bar{n}] \geq \calE [n] + \intxt{k[n](\bar{n} - n)} = \calE [n],
\]
saying that for any given average $a$, the unique concentration $n$ such that $\left< n \right> =a$ and $Be\cdot\nabla_x k[n]=0$, satisfies
\[
\calE [n] = \min_{\left< \bar{n}\right>=a}\calE[\bar{n}].
\]
We denote by $F$ the application which maps $a\in \mathrm{ker}(Be\cdot\nabla_x)$ to $n$ such that $\left< n \right> =a$, $Be\cdot\nabla_x k[n] =0$.
%%
\begin{lemma}$\;$\\
\label{FirstVarBis}
The application $a\in \mathrm{ker}(Be\cdot\nabla_x) \to \calE [n=F(a)] $ is convex and its first variation is $a\to k[n=F(a)]$.
\end{lemma}
%%
\begin{proof}$\;$\\
Consider $a_1, a_2\in\mathrm{ker}(Be\cdot\nabla_x) $ and $\lambda_1, \lambda_2\in[0,1]$ such that $\lambda_1 + \lambda_2 =1$. We have
\[
\lambda_1 \calE [F(a_1)] + \lambda_2 \calE [F(a_2)] \geq \calE [\lambda_1 F(a_1) + \lambda_2 F(a_2)]
\]
since $\calE$ is convex and 
\[
\calE [F(\lambda_1 a_1 + \lambda_2 a_2)] = \min_{\left< \bar{n}\right> = \lambda_1 a_1 + \lambda_2 a_2}\calE [\bar{n}] \leq \calE [\lambda_1 F(a_1) + \lambda_2 F(a_2)]
\]
because
\[
\left< \lambda_1 F(a_1) + \lambda_2 F(a_2)  \right> = \lambda_1 \left< F(a_1) \right> + \lambda_2 \left< F(a_2)\right> = \lambda_1 a_1 + \lambda_2 a_2.
\]
Consider now $a, z\in \mathrm{ker}(Be\cdot\nabla_x)$ and $h\in\R$. The convexity of $\calE$ implies
\begin{align*}
\calE [F(a+hz)] - \calE[F(a)] &\geq \intxt{k[F(a)][F(a+hz)-F(a)]}\\
&= \intxt{k[F(a)]\left< F(a+hz)-F(a) \right>}\\
&= \intxt{k[F(a)][\left< F(a+hz) \right> - \left< F(a) \right>]}\\
&= \intxt{k[F(a)][(a+hz) - a]}\\
&= h \intxt{k[F(a)]z(x)}.
\end{align*}
Passing to the limit when $h\searrow 0$ and $h\nearrow 0$ we deduce that
\[
\lim_{h\to 0}\dfrac{\calE[F(a+hz)]-\calE[F(a)]}{h} = \intxt{k[F(a)]z}.
\]
\end{proof}
Combining the results in Proposition \ref{ZeroAve}, Lemma \ref{FirstVarBis}, the limit model \eqref{equ:EquLimitKn}, \eqref{equ:EquConsSec3} becomes
\[
\partial_t a + \left<\Divx\left(\dfrac{ae}{\omega_c}\wedge \nabla_x k[F(a)]\right) \right> =0,\,\, n=F(a).
\]
As $k[F(a)]\in \mathrm{ker}(Be\cdot\nabla_x)$, we obtain by \eqref{equ:AveConsFlowY}
\[
\left<\Divx\left(\dfrac{ae}{\omega_c}\wedge \nabla_x k[F(a)]\right)\right> = \left<\rot_x\left( \dfrac{ae}{\omega_c}\right)\cdot \nabla_x k[F(a)] \right> = \left<\rot_x\left( \dfrac{ae}{\omega_c}\right) \right>\cdot\nabla_x k[F(a)],
\]
and therefore the previous limit model also writes
\begin{equation}
\label{equ:AveRot}
\partial_t a + \left< \rot_x\left( \dfrac{ae}{\omega_c}\right) \right>\cdot\nabla_x k[F(a)]=0,\,\, n=F(a).
\end{equation}
%%

\section{A commutation formula for angular vector fields}
\label{AngVectFields}
The last step will concern a commutation formula between the operators $\left<\cdot\right>$ and $\rot_x$. We establish this formula for the special class of vector fields which present angle variables. In particular, this formula will apply for tokamak magnetic fields. We start with a very simple example. Consider the vector field $b(y) \cdot\nabla_y = y_2\partial_{y_1} - y_1 \partial_{y_2}, y=(y_1,y_2)\in\R^2$, whose characteristic flow is $2\pi$-periodic
\begin{align*}
Y(s;y) = \calR(-s)y = \begin{pmatrix}
\cos s & \sin s\\
-\sin s &  \cos s
\end{pmatrix}y ,\,\,(s,y)\in\R\times\R^2.
\end{align*}
The gradient of any invariant function $\psi$, that is a function satisfying $\psi(Y(s;\cdot)) = \psi, s\in\R$, verifies
\begin{equation}
\label{equ:GradInvFunc}
{^t}\partial Y(s;\cdot)(\nabla\psi)(Y(s;\cdot)) = \nabla\psi,\,\, s\in\R.
\end{equation}
There are other vector fields verifying similar properties. Let us consider the angle $\theta = \theta(y)\in [0,2\pi[$ given by
\[
y_1 = |y|\cos\theta(y),\,\, y_2 = |y|\sin\theta(y),\,\, y\in\R^2 \backslash\left\{ (0,0)\right\}.
\]
The function $\theta$ is smooth in $D = \R^2 \backslash (\R_+ \times \left\{0\right\})$ and we have
\[
\nabla_y\theta = - \dfrac{(y_2,-y_1)}{|y|^2} = -\dfrac{b(y)}{|y|^2},\,\,y\in D.
\]
The function $\theta$ is discontinuous across $\R^\star _+ \times \left\{ 0 \right\}$
\[
\lim_{y_1\to z_1, y_2\searrow 0} \theta(y) = 0,\,\,\,\,\,\lim_{y_1\to z_1, y_2 \nearrow 0} \theta(y) =2\pi,\,\,z_1 >0,
\]
but its gradient, which is well defined on $D$ is the restriction of a smooth vector field on $\R^2 \backslash\left\{ (0,0)\right\}$
\[
\nu(y) = - \dfrac{(y_2,-y_1)}{|y|^2},\,\, y\in \R^2 \backslash\left\{ (0,0)\right\}.
\]
For any $y\in D$ and $|s|$ small enough we have
\[
\dfrac{\mathrm{d}}{\mathrm{d}s} \theta(Y(s;y)) = b(Y(s;y))\cdot (\nabla\theta)(Y(s;y)) =-1,
\]
implying that $\theta(Y(s;y)) = \theta(y) -s ,y\in D$ and $|s|$ small enough. Taking the gradient with respect to $y$ we obtain
\[
{^t}\partial Y(s;y)(\nabla\theta)(Y(s;y)) = \nabla\theta(y),
\]
or 
\begin{equation}
\label{equ:GradAngVect}
{^t}\partial Y(s;y)\nu(Y(s;y)) = \nu(y),\,\, y\in D,\,\, |s| \,\mathrm{small\,enough}.
\end{equation}
Actually it is easily seen that the previous formula holds true for any $y\in\R^2\backslash\left\{ (0,0)\right\}$ and $s\in\R$. The vector field $\nu$ also satisfies
\[
\mathrm{div}_y {^t}\nu (y) =0,
\]
but it is not the gradient of a smooth function $\tilde{\theta}$ on $\R^2\backslash\left\{ (0,0)\right\}$, because, in that case, for any $y\in\R^2\backslash\left\{ (0,0)\right\}$, we would obtain
\[
-1 = \dfrac{1}{2\pi}\intct{(b\cdot\nu)(Y(s;y))} = \dfrac{1}{2\pi}\intct{(b\cdot\nabla\tilde{\theta})(Y(s;y))} =  \dfrac{1}{2\pi}\intct{\dfrac{\mathrm{d}}{\mathrm{d}s}\tilde{\theta}(Y(s;y))} =0.
\]
Generally, given a smooth divergence free vector field $b\cdot\nabla_y$ in $\R^3$, with global characteristic flow $Y=Y(s;y), (s,y)\in\R\times\R^3$, we call angular vector field in $D\in\R^3$ any vector field $\nu\cdot\nabla_y$ satisfying
\[
b(y)\cdot \nu(y) = C,\,\, {^t}\partial Y(s;y)\nu(Y(s;y)) = \nu(y),\,\, \rot_y\nu =0,\,\,(s,y)\in\R\times D,
\]
for some constant $C\in \R^\star$, where $D$ is an open subset of $\R^3$, which is left invariant by the flow $\mathit{i.e.,}$ $Y(s;D)=D, s\in\R.$ We intend to establish the following commutation formula.
%%
\begin{pro}$\;$\\
\label{AngField}
Let us consider a vector field $b\cdot\nabla_y$ in $\R^3$ satisfying \eqref{equ:EquLipDiv}, \eqref{equ:EquGrowth} with $S$-periodic characteristic flow $Y=Y(s;y)$, $(s,y)\in \R\times\R^3$. We denote by $\eta \cdot \nabla_y$ the gradient of an invariant function with respect to the flow $Y$, or an angular vector field, in some open subset $D$ of $\R^3$, which is left invariant by the flow $Y$. Therefore, for any $C^1$ function $\alpha = \alpha(y)$, we have
\begin{equation}
\label{equ:AngField}
\left< \nabla_y \alpha \wedge \eta \right> = \nabla_y \left< \alpha\right>\wedge \eta\,\, \mathrm{in}\,\, D.
\end{equation}
In particular, if $\alpha\in\mathrm{ker}(b\cdot\nabla_y)$, then $( \nabla_y \alpha \wedge \eta )\cdot \nabla_y$ is in involution with respect to $b\cdot\nabla_y$ in $D$.
\end{pro}
%%
We will use the following lemmas.
%%
\begin{lemma}$\;$\\
\label{VectProd}
We denote by $M[e]$ the matrix of the linear transformation $v \to e\wedge v, v\in\R^3$, that is $M[e]v = e \wedge v, v\in\R^3$. For any $e\in \mathbb{S}^2$, and $\xi, \eta\in \R^3$ such that $\xi\cdot e =0$, we have
\[
\xi\wedge \eta = (e\otimes M[e]\xi - M[e]\xi\otimes e)\eta.
\]
\end{lemma}
%%
\begin{proof}$\;$\\
By direct computations one gets
\begin{align*}
(e\otimes M[e]\xi - M[e]\xi\otimes e)\eta &= ((e\wedge \xi)\cdot\eta)e - (e\cdot\eta)e\wedge \xi \\
&= ((\xi\wedge \eta)\cdot e)e + (\eta\cdot e)\xi\wedge e \\
&= e\otimes e(\xi\wedge (\eta-(\eta\cdot e)e)) + (\eta\cdot e)\xi\wedge e \\
&= \xi\wedge (\eta-(\eta\cdot e)e) + (\eta\cdot e)\xi\wedge e \\
&= \xi \wedge \eta,
\end{align*}
where we have used that $\xi \wedge (\eta - (\eta\cdot e)e) \in \R e$, since $\xi \cdot e =0$.
\end{proof}
%%
For any function or vector field, the notation $F_s$ stands for $F\circ Y(s;\cdot)$.
\begin{lemma}$\;$\\
\label{Identity}
Let us consider a vector field $b\cdot \nabla_y$ in $\R^3$ satisfying \eqref{equ:EquLipDiv}, \eqref{equ:EquGrowth} with $S$-periodic characteristic flow $Y=Y(s;y)$, $(s,y)\in \R\times\R^3$. We denote by $M[e]$ the matrix of the linear transformation $v \to e\wedge v, v\in\R^3$, that is $M[e]v = e \wedge v, v\in\R^3$. Then, for any function u such that $u\in \mathrm{ker}(b\cdot\nabla_y)$,
we have the equality
\[
(I_3 -e_s\otimes e_s)\dfrac{\partial Y(s;\cdot)M[e]{^t}\partial Y(s;\cdot)}{|b|}(I_3 -e_s\otimes e_s)(\nabla u)_s = \dfrac{M[e_s]}{|b_s|} (\nabla u)_s,\,\, e=\dfrac{b}{|b|}.
\]
\end{lemma}
%%
\begin{proof}$\;$\\
For any invariant functions $\alpha = \alpha(y), \beta = \beta(y)$ with respect to the flow $Y$ we have $\nabla_y \alpha \wedge \nabla_y \beta \in \R e$ and $\mathrm{div}_y (\nabla_y \alpha \wedge \nabla_y \beta) = 0$. Therefore there is $\lambda \in \mathrm{ker}(b\cdot \nabla_y)$ such that $\nabla_y \alpha \wedge \nabla_y \beta = \lambda b$, saying that the vector field $\nabla_y \alpha \wedge \nabla_y \beta$ is in involution with respect to $b\cdot\nabla_y$. We have
\[
\partial Y(s;\cdot) \nabla \alpha \wedge \nabla \beta = (\nabla \alpha)_s \wedge (\nabla \beta)_s.
\]
Therefore, by Lemma \ref{VectProd} we obtain
\[
\partial Y(s;\cdot)(e\otimes M[e]\nabla \alpha - M[e]\nabla \alpha\otimes e)\nabla \beta = (e_s\otimes M[e_s](\nabla \alpha)_s - M[e_s](\nabla \alpha)_s\otimes e_s)(\nabla \beta)_s ,
\]
which reduce, thanks to the equalities $e\cdot \nabla\beta =0$, $e_s\cdot (\nabla\beta)_s =0$ to
\[
\partial Y(s;\cdot)(e\otimes M[e]\nabla\alpha)\nabla\beta = (e_s\otimes M[e_s](\nabla \alpha)_s)(\nabla\beta)_s .
\]
As $\alpha$ and $\beta$ are left invariant by the flow $Y$, we have
\[
\nabla \alpha = \nabla (\alpha_s) = {^t}\partial Y(s;\cdot)(\nabla\alpha)_s, \,\, \nabla\beta = \nabla(\beta_s) = {^t}\partial Y(s;\cdot)(\nabla\beta)_s ,
\]
implying that 
\[
\partial Y(s;\cdot)(e\otimes M[e]{^t}\partial Y(s;\cdot)(\nabla\alpha)_s){^t}\partial Y(s;\cdot)(\nabla\beta)_s = (e_s\otimes M[e_s](\nabla \alpha)_s)(\nabla\beta)_s .
\]
Observe that
\[
\partial Y(s;\cdot) e = \dfrac{\partial Y(s;\cdot)b}{|b|} = \dfrac{b_s}{|b|}=\dfrac{|b_s|}{|b|}e_s,\,\,\mathrm{since}\, [b,b]=0 ,
\]
and therefore we obtain
\[
\left( e_s \otimes \dfrac{\partial Y(s;\cdot)M[e]{^t}\partial Y(s;\cdot)}{|b|}(\nabla\alpha )_s  \right)(\nabla\beta)_s = \left( e_s \otimes \dfrac{M[e_s](\nabla\alpha)_s}{|b_s|} \right)(\nabla\beta)_s ,
\]
or equivalently
\[
\left(\dfrac{\partial Y(s;\cdot)M[e]{^t}\partial Y(s;\cdot)}{|b|} -  \dfrac{M[e_s]}{|b_s|} \right) (\nabla \alpha)_s \in \R e_s .
\]
Finally we have 
\[
(I_3 - e_s \otimes e_s) \left(\dfrac{\partial Y(s;\cdot)M[e]{^t}\partial Y(s;\cdot)}{|b|} -  \dfrac{M[e_s]}{|b_s|} \right) (\nabla \alpha)_s  =0 ,
\]
for any invariant function $\alpha$, and our conclusions follows.
\end{proof}
%%
\begin{lemma}$\;$\\
\label{ZeroAveVectField}
Let us consider a vector field $b\cdot\nabla_y$ in $\R^3$ satisfying \eqref{equ:EquLipDiv}, \eqref{equ:EquGrowth} with $S$-periodic characteristic flow $Y=Y(s;y),(s,y)\in\R\times\R^3$, which possesses angular vector field $\nu$ in some invariant open subset $D \subset \R^3$. A vector field $c\cdot\nabla_y$ has zero average in $D$ iff $\left< c\cdot\nu \right> =0$ in $D$ and $\left<  c\cdot \nabla_y u\right> =0$ in $D$ for any function $u$ such that $\mathds{1}_D u \in\mathrm{ker}(b\cdot\nabla_y)$.
\end{lemma}
%%
\begin{proof}$\;$\\
By formula \eqref{equ:AveConsFlowY} we know that for any function $u$ which is left invariant by $Y$ in $D$, we have $\left<c\cdot\nabla_y u\right>= \left< c\right>\cdot\nabla_y u$ in $D$. Similarly, for any $y\in D$ we write
\begin{align*}
\left< c\cdot\nu\right>(y) &= \dfrac{1}{S}\intst{c(Y(s;y))\cdot \nu(Y(s;y))}\\
&= \dfrac{1}{S} \intst{\partial Y(-s;Y(s;y)) c(Y(s;y)) \cdot {^t} \partial_y Y(s;y)\nu(Y(s;y))} \\
&= \dfrac{1}{S} \intst{\partial Y(-s;Y(s;y)) c(Y(s;y))}\cdot \nu(y) \\
&= \left< c\right>(y) \cdot \nu(y).
\end{align*}
Clearly, if $\left< c\right> =0$ in $D$, then $\mathds{1}_D \left< c\cdot \nabla_y u \right> =0$ for any function $u$ such that $\mathds{1}_D u\in \mathrm{ker}(b\cdot\nabla_y)$ and $\mathds{1}_D \left< c \cdot \nu\right> =0$. Conversely, if $\mathds{1}_D \left< c \cdot \nabla_y u\right> =0$ for any function $u$ such that $\mathds{1}_D u \in\mathrm{ker}(b\cdot\nabla_y)$ and $\mathds{1}_D \left< c \cdot \nu\right> =0$, then $\mathds{1}_D \left< c\right>\cdot\nabla_y u =0$, $\mathds{1}_D \left< c\right>\cdot \nu =0$. We deduce that there is a function $\lambda = \lambda(y)$ in $D$ such that
\[
\left< c\right>(y)= \lambda(y)b(y),\,\, y\in D.
\]
Taking the scalar product by $\nu(y), y\in D$, we obtain
\[
0 = \left< c\right>(y)\cdot \nu(y) = \lambda(y) b(y)\cdot \nu(y) = \lambda(y) C,\,\, y\in D.
\]
Since $C\in \R^\star$, we deduce that $\lambda$ vanishes in $D$ and $\mathds{1}_D\left< c\right> =0$.
\end{proof}
%%
We are ready to prove the commutation formula \eqref{equ:AngField}.
\begin{proof}(of Proposition \ref{AngField})$\;$\\
All the computations are performed in $D$.\\
 We assume for the moment that $\alpha\in \mathrm{ker}(b\cdot\nabla_y)$ and we prove that $\nabla \alpha \wedge \eta$ is in involution with respect to $b\cdot\nabla_y$. We have by Lemma \ref{VectProd} and Lemma \ref{Identity}
\begin{align*}
\partial Y(s;\cdot)&(\nabla \alpha \wedge \eta) = \partial Y(s;\cdot)(e\otimes M[e]\nabla\alpha - M[e]\nabla\alpha\otimes e)\eta \\
&= \left[ b_s \otimes  \dfrac{\partial Y(s;\cdot)M[e]{^t}\partial Y(s;\cdot)}{|b|}(\nabla\alpha )_s - \dfrac{\partial Y(s;\cdot)M[e]{^t}\partial Y(s;\cdot)}{|b|}(\nabla\alpha )_s \otimes b_s \right]\eta_s \\
&= \left[ b_s \otimes (I_3 - e_s\otimes e_s)\dfrac{\partial Y(s;\cdot)M[e]{^t}\partial Y(s;\cdot)}{|b|}(I_3 - e_s\otimes e_s)(\nabla\alpha )_s  \right.\\
&\left.  - (I_3 - e_s\otimes e_s)\dfrac{\partial Y(s;\cdot)M[e]{^t}\partial Y(s;\cdot)}{|b|}(I_3 - e_s\otimes e_s) (\nabla\alpha )_s \otimes b_s \right] \eta_s \\
&= (e_s \otimes M[e_s](\nabla \alpha)_s - M[e_s](\nabla \alpha)_s\otimes e_s)\eta_s \\
&= (\nabla\alpha)_s \wedge \eta_s,
\end{align*}
where we have used that 
\[
\left[ b_s \otimes ( - e_s\otimes e_s)\dfrac{\partial Y(s;\cdot)M[e]{^t}\partial Y(s;\cdot)}{|b|}(\nabla\alpha )_s  
  + ( e_s\otimes e_s)\dfrac{\partial Y(s;\cdot)M[e]{^t}\partial Y(s;\cdot)}{|b|} (\nabla\alpha )_s \otimes b_s \right] \eta_s =0.
\]
Assume now that $\left< \alpha \right> =0$ and we prove that $\left<\nabla\alpha \wedge \eta \right>=0$. 
If $\eta = \nabla \beta$ for some function $\beta$, which is left invariant by $Y$ in $D$ we have
\begin{align*}
&\partial Y (-s;Y(s;\cdot))\eta_s \wedge (\nabla\alpha)_s \\
&= \partial Y (-s;Y(s;\cdot))(e_s\otimes M[e_s](\nabla\beta)_s - M[e_s](\nabla\beta)_s\otimes e_s)(\nabla\alpha)_s\\
&= \left( b \otimes  \dfrac{\partial Y(-s;Y(s;\cdot))M[e_s]{^t}\partial Y(-s;Y(s;\cdot))}{|b_s|}\nabla\beta \right. \\
&\left. - \dfrac{\partial Y(-s;Y(s;\cdot))M[e_s]{^t}\partial Y(-s;Y(s;\cdot))}{|b_s|}\nabla\beta \otimes b \right)\nabla(\alpha_s) \\
&= \left[ b \otimes (I_3 - e\otimes e)\dfrac{\partial Y(-s;Y(s;\cdot))M[e_s]{^t}\partial Y(-s;Y(s;\cdot))}{|b_s|}(I_3 - e\otimes e)\nabla\beta  \right.\\
&\left.  - (I_3 - e\otimes e)\dfrac{\partial Y(-s;Y(s;\cdot))M[e_s]{^t}\partial Y(-s;Y(s;\cdot))}{|b_s|}(I_3 - e\otimes e) \nabla\beta \otimes b \right] \nabla\alpha_s  \\
&= \left( b\otimes \dfrac{M[e]}{|b|}\nabla\beta - \dfrac{M[e]}{|b|} \nabla\beta\otimes b\right)\nabla\alpha_s\\
&= (e \otimes M[e]\nabla\beta - M[e]\nabla\beta \otimes e)\nabla\alpha_s\\
&= \nabla\beta \wedge \nabla\alpha_s,
\end{align*}
where we used the following formulas in the calculations above
\[
\partial Y(s;\cdot) b = b_s \Leftrightarrow b = \partial Y(-s;Y(s,\cdot))b_s ,
\]
and 
\[
\nabla(\alpha_s) = {^t}\partial Y(s;\cdot)(\nabla\alpha)_s \Leftrightarrow {^t}\partial Y(-s;Y(s;\cdot)) \nabla(\alpha_s) = (\nabla\alpha)_s,
\]
\[
\nabla\beta = \nabla(\beta_s) = {^t}\partial Y(s;\cdot)(\nabla\beta)_s \Leftrightarrow {^t}\partial Y(-s;Y(s;\cdot)) \nabla\beta = (\nabla\beta)_s.
\]
We obtain
\[
\left< \nabla\beta \wedge \nabla\alpha \right> = \dfrac{1}{S}\intst{\nabla\beta \wedge \nabla\alpha _s} = \nabla\beta \wedge \nabla \left< \alpha \right> =0.
\]
If $\eta$ is an angular vector field $\nu$ in $D$, we appeal to Lemma \ref{ZeroAveVectField}. Obviously we have $\left< (\nabla\alpha\wedge \nu)\cdot \nu\right> =0$ and for any function $u$ such that $\mathds{1}_D u\in\mathrm{ker}(b\cdot\nabla_y)$, we can write since $(\nabla_y u\wedge \nu)\cdot \nabla_y$ is in involution with $b\cdot\nabla_y$ in $D$ cf. the first part of this proof, and thanks to \eqref{equ:AveInvoFieldb}
\[
\left< (\nabla_y \alpha\wedge \nu)\cdot \nabla_y u\right> = - \left< (\nabla_y u \wedge \nu )\cdot \nabla_y \alpha\right> = - (\nabla_y u \wedge \nu )\cdot \nabla_y \left<\alpha\right> =0.
\]
Therefore we deduce that
\[
\left< \nabla\alpha \wedge \nu \right> =0.
\]
Finally, for any function $\alpha$ we have
\begin{align*}
\left< \nabla\alpha \wedge \eta \right> &= \left< \nabla \left<\alpha\right> \wedge \eta \right> + \left< \nabla(\alpha -\left<\alpha\right>)  \wedge \eta \right> \\&= \dfrac{1}{S}\intst{\partial Y(-s;Y(s;\cdot))(\nabla \left<\alpha\right>)_s \wedge \eta_s}\\
&= \dfrac{1}{S} \intst{\partial Y(-s;Y(s;\cdot))\partial Y(s;\cdot)(\nabla\left<\alpha\right> \wedge \eta)}\\
&= \dfrac{1}{S} \intst{\nabla\left<\alpha\right> \wedge \eta}\\& = \nabla\left<\alpha\right> \wedge \eta.
\end{align*}
\end{proof}
%%
\section{Tokamak magnetic fields}
\label{TokMagField}
In this section we apply the previous results to some examples of magnetic fields. We start by a simplified framework, that of a magnetic field, whose magnetic lines wind on cylindrical surfaces.
%%
\subsection{Cylindrical case}
\label{Cylin}
We consider the magnetic field $\textbf{B} = Be = B_0 \left( \dfrac{x_2}{R_0},-\dfrac{x_1}{R_0},1 \right)$, $x=(x_1,x_2,x_3)=(\bar{x},x_3)\in \R^3$, where $B_0,R_0$ are some reference values for the magnetic field and length. The characteristic flow is given by
\[
(\bar{X}(s;\bar{x}),X_3(s;x_3)) = \left( \calR\left(-s \dfrac{B_0}{R_0}\right)\bar{x},x_3 + sB_0\right),\,\,(s,\bar{x},x_3)\in \R\times\R^3 ,
\]
where 
\[
\calR(\theta) = \begin{pmatrix}
\cos \theta & -\sin \theta\\
\sin \theta &  \cos \theta
\end{pmatrix},\,\,\theta\in\R.
\]
We have two angular vector fields
\[
\nu_\theta = \dfrac{(x_2,-x_1,0)}{x_1 ^2 + x_2 ^2},\, \bar{x}\neq 0,\,\, \nu_{\parallel} = (0,0,1).
\]
All the functions are supposed periodic with respect to $x_3$. Taking $S=2\pi R_0/B_0$, we define the average operator for a function $u$ by
\[
\left< u \right>(x) = \dfrac{1}{S}\intst{u(\bar{X}(s;\bar{x}),X_3(s;x_3))} = \dfrac{1}{S}\intst{u\left(\calR\left(-s \dfrac{2\pi}{S}\right)\bar{x}, x_3 + s\dfrac{2\pi}{S}R_0 \right)},
\]
and for a vector field $c\cdot\nabla_x = \bar{c}\cdot\nabla_{\bar{x}} + c_3 \partial x_3$ by
\begin{align*}
\left< c\right>(x) &= \dfrac{1}{S}\intst{\begin{pmatrix}
   \begin{array}{cr}
\calR(s\frac{2\pi}{S}) & \begin{matrix} 0\\ 0 \end{matrix} \\
  \begin{matrix} 0 && 0 \end{matrix} & 1
   \end{array}
\end{pmatrix}c\left(\calR\left(-s \dfrac{2\pi}{S}\right)\bar{x}, x_3 + s\dfrac{2\pi}{S}R_0 \right)}\\
&= \dfrac{1}{S}\intst{\begin{pmatrix} \calR(s\frac{2\pi}{S})\bar{c}\left(\calR\left(-s \frac{2\pi}{S}\right)\bar{x}, x_3 + s\frac{2\pi}{S}R_0 \right)\\
c_3\left(\calR\left(-s \frac{2\pi}{S}\right)\bar{x}, x_3 + s\frac{2\pi}{S}R_0 \right)
\end{pmatrix}}.
\end{align*}
We use the following decomposition of $Be\cdot\nabla_x$
\[
Be = \dfrac{B_0}{R_0}|\bar{x}|^2 \nu_{\theta} + B_0 \nu_{\parallel},\,\, |\bar{x}|>0.
\]
Thanks to Proposition \ref{AngField}, we compute the term $\left< \rot_x\left( \dfrac{ae}{\omega_c}\right) \right>$ appearing in the limit model \eqref{equ:AveRot}. Observe that
\[
\rot_x\left( \dfrac{ae}{\omega_c}\right) = \rot_x\left[ \dfrac{a}{B\omega_c}\left(\dfrac{B_0}{R_0}|\bar{x}|^2 \nu_{\theta} + B_0 \nu_{\parallel} \right) \right] = \nabla_x \left( \dfrac{aB_0|\bar{x}|^2}{B\omega_cR_0}\right)\wedge \nu_\theta + \nabla_x \left( \dfrac{aB_0}{B\omega_c}\right)\wedge \nu_\parallel ,
\]
and therefore
\begin{align*}
\left< \rot_x\left( \dfrac{ae}{\omega_c}\right) \right> &= \nabla_x \left<\dfrac{aB_0 |\bar{x}|^2}{B\omega_c R_0} \right> \wedge \nu_\theta + \nabla_x \left<\dfrac{aB_0 }{B\omega_c } \right>\wedge \nu_\parallel \\
&= \nabla_x \left( \dfrac{aB_0 |\bar{x}|^2}{B\omega_c R_0} \right)\wedge \nu_\theta + \nabla_x \left(\dfrac{aB_0 }{B\omega_c } \right)\wedge \nu_\parallel  =  \rot_x\left( \dfrac{ae}{\omega_c}\right),
\end{align*}
since the functions $a$, $B\omega_c$ and $|\bar{x}|^2$ belong to $\mathrm{ker}(Be\cdot\nabla_x)$. We obtain
\[
\left< \rot_x\left( \dfrac{ae}{\omega_c}\right) \right> \cdot \nabla_x k[F(a)] = \rot_x\left( \dfrac{ae}{\omega_c}\right)\cdot \nabla_x k[F(a)] = \Divx\left(\dfrac{ae}{\omega_c}\wedge \nabla_x k[F(a)] \right).
\]
In that case, the vector field $\rot_x\left(\frac{ae}{\omega_c} \right)$ is in involution with $Be\cdot\nabla_x$, and \eqref{equ:AveRot} becomes
\[
\partial_t a + \Divx\left(\dfrac{ae}{\omega_c}\wedge \nabla_x k[F(a)] \right) =0,\,\, n =F(a).
\]
In this case we work in the $2\pi R_0$-periodic domain with respect to $x_3$, $\R^2 \times \mathbb{T}^1$, where $\T^1 = \R/(2\pi R_0 \Z)$. The potential $\Phi$ solves the Poisson equation
\[
-\epsilon_0 \Delta_x \Phi = qn,\,\, x\in \R^2\times\T^1,
\]
with the boundary condition
\[
\lim_{|\bar{x}|\to\infty}\Phi(\bar{x},x_3)=0,\,\, x_3\in \T^1.
\]
The Jacobian matrix of the flow $X(s,x) = (\bar{X}(s;\bar{x}),X_3(s;x_3))$ is orthogonal
\[
\partial_x X(s;x) = \begin{pmatrix}
   \begin{array}{cr}
\calR(-s\frac{B_0}{R_0}) & \begin{matrix} 0\\ 0 \end{matrix} \\
  \begin{matrix} 0 &&´ 0 \end{matrix} & 1
   \end{array}
\end{pmatrix},
\]
which implies that the Laplace operator commutes with the translations along the flow, that is
\[
\Delta_x u_s = (\Delta_x u)_s ,
\]
for any smooth function $u$. Indeed, for any $\psi \in C^1_c(\R^2\times \T^1)$ we have
\begin{align*}
\intxlt{\Delta_x u_s \psi_s} &= - \intxlt{\nabla_x u_s \cdot \nabla_x \psi_s}\\
&= - \intxlt{{^t}\partial X(s;x)(\nabla u)_s \cdot {^t}\partial X(s;x)(\nabla \psi)_s } \\
&= - \intxlt{(\nabla u )_s\cdot (\nabla \psi)_s}\\
&= - \intxlt{\nabla u \cdot \nabla \psi}\\
&= \intxlt{\Delta_x u \,\psi}\\
&= \intxlt{(\Delta_x u)_s \psi_s},
\end{align*}
saying that $\Delta_x u_s = (\Delta_x u)_s$. If $\Phi[n]$ is the potential corresponding to the $2\pi R_0$-periodic concentration $n$ with respect to $x_3$, then
\[
-\epsilon_0 \Delta_x(\Phi[n])_s = -\epsilon_0 (\Delta_x \Phi[n])_s = q n_s ,
\]
for any $x_3$ we have
\[
\lim_{|\bar{x}|\to +\infty} \Phi[n](X(s;x)) = \lim_{|\bar{x}|\to +\infty}\Phi[n](\bar{X}(s;\bar{x}),X_3(s;x_3)) =0,\,\, \mathrm{because}\, |\bar{X}(s,\bar{x})| = |\bar{x}|,
\]
and $(\Phi[n])_s$ is $2\pi R_0$-periodic with respect to $x_3$
\begin{align*}
\Phi[n](\bar{X}(s;\bar{x}),X_3(s;x_3 +2\pi R_0 )) &= \Phi[n](\bar{X}(s;\bar{x}),X_3(s;x_3) +2\pi R_0 )\\
&= \Phi[n](\bar{X}(s;\bar{x}),X_3(s;x_3 )) = (\Phi[n])_s (x).
\end{align*}
Therefore we have $(\Phi[n])_s = \Phi[n_s]$. In particular, if $n\in \mathrm{ker}(Be\cdot\nabla_x)$ then $\Phi[n]\in \mathrm{ker}(Be\cdot\nabla_x)$. By construction $n=F(a)$ is the unique concentration such that $\left< n\right> =a$, $Be\cdot\nabla_x k[n] =0$. Clearly we have $\left< a\right> =a$ and $k[a] = \sigma(1 + \ln a) + \frac{q}{m}\Phi[a]\in \mathrm{ker}(Be\cdot\nabla_x)$ and thus $n=F(a)=a$ for any $a \in \mathrm{ker}(Be\cdot\nabla_x)$. The constraint in \eqref{equ:EquConsSec3} is automatically satisfied. In that case, our limit model simply writes
\begin{equation}
\label{equ:EquCyl1}
\partial_t n + \Divx\left( \dfrac{n e}{\omega_c}\wedge \nabla_x k[n] \right) =0,\,\, (t,x)\in \R_+ \times \R^2\times \T^1.
\end{equation}
%%
\begin{remark}$\;$\\
\label{PropaCons}
 The equation \eqref{equ:EquCyl1} propagates the constraint $Be\cdot\nabla_x n =0$.  When the magnetic field is uniform $\mathit{i.e.,}$ $Be = (0,0,1)$, it is not difficult to check that if $Be\cdot\nabla_x n (t,x) = 0$ holds at $t=0$, then it will do so for all time in which the solution exists. Thus, the constraint \eqref{equ:EquConsSec3} can be understood as a mere constraint on the initial data.
\end{remark}
Since we know that at any time $t$, $n(t)$ belongs to $\mathrm{ker}(Be\cdot\nabla_x)$, we can reduce the above model to a two dimensional problem. We appeal to the invariants of the flow $X$
\[
\calR\left( \dfrac{X_3(s;x_3)}{R_0} \right)\bar{X}(s;\bar{x}) = \calR \left( \dfrac{x_3 +  s B_0}{R_0}\right)\calR \left( -s\dfrac{B_0}{R_0}\right)\bar{x} = \calR \left(\dfrac{x_3}{R_0} \right)\bar{x}.
\]
We introduce the new unknown function $N=N(t,\bar{y} =(y_1,y_2))$ such that
\[
n(t,x) =  N(t,\bar{y} = \calR(x_3 / R_0 )\bar{x}),
\]
and we are looking for the model satisfied by $N=N(t,\bar{y})$.
%%
\begin{lemma}$\;$\\
\label{Laplace}
Let us consider a smooth function $U = U(\bar{y}),\bar{y}\in\R^2$, and $u(x) =U(\calR (x_3/ R_0)\bar{x})$, $x\in \R^2 \times \T^1$. We have
\[
\Delta_x u = \left[ \mathrm{div}_{\bar{y}}\left( I_2 + \dfrac{^\perp \bar{y} \otimes {^\perp}\bar{y}}{R_0 ^2} \right)\nabla_{\bar{y}}U \right](\bar{y}= \calR(x_3 /R_0)\bar{x}).
\]
\end{lemma}
\begin{proof}$\;$\\
Consider $\Psi \in C^1 _c(\R^2)$ and $\psi(x) = \Psi(\calR(x_3 /R_0)\bar{x})$, $x\in \R^2 \times \T^1$. Integrating by parts, thanks to the $x_3$-periodicity, one gets
\begin{align*}
\intxlt{\Delta_x u\, \psi(x)} &= - \intxlt{\nabla_x u \cdot \nabla_x \psi}\\
&= -\intxlt{\dfrac{^t \partial \bar{y}}{\partial x}(\nabla_{\bar{y}}U)(\calR(x_3 /R_0)\bar{x})\cdot \dfrac{^t \partial \bar{y}}{\partial x}(\nabla_{\bar{y}}\Psi)(\calR(x_3 /R_0)\bar{x})}\\
&= -\intxlt{\dfrac{ \partial \bar{y}}{\partial x}\dfrac{^t \partial \bar{y}}{\partial x}(\nabla_{\bar{y}}U)(\calR(x_3 /R_0)\bar{x})\cdot (\nabla_{\bar{y}}\Psi)(\calR(x_3 /R_0)\bar{x})},
\end{align*}
where $\frac{ \partial \bar{y}}{\partial x}$ is the Jacobian matrix of the apllication $x \to \calR(x_3 /R_0)\bar{x}$
\[
\dfrac{ \partial \bar{y}}{\partial x} = \left(\calR(x_3 /R_0), \calR(x_3 /R_0 + \pi/2)\dfrac{\bar{x}}{R_0}  \right)\in \calM_{2,3}(\R).
\]
The matrix product $\frac{ \partial \bar{y}}{\partial x}\frac{ ^t\partial \bar{y}}{\partial x}$ writes
\[
\frac{ \partial \bar{y}}{\partial x}\frac{ ^t\partial \bar{y}}{\partial x} = I_2 + \calR\left( \dfrac{x_3}{R_0}\right)\dfrac{^\perp \bar{x}}{R_0}\otimes  \calR\left( \dfrac{x_3}{R_0}\right)\dfrac{^\perp \bar{x}}{R_0},\,\, ^\perp\bar{x} = (x_2, -x_1),
\]
and we obtain
\begin{align*}
&\intxlt{\Delta_x u \, \psi(x)} \\
&= - \intxlt{\left[ I_2 + \calR\left( \dfrac{x_3}{R_0}\right)\dfrac{^\perp \bar{x}}{R_0}\otimes  \calR\left( \dfrac{x_3}{R_0}\right)\dfrac{^\perp \bar{x}}{R_0} \right](\nabla_{\bar{y}}U)(\calR(\dfrac{x_3}{R_0})\bar{x})\cdot (\nabla_{\bar{y}}\Psi)(\calR(\dfrac{x_3}{ R_0})\bar{x})}\\
&=- \int_{\T^1}{\int_{\R^2}{\left( I_2 + \dfrac{^\perp \bar{y} \otimes {^\perp} \bar{y}}{R_0 ^2} \right)\nabla_{\bar{y}}U(\bar{y})\cdot\nabla_{\bar{y}}\Psi(\bar{y})} }\mathrm{d}\bar{y}\,\mathrm{d}x_3\\
&= 2\pi R_0 \int_{\R^2}{\left( \mathrm{div}_{\bar{y}}\left( I_2 + \dfrac{^\perp \bar{y} \otimes {^\perp} \bar{y}}{R_0 ^2} \right)\nabla_{\bar{y}}U \right) \Psi(\bar{y})}\mathrm{d}\bar{y}\\
&= \intxlt{\left[ \mathrm{div}_{\bar{y}}\left( I_2 + \dfrac{^\perp \bar{y} \otimes {^\perp} \bar{y}}{R_0 ^2} \right)\nabla_{\bar{y}}U \right]\left(\bar{y} = \calR\left( \dfrac{x_3}{R_0}\right)\bar{x} \right)\psi(x)}.
\end{align*}
The previous computation shows that $\Delta_x u - \left[\mathrm{div}_{\bar{y}}\left( I_2 + \frac{^\perp \bar{y} \otimes {^\perp} \bar{y}}{R_0 ^2} \right)\nabla_{\bar{y}}U \right]\left(\bar{y} = \calR\left( \frac{x_3}{R_0}\right)\bar{x} \right) $  is orthogonal on $\mathrm{ker}(Be\cdot\nabla_x)$. But this function belongs to $\mathrm{ker}(Be\cdot\nabla_x)$, because $u$ belongs to $\mathrm{ker}(Be\cdot\nabla_x)$, together with $\Delta_x u$, since the Laplace operator commutes with the flow $X$. Finally we obtain
\[
\Delta_x u = \left[\mathrm{div}_{\bar{y}}\left( I_2 + \frac{^\perp \bar{y} \otimes {^\perp} \bar{y}}{R_0 ^2} \right)\nabla_{\bar{y}}U \right]\left(\bar{y} = \calR\left( \frac{x_3}{R_0}\right)\bar{x} \right).
\]
\end{proof}
%%
\begin{lemma}$\;$\\
\label{ConservLamw}
Let us consider two smooth functions $U= U(\bar{y}), W = W(\bar{y}), \bar{y}\in \R^2$ and $u(x) = U(\calR(x_3 /R_0)\bar{x})$, $w(x) = W(\calR(x_3 /R_0)\bar{x})$, $x\in \R^2 \times \T^1$. We have
\[
\Divx\left( \dfrac{u e}{\omega_c}\wedge\nabla_x w \right) =  \left[ \mathrm{div}_{\bar{y}}\left( \dfrac{U}{\omega_0}\calR(\pi/2)\nabla_{\bar{y}}W \right) \right]\left(\bar{y} = \calR\left( \frac{x_3}{R_0}\right)\bar{x} \right),\,\, \omega_0 =\dfrac{qB_0}{m}.
\]
\end{lemma}
\begin{proof}$\;$\\
As before, we perform the computation in distribution sense. We already know that the vector field $\rot_x\left(\frac{ue}{\omega_c} \right)\cdot\nabla_x$ is in involution with $Be\cdot\nabla_x$, and therefore 
\[
\Divx\left( \dfrac{u e}{\omega_c}\wedge\nabla_x w \right) = \rot_x\left(\dfrac{ue}{\omega_c} \right)\cdot\nabla_x w \in \mathrm{ker}(Be\cdot\nabla_x),
\]
it is enough to consider test functions $\psi(x) = \Psi(\calR(x_3 / R_0)\bar{x}), \Psi \in C^1_c(\R^2)$
\begin{align*}
\intxlt{&\Divx\left( \dfrac{u e}{\omega_c}\wedge\nabla_x w \right) \psi(x)} = - \intxlt{\left( \dfrac{u e}{\omega_c}\wedge\nabla_x w \right)  \cdot \nabla_x \psi(x)}\\
&=- \intxlt{\dfrac{u}{\omega_c}M[e]\nabla_x \omega \cdot \nabla_x \psi}\\
&= - \intxlt{\dfrac{U(\calR(x_3 /R_0)\bar{x})}{\omega_c} \dfrac{\partial\bar{y}}{\partial x}M[e]\dfrac{^t\partial\bar{y}}{\partial x}(\nabla_{\bar{y}}W)(\calR(x_3 /R_0)\bar{x})\cdot (\nabla_{\bar{y}}\Psi)(\calR(x_3 /R_0)\bar{x})}.
\end{align*}
By direct computations we obtain
\[
\dfrac{1}{\omega_c}\dfrac{\partial\bar{y}}{\partial x}M[e]\dfrac{^t\partial\bar{y}}{\partial x} = \dfrac{1}{\omega_0}\calR\left(\dfrac{\pi}{2}\right),\,\,\omega_0 =\dfrac{qB_0}{m},
\]
and therefore the previous calculations lead to
\begin{align*}
\intxlt{&\Divx\left( \dfrac{u e}{\omega_c}\wedge\nabla_x w \right) \psi(x)} = 2\pi R_0 \int_{\R^2}{\dfrac{U(\bar{y})}{\omega_0}{^\perp}\nabla_{\bar{y}}W\cdot \nabla_{\bar{y}}\Psi}\mathrm{d}\bar{y}\\
&= -2\pi R_0 \int_{\R^2}{\mathrm{div}_{\bar{y}}\left( \dfrac{U(\bar{y})}{\omega_0}{^\perp}\nabla_{\bar{y}}W\right)\Psi(\bar{y})}\mathrm{d}\bar{y}\\
&= - \intxlt{\left[ \mathrm{div}_{\bar{y}}\left( \dfrac{U(\bar{y})}{\omega_0}{^\perp}\nabla_{\bar{y}}W\right) \right]\left( \bar{y}= \calR\left( \dfrac{x_3}{R_0}\right)\right)\psi(x)}.
\end{align*}
We deduce that
\[
\Divx\left( \dfrac{u e}{\omega_c}\wedge\nabla_x w \right) =  \mathrm{div}_{\bar{y}}\left( \dfrac{U}{\omega_0}{^\perp}\nabla_{\bar{y}}W \right).
\]
\end{proof}
Combining Lemma \ref{Laplace} and Lemma \ref{ConservLamw} we derive the limit model with respect to the new unknown $N$. The potential $\Phi = \phi[n]$ writes $\phi(t,x) = \Phi(t,\bar{y}= \calR(x_3 /R_0)\bar{x})$ where $\Phi(t,\bar{y})$ solves the elliptic equation
\[
- \epsilon_0 \mathrm{div}_{\bar{y}}\left[ \left(  I_2 + \dfrac{{^\perp} \bar{y} \otimes {^\perp} \bar{y}}{R_0 ^2}\right)\nabla_{\bar{y}}\Phi(t,\bar{y})\right] = q N(t,\bar{y}),\,\, \bar{y}\in\R^2.
\] We supplement this elliptic equation by the condition $\lim_{|\bar{y}|\to +\infty}\Phi(t,\bar{y})=0$ and we denote by $\Phi[N]$ the solution corresponding to the concentration $N$. We introduce $K[N]=\sigma(1+\ln N) + \frac{q}{m}\Phi[N]$. The time evolution for the concentration $N$ is given by
\[
\partial_t N + \mathrm{div}_{\bar{y}}\left(\dfrac{N}{\omega_0}\calR\left(\dfrac{\pi}{2} \right)\nabla_{\bar{y}}K[N]  \right) =0,\,\,(t,\bar{y})\in\R_+ \times\R^2 ,
\]
and the initial condition
\[
N(0,\bar{y}) = N_{\mathrm{in}}(\bar{y}),\,\, \bar{y}\in \R^2 ,
\]
where $n_{\mathrm{in}}(x) = N_{\mathrm{in}}(\calR(x_3 /R_0)\bar{x}),x\in\R^2\times\T^1$.

%%
\subsection{The toroidal case}
\label{Toroidal}
We consider now a magnetic field whose magnetic lines wind on toroidal surfaces (called magnetic surface). We denote by $\varphi$ the toroidal angle in the plan $x_1 O x_2$, by $\theta$ the poloidal angle and $R_0$ is the mean radius of the torus, as shown in Figure \ref{fig:mesh1}
\[
x_1 = (R_0 + r\cos\theta)\cos\varphi,\,\, x_2 = (R_0 + r\cos\theta)\sin\varphi,\,\, x_3 = r\sin\theta.
\]
\begin{figure}[h]
    \centering
    \includegraphics[scale=0.4]{./Toroidal_coordinates.png}
    \caption{Toroidal and poloidal coordinates (Source: FusionWiki)}
    \label{fig:mesh1}
\end{figure}
The magnetic field writes cf. \cite{Lutz2013}
\[
Be = \dfrac{B_0 r}{f_q (R_0 + r\cos\theta)}e_\theta + B_0 e_\varphi,\,\, r < r_0 < R_0 ,
\]
where $e_\varphi$ and $e_\theta$ stand the unit vectors of toroidal and poloidal coordinates system
\[
e_\varphi = \dfrac{\frac{\partial}{\partial \varphi}}{\left|\frac{\partial}{\partial \varphi} \right|} = (-\sin\varphi,\cos\varphi,0)= \dfrac{(-x_2,x_1,0)}{\sqrt{x_1 ^2 +x_2 ^2}},
\]
\[
e_\theta = \dfrac{\frac{\partial}{\partial \theta}}{\left|\frac{\partial}{\partial \theta} \right|} =  (-\sin\theta\cos\varphi, -\sin\theta\sin\varphi,\cos\theta) = \left( -\dfrac{(x_3 x_1, x_3 x_2)}{r\sqrt{x_1 ^2 + x_2 ^2}}, \dfrac{\sqrt{x_1 ^2 + x_2 ^2} -R_0}{r}\right),
\]
with
\[
\dfrac{\partial}{\partial \varphi} = (-(R_0 + r\cos\theta)\sin\varphi,(R_0 + r\cos\theta)\cos\varphi,0),
\]
\[
\dfrac{\partial}{\partial \theta} = (-r\sin\theta\cos\varphi,- r\sin\theta\sin\varphi,r\cos\theta).
\]
Here $f_q$ is the quality factor, that is the number of toroidal winds of a magnetic line, corresponding to one poloidal wind. The magnetic field lines are either closed or dense on magnetic surface, depending whether the quality factor $f_q$ is rational, ($\mathit{i.e.,}$ $f_q = n/m$, $m, n$ are integers) or not. If $f_q$ is rational, the field line is closed otherwise the field line is dense on a magnetic surface. It is obvious that a field line on a magnetic surface with $f_q =n/m$ closes itself after traveling $n$ toroidal turns and $m$ poloidal turns.\\
In Cartesian coordinates, the magnetic field writes
\[
\dfrac{Be}{B_0} = \left(\dfrac{- x_2}{\sqrt{x_1 ^2 + x_2 ^2}}, \dfrac{x_1}{\sqrt{x_1 ^2 + x_2 ^2}}, 0 \right) + \dfrac{1}{f_q}\dfrac{r}{\sqrt{x_1 ^2 + x_2 ^2}}\left( \dfrac{- x_3 x_1}{r\sqrt{x_1 ^2 + x_2 ^2}}, \dfrac{-x_3 x_2}{r \sqrt{x_1 ^2 + x_2 ^2}}, \dfrac{\sqrt{x_1 ^2 + x_2 ^2}-R_0}{r} \right).
\]
Both the fields $e_\varphi \cdot \nabla_x$, $e_\theta \cdot \nabla_x$ leave invariant the function $r^2 = (\sqrt{x_1 ^2 + x_2 ^2} - R_0)^2 + x_3 ^2$ and therefore we have $B e\cdot \nabla_x r = 0$. We denote by $X(s;x)$ the characteristic flow of $Be\cdot\nabla_x$. We have
\[
r \cos\theta \dfrac{\marmd\theta}{\marmd s} = \dfrac{\marmd X_3}{\marmd s} = Be\cdot e_3 = B_0 \dfrac{r \cos \theta}{f_q(R_0 + r\cos\theta)},
\]
implying that 
\begin{equation}
\label{equ:EquEvoPol}
\dfrac{\marmd\theta}{\marmd s} = \dfrac{B_0}{f_q(R_0 + r\cos\theta)}.
\end{equation}
In order to determine the evolution of the toroidal angle $\varphi$, we write
\[
- r \sin\theta \dfrac{\marmd\theta}{\marmd s}\cos\varphi - (R_0 + r\cos\theta)\sin\varphi\dfrac{\marmd\varphi}{\marmd s} = \dfrac{\marmd X_1}{\marmd s} = -B_0 \sin\varphi - B_0 \dfrac{r\sin\theta \cos\varphi}{f_q (R_0 + r\cos\theta)},
\]
leading to
\begin{equation}
\label{equ:EquEvoTor}
\dfrac{\marmd\varphi}{\marmd s} = \dfrac{B_0}{R_0 + r \cos\theta}.
\end{equation}
The differential equation \eqref{equ:EquEvoPol} also writes
\begin{equation}
\label{equ:EquEvoPloBis}
\dfrac{\marmd}{\marmd s}(R_0 \theta + r\sin\theta) = \dfrac{B_0}{f_q  },
\end{equation}
and thus we obtain
\[
R_0 \theta(s) + r\sin\theta(s) = R_0\theta(0) + r \sin\theta(0) + s\dfrac{B_0}{f_q}.
\]
As $R_0 + r\cos\theta \geq R_0 -r \geq R_0 - r_0 >0$, the poloidal angle $\theta$ is increasing and there is $S >0$ such that $\theta(S) = \theta(0) + m 2\pi$, $m\in\Z\backslash \left\{0\right\}$. The number $S$ comes by the above equality that
 \[
 R_0 m 2\pi = R_0 (\theta(S) - \theta(0)) = S \frac{B_0}{f_q},
 \]
 and thus $S = f_q\frac{m 2\pi R_0}{B_0}$. By \eqref{equ:EquEvoPol} and \eqref{equ:EquEvoTor} we have $\frac{\marmd }{\marmd s}(\varphi - f_q \theta) =0$, implying that
\[
\varphi(S) - \varphi(0) = f_q (\theta(S) - \theta(0)) = f_q m 2\pi.
\]
Therefore the magnetic lines wind $f_q$ times along the toroidal angle while doing one wind along the poloidal angle. As $r$ is left invariant by the flow $X$, we have $r(S) =r(0)$ and thus $X(S;x) = X(0;x) = x$ if $f_q = \frac{n}{m}, n\in \Z\backslash \left\{0\right\} $, saying that the characteristic flow is $S$-periodic. Observe that
\[
\Divx\left(\dfrac{- x_2}{\sqrt{x_1 ^2 + x_2 ^2}}, \dfrac{x_1}{\sqrt{x_1 ^2 + x_2 ^2}}, 0 \right) =0,
\]
\[
\Divx\left( \dfrac{- x_3 x_1}{x_1 ^2 + x_2 ^2}, \dfrac{-x_3 x_2}{x_1 ^2 + x_2 ^2}, \dfrac{\sqrt{x_1 ^2 + x_2 ^2}-R_0}{\sqrt{x_1 ^2 + x_2 ^2}} \right) =0,
\]
and therefore the magnetic field $Be\cdot\nabla_x$ is divergence free. We are looking for angular vector fields. Motivated by \eqref{equ:EquEvoPloBis}, we consider
\[
\nu_\theta = R_0 \nabla_x \theta + r \cos\theta\nabla_x\theta + \sin\theta\nabla_x r.
\]
Since $\frac{\marmd}{\marmd s}\varphi = f_q \frac{\marmd}{\marmd s}\theta$, we also have from \eqref{equ:EquEvoPloBis} that $\frac{\marmd}{\marmd s}\left\{ R_0 \varphi + f_q r \sin\theta \right\}=B_0$ and we take
\[
\nu_\varphi = R_0 \nabla_x \varphi + f_q r\cos\theta \nabla_x\theta + f_q \sin\theta\nabla_x r.
\]
%%
\begin{pro}$\;$\\
\label{DecompositionBe}
The vector fields $\nu_\varphi,\nu_\theta$ are angular. The magnetic field 
$Be\cdot\nabla_x$ writes
\[
Be = \alpha_\varphi \nu_\varphi + \alpha_\theta \nu_\theta + \alpha_r \nabla_x r ,
\]
with 
\[
\alpha_\varphi = \dfrac{B_0 |\bar{x}|}{R_0},\,\, \alpha_\theta = \dfrac{B_0 r^2}{f_q |\bar{x}|^2} - B_0 f_q \dfrac{|\bar{x}|-R_0}{R_0},\,\, \alpha_r= -B_0 x_3 \left( \dfrac{r}{f_q |\bar{x}|^2} + \dfrac{f_q}{r}\right).
\]
\end{pro}
%%
\begin{proof}$\;$\\
It is easily seen that the vector $\nu_\varphi$ and $\nu_\theta$ are angular fields. For the decomposition of magnetic field $Be$, notice that, in the definition of $\nu_\varphi, \nu_\theta  $, the gradients of the angles $\varphi, \theta$ are understood as the continuous vector fields
\[
\nabla_x\varphi = (-\sin\varphi, \cos\varphi,0) = \left(\dfrac{- x_2}{x_1 ^2 + x_2 ^2}, \dfrac{x_1}{x_1 ^2 + x_2 ^2}, 0 \right) ,
\]
and
\[
\nabla_x\theta = \dfrac{1}{r} (-\sin\theta\cos\varphi, -\sin\theta\sin\varphi,\cos\theta) = \dfrac{1}{r}\left( \dfrac{- x_3 x_1}{r\sqrt{x_1 ^2 + x_2 ^2}}, \dfrac{-x_3 x_2}{r \sqrt{x_1 ^2 + x_2 ^2}}, \dfrac{\sqrt{x_1 ^2 + x_2 ^2}-R_0}{r} \right).
\]
By direct computations we have 
\[
|\nabla_x\varphi|^2 = \dfrac{1}{x_1 ^2 +x_2 ^2},\,\, |\nabla_x\theta|^2 = \dfrac{1}{r^2},\,\,|\nabla_x r|^2 = 1 ,
\]
\[
\nabla_x\varphi \cdot \nabla_x\theta =\nabla_x\varphi\cdot \nabla_x r  = \nabla_x\theta \cdot\nabla_x r =0 ,
\]
and 
\[
Be\cdot \nabla_x\varphi = \dfrac{B_0}{\sqrt{x_1 ^2 +x_2 ^2}},\,\, Be\cdot \nabla_x \theta = \dfrac{B_0}{f_q |\bar{x}|},\,\, Be\cdot\nabla_x r =0,
\]
and 
\[
\nu_\varphi\cdot \nabla_x\varphi = R_0 |\nabla_x\varphi|^2 = \dfrac{R_0}{|\bar{x}|^2}, \,\, \nu_\varphi\cdot \nabla_x\theta = f_q r \cos\theta |\nabla_x\theta|^2 = \dfrac{f_q \cos\theta}{r},
\]
\[
\nu_\varphi\cdot\nabla_x r = f_q \sin\theta |\nabla_x r|^2 = f_q \sin\theta,\,\, \nu_\theta \cdot\nabla_x\varphi = 0,\,\, \nu_\theta \cdot \nabla_x\theta = |\bar{x}|^2|\nabla_{x}\theta|^2 = \dfrac{|\bar{x}|}{r^2},
\]
\[
\nu_\theta \cdot \nabla_x r = \sin\theta |\nabla_x r|^2 = \sin\theta.
\]
Taking the scalar product with $\nabla_x\varphi, \nabla_x\theta, \nabla_x r$ we obtain the coefficients $\alpha_\varphi, \alpha_\theta, \alpha_r$ in the decomposition of the magnetic field $Be\cdot\nabla_x$.
\end{proof}
By Proposition \ref{DecompositionBe}, we have
\begin{align*}
\rot_x\left( \dfrac{ae}{\omega_c}\right) &= \rot_x \left[\dfrac{a}{B\omega_c}(\alpha_\varphi \nu_\varphi + \alpha_\theta \nu_\theta + \alpha_r \nabla_x r)  \right]\\
&=\nabla_x \left( \dfrac{a\alpha_\varphi}{B\omega_c} \right)\wedge \nu_\varphi  + \nabla_x \left( \dfrac{a\alpha_\theta}{B\omega_c} \right)\wedge \nu_\theta + \nabla_x \left( \dfrac{a\alpha_r}{B\omega_c} \right)\wedge \nu_r ,
\end{align*} 
and thanks to Proposition \ref{AngField}, we obtain
\begin{align*}
\left<  \rot_x\left( \dfrac{ae}{\omega_c}\right)\right> &= \nabla_x \left< \dfrac{a\alpha_\varphi}{B\omega_c} \right>\wedge \nu_\varphi  + \nabla_x \left< \dfrac{a\alpha_\theta}{B\omega_c} \right>\wedge \nu_\theta + \nabla_x \left< \dfrac{a\alpha_r}{B\omega_c} \right>\wedge \nu_r \\
&= \rot_x \left(  \left< \dfrac{a\alpha_\varphi}{B\omega_c} \right> \nu_\varphi  +  \left< \dfrac{a\alpha_\theta}{B\omega_c} \right> \nu_\theta +  \left< \dfrac{a\alpha_r}{B\omega_c} \right> \nu_r\right).
\end{align*}
We can write
\[
\left<  \rot_x\left( \dfrac{ae}{\omega_c}\right)\right> \cdot\nabla_x k[F(a)] = \rot_x \left[ a \left( \left< \dfrac{\alpha_\varphi}{B\omega_c} \right> \nu_\varphi  +  \left< \dfrac{\alpha_\theta}{B\omega_c} \right> \nu_\theta +  \left< \dfrac{\alpha_r}{B\omega_c} \right> \nu_r\right) \right]\cdot\nabla_x k[F(a)],
\]
and the limit model \eqref{equ:EquAveLim} becomes
\[
\partial_t a + \Divx \left[ a \left( \left< \dfrac{\alpha_\varphi}{B\omega_c} \right> \nu_\varphi  +  \left< \dfrac{\alpha_\theta}{B\omega_c} \right> \nu_\theta +  \left< \dfrac{\alpha_r}{B\omega_c} \right> \nu_r\right) \wedge \nabla_x k[F(a)]\right] =0,\,\,n=F(a).
\]


%%
\section{Convergence result}
We concentrate now on the asymptotic behavior as $\eps \searrow 0$ of the family of weak solutions $(f^\eps, E[f^\eps])_{\eps>0}$ of the Vlasov-Poisson-Fokker-Planck system \eqref{equ:VPFP-Scale}, \eqref{equ:PoissonEpsi}, and \eqref{equ:Initial} and we establish rigorously the connection to the fluid model \eqref{equ:gyro-kinetic}, \eqref{equ:constraint}, and \eqref{equ:PoissonLimit}. \\
We are looking a model for the concentration $n^\eps = n[f^\eps] =\intvt{f^\eps}$, similar to the equation \eqref{equ:gyro-kinetic} of the limit concentration $n$ and we perform the balance of the relative entropy between $n^\eps$ and $n$. As usual, these computations require the smoothness of the solution for the limit model. We justify the asymptotic behavior of  $(f^\eps, E[f^\eps])_{\eps>0}$ when $\eps \searrow 0$, provided that there is a smooth solution $(n, E[n] = -\nabla_x\Phi[n])$ for the fluid model \eqref{equ:gyro-kinetic}, \eqref{equ:constraint}, and \eqref{equ:PoissonLimit}. We do not concentrate on the well posedness of this fluid model, nevertheless we refer to Section \ref{Cylin} for some examples of smooth solutions. We are working with weak solutions $(f^\eps,E[f^\eps])_{\eps>0}$.

The balance for the number of particles writes
\begin{equation}
\label{equ:ParticleDens}
\partial_t n^\eps + \dfrac{1}{\eps}\Divx j^\eps =0,\,\, j^\eps = j[f^\eps]=\intvt{f^\eps v}.
\end{equation}
We are using the balance momentum as well
\begin{equation}
\label{equ:EquMomentum}
\eps \partial_t j^\eps + \Divx \intvt{f^\eps v\otimes v} - \dfrac{q}{m}n^\eps E[f^\eps] - \dfrac{\omega_c}{\eps}j^\eps \wedge e = - \dfrac{j^\eps}{\tau},
\end{equation}
which allows us to express the orthogonal component of $j^\eps$
\begin{align*}
\dfrac{j^\eps -(j^\eps\cdot e)e}{\eps} &= \dfrac{n^\eps e}{\omega_c}\wedge \left(\sigma \dfrac{\nabla_x n^\eps}{n^\eps}+ \dfrac{q}{m}\nabla_x \Phi[f^\eps] \right)\\
&+ \dfrac{e}{\omega_c}\wedge \left[ \Divx\intvt{(\sigma \nabla_v f^\eps + v f^\eps)\otimes v}+ \eps\partial_t j^\eps + \dfrac{j^\eps}{\tau}  \right]\\
&= \dfrac{n^\eps e}{\omega_c}\wedge \nabla_x k[n^\eps] + \dfrac{e}{\omega_c}\wedge F^\eps ,
\end{align*}
where we denote
\[
F^\eps =  \Divx\intvt{(\sigma \nabla_v f^\eps + v f^\eps)\otimes v}+ \eps\partial_t j^\eps + \dfrac{j^\eps}{\tau},
\]
and in the above computation, we have used that $\Divx\intvt{\sigma \nabla_v f^\eps \otimes v} = - \sigma \nabla_x n^\eps$.\\
Observe that 
\begin{align*}
\dfrac{1}{\eps}\Divx j^\eps &= \Divx \dfrac{j^\eps - (j^\eps\cdot e)e}{\eps} + \Divx\left[ \dfrac{(j^\eps\cdot e)Be}{B\eps}\right]\\
&= \Divx\left( \dfrac{n^\eps e}{\omega_c}\wedge \nabla_x k[n^\eps] \right) + \Divx\left( \dfrac{e}{\omega_c}\wedge F^\eps\right)+ Be\cdot \nabla_x \left[ \dfrac{(j^\eps\cdot e)}{B\eps}\right],
\end{align*}
and finally, thanks to \eqref{equ:ParticleDens}, we obtain a similar model for $n^\eps$, as in \eqref{equ:gyro-kinetic}
\begin{equation}
\label{equ:EquDensityEps}
\partial_t n^\eps + \Divx\left( \dfrac{n^\eps e}{\omega_c}\wedge \nabla_x k[n^\eps] \right) + \Divx\left( \dfrac{e}{\omega_c}\wedge F^\eps\right)+ Be\cdot \nabla_x p^\eps =0,\,\,\, p^\eps = \dfrac{j^\eps\cdot e}{B\eps}.
\end{equation}
We are also looking for a equation, analogous to \eqref{equ:constraint}, in order to complete the evolution equation \eqref{equ:EquDensityEps}, involving the Lagrange multiplier $p^\eps$. Considering the parallel component in the momentum balance \eqref{equ:EquMomentum}, we obtain
\[
\sigma e\cdot \nabla_x n^\eps + \dfrac{q}{m} n^\eps e\cdot \nabla_x \Phi[n^\eps] + e\cdot F^\eps =0.
\]
Thanks to \eqref{equ:constraint}, the above equation also writes
\begin{equation}
\label{equ:EquParallel}
e\cdot \nabla_x \left(\sigma \dfrac{n^\eps -n}{n} + \dfrac{q}{m}(\Phi[n^\eps]-\Phi[n]) \right) + \dfrac{e}{n}\cdot F^\eps = \dfrac{q}{m}\dfrac{(n^\eps -n)(E[n^\eps]-E[n])}{n}\cdot e.
\end{equation}
We intend to estimate the modulated energy of $n^\eps$ with respect to $n$  by writing $\calE[n^\eps|n]$ as
\begin{align}
\label{equ:EntropyDens}
\calE[n^\eps|n] &= \sigma \intxt{n h\left(\dfrac{n^\eps}{n} \right)} + \dfrac{\epsilon_0}{2m}\intxt{|\nabla_x \Phi[n^\eps]-\nabla_x\Phi[n]|^2}\nonumber\\
&= \intxt{(\sigma n^\eps \ln n^\eps +\dfrac{\epsilon_0}{2m} |\nabla_x\Phi[n^\eps]|^2)} - \intxt{(\sigma n \ln n +\dfrac{\epsilon_0}{2m} |\nabla_x\Phi[n]|^2)}\nonumber\\
&- \intxt{\left\{ \sigma(1+\ln n)+ \dfrac{q}{m}\Phi[n]\right\}(n^\eps -n)}\nonumber\\
&:= \calE[n^\eps] -\calE[n] - \intxt{k[n](n^\eps -n)}.
\end{align}
We introduce as well the modulated energy of $f^\eps$ with respect to $n^\eps M$, given by
\begin{align*}
\sigma&\intvxt{n^\eps M h\left(\dfrac{f^\eps}{n^\eps M} \right)}+ \dfrac{\epsilon_0}{2m}\intxt{\underbrace{|\nabla_x \Phi[f^\eps]-\nabla_x\Phi[n^\eps M]|^2}_{=0}}\\
&= \sigma \intvxt{f^\eps \ln f^\eps - f^\eps \ln n^\eps + f^\eps \ln (2\pi\sigma)^{3/2}+ f^\eps \frac{|v|^2}{2\sigma}}\\
&= \intvxt{\sigma f^\eps \ln f^\eps + f^\eps \frac{|v|^2}{2}} + \dfrac{\epsilon_0}{2m}\intxt{|\nabla_x \Phi[f^\eps]|^2} \\
&- \intxt{\sigma n^\eps \ln n^\eps} - \dfrac{\epsilon_0}{2m}\intxt{|\nabla_x \Phi[n^\eps]|^2} + \sigma \ln(2\pi\sigma)^{3/2}\intvxt{f^\eps}\\
&= \calE[f^\eps] - \calE[n^\eps]+ \sigma \ln(2\pi\sigma)^{3/2}\intvxt{f^\eps}.
\end{align*}
Thanks to the free energy balance \eqref{equ:EquFreeEne} and mass conservation of \eqref{equ:VPFP-Scale} one gets
\begin{align}
\label{equ:BalanEnerDens}
\calE[n^\eps(t)] - \calE[n^\eps(0)] &+ \sigma\intvxt{n^\eps(t) M h\left(\dfrac{f^\eps(t)}{n^\eps(t) M} \right)}\\
&-\sigma\intvxt{n^\eps(0) M h\left(\dfrac{f^\eps(0)}{n^\eps(0) M} \right)}\nonumber\\
& =  - \dfrac{1}{\eps\tau}\inttvxt{\dfrac{|\sigma \nabla_v f^\eps + v f^\eps|^2}{f^\eps}}\nonumber.
\end{align}
Thanks to Proposition \ref{BalLiMod} and combining \eqref{equ:EntropyDens}, \eqref{equ:BalanEnerDens} leads to
\begin{align}
\label{BalModEnerDens}
&\calE[n^\eps(t)|n(t)] +  \sigma\intvxt{n^\eps(t) M h\left(\dfrac{f^\eps(t)}{n^\eps(t) M} \right)} + \dfrac{1}{\eps\tau}\inttvxt{\dfrac{|\sigma \nabla_v f^\eps + v f^\eps|^2}{f^\eps}}\nonumber\\
&= \calE[n^\eps(0)|n(0)] +  \sigma\intvxt{n^\eps(0) M h\left(\dfrac{f^\eps(0)}{n^\eps(0) M} \right)} - \int_{0}^{t}\dfrac{\marmd }{\marmd s}\intxt{k[n](n^\eps -n)}\marmd s.
\end{align}
The next task is to evaluate the time derivative of $\intxt{k[n](n^\eps -n)}$. Notice that for any smooth concentration $n$, we can write 
\begin{align*}
\dfrac{n e}{\omega_c} \wedge \nabla_x k[n] &= \dfrac{n e}{\omega_c}\wedge \left(\sigma \dfrac{\nabla_x n}{n}+ \dfrac{q}{m}\nabla_x\Phi[n] \right)\\
&= \dfrac{\sigma e}{\omega_c}\wedge \nabla_x n + \dfrac{n e}{B}\wedge \nabla_x \Phi[n]\\
&= n V[n] - \sigma \rot_x \left( \dfrac{ne}{\omega_c} \right),
\end{align*}
where $V[n] = \sigma \rot_x\left(\frac{e}{\omega_c} \right)+\frac{e\wedge \nabla_x\Phi[n]}{B}$. Clearly, we have
\begin{equation}
\label{equ:EquDivVit}
\Divx\left( \dfrac{ne}{\omega_c}\wedge \nabla_x k[n]\right) = \Divx(nV[n]).
\end{equation}
%%
\begin{pro}$\;$\\
\label{TimeDeri}
With the notations in \eqref{equ:gyro-kinetic}, \eqref{equ:constraint}, \eqref{equ:EquDensityEps} and \eqref{equ:EquParallel} we have the equality
\begin{align*}
&\dfrac{\marmd}{\marmd t}\intxt{k[n(t)](n^\eps(t,x)-n(t,x))} \\&= \intxt{\left(p\dfrac{Be}{n} + \dfrac{e}{\omega_c}\wedge \nabla_x k[n] \right)\cdot \left( \dfrac{q}{m}(n^\eps -n)(E[n^\eps]-E[n]) -F^\eps \right)}.
\end{align*}
\end{pro}
%%
\begin{proof}$\;$\\
By straightforward computations, we obtain
\begin{align}
\label{equ:FirstTimeDeri}
&\dfrac{\marmd}{\marmd t}\intxt{k[n](n^\eps-n)}\nonumber\\ &= \intxt{\left(\sigma \dfrac{\partial_t n}{n}+\dfrac{q}{m}\partial_t \Phi[n] \right)(n^\eps -n)}+\intxt{k[n](\partial_t n^\eps -\partial_t n)}\nonumber \\
&= \intxt{\partial_t n \left(\sigma \dfrac{n^\eps - n}{n} + \dfrac{q}{m}(\Phi[n^\eps]-\Phi[n]) \right)}\\
&+ \intxt{k[n]\left[ \Divx\left( \dfrac{ne}{\omega_c}\wedge \nabla_x k[n]\right) - \Divx \left( \dfrac{n^\eps e}{\omega_c}\wedge \nabla_x k[n^\eps]\right) - \Divx \left( \dfrac{e}{\omega_c}\wedge F^\eps \right) \right]},\nonumber
\end{align}
where in the last integral we have used the contraint $Be\cdot\nabla_x k[n]=0$ which allows us to deduce that
\[
\intxt{k[n](Be\cdot\nabla_x p - Be\cdot\nabla_x p^\eps)} =0.
\]
Thanks to \eqref{equ:EquDivVit}, \eqref{equ:EquParallel} we have
\begin{align}
\label{equ:FirstTimeDerBis}
&\intxt{\partial_t n \left(\sigma \dfrac{n^\eps - n}{n} + \dfrac{q}{m}(\Phi[n^\eps]-\Phi[n]) \right)}\nonumber\\
&= - \intxt{\Divx\left( \dfrac{ne}{\omega_c}\wedge \nabla_x k[n]\right)\left(\sigma \dfrac{n^\eps - n}{n} + \dfrac{q}{m}(\Phi[n^\eps]-\Phi[n]) \right) }\\
&- \intxt{Be\cdot\nabla_x p\left(\sigma \dfrac{n^\eps - n}{n} + \dfrac{q}{m}(\Phi[n^\eps]-\Phi[n]) \right)}\nonumber\\
&= -\intxt{\Divx(nV[n])\left(\sigma \dfrac{n^\eps - n}{n} \right)} - \intxt{\left( \dfrac{ne}{\omega_c}\wedge \nabla_x k[n]\right)\cdot\dfrac{q}{m}(E[n^\eps]-E[n])}\nonumber\\
&+ \intxt{\dfrac{pBe}{n}\cdot \left[\dfrac{q}{m}(n^\eps -n)(E[n^\eps] -E[n])-F^\eps \right]}\nonumber\\
&=-\sigma \intxt{\Divx\left( \dfrac{e\wedge \nabla_x\Phi[n]}{B}\right)(n^\eps -n)} - \sigma \intxt{\nabla_x \ln n \cdot V[n](n^\eps -n)}\nonumber\\
&- \intxt{\left( \dfrac{ne}{B}\wedge \nabla_x k[n]\right)\cdot(E[n^\eps]-E[n])} + \intxt{\dfrac{pBe}{n}\cdot \left[\dfrac{q}{m}(n^\eps -n)(E[n^\eps] -E[n])-F^\eps \right]}\nonumber\\
&= - \intxt{\nabla_x k[n]\cdot V[n](n^\eps -n)} - \intxt{\left( \dfrac{ne}{B}\wedge \nabla_x k[n]\right)\cdot(E[n^\eps]-E[n])}\nonumber\\
&+ \intxt{\dfrac{pBe}{n}\cdot \left[\dfrac{q}{m}(n^\eps -n)(E[n^\eps] -E[n])-F^\eps \right]}.\nonumber
\end{align}
Thanks to \eqref{equ:EquDivVit} again, the last integral in \eqref{equ:FirstTimeDeri} writes easily
\begin{align}
\label{equ:LastInte}
&\intxt{k[n]\left[ \Divx\left( \dfrac{ne}{\omega_c}\wedge \nabla_x k[n]\right)  - \Divx \left( \dfrac{n^\eps e}{\omega_c}\wedge \nabla_x k[n^\eps]\right) - \Divx \left( \dfrac{e}{\omega_c}\wedge F^\eps \right)\right]}\\
&= \intxt{\nabla_x k[n]\cdot (n^\eps V[n^\eps] - n V[n])} - \intxt{\left( \dfrac{e}{\omega_c}\wedge \nabla_x k[n]\right)\cdot F^\eps}\nonumber.
\end{align}
Observe that 
\begin{align*}
n^\eps V[n^\eps] - nV[n]- (n^\eps -n)V[n] &= n^\eps  \dfrac{e\wedge \nabla_x \Phi[n^\eps]}{B} - n \dfrac{e\wedge \nabla_x \Phi[n]}{B} - (n^\eps -n)\dfrac{e\wedge \nabla_x \Phi[n]}{B}\\
&= n^\eps  \dfrac{e\wedge (\nabla_x \Phi[n^\eps]-\nabla_x\Phi[n])}{B} ,
\end{align*}
and finally \eqref{equ:FirstTimeDeri}, \eqref{equ:FirstTimeDerBis} and \eqref{equ:LastInte} yield the result.
\end{proof}
%%
Coming back to \eqref{BalModEnerDens}, the modulated energy balance becomes
\begin{align}
\label{BalModEnerDensBis}
\calE[n^\eps(t)|n(t)] &+  \sigma\intvxt{n^\eps(t) M h\left(\dfrac{f^\eps(t)}{n^\eps(t) M} \right)} + \dfrac{1}{\eps\tau}\inttvxt{\dfrac{|\sigma \nabla_v f^\eps + v f^\eps|^2}{f^\eps}}\nonumber\\
&= \calE[n^\eps(0)|n(0)] +  \sigma\intvxt{n^\eps(0) M h\left(\dfrac{f^\eps(0)}{n^\eps(0) M} \right)} \\
&- \int_{0}^{t}{\intxt{W[n]\cdot \left( \dfrac{q}{m}(n^\eps -n)(E[n^\eps]-E[n])-F^\eps\right)}}\mathrm{d}s,\nonumber
\end{align}
where $W[n] = \frac{pBe}{n}+\frac{e}{\omega_c}\wedge \nabla_x k[n]$. In order to apply Gronwall lemma, we estimate the terms in the last integral of \eqref{BalModEnerDensBis}. Thanks to the formula
\begin{align*}
 \dfrac{q}{m}(n^\eps -n)(E[n^\eps]-E[n]) &= \dfrac{\epsilon_0}{m}[\Divx(E[n^\eps]-E[n])](E[n^\eps]-E[n])\\
 &= \dfrac{\epsilon_0}{m} \Divx \left( (E[n^\eps]-E[n])\otimes (E[n^\eps]-E[n]) - \dfrac{|E[n^\eps]-E[n]|^2}{2}I_3 \right),
\end{align*}
we obtain
\begin{align*}
- \intxt{W[n]&\cdot \left( \dfrac{q}{m}(n^\eps -n)(E[n^\eps]-E[n])\right)} \\
&= \dfrac{\epsilon_0}{m}\intxt{ \left((E[n^\eps]-E[n])\otimes (E[n^\eps]-E[n]) - \dfrac{|E[n^\eps]-E[n]|^2}{2}I_3\right): \partial_x W[n]}\\
&\leq \|\partial_x W[n] \|_{L^\infty(\R^3)}\dfrac{\epsilon_0}{m}\left(1+ \dfrac{\sqrt{3}}{2} \right)\intxt{|E[n^\eps]-E[n]|^2},
\end{align*}
where for any matrix $P\in \calM_{3,3}(\R)$, the notation $\| P\|$ stands for $(P:P)^{1/2}$. Similarly, we have for some value $C$ to be precised later on
\begin{align*}
&\intxt{W[n]\cdot \Divx \intvt{(\sigma \nabla_v f^\eps + f^\eps v)\otimes v}}\\
 &= - \intxt{\partial_x W[n]: \intvt{(\sigma \nabla_v f^\eps + f^\eps v)\otimes v}}\\
 &\leq \|\partial_x W[n]  \|_{L^\infty((0,T)\times\R^3)}\left[\dfrac{1}{2\eps\tau C}\intvxt{\dfrac{|\sigma \nabla_v f^\eps + f^\eps v |^2}{f^\eps}}+ \eps \tau C \intvxt{f^\eps \dfrac{|v|^2}{2}} \right].
\end{align*}
Since $j^\eps = \intvt{(\sigma \nabla_v f^\eps + f^\eps v )}$ we have
\begin{align*}
&\int_{0}^{t}\intxt{W[n(s)]\cdot (\eps \partial_s j^\eps + \dfrac{j^\eps}{\tau})}\marmd s \\
&=  \eps \intxt{W[n(t)]\cdot j^\eps(t,x)} - \eps \intxt{W[n(0)]\cdot j^\eps(0,x)}\\
&+ \inttvxt{[\sigma \nabla_v f^\eps + f^\eps(s,x,v)v]\cdot \left[ \dfrac{W[n(s)]}{\tau} - \eps \partial_s W[n(s)] \right]}\\
&\leq \sqrt{\eps} \intvxt{(f^\eps(0,x,v) + f^\eps(t,x,v))\left(\eps\dfrac{|v|^2}{2}+ \dfrac{\| W[n]\|^2 _{L^\infty}}{2} \right)}\\
&+ \left[\eps \| \partial_s W[n] \|_{L^\infty} + \dfrac{\|W[n]\|_{L^\infty}}{\tau}  \right]\inttvxt{\left\{ \dfrac{1}{2\eps C}\dfrac{|\sigma \nabla_v f^\eps + f^\eps v |^2}{f^\eps}+ \dfrac{\eps C}{2}f^\eps \right\}}.
\end{align*}
Plugging the above computations in \eqref{BalModEnerDensBis}, the modulated energy balance becomes for $t\in[0,T]$
\begin{align*}
&\calE[n^\eps(t)|n(t)] +  \sigma\intvxt{n^\eps(t) M h\left(\dfrac{f^\eps(t)}{n^\eps(t) M} \right)} \\& + \dfrac{1}{\eps\tau}\left(1-\dfrac{\|W[n]\|_{L^\infty}}{2C}-\dfrac{\eps \tau \| \partial_s W[n] \|_{L^\infty} }{2C}  -\dfrac{\|W[n]\|_{L^\infty}}{2C}  \right)\inttvxt{\dfrac{|\sigma \nabla_v f^\eps + v f^\eps|^2}{f^\eps}}\\
&\leq \calE[n^\eps(0)|n(0)]+ \sigma\intvxt{n^\eps(0) M h\left(\dfrac{f^\eps(0)}{n^\eps(0) M} \right)} \\
&+ \|\partial_x W[n] \|_{L^\infty}\left(2+ \sqrt{3} \right)\dfrac{\epsilon_0}{2m}\intxt{|E[n^\eps]-E[n]|^2}\\
&+\eps \dfrac{ \tau C}{2}\|\partial_x W[n] \|_{L^\infty} \intTvxt{f^\eps |v|^2} + \sqrt{\eps}\sup_{0\leq t\leq T}\eps \intvxt{f^\eps |v|^2}\\
&+ \sqrt{\eps} \left[ \sqrt{\eps}\dfrac{CT}{2}\left( \eps \| \partial_s W[n] \|_{L^\infty} + \dfrac{\|W[n]\|_{L^\infty}}{\tau}\right)+ \|W[n]\|_{L^\infty}^2 \right]\intvxt{f^\eps(0,x,v)}.
\end{align*}
Taking $0 < \eps \leq 1$ and $C$ large enough, we obtain by Lemma \ref{KinEne} and \eqref{equ:EntropyDens}, for some constant $C_T$, $0\leq t\leq T$, $0< \eps \leq 1$
\begin{align*}
&\calE[n^\eps(t)|n(t)] +  \sigma\intvxt{n^\eps(t) M h\left(\dfrac{f^\eps(t)}{n^\eps(t) M} \right)} + \dfrac{1}{2\eps\tau}\inttvxt{\dfrac{|\sigma \nabla_v f^\eps + v f^\eps|^2}{f^\eps}}\\
&\leq  \calE[n^\eps(0)|n(0)] +  \sigma\intvxt{n^\eps(0) M h\left(\dfrac{f^\eps(0)}{n^\eps(0) M} \right)} + C_T \int_{0}^{t}\calE[n^\eps(s)|n(s)]\marmd s + C_T \sqrt{\eps}.
\end{align*}
Applying Gronwall lemma, we deduce that for $0\leq t\leq T$, $0< \eps \leq 1$
\begin{align*}
\calE[n^\eps(t)|n(t)] +&  \sigma\intvxt{n^\eps(t) M h\left(\dfrac{f^\eps(t)}{n^\eps(t) M} \right)} + \dfrac{1}{2\eps\tau}\inttvxt{\dfrac{|\sigma \nabla_v f^\eps + v f^\eps|^2}{f^\eps}}\\
&\leq  \left[\calE[n^\eps(0)|n(0)] +  \sigma\intvxt{n^\eps(0) M h\left(\dfrac{f^\eps(0)}{n^\eps(0) M} \right)}  + C_T \sqrt{\eps}\right]e^{C_T t}.
\end{align*}
The above inequality says that the particle density $f^\eps$ remains close to the Maxwellian with the same concentration, $\mathit{i.e.,}$ $n^\eps(t)M$, and $n^\eps(t)$ stays near $n(t)$, provided that analogous behaviour occur for the initial conditions. Therefore, we are ready to prove our main theorem.
%%
\begin{proof}(of Theorem \ref{MainThm})$\;$\\
We justify the convergence of $f^\eps$ toward $nM$ in $L^\infty(0,T;L^1(\R^3\times\R^3))$, the other convergences being obvious. We use the Csis\'ar -Kullback inequality in order to control the $L^1$ norm by the relative entropy, cf. \cite{Csi1967, Kul1967}
\[
\int_{\R^n}|g-g_0|\marmd x \leq 2 \max \left\{ \left( \int_{\R^n}g_0\marmd x \right)^{1/2}, \left( \int_{\R^n}g\marmd x\right)^{1/2} \right\}\left(\int_{R^n}g_0  h\left(\dfrac{g}{g_0} \right)\marmd x \right)^{1/2}
\]
for any non negative integrable functions $g_0,g: \R^n \to \R$. Applying two times the  Csis\'ar -Kullback inequality we obtain
\begin{align*}
&\intvxt{|f^\eps(t,x,v) -n(t,x)M(v)|} \\ &\leq \intvxt{|f^\eps(t,x,v)- n^\eps(t,x)M(v)|} + \intxt{|n^\eps(t,x)-n(t,x)|}\\
&\leq 2 \sqrt{M_{\mathrm{in}}} \left(\intvxt{ n^\eps(t)M(v) h\left(\dfrac{f^\eps(t)}{n^\eps(t)M} \right)}\right)^{1/2} \\&+ 2 \max\left\{ \sqrt{M_{\mathrm{in}}},\sqrt{\|n_{\mathrm{in}}\|_{L^1(\R^3)}} \right\} \left( \intxt{n(t) h \left(\dfrac{n^\eps(t)}{n(t)}\right)}  \right)^{1/2}  \to 0 ,\,\,\mathrm{as}\,\eps\searrow 0.
\end{align*}
\end{proof}
%%
In the same manner we perform the balance of the relative between two smooth solutions of the limit model.
%%
\begin{pro}$\;$\\
\label{UniLimMod}
Assume that $n$, $\tilde{n}$ are smooth solutions of \eqref{equ:gyro-kinetic},  \eqref{equ:constraint} and \eqref{equ:PoissonLimit} such that $n_{\mathrm{in}}, \tilde{n}_{\mathrm{in}} \geq 0$, $n_{\mathrm{in}}, \tilde{n}_{\mathrm{in}} \in L^1(\R^3)$, $\nabla_x \Phi[n_{\mathrm{in}}], \nabla_x\Phi[\tilde{n}_{\mathrm{in}}]\in L^2(\R^3)$, $\partial_x W[n]\in L^1(0,T;L^{\infty}(\R^3))$, $k[n_{\mathrm{in}}], k[\tilde{n}_{\mathrm{in}}] \in \mathrm{ker}(Be\cdot\nabla_x)$. Then we have the inequality
\[
\calE[\tilde{n}(t)|n(t)] \leq \calE[\tilde{n}_{\mathrm{in}}|n_{\mathrm{in}}]\exp((2+\sqrt{3})\|\partial_x W[n]\|_{L^1(0,T;L^\infty(\R^3))}),\,\, 0\leq t\leq T.
\]
In particular, there is at most one smooth solution.
\end{pro}
%%
\begin{proof}$\;$\\
By \eqref{equ:EntropyDens} we know that
\begin{align*}
\calE[\tilde{n}|n] &= \calE[\tilde{n}] - \calE[n] - \intxt{k[n](\tilde{n}-n)}\\
&= \sigma \intxt{n h\left(\dfrac{\tilde{n}}{n} \right)} + \dfrac{\varepsilon_0}{2m}\intxt{|\nabla_x \Phi[\tilde{n}] - \nabla_x\Phi[n]|^2}.
\end{align*}
Thanks to the constraint $Be\cdot\nabla_x k[n] =0, Be\cdot\nabla_x k[\tilde{n}] = 0$, we can write
\[
e\cdot \nabla_x \left( \sigma \dfrac{\tilde{n} -n}{n} + \dfrac{q}{m}(\Phi[\tilde{n}]-\Phi[n]) \right)  = \dfrac{q}{m}\dfrac{(\tilde{n} -n)(E[\tilde{n}]-E[n])}{n}\cdot e.
\]
As in the proof of Proposition \ref{TimeDeri}, we observe that
\begin{align*}
\dfrac{\marmd}{\marmd t}\intxt{k[n](\tilde{n}-n)} &= \intxt{\partial_t n \left(\sigma \dfrac{\tilde{n} - n}{n} + \dfrac{q}{m}(\Phi[\tilde{n}]-\Phi[n]) \right)}\\
&+ \intxt{k[n]\left[ \Divx\left( \dfrac{ne}{\omega_c}\wedge \nabla_x k[n]\right)- \Divx \left( \dfrac{\tilde{n} e}{\omega_c}\wedge \nabla_x k[\tilde{n}]\right) \right]  }\\
&= \intxt{W[n]\cdot \dfrac{q}{m}(\tilde{n}-n)(E[\tilde{n}]-E[n])},
\end{align*}
and the balance for the relative entropy becomes
\begin{align*}
\calE[\tilde{n}(t)|n(t)] - \calE[\tilde{n}(0)|n(0)] &= -\dfrac{q}{m}\int_{0}^{t}\intxt{(\tilde{n}-n)(E[\tilde{n}]-E[n])\cdot W[n]}\marmd s\\
&\leq \| \partial_x W[n]\|_{L^\infty(\R^3)}(2+\sqrt{3})\dfrac{\epsilon_0}{2m}\int_{0}^{t}\intxt{|E[n^\eps]-E[n]|^2}\marmd s \\
&\leq \| \partial_x W[n]\|_{L^\infty(\R^3)}(2+\sqrt{3})\int_{0}^{t}\calE[\tilde{n}(s)|n(s)]\marmd s.
\end{align*}
Applying Gronwall lemma completes the proof.
\end{proof}





%%
\section{Example of smooth solutions for limit model}
In this section we construct smooth solution for the limit model obtained in Section \ref{Cylin}. We focus on the existence of the limit model 
\begin{equation}
\label{equ:EquCylinModLim}
\partial_t n + \Divx\left( \dfrac{ne}{\omega_c}\wedge \nabla_x k[n]\right) =0,\,\,k[n] = \sigma(1+\ln n) + \dfrac{q}{m}\Phi[n],\,\,(t,x)\in \R_+ \times \R^2 \times \T^1 ,
\end{equation}
where $\Phi[n] $ stands for the Poisson electric potential which solves
\begin{equation}
\label{equ:PoiCylinLim}
 -\epsilon_0 \Delta_x \Phi[n(t)](x) = q\, n(t,x),\,\,(t,x)\in \R_+ \times \R^2 \times \T^1.
\end{equation}
Denoting $E[n(t)] = -\nabla_x \Phi[n(t)]$ the electric field derives from the potential $\Phi[n(t)]$. 
We supplement our model by the initial condition
\begin{equation}
\label{equ:IniCylinLim}
n(0,x) = n_{\mathrm{in}}(x),\,\, x\in \R^2 \times \T^1,
\end{equation}
where $n_{\mathrm{in}}$ is a smooth function and  belongs to $ \mathrm{ker}(Be\cdot\nabla_x)$. The external magnetic field we consider here $Be = (x_2, -x_1,1)$. Notice that the vector field $e/B \in W^{2,\infty}((\R^2\times\T^1))$.\\
We follow the same arguments as in the well-posedness proof for the Vlasov-Poisson problem with an external magnetic field, as discussed in \cite{BosSIAM2019, Bos2020}. Our goal is to obtain a priori bounds for the $L^\infty$ norm of $E[n]$ and $ \partial_x E[n]$, not in the full space $\R^3$, but in $\R^2\times\T^1$. These bounds rely on estimating the fundamental solution of Laplace's equation on $\R^2\times \T^1$. Therefore, we begin by investigating the Poisson equation for a given density in this domain and finding a fundamental solution for this purpose. 
%%
\subsection{Fundamental solution of Laplace's equation on $\R^2\times\T^1$}
Consider a function $\Xi: \R^2\times\T^1 \to \R$ satisfying
\begin{equation}
\label{equ:EquPoiPerZ}
-\Delta _x \Xi = \delta_0(\bar{x},x_3),\,\,x=(\bar{x}=(x_1,x_2),x_3)\in \R^2\times\T^1,
\end{equation}
in the sense of distributions, where $\delta_0(x)$ denotes the Dirac measure on $\R^2\times\T^1$ giving unit mass to the point $0$.
%%
\begin{lemma}$\;$\\
\label{PerFundSol}
Let $x = (\bar{x},x_3) \in \R^2\times \T^1$. Then
\[
\Xi (x) = -\dfrac{1}{4\pi ^2} \ln (|\bar{x}|) + \Gamma (x),
\]
satifies \eqref{equ:EquPoiPerZ}, where
\[
\Gamma(x) = \int_{0}^{\infty}\dfrac{1}{4\pi t}e^{{-|\bar{x}|^2}/{4t}} \dfrac{1}{\pi}\left[  \sum_{n=1}^{\infty} e^{-n^2 t}\cos(n x_3)\right]\marmd t .
\]
\end{lemma}
%%
\begin{proof}$\;$\\
We have
%&= \delta_0(\bar{x},x_3)= \delta_0(\bar{x})\delta_0(x_3 )\nonumber\\
%&= \delta_0(\bar{x})\dfrac{1}{2\pi}\sum_{n\in\Z} e^{i n x_3}\nonumber\\
%&
\begin{align}
\label{equ:SumPerZ}
-\Delta_x \Xi = \dfrac{1}{2\pi}\sum_{n\in\Z} \delta_0(\bar{x})e^{i n x_3},
\end{align}
where we have used the Poisson summation formula
\[
\delta_0(x_3) = \dfrac{1}{2\pi}\sum_{n\in\Z} e^{i n x_3}.
\]
Indeed, the $\delta_0(x_3)$ is periodic with period $2\pi$, it can be represented as a Fourier series
\[
\delta_0(x_3) = \sum_{n\in\Z}c_n e^{in x_3},
\]
where the Fourier coefficients are
\begin{align*}
c_n  &= \dfrac{1}{2\pi} \int_{-\pi}^{\pi}\delta_0(x_3 ) e^{-in x_3}\marmd x_3 = \dfrac{1}{2\pi} e^{-i0x_3} = \dfrac{1}{2\pi}.
\end{align*}
On the other hand, as $\Xi$ is periodic in $x_3$ of period $2\pi$, we also have
\begin{equation*}
\Xi(x) = \sum_{n\in \Z} \beta_n(\bar{x}) e^{inx_3},
\end{equation*}
therefore
\begin{equation}
\label{equ:EquLap}
-\Delta_x \Xi(\bar{x},x_3) = \sum_{n\in\Z}(-\Delta_{\bar{x}}\beta_n(\bar{x}) + n^2 \beta_n(\bar{x}))e^{inx_3}.
\end{equation}
Comparing \eqref{equ:SumPerZ} and \eqref{equ:EquLap} yields the following linear elliptic equation in the whole space $\R^2$ for any $n\in\Z \backslash\left\{0\right\}$
\begin{equation}
\label{equ:EquCoeffs}
-\Delta_{\bar{x}}\beta_n(\bar{x}) + n^2 \beta_n(\bar{x}) = \dfrac{1}{2\pi}\delta_0(\bar{x}),\,\, \bar{x}\in\R^2.
\end{equation}
A solution to \eqref{equ:EquCoeffs} can be found by using the Fourier transform for linear equation. It is known that the solution to this equation is given in term of the Bessel potential $G(\bar{x})$ as $\beta_n(\bar{x}) = \frac{1}{2\pi}(U \star \delta_0)(\bar{x})$, cf. \cite{Evans} where 
$
U(\bar{x}) = \int_{0}^{\infty}\frac{1}{4\pi t}e^{-|\bar{x}|^2/{4t}}e^{-n^2 t}\marmd t.
$
Thus, we have the solution formula
\[
\beta_n(\bar{x}) = \dfrac{1}{2\pi}\int_{0}^{t}\dfrac{1}{4\pi t}e^{-|\bar{x}|^2/{4t}}e^{-n^2 t}\marmd t.
\]
In the case $n=0$, the equation \eqref{equ:EquCoeffs} becomes the Laplace equation on $\R^2$. It is well known that the fundamental solution is given by $-\frac{1}{4\pi ^2}\ln |\bar{x}|$.
Finally, we obtain
\begin{align*}
\Xi(x) &= -\dfrac{1}{4\pi^2} \ln (|x|) + \sum_{n\in \Z\backslash\left\{0\right\}}\dfrac{1}{2\pi}\int_{0}^{t}\dfrac{1}{4\pi t}e^{-|\bar{x}|^2/{4t}}e^{-n^2 t}\marmd t \, e^{inx_3}\\
&= -\dfrac{1}{4\pi^2} \ln (|x|) + \int_{0}^{t} \dfrac{1}{4\pi t}e^{-|\bar{x}|^2/{4t}} \dfrac{1}{2\pi}\sum_{n\in \Z\backslash\left\{0\right\}}e^{-n^2 t}e^{inx_3}\marmd t\\
&=  -\dfrac{1}{4\pi^2} \ln (|x|) + \int_{0}^{t} \dfrac{1}{4\pi t}e^{-|\bar{x}|^2/{4t}} \dfrac{1}{\pi}\left[ \sum_{n=1}^{\infty}e^{-n^2 t}\cos(n x_3) \right]\marmd t.
\end{align*}
\end{proof}
%%
Let us denote $\Gamma_{1,2}(t,\bar{x}) := \frac{1}{4\pi t}e^{-|\bar{x}|^2/{4t}}$ and $\Gamma_3 (t,x_3) := \frac{1}{2\pi}\left[1 + 2 \sum_{n=1}^{\infty}e^{-n^2 t}\cos(n x_3) \right]$. It is know that $\Gamma_{1,2}$ is a heat kernel on $\R^2$ of the heat equation 
\begin{align*}
\left\{
    \begin{array}{ll}
      \partial_t \Gamma_{1,2}(t,\bar{x}) - \Delta_{\bar{x}}\Gamma_{1,2}(t,\bar{x}) = 0,\,\, (t,\bar{x})\in \R_+\times \R^2 ,\\
\hspace*{30mm}\Gamma_{1,2}|_{t=0}(\bar{x}) = \delta_0(\bar{x}),
    \end{array}
  \right. 
\end{align*}
while $\Gamma_3$ is a heat kernel on $\T^1$ of
\begin{align*}
\left\{
    \begin{array}{ll}
\partial_t \Gamma_{3}(t,x_3) - \partial_{x_3}^2\Gamma_{3}(t,x_3) =0,\,\, (t,x_3)\in \R_+\times \T^1 , \\
\hspace*{32mm}\Gamma_{3}|_{t=0}(x_3) = \delta_0(x_3).
 \end{array}
  \right. 
\end{align*}
For a proof of this property, we refer to \cite{CarPanZa20}.
We define now $G(t,x):=\Gamma_{1,2}(t,\bar{x}) \Gamma_3(t,x_3)$. Then $G$ is the fundamental solution of heat equation on $\R^2\times \T^1$, that means
\begin{align*}
\left\{
    \begin{array}{ll}
\partial_t G - \Delta_x G =0,\,\, (t,x=(\bar{x},x_3))\in \R_+\times\R^2\times \T^1 ,\\
G|_{t=0}(x) = \delta_0(x).
 \end{array}
  \right. 
\end{align*}
 Thus, we have that the function $\Gamma$ in the fundamental solution for Laplace's equation \eqref{equ:EquPoiPerZ} is related to the previous solution of heat equation as 
\begin{equation}
\label{equ:EquLapHeat}
\Gamma(x) = \int_{0}^{\infty}\Gamma_{1,2}(t,\bar{x})\left[\Gamma_3(t,x_3) -\frac{1}{2\pi} \right]\marmd t.
\end{equation}
%%
\begin{remark}$\;$\\
The heat kernel $\Gamma_3$ on $\T^1$ can also be given by the heat kernel $k_t(x_3) = {(4\pi t)}^{-1/2}e^{-x_3 ^2/{4t}}$ on the real line $\R$ as follows
\begin{equation}
\label{equ:HeatKerR}
\Gamma_3 (x_3) = \dfrac{1}{2\pi} g_t(x_3): = \dfrac{1}{2\pi}\left[2\pi \sum_{n\in \Z} k_t(x_3 + 2\pi n)\right],\,\, x_3 \in \T^1.
\end{equation}
Indeed, the function $g_t \in L^1(\T^1)$ since 
\[
\| g_t \|_{\T^1} = \int_{\T^1} g_t \marmd m(x_3) = \sum_{n\in\Z} \int_{\T^1} k_t(x_3 + 2\pi n) \marmd x_3 = \int_{\R} k_t(x_3) \marmd x_3 =1 ,
\]
where $\marmd m(x_3)$ is Haar measure on $\T^1$, $ \marmd m(x_3) = 1/{(2\pi)}\,\marmd x_3$. Thus, the periodic function $g_t$ can be written in the form of the Fourier serie
\[
g_t(x_3) =  \sum_{n\in \Z} \hat{g}_t(n) e^{i n x_3},
\]
where $(\hat{g}_t (n))_{n\in\Z}$ is the sequence of the Fourier coeffiecients which is given by
\begin{align*}
\hat{g}_t(n) &= \dfrac{1}{2\pi}\int_{\T^1} g_t(x_3) e^{-i n x_3} \marmd m(x_3) 
= \dfrac{1}{4\pi^2}\sum_{n\in\Z}\int_{\T^1} k_t(x_3 +2\pi n) e^{-i n (x_3 + 2\pi n)} \marmd x_3 \\
&=\dfrac{1}{4\pi^2}\int_{\R}k_t(x_3)e^{-i n x_3}\marmd x_3 = \dfrac{1}{4\pi^2}\hat{k}_t(n) =\dfrac{1}{2\pi}\left[\dfrac{1}{2\pi} e^{-n^2 t}\right],
\end{align*}
where $\hat{k}_t(n)$ is the Fourier transform of  the function $k_t(x_3)$.
\end{remark}
%%
Since we need the bounds of the function $\Gamma$ and its derivatives in the following, we must to estimate the function $\Gamma_3 -\frac{1}{2\pi}$ and also the first and second derivates of $\Gamma_3$ from \eqref{equ:EquLapHeat}. We shall use the arguments in \cite{Maheux} to obtain the bound of $|\Gamma_3 -\frac{1}{2\pi}|$. Firstly, using the formula \eqref{equ:HeatKerR}, we can rewrite the function $\Gamma_3$ on $\T^1$ as follows:
%%
\begin{lemma}$\;$\\
\label{EquiFormHeatKer}
For any $t>0$ and for any $x_3\in\T^1$, we have
\begin{equation*}
g_t(x_3) = \sqrt{\dfrac{\pi}{t}}\exp\left(\dfrac{- x_3^2}{4t} \right)\left( 1+ 2 \sum_{n\geq 1}\exp\left(\dfrac{-\pi^2 n^2}{t} \right)\cosh\left(\dfrac{\pi n x_3}{t}\right) \right).
\end{equation*}
\end{lemma}
%%
\begin{proof}$\;$\\
Using the definition of $g_t(x_3)$, we have
\begin{align*}
g_t(x_3) &= 2\pi \sum_{n\in\Z}\dfrac{1}{(4\pi t)^{1/2}}\exp\left(-\dfrac{(x_3 +2n\pi)^2}{4t}\right)\\
&=\sqrt{\dfrac{\pi}{t}}\exp\left(\dfrac{- x_3^2}{4t} \right)\sum_{n\in\Z} \exp\left( -\dfrac{\pi n x_3}{t}\right) \exp\left( \dfrac{-\pi^2 n^2 }{t}\right)\\
&= \sqrt{\dfrac{\pi}{t}}\exp\left(\dfrac{- x_3^2}{4t} \right)\\
& \left( 1+ \sum_{n\geq 1}\left(\exp\left( \dfrac{\pi n x_3}{t} \right) + \exp\left(-\dfrac{\pi n x_3}{t} \right) \right)  \exp\left( \dfrac{-\pi^2 n^2 }{t}\right)\right),
\end{align*}
which gives the claim, using $\cosh(y) = \frac{e^y + e^{-y}}{2}$.
\end{proof}
%%
Next, using Lemma \ref{EquiFormHeatKer}, we obtain the following estimate
\begin{lemma}$\;$\\
For any $t>0$ and any $x_3\in \T^1 =[-\pi,\pi]$, we have
\begin{equation}
\label{FirstEstHeatKer}
\exp\left(\dfrac{- x_3^2}{4t} \right) g_t(0) \leq g_t(x_3) \leq \left[\sqrt{\dfrac{\pi}{t}} + g_t(0) \right] \exp\left(\dfrac{ - x_3^2}{4t} \right).
\end{equation}
\end{lemma}
%%
\begin{proof}$\;$\\
Using Lemma \ref{EquiFormHeatKer} and the fact that  from $\cosh(y) \geq 1, y\in\R$ we get the lower bound. Indeed, for any $t>0$
\[
g_t(x_3) \geq \exp\left(-\dfrac{ x_3^2}{4t} \right)  \sqrt{\dfrac{\pi}{t}} \left( 1+ 2 \sum_{n\geq 1}\exp\left(\dfrac{-\pi^2 n^2}{t} \right)\right) = \exp\left(-\dfrac{ x_3^2}{4t} \right) g_t(0).
\]
For the upper bound, let us write
\[
S(x_3) = 1+ 2 \sum_{n\geq 1}\exp\left(\dfrac{-\pi^2 n^2}{t} \right)\cosh\left(\dfrac{\pi n x_3}{t}\right).
\]
For any $n\geq 1$, using $|x_3|\leq \pi$
\[
2 \cosh\left(\dfrac{\pi n x_3}{t}\right) \leq 2 \cosh\left(\dfrac{\pi^2 n }{t}\right) = \exp\left( \dfrac{\pi^2 n }{t} \right) + \exp\left(-\dfrac{\pi^2 n }{t} \right) \leq 1+ \exp\left( \dfrac{\pi^2 n }{t}\right).
\]
Therefore
\begin{align*}
S(x_3) &\leq 1 + \sum_{n\geq 1}\exp\left(\dfrac{-\pi^2 n^2}{t} \right)\left( 1+ \exp\left( \dfrac{\pi^2 n }{t}\right) \right)\\
&= 1 + \sum_{n\geq 1} \exp\left(\dfrac{-\pi^2 n^2}{t} \right) + \exp\left(- \dfrac{\pi^2 n(n-1) }{t}\right)\\
&\leq 1 + \sum_{n\geq 1} \exp\left(\dfrac{-\pi^2 n^2}{t} \right) + \exp\left(- \dfrac{\pi^2 (n-1)^2 }{t}\right)\\
&= 2 + 2 \sum_{n\geq 1} \exp\left(\dfrac{-\pi^2 n^2}{t} \right)\\
&= 1+ \sqrt{\dfrac{t}{\pi}}g_t(0).
\end{align*}
Together with  $g_t(x_3) = \sqrt{\dfrac{\pi}{t}}\exp\left(\dfrac{- x_3^2}{4t} \right) S(x_3)$ implies the upper bound we wanted to prove.
\end{proof}
%%
We need the estimate of the function $g_t(x_3)$ at $x_3 =0$.
\begin{lemma}$\;$\\
\label{Estginit}
For any $t>0$, we have
\[
\sqrt{\dfrac{\pi}{t}} \leq g_t(0) \leq 1+ \sqrt{\dfrac{\pi}{t}},
\]
and 
\[
2 e^{-t} \leq g_t(0) -1 \leq \dfrac{2 e^{-t}}{1- e^{-t}}.
\]
Consequently, there exist positive constants $C_1, C_2$ such that $ | g_t(0) -1| \leq C_1 \frac{e^{-C_2 t}}{\sqrt{t}}$.
\end{lemma}
%%
\begin{proof}$\;$\\
Using Lemma \ref{EquiFormHeatKer} with $x_3 =0$ gives $g_t(0)\geq \sqrt{\frac{\pi}{t}}$. By formula \eqref{equ:HeatKerR} we have 
\[
g_t(0)-1 = 2 \phi(t),\,\,\,\phi(t) = 2\sum_{n\geq 1} e^{-n^2 t}.
\]
Since $e^{- x^2 t}$ is positive and decreasing, bounding a sum by an integral we get
\[
\phi(t)\leq \int_{0}^{\infty}e^{-x^2 t}\mathrm{d} x = \dfrac{1}{\sqrt{t}} \int_{0}^{\infty}e^{-x^2 }\mathrm{d} x = \dfrac{1}{2}\sqrt{\dfrac{\pi}{t}},
\]
hence $g_t(0) \leq 1+ \sqrt{\dfrac{\pi}{t}}$. Moreover, $\phi(t)\geq e^{-t}$ we have $g_t(0) -1 \geq 2 e^{-t}$. Finally, since $e^{-n^2 t}\leq e^{-n t}$, for any $n \geq 1$, we deduce that
$
\phi(t) \leq 2 \sum_{n\geq 1} e^{-n t} = \dfrac{2 e^{-t}}{1 - e^{-t}},
$
which gives $g_t(0)-1 \le\frac{2 e^{-t}}{1 - e^{-t}}$. To finish Lemma, it remains to prove $| g_t(0) -1| \leq \frac{e^{-t}}{\sqrt{t}}$, for any $t>0$. Indeed, we have
\begin{align*}
| g_t(0) -1| &=  (g_t(0) -1)\mathds{1}_{\left\{0<t<1\right\}}+  (g_t(0) -1)\mathds{1}_{\left\{t\geq1\right\}}\\
&\leq \sqrt{\dfrac{\pi}{t}} e^{-t} e^{t}\mathds{1}_{\left\{0<t<1\right\}} + \dfrac{2 e^{-t}}{1- e^{-t}}\sqrt{t}\dfrac{1}{\sqrt{t}}\mathds{1}_{\left\{t\geq1\right\}}\\
&\leq C_1 \dfrac{e^{-C_2 t}}{\sqrt{t}},
\end{align*}
for some positive constants $C_1$ and $C_2$.
\end{proof}
%%
Now, the following lemma provides estimates of $\Gamma_3 - \frac{1}{2\pi}$ and its derivatives on $\T^1$. 
\begin{lemma}$\;$\\
\label{BoundHeatT1}
Let $\Gamma_3(t,x_3) = \frac{1}{2\pi}\left[1 + 2 \sum_{n=1}^{\infty}e^{-n^2 t}\cos(n x_3) \right]$ be the heat kernel on $\T^1$. Then there exist constants $C_1, C_2$ and $C_3$  which can change from line to line such that
\begin{equation}
\label{FirstEstHeatKerbis}
\left| \Gamma_3 (t,x_3) - \dfrac{1}{2\pi} \right| \leq C_1\dfrac{1}{\sqrt{t}}e^{-C_2 t} e^{-C_3 x_3 ^2 /{4t}},\,\, t>0, x_3 \in\T^1,
\end{equation}
\begin{equation}
\label{SecEstHeatKer}
|\partial_{x_3}\Gamma_3 (t,x_3)| \leq C_1\dfrac{1}{t}e^{-C_2 t} e^{-C_3 x_3 ^2 /{4t}},\,\, t>0, x_3 \in\T^1,
\end{equation}
\begin{equation}
\label{ThirdEstHeatKer}
|\partial_{x_3}^2 \Gamma_3 (t,x_3)| \leq C_1\dfrac{1}{t^{3/2}}e^{-C_2 t} e^{-C_3 x_3 ^2 /{4t}},\,\, t>0, x_3\in\T^1.
\end{equation} 
\end{lemma}
\begin{proof}$\;$\\
Readers can find these results in \cite{CarPanZa20}, even when $\T^1$ is replaced by more general compact manifold, cf. \cite{TianZa, Zhang}. We provide here the main lines of the proof.\\
For the bound \eqref{FirstEstHeatKerbis}, it is easily obtained from the consequence of Lemma \ref{Estginit} for $t \geq 1$. If $t \leq 1$, first using \eqref{equ:HeatKerR} yields
$
\Gamma_3(t,x_3) -\frac{1}{2\pi} = \frac{1}{2\pi}(g_t(x_3) -1)
$
then \eqref{FirstEstHeatKer} we have
\begin{equation}
\label{LowUppHeatKer}
\dfrac{1}{2\pi}\left[\exp\left(\dfrac{- x_3^2}{4t} \right) g_t(0)-1\right] \leq \Gamma_3(t,x_3) -\dfrac{1}{2\pi} \leq \dfrac{1}{2\pi} \left[ \exp\left(\dfrac{ - x_3^2}{4t} \right)\left(\sqrt{\dfrac{\pi}{t}} + g_t(0) -1\right)\right].
\end{equation}
Using the upper bound in \eqref{LowUppHeatKer} and Lemma \ref{Estginit} we deduce that
\begin{align*}
\Gamma_3(t,x_3) -\dfrac{1}{2\pi} \leq \dfrac{1}{2\pi}\exp\left(\dfrac{ - x_3^2}{4t} \right)\sqrt{\dfrac{\pi}{t}} + \dfrac{1}{2\pi}\exp\left(\dfrac{ - x_3^2}{4t} \right) \dfrac{2 e^{-t}}{1- e^{-t}}.
\end{align*}
If $t \in [\delta_0, 1]$, for some $\delta_0\in (0,1)$, it's not hard to show from the previous inequality that there exist positive constants $C_1, C_2$ and $ C_3$ such that $\Gamma_3(t,x_3) -\frac{1}{2\pi} \leq C_1\frac{1}{\sqrt{t}}e^{-C_2 t} e^{-C_3 x_3 ^2 /{4t}}$. On the other hand, for any positive test function $\varphi\in C^\infty_0(\R)$, since $\lim_{t\to 0^+}\left<\Gamma_3 -\frac{1}{2\pi}, \varphi \right> = (1-1/2\pi)\varphi(x_3)$ and $\lim_{t\to 0^+} \left< k_t,\varphi \right> = \varphi(x_3)$, where $k_t(x_3) = (4\pi t)^{-1/2}e^{-x_3 ^2 /{4t}}$ is the heat kernel on $\R$, we deduce that we can choose the positive constants as above to obtain the previous estimate of $\Gamma_3 -1/2$ as $t\to 0^+$. Together, these arguments give us the upper bound of \eqref{FirstEstHeatKerbis}. 
Similarly, by using the lower bound in \eqref{LowUppHeatKer}, we infer the lower bound in \eqref{FirstEstHeatKerbis}. Therefore, we obtain the estimate \eqref{FirstEstHeatKerbis}. Now, for the estimates \eqref{SecEstHeatKer} and \eqref{ThirdEstHeatKer}, we apply  Lemma $2.1$ in \cite{TianZa}, which can be extended to the parabolic case, see Lemma $2.3$ in \cite{TianZa}
\[
|\nabla_x u(t,x_3)| \leq \dfrac{C}{r}\left( \dfrac{1}{r^4}\int_{t-r}^{r}\int_{|y- x_3|<r}|u(s,y)|^2 \mathrm{d}y\mathrm{d}s \right)^{1/2},
\]
where $u$ is asolution of the heat equation $\partial_t u -\Delta_{x_3}u =0$ in the domain $ [t-r^2,t]\times B(x_3,r)$, with $ r =\sqrt{t}/2 $ for any fixed point $(t,x_3)\in \R\times\T^1$.
\end{proof}
%%
In the next lemma, we provide estimates for the function $\Gamma$ and its derivatives using the relation \eqref{equ:EquLapHeat} and the inequalities \eqref{FirstEstHeatKerbis}, \eqref{SecEstHeatKer} and \eqref{ThirdEstHeatKer}.
%%
\begin{lemma}$\;$\\
\label{EstFundSol}
Let $\Gamma(x)$ be the function on $\R^2\times \T^1$ provided by Lemma \ref{PerFundSol}. Then we have the following estimates
\[
|\Gamma(x)| \leq  \dfrac{C}{|x|},\,\,
 |\partial_x\Gamma(x)| \leq \dfrac{C}{|x|^2},\,\, |D^2_x \Gamma(x)|\leq \dfrac{C}{|x|^3},
\]
where $D^2_x$ denotes the second order derivative. Here, $C$ stands for a positive constant, which can vary in each estimate.
\end{lemma}
%%
\begin{proof}$\;$\\
We will first estimate $\Gamma(x)$. Thanks to \eqref{equ:EquLapHeat} and \eqref{FirstEstHeatKerbis}, we deduce that
\begin{align*}
|\Gamma(x)| &\leq \dfrac{C_1}{4\pi} \int_{0}^{\infty}t^{-3/2} e^{-C_2 t}e^{-|\bar{x}|^2/{4t}}e^{- C_3 x _3 ^2 /{4t}}\marmd t \\
&\leq \dfrac{C_1}{4\pi} \int_{0}^{\infty}t^{-3/2} e^{-C_2 t} e^{-C_3 '|x|^2/{t}}\marmd t,\,\,\,\, C_3 ' = \min{(1, C_3)}/4\\
&= \dfrac{C_1}{4\pi} e^{ -2 \sqrt{C_2 C' _3} |x|} \int_{0}^{\infty}t^{-3/2}e^{-C_2 t + 2\sqrt{C_2 C' _3} |x| - C' _3 |x|^2/{t}} \marmd t \\
&=  \dfrac{C_1}{4\pi} e^{ -2 \sqrt{C_2 C' _3 }|x|} \int_{0}^{\infty} e^{-\left(\frac{\sqrt{C'_3}|x| - \sqrt{C_2 }t}{\sqrt{t}} \right)^2} 2\marmd (- t^{-1/2})\\
&=  \dfrac{C_1}{2\pi} e^{ -2 \sqrt{C_2 C' _3 }|x|} \int_{0}^{\infty} e^{-\left(\sqrt{C' _3}|x| u - \sqrt{C_2}u^{-1} \right)^2} \marmd u,\,\,\, u = t^{-1/2}\\
&=  \dfrac{C_1}{2\pi}\dfrac{1}{\sqrt{C' _3}|x|} e^{ -2 \sqrt{C_2 C' _3 }|x|} \int_{0}^{\infty} e^{- \left(\theta - \sqrt{C_2 C' _3 }|x|\theta^{-1} \right)^2} \marmd \theta,\,\,\, \theta = \sqrt{C' _3}|x| u\\
& \leq \dfrac{C}{|x|},
\end{align*}
for some positive constant $C$, where we have used that 
\begin{equation}
\label{equ:GaussInt}
\int_{0}^{\infty} e^{- \left(\theta - \sqrt{C_2 C' _3 }|x|\theta^{-1} \right)^2} \marmd \theta = \dfrac{\sqrt{\pi}}{2},
\end{equation}
see the proof of Lemma \ref{GaussIntegral} in Appendix.

Next we estimate  $\nabla_x\Gamma(x)$. Taking the derivative with respect to $x$ in the formula \eqref{equ:EquLapHeat} we deduce that
\begin{align*}
|\nabla_x \Gamma (x)| \leq \int_{0}^{\infty}|\nabla_{\bar{x}} \Gamma_{1,2}(t,\bar{x})||\Gamma_3(t,x_3)-\frac{1}{2\pi}|\marmd t + \int_{0}^{\infty}|\Gamma_{1,2}(t,\bar{x})||\partial_{x_3}\Gamma_3(t,x_3)|\marmd t.
\end{align*}
A simple computation show that
\[
\nabla_{\bar{x}} \Gamma_{1,2}(t,\bar{x}) = \dfrac{-|\bar{x}|}{8\pi t^2}e^{-|\bar{x}|^2/{4t}},
\]
and thanks to the estimates \eqref{FirstEstHeatKerbis} and \eqref{SecEstHeatKer} we obtain 
\begin{align*}
|\nabla_x \Gamma (x)|  &\leq  \dfrac{C_1|\bar{x}|}{8\pi}\int_{0}^{\infty} t^{-5/2}e^{-C_2 t}e^{-|\bar{x}|^2/{4t}}e^{- C_3 x _3 ^2 /{4t}}\marmd t + \dfrac{C_1}{4\pi}\int_{0}^{\infty}t^{-2}e^{-C_2 t}e^{-|\bar{x}|^2/{4t}}e^{- C_3 x _3 ^2 /{4t}}\marmd t\\
&\leq \dfrac{C_1|\bar{x}|}{8\pi}\int_{0}^{\infty} t^{-5/2}e^{-C_2 t}e^{-C'_3|x|^2/t}\marmd t + \dfrac{C_1}{4\pi} \int_{0}^{\infty}t^{-2}e^{-C_2 t}e^{-C'_3|x|^2/t}\marmd t ,
\end{align*}
where $ C'_3 = \min(1,C_3)/4 $. Using $\sup_{\R_+ ^\star}q(t) = q(C'_3|x|^2)$ where $q(t) =t^{-1/2}e^{-C'_3 |x|^2 /{2t}}$ for the first integral on the last line of the previous inequality, we deduce that
\begin{align*}
|\nabla_x \Gamma (x)|  &\leq \left(\dfrac{C_1}{8\pi}\dfrac{1}{\sqrt{C'_3 e}} + \dfrac{C_1}{4\pi}\right)\int_{0}^{\infty} t^{-2}e^{-C'_3|x^2|/2t} \marmd t\\
&= \left(\dfrac{C_1}{8\pi}\dfrac{1}{\sqrt{C'_3 e}} + \dfrac{C_1}{4\pi}\right)\dfrac{2}{C'_3|x|^2}\int_{0}^{\infty} \frac{\marmd }{\marmd t}e^{-C'_3|x^2|/2t}\marmd t\\
&\leq \dfrac{C}{|x|^2},
\end{align*}
for some positive constant $C$.

Finally, we estimate $D^2_x \Gamma(x)$. By direct computation in \eqref{FirstEstHeatKerbis}, we have
\begin{align*}
|D^2_x \Gamma(x)| &\leq \int_{0}^{\infty} |D^2_{\bar{x}} \Gamma_{1,2}(t,\bar{x})||\Gamma_3(t,x_3)-\frac{1}{2\pi}|\marmd t + \int_{0}^{\infty} |\nabla_{\bar{x}} \Gamma_{1,2}(t,\bar{x})||\partial_{x_3}\Gamma_3(t,x_3)|\marmd t \\
&+ \int_{0}^{\infty} |\Gamma_{1,2}(t,\bar{x})||\partial^2 _{x_3}\Gamma_3(t,x_3)|\marmd t.
\end{align*}
Since
\[
D^2_{\bar{x}} \Gamma_{1,2}(t,\bar{x}) = \dfrac{1}{8\pi t^2}\left[-I_2 + \dfrac{\bar{x}\otimes\bar{x}}{2t} \right]e^{-|\bar{x}|^2/{4t}},
\]
it implies that
\begin{align*}
|D^2_{\bar{x}} \Gamma_{1,2}(\bar{x})| \leq \dfrac{1}{8\pi t^2}e^{-|\bar{x}|^2/{4t}} + \dfrac{|\bar{x}|^2}{16\pi t^3} e^{-|\bar{x}|^2/{4t}}.
\end{align*}
Using the inequalities \eqref{FirstEstHeatKerbis}, \eqref{SecEstHeatKer} and \eqref{ThirdEstHeatKer} we deduce that
\begin{align*}
|D^2_x \Gamma(x)| &\leq \dfrac{C_1}{8\pi}\int_{0}^{\infty}t^{-5/2}e^{-C_2 t}e^{-|\bar{x}|^2/{4t}}e^{- C_3 x _3 ^2 /{4t}}\marmd t \\&+  \dfrac{C_1|\bar{x}|^2}{16\pi}\int_{0}^{\infty}t^{-7/2}e^{-C_2 t}e^{-|\bar{x}|^2/{4t}}e^{- C_3 x _3 ^2 /{4t}}\marmd t \\
&+ \dfrac{C_1|\bar{x}|}{8\pi}\int_{0}^{\infty}t^{-3}e^{-C_2 t}e^{-|\bar{x}|^2/{4t}}e^{- C_3 x _3 ^2 /{4t}}\marmd t\\
&+ \dfrac{C_1}{4\pi}\int_{0}^{\infty}t^{-5/2}e^{-C_2 t}e^{-|\bar{x}|^2/{4t}}e^{- C_3 x _3 ^2 /{4t}}\marmd t\\
& =: I_1 + I_2 + I_3 + I_4.
\end{align*}
The estimates of the integrals $I_1$ and $I_4$ are performed as above. Thus we get 
\[
I_1 \leq \dfrac{C}{|x|^3},\,\, I_4 \leq \dfrac{C}{|x|^3}.
\]
For the integral $I_3$, we have that
\[
I_3 \leq \dfrac{C_1|x|}{8\pi^2}\int_{0}^{\infty}t^{-3}e^{-C_2 t}e^{-C'_3|x|^2/{t}}\marmd t,\,\, C'_3 =\min(1,C_3)/4.
\]
Using again $\sup_{\R_+ ^\star}h(t) = h(C'_3|x|^2)$ where $h(t) =t^{-1/2}e^{-C'_3 |x|^2 /{2t}}$, we obtain that
$
I_3 \leq \dfrac{C}{|x|^3}
$, for some positive constant $C$. 
Similarly for integral $I_2$, we also have
\begin{align*}
I_2 &\leq  \dfrac{C_1|x|^2}{32\pi^2}\int_{0}^{\infty}t^{-7/2}e^{-C_2 t}e^{-C'_3|x|^2/{t}}\marmd t,\,\, C'_3=\min(1,C_3)/4\\
&\leq \dfrac{C_1|x|^2}{32\pi^2}\dfrac{1}{\sqrt{C'_3 e}|x|} \int_{0}^{\infty}t^{-3}e^{-C_2 t}e^{-C'_3|x|^2 /2t} \marmd t\\
&\leq \dfrac{C}{|x|^3}.
\end{align*}
Together, the estimations of $I_i$, for any $ i=1,...,4$ will provide us the estimate of $D^2_{x}\Gamma(x)$.
\end{proof}

Thanks to  Lemma \ref{PerFundSol} and the $L^\infty$ estimate for the function $\Gamma$ in Lemma \ref{EstFundSol}, and following the same arguments as in the proof for Poisson's equation in $\R^3$, we can show that the solution of the Poisson equation \eqref{equ:PoiCylinLim} is given by
\begin{equation}
\label{equ:PoissonPoten}
\Phi[n](x) = \dfrac{q}{\epsilon_0}\Xi \star n(x) = \dfrac{q}{\epsilon_0}\intyzt{\Xi(x-y)n(y)}. 
\end{equation} 

%%
\subsection{Estimations for the electric field and its gradient on $\R^2 \times \T^1$}
We give now some estimates of the electric field $E[n] = -\nabla_x\Phi[n]$ which can be proved by treating the singular term in the fundamental solution $\Xi$ as in, cf. \cite{Batt} for the space domain $x\in\R^3$ and \cite{ReiRen1993} for $x\in\T^3$.
%%
\begin{lemma}$\;$\\
\label{FirstEstiEle}
Let $n$ be a positive concentration and belongs to $ L^1(\R^2\times \T^1) \cap L^\infty(\R^2\times \T^1)$. Then, there exists a constant $C>0$ such that the electric field $E[n]$ satisfies the following estimate:
\[
\| E[n] \|_{L^\infty} \leq C (\| n\|_{L^\infty} + \| n\|_{L^1}).
\]
\end{lemma}
\begin{proof}$\;$\\
For any $x = (\bar{x},x_3)\in \R^2 \times [-\pi,\pi]$, by the formula \eqref{equ:PoissonPoten} we have
\begin{align*}
\nabla_x \Phi[n](x) &= -\dfrac{q}{4\pi^2 \epsilon_0}\intyzt{\nabla_{\bar{x}}\ln |\bar{x} - \bar{y}| n(y)} + \dfrac{q}{\epsilon_0}\intyzt{\nabla_x \Gamma(x-y)\,n(y)}\\
&=-\dfrac{q}{4\pi^2 \epsilon_0}\intyzt{\dfrac{\bar{x} -\bar{y}}{|\bar{x}-\bar{y}|^2} n(y)} +  \dfrac{q}{\epsilon_0} \int_{x_3 -\pi}^{x_3 + \pi}\intyyt{\nabla_x\Gamma(x-y)\,n(y)}\marmd y_3.
\end{align*}
The first integral in the previous expression can be estimated as
\begin{align*}
&\dfrac{q}{4\pi^2 \epsilon_0} \left( \intyzt{\nabla_{\bar{x}}\ln |\bar{x} - \bar{y}| \mathds{1}_{\left\{ |\bar{x} - \bar{y}|\leq 1 \right\}} n(y)} + \intyzt{\nabla_{\bar{x}}\ln |\bar{x} - \bar{y}| \mathds{1}_{ \left\{ |\bar{x} - \bar{y}|\geq 1 \right\}} n(y)} \right)\\
& \leq C (\|n \|_{L^\infty} + \|n \|_{L^1}).
\end{align*}
For the second integral, we make a decomposition of $\R^2\times\T^1$ in the following way
\[
\R^2\times [x_3 -\pi, x_3 + \pi]: = I \cup J ,
\]
where 
\[
I= \left\{ y\in \R^3: |x-y| \geq 1 \right\} \cap \R^2\times [x_3 -\pi, x_3 + \pi]   ,
 \]
 \[
J= \left\{ y\in\R^3: |x-y| \leq 1 \right\}.
 \]
It is obviously that $J \subseteq \R^2\times [x_3 -\pi, x_3 + \pi] $. Thus the last integral in the previous equality can be written
\[
\int_{x_3 -\pi}^{x_3 + \pi}\intyyt{\nabla_x\Gamma(x-y)\,n(y)}\marmd y_3 = \int_{I} \nabla_x\Gamma(x-y)\,n(y) \marmd y + \int_{J}\nabla_x\Gamma(x-y)\,n(y) \marmd y.
\]
Thanks to Lemma \ref{EstFundSol}, we deduce that
\begin{align*}
\dfrac{q}{\epsilon_0}\int_{x_3 -\pi}^{x_3 + \pi}\intyyt{\nabla_x\Gamma(x-y)\,n(y)}\marmd y_3 &\leq C \left[ \int_{I} \dfrac{1}{|x-y|^2} n(y)\marmd y + \int_{J}\dfrac{1}{|x-y|^2} n(y)\marmd y \right] \\
&\leq C\left[ \int_{x_3 -\pi}^{x_3 + \pi}\intyyt{n(y)}\marmd y_3 + \int_{|x-y|\leq 1}  \dfrac{1}{|x-y|^2} n(y) \marmd y \right]\\
&\leq C \left[ \intyzt{n(y)} + 4\pi \| n\|_{L^\infty}\right]\\
&\leq C (\|n\|_{L^1} + \| n\|_{L^\infty}),
\end{align*}
where we have used that  $\int_{|x-y|\leq 1} \frac{1}{|x-y|^2} \marmd y =4\pi$. Combining these estimates, we obtain the desired result in Lemma.
\end{proof}
%%
\begin{lemma}$\;$\\
\label{SecondEstEle}
Let $n \in L^1(\R^2\times \T^1)\cap W^{1,\infty}(\R^2\times\T^1)$ and $n\geq 0$. There exists a constant $C>0$ such that the gradient of the electric field $E[n]$ satisfies the following estimates
\[
\| \nabla_x E[n] \|_{L^\infty} \leq C \left( 1 + \| n\|_{L^\infty}(1+\ln ^{+}( \|\nabla_x n \|_{L^\infty})) + \|n\|_{L^1} \right),
\]
where the notation $\ln^+$ stands for the positive part of $\ln$.
\end{lemma}
\begin{proof}$\;$\\
Observe that
\begin{align*}
\Phi[n](x) &= -\dfrac{q}{4\pi^2 \epsilon_0}\intyzt{\ln ( |\bar{x} - \bar{y}|) n(y)} +  \dfrac{q}{\epsilon_0}\intyzt{\Gamma(x-y)\,n(y)}   \\
&= -\dfrac{q}{4\pi^2 \epsilon_0}\intyzt{\ln (|\bar{y}|) n(\bar{x}-\bar{y},x_3 -y_3)} + \dfrac{q}{\epsilon_0}\intyzt{\Gamma(y)\,n(x-y)},
\end{align*}
because the functions $\Gamma$ and $n$ are periodic with respect to $x_3$ of period $2\pi$. We estimate now $\partial_{x_1}^2 \Phi[n](x)$. In other cases, we can do the same. Taking the derivative in the variable $x_1$ of the above equality, we have
\begin{align*}
\partial_{x_1} \Phi[n](x) &= -\dfrac{q}{4\pi^2 \epsilon_0}\intyzt{\ln (|\bar{y}|) \partial_{x_1} n(\bar{x}-\bar{y},x_3 -y_3)} + \dfrac{q}{\epsilon_0}\intyzt{\Gamma(y)\partial_{x_1}n (x-y)}\\
&=\dfrac{q}{4\pi^2 \epsilon_0}\intyzt{\ln (|\bar{y}|) \partial_{y_1} n(\bar{x}-\bar{y},x_3 -y_3)} - \dfrac{q}{\epsilon_0}\intyzt{\Gamma(y)\partial_{y_1}n (x-y)}\\
&= \dfrac{q}{4\pi^2 \epsilon_0}\intyzt{\ln (|\bar{x}-\bar{y}|) \partial_{y_1} n(\bar{y},y_3)} - \dfrac{q}{\epsilon_0}\intyzt{\Gamma(x-y)\partial_{y_1}n(y)},
\end{align*}
which implies that
\begin{align*}
\partial_{x_1}^2 \Phi[n](x) &= \dfrac{q}{4\pi^2 \epsilon_0}\intyzt{ \partial_{x_1}\ln (|\bar{x}-\bar{y}|) \partial_{y_1} n(\bar{y},y_3)} -\dfrac{q}{\epsilon_0}\intyzt{\partial_{x_1}\Gamma(x-y)\partial_{y_1}n(y)}\\
&= \dfrac{q}{4\pi^2 \epsilon_0}\intyzt{ \dfrac{x_1 -y_1}{|\bar{x}-\bar{y}|^2} \partial_{y_1} n(\bar{y},y_3)}  -\dfrac{q}{\epsilon_0}\int_{x_3 -\pi}^{x_3 +\pi}\intyyt{\partial_{x_1}\Gamma(x-y)\partial_{y_1}n(y)}\marmd y_3\\
&=: K_1 + K_2. 
\end{align*}
The estimation of $K_1$, see \cite{Batt}. We estimate now $K_2$.
Let $r, R >0$ such that $0< r < R < \infty$ verify
\[
\left\{ y\in\R^3 : |x-y| < R\right\} \subset \R^2 \times [x_3 - \pi, x_3 + \pi].
\]
Then we make a decomposition of $\R^2 \times [x_3 - \pi, x_3 + \pi]$ in the following way
 \[\R^2 \times [x_3 - \pi, x_3 + \pi] := J_1 \cup J_2 \cup J_3 ,\] where
\[
J_1 =\left\{ y\in\R^3 : |x-y| > R\right\} \cap \R^2 \times [x_3 - \pi, x_3 + \pi] ,
\]
\[
J_2 = \left\{ y\in\R^3 : r < |x-y| < R\right\},\,\,\,\,
J_3 = \left\{ y\in\R^3 : |x-y| < r\right\}.
\]
For the integral on $J_1$, thanks to the integration by parts with respect to $y_1$ and noticing that the boundary of $J_1$ is $\partial J_1 = \left\{ y\in\R^3: |x-y| = R\right\} \cup \R^2\times \left\{x_3 -\pi, x_3 +\pi \right\} $, we get
\begin{align}
\label{equ:EquJ1}
&- \int_{J_1}\partial_{x_1}\Gamma(x-y)\partial_{y_1}n(y)\marmd\bar{y}\marmd y_3 \nonumber \\ &=  \int_{J_1}\partial_{y_1}\partial_{x_1}\Gamma(x-y) n(y) \marmd\bar{y}\marmd y_3 - \int_{|x-y|=R} \partial_{x_1}\Gamma(x-y) n(y) \dfrac{-(x_1 - y_1)}{|x-y|}\marmd \sigma(y)\nonumber\\
&- \intyyt{\underbrace{\left[ \partial_{x_1} \Gamma(x-(\bar{y},x_3 +\pi))n(\bar{y},x_3 + \pi) - \partial_{x_1} \Gamma(x-(\bar{y},x_3 -\pi))n(\bar{y},x_3 -\pi) \right]}_{=0}}.
\end{align}
Similarly, the integral on $J_2$ can be written as
\begin{align}
\label{equ:EquJ2}
&- \int_{J_2}\partial_{x_1}\Gamma(x-y)\partial_{y_1}n(y)\marmd\bar{y}\marmd y_3 \nonumber\\
&= \int_{J_2}\partial_{y_1}\partial_{x_1}\Gamma(x-y) n(y) \marmd\bar{y}\marmd y_3 - \int_{|x-y|= R} \partial_{x_1}\Gamma(x-y) n(y) \dfrac{(x_1 - y_1)}{|x-y|}\marmd \sigma(y)\nonumber\\
&- \int_{|x-y|= r} \partial_{x_1}\Gamma(x-y) n(y) \dfrac{-(x_1 - y_1)}{|x-y|}\marmd \sigma(y).
\end{align}
For the integral on $J_3$, since $\partial_{y_1}n(y) = \partial_{y_1}[n(y)-n(x)]$ and then using the integration by parts, we obtain
\begin{align}
\label{equ:EquJ3}
&- \int_{J_3}\partial_{x_1}\Gamma(x-y)\partial_{y_1}n(y)\marmd\bar{y}\marmd y_3
= - \int_{J_3}\partial_{x_1}\Gamma(x-y)\partial_{y_1}[n(y)-n(x)]\marmd\bar{y}\marmd y_3 \nonumber\\
&= \int_{J_3}\partial_{y_1}\partial_{x_1}\Gamma(x-y) [n(y)-n(x)] \marmd\bar{y}\marmd y_3 - \int_{|x-y|=r}\partial_{x_1}\Gamma(x-y) [n(y)-n(x)]\dfrac{(x_1 - y_1)}{|x-y|}\marmd \sigma(y).
\end{align}
Combining the equalities \eqref{equ:EquJ1}, \eqref{equ:EquJ2} and \eqref{equ:EquJ3} we deduce that
\begin{align*}
\partial_{x_1}^2 \Phi[n](x)&=\int_{J_1}\partial_{y_1}\partial_{x_1}\Gamma(x-y) n(y) \marmd\bar{y}\marmd y_3 + \int_{J_2}\partial_{y_1}\partial_{x_1}\Gamma(x-y) n(y) \marmd\bar{y}\marmd y_3\\
&+ \int_{J_3}\partial_{y_1}\partial_{x_1}\Gamma(x-y) [n(y)-n(x)] \marmd\bar{y}\marmd y_3 \\&+ \int_{|x-y|=r}\partial_{x_1}\Gamma(x-y) n(x)\dfrac{(x_1 - y_1)}{|x-y|}\marmd \sigma(y)\\
&:= I_1 + I_2 + I_3 + I_4.
\end{align*}
Thanks to Lemma \ref{EstFundSol}, we will estimate the integrals $I_i$, for any $i=1,...,4$.\\
For the integral $I_4$, using the $L^\infty$ estimate of $\partial_x \Gamma$ we have
\[
I_4 \leq C \int_{|x-y| =r} \dfrac{1}{|x-y|^2}\marmd \sigma(y) \| n \|_{L^\infty}= 4\pi C \|n\|_{L^\infty}.
\]
For the integral $I_3$, using the $L^\infty$ estimate of $\partial_x ^2 \Gamma$ we also get
\[
I_3 \leq C \int_{|x-y| < r} \dfrac{1}{|x-y|^3}|x-y| \marmd y \|\nabla_x n\|_{L^\infty} = 2\pi^2 C r\|\nabla_x n\|_{L^\infty}.
\]
Similarly for the integral $I_2$ and the integral $I_1$, we obtain
\[
I_2 \leq C \int_{r< |x-y| < R} \dfrac{1}{|x-y|^3}\marmd y \|n\|_{L^\infty} = 2\pi^2 C\ln(R/r)  \|n\|_{L^\infty},
\]
\[
I_1 \leq C \int_{J_1} \dfrac{1}{|x-y|^3}n(y)\marmd y \leq \dfrac{C}{R^3}\|n\|_{L^1}.
\]
Finally, together these estimates of $I_i$, for any $i=1,...,4$ we obtain
\[
K_2 \leq C \left(  \dfrac{1}{R^3}\|n\|_{L^1}+  \ln(R/r)  \|n\|_{L^\infty}+ r\|\nabla_x n\|_{L^\infty}+  \|n\|_{L^\infty}  \right).
\]
Taking $r= \frac{1}{1 + \|\nabla n\|_{L^\infty}}$ and $R=1$ which gives us the result of the lemma.
\end{proof}
%%
\subsection{Local existence of smooth solutions}
Let's begin to establish strong solutions for the limit model. It is sufficient to  construct a solution on some time interval $[0,T]$, $T>0$. We present only the main arguments, the other details being left to the reader. We assume that the initial condition $n_{\mathrm{in}}$ satisfies the hypotheses
\begin{enumerate}
\item[H1)] $n_{\mathrm{in}} \geq 0$,
\item[H2)] $n_{\mathrm{in}}\in W^{1,\infty}(\R^2\times\T^1)\cap W^{1,1}(\R^2\times\T^1)$.
\end{enumerate}
%%
\paragraph*{Solution integrated along the characteristics}$\;$\\
A standard computation, we can rewrite the equation \eqref{equ:EquCylinModLim} as
\begin{equation}
\label{equ:EquCylinModLimBis}
\partial_t n + \left( E \wedge \dfrac{e}{B} \right)\cdot\nabla_x n + \sigma \rot_x \left(\dfrac{e}{\omega_c} \right)\cdot \nabla_x n - \rot_x\left( \dfrac{e}{B} \right)\cdot E n =0,\,\, (t,x)\in \R_+\times \R^2\times\T^1.
\end{equation}
For any smooth field $E\in L^\infty(0,T;W^{1,\infty}(\R^2\times\T^1))$, we consider the associated characteristics flow of this equation
\begin{align}
\label{equ:EquChaLim}
\left\{
    \begin{array}{ll}
\dfrac{\marmd}{\marmd t}\Pi(t;s,x) = E(t,\Pi(t;s,x))\wedge \dfrac{e(\Pi(t;s,x))}{B(\Pi(t;s,x))}+ \sigma \rot_x\left( \dfrac{e}{\omega_c} \right)(\Pi(t;s,x)),\\
\Pi(s;s,x) = x\in \R^2\times\T^1,
\end{array}
\right.
\end{align}
where $\Pi(t;s,x)$ is the solution of the ODE, $t$ represents the time variable, $s$ is the initial time and $x$ is the initial position. $\Pi(s;s,x) =x$ is our initial condition. Notice that the vector field $\frac{e}{B}$ is also smooth and belongs to $W^{2,\infty}(\R^2\times\T^1)$. Therefore, the characteristics in \eqref{equ:EquChaLim} are well defined for any $(s,x)\in [0,T]\times\R^2\times\T^1$ and there are smooth with respect to $x$. From \eqref{equ:EquChaLim}, the equation \eqref{equ:EquCylinModLimBis} can be written as
\[
\dfrac{\marmd }{\marmd t}n(t,\Pi(t;s,x)) - \rot_x\left(\dfrac{e}{B}\right)(\Pi(t;s,x))\cdot E(t,\Pi(t;s,x))n(t,\Pi(t;s,x)) = 0.
\]
The solution of the transport equation \eqref{equ:EquCylinModLimBis} is given by
\begin{equation}
\label{SolChaPi}
n(t,x) = n_{\mathrm{in}}(\Pi(0;t,x)) \exp\left( \int_{0}^{t}\rot_x\left(\dfrac{e}{B} \right)(\Pi(s;t,x))\cdot E(s,\Pi(s;t,x))\marmd s \right).
\end{equation}
%%
\paragraph*{Conservation law on a volume}$\;$\\
We have the following conservation law
\begin{align}
\label{ConserVolu}
\intxlt{n(t,x)} = \intxlt{n_{\mathrm{in}}(x)},\,\, 0\leq t\leq T.
\end{align}
Indeed, we denote $J(t;s,x)$ is the Jacobian matrix  of $\Pi(t;s,x)$ with respect to $x$ at $(t;s,x)$. The determinant of the Jacobian matrix $J(t;s,x)$ is given by
\begin{align*}
\left\{
\begin{array}{ll}
\dfrac{\marmd}{\marmd t} \mathrm{det} \left( J(t;s,x) \right)  = \Divx \left( E(t) \wedge \dfrac{e}{B} + \sigma\rot_x \left(\dfrac{e}{\omega_c}\right)\right)(\Pi(t;s,x)) \mathrm{det} \left( J(t;s,x) \right),\\
\mathrm{det}(J(t;t,x)) =1.
\end{array}
\right.
\end{align*} 
Hence, we obtain
\[
 \mathrm{det} \left( J(t;s,x) \right) = \exp\left( -\int_{0}^{t}\rot_x\left( \dfrac{e}{B}\right)(\Pi(\theta;s,x))\cdot E(\theta,\Pi(\theta;s,x))\marmd \theta\right).
\]
Integrating the equality \eqref{SolChaPi} with respect to $x$ and then changing the variable $x$ to $\Pi(t;0,x)$, we obtain
\begin{align*}
&\intxlt{n(t,x)} \\
&= \intxlt{n_{\mathrm{in}}(x) \exp\left( \int_{0}^{t}\rot_x\left( \dfrac{e}{B}\right)(\Pi(s;0,x))\cdot E(s,\Pi(s;0,x))\marmd s\right) \mathrm{det} \left( J(t;0,x) \right)}\\
&= \intxlt{n_{\mathrm{in}}(x)}.
\end{align*}
%%
\paragraph*{A priori estimates}
\paragraph*{The bound in $L^\infty(0,T;W^{1,\infty}(\R^2\times\T^1))$ of the solutions}$\;$\\
We have the following bounds
\begin{equation}
\label{NormInfty}
\sup_{t\in[0,T]}\|n(t)\|_{L^\infty(\R^2\times\T^1)} \leq  \|n_{\mathrm{in}}\|_{L^\infty(\R^2\times\T^1)}  \exp(C_B T\sup_{t\in [0,T]}\|E(t)\|_{L^\infty}),
\end{equation}
\begin{align}
\label{GradNormInfty}
\|\nabla_x n(t)\|_{L^\infty(\R^2\times\T^1)} \leq &( \|n_{\mathrm{in}}\|_{L^\infty} +  \exp( C_0 T(1 + \sup_{t\in[0,T]}\|E(t)\|_{W^{1,\infty}}) )  \|\nabla_x n_{\mathrm{in}}\|_{L^\infty} )\\
&\exp(C_0T(1+\sup_{t\in[0,T]}\|E(t)\|_{W^{1,\infty}})),\nonumber
\end{align}
where we denote the constants $C_B = \|e/B\|_{W^{2,\infty}(\R^2\times\T^1)}$ and $C_0=C(\sigma,q,m,B)$.\\
We will first prove \eqref{NormInfty}. By the formular \eqref{SolChaPi}, for any $t\in[0,T]$ we have
\begin{align*}
\|n(t)\|_{L^\infty} &\leq \|n_{\mathrm{in}}\|_{L^\infty} \left\| \exp\left( \int_{0}^{t}\rot_x\left(\dfrac{e}{B} \right)(\Pi(s;t,x))\cdot E(s,\Pi(s;t,x))\marmd s \right) \right\|_{L^\infty}\\
&\leq \|n_{\mathrm{in}}\|_{L^\infty}\exp(T\|\partial_x(e/B)\|_{L^\infty}\sup_{t\in [0,T]}\|E(t)\|_{L^\infty})\\
&\leq \|n_{\mathrm{in}}\|_{L^\infty} \exp(C_B T\sup_{t\in [0,T]}\|E(t)\|_{L^\infty}).
\end{align*}
We then prove \eqref{GradNormInfty}. By taking the derivative with respect to $x$ in the formula \eqref{SolChaPi}, we imply
\[
\|\nabla_x n(t)\|_{L^\infty}\leq \|n_{\mathrm{in}}(\Pi(0;t,\cdot))\|_{W^{1,\infty}}\left\|\exp\left( \int_{0}^{t}\rot_x\left(\dfrac{e}{B} \right)(\Pi(s;t,\cdot))\cdot E(s,\Pi(s;t,\cdot))\marmd s \right) \right\|_{W^{1,\infty}}.
\]
We estimate now $\|n_{\mathrm{in}}(\Pi(0;t,x))\|_{W^{1,\infty}}$. Since
\[
\|n_{\mathrm{in}}(\Pi(0;t,\cdot))\|_{W^{1,\infty}} \leq \|n_{\mathrm{in}}\|_{L^\infty} + \|\partial_x \Pi(0;t,\cdot)\|_{L^\infty}\|\nabla n_{\mathrm{in}}\|_{L^\infty},
\]
therefore it remains to estimate $\sup_{t\in [0,T]}\|\partial_x \Pi(0;t,\cdot)\|_{L^\infty}$. Taking the derivative with respect to $x$ in \eqref{equ:EquChaLim}, we deduce that
\begin{align*}
\|\partial_x \Pi(0;t,\cdot)\|_{L^\infty} &\leq 1 + \int_{0}^{t}(\|E(s)\wedge (e/B) \|_{W^{1,\infty}} + \sigma\|e/\omega_c\|_{W^{2,\infty}} )\|\partial_x \Pi(0;s,\cdot)\|_{L^\infty}\marmd s\\
&\leq 1 + (1+C(\sigma, q, m, C_B)) \int_{0}^{t}(1 + \|E(s)\|_{W^{1,\infty}})\|\partial_x \Pi(0;s,\cdot)\|_{L^\infty}\marmd s,
\end{align*}
for some constant $C(\sigma, q ,m)$ depending on $\sigma, q, m$. Thanks to Grönwall's inequality, we have
\begin{equation}
\label{GradCharPi}
\|\partial_x \Pi(0;t,\cdot)\|_{L^\infty} \leq \exp( (1+C(\sigma, q, m,C_B)) t(1 + \sup_{t\in[0,T]}\|E(t)\|_{W^{1,\infty}}) ),\,\, t\in[0,T],
\end{equation}
which implies for anty $t\in [0,T]$ that
\[
\|n_{\mathrm{in}}(\Pi(0;t,\cdot))\|_{W^{1,\infty}} \leq \|n_{\mathrm{in}}\|_{L^\infty} +  \exp((1+ C(\sigma,q,m,C_B)) T(1 +\sup_{t\in[0,T]} \|E(t)\|_{W^{1,\infty}}))  \|\nabla n_{\mathrm{in}}\|_{L^\infty}.
\]
Next we estimate the following norm
\[
I(t):= \left\|\exp\left( \int_{0}^{t}\rot_x\left(\dfrac{e}{B} \right)(\Pi(s;t,\cdot))\cdot E(s,\Pi(s;t,\cdot))\marmd s \right) \right\|_{W^{1,\infty}}.
\]
A straightforward computations and then using \eqref{GradCharPi} yield
\begin{align*}
I(t)&\leq \left\|\exp\left( \int_{0}^{t}\rot_x\left(\dfrac{e}{B} \right)\cdot E(s)\marmd s \right) \right\|_{L^{\infty}}\left(1+ \int_{0}^{t} \left\| \rot_x\left(\dfrac{e}{B} \right)\cdot E(s)|_{\Pi(s;t,x)}\right\|_{W^{1,\infty}} \marmd s \right)\\
&\leq \exp(C_B t \sup_{t\in[0,T]}\|E(t)\|_{L^\infty})\left( 1 + \int_{0}^{t} C_B \|E(s) \|_{W^{1,\infty}}(1+\|\partial_x \Pi(s;t,\cdot)\|_{L^\infty}) \marmd s\right) \\
&\leq \exp(C_B t \sup_{t\in[0,T]}\|E(t)\|_{L^\infty})\\& \left(1+ \int_{0}^{t}C_B\sup_{[0,T]}\|E(s)\|_{W^{1,\infty}}\exp((1+C(\sigma,q,m,C_B) t(1+ \sup_{[0,T]}\|E(s)\|_{W^{1,\infty}}))\marmd s\right)\\
&\leq \exp(C_B t \sup_{t\in[0,T]}\|E(t)\|_{L^\infty})\\
&(1 +C_B t \sup_{[0,T]}\|E(s)\|_{W^{1,\infty}}\exp((1+C(\sigma,q,m,C_B)) t(1+ \sup_{t\in[0,T]}\|E(t)\|_{W^{1,\infty}})))\\
&\leq \exp(C(\sigma, q,m,B)t(1+\sup_{t\in[0,T]}\|E(t)\|_{W^{1,\infty}})),
\end{align*}
for some constant $C(\sigma, q,m,B)$ depending on $\sigma,q,m, C_B$.
Combining these estimates yield
\[
\|\nabla_x n(t)\|_{L^\infty}\leq \left( \|n_{\mathrm{in}}\|_{L^\infty} +  \exp\left( C_0 T(1 + \|E\|_{W^{1,\infty}}) \right)  \|\nabla n_{\mathrm{in}}\|_{L^\infty} \right)\exp(C_0T(1+\|E\|_{W^{1,\infty}})),
\]
where we used the notation $C_0$ for a universal constant depending on $\sigma,q,m,B$.
%%
\paragraph*{The bound in $L^\infty(0,T;W^{1,1}(\R^2\times\T^1))$ of the solutions}
\begin{equation}
\label{NormL1}
\|n(t)\|_{L^1} = \| n_{\mathrm{in}}\|_{L^1},\,\, t\in[0,T],
\end{equation}
\begin{equation}
\label{GradNormL1}
\|\nabla_x n\|_{L^1} \leq \exp( C_0 t(1 + \sup_{t\in[0,T]}\|E\|_{W^{1,\infty}}))( \|\nabla n_{\mathrm{in}}\|_{L^1} +  t C_0 \sup_{t\in[0,T]}\|E(t)\|_{W^{1,\infty}} \|n_{\mathrm{in}}\|_{L^1} ),
\end{equation}
where $C_0$ is the constant depending on $\sigma, q, m, B$. Now we will prove \eqref{GradNormL1}.
By taking the derivative with respect to $x$ in \eqref{SolChaPi}, we have
\begin{align*}
\nabla_x n(t,x) = \exp\left( \int_{0}^{t}\rot_x\left(\frac{e}{B} \right)\cdot E(s)|_{\Pi(s;t,x)}\marmd s \right) \left[{^t}\partial_x \Pi(0;t,x)\nabla_x n_{\mathrm{in}}(\Pi(0;t,x)) \right.\\
\left. + n_{\mathrm{in}}(\Pi(0;t,x))   \int_{0}^{t}{^t}\partial_x \Pi(s;t,x)\nabla_x \left\{\rot_x\left(\frac{e}{B} \right)\cdot E(s)\right\}|_{\Pi(s;t,x)}\marmd s  \right].
\end{align*}
Then, we integrate with respect to $x$ and change the variable $x$ to $\Pi(t;0,x)$. Notice that the Jacobian formula is given by
\[
\exp\left(- \int_{0}^{t}\rot_x\left(\frac{e}{B} \right)\cdot E(s)|_{\Pi(s;0,x)}\marmd s \right),
\]
therefore we deduce that
\begin{align*}
\intxlt{|\nabla_x n|} &\leq \intxlt{\|\partial_x \Pi(0;t,\cdot)\|_{L^\infty}|\nabla_x n_{\mathrm{in}}|}\\
&+ C_B \intxlt{n_{\mathrm{in}}(x)\int_{0}^{t}\|\partial_x \Pi(s;t,\cdot)\|_{L^\infty} \|E(s,\cdot)\|_{W^{1,\infty}}\marmd s}.
\end{align*}
Thanks to \eqref{GradCharPi} we obtain
\begin{align*}
\intxlt{|\nabla_x n|}\leq \exp( C_0 t(1 + \sup_{t\in[0,T]}\|E\|_{W^{1,\infty}}) )  (\|\nabla n_{\mathrm{in}}\|_{L^1} +  t C_0 \sup_{t\in[0,T]}\|E(t)\|_{W^{1,\infty}} \|n_{\mathrm{in}}\|_{L^1}),
\end{align*}
where we use same the notation $C_0$ for a universal constant depending on $\sigma,q,m, B$.
%%
\paragraph*{Local existence of solutions}$\;$\\
We define
\[
\Sigma : = \left\{ E\in L^\infty(0,T;W^{1,\infty}(\R^2\times\T^1)): \sup_{[0,T]}\|E(t)\|_{L^\infty}\leq M_1,\,\, \sup_{[0,T]}\|\partial_x E(t) \|_{L^\infty} \leq M_2  \right\},
\]
where $M_i, i=1, 2$ are two constants to be fixed later. Given an electric field $E$ in $\Sigma$. We consider the solution by characteristic of the equation \eqref{equ:EquCylinModLimBis} on $\R^2\times\T^1$, corresponding to the electric field $E$ and denote by $n^{E}$ which is given by the formula \eqref{SolChaPi}. We construct  the following map $\calF$ on $\Sigma$, whose fixed point gives the solution of the system \eqref{equ:EquCylinModLimBis}, \eqref{equ:PoiCylinLim} and \eqref{equ:IniCylinLim} at least locally in time such solutions exist
\begin{equation}
\label{MapFixPoint}
E \to \calF(E) = \dfrac{q}{\epsilon_0}\intylt{\nabla_x\Xi(x-y)n^E(t,y)}.
\end{equation}
We will prove that the map $\calF$ is left invariant on the set $\Sigma$ for a convenient choice of the constants $M_1$ and $M_2$, then we want to establish an estimate like
\begin{equation}
\label{MapContract}
\|\calF E(t) - \calF\tilde{E}(t)\|_{L^{\infty}} \leq C_T \int_{0}^{t} \|E(s) -\tilde{E}(s)\|_{L^\infty}\marmd s,\,\, E, \tilde{E}\in\Sigma,\, t\in[0,T],
\end{equation}
for some constant $C_T$, not depending on $E$ and $\tilde{E}$. After that, the existence of the system \eqref{equ:EquCylinModLimBis}, \eqref{equ:PoiCylinLim} and \eqref{equ:IniCylinLim} immediately, based on the construction of an iterative method for $\calF$.
%%
\begin{lemma}$\;$\\
\label{ClosedSet}
There exist positive constants $M_1$,  $M_2$ and $T=T(M_1, M_2)$ such that $\calF(\Sigma) \subset \Sigma$.
\end{lemma}
%%
\begin{proof}$\;$\\
Let $E\in \Sigma$. Thanks to Lemma \ref{FirstEstiEle} and the formulas \eqref{NormInfty} and \eqref{NormL1}, we have
 \begin{align*}
 \|\calF(E)(t,\cdot) \|_{L^\infty} &\leq C( \|n_{\mathrm{in}}\|_{L^\infty}  \exp(C_B T\sup_{t\in [0,T]}\|E(t)\|_{L^\infty})  + \|n_{\mathrm{in}}\|_{L^1} )\\
 &\leq C(\|n_{\mathrm{in}}\|_{L^\infty} + \|n_\mathrm{in}\|_{L^1})( \exp(C_B T\sup_{t\in [0,T]}\|E(t)\|_{L^\infty}) + 1  )\\
 &\leq C(\|n_{\mathrm{in}}\|_{L^\infty} + \|n_{\mathrm{in}}\|_{L^1}) \exp(C_B T\sup_{t\in [0,T]}\|E(t)\|_{L^\infty} +1).
 \end{align*}
Here, we fix $M_1$ as a constant such that ${C e^{2}}{(\|n_{\mathrm{in}}\|_{L^\infty} + \|n_{\mathrm{in}}\|_{L^1})} \leq M_1$ , and we choose $T = \frac{1}{\max(C_B, C_0)( M_1 +M_2)}$, where $C_0 = C(\sigma, q, m, B)$ is a universal constant. Hence, we obtain
\[
\sup_{t\in[0,T]}\|\calF(E)(t,\cdot) \|_{L^\infty} \leq  M_1.  
\]
The bound of $L^\infty$ norm for the density $n(t)$ in \eqref{NormInfty} becomes
\begin{equation}
\label{NormInftyBis}
\| n(t)\|_{L^\infty} \leq e \|n_{\mathrm{in}}\|_{L^\infty}.
\end{equation}
It remains to estimate $\| \partial_x \calF(E)(t,\cdot)\|_{L^\infty}$. Thanks to Lemma \ref{SecondEstEle}, we need to estimate $\ln^+ (\|\nabla_x n(t)\|)$. By the formula \eqref{GradNormInfty} we have
\begin{align*}
\ln^+(\| \nabla_x n(t) \|_{L^\infty}) &\leq \ln^{+}( \|n_{\mathrm{in}}\|_{L^\infty} +  \exp( C_0 T(1 + \sup_{t\in[0,T]}\|E(t)\|_{W^{1,\infty}}))  \|\nabla_x n_{\mathrm{in}}\|_{L^\infty})\\& + C_0 T (1+\sup_{t\in[0,T]}\|E(t)\|_{W^{1,\infty}})\\
&\leq \ln^+ (\|n_{\mathrm{in}}\|_{W^{1,\infty}}(1+ \exp( C_0 T(1 + \sup_{t\in[0,T]}\|E(t)\|_{W^{1,\infty}}) )) )\\
&+ C_0 T (1+\sup_{t\in[0,T]}\|E(t)\|_{W^{1,\infty}})\\
&\leq \ln^+(\|n_{\mathrm{in}}\|_{W^{1,\infty}})+ 1 + 2C_0 T(1+\sup_{t\in[0,T]}\|E(t)\|_{W^{1,\infty}}).
\end{align*}
Thus, together with \eqref{NormInftyBis} we deduce that
\begin{align*}
\| \partial_x \calF(E)(t,\cdot)\|_{L^\infty} &\leq C (1+ e \|n_{\mathrm{in}}\|_{L^\infty} (2+ \ln^+(\|n_{\mathrm{in}}\|_{W^{1,\infty}}) + 2C_0 T(1+\sup_{t\in[0,T]}\|E(t)\|_{W^{1,\infty}} ) + \|n_{\mathrm{in}}\|_{L^1} )\\
&\leq 2C(1+ e\|n_{\mathrm{in}}\|_{L^\infty}(2+ \ln^+(\|n_{\mathrm{in}}\|_{W^{1,\infty}}))+ \|n_{\mathrm{in}}\|_{L^1})(1+ C_0 T \sup_{t\in[0,T]}\|E(t)\|_{W^{1,\infty}}).
\end{align*}
Here, we fix $M_2$ as a constant such that $2C(1+ e\|n_{\mathrm{in}}\|_{L^\infty}(2+ \ln^+(\|n_{\mathrm{in}}\|_{W^{1,\infty}}))+ \|n_{\mathrm{in}}\|_{L^1}) \leq \frac{M_2}{2}$ and we take $T = \frac{1}{\max(C_B, C_0)( M_1 +M_2)}$. Therefore we obtain
\[
\| \partial_x \calF(E)(t,\cdot)\|_{L^\infty} \leq \dfrac{M_2}{2}2 = M_2.
\]
\end{proof}
Now we establish the inequality \eqref{MapContract}. Consider $E$ and $\tilde{E}\in\Sigma$ and denote by $n^{E}$ and $\tilde{n}^{\tilde{E}}$ the solutions by characteristics of \eqref{equ:EquCylinModLimBis}, \eqref{equ:IniCylinLim} with the same initial data $n_\mathrm{in}$ corresponding to the electric fields $E$ and $\tilde{E}$, respectively. It is easily seen from Lemma \ref{FirstEstiEle} that
\[
\| \calF(E)(t) -\calF(\tilde{E})(t)\|_{L^\infty} \leq C (\|n^E(t)-\tilde{n}^{\tilde{E}}(t)\|_{L^\infty} + \|n^E(t) - \tilde{n}^{\tilde{E}}(t)\|_{L^1}).
\]
Notice that the constant $C$ is not depend on $E$ and $\tilde{E}$. 
%%
\begin{lemma}$\;$\\
\label{DiffNormInfty}
We have 
\[
\|n^E(t)-\tilde{n}^{\tilde{E}}(t)\|_{L^\infty} \leq C \int_{0}^{t}\| E(s) - \tilde{E}(s)\|_{L^\infty}\marmd s ,
\]
for some positive constant $C$, not depending on $E,\tilde{E}$.
\end{lemma}
%%
\begin{proof}$\;$\\
Thanks to \eqref{SolChaPi}, we deduce that
\begin{align*}
&|n^{E}(t,x) - \tilde{n}^{\tilde{E}}(t,x)| \\
&\leq |n_{\mathrm{in}}(\Pi^{E}(0;t,x))-n_{\mathrm{in}}(\tilde{\Pi}^{\tilde{E}}(0;t,x)) | \exp\left( \int_{0}^{t}\rot_x\left(\dfrac{e}{B} \right)\cdot E(s)|_{\Pi^E(s;t,x)}\marmd s \right)\\
&+ n_{\mathrm{in}}(\tilde{\Pi}^{\tilde{E}}(0;t,x))\left[\exp\left( \int_{0}^{t}\rot_x\left(\dfrac{e}{B} \right)\cdot E(s)|_{\Pi^E(s;t,x)}\marmd s \right) - \exp\left( \int_{0}^{t}\rot_x\left(\dfrac{e}{B} \right)\cdot \tilde{E}(s)|_{\tilde{\Pi}^{\tilde{E}}(s;t,x)}\marmd s \right)\right]
\\& =: I_1 + I_2,
\end{align*}
where $\Pi^E$ and $\tilde{\Pi}^{\tilde{E}}$ denote the characteristic of \eqref{equ:EquChaLim} corresponding to the vector fields $E$ and $\tilde{E}$. We estimate now the integral $I_1$. Since
\[
|n_{\mathrm{in}}(\Pi^{E}(0;t,x))- n_{\mathrm{in}}(\tilde{\Pi}^{\tilde{E}}(0;t,x)) | \leq |\Pi^{E}(0;t,x) - \tilde{\Pi}^{\tilde{E}}(0;t,x) |\|\nabla_x n_{\mathrm{in}}\|_{L^\infty},
\]
so we need to estimate $\sup_{t,s\in[0,T]} \|\Pi^{E}(t;s,\cdot) - \tilde{\Pi}^{\tilde{E}}(t;s,\cdot) \|_{L^\infty} $. We claim that
\begin{align}
\label{DiffCharc}
\sup_{t,s\in[0,T]} \|\Pi^{E}(t;s,\cdot) - \tilde{\Pi}^{\tilde{E}}(t;s,\cdot) \|_{L^\infty}  \leq C_0 e^{C_0 T(1+M_1+M_2)} \int_{0}^{t} \|E(s,\cdot)-\tilde{E}(s,\cdot) \|_{L^\infty}\marmd s,
\end{align}
for some constant $C_0$ depending on $\sigma,q,m, B$. Indeed, from the equations in \eqref{equ:EquChaLim} we imply that
\begin{align*}
\dfrac{\marmd }{\marmd t}(\Pi^{E} - \tilde{\Pi}^{\tilde{E}})(t;s,x) &= \left(E(t)\wedge\dfrac{e}{B}|_{\Pi(t;s,x)} -\tilde{E}(t)\wedge \dfrac{e}{B}|_{{\Pi}}(t;s,x)\right)\\
& + \left( \tilde{E}(t)\wedge \dfrac{e}{B}|_{{\Pi}}(t;s,x) - \tilde{E}(t)\wedge \dfrac{e}{B}|_{\tilde{\Pi}}(t;s,x)\right)\\
&+ \sigma \rot_x \left(\dfrac{e}{\omega_c}\right)(\Pi(t;s,x)) - \sigma \rot_x \left(\dfrac{e}{\omega_c}\right)(\tilde{\Pi}(t;s,x)),\\
(\Pi^{E} - \tilde{\Pi}^{\tilde{E}})(s;s,x) &=0.
\end{align*}
Integrating between $s$ and $t$ we find
\begin{align*}
|(\Pi^{E} - \tilde{\Pi}^{\tilde{E}})(t;s,x)| &\leq C_0\int_{0}^{t} \|E(s)- \tilde{E}(s)\|_{L^\infty}\marmd s \\ & + C_0\int_{0}^{t}\|\tilde{E}\|_{W^{1,\infty}}|(\Pi^{E} - \tilde{\Pi}^{\tilde{E}})(\tau;s,x)|\marmd \tau 
+ C_0 \int_{0}^{t}|(\Pi^{E} - \tilde{\Pi}^{\tilde{E}})(\tau;s,x)|\marmd \tau.
\end{align*}
Notice that $\sup_{t\in[0,T]}\|\tilde{E}(t)\|_{W^{1,\infty}}\leq M_1 +M_2$, since $\tilde E \in \Sigma$.
Then, the Gronwall lemma allows us to conclude that \eqref{DiffCharc} holds. Therefore we have
\[
I_1 \leq  C \int_{0}^{t} \|E(s,\cdot)-\tilde{E}(s,\cdot) \|_{L^\infty}\marmd s.
\]
Next, we estimate the integral $I_2$. We utilize the inequality $|e^{x}- e^{y}| \leq e^{x+y}|x-y|$, valid for any $x,y\in\R$. Applying the same argument as in the estimate of $I_1$, we obtain 
\[
I_2 \leq C   \int_{0}^{t}\|E(s,\cdot)-\tilde{E}(s,\cdot) \|_{L^\infty}\marmd s.
\]
Notice that, for the sake of simplicity, we use the same notation $C$ in the inequalities for both $I_1$ and $I_2$ standing for a universal constant depending on $T,M_1,M_2,B,n_\mathrm{in}$. Finally, we combine the estimate for the integrals $I_1$ and $I_2$ to derive the result.
\end{proof}
%%
\begin{lemma}$\;$\\
\label{DiffNormL1}
We have 
\[
\|n^E(t)-\tilde{n}^{\tilde{E}}(t)\|_{L^1} \leq C \int_{0}^{t}\| E(s) - \tilde{E}(s)\|_{L^\infty}\marmd s ,
\]
for some positive constant $C$, not depending on $E,\tilde{E}$.
\end{lemma}
%%
\begin{proof}$\;$\\
Since $n^E$ and $\tilde{n}^{\tilde{E}}$ are the solutions of \eqref{equ:EquCylinModLimBis} therefore we deduce that
\begin{align*}
\partial_t (n^E(t) - \tilde{n}^{\tilde{E}}(t))  + \left( (E -\tilde{E}) \wedge \dfrac{e}{B} \right)\cdot\nabla_x n^E  + \left( \tilde{E}\wedge \dfrac{e}{B}\right)\cdot\nabla_x(n^E -\tilde{n}^{\tilde{E}})   \\
+ \sigma \rot_x \left(\dfrac{e}{\omega_c} \right)\cdot \nabla_x (n^E - \tilde{n}^{\tilde{E}}) - \rot_x\left( \dfrac{e}{B} \right)\cdot (E- \tilde{E}) n^E - \rot_x\left( \dfrac{e}{B} \right)\cdot \tilde{E} (n^E -\tilde{n}^{\tilde{E}} )  =0,\\
n^E (0)- \tilde{n}^{\tilde{E}}(0) =0.
\end{align*}
Multiplying this equation by $\mathrm{sign}(n^E -\tilde{n}^{\tilde{E}})$, then integrating with respect to $x$, we obtain
\begin{align*}
\partial_t \intxlt{|n^E(t) -\tilde{n}^{\tilde{E}}(t)|} + \intxlt{\mathrm{sign}(n^E -\tilde{n}^{\tilde{E}})\left( (E -\tilde{E})  \wedge \dfrac{e}{B} \right)\cdot\nabla_x n^E}\\
- \intxlt{\mathrm{sign}(n^E -\tilde{n}^{\tilde{E}})\rot_x\left( \dfrac{e}{B} \right)\cdot (E- \tilde{E}) n^E}
- \intxlt{\rot_x\left( \dfrac{e}{B} \right)\cdot \tilde{E} |n^E -\tilde{n}^{\tilde{E}}|} =0.
\end{align*}
Using the inequality \eqref{GradNormL1} and a straightforward estimations yield
\begin{align*}
\partial_t \intxlt{|n^E(t) -\tilde{n}^{\tilde{E}}(t)|} \leq C \left( \|E(t) -\tilde{E}(t) \|_{L^\infty} +  \intxlt{|n^E -\tilde{n}^{\tilde{E}}|}\right),
\end{align*}
for some positive constant $C$ depending only on $q,m,B, M_1,M_2, n_{\mathrm{in}}$. 
Integrating between $0$ and $t$ and thanks to the Gronwall lemma we obtain the desired result.
\end{proof}
Based on these arguments and Proposition \ref{UniLimMod}, we establish the following result
%%
\begin{pro}$\;$\\
Assume that the initial condition $n_{\mathrm{in}}$ satisfies the hypotheses $H1$ and $H2$. There exists $T>0$ and a local time strong solution $(n,E)$ on $[0,T]$ for the limit model \eqref{equ:EquCylinModLim}, \eqref{equ:PoiCylinLim} and \eqref{equ:IniCylinLim}. The solution is unique and satisfies
\begin{align*}
n\geq 0,\,\, n\in L^{\infty}(0,T;W^{1,\infty}(\R^2\times\T^1))\cap L^\infty(0,T;W^{1,1}(\R^2\times\T^1)),\\
E\in L^\infty(0,T;W^{1,\infty}(\R^2\times\T^1)).
\end{align*} 
\end{pro}
%%
\section*{Appendix}
\begin{lemma}$\;$\\
\label{GaussIntegral}
For any $r\in\R_+$, we have
\[
\int_{0}^{\infty} e^{-(\theta -r \theta^{-1})^2} \marmd \theta = \dfrac{\sqrt{\pi}}{2}.
\]
\end{lemma}
\begin{proof}$\;$\\
Let us denote $I(r) = \int_{0}^{\infty} e^{-(\theta -r \theta^{-1})^2} \marmd \theta$. It is easily seen that 
\begin{align*}
 I(0) =  \int_{0}^{\infty}e^{-\theta^2} =\dfrac{\sqrt{\pi}}{2}.
\end{align*}
For any $r>0$, by taking the derivative with respect to $r$, we obtain that
\begin{align*}
I'(r) = 2\int_{0}^{\infty} \theta^{-1}(\theta - r \theta^{-1})e^{-(\theta -r\theta^{-1})^2}\marmd \theta
\end{align*}
which yields
\begin{equation}
\label{DerivGauss}
I'(r) = 2 I(r) - 2 r \int_{0}^{\infty}\theta^{-2} e^{-(\theta - r\theta^{-1})^2}\marmd \theta.
\end{equation}
Observer that
\[
 2 r \int_{0}^{\infty}\theta^{-2} e^{-(\theta - r\theta^{-1})^2}\marmd \theta =  - 2 \int_{0}^{\infty} e^{-(\theta - r\theta^{-1})^2}\marmd (r\theta^{-1}) 
\]
and by changing the variable $u = r\theta^{-1}$ one gets
\[
2 r \int_{0}^{\infty}\theta^{-2} e^{-(\theta - r\theta^{-1})^2}\marmd \theta = -2 \int_{\infty}^{0}e^{-(r u^{-1} -u)^2}\marmd u = 2 I(r).
\]
Substituting in \eqref{DerivGauss}, we have
\[
I'(r)= 0, \,\, r>0
\]
which implies $I(r)$ is independant of value of $r$ and thus $I(r) = \frac{\sqrt{\pi}}{2}$, for any $r\in \R_+$.
\end{proof}
%%
\paragraph*{Acknowledgement}$\;$\\
This work has been carried out within the framework of the EUROfusion Consortium, funded by the European Union via the Euratom Research and Training Programme (Grant Agreement No 101052200 — EUROfusion). Views and opinions expressed are however those of the author(s) only and do not necessarily reflect those of the European Union or the European Commission. Neither the European Union nor the European Commission can be held responsible for them.



\begin{thebibliography}{99}
\bibitem{AbdaHajj08} N.B. Abdallah, R. El Hajj,  Diffusion and guiding center approximation for particle transport in
strong magnetic fields, Kinet. Relat. Models, 1 (2008), 331-354.

\bibitem{BarGolToanSen16}C.W. Bardos, F. Golse, Toan T. Nguyen, R. Sentis, The Maxwell–Boltzmann approximation for ion kinetic modeling,  Physica D: Nonlinear Phenomena, 376-377 (2016),  94-107.

\bibitem{Batt} J. Batt, Global symmetric solutions of the initial value problem of stellar dynamics, J. Diff. Equations, 25(1977) 342-364.

\bibitem{BerVas2005} F. Berthelin, A. Vasseur,
From Kinetic Equations to Multidimensional Isentropic Gas Dynamics Before Shocks,
SIAM Journal on Mathematical Analysis. 36(2005) 1807-1835.

\bibitem{BoniCarSoler1997} L. L. Bonilla, J. A. Carrillo and J. Soler, Asymptotic behaviour of the initial boundary
value problem for the three dimensional Vlasov–Poisson–Fokker–Planck system, SIAM J. Appl. Math. 57 (1997) 1343–1372.

\bibitem{Bos2007} M. Bostan, The Vlasov–Maxwell System with Strong Initial Magnetic Field: Guiding-Center Approximation.
Multiscale Modeling \& Simulation,
6 (2007) 1026-1058.

\bibitem{BosGou08} M. Bostan, T. Goudon, High-electric-field limit for the Vlasov-Maxwell-Fokker-Planck system.
Annales de l'I.H.P. Analyse non linéaire, 25 (2008) 1221-1251. 

\bibitem{BogMit61} N.N. Bogoliubov, and Y.A. Mitropolsky, 
Asymptotic Methods in the Theory of Non-linear Oscillations, 
Published by Gordon \& Breach, New York, 1961.

\bibitem{BosTraEquSin} M. Bostan,
Transport equations with disparate advection fields. Application to the gyrkinetic models in plasma physics,
Journal of Differential Equations,
249 (2010), 1620-1663.

\bibitem{BosAsyAna} M. Bostan,
The Vlasov-Poisson system with strong external magnetic field. Finite Larmor radius regime, Asymptot. Anal., 61 (2009), 91-123.

\bibitem{BosCollis2010} M. Bostan, Collisional models for strongly magnetized plasmas. The gyrokinetic Fokker-Planck equation, Libertas Math, 30 (2010), 99-117. 

\bibitem{BosGuiCen3D} M. Bostan,
Gyrokinetic Vlasov Equation in Three Dimensional Setting. Second Order Approximation, Multiscale Modeling \& Simulation, 8 (2010), 1923-1957.

\bibitem{BosFinHauCRAS} M. Bostan, A. Finot, M. Haurray, The effective Vlasov-Poisson system for strongly magnetized plasma,
 C. R. Acad. Sci. Paris, Ser. I 354(2016) 771-777.

\bibitem{BosFin16} M. Bostan, A. Finot, The Effective Vlasov-Poisson System for the Finite Larmor Radius Regime,
SIAM J. Multiscale Model. Simul., 14 (2016), 1238-1275.

\bibitem{BosSIAM09} M. Bostan, Transport of Charged Particles Under Fast Oscillating Magnetic Fields, SIAM Journal on Mathematical Analysis, 44 (2012) 1415-1447.

\bibitem{BosLarmor2016}M. Bostan, High magnetic field equilibria for the Fokker-Planck-Landau equation, Ann. Inst. H. Poincaré Anal. Non Linéaire, Vol. 33 (2016) 899-931.

\bibitem{BosSIAM16} M. Bostan, Multi-scale analysis for linear first order PDEs. The finite Larmor radius regime, SIAM J. Math. Anal. 48 (2016) 2133-2188.

\bibitem{BosSIAM2019} M. Bostan, Asymptotic Behavior for the Vlasov--Poisson Equations with Strong External Magnetic Field. Straight Magnetic Field Lines, SIAM Journal on Mathematical Analysis, 51 (2019).

\bibitem{Bos2020} M. Bostan, Asymptotic behavior for the Vlasov-Poisson equations with strong external curved magnetic field. Part I : well prepared initial conditions, hal.inria.fr/hal-02088870/.

\bibitem{BosQueBolt2014} M. Bostan, C. Caldini-Queiros, Finite Larmor radius approximation for collisional magnetic
confinement. Part I: the linear Boltzmann equation, Quart. Appl. Math., Vol. LXXII, No. 2(2014) 323-345.

\bibitem{BosQueFPL2014} M. Bostan, C. Caldini-Queiros, Finite Larmor radius approximation for collisional magnetic
confinement. Part II: the Fokker-Planck-Landau equation, Quart. Appl. Math., Vol. LXXII, No. 3 (2014) 513-548.

\bibitem{BosGam2012}M. Bostan, I.M. Gamba, Impact of strong magnetic fields on collision mechanism for transport of
charged particles, J. Stat. Phys., 148(2012), 856–895.

\bibitem{BosTuan} M. Bostan, A-T. Vu, Asymptotic behavior of the
two-dimensional Vlasov-Poisson-Fokker-Planck equation
with strong external magnetic field, Preprint.

\bibitem{Bou1993} F. Bouchut, Existence and uniqueness of a global smooth solution for the Vlasov-Poisson-Fokker-Planck system in three dimension, J. Funct. Anal. 111(1993) 239-258.

\bibitem{Bou1995} F. Bouchut, Smoothing effect for the nonliear Vlasov-Poisson-Fokker-Planck system, J. Differential Equations, 122(1995) 225-238.

\bibitem{BouDol1995} F. Bouchut and J. Dolbeaut, On long asymptotics of the Vlasov–Fokker–Planck equa-
tion and of the Vlasov–Poisson–Fokker–Planck system with Coulombic and Newtonian
potentials, Differential Integ. Equations 8 (1995) 487–515.

\bibitem{Bre2000} Y. Brenier, Convergence of the Vlasov-Poisson system to the incompressible Euler equations, Comm. Paryial Differential Equations 25(2000) 737-754.

\bibitem{BreMauPue2003} Y. Brenier, N. Mauser, M. Puel, Incompressible Euler and e-MHD scaling limits of the Vlasov-Maxwell system, Commun. Math. Sci. 1(2003) 437-447.

\bibitem{CarPanZa20} B. Carrillo, X. Pan, Qi S. Zhang, Decay and vanishing of some axially symmetric D-solutions of the Navier-Stokes equations, Journal of Functional Analysis, 279(2020).

\bibitem{CarChoiJung2021}J.-A. Carrillo, Y.-P. Choi, J. Jung, Quantifying the hydrodynamic limit of Vlasov-type equations with alignment and nonlocal forces,  Math. Models and Methods Appl. Sci Vol. 31(2021) 327-408. 

\bibitem{CarSol1995} J.-A. Carrillo, J. Soler, On the initial value problem for the Vlasov-Poisson-Fokker-Planck system with initial data in $L^p$ spaces, Math. Methods Appl. Sci 18(1995) 825-839.

\bibitem{Csi1967} I. Csisz\'ar , Information-type measures of difference of probability distributons and indirect observation, Studia Sci. Math. Hungar, 2(1967) 299-318.

\bibitem{Cha1949} S. Chandrasekhar, Brownian motion, dynamic friction and stellar dynamics, Rev. Mod. Physics 21(1949) 383-388.

\bibitem{ChoiJeong2023} Y.-P. Choi, and I.-J. Jeong, Global-in-time existence of weak solutions for
Vlasov–Manev–Fokker–Planck system, Kinetic and Related Models 16(2023) 41-53.

\bibitem{Deg1986} P. Degond, Global existence of smooth solutions for the Vlasov-Fokker-Planck equation in 1 and 2 space dimension, Annales Scientifiques de l'ENS, S\'erie 4, 19(1986) 519-542.

\bibitem{DegFil16} P. Degond, F. Filbet, On the Asymptotic Limit of the Three Dimensional Vlasov–Poisson System for Large Magnetic Field: Formal Derivation,
 Journal of Statistical Physics 165(2016), 765–784.

\bibitem{BerDel2000} J.-L. Delcroix, A. Bers, Physiquue des plasmas, EDP Sciences 2000.

\bibitem{Evans}  L.-C. Evans, Partial Differential Equations: Second Edition (Graduate Studies in Mathematics) 2nd Edition, American Mathematical Soc., 2010 - 749 pages.

\bibitem{FreSon00} E. F\'enod, E. Sonnend\" ucker, Long time behavior of two-dimensional Vlasov equation with strong external magnetic field, Math. Models Methods Appl. Sci. 10(2000) 539-553.

\bibitem{GolSaintMag1999}F. Golse, L. Saint-Raymond, The Vlasov-Poisson system with strong magnetic field, J. Math. Pures Appl. 78(1999) 791-817.

\bibitem{GolSaintQuas2003} F. Golse, L. Saint-Raymond, The Vlasov-Poisson system with strong magnetic field in quasineutral regime, Math. Models Methods Appl. Sci., 13(2003) 661-714.

\bibitem{GouJabVas2004} T. Goudon, P.-E. Jabin, A. Vasseur, Hydrodynamic limits for the Vlasov-Navier-Stokes equations. Part II: Fine particles regimes, Indiana Univ. Math. J. 53(2004) 1517-1536.

\bibitem{ReiRen1993} G. Rein, A.-D. Rendall, Global existence of classical solutions to the vlasov-poisson system in a three-dimensional, cosmological setting, Archive for Rational Mechanics and Analysis, 126(1994) 183–201.

\bibitem{HerRod2019} M. Herda, L.M. Rodrigues, Anisotropic Boltzmann-Gibbs dynamics of strongly magnetized Vlasov-Fokker-Planck equations, Kinetic and Related Models, 2019. 

\bibitem{Kul1967} S. Kullback, A lower bound for discrimination in information in terms of variation, IEEE TRans. Information Theory 4(1967) 126-127.

\bibitem{LeeGyro1983}W. W. Lee, Gyrokinetic approach in particle simulation, Physics of Fluids, 26(1983), 556–562.

\bibitem{LittHam1981} R. G. Littlejohn, Hamiltonian formulation of guiding center motion, Phys. Fluids, 24(1981), 1730–1749.

\bibitem{Lutz2013} M. Lutz, Etude math\'ematique et numérique d'un mod\`ele gyrocin\' etique incluant des effets \'electromagn\'etique pour la simulation d'un plasma de Tokamak, Th\`ese de Doctorat.

\bibitem{Miot} E. Miot, On the gyrokinetic limit for the two-dimensional Vlasov-Poisson system,
arXiv:1603.04502.

\bibitem{Maheux} P. Maheux, Notes on Heat Kernels on Infinite dimensional
Torus. Lecture note.

\bibitem{Negu} C. Negulescu, Kinetic Modelling of Strongly Magnetized Tokamak Plasmas with Mass Disparate Particles. The Electron Boltzmann Relation, Multiscale Modeling \& Simulation, 16(2018), 1732-1755.


\bibitem{ODwVic1990} B.O'Dwyer, H.D. Victory Jr., On classocal solutions of the Vlasov-Poisson-Fokker-Planck system, Indiana Univ. Math. J. 39(1990) 105-156.

\bibitem{PueSaint2004} M. Puel, L. Saint-Raymond, Quasineutral limit for the relativistic Vlasov-Maxwell system, Asymptot. Anal. 40(2004) 303-352.

\bibitem{ReinWeckler1990} G. Rein, J. Weckler, Generic global classical solutions of the Vlasov-Poisson-Fokker-Planck system in three dimensions, Diff. Equations 99(1992) 59-77.

\bibitem{Saint2002}L. Saint-Raymond, Control of large velocities in the two-dimensional gyro-kinetic approximation, J. Math. Pures Appl. 81(2002) 379-399.

\bibitem{Saint2003} L. Saint-Raymond, Convergence of solutions to the Boltzmann equation in the incompressible Euler lomit, Arch. Ration. Mech. Anal. 166(2003) 47-80.

\bibitem{TianZa} G. Tian, Qi S. Zhang, Isoperimetric inequality under Kähler Ricci flow,  American Journal of Mathematics, 136(2014), 1155-1173. 

\bibitem{Vic1991} H.D. Victory Jr., On the existence of global weak solutions for the Vlasov-Poisson-Fokker-Planck system, J. Math. Anal. Appl. 160(1991) 525-555.

\bibitem{Yau1991} H.T. Yau, relative entropy and hydrodynamics of Ginzburg-Landau models, Lett. Math. Phys. 22(1991) 63-80.

\bibitem{Zhang} Qi S. Zhang, A formula for backward and control problems of the heat equation, arxiv.org/abs/2005.08375.

\end{thebibliography}




\end{document}
