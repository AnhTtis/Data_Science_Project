To test our implementation, we set up several experiments that demonstrate how each component performs and how IRIS-NP scales.  In all experiments shown SNOPT \cite{gill2005snopt} was used to solve the nonlinear counterexample search.

\subsection{Collision Pair Ordering Ablation}
\label{sec:ex/ordering}
To study the importance of the order in which collision pairs are considered, we compare generating regions when the collision pairs are ordered by task-space distance as described in Section \ref{sec:ordering} against generating regions when collision pairs are unordered.
For this comparison, we use a KUKA LBR iiwa with seven degrees of freedom, surrounded by randomly placed pillars as depicted in Figure \ref{fig:orient_con}.

\iffalse
\begin{figure}[t]
\centering
\includegraphics[width=0.35\textwidth]{figures/iiwa_env.png}
\label{fig:iiwa_env}
\caption{The environment used for testing the collision pair ordering. The arm is a seven degree of freedom KUKA LBR iiwa with welded gripper, in a forest of randomly placed pillars.
}
\end{figure}
\fi

Using 100 randomly selected seed points that do not place the robot into collision with either itself or the environment, we generated regions about each of these seed point.
Region generation was done with collision pairs sorted by their distance in task space when at the seed configuration and with collision pairs unsorted.
For the termination settings, we required that the sample point stay inside the region, had an iteration limit of 5 and a minimum growth rate between iterations of 2\%.
The configuration-space margin was set to 0.01 and we used 1 infeasible sample per collision pair.

\begin{table}[t]
\caption{Ordering the collision pairs by distance results in regions that are generated faster, with fewer faces and without significant change in the volume of the inscribed ellipse.
The average number of faces of the polytope and the average run time are shown for regions generated from 100 random seeds.
Because some sampled configurations are in very open spaces and others are in close proximity to many obstacles we compare the largest volume regions.
}
\centering
\begin{tabular}{| c | c | c | c |}
 \hline
 & Mean Polytope Faces & Mean Run Time & Maximum Volume\\ 
 \hline
 Ordered   & 146 $\pm$ 29 & 41.4 $\pm$ 11.7 & 86.02 \\  
 Unordered & 181 $\pm$ 27 & 54.2 $\pm$ 8.4 & 89.35 \\
 \hline
\end{tabular}
\label{tab:ordering}
\end{table}

The results of this ablation are shown in Table \ref{tab:ordering}.
On average the regions generated using ordered collision pairs were generated faster and had less faces than the regions generated with unsorted collision pairs.
In addition, there was not a significant difference in the volume of the maximum inscribed ellipse, suggesting that the volume of the overall regions were comparable.

\subsection{Probabilistic Certification}

\begin{figure}[t]
\centering
\includegraphics[width=0.35\textwidth]{figures/infeasible_samples.pdf}
\caption{As the number of consecutive infeasible counterexample searches required before continuing increases, the percent of samples within the region that are in collision heads to zero. This trend allows the user to trade off the run time of region generation versus probabilistic certification of the region. Note that this specific environment and seed configuration was explicitly designed to make it very difficult for IRIS-NP to generate completely collision-free regions. In most practical environments, we find that a single infeasible sample is sufficient to eliminate all collisions from the region.
}
\label{fig:certificate}
\end{figure}

For this section, we look at the ability to provide a probabilistic certificate of the regions generated with IRIS-NP.
As mentioned previously, because the process of adding hyperplanes to separate obstacles relies on a nonlinear optimization, we cannot guarantee that there are no collisions inside the region, even when the solver reports infeasible.
A possible solution is to increase the number of consecutive infeasible optimizations that are solved from randomly sampled initial guesses before ending the search for hyperplanes.

To test if this reduced the number of colliding configurations inside the region, we set up a simple environment with a 4 link arm in the plane, with circular obstacles around the robot.
The robot, environment, and seed were selected to make it very difficult to generate a region that was completely collision free.
In practice, most of the regions generated for real problems had few if any collisions in the region, even with just a single infeasible sample.
Regions were generated about the same seed configuration while varying the number of consecutive infeasible counterexample searches to perform.
For the termination settings, we required that the sample point stay inside the region, had an iteration limit of 5 and a minimum growth rate between iterations of 2\%.
The configuration-space margin was set to 0.01.
We then randomly sampled configurations within the region (using rejection sampling) to calculate the percentage of configurations within the region that are in collision.
The results are shown in Figure \ref{fig:certificate}.

As we increased the number of infeasible problems the solver must solve before moving on to a different collision pair, the percent of colliding configurations that were in the region dropped.
The the decrease in percent of colliding configurations was not monotonic but that is likely due to the interplay between adding more hyperplanes during the counterexample step and the metric used to add hyperplanes, which is changed on the major iterations of IRIS-NP when we calculate a new maximum inscribed ellipse.
We do not claim the process is monotonic, only that the percent of colliding configurations within the region converges to zero in the limit.

\subsection{Additional Constraints}

\begin{figure}[t]
\centering
\begin{subfigure}[t]{0.23\textwidth}
\centering
\includegraphics[width=\textwidth]{figures/no_orient_constraints.png}
\end{subfigure}
\
\begin{subfigure}[t]{0.23\textwidth}
\centering
\includegraphics[width=\textwidth]{figures/orient_constraints.png}
\end{subfigure}
\caption{A comparison of the convex collision-free regions generated without (left) and with (right) an additional gripper orientation constraint designed to prevent a held mug from spilling its contents. The red pillars are task-space obstacles that both the robot and mug must not collide with.
}
\label{fig:orient_con}
\end{figure}

As mentioned in Section \ref{sec:add_constraints}, using the same machinery that is used to generate convex collision-free regions, IRIS-NP can support additional nonlinear constraints on the configuration of the robot.
One such constraint is an orientation constraint on the end effector that prevents a grasped mug from spilling its contents. We demonstrate the support for additional constraints using the same environment as was described in Section \ref{sec:ex/ordering}.
We compared generating a region that is collision free with generating a region that has the additional constraint that the gripper's orientation must be kept within 0.15 radians of level.
Both regions were generated using a seed point where the arm is reaching around and between pillars, yielding the potential for multiple collisions.
For the termination settings, we do not require the sample point stay inside the region, the iteration limit is 5 and the minimum growth rate between iterations is 2\%. The configuration-space margin was set to 0.01 and we used 3 infeasible sample per collision pair.

A few configurations from each region are shown side by side in Figure \ref{fig:orient_con}.
IRIS-NP is able to quickly optimize a region that obeys the constraint, keeping the gripper close to level and the held mug upright everywhere inside the region.
As expected, the region without the gripper orientation constraint is larger as there are more configurations that are collision-free but violate the orientation constraint.
Unexpectedly, the region with the added constraint is generated faster than the region without it, 110 seconds and 129 seconds respectively.
This is not due to the number of faces added to the polytope, as the region with the constraint has more at 576 faces than the one that does not at 172 faces.
The speedup is likely due to the fact that the added constraint quickly limited the set of configurations that could lie inside the region, requiring fewer iterations to maximize the volume of the region.

\subsection{Scaling to High Dimensions}
\begin{figure}[t]
\centering
\includegraphics[width=0.35\textwidth]{figures/bimanual.png}
\caption{A bimanual environment consisting of two KUKA LBR iiwa and a shelving unit. IRIS-NP can scale well to this 14 dimensional environment, constructing some regions in as little as 1 minute.
}
\label{fig:bimanual}
\end{figure}

Since IRIS-NP relies on a local nonlinear optimization to grow large convex regions, it can scale well to robots with larger numbers of degrees of freedom. To demonstrate this, we generate regions for a bimanual manipulator consisting of two KUKA LBR iiwa for a total of fourteen degrees of freedom. The environment, shown in Figure \ref{fig:bimanual} contains a shelving unit that the arms can reach into. For this environment, the region must not only confirm that neither arm is in collision with any of the environment obstacles, it must also confirm that the two arms do not collide with each other.
IRIS-NP generates large regions in this environment in 20 minutes for large open regions of the configuration space and as fast as 1 minute for the more constrained regions in the shelves.
