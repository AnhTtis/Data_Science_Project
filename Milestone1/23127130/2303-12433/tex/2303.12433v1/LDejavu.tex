\documentclass[12pt,reqno,a4paper]{amsart}
\usepackage{graphicx,caption,amssymb,bbm}
%%\usepackage{mathabx}
%%\usepackage{amscd,textcomp}

%%\usepackage[hypertex]{hyperref}
\usepackage{hyperref}

%% Defining nice curly \ll
\DeclareFontFamily{U}{mathb}{\hyphenchar\font45}
\DeclareFontShape{U}{mathb}{m}{n}{
      <5> <6> <7> <8> <9> <10> gen * mathb
      <10.95> mathb10 <12> <14.4> <17.28> <20.74> <24.88> mathb12
}{}
\DeclareSymbolFont{mathb}{U}{mathb}{m}{n}
\DeclareMathSymbol{\llcurly}{3}{mathb}{"CE}
\DeclareMathSymbol{\ggcurly}{\mathrel}{mathb}{"CF}

\begin{document}

\let\kappa=\varkappa
\let\eps=\varepsilon
\let\phi=\varphi
\let\p\partial
\let\lle=\preccurlyeq
\let\ulle=\curlyeqprec

\def\Z{\mathbb Z}
\def\R{\mathbb R}
\def\C{\mathbb C}
\def\Q{\mathbb Q}
\def\P{\mathbb P}
\def\HH{\mathsf{H}}
\def\XX{\mathcal{X}}

%% small mathbb using bbm package
\def\bbk{\mathbbm{k}}

\def\conj{\overline}
\def\Beta{\mathrm{B}}
\def\const{\mathrm{const}}
\def\ov{\overline}
\def\wt{\widetilde}
\def\wh{\widehat}

\renewcommand{\Im}{\mathop{\mathrm{Im}}\nolimits}
\renewcommand{\Re}{\mathop{\mathrm{Re}}\nolimits}
\newcommand{\codim}{\mathop{\mathrm{codim}}\nolimits}
\newcommand{\Aut}{\mathop{\mathrm{Aut}}\nolimits}
\newcommand{\lk}{\mathop{\mathrm{lk}}\nolimits}
\newcommand{\sign}{\mathop{\mathrm{sign}}\nolimits}
\newcommand{\rk}{\mathop{\mathrm{rk}}\nolimits}

\def\id{\mathrm{id}}
\def\Leg{\mathrm{Leg}}
\def\Jet{{\mathcal J}}
\def\sS{{\mathcal S}}
\def\lcan{\lambda_{\mathrm{can}}}
\def\ocan{\omega_{\mathrm{can}}}
\def\bgamma{\boldsymbol{\gamma}}

\renewcommand{\mod}{\mathrel{\mathrm{mod}}}


\newtheorem{mainthm}{Theorem}
\renewcommand{\themainthm}{{\Alph{mainthm}}}
\newtheorem{thm}{Theorem}[section]
\newtheorem{lem}[thm]{Lemma}
\newtheorem{prop}[thm]{Proposition}
\newtheorem{cor}[thm]{Corollary}


\theoremstyle{definition}
\newtheorem{exm}[thm]{Example}
\newtheorem{rem}[thm]{Remark}
\newtheorem{df}[thm]{Definition}
\newtheorem{que}[thm]{Question}
\newtheorem{conje}[thm]{Conjecture}

\numberwithin{equation}{section}
\newcommand{\aftersubsec}{\hfill\nopagebreak\par\smallskip\noindent}

\title{Legendrian links and d\'ej\`a vu moments}
\author[Nemirovski]{Stefan Nemirovski}
\thanks{This work was partially supported by the SFB TRR 191 {\it Symplectic Structures in Geometry, Algebra and
Dynamics}, funded by the DFG (Projektnummer 281071066~--~TRR 191).}
\address{%
Steklov Mathematical Institute, Gubkina 8, 119991 Moscow, Russia;\hfill\break
\phantom{\& }Fakult\"at f\"ur Mathematik, Ruhr-Universit\"at Bochum, 44780 Bochum, \phantom{\& }Germany}
\email{stefan@mi-ras.ru}

\begin{abstract}
A Legendrian link is called a d\'ej\`a vu link if its components can be connected
by a positive Legendrian isotopy but this isotopy cannot be embedded. This is the
contact geometric analogue of a pair of events in a spacetime such that there are
d\'ej\`a vu moments on every future-directed timelike path between them. 
We construct d\'ej\`a vu links in several geometrically relevant situations
and discuss their basic properties.
\end{abstract}

\maketitle




\section{Introduction and overview}
Let $(Y,\xi)$ be a contact manifold with a co-oriented contact structure.
An isotopy of Legendrian submanifolds $\iota:L\times[0,1]\to Y$
is said to be {\it embedded\/} if the map $\iota$ is an embedding 
and {\it positive\/} if the trajectories of points on $L$ 
are positively transverse to the co-oriented contact distribution, 
see \S\ref{LegIso} for a discussion of different types of 
Legendrian isotopies. Every positive isotopy is embedded locally. 
A basic example of a positive isotopy is given by the action of the Reeb flow 
of a contact form defining~$\xi$. This isotopy is embedded if 
there are no Reeb chords of action $\le 1$ for $L_0=\iota(L\times\{0\})$.

For the purposes of this paper, a {\it Legendrian link\/} is 
an ordered pair $(\Lambda_1,\Lambda_2)$ of disjoint closed 
connected Legendrian submanifolds in~$Y$.

\begin{df}
A Legendrian link $(\Lambda_1,\Lambda_2)$ is called a {\it d\'ej\`a vu link\/} if
it satisfies the following two conditions:
\begin{itemize}
\item[(P)] There is a positive isotopy connecting $\Lambda_1$ to $\Lambda_2$. 
\item[(DjV)] There is {\it no\/} embedded positive isotopy connecting $\Lambda_1$ to $\Lambda_2$.
\end{itemize}
\end{df}

The definition and terminology are motivated by the connection with
Lorentz geometry recalled in \S\ref{spacetimes}. The {\it sky\/}  
(or {\it celestial sphere\/}) of a point in a reasonable spacetime~$\XX$
is a Legendrian sphere in the contact manifold of light rays of~$\XX$.
Future-directed timelike curves in $\XX$ induce positive Legendrian
isotopies of skies. If the skies
of two points in $\XX$ form a d\'ej\`a vu link, then every future-directed
curve connecting these points contains {\it d\'ej\`a vu moments}, 
that is, pairs of distinct points lying on the same light ray, 
see \S\ref{djvLorentz}. 

If the Legendrian isotopy class of the components $\Lambda_1$ and $\Lambda_2$ is fixed, 
there is only one Legendrian isotopy class of {\it non\/} d\'ej\`a vu links satisfying
condition~(P) by Corollary~\ref{NonDjV}. Legendrian links satisfying~(P)
and not smoothly isotopic to links in this `primitive' class are clearly d\'ej\`a vu. 
Proposition~\ref{DjVS1Top} shows that this observation can lead to 
a topological description of d\'ej\`a vu links in certain cases 
but being d\'ej\`a vu is not a topological property in general, 
see Example~\ref{SmVsLeg}. 

The components of a d\'ej\`a vu link may sometimes be connected 
by an embedded but non-positive Legendrian isotopy, see \S\S\ref{yxl} 
and~\ref{DjVS1}. It seems unlikely, however, that this could happen
if the Legendrian isotopy class of the components of the link
is orderable. A partial result in this direction is obtained 
in Proposition~\ref{DejavuOrdNoEmb} based on a partial answer
to Question~\ref{QMonNonneg}. 

D\'ej\`a vu links with components in the most basic orderable Legendrian
isotopy class of the zero section of a one-jet bundle are constructed
in~\S\ref{EmbJet} using generating function methods recalled briefly 
in~\S\ref{QIF}. It is a simple example of a problem in which one must use 
the `non-persistent' finite bars of the barcode of a generating 
function rather than the infinite bars and the associated spectral
invariants.





\section{Legendrian submanifolds}

\subsection{Legendrian isotopies}
\label{LegIso}
\aftersubsec
A parametrised Legendrian isotopy in a contact manifold $(Y,\xi)$
is a smooth map
$$
\iota:L\times [0,1] \longrightarrow Y
$$
such that $\iota|_{L\times\{t\}}:L\times\{t\} \hookrightarrow Y$ is an embedding
and the submanifold $\Lambda_t=\iota(L\times\{t\})$ is Legendrian for all $t\in [0,1]$.
Two parametrised isotopies are equivalent if they differ by 
a diffeomorphism $\phi$ of $L\times [0,1]$ such that $\phi(L\times \{t\})=L\times \{t\}$
for all $t\in [0,1]$. A Legendrian isotopy is an equivalence class of parametrised 
Legendrian isotopies. 

Let us assume henceforth that all Legendrian submanifolds are {\it closed}.
By the Legendrian isotopy extension theorem~\cite[Theorem~2.6.2]{Ge},
a~Legendrian isotopy of closed Legendrian submanifolds can be extended
to a compactly supported contact isotopy of the ambient contact
manifold. In particular, Legendrian isotopic links are ambiently
contactomorphic.

A Legendrian isotopy parametrised by $\iota:L\times [0,1] \longrightarrow Y$ 
in a co-oriented contact manifold $(Y,\xi=\ker\alpha)$ is called 
{\it non-negative\/} if $\iota^*\alpha(\frac{\p}{\p t})\ge 0$ on $L\times [0,1]$.
If the inequality is everywhere strict, the isotopy is called {\it positive}.
Both properties do not depend on the parametrisation and on the choice
of a contact form defining the co-oriented contact structure;
they are also invariant under (co-orientation preserving) contactomorphisms.

We write $\Lambda\lle\Lambda'$ if there is a non-negative Legendrian isotopy 
from $\Lambda$ to $\Lambda'$ and $\Lambda\llcurly\Lambda'$ if
there is a positive one. The relation $\lle$ is clearly reflexive and transitive;
if it is also antisymmetric on a Legendrian isotopy class $\mathcal{L}$,
then this class is called {\it orderable}. By~\cite[Proposition 4.7]{ChNe3},
$\mathcal{L}$ is orderable if and only if it does not contain a positive 
Legendrian loop, i.e.\ if and only if $\llcurly$ is {\it not\/} reflexive on~$\mathcal{L}$.

A Legendrian isotopy $\{\Lambda_t\}_{t\in [0,1]}$ is called $\lle$-{\it monotone\/}
if $\Lambda_{t_1}\lle \Lambda_{t_2}$ for all $0\le t_1<t_2\le 1$. 

\begin{lem}
\label{EmbIncrease}
A Legendrian isotopy $\{\Lambda_t\}_{t\in [0,1]}$ such that $\Lambda_0\lle \Lambda_1$ 
and $\Lambda_{t_1}\cap\Lambda_{t_2}=\varnothing$ for all $t_1\ne t_2$
is $\lle$-monotone.
\end{lem}

\begin{proof}
The Legendrian links $(\Lambda_{t_1},\Lambda_{t_2})$ are Legendrian
isotopic and hence contactomorphic for all $0\le t_1<t_2\le 1$. 
The result follows because $\lle$ is preserved by contactomorphisms.
\end{proof}

\begin{rem}
Recall from \cite[Lemma 2.2]{ChNe2} that $\llcurly$ is equivalent to $\lle$ for disjoint Legendrians.
Hence, the isotopy in the lemma is in fact $\llcurly$-monotone.
\end{rem}

A non-negative isotopy is obviously $\lle$-monotone but the converse
is not true in general.

\begin{exm}
\label{EmbNonOrd}
Let $\mathcal{L}$ be a {\it non\/}orderable Legendrian isotopy class
(e.g.\ any class containing a {\it loose\/} Legendrian~\cite{Li}
or any class in a contact manifold admitting a periodic Reeb flow). 
Then there is a positive Legendrian loop based at every $\Lambda\in\mathcal{L}$, 
which implies that $\Lambda\llcurly\Lambda'$
for every $\Lambda'$ sufficiently $C^1$-close to $\Lambda$,
cf.\ \cite[Proof of Corollary 8.1]{ChNe2}. If now $\{\Lambda_t\}_{t\in [0,1]}$
is any Legendrian isotopy in $\mathcal{L}$ such that $\Lambda_t$ are pairwise
disjoint, then it is $\llcurly$-monotone by the proof of Lemma~\ref{EmbIncrease}.
\end{exm}

\begin{que}
\label{QMonNonneg}
Are $\lle$-monotone isotopies non-negative in every {\it orderable\/} Legendrian isotopy class?
\end{que}

For the Legendrian isotopy class of the zero section of the $1$-jet bundle~$\Jet^1(L)$ 
of a closed manifold~$L$, the positive answer to the above question 
follows easily from~\cite[Corollary 5.4]{ChNe1}, 
which is essentially equivalent to the orderability 
of that class~\cite[Corollary 5.5]{ChNe1}. There is 
another case in which we are now going to show that $\lle$-monotone 
Legendrian isotopies are non-negative.

\begin{lem}
\label{SphMonNonneg}
Let $\mathcal{L}$ be an orderable Legendrian isotopy class of spheres.
A $\lle$-monotone Legendrian isotopy $\{\Lambda_t\}_{t\in [0,1]}\subset\mathcal{L}$  
is non-negative.
\end{lem}

\begin{proof}
Suppose that the isotopy is not non-negative at some $\tau\in [0,1]$ and denote $\Lambda=\Lambda_\tau$.
Fix a contactomorphism $\Psi$ from a neighbourhood $U\supset\Lambda$ to a tubular neighbourhood of the zero section 
in $\Jet^1(\Lambda)$ mapping $\Lambda$ onto the zero section~$\mathrm{O}$, see~\cite[Example 2.5.11]{Ge}. 
For $t$ close enough to $\tau$, the Legendrian
$\Lambda_t$ corresponds to the graph of the $1$-jet of a smooth function $f_t:\Lambda\to\R$.
We may now assume that $f_t$ is $C^1$-small and negative somewhere on $\Lambda$
but $\Lambda\lle \Lambda_t= \Psi^{-1}(j^1(f_t))$. 

Let $F$ be a smooth function on $\Lambda$ such that 
\begin{itemize}
\item[a)] $F\ge f_t$;
\item[b)] $\{F<0\}$ is a ball in $\Lambda$;
\item[c)] zero is not a critical value of $F$ (i.e.\ $j^1(F)\cap \mathrm{O}=\varnothing$);
\item[d)] $j^1(F)\subset \Psi(U)$.
\end{itemize}
$F$ may be defined as a regularised maximum (see \cite[Lemma I.5.18]{De}) 
of $f_t$ and a $C^1$-small function $\phi$ such that $\{\phi\le 0\}$ is a closed ball 
contained in~$\{f_t<0\}\ne\varnothing$.

Property (a) implies that $\Lambda\lle \Psi^{-1}(j^1(f_t))\lle \Psi^{-1}(j^1(F))$. 
The space of functions satisfying (b)--(d) is connected for any $\Lambda$. 
If $\Lambda\cong S^n$, the function $-F$ satisfies (b)--(d) too.
Hence, the links $(\mathrm{O},j^1(F))$ and $(\mathrm{O},j^1(-F))$
are Legendrian isotopic in $\Psi(U)$. The latter link is Legendrian
isotopic to $(j^1(F),\mathrm{O})$ via the `shift' contact isotopy
$$
(q,p,u)\mapsto \left(q,p+s\tfrac{\p F}{\p q}, u+sF(q)\right), \quad s\in [0,1].
$$
Thus, $(\Lambda,\Psi^{-1}(j^1(F)))$ is Legendrian isotopic to $(\Psi^{-1}(j^1(F)),\Lambda)$
and therefore we have both $\Lambda\lle \Psi^{-1}(j^1(F))$ and $\Psi^{-1}(j^1(F))\lle\Lambda$, 
which contradicts the assumption that $\mathcal{L}$ is orderable.
\end{proof}

\begin{rem}[Irreversible Legendrian links]
The key point in the above proof will not work 
if $\Lambda$ is not a homology sphere, i.e.\ if there
exists a non-zero homology class $\beta\in\HH_k(\Lambda;\bbk)$
of degree $k\ne 0, \dim\Lambda$. Namely, the links
$(\mathrm{O},j^1(F))$ and $(j^1(F),\mathrm{O})$
will {\it not\/} be Legendrian isotopic in $\Jet^1(\Lambda)$
for any function $F$ satisfying (b) and~(c).
To see this, observe that if $S_F$ is any 
quadratic at infinity generating function for $j^1(F)$
(see \S\ref{QIF}), then $c_\beta(S_F)=c_\beta(F)>0$
and $c_\beta(-S_F)=c_\beta(-F)<0$, where $c_\beta$ is the spectral invariant
defined in Remark~\ref{spectr}. Hence, one can apply
Traynor's argument from~\cite[\S 5]{Tr}
with $c_\beta$ instead of $c_+=c_{[\Lambda]}$.
\end{rem}

\begin{rem}[Lorentzian comparison]
The statement analogous to Lemma~\ref{SphMonNonneg} in Lorentz geometry
(in the sense explained in~\S\ref{spacetimes}) asserts that a causally
monotone curve in a strongly causal spacetime is future-directed,
see e.g.~\cite[Theorem 3.24]{MS}. Note that orderability is formally
analogous to causality, so this is another instance suggesting that
there is no `strong causality' in the Legendrian world, cf.~\cite[Remark~4.2]{ChNe4}.
\end{rem}

A Legendrian isotopy parametrised by $\iota:L\times [0,1] \longrightarrow Y$ 
is called {\it immersed\/} or {\it embedded\/} if the map $\iota$ is an immersion
or an embedding. These notions are obviously independent of the choice of
a parametrisation. The property of a Legendrian isotopy to be {\it immersed\/}
can also be expressed in terms of the pull-back of a contact form.

\begin{lem}
\label{Immersed}
A Legendrian isotopy $\iota:L\times [0,1]\to (Y, \xi=\ker\alpha)$
is \textbf{not\/} immersed at a point $(q,\tau)$ if and only if $(q,\tau)$ is a
critical point of the function $\iota^*\alpha(\frac{\p}{\p t})$ 
restricted to $L\times\{\tau\}$ with critical value zero.
\end{lem}

\begin{proof}
The condition $\iota^*\alpha(\frac{\p}{\p t})=0$ is equivalent to $\iota_*\frac{\p}{\p t}\in\xi_{\iota(q,\tau)}$.
The point $(q,\tau)$ is critical for $\iota^*\alpha(\frac{\p}{\p t})$ on $L\times\{\tau\}$ if and only if
$$
X\left(\iota^*\alpha(\tfrac{\p}{\p t})\right) = 0
$$
at $(q,\tau)$ for all {\it vertical\/} vector fields $X$ on~$L\times [0,1]$.
For a vertical~$X$, the commutator $[X,\frac{\p}{\p t}]$
is also vertical and $\iota^*\alpha(X)=\alpha(\iota_*X)=0$. Hence,  
$$
\begin{array}{rcl}
%%%% 0& =& \\[2pt] 
X\left(\iota^*\alpha(\tfrac{\p}{\p t})\right)&=& 
d(\iota^*\alpha)\left(X,\tfrac{\p}{\p t}\right) + \tfrac{\p}{\p t}(\iota^*\alpha(X)) + \iota^*\alpha\left([X,\tfrac{\p}{\p t}]\right)\\[2pt]
&=& d(\iota^*\alpha)\left(X,\tfrac{\p}{\p t}\right)\\[2pt]
&=& d\alpha\left(\iota_*X,\iota_*\tfrac{\p}{\p t}\right).
\end{array}
$$
Thus, our assumptions are equivalent to $\iota_*\frac{\p}{\p t}$ 
being skew-orthogonal to the Lagrangian subspace $\iota_*(T_q L)$ in the symplectic
vector space $(\xi_{\iota(q,\tau)}, d\alpha)$. This means precisely 
that $\iota_*\frac{\p}{\p t}\in \iota_*(T_q L)$  and the rank 
of $\iota$ is not maximal at~$(q,\tau)$.
\end{proof}

As an application, we show that being immersed characterises positive 
Legendrian isotopies among non-negative ones.

\begin{cor}
\label{NonnegEmbPos}
A non-negative Legendrian isotopy is positive if and only if it is immersed.
\end{cor}

\begin{proof}
A positive Legendrian isotopy is obviously immersed. To prove the `if' part,
let $\iota:L\times [0,1]\to (Y, \xi=\ker\alpha)$ be a parametrisation 
of an immersed non-negative isotopy. Suppose that the isotopy isn't positive. 
Then $\iota^*\alpha(\frac{\p}{\p t})=0$ 
at some point $(q,\tau)\in L\times[0,1]$.
Since $\iota^*\alpha(\frac{\p}{\p t})\ge 0$, it follows that this function
attains its minimum equal to zero at $(q,\tau)$. Hence, the isotopy 
is not immersed at that point by Lemma~\ref{Immersed}.
\end{proof}

\begin{rem} 
A related argument may be found in~\cite[Lemma~4.12(i)]{GKS}.
\end{rem}

The existence of an embedded isotopy between two Legendrians
may be inferred from a seemingly weaker assumption.

\begin{lem}
\label{ExistEmb}
If $\Lambda_0\cap\Lambda_t=\varnothing$ for all $t>0$
in a Legendrian isotopy $\{\Lambda_t\}_{t\in [0,1]}$,
then $\Lambda_0$ and $\Lambda_1$ are connected
by an embedded Legendrian isotopy. 
If $\{\Lambda_t\}_{t\in [0,1]}$ is also positive at $t=0$,
then $\Lambda_0$ and $\Lambda_1$ are connected
by a positive embedded Legendrian isotopy.
\end{lem}

\begin{proof}
If we identify a neighbourhood
of $\Lambda_0$ with a neighbourhood of the zero section 
in $\Jet^1(\Lambda_0)$, then $\Lambda_\tau$ for small enough $\tau>0$ 
corresponds to $j^1(f_\tau)$ for a smooth function $f_\tau$ on $\Lambda_0$
such that zero is not its critical value. Hence, an embedded
isotopy connecting $\Lambda_0$ to $\Lambda_\tau$ can be defined 
by $\{j^1(sf_\tau)\}_{s\in [0,1]}$. If the given isotopy 
is positive at $t=0$, then it is positive and embedded on $[0,\tau]$
for small enough~$\tau>0$. 
(This corresponds to $f_{\tau'}>f_{\tau''}>0$ on $\Lambda_0$ for $\tau\ge\tau'>\tau''>0$.)
It remains to observe that $\Lambda_\tau$ is Legendrian isotopic
to $\Lambda_1$ in the complement of $\Lambda_0$ and therefore
there exists a contact isotopy $\{\phi_t\}_{t\in [\tau,1]}$ 
such that $\Lambda_0\cap\mathop{\mathrm{supp}}\phi_t =\varnothing$
and $\phi_t(\Lambda_\tau)=\Lambda_t$. Applying $\phi_1$
to the Legendrian isotopies from $\Lambda_0$ to $\Lambda_\tau$ 
constructed above completes the proof.
\end{proof}

\begin{rem}
\label{EmbHomotopic}
The embedded isotopy obtained in the lemma is homotopic
to the original Legendrian isotopy through Legendrian
isotopies. If the original isotopy is positive,
this homotopy is through positive isotopies.
\end{rem}

Since the space of positive functions on a manifold is
connected, the proof of Lemma~\ref{ExistEmb} implies 
the following result.

\begin{cor}
\label{NonDjV}
All \textbf{non} d\'ej\`a vu Legendrian links $(\Lambda,\Lambda')$
with $\Lambda\llcurly\Lambda'$ and a given $\Lambda$
are Legendrian isotopic by an isotopy fixing $\Lambda$.
\end{cor}

It follows that the Legendrian isotopy class of a non d\'ej\`a vu link
with $\Lambda\llcurly\Lambda'$ is completely determined by the Legendrian
isotopy class of its components. One way of representing this class
of links is to take the link formed by a Legendrian and its sufficiently small
shift along the Reeb flow of a contact form. 

\begin{prop}
\label{DejavuOrdNoEmb}
Suppose that $\mathcal{L}$ is either the Legendrian isotopy class 
of the zero section in $\Jet^1(L)$ or an orderable Legendrian
isotopy class of spheres. Let $(\Lambda,\Lambda')$ be 
a Legendrian link with components in~$\mathcal{L}$.
\begin{itemize}
\item[(i)] 
If $\Lambda\llcurly\Lambda'$, then every embedded isotopy
from $\Lambda$ to $\Lambda'$ is positive.
\item[(ii)]
If $(\Lambda,\Lambda')$ is a d\'ej\`a vu Legendrian link,
then for every Legendrian isotopy $\{\Lambda_t\}_{t\in [0,1]}$ with
$\Lambda_0=\Lambda$ and $\Lambda_1=\Lambda'$ there exists
a $t_0>0$ such that $\Lambda\cap\Lambda_{t_0}\neq\varnothing$
and a $t_1<1$ such that $\Lambda'\cap\Lambda_{t_1}\neq\varnothing$.
\end{itemize}
\end{prop}

\begin{proof}
(i) An embedded Legendrian isotopy from $\Lambda$ to $\Lambda'$ is $\lle$-monotone by Lemma~\ref{EmbIncrease}.
Any $\lle$-monotone isotopy in~$\mathcal{L}$ is non-negative 
by Lemma~\ref{SphMonNonneg} and the discussion preceding it. 
Finally, every embedded non-negative isotopy is positive by Corollary~\ref{NonnegEmbPos}.

\smallskip
\noindent
(ii) Assume that there is a Legendrian isotopy connecting $\Lambda$ to $\Lambda'$
which is disjoint from $\Lambda$ for all $t>0$ or 
from $\Lambda'$ for all $t<1$. Then by Lemma~\ref{ExistEmb} 
there is an embedded isotopy connecting $\Lambda$ to $\Lambda'$.
Applying (i) we obtain a contradiction with the definition of a d\'ej\`a vu link.
\end{proof}

\begin{rem}
Proposition~\ref{DejavuOrdNoEmb} holds true for every Legendrian isotopy class
in which $\lle$-monotone isotopies are non-negative. So it will hold for every 
orderable Legendrian isotopy class if Question~\ref{QMonNonneg} can be
answered in the positive. 
\end{rem}

\begin{rem} 
\label{UODejavuOrdNoEmb}
A version of Proposition~\ref{DejavuOrdNoEmb} 
can be formulated for {\it universally orderable\/}
Legendrian isotopy classes, see~\cite[\S 4.3]{ChNe3}. Namely,
one has to assume that the Legendrian isotopies in (i) and~(ii)
are {\it a~priori\/} homotopic through Legendrian isotopies 
to positive isotopies. A careful inspection of the proofs in this section  
shows that the modified statement is true for a universally orderable 
class of Legendrian spheres. On the other hand, the result can be false 
without this additional assumption, see~\S\ref{yxl} and~\S\ref{DjVS1}.
\end{rem}

\subsection{Quadratic at infinity functions and Legendrians}
\label{QIF}
\aftersubsec
Let $L$ be a closed connected manifold. A smooth function 
$$
S=S(q,\xi):L\times\R^N\to\R
$$ 
is said to be {\it quadratic at infinity\/} if 
$$
S(q,\xi)=\sigma(q,\xi) + Q(\xi),
$$
where $\sigma$ has compact support in $L\times \R^N$ 
and $Q$ is a non-degenerate quadratic 
form on~$\R^N_\xi$. Denote by
$$
S^c:=\{(q,\xi)\in L\times\R^N\mid S(q,\xi)< c\}
$$
the sublevel sets of $S$ and let $S^{-\infty}$ be the set $S^c$ 
for a sufficiently negative~$c\ll 0$. The following standard lemma 
is an immediate consequence of the isotopy extension theorem.

\begin{lem}
\label{stability}
Let $S_t = \sigma_t + Q$, $t\in [0,1]$ be a smooth family of quadratic at infinity
functions. Suppose that $a\in\R$ is not a critical value of $S_t$ for all~$t$.
Then the inclusions $\imath^a_t:S^a_t\hookrightarrow L\times\R^N$ are
isotopic by a compactly supported isotopy constant on $S_t^{-\infty}$. 
In particular, the relative homology groups $\HH_*(S_t^a,S_t^{-\infty};\bbk)$ 
are isomorphic for all~$t$ and the induced homomorphisms
$$
\left(\imath^a_t\right)_*: \HH_*(S_t^a,S_t^{-\infty};\bbk) \longrightarrow \HH_*(L\times\R^N,S_t^{-\infty};\bbk)
$$
do not depend on~$t$.
\end{lem}

\begin{rem}[Persistence and barcodes]
\label{pers}
If $S$ is a Morse function (i.e.\ its critical points are non-degenerate), 
then the relative homology groups $\HH_*(S^c,S^{-\infty};\bbk)$, $c\in\R$, 
together with the homomorphisms induced by the inclusions 
$\imath^{c,c'}:S^c\hookrightarrow S^{c'}$ for $c\le c'$
form a {\it persistence module\/} in the sense of \cite[Definition~1.1.1]{PRSZ}.
Lemma~\ref{stability} implies that if two quadratic at infinity Morse functions 
can be connected by a family such that $a\in\R$ is never a critical value, then 
in the {\it barcodes\/} associated to their persistence modules 
by \cite[Theorem 2.1.1]{PRSZ} the number of bars containing $a$ is the same.
\end{rem}

\begin{rem}[Spectral invariants]
\label{spectr}
Let $\R^N=V_+\times V_-$ be a decomposition into linear subspaces such that $Q$ 
is positive definite on~$V_+$ and negative definite on~$V_-$. The dimension
$\nu=\dim V_-$ is the (negative) index of~$Q$. The map
$$
\HH_*(L;\bbk) \ni \beta \longmapsto \beta\times [V_-]\in\HH_{*+\nu}(L\times\R^N,S^{-\infty};\bbk)
$$
is an isomorphism between the homology of $L$ and the shifted homology
of $L\times\R^N$ relative to~$S^{-\infty}$. Following Viterbo~\cite[\S2]{Vi}, 
one can therefore associate critical values of~$S$ to homology classes on~$L$.
Namely, for $\beta\in \HH_k(L;\bbk)$, define
$$
c_\beta(S):=\inf\bigl\{c\in\R\mid \beta\times [V_-]\in \left(\imath^c\right)_*\HH_{k+\nu}(S^c,S^{-\infty};\bbk)\bigr\},
$$
where $\imath^c:S^c\to L\times\R^N$ is the inclusion. This is a critical value of $S$
by basic Morse theory. Moreover, $c_\beta(S_t)$ is a continuous function of $t$ 
for any smooth family of quadratic at infinity functions, cf.~\cite[Proposition 2.5]{Vi}.
In the barcode associated to $S$, the numbers $c_\beta(S)$ are precisely the endpoints
of the {\it infinite\/} bars.
\end{rem}

Let $\Jet^1(L)$ denote the $1$-jet bundle of~$L$ equipped with 
the standard contact form $du-\lcan$,
where $u$ is the fibre coordinate in $\Jet^0(L)$ and $\lcan=p\,dq$
is the Liouville form on $T^*L$.

\begin{df} A smooth function 
$$
S=S(q,\xi):L\times\R^N\to\R
$$ 
is  a {\it generating function\/} for a Legendrian submanifold $\Lambda\subset\Jet^1(L)$ 
if zero is a regular value of the partial differential $d_\xi S$ and the map
\begin{equation}
\label{gendef}
\{d_\xi S(q,\xi)=0\}\ni (q,\xi) \longmapsto (q,d_q S(q,\xi),S(q,\xi))\in \Jet^1(L)
\end{equation}
is a diffeomorphism onto~$\Lambda$. 
\end{df}

Note that (Morse) critical points of $S$
as a function of both variables $(q,\xi)$ are in one-to-one 
correspondence with (transverse) intersection points of $\Lambda$
with $\{p=0\}\subset\Jet^1(L)$. In particular, the intersection
of $\Lambda$ with the zero section is parametrised by the
critical points of $S$ with critical value zero.

By Chekanov's theorem~\cite{Che}, for any Legendrian isotopy  
$\{\Lambda_t\}_{t\in [0,1]}$ such that $\Lambda_0$ is the zero section $\mathrm{O}\subset\Jet^1(L)$, 
there exists a smooth family of {\it quadratic at infinity\/} 
generating functions 
$$
S_t = \sigma_t + Q :L\times\R^N\to\R
$$ 
for $\Lambda_t$ with $\sigma_0\equiv 0$. 
Furthermore, this family is unique up to stabilisations and fibrewise
diffeomorphisms by the Viterbo--Th\'eret theorem~\cite{Vi, Th1, Th2}.

\section{Examples of d\'ej\`a vu links}

\subsection{D\'ej\`a vu \textit{versus\/} Wiedersehen}
\label{yxl}
\aftersubsec
Let $M$ be a connected manifold, $\dim M\ge 2$, 
and let $ST^*M$ be its co-sphere bundle with the canonical contact structure. 
The fibres of $ST^*M$ are Legendrian spheres and the
links formed by any two of them are Legendrian isotopic.

Suppose that there is a Riemannian metric 
on $M$ making it into a $Y^x_\ell$-manifold, i.e.\ such that all unit speed
geodesics from $x\in M$ return to $x$ at time~$\ell>0$.
(Examples include compact rank one symmetric spaces and
certain exotic spheres, see~\cite[\S 7.C]{Be}.)
Then the co-geodesic flow of this metric defines a positive
Legendrian loop in $ST^*M$ based at the fibre $ST_x^*M$.
Hence, there exists a positive Legendrian isotopy connecting 
(any) two fibres, see~\cite[\S 8]{ChNe2}.

In the special case when $M$ is the standard sphere,
we can do better. If $x\ne y$ are two points in~$M$, 
then they are antipodal for the pull-back of the
standard round metric by a suitable diffeomorphism. 
The co-geodesic flow of this metric defines an 
{\it embedded\/} positive Legendrian isotopy 
from $ST_x^*M$ to $ST_y^*M$. 

\begin{prop}
\label{BSnondjv}
Let $M$ be a connected manifold not homeomorphic to the sphere.
If there is a positive Legendrian isotopy connecting 
two different fibres of $ST^*M$, then they form 
a d\'ej\`a vu Legendrian link.
\end{prop}

\begin{proof}
Let us assume that there is an embedded positive Legendrian
isotopy from $ST_x^*M$ to $ST_y^*M$, $x\ne y\in M$, 
and prove that $M$ must be homeomorphic to the sphere.

As in the proof of Lemma~\ref{ExistEmb}, we may arrange that
the embedded isotopy has a standard form near its endpoints.
In the case of $ST^*M$, this standard form may be taken to be
the action of the co-geodesic flow. It follows that if 
$\pi:ST^*M\to M$ is the bundle projection and $\iota:S^{n-1}\times [0,1]\to ST^*M$
is a parametrisation of the isotopy, then the map $\pi\circ\iota: S^{n-1}\times (0,1)\to M$ 
extends to a smooth map $f:S^n\to M$, $n=\dim M$. Moreover, there is a contractible 
neighbourhood $U\ni x$ such that $f: f^{-1}(U)\to U$ is a diffeomorphism.

The result now follows from a standard topological argument, cf. e.g.~\cite[\S 3.8]{McD}. 
Let $p:\wt{M}\to M$ be the universal covering, then $f$ lifts to a map $\wt{f}:S^n\to\wt{M}$. 
The map $\wt{f}$ is one-to-one over $\wt{f}(f^{-1}(U))$ and therefore induces an isomorphism 
on the (non-trivial) compactly supported top degree cohomology with any constant coefficients.
Hence, $\wt{f}$ is surjective onto $\wt{M}$ and therefore $p$ is one-to-one over $U$, 
which implies that $M=\wt{M}$ is simply connected and compact.
If $\alpha\in\HH^k(M;\bbk)$ is a non-zero cohomology class with $k\ne 0,n$ 
and coefficients in any field~$\bbk$,
then by Poincar\'e duality (on the compact orientable manifold~$M$) 
there is a class $\beta\in \HH^{n-k}(M;\bbk)$ such that $\alpha\cup\beta\ne 0\in \HH^n(M;\bbk)$. 
Then $0\ne f^*(\alpha\cup\beta)=f^*(\alpha)\cup f^*(\beta)$,
which is impossible since the cohomology of $S^n$ is trivial in degrees $k\ne 0,n$.
Thus the cohomology of $M$ with any (e.g.\ integer) coefficients 
vanishes in all degrees $k\ne 0, n$. This shows that $M$ is homeomorphic
to the sphere by the classification of surfaces for $n=2$
and by the theorems of Perelman, Freedman, and Smale for
$n=3, 4$, and~$\ge 5$, respectively.
\end{proof}

The existence of a positive Legendrian loop based at a fibre 
of $ST^*M$ implies that the universal cover $\wt{M}$ is 
compact~\cite{ChNe2} and the integral cohomology ring $\HH^*(\wt{M};\Z)$
is isomorphic to that of a CROSS~\cite{FrLaSch}. If the
loop does not intersect the fibre except at its basepoint, 
then $M$ is either simply connected or homotopy equivalent
to the real projective space $\R\mathrm{P}^n$ by~\cite{Da}.
The above proposition shows that the existence of an
embedded positive Legendrian isotopy between two fibres imposes
an even stronger topological condition on~$M$.
Note also that the proof does not use any tools from
contact topology such as generating functions or
Rabinowitz--Floer homology.

There are obvious embedded Legendrian isotopies
between two fibres of $ST^*M$ obtained by moving
the first fibre along an embedded curve
connecting the corresponding points in~$M$.
(Such isotopies are not positive or non-negative,
see~\cite[\S 6]{ChNe1}.) The Legendrian isotopy class 
of the fibre of $ST^*M$ is always universally orderable~\cite{ChNe3}
and so Remark~\ref{UODejavuOrdNoEmb} shows that 
those obvious isotopies are not homotopic to
a positive one.


\subsection{D\'ej\`a vu Legendrian links in~$T^*L\times S^1$ ({cf.~\cite[\S 8]{CS}})}
\label{DjVS1}
\aftersubsec
Let $Y:= T^*L\times S^1$ be the quotient of $\Jet^1(L)$ by the $\Z$-action 
generated by the shift~$(q,p,u)\mapsto (q,p,u+1)$ and denote by $\pi:\Jet^1(L)\to Y$ 
the projection to the quotient. The canonical contact form $du-\lcan$ is invariant under 
this action and descends to a contact form with periodic Reeb flow on~$Y$. 
In particular, every Legendrian isotopy class in $Y$ is nonorderable. 

Let $f:L\to\R$ be a smooth function such that 
\begin{itemize}
\item[1)] $\max_L |f|<\tfrac{1}{2}$;
\item[2)] zero is not a critical value of~$f$;
\item[3)] $f$ changes sign on $L$.
\end{itemize}
The first two conditions guarantee that $\pi(j^1(f))$ is a Legendrian submanifold of~$Y$ 
disjoint from $\pi(\mathrm{O})$.

\begin{prop}
$\bigl(\pi(\mathrm{O}),\pi(j^1(f))\bigr)$ is a d\'ej\`a vu link in~$Y$.
\end{prop}

\begin{proof}
First, taking the projection to $Y$ of the positive linear isotopy from $\mathrm{O}$ to $j^1(f+1)$ 
shows that $\pi(\mathrm{O})\llcurly\pi(j^1(f))$. 
Secondly, we need to prove that a positive isotopy from $\pi(\mathrm{O})$ to $\pi(j^1(f))$ 
cannot be embedded. Such an isotopy lifts to a positive isotopy 
from $\mathrm{O}$ to $j^1(f+k)$ for some $k\in\Z$. 
By \cite[Corollary~5.4]{ChNe1}, we have $f+k\ge 0$ on $L$ and therefore $k>0$. 
Consider now a family $S_t$, $t\in [0,1]$, of quadratic at infinity generating functions 
for the lifted isotopy and the spectral invariant $c_{[L]}$ associated 
to the fundamental class $[L]\in\HH_{\dim L}(L;\Z/2)$.
Then $c_{[L]}(S_0)=0$ and $c_{[L]}(S_1)=\max_L (f+k) >k$ so that
$c_{[L]}(S_\tau)=k$ for some $\tau\in (0,1)$ by the continuity of the spectral invariant.
Hence, $k$ is a critical value of $S_\tau$ or, in other words, the Legendrian generated 
by $S_\tau$ intersects $j^1(k)$. The projection of this Legendrian to $Y$ intersects 
$\pi(j^1(k))=\pi(\mathrm{O})$ and thus the positive isotopy from $\pi(\mathrm{O})$ 
to $\pi(j^1(f))$ in $Y$ is not embedded as claimed. 
\end{proof}

The linear isotopy from the zero section $\mathrm{O}$ to $j^1(f)$ projects 
to an {\it embedded\/} Legendrian isotopy from $\pi(\mathrm{O})$ 
to $\pi(j^1(f))$ in~$Y$, which shows that Proposition~\ref{DejavuOrdNoEmb} 
does not hold in this case. Note, however, that such an embedded isotopy
cannot be homotopic to a positive isotopy. (Otherwise it would lift to
an isotopy from $\mathrm{O}$ to $j^1(f+k)$ for $k>0$ and the last
part of the proof of the proposition would again lead to a contradiction.)
The Legendrian isotopy class of $\pi(\mathrm{O})$ is {\it universally\/}
orderable, so this agrees with Remark~\ref{UODejavuOrdNoEmb}.





\subsection{D\'ej\`a vu Legendrian links in $\Jet^1(S^1)$}
\label{EmbJetS1}
\aftersubsec
The starting point of the construction is the positive Legendrian
isotopy in $\Jet^1(\R)$ depicted on Fig.~\ref{PosIso}. The figure shows
the wavefronts, i.e.\ the projections of the Legendrians 
to the $(q,u)$-plane. The dashed line represents the zero section $\mathrm{O}$.
The solid wavefront is evolving by a Legendrian isotopy that is
positive because at every moment the points on the wavefront are
moving upwards with respect to the tangent line to the wavefront.
The Legendrian $\Lambda$ in (d) has a single transverse critical 
point (i.e.\ intersection with $\{p=0\}$) in $u\le 0$. 

Outside of a suitably chosen segment in $\R$ the isotopy coincides
with the linear isotopy from $\mathrm{O}$ to the $1$-jet of
a positive constant function. Therefore it can be completed to a
positive isotopy in $\Jet^1(S^1)$. 

\begin{figure}[ht]
\centering
\captionsetup{margin=-1cm}
\includegraphics[scale=0.6]{./PosIso.pdf}
\caption{A negative critical value from a positive isotopy.} 
\label{PosIso}
\end{figure}

The Legendrian link $(\mathrm{O},\Lambda)$ is a d\'ej\`a vu link in~$\Jet^1(S^1)$
for a purely topological reason. Indeed, suppose not. Then $\Lambda$ is 
Legendrian isotopic to $j^1(1)$ in the complement of $\mathrm{O}$ by Corollary~\ref{NonDjV}.
The two knots are however not even homotopic there because the winding number 
of $\Lambda$ around $\mathrm{O}$ defined as the degree of the projection to the $(u,p)$-plane 
minus the origin is~$\pm 1$. 

\begin{figure}[ht]
\centering
\captionsetup{margin=-1cm}
\includegraphics[scale=0.6]{./PosIsoDeg0.pdf}
\caption{D\'ej\`a vu with zero winding.} 
\label{PosIsoDeg0}
\end{figure}

The construction of $\Lambda$ can be modified to make its winding number 
around the zero section vanish. It is enough to create a pair of cusps 
during the isotopy from (c) to~(d), see Fig.~\ref{PosIsoDeg0}. 
The link $(\mathrm{O},\Lambda')$ is d\'ej\`a vu by Proposition~\ref{DjVdimn}
but one can also show that it is not even smoothly isotopic to $(\mathrm{O}, j^1(1))$. 

These two examples illustrate the following general fact based on
the results of Ding and Geiges~\cite{DG} and similar to~\cite[Theorem B]{ChNe1}.

\begin{prop}
\label{DjVS1Top}
A Legendrian link $(\Lambda_1,\Lambda_2)$ in $\Jet^1(S^1)$ such that its components 
are Legendrian isotopic to the zero section and $\Lambda_1\llcurly \Lambda_2$ 
is a d\'ej\`a vu link if and only if it is not smoothly isotopic to~$(\mathrm{O}, j^1(1))$.
\end{prop}

\begin{proof}
The `if' part follows from the definitions because as observed above 
a non d\'ej\`a vu link is even Legendrian isotopic to $(\mathrm{O}, j^1(1))$ 
by Corollary~\ref{NonDjV}.

A Legendrian link in $\Jet^1(S^1)$ such that its components are Legendrian isotopic
to the zero section is smoothly isotopic to $(\mathrm{O}, j^1(1))$
if and only if it is Legendrian isotopic to either $(\mathrm{O}, j^1(1))$
or $(j^1(1),\mathrm{O})$ by the main result of~\cite{DG}.

However, $(\Lambda_1,\Lambda_2)$ cannot be Legendrian isotopic to $(j^1(1),\mathrm{O})$
because $j^1(1)\ggcurly\mathrm{O}$ and the class of the zero section is orderable \cite{ChNe1,CFP}.
So if $(\Lambda_1,\Lambda_2)$ is smoothly isotopic to $(\mathrm{O}, j^1(1))$,
then it is Legendrian isotopic to it and is not d\'ej\`a vu,
which proves the `only if' part. 
\end{proof} 

This topological characterisation of condition (DjV) in the definition of 
d\'ej\`a vu links is specific to dimension~$3$, see Example~\ref{SmVsLeg}.




%% \begin{prop}
%% \label{DjVdim1}
%% $(\mathrm{O},\Lambda')$ is a d\'ej\`a vu link in~$\Jet^1(S^1)$.
%% \end{prop}



\subsection{D\'ej\`a vu Legendrian links in $\Jet^1(L)$}
\label{EmbJet}
\aftersubsec
In order to construct d\'ej\`a vu links in $\Jet^1(L)$
for an arbitrary closed manifold $L$ of dimension $n$, 
we `thicken' a given Legendrian in $\Jet^1(\R)$ 
to a similar Legendrian in $\Jet^1(\R^n)$ using 
the Legendrian suspension construction~\cite[\S 3.3]{ChNe3}.
We will only consider Legendrians in $\Jet^1(\R^n)$
that are equal to 1-jets of functions outside of
a compact set and Legendrian isotopies within
this class.
  
Let $\Lambda\subset\Jet^1(\R)$ be a Legendrian properly
isotopic to the zero section and equal to the 1-jet of
a bounded positive function outside of a compact subset.
Choose a positive Legendrian isotopy $\{\Lambda_t\}_{t\in [0,1]}$
in $\Jet^1(\R)$ such that $\Lambda_0=\Lambda$
and $\Lambda_1=j^1(C)$ for a positive constant~$C$.
(Such an isotopy can be constructed by taking any isotopy
from $\Lambda$ to $\mathrm{O}$ and composing it with 
appropriate positive shifts in the $u$-direction, see e.g.~\cite[\S 4]{CS}.)
Let $\chi:[0,+\infty)\to [0,1]$ be a smooth function such that
$$
\chi(t)=\left\{
\begin{array}{rl}
t& \text{ for } t<\eps\\
1&  \text{ for }t\ge 1-\eps
\end{array}
\right.
\quad\text{ and }\quad \chi'(t)>0 \text{ for } t< 1-\eps
$$
and consider the Legendrian family $\{\Lambda_{\chi(\|x\|^2)}\}_{x\in\R^{n-1}}$
with base $\R^{n-1}$, see~\cite[\S 3.1]{ChNe3}.
(Here $\|x\|$ denotes the Euclidean norm of $x\in\R^{n-1}$.) 
This family is transverse on $0<\|x\|^2<1-\eps$ and constant on $\|x\|^2\ge 1-\eps$.
Its Legendrian suspension is a Legendrian submanifold 
$$
\wt{\Lambda}\subset \Jet^1(\R^n)=\Jet^1(\R)\times T^*\R^{n-1}
$$
with the following properties:
\begin{enumerate}
\item $\wt{\Lambda}$ is the 1-jet of a positive function outside of a compact set~$K$.
\item Critical points of $\wt{\Lambda}$ in $K$ are of the form $((q,0),0,u)\in\Jet^1(\R^n)$,
where $(q,0,u)\in\Jet^1(\R)$ is a critical point of $\Lambda$.
\item If $\mathrm{O}\llcurly\Lambda$, then $\mathrm{O}\llcurly\wt{\Lambda}$ .
\end{enumerate}
By property~(1), $\wt{\Lambda}$ can be completed to a Legendrian in $\Jet^1(L)$ 
for any $n$-dimensional closed manifold $L$ so that no additional critical 
points in $\{u\le 0\}$ are created and property~(3) is preserved.

\begin{prop}
\label{DjVdimn}
Let $\Lambda$ be a Legendrian in $\Jet^1(\R)$ such that $\mathrm{O}\llcurly\Lambda$
and $\Lambda$ has $\kappa\ge 1$ transverse critical points in $\{u\le 0\}$ with the same
negative critical value. {\rm (\/}For instance, one may take the Legendrian $\Lambda'$
in Fig.~{\rm \ref{PosIsoDeg0}} and $\kappa=2$.{\rm )\/} 
Then $(\mathrm{O},\wt{\Lambda})$ is a d\'ej\`a vu link in~$\Jet^1(L)$.
\end{prop}

\begin{proof}
Any quadratic at infinity generating function $S$ for $\wt{\Lambda}$
will have $\kappa$ Morse critical points in $\{S\le 0\}$
with the same negative critical value. 
Therefore, the relative homology 
$\HH_*(S^0,S^{-\infty};\Z/2)$ will have $\kappa$ 
independent generators (in the degrees equal
to the indices of those critical points).

On the other hand, let $\{S_t\}_{t\in [0,1]}$ be a family 
of quadratic at infinity generating functions for 
a positive isotopy from $\mathrm{O}$ to $\wt{\Lambda}$. 
For small~$t$, the Legendrians in the isotopy
are 1-jets of positive functions and hence
all critical values of $S_t$ are positive 
and $\HH_*(S_t^0,S_t^{-\infty};\Z/2)=0$. 
So by Lemma~\ref{stability}, zero must be a critical
value of $S_\tau$ for some $\tau>0$ and the corresponding
Legendrian intersects $\mathrm{O}$, which shows that
the isotopy is not embedded.
\end{proof}

The argument in the proof uses the positivity of the
embedded isotopy only for small $t>0$. This agrees
with Lemma~\ref{ExistEmb}.

Spectral invariants are increasing along positive isotopies 
(see \cite[Proposition~2]{CFP} or \cite[Lemma~5.2]{ChNe1}) and hence cannot be used 
to detect the intersection of such an isotopy with $\mathrm{O}$. 
In other words, the image of the homomorphism 
$\HH_*(S^0,S^{-\infty};\bbk)\to \HH_*(L\times\R^N,S^{-\infty};\bbk)$
is trivial and therefore to invoke Lemma~\ref{stability}
we need to know that the kernel of this homomorphism is non-trivial.
This general principle may be stated in terms of 
the {\it finite bars\/} in the barcode of a generating function.

\begin{prop}
Let $\Lambda\subset\Jet^1(L)$ be a Legendrian submanifold
such that $\mathrm{O}\llcurly\Lambda$ and $\mathrm{O}\cap\Lambda=\varnothing$.
If for some {\rm (\/}and then any\/{\rm )} quadratic at infinity
generating Morse function of $\Lambda$ there is a finite bar
in its barcode containing $0\in\R$, then $(\mathrm{O},\Lambda)$
is a d\'ej\`a vu link.
\end{prop}




\begin{exm}[Smooth \textit{vs}\/ Legendrian links]
\label{SmVsLeg}
For $n\ge 2$ and $L=S^n$, a d\'ej\`a vu link $(\mathrm{O},\Lambda)$
{\it can\/} be smoothly isotopic to $(\mathrm{O},j^1(1))$ 
by an isotopy fixing $\mathrm{O}$.
The complement to the zero section in $\Jet^1(S^n)$ 
is diffeomorphic to $S^n\times (\R^{n+1}-\{0\})$
by the hodograph contactomorphism discussed in 
\S\ref{EmbST}. Taking $n\ge 2$, $k=0$ and $m=2n+1>2n-k$
in part~(b) of Haefliger's Th\'eor\`eme d'existence~\cite[p.~47]{H}, 
we see that embedded $n$-spheres in $S^n\times (\R^{n+1}-\{0\})$
are isotopic if and only if they are homotopic,
which can be inferred from the degrees of their projections to
$S^n$ and $\R^{n+1}-\{0\}$. For instance, $j^1(1)$ is
smoothly isotopic to the Legendrian $\wt{\Lambda'}\subset\Jet^1(S^n)$
obtained from the non-winding Legendrian $\Lambda'\subset\Jet^1(\R)$
shown in Fig.~\ref{PosIsoDeg0}. So the links $(\mathrm{O}, \wt{\Lambda'})$
and $(\mathrm{O},j^1(1))$ are smoothly isotopic in $\Jet^1(S^n)$ by an isotopy fixing~$\mathrm{O}$.
This example is similar to the example of smoothly 
unlinked but Legendrian linked $2$-spheres in~\cite[\S 6]{NT}.
\end{exm}




\subsection{D\'ej\`a vu Legendrian links in $ST^*\R^n$}
\label{EmbST}
\aftersubsec
The 1-jet bundle of the $(n-1)$-sphere is contactomorphic
to the co-sphere bundle of $\R^n$. Explicitly, 
let $\langle\cdot,\cdot\rangle$ denote the standard scalar product on $\R^n$
and let $S^{n-1}\subset\R^n$ be the unit sphere.
The map
$$
\R^n\times S^{n-1}\ni (x,q)\longmapsto\langle q,\cdot \rangle \in ST_x^*\R^n
$$
is a trivialisation of $ST^*\R^n$.
The {\it hodograph contactomorphism\/} is defined by the formula
$$
ST^*\R^n\ni(x,q) \longmapsto
(q,\langle x,\cdot \rangle|_{T_qS^{n-1}}, \langle x,q\rangle)\in\Jet^1(S^{n-1})
$$
The fibre of $ST^*\R^n$ over the origin is mapped to the zero section.

\begin{cor}
Legendrian d\'ej\`a vu links exist in the Legendrian isotopy class
of the fibre of $ST^*\R^n$.
\end{cor}

\begin{figure}[ht]
\centering
\captionsetup{margin=-1cm}
\includegraphics[scale=0.6]{./PosIsoST.pdf}
\caption{D\'ej\`a vu Legendrian link in $ST^*\R^2$.} 
\label{PosIsoST}
\end{figure}

The hodograph pre-image in $ST^*\R^2$ of the Legendrian isotopy in Fig.~\ref{PosIso} 
is shown in Fig.~\ref{PosIsoST}. Here again we are drawing wavefronts,
which in this case are co-oriented projections of Legendrians to~$\R^2$.
The black dot represents the fibre over the origin. The critical point 
with negative $u$ in $\Jet^1(S^1)$ on Fig.~\ref{PosIso}(d) corresponds 
to the white dot on the wavefront of $\Lambda$ in~$\R^2$.
At this point, the co-orientation normal $q\in S^1$ 
is exactly opposite to the vector $x\in\R^2$ from the origin.
In this representation it is even more clear that the isotopy 
to $\Lambda$ is positive because the wavefronts in $\R^2$ 
are moving in the direction of their co-orientation, see~\cite[Example 2.2]{ChNe1}. 

\begin{figure}[ht]
\centering
\captionsetup{margin=-1cm}
\includegraphics[scale=0.6]{./PosIsoSTDeg0.pdf}
\caption{D\'ej\`a vu Legendrian link without winding in $ST^*\R^{2n+1}$.} 
\label{PosIsoSTDeg0}
\end{figure}

Legendrians with the same properties in $ST^*\R^n$ for $n\ge 3$
can be obtained by applying the `finger move' on Fig.~\ref{PosIsoST} 
to the Legendrian whose wavefront in $\R^n$ is the outwardly co-oriented 
$(n-1)$-sphere around the origin. ({\it Warning\/}: These Legendrians 
will {\it not\/} be the hodograph images of the thickened Legendrians 
in $\Jet^1(S^{n-1})$ constructed by the suspension trick in \S\ref{EmbJet}.) 
Creating an additional $(n-2)$-sphere of cusps as on Fig.~\ref{PosIsoSTDeg0}
gives a d\'ej\`a vu link which for {\it odd\/} $n\ge 3$ is 
{\it smoothly\/} isotopic to the {\it non\/} d\'ej\`a vu link on Fig.~\ref{PosIsoST}(a) 
by the argument in Example~\ref{SmVsLeg}. For even $n$, one needs to create
two such `swallowtails', cf.~\cite[p.~257]{NT}.

\section{Lorentz geometry}

\subsection{Spacetimes and spaces of null geodesics}
\label{spacetimes}
\aftersubsec
A {\it spacetime\/} is a connected time-oriented Lorentz manifold $(\XX,\langle\text{ },\!\text{ }\rangle)$.
The Lorentz metric is taken to be of signature $(+,-,\dots,-)$ so that $\langle v,v\rangle>0$
for timelike vectors and $\langle v,v\rangle<0$ for spacelike vectors.
The time-orientation is a continuous choice of the future hemicone
$$
C^{\uparrow}_x\subset \{v\in T_x\XX\mid \langle v,v\rangle\ge 0, v\ne 0\}
$$
in the cone of non-spacelike vectors at each point~$x\in\XX$.
The vectors in $C^\uparrow_x$ are called {\it future-pointing}.
A piecewise smooth curve in $\XX$ is {\it future-directed\/}
if all its tangent vectors are future-pointing.

The {\it causality relation\/} $\le$  on $\XX$ is defined
by setting $x\le y$ if either $x=y$ or there is a future-directed
curve connecting $x$ to~$y$. 
The {\it chronology relation\/} $\ll$ is defined similarly
by writing $x\ll y$ if there is a future-directed timelike curve 
connecting $x$ to~$y$.

$\XX$ is {\it causal\/} if it does not contain closed 
future-directed curves. (This is equivalent to requiring
that $\le$ is a partial order.) A causal spacetime $\XX$ 
is {\it globally hyperbolic\/} if the causal interval $\{z\in\XX\mid x\le z\le y\}$ 
is compact for every~$x,y\in\XX$, see~\cite{BeSa2}
for the equivalence of this definition and the more classical one~\cite[Definition 5.24]{Pe} or~\cite[\S 6.6]{HaEl}. 
By the smooth splitting theorem~\cite{BeSa1}, 
a globally hyperbolic spacetime is foliated by smooth spacelike 
Cauchy (hyper)surfaces, where a {\it Cauchy surface\/} is a subset 
of a spacetime such that every endless future-directed curve 
intersects it exactly once.

The {\it space of null geodesics\/} of $\XX$ is the set $\mathfrak N_\XX$ 
of equivalence classes of endless future-directed null geodesics 
up to an orientation preserving affine reparametrisation~\cite{Lo2}. 
For a globally hyperbolic $\XX$ of dimension $\ge 3$, this space is a contact manifold 
and to every Cauchy surface $M\subset\XX$ there is associated 
a contactomorphism $\rho_M:\mathfrak N_\XX\overset{\cong}{\longrightarrow} ST^*M$
onto the co-sphere bundle of $M$, see e.g.\ \cite[pp.~252--253]{NT}
or \cite[\S\S 1-2]{ChNe4}.  

The set $\mathfrak S_x\subset\mathfrak N_\XX$ of all null geodesics 
passing through a point $x\in\XX$ is a Legendrian sphere in $\mathfrak N_\XX$
called the {\it sky\/} (or the {\it celestial sphere\/}) of that point.
For any Cauchy surface $M\subset\XX$ and a point $x\in M$, 
$\rho_M(\mathfrak S_x)=ST_x^*M$ and hence all skies are
mapped by $\rho_M$ to the Legendrian isotopy class of the fibre of~$ST^*M$, see \cite[\S 4]{ChNe1}.

The twistor map $x\mapsto\mathfrak S_x$ has the following properties
summarised in this form in~\cite[\S4.2]{ChNe5}.
If $x\le y$, then $\mathfrak S_x\lle\mathfrak S_y$, and if $x\ll y$, then
$\mathfrak S_x\llcurly\mathfrak S_y$, where $\lle$ and $\llcurly$
are the relations on Legendrians introduced in~\S\ref{LegIso}.
Moreover, if the Legendrian isotopy class of the fibre of $ST^*M$
is orderable (for instance, if $M$ is non-compact~\cite{ChNe2}), then 
the converse implications hold as well.

\subsection{D\'ej\`a vu moments}
\label{djvLorentz}
\aftersubsec
Let $\gamma=\gamma(t)$ be a future-directed timelike curve 
in a causal spacetime~$\XX$. A {\it d\'ej\`a vu moment\/} is
a point $x_+=\gamma(t_+)$ such that there exists a point
$x_-=\gamma(t_-)$ with $t_-<t_+$ and a null geodesic 
connecting $x_-$ to~$x_+$. (Causality implies that this null
geodesic will be future-directed.)
The curve $\gamma$ may be thought of as the world line
of an observer who is receiving the same light ray 
at inner time $t_+$ as at the time $t_-$ in the past.

\begin{prop}
\label{djvlinks2moments}
If the skies of two points in a globally hyperbolic spacetime~$\XX$ form a d\'ej\`a vu 
Legendrian link $(\mathfrak S_x,\mathfrak S_y)$, then there are d\'ej\`a vu 
moments on every future-directed timelike curve from $x$ to~$y$. 
\end{prop}

\begin{proof}
Such a curve $\gamma=\gamma(t)$, $t\in[0,1]$, 
defines a piecewise smooth isotopy of skies $\mathfrak S_{\gamma(t)}$
connecting $\mathfrak S_x$ to $\mathfrak S_y$
that is positive on its smooth segments by~\cite[Proposition 4.3]{ChNe5}.
Since there is no positive embedded isotopy between those skies,
it follows from Lemma~\ref{ExistEmb} that there exist $t_0\in(0,1]$ and $t_1\in[0,1)$
such that $\mathfrak S_x\cap\mathfrak S_{\gamma(t_0)}\ne\varnothing$
and $\mathfrak S_{\gamma(t_1)}\cap\mathfrak S_y\ne\varnothing$.
But this means exactly that there are null geodesics connecting 
$x$ to $\gamma(t_0)$ and $\gamma(t_1)$ to~$y$. The skies of $x$
and $y$ are disjoint, so they are not connected by a null geodesic.
Hence, we get at least two d\'ej\`a vu moments at $\gamma(t_0)$
and $y=\gamma(1)$.
\end{proof} 

\begin{rem}
The proof of Proposition~\ref{djvlinks2moments} shows, in other words, that every future-directed timelike curve from $x$ to $y$ 
must intersect the exponentiated future null cone of $x$ and the exponentiated
past null cone of $y$. 
\end{rem}

D\'ej\`a vu links in $ST^*\R^n$ of the types shown on Fig.~\ref{PosIsoST}(d) and Fig.~\ref{PosIsoSTDeg0} 
appear as links of skies in globally hyperbolic static spacetimes of the form $(\R^n\times\R, -g\oplus dt^2)$,
where $g$ is a `bumpy' Riemnannian metric on $\R^n$ creating scattering obstacles. 
More advanced examples of this kind may be derived from~\cite[Fig.~5]{NT}. 

If the Legendrian isotopy class of skies
(i.e.\ the fibre class in $ST^*M$ for a Cauchy surface $M\subset\XX$) is {\it orderable}, 
then the proposition can be strengthened in two ways. 
First, there actually exists 
a future-directed timelike curve from $x$ to $y$ because 
$\mathfrak S_x\llcurly \mathfrak S_y$ implies $x\ll y$.
Secondly, intersections with the exponentiated null cones
exist for {\it every\/} curve connecting $x$ and $y$ 
by Proposition~\ref{DejavuOrdNoEmb}(ii). The first part
of the following example shows that without the orderability
assumption both assertions may be false.
 
\begin{exm}
Let $(M,g)$ be a Riemannian $Y^x_\ell$-manifold, which implies that
the fibre class in $ST^*M$ is {\it not\/} orderable, see \S\ref{yxl}.
The Lorentz direct product $(M\times\R,-g\oplus dt^2)$ is a globally
hyperbolic spacetime.

\smallskip
\noindent
{\bf (i)} If $M$ is not homeomorphic to the sphere, then the skies 
of two points $x=(\underline{x},0)$ and $y=(\underline{y},0)$ 
on the same Cauchy surface $M\times\{0\}$ form a d\'ej\`a vu link by Proposition~\ref{BSnondjv}. 
At the same time, the points are not causally related and can be connected by a (spacelike) curve 
inside $M\times\{0\}$ which does not intersect their exponentiated null cones.

\smallskip
\noindent
{\bf (ii)}  If $M$ is diffeomorphic to the sphere, then $(\mathfrak{S}_x,\mathfrak{S}_{y})$ 
is {\it not\/} a d\'ej\`a vu link. Let us consider the point $y'=(\underline{y},\ell)$.
If $\underline{y}$ is close to $\underline{x}$, then obviously $x\ll y'$. 
Note also that $\mathfrak{S}_{y'}=\mathfrak{S}_y$ because null geodesics in~$\XX$
are of the form $(\beta(s),s)$, where $\beta=\beta(s)$ is a naturally parametrised Riemannian geodesic in $(M,g)$. 
So $(\mathfrak{S}_x,\mathfrak{S}_{y'})$ is {\it not\/} a d\'ej\`a vu link. 
Nevertheless, the conclusion of Proposition~\ref{djvlinks2moments} 
is valid for $x$ and~$y'$ (non-vacuously because $x\ll y'$). To see this, 
note that $x\ll x'=(\underline{x},\ell)$ whereas $y'$ is causally unrelated to $x'$. 
Hence, every curve connecting $x$ and $y'$ intersects the boundary of $I^-(x')=\{z\in\XX\mid z\ll x'\}$. 
In a globally hyperbolic spacetime, this boundary is contained in the 
{\it closed\/} set $J^-(x')=\{z\in\XX\mid z\le x'\}$  
and is covered by null geodesics passing through $x'$ (and hence through~$x$) 
by~\cite[Proposition 3.71]{MS} and \cite[Proposition 2.20]{Pe}. 
It follows that every future-directed curve from $x$ to $y'$ 
must intersect the exponentiated future null cone of~$x$.
Applying the same argument to $I^+(y)=\{z\in\XX\mid y\ll z\}$ 
shows that such a curve must also intersect the exponentiated
past null cone of~$y'$.\qed
\end{exm}

The second part of the preceding example shows that the converse
to Proposition~\ref{djvlinks2moments} need not hold in general.
The example is very special, however, so one may venture the
following (optimistic) conjecture:

\begin{conje}
Let $\XX$ be a globally hyperbolic spacetime such that its Cauchy surface
is {\it not\/} homeomorphic to the sphere. If there are d\'ej\`a vu moments
on every future-directed timelike curve connecting two causally related points $x,y\in\XX$,
then their skies either intersect or form a d\'ej\`a vu Legendrian link
in~$\mathfrak{N}_\XX$.
\end{conje}

If this conjecture is true, it will have have a number of purely
geometric consequences which may be used to support (or refute) it.
Assume, for instance, that $x$ is connected to $y$ by a future-directed 
timelike curve that does not intersect the exponentiated
future null cone of $x$. Then $\mathfrak{S}_x\llcurly\mathfrak{S}_y$
but the link $(\mathfrak{S}_x,\mathfrak{S}_y)$ is not d\'ej\`a vu
by Lemma~\ref{ExistEmb}. The conjecture would imply
that there exists a (maybe different) future-directed timelike curve from $x$ to~$y$
without d\'ej\`a vu moments and, in particular, not intersecting
the exponentiated past null cone of~$y$.


\begin{thebibliography}{99}
\bibitem{BeSa1}
A.~Bernal, M.~S\'anchez, {\em On smooth Cauchy hypersurfaces and Geroch's splitting theorem},
Comm.~Math.~Phys.~{\bf 243} (2003), 461--470.
\bibitem{BeSa2}
A.~Bernal, M.~S\'anchez, 
{\it Globally hyperbolic spacetimes can be defined as ``causal'' instead of
``strongly causal''},  Classical Quantum Gravity {\bf 24} (2007), 745--750.
\bibitem{Be}
A.~L.~Besse, {\em Manifolds all of whose geodesics are closed},
with appendices by D.~B.~A.~Epstein, J.-P.~Bourguignon,
L.~B\'erard-Bergery, M.~Berger and J.~L.~Kazdan.
Ergebnisse der Mathematik und ihrer Grenzgebiete, 93. Springer-Verlag, Berlin-New York, 1978.
\bibitem{Che} 
Yu. V. Chekanov, 
{\it Critical points of quasifunctions, and generating families of Legendrian manifolds}, 
Funktsional. Anal. i Prilozhen.  {\bf 30}:2 (1996), 56--69 (Russian);
English transl. in Funct. Anal. Appl. {\bf 30}:2 (1996), 118--128.
\bibitem{ChNe1}
V. Chernov, S. Nemirovski,
{\it Legendrian links, causality, and the Low conjecture},
Geom. Funct. Anal. {\bf 19} (2010), 1320--1333.
\bibitem{ChNe2}
V. Chernov, S. Nemirovski,
{\it Non-negative Legendrian isotopy in $ST^*M$},
Geom. Topol. {\bf 14} (2010), 611--626.
\bibitem{ChNe3}
V. Chernov, S. Nemirovski,
{\it Universal orderability of Legendrian isotopy classes},
J. Symplectic Geom. {\bf 14} (2016), 149--170.
\bibitem{ChNe4}
V. Chernov, S. Nemirovski,
{\it Redshift and contact forms},
J. Geom. Phys. {\bf 123} (2018), 379--384.
\bibitem{ChNe5}
V. Chernov, S. Nemirovski,
{\it Interval topology in contact geometry},
Commun. Contemp. Math. {\bf 22} (2020), 1950042, 19~pp.
\bibitem{CFP}
V. Colin, E. Ferrand, P. Pushkar, 
{\it Positive isotopies of Legendrian submanifolds and applications},
Int. Math. Res. Not. IMRN (2017), no. 20, 6231--6254.
\bibitem{CS}
V. Colin, S. Sandon,
{\it The discriminant and oscillation lengths for contact and Legendrian isotopies},
J. Eur. Math. Soc. \textbf{17} (2015), 1657--1685.
\bibitem{Da}
L.~Dahinden,
{\it The Bott-Samelson theorem for positive Legendrian isotopies},
Abh. Math. Semin. Univ. Hambg. {\bf 88} (2018), 87--96. 
\bibitem{De}
J.-P.~Demailly, {\it Complex Analytic and Differential Geometry}, 
Ebook available at {\tt www-fourier.ujf-grenoble.fr/$\sim$demailly/manuscripts/agbook.pdf}
\bibitem{DG}
F. Ding, H. Geiges, {\it Legendrian helix and cable links}, 
Commun. Contemp. Math. {\bf 12} (2010), 487--500.
\bibitem{FrLaSch}
U.~Frauenfelder, C.~Labrousse, F.~Schlenk,
{\it Slow volume growth for Reeb flows on spherizations
and contact Bott--Samelson theorems},
J. Topol. Anal. {\bf 7} (2015), 407--451.
\bibitem{Ge}
H. Geiges, {\it An introduction to contact topology},
Cambridge Studies in Advanced Mathematics, 109. Cambridge University Press, Cambridge, 2008.
\bibitem{GKS}
S.~Guillermou, M.~Kashiwara, P.~Schapira,
{\it Sheaf quantization of Hamiltonian isotopies and applications to nondisplaceability problems},
Duke Math. J. {\bf 161} (2012), 201--245.
\bibitem{H}
A.~Haefliger,
{\it Plongements diff\'{e}rentiables de vari\'{e}t\'{e}s dans vari\'{e}t\'{e}s},
Comment. Math. Helv., {\bf 36} (1961), 47--82.
\bibitem{HaEl}
S.~W.~Hawking, G.~F.~R.~Ellis, {\em The large scale structure of space-time},
Cambridge Monographs on Mathematical Physics, No.~1,
Cambridge University Press, London--New York, 1973.
\bibitem{Li}
G.~Liu, 
{\it Positive loops of loose Legendrian embeddings and applications},
J. Symplectic Geom. {\bf 18} (2020), 867--887.
\bibitem{Lo2}
R.~J.~Low, {\em The space of null geodesics}, Proceedings of the
Third World Congress of Nonlinear Analysts, Part 5 (Catania, 2000).
Nonlinear Anal. {\bf 47} (2001), 3005--3017.
\bibitem{McD} D. McDuff, {\it Symplectic manifolds with contact type boundaries},
Invent. Math. {\bf 103} (1991), 651--671.
\bibitem{MS}
E.~Minguzzi, M.~Sanchez,
{\it The causal hierarchy of spacetimes}, 
Recent developments in pseudo-Riemannian geometry, 299--358,
ESI Lect. Math. Phys., Eur. Math. Soc., Z\"urich, 2008. 
\bibitem{NT}
J.~Nat\'ario, P.~Tod,
{\em Linking, Legendrian linking and causality},
Proc.~London Math.~Soc.~(3) {\bf 88} (2004), 251--272.
\bibitem{Pe}
R. Penrose, 
{\it Techniques of differential topology in relativity},
Conference Board of the Mathematical Sciences Regional Conference Series in Applied Mathematics, No. 7. 
Society for Industrial and Applied Mathematics, Philadelphia, Pa., 1972.
\bibitem{PRSZ}
L.~Polterovich, D.~Rosen, K.~Samvelyan, J.~Zhang, 
{\it Topological persistence in geometry and analysis},
University Lecture Series, 74. American Mathematical Society, Providence, RI, 2020, xi+128~pp. 
\bibitem{Th1} 
D.~Th\'eret, 
{\it Utilisation des fonctions g\'en\'eratrices en g\'eom\'etrie symplectique globale}, 
PhD Thesis, Universit\'e Denis Diderot (Paris 7), 1995.
\bibitem{Th2} 
D.~Th\'eret,
{\it A complete proof of Viterbo’s uniqueness theorem on generating functions}, 
Topology Appl. {\bf 96} (1999), 249--266.
\bibitem{Tr} 
L.~Traynor, 
{\it Legendrian circular helix links},
Math. Proc. Cambridge Philos. Soc. {\bf 122} (1997), 301--314.
\bibitem{Vi} C. Viterbo, {\it Symplectic topology as the geometry of generating functions},
Math. Ann. {\bf 292} (1992), 685--710.
\end{thebibliography}

\end{document}
