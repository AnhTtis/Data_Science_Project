%===============================================================================
% ifacconf.tex 2022-02-11 jpuente  
% Template for IFAC meeting papers
% Copyright (c) 2022 International Federation of Automatic Control
%===============================================================================
\documentclass{ifacconf}

\usepackage{graphicx}      % include this line if your document contains figures
\usepackage{natbib}        % required for bibliography
\usepackage{siunitx}
\usepackage{tikz}
\usepackage{pgf}
\usepackage{amsfonts}
\usepackage[acronym, toc]{glossaries}
\usepackage{mathtools}
\usepackage{bm}
\usepackage{siunitx}
\usepackage{subcaption}
\usepackage{multirow}
\usepackage{makecell}
\usepackage{xcolor}
\usepackage{caption}


% GLOSSARY
\makeglossaries
\section*{Glossary}

\begin{tabularx}{\columnwidth}{r X}
    $d$ & Local qudit bond dimension. \\
    $L$ & Number of layers in the NoRA network. \\
    $n_{\ell}$ & Number of qudits being acted on non-trivially at layer $\ell$. \\
    $N$ & Number of physical qudits. Equal to $n_L$. \\
    $k$ & Number of (ground-state) logical qudits. Equal to $n_0$. \\
    $\Delta n_{\ell}$ & Number of new (thermal) qudits introduced at layer $\ell$. Equal to $n_{\ell} - n_{\ell - 1}$. \\
    $D$ & Circuit depth for a single layer of NoRA (layer-independent). \\
    $r$ & Qudit growth rate of the NoRA network with specified scaling. \\
    $q$ & Locality of the circuit at a single later (layer-independent).  \\
    $w_{\ell}$ & Operator weight of a (stabilizer) Pauli string at layer $\ell$. \\
    $g$ & Effective weight growth rate for a single random $q$-local Clifford. Approximately $q \cdot (d^2 - 1)/d^2$. \\
    $D_{\text{sat}}$ & Minimum layer circuit depth necessary to achieve approximate weight saturation. Approximately $ \approx \log_g r$. \\
    $\delta$ & Code distance of a $[[N, k, \delta]]$ error-correcting code. \\
    $\delta_{\text{qsb}}$ & Maximum possible code distance as predicted by the quantum singleton bound. Equal to $(N - k)/2 + 1$. \\
    $\delta / N$ & Relative code distance.
    
\end{tabularx}

\begin{acronym}
    \acro{LDA}{\emph{Latent Dirichlet Allocation}}
    \acro{CMT}{\emph{Conference Management Toolkit}}
    \acro{TPMS}{\emph{The Toronto Paper Matching System}}
    \acro{MCMF}{\emph{MinCost-MaxFlow}}
\end{acronym}

\graphicspath{{images/}}

% tikz
\input{./thesis_styles.tikzstyles}
\usetikzlibrary{backgrounds}
\usetikzlibrary{arrows}
\usetikzlibrary{shapes,shapes.geometric,shapes.misc}

% this style is applied by default to any tikzpicture included via \tikzfig
\tikzstyle{tikzfig}=[baseline=-0.25em,scale=0.5]

% these are dummy properties used by TikZiT, but ignored by LaTex
\pgfkeys{/tikz/tikzit fill/.initial=0}
\pgfkeys{/tikz/tikzit draw/.initial=0}
\pgfkeys{/tikz/tikzit shape/.initial=0}
\pgfkeys{/tikz/tikzit category/.initial=0}

% standard layers used in .tikz files
\pgfdeclarelayer{edgelayer}
\pgfdeclarelayer{nodelayer}
\pgfsetlayers{background,edgelayer,nodelayer,main}

% style for blank nodes
\tikzstyle{none}=[inner sep=0mm]

% include a .tikz file
\newcommand{\tikzfig}[1]{%
{\tikzstyle{every picture}=[tikzfig]
\IfFileExists{#1.tikz}
  {\resizebox{.95\linewidth}{!}{\input{#1.tikz}}}
  {%
    \IfFileExists{./images/#1.tikz}
      {\resizebox{.95\linewidth}{!}{\input{./images/#1.tikz}}}
      {\tikz[baseline=-0.5em]{\node[draw=red,font=\color{red},fill=red!10!white] {\textit{#1}};}}%
  }}%
}

% the same as \tikzfig, but in a {center} environment
\newcommand{\ctikzfig}[1]{%
  \begin{center}\rm
    \tikzfig{#1}
  \end{center}
}

\newcommand{\insertfig}[2]{%
  \begin{figure}[h]
    \centering
    \tikzfig{#1}
    \caption{#2}
    \label{fig:#1}
  \end{figure}
}

% fix strange self-loops, which are PGF/TikZ default
\tikzstyle{every loop}=[]

\DeclareMathOperator{\arctantwo}{arctan2}
\DeclareMathOperator{\dist}{dist}
\DeclareMathOperator{\minimize}{minimize}
%===============================================================================
\begin{document}
\begin{frontmatter}

\title{
    Vision Based Docking of Multiple Satellites with an Uncooperative Target \thanksref{footnoteinfo}
    % An advanced control scheme for visual docking of multiple satellites with an uncooperative target %\thanksref{footnoteinfo}
} 
% Title, preferably not more than 10 words.

\thanks[footnoteinfo]{© 2023 Fragiskos Fourlas, Vignesh Kottayam Viswanathan, Sumeet Satpute, and George Nikolakopoulos. This work has been accepted to IFAC for publication under a Creative Commons Licence CC-BY-NC-ND}

\author{Fragiskos Fourlas}, 
\author{Vignesh Kottayam Viswanathan},
\author{Sumeet Satpute}, and
\author{George Nikolakopoulos}


% \textit{Luleå University of Technology, Luleå, 97187 Sweden}

\address{Robotics and AI Team\\ Lule\aa\,\,University of Technology, Lule\aa\, Sweden\\E-mail: \{frafou; vigkot; sumsat; geonik \}@ltu.se}
% \address[Second]{Colorado State University, 
%   Fort Collins, CO 80523 USA (e-mail: author@lamar. colostate.edu)}
% \address[Third]{e-mail: george.nikolakopoulos@ltu.se}



\begin{abstract}                % Abstract of not more than 250 words.
With the ever growing number of space debris in orbit, the need to prevent further space population is becoming more and more apparent. Refueling, servicing, inspection and deorbiting of spacecraft are some example missions that require precise navigation and docking in space. Having multiple, collaborating robots handling these tasks can greatly increase the efficiency of the mission in terms of time and cost. This article will introduce a modern and efficient control architecture for satellites on collaborative docking missions. The proposed architecture uses a centralized scheme that combines state-of-the-art, ad-hoc implementations of algorithms and techniques to maximize robustness and flexibility. It is based on a Model Predictive Controller (MPC) for which efficient cost function and constraint sets are designed to ensure a safe and accurate docking. A simulation environment is also presented to validate and test the proposed control scheme. 

\end{abstract}

\begin{keyword}
Nonlinear cooperative control, Nonlinear model predictive control, Spacecraft docking, Visual navigation.
\end{keyword}

\end{frontmatter}
%===============================================================================


\captionsetup{width=.45\textwidth}

\section{Introduction} \label{ch:chap_1}
\section{Introduction}

The increasing complexity of source code poses a key challenge to the reliability of large-scale software systems. Software bugs in these systems can lead to safety issues~\cite{bug_safety} for users around the world as well as cause non-negligible financial losses~\cite{bug_loss}. As such, developers have to spend a large amount of time and effort on bug fixing. Consequently, \aprfull (\apr), designed to automatically generate patches to fix software bugs, has attracted wide attention from both academia and industry~\cite{long2016prophet, legoues2012genprog, long2015spr, lou2020can, tufano2018empstudy}. 


To achieve \apr, one popular approach is known as Generate-and-Validate (G\&V)~\cite{qi2015gv, ghanbari2019prapr, lou2020can, le2016hdrepair, legoues2012genprog, wen2018capgen, hua2018sketchfix, martinez2016astor, koyuncu2020fixminder, liu2019tbar, liu2019avatar}, which is typically based on the following pipeline: First, fault localization techniques~\cite{wong2016fl, abreu2007ochiai, zhang2013injecting, papadakis2015metallaxis, li2019deepfl, li2017transforming} are applied to determine the suspicious locations in programs where bugs are likely to exist. Then, the buggy locations are used by the \apr tools to generate a list of patches that replace buggy lines with correct lines. Afterward, each patch is validated against the original test suite to identify any \emph{plausible patches} (i.e., passing all tests in the test suite). Finally, to determine the \emph{correct patches}, developers examine the list of plausible patches to see if any of them can correctly fix the bug. 

Traditional \apr tools can mainly be categorized into heuristic-based~\cite{legoues2012genprog, le2016hdrepair, wen2018capgen}, constraint-based~\cite{mechtaev2016angelix, le2017s3, demacro2014nopol, long2015spr} and \template~\cite{ghanbari2019prapr, hua2018sketchfix, martinez2016astor, liu2019tbar, liu2019avatar}. Among these traditional tools, \template \apr tools~\cite{ghanbari2019prapr, liu2019tbar, benton2020effectiveness} have been able to achieve state-of-the-art results. \Template \apr tools typically leverage pre-defined templates (e.g., adding a nullness check) for bug fixing. However, since these fix templates are typically handcrafted, the number and types of bugs they are able to fix can be limited. 



To address the limitations of traditional \apr, researchers have proposed various \learning \apr tools~\cite{li2020dlfix, chen2018sequencer, jiang2021cure, lutellier2020coconut, zhu2021recoder, ye2022rewardrepair} based on the \nmtfull (\nmt) architecture~\cite{sutskever2014mt} where the input is the buggy code snippets and the goal is to translate the buggy code snippets into a fixed version. To accomplish this, \learning \apr tools require supervised training datasets with pairs of both buggy and fixed code snippets in order to learn how to perform this translation step. These training data are usually obtained by mining historical bug fixes using heuristics/keywords~\cite{dallmeier2007benchmark}, which can be imprecise for identifying bug-fixing commits; even the actual bug-fixing commits can include irrelevant code changes, leading to further pollution in the dataset~\cite{xia2022alpharepair}.
% 
Moreover, it can be hard for such \apr tools to generalize and fix bug types unseen during training. 



To better leverage recent advances in \plmfull{s} (\plm{s}), researchers~\cite{xia2022alpharepair, xia2023repairstudy, kolak2022patch, prenner2021codexws} have directly applied \plm{s} to generate patches without bug-fixing datasets. These \llm-based \apr tools work by either directly generating a complete code function~\cite{prenner2021codexws, xia2023repairstudy} or predict/infill the correct code snippet given its surrounding context~\cite{xia2022alpharepair, xia2023repairstudy}. By directly using \llm{s} that are pre-trained on billions of open-source code snippets, \llm-based \apr tools can achieve state-of-the-art performance on many repair datasets~\cite{xia2022alpharepair}. 


% 
%
%

Traditional \apr tools have long used the insight of the \emph{plastic surgery hypothesis}~\cite{barr2014plastic} where it states that the code ingredients to fix a bug already exist within the same project. Traditional \apr tools have manually designed pattern-~\cite{ghanbari2019prapr, saha2017elixir} or heuristic-based~\cite{jiang2018simfix, legoues2012genprog} approaches to finding and using such relevant code ingredients to generate fixes for bugs. However, the plastic surgery hypothesis has been largely ignored in \llm-based \apr. In fact, \llm provides a unique opportunity to fully automate the plastic surgery hypothesis idea via fine-tuning (learning project-specific information via model updates from the buggy project) and prompting (directly providing relevant code ingredients to the model), and make it directly applicable to different languages (since the \llm{s} are typically multi-lingual).%
Moreover, despite the intensive manual efforts involved, traditional \apr tools still cannot fully leverage project-specific information due to large search space for leveraging/composing existing code ingredients. In contrast, the project-specific information can effectively leveraged by \llm{s} due to their power in code understanding/vectorization, e.g., even partial/imprecise information may still guide \llm{s} in correct patch generation!
 To this end, we ask the question: \emph{How useful is the plastic surgery hypothesis in the era of \plm{s}}?








\mypara{Our Work.} To answer the question, we present \ourtech{\xspace} -- a \llm-based approach that automatically utilizes the plastic surgery hypothesis by systematically combining multiple fine-tuning and prompting strategies for \apr. \ourtech fine-tunes \plm{s} using two novel domain-specific training strategies: \textbf{\epfinetune} -- we fine-tune using the original buggy project by aggressively masking out a high percentage of tokens, which allows \plm to learn project-specific code tokens and programming styles; and \textbf{\rofinetune} -- which only masks out a single continuous code sequence per training sample, allowing the model to get used to the final \csapr task of predicting a single continuous code sequence. Furthermore, we directly leverage the ability for \plm{s} to understand natural language instructions and introduce a novel prompting strategy, \textbf{\idprompting}, which uses information retrieval and static analysis to obtain a list of relevant identifiers for the buggy lines. While such relevant identifiers are critical for fixing some difficult bugs, they may not be seen by the \llm during inference due to limited context window size. Through the use of prompting, we directly tell the model to use these extracted identifiers (relevant code ingredients) to generate the correct code. Finally, to perform repair, we combine all four model variants (including the base model, both fine-tuned models and the base model with prompting) for the final repair.





While our insight of leveraging the plastic surgery hypothesis for \llm-based \apr is generalizable across different types of \plm{s}, to implement \ourtech, we choose a recent \plm{\xspace}, \ctfive~\cite{wang2021codet5}, which is pre-trained on millions of open-source code snippets. \ctfive is an encoder-decoder model trained using \mspfull (\msp) objective where a percentage of tokens are masked out and each continuous masked token sequence is referred to as a masked span. Also, although we only extract relevant identifiers from the current buggy project (since this paper focuses on the plastic surgery hypothesis), our work can be easily extended to obtain other code information (such as relevant statements or functions) from other sources, such as  the massive pre-training corpora~\cite{husain2020codesearchnet} or historical bug-fixing datasets~\cite{jiang2019infer}, which can provide more coding knowledge for \llm{s}. Besides, although we mainly focus on using traditional string comparison algorithms for information retrieval in this paper, these techniques can be easily replaced by other frequency-based retrieval~\cite{robertson2009probabilistic} and neural search (or embedding-based search)~\cite{reimers2019sentence}.
  In summary, this paper makes the following contributions:


%


\begin{itemize}[noitemsep, leftmargin=*, topsep=0pt]
    \item \textbf{Dimension.} This paper is the first to revisit the important plastic surgery hypothesis in the era of \llm{s}. It opens up a new dimension for \llm-based \apr to incorporate previously neglected information from the buggy project itself to boost \apr performance. Furthermore, it demonstrates the promising future of retrieval-based prompting for modern \llm-based \apr.
    \item \textbf{Implementation.} We implement \ourtech based on the recent \ctfive model. We augment the model using two novel fine-tuning strategies: \epfinetune and \rofinetune, along with a novel prompting strategy based on information retrieval and static analysis: \idprompting. We combine the patches generated by all four models together and perform patch ranking to speed up \apr.% 
    \item \textbf{Evaluation Study.} We conduct an extensive evaluation against state-of-the-art \apr tools. On the widely studied \dfj 1.2 and 2.0 datasets~\cite{just2014dfj}, \ourtech is able to achieve the new state-of-the-art results of 89 and 44 correct bug fixes (15 and 8 more than best baseline) respectively.  Furthermore, we perform a broad ablation study to justify our design. \ourtech demonstrates for the first time that the plastic surgery hypothesis can substantially boost \llm-based \apr and advance state-of-the-art \apr, while being fully automated and general. Moreover, even partial/imprecise code ingredients may still effectively guide \llm{s} for \apr!
\end{itemize}



%% There are a number of predefined theorem-like environments in
%% ifacconf.cls:
%%
%% \begin{thm} ... \end{thm}            % Theorem
%% \begin{lem} ... \end{lem}            % Lemma
%% \begin{claim} ... \end{claim}        % Claim
%% \begin{conj} ... \end{conj}          % Conjecture
%% \begin{cor} ... \end{cor}            % Corollary
%% \begin{fact} ... \end{fact}          % Fact
%% \begin{hypo} ... \end{hypo}          % Hypothesis
%% \begin{prop} ... \end{prop}          % Proposition
%% \begin{crit} ... \end{crit}          % Criterion
%% \begin{pf} ... \end{pf}              % Proof

\section{Problem Formulation}\label{ch:chap2}
\section{The Semi-Oblivious Chase Procedure}\label{sec:semi}
%

The semi-oblivious chase (or simply chase) takes as input a database $D$ and a set $\dep$ of TGDs, and constructs an instance that contains $D$ and satisfies $\dep$.
%
A central notion in this context is that of trigger.
%are those of trigger, active trigger, and trigger application.

\begin{definition}%[\textbf{Trigger Application}]
	Given a set $\dep$ of TGDs and an instance $I$, a {\em trigger} for $\dep$  on $I$ is a pair $(\sigma,h)$, where $\sigma \in \dep$ and $h$ is a homomorphism from $\body{\sigma}$ to $I$.
	%
	The {\em result} of $(\sigma,h)$, denoted $\result{\sigma}{h}$, is the set $\mu(\head{\sigma})$, where $\mu : \var{\head{\sigma}} \ra \ins{C} \cup \ins{N}$ is defined as follows:
	%
	%$\mu(x) = h(x)$ if $x \in \fr{\sigma}$, and $\mu(x) = \bot_{\sigma,h_{|\fr{\sigma}}}^{x}$ otherwise,
	\[
	\mu(x)\
	=\ \left\{
	\begin{array}{ll}
	h(x) & \quad \text{if } x \in \fr{\sigma}\\
	&\\
	\bot_{\sigma,h_{|\fr{\sigma}}}^{x} & \quad \text{otherwise}
	\end{array} \right.
	\]
	where $\bot_{\sigma,h_{|\fr{\sigma}}}^{x} \in \ins{N}$.  Let $T(\dep,I)$ be the set of triggers for $\dep$ on $I$.	\hfill\markfull
\end{definition}




Observe that in the definition of $\result{\sigma}{h}$, each existentially quantified variable $x$ of $\head{\sigma}$ is mapped by $\mu$ to a null value of $\ins{N}$ whose name is uniquely determined by the trigger $(\sigma,h)$ and the variable $x$ itself. This means that, given a trigger $(\sigma,h)$, we can unambiguously construct the set of atoms $\result{\sigma}{h}$.
%
The central idea of the chase is, starting from a database $D$, to exhaustively apply triggers for the given set $\dep$ of TGDs on the instance constructed so far.
%
More precisely, given a database $D$ and a set $\dep$ of TGDs, let
\[
\mathsf{chase}^{0}(D,\dep)\ =\ D,
\]
and for each $i>0$, let
\[
\mathsf{chase}^{i}(D,\dep)\ =\ \mathsf{chase}^{i-1}(D,\dep)\ \cup\ \bigcup_{(\sigma,h) \in S} \result{\sigma}{h},
\]
where $S = T(\dep,\mathsf{chase}^{i-1}(D,\dep))$. 
%
We finally define {\em the result of the chase of $D$ w.r.t.~$\dep$} as the (possibly infinite) instance
\[
\chase{D}{\dep}\ =\ \bigcup_{i \geq 0} \mathsf{chase}^{i}(D,\dep).
\]


\ignore{
The semi-oblivious chase procedure (or simply chase) takes as input a database $D$ and a set $\dep$ of TGDs, and constructs an instance that contains $D$ and satisfies $\dep$.
%
Central notions in this context are those of trigger, active trigger, and trigger application.

\begin{definition}%[\textbf{Trigger Application}]
	Given a set $\dep$ of TGDs and an instance $I$, a {\em trigger} for $\dep$  on $I$ is a pair $(\sigma,h)$, where $\sigma \in \dep$ and $h$ is a homomorphism from $\body{\sigma}$ to $I$.
	%
	The {\em result} of $(\sigma,h)$, denoted $\result{\sigma}{h}$, is the set $\mu(\head{\sigma})$, where $\mu : \var{\head{\sigma}} \ra \ins{C} \cup \ins{N}$ is defined as follows:
	%
	%$\mu(x) = h(x)$ if $x \in \fr{\sigma}$, and $\mu(x) = \bot_{\sigma,h_{|\fr{\sigma}}}^{x}$ otherwise,
	\[
	\mu(x)\
	=\ \left\{
	\begin{array}{ll}
	h(x) & \quad \text{if } x \in \fr{\sigma}\\
	&\\
	\bot_{\sigma,h_{|\fr{\sigma}}}^{x} & \quad \text{otherwise}
	\end{array} \right.
	\]
	where $\bot_{\sigma,h_{|\fr{\sigma}}}^{x}$ is a null value from $\ins{N}$.
	%
	The trigger $(\sigma,h)$ is {\em active} if $\result{\sigma}{h} \not\subseteq I$.
	%
	The {\em application} of $(\sigma,h)$ to $I$ returns the instance $J = I \cup \result{\sigma}{h}$ and is denoted as $I \app{\sigma}{h} J$.
	\hfill\markfull
\end{definition}


Observe that in the definition of $\result{\sigma}{h}$ above, each existentially quantified variable $x$ of $\head{\sigma}$ is mapped by $\mu$ to a null value of $\ins{N}$ whose name is uniquely determined by the trigger $(\sigma,h)$ and the variable $x$ itself. This means that, given a trigger $(\sigma,h)$, we can unambiguously extract the set of atoms 
$\result{\sigma}{h}$.



%\medskip

%\noindent
%\textbf{Semi-Oblivious Chase.}
The central idea of the chase is, starting from a database $D$, to exhaustively apply active triggers for the given set $\dep$ of TGDs on the instance constructed so far. This is formalized via the notion of (semi-oblivious) chase derivation, which can be finite or infinite.


\begin{definition}
	Consider a database $D$ and a set $\dep$ of TGDs.
	%We consider the two cases where a derivation is finite or infinite:
	\begin{itemize}
		\item A finite sequence $(I_i)_{0 \leq i \leq n}$ of instances, with $D = I_0$ and $n \geq 0$, is a {\em chase derivation} of $D$ w.r.t.~$\dep$ if, for each $i \in \{0,\ldots,n-1\}$, there is an active trigger $(\sigma,h)$ for $\dep$ on $I_i$ with $I_i \app{\sigma}{h} I_{i+1}$, and there is no active trigger for $\dep$ on $I_n$. The {\em result} of such a chase derivation is the instance $I_n$.
		
		
		\item An infinite sequence $(I_i)_{i \geq 0}$ of instances, with $D = I_0$, is a {\em chase derivation} of $D$ w.r.t.~$\dep$ if, for each $i \geq 0$, there is an active trigger $(\sigma,h)$ for $\dep$ on $I_i$ such that $I_i \app{\sigma}{h} I_{i+1}$. Moreover, $(I_i)_{i \geq 0}$ is {\em fair} if, for each $i \geq 0$, and for every active trigger $(\sigma,h)$ for $\dep$ on $I_i$, there exists $j > i$ such that $(\sigma,h)$ is not an active trigger for $\dep$ on $I_j$. 
		%The latter is known as the {\em fairness condition}, and guarantees that all the active triggers will be deactivated. %
		The {\em result} of such a chase derivation is the instance $\bigcup_{i \geq 0} \, I_i$.
	\end{itemize}
	%
	%The {\em result} of a chase derivation is defined as the union of all the instances occurring in it. 
	A chase derivation is {\em valid} if it is finite or infinite and fair.  \hfill\markfull
\end{definition}


Let us stress that infinite but unfair chase derivations are not considered as valid ones since they do not serve the main purpose of the chase, that is, to build an instance that satisfies the given set of TGDs. Indeed, given the set $\dep$ consisting of the TGDs
\[
\sigma\ =\ R(x,y) \ra \exists z \, R(y,z) \qquad \sigma'\ =\ R(x,y) \ra P(x,y),
\]
the result of the unfair chase derivation of $D = \{R(a,b)\}$ w.r.t.~$\dep$ that involves only triggers of the form $(\sigma,\cdot)$, i.e., only the TGD $\sigma$ is used, does not satisfy $\sigma'$, and thus, it does not satisfy $\dep$.
%
Interestingly, for every database $D$ and set $\dep$ of TGDs, any two valid chase derivations of $D$ w.r.t.~$\dep$ have always the same result, which implies that all valid chase derivations are either finite or infinite~\cite{GrOn18}. Therefore, in the rest of the paper, we can safely refer to {\em the} result of the chase of $D$ w.r.t. $\dep$, which we will denote by $\chase{D}{\dep}$. 
}


%\subsection{Non-Uniform Chase Termination}\label{sec:problem}
%

\medskip

\noindent
\textbf{Chase Termination.}
The result of the chase may be infinite even for very simple settings: it is easy to see that for $D = \{R(a,b)\}$ and $\dep = \{R(x,y) \ra \exists z \, R(y,z)\}$, $\chase{D}{\dep}$ is infinite.
%; in particular, $\chase{D}{\dep} = \{R(a,b),R(b,\bot_1),R(\bot_1,\bot_2),R(\bot_2,\bot_3),\ldots\}$, where $\bot_1,\bot_2,\ldots$ are null values.
%
This leads to the following problem, parameterized by a class $\class{C}$ of TGDs such as $\class{SL}$ (the class of simple-linear TGDs) and $\class{L}$ (the class of linear TGDs):


\medskip

\begin{center}
	\fbox{
		\begin{tabular}{ll}
			%{\small PROBLEM} : & %$\mathsf{ChaseTermination}(\class{C})$
			%\\
			{\small INPUT} : & A database $D$ and a set $\dep$ of TGDs from $\class{C}$.
			\\
			{\small QUESTION} : &  Is the instance $\chase{D}{\dep}$ finite?
	\end{tabular}}
\end{center}

\medskip

\noindent This problem has been recently studied in~\cite{CaGP22} for the classes of simple-linear and linear TGDs. Interestingly, for both classes, the finiteness of the result of the chase has been syntactically characterized by exploiting the notion of non-uniform weak-acyclicity. 
%
We proceed to recall this acyclicity notion, and then present the characterizations established in~\cite{CaGP22}, which in turn lead to simple algorithms for checking the finiteness of the result of the chase.
%
Note that, for the sake of clarity, in the rest of the paper we assume TGDs with a non-empty frontier, i.e., we assume that there is at least one variable in a TGD $\sigma$ that occurs both in $\body{\sigma}$ and $\head{\sigma}$. This assumption can be made without loss of generality since, given a database $D$ and a set $\dep$ of TGDs, we can easily construct a set $\dep'$ of TGDs with a non-empty frontier by slightly modifying $\dep$ such that $\chase{D}{\dep}$ is finite iff $\chase{D}{\dep'}$ is finite.


\medskip

\noindent
\textbf{Non-Uniform Weak-Acyclicity.} Weak-acyclicity was introduced in~\cite{FKMP05} as the main formalism for data exchange purposes, which guarantees the finiteness of the result of the chase for {\em every} input database. Non-uniform weak-acyclicity is the database-dependent variant of weak-acyclicity introduced in~\cite{CaGP22}. We proceed to give the formal definitions.
%
We first need to recall the notion of the {\em dependency graph} of a set $\dep$ of TGDs, 
%which symbolically encodes how terms may propagate during the chase.
%The {\em dependency graph} of set $\dep$ of TGDs 
defined as a directed multigraph $\depg{\dep}=(N,E)$, where $N = \pos{\sch{\dep}}$ and $E$ contains {\em only} the following edges.
%
For each TGD $\sigma \in \dep$ with $\head{\sigma} = \{\alpha_1,\ldots,\alpha_k\}$, for each $x \in \frontier{\sigma}$, and for each position $\pi \in \posvar{\body{\sigma}}{x}$:
\begin{itemize}
	\item For each $i \in [k]$ and for each $\pi' \in \posvar{\alpha_i}{x}$, there exists a \emph{normal} edge $(\pi,\pi') \in E$.
	%
	\item For each existentially quantified variable $z$ in $\sigma$, $i \in [k]$, and $\pi' \in \posvar{\alpha_i}{z}$, there is a \emph{special} edge $(\pi,\pi') \in E$.
\end{itemize}
%
We further need to define when a predicate is reachable from another predicate. 
%
Given predicates $R,P \in \sch{\dep}$, {\em $P$ is reachable from $R$ (w.r.t.~$\dep$)} if $R = P$, or there exists a path in $\depg{\dep}$ from a position of the form $(R,i)$ to a position of the form $(P,j)$.
%
%we write $R \ra_\dep P$  if $R = P$, or there exists a TGD $\sigma \in \dep$ such that $R$ occurs in $\body{\sigma}$ and $P$ occurs in $\head{\sigma}$. We say that {\em $P$ is reachable from $R$ (w.r.t.~$\dep$)}, denoted $R \reach{\dep} P$, if (i) $R \ra_\dep P$, or (ii) there exists $T \in \sch{\dep}$ such that $R \reach{\dep} T$ and $T \ra_\dep P$.
%in $\depg{\dep}$, denoted $R \reach{\dep} P$, if there exists a path in $\depg{\dep}$ from a position $(R,i)$ to a position $(P,j)$, for some $i \in [\arity{R}]$ and $j \in [\arity{P}]$.
Given a database $D$, we say that a (not necessarily simple and possibly cyclic) path $C$ in $\depg{\dep}$ is \emph{$D$-supported} if there exists an atom $R(\bar t) \in D$ and a node of the form $(P,i)$ in $C$ such that $P$ is reachable from $R$.
%
We are now ready to recall (non-uniform) weak-acyclicity.



\begin{definition}\label{def:dwa}
	Consider a database $D$ and a set $\dep$ of TGDs. We say that $\dep$ is {\em weakly-acyclic w.r.t.~$D$}, or {\em $D$-weakly-acyclic}, if there is no $D$-supported cycle in $\depg{\dep}$ with a special edge. 
	%
	We say that $\dep$ is {\em weakly-acyclic} if there is no cycle in $\depg{\dep}$ with a special edge. \hfill\markfull
\end{definition}


\smallskip

\noindent
\textbf{Characterizing the Finiteness of the Chase.}
It is not very difficult to show that whenever a set $\dep$ of TGDs (not necessarily linear) is $D$-weakly-acyclic, then the instance $\chase{D}{\dep}$ is finite. In other words, the $D$-weak-acyclicity of $\dep$ is a sufficient condition for the finiteness of $\chase{D}{\dep}$. What is more interesting is that, assuming that $\dep$ is a set of simple-linear TGDs, the $D$-weak-acyclicity of $\dep$ is also a necessary condition for the finiteness of $\chase{D}{\dep}$. This leads to the following characterization established in~\cite{CaGP22}:

\begin{theorem}\label{the:characterization-simple-linear}
	Consider a database $D$ and a set $\dep \in \class{SL}$ of TGDs. It holds that $\chase{D}{\dep}$ is finite iff $\dep$ is $D$-weakly-acyclic.
\end{theorem}

For linear TGDs, it turned out that non-uniform weak-acyclicity is not powerful enough for characterizing the finiteness of the chase instance. Here is an example given in~\cite{CaGP22} that illustrates this fact:
%This is illustrated by the following example.


\begin{example}
	Consider the database $D = \{R(a,b)\}$ and the singleton set $\dep$ consisting of the (non-simple) linear TGD
	\[
	R(x,x)\ \ra\ \exists z \, R(z,x). 
	\]
	It is easy to see that there is no trigger for $\dep$ on $D$. This means that $\chase{D}{\dep} = D$ is finite, whereas $\dep$ is {\em not} $D$-weakly-acyclic. \hfill\markfull
\end{example}


To obtain a characterization analogous to Theorem~\ref{the:characterization-simple-linear}, the authors of~\cite{CaGP22} used the technique of {\em simplification} to convert linear TGDs into simple-linear TGDs, while preserving the finiteness of the chase instance. We proceed to recall this technique.
%
Let $\bar t = (t_1,\ldots,t_n)$ be a tuple of (not necessarily distinct) terms. We write $\unique{\bar t}$ for the tuple obtained from $\bar t$ by keeping only the first occurrence of each term in $\bar t$.
%
For example, if $\bar t = (x,y,x,z,y)$, then $\unique{\bar t} = (x,y,z)$.
%
For each $i \in [n]$, the \emph{identifier of $t_i$ in $\bar t$}, denoted $\id{\bar t}{t_i}$, is the integer that identifies the position of $\unique{\bar t}$ at which $t_i$ appears. 
%
We write $\id{}{\bar t}$ for the tuple $(\id{\bar t}{t_1},\ldots,\id{\bar t}{t_n})$.
%
For example, if $\bar t = (x,y,x,z,y)$, then $\id{}{\bar t} = (1,2,1,3,2)$.
%
For an atom $\alpha = R(\bar t)$, the {\em simplification of $\alpha$}, denoted $\simple{\alpha}$, is the atom $R_{\id{}{\bar t}}(\unique{\bar t})$, whereas the {\em shape of $\alpha$}, denoted $\shape{\alpha}$, is the predicate $R_{\id{}{\bar t}}$. We can naturally refer to the simplification and the shape of a set of atoms.
%
For a tuple of variables $\bar x = (x_1,\ldots,x_n)$, a \emph{specialization of $\bar x$} is a function $f$ from $\bar x$ to $\bar x$ such that $f(x_1) = x_1$, and $f(x_i) \in \{f(x_1),\ldots,f(x_{i-1}),x_i\}$, for each $i \in \{2,\ldots,n\}$.
We write $f(\bar x)$ for $(f(x_1),\ldots,f(x_n))$. We are now ready to recall how a set of linear TGDs is converted into a set of simple-linear TGDs.

\begin{definition}\label{def:simplification}
	Consider a linear TGD $\sigma$ of the form
	\[
	R(\bar x) \ra \exists \bar z\, \psi(\bar y,\bar z), 
	\]
	where $\bar y \subseteq \bar x$, and a specialization $f$ of $\bar x$. The {\em simplification of $\sigma$ induced by $f$} is the simple-linear TGD
	\[
	\simple{R(f(\bar x))} \rightarrow \exists \bar z\, \simple{\psi(f(\bar y),\bar z)}.
	\]
	We write $\simple{\sigma}$ for the set of all simplifications of $\sigma$ induced by some specialization of $\bar x$.
	%
	For a set $\dep \in \class{L}$ of TGDs, the {\em simplification of $\dep$} is defined as the set
	\[
	\simple{\dep}\ =\ \bigcup_{\sigma \in \dep} \simple{\sigma}
	\]
	consisting only of simple-linear TGDs. \hfill\markfull
\end{definition}

We can now recall the characterization for the finiteness of the chase instance for linear TGDs, established in~\cite{CaGP22}, which is similar to the one for simple-linear TGDs, with the key difference that first we need to simplify both the database and the set of linear TGDs:

\begin{theorem}\label{the:characterization-linear}
	Consider a database $D$ and a set $\dep \in \class{L}$ of TGDs. Then, $\chase{D}{\dep}$ is finite iff $\simple{\dep}$ is $\simple{D}$-weakly-acyclic.
\end{theorem}

It is clear that Theorems~\ref{the:characterization-simple-linear} and~\ref{the:characterization-linear} provide simple algorithms for checking whether the chase instance is finite. In particular, given a database $D$ and a set $\dep$ of simple-linear TGDs, we simply need to check whether $\dep$ is $D$-weakly-acyclic, in which case the algorithm returns \true; otherwise, it returns \false. The same holds when $\dep$ is a set of linear TGDs, with the difference that the algorithm first needs to simplify $D$ and $\dep$, and then perform the acyclicity check.
%
Our goal is to experimentally evaluate the above algorithms with the aim of understanding which input parameters affect their performance, clarifying whether they can be applied in a practical context, and revealing their performance limitations. Of course, a naive implementation of the above algorithms, especially for linear TGDs where the expensive simplification must be applied, will lead to poor performance, and thus, will not be very useful towards our goal. Hence, we need to somehow convert the above theoretical algorithms into practical algorithms that are amenable to efficient implementations. This is the subject of the next section.

\section{Proposed Methodology}\label{ch:chap2.5}
\section{Method}
\label{s:method}

We consider the 3D euclidean space $\Real^3$ with points $p=(x,y,z)\in\Real^3$. We discretize the unit cube $\gC=[0,1]^3$ with a 3D voxel grid $\gG=\set{p_I}$, with nodes $p_I$ indexed by $I=(i,j,k)$, $i,j,k\in [n]=\set{1,\ldots,n}$, \ie, $p_I=(x_{ijk},y_{ijk},z_{ijk})$. We denote by $h=n^{-1}$, and by $N=n^3$ the total number of nodes.   
We represent our reconstructed surface as a zero level of a scalar function $f$ defined over the cube $\gC$. $f$ is defined by prescribing its values at the grid's nodes $f_I\in\Real$ and trilinear interpolating in each voxel. We will denote by $f(p)$ the interpolated value at point $p$. 

Given an input point cloud consisting of $m$ points $q_k\in\Real^3$ with or without (unit norm) normals $n_k\in \Real^3$, $k\in [m]$, our goal is to compute $f$ so that its zero level set approximates the unknown surface, \ie, 
\begin{equation}
    \gS = \set{p\in\gC \ \vert \ f(p)=0}.
\end{equation}
Our approach to compute $f$ is to minimize a loss function of the form
\begin{equation}
    \gL = \gL_{\text{data}} + \gL_{\text{prior}}
\end{equation}
where 
\begin{equation}\label{e:loss_data}
    \gL_{\text{data}} = \frac{\lambda_{\text{p}}}{m}\sum_{k=1}^m \abs{f(q_k)}^2 + \frac{\lambda_{\text{n}}}{m}\sum_{k=1}^m \norm{\nabla f(q_k) - n_k}^2
\end{equation}
where $\norm{\cdot}$ is the standard euclidean norm in $\Real^3$, $\nabla f(p) \in \Real^3$ is the gradient of $f$ sampled at point $p$. Note that $\nabla f$ is defined in interior of voxels, which is generically where the input points $q_k$ resides. $\gL_{\text{data}}$ is the standard data loss encouraging the zero level to pass through the input points $q_k$, and its normals (defined by gradients of $f$) to coincide with input normals $n_k$. 

The prior, $\gL_{\text{prior}}$, is the main contribution of this work, where we combine two novel losses,
\begin{equation}
    \gL_{\text{prior}} = \lambda_{\text{v}} \gL_{\text{viscosity}} + \lambda_{\text{c}} \gL_{\text{coarea}}
\end{equation}
Intuitively, the viscosity loss optimizes for a smooth Signed Distance Function (SDF) solutions, avoiding auxiliary bad minima of the Eikonal equation, while the coarea loss strives to minimize the area of the zero level surface. Our loss has $4$ hyper-parameters $\lambda_{\text{p}},\lambda_{\text{n}},\lambda_{\text{v}},\lambda_{\text{c}}$. We provide more details on these priors next. 


\subsection{Viscosity Loss}\label{ss:viscosity_loss}
The goal of the viscosity loss is to make $f$ approximate an SDF over $\gC$. Given boundary conditions asking $f$ to vanish on some closed compact surface $\gS$, the SDF solves the Eikonal equation PDE, \ie, $\norm{\nabla f(p)}=1$, in a certain well defined sense (viscosity). This motivated some previous work to directly optimize the Eikonal loss \citep{gropp2020implicit,sitzmann2020implicit}
\begin{equation}\label{e:loss_eikonal}
    \gL_{\text{eikonal}} = \int_\gC \Big (\norm{\nabla f(p)}-1\Big )^2 dp
\end{equation}
\begin{wrapfigure}[14]{r}{0.28\textwidth}\vspace{-15pt}
  \begin{center}
    \includegraphics[width=0.25\textwidth]{figs/illustrations/eikonl_1d.png}
  \end{center}
  \caption{Two global minimizers of the Eikonal loss over a grid in 1D. Top solution is not an SDF. }\label{fig:eikonal_1d}
\end{wrapfigure}
Unfortunately, the Eikonal loss has many undesirable minima which are not SDFs. Figure \ref{fig:eikonal_1d} shows a 1D example: both depicted solutions (denoted $f$) vanish at the input points $q_1,q_2$ (black points) and globally minimize the Eikonal loss over the grid (grid points are shown in blue). The INR works mentioned above use neural networks for representing $f$ which injects an inductive bias avoiding these bad minima, however on grids, minimizing \eqref{e:loss_eikonal} cannot avoid these solutions. See, \eg, middle column in Figure \ref{fig:teaser}. 

More classical Eikonal solvers do work with grids however use mostly fast marching or sweeping methods \citep{osher1988fronts,sethian1996fast,zhao2005fast,chacon2012fast}. Namely, use a special discretization of the Eikonal equation favoring the viscosity  solution of the Eikonal \cite{rouy1992viscosity}, and update node values according to a moving front \cite{sethian1996fast}. Since this discretization is up-wind (will only propagate values in one direction) and requires choosing the maximal among its solution, its success in adaptation to a loss is not clear. 

We use a different approach to build a loss favoring SDF solutions over grids motivated by the vanishing viscosity method \cite{crandall1983viscosity}. Namely, adding to the Eikonal PDE a small perturbation of the Laplacian of $f$ (denoted by $\Delta f$), \ie, $\norm{\nabla f(p)}-1 - \eps\Delta f(p)=0$, makes the PDE semi-linear elliptic \citep{calder2018lecture}, and hence with suitable boundary conditions it is uniquely solvable inside $\gS$ with a smooth solution, approaching the viscosity positive distance function to the boundary as $\eps\too 0$. Similarly, for $1-\norm{\nabla f(p)} - \eps \Delta f(p)=0$ the solution approaches the negative distance function inside the domain. 
Motivated by the vanishing viscosity principle we suggest the following viscosity loss:
\begin{equation}\label{e:loss_viscosity_eikonal}
\gL_{\text{viscosity}} = \int_\gC \Big((\norm{\nabla f (p)}-1)\mathrm{sign}(f(p)) - \eps \Delta f(p)\Big)^2 dp
\end{equation}
We discretize this loss over the grid $\gG$ by replacing the first order derivatives and second order derivatives with symmetric finite  differences, \ie,
\begin{align*}
    D_x f_I=D_x f_{i,j,k} = \frac{f_{i+1,j,k}-f_{i-1,j,k}}{2h}, \quad D^2_x f_I = D^2_x f_{i,j,k}=\frac{f_{i+1,j,k}-2f_{i,j,k}+f_{i-1,j,k}}{h^2}
\end{align*}
and similarly for $D_y$ and $D_z$. We use these discrete operators to approximate the gradient $\widehat{\nabla} f(p_I) = (D_x f_I, D_y f_I, D_z f_I)$ and Laplacian $\widehat{\Delta}f(p_I) = D_x^2f_I + D_y^2 f_I + D_z^2 f_I$. The discretized viscosity loss now takes the form
\begin{equation}
    \widehat{\gL}_{\text{viscosity}} = \frac{1}{N}\sum_{I} \Big((\|\widehat{\nabla} f (p_I)\|-1)\mathrm{sign}(f(p_I)) - \eps \widehat{\Delta} f(p_I)\Big)^2
\end{equation}



\subsection{Coarea loss}\label{ss:coarea_loss}
The coarea loss is approximating the area of the zero level set, and therefore incorporating it in the optimization pushes the reconstructed surface to be economic in area. 

First, similarly to  \citep{yariv2021volume} we use the centered Laplace CDF
\begin{equation}
   \Psi\beta(s)= \begin{cases}
   \frac{1}{2}\exp\parr{\frac{s}{\beta}} & s\leq 0 \\ 1-\frac{1}{2}\exp\parr{-\frac{s}{\beta}} & s\geq  0
   \end{cases}
\end{equation} to transform the SDF $f$ to a smooth approximation of the indicator function:
\begin{equation}
    \chi_\beta(p)=\Psi\beta (-f(p))
\end{equation}
As $\beta\too 0$, $\chi_\beta$ converges to an indicator function leading to $1$ inside $\gS$ and $0$ outside. The coarea loss is defined as 
\begin{equation}
    \gL_{\text{coarea}} = \int_\gC \norm{\nabla \chi_\beta (p)} dp
\end{equation}
To understand why this loss approximates the area of $\gS$ we can use the coarea formula \citep{rindler2018calculus}:
\begin{equation}\label{e:coarea}
    \int_\gC \norm{\nabla \chi_\beta(p)}dp = \int_{-\infty}^{\infty} \mathrm{area}(\chi_\beta^{-1}(s))ds,
\end{equation}
where $\chi_\beta^{-1}(s)=\set{p\ \vert \ \chi_\beta(p)=s}$ is the preimage of the value $s$. Since $\chi_x(p)\in [0,1]$ the r.h.s.~integral can be restricted to the interval $[0,1]$, and therefore the coarea loss averages the area of the level sets of $\chi_\beta$. Next,  $$\chi_\beta^{-1}(s)= \set{p\ \vert \ \Psi\beta (-f(p)) = s } = \{p\ \vert \ f(p) = -\Psi\beta^{-1} (s) \} = f^{-1}(-\Psi\beta^{-1} (s)),$$
\begin{wrapfigure}[11]{r}{0.28\textwidth}\vspace{-20pt}
  \begin{center}
  \includegraphics[width=0.25\textwidth]{figs/semi.png}
  \end{center}
  \caption{Reconstruction of a semisphere point cloud (white dots) without (left) and with (right) coarea loss. }\label{fig:coarea_semisphere}
\end{wrapfigure}

which shows that the level set $s\in (0,1)$ of $\chi_\beta$ is the level set $-\Psi\beta^{-1}(s)$ of the SDF $f$. As $\beta\too 0$, $-\Psi\beta^{-1}(s)\too 0$ for all $s\in (0,1)$ (and uniformly in $(\eps,1-\eps)$ for fixed $\eps>0$). Therefore the average of the level set area (\ie, the r.h.s.~of \eqref{e:coarea}) converges to the area of $f^{-1}(0)=\gS$. Figure \ref{fig:teaser} (right) shows how removing the coarea loss introduces an extraneous zero level set, and hence results in an undesired surface part. Figure \ref{fig:coarea_semisphere} shows a comparison of a reconstruction of semisphere with and without coarea. In the experiments section we provide more ablation tests with the coarea and viscosity losses.

To discretize the coarea loss we let $w_I$ denote the centers of grid's voxels, and note that $\nabla \chi_\beta(w_I) = \Phi_\beta(-f(w_I))\nabla f(w_I)$, where 
\begin{equation*}
    \Phi_\beta(s) = \frac{1}{2\beta}\exp\parr{\frac{\abs{s}}{\beta}}
\end{equation*}
is the PDF of the Laplace distribution, and $\nabla f(w_I)$ is computed as a linear combination of the voxel's corner values $f_{I_1},\ldots,f_{I_8}$, see more details in the Appendix. We end up with the discretized loss:
\begin{equation}
    \widehat{\gL}_{\text{coarea}} = \frac{1}{N}\sum_{I}\Phi_\beta(-f(w_I))\norm{\nabla f(w_I)}
\end{equation}
This loss is usually incorporated with a small hyper-parameter $\lambda_{\text{c}}$ with the purpose of eliminating redundant surface parts.



\section{Controller} \label{ch:chap3}
% \acrfull{mpc} is an advanced form of closed-loop control that calculates the optimal input sequence for a dynamic-sytem to reach a desired state.
% \acrshort{mpc} works really well with complex, non-linear, \acrshort{mimo} systems that may have interactions between their inputs and outputs.
% It is also able to handle constraints on the system's input, state and output.
% This enables the controller to simulate the bounded inputs of a realistic model as well as implement a collision avoidance system inside the controller itself.

% \acrshort{mpc} works by using a mathematical model of the plant to predict the system's behavior in the future.
% At time $t$, the system's state is sampled by means of various sensors and a control strategy is calculated to minimize a cost function over a short horizon $[t, t+T]$.
% An optimization algorithm is used to calculate a sequence of control inputs $u_k$ that will drive the system to a state that minimizes the cost function.
% Only the first step of the control strategy, $u_0$, is used to drive the system.
% Then, in the next control iteration, the plant's state is sampled again and a new control strategy is calculated.
% This continuous shifting of the horizon to the future, eventually drives the system to the desired state.

% \insertfig{1_mpc}{Block diagram of a \acrshort{mpc} controller}

% In this project, the chasers need to maintain a fixed pose relative to the \gls{target} and avoid collision with the rest of the chasers and the target.
% To achieve this, both the future states of the chasers and the target need to be predicted and their relative pose needs to be maintained in every time-step of the horizon.
% Details about the system dynamics of both a \gls{chaser} and the target in can be found in Section \ref{ch:chap3:sec2} and the cost function that implements this concept in Section \ref{ch:chap3:sec3}.
% Additional constraints are added to the optimization problem to ensure collision avoidance.
% Details about the constraints enforced on the optimizer can be found in Section \ref{ch:chap3:sec4}.

% To sum up, an \acrshort{mpc} controller uses the following principles:
% \begin{itemize}
%     \item A model of the controlled dynamic system.
%     \item A cost function $J$ over the horizon $[t, t+T]$.
%     \item An optimization algorithm that minimizes $J$ using the control input $u$ while obeying a set of constraints.
% \end{itemize}

As previously mentioned, the control architecture uses a MPC backbone.
In this section, the cost function and constraints used to formulate the MPC will be presented.




% \subsection{Euler Method} \label{ch:chap3:euler}
% Now the system dymanics need to be descretized and solved to be used with the MPC controller.
% For that purpose, the Euler method is used (Equation \ref{eq:3_euler_method}).
% The local error of this method is proportinal to the square of the step size, and the global error proportional to the step size.
% This means that a smaller step size will result in a more accurate solution but for the same horizon duration, smaller step size will result in higher computational complexity and thus an increase on processing time.
% A good balance between time-step and horizon length is necessary to obtain a good \acrshort{mpc} prediction and keep the calculation time low.

% \begin{equation} \label{eq:3_euler_method}
%     \begin{array}{c}
%         \text{For a dynamic system defined as: } \dot{x} = f(x,u)\\
%         x_{k+1} = x_k + T_s f(x_k,u_k)
%     \end{array}
% \end{equation}

% From Equation \ref{eq:3_chaser_dynamics}, it is obvious that the control inputs that will be used in the descritized equations for each chaser are:

% \begin{equation}
%     \begin{split}
%         u   &= [F_x, F_y, F_z, \tau_x, \tau_y, \tau_z]^T\\
%             &= [\bm{F}, \bm{\tau}]^T
%     \end{split}
% \end{equation}

\subsection{Cost Function} \label{ch:chap3:sec3}

% As mentioned in Section \ref{ch:chap3}, a cost function $J$ must be determined for the \acrshort{mpc} controller to work.
% This cost function will be minimized by the optimization algorithm using $u$ to drive the plant to the desired state.

% Since the mission requires each chaser $i$ to be at a pose ${x_{ref}^i}$ relative to the target, the cost function could be defined as the distance between the desired pose ${x_{ref}^i}$ and the chaser's pose $p_{ch} = [\bm{x}, \bm{q}]$.
% But $ref^i$ changes while the target is moving during the time horizon.
% Therefore, the MPC controller predicts the movement of the target as well, and uses these predicted future poses to calculate what the desired pose $ref^i_k$ will be in every time-step $k$ of the horizon.
% An appropriate cost function, that uses these predictions to calculate an optimal control sequence is $J$ in \eqref{eq:3_cost_function}.

The chasers state vectors are defined as $x^i = [p^i, q^i]^\top$ and the corresponding control action as $u^i = [F^i, \tau^i]^\top$.
The system dynamics are discretized with a sampling time of $dt$ using the forward Euler method to obtain $x^i_{k+1} = \zeta(x^i_k, u^i_k)$. 
The target's state vector is defined as $x^{tar} = [p^{tar}, q^{tar}]$, where $p^{tar}$ is the position and $q^{tar}$ the quaternion attitude represention of the target.
They are descritized in a manner similar to the chaser's states.
The reference poses for the chasers are calculated by transfering the target's body frame using $x^{{off}_i} = [p^{{off}_i}, q^{{off}_i}]$, where $p^{{off}_i}$ is a translation and $q^{{off}_i}$ is a rotation quaternion.
These calculations are performed as such: $p^{{ref}_i}_k = p^{tar}_k + q^{tar}_k \otimes p^{{off}_i} \otimes {q^{tar}_k}^\ast$ and $q^{{ref}_i}_k = q^{tar}_k q^{{off}_i}$.
This discrete model is used as the prediction model of the NMPC.
The prediction is performed over a receding horizon of $D = T/dt$ steps, where $T$ is the horizon duration in seconds.

A cost function is defined such that, when minimized in the current time and the predicted horizon, an optimal set of control actions $u^i_k$ will be calculated.
Let $x_{k+j|k}$ and $x^{tar}_{k+j|k}$ be the predicted chaser and target states at time step $k+j$ respectively, calculated in time step $k$.
The corresponding control actions are $u_{k+j|k}$ and reference states $x^{{ref}_i}_{k+j|k}$.
Also, let $\bm{x}_k$ and $\bm{u}_k$ be the predicted states and control actions for the whole horizon duration calculated at time step $k$.
The cost function is formulated as follows:

\begin{multline} \label{eq:3_cost_function}
    J(\bm{x}_k, \bm{u}_k, u_{k-1|k}) = \sum_{i=0}^{N-1} \{ \sum_{j=0}^{D} \{ \\ 
    \underbrace{(1 - \frac{\alpha \cdot k}{D} )}_\text{Falloff \%} \cdot [\underbrace{{\| p^{{ref}_i}_{k+j|k} - p^i_{k+j|k} \|}^2 Q_p}_\text{Position cost} + \underbrace{{\| q^i_{k+j|k} \otimes {q^{{ref}_i}_{k+j|k}}^\ast \|}^2 Q_q}_\text{Orientation cost}] \\
    + \underbrace{\| u^i_{ref} - u^i_{k+j|k} \|^2 Q_u \}}_\text{Input cost} \\ 
    + \underbrace{{\| p^{{ref}_i}_{k+D|k} - p^i_{k+D|k} \|}^2 Q^f_p + {\| q^i_{k+D|k} \otimes {q^{{ref}_i}_{k+D|k}}^\ast \|}^2 Q^f_q}_\text{Final state cost} \}
\end{multline}

where $Q_p, Q^f_p \in \mathbb{R}^{3\times3}$, $Q_q, Q^f_q \in \mathbb{R}^{4\times4}$ and $Q_u \in \mathbb{R}^{6\times6}$ are positive definite weight matrices for the position and orientation states and inputs respectively. 
In \eqref{eq:3_cost_function}, the first term represents the state cost which penalizes deviation from the reference state at time-step $k$, $x^{{ref}_i}_k$.
The Falloff term $\alpha$, is an adaptive weight to penalize overshoot errors.
% The idea was that error propagation of velocity measurements can accumulate and become proportional to the horizon duration.
The second term represents the input cost which penalizes deviation from the steady-state input $u_{ref} = 0$ that describes constant-velocity movement.
Finally, the final state-cost applies an extra penalty for deviation of the state from the reference at the end of the horizon period.

% where %\footnote{Powers in these equations are element-wise operations}: 
% \begin{itemize}
%     % \item $N =$ The number of chasers
%     % \item $D = T / dt$ The horizon length
%     % \item $J_{p_{i,k}} = Q_p \cdot \left\{ \begin{array}{c} 
%     %     ({ref}_k - p_{i_k})^2 \text{ , for $x$, $\dot{x}$ and $\omega$}\\ 
%     %     (p_{i_k} \otimes {ref}_k^*)^2 \text{ , for $\bm{q}$}
%     % \end{array} \right. $ \\The cost representing the position error of chaser $i$ on time $k$
%     \item $Q_p, Q_{fp} \in \mathbb{R}^{3\times3}$, $Q_q, Q_{fq} \in \mathbb{R}^{4\times4}$ and $Q_u \in \mathbb{R}^{6\times6}$ are positive definite weight matrices for the position and orientation states and inputs respectively.
%     \item $J_{u_{i,k}} = Q_u \cdot u_{i_k}$ The control cost on time $k$ for chaser $i$
%     \item $J_{f_{i}} = Q_f \cdot J_{i_{T}}$ The cost representing the final state error of chaser $i$
%     \item $Q_p, Q_u, Q_f =$ Weight matrices for the cost functions
%     \item ${ref}_k =$ The predicted pose of the \gls{target} at time $k$
%     \item $\alpha$ = Falloff percentage
% \end{itemize}

\subsection{Constraints} \label{ch:chap3:sec4}

% Appropriate optimization constraints must force the controller to respect the hardware limits for the control inputs \eqref{eq:3_constr_max}.
% They can also force the optimizer to limit the plant output to avoid collisions.
A minimum chaser-chaser and chaser-target distance $d_{min}$ is enforced by \eqref{eq:3_constr_dist1} and \eqref{eq:3_constr_dist2}.
\eqref{eq:3_constr_dist3} prevents the chasers from moving too far away from the target by setting their maximum distance to $d_{max}$.
The implemented constraints are the following:
\begin{subequations}
    \begin{equation}\label{eq:3_constr_max}
        u^i_{k+j|k} = [\bm{F}^i_{k+j|k}, \bm{\tau}^i_{k+j|k}] \in [-F_{max}, F_{max}] \times [-\tau_{max}, \tau_{max}]
    \end{equation}
    
    \begin{equation}\label{eq:3_constr_dist1}
        \dist{(p^{tar}_{k+j|k}, p^m_{k+j|k})} \geq d_{min}\; \forall\; j \in [0, D]
    \end{equation}
    
    \begin{equation}\label{eq:3_constr_dist2}
        \dist{(p^m_{k+j|k}, p^n_{k+j|k})} \geq d_{min} \forall\; j \in [0, D],\; m \neq n
    \end{equation}
    
    \begin{equation}\label{eq:3_constr_dist3}
        \dist{(p^{{ref}_m}_{k+j|k}, p^m_{k+j|k})} \leq d_{max}\; \forall\; j \in [0, D]
    \end{equation}
\end{subequations}
where $dist$ is a function of Eucledean distance.

\subsection{Optimizer} \label{ch:chap3:sec5}

% An optimization problem is defined as the calculation of the extrema of an objective function $f(x)$ over a set of real variables $x$.
% The problem might be subject to a set of conditions defined as a system of equalities and inequalities called constraints.
% \eqref{eq:3_optimization} is the mathematic representation of the problem where $m,p \in \mathbb{Z^+}$ and $f, g_i, h_j$ are real functions on $X \subseteq \mathbb{R}^n$.
% \acrfull{nlp} is the process of solving an optimization problem where at least one of $f, g_i, h_j$ is non-linear \cite{math_programming}.

% \begin{equation} \label{eq:3_optimization}
%     \begin{split}
%         \min_{x \in X \subseteq \mathbb{R}^n} \quad & f(x)\\
%         \text{subject to} \quad & g_i(x) \ge 0 \; \forall i \in \{1, \dots, m\}\\
%         & h_j(x) = 0 \; \forall j \in \{1, \dots, p\}
%     \end{split}
% \end{equation}

\acrfull{open} was used as the MPC cost function optimizer.
\acrshort{open} is a framework developed by \cite{opengen} and is based on the PANOC \citep{panoc} optimization algorithm.
PANOC is an extremely fast, the state-of-the-art optimizer for real-time, embedded applications.
A powerful symbolic mathematics library called CasADi \citep{casadi} is used to define the optimization problem and perform under-the-hood operations.
% OpEn runs in Rust, a programming language that is ideal for embeded, real-time applications and can be interfaced for use with many high level programming languages.
% These make it the state-of-the-art option for real-time applications such as robotics, autonomous vehicles and \acrshort{uav}s where running an optimizer is necessary to close a control loop in every time-step.

% In contrast to other popular libraries, \acrshort{open} uses an algorithm called PANOC.
% PANOC is an algorithm introduced by \cite{panoc} that involves only very simple iterations making it extremely fast.
% It is fundamentally different from other popular, iterative algorithms as it implements a method called proximal averaged Newton-type method.
% PANOC is capable of solving \acrshort{nlp} problems in the form described by \eqref{eq:3_nlp_panoc}.
% (As described in the library's \href{https://alphaville.github.io/optimization-engine/docs/open-intro}{documentation page}).
% \begin{equation}\label{eq:3_nlp_panoc}
%     \begin{split}
%         \mathbb{P}(p) \; \text{:} \; \min_{u \in \mathbb{R}^{n_u}} \quad & f(u, p)\\
%         \text{subject to} \quad & u \; \in \; U\\
%         & F_1(u, p) \; \in \; C\\
%         & F_2(u, p) \; = \; 0
%     \end{split}
% \end{equation}
% Where $u \in \mathbb{R}^{n_u}$ is the vector decision variables of the problem, $p \in \mathbb{R}^{n_p}$ is the parameter vector and $F_1$ and $F_2$ describe two types of constraints handled with different methods in the algorithm.

\section{Simulation Results} \label{ch:chap4}
The following section presents the results of the simulations that were performed to test the control architecture.
First of all, in Section \ref{ch:chap4:sec1} different parameters of the MPC controller are tested and compared in a single chaser scenario.
Then, in Section \ref{ch:chap4:sec2} multi chaser scenarios are performed and the docking sequence is tested.
Finally, in Section \ref{ch:chap4:sec3} a clear demonstration of the collision avoidance capabilities are presented.
In every scenario the velocities of the target are $v=[1.50, 0.75, 3.00]\times10^{-2}\text{m/s}$ and $\omega=[1.50, 4.50, 3.00]\times10^{-2}\text{rad/s}$ and the simulations were recorded over 2 minutes.
A video demonstration of these simulations can be found at \textcolor{blue}{youtu.be/86JT4M7eyGc}.

\subsection{Single Chaser \& MPC Tuning} \label{ch:chap4:sec1}

The tuneable parameters of a MPC controller are the cost function parameters described in \eqref{eq:3_cost_function}.
The error terms of each state variable, scaled by their weight $Q_p$, need to be close to each other to avoid some to be prioritized over others.
For example in cases where the chasers start far away from the target, the position error will be degrees of magnitude higher than the quaternion error which is bounded in $[-1, 1]$.
% In this case the chaser begins near the target drifts only a couple of meters away from the $ref$ pose. 
The $Q_u$ weights penalize the use of the control inputs and can be tuned by trail and error to avoid over-actuation of the chaser.
Finally the $Q_f$ weights penalize the final state error to avoid overshooting the target and must be tuned higher than $Q_p$.

\begin{table}[h]
    \centering
    \begin{tabular}{|c|c|c|c|}
        \hline
         & $Q_p$ & $Q_u$ & $Q_f$ \\
        \hline
         Position/Force & $65$ & $3.5$ & $3250$\\
        \hline
         Orientation/Torque & $35$ & $40$ & $1750$\\
        \hline
    \end{tabular}
    \vspace*{.15cm}
    \caption{MPC Tuning Weights}
    \label{tab:weights}
\end{table}
\vspace*{-.25cm}

The values of these weights are presented in Table \ref{tab:weights} and were determined after repetitive testing.
The rest of the cost function parameters have more interesting effects on the performance of the controller and thus are further analyzed in the following subsections.
As a baseline for comparison, Figure \ref{fig:baseline_1p} shows the results of the MPC controller with $T=3\text{s}$, $dt=0.1\text{s}$ and $\alpha=0\%$.
 % $T=\text{\qty{3}{\second}}$, $dt=\text{\qty{0.1}{\second}}$ and $\alpha=0\%$.
Every test was performed and recorded three times for more accurate results.
Table \ref{tab:1p_tests}  presents the results of the simulations.

\begin{figure}
    \centering
    \resizebox{.49\textwidth}{!}{
        \input{images/1p_10_3_0.pgf}
    }
    \caption{Baseline runs with 1 chaser, $T=3\text{s}$, $dt=0.1\text{s}$ and $\alpha=0\%$. \textbf{Top:} Chaser and target trajectories, \textbf{Bottom:} Pose errors.}
    % \caption{Baseline runs with 1 chaser, $T=\text{\qty{3}{\second}}$, $dt=\text{\qty{0.1}{\second}}$ and $\alpha=0\%$. \textbf{Top:} Chaser and target trajectories, \textbf{Bottom:} Pose errors.}
    \label{fig:baseline_1p}
\end{figure}

\begin{table}[h]
    \centering
    \begin{tabular}{|m{1.5cm}|c|c|c|c|c|}
        \hline
        $\bm{Comment}$ & $\bm{T}$(s) & $\bm{dt}$(s) & $\bm{\alpha}$(\%) & \thead{Position\footnotemark \\ $\bm{MSE}$} & \thead{Orientation \\ $\bm{MSE}$}\\ 
        \hline
        Baseline & $3$ & $1/10$ & $0\%$ & $1.64$ & $3.94$ \\
        \hline
        \multirow{2}{*}{$\bm{T}$ test}
            & $0.5$  & $1/10$ & $0\%$ & $64.8$\footnotemark & $229$ \\
            \cline{2-6}
            & $10$  & $1/10$ & $0\%$ & $126$\footnotemark & $943$ \\
        \hline
        \multirow{2}{*}{$\bm{dt}$ test}
            & $3$  & $1/5$ & $0\%$ & $1.73$ & $4.87$ \\
            \cline{2-6}
            & $3$  & $1/30$ & $0\%$ & $1.42$ & $3.89$ \\
        \hline
        \multirow{3}{*}{$\bm{\alpha}$ test}
            & $3$  & $1/10$ & $10\%$ & $1.65$ & $4.09$\\
            \cline{2-6}
            & $3$  & $1/10$ & $30\%$ & $1.69$ & $4.07$ \\
            \cline{2-6}
            & $3$  & $1/10$ & $60\%$ & $1.75$ & $4.25$ \\
            \cline{2-6}
            & $3$  & $1/30$ & $30\%$ & $1.59$ & $4.02$ \\
        \hline
    \end{tabular}
    \vspace*{.15cm}
    \caption{Single Chaser Tests. Position MSE: $m^2 \times 10^{-2}$, Orientation MSE: ${rad}^2 \times 10^{-3}$}
    \label{tab:1p_tests}
\end{table}
\vspace*{-.25cm}
\footnotetext[2]{Average over 3 runs.}
\footnotetext[3]{1/3 runs failed.}
\footnotetext[4]{2/3 runs failed.}

\subsubsection{Horizon Period:} \label{ch:chap4:sec1:sub1}
$T$ is the time in seconds that the controller will predict the future of the system.
It is clear that $T$ is the most important parameter for tuning the MPC controller.
Having too small a horizon allows the controller to predict only the short term results of the plant's input and too large a horizon will make the controller too slow.
Both tests with $T=0.5\text{s}$ and $T=10\text{s}$ had failed runs.

\subsubsection{Prediction Time Step:} \label{ch:chap4:sec1:sub2}
$dt$ is the sampling for the horizon period. 
Smaller $dt$ values will make the controller more accurate since it improves the performance of the forward Euler method. 
Decreasing $dt$ also adds more steps to the MPC and thus more variables in the optimization problem, increasing the calculation time for a solution.

\subsubsection{Falloff percentage:} \label{ch:chap4:sec1:sub3}
The experimental term $\alpha$ is tested for performance improvement.
The test results show that $\alpha$ actually increases the MSE of every scenario and thus is not used in the final controller.

% \begin{figure}
%     \centering
%     \resizebox{.49\textwidth}{!}{
%         \input{images/alpha_tests.pgf}
%     }
%     \caption{Results with 1 chaser, $T=\text{\qty{3}{\second}}$, $dt=\text{\qty{0.1}{\second}}$ and variable $\alpha$.}
%     \label{fig:alpha_tests}
% \end{figure}

\subsection{Multi Chaser} \label{ch:chap4:sec2}

For multi chaser scenarios, two tests were performed.
The first test is a simple tracking scenario where two chasers communicate their estimations of the target's pose to each other to maintain a pre-defined pose relative to the target.
In Figure \ref{fig:2p_test}, a spike in the MSE plot is observed at around 100s, which is caused by a Euler angle singularity introduced by the PnP solver of OpenCV.
This singularity can be interpreted as a random interference in the system which the controller was able to handle, hence showcasing it's robustness.

\begin{figure}
    \centering
    \resizebox{.49\textwidth}{!}{
        \input{images/2p_10_3_60.pgf}
    }
    \caption{Two Chaser tracking mission. \textbf{Top:} Chaser and target trajectories, \textbf{Bottom left:} Chaser position error, \textbf{Bottom right:} Chaser orientation error}
    \label{fig:2p_test}
\end{figure}

The second test is a more complex scenario where four chasers initially track the target.
After about 60s Chaser 0 and 3 receive an approach command and start docking the target.
Their relative pose is maintained throughout the docking process but their position is gradually scaled to decrease their distance to the target.
Figure \ref{fig:4p_test} showcases the distance each chaser maintains from the target.
Of course the pose error increases in the docking process since the chaser's position is compared to their initial $ref$ pose and not the docking pose.

\begin{figure}
    \centering
    \resizebox{.49\textwidth}{!}{
        \input{images/4p_dock.pgf}
    }
    \caption{Four Chaser tracking mission with Chaser 0 and 3 docking after 60 seconds. \textbf{Top: } Chasers pose error, \textbf{Bottom: } Docking evaluation}
    \label{fig:4p_test}
\end{figure}

\subsection{Collision Avoidance} \label{ch:chap4:sec3}

To showcase the collision avoidance capabilities of the controller, a simple scenario was created where two chasers are initially just tracking a stationary target.
A command is given to Chaser 0 to change it's $ref$ pose to a position on the opposite side of the target.
The closest route to that pose would be a direct line through the target.
However, the controller is able to find a path over the target, respecting a limit of $d_{min}=0.35$m.
Chaser 1 maintains it's initial $ref$ pose relative to the static target, observing the scene from afar.
This is crucial since Chaser 0 will inevitably loose sight of the target and be unable to estimate it's pose alone.
The pose estimation from Chaser 1 is used by Chaser 0 during the maneuver, showcasing one of the benefits of the multi chaser approach.

\begin{figure}
    \centering
    \resizebox{.49\textwidth}{!}{
        \input{images/2p_collision.pgf}
    }
    \caption{Two Chaser, collision avoidance mission. Chaser 0 maneuvers over the target while Chaser 1 assists with visual pose estimation. \textbf{Top: } Chaser trajectories and Target pose,
    \textbf{Bottom: } Collision avoidance evaluation}
    \label{fig:collision_avoidance}
\end{figure}

% \section{Experimental Results} \label{ch:chap5}
% After successfully testing the controller setup and the designed processing flow in the simulation environment, it was decided to test the architecture in a realistic environment.
The setup that was available for experimentation was sufficient to only experiment with one agent.
The robot that was used, is capable of 3-\acrshort{dof} motion (2D movement and rotation).
% Details about the used robotic platform can be found on Section \ref{ch:chap5:sec1:sub1}.
The controller explained in Section \ref{ch:chap3} is modified for movement in two dimensions.
% Details about the modified controller can be found on Section \ref{ch:chap5:sec1:sub2}.
Finally, since the scenario assumes that the chaser odometry is known, the laboratory's visual positioning system was used.

\subsection{The Slider}\label{ch:chap5:sec1:sub1}

The robotic platform that was used for the experiment is designed by the Robotics \& AI group of Luleå University of Technology and is called the Slider \citep{slider}.
The Slider is a 2D floating, satellite platform designed for in-lab experimentation simulating a satellite's motion in space.
The platform is designed to smoothly maneuver on a frictionless surface such as an epoxy coated table or, in the particular case of this experiment, a large piece of glass pane.
The glass pane that was used as a sliding table was $1\text{m}^2$ in size.

\begin{figure}[h]
    \includegraphics[width=.45\textwidth]{slider_top.png}
    \centering
    \caption{Render of the Slider's top view.}
    \label{fig:5_slider_top}
\end{figure}

% The platform relies on a pneumatic system consisting of a pressure tank, air-bearings and thrusters.
% The setup used mains pressured air to supply the pneumatic system Which introduced limitations and disturbances to the system.
% The flexible pipes connected to the robot to supply it with pressure applied forces, pulling it when it moved towards the edges of the sliding table.

% \begin{figure}[h]
%     \includegraphics[width=1.0\textwidth]{slider_blueprint.pdf}
%     \centering
%     \caption{Complete blueprint of the Slider platform}
%     \label{fig:5_slider_blueprint}
% \end{figure}

% Frictionless movement is achieved by using three air-bearings at the bottom of the platform.
% Eight thrusters are placed around the platform according to 
% %Fig. \ref{fig:5_slider_blueprint} and 
% Table \ref{tab:5_thruster_pos} that create a force of \qty{.7}{\newton} each.
% Due to their position, each thruster also produces $0.7 \cdot 140 \times 10^{-3}=\text{\qty{98e-3}{\newton\meter}}$ of torque on either direction depending on their orientation.
% Table \ref{tab:5_slider_tsl} explains which thrusters need to be powered to move the platform in standard directions.
% From this table, it is apparent that the maximum force that can be applied in each direction is $2 \cdot 0.7 = \text{\qty{1.4}{\newton}}$ and the maximum torque is $4 \cdot 98 \times 10^{-3} = \text{\qty{392e-3}{\newton\meter}}$.
% Table \ref{tab:5_slider_parameters} contains the rest of the necessary parameters of the Slider platform.

% \begin{table}[h]
%     \centering
%     \begin{tabular}{|c||p{.35cm}|p{.35cm}|p{.35cm}|p{.35cm}|p{.35cm}|p{.35cm}|p{.35cm}|p{.35cm}|}
%         \hline
%         $\bm{Direction}$ & \multicolumn{8}{|c|}{$\bm{Thrusters}$} \\ 
%         \hline
%         & $T_{11}$ & $T_{12}$ & $T_{21}$ & $T_{22}$ & $T_{31}$ & $T_{32}$ & $T_{41}$ & $T_{42}$ \\
%         \hline
%         Forward & & & \checkmark & & \checkmark & & &\\
%         \hline
%         Backward & \checkmark & & & & & & \checkmark &\\
%         \hline
%         Left & & \checkmark & & \checkmark & & & &\\
%         \hline
%         Right & & & & & & \checkmark & & \checkmark\\
%         \hline
%         Clockwise & \checkmark & & & \checkmark & \checkmark & & & \checkmark\\
%         \hline
%         C-Clockwise & & \checkmark & \checkmark & & & \checkmark & \checkmark &\\
%         \hline
%     \end{tabular}
%     \vspace*{.15cm}
%     \caption{Thruster activation logic}
%     \label{tab:5_slider_tsl}
% \end{table}
% \vspace*{-.25cm}

% \begin{table}[h]
%     \centering
%     \begin{tabular}{|c||c|c|}
%         \hline
%         $\bm{Thruster}$ & $\bm{Position}$ (\unit{\mm}) & $\bm{Orientation}$ (\unit{\deg}) \\
%         \hline
%         $\mathbf{T_{11}}$ & $(195, -140)$ & $0$ \\
%         \hline
%         $\mathbf{T_{12}}$ & $(140, -195)$ & $270$ \\
%         \hline
%         $\mathbf{T_{21}}$ & $(-195, -140)$ & $180$ \\
%         \hline
%         $\mathbf{T_{22}}$ & $(-140, -195)$ & $270$ \\
%         \hline
%         $\mathbf{T_{31}}$ & $(-195, 140)$ & $180$ \\
%         \hline
%         $\mathbf{T_{32}}$ & $(-140, 195)$ & $90$ \\
%         \hline
%         $\mathbf{T_{41}}$ & $(195, 140)$ & $0$ \\
%         \hline
%         $\mathbf{T_{42}}$ & $(195, 195)$ & $90$ \\
%         \hline
        
%     \end{tabular}
%     \vspace*{.15cm}
%     \caption{Thruster Position \& Orientation}
%     \label{tab:5_thruster_pos}
% \end{table}
% \vspace*{-.25cm}

The platform was equiped with a webcam to provide a visual feedback pointing towards the front of the robot.
Special, reflective markers were also mounted on the robot to enable odometry from the lab's visual positioning system.
Table \ref{tab:5_slider_parameters} contains some of Slider's hardware specifications.

\begin{table}[h]
    \centering
    \begin{tabular}{|c|c|}
        \hline
        $\bm{Parameters}$ & $\bm{Values}$ \\ 
        \hline
        Mass ($m$) & \qty{4.436}{\kg} \\ 
        \hline
        Moment of Inertia ($I_{zz}$) & \qty{1.092}{\kg\meter\squared} \\ 
        \hline
        Minimum thruster on-time & \qty{10}{\ms} \\ 
        \hline
        Maximum force ($F_{max}$) & \qty{1.4}{\newton} \\ 
        \hline
        Maximum torque ($\tau_{max}$) & \qty{0.392}{\newton\meter} \\ 
        \hline
    \end{tabular}
    \vspace*{.15cm}
    \caption{Slider system parameters}
    \label{tab:5_slider_parameters}
\end{table}
\vspace*{-.25cm}

The processing flow has to be split between the on-board computer and the central computer and Fig. \ref{fig:1_chaser_flow} needs to be updated.
Fig. \ref{fig:slider_flow} explains which parts of the processing flow happen on the on-board computer and which are on the central computer.
The estimation combiner is now only used for filtering since only one chaser is operating.

\insertfig{slider_flow}{Updated processing flow for Slider}

\subsection{Controller Setup}\label{ch:chap5:sec1:sub2}

Since the Slider is a 2D platform, the controller described in Section \ref{ch:chap3} needs to be modified.
To begin with, the system state now does not include position on z-axis or rotation on x and y-axis: $x=[x, y, q_w, q_x, q_y, q_z, \dot{x}, \dot{y}, \omega]$
Unfortunately, all four quaternion parts are needed to describe the rotation.
Next, Equation \ref{eq:3_target_dynamics} and \ref{eq:3_chaser_dynamics} can be used as is but with $z = v_z = \omega_z = 0$ set constant.

The input vector is updated to only contain two forces and a torque: $u = [F_x, F_y, \tau]$.
In \eqref{eq:3_chaser_dynamics}, $F_z = \tau_x = \tau_y = 0$ are set constant to simulate the uncontrollability their corresponding states.
The cost function in \eqref{eq:3_cost_function} can be used as is with the parts involving uncontrollable states ignored.
Finally, constraints \eqref{eq:3_constr_dist1} and \eqref{eq:3_constr_dist2} are removed.
The system has little space for maneuvers and, in order to test the controller, Slider must get really close to the target which will interfere with the collision avoidance constraints.

\subsection{Thruster Selection Logic}\label{ch:chap5:sec1:sub3}

Slider is an over-actuated system, since it uses 8 thrusters and is able to perform 6 distinct movements (3DoF + opposite).
This requires a complex thruster selection logic to determine which thrusters to activate to produce the requested force and torque.
% First of all, Equation \ref{eq:5_thruster_mapping} explains the relation between $F_x$, $F_y$ and $\tau$ and the force applied by each thruster.
% \begin{subequations}\label{eq:5_thruster_mapping}
%     \begin{equation}
%         Ax=b \Leftrightarrow [A_{11} \cdots A_{42}]x=b
%     \end{equation}
%     \begin{equation}
%         A_{AB} = \begin{bmatrix}
%             \cos{(\beta_{AB})} \\ \sin{(\beta_{AB})} \\ (r_{T_{AB}}^y \cos{(\beta_{AB})} - r_{T_{AB}}^x \cos{(\beta_{AB})})
%         \end{bmatrix}
%     \end{equation}
%     \begin{equation}
%         x = \begin{bmatrix}
%             T_{11} & T_{12} & T_{21} & T_{22} & T_{31} & T_{32} & T_{41} & T_{42}
%         \end{bmatrix}^T
%     \end{equation}
%     \begin{equation}
%         b = \begin{bmatrix}
%             F_x &
%             F_y &
%             \tau
%         \end{bmatrix}^T
%     \end{equation}
% \end{subequations}
% Where $r_{T_{AB}}^x$, $r_{T_{AB}}^y$ and $\beta_{AB}$ are the position and orientation of the thrusters that can be found in Table \ref{tab:5_thruster_pos}.
The problem of selecting the thrusters to activate for a given force-torque command is reduced to an optimization problem as described in the original Slider paper.
% \begin{equation}
%     \begin{split}
%         \min_x \quad & J = x^T \cdot x \\
%         \text{s.t.} \quad & Ax=b\\
%         \quad & 0 \le x \le x_{max}
%     \end{split}
% \end{equation}
The solution to the optimization problem is the optimal thruster output that provides the necessary force and torque output while minimizing the actuation effort and respecting the hardware limitations.
The output of the TSL is then passed through a PWM generator.

\subsection{Results}
Two scenarios were set up in the laboratory environment to evaluate the performance of the controller.
First, a target is placed down and held stationary near the Slider.
The slider is then allowed to track it and after a while the docking command is given.
Second, a target is initially moved by hand and the Slider is allowed to track it for a while. 
Then it is placed down and held stationary before the docking command is given.
Both scenarios are tested twice: Firstly using the ground truth odometry from the lab's positioning system to evaluate the controller with ideal pose estimation and then using the ArUco based estimation.

\begin{figure}
    \centering
    \resizebox{.49\textwidth}{!}{
        \input{images/dock_experiment_4.pgf}
    }
    \caption{Stationary target scenario using ground truth odometry. Docking begins at $t=\text{\qty{20}{\second}}$. \textbf{Top: } Slider and target trajectories \textbf{Middle: } Pose error \textbf{Bottom: } Docking evaluation}
    \label{fig:sl_dock_4}
\end{figure}

It should be emphasized that this setup is very susceptible to external disturbances.
In addition, there were other factors that negatively impacted the test results.
Most notably:
\begin{itemize}
    \item The sliding table is affected by the slope of the floor and the wooden board that supports it.
    \item The tubes used to provide the Slider with pressurised air, pull the platform towards the center of the sliding table.
    \item The lab's positioning system is not 100\% accurate.
    \item The controller is designed to track a target moving with constant velocity. Tracking a stationary target makes the process susceptible to noise.
    \item There is not enough room for a second chaser, or for complex maneuvers allowing the tracking of a constantly moving target.
\end{itemize}
Previous experimentations on the platform suggest that a \qty{5}{\centi\meter} deviation from setpoints is typical and expected.
With this in mind, we conclude that in all scenarios the chaser successfully docks the target at a docking distance of \qty{20}{\centi\meter} within the expected range of error.

\begin{figure}
    \centering
    \resizebox{.49\textwidth}{!}{
        \input{images/dock_experiment_5.pgf}
    }
    \caption{Initially moving target scenario using ground truth odometry. Docking begins at $t=\text{\qty{85}{\second}}$. \textbf{Top: } Slider and target trajectories \textbf{Middle: } Pose error \textbf{Bottom: } Docking evaluation}
    \label{fig:sl_dock_5}
\end{figure}

In Fig. \ref{fig:sl_dock_4} and \ref{fig:sl_dock_5} spikes in the error are observed in $t=\text{\qty{32}{\second}}$ and $t=\text{\qty{64}{\second}}$ respectively.
These are typical errors of the laboratory's visual positioning system that occur when the software confuses one of the reflective markers on the Slider and miscalculates it's orientation.
In the scenario of Fig. \ref{fig:sl_dock_4} the disturbance occurs during the docking phase while in the scenario of Fig. \ref{fig:sl_dock_5} it occurs during the tracking phase.
In both cases the controller is able to handle the disturbance and recover back to normal operation.

\begin{figure}
    \centering
    \resizebox{.49\textwidth}{!}{
        \input{images/dock_experiment_6.pgf}
    }
    \caption{Stationary target scenario using ArUco pose estimator. Docking begins at $t=\text{\qty{32}{\second}}$. \textbf{Top: } Slider and target trajectories \textbf{Middle: } Pose error \textbf{Bottom: } Docking evaluation}
    \label{fig:sl_dock_6}
\end{figure}

\begin{figure}
    \centering
    \resizebox{.49\textwidth}{!}{
        \input{images/dock_experiment_8.pgf}
    }
    \caption{Initially moving target scenario using ArUco pose estimator. Docking begins at $t=\text{\qty{30}{\second}}$. \textbf{Top: } Slider and target trajectories \textbf{Middle: } Pose error \textbf{Bottom: } Docking evaluation}
    \label{fig:sl_dock_8}
\end{figure}

\section{Conclusion}
In this article we propose a reliable centralized NMPC-based control architecture that can assist in docking of multiple chaser spacecraft to a tumbling target in Space. The tumbling target pose is estimated using a single monocular camera on-board each chaser spacecraft. The estimated pose is shared between the chaser spacecraft to improve the fused pose estimate of the target when the target moves of the field of view of one of the chaser spacecraft. Multiple simulation scenarios were tested to determine the efficacy of the proposed controller.  

% \begin{ack}
% Place acknowledgments here. 
% \end{ack}

\bibliography{ifacconf}

% \appendix
% \section{A summary of Latin grammar}    % Each appendix must have a short title.
% \section{Some Latin vocabulary}              % Sections and subsections are supported  
                                                                         % in the appendices.
\end{document}
