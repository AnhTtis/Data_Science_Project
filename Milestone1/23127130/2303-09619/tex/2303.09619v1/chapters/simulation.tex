The following section presents the results of the simulations that were performed to test the control architecture.
First of all, in Section \ref{ch:chap4:sec1} different parameters of the MPC controller are tested and compared in a single chaser scenario.
Then, in Section \ref{ch:chap4:sec2} multi chaser scenarios are performed and the docking sequence is tested.
Finally, in Section \ref{ch:chap4:sec3} a clear demonstration of the collision avoidance capabilities are presented.
In every scenario the velocities of the target are $v=[1.50, 0.75, 3.00]\times10^{-2}\text{m/s}$ and $\omega=[1.50, 4.50, 3.00]\times10^{-2}\text{rad/s}$ and the simulations were recorded over 2 minutes.
A video demonstration of these simulations can be found at \textcolor{blue}{youtu.be/86JT4M7eyGc}.

\subsection{Single Chaser \& MPC Tuning} \label{ch:chap4:sec1}

The tuneable parameters of a MPC controller are the cost function parameters described in \eqref{eq:3_cost_function}.
The error terms of each state variable, scaled by their weight $Q_p$, need to be close to each other to avoid some to be prioritized over others.
For example in cases where the chasers start far away from the target, the position error will be degrees of magnitude higher than the quaternion error which is bounded in $[-1, 1]$.
% In this case the chaser begins near the target drifts only a couple of meters away from the $ref$ pose. 
The $Q_u$ weights penalize the use of the control inputs and can be tuned by trail and error to avoid over-actuation of the chaser.
Finally the $Q_f$ weights penalize the final state error to avoid overshooting the target and must be tuned higher than $Q_p$.

\begin{table}[h]
    \centering
    \begin{tabular}{|c|c|c|c|}
        \hline
         & $Q_p$ & $Q_u$ & $Q_f$ \\
        \hline
         Position/Force & $65$ & $3.5$ & $3250$\\
        \hline
         Orientation/Torque & $35$ & $40$ & $1750$\\
        \hline
    \end{tabular}
    \vspace*{.15cm}
    \caption{MPC Tuning Weights}
    \label{tab:weights}
\end{table}
\vspace*{-.25cm}

The values of these weights are presented in Table \ref{tab:weights} and were determined after repetitive testing.
The rest of the cost function parameters have more interesting effects on the performance of the controller and thus are further analyzed in the following subsections.
As a baseline for comparison, Figure \ref{fig:baseline_1p} shows the results of the MPC controller with $T=3\text{s}$, $dt=0.1\text{s}$ and $\alpha=0\%$.
 % $T=\text{\qty{3}{\second}}$, $dt=\text{\qty{0.1}{\second}}$ and $\alpha=0\%$.
Every test was performed and recorded three times for more accurate results.
Table \ref{tab:1p_tests}  presents the results of the simulations.

\begin{figure}
    \centering
    \resizebox{.49\textwidth}{!}{
        \input{images/1p_10_3_0.pgf}
    }
    \caption{Baseline runs with 1 chaser, $T=3\text{s}$, $dt=0.1\text{s}$ and $\alpha=0\%$. \textbf{Top:} Chaser and target trajectories, \textbf{Bottom:} Pose errors.}
    % \caption{Baseline runs with 1 chaser, $T=\text{\qty{3}{\second}}$, $dt=\text{\qty{0.1}{\second}}$ and $\alpha=0\%$. \textbf{Top:} Chaser and target trajectories, \textbf{Bottom:} Pose errors.}
    \label{fig:baseline_1p}
\end{figure}

\begin{table}[h]
    \centering
    \begin{tabular}{|m{1.5cm}|c|c|c|c|c|}
        \hline
        $\bm{Comment}$ & $\bm{T}$(s) & $\bm{dt}$(s) & $\bm{\alpha}$(\%) & \thead{Position\footnotemark \\ $\bm{MSE}$} & \thead{Orientation \\ $\bm{MSE}$}\\ 
        \hline
        Baseline & $3$ & $1/10$ & $0\%$ & $1.64$ & $3.94$ \\
        \hline
        \multirow{2}{*}{$\bm{T}$ test}
            & $0.5$  & $1/10$ & $0\%$ & $64.8$\footnotemark & $229$ \\
            \cline{2-6}
            & $10$  & $1/10$ & $0\%$ & $126$\footnotemark & $943$ \\
        \hline
        \multirow{2}{*}{$\bm{dt}$ test}
            & $3$  & $1/5$ & $0\%$ & $1.73$ & $4.87$ \\
            \cline{2-6}
            & $3$  & $1/30$ & $0\%$ & $1.42$ & $3.89$ \\
        \hline
        \multirow{3}{*}{$\bm{\alpha}$ test}
            & $3$  & $1/10$ & $10\%$ & $1.65$ & $4.09$\\
            \cline{2-6}
            & $3$  & $1/10$ & $30\%$ & $1.69$ & $4.07$ \\
            \cline{2-6}
            & $3$  & $1/10$ & $60\%$ & $1.75$ & $4.25$ \\
            \cline{2-6}
            & $3$  & $1/30$ & $30\%$ & $1.59$ & $4.02$ \\
        \hline
    \end{tabular}
    \vspace*{.15cm}
    \caption{Single Chaser Tests. Position MSE: $m^2 \times 10^{-2}$, Orientation MSE: ${rad}^2 \times 10^{-3}$}
    \label{tab:1p_tests}
\end{table}
\vspace*{-.25cm}
\footnotetext[2]{Average over 3 runs.}
\footnotetext[3]{1/3 runs failed.}
\footnotetext[4]{2/3 runs failed.}

\subsubsection{Horizon Period:} \label{ch:chap4:sec1:sub1}
$T$ is the time in seconds that the controller will predict the future of the system.
It is clear that $T$ is the most important parameter for tuning the MPC controller.
Having too small a horizon allows the controller to predict only the short term results of the plant's input and too large a horizon will make the controller too slow.
Both tests with $T=0.5\text{s}$ and $T=10\text{s}$ had failed runs.

\subsubsection{Prediction Time Step:} \label{ch:chap4:sec1:sub2}
$dt$ is the sampling for the horizon period. 
Smaller $dt$ values will make the controller more accurate since it improves the performance of the forward Euler method. 
Decreasing $dt$ also adds more steps to the MPC and thus more variables in the optimization problem, increasing the calculation time for a solution.

\subsubsection{Falloff percentage:} \label{ch:chap4:sec1:sub3}
The experimental term $\alpha$ is tested for performance improvement.
The test results show that $\alpha$ actually increases the MSE of every scenario and thus is not used in the final controller.

% \begin{figure}
%     \centering
%     \resizebox{.49\textwidth}{!}{
%         \input{images/alpha_tests.pgf}
%     }
%     \caption{Results with 1 chaser, $T=\text{\qty{3}{\second}}$, $dt=\text{\qty{0.1}{\second}}$ and variable $\alpha$.}
%     \label{fig:alpha_tests}
% \end{figure}

\subsection{Multi Chaser} \label{ch:chap4:sec2}

For multi chaser scenarios, two tests were performed.
The first test is a simple tracking scenario where two chasers communicate their estimations of the target's pose to each other to maintain a pre-defined pose relative to the target.
In Figure \ref{fig:2p_test}, a spike in the MSE plot is observed at around 100s, which is caused by a Euler angle singularity introduced by the PnP solver of OpenCV.
This singularity can be interpreted as a random interference in the system which the controller was able to handle, hence showcasing it's robustness.

\begin{figure}
    \centering
    \resizebox{.49\textwidth}{!}{
        \input{images/2p_10_3_60.pgf}
    }
    \caption{Two Chaser tracking mission. \textbf{Top:} Chaser and target trajectories, \textbf{Bottom left:} Chaser position error, \textbf{Bottom right:} Chaser orientation error}
    \label{fig:2p_test}
\end{figure}

The second test is a more complex scenario where four chasers initially track the target.
After about 60s Chaser 0 and 3 receive an approach command and start docking the target.
Their relative pose is maintained throughout the docking process but their position is gradually scaled to decrease their distance to the target.
Figure \ref{fig:4p_test} showcases the distance each chaser maintains from the target.
Of course the pose error increases in the docking process since the chaser's position is compared to their initial $ref$ pose and not the docking pose.

\begin{figure}
    \centering
    \resizebox{.49\textwidth}{!}{
        \input{images/4p_dock.pgf}
    }
    \caption{Four Chaser tracking mission with Chaser 0 and 3 docking after 60 seconds. \textbf{Top: } Chasers pose error, \textbf{Bottom: } Docking evaluation}
    \label{fig:4p_test}
\end{figure}

\subsection{Collision Avoidance} \label{ch:chap4:sec3}

To showcase the collision avoidance capabilities of the controller, a simple scenario was created where two chasers are initially just tracking a stationary target.
A command is given to Chaser 0 to change it's $ref$ pose to a position on the opposite side of the target.
The closest route to that pose would be a direct line through the target.
However, the controller is able to find a path over the target, respecting a limit of $d_{min}=0.35$m.
Chaser 1 maintains it's initial $ref$ pose relative to the static target, observing the scene from afar.
This is crucial since Chaser 0 will inevitably loose sight of the target and be unable to estimate it's pose alone.
The pose estimation from Chaser 1 is used by Chaser 0 during the maneuver, showcasing one of the benefits of the multi chaser approach.

\begin{figure}
    \centering
    \resizebox{.49\textwidth}{!}{
        \input{images/2p_collision.pgf}
    }
    \caption{Two Chaser, collision avoidance mission. Chaser 0 maneuvers over the target while Chaser 1 assists with visual pose estimation. \textbf{Top: } Chaser trajectories and Target pose,
    \textbf{Bottom: } Collision avoidance evaluation}
    \label{fig:collision_avoidance}
\end{figure}