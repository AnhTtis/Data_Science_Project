After successfully testing the controller setup and the designed processing flow in the simulation environment, it was decided to test the architecture in a realistic environment.
The setup that was available for experimentation was sufficient to only experiment with one agent.
The robot that was used, is capable of 3-\acrshort{dof} motion (2D movement and rotation).
% Details about the used robotic platform can be found on Section \ref{ch:chap5:sec1:sub1}.
The controller explained in Section \ref{ch:chap3} is modified for movement in two dimensions.
% Details about the modified controller can be found on Section \ref{ch:chap5:sec1:sub2}.
Finally, since the scenario assumes that the chaser odometry is known, the laboratory's visual positioning system was used.

\subsection{The Slider}\label{ch:chap5:sec1:sub1}

The robotic platform that was used for the experiment is designed by the Robotics \& AI group of Luleå University of Technology and is called the Slider \citep{slider}.
The Slider is a 2D floating, satellite platform designed for in-lab experimentation simulating a satellite's motion in space.
The platform is designed to smoothly maneuver on a frictionless surface such as an epoxy coated table or, in the particular case of this experiment, a large piece of glass pane.
The glass pane that was used as a sliding table was $1\text{m}^2$ in size.

\begin{figure}[h]
    \includegraphics[width=.45\textwidth]{slider_top.png}
    \centering
    \caption{Render of the Slider's top view.}
    \label{fig:5_slider_top}
\end{figure}

% The platform relies on a pneumatic system consisting of a pressure tank, air-bearings and thrusters.
% The setup used mains pressured air to supply the pneumatic system Which introduced limitations and disturbances to the system.
% The flexible pipes connected to the robot to supply it with pressure applied forces, pulling it when it moved towards the edges of the sliding table.

% \begin{figure}[h]
%     \includegraphics[width=1.0\textwidth]{slider_blueprint.pdf}
%     \centering
%     \caption{Complete blueprint of the Slider platform}
%     \label{fig:5_slider_blueprint}
% \end{figure}

% Frictionless movement is achieved by using three air-bearings at the bottom of the platform.
% Eight thrusters are placed around the platform according to 
% %Fig. \ref{fig:5_slider_blueprint} and 
% Table \ref{tab:5_thruster_pos} that create a force of \qty{.7}{\newton} each.
% Due to their position, each thruster also produces $0.7 \cdot 140 \times 10^{-3}=\text{\qty{98e-3}{\newton\meter}}$ of torque on either direction depending on their orientation.
% Table \ref{tab:5_slider_tsl} explains which thrusters need to be powered to move the platform in standard directions.
% From this table, it is apparent that the maximum force that can be applied in each direction is $2 \cdot 0.7 = \text{\qty{1.4}{\newton}}$ and the maximum torque is $4 \cdot 98 \times 10^{-3} = \text{\qty{392e-3}{\newton\meter}}$.
% Table \ref{tab:5_slider_parameters} contains the rest of the necessary parameters of the Slider platform.

% \begin{table}[h]
%     \centering
%     \begin{tabular}{|c||p{.35cm}|p{.35cm}|p{.35cm}|p{.35cm}|p{.35cm}|p{.35cm}|p{.35cm}|p{.35cm}|}
%         \hline
%         $\bm{Direction}$ & \multicolumn{8}{|c|}{$\bm{Thrusters}$} \\ 
%         \hline
%         & $T_{11}$ & $T_{12}$ & $T_{21}$ & $T_{22}$ & $T_{31}$ & $T_{32}$ & $T_{41}$ & $T_{42}$ \\
%         \hline
%         Forward & & & \checkmark & & \checkmark & & &\\
%         \hline
%         Backward & \checkmark & & & & & & \checkmark &\\
%         \hline
%         Left & & \checkmark & & \checkmark & & & &\\
%         \hline
%         Right & & & & & & \checkmark & & \checkmark\\
%         \hline
%         Clockwise & \checkmark & & & \checkmark & \checkmark & & & \checkmark\\
%         \hline
%         C-Clockwise & & \checkmark & \checkmark & & & \checkmark & \checkmark &\\
%         \hline
%     \end{tabular}
%     \vspace*{.15cm}
%     \caption{Thruster activation logic}
%     \label{tab:5_slider_tsl}
% \end{table}
% \vspace*{-.25cm}

% \begin{table}[h]
%     \centering
%     \begin{tabular}{|c||c|c|}
%         \hline
%         $\bm{Thruster}$ & $\bm{Position}$ (\unit{\mm}) & $\bm{Orientation}$ (\unit{\deg}) \\
%         \hline
%         $\mathbf{T_{11}}$ & $(195, -140)$ & $0$ \\
%         \hline
%         $\mathbf{T_{12}}$ & $(140, -195)$ & $270$ \\
%         \hline
%         $\mathbf{T_{21}}$ & $(-195, -140)$ & $180$ \\
%         \hline
%         $\mathbf{T_{22}}$ & $(-140, -195)$ & $270$ \\
%         \hline
%         $\mathbf{T_{31}}$ & $(-195, 140)$ & $180$ \\
%         \hline
%         $\mathbf{T_{32}}$ & $(-140, 195)$ & $90$ \\
%         \hline
%         $\mathbf{T_{41}}$ & $(195, 140)$ & $0$ \\
%         \hline
%         $\mathbf{T_{42}}$ & $(195, 195)$ & $90$ \\
%         \hline
        
%     \end{tabular}
%     \vspace*{.15cm}
%     \caption{Thruster Position \& Orientation}
%     \label{tab:5_thruster_pos}
% \end{table}
% \vspace*{-.25cm}

The platform was equiped with a webcam to provide a visual feedback pointing towards the front of the robot.
Special, reflective markers were also mounted on the robot to enable odometry from the lab's visual positioning system.
Table \ref{tab:5_slider_parameters} contains some of Slider's hardware specifications.

\begin{table}[h]
    \centering
    \begin{tabular}{|c|c|}
        \hline
        $\bm{Parameters}$ & $\bm{Values}$ \\ 
        \hline
        Mass ($m$) & \qty{4.436}{\kg} \\ 
        \hline
        Moment of Inertia ($I_{zz}$) & \qty{1.092}{\kg\meter\squared} \\ 
        \hline
        Minimum thruster on-time & \qty{10}{\ms} \\ 
        \hline
        Maximum force ($F_{max}$) & \qty{1.4}{\newton} \\ 
        \hline
        Maximum torque ($\tau_{max}$) & \qty{0.392}{\newton\meter} \\ 
        \hline
    \end{tabular}
    \vspace*{.15cm}
    \caption{Slider system parameters}
    \label{tab:5_slider_parameters}
\end{table}
\vspace*{-.25cm}

The processing flow has to be split between the on-board computer and the central computer and Fig. \ref{fig:1_chaser_flow} needs to be updated.
Fig. \ref{fig:slider_flow} explains which parts of the processing flow happen on the on-board computer and which are on the central computer.
The estimation combiner is now only used for filtering since only one chaser is operating.

\insertfig{slider_flow}{Updated processing flow for Slider}

\subsection{Controller Setup}\label{ch:chap5:sec1:sub2}

Since the Slider is a 2D platform, the controller described in Section \ref{ch:chap3} needs to be modified.
To begin with, the system state now does not include position on z-axis or rotation on x and y-axis: $x=[x, y, q_w, q_x, q_y, q_z, \dot{x}, \dot{y}, \omega]$
Unfortunately, all four quaternion parts are needed to describe the rotation.
Next, Equation \ref{eq:3_target_dynamics} and \ref{eq:3_chaser_dynamics} can be used as is but with $z = v_z = \omega_z = 0$ set constant.

The input vector is updated to only contain two forces and a torque: $u = [F_x, F_y, \tau]$.
In \eqref{eq:3_chaser_dynamics}, $F_z = \tau_x = \tau_y = 0$ are set constant to simulate the uncontrollability their corresponding states.
The cost function in \eqref{eq:3_cost_function} can be used as is with the parts involving uncontrollable states ignored.
Finally, constraints \eqref{eq:3_constr_dist1} and \eqref{eq:3_constr_dist2} are removed.
The system has little space for maneuvers and, in order to test the controller, Slider must get really close to the target which will interfere with the collision avoidance constraints.

\subsection{Thruster Selection Logic}\label{ch:chap5:sec1:sub3}

Slider is an over-actuated system, since it uses 8 thrusters and is able to perform 6 distinct movements (3DoF + opposite).
This requires a complex thruster selection logic to determine which thrusters to activate to produce the requested force and torque.
% First of all, Equation \ref{eq:5_thruster_mapping} explains the relation between $F_x$, $F_y$ and $\tau$ and the force applied by each thruster.
% \begin{subequations}\label{eq:5_thruster_mapping}
%     \begin{equation}
%         Ax=b \Leftrightarrow [A_{11} \cdots A_{42}]x=b
%     \end{equation}
%     \begin{equation}
%         A_{AB} = \begin{bmatrix}
%             \cos{(\beta_{AB})} \\ \sin{(\beta_{AB})} \\ (r_{T_{AB}}^y \cos{(\beta_{AB})} - r_{T_{AB}}^x \cos{(\beta_{AB})})
%         \end{bmatrix}
%     \end{equation}
%     \begin{equation}
%         x = \begin{bmatrix}
%             T_{11} & T_{12} & T_{21} & T_{22} & T_{31} & T_{32} & T_{41} & T_{42}
%         \end{bmatrix}^T
%     \end{equation}
%     \begin{equation}
%         b = \begin{bmatrix}
%             F_x &
%             F_y &
%             \tau
%         \end{bmatrix}^T
%     \end{equation}
% \end{subequations}
% Where $r_{T_{AB}}^x$, $r_{T_{AB}}^y$ and $\beta_{AB}$ are the position and orientation of the thrusters that can be found in Table \ref{tab:5_thruster_pos}.
The problem of selecting the thrusters to activate for a given force-torque command is reduced to an optimization problem as described in the original Slider paper.
% \begin{equation}
%     \begin{split}
%         \min_x \quad & J = x^T \cdot x \\
%         \text{s.t.} \quad & Ax=b\\
%         \quad & 0 \le x \le x_{max}
%     \end{split}
% \end{equation}
The solution to the optimization problem is the optimal thruster output that provides the necessary force and torque output while minimizing the actuation effort and respecting the hardware limitations.
The output of the TSL is then passed through a PWM generator.

\subsection{Results}
Two scenarios were set up in the laboratory environment to evaluate the performance of the controller.
First, a target is placed down and held stationary near the Slider.
The slider is then allowed to track it and after a while the docking command is given.
Second, a target is initially moved by hand and the Slider is allowed to track it for a while. 
Then it is placed down and held stationary before the docking command is given.
Both scenarios are tested twice: Firstly using the ground truth odometry from the lab's positioning system to evaluate the controller with ideal pose estimation and then using the ArUco based estimation.

\begin{figure}
    \centering
    \resizebox{.49\textwidth}{!}{
        \input{images/dock_experiment_4.pgf}
    }
    \caption{Stationary target scenario using ground truth odometry. Docking begins at $t=\text{\qty{20}{\second}}$. \textbf{Top: } Slider and target trajectories \textbf{Middle: } Pose error \textbf{Bottom: } Docking evaluation}
    \label{fig:sl_dock_4}
\end{figure}

It should be emphasized that this setup is very susceptible to external disturbances.
In addition, there were other factors that negatively impacted the test results.
Most notably:
\begin{itemize}
    \item The sliding table is affected by the slope of the floor and the wooden board that supports it.
    \item The tubes used to provide the Slider with pressurised air, pull the platform towards the center of the sliding table.
    \item The lab's positioning system is not 100\% accurate.
    \item The controller is designed to track a target moving with constant velocity. Tracking a stationary target makes the process susceptible to noise.
    \item There is not enough room for a second chaser, or for complex maneuvers allowing the tracking of a constantly moving target.
\end{itemize}
Previous experimentations on the platform suggest that a \qty{5}{\centi\meter} deviation from setpoints is typical and expected.
With this in mind, we conclude that in all scenarios the chaser successfully docks the target at a docking distance of \qty{20}{\centi\meter} within the expected range of error.

\begin{figure}
    \centering
    \resizebox{.49\textwidth}{!}{
        \input{images/dock_experiment_5.pgf}
    }
    \caption{Initially moving target scenario using ground truth odometry. Docking begins at $t=\text{\qty{85}{\second}}$. \textbf{Top: } Slider and target trajectories \textbf{Middle: } Pose error \textbf{Bottom: } Docking evaluation}
    \label{fig:sl_dock_5}
\end{figure}

In Fig. \ref{fig:sl_dock_4} and \ref{fig:sl_dock_5} spikes in the error are observed in $t=\text{\qty{32}{\second}}$ and $t=\text{\qty{64}{\second}}$ respectively.
These are typical errors of the laboratory's visual positioning system that occur when the software confuses one of the reflective markers on the Slider and miscalculates it's orientation.
In the scenario of Fig. \ref{fig:sl_dock_4} the disturbance occurs during the docking phase while in the scenario of Fig. \ref{fig:sl_dock_5} it occurs during the tracking phase.
In both cases the controller is able to handle the disturbance and recover back to normal operation.

\begin{figure}
    \centering
    \resizebox{.49\textwidth}{!}{
        \input{images/dock_experiment_6.pgf}
    }
    \caption{Stationary target scenario using ArUco pose estimator. Docking begins at $t=\text{\qty{32}{\second}}$. \textbf{Top: } Slider and target trajectories \textbf{Middle: } Pose error \textbf{Bottom: } Docking evaluation}
    \label{fig:sl_dock_6}
\end{figure}

\begin{figure}
    \centering
    \resizebox{.49\textwidth}{!}{
        \input{images/dock_experiment_8.pgf}
    }
    \caption{Initially moving target scenario using ArUco pose estimator. Docking begins at $t=\text{\qty{30}{\second}}$. \textbf{Top: } Slider and target trajectories \textbf{Middle: } Pose error \textbf{Bottom: } Docking evaluation}
    \label{fig:sl_dock_8}
\end{figure}