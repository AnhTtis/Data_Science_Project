\subsection{Motivation}
The subject of autonomous rendezvous and docking in space has become popular during the last decades. 
Successful proximity operations can be utilized to extend the life span of satellites and help with deorbiting when they reach end of life (EOL). The need of accurate pose estimation of uncooperative targets in space for navigation and docking is highlighted by the European Space Agency (ESA) hosting their Pose Estimation Challenge in 2019 \citep{esa_challenge}. This challenge also enforced the use of only a single camera, highlighting the importance of simple, compact and cheap sensors for this kind of applications. Most of the published research on this topic focus on single-agent docking missions. With the recent advancements on the launch of small platforms and cansats, the use of multiple, smaller and simpler satellites can greatly increase the efficiency of the mission in terms of time and cost, which is the main motivation of the present article.

\subsection{Related Works}
A family of solutions for the ESA challenge used Convolutional Neural Networks (CNN) to estimate the pose of the target. However, the most popular implementations either require an a priori knowledge of the targets wireframe e.g. \citep{cnn1}.
Some solutions have also been presented that lift this limitation e.g. \citep{cnn2}. 
Nevertheless, the use of CNNs for pose estimation is still a relatively new field and require huge amounts of data for training.

More conventional solutions for the problem have been proposed in the literature. \cite{image_based_control_Zhao} splits these solutions into two categories: image-based and position-based. An example of an image-based solution, a controller built directly on the image taken by the sensor, is proposed by \cite{Huang2017}.
This work focuses on detecting well defined edges on spacecraft parts, such as brackets, where a mechanical grabber can easily attach.
\cite{pose_estimation_depthcam_Harvey} propose matching a 3D model of the target with the output of on-board sensors but this method requires more complex LiDAR sensors.
An effective and accurate way to estimate the pose of the target is to use markers on its surface.
\cite{marker_docking_nasa} evaluated the use of special, reflective markers for docking purposes but still use complex combinations of LiDAR and camera sensors.

An example of a position-based approach is proposed by \cite{optical_aided_rendezvous_Renato} and is one of the most promising solutions in literature.
It uses a combination of a pose estimation method based on the Structure from Motion (SfM) algorithm and an optimal control scheme, removing the need for markers and using only a monocular camera.
But still, this work focuses on a single-agent docking scenario.
Examples of docking multiple satellites exist in the literature but mostly focus on them docking with each other and not with an uncooperative target e.g. \citep{ardc} and \citep{multi_6dog_gnc}.

\subsection{Contributions}

%This work will attempt to combine the advantages of the aforementioned papers and propose improvements by using modern, state-of-the-art algorithms and methods.
%A complete, flexible control architecture for multiple satellites (chasers) docking a single, uncooperative target is proposed.The controller uses only a single camera for 3D target odometry estimation and is based on Non-linear Model Predictive Control (NMPC).
%More specifically, the ArUco method \citep{aruco_1} for marker detection is used in combination with the EPnP \citep{epnp} algorithm for pose estimation.
%A cost function and set of constraints for the NMPC is created ad-hoc.
%The PANOC non-linear optimization algorithm %\citep{panoc} is used for the implementation of the NMPC.
%A simulation environment is also developed to test and evaluate collaborative docking scenarios.

The proposed framework in this work aims to integrate and present improvements based on aforementioned state-of-art works. Thus, the main contributions in this work are as follows:
\begin{enumerate}
    \item We propose a novel centralized Nonlinear Model Predictive Controller architecture for multiple chaser spacecraft towards collaborative tracking and docking with an uncooperative, tumbling target spacecraft.
    \item A robust, fault-tolerant, vision-based pose estimation framework
    \item A comprehensive evaluations of the framework in a realistic simulation environment is carried out to prove the efficacy of the proposed work.
\end{enumerate}

%%
% The architecture will be evaluated both in simulation and in a laboratory environment.