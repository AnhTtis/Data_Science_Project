\documentclass[journal,10pt,twocolumn,draftclsnofoot,]{IEEEtran}
\usepackage[a4paper, total={6.8in, 10in}]{geometry}
\usepackage[pdftex]{graphicx}
\linespread{1}

\usepackage{amssymb}
\usepackage{caption}
\usepackage{comment}
\usepackage{multirow}
\usepackage{romannum}
\usepackage{subcaption}
\usepackage{graphicx}
\usepackage{booktabs}
\usepackage{multirow}
\usepackage{array}
\usepackage{hyperref}
\usepackage{xcolor}
\newcolumntype{P}[1]{>{\centering\arraybackslash}p{#1}}


\begin{document}


% \noindent\textbf{Dear Respected Reviewers}\\
\noindent{\large \textbf{Dear Respected Reviewers}}

\vspace{3pt}

We would like to extend our appreciation for taking the time and effort to provide such valuable feedback. We have carefully read all insightful comments from the respected reviewers and we have addressed each comment below.

\vspace{3pt}
\noindent\textbf{Response to Reviewer \#3}

\vspace{3pt}





% \noindent\textcolor{blue}{\textbf{R3 C1:} \textit{Fairness metrics need to be defined in the experiment section. Section 3 only defined the ``fair'' cases, but not the metrics.}}

\noindent\textcolor{blue}{\textbf{R3 C1:} \textit{Definition of fairness metrics.}}


% \vspace{3pt}

\noindent\textbf{Response:}
Similar to other studies on fairness [12, 13], we use the difference in the probabilities (as presented in equations of Section 3 as our metrics. In perfectly fair cases these differences are equal to 0 however in biased decision making these values are often greater than 0 which we report in the Results section. We will further clarify this in the final paper.


\vspace{3pt}

\noindent\textcolor{blue}{\textbf{R3 C2:} \textit{The sizes (and complication level) of SR models.}}
% need to be clearly described for readers to understand the conclusions of the paper.}}

\noindent\textbf{Response:}    
% The complication level of the models are decided based on the sophistication of the architectures. For example the addition of the concept of the skip connections to the architecture (The difference between resnet and se-resnet models with VGG-M-40 model). Inside the resnet and se-resnet based models we compare the 2 variations of `V2' and `L' with the variation 'V2' being larger compared to `L' in terms of number of parameters. We will add this statement to the paper along with the comparison between the number of parameters for resnet and se-resnet-based models. 
Thank you for this interesting suggestion. While the complete architectures of each of these models are provided in their respective references (Nagrani et al. [4], Chung et al. [5]) We have now calculated the parameters and provide them in Table 1, which we will add to the manuscript and discuss.

\begin{table}[h]
\centering
\footnotesize
\caption{Comparison between the number of parameters on ResNet and SE-ResNet-based models.}
\label{table:hyp}
\begin{tabular}{l l} 
\hline
Method & Num. of parameters\\
% \hline
\hline
  ResNet34L &  1.4M \\ %change citations
  ResNet34V2 & 2.0M\\
  SEResNet34L & 1.4M \\
  SEResNet34V2 & 2.0M \\
\hline

\end{tabular}
\vspace{-10pt}
\end{table}


\vspace{3pt}

\noindent\textcolor{blue}{\textbf{R3 C3:}\textit{ Statistical tests.}} 

\noindent\textbf{Response:}
Thank you for this suggestion. We have started working on running the tests, but unfortunately they did not make it to the deadline. We should point out that given 4 sets of independent variables (DNN model, loss function, fairness measure, protected group), providing a full statistical analysis would be very challenging given the limited space. We will try to include a few important ones, while providing a full statistical analysis in the expansion of the work, which we are currently working on.

% No Reponse yet.
% Don't know how to answer this all we did was statistical tests what else are we supposed to do?




\vspace{3pt}














\noindent\textbf{Response to Reviewer \#4}

\vspace{3pt}

\noindent\textbf{\textcolor{blue}{R4 C1:}}
\textcolor{blue}{\textit{Interesting paper, relevant problem, comprehensive study. Minor English review is suggested to catch grammar errors.}}

\noindent\textbf{Response:}
Thank you for this feedback. We have now reviewed and corrected a few grammatical errors for the final version of the paper. 
\vspace{3pt}

\noindent\textbf{Response to Reviewer \#5}

\vspace{3pt}

\noindent\textbf{\textcolor{blue}{R5 C1:}}
\textcolor{blue}{\textit{Fairness measurements.}}

\noindent\textbf{Response:}
% Equations 1, 2, and 3 are the formulations of the metrics used in this paper. 
As done by other studies on fairness [12], [13], we use the difference in the probabilities (as presented in equations of section 3) as our metrics. In perfectly fair cases these differences are equal to 0 however in biased decision making these values are often greater than 0 which we report in the results section. We will further clarify this in the final paper.

\vspace{3pt}

\noindent\textbf{\textcolor{blue}{R5 C2:}}
\textcolor{blue}{\textit{The fairness changes with different thresholds.}}

\noindent\textbf{Response:}
Thank you for this suggestion. This is a very interesting idea which we will explore as we work on expansion of the work. We would also like to point out that an analysis of fairness with regards to thresholds have been performed in a few other studies such as [13]. 
% In this paper we aimed to study the fairness of the models and training protocols at their best performing conditions. On the other hand, an analysis of fairness with regards to thresholds have already been performed in other studies such as [13]. 
\vspace{3pt}

\noindent\textbf{\textcolor{blue}{R5 C3:}}
\textcolor{blue}{\textit{The experiments on other datasets should also be attempted.}}

\noindent\textbf{Response:}
Thank you for this suggestion. While we used Voxceleb as the standard defacto dataset used for speaker recognition, we will include additional datasets such as SRE21 and SITW for expansion of the work.
% are currently performing the experiments on other datasets and will collect the results of these experiments into the expansion of this work.  
\vspace{3pt}

\noindent\textbf{Response to Reviewer \#6}

\vspace{3pt}

\noindent\textbf{\textcolor{blue}{R6 C1:}}
\textcolor{blue}{\textit{Reformulate the end of Section 2.}}

\noindent\textbf{Response:}
Thank you for pointing this out. We will revise this as you suggested to avoid the implication that our work completes studies on fairness in SR systems.

\vspace{3pt}

\noindent\textbf{\textcolor{blue}{R6 C2, C3, C6:}}
\textcolor{blue}{\textit{On fairness metrics, EER, and DCF.}}

\noindent\textbf{Response:}
Since in order to measure EER for measuring the performance of an SR system, different thresholds need to be selected, this approach has not been widely used for speech-related fairness research in the literature. As a result, we resorted to the fundamental definition of fairness to carry out the intended experiments. As you suggested, we will heavily shorten 3.2 and instead (1) clarify our motivation for using the metrics that were used and provide stronger grounding through seminal prior works such as [12] and [13], and further explain potential issues with using EER or DCF; (2) analysis of the DNN models. 

\vspace{3pt}


% \noindent\textbf{\textcolor{blue}{R6 C3 and C6:}}
% \textcolor{blue}{\textit{Intention of using different metrics and not using EER.}}

% \noindent\textbf{Response:}
% We wanted to keep the threshold constant for all of our experiments and EER values are very dependant on selecting different thresholds for the final score of SR systems. 
% \vspace{3pt}


% \noindent\textbf{\textcolor{blue}{R6 C4:}}
% \textcolor{blue}{\textit{Removing section 3.2.}}

% \noindent\textbf{Response:}
% Thank you for the feedback. We have now removed this section and dedicated this space to a more detailed analysis of size and complication level of the models as well as providing context on possible effects of the loss functions.  
% \vspace{3pt}

\noindent\textbf{\textcolor{blue}{R6 C4:}}
\textcolor{blue}{\textit{Other types of models.}}

\noindent\textbf{Response:} 
Thank you for this interesting suggestion. For expansion of the work, we will consider other forms of models beyond DNNs.
\vspace{3pt}



\noindent\textbf{\textcolor{blue}{R6 C5:}}
\textcolor{blue}{\textit{Context on the importance of the loss functions.}}

\noindent\textbf{Response:}
The 2 types of loss functions, namely classification and metric learning, have 2 different approaches when training the DNNs. Classification losses take each utterance and its speaker objectively and treat it separately from other speakers. However, metric learning loss functions compare utterances from different speakers with each other. We expected that involving one or multiple speakers during the training of DNNs would have an effect on the fairness of the models. We will add this discussion to the paper.  
\vspace{3pt}

\noindent\textbf{\textcolor{blue}{R6 C7:}}
\textcolor{blue}{\textit{Description for Table 1 and significance.}}

\noindent\textbf{Response:}
Thank you for pointing this out. We will revise the caption to be more descriptive and self-contained. 
% \vspace{3pt}
% \noindent\textbf{\textcolor{blue}{R6 C8:}}
% \textcolor{blue}{\textit{Significance of the values in the table 1.}}
% \noindent\textbf{Response:}
% Thank you for your feed back. 
Regarding the significance of numbers, we would like to point out that while we are currently performing statistical significance tests on the results, we believe the comparative results of these probability values on their own are quite meaningful. 
% any amount of bias in decision making is not desirable and therefore  is significant. However, the decision on what amount of bias is considered a large bias is subjective. In this study we aimed to compare the different models and loss functions to find a combination with the least amount of bias. 
\vspace{3pt}

\noindent\textbf{\textcolor{blue}{R6 C8:}}
\textcolor{blue}{\textit{A study based on ethnicity of the speakers.}}

\noindent\textbf{Response:}
Thank you for the interesting suggestion. We agree that this would be an interesting follow up study. 
% We will include the ethnicity of the speakers as a factor in the expansion of this work. 
\vspace{3pt}

\noindent\textbf{\textcolor{blue}{R6 C9:}}
\textcolor{blue}{\textit{Impact of data and balancing.}}

\noindent\textbf{Response:}
Thank you for this interesting suggestion. 
% We are currently performing further experiments regarding the effects of different methods to reduce the bias in SR systems. 
As a key part of our efforts on extending this work to include new ways of mitigating/reducing bias, 
% is the analysis on different data manipulation techniques.
we have been exploring different sampling and augmentation techniques. 
%The result of this study will be published in the expansion of this work. 
\vspace{3pt}



\end{document}