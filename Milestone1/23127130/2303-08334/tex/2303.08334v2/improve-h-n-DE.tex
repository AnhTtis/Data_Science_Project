%%%%%%%%%%%%%%%%%%%%%%%%%%%%%%%%%%%%%%%%%%%%%%%%%%%%%%%%%%%%%%%%%%%%%
%%                                                                 %%
%% Please do not use \input{...} to include other tex files.       %%
%% Submit your LaTeX manuscript as one .tex document.              %%
%%                                                                 %%
%% All additional figures and files should be attached             %%
%% separately and not embedded in the \TeX\ document itself.       %%
%%                                                                 %%
%%%%%%%%%%%%%%%%%%%%%%%%%%%%%%%%%%%%%%%%%%%%%%%%%%%%%%%%%%%%%%%%%%%%%

%%\documentclass[referee,sn-basic]{sn-jnl}% referee option is meant for double line spacing

%%=======================================================%%
%% to print line numbers in the margin use lineno option %%
%%=======================================================%%

%%\documentclass[lineno,sn-basic]{sn-jnl}% Basic Springer Nature Reference Style/Chemistry Reference Style

%%======================================================%%
%% to compile with pdflatex/xelatex use pdflatex option %%
%%======================================================%%

%%\documentclass[pdflatex,sn-basic]{sn-jnl}% Basic Springer Nature Reference Style/Chemistry Reference Style

%%\documentclass[sn-basic]{sn-jnl}% Basic Springer Nature Reference Style/Chemistry Reference Style
\documentclass[pdflatex,sn-mathphys]{sn-jnl}% Math and Physical Sciences Reference Style
%%\documentclass[sn-aps]{sn-jnl}% American Physical Society (APS) Reference Style
%%\documentclass[sn-vancouver]{sn-jnl}% Vancouver Reference Style
%%\documentclass[sn-apa]{sn-jnl}% APA Reference Style
%%\documentclass[sn-chicago]{sn-jnl}% Chicago-based Humanities Reference Style
%%\documentclass[sn-standardnature]{sn-jnl}% Standard Nature Portfolio Reference Style
%%\documentclass[default]{sn-jnl}% Default
%%\documentclass[default,iicol]{sn-jnl}% Default with double column layout

%%%% Standard Packages
%%<additional latex packages if required can be included here>
%%%%

%%%%%=============================================================================%%%%
%%%%  Remarks: This template is provided to aid authors with the preparation
%%%%  of original research articles intended for submission to journals published 
%%%%  by Springer Nature. The guidance has been prepared in partnership with 
%%%%  production teams to conform to Springer Nature technical requirements. 
%%%%  Editorial and presentation requirements differ among journal portfolios and 
%%%%  research disciplines. You may find sections in this template are irrelevant 
%%%%  to your work and are empowered to omit any such section if allowed by the 
%%%%  journal you intend to submit to. The submission guidelines and policies 
%%%%  of the journal take precedence. A detailed User Manual is available in the 
%%%%  template package for technical guidance.
%%%%%=============================================================================%%%%

\jyear{2021}%

%% as per the requirement new theorem styles can be included as shown below
\theoremstyle{thmstyleone}%
%%\newtheorem{theorem}{Theorem}%  meant for continuous numbers
\newtheorem{theorem}{Theorem}[section]% meant for sectionwise numbers
%% optional argument [theorem] produces theorem numbering sequence instead of independent numbers for Proposition
\newtheorem{proposition}[theorem]{Proposition}% 
\newtheorem{lemma}[theorem]{Lemma}% 
%%\newtheorem{proposition}{Proposition}% to get separate numbers for theorem and proposition etc.

\theoremstyle{thmstyletwo}%
\newtheorem{example}{Example}%
\newtheorem{remark}{Remark}%

\theoremstyle{thmstylethree}%
\newtheorem{definition}{Definition}%

\raggedbottom
%%\unnumbered% uncomment this for unnumbered level heads

\usepackage{mathtools}
\usepackage{upgreek}
\DeclarePairedDelimiter{\abs}{\lvert}{\rvert}
\newcommand{\domD}{\mathscr{D}}
\newcommand{\dd}{\mathrm{d}}
\newcommand{\LC}{\mathbf{L}^{\mathrm{DE}}}
\renewcommand{\Im}{\operatorname{Im}}
\renewcommand{\Re}{\operatorname{Re}}
\renewcommand{\pi}{\uppi}
\DeclareMathOperator{\sinc}{sinc}
\DeclareMathOperator{\arsinh}{arsinh}
\DeclareMathOperator{\OO}{O}
\DeclareMathOperator{\ee}{e}
\numberwithin{equation}{section}
\begin{document}

\title[Improvement of selection formulas of mesh size and truncation numbers]{Improvement of selection formulas of mesh size and truncation numbers for the DE-Sinc approximation and its theoretical error bound}

%%=============================================================%%
%% Prefix	-> \pfx{Dr}
%% GivenName	-> \fnm{Joergen W.}
%% Particle	-> \spfx{van der} -> surname prefix
%% FamilyName	-> \sur{Ploeg}
%% Suffix	-> \sfx{IV}
%% NatureName	-> \tanm{Poet Laureate} -> Title after name
%% Degrees	-> \dgr{MSc, PhD}
%% \author*[1,2]{\pfx{Dr} \fnm{Joergen W.} \spfx{van der} \sur{Ploeg} \sfx{IV} \tanm{Poet Laureate} 
%%                 \dgr{MSc, PhD}}\email{iauthor@gmail.com}
%%=============================================================%%

\author*[1]{\fnm{Tomoaki} \sur{Okayama}}\email{okayama@hiroshima-cu.ac.jp}
%ORCID: 0000-0001-9942-1670
\author[2]{\fnm{Shota} \sur{Ogawa}}%\email{tmkkmtmitrotfnfkikstif@i.softbank.jp}

\affil*[1]{\orgdiv{Graduate School of Information Sciences},
\orgname{Hiroshima City University}, \orgaddress{\street{3-4-1,
Ozuka-higashi}, \city{Asaminami-ku}, \postcode{731-3194},
\state{Hiroshima}, \country{Japan}}}

\affil[2]{\orgdiv{} \orgname{Systemers Inc.},
\orgaddress{\street{Kurotaki Building 3F, 3-9-7, Shimbashi},
\city{Minato-ku}, \postcode{105-0004}, \state{Tokyo}, \country{Japan}}}

%%==================================%%
%% sample for unstructured abstract %%
%%==================================%%

\abstract{
The Sinc approximation applied to
double-exponentially decaying functions
is referred to as the DE-Sinc approximation.
Because of its high efficiency,
this method has been used in various applications.
In the Sinc approximation, its mesh size and truncation numbers should be
optimally selected to achieve its best performance. However,
%50
the standard selection formula has only been ``near-optimally'' selected
because the optimal formula of the mesh size cannot be expressed
in terms of elementary functions of truncation numbers.
In this study, we propose two improved selection formulas.
The first one is based on the concept by an earlier research
%100
that resulted in a better selection formula
for the double-exponential formula.
The formula performs slightly better than the standard one,
but is still not optimal.
As a second selection formula,
we introduce a new parameter to propose truly optimal selection formula.
We provide explicit error bounds for both selection formulas.
%151
Numerical comparisons show that the first formula gives
a better error bound than the standard formula,
and the second formula gives a much better error bound
than the standard and first formulas.
%183
}

%%================================%%
%% Sample for structured abstract %%
%%================================%%

% \abstract{\textbf{Purpose:} The abstract serves both as a general introduction to the topic and as a brief, non-technical summary of the main results and their implications. The abstract must not include subheadings (unless expressly permitted in the journal's Instructions to Authors), equations or citations. As a guide the abstract should not exceed 200 words. Most journals do not set a hard limit however authors are advised to check the author instructions for the journal they are submitting to.
% 
% \textbf{Methods:} The abstract serves both as a general introduction to the topic and as a brief, non-technical summary of the main results and their implications. The abstract must not include subheadings (unless expressly permitted in the journal's Instructions to Authors), equations or citations. As a guide the abstract should not exceed 200 words. Most journals do not set a hard limit however authors are advised to check the author instructions for the journal they are submitting to.
% 
% \textbf{Results:} The abstract serves both as a general introduction to the topic and as a brief, non-technical summary of the main results and their implications. The abstract must not include subheadings (unless expressly permitted in the journal's Instructions to Authors), equations or citations. As a guide the abstract should not exceed 200 words. Most journals do not set a hard limit however authors are advised to check the author instructions for the journal they are submitting to.
% 
% \textbf{Conclusion:} The abstract serves both as a general introduction to the topic and as a brief, non-technical summary of the main results and their implications. The abstract must not include subheadings (unless expressly permitted in the journal's Instructions to Authors), equations or citations. As a guide the abstract should not exceed 200 words. Most journals do not set a hard limit however authors are advised to check the author instructions for the journal they are submitting to.}

\keywords{Sinc approximation, double-exponential transformation, error bound,
mesh size, truncation number}

%%\pacs[JEL Classification]{D8, H51}

%%\pacs[MSC Classification]{35A01, 65L10, 65L12, 65L20, 65L70}

\maketitle

\section{Introduction and summary}\label{sec1}

The Sinc approximation is based on Shannon's sampling formula
\begin{equation}
 F(x)\approx \sum_{k=-\infty}^{\infty}
 F(kh) \sinc\left(\frac{x-kh}{h}\right),
\label{eq:sinc-approx-inf}
\end{equation}
where $\sinc x=\sin(\pi x)/(\pi x)$, and $h$ represents a mesh size.
If $F$ is analytic and absolutely integrable
on a strip complex region
\begin{equation}
\domD_d=\{\zeta\in\mathbb{C} : \abs{\Im\zeta}< d\}\quad (d>0),
\label{eq:domDd}
\end{equation}
this approximation is capable of achieving exponential convergence,
denoted as $\OO(\ee^{-\pi d/h})$.
Furthermore, if $\abs{F(x)}$ decays exponentially as $x\to\pm\infty$,
i.e., $\abs{F(x)}=\OO(\ee^{-\mu\abs{x}})$ $(\mu>0)$,
the infinite sum in~\eqref{eq:sinc-approx-inf} may
be truncated at some truncation numbers $M$ and $N$ as
\begin{equation}
\sum_{k=-\infty}^{\infty}
 F(kh) \sinc\left(\frac{x-kh}{h}\right)
\approx \sum_{k=-M}^{N}
 F(kh) \sinc\left(\frac{x-kh}{h}\right).
\label{eq:sinc-approx-finite}
\end{equation}
Its error rate is estimated as $\OO(\ee^{-\mu n h} / h)$,
where $n = \max\{M, N\}$.
The overall error is estimated by
the sum of the error in~\eqref{eq:sinc-approx-inf}
(termed the discretization error) and
that in~\eqref{eq:sinc-approx-finite} (termed the truncation error).
Thus,
by setting the exponential rates of the two errors equal to each other,
we can derive the optimal formula for
the mesh size $h$ with respect to $n$
as
\[
 \ee^{-\pi d /h} = \ee^{-\mu n h}
\quad \Leftrightarrow \quad
h = \sqrt{\frac{\pi d}{\mu n}},
\]
resulting in $\OO(\sqrt{n}\ee^{-\sqrt{\pi d \mu n}})$
as the final (overall) error rate.
The Sinc approximation for
exponentially decaying functions is called
the SE-Sinc approximation,
and it has been extensively developed
and analyzed by Stenger~\cite{Stenger,stenger00:_summar}.

Recently, it has been reported~\cite{SugiMatsu,TanaSugiMuro} that
if $\abs{F(x)}$ decays double exponentially as $x\to\pm\infty$,
i.e., $\abs{F(x)}=\OO(\ee^{-(\pi/2)\mu \exp(\abs{x})})$,
the approximation error in~\eqref{eq:sinc-approx-finite}
is $\OO(\ee^{-(\pi/2)\mu\exp(nh)})$,
which is considerably smaller than that
in the case where $\abs{F(x)}$ decays single exponentially.
The Sinc approximation for
double-exponentially decaying functions is called
the DE-Sinc approximation.
In this case, the optimal formula of
the mesh size $h$ with respect to $n$
is given by the equation
\begin{equation}
 \ee^{-\pi d /h} = \ee^{-(\pi/2)\mu\exp(nh)},
\label{eq:double-expo-case}
\end{equation}
but a closed form for $h$
in terms of elementary functions of $n$
cannot be obtained.
Instead, several authors have employed the following near-optimal formula
\begin{equation}
 h = \frac{\log(2 d n/\mu)}{n}.
\label{eq:standard-h-DE}
\end{equation}
By the formula, the left-hand side in~\eqref{eq:double-expo-case}
becomes $\ee^{-\pi d n/\log(2 d n/\mu)}$,
and the right-hand side in~\eqref{eq:double-expo-case}
becomes $\ee^{-\pi d n}$.
Although these two rates are approximately equal,
the former is slightly lower.
Consequently, the final error rate of the DE-Sinc approximation
is $\OO(\ee^{-\pi d n/\log(2 d n/\mu)})$.
%as the final error rate of the DE-Sinc approximation.
We note that not only its error rate
but also its computable error bound has been provided~\cite{OkaMatsuSugi,Oka1},
which is useful for a computation with guaranteed accuracy.

Considering the above background, this study improves the formula
that relates $h$ and $n$ in the case of the DE-Sinc approximation.
We propose two improved formulas.
The first one is based on the concept by an earlier research~\cite{OkaKuro},
which focused on the double-exponential formula.
Their improved formula between $h$ and $n$
reduced the gap between the discretization and truncation errors.
As a result,
a sharper error bound for the double-exponential formula
was obtained than the existing one.
Thus, utilizing the same idea,
we provide a sharper error bound for the DE-Sinc approximation
than the existing one~\cite{OkaMatsuSugi,Oka1}.

Our second formula is based on a distinct approach.
Thus far, $n$ has denoted a maximum number of truncation numbers $M$ and $N$,
i.e., $n=\max\{M, N\}$.
In the new formula, we regard $n$ as a free parameter
that does not have to be equal to $M$ or $N$.
Then, we can derive formulas for $h$, $M$, and $N$
with respect to $n$, such that
the discretization and truncation errors
have exactly the same rate.
In consequence, we give a much better error bound
for the DE-Sinc approximation.
Numerical experiments also support our theoretical results.

The remaining sections of this paper are structured as follows.
In Section~\ref{sec2},
after reviewing an existing theorem,
we state two new theorems (main results of this paper).
In Section~\ref{sec3},
we show some numerical examples.
In Section~\ref{sec4},
we provide proofs of the main results.
%In Section~\ref{sec5}, we provide the conclusions of this study.


\section{Existing and new results}\label{sec2}

In this paper,
we consider functions $F(\zeta)$
that are analytic in $\domD_d$ (recall~\eqref{eq:domDd})
and decay double-exponentially as $\Re\zeta\to\pm\infty$.
More specifically,
we consider the following function space.

\begin{definition}
Let $L$, $R$, $\alpha$, and $\beta$
be positive constants,
and let $d$ be a constant with $0<d<\pi/2$.
Then, $\LC_{L,R,\alpha,\beta}(\domD_d)$ denotes
a family of functions $F$ that are analytic in $\domD_d$,
and satisfy
\begin{align}
 \abs{F(\zeta)}&\leq
\frac{L}{\abs{1+\ee^{-\pi\sinh\zeta}}^{\alpha}\abs{1+\ee^{\pi\sinh\zeta}}^{\beta}}\nonumber
\intertext{for all $\zeta\in\domD_d$, and}
 \abs{F(x)}&\leq
\frac{R}{(1+\ee^{-\pi\sinh x})^{\alpha}(1+\ee^{\pi\sinh x})^{\beta}}
\label{eq:double-exp-decay}
\end{align}
for all $x\in\mathbb{R}$.
\end{definition}

The existing error bound
for the DE-Sinc approximation is written as follows.

\begin{theorem}[Okayama~{\cite[Theorem 4.8]{Oka1}}]
\label{thm:exist}
Assume that $F\in\LC_{L,R,\alpha,\beta}(\domD_d)$.
Let $\mu=\min\{\alpha,\beta\}$,
let $\nu=\max\{\alpha,\beta\}$,
let $h$ be defined as in~\eqref{eq:standard-h-DE},
and let $M$ and $N$ be defined as
\begin{align}
\left\{
\begin{aligned}
  M&=n, &N&= n - \left\lfloor\frac{1}{h}\log\left(\frac{\beta}{\alpha}\right)\right\rfloor&&
(\text{if}\,\,\,\mu=\alpha),\\
  N&=n, &M&= n - \left\lfloor\frac{1}{h}\log\left(\frac{\alpha}{\beta}\right)\right\rfloor&&
(\text{if}\,\,\,\mu=\beta).
\end{aligned}
\right.
\label{eq:standard-M-N}
\end{align}
Furthermore, let $n$ be taken sufficiently large so that
$n\geq (\nu\ee)/(2d)$ holds.
Then, it holds that
\[
 \sup_{x\in\mathbb{R}}
\left\lvert
F(x) - \sum_{k=-M}^N F(kh)\sinc\left(\frac{x - kh}{h}\right)
\right\rvert
\leq C \ee^{-\pi d n/\log(2 d n/\mu)},
\]
where $C$ is a constant independent of $n$, expressed as
\[
 C
=\frac{2}{\pi d}
\left[
\frac{2L}{\pi\mu(1 - \ee^{-\pi\mu\ee})\cos^{\alpha+\beta}((\pi/2)\sin d)\cos d}
+ R\ee^{\pi\nu/2}
\right].
\]
\end{theorem}

This theorem states not only the convergence rate
$\OO(\ee^{-\pi d n/\log(2 d n/\mu)})$,
but also the computable error bound by revealing
the explicit form of the constant $C$.

In the same form,
the first main result of this paper is written as follows.
The proof is given in Section~\ref{sec:proof1}.
For convenience,
we introduce functions $p(x)$ and $q(x)$ as
\begin{align}
 p(x) &= \frac{x}{\arsinh(x / \arsinh x)}, \label{eq:p-x}\\
 q(x) &= \frac{x}{\arsinh x}. \label{eq:q-x}
\end{align}

\begin{theorem}
\label{thm:new1}
Assume that $F\in\LC_{L,R,\alpha,\beta}(\domD_d)$.
Let $\mu=\min\{\alpha,\beta\}$,
let $h$ be defined as
\begin{equation}
h = \frac{\arsinh(q(dn/\mu))}{n},
\label{eq:improve-h-DE-1}
\end{equation}
and let $M$ and $N$ be defined as
\begin{align}
\left\{
\begin{aligned}
  M&=n, &N&= \left\lceil\frac{1}{h}\arsinh\left(\frac{\alpha}{\beta}q\left(\frac{dn}{\mu}\right)\right)\right\rceil&&
(\text{if}\,\,\,\mu=\alpha),\\
  N&=n, &M&= \left\lceil\frac{1}{h}\arsinh\left(\frac{\beta}{\alpha}q\left(\frac{dn}{\mu}\right)\right)\right\rceil&&
(\text{if}\,\,\,\mu=\beta).
\end{aligned}
\right.
\label{eq:improve-M-N-1}
\end{align}
Then, it holds that
\[
 \sup_{x\in\mathbb{R}}
\left\lvert
F(x) - \sum_{k=-M}^N F(kh)\sinc\left(\frac{x - kh}{h}\right)
\right\rvert
\leq C \ee^{-\pi d n/\arsinh(d n/\mu)},
\]
where $C$ is a constant independent of $n$, expressed as
\[
 C = \frac{2}{\pi d}
\left[
\frac{2L\ee^{-\pi\mu(p(d/\mu)-q(d/\mu))}}{\pi\mu(1 - \ee^{-2\pi\mu p(d/\mu)})\cos^{\alpha+\beta}((\pi/2)\sin d)\cos d}
+ \frac{\pi}{2}R
\right].
\]
\end{theorem}

The convergence rate given by Theorem~\ref{thm:new1} is
$\OO(\ee^{-\pi d n/\arsinh(dn/\mu)})$,
which is asymptotically equal to the previous rate:
$\OO(\ee^{-\pi d n/\log(2dn/\mu)})$
(note that
$\arsinh x=\log(x + \sqrt{1 + x^2})\simeq \log (2x)$ as $x\to\infty$).
However, the constant $C$ in Theorem~\ref{thm:new1} may be smaller than
that in Theorem~\ref{thm:exist},
which is also confirmed in numerical experiments provided in the next section.
Furthermore, in Theorem~\ref{thm:exist},
there is an artificial condition on $n$:
$n\geq (\nu\ee)/(2d)$,
which is eliminated in Theorem~\ref{thm:new1}.

In both Theorems~\ref{thm:exist} and~\ref{thm:new1},
the integer $n$ satisfies
$n = \max\{M, N\}$,
i.e., either $M$ or $N$ equals to $n$.
In the next theorem,
both $M$ and $N$ may be taken smaller than $n$.
This theorem is the second main result of this paper.
The proof is given in Section~\ref{sec:proof2}.

\begin{theorem}
\label{thm:new2}
Assume that $F\in\LC_{L,R,\alpha,\beta}(\domD_d)$.
Let $\mu=\min\{\alpha,\beta\}$,
let $h$ be defined as
\begin{equation}
h = \frac{\arsinh(dn/\mu)}{n},
\label{eq:improve-h-DE-2}
\end{equation}
and let $M$ and $N$ be defined as
\begin{align}
\begin{aligned}
  M&=\left\lceil\frac{1}{h}\arsinh\left(\frac{\mu}{\alpha}q\left(\frac{dn}{\mu}\right)\right)\right\rceil,
 &N&= \left\lceil\frac{1}{h}\arsinh\left(\frac{\mu}{\beta}q\left(\frac{dn}{\mu}\right)\right)\right\rceil.
\end{aligned}
\label{eq:improve-M-N-2}
\end{align}
Then, it holds that
\[
 \sup_{x\in\mathbb{R}}
\left\lvert
F(x) - \sum_{k=-M}^N F(kh)\sinc\left(\frac{x - kh}{h}\right)
\right\rvert
\leq C \ee^{-\pi d n/\arsinh(d n/\mu)},
\]
where $C$ is a constant independent of $n$, expressed as
\[
C= \frac{2}{\pi d}
\left[
\frac{2L}{\pi\mu(1 - \ee^{-2\pi\mu q(d/\mu)})\cos^{\alpha+\beta}((\pi/2)\sin d)\cos d}
+ R
\right].
\]
\end{theorem}

For simplicity, let us consider the case where $\alpha=\beta$.
In this case, from both formulas~\eqref{eq:standard-M-N}
and~\eqref{eq:improve-M-N-1},
substituting the formula of $h$,
we obtain $M=N=n$.
In contrast, from the formula~\eqref{eq:improve-M-N-2},
substituting the formula of $h$,
%~\eqref{eq:improve-h-DE-2}
we obtain
\[
 M=N=\left\lceil
n\cdot \frac{\arsinh(q(dn/\mu))}{\arsinh(dn/\mu)}
\right\rceil.
\]
Because $q(x)< x$ for $x> \sinh 1$,
this equation implies that $M=N < n$ for sufficiently large $n$.
Although the convergence rates
given by Theorems~\ref{thm:new1} and~\ref{thm:new2}
seem exactly the same,
we may obtain a better result in the case of Theorem~\ref{thm:new2},
because the number of function evaluations
$M+N+1$ may be smaller than that in the case of Theorem~\ref{thm:new1}.
This is also confirmed in numerical experiments provided in the next section.

\section{Numerical experiments}
\label{sec3}

In this section, we present numerical results
using the formulas described
in Theorems~\ref{thm:exist}--\ref{thm:new2}.
We implemented the programs in C
with double-precision arithmetic.

The numerical experiments are conducted on the following two functions:
\begin{align}
 f_1(t) &= (1 - t^2)^{1/2}, \label{eq:f_1}\\
 f_2(t) &= (1 + t^2)^{1/2}(1 + t)^{1/2}(1 - t)^{3/4}, \label{eq:f_2}
\end{align}
which are defined on $[-1, 1]$.
The second one was also conducted in
Okayama et al.~\cite[Example 3.1]{OkaMatsuSugi}.
Before applying the Sinc approximation,
we employ the double-exponential transformation given by
\[
 t = \phi(x) = \tanh\left(\frac{\pi}{2}\sinh x\right).
\]
That is, putting $F_i(x)=f_i(\phi(x))$, we obtain the
DE-Sinc approximation formula
\[
 F_i(x) \approx\sum_{k=-M}^N F_i(kh)\sinc\left(\frac{x - kh}{h}\right).
\]
The function $F_1$ satisfies
the assumptions in Theorems~\ref{thm:exist}--\ref{thm:new2}
with $L=R=2$, $\alpha=\beta=1/2$, and $d=3/2$.
The function $F_2$
satisfies
the assumptions in Theorems~\ref{thm:exist}--\ref{thm:new2}
with $L=R=4$, $\alpha=1/2$, $\beta=3/4$, and $d=\pi/6$.
We examined its approximation error
on the following 1999 points:
\[
 x = \phi^{-1}\left(\frac{i}{1000}\right)\quad
 (i = -999,\,-998,\,\ldots,\,-1,\,0,\,1,\,\ldots,\,998,\,999),
\]
and investigated the maximum error among these points.
The results are shown in Figs.~\ref{fig:1} and~\ref{fig:2}.
In both figures,
the error bounds given in Theorems~\ref{thm:exist}--\ref{thm:new2}
include corresponding observed errors.
Furthermore, the figures also demonstrate that
the constant $C$ in Theorem~\ref{thm:new1}
is improved compared to that in Theorem~\ref{thm:exist},
although the convergence rate remains unchanged.
In contrast,
the convergence rate by Theorem~\ref{thm:new2}
is higher than that by Theorems~\ref{thm:exist} and~\ref{thm:new1}.
This reflects the fact that $M+N+1$ (the number of function evaluations)
in Theorem~\ref{thm:new2} is smaller than
that in Theorems~\ref{thm:exist} and~\ref{thm:new1}
(note that the horizontal axis is not $n$, but $M+N+1$).

\begin{figure}[htbp]
\centering
\includegraphics[scale=.9]{Fig1.eps}
\caption{Observed error and error bound for the function $f_1$ in~\eqref{eq:f_1}.}
\label{fig:1}
\end{figure}
\begin{figure}[htbp]
\centering
\includegraphics[scale=.9]{Fig2.eps}
\caption{Observed error and error bound for the function $f_2$ in~\eqref{eq:f_2}.}
\label{fig:2}
\end{figure}

\section{Proofs}
\label{sec4}

\subsection{Sketch of the proofs}

In this section, we provide the proofs
for Theorems~\ref{thm:new1} and~\ref{thm:new2}.
In both cases,
as outlined in Section~\ref{sec1},
we divide the approximation error into
the sum of the
discretization and truncation errors.
More precisely, we estimate the error as
\begin{align*}
& \left\lvert
  F(x) - \sum_{k=-M}^N F(kh)\sinc\left(\frac{x - kh}{h}\right)
 \right\rvert\\
&\leq
 \left\lvert
  F(x) - \sum_{k=-\infty}^{\infty} F(kh)\sinc\left(\frac{x - kh}{h}\right)
 \right\rvert\\
&\quad+\left\lvert
\sum_{k=-\infty}^{-M-1}F(kh)\sinc\left(\frac{x - kh}{h}\right)
+
\sum_{k=N+1}^{\infty}F(kh)\sinc\left(\frac{x - kh}{h}\right)
 \right\rvert.
\end{align*}
As for the first term (the discretization error),
we use the following bound.

\begin{theorem}[Okayama~{\cite[Part of Theorem 4.8]{Oka1}}]
\label{thm:discrete}
Assume that $F\in\LC_{L,R,\alpha,\beta}(\domD_d)$.
Then, it holds that
\begin{align*}
& \sup_{x\in\mathbb{R}}\left\lvert
F(x) - \sum_{k=-\infty}^{\infty}F(kh)\sinc\left(\frac{x - kh}{h}\right)
\right\rvert\\
&\leq \frac{4L}{\pi^2 d \mu(1 - \ee^{-2\pi d/h})\cos^{\alpha+\beta}((\pi/2)\sin d)\cos d}\ee^{-\pi d/h},
\end{align*}
where $\mu=\min\{\alpha,\beta\}$.
\end{theorem}

As for the second term (the truncation error),
we give the following bound.
The proof is provided in Section~\ref{lem:new}.

\begin{lemma}
\label{lem:new}
Assume that $F\in\LC_{L,R,\alpha,\beta}(\domD_d)$.
Let $\mu=\min\{\alpha,\beta\}$,
and let $M$ and $N$ be defined as in~\eqref{eq:improve-M-N-2}.
Then, it holds that
\begin{align*}
&\sup_{x\in\mathbb{R}}\left\lvert
\sum_{k=-\infty}^{-M-1}F(kh)\sinc\left(\frac{x - kh}{h}\right)
+
\sum_{k=N+1}^{\infty}F(kh)\sinc\left(\frac{x - kh}{h}\right)
 \right\rvert\\
&\leq \frac{2R}{\pi\mu h\sqrt{1 + \{q(dn/\mu)\}^2}}\ee^{-\pi\mu q(dn/\mu)}.
\end{align*}
\end{lemma}

We then equate the exponential rates of the two errors:
$\ee^{-\pi d/h}$ and $\ee^{-\pi\mu q(dn/\mu)}$.
In Theorem~\ref{thm:new1},
we set the mesh size $h$ as in~\eqref{eq:improve-h-DE-1}.
In this case, the formula~\eqref{eq:improve-M-N-2}
is rewritten as~\eqref{eq:improve-M-N-1},
from which $n = \max\{M, N\}$ holds.
Furthermore,
we have
\begin{align*}
 (\text{the discretization error}) &= \OO(\ee^{-\pi d n/\arsinh(q(dn/\mu))}),\\
 (\text{the truncation error}) &= \OO(\ee^{-\pi d n/\arsinh(dn/\mu)}),
\end{align*}
and the latter order is slightly worse than the former.
Therefore, the convergence rate of the overall error is given as
$\OO(\ee^{-\pi d n/\arsinh(dn/\mu)})$.
The detailed proof is provided in Section~\ref{sec:proof1}.

In Theorem~\ref{thm:new2},
we set the mesh size $h$ as in~\eqref{eq:improve-h-DE-2}.
In this case, although $n=\max\{M, N\}$ does not hold,
we have
\begin{align*}
 (\text{the discretization error}) &= \OO(\ee^{-\pi d n/\arsinh(dn/\mu)}),\\
 (\text{the truncation error}) &= \OO(\ee^{-\pi d n/\arsinh(dn/\mu)}),
\end{align*}
and both orders are exactly the same.
Thus, the convergence rate of the overall error is given as
$\OO(\ee^{-\pi d n/\arsinh(dn/\mu)})$.
The detailed proof is provided in Section~\ref{sec:proof2}.

\subsection{Proof of Lemma~\ref{lem:new}}

We show Lemma~\ref{lem:new} as follows.
\begin{proof}
First, using $\abs{\sinc x}\leq 1$, we have
\begin{align*}
&\left\lvert
\sum_{k=-\infty}^{-M-1}F(kh)\sinc\left(\frac{x - kh}{h}\right)
+
\sum_{k=N+1}^{\infty}F(kh)\sinc\left(\frac{x - kh}{h}\right)
 \right\rvert\\
&\leq \sum_{k=-\infty}^{-M-1}\abs{F(kh)} + \sum_{k=N+1}^{\infty}\abs{F(kh)}.
\end{align*}
Using~\eqref{eq:double-exp-decay}, we estimate the second term as
\begin{align*}
 \sum_{k=N+1}^{\infty}\abs{F(kh)}
&\leq R\sum_{k=N+1}^{\infty}
\frac{1}{(1+\ee^{-\pi\sinh(kh)})^{\alpha}(1+\ee^{\pi\sinh(kh)})^{\beta}}\\
&=R\sum_{k=N+1}^{\infty}
\frac{\ee^{-\pi\beta\sinh(kh)}}{(1+\ee^{-\pi\sinh(kh)})^{\alpha+\beta}}\\
&\leq R\sum_{k=N+1}^{\infty}
\frac{\ee^{-\pi\beta\sinh(kh)}}{(1 + 0)^{\alpha+\beta}}\\
&\leq \frac{R}{h}\int_{Nh}^{\infty}\ee^{-\pi\beta\sinh x}\dd x\\
&\leq \frac{R}{h\pi\beta \cosh(Nh)}
\int_{Nh}^{\infty}\pi\beta\cosh(x)\ee^{-\pi\beta\sinh x}\dd x\\
&= \frac{R}{h\pi\beta\cosh(Nh)}\ee^{-\pi\beta\sinh(Nh)}.
\end{align*}
Furthermore, using~\eqref{eq:improve-M-N-2}, we have
\begin{align*}
& \frac{R}{h\pi\beta\cosh(Nh)}\ee^{-\pi\beta\sinh(Nh)}\\
&\leq \frac{R}{h\pi\beta\cosh(\arsinh((\mu/\beta)q(dn/\mu)))}
\ee^{-\pi\beta\sinh(\arsinh((\mu/\beta)q(dn/\mu)))}\\
&=\frac{R}{h\pi\beta\sqrt{1 + \{(\mu/\beta)q(dn/\mu)\}^2}}
\ee^{-\pi\mu q(dn/\mu)}\\
&=\frac{R}{h\pi\sqrt{\beta^2 + \mu^2\{q(dn/\mu)\}^2}}
\ee^{-\pi\mu q(dn/\mu)}\\
&\leq\frac{R}{h\pi\sqrt{\mu^2 + \mu^2\{q(dn/\mu)\}^2}}
\ee^{-\pi\mu q(dn/\mu)}\\
&=\frac{R}{h\pi\mu\sqrt{1 + \{q(dn/\mu)\}^2}}
\ee^{-\pi\mu q(dn/\mu)}.
\end{align*}
In the same way,
using~\eqref{eq:double-exp-decay} and~\eqref{eq:improve-M-N-2},
we estimate the first term as
\[
 \sum_{k=-\infty}^{-M-1}\abs{F(kh)}
\leq \frac{R}{h\pi\mu\sqrt{1 + \{q(dn/\mu)\}^2}}
\ee^{-\pi\mu q(dn/\mu)}.
\]
Thus, we obtain the desired inequality.
\end{proof}

\subsection{Proof of Theorem~\ref{thm:new1}}
\label{sec:proof1}

For the proof of Theorem~\ref{thm:new1},
we use the following two propositions.

\begin{proposition}[Okayama and Kurogi~{\cite[Proposition 8]{OkaKuro}}]
\label{prop:p-q-r}
Let $p(x)$ and $q(x)$ be defined by~\eqref{eq:p-x}
and~\eqref{eq:q-x}, respectively.
Then, $p(x)$ and $r(x) = p(x) - q(x)$
are monotonically increasing functions for $x\geq 0$.
\end{proposition}
\begin{proposition}
\label{prop:sinh}
For all $t\geq 0$, we have
\begin{equation}
 \sinh t\geq t \sinh\left(\frac{2}{\pi}t\right).
\label{eq:sinh}
\end{equation}
\end{proposition}
\begin{proof}
Setting $f(t)=\sinh(t) - t\sinh((2/\pi) t)$,
we show $f(t)\geq 0$ for $t\geq 0$, from which we obtain the conclusion.
Using
\begin{align*}
&\sinh t\\
&=\sinh\left(\frac{2}{\pi}t + \left(1 - \frac{2}{\pi}\right)t\right)\\
&=\sinh\left(\frac{2}{\pi}t\right)
\cosh\left(\left(1 - \frac{2}{\pi}\right)t\right)
+\cosh\left(\frac{2}{\pi}t\right)
\sinh\left(\left(1 - \frac{2}{\pi}\right)t\right)\\
&=\sinh\left(\frac{2}{\pi}t\right)
\cosh\left(\left(1 - \frac{2}{\pi}\right)t\right)
+\left\{\sinh\left(\frac{2}{\pi}t\right) + \exp\left(-\frac{2}{\pi}t\right)\right\}
\sinh\left(\left(1 - \frac{2}{\pi}\right)t\right)\\
&=\sinh\left(\frac{2}{\pi}t\right)
\exp\left(\left(1 - \frac{2}{\pi}\right)t\right)
+\exp\left(-\frac{2}{\pi}t\right)
\sinh\left(\left(1 - \frac{2}{\pi}\right)t\right),
\end{align*}
we have
\begin{align*}
 f(t) &= g(t)\sinh\left(\frac{2}{\pi}t\right) + h(t),
\end{align*}
where
\begin{align*}
 g(t) &= \exp\left(\left(1 - \frac{2}{\pi}\right)t\right) - t,\\
 h(t) &= \exp\left(-\frac{2}{\pi}t\right)
\sinh\left(\left(1 - \frac{2}{\pi}\right)t\right).
\end{align*}
Here, we set $\gamma$ and $\delta$ as
\begin{align*}
 \gamma &= \frac{\pi}{\pi - 2}\log\left(\frac{\pi}{\pi - 2}\right),\\
 \delta &= \frac{\pi}{2(\pi - 2)}\log\left(\frac{\pi}{4 - \pi}\right),
\end{align*}
which satisfy $g'(\gamma)=0$ and $h'(\delta)=0$.
In what follows, we consider three cases:
(i) $0\leq t < \pi \log 2$,
(ii) $\pi\log 2\leq t < \pi\log 3$, and
(iii) $\pi \log 3 \leq t$, and we show $f(t)\geq 0$ in each case.

(i) Consider the case $0\leq t < \pi \log 2$.
Because $g(t)$ is monotonically decreasing
for $t\in (0, \gamma)$,
noting $\pi\log 2< \gamma$,
we have
\[
 f(t) \geq g(\pi\log 2)\sinh\left(\frac{2}{\pi}t\right) + h(t).
\]
Furthermore, from $g(\pi\log 2)>0$,
$\sinh((2/\pi)t)\geq 0$ and $h(t)\geq 0$, we have $f(t)\geq 0$.

(ii) Consider the case $\pi\log 2\leq t < \pi\log 3$.
Because $g(t)$ has its minimum at $t = \gamma$,
noting $\gamma\in (\pi\log2, \pi\log 3)$,
we have
\[
 f(t)\geq g(\gamma)\sinh\left(\frac{2}{\pi}t\right) + h(t).
\]
Because $g(\gamma) < 0$,
$g(\gamma)\sinh((2/\pi)t)$ is a monotonically decreasing function.
Furthermore, because $h(t)$ is monotonically decreasing for $t\geq \delta$,
noting $\delta < \pi\log 2$, we have
\[
 g(\gamma)\sinh\left(\frac{2}{\pi}t\right) + h(t)
\geq g(\gamma)\sinh\left(\frac{2}{\pi}\cdot\pi\log 3\right) + h(\pi\log 3)
> 0.
\]

(iii) Consider the case $\pi \log 3 \leq t$.
Because $g(t)$ is monotonically increasing
for $t\geq \gamma$,
noting $\gamma < \pi\log 3$, we have
\[
 f(t)\geq g(\pi\log 3)\sinh\left(\frac{2}{\pi}t\right) + h(t).
\]
Furthermore, because $g(\pi\log 3)>0$,
$\sinh((2/\pi) t)>0$, and $h(t)>0$,
we have $f(t)>0$.
This completes the proof.
\end{proof}

Using these propositions,
we prove Theorem~\ref{thm:new1} as follows.

\begin{proof}
From Theorem~\ref{thm:discrete} and Lemma~\ref{lem:new},
substituting~\eqref{eq:improve-h-DE-1},
we have
\begin{align}
& \sup_{x\in\mathbb{R}}\left\lvert
F(x) - \sum_{k=-M}^N F(kh)\sinc\left(\frac{x - kh}{h}\right)
\right\rvert\nonumber\\
&\leq\frac{4L}{\pi^2 d \mu\cos^{\alpha+\beta}((\pi/2)\sin d)\cos d}
\cdot\frac{\ee^{-\pi\mu p(dn/\mu)}}{(1 - \ee^{-2\pi\mu p(dn/\mu)})}\nonumber\\
&\quad+ \frac{2 R (dn/\mu)}{\pi d\arsinh(q(dn/\mu))\sqrt{1 + \{q(dn/\mu)\}^2}}
\ee^{-\pi\mu q(dn/\mu)}.
\label{eq:bound-disc-trunc}
\end{align}
Here, for $t>0$, the inequality~\eqref{eq:sinh} is rewritten as
\begin{align*}
 \frac{\sinh t}{t}&\geq \sinh\left(\frac{2}{\pi}t\right)\\
\Leftrightarrow\quad
\arsinh\left(\frac{\sinh t}{t}\right)&\geq \frac{2}{\pi} t\\
\Leftrightarrow\quad
\frac{t}{\arsinh\left(\frac{\sinh t}{t}\right)}
&\leq \frac{\pi}{2}.
\end{align*}
Putting $t=\arsinh(dn/\mu)$ in the last inequality, we have
\[
 \frac{\arsinh(dn/\mu)}{\arsinh\left(\frac{dn/\mu}{\arsinh(dn/\mu)}\right)}
=\frac{\arsinh(dn/\mu)}{\arsinh(q(dn/\mu))}\leq \frac{\pi}{2}.
\]
Therefore, the second term in~\eqref{eq:bound-disc-trunc}
is bounded as
\begin{align*}
&\frac{2 R (dn/\mu)}{\pi d\arsinh(q(dn/\mu))\sqrt{1 + \{q(dn/\mu)\}^2}}
\ee^{-\pi\mu q(dn/\mu)}\\
&=\frac{2R}{\pi d}\cdot
\frac{\arsinh(dn/\mu)}{\arsinh(q(dn/\mu))}\cdot
\frac{q(dn/\mu)}{\sqrt{1 + \{q(dn/\mu)\}^2}}
\ee^{-\pi\mu q(dn/\mu)}\\
&\leq \frac{2R}{\pi d}\cdot\frac{\pi}{2}\cdot
\frac{q(dn/\mu)}{\sqrt{0 + \{q(dn/\mu)\}^2}}\ee^{-\pi\mu q(dn/\mu)}\\
&=\frac{R}{d}\ee^{-\pi\mu q(dn/\mu)}.
\end{align*}
Furthermore, from Proposition~\ref{prop:p-q-r},
the first term in~\eqref{eq:bound-disc-trunc}
is bounded as
\begin{align*}
&\frac{4L}{\pi^2 d \mu\cos^{\alpha+\beta}((\pi/2)\sin d)\cos d}
\cdot\frac{\ee^{-\pi\mu p(dn/\mu)}}{(1 - \ee^{-2\pi\mu p(dn/\mu)})}\\
&=\frac{4L}{\pi^2 d \mu\cos^{\alpha+\beta}((\pi/2)\sin d)\cos d}
\cdot
\frac{\ee^{-\pi\mu \{p(dn/\mu) - q(dn/\mu)\}}}{(1 - \ee^{-2\pi\mu p(dn/\mu)})}
\ee^{-\pi \mu q(dn/\mu)}\\
&\leq\frac{4L}{\pi^2 d \mu\cos^{\alpha+\beta}((\pi/2)\sin d)\cos d}
\cdot
\frac{\ee^{-\pi\mu \{p(d/\mu) - q(d/\mu)\}}}{(1 - \ee^{-2\pi\mu p(d/\mu)})}
\ee^{-\pi \mu q(dn/\mu)}.
\end{align*}
This completes the proof.
\end{proof}

\subsection{Proof of Theorem~\ref{thm:new2}}
\label{sec:proof2}

We prove Theorem~\ref{thm:new2} as follows.
\begin{proof}
From Theorem~\ref{thm:discrete} and Lemma~\ref{lem:new},
substituting~\eqref{eq:improve-h-DE-2},
we have
\begin{align}
& \sup_{x\in\mathbb{R}}\left\lvert
F(x) - \sum_{k=-M}^N F(kh)\sinc\left(\frac{x - kh}{h}\right)
\right\rvert\nonumber\\
&\leq\frac{4L}{\pi^2 d \mu\cos^{\alpha+\beta}((\pi/2)\sin d)\cos d}
\cdot\frac{\ee^{-\pi\mu q(dn/\mu)}}{(1 - \ee^{-2\pi\mu q(dn/\mu)})}\nonumber\\
&\quad+ \frac{2 R (dn/\mu)}{\pi d\arsinh(dn/\mu)\sqrt{1 + \{q(dn/\mu)\}^2}}
\ee^{-\pi\mu q(dn/\mu)}.
\label{eq:bound-disc-trunc-2}
%&=
%\frac{2}{\pi d}\left[
%\frac{2L}{\pi\mu\cos^{\alpha+\beta}((\pi/2)\sin d)\cos d (1 - \ee^{-2\pi\mu q(dn/\mu)})}
%+\frac{R q(dn/\mu)}{\sqrt{1 + \{q(dn/\mu)\}^2}}
%\right]\ee^{-\pi\mu q(dn/\mu)}.
\end{align}
The second term in~\eqref{eq:bound-disc-trunc-2}
is bounded as
\begin{align*}
&\frac{2 R (dn/\mu)}{\pi d\arsinh(dn/\mu)\sqrt{1 + \{q(dn/\mu)\}^2}}
\ee^{-\pi\mu q(dn/\mu)}\\
&=\frac{2R}{\pi d}\cdot\frac{q(dn/\mu)}{\sqrt{1 + \{q(dn/\mu)\}^2}}
\ee^{-\pi\mu q(dn/\mu)}\\
&\leq\frac{2R}{\pi d}\cdot\frac{q(dn/\mu)}{\sqrt{0 + \{q(dn/\mu)\}^2}}
\ee^{-\pi\mu q(dn/\mu)}\\
&=\frac{2R}{\pi d}\ee^{-\pi\mu q(dn/\mu)}.
\end{align*}
Furthermore,
% from Proposition~\ref{prop:p-q-r},
because $q(x)$ monotonically increases for $x\geq 0$,
the first term in~\eqref{eq:bound-disc-trunc-2}
is bounded as
\begin{align*}
&\frac{4L}{\pi^2 d \mu\cos^{\alpha+\beta}((\pi/2)\sin d)\cos d}
\cdot\frac{\ee^{-\pi\mu q(dn/\mu)}}{(1 - \ee^{-2\pi\mu q(dn/\mu)})}\\
&\leq\frac{4L}{\pi^2 d \mu\cos^{\alpha+\beta}((\pi/2)\sin d)\cos d}
\cdot\frac{\ee^{-\pi\mu q(dn/\mu)}}{(1 - \ee^{-2\pi\mu q(d/\mu)})}.
\end{align*}
This completes the proof.
\end{proof}

\subsection{Remark on the function $q(x)$}

So far, we have defined the function $q(x)$ by~\eqref{eq:q-x}.
However, in view of the proofs above,
one may think that we can choose another function for $q(x)$.
That is certainly true.
For instance, if we define $q(x)$ as $q(x)=x$,
we can derive the following theorem instead of Theorem~\ref{thm:new2}.
The proof is omitted here because it can be shown in exactly the same way
as Theorem~\ref{thm:new2}.

\begin{theorem}
\label{thm:new3}
Assume that $F\in\LC_{L,R,\alpha,\beta}(\domD_d)$.
Let $\mu=\min\{\alpha,\beta\}$,
let $h$ be defined as
\begin{equation}
h = \frac{1}{n},
\label{eq:improve-h-DE-3}
\end{equation}
and let $M$ and $N$ be defined as
\begin{align}
\begin{aligned}
  M&=\left\lceil\frac{1}{h}\arsinh\left(\frac{dn}{\alpha}\right)\right\rceil,
 &N&= \left\lceil\frac{1}{h}\arsinh\left(\frac{dn}{\beta}\right)\right\rceil.
\end{aligned}
\label{eq:improve-M-N-3}
\end{align}
Then, it holds that
\[
 \sup_{x\in\mathbb{R}}
\left\lvert
F(x) - \sum_{k=-M}^N F(kh)\sinc\left(\frac{x - kh}{h}\right)
\right\rvert
\leq C \ee^{-\pi d n},
\]
where $C$ is a constant independent of $n$, expressed as
\[
C= \frac{2}{\pi d}
\left[
\frac{2L}{\pi\mu(1 - \ee^{-2\pi d})\cos^{\alpha+\beta}((\pi/2)\sin d)\cos d}
+ R
\right].
\]
\end{theorem}

The convergence rate given by this theorem is
$\OO(\ee^{-\pi d n})$, which appears to be faster than
that by Theorem~\ref{thm:new2}: $\OO(\ee^{-\pi d n/\arsinh(dn/\mu)})$.
However, it is important to note that these rates are given
with respect to $n$,
not with respect to the number of function evaluations ($M+N+1$).
According to~\eqref{eq:improve-h-DE-3}
and~\eqref{eq:improve-M-N-3},
we derive $M=N=\OO(n\log n)$,
which grows more rapidly than $n$.

In this way, by choosing a higher order function for $q(x)$,
we can obtain a higher convergence rate with respect to $n$.
However, in exchange for the higher convergence rate,
we obtain higher growth of $M+N+1$.
As a consequence, we cannot improve the convergence rate
with respect to $M+N+1$.
Although Theorem~\ref{thm:new3} is meaningful
as an error bound with simpler formulas of $h$, $M$, and $N$,
we should note that the convergence rate is given
not with respect to $M+N+1$ but with respect to $n$.

\backmatter

%\bmhead{Supplementary information}
%
%If your article has accompanying supplementary file/s please state so here. 
%
%Authors reporting data from electrophoretic gels and blots should supply the full unprocessed scans for key as part of their Supplementary information. This may be requested by the editorial team/s if it is missing.
%
%Please refer to Journal-level guidance for any specific requirements.

\bmhead{Acknowledgments}
This work was partially supported by the
JSPS Grant-in-Aid for Young Scientists (B)
Number JP17K14147.

%\section*{Statements and Declarations}
%
%\begin{itemize}
% \item \textbf{Competing Interests:}
% Authors are required to disclose financial or non-financial interests
% that are directly or indirectly related to the work submitted for publication.
% Please refer to ``Competing Interests and Funding'' below for more information
% on how to complete this section.
%\end{itemize}

%%===========================================================================================%%
%% If you are submitting to one of the Nature Portfolio journals, using the eJP submission   %%
%% system, please include the references within the manuscript file itself. You may do this  %%
%% by copying the reference list from your .bbl file, paste it into the main manuscript .tex %%
%% file, and delete the associated \verb+\bibliography+ commands.                            %%
%%===========================================================================================%%

\bibliography{sn-bibliography}% common bib file
%% if required, the content of .bbl file can be included here once bbl is generated
%%\input sn-article.bbl

%% Default %%
%%\input sn-sample-bib.tex%

\end{document}
