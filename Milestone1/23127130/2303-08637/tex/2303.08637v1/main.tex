\documentclass[prb,aps,twocolumn,superscriptaddress,10pt]{revtex4-2}
\usepackage[utf8]{inputenc}
\usepackage[T1]{fontenc}
\usepackage{color}
\usepackage{amsmath}
\usepackage{amsfonts}
\usepackage{xr-hyper}
\usepackage{hyperref}
\usepackage{amsthm}
\usepackage{bm}
\usepackage{float}
\usepackage{textcomp}
\usepackage{upgreek}
\usepackage{textgreek}
\usepackage{setspace}
\usepackage[percent]{overpic}
\usepackage[table]{xcolor}


\hypersetup{
  colorlinks = true,
  allcolors= blue,
}
\usepackage{tikz,pgfplots}
\usepackage{blkarray}
\usetikzlibrary{patterns}
\usetikzlibrary{fadings}
\usetikzlibrary{decorations.pathmorphing,patterns}
\tikzset{snake it/.style={decorate, decoration=snake}}
\usetikzlibrary{arrows, decorations.markings}
\pgfplotsset{compat=1.14}
\tikzset{
vecArrow/.style={
  thick,
  decoration={markings,mark=at position
   1 with {\arrow[scale=2,thin]{open triangle 60}}},
  double distance=1.4pt, shorten >= 10.5pt,
  preaction = {decorate},
  postaction = {draw,line width=1.4pt, white,shorten >= 4.5pt}
  },
innerWhite/.style={
  semithick,
  white,
  line width=1.4pt,
  shorten >= 4.5pt
  }
}

\definecolor{orange}{rgb}{1,0.5,0}
\definecolor{darkgreen}{rgb}{0,0.4,0.1}
\definecolor{cola}{HTML}{5FBDFF}
\definecolor{colb}{HTML}{118DFE}
\definecolor{cold}{HTML}{EEFF0C}
\definecolor{cole}{HTML}{FC9E0A}


\newcommand{\todor}[1]{{\color{black}{\bf #1}}}
\newcommand{\todoo}[1]{{\color{orange}{\bf #1}}}
\newcommand{\todob}[1]{{\color{blue}{\bf #1}}}
\newcommand{\revadd}[1]{{\color{black}{#1}}}
\newcommand{\revrem}[1]{{\color{blue}{\sout{#1}}}}
\newcommand{\revision}[2]{{\color{blue}{{\sout{#1}}}} {\color{black}{#2}} }
\newcommand{\WidthFigure}{\columnwidth}
\newcommand{\folder}{./}

\makeatletter
\newcommand{\doublehat}[1]{% 
\begingroup%
  \let\macc@kerna\z@%
  \let\macc@kernb\z@%
  \let\macc@nucleus\@empty%
  \hat{\raisebox{.3ex}{\vphantom{\ensuremath{#1}}}\smash{\hat{#1}}}%
\endgroup%
}
\makeatother

\makeatletter
\newcommand{\doublehatSub}[1]{% 
\begingroup%
  \let\macc@kerna\z@%
  \let\macc@kernb\z@%
  \hat{\raisebox{-.07ex}{\vphantom{\ensuremath{#1}}}\smash{\hat{#1}}}%
\endgroup%
}
\makeatother

\DeclareRobustCommand{\utilde}[1]{{\underaccent{\tilde}{#1}}}
\DeclareRobustCommand{\atilde}[1]{{\tilde{#1}}}
\makeatletter
\DeclareFontFamily{OMX}{MnSymbolE}{}
\DeclareSymbolFont{MnLargeSymbols}{OMX}{MnSymbolE}{m}{n}
\SetSymbolFont{MnLargeSymbols}{bold}{OMX}{MnSymbolE}{b}{n}
\DeclareFontShape{OMX}{MnSymbolE}{m}{n}{
    <-6>  MnSymbolE5
   <6-7>  MnSymbolE6
   <7-8>  MnSymbolE7
   <8-9>  MnSymbolE8
   <9-10> MnSymbolE9
  <10-12> MnSymbolE10
  <12->   MnSymbolE12
}{}
\DeclareFontShape{OMX}{MnSymbolE}{b}{n}{
    <-6>  MnSymbolE-Bold5
   <6-7>  MnSymbolE-Bold6
   <7-8>  MnSymbolE-Bold7
   <8-9>  MnSymbolE-Bold8
   <9-10> MnSymbolE-Bold9
  <10-12> MnSymbolE-Bold10
  <12->   MnSymbolE-Bold12
}{}

\let\llangle\@undefined
\let\rrangle\@undefined
\DeclareMathDelimiter{\llangle}{\mathopen}%
                     {MnLargeSymbols}{'164}{MnLargeSymbols}{'164}
\DeclareMathDelimiter{\rrangle}{\mathclose}%
                     {MnLargeSymbols}{'171}{MnLargeSymbols}{'171}
\makeatother
\DeclareUnicodeCharacter{300}{à}
\newcommand{\Fockket}[1]{\left\Vert\hspace*{-0.6mm}\{\hspace*{-0.6mm}n{#1}_{\vec{\bm{q}}_{c#1}\vec{s}_{c#1} }\hspace*{-0.6mm}\}\hspace*{-0.6mm}\right\rrangle}
\newcommand{\Fockbra}[1]{\left\llangle\hspace*{-0.6mm}\{\hspace*{-0.6mm}{n#1}_{\vec{\bm{q}}_{c#1}\vec{s}_{c#1} }\hspace*{-0.6mm}\}\hspace*{-0.6mm}\right\Vert}

\DeclareMathAlphabet{\mathsfit}{\encodingdefault}{\sfdefault}{m}{sl}
\SetMathAlphabet{\mathsfit}{bold}{\encodingdefault}{\sfdefault}{bx}{sl}
\newcommand{\tens}[1]{\bm{\mathsfit{#1}}}
\newcommand{\tenscomp}[1]{\mathsfit{#1}}
\newcommand{\Trho}{\text{\textrho}}

\makeatletter
\usepackage{comment}
\let\wfs@comment@comment\comment
\let\comment\@undefined

\usepackage[final]{changes}
\let\wfs@changes@comment\comment
\let\comment\@undefined

\newcommand\comment{%
    \ifthenelse{\equal{\@currenvir}{comment}}
    {\wfs@comment@comment}
    {\wfs@changes@comment}%
}
\makeatother

\definecolor{dgreen}{rgb}{0,0.45,0}


\makeatletter
\@namedef{Changes@AuthorColor}{red}
\colorlet{Changes@Color}{red}
\setaddedmarkup{\textcolor{orange}{#1}}
\makeatother


\setlength {\marginparwidth }{2cm} 
\begin{document}


\title{
Vibrational and thermal properties of amorphous alumina from first principles
}

\author{Angela F. Harper}
\altaffiliation{Current address: Fritz-Haber Institute of the Max Planck Society, Berlin (DE)}
\affiliation{Theory of Condensed Matter Group, Cavendish Laboratory, University of Cambridge (UK)}

\author{Kamil Iwanowski}
\affiliation{Theory of Condensed Matter Group, Cavendish Laboratory, University of Cambridge (UK)}

\author{Mike C. Payne}
\affiliation{Theory of Condensed Matter Group, Cavendish Laboratory, University of Cambridge (UK)}

\author{Michele Simoncelli}
\email{ms2855@cam.ac.uk}
\affiliation{Theory of Condensed Matter Group, Cavendish Laboratory, University of Cambridge (UK)}


\begin{abstract}
Alumina is historically a difficult material to characterize as a result of both its amorphous nature and the large range of densities and local atomic coordination topologies present. Alumina is ubiquitously employed as a high-dielectric-constant material in electronic devices, thus understanding the microscopic physics governing its structural and heat-transport properties is of key relevance for e.g. the design of heat management in these devices.
Here we rely on first-principles techniques to characterize the structural, vibrational, and thermal properties of amorphous alumina across a wide range of densities, ranging from the record-low value of 2.28 g/cm$^3$ to the record-high value of 3.49 g/cm$^3$.
We show that the large variety of atomic coordination topologies coexisting in this material are responsible for the emergence of significant structural disorder at the sub-nanometer length scale. 
We apply the recently developed Wigner formulation of thermal transport in solids to characterize 
how the interplay between such strong topological disorder and anharmonicity affects heat transport, showing that topological disorder dominates over anharmonicity in determining the thermal conductivity of amorphous alumina.
\end{abstract}
\maketitle


\section{Introduction} % (fold)
\label{sec:intro}

Alumina (Al$_2$O$_3$) is a deceptively complex material; there are at least 9 known metastable polymorphs \cite{levin_1998_polymorphs}, an amorphous phase which possesses its own unique properties as a result of local disorder \cite{tavakoli2013amorphous}, and a number of predicted non-stoichiometric structures \cite{li_effects_2020}. 
Alumina has applications in catalysis \cite{shafiq2022recent,oh2019sustainable}, energy storage devices including Li-ion batteries \cite{jin2019li4ti5o12}, Li-metal anodes \cite{qu2019air}, and solid state electrolytes \cite{randau2021additive}, as well as many other diverse technological applications \cite{paterson_thermal_2020,mavric_advanced_2019}. 
In particular, amorphous alumina (am-Al$_2$O$_3$) deposited by atomic layer deposition (ALD) has become a subject of extensive research, as the ALD process allows experimentalists to deposit individual atomic layers of material on a device, with precise control over film thickness and morphology. 
Especially in thin films, nanoparticles, and layered devices, the thermal properties of alumina are a key driver for optimal device operation \cite{paterson_thermal_2020,scott2018thermalal2o3}.


Despite these advances in device performance, there are several open fundamental questions on how the structural properties of am-Al$_2$O$_3$ affect its heat-transport properties. Solid-state $^{27}$Al nuclear magnetic resonance identifies the three main aluminum coordination environments in am-Al$_2$O$_3$ \cite{fharper_modelling_2023,Lee_2010_NMR,sarou_2013_thermal,hashimoto_structure_2022}, which play a role in the electronic transport as evidenced by X-Ray absorption spectroscopy and first principles electronic density of states calculations \cite{Dicks_2019,Leung_2021}. {However, neither of these experimental techniques are able to distinguish oxygen coordination environments in am-Al$_2$O$_3$.}
The literature on the relationships between structural and thermal properties of am-Al$_2$O$_3$ at technologically relevant temperatures (\textit{i.e.} at $T> 30$ K, in the so called above-the-plateau regime \cite{cahill_thermal_1987,allen1989thermal,simoncelli_thermal_2022}) is less developed.
Recently, \citet{li_effects_2020}  modeled several am-AlO$_x$ phases using machine learned force-fields and computed the room-temperature thermal conductivity using Green-Kubo (GKMD) and non-equilibrium molecular dynamics (NEMD) approaches. While providing novel  insights on the physics governing thermal transport in amorphous alumina at room temperature, this work could not explore how the thermal conductivity varies with temperature---in fact, GKMD and NEMD are both governed by classical equipartition and thus yield progressively less accurate predictions as temperature decreases (\textit{i.e.} when the specific heat deviates significantly from the classical limit) \cite{PhysRevMaterials.3.085401}. 

Here we employ first-principles calculations to characterize the structural, vibrational, and thermal properties of am-Al$_2$O$_3$ across a wide range of densities ($2.28{\leq}\rho{\leq}3.49$ g/cm$^3$). In Sec.~\ref{sec:structural_properties_of_amor} we discuss how the atomic coordination topology of am-Al$_2$O$_3$ changes with density, and how these changes affect the vibrational properties.
These vibrational properties are then used in Sec.~\ref{sec:thermal_properties} to parametrize the recently developed Wigner formulation for thermal transport \cite{simoncelli2019unified,simoncelli2021Wigner}, which can be used to calculate heat conduction in amorphous solids accounting comprehensively for the effects of structural disorder, anharmonicity, and quantum Bose-Einstein statistics of vibrations \cite{simoncelli_thermal_2022}. We employ the convergence-acceleration protocol discussed in Ref.~\cite{simoncelli_thermal_2022} to compute the bulk limit of the thermal conductivity of am-Al$_2$O$_3$, highlighting the agreement between our calculations and experiments performed at various densities and temperatures.
We describe how the interplay between structural disorder and anharmonicity affects thermal transport in am-Al$_2$O$_3$, showing that the thermal conductivity of am-Al$_2$O$_3$ is mainly determined by structural disorder at all the densities and temperatures studied ($50{\leq}T{\leq}700$ K).
We investigate the microscopic mechanisms that govern the increase of the conductivity with density, showing that the thermal conductivity increase is caused by a simultaneous increase  of the vibrational density of states and of the diffusivity of vibrational modes.
This work provides fundamental insights into the microscopic physics governing thermal transport in am-Al$_2$O$_3$, using the recently developed Wigner formulation \cite{simoncelli2019unified,simoncelli2021Wigner} to elucidate how structural disorder and anharmonicity affect the magnitude and temperature dependence of the thermal conductivity, thus providing information that could not be obtained from molecular-dynamics-based methods governed by classical equipartition \cite{PhysRevMaterials.3.085401}.



\section{Structural and vibrational properties}
\label{sec:structural_properties_of_amor}

According to the literature \cite{aarhammar2011unveiling,li_effects_2020}, the density of am-Al$_2$O$_3$ can be as low as 2.1 g/cm$^3$ or as high as 3.6 g/cm$^3$. In order to generate structures throughout this range, we chose to extract the structures from Ref. \cite{fharper_modelling_2023} at 3.17 g/cm$^3$, 3.30 g/cm$^{3}$ and 3.49 g/cm$^{3}$, and additionally generate two low-density am-Al$_2$O$_3$ phases using the melt-quench procedure as laid out in Ref. \cite{fharper_modelling_2023}, at 2.28 g/cm$^3$ and 2.98 g/cm$^{3}$. Each 120-atom structure was generated using \textit{ab initio} molecular dynamics, initially melted at a temperature of 4000\,K, then quenched to a temperature of 300\,K, and finally equilibrated in the NVT ensemble (see Appendix~\ref{sec:computational_details} for details). 
To the best of our knowledge, our 2.28 g/cm$^3$ model of am-Al$_2$O$_3$ has a record-low density, as the lowest density model we found in the literature is the 2.9 g/cm$^3$ model from \citet{lizarraga_2011_lowdensity}. Moreover, the 3.49 g/cm$^{3}$ am-Al$_2$O$_3$ structure from \cite{fharper_modelling_2023} is also the highest density model structure, with the next highest density structure from \citet{tane_2011_nanovoid} at 3.34 g/cm$^3$. 
We believe that such a wide range of model densities will span the rich variety of local atomic environments and coordination topologies present in am-Al$_2$O$_3$, thus allowing us to explore how these 
affect the vibrational and thermal properties.

\subsection{Coordination Topology} % (fold)
\label{sub:coordination_topology}

The amorphous phase of Al$_2$O$_3$ is a disordered solid with perhaps the richest known coordination topology \cite{fharper_modelling_2023,hashimoto_structure_2022,shi2019structure}. To the best of our knowledge, the relationship between the density and coordination environments of am-Al$_2$O$_3$, especially at densities close to extreme values of 2.1 and 3.6 g/cm$^3$, has not previously been extensively discussed.
Therefore, we start by characterizing the coordination topology of our am-Al$_2$O$_3$ models with densities ranging from 2.28 to 3.49 g/cm$^3$, showing in Fig.~\ref{fig:coord} the probability distribution of finding $x$-fold-coordinated oxygen ($\rm O_x$) or $y$-fold-coordinated aluminum ($\rm Al_y$) in our models.
We found that six different coordination environments are present in our models---$\rm O_x$ with $x\in 2,3,4$, and $\rm Al_y$ with $y\in 4,5,6$---and at least five out of these six coordination topologies coexist at every model density analyzed.

\begin{figure}[t]
  \centering
\includegraphics[width=\WidthFigure]{\folder/amorphous_alumina_coordination_histogram.pdf}
  \caption{\textbf{Coordination of oxygen and aluminum in am-Al$_2$O$_3$ at various densities.} 
  The histograms show the proportion of oxygen (top) or aluminum (bottom) with a certain coordination number in the disordered atomistic models analyzed (different colors distinguish different densities).
  Coordination varies from 2 to 4 for O, and from 4 to 6 for Al.}
  \label{fig:coord}
\end{figure}


As shown in the top panel of Fig. \ref{fig:coord}, at increasing densities
the concentration of twofold-coordinated oxygens (O$_2$) generally decreases in favor of fourfold-coordinated oxygens (O$_4$);
changes in density have weaker effects on the concentration of threefold-coordinated oxygens (O$_3$).
Turning our attention to Al (see the bottom panel of Fig. \ref{fig:coord})  we see that increasing density yields a reduction in the concentration of Al$_4$ environments, which is compensated by an increase of Al$_6$ environments. In contrast, the Al$_5$ environment displays weaker variations with density; we note that these Al$_5$ environments are characteristic of the amorphous phase of Al$_2$O$_3$ and absent in the crystalline phases \cite{paz_identification_2014}.

Finally, we highlight how the ultra-low-density (2.28 g/cm$^3$) model contains a high proportion of O$_2$ bridging oxygens and a majority of Al$_4$ environments, with no Al$_6$ environments. This is consistent with the trend that at lower densities Al becomes primarily tetrahedrally coordinated in Al$_4$ environments \cite{mavric_advanced_2019}. 


The coexistence (mixing) of at least five different coordination environments in all the am-Al$_2$O$_3$ models studied underscores the presence of strong structural disorder in this material. To quantify these expectations, we plot in Fig.~\ref{fig:al2o3_rdf} the total radial distribution function (RDF) for all the models analyzed. We highlight how the RDF converges to one over a distance shorter than 5 \AA, lacking clear signatures of medium-range order (\textit{i.e.} of oscillations in the RDF in the range 5-20 \AA \space \cite{elliot_mro}) and thus indicating that the typical lengthscale of structural disorder in am-Al$_2$O$_3$ is comparable to the size of the topological units (we recall that the typical Al-O bonding distance spans the range from 1.7 to 2.1 \AA \cite{young_probing_2020}).
We note in passing that the linear size of our 120-atom models is always larger than 10 \AA, further supporting our assertion that atomistic models containing hundreds of atoms are sufficiently large to capture the structural properties of strongly disordered solids \cite{simoncelli_thermal_2022,Fiorentino_2023}.


\begin{figure}[t!]
  \centering
\includegraphics[width=\WidthFigure]{\folder/alumina_radial_distribution_function3.pdf}\\[-3mm]
  \caption{
  \textbf{Radial distribution function at various densities.}  Solid lines, total RDF of the am-Al$_2$O$_3$ generated and studied in this work (a rigid shift is used to distinguish the RDF of different models).
   We highlight how the RDF of am-Al$_2$O$_3$ converges to one (horizontal black lines) over a distance shorter than 5 \AA,  indicating the presence of strong structural disorder~\cite{elliot_mro}.
  }
  \label{fig:al2o3_rdf}
\end{figure}


\subsection{Vibrational properties} % (fold)
\label{sub:vibrational_properties}

The strong variations in the coordination topology observed in the different structures of am-Al$_2$O$_3$ (Fig.~\ref{fig:coord}) are intuitively expected to affect the atomic vibrational excitations.
Thus, in this section we investigate quantitatively how the vibrational properties of am-Al$_2$O$_3$ change with the coordination topology. We used density-functional perturbation theory (DFPT) \cite{baroni_phonons_2001} to compute the energy $\hbar\omega_{\bm{q}s}$ of each vibrational mode $\bm{q}s$, and the corresponding displacement pattern $\mathcal{E}_{\bm{q}s}^{b\alpha}$, which describes how atom $b$ moves in direction $\alpha$ when the vibration $\bm{q}s$ is excited. We label vibrational modes using both the wavevector $\bm{q}$ and the mode index $s$ for the sake of generality, keeping in mind that $\bm{q}$ is necessary only in crystals and in finite-size models of disordered solids, while in ``ideal glasses'' (namely, an astronomically large set of atoms whose arrangement lacks long-range order) one would obtain $\bm{q}=\bm{0}$ only and $s$ would be the only label for vibrations \cite{simoncelli_thermal_2022}.
Then, we use these quantities to investigate if there is a relationship between atomic displacements and coordination topology, computing the root mean square displacement of every atom $b$ \cite{wallace1998thermodynamics}, 
\begin{equation}
\label{eq:RMSD}
    \text{RMSD}(b,T){=}\sqrt{\frac{\hbar}{N_c m_b} \sum_{\bm{q}s} \frac{1}{\omega_{\bm{q}s}}\Big[\tfrac{1}{2} + \tenscomp{N}_{\bm{q}s} \Big] \sum_{\alpha} |\mathcal{E}_{\bm{q}s}^{b\alpha}|^2},
    \raisetag{5pt}
\end{equation}
where  $\tenscomp{N}_{\bm{q}s} {=} [{\exp{(\hbar \omega_{\bm{q}s} / k_B T)} {-} 1}]^{-1}$ is the Bose-Einstein distribution at temperature $T$, $m_b$ is the mass of atom $b$, and $N_c$
is a normalization factor\footnote{N is equal to the number of $\bm{q}$ points used to sample the Brillouin Zone; in the case of an ideal glass (astronomically large disordered simulation cell) the Brillouin Zone reduces to the point $\bm{q}=\bm{0}$ only and $N=1$.}.
\begin{figure}[b!]
  \centering
\includegraphics[width=\WidthFigure]{\folder/amorphous_alumina_MSD.pdf}\\[-3mm]
  \caption{\textbf{Average atomic root mean square displacement in am-Al$_2$O$_3$ at various densities.} The average RMSD was calculated at  $T=300K$ using Eq.~(\ref{eq:RMSD}) as discussed in the main text. For oxygen the average RMSD decreases as coordination increases; aluminum displays a much weaker anticorrelation between RMSD and coordination number in models with $\rho \leq 3.17$ g/cm$^3$, such an anticorrelation disappears in models with $\rho \geq 3.30$ g/cm$^3$. Error bars represent the standard deviation of the average RMSD, and are not reported in the cases in which only a single coordination environment was detected.}
  \label{fig:MSD_atoms}
\end{figure}


To see if a relationship between RMSD and coordination topology exists, we compute the average ${\rm RMSD}(b,T)$ over atoms with equal coordination. Fig.~\ref{fig:MSD_atoms} shows that the average RMSD of O$_x$ atoms decreases as coordination increases in all the atomistic models analyzed;  the average RMSD of Al$_y$ is anticorrelated with coordination in models with $\rho{\leq} 3.17$ g/cm$^3$, and displays deviations from this anticorrelated trend in models with $\rho = 3.30$ g/cm$^3$ (where all Al atoms have a very similar average RMSD, regardless of their coordination) and with
$\rho = 3.49$ g/cm$^3$ (where Al$_6$ has average RMSD slightly larger than Al$_5$).
Interestingly, as density increases the average RMSD of O$_4$ approaches the RMSD of Al$_4$---considering the significant differences between the mass of oxygen and aluminum ($m_{Al}/m_O\approx 1.7$), we note that this behavior departs from the trend  RMSD $\propto \sqrt{{m_b}^{-1}}$  ubiquitously observed in solids with simple coordination topology. This shows that the presence of complex coordination topologies can have non-trivial effects on the vibrational properties.



Having investigated how the constraints imposed by the coordination topology affect the average RMSDs, we now study how coordination influences the vibrational frequencies. 
To this aim, we compute the vibrational density of states (VDOS), $g(\omega){=}({\mathcal{V}N_{\rm c}})^{-1}\sum_{\bm{q},s}\delta(\omega{-}\omega_{\bm{q}s})$ (here, $\mathcal{V}$ is the volume of the cell used to simulate the disordered system), we use the eigenvectors to decompose the VDOS into partial (single-atom) contributions (PDOS), and finally we integrate the single-atom PDOS using an indicator function that allows us to resolve different coordination environments: 
\begin{equation}
    \label{eq:pVDOS}
    g_{t_x}(\omega) = \frac{1}{\mathcal{V} N_c} \sum_{{\bm{q}s}} \delta(\omega - \omega_{\bm{q}s}) \sum_{b, \alpha} |\mathcal{E}^{b\alpha}_{{\bm{q}s}}|^2 \delta_{b , t_x},
\end{equation}
where $\delta_{b , t_x}$ is indicator function equal to one if the atom $b$ is of type $t$ and has coordination $x$ (e.g., $t_x=O_3$ for threefold-coordinated oxygen), and zero otherwise.
Thus, $g_{t_x}(\omega)$ allows us to resolve how the coordination topology affects the VDOS.


\begin{figure}[t]
  \centering
\includegraphics[width=\WidthFigure]{\folder/amorphous_alumina_dos_without_o5_all_5.pdf}\\[-3mm]
  \caption{
  \textbf{Vibrational density of states for various densities of am-Al$_2$O$_3$.} Solid black gives the total VDOS. The colored solid lines distinguish coordination environments for Al atoms: green is Al$_4$, red is Al$_5$, and yellow is Al$_6$. Dashed colored lines are used for coordination environments of O atoms: cyan for O$_2$, blue for O$_3$, and purple for O$_4$.
  }
  \label{fig:VDOS}
\end{figure}
Results for $g_{t_x}(\omega)$ are shown in Fig.~\ref{fig:VDOS}. 
We highlight how the PDOS for different coordination environments has a shape that is similar (in the mathematical sense) to models having different densities. In particular, different coordination environments for oxygen have fingerprints in the coordination-resolved PDOS: O$_3$ is characterized by a bimodal PDOS with a local minimum around 500 cm$^{-1}$, while O$_2$ and O$_4$ have a unimodal PDOS with peaks at about 150 and 500  cm$^{-1}$, respectively. In contrast, the coordination-resolved PDOS for Al$_4$, Al$_5$, and Al$_6$ are quite all similar in shape; we note that this is in sharp contrast to the crystalline $\theta$-Al$_2$O$_3$ phase, which contains only Al$_4$ tetrahedra and Al$_6$ octahedra and displays a clear distinction between high-frequency breathing modes associated with Al$_4$ environments, and low-frequency bending modes associated with Al$_6$ environments \cite{lodziana2003dynamical}. 


Varying the density of am-Al$_2$O$_3$ causes a variation in the relative magnitude of the Al$_x$ PDOS: at 2.28 g/cm$^3$, the Al$_4$ PDOS is always larger in magnitude than Al$_5$; at increasing density, the relative magnitude of the PDOS of Al$_4$ and Al$_5$ progressively reverses, with the highest-density 3.49 g/cm$^3$ model featuring a PDOS for Al$_5$ always larger than that for Al$_4$.

We note that the O$_3$ and Al$_5$ coordination environments have a significant PDOS in all the models analyzed, regardless of the models' density. 
In contrast, the PDOS of O$_2$, O$_4$, Al$_4$, Al$_6$, display much stronger variations with density---increasing density yields a progressive suppression of the O$_2$ and Al$_4$ environments, compensated by the progressive appearance of O$_4$ and Al$_6$ environments.  

In the next section we will discuss how these vibrational properties affect the thermal properties.



\section{Thermal properties} % (fold)
\label{sec:thermal_properties}

\subsection{Thermal conductivity of glasses} % (fold)
\label{ssec:Wigner_formulation}

In this section we summarize the salient features of the Wigner formulation of thermal transport \cite{simoncelli2019unified}, which describes heat conduction, accounting for the interplay between structural disorder, anharmonicity, and quantum Bose-Einstein statistics of atomic vibrations. This allows us to describe the thermal conductivity of solids ranging from crystals \cite{PhysRevX.10.011019,simoncelli2021Wigner,Lucente} to glasses \cite{simoncelli_thermal_2022,liu_unraveling_nodate}.

For amorphous systems, the Wigner formulation of thermal transport yields the following conductivity expression:
\begin{equation}
\begin{split}
\kappa{=}\frac{1}{\mathcal{V}{N_{\rm c}} }{\sum_{\bm{q},s,s'}}&\!
\frac{\omega_{\bm{q}s}{+}\omega_{\bm{q}s'}}{4}\!\left(\frac{C_{\bm{q}s}}{\omega_{\bm{q}s}}{+}\frac{C_{\bm{q}s'}}{\omega_{\bm{q}s'}}\right)\!
\frac{\rVert\tens{v}(\bm{q})_{s,s'}\lVert^2}{3}\\
\times&\pi\mathcal{F}_{[\eta;\Gamma_{\bm{q}s}{+}\Gamma_{\bm{q}s'}]}(\omega_{\bm{q}s}-\omega_{\bm{q}s'})\;,
\label{eq:thermal_conductivity_combined}
\end{split}
\raisetag{5mm}
\end{equation}
where $\hbar\omega_{\bm{q}s}$ denotes the energy of the vibration $\bm{q}s$ and $\hbar\Gamma_{\bm{q}s}$ its linewidth (energy perturbation to the harmonic energy level due to anharmonicity~\cite{paulatto2013anharmonic,fugallo2013ab,phono3py,alamode,cepellotti_phoebe_2022,carrete_almabte_2017,kaldo} and isotopic-mass disorder~\cite{tamura_isotope_1983}); 
$\rVert\tens{v}(\bm{q})_{s,s'}\lVert^2{=}\sum_{\alpha=1}^3\tenscomp{v}^{\alpha}(\bm{q})_{s,s'}\tenscomp{v}^{\alpha}(\bm{q})_{s',s}$
is the square modulus of the velocity operator \cite{simoncelli2021Wigner} between eigenstates $s$ and $s'$ at fixed $\bm{q}$ ($\alpha$ denotes a Cartesian direction, and since vitreous solids are in general isotropic, the scalar conductivity~(\ref{eq:thermal_conductivity_combined}) is computed as the average trace of the tensor $\kappa^{\alpha\beta}$ \cite{simoncelli_thermal_2022});
\begin{equation}
C_{\bm{q}s}{=}C[\omega_{\bm{q}s},T]{=}\frac{\hbar^2\omega^2_{\bm{q}s} }{k_{\rm B} T^2} {\tenscomp{N}}_{\bm{q}s}\big({\tenscomp{N}}_{\bm{q}s}{+}1\big)  
\label{eq:quantum_specific_heat_A}
\end{equation}
is the quantum specific heat of the mode $\bm{q}s$; 
$\mathcal{V}$, $N_{\rm c}$, and $\tenscomp{N}_{\bm{q}s}$ are, in order, the simulation's cell volume, normalization factor 
and Bose-Einstein distribution already discussed in Sec.~\ref{sub:vibrational_properties}.
Finally, $\mathcal{F}_{[\eta;\Gamma_{\bm{q}s}{+}\Gamma_{\bm{q}s'}]}(\omega_{\bm{q}s}-\omega_{\bm{q}s'})$ is a Voigt distribution, \textit{i.e.} a two-parameter distribution that reduces to a Lorentzian with  FWHM $\Gamma_{\bm{q}s}{+}\Gamma_{\bm{q}s'}$ when $\Gamma_{\bm{q}s}{+}\Gamma_{\bm{q}s'}\gg \eta$, and to a Gaussian representation of the Dirac delta with variance $\eta^2\pi/2$ in the opposite limit ($\Gamma_{\bm{q}s}{+}\Gamma_{\bm{q}s'}\ll \eta$). 

As discussed in detail in Ref.~\cite{simoncelli_thermal_2022}, Eq.~(\ref{eq:thermal_conductivity_combined}) describes thermal transport as originating from couplings between vibrations that have an energy difference smaller than the broadening of the Voigt profile. Such a broadening is determined by $\eta$ in the low-temperature (harmonic) limit where anharmonicity phases out ($\Gamma_{\bm{q}s}{+}\Gamma_{\bm{q}s'}\to 0\;\forall\;\bm{q},s$). Setting $\eta$ to a value slightly larger than the average energy-level spacing $\hbar\Delta\omega_{\rm avg}=\frac{\hbar\omega_{\rm max}}{3\cdot N_{\rm at}}$ ($\hbar\omega_{\rm max}$ is the maximum vibrational energy, $3N_{\rm at}$ is the number of vibrational energy levels, \textit{i.e.} three times the number of atoms in the system's reference cell) accounts for the physical property that heat transfer via a wave-like tunneling between neighboring (quasi-degenerate) vibrational eigenstates can occur~\footnote{A necessary condition for this to happen is to have vibrations that are not localized in the Anderson sense \cite{Anderson_localization}, \textit{i.e.} to have non-zero velocity operator elements in Eq.~(\ref{eq:thermal_conductivity_combined}).} even in the limit of vanishing anharmonicity, implying that in such a limit Eq.~(\ref{eq:thermal_conductivity_combined}) reduces to the harmonic Allen-Feldman (AF) thermal conductivity for glasses \cite{allen1989thermal,allen1993thermal}.
In contrast, at temperatures where the anharmonic linewidth are much larger than the computational broadening $\eta$, the Voigt profile becomes a Lorentzian with full width at half maximum (FWHM) $\Gamma_{\bm{q}s}{+}\Gamma_{\bm{q}s'}$, effectively reducing to the anharmonic Wigner conductivity expression~\cite{simoncelli2021Wigner}.
Overall, the Voigt profile in Eq.~(\ref{eq:thermal_conductivity_combined}) ensures that the low-temperature harmonic Allen-Feldman limit is correctly described, and the effects of anharmonicity are considered only when they are not affected by finite-size effects \cite{simoncelli2021Wigner}. 

We note that in actual calculations, a glass is approximately described as a crystal having a primitive cell containing a large but finite number of atoms $N_{\rm at}$. Thus, the Brillouin Zone (BZ) corresponding to the (large) finite-size model does not reduce to the aforementioned ``ideal glass'' limit ($\bm{q}{=}\bm{0}$ only), but has a (small) finite volume.
Recent work \cite{simoncelli_thermal_2022} has shown that when the lengthscale of structural disorder is shorter than the size of the simulation cell, periodic boundary conditions and Fourier interpolation over the small BZ of the glass can be exploited to improve the sampling of the vibrational properties, thus accurately extrapolating the bulk limit for the thermal conductivity. 
In Fig.~\ref{fig:al2o3_rdf} we show that Fourier interpolation can be employed to improve the sampling of the vibrational eigenstates in all the 120-atom models of am-Al$_2$O$_3$ studied in this work, since all these models have a radial distribution function that approaches $g(r)=1$ at distances significantly shorter than the linear size of the simulation cell.


Finally, it is worth mentioning that the Wigner thermal conductivity expression~(\ref{eq:thermal_conductivity_combined}) can be derived also from a many-body Green-Kubo approach~\cite{caldarelli_many-body_2022,PhysRevB.107.054311}, and such an expression has been recently employed, in combination with interatomic potentials, to describe the thermal properties of several glasses~\cite{isaeva2019modeling,lundgren_mode_2021} \footnote{More precisely , Refs.~\cite{isaeva2019modeling,lundgren_mode_2021} evaluated the bulk limit of the conductivity relying on empirical interatomic potentials and atomistic models containing thousands of atoms, \textit{i.e.}  having a size large enough to achieve computational convergence by evaluating Eq.~(\ref{eq:thermal_conductivity_combined}) at $\bm{q}=\bm{0}$ only and without relying on the Voigt regularization. See Ref.~\cite{simoncelli_thermal_2022} for details on the conditions under which evaluating Eq.~(\ref{eq:thermal_conductivity_combined}) yields equivalent results with or without relying on the Fourier interpolation and Voigt regularization.}.


\subsection{Numerical results} % (fold)
\label{sub:numerical_results}
\subsubsection{Thermal conductivity} % (fold)
\label{ssub:thermal_conductivity}

We use first-principles calculations to determine all the parameters entering in the regularized Wigner thermal conductivity expression (see Appendix~\ref{sec:computational_details} for details), thus we evaluate such an expression for each am-Al$_2$O$_3$ structure 
in the temperature range from 50\,K to 700\,K. We focus on this temperature  range because it is the most relevant for technological applications related to electronics \cite{jin2019li4ti5o12,qu2019air,randau2021additive}, and can be studied with the 120-atom models at our disposal \cite{simoncelli_thermal_2022}.



\begin{figure}[b]
  \centering
\includegraphics[width=\WidthFigure]{\folder/amorphous_alumina_AF_Wigner_comparison3.pdf}
  \caption{\textbf{Wigner vs Allen-Feldman thermal conductivity of am-Al$_2$O$_3$}.
  Blue, green and red denote am-Al$_2$O$_3$ models with densities 2.28, 2.98 and 3.49 g/cm$^3$ respectively. The solid lines are predictions from the anharmonic regularized Wigner transport equation (rWTE), the dashed lines are obtained from the harmonic Allen-Feldman theory (AF). 
  The effects of anharmonicity are overall small, and become weaker as density decreases. 
  Scatter points are experiments: red triangles are taken from \citet{Lee1995} (DC sputtering), green circles from \citet{Monachon2015} (ALD on Si substrate), and the green square is from \citet{gorham_density_2014} (ALD on Si substrate). 
 Theory and experiments are in reasonably good agreement over the entire temperature range analyzed.}
  \label{fig:Wigner_vs_AllenFeldman_tc}
\end{figure}
We show in Fig.~\ref{fig:Wigner_vs_AllenFeldman_tc} the predictions for the thermal conductivity obtained using the harmonic AF or anharmonic rWTE formulations (see Appendix~\ref{sec:convergence_of_the_allen} for details on the convergence test for the broadening parameter $\eta$ and for the calculation of the anharmonic linewidths). We find that the effects of anharmonicity are in general weak---AF and rWTE differ at most by 10$\%$ in the highest-density (3.49\,g/cm$^3$) model. These differences become smaller as the density decreases, and are practically invisible in the lowest-density (2.28\,g/cm$^3$) model.
The good agreement between AF and rWTE shows that in am-Al$_2$O$_3$ the vibrations' damping due to disorder is strong enough to dominate over other possible damping mechanisms such as anharmonicity. In addition, we highlight how both AF and rWTE predict an increasing-up-to-saturation trend for the temperature-conductivity curve; such a saturating trend is inherited from that of the specific heat (more on this later in Sec.~\ref{ssub:diffusivity}).
Our calculations shed light on the thermal properties of am-Al$_2$O$_3$ below room temperature, where the quantum Bose-Einstein statistics of vibrations has major effects on thermal transport. This is a step forward compared to previous studies based on molecular dynamics (MD) \cite{li_effects_2020}, which were governed by classical equipartition and thus had to be limited to the high-temperature regime \cite{PhysRevMaterials.3.085401} (\textit{i.e.} at temperature large enough to have a quantum specific heat effectively very close to the constant classical limit).

We compare our calculations with experimental measurements from Refs.~\cite{Monachon2015,gorham_density_2014} (ALD samples grown on Si substrate and having density 3.0\,g/cm$^3$) and Ref.~\cite{Lee1995} (DC-sputtered samples with density 3.51\,g/cm$^3$).
The differences between the conductivities of samples at similar density taken from different references (e.g., empty circle and empty square at $T{\sim}$300 K) may originate from many factors, including e.g. sample deposition technique, size, and method used to measure the conductivity \cite{zhou_2020_thermal}. 
None of these factors are accounted for in our theoretical calculations; thus, the differences observed between experiments on samples having similar density and size larger than the nanometre length scale (\textit{i.e.} the linear size of our atomistic models) are representative of the discrepancies that should be considered acceptable when comparing theory and experiments. 
Keeping these considerations in mind, we conclude that our theoretical predictions are in acceptable agreement with experiments.




We now turn our attention to the dependence of the conductivity on the density. 
In Fig.~\ref{fig:k_vs_density} we plot the room-temperature rWTE conductivity as a function of density (we note from Fig.~\ref{fig:Wigner_vs_AllenFeldman_tc} that at 300 K the rWTE and AF conductivities are practically indistinguishable across the entire density range analyzed), finding an approximately linear increase of the conductivity with density ($\kappa(\rho)_{300K}\approx a\cdot\rho +b$, where $a {=} 0.637 {\pm} 0.004 \tfrac{\rm W\cdot cm^3}{\rm m\cdot K \cdot g}$, $b {=} {-}0.495 {\pm}0.014 \tfrac{\rm W}{\rm m\cdot K }$). 
We compare our predictions with the experiments from \citet{gorham_density_2014}, who characterized the room-temperature thermal conductivity of ALD am-Al$_2$O$_3$ grown on Si substrate~\footnote{We note, in passing, that Refs. \cite{Monachon2015,Lee2017,gorham_density_2014} did not observe a significant dependence of the thermal conductivity from the substrate on which the am-Al$_2$O$_3$ sample was grown.} at densities ranging from 2.66 g/cm$^3$ to 3.12 g/cm$^3$. 
To give an idea about the conductivity variability found when comparing independent experiments, we also report the room-temperature conductivities extracted from the dataset of \citet{Lee1995} (1995, DC-sputtered) and \citet{Lee2017} (ALD on Si, 2017) already presented in Fig.~\ref{fig:Wigner_vs_AllenFeldman_tc}.
Considering the experimental error bars and the variance observed between independent experiments, we conclude that 
the linear increase of conductivity with density emerging from our calculations can be considered in good agreement with the increasing trend observed in experiments.


\begin{figure}[t]
  \centering
  \includegraphics[width=\WidthFigure]{\folder/amorphous_alumina_cond_density3.pdf}
  \caption{
\textbf{Thermal conductivity as a function of density.} 
  The solid green line is the theoretical rWTE conductivity computed at 300 K. 
  Empty scatter points are experiments performed at 300 K by \citet{gorham_density_2014} in 2014 (green squares, ALD samples grown on Si), by \citet{Lee2017} in 2017 (blue diamond, ALD samples grown on Si), and by \citet{Lee1995} in 1995 (red triangle, DC sputtering).}
  \label{fig:k_vs_density}
\end{figure} 



\subsubsection{Thermal diffusivity} % (fold)
\label{ssub:diffusivity}

To gain microscopic insights into the heat transport mechanisms in am-Al$_2$O$_3$, it is useful to resolve the amount of heat carried by each atomic vibration and its diffusion rate. 
This information can be obtained by recasting the rWTE conductivity expression~(\ref{eq:thermal_conductivity_combined}) as \cite{simoncelli_thermal_2022}
\begin{equation}
  \kappa(T)=\int_{0}^{\omega_{\rm max}} g(\omega) C(\omega,T) D(\omega,T) d\omega\;,
  \label{eq:kappa_diff}
\end{equation}
where $\omega_{\rm max}$ is the maximum vibrational frequency of the system, $g(\omega)$ is the VDOS discussed in Sec.~\ref{sub:vibrational_properties}), $C(\omega,T)$ is the specific heat for a vibration with frequency $\omega$ at temperature $T$ (see Eq.~(\ref{eq:quantum_specific_heat_A})), and $D(\omega,T)$ is the rWTE diffusivity \cite{simoncelli_thermal_2022}, 
\begin{align}
D(\omega,T){=}&[g(\omega){\mathcal{V}N_{\rm c}}]^{-1}\sum_{\bm{q},s} D_{\bm{q}s} \delta(\omega{-}\omega_{\bm{q}s}),\label{eq:diff_omega} \\
D_{\bm{q}s}{=}&{\sum_{s'}}
\frac{\omega_{\bm{q}s}{+}\omega_{\bm{q}s'} }{{2[C_{\bm{q}s}{+}C_{\bm{q}s'}] }}
\!\left[\!\frac{C_{\bm{q}s}}{\omega_{\bm{q}s}}{+}\frac{C_{\bm{q}s'}}{\omega_{\bm{q}s'}}\!\right]\!\!
\frac{\lVert\tens{v}(\bm{q})_{s,s'}\lVert^2\!}{3}\label{eq:diffusivity_q_s}\\
&\hspace{5mm}\times\pi\mathcal{F}_{[\eta;\Gamma_{\bm{q}s}{+}\Gamma_{\bm{q}s'}]}(\omega_{\bm{q}s}-\omega_{\bm{q}s'})\;.\nonumber
\end{align}
The expression for $D_{\bm{q}s}$ is determined by factorizing the specific heat $C_{\bm{q}s}$ in Eq.~(\ref{eq:thermal_conductivity_combined}) and by the requirement that in the coupling between two vibrations ${\bm{q}s}$ and ${\bm{q}s'}$ each contributes to the coupling with a weight equal to the relative specific heat \cite{simoncelli2021Wigner} (e.g. for vibration ${\bm{q}s}$ the weight is $\tfrac{C_{\bm{q}s}}{C_{\bm{q}s}+C_{\bm{q}s'}}$, and correspondingly for vibration ${\bm{q}s'}$ the weight is $\tfrac{C_{\bm{q}s'}}{C_{\bm{q}s}+C_{\bm{q}s'}}$).
In the harmonic AF limit $\eta{\gg}\Gamma_{\bm{q}s}{+}\Gamma_{\bm{q}s'}{\to} 0$, thus the Voigt distribution in Eq.~(\ref{eq:diffusivity_q_s}) reduces to the Gaussian representation of the Dirac $\delta$, accounting only for couplings between quasi-degenerate vibrational eigenstates and effectively reducing to the temperature-independent AF diffusivity~\cite{allen1989thermal,allen1993thermal}. 
We note that in the context of Eq.~\ref{eq:kappa_diff} the only difference between AF and rWTE originates from the diffusivity $D(\omega, T)$, implying that differences between AF and rWTE conductivities derive exclusively from differences between the AF and rWTE diffusivities.
Thus, recalling that Fig.~\ref{fig:Wigner_vs_AllenFeldman_tc} showed small or negligible effects of anharmonicity on the thermal conductivity of am-Al$_2$O$_3$, hereafter we will focus on the temperature-\textit{independent} AF limit of the diffusivity $D(\omega)$ to simplify the discussion (\textit{i.e.} Eqs.~(\ref{eq:diff_omega},\ref{eq:diffusivity_q_s}) evaluated with $\Gamma_{\bm{q}s}=0\;\forall \bm{q}s$ and $\eta$ determined from the convergence plateau as discussed in the Appendix \ref{sec:convergence_of_the_allen}).



Eq.~(\ref{eq:kappa_diff}) shows that the contributions to heat transport of atomic vibrational modes with frequency $\omega$ is determined by their density of states $g(\omega)$, amount of heat carried $C(\omega,T)$ and rate of diffusion $D(\omega)$.
Keeping this in mind, and considering that: (i) the analysis of the VDOS discussed earlier in Fig.~\ref{fig:VDOS} showed that the VDOS increases with density; (ii) the analysis of the conductivity as a function of density reported in Fig.~\ref{fig:k_vs_density} also showed a conductivity increasing (linearly) with density;
it is natural to ask to what extent the conductivity increase observed in Fig.~\ref{fig:k_vs_density} derives from the increase in the VDOS with density, and how density affects the diffusivity of vibrations.  
\begin{figure}[t]
  \centering
\includegraphics[width=\WidthFigure]{\folder/fig6_rev.pdf}
  \caption{\textbf{AF diffusivity of am-Al$_2$O$_3$ at various densities,} computed using Eq.~\ref{eq:diff_omega} in the AF limit (see text for details). 
  The diffusivity tends to increase with density, especially at low frequencies (0 - 100 cm$^{-1}$ range) and between 400 and 600 ${\rm cm}^{-1}$. Inset, quantum specific heat as a function of temperature.}
  \label{fig:diffusivity}
\end{figure}
Analyzing the frequency-resolved AF diffusivity $D(\omega)$ reported in Fig.~\ref{fig:diffusivity} for all our models of am-Al$_2$O$_3$ allows us to address these questions. 
We see that an overall increase of diffusivity with density is visible when comparing low-density ($\rho=$2.28 g/cm$^3$), medium-density ($\rho=$2.98 g/cm$^3$) and high-density ($\rho\geq 3.17$ g/cm$^3$) models, especially at low frequencies (from 0 to 100 cm$^{-1}$). 
More precisely, comparing the diffusivity of the highest-density $\rho= 3.49$ g/cm$^3$ model with that of the lowest-density $\rho= 2.28$ g/cm$^3$ model, we find that the highest-density model has a diffusivity that is a factor $\sim$2 larger than that of the lowest-density model in the low-frequency region (0-100 cm$^{-1}$) and a up to a factor 1.5 larger in the region between 400 and 600 cm$^{-1}$. The inset of  Fig.~\ref{fig:diffusivity} shows the quantum specific heat $C(\omega,T)$ at various temperatures, whose frequency dependence at fixed temperature $T$ is indicative of the portion of vibrational spectrum that significantly contributes to heat transport at that temperature.
We highlight how the low-frequency vibrational modes that have density-dependent diffusivity are significantly populated at all the temperatures considered, implying that the increase of the thermal conductivity with density observed in Fig.~\ref{fig:k_vs_density} receives contributions also from increases in the diffusivity.
We also note that the saturation of the specific heat shown in the inset of Fig.~\ref{fig:diffusivity} drives the saturation of the AF thermal conductivity at high temperature (Fig.~\ref{fig:Wigner_vs_AllenFeldman_tc}). Given that the effects of anharmonicity are unimportant for thermal transport in am-Al$_2$O$_3$, the saturation of the rWTE conductivity has to be attributed to the saturation of the specific heat.
Finally, we highlight how the classical limit (dashed line in the inset of Fig.~\ref{fig:diffusivity}) is not yet reached at temperatures as high as 1200 K; this underscores the importance of correctly accounting for the quantum Bose-Einstein statistics of vibrations to describe thermal transport in am-Al$_2$O$_3$ at technologically relevant temperatures.


In summary, we have found that both VDOS (Fig.~\ref{fig:VDOS}) and diffusivity (Fig.~\ref{fig:diffusivity}) increase with density.
In order to estimate how much the increase of conductivity with density shown in Figs.~\ref{fig:Wigner_vs_AllenFeldman_tc},\ref{fig:k_vs_density} depends on increase of the VDOS and how much on the increase in diffusivity, we computed the conductivity  artificially combining the VDOS and diffusivity of the highest and lowest-density models in the following ways: (i) using the VDOS of the 3.49 g/cm$^3$ model and the AF diffusivity of the 2.28 g/cm$^3$ model; (ii) using the VDOS of the 2.28 g/cm$^3$ model and the AF diffusivity of the 3.49 g/cm$^3$ model. The comparison between these artificial conductivities and the exact ones (taken from Fig.~\ref{fig:Wigner_vs_AllenFeldman_tc}) are reported in Fig.~\ref{fig:artificial_conductivity}, and show that 
these two artificial calculations yield
a conductivity that lies approximately halfway between the actual conductivities of the lowest-density and highest-density models over a wide temperature range, demonstrating that variations of the thermal conductivity with density are determined by both variations in the VDOS and in the diffusivity.

\begin{figure}
  \centering
\includegraphics[width=\WidthFigure]{\folder/amorphous_alumina_AF_vDOS_diff_contributions.pdf}\\[-3mm]
  \caption{\textbf{Microscopic mechanisms underlying conductivity increase with density.}
  Solid lines are exact AF conductivity calculation for the highest-density 3.49 g/cm$^3$ model (orange, $g_{3.49} D_{3.49}$) and lowest-density 2.28 g/cm$^3$ model (cyan, $g_{2.28} D_{2.28}$) taken from Fig.~\ref{fig:Wigner_vs_AllenFeldman_tc}.
  The dashed lines show results of artificial conductivity calculations, performed using Eq.~(\ref{eq:kappa_diff}) 
  with 
  the VDOS of the highest-density model and the diffusivity of the lowest-density model (purple, $g_{3.49} D_{2.28}$),  or the VDOS of the  lowest-density model and the diffusivity of the higher-density model (grey, $g_{2.28} D_{3.49}$).
  The artificial calculations yield conductivities lying approximately halfway between the exact limits, indicating that 
  the increase of conductivity with density is determined by both an increases in vDOS and by an increase in diffusivity (in similar proportion).}
  \label{fig:artificial_conductivity}
\end{figure}





\section{Conclusions}

The ubiquitous use of am-Al$_2$O$_3$ in electronic devices and the several open fundamental questions on how its atomistic structural and vibrational properties determine its macroscopic thermal properties prompted us to study this material from a first principles level of theory. 
We generated and characterized atomistic models of am-Al$_2$O$_3$ from AIMD with densities ranging from the record-low value of 2.28 g/cm$^3$ to the record-high value of 3.49 g/cm$^3$, describing how the atomic coordination topology varies with density. 
We have shown that at least five different atomic coordination environments coexist in am-Al$_2$O$_3$, and these lead to significant structural disorder already at the sub-nanometre lengthscale. 
We have discussed how the atomic coordination topology affects the vibrational properties, showing that
different coordination environments for oxygen have fingerprints on the coordination-resolved PDOS---O$_3$ is characterized by a bimodal PDOS with a local minimum around 500 cm$^{-1}$, while O$_2$ and O$_4$ feature by a unimodal PDOS peaked at about 150 and 500 cm$^{-1}$, respectively. In contrast, the coordination-resolved PDOS for Al$_4$, Al$_5$, and Al$_6$ are quite similar in shape at each density, but vary in their relative magnitudes. 
We have described the thermal properties using the recently introduced Voigt-regularized Wigner formulation (rWTE) \cite{simoncelli_thermal_2022}, accounting comprehensively for the effects of structural disorder, anharmonicity, and quantum Bose-Einstein statistics. 
We have shed light on the microscopic physics underlying thermal transport in am-Al$_2$O$_3$, discussing the dominant role played by strong structural disorder (emerging from having at least five coexisting different atomic coordination topologies) and the negligible role played by anharmonicity.

Specifically, we showed that the harmonic Allen-Feldman theory---evaluated using the convergence-acceleration protocol discussed in Ref.~\cite{simoncelli_thermal_2022}---yields predictions in close agreement with the anharmonic rWTE protocol and with experiments.
We discussed how the increase in the thermal conductivity observed with density derives from an increase of the vibrational density of states with density, as well as from an increase of the diffusivity with density.
Importantly, we have investigated the thermal properties also below room temperature ($T\gtrsim 50 K$), where the the quantum Bose-Einstein statistics of vibrations yields a specific heat significantly different from the classical limit, providing information on the thermal conductivity in a regime inaccessible by molecular-dynamics-based methods \cite{li_effects_2020,PhysRevMaterials.3.085401}, which are governed by classical equipartition \cite{PhysRevMaterials.3.085401} and thus limited to high temperatures. 
Ultimately, this study further validates the capability of the recently developed rWTE protocol \cite{simoncelli_thermal_2022} to describe the thermal properties of strongly disordered glasses using atomistic models containing hundreds of atoms, and thus within the reach of first-principles techniques.

\section*{Acknowledgements}
A. F. H. acknowledges the financial support of the Gates Cambridge Trust and the Winton Programme for the Physics of Sustainability, University of Cambridge.
M. S. acknowledges support from Gonville and Caius College, and from the SNSF project P500PT\_203178. The calculations presented in this work have been performed using computational resources provided by the Cambridge Tier-2 system operated by the University of Cambridge Research Computing Service (www.hpc.cam.ac.uk) funded by EPSRC Tier-2 capital grant EP/T022159/1.

%\bibliography{citations}


%apsrev4-2.bst 2019-01-14 (MD) hand-edited version of apsrev4-1.bst
%Control: key (0)
%Control: author (8) initials jnrlst
%Control: editor formatted (1) identically to author
%Control: production of article title (0) allowed
%Control: page (0) single
%Control: year (1) truncated
%Control: production of eprint (0) enabled
\providecommand{\noopsort}[1]{}\providecommand{\singleletter}[1]{#1}%
\begin{thebibliography}{74}%
\makeatletter
\providecommand \@ifxundefined [1]{%
 \@ifx{#1\undefined}
}%
\providecommand \@ifnum [1]{%
 \ifnum #1\expandafter \@firstoftwo
 \else \expandafter \@secondoftwo
 \fi
}%
\providecommand \@ifx [1]{%
 \ifx #1\expandafter \@firstoftwo
 \else \expandafter \@secondoftwo
 \fi
}%
\providecommand \natexlab [1]{#1}%
\providecommand \enquote  [1]{``#1''}%
\providecommand \bibnamefont  [1]{#1}%
\providecommand \bibfnamefont [1]{#1}%
\providecommand \citenamefont [1]{#1}%
\providecommand \href@noop [0]{\@secondoftwo}%
\providecommand \href [0]{\begingroup \@sanitize@url \@href}%
\providecommand \@href[1]{\@@startlink{#1}\@@href}%
\providecommand \@@href[1]{\endgroup#1\@@endlink}%
\providecommand \@sanitize@url [0]{\catcode `\\12\catcode `\$12\catcode
  `\&12\catcode `\#12\catcode `\^12\catcode `\_12\catcode `\%12\relax}%
\providecommand \@@startlink[1]{}%
\providecommand \@@endlink[0]{}%
\providecommand \url  [0]{\begingroup\@sanitize@url \@url }%
\providecommand \@url [1]{\endgroup\@href {#1}{\urlprefix }}%
\providecommand \urlprefix  [0]{URL }%
\providecommand \Eprint [0]{\href }%
\providecommand \doibase [0]{https://doi.org/}%
\providecommand \selectlanguage [0]{\@gobble}%
\providecommand \bibinfo  [0]{\@secondoftwo}%
\providecommand \bibfield  [0]{\@secondoftwo}%
\providecommand \translation [1]{[#1]}%
\providecommand \BibitemOpen [0]{}%
\providecommand \bibitemStop [0]{}%
\providecommand \bibitemNoStop [0]{.\EOS\space}%
\providecommand \EOS [0]{\spacefactor3000\relax}%
\providecommand \BibitemShut  [1]{\csname bibitem#1\endcsname}%
\let\auto@bib@innerbib\@empty
%</preamble>
\bibitem [{\citenamefont {Levin}\ and\ \citenamefont
  {Brandon}(1998)}]{levin_1998_polymorphs}%
  \BibitemOpen
  \bibfield  {author} {\bibinfo {author} {\bibfnamefont {I.}~\bibnamefont
  {Levin}}\ and\ \bibinfo {author} {\bibfnamefont {D.}~\bibnamefont
  {Brandon}},\ }\bibfield  {title} {\bibinfo {title} {Metastable alumina
  polymorphs: Crystal structures and transition sequences},\ }\href
  {https://doi.org/https://doi.org/10.1111/j.1151-2916.1998.tb02581.x}
  {\bibfield  {journal} {\bibinfo  {journal} {Journal of the American Ceramic
  Society}\ }\textbf {\bibinfo {volume} {81}},\ \bibinfo {pages} {1995}
  (\bibinfo {year} {1998})}\BibitemShut {NoStop}%
\bibitem [{\citenamefont {Tavakoli}\ \emph {et~al.}(2013)\citenamefont
  {Tavakoli}, \citenamefont {Maram}, \citenamefont {Widgeon}, \citenamefont
  {Rufner}, \citenamefont {Van~Benthem}, \citenamefont {Ushakov}, \citenamefont
  {Sen},\ and\ \citenamefont {Navrotsky}}]{tavakoli2013amorphous}%
  \BibitemOpen
  \bibfield  {author} {\bibinfo {author} {\bibfnamefont {A.~H.}\ \bibnamefont
  {Tavakoli}}, \bibinfo {author} {\bibfnamefont {P.~S.}\ \bibnamefont {Maram}},
  \bibinfo {author} {\bibfnamefont {S.~J.}\ \bibnamefont {Widgeon}}, \bibinfo
  {author} {\bibfnamefont {J.}~\bibnamefont {Rufner}}, \bibinfo {author}
  {\bibfnamefont {K.}~\bibnamefont {Van~Benthem}}, \bibinfo {author}
  {\bibfnamefont {S.}~\bibnamefont {Ushakov}}, \bibinfo {author} {\bibfnamefont
  {S.}~\bibnamefont {Sen}},\ and\ \bibinfo {author} {\bibfnamefont
  {A.}~\bibnamefont {Navrotsky}},\ }\bibfield  {title} {\bibinfo {title}
  {Amorphous alumina nanoparticles: structure, surface energy, and
  thermodynamic phase stability},\ }\href
  {https://pubs.acs.org/doi/10.1021/jp405820g} {\bibfield  {journal} {\bibinfo
  {journal} {The Journal of Physical Chemistry C}\ }\textbf {\bibinfo {volume}
  {117}},\ \bibinfo {pages} {17123} (\bibinfo {year} {2013})}\BibitemShut
  {NoStop}%
\bibitem [{\citenamefont {Li}\ \emph {et~al.}(2020)\citenamefont {Li},
  \citenamefont {Ando},\ and\ \citenamefont {Watanabe}}]{li_effects_2020}%
  \BibitemOpen
  \bibfield  {author} {\bibinfo {author} {\bibfnamefont {W.}~\bibnamefont
  {Li}}, \bibinfo {author} {\bibfnamefont {Y.}~\bibnamefont {Ando}},\ and\
  \bibinfo {author} {\bibfnamefont {S.}~\bibnamefont {Watanabe}},\ }\bibfield
  {title} {\bibinfo {title} {Effects of density and composition on the
  properties of amorphous alumina: {A} high-dimensional neural network
  potential study},\ }\href {https://doi.org/10.1063/5.0026289} {\bibfield
  {journal} {\bibinfo  {journal} {The Journal of Chemical Physics}\ }\textbf
  {\bibinfo {volume} {153}},\ \bibinfo {pages} {164119} (\bibinfo {year}
  {2020})}\BibitemShut {NoStop}%
\bibitem [{\citenamefont {Shafiq}\ \emph {et~al.}(2022)\citenamefont {Shafiq},
  \citenamefont {Shafique}, \citenamefont {Akhter}, \citenamefont {Yang},\ and\
  \citenamefont {Hussain}}]{shafiq2022recent}%
  \BibitemOpen
  \bibfield  {author} {\bibinfo {author} {\bibfnamefont {I.}~\bibnamefont
  {Shafiq}}, \bibinfo {author} {\bibfnamefont {S.}~\bibnamefont {Shafique}},
  \bibinfo {author} {\bibfnamefont {P.}~\bibnamefont {Akhter}}, \bibinfo
  {author} {\bibfnamefont {W.}~\bibnamefont {Yang}},\ and\ \bibinfo {author}
  {\bibfnamefont {M.}~\bibnamefont {Hussain}},\ }\bibfield  {title} {\bibinfo
  {title} {Recent developments in alumina supported hydrodesulfurization
  catalysts for the production of sulfur-free refinery products: A technical
  review},\ }\href {https://doi.org/10.1080/01614940.2020.1780824} {\bibfield
  {journal} {\bibinfo  {journal} {Catalysis Reviews}\ }\textbf {\bibinfo
  {volume} {64}},\ \bibinfo {pages} {1} (\bibinfo {year} {2022})}\BibitemShut
  {NoStop}%
\bibitem [{\citenamefont {Oh}\ \emph {et~al.}(2019)\citenamefont {Oh},
  \citenamefont {Bathula}, \citenamefont {Park},\ and\ \citenamefont
  {Suh}}]{oh2019sustainable}%
  \BibitemOpen
  \bibfield  {author} {\bibinfo {author} {\bibfnamefont {J.}~\bibnamefont
  {Oh}}, \bibinfo {author} {\bibfnamefont {H.~B.}\ \bibnamefont {Bathula}},
  \bibinfo {author} {\bibfnamefont {J.~H.}\ \bibnamefont {Park}},\ and\
  \bibinfo {author} {\bibfnamefont {Y.-W.}\ \bibnamefont {Suh}},\ }\bibfield
  {title} {\bibinfo {title} {A sustainable mesoporous palladium-alumina
  catalyst for efficient hydrogen release from n-heterocyclic liquid organic
  hydrogen carriers},\ }\href
  {https://www.nature.com/articles/s42004-019-0167-7} {\bibfield  {journal}
  {\bibinfo  {journal} {Communications chemistry}\ }\textbf {\bibinfo {volume}
  {2}},\ \bibinfo {pages} {68} (\bibinfo {year} {2019})}\BibitemShut {NoStop}%
\bibitem [{\citenamefont {Jin}\ \emph {et~al.}(2019)\citenamefont {Jin},
  \citenamefont {Yu}, \citenamefont {Gao}, \citenamefont {He}, \citenamefont
  {White},\ and\ \citenamefont {Liang}}]{jin2019li4ti5o12}%
  \BibitemOpen
  \bibfield  {author} {\bibinfo {author} {\bibfnamefont {Y.}~\bibnamefont
  {Jin}}, \bibinfo {author} {\bibfnamefont {H.}~\bibnamefont {Yu}}, \bibinfo
  {author} {\bibfnamefont {Y.}~\bibnamefont {Gao}}, \bibinfo {author}
  {\bibfnamefont {X.}~\bibnamefont {He}}, \bibinfo {author} {\bibfnamefont
  {T.~A.}\ \bibnamefont {White}},\ and\ \bibinfo {author} {\bibfnamefont
  {X.}~\bibnamefont {Liang}},\ }\bibfield  {title} {\bibinfo {title}
  {Li$_4${T}i$_5${O}$_{12}$ coated with ultrathin aluminum-doped zinc oxide
  films as an anode material for lithium-ion batteries},\ }\href
  {https://doi.org/10.1016/j.jpowsour.2019.226859} {\bibfield  {journal}
  {\bibinfo  {journal} {Journal of Power Sources}\ }\textbf {\bibinfo {volume}
  {436}},\ \bibinfo {pages} {226859} (\bibinfo {year} {2019})}\BibitemShut
  {NoStop}%
\bibitem [{\citenamefont {Qu}\ \emph {et~al.}(2019)\citenamefont {Qu},
  \citenamefont {Jia}, \citenamefont {Wang}, \citenamefont {Li}, \citenamefont
  {Yao}, \citenamefont {Li}, \citenamefont {Liu}, \citenamefont {Zou},
  \citenamefont {Zhou}, \citenamefont {Wang} \emph {et~al.}}]{qu2019air}%
  \BibitemOpen
  \bibfield  {author} {\bibinfo {author} {\bibfnamefont {S.}~\bibnamefont
  {Qu}}, \bibinfo {author} {\bibfnamefont {W.}~\bibnamefont {Jia}}, \bibinfo
  {author} {\bibfnamefont {Y.}~\bibnamefont {Wang}}, \bibinfo {author}
  {\bibfnamefont {C.}~\bibnamefont {Li}}, \bibinfo {author} {\bibfnamefont
  {Z.}~\bibnamefont {Yao}}, \bibinfo {author} {\bibfnamefont {K.}~\bibnamefont
  {Li}}, \bibinfo {author} {\bibfnamefont {Y.}~\bibnamefont {Liu}}, \bibinfo
  {author} {\bibfnamefont {W.}~\bibnamefont {Zou}}, \bibinfo {author}
  {\bibfnamefont {F.}~\bibnamefont {Zhou}}, \bibinfo {author} {\bibfnamefont
  {Z.}~\bibnamefont {Wang}}, \emph {et~al.},\ }\bibfield  {title} {\bibinfo
  {title} {Air-stable lithium metal anode with sputtered aluminum coating layer
  for improved performance},\ }\href
  {https://doi.org/10.1016/j.electacta.2019.05.138} {\bibfield  {journal}
  {\bibinfo  {journal} {Electrochimica Acta}\ }\textbf {\bibinfo {volume}
  {317}},\ \bibinfo {pages} {120} (\bibinfo {year} {2019})}\BibitemShut
  {NoStop}%
\bibitem [{\citenamefont {Randau}\ \emph {et~al.}(2021)\citenamefont {Randau},
  \citenamefont {Walther}, \citenamefont {Neumann}, \citenamefont {Schneider},
  \citenamefont {Negi}, \citenamefont {Mogwitz}, \citenamefont {Sann},
  \citenamefont {Becker-Steinberger}, \citenamefont {Danner}, \citenamefont
  {Hein} \emph {et~al.}}]{randau2021additive}%
  \BibitemOpen
  \bibfield  {author} {\bibinfo {author} {\bibfnamefont {S.}~\bibnamefont
  {Randau}}, \bibinfo {author} {\bibfnamefont {F.}~\bibnamefont {Walther}},
  \bibinfo {author} {\bibfnamefont {A.}~\bibnamefont {Neumann}}, \bibinfo
  {author} {\bibfnamefont {Y.}~\bibnamefont {Schneider}}, \bibinfo {author}
  {\bibfnamefont {R.~S.}\ \bibnamefont {Negi}}, \bibinfo {author}
  {\bibfnamefont {B.}~\bibnamefont {Mogwitz}}, \bibinfo {author} {\bibfnamefont
  {J.}~\bibnamefont {Sann}}, \bibinfo {author} {\bibfnamefont {K.}~\bibnamefont
  {Becker-Steinberger}}, \bibinfo {author} {\bibfnamefont {T.}~\bibnamefont
  {Danner}}, \bibinfo {author} {\bibfnamefont {S.}~\bibnamefont {Hein}}, \emph
  {et~al.},\ }\bibfield  {title} {\bibinfo {title} {On the additive
  microstructure in composite cathodes and alumina-coated carbon microwires for
  improved all-solid-state batteries},\ }\href
  {https://pubs.acs.org/doi/10.1021/acs.chemmater.0c04454} {\bibfield
  {journal} {\bibinfo  {journal} {Chemistry of Materials}\ }\textbf {\bibinfo
  {volume} {33}},\ \bibinfo {pages} {1380} (\bibinfo {year}
  {2021})}\BibitemShut {NoStop}%
\bibitem [{\citenamefont {Paterson}\ \emph {et~al.}(2020)\citenamefont
  {Paterson}, \citenamefont {Singhal}, \citenamefont {Tainoff}, \citenamefont
  {Richard},\ and\ \citenamefont {Bourgeois}}]{paterson_thermal_2020}%
  \BibitemOpen
  \bibfield  {author} {\bibinfo {author} {\bibfnamefont {J.}~\bibnamefont
  {Paterson}}, \bibinfo {author} {\bibfnamefont {D.}~\bibnamefont {Singhal}},
  \bibinfo {author} {\bibfnamefont {D.}~\bibnamefont {Tainoff}}, \bibinfo
  {author} {\bibfnamefont {J.}~\bibnamefont {Richard}},\ and\ \bibinfo {author}
  {\bibfnamefont {O.}~\bibnamefont {Bourgeois}},\ }\bibfield  {title} {\bibinfo
  {title} {Thermal conductivity and thermal boundary resistance of amorphous
  {Al}$_{\textrm{2}}${O}$_{\textrm{3}}$ thin films on germanium and sapphire},\
  }\href {https://doi.org/10.1063/5.0004576} {\bibfield  {journal} {\bibinfo
  {journal} {Journal of Applied Physics}\ }\textbf {\bibinfo {volume} {127}},\
  \bibinfo {pages} {245105} (\bibinfo {year} {2020})}\BibitemShut {NoStop}%
\bibitem [{\citenamefont {Mavrič}\ \emph {et~al.}(2019)\citenamefont
  {Mavrič}, \citenamefont {Valant}, \citenamefont {Cui},\ and\ \citenamefont
  {Wang}}]{mavric_advanced_2019}%
  \BibitemOpen
  \bibfield  {author} {\bibinfo {author} {\bibfnamefont {A.}~\bibnamefont
  {Mavrič}}, \bibinfo {author} {\bibfnamefont {M.}~\bibnamefont {Valant}},
  \bibinfo {author} {\bibfnamefont {C.}~\bibnamefont {Cui}},\ and\ \bibinfo
  {author} {\bibfnamefont {Z.~M.}\ \bibnamefont {Wang}},\ }\bibfield  {title}
  {\bibinfo {title} {Advanced applications of amorphous alumina: {From} nano to
  bulk},\ }\href {https://doi.org/10.1016/j.jnoncrysol.2019.119493} {\bibfield
  {journal} {\bibinfo  {journal} {Journal of Non-Crystalline Solids}\ }\textbf
  {\bibinfo {volume} {521}},\ \bibinfo {pages} {119493} (\bibinfo {year}
  {2019})}\BibitemShut {NoStop}%
\bibitem [{\citenamefont {Scott}\ \emph {et~al.}(2018)\citenamefont {Scott},
  \citenamefont {Gaskins}, \citenamefont {King},\ and\ \citenamefont
  {Hopkins}}]{scott2018thermalal2o3}%
  \BibitemOpen
  \bibfield  {author} {\bibinfo {author} {\bibfnamefont {E.~A.}\ \bibnamefont
  {Scott}}, \bibinfo {author} {\bibfnamefont {J.~T.}\ \bibnamefont {Gaskins}},
  \bibinfo {author} {\bibfnamefont {S.~W.}\ \bibnamefont {King}},\ and\
  \bibinfo {author} {\bibfnamefont {P.~E.}\ \bibnamefont {Hopkins}},\
  }\bibfield  {title} {\bibinfo {title} {Thermal conductivity and thermal
  boundary resistance of atomic layer deposited high-k dielectric aluminum
  oxide, hafnium oxide, and titanium oxide thin films on silicon},\ }\href
  {https://doi.org/10.1063/1.5021044} {\bibfield  {journal} {\bibinfo
  {journal} {APL Materials}\ }\textbf {\bibinfo {volume} {6}},\ \bibinfo
  {pages} {058302} (\bibinfo {year} {2018})}\BibitemShut {NoStop}%
\bibitem [{\citenamefont {F. Harper}\ \emph {et~al.}(2023)\citenamefont
  {F. Harper}, \citenamefont {P. Emge}, \citenamefont {M. Magusin},
  \citenamefont {P. Grey},\ and\ \citenamefont
  {J. Morris}}]{fharper_modelling_2023}%
  \BibitemOpen
  \bibfield  {author} {\bibinfo {author} {\bibfnamefont {A.}~\bibnamefont
  {F. Harper}}, \bibinfo {author} {\bibfnamefont {S.}~\bibnamefont
  {P. Emge}}, \bibinfo {author} {\bibfnamefont {P.~C.~M.}\ \bibnamefont
  {M. Magusin}}, \bibinfo {author} {\bibfnamefont {C.}~\bibnamefont
  {P. Grey}},\ and\ \bibinfo {author} {\bibfnamefont {A.}~\bibnamefont
  {J. Morris}},\ }\bibfield  {title} {\bibinfo {title} {Modelling amorphous
  materials via a joint solid-state {NMR} and {X}-ray absorption spectroscopy
  and {DFT} approach: application to alumina},\ }\href
  {https://doi.org/10.1039/D2SC04035B} {\bibfield  {journal} {\bibinfo
  {journal} {Chemical Science}\ }\textbf {\bibinfo {volume} {14}},\ \bibinfo
  {pages} {1155} (\bibinfo {year} {2023})}\BibitemShut {NoStop}%
\bibitem [{\citenamefont {Lee}\ \emph {et~al.}(2010)\citenamefont {Lee},
  \citenamefont {Park}, \citenamefont {Yi},\ and\ \citenamefont
  {Moon}}]{Lee_2010_NMR}%
  \BibitemOpen
  \bibfield  {author} {\bibinfo {author} {\bibfnamefont {S.~K.}\ \bibnamefont
  {Lee}}, \bibinfo {author} {\bibfnamefont {S.~Y.}\ \bibnamefont {Park}},
  \bibinfo {author} {\bibfnamefont {Y.~S.}\ \bibnamefont {Yi}},\ and\ \bibinfo
  {author} {\bibfnamefont {J.}~\bibnamefont {Moon}},\ }\bibfield  {title}
  {\bibinfo {title} {Structure and disorder in amorphous alumina thin films:
  Insights from high-resolution solid-state {NMR}},\ }\href
  {https://doi.org/10.1021/jp105306r} {\bibfield  {journal} {\bibinfo
  {journal} {The Journal of Physical Chemistry C}\ }\textbf {\bibinfo {volume}
  {114}},\ \bibinfo {pages} {13890} (\bibinfo {year} {2010})}\BibitemShut
  {NoStop}%
\bibitem [{\citenamefont {Sarou-Kanian}\ \emph {et~al.}(2013)\citenamefont
  {Sarou-Kanian}, \citenamefont {Gleizes}, \citenamefont {Florian},
  \citenamefont {Samélor}, \citenamefont {Massiot},\ and\ \citenamefont
  {Vahlas}}]{sarou_2013_thermal}%
  \BibitemOpen
  \bibfield  {author} {\bibinfo {author} {\bibfnamefont {V.}~\bibnamefont
  {Sarou-Kanian}}, \bibinfo {author} {\bibfnamefont {A.~N.}\ \bibnamefont
  {Gleizes}}, \bibinfo {author} {\bibfnamefont {P.}~\bibnamefont {Florian}},
  \bibinfo {author} {\bibfnamefont {D.}~\bibnamefont {Samélor}}, \bibinfo
  {author} {\bibfnamefont {D.}~\bibnamefont {Massiot}},\ and\ \bibinfo {author}
  {\bibfnamefont {C.}~\bibnamefont {Vahlas}},\ }\bibfield  {title} {\bibinfo
  {title} {Temperature-dependent 4-, 5- and 6-fold coordination of aluminum in
  mocvd-grown amorphous alumina films: A very high field $^{27}${Al-NMR}
  study},\ }\href {https://doi.org/10.1021/jp4077504} {\bibfield  {journal}
  {\bibinfo  {journal} {The Journal of Physical Chemistry C}\ }\textbf
  {\bibinfo {volume} {117}},\ \bibinfo {pages} {21965} (\bibinfo {year}
  {2013})}\BibitemShut {NoStop}%
\bibitem [{\citenamefont {Hashimoto}\ \emph {et~al.}(2022)\citenamefont
  {Hashimoto}, \citenamefont {Onodera}, \citenamefont {Tahara}, \citenamefont
  {Kohara}, \citenamefont {Yazawa}, \citenamefont {Segawa}, \citenamefont
  {Murakami},\ and\ \citenamefont {Ohara}}]{hashimoto_structure_2022}%
  \BibitemOpen
  \bibfield  {author} {\bibinfo {author} {\bibfnamefont {H.}~\bibnamefont
  {Hashimoto}}, \bibinfo {author} {\bibfnamefont {Y.}~\bibnamefont {Onodera}},
  \bibinfo {author} {\bibfnamefont {S.}~\bibnamefont {Tahara}}, \bibinfo
  {author} {\bibfnamefont {S.}~\bibnamefont {Kohara}}, \bibinfo {author}
  {\bibfnamefont {K.}~\bibnamefont {Yazawa}}, \bibinfo {author} {\bibfnamefont
  {H.}~\bibnamefont {Segawa}}, \bibinfo {author} {\bibfnamefont
  {M.}~\bibnamefont {Murakami}},\ and\ \bibinfo {author} {\bibfnamefont
  {K.}~\bibnamefont {Ohara}},\ }\bibfield  {title} {\bibinfo {title} {Structure
  of alumina glass},\ }\href {https://doi.org/10.1038/s41598-021-04455-6}
  {\bibfield  {journal} {\bibinfo  {journal} {Scientific Reports}\ }\textbf
  {\bibinfo {volume} {12}},\ \bibinfo {pages} {516} (\bibinfo {year} {2022})},\
  \bibinfo {note} {number: 1 Publisher: Nature Publishing Group}\BibitemShut
  {NoStop}%
\bibitem [{\citenamefont {Dicks}\ \emph {et~al.}(2019)\citenamefont {Dicks},
  \citenamefont {Cottom}, \citenamefont {Shluger},\ and\ \citenamefont
  {Afanas’ev}}]{Dicks_2019}%
  \BibitemOpen
  \bibfield  {author} {\bibinfo {author} {\bibfnamefont {O.~A.}\ \bibnamefont
  {Dicks}}, \bibinfo {author} {\bibfnamefont {J.}~\bibnamefont {Cottom}},
  \bibinfo {author} {\bibfnamefont {A.~L.}\ \bibnamefont {Shluger}},\ and\
  \bibinfo {author} {\bibfnamefont {V.~V.}\ \bibnamefont {Afanas’ev}},\
  }\bibfield  {title} {\bibinfo {title} {The origin of negative charging in
  amorphous {A}l$_2${O}$_3$ films: the role of native defects},\ }\href
  {https://doi.org/10.1088/1361-6528/ab0450} {\bibfield  {journal} {\bibinfo
  {journal} {Nanotechnology}\ }\textbf {\bibinfo {volume} {30}},\ \bibinfo
  {pages} {205201} (\bibinfo {year} {2019})}\BibitemShut {NoStop}%
\bibitem [{\citenamefont {Leung}(2021)}]{Leung_2021}%
  \BibitemOpen
  \bibfield  {author} {\bibinfo {author} {\bibfnamefont {K.}~\bibnamefont
  {Leung}},\ }\bibfield  {title} {\bibinfo {title} {First principles, explicit
  interface studies of oxygen vacancy and chloride in alumina films for
  corrosion applications},\ }\href {https://doi.org/10.1149/1945-7111/abe7a5}
  {\bibfield  {journal} {\bibinfo  {journal} {Journal of The Electrochemical
  Society}\ }\textbf {\bibinfo {volume} {168}},\ \bibinfo {pages} {031511}
  (\bibinfo {year} {2021})}\BibitemShut {NoStop}%
\bibitem [{\citenamefont {Cahill}\ and\ \citenamefont
  {Pohl}(1987)}]{cahill_thermal_1987}%
  \BibitemOpen
  \bibfield  {author} {\bibinfo {author} {\bibfnamefont {D.~G.}\ \bibnamefont
  {Cahill}}\ and\ \bibinfo {author} {\bibfnamefont {R.~O.}\ \bibnamefont
  {Pohl}},\ }\bibfield  {title} {\bibinfo {title} {Thermal conductivity of
  amorphous solids above the plateau},\ }\href
  {https://doi.org/10.1103/PhysRevB.35.4067} {\bibfield  {journal} {\bibinfo
  {journal} {Physical Review B}\ }\textbf {\bibinfo {volume} {35}},\ \bibinfo
  {pages} {4067} (\bibinfo {year} {1987})},\ \bibinfo {note} {publisher:
  American Physical Society}\BibitemShut {NoStop}%
\bibitem [{\citenamefont {Allen}\ and\ \citenamefont
  {Feldman}(1989)}]{allen1989thermal}%
  \BibitemOpen
  \bibfield  {author} {\bibinfo {author} {\bibfnamefont {P.~B.}\ \bibnamefont
  {Allen}}\ and\ \bibinfo {author} {\bibfnamefont {J.~L.}\ \bibnamefont
  {Feldman}},\ }\bibfield  {title} {\bibinfo {title} {{Thermal Conductivity of
  Glasses: Theory and Application to Amorphous Si}},\ }\href
  {https://doi.org/10.1103/PhysRevLett.62.645} {\bibfield  {journal} {\bibinfo
  {journal} {Phys. Rev. Lett.}\ }\textbf {\bibinfo {volume} {62}},\ \bibinfo
  {pages} {645} (\bibinfo {year} {1989})}\BibitemShut {NoStop}%
\bibitem [{\citenamefont {Simoncelli}\ \emph
  {et~al.}(2022{\natexlab{a}})\citenamefont {Simoncelli}, \citenamefont
  {Mauri},\ and\ \citenamefont {Marzari}}]{simoncelli_thermal_2022}%
  \BibitemOpen
  \bibfield  {author} {\bibinfo {author} {\bibfnamefont {M.}~\bibnamefont
  {Simoncelli}}, \bibinfo {author} {\bibfnamefont {F.}~\bibnamefont {Mauri}},\
  and\ \bibinfo {author} {\bibfnamefont {N.}~\bibnamefont {Marzari}},\
  }\bibfield  {title} {\bibinfo {title} {Thermal conductivity of glasses above
  the plateau: first-principles theory and applications},\ }\href
  {http://arxiv.org/abs/2209.11201} {\bibfield  {journal} {\bibinfo  {journal}
  {arXiv.2209.11201}\ } (\bibinfo {year} {2022}{\natexlab{a}})}\BibitemShut
  {NoStop}%
\bibitem [{\citenamefont {Puligheddu}\ \emph {et~al.}(2019)\citenamefont
  {Puligheddu}, \citenamefont {Xia}, \citenamefont {Chan},\ and\ \citenamefont
  {Galli}}]{PhysRevMaterials.3.085401}%
  \BibitemOpen
  \bibfield  {author} {\bibinfo {author} {\bibfnamefont {M.}~\bibnamefont
  {Puligheddu}}, \bibinfo {author} {\bibfnamefont {Y.}~\bibnamefont {Xia}},
  \bibinfo {author} {\bibfnamefont {M.}~\bibnamefont {Chan}},\ and\ \bibinfo
  {author} {\bibfnamefont {G.}~\bibnamefont {Galli}},\ }\bibfield  {title}
  {\bibinfo {title} {{Computational prediction of lattice thermal conductivity:
  A comparison of molecular dynamics and Boltzmann transport approaches}},\
  }\href {https://doi.org/10.1103/PhysRevMaterials.3.085401} {\bibfield
  {journal} {\bibinfo  {journal} {Phys. Rev. Materials}\ }\textbf {\bibinfo
  {volume} {3}},\ \bibinfo {pages} {085401} (\bibinfo {year}
  {2019})}\BibitemShut {NoStop}%
\bibitem [{\citenamefont {{Simoncelli, M. and Marzari, N. and Mauri,
  F.}}(2019)}]{simoncelli2019unified}%
  \BibitemOpen
  \bibfield  {author} {\bibinfo {author} {\bibnamefont {{Simoncelli, M. and
  Marzari, N. and Mauri, F.}}},\ }\bibfield  {title} {\bibinfo {title}
  {{Unified theory of thermal transport in crystals and glasses}},\ }\href
  {https://doi.org/10.1038/s41567-019-0520-x} {\bibfield  {journal} {\bibinfo
  {journal} {Nat. Phys.}\ }\textbf {\bibinfo {volume} {15}},\ \bibinfo {pages}
  {809} (\bibinfo {year} {2019})}\BibitemShut {NoStop}%
\bibitem [{\citenamefont {Simoncelli}\ \emph
  {et~al.}(2022{\natexlab{b}})\citenamefont {Simoncelli}, \citenamefont
  {Marzari},\ and\ \citenamefont {Mauri}}]{simoncelli2021Wigner}%
  \BibitemOpen
  \bibfield  {author} {\bibinfo {author} {\bibfnamefont {M.}~\bibnamefont
  {Simoncelli}}, \bibinfo {author} {\bibfnamefont {N.}~\bibnamefont
  {Marzari}},\ and\ \bibinfo {author} {\bibfnamefont {F.}~\bibnamefont
  {Mauri}},\ }\bibfield  {title} {\bibinfo {title} {Wigner formulation of
  thermal transport in solids},\ }\href
  {https://doi.org/10.1103/PhysRevX.12.041011} {\bibfield  {journal} {\bibinfo
  {journal} {Phys. Rev. X}\ }\textbf {\bibinfo {volume} {12}},\ \bibinfo
  {pages} {041011} (\bibinfo {year} {2022}{\natexlab{b}})}\BibitemShut
  {NoStop}%
\bibitem [{\citenamefont {{\AA}rhammar}\ \emph {et~al.}(2011)\citenamefont
  {{\AA}rhammar}, \citenamefont {Pietzsch}, \citenamefont {Bock}, \citenamefont
  {Holmstr{\"o}m}, \citenamefont {Araujo}, \citenamefont {Gr{\aa}sj{\"o}},
  \citenamefont {Zhao}, \citenamefont {Green}, \citenamefont {Peery},
  \citenamefont {Hennies} \emph {et~al.}}]{aarhammar2011unveiling}%
  \BibitemOpen
  \bibfield  {author} {\bibinfo {author} {\bibfnamefont {C.}~\bibnamefont
  {{\AA}rhammar}}, \bibinfo {author} {\bibfnamefont {A.}~\bibnamefont
  {Pietzsch}}, \bibinfo {author} {\bibfnamefont {N.}~\bibnamefont {Bock}},
  \bibinfo {author} {\bibfnamefont {E.}~\bibnamefont {Holmstr{\"o}m}}, \bibinfo
  {author} {\bibfnamefont {C.~M.}\ \bibnamefont {Araujo}}, \bibinfo {author}
  {\bibfnamefont {J.}~\bibnamefont {Gr{\aa}sj{\"o}}}, \bibinfo {author}
  {\bibfnamefont {S.}~\bibnamefont {Zhao}}, \bibinfo {author} {\bibfnamefont
  {S.}~\bibnamefont {Green}}, \bibinfo {author} {\bibfnamefont
  {T.}~\bibnamefont {Peery}}, \bibinfo {author} {\bibfnamefont
  {F.}~\bibnamefont {Hennies}}, \emph {et~al.},\ }\bibfield  {title} {\bibinfo
  {title} {Unveiling the complex electronic structure of amorphous metal
  oxides},\ }\href {https://www.pnas.org/doi/10.1073/pnas.1019698108}
  {\bibfield  {journal} {\bibinfo  {journal} {Proceedings of the National
  Academy of Sciences}\ }\textbf {\bibinfo {volume} {108}},\ \bibinfo {pages}
  {6355} (\bibinfo {year} {2011})}\BibitemShut {NoStop}%
\bibitem [{\citenamefont {Liz\'arraga}\ \emph {et~al.}(2011)\citenamefont
  {Liz\'arraga}, \citenamefont {Holmstr\"om}, \citenamefont {Parker},\ and\
  \citenamefont {Arrouvel}}]{lizarraga_2011_lowdensity}%
  \BibitemOpen
  \bibfield  {author} {\bibinfo {author} {\bibfnamefont {R.}~\bibnamefont
  {Liz\'arraga}}, \bibinfo {author} {\bibfnamefont {E.}~\bibnamefont
  {Holmstr\"om}}, \bibinfo {author} {\bibfnamefont {S.~C.}\ \bibnamefont
  {Parker}},\ and\ \bibinfo {author} {\bibfnamefont {C.}~\bibnamefont
  {Arrouvel}},\ }\bibfield  {title} {\bibinfo {title} {Structural
  characterization of amorphous alumina and its polymorphs from
  first-principles xps and nmr calculations},\ }\href
  {https://doi.org/10.1103/PhysRevB.83.094201} {\bibfield  {journal} {\bibinfo
  {journal} {Phys. Rev. B}\ }\textbf {\bibinfo {volume} {83}},\ \bibinfo
  {pages} {094201} (\bibinfo {year} {2011})}\BibitemShut {NoStop}%
\bibitem [{\citenamefont {Tane}\ \emph {et~al.}(2011)\citenamefont {Tane},
  \citenamefont {Nakano}, \citenamefont {Nakamura}, \citenamefont {Ogi},
  \citenamefont {Ishimaru}, \citenamefont {Kimizuka},\ and\ \citenamefont
  {Nakajima}}]{tane_2011_nanovoid}%
  \BibitemOpen
  \bibfield  {author} {\bibinfo {author} {\bibfnamefont {M.}~\bibnamefont
  {Tane}}, \bibinfo {author} {\bibfnamefont {S.}~\bibnamefont {Nakano}},
  \bibinfo {author} {\bibfnamefont {R.}~\bibnamefont {Nakamura}}, \bibinfo
  {author} {\bibfnamefont {H.}~\bibnamefont {Ogi}}, \bibinfo {author}
  {\bibfnamefont {M.}~\bibnamefont {Ishimaru}}, \bibinfo {author}
  {\bibfnamefont {H.}~\bibnamefont {Kimizuka}},\ and\ \bibinfo {author}
  {\bibfnamefont {H.}~\bibnamefont {Nakajima}},\ }\bibfield  {title} {\bibinfo
  {title} {Nanovoid formation by change in amorphous structure through the
  annealing of amorphous {A}l$_2${O}$_3$ thin films},\ }\href
  {https://doi.org/https://doi.org/10.1016/j.actamat.2011.04.008} {\bibfield
  {journal} {\bibinfo  {journal} {Acta Materialia}\ }\textbf {\bibinfo {volume}
  {59}},\ \bibinfo {pages} {4631} (\bibinfo {year} {2011})}\BibitemShut
  {NoStop}%
\bibitem [{\citenamefont {Shi}\ \emph {et~al.}(2019)\citenamefont {Shi},
  \citenamefont {Alderman}, \citenamefont {Berman}, \citenamefont {Du},
  \citenamefont {Neuefeind}, \citenamefont {Tamalonis}, \citenamefont {Weber},
  \citenamefont {You},\ and\ \citenamefont {Benmore}}]{shi2019structure}%
  \BibitemOpen
  \bibfield  {author} {\bibinfo {author} {\bibfnamefont {C.}~\bibnamefont
  {Shi}}, \bibinfo {author} {\bibfnamefont {O.~L.}\ \bibnamefont {Alderman}},
  \bibinfo {author} {\bibfnamefont {D.}~\bibnamefont {Berman}}, \bibinfo
  {author} {\bibfnamefont {J.}~\bibnamefont {Du}}, \bibinfo {author}
  {\bibfnamefont {J.}~\bibnamefont {Neuefeind}}, \bibinfo {author}
  {\bibfnamefont {A.}~\bibnamefont {Tamalonis}}, \bibinfo {author}
  {\bibfnamefont {J.~R.}\ \bibnamefont {Weber}}, \bibinfo {author}
  {\bibfnamefont {J.}~\bibnamefont {You}},\ and\ \bibinfo {author}
  {\bibfnamefont {C.~J.}\ \bibnamefont {Benmore}},\ }\bibfield  {title}
  {\bibinfo {title} {The structure of amorphous and deeply supercooled liquid
  alumina},\ }\href {https://doi.org/10.3389/fmats.2019.00038} {\bibfield
  {journal} {\bibinfo  {journal} {Frontiers in Materials}\ }\textbf {\bibinfo
  {volume} {6}},\ \bibinfo {pages} {38} (\bibinfo {year} {2019})}\BibitemShut
  {NoStop}%
\bibitem [{\citenamefont {Paz}\ \emph {et~al.}(2014)\citenamefont {Paz},
  \citenamefont {Lebedeva}, \citenamefont {Tokatly},\ and\ \citenamefont
  {Rubio}}]{paz_identification_2014}%
  \BibitemOpen
  \bibfield  {author} {\bibinfo {author} {\bibfnamefont {A.~P.}\ \bibnamefont
  {Paz}}, \bibinfo {author} {\bibfnamefont {I.~V.}\ \bibnamefont {Lebedeva}},
  \bibinfo {author} {\bibfnamefont {I.~V.}\ \bibnamefont {Tokatly}},\ and\
  \bibinfo {author} {\bibfnamefont {A.}~\bibnamefont {Rubio}},\ }\bibfield
  {title} {\bibinfo {title} {Identification of structural motifs as tunneling
  two-level systems in amorphous alumina at low temperatures},\ }\href
  {https://doi.org/10.1103/PhysRevB.90.224202} {\bibfield  {journal} {\bibinfo
  {journal} {Physical Review B}\ }\textbf {\bibinfo {volume} {90}},\ \bibinfo
  {pages} {224202} (\bibinfo {year} {2014})}\BibitemShut {NoStop}%
\bibitem [{\citenamefont {Elliott}(1984)}]{elliot_mro}%
  \BibitemOpen
  \bibfield  {author} {\bibinfo {author} {\bibfnamefont {S.~R.}\ \bibnamefont
  {Elliott}},\ }\href@noop {} {\emph {\bibinfo {title} {Physics of amorphous
  materials}}}\ (\bibinfo  {publisher} {Longman Group Ltd.},\ \bibinfo {year}
  {1984})\BibitemShut {NoStop}%
\bibitem [{\citenamefont {Young}\ \emph {et~al.}(2020)\citenamefont {Young},
  \citenamefont {Bedford}, \citenamefont {Yanguas-Gil}, \citenamefont
  {Letourneau}, \citenamefont {Coile}, \citenamefont {Mandia}, \citenamefont
  {Aoun}, \citenamefont {Cavanagh}, \citenamefont {George},\ and\ \citenamefont
  {Elam}}]{young_probing_2020}%
  \BibitemOpen
  \bibfield  {author} {\bibinfo {author} {\bibfnamefont {M.~J.}\ \bibnamefont
  {Young}}, \bibinfo {author} {\bibfnamefont {N.~M.}\ \bibnamefont {Bedford}},
  \bibinfo {author} {\bibfnamefont {A.}~\bibnamefont {Yanguas-Gil}}, \bibinfo
  {author} {\bibfnamefont {S.}~\bibnamefont {Letourneau}}, \bibinfo {author}
  {\bibfnamefont {M.}~\bibnamefont {Coile}}, \bibinfo {author} {\bibfnamefont
  {D.~J.}\ \bibnamefont {Mandia}}, \bibinfo {author} {\bibfnamefont
  {B.}~\bibnamefont {Aoun}}, \bibinfo {author} {\bibfnamefont {A.~S.}\
  \bibnamefont {Cavanagh}}, \bibinfo {author} {\bibfnamefont {S.~M.}\
  \bibnamefont {George}},\ and\ \bibinfo {author} {\bibfnamefont {J.~W.}\
  \bibnamefont {Elam}},\ }\bibfield  {title} {\bibinfo {title} {Probing the
  {Atomic}-{Scale} {Structure} of {Amorphous} {Aluminum} {Oxide} {Grown} by
  {Atomic} {Layer} {Deposition}},\ }\href
  {https://doi.org/10.1021/acsami.0c01905} {\bibfield  {journal} {\bibinfo
  {journal} {ACS Applied Materials \& Interfaces}\ }\textbf {\bibinfo {volume}
  {12}},\ \bibinfo {pages} {22804} (\bibinfo {year} {2020})}\BibitemShut
  {NoStop}%
\bibitem [{\citenamefont {Fiorentino}\ \emph {et~al.}(2023)\citenamefont
  {Fiorentino}, \citenamefont {Perego},\ and\ \citenamefont
  {Baroni}}]{Fiorentino_2023}%
  \BibitemOpen
  \bibfield  {author} {\bibinfo {author} {\bibfnamefont {A.}~\bibnamefont
  {Fiorentino}}, \bibinfo {author} {\bibfnamefont {P.}~\bibnamefont {Perego}},\
  and\ \bibinfo {author} {\bibfnamefont {S.}~\bibnamefont {Baroni}},\
  }\bibfield  {title} {\bibinfo {title} {Hydrodynamic finite-size scaling of
  the thermal conductivity in glasses},\ }\href
  {https://arxiv.org/abs/2303.07010} {\bibfield  {journal} {\bibinfo  {journal}
  {arXiv.2303.07010}\ } (\bibinfo {year} {2023})}\BibitemShut {NoStop}%
\bibitem [{\citenamefont {Baroni}\ \emph {et~al.}(2001)\citenamefont {Baroni},
  \citenamefont {de~Gironcoli}, \citenamefont {Dal~Corso},\ and\ \citenamefont
  {Giannozzi}}]{baroni_phonons_2001}%
  \BibitemOpen
  \bibfield  {author} {\bibinfo {author} {\bibfnamefont {S.}~\bibnamefont
  {Baroni}}, \bibinfo {author} {\bibfnamefont {S.}~\bibnamefont
  {de~Gironcoli}}, \bibinfo {author} {\bibfnamefont {A.}~\bibnamefont
  {Dal~Corso}},\ and\ \bibinfo {author} {\bibfnamefont {P.}~\bibnamefont
  {Giannozzi}},\ }\bibfield  {title} {\bibinfo {title} {Phonons and related
  crystal properties from density-functional perturbation theory},\ }\href
  {https://doi.org/10.1103/RevModPhys.73.515} {\bibfield  {journal} {\bibinfo
  {journal} {Reviews of Modern Physics}\ }\textbf {\bibinfo {volume} {73}},\
  \bibinfo {pages} {515} (\bibinfo {year} {2001})}\BibitemShut {NoStop}%
\bibitem [{\citenamefont {Wallace}(1998)}]{wallace1998thermodynamics}%
  \BibitemOpen
  \bibfield  {author} {\bibinfo {author} {\bibfnamefont {D.}~\bibnamefont
  {Wallace}},\ }\href {https://books.google.co.uk/books?id=qLzOmwSgMIsC} {\emph
  {\bibinfo {title} {Thermodynamics of Crystals}}},\ Dover books on physics\
  (\bibinfo  {publisher} {Dover Publications},\ \bibinfo {year}
  {1998})\BibitemShut {NoStop}%
\bibitem [{Note1()}]{Note1}%
  \BibitemOpen
  \bibinfo {note} {N is equal to the number of $\protect \bm {q}$ points used
  to sample the Brillouin Zone; in the case of an ideal glass (astronomically
  large disordered simulation cell) the Brillouin Zone reduces to the point
  $\protect \bm {q}=\protect \bm {0}$ only and $N=1$.}\BibitemShut {Stop}%
\bibitem [{\citenamefont {{\L}odziana}\ and\ \citenamefont
  {Parli{\'n}ski}(2003)}]{lodziana2003dynamical}%
  \BibitemOpen
  \bibfield  {author} {\bibinfo {author} {\bibfnamefont {Z.}~\bibnamefont
  {{\L}odziana}}\ and\ \bibinfo {author} {\bibfnamefont {K.}~\bibnamefont
  {Parli{\'n}ski}},\ }\bibfield  {title} {\bibinfo {title} {Dynamical stability
  of the $\alpha$ and $\theta$ phases of alumina},\ }\href
  {https://journals.aps.org/prb/abstract/10.1103/PhysRevB.67.174106} {\bibfield
   {journal} {\bibinfo  {journal} {Physical Review B}\ }\textbf {\bibinfo
  {volume} {67}},\ \bibinfo {pages} {174106} (\bibinfo {year}
  {2003})}\BibitemShut {NoStop}%
\bibitem [{\citenamefont {Simoncelli}\ \emph {et~al.}(2020)\citenamefont
  {Simoncelli}, \citenamefont {Marzari},\ and\ \citenamefont
  {Cepellotti}}]{PhysRevX.10.011019}%
  \BibitemOpen
  \bibfield  {author} {\bibinfo {author} {\bibfnamefont {M.}~\bibnamefont
  {Simoncelli}}, \bibinfo {author} {\bibfnamefont {N.}~\bibnamefont
  {Marzari}},\ and\ \bibinfo {author} {\bibfnamefont {A.}~\bibnamefont
  {Cepellotti}},\ }\bibfield  {title} {\bibinfo {title} {Generalization of
  fourier's law into viscous heat equations},\ }\href
  {https://doi.org/10.1103/PhysRevX.10.011019} {\bibfield  {journal} {\bibinfo
  {journal} {Phys. Rev. X}\ }\textbf {\bibinfo {volume} {10}},\ \bibinfo
  {pages} {011019} (\bibinfo {year} {2020})}\BibitemShut {NoStop}%
\bibitem [{\citenamefont {Di~Lucente}\ \emph {et~al.}(2023)\citenamefont
  {Di~Lucente}, \citenamefont {Simoncelli},\ and\ \citenamefont
  {Marzari}}]{Lucente}%
  \BibitemOpen
  \bibfield  {author} {\bibinfo {author} {\bibfnamefont {E.}~\bibnamefont
  {Di~Lucente}}, \bibinfo {author} {\bibfnamefont {M.}~\bibnamefont
  {Simoncelli}},\ and\ \bibinfo {author} {\bibfnamefont {N.}~\bibnamefont
  {Marzari}},\ }\bibfield  {title} {\bibinfo {title} {Crossover from boltzmann
  to wigner thermal transport in thermoelectric skutterudites},\ }\href
  {https://doi.org/10.48550/arXiv.2303.07019} {\bibfield  {journal} {\bibinfo
  {journal} {arXiv.2303.07019}\ } (\bibinfo {year} {2023})}\BibitemShut
  {NoStop}%
\bibitem [{\citenamefont {Liu}\ \emph {et~al.}(2023)\citenamefont {Liu},
  \citenamefont {Liang}, \citenamefont {Yang}, \citenamefont {Yang},
  \citenamefont {Yang}, \citenamefont {Song}, \citenamefont {Mei},
  \citenamefont {Csányi},\ and\ \citenamefont {Cao}}]{liu_unraveling_nodate}%
  \BibitemOpen
  \bibfield  {author} {\bibinfo {author} {\bibfnamefont {Y.}~\bibnamefont
  {Liu}}, \bibinfo {author} {\bibfnamefont {H.}~\bibnamefont {Liang}}, \bibinfo
  {author} {\bibfnamefont {L.}~\bibnamefont {Yang}}, \bibinfo {author}
  {\bibfnamefont {G.}~\bibnamefont {Yang}}, \bibinfo {author} {\bibfnamefont
  {H.}~\bibnamefont {Yang}}, \bibinfo {author} {\bibfnamefont {S.}~\bibnamefont
  {Song}}, \bibinfo {author} {\bibfnamefont {Z.}~\bibnamefont {Mei}}, \bibinfo
  {author} {\bibfnamefont {G.}~\bibnamefont {Csányi}},\ and\ \bibinfo {author}
  {\bibfnamefont {B.}~\bibnamefont {Cao}},\ }\bibfield  {title} {\bibinfo
  {title} {Unraveling {Thermal} {Transport} {Correlated} with {Atomistic}
  {Structures} in {Amorphous} {Gallium} {Oxide} via {Machine} {Learning}
  {Combined} with {Experiments}},\ }\href
  {https://doi.org/10.1002/adma.202210873} {\bibfield  {journal} {\bibinfo
  {journal} {Advanced Materials}\ }\textbf {\bibinfo {volume} {n/a}},\ \bibinfo
  {pages} {2210873} (\bibinfo {year} {2023})}\BibitemShut {NoStop}%
\bibitem [{\citenamefont {Paulatto}\ \emph {et~al.}(2013)\citenamefont
  {Paulatto}, \citenamefont {Mauri},\ and\ \citenamefont
  {Lazzeri}}]{paulatto2013anharmonic}%
  \BibitemOpen
  \bibfield  {author} {\bibinfo {author} {\bibfnamefont {L.}~\bibnamefont
  {Paulatto}}, \bibinfo {author} {\bibfnamefont {F.}~\bibnamefont {Mauri}},\
  and\ \bibinfo {author} {\bibfnamefont {M.}~\bibnamefont {Lazzeri}},\
  }\bibfield  {title} {\bibinfo {title} {{Anharmonic properties from a
  generalized third-order ab initio approach: Theory and applications to
  graphite and graphene}},\ }\href {https://doi.org/10.1103/PhysRevB.87.214303}
  {\bibfield  {journal} {\bibinfo  {journal} {Phys. Rev. B}\ }\textbf {\bibinfo
  {volume} {87}},\ \bibinfo {pages} {214303} (\bibinfo {year}
  {2013})}\BibitemShut {NoStop}%
\bibitem [{\citenamefont {Fugallo}\ \emph {et~al.}(2013)\citenamefont
  {Fugallo}, \citenamefont {Lazzeri}, \citenamefont {Paulatto},\ and\
  \citenamefont {Mauri}}]{fugallo2013ab}%
  \BibitemOpen
  \bibfield  {author} {\bibinfo {author} {\bibfnamefont {G.}~\bibnamefont
  {Fugallo}}, \bibinfo {author} {\bibfnamefont {M.}~\bibnamefont {Lazzeri}},
  \bibinfo {author} {\bibfnamefont {L.}~\bibnamefont {Paulatto}},\ and\
  \bibinfo {author} {\bibfnamefont {F.}~\bibnamefont {Mauri}},\ }\bibfield
  {title} {\bibinfo {title} {{Ab initio variational approach for evaluating
  lattice thermal conductivity}},\ }\href
  {https://doi.org/10.1103/PhysRevB.88.045430} {\bibfield  {journal} {\bibinfo
  {journal} {Phys. Rev. B}\ }\textbf {\bibinfo {volume} {88}},\ \bibinfo
  {pages} {045430} (\bibinfo {year} {2013})}\BibitemShut {NoStop}%
\bibitem [{\citenamefont {Togo}\ \emph {et~al.}(2015)\citenamefont {Togo},
  \citenamefont {Chaput},\ and\ \citenamefont {Tanaka}}]{phono3py}%
  \BibitemOpen
  \bibfield  {author} {\bibinfo {author} {\bibfnamefont {A.}~\bibnamefont
  {Togo}}, \bibinfo {author} {\bibfnamefont {L.}~\bibnamefont {Chaput}},\ and\
  \bibinfo {author} {\bibfnamefont {I.}~\bibnamefont {Tanaka}},\ }\bibfield
  {title} {\bibinfo {title} {Distributions of phonon lifetimes in brillouin
  zones},\ }\href {https://doi.org/10.1103/PhysRevB.91.094306} {\bibfield
  {journal} {\bibinfo  {journal} {Phys. Rev. B}\ }\textbf {\bibinfo {volume}
  {91}},\ \bibinfo {pages} {094306} (\bibinfo {year} {2015})}\BibitemShut
  {NoStop}%
\bibitem [{\citenamefont {Tadano}\ \emph {et~al.}(2014)\citenamefont {Tadano},
  \citenamefont {Gohda},\ and\ \citenamefont {Tsuneyuki}}]{alamode}%
  \BibitemOpen
  \bibfield  {author} {\bibinfo {author} {\bibfnamefont {T.}~\bibnamefont
  {Tadano}}, \bibinfo {author} {\bibfnamefont {Y.}~\bibnamefont {Gohda}},\ and\
  \bibinfo {author} {\bibfnamefont {S.}~\bibnamefont {Tsuneyuki}},\ }\bibfield
  {title} {\bibinfo {title} {{Anharmonic force constants extracted from
  first-principles molecular dynamics: applications to heat transfer
  simulations}},\ }\href {https://doi.org/10.1088/0953-8984/26/22/225402}
  {\bibfield  {journal} {\bibinfo  {journal} {J. Phys. Condens. Matter}\
  }\textbf {\bibinfo {volume} {26}},\ \bibinfo {pages} {225402} (\bibinfo
  {year} {2014})}\BibitemShut {NoStop}%
\bibitem [{\citenamefont {Cepellotti}\ \emph {et~al.}(2022)\citenamefont
  {Cepellotti}, \citenamefont {Coulter}, \citenamefont {Johansson},
  \citenamefont {Fedorova},\ and\ \citenamefont
  {Kozinsky}}]{cepellotti_phoebe_2022}%
  \BibitemOpen
  \bibfield  {author} {\bibinfo {author} {\bibfnamefont {A.}~\bibnamefont
  {Cepellotti}}, \bibinfo {author} {\bibfnamefont {J.}~\bibnamefont {Coulter}},
  \bibinfo {author} {\bibfnamefont {A.}~\bibnamefont {Johansson}}, \bibinfo
  {author} {\bibfnamefont {N.~S.}\ \bibnamefont {Fedorova}},\ and\ \bibinfo
  {author} {\bibfnamefont {B.}~\bibnamefont {Kozinsky}},\ }\bibfield  {title}
  {\bibinfo {title} {Phoebe: a high-performance framework for solving phonon
  and electron {Boltzmann} transport equations},\ }\href
  {https://doi.org/10.1088/2515-7639/ac86f6} {\bibfield  {journal} {\bibinfo
  {journal} {Journal of Physics: Materials}\ }\textbf {\bibinfo {volume} {5}},\
  \bibinfo {pages} {035003} (\bibinfo {year} {2022})},\ \bibinfo {note}
  {publisher: IOP Publishing}\BibitemShut {NoStop}%
\bibitem [{\citenamefont {Carrete}\ \emph {et~al.}(2017)\citenamefont
  {Carrete}, \citenamefont {Vermeersch}, \citenamefont {Katre}, \citenamefont
  {van Roekeghem}, \citenamefont {Wang}, \citenamefont {Madsen},\ and\
  \citenamefont {Mingo}}]{carrete_almabte_2017}%
  \BibitemOpen
  \bibfield  {author} {\bibinfo {author} {\bibfnamefont {J.}~\bibnamefont
  {Carrete}}, \bibinfo {author} {\bibfnamefont {B.}~\bibnamefont {Vermeersch}},
  \bibinfo {author} {\bibfnamefont {A.}~\bibnamefont {Katre}}, \bibinfo
  {author} {\bibfnamefont {A.}~\bibnamefont {van Roekeghem}}, \bibinfo {author}
  {\bibfnamefont {T.}~\bibnamefont {Wang}}, \bibinfo {author} {\bibfnamefont
  {G.~K.}\ \bibnamefont {Madsen}},\ and\ \bibinfo {author} {\bibfnamefont
  {N.}~\bibnamefont {Mingo}},\ }\bibfield  {title} {\bibinfo {title} {{almaBTE}
  : {A} solver of the space–time dependent {Boltzmann} transport equation for
  phonons in structured materials},\ }\href
  {https://doi.org/10.1016/j.cpc.2017.06.023} {\bibfield  {journal} {\bibinfo
  {journal} {Computer Physics Communications}\ }\textbf {\bibinfo {volume}
  {220}},\ \bibinfo {pages} {351} (\bibinfo {year} {2017})}\BibitemShut
  {NoStop}%
\bibitem [{\citenamefont {Barbalinardo}\ \emph {et~al.}(2020)\citenamefont
  {Barbalinardo}, \citenamefont {Chen}, \citenamefont {Lundgren},\ and\
  \citenamefont {Donadio}}]{kaldo}%
  \BibitemOpen
  \bibfield  {author} {\bibinfo {author} {\bibfnamefont {G.}~\bibnamefont
  {Barbalinardo}}, \bibinfo {author} {\bibfnamefont {Z.}~\bibnamefont {Chen}},
  \bibinfo {author} {\bibfnamefont {N.~W.}\ \bibnamefont {Lundgren}},\ and\
  \bibinfo {author} {\bibfnamefont {D.}~\bibnamefont {Donadio}},\ }\bibfield
  {title} {\bibinfo {title} {{Efficient anharmonic lattice dynamics
  calculations of thermal transport in crystalline and disordered solids}},\
  }\href@noop {} {\bibfield  {journal} {\bibinfo  {journal} {Journal of Applied
  Physics}\ }\textbf {\bibinfo {volume} {128}},\ \bibinfo {pages} {135104}
  (\bibinfo {year} {2020})}\BibitemShut {NoStop}%
\bibitem [{\citenamefont {Tamura}(1983)}]{tamura_isotope_1983}%
  \BibitemOpen
  \bibfield  {author} {\bibinfo {author} {\bibfnamefont {S.-i.}\ \bibnamefont
  {Tamura}},\ }\bibfield  {title} {\bibinfo {title} {Isotope scattering of
  dispersive phonons in {Ge}},\ }\href
  {https://doi.org/10.1103/PhysRevB.27.858} {\bibfield  {journal} {\bibinfo
  {journal} {Physical Review B}\ }\textbf {\bibinfo {volume} {27}},\ \bibinfo
  {pages} {858} (\bibinfo {year} {1983})}\BibitemShut {NoStop}%
\bibitem [{Note2()}]{Note2}%
  \BibitemOpen
  \bibinfo {note} {A necessary condition for this to happen is to have
  vibrations that are not localized in the Anderson sense \cite
  {Anderson_localization}, \protect \textit {i.e.} to have non-zero velocity
  operator elements in Eq.~(\ref
  {eq:thermal_conductivity_combined}).}\BibitemShut {Stop}%
\bibitem [{\citenamefont {Allen}\ and\ \citenamefont
  {Feldman}(1993)}]{allen1993thermal}%
  \BibitemOpen
  \bibfield  {author} {\bibinfo {author} {\bibfnamefont {P.~B.}\ \bibnamefont
  {Allen}}\ and\ \bibinfo {author} {\bibfnamefont {J.~L.}\ \bibnamefont
  {Feldman}},\ }\bibfield  {title} {\bibinfo {title} {{Thermal conductivity of
  disordered harmonic solids}},\ }\href
  {https://doi.org/10.1103/PhysRevB.48.12581} {\bibfield  {journal} {\bibinfo
  {journal} {Phys. Rev. B}\ }\textbf {\bibinfo {volume} {48}},\ \bibinfo
  {pages} {12581} (\bibinfo {year} {1993})}\BibitemShut {NoStop}%
\bibitem [{\citenamefont {Caldarelli}\ \emph {et~al.}(2022)\citenamefont
  {Caldarelli}, \citenamefont {Simoncelli}, \citenamefont {Marzari},
  \citenamefont {Mauri},\ and\ \citenamefont
  {Benfatto}}]{caldarelli_many-body_2022}%
  \BibitemOpen
  \bibfield  {author} {\bibinfo {author} {\bibfnamefont {G.}~\bibnamefont
  {Caldarelli}}, \bibinfo {author} {\bibfnamefont {M.}~\bibnamefont
  {Simoncelli}}, \bibinfo {author} {\bibfnamefont {N.}~\bibnamefont {Marzari}},
  \bibinfo {author} {\bibfnamefont {F.}~\bibnamefont {Mauri}},\ and\ \bibinfo
  {author} {\bibfnamefont {L.}~\bibnamefont {Benfatto}},\ }\bibfield  {title}
  {\bibinfo {title} {Many-body {Green}'s function approach to lattice thermal
  transport},\ }\href {https://doi.org/10.1103/PhysRevB.106.024312} {\bibfield
  {journal} {\bibinfo  {journal} {Physical Review B}\ }\textbf {\bibinfo
  {volume} {106}},\ \bibinfo {pages} {024312} (\bibinfo {year}
  {2022})}\BibitemShut {NoStop}%
\bibitem [{\citenamefont {Fiorentino}\ and\ \citenamefont
  {Baroni}(2023)}]{PhysRevB.107.054311}%
  \BibitemOpen
  \bibfield  {author} {\bibinfo {author} {\bibfnamefont {A.}~\bibnamefont
  {Fiorentino}}\ and\ \bibinfo {author} {\bibfnamefont {S.}~\bibnamefont
  {Baroni}},\ }\bibfield  {title} {\bibinfo {title} {From green-kubo to the
  full boltzmann kinetic approach to heat transport in crystals and glasses},\
  }\href {https://doi.org/10.1103/PhysRevB.107.054311} {\bibfield  {journal}
  {\bibinfo  {journal} {Phys. Rev. B}\ }\textbf {\bibinfo {volume} {107}},\
  \bibinfo {pages} {054311} (\bibinfo {year} {2023})}\BibitemShut {NoStop}%
\bibitem [{\citenamefont {Isaeva}\ \emph {et~al.}(2019)\citenamefont {Isaeva},
  \citenamefont {Barbalinardo}, \citenamefont {Donadio},\ and\ \citenamefont
  {Baroni}}]{isaeva2019modeling}%
  \BibitemOpen
  \bibfield  {author} {\bibinfo {author} {\bibfnamefont {L.}~\bibnamefont
  {Isaeva}}, \bibinfo {author} {\bibfnamefont {G.}~\bibnamefont
  {Barbalinardo}}, \bibinfo {author} {\bibfnamefont {D.}~\bibnamefont
  {Donadio}},\ and\ \bibinfo {author} {\bibfnamefont {S.}~\bibnamefont
  {Baroni}},\ }\bibfield  {title} {\bibinfo {title} {{Modeling heat transport
  in crystals and glasses from a unified lattice-dynamical approach}},\ }\href
  {https://doi.org/10.1038/s41467-019-11572-4} {\bibfield  {journal} {\bibinfo
  {journal} {Nat. Commun.}\ }\textbf {\bibinfo {volume} {10}},\ \bibinfo
  {pages} {3853} (\bibinfo {year} {2019})}\BibitemShut {NoStop}%
\bibitem [{\citenamefont {Lundgren}\ \emph {et~al.}(2021)\citenamefont
  {Lundgren}, \citenamefont {Barbalinardo},\ and\ \citenamefont
  {Donadio}}]{lundgren_mode_2021}%
  \BibitemOpen
  \bibfield  {author} {\bibinfo {author} {\bibfnamefont {N.~W.}\ \bibnamefont
  {Lundgren}}, \bibinfo {author} {\bibfnamefont {G.}~\bibnamefont
  {Barbalinardo}},\ and\ \bibinfo {author} {\bibfnamefont {D.}~\bibnamefont
  {Donadio}},\ }\bibfield  {title} {\bibinfo {title} {Mode localization and
  suppressed heat transport in amorphous alloys},\ }\href
  {https://doi.org/10.1103/PhysRevB.103.024204} {\bibfield  {journal} {\bibinfo
   {journal} {Physical Review B}\ }\textbf {\bibinfo {volume} {103}},\ \bibinfo
  {pages} {024204} (\bibinfo {year} {2021})}\BibitemShut {NoStop}%
\bibitem [{Note3()}]{Note3}%
  \BibitemOpen
  \bibinfo {note} {More precisely , Refs.~\cite
  {isaeva2019modeling,lundgren_mode_2021} evaluated the bulk limit of the
  conductivity relying on empirical interatomic potentials and atomistic models
  containing thousands of atoms, \protect \textit {i.e.} having a size large
  enough to achieve computational convergence by evaluating Eq.~(\ref
  {eq:thermal_conductivity_combined}) at $\protect \bm {q}=\protect \bm {0}$
  only and without relying on the Voigt regularization. See Ref.~\cite
  {simoncelli_thermal_2022} for details on the conditions under which
  evaluating Eq.~(\ref {eq:thermal_conductivity_combined}) yields equivalent
  results with or without relying on the Fourier interpolation and Voigt
  regularization.}\BibitemShut {Stop}%
\bibitem [{\citenamefont {Lee}\ \emph {et~al.}(1995)\citenamefont {Lee},
  \citenamefont {Cahill},\ and\ \citenamefont {Allen}}]{Lee1995}%
  \BibitemOpen
  \bibfield  {author} {\bibinfo {author} {\bibfnamefont {S.-M.}\ \bibnamefont
  {Lee}}, \bibinfo {author} {\bibfnamefont {D.~G.}\ \bibnamefont {Cahill}},\
  and\ \bibinfo {author} {\bibfnamefont {T.~H.}\ \bibnamefont {Allen}},\
  }\bibfield  {title} {\bibinfo {title} {Thermal conductivity of sputtered
  oxide films},\ }\href {https://doi.org/10.1103/PhysRevB.52.253} {\bibfield
  {journal} {\bibinfo  {journal} {Phys. Rev. B}\ }\textbf {\bibinfo {volume}
  {52}},\ \bibinfo {pages} {253} (\bibinfo {year} {1995})}\BibitemShut
  {NoStop}%
\bibitem [{\citenamefont {Monachon}\ and\ \citenamefont
  {Weber}(2015)}]{Monachon2015}%
  \BibitemOpen
  \bibfield  {author} {\bibinfo {author} {\bibfnamefont {C.}~\bibnamefont
  {Monachon}}\ and\ \bibinfo {author} {\bibfnamefont {L.}~\bibnamefont
  {Weber}},\ }\bibfield  {title} {\bibinfo {title} {Influence of a nanometric
  {A}l$_2${O}$_3$ interlayer on the thermal conductance of an al/(si, diamond)
  interface},\ }\href {https://doi.org/https://doi.org/10.1002/adem.201400060}
  {\bibfield  {journal} {\bibinfo  {journal} {Advanced Engineering Materials}\
  }\textbf {\bibinfo {volume} {17}},\ \bibinfo {pages} {68} (\bibinfo {year}
  {2015})}\BibitemShut {NoStop}%
\bibitem [{\citenamefont {Gorham}\ \emph {et~al.}(2014)\citenamefont {Gorham},
  \citenamefont {Gaskins}, \citenamefont {Parsons}, \citenamefont {Losego},\
  and\ \citenamefont {Hopkins}}]{gorham_density_2014}%
  \BibitemOpen
  \bibfield  {author} {\bibinfo {author} {\bibfnamefont {C.~S.}\ \bibnamefont
  {Gorham}}, \bibinfo {author} {\bibfnamefont {J.~T.}\ \bibnamefont {Gaskins}},
  \bibinfo {author} {\bibfnamefont {G.~N.}\ \bibnamefont {Parsons}}, \bibinfo
  {author} {\bibfnamefont {M.~D.}\ \bibnamefont {Losego}},\ and\ \bibinfo
  {author} {\bibfnamefont {P.~E.}\ \bibnamefont {Hopkins}},\ }\bibfield
  {title} {\bibinfo {title} {Density dependence of the room temperature thermal
  conductivity of atomic layer deposition-grown amorphous alumina
  ({{A}l$_2${O}$_3$})},\ }\href {https://doi.org/10.1063/1.4885415} {\bibfield
  {journal} {\bibinfo  {journal} {Applied Physics Letters}\ }\textbf {\bibinfo
  {volume} {104}},\ \bibinfo {pages} {253107} (\bibinfo {year}
  {2014})}\BibitemShut {NoStop}%
\bibitem [{\citenamefont {Zhou}\ \emph {et~al.}(2020)\citenamefont {Zhou},
  \citenamefont {Cheng}, \citenamefont {Chen}, \citenamefont {Xie},
  \citenamefont {Wang},\ and\ \citenamefont {Zhang}}]{zhou_2020_thermal}%
  \BibitemOpen
  \bibfield  {author} {\bibinfo {author} {\bibfnamefont {W.-X.}\ \bibnamefont
  {Zhou}}, \bibinfo {author} {\bibfnamefont {Y.}~\bibnamefont {Cheng}},
  \bibinfo {author} {\bibfnamefont {K.-Q.}\ \bibnamefont {Chen}}, \bibinfo
  {author} {\bibfnamefont {G.}~\bibnamefont {Xie}}, \bibinfo {author}
  {\bibfnamefont {T.}~\bibnamefont {Wang}},\ and\ \bibinfo {author}
  {\bibfnamefont {G.}~\bibnamefont {Zhang}},\ }\bibfield  {title} {\bibinfo
  {title} {Thermal conductivity of amorphous materials},\ }\href
  {https://doi.org/https://doi.org/10.1002/adfm.201903829} {\bibfield
  {journal} {\bibinfo  {journal} {Advanced Functional Materials}\ }\textbf
  {\bibinfo {volume} {30}},\ \bibinfo {pages} {1903829} (\bibinfo {year}
  {2020})}\BibitemShut {NoStop}%
\bibitem [{Note4()}]{Note4}%
  \BibitemOpen
  \bibinfo {note} {We note, in passing, that Refs. \cite
  {Monachon2015,Lee2017,gorham_density_2014} did not observe a significant
  dependence of the thermal conductivity from the substrate on which the
  am-Al$_2$O$_3$ sample was grown.}\BibitemShut {Stop}%
\bibitem [{\citenamefont {Lee}\ \emph {et~al.}(2017)\citenamefont {Lee},
  \citenamefont {Choi}, \citenamefont {Kim}, \citenamefont {Kim}, \citenamefont
  {Lee}, \citenamefont {Im}, \citenamefont {Kwon}, \citenamefont {Seo},
  \citenamefont {Shin},\ and\ \citenamefont {Moon}}]{Lee2017}%
  \BibitemOpen
  \bibfield  {author} {\bibinfo {author} {\bibfnamefont {S.-M.}\ \bibnamefont
  {Lee}}, \bibinfo {author} {\bibfnamefont {W.}~\bibnamefont {Choi}}, \bibinfo
  {author} {\bibfnamefont {J.}~\bibnamefont {Kim}}, \bibinfo {author}
  {\bibfnamefont {T.}~\bibnamefont {Kim}}, \bibinfo {author} {\bibfnamefont
  {J.}~\bibnamefont {Lee}}, \bibinfo {author} {\bibfnamefont {S.~Y.}\
  \bibnamefont {Im}}, \bibinfo {author} {\bibfnamefont {J.~Y.}\ \bibnamefont
  {Kwon}}, \bibinfo {author} {\bibfnamefont {S.}~\bibnamefont {Seo}}, \bibinfo
  {author} {\bibfnamefont {M.}~\bibnamefont {Shin}},\ and\ \bibinfo {author}
  {\bibfnamefont {S.~E.}\ \bibnamefont {Moon}},\ }\bibfield  {title} {\bibinfo
  {title} {Thermal conductivity and thermal boundary resistances of ald
  {Al}$_2${O}$_3$ films on {S}i and sapphire},\ }\href
  {https://doi.org/10.1007/s10765-017-2308-5} {\bibfield  {journal} {\bibinfo
  {journal} {International Journal of Thermophysics}\ }\textbf {\bibinfo
  {volume} {38}},\ \bibinfo {pages} {176} (\bibinfo {year} {2017})}\BibitemShut
  {NoStop}%
\bibitem [{\citenamefont {Anderson}(1958)}]{Anderson_localization}%
  \BibitemOpen
  \bibfield  {author} {\bibinfo {author} {\bibfnamefont {P.~W.}\ \bibnamefont
  {Anderson}},\ }\bibfield  {title} {\bibinfo {title} {Absence of diffusion in
  certain random lattices},\ }\href {https://doi.org/10.1103/PhysRev.109.1492}
  {\bibfield  {journal} {\bibinfo  {journal} {Phys. Rev.}\ }\textbf {\bibinfo
  {volume} {109}},\ \bibinfo {pages} {1492} (\bibinfo {year}
  {1958})}\BibitemShut {NoStop}%
\bibitem [{\citenamefont {Kresse}\ and\ \citenamefont
  {Hafner}(1993)}]{kresse1993ab}%
  \BibitemOpen
  \bibfield  {author} {\bibinfo {author} {\bibfnamefont {G.}~\bibnamefont
  {Kresse}}\ and\ \bibinfo {author} {\bibfnamefont {J.}~\bibnamefont
  {Hafner}},\ }\bibfield  {title} {\bibinfo {title} {Ab initio molecular
  dynamics for liquid metals},\ }\href
  {https://journals.aps.org/prb/abstract/10.1103/PhysRevB.47.558} {\bibfield
  {journal} {\bibinfo  {journal} {Physical review B}\ }\textbf {\bibinfo
  {volume} {47}},\ \bibinfo {pages} {558} (\bibinfo {year} {1993})}\BibitemShut
  {NoStop}%
\bibitem [{\citenamefont {Giannozzi}\ \emph {et~al.}(2009)\citenamefont
  {Giannozzi}, \citenamefont {Baroni}, \citenamefont {Bonini}, \citenamefont
  {Calandra}, \citenamefont {Car}, \citenamefont {Cavazzoni}, \citenamefont
  {Ceresoli}, \citenamefont {Chiarotti}, \citenamefont {Cococcioni},
  \citenamefont {Dabo}, \citenamefont {Corso}, \citenamefont {Gironcoli},
  \citenamefont {Fabris}, \citenamefont {Fratesi}, \citenamefont {Gebauer},
  \citenamefont {Gerstmann}, \citenamefont {Gougoussis}, \citenamefont
  {Kokalj}, \citenamefont {Lazzeri}, \citenamefont {Martin-Samos},
  \citenamefont {Marzari}, \citenamefont {Mauri}, \citenamefont {Mazzarello},
  \citenamefont {Paolini}, \citenamefont {Pasquarello}, \citenamefont
  {Paulatto}, \citenamefont {Sbraccia}, \citenamefont {Scandolo}, \citenamefont
  {Sclauzero}, \citenamefont {Seitsonen}, \citenamefont {Smogunov},
  \citenamefont {Umari},\ and\ \citenamefont
  {Wentzcovitch}}]{giannozzi_quantum_2009}%
  \BibitemOpen
  \bibfield  {author} {\bibinfo {author} {\bibfnamefont {P.}~\bibnamefont
  {Giannozzi}}, \bibinfo {author} {\bibfnamefont {S.}~\bibnamefont {Baroni}},
  \bibinfo {author} {\bibfnamefont {N.}~\bibnamefont {Bonini}}, \bibinfo
  {author} {\bibfnamefont {M.}~\bibnamefont {Calandra}}, \bibinfo {author}
  {\bibfnamefont {R.}~\bibnamefont {Car}}, \bibinfo {author} {\bibfnamefont
  {C.}~\bibnamefont {Cavazzoni}}, \bibinfo {author} {\bibfnamefont
  {D.}~\bibnamefont {Ceresoli}}, \bibinfo {author} {\bibfnamefont {G.~L.}\
  \bibnamefont {Chiarotti}}, \bibinfo {author} {\bibfnamefont {M.}~\bibnamefont
  {Cococcioni}}, \bibinfo {author} {\bibfnamefont {I.}~\bibnamefont {Dabo}},
  \bibinfo {author} {\bibfnamefont {A.~D.}\ \bibnamefont {Corso}}, \bibinfo
  {author} {\bibfnamefont {S.~d.}\ \bibnamefont {Gironcoli}}, \bibinfo {author}
  {\bibfnamefont {S.}~\bibnamefont {Fabris}}, \bibinfo {author} {\bibfnamefont
  {G.}~\bibnamefont {Fratesi}}, \bibinfo {author} {\bibfnamefont
  {R.}~\bibnamefont {Gebauer}}, \bibinfo {author} {\bibfnamefont
  {U.}~\bibnamefont {Gerstmann}}, \bibinfo {author} {\bibfnamefont
  {C.}~\bibnamefont {Gougoussis}}, \bibinfo {author} {\bibfnamefont
  {A.}~\bibnamefont {Kokalj}}, \bibinfo {author} {\bibfnamefont
  {M.}~\bibnamefont {Lazzeri}}, \bibinfo {author} {\bibfnamefont
  {L.}~\bibnamefont {Martin-Samos}}, \bibinfo {author} {\bibfnamefont
  {N.}~\bibnamefont {Marzari}}, \bibinfo {author} {\bibfnamefont
  {F.}~\bibnamefont {Mauri}}, \bibinfo {author} {\bibfnamefont
  {R.}~\bibnamefont {Mazzarello}}, \bibinfo {author} {\bibfnamefont
  {S.}~\bibnamefont {Paolini}}, \bibinfo {author} {\bibfnamefont
  {A.}~\bibnamefont {Pasquarello}}, \bibinfo {author} {\bibfnamefont
  {L.}~\bibnamefont {Paulatto}}, \bibinfo {author} {\bibfnamefont
  {C.}~\bibnamefont {Sbraccia}}, \bibinfo {author} {\bibfnamefont
  {S.}~\bibnamefont {Scandolo}}, \bibinfo {author} {\bibfnamefont
  {G.}~\bibnamefont {Sclauzero}}, \bibinfo {author} {\bibfnamefont {A.~P.}\
  \bibnamefont {Seitsonen}}, \bibinfo {author} {\bibfnamefont {A.}~\bibnamefont
  {Smogunov}}, \bibinfo {author} {\bibfnamefont {P.}~\bibnamefont {Umari}},\
  and\ \bibinfo {author} {\bibfnamefont {R.~M.}\ \bibnamefont {Wentzcovitch}},\
  }\bibfield  {title} {\bibinfo {title} {{QUANTUM} {ESPRESSO}: a modular and
  open-source software project for quantum simulations of materials},\ }\href
  {https://doi.org/10.1088/0953-8984/21/39/395502} {\bibfield  {journal}
  {\bibinfo  {journal} {Journal of Physics: Condensed Matter}\ }\textbf
  {\bibinfo {volume} {21}},\ \bibinfo {pages} {395502} (\bibinfo {year}
  {2009})}\BibitemShut {NoStop}%
\bibitem [{\citenamefont {Giannozzi}\ \emph {et~al.}(2017)\citenamefont
  {Giannozzi}, \citenamefont {Andreussi}, \citenamefont {Brumme}, \citenamefont
  {Bunau}, \citenamefont {Buongiorno~Nardelli}, \citenamefont {Calandra},
  \citenamefont {Car}, \citenamefont {Cavazzoni}, \citenamefont {Ceresoli},
  \citenamefont {Cococcioni}, \citenamefont {Colonna}, \citenamefont
  {Carnimeo}, \citenamefont {Dal~Corso}, \citenamefont {de~Gironcoli},
  \citenamefont {Delugas}, \citenamefont {DiStasio}, \citenamefont {Ferretti},
  \citenamefont {Floris}, \citenamefont {Fratesi}, \citenamefont {Fugallo},
  \citenamefont {Gebauer}, \citenamefont {Gerstmann}, \citenamefont {Giustino},
  \citenamefont {Gorni}, \citenamefont {Jia}, \citenamefont {Kawamura},
  \citenamefont {Ko}, \citenamefont {Kokalj}, \citenamefont {Küçükbenli},
  \citenamefont {Lazzeri}, \citenamefont {Marsili}, \citenamefont {Marzari},
  \citenamefont {Mauri}, \citenamefont {Nguyen}, \citenamefont {Nguyen},
  \citenamefont {Otero-de-la Roza}, \citenamefont {Paulatto}, \citenamefont
  {Poncé}, \citenamefont {Rocca}, \citenamefont {Sabatini}, \citenamefont
  {Santra}, \citenamefont {Schlipf}, \citenamefont {Seitsonen}, \citenamefont
  {Smogunov}, \citenamefont {Timrov}, \citenamefont {Thonhauser}, \citenamefont
  {Umari}, \citenamefont {Vast}, \citenamefont {Wu},\ and\ \citenamefont
  {Baroni}}]{giannozzi_advanced_2017}%
  \BibitemOpen
  \bibfield  {author} {\bibinfo {author} {\bibfnamefont {P.}~\bibnamefont
  {Giannozzi}}, \bibinfo {author} {\bibfnamefont {O.}~\bibnamefont
  {Andreussi}}, \bibinfo {author} {\bibfnamefont {T.}~\bibnamefont {Brumme}},
  \bibinfo {author} {\bibfnamefont {O.}~\bibnamefont {Bunau}}, \bibinfo
  {author} {\bibfnamefont {M.}~\bibnamefont {Buongiorno~Nardelli}}, \bibinfo
  {author} {\bibfnamefont {M.}~\bibnamefont {Calandra}}, \bibinfo {author}
  {\bibfnamefont {R.}~\bibnamefont {Car}}, \bibinfo {author} {\bibfnamefont
  {C.}~\bibnamefont {Cavazzoni}}, \bibinfo {author} {\bibfnamefont
  {D.}~\bibnamefont {Ceresoli}}, \bibinfo {author} {\bibfnamefont
  {M.}~\bibnamefont {Cococcioni}}, \bibinfo {author} {\bibfnamefont
  {N.}~\bibnamefont {Colonna}}, \bibinfo {author} {\bibfnamefont
  {I.}~\bibnamefont {Carnimeo}}, \bibinfo {author} {\bibfnamefont
  {A.}~\bibnamefont {Dal~Corso}}, \bibinfo {author} {\bibfnamefont
  {S.}~\bibnamefont {de~Gironcoli}}, \bibinfo {author} {\bibfnamefont
  {P.}~\bibnamefont {Delugas}}, \bibinfo {author} {\bibfnamefont {R.~A.}\
  \bibnamefont {DiStasio}}, \bibinfo {author} {\bibfnamefont {A.}~\bibnamefont
  {Ferretti}}, \bibinfo {author} {\bibfnamefont {A.}~\bibnamefont {Floris}},
  \bibinfo {author} {\bibfnamefont {G.}~\bibnamefont {Fratesi}}, \bibinfo
  {author} {\bibfnamefont {G.}~\bibnamefont {Fugallo}}, \bibinfo {author}
  {\bibfnamefont {R.}~\bibnamefont {Gebauer}}, \bibinfo {author} {\bibfnamefont
  {U.}~\bibnamefont {Gerstmann}}, \bibinfo {author} {\bibfnamefont
  {F.}~\bibnamefont {Giustino}}, \bibinfo {author} {\bibfnamefont
  {T.}~\bibnamefont {Gorni}}, \bibinfo {author} {\bibfnamefont
  {J.}~\bibnamefont {Jia}}, \bibinfo {author} {\bibfnamefont {M.}~\bibnamefont
  {Kawamura}}, \bibinfo {author} {\bibfnamefont {H.-Y.}\ \bibnamefont {Ko}},
  \bibinfo {author} {\bibfnamefont {A.}~\bibnamefont {Kokalj}}, \bibinfo
  {author} {\bibfnamefont {E.}~\bibnamefont {Küçükbenli}}, \bibinfo {author}
  {\bibfnamefont {M.}~\bibnamefont {Lazzeri}}, \bibinfo {author} {\bibfnamefont
  {M.}~\bibnamefont {Marsili}}, \bibinfo {author} {\bibfnamefont
  {N.}~\bibnamefont {Marzari}}, \bibinfo {author} {\bibfnamefont
  {F.}~\bibnamefont {Mauri}}, \bibinfo {author} {\bibfnamefont {N.~L.}\
  \bibnamefont {Nguyen}}, \bibinfo {author} {\bibfnamefont {H.-V.}\
  \bibnamefont {Nguyen}}, \bibinfo {author} {\bibfnamefont {A.}~\bibnamefont
  {Otero-de-la Roza}}, \bibinfo {author} {\bibfnamefont {L.}~\bibnamefont
  {Paulatto}}, \bibinfo {author} {\bibfnamefont {S.}~\bibnamefont {Poncé}},
  \bibinfo {author} {\bibfnamefont {D.}~\bibnamefont {Rocca}}, \bibinfo
  {author} {\bibfnamefont {R.}~\bibnamefont {Sabatini}}, \bibinfo {author}
  {\bibfnamefont {B.}~\bibnamefont {Santra}}, \bibinfo {author} {\bibfnamefont
  {M.}~\bibnamefont {Schlipf}}, \bibinfo {author} {\bibfnamefont {A.~P.}\
  \bibnamefont {Seitsonen}}, \bibinfo {author} {\bibfnamefont {A.}~\bibnamefont
  {Smogunov}}, \bibinfo {author} {\bibfnamefont {I.}~\bibnamefont {Timrov}},
  \bibinfo {author} {\bibfnamefont {T.}~\bibnamefont {Thonhauser}}, \bibinfo
  {author} {\bibfnamefont {P.}~\bibnamefont {Umari}}, \bibinfo {author}
  {\bibfnamefont {N.}~\bibnamefont {Vast}}, \bibinfo {author} {\bibfnamefont
  {X.}~\bibnamefont {Wu}},\ and\ \bibinfo {author} {\bibfnamefont
  {S.}~\bibnamefont {Baroni}},\ }\bibfield  {title} {\bibinfo {title} {Advanced
  capabilities for materials modelling with {Quantum} {ESPRESSO}},\ }\href
  {https://doi.org/10.1088/1361-648X/aa8f79} {\bibfield  {journal} {\bibinfo
  {journal} {Journal of Physics: Condensed Matter}\ }\textbf {\bibinfo {volume}
  {29}},\ \bibinfo {pages} {465901} (\bibinfo {year} {2017})}\BibitemShut
  {NoStop}%
\bibitem [{\citenamefont {Prandini}\ \emph {et~al.}(2018)\citenamefont
  {Prandini}, \citenamefont {Marrazzo}, \citenamefont {Castelli}, \citenamefont
  {Mounet},\ and\ \citenamefont {Marzari}}]{prandini2018precision}%
  \BibitemOpen
  \bibfield  {author} {\bibinfo {author} {\bibfnamefont {G.}~\bibnamefont
  {Prandini}}, \bibinfo {author} {\bibfnamefont {A.}~\bibnamefont {Marrazzo}},
  \bibinfo {author} {\bibfnamefont {I.~E.}\ \bibnamefont {Castelli}}, \bibinfo
  {author} {\bibfnamefont {N.}~\bibnamefont {Mounet}},\ and\ \bibinfo {author}
  {\bibfnamefont {N.}~\bibnamefont {Marzari}},\ }\bibfield  {title} {\bibinfo
  {title} {Precision and efficiency in solid-state pseudopotential
  calculations},\ }\href {https://www.nature.com/articles/s41524-018-0127-2}
  {\bibfield  {journal} {\bibinfo  {journal} {npj Computational Materials}\
  }\textbf {\bibinfo {volume} {4}},\ \bibinfo {pages} {72} (\bibinfo {year}
  {2018})}\BibitemShut {NoStop}%
\bibitem [{\citenamefont {Th\'ebaud}\ \emph {et~al.}(2022)\citenamefont
  {Th\'ebaud}, \citenamefont {Berlijn},\ and\ \citenamefont
  {Lindsay}}]{PhysRevB.105.134202}%
  \BibitemOpen
  \bibfield  {author} {\bibinfo {author} {\bibfnamefont {S.}~\bibnamefont
  {Th\'ebaud}}, \bibinfo {author} {\bibfnamefont {T.}~\bibnamefont {Berlijn}},\
  and\ \bibinfo {author} {\bibfnamefont {L.}~\bibnamefont {Lindsay}},\
  }\bibfield  {title} {\bibinfo {title} {Perturbation theory and thermal
  transport in mass-disordered alloys: Insights from green's function
  methods},\ }\href {https://doi.org/10.1103/PhysRevB.105.134202} {\bibfield
  {journal} {\bibinfo  {journal} {Phys. Rev. B}\ }\textbf {\bibinfo {volume}
  {105}},\ \bibinfo {pages} {134202} (\bibinfo {year} {2022})}\BibitemShut
  {NoStop}%
\bibitem [{\citenamefont {Garg}\ \emph {et~al.}(2011)\citenamefont {Garg},
  \citenamefont {Bonini}, \citenamefont {Kozinsky},\ and\ \citenamefont
  {Marzari}}]{PhysRevLett.106.045901}%
  \BibitemOpen
  \bibfield  {author} {\bibinfo {author} {\bibfnamefont {J.}~\bibnamefont
  {Garg}}, \bibinfo {author} {\bibfnamefont {N.}~\bibnamefont {Bonini}},
  \bibinfo {author} {\bibfnamefont {B.}~\bibnamefont {Kozinsky}},\ and\
  \bibinfo {author} {\bibfnamefont {N.}~\bibnamefont {Marzari}},\ }\bibfield
  {title} {\bibinfo {title} {{Role of Disorder and Anharmonicity in the Thermal
  Conductivity of Silicon-Germanium Alloys: A First-Principles Study}},\ }\href
  {https://doi.org/10.1103/PhysRevLett.106.045901} {\bibfield  {journal}
  {\bibinfo  {journal} {Phys. Rev. Lett.}\ }\textbf {\bibinfo {volume} {106}},\
  \bibinfo {pages} {045901} (\bibinfo {year} {2011})}\BibitemShut {NoStop}%
\bibitem [{\citenamefont {Togo}(2023)}]{togo_first-principles_2023}%
  \BibitemOpen
  \bibfield  {author} {\bibinfo {author} {\bibfnamefont {A.}~\bibnamefont
  {Togo}},\ }\bibfield  {title} {\bibinfo {title} {First-principles {Phonon}
  {Calculations} with {Phonopy} and {Phono3py}},\ }\href
  {https://doi.org/10.7566/JPSJ.92.012001} {\bibfield  {journal} {\bibinfo
  {journal} {Journal of the Physical Society of Japan}\ }\textbf {\bibinfo
  {volume} {92}},\ \bibinfo {pages} {012001} (\bibinfo {year}
  {2023})}\BibitemShut {NoStop}%
\bibitem [{\citenamefont {Monacelli}\ \emph {et~al.}(2021)\citenamefont
  {Monacelli}, \citenamefont {Bianco}, \citenamefont {Cherubini}, \citenamefont
  {Calandra}, \citenamefont {Errea},\ and\ \citenamefont
  {Mauri}}]{monacelli_stochastic_2021}%
  \BibitemOpen
  \bibfield  {author} {\bibinfo {author} {\bibfnamefont {L.}~\bibnamefont
  {Monacelli}}, \bibinfo {author} {\bibfnamefont {R.}~\bibnamefont {Bianco}},
  \bibinfo {author} {\bibfnamefont {M.}~\bibnamefont {Cherubini}}, \bibinfo
  {author} {\bibfnamefont {M.}~\bibnamefont {Calandra}}, \bibinfo {author}
  {\bibfnamefont {I.}~\bibnamefont {Errea}},\ and\ \bibinfo {author}
  {\bibfnamefont {F.}~\bibnamefont {Mauri}},\ }\bibfield  {title} {\bibinfo
  {title} {The stochastic self-consistent harmonic approximation: calculating
  vibrational properties of materials with full quantum and anharmonic
  effects},\ }\href {https://doi.org/10.1088/1361-648X/ac066b} {\bibfield
  {journal} {\bibinfo  {journal} {Journal of Physics: Condensed Matter}\
  }\textbf {\bibinfo {volume} {33}},\ \bibinfo {pages} {363001} (\bibinfo
  {year} {2021})}\BibitemShut {NoStop}%
\bibitem [{\citenamefont {Tadano}\ and\ \citenamefont
  {Saidi}(2022)}]{tadano_first-principles_2022}%
  \BibitemOpen
  \bibfield  {author} {\bibinfo {author} {\bibfnamefont {T.}~\bibnamefont
  {Tadano}}\ and\ \bibinfo {author} {\bibfnamefont {W.~A.}\ \bibnamefont
  {Saidi}},\ }\bibfield  {title} {\bibinfo {title} {First-{Principles} {Phonon}
  {Quasiparticle} {Theory} {Applied} to a {Strongly} {Anharmonic} {Halide}
  {Perovskite}},\ }\href {https://doi.org/10.1103/PhysRevLett.129.185901}
  {\bibfield  {journal} {\bibinfo  {journal} {Physical Review Letters}\
  }\textbf {\bibinfo {volume} {129}},\ \bibinfo {pages} {185901} (\bibinfo
  {year} {2022})}\BibitemShut {NoStop}%
\bibitem [{\citenamefont {Jain}(2020)}]{jain_multichannel_2020}%
  \BibitemOpen
  \bibfield  {author} {\bibinfo {author} {\bibfnamefont {A.}~\bibnamefont
  {Jain}},\ }\bibfield  {title} {\bibinfo {title} {Multichannel thermal
  transport in crystalline {Tl} 3 {VSe} 4},\ }\href
  {https://doi.org/10.1103/PhysRevB.102.201201} {\bibfield  {journal} {\bibinfo
   {journal} {Physical Review B}\ }\textbf {\bibinfo {volume} {102}},\ \bibinfo
  {pages} {201201} (\bibinfo {year} {2020})}\BibitemShut {NoStop}%
\bibitem [{\citenamefont {Feng}\ \emph {et~al.}(2017)\citenamefont {Feng},
  \citenamefont {Lindsay},\ and\ \citenamefont {Ruan}}]{feng_four-phonon_2017}%
  \BibitemOpen
  \bibfield  {author} {\bibinfo {author} {\bibfnamefont {T.}~\bibnamefont
  {Feng}}, \bibinfo {author} {\bibfnamefont {L.}~\bibnamefont {Lindsay}},\ and\
  \bibinfo {author} {\bibfnamefont {X.}~\bibnamefont {Ruan}},\ }\bibfield
  {title} {\bibinfo {title} {Four-phonon scattering significantly reduces
  intrinsic thermal conductivity of solids},\ }\href
  {https://doi.org/10.1103/PhysRevB.96.161201} {\bibfield  {journal} {\bibinfo
  {journal} {Physical Review B}\ }\textbf {\bibinfo {volume} {96}},\ \bibinfo
  {pages} {161201} (\bibinfo {year} {2017})}\BibitemShut {NoStop}%
\bibitem [{\citenamefont {Schirmacher}(2006)}]{schirmacher2006thermal}%
  \BibitemOpen
  \bibfield  {author} {\bibinfo {author} {\bibfnamefont {W.}~\bibnamefont
  {Schirmacher}},\ }\bibfield  {title} {\bibinfo {title} {Thermal conductivity
  of glassy materials and the boson peak},\ }\href
  {https://doi.org/10.1209/epl/i2005-10471-9} {\bibfield  {journal} {\bibinfo
  {journal} {EPL (Europhysics Letters)}\ }\textbf {\bibinfo {volume} {73}},\
  \bibinfo {pages} {892} (\bibinfo {year} {2006})}\BibitemShut {NoStop}%
\bibitem [{\citenamefont {Lubchenko}\ and\ \citenamefont
  {Wolynes}(2003)}]{lubchenko2003origin}%
  \BibitemOpen
  \bibfield  {author} {\bibinfo {author} {\bibfnamefont {V.}~\bibnamefont
  {Lubchenko}}\ and\ \bibinfo {author} {\bibfnamefont {P.~G.}\ \bibnamefont
  {Wolynes}},\ }\bibfield  {title} {\bibinfo {title} {The origin of the boson
  peak and thermal conductivity plateau in low-temperature glasses},\ }\href
  {https://doi.org/10.1073/pnas.252786999} {\bibfield  {journal} {\bibinfo
  {journal} {Proc. Natl. Acad. Sci. U.S.A.}\ }\textbf {\bibinfo {volume}
  {100}},\ \bibinfo {pages} {1515} (\bibinfo {year} {2003})}\BibitemShut
  {NoStop}%
\bibitem [{\citenamefont {Wang}\ \emph {et~al.}(2019)\citenamefont {Wang},
  \citenamefont {Ninarello}, \citenamefont {Guan}, \citenamefont {Berthier},
  \citenamefont {Szamel},\ and\ \citenamefont
  {Flenner}}]{wang_low-frequency_2019}%
  \BibitemOpen
  \bibfield  {author} {\bibinfo {author} {\bibfnamefont {L.}~\bibnamefont
  {Wang}}, \bibinfo {author} {\bibfnamefont {A.}~\bibnamefont {Ninarello}},
  \bibinfo {author} {\bibfnamefont {P.}~\bibnamefont {Guan}}, \bibinfo {author}
  {\bibfnamefont {L.}~\bibnamefont {Berthier}}, \bibinfo {author}
  {\bibfnamefont {G.}~\bibnamefont {Szamel}},\ and\ \bibinfo {author}
  {\bibfnamefont {E.}~\bibnamefont {Flenner}},\ }\bibfield  {title} {\bibinfo
  {title} {Low-frequency vibrational modes of stable glasses},\ }\href
  {https://doi.org/10.1038/s41467-018-07978-1} {\bibfield  {journal} {\bibinfo
  {journal} {Nature Communications}\ }\textbf {\bibinfo {volume} {10}},\
  \bibinfo {pages} {26} (\bibinfo {year} {2019})}\BibitemShut {NoStop}%
\end{thebibliography}%



\begin{center}
\textbf{\large Appendix}
\end{center}

\setcounter{equation}{0}
\setcounter{figure}{0}
\setcounter{table}{0}
\setcounter{page}{1}
\makeatletter
\renewcommand{\theequation}{A\arabic{equation}}
\renewcommand{\thefigure}{A\arabic{figure}}

\appendix

\section{Structure generation and vibrational properties} % (fold)
\label{sec:computational_details}
The structures were generated using the melt quench procedure described in Ref.~\cite{fharper_modelling_2023}, using VASP v5.4 \cite{kresse1993ab}, for consistency with those generated in Ref.~\cite{fharper_modelling_2023}.
In order to apply the computational protocol of Ref.~\cite{simoncelli_thermal_2022} to study the thermal properties, the vibrational frequencies have to be interpolated in Fourier space. 
Applying Fourier interpolation to the vibrational frequencies of disordered atomistic models containing less than 200 atoms yields more accurate results when a mesh denser than the point $\bm{q}=\bm{0}$ only is used as starting point. 
Therefore, in order to compute the vibrational properties of am-Al$_2$O$_3$ on a mesh denser than $\bm{q}=\bm{0}$ only and to use the most accurate density-functional perturbation theory (DFPT) technique \cite{baroni_phonons_2001}, the vibrational properties are computed using Quantum ESPRESSO \cite{giannozzi_quantum_2009,giannozzi_advanced_2017} on a $2\times2\times2$ $\bm{q}$-mesh (we recall that at present the DFPT implementation in VASP is restricted to calculations at $\bm{q}{=}\bm{0}$). Quantum ESPRESSO calculations were carried out using PBE functional, with pseudopotentials from the standard solid-state pseudopotential libraries (SSSP) precision library \cite{prandini2018precision}.
To compute the vibrational properties, the cell and atomic positions of all the am-Al$_2$O$_3$ models were relaxed using a threshold for forces of 2$\times$10$^{-4}$\,Ry/Bohr and of $0.01$ kBar for pressure (\texttt{vc-relax} command).


\section{Coordination environment} % (fold)
\label{sec:coord_rdf_diff}
{To determine the coordination topology we calculated how many atoms are in the sphere of radius 2.2 \AA \space for the lowest-density structure and in the sphere of radius 2.35 \AA \space for higher-density structures. This was done to ensure the cutoff is in the region of the first valley of RDF in Fig. \ref{fig:al2o3_rdf}.}


\section{Thermal conductivity calculations}
\subsection{Convergence of the Allen-Feldman theory} % (fold)
\label{sec:convergence_of_the_allen}
\begin{figure*}
  \centering
\includegraphics[width=\textwidth]{\folder/amorphous_alumina_convergence_plateau_2x3_7x7x7.pdf}\\[-3mm]
  \caption{\textbf{Convergence of the AF conductivity for am-Al$_2$O$_3$ with respect to the broadening $\eta$ for the Dirac $\delta$,} The three upper panels show convergence plateaus for $5\times5\times5$ (lines) and $7\times7\times7$ (scatter points) meshes of models with densities 2.28, 2.98 and 3.49 $g / cm^3$. Black (red) correspond to evaluating the AF conductivity using the Gaussian (Lorentzian) representation of the Dirac delta function. 
  The three lower panels contain calculations at $\bm{q}=\bm{0}$ only for models with densities 2.28, 2.98 and 3.49 $g / cm^3$.  Black and red denote Gaussian and Lorentzian, as in the panels above. The broadening $\eta$ is chosen as the value approximately determining the beginning of the convergence plateau \cite{simoncelli_thermal_2022}, and it is set to 6.0 $cm^{-1}$ for models with density up to 3.30 $g / cm^3$ and 8.34 for the model with density $3.49 g/cm^3$. We see that the $5\times5\times5$ $\bm{q}$ mesh is dense enough to achieve computational convergence with respect to mesh size, since results obtained on a $7\times7\times7$ mesh are practically indistinguishable from those obtained using a $5\times5\times5$ mesh. Calculations at $\bm{q}=\bm{0}$  only are far from computational convergence and consequently underestimate the conductivity. 
}
  \label{fig:harm_theory_plateau}
\end{figure*}
In order to calculate the bulk limit of the thermal conductivity of a strongly disordered glass such as am-Al$_2$O$_3$, we rely on the convergence-acceleration protocol discussed in Ref.~\cite{simoncelli_thermal_2022} both for the AF and rWTE conductivities. The capability of such a protocol to accurately extrapolate the bulk limit of the thermal conductivity of strongly disordered glasses from finite-size models containing hundreds of atoms was thoroughly validated in Ref.~\cite{simoncelli_thermal_2022}.
{The protocol requires determining the broadening parameters $\eta$ for the Voigt profile appearing in Eq.~(\ref{eq:thermal_conductivity_combined}) as a value determining the beginning of the convergence plateau shown in Fig.~\ref{fig:harm_theory_plateau} (see Sec.~\ref{ssec:Wigner_formulation} for details). 

All the am-Al$_2$O$_3$ models analyzed display a clear and broad convergence plateau for the AF conductivity.
The three upper panels in Fig. \ref{fig:harm_theory_plateau} show results obtained employing a $5\times5\times5$ or $7\times7\times7$ $\bm{q}$-mesh; the good agreement between these two calculations indicates that computational convergence has been achieved.
 
 The  three bottom panels of Fig.~\ref{fig:harm_theory_plateau} show that that a calculation performed at $\bm{q}=0$ only for a 120-atom model  is far from computational convergence and underestimates the thermal conductivity.
The values of $\eta$ that we determined from the convergence test discussed in Fig. \ref{fig:harm_theory_plateau} and that  we employed in our calculations are reported in Table \ref{tab:broadening}.

\begin{table}[h]
    \centering
    \begin{tabular}{|c|c|c|c|c|c|}
    \hline
    $\rho$ ($g/cm^3$) & 2.28 & 2.98 & 3.17 & 3.30 & 3.49 \\
    \hline
    $\hbar\eta$ ($cm^{-1}$) & 6.0 & 6.0 & 6.0 & 6.0 & 8.34 \\
    \hline
    \end{tabular}
    \caption{Broadening parameters $\eta$ used for the Gaussian representation of the Dirac $\delta$  function appearing in the AF conductivity expression, and for the Voigt distribution appearing in the rWTE expression.}
    \label{tab:broadening}
\end{table}


 


\subsection{Anharmonic Linewidths} % (fold)
\label{sec:linewidths}

\begin{figure*}
  \centering
\includegraphics[width=\textwidth]{\folder/amorphous_alumina_linewidth_comparison.pdf}\\[-3mm]
  \caption{\textbf{Effect of temperature on anharmonic linewidths of am-Al$_2$O$_3$ at various densities and temperatures.} The scatter points represent the linewidths computed at $\bm{q}=\bm{0}$, and the solid lines are coarse grained functions $\Gamma_a[\omega, T]$ used to approximately describe the anharmonic linewidths as a single-valued functions of frequency, thus to estimate the effects of anharmonicity at a reduced computational cost\cite{PhysRevB.105.134202,PhysRevLett.106.045901,simoncelli2021Wigner}. The purple region denotes the overdamped regime $\Gamma > \omega$ \cite{simoncelli2021Wigner,caldarelli_many-body_2022}. The gray dashed lines show the average spacing between vibrational energy levels. We note that the linewidths of am-Al$_2$O$_3$ are similar to those found other oxide glasses, e.g., vitreous silica \cite{simoncelli_thermal_2022}. 
  %The approximated single-valued function $\Gamma(\omega)$ for densities 2.28 and 2.98 $g/cm^3$ were obtained using method A (See subsection \ref{subsec:method_a}) and for density 3.49 $g/cm^3$ using method B. (See subsection \ref{subsec:method_b}).
  }
  \label{fig:temp_v_linewidth}
\end{figure*}
The linewidths were computed using \texttt{phono3py} \cite{phono3py,togo_first-principles_2023} and the standard perturbative treatment of anharmonicity, \textit{i.e.} vibrational frequencies were considered to be independent from temperature (\textit{i.e.} it neglects thermal expansion and the renormalization of frequencies due to anharmonicity \cite{monacelli_stochastic_2021,tadano_first-principles_2022,jain_multichannel_2020,feng_four-phonon_2017,PhysRevMaterials.3.085401}), and the linewidths were computed considering exclusively the cubic terms in the Taylor expansion of the interatomic potential \cite{simoncelli2021Wigner,simoncelli_thermal_2022} and the contribution due to isotopic-mass disorder \cite{tamura_isotope_1983}. 
%We employed the approximations detailed in Ref.~\cite{simoncelli_thermal_2022} to account for anharmonicity at a reduced computational cost. 

Fig. \ref{fig:temp_v_linewidth} show the linewidths calculated at $\bm{q}=\bm{0}$ at various densities and temperatures. {We see that at 50K a significant proportion of vibrational modes have linewidths smaller than the average energy-level spacing. As mentioned in section \ref{sec:thermal_properties}, these linewidths are regularised by the Voigt distribution, which ensures that heat transfer between neighboring vibrational eigenstates can always occur, implying that the effects of anharmonicity are accounted for only when they are not affected by finite-size effects \cite{simoncelli_thermal_2022}.

The solid lines in Fig. \ref{fig:temp_v_linewidth} 
are the  functions $\Gamma_a[\omega]$ that approximately describe the anharmonic linewidths as a single-valued functions of frequency, obtained 
following the approaches discussed in Ref.~\cite{simoncelli2021Wigner} (see also Refs.~\cite{PhysRevB.105.134202,PhysRevLett.106.045901,Fiorentino_2023} for similar approximated treatments of anharmonicity).
The approximated function $\Gamma_a[\omega]$ is employed to compute the anharmonic linewidths as a function of frequency when the Fourier interpolation is used to extrapolate the bulk limit of the thermal conductivity~\ref{eq:thermal_conductivity_combined}, following the protocol discussed in Ref.~\cite{simoncelli2021Wigner}.

\subsection{rWTE conductivity calculation}
The quantities needed to evaluate the rWTE conductivity~(\ref{eq:thermal_conductivity_combined}) were computed as follows:
(i) the $\eta$ parameter was computed as discussed in Sec.~\ref{sec:convergence_of_the_allen}; 
(ii) the software \texttt{phono3py} \cite{phono3py,togo_first-principles_2023} was used to evaluate frequencies and velocity operators on a $5\times5\times5$ $\bm{q}$ mesh;
(iii) the linewidths were evaluated from the frequencies determined at the previous point using the function $\Gamma_a(\omega)$ discussed in Sec.~\ref{sec:linewidths}.

We checked that increasing the size of the $\bm{q}$ mesh from $5\times5\times5$ to $7\times7\times7$ produced practically indistinguishable results in the RMDS, VDOS, and thermal conductivities. Therefore, all the results discussed in the main text are evaluated on a $5\times5\times5$ $\bm{q}$ mesh.


We limited our calculations to temperatures higher than 50 K, since at temperatures lower than 50 K the thermal properties of am-Al$_2$O$_3$ are dominated by low-frequency vibrational modes that are likely to feature glassy anomalies \cite{schirmacher2006thermal,lubchenko2003origin,wang_low-frequency_2019}; accurately sampling these low-frequencies vibrational modes requires using atomistic models containing thousands of atoms, and it is therefore beyond the scope of the present work.

{$D(\omega)$ data for the plot in Fig. \ref{fig:diffusivity} was calculated using Eq.~(\ref{eq:diff_omega}) with $\delta$-function smeared to a Gaussian with variance $\frac{\pi}{2} \eta_0^2$, with $\hbar\eta_0 = 8.34$ $cm^{-1}$.}



\end{document}
