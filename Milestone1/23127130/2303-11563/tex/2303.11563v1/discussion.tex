%Any discussion points?
This paper proposes \ourmethod, a novel approach for learning heterogeneous dynamic embeddings of patients, doctors, rooms, and medications from diverse hospital operation data streams.
These embeddings capture similarities among entities 
% doctors, rooms, patients, and medications 
based on static attributes and dynamic interactions. 
Consequently, these embeddings can serve as input to a variety of prediction tasks to improve clinical decision-making and patient care.
Our results show that on a variety of prediction tasks \ourmethod substantially outperforms baselines and produces embeddings meaningful to clinical experts. 

As we see it, our work has limitations.
% In the interaction data,
% When we performed this analysis, 
All our patient-physician interactions are associated with procedures. 
While such procedure-related interactions are essential, we need to consider a more extensive, richer set of patient-physician interactions.
Such interactions can be extracted from the clinical notes from hospitals.
Each clinical note provides much context for each interaction, and one possible way to label the interaction with this context is to compute embeddings of clinical notes and attach these as weights or features to the interactions.
In the longer term, we are interested in deploying \ourmethod on top of the existing electronic medical record system at the hospital to provide clinical support. To reach this goal, we need a way to make the embeddings learned by \ourmethod interpretable to healthcare professionals. Specifically, we need to identify and explain the factors that cause two healthcare entities to be close to (or far from) each other. We propose to do so by introducing prototypes~\cite{dovsilovic2018explainable} and adding an explainability module for feature ablation.
\begin{comment}
As we see it, our work has two main limitations. First, due to limitations of our dataset when we performed this analysis, all our patient-physician interactions are associated with procedures. While such procedure-related interactions are quite important, clearly, there is a larger, richer set patient-physician interactions that we need to take into account. We have recently obtained hospital clinical notes data from which we have begun to extract many additional interactions. Each clinical note provides a lot of context for an interaction and on possible way to label the interaction with this context is to compute embeddings of clinical notes and attach these as weights or features to the interactions.
A second limitation of \ourmethod is the interpretability of the computed embeddings.
In the longer term, we are interested in deploying \ourmethod on top of the existing electronic medical record system at the hospital as a way of providing clinical support. To reach this goal, we need a way to make the embeddings learned by \ourmethod interpretable to healthcare professionals. Specifically, we need to be able to identify and explain the factors that cause two healthcare entities to be close to (or far from) each other. We propose to do so by introducing prototypes~\cite{dovsilovic2018explainable} and adding explainability module for feature ablation.
\end{comment}

%\textbf{Any thoughts on how to deal with this? --Sriram}
%\end{itemize}
%\ba{We could address this in the following two ways. (i) Introduce protoype-based embeddings. Prototypes epitomize certain behaviors which collectively cover the entire dataset. For example each member of a set of doctors chosen as prototypes will have different specialty. Now, the distance of a particular doctor's embedding to the embeddings of the prototype doctors will help us interpret the doctors embedding. (ii) Do feature ablation/importance analysis. Remove one dimension at a time and see how results on different tasks  change. This will help us get a sense of which latent dimension is associated with what task.}


Besides the future work mentioned above, 
% which is implied by the limitations of our work, 
we see a direction to expand our work. 
We can apply \ourmethod to other prediction tasks in the healthcare setting. 
For example, predicting the risk of readmission (see \cite{min2019predictive})
prior to discharge and predicting the 
length of stay in the hospital early during a patient visit (see \cite{TURGEMAN2017376}) are both tasks that can enable additional clinical resources for high-risk patients.

\begin{comment}
Besides the future work mentioned above that are implied by the limitations of our work, we see two directions in which to expand our work. First, we will apply \ourmethod to other prediction tasks. For example, predicting the risk of readmission (see for e.g.,
% \cite{MinYuWang}) 
\cite{min2019predictive})
prior to discharge and predicting the 
length of stay in the hospital early during a patient visit (see for e.g., \cite{TURGEMAN2017376}) are both tasks that can enable additional clinical resources for high risk patients.
Second, we aim to perform comparison with predictions that use carefully hand-crafted feature sets specific to the task at hand. Of course, these feature sets can be added to \ourmethod (as static features), in which case we will be evaluating how much gain there is in layering \ourmethod on top of these other methods.
\end{comment}

\begin{comment}
Patients receive various types of care during their visit, such as medication prescriptions and procedures that are targeted to treat the diagnosed disease of the patient.
As this medical history of the patient is related to the patient's susceptibility to a specific disease, this information needs to be well preserved as the record of the patient.
Learning dynamic patient embedding is crucial as it represents the patient based on the care the patient received in the hospital.
The novelty of this paper is that the proposed method, HIDEP captures not only the prescriptions and procedures that patients received but also the doctors that interacted with the patient as well as the rooms that patients visit in the hospital.
What is more, HIDEP also learns dynamic embedding of doctors, medications, and rooms.

% CDI classification. implications
Our results show that HIDEP performs the best in CDI prediction task.
CDI is a common infectious disease that one can get exposed to C.diff by exposure to other infectious individuals or contaminated areas in hospitals.
Not everyone with C.diff gets infected; in fact, those that are immunocompromised are likely to get infected.
HIDEP\textsubscript{P-Doctor,P-Room} successfully captures patient exposure to rooms that may have been contaminated by C.diff or exposure to doctors who may as well be infectious.
JODIE\textsubscript{P-Room} also performed well in this task, which confirms the nature of CDI transmission, and that dynamic room embeddings itself were able to capture the rooms that were potentially contaminated by infectious patients or healthcare personnel.

% Severity. patient-room, patient-doctor. 
Our results in predicting the severity risk of the next visit also show that patient interaction with rooms and doctors are crucial in the prediction task.
An explanation could be that patients who received procedures by physicians to treat severe disease in the previous visit are likely to be placed in recovery rooms.
Patient embedding can capture these series of events, and the severity of the patients' future visits can be predicted.
On the other hand, mortality risk prediction results show that medication prescriptions and interaction with doctors, including procedures in the interaction is crucial in this task.

There are some directions to improve the performance of HIDEP.
One direction is to modify the input to the model.
For the patient-medication interaction, we may substitute the medication id to a higher level medication category in the hierarchy to allow a smaller set of medications to be updated frequently.
For patient-doctor interaction, the interaction feature is a one-hot vector indicating the procedure given in this interaction; due to the high dimension and the sparsity, we may apply dimension reduction algorithm such as PCA to the feature vector.
We may also utilize room proximity information to force the rooms in the nearby area in hospitals, e.g., rooms in the same unit, to be trained similarly.

Another direction is to modify the architecture of HIDEP.
For learning the patient embeddings using different types of interactions, every interaction type does not give the same importance when updating the patient embedding when a new interaction is made.
Therefore, we might incorporate the attention mechanism that learns the importance of inputs so that more important interaction affects more on the updated patient embeddings.
This is also beneficial for interpreting the models. The learned attention can give information on which types of interactions are more important than others and what behaviors of patients affect more on the patient embeddings.
\end{comment}
