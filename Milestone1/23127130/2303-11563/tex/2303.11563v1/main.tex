\documentclass[conference]{IEEEtran}
\IEEEoverridecommandlockouts
% The preceding line is only needed to identify funding in the first footnote. If that is unneeded, please comment it out.
\usepackage{cite}
\usepackage{amsmath,amssymb,amsfonts}
\usepackage{algorithmic}
\usepackage{graphicx}
\usepackage{textcomp}
\usepackage{xcolor}

% ================================================
% -------------------
% Packages
\usepackage{hyperref}
\usepackage{multirow}
\usepackage{makecell}
\usepackage{graphicx}
\usepackage{comment}
\usepackage{xspace}
\usepackage{xcolor}
\usepackage{array}
\usepackage{caption}
\usepackage{subcaption}
\usepackage{wrapfig}
\usepackage{mathtools}
\usepackage{amsfonts}
\usepackage{caption,tabularx,booktabs}
\usepackage[final]{pdfpages}
% -------------------
% newcommand
\newcommand{\ourmethodFull}{\textsc{\textbf{D}ynamic \textbf{E}mbedding of Health\textbf{C}are \textbf{Ent}ities}\xspace}

\newcommand{\ourmethod}{\textsc{DECEnt}\xspace}
\newcommand{\ourmethodp}{\textsc{DECEnt+}\xspace}
\newcommand{\domain}{\textsc{Domain}\xspace}
\newcommand{\nodevec}{\textsc{Node2Vec}\xspace}
\newcommand{\deepwalk}{\textsc{DeepWalk}\xspace}
\newcommand{\rnn}{\textsc{RNN}\xspace}
\newcommand{\lstm}{\textsc{LSTM}\xspace}
\newcommand{\dynamictriad}{\textsc{CTDNE}\xspace}
\newcommand{\jodie}{\textsc{JODIE}\xspace}
\newcommand{\jodiep}{\textsc{JODIE++}\xspace}

\newcommand{\entities}{\mathcal{\tau}}
\newcommand{\doc}{\mathcal{D}}
\newcommand{\pat}{\mathcal{P}}
\newcommand{\room}{\mathcal{R}}
\newcommand{\med}{\mathcal{M}}

\newcommand{\Int}{\mathcal{S}}
\newcommand{\Med}{\mathcal{MD}}
\newcommand{\Pro}{\mathcal{PR}}
\newcommand{\Tra}{\mathcal{TR}}

\newcommand{\ps}{\mathbf{p}}
\newcommand{\pd}{\hat{\mathbf{p}}}
\newcommand{\ed}{\hat{\mathbf{e}}}
\newcommand{\pred}{\tilde{\mathbf{e}}}
\newcommand{\es}{\bar{\mathbf{e}}}

\newcommand{\W}{\mathbf{W}}

\newcommand{\prodmod}{\textsc{PM}\xspace}
\newcommand{\medmod}{\textsc{MM}\xspace}
\newcommand{\transmod}{\textsc{TM}\xspace}
\newcommand{\predmod}{\textsc{PRED}\xspace}

% \newcounter{theorem} % NOTE: Added by Hankyu for KDD applied track submission
\newtheorem{problem}{Problem}

\newcommand\ba[1]{\textcolor{red}{Bijaya says: #1}}
\newcommand\srp[1]{\textcolor{blue}{Sriram says: #1}}
\newcommand\hankyu[1]{\textcolor{purple}{#1}}
% =======================================

\def\BibTeX{{\rm B\kern-.05em{\sc i\kern-.025em b}\kern-.08em
    T\kern-.1667em\lower.7ex\hbox{E}\kern-.125emX}}


\begin{document}



\title{Dynamic Healthcare Embeddings for Improving Patient Care
%{\footnotesize \textsuperscript{*}For the CDC MInD Healthcare Network}
%\thanks{This work was funded by the CDC MInD Healthcare Network grant U01CK000594 and NSF grant 1955939.Support for this research is provided by the Centers for Disease Control and Prevention MInD Healthcare Network (Awards U01CK000531 and U01CK000594 with respective Covid-19 supplements)}
}

% \begin{comment}
\author{\IEEEauthorblockN{Hankyu Jang}
\IEEEauthorblockA{\textit{Department of Computer Science} \\
\textit{University of Iowa}\\
Iowa City, Iowa, USA \\
hankyu-jang@uiowa.edu}
\and
\IEEEauthorblockN{Sulyun Lee}
\IEEEauthorblockA{\textit{Interdisciplinary Graduate Program in Informatics} \\
\textit{University of Iowa}\\
Iowa City, Iowa, USA \\
sulyun-lee@uiowa.edu}
\and
\IEEEauthorblockN{D.~M.~Hasibul Hasan}
\IEEEauthorblockA{\textit{Department of Computer Science} \\
\textit{University of Iowa}\\
Iowa City, Iowa, USA \\
dmhasibul-hasan@uiowa.edu}
\and
\IEEEauthorblockN{Philip M.~Polgreen}
\IEEEauthorblockA{\textit{Department of Internal Medicine} \\
\textit{University of Iowa}\\
Iowa City, Iowa, USA \\
philip-polgreen@uiowa.edu}
\and
\IEEEauthorblockN{Sriram V.~Pemmaraju}
\IEEEauthorblockA{\textit{Department of Computer Science} \\
\textit{University of Iowa}\\
Iowa City, Iowa, USA \\
sriram-pemmaraju@uiowa.edu\\
*For the CDC MInD Healthcare Network
}
\and
\IEEEauthorblockN{Bijaya Adhikari}
\IEEEauthorblockA{\textit{Department of Computer Science} \\
\textit{University of Iowa}\\
Iowa City, Iowa, USA \\
bijaya-adhikari@uiowa.edu}
% \and
% \IEEEauthorblockN{*For the CDC MInD Healthcare Network}\\
% \IEEEauthorblockA{}
}
% \end{comment}

% \author{
% \IEEEauthorblockN{Hankyu Jang\authorrefmark{1}, ...}\\
% \IEEEauthorblockA{\authorrefmark{1}Department of Computer Science\\ University of Iowa, Iowa City, Iowa\\Email: hankyu-jang@uiowa.edu}
% }

% another format
\begin{comment}
\author{
\IEEEauthorblockN{Michael Shell\authorrefmark{1}
, Homer Simpson\authorrefmark{2}, James Kirk\authorrefmark{3}, Montgomery Scott\authorrefmark{3} and Eldon Tyrell\authorrefmark{4}}\\
\IEEEauthorblockA{\authorrefmark{1}School of Electrical and Computer Engineering\\
Georgia Institute of Technology, Atlanta, Georgia 30332--0250\\
Email: mshell@ece.gatech.edu}
\IEEEauthorblockA{\authorrefmark{2}Twentieth Century Fox , Springfield, USA\\
Email: homer@thesimpsons.com}
\IEEEauthorblockA{\authorrefmark{3}Starfleet Academy, San Francisco, California 96678-2391\\
Telephone: (800) 555--1212, Fax: (888) 555--1212}
\IEEEauthorblockA{\authorrefmark{4}Tyrell Inc., 123 Replicant Street, Los Angeles, California 90210--4321}}
\end{comment}

\maketitle



\begin{abstract}


Over the past few years, there has been a significant amount of research focused on studying the ReLU activation function, with the aim of achieving neural network convergence through over-parametrization. However, recent developments in the field of Large Language Models (LLMs) have sparked interest in the use of exponential activation functions, specifically in the attention mechanism.

Mathematically, we define the neural function $F: \R^{d \times m} \times  \mathbb{R}^d \rightarrow \mathbb{R}$ using an exponential activation function. Given a set of data points with labels $\{(x_1, y_1), (x_2, y_2), \dots, (x_n, y_n)\} \subset \mathbb{R}^d \times \mathbb{R}$ where $n$ denotes the number of the data. Here $F(W(t),x)$ can be expressed as $F(W(t),x) := \sum_{r=1}^m a_r \exp(\langle w_r, x \rangle)$, where $m$ represents the number of neurons, and $w_r(t)$ are weights at time $t$. It's standard in literature that $a_r$ are the fixed weights and it's never changed during the training. We initialize the weights $W(0) \in \mathbb{R}^{d \times m}$ with random Gaussian distributions, such that $w_r(0) \sim \mathcal{N}(0, I_d)$ and initialize $a_r$ from random sign distribution for each $r \in [m]$.

Using the gradient descent algorithm, we can find a weight $W(T)$ such that $\| F(W(T), X) - y \|_2 \leq \epsilon$ holds with probability $1-\delta$, where $\epsilon \in (0,0.1)$ and $m = \Omega(n^{2+o(1)}\log(n/\delta))$. To optimize the over-parametrization bound $m$, we employ several tight analysis techniques from previous studies [Song and Yang arXiv 2019, Munteanu, Omlor, Song and Woodruff ICML 2022]. 

 

\end{abstract}

\begin{IEEEkeywords}
dynamic embedding, heterogeneous networks, patient embedding, healthcare analytics
\end{IEEEkeywords}

\section{Introduction}
\label{section:introduction}
\section{Introduction}
\label{sec:introduction}
% \begin{itemize}
%     % Diffusion of FL
%     \item {\st{Diffusion of FL}}
%     % Security threats to FL
%     \item {\st{Security threats to FL with particular focus on model poisoning}}
%     % Limitations of existing countermeasures
%     \item {\st{Current countermeasures (e.g., KRUM) and their limitations}}
%     % Proposed method and its advantages
%     \item {\st{Intuitive description of the proposed method and its difference (i.e., advantages) w.r.t. state of the art}}
%     % Main contributions
%     \item {\st{Summary of the main contributions of this work}}
%     % Paper's structure and organization
%     \item {\st{Paper's structure and organization}}
% \end{itemize}

% Diffusion of FL
Recently, {\em federated learning} (FL) has emerged as the leading paradigm for training distributed, large-scale, and privacy-preserving machine learning (ML) systems~\cite{mcmahan2017googleai,mcmahan2017aistats}. 
The core idea of FL is to allow multiple edge clients to collaboratively train a shared, global model without disclosing their local private training data.
%Specifically, an FL system consists of a central server and many edge clients; 
A typical FL round involves the following steps: {\em(i)} the server randomly picks some clients and sends them the current, global model; {\em(ii)} each selected client locally trains its model with its own private data; then, it sends the resulting local model to the server;\footnote{Whenever we refer to global/local model, we mean global/local model {\em parameters}.} {\em(iii)} the server updates the global model by computing an \emph{aggregation function}, usually the average (FedAvg), on the local models received from clients.
% \begin{enumerate}
%     \item[{\em(i)}] the server sends the current, global model to the clients and appoints some of them for training;
%     \item[{\em(ii)}] each selected client locally trains its copy of the global model with its own private data; then, it sends the resulting local model back to the server;\footnote{Whenever we refer to global/local model, we mean global/local model {\em parameters}.}
%     \item[{\em(iii)}] the server updates the global model by computing an \emph{aggregation function} on the local models received from clients (by default, the average, also referred to as FedAvg~\cite{mcmahan2017aistats}).
% \end{enumerate}
This process goes on until the global model converges. %(e.g., after a certain number of rounds or other similar stopping criteria).
%\\
% The advantages of FL over the traditional, centralized learning paradigm are undoubtedly clear in terms of flexibility/scalability (clients can join/disconnect from the FL network dynamically), network communications (only model weights\footnote{We will use \textit{parameters} and \textit{weights} interchangeably.} are exchanged between clients and server), and privacy (each client's private training data is kept local at the client's end and not uploaded to the server).
\\
% Security threats to FL
%However, the growing adoption of FL also raises security concerns~\cite{costa2022covert}, particularly about its confidentiality, integrity, and availability.
Although its advantages over standard ML, FL also raises security concerns~\cite{costa2022covert}. %, particularly about its confidentiality, integrity, and availability~\cite{costa2022covert}.
% OLD, LONG VERSION
% Indeed, some work deals with privacy leakage that may expose the local data of some clients~\cite{melis2019sp}. 
% A large body of work, instead, investigates attacks that usually aim to detriment the predictive accuracy of the learned global model. For instance, \emph{data poisoning} attacks achieve this goal by letting an adversary pollute the training set of some corrupt FL clients with maliciously crafted examples~\cite{jagielski2018sp}.
% Similarly, in \emph{model poisoning} the attacker attempts to tweak the global model weights~\cite{bhagoji2019pmlr} by directly perturbing the local model's weights of some infected FL clients before these are sent to the central server for aggregation, usually via so-called Byzantine attacks. 
% It turns out that Byzantine model poisoning attacks severely impact standard FedAvg; therefore, more robust aggregation functions must be designed to make FL systems secure.
Here, we focus on \emph{untargeted model poisoning} attacks~\cite{bhagoji2019pmlr}, where an adversary attempts to tweak the global model weights %\footnote{We will use the terms \textit{parameters} and \textit{weights} interchangeably.} 
by directly perturbing the local model's parameters of some infected clients before these are sent to the central server for aggregation.
In doing so, the adversary aims to jeopardize the global model \textit{indiscriminately} at inference time.
Such model poisoning attacks severely impact standard FedAvg; therefore, more robust aggregation functions must be designed to secure FL systems.
\\
% In this paper, we focus on designing a novel robust aggregation scheme at the server's end to contrast the effect of Byzantine model poisoning attacks.
%
% Current countermeasures and their limitations
%Several countermeasures have been proposed in the literature to combat model poisoning attacks on FL systems.
% Some methods use simple statistics more robust than plain average to smooth the impact of malicious updates (e.g., Trimmed Mean and FedMedian~\cite{yin2018icml}). 
% Other defenses implement outlier detection techniques to discard malicious updates from the aggregation performed at the server's end. Those are either based on heuristics (e.g., Krum/Multi-Krum~\cite{blanchard2017nips} and Bulyan~\cite{mhamdi2018pmlr}) or data-driven approaches (e.g., K-means clustering~\cite{shen2016acm} or DnC via spectral analysis~\cite{shejwalkar2021ndss}). 
% Finally, some strategies rely on a centralized ``source of trust'' to spot potential malicious updates (e.g., FLTrust~\cite{cao2020fltrust}).
% Several countermeasures have been proposed in the literature to combat model poisoning attacks on FL systems, i.e., to discard possible malicious local updates from the aggregation performed at the server's end. 
% These techniques range from simple statistics more robust than plain average (e.g., Trimmed Mean and FedMedian~\cite{yin2018icml}) to outlier detection heuristics (e.g., Krum/Multi-Krum~\cite{blanchard2017nips} and Bulyan~\cite{mhamdi2018pmlr}) or data-driven approaches (e.g., spectral analysis via K-means clustering~\cite{shen2016acm} or spectral analysis), or methods based on ``source of trust'' (e.g., FLTrust~\cite{cao2020fltrust}).
% OLD, LONG VERSION
%Several countermeasures have been proposed in the literature to combat Byzantine model poisoning attacks on FL systems.
% Descriptive statistics
% For example, Trimmed Mean and FedMedian aggregate local model updates using more robust statistics than standard average~\cite{yin2018icml}.
%
% % Heuristics for outlier detection
% Many existing Byzantine-resilient strategies implement some outlier detection heuristics to discard the model updates sent by potentially malicious clients from the input of the aggregation function.
% One of the most popular heuristics is Krum~\cite{blanchard2017nips}.
% This strategy tries to mitigate the impact of Byzantine attacks by selecting as a global model the local model with the smallest sum of Euclidean distances to {\em all} the other local models.
% Although powerful, Krum requires the server to know (or, at least, estimate) the number of malicious FL clients upfront, which is generally impossible in a realistic attack scenario. %
% Moreover, Krum may become ineffective for complex, high-dimensional model parameter spaces due to the curse of dimensionality.
% Bulyan~\cite{mhamdi2018pmlr} tries to overcome this issue by combining Krum with a variant of Trimmed Mean.
% % Data-driven outlier detection
% Other strategies use data-driven outlier detection techniques -- e.g., via K-means clustering~\cite{shen2016acm} -- to spot potential malicious local model updates. 
% %For instance, Shen et al. propose to cluster local model updates with K-means and thus identify outliers.
%
% % Other techniques
% As far as the server is concerned, any local model received can be from a potential malicious client. 
% FLTrust~\cite{cao2020fltrust} assumes the server acts as a client, i.e., trains a local model on an additional {\em trustworthy} dataset at the server's end and compares it against all the local models from other clients. 
% This way, the server can rely on some ``source of trust'' when discarding potentially malicious clients.
%\\
% Limitations of existing Byzantine-resilient strategies
Unfortunately, existing defense mechanisms either rely on simple heuristics (e.g., Trimmed Mean and FedMedian by~\cite{yin2018icml}) or need strong and unrealistic assumptions to work effectively (e.g., foreknowledge or estimation of the number of malicious clients in the FL system, as for Krum/Multi-Krum~\cite{blanchard2017nips} and Bulyan~\cite{mhamdi2018pmlr}, which, however, cannot exceed a fixed threshold).
Furthermore, outlier detection methods using K-means clustering~\cite{shen2016acm} or spectral analysis like DnC~\cite{shejwalkar2021ndss} do not directly consider the temporal evolution of local model updates received.
Finally, strategies like FLTrust~\cite{cao2020fltrust} require the server to collect its own dataset and act as a proper client, thereby altering the standard FL protocol.
\\
% OLD, LONG VERSION
% Overall, existing Byzantine-resilient strategies are either simple heuristics (e.g., FedMedian) or, if they are more complex, they rely on strong and unrealistic assumptions to work effectively (e.g., knowing the number of malicious clients in the FL system in advance, as for Krum and alike).
% Furthermore, data-driven outlier detection methods do not consider the temporary evolution of local model updates received (e.g., K-means clustering). 
% Finally, strategies like FLTrust requires the server to collect its own dataset and act as a proper client, thereby altering the standard FL protocol.
%
% Description of the proposed method
This work introduces a novel pre-aggregation \textit{filter} robust to untargeted model poisoning attacks. Notably, this filter $(i)$ operates without requiring prior knowledge or constraints on the number of malicious clients and $(ii)$ inherently integrates temporal dependencies. 
The FL server can employ this filter as a preprocessing step before applying \textit{any} aggregation function, be it standard like FedAvg or robust like Krum or Bulyan.
Specifically, we formulate the problem of identifying corrupted updates as a multidimensional (i.e., matrix-valued) time series anomaly detection task. 
The key idea is that legitimate local updates, resulting from well-calibrated iterative procedures like stochastic gradient descent (SGD) with an appropriate learning rate, show \textit{higher predictability} compared to malicious updates. This hypothesis stems from the fact that the sequence of gradients (thus, model parameters) observed during legitimate training exhibit regular patterns, as validated in Section~\ref{subsec:intuition}. %until convergence. 
%This regularity may be more pronounced for smooth convex loss functions, but it can still be captured within an appropriate time window, even for more complex and convoluted loss surfaces. 
%We provide evidence of this claim in Appendix~B, where we show that the average mutual information (i.e., ``predictability''), calculated over pairs of legitimate model updates sent at different FL rounds, is significantly higher than the corresponding computation for a malicious client.
\\
Inspired by the matrix autoregressive (MAR) framework for multidimensional time series forecasting~\cite{chen2021je}, we propose the FLANDERS ({\em \textbf{F}ederated \textbf{L}earning meets \textbf{AN}omaly \textbf{DE}tection for a \textbf{R}obust and \textbf{S}ecure}) filter.
The main advantages of FLANDERS over existing strategies like FLDetector~\cite{zhao2020multivariate} are its resilience to large-scale attacks, where $50\%$ or more FL participants are hostile, and the capability of working under realistic non-iid scenarios.
We attribute such a capability to two key factors: $(i)$ FLANDERS works without knowing a priori the ratio of corrupted clients, and $(ii)$ it embodies temporal dependencies between intra- and inter-client updates, quickly recognizing local model drifts caused by evil players. Below, we summarize our main contributions:

\begin{itemize}
\item[{\em(i)}]
We provide empirical evidence that the sequence of models sent by legitimate clients is more predictable than those of malicious participants performing untargeted model poisoning attacks.
\\
\item[{\em(ii)}] 
We introduce FLANDERS, the first pre-aggregation filter for FL robust to untargeted model poisoning based on multidimensional time series anomaly detection.
\\
\item[{\em(iii)}] 
We integrate FLANDERS into Flower,\footnote{\scriptsize{\url{https://flower.dev/}}} a popular FL simulation framework for reproducibility.
\\
\item[{\em(iv)}] 
We show that FLANDERS improves the robustness of the existing aggregation methods under multiple settings: different datasets, client's data distribution (non-iid), models, and attack scenarios.
\\
\item[{\em(v)}] 
We publicly release all the implementation code of FLANDERS along with our experiments.\footnote{\scriptsize{\url{https://anonymous.4open.science/r/flanders_exp-7EEB}}}
\end{itemize}

% Paper's structure and organization
The remainder of the paper is structured as follows. %some related work and the current state-of-the-art solutions to security issues that FL entails. 
Section~\ref{sec:background} covers background and preliminaries. 
In Section~\ref{sec:related}, we discuss related work.
Section~\ref{sec:problem} and Section~\ref{sec:method} describe the problem formulation and the method proposed. % to tackle it. 
Section~\ref{sec:experiments} gathers experimental results. %, and Section~\ref{sec:limitations} discusses some limitations of this work.
Finally, we conclude in Section~\ref{sec:conclusion}.
 %discusses the limitations of this work and draws future research directions.
%reports conclusions and draws perspectives for future research directions.

%%%%%%% OLD %%%%%%%
%to overcome the resilience of Byzantine failures in distributed Stochastic Gradient Descent computations. 
% The strength of Krum is its time complexity, which is linear in the gradient dimension. 
% However, the robustness of the approach is guaranteed for gradient-based learning applications only when the majority of the clients are not compromised. 
% Besides, the aggregation mechanism of Krum, as well as that of similar methods, is robust from a coarse-grained perspective and does not provide solutions to errors and perturbations that may occur at inference time.
%A related approach to~\cite{blanchard2017nips} is the work of Su et al.~\cite{su2016dc}. Here, the authors propose an iterated approximate agreement to tackle a multi-layer scenario attacked by Byzantine agents. 
%However, the method works efficiently on the sole discrete context and it is inapplicable to continuous state environments.
%\gabri{Maybe, we should just talk about the main limitations of existing countermeasures without digging into their details (or, we can just mention Krum as this is the most popular one). I will move the description of all these methods to the Related Work section.}

\section{Preliminaries}
\label{section:preliminaries}
\section{Notation and Preliminaries}\label{sec_prel}
Let $\mathbb{Z}_{>0}$ denote the set of positive integers and let $\mathbb{Z}_{[a,b]}$ denote the set of integers in the interval $[a,b]$. The $m\times m$ identity matrix is denoted by $I_m$ and its columns by $e_i$ for $i\in\mathbb{Z}_{[1,m]}$. We use $\mathbf{0}$ to denote a vector or a matrix of zeros of appropriate dimensions. For a sequence $\{z_k\}_{k=0}^{N-1}$ with $z_k\in\mathbb{R}^\eta$, we denote its stacked vector as $z = \begin{bmatrix}z_0^\top &z_1^\top & \dots & z_{N-1}^\top\end{bmatrix}^\top$ and a stacked window of it as $z_{[l,j]} = \begin{bmatrix}z_l^\top &z_{l+1}^\top & \dots & z_{j}^\top\end{bmatrix}^\top$ with $0\leq l<j$.\par
Persistence of excitation of a sequence and its extension to multiple sequences \cite{vanWaarde20} are defined as follows.
\begin{definition} The sequence \(\{z_k\}_{k=0}^{N-1}\), $z_k\in\mathbb{R}^{\eta}$, is said to be persistently exciting of order \(L\) if \(\textup{rank}(\mathscr{H}_{L}(z))=\eta L\), where $\mathscr{H}_L(z) = \begin{bmatrix}
		z_{[0,L-1]} & z_{[1,L]} & \cdots & z_{[N-L,N-1]}
	\end{bmatrix}$.
	\label{def_PE}
\end{definition}
\begin{definition}[\cite{vanWaarde20}]\label{def_cPE}
	The sequences $\{z_k^{(j)}\}_{k=0}^{N_j-1}$, with $z_k^{(j)}\in\mathbb{R}^\eta$ and $j\in\mathbb{Z}_{[1,r]}$, are said to be \textit{collectively persistently exciting} of order $L$ if rank$(\mathcal{H}_L(\mathscr{Z}))=\eta L$, where $\mathscr{Z} = \begin{bmatrix}
		(z^{(1)})^\top & \cdots & (z^{(r)})^\top
	\end{bmatrix}^\top,$ and
	\begin{equation*}
		\mathcal{H}_L(\mathscr{Z}) = \begin{bmatrix}
			\mathscr{H}_L(z^{(1)}) & \cdots & \mathscr{H}_L(z^{(r)})
		\end{bmatrix}.
	\end{equation*}
\end{definition}

\section{Method}
\label{section:method}
\section{Method}
\label{sec:method}

% \ml{``Inconsistent'' to ``large variation''}

% In this section, we propose our methods based on the observations in Section \ref{sec:motivation}.
In this section, we propose two techniques to further enhance the strong baseline to capture the variation of activation distributions better.
We first introduce spatial re-scaling to adapt the network to pixel-to-pixel variation.
We then propose channel-wise shifting and re-scaling to better capture the channel-to-channel variation.
Meanwhile, as both of the two methods are image-dependent, the image-to-image variation can be captured naturally.
By combining the two methods with our strong baseline, we build our enhanced BNN for SR, named EBSR.

% Because the activation distributions among pixels, channels and images have large variations \red{**are highly inconsistent} in SR networks, we introduce spatial re-scaling to adapt to pixel-wise variations and channel shift and re-scaling to adapt to channel-wise variations. And both of them are image-dependent to adapt to image-wise variations, which means during inference our network re-scales and shifts the distributions of activations flexibly for different input images. Based on these methods, we build an enhanced binary neural network for image super-resolution (EBSR).

% According to [3], the difference of activation magnitudes indicates different scaling factors are needed for each pixel.

\subsection{Spatial Re-scaling}
% It is better to use different scaling factors for different pixels to reduce the quantization error and retain more detailed information for image super-resolution. 

% \ml{In the main method, we do not need to introduce the previous works but can focus on introducing our own method. Channel rescaling in Real-to-binary Net is not relevant in this context.}

% Re-scaling the output of binary convolutions was proposed at the birth of BNN in XNOR-Net \cite{rastegari2016xnor} to reduce quantization error and improve accuracy for image classification tasks.
% It is computed as below:
% \begin{equation}
% \mathcal{A} * \mathcal{W} \approx(\operatorname{sign}(\mathcal{A}) \circledast \operatorname{sign}(\mathcal{W})) \odot \mathcal{K} \alpha
% \label{eq:xnor-net rescale}
% \end{equation}
% where $\circledast$ denotes the binary convolution and $\odot$ denotes the element-wise multiplication.
% $\mathcal{A}$, $\mathcal{W}$, $\alpha$, and $\mathcal{K}$ denote the activation, weight, weight scaling factor, and activation scaling factor, respectively.
%  Later in XNOR-Net++ \cite{bulat2019xnor}, Bulat et al. fuse the activation and weight scaling factors into a single one that is learned end-to-end based on gradients and this improves the classification accuracy on ImageNet dataset.

% % It is computed as Eq.~\ref{eq:xnor-net rescale}, where $\circledast$ denotes 
% %  the binary convolution and $\odot$ denotes the element-wise multiplication. The binary convolution of $\mathcal{A}$ and $\mathcal{W}$ is rescaled by the weight scaling factor $\alpha$ and the activation scaling factor $\mathcal{K}$, both of which are calculated analytically.


% \zc{Similarly, you should explain the meaning of A, W and the operators $\circledast$ in the formula}
% Then in Real-to-binary Net \cite{martinez2020training}, Martinez et al. used a data-driven channel re-scaling module that takes the pre-convolution activations as input to predict the activation scaling factor. Unlike that in XNOR-Net++ \cite{bulat2019xnor}, these scaling factors are not fixed during inference but rather inferred from data. By doing this, they further improved the classification accuracy on ImageNet over XNOR-Net++. 
As is shown in Figure \ref{fig:pixel}, activation distributions have large pixel-to-pixel variation in SR networks
and the difference of activation magnitudes indicates different scaling factors are preferred for different pixels.
Inspired by \cite{martinez2020training}, we propose spatial re-scaling to better adapt the network to the spatial variation
of activation distributions in SR networks.
% fit the various pixel-wise distributions in SR networks.
We take the real-valued activations $A$ before convolution as input and predict pixel-wise scaling factors $S(A)$, which re-scale the binary convolution output. Spatial re-scaling process can be formulated as follows:
\begin{equation}
A * W \approx(\operatorname{sign}(A) \circledast \operatorname{sign}(W)) \odot \alpha \odot S(A)
\label{eq:spatial rescale}
\end{equation}
where $\circledast$ denotes 
the binary convolution and $\odot$ denotes the element-wise multiplication. $A$, $W$, $\alpha$, and $S\left(A\right)$ denote real-valued activations, weights, the scaling factor of weights, and the spatial-wise scaling factor of activations respectively. $S\left(A\right) \in \mathbb{R}^{1\times H\times W}$ can be calculated with a convolution and a sigmoid function.
% as $\sigma\left( CONV\left(A\right)\right)$. 
As shown in Figure \ref{fig:method}(a), real-valued activations first go through a convolution layer,
which has an input channel of $C$ and an output channel of 1, 
and then pass through a sigmoid function to produce the scaling factors $S(A)$ along the spatial dimension.
During inference, the scaling factor will change dynamically according to different input feature maps.
By re-scaling binary convolution output using $S(A)$, we can reduce the quantization error and the original pixel-wise information in FP activation
will be preserved much better.
Spatial re-scaling leads to a large PSNR improvement of 0.24 dB (from 30.30 dB to 31.54 dB) on Set5 and 0.22 dB (from 25.09 dB to 25.31 dB)
on Urban100 compared with our strong baseline. 

\subsection{Channel-wise Shifting and Re-scaling}

\begin{table}[!tb]
\centering
\caption{Comparison between whether to fuse channel-wise shifting and re-scaling or not based on our baseline with spatial re-scaling. }
\label{tab:fusing}

\scalebox{0.65}{
\begin{tabular}{c|cc|cc|cc}
\hline
\multirow{2}{*}{Method}     & \multirow{2}{*}{OPs} & \multirow{2}{*}{Params} & \multicolumn{2}{c|}{Set5} & \multicolumn{2}{c}{Urban100} \\ \cline{4-7} 
                            &                      &                         & PSNR        & SSIM        & PSNR          & SSIM         \\ \hline
Baseline + spatial re-scale & 2.16G                & 0.05M                   & 31.54       & 0.883       & 25.31         & 0.759        \\
+ channel-wise shift and re-scale             & 2.34G                & 0.09M                   & 31.61       & 0.885       & 25.35         & 0.761        \\
+ w/ fusing                   & 2.27G                & 0.08M                   & \textbf{31.64}       & \textbf{0.885}       & \textbf{25.36}         & \textbf{0.761}        \\ \hline
\end{tabular}
}
\end{table}

In SR networks, activation distributions exhibit larger channel-to-channel variation (Figure \ref{fig:chl}).
Both the mean and magnitude of the activation distributions vary significantly across channels.
% Thus we use channel-wise shifting and re-scaling to adapt to various channel-wise distributions. 
\cite{martinez2020training} has proposed the data-driven channel re-scaling, 
but our method differs from them in further introducing data-driven thresholds to handle the channel-wise variation of both mean and magnitude.
Since the blocks to generate the scaling factors and thresholds are very similar, we further propose to fuse them into one module.
% and fusing channel-wise shifting and re-scaling into one module.
We evaluate the effect of fusing the two blocks in Table \ref{tab:fusing}.
With channel-wise shifting and re-scaling fused, our models have fewer operations and parameters overhead and slightly higher performance.

For the specific process, we take the real-valued activations as input and predict different thresholds and scaling factors for each channel. They are also image dependent, e.g., $\beta_{i}$ in Eq.\ref{eq:act_binarize} is no longer fixed during inference but generated according to different input feature maps. Channel-wise shifting and re-scaling can be formulated as follows:
\begin{equation}
A * W \approx(\operatorname{sign}(A-C_s(A)) \circledast \operatorname{sign}(W)) \odot \alpha \odot C_r(A)
\label{eq:channel-wise_shift_and_rescale}
\end{equation}
where $\circledast$ denotes 
the binary convolution and $\odot$ denotes the element-wise multiplication. $C_s(A), C_r(A) \in \mathbb{R}^{C\times1\times1}$ denote the channel-wise threshold and scaling factor, respectively. 
We show the block diagram in Figure \ref{fig:method}(b).
The real-valued input feature map is first squeezed to a ${C\times1\times1}$ vector by a global average pooling (GAP) layer.
The subsequent fully connected layers and ReLU learn the channel-wise information and output a ${2C\times1\times1}$ vector.
Then the ${2C\times1\times1}$ vector is split into two ${C\times1\times1}$ vectors.
We use the first $C$ channels as the channel-wise bias and pass the last $C$ channels through a sigmoid layer 
as the channel-wise scaling factor, which are used to shift the real-valued activations and re-scale the binary convolution output, respectively. 


% \ml{We can mention previously, channel-wise re-scale has been proposed. We propose to fuse them. Add the comparison between fuse v.s. no fuse.}

\begin{figure}[!tbp]%
  \centering
    \includegraphics[width=0.4\textwidth]{fig/methods.png}
  
% \subfloat[channel-wise shifting\&re-scale]{
%     \label{subfig:channel-wise shifting and re-scale}
%     \includegraphics[width=0.2\textwidth]{fig/chl shift and rescale.png}
%   }

  \caption{Block diagram for spatial re-scaling, and channel-wise shifting and re-scaling.} 
  % Input A is the real-valued activation tensor and C, H, and W denote its dimension. GAP stands for global average pooling. The reduction ratio r is set to 16 for a better trade-off between the performance and the number of operations and parameters.}
  \label{fig:method}
\end{figure}


\subsection{Network Structure}

Combining the spatial re-scaling and the channel-wise shifting and re-scaling methods, we construct the enhanced convolution layer (E-Conv).
Then we build our EBSR model based on E-Conv.
In Figure \ref{fig:E-conv}, we compare the binary convolution layer used in the baseline network and our proposed E-Conv.
We use spatial and channel-wise scaling factors to re-scale the binary convolution output,
and use channel-wise shifting to learn appropriate thresholds for each channel before binarization.
The scaling factors and threshold used in E-Conv are learnable and depend on the real-valued input activations.
In this way, our proposed EBSR can adapt to pixel-to-pixel, channel-to-channel, and image-to-image variations
to reduce the large binarization error and preserve more details.
% In this way, our proposed E-Conv reduces the large quantization error caused by binarization and keeps the original information of input feature maps to a large extent.


\begin{figure}[!tb]%
  \centering

    \includegraphics[width=0.5\textwidth]{fig/E-conv.png}

  \caption{Comparison of (a) the binary convolution layer with a skip connection used in our baseline network and (b) the proposed E-Conv.}
  \label{fig:E-conv}
\end{figure}


Figure \ref{fig:network} shows the basic block based on the E-Conv and our EBSR composed of the basic blocks. Following existing works, the convolution layers in the head and tail modules are not binarized. We choose the lightweight EDSR which has 16 basic blocks and 64 channels, and EDSR which has 32 basic blocks and 256 channels as our backbones, which correspond to EBSR-light and EBSR, respectively.

\begin{figure}[!tb]%
  \centering
  {
    \includegraphics[width=0.35\textwidth]{fig/network.png}
  }
  
  \caption{The structure of our proposed EBSR.  Convolution layers in purple are real-valued vanilla 3x3 convolutions.}
  \label{fig:network}
\end{figure}

\section{Experiment}
\label{section:experiment}
\section{Experimental Results}
\label{sec:experiments}
\subsection{Training Details}
\cite{Kalantari2017DeepHD} provides the first dataset specifically designed for multi-exposure HDR fusion under large motion. It consists of 74 training sets, which we use to supervise the training of our model. We crop the input images to patches of size \(256 \times 256\) at a step size of 64. This totally generates 20128 training samples. To augment training samples, we randomly rotate and flip the training images. The training adopts Adam optimizer. The learning rate is initialized to \(10^{-4}\) and is reduced to \(10^{-5}\) after 20 epochs. It is observed that 40 epochs are sufficient for the training to converge.    

\subsection{Numerical Evaluation}
We numerically measure the performance of our method on the 15 test sets of \cite{Kalantari2017DeepHD}, by Peak Signal-to-Noise Ratio (PSNR) and Structure Similarity, computed in both tonemapping domain (-\(\mu\)) and HDR linear domain (-L). Visual difference metric HDR-VDP-2 is also adopted, where the parameters are set as same as in previous works \cite{wu2018end} and \cite{niu2021hdrgan}. 

Table \ref{table_metrics} compares our model with state-of-the-art models. For \cite{yan2020nonlocal} and \cite{xiong2021hierarchical}, we use the results reported in the publications. Note that \cite{sen2012robust} and \cite{hu2013hdr} are not machine learning based methods. Moreover,  \cite{Kalantari2017DeepHD} and \cite{wu2018end} apply optical flow and homography transformation to preprocess the input images respectively, and hence entail extra computation. 

Table \ref{table_metrics} shows that our method outperforms competing method in terms of PSNR-L, SSIM-$\mu$, SSIM-L and HDR-VDP-2. It ranks the second best in PSNR-$\mu$, being slightly (0.1dB) inferior to \cite{xiong2021hierarchical}. Note that \cite{xiong2021hierarchical} utilizes a pretrained model to detect ghosting regions for training, whereas our method does not require any pretrained model. The high PSNR and SSIM scores varify that our model has strong HDR reconstruction ability and can accurately restore the radiance and structure of the scene in both tonemapping domain and HDR linear domain. Furthermore, its high performance in term of HDR-VDP-2\cite{mantiuk2011hdr} performance indicates that our method can generate HDR image visually close to the target image.

\begin{table*}[ht]
\centering
\begin{tabular}{l|c|c|c|c|c}
\hline
& PSNR-$\mu$ & PSNR-L & SSIM-$\mu$ & SSIM-L & HDR-VDP-2 \\
\hline
\bfseries Sen & 40.97 & 38.36 & 0.9830 & 0.9746 & 60.60\\
\hline
\bfseries Hu  & 35.65 & 30.80 & 0.9725 & 0.9491 & 58.34\\
\hline
\bfseries Kalantari & 42.69 & 41.22 & 0.9888 & 0.9845 & 65.05\\
\hline
\bfseries DeepHDR& 41.99 & 41.22 & 0.9878 & 0.9859 & \underline{65.91}\\
\hline
\bfseries AHDR & 43.62 & 41.03 & 0.9900  &\underline{0.9883} & 63.85 \\
\hline 
\bfseries NHDRRNet& 42.414 & - & 0.9887 & - & 61.21 \\
\hline 
\bfseries HDR-GAN &43.92 & \underline{41.57} &\underline{0.9905} &0.9865 & 65.45\\
\hline 
\bfseries HFNet & \textbf{44.28} & 41.47 & - & - & - \\
\hline 
\bfseries Ours & \underline{44.18} & \textbf{42.19}&\textbf{0.9912} & \textbf{0.9883}& \textbf{67.07} \\
\hline
\end{tabular}
\caption{Numerical performance of the proposed model, evaluated on the dataset by Kalantari-Ramamoorthi. The best and second best results for each metric are marked in \textbf{bold} and \underline{underlined}, respectively}
\label{table_metrics}
\end{table*}

\subsection{Visual Performance Evaluation}

\begin{figure*}[!htb]
\centering
\includegraphics[width=\textwidth]{experiments/kalantari_test.png}
\caption{Visual comparison on the test set of Kalantari-Ramamoorthi dataset. Zoom-in views of reconstruction by each method are presented on the saturated regions that contain moving objects. Our network built with gated Swin Transformer yields noticeably better visual results than other methods.}
\label{fig_kalantari_test}
\end{figure*}
Fig. \ref{fig_kalantari_test} present the visual performance of our method and comparable methods on two examples from \cite{Kalantari2017DeepHD}. We present the zoom-in views of two challenging cases, where large saturated regions contain substantial non-rigid motion in the reference image. The two patch-based methods do not reconstruct the missing details in the saturated regions, as they heavily rely on the details provided by the reference image for registration. Image reconstructed by the optical flow based method \cite{Kalantari2017DeepHD} suffers motion blur artifacts. This is because the convolutions of DeepHDR and HDR-GAN have limited receptive fields, and hence are hampered to repair missing content in misaligned regions by aligned regions. The gating mechanism of AHDR is only applied to low-level features, so the high-level outliers may deteriorate the HDR fusion. In contrast to comparable methods, our model remarkably overcomes the ghosting artifacts.

\begin{figure}[ht]
\centering
\includegraphics[width=\columnwidth]{experiments/sen_test.pdf}
\caption{Visual performance comparison on example images from the dataset by Sen et al. Zoom in views on challenging areas are presented. Although the ground truth is unavailable, it can be clearly observed that our method visually performs better than comparable methods.}
\label{sen_test}
\end{figure}

\begin{figure}[ht]
\centering
\includegraphics[width=\columnwidth]{experiments/tursun_test.pdf}
\caption{Visual performance comparison on example images from the dataset by Tursun et al. Compared to state of the art methods, our method suffers less ghosting artifact.}
\label{tursun_test}
\end{figure}

Fig.\ref{sen_test} and Fig.\ref{tursun_test} present visual performance of our method on two examples from benchmark datasets \cite{sen2012robust} and \cite{tursun2016objective}. As these test datasets   do not provide ground truth image. we mark the visual difference on the results generated by different methods. It can be seen that our method suffers less artifacts than other methods in various scenes with various motion patterns, achieving better visual results. Our method creates high-quality HDR more robustly and generalizes well. 

\subsection{Ablation Study}

\begin{table}[h]
\centering
\resizebox{\columnwidth}{!}{
\begin{tabular}{l|c|c|c|c|c}
\hline
                         & PSNR-$\mu$ & PSNR-l & SSIM-$\mu$ & SSIM-l & HDR-VDP-2 \\ \hline
restormer(w/o ssim loss) & 44.00  & 41.5   & 0.9906 & 0.9873 & 64.72  \\ \hline
Ours(w/o ssim loss)      & 44.07  & 41.83  & 0.9909 & 0.9879 &  64.78  \\ \hline
Ours                     & 44.18  & 42.19  & 0.9912 & 0.9883 & 67.07      \\ \hline
\end{tabular}
}
\caption{Experimental results of ablation study. We compare using Gated Swin Transformer v.s. Gated Transformer, and the combined loss function v.s. the traditional $l_{1}$ norm loss function.}
\label{table_ablation_block_loss}
\end{table}

We verify various components of our method, including Swin Transformer, loss function, and gating mechanism by ablation study.

\subsubsection{Ablation Study on Block Design}
Our model has similar architecture to Restormer, which uses modified Transformer, whereas we use modified Swin Transformer as the building unit. For comparison, we replace the residual modules in each block in our model with multiple transformer layers as in Restormer, with same number of transformer layers. Table \ref{table_ablation_block_loss} presents the results, which show that using Swin Transformer achieves superior performance in all measures. The reason is that the attention module of Restormer is computed channel-wise, but forgoes the cross-exposure spatial dependency to repair the non-aligned area. 

\subsubsection{Ablation Study on Loss Function}
We trained our model under different loss function configurations, as shown in \ref{table_ablation_block_loss}. The results validate that the SSIM loss benefits detail reconstruction.

\subsubsection{Ablation Study on Gating Mechanism}
\begin{table}[h]
\resizebox{\columnwidth}{!}{
\begin{tabular}{l|c|c|c|c|c}
\hline
           & PSNR-$\mu$ & PSNR-l & SSIM-$\mu$ & SSIM-l & HDR-VDP-2 \\ \hline
w/o gating & 43.14  & 41.03  & 0.9904 & 0.9868 &     64.88      \\ \hline
one gating & 43.44  & 41.42  & 0.9909 & 0.9882 &    67.13   \\ \hline
Ours       & 43.61  & 41.74  & 0.9909 & 0.9881 & 66.96     \\ \hline
\end{tabular}
}
\caption{Ablation experimental results to verify the effectiveness of the gating mechanism}
\label{table_ablation_gating}
\end{table}

The gating mechanism is an important component in our model. Ablation study is conducted in the gating mechanism as follows.

\textbf{w/o gating}: The gating mechanism is not used in the feed forward network of all transformer layers in the model, that it, our GST unit degenerate to the vanilla Swin Transformer.

\textbf{one gating}: The gating mechanism is only used in the first Swin Transformer layers subsequent to the embedding layer, but not used for other layers. 

 Table \ref{table_ablation_gating} shows the results of the ablation experiments, where the model is trained for 20 epochs. By removing the gating mechanism, the network relies on self-attention for image alignment, resulting in the lowest performance. On top of it, adding gates to low level layers notably improves the HDR reconstruction. Furthermore, by integrating the gating mechanism with all Swin Transformer layers, the model effectively inpaints information in non-aligned regions and obtains the highest HDR reconstruction results, thus validates the effectiveness of the gating mechanism in our model.

\subsection{Evaluating the Embeddings}
Results described earlier in the paper show that embeddings obtained via \ourmethod are excellent for a variety of predictive tasks. 
In this subsection, we present findings from a qualitative exploration of the embeddings. 
This was done in collaboration with a medical doctor at the hospital who specializes in infectious diseases.

We start our exploration by computing \textit{dispersions} for subsets of healthcare entities.
% Specifically, 
Suppose that the set of doctors $\doc$ is partitioned into subsets $\doc_1, \doc_2, \ldots, \doc_d$.
For any $1 \le i < j \le d$ and time $t$ we define the pairwise \textit{doctor dispersion} between $\doc_i$ and $\doc_j$ as
\begin{align}
\small
    disp_{\doc, t}(i, j) = \frac{\sum_{d \in \doc_i, d' \in \doc_j} ||\ed_{d,t} - \ed_{d',t}||_2}{|\doc_i|\cdot |\doc_j|}
\end{align}
% $$disp_{\doc, t}(i, j) = \frac{\sum_{d \in \doc_i, d' \in \doc_j} ||\ed_{d,t} - \ed_{d',t}||_2}{|\doc_i|\cdot |\doc_j|}.$$ 

Recall that $\ed_{d,t}$ denotes the embedding of a doctor $d$ at time $t$.
If the partition $\doc_1, \doc_2, \ldots, \doc_d$ represents a grouping of doctors by specialty (e.g., pediatrics, cardiology, anesthesia, neurology, etc.) then the pairwise
doctor dispersions provide a measure of the average distance between doctors from different specialties in the computed embeddings.

\begin{comment}
\begin{figure}[t!]
    \centering
    \includegraphics[width=\columnwidth]{plots/item_emb_validation/physician_embedding_manual_rs21.png}
    \caption{2-d projection of doctor embeddings learned by DECEnt. On average, pediatricians are further away from the rest of the doctors compared to doctors in Internal Medicine.}
    \label{fig:DECEnt_doctor_2d}
\end{figure}
\end{comment}

% \begin{comment}
% \begin{figure}{0.48\textwidth}
\begin{figure}[t!]
\centering
% \includegraphics[width=0.27\textwidth]{plots/item_emb_validation/physician_embedding_manual_rs21.png}
\includegraphics[width=0.8\columnwidth]{plots/item_emb_validation/physician_embedding_manual_rs21.png}
\caption{Doctor embeddings learned by \ourmethod.
On average pediatricians are further away from the rest of the doctors compared to doctors in Internal Medicine.
}
\label{fig:DECEnt_doctor_2d}
\end{figure}
% \end{comment}



One of the findings from the pairwise dispersion also confirms what is generally known about hospital operations. 
For example, doctors in general internal medicine and anesthesia interact with a wide variety of patients, and thus doctors in these specialties are close to those in other specialties. On the other hand, patients of doctors in Family Practice and Otolaryngology (ears, nose, and throat) tend not to require referral to other specialties. 
As a result, doctors in these specialties are relatively far away from doctors in other specialties.
In Figure \ref{fig:DECEnt_doctor_2d}, we plot the doctor embeddings learned by DECEnt projecting them to 2-d using t-SNE \cite{van2008visualizing}.
% We see a clear separation of the doctors that provide care to infants or children with internal medicine physicians and other physicians. which confirms that \ourmethod successfully captures the dynamics in care.

% Similarly, we compute pairwise \textit{medication dispersions} (induced by a partition of medications into classes) and pairwise \textit{room dispersions} (induced by a partition of rooms into units) and provide evaluations on the trained embeddings in the Appendix.
% See Section~\ref{appendix:evaluating_the_embeddings} i
% In the Appendix, we provide additional evaluations on the trained embeddings.
% \input{appendix_additional_evaluation}

\section{Related Work}
\label{section:related_works}
\setlength{\tabcolsep}{1.6mm}{
\renewcommand\arraystretch{1.1}
\begin{table}[ht]
  \centering
  \scalebox{0.9}{
  \begin{tabular}{llcccc}
    \toprule
    &\multirow{2}*{Methods} & \multirow{2}*{Sal.} &   \multicolumn{2}{c}{VOC} & MS~COCO \\
    \cmidrule(r){4-5}\cmidrule(r){6-6}
    &&&\texttt{val}&\texttt{test}&\texttt{val}\\
    \hline
    \multirow{13}*{\rotatebox{90}{ResNet-50}}
    &IRN~\cite{irn}          \tiny{CVPR'19}     &              & 63.5       & 64.8          & 42.0  \\
    &LayerCAM~\cite{layercam}\tiny{TIP'21}      &              & 63.0       & 64.5          & -     \\
    &AdvCAM~\cite{advcam}    \tiny{CVPR'21}     &              & 68.1       & 68.0          & 44.2  \\
    &RIB~\cite{rib}          \tiny{NeurIPS'21}  &              & 68.3       & 68.6          & 44.2  \\
    &ReCAM~\cite{recam}      \tiny{CVPR'22}     &              & 68.5       & 68.4          & 42.9  \\
    % \rowcolor{Gray}
    &\cellcolor{Gray}IRN+\texttt{LPCAM}    &\cellcolor{Gray} & \cellcolor{Gray}68.6    & \cellcolor{Gray}68.7      & \cellcolor{Gray}44.5  \\
    &SIPE~\cite{sipe}        \tiny{CVPR'22}     &              & 68.8       & 69.7          & 40.6  \\
    &OOD~\cite{ood}+Adv      \tiny{CVPR'22}     &              & 69.8       & 69.9          & -     \\
    &AMN~\cite{amn}          \tiny{CVPR'22}     &              & 69.5       & 69.6          & 44.7  \\
    &\cellcolor{Gray}AMN+\texttt{LPCAM}    &\cellcolor{Gray} & \cellcolor{Gray}70.1    &\cellcolor{Gray} 70.4      & \cellcolor{Gray}45.5  \\ 
    &ESOL~\cite{esol}        \tiny{NeurIPS'22}  &              & 69.9$^*$   & 69.3$^*$      & 42.6  \\
    &CLIMS~\cite{clims}      \tiny{CVPR'22}     &              & 70.4$^*$   & 70.0$^*$      & -     \\
    &EDAM~\cite{edam}        \tiny{CVPR'21}     &\checkmark    & 70.9$^*$   & 71.8$^*$      & -     \\
    &\cellcolor{Gray}EDAM+\texttt{LPCAM}  &\cellcolor{Gray}\checkmark & \cellcolor{Gray}71.8$^*$ &\cellcolor{Gray} 72.1$^*$& \cellcolor{Gray}42.1\\
    \hline
    \multirow{9}*{\rotatebox{90}{WideResNet-38}}
    &Spatial-BCE~\cite{sbce} \tiny{ECCV'22}     &              & 70.0       & 71.3      & 35.2  \\
    &BDM~\cite{bdm}          \tiny{ACMMM'22}    &\checkmark    & 71.0       & 71.0      & 36.7  \\ 
    &RCA~\cite{rca}+OOA      \tiny{CVPR'22}     &\checkmark    & 71.1       & 71.6      & 35.7  \\
    &RCA~\cite{rca}+EPS      \tiny{CVPR'22}     &\checkmark    & 72.2       & 72.8      & 36.8  \\
    &HGNN~\cite{hgnn}        \tiny{ACMMM'22}    &\checkmark         & 70.5$^*$   & 71.0$^*$  & 34.5  \\ 
    &EPS~\cite{eps}          \tiny{CVPR'21}     &\checkmark         & 70.9$^*$   & 70.8$^*$  & -     \\
    &RPIM~\cite{rpim}        \tiny{ACMMM'22}    &\checkmark         & 71.4$^*$   & 71.4$^*$  & -     \\ 
    &L2G~\cite{l2g}          \tiny{CVPR'22}     &\checkmark         & 72.1$^*$   & 71.7$^*$  & 44.2  \\
    \hline
    \multirow{2}*{\rotatebox{90}{\small{DeiT-S}}}
    &MCTformer~\cite{mctformer}    \tiny{CVPR'22}     &                 & 71.9$^{\dag}$  & 71.6$^{\dag}$   & 42.0  \\
    &\cellcolor{Gray}MCTformer+\texttt{LPCAM}      &\cellcolor{Gray} & \cellcolor{Gray}72.6$^{\dag}$  & \cellcolor{Gray}72.4$^{\dag}$  &\cellcolor{Gray} 42.8 \\
    \bottomrule
  \end{tabular}}
  \vspace{-2mm}
  \caption{The mIoU results (\%) based on DeepLabV2 on VOC and MS~COCO. The side column shows three backbones of multi-label classification model. ``Sal.'' denotes using saliency maps. * denotes the segmentation model is pre-trained on MS~COCO. $^\dag$ denotes the segmentation model is pre-trained on VOC.
  }
  \vspace{-6mm}
  \label{table_related}
\end{table}
}



\section{Discussion and Limitations}
\label{section:discussion}
We provide some comments on the growth conditions which constituted the majority of our analysis in sections \ref{sec:Hmixing} and \ref{sec:Hsigma}. In the simplest cases of Lemma \ref{lemma:unstableGrowth}, growth was established in an analogous fashion to the old one-step expansion condition (\ref{eq:oldOneStepExpansion}), finding the relevant Jacobians $M_j$ and checking that their expansion factors $K(M_j)$ satisfy
\begin{equation}
    \label{eq:discussionOneStep}
    \sum_j \frac{1}{K(M_j)} <1.
\end{equation}
For the more complicated cases, the inductive method used to establish growth near the accumulation points in Lemma \ref{lemma:unstableGrowth} and the weakened one-step expansion condition (\ref{eq:oneStep}) both address the same fundamental issue: the splitting of unstable curves by singularities into an unbounded number of small components. They circumvent this obstacle in rather different ways, however. While (\ref{eq:oneStep}) generalises (\ref{eq:discussionOneStep}) to ensure an growth of unstable curves `on average' (see \cite{chernov_statistical_2009} for a precise statement), our inductive method is a more direct adaptation of (\ref{eq:discussionOneStep}), using it to generate contradictory geometric conditions which a hypothetical non-growing unstable curve must satisfy. It may be possible to prove Theorem \ref{sec:Hmixing} using (\ref{eq:oneStep}) as the basis for growth. Since we required (\ref{eq:oneStep}) anyway for proving Theorem \ref{thm:HsigmaExp}, this could potentially condense our analysis, but only to a minor extent. A convenience of the method used in section \ref{sec:Hmixing} is that, by way of the `simple intersection' property, it naturally gives geometric information on the images of manifolds, useful for proving the property \textbf{(M)} of Theorem \ref{thm:katok-strelcyn}.

We expect that essentially analogous analysis can be applied to establish mixing properties in a wide class of piecewise linear non-uniformly hyperbolic maps, including those (like the OTM) which sit on the boundary of ergodicity and beyond. While we have relied on the precise partition structure of $H_\sigma$, its fundamental feature (self-similar sequences of elements $A^k$, sharing boundaries with its neighbours $A^{k-1},A^{k+1}$ and accumulating onto some point $p$) is quite typical to return map systems. See, for example, those of various stadium billiards \cite{chernov_chaotic_2006,chernov_improved_2008,chernov_statistical_2009} and LTMs \cite{springham_polynomial_2014}. Indeed, the same method can be used to prove the Bernoulli property for non-monotonic LTMs \cite{myers_hill_mixing_2022}, where monotonicity of the manifold images cannot be assumed and the classical argument \cite{sturman_mathematical_2006} fails. The OTM is the pointwise limit of these maps as the boundary shrinks to null measure. It further has utility in proving growth conditions for maps which are uniformly hyperbolic but possess regions $A_j$ where the hyperbolicity is very weak, signified by $K(M_j) \approx 1$, so that (\ref{eq:discussionOneStep}) fails. Typically this leads to suboptimal bounds on mixing windows, see e.g. \cite{wojtkowski_model_1981,przytycki_ergodicity_1983,myers_hill_family_2022}. The map $H_{(\eta,\eta)}$ for $\eta \approx 1/2$ is another example, possessing weak hyperbolicity over $A_2, A_3$. Letting $\varepsilon = |\eta-1/2|>0$, there is an upper bound $N = N(\varepsilon)$ on escape times from the intersections $A_2\cap \sigma, A_3 \cap \sigma$. The growth lemma then follows by applying the inductive step roughly $N$ times and can be established for arbitrarily small $\varepsilon$, opening the door to establishing optimal mixing windows.

The above gives two examples of piecewise linear perturbations to $H$ where mixing with respect to Lebesgue is preserved and our methods can be applied. Nonlinear perturbations to the shear profiles complicate the analysis in several ways. Firstly as the map's Jacobians takes on a broader range of values, cone invariance becomes an increasingly harder condition to establish. Cones must be widened, giving looser bounds on expansion factors, which may already be weak due to new regions of weaker stretching. This, together with the change from polygonal to curvilinear return time partition elements and nonlinear local manifolds, adds some complexity to showing growth conditions. This does not rule out certain (small) nonlinear perturbations however. There is some leeway in the inequalities which govern cone invariance and growth of local manifolds, the latter of which is not too dissimilar from the piecewise linear setting (see Lemmas \ref{lemma:piecewiseApprox}, \ref{lemma:componentLength}). Certain small perturbations would not alter the \emph{topological} structure of the return time partition, i.e. which elements share boundaries, the key information needed for setting up the induction. Finally while the partition elements would no longer be polygonal, only coarse geometric information is required for verifying each inductive step. Following the above, a potential perturbation could be to replace the linear portions of each shear by a cubic, perturbing the tent profile
\[  f(t) = \begin{cases} 2t & 0 \leq t \leq 1/2, \\ 2(1-t) & 1/2 \leq t \leq 1 ,\end{cases} \]
of the OTM shears to
\[  f_a(t) = \begin{cases} \frac{1}{8} t \left(16 - a + 6at - 8at^{2} \right) & 0 \leq t \leq 1/2, \\ \frac{1}{8}\left(1-t\right)\left( 16 - a + 6a\left(1-t\right) - 8a\left(1-t\right)^{2}\right)  & 1/2 \leq t \leq 1, \end{cases}   \]
for $a>0$. For small enough $a$ the gradient range $f'(t)$ is restricted to small neighbourhoods of $\{ 2, -2\}$ and the escape time partition retains a similar structure. We illustrate this in Figure \ref{fig:perturbations}, showing escapes from the square $S_3$ under the map $G \circ F$, equivalent to escapes from the perturbed $A_3$ under the $G \circ F$, but with a cleaner geometry for comparison. When $a$ is too large the analogy to the OTM breaks down. At $a=16$ the map is twice differentiable everywhere and features a new source of slowed mixing, the Jacobian is the identity at the corner points $x,y \in \{  0, 1/2 \}$ giving locally parabolic behaviour (visible in the escape time partition). 

\begin{figure}
    \centering
    \includegraphics[width=0.24 \linewidth]{0.png}
    \includegraphics[width=0.24 \linewidth]{4.png}
    \includegraphics[width=0.24 \linewidth]{8.png}
    \includegraphics[width=0.24 \linewidth]{16.png}
    \caption{Partition of escape times from $S_3$ under the mapping $F \circ G$ for $a= 0,4,8,16$. }
    \label{fig:perturbations}
\end{figure}


\par \noindent \textbf{Acknowledgements:}
The authors acknowledge funding from the CDC MInD Healthcare Network grant U01CK000594 and NSF grant 1955939.
The authors thank feedback from other University of Iowa CompEpi group members.
%Support for this research is provided by the Centers for Disease Control and Prevention MInD Healthcare Network (Awards U01CK000531 and U01CK000594 with respective Covid-19 supplements).
%%
%% The next two lines define the bibliography style to be used, and
%% the bibliography file.

\newpage
\bibliographystyle{IEEEtran}
% \bibliography{bibliography}

\begin{thebibliography}{1}

% Generated by IEEEtran.bst, version: 1.14 (2015/08/26)
% \begin{thebibliography}{10}
\providecommand{\url}[1]{#1}
\csname url@samestyle\endcsname
\providecommand{\newblock}{\relax}
\providecommand{\bibinfo}[2]{#2}
\providecommand{\BIBentrySTDinterwordspacing}{\spaceskip=0pt\relax}
\providecommand{\BIBentryALTinterwordstretchfactor}{4}
\providecommand{\BIBentryALTinterwordspacing}{\spaceskip=\fontdimen2\font plus
\BIBentryALTinterwordstretchfactor\fontdimen3\font minus
  \fontdimen4\font\relax}
\providecommand{\BIBforeignlanguage}[2]{{%
\expandafter\ifx\csname l@#1\endcsname\relax
\typeout{** WARNING: IEEEtran.bst: No hyphenation pattern has been}%
\typeout{** loaded for the language `#1'. Using the pattern for}%
\typeout{** the default language instead.}%
\else
\language=\csname l@#1\endcsname
\fi
#2}}
\providecommand{\BIBdecl}{\relax}
\BIBdecl

\bibitem{choi2018mime}
E.~Choi, C.~Xiao, W.~Stewart, and J.~Sun, ``Mime: Multilevel medical embedding
  of electronic health records for predictive healthcare,'' in \emph{NIPS},
  2018.

\bibitem{CDCCDI}
C.~for Disease~Control and Prevention, \emph{Clostridioides difficile
  Infection}, Reviewed Nov 13, 2019 (accessed June 10, 2020).

\bibitem{dai2016deep}
H.~Dai, Y.~Wang, R.~Trivedi, and L.~Song, ``Deep coevolutionary network:
  Embedding user and item features for recommendation,'' in \emph{DLRS}, 2017.

\bibitem{kumar2019predicting}
S.~Kumar, X.~Zhang, and J.~Leskovec, ``Predicting dynamic embedding trajectory
  in temporal interaction networks,'' in \emph{ACM SIGKDD}, 2019.

\bibitem{zhou2018dynamic}
L.~Zhou, Y.~Yang, X.~Ren, F.~Wu, and Y.~Zhuang, ``Dynamic network embedding by
  modeling triadic closure process,'' in \emph{AAAI}, 2018.

\bibitem{Bourgain1986TheMI}
J.~Bourgain, ``The metrical interpretation of superreflexivity in banach
  spaces,'' \emph{Israel Journal of Mathematics}, vol.~56, pp. 222--230, 1986.

\bibitem{kingma2014adam}
D.~P. Kingma and J.~Ba, ``Adam: A method for stochastic optimization,'' in
  \emph{ICLR}, 2015.

\bibitem{grover2016node2vec}
A.~Grover and J.~Leskovec, ``node2vec: Scalable feature learning for
  networks,'' in \emph{ACM SIGKDD}, 2016.

\bibitem{perozzi2014deepwalk}
B.~Perozzi, R.~Al-Rfou, and S.~Skiena, ``Deepwalk: Online learning of social
  representations,'' in \emph{ACM SIGKDD}, 2014.

\bibitem{Nguyen2018CTDNE}
G.~H. Nguyen, J.~B. Lee, R.~A. Rossi, N.~K. Ahmed, E.~Koh, and S.~Kim,
  ``Continuous-time dynamic network embeddings,'' in \emph{IW3C2 WWW}, 2018.

\bibitem{li2019using}
B.~Y. Li, J.~Oh, V.~B. Young, K.~Rao, and J.~Wiens, ``Using machine learning
  and the electronic health record to predict complicated clostridium difficile
  infection,'' in \emph{OFID}, 2019.

\bibitem{min2019predictive}
X.~Min, B.~Yu, and F.~Wang, ``Predictive modeling of the hospital readmission
  risk from patients’ claims data using machine learning: a case study on
  copd,'' \emph{Sci. Rep.}, 2019.

\bibitem{monsalve2015improving}
M.~Monsalve, S.~Pemmaraju, S.~Johnson, and P.~M. Polgreen, ``Improving risk
  prediction of clostridium difficile infection using temporal event-pairs,''
  in \emph{IEEE ICHI}, 2015.

\bibitem{van2008visualizing}
L.~Van~der Maaten and G.~Hinton, ``Visualizing data using t-sne.'' \emph{JMLR},
  2008.

\bibitem{Torricelli2020}
M.~Torricelli, M.~Karsai, and L.~Gauvin, ``weg2vec: Event embedding for
  temporal networks,'' \emph{Sci. Rep.}, 2020.

\bibitem{Tran2015eNRBM}
T.~Tran, T.~D. Nguyen, D.~Phung, and S.~Venkatesh, ``{{L}earning vector
  representation of medical objects via {E}{M}{R}-driven nonnegative restricted
  {B}oltzmann machines (e{N}{R}{B}{M})},'' \emph{J Biomed Inform}, 2015.

\bibitem{Zhu2016CNN}
Z.~Zhu, C.~Yin, B.~Qian, Y.~Cheng, J.~Wei, and F.~Wang, ``Measuring patient
  similarities via a deep architecture with medical concept embedding,'' in
  \emph{IEEE ICDM}, 2016.

\bibitem{FarajtabarWGLZS15COEVOLVE}
M.~Farajtabar, Y.~Wang, M.~Gomez~Rodriguez, S.~Li, H.~Zha, and L.~Song,
  ``Coevolve: A joint point process model for information diffusion and network
  co-evolution,'' \emph{Advances in Neural Information Processing Systems},
  vol.~28, 2015.

\bibitem{makar2018learning}
M.~Makar, J.~Guttag, and J.~Wiens, ``Learning the probability of activation in
  the presence of latent spreaders,'' in \emph{AAAI}, 2018.

\bibitem{sherman2017leveraging}
E.~Sherman, H.~Gurm, U.~Balis, S.~Owens, and J.~Wiens, ``Leveraging clinical
  time-series data for prediction: a cautionary tale,'' in \emph{AMIA}, 2017.

\bibitem{adhikari2019fast}
B.~Adhikari, B.~Lewis, A.~Vullikanti, J.~M. Jim{\'e}nez, and B.~A. Prakash,
  ``Fast and near-optimal monitoring for healthcare acquired infection
  outbreaks,'' \emph{PLoS CompBio}, 2019.

\bibitem{sundareisan2015hidden}
S.~Sundareisan, J.~Vreeken, and B.~A. Prakash, ``Hidden hazards: Finding
  missing nodes in large graph epidemics,'' in \emph{SDM}, 2015.

\bibitem{jang2019evaluating}
H.~Jang, S.~Justice, P.~M. Polgreen, A.~M. Segre, D.~K. Sewell, and S.~V.
  Pemmaraju, ``Evaluating architectural changes to alter pathogen dynamics in a
  dialysis unit,'' in \emph{IEEE/ACM ASONAM}, 2019.

\bibitem{wang2016structural}
D.~Wang, P.~Cui, and W.~Zhu, ``Structural deep network embedding,'' in
  \emph{ACM SIGKDD}, 2016.

\bibitem{yu2018learning}
W.~Yu, C.~Zheng, W.~Cheng, C.~C. Aggarwal, D.~Song, B.~Zong, H.~Chen, and
  W.~Wang, ``Learning deep network representations with adversarially
  regularized autoencoders,'' in \emph{ACM SIGKDD}, 2018.

\bibitem{Miotto2016DeepPatient}
R.~Miotto, L.~Li, B.~A. Kidd, and J.~T. Dudley, ``Deep patient: An unsupervised
  representation to predict the future of patients from the electronic health
  records,'' \emph{Sci. Rep.}, 2016.

\bibitem{dovsilovic2018explainable}
F.~K. Do{\v{s}}ilovi{\'c}, M.~Br{\v{c}}i{\'c}, and N.~Hlupi{\'c}, ``Explainable
  artificial intelligence: A survey,'' in \emph{MIPRO}.\hskip 1em plus 0.5em
  minus 0.4em\relax IEEE, 2018.

\bibitem{TURGEMAN2017376}
L.~Turgeman, J.~H. May, and R.~Sciulli, ``Insights from a machine learning
  model for predicting the hospital length of stay (los) at the time of
  admission,'' \emph{Expert Syst. Appl.}, 2017.

\end{thebibliography}

\end{document}