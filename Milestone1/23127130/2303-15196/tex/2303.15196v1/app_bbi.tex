\section{Bouncing Born--Infeld (BBI) optimiser}\label{app:bbi}

Usual machine learning optimisation algorithms can be naturally described in physics terms as a particle moving down an irregular hill. Stochastic gradient descent with momentum is a standard example, as it can be viewed as a noisy and discretised version of a particle motion. Crucially, this is a friction-based evolution, such that the particle stops when there is insufficient kinetic energy to escape a minimum in the potential. \cite{de2022born} proposed an energy-conserving algorithm in which there is no friction and where the optimisation process slows down near the minima as this region dominates the phase space volume of the system. The algorithm of \cite{de2022born} is based on the relativistic Born--Infeld dynamics \citep{BornInfeld1934}, where the total potential energy ($V$) depends on the speed limit as $V=v_{\textrm{rel}}^2$, such that as $V \rightarrow 0$ the particle stops. For completeness, we summarise below BBI update rules \citep{de2022born}:
    \begin{subequations}
        \begin{align}
            \boldsymbol{\Pi}_{i+1}  &=   \boldsymbol{\Pi}_{i} - \frac{1}{2} \boldsymbol{\nabla} V_i \Delta t \left(\frac{V_i}{E} + \frac{E}{V_i}\right), \\
          \boldsymbol{\Theta}_{i+1} &=  \boldsymbol{\Theta}_{i} + \boldsymbol{\Pi}_{i+1} \Delta t \frac{V_i}{E},
         \end{align}
    \end{subequations}
with the nomenclature
\begin{itemize}[itemindent=1.2cm]
    \item[$\boldsymbol{\Theta}_{i}$] parameter vector with components $\theta_i$
    \item[$\boldsymbol{\Pi}_{i}$] momentum vector with components $\pi_i$
    \item[$i$] optimisation step number
    \item[$\Delta t$] optimisation step size (learning rate)
    \item[$V_i= V(\boldsymbol{\Theta}_i)$] the potential of the $i$-th step
    \item[$E = V_0 + \delta E$] constant dependent on the initialisation, and the additional initial energy parameter $\delta E$
    \item[$\boldsymbol{\Pi}_0$] the initial momentum, set as $ - \frac{\boldsymbol{\nabla} V(\boldsymbol{\Theta}_0)}{\mid\boldsymbol{\nabla} V(\boldsymbol{\Theta}_0)\mid} \sqrt{\frac{E^2}{V_0} -V_0}$.
\end{itemize}

As $E$ is constant, a particle can be trapped in long-lived orbits in motion. To  avoid such stable orbits and boost chaotic mixing, random bounces are introduced  by generating a new random momentum vectors with the same absolute momentum. There are three additional hyperparameters controlling bouncing: number of bounces ($N_b$), fixed timesteps for bounces ($T_0$), and progress-dependent timesteps for bounces ($T_1$).  Additionally momentum can be re-scaled to conserve energy lost  due to discretisation effects.