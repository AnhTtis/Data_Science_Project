\documentclass[11pt]{article}
\usepackage{pdfsync}
\usepackage{enumitem}
\usepackage{cite}
\usepackage{epsfig,epsf}
\usepackage{amsmath}
\usepackage{amsfonts}
\usepackage{amssymb}
\usepackage{amsthm}
\usepackage{epstopdf}
\usepackage{dsfont}
\usepackage{mathrsfs}
\catcode`\@=11
%\usepackage{eucal}
\usepackage{empheq}

\newtheorem{thm}{Theorem}
\newtheorem{lemma}{Lemma}

\usepackage[colorlinks=true,
linkcolor=purple,
anchorcolor=magenta,
citecolor=blue
]{hyperref}
\usepackage{slashed}
\usepackage{color,colordvi}
\usepackage{tcolorbox}
%\usepackage{marvosym}
\usepackage{nicefrac,xfrac}
\textwidth 173mm
\textheight 230mm
\topmargin -40pt
\oddsidemargin -0.45cm
\evensidemargin -0.45cm
%%%%%%%%%%%%%%%%%%%%%%%%%%%%%%%%%%%%%%%%%%%%%%%%%%%%%%%%%%%%%%%%%%%%%%%%%%%%%%%%%%%%%%%%%%%%%%%%%%%%%%%%%%%%%%%%%%%%%%
%\renewcommand{\theequation}{\arabic{section}.\arabic{equation}}
%\renewcommand{\thefootnote}{\fnsymbol{footnote}}
%\renewcommand{\thetable}{\arabic{table}}
%%%%%%%%%%%%%%%%%%%%%%%%%%%%%%%%%%%%%%%%%%%%%%%%%%%%%%%%%%%%%
\DeclareMathOperator{\sign}{sign}

%\DeclareMathOperator{\Li}{Li}

%\newcommand{\ntag}{\notag\\ &\quad}



%%%%%%%%%%%%%%%%%%%%%%%%%%%%%%%%%%%%%%%%%%%%%%%%%%%%

%\newcommand{\deriv}{\stackrel{\leftrightarrow}{D}}
%\newcommand{\derleft}{\stackrel{\leftarrow}{D}}
%\newcommand{\derright}{\stackrel{\rightarrow}{D}}
\newcommand \widebar [1] {\overline{#1}}
%%%%%%%%%%%%%%%%%%%%%%%%%%%%%%%%%%%%%%%%%%%%%%%%%%%%
%\def \qqquad {\qquad\quad}
%\def \qqqquad {\qquad\qquad}
%%%%%%%%%%%%%%%%% Macros %%%%%%%%%%%%%%%%%%%%%%%%
%\newcommand{\widefbox}{\addtolength{\fboxsep}{5pt}\fbox}
%\newcommand*\widefbox[1]{\fbox{\hspace{1em}#1\hspace{1em}}}

%\definecolor{myblue}{rgb}{.8, .8, 1}
%\definecolor{shadecolor}{rgb}{0.92,0.9,0.9}
% \definecolor{light-gray}{rgb}{0.827451,0.827451,0.827451}

%\newcommand*\mybluebox[1]{%
%\colorbox{myblue}{\hspace{1em}#1\hspace{1em}}}


\def \Tr {\mbox{Tr\,}}
\def \tr {\mbox{tr}}
%\newcommand \vev [1] {\langle{#1}\rangle}
%\newcommand \VEV [1] {\left\langle{#1}\right\rangle}
%\newcommand \rvev [1] {\langle{#1}\rangle}
%\newcommand \ket [1] {|{#1}\rangle}
%\newcommand \bra [1] {\langle {#1}|}
%\def \Li {\text{Li\,}}

%\def\inbar{\,\vrule height1.5ex width.4pt depth0pt}
%\def\IC{\relax\hbox{$\inbar\kern-.3em{\rm C}$}}
%\def\IZ{\relax{\hbox{\cmss Z\kern-.4em Z}}}
%\def\IR{{\hbox{{\rm I}\kern-.2em\hbox{\rm R}}}}
%\def\R{{\tiny \IR}}
%\def\IP{{\hbox{{\rm I}\kern-.2em\hbox{\rm P}}}}
%\def\II{\hbox{{1}\kern-.25em\hbox{l}}}


%\newcommand{\twist}{ t}

\numberwithin{equation}{section}
\renewcommand{\thefootnote}{\fnsymbol{footnote}}
%%%%%%%%%%%%%%%%%%%%%%%%%%%%%%%%%%%%%%%%%%%%%%%%%%


\begin{document}

\begin{titlepage}

\begin{flushright}
\begin{tabular}{l}
\large  MPP-2023-50
\end{tabular}
\end{flushright}


\vskip3cm

\begin{center}
{\LARGE \bf
  {Unitarity of the SoV transform for %noncompact  %\\[2mm]
  $\mathbf{SL}(2,\mathbb C)$ spin chains. } }

\vspace{1cm}

 {\sc A.N.~Manashov}${}$
\\[0.5cm]

 {\it Max-Planck-Institut f\"ur Physik, Werner-Heisenberg-Institut, 80805 M\"unchen, Germany.}

\vskip2cm

{\bf Abstract:\\[10pt]} \parbox[t]{\textwidth}{
We prove  the unitarity of the separation of variables transform for   $\mathrm{SL}(2,\mathbb C)$ spin chains by a method based on the use
of Gustafson integrals. }
\end{center}

\end{titlepage}

\setcounter{footnote}{0}


%%%%%%%%%%%%%%%%%%%%%%%%%%%%%%%%%%%%%%%%%%%%%%%%%%%%%%%%%%%%%%%%%%%%%%%%%%%%%%%%%%%%%%%%%%%%%%%%%%%%%%%%%%%%%%%%%%%%%%%%%%%%%%%%%%%%%%%%%%%%%
\section{Introduction}
\label{sect:intr}
%%%%%%%%%%%%%%%%%%%%%%%%%%%%%%%%%%%%%%%%%%%%%%%%%%%%%%%%%%%%%%%%%%%%%%%%%%%%%%%%%%%%%%%%%%%%%%%%%%%%%%%%%%%%%%%%%%%%%%%%%%%%%%%%%%%%%%%%%%%%%


Theory of quantum integrable models is an important part of modern theoretical physics. The solution of such models relies on the Quantum
Inverse Scattering Method (QISM) which includes such techniques as the Algebraic Bethe Ansatz (ABA)~\cite{MR549615} and Separation of
Variables (SoV)\cite{MR802110,MR1239668}. The ABA  allows one to effectively  calculate energies and eigenstates of integrable models and
to address more complicated problems such as calculating norms~\cite{MR677006}, scalar products~\cite{MR1007797} and correlation
functions~\cite{MR763763,MR1741654}. Models with  infinite dimensional Hilbert spaces, the Toda chain~\cite{MR626693} being the most famous
example, are, however,  beyond ABA's grasp. The solution of such models relies on the SoV method proposed by
Sklyanin~\cite{MR802110,MR1239668}. The method consists in constructing a map between  the original Hilbert space, $\mathds H_\text{org}$,
in which the model is formulated, and an auxiliary Hilbert space, $\mathds H_\text{SoV}$. This map  is constructed in such a way that a
multidimensional spectral problem associated with the original Hamiltonian is reduced to a one-dimensional problem on an auxiliary Hilbert
space which usually takes the form of the Baxter $T-Q$ relation. Technically  constructing the SoV representation is equivalent to finding
the eigenfunctions of an element of the monodromy matrix associated with the model. For the Toda chain it was done by Kharchev and Lebedev
\cite{MR1751619,MR1831292}. Later, a regular method  for obtaining eigenfunctions for models with an R-matrix  of the rank one~\footnote{
In recent years, significant progress has been made in constructing SoV representations for higher rank finite-dimensional models, see
refs.~\cite{MailletNiccoli18,MailletNiccoli19,RyanVolin19,GromovSizov17,GromovRyan20}} was developed in~\cite{Derkachov:2001yn}, and at
present the SoV representation is known for a number of
models~\cite{Derkachov:2002tf,Derkachov:2003qb,BytskoTeschner06,Silantyev,MR3230255}.

In order to be sure that the spectral problems in the original and auxiliary Hilbert spaces are equivalent it is necessary to show that the
corresponding map, $\mathds H_\text{SoV}\mapsto \mathds H_\text{org}$, is unitary (or that the eigenfunctions form a complete set in
$\mathds H_\text{org}$). If $\dim \mathds H_\text{org} < \infty$ the problem can be solved, at least in principle,  by counting the
dimensions of the Hilbert spaces. For the models with infinite dimensional Hilbert space, such as the Toda chain, the noncompact
$\mathrm{SL}(2,\mathbb C)$ spin chain, etc.,  the task becomes more difficult. For the Toda chain unitarity was first established by using
harmonic analysis of Lie groups techniques~\cite{Semenov-Tian-Shansky1994,Wallach92}. However this method  is quite sophisticated and can
hardly be  generalized to more complicated cases. The rigorous proof of the unitarity of the SoV transform for the Toda chain based on the
use of natural objects for the QISM was given by Kozlowski~\cite{Kozlowski15}. This technique was later applied to the modular $XXZ$
magnet~\cite{DerkachovKozlowskiManashov19}. Later it was realized~\cite{DerkachovManashov17} that there exists a close relation between
$\mathrm{SL}(2,\mathbb R)$ symmetric spin chains and the multidimensional Mellin-Barnes integrals studied by
Gustafson~\cite{Gustafson92,Gustafson94} that allowed to greatly simplify  the proof of the unitarity of the SoV transform for
$\mathrm{SL}(2,\mathbb R)$ symmetric spin chains~\cite{DerkachovKozlowskiManashov21}.

In the present paper we apply this technique to the analysis of the noncompact spin chains with  the $\mathrm{SL}(2,\mathbb C)$ symmetry
group. Such models appear in the studies of the Regge limit of scattering amplitudes  in  gauge theories, in QCD in
particular~\cite{Lipatov:1993yb,Lipatov:1993qn,Faddeev:1994zg,Lipatov:2009nt,Bartels:2011nz}, see
also~\cite{DerkachovKazakovOlivucci19,DerkachovOlivucci20,DerkachovOlivucci21} for recent developments. The SoV representation for the
$\mathrm{SL}(2,\mathbb C)$ spin chains was constructed in~\cite{Derkachov:2001yn}. The generalization of Gustafson integrals relevant for
the $\mathrm{SL}(2,\mathbb C)$ spin chains was  obtained  recently in~\cite{Derkachov20}.  Based on these results, we present below a proof
of  unitarity of the SoV transform for a generic $\mathrm{SL}(2,\mathbb C)$ spin chain.

The paper is organized as follows: in sect.~\ref{preliminaries} we  recall  elements of the QISM relevant for further analysis. The
eigenfunctions of the elements of the monodromy matrix are constructed in sect.~\ref{sect:eigenfunctions}. In sect.~\ref{sect:etc} we
calculate several scalar products of the eigenfunctions and discuss their properties. Sect.~\ref{sect:SoV} contains the proof of unitarity
of the SoV transform.  Sect.~\ref{sect:summary} is reserved for a summary and several appendices contain a discussion of technical details.

%%%%%%%%%%%%%%%%%%%%%%%%%%%%%%%%%%%%%%%%%%%%%%%%%%%%%%%%%%%%%%%%%%%%%%%%%%%%%%%%%%%%%%%%%%%%%%%%%%%%%%%%%%%%%%%%%%%%%%%%%%%%%%%%%%%%%%%%%%%%%
\section{$\mathrm{SL}(2,\mathbb C)$ spin chains}\label{preliminaries}
%%%%%%%%%%%%%%%%%%%%%%%%%%%%%%%%%%%%%%%%%%%%%%%%%%%%%%%%%%%%%%%%%%%%%%%%%%%%%%%%%%%%%%%%%%%%%%%%%%%%%%%%%%%%%%%%%%%%%%%%%%%%%%%%%%%%%%%%%%%%%

 Spin chains are quantum mechanical systems whose dynamical variables are spin generators. We  consider  models with
spin generators belonging to the unitary continuous principle series representation, $\mathrm T^{(s_k,\bar s_k)}$, of the unimodular group
of complex two by two matrices. Namely, each site of the chain is equipped with two sets of generators, holomorphic ($S^\alpha$) and
anti-holomorphic ones ($\bar S^\alpha$),
%
%
\begin{subequations}
%
%
\begin{align}
%
S^-_k=-\partial_{z_k}, &&S^0_k=z_k\partial_{z_k} + s_k, && S^+_k=z_k^2\partial_{z_k} + 2 s_k z_k,\\
\bar S^-_k=-\partial_{\bar z_k}, &&S^0_k=\bar z_k \partial_{\bar z_k} + \bar s_k, && \bar S^+_k=\bar z_k^2
 \partial_{\bar z_k} + 2\bar s_k \bar z_k.
%
\end{align}
%
\end{subequations}
%
The generators $S^\alpha_k(\bar S^\alpha_k)$ satisfy the standard $\mathrm{sl}(2)$ commutation relations, while the generators at different
sites and holomorphic and anti-holomorphic generators commute, $[S^\alpha_k, \bar S^{\alpha'}_k]=0$. The parameters $s_k, \bar s_k$
specifying the representation take the form~\cite{MR0207913}
%
\begin{align}
%
s_k=\frac{1+n_k}{2}+i\rho_k, && \bar s_k=\frac{1- n_k}{2}+i\rho_k\,,
%
\end{align}
%
where $n_k$ is an integer or half-integer number and $\rho_k$ is real, so that
%
\begin{align}
%
s_k+\bar s_k^*=1 &&\text{ and }&& s_k-\bar s_k=n_k\in \mathbb Z/2.
%
\end{align}
%
The later condition comes from the requirement for the finite group transformations to be well defined while the former one guarantees the
unitary character of transformations and anti-hermiticity of
the generators, $(S_k^\alpha)^\dagger=-\bar S_k^\alpha$.


 The Hilbert space of the model is given by the direct product of the  Hilbert spaces at each node. For a chain of length $N$, $\mathds
H_N=\bigotimes_{k=1}^N \mathcal H_k$, where $\mathcal H_k = L_2(\mathbb C)$.
%

\vskip 3mm

In the QISM~\cite{MR549615,MR562799,MR671263,MR1239668} the dynamics of the model is determined by  a family of mutually commuting
operators.
 Namely, one defines the so-called $L$-operators,
%
\begin{align}
%
%
L_k(u)= u + i\begin{pmatrix}
%
S_k^0 & S_k^-\\
S_k^+ & -S_0^-
%
\end{pmatrix},
%
&&
\bar L_k(\bar u)= \bar u + i \begin{pmatrix}
%
\bar S_k^0 & \bar S_k^-\\
\bar S_k^+ & -\bar S_0^-
%
\end{pmatrix},
\end{align}
%
which are the basic building blocks in the QISM.  The complex variables $u,\bar u$ are  called spectral parameters. The next important
object -- a monodromy matrix -- is given by the product of $L$ operators
%
\begin{align}\label{monodromymatrix}
%
T_N(u) & = L_1(u+\xi_1)L_2(u+\xi_2)\ldots L_N(u+\xi_N), \notag\\
\bar T_N(\bar u) & = \bar L_1(\bar u+\bar \xi_1)\bar L_2(\bar u+\bar \xi_2)\ldots \bar L_N(\bar u+\bar \xi_N),
%
\end{align}
%
where  $\xi_k$, $\bar \xi_k$ are the so-called impurity parameters~\footnote{
 As it  can already be  noticed any formula in the holomorphic sector has its exact copy in the anti-holomorphic one. Therefore, from now on
we write explicitly only holomorphic  formulae tacitly implying its anti-holomorphic counterarts.
% only the holomorphic variant of equations will be written down explicitly and the anti-holomorphic counterparts will be tacitly implied. }.
}. The entries of the monodromy matrix,
%
\begin{align}
%
%
T_N(u)=\begin{pmatrix}
%
A_N(u) & B_N(u)\\
C_N(u) & D_N(u)
%
\end{pmatrix},
%
%
\end{align}
%
are polynomials in $u$  with the operator valued coefficients, e.g.
%
\begin{align}\label{ANBN}
%
A_N(u)= u^N + u^{N-1}\left(i S^0+\Xi\right) + \sum_{k=2}^{N} u^{N-k} a_k, && B_N(u)= u^{N-1} i S^-  +  \sum_{k=2}^{N} u^{N-k} b_k.
%
\end{align}
%
where  $\Xi=\sum_{k=1}^N \xi_k$ and  $S^0, S^-$ are the total generators:
%
\begin{align}\label{totalS}
%
S^\alpha  & = S^{\alpha}_1 + \cdots + S^\alpha_N.
%
\end{align}
%
The entries of the monodromy matrix form  commuting operator families~\cite{MR549615,MR1616371}
%
\begin{align}
%
[A_N(u),A_N(v)]=[B_N(u),B_N(v)]=[C_N(u),C_N(v)]=[D_N(u),D_N(v)] =0.
%
\end{align}
%
In particular, each entry commutes with the corresponding total generator, $S^\alpha$,
%
\begin{align}
%
[S^0,A_N(u)]=[S^0,D_N(u)]=0 &&\text{ and } &&[S^-,B_N(u)]=[S^+,C_N(u)]=0.
%
\end{align}
%
The same equations hold for the anti-holomorphic operators $\bar A_N, \bar B_N, \bar C_N, \bar D_N$ and, of course, the holomorphic and
anti-holomorphic operators commute. Moreover it can be checked that if the impurity parameters satisfy the constraint $\bar \xi_k=\xi_k^*$
for all $k$, the following relations between  holomorphic and anti-holomorphic operators hold
%
\begin{align}
%
(A_N(u))^\dagger = \bar A_N(u^*), && (B_N(u))^\dagger = \bar B_N(u^*),
%
\end{align}
%
etc. This ensures that the operators $a_k$ and $\bar a_k$ in the expansion of $A_N(u)$,~Eq.~\eqref{ANBN}, and  $\bar A_N(u)$, are adjoint
to each other $a_k^\dagger =\bar a_k$ ($b_k^\dagger =\bar b_k$ etc.)

The commutativity of the operators $A_N(u), B_N(u),C_N(u), D_N(u)$  implies that  the  following families of  self-adjoint operators
%
\begin{align*}
%
\mathfrak A_N=\{iS^0, i\bar S^0, a_k+\bar a_k, i(a_k-\bar a_k)\,\,, k=2,\ldots,N\},
\\
\mathfrak B_N=\{iS^-, i\bar S^-, b_k+\bar b_k, i(b_k-\bar b_k)\,\,, k=2,\ldots,N\},
%
\end{align*}
(and similarly for others) are commutative  and can be diagonalized  simultaneously~\footnote{The impurity parameters must also  satisfy
the condition $i(\xi_k -\bar \xi_k)=r_k$, where $r_k$ are (half)integers.}. The corresponding eigenfunctions provide a convenient basis --
Sklyanin's representation of Separated Variables (SoV) -- for the analysis of  spin chain models~\cite{MR1239668}.

The operators $B_N$ and $C_N$, ($A_N$ and $D_N$) are related to each other by the inversion transformation, see ref.~\cite{MR3230255} for
detail, so  it is sufficient to construct eigenfunctions for the operators $B_N$ and $A_N$.  The eigenfunctions of $B_N$ for the
homogeneous chain were constructed in ref.~\cite{Derkachov:2001yn} and later on for the operator $A_N$,~\cite{MR3230255}. Extending this
approach to the inhomogeneous case is rather straightforward.





%%%%%%%%%%%%%%%%%%%%%%%%%%%%%%%%%%%%%%%%%%%%%%%%%%%%%%%%%%%%%%%%%%%%%%%%%%%%%%%%%%%%%%%%%%%%%%%%%%%%%%%%%%%%%%%%%%%%%%%%%%%%%%%%%%%%%%%%%%%%%
\section{Eigenfunctions}\label{sect:eigenfunctions}
%%%%%%%%%%%%%%%%%%%%%%%%%%%%%%%%%%%%%%%%%%%%%%%%%%%%%%%%%%%%%%%%%%%%%%%%%%%%%%%%%%%%%%%%%%%%%%%%%%%%%%%%%%%%%%%%%%%%%%%%%%%%%%%%%%%%%%%%%%%%%

In this section we present explicit expressions for the eigenfunctions of the operators  $B_N$ and $A_N$  for a generic inhomogeneous spin
chain with impurities. We start with the  operator $B_N$ where the construction follows the lines of ref.~\cite{Derkachov:2001yn} with
minimal modifications.

%%%%%%%%%%%%%%%%%%%%%%%%%%%%%%%%%%%%%%%%%%%%%%%%%%%%%%%%%%%%%%%%%%%%%%%%%%%%%%%%%%%%%%%%%%%%%%%%%%%%%%%%%%%%%%%%%%%%%%%%%%%%%%%%%%%%%%%%%%%%%
\subsection{$B_N$ operator}\label{subs:BN}
%%%%%%%%%%%%%%%%%%%%%%%%%%%%%%%%%%%%%%%%%%%%%%%%%%%%%%%%%%%%%%%%%%%%%%%%%%%%%%%%%%%%%%%%%%%%%%%%%%%%%%%%%%%%%%%%%%%%%%%%%%%%%%%%%%%%%%%%%%%%%

Let  $\Lambda_n$  be an integral (layer) operator which maps functions of $n-1$ variables into functions of $n$ variables and depends on
the spectral parameters $x,\bar x$ and the complex vectors $\gamma,\bar \gamma$ of dimension  $2n-2$
%
\begin{align}\label{LNoperator}
%
[\Lambda_n(x|\gamma)f](z_1,\ldots,z_n) &= \idotsint \Lambda_n(x|\gamma)(z_1,\ldots,z_n|w_1,\ldots,w_{n-1})
f(w_1,\ldots,w_{n-1})\prod_{k=1}^{n-1} d^2 w_k. &
%
\end{align}
%
The kernel is given by the following expression
%
\begin{align}\label{LNkernel}
%
\Lambda_n(x|\gamma)(z_1,\ldots,z_n|w_1,\ldots,w_{n-1}) &=\prod_{k=1}^{n-1} D_{\gamma_{2k-1}-ix}(z_{k}-w_{k})
D_{\gamma_{2k}+ix}(z_{k+1}-w_{k}), &
%
\end{align}
where the function $D_\alpha(z)$ (propagator) is defined as follows
%
\begin{align}\label{Dalpha}
%
D_{\alpha}(z)\equiv D_{\alpha,\bar\alpha}(z,\bar z) = {z^{-\alpha} \bar z^{-\bar\alpha}}.
%
\end{align}
%
We will assume that the  indices $\alpha,\bar\alpha$ satisfy the condition $[\alpha] \equiv \alpha-\bar\alpha\in \mathbb Z$ so that the
propagator is a single-valued function on the complex plane. It implies that the parameters $\gamma_k$  and $x$  have the form
%
\begin{align}\label{gammaxform}
%
%
\gamma_k=\frac12+\frac{r_k}{2}+i\sigma_k, && \bar \gamma_k=\frac12- \frac{r_k}{2}+i\sigma_k\,, &&
x=\frac{im}2 + \nu, && \bar x=-\frac{im}2 + \nu.
%
\end{align}
%
The numbers $\{m, r_1,\ldots, r_{2N-2}\}$  are either integer  or half-integer and depending on this we call the corresponding variables
integer (half-integer).
 The  continuous parameters  $\sigma_k$ and $\nu$  are subject to the constraints
%
\begin{align}
%
\text{Im}(\sigma_{2k+1} - \nu)> -\nicefrac 12 &&\text{and}&& \text{Im}(\sigma_{2k} + \nu)> - \nicefrac12
%
\end{align}
%
which guarantee the convergence of the integral~\eqref{LNoperator} for  a smooth function $f$ with  finite support.
% can take complex values which are bounded by the  requirement of convergence
%of the integral~\eqref{LNoperator}. For a smooth function $f$ with  finite support  this requirement results in the following constraint:
In the case we are most interested in, $\gamma_k + \bar\gamma_k=1$, the parameters $\sigma_k\in \mathbb{R}$, and the variable $\nu$  lies
in the strip $-\nicefrac12<\mathrm{Im}\nu<\nicefrac12$.

\vskip 3mm

The operators $\Lambda_n$ possess two important properties:
%
\begin{enumerate}[label=(\roman*)]
%
\item
%
%
 Let  $\rho$  be a map which takes  $M$-dimensional vectors
%
\begin{align}
%
\gamma=(\gamma_1,\ldots,\gamma_{M}), &&
\bar \gamma=(\bar \gamma_1,\ldots\bar \gamma_{M})%
\end{align}
%
to   vectors  of dimension $M-2$ as follows
%
\begin{align}
%
\rho \gamma=(\gamma'_2,\gamma'_3,\ldots,\gamma'_{M-1}) &&
\rho \bar \gamma=(\bar \gamma'_2,\bar \gamma'_3,,\ldots,\bar \gamma'_{M-1}),
%
\end{align}
%
where  $a' \equiv 1-a$. It can be shown that the operators $\Lambda_n$ and $\Lambda_{n-1}$ obey the following exchange relation
%
\begin{align}\label{Bexchange}
%
\Lambda_n(u|\gamma)\,\Lambda_{n-1}(v|\rho \gamma) & = \omega_n(\gamma, u,v)\, \Lambda_n(v|\gamma)\,\Lambda_{n-1}(u|\rho \gamma).
%
\end{align}
%
Here $\gamma (\bar\gamma)$ is $2n-2$ dimensional vector and the factor $\omega_n$ is given by the following expression

%
\begin{align}\label{omegan}
%
\omega_n(\gamma, u,v) & = \prod_{m=1}^{n-1}
\boldsymbol \Gamma\left[
\frac{\gamma_{2m-1}-iv,\bar\gamma_{2m}+i\bar v}
{\gamma_{2m-1}-iu,\bar\gamma_{2m}+i\bar u}\right]
                                                  =
%
\prod_{m=1}^{n-1} \boldsymbol \Gamma\left[
\frac{\bar \gamma_{2m-1}-i\bar v,\gamma_{2m}+i v}
{\bar \gamma_{2m-1}-i\bar u,\gamma_{2m}+i u}\right],
\end{align}
%
%
where
%
\begin{align}
%
\boldsymbol \Gamma\left[\frac{a_1,a_2,\ldots,a_n}{b_1,b_2,\ldots,b_m}\right]\equiv
\frac{\prod_{k=1}^n\boldsymbol\Gamma[a_k]}
{\boldsymbol\prod_{k=1}^m\boldsymbol\Gamma[b_k]}
%
\end{align}
%
and  $\boldsymbol \Gamma$ is the Gamma function of the  complex field $\mathbb C$~\cite{MR2125927}
%
\begin{align}
%
\boldsymbol\Gamma[u] \equiv \boldsymbol\Gamma[u,\bar u] =
{\Gamma(u)}/{\Gamma(1-\bar u)}.
%
\end{align}
%
%%
The relation~\eqref{Bexchange} is a direct consequence of the exchange relation for the propagators, see~\eqref{exchange-rel}. Its proof
is exactly the same as for the homogeneous spin chain. For more details see refs.~\cite{Derkachov:2001yn,MR3230255}.

\item Let us choose  the vector $\gamma$ as follows
%
\begin{align}\label{gammaN}
%
\gamma =(s_1-i\xi_1,s_2+i\xi_2,s_2-i\xi_2,\ldots,s_{N-1} + i\xi_{N-1},s_{N-1} - i\xi_{N-1},s_N + i\xi_N),
\notag\\
%
\bar\gamma =(\bar s_1-i\bar\xi_1,\bar s_2+i\bar\xi_2,\bar s_2-i\bar\xi_2,\ldots,\bar s_{N-1} + i\bar\xi_{N-1},
\bar s_{N-1} - i\bar\xi_{N-1},\bar s_N + i\bar \xi_N),
\end{align}
%
where $s_k$ and $\xi_k$ are the spins and impurity parameters of the spin chain, respectively. For such a choice of the vector  $\gamma$
the operator $B_N(x)$ annihilates $\Lambda_N(x|\gamma)$~\cite{Derkachov:1999pz,Derkachov:2001yn}
%
\begin{align}\label{Beq}
%
B_N(x) \Lambda_N(x|\gamma)=0.
%
\end{align}

\end{enumerate}



\vskip 3mm



Let us define a function
%
\begin{align}
\label{PsiBdefinition}
%%
\Psi^{(N)}_{p,x}(z)
\equiv \Psi^{(N)}_{p,x_1,\ldots,x_{N-1}}  (z_1,\ldots, z_N)
& = \pi^{-N^2/2}  |p|^{N-1}
\int d^2 z    U_{x_1,\ldots,x_{N-1}}(z_1,\ldots,z_N| z)\,
e^{i (p z +\bar p \bar z)}\,,
%%
\end{align}
%
where the kernel  $ U_{x_1, \ldots, x_{N-1} }$ is given by the product of the layer operators,
%
%
\begin{align}
%
U_{x_1,\ldots,x_{N-1}} &=
 \varpi (x|\gamma)
\Lambda_N (x_1|\gamma)
        \Lambda_{N-1}(x_2|\rho\gamma)
                    \Lambda_{N-2}(x_3|\rho^2\gamma)
                            \ldots\Lambda_2(x_{N-1}|\rho^{N-2}\gamma),
%
%
\end{align}
%
and $\gamma $ is given by~Eq.~\eqref{gammaN}.
%
%
Equation~\eqref{Bexchange} guarantees that  $U_{x_1,\ldots,x_{N-1}}\sim U_{x_{i_1},\ldots,x_{i_{N-1}}}$ for any permutation of
$x_1,\ldots, x_{N-1}$. The kernel $U_x$ becomes totally symmetric for the following choice of the prefactor  $\varpi (x|\gamma)$:
%
\begin{align}\label{varphifactor}
%
\varpi({x}|\gamma)  &=\varpi(x_1,\ldots,x_{N-1}|\gamma)=
\prod_{m=1}^{N-1} \prod_{k=1}^{m}\varpi(x_k|\rho^{m-1}\gamma),
\end{align}
%
where
\begin{align}
%
\varpi(x|\gamma)=\varpi(x|\gamma_1,\ldots,\gamma_{2n}) & =
    \prod_{m=1}^{n} \boldsymbol\Gamma[\gamma_{2m-1}-ix,\bar\gamma_{2m} + i\bar x].
\end{align}
%
Thus the function $ \Psi^{(N)}_{p,x_1,\ldots,x_{N-1}} $ is  a symmetric function of the variables $x_1,\ldots,x_{N-1}$.
%
 Together with Eq.~\eqref{Beq} it implies that
%
\begin{align}
%
B_N(x_k)\Psi^{(N)}_{p,x_1,\ldots,x_{N-1}} =0 && \text{for}&& k=1,\ldots, N-1.
%
\end{align}
%
Invariance of the kernel  $U_{x_1m\ldots,x_{N-1}}(z_1,\ldots, z_N|z)$ is under shifts
%
\begin{align}
%
U_{x_1\ldots,x_{N-1}}(z_1+w,\ldots, z_N+w|z+w) =U_{x_1\ldots,x_{N-1}}(z_1,\ldots, z_N|z)
%
\end{align}
%
results in
%
\begin{align}\label{Seq}
%
iS^{-}\, \Psi^{(N)}_{p,x_1,\ldots,x_{N-1}}  = p \, \Psi^{(N)}_{p,x_1,\ldots,x_{N-1}},  &&
i\bar S^{-}\, \Psi^{(N)}_{p,x_1,\ldots,x_{N-1}}  = \bar p \, \Psi^{(N)}_{p,x_1,\ldots,x_{N-1}}.
\end{align}
%
It follows then from Eqs.~\eqref{ANBN}, \eqref{Beq} and \eqref{Seq} that~\footnote{ We recall that the variables $x_k, \bar x_k$,
$k=1,\ldots, N-1$ take the form
%
%
$x_k={in_k}/{2}+\nu_k, $ $ \bar x_k=-{in_k}/{2}+\nu_k$,
%
%
where, depending on the spin and impurities parameters, all $n_k$ are either integer or half-integer numbers. }
%
%
\begin{align}
%
B_N(u)\Psi^{(N)}_{p,x}(z)=p\prod_{k=1}^{N-1}(u-x_k)\Psi^{(N)}_{p,x}(z), &&
\bar B_N(\bar u)\Psi^{(N)}_{p,x}(z)= \bar p\prod_{k=1}^{N-1}(\bar u-\bar x_k)\Psi^{(N)}_{p,x}(z).
%
\end{align}
%

\vskip 3mm

For  $N=1$ the functions $\Psi_{p}^{(1)}(z,\bar z) = \pi^{-1/2}\,e^{i(p z + \bar p\bar z)}$ form the complete orthonormal system in
$\mathds H_1=L_2(\mathbb C)$. The aim of this paper is to extend this statement  to  $N>1$. Namely, we will show in sect.~\ref{sect:SoV}
that if the spins and impurities parameters of the spin chain obey the ``unitarity'' condition,
%
\begin{align}\label{gammaunitarity}
%
\gamma_k+\bar\gamma_k^*=1,
%
\end{align}
for all $k$ (  $\gamma_k$ has the form~\eqref{gammaxform} with $\sigma_k\in\mathbb R$ ) then the set of functions $\{\Psi^{(N)}_{p,x},\,\,
x_k=\bar x_k^* \,\,(\nu_k\in \mathbb R),\, k=1,\ldots, N-1\}$ is complete in $\mathds H_N=(\bigotimes L_2(\mathbb C))^N$.

Note that  the functions $\Psi_{p,x}^{(N)}$ are well defined for the complex parameters $\nu_k$ in the vicinity of the real line. For
further analysis, it will be useful to consider  regularized functions, $\Psi_{p,x}^{(N),\epsilon}$, by relaxing the last of  the
conditions~\eqref{gammaunitarity} to $\gamma_{2N-2}+\bar\gamma_{2N-2}^*=1+2\epsilon$. This can be achieved by  shifting the impurity
parameter $\xi_N\to\xi_N - i\epsilon$~\footnote{ Of course, one also can regularize the function by shifting the parameter $\gamma_1$
instead of $\gamma_{2N-2}$, $\gamma_{1}+\bar\gamma_{1}^*=1+2\epsilon$. }, i.e.
%
\begin{align}\label{Psiepsilon}
%
\Psi_{p,x}^{(N),\epsilon}(z) \overset{\text{def}}{=} \Psi_{p,x}^{(N)}(z)\Big|_{\xi_N\to\xi_N - i\epsilon}\,.
%
\end{align}
%




%%%%%%%%%%%%%%%%%%%%%%%%%%%%%%%%%%%%%%%%%%%%%%%%%%%%%%%%%%%%%%%%%%%%%%%%%%%%%%%%%%%%%%%%%%%%%%%%%%%%%%%%%%%%%%%%%%%%%%%%%%%%%%%%%%%%%%%%%%%%%
\subsection{$A_N$ operator}\label{subs:AN}
%%%%%%%%%%%%%%%%%%%%%%%%%%%%%%%%%%%%%%%%%%%%%%%%%%%%%%%%%%%%%%%%%%%%%%%%%%%%%%%%%%%%%%%%%%%%%%%%%%%%%%%%%%%%%%%%%%%%%%%%%%%%%%%%%%%%%%%%%%%%%
Construction of the eigenfunctions of the operator $A_N$   follows  the scheme described in the previous subsection. We  define a layer
operator $ \Lambda^\prime_n$ which maps functions of $n-1$ variables into functions of $n$ variables
%
\begin{align}\label{LNoperatorA}
%
[\Lambda^\prime_n(x|\gamma)f](z_1,\ldots,z_n) &= \idotsint \Lambda^\prime_n(x|\gamma)(z_1,\ldots,z_n|w_1,\ldots,w_{n-1})
f(w_1,\ldots,w_{n-1})\prod_{k=1}^{n-1} d^2 w_k,
%
\end{align}
%
where the kernel is given by the following expression
%
\begin{align}\label{LNprimekernel}
%
\Lambda^\prime_n(x|\gamma)(z_1,\ldots,z_n|w_1,\ldots,w_{n-1}) =  D_{\gamma_{2N-1}-ix}(z_{N})\,
\prod_{k=1}^{n-1} D_{\gamma_{2k-1}-ix}(z_{k}-w_{k}) D_{\gamma_{2k}+ix}(z_{k+1}-w_{k}).
%
\end{align}
The layer operator $ \Lambda^\prime_n$ depends on the spectral parameters $x(\bar x)$ and the vector $\gamma(\bar \gamma)$ of dimension
$2n-1$ which have the form  Eq.~\eqref{gammaxform}.

These operators satisfy the exchange relation
%
\begin{align}
%
\Lambda^\prime_n(u|\gamma)\Lambda^\prime_{n-1}(v|\rho \gamma) & = \omega_n(\gamma, u,v)
\Lambda^\prime_n(v|\gamma)\Lambda^\prime_{n-1}(u|\rho \gamma),
%
\end{align}
and the factor $\omega_n$ is defined in Eq.~\eqref{omegan}.

Let $\Phi^{(N)}_{x}(z)$ be the following function
%
\begin{align}
%
\Phi^{(N)}_{x}(z) & \equiv   \Phi^{(N)}_{x_1,\ldots,x_{N}}(z_1,\ldots, z_N)
\notag\\
& = \pi^{-N^2/2}
\varpi (x|\gamma)\left[
\Lambda^\prime_N (x_1|\gamma)
        \Lambda^\prime_{N-1}(x_2|\rho\gamma)
                            \ldots
                                    \Lambda^\prime_1(x_{N}|\rho^{N-1}\gamma)\right](z_1,\ldots,z_N)\,,
%
%
\end{align}
where $\gamma$ is $2N-1$ dimensional vector and the   prefactor $\varpi$ is given by  Eq.~\eqref{varphifactor}. For such a choice of
$\varpi$ the function $\Phi^{(N)}_{x}$ is a symmetric function of the variables $x_1,\ldots,x_N$.

It can be shown that  the operator $A_N(x)$ annihilates the layer operator $\Lambda^\prime_N(x|\gamma)$,
%
\begin{align}\label{Aeq}
%
A_N(x) \Lambda^\prime_N(x|\gamma)=0,
%
\end{align}
%
for the following choice of the vector $\gamma$
\begin{align}\label{gammaNA}
%
\gamma =(s_1-i\xi_1,s_2+i\xi_2,s_2-i\xi_2,\ldots,
s_N + i\xi_N, s_N - i\xi_N),
\notag\\
%
\bar\gamma =(\bar s_1-i\bar\xi_1,\bar s_2+i\bar\xi_2,\bar s_2-i\bar\xi_2,\ldots,
\bar s_N + i\bar \xi_N,\bar s_N - i\bar \xi_N).
\end{align}
%
Taking into account polynomiality of  $A_N(u)$, see Eq.~\eqref{ANBN}, one obtains
%
%
\begin{align}
%
A_N(u)\Phi^{(N)}_{x}(z)=\prod_{k=1}^{N}(u-x_k)\Phi^{(N)}_{x}(z), &&
\bar A_N(\bar u)\Phi^{(N)}_{x}(z)= \prod_{k=1}^{N}(\bar u-\bar x_k)\Phi^{(N)}_{x}(z).
%
\end{align}
%
Again, the variables $x_k, \bar x_k$ are integers (half-integers) for all $k$. We will show that these functions, $\{\Phi^{(N)}_{x}(z),  \
x_k=\bar x_k^*, \ k=1,\ldots, N \}$,  form a complete set in the Hilbert space $\mathds H_N$.




%%%%%%%%%%%%%%%%%%%%%%%%%%%%%%%%%%%%%%%%%%%%%%%%%%%%%%%%%%%%%%%%%%%%%%%%%%%%%%%%%%%%%%%%%%%%%%%%%%%%%%%%%%%%%%%%%%%%%%%%%%%%%%%%%%%%%%%%%%%%%
%%%%%%%%%%%%%%%%%%%%%%%%%%%%%%%%%%%%%%%%%%%%%%%%%%%%%%%%%%%%%%%%%%%%%%%%%%%%%%%%%%%%%%%%%%%%%%%%%%%%%%%%%%%%%%%%%%%%%%%%%%%%%%%%%%%%%%%%%%%%%
\section{Scalar products, momentum representation, etc.}\label{sect:etc}
%%%%%%%%%%%%%%%%%%%%%%%%%%%%%%%%%%%%%%%%%%%%%%%%%%%%%%%%%%%%%%%%%%%%%%%%%%%%%%%%%%%%%%%%%%%%%%%%%%%%%%%%%%%%%%%%%%%%%%%%%%%%%%%%%%%%%%%%%%%%%
%%%%%%%%%%%%%%%%%%%%%%%%%%%%%%%%%%%%%%%%%%%%%%%%%%%%%%%%%%%%%%%%%%%%%%%%%%%%%%%%%%%%%%%%%%%%%%%%%%%%%%%%%%%%%%%%%%%%%%%%%%%%%%%%%%%%%%%%%%%%%

The functions constructed in the previous section are given by multi-dimensional integrals. In this section we show that these integrals
converge for the parameters $\nu_k$ in the vicinity of real axis. To this end it will be quite helpful, as was advocated in
ref.~\cite{Derkachov:2001yn}, to visualize  the integrals  as  Feynman diagrams. The examples for $N=3$ are shown in
Fig.~\ref{diag:examples}.
%
%
\begin{figure}[t]
%
%
\begin{center}
%
\includegraphics[width=0.72\linewidth]{SLN}
%
\end{center}
%
\caption{The diagrammatic representation for the function $\Psi$ (left( and $\Phi$ (right) for $N=3$.
The arrow
from $z$ to $w$ with an index $\alpha$ stands for the propagator $D_{\alpha}(z-w)$, Eq.~\eqref{Dalpha}.
}
\label{diag:examples}
%
\end{figure}
%
%
It will be convenient to convert  diagrams (functions)  to  momentum space  %%
\begin{align}
%
\Psi(z_1,\ldots, z_N)={\pi^{-N}}\idotsint \widetilde \Psi(p_1,\ldots,p_N) e^{i \sum_{k=1}^N (p_kz_k + \bar p_k \bar z_k)}
\, d^2p_1\ldots d^2p_N\,.
%
\end{align}
%
In momentum space the function $\Psi^{(N),\epsilon}_{p, x}$, Eq.~\eqref{Psiepsilon}, takes the form
%
\begin{align}
%
\widetilde \Psi^{(N),\epsilon}_{p, x}(p_1,\ldots, p_N) & = \delta^{(2)}\left(p-\sum_{k=1}^Np_k\right)
\Psi^{(N),\epsilon}_{x}(p_1,\ldots, p_N).
%
\end{align}
%
Let us remark here that the ''$\epsilon$" regularization  is reduced to a multiplication by the factor $(p_N \bar p_N)^\epsilon$
%
\begin{align}\label{Psiespilon}
%
\Psi^{(N),\epsilon}_{x}(p_1,\ldots, p_N) & =(p_N \bar p_N)^\epsilon \,\Psi^{(N)}_{x}(p_1,\ldots, p_N)\,.
%
\end{align}
%
%
The  function $\Psi^{(N),\epsilon}_{x}$ can be read from the Feynman diagram in Fig.\ref{diag:examples} as follows
%
\begin{align}\label{PsiNdefinition}
%
\Psi^{(N),\epsilon}_{x}(p_1,\ldots,p_N) =\idotsint \mathcal J_x^{\epsilon}(\{p_k\},\{\ell_{ij}\})\, \prod_{1\leq j\leq i\leq N-2} d^2\ell_{ij},
%
\end{align}
%
with the integrand $\mathcal J_x^{\epsilon}(\{p_k\},\{\ell_{ij}\})$  given by the product of the propagators, $D_\alpha(k)$. Up to a
momentum independent factor
%
\begin{align}
%
\mathcal J_x^{\epsilon}(\{p_k\},\{\ell_{ij}\})\simeq \prod_{k=1}^{N-1}\prod_{j=1}^k D_{\alpha_{kj}}(\ell_{k,j}-\ell_{k-1,j-1})
D_{\beta_{kj}}(\ell_{k-1,j}-\ell_{k,j})\,,
%
\end{align}
%
where $\ell_{k0}\equiv0$, $\ell_{k-1,k}\equiv p$ and $\ell_{N-1,j}=(p_1+ \ldots + p_j)$. The indices $\alpha_{kj}$, $\beta_{kj}$ take the
following values
%
\begin{align}
%
\alpha_{kj} =\gamma_{2j-1}^{(N-k)}+ix_{N-k}, && \beta_{kj} =\gamma_{2j}^{(N-k)}-ix_{N-k},
%
\end{align}
%
where we introduced the notations:
%
\begin{align}
%
 a^{(1)}=a^\prime =1-a &&\text{ and  } && a^{(k+1)}=1-a^{(k)}.
%
\end{align}
%
In many cases,  Feynman diagrams can be evaluated diagrammatically. In particular, the  computation of  diagrams for the scalar product of
$\Psi$ ($\Phi$) functions is based on the successive application of the exchange relation~\eqref{exchange-rel} to the diagram.



Let us consider the scalar product of two functions $\Psi^{(N),\epsilon}_{p, x}$ and $\Psi^{(N),\epsilon'}_{q, y}$
%
\begin{align}\label{BBproduct}
%
\Big(\Psi^{(N),\epsilon'}_{q, y},\Psi^{(N),\epsilon}_{p, x}\Big) & =  \pi \delta^2(p-q) (p\bar p)^{\epsilon+\epsilon'}
I^{\epsilon,\epsilon'}(x,y),
\end{align}
where
\begin{align}\label{Ie}
I^{\epsilon,\epsilon'}(x,y) & = \frac1\pi (p\bar p)^{-\epsilon -\epsilon'}\idotsint \, \delta^{(2)}\left(p-\sum_k p_k\right)
 \Psi^{(N),\epsilon}_{x}(\vec{p}) \left(\Psi^{(N),\epsilon'}_{y}(\vec{p}) \right)^\dagger \prod_{j=1}^N d^2p_j.
%
\end{align}
%
The function   $I^{\epsilon,\epsilon'}_p(x,y)$ is given by the Feynman diagram shown in Fig.~\ref{diag:scalarproducts} (left panel), which
is a multidimensional integral
%
\begin{align}\label{intI}
%
I^{\epsilon,\epsilon'}_p(x,y) & = \idotsint\mathcal I^{\epsilon\epsilon'}_{x,y}(p,\{\ell_{pr}\}|\gamma)\prod_{p,r=1}^{N-1} d^2\ell_{pr}
%
\end{align}
%
with the integrand given by the product of the propagators. The diagram can be evaluated in a closed form by  successively applying  the
exchange relation~\eqref{exchange-rel}, that is equivalent to calculating  the loop integrals in a certain order. The answer takes the form
\begin{align}\label{twoform}
%
I^{\epsilon,\epsilon'}(x,y) & = \mathrm C_N(\gamma) \,
     \boldsymbol\Gamma\left[\frac{\epsilon+\epsilon' + i X -i\bar Y^*}
     {\epsilon+\epsilon'}\right]
        \frac
        {
       \prod_{k,j=1}^{N-1} \boldsymbol\Gamma[i(y_k^*-\bar x_j)]}{
        \prod_{k=1}^{N-1} \bar \phi_N(\bar x_k) (\phi_N(y_k))^*}
%
\notag\\
&=\mathrm C_N(\gamma) \,
 \boldsymbol\Gamma\left[
     \frac{\epsilon+\epsilon' + i\bar X -i Y^*}{\epsilon+\epsilon'}
     \right]
        \frac{
       \prod_{k,j=1}^{N-1} \boldsymbol\Gamma[i(\bar y_k^* - x_j)]}
        {\prod_{k=1}^{N-1}  \phi_N( x_k) (\bar \phi_N(\bar y_k))^*},
\end{align}
%
where $X=\sum_{k=1}^{N-1} x_k$, $Y=\sum_{k=1}^{N-1} Y_k$ and
%
\begin{align}
 \phi_N( x) & =\boldsymbol\Gamma\left[\gamma_{2N-3} -i x,\gamma^{(1)}_{2N-4} -i x, \gamma_{2N-5} -i x,\ldots, \gamma_{N}^{(N-3)}-i x\right]\,,
%
\notag\\
%
\bar \phi_N(\bar x) & =\boldsymbol\Gamma\left[
\bar \gamma_{2N-3} -i\bar x, \bar \gamma^{(1)}_{2N-4} -i\bar x, \bar \gamma_{2N-5} -i\bar x,\ldots,
\bar\gamma_{N}^{(N-3)}-i\bar x\right]\,.
%
\end{align}
%
For the sign factor $\mathrm C_N(\gamma)$  we get :
\begin{align}\label{BBCN}
%
\mathrm C_{N}(\gamma_1,\gamma_2,\ldots,\gamma_{2N-2})
&=
%
\begin{cases}
%
1 & \text{ odd } N
%
\\
%
(-1)^{\sum_{k=1}^{N-3}\left[\gamma_{2N-2-k}^{(k-1)}-\gamma_{N}^{(N-3)}\right]}
& \text{ even } N.
\end{cases}
%%
\end{align}
%
Here $[a]\equiv a-\bar a$. Details of the calculation can be found in~\ref{app:scalarproduct}.



\vskip 3mm

 Let  us show now  that  integrations in~\eqref{intI} can be done  in an arbitrary order. The integrand in~\eqref{intI},
 $\mathcal I^{\epsilon\epsilon'}_{x,y}(p,\{\ell_{pr}\}|\gamma)$, is given by
the product of the propagators $D_{\alpha}(k)$, with each index being of the form  $\alpha=  \frac12 +\frac{n}2 + i\sigma$, momentum $k$
being a linear combination of  loop momenta, $\ell_{ij}$, and the external momentum $p$. Since
%
\begin{align}
%
\big | D_\alpha(k)\big | =\big | k^{-\alpha} \bar k^{-\bar \alpha}\big | = |k|^{-1 +2\, \text{Im}\sigma}=D_{\nicefrac12-\text{Im}\sigma}(k)
%
\end{align}
%
then for the  parameters $\gamma$ satisfying the  unitarity condition~\eqref{gammaunitarity}, and $x_k,y_k$ having the form
%
\begin{align}\label{xydef}
%
x_k={in_k}/{2}+\nu_k, &&  y_k={im_k}/{2}+\mu_k,
%
\end{align}
%
one obtains for the modulus of the integrand
%
\begin{align}
%
\left|\mathcal I^{\epsilon\epsilon'}_{x,y}(p,\{\ell_{pr}\}|\gamma)\right| =
\mathcal I^{\epsilon\epsilon'}_{\underline{x},\underline{y}}(p,\{\ell_{pr}\}|\underline{\gamma})>0\,,
%
\end{align}
%
where the underlined variables are: $\underline{\gamma}=(\nicefrac12,\ldots,\nicefrac12)$,
%
\begin{align*}
%
(\underline{x})_k=\mathrm{Im}(\nu_k)=\epsilon_k,  && (\underline{y})_k=\mathrm{Im}(\mu_k)=\epsilon'_k.
%
\end{align*}
%
Thus the integral  of $\vert\mathcal I^{\epsilon\epsilon'}_{x,y}(p,\{\ell_{pr}\}|\gamma)\vert$ is a particular case of the integral
 \eqref{intI} which
was calculated by performing loop integrations in a certain order. Since all integrals  converge under the conditions
%
\begin{align}
%
\epsilon_{kj}\equiv \epsilon_k+\epsilon'_j >0&&\text{for }&& k,j=1,\ldots, N-1&&\text{and} && \epsilon+\epsilon'
>\sum_{k=1}^{N-1} (\epsilon_k+\epsilon'_k),
%
\end{align}
%
by Fubini theorem, the integral~\eqref{intI} exists and the integrations can be done in an arbitrary order.

The following statements can immediately be deduced from this result:
%
\begin{itemize}
%
%
\item  For any bounded function $\varphi(p,x)$ with a finite support the function
%
\begin{align}\label{Psixepsilon}
%
\Psi^\epsilon_\varphi = \idotsint \varphi(p,x)\, \Psi^{(N),\epsilon}_{p, x^\epsilon}\, d^2p\, \mathcal D x_1\ldots \mathcal D x_{N-1}\,,
%
\end{align}
%
where $x^\epsilon =(x_1 + i\epsilon_1,\ldots, x_{N-1}+ i\epsilon_{N-1})$,   $x_k=in_k/2+\nu_k$,    $\epsilon_k>0$,
   $\epsilon>\sum_{k=1}^{N-1} \epsilon_k$ and
%
%
$$
 \int \mathcal D x_k \equiv \sum_{n_k=-\infty}^\infty \int_{-\infty}^{\infty} d\nu_k,
$$
%
%
belongs to the Hilbert space $\mathds H_N$, $\|\Psi^\epsilon_\varphi\|^2 <\infty$, for sufficiently small $\epsilon$.

\item It follows from the finiteness of the integral $I^{\epsilon,\epsilon'}_p(x,y)$, Eq.~\eqref{Ie}, that  the function
    $\Psi_{x}^{(N),\epsilon}(\vec p)$, Eq.~\eqref{PsiNdefinition}, exists almost for all $\vec{p}$ for the separated variables $x_k$
    close to the real axis:
        $$
            \text{Im}\,\nu_k = \frac12\text{Im}(x_k+\bar x_k)   \sim 0 \qquad \text{for all $k$}
        $$
        and $\Psi_{x}^{(N),\epsilon}(\vec p)$ is a continuous function of $\nu_k$ in this region. Indeed, let us fix $m<N$ and put
        $u_m=\text{Re}\nu_m$ and $v_m=\text{Im}\nu_m $, $ |v_m| <\delta$.  One  gets the following estimate for the
        integrand~\eqref{intI}
%
\begin{align}\label{estimate1}
%
\left|\mathcal J_x^{\epsilon}(\{p_k\},\{\ell_{ij}\})\right|< \left|\mathcal J_{x_+}^{\epsilon}(\{p_k\},\{\ell_{ij}\})\right|
+\left|\mathcal J_{x_-}^{\epsilon}(\{p_k\},\{\ell_{ij}\})\right|,
%
\end{align}
%
where $x_\pm$ are defined as follows: for $k\neq m$  $(x_\pm)_k=x_k$ and for $k=m$   $(x_\pm)_m=u_m \pm i\delta$. The integrals of the
functions on the rhs of~\eqref{estimate1} are finite for sufficiently small $\delta$. It follows then from the Lebesgue theorem that the
function $\Psi_{x}^{(N),\epsilon}(\vec p)$ is continuous in the variable $\nu_m$~\footnote{Since the integrand is analytic function of
$\nu_k$ $\Psi_{x}^{(N),\epsilon}(\vec p)$ is an analytic function of $\nu_k$ in the vicinity of the real axis.}.
\end{itemize}

\vskip 5mm


The scalar product of the functions $\Psi^{(N)}_{p,y}$  and  $\Phi^{(N)}_{x}$ constructed in~sect.~\ref{subs:AN} can be calculated in a
similar way. Note that there is no need to introduce ``$\epsilon$" regulator here. The corresponding integral is absolutely convergent when
$\text{Im} (\nu_k +\mu_j)>0$ for all $k, j$, ( $x_k, y_j$ given by~\eqref{xydef} ). %%
%
{The  scalar product takes the form}
%
\begin{align}\label{ABproduct}
%
\Big(\Psi^{(N)}_{p,y}|\Phi^{(N)}_{x} \Big) & { = }\mathrm C^{AB}_N(\gamma)\,|p|^{N-1}
 \,(-ip)^{-A_N - iX} (i\bar p)^{-\bar A_N - i\bar X}
\frac{
\prod_{k=1}^N \prod_{j=1}^{N-1} \boldsymbol\Gamma[i( \bar y_j^* - x_k)]}
{\left(\prod_{j=1}^{N}\vartheta_N(x_j)\right) \left(\prod_{j=1}^{N-1}\bar\vartheta_N(\bar y_j)\right)^\dagger
}
,
%
\end{align}
%%
where
%
\begin{align}
%
\vartheta_N(x)=
\prod_{k=1}^{N}
\boldsymbol \Gamma\left[\gamma^{(k-1)}_{2N-k}-i x_j\right]\,, && \bar \vartheta_N(\bar x)=
\prod_{k=1}^{N}
\boldsymbol\Gamma\left[\bar \gamma^{(k-1)}_{2N-k}-i\bar  x_j\right],
%
\end{align}
%
$A_N=\sum_{k=N}^{2N-1} \gamma_k^{(k)}$, $X=\sum_{k=1}^N x_k$ and
%
\begin{align}
%
\mathrm C^{AB}_N(\gamma_1,\ldots,\gamma_{2N-1})  &=
\begin{cases}
%%
1 & \text{odd } N\\
%%
(-1)^{\sum_{k=1}^{N}\left[\gamma_{2N-k}^{(k-1)}-\gamma_{N}^{(N-1)}\right]}
%%
&
\text{even } N\,.
%%
\end{cases}
\end{align}
%
%
Similar to  the previous case one can argue that $\Phi^{(N)}_{x}$ is a continuous function of $\nu_k$ in the vicinity of  the real axis.


\vskip 3mm

Finally, the scalar product of the functions $\Psi^{(N+1)}_{p,x}(z_1,\ldots, z_{N+1})$ and $\Psi^{(N)}_{q_1,y}(z_1,\ldots,z_N)\otimes
\Psi^{(1)}_{q_2}(z_{N+1})$ which we need in the proof of Theorem~\ref{theorem}, takes the form
%
%
%
\begin{align}\label{bbN}
%
\left(\Psi^{N}_{q_1,y}\otimes \Psi^{(1)}_{q_2},\Psi^{(N+1)}_{p,x}\right) & =
\mathrm C_{NN+1}(\gamma)\, \pi\,\delta^{(2)}(p-q_1-q_2)\,
|p|^N\, |q_1|^{N-1} \times \,
\notag\\
&\quad
(ip)^{-\bar A_{N+1}^*  }
(-i\bar p)^{-A_{N+1}^*  }\, (iq_2)^{-\gamma'_{2N}}(-i\bar q_2)^{-\bar \gamma'_{2N}}
(-iq_1)^{-A_N } (i\bar q_1)^{-\bar A_N  }\, \times
\notag
\\[2mm]
&\quad
\left(1+\frac {q_1} {q_2}\right)^{i\bar Y^*}
\left(1+\frac{\bar q_1}{\bar q_2}\right)^{i Y^*}
\left(-\frac {q_2} {q_1}\right)^{i X}
\left(-\frac{\bar q_2}{\bar q_1}\right)^{i \bar X}\times
\notag\\
&\quad \frac{\prod_{k=1}^{N-1}\prod_{j=1}^{N}\boldsymbol\Gamma\left[i(\bar y_k^*-x_j)\right]}
{ \left(\prod_{j=1}^N\prod_{k=1}^{N-1} \boldsymbol\Gamma\left[\gamma_{2N-k}^{(k-1)}-ix_j\right]\right)
\left(\prod_{k=1}^N\prod_{j=1}^{N-1} \boldsymbol\Gamma\left[\bar\gamma_{2N-k}^{(k-1)}-i\bar y_j\right]\right)^\dagger }
\,,
%
\end{align}
%%%
where
%
\begin{align}\label{ANdef}
%
A_N=\sum_{m=N}^{2N-1}\gamma_m^{(m)}\,, &&  A_{N+1}=A_N-\gamma_N^{(N)}=\sum_{m=N+1}^{2N-1}\gamma_m^{(m)}
%
\end{align}
%
and
%
\begin{align}
%
\mathrm C_{NN+1}(\gamma_1,\ldots,\gamma_{2N})=\begin{cases}
1 & \text{for odd}\ N\\
    (-1)^{\sum_{k=1}^{N-1}\left[\gamma_{2N-k}^{(k-1)}-\gamma_N^{(N-1)}\right]} & \text{for even}\ N\,.
%
\end{cases}
\end{align}
The calculation is almost the same as in the previous cases so we omit the  details.


%%%%%%%%%%%%%%%%%%%%%%%%%%%%%%%%%%%%%%%%%%%%%%%%%%%%%%%%%%%%%%%%%%%%%%%%%%%%%%%%%%%%%%%%%%%%%%%%%%%%%%%%%%%%%%%%%%%%%%%%%%%%%%%%%%%%%%%%%%%%%
\section{SoV representation}\label{sect:SoV}
%%%%%%%%%%%%%%%%%%%%%%%%%%%%%%%%%%%%%%%%%%%%%%%%%%%%%%%%%%%%%%%%%%%%%%%%%%%%%%%%%%%%%%%%%%%%%%%%%%%%%%%%%%%%%%%%%%%%%%%%%%%%%%%%%%%%%%%%%%%%%

In the previous section we constructed the functions $\Psi_{p,x}^{(N)}$ and $\Phi^{(N)}_{x}$ associated with the entries $B_N$ and $A_N$ of
the monodromy matrix~\eqref{monodromymatrix}.  For  a given vector $\Psi \in \mathds H_N$ we define two functions by projecting it on
$\Psi_{p,x}^{(N)}$ and $\Phi^{(N)}_{x}$:
%
\begin{align}
%
\varphi(p,x_1,\ldots,x_{N-1})=\left(\Psi_{p,x}^{(N)},\Psi\right), &&
\chi(x_1,\ldots,x_N)=\left(\Phi_{x}^{(N)},\Psi\right).
%
\end{align}
%
These functions are symmetric functions of  the variables $x$.  It was shown by Sklyanin~\cite{MR1239668} that the transformation
$\Psi\mapsto \varphi\, (\Psi\mapsto \chi)$  reduces the original  multi-dimensional spectral problem for the transfer matrix to the set of
one-dimensional spectral problems that greatly simplifies  the analysis. We want to show that the maps $\Psi\mapsto \varphi$  and
$\Psi\mapsto \chi$ can be extended to the isomorphism between the Hilbert spaces, $\mathds H_N \mapsto \mathbb H_{\text{SoV}}$.


Let us define
%
\begin{align}\label{symscprod}
%
\left(\varphi_1,\varphi_2\right)_{B_N} &=
\int_{\mathbb R\times \mathbb R} \int_{\mathscr D^\sigma_{N-1}} (\varphi_1(p,x))^\dagger \varphi_2(p,x)\, \mu_{N-1}\left(x\right)
d^2p\, d\mu^{B}_{N-1}(x),
\notag\\
\left(\chi_1,\chi_2\right)_{A_N} &=
 \int_{\mathscr D^\sigma_{N}} (\chi_1(x))^\dagger \chi_2(x)\,
 d\mu^{A}_{N}(x).
%
\end{align}
%
The variables  $x_k,\bar x_k$ take the form $x_k=in_k/2+ \nu_k$, $\bar x_k=-in_k/2+ \nu_k$, where all $n_k$ are either integers or
half-integers,
%
\begin{align}
%
n_k\in \mathbb Z^\sigma\equiv \mathbb Z+\frac\sigma 2, && \sigma=0,1
%
\end{align}
%
and
%
\begin{align}
%
\mathscr D^\sigma_N \equiv \left(\mathbb R \times Z^\sigma\right)^N.
%
\end{align}
%
%
The  measures are defined as follows
%
\begin{align}
%
 d\mu^{B(A)}_N(x)=\mu^{B(A)}_N(x) \prod_{k=1}^N\mathcal Dx_k. && \mu^{B(A)}_{N}(x) =c_N^{B(A)} \mu_{N}(x).
%
\end{align}
%
The symbol $\mathcal D x$ stands for
%
\begin{align}
%
\int \mathcal D x  \equiv \sum_{n\in \mathbb Z^\sigma} \int_{-\infty}^\infty d\nu.
%
\end{align}
%
The weight function $\mu_{N}(x)$ is given  by the following expression
%
\begin{align}
%
\mu_{N}(x_1,\ldots,x_N) & =\prod_{1\leq k<j\leq N} x_{kj}\bar x_{kj}  =\prod_{1\leq k<j\leq N}\left(\nu_{kj}^2+\frac14 n_{kj}^2\right),
%
\end{align}
where $x_{kj}=x_k-x_j$, $\nu_{kj}= \nu_{k}-\nu_j$, $n_{kj}=n_k-n_j$ while the coefficients $c_N^{B(A)}$ take the form
\begin{align}
%
\left(c_{N}^B\right)^{-1}=\frac12{(2\pi)^{N+1} N!},&& \left(c_N^A\right)^{-1} ={(2\pi)^N N!}.
%
\end{align}
%
%

\vskip 3mm

Let  $\mathds H_N^{B,\sigma}$, $\mathds H_N^{A,\sigma}$ be the Hilbert spaces of symmetric functions corresponding to the scalar
products~\eqref{symscprod}:
%
\begin{subequations}
%
%
%
\begin{align}\label{HB}
%
\mathds H_N^{B,\sigma} &= L^2(\mathbb R\times\mathbb R)
 \otimes L^2_{\mathrm{sym}}\left( \mathscr D^\sigma_{N-1},
 d\mu^{B}_{N-1}(x)\right),
%
\\
\label{HA}
\mathds H_N^{A,\sigma} &=
 L^2_{\mathrm{sym}}\left( \mathscr D^\sigma_N,
 d\mu^A_{N}(x)\right).
\end{align}
%
%
\end{subequations}


Given that  $\varphi(p,x)$  and $\chi(x)$ are  smooth and compactly supported functions on $\mathbb R^2\times \mathscr D^\sigma_{N-1} $
and $\mathscr D^\sigma_N$, respectively, we introduce transforms $\mathrm T_N^{B}:\varphi\mapsto \Psi_\varphi$ and $\mathrm
T_N^{A}:\chi\mapsto \Psi_\chi$,
%
\begin{subequations}
\begin{align}
\label{TB}
%
\Psi_\varphi(z)\equiv [\mathrm T_N^{B}\varphi](z) & =
\int_{\mathbb R^2}\int_{\mathscr D^{\sigma}_{N-1}}
\varphi(p,x)\,
\Psi_{p,x}^{(N)}(z) \, d^2p\, d\mu^{B}_{N-1}(x)\,,
\\
\label{TA}
\Phi_\chi(z)\equiv [\mathrm T_N^{A}\chi](z) & =
\int_{\mathscr D^{\sigma}_N} \chi(x)\,\Phi_{p,x}^{(N)}(z) \,  d\mu^{A}_{N}(x)\,.
%
\end{align}
%
%
\end{subequations}
%
Note that the function $\Psi_\varphi$ depends on the vector $\gamma$,  Eq.~\eqref{gammaN}, which appears in the definition of the function
$\Psi_{p,x}^{(N)}$. That is $\mathrm T_N^{B} \equiv \mathrm T_N^B(\gamma)$ and the same applies to the operator $\mathrm T_N^A$. In order
to not overload the notation we do not display this dependence explicitly.


%%%%%%%%%%%%%%%%%%%%%%%%%%%%%%%%%%%%%%%%%%%%%%%%%%%%%%%%%%%%%%%%%%%%%%%%%%%%%%%%%%%%%%%%%%%%%%%%%%%%%%%%%%%%%%%%%%%%%%%%%%%%%%%%%%%%%%%%%%%%%
\subsection{ $B$ system }
%%%%%%%%%%%%%%%%%%%%%%%%%%%%%%%%%%%%%%%%%%%%%%%%%%%%%%%%%%%%%%%%%%%%%%%%%%%%%%%%%%%%%%%%%%%%%%%%%%%%%%%%%%%%%%%%%%%%%%%%%%%%%%%%%%%%%%%%%%%%%

We begin the proof of the unitarity of the transform  $ \mathrm T_N^B $  with the following lemma:

\begin{lemma}\label{firstlemma}

For any smooth fast decreasing  function $\varphi$ on $\mathbb{R}^2\times\mathscr D^\sigma_{N-1}$ the function $ \mathrm T_{N}^{B}\varphi$
belongs to the Hilbert space $\mathds H_N$ and it holds
%
%
%
\begin{align}\label{unit-lemma}
%
\|\mathrm T_{N}^{B}\varphi\|_{\mathds H_N}^2 =\|{\varphi}\|_{\mathds H_N^{B,\pm}}^2 = \int_{\mathfrak D_N^\pm}
|\varphi(p, x)|^2\, d^2 p
\,d\mu_{N-1}^{B}( x)\,.
\end{align}
%
\end{lemma}
%
%
\noindent {\bf Proof:} Let  $\Psi_\varphi^\epsilon$ be a function defined by Eq.~\eqref{TB}   with $\Psi_{p,x}^{(N)}$  replaced by
$\Psi_{p,x^\epsilon}^{(N),\epsilon}$, see Eqs.~\eqref{Psiespilon} and \eqref{Psixepsilon}. It can be shown that
$\Psi_\varphi^\epsilon(\vec{p})\underset{\epsilon\to 0}{\mapsto} \Psi_\varphi(\vec{p})$ almost everywhere. Next, taking into account
Eq.~\eqref{BBproduct} one gets
\begin{align}\label{PsiPsi}
%
\left(\Psi_\varphi^\epsilon,\Psi_{\varphi'}^{\epsilon'}\right)_{\mathds H_N}
    =\pi \int d^2p \int d\mu_{N-1}^{B}(x)\int d\mu_{N-1}^{B}(x') (p\bar p)^{\epsilon+\epsilon'}
    \,\varphi(p,x)\,(\varphi^\prime(p,x'))^\dagger\, I^{\epsilon,\epsilon'}(x,x'),
%
\end{align}
%
with $ I^{\epsilon,\epsilon'}(x,x')$  given by Eq.~\eqref{twoform}.
 Let us assume that the function $\varphi(\varphi^\prime)$
has the form
%
\begin{align}\label{psifac}
%
\varphi(p,x_1,\ldots,x_{N-1})   = \kappa(p) \phi(x_1,\ldots,x_{N-1}),
\end{align}
where $\phi(x_1,\ldots,x_{N-1})$ is a symmetric function
\begin{align}\label{ffact}
\phi(x_1,\ldots,x_{N-1}) =
\sum_{S_{N-1}} \phi_1(x_{i_1})\ldots  \phi_{N-1}(x_{i_{N-1}})
%
\end{align}
%
and the sum goes over all permutations. We also assume that the functions $\phi_k(x_k)=\phi_k(n_k,\nu_k)$  are local in $n_k$,
$\phi(n_k,\nu_k)=\delta_{n_k,m_k}\phi_k(\nu_k)$ and $\phi_k(\nu_k)$ is an analytic function of $\nu_k$ in some strip
$|\mathrm{Im}\nu_k|<\delta_k$ which vanishes sufficiently fast at $\nu_k\to \pm\infty$. Such functions form a dense subspace in the Hilbert
space $\mathds H_N^{B,\sigma}$.  Since the  momentum integral in \eqref{PsiPsi}  factorizes one has to consider the integrals over
 $x_k=(n_k, \nu_k)$,
$x'_k=(n'_k, \nu'_k)$ which have the form
%
\begin{align}\label{SumInt}
%
 \int d\mu_{N-1}^{B}(x)\int d\mu_{N-1}^{B}(x')  \cdots
 \equiv \prod_{j=1}^{N-1}\sum_{n_j\in \mathbb Z+\frac\sigma2}\sum_{n'_j\in \mathbb Z+\frac\sigma2}
 \int_{-\infty}^\infty\cdots \int_{-\infty}^\infty \,
  \mu_{N-1}^{B}(\vec{n},\vec{\nu})\mu_{N-1}^{B}(\vec{n}',\vec{\nu}')\prod_{k=1}^{N-1}d\nu_k d\nu'_k \cdots.
%
\end{align}
%
According to our   assumptions only finite number of terms contribute to the sum in~\eqref{SumInt}. Let us study behaviour of a particular
term in the sum  in the limit $\epsilon,\epsilon'\mapsto 0$. The functions $\phi$, $\phi^\prime$ are smooth and fast decreasing functions
of $\nu, \nu'$. The function $ I^{\epsilon,\epsilon'}(x,x')$ contains the factor $\boldsymbol \Gamma[\epsilon+\epsilon' +i\bar X-i(X')^*]/
\boldsymbol \Gamma[\epsilon+\epsilon']$ and the product of the $\boldsymbol \Gamma$-functions
%
\begin{align}\label{afactor}
\boldsymbol\Gamma\left[i((\bar x'_k)^*-x_j)\right]=
\boldsymbol\Gamma\left[\frac{n'_k}2 - \frac{n_j}2 + i(\nu'_k-\nu_j) + \epsilon_{jk}\right]
=\frac{\Gamma\left(\frac{n'_k}2 - \frac{n_j}2 + i(\nu'_k-\nu_j) + \epsilon_{jk}\right)}{
\Gamma(\left(1+\frac{n'_k}2 - \frac{n_j}2 - i(\nu'_k-\nu_j) - \epsilon_{jk}\right)},
\end{align}
where $\epsilon_{jk}\equiv\epsilon_j+\epsilon'_k$. In the  $\epsilon'_k, \epsilon_j\to 0$ this function becomes singular at $\nu'_k =\nu_j$
if $n'_k=n_j$. Let us shift the contours of integrations over $\nu'_k$ variables to the upper half-plane,
$\mathrm{Im}\nu'_k=\delta>\epsilon_{jk}$, and pick up the residues at the corresponding poles. After this we can send $\epsilon'_k,
\epsilon_j\mapsto 0$. Let us consider a generic contribution arising after this rearrangement. It has the form
%
\begin{align}
%
\int_{C_\delta} \cdots \int_{C_\delta}
     \prod_{k=1}^{M} d\nu'_{i_k} f\left( x_1,\ldots,x_{N-1}, S(x'_1),\ldots, S(x'_{N-1})\right)\,,
%
\end{align}
%
where $ S(x'_k)= x'_k $ if $k \in (i_1,\ldots i_M)$ and $S(x'_k)= x_{p_k}$ if $k$ does not belong to this set. The integrand $f$ is given
by the product of the functions $\phi_k$, $\phi^\prime_k$,  $\boldsymbol\Gamma$~-~functions~\eqref{afactor} and the factor $A=\boldsymbol
\Gamma[\epsilon+\epsilon' +i\bar X-i(X')^*]/ \boldsymbol \Gamma[\epsilon+\epsilon']$.
 All these factors
are regular on the  contours of integration. Moreover, if $M\geq 1$ the last factor, $A$, tends to zero at $\epsilon,\epsilon'\mapsto 0$.
Thus the only non-vanishing contribution comes from the term with $M=0$, i.e. when all $x'_k\mapsto x_{i_k}$ for $k=1,\ldots, N-1$. It
takes the form
%
%
\begin{align}
%
\left(\Psi_\varphi^\epsilon,\Psi_{\varphi'}^{\epsilon'}\right)_{\mathds H_N}
    =  \int d^2p \int d\mu_{N-1}^{B}(x)
    \,\varphi(p,x)\,(\varphi^\prime(p,x))^\dagger\,
    +O(\epsilon+\epsilon')
%
\end{align}
%
that results in the following estimate for the norm of the function $\Psi^\epsilon_\varphi$,
%
\begin{align}
%
\|\Psi^\epsilon_\varphi\|^2_{\mathds H_N} &= K + O(\epsilon),
%
\end{align}
%
where
%
\begin{align}
%
K =\|{\varphi}\|_{\mathds H_N^{B,\sigma}}^2 \equiv \int_{\mathbb R^2}\int_{\mathscr D^{\sigma}_{N-1}}
|\varphi(p, x)|^2\, d^2 p
\,d\mu_{N-1}^{B}( x)\,.
%
\end{align}
%
Since $\Psi_\varphi^\epsilon(\vec{p})\mapsto \Psi_\varphi(\vec{p})$ at $\epsilon \to 0$ it follows from Fatou's theorem that
$\|\Psi_\varphi\|^2_{\mathds H_N} < K$. At the same time the inequality
%
\begin{align}
%
\|\Psi_\varphi-\Psi_\varphi^\epsilon \|^2_{\mathds H_N} \geq 0
%
\end{align}
%
implies  $\|\Psi_\varphi\|^2_{\mathds H_N}\geq K$ that results in $\|\Psi_\varphi\|^2_{\mathds H_N}=K$.

Since the set of functions~\eqref{psifac}, \eqref{ffact} is dense in the Hilbert spaces $\mathds H_N^{B\pm}$ the transformation $\mathrm
T_N^{B}$ can be extended to the entire Hilbert space $\mathds H_N^{B,\pm}$ and \eqref{unit-BA} holds for any function $\varphi \in \mathds
H_N^{B,\pm}$. $\blacksquare$

Taking this result into account we formulate the following theorem.

 \vskip 3mm

%
\begin{thm}\label{theorem}
%
The map $\mathrm T_N^B$ defined in Eq.~\eqref{TB} can be extended to the linear bijective isometry of the Hilbert spaces, $\mathbb
H_N^{B,\sigma} \mapsto \mathbb H_N$, i.e.
%
\begin{subequations}
%
%
\begin{align}\label{unit-BA}
%
\|\mathrm T_{N}^{B}\varphi\|_{\mathds H_N}^2 &=\|{\varphi}\|_{\mathds H_N^{B,\sigma}}^2
\\
\intertext{and}
\label{unit-Range}
\mathcal R\left( \mathrm T_{N}^{B}\right)
&= \mathds H_N.
%
\end{align}
\end{subequations}
\end{thm}

\vskip 3mm

\noindent {\bf Proof:} Eq.~\eqref{unit-BA} is a direct consequence of Lemma~\ref{firstlemma}. It  implies that $\|\mathrm T_{N}^{B}\|=1$,
hence $\mathcal R\left( \mathrm T_{N}^{B}\right)$ is a closed subspace in $\mathds H_N$ and  $\mathds H_N= \mathcal R\left( \mathrm
T_{N}^{B}\right)\oplus \mathcal R\left( \mathrm T_{N}^{B}\right)^\perp$. Since $ \mathcal R\left( \mathrm T_{N}^{B}\right)^\perp =
\ker\left(\mathrm T_N^B\right)^*$ in order to prove \eqref{unit-Range} it is enough to show that $ \ker\left(\mathrm T_N^B\right)^*=0$.


\vskip 1mm

We prove this statement using induction on $N$. For $N=1$ the map $\mathrm T_{N=1}^B$ is a two-dimensional Fourier transform, hence
Eq.~\eqref{unit-Range} is true. Let us now assume that $\mathcal R\left( \mathrm T_{N}^{B}\right) = \mathds H_N$ and prove that it implies
$\mathcal R\left( \mathrm T_{N+1}^{B}\right)= \mathds H_{N+1}$. As was stated above it is sufficient to prove that  $\ker\left(\mathrm
T_{N+1}^B\right)^*=0$. To this end let us consider the map
%
\begin{align}
%
\mathrm S_N =\left(\mathrm T_{N+1}^B\right)^* \left(\mathrm T_{N}^B\otimes \mathrm T_{1}^B\right), &&
\mathds H_{N}^{B,\sigma}\otimes L^2(\mathbb R^2)\overset{\mathrm T_{N}^B\otimes \mathrm T_{1}^B}{\longmapsto} \mathds H_{N+1}
\overset{(\mathrm T_{N+1}^B)^*}{\longmapsto}\mathds H_{N+1}^{B,\sigma}.
%
\end{align}
%
Since by the assumption  $\mathrm T_{N}^B\otimes \mathrm T_{1}^B$ is a bijective isometry $\ker \mathrm S_N=0$ if and only if $\ker
\left(\mathrm T_{N+1}^B\right)^* = 0 $.

The adjoint operator $(\mathrm T_{N+1}^B)^*$ is a bounded operator which acts on a vector $\Psi\in \mathds H_{N+1}$ by projecting it on the
eigenfunction $\Psi^{(N+1)}_{p,x}$,
%
\begin{align}\label{def-varphi}
%
\left(\mathrm T_{N+1}^B\right)^* \Psi = \left(\Psi^{(N+1)}_{p,x},\Psi\right)_{\mathds H_{N+1}}=
\left(\Psi^{(N+1)}_{p,x},\mathrm P_{N+1}\Psi\right)_{\mathds H_{N+1}}\equiv\varphi(p,x),
%
\end{align}
%
where $\mathrm P_{N+1}$ is the projector on $\mathcal R(\mathrm T^B_{N+1})$. It follows from \eqref{def-varphi} that
%
\begin{align}\label{firstnormestimate}
%
\Vert \varphi\Vert_{\mathds H_{N+1}^{B,\sigma}}^2 & =\int_{\mathbb R^2}\int_{\mathscr D^{\sigma}_{N}}\vert \varphi(p,x)\vert^2 d^2p d\mu_N^B(x) =
\Vert \mathrm P_{N+1}\Psi\Vert^2_{\mathds H_{N+1}} \leq \Vert \Psi\Vert^2_{\mathds H_{N+1}}.
%
\end{align}
%
For $\phi\in \mathds H_{N}^{B,\sigma}\otimes L^2(\mathbb R^2)$ the function $\Psi_\phi=\left(\mathrm T_{N}^B\otimes \mathrm
T_{1}^B\right)\phi$ reads
%For a smooth  rapidly decreasing function  $\phi$ on $\mathbb{R}^2\times\mathbb{R}^2\times\left(\mathbb{R}\times Z^\sigma\right)^{N-1}$ we
%define a function $\Psi_\phi$ as
%
\begin{align}\label{TNT1}
%
 \Psi_\phi(z) & %\equiv \left(\mathrm T_{N}^B\otimes \mathrm T_{1}^B\right)\phi (z) &
            =  \int_{\mathbb{R}^2\otimes\mathbb{R}^2} \int_{\mathscr D^{\sigma}_{N-1}}
            \Psi^{(N)}_{q_1,x}(z_1,\ldots,z_N)\, \Psi^{(1)}_{q_2}(z_{N+1})\,
\phi(q_1,q_2,x)     d^2q_1 d^2q_2 d\mu_{N-1}^B(x).
%
&
\end{align}
%
Replacing $ \Psi^{(N)}_{q_1,x}\mapsto \Psi^{(N),\epsilon}_{q_1,x}$ in \eqref{TNT1} we define a new function, $\Psi_\phi^\epsilon$.
% be a function given by the expression~\eqref{TNT1} with $ \Psi^{(N)}_{q_1,x}\mapsto \Psi^{(N),\epsilon}_{q_1,x}$,
%see Eq.~\eqref{Psiepsilon}.
Since by Lemma~\ref{firstlemma} $\Psi_\phi^\epsilon\underset{\epsilon\to 0^+}{\longrightarrow} \Psi_\phi$ in $\mathds H_{N+1}$ for smooth
rapidly decreasing functions  we obtain
%
\begin{align}\label{varphiSN}
%
\varphi(p,x)=[ S_N \phi](p,x)= \left(\Psi^{(N+1)}_{p,x},\Psi_\phi\right)_{\mathds H_{N+1}}=\lim_{\epsilon\to 0^+}
\left(\Psi^{(N+1)}_{p,x},\Psi^\epsilon_\phi\right)_{\mathds H_{N+1}}\equiv \lim_{\epsilon\to 0^+}\varphi_\epsilon(p,x),
\end{align}
%
where
%
\begin{align}\label{varphiSepsilon}
%
\varphi_\epsilon(p,x) & =%\lim_{\epsilon\to 0^+}
\int_{\mathbb R^2\times \mathbb R^2 }\int_{\mathscr D^{\sigma}_{N-1}}
\, S^\epsilon_N(p,x|q_1,q_2,x')\,\phi(q_1,q_2,x') \,     d^2q_2 d^2q_1 d\mu_{N-1}^B(x').
%
\end{align}
%
The kernel $S_N^\epsilon$ reads
%
\begin{align}\label{Sepsilon}
%
S^\epsilon_N(p,x|q_1,q_2,x') & = \left(\Psi^{(N+1)}_{p,x},\Psi^{(N)}_{q_1,x'_\epsilon}\otimes \Psi^{(1)}_{q_2}\right),
%
\end{align}
%
see Eq.~\eqref{bbN}, and $x'_\epsilon = \left(x'_1+i\epsilon_1,\ldots,x'_{N-1}+i\epsilon_{N-1}\right)$. We assume that function $\phi$
takes the form
%
\begin{align}\label{phidef}
%
\phi(q_1,q_2,x_1,\ldots,x_{N-1})   = \kappa_1(q_1)\kappa_2(q_2) \sum_{S_{N-1}} \phi_1(x_{i_1})\ldots  \phi_{N-1}(x_{i_{N-1}}),
%
\end{align}
%
where the sum  goes over all permutations and that the functions $\phi_k$  are local in ``$n$" variable, that is
$\phi_k(x_k)=\phi_k(n_k,\nu_k)=\delta_{n_k m_k}\phi_{n_k}(\nu_k)$ and $\phi_{n_k}$ are compactly supported.
%and the functions $\phi_{k}$ satisfy the same conditions as in Lemma~\ref{firstlemma}, see Eq.~\eqref{ffact}.

%Note that $\|\varphi\|^2_{\mathds H_{N+1}^{B,\sigma}} =\lim_{\epsilon\to 0^+}\|\varphi_\epsilon\|^2_{\mathds H_{N+1}^{B,\sigma}}$, where
%\begin{align}\label{varphiBdef}
%%
%\|\varphi_\epsilon\|^2_{\mathds H_{N+1}^{B,\sigma}} & =\int_{\mathbb R^2}\int_{\mathscr D^{\sigma}_{N}}
%|\varphi_\epsilon(p,y)|^2  d^2p\,d\mu_{N}^B(y).
%         %= \lim_{\epsilon\to 0^+}\int_{\mathbb R^2}\int_{\mathscr D^{\sigma}_{N}} |\varphi_\epsilon(p,y)|^2  d^2p\,d\mu_{N}^B(y).
%         % \|\phi\|^2_{\mathds H_{N}^{B,\sigma}\otimes L^2(\mathbb R^2)}
%%
%\end{align}
%We want to prove that  $\|\varphi\|^2_{\mathds H_{N+1}^{B,\sigma}} = \|\phi\|^2_{\mathds H_{N}^{B,\sigma}\otimes L^2(\mathbb R^2)}$.
%%
%%
 The function $\varphi(p,y)$ does not decrease sufficiently fast for large $y_k$ in order to justify changing the order of integration after
substituting $\varphi_\epsilon(p,y)$ in the form~\eqref{varphiSN}, \eqref{varphiSepsilon} into \eqref{firstnormestimate}. To overcome this
difficulty we, following the lines of ref.~\cite{DerkachovKozlowskiManashov21},  consider the integral
%
\begin{align}
%
%\|\varphi\|^2_{\mathds H_{N+1}^{B,\sigma}} & =\lim_{M\to\infty}
I_Z(\varphi) &=
\int_{\mathbb R^2}\int_{\mathscr D^{\sigma}_{N}} |
\varphi(p,y)|^2 \Omega_Z(y)  d^2p\,d\mu_{N}^B(y)\,, %= \lim_{M\to\infty}  F_M(\epsilon).
%
\end{align}
%%
% by  replacing
%%
%\begin{align}
%%
%\vert\varphi_\epsilon(p,y)\vert^2 \mapsto \vert\varphi_\epsilon(p,y)\vert^2 \Omega(Z,y),
%\end{align}
%%
where
\begin{align}
%
%
        \Omega_Z(y)=\prod_{k=1}^{N} \frac{\boldsymbol\Gamma\left[ Z+iy_k, Z-iy_k\right]}
        {\boldsymbol\Gamma\left[Z,Z\right]},   && Z=\bar Z =\frac12 + i M.
%
\end{align}
%
%and
%
%$
%%\begin{align}
%%
%Z=\bar Z =\frac12 + i M.
%%
%%
%%\end{align}
%%
%$
For $y_k^*=\bar y_k$ the factor  $\Omega$ is a pure phase, $\vert\Omega_Z(y)\vert = 1$ and  $\Omega_Z(y)\mapsto 1$ when $M\to \infty$, $y$
is fixed. Since the integral \eqref{firstnormestimate} is convergent  % one gets
%
\begin{align}
%
\|\varphi\|^2_{\mathds H_{N+1}^{B,\sigma}} =\lim_{M\to\infty}
\int_{\mathbb R^2}\int_{\mathscr D^{\sigma}_{N}} |
\varphi(p,y)|^2\, \Omega_{Z}(y)  d^2p\,d\mu_{N}^B(y).
%
\end{align}
%

%%
%\begin{align}
%%
%\|\varphi_\epsilon\|^2_{\mathds H_{N+1}^{B,\sigma}} & =\lim_{M\to\infty} \int_{\mathbb R^2}\int_{\mathscr D^{\sigma}_{N}} |
%\varphi_\epsilon(p,y)|^2 \Omega(Z,y)  d^2p\,d\mu_{N}^B(y). %= \lim_{M\to\infty}  F_M(\epsilon).
%%
%\end{align}
%%
%and
% Note, that $F_M(\epsilon){\rightarrow} F_{M}(0)$  uniformly in $M$, therefore $\epsilon$ and $M$ limits can be interchanged.

It follows from Eqs.~\eqref{varphiSepsilon}, \eqref{Sepsilon} and \eqref{bbN} that for compactly supported functions $\phi_k$ the function
$f(\nu)=|\varphi_\epsilon(p,y)|^2$ is an analytic function of $\nu_k$ in the vicinity of the real axis for sufficiently large $\nu_k$.
%
% large $\nu_k$ ($y_k=n_k/2+i\nu_k$) the function $f(\nu)=|\varphi_\epsilon(p,y)|^2$ is an analytic function of $\nu_k$ in the vicinity of
%the real axis.
Thus we can write %%
%
\begin{align}\label{IomegaZ}
%
I_Z(\varphi) =\lim_{\omega\to 0} I^\omega_Z(\varphi) =
\lim_{\omega\to 0}\int_{\mathbb R^2}\int_{\mathscr D^{\sigma,\omega}_{N}} |
\varphi(p,y)|^2\, \Omega_{Z-\omega}(y)  d^2p\,d\mu_{N}^B(y),
%
\end{align}
%%
where  the integration contours over $\nu_k$ are deformed in order to separate the poles due to the Gamma functions, $\boldsymbol\Gamma
\left[ Z-\omega\pm iy_k\right]$, in the factor $\Omega$. The integral $I^\omega_Z(\varphi)$ is an analytic function of $\omega$.
Substituting $\varphi(p,y)$ in~\eqref{IomegaZ} in the form~\eqref{varphiSepsilon} one can show that for $\text{Re}\,\omega>1$ the integrals
over $y$ decay fast enough to allow the change of the order of integration over $x,x'$ and $y$. Thus we obtain
%
\begin{align}\label{Iomega}
%
I^\omega_Z(\varphi) & = \lim_{\epsilon,\epsilon'\to 0^+} \int_{\mathbb R^2\times \mathbb R^2}
\int_{\mathscr D^{\sigma}_{N-1}\times \mathscr D^{\sigma}_{N-1}}\!
 \delta^{(2)}(q_1+q_2-q'_1-q'_2)\, \phi(q_1,q_2,x)\,
\left(\phi(q'_1,q'_2,x')\right)^\dagger \,\left|\frac{q_1+q_2}{q_1 q'_2}\right|^{2}\left|\frac{q'_1}{q_1}\right|^{N-1}\!\!\times
%
\notag\\[7pt]
&\quad
\left(1+\frac{q'_1}{q'_2}\right)^{iX'}\left(1+\frac{\bar q'_1}{\bar q'_2}\right)^{i\bar X'}\left(1+\frac{q_1}{q_2}\right)^{-iX}
\left(1+\frac{\bar q_1}{\bar q_2}\right)^{-i\bar X}
\left(\frac{q_1}{q'_1}\right)^{A_N}\left(\frac{\bar q_1}{\bar q'_1}\right)^{\bar A_N}
\left(\frac{q'_2}{q_2}\right)^{\gamma_{2N}}\left(\frac{\bar q'_2}{\bar q_2}\right)^{\bar \gamma_{2N}}
\,\times
\notag\\[7pt]
&\quad
R(x,x')  \,
J^{(\epsilon)}_\omega(Z,\zeta,x,x')
%
\,d^2q_1\, d^2q_2\, d^2q'_1\, d^2q'_2 \,d\mu_{N-1}^B(x)\, d\mu_{N-1}^B(x')\,,
%
\end{align}
%
where $\zeta = \dfrac{q_1 q'_2}{q_2 q'_1}$, $A_N$ is defined in Eq.~\eqref{ANdef},
%
\begin{align}
%
R(x,x') =\prod_{k=1}^N\prod_{j=1}^{N-1}
{\boldsymbol\Gamma\left[\bar\gamma^{(k-1)}_{2N-k}-i x'_j\right]} / {\boldsymbol\Gamma\left[\bar\gamma^{(k-1)}_{2N-k}-i x_j\right]}
%
\end{align}
%
and
%
\begin{align}\label{Jomega}
%
J^{(\epsilon,\epsilon')}_\omega(Z,\zeta,x,x') & =\pi^2\int_{\mathscr D^{\omega,\sigma}_{N}}
\zeta^{iY}\bar \zeta^{i\bar Y}\prod_{j=1}^N
\frac{\boldsymbol \Gamma[Z-\omega \pm iy_j]}{\boldsymbol\Gamma^2(Z)}{\prod_{k=1}^{N-1}
\boldsymbol \Gamma[i( \bar x_k - \bar y_j)]\boldsymbol \Gamma[i( y_j - x'_k)]} d\mu_{N}^B(y).
%
\end{align}
%
We recall that the variables $\nu_k,\nu'_k$, ($x_k= in_k/2+\nu_k$, $x'_k=in'_k/2+\nu'_k$) have  small negative (positive) imaginary parts,
$\text{Im}\nu_k=-\epsilon_k$,  $\text{Im}\nu'_k=\epsilon'_k$, which must be send to zero at the end of the calculation.

%It is quite remarkable
 The integral~\eqref{Jomega} can be obtained in the closed form with the help of Eq.~\eqref{GustafsonII}. Indeed,
%
\begin{align}
%
\prod_{1\leq j\neq k\leq N} \frac 1{\boldsymbol \Gamma[i(y_k-y_j)]} & =\mu_N(y) \, (-1)^{\sum_{k<j} [i(y_k-y_j)]},
\notag\\
\prod_{j=1}^N\prod_{k=1}^{N-1}\boldsymbol \Gamma[i(\bar x_k -\bar y_j)] &=
\prod_{j=1}^N\prod_{k=1}^{N-1}\boldsymbol \Gamma[i( x_k -y_j)] (-1)^{\sum_{j=1}^N\sum_{k=1}^{N-1} [i(y_j-x_k)]}\,,
%
\end{align}
%
where $y_k = i m_k/2 + \nu_k$, $\bar y_k = -i m_k/2 + \nu_k$ and we recall that $[iy_k]= i(y_k-\bar y_k)=-m_k$. Taking into account that
%
\begin{align}
%
(-1)^{\sum_{k<j} [i(y_k-y_j)]} (-1)^{\sum_{j=1}^N\sum_{k=1}^{N-1} [i(y_j-x_k)]} =(-1)^{\sum_{1\leq k<j\leq N-1} [i(x_k-x_j)]}
%
\end{align}
%
one finds that the integral \eqref{Jomega} is nothing else as Gustafson's integral~\eqref{GustafsonII} [$u_k\to iy_k$ for all~$k$,
 $\{z_1,\ldots,z_N\}\mapsto \{ix_1,\ldots,i x_{N-1}, Z-\omega\}$ and  $\{w_1,\ldots,w_N\} \mapsto \{ -ix'_1,\ldots,-i x'_{N-1}, Z-\omega
 \}$].
Thus we obtain for $ J^{(\epsilon,\epsilon')}_\omega$:
%
\begin{align}\label{Jomega2}
%
J^{(\epsilon,\epsilon')}_\omega(Z,\zeta,x,x')& = \pi\,(-1)^{\sum_{ k<j} [i(x_k-x_j)]}
\frac{\boldsymbol\Gamma[2Z-2\omega]}{\boldsymbol\Gamma^2[Z]}
\frac{\zeta^{Z - \omega + iX}}{(1 + \zeta)^{2(Z-\omega) + i(X  - X')}}
    \frac{\bar\zeta^{\bar Z - \omega + i\bar X}}{(1 + \bar \zeta)^{2(\bar Z - \omega) + i(\bar X - \bar X')}}\times
    \notag\\
    &\quad
\prod_{k=1}^{N-1}
\frac{\boldsymbol\Gamma\left[Z-\omega+ix_k,Z-\omega-ix_k'\right]}{{\boldsymbol\Gamma[Z,Z]}}
\prod_{k,j=1}^{N-1}\boldsymbol\Gamma[i(x_k-x'_j)]\,.
%
\end{align}
%
Let us substitute this expression into~\eqref{Iomega} and calculate the corresponding limits. First of all, since all factors containing
$\omega$ are regular at $\omega,\epsilon_k,\epsilon'_k\to 0$ one can interchange the limits and first send $\omega\to 0$.

At $M\to\infty$ the integral over $q,q'$ is dominated by the contribution from the stationary point at $\zeta=1$,
%
\begin{align}
%
\frac{\boldsymbol\Gamma[1+2iM]}{\boldsymbol\Gamma^2[\frac12+iM]}
\int d^2\zeta \frac{(\zeta\bar\zeta)^{iM+\frac12}}{ ((1+\zeta)(1+\bar\zeta))^{1+2iM} }\varphi(\zeta) & \underset{M\to \infty}{=}
        \pi \varphi(1 )\left( 1 + O\left(\frac1{M^{1/2}}\right)\right).
%
\end{align}
%
Taking this into account and expanding the first factor in the second line in~\eqref{Jomega2} one gets for \eqref{Iomega}
\begin{align}\label{Iomega=0}
%
I_{\omega=0}(Z) & = \lim_{\epsilon,\epsilon'\to 0^+}
\int_{\mathscr D^{\sigma}_{N-1}\times \mathscr D^{\sigma}_{N-1}}\!
 \phi(q_1,q_2,x)\,
\left(\phi(q_1,q_2,x')\right)^\dagger
%
\,\times
\notag\\[7pt]
&\quad  \pi\,(-1)^{\sum_{ k<j} [i(x_k-x_j)]} i^{N-N'}
 R(x,x') \left(1+\frac{q_1}{q_2}\right)^{i(X'-X)}
\left(1+\frac{\bar q_1}{\bar q_2}\right)^{i(\bar X'-\bar X)}\times
\notag\\[7pt]
&\quad
\left(\frac M 2\right)^{2i(\mathcal V-\mathcal V')}
\prod_{k,j=1}^{N-1}\boldsymbol\Gamma[i(x_k-x'_j)+ \epsilon_{kj}]
%
\,d^2q_1\, d^2q_2\, d\mu_{N-1}^B(x)\, d\mu_{N-1}^B(x')   + \ldots\,,
%
\end{align}
%
where ellipses stand for terms vanishing at $M\to\infty$ and
%
\begin{align*}
%
 x_k=\frac{ in_k}2+ \nu_k, && x'_k= \frac{in'_k}2+ \nu'_k, && \epsilon_{kj}=\epsilon_k+\epsilon'_j, && X=\sum_{k=1}^{N-1}x_k,
 && \mathcal V=\sum_{k=1}^{N-1}\nu_k, &&N=\sum_{k=1}^{N-1}n_k,
%
\end{align*}
%
etc. The analysis of this integral is similar to the analysis of the integral~\eqref{PsiPsi}~\footnote{ We do it assuming that the
functions $\phi_k(x_k)$ have the properties discussed around Eq.~\eqref{psifac}.
 }.
 In the limit
$\epsilon,\epsilon'\to 0$ the poles of the Gamma functions, $x_k=x'_j$, approach the integration contour, while  all other factors remain
regular. Let us shift the integration contour in $x_k$ to the upper complex half-plane picking up the residues  at the poles at $x_k=x'_j$.
We recall that the Gamma functions develop poles only when $n_k=n'_j$, otherwise they are regular at $\nu_k=\nu'_j$. Afterwards we can send
$\epsilon,\epsilon'\to 0$. The answer is given by the sum of  terms
%
\begin{align}\label{MM}
%
\idotsint
M^{i\sum_{k=1}^m (\nu_{i_k}-\nu'_{j_k})}\times f_m(x,x')d\nu_{i_1}\ldots d\nu_{i_m} d\nu'_1\ldots d\nu'_{N-1},
%
\end{align}
%
where $f_m(x,x')$ is a smooth function. Note, the contours of integration  over $\nu$ variables  lay in the upper half-plane, so that
$\vert M^{i\sum_{k=1}^m (\nu_{i_k}-\nu'_{j_k})}\vert <1$ in the integration region. Since the functions $ f_m(x,x')$ are smooth functions
all such terms with $m>0$ vanish after integration in the limit $M\to\infty$. Thus the only contribution with $m=0$, i.e. when
$x_k=x'_{k_j}$, survives in this limit. Then one obtains after some algebra
%
\begin{align}
%
\Vert\varphi\Vert^2_{\mathds H_{N+1}^{B,\sigma}} & =\Vert S_N\phi\Vert^2_{\mathds H_{N+1}^{B,\sigma}}=
 \int_{\mathbb R^2\times \mathbb R^2}\int_{\mathscr D^{\sigma}_{N-1}}
\vert \phi(q_1,q_2,x)\vert^2\,  d^2q_1 d^2q_2\,d\mu_{N-1}^B(x)  =  \|\phi\|^2_{\mathds H_{N}^{B,\sigma}\otimes L^2(\mathbb R^2)}\,.
%
\end{align}
%
Since the space of functions~\eqref{phidef} dense in $\mathds H_{N}^{B,\sigma}\otimes L^2(\mathbb R^2)$ this relation can be extended to
the whole Hilbert space. Thus one concludes that $\ker S_N=0$, and, hence, $\ker\left(\mathrm T_{N+1}^B\right)^*=0$.  $\blacksquare$

%%%%%%%%%%%%%%%%%%%%%%%%%%%%%%%%%%%%%%%%%%%%%%%%%%%%%%%%%%%%%%%%%%%%%%%%%%%%%%%%%%%%%%%%%%%%%%%%%%%%%%%%%%%%%%%%%%%%%%%%%%%%%%%%%%%%%%%%%%%%%
\subsection{$A$ system}\label{sect:Asystem}
%%%%%%%%%%%%%%%%%%%%%%%%%%%%%%%%%%%%%%%%%%%%%%%%%%%%%%%%%%%%%%%%%%%%%%%%%%%%%%%%%%%%%%%%%%%%%%%%%%%%%%%%%%%%%%%%%%%%%%%%%%%%%%%%%%%%%%%%%%%%%

Using the results of the previous section it becomes quite  easy to prove the unitarity of $\mathrm T_N^A$ transform.  First, we prove an
analogue of the Lemma~\ref{firstlemma}
%
\begin{lemma}\label{secondlemma}
%
For any smooth fast decreasing function $\chi$ on $\mathscr{D}^\sigma_{N}$ the function $ \mathrm T_{N}^{A}\chi$, Eq.~\eqref{TA}, belongs
to the Hilbert space $\mathds H_N$ and it holds
%
%
\begin{align}\label{unit-lemma-A}
%
\|\mathrm T_{N}^{A}\chi\|_{\mathds H_N}^2 =\Vert{\chi}\Vert_{\mathds H_N^{A,\sigma}}^2 = \int_{\mathscr D^{\sigma}_{N}}
|\chi(x)|^2
\,d\mu_{N}^{A}( x)\,.
%
\end{align}
%
\end{lemma}
%
\noindent {\bf Proof:}  The proof is similar to the proof of the lemma~\ref{firstlemma}. It  suffices to prove~\eqref{unit-lemma-A} for
functions of the form
%
\begin{align}\label{chidef}
%
\chi(x_1,\ldots,x_{N})   =  \sum_{S_{N}} \chi_1(x_{i_1})\ldots  \chi_{N}(x_{i_{N}}),
    && \chi_k(x_k)=\chi_k(n_k,\nu_k)=\delta_{n_km_k}\chi_k(\nu_k).
%
\end{align}
%
We assume  that the functions $\chi_k(\nu)$ are analytic in some strip near the real axis. Let us calculate the projection
%
\begin{align}\label{varphiPhi}
%
\varphi_\chi(p,y) & = \left( \Psi_{p,y}^{(N)}, \Phi_\chi \right) =\lim_{\epsilon\to 0}
            \int_{ \mathscr{D}^\sigma_N } \left( \Psi_{p,y}^{(N)},\Phi_{x+i\epsilon}^{(N)}\right) \chi(x) d\mu^A_{N}(x).
%
\end{align}
%
Here we have given the variables $x_k\to x_k+i\epsilon_k$, $\epsilon_k=\bar\epsilon_k>0$  small imaginary parts which allows us to change
the order of integration. In order to  show that
  $ \| \varphi_\chi\|_{\mathrm{H}_N^{B,\sigma}}  = \| \chi \|_{\mathrm{H}_N^{A,\sigma} } $ we write
%
\begin{align}\label{normA}
%
\Vert \varphi_\chi\Vert^2_{\mathds H_{N}^{B,\sigma}} &= \int_{\mathbb R^2} \int_{\mathscr D^{\sigma}_{N-1}}
 \vert \varphi(p,y)\vert^2 d^2p d\mu_{N-1}^B(y) =
 \lim_{\sigma\to 0} \int_{\mathbb R^2} e^{-\sigma|p|^2}\left( \int_{\mathscr D^{\sigma}_{N-1}}\!
 \vert \varphi(p,y)\vert^2 d\mu_{N-1}^B(y)\right) d^2p.
%
\end{align}
%
Using the representation ~\eqref{varphiPhi} for  $\varphi_\chi(p,y)$ we first evaluate the $y$-integral~\footnote{The $x,x',y$ integral can
be interchanged since the integral of modulus is convergent.}. This integral coincides with the so-called $\mathrm{SL}(2,\mathbb C)$
Gustafson integral and can be evaluated in a closed form~\eqref{GustafsonI} resulting in
%yielding for~\eqref{normA}
\begin{align}
%
\Vert \varphi_\chi\Vert^2_{\mathds H_{N}^{B,\sigma}}  & =\frac1\pi
 \lim_{\sigma\to 0}  \lim_{\epsilon,\epsilon'\to 0^+}\int_{\mathbb R^2} \int_{\mathscr D^{\sigma}_{N}\times \mathscr
D^{\sigma}_{N}} e^{-\sigma|p|^2}  i^{N-N'} p^{i(X'-X)-1+\mathcal E+\mathcal E'}
\bar p^{i(\bar X'- \bar X)-1+ \mathcal E+\mathcal E'}(-1)^{\sum_{k<j}[i(x'_k -x'_j)]}
\notag\\
&\quad
\frac{\chi(x)}{\left(\prod_{j=1}^{N}\vartheta_N(x_j)\right)}\!
\left(\frac{\chi(x')}{\left(\prod_{j=1}^{N}\vartheta_N(x'_j)\right)}\right)^\dagger\!
\frac{\prod_{k,j=1}^N \boldsymbol\Gamma[i(x'_k-x_j)+\epsilon_{jk}]}{\boldsymbol\Gamma[i( X'- X )+\mathcal E+\mathcal E']}
 d\mu^A_N(x) d\mu^A_N(x') d^2p,
%
\end{align}
%
where $X=\sum_{k=1}^N x_k$, $N=\sum_{k=1}^N n_k$, $\mathcal E=\sum_{k=1}^N\epsilon_k$, $\epsilon_{jk}=\epsilon_j+\epsilon'_k$, etc. For the
momentum integral one gets
%
\begin{align}
%
\pi\delta_{NN'} \sigma^{i(\mathcal V-\mathcal V') - \mathcal E -\mathcal E'} \Gamma(i(\mathcal V'-\mathcal V)+\mathcal E +\mathcal E'),
%
\end{align}
%
where $\Gamma$ is  Euler's gamma function. Thus
%
\begin{align}
%
\Vert \varphi_\chi\Vert^2_{\mathds H_N^{B,\sigma}}  & =
\lim_{\sigma\to 0}  \lim_{\epsilon,\epsilon'\to 0^+} \int_{\mathscr D^{\sigma}_{N}\times \mathscr
D^{\sigma}_{N}}
(-1)^{\sum_{k<j}[i(x'_k -x'_j)]}\delta_{NN'} \sigma^{i(\mathcal V-\mathcal V')}
\prod_{k,j=1}^N \boldsymbol\Gamma[i(x'_k-x_j)+\epsilon_{jk}]
\notag\\
&\quad\Gamma(1+i(\mathcal V-\mathcal V'))
\frac{\chi(x)}{\left(\prod_{j=1}^{N}\vartheta_N(x_j)\right)}\!
\left(\frac{\chi(x')}{\left(\prod_{j=1}^{N}\vartheta_N(x'_j)\right)}\right)^\dagger\!
 d\mu^A_N(x) d\mu^A_N(x'),
%
\end{align}
%
where  we put $\epsilon_k,\epsilon'_k=0$ in all nonsingular factors. The analysis of this integral in the $\sigma, \epsilon,\epsilon'\to0$
limit is  exactly the same as in Theorem~\ref{theorem}, see discussion around Eq.~\eqref{Iomega=0}, and  results in
%
\begin{align}\label{mapAnorm}
%
\Vert \Phi_\chi\Vert^2_{\mathds H_N} & =
\Vert \varphi_\chi\Vert^2_{\mathds H_N^{B,\sigma}} = \int_{\mathscr D^{\sigma}_{N}} \vert\varphi(x)\vert^2 d\mu^A_N(x).
%
\end{align}
%
%
Since the space of the functions~\eqref{chidef} is dense in $\mathds H_N^{A,\sigma}$ the relation~\eqref{mapAnorm} extends to the whole
Hilbert space. $\blacksquare$


Finally, we formulate the analog of  Theorem~\ref{theorem} for the map $\mathrm T_N^A$.
%
\begin{thm}\label{theoremA}

The map $\mathrm T_N^A$ defined in Eq.~\eqref{TA} can be extended to the linear bijective isometry of the Hilbert spaces, $\mathbb
H_N^{A,\sigma} \mapsto \mathbb H_N$, i.e.
%
\begin{subequations}
%
%
\begin{align}\label{unit-AA}
%
\|\mathrm T_{N}^{A}\chi\|_{\mathds H_N}^2 &=\|{\varphi}\|_{\mathds H_N^{A,\sigma}}^2
\\
\intertext{and}
\label{unit-RangeA}
\mathcal R\left( \mathrm T_{N}^{A}\right)
&= \mathds H_N.
%
\end{align}
\end{subequations}

\end{thm}

\noindent {\bf Proof:} As in the Theorem~\ref{theorem} we only need to prove Eq.~\eqref{unit-RangeA}. As was discussed in
%Theorem~\ref{theorem}
earlier Eq.~\eqref{unit-RangeA} is equivalent to the statement  that $\ker (\mathrm T_N^A)^*=0$ or to the assertion $\ker \mathbf S_N=0$,
where $\mathbf S_N= (\mathrm T_N^A)^* \mathrm T_N^B$. In order to prove this it suffices to show that $\Vert \mathbf S_N
\varphi\Vert_{\mathds H^{A,\sigma}_N} = \Vert  \varphi\Vert_{\mathds H^{B,\sigma}_N}$. The proof of this statement repeats step by step the
proof given in the Theorem~\ref{theorem}, and on the technical level is reduced  to the evaluation of the integral~\eqref{Jomega}.
$\blacksquare$


%%%%%%%%%%%%%%%%%%%%%%%%%%%%%%%%%%%%%%%%%%%%%%%%%%%%%%%%%%%%%%%%%%%%%%%%%%%%%%%%%%%%%%%%%%%%%%%%%%%%%%%%%%%%%%%%%%%%%%%%%%%%%%%%%%%%%%%%%%%%%
\section{Summary}\label{sect:summary}
%%%%%%%%%%%%%%%%%%%%%%%%%%%%%%%%%%%%%%%%%%%%%%%%%%%%%%%%%%%%%%%%%%%%%%%%%%%%%%%%%%%%%%%%%%%%%%%%%%%%%%%%%%%%%%%%%%%%%%%%%%%%%%%%%%%%%%%%%%%%%

In this work we consider a generic inhomogeneous $\mathrm{SL}(2,\mathbb C)$ spin chain with impurities and construct the eigenfunctions of
the $B$ and $A$ entries of the monodromy matrix. We prove the unitarity of the SoV transform associated with these systems or,
equivalently, the completeness of the corresponding systems in the Hilbert space of the model. Namely, the following identities hold in the
sense of distributions,
%
%
\begin{align}
%
\int_{\mathbb R^2} \int_{\mathscr D^\sigma_{N-1}} \Psi^{(N)}_{p,x}(z)(\Psi^{(N)}_{p,x}(z'))^\dagger\,
d^2p \,d\mu_N^B(x) &=\prod_{k=1}^N \delta^2(z_k-z'_k),
\notag\\
\int_{\mathscr D^\sigma_N} \Phi^{(N)}_x(z)(\Phi^{(N)}_{x}(z'))^\dagger d\mu_N^A(x) &=\prod_{k=1}^N \delta^2(z_k-z'_k),
%
\end{align}
%
and
\begin{align}
\int_{\mathbb C^N} \Psi^{(N)}_{p,x}(z) (\Psi^{(N)}_{p',x'}(z))^\dagger \prod_{k=1}^N d^2z_k  &=
 (\mu^B_N(x))^{-1}\,\delta^{2}(p-p') \delta^{N-1}(x, x'),
%
\notag\\
\int_{\mathbb C^N} \Phi^{(N)}_x(z) (\Phi^{(N)}_{x'}(z))^\dagger \prod_{k=1}^N d^2z_k  &=
 (\mu^A_N(x))^{-1}\, \delta^N(x, x'),
\end{align}
where
%
\begin{align}\label{deltaN}
%
\delta^N(x, x')&=\frac1{N!}\sum_{w\in S_N} \delta^N(x'-wx) \quad, \qquad w x=(x_{w_1},\dots, x_{w_N} )
%
\end{align}
%
and $\delta^N(x'-x)=\prod_{k=1}^N \delta^2(x'_k-x_k)$, $\delta^2(x'-x)=\delta_{nn'}\delta(\nu-\nu')$.

The method  relies heavily on the use of multidimensional Mellin-Barnes integrals which generalize integrals calculated  by
R.~A.~Gustafson~\cite{Gustafson94}. The attractive feature of our approach is that it does not depends on the details of the spin chain
such as spins and inhomogeneity parameters. We believe that this technique can also be used to prove the unitarity of the SoV transform for
the open $\mathrm {SL} (2,\mathbb C)$ spin chain.


%%%%%%%%%%%%%%%%%%%%%%%%%%%%%%%%%%%%%%%%%%%%%%%%%%%%%%%%%%%%%%%%%%%%%%%%%%%%%%%%%%%%%%%%%%%%%%%%%%%%%%%%%%%%%%%%%%%%%%%%%%%%%%%%%%%%%%%%%%%%%
%%%%%%%%%%%%%%%%%%%%%%%%%%%%%%%%%%%%%%%%%%%%%%%%%%%%%%%%%%%%%%%%%%%%%%%%%%%%%%%%%%%%%%%%%%%%%%%%%%%%%%%%%%%%%%%%%%%%%%%%%%%%%%%%%%%%%%%%%%%%%
\subsection*{Acknowledgements}
%%%%%%%%%%%%%%%%%%%%%%%%%%%%%%%%%%%%%%%%%%%%%%%%%%%%%%%%%%%%%%%%%%%%%%%%%%%%%%%%%%%%%%%%%%%%%%%%%%%%%%%%%%%%%%%%%%%%%%%%%%%%%%%%%%%%%%%%%%%%%
The author is grateful to S. \'{E}. Derkachov for fruitful discussions and T. A. Sinkevich for critical remarks.
%This work  was supported by the DFG grant for the Research Unit FOR 2926 and the DFG grants MO 1801/4-1.

%%%%%%%%%%%%%%%%%%%%%%%%%%%%%%%%%%%%%%%%%%%%%%%%%%%%%%%%%%%%%%%%%%%%%%%%%%%%%%%%%%%%%%%%%%%%%%%%%%%%%%%%%%%%%%%%%%%%%%%%%%%%%%%%%%%%%%%%%%%%%
\appendix

\newcommand{\appsection}[1]{\let\oldthesection\thesection
  \renewcommand{\thesection}{Appendix \oldthesection}
  \section{#1}\let\thesection\oldthesection
  }
%%%%%%%%%%%%%%%%%%%%%%%%%%%%%%%%%%%%%%%%%%%%%%%%%%%%%%%%%%%%%%%%%%%%%%%%%%%%%%%%%%%%%%%%%%%%%%%%%%%%%%%%%%%%%%%%%%%%%%%%%%%%%%%%
\appsection{The diagram technique}\label{sect:Diagram}
%%%%%%%%%%%%%%%%%%%%%%%%%%%%%%%%%%%%%%%%%%%%%%%%%%%%%%%%%%%%%%%%%%%%%%%%%%%%%%%%%%%%%%%%%%%%%%%%%%%%%%%%%%%%%%%%%%%%%%%%%%%%%%%%

Throughout this paper we used a diagrammatic representation for  the functions under consideration.  The calculation of relevant scalar
products is, most conveniently, performed diagrammatically   with the help of a few simple identities. Below we give some of these rules
(see also ref.~\cite{Derkachov:2001yn}).
%

%
\begin{enumerate}[label= (\roman*)]

\item  An arrow  with the index $\alpha$ directed from $w$ to~$z$ stands  for  a propagator $D_\alpha(z-w)=[z-w]^{-\alpha}$ :

\centerline{\includegraphics[width=0.38\linewidth]{App}}
%
\item The Fourier transform reads
%
\begin{align}\label{Fourier}
\int d^2 z e^{i(pz+\bar p\bar z)}D_\alpha(z)=\pi\, {i^{\alpha-\bar\alpha}}a(\alpha)\, D_{1-\alpha}(p)\,,
\end{align}
%
where the function $a(\alpha)\equiv 1/\boldsymbol\Gamma[\alpha]=\Gamma(1-\bar\alpha)/\Gamma(\alpha)$. %%
%%
%

\item Chain rule
%
\begin{align}\label{Chain}
\int\frac{ d^2 w}{[z_1-w]^\alpha [w-z_2]^{\beta}}=
\pi
\frac{a(\alpha,\beta)}{a(\gamma)}
\frac{1}{[z_1-z_2]^{\gamma}}\,,
\end{align}
%
where $\gamma=\alpha+\beta-1$. Its  diagrammatic form is\\[2mm]

\centerline{\includegraphics[width=0.42\linewidth]{Chain}}


%
\item Star-triangle relation
%
\begin{equation}
%
\vcenter{\hbox{
\includegraphics[width=0.57\linewidth]{StarTriangle}
}}
%
%
\end{equation}
%
\item Exchange relation
\begin{equation}
\label{exchange-rel}
\vcenter{\hbox{\includegraphics[width=0.57\linewidth]{Cross}}}
\end{equation}
where $\alpha+\beta=\alpha'+\beta'$.

%
\end{enumerate}
%




%%%%%%%%%%%%%%%%%%%%%%%%%%%%%%%%%%%%%%%%%%%%%%%%%%%%%%%%%%%%%%%%%%%%%%%%%%%%%%%%%%%%%%%%%%%%%%%%%%%%%%%%%%%%%%%%%%%%%%%%%%%%%%%%%%%%%%%%%%%%%
\appsection{Scalar products}\label{app:scalarproduct}
%%%%%%%%%%%%%%%%%%%%%%%%%%%%%%%%%%%%%%%%%%%%%%%%%%%%%%%%%%%%%%%%%%%%%%%%%%%%%%%%%%%%%%%%%%%%%%%%%%%%%%%%%%%%%%%%%%%%%%%%%%%%%%%%%%%%%%%%%%%%%
%


Here we discuss  the calculation of scalar products of $\Psi^{(N),\epsilon}_{p,x}$ and $\Phi_{x}^{(N)}$ functions. The diagrams for the
scalar products~\eqref{BBproduct}, \eqref{ABproduct} are shown in Fig.~\ref{diag:scalarproducts}. The leftmost vertex on both diagrams has
only two propagators attached to it. We call such a vertex -- free vertex. On the first step one integrates  over the free vertex (on both
diagrams) using the chain relation for  propagators and move the  resulting line to the right with the help of the exchange relation. After
that  two new  free vertices appear and one repeat the same procedure again. In this way one can integrate over all vertices on the left
edge of both diagrams (they are shown by black blobs). Keeping trace of all factors arising in the process one represent the initial
diagram $D$ as
%
\begin{align}\label{firststep}
%
D_N\big(\{x_1,x_2,\ldots\}, \{y_1, y_2, \ldots\}, \{\gamma_1, \gamma_2,\ldots\} \big) = f(x_1,y_1,\gamma)\,
D_N^\prime \big (\{x_2,\ldots\}, \{y_2,\ldots\}, \{\gamma_3,\ldots\} \big ).
%
\end{align}
%
Taking into account that  the function $\Psi_{p,x}^{(N)}$ and $\Phi_x^{(N)}$ are symmetric functions of the separated variables it follows
from \eqref{firststep} that
%
\begin{align}\label{DCN}
%
D_N\big(\{x_1,x_2,\ldots\}, \{y_1, y_2, \ldots\}, \{\gamma_1, \gamma_2,\ldots\} \big) = C_N(\gamma) \prod_{k,j} f(x_k,y_j,\gamma)\,.
%
\end{align}
%
The factor $C_N(\gamma)$ does not depend on $x,y$ variables. The easiest way to fix it is to evaluate both sides of \eqref{DCN} for special
values of $ x, y$. For example, one can take $x_k\to x$ and $y_k\to \bar x^*$. Both sides, in this limits, contain divergent factors,
$\boldsymbol\Gamma[i(\bar y^*_j- x_k)]$ which cancel out. It is easy check that the result of the integration over any free vertex in this
limit (after removing this singular factor) gives one. Therefore the equation on $\mathrm C_N(\gamma)$ for the scalar
product~\eqref{twoform} takes the form
%
\begin{align}
%
1=\mathrm C_N(\gamma)   (\chi( x) (\bar \chi(\bar x^*))^*)^{N-1}=\mathrm C_N(\gamma)\,
(-1)^{(N-1)\sum_{k=0}^{N-3}[\gamma_{2N-3-k}^{(k)}-ix]}\,.
%
\end{align}
%
Since $[\gamma_m^{(k)}-ix]$ is an integer number one gets that $\mathrm C_N=1$ for odd $N$, while for even $N$
\begin{align}
\sum_{k=0}^{N-3}[\gamma_{2N-3-k}^{(k)}-ix] & =\sum_{k=0}^{N-3}([\gamma_{2N-3-k}^{(k)} -\gamma_N^{(N-3)}]+[\gamma_N^{(N-3)}-ix])
\notag\\
&=\sum_{k=0}^{N-3}[\gamma_{2N-3-k}^{(k)} -\gamma_N^{(N-3)}] +(N-2)[\gamma_N^{(N-3)} - ix].
\end{align}
Taking into account that the last term in the above equation  is an even number one gets that $\mathrm C_N(\gamma)$ is given by the
expression~\eqref{BBCN}. For the second diagram the analysis follows exactly the same lines.

%
\begin{figure}[t]
%
%
\begin{center}
%
\includegraphics[width=0.87\linewidth]{scABB}
%
\end{center}
%
\caption{Examples of diagrams for scalar products,
Eqs.~\eqref{BBproduct}, \eqref{ABproduct} for $N=4$.
}
\label{diag:scalarproducts}
%
\end{figure}

%%%%%%%%%%%%%%%%%%%%%%%%%%%%%%%%%%%%%%%%%%%%%%%%%%%%%%%%%%%%%%%%%%%%%%%%%%%%%%%%%%%%%%%%%%%%%%%%%%%%%%%%%%%%%%%%%%%%%%%%%%%%%%%%%%%%%%%%%%%%%
\appsection{Gustafson's integral reduction}\label{app:gustafson}
%%%%%%%%%%%%%%%%%%%%%%%%%%%%%%%%%%%%%%%%%%%%%%%%%%%%%%%%%%%%%%%%%%%%%%%%%%%%%%%%%%%%%%%%%%%%%%%%%%%%%%%%%%%%%%%%%%%%%%%%%%%%%%%%%%%%%%%%%%%%%
The extension of the first Gustafson integral~\cite[Theorem 5.1]{Gustafson94} to the complex case was obtained in~\cite{Derkachov20}. It
takes the form
%
\begin{align}\label{GustafsonI}
%
\prod_{j=1}^N\sum_{n_j\in\mathbb{Z}+\frac{\sigma}{2}}
\int_{-i\infty}^{i\infty}
\frac{\prod_{m=1}^{N+1}\prod_{k=1}^{N}
\boldsymbol{\Gamma}(z_m-u_k)\boldsymbol{\Gamma}(u_k+w_m)}{ \prod_{m<j}
\boldsymbol{\Gamma}(u_m- u_j)\boldsymbol{\Gamma}(u_j- u_m)}
\prod_{p=1}^{N}
\frac{d\nu_p}{2\pi i}
=\frac{N!\prod_{k,j=1}^{N+1}\boldsymbol{\Gamma}(z_k+ w_j)}{
\boldsymbol{\Gamma}\left(\sum_{k=1}^{N+1} (z_k+w_k) \right)},
%
\end{align}
%
where $\boldsymbol \Gamma$ is the Gamma function of the  complex field $\mathbb C$~\cite{MR2125927}
%
\begin{align}
%
\boldsymbol\Gamma(u) \equiv \boldsymbol\Gamma(u,\bar u) = \frac{\Gamma(u)}{\Gamma(1-\bar u)}=\frac1{a(u)}.
%
\end{align}
%
The variables $u_k,w_m, z_m$ have the form
%
\begin{align*}
%
u_k&=\frac{n_k}2+ \nu_k, & z_m&=\frac{n_m}2+ x_m, & w_m&=\frac{\ell_m}2+ y_m\,,
\notag\\
\bar u_k&=-\frac{n_k}2+ \nu_k, & \bar z_m&=-\frac{n_m}2+ x_m, & \bar w_m&=-\frac{\ell_m}2+ y_m\,.
%
\end{align*}
and the integration contours over $\nu_k$ separate the series of  poles associated with the $\boldsymbol \Gamma$--functions:
$\boldsymbol\Gamma(z_m-u_k)$ and $\boldsymbol{\Gamma}(u_k+w_m)$, see ref.~\cite{Derkachov20} for more detail. The integral converges for
$\sum_{m=1}^{N+1} \mathrm{Re}(z_m+w_m) <1$.

Let us put
%
\begin{align}
%
z_{N+1} & =M\left(\frac12+i x\right), & \bar z_{N+1}  & =M\left(-\frac12+i x\right),
\notag\\
w_{N+1} & =M'\left(\frac12+i x'\right), & \bar w_{N+1} & =M'\left(-\frac12+i x'\right)
%
\end{align}
%
and send $M,M'\to\infty$ keeping $M/M'=\xi$ fixed, so that $w_{N+1}/z_{N+1}\mapsto \zeta$ and  $\bar w_{N+1}/\bar z_{N+1}\mapsto \bar
\zeta$.

Dividing both sides of~\eqref{GustafsonI} by $(\boldsymbol \Gamma(z_{N+1})\boldsymbol \Gamma(w_{N+1}))^N$ we get in this limit %%
%
\begin{align}\label{GustafsonII}
%
\frac1{N!}\prod_{j=1}^N\sum_{n_j\in\mathbb{Z}+\frac{\sigma}{2}}
\int_{-i\infty}^{i\infty} [\zeta]^U
\frac{\prod_{m,k=1}^{N}
\boldsymbol{\Gamma}(z_m-u_k)\boldsymbol{\Gamma}(u_k+w_m)}{ \prod_{m<j}
\boldsymbol{\Gamma}(u_m- u_j)\boldsymbol{\Gamma}(u_j- u_m)}
\prod_{p=1}^{N}\frac{d\nu_p}{2\pi i}
=\frac{[\zeta]^Z}{[1+\zeta]^{Z+W}}\prod_{k,j=1}^{N}\boldsymbol{\Gamma}(z_k+ w_j),
%
\end{align}
%
where $|\arg \zeta|<\pi$, $Z=\sum_{k=1}^N z_k$,  $W=\sum_{k=1}^N w_k$ and we recall that $[\zeta]^U\equiv \zeta^U \bar\zeta^{\bar U}$.

%%%%%%%%%%%%%%%%%%%%%%%%%%%%%%%%%%%%%%%%%%%%%%%%%%%%%%%%%%%%%%%%%%%%%%%%%%%%%%%%%%%%%%%%%%%%%%%%%%%%%%%%%%%%%%%%%%%%%%%%%%%%%%%%%%%%%%%%%%%%%
%\bibliography{ref_sov}
%\bibliographystyle{ieeetr.bst}

\begin{thebibliography}{10}

\bibitem{MR549615} E.~K. Sklyanin, L.~A. Tahtad\v{z}jan, and L.~D. Faddeev, ``Quantum inverse
  problem method. {I},'' {\em Teoret. Mat. Fiz.}, vol.~40, no.~2, pp.~194--220,
  1979.

\bibitem{MR802110} E.~K. Sklyanin, ``The quantum {T}oda chain,'' in {\em Nonlinear equations in
  classical and quantum field theory ({M}eudon/{P}aris, 1983/1984)}, vol.~226
  of {\em Lecture Notes in Phys.}, pp.~196--233, Springer, Berlin, 1985.

\bibitem{MR1239668} E.~K. Sklyanin, ``Quantum inverse scattering method. {S}elected topics,'' in
  {\em Quantum group and quantum integrable systems}, Nankai Lectures Math.
  Phys., pp.~63--97, World Sci. Publ., River Edge, NJ, 1992.

\bibitem{MR677006} V.~E. Korepin, ``Calculation of norms of {B}ethe wave functions,'' {\em Comm.
  Math. Phys.}, vol.~86, no.~3, pp.~391--418, 1982.

\bibitem{MR1007797} N.~A. Slavnov, ``Calculation of scalar products of wave functions and
  form-factors in the framework of the algebraic {B}ethe ansatz,'' {\em Teoret.
  Mat. Fiz.}, vol.~79, no.~2, pp.~232--240, 1989.

\bibitem{MR763763} A.~G. Izergin and V.~E. Korepin, ``The quantum inverse scattering method
  approach to correlation functions,'' {\em Comm. Math. Phys.}, vol.~94, no.~1,
  pp.~67--92, 1984.

\bibitem{MR1741654} N.~Kitanine, J.~M. Maillet, and V.~Terras, ``Correlation functions of the
  {$XXZ$} {H}eisenberg spin-{${1\over2}$} chain in a magnetic field,'' {\em
  Nuclear Phys. B}, vol.~567, no.~3, pp.~554--582, 2000.

\bibitem{MR626693} M.~C. Gutzwiller, ``The quantum mechanical {T}oda lattice. {II},'' {\em Ann.
  Physics}, vol.~133, no.~2, pp.~304--331, 1981.

\bibitem{MR1751619} S.~Kharchev and D.~Lebedev, ``Integral representation for the eigenfunctions of
  a quantum periodic {T}oda chain,'' {\em Lett. Math. Phys.}, vol.~50, no.~1,
  pp.~53--77, 1999.

\bibitem{MR1831292} S.~Kharchev and D.~Lebedev, ``Integral representations for the eigenfunctions
  of quantum open and periodic {T}oda chains from the {QISM} formalism,''
  vol.~34, pp.~2247--2258, 2001.
\newblock Kowalevski Workshop on Mathematical Methods of Regular Dynamics
  (Leeds, 2000).

\bibitem{MailletNiccoli18} J.~M. Maillet and G.~Niccoli, ``On quantum separation of variables,'' {\em J.
  Math. Phys.}, vol.~59, no.~9, pp.~091417, 47, 2018.

\bibitem{MailletNiccoli19} J.~M. Maillet and G.~Niccoli, ``On quantum separation of variables beyond
  fundamental representations,'' {\em SciPost Phys.}, vol.~10, no.~2, pp.~Paper
  No. 026, 38, 2021.

\bibitem{RyanVolin19} P.~Ryan and D.~Volin, ``Separated variables and wave functions for rational
  {${gl}(N)$} spin chains in the companion twist frame,'' {\em J. Math. Phys.},
  vol.~60, no.~3, pp.~032701, 23, 2019.

\bibitem{GromovSizov17} N.~Gromov, F.~Levkovich-Maslyuk, and G.~Sizov, ``New construction of
  eigenstates and separation of variables for {${\rm SU}(N)$} quantum spin
  chains,'' {\em J. High Energy Phys.}, no.~9, pp.~111, front matter+39, 2017.

\bibitem{GromovRyan20} N.~Gromov, F.~Levkovich-Maslyuk, and P.~Ryan, ``Determinant form of correlators
  in high rank integrable spin chains via separation of variables,'' {\em J.
  High Energy Phys.}, no.~5, pp.~Paper No. 169, 79, 2021.

\bibitem{Derkachov:2001yn} S.~E. Derkachov, G.~P. Korchemsky, and A.~N. Manashov, ``Noncompact
  {H}eisenberg spin magnets from high-energy {QCD}. {I}. {B}axter
  {$Q$}-operator and separation of variables,'' {\em Nuclear Phys. B},
  vol.~617, no.~1-3, pp.~375--440, 2001.

\bibitem{Derkachov:2002tf} S.~E. Derkachov, G.~P. Korchemsky, and A.~N. Manashov, ``Separation of
  variables for the quantum {${\rm SL}(2,\Bbb R)$} spin chain,'' {\em J. High
  Energy Phys.}, no.~7, pp.~047, 30, 2003.

\bibitem{Derkachov:2003qb} S.~E. Derkachov, G.~P. Korchemsky, and A.~N. Manashov, ``Baxter {$\Bbb
  Q$}-operator and separation of variables for the open {${\rm SL}(2,{\Bbb
  R})$} spin chain,'' {\em J. High Energy Phys.}, no.~10, pp.~053, 31, 2003.

\bibitem{BytskoTeschner06} A.~G. Bytsko and J.~Teschner, ``Quantization of models with non-compact quantum
  group symmetry: modular {$XXZ$} magnet and lattice sinh-{G}ordon model,''
  {\em J. Phys. A}, vol.~39, no.~41, pp.~12927--12981, 2006.

\bibitem{Silantyev} A.~V. Silant'ev, ``Transition function for the {T}oda chain,'' {\em Teoret.
  Mat. Fiz.}, vol.~150, no.~3, pp.~371--390, 2007.

\bibitem{MR3230255} S.~E. Derkachov and A.~N. Manashov, ``Iterative construction of eigenfunctions
  of the monodromy matrix for {$\mathrm{SL}(2,\Bbb C)$} magnet,'' {\em J. Phys.
  A}, vol.~47, no.~30, pp.~305204, 25, 2014.

\bibitem{Semenov-Tian-Shansky1994} M.~A. Semenov-Tian-Shansky, {\em Quantization of Open Toda Lattices},
  pp.~226--259.
\newblock Berlin, Heidelberg: Springer Berlin Heidelberg, 1994.

\bibitem{Wallach92} N.~R. Wallach, {\em Real reductive groups. {II}}, vol.~132 of {\em Pure and
  Applied Mathematics}.
\newblock Academic Press, Inc., Boston, MA, 1992.

\bibitem{Kozlowski15} K.~K. Kozlowski, ``Unitarity of the {S}o{V} transform for the {T}oda chain,''
  {\em Comm. Math. Phys.}, vol.~334, no.~1, pp.~223--273, 2015.

\bibitem{DerkachovKozlowskiManashov19} S.~E. Derkachov, K.~K. Kozlowski, and A.~N. Manashov, ``On the separation of
  variables for the modular {XXZ} magnet and the lattice sinh-{G}ordon
  models,'' {\em Ann. Henri Poincar\'{e}}, vol.~20, no.~8, pp.~2623--2670,
  2019.

\bibitem{DerkachovManashov17} S.~E. Derkachov and A.~N. Manashov, ``Spin chains and {G}ustafson's
  integrals,'' {\em J. Phys. A}, vol.~50, no.~29, pp.~294006, 20, 2017.

\bibitem{Gustafson92} R.~A. Gustafson, ``Some {$q$}-beta and {M}ellin-{B}arnes integrals with many
  parameters associated to the classical groups,'' {\em SIAM J. Math. Anal.},
  vol.~23, no.~2, pp.~525--551, 1992.

\bibitem{Gustafson94} R.~A. Gustafson, ``Some {$q$}-beta and {M}ellin-{B}arnes integrals on compact
  {L}ie groups and {L}ie algebras,'' vol.~341, no.~1, pp.~69--119, 1994.

\bibitem{DerkachovKozlowskiManashov21} S.~E. Derkachov, K.~K. Kozlowski, and A.~N. Manashov, ``Completeness of {S}o{V}
  representation for {$\mathrm{SL}(2,\mathbb R)$} spin chains,'' {\em SIGMA
  Symmetry Integrability Geom. Methods Appl.}, vol.~17, pp.~Paper No. 063, 26,
  2021.

\bibitem{Lipatov:1993yb} L.~N. Lipatov, ``{Asymptotic behavior of multicolor QCD at high energies in
  connection with exactly solvable spin models},'' {\em JETP Lett.}, vol.~59,
  pp.~596--599, 1994.

\bibitem{Lipatov:1993qn} L.~N. Lipatov, ``{High-energy asymptotics of multicolor QCD and two-dimensional
  conformal field theories},'' {\em Phys. Lett. B}, vol.~309, pp.~394--396,
  1993.

\bibitem{Faddeev:1994zg} L.~D. Faddeev and G.~P. Korchemsky, ``{High-energy QCD as a completely
  integrable model},'' {\em Phys. Lett. B}, vol.~342, pp.~311--322, 1995.

\bibitem{Lipatov:2009nt} L.~N. Lipatov, ``{Integrability of scattering amplitudes in N=4 SUSY},'' {\em
  J. Phys. A}, vol.~42, p.~304020, 2009.

\bibitem{Bartels:2011nz} J.~Bartels, L.~N. Lipatov, and A.~Prygarin, ``{Integrable spin chains and
  scattering amplitudes},'' {\em J. Phys. A}, vol.~44, p.~454013, 2011.

\bibitem{DerkachovKazakovOlivucci19} S.~Derkachov, V.~Kazakov, and E.~Olivucci, ``Basso-{D}ixon correlators in
  two-dimensional fishnet {CFT},'' {\em J. High Energy Phys.}, no.~4, pp.~032,
  31, 2019.

\bibitem{DerkachovOlivucci20} S.~Derkachov and E.~Olivucci, ``Exactly solvable magnet of conformal spins in
  four dimensions,'' {\em Phys. Rev. Lett.}, vol.~125, no.~3, pp.~031603, 7,
  2020.

\bibitem{DerkachovOlivucci21} S.~Derkachov and E.~Olivucci, ``Exactly solvable single-trace four point
  correlators in {$\chi \rm CFT_4$},'' {\em J. High Energy Phys.}, no.~2,
  pp.~Paper No. 146, 84, 2021.

\bibitem{Derkachov20} S.~E. Derkachov and A.~N. Manashov, ``On complex gamma-function integrals,''
  {\em SIGMA Symmetry Integrability Geom. Methods Appl.}, vol.~16, pp.~Paper
  No. 003, 20, 2020.

\bibitem{MR0207913} I.~M. Gel'fand, M.~I. Graev, and N.~Y. Vilenkin, {\em Generalized functions.
  {V}ol. 5: {I}ntegral geometry and representation theory}.
\newblock Translated from the Russian by Eugene Saletan, Academic Press, New
  York-London, 1966.

\bibitem{MR562799} L.~A. Tahtad\v{z}jan and L.~D. Faddeev, ``The quantum method for the inverse
  problem and the {$XYZ$} {H}eisenberg model,'' {\em Uspekhi Mat. Nauk},
  vol.~34, no.~5(209), pp.~13--63, 256, 1979.

\bibitem{MR671263} P.~P. Kulish and E.~K. Sklyanin, ``Quantum spectral transform method. {R}ecent
  developments,'' in {\em Integrable quantum field theories ({T}v\"{a}rminne,
  1981)}, vol.~151 of {\em Lecture Notes in Phys.}, pp.~61--119, Springer,
  Berlin-New York, 1982.

\bibitem{MR1616371} L.~D. Faddeev, ``How the algebraic {B}ethe ansatz works for integrable
  models,'' in {\em Sym\'{e}tries quantiques ({L}es {H}ouches, 1995)},
  pp.~149--219, North-Holland, Amsterdam, 1998.

\bibitem{MR2125927} I.~M. Gel'fand, M.~I. Graev, and V.~S. Retakh, ``Hypergeometric functions over
  an arbitrary field,'' {\em Uspekhi Mat. Nauk}, vol.~59, no.~5(359),
  pp.~29--100, 2004.

\bibitem{Derkachov:1999pz} S.~E. Derkachov, ``Baxter's {$Q$}-operator for the homogeneous {$XXX$} spin
  chain,'' {\em J. Phys. A}, vol.~32, no.~28, pp.~5299--5316, 1999.

\end{thebibliography}


%\bibliographystyle{sigma.bst}
\end{document}
