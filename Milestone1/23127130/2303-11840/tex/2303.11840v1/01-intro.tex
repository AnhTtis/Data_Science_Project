\section{Introduction}
Facial expression is an important indicator of our psychological and physical state, making facial expression recognition (FER) a subject of concern. Due to high similarities across different expressions \cite{2021Feature}, disturbing factors \cite{2020Ruan}, and micro-facial movements that are difficult to perceive \cite{WEI2021159}, accurate and reliable results of FER are difficult to achieve.
\par
%The previous methods have made significant efforts on FER, most of them pay attention to one of the two orientations: disturbance-disentangling and effective features extracting. 

In recent years, deep learning has been widely applied in FER and has demonstrated impressive performance, which mainly focus on two orientations: disturbance-disentangling and effective feature extracting. 
The former one aims to eliminate the impact of disturbing factors such as illumination and identity \cite{2017Identity,2020Ruan,RAN}. The latter one focuses on the process of extracting expression features, trying to capture fine-grained features \cite{2021Feature}, transformed deep and shallow features \cite{BOUGOURZI2020113459}, expressive information \cite{2018Yang} or dynamic information \cite{2012Atlases,2014Liu,KHAN2017427}.      

\par
Despite the huge success and promise, existing FER methods generally ignore the fact that neutral information hidden in neutral facial expression is shared by both facial expression image and corresponding neutral expression image.
Here, neutral information includes initial state of expression and disturbance information from disturbing factors. Disturbance information can confuse networks during training stage, resulting in that the networks misjudge targets as other similar expressions, \emph{e.g.}, appearance characteristics such as wrinkles are usually mistaken for facial movements. Their redundancy will also reduce the weight of micro expression features, making micro-facial movements more difficult to perceive, \emph{e.g.}, the subtle motions of micro-facial expression are erased during the convolution process, so they play minor parts in the final features extracted by networks. 

\begin{figure}
\centering
\includegraphics[scale=0.6]{image/examples.jpg}
\caption{Examples to show the harm caused by disturbance information. Images are from the Oulu-CASIA dataset.}
\label{fig:Examples}
\end{figure}

\par

Figure \ref{fig:Examples} shows an example with harmful disturbance information. In the left side of good examples, we can see that bulging mouth is an important characteristic of angry expressions while squinting eyes and pouting are distinguished characteristics of disgust expressions. 
%Then, given the example with harmful disturbing factors, 
In the right side of the example with harmful disturbing factors, the young man's anger face is easily misidentified as “disgust” because of his “squinting eyes”. 
%So the anger face on the right can easily be misidentified as “disgust” because of its “closed eyes”. 
However, according to his neutral face, “squinting eyes” is not his facial movement but his identity information of small eyes. It can be seen that facial expression image (\emph{e.g.}, anger face) and neutral expression image from the same person share similar disturbance information.

\par 
In contrast to disturbance information, initial state of expression only exists in neutral expression, but plays a favorable role in FER. With its help, we can extract deviation information and capture features from facial movements (such as raising eyebrows and slightly widening eyes) that may be easily covered up by identity. 
\par
Inspired by the above observations, in this paper we propose a \underline{S}elf-\underline{P}aced \underline{N}eutral Expression-\underline{D}isentangled \underline{L}earning (SPNDL) model, which exploits neutral facial expression to remove disturbance information and obtain initial state of expression.
%we aim at exploiting neutral facial expression to remove disturbance information and obtain initial state of expression. To this end, in this paper we propose a \underline{S}elf-\underline{P}aced \underline{N}eutral Expression-\underline{D}isentangled \underline{L}earning (SPNDL) model. 
We first employ a backbone convolutional neutral network (CNN) to extract basic features of input paired images, consisting of the facial expression image (called target image) to be predicted and its corresponding neutral expression image (called reference image).
%Besides the facial expression image to be predicted, called target image by us, its corresponding neutral expression image is chosen as reference image. 
These paired images are input to the same backbone network simultaneously, outputting two paired basic features.
Because the target image and its reference image share similar disturbance information, a simple but effective subtraction operation is applied to these paired basic features to obtain neutral expression-disentangled features (NDFs).
As initial state of expression exists in reference image, and peak state of expression exists in target image, the obtained NDFs contain the deviation information. 
In particular, to alleviate the distribution difference between two paired basic features, we apply a normalization operation between the output features of target image and reference image in each layer of backbone CNN. 
\par
Deep learning based FER methods solve non-convex optimization problems and are easily stuck into poor local solutions.
To alleviate this issue, we propose a SPL strategy for FER, which learns samples with easy NDFs first, and then gradually adapts to hard ones. This strategy can reduce the impacts of low-quality samples and paired samples with inconsistently distributed NDFs.
The main contributions of this paper can be summarized as follows:
\begin{enumerate}[label=\arabic*), leftmargin=*]
    \item We propose a neutral expression-disentangled method to remove disturbance information from facial expressions and capture the deviation information of facial movement, which is low-cost and can effectively improve the performance of facial expression recognition.
     \item We propose a self-paced learning strategy to alleviate the non-convex optimization issue, and simultaneously suppress the negative impacts of inconsistently distributed NDFs and low-quality data.
     \item Our method has no hyperparameters beyond those of deep learning models, making it easy to be trained. We achieved 99.69$\%$, 90.14$\%$ and 89.08$\%$ recognition accuracies respectively on three popular databases (\emph{i.e.}, CK+, Oulu-CASIA, and RAF-DB), which outperforms the state-of-the-art FER methods on the first two databases and gains competitive performance on the last database.  
\end{enumerate}
