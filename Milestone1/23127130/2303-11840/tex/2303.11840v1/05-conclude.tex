\section{Conclusion and Future Work}
In this paper, we have proposed a self-paced neutral expression-disentangled learning (SPNDL) model, which is hyperparameter-free in the sense. Our method achieves state-of-the-art FER performance on the CK+ and Oulu-CASIA databases and competitive results on RAF-DB.
First, basic features of target image and the corresponding neutral image are extracted, both of which share similar disturbance information. To alleviate the inconsistently distributing problem of these basic features, we apply a normalization operation in each layer of the backbone CNN. Second, NDFs are obtained by a subtraction operation, which can suppress the negative impact of disturbing factors in facial expression images, and simultaneously introduce the information of expression's initial state. Finally, a SPL strategy is proposed in training stage, letting the network learn like human beings -- starting with easy NDFs and building up to complex ones. 

Our future work will focus on how to more effectively learn neutral expression-disentangled features when there are no corresponding neutral expression images, especially on large-scale in-the-wild databases where there are more disturbing factors.
%it is still hard to achieve state-of-the-art results.
%which means SPNDL does not work well on in-the-wild databases. 
%Thus, our future work will focus on extending our framework to large-scale databases where there are more disturbing factors and no corresponding neutral expression images.