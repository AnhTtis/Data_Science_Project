%%\documentclass[preprint,12pt]{elsarticle}

%% Use the option review to obtain double line spacing
 \documentclass[1+,12pt]{elsarticle}

%% Use the options 1p,twocolumn; 3p; 3p,twocolumn; 5p; or 5p,twocolumn
%% for a journal layout:
%% \documentclass[final,1p,times]{elsarticle}
%% \documentclass[final,1p,times,twocolumn]{elsarticle}
%% \documentclass[final,3p,times]{elsarticle}
%% \documentclass[final,3p,times,twocolumn]{elsarticle}
%% \documentclass[final,5p,times]{elsarticle}
%% \documentclass[final,5p,times,twocolumn]{elsarticle}

%% For including figures, graphicx.sty has been loaded in
%% elsarticle.cls. If you prefer to use the old commands
%%\usepackage{epsfig}

%% The amssymb package provides various useful mathematical symbols
\usepackage{amssymb}
%% The amsthm package provides extended theorem environments
\usepackage{amsthm}
\usepackage{enumitem}
\usepackage{booktabs}
\usepackage{bm}
\usepackage{multirow}
\usepackage{enumitem}
\usepackage{bbding}
\usepackage{makecell}
\usepackage{subcaption}
\usepackage{float}
\usepackage{amsmath}

%% The lineno packages adds line numbers. Start line numbering with
%% \begin{linenumbers}, end it with \end{linenumbers}. Or switch it on
%% for the whole article with \linenumbers.
%% \usepackage{lineno}

\begin{document}

\begin{frontmatter}

%% Title, authors and addresses

%% use the tnoteref command within \title for footnotes;
%% use the tnotetext command for theassociated footnote;
%% use the fnref command within \author or \address for footnotes;
%% use the fntext command for theassociated footnote;
%% use the corref command within \author for corresponding author footnotes;
%% use the cortext command for theassociated footnote;
%% use the ead command for the email address,
%% and the form \ead[url] for the home page:
%% \title{Title\tnoteref{label1}}
%% \tnotetext[label1]{}
%% \author{Name\corref{cor1}\fnref{label2}}
%% \ead{email address}
%% \ead[url]{home page}
%% \fntext[label2]{}
%% \cortext[cor1]{}
%% \affiliation{organization={},
%%             addressline={},
%%             city={},
%%             postcode={},
%%             state={},
%%             country={}}
%% \fntext[label3]{}



\title{Self-Paced Neutral Expression-Disentangled Learning for
Facial Expression Recognition\tnoteref{t1}}
\tnotetext[t1]{This work was supported in part by Sichuan Science and Technology Program (Nos. 2021YFS0172, 2022YFS0047, and 2022YFS0055), Medico-Engineering Cooperation Funds from University of Electronic Science and Technology of China (No. ZYGX2021YGLH022), and Opening Funds from Radiation Oncology Key Laboratory of Sichuan Province (No. 2021ROKF02). Lifang He was supported by Lehigh’s accelerator grant S00010293.}

%% use optional labels to link authors explicitly to addresses:
\author[1]{Zhenqian Wu}
\ead{russius@163.com}
\author[1]{Xiaoyuan Li}
\ead{lilan1196769290@163.com}
\author[1]{Yazhou Ren\corref{cor1}}
\ead{yazhou.ren@uestc.edu.cn}
\author[1]{Xiaorong Pu}
\ead{puxiaor@uestc.edu.cn}
\author[1]{Xiaofeng Zhu}
\ead{seanzhuxf@gmail.com}
\author[2]{Lifang He}
\ead{lih319@lehigh.edu}

\cortext[cor1]{Corresponding author}

\affiliation[1]{organization={School of Computer Science and Engineering, University of Electronic Science and Technology of China},
             city={Chengdu 611731},
             %%postcode={611731},
             country={China}}
             
\affiliation[2]{organization={Department of Computer Science and Engineering, Lehigh University},
             city={Bethlehem},
             %%postcode={18015},
             state={PA 18015},
             country={USA}}

%% \affiliation[label2]{organization={},
%%             addressline={},
%%             city={},
%%             postcode={},
%%             state={},
%%             country={}}


%%%%%%%%% ABSTRACT
\begin{abstract}
The accuracy of facial expression recognition is typically affected by the following factors: high similarities across different expressions, disturbing factors, and micro-facial movement of rapid and subtle changes. One potentially viable solution for addressing these barriers is to exploit the neutral information concealed in neutral expression images. To this end, in this paper we propose a \underline{S}elf-\underline{P}aced \underline{N}eutral Expression-\underline{D}isentangled \underline{L}earning (SPNDL) model. SPNDL disentangles neutral information from facial expressions, making it easier to extract key and deviation features. 
Specifically, it allows to capture discriminative information among similar expressions and perceive micro-facial movements. In order to better learn these neutral expression-disentangled features (NDFs) and to alleviate the non-convex optimization problem, a self-paced learning (SPL) strategy based on NDFs is proposed in the training stage. SPL learns samples from easy to complex by increasing the number of
samples selected into the training process, which enables to effectively suppress the negative impacts introduced by low-quality samples and inconsistently distributed NDFs. Experiments on three popular databases (\emph{i.e.}, CK+, Oulu-CASIA, and RAF-DB) show the effectiveness of our proposed method.

\end{abstract}

%%Graphical abstract
%\begin{graphicalabstract}
%\includegraphics{grabs}
%\end{graphicalabstract}

%%Research highlights
%\begin{highlights}
%\item Neutral information in neutral expression is utilized for %facial expression recognition.
%\item Deviation information is extracted from facial movements while %disturbance information is removed.
%\item A self-paced learning strategy based on neutral %expression-disentangled features is proposed.
%\item A variation of our proposed method is adopted for in-the-wild %databases. 
%\end{highlights}

\begin{keyword}
%% keywords here, in the form: keyword \sep keyword
Facial expression recognition  \sep Disturbance-disentangling \sep Self-paced learning \sep Feature extracting
%% PACS codes here, in the form: \PACS code \sep code

%% MSC codes here, in the form: \MSC code \sep code
%% or \MSC[2008] code \sep code (2000 is the default)

\end{keyword}

\end{frontmatter}

%% \linenumbers

%% main text
Humans are multimodal learners. We communicate with each other about things that we have experienced and knowledge we have gained using our senses---most commonly including sight as well as hearing, touch, smell, and taste. Our communication channel is limited to a single modality---spoken language, signed language, or text---but a reader or listener is expected to use his or her imagination to visualize and reason about the content being described. In general, language is used to describe scenes, events, and images; the words used to describe these are used to conjure up a visual impression in the listener. Therefore, it is natural to consider the types of visual reasoning used in understanding language, and to ask how well we can currently model them with computational methods.

Consider, for instance, the questions in Figure \ref{fig:teaser}. Concreteness is typically correlated with how well a concept can be visually imagined. For example, a concrete word such as \emph{present} often has a unique visual representation. In addition, common associations such as \emph{ocean}$\rightarrow$\emph{blue} (color) and \emph{corn chip}$\rightarrow$\emph{triangle} (shape) reflect properties of an imagined visual representation of the item in question. These properties may be difficult to infer from text alone without prior knowledge gained from visual input; for instance, a number of studies have investigated the partial ability of blind English speakers to predict color associations and how it differs from the intuition of sighted speakers\footnote{This phenomenon is illustrated in \href{https://www.youtube.com/watch?v=59YN8_lg6-U}{this interview} with Tommy Edison, a congenitally blind man, in which he describes his understanding and frequent confusion regarding color associations.}~\cite{van2021blind, saysani2021seeing, saysani2018colour, shepard1992representation, marmor1978age}. 

There has been a wealth of recent research vision-and-language (V\&L) tasks involving both text and image data, and the use of vision-language pretraining (VLP) to create models that are able to reason jointly about both of these modalities together~\cite{chen2020uniter,kim2021vilt,li2021align,chen2022vlp}. Notable in this regard is CLIP~\cite{radford2021learning}, consisting of paired text and image encoders jointly trained on a contrastive objective, that learns to align text and image embeddings in a shared semantic space. On the other hand, text encoder models such as BERT~\cite{devlin2018bert} learn to reason about text in a unimodal vacuum, with knowledge derived from pretraining tasks that only involve textual data.




Prior work has investigated the performance of multimodally trained text encoders on various natural language understanding (NLU) tasks with mixed results, sometimes finding that they are outperformed by unimodal models~\cite{iki2021effect} and at other times suggesting improved performance~\cite{wang2021simvlm}. However, these works fine-tune the models under consideration on NLU tasks before evaluation, making it difficult to disentangle the effects of multimodal pretraining and fine-tuning configuration on the observed performance. Additionally, these works do not address the distinction between NLU tasks requiring implicit visual reasoning and ones that are purely non-visual. We refer to natural language inference involving implicit visual reasoning as \emph{visual language understanding} (VLU) and propose a suite of VLU tasks that may be used to evaluate visual reasoning capabilities of pretrained text encoders, focusing primarily on zero-shot methods.

We compare multimodally trained text encoders such as that of CLIP to BERT and other unimodally trained text encoders, evaluating their performance on our suite of VLU tasks. We evaluate these models in without modifying their internal weights in order to probe their knowledge obtained during pretraining. A key design aspect of these tests is the probing method used to evaluate knowledge. Previous work has probed the knowledge of BERT and similar models using a masked language modelling (MLM) paradigm~\cite{petroni2019language, rogers2020primer}, but this cannot be directly applied to CLIP since it was not pretrained with MLM. We therefore propose a new zero-shot probing method that we term \emph{Stroop probing}. This is based on the psychological Stroop effect~\cite{macleod1991half} (described in Section \ref{sec:probing}), which suggests that salient items should have a stronger interference effect on the representation of their context.

Strikingly, we find that the multimodally trained text encoders under consideration outperform unimodally trained text encoders on VLU tasks, both when comparing to much larger encoders as well as ones of comparable size. We also compare these models on baseline NLU tasks that do not involve visual reasoning and find that models such as CLIP underperform on these tasks, demonstrating that they do not have a global advantage on NLU tasks. We conclude that exposure to images during pretraining improves performance on text-only tasks that require visual reasoning. Furthermore, our findings isolate the effect of the text component of multimodal models for tasks such as text to image generation, providing principled guidelines for understanding the knowledge that such models inject into downstream vision tasks.



\section{Related Work}
\label{sec:related}

We now briefly summarize prior work on standard video compression, learned image compression, and learned video compression.

\textbf{Standard Video Compressors:} Video compression has been attracting attention from the early 2000s due to increasing video content in live streaming and real-time communication. More recently, with the wide spread of COVID-19 across the world, we have seen the importance of reliable and fast communication of high-resolution video content for remote lectures and remote meetings. Standard video compression frameworks typically consist of transform coding, intra-prediction, motion prediction, motion compensation, and entropy coding blocks. The large number of blocks to be designed and optimized for the best rate-distortion performance resulted in many different video coding standards such as ISO/IEC MPEG series \cite{tudor1995mpeg, haskell1996digital, sikora1997mpeg, li2001overview}, ITU-T H.26x series \cite{wiegand2003overview, sullivan2012overview, sze2014high, sullivan2004h, vetro2011overview}, AVS series \cite{yu2009overview, ma2013overview, zhang2019recent}, VP9 \cite{mukherjee2013latest}, and AV1 \cite{chen2018overview, han2021technical}. In our experiments, we use H.264 as the standard video codec between the neural pre-processor and post-processor. Hence, the differential proxy in Figure~\ref{fig:diagram} is designed to best approximate H.264 while maintaining the computational efficiency. \berivan{revisit the baselines once we get the new results.} 

\textbf{Learned Image Compression:}
Since the early work on learned image compression \cite{toderici2015variable}, the rate-distortion performance of learned compressors have gradually outperformed the standard image codecs \cite{balle2016end, balle2016end2, balle2018efficient, balle2016end_pcs, balle2018variational, hu2021learning, minnen2018joint, toderici2017full, mentzer2020high}. This was achieved thanks to the non-linear transforms \cite{balle2020nonlinear} learned through the end-to-end optimization of the Lagrangian loss function $L(\theta)=D(\theta)+\lambda R(\theta)$, where $D(\theta)$ is an expected distortion, $R(\theta)$ is an expected rate, and $\lambda >0$ is a Lagrange multiplier, using a differentiable proxy for the quantizer, mostly modeled as additive uniform noise or bypassed through a straight-through estimator. While they exceed the performance of the standard image codec, learned image compressors require a much greater complexity. Sandwiched image compression \cite{guleryuz2021sandwiched} is one way of breaking the tension between the flexibility and the computational complexity that the neural networks bring. Having a standard image codec between two lightweight neural pre- and post-processor networks, sandwiched image model improved the rate-distortion performance over the standard codec used alone by learning non-linear transforms without increasing the computational complexity much compared to other learned image compression frameworks. Our experiment on compressing a video clip by applying the sandwiched image model on each frame individually verifies that sandwiched image model gives more than 25 dB better reconstructions than the image codec applied alone. See Figure~\ref{fig:intra-comparison} for the experiments with JPEG and H264 (intra codec version) used with grayscale coding. With the sandwiched video model, our goal is to achieve better rate-distortion points than the sandwiched image baseline by training the model with a carefully designed video codec proxy.


\begin{figure}
    \centering
    \subfigure[\em ]{\includegraphics[width=0.45\linewidth]{ICIP/figures/jpeg.png}}
    \subfigure[\em ]{\includegraphics[width=0.45\linewidth]{ICIP/figures/h264_intra.png}}
    \caption{Comparison of (a) JPEG vs. sandwiched JPEG and (b) H264-intra vs sandwiched H264-intra.}
    \label{fig:intra-comparison}
    \vspace{-0.2in}
\end{figure}

\textbf{Learned Video Compression:}
\berivan{Fernando's paper?}



\section{Preliminaries}
\noindent\textbf{Denoising Diffusion Probabilistic Models with classifier-free guidance:}
Diffusion models are probabilistic models that approximate the data distribution by iteratively adding noise and denoising through a forward/reverse Gaussian Diffusion Process~\citep{ddpm,song2020score}. The forward process applies noise at each time step $t\in{0,...,T}$ to the data distribution $\mathbf{x}_{0}$, creating a noisy sample $\mathbf{x}_t$ where $\mathbf{x}_t = \sqrt{\bar{\alpha}_t}\mathbf{x}_0+\sqrt{1-\bar{\alpha}_t}\bm{\epsilon}$ ($\bm{\epsilon}\sim\mathcal{N}(\boldsymbol{0},\boldsymbol{I})$), and $\bar{\alpha}_t$ is the accumulation of the noise schedule $\alpha_{0:T}$ defined by $\bar{\alpha}_t=\prod^t_{s=1}\alpha_s$. To denoise images, the diffusion process uses a reparameterized variant of Gaussian noise prediction $\bm{\epsilon}_\theta(\mathbf{x}_t,t)$ targeting Gaussian noise $\bm{\epsilon}$. The reverse process $p(\mathbf{x}_{t-1}|\mathbf{x}_{t})$ of the Markov Chain generates new samples from Gaussian noise, which is approximated by Bayes' theorem as $q(\mathbf{x}_{t-1}|\mathbf{x}_t,\mathbf{x}_0)$, where $\mathbf{x}_0$ is derived from the forward process as $\mathbf{x}_0 = \frac{1}{\sqrt{\bar{\alpha}_t}}(\mathbf{x}_t-\sqrt{1-\bar{\alpha}_t\bm{\epsilon}_\theta(\mathbf{x}_t,t)})$.

Classifier-free guidance~\citep{clsfree} is introduced for conditional diffusion models to generate images without requiring an extra image classifier. A conditional model with a parameterized reverse process $p(\mathbf{x}_{t-1}|\mathbf{x}_t,\mathbf{c})$ uses a conditional identifier $\mathbf{c}$ through $\bm{\epsilon}_{\theta}(\mathbf{x}_t,t,\mathbf{c})$. To predict an unconditional score, the conditional identifier is replaced with a null token $\O$ and denoted as $\bm{\epsilon}_{\theta}(\mathbf{x}_t,t,\mathbf{c}=\O)$. Classifier-free guidance can then be approximated as a linear combination of conditional and unconditional predictions:
\vspace{-3pt}
\begin{equation}
   \bm{\tilde{\epsilon}}_{\theta}(\mathbf{x}_t,t,\mathbf{c}) = (1+w)\bm{\epsilon}_{\theta}(\mathbf{x}_t,t,\mathbf{c})-w\bm{\epsilon}_{\theta}(\mathbf{x}_t,t,\mathbf{c}=\O),
   \vspace{-3pt}
\end{equation}
where $w$ is the guidance scale. Text-video and text-image-based diffusion models~\citep{ldm,imagen,glide,vdm,makeavideo} use DDPM with classifier-free guidance. This diffusion method can be adapted to various tasks with flexibility.

\noindent\textbf{Latent Diffusion Models:} 
Compared with image diffusion, video diffusion has significantly higher computation costs because it needs to process multiple frames.
Recent works have explored the computation-efficient version of diffusion modeling, such as latent diffusion model (LDM)~\citep{ldm}. LDM proposes the VAE-based latent diffusion, including a KL-regularized autoencoder for encoding/decoding latent representation $\bm{\varepsilon}(\mathbf{x})$, and a diffusion model to operate on the latent space $\mathbf{z}_t$.
For the conditional generation, LDM introduces a domain-specific encoder $\bm{\tau}_\theta$ to the projection of condition $\mathbf{y}$ for various modality generations. Thus, the objective of LDM is: 
\vspace{-5pt}
\begin{equation}
    \vspace{-10pt}
    L_{\mathrm{LDM}} = \mathbb{E}_{t,\bm{\varepsilon}(\mathbf{x}),\mathbf{y},\bm{\epsilon}\sim\mathcal{N}(\boldsymbol{0},\boldsymbol{I})}\Bigr[\|\bm{\epsilon} - \bm{\epsilon}_\theta(\bm{z}_t,t,\bm{\tau}_\theta(\mathbf{y}))\|^2\Bigr]
\end{equation}

\section{Methodology}\label{sec:method}
In this paper, we aim to explore an efficient diffusion method to predict coherent video frames guided by language instructions, which requires learning to parse natural language, understand the scene, and ground the language and scene together. However, it is challenging to directly apply conventional video diffusion models for TVP due to the following problems: (1) The limited labeled text-video data and computational resources. (2) Low fidelity of frame generation. (3) Lack of fine-grained instruction for each frame in the task-level videos.
 
%\gu{Specifically, inheriting from I3D~\cite{i3d} technique, we build an inflated 3D U-Net to extend the prior knowledge contained in Stable Diffusion across the frames to generate high-quality and coherent frames by inserting computation-efficient spatial-temporal attention layers (Sec.~\ref{sec:efficientnet}).} As for the language conditioning model, we propose a novel Frame Sequential Text (FSText) Decomposer to adaptively decompose the text instruction into sub-conditions for each frame (Sec.~\ref{sec:fstext}).

\subsection{Overview of Seer}
\label{sec:inflate}
Motivated by the robust generative capabilities of text-to-image (T2I) diffusion models, we leverage the prior knowledge implied in pretrained T2I models by inflating the 2D U-Net~\citep{ldm} and incorporating temporally consistent layers. However, the inflated video diffusion model guided solely by coarse global language instruction tends to generate irrelevant T2I outcomes and fails to maintain temporal coherency between video frames. To address this limitation and provide precise and controllable guidance for our inflated model, we introduce a novel temporal decomposition component for language instruction, this component decomposes global instruction as temporally aligned sub-instruction for delicate task-level guidance, which significantly enhances the fidelity and coherency of predicted video.

Our Seer method comprises two main components: the video diffusion and the language conditioning modules. We propose to enhance these two components to facilitate high-fidelity frame synthesis and the temporal alignment of text instructions, respectively. Specifically, as shown in Figure~\ref{fig:pipeline} (a), we utilize two pathways to implement the conditional diffusion process guided by reference frames and language: \textbf{1)} We incorporate the spatial-temporal module discussed in Section~\ref{sec:efficientnet} into the Inflated 3D U-Net. This integration enables the propagation of contextual information from reference frames to future frames within the spatial-temporal space, allowing for coherent motion prediction based on the reference frames. \textbf{2)} To plan continuous motion with fine-grained language guidance, we introduce a Frame Sequential Text (FSText) Decomposer in Section~\ref{sec:fstext}. This module transforms global language instructions into multi-timestep sub-instructions that are synchronized with video. Subsequently, we inject these frame-wise subinstruction tokens into the intermediate latent space of the video frames at each time step.
With this design, we merely train the spatial-temporal layers and FSText module from scratch while freezing the remaining pretrained modules within our 3D inflated U-Net. These two modules are jointly trained by the diffusion objective, where $f_\theta$ is our FSText decomposer, $\bm{\tau}$ is the frozen CLIP text encoder, and $\mathbf{y}$ is the input text:
\begin{equation}
    L_{\mathrm{diffusion}} = \mathbb{E}_{t,\bm{\varepsilon}(\mathbf{x}),\mathbf{y},\bm{\epsilon}\sim\mathcal{N}(\boldsymbol{0},\boldsymbol{I})}\Bigr[\|\bm{\epsilon} - \bm{\epsilon}_\theta(\mathbf{z}_t,t,f_\theta(\bm{\tau}(\mathbf{y})))\|^2\Bigr],
\end{equation}


%\chuan{I find this paragraph is similar to the last paragraph of page 4. Just keep one to reduce redundancy.}
%Since the T2I latent diffusion models consist of two main components: the image diffusion module and the language conditioning module~\cite{ldm}. We propose to inflate these two parts to perform the synthesis of high-fidelity frames and the temporal decomposition of text instructions, respectively. \gu{Specifically, as shown in Figure~\ref{fig:pipeline} (a), we incorporate the computation-efficient spatial-temporal module into the Inflated 3D U-Net for optimizing temporal consistency in Section~\ref{sec:efficientnet}, and we propose a Frame Sequential Text (FSText) Decomposer for text conditioning module in Section~\ref{sec:fstext}.} Overall, we adopt two pathways to implement the conditional diffusion process of language guidance and reference frames. During training, we stack the latent space of the reference frames with the noisy latent space of the remaining frames along the temporal dimension. During inference, we predict future frames by propagating the prior reference frames and Gaussian noise through the Inflated 3D U-Net. For text conditioning, we employ FSText Decomposer to incorporate the text condition into the diffusion model.





\subsection{Data \& Computation-efficient 3D Network}\label{sec:efficientnet}
To design a computation-efficient visual backbone for our video diffusion model,  we refer to some relevant works on lifting 2D to 3D video modeling~\citep{i3d} and efficient attention computation~\citep{swin, videoswin}. In general, we leverage the latent diffusion model (LDMs) pretrained on T2I tasks to build a text-video model. Our inflated 3D U-Net consists of two principal components as illustrated in Figure~\ref{fig:pipeline} (b): \textbf{1)} The 3D spatial layers, where we draw inspiration from I3D~\citep{i3d} and enhance the 2D convolution kernel from ($3 \times 3$) to a 3D counterpart ($1 \times 3 \times 3$)  with an added video frames axis from the pre-trained 2D modules, consisting of a series of 2D ResNet blocks and Spatial Attention Blocks.
\begin{wrapfigure}[17]{r}{0.42\textwidth}
\centering
\vspace{-10pt}
\includegraphics[width=0.8\linewidth]{fig/wintempatten.pdf}
\vspace{-10pt}
\caption{Variants of temporal attention, only the blue tokens attend to the current token in the red box. Red dashed arrows indicate the direction of attention. And the orange boxes indicate the local window region ($2\times 2$ window in this case).}
\label{fig:tempattn}
\end{wrapfigure}
\textbf{2)} The temporal layers, play a crucial role in our visual backbone for propagating contextual information from the reference frame's image prior across the temporal sequence. We investigated various temporal attention and incorporated them into our 3D U-Net architecture. Our empirical observations indicate that bi-directional temporal attention tends to disregard guidance from reference frames, and both bi-directional and directed temporal attention struggle to capture dependencies among spatial regions, as discussed in Section~\ref{sec:ablate:temp}. To address these limitations while reducing complexity, we employ an efficient approach that builds upon the concept of window attention~\citep{swin} in 3D space: the implementation of local window attention in an autoregressive manner across spatial-temporal dimensions. As illustrated in Figure~\ref{fig:tempattn}, we establish fixed local windows for each spatial region with a window size of $m \times m$ relative to the global frame sequence $n$. Within this framework, we compute self-attention using a causal mask, considering both local spatial and global temporal dimensions within the 3D space. This effectively constrains pixel propagation from the future temporal-spatial sequence.

Finally, We maintain the acquired knowledge from the 2D modules by freezing all pretrained weights and exclusively training the spatiotemporal attention layers during fine-tuning. Overall, through a combination of frozen pre-trained spatial layers and lightweight spatiotemporal layers, our inflated 3D U-Net not only retains crucial knowledge but also enhances fine-tuning efficiency.

%Our empirical findings indicate that bi-directional temporal attention tends to disregard visual guidance from reference frames in time sequence (as discussed in Section~\ref{sec:ablate:temp}), and both bi-directional and directed temporal attention also miss out on capturing dependencies among nearby spatial regions, resulting in suboptimal frame quality. To address these limitations and enhance generation while reducing complexity, we employ an efficient approach that builds upon the concept of window attention~\cite{swin} in 3D space: the implementation of local window attention in an autoregressive manner across spatial-temporal dimensions. As illustrated in Figure~\ref{fig:tempattn}, we establish fixed local windows for each spatial region with a window size of $m \times m$ relative to the global frame sequence $n$. Within this framework, we compute self-attention with a causal mask across both local spatial and global temporal space within the 3D space, effectively constraining the pixel propagation from the future temporal-spatial sequence.
%The utilization of text-to-image (T2I) priors enhances the generative and imaginative capabilities of video generative models~\citep{makeavideo}. In this spirit, we leverage the latent diffusion model (LDMs) pretrained on T2I tasks, inflating it along the temporal axis. Our proposed latent diffusion model consists of two principal components as illustrated in Figure~\ref{fig:pipeline} (b): \textbf{1)} The 3D spatial layers inflated from the pre-trained image diffusion module, consisting of a series of 2D ResNet blocks and Spatial Attention Blocks. To adapt these 2D modules for 3D processing, we draw inspiration from I3D~\cite{i3d} to enhance the 2D convolution kernel from ($3 \times 3$) to a 3D counterpart ($1 \times 3 \times 3$)  with an added video frames axis. \textbf{2)} The temporal layers propagate contextual information from the image prior across the temporal sequence. We maintain the learned text-to-image knowledge from 2D modules by freezing all pre-trained weights during fine-tuning. This strategy not only retains crucial knowledge but also enhances fine-tuning efficiency.
%In comparison to plain spatial-temporal attention, the application of local windows to spatial regions significantly reduces computational overhead while delivering high-fidelity generation results. Notably, we observe that adopting SAWT-Atten has only a marginal $2.13\%$ computation speed lag compared to directed and bidirectional temporal attention, as shown in Appendix Table~\ref{table:speed_temp}.
%To address these limitations and enhance generation while reducing complexity, we employ an efficient approach: implementing local window attention in an autoregressive manner on both temporal and spatial spaces. Specifically, in this paper, extending from window attention~\cite{swin} in 3D space, we adopt a fixed window strategy with window sizes of $8\times8$, $4\times4$, $4\times4$, and $4\times4$ at stages 1, 2, 3, and 4 of U-Net encoder, respectively, within the U-Net network which utilizes multi-scale features. This strategy replaces the shifted window technique in Swin-Attention~\cite{swin}. Within the Scaled Autoregressive Window Temporal Attention (SAWT-Attn) layer (illustrated in Figure~\ref{fig:tempattn}), we extend vanilla temporal attention into spatial space through window attention for each spatial area with the local window of size $m \times m$, alongside the global video frame sequence $n$ across this window. The SAWT-Attn layer conducts self-attention on this extended sequence with a causal mask, integrating both spatial and temporal dimensions and restricting the model from learning future temporal-spatial tokens.
%As it operates in both spatial and temporal spaces, frame generation attends not only to prior frames but also to adjacent spatial regions, resulting in high-fidelity generation performance.
%In this context, for a video clip with $n$ frames, each is projected into $n\times H/K \times W/K \times 4$ latent vectors. Here $n$ signifies the frame count, $K$ indicates downsample ratio of VAE encoder, $(H/K, W/K)$ represents spatial dimensions, and $4$ corresponds to the number of latent channels.

\subsection{Frame Sequential Text Decomposer}\label{sec:fstext}
For the language conditioning module, since our 3D inflated U-Net is built upon a pretrained text-to-image model, we noticed that using a text-to-image prior alongside a global instruction tends to provide strong semantic guidance, which can override the scene in reference frames, deviating from the intended guidance for prediction based on the existing scenes.
To address the above limitation and better capture long-term dependencies from both text and reference frames, we introduce the Frame Sequential Text (FSText) Decomposer. This novel approach decomposes the global instruction into fine-grained sub-instructions, aligning with each frame. We further explore the interpretability of sub-instruction embeddings in Section~\ref{sec:results:subins}.
%the existing methods~\citep{magicvideo,makeavideo,tuneavideo} simply encode a single text embedding for the whole video with a CLIP text encoder~\cite{radford2021clip}.However, since text instructions often pertain to the overall task, understanding progress at each time step becomes challenging with a global instruction embedding
\begin{figure*}
\centering
\vspace{-30pt}
\includegraphics[width=0.9\linewidth]{fig/seq_text_transformer.pdf}
\vspace{-10pt}
\caption{Frame Sequential Text Decomposer is shown in (a). We start by initializing the weight of the network to project identity vectors from CLIP text tokens. We then optimize the generated text tokens via the diffusion process (b), where frame-individual cross-attention is denoted by ``fic-attn."}
\label{fig:fseq}
\vspace{-10pt}
\end{figure*}
To derive a sequence of temporally aligned sub-instruction embeddings from the global instruction generated by the CLIP text encoder~\cite{radford2021clip}, we employ a transformer-based temporal network designed to fulfill three essential properties for meaningful sub-instructions: \textbf{1)} Contextual aggregation, which ensures that the inner tokens of each sub-instruction aggregate contextual information within the sentence. \textbf{2)} Semantic inheritance, the semantic information of these sub-instructions is inherited directly from the global instruction \textbf{3)} Temporal consistency ensures alignment between the sub-instructions and the time sequence, thereby facilitating the generation of temporally consistent video. Based on these properties,  our network consists of  three key components: \textbf{a)} To achieve the property of contextual aggregation, we employ Text-Sequential Attention, akin to BERT, a bidirectional self-attention layer~\citep{bert} to capture global dependencies among different positions within text sentences. \textbf{b)} To ensure semantic inheritance, we use Cross-Attention, responsible for projecting the global instruction's textual sequence onto the inner tokens of each sub-instruction, this component ensures that all sub-instructions contain essential global instruction signals for guiding video frame generation. \textbf{c)} To maintain temporal consistency, we adopt temporal Attention, a directed attention layer to capture temporal dependencies along the frame axis, which enhances temporal consistency among the generated sub-instructions throughout the video.
\begin{wrapfigure}[10]{r}{0.4\textwidth}
\centering
\vspace{-10pt}
\includegraphics[width=0.9\linewidth]{fig/fstext_module.pdf}
\vspace{-10pt}
\caption{The FSText attention of sub-instruction tokens.}
\label{fig:fstextpipline}
\end{wrapfigure}
Specifically, as shown in Figure~\ref{fig:fstextpipline}, we start with a global CLIP text embedding, denoted as $(l, C)$, where $l$ signifies the text sentence length and $C$ is the channel size, we initialize learnable tokens with shape $(n, l, C)$ where $n$ denotes the number of frames. The tokens are fed into the text sequential attention layer to perform self-attention along the $l$ axis. Subsequently, the cross-attention layer employs these learnable tokens as queries and the global text embedding as keys and values, resulting in a one-to-multiple projection from the global text into $n$ time steps. This yields $(n, l, C)$ tokens for $n$ frame containing task instruction information. Finally, the temporal attention layer conducts directed attention along the $n$ axis for each token in the textual sequence, transforming the macro-instruction progress into frame-specific guidance.

After getting $n$ sub-instruction embeddings corresponding to each frame, the next step is to inject this guidance into the diffusion process, which is commonly completed by a cross-attention layer. As shown in Figure~\ref{fig:fseq} (b), different from the existing works~\citep{magicvideo,tuneavideo} that calculate the cross-attention between the global instruction embedding and $n$ frames. In our cross-attention layer, where cross-attention is calculated separately between visual latent vectors and sub-instruction embeddings for each frame, and the results from all frames are then concatenated, an attention mechanism we refer to as frame-individual cross-attention (\textit{fic-attn}).

\paragraph{Initialization~~} We find initialization is critical to FSText decomposer. Especially, the random initialization fails to approximate the distribution of text embeddings in the pretrained T2I model and results in poor performance. To guarantee the sub-instruction embeddings become a close approximation of the CLIP text embedding, we employ an initialization strategy by enforcing the FSText decomposer to be an identity function (Note that this initialization step is completed before the diffusion process. We ablate this design in Section~\ref{sec:ablate:fstext}). It can be achieved by this objective:
\begin{equation}
    L_{\mathrm{identity}} = \|f_\theta(\bm{\tau}(\mathbf{y})) - \bm{\tau}(\mathbf{y})\|^2
\end{equation}

%\subsection{Inflated 3D U-Net with Autoregressive Spatial-Temporal Attention}\label{sec:tempoal}
%We inflate the Text-to-Image (T2I) pre-trained 2D U-Net to our Inflated 3D U-Net as illustrated in Figure~\ref{fig:pipeline} (b). A standard 2D U-Net block of LDMs consists of a series of 2D ResNet blocks and Spatial Attention Blocks including spatial self-attention and cross-attention modules. Similar to~\cite{vdm}, we replace the $3 \times 3$ 2D convolution kernel with a $1 \times 3 \times 3$ 3D convolution kernel with an additional axis of video frames. Additionally, to further boost the performance of capturing the inter-frame dependency, we incorporate temporal attention after every spatial cross-attention layer. In Figure~\ref{fig:tempattn}, we explore various types of temporal attention, including: (1) bi-directional temporal attention~\cite{vdm,makeavideo,imagenvideo}, which employs a full self-attention across all tokens along the temporal dimension; (2) directed temporal attention~\cite{magicvideo}, which uses a masked attention mechanism that follows the direction of the video sequence along the temporal dimension; and (3) autoregressive spatial-temporal attention: a novel technique proposed by us, which uses causal attention to autoregressively generates the frames on both spatial and temporal dimensions by flattening the tokens into a long sequence.

%We empirically observe that the two existing temporal attention layers cannot achieve promising performance on the TVP task. Bi-directional temporal attention tends to neglect the visual content guidance of the reference frames during the generation process (see Section~\ref{sec:ablate:temp}). 
%And the directed temporal attention fails to capture the dependency of nearby spatial regions and thus generates low-quality frames, while it adheres to the temporal sequence constraint.

%To handle the limitations of bi-directional and directed temporal attention, we introduce the Autoregressive Spatial-Temporal Attention (AST-Attn) mechanism shown in Figure~\ref{fig:tempattn}.
%Given $n$ frames video, a video clip is projected into $n\times s$ latent vectors (where $s$ is the length of a latent vector in each frame) by the pre-trained VAE encoder. We flatten the latent vectors of both temporal and spatial dimensions ($n\times s$) into one dimension. 
%Then, AST-Attn performs self-attention on this long sequence with a causal mask that prevents the model from learning from future temporal-spatial tokens. Because it performs in both spatial and temporal spaces, the frame generation will attend to not only the previous frames but also the nearby spatial regions, which results in high-fidelity generation performance.
%While the calculation of Autoregressive Spatial-Temporal Attention (AST-Attn) involves both temporal and spatial dimensions, its computational complexity remains manageable due to the design of the Inflated 3D U-Net, which maintains the complexity of spatial compression rates and channel depths within a controllable range. Specifically, in the AST-Attn layers of Inflated 3D U-Net, higher-resolution features have more spatial tokens but are computed in lower embedding dimensions, while lower-resolution features have fewer spatial tokens but are computed in higher embedding dimensions. In practice, we observe that adopting AST-Attn has only a $0.4\%$ computation speed lag compared to directed and bidirectional temporal attention.

%By incorporating Autoregressive Spatial-Temporal Attention in the Inflated 3D U-Net, we can generate high-fidelity and coherent video frames with minimal fine-tuning. Specifically, we merely fine-tune the proposed autoregressive spatial-temporal attention layers and freeze the rest of the pre-trained layers in our Inflated 3D U-Net. 
\iffalse
\begin{figure}
\centering
\includegraphics[width=1.0\linewidth]{fig/inflate_unet.pdf}
\caption{The overview of inflated 3D U-Net, we inflate the pre-trained T2I latent diffusion model (LDM) by expanding the 2D Conv kernel to 3D kernels and connecting the cross-attention layer with the trainable causal temporal attention layer.}
\label{fig:inflate3d}
\end{figure}
\fi




\section{Experiments}
In this section, we evaluate the proposed method Seer on the text-conditioned video prediction task. 
We compare against various recent methods and conduct ablation studies on the techniques presented in Section~\ref{sec:method}.

\begin{figure}
\centering
\includegraphics[width=1.0\linewidth]{fig/prediction.pdf}
\caption{Visualization results of text-conditioned video prediction (conditioned on first 2 frames) on Something-Something V2. TAV refers to Tune-A-Video.}
\vspace{-10pt}
\label{fig:tvp:sthv2}
\end{figure}
\begin{table*}
\centering\small
\tablestyle{2pt}{1.1}
\setlength{\tabcolsep}{6pt}
{\caption{\textbf{ Text-conditioned video prediction (TVP) results on Something-Something V2 (SSv2) and Bridgedata (Bridge).} We report the FVD and KVD metrics of each method. We can see that our method Seer achieves the lowest FVD and KVD values in both SSv2 and Bridgedata, illustrating our superior performance on the challenging TVP task.
}
\label{table:tvp}}
\vspace*{-3mm}
\begin{tabular}{cccc|cc|cc}
\specialrule{.1em}{.05em}{.05em} 
 \multirow{2}{*}{Method} & \multirow{2}{*}{Pre.-weight} & \multirow{2}{*}{Text} & \multirow{2}{*}{Resolution} & \multicolumn{2}{c|}{\textbf{SSv2} (ref. = 2)} & \multicolumn{2}{c}{\textbf{Bridge} (ref. = 1)}\\
   &  &  &  & FVD$\downarrow$ & KVD$\downarrow$  & FVD$\downarrow$ & KVD$\downarrow$\\
 \hline
TATS~\cite{tats} & video & No  & $128\times 128$ & 428.1 & 2177 & 1253 & 6213\\
 MCVD~\cite{mcvd} & No & No  & $256\times 256$ & 1407 & 3.80 & 1427 & 2.50\\
Tune-A-Video~\cite{tuneavideo} & txt-img & Yes & $256\times 256$ & 291.4 & 0.91 & 515.7 & 2.01\\
Seer (Ours) & txt-img & Yes & $256\times 256$ & $\bf 200.1$ & $\bf 0.30$ & $\bf 507.3$ & $\bf 1.37$\\
\specialrule{.1em}{.05em}{.05em} 
\end{tabular}
\vspace{-10pt}
\end{table*}
\begin{figure}
\centering
\includegraphics[width=1.0\linewidth]{fig/tvp_bridgedata.pdf}
\caption{Visualization of Text-conditioned Video Prediction on Bridgedata. ``Ref." refers to reference frames and TAV refers to Tune-A-Video.}
\vspace{-10pt}
\label{fig:tvp:bridge}
\end{figure}


\subsection{Datasets}
We conduct experiments on two text-video datasets: Something Something-V2 (SSv2)~\cite{sthv2}, which contains videos of human behaviors involving everyday objects accompanied by language instructions, and BridgeData~\cite{bridge} that is rendered by a photo-realistic kitchen simulator with text prompts.
Because the SSv2 validation set is too large (with over 200k samples), we follow \cite{ucf101} to evaluate the first 2048 samples during evaluation to save testing time.
For BridgeData, we split the dataset into an $80\%$ training set and $20\%$ validation set, and evaluate all validation samples. To reduce complexity, we downsample each video clip to 12 frames for SSv2 and 16 frames for BridgeData during both training and evaluation. Moreover, to provide a fair comparison with recent unreleased video generative model baselines~\cite{vdm,lvdm,magicvideo,makeavideo}, we also included results on the UCF-101 dataset~\cite{ucf101} in Appendix~\ref{appendix:sec:ucf101}.

\subsection{Implementation Details}
We use the pre-trained weights of the Tex-to-Image Latent Diffusion Model (LDM), Stable Diffusion-v1.5~\cite{ldm}
, to initialize the VAE, ResNet Blocks and Spatial-Cross Attention layers of the 3D U-Net. We freeze both the pre-trained VAE and the pre-trained modules of the 3D U-Net, and only fine-tune the Autoregressive Spatial-Temporal Attention Layers. To fine-tune the FSText Decomposer, we initialized it as the identity function of the CLIP text embedding, as described in Section~\ref{sec:fstext}. We train the models on Something Something-V2 and BridgeData with an image resolution of $256 \times 256$ for 20k training steps. In the evaluation stage, we speed up the sampling process with the fast sampler DDIM~\cite{ddim} and denoise the prediction with conditional guidance of 7.5 for 30 timesteps. Please refer to Appendix~\ref{appendix:sec:impl} for more details on hyperparameters.

\subsection{Evaluation Settings}
\noindent\textbf{Baselines.~~}\label{sec:baseline} We compare Seer with three publicly released baselines for video generation (1) conditional video diffusion methods: \textit{Tune-A-Video}~\cite{tuneavideo} and \textit{Masked Conditional Video Diffusion} (MCVD)\cite{mcvd}; and (2) autoregressive-based transformer method: \textit{Time-Agnostic VQGAN and Time-Sensitive Transformer} (TATS)\cite{tats}. Since Tune-A-Video is also the Text-to-Image inflated video diffusion model, and both MCVD and TATS are long video generative models for video prediction, they conform to our benchmark that requires predicting task-level movements. We further fine-tune Tune-A-Video, TATS, and train MCVD on the training sets of SSv2 and Bridgedata for 300k training steps.

\noindent\textbf{Machine Evaluation.~~}\label{sec:exp:tvp} We evaluate the text-conditioned video prediction of several baseline methods on Something Something-V2 (SSv2) (with 2 reference frames) and Bridgedata (with 1 reference frame). Additionally, we conduct several ablation studies of our proposed modules on SSv2. We report the Fréchet Video Distance (FVD) and Kernel Video Distance (KVD) metrics in our evaluation. FVD and KVD are calculated with the Kinetics-400 pre-trained I3D model~\cite{i3d}. We evaluate FVD and KVD on 2,048 SSv2 samples and 5,558 Bridgedata samples in the validation sets. For FVD metrics, we follow the evaluation code of VideoGPT~\cite{videogpt}. We further evaluate the class-conditioned video prediction of our method on the UCF-101 dataset~\cite{ucf101} and present the comparison results in Appendix~\ref{appendix:sec:ucf101}.

\begin{figure}
\centering
\includegraphics[width=0.9\linewidth]{fig/human_eval.pdf}
\caption{Human evaluation results. Preference percentage for text-conditioned video manipulation on SSv2.}
\vspace{-8pt}
\label{fig:humaneval}
\end{figure}

\noindent\textbf{Human Evaluation.~~}\label{sec:exp:humaneval} Besides evaluating the models on the standard validation sets, we also manually modify the text prompts to provide richer testing results, called text-conditioned video manipulation. Because of the absence of ground-truth frames, we conducted a human evaluation of text-conditioned video manipulation using 99 video clips from the validation set of SSv2. We manually modified partial text prompts and generated 99 predicted videos for each method. Then, we invited 54 anonymous evaluators to rate the quality of the prediction, with a higher priority placed on the semantic contents in the videos and an intermediate priority placed on the fidelity of the video frames. We report the percentage of overall preference choices among the 99 video clips. More details are introduced in Appendix~\ref{appendix:sec:humaneval}.

\begin{figure}
\centering
\includegraphics[width=1.0\linewidth]{fig/manipulation.pdf}
\caption{Visualization of Text-conditioned Video Prediction with the original (a) and manually modified (b) text prompts on Something-Something V2. ``Ref." is reference frames and TAV refers to Tune-A-Video.}
\vspace{-10pt}
\label{fig:tvm:sthv2}
\end{figure}

\subsection{Main Results}\label{sec:main-results}
\noindent\textbf{Quantitative Results.~~} Table~\ref{table:tvp} presents the text-condtioned video prediction results on Something Something-V2 (SSv2) and BridgeData. Seer achieves the best performance among all baselines, with the lowest Fréchet Video Distance (FVD) of 200.1 and Kinematic Distance (KVD) of 0.3 in SSv2, and the lowest FVD of 507 and KVD of 1.37 in BridgeData. Notably, Seer and Tune-A-Video both incorporate text conditioning, and the results highlight Seer's superior text-video alignment performance, especially on SSv2.

The results of the human evaluation in the text-conditioned video manipulation experiment are shown in Figure~\ref{fig:humaneval}. Our proposed Seer outperforms the other baselines in terms of both semantic content and fidelity of video, with a preference rate of at least $72.2\%$ in comparison. This indicates that Seer is effective in generating high-quality video clips that are faithful to the input text prompts.

\begin{table}
\centering\small
\tablestyle{2pt}{1.0}
\setlength{\tabcolsep}{4pt}
\floatsetup{floatrowsep=qquad, captionskip=1.5pt}
\begin{floatrow}[2]
\ttabbox
{\begin{tabular}{c|cc}
init. weight & FVD$\downarrow$ & KVD$\downarrow$\\
 \hline
 \textit{random} & 367.9 & 0.75\\
\textit{identity}(Ours) & 200.1 & 0.30\\
\end{tabular}
{\caption[ftext]{\textbf{Init. weight ablation results of FSText}}
\label{table:ablation:weight}
}
}
\hfill
\ttabbox
{\begin{tabular}{c|cc}
temp. attn. & FVD$\downarrow$ & KVD$\downarrow$\\
 \hline
 \textit{bi-direct.} & 258.2 & 0.56\\
 \textit{directed.} & 222.3 & 0.40\\
 \textit{autoreg.}(Ours) &200.1 & 0.30\\
\end{tabular}}
{\caption[temp]{\textbf{Ablation study of temporal attention}}
\label{table:ablation:tempattn}
}
\end{floatrow}

\begin{floatrow}[1]
\ttabbox[1.0\linewidth]
{\begin{tabular}{cc|cc}
\specialrule{.1em}{.05em}{.05em} 
fine-tune& FSText. & FVD$\downarrow$ & KVD$\downarrow$\\
 \hline
 \textit{temp-attn.} & & 328.2 & 1.26\\
 \textit{cross}+\textit{temp-attn.} & & 249.9 & 0.73\\
 \textit{temp-attn.}(Ours) & \checkmark &200.1 & 0.30\\
 \textit{cross}+\textit{temp-attn.} & \checkmark & 1807 & 5.12\\
\specialrule{.1em}{.05em}{.05em} 
\end{tabular}}
{\caption[fine]{\textbf{Ablation study of Fine-tune settings}}
\label{table:ablation:finetune}
}
\end{floatrow}
\end{table}


\iffalse
\begin{table}
\centering\small
\tablestyle{2pt}{1.0}
\setlength{\tabcolsep}{5pt}
\caption{\textbf{ Fine-tune settings and component design }.}
\label{table:ablation:finetune}
\vspace*{-3mm}
\begin{tabular}{cc|cc}
\specialrule{.1em}{.05em}{.05em} 
fine-tune& FSText. & FVD$\downarrow$ & KVD$\downarrow$\\
 \hline
 \textit{temp-attn.} & & 328.2 & 1.26\\
 \textit{cross}+\textit{temp-attn.} & & 249.9 & 0.73\\
 \textit{temp-attn.}(Ours) & \checkmark &200.1 & 0.30\\
 \textit{cross}+\textit{temp-attn.} & \checkmark & 1807 & 5.12\\
\specialrule{.1em}{.05em}{.05em} 
\end{tabular}
\end{table}
\fi


\noindent\textbf{Qualitative Results.~~} Figure~\ref{fig:tvp:sthv2} compares the text-conditioned video prediction (TVP) performance of Seer and Tune-A-Video on Something Something-V2 (SSv2). While Tune-A-Video can align simple text prompts with video in some cases, it struggles to consistently track the spatial appearance of reference frames in later predictions. For instance, in the ``taking glass from desk" samples, Tune-A-Video fails to generate a coherent motion trajectory and corrupts the pixels in the background, generating a new video instead of predicting from the reference frames. In contrast, Seer generates relatively coherent motion and better aligns the predictions with text prompts. Additionally, Seer can generate hidden objects by leveraging the imaging capability of the pretrained text-to-image diffusion model, which flexibly tackles occlusion issues in video prediction. In the ``tearing a piece of paper into two pieces" sample, Seer accurately predicts that a man is hidden behind the paper and generates coherent frames including the man's face. Figure~\ref{fig:tvp:bridge} compares Seer and Tune-A-Video's TVP performance on Bridgedata, illustrating that Seer achieves better text-video alignment, including the alignment of instructed behavior and target objects in future frames, and predicts a more coherent video with higher fidelity.

Figure~\ref{fig:tvm:sthv2} shows a comparison of Seer and Tune-A-Video for text-conditioned video prediction and manipulation on Something Something-V2 (SSv2). Tune-A-Video tends to mainly focus on Text-to-Video alignment, usually ignoring the directional temporal movement of the video. For example, in the case of ``turning the camera left while filming wall mounted fan", Tune-A-Video generates a semantic movement when the word "left" is replaced with "right" in the sentence, but fails to maintain temporal consistency in the video background during the transition from the middle to the last frame. In contrast, Seer performs better in handling the temporal dynamics of the video and achieving more precise text-video alignment in video manipulation.


\subsection{Ablation study}
In this section, we evaluate the effect of different components of our method in the TVP task on the SSv2 dataset.

\noindent\textbf{FSText Decomposer.~~}\label{sec:ablate:fstext}
Table~\ref{table:ablation:weight} compares different weight initialization strategies of FSText decomposer. The results demonstrate that using identity initialization described in Section~\ref{sec:fstext} yields higher prediction quality compared with random initialization. This finding demonstrates that identity initialization is necessary for the temporal-text projection of FSText decomposer. We also provide additional ablation results of FSText decomposer in Appendix~\ref{appendix:sec:fstext}.

\noindent\textbf{Temporal Attention.~~}\label{sec:ablate:temp}
As shown in Table~\ref{table:ablation:tempattn} studies the effectiveness of different types of temporal attention. Our autoregressive spatial-temporal attention (autoreg.) outperforms both bi-directional temporal attention (bi-direct.) and directed temporal attention (directed.), resulting in the lowest FVD and KVD scores. We also find that directed temporal attention further improves video prediction performance compared to bi-directional temporal attention because it utilizes the inductive bias of sequential generation.

\noindent\textbf{Fine-tune Setting.~~}
We compare various fine-tuning settings of 3D Inflated U-Net, and the results are presented in Table~\ref{table:ablation:finetune}. Our default setting involves fine-tuning both FSText decomposer (FSText.) and autoregressive spatial-temporal attention (AST-Attn.) layers (\textit{temp-attn.}), while freezing the remaining modules in 3D U-Net. For the ``\textit{temp-attn.}" setting, we only finetune the AST-Attn. layers and freeze all other components. In the "\textit{cross+temp-attn.}" setting, we jointly update the parameters of Spatial-Cross Attention layers and AST-Attn. layers. We further fine-tune the "\textit{cross+temp-attn.}" together with FSText decomposer. We observe that our default setting achieves the highest quality of video prediction among all these settings, indicating that fine-tuning FSText decomposer is critical. Based on our default setting, further fine-tuning "\textit{cross+temp-attn.}" causes the performance of Seer to drop a lot, even the worst among all fine-tuning settings. These results suggest that the optimization of the FSText decomposer is strongly guided by the frozen conditional diffusion prior, and additional fine-tuning of cross-attention leads to uncontrollable guidance to the FSText decomposer.

\iffalse
\begin{table}
\centering\small
\tablestyle{2pt}{1.0}
\setlength{\tabcolsep}{5pt}
\caption{\textbf{ Type of temporal attention }}
\label{table:ablation:tempattn}
\vspace*{-3mm}
\begin{tabular}{c|cc}
temp. attn. & FVD$\downarrow$ & KVD$\downarrow$\\
 \hline
 \textit{bi-direct.} & 258.2 & 0.56\\
 \textit{directed.} & 222.3 & 0.40\\
 \textit{autoreg.}(Ours) &200.1 & 0.30\\
\specialrule{.1em}{.05em}{.05em} 
\end{tabular}
\end{table}
\fi
\section{Conclusion and Future Work}
In this paper, we have proposed a self-paced neutral expression-disentangled learning (SPNDL) model, which is hyperparameter-free in the sense. Our method achieves state-of-the-art FER performance on the CK+ and Oulu-CASIA databases and competitive results on RAF-DB.
First, basic features of target image and the corresponding neutral image are extracted, both of which share similar disturbance information. To alleviate the inconsistently distributing problem of these basic features, we apply a normalization operation in each layer of the backbone CNN. Second, NDFs are obtained by a subtraction operation, which can suppress the negative impact of disturbing factors in facial expression images, and simultaneously introduce the information of expression's initial state. Finally, a SPL strategy is proposed in training stage, letting the network learn like human beings -- starting with easy NDFs and building up to complex ones. 

Our future work will focus on how to more effectively learn neutral expression-disentangled features when there are no corresponding neutral expression images, especially on large-scale in-the-wild databases where there are more disturbing factors.
%it is still hard to achieve state-of-the-art results.
%which means SPNDL does not work well on in-the-wild databases. 
%Thus, our future work will focus on extending our framework to large-scale databases where there are more disturbing factors and no corresponding neutral expression images.

%% The Appendices part is started with the command \appendix;
%% appendix sections are then done as normal sections
%% \appendix

%% \section{}
%% \label{}

\bibliographystyle{elsarticle-num} 
\bibliography{egbib}

\end{document}
%%
%% End of file `elsarticle-template-num.tex'.
