% \documentclass[main.tex]{subfiles}
\label{sec:intro}
% \begin{document}

\paragraph{Transient faults in asynchronous circuits}
Several studies have explored the effects of transient faults on asynchronous circuits. Ways for detection and mitigation techniques with some form of redundancy have been proposed alongside.

The authors in~\cite{lafrieda2004fault} perform a thorough analysis of single-event transient (SET) effects, among other types of faults, in QDI circuits. The fault's impact is first presented at the gate level, then on communication channels, translating the fault to a deadlock. They elaborate other possible errors on a high level in terms of synchronization failure, token generation and token consumption.
An efficient failure detection method for QDI circuits is presented in~\cite{peng2005efficient}. The method brings the circuit to a fail-safe state in the presence of hard and soft errors. The authors investigate the probability for a glitch to propagate through a state-holding element in asynchronous circuits.
In~\cite{monnet2007formal}, the authors propose a formal method to model the behavior of QDI circuits in the presence of transient faults. They use symbolic simulation to provide an exhaustive list of possible effects and analyze which of these cases are theoretically reachable. Their model, however, does not support delay parameters which could reduce the set of states that are physically reachable, further proving the resistance of a design against single-event upsets (SEUs).
They also show in~\cite{monnet2005asynchronous} the Muller C-element fault sensitivity and then specify a global sensitivity criterion to SETs for asynchronous circuits. They provide a behavioral analysis, with distinct classifications, of QDI circuits in the presence of faults.
With the help of signal transition graphs (STGs), the authors in~\cite{bainbridge2009glitch} \emph{informally} analyze SEUs due to glitches on QDI network-on-chip links. They propose several mitigation techniques with a focus on reducing the latch's sensitive window to a glitch.
Some of these techniques are tested and compared against other proposed variations in \cite{huemer2020QDIwindows}, \cite{behal2021towards}, and \cite{tabassam2022set}. The assessment there is based on extensive fault injection simulations into different QDI buffer styles, in order to identify the main culprits of the circuit. They provide a quantitative analysis to determine the windows of vulnerability to SETs and the impact of certain parameter choices on the resilience of the circuit. However, the analysis is done based on a regular timing grid, which causes linear complexity in time and in resolution, and cannot exclude the potential of overlooking relevant windows between the grid points.

\paragraph{Hazards in PRSs}
QDI circuits can be modeled on different levels. A
Production Rule Set (PRS), introduced by Martin~\cite{martin1986compiling}, is the canonical representation of a QDI circuit from which one can easily reach an equivalent implementation in CMOS technology. PRSs do not normally support hazards, and by guaranteeing \emph{stability} and \emph{non-interference} characteristics~\cite{jang2005seu}, a PRS execution is assumed to be hazard-free. The authors consider an SEU as flipping of a variable's value and model it in so called transition graphs to identify deadlock or abnormal behavior.
\cite{katelman2012rewriting} extends the semantics of PRSs in order to be able to address hazards as circuit failures, but it is limited to checking the hazard-freedom property of a circuit. 

These papers are focused on the \emph{possibility} of failure and are restricted to precedence of events, without explicitly considering timing.
Our work enables further propagation of what we define as a glitch in order to check whether it has reached the final outputs of a circuit and, based on actual timing information, \emph{quantify} this proportion of failure.

% \end{document}