% \documentclass[main.tex]{subfiles}
\label{sec:intro}
%\begin{document}

By means of a formal proof we have established that the regular operation of circuits can be decomposed into time windows within which faults are equivalent in that their effect (as perceived at some selected monitoring signals) remains the same. These time windows are bounded by an arbitrary bound on the left and a regular signal transition on the right. 
Consequently, for determining the effect of transient faults on a circuit, a single bisection between each signal transition is sufficient to determine all sensitivity windows.

The approach has two advantages over standard sweeping approaches to find sensitive regions: (i) it provably finds all sensitivity windows, no matter how small they are. Sweeping by contrast always leaves the possibility open that a small window may exist between two samples.
(ii) It outperforms sweeping in that a small grid of samples is not necessary: many (large) windows require only a single sample via our method.

Based on this result we have developed a Python-based tool that, starting from a production-rule based circuit description, systematically explores its resilient and its vulnerable windows (along with the respective fault effects). 
The relative size of the windows is then used to predict the proportion of (random) faults that will be effective, and thus, given a fault rate, the failure rate.
Since our approach allows identifying the windows individually, it is possible to attach weights to the diverse nodes to account for different susceptibility (drive strength, e.g.) in the overall prediction.

We have illustrated the function of our tool on several examples of typical QDI circuits which showed that the tool is efficient and allows for fast analysis with a good scaling towards complex circuits.

While currently only relatively simple circuits were targeted to allow keeping a focus on the principle of our approach, a next step will be to extend the set of targets to larger and more complex circuits.
Another extension of our approach will be to determine the constituent parameters for the window sizes. Since we determine all windows individually in our automated process, backtracking to the origins of the relevant signal transitions is possible. With that information we can determine in detail how individual parameters like circuit delays or pipeline load influence resilience and hence elaborate targeted optimizations.
Finally, work on improving the performance of the implementation is planned: the proposed algorithm is easily parallelizable since windows can be determined independently and hence concurrently.


%\end{document}