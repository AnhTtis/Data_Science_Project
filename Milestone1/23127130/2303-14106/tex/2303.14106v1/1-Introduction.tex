% \documentclass[main.tex]{subfiles}
\label{sec:intro}
% \begin{document}

It is well known that synchronous circuits exhibit a natural resilience against transient faults through masking. Specifically, the relevant effects are electrical masking (short fault pulses are filtered by low-pass behavior of gates and interconnect), logical masking (depending on other input levels, the logic level of the faulty input may be irrelevant for the gate output) and temporal masking (the flip flop ``samples'' its data input at the active clock edges while ignoring faults that happen between these). However, synchronous circuits have little resilience against (fault) effects that impact the timing.
In contrast, asynchronous, specifically QDI, circuits exhibit large, ideally unlimited, tolerance against timing variations. This is due to their event-driven operation principle. Unfortunately, this very event driven operation makes them prone to transient faults. Electrical masking and logical masking mitigate many of the fault effects, just like in the synchronous case. Whether temporal masking occurs, however, is not easy to answer. Previous works have shown that asynchronous pipelines, e.g., have data accepting windows during which they are susceptible to fault pulses. The size of these windows depends on several parameters, most notably the mode of pipeline operation (bubble-limited / balanced / token-limited). For unbalanced operation these windows may reach considerable size, making the circuit clearly more susceptible to faults than in the synchronous case with its ``instantaneous sampling''. That is why several mitigation methods \cite{bainbridge2009glitch} aim at minimizing the data accepting windows. In any case there is some effect equivalent to temporal masking, and most often it is constituted by Muller C-elements (MCEs): While in \emph{combinational} mode of operation (matching inputs), the MCE ignores fault pulses on any input, not even a pulse at the output can flip its state. In \emph{storage} mode (non-matching inputs), however, the MCE's state can be easily flipped by a fault pulse at one of the inputs or at the output (directly at the keeper). So apparently, the share of time during which an MCE is in combinational mode determines the masking provided by it. In a reasonably complex practical setting, however, this insight is hard to map to a general prediction of the whole circuit.


\paragraph{Contributions and organization}
Related studies have already explored the resilience of asynchronous circuits against transient (and permanent) faults and produced interesting results for specific cases, as well as some general insights. However, important answers are still missing. The data accepting windows, e.g., have been qualitatively described, and they have been experimentally determined and visualized for specific parameters by injecting faults over a regularly spaced time grid. However, to the best of our knowledge, no systematic exploration has been performed.
In this paper we present an approach to efficiently and precisely identify the sensitive windows (position and size) over time for all nodes individually, with a guarantee to find all windows of vulnerability, no matter how small. 

To this end, we introduce our circuit model in Section~\ref{sec:model}.
In Section~\ref{sec:results} we start with basic consistency results
  of the model, followed by our main technical result: the definition of value regions in executions along with
  a proof of the equivalence of glitches within those regions (Theorem~\ref{thm:main}).
Based on this result we then present our tool for sensitivity-window exploration (Section~\ref{sec:tool}) and apply it to
  different QDI circuits for illustration, where the ability to break down the sensitivity analysis to the signal level proves beneficial.
We conclude in Section~\ref{sec:conclusion}.


% \end{document}