%%%%%%%%%%%%%%%%%%%%%%%%%%%%%%%%%%%%%%
% 
% TODO:
% 1.
% 2.
%
% 
%%%%%%%%%%%%%%%%%%%%%%%%%%%%%%%%%%%%%%

\documentclass[conference]{IEEEtran}
\IEEEoverridecommandlockouts
% The preceding line is only needed to identify funding in the first footnote. If that is unneeded, please comment it out.
\usepackage{cite}
\usepackage{pdfpages}
\usepackage{amsmath,amssymb,amsfonts}
\usepackage{algorithmic}
\usepackage{graphicx}

\usepackage{textcomp}
\usepackage{xcolor}
\usepackage{pdflscape}
\usepackage[a4paper, total={184mm,239mm}]{geometry}
\def\BibTeX{{\rm B\kern-.05em{\sc i\kern-.025em b}\kern-.08em
    T\kern-.1667em\lower.7ex\hbox{E}\kern-.125emX}}
    
\usepackage{subfiles}
\usepackage{tikz} 
\usepackage{pbox}
% \usepackage{caption}
% \usepackage{subcaption}

% todo notes
\usepackage{todonotes}

\usepackage{multirow}
\usepackage{diagbox}

\usepackage{amsmath,amsthm}

\newcommand{\meta}{\text{X}}
\newcommand{\IR}{\mathbf{R}}
\newcommand{\IN}{\mathbf{N}}
\newcommand{\IB}{\mathbf{B}}
\newcommand{\BX}{\mathbf{B}_\meta}

\newcommand{\calI}{{\cal I}}
\newcommand{\calL}{{\cal L}}
\newcommand{\calO}{{\cal O}}

\renewcommand{\paragraph}[1]{{\medskip\noindent\bf #1.}}
\newtheorem{thm}{Theorem}
\newtheorem{lem}{Lemma}


\begin{document}

\title{On the Susceptibility of QDI Circuits to\\Transient Faults\\
\thanks{This research was partially supported by the project ENROL (grant I 3485-N31) of the Austrian Science Fund (FWF) as well as the Doctoral College on Resilient Embedded Systems (DC-RES) and the ANR project DREAMY (ANR-21-CE48-0003).}
}

\author{%
\IEEEauthorblockN{Raghda El Shehaby}
\IEEEauthorblockA{TU Wien, Institute of Computer Engineering\\
Vienna, Austria}\\
%
\IEEEauthorblockN{Matthias F\"ugger}
\IEEEauthorblockA{CNRS \& LMF, ENS Paris-Saclay, Université Paris-Saclay \& Inria\\
France}\\
%
\IEEEauthorblockN{Andreas Steininger}
\IEEEauthorblockA{TU Wien, Institute of Computer Engineering\\
Vienna, Austria}
}

\maketitle

\begin{abstract}
By design, quasi delay-insensitive (QDI) circuits exhibit higher resilience against timing variations as compared to their synchronous counterparts.
Since computation in QDI circuits is event-based rather than clock-triggered, spurious events due to transient faults such as radiation-induced glitches, a priori are of higher concern in QDI circuits.

In this work we propose a formal framework with the goal to gain a deeper understanding on how susceptible QDI circuits are to transient faults.
We introduce a worst-case model for transients in circuits.
We then prove an equivalence of faults within this framework and use this result to provably exhaustively check QDI circuits, a linear Muller pipeline and a cyclic Muller pipeline, for their susceptibility to produce non-stable output signals.
\end{abstract}

\begin{IEEEkeywords}
transient faults, QDI circuits, automatic evaluation
\end{IEEEkeywords}

\section{Introduction}
\subfile{1-Introduction}

\section{Related Work}
\subfile{2-RelatedWork}

\section{Model}
\label{sec:model}
\subfile{3-Model}

\section{Results}
\label{sec:results}
\subfile{4-Results}

\section{Conclusion \& Future Work}
\label{sec:conclusion}
\subfile{5-Conclusion}

\bibliographystyle{IEEEtran}
\bibliography{main}

\end{document}
