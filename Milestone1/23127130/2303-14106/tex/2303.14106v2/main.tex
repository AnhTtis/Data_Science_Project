% \documentclass[conference]{IEEEtran}
\documentclass[runningheads]{llncs}
% \IEEEoverridecommandlockouts
% The preceding line is only needed to identify funding in the first footnote. If that is unneeded, please comment it out.
\usepackage{cite}
\usepackage{pdfpages}
\usepackage{amsmath,amssymb,amsfonts}
\usepackage{algorithmic}
\usepackage{graphicx}

\usepackage{textcomp}
\usepackage{xcolor}
\usepackage{pdflscape}
% \usepackage[a4paper, total={184mm,239mm}]{geometry}

% \usepackage{geometry}
% \geometry{
%   a4paper,         % or letterpaper
%   textwidth=12.2cm,  % llncs has 12.2cm
%   textheight=19.3cm, % llncs has 19.3cm
%   heightrounded,   % integer number of lines
%   hratio=1:1,      % horizontally centered
%   vratio=2:3,      % not vertically centered
% }

\def\BibTeX{{\rm B\kern-.05em{\sc i\kern-.025em b}\kern-.08em
    T\kern-.1667em\lower.7ex\hbox{E}\kern-.125emX}}
    
\usepackage{subfiles}
\usepackage{tikz} 
\usepackage{pbox}
% \usepackage{caption}
% \usepackage{subcaption}

% todo notes
\usepackage{todonotes}

%\usepackage{tabularray}
\usepackage{multirow}
\usepackage{diagbox}

% \usepackage{amsmath,amsthm}
\usepackage{amsmath}

\newcommand{\meta}{\text{X}}
\newcommand{\IR}{\mathbf{R}}
\newcommand{\IN}{\mathbf{N}}
\newcommand{\IB}{\mathbf{B}}
\newcommand{\BX}{\mathbf{B}_\meta}

\newcommand{\calI}{{\cal I}}
\newcommand{\calL}{{\cal L}}
\newcommand{\calO}{{\cal O}}

\renewcommand{\paragraph}[1]{{\medskip\noindent\bf #1.}}
\newtheorem{thm}{Theorem}
\newtheorem{lem}{Lemma}


\begin{document}

\title{On the Susceptibility of QDI Circuits to\\Transient Faults\thanks{This research was partially supported by the project ENROL (grant I 3485-N31) of the Austrian Science Fund (FWF), the Doctoral College on Resilient Embedded Systems (DC-RES), the ANR project DREAMY (ANR-21-CE48-0003), and the French government's excellence scholarships for research visits.}
}

% \author{%
% \IEEEauthorblockN{Raghda El Shehaby}
% \IEEEauthorblockA{TU Wien, Institute of Computer Engineering\\
% Vienna, Austria}\\
% %
% \IEEEauthorblockN{Matthias F\"ugger}
% \IEEEauthorblockA{CNRS \& LMF, ENS Paris-Saclay, Université Paris-Saclay \& Inria\\
% France}\\
% %
% \IEEEauthorblockN{Andreas Steininger}
% \IEEEauthorblockA{TU Wien, Institute of Computer Engineering\\
% Vienna, Austria}
% }

\author{Raghda El Shehaby\inst{1}\orcidID{0009-0000-6653-9074}
\and
Matthias F\"ugger\inst{2}\orcidID{0000-0001-5765-0301}
\and
Andreas Steininger\inst{1}\orcidID{0000-0002-3847-1647}
}

\authorrunning{El Shehaby \emph{et al.}}

\institute{
  TU Wien, Institute of Computer Engineering\\
  \and
  CNRS \& LMF, ENS Paris-Saclay, Université Paris-Saclay \& Inria
}

\maketitle

\begin{abstract}
By design, quasi delay-insensitive (QDI) circuits exhibit hi\-gher resilience against timing variations as compared to their synchronous counterparts.
Since computation in QDI circuits is event-based rather than clock-triggered, spurious events due to transient faults such as radiation-induced glitches, a priori are of higher concern in QDI circuits.

In this work we propose a formal framework with the goal to gain a deeper understanding on how susceptible QDI circuits are to transient faults.
We introduce a worst-case model for transients in circuits.
We then prove an equivalence of faults within this framework and use this result to provably exhaustively check a widely used QDI circuit, a linear Muller pipeline, for its susceptibility to produce non-stable output signals.

\keywords{transient faults, QDI circuits, automatic evaluation}

\end{abstract}



\section{Introduction}
\subfile{1-Introduction}

\section{Related Work}
\subfile{2-RelatedWork}

\section{Model}
%\label{sec:model}
\subfile{3-Model}

\section{Results}
%\label{sec:results}
\subfile{4-Results}

\section{Conclusion}
%\label{sec:conclusion}
\subfile{5-Conclusion}

\bibliographystyle{IEEEtran}
\bibliography{main}

\clearpage
\appendix
\subfile{Appendix}

\end{document}
