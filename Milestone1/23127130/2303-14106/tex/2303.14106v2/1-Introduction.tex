% \documentclass[main.tex]{subfiles}
\label{sec:intro}
% \begin{document}

A transient fault in a circuit is a temporarily incorrect value at a circuit's signal, e.g., induced by radiation.
It is well known that synchronous (i.e., clocked) circuits exhibit a natural resilience against transient faults through masking. Specifically, the relevant effects are electrical masking (short fault pulses are filtered by low-pass behavior of subsequent gates and interconnect), logical masking (depending on other input levels, the logic level of the faulty input may be irrelevant for the gate output) and temporal masking (the flip-flop samples its data input at the active clock edges while ignoring faults that happen between these). However, synchronous circuits have little resilience against (fault) effects that impact the timing.
By contrast, asynchronous (i.e., self-timed, handshake-based) and in particular quasi delay-insensitive (QDI) \cite{martin1986compiling} circuits exhibit large, ideally unlimited, tolerance against timing variations by construction. This is due to their event-driven operation principle.
Unfortunately, this very event driven operation makes them prone to transient faults. While electrical masking and logical masking mitigate fault effects like in the synchronous case, it is not obvious whether considerable temporal masking occurs. Previous works have shown that asynchronous pipelines, e.g., have data accepting windows during which they are susceptible to fault pulses. The size of these windows depends on several parameters, most notably the mode of pipeline operation (bubble-limited/balanced/token-limited). For unbalanced operation these windows may reach considerable size, making the circuit clearly more susceptible to faults than in the synchronous case with its instantaneous sampling. That is why several mitigation methods \cite{bainbridge2009glitch} aim at minimizing the data accepting windows. In any case there is some effect equivalent to temporal masking, and most often it is constituted by Muller C-elements (MCEs): During the \emph{combinational} mode of operation (matching inputs), the MCE ignores fault pulses on any input and not even a pulse at the output can flip its state. In \emph{storage} mode (non-matching inputs), however, the MCE's state can be easily flipped by a fault pulse at one of the inputs or at the output (directly at the keeper). So apparently, the share of time during which an MCE is in combinational mode determines the masking provided by it. In a reasonably complex practical setting, however, this insight is hard to map to a general prediction of the whole circuit.

Given an asynchronous circuit, a natural question thus is at which signals and at which times the circuit is susceptible to a transient fault.
In this paper we present an approach to efficiently and provably exhaustively answer this question. 

\paragraph{Organization}
We discuss related work in Section~\ref{sec:relatedwork} and introduce our circuit model in Section~\ref{sec:model}.
In Section~\ref{sec:results} we start with basic consistency results
  of the model, followed by our main technical result: the definition of value regions in executions along with a proof of the equivalence of glitches within those regions (Theorem~\ref{thm:main}).
Based on this result we then present our tool for sensitivity-window exploration (Section~\ref{sec:tool}) and apply it to
a  widely used QDI circuit for illustration.
We conclude in Section~\ref{sec:conclusion}.


% \end{document}