%!TEX root = ../main.tex

\section{Introduction}%
\label{sec:introduction}
Today's vehicles accommodate a large number of interconnected \acp{ECU}, the software of which is expected to undergo 
faster development cycles with frequent updates and service reconfigurations. 
The introduction of a \ac{SOA} promises to support this increasing dynamic~\cite{kobct-osoas-17}.
The \ac{IVN} plays a fundamental role in the performance, safety, and security of these interconnected services~\cite{mrfm-svcjr-21}. 
Ethernet emerges as the next generation \ac{IVN} technology to extend capacity and flexibility, while replacing existing bus systems.
\ac{TSN} enhances Ethernet with \ac{QoS} features, e.g., deterministic communication and redundancy.

One major challenge of future \acp{IVN} is to transform from statically pre-configured into service-oriented networks that support dynamic service adaptation.
With its centralized control plane, \ac{SDN} promises a dynamic and flexible \ac{IVN}~\cite{hhlng-saeea-20}. 
In previous work, we presented \ac{TSSDN} that supports \ac{QoS} and security requirements of in-vehicle applications~\cite{hmks-snsti-19,hmks-stsnv-23}.
An SDN controller can reconfigure the network according to service availability, including updates or failures.
\ac{TSN} resource partitioning~\cite{lbp-bptjr-21} allows adding new traffic flows dynamically without affecting a static configuration defined at design time.
Additionally, failover mechanisms and seamless service mobility~\cite{erf-sbrjr-21} can improve the robustness of the \ac{IVN}.

Evolving automotive systems is challenging, as industry practices and backward interoperability need to be met at reasonable overhead~\cite{mrfm-svcjr-21}.
The \ac{SOME/IP} is a widely used protocol for automotive \acp{SOA} standardized by AUTOSAR~\cite{autosar-someip-22}.
\ac{SOME/IP} offers \ac{SD}~\cite{autosar-someip-sd-22} as a complementary service.
SDN-supported SOA in vehicles opens powerful potentials.
Options range from discovery optimizations to \ac{QoS}, security, and robustness improvements.
To support in-car \ac{SOA}, the \ac{SDN} controller must know all services on the \ac{IVN}, and thus support automotive protocols for service discovery.

In this work, we present a network control scheme for the \ac{SOME/IP} \ac{SD} based on the \ac{SDN} paradigm. 
We design an \ac{SDN} controller application that fully supports \ac{SOME/IP} \ac{SD}.
It intercepts discovery messages, learns about services, directly responds to requests, and sets up paths automatically.
Our approach is completely transparent to existing \ac{SOME/IP} implementations.
We discuss further potentials of supporting \ac{SOME/IP} communication with SDN (service discovery optimization, service mobility, \ac{QoS} improvements, discovery protection).
In simulation, we compare the scalability of our approach to non-optimized SDN and standard Ethernet.

The remainder of this work is structured as follows:
In Sec.~\ref{sec:background_and_related_work}, we discuss related work.
Sec.~\ref{sec:concept} introduces the \ac{SOME/IP} \ac{SD} mechanism, the \ac{SDN} paradigm, our methodology to support automotive \acp{SOA} with \ac{SDN}, and further potentials enabled by their combination. 
Sec.~\ref{sec:eval} evaluates the performance of the proposed \ac{SDN} supported \ac{SOME/IP} \ac{SD}.
Finally, Sec.~\ref{sec:conclusion_and_outlook} concludes this work and outlines future work.

