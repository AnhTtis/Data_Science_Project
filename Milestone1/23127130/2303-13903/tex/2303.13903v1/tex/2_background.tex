%!TEX root = ../main.tex

\section{Background and Related Work}
\label{sec:background_and_related_work}
Today, an \ac{IVN} is a complex distributed system with a multitude of devices and services communicating via a combination of bus systems and switched Ethernet.
Combined with \acf{TSN}, Automotive Ethernet is a promising candidate for the backbone of next generation \acp{IVN}.
Gateways that translate between different networks (e.g., CAN and Ethernet) and protocols are commonly used to enable interoperability and backward compatibility~\cite{iks-aiejr-22}.

\subsection{Service-Oriented Architectures in Vehicles}
\ac{SOA} is proposed to enable flexible and dynamic application placement, and frequent updates in automotive networks~\cite{kobct-osoas-17}.
In-vehicle applications can act as service providers to make certain functions and data available to consuming applications.
While dynamic service discovery can improve flexibility of, for example, navigation, infotainment, diagnostics, and driver assistance services, safety-critical services may still require dedicated, static connections to ensure reliability~\cite{chrmk-qaasc-19}.

Two major candidates are considered as the communication protocol for \acp{SOA} in vehicles~\cite{iks-aiejr-22}:
(1) \ac{SOME/IP}~\cite{autosar-someip-22} is explicitly tailored to the automotive environment and enables service-oriented communication via TCP/UDP-IP.
(2) \ac{DDS}~\cite{omg-dds-15} from the Object Management Group is a viable alternative available in the AUTOSAR platform, but not designed for automotive applications and not widely used by automotive companies~\cite{iks-aiejr-22}.
This work focuses on \ac{SOME/IP} due to its widespread deployment in the automotive domain.
Nevertheless, our work translates to other protocols such as \ac{DDS}.

In previous work~\cite{chrmk-qaasc-19}, we assessed the design space of vehicular services and proposed a mechanism to enable \ac{QoS} within a vehicular middleware.
Since rollout of \ac{SOME/IP}-\ac{SD} in AUTOSAR 4.1, \ac{SOA} is a standard feature for future \acp{IVN}.
Kampmann et al.~\cite{kakww-dsosa-19} propose containerized services to be placed and activated on dynamic allocated hardware resources during runtime.
The dynamic nature of \acp{SOA} poses challenges to a traditionally pre-configured \ac{IVN}, as it must adapt to changes in service availability during runtime.

\subsection{Supporting Automotive Network Functions with SDN}
\acf{SDN} has the potential to increase the flexibility and performance of networks~\cite{mabpp-oeicn-08}, in particular in well-known environments. 
\ac{SDN} centralizes the control plane and separates it from the data plane, allowing a central controller to perform network functions such as routing, firewalling and load balancing for the local network.

In cars, \ac{SDN} can enable a reconfigurable and flexible network architecture that adapts to changes in the network, e.g., software updates and downloadable drive assistance systems~\cite{hhlng-saeea-20}.
In previous work~\cite{hmks-snsti-19,hmks-stsnv-23}, we presented \ac{TSSDN}, an integration of SDN with TSN, and showed how \ac{SDN} can significantly enhance \ac{IVN} security.
Ergen\c{c} et al.~\cite{erf-sbrjr-21}, illustrate service-based resilience for \acp{IVN} by configuring backup nodes for critical services.
In case of failures, those can be activated, which also requires changes in the network configuration. 

Intercepting packets of network control protocols is a common approach to optimize networking objectives via \ac{SDN}.
Examples include the management of the \ac{ARP}~\cite{ap-saejr-16} or IP multicast routing~\cite{ima-smsjr-18}.
Bertaux et al.~\cite{bhmba-dbcjr-14} present a first design for an \ac{SDN} application that dynamically allocates network resources for \ac{DDS} applications.
Such a mechanism is missing in \ac{SOME/IP} and could enable the \ac{IVN} to adapt to changes during runtime.

In this work, we present a concept for an \ac{SDN} controller application that fully supports the \ac{SOME/IP} \ac{SD} protocol. 
It can intercept and handle service announcements and subscriptions to manage network resources but remains fully transparent for existing \ac{SOME/IP} implementations and applications.
