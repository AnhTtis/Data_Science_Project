%!TEX root = ../main.tex

%\section{Performance Analysis of the Prototype}
\section{Performance analysis}
\label{sec:eval}
Our study compares the performance of the presented concept compared to non-optimized SDN forwarding and standard Ethernet switches in simulation.
Our key performance metric is the time to set up all subscriptions in a network, from first producer until the last subscription is established.

\subsection{Simulation Environment}

Our simulation environment is based on the OMNeT++ simulator~\cite{omnet-inet} with the INET framework, and the OpenFlowOMNeTSuite~\cite{kj-oeojr-13}.
For our implementation, we use the CoRE4INET, SDN4CoRE~\cite{hmks-smsdn-19}, and SOA4CoRE~\cite{chrmk-qaasc-19} frameworks which our research group maintains.
They are open source on \mbox{\textit{\url{github.com/CoRE-RG}}}.
SOA4CoRE implements the \ac{SOME/IP} and \ac{SOME/IP} \ac{SD} protocol, and our proposed controller application for \ac{SOME/IP} \ac{SD} is added to SDN4CoRE.

\subsection{Evaluation Scenario}
Fig.~\ref{fig:scalingtopo} shows the topology used for the comparison.
It consists of switches, producer nodes, and consumer nodes connected via \SI{1}{\giga\bit\per\second} Ethernet links.
All switches are connected to a controller in the SDN variants.

To investigate scalability, we vary the number of producer (P) and consumer nodes (C) from 1 to 50, and the number of switches between them from 1 to 5, since we expect that this number is not exceeded in a real \ac{IVN}.
Each producer has one publisher service.
Each consumer has one subscriber service per publisher.
We simulate all 192 parameter combinations and measure the time to set up all subscriptions.

The simulation models provide a wide range of configuration parameters.
Switches have a hardware forwarding delay of \SI{8}{\micro\second}.
The SDN controller uses the OpenFlow protocol to configure the switches.
The OpenFlow messages processing time in the controller application is \SI{100}{\micro\second}, based on the worst-case performance of the best-performing controller implementation we have evaluated in previous work~\cite{rhmks-rapesc-20}.
We assume the switch processing time is similar and set it to \SI{100}{\micro\second}, but could not find any data in the literature on the performance of OpenFlow processing on switches.
The controller and the switches can handle multiple OpenFlow packets in parallel.

\begin{figure}
    \centering
    \includegraphics[width=1.0\linewidth, trim=0.62cm 0.62cm 0.62cm 0.72cm, clip=true]{scaling_topo.pdf}
    \caption{Topology for the scalability comparison with 1 to 5 switches (S), 1 to 50 producers (P), and 1 to 50 consumer nodes (C) with one subscriber service per producer (P).}
    \label{fig:scalingtopo}
    \vspace{-7pt}
\end{figure}

\subsection{Results}

\begin{figure*}
	\centering
    \begin{tikzpicture}
        \begin{axis}[width=.32\linewidth, height=.28\linewidth,
            title={1 switch},
            every axis title/.style={below right,at={(0,1)},draw=black,fill=white},
            change y base,
            y unit=\second,
			ylabel=log(startup time),
			xlabel={Producers},
            x unit=\#,
            xmin=1, xmax=50,
            ymin=0.00001, ymax=0.1,
            ymode=log,
            xtick={1,10,20,30,40,50},
            ytick={0.00001,0.0001,0.001,0.01,0.1,1,10},
            xtick pos=bottom,
            ]
            \addplot [CoreGreen, dashed, thick] table [x, y, col sep=comma] {./data/scaling/ethernet_S=1_C=1.csv};\label{plots:eth_1}
            \addplot [CoreRed, dashed, thick] table [x, y, col sep=comma] {./data/scaling/parallel_100c_100s_sdn_S=1_C=1.csv};\label{plots:sdn_1}
            \addplot [CoreBlue, dashed, thick] table [x, y, col sep=comma] {./data/scaling/parallel_100c_100s_vanilla_S=1_C=1.csv};\label{plots:vanilla_1}
            \addplot [CoreGreen, thick] table [x, y, col sep=comma] {./data/scaling/ethernet_S=1_C=50.csv};\label{plots:eth_50}
            \addplot [CoreRed, thick] table [x, y, col sep=comma] {./data/scaling/parallel_100c_100s_sdn_S=1_C=50.csv};\label{plots:sdn_50}
            \addplot [CoreBlue, thick] table [x, y, col sep=comma] {./data/scaling/parallel_100c_100s_vanilla_S=1_C=50.csv};\label{plots:vanilla_50}
        \end{axis}
    \end{tikzpicture}
    \begin{tikzpicture}
        \begin{axis}[width=.32\linewidth, height=.28\linewidth,
            title={2 switches},
            every axis title/.style={below right,at={(0,1)},draw=black,fill=white},
            change y base,
			xlabel={Producers},
            x unit=\#,
            xmin=1, xmax=50,
            ymin=0.00001, ymax=0.1,
            ymode=log,
            xtick={1,10,20,30,40,50},
            ytick={0.00001,0.0001,0.001,0.01,0.1,1,10},
            xtick pos=bottom,
            ]
            \addplot [CoreGreen, dashed, thick] table [x, y, col sep=comma] {./data/scaling/ethernet_S=2_C=1.csv};
            \addplot [CoreRed, dashed, thick] table [x, y, col sep=comma] {./data/scaling/parallel_100c_100s_sdn_S=2_C=1.csv};
            \addplot [CoreBlue, dashed, thick] table [x, y, col sep=comma] {./data/scaling/parallel_100c_100s_vanilla_S=2_C=1.csv};
            \addplot [CoreGreen, thick] table [x, y, col sep=comma] {./data/scaling/ethernet_S=2_C=50.csv};
            \addplot [CoreRed, thick] table [x, y, col sep=comma] {./data/scaling/parallel_100c_100s_sdn_S=2_C=50.csv};\label{plots:sdn_50}
            \addplot [CoreBlue, thick] table [x, y, col sep=comma] {./data/scaling/parallel_100c_100s_vanilla_S=2_C=50.csv};
        \end{axis}
    \end{tikzpicture}
    \begin{tikzpicture}
        \begin{axis}[width=.32\linewidth, height=.28\linewidth,
            title={5 switches},
            every axis title/.style={below right,at={(0,1)},draw=black,fill=white},
            change y base,
			xlabel={Producers},
            x unit=\#,
            xmin=1, xmax=50,
            ymin=0.00001, ymax=0.1,
            ymode=log,
            xtick={1,10,20,30,40,50},
            ytick={0.00001,0.0001,0.001,0.01,0.1,1,10},
            xtick pos=bottom,
            ]
            \addplot [CoreGreen, dashed, thick] table [x, y, col sep=comma] {./data/scaling/ethernet_S=5_C=1.csv};
            \addplot [CoreRed, dashed, thick] table [x, y, col sep=comma] {./data/scaling/parallel_100c_100s_sdn_S=5_C=1.csv};
            \addplot [CoreBlue, dashed, thick] table [x, y, col sep=comma] {./data/scaling/parallel_100c_100s_vanilla_S=5_C=1.csv};
            \addplot [CoreGreen, thick] table [x, y, col sep=comma] {./data/scaling/ethernet_S=5_C=50.csv};
            \addplot [CoreRed, thick] table [x, y, col sep=comma] {./data/scaling/parallel_100c_100s_sdn_S=5_C=50.csv};\label{plots:sdn_50}
            \addplot [CoreBlue, thick] table [x, y, col sep=comma] {./data/scaling/parallel_100c_100s_vanilla_S=5_C=50.csv};
        \end{axis}
    \end{tikzpicture}
    \begin{tikzpicture}
        \hspace{8pt}
        \matrix[
            font=\footnotesize,
            matrix of nodes,
            align=left,
            anchor=west,
            inner sep=0.1em,
            draw
            ]
            {
                \ref{plots:eth_1}&{\hspace{-1pt}w/o SDN, \hspace{4pt}1 consumer \hspace{3pt}per producer}&[5pt]
                \ref{plots:sdn_1}&{\hspace{-1pt}SDN optimized, \hspace{4pt}1 consumer \hspace{3pt}per producer}&[5pt]
                \ref{plots:vanilla_1}&{\hspace{-1pt}SDN vanilla, \hspace{4pt}1 consumer \hspace{3pt}per producer}\\
                \ref{plots:eth_50}&{w/o SDN, 50 consumers per producer}&[5pt]
                \ref{plots:sdn_50}&{SDN optimized, 50 consumers per producer}&[5pt]
                \ref{plots:vanilla_50}&{SDN vanilla, 50 consumers per producer}\\
            };
    \end{tikzpicture}
    \caption{
        Comparison of the presented approach (SDN optimized) to non-optimized SDN forwarding (SDN vanilla) and standard Ethernet switches (w/o SDN) in terms of scalability.
        The logarithmic scale depicts the total time to set up all subscriptions for an increasing number of producers with 1 and 50 consumers per producer.
    }
    \label{fig:ethvsSDNovsSDNv}
    \vspace{-8pt}
\end{figure*} 

Fig.~\ref{fig:ethvsSDNovsSDNv} compares the setup time of the presented approach (SDN optimized), non-optimized SDN forwarding (SDN vanilla), and standard Ethernet (w/o SDN) for 1 to 50 producers on 1, 2, and 5 switches on a logarithmic scale. 
For simplicity, the graphs only depict results for 1 and 50 consumers per producer, the others show a similar trend.

In all cases, we observe an approximately linear increase in setup time with the number of producers for all three approaches.
Compared to standard Ethernet switching, the SDN solutions are about one order of magnitude slower, which is to be expected due to the delay caused by forwarding each \ac{SOME/IP} \ac{SD} packet to the central \ac{SDN} controller. 
This delay is highly dependent on the OpenFlow processing time.
Optimized controllers, and switches might come closer to the Ethernet performance.
Nevertheless, the additional delay only affects the setup time of the subscriptions and not the actual data transfer of the service.

For setting up one subscription over 5 switches, the optimized SDN approach takes \SI{0.6}{\milli\second}, non-optimized SDN \SI{2.2}{\milli\second} and the non-SDN variant only \SI{0.1}{\milli\second}.
This is a good indicator that the time to migrate a service from one node to another is less than \SI{1}{\milli\second}.
The setup of subscriptions for 50 publishers with 50 subscribers each (2500 subscriptions), is considered a cold start as all services announce their availability and their required subscriptions at the same time.
The setup for all connections takes about \SI{5.5}{\milli\second} without SDN, \SI{16.4}{\milli\second} with vanilla SDN, and \SI{9.3}{\milli\second} with our optimized SDN approach. 

Overall, the presented approach of a SOME/IP-aware SDN controller improves performance by up to 50\% compared to a non-optimized SDN.
Although a migration time of \SI{1}{\milli\second} is acceptable for most in-vehicle services, it may not be acceptable for safety-critical services, e.g., collision detection, which likely require configured routes and redundancy.
Most services, however, are discovered when the vehicle is started.
The fastest required service availability in current cars is about \SI{200}{\milli\second} and even lower for, e.g., infotainment services.
Therefore, the setup time for all in-vehicle connections (2500 services in less than \SI{10}{\milli\second}) is acceptable for all kinds of services, including safety-critical services. 
