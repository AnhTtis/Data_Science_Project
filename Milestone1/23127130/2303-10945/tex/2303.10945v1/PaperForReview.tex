% CVPR 2022 Paper Template
% based on the CVPR template provided by Ming-Ming Cheng (https://github.com/MCG-NKU/CVPR_Template)
% modified and extended by Stefan Roth (stefan.roth@NOSPAMtu-darmstadt.de)

\documentclass[10pt,twocolumn,letterpaper]{article}

%%%%%%%%% PAPER TYPE  - PLEASE UPDATE FOR FINAL VERSION
% \usepackage[review]{cvpr}      % To produce the REVIEW version
% \usepackage{cvpr}              % To produce the CAMERA-READY version
\usepackage[pagenumbers]{cvpr} % To force page numbers, e.g., for an arXiv version

% Include other packages here, before hyperref.
\usepackage{graphicx}
\usepackage{amsmath}
\usepackage{amssymb}
\usepackage{booktabs}
\usepackage{ifpdf}
\usepackage{algorithmic}
\usepackage{algorithm}
\usepackage{ulem}
\usepackage{CJKulem}
% \usepackage{cite}
\usepackage{color}
\usepackage{times}
\usepackage{epsfig}
\usepackage{url}

\usepackage{amsmath}
\usepackage{amssymb}
\usepackage{diagbox}
\usepackage{multirow}
%\usepackage{floatrow}
\usepackage{mathrsfs}
\usepackage{array}
%\usepackage{subcaption}
%\usepackage{subfig}
%\usepackage{booktabs}
\usepackage{float}
%\usepackage[numbers,sort&compress]{natbib}
\usepackage{url}
\usepackage{array}
\usepackage{ulem}
\usepackage{soul}
\usepackage{algorithm}
\usepackage{algorithmic}
% \renewcommand{\algorithmicrequire}{\textbf{Input:}}  % Use Input in the format of Algorithm
% \renewcommand{\algorithmicensure}{\textbf{Output:}} % Use Output in the format of Algorithm
% It is strongly recommended to use hyperref, especially for the review version.
% hyperref with option pagebackref eases the reviewers' job.
% Please disable hyperref *only* if you encounter grave issues, e.g., with the
% file validation for the camera-ready version.
%
% If you comment hyperref and then uncomment it, you should delete
% ReviewTempalte.aux before re-running LaTeX.
% (Or just hit 'q' on the first LaTeX run, let it finish, and you
%  should be clear).
\usepackage[pagebackref,breaklinks,colorlinks]{hyperref}
\usepackage{hyperref}

% Support for easy cross-referencing
\usepackage[capitalize]{cleveref}
\crefname{section}{Sec.}{Secs.}
\Crefname{section}{Section}{Sections}
\Crefname{table}{Table}{Tables}
\crefname{table}{Table}{Tabs.}

\makeatletter
\newcommand{\rmnum}[1]{\romannumeral #1}
\newcommand{\Rmnum}[1]{\expandafter\@slowromancap\romannumeral #1@}
\makeatother

%%%%%%%%% PAPER ID  - PLEASE UPDATE
\def\cvprPaperID{} % *** Enter the CVPR Paper ID here
\def\confName{}
\def\confYear{}

\normalem


\begin{document}
%%%%%%%%% TITLE - PLEASE UPDATE
\title{Open-World Pose Transfer via Sequential Test-Time Adaption}

\author{Junyang Chen\textsuperscript{1} \quad Xiaoyu Xian\textsuperscript{2} \quad Zhijing Yang\textsuperscript{1} \quad Tianshui Chen\textsuperscript{1} \quad Yongyi Lu\textsuperscript{1} \quad Yukai Shi\textsuperscript{1*} \\ Jinshan Pan\textsuperscript{3} \quad Liang Lin\textsuperscript{4}\\
\textsuperscript{1}Guangdong University of Technology \quad
\textsuperscript{2}CRRC Academy\\
\textsuperscript{3} Nanjing University of Science and Technology \quad
\textsuperscript{4} Sun Yat-sen University\\
{\tt\small \{jychen9811, tianshuichen, yylu1989, sdluran\}@gmail.com }\\ {\tt\small  \{yzhj, ykshi\}@gdut.edu.cn \quad xxy@crrc.tech \quad linliang@ieee.org}}
% For a paper whose authors are all at the same institution,
% omit the following lines up until the closing ``}''.
% Additional authors and addresses can be added with ``\and'',
% just like the second author.
% To save space, use either the email address or home page, not both

% \maketitle
\twocolumn[{%
\renewcommand\twocolumn[1][]{#1}%
\maketitle
\begin{center}
\hsize=\textwidth % cvpr 需要
\centering
\includegraphics[width=1.0\linewidth]{fig/abstract_row.pdf}
% \vspace{-4mm}
\captionof{figure}{Visualization of open-world pose transfer (OWPT). With open-world references, it can be observed that typical pose transfer method (\emph{e.g.} NTED~\cite{Pose_one_4}) exhibits a twisty pattern. In this sense, we call for a solid model that can handle open-world instances beyond a specific dataset. }
\label{fig:OWPT}
%we call for a solid model can transfer any identity freely beyond a specific dataset.}
\end{center}
}]

% \footnotetext[1]{More demo is available at: https://github.com/conferenceAnonymous/TTA-Pose-Transfer}




% \begin{figure*}[!h]
%     % \vspace{-2mm}
% 	\begin{center}
	    
% 		\includegraphics[width=1.0\linewidth]{fig/abstract_row.pdf}
% 	\end{center}
%  	\vspace{-5mm}
% 	\caption{Visualization of open-world pose transfer (OWPT). With open-world references, it can be observed that typical pose transfer method (\emph{e.g.} NTED~\cite{Pose_one_4}) exhibits a twisty pattern. In this sense, we call for a solid model that can handle open-world instances beyond a specific dataset. }%trained on DeepFashion~\cite{liu2016deepfashion} dataset. The pre-trained model performs reasonably well on the source domain data. However, the synthesis results are easily violated on the OOD data. Hence, we first propose Open-World Pose Transfer task to investigate 'How to keep the pre-trained model performance on the OOD data'.}
% 	\label{fig:introduce_image}
% 	\label{fig:OWPT}
% 	\vspace{-6mm}
% \end{figure*}


%%%%%%%%% ABSTRACT


% \begin{figure*}[t]
% 	\begin{center}
% 		\includegraphics[width=0.95\linewidth]{fig/abstract.pdf}
% 	\end{center}
% 	%\vspace{-3mm}
% 	\caption{Comparison between with and without self-supervised appearance guidance on the pre-trained model.}
% 	\label{fig:comp_SSL_init}
% 	\vspace{-3mm}
% \end{figure*}
\begin{abstract}
Pose transfer aims to transfer a given person into a specified posture, has recently attracted considerable attention. A typical pose transfer framework usually employs representative datasets to train a discriminative model, which is often violated by out-of-distribution (OOD) instances. Recently, test-time adaption (TTA) offers a feasible solution for OOD data by using a pre-trained model that learns essential features with self-supervision. However, those methods implicitly make an assumption that all test distributions have a unified signal that can be learned directly. In open-world conditions, the pose transfer task raises various independent signals: OOD appearance and skeleton, which need to be extracted and distributed in speciality. To address this point, we develop a SEquential Test-time Adaption (SETA). In the test-time phrase, SETA extracts and distributes external appearance texture by augmenting OOD data for self-supervised training. To make non-Euclidean similarity among different postures explicit, SETA uses the image representations derived from a person re-identification (Re-ID) model for similarity computation. By addressing implicit posture representation in the test-time sequentially, SETA greatly improves the generalization performance of current pose transfer models. In our experiment, we first show that pose transfer can be applied to open-world applications, including Tiktok reenactment and celebrity motion synthesis.
\end{abstract}



\section{Introduction}
Pose transfer aims at transforming a source person into a target posture, while maintaining the original appearance. Previously, some pose transfer works~\cite{Pose_one_1, Pose_one_2, Pose_one_3, Pose_one_4, Pose_one_5, Pose_one_6} achieve charming results on a specific dataset~\cite{liu2016deepfashion} by assuming the prior conditions of the target sample are very similar to the training sample. As shown in Fig.~\ref{fig:OWPT}, this assumption could be easily violated in practice due to the out-the-distribution (OOD) data from real-world applications. How to make those pose transfer models great again on OOD data is still a non-trivial challenge.  


% \begin{figure}[!t]
% \centering
% \includegraphics[width=1.0\linewidth,height=0.9\linewidth]{fig/abstract4.pdf}
% \vspace{-3mm}
% \caption{Visualization of open-world pose transfer (OWPT). With open-world references, it can be observed that typical pose transfer method (\emph{e.g.} NTED~\cite{Pose_one_4}) exhibits a twisty pattern. In this sense, we call for a solid model that can handle open-world instances beyond a specific dataset.} 
% %\label{fig:introduce_image}
% \label{fig:OWPT}
% \vspace{-5mm}
% \end{figure}


Recently, with the increasing attention of pose transfer, many datasets~\cite{SHHQ,DeepFashion2,Tiktok} are proposed for various contexts, which makes those datasets vastly different from each other across many domains. DeepFashion~\cite{liu2016deepfashion} contains more pretty consumers with fashion clothes, which makes SHHQ dataset~\cite{SHHQ} differ significantly from it in terms of clothing, age and posture. Tiktok~\cite{Tiktok} dataset has more dance routines that make itself different from DeepFashion in both posture and appearance. 

To show the discriminative characteristic of pose transfer datasets~\cite{SHHQ,DeepFashion2,Tiktok} with more quantitative evidence, we obtain the high-level feature by a person ReID~\cite{ReID_model} model and visualize them in Fig.~\ref{fig:motivation} (a). More specifically, the ReID model is trained on Market-1501~\cite{market_1501}, and the high-level features are fetched from the $1^{st}$ layer. As can be seen in Fig.~\ref{fig:motivation} (a), each dataset exhibits a distinct yet independent pattern, from which we realize such bias literally can be inherited by deep models. Once a discriminative model is trained on a specified dataset, the inherited bias affects its ability to process OOD data. Considering the application of pose transfer tasks in real-world scenarios, it is necessary to make pose transfer models overcome such bias by resolving the distribution shift between the source domain and test domains.


To alleviate the distribution shift problem, a family of methods~\cite{DA1,DA2,DA3,DA4,DA5,DA10,DA11,DA12} based on domain adaption (DA) have been proposed, which assume the target data are accessible during model adaption. Taking advantage of this privileged data, the generalization ability is significantly enhanced. However, user data is often considered private content, which makes it unfeasible for upload, annotation and re-training. Domain generalization~\cite{DG1,DG2,DG3,DG4,DG5,DG6} (DG) addresses domain shift without pre-fetching OOD data. It extends the diversity of datasets to learn more generalized features. Nevertheless, constructing an expensive dataset does not always hold in practice, new cases always appear to cause new trouble. Recently, test-time adaption (TTA) methods~\cite{Tent, TTT} blur the boundary between DA and DG by assuming OOD data can be used in local device without annotation. As suggested in \cite{TTT}, in TTA, a pre-trained model learns essential feature representations from OOD data with self-supervision~\cite{gandelsman2022test}. In this sense, TTA~\cite{TTT,Tent} methods simply optimize model for the test distribution straightway. This flexible learning paradigm makes strong adaptability toward OOD data. However, those methods implicitly make an assumption that \emph{all test distributions have a unified signal that can be learned directly.}

In open-world conditions, the pose transfer task raises various yet non-trivial signals: out-the-distribution (OOD) appearance and skeleton. This means there are various domain knowledge required to be extracted in speciality. As shown in Fig.~\ref{fig:motivation} (b), we train NTED~\cite{Pose_one_4} with DeepFashion dataset, and apply the trained model on DeepFashion2, SHHQ and Tiktok datasets for inference. As can be seen in Fig.~\ref{fig:motivation} (b), the generated results easily show a twisty pattern with OOD appearance/skeleton. However, typical TTA methods were not designed to learn disentangled signals individually and originally. Therefore, we develop a SEquential Test-time Adaption (SETA) for independent signals adaption. Meanwhile, we develop an appearance adapter for external appearance texture extraction and distribution. Since the postures derived from OOD skeletons are non-Euclidean, we employ a person Re-ID~\cite{ReID_model} to extract the representations of each 
posture to compute the consistency over motion. SETA significantly improves the generalization performance toward OOD instances by fetching independent OOD signals sequentially. The contributions are summarized as follows:


\begin{figure*}[ht]
    % \vspace{-2mm}
	\begin{center}
	    
		\includegraphics[width=0.95\linewidth,height=0.35\linewidth]{fig/intro_img.pdf}
	\end{center}
 	\vspace{-5mm}
	\caption{(a) To investigate the domain gap between source and OOD datasets, we obtain the high-level feature by a person ReID model~\cite{ReID_model} and use t-SNE for visualization. (b) The domain generalization of NTED~\cite{Pose_one_4}. Typical pose transfer model performs reasonably well on the source domain, however, the generated results could be easily violated with OOD input. We first propose an Open-World Pose Transfer (OWPT) framework to investigate the domain generalization of a pre-defined model toward OOD appearance and skeleton. }%trained on DeepFashion~\cite{liu2016deepfashion} dataset. The pre-trained model performs reasonably well on the source domain data. However, the synthesis results are easily violated on the OOD data. Hence, we first propose Open-World Pose Transfer task to investigate 'How to keep the pre-trained model performance on the OOD data'.}
	\label{fig:motivation}
	\vspace{-6mm}
\end{figure*}


%During SETA, we consider different domain signal as a `task' using the TTA. Our goal is to learn the disentangled signals via the sequential tasks so that the updated parameters can strongly adapt toward OOD data, and generate more realistic results (See Fig.~\ref{fig:OWPT}).

% Formerly, pose transfer could be easily violated in practice due to the Out-Of-Distribution (OOD) input. To improve the practicability, we first build pose transfer under a open-world settings (OWPT), by which the model can handle arbitrary characteristic and skeleton.

%(2).    Test-time adaption (TTA) is renewed for a multiple domain scene. In our model, multiple domain knowledge ( e.g., identity, appearance, posture ) are imported by a SEquential Test-Time Adaption (SETA) strategy.  
(1)    Test-time adaption (TTA) is renewed for a various domain scene. In our model, multiple OOD domain knowledge ( \emph{e.g.} appearance, posture ) are SEquentially learned and distributed in Test-Time Adaption (SETA).  

(2)    To learn implicit appearance representation among OOD postures, SETA cleverly employs the image representations derived from a person re-identification (Re-ID) model to obtain non-Euclidean consistency over motion.

(3)    Pose transfer is further extended for an open-world environment for the first time. For the first time, we conduct pose transfer into open-world applications, including Tiktok reenactment and celebrity motion synthesis.


\vspace{-2mm}
\section{Related Work}


\paragraph{Test-Time Adaption.} Test-Time Adaption (TTA) aims to use the existing model to quickly adapt to OOD data during the test stage. Recently, some TTA methods~\cite{Tent, TTT, Lilun1, Lilun2, Lilun3, Lilun4} have been proposed for model generalization to OOD data. Sun \emph{et al}.~\cite{TTT} propose to apply the self-supervised proxy task to update model parameters on target data. Wang \emph{et al}.~\cite{Tent} introduce a entropy minimization method to optimize the parameters of the batch normalization layers. As the promising application prospect, TTA has been extended to several tasks~\cite{Deblurring, Dehazing, Super_resolution, RL, MM_TTA}. Chi \emph{et al}.~\cite{Deblurring} propose a meta-auxiliary learning paradigm for fast updating model parameters in dynamic scene deblurring task. Liu \emph{et al}.~\cite{Dehazing} update model parameters with the self-training signals from the proposed self-reconstruction method. Shin \emph{et al}.~\cite{MM_TTA} introduce cross-modal pseudo labels as self-training signals. These previous works always focus on single kind of self-training signal. However, when facing various OOD domain knowledge, the typical paradigm of TTA needs to be re-examined. In comparison to the aforementioned works, we need to provide various self-training signals in open world pose transfer. Hence, we develop a Sequential Test-Time Adaption for learning multiple domain knowledge with various self-training signals.

\vspace{-4mm}

\paragraph{Pose Transfer.} Pose Transfer has been an attractive topic in the image synthesis community since Ma \emph{et al}.~\cite{Pose_1} proposed. Up to now, many pipelines have been proposed, which can be classified as multi-stage approaches~\cite{Pose_multi_1, Pose_multi_2, Pose_multi_3, Pose_multi_4} and one-stage methods~\cite{Pose_one_1, Pose_one_2, Pose_one_3, Pose_one_4, Pose_one_5, Pose_one_6}. The former presents a coarse-to-fine structure which utilizes coarse shape or foreground masks to ensure generalization on arbitrary poses. However, these methods are not efficient at inference and require additional computing power. ADGAN~\cite{Pose_one_2} ameliorates this issue by extracting expressive textures vectors from different semantic entities to synthesize the target image. CASD~\cite{Pose_one_3} introduces attention-based methods to distribute the semantic vectors to the target poses. Compared to these parser-based methods, NTED~\cite{Pose_one_4} applies sparse attention-based operation to extract the semantic textures without the assistance of the external parser. Although these methods have been validated well on the DeepFashion dataset~\cite{liu2016deepfashion}, there is no relevant research to extend these pre-trained models to OOD dataset. Our work first explores the performance of these models on OOD data. However, the performance still remains a large gap between photo-realistic image due to the pre-trained models overfitting on DeepFashion dataset~\cite{liu2016deepfashion}.

\vspace{-4mm}
\section{Preliminaries}

\subsection{Pose Transfer}
Given a reference image, traditional pose transfer methods aim at synthesizing high fidelity images with different poses. Most pose transfer approaches adopt a similar pipeline: a \emph{texture encoder} extracts appearance characters from the reference images, a \emph{skeleton encoder} describes the semantic distribution from the target poses. Finally, a \emph{generator} is used to produce high fidelity images via transferring the appearance texture to the target poses under the semantic distribution guidance. 

\subsection{Bottlenecks}
Traditional pose transfer approaches have the ability to extract and distribute the human appearance texture within a fixed dataset. However, as shown in Fig.~\ref{fig:motivation}, this ability is limited by the domain gap. In the inference stage, since the training data and test data are drawn from different distribution, even a minor differences turn out to weaken state-of-the-art approaches. Thus, \emph{`How to fetch desired signals from OOD samples'} is crucial to pave the way for open-world pose transfer task.


\section{Proposed Method}
Given a discriminative model $f_{\theta}$ trained on a representative dataset~\cite{liu2016deepfashion}, our goal is to generate the realistic images on out-of-distributions (OOD) data. In this section, we introduce an effective solution to learn the disentangled OOD signals (\emph{i.e.} OOD apearance signals, OOD skeleton signals) sequentially. First, we update the pre-trained parameters ${\theta}$ via OOD appearance signals, which are fetched from OOD data and its augmentation. Then, we introduce the ReID model to obtain the consistency over motion during pose transformation. Finally, we fetch OOD skeleton signals from the consistency over motion to update the previously adapted parameters.

% Typical pose transfer pipelines~\cite{Pose_one_4,Pose_one_3,Pose_one_2} apply a reconstruction term $L_{rec}$, a perceptual term $L_{perc}$~\cite{VGGloss}, an adversarial term $L_{adv}$ and an attention term $L_{att}$~\cite{Pose_one_4} as objective function. The formulation is given as follows:
% \begin{equation}
% L_{Training}=L_{rec} + L_{perc} + L_{adv} + L_{att}.
% \label{Training_loss}
% \end{equation}
% where $L_{rec}$ is formulated as the L1 distance, $L_{perc}$ is computed on the VGG layers, $L_{adv}$ is calculated with the discriminators, and $L_{att}$ is used to calculate L1 distance at each operation layer.




\begin{figure*}[t]
     \vspace{-1mm}
	\begin{center}
		\includegraphics[width=0.95\linewidth,height=0.33\linewidth]{fig/pipeline.pdf}
	\end{center}
 	\vspace{-5mm}
	\caption{(a) Overview of Sequential Test-Time Adaption. (b) We optimize the pre-trained parameters $\theta$ by external appearance signals, which fetched from OOD data with augmentation, in the test stage. Note that $\hat{I}^{id} = f_{\theta}(I^{aug},P^{id})$ and $\hat{I}^{aug} = f_{\theta}(I^{id},P^{aug})$. $P^{id}$ and $P^{aug}$ are the skeleton of $I^{id}$ and $I^{aug}$. (c) We introduce arbitrary OOD skeletons to generate the pose transfer images with the updated parameters $\hat{\theta}$. A person ReID model~\cite{ReID_model} is used to obtain the consistency over motion between the input image and generated images. Then, we fetch the implicit posture representation from the consistency to optimize $\hat{\theta}$ in the test-time adaption stage for OOD skeletons. }
% 	\caption{(a) Overview of Sequential Test-Time Adaption. First, we use a composition of data augmentation operations to generate augmented samples. Second, we optimize the pre-trained parameters $\theta$ by external appearance signals fetched from OOD data and its derivatives in the test-time stage. Third, we introduce arbitrary OOD skeletons to generate the pose transfer images with the updated parameters $\hat{\theta}$. A person ReID model~\cite{ReID_model} is used to obtain the consistency over motion between input image and generated images. Finally, we fetch the self-training signals from the consistency to optimize the updated parameters $\hat{\theta}$ in the test-time adaption stage for OOD skeletons. (b) Illustration of the self-training process for OOD appearance domain adaption. $\hat{I}^{id} = f_{\theta}(I^{aug},P^{id})$ and $\hat{I}^{aug} = f_{\theta}(I^{id},P^{aug})$. $P^{id}$ and $P^{aug}$ are the skeleton of $I^{id}$ and $I^{aug}$. (c) As the person image is deployed into new posture, skeleton adapter cleverly applies an automatic similarity learning with a person ReID model for global- and local- similarity computation. }%We update the model parameters via the fine-tuning signals provided from Identity loss.}
	
% 	(a) Illustrations of the stochastic data augmentation operators. Our model learns richer representation by transferring data stochastically. (b) Illustration of the proposed self-supervised representation learning for open-world appearance domain adaption. Specifically, open-world images are transferred by a inner loop, which includes the fixed, rotate, scale and erase metrics. Then, we update model's parameters based on open-world instances and their augmented representations with a self-supervised manner.
	\label{fig:pipeline}
 	\vspace{-5mm}
\end{figure*}

% i) Appearance  

% ii) Motion

\subsection{Open Appearance}

Inspired by recent progress in TTA, we develop an \textbf{\emph{appearance adapter}} to learn essential knowledge from OOD appearance sequentially.

\vspace{-4mm}

\paragraph{Fetch Appearance Signals.}
In test-time phrase, given a person image $I^{id}$, we apply the appearance adapter to learn OOD appearance signals. First, we use data augmentation to generate samples for self-supervised training. Each augmentation can transform data stochastically with some internal parameters (\emph{e.g.} rotation angle, scale degree, mask ratio). The augmented samples are denoted as $I^{aug}$. As shown in Fig.~\ref{fig:pipeline} (b), given a pair of training images $(I^{id}, I^{aug})$, a pre-trained model $f_\theta$ with parameter $\theta$, we update $f_\theta$ to learn the signals from OOD appearance domain based on the loss $L_{Appe}$, which includes reconstruction loss $L_{rec}$, perceptual loss~\cite{VGGloss} $L_{perc}$ and 
attention loss~\cite{Pose_one_4} $L_{att}$. Thus, the appearance adaptive loss is given as follows:  
\begin{equation}
    % \vspace{-5mm}
    \begin{split}
        L_{Appe}=L_{rec} + L_{perc} + L_{att}.
    \end{split}
\end{equation}
where $L_{rec}$ is formulated as the L1 distance, $L_{perc}$ is computed on the VGG layers, and $L_{att}$ is used to calculate L1 distance at each operation layer.

In test-time phrase, we optimize the self-supervised loss $L_{Appe}$ over OOD data and its derivatives that drawn from the test distribution $Q$, which is defined as follows:
% The formulation of adaption for appearance domain is defined as follows:
% Thus, we optimize $\theta$ to personalize the appearance domain of OOD data by $L_{Appe}$, formulated as
% \vspace{-3mm}
\begin{equation}
    % \vspace{-5mm}
    \begin{split}
    \hat{\theta} = &\mathop{\arg\min}_{\theta} \mathbb{E}_Q\begin{bmatrix}
        L_{Appe}(\hat{I}^{id}, I^{id}; \theta)+\\
        L_{Appe}(\hat{I}^{aug}, I^{aug}; \theta)
        \end{bmatrix}.
    \end{split}
\end{equation}
where $\hat{I}^{id} = f_{\theta}(I^{aug},P^{id})$ and $\hat{I}^{aug} = f_{\theta}(I^{id},P^{aug})$, $P^{id}$ and $P^{aug}$ are the skeleton of $I^{id}$ and $I^{aug}$. $\theta$ and $\hat{\theta}$ indicate the pre-trained model parameters and the adapted parameters learned from OOD appearance signals, and $E_Q$ is evaluated on an OOD appearance distribution $Q$. 

% \begin{figure}[t]
% 	\centering
% 		\includegraphics[width=0.95\linewidth,height=0.7\linewidth]{fig/SSL_v2.pdf}
%  	\vspace{0mm}
% 	\caption{Visual comparison of pre-trained model (\emph{i.e.} NTED~\cite{Pose_one_4}) with and without TTA on the appearance domain.}
% 	\label{fig:comp_SSL_init}
%  	\vspace{-3mm}
% \end{figure}

% \begin{equation}
%     % \vspace{-5mm}
%     \begin{split}
%     \hat{\theta} = &\mathop{\min}_{\theta} E[\ \  \frac{1}{2N}\sum_{i=1}^{N} \| L_{Appe}(f_{\theta}(I^{id}, P^{aug}_i), I^{aug}_i) + \\ 
%     &\quad L_{Appe}(f_{\theta}(I^{aug}_i, P^{id}), I^{id})\|] 
%     \end{split}
% \end{equation}

% where $\theta$ and $\hat{\theta}$ indicates the pre-trained model parameters and the adaptive parameters learned from OOD appearance siganls, respectively. $N$ is the number of the augmented samples. $P^{id}$ and $P^{aug}_i$ are the skeleton of $I^{id}$ and $I_{i}^{aug}$.

% \begin{equation}
% \tilde{\theta} \gets \theta - \alpha\nabla_\theta L_{ori}(x_t, x_{aug};\theta).
% \label{gradients_1}
% \end{equation}
% \vspace{-4mm}

% where $\alpha$ is the adaptation learning rate. Here $\tilde{\theta}$ can be seen as the model parameters adapted to the input $x_t$ via \textbf{SSLA}. 
\paragraph{Open Appearance Deployment.} The updated model $f_{\hat{\theta}}$ has been learned specifically to facilitate adaptation to OOD appearance domain. In inference stage, the appearance texture of OOD data could be extracted and then distributed according to the semantic distribution of skeleton via $f_{\hat{\theta}}$. 

% As illustrated in Fig.~\ref{fig:comp_SSL_init}, in comparison with the pre-trained model, the results with appearance adapter perform reasonably well in maintaining the appearance texture of OOD data. 






\subsection{Open Skeleton}

\paragraph{Fetch Skeleton Signals.}
To learn the implicit posture representation in pose transformation, we develop a \textbf{\emph{skeleton adapter}} to fetch OOD skeleton signals from the consistency over motion. As shown in Fig. \ref{fig:pipeline} (c), non-Euclidean consistency exists between the original and transferred person image.
% As shown in Fig.~\ref{fig:comp_appe_skl}, the appearance texture of `w/o skeleton adapter' are inconsistent, owing to the motion from pose transformation. To overcome this issue, we develop a \textbf{\emph{skeleton adapter}} to fetch OOD skeleton signals from the consistency over motion. As shown in Fig. \ref{fig:pipeline} (c), non-Euclidean consistency exists between the original and transferred person image.
% As shown in Fig.~\ref{fig:comp_appe_skl}, the appearance texture of `w/o skeleton adapter' are inconsistent, owing to the motion from pose transformation. To overcome this issue, we develop a \textbf{\emph{skeleton adapter}} to fetch OOD skeleton signals from the consistency over motion. As shown in Fig. \ref{fig:pipeline} (c), non-Euclidean consistency exists between the original and transferred person image.

% To learn the implicit posture representation,
%放几个比较明显的failure cases,尽量变成横排;是做一个说明OOD skeletons 在之前appearance做好的case上,他效果变得不好的例子
% \paragraph{Compute the similarity via \textcolor{red}{consistent Identity loss}:}

%% low low的感觉motion consistent loss

\begin{table*}[t]
	\centering
	\vspace{2mm}
    \caption{Evaluation results on Open-World Pose Transfer (OWPT). }
    \vspace{0mm}
    	   \setlength{\tabcolsep}{2mm}
	   {
\begin{tabular}{ccccccccccccc}
\toprule[1pt]
\multicolumn{1}{c}{\multirow{2}{*}{Method}} & \multicolumn{1}{c}{\multirow{2}{*}{OWPT}} & \multicolumn{3}{c}{SHHQ} &  & \multicolumn{3}{c}{DeepFashion 2} &  & \multicolumn{3}{c}{Tiktok} \\ \cline{3-5} \cline{7-9} \cline{11-13} 
\multicolumn{2}{c}{}                        & SSIM~{\color{red}$\uparrow$}   & LPIPS~{\color{red}$\downarrow$}   & FID~{\color{red}$\downarrow$}   &  & SSIM~{\color{red}$\uparrow$}      & LPIPS~{\color{red}$\downarrow$}     & FID~{\color{red}$\downarrow$}     &  & SSIM~{\color{red}$\uparrow$}    & LPIPS~{\color{red}$\downarrow$}   & FID~{\color{red}$\downarrow$}   \\ \hline
\multirow{2}{*}{ADGAN}          & w/o SETA           &0.586        &0.425         &73.04       &  &0.608           &0.444           &75.53         &  &0.657         &  0.291        &77.14       \\
                                & w/ SETA          &\textbf{0.901}         & \textbf{0.079}       &\textbf{35.67}       &  &\textbf{0.885}           & \textbf{0.146}          &\textbf{65.09}         &  &\textbf{0.834}         & \textbf{0.133}         &\textbf{68.35}       \\ \hline
\multirow{2}{*}{CASD}           & w/o SETA          & 0.728        & 0.198         & 36.09      &   &  0.702         &0.268           & 45.09         &  &0.671         & 0.277         & 63.33      \\
                                & w/ SETA           &\textbf{0.933}        &\textbf{0.042}         &\textbf{18.66}       &  & \textbf{0.916}          &\textbf{0.056}          & \textbf{32.15}         &  &\textbf{0.819}         & \textbf{0.109}         & \textbf{55.21}      \\ \hline
\multirow{2}{*}{NTED}           & w/o SETA         & 0.723        & 0.211         &40.25       &  &0.686           & 0.284          &53.457         &  &0.678         &0.259          &72.68        \\
                                & w/ SETA          & \textbf{0.890}       &\textbf{0.049}         &\textbf{19.07}       &  &\textbf{0.859}           & \textbf{0.089}         & \textbf{33.74}        &  &\textbf{0.837}         &\textbf{0.097}          &\textbf{35.56}       \\
\bottomrule[1pt]

\label{tab:quantitative}
\end{tabular}}
\vspace{-8mm}
\end{table*}



\paragraph{Global Consistency over Motion.} We fetch OOD skeleton signals from the consistency over motion. First, we use the model $f_{\hat{\theta}}$ to generate the pose transfer image $\hat{I}^{tar}$ from the reference image $I^{id}$ and target skeleton $P^{tar}$. \emph{A person re-identification model has potential to search the images of same person with different posture}, with the help of that, we obtain the consistency over motion. Specifically, we apply a ReID model~\cite{ReID_model} on the reference person image and the generated image to obtain features for similarity computation~\cite{ReID_loss}:


\begin{equation}
L_{Content}=\sum_{t}\|\phi_t(\hat{I}^{tar})-\phi_t(I^{id})\|_2.
\label{REID}
\end{equation}
where $\phi_t$ represents the $t$-th layer of ReID model~\cite{ReID_model}. 

\vspace{-4mm}

\paragraph{Local Consistency over Motion.} In addition, we also enforce the local correspondence between each body part of the generated image and the reference image. However, the regions of the same body part in different poses usually have different sizes and shapes, which prevents us from computing loss in Euclidean space (\emph{e.g.} SSIM, L2 and perceptual loss). Inspired by \cite{Gram}, we use the Gram matrix to calculate the local similarity loss, which is not restricted to Euclidean space. Thus, we compute the Gram-matrix similarity using ReID's feature within each body-part as:
\begin{equation}
L_{GRAM}=\sum_{q}\|G(\hat{M}_q\odot\phi_t(\hat{I}^{tar}))-G({M}_q\odot\phi_t(I^{id}))\|_2.
% \label{ACL}
\end{equation}
where $M_q$ and $\hat{M}_q$ are the human parsing results of $I^{id}$ and $\hat{I}^{tar}$ estimated by~\cite{Parsing}, $G$ is the Gram matrix, $\odot$ denotes the element-wise multiplication. We use the features from the first layer (\emph{i.e.} $t = 1$) empirically. Thus, we calculate the consistency loss over motion via the global and local correspondence as:
\begin{equation}
% \vspace{-1mm}
L_{COM}=L_{Content} + L_{GRAM}.
\label{motion_consistency}
% \vspace{-1mm}
\end{equation}



% \begin{figure}[t]
% \centering
% \includegraphics[width=0.95\linewidth,height=0.4\linewidth]{fig/shape_misalign.pdf}
% \vspace{2mm}
% \caption{Visual comparison of reference- and generated- images with semantics index.}
% \label{fig:shape_misalign}
% \vspace{-5mm}
% \end{figure}

% 例1





Note that $L_{COM}$ is a function for updating model parameters $\hat{\theta}$, which is used to learn external skeleton signals during pose transformation. In test-time phase, we can optimize $L_{COM}$ over OOD skeleton drawn from a test distribution $P$, which is defined as follows:
% Note that $L_{Ske}$ is a function for updating $\hat{\theta}$, which used to facilitate adaptation to OOD data during pose transformation. Thus, The formulation of adaption process is defined as follows:
\begin{equation}
    % \vspace{-2mm}
    \begin{split}
    \tilde{\theta} = &\mathop{\arg\min}_{\hat{\theta}} \mathbb{E}_P\begin{bmatrix}
        L_{COM}(\hat{I}^{tar}, I^{id}; \hat{\theta})
        \end{bmatrix}.
    \end{split}
\end{equation}
where $\hat{I}^{tar} = f_{\hat{\theta}}(I^{id},P^{tar})$. $P^{tar}$ is OOD skeletons. $\tilde{\theta}$ indicates the updated parameters learned from OOD skeleton signals. $E_P$ is evaluated on an OOD skeleton distribution $P$. 

% \begin{equation}
%     \begin{split}
%     \tilde{\theta} = \mathop{\arg\min}_{\hat{\theta}} \ \  \frac{1}{K}\sum_{v=1}^{K} \| L_{Ske}(f_{\hat{\theta}}(I^{id}, P^{tar}_v), I^{id}) \| 
%     \end{split}
% \end{equation}

% where $\tilde{\theta}$ indicates the updated parameters learned from OOD skeleton signals. $K$ denotes the set of OOD skeletons and $P^{tar}_v$ is the v-th OOD skeletons. 

% \emph{Why $L_{Ske}$ realize a automatic similarity learning between two images with different posture?}
% As $I^{id}$ is deployed into new posture w.r.t $P^{tar}_v$, $L_{Ske}$ still apply an automatic similarity learning by a person re-identification manner with $\phi$.

Ideally, we update the pre-trained parameters $\theta$ via the sequential OOD signals (\emph{i.e.} appearance$\rightarrow$skeleton). The sequential optimization is given as follows:
% \begin{equation}
% \tilde{\theta} \gets \theta - \alpha\nabla_\theta L_{ori}(x_t, x_{aug};\theta).
% \label{gradients_1}
% \end{equation}
% where $\alpha$ is the adaptation learning rate. Here $\tilde{\theta}$ can be seen as the model parameters adapted to the input $x_t$ via \textbf{SSLA}. 
\begin{equation}
    % \vspace{-1mm}
    \begin{split}
    \hat{\theta} \gets \theta-\alpha\nabla_\theta L_{Appe}(\hat{I}^{id}, \hat{I}^{aug}, I^{id}, I^{aug}; \theta)
    \end{split}
\end{equation}
\begin{equation}
    \begin{split}
    \tilde{\theta} \gets \hat{\theta}-\beta{\nabla_{\hat{\theta}}}L_{COM}(\hat{I}^{tar},I^{id};\hat{\theta})
    \end{split}
\end{equation}
where $\alpha$ and $\beta$ are the adaptation learning rates. We present a summarization of SETA in Algorithm~\ref{alg:Framwork}.
% \begin{equation}
% \hat{\theta} \gets \tilde{\theta} - \alpha\nabla_{\tilde{\theta}} L_{cons}(x_t, \hat{x}_t;\tilde{\theta}).
% \label{gradients_2}
% \end{equation}



% \paragraph{}
% The general pipelines of pose transfer are divided into two steps. First, extracting appearance texture from reference person. Then, distributing them to the target skeleton. In the Test-Time Adaption stage, we have three options to update the model.




% \begin{figure*}[ht]
% 	\begin{center}
% 		\includegraphics[width=0.95\linewidth]{fig/SSL_aux_loss.pdf}
% 	\end{center}
% 	%\vspace{-3mm}
% 	\caption{.}
% 	\label{fig:network}
% 	\vspace{-3mm}
% \end{figure*}
\renewcommand{\algorithmicrequire}{\textbf{Input:}}  % Use Input in the format of Algorithm
\renewcommand{\algorithmicensure}{\textbf{Output:}} % Use Output in the format of Algorithm



\begin{figure*}[t]
	\begin{center}
		\includegraphics[width=0.95\linewidth]{fig/quanlitation.pdf}
	\end{center}
 	\vspace{-5mm}
	\caption{Qualitative comparison of using our proposed method on different datasets. Pose transfer frameworks generate more realistic on ODD references with SETA.}
	\label{fig:quanlitation}
	\vspace{-6mm}
\end{figure*}



\section{Experiments}

In this section, we describe our experimental setups and evaluate our proposed methods on benchmark datasets with various pre-trained models. We also apply our methods to other human generation tasks, such as celebrity motion synthesis and skeleton-driven tiktok reenactment, to show the potential extensibility of our approaches.
% OOD Scene

\begin{algorithm}[t]
\caption{SETA algorithm.}
\label{alg:Framwork}
\begin{algorithmic}[1] %这个1 表示每一行都显示数字
    \REQUIRE $\alpha$, $\beta$:  learning rates\\
    \REQUIRE $I^{id}$, $P^{id}$: OOD person image and skeleton \\
    %\REQUIRE $I^{aug}_i$, $P^{aug}_i$:  augmented person image and skeletons\\
    \REQUIRE $P^{tar}_v$: target skeleton \\
    \REQUIRE  $\theta$: pre-trained model
    
    \STATE Sample an batch of OOD data in $\{I^{id}, P^{tar}_v\}^K_{v=1}$;
    \STATE Augment $\left \{ I^{id},P^{id} \right \}$ to generate set $\left \{ I^{aug},P^{aug} \right \}$;
    \STATE Compute $L_{Appe}(I^{id},P^{id},I^{aug},P^{aug})$;
    \FOR { $\left \{ I^{aug}_i,P^{aug}_i \right \}$ in $ \left \{ I^{aug},P^{aug} \right \}$}
    \STATE Generate fake images:
    \STATE $\hat{I}^{id} = f_{\theta}(I^{aug}_i,P^{id})$ and $\hat{I}^{aug}_i = f_{\theta}(I^{id},P^{aug}_i)$;
    \STATE Update parameters with gradient descent:\\ 
    $\hat{\theta} \gets \theta-\alpha\nabla_\theta L_{Appe}(\hat{I}^{id}, \hat{I}^{aug}_i, I^{id}, I^{aug}_i; \theta)$;
    \ENDFOR
    \WHILE {$v \le  K$}
        \STATE Generate fake images: $\hat{I}^{tar}_v = f_{\hat{\theta}}(I^{id},P^{tar}_v)$;
        \STATE Compute $L_{COM}(\hat{I}^{tar}_v, I^{id})$;
        %\FOR { $P^{tar}_v$}
        \STATE Update parameters with gradient descent:\\ $\tilde{\theta} \gets \hat{\theta}-\beta{\nabla_{\hat{\theta}}}L_{COM}(\hat{I}^{tar}_v,I^{id};\hat{\theta})$;
       % \ENDFOR
    
    \ENDWHILE
    \ENSURE ~~\\ %算法的输出:Output
        Updated model parameter $\tilde{\theta}$.
\end{algorithmic}
\end{algorithm}

\subsection{Implementation Details}

In our experiments, the pre-trained models from NTED, CASD and ADGAN are trained on the In-shop Clothes Retrieval Benchmark of the DeepFashion dataset~\cite{liu2016deepfashion} (\emph{e.g.} images of the same person are paired). During test-time adaption stage, the rotation angle is set to \{20,10,5,-5,-10,-20\}, scale degrees is set to [-0.2,0.2], and the mask ratio is set to 50\%. For OOD appearance adaptation step, we perform 30 training iterations. Then, we perform 5 iterations with skeleton domain adapter with OOD skeletons. Both two stages use the initial learning rate as $2 \times 10^{-3}$ of all networks except the NTED~\cite{Pose_one_4}, which is set to be $1 \times 10^{-3}$. The Adam solver is used for both adaption stage with hyper-parameter $\beta_1=0.5$, and $\beta_2=0.99$. Note that $\beta_1=0$ in NTED. All the experiments are conducted on Nvidia V100 GPUs.

% \begin{figure*}[t]
% 	\begin{center}
% 		\includegraphics[width=0.95\linewidth]{fig/NTED_ablation.pdf}
% 	\end{center}
% 	%\vspace{-3mm}
% 	\caption{Ablation study of data augmentation in Open-Appearance.}
% 	\label{fig:NTED_ablation}
% 	\vspace{-3mm}
% \end{figure*}





\subsection{Evaluation Datasets and Metrics}
% \vspace{-3mm}
Under the Open-World Pose Transfer (OWPT) setting, we employ SHHQ~\cite{SHHQ}, DeepFashion 2~\cite{DeepFashion2} and Tiktok~\cite{Tiktok} datasets for evaluation. SHHQ~\cite{SHHQ} is currently the largest dataset of human whole body, which consists of various appearance and poses styles person images. DeepFashion 2~\cite{DeepFashion2} contains lots of Asians in comprehensive fashion outfits. Tiktok~\cite{Tiktok} dataset consists of dance videos that capture a single person performing dance moves.  We use 30 videos from Tiktok dataset, 2536 images from SHHQ, and 1557 images from DeepFashion 2 as the evaluation datasets. For all images, we process them into image resolution of 256~$\times$~176. We adopt SSIM\cite{SSIM}, LPIPS\cite{Lpips} and FID\cite{FID} (Frechet Inception Distance) as the evaluation metrics.%Since each large-scale dataset comprises tens of thousands of images, to simplify the test-time adaption process, we select part of the images from each dataset. We select 30 videos from Tiktok dataset, 2536 images from SHHQ and 1557 images from DeepFashion 2 as the evaluation datasets. We will provide the list of test images on our github for the community to reproduce. For all images, we process them into image resolution of 256~$\times$~176. We adopt SSIM\cite{SSIM}, LPIPS\cite{Lpips} and FID\cite{FID} (Frechet Inception Distance) as the evaluation metrics.



\subsection{Comparisons}
% \href{https://github.com/conferenceAnonymous/TTA-pose-transfer}{

\paragraph{Quantitative Comparison.} Under the OWPT setting, the performance of baseline approaches with our proposed sequential test-time adaption is reported under the term `w/ SETA'. As shown in Table~\ref{tab:quantitative}, our methods consistently outperform the existing approaches. The quantitative results on SSIM and LPIPS demonstrate that our approach obtains better image similarity toward OOD data. In addition, our approaches outperform the baseline methods with lower FID scores, which indicate better-quality images are shown by SETA.
% Please add the following required packages to your document preamble:
% \usepackage{multirow}

\begin{figure*}[t]
\centering
\includegraphics[width=0.95\linewidth,height=0.28\linewidth]{fig/video.pdf}
% \vspace{-2mm}
\caption{Skeleton-driven results on the Tiktok dataset. Given target skeletons, our method can generate realistic dance sequences.}
\label{fig:video}
\vspace{-3mm}
\end{figure*}

\begin{figure*}[t]
\centering
\includegraphics[width=0.95\linewidth,height=0.23\linewidth]{fig/celebrity.pdf}
\vspace{-2mm}
\caption{Examples of celebrity motion synthesis. Our algorithm transfer Morgan Freeman and Elon Musk into desired postures.}
\label{fig:celebrity}
\vspace{-5mm}
\end{figure*}

\vspace{-4mm}
\paragraph{Qualitative Comparison.} To further validate the proposed SEquntial Test-Time Adaption method under OWPT setting, we perform visual comparison of our method with recent proposed pose transfer methods in Fig.~\ref{fig:quanlitation}, including ADGAN~\cite{Pose_one_2}, CASD~\cite{Pose_one_3} and NTED~\cite{Pose_one_4}. As shown in the first and second rows of Fig.~\ref{fig:quanlitation}, when the appearance texture and body shape of reference person are different from the source dataset, these attributes of the generated results do not match to each other. Therefore, pre-trained models fail to preserve appearance texture, which demonstrate the challenge from OOD data. In the last row of Fig.~\ref{fig:quanlitation}, since the race appearance of Deepfashion 2 is different from the source dataset, images generated by baseline models fail to keep consistent appearances. This reveals pre-trained models could not be applied to the unknown data as they are limited by the prior knowledge inherit from source dataset.

In comparison, the generated results `w/ SETA' demonstrate the effectiveness of SEquential Test-Time Adaption under OWPT setting. As shown in Fig.~\ref{fig:quanlitation}, the adapted model has learned the OOD appearance signals, which could preserve the gender and clothes texture well. Benefited from the skeleton adapter, the updated model is able to distribute the texture to the target skeleton reasonably. Since we do not need additional pair labels of the input image during TTA stage, SETA can adaptively extend to various single human images of OOD data.



% \subsection{Open-World Applications}

% In this section, we verify the practicality of the proposed methods on two popular applications.

% \emph{Skeleton-driven Tiktok Reenactment} aims to generate single-person dance videos by a reference person image and sequence dance skeletons. We conduct experiment on the Tiktok dataset~\cite{Tiktok} and 30 videos are used for testing. Example reenactments generated by SETA are shown in Fig.~\ref{fig:video}, which synthesis delightful visual quality even with complex poses input.

% % #\paragraph{Open-World Appearance from Celebrity}
% \emph{Open-World Celebrity Motion Synthesis} aims to generate arbitrary poses with a celebrity image. We apply SETA to learn the celebrity's appearance and generate different pose views. We conduct experiments on the Morgan Freeman and Elon Reeve Musk. As demonstrated in Fig.~\ref{fig:celebrity}, our method is able to synthesize realistic results for celebrity motion application.

\vspace{-3mm}
\subsection{Skeleton-driven Tiktok Reenactment}
\vspace{-2mm} In this subsection, we show that our model can generate coherent single-person dance videos with delightful visual performance. We first extract appearance signals and skeleton signals from sequential motions with various poses of source video by our SETA. Then, we transform a reference person image into sequence dance skeletons. We conduct experiment on the Tiktok dataset~\cite{Tiktok} and 30 videos are used for testing. As shown in the second row of Fig.~\ref{fig:quanlitation}, state-of-the-art methods~\cite{Pose_one_4, Pose_one_3, Pose_one_2} fail to keep the source identity, leading inability to generate the realism of the produced videos. In contrast, combined with SETA, the pose transfer model~\cite{Pose_one_4, Pose_one_3, Pose_one_2} can generate realistic results along with accurate movements while still preserving the source identity. Qualitative reenactments results generated by SETA are shown in Fig.~\ref{fig:video}, which synthesis delightful visual quality even with complex poses input. We provide more skeleton-driven tiktok~\cite{Tiktok} driven reenactment samples in Fig.~\ref{fig:video_comp}.

\subsection{Open-World Celebrity Motion Synthesis} In this subsection, we aim to generate high-resolution celebrity image with arbitrary poses. We apply SETA to learn the celebrity's appearance and generate different pose views with $512 \times 352$ resolution. Experiments conducted on Morgan Freeman and Elon Reeve Musk are provided in Fig.~\ref{fig:celebrity}. It can be seen that our SETA can generate realistic results with well-preserved source identity details, such as clothes, human body parts and facial expression. To this end, SETA is able to synthesize high-resolution realistic results for celebrity motion application. More celebrity motion synthesis samples are shown in Fig.~\ref{fig:celebrity_1} and Fig.~\ref{fig:celebrity_2}.




% \begin{figure}[t]
% 	\centering
% 		\includegraphics[width=0.95\linewidth,height=0.7\linewidth]{fig/SSL_v2.pdf}
%  	\vspace{-3mm}
% 	\caption{Visual comparison of pre-trained model (\emph{i.e.} NTED~\cite{Pose_one_4}) with and without TTA on the appearance domain.}
% 	\label{fig:comp_SSL_init}
%  	\vspace{-6mm}
% \end{figure}

\subsection{Analysis and Discussion}
We perform ablation studies to further investigate various aspects of the proposed approach. For evaluation of visual performance, we recruited 25 volunteers to collect human feedback on the synthetic results as MOS (Mean Opinion Score). Specially, 300 pairs from SHHQ dataset are randomly selected. Each volunteer is asked to select the generated result with the best visual performance from each group of images.
% All experiments are based on a baseline model ( \emph{i.e.} NTED~\cite{Pose_one_4}). Note that we also provide more ablation study on ADGAN and CASD in supplementary materials.

% \begin{figure}[t]
% \centering
% \includegraphics[width=0.95\linewidth]{fig/domain.pdf}
% \vspace{-2mm}
% \caption{We use two different TTA orders for domain transition. }
% \label{fig:domain}
% \vspace{-5mm}
% \end{figure}

% Since we first propose SEquential Test-Time Adaption for pose transfer, we study the sequential relationship between two domains. (1) Could we fetch both the OOD appearance signals and the OOD skeleton signals in the test-time adaption stage? (2)since we do not have the the pair label of OOD data, updating skeleton domain first with the incorrect appearance texture will lead to catastrophic forgetting, which is  demonstrated in Fig.~\ref{fig:domain}. Inspired by the general pipelines of pose transfer, which first extract the appearance texture and then distribute them according the semantic distribution of target skeletons. We first update the appearance encoder to fit the appearance domain of OOD data. Then we enhance the model to delicately distribute the texture into the related semantic of target skeletons. The entire training procedure is elaborated in Algorithm.~\ref{alg:Framwork} and Fig.~\ref{fig:pipeline}. In this way, we not only makes the connection between the two domains more natural, but also avoids catastrophic forgetting caused by unstable adaption.



% ok

%\paragraph{Analysis of Data Augmentation in Appearance Adapter:} To validate the effectiveness of Data Augmentation (DA) in Open-Appearance, we design a baseline method (\emph{i.e.} `w/o DA') for verification. As demonstrated in Table~\ref{tab:NTED_ablation}, `w/ DA' continuously produce better results.
% \begin{table}[t]
% 	\centering
% 	  \caption{Ablation studies of two adapters. Lower FID indicates better results. (a) denotes NTED + Appearance Adapter. (b) denotes NTED + Appearance Adapter + Skeleton Adapter.}
% 	   %   \vspace{-1mm}
%                 \begin{tabular}{cccc}
%                 \hline
%                 Methods & SHHQ  & DeepFashion 2 & Tiktok \\ \hline
%                 NTED    & 40.25 & 53.45 & 72.68  \\ \hline
%                 (a)     & 19.57 & 34.37 & 37.83  \\ \hline
%                 (b)     & \textbf{19.07} & \textbf{33.74} & \textbf{35.56}  \\ \hline
%                 \end{tabular}
% \label{tab:ablation_study_detail}
% % \vspace{-7mm}
% \end{table}
% \paragraph{Ablations of Adapters:} We show the ablation studies on the effects of two adapters. They are presented as follows: (a) NTED + Appearance Adapter: where the appearance adapter is used to extract the appearance texture on OOD data; (b) NTED + Appearance Adapter + Skeleton Adapter: where the skeleton adapter is used to distribute the texture according to the target skeleton; As depicted in Table~\ref{tab:ablation_study_detail}, combined with skeleton adapter, the proposed method achieves 0.5 FID gains in SHHQ~\cite{SHHQ}, 0.63 FID gains in DeepFashion 2~\cite{DeepFashion2} and 2.27 FID gains in Tiktok~\cite{Tiktok}.
% \paragraph{Ablations of Skeleton Adapters.} i) quantitative ii) qualitative iii) human feedback



% \paragraph{Effectiveness of Appearance Adapters.} 
% The proposed appearance adapter is used to adapt to OOD appearance smoothly. (1) As illustrated in second column of Fig.~\ref{fig:comp_appe_skl}, the results with appearance adapter perform reasonably well in maintaining the appearance texture of OOD data, which is help to train skeleton adapter stably. (2) As depicted in Table~\ref{tab:ablation_study_detail}, joint training with skeleton adapter could achieve 0.049 LPIPS and 19.07 FID for NTED~\cite{Pose_one_4} on SHHQ~\cite{SHHQ}. Similarly, it achieves 0.042 LPIPS and 18.66 FID for CASD~\cite{Pose_one_3}.

\begin{table}[t]
	\centering
	\scriptsize
	  \caption{Ablation study of SETA. Lower LPIPS indicates better results. Higher MOS indicates that humans prefer.}
 	  \vspace{2mm}
	   %   \vspace{0mm}
	   	   \setlength{\tabcolsep}{0.3mm}\renewcommand{\arraystretch}{1.5}
	   {
\begin{tabular}{cccccc}
\toprule[1pt]
\multirow{2}{*}{\footnotesize Methods}                & \multicolumn{2}{c}{\footnotesize NTED} &  & \multicolumn{2}{c}{\footnotesize CASD} \\ \cline{2-3} \cline{5-6} 
                                        &\footnotesize LPIPS       & \footnotesize MOS       &  & \footnotesize LPIPS            & \footnotesize  MOS          \\ \hline
 \footnotesize Pre-trained Model                    & \footnotesize 0.211       & \footnotesize 3.34\%      &  & \footnotesize 0.198           & \footnotesize 4.41\%          \\ \hline
  \footnotesize w/ Appearance Adapter            & \footnotesize 0.084       & \footnotesize 20.53\%      &  & \footnotesize 0.079           & \footnotesize 23.43\%          \\ \hline
 \footnotesize w/ Appearance Adapter + Skeleton Adapter  & \footnotesize 0.049       & \footnotesize 76.13\%      &  & \footnotesize 0.042          & \footnotesize 72.17\%          \\ 
\bottomrule[1pt]
\end{tabular}}
\label{tab:ablation_study_detail}
 \vspace{-7mm}
\end{table}


\begin{figure*}[t]
\centering
\includegraphics[width=1\linewidth,height=0.30\linewidth]{fig/SSL_AUX_feature.pdf}
\vspace{-7mm}
\caption{Qualitative effects of skeleton adapter during pose transformation. Given a reference person image and a target skeleton, we show the feature representations and visualization results of the skeleton adapter. $\hat{\theta}$ denotes the parameters of NTED~\cite{Pose_one_4} has been updated by OOD appearance signals.}%The visual performance demonstrates that the motion correspondence modeling better with skeleton adapter.}
\label{fig:comp_appe_skl}
\vspace{-7mm}
\end{figure*}
\vspace{-4mm}
\paragraph{Effectiveness of Appearance Adapters.} 
The proposed appearance adapter is mainly used to provide a stable appearance transition. (1) Table~\ref{tab:ablation_study_detail} shows quantitative evaluations on SHHQ~\cite{SHHQ} dataset. Appearance adapter could achieve 0.127 LPIPS gains and 20.53\% MOS for NTED~\cite{Pose_one_4}. Similarly, it achieves 0.119 LPIPS gains and 23.43\% MOS for CASD~\cite{Pose_one_3}. The quantitative results shows that using this module is helpful to obtains better realistic results. (2) As illustrated in the second and fifth columns of Fig.~\ref{fig:comp_appe_skl}, the results with appearance adapter perform reasonably well in maintaining the global appearance texture of various OOD data, which is help to train skeleton adapter stably.

\vspace{-3mm}
\paragraph{Effectiveness of Skeleton Adapter.} 
To verify the skeleton adapter comprehensively, we perform it to learn implicit posture representation and make the analyses from \emph{quantitation, human feedback, and visual quality}: (1) As depicted in Table~\ref{tab:ablation_study_detail}, joint training with skeleton adapter can further improve LPIPS performance by 41.66\% for NTED and 46.84\% for CASD. We also collect human feed back for skeleton adapters. The synthesis results achieves nearly 76.13\% MOS for NTED and 72.17\% for CASD. It suggests that skeleton adapter can keep the most useful features and maintain a stable feature-warping process in pose transformation stage. (2) As shown in the first row of Fig.~\ref{fig:comp_appe_skl}, `w/o skeleton adapter ($\hat{\theta}$)' mistakenly ignores some noticeable features, leading to attribute missing and distorted embroideries. In the second row, images generated by `w/o skeleton adapter ($\hat{\theta}$)' exist the clear artifacts, which are caused by erroneous feature warping in pose transition. Generally, only fetched appearance signals from augmented derivatives, the pre-trained model cannot handle highly non-rigid deformation when encounter various OOD skeleton. Therefore, we combine the pre-trained model with skeleton adapter to keep motion consistency. As shown in Fig.~\ref{fig:comp_appe_skl}, `w/ skeleton adapter ($\hat{\theta}$)' could maintain the human features in various OOD skeleton, which is well impressive as none of pose transfer methods have done it before.

\begin{table}[t]
	\centering
	\scriptsize
	  \caption{Analysis for the number of updating iterations in test-time training.}
 	  \vspace{2mm}
	   %   \vspace{0mm}
	   
	   %\resizebox{\textwidth}{5mm}{
	   \setlength{\tabcolsep}{1.3mm}\renewcommand{\arraystretch}{1.5}
	   {
	   
\begin{tabular}{ccccccc}
\toprule[1pt]
\multirow{2}{*}{\begin{tabular}[c]{@{}c@{}}\small Test-time\\ \small Training\end{tabular}} & \multicolumn{2}{c}{\small NTED} & \multicolumn{2}{c}{\small CASD} & \multicolumn{2}{c}{\small ADGAN} \\ \cline{2-7} 
                                                                           & \small LPIPS        & \small Time     & \small LPIPS        & \small Time     & \small LPIPS        &\small Time      \\ \hline
\small w/o SETA                                                                  & \small 0.211      & \small 0.68$s$        & \small 0.198      & \small 1.12$s$       & \small 0.425      & \small 1.36$s$         \\ \hline
\small 5 iterations                                                                 & \small 0.105      & \small 10.87$s$       &  \small 0.096          &  \small 24.8$s$           & \small 0.163      & \small 26.5$s$         \\ \hline
\small 10 iterations                                                                 & \small 0.077      & \small 14.21$s$       & \small 0.073     & \small 30.95$s$       & \small 0.115      & \small 33.08$s$        \\ \hline
\small 30 iterations                                                                 & \small 0.049      & \small 28.48$s$       & \small 0.042           & \small  60.46$s$        & \small 0.079      & \small 61.87$s$      \\ 
\bottomrule[1pt]
\end{tabular}}
        % }
\label{tab:time_ablation}
\vspace{-7mm}
\end{table}

% \paragraph{Effectiveness of Skeleton Adapter.} 
% To verify the skeleton adapter quantitatively, we perform it to learn implicit posture representation and make the analyses from \emph{quantitation, visual quality and human feedback}: (1) As depicted in Table~\ref{tab:ablation_study_detail}, `w/ Skeleton Adapter' achieves 0.065 LPIPS for NTED~\cite{Pose_one_4} and 0.063 LPIPS for . And joint training with appearance adapter can further improve LPIPS performance by 27.94\% for NTED and 28.81\% for CASD. It suggests that skeleton adapter can keep the most useful features and maintain a stable feature-warping process in pose transformation stage. (2) As shown in the first row of Fig.~\ref{fig:comp_appe_skl}, `w/o skeleton adapter ($\hat{\theta}$)' mistakenly ignores some noticeable features, leading to attribute missing and distorted embroideries. In the second row, images generated by `w/o skeleton adapter ($\hat{\theta}$)' exist the clear artifacts, which are caused by erroneous feature warping in pose transition. Actually, only fetched appearance signals from augmented derivatives, the pre-trained model cannot handle highly non-rigid deformation when encounter various OOD skeleton. Therefore, we combine the pre-trained model with skeleton adapter to keep motion consistency. As shown in Fig.~\ref{fig:comp_appe_skl}, `w/ skeleton adapter ($\hat{\theta}$)' could maintain the human features in various OOD skeleton, which is well impressive as none of pose transfer methods have done it before.
% (3) We further collect human feed back for both adapters. And the synthesis results with both adapters achieves nearly 58.25\% MOS for NTED and 57.12\% for CASD, which demonstrates that the skeleton adapter can improve the visual performance in pose transition stage.
% (3) We further collect human feed back of skeleton adapter by recruiting 25 volunteers. And the synthesis results with skeleton adapter achieves nearly 69.38\% support rate, {\color{blue} while xxx achieves xx }. 

% Since we achieve satisfactory visual results and better metrics (\emph{i.e.} LPIPS and FID), the effectiveness of skeleton adapter is verified.

% \paragraph{Effectiveness of Skeleton Adapter.} 
% To verify the skeleton adapter quantitatively, we perform it to learn implicit posture representation and make the analyses from \emph{quantitation, visual quality and human feedback}: (1) As depicted in Table~\ref{tab:ablation_study_detail}, `w/ Skeleton Adapter' achieves nearly 70.6 \% LPIPS and 48.1 \% FID gains for NTED~\cite{Pose_one_4} on SHHQ datasets. Similarly, it achieves (2) As shown in the first row of Fig.~\ref{fig:comp_appe_skl}, `w/o skeleton adapter ($\hat{\theta}$)' mistakenly ignores some noticeable features, leading to attribute missing and distorted embroideries. In the second row, images generated by `w/o skeleton adapter ($\hat{\theta}$)' exist the clear artifacts, which are caused by erroneous feature warping in pose transition. Actually, only fetched appearance signals from augmented derivatives, the pre-trained model cannot handle highly non-rigid deformation when encounter various OOD skeleton. Therefore, we combine the pre-trained model with skeleton adapter to keep motion consistency. As shown in Fig.~\ref{fig:comp_appe_skl}, `w/ skeleton adapter ($\hat{\theta}$)' could maintain the human features in various OOD skeleton, which is well impressive as none of pose transfer methods have done it before. (3) We further collect human feed back of skeleton adapter by recruiting 25 volunteers. And the synthesis results with skeleton adapter achieves nearly 69.38\% support rate, {\color{blue} while xxx achieves xx }. 
% Since we achieve satisfactory visual results and better metrics (\emph{i.e.} LPIPS and FID), the effectiveness of skeleton adapter is verified.


% 'w/o skeleton adapter ($\hat{\theta}$)' produces OOD appearance according to target skeleton, the model itself still mistakenly ignores some noticeable features and raises unreasonable warping.

% without training in a stable appearance transitions is 22.59 and 0.091, with stable transfer is 20.87 and 0.062, which achieves 1.72 FID and 0.029 LPIPS gains. Since only using the skeleton adapter can also achieve nearly 56 \% LPIPS and 190 \% FID gains, the effectiveness of the skeleton adapter is verified.

% As discussed above, skeleton adapter struggles to fetch finetune signals in unstable appearance transitions stage.
% In comparision with FID and LPIPS on the results predicted by As depicted in Table~\ref{tab:ablation_study_detail}, combined with skeleton adapter, pose transfer models can adapt generate high percepetual. Specifically, appearance adapter achieves 0.5 and 0.63 FID gains for NTED~\cite{Pose_one_4} on two datasets. Similarly, it achieves 0.46 and 0.77 FID gains for CASD~\cite{Pose_one_3}. With appearance adapter and skeleton adapter, NTED and CASD obtain a two-fold increase with FID index on SHHQ.


% \begin{table}[t]
% 	\centering
% 	\scriptsize
% 	  \caption{Ablation study of SETA. Lower FID indicates better results. }
%  	  \vspace{-2mm}
% 	   %   \vspace{0mm}
% 	   	   \setlength{\tabcolsep}{0.3mm}\renewcommand{\arraystretch}{1.5}
% 	   {
% \begin{tabular}{cccccc}
% \toprule[1pt]
% \multirow{2}{*}{\footnotesize Methods}                & \multicolumn{2}{c}{\footnotesize SHHQ} &  & \multicolumn{2}{c}{\footnotesize DeepFashion2} \\ \cline{2-3} \cline{5-6} 
%                                         &\footnotesize  NTED        & \footnotesize CASD       &  & \footnotesize NTED            & \footnotesize CASD           \\ \hline
%  \footnotesize Pre-trained model                       & \footnotesize 40.25       & \footnotesize 34.09      &  & \footnotesize 53.45           & \footnotesize 45.09          \\ \hline
%  \footnotesize w/ Appearance Adapter            & \footnotesize 19.57       & \footnotesize 17.67      &  & \footnotesize 34.37           & \footnotesize 32.92          \\ \hline
%   \footnotesize w/ Skeleton Adapter            & \footnotesize 22.59       & \footnotesize xxxx      &  & \footnotesize xxxx           & \footnotesize xxxx          \\ \hline
%  \footnotesize w/ Appearance Adapter + Skeleton Adapter & \footnotesize 19.07       & \footnotesize 17.21      &  & \footnotesize 33.74           & \footnotesize 32.15          \\ 
% \bottomrule[1pt]
% \end{tabular}}
% \label{tab:ablation_study_detail}
%  \vspace{-1mm}
% \end{table}


% \begin{table}[t]
% 	\centering
% 	\scriptsize
% 	  \caption{Ablation study of SETA. Lower FID and LPIPS indicates better results. }
%  	  \vspace{2mm}
% 	   %   \vspace{0mm}
% 	   	   \setlength{\tabcolsep}{0.3mm}\renewcommand{\arraystretch}{1.5}
% 	   {
% \begin{tabular}{cccccc}
% \toprule[1pt]
% \multirow{2}{*}{\footnotesize Methods}                & \multicolumn{2}{c}{\footnotesize NTED} &  & \multicolumn{2}{c}{\footnotesize CASD} \\ \cline{2-3} \cline{5-6} 
%                                         &\footnotesize LPIPS       & \footnotesize MOS       &  & \footnotesize LPIPS            & \footnotesize  MOS          \\ \hline
%  \footnotesize Pre-trained Model                    & \footnotesize 0.211       & \footnotesize 1.31\%      &  & \footnotesize 0.198           & \footnotesize 2.41\%          \\ \hline
%   \footnotesize w/ Appearance Adapter            & \footnotesize 0.068       & \footnotesize 19.54\%      &  & \footnotesize 0.059           & \footnotesize 21.42\%          \\ \hline
 
%   \footnotesize w/ Skeleton Adapter            & \footnotesize 0.065       & \footnotesize 20.89\%      &  & \footnotesize 0.063           & \footnotesize 19.05\%          \\ \hline
%  \footnotesize w/ Appearance Adapter + Skeleton Adapter  & \footnotesize 0.049       & \footnotesize 58.25\%      &  & \footnotesize 0.042          & \footnotesize 57.12\%          \\ 
% \bottomrule[1pt]
% \end{tabular}}
% \label{tab:ablation_study_detail}
%  \vspace{-4mm}
% \end{table}



\vspace{-4mm}
\paragraph{Discussion of Domain Succession.}
Since we first use two domain information in TTA, we study the sequential relationship between two domains. (1) \emph{Could we fetch both OOD appearance and skeleton signals simultaneously to update the pre-trained model?} Since the pair-wise label of OOD appearance and skeleton are absent, we could not obtain the disentangled signals synchronously. (2) \emph{Could we fetch OOD skeleton signals at first?} Because the pre-trained model could not handle OOD appearance beyond the DeepFashion dataset, it is hard to directly realize the motion consistency between the input image and the generated images. Thus, we recommend to use skeleton adapter to learn implicit posture representation from augmented derivatives at first step, which is beneficial to provide a stable pose transition for learning skeleton signals. (3) \emph{Could we fetch OOD appearance signals at first time?} Inspired by the typical pipelines pose transfer, which first extract the appearance texture and then distribute them to the target skeletons. We first learn OOD appearance signals from OOD data, which exhibits a strong adaptability for the appearance domain. Then, we generate the pose transfer image with OOD appearance, and fetch OOD skeleton signals by using the consistency over motion.  By this means, we not only make the connection between the two domains naturally, but also avoid catastrophic forgetting caused by unstable adaption.




\vspace{-3mm}
\paragraph{Analysis of Test-time Updating Iterations.}

To strike a trade-off between performance and update speed, we conduct various experiments on SHHQ for update iterations in the appearance adaption stage for SETA. The quantitative results of the update iterations are presented in Table~\ref{tab:time_ablation}. It can be observed that larger update iterations yield better generated results, but also bring more time consumption on adaption stage.

\vspace{-2mm}

\section{Conclusion and Limitation}

In this paper, we first extend the pose transfer task to the open-world environment. Specifically, we propose SEquential Test-time Adaption (SETA) to learn non-trivial signals in open-world condition. Extensive evaluations clearly verify the effectiveness of SETA over the state-of-the-art methods with more similar identity, less twisty pattern and greater generalization ability.

Though SETA has learned the non-trivial signals in open-world condition, some detail textures (\emph{e.g.} cloth, makeup) still contain artifacts. In following work, we will apply generative model with ultra high-resolution (UHR) images in our algorithm for a further experiment.

% \paragraph{Open-World Skeleton from TikTok}



%%%%%%%%% REFERENCES
{\small
\normalem
\bibliographystyle{ieee_fullname}
\bibliography{egbib}
}

%\newpage
\appendix

%%%%%%%%% TITLE - PLEASE UPDATE
\newpage
% \paragraph{{\large Appendix}}

% \begin{itemize}
% \item In Section~\ref{sec:Visual_tiktok}, we provide more skeleton-driven tiktok~\cite{Tiktok} driven reenactment samples.
% \item In Section~\ref{sec:Celebrity_Motion}, we provide more celebrity motion synthesis samples.

% \end{itemize}


% \section{Skeleton-driven Tiktok Reenactment}
% \label{sec:Visual_tiktok}
% Reenactments generated by SETA are shown in Fig.~\ref{fig:video_comp}, which synthesis delightful visual quality even with complex poses input.


\begin{figure*}[h]
	\begin{center}
		\includegraphics[width=1\linewidth]{fig/tiktok_comp.pdf}
	\end{center}
	\caption{Skeleton-driven results on the Tiktok dataset~\cite{Tiktok}. Given target skeletons, our method can generate realistic dance sequences.}
	\label{fig:video_comp}
\end{figure*}


% \section{Open-World Celebrity Motion Synthesis}
% \label{sec:Celebrity_Motion}
% We conduct experiments on Stephen Curry,  Kobe Bean Bryant, Elon Reeve Musk and Morgan Freeman. As demonstrated in Fig.~\ref{fig:celebrity_1} and Fig.~\ref{fig:celebrity_2}, combined with NTED~\cite{Pose_one_4}, our method is able to synthesize realistic results for celebrity motion application. Note that, the Fig. 9 in the main paper is generated by using image matting~\cite{paddleseg2019} and image harmonization~\cite{Harmonization}.

\begin{figure*}[t]
	\begin{center}
		\includegraphics[width=1\linewidth]{fig/celebrity_1.pdf}
	\end{center}
    \caption{Examples of celebrity motion synthesis. Our algorithm transfer Stephen Curry,  Kobe Bean Bryant, Elon Reeve Musk and Morgan Freeman into desired postures.}
	\label{fig:celebrity_1}
\end{figure*}
    %\end{titlepage}
\newpage
\begin{figure*}[t]
	\begin{center}
		\includegraphics[width=1\linewidth]{fig/celebrity_2.pdf}
	\end{center}
    \caption{Examples of celebrity motion synthesis. Our algorithm transfer Stephen Curry,  Kobe Bean Bryant, Elon Reeve Musk and Morgan Freeman into desired postures.}
	\label{fig:celebrity_2}
\end{figure*}



\end{document}







