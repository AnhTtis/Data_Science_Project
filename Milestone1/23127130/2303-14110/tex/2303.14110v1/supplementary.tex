\documentclass[11pt,a4paper]{article}
\usepackage[left=1.5cm,right=1.5cm,top=1cm,bottom=1.5cm]{geometry}
\usepackage{times}
\pagestyle{plain}
\usepackage{graphicx}
\usepackage{makecell}
\usepackage{float}
\graphicspath{./}


\usepackage{amsmath}
\usepackage{amssymb}
\usepackage{color}
\usepackage{tabularx} 
\usepackage{comment}
\usepackage{multirow}
\usepackage{multicol}
\usepackage[flushleft]{threeparttable}


%%%%%%%%%%%%%%%%%%%%%%%%%%%%%% User specified LaTeX commands.
\renewcommand{\arraystretch}{1.2}
\renewcommand{\thetable}{S\arabic{table}}  
\renewcommand{\thefigure}{S\arabic{figure}}
\renewcommand{\thesection}{S\arabic{section}}
\definecolor{lr}{rgb}{1.0,0.3,0.3}
\definecolor{dg}{rgb}{0.0,0.5,0.0}
\makeatletter
\makeatother


\title{Supplementary Information \\ {\normalsize for the paper entitled} \\ "Symmetric carbon tetramers forming chemically stable spin qubits in hBN"}

\author{Zsolt Benedek$^{1,2,3}$, Rohit Babar$^{1,2}$, \'{A}d\'{a}m Ganyecz$^{1,2}$, Tibor Szilv\'{a}si$^{3}$, \"Ors Legeza$^1$, Gergely Barcza$^{1,2,3,*}$,\\ Viktor Iv\'{a}dy$^{2,4,5,*}$} 

\begin{document}
\maketitle
\date{
\noindent
$^1$ Strongly Correlated Systems Lend\"{u}let Research Group, Wigner Research Centre for Physics, PO Box 49, H-1525, Budapest, Hungary\\
$^2$ MTA-ELTE Lend\"{u}let "Momentum" NewQubit Research Group, P\'{a}zm\'{a}ny P\'eter, S\'et\'{a}ny 1/A, 1117 Budapest, Hungary\\
$^3$ Department of Chemical and Biological Engineering, The University of Alabama, Tuscaloosa, Alabama 35487, United States\\
$^4$ Department of Physics of Complex Systems, E\"otv\"os Loránd University, Egyetem t\'er 1-3, H-1053 Budapest, Hungary\\
$^5$ Department of Physics, Chemistry and Biology, Link\"oping University, SE-581 83 Link\"oping, Sweden\\
$^*$email: barcza.gergely@wigner.hu, ivady.viktor@ttk.elte.hu
}



%%%%%%%%%%%%%%%%%%%%%%%%%%%%%%%%%%%%%%%%%%%%%%%%%%%%%%%%%%%%%%%%%%%%%%%%%%%%%%%

\tableofcontents

\section{Detailed description of the computational methodology (molecular models)} 

\subsection{Construction of molecular (flake) models}

%
\begin{figure}[H]
\begin{center}
	\includegraphics[width=0.90\textwidth]{figs/FigSM1.pdf}
	\caption{ Molecular structure of the considered flake models. In the ball-and-stick representation generated by Chemcraft program package\cite{chemcraft}, hydrogens, borons, carbons and nitrogens are colored white, pink, gray, and blue, respectively.  We note that Flake 2 and Flake 3 are referred to as "small flake" and "large flake", respectively, in the article.}
	\label{fig:flakes}  
\end{center}
\end{figure}
%
In order to study the C4 defects based on finite molecular models, we designed three flake structures. As shown in Figure \ref{fig:flakes}, we initially formulated a minimal model (FLAKE 1) and systematically increased the number of pristine boron nitride hexagons around the defect, generating FLAKE 2 and FLAKE 3. As the first step of our investigations, we optimized the geometry of these models in their ground triplet state ($^{3}A_2^\prime$).

\subsection{Ground state orbital diagrams, shape of frontier (C4 centered) molecular orbitals}

\begin{figure}[H]
\begin{center}
	\includegraphics[width=\columnwidth]{figs/FigSM2.pdf}
	\caption{Orbital diagram of C4N and C4B defects, (computed at the ground state geometry of  FLAKE 2 model, ROKS-PBE/cc-pVDZ level of theory), shown together with the isosurface and symmetry of C4 centered defect molecular orbitals.  }
        \label{fig:supplementary-orbital_diagram}  
\end{center}
\end{figure}

\subsection{Identification of excited states (TD-DFT)}

To identify the lowest-lying excited electronic states, we carried out  time-dependent density functional theory (TD-DFT)\cite{Petersilka_1996}  calculations  on the ground state geometry of FLAKE 1 using the ORCA program suite. 10 triplet roots and 10 spin-flipped singlet roots were requested at TD-PBE0/cc-pVDZ level of theory. 
Even though such a calculation is not expected to provide accurate energies (and electronic properties),  it still reliably provides the most relevant orbital-to-orbital excitations to be studied.  
In Table \ref{tab:tddft_energies}, we summarize the C4 center related excitations found among the 10 roots. (The rest of the obtained roots typically describe electron transitions to/from rather delocalized orbitals belonging to the conduction/valence band.) 

\begin{table}[H]
\begin{center}
\caption{\label{tab:tddft_energies} Composition and energy of excited states based on TD-PBE0/cc-pVDZ calculations on the ground state geometry.}
 \begin{tabular}{c|c|c|ccc}
 \hline
 Defect & Leading excitation(s) (weight) & Degeneracy & Identified state & Relative energy (eV) \\ \hline
\multirow{4}{*}{C4N}   & Ground state & 1 & $^{3}A_2^\prime$ & 0.00\\
& $e^{\prime\prime}[\alpha] \rightarrow  e^{\prime\prime}[\beta]$ (88\%) & 2 & $^{1}E^\prime$ & 1.18\\
& $e^{\prime\prime}[\alpha] \rightarrow e^{\prime\prime}[\beta]$ (98\%) & 1 & $^{1}A_1^{\prime}$ & 1.54\\
& $e^{\prime\prime}[\alpha] \rightarrow a_2^{\prime\prime}(s)[\alpha]$ (74\%)& 2 & $^{3}E^{\prime}$ & 2.22\\
\hline
\multirow{6}{*}{C4B}   & Ground state & 1 & $^{3}A_2^\prime$ &  0.00\\
& $e^{\prime\prime}[\alpha] \rightarrow  e^{\prime\prime}[\beta]$ (92\%) & 2 & $^{1}E^\prime$ & 1.37\\
& $e^{\prime\prime}[\alpha] \rightarrow e^{\prime\prime}[\beta]$ (99\%) & 1 & $^{1}A_1^{\prime}$ & 1.90\\ 
& $e^{\prime}[\beta] \rightarrow e^{\prime\prime}[\beta]$ (91\%) &  1 & $^{3}A_2^{\prime\prime}$  & 3.15\\
& $e^{\prime}[\beta] \rightarrow e^{\prime\prime}[\beta]$ (73\%), $a_2^{\prime}[\beta] \rightarrow e^{\prime\prime}[\beta]$ (19\%) & 2 & $^{3}E^{\prime\prime}$ & 3.15\\
&$a_2^{\prime\prime}(s)[\beta] \rightarrow e^{\prime\prime}[\beta] (81\%)$ & 2 & $^{3}E^{\prime}$ & 3.27\\ 

\hline
 \end{tabular}
\end{center}
\end{table}

\subsection{Choice of active orbitals by orbital analysis of the DMRG solution}
\label{sect:dmrg}

To identify the chemically most relevant orbitals  of the lowest vertical excitations in a "reference-free" unbiased manner, we performed exploratory density matrix renormalization group (DMRG)\cite{White-1999} calculations which is a  multireference wave-function method capable to treat several dozens of strongly correlated orbitals\cite{Szalay-2015a,Olivares-2015,Baiardi2020}.

We studied the full valence space of FLAKE 1 of C4N and C4B defects. In this model $\sigma_{sp^2}$,$\sigma*_{sp^2}$, $\pi_{p_z}$ and $\pi*_{p_z}$ orbitals between B, C and N atoms were considered which were selected from  the full set of  spin-restricted open-shell  Kohn-Sham orbitals following Pipek-Mezey orbital localization\cite{Pipek-1989}. We note that terminating N-H and B-H bonds were omitted from the simulations as they are not relevant in the understanding of the defect embedded in macroscopic hBN layer. 

Accordingly, for both C4N and C4B, we correlated 76 electrons on  76 orbitals in the DMRG simulations considering the complete active space (CAS) protocol\cite{Roos1987}. Based on the state-averaged mutual information pattern of the target states, which had been obtained by cheap preliminary DMRG calculations, we determined an optimal DMRG orbital ordering for both defects\cite{barcza_2011}.
In the warmup sweep of the DMRG, environmental were constructed according to the restricted active space protocol in order to retrieve correlation effects of all orbitals  from the initial DMRG steps\cite{barcza_2022b}.
The DMRG truncation is based on the spectrum of the state-averaged reduced density matrix.
The relevant orbitals of the excitations, can already be predicted from  DMRG calculation with low bond dimension\cite{stein_2016,barcza_2022a}. Correspondingly,  in order to profile the convergence of the relative energies, we performed DMRG simulations with fixed $M=100,200,500$ number of block states.
Also note that  we found that the relative  DMRG spectrum predicted for $M$ number of retained states reaches chemical accuracy within 5 DMRG sweeps.


\begin{table}[H]
\begin{center}
\caption{\label{tab:CAS_DMRG_energy} Vertical energies  of FLAKE1 obtained in CAS based description. DMRG simulations were
 performed on the valence space of 76 spatial orbitals and 76 electrons for increasing $M$ block states.
All values were calculated at the geometry of the ground state ($^{3}A_2^\prime$) of FLAKE1 in ROKS basis, and are presented in eV units. }
 \begin{tabular}{c|c|ccc}
 \hline
 Defect & Electronic state & DMRG(M=100)  & DMRG(M=200)  & DMRG(M=500)  \\ \hline
\multirow{4}{*}{C4N}   & $^{3}A_2^\prime$ & 0.00 & 0.00 & 0.00  \\ 
& $^{1}E^\prime$ &  0.85 & 0.84 & 0.80\\
& $^{1}A_1^\prime$ &    2.50 & 2.38 & 2.27\\
& $^{3}E^\prime$ &  4.67 & 4.56 & 4.32\\ 
\hline
\multirow{6}{*}{C4B}   & $^{3}A_2^\prime$ & 0.00 & 0.00 & 0.00 \\ 
& $^{1}E^\prime$ &  1.35 & 1.27 & 1.23\\
& $^{1}A_1^\prime$ &    4.05 & 3.86 & 3.45 \\
& $^{3}A_2^{\prime\prime}$  & 5.11 &  5.02 & 4.96\\
& $^{3}E^{\prime\prime}$ &  4.87 & 4.82 & 4.45 \\
& $^{3}E^\prime$ &   5.02 &  4.87 & 4.58 \\ \hline
 \end{tabular}
 \end{center}
\end{table}

In line with the previous TD-DFT results, we requested 3 singlet and 3 triplet roots for C4N, while 3 singlet and 6 triplet roots for C4B. The DMRG energies are summarized in Table~\ref{tab:CAS_DMRG_energy}. 
We found that increasing the number of block states by a factor of five (M=100 vs 500), the excitation energies typically decrease only by 1-10\%.

The chemical relevance of the molecular orbitals was assessed by the  single-orbital entropy profile\cite{barcza_2011,stein_2016} for each electronic state of interest. 
We found that for C4N defect the obtained DMRG roots were found to characteristically correspond to the TD-DFT electronic states of Table~\ref{tab:tddft_energies}, here we identified the  seven $p_z$ orbitals localized on the defect carbons (indexed 37, 38, 39, 67) and the neighboring  borons (61, 62, 63) with substantial orbital entropy, see left panel of Fig.\ref{fig:DMRG_entropy} and Fig.\ref{fig:active_orbitals}.
In the case of C4B, owing to the weaker bonding of the carbons to the neighboring electron-deficient borons, we found that $\sigma$ system also plays substantial role, i.e., besides the $p_z$ orbitals on the carbons (37, 38, 39, 40) and on inner N atoms (34, 35, 36) $\sigma$ orbitals forming C-C (1, 22, 23) and neighboring B-C (2, 3, 10, 15, 24, 29) bonds have outstanding entropy contribution, see right panel of Fig.\ref{fig:DMRG_entropy} and Fig.\ref{fig:active_orbitals}. 

Accordingly, we selected a 4-electron 7-orbital active space for C4N, and a 28-electron 16-orbital active space for C4B. 

%
\begin{figure}[H]
\begin{center}
	\includegraphics[width=0.49\textwidth]{figs/FigSM3.pdf}
    \includegraphics[width=0.49\textwidth]{figs/FigSM4.pdf}
 \caption{(Left) Single orbital entropy $S_i$ of each target state of the FLAKE1 C4N defect obtained by DMRG keeping M=100 block states.
 (Right) Similar figure but for FLAKE1 C4B. }
	\label{fig:DMRG_entropy}  
\end{center}
\end{figure}
%

%
\begin{figure}[H]
\begin{center}
\includegraphics[width=\textwidth]{figs/FigSM5.pdf}
 \caption{Shape of localized DFT orbitals of FLAKE 1 selected for CASSCF active space.  For better visibility, the presented orbitals are grouped according to their characteristic structure. Number of assigned electrons are shown in parentheses.}
	\label{fig:active_orbitals}  
\end{center}
\end{figure}
%


\subsection{Choice of the size of the flake model (CASSCF-NEVPT2)}
The single-point electronic properties  discussed in the paper were determined at CASSCF/cc-pVTZ level of theory while the  single-point energies are corrected by the second-order N-electron valence perturbation theory (NEVPT2)\cite{nevpt2}. State-average calculations were performed, requesting multiple singlet and triplet roots.
The active space was selected from localized PBE0/cc-pVTZ molecular orbitals corresponding to density matrix renormalization group predictions as described in Sect.~\ref{sect:dmrg}. 

According to preliminary CASSCF-NEVPT2 calculations with different flake model sizes (Table \ref{tab:flake_convergence_casscf-nevpt2}), the  relative energy of the lowest-lying  vertical electronic states already reaches convergence at the size of FLAKE 2. 
On the other hand, it was previously shown that the calculation of phonons is more sensitive to the system size and requires a sufficiently large model~\cite{reimers_photoluminescence_2020}. Thus, we decided on computing the vibrational spectrum of the defect based on FLAKE 3, while other (relaxed) molecular properties were obtained using FLAKE 2.

\begin{table}[H]
\begin{center}
\caption{\label{tab:flake_convergence_casscf-nevpt2} Vertical energy spectrum obtained CASSCF-NEVPT2/cc-pVTZ energy levels for increasing flake sizes. Note that DLPNO-NEVPT2 perturbation\cite{dlpno-nevpt2} was applied for FLAKE 3 due to the unreasonably large computational cost of non-approximated NEVPT2. The active space ((4,7) or (28,16) for C4N and C4B defects, respectively) was selected from the set of localized Kohn-Sham orbitals as described in the previous section. All values were calculated at the geometry of the ground state ($^{3}A_2^\prime$), and are presented in eV units. (See Fig. \ref{fig:flakes} for the visualization of the flake models.)  As the difference between FLAKE 2 and FLAKE 3 excitation energies remains below 0.15 eV (i. e. the expected error margin of CASSCF-NEVPT2) in all cases, FLAKE 2 can be considered as a reasonable model for computing energies and electronic molecular properties. } 
 \begin{tabular}{c|c|ccc }
 \hline
 Defect & Electronic state & FLAKE 1 & FLAKE 2 & FLAKE 3  \\ \hline
\multirow{4}{*}{C4N}   & $^{3}A_2^\prime$ & 0 & 0  & 0   \\
& $^{1}E^\prime$ & 0.64 & 0.70 & 0.69   \\
& $^{1}A_1^\prime$ & 1.18 & 1.22 & 1.09  \\
& $^{3}E^\prime$ & 2.32 & 2.94 & 3.06   \\
\hline
\multirow{6}{*}{C4B}  & $^{3}A_2^\prime$ & 0 & 0 & 0 \\ 
& $^{1}E^\prime$ &  0.76 & 0.75 & 0.72   \\
& $^{1}A_1^\prime$  & 1.17 & 1.08 &  1.04 \\
& $^{3}A_2^{\prime\prime}$  & 3.17 & 2.78 &  2.65\\
& $^{3}E^{\prime\prime}$ & 3.10 & 2.78 &  2.67 \\
& $^{3}E^\prime$ & 3.66 & 3.71 &  3.66  \\ \hline
 \end{tabular}
 \end{center}
\end{table}



\subsection{Relaxation of excited states: geometry optimization, vibrational analysis (TD-DFT), single-point energies (CASSCF-NEVPT2)}

The identified singlet excited states ($^{1}E^\prime$, $^{1}A_1^\prime$) and the lowest-lying triplet excited states (C4N: $^{3}E^\prime$; C4B: $^{3}A_2^{\prime\prime}$, $^{3}E^{\prime\prime}$) were relaxed to their optimized geometry at TD-PBE0/cc-pVDZ level of theory, by following the TD-DFT root corresponding to the desired electron configuration. It is remarkable that the (TD-)DFT geometry optimization process automatically accounts for Jahn-Teller distortion effects (if applicable) - in this way, we discovered that the optimization of $^{3}A_2^{\prime\prime}$ and $^{3}E^{\prime\prime}$ gives the same geometry of C$_{2v}$ symmetry, which we handle as the fist triplet excited state of C4B in the following.   
After the optimization process, vibrational analysis (FLAKE 3, PBE0/cc-pVDZ level) and single-point energy calculations (FLAKE 2, CASSCF-NEVPT2/cc-pVTZ) were performed on the obtained relaxed structures. We note that the vibrational spectrum was calculated by freezing the N-H and B-H bonds using the partial Hessian technique\cite{partial_hess}, as these bonds are not present in actual hBN samples.  

\subsection{Photoluminescence rates and photoluminescence spectra}

The rate and spectrum of photoluminescence, corresponding to the relaxation of the triplet excited state to the triplet ground state, was calculated using the excited state dynamics (ESD) module of ORCA\cite{orca_esd}. In this framework, the dynamics are calculated from first principles, assuming harmonic nuclear movement and using the analytic solution of the path integral of the multidimensional harmonic oscillator to solve Fermi's golden rule for photon emission. In this work, we applied the adiabatic hessian model in ESD, that is, geometry and Hessian matrix for both the ground state and the excited geometry was provided from previous DFT calculations, as described above. The transition dipole moment vector and the adiabatic energy difference required for ESD was taken from the CASSCF-NEVPT2 results. The linewidth for the spectrum was set to 300 $cm^{-1}$ in order to simulate the usual experimental setup.

We note that we applied the Frank-Condon approximation, that is, the size of the transition dipole moment vector was considered to be constant and insensitive to the geometrical changes caused by vibrations.

\subsection{Spin-orbit and spin-spin couplings, zero-field splitting, intersystem crossing rates}

Spin-orbit coupling (SOC) matrix elements, spin-spin coupling (SSC) and zero-field splitting tensors were calculated in the framework of the quasi-degenerate perturbation theory (QDPT)\cite{qdpt}, in the basis of the obtained CASSCF roots. Dynamic correlation energy was taken into account by using NEVPT2 corrected diagonal energies in the QDPT matrix. 

Intersystem crossing rates can be computed analogously to photoluminescence rates as described in the previous section, except that the the probability of transition between states is described by SOC matrix elements.

\section{Comparison of periodic and flake model predictions}

\begin{table}[H]
\begin{center}
\caption{\label{tab:Bonds} Comparison of the carbon-carbon bond lengths for ground and excited state geometries calculated within different models. All values are in \AA.}
 \begin{tabular}{c|c|ccc}
 \hline
 Defect &  State &  Flake (PBE0/cc-pVDZ) &  Sheet (HSE06) & Bulk (HSE06) \\ \hline
\multirow{4}{*}{C4N} 
& $^{3}A_2^\prime$ & 1.416 & 1.415 & 1.415\\
& $^{3}E^\prime$ & 1.435, 1.436, 1.443 & 1.430, 1.430, 1.441 & 1.433, 1.434, 1.448\\
%%&&&\\ 
& $^{1}E^\prime$ & 1.381, 1.437, 1.437 & 1.383, 1.433, 1.434 & 1.391,1.415, 1.444\\  
& $^{1}A_1^\prime$ & 1.418, 1.418, 1.418 & - & -\\
\hline
\multirow{4}{*}{C4B}  
& $^{3}A_2^\prime$ & 1.418 & 1.408 & 1.408\\
& $^{3}A_2^{\prime\prime}$ & 1.383,1.383,1.487 & 1.368, 1.384, 1.462 & 1.405, 1.407, 1.435\\
%&&&\\ 
& $^{1}E^\prime$ & 1.393,1.430,1.430 & 1.392, 1.393, 1.443 & 1.382, 1.405, 1.440\\
& $^{1}A_1^\prime$ & 1.416,1.416,1.416 & - & -\\
\hline
 \end{tabular}
\end{center}
\end{table}

\begin{table}[H]
\begin{center}

\caption{\label{tab:Energy} Comparison of the zero phonon line energies calculated within different models. All values are in eV.}  
 \begin{tabular}{c|c|ccc}
  \hline
  Defect & Transition &  Flake (CASSCF/cc-pVTZ) &  Sheet (HSE06) & Bulk (HSE06) \\ \hline
\multirow{2}{*}{C4N}  
& $^{3}A_2^\prime$ $\rightarrow$ $^{3}E^\prime$  & 2.32 & 1.98 & 1.99\\
& $^{1}E^\prime$ $\rightarrow$ $^{1}A_1^\prime$ & 0.52 & - & -\\
\hline
\multirow{2}{*}{C4B}  
& $^{3}A_2^\prime$ $\rightarrow$ $^{3}A_2^{\prime\prime}$ & 2.37 & 2.40 & 2.44\\
& $^{1}E^\prime$ $\rightarrow$ $^{1}A_1^\prime$ & 0.35 & - & -\\
 \hline
\end{tabular}
  \end{center}
\end{table}

\section{Supplementary figures and tables}

\subsection{Excited state distortion in single layer models (C4N defect)}
\label{sec:supplementary-exc_distortion} 

\begin{figure}[h!]
\begin{center}
	\includegraphics[width=0.80\columnwidth]{figs/FigSM6.pdf}
	\caption{ Excited state configurations of the C4N defect in single layer hBN. \textbf{a} Structure of the polaronic-like distortion in the $^3E^{\prime}$ excited state in single layer hBN. \textbf{b} Localization of the highly distorted $a_2^{\prime\prime}(h)$ single particle state on the out-of-plane boron atom. }
	\label{fig:sb}  
\end{center}
\end{figure}

We note that we observe different geometry distortions for the triplet excited state of the C4N defect  in multi and single layer models. In the latter case, the highly symmetric structure spontaneously distorts beyond the Jahn-Teller effect and relaxes into a polaronic-like state where one of the second nearest neighbor boron pops out of the plane, see Fig.~\ref{fig:sb}a. In  this configuration the partially occupied $a_2^{\prime\prime}(s)$ state gets large distorted and localizes only on the p$_z$ orbital of the out of plane boron atom, Fig.~\ref{fig:sb}b. These results are consistently obtained in our periodic hybrid-DFT and CASSCF-NEVPT2 calculations. Using the latter method, we obtain an energy for the polaronic-like distortion $\sim0.5$~eV lower than the symmetric excited state configuration. On the other hand, the out-of-plane relaxation becomes energetically unfavourable in bulk and multi layer structures. Since the multi-layer  configuration is currently more relevant for the applications than the single layer configuration, we study the former in more details.


\subsection{Flakes buckled due to strain (C4N defect)}
\label{sup:sec:buckled_geom}
\begin{figure}[H]
\begin{center}
	\includegraphics[width=\textwidth]{figs/FigSM7.pdf}
	\caption{Visualization of the geometry of the ${}^{3}A^{'}_2$ electronic state at different stages of in-plane compression. Left: equilibrium geometry. Middle: 3\% contraction (i.e. distances between the central carbon and the outside atoms reduced by 3\% compared to equilibrium). Right: 5\% contraction (i.e. distances between the central carbon and the outside atoms reduced by 5\% compared to equilibrium). $\phi$ refers to the dihedral angle given by the position of the 4 carbon atoms. The geometries are presented using the molecular model FLAKE 2 (see \ref{fig:flakes}). Hydrogen atoms are omitted for clarity.}
        \label{fig:buckled_flakes}  
\end{center}
\end{figure}


\subsection{Vibrational modes coupled to the PL transition (C4N defect)}
\label{sup:sec:modes}
\begin{figure}[H]
\begin{center}
	\includegraphics[width=\columnwidth]{figs/FigSM8.pdf}
	\caption{Supplement to Fig. 4.: Visualization of the dominant vibrational modes of the phonon sideband of ${}^{3}E^{'} \rightarrow {}^{3}A^{'}_2$ transition}
        \label{fig:modes}  
\end{center}
\end{figure}


\subsection{Estimated PL and ISC rates for buckled flakes - demonstration of spin polarization and ODMR contrast (C4N defect)}
\label{sup:sec:transitionrates}
\begin{figure}[H]
\begin{center}
	\includegraphics[width=0.8\columnwidth]{figs/FigSM9.pdf}
	\caption{Estimated transition rates between the electronic states of the buckled flake ($\phi$=6.9°). The rates were computed using the Excited State Dynamics module of ORCA, based on the CASSCF(4,7)-NEVPT2 electronic energies shown in the Figure, the SOC matrix elements shown in Table 1 and the vibrational spectra of the flakes computed at TDDFT level. The thickness of reaction arrows is proportional to the order of magnitude of the represented transition rate. It can be observed that PL and ISC rates are comparable (enabling readout) and that the relaxation of ${}^{3}E^{'}$ through the singlet block results almost exclusively in the +/-1 spin sublevel of ${}^{3}A^{'}_2$ (enabling inicialization and manipulation).}
	\label{fig:tr}  
\end{center}
\end{figure}

\subsection{SOC matrix elements on displaced geometries (demonstration of Herzberg-Teller effect)}

\begin{table}[h!]
\begin{center}
\caption{\label{sup:tab:HTeffect} Spin-orbit coupling matrix elements (SOCMEs, absolute values) as obtained on CASSCF-NEVPT2 level of theory for equilibrium (${^3}A_2^{\prime}$) and displaced geometries. The displacement is given relative to the amplitude of the normal vibrational mode visualized in Figure \ref{sup:fig:out-of-plane-vibration}, displacement 0.00 denotes the equilibrium. All energy values are in GHz. Highlighted are the trends indicating the possibility of Herzberg-Teller transitions.}
 \begin{tabular}{|c|ccc|}
 \hline
 \multirow{2}{*}{  Transition} & \multicolumn{3}{c|}{Displacement }  \\ \cline{2-4}
 &  0.00 & 0.05  & 0.1 \\\hline
 \bf{$^3E^{\prime}(m_s = \pm1) \rightarrow {^1}A_1^{\prime}$} & \bf{0} & \bf{5.70} & \bf{9.81} \\
 $^3E^{\prime} (m_s = 0) \rightarrow {^1}A_1^{\prime}$ & 0  & 0 & 0   \\
 \bf{$^3E^{\prime} (m_s = \pm1) \rightarrow {^1}E^{\prime}$} & \bf{0} &  \bf{7.77} & \bf{12.99}   \\
 $^3E^{\prime} (m_s = 0) \rightarrow {^1}E^{\prime}$ & 0.27 & 0.24 & 0.27  \\
 $^1A_1^{\prime} \rightarrow {^3}A_2^{\prime} (m_s = \pm1)$ & 0 & 0 & 0  \\
 $^1A_1^{\prime}\rightarrow {^3}A_2^{\prime} (m_s = 0)$ & 2.82 & 1.98 &  1.26 \\
 \bf{$^1E^{\prime} \rightarrow {^3}A_2^{\prime} (m_s = \pm1)$} & \bf{0} & \bf{3.81} &  \bf{6.24}  \\
 $^1E^{\prime} \rightarrow {^3}A_2^{\prime} (m_s = 0)$ & 0  & 0 & 0  \\ \hline
 \end{tabular}
\end{center}
\end{table}

\begin{figure}[H]
\begin{center}
	\includegraphics[width=0.8\textwidth]{figs/FigSM10.pdf}
	\caption{Visualization of the out-of plane vibrational mode of the C4 center (E = 613 $cm^{-1}$). For clarity reasons, only the center of the flake is shown.}
        \label{sup:fig:out-of-plane-vibration}  
\end{center}
\end{figure}





\subsection{Hyperfine tensors of nuclear spins}
\label{sup:sec:hyperfine}
\begin{table}[H]
%\begin{center}
\caption{\label{sup:tab:hyperfine} Supplement to Table 2:  (Left) C4N  hyperfine tensors of all coupled nuclear spins. All values are in MHz. (Right) Similar but for CBN.}
\begin{tabular}{cc}%
 \begin{tabular}[t]{|c|cccccc| }
 \hline
 site & $A_{xx}$ & $A_{yy}$ & $A_{zz} = A_z$ & $A_{xy}$ & $A_{xz}$ & $A_{yz}$ \\ 
 \hline
C0 & -33.4 & -33.4 & -57.6 & 0.0 & 0.0 & 0.0  \\ \hline
C1$_a$ & 8.7 &  8.6 & 95.0 &  -0.1  &  0.0 & 0.0 \\
C1$_a$ & 8.6 &  8.8 &  95.0 &  0.0  &  0.0 & 0.0 \\
C1$_a$ & 8.7 &  8.6 &  95.0 &   0.1  & 0.0 & 0.0 \\ 
N1$_a$ &  -3.2 &  -3.3 &  0.3  & 0.1  &  0.0  & 0.0   \\
N1$_a$ &  -3.3 &  -3.1 &  0.3  & 0.0  &  0.0  &  0.0    \\
N1$_a$ &  -3.2 &  -3.3 &  0.3  &  0.1  &  0.0  &  0.0   \\
N1$_a$ &  -3.2 &  -3.3 &  0.3  &  -0.1  &  0.0  &  0.0    \\
N1$_a$ &  -3.2 &  -3.3 &  0.3  &  -0.1  &   0.0  & 0.0    \\
N1$_a$ &  -3.3 &  -3.1 &  0.3  &  0.0  &  0.0  &  0.0    \\

B2$_a$ &  -0.3 &  -0.5  &  1.7 &  -0.1 &   0.0 &  0.0  \\
B2$_a$ &  -0.3 &  -0.5 &   1.7 &   0.1 &  0.0 &   0.0 \\
B2$_a$ &  -0.6 &  -0.3  &  1.7 &  0.0  & 0.0 &  0.0  \\

B3$_a$  &  0.2 &  -0.9  &  1.5  &  0.3   & 0.0  &  0.0 \\    
B3$_a$  &  0.2 & -0.9  &  1.5 &  -0.3  &  0.0  &  0.0 \\   
B3$_a$ &  -0.9  & 0.2  &  1.5 &  -0.3  & 0.0  &  0.0 \\    
B3$_a$ &  -0.9  &  0.2  &  1.5  &  0.3  &  0.0  &  0.0 \\   
B3$_a$ &  -0.3  & -0.3  &  1.5  &  0.7  &  0.0  &  0.0  \\
B3$_a$ &  -0.3  &  -0.3  &  1.5  & -0.7  &   0.0  &   0.0 \\ \hline
B$_A$ &  -0.3 &  -0.3  &  0.6 & 0.0 & 0.0   & 0.0 \\
B$_B$ & -0.3 &  -0.3 & 0.6 & 0.0 & 0.0 &  0.0 \\ \hline
 \end{tabular}


&

 \begin{tabular}[t]{|c|cccccc| }
 \hline
 site & $A_{xx}$ & $A_{yy}$ & $A_{zz} = A_z$ & $A_{xy}$  & $A_{xz}$ & $A_{yz}$ \\ \hline
C$_c$ & -22.9 & -22.9 & -48.0  &  0.0  &  0.0 &  0.0  \\ 
C$_{sa}$ &-2.4 &  -2.5 &  68.9  & -0.1 &  0.0  &  0.0  \\
C1$_{sb}$ &   -2.6 &  -2.6 &  68.9 &   0.0  & 0.0 &  0.0   \\
C1$_{sc}$ &   -2.4 &  -2.5 &  68.9  &  0.1  &  0.0   & 0.0 \\
B$_{1a}$ &  -10.6  & -8.5 &  -6.8  & -0.2  &  0.0  &  0.0   \\
B$_{1b}$ &   -9.2  & -9.9  & -6.8  &  1.0 &  0.0 &   0.0   \\
B$_{1c}$ &   -8.9  & -10.2  &  -6.8  &  0.8 &  0.0 &  0.0   \\
B$_{1d}$ &   -9.2  & -9.9  & -6.8 &  -1.0  &  0.0  &  0.0   \\
B$_{1e}$ &   -8.9 & -10.2 &  -6.8  & -0.8  &  0.0  &  0.0   \\
B$_{1f}$ &  -10.6  & -8.5  & -6.8  &  0.2 &  0.0  &  0.0   \\
 B$_{5a}$ &   -0.5 &  -0.9 &  -1.1  &  0.3  &  0.0  &  0.0   \\
B$_{5b}$ &   -0.5  & -0.9 &  -1.1  & -0.3  &  0.0 &  0.0   \\
B$_{5c}$ &   -1.1  &  -0.4  & -1.1  &  0.0  &  0.0  & 0.0   \\
N$_{2a}$ &    -0.3  & -0.4  &  1.5   & 0.1 &  0.0  &  0.0    \\
N$_{2b}$ &    -0.3  &  -0.4  &  1.5   & -0.1  &  0.0  &  0.0    \\
N$_{2c}$ &    -0.5  &  -0.3  &  1.5  & 0.0  &  0.0  &  0.0    \\
N$_{3a}$ &   -0.3  & -0.2  &  1.1  &   0.1  &  0.0  &  0.0    \\
N$_{3b}$ &   -0.1  &  -0.4  &  1.1  &  0.0  &  0.0 &  0.0    \\
N$_{3c}$ &   -0.1  & -0.4  &  1.1  &  0.0 &  0.0 &  0.0    \\
N$_{3d}$ &   -0.3  & -0.3  &  1.1  & -0.1  &  0.0  &  0.0    \\
N$_{3e}$ &   -0.3 &  -0.2 &   1.1   & -0.1  &  0.0 & 0.0    \\
N$_{3f}$ &   -0.3  & -0.2  &  1.1   & 0.1 &  0.0  & 0.0    \\
 B$_{6a}$ &   -0.5  & -0.4 &  -0.7  &  0.3  &  0.0  &  0.0    \\
 B$_{6b}$ &   -0.2 &  -0.7 &  -0.7  &  0.1   & 0.0  &  0.0    \\
B$_{6c}$ &   -0.2 &  -0.7 &  -0.7 &  -0.1  & 0.0 &   0.0    \\
B$_{6d}$ &   -0.5 &  -0.4 &  -0.7  & -0.3   & 0.0  &  0.0    \\
 B$_{6e}$ &   -0.7 &  -0.2  & -0.7  & -0.2  &  0.0 &  0.0    \\
B$_{6f}$ &   -0.7  & -0.2 &  -0.7  &  0.2   & 0.0  &  0.0    \\
 B$_{4a}$ &   -0.4  &  -0.8  &  -0.2  &  -0.3  &  0.0  & 0.0    \\
B$_{4b}$ &   -0.4 &  -0.8 &  -0.2 &   0.3  & 0.0  & 0.0    \\
 B$_{4c}$ &   -1.0 &  -0.2 &  -0.2 &   0.0  &  0.0  &  0.0    \\
  B$_{Ba}$ &   -0.2 &  -0.2 &   0.6  &  0.0  &  0.2  &  0.1   \\
 B$_{Bb}$ &   -0.2 &  -0.2  &  0.6 &  0.0  & -0.2  &  0.1   \\
 B$_{Bc}$ &   -0.2 &  -0.2  &  0.6  &  0.0   & 0.0 &  -0.2    \\
 B$_{Aa}$ &   -0.2 &  -0.2 &   0.6 &   0.0  & -0.2 &  -0.1    \\
 B$_{Ab}$ &   -0.2 &  -0.2  &  0.6 &  0.0   & 0.2 &  -0.1   \\
 B$_{Ac}$ &   -0.2 &  -0.2  &  0.6  &  0.0   & 0.0  &  0.2    \\
 \hline
 \end{tabular}
  \tabularnewline
\end{tabular}
\end{table}


\section{Inputs and optimized geometries}
\subsection{Sample input files}

\subsubsection{Geometry optimization and partial vibrational analysis using ORCA 5.0.3.}
\label{input_opt}

! RIJCOSX PBE0 d3bj cc-pVDZ def2/J Opt numfreq \\
\%tddft \quad \quad \quad \quad \quad \quad \quad \quad \quad \quad \#Note: the \%tddft block was omitted in the case of ground electronic states \\
nroots 10 \\
iroot [number of root of interest] \\
sf ["true" (singlet states) or "false" (triplet states)] \\
end \\
\%freq \\
partial\_hess \{ [Atom number of H atoms]\} end \\
dx 0.01 \quad \quad \quad \quad \quad \quad \quad \quad \quad \quad \#Note: Low step size in order to avoid root flipping\\
end \\
\\
* xyzfile 0 3 [Geometry guess]

\subsubsection{Single-point energy, SOC and SSC calculations using ORCA 5.0.3.}
\label{input_sp}

! cc-pvtz normalprint moread cc-pvtz/c rijcosx def2/j nevpt2\\
\%moinp [gbw file containing localized PBE0/cc-pVTZ orbitals, with active orbitals rotated together] \\
\%casscf \\
nel [number of electrons: 4 (C4N) or 28 (C4B)] \\
norb [number of active orbitals: 7 (C4N) or 16 (C4B)] \\
mult 3,1 \\
nroots [triplet and singlet roots requested: 3,3 (C4N) or 6,6 (C4B)] \\
orbstep superci \\
switchstep diis \\
shiftup 2.0 \\
shiftdn 2.0 \\
minshift 0.6 \\
maxiter 150 \\
rel \\
dosoc true \\
dossc true \\ 
printlevel 4 \\
end \\
\\
* xyzfile 0 3 [Optimized geometry]

\subsubsection{Photoluminescence rate and spectrum calculation using ORCA 5.0.3.}
\label{input_pl}

!ESD(FLUOR) NOITER \\
\%ESD \\
GSHESSIAN ["hess" file of ground state (FLAKE 3)] \\
ESHESSIAN ["hess" file of excited state (FLAKE 3)] \\
DELE [Energy difference at CASSCF-NEVPT2 level (FLAKE 2)] \\
TDIP [Transition dipole moment between the states of interest at CASSCF-NEVPT2 level (FLAKE 2)] \\
tcutfreq 100 \\
temp 0 \\
linew 300 \\
printlevel 4 \\
END \\
\\
* xyzfile 0 3 [Optimized geometry of ground state (FLAKE 3)]

\subsubsection{Intersystem crossing rate calculation using ORCA 5.0.3.}
\label{input_isc}

!ESD(ISC) NOITER \\
\%ESD \\
ISCFSHESSIAN ["hess" file of final state] \\
ISCISHESSIAN ["hess" file of initial state] \\
DELE [Energy difference at CASSCF-NEVPT2 level] \\
SOCME [SOC matrix element between the states of interest at CASSCF-NEVPT2 level] \\
coordsys cartesian \\
ifreqflag remove \\
tcutfreq 100 \\
temp 0 \\
printlevel 4 \\
END
\\
* xyzfile 0 3 [Optimized geometry of final state]

\subsection{Optimized geometries}

The geometries of molecular models (all relevant electronic states optimized at (TD-)PBE0/cc-pVDZ level) and the atomic coordinates of supercell units can be found in the supplementary file "geometries.pdf". All xyz coordinates are given in angstroems.



\input{supplementary_ref.bbl}


\end{document}