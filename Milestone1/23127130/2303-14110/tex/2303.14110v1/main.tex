\documentclass{nature-pre}
\bibliographystyle{naturemag}
\usepackage{times}

\pagestyle{plain}
\usepackage{graphicx}
\usepackage{amsmath}
\usepackage{amssymb}
\usepackage{multirow}




\graphicspath{./}
 



\title{Symmetric carbon tetramers forming chemically stable spin qubits in hBN}
\author{Zsolt Benedek$^{1,2,3}$, Rohit Babar$^{1,2}$, \'{A}d\'{a}m Ganyecz$^{1,2}$, Tibor Szilv\'{a}si$^{3}$, \"Ors Legeza$^1$, Gergely Barcza$^{1,2,3,*}$, Viktor Iv\'{a}dy$^{2,4,5,*}$} 

\begin{document}
\maketitle

\begin{affiliations}
\item  {Strongly Correlated Systems Lend\"{u}let Research Group, Wigner Research Centre for Physics, PO Box 49, H-1525, Budapest, Hungary}
\item {MTA-ELTE Lend\"{u}let "Momentum" NewQubit Research Group, P\'{a}zm\'{a}ny P\'eter, S\'et\'{a}ny 1/A, 1117 Budapest, Hungary}
\item{Department of Chemical and Biological Engineering, The University of Alabama, Tuscaloosa, Alabama 35487, United States}
\item{Department of Physics of Complex Systems, E\"otv\"os Loránd University, Egyetem t\'er 1-3, H-1053 Budapest, Hungary}
\item   {Department of Physics, Chemistry and Biology, Link\"oping University, SE-581 83 Link\"oping, Sweden}
\item[*] email: barcza.gergely@wigner.hu, ivady.viktor@ttk.elte.hu
\end{affiliations}

\date{\today}


\newpage

\begin{abstract}
Point defect quantum bits in semiconductors have the potential to revolutionize sensing at atomic scales.  Currently, vacancy related defects, such as the NV center in diamond and the VB$^-$ in hexagonal boron nitride (hBN), are at the forefront of high spatial resolution and low dimensional sensing. On the other hand, vacancies' reactive nature and instability at the surface limit further developments. Here, we study the symmetric carbon tetramers in hBN and propose them as a chemically stable spin qubit for sensing in low dimensions. We utilize periodic-DFT and quantum chemistry approaches to reliably and accurately predict the electronic, optical, and spin properties of the studied defect. We show that the nitrogen centered symmetric carbon tetramer gives rise to spin state dependent optical signals with strain sensitive intersystem crossing rates. Furthermore, the weak hyperfine coupling of the defect to their spin environments results in a reduced electron spin resonance linewidth that may enhance sensitivity.
\end{abstract}


\newpage

%%%%%%%%%%%%%%%%%%%%%%%%%%%%%%%%%%%%%%%%%%%%%%%%%%%%%%%%%%%%%%%%%%%%%%%%%%%%%%%


\section*{Introduction}
Condensed matter physics in low dimensions is already a vast, yet rapidly growing field. Especially, transition metal dichalcogenides\cite{wang_electronics_2012,manzeli_2d_2017} and complex van der Waals heterostructures\cite{geim_van_2013,liang_van_2020} develop with unprecedented pace. The study of these nanometer-scale structures and related phenomena demands novel high-spatial resolution sensing devices  operating in a wide temperature range and sensitive to various external fields. Point defect quantum bits in semiconductors, such as the NV center in diamond\cite{Jelezko:PRL200492,DohertyNVreview} and the silicon vacancy in silicon carbide\cite{Widmann2014}, have already demonstrated outstanding capabilities in high-spatial resolution sensing and fulfilled many of these requirements.\cite{taylor_high-sensitivity_2008,schirhagl_nitrogen-vacancy_2014,barry_sensitivity_2020,sturner_integrated_2020,zhang_toward_2021} The distance of the sensor from the targeted system is of crucial importance for high-spatial resolution sensing. Therefore, further improvements require point defect sensors to be engineered closer to the surface or to be directly integrated into various low-dimensional structures. The currently available point defect qubit sensors in 3D semiconductors are not optimal for such applications due to their inherently bulk nature and strong dependence on surface chemistry.\cite{kaviani_proper_2014,kim_decoherence_2015,dwyer_probing_2022} The development of point defect qubits in layered van der Waals semiconductors may provide a way to overcome this obstacle as their surface is chemically stable and the thickness of the host material, and thus the distance of the qubits and the surface, can be engineered straightforwardly by exfoliation.\cite{tetienne_quantum_2021,healey_quantum_2022,kumar_magnetic_2022} Furthermore, van der Waals semiconductors with spin qubits can implement atomic thin sensors with advanced capabilities.\cite{gottscholl_spin_2021,liu_temperature-dependent_2021,tetienne_quantum_2021,healey_quantum_2022,kumar_magnetic_2022,lyu_strain_2022}

Hexagonal boron nitride (hBN) is a layered wide-band gap semiconductors, which is often used in van der Waals heterostructures. Its large, close to 6~eV band gap accommodates numerous optically active electronic states of structural defects and impurities.\cite{caldwell_photonics_2019,sajid_single-photon_2020} Exfoliated hBN samples may contain point defects in such a low number that even individual color centers can be observed with confocal microscopy techniques. Numerous single photon emitters were demonstrated in hBN that has opened a new field.\cite{tran_quantum_2016,caldwell_photonics_2019} Point defect quantum bits form a special class of color centers that carry high spin ground and optically excited states and feature a spin dependent optical emission. This phenomena makes optical detection of magnetic resonance (ODMR) measurements possible. ODMR signal of different spin qubits have already been reported\cite{gottscholl_initialization_2020,chejanovsky_single-spin_2021,mendelson_identifying_2021,stern_room-temperature_2022} and predicted\cite{sajid_vncb_2020,babar_quantum_2021,bhang_first-principles_2021,liu_spin-active_2022} in hBN.  One of the observed ODMR centers have been identified as the negatively charged boron vacancy center (VB$^-$ center).\cite{gottscholl_initialization_2020,ivady_ab_2020,haykal_decoherence_2022,liu_coherent_2022}  Identification of other ODMR centers remains elusive, despite the numerous experimental\cite{mendelson_identifying_2021,chejanovsky_single-spin_2021,stern_room-temperature_2022} and theoretical works\cite{jara_first-principles_2021,li_carbon_2022,golami_ab_2022,marek_thermodynamics_2022}. Recently, it has been demonstrated that the formation of these ODMR center is directly related to carbon contamination in hBN.\cite{mendelson_identifying_2021} 

The electronic, optical, and spin properties of the VB$^-$ center has been comprehensively studied in the literature in numerous experimental\cite{gottscholl_initialization_2020,gao_high-contrast_2021,murzakhanov_electronnuclear_2022,haykal_decoherence_2022,liu_coherent_2022,gao_nuclear_2022} and theoretical studies\cite{ivady_ab_2020,sajid_edge_2020,reimers_photoluminescence_2020,barcza_dmrg_2021}. While this center has already been successfully used in various sensing applications\cite{gottscholl_spin_2021,liu_temperature-dependent_2021,tetienne_quantum_2021,healey_quantum_2022,lyu_strain_2022}, single defect measurements have not been demonstrated yet, expectantly due to the center's low photo luminescence (PL) emission rate\cite{ivady_ab_2020}. Furthermore, as reactive nitrogen dangling bonds give rise to the electronic states of the VB$^-$ defect that may be terminated by mobile interstitial and adatoms on the surface, the VB$^-$ center is expectedly chemically unstable in few layer hBN samples. However, antisites and impurities with satisfied covalent bonds may form chemically stable structures with optically addressable electronic states. Spin qubits realized by such defects may remain functional even on the surface and in atomically thin layers. While antisites give rise to only a few structures with no relevant spin properties\cite{WestonPhysRevB2018}, carbon impurities are common in hBN, give rise to numerous complex structures, and form  expectedly the unidentified ODMR centers. 


Here, we theoretically study the neutral charge state of the nitrogen and the boron centred symmetric carbon tetramer structures in hBN, i.e.\ (C$_{\text{B}}$)$^3$-C$_{\text{N}}$ and (C$_{\text{N}}$)$^3$-C$_{\text{B}}$ that we dub as C4N and C4B, respectively. The electronic structures of the defects consist of a triplet ground state, an optically allowed triplet excited state, and two singlet states between the triplets. For the C4B defect,  we obtain large inter-system crossing rate from the triplet excited state to the singlet manifold and strain dependent inter-system crossing rate to the ground state. For the C4N defect, the spin selective decay is enabled by out-of-plane distortions. Therefore, strain can be used to engineer the defects' inter-system crossing rates and contrast.  In addition, the carbon tetramers  gives rise to narrow magnetic resonance lines, due to the localization of the spin density on the spinless carbon atoms.  Relying on our results, we propose the C4N and C4B defects in hBN as chemically stable spin qubits for improving sensing in low dimensions.

The thermodynamic properties of carbon complexes have been  studied recently in the literature.\cite{marek_thermodynamics_2022,huang_carbon_2022} Importantly, the symmetric C4N and C4B tetramers, see Fig.~\ref{fig:fig1}a and b, possess a surprisingly low formation energy. This can be explained by the Baird-aromatic stabilization of C4 containing hBN  (i.e. there are two unpaired electrons and altogether $4k$ electrons in the delocalized $\pi$ system of any selected set of rings; $k$ denotes an arbitrary integer), which is similarly favorable to the Hückel aromaticity of pure hBN (i. e. there are $4k+2$ electrons in the delocalized $\pi$ system, all of which are paired). We note here that based on Baird's and Hückel's rules \cite{aromaticity}, the general conclusion can be drawn that even number of carbon atoms can, while odd number of carbon atoms cannot maintain aromaticity, implying larger formation energies in the latter case.

The formation energy of the neutral C4N and C4B defects are 2.5~eV (8.3~eV) and 8.7~eV (2.9~eV) in N-rich (B-rich) growth conditions and further decreases in charged configurations.\cite{marek_thermodynamics_2022} The most relevant charge transition levels are $E(++|+) =  2.38$~eV, $E(+|0)= 3.28$~eV, and $ E(0|-)= 5.59$~eV for the C4N defect and $E(+|0) = 0.48 $~eV, $ E(0|-) = 3.03$~eV, and $ E(-|--) = 3.90$~eV for the C4B defect measured from the valence band maximum.\cite{marek_thermodynamics_2022} The neutral charge state of the C4N (C4B) defect is thus stable in the upper half (lower half) of the band gap, where the Fermi energy is located in N-rich (N-poor) growth conditions.\cite{marek_thermodynamics_2022}





\begin{figure}[!ht]
\begin{center}
	\includegraphics[width=0.6\columnwidth]{figs/Fig1.pdf}
	\caption{ Atomic and electronic structure of symmetric carbon tetramers in hBN. {\bf a} and {\bf b} show the ground state atomic configuration of the nitrogen site centered C4N defect and the boron site centered C4B defect, respectively. Both defects exhibit D$_{3h}$ point group symmetry.  {\bf c} and  {\bf d} depict the corresponding single particle electronic structures. Yellow, red, and blue lines represents conduction band, valence band, and defect energy levels, respectively. Up and down arrows indicate the occupation of the defect states. The localized defect orbitals are depicted in Supplementary Figure S2.}
	\label{fig:fig1}  
\end{center}
\end{figure}



\section*{Results}

The single particle electronic structures of the neutral ground state of the C4N and C4B defects are depicted in Fig.~\ref{fig:fig1}c and d, respectively. In-plane sp2 bonding states of the atoms are fully occupied and most of them fall deep in the valence band. The four $p_z$ orbitals of the carbon atoms form defect states that appear inside the band gap.  In the neutral charge state of the C4N defect, the most relevant single particle defects states are the fully occupied $a_2^{\prime\prime}\left( c \right)$  state, the half occupied $e^{\prime\prime}$  states, and the empty $a_2^{\prime\prime}\left( s \right)$, see Fig.~\ref{fig:fig1}c. As visualized in Supplementary Figure S2, the $a_2^{\prime\prime}$ and $e^{\prime\prime}$ orbitals are primarily located on the central carbon atom and on the three side carbon atoms, respectively. $a_2^{\prime\prime}\left( c \right)$ orbital is strongly localized on the central carbon, while $a_2^{\prime\prime}\left( s \right)$ orbital is slightly delocalized and has significant contributions from the $p_z$ orbitals of the surrounding three boron atoms. The most relevant defects states for the C4B defect  are the fully occupied $e^{\prime}$ and $a_2^{\prime\prime}\left( s \right)$  states, the half occupied $e^{\prime\prime}$  state, and the empty $a_2^{\prime\prime}\left( c \right)$, see Fig.~\ref{fig:fig1}d and Fig.~S2. The occupied defect states of the C4B defect can be found closer to the valence band compared to the case of the C4N defect. For the C4B defect, a fully occupied in-plane bonding $e^{\prime}$ state can be found close to the valance band maximum and it plays an important role in the optical excitation process. Since the double degenerate $e^{\prime\prime}$ state is occupied by two electrons with parallel spin, both defects exhibit a triplet ground state.

\begin{figure}[!ht]
\begin{center}
	\includegraphics[width=0.9\columnwidth]{figs/Fig2.pdf}
	\caption{ Many-body electronic structure of the symmetric C4N and C4B defects. {\bf a} Many-body spectrum of the C4N defect separated into singlet and triplet manifolds. Arrows connecting the states indicate possible optical and spin-orbit interaction mediated non-radiative transitions.  {\bf b} -  {\bf e} Slater determinant expression of the many body states of the C4N defect. Squares with numbers indicate the occupancy of each of the localized defect states, while groups of squares represent different Slater determinants. The percentages below the Slater determinants represent the weight of the determinant in the expression. {\bf f} Many-body spectrum of the C4B defect. {\bf g} -  {\bf k} Slater determinant expression of the many body states of the C4B defect.  }
	\label{fig:fig2}  
\end{center}
\end{figure}


The many-body electronic structures of the symmetric carbon tetramers are schematically visualized in Fig.~\ref{fig:fig2}, where the energy gaps are obtained with an anticipated error margin of  $\pm0.15$~eV\cite{NEVPT2_error} on CASSCF-NEVPT2 level of theory\cite{nevpt2}. As can be seen, four (five) states can be found in the low energy part of the electronic structure of the C4N (C4B) defect. The ground states of both defects can be described to a large degree ($>\!93$\%) by the single Slater determinant obtained in the single particle picture in DFT, see Fig.~\ref{fig:fig2}c-d. For C4N, a $^3E^{\prime}$ triplet state can be found 2.32~eV above the $^3A_2^{\prime}$ ground state. (The corresponding zero phonon photoluminescence (ZPL) energy, which  is obtained by adding the zero-point energy contribution of the local vibrational modes to the 2.32~eV adiabatic energy difference, was found to be 2.18~eV.) The $^3E^{\prime}$ optically excited state is largely described by the determinant corresponding to the $e_x^{\prime\prime} \rightarrow a_2^{\prime\prime} \! (s)$ transition, however, other determinants of single excitation mix with the leading term as depicted in Fig.~\ref{fig:fig2}\textbf{c}. Transition between the triplets is enabled by  parallel to $c$ polarized photon absorption and emission. The $^3E^{\prime}$ state is Jahn-Teller (JT) unstable and the optimized structure is slight distorted, see Fig.~\ref{fig:fig2}\textbf{a}. Due to the small JT distortion, we expect a dynamic Jahn-Teller effect, where the vibronic $^3\widetilde{E}^{\prime}$ state exhibits effective high symmetry. (Here, we note that an alternative, out-of-plane distorted triplet excited state geometry was also observed in some calculations, see Supplementary Information, Figure~S5. This effect is however peculiar to single, separated hBN layers.) In between the triplet states, there are two singlet excited states, a $^1E^{\prime}$ state and a $^1A_1^{\prime}$ state 0.61~eV and 1.13~eV above the ground state energy level, respectively. 

For the C4B defect, we obtain a similar many-body electronic structure as for the C4N defect; however, with an additional dark triplet excited state in-between the ground state and the $^3E^{\prime}$ triplet excited state, see Fig.~\ref{fig:fig2}a and f. The lower lying $^3A_2^{\prime\prime}$ excited state and the optically excited $^3E^{\prime}$ state  is expected to be found  {2.37}~eV and $\sim${3.0}~eV above the ground state. (Note that the latter value is an estimation based on the vertical excitation energy of 3.7 eV at the $^3A_2^{\prime}$ geometry and approximate structural relaxation energy as the relaxation to $^3E^{\prime}$ geometry could not be performed due to the delocalization of the characteristic orbitals.) These states are described to a large degree by an $e^{\prime} \rightarrow e^{\prime\prime}$ transition and $a_2^{\prime} \rightarrow e^{\prime\prime}$ transition, respectively, see Fig.~\ref{fig:fig2}h and i. Both states are JT unstable and the point group symmetry of the optimized excited state structure reduces to C$_{2\text{v}}$. The relaxed lowest energy triplet excited state belongs to the $B_2$ irreducible representation of C$_{2\text{v}}$. In D$_{3\text{h}}$ symmetry, no optical transition is possible between the ground and the $^3A_2^{\prime\prime}$ excited state. This property is largely preserved even in the JT distorted $^3B_2^{\prime\prime}$ state. Transition to the higher lying $^3E^{\prime}$ state is possible, similarly to the C4N defect; however, due to the proximity of the $^3B_2^{\prime\prime}$ state a rapid non-radiative decay to this lower lying triplet state is expected.  Therefore, the C4B defect can be optically excited, but no PL emission is possible from this defect in the visible range. In addition to the triplets, we find two singlets, a $^1E^{\prime}$ state and a $^1A_1^{\prime}$ state 0.72~eV and 1.07~eV above the ground state, respectively. 

Slater determinant expansion of the singlet states are also provided in Fig.~\ref{fig:fig2}\textbf{b}-\textbf{e} and Fig.~\ref{fig:fig2}\textbf{g}-\textbf{k} for the C4N and C4B defects. The most relevant $^1E^{\prime}$ and the $^1A_1^{\prime}$ states are the singlets that correspond to the ground state occupancy of the single particle orbitals. These singlets mix with other excited determinants to a similar degree as the optically excited states. The $^1E^{\prime}$ state is a Jahn-Teller unstable states and it goes through a severe structure distortion. In the optimized configuration one of the C-C bonds shortens that splits the $^1E^{\prime}$ state into a $^1A_1$ and a $^1B_1$ states in C$_{2v}$ symmetry, see Fig.~\ref{fig:fig2}a and f. 

For the radiative lifetime of the triplet excited states of the C4N defect, we obtain 80.5~ns at 0~K using 2.13 refractive index for a 590~nm photon. From the computed dipole moment of the ${}^{3}E^{\prime} \rightarrow {}^{3}A^{\prime}_2$ transition and harmonic vibrational analysis of the two individual electronic states, we calculate the phonon side band (PSB) of the C4N defect, see Fig.~\ref{fig:pl}.  We obtain 1.8 for the Huang Rhys factor (HR factor) that correspond to Debye Waller factors (DW factors) of 0.165. The relatively high DW factor for C4N defect is due to the high symmetry and small distortion of the excited state geometry. In addition, in $D_{3h}$ symmetry we expect weak coupling to electric field for the C4N structure.\cite{zhigulin_stark_2022}

\begin{figure}[!ht]
\begin{center}
	\includegraphics[width=0.80\columnwidth]{figs/Fig3.pdf}
	\caption{ Photoluminescence spectrum of the symmetric C4N defect in hBN compared with an experimental signal\cite{mendelson_identifying_2021} of the unidentified carbon related spin qubit in hBN. For better comparison the  theoretical ZPL emission is aligned  with the first maximum of the experimental spectrum. The theoretical spectrum is broadened to mimic temperature and inhomogeneous effects. Blue and green impulses show the partial Huang Rhys factors of one and two photon processes. }
	\label{fig:pl}  
\end{center}
\end{figure}

Partial HR factors of the vibrational modes indicate that the PL transitions couple strongest to in-plane carbon-carbon bond stretching modes that drive the JT distorted excited state into the symmetric ground state configuration (see section S3.3 of the Supplementary Information for the visualization of the vibrations). The energy of both modes is 0.196~eV (1583 cm$^{-1}$), which corresponds to the generally observed range of aromatic C-C stretching in infrared spectroscopy.\cite{aromatic_stretching} On the other hand, additional lower energy modes also couple to the defect that broaden and shift the maximum of the resonance peaks of the calculated PL spectrum. The phonon side-band of the PL emission of the C4N defect is compared with the experimental PL spectrum\cite{mendelson_identifying_2021,stern_room-temperature_2022} of the unidentified carbon related quantum bit, see Figure~\ref{fig:pl}. The theoretical and experimental spectra agree very well.


\begin{table}
\begin{center}
\caption{\label{tab:LSdata} Spin-orbit coupling matrix elements (SOCMEs) as obtained on CASSCF-NEVPT2 level of theory for 0~K optimised and distorted atomic configurations. The distortion is quantified by the dihedral angle ($\phi$) corresponding to the position of the 4 carbon atoms in the ground state. See section S3.2 of SI for the visualization of distorted geometries. All values are in GHz. We note that the matrix element depends on the geometry. Herein, the average of initial-state and final-state SOCMEs is given. When a zero and a non-zero SOCMEs are obtained for the involved inital and final electronic states, the non-zero value is provided.}
 \begin{tabular}{|c|ccc|| c|c| }
 \hline
 \multicolumn{4} {|c||}{ C4N } & \multicolumn{2} {c|}{ C4B } \\ \hline
 \multirow{3}{*}{  Transition} & \multicolumn{3}{c||}{matrix element } & \multirow{3}{*}{  Transition} &  matrix \\ \cline{2-4}
 &  0~K static & Buckling  & Buckling & & element \\ \cline{6-6}
 &  ($\phi$=0°) &  ($\phi$=2.7°) & ($\phi$=6.9°) & &  0~K static \\ \hline
 $^3E^{\prime}(m_s = \pm1) \rightarrow {^1}A_1^{\prime}$ & 0 & 16.03 & 31.95 & $^3A_2^{\prime\prime}(m_s = \pm1) \rightarrow {^1}A_1^{\prime} $ &  27.08 \\
 $^3E^{\prime} (m_s = 0) \rightarrow {^1}A_1^{\prime}$ & 0.03  & 0.03 & 0.18 &  $^3A_2^{\prime\prime} (m_s = 0) \rightarrow {^1}A_1^{\prime}$ & 0  \\
 $^3E^{\prime} (m_s = \pm1) \rightarrow {^1}E^{\prime}$ & 0 &  7.67 & 13.26 & $^3A_2^{\prime\prime} (m_s = \pm1) \rightarrow {^1}E^{\prime}$ & 48.81  \\
 $^3E^{\prime} (m_s = 0) \rightarrow {^1}E^{\prime}$ & 0.45 & 4.65 & 14.94 & $^3A_2^{\prime\prime} (m_s = 0) \rightarrow {^1}E^{\prime}$ & 0  \\
 $^1A_1^{\prime} \rightarrow {^3}A_2^{\prime} (m_s = \pm1)$ & 0 & 0.26 & 0.15 & $^1A_1^{\prime} \rightarrow {^3}A_2^{\prime} (m_s = \pm1)$ & 0  \\
 $^1A_1^{\prime}\rightarrow {^3}A_2^{\prime} (m_s = 0)$ & 2.76 & 2.61 &  11.25 & $^1A_1^{\prime}\rightarrow {^3}A_2^{\prime} (m_s = 0)$ & 2.71 \\
 $^1E^{\prime} \rightarrow {^3}A_2^{\prime} (m_s = \pm1)$ & 0 & 6.41 &  9.53 & $^1E^{\prime} \rightarrow {^3}A_2^{\prime} (m_s = \pm1)$ & 0 \\
 $^1E^{\prime} \rightarrow {^3}A_2^{\prime} (m_s = 0)$ & 0.27  & 0.24 & 1.47 & $^1E^{\prime} \rightarrow {^3}A_2^{\prime} (m_s = 0)$ & 0.06 \\ \hline
 \end{tabular}
\end{center}
\end{table}

Point defect quantum bits are a special type of color centers that exhibit high spin ground state and spin state dependent optical emission through spin selective non-radiative decay processes from the optically excited state to the ground state. As  carbon tetramers possess high spin ground state and singlet shelving states between the triplet excited and ground states, they may implement optically addressable spin quantum bits in hBN that we investigate in the following. 

In order to obtain an efficient spin state dependent non-radiative decay channel through the singlets, the triplet and the singlet manifolds should be coupled by strong spin-orbit coupling (SOC) matrix elements. For the C4B defect we obtain 48.81~GHz (27.08~GHz) spin-orbit coupling matrix elements between the $^3A_2^{\prime\prime}$($m_S = \pm1$) and the ${^1}E^{\prime}$ (${^1}A_1^{\prime}$) states, see Table~\ref{tab:LSdata}. Spin-orbit interaction thus gives rise to a decay channel with approximately 0.07~MHz decay rate for the dark excited triplet of the C4B defect. The spin-orbit coupling matrix elements between the $^{1}E^{\prime}$ state and the ground state are weaker, enabled mostly by the Jahn-Teller distortion of the low lying singlet state. Consequently, the $^{1}E^{\prime}$ state is long-lived compared to the triplet excited state. These results indicate that the C4B defect can be spin polarized through an optical excitation to the ${^3}E^{\prime}$ state and subsequent non-radiative and spin selective decay through the ${^3}A_2^{\prime\prime}$, ${^1}A_1^{\prime}$, and ${^1}E^{\prime}$ states. Due to this behaviour, the C4B defect may observed in electron spin resonance (ESR) measurement in low concentrations under $\sim$3.0~eV excitation. In addition, due to the strict spin selectivity of the non-radiative decay from the ${^3}A_2^{\prime\prime}$ excited state and the expectedly long lifetime of the ${^3}A_2^{\prime\prime}$($m_S = 0$) state, the defect may be suitable for photoelectron detected magnetic resonance (PDMR) read-out of the spin states.
 


The C4N defect's lowest energy triplet excited state has a different symmetry than the excited state of the C4B defect, therefore, the spin-orbit coupling between the triplets and the singlets is forbidden in first order approximation in D$_{3h}$ symmetry. Indeed, considering the 0~K static atomic structure and corresponding many-particle electronic states, we obtain either zero or small spin-orbit coupling matrix elements between the states on CASSCF-NEVPT2 level of theory, see Table~\ref{tab:LSdata}, left column. The non-zero elements are due to the JT effect that give rise to weak couplings between states. On the other hand, strain and out-of-plane distortions may break the symmetry that could give rise to a sizable increases of the spin-orbit coupling matrix elements for C4N. For example, in-plane compression of hBN flakes causes the sample to buckle, which can be quantified by the dihedral angle given by the position of the 4 carbon atoms (denoted as $\phi$ in Table~\ref{tab:LSdata}). $\phi$ equals to 0° in the equilibrium geometry as all C atoms are located in one plane, but it shows an increasing deviation from 0° when enforcing the side of the flake - i.e.\ the outside B and N atoms - to remain in fixed position closer and closer to the center (see section S3.2 of Supplementary Information for representative geometries).  Buckling of hBN, which is commonly observed in experiments \cite{stern_room-temperature_2022}, can give rise to spin-orbit matrix elements in the 10-30~GHz range that are comparable with NV center's matrix elements.\cite{thiering_ab_2017} The increased SOC matrix elements give rise to spin dependent non-radiative dacay channels that may facilitate optical read-out of the spin state.


As a demonstrative example, we computed the photoluminescence and inter-system crossing rates between the electronic states of the buckled flake at $\phi$=6.9° - see section S3.4 of the Supplementary Information. The obtained data indicate polarizability in $m_S = \pm 1 $ state. These results demonstrate that the C4N defect can implement an optically addressable spin qubit, whose spin dependent optical signal is enabled mostly by local strain.

Apart from strain induced geometry distortions, the out-of-plane vibration of the C4 center may also give rise to significant SOC matrix elements and - consequently - intersystem crossing rates in the case of C4N. Namely, even though the matrix elements are zero (or close to zero) in equilibrium geometry, it does not hold true for non-equilibrium geometries where the system spends a significant amount of time due to its constant vibration (See Table S6 for SOC matrix elements computed at geometries displaced along the out-of-plane normal mode). Thus, a Herzberg-Teller transition\cite{Herzberg-Teller} is possible, the rate of which strongly depends on the occupancy of vibrational levels (i. e. the temperature).  


\begin{table}
\begin{center}
\caption{\label{tab:HFdata} Hyperfine tensors of C4N and C4B defect for the strongest coupled nuclear spins. Location of the considered nuclear spins are visualized in Fig.~\ref{fig:spin}.  In the calculations, we use $^{13}$C, $^{11}$B, and $^{14}$N isotopes. The table provides the hyperfine tensor for one of the symmetrically equivalent positions and N gives  the number of equivalent positions in the lattice. The complete table of hyperfine tensors can be found in Table~S3.6 of the Supplementary Information. All values are in MHz. }
 \begin{tabular}{|c|c|cccc|| c|c|cccc| }
 \hline
 \multicolumn{6}{|c||}{C4N}  &   \multicolumn{6}{c|}{C4B} \\ \hline
 site & N & $A_{xx}$ & $A_{yy}$ & $A_{zz} = A_z$ & $A_{xy}$ & site & N & $A_{xx}$ & $A_{yy}$ & $A_{zz} = A_z$ & $A_{xy}$  \\ \hline
C$_c$ & 1 & -33.4 & -33.4 & -57.6 & 0.0 & C$_c$ & 1 & -22.9 & -22.9 & -48.0 & 0.0 \\ 
C$_s$ & 3 & 8.6 &  8.8 &  95.0 &  0.0 & C$_s$ & 3 & -2.6 & -2.6 & 68.9 & 0.0  \\
N$_1$ & 6 & -3.2 &  -3.3 &  0.3  & 0.1 & B$_1$ & 6 & -10.6 & -8.5 & -6.8 & -0.2 \\
B$_2$ & 3 & -0.6 &  -0.3  &  1.7 &  0.0 & N$_2$ & 3 & -0.5 & -0.3 & 1.5 & 0.0 \\
B$_3$  & 6 & 0.2 &  -0.9  &  1.5  &  0.3 & N$_3$ & 6 & -0.1 & -0.4 & 1.1 & 0.0  \\ \hline
B$_{A/B}$ & 2 &  -0.3 &  -0.3  &  0.6 & 0.0 & B$_{A/B}$ & 6 & -0.2 & -0.2 & 0.6 & 0.0 \\ \hline
 \end{tabular}
\end{center}
\end{table}

\begin{figure}[!ht]
\begin{center}
	\includegraphics[width=0.70\columnwidth]{figs/Fig4.pdf}
	\caption{ Spin densities and spin resonance signals of symmetric carbon tetramers. \textbf{a} and \textbf{b} depict the spin density of the C4N defect from top and side views. \textbf{c} depicts the spin density of the C4B defect from the top view. \textbf{d} Electron spin resonance signal of the C4N defect at $B = 0$ with no carbon nuclear spins in $^{10}$B (blue) and $^{11}$B (green) containing samples. \textbf{e} Electron spin resonance signal of the C4N defect when it includes four $^{13}$C nuclear spins. \textbf{f} Simulated electron spin resonance signal of the C4B defect with no carbon nuclear spins in $^{10}$B and $^{11}$B containing samples. \textbf{e} Spin resonance signal of the C4B defect including four $^{13}$C nuclear spins.}
	\label{fig:spin}  
\end{center}
\end{figure}

Finally, we present our result on the spin properties of the C4N and C4B defects. For the zero-field splitting (ZFS) parameter $D$ of the C4N defect we obtain 820~MHz and 840~MHz with CASSCF-NEVPT2 and periodic hybrid-DFT with spin contamination error correction\cite{biktagirov_spin_2020}, respectively. The difference of the values indicate the error margin of our ZFS calculations. For the C4B defect we obtain a slightly reduced $D = 660$~MHz ZFS value on CASSCF-NEVPT2 level of theory.  Due to the D$_{3h}$ symmetry, the quantization axis of the defects is parallel to the $c$-axis and no $E$ splitting is observed in unstrained configurations. The spin density of the defects and the locations of the most relevant nuclear spins  are depicted in Fig.~\ref{fig:spin}a-c. The corresponding hyperfine coupling parameters, obtained with periodic hybrid-DFT in a bulk model, are provided in Table~\ref{tab:HFdata} and in section S3.6 of the Supplementary Information. Most notably, the spin density localizes mainly on the carbon atoms and only secondary localization can be found on the first neighbor nitrogen atoms. Accordingly, the carbon hyperfine parameters an order of magnitude larger than the rest of the coupling parameters. Using the theoretical spin coupling parameters, the predicted electron spin resonance (ESR) spectra at zero magnetic field for different isotope abundances are depicted in Fig.~\ref{fig:spin}d-g. In natural abundance, $^{13}$C nuclear spin can be found only with 1.07\% probability, thus in most configurations no carbon spins are included in the structures. When completely ignoring carbon nuclear spins, we obtain a narrow homogeneous ESR signal, where the line width is determined by the boron hyperfine coupling tensor and the boron isotope abundance. The full widths of the resonance peaks at half maximum are 12~MHz (8~MHz) and 31~MHz (24~MHz) for the C4N and the C4B defects in ${^{10}}$BN (${^{11}}$BN) sample. The narrow ESR line width may make the C4N and C4B defects attractive candidates for sensing. When carbon nuclear spins are included with 100\% $^{13}C$ abundance, we observe a characteristic 8 peak hyperfine structure with 95.0 and 57.6 MHz splittings for the C4N defect, see Fig.~\ref{fig:spin}e, and a characteristic 5 broad peak structure with $\sim$60~MHz splitting. $^{13}$C enriched carbon contamination and observation of the carbon related hypefine structure can be used to unambiguously identify the C4N and C4B defects. 




\section*{Discussions}

\bigbreak
\noindent \textit{Sensing applications}

Recently, few nanometers thick sensing foils has been developed by using spin qubit-containing hBN sheets.\cite{tetienne_quantum_2021,healey_quantum_2022,kumar_magnetic_2022} In order to maximally utilize the layered structure of hBN and to further boost sensitivity and spatial resolution in such experiments, single sheet hBN flakes with stable point defect spin qubits are needed. We propose here the charge neutral symmetric carbon tetramers for implementing such chemically stable quantum bits in hBN. Due to the bond-formations, the carbon tetramers can give rise to atomic thin sensing foils that may be stable even at ambient conditions. Furthermore, the narrow electron spin resonance linewidth of the nitrogen centered carbon tetramer may lead to high sensitivity.

Robust spin qubits could also be highly beneficial for hBN-based dynamic nuclear polarization.\cite{gao_nuclear_2022} We show here that carbon tetramers can be spin polarized by optical illumination, leading to a spin polarization sources that may couple to nuclear spins outside the hBN host. Since the symmetric carbon tetramers can be stable at the surface in close proximity to other molecules, e.g. the NMR agent tetramethylsilane, efficient polarization transfer could be achieved between the quantum defects and the molecules on the surface. High-temperature and low-magnetic nuclear hyperpolarization mechanism are long-time sought for for boosting the sensitivity of conventional NMR and MRI applications.

Stable spin qubits are key components of both of the above-mentioned applications. Since, the symmetric carbon tetramers improve on the state of the art in this respect, they could lead to advances in nanoscale sensing.


\bigbreak
\noindent \textit{Fabrication}

Previous computational studies\cite{marek_thermodynamics_2022} have demonstrated that carbon tetramers exhibit low formation energy in N-rich samples that may imply the formation of these complex defects in observable concentrations ($\sim 10^{14}$ cm$^{-3}$) in hBN. After growth treatments, however, may give rise to concentrations well exceeding the thermal equilibrium values and enable on-demand fabrication in few layer samples. In this respect, carbon implantation maybe of high importance that can create carbon related point defect quantum bits in hBN.\cite{mendelson_identifying_2021} Furthermore, scanning transmission electron microscope (STEM) is also of high potential for sub-nanometer precision creation of defects.\cite{park_atomically_2021} STEM mapping combined with a subsequent annealing steps allows the creation of the carbon complexes in single layer hBN samples.\cite{park_atomically_2021} These processes open up new directions for tailored fabrication of carbon clusters, including the symmetric C4N defect, in hBN. 


\bigbreak
\noindent \textit{Summary}

In this paper we comprehensively studied the neutral nitrogen and boron site centered symmetric carbon tetramers in hBN. By using complementary first principles methods, we predicted the electronic structure as well as the optical and spin properties of the defects. We showed that symmetric carbon tetrameters exhibit high spin ground state, optical transition in the visible range, singlet shelving states, and strain dependent non-radiative decay channels. We concluded that the C4N and C4B defects can implement chemically stable spin qubit that may remain functional on the surface even at ambient conditions, in contrast to the established vacancy related quantum defects. Furthermore, strain can be used to tailor the spin contrast of the defects, which is a new feature for spin quantum bits in hBN. 


\section*{Methods}

In this study, we carried out first principles calculations of both periodic, supercell model and molecule model of the symmetric C4 defects in hBN. The two approaches allowed independent computational studies on different levels of theory, ensuring the reliability of our results.
Below we briefly summarize the computational protocols; the detailed description of the methodology (including sample input files for the sake of reproducibility) can be found in the Supplementary Information.

\begin{figure}[!ht]
\begin{center}
	\includegraphics[width=0.80\columnwidth]{figs/Fig5.pdf}
	\caption{ Illustration of supercell and molecular models used in the present study for the case of C4N defect.}
	\label{fig:models}  
\end{center}
\end{figure}

\bigbreak
\noindent \textit{  Computations with supercell models}

Kohn-Sham density functional theory was employed to study the periodic models of carbon tetramer defects with VASP package\cite{VASP}. Here, we considered a plane wave basis set of 450 eV, PAW\cite{PAW} core potentials, and HSE06 hybrid exchange-correlation functional\cite{HSE03} with 0.32 exact exchange fraction\cite{WestonPhysRevB2018}. The defects were modeled in 162-atom (monolayer) and 768-atom (bulk) supercells (Fig. \ref{fig:models}, top). To account for van der Waals interaction, we included the D3 correction by Grimme et al\cite{DFT-D3-Grimme}. The ZPL energies were calculated from the energy difference between ground and excited states, where the excited states were obtained using spin-conserved constrained DFT method\cite{dSCF} without enforcing symmetry restrictions.


\bigbreak
\noindent \textit{ Computations with molecule models }

As a reasonable compromise between cost and accuracy, two hBN flakes of different size were used to investigate the C4 defects as a finite molecule. Electronic energies and properties (e.g. transition dipole moments) were computed on a smaller model (Fig. \ref{fig:models}, bottom left), while phononic effects were calculated on a larger system (Fig. \ref{fig:models}, bottom right). Test calculations justifying the choice of flake size can be found in the Supplementary Information. All calculations were carried out using the ORCA 5.0.3. program\cite{neese2022software}.

 Geometry optimizations and vibrational analyses were performed using density functional theory, at PBE0/cc-pVDZ level\cite{pbe0func,cc-pVDZ} with D3(BJ) dispersion correction\cite{d3bj}. In the case of excited states, time-dependent density functional theory (TD-DFT)\cite{td-dft} was requested with 10 roots. Singlet excited states were generated from the triplet ground state by spin-flip\cite{spin-flip}.

Single-point energies and electronic properties were determined at CASSCF/cc-pVTZ level of theory\cite{casscf,cc-pVDZ} while dynamic correlation energy was taken into account by second-order N-electron valence perturbation theory (NEVPT2)\cite{nevpt2}. 
The active orbital space for CASSCF was constructed based on both time-dependent density functional theory results obtained by ORCA and density matrix renormalization group (DMRG)\cite{reiher-dmrg,legeza-dmrg} calculations using the Budapest-DMRG package\cite{budapest_qcdmrg}. 

 
\section*{Acknowledgments} 

This research was  supported by the National Research, Development, and Innovation Office of Hungary  within the Quantum Information National Laboratory of Hungary (Grant No. 2022-2.1.1-NL-2022-00004) and within grants FK 135496 and FK 145395.
V.I. also appreciates support from the Knut and Alice Wallenberg Foundation through WBSQD2 project (Grant No.\ 2018.0071). O.L. was supported by the Center for Scalable and Predictive methods for Excitation and Correlated phenomena (SPEC), funded as part of the Computational Chemical Sciences Program by the U.S. Department of Energy (DOE), Office of Science, Office of Basic Energy Sciences, Division of Chemical Sciences, Geosciences, and Biosciences at Pacific Northwest National Laboratory.
The calculations were performed on resources provided by the Swedish National Infrastructure for Computing (SNIC) at the National Supercomputer Centre (NSC).
We acknowledge KIF\"U for awarding us access to computational resource based in Hungary.
Z.B. and T.S. would like to thank the University of Alabama and the Office of Information Technology for providing high performance computing resources and support that have contributed to these research results. 

\section*{Data availability}

The data that support the findings of this study are available from the authors upon reasonable request.

\section*{Author contributions}

Z.B., R.B, and A.G. carried out the first principles calculations, O.L. developed the DMRG program package. V.I., Z.B., and G.B. wrote the manuscript with inputs from all coauthors. The work was supervised by V.I. and G.B. 

\section*{Competing interests}

The authors declare no competing interests.


\section*{References}
\bigbreak
\input{main_ref.bbl}

\end{document}