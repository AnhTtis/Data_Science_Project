
\begin{figure}[H]
\centering
\includegraphics[width=\textwidth]{feature_tracker_uq/kitti_figs/kitti_scene.png}
\includegraphics[width=0.48\textwidth]{feature_tracker_uq/kitti_figs/kitti_track_throwout.pdf}
\caption{\textbf{KITTI Dataset: We will throw out the 10\% of tracks from each scene with the most error.} The bottom figure plots the histogram density of the maximum L2 error of all feature tracks of a single scene in log scale. The corresponding scene is pictured on top. The outlier errors are caused by noisy data in the depth image collection process.}
\label{fig:kitti_error_throwout}
\end{figure}


\begin{figure}[H]
    \centering
    \includegraphics[width=4in]{feature_tracker_uq/kitti_figs/kitti_1.00_feature_lifetime.pdf}
    \caption{\textbf{KITTI Dataset: Most features live for less than five frames.} The distribution of feature lifetimes is plotted as a log-scale histogram for both the Lucas-Kanade and Correspondence-Based Tracker at nominal speed. The Lucas-Kanade Tracker produces a long tail of features with longer lifetimes. Features with long-lifetimes are those far away from the car's camera, in the center of the image.}
    \label{fig:kitti_feature_lifetime}
\end{figure}


\begin{figure}[H]
    \centering
    \subfigure[Lucas-Kanade]{\includegraphics[width=0.48\textwidth]{feature_tracker_uq/kitti_figs/kitti_avg_feats_LK.pdf}}
    \subfigure[Correspondence]{\includegraphics[width=0.48\textwidth]{feature_tracker_uq/kitti_figs/kitti_avg_feats_match.pdf}}
    \caption{Feature lifetime is plotted on the horizontal axis. The vertical axis, in log scale, shows the number of features in all 28 scenes that were tracked for at least that many frames. In both plots the number of features drops very fast. Note that for speeds greater than 8.00, the Lucas-Kanade tracker fails to match any features past one frame. \textbf{In subsequent analyses on the KITTI dataset, we only compute mean errors and covariances at timesteps with at least 100 features. We also only analyze speeds 1.00, 2.00, and 3.00 because higher speeds would otherwise be limited to $\le$ two timesteps.}}
    \label{fig:kitti_avg_feats}
\end{figure}


\begin{figure}[H]
    \centering
    \subfigure[Lucas-Kanade]{\includegraphics[width=0.48\textwidth]{feature_tracker_uq/kitti_figs/kitti_LK_percent_outlier.pdf}}
    \subfigure[Correspondence]{\includegraphics[width=0.48\textwidth]{feature_tracker_uq/kitti_figs/kitti_match_percent_outlier.pdf}}
    \caption{\textbf{KITTI Dataset: Outlier ratios are above 40 percent.} Outlier ratios per frame are shown as box-and-whisker plots for the Lucas-Kanade tracker on the left and the Correspondence tracker on the right. For the Lucas-Kanade tracker, outlier ratios remain a constant 40 percent. For the correspondence tracker, outlier ratios are higher, around 50 percent, for lower speeds and then decrease. The decreases exists not because of improvements in feature matching with higher speeds, but because fewer features are matched at all.}
    \label{fig:kitti_outlier_ratio}
\end{figure}


\begin{figure}[H]
    \centering
    \subfigure[$\nu(t)$, Horizontal Coordinate]{\includegraphics[width=0.48\textwidth]{feature_tracker_uq/kitti_figs/kitti_speed1.00_x.pdf}}
    \subfigure[$\nu(t)$, Vertical Coordinate]{\includegraphics[width=0.48\textwidth]{feature_tracker_uq/kitti_figs/kitti_speed1.00_y.pdf}}
    \caption{\textbf{KITTI Dataset: The zero-mean assumption approximately holds for both the Lucas-Kanade Tracker and the Correspondence Tracker at nominal speed.} Lines shown are horizontal (left) and vertical (right) coordinates of mean error $\nu(t)$ calculated using tracks averaged over all scenes; calculation is cutoff at 24 frames for the Lucas-Kanade Tracker and 6 frames for the Correspondence Tracker so that averages can be computed with at least 100 features. Mean errors remain at roughly zero.}
    \label{fig:kitti_1.00_meanerror}
\end{figure}


\begin{figure}[H]
    \centering
    \subfigure[$\eta(t)$, Horizontal Coordinate]{\includegraphics[width=0.48\textwidth]{feature_tracker_uq/kitti_figs/kitti_speed1.00_x_abs.pdf}}
    \subfigure[$\eta(t)$, Vertical Coordinate]{\includegraphics[width=0.48\textwidth]{feature_tracker_uq/kitti_figs/kitti_speed1.00_y_abs.pdf}}
    \subfigure[$\Phi(t)$, Horizontal Coordinate]{\includegraphics[width=0.48\textwidth]{feature_tracker_uq/kitti_figs/kitti_speed1.00_x_cov.pdf}}
    \subfigure[$\Phi(t)$, Vertical Coordinate]{\includegraphics[width=0.48\textwidth]{feature_tracker_uq/kitti_figs/kitti_speed1.00_y_cov.pdf}}  
    \caption{\textbf{KITTI Dataset: The Lucas-Kanade Tracker drifts considerably more than the Correspondence Tracker in all directions.} Lines shown are horizontal (left column) and vertical (right column) coordinates of mean absolute error $\eta(t)$ (top row) and covariance $\Phi(t)$ (bottom row) calculated using tracks averaged over all scenes; calculation is cutoff at 24 frames for Lucas-Kanade Tracker and 6 frames for the Correspondence Tracker so that averages can be computed with at least 100 features. Both mean absolute error and covariance are roughly constant when using the Correspondence Tracker. On the other hand, both drift slightly upwards and then level off when using the Lucas-Kanade Tracker.}
    \label{fig:kitti_1.00_error_cov}
\end{figure}


\begin{figure}[H]
    \centering
    \subfigure[$\nu(t)$, Horizontal Coordinate]{
        \includegraphics[width=0.48\textwidth]{feature_tracker_uq/kitti_figs/kitti_LK_x.pdf}
        \includegraphics[width=0.48\textwidth]{feature_tracker_uq/kitti_figs/kitti_LK_x_boxplot.pdf}
    }
    \subfigure[$\nu(t)$, Vertical Coordinate]{
        \includegraphics[width=0.48\textwidth]{feature_tracker_uq/kitti_figs/kitti_LK_y.pdf}
        \includegraphics[width=0.48\textwidth]{feature_tracker_uq/kitti_figs/kitti_LK_y_boxplot.pdf}
    }
    \caption{\textbf{KITTI Dataset: Mean tracking errors increase with speed when using the Lucas-Kanade Tracker.}
    The left column contains plots of the horizontal (top row) and vertical (bottom row) components of the mean tracking error $\nu(t)$ at each timestep $t$ after initial feature detection at multiple speeds. Each dot corresponds to a processed frame; lines for higher speeds contain data from fewer frames and therefore show fewer dots. The right column plots the ordinate values of each line for $t>0$ in the left figures as a box plot: means are shown as green triangles and medians are shown as orange lines.
    The mean and median values of the horizontal and vertical coordinates of $\nu(t)$ increases by about two pixels when speed is increased from 2.00 to 3.00. There is no such increase in $\nu(t)$ when speed is increased from 1.00 to 2.00.} \label{fig:kitti_LK_meanerror}
\end{figure}


\begin{figure}[H]
    \centering
    \subfigure[$\eta(t)$, Horizontal Coordinate]{
        \includegraphics[width=0.48\textwidth]{feature_tracker_uq/kitti_figs/kitti_LK_x_abs.pdf}
        \includegraphics[width=0.48\textwidth]{feature_tracker_uq/kitti_figs/kitti_LK_x_abs_boxplot.pdf}
    }
    \subfigure[$\eta(t)$, Vertical Coordinate]{
        \includegraphics[width=0.48\textwidth]{feature_tracker_uq/kitti_figs/kitti_LK_y_abs.pdf}
        \includegraphics[width=0.48\textwidth]{feature_tracker_uq/kitti_figs/kitti_LK_y_abs_boxplot.pdf}
    } 
    \caption{\textbf{KITTI Dataset: Mean absolute errors increase with speed when using the Lucas-Kanade Tracker.}
    The left column contains plots of the horizontal (top row) and vertical (bottom row) components of the mean absolute error $\eta(t)$ at each timestep $t$ after initial feature detection at multiple speeds. Each dot corresponds to a processed frame; lines for higher speeds contain data from fewer frames and therefore show fewer dots. The right column plots the ordinate values of each line for $t>0$ in the left figures as a box plot: means are shown as green triangles and medians are shown as orange lines.
    The mean and median values of $\eta(t)$ jump when speed is increased from 2.00 to 3.00. Left column plots show that $\eta(t)$ is approximately unchanged when speed is increased from 1.00 to 2.00. Since the box plot for speed=1.00 contains more points at larger values of $t$ than the box plot for speed=2.00, the mean and median values in the box plot decrease.
    }
    \label{fig:kitti_LK_MAE}
\end{figure}


\begin{figure}[H]
    \centering
    \subfigure[$\Phi(t)$, Horizontal Coordinates]{
        \includegraphics[width=0.48\textwidth]{feature_tracker_uq/kitti_figs/kitti_LK_x_cov.pdf}
        \includegraphics[width=0.48\textwidth]{feature_tracker_uq/kitti_figs/kitti_LK_x_cov_boxplot.pdf}
    }
    \subfigure[$\Phi(t)$, Vertical Coordinates]{
        \includegraphics[width=0.48\textwidth]{feature_tracker_uq/kitti_figs/kitti_LK_y_cov.pdf}
        \includegraphics[width=0.48\textwidth]{feature_tracker_uq/kitti_figs/kitti_LK_y_cov_boxplot.pdf}
    }
    \caption{\textbf{KITTI Dataset: Covariances increase with speed when using the Lucas-Kanade Tracker.}
    The left column contains plots of the horizontal (top row) and vertical (bottom row) components of the covariance $\Phi(t)$ at each timestep $t$ after initial feature detection at multiple speeds. Each dot corresponds to a processed frame; lines for higher speeds contain data from fewer frames and therefore show fewer dots. The right column plots the ordinate values of each line for $t>0$ in the left figures as a box plot: means are shown as green triangles and medians are shown as orange lines.
    We see a linear increase in covariance in the horizontal coordinate with speed. The increase in the vertical coordinate follows the same trend noted in Figures \ref{fig:kitti_LK_meanerror} and \ref{fig:kitti_LK_MAE}.
    }
    \label{fig:kitti_LK_cov}
\end{figure}


\begin{figure}[H]
    \centering
    \subfigure[$\nu(t)$, Horizontal Coordinate]{
        \includegraphics[width=0.48\textwidth]{feature_tracker_uq/kitti_figs/kitti_match_x.pdf}
        \includegraphics[width=0.48\textwidth]{feature_tracker_uq/kitti_figs/kitti_match_x_boxplot.pdf}
    }
    \subfigure[$\nu(t)$, Vertical Coordinate]{
        \includegraphics[width=0.48\textwidth]{feature_tracker_uq/kitti_figs/kitti_match_y.pdf}
        \includegraphics[width=0.48\textwidth]{feature_tracker_uq/kitti_figs/kitti_match_y_boxplot.pdf}
    }
    \caption{\textbf{KITTI Dataset: Mean errors are unaffected by speed when using the Correspondence Tracker.}
    The left column contains plots of the horizontal (top row) and vertical (bottom row) components of the mean tracking error $\nu(t)$ at each timestep $t$ after initial feature detection at multiple speeds. Each dot corresponds to a processed frame; lines for higher speeds contain data from fewer frames and therefore show fewer dots. The right column plots the ordinate values of each line for $t>0$ in the left figures as a box plot: means are shown as green triangles and medians are shown as orange lines. Compared to the results for the Lucas-Kanade Tracker in Figure \ref{fig:kitti_LK_meanerror}, mean errors do not change when speed is increased from 1.00 to 3.00.
    }
    \label{fig:kitti_match_meanerror}
\end{figure}


\begin{figure}[H]
    \centering
    \subfigure[$\eta(t)$, Horizontal Coordinate]{
        \includegraphics[width=0.48\textwidth]{feature_tracker_uq/kitti_figs/kitti_match_x_abs.pdf}
        \includegraphics[width=0.48\textwidth]{feature_tracker_uq/kitti_figs/kitti_match_x_abs_boxplot.pdf}
    }
    \subfigure[$\eta(t)$, Vertical Coordinate]{
        \includegraphics[width=0.48\textwidth]{feature_tracker_uq/kitti_figs/kitti_match_y_abs.pdf}
        \includegraphics[width=0.48\textwidth]{feature_tracker_uq/kitti_figs/kitti_match_y_abs_boxplot.pdf}
    }
    \caption{\textbf{KITTI Dataset: Mean absolute errors are unaffected by speed when using the Correspondence Tracker.}
    The left column contains plots of the horizontal (top row) and vertical (bottom row) components of the mean absolute error $\eta(t)$ at each timestep $t$ after initial feature detection at multiple speeds. Each dot corresponds to a processed frame; lines for higher speeds contain data from fewer frames and therefore show fewer dots. The right column plots the ordinate values of each line for $t>0$ in the left figures as a box plot: means are shown as green triangles and medians are shown as orange lines. Compared to the results for the Lucas-Kanade Tracker in Figure \ref{fig:kitti_LK_MAE}, mean errors do not change when speed is increased from 1.00 to 3.00.}
    \label{fig:kitti_match_MAE}
\end{figure}


\begin{figure}[H]
    \centering
    \subfigure[$\Phi(t)$, Horizontal Coordinates]{
        \includegraphics[width=0.48\textwidth]{feature_tracker_uq/kitti_figs/kitti_match_x_cov.pdf}
        \includegraphics[width=0.48\textwidth]{feature_tracker_uq/kitti_figs/kitti_match_x_cov_boxplot.pdf}
    }
    \subfigure[$\Phi(t)$, Vertical Coordinates]{
        \includegraphics[width=0.48\textwidth]{feature_tracker_uq/kitti_figs/kitti_match_y_cov.pdf}
        \includegraphics[width=0.48\textwidth]{feature_tracker_uq/kitti_figs/kitti_match_y_cov_boxplot.pdf}
    }
    \caption{\textbf{KITTI Dataset: Covariance is unaffected by speed when using the Correspondence Tracker.}  
    The left column contains plots of the horizontal (top row) and vertical (bottom row) components of the mean absolute error $\eta(t)$ at each timestep $t$ after initial feature detection at multiple speeds. Each dot corresponds to a processed frame; lines for higher speeds contain data from fewer frames and therefore show fewer dots. The right column plots the ordinate values of each line for $t>0$ in the left figures as a box plot: means are shown as green triangles and medians are shown as orange lines. 
    Compared to the results for the Lucas-Kanade Tracker in Figure \ref{fig:kitti_LK_cov}, covariances do not change when speed is increased from 1.00 to 2.00. Covariances show an increase of about 2 pixels when speed is increased from 2.00 to 3.00, however.
    }
    \label{fig:kitti_match_cov}
\end{figure}




