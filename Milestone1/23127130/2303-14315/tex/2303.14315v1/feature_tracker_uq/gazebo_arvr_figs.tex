
\begin{figure}[H]
\centering
\includegraphics[width=0.48\textwidth]{feature_tracker_uq/gazebo_linear_figs/gazebo_wall.png}
\includegraphics[width=0.48\textwidth]{feature_tracker_uq/gazebo_arvr_figs/gazebo_arvr_track_throwout.pdf}
\caption{\textbf{Gazebo AR/VR Dataset: We will throw out the 10\% of tracks from each scene with the most error.} The right figure plots the histogram density of the maximum L2 error of all feature tracks of a single scene in log scale. The corresponding scene is pictured on the left. The outlier errors are caused by poor depth association that is the result of depth images and AR/VR images not being collected synchronously in the Gazebo simulation environment.}
\label{fig:gazebo_arvr_error_throwout}
\end{figure}



\begin{figure}[H]
    \centering
    \includegraphics[width=4in]{feature_tracker_uq/gazebo_arvr_figs/gazebo_arvr_1.00_feature_lifetime.pdf}
    \caption{\textbf{Gazebo AR/VR Dataset: Most features live for less than 10 frames.} The distribution of feature lifetimes is plotted as a log-scale histogram for both the Lucas-Kanade and Correspondence Tracker at nominal speed. Many features live for less than 10 frames, especially when the Correspondence Tracker is used. However, Lucas-Kanade produces a long tail of features with longer lifetimes.}
    \label{fig:gazebo_arvr_feature_lifetime}
\end{figure}


\begin{figure}[H]
    \subfigure[Lucas-Kanade]{\includegraphics[width=0.48\textwidth]{feature_tracker_uq/gazebo_arvr_figs/gazebo_arvr_avg_feats_LK.pdf}}
    \subfigure[Correspondence]{\includegraphics[width=0.48\textwidth]{feature_tracker_uq/gazebo_arvr_figs/gazebo_arvr_avg_feats_match.pdf}}
    \caption{Feature Lifetime is plotted on the horizontal axis. The vertical axis, in log scale, shows the number of features in all 11 scenes that lived at least that long for every tested speed. The number of features drops very fast when the Correspondence Tracker is used. On the other hand, many features have a long lifetime when the Lucas-Kanade tracker is used, since the scene is persistent. \textbf{In subsequent analyses, we only compute mean errors and covariances at timesteps with at least 100 features on the Gazebo AR/VR Dataset.}}
    \label{fig:gazebo_arvr_avg_feats}
\end{figure}


\begin{figure}[H]
    \centering
    \subfigure[Lucas-Kanade]{\includegraphics[width=0.48\textwidth]{feature_tracker_uq/gazebo_arvr_figs/gazebo_arvr_LK_percent_outlier.pdf}}
    \subfigure[Correspondence]{\includegraphics[width=0.48\textwidth]{feature_tracker_uq/gazebo_arvr_figs/gazebo_arvr_match_percent_outlier.pdf}}
    \caption{\textbf{Gazebo AR/VR Dataset: Outlier Ratio increases with speed when using the Lucas-Kanade Tracker and are constant when using the Correspondence Tracker.} Outlier ratios are shown as box-and-whisker plots for tested speeds for the Lucas-Kanade tracker on the left and the Correspondence Tracker on the right. For lower speeds, the Lucas-Kanade tracker produces fewer outliers. Outlier ratios then increase with speed. On the other hand, the outlier ratio for the Correspondence Tracker remains constant, between 30 and 40 percent.}
    \label{fig:gazebo_arvr_outlier_ratio}
\end{figure}


\begin{figure}[H]
    \centering
    \subfigure[$\nu(t)$, Horizontal Coordinate]{\includegraphics[width=0.48\textwidth]{feature_tracker_uq/gazebo_arvr_figs/gazebo_arvr_speed1.00_x.pdf}}
    \subfigure[$\nu(t)$, Vertical Coordinate]{\includegraphics[width=0.48\textwidth]{feature_tracker_uq/gazebo_arvr_figs/gazebo_arvr_speed1.00_y.pdf}}
    \caption{\textbf{Gazebo AR/VR Dataset: At nominal speed, the Lucas-Kanade Tracker accumulates drift that changes with motion. The Correspondence Tracker's error is not zero-mean, but tracks do not live long enough to accumulate.} Lines shown are horizontal (left) and vertical (right) coordinates of mean error $\nu(t)$ calculated using tracks averaged over all scenes; calculation is cut off at 58 frames for the Lucas-Kanade Tracker and 247 frames for the Correspondence Tracker so that averages can be computed with at least 100 features.}
    \label{fig:gazebo_arvr_1.00_meanerror}
\end{figure}


\begin{figure}[H]
    \centering
    \subfigure[$\eta(t)$, Horizontal Coordinate]{\includegraphics[width=0.48\textwidth]{feature_tracker_uq/gazebo_arvr_figs/gazebo_arvr_speed1.00_x_abs.pdf}}
    \subfigure[$\eta(t)$, Vertical Coordinate]{\includegraphics[width=0.48\textwidth]{feature_tracker_uq/gazebo_arvr_figs/gazebo_arvr_speed1.00_y_abs.pdf}}   
    \subfigure[$\Phi(t)$, Horizontal Coordinate]{\includegraphics[width=0.48\textwidth]{feature_tracker_uq/gazebo_arvr_figs/gazebo_arvr_speed1.00_x_cov.pdf}}
    \subfigure[$\Phi(t)$, Vertical Coordinate]{\includegraphics[width=0.48\textwidth]{feature_tracker_uq/gazebo_arvr_figs/gazebo_arvr_speed1.00_y_cov.pdf}}      
    \caption{\textbf{Gazebo AR/VR Dataset: The Lucas-Kanade tracker drifts considerably more than the Correspondence Tracker.} Lines shown are horizontal (left column) and vertical (right column) coordinates of mean absolute error $\eta(t)$ (top row) and covariance $\Phi(t)$ (bottom row) calculated using tracks averaged over all scenes; calculation is cutoff at 247 frames for Lucas-Kanade Tracker and 17 frames for the Correspondence Tracker so that averages can be computed with at least 100 features. Both mean absolute error and covariance are roughly constant and small when using the Correspondence Tracker. On the other hand, the horizontal coordinate of $\eta(t)$ and $\Phi(t)$ drifts and changes with motion when using the Lucas-Kanade Tracker.}
    \label{fig:gazebo_arvr_1.00_error_cov}
\end{figure}


\begin{figure}[H]
    \centering
    \subfigure[$\nu(t)$, Horizontal Coordinate]{
        \includegraphics[width=0.48\textwidth]{feature_tracker_uq/gazebo_arvr_figs/gazebo_arvr_LK_x.pdf}
        \includegraphics[width=0.48\textwidth]{feature_tracker_uq/gazebo_arvr_figs/gazebo_arvr_LK_x_boxplot.pdf}
    }
    \subfigure[$\nu(t)$, Vertical Coordinate]{
        \includegraphics[width=0.48\textwidth]{feature_tracker_uq/gazebo_arvr_figs/gazebo_arvr_LK_y.pdf}
        \includegraphics[width=0.48\textwidth]{feature_tracker_uq/gazebo_arvr_figs/gazebo_arvr_LK_y_boxplot.pdf}
    }
    \caption{\textbf{Gazebo AR/VR Dataset: Mean tracking errors are not affected by speed when using the Lucas-Kanade Tracker.}
    The left column contains plots of the horizontal (top row) and vertical (bottom row) components of the mean tracking error $\nu(t)$ at each timestep $t$ after initial feature detection at multiple speeds. Each dot corresponds to a processed frame; lines for higher speeds contain data from fewer frames and therefore show fewer dots. The right column plots the ordinate values of each line for $t>0$ in the left figures as a box plot: means are shown as green triangles and medians are shown as orange lines.
    The box plots show that over time, mean and median errors are not much affected by speed in both the horizontal and vertical coordinates. This is also illustrated by the fact that in the left column, all lines are on top of one another. In the left plots, some of the lines at higher speeds (brown, pink, and gray) show less mean error than the lines at lower speeds, indicating that the number of frames, and not just absolute distance, also affects the total drift.
    }
    \label{fig:gazebo_arvr_LK_meanerror}
\end{figure}


\begin{figure}[H]
    \centering
    \subfigure[$\eta(t)$, Horizontal Coordinate]{
        \includegraphics[width=0.48\textwidth]{feature_tracker_uq/gazebo_arvr_figs/gazebo_arvr_LK_x_abs.pdf}
        \includegraphics[width=0.48\textwidth]{feature_tracker_uq/gazebo_arvr_figs/gazebo_arvr_LK_x_abs_boxplot.pdf}
    }
    \subfigure[$\eta(t)$, Vertical Coordinate]{
        \includegraphics[width=0.48\textwidth]{feature_tracker_uq/gazebo_arvr_figs/gazebo_arvr_LK_y_abs.pdf}
        \includegraphics[width=0.48\textwidth]{feature_tracker_uq/gazebo_arvr_figs/gazebo_arvr_LK_y_abs_boxplot.pdf}
    }
    \caption{\textbf{Gazebo AR/VR Dataset: Mean absolute errors are not affected by speed when using the Lucas-Kanade Tracker.}
    The left column contains plots of the horizontal (top row) and vertical (bottom row) components of the mean absolute error $\eta(t)$ at each timestep $t$ after initial feature detection at multiple speeds. Each dot corresponds to a processed frame; lines for higher speeds contain data from fewer frames and therefore show fewer dots. The right column plots the ordinate values of each line for $t>0$ in the left figures as a box plot: means are shown as green triangles and medians are shown as orange lines.
    The left-column plots show that that lines of $\eta(t)$ for different speeds are largely on top of one another. Some of the lines at higher speeds (brown, pink, and gray) have lower mean absolute errors than the lines at lower speeds, indicating that the number of frames, and not just absolute distance, also affects the total drift.
    }
    \label{fig:gazebo_arvr_LK_MAE}
\end{figure}



\begin{figure}[H]
    \centering
    \subfigure[$\Phi(t)$, Horizontal Coordinate]{
        \includegraphics[width=0.48\textwidth]{feature_tracker_uq/gazebo_arvr_figs/gazebo_arvr_LK_x_cov.pdf}
        \includegraphics[width=0.48\textwidth]{feature_tracker_uq/gazebo_arvr_figs/gazebo_arvr_LK_x_cov_boxplot.pdf}
    }
    \subfigure[$\Phi(t)$, Vertical Coordinate]{
        \includegraphics[width=0.48\textwidth]{feature_tracker_uq/gazebo_arvr_figs/gazebo_arvr_LK_y_cov.pdf}
        \includegraphics[width=0.48\textwidth]{feature_tracker_uq/gazebo_arvr_figs/gazebo_arvr_LK_y_cov_boxplot.pdf}
    }
    \caption{\textbf{Gazebo AR/VR Dataset: Covariance is not affected by speed when using the Lucas-Kanade Tracker.}
    The left column contains plots of the horizontal (top row) and vertical (bottom row) components of the covariance $\Phi(t)$ at each timestep $t$ after initial feature detection at multiple speeds. Each dot corresponds to a processed frame; lines for higher speeds contain data from fewer frames and therefore show fewer dots. The right column plots the ordinate values of each line for $t>0$ in the left figures as a box plot: means are shown as green triangles and medians are shown as orange lines.
    The left-column plots show that that lines of $\eta(t)$ for different speeds are largely on top of one another. The right-column plots show that the corresponding box plots remain similar until speed is increased to 10.00. Then, variation in $\Phi(t)$ (i.e., the height of the boxes) shrinks because there are fewer points over time. The location of the green triangles in the box plots also changes because there are fewer points with low values of $t$ and low-errors contributing to it. Some of the lines at higher speeds (brown, pink, and gray) show less covariance than the lines at lower speeds, indicating that the number of frames, and not just absolute distance, also affects the total drift.
    }
    \label{fig:gazebo_arvr_LK_cov}
\end{figure}




\begin{figure}[H]
    \centering
    \subfigure[$\nu(t)$, Horizontal Coordinate]{
        \includegraphics[width=0.48\textwidth]{feature_tracker_uq/gazebo_arvr_figs/gazebo_arvr_match_x.pdf}
        \includegraphics[width=0.48\textwidth]{feature_tracker_uq/gazebo_arvr_figs/gazebo_arvr_match_x_boxplot.pdf}
    }
    \subfigure[$\nu(t)$, Vertical Coordinate]{
        \includegraphics[width=0.48\textwidth]{feature_tracker_uq/gazebo_arvr_figs/gazebo_arvr_match_y.pdf}
        \includegraphics[width=0.48\textwidth]{feature_tracker_uq/gazebo_arvr_figs/gazebo_arvr_match_y_boxplot.pdf}
    }
    \caption{\textbf{Gazebo AR/VR Dataset: Mean errors are unaffected by speed when using the Correspondence Tracker until tracking failure occurs.}
    The left column contains plots of the horizontal (top row) and vertical (bottom row) components of the mean tracking error $\nu(t)$ at each timestep $t$ after initial feature detection at multiple speeds. Each dot corresponds to a processed frame; lines for higher speeds contain data from fewer frames and therefore show fewer dots. The right column plots the ordinate values of each line for $t>0$ in the left figures as a box plot: means are shown as green triangles and medians are shown as orange lines. 
    In both the horizontal and vertical coordinates, mean errors over time are largely the same until speed=5.00. Then, tracking failures cause larger errors.
    }
    \label{fig:gazebo_arvr_match_meanerror}
\end{figure}



\begin{figure}[H]
    \centering
    \subfigure[$\eta(t)$, Horizontal Coordinate]{
        \includegraphics[width=0.48\textwidth]{feature_tracker_uq/gazebo_arvr_figs/gazebo_arvr_match_x_abs.pdf}
        \includegraphics[width=0.48\textwidth]{feature_tracker_uq/gazebo_arvr_figs/gazebo_arvr_match_x_abs_boxplot.pdf}
    }
    \subfigure[$\eta(t)$, Vertical Coordinate]{
        \includegraphics[width=0.48\textwidth]{feature_tracker_uq/gazebo_arvr_figs/gazebo_arvr_match_y_abs.pdf}
        \includegraphics[width=0.48\textwidth]{feature_tracker_uq/gazebo_arvr_figs/gazebo_arvr_match_y_abs_boxplot.pdf}
    }
    \caption{\textbf{Gazebo AR/VR Dataset: Mean absolute errors increase with speed when using the Correspondence Tracker.}
    The left column contains plots of the horizontal (top row) and vertical (bottom row) components of the mean absolute error $\eta(t)$ at each timestep $t$ after initial feature detection at multiple speeds. Each dot corresponds to a processed frame; lines for higher speeds contain data from fewer frames and therefore show fewer dots. The right column plots the ordinate values of each line for $t>0$ in the left figures as a box plot: means are shown as green triangles and medians are shown as orange lines. For both the horizontal and vertical coordinates, there is a steady rise in the mean absolute error in the right column plots.
    }
    \label{fig:gazebo_arvr_match_MAE}
\end{figure}



\begin{figure}[H]
    \centering
    \subfigure[$\Phi(t)$, Horizontal Coordinate]{
        \includegraphics[width=0.48\textwidth]{feature_tracker_uq/gazebo_arvr_figs/gazebo_arvr_match_x_cov.pdf}
        \includegraphics[width=0.48\textwidth]{feature_tracker_uq/gazebo_arvr_figs/gazebo_arvr_match_x_cov_boxplot.pdf}
    }
    \subfigure[$\Phi(t)$, Vertical Coordinate]{
        \includegraphics[width=0.48\textwidth]{feature_tracker_uq/gazebo_arvr_figs/gazebo_arvr_match_y_cov.pdf}
        \includegraphics[width=0.48\textwidth]{feature_tracker_uq/gazebo_arvr_figs/gazebo_arvr_match_y_cov_boxplot.pdf}
    }
    \caption{\textbf{Gazebo AR/VR Dataset: Covariance increases with speed when using the Correspondence Tracker.}
    The left column contains plots of the horizontal (top row) and vertical (bottom row) components of the mean tracking error $\Phi(t)$ at each timestep $t$ after initial feature detection at multiple speeds. Each dot corresponds to a processed frame; lines for higher speeds contain data from fewer frames and therefore show fewer dots. The right column plots the ordinate values of each line for $t>0$ in the left figures as a box plot: means are shown as green triangles and medians are shown as orange lines. For both the horizontal and vertical coordinates, there is a steady rise in the covariance in the right column plots.
    }
    \label{fig:gazebo_arvr_match_cov}
\end{figure}

