

\begin{figure}[H]
\centering
\includegraphics[width=0.48\textwidth]{feature_tracker_uq/dtu_figs/dtu_scene_001_keyframe.png}
\includegraphics[width=0.48\textwidth]{feature_tracker_uq/dtu_figs/dtu_track_throwout.pdf}
\caption{\textbf{DTU Point Features Dataset: We will throw out the 10\% of tracks with the most error from each scene.} The right figure plots the histogram density of all feature tracks' maximum L2 error in log scale. The corresponding scene is pictured on the left. Outliers in the blue histogram are caused by noisy depth measurements and the imperfect association of features with laser scan points.}
\label{fig:dtu_error_throwout}
\end{figure}


\begin{figure}[H]
    \centering
    \includegraphics[width=4.00in]{feature_tracker_uq/dtu_figs/dtu_diffuse_1.00_feature_lifetime_hist.pdf}
    \caption{\textbf{DTU Point Features Dataset: Feature lifetimes generated by the Lucas-Kande Tracker is a long tailed distribution.} The histograms above plot feature lifetime density in log scale for scenes with diffuse lighting and no skipped frames (speed=1.00) for the Lucas-Kanade (blue) and Correspondence (orange) Trackers. There is a long tail of tracks with longer lifetimes when we use a sparse optical flow rather than correspondences.}
    \label{fig:dtu_track_lifetime}
\end{figure}


\begin{figure}[H]
    \centering
    \subfigure[Lucas-Kanade]{\includegraphics[width=0.48\textwidth]{feature_tracker_uq/dtu_figs/dtu_lk_diffuse_variable_speed_percent_outlier.pdf}}
    \subfigure[Correspondence]{\includegraphics[width=0.48\textwidth]{feature_tracker_uq/dtu_figs/dtu_match_diffuse_variable_speed_percent_outlier.pdf}}
    \caption{\textbf{DTU Point Features Dataset: Outlier Ratio Depends on Speed.} In the box-and-whisker plots above, the orange line is the median, the green triangle is the mean, and the box extends from the first to the third quartiles. The whiskers extend up to 1.5x the length of the boxes. Outlier ratios increase with speed for all tested feature trackers to a point, and then falls slightly. Each box-and-whisker is computed using features from all 60 scenes, one tracker, and one speed. Outlier ratios then decrease at higher speeds not because the tracker is more accurate, but because the percentage of features that fail to be tracked from frame to frame increases.}
    \label{fig:dtu_track_outliers_speed}
\end{figure}


\begin{figure}[H]
    \centering
    \subfigure[Lucas-Kanade]{
        \includegraphics[width=0.48\textwidth]{feature_tracker_uq/dtu_figs/dtu_LK_variable_light_speed1.00_percent_outlier_LR.pdf}
        \includegraphics[width=0.48\textwidth]{feature_tracker_uq/dtu_figs/dtu_LK_variable_light_speed1.00_percent_outlier_BF.pdf}
    }
    \subfigure[Correspondence]{
        \includegraphics[width=0.48\textwidth]{feature_tracker_uq/dtu_figs/dtu_match_variable_light_speed1.00_percent_outlier_LR.pdf}
        \includegraphics[width=0.48\textwidth]{feature_tracker_uq/dtu_figs/dtu_match_variable_light_speed1.00_percent_outlier_BF.pdf}
    }   
    \caption{\textbf{DTU Point Features Dataset: Outlier ratio does not depend on the existence of directional lighting.} In the box-and-whisker plots above, the orange line is the median, the green triangle is the mean, and the box extends from the first to the third quartiles. The whiskers extend up to 1.5x the length of the boxes. Each box-and-whisker plot is computed using features from all 60 scenes, one tracker, speed=1.00, and one of the lighting conditions in Figure \ref{fig:dtu_light_stage}. The distribution of outlier ratio is approximately the same for all lighting conditions.}
    \label{fig:dtu_track_outliers_lights}
\end{figure}


\begin{figure}[H]
    \centering
    \subfigure[Lucas-Kanade]{
        \includegraphics[width=0.48\textwidth]{feature_tracker_uq/dtu_figs/dtu_LK_lightsLR_feature_lifetime.pdf}
        \includegraphics[width=0.48\textwidth]{feature_tracker_uq/dtu_figs/dtu_LK_lightsBF_feature_lifetime.pdf}
    }
    \subfigure[Correspondence]{
        \includegraphics[width=0.48\textwidth]{feature_tracker_uq/dtu_figs/dtu_match_lightsLR_feature_lifetime.pdf}
        \includegraphics[width=0.48\textwidth]{feature_tracker_uq/dtu_figs/dtu_match_lightsBF_feature_lifetime.pdf}
    }
    \caption{\textbf{DTU Point Features Dataset: Feature lifetime does not depend on the existence of directional lighting.}  In the box-and-whisker plots above, the orange line is the median, the green triangle is the mean, and the box extends from the first to the third quartiles. The whiskers extend up to 1.5x the length of the boxes. Outlier ratios increase with speed for all tested feature trackers to a point, and then falls slightly. Each box-and-whisker is computed using features from all 60 scenes, one tracker, and one speed. The distribution of feature lifetime is approximately the same for all lighting conditions.}
    \label{fig:dtu_lighting_feature_lifetimes}
\end{figure}



\begin{figure}[H]
    \centering
    \subfigure[Lucas-Kanade]{\includegraphics[width=0.48\textwidth]{feature_tracker_uq/dtu_figs/dtu_LK_diffuse_avg_feats.pdf}}
    \subfigure[Correspondence]{\includegraphics[width=0.48\textwidth]{feature_tracker_uq/dtu_figs/dtu_match_diffuse_avg_feats.pdf}}
    \caption{Each curve shows the total number of tracked features at each timestep for the Lucas-Kanade Tracker (left) and the Correspondence Tracker (right) in log scale. Each dot on a curve is a frame in the sequence and each curve is computed using all features visible in the Key Frame under diffuse lighting and one speed. The number of features that can be used to compute mean $\mu(t)$ and covariance $\Sigma(t)$ declines quickly away from the Key Frame when using the Correspondence Tracker. When using the Lucas-Kanade Tracker, a slower speed means that more features are tracked for more frames. When using the Correspondence Tracker, the number of features tracked is dependent on the number of frames as well as the speed for the reasons noted in Section \ref{sec:feature_tracker_configuration}. The closer two frames are (i.e., the slower the speed), the fewer features are dropped between them. This is consistent with previously known results about the precision and recall of feature descriptors \cite{mikolajczyk_performance_2005, schonberger_comparative_eval_2017, WU2017150}. \textbf{We limit calculations of mean error $\mu(t)$, mean absolute error, $\kappa(t)$, and covariance $\Sigma(t)$ to timesteps that contain at least 100 features.}}
    \label{fig:dtu_active_features}
\end{figure}


\begin{figure}[H]
    \centering
    \subfigure[$\mu(t)$, Horizontal Coordinate]{\includegraphics[width=0.48\textwidth]{feature_tracker_uq/dtu_figs/dtu_diffuse_speed1.00_x.pdf}}
    \subfigure[$\mu(t)$, Vertical Coordinate]{\includegraphics[width=0.48\textwidth]{feature_tracker_uq/dtu_figs/dtu_diffuse_speed1.00_y.pdf}}
    \caption{\textbf{DTU Point Features Dataset: At nominal speed and with diffuse lighting, the tracker used has little effect on $\mu(t)$.} Lines shown are mean feature track errors $\mu(t)$ at each timestep $t$ calculated over all scenes. The blue lines are feature track errors calculated using the Lucas-Kanade Tracker and the orange lines are feature track errors calculated using the Correspondence Tracker. Lines are cut-off to timesteps where at least 100 features with 3D data are available (see Fig. \ref{fig:dtu_active_features}). The orange lines are on top of the blue lines, therefore the tracker used does not affect mean error.}
    \label{fig:dtu_diffuse_1.00_meanerror}
\end{figure}


\begin{figure}[H]
    \centering
   \subfigure[$\kappa(t)$, Horizontal Coordinate]{\includegraphics[width=0.48\textwidth]{feature_tracker_uq/dtu_figs/dtu_diffuse_speed1.00_x_abs.pdf}}
    \subfigure[$\kappa(t)$, Vertical Coordinate]{\includegraphics[width=0.48\textwidth]{feature_tracker_uq/dtu_figs/dtu_diffuse_speed1.00_y_abs.pdf}}
    \subfigure[$\Sigma(t)$, Horizontal Covariance]{\includegraphics[width=0.48\textwidth]{feature_tracker_uq/dtu_figs/dtu_diffuse_speed1.00_x_cov.pdf}}
    \subfigure[$\Sigma(t)$, Vertical Coordinate]{\includegraphics[width=0.48\textwidth]{feature_tracker_uq/dtu_figs/dtu_diffuse_speed1.00_y_cov.pdf}}
    \caption{\textbf{DTU Point Features Dataset: At nominal speed and under diffuse lighting, the tracker used does affect mean absolute error $\kappa(t)$ and covariance $\Sigma(t)$.} Lines shown are horizontal and vertical coordinates of $\kappa(t)$ (top row), and $\Sigma(t)$ (bottom row) calculated using all tracks from all scenes. Each dot corresponds to a single frame. Mean absolute error and covariance for the Correspondence Tracker are roughly constant with respect to time, while the same values for the Lucas-Kanade Tracker increases steadily with time away from the Key Frame.}
    \label{fig:dtu_diffuse_1.00_MAE_cov}
\end{figure}


\begin{figure}[H]
    \centering
    \subfigure[$\mu(t)$, Horizontal Coordinate]{
        \includegraphics[width=0.48\textwidth]{feature_tracker_uq/dtu_figs/dtu_lk_diffuse_variable_speed_x.pdf}
        \includegraphics[width=0.48\textwidth]{feature_tracker_uq/dtu_figs/dtu_lk_diffuse_variable_speed_x_boxplot.pdf}
    }
    \subfigure[$\mu(t)$, Vertical Coordinate]{
        \includegraphics[width=0.48\textwidth]{feature_tracker_uq/dtu_figs/dtu_lk_diffuse_variable_speed_y.pdf}
        \includegraphics[width=0.48\textwidth]{feature_tracker_uq/dtu_figs/dtu_lk_diffuse_variable_speed_y_boxplot.pdf}
    }
    \caption{\textbf{DTU Point Features Dataset: When using the Lucas-Kanade Tracker with diffuse lighting, speed affects mean error.} The left column contains plots of the horizontal (top row) and vertical (bottom row) coordinates of mean error $\mu(t)$ at each timestep and multiple speeds. Each dot corresponds to a processed frame; lines for higher speeds contain data from fewer frames and therefore show fewer dots. The right column plots the ordinate value of each line in the left figures as a box plot: means are shown as green triangles and medians are shown as orange lines. As speed is increased, the slope of the horizontal components of $\mu(t)$ in the left plots (eq. \eqref{eq:mean_error_at_time}) decreases and the height of each box in the right plot decreases, i.e. the absolute magnitude of $\mu(t)$ slighty decreases.  This trend indicates the existence of two speed-related components that affect $\mu(t)$: the difference between frames and the number of frames that have elapsed; the former has a much larger effect than the latter. The latter occurs because the exact point that the Lucas-Kanade Tracker tracks drifts with each frame. Fewer frames means that the tracked point has fewer opportunities to drift.}
    \label{fig:dtu_LK_mean_varyspeed}
\end{figure}



\begin{figure}[H]
    \centering
    \subfigure[$\kappa(t)$, Horizontal Coordinate]{
        \includegraphics[width=0.48\textwidth]{feature_tracker_uq/dtu_figs/dtu_lk_diffuse_variable_speed_x_abs.pdf}
        \includegraphics[width=0.48\textwidth]{feature_tracker_uq/dtu_figs/dtu_lk_diffuse_variable_speed_x_abs_boxplot.pdf}
    }
    \subfigure[$\kappa(t)$, Vertical Coordinate]{
        \includegraphics[width=0.48\textwidth]{feature_tracker_uq/dtu_figs/dtu_lk_diffuse_variable_speed_y_abs.pdf}
        \includegraphics[width=0.48\textwidth]{feature_tracker_uq/dtu_figs/dtu_lk_diffuse_variable_speed_y_abs_boxplot.pdf}
    } 
    \caption{\textbf{DTU Point Features Dataset: When using the Lucas-Kanade Tracker with diffuse lighting, speed affects mean absolute error.} The left column contains plots of the horizontal (top row) and vertical (bottom row) coordinates of mean absolute error $\kappa(t)$ at each timestep and multiple speeds. Each dot corresponds to a processed frame; lines for higher speeds contain data from fewer frames and therefore show fewer dots. The right column plots the ordinate value of each line in the left figures as a box plot: means are shown as green triangles and medians are shown as orange lines. As speed is increased, the mean absolute error at each timestep slightly decreases. This indicates the existence of two speed-related components that affect $\kappa(t)$: the difference between frames and the number of frames that have elapsed; the former has a much larger effect than the latter. The latter occurs because the exact point that the Lucas-Kanade Tracker tracks drifts with each frame. Fewer frames means that the tracked point has fewer opportunities to drift.}
    \label{fig:dtu_LK_MAE_varyspeed}
\end{figure}



\begin{figure}[H]
    \centering
   \subfigure[$\Sigma(t)$, Horizontal Covariance]{
        \includegraphics[width=0.48\textwidth]{feature_tracker_uq/dtu_figs/dtu_lk_diffuse_variable_speed_x_cov.pdf}
        \includegraphics[width=0.48\textwidth]{feature_tracker_uq/dtu_figs/dtu_lk_diffuse_variable_speed_x_cov_boxplot.pdf}
    }
    \subfigure[$\Sigma(t)$, Vertical Coordinate]{
        \includegraphics[width=0.48\textwidth]{feature_tracker_uq/dtu_figs/dtu_lk_diffuse_variable_speed_y_cov.pdf}
        \includegraphics[width=0.48\textwidth]{feature_tracker_uq/dtu_figs/dtu_lk_diffuse_variable_speed_y_cov_boxplot.pdf}
    }
    \caption{\textbf{DTU Point Features Dataset: When using the Lucas-Kanade Tracker with diffuse lighting, speed affects covariance.} The left column contains plots of the square root of the horizontal (top row) and vertical (bottom row) coordiantes of $\Sigma(t)$ at each timestep and multiple speeds. Each dot cooresponds to a processed frame; lines for higher speeds contain data from fewer frames and therefore show fewer dots. The right column plots the ordinate value of each line in the left figures as a box plot: means are shown as green triangles and medians are shown as orange lines. As speed is increased, the covariance of both the horizontal and vertical coordiantes slightly decreases; the lines in the left plot become slightly less steep and mean values of covariance get slightly smaller. This indicates the existence of two speed-related components to these statistics: the difference between frames and the number of frames that have elapsed; the former has a much larger effect than the latter. The latter occurs because the exact point that the Lucas-Kanade Tracker tracks drifts with each frame. Fewer frames means that the tracked point has fewer opportunities to drift.}
    \label{fig:dtu_LK_cov_varyspeed}
\end{figure}


\begin{figure}[H]
    \centering
    \subfigure[$\mu(t)$, Horizontal Coordinate]{
        \includegraphics[width=0.48\textwidth]{feature_tracker_uq/dtu_figs/dtu_match_diffuse_variable_speed_x.pdf}
        \includegraphics[width=0.48\textwidth]{feature_tracker_uq/dtu_figs/dtu_match_diffuse_variable_speed_x_boxplot.pdf}
    }
    \subfigure[$\mu(t)$, Vertical Coordinate]{
        \includegraphics[width=0.48\textwidth]{feature_tracker_uq/dtu_figs/dtu_match_diffuse_variable_speed_y.pdf}
        \includegraphics[width=0.48\textwidth]{feature_tracker_uq/dtu_figs/dtu_match_diffuse_variable_speed_y_boxplot.pdf}
    }
    \caption{\textbf{DTU Point Features Dataset: When using the Correspondence Tracker with diffuse lighting, mean error $\mu(t)$ is not affected by speed.}  The left column contains plots of the horizontal coordinate (top row) and vertical coordinate (bottom row) of $\mu(t)$ at each timestep and multiple speeds. Each dot corresponds to a processed frame; lines for higher speeds contain data from fewer frames and therefore show fewer dots. The right column plots the ordinate value of each line in the left figures as a box plot: means are shown as green triangles and medians are shown as orange lines. Both the line and box plots only contain timesteps that contain at least 100 tracked features (see Fig. \ref{fig:dtu_active_features}), leading to some asymmetry of the lines about the Key Frame, i.e. the line plot of the horizontal coordinate of $\mu(t)$ at speed=8.00 starts at timestep 0, but ends at timestep 40 even though the Key Frame is at timestep 24. As speed is increased, there is no change in both the horizontal and vertical coordinates, as all lines in the left column plots are on top of one another. The boxes in the box plots of the horizontal coordinate are taller for higher speeds because the time cutoff for those speeds is longer than for the lower speeds, allowing more error to appear in the tracked features.}
    \label{fig:dtu_match_diffuse_mean_error_varyspeed}
\end{figure}



\begin{figure}[H]
    \centering
    \subfigure[$\kappa(t)$, Horizontal Coordinate]{
        \includegraphics[width=0.48\textwidth]{feature_tracker_uq/dtu_figs/dtu_match_diffuse_variable_speed_x_abs.pdf}
        \includegraphics[width=0.48\textwidth]{feature_tracker_uq/dtu_figs/dtu_match_diffuse_variable_speed_x_abs_boxplot.pdf}
    }
    \subfigure[$\kappa(t)$, Vertical Coordinate]{
        \includegraphics[width=0.48\textwidth]{feature_tracker_uq/dtu_figs/dtu_match_diffuse_variable_speed_y_abs.pdf}
        \includegraphics[width=0.48\textwidth]{feature_tracker_uq/dtu_figs/dtu_match_diffuse_variable_speed_y_abs_boxplot.pdf}
    }
    \caption{\textbf{DTU Point Features Dataset: When using the Correspondence Tracker with diffuse lighting, mean absolute error $\kappa(t)$ increases in the horizontal direction, but not the vertical direction, as speed is increased.}
    The left column contains plots of the horizontal coordinate (top row) and vertical coordinate (bottom row) of $\kappa(t)$ at each timestep and multiple speeds. Each dot corresponds to a processed frame; lines for higher speeds contain data from fewer frames and therefore show fewer dots. The right column plots the ordinate value of each line in the left figures as a box plot: means are shown as green triangles and medians are shown as orange lines. Both the line and box plots only contain timesteps that contain at least 100 tracked features (see Fig. \ref{fig:dtu_active_features}). The mean and median values of the horizontal coordinate of $\kappa(t)$ increases as speed is increased.
    }
    \label{fig:dtu_match_diffuse_MAE_varyspeed}
\end{figure}


\begin{figure}[H]
    \centering
    \subfigure[$\Sigma(t)$, Horizontal Covariance]{
      \includegraphics[width=0.48\textwidth]{feature_tracker_uq/dtu_figs/dtu_match_diffuse_variable_speed_x_cov.pdf}
      \includegraphics[width=0.48\textwidth]{feature_tracker_uq/dtu_figs/dtu_match_diffuse_variable_speed_x_cov_boxplot.pdf}
    }
    \subfigure[$\Sigma(t)$, Vertical Coordinate]{
      \includegraphics[width=0.48\textwidth]{feature_tracker_uq/dtu_figs/dtu_match_diffuse_variable_speed_y_cov.pdf}
      \includegraphics[width=0.48\textwidth]{feature_tracker_uq/dtu_figs/dtu_match_diffuse_variable_speed_y_cov_boxplot.pdf}
    }
    \caption{\textbf{DTU Point Features Dataset: When using the Correspondence Tracker with diffuse lighting, covariance $\Sigma(t)$ increases in the horizontal direction, but not the vertical direction, as speed is increased.}  
    The left column contains plots of the square root of the horizontal coordinate (top row) and vertical coordinate (bottom row) of $\Sigma(t)$ at each timestep and multiple speeds. Each dot corresponds to a processed frame; lines for higher speeds contain data from fewer frames and therefore show fewer dots. The right column plots the ordinate value of each line in the left figures as a box plot: means are shown as green triangles and medians are shown as orange lines. Both the line and box plots only contain timesteps that contain at least 100 tracked features (see Fig. \ref{fig:dtu_active_features}). The mean and median values of the horizontal value of $\Sigma(t)$ as speed is increased.
    }
    \label{fig:dtu_match_diffuse_cov_varyspeed}
\end{figure}


\begin{figure}[H]
    \centering
    \subfigure[$\mu(t)$, Horizontal Coordinate]{
        \includegraphics[width=0.48\textwidth]{feature_tracker_uq/dtu_figs/dtu_LK_lightsLR_speed1.00_x.pdf}
        \includegraphics[width=0.48\textwidth]{feature_tracker_uq/dtu_figs/dtu_LK_lightsBF_speed1.00_x.pdf}
    }
    \subfigure[$\mu(t)$, Vertical Coordinate]{
        \includegraphics[width=0.48\textwidth]{feature_tracker_uq/dtu_figs/dtu_LK_lightsLR_speed1.00_y.pdf}
        \includegraphics[width=0.48\textwidth]{feature_tracker_uq/dtu_figs/dtu_LK_lightsBF_speed1.00_y.pdf}
    }
    \caption{\textbf{DTU Point Features Dataset: The existence of directional lighting does not change trends in mean error $\mu(t)$ when using the Lucas-Kande Tracker at nominal speed.} We compute $\mu(t)$ using diffuse lighting (black lines) and each of the directional lighting conditions listed in Figure \ref{fig:dtu_light_stage} using all tracks from all 60 scenes. Results for the horizontal coordinate are in the top row and results for the  vertical coordinate are in the bottom row. Timesteps are limited to those that contain at least 100 features. The variation of $\mu(t)$ due to the existence of directional lighting is at most 10 percent the size of the variation common to all plotted lines. The effect of directional lighting is relatively small because changes between adjacent frames are small whether or not the scene contains directional lighting.}
    \label{fig:dtu_lighting_mu_LK}
\end{figure}


\begin{figure}[H]
    \centering
    \subfigure[$\mu(t)$, Horizontal Coordinate]{
        \includegraphics[width=0.48\textwidth]{feature_tracker_uq/dtu_figs/dtu_match_lightsLR_speed1.00_x.pdf}
        \includegraphics[width=0.48\textwidth]{feature_tracker_uq/dtu_figs/dtu_match_lightsBF_speed1.00_x.pdf}
    }
    \subfigure[$\mu(t)$, Vertical Coordinate]{
        \includegraphics[width=0.48\textwidth]{feature_tracker_uq/dtu_figs/dtu_match_lightsLR_speed1.00_y.pdf}
        \includegraphics[width=0.48\textwidth]{feature_tracker_uq/dtu_figs/dtu_match_lightsBF_speed1.00_y.pdf}
    }
    \caption{\textbf{DTU Point Features Dataset: The existence of directional lighting does not change trends in mean error $\mu(t)$ when using the Correspondence Tracker at nominal speed.} We compute $\mu(t)$ using diffuse lighting (black lines) and each of the directional lighting conditions listed in Figure \ref{fig:dtu_light_stage} using all tracks from all 60 scenes. Results of the horizontal coordinate are shown in the top row and results for the vertical coordinate are shown in the bottom row. Timesteps are limited to those that contain at least 100 features. The variation of $\mu(t)$ due to the existence of directional lighting is at most 10 percent of the variation common to all plotted lines. The effects of directional lighting is relatively small because changes between adjacent frames are small whether or not the scene contains directional lighting.}
    \label{fig:dtu_lighting_mu_match}
\end{figure}


\begin{figure}[H]
    \centering
    \subfigure[$\kappa(t)$, Horizontal Coordinate]{
        \includegraphics[width=0.48\textwidth]{feature_tracker_uq/dtu_figs/dtu_LK_lightsLR_speed1.00_x_abs.pdf}
        \includegraphics[width=0.48\textwidth]{feature_tracker_uq/dtu_figs/dtu_LK_lightsBF_speed1.00_x_abs.pdf}
    }
    \subfigure[$\kappa(t)$, Vertical Coordinate]{
        \includegraphics[width=0.48\textwidth]{feature_tracker_uq/dtu_figs/dtu_LK_lightsLR_speed1.00_y_abs.pdf}
        \includegraphics[width=0.48\textwidth]{feature_tracker_uq/dtu_figs/dtu_LK_lightsBF_speed1.00_y_abs.pdf}
    }
    \caption{\textbf{DTU Point Features Dataset: The existence of directional lighting does not change trends in mean absolute error $\kappa(t)$ when using the Lucas-Kanade Tracker at nominal speed.} We compute $\mu(t)$ at each timestep using diffuse lighting (black lines) and each of the directional lighting conditions listed in Figure \ref{fig:dtu_light_stage} using all tracks from all 60 scenes. Timesteps are limited to those that contain at least 100 features. The variation of $\kappa(t)$ due to the existence of directional lighting is at most 10 percent of the variation common to all plotted lines. The effect of directional lighting is relatively small because changes between adjacent frames are small whether or not the scene contains directional lighting.}
    \label{fig:dtu_lighting_omega_LK}
\end{figure}


\begin{figure}[H]
    \centering
    \subfigure[$\kappa(t)$, Horizontal Coordinate]{
        \includegraphics[width=0.48\textwidth]{feature_tracker_uq/dtu_figs/dtu_match_lightsLR_speed1.00_x_abs.pdf}
        \includegraphics[width=0.48\textwidth]{feature_tracker_uq/dtu_figs/dtu_match_lightsBF_speed1.00_x_abs.pdf} 
    }
    \subfigure[$\kappa(t)$, Vertical Coordinate]{
        \includegraphics[width=0.48\textwidth]{feature_tracker_uq/dtu_figs/dtu_match_lightsLR_speed1.00_y_abs.pdf} 
        \includegraphics[width=0.48\textwidth]{feature_tracker_uq/dtu_figs/dtu_match_lightsBF_speed1.00_y_abs.pdf} 
    }
    \caption{\textbf{DTU Point Features Dataset: The existence of directional lighting does not change trends in mean absolute error $\kappa(t)$ when using the Correspondence Tracker at nominal speed.} We compute $\kappa(t)$ using diffuse lighting (black lines) and each of the directional lighting conditions listed in Figure \ref{fig:dtu_light_stage} using all tracks from all 60 scenes. Timesteps are limited to those that contain at least 100 features. The variation of $\kappa(t)$ due to the existence of directional lighting is at most 10 percent of the variation common to all plotted lines. The effect of directional lighting is relatively small because changes between adjacent frames are small whether or not the scene contains directional lighting.}
    \label{fig:dtu_lighting_omega_match}
\end{figure}



\begin{figure}[H]
    \centering
    \subfigure[$\Sigma(t)$, Horizontal Coordinate]{
        \includegraphics[width=0.48\textwidth]{feature_tracker_uq/dtu_figs/dtu_LK_lightsLR_speed1.00_x_cov.pdf}
        \includegraphics[width=0.48\textwidth]{feature_tracker_uq/dtu_figs/dtu_LK_lightsBF_speed1.00_x_cov.pdf}
    }
    \subfigure[$\Sigma(t)$, Vertical Coordinate]{
        \includegraphics[width=0.48\textwidth]{feature_tracker_uq/dtu_figs/dtu_LK_lightsLR_speed1.00_y_cov.pdf}
        \includegraphics[width=0.48\textwidth]{feature_tracker_uq/dtu_figs/dtu_LK_lightsBF_speed1.00_y_cov.pdf}
    }
    \caption{\textbf{DTU Point Features Dataset: The existence of directional lighting does not change trends in covariance $\Sigma(t)$ when using the Lucas-Kanade Tracker at nominal speed.} We compute $\Sigma(t)$ using diffuse lighting (black lines) and each of the directional lighting conditions listed in Figure \ref{fig:dtu_light_stage} using all tracks from all 60 scenes. Timesteps are limited to those that contain at least 100 features. 
    The variation of $\Sigma(t)$ due to the existence of directional lighting is at most 10 percent of the variation common to all plotted lines. The effect of directional lighting is relatively small because changes between adjacent frames are small whether or not the scene contains directional lighting. The blip in the bottom-right figure is due to one specific scene where the AGAST tracker finds very few features, causing a failure in tracking and outlier rejection, and then calculation of $\Sigma(t)$ downstream. }
    \label{fig:dtu_lighting_sigma_LK}
\end{figure}



\begin{figure}[H]
    \centering
    \subfigure[$\Sigma(t)$, Horizontal Coordinate]{
        \includegraphics[width=0.48\textwidth]{feature_tracker_uq/dtu_figs/dtu_match_lightsLR_speed1.00_x_cov.pdf}
        \includegraphics[width=0.48\textwidth]{feature_tracker_uq/dtu_figs/dtu_match_lightsBF_speed1.00_x_cov.pdf} 
    }
    \subfigure[$\Sigma(t)$, Vertical Coordinate]{
        \includegraphics[width=0.48\textwidth]{feature_tracker_uq/dtu_figs/dtu_match_lightsLR_speed1.00_y_cov.pdf} 
        \includegraphics[width=0.48\textwidth]{feature_tracker_uq/dtu_figs/dtu_match_lightsBF_speed1.00_y_cov.pdf} 
    }
    \caption{\textbf{DTU Point Features Dataset: The existence of directional lighting does not change trends in covariance $\Sigma(t)$ when using the Correspondence Tracker at nominal speed.} We compute $\Sigma(t)$ using diffuse lighting (black lines) and each of the directional lighting conditions listed in Figure \ref{fig:dtu_light_stage} using all tracks from all 60 scenes. Timesteps are limited to those that contain at least 100 features. The variation of $\Sigma(t)$ due to the existence of directional lighting is at most 10 percent of the variation common to all plotted lines. The effect of directional lighting is relatively small because changes between adjacent frames are small whether or not the scene contains directional lighting.}
    \label{fig:dtu_lighting_sigma_match}
\end{figure}


\begin{figure}[H]
    \centering
    \subfigure[Lucas-Kanade]{
        \includegraphics[width=0.48\textwidth]{feature_tracker_uq/dtu_figs/dtu_LK_sideways_x.pdf}
        \includegraphics[width=0.48\textwidth]{feature_tracker_uq/dtu_figs/dtu_LK_sideways_y.pdf}
    }
    \subfigure[Correspondence]{
        \includegraphics[width=0.48\textwidth]{feature_tracker_uq/dtu_figs/dtu_match_sideways_x.pdf} 
        \includegraphics[width=0.48\textwidth]{feature_tracker_uq/dtu_figs/dtu_match_sideways_y.pdf} 
    }
    \caption{\textbf{DTU Point Features Dataset: Mean errors are larger about the direction of motion for both the Lucas-Kanade and Correspondence Trackers.} In Figures \ref{fig:dtu_LK_mean_varyspeed}, \ref{fig:dtu_match_diffuse_mean_error_varyspeed}, \ref{fig:dtu_lighting_mu_LK}, and \ref{fig:dtu_lighting_mu_match}, the horizontal component (left column) of $\mu(t)$ was always larger than the vertical component (right column). When images are rotated 90 degrees counterclockwise (``sideways''), the trend is reversed. Errors shown above are computed for the Lucas-Kanade Tracker at nominal speed and in diffuse lighting.}
    \label{fig:dtu_mean_error_sideways}
\end{figure}


\begin{figure}[H]
    \centering
    \subfigure[$\kappa(t)$, Horizontal Coordinate]{\includegraphics[width=0.48\textwidth]{feature_tracker_uq/dtu_figs/dtu_LK_sideways_x_abs.pdf}}
    \subfigure[$\kappa(t)$, Vertical Coordinate]{\includegraphics[width=0.48\textwidth]{feature_tracker_uq/dtu_figs/dtu_LK_sideways_y_abs.pdf}}
    \caption{\textbf{DTU Point Features Dataset: Mean absolute errors are larger about the direction of motion when using the Lucas-Kanade Tracker.} In Figures \ref{fig:dtu_LK_mean_varyspeed}, \ref{fig:dtu_match_diffuse_mean_error_varyspeed}, \ref{fig:dtu_lighting_mu_LK}, and \ref{fig:dtu_lighting_mu_match}, the horizontal component of $\kappa(t)$ was always larger than the vertical component. When images are rotated 90 degress counterclockwise (``sideways''), the trend is reversed. Mean absolute errors shown above are computed for the Lucas-Kanade Tracker at nominal speed and in diffuse lighting. }
    \label{fig:dtu_abs_error_sideways}
\end{figure}

\begin{figure}[H]
    \centering
    \subfigure[$\kappa(t)$, Horizontal Coordinate]{\includegraphics[width=0.48\textwidth]{feature_tracker_uq/dtu_figs/dtu_match_sideways_x_abs.pdf}}
    \subfigure[$\kappa(t)$, Vertical Coordinate]{\includegraphics[width=0.48\textwidth]{feature_tracker_uq/dtu_figs/dtu_match_sideways_y_abs.pdf}}
    \caption{\textbf{DTU Point Features Dataset: The direction of motion does not affect mean absolute error when using the Correspondence Tracker.} In Figures \ref{fig:dtu_match_diffuse_MAE_varyspeed}, and \ref{fig:dtu_lighting_omega_match}, the difference between the horizontal and vertical components of $\kappa(t)$ was a fraction of the size of $\kappa(t)$ in both components. When images are rotated 90 degrees counterclockwise (``sideways''), the trend is the same. Errors shown above are computed for the Correspondence Tracker at nominal speed and in diffuse lighting. }
    \label{fig:dtu_match_abs_error_sideways}
\end{figure}


\begin{figure}[H]
    \centering
    \subfigure[$\Sigma(t)$, Horizontal Coordinate]{\includegraphics[width=0.48\textwidth]{feature_tracker_uq/dtu_figs/dtu_LK_sideways_x_cov.pdf}}
    \subfigure[$\Sigma(t)$, Vertical Coordinate]{\includegraphics[width=0.48\textwidth]{feature_tracker_uq/dtu_figs/dtu_LK_sideways_y_cov.pdf}}
    \caption{\textbf{DTU Point Features Dataset: Covariances are larger about the direction of motion when using the Lucas-Kanade Tracker.} In Figures \ref{fig:dtu_LK_cov_varyspeed}, \ref{fig:dtu_match_diffuse_cov_varyspeed}, \ref{fig:dtu_lighting_sigma_LK}, and \ref{fig:dtu_lighting_sigma_match}, the horizontal component of $\Sigma(t)$ was always larger than the vertical component. When images are rotated 90 degrees counterclockwise (``sideways"), the trend is reversed for both errors (top row) and covariance (bottom row). Errors above are computed for the Lucas-Kanade Tracker at nominal speed and in diffuse lighting.}
    \label{fig:dtu_error_sideways}
\end{figure}

\begin{figure}[H]
    \centering
    \subfigure[$\Sigma(t)$, Horizontal Coordinate]{\includegraphics[width=0.48\textwidth]{feature_tracker_uq/dtu_figs/dtu_match_sideways_x_cov.pdf}}
    \subfigure[$\Sigma(t)$, Vertical Coordinate]{\includegraphics[width=0.48\textwidth]{feature_tracker_uq/dtu_figs/dtu_match_sideways_y_cov.pdf}}
    \caption{\textbf{DTU Point Features Dataset: The direction of motion does not affect covariance when using the Correspondence Tracker.} In Figures \ref{fig:dtu_match_diffuse_cov_varyspeed}, and \ref{fig:dtu_lighting_sigma_match}, the difference between the horizontal and vertical components of $\Sigma(t)$ was a fraction of the size of $\Sigma(t)$ in both components. When images are rotated 90 degrees counterclockwise (``sideways''), the trend is the same. Errors shown above are computed for the Correspondence Tracker at nominal speed and in diffuse lighting. }
    \label{fig:dtu_match_cov_sideways}
\end{figure}


\begin{figure}[H]
    \centering
    \subfigure[Lucas-Kanade]{
        \includegraphics[width=0.48\textwidth]{feature_tracker_uq/dtu_figs/dtu_LK_variable_light_speed2.00_percent_outlier_LR.pdf}
        \includegraphics[width=0.48\textwidth]{feature_tracker_uq/dtu_figs/dtu_LK_variable_light_speed2.00_percent_outlier_BF.pdf}
    }
    \subfigure[Correspondence]{
        \includegraphics[width=0.48\textwidth]{feature_tracker_uq/dtu_figs/dtu_match_variable_light_speed2.00_percent_outlier_LR.pdf}
        \includegraphics[width=0.48\textwidth]{feature_tracker_uq/dtu_figs/dtu_match_variable_light_speed2.00_percent_outlier_BF.pdf}
    }
    \caption{\textbf{At twice nominal speed, the existence of directional lighting does not affect outlier ratio.} In the box-and-whisker plots above, the orange line is the median, the green triangle is the mean, and the box extends from the first to the third quartiles. The whiskers extend up to 1.5x the length of the boxes. Each box-and-whisker plot is computed using features from all 60 scenes, one tracker, speed=2.00, and one of the lighting conditions in Figure \ref{fig:dtu_light_stage}. The distribution of outlier ratio is approximately the same for all lighting conditions.}
    \label{fig:dtu_speed2.00_percent_outlier}
\end{figure}


\begin{figure}[H]
    \centering
    \subfigure[Lucas-Kanade]{
        \includegraphics[width=0.48\textwidth]{feature_tracker_uq/dtu_figs/dtu_LK_variable_light_speed4.00_percent_outlier_LR.pdf}
        \includegraphics[width=0.48\textwidth]{feature_tracker_uq/dtu_figs/dtu_LK_variable_light_speed4.00_percent_outlier_BF.pdf}
    }
    \subfigure[Correspondence]{
        \includegraphics[width=0.48\textwidth]{feature_tracker_uq/dtu_figs/dtu_match_variable_light_speed4.00_percent_outlier_LR.pdf}
        \includegraphics[width=0.48\textwidth]{feature_tracker_uq/dtu_figs/dtu_match_variable_light_speed4.00_percent_outlier_BF.pdf}
    }
    \caption{\textbf{At four times nominal speed, the existence of directional lighting does not affect outlier ratio.} In the box-and-whisker plots above, the orange line is the median, the green triangle is the mean, and the box extends from the first to the third quartiles. The whiskers extend up to 1.5x the length of the boxes. Each box-and-whisker plot is computed using features from all 60 scenes, one tracker, speed=4.00, and one of the lighting conditions in Figure \ref{fig:dtu_light_stage}. The distribution of outlier ratio is approximately the same for all lighting conditions.}
    \label{fig:dtu_speed4.00_percent_outlier}
\end{figure}

\begin{figure}[H]
    \centering
    \subfigure[Lucas-Kanade]{
        \includegraphics[width=0.48\textwidth]{feature_tracker_uq/dtu_figs/dtu_LK_variable_light_speed8.00_percent_outlier_LR.pdf}
        \includegraphics[width=0.48\textwidth]{feature_tracker_uq/dtu_figs/dtu_LK_variable_light_speed8.00_percent_outlier_BF.pdf}
    }
    \subfigure[Correspondence]{
        \includegraphics[width=0.48\textwidth]{feature_tracker_uq/dtu_figs/dtu_match_variable_light_speed8.00_percent_outlier_LR.pdf}
        \includegraphics[width=0.48\textwidth]{feature_tracker_uq/dtu_figs/dtu_match_variable_light_speed8.00_percent_outlier_BF.pdf}
    }
    \caption{\textbf{At eight times nominal speed, the existence of directional lighting does not affect outlier ratio.} In the box-and-whisker plots above, the orange line is the median, the green triangle is the mean, and the box extends from the first to the third quartiles. The whiskers extend up to 1.5x the length of the boxes. Each box-and-whisker plot is computed using features from all 60 scenes, one tracker, speed=8.00, and one of the lighting conditions in Figure \ref{fig:dtu_light_stage}. The distribution of outlier ratio is approximately the same for all lighting conditions.}
    \label{fig:dtu_speed8.00_percent_outlier}
\end{figure}

\begin{figure}[H]
    \centering
    \subfigure[Lucas-Kanade]{
        \includegraphics[width=0.48\textwidth]{feature_tracker_uq/dtu_figs/dtu_LK_variable_light_speed12.00_percent_outlier_LR.pdf}
        \includegraphics[width=0.48\textwidth]{feature_tracker_uq/dtu_figs/dtu_LK_variable_light_speed12.00_percent_outlier_BF.pdf}
    }
    \subfigure[Correspondence]{
        \includegraphics[width=0.48\textwidth]{feature_tracker_uq/dtu_figs/dtu_match_variable_light_speed12.00_percent_outlier_LR.pdf}
        \includegraphics[width=0.48\textwidth]{feature_tracker_uq/dtu_figs/dtu_match_variable_light_speed12.00_percent_outlier_BF.pdf}
    }
    \caption{\textbf{At twelve times nominal speed, the existence of directional lighting does not affect outlier ratio.} In the box-and-whisker plots above, the orange line is the median, the green triangle is the mean, and the box extends from the first to the third quartiles. The whiskers extend up to 1.5x the length of the boxes. Each box-and-whisker plot is computed using features from all 60 scenes, one tracker, speed=12.00, and one of the lighting conditions in Figure \ref{fig:dtu_light_stage}. The distribution of outlier ratio is approximately the same for all lighting conditions.}
    \label{fig:dtu_speed12.00_percent_outlier}
\end{figure}



\begin{figure}[H]
    \centering
    \subfigure[Horizontal Coordinate]{
        \includegraphics[width=0.48\textwidth]{feature_tracker_uq/dtu_figs/dtu_LK_lightsLR_speed2.00_x.pdf}
        \includegraphics[width=0.48\textwidth]{feature_tracker_uq/dtu_figs/dtu_LK_lightsBF_speed2.00_x.pdf}
    }
    \subfigure[Vertical Coordinate]{
        \includegraphics[width=0.48\textwidth]{feature_tracker_uq/dtu_figs/dtu_LK_lightsLR_speed2.00_y.pdf}
        \includegraphics[width=0.48\textwidth]{feature_tracker_uq/dtu_figs/dtu_LK_lightsBF_speed2.00_y.pdf}
    }
    \caption{\textbf{DTU Point Features Dataset: At twice nominal speed, lighting condition does not change trends in mean error $\mu(t)$  when using the Lucas-Kanade Tracker.} We compute $\mu(t)$ at each timestep using diffuse lighting (black lines) and each of the directional lighting conditions listed in Figure \ref{fig:dtu_light_stage} using all tracks from all 60 scenes.  The variation of $\mu(t)$ due to the existence of directional lighting is at most 10 percent of the variation common to all plotted lines. The effect of directional lighting is relatively small because changes between adjacent frames are small whether or not the scene contains directional lighting.}
    \label{dtu_LK_mu_speed2.00}
\end{figure}


\begin{figure}[H]
    \centering
    \subfigure[Horizontal Coordinate]{
        \includegraphics[width=0.48\textwidth]{feature_tracker_uq/dtu_figs/dtu_LK_lightsLR_speed4.00_x.pdf}
        \includegraphics[width=0.48\textwidth]{feature_tracker_uq/dtu_figs/dtu_LK_lightsBF_speed4.00_x.pdf}
    }
    \subfigure[Vertical Coordinate]{
        \includegraphics[width=0.48\textwidth]{feature_tracker_uq/dtu_figs/dtu_LK_lightsLR_speed4.00_y.pdf}
        \includegraphics[width=0.48\textwidth]{feature_tracker_uq/dtu_figs/dtu_LK_lightsBF_speed4.00_y.pdf}
    }
    \caption{\textbf{DTU Point Features Dataset: At four times nominal speed, lighting condition does not change trends in mean error $\mu(t)$ when using the Lucas-Kanade Tracker.} We compute $\mu(t)$ at each timestep using diffuse lighting (black lines) and each of the directional lighting conditions listed in Figure \ref{fig:dtu_light_stage} using all tracks from all 60 scenes. The variation of $\mu(t)$ due to the existence of directional lighting is at most 10 percent of the variation common to all plotted lines. The effect of directional lighting is relatively small because changes between adjacent frames are small whether or not the scene contains directional lighting.}
    \label{dtu_LK_mu_speed4.00}
\end{figure}



\begin{figure}[H]
    \centering
    \subfigure[Horizontal Coordinate]{
        \includegraphics[width=0.48\textwidth]{feature_tracker_uq/dtu_figs/dtu_LK_lightsLR_speed8.00_x.pdf}
        \includegraphics[width=0.48\textwidth]{feature_tracker_uq/dtu_figs/dtu_LK_lightsBF_speed8.00_x.pdf}
    }
    \subfigure[Vertical Coordinate]{
        \includegraphics[width=0.48\textwidth]{feature_tracker_uq/dtu_figs/dtu_LK_lightsLR_speed8.00_y.pdf}
        \includegraphics[width=0.48\textwidth]{feature_tracker_uq/dtu_figs/dtu_LK_lightsBF_speed8.00_y.pdf}
    }
    \caption{\textbf{DTU Point Features Dataset: At eight times nominal speed, lighting condition does not change trends in mean error $\mu(t)$ when using the Lucas-Kanade Tracker.} We compute $\mu(t)$ at each timestep using diffuse lighting (black lines) and each of the directional lighting conditions listed in Figure \ref{fig:dtu_light_stage} using all tracks from all 60 scenes. The variation of $\mu(t)$ due to the existence of directional lighting is at most 10 percent of the variation common to all plotted lines. The effect of directional lighting is relatively small because changes between adjacent frames are small whether or not the scene contains directional lighting.}
    \label{dtu_LK_mu_speed8.00}
\end{figure}


\begin{figure}[H]
    \centering
    \subfigure[Horizontal Coordinate]{
        \includegraphics[width=0.48\textwidth]{feature_tracker_uq/dtu_figs/dtu_LK_lightsLR_speed12.00_x.pdf}
        \includegraphics[width=0.48\textwidth]{feature_tracker_uq/dtu_figs/dtu_LK_lightsBF_speed12.00_x.pdf}
    }
    \subfigure[Vertical Coordinate]{
        \includegraphics[width=0.48\textwidth]{feature_tracker_uq/dtu_figs/dtu_LK_lightsLR_speed12.00_y.pdf}
        \includegraphics[width=0.48\textwidth]{feature_tracker_uq/dtu_figs/dtu_LK_lightsBF_speed12.00_y.pdf}
    } 
    \caption{\textbf{DTU Point Features Dataset: At twelve times nominal speed, lighting condition does not change trends in mean error $\mu(t)$  when using the Lucas-Kanade Tracker.} We compute $\mu(t)$ at each timestep using diffuse lighting (black lines) and each of the directional lighting conditions listed in Figure \ref{fig:dtu_light_stage} using all tracks from all 60 scenes. The variation of $\mu(t)$ due to the existence of directional lighting is at most 10 percent of the variation common to all plotted lines. The effect of directional lighting is relatively small because changes between adjacent frames are small whether or not the scene contains directional lighting.}
    \label{fig:dtu_LK_mu_speed12.00}
\end{figure}


\begin{figure}[H]
    \centering
    \subfigure[Horizontal Coordinate]{
        \includegraphics[width=0.48\textwidth]{feature_tracker_uq/dtu_figs/dtu_LK_lightsLR_speed2.00_x_abs.pdf}
        \includegraphics[width=0.48\textwidth]{feature_tracker_uq/dtu_figs/dtu_LK_lightsBF_speed2.00_x_abs.pdf}
    }
    \subfigure[Vertical Coordinate]{
        \includegraphics[width=0.48\textwidth]{feature_tracker_uq/dtu_figs/dtu_LK_lightsLR_speed2.00_y_abs.pdf}
        \includegraphics[width=0.48\textwidth]{feature_tracker_uq/dtu_figs/dtu_LK_lightsBF_speed2.00_y_abs.pdf}
    }
    \caption{\textbf{DTU Point Features Dataset: At twice nominal speed, lighting condition does not change trends in mean absolute error $\kappa(t)$ when using the Lucas-Kanade Tracker.} We compute $\kappa(t)$ using diffuse lighting (black lines) and each of the directional lighting conditions listed in Figure \ref{fig:dtu_light_stage} using all tracks from all 60 scenes.  The variation of $\kappa(t)$ due to the existence of directional lighting is at most 10 percent of the variation common to all plotted lines. The effect of directional lighting is relatively small because changes between adjacent frames are small whether or not the scene contains directional lighting.}
    \label{dtu_LK_kappa_speed2.00}
\end{figure}


\begin{figure}[H]
    \centering
    \subfigure[Horizontal Coordinate]{
        \includegraphics[width=0.48\textwidth]{feature_tracker_uq/dtu_figs/dtu_LK_lightsLR_speed4.00_x_abs.pdf}
        \includegraphics[width=0.48\textwidth]{feature_tracker_uq/dtu_figs/dtu_LK_lightsBF_speed4.00_x_abs.pdf}
    }
    \subfigure[Vertical Coordinate]{
        \includegraphics[width=0.48\textwidth]{feature_tracker_uq/dtu_figs/dtu_LK_lightsLR_speed4.00_y_abs.pdf}
        \includegraphics[width=0.48\textwidth]{feature_tracker_uq/dtu_figs/dtu_LK_lightsBF_speed4.00_y_abs.pdf}
    }
    \caption{\textbf{DTU Point Features Dataset: At four times nominal speed, lighting condition does not change trends in mean absolute error $\kappa(t)$ when using the Lucas-Kanade Tracker.} We compute $\kappa(t)$ using diffuse lighting (black lines) and each of the directional lighting conditions listed in Figure \ref{fig:dtu_light_stage} using all tracks from all 60 scenes.   The variation of $\kappa(t)$ due to the existence of directional lighting is at most 10 percent of the variation common to all plotted lines. The effect of directional lighting is relatively small because changes between adjacent frames are small whether or not the scene contains directional lighting.}
    \label{dtu_LK_kappa_speed4.00}
\end{figure}



\begin{figure}[H]
    \centering
    \subfigure[Horizontal Coordinate]{
        \includegraphics[width=0.48\textwidth]{feature_tracker_uq/dtu_figs/dtu_LK_lightsLR_speed8.00_x_abs.pdf}
        \includegraphics[width=0.48\textwidth]{feature_tracker_uq/dtu_figs/dtu_LK_lightsBF_speed8.00_x_abs.pdf}
    }
    \subfigure[Vertical Coordinate]{
        \includegraphics[width=0.48\textwidth]{feature_tracker_uq/dtu_figs/dtu_LK_lightsLR_speed8.00_y_abs.pdf}
        \includegraphics[width=0.48\textwidth]{feature_tracker_uq/dtu_figs/dtu_LK_lightsBF_speed8.00_y_abs.pdf}
    }
    \caption{\textbf{DTU Point Features Dataset: At eight times nominal speed, lighting condition does not change trends in $\kappa(t)$ when using the Lucas-Kanade Tracker.} We compute $\kappa(t)$ using diffuse lighting (black lines) and each of the directional lighting conditions listed in Figure \ref{fig:dtu_light_stage} using all tracks from all 60 scenes.  The variation of $\kappa(t)$ due to the existence of directional lighting is at most 10 percent of the variation common to all plotted lines. The effect of directional lighting is relatively small because changes between adjacent frames are small whether or not the scene contains directional lighting.}
    \label{dtu_LK_kappa_speed8.00}
\end{figure}


\begin{figure}[H]
    \centering
    \subfigure[Horizontal Coordinate]{
        \includegraphics[width=0.48\textwidth]{feature_tracker_uq/dtu_figs/dtu_LK_lightsLR_speed12.00_x_abs.pdf}
        \includegraphics[width=0.48\textwidth]{feature_tracker_uq/dtu_figs/dtu_LK_lightsBF_speed12.00_x_abs.pdf}
    }
    \subfigure[Vertical Coordinate]{
        \includegraphics[width=0.48\textwidth]{feature_tracker_uq/dtu_figs/dtu_LK_lightsLR_speed12.00_y_abs.pdf}
        \includegraphics[width=0.48\textwidth]{feature_tracker_uq/dtu_figs/dtu_LK_lightsBF_speed12.00_y_abs.pdf}
    } 
    \caption{\textbf{DTU Point Features Dataset: At twelve times nominal speed, lighting condition does not change trends in mean absolute error $\kappa(t)$ when using the Lucas-Kanade Tracker.} We compute $\kappa(t)$ using diffuse lighting (black lines) and each of the directional lighting conditions listed in Figure \ref{fig:dtu_light_stage} using all tracks from all 60 scenes. The variation of $\kappa(t)$ due to the existence of directional lighting is at most 10 percent of the variation common to all plotted lines. The effect of directional lighting is relatively small because changes between adjacent frames are small whether or not the scene contains directional lighting.
}
    \label{fig:dtu_LK_kappa_speed12.00}
\end{figure}



\begin{figure}[H]
    \centering
    \subfigure[Horizontal Coordinate]{
        \includegraphics[width=0.48\textwidth]{feature_tracker_uq/dtu_figs/dtu_LK_lightsLR_speed2.00_x_cov.pdf}
        \includegraphics[width=0.48\textwidth]{feature_tracker_uq/dtu_figs/dtu_LK_lightsBF_speed2.00_x_cov.pdf}
    }
    \subfigure[Vertical Coordinate]{
        \includegraphics[width=0.48\textwidth]{feature_tracker_uq/dtu_figs/dtu_LK_lightsLR_speed2.00_y_cov.pdf}
        \includegraphics[width=0.48\textwidth]{feature_tracker_uq/dtu_figs/dtu_LK_lightsBF_speed2.00_y_cov.pdf}
    }
    \caption{\textbf{DTU Point Features Dataset: At twice nominal speed, lighting condition does not change trends in covariance $\Sigma(t)$  when using the Lucas-Kanade Tracker.} We compute $\Sigma(t)$ using diffuse lighting (black lines) and each of the directional lighting conditions listed in Figure \ref{fig:dtu_light_stage} using all tracks from all 60 scenes. 
    The variation of $\Sigma(t)$ due to the existence of directional lighting is at most 10 percent of the variation common to all plotted lines. The effect of directional lighting is relatively small because changes between adjacent frames are small whether or not the scene contains directional lighting.}
    \label{fig:dtu_LK_cov_speed2.00}
\end{figure}


\begin{figure}[H]
    \centering
    \subfigure[Horizontal Coordinate]{
        \includegraphics[width=0.48\textwidth]{feature_tracker_uq/dtu_figs/dtu_LK_lightsLR_speed4.00_x_cov.pdf}
        \includegraphics[width=0.48\textwidth]{feature_tracker_uq/dtu_figs/dtu_LK_lightsBF_speed4.00_x_cov.pdf}
    }
    \subfigure[Vertical Coordinate]{
        \includegraphics[width=0.48\textwidth]{feature_tracker_uq/dtu_figs/dtu_LK_lightsLR_speed4.00_y_cov.pdf}
        \includegraphics[width=0.48\textwidth]{feature_tracker_uq/dtu_figs/dtu_LK_lightsBF_speed4.00_y_cov.pdf}
    }
    \caption{\textbf{DTU Point Features Dataset: At four times nominal speed, lighting condition does not change trends in covariance $\Sigma(t)$  when using the Lucas-Kanade Tracker.} We compute $\Sigma(t)$ using diffuse lighting (black lines) and each of the directional lighting conditions listed in Figure \ref{fig:dtu_light_stage} using all tracks from all 60 scenes. The variation of $\Sigma(t)$ due to the existence of directional lighting is less than 10 percent of the variation common to all plotted lines for all but one lighting condition. The effect of directional lighting is relatively small because changes between adjacent frames are small whether or not the scene contains directional lighting.
    The larger-than average covariance for lighting condition BF7 is caused by a single scene where feature tracking fails.}
    \label{fig:dtu_LK_cov_speed4.00}
\end{figure}



\begin{figure}[H]
    \centering
    \subfigure[Horizontal Coordinate]{
        \includegraphics[width=0.48\textwidth]{feature_tracker_uq/dtu_figs/dtu_LK_lightsLR_speed8.00_x_cov.pdf}
        \includegraphics[width=0.48\textwidth]{feature_tracker_uq/dtu_figs/dtu_LK_lightsBF_speed8.00_x_cov.pdf}
    }
    \subfigure[Vertical Coordinate]{
        \includegraphics[width=0.48\textwidth]{feature_tracker_uq/dtu_figs/dtu_LK_lightsLR_speed8.00_y_cov.pdf}
        \includegraphics[width=0.48\textwidth]{feature_tracker_uq/dtu_figs/dtu_LK_lightsBF_speed8.00_y_cov.pdf}
    }
    \caption{\textbf{DTU Point Features Dataset: At eight times nominal speed, lighting condition does not change trends in covariance $\Sigma(t)$ when using the Lucas-Kanade Tracker.} We compute $\Sigma(t)$ using diffuse lighting (black lines) and each of the directional lighting conditions listed in Figure \ref{fig:dtu_light_stage} using all tracks from all 60 scenes. The variation of $\Sigma(t)$ due to the existence of directional lighting is at most 10 percent of the variation common to all plotted lines. The effect of directional lighting is relatively small because changes between adjacent frames are small whether or not the scene contains directional lighting.}
    \label{fig:dtu_LK_cov_speed8.00}
\end{figure}


\begin{figure}[H]
    \centering
    \subfigure[Horizontal Coordinate]{
        \includegraphics[width=0.48\textwidth]{feature_tracker_uq/dtu_figs/dtu_LK_lightsLR_speed12.00_x_cov.pdf}
        \includegraphics[width=0.48\textwidth]{feature_tracker_uq/dtu_figs/dtu_LK_lightsBF_speed12.00_x_cov.pdf}
    }
    \subfigure[Vertical Coordinate]{
        \includegraphics[width=0.48\textwidth]{feature_tracker_uq/dtu_figs/dtu_LK_lightsLR_speed12.00_y_cov.pdf}
        \includegraphics[width=0.48\textwidth]{feature_tracker_uq/dtu_figs/dtu_LK_lightsBF_speed12.00_y_cov.pdf}
    }
    \caption{\textbf{DTU Point Features Dataset: At twelve times nominal speed, lighting condition does not change trends in covariance $\Sigma(t)$ when using the Lucas-Kanade Tracker.} We compute $\Sigma(t)$ using diffuse lighting (black lines) and each of the directional lighting conditions listed in Figure \ref{fig:dtu_light_stage} using all tracks from all 60 scenes. 
    At twelve times nominal speed, tracking failures cause large covariances to appear for some lighting conditions. Otherwise, the variation of $\Sigma(t)$ due to the existence of directional lighting is at most 10 percent of the variation common to all plotted lines. The effect of directional lighting is relatively small because changes between adjacent frames are small whether or not the scene contains directional lighting.
    }
    \label{fig:dtu_LK_cov_speed12.00}
\end{figure}



\begin{figure}[H]
    \centering
    \subfigure[Horizontal Coordinate]{
        \includegraphics[width=0.48\textwidth]{feature_tracker_uq/dtu_figs/dtu_match_lightsLR_speed2.00_x.pdf}
        \includegraphics[width=0.48\textwidth]{feature_tracker_uq/dtu_figs/dtu_match_lightsBF_speed2.00_x.pdf}
    }
    \subfigure[Vertical Coordinate]{
        \includegraphics[width=0.48\textwidth]{feature_tracker_uq/dtu_figs/dtu_match_lightsLR_speed2.00_y.pdf}
        \includegraphics[width=0.48\textwidth]{feature_tracker_uq/dtu_figs/dtu_match_lightsBF_speed2.00_y.pdf}
    }
    \caption{\textbf{DTU Point Features Dataset: At twice nominal speed, lighting condition does not change trends in mean error $\mu(t)$ when using the Correspondence Tracker.} We compute $\mu(t)$ at each timestep using diffuse lighting (black lines) and each of the directional lighting conditions listed in Figure \ref{fig:dtu_light_stage} using all tracks from all 60 scenes. Lines are limited to timesteps containing at least 100 features. The variation of $\mu(t)$ due to the existence of directional lighting is at most 10 percent of the variation common to all plotted lines. The effect of directional lighting is relatively small because changes between adjacent frames are small whether or not the scene contains directional lighting.}
    \label{dtu_match_mu_speed2.00}
\end{figure}


\begin{figure}[H]
    \centering
    \subfigure[Horizontal Coordinate]{
        \includegraphics[width=0.48\textwidth]{feature_tracker_uq/dtu_figs/dtu_match_lightsLR_speed4.00_x.pdf}
        \includegraphics[width=0.48\textwidth]{feature_tracker_uq/dtu_figs/dtu_match_lightsBF_speed4.00_x.pdf}
    }
    \subfigure[Vertical Coordinate]{
        \includegraphics[width=0.48\textwidth]{feature_tracker_uq/dtu_figs/dtu_match_lightsLR_speed4.00_y.pdf}
        \includegraphics[width=0.48\textwidth]{feature_tracker_uq/dtu_figs/dtu_match_lightsBF_speed4.00_y.pdf}
    }
    \caption{\textbf{DTU Point Features Dataset: At four times nominal speed, lighting condition does not change trends in mean error $\mu(t)$ when using the Correspondence Tracker.} We compute $\mu(t)$ at each timestep using diffuse lighting (black lines) and each of the directional lighting conditions listed in Figure \ref{fig:dtu_light_stage} using all tracks from all 60 scenes. Lines are limited to timesteps containing at least 100 features. The variation of $\mu(t)$ due to the existence of directional lighting is at most 10 percent of the variation common to all plotted lines. The effect of directional lighting is relatively small because changes between adjacent frames are small whether or not the scene contains directional lighting.}
    \label{dtu_match_mu_speed4.00}
\end{figure}



\begin{figure}[H]
    \centering
    \subfigure[Horizontal Coordinate]{
        \includegraphics[width=0.48\textwidth]{feature_tracker_uq/dtu_figs/dtu_match_lightsLR_speed8.00_x.pdf}
        \includegraphics[width=0.48\textwidth]{feature_tracker_uq/dtu_figs/dtu_match_lightsBF_speed8.00_x.pdf}
    }
    \subfigure[Vertical Coordinate]{
        \includegraphics[width=0.48\textwidth]{feature_tracker_uq/dtu_figs/dtu_match_lightsLR_speed8.00_y.pdf}
        \includegraphics[width=0.48\textwidth]{feature_tracker_uq/dtu_figs/dtu_match_lightsBF_speed8.00_y.pdf}
    }
    \caption{\textbf{DTU Point Features Dataset: At eight times nominal speed, lighting condition does not change trends in mean error $\mu(t)$  when using the Correspondence Tracker.} We compute $\mu(t)$ at each timestep using diffuse lighting (black lines) and each of the directional lighting conditions listed in Figure \ref{fig:dtu_light_stage} using all tracks from all 60 scenes. Lines are limited to timesteps containing at least 100 features. The variation of $\mu(t)$ due to the existence of directional lighting is smaller than the variation common to all plotted lines. The effect of directional lighting is relatively small because changes between adjacent frames are small whether or not the scene contains directional lighting.}
    \label{dtu_match_mu_speed8.00}
\end{figure}


\begin{figure}[H]
    \centering
    \subfigure[Horizontal Coordinate]{
        \includegraphics[width=0.48\textwidth]{feature_tracker_uq/dtu_figs/dtu_match_lightsLR_speed12.00_x.pdf}
        \includegraphics[width=0.48\textwidth]{feature_tracker_uq/dtu_figs/dtu_match_lightsBF_speed12.00_x.pdf}
    }
    \subfigure[Vertical Coordinate]{
        \includegraphics[width=0.48\textwidth]{feature_tracker_uq/dtu_figs/dtu_match_lightsLR_speed12.00_y.pdf}
        \includegraphics[width=0.48\textwidth]{feature_tracker_uq/dtu_figs/dtu_match_lightsBF_speed12.00_y.pdf}
    } 
    \caption{\textbf{DTU Point Features Dataset: At twelve times nominal speed, lighting condition does not change trends in mean error $\mu(t)$  when using the Correspondence Tracker.} We compute $\mu(t)$ at each timestep using diffuse lighting (black lines) and each of the directional lighting conditions listed in Figure \ref{fig:dtu_light_stage} using all tracks from all 60 scenes. Lines are limited to timesteps containing at least 100 features. 
    With the exception of one lighting condition, the variation of $\mu(t)$ due to the existence of directional lighting is at most 10 percent of the variation common to all plotted lines. The effect of directional lighting is relatively small because changes between adjacent frames are small whether or not the scene contains directional lighting. 
    The large variation in lighting condition LR6 is caused by tracking failures. }
    \label{fig:dtu_match_mu_speed12.00}
\end{figure}


\begin{figure}[H]
    \centering
    \subfigure[Horizontal Coordinate]{
        \includegraphics[width=0.48\textwidth]{feature_tracker_uq/dtu_figs/dtu_match_lightsLR_speed2.00_x_abs.pdf}
        \includegraphics[width=0.48\textwidth]{feature_tracker_uq/dtu_figs/dtu_match_lightsBF_speed2.00_x_abs.pdf}
    }
    \subfigure[Vertical Coordinate]{
        \includegraphics[width=0.48\textwidth]{feature_tracker_uq/dtu_figs/dtu_match_lightsLR_speed2.00_y_abs.pdf}
        \includegraphics[width=0.48\textwidth]{feature_tracker_uq/dtu_figs/dtu_match_lightsBF_speed2.00_y_abs.pdf}
    }
    \caption{\textbf{DTU Point Features Dataset: At twice nominal speed, lighting condition does not change trends in mean absolute error $\kappa(t)$  when using the Correspondence Tracker.} We compute $\kappa(t)$ using diffuse lighting (black lines) and each of the directional lighting conditions listed in Figure \ref{fig:dtu_light_stage} using all tracks from all 60 scenes. Lines are limited to timesteps containing at least 100 features.  The variation of $\kappa(t)$ due to the existence of directional lighting is at most 10 percent of the variation common to all plotted lines. The effect of directional lighting is relatively small because changes between adjacent frames are small whether or not the scene contains directional lighting. }
    \label{dtu_match_kappa_speed2.00}
\end{figure}


\begin{figure}[H]
    \centering
    \subfigure[Horizontal Coordinate]{
        \includegraphics[width=0.48\textwidth]{feature_tracker_uq/dtu_figs/dtu_match_lightsLR_speed4.00_x_abs.pdf}
        \includegraphics[width=0.48\textwidth]{feature_tracker_uq/dtu_figs/dtu_match_lightsBF_speed4.00_x_abs.pdf}
    }
    \subfigure[Vertical Coordinate]{
        \includegraphics[width=0.48\textwidth]{feature_tracker_uq/dtu_figs/dtu_match_lightsLR_speed4.00_y_abs.pdf}
        \includegraphics[width=0.48\textwidth]{feature_tracker_uq/dtu_figs/dtu_match_lightsBF_speed4.00_y_abs.pdf}
    }
    \caption{\textbf{DTU Point Features Dataset: At four times nominal speed, lighting condition does not change trends in mean absolute error $\kappa(t)$ remains independent of lighting condition when using the Correspondence Tracker.} We compute $\kappa(t)$ using diffuse lighting (black lines) and each of the directional lighting conditions listed in Figure \ref{fig:dtu_light_stage} using all tracks from all 60 scenes. Lines are limited to timesteps containing at least 100 features. There are no significant differences between lines. The effect of directional lighting is small because changes from frame-to-frame are small.}
    \label{dtu_match_kappa_speed4.00}
\end{figure}



\begin{figure}[H]
    \centering
    \subfigure[Horizontal Coordinate]{
        \includegraphics[width=0.48\textwidth]{feature_tracker_uq/dtu_figs/dtu_match_lightsLR_speed8.00_x_abs.pdf}
        \includegraphics[width=0.48\textwidth]{feature_tracker_uq/dtu_figs/dtu_match_lightsBF_speed8.00_x_abs.pdf}
    }
    \subfigure[Vertical Coordinate]{
        \includegraphics[width=0.48\textwidth]{feature_tracker_uq/dtu_figs/dtu_match_lightsLR_speed8.00_y_abs.pdf}
        \includegraphics[width=0.48\textwidth]{feature_tracker_uq/dtu_figs/dtu_match_lightsBF_speed8.00_y_abs.pdf}
    }
    \caption{\textbf{DTU Point Features Dataset: At eight times nominal speed, lighting condition does not change trends in mean absolute error $\kappa(t)$ when using the Correspondence Tracker.} We compute $\kappa(t)$ using diffuse lighting (black lines) and each of the directional lighting conditions listed in Figure \ref{fig:dtu_light_stage} using all tracks from all 60 scenes. Lines are limited to timesteps containing at least 100 features. The variation of $\kappa(t)$ due to the existence of directional lighting is at most 10 percent of the variation common to all plotted lines. The effect of directional lighting is relatively small because changes between adjacent frames are small whether or not the scene contains directional lighting. }
    \label{dtu_match_kappa_speed8.00}
\end{figure}


\begin{figure}[H]
    \centering
    \subfigure[Horizontal Coordinate]{
        \includegraphics[width=0.48\textwidth]{feature_tracker_uq/dtu_figs/dtu_match_lightsLR_speed12.00_x_abs.pdf}
        \includegraphics[width=0.48\textwidth]{feature_tracker_uq/dtu_figs/dtu_match_lightsBF_speed12.00_x_abs.pdf}
    }
    \subfigure[Vertical Coordinate]{
        \includegraphics[width=0.48\textwidth]{feature_tracker_uq/dtu_figs/dtu_match_lightsLR_speed12.00_y_abs.pdf}
        \includegraphics[width=0.48\textwidth]{feature_tracker_uq/dtu_figs/dtu_match_lightsBF_speed12.00_y_abs.pdf}
    } 
    \caption{\textbf{DTU Point Features Dataset: At twelve times nominal speed, lighting condition does not change trends in mean absolute error $\kappa(t)$  when using the Correspondence Tracker.} We compute $\kappa(t)$ using diffuse lighting (black lines) and each of the directional lighting conditions listed in Figure \ref{fig:dtu_light_stage} using all tracks from all 60 scenes. Lines are limited to timesteps containing at least 100 features. With the exception of lighting condition BF6, the variation of $\kappa(t)$ due to the existence of directional lighting is at most 10 percent of the variation common to all plotted lines. The effect of directional lighting is relatively small because changes between adjacent frames are small whether or not the scene contains directional lighting.
    }
    \label{fig:dtu_match_kappa_speed12.00}
\end{figure}



\begin{figure}[H]
    \centering
    \subfigure[Horizontal Coordinate]{
        \includegraphics[width=0.48\textwidth]{feature_tracker_uq/dtu_figs/dtu_match_lightsLR_speed2.00_x_cov.pdf}
        \includegraphics[width=0.48\textwidth]{feature_tracker_uq/dtu_figs/dtu_match_lightsBF_speed2.00_x_cov.pdf}
    }
    \subfigure[Vertical Coordinate]{
        \includegraphics[width=0.48\textwidth]{feature_tracker_uq/dtu_figs/dtu_match_lightsLR_speed2.00_y_cov.pdf}
        \includegraphics[width=0.48\textwidth]{feature_tracker_uq/dtu_figs/dtu_match_lightsBF_speed2.00_y_cov.pdf}
    }
    \caption{\textbf{DTU Point Features Dataset: At twice nominal speed, lighting condition does not change trends in covariance $\Sigma(t)$  when using the Correspondence Tracker.} We compute $\Sigma(t)$ using diffuse lighting (black lines) and each of the directional lighting conditions listed in Figure \ref{fig:dtu_light_stage} using all tracks from all 60 scenes. Timesteps are limited to those with at least 100 features. The variation of $\Sigma(t)$ due to the existence of directional lighting is at most 10 percent of the variation common to all plotted lines. The effect of directional lighting is relatively small because changes between adjacent frames are small whether or not the scene contains directional lighting.}
    \label{fig:dtu_match_cov_speed2.00}
\end{figure}


\begin{figure}[H]
    \centering
    \subfigure[Horizontal Coordinate]{
        \includegraphics[width=0.48\textwidth]{feature_tracker_uq/dtu_figs/dtu_match_lightsLR_speed4.00_x_cov.pdf}
        \includegraphics[width=0.48\textwidth]{feature_tracker_uq/dtu_figs/dtu_match_lightsBF_speed4.00_x_cov.pdf}
    }
    \subfigure[Vertical Coordinate]{
        \includegraphics[width=0.48\textwidth]{feature_tracker_uq/dtu_figs/dtu_match_lightsLR_speed4.00_y_cov.pdf}
        \includegraphics[width=0.48\textwidth]{feature_tracker_uq/dtu_figs/dtu_match_lightsBF_speed4.00_y_cov.pdf}
    }
    \caption{\textbf{DTU Point Features Dataset: At four times nominal speed, lighting condition does not change trends in covariance $\Sigma(t)$  when using the Correspondence Tracker.} We compute $\Sigma(t)$ using diffuse lighting (black lines) and each of the directional lighting conditions listed in Figure \ref{fig:dtu_light_stage} using all tracks from all 60 scenes. Timesteps are limited to those with at least 100 features.  The variation of $\Sigma(t)$ due to the existence of directional lighting is at most 10 percent of the variation common to all plotted lines. The effect of directional lighting is relatively small because changes between adjacent frames are small whether or not the scene contains directional lighting.}
    \label{fig:dtu_match_cov_speed4.00}
\end{figure}



\begin{figure}[H]
    \centering
    \subfigure[Horizontal Coordinate]{
        \includegraphics[width=0.48\textwidth]{feature_tracker_uq/dtu_figs/dtu_match_lightsLR_speed8.00_x_cov.pdf}
        \includegraphics[width=0.48\textwidth]{feature_tracker_uq/dtu_figs/dtu_match_lightsBF_speed8.00_x_cov.pdf}
    }
    \subfigure[Vertical Coordinate]{
        \includegraphics[width=0.48\textwidth]{feature_tracker_uq/dtu_figs/dtu_match_lightsLR_speed8.00_y_cov.pdf}
        \includegraphics[width=0.48\textwidth]{feature_tracker_uq/dtu_figs/dtu_match_lightsBF_speed8.00_y_cov.pdf}
    }
    \caption{\textbf{DTU Point Features Dataset: At eight times nominal speed, lighting condition does not change trends in covariance $\Sigma(t)$ when using the Correspondence Tracker.} We compute $\Sigma(t)$ using diffuse lighting (black lines) and each of the directional lighting conditions listed in Figure \ref{fig:dtu_light_stage} using all tracks from all 60 scenes. Timesteps are limited to those with at least 100 features. The variation of $\Sigma(t)$ due to the existence of directional lighting is at most 10 percent of the variation common to all plotted lines. The effect of directional lighting is relatively small because changes between adjacent frames are small whether or not the scene contains directional lighting.}
    \label{fig:dtu_match_cov_speed8.00}
\end{figure}


\begin{figure}[H]
    \centering
    \subfigure[Horizontal Coordinate]{
        \includegraphics[width=0.48\textwidth]{feature_tracker_uq/dtu_figs/dtu_match_lightsLR_speed12.00_x_cov.pdf}
        \includegraphics[width=0.48\textwidth]{feature_tracker_uq/dtu_figs/dtu_match_lightsBF_speed12.00_x_cov.pdf}
    }
    \subfigure[Vertical Coordinate]{
        \includegraphics[width=0.48\textwidth]{feature_tracker_uq/dtu_figs/dtu_match_lightsLR_speed12.00_y_cov.pdf}
        \includegraphics[width=0.48\textwidth]{feature_tracker_uq/dtu_figs/dtu_match_lightsBF_speed12.00_y_cov.pdf}
    }
    \caption{\textbf{DTU Point Features Dataset: At twelve times nominal speed, lighting condition does not change trends in covariance $\Sigma(t)$  when using the Correspondence Tracker.} We compute $\Sigma(t)$ using diffuse lighting (black lines) and each of the directional lighting conditions listed in Figure \ref{fig:dtu_light_stage} using all tracks from all 60 scenes. Timesteps are limited to those with at least 100 features. With the exception of feature track failures in lighting condition BF6, the variation of $\Sigma(t)$ due to the existence of directional lighting is a fraction of the variation common to all plotted lines. The effect of directional lighting is relatively small because changes between adjacent frames are small whether or not the scene contains directional lighting. }
    \label{fig:dtu_match_cov_speed12.00}
\end{figure}






