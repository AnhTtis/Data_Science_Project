

\begin{figure}[H]
\centering
\includegraphics[width=0.48\textwidth]{feature_tracker_uq/gazebo_linear_figs/gazebo_wall.png}
\includegraphics[width=0.48\textwidth]{feature_tracker_uq/gazebo_linear_figs/gazebo_linear_track_throwout.pdf}
\caption{\textbf{Gazebo Linear Dataset: We will throw out the 20\% of tracks with the most error instead of the 10\% of tracks.} The right figure plots the histogram density of the maximum L2 error of all feature tracks of one scene in log scale. The corresponding scene is pictured on the left. The large errors  that still remain after removing the 10\% of tracks with the most errors are caused by track propagation along smooth edges when the AGAST feature detector does not select perfect corners, as well as the asynchronous collection of RGB and depth images in the Gazebo simulator. The errors caused by track propagation along smooth edges are unlikely to occur in real world data, where backgrounds and textures are less ideal.}
\label{fig:gazebo_linear_error_throwout}
\end{figure}


\begin{figure}[H]
    \centering
    \includegraphics[width=4in]{feature_tracker_uq/gazebo_linear_figs/gazebo_linear_1.00_feature_lifetime.pdf}
    \caption{\textbf{Gazebo Linear Dataset: Feature Lifetime is usually $\leq$ five frames.} The distribution of feature lifetimes is plotted as a log-scale histogram for both the Lucas-Kanade and Correspondence Tracker at nominal speed. Many features live for less $\leq$ five frames, especially when the Correspondence Tracker is used. However, Lucas-Kanade produces a long tail of features with longer lifetimes.}
    \label{fig:gazebo_linear_feature_lifetime}
\end{figure}


\begin{figure}[H]
    \subfigure[Lucas-Kanade]{\includegraphics[width=0.48\textwidth]{feature_tracker_uq/gazebo_linear_figs/gazebo_linear_avg_feats_LK.pdf}}
    \subfigure[Correspondence]{\includegraphics[width=0.48\textwidth]{feature_tracker_uq/gazebo_linear_figs/gazebo_linear_avg_feats_match.pdf}}
    \caption{Feature Lifetime is plotted on the horizontal axis. The vertical axis, in log scale, shows the number of features in all 11 scenes that lived at least that long for every tested speed. The number of features drops very fast, especially when the Correspondence Tracker is used. \textbf{In subsequent analyses, we only compute means errors and covariances at timesteps with at least 500 features on the Gazebo Linear Dataset.}}
    \label{fig:gazebo_linear_avg_feats}
\end{figure}


\begin{figure}[H]
    \centering
    \subfigure[Lucas-Kanade]{\includegraphics[width=0.48\textwidth]{feature_tracker_uq/gazebo_linear_figs/gazebo_linear_LK_percent_outlier.pdf}}
    \subfigure[Correspondence]{\includegraphics[width=0.48\textwidth]{feature_tracker_uq/gazebo_linear_figs/gazebo_linear_match_percent_outlier.pdf}}
    \caption{\textbf{Gazebo Linear Dataset: Outlier Ratios are a function of speed when using the Lucas-Kanade Tracker and constant for the Correspondence Tracker.} Outlier ratios per frame are shown as box-and-whisker plots for tested speeds for the Lucas-Kanade tracker on the left and the Correspondence Tracker on the right. Mean values are shown as green triangles and median values are shown as orange lines. For lower speeds, the Lucas-Kanade tracker produces fewer outliers. Outlier ratios then increase with speed. On the other hand, the outlier ratio for the Correspondence Tracker remains constant, at around 40 percent.}
    \label{fig:gazebo_linear_outlier_ratio}
\end{figure}


\begin{figure}[H]
    \centering
    \subfigure[$\nu(t)$, Horizontal Coordinate]{\includegraphics[width=0.48\textwidth]{feature_tracker_uq/gazebo_linear_figs/gazebo_linear_speed1.00_x.pdf}}
    \subfigure[$\nu(t)$, Vertical Coordinate]{\includegraphics[width=0.48\textwidth]{feature_tracker_uq/gazebo_linear_figs/gazebo_linear_speed1.00_y.pdf}}
    \caption{\textbf{Gazebo Linear Dataset: At nominal speed, the Lucas-Kanade Tracker slowly accumulates negative error in the horizontal direction. The Correspondence Tracker has zero mean error.} Lines shown are horizontal (left) and vertical (right) coordinates of mean error $\nu(t)$ calculated using tracks averaged over all scenes; calculation is cut off at 58 frames for the Lucas-Kanade Tracker and 9 frames for the Correspondence Tracker so that averages can be computed with at least 500 features.}
    \label{fig:gazebo_linear_1.00_meanerror}
\end{figure}


\begin{figure}[H]
    \centering
    \subfigure[$\eta(t)$, Horizontal Coordinate]{\includegraphics[width=0.48\textwidth]{feature_tracker_uq/gazebo_linear_figs/gazebo_linear_speed1.00_x_abs.pdf}}
    \subfigure[$\eta(t)$, Vertical Coordinate]{\includegraphics[width=0.48\textwidth]{feature_tracker_uq/gazebo_linear_figs/gazebo_linear_speed1.00_y_abs.pdf}}   
    \subfigure[$\Phi(t)$, Horizontal Coordinate]{\includegraphics[width=0.48\textwidth]{feature_tracker_uq/gazebo_linear_figs/gazebo_linear_speed1.00_x_cov.pdf}}
    \subfigure[$\Phi(t)$, Vertical Coordinate]{\includegraphics[width=0.48\textwidth]{feature_tracker_uq/gazebo_linear_figs/gazebo_linear_speed1.00_y_cov.pdf}}      
    \caption{\textbf{Gazebo Linear Dataset: The Lucas-Kanade tracker drifts considerably more than the Correspondence Tracker, but only in the horizontal direction.} Lines shown are horizontal (left column) and vertical (right column) coordinates of mean absolute error $\eta(t)$ (top row) and covariance $\Phi(t)$ (bottom row) calculated using tracks averaged over all scenes; calculation is cut off at 58 frames for Lucas-Kanade Tracker and 9 frames for the Correspondence Tracker so that averages can be computed with at least 500 features. Both mean absolute error and covariance are constant when using the Correspondence Tracker. On the other hand, the horizontal coordinate of $\eta(t)$ and $\Phi(t)$ drifts upwards when using the Lucas-Kanade Tracker.}
    \label{fig:gazebo_linear_1.00_error_cov}
\end{figure}


\begin{figure}[H]
    \centering
    \subfigure[$\nu(t)$, Horizontal Coordinate]{
        \includegraphics[width=0.48\textwidth]{feature_tracker_uq/gazebo_linear_figs/gazebo_linear_LK_x.pdf}
        \includegraphics[width=0.48\textwidth]{feature_tracker_uq/gazebo_linear_figs/gazebo_linear_LK_x_boxplot.pdf}
    }
    \subfigure[$\nu(t)$, Vertical Coordinate]{
        \includegraphics[width=0.48\textwidth]{feature_tracker_uq/gazebo_linear_figs/gazebo_linear_LK_y.pdf}
        \includegraphics[width=0.48\textwidth]{feature_tracker_uq/gazebo_linear_figs/gazebo_linear_LK_y_boxplot.pdf}
    }
    \caption{\textbf{Gazebo Linear Dataset: Mean errors increase with speed when using the Lucas-Kanade Tracker.}
    The left column contains plots of the horizontal (top row) and vertical (bottom row) components of the mean tracking error $\nu(t)$ at each timestep $t$ after initial feature detection at multiple speeds. Each dot corresponds to a processed frame; lines for higher speeds contain data from fewer frames and therefore show fewer dots. The right column plots the ordinate values of each line for $t>0$ in the left figures as a box plot: means are shown as green triangles and medians are shown as orange lines.
    The top-right shows that mean errors in the horizontal coordinate become more negative as speed is increased from 1.00 to 8.00. The mean error then decreases for speeds=10.00 (brown line), 15.00 (pink line), and 20.00 (gray line), showing that both the number of elapsed frames, and the speed are both factors that affect $\nu(t)$. For all speeds, mean error is close to zero in the vertical coordinate.
    }
    \label{fig:gazebo_linear_LK_meanerror}
\end{figure}


\begin{figure}[H]
    \centering
    \subfigure[$\eta(t)$, Horizontal Coordinate]{
        \includegraphics[width=0.48\textwidth]{feature_tracker_uq/gazebo_linear_figs/gazebo_linear_LK_x_abs.pdf}
        \includegraphics[width=0.48\textwidth]{feature_tracker_uq/gazebo_linear_figs/gazebo_linear_LK_x_abs_boxplot.pdf}
    }
    \subfigure[$\eta(t)$, Vertical Coordinate]{
        \includegraphics[width=0.48\textwidth]{feature_tracker_uq/gazebo_linear_figs/gazebo_linear_LK_y_abs.pdf}
        \includegraphics[width=0.48\textwidth]{feature_tracker_uq/gazebo_linear_figs/gazebo_linear_LK_y_abs_boxplot.pdf}
    }
    \caption{\textbf{Gazebo Linear Dataset: Mean absolute errors increase with speed when using the Lucas-Kanade Tracker.}
    The left column contains plots of the horizontal (top row) and vertical (bottom row) components of the mean absolute error $\eta(t)$ at each timestep $t$ after initial feature detection at multiple speeds. Each dot corresponds to a processed frame; lines for higher speeds contain data from fewer frames and therefore show fewer dots. The right column plots the ordinate values of each line for $t>0$ in the left figures as a box plot: means are shown as green triangles and medians are shown as orange lines.
    Mean absolute errors in the horizontal coordinate increase as speed is increased from 1.00 to 8.00. $\eta(t)$ then decreases for speeds=10.00 (brown line), 15.00 (pink line), and 20.00 (gray line), showing that both the number of elapsed frames, and the speed are both factors that affect $\nu(t)$. For all speeds, mean absolute error is close to zero in the vertical coordinate.    
    }
    \label{fig:gazebo_linear_LK_MAE}
\end{figure}



\begin{figure}[H]
    \centering
    \subfigure[$\Phi(t)$, Horizontal Coordinate]{
        \includegraphics[width=0.48\textwidth]{feature_tracker_uq/gazebo_linear_figs/gazebo_linear_LK_x_cov.pdf}
        \includegraphics[width=0.48\textwidth]{feature_tracker_uq/gazebo_linear_figs/gazebo_linear_LK_x_cov_boxplot.pdf}
    }
    \subfigure[$\Phi(t)$, Vertical Coordinate]{
        \includegraphics[width=0.48\textwidth]{feature_tracker_uq/gazebo_linear_figs/gazebo_linear_LK_y_cov.pdf}
        \includegraphics[width=0.48\textwidth]{feature_tracker_uq/gazebo_linear_figs/gazebo_linear_LK_y_cov_boxplot.pdf}
    }
    \caption{\textbf{Gazebo Linear Dataset: Covariance increases with speed when using the Lucas-Kanade Tracker.}
    The left column contains plots of the horizontal (top row) and vertical (bottom row) components of the covariance $\Phi(t)$ at each timestep $t$ after initial feature detection at multiple speeds. Each dot corresponds to a processed frame; lines for higher speeds contain data from fewer frames and therefore show fewer dots. The right column plots the ordinate values of each line for $t>0$ in the left figures as a box plot: means are shown as green triangles and medians are shown as orange lines.
    Covariance increases in the horizontal coordinate increase as speed is increased from 1.00 to 8.00. The covariance then decreases for speeds=10.00 (brown line), 15.00 (pink line), and 20.00 (gray line), showing that both the number of elapsed frames, and the speed are both factors that affect $\Phi(t)$. For all speeds, covariance is close to zero in the vertical coordinate.
    }
    \label{fig:gazebo_linear_LK_cov}
\end{figure}




\begin{figure}[H]
    \centering
    \subfigure[$\nu(t)$, Horizontal Coordinate]{
        \includegraphics[width=0.48\textwidth]{feature_tracker_uq/gazebo_linear_figs/gazebo_linear_match_x.pdf}
        \includegraphics[width=0.48\textwidth]{feature_tracker_uq/gazebo_linear_figs/gazebo_linear_match_x_boxplot.pdf}
    }
    \subfigure[$\nu(t)$, Vertical Coordinate]{
        \includegraphics[width=0.48\textwidth]{feature_tracker_uq/gazebo_linear_figs/gazebo_linear_match_y.pdf}
        \includegraphics[width=0.48\textwidth]{feature_tracker_uq/gazebo_linear_figs/gazebo_linear_match_y_boxplot.pdf}
    }
    \caption{\textbf{Gazebo Linear Dataset: Mean errors are unaffected by speed when using the Correspondence Tracker until tracking failure occurs.}
    The left column contains plots of the horizontal (top row) and vertical (bottom row) components of the mean tracking error $\nu(t)$ at each timestep $t$ after initial feature detection at multiple speeds. Each dot corresponds to a processed frame; lines for higher speeds contain data from fewer frames and therefore show fewer dots. The right column plots the ordinate values of each line for $t>0$ in the left figures as a box plot: means are shown as green triangles and medians are shown as orange lines. In the horizontal coordinate, mean errors remain near zero as speed is increased from 1.00 to 15.00. Mean errors are larger when speed=20.00. The mean error is close to zero in the vertical coordinate.
    }
    \label{fig:gazebo_linear_match_meanerror}
\end{figure}



\begin{figure}[H]
    \centering
    \subfigure[$\eta(t)$, Horizontal Coordinate]{
        \includegraphics[width=0.48\textwidth]{feature_tracker_uq/gazebo_linear_figs/gazebo_linear_match_x_abs.pdf}
        \includegraphics[width=0.48\textwidth]{feature_tracker_uq/gazebo_linear_figs/gazebo_linear_match_x_abs_boxplot.pdf}
    }
    \subfigure[$\eta(t)$, Vertical Coordinate]{
        \includegraphics[width=0.48\textwidth]{feature_tracker_uq/gazebo_linear_figs/gazebo_linear_match_y_abs.pdf}
        \includegraphics[width=0.48\textwidth]{feature_tracker_uq/gazebo_linear_figs/gazebo_linear_match_y_abs_boxplot.pdf}
    }
    \caption{\textbf{Gazebo Linear Dataset: Mean absolute errors increase with speed when using the Correspondence Tracker.}
    The left column contains plots of the horizontal (top row) and vertical (bottom row) components of the mean absolute error $\eta(t)$ at each timestep $t$ after initial feature detection at multiple speeds. Each dot corresponds to a processed frame; lines for higher speeds contain data from fewer frames and therefore show fewer dots. The right column plots the ordinate values of each line for $t>0$ in the left figures as a box plot: means are shown as green triangles and medians are shown as orange lines. In the horizontal coordinate, mean absolute errors increase slowly with speed at first; increases are larger from speed=10.00 to speed=15.00 and speed=15.00 to speed=20.00. The mean absolute error is approximately 0 in the vertical coordinate.
    }
    \label{fig:gazebo_linear_match_MAE}
\end{figure}



\begin{figure}[H]
    \centering
    \subfigure[$\Phi(t)$, Horizontal Coordinate]{
        \includegraphics[width=0.48\textwidth]{feature_tracker_uq/gazebo_linear_figs/gazebo_linear_match_x_cov.pdf}
        \includegraphics[width=0.48\textwidth]{feature_tracker_uq/gazebo_linear_figs/gazebo_linear_match_x_cov_boxplot.pdf}
    }
    \subfigure[$\Phi(t)$, Vertical Coordinate]{
        \includegraphics[width=0.48\textwidth]{feature_tracker_uq/gazebo_linear_figs/gazebo_linear_match_y_cov.pdf}
        \includegraphics[width=0.48\textwidth]{feature_tracker_uq/gazebo_linear_figs/gazebo_linear_match_y_cov_boxplot.pdf}
    }
    \caption{\textbf{Gazebo Linear Dataset: Covariance increases speed when using the Correspondence Tracker.}
    The left column contains plots of the horizontal (top row) and vertical (bottom row) components of the mean tracking error $\Phi(t)$ at each timestep $t$ after initial feature detection at multiple speeds. Each dot corresponds to a processed frame; lines for higher speeds contain data from fewer frames and therefore show fewer dots. The right column plots the ordinate values of each line for $t>0$ in the left figures as a box plot: means are shown as green triangles and medians are shown as orange lines. In the horizontal coordinate, covariance increases slowly with speed at first; increases are larger from speed=10.00 to speed=15.00 and speed=15.00 to speed=20.00. The covariance is close to zero in the vertical coordinate.
    }
    \label{fig:gazebo_linear_match_cov}
\end{figure}



\begin{figure}
    \centering
    \subfigure[$\nu(t)$, Horizontal Coordinate]{\includegraphics[width=0.48\textwidth]{feature_tracker_uq/gazebo_linear_figs/gazebo_linear2_speed1.00_x.pdf}}
    \subfigure[$\nu(t)$, Vertical Coordinate]{\includegraphics[width=0.48\textwidth]{feature_tracker_uq/gazebo_linear_figs/gazebo_linear2_speed1.00_y.pdf}}
    \caption{\textbf{Gazebo Linear Dataset: The Lucas-Kanade Tracker drifts opposite the direction of motion.} Lines above contain $\nu(t)$ computed from tracks using the Lucas-Kanade Tracker. In the black lines, the quadrotor is flying horizontally from left to right, as is the case in the rest of the experiments on the Gazebo Linear Dataset. In the blue lines, the quadrotor is flying horizontally from right to left while observing the same scene; the scene is not mirror-imaged, so the features tracked in the two trajectories are not identical. Once again, there is nearly no mean error in the vertical direction. However, mean horizontal error is positive instead of negative.}
    \label{fig:gazebo_linear_backwards}
\end{figure}