% !TeX spellcheck = en_US

%% ****** Start of file apstemplate.tex ****** %
%%
%%
%%   This file is part of the APS files in the REVTeX 4.2 distribution.
%%   Version 4.2a of REVTeX, January, 2015
%%
%%
%%   Copyright (c) 2015 The American Physical Society.
%%
%%   See the REVTeX 4 README file for restrictions and more information.
%%
%
% This is a template for producing manuscripts for use with REVTEX 4.2
% Copy this file to another name and then work on that file.
% That way, you always have this original template file to use.
%
% Group addresses by affiliation; use superscriptaddress for long
% author lists, or if there are many overlapping affiliations.
% For Phys. Rev. appearance, change preprint to twocolumn.
% Choose pra, prb, prc, prd, pre, prl, prstab, prstper, or rmp for journal
%  Add 'draft' option to mark overfull boxes with black boxes
%  Add 'showkeys' option to make keywords appear
%\documentclass[aps,prl,preprint,groupedaddress]{revtex4-2}
\documentclass[aps,prl,preprint,superscriptaddress]{revtex4-2}
%\documentclass[aps,prl,reprint,groupedaddress]{revtex4-2}

% You should use BibTeX and apsrev.bst for references
% Choosing a journal automatically selects the correct APS
% BibTeX style file (bst file), so only uncomment the line
% below if necessary.
\bibliographystyle{apsrev4-2}
\usepackage[colorlinks = true,
            linkcolor = black,
            urlcolor  = blue,
            citecolor = black,
            anchorcolor = black]{hyperref}
\usepackage{graphicx,xcolor,amsmath,booktabs}

\usepackage{color}
%\usepackage{ulem}
\newcommand{\blue}{\color{blue}}
\newcommand{\red}{\color{red}}
\newcommand{\green}{\color{green}}

\usepackage[euler]{textgreek}
\usepackage[separate-uncertainty=true, retain-explicit-plus=true]{siunitx}


%\usepackage[activate={true,nocompatibility},final,tracking=true,kerning=true,spacing=true]{microtype}

\renewcommand{\figurename}{SUPPL. FIG.}
\renewcommand{\tablename}{SUPPL. TABLE}
\newcommand*{\myalign}[2]{\multicolumn{1}{#1}{#2}}

% \usepackage[mathlines]{lineno}% Enable numbering of text and display math
% \linenumbers\relax % Commence numbering lines

\begin{document}

% Use the \preprint command to place your local institutional report
% number in the upper righthand corner of the title page in preprint mode.
% Multiple \preprint commands are allowed.
% Use the 'preprintnumbers' class option to override journal defaults
% to display numbers if necessary
%\preprint{}

%Title of paper
\title{Ultrafast Opto-magnetic Effects in the Extreme Ultraviolet Spectral Range \texorpdfstring{\\[10pt] SUPPLEMENTAL MATERIAL}{(SUPPLEMENTAL MATERIAL)}}

% repeat the \author .. \affiliation  etc. as needed
% \email, \thanks, \homepage, \altaffiliation all apply to the current
% author. Explanatory text should go in the []'s, actual e-mail
% address or url should go in the {}'s for \email and \homepage.
% Please use the appropriate macro foreach each type of information

% \affiliation command applies to all authors since the last
% \affiliation command. The \affiliation command should follow the
% other information
% \affiliation can be followed by \email, \homepage, \thanks as well.
\author{Martin Hennecke}
\email[]{hennecke@mbi-berlin.de}
%\homepage[]{Your web page}
%\thanks{}
%\altaffiliation{}
\author{Clemens von Korff Schmising}
\author{Kelvin Yao}
\affiliation{Max-Born-Institut f{\"u}r Nichtlineare Optik und Kurzzeitspektroskopie, Max-Born-Stra{\ss}e 2A, 12489 Berlin, Germany}
\author{Emmanuelle Jal}
\author{Boris Vodungbo}
\author{Valentin Chardonnet}
\affiliation{Sorbonne Universit{\'e}, CNRS, Laboratoire de Chimie Physique -- Mati{\`e}re et Rayonnement, LCPMR, 75005 Paris, France}
\author{Katherine L{\'e}gar{\'e}}
\affiliation{Institut National de la Recherche Scientifique, INRS-EMT, Varennes, Quebec J3X 1P7, Canada}
\author{Flavio Capotondi}
\author{Denys Naumenko}
\author{Emanuele Pedersoli}
\author{Ignacio Lopez-Quintas}
\altaffiliation[Present address: ]{Grupo de Investigaci\'{o}n en Aplicaciones del L\'{a}ser y Fot\'{o}nica, Departamento de F\'{i}sica Aplicada, University of Salamanca, 37008 Salamanca, Spain}
\affiliation{FERMI, Elettra-Sincrotrone Trieste, 34149 Basovizza, Trieste, Italy}
\author{Ivaylo P.\ Nikolov}
\author{Lorenzo Raimondi}
\affiliation{FERMI, Elettra-Sincrotrone Trieste, 34149 Basovizza, Trieste, Italy}
\author{Giovanni De Ninno}
\affiliation{FERMI, Elettra-Sincrotrone Trieste, 34149 Basovizza, Trieste, Italy}
\affiliation{Laboratory of Quantum Optics, University of Nova Gorica, 5001 Nova Gorica, Slovenia}
\author{Leandro Salemi}
\affiliation{Department of Physics and Astronomy, Uppsala University, P.O.\ Box 516, SE-751 20 Uppsala, Sweden}
\author{Sergiu Ruta}
\affiliation{Department of Physics, University of York, York YO10 5DD, United Kingdom}
\affiliation{College of Business, Technology and Engineering, Sheffield Hallam University, Howard Street, Sheffield, S1 1WB, United Kingdom}
\author{Roy Chantrell}
\affiliation{Department of Physics, University of York, York YO10 5DD, United Kingdom}
\author{Thomas Ostler}
\affiliation{College of Business, Technology and Engineering, Sheffield Hallam University, Howard Street, Sheffield, S1 1WB, United Kingdom}
\author{Bastian Pfau}
\author{Dieter Engel}
\affiliation{Max-Born-Institut f{\"u}r Nichtlineare Optik und Kurzzeitspektroskopie, Max-Born-Stra{\ss}e 2A, 12489 Berlin, Germany}
\author{Peter M.\ Oppeneer}
\affiliation{Department of Physics and Astronomy, Uppsala University, P.O.\ Box 516, SE-751 20 Uppsala, Sweden}
\author{Stefan Eisebitt}
\affiliation{Max-Born-Institut f{\"u}r Nichtlineare Optik und Kurzzeitspektroskopie, Max-Born-Stra{\ss}e 2A, 12489 Berlin, Germany}
\affiliation{Institut f{\"u}r Optik und Atomare Physik, Technische Universit{\"a}t Berlin, Stra{\ss}e des 17. Juni 135, 10623 Berlin, Germany}
\author{Ilie Radu}
\email[]{radu@mbi-berlin.de}
\altaffiliation[Present address: ]{European XFEL GmbH, Holzkoppel 4, 22869 Schenefeld, Germany}
\affiliation{Max-Born-Institut f{\"u}r Nichtlineare Optik und Kurzzeitspektroskopie, Max-Born-Stra{\ss}e 2A, 12489 Berlin, Germany}
%\affiliation{now with European XFEL}

%Collaboration name if desired (requires use of superscriptaddress
%option in \documentclass). \noaffiliation is required (may also be
%used with the \author command).
%\collaboration can be followed by \email, \homepage, \thanks as well.
%\collaboration{}
%\noaffiliation

\date{\today}

%\begin{abstract}
%% insert abstract here
%\end{abstract}

% insert suggested keywords - APS authors don't need to do this
%\keywords{}

%\maketitle must follow title, authors, abstract, and keywords
\maketitle

% body of paper here - Use proper section commands
% References should be done using the \cite, \ref, and \label commands
%\section{}
% Put \label in argument of \section for cross-referencing
%\section{\label{}}
%\subsection{}
%\subsubsection{}

%\newpage

%%%%%%%%%%%%%%%%%%%%%%%%%%%%%%%%%%%%%%%%%%%%%%%%%%%%%%%
%\section{Supplemental Material}
%%%%%%%%%%%%%%%%%%%%%%%%%%%%%%%%%%%%%%%%%%%%%%%%%%%%%%%

%%%%%%%%%%%%%%%%%%%%%%%%%%%%%%%%%%%%%%%%%%%%%%%%%%%%%%%
\section{1. Experimental techniques and data acquisition}
%%%%%%%%%%%%%%%%%%%%%%%%%%%%%%%%%%%%%%%%%%%%%%%%%%%%%%%

The time-resolved pump-probe measurements shown in the main article were carried out at the DiProI end-station using the FEL-1 beamline of the seeded free-electron laser (FEL) FERMI \cite{Capotondi:ig5025}, employing the magneto-optical Faraday and Kerr effect to measure the time evolution of the GdFeCo sample magnetization as a function of XUV excitation photon energy, polarization and fluence.
The magnetization dynamics of the sample were probed under an incidence angle of \SI{45}{\degree}, employing linearly polarized, \SI{\approx 90}{fs} (FWHM), \SI{400}{nm} optical pulses generated by a frequency-doubled Ti:sapphire laser system.
Since the laser system shares an oscillator with a similar Ti:sapphire laser that provides the seeding pulses for initiating the FEL lasing process, the XUV pump and optical probe pulses are intrinsically synchronized and the jitter between the two pulses is reduced to \SI{\approx 10}{fs} \cite{Cinquegrana2021}.
Both Faraday and Kerr rotation of the polarization axis of the probing pulses were measured simultaneously in transmission and reflection geometry, respectively, by two independent polarization-sensitive detection setups using Wollaston prisms and balanced photo diodes.
An electromagnet applying a saturating magnetic field of \SI{\pm 8}{mT} perpendicular to the sample plane was used to restore the initial magnetization state of the sample after each pump-probe cycle.
Additionally, the magnetic field was flipped every 200 pulses to obtain a magnetic contrast corresponding to the difference between the Faraday and Kerr signals measured for opposite magnetic field directions.
The delay between pump and probe pulses was adjusted via an optical delay stage in the probe beam path. 
Both the FEL and the optical laser system were running at a repetition rate of \SI{50}{Hz}. 
By seeding only every second accelerated electron bunch in the modulator section of the FEL, the XUV pump repetition rate is effectively reduced to \SI{25}{Hz}, allowing for an interleaved measurement of the pumped and unpumped states of the sample. 
Fast oscilloscopes triggered by the time base of the FEL were used to record and split the pumped and unpumped signals of the balanced photo diodes.
The acquired Faraday rotation signals, i.e., the pumped and unpumped state for opposite magnetic field directions as a function of pump-probe delay and XUV polarization, are shown exemplarily in Suppl.~Fig.~\ref{Suppl_1_Figure} for an incident excitation fluence of \SI{4.7}{mJ/cm^2} at a photon energy of \SI{64.0}{eV}.

\begin{figure}
	\centering
	\includegraphics[width=0.80\linewidth]{Suppl_1_Figure}
	\caption{Transient magneto-optical Faraday rotation upon excitation with an incident fluence of \SI{4.7}{mJ/cm^2} and a photon energy of \SI{64.0}{eV}, recorded for opposite magnetic field directions as a function of pump-probe delay and XUV polarization.
	The pumped and unpumped states are acquired by probing at \SI{50}{Hz} repetition rate while pumping at only \SI{25}{Hz}.
	\label{Suppl_1_Figure}}
\end{figure}

\begin{table}
	\centering
	\begin{tabular}{|c|c|l|c|}
		\hline 
		\textbf{Energy (eV)} & \textbf{Wavelength (nm)} & \myalign{c|}{\textbf{FEL seed (nm)}} & \textbf{Pulse length (fs)} \\ 
		\hline\hline
		$51.00 \pm 0.02$ & $24.31 \pm 0.01$ & 243.14 (10. harm.)  & $\approx 92$ \\%92.1 \\ 
		\hline 
		$54.10 \pm 0.02$ & $22.92 \pm 0.01$ & 251.90 (11. harm) & $\approx 89$ \\%89.2 \\ 
		\hline 
		$56.10 \pm 0.02$ & $22.10 \pm 0.01$ & 243.14 (11. harm)  & $\approx 89$ \\%89.2 \\ 
		\hline 
		$64.00 \pm 0.04$ & $19.37 \pm 0.01$ & 251.90 (13. harm)  & $\approx 84$ \\%84.3 \\ 
		\hline 
	\end{tabular} 
	\caption{XUV photon energies and wavelengths used for excitation by tuning the FEL seeding laser wavelength and harmonics. 
	The spectral bandwidth (energy resolution) is given by fitting the XUV spectrometer measurements with a Gaussian peak and taking the FWHM of the peak. 
	The pulse lengths are approximated from the seeding laser pulse duration of \SI{\approx 170}{fs} and its harmonic order~\cite{PhysRevX.7.021043}. 
	\label{Suppl_1_Table}}
\end{table}

In order to excite the sample using different XUV photon energies, the wavelength of the FEL was adjusted by either changing the FEL seeding laser wavelength and the undulators gap between the magnetic sections, or by changing the harmonic order of the emitted radiation.
Suppl.~Table~\ref{Suppl_1_Table} shows the FEL parameters used in the experiment. 
A spectrometer in the XUV beam path was used to record the spectrum of the FEL shots in order to determine the spectral bandwidth, i.e., the energy resolution, by fitting the spectrum with a Gaussian function. 
The FEL pulse durations were determined according to Ref.~\onlinecite{PhysRevX.7.021043}, scaling with the seeding laser pulse length of \SI{\approx 170}{fs} and inversely with the harmonic order of the FEL. 
The resulting pulse lengths show only negligible dependence on the XUV wavelengths used in our experiment and are thus approximated by \SI{\approx 90}{fs}. 
At each photon energy, the polarization of the XUV was alternated between linear horizontal, $\sigma_-$ and $\sigma_+$ by moving the undulator of the FEL. 
The degree of circular polarization after transmission through the DiProI beamline was characterized in Ref.~\onlinecite{PhysRevX.4.041040} and shown to be consistently above \SI{90}{\percent} for both $\sigma_-$ and $\sigma_+$ up to XUV wavelengths of \SI{60}{nm}, approaching almost \SI{\approx 100}{\percent} in the lower wavelength range below \SI{25}{nm}. 
Thus, tuning the XUV wavelength in a range from 19.37 to \SI{24.31}{nm}, any change in helicity due to the different wavelengths is expected to be on the order of \SI{\approx 1}{\percent} or less, thus not significantly impacting the polarization state of the XUV pulses. 
Previous pump-probe studies reported by the authors of Ref.~\onlinecite{Gutt2017}, which were carried out at the same end station utilizing a similar wavelength and fluence range, have further demonstrated the very high reproducibility and intensity correlation when switching the helicity of the FEL radiation between $\sigma_-$ and $\sigma_+$.
Changing the incident pump fluence on the sample was accomplished by attenuating the FEL pulse energies using solid-state aluminum filters of different thicknesses (\SIrange[range-phrase=--, range-units=single]{100}{500}{\nano\meter}) for rough adjustments and a variable pressure inside of a gas absorber for fine-tuning. 
The resulting attenuation, average pulse energy and shot-to-shot fluctuations of the FEL could be monitored by two $I_0$ gas monitor detectors (GMD) before and after the attenuator.

For optimizing the pump-probe conditions, the spot sizes (FWHM) of the FEL pump and optical probe pulses were adjusted and measured directly in the sample plane by covering parts of the sample with a layer of fluorescent paint and profiling the beam spots on a camera, leading to an uncertainty of \SIrange[range-phrase=--, range-units=single]{\approx 5}{10}{\percent} in FWHM due to this method. 
The XUV pump spot size was adjusted to \SIrange[range-phrase=$\times$, range-units=single]{300}{300}{\micro\meter\squared} using a Kirkpatrick-Baez (KB) mirror focusing system in front of the experimental chamber. 
The spot size of the probing laser was tuned to \SIrange[range-phrase=$\times$, range-units=single]{85}{85}{\micro\meter\squared} via optical lenses.
The delay between pump and probe pulses was adjusted via an optical delay stage in the probe beam path.
For spatial overlap, the probing spot was centered within the much larger pump spot, in order to probe a homogeneously pumped area. 
Additionally, a YAG screen placed before the KB mirrors was used to validate the XUV beam size and position before the focusing optics. 
Spot sizes and pump-probe overlap were checked regularly throughout the experiment in order to assure reliable and stable pump-probe conditions and minimize any systematic error that could emerge from temporal drifts or a change of FEL parameters, especially after changing photon energy and polarization.

%%%%%%%%%%%%%%%%%%%%%%%%%%%%%%%%%%%%%%%%%%%%%%%%%%%%%%%
\section{2. Data sorting and treatment}
%%%%%%%%%%%%%%%%%%%%%%%%%%%%%%%%%%%%%%%%%%%%%%%%%%%%%%%

\begin{figure}
	\centering
	\includegraphics[width=0.8\linewidth]{Suppl_2_Figure}
	\caption{Analysis of the shot-resolved FEL pulse energies recorded by the $I_0$ gas monitor detector. 
	The graph shows exemplary the data recorded for linearly polarized XUV radiation at a photon energy of \SI{54.1}{eV}. 
	(a) FEL pulse energies per shot as recorded for an average pulse energy of \SI{5.0}{\micro\joule}. Only every twentieth shot is plotted for better visibility. 
	(b) Histograms of the statistical distribution of XUV pulse energies recorded during three subsequent pump-probe delay scans using different average excitation fluences.
	The Faraday/Kerr data was sorted by averaging only over those data points where the sample was excited by FEL shots with the same pulse energy, as defined by a \SI{1}{\micro\joule} grid (shown as dashed lines). 
	\label{Suppl_2_Figure}}
\end{figure}

The recorded time-resolved Faraday and Kerr data were sorted by incident excitation fluence using the pulse energies of the FEL shots recorded by the $I_0$ GMD. 
Shot-to-shot fluctuations of the FEL source lead to a statistical distribution of pulse energies around an average value targeted by the attenuator settings. 
Thus, each data set collected during a single pump-probe delay scan contains a large amount of fluence-dependent information that can be extracted.

Suppl.~Fig.~\ref{Suppl_2_Figure} shows exemplary the histograms of FEL pulse energies per shot of three consecutive pump-probe delay scans recorded at different average excitation fluences. 
Sorting the data by pulse energy using a step size of \SI{1}{\micro\joule} and averaging only over the Faraday and Kerr probe belonging to the same interval of $(x\pm0.5)$\,\textmu J allows to obtain a fluence-dependence with a high density of points.
The binning window widths, resulting in an uncertainty of the excitation fluence, were chosen as a trade-off between the density of points in the fluence diagram and the signal-to-noise ratio of the averaged Faraday and Kerr signals which is limited by the number of shots that fall into the intervals. 
Taking also the beamline transmission of \SI{\approx 60}{\percent} into account, the pulse energies can be divided by the XUV spot size to obtain the incident fluence in units of mJ/cm$^2$ on the sample.

\begin{figure}
	\centering
	\includegraphics[width=1\linewidth]{Suppl_3_Figure}
	\caption{Transient magnetization dynamics of the system induced by XUV pulses of \SI{54.1}{eV} photon energy, probed by the normalized magneto-optical Faraday rotation as a function of pump-probe delay, XUV polarization and incident excitation fluence.
	The magnetization is normalized to the equilibrium magnetization in the unexcited state ($M/M_0$).
	\label{Suppl_3_Figure}}
\end{figure}

Suppl.~Fig.~\ref{Suppl_3_Figure} shows the full data set of the polarization- and fluence-dependent magnetization dynamics induced by resonant \SI{54.1}{eV} excitation as a function of pump-probe delay, obtained from sorting the time-resolved magneto-optical Faraday rotation after XUV pump pulse energy.
The data was fitted using a double exponential fit function (solid lines in Suppl.~Fig.~\ref{Suppl_3_Figure}), taking into account the ultrafast demagnetization, the subsequent relaxation process and the experimental time resolution:
\begin{equation} \label{fitfunction}
f(t)=g(t)\otimes
\begin{cases}
\left.A\right., & t \leq 0\\
\left.A-B\left[1-\exp\left(-\frac{t}{\tau_{\text{B}}}\right)\right]+C\left[1-\exp\left(-\frac{t}{\tau_{\text{C}}}\right)\right]\right., & t>0
\end{cases}
\end{equation}
with $A$ corresponding to the unpumped Faraday rotation at negative delays, $B$ and $C$ to the amplitudes of the two exponential components modeling the ultrafast demagnetization and subsequent relaxation processes and $\tau_{\text{B}}$ and $\tau_{\text{C}}$ to the respective exponential time constants.
By convolution with a Gaussian function $g(t)$, the experimental time resolution of \SI{\approx 280}{fs} (FWHM) is taken into account, which was determined using the method described in Ref.~\onlinecite{ZIOLEK2001467} from the pulse duration of the pump and probe pulses as well as the experimental geometry, i.e., the angle between the two beams and their footprints on the sample. 

%%%%%%%%%%%%%%%%%%%%%%%%%%%%%%%%%%%%%%%%%%%%%%%%%%%%%%%
\section{3. Faraday vs.\ Kerr probing}
%%%%%%%%%%%%%%%%%%%%%%%%%%%%%%%%%%%%%%%%%%%%%%%%%%%%%%%

\begin{figure}
	\centering
	\includegraphics[width=1\linewidth]{Suppl_4_Figure}
	\caption{Maximum demagnetization amplitudes ($D$) obtained from the transient magneto-optical \textit{Faraday} rotation upon $\sigma_{\pm}$-polarized excitation (upper panels, red and green) and their difference $\Delta M$ (lower panels, yellow) as a function of incident fluence and XUV photon energy.
	\label{Suppl_4_Figure}}
\end{figure}

\begin{figure}
	\centering
	\includegraphics[width=1\linewidth]{Suppl_5_Figure}
	\caption{Maximum demagnetization amplitudes ($D$) obtained from the transient magneto-optical \textit{Kerr} rotation upon $\sigma_{\pm}$-polarized excitation (upper panels, red and green) and their difference $\Delta M$ (lower panels, yellow) as a function of incident fluence and XUV photon energy.
	\label{Suppl_5_Figure}}
\end{figure}

Suppl.~Fig.~\ref{Suppl_4_Figure} and \ref{Suppl_5_Figure} show the maximum demagnetization amplitudes $D$ upon $\sigma_{\pm}$ and linearly polarized excitation ($D = 1 - \text{min}[M/M_0]$) as well as the differences $\Delta M = D[\sigma_-] - D[\sigma_+]$ as a function of incident excitation fluence for the XUV excitation photon energies of 51.0, 54.1, 56.1 and \SI{64.0}{eV}, comparing magneto-optical Faraday and Kerr probing. 
Note that all values are normalized to the equilibrium magnetization in the unexcited state.
The data were fitted using sigmoid functions serving as a guide to the eye. 
The shaded areas in the upper panels correspond to a \SI{90}{\percent} confidence interval as an estimation of the experimental uncertainty. 
The shaded areas in the lower panels correspond to the different fluence regimes that within the experimental uncertainty indicate the helicity-dependent effect to scale almost linearly with the fluence (white area) until a saturated state is reached where it almost stays constant or starts to decrease again (gray area). 
The dashed lines serve as a guide to the eye and correspond to linear trends fitted to the different fluence regimes. 
Comparing the Faraday and Kerr data reveals a slight difference in demagnetization amplitudes, which can be related to the different information depths of the Faraday and Kerr measurements, corresponding either to the whole depth of the sample in transmission or the penetration depth of the optical light in reflection geometry, respectively.
Furthermore, the helicity-dependent effect probed by the Kerr rotation undergoes a stronger attenuation in the saturated regime, as the measured demagnetization amplitudes approach the fully demagnetized state for lower excitation fluences compared to the Faraday data.

%%%%%%%%%%%%%%%%%%%%%%%%%%%%%%%%%%%%%%%%%%%%%%%%%%%%%%%
\section{4. Static XUV and XMCD spectroscopy}
%%%%%%%%%%%%%%%%%%%%%%%%%%%%%%%%%%%%%%%%%%%%%%%%%%%%%%%

\begin{figure}
	\centering
	\includegraphics[width=.5\linewidth]{Suppl_6_Figure}
	\caption{Static XUV absorption (XAS) and XMCD asymmetry spectra of the $\textrm{Gd}_{24}\textrm{Fe}_{67}\textrm{Co}_{9}$ and $\textrm{Gd}_{24}\textrm{Fe}_{76}$ samples, measured in the photon energy range of the Fe $\text{M}_{3,2}$ resonance.
	The small peak at \SI{62.0}{eV} in the XAS of the GdFeCo sample (upper panel) arises due to resonant excitation of the small Co constituent, which is not present in the XAS of the GdFe sample (lower panel).
	On the contrary, the XMCD of the GdFeCo sample is suppressed at \SI{62.0}{eV} compared to GdFe, which can be attributed to the opposite polarity of the Fe and Co XMCD at this photon energy (see, e.g., Ref.~\onlinecite{PhysRevLett.122.217202} for XMCD spectra of elemental Fe and Co), cancelling out each other in their superposition due to the different concentration of Fe and Co in the alloy.
	\label{Suppl_6_Figure}}
\end{figure}

In order to characterize the studied $\textrm{Gd}_{24}\textrm{Fe}_{67}\textrm{Co}_{9}$ sample in the XUV spectral range and to determine the magnitude of the dichroic absorption expected due to the XMCD effect, static XUV absorption spectroscopy (XAS) was carried out using circularly polarized radiation at the Fe $\text{M}_{3,2}$ resonance, employing the ALICE reflectometer \cite{alice} at the PM3 beamline of BESSY II \cite{Kachel:ie5135}.
The XMCD spectrum shown in Fig.~1b of the main article thereby corresponds to the asymmetry ($A$) of the transmitted XUV intensities ($T_{\uparrow,\downarrow}$) recorded for opposite magnetic field directions perpendicular to the sample plane, which is calculated by $A = (T_\uparrow - T_\downarrow) / (T_\uparrow + T_\downarrow)$.
The absorption spectra for opposite fields ($B_{\uparrow,\downarrow}$) are determined from the XUV transmission of the sample normalized to the beamline spectrum ($T_0$) by $B_{\uparrow,\downarrow} = -\text{ln}(T_{\uparrow,\downarrow}/T_0)$. 
For comparison, this characterization was also carried out for a similar $\textrm{Gd}_{24}\textrm{Fe}_{76}$ sample, i.e., without the Co ingredient but otherwise identical composition.
The comparison is shown in Suppl.~Fig.~\ref{Suppl_6_Figure}, showing the influence of the small fraction of Co atoms on the XAS and XMCD.
In order to correct for the limited degree of circular polarization approaching the lower photon energy limits of the beamline, the magnetic part $\Delta \beta$  of the absorptive refractive index was calculated from the measured XMCD for the effective thickness of the Fe content within the alloy and normalized to reference XMCD measurements on pure Fe systems \cite{PhysRevLett.122.217202}.
The corrected XMCD magnitude at the Fe $\text{M}_{3,2}$ resonance agrees with the values reported in the literature for comparable GdFe systems \cite{app10217580,hessing_phd_2021}.

%%%%%%%%%%%%%%%%%%%%%%%%%%%%%%%%%%%%%%%%%%%%%%%%%%%%%%%
\section{5. Atomistic spin dynamics (ASD) simulations}
%%%%%%%%%%%%%%%%%%%%%%%%%%%%%%%%%%%%%%%%%%%%%%%%%%%%%%%

In order to quantify the effect of the finite XMCD present in case of fully resonant excitation at 54.1 and \SI{56.1}{eV} on the helicity-dependent dynamics, atomistic spin dynamics (ASD) simulations were carried out \cite{Radu2011}, simulating the influence of the dichroic absorption on the helicity-dependent demagnetization amplitudes.
The magnitude of the dichroic absorption was set according to the XMCD asymmetry obtained from the static XUV and XMCD spectroscopy (see also Section~4 and Fig.~1b of the main article), i.e., \SI{-6.2 \pm 0.5}{\percent} in case of \SI{54.1}{eV}, \SI{-3.0 \pm 0.5}{\percent} in case of \SI{56.1}{eV} and \SI{0.5 \pm 0.5}{\percent} in case of \SI{51.0}{eV} excitation, respectively (note that for \SI{64.0}{eV}, the XMCD is fully zero, thus no dichroic absorption is expected).
The resulting simulated fluence-dependencies of the maximum demagnetization amplitudes upon $\sigma_{\pm}$-polarized excitation are shown in Suppl.~Fig.~\ref{Suppl_7_Figure} together with their difference $\Delta M$. 

\begin{figure}
	\centering
	\includegraphics[width=1\linewidth]{Suppl_7_Figure}
	\caption{Maximum demagnetization amplitudes ($D$) obtained from ASD simulations for $\sigma_{\pm}$- and linearly polarized excitation (upper panels, red, blue and green) as a function of simulated excitation fluence and magnitude of XMCD that was set according to the static XMCD spectroscopy of the sample.
	The lower panels show the corresponding difference $\Delta M$ (purple).
	The error bars are obtained by varying the input values for the XMCD magnitude by \SI{\pm 0.5}{\percent}.
	\label{Suppl_7_Figure}}
\end{figure}

\begin{figure}
	\centering
	\includegraphics[width=1\linewidth]{Suppl_8_Figure}
	\caption{Experimentally observed (yellow color, compare Suppl.~Fig.~\ref{Suppl_4_Figure}) and simulated (purple color, compare Suppl.~Fig.~\ref{Suppl_7_Figure}) $\Delta M$ values as a function of the maximum demagnetization amplitudes induced by linearly polarized excitation of the same fluence.
	\label{Suppl_8_Figure}}
\end{figure}

Simulating also the dynamics excited by \textit{linearly} polarized radiation allows better comparability between experiment and simulation by comparing the simulated XMCD-induced and experimentally obtained helicity-dependence ($\Delta M$) based on the respective fluences that lead to the same amount of demagnetization in the linearly polarized case.
Both the experimentally observed (yellow color) and simulated (purple color) $\Delta M$ values are shown in Suppl.~Fig.~\ref{Suppl_8_Figure} as a function of the maximum demagnetization amplitudes induced by linearly polarized excitation of the same fluence.
The comparison between experiment and ASD simulations presented in Fig.~4 of the main article is based on the largest $\Delta M$ observed in the experiment for each XUV photon energy and the interpolated value of $\Delta M$ taken from the ASD simulation resembling the respective magnitude of XMCD.

The ASD simulations are done using the VAMPIRE software package \cite{Evans2014,vampire-url} based on the Landau-Lifshitz-Gilbert (LLG) equation. 
The system Hamiltonian is given by:
\begin{align}
    \mathcal{H} = & - \sum_{i<j } (\mathbf{S}_i\,\mathbf{J}_{ij} \,  \mathbf{S}_j\:)-k_\text{u} \sum_{i} (\mathbf{S}_i\, \cdot  \mathbf{e})^2 
\end{align}
where $\mathbf{S}_i$, $\mathbf{S}_j$ are the normalised spin vectors on $i, j$ sites, $\mathbf{{J}}_{ij}$ is the exchange constant and $k_{\text{u}}$ is the uniaxial magnetocrystalline anisotropy energy  constant per site. 
We used similar parametrization as reported in literature \cite{Ostler2012,Iacocca2019}: $\mu_{\text{Fe}}=1.92 \mu_\text{B}$ and $\mu_{\text{Gd}}=7.63 \mu_\text{B}$, the anisotropy is taken as $k_\text{u}=8.07246 \cdot 10^{-24}$\,J and the exchange interactions are $J_{\text{Fe--Fe}}=2.835\cdot 10^{-21}$\,J, $J_{\text{Gd--Gd}}=1.26\cdot 10^{-21}$\,J and $J_{\text{Fe--Gd}}=-1.09\cdot 10^{-21}$\,J.
We incorporate the rapid change in thermal energy of a system under the influence of a femtosecond laser pulse using the two-temperature model \cite{Jiang2005}:
\begin{align}
T_\text{e} C_\text{e} \frac{dT_\text{e}(t)}{dt} &=-G_\text{e--ph} \left[T_\text{ph}(t)-T_\text{e}(t)\right] + P_\text{abs}(t), \\
C_\text{ph} \frac{dT_\text{ph}(t)}{dt}&=G_\text{e--ph} \left[T_\text{e}(t)-T_\text{ph}(t)\right], \label{2TM2} 
\end{align}
where $C_\text{e}=225$\,Jm$^{-3}$K$^{-2}$, $C_\text{ph}=3.1 \cdot10^6$\,Jm$^{-3}$K$^{-1}$ and $G_\text{e--ph}=2.5\cdot10^{17}$\,Wm$^{-3}$K$^{-1}$.
The spin system is coupled to the electron temperature $T_\text{e}$ and the quantity $P_\text{abs}(t)$ corresponds to the laser power density absorbed by the electronic system and depends on laser fluence and light polarisation.
The XMCD effect is included by modifying $P_\text{abs}(t)$ proportional to the percentage obtained from the experiments.  

%%%%%%%%%%%%%%%%%%%%%%%%%%%%%%%%%%%%%%%%%%%%%%%%%%%%%%%
\section{6. {{AB INITIO}} calculations of the Inverse Faraday Effect}
%%%%%%%%%%%%%%%%%%%%%%%%%%%%%%%%%%%%%%%%%%%%%%%%%%%%%%%

In order to compare our experimental results to theoretical predictions of an opto-magnetic effect in the XUV spectral range, we carried out first \textit{ab initio} calculations of the IFE of GdFeCo alloy, in the vicinity of the Fe $\text{M}_{3,2}$ resonance, using the second-order response theory formalism as derived in Refs.~\cite{PhysRevB.89.014413,PhysRevLett.117.137203}.
The calculations were performed with the full-potential, all-electron code WIEN2k \cite{Blaha2018}, with spin-orbit interaction included. 

The system was modeled as a stoichiometric $\text{GdFe}_2\text{Co}$ unit cell in the $\text{AuCu}_3$ structure, i.e., as $\text{GdFe}_3$, but with one Fe atom replaced by Co. 
The lattice parameters were first optimized using the VASP code \cite{Kresse1996}, after which the electronic structure and IFE was computed with the WIEN2k program. 
To capture the strong electron correlations in the Gd 4f shell, we employed the GGA+$U$ method, in the atomic limit version, with parameters $U = \SI{7}{eV}$ and $J = \SI{1}{eV}$, in combination with the generalized gradient approximation (GGA) \cite{Perdew1996} to the exchange-correlation potential. 
The calculated bandstructure is shown in Suppl.~Fig.~\ref{Suppl_9_Figure}. 
The occupied Gd 4f states are located at \SI{-6}{eV} and the unoccupied 4f states start at \SI{+4}{eV}. 
The spin-polarized Fe and Co 3d states that are accessed by XUV photon excitation of the 3p states are in the energy window of 0 to \SI{4}{eV}.

\begin{figure}
	\includegraphics[width=0.75\linewidth]{Suppl_9_Figure.jpg}
	\caption{Calculated spin-polarized bandstructure of $\text{GdFe}_2\text{Co}$ along high-symmetry lines of the simple tetragonal Brillouin zone, with special symmetry points as indicated.}  
	\label{Suppl_9_Figure}
\end{figure}

To compute the IFE, the exchange splitting of the 3p semi-core states of Fe and Co atoms was included as well \cite{Valencia2010}.
The opto-magnetic IFE response for left ($\sigma_-$) and right ($\sigma_+$) circularly polarized excitation is expressed as
\begin{equation}
\vec{M}_{\text{ind}} = i\, \mathcal{K}^{\text{IFE}}_{\sigma_{\pm}} (\omega) \left[ \vec{E}(\omega)
\times \vec{E}^{\star}(\omega) \right],
\end{equation}
where $\vec{E}(\omega)$ is the electric field of the XUV light and $\mathcal{K}^{\text{IFE}}_{\sigma_{\pm}} (\omega)$ the helicity-dependent IFE constant \cite{PhysRevLett.117.137203}. 
\begin{figure}
	\includegraphics[width=0.50\linewidth]{Suppl_10_Figure}
	\caption{Static \textit{ab initio} calculations of the Inverse Faraday Effect (IFE) constant $\mathcal{K}^{\text{IFE}}$ in the XUV spectral range for a $\text{GdFe}_2\text{Co}$ unit cell. 
	The upper panel shows the calculated total opto-magnetic response to $\sigma_{\pm}$-polarized XUV radiation as a function of photon energy, i.e., the sum over the spin ($S$) and orbital ($L$) responses to left ($\sigma_-$) and right ($\sigma_+$) circularly polarized excitation, respectively.
	The lower panel shows the corresponding difference $\Delta_{\text{IFE}}$ between the total opto-magnetic constants, which is also shown in Fig.~4 of the main article for qualitative comparison to the experimental data.
	\label{Suppl_10_Figure}}
\end{figure}
The computed opto-magnetic IFE response is shown in Suppl.~Fig.~\ref{Suppl_10_Figure}.
The photon energy- and helicity-dependent IFE constants are given for the total optically induced magnetization $M_{\text{ind}}$, i.e., they contain both spin and orbital contributions (see Ref.~\onlinecite{PhysRevLett.117.137203}). 
The IFE constants for $\sigma_{-}$ and $\sigma_{+}$ circular polarizations are clearly distinct in the photon energy range of the Fe and Co M absorption edges while for energies outside of this range the distinction between $\sigma_{-}$ and $\sigma_{+}$ polarizations vanishes. 
The bottom panel shows $\Delta_{\text{IFE}}$, i.e., the difference between the opto-magnetic constants for the two helicities. 
To understand this spectrum, we can first note that, in the absence of spin magnetization in the alloy, the IFE spectra for $\sigma_{-}$ and $\sigma_{+}$ polarizations would be identical but with opposite sign. 
This feature can be approximately recognized in Suppl.~Fig.~\ref{Suppl_10_Figure}. 
When there is spin magnetization present, the exchange splitting of the 3d valence bands leads to an energy-dependent shift in the individual IFE spectra, such that
$\mathcal{K}^{\text{IFE}}_{\sigma_{+}}(\omega) \neq -\mathcal{K}^{\text{IFE}}_{\sigma_{-}}(\omega)$. 
A pronounced difference between $\mathcal{K}^{\text{IFE}}_{\sigma_{+}}(\omega)$ and $-\mathcal{K}^{\text{IFE}}_{\sigma_{-}}(\omega)$ occurs at 53--\SI{55}{eV} (top panel of Suppl.~Fig.~\ref{Suppl_10_Figure}), i.e., where the $\text{3p} \to \text{3d}$ resonance occurs for Fe in the calculations. 
Around 58--\SI{60}{eV}, a similar difference is expected to appear near the Co 3p absorption edge. 
However, there will be an overlap of the M edges of Fe and Co in this photon energy range which can partially cancel out each other. 
In addition, we mention that previously, it was found that the \textit{ab initio} computed XUV spectra of 3d transition metals compared well with measured spectra, but the computed energy onset of the M$_{3,2}$ edge differs from the measured onset position by about \SI{2.5}{eV} \cite{Turgut2016}. 
This difference occurs because the \textit{ab initio} calculated energy positions of the semi-core levels can deviate from the real energy positions and, in addition, there is an effect of the core hole in the XUV excitation which can lead to a shift in the binding energy of the semi-core levels \cite{Valencia2010,Turgut2016}. 
In the main article, the $\Delta_{\text{IFE}}$ spectrum has therefore been shifted by \SI{+2.5}{eV} to align with the experimental onset of the Fe M edge.
We remark further that the computed IFE spectrum is valid in the quasi-static approximation, i.e., assuming that the computed IFE response is modulated only by the envelope of the XUV pulse. 
The employed second-order response theory is furthermore expected to be valid for reasonable, but not too high laser fluences. 

%%%%%%%%%%%%%%%%%%%%%%%%%%%%%%%%%%%%%%%%%%%%%%%%%%%%%%%
\section{7. Comparison to circularly polarized VIS/NIR-pumping}
%%%%%%%%%%%%%%%%%%%%%%%%%%%%%%%%%%%%%%%%%%%%%%%%%%%%%%%

\begin{figure}
	\includegraphics[width=0.75\linewidth]{Suppl_11_Figure}
	\caption{(a) Laser-induced magnetization dynamics ($M/M_0$) driven by circularly polarized fs NIR-pulses (1.55\,eV) with opposite helicities ($\sigma_{\pm}$) for two different fluences 5 and \SI{11}{mJ/cm^2}. 
	(b) Maximum demagnetization amplitudes ($D$) measured upon $\sigma_{\pm}$-pumping (upper panels) and their difference $\Delta M$ (lower panels) as a function of incident fluence.
	All values are normalized to the equilibrium magnetization in the unexcited state.
	\label{Suppl_11_Figure}}
\end{figure}

In order to \textit{directly} compare the XUV-IFE to the IFE induced by VIS/NIR-pumping of the valence band, additional time-resolved measurements were performed on the very same GdFeCo sample that was used in the XUV-studies presented in the main article.
Employing a table-top time-resolved MOKE setup (1.55\,eV pump -- 3.1\,eV probe), the laser-induced demagnetization dynamics were systematically studied as a function of excitation fluence and helicity.
The pump pulses were circularly polarized using a $\lambda/4$-wave plate, alternating their helicity between $\sigma_-$ and $\sigma_+$ with $\leq0.1$\,$\%$ fluence deviation.
The results are shown in Suppl.~Fig.~\ref{Suppl_11_Figure}, revealing an helicity-dependent effect on the order of $\approx 1$\,$\%$ of the equilibrium magnetization.
The magnitude of the effect was obtained in the same way as for the XUV-IFE studies presented in the main article, i.e., from the difference $\Delta M$ between the maximum demagnetization amplitudes $D$ induced by the opposite helicities $\sigma_{\pm}$.
These measurements show that the helicity-dependent effects induced by VIS/NIR light are by an order of magnitude smaller compared to the XUV-IFE measured on the same GdFeCo sample, strongly suggesting that the IFE scales indeed with the spin-orbit coupling, which is much stronger for the core-levels than for valence band electrons (e.g., \SI{1.1}{eV} vs.\ \SI{65}{meV} in case of Fe 3p and 3d electrons, respectively \cite{Vijayakumar1996}).

% If in two-column mode, this environment will change to single-column
% format so that long equations can be displayed. Use
% sparingly.
%\begin{widetext}
% put long equation here
%\end{widetext}

% figures should be put into the text as floats.
% Use the graphics or graphicx packages (distributed with LaTeX2e)
% and the \includegraphics macro defined in those packages.
% See the LaTeX Graphics Companion by Michel Goosens, Sebastian Rahtz,
% and Frank Mittelbach for instance.
%
% Here is an example of the general form of a figure:
% Fill in the caption in the braces of the \caption{} command. Put the label
% that you will use with \ref{} command in the braces of the \label{} command.
% Use the figure* environment if the figure should span across the
% entire page. There is no need to do explicit centering.

% \begin{figure}
% \includegraphics{}%
% \caption{\label{}}
% \end{figure}

% Surround figure environment with turnpage environment for landscape
% figure
% \begin{turnpage}
% \begin{figure}
% \includegraphics{}%
% \caption{\label{}}
% \end{figure}
% \end{turnpage}

% tables should appear as floats within the text
%
% Here is an example of the general form of a table:
% Fill in the caption in the braces of the \caption{} command. Put the label
% that you will use with \ref{} command in the braces of the \label{} command.
% Insert the column specifiers (l, r, c, d, etc.) in the empty braces of the
% \begin{tabular}{} command.
% The ruledtabular enviroment adds doubled rules to table and sets a
% reasonable default table settings.
% Use the table* environment to get a full-width table in two-column
% Add \usepackage{longtable} and the longtable (or longtable*}
% environment for nicely formatted long tables. Or use the the [H]
% placement option to break a long table (with less control than 
% in longtable).
% \begin{table}%[H] add [H] placement to break table across pages
% \caption{\label{}}
% \begin{ruledtabular}
% \begin{tabular}{}
% Lines of table here ending with \\
% \end{tabular}
% \end{ruledtabular}
% \end{table}

% Surround table environment with turnpage environment for landscape
% table
% \begin{turnpage}
% \begin{table}
% \caption{\label{}}
% \begin{ruledtabular}
% \begin{tabular}{}
% \end{tabular}
% \end{ruledtabular}
% \end{table}
% \end{turnpage}

% Specify following sections are appendices. Use \appendix* if there
% only one appendix.
%\appendix
%\section{}

% If you have acknowledgments, this puts in the proper section head.
\begin{acknowledgments}
% put your acknowledgments here.
\end{acknowledgments}

% Create the reference section using BibTeX:
%\bibliography{paper.bib}

%apsrev4-2.bst 2019-01-14 (MD) hand-edited version of apsrev4-1.bst
%Control: key (0)
%Control: author (72) initials jnrlst
%Control: editor formatted (1) identically to author
%Control: production of article title (-1) disabled
%Control: page (0) single
%Control: year (1) truncated
%Control: production of eprint (0) enabled
\begin{thebibliography}{25}%
	\makeatletter
	\providecommand \@ifxundefined [1]{%
	 \@ifx{#1\undefined}
	}%
	\providecommand \@ifnum [1]{%
	 \ifnum #1\expandafter \@firstoftwo
	 \else \expandafter \@secondoftwo
	 \fi
	}%
	\providecommand \@ifx [1]{%
	 \ifx #1\expandafter \@firstoftwo
	 \else \expandafter \@secondoftwo
	 \fi
	}%
	\providecommand \natexlab [1]{#1}%
	\providecommand \enquote  [1]{``#1''}%
	\providecommand \bibnamefont  [1]{#1}%
	\providecommand \bibfnamefont [1]{#1}%
	\providecommand \citenamefont [1]{#1}%
	\providecommand \href@noop [0]{\@secondoftwo}%
	\providecommand \href [0]{\begingroup \@sanitize@url \@href}%
	\providecommand \@href[1]{\@@startlink{#1}\@@href}%
	\providecommand \@@href[1]{\endgroup#1\@@endlink}%
	\providecommand \@sanitize@url [0]{\catcode `\\12\catcode `\$12\catcode
	  `\&12\catcode `\#12\catcode `\^12\catcode `\_12\catcode `\%12\relax}%
	\providecommand \@@startlink[1]{}%
	\providecommand \@@endlink[0]{}%
	\providecommand \url  [0]{\begingroup\@sanitize@url \@url }%
	\providecommand \@url [1]{\endgroup\@href {#1}{\urlprefix }}%
	\providecommand \urlprefix  [0]{URL }%
	\providecommand \Eprint [0]{\href }%
	\providecommand \doibase [0]{https://doi.org/}%
	\providecommand \selectlanguage [0]{\@gobble}%
	\providecommand \bibinfo  [0]{\@secondoftwo}%
	\providecommand \bibfield  [0]{\@secondoftwo}%
	\providecommand \translation [1]{[#1]}%
	\providecommand \BibitemOpen [0]{}%
	\providecommand \bibitemStop [0]{}%
	\providecommand \bibitemNoStop [0]{.\EOS\space}%
	\providecommand \EOS [0]{\spacefactor3000\relax}%
	\providecommand \BibitemShut  [1]{\csname bibitem#1\endcsname}%
	\let\auto@bib@innerbib\@empty
	%</preamble>
	\bibitem [{\citenamefont {Capotondi}\ \emph {et~al.}(2015)\citenamefont
	  {Capotondi}, \citenamefont {Pedersoli}, \citenamefont {Bencivenga},
	  \citenamefont {Manfredda}, \citenamefont {Mahne}, \citenamefont {Raimondi},
	  \citenamefont {Svetina}, \citenamefont {Zangrando}, \citenamefont
	  {Demidovich}, \citenamefont {Nikolov}, \citenamefont {Danailov},
	  \citenamefont {Masciovecchio},\ and\ \citenamefont
	  {Kiskinova}}]{Capotondi:ig5025}%
	  \BibitemOpen
	  \bibfield  {author} {\bibinfo {author} {\bibfnamefont {F.}~\bibnamefont
	  {Capotondi}}, \bibinfo {author} {\bibfnamefont {E.}~\bibnamefont
	  {Pedersoli}}, \bibinfo {author} {\bibfnamefont {F.}~\bibnamefont
	  {Bencivenga}}, \bibinfo {author} {\bibfnamefont {M.}~\bibnamefont
	  {Manfredda}}, \bibinfo {author} {\bibfnamefont {N.}~\bibnamefont {Mahne}},
	  \bibinfo {author} {\bibfnamefont {L.}~\bibnamefont {Raimondi}}, \bibinfo
	  {author} {\bibfnamefont {C.}~\bibnamefont {Svetina}}, \bibinfo {author}
	  {\bibfnamefont {M.}~\bibnamefont {Zangrando}}, \bibinfo {author}
	  {\bibfnamefont {A.}~\bibnamefont {Demidovich}}, \bibinfo {author}
	  {\bibfnamefont {I.}~\bibnamefont {Nikolov}}, \bibinfo {author} {\bibfnamefont
	  {M.}~\bibnamefont {Danailov}}, \bibinfo {author} {\bibfnamefont
	  {C.}~\bibnamefont {Masciovecchio}},\ and\ \bibinfo {author} {\bibfnamefont
	  {M.}~\bibnamefont {Kiskinova}},\ }\href
	  {https://doi.org/10.1107/S1600577515004919} {\bibfield  {journal} {\bibinfo
	  {journal} {Journal of Synchrotron Radiation}\ }\textbf {\bibinfo {volume}
	  {22}},\ \bibinfo {pages} {544} (\bibinfo {year} {2015})}\BibitemShut
	  {NoStop}%
	\bibitem [{\citenamefont {Cinquegrana}\ \emph {et~al.}(2021)\citenamefont
	  {Cinquegrana}, \citenamefont {Demidovich}, \citenamefont {Kurdi},
	  \citenamefont {Nikolov}, \citenamefont {Sigalotti}, \citenamefont {Susnjar},\
	  and\ \citenamefont {Danailov}}]{Cinquegrana2021}%
	  \BibitemOpen
	  \bibfield  {author} {\bibinfo {author} {\bibfnamefont {P.}~\bibnamefont
	  {Cinquegrana}}, \bibinfo {author} {\bibfnamefont {A.}~\bibnamefont
	  {Demidovich}}, \bibinfo {author} {\bibfnamefont {G.}~\bibnamefont {Kurdi}},
	  \bibinfo {author} {\bibfnamefont {I.}~\bibnamefont {Nikolov}}, \bibinfo
	  {author} {\bibfnamefont {P.}~\bibnamefont {Sigalotti}}, \bibinfo {author}
	  {\bibfnamefont {P.}~\bibnamefont {Susnjar}},\ and\ \bibinfo {author}
	  {\bibfnamefont {M.~B.}\ \bibnamefont {Danailov}},\ }\href
	  {https://doi.org/10.1017/hpl.2021.49} {\bibfield  {journal} {\bibinfo
	  {journal} {High Power Laser Science and Engineering}\ }\textbf {\bibinfo
	  {volume} {9}},\ \bibinfo {pages} {e61} (\bibinfo {year} {2021})}\BibitemShut
	  {NoStop}%
	\bibitem [{\citenamefont {Finetti}\ \emph {et~al.}(2017)\citenamefont
	  {Finetti}, \citenamefont {H\"oppner}, \citenamefont {Allaria}, \citenamefont
	  {Callegari}, \citenamefont {Capotondi}, \citenamefont {Cinquegrana},
	  \citenamefont {Coreno}, \citenamefont {Cucini}, \citenamefont {Danailov},
	  \citenamefont {Demidovich}, \citenamefont {De~Ninno}, \citenamefont
	  {Di~Fraia}, \citenamefont {Feifel}, \citenamefont {Ferrari}, \citenamefont
	  {Fr\"ohlich}, \citenamefont {Gauthier}, \citenamefont {Golz}, \citenamefont
	  {Grazioli}, \citenamefont {Kai}, \citenamefont {Kurdi}, \citenamefont
	  {Mahne}, \citenamefont {Manfredda}, \citenamefont {Medvedev}, \citenamefont
	  {Nikolov}, \citenamefont {Pedersoli}, \citenamefont {Penco}, \citenamefont
	  {Plekan}, \citenamefont {Prandolini}, \citenamefont {Prince}, \citenamefont
	  {Raimondi}, \citenamefont {Rebernik}, \citenamefont {Riedel}, \citenamefont
	  {Roussel}, \citenamefont {Sigalotti}, \citenamefont {Squibb}, \citenamefont
	  {Stojanovic}, \citenamefont {Stranges}, \citenamefont {Svetina},
	  \citenamefont {Tanikawa}, \citenamefont {Teubner}, \citenamefont {Tkachenko},
	  \citenamefont {Toleikis}, \citenamefont {Zangrando}, \citenamefont {Ziaja},
	  \citenamefont {Tavella},\ and\ \citenamefont
	  {Giannessi}}]{PhysRevX.7.021043}%
	  \BibitemOpen
	  \bibfield  {author} {\bibinfo {author} {\bibfnamefont {P.}~\bibnamefont
	  {Finetti}}, \bibinfo {author} {\bibfnamefont {H.}~\bibnamefont {H\"oppner}},
	  \bibinfo {author} {\bibfnamefont {E.}~\bibnamefont {Allaria}}, \bibinfo
	  {author} {\bibfnamefont {C.}~\bibnamefont {Callegari}}, \bibinfo {author}
	  {\bibfnamefont {F.}~\bibnamefont {Capotondi}}, \bibinfo {author}
	  {\bibfnamefont {P.}~\bibnamefont {Cinquegrana}}, \bibinfo {author}
	  {\bibfnamefont {M.}~\bibnamefont {Coreno}}, \bibinfo {author} {\bibfnamefont
	  {R.}~\bibnamefont {Cucini}}, \bibinfo {author} {\bibfnamefont {M.~B.}\
	  \bibnamefont {Danailov}}, \bibinfo {author} {\bibfnamefont {A.}~\bibnamefont
	  {Demidovich}}, \bibinfo {author} {\bibfnamefont {G.}~\bibnamefont
	  {De~Ninno}}, \bibinfo {author} {\bibfnamefont {M.}~\bibnamefont {Di~Fraia}},
	  \bibinfo {author} {\bibfnamefont {R.}~\bibnamefont {Feifel}}, \bibinfo
	  {author} {\bibfnamefont {E.}~\bibnamefont {Ferrari}}, \bibinfo {author}
	  {\bibfnamefont {L.}~\bibnamefont {Fr\"ohlich}}, \bibinfo {author}
	  {\bibfnamefont {D.}~\bibnamefont {Gauthier}}, \bibinfo {author}
	  {\bibfnamefont {T.}~\bibnamefont {Golz}}, \bibinfo {author} {\bibfnamefont
	  {C.}~\bibnamefont {Grazioli}}, \bibinfo {author} {\bibfnamefont
	  {Y.}~\bibnamefont {Kai}}, \bibinfo {author} {\bibfnamefont {G.}~\bibnamefont
	  {Kurdi}}, \bibinfo {author} {\bibfnamefont {N.}~\bibnamefont {Mahne}},
	  \bibinfo {author} {\bibfnamefont {M.}~\bibnamefont {Manfredda}}, \bibinfo
	  {author} {\bibfnamefont {N.}~\bibnamefont {Medvedev}}, \bibinfo {author}
	  {\bibfnamefont {I.~P.}\ \bibnamefont {Nikolov}}, \bibinfo {author}
	  {\bibfnamefont {E.}~\bibnamefont {Pedersoli}}, \bibinfo {author}
	  {\bibfnamefont {G.}~\bibnamefont {Penco}}, \bibinfo {author} {\bibfnamefont
	  {O.}~\bibnamefont {Plekan}}, \bibinfo {author} {\bibfnamefont {M.~J.}\
	  \bibnamefont {Prandolini}}, \bibinfo {author} {\bibfnamefont {K.~C.}\
	  \bibnamefont {Prince}}, \bibinfo {author} {\bibfnamefont {L.}~\bibnamefont
	  {Raimondi}}, \bibinfo {author} {\bibfnamefont {P.}~\bibnamefont {Rebernik}},
	  \bibinfo {author} {\bibfnamefont {R.}~\bibnamefont {Riedel}}, \bibinfo
	  {author} {\bibfnamefont {E.}~\bibnamefont {Roussel}}, \bibinfo {author}
	  {\bibfnamefont {P.}~\bibnamefont {Sigalotti}}, \bibinfo {author}
	  {\bibfnamefont {R.}~\bibnamefont {Squibb}}, \bibinfo {author} {\bibfnamefont
	  {N.}~\bibnamefont {Stojanovic}}, \bibinfo {author} {\bibfnamefont
	  {S.}~\bibnamefont {Stranges}}, \bibinfo {author} {\bibfnamefont
	  {C.}~\bibnamefont {Svetina}}, \bibinfo {author} {\bibfnamefont
	  {T.}~\bibnamefont {Tanikawa}}, \bibinfo {author} {\bibfnamefont
	  {U.}~\bibnamefont {Teubner}}, \bibinfo {author} {\bibfnamefont
	  {V.}~\bibnamefont {Tkachenko}}, \bibinfo {author} {\bibfnamefont
	  {S.}~\bibnamefont {Toleikis}}, \bibinfo {author} {\bibfnamefont
	  {M.}~\bibnamefont {Zangrando}}, \bibinfo {author} {\bibfnamefont
	  {B.}~\bibnamefont {Ziaja}}, \bibinfo {author} {\bibfnamefont
	  {F.}~\bibnamefont {Tavella}},\ and\ \bibinfo {author} {\bibfnamefont
	  {L.}~\bibnamefont {Giannessi}},\ }\href
	  {https://doi.org/10.1103/PhysRevX.7.021043} {\bibfield  {journal} {\bibinfo
	  {journal} {Phys. Rev. X}\ }\textbf {\bibinfo {volume} {7}},\ \bibinfo {pages}
	  {021043} (\bibinfo {year} {2017})}\BibitemShut {NoStop}%
	\bibitem [{\citenamefont {Allaria}\ \emph {et~al.}(2014)\citenamefont
	  {Allaria}, \citenamefont {Diviacco}, \citenamefont {Callegari}, \citenamefont
	  {Finetti}, \citenamefont {Mahieu}, \citenamefont {Viefhaus}, \citenamefont
	  {Zangrando}, \citenamefont {De~Ninno}, \citenamefont {Lambert}, \citenamefont
	  {Ferrari}, \citenamefont {Buck}, \citenamefont {Ilchen}, \citenamefont
	  {Vodungbo}, \citenamefont {Mahne}, \citenamefont {Svetina}, \citenamefont
	  {Spezzani}, \citenamefont {Di~Mitri}, \citenamefont {Penco}, \citenamefont
	  {Trov\'o}, \citenamefont {Fawley}, \citenamefont {Rebernik}, \citenamefont
	  {Gauthier}, \citenamefont {Grazioli}, \citenamefont {Coreno}, \citenamefont
	  {Ressel}, \citenamefont {Kivim\"aki}, \citenamefont {Mazza}, \citenamefont
	  {Glaser}, \citenamefont {Scholz}, \citenamefont {Seltmann}, \citenamefont
	  {Gessler}, \citenamefont {Gr\"unert}, \citenamefont {De~Fanis}, \citenamefont
	  {Meyer}, \citenamefont {Knie}, \citenamefont {Moeller}, \citenamefont
	  {Raimondi}, \citenamefont {Capotondi}, \citenamefont {Pedersoli},
	  \citenamefont {Plekan}, \citenamefont {Danailov}, \citenamefont {Demidovich},
	  \citenamefont {Nikolov}, \citenamefont {Abrami}, \citenamefont {Gautier},
	  \citenamefont {L\"uning}, \citenamefont {Zeitoun},\ and\ \citenamefont
	  {Giannessi}}]{PhysRevX.4.041040}%
	  \BibitemOpen
	  \bibfield  {author} {\bibinfo {author} {\bibfnamefont {E.}~\bibnamefont
	  {Allaria}}, \bibinfo {author} {\bibfnamefont {B.}~\bibnamefont {Diviacco}},
	  \bibinfo {author} {\bibfnamefont {C.}~\bibnamefont {Callegari}}, \bibinfo
	  {author} {\bibfnamefont {P.}~\bibnamefont {Finetti}}, \bibinfo {author}
	  {\bibfnamefont {B.}~\bibnamefont {Mahieu}}, \bibinfo {author} {\bibfnamefont
	  {J.}~\bibnamefont {Viefhaus}}, \bibinfo {author} {\bibfnamefont
	  {M.}~\bibnamefont {Zangrando}}, \bibinfo {author} {\bibfnamefont
	  {G.}~\bibnamefont {De~Ninno}}, \bibinfo {author} {\bibfnamefont
	  {G.}~\bibnamefont {Lambert}}, \bibinfo {author} {\bibfnamefont
	  {E.}~\bibnamefont {Ferrari}}, \bibinfo {author} {\bibfnamefont
	  {J.}~\bibnamefont {Buck}}, \bibinfo {author} {\bibfnamefont {M.}~\bibnamefont
	  {Ilchen}}, \bibinfo {author} {\bibfnamefont {B.}~\bibnamefont {Vodungbo}},
	  \bibinfo {author} {\bibfnamefont {N.}~\bibnamefont {Mahne}}, \bibinfo
	  {author} {\bibfnamefont {C.}~\bibnamefont {Svetina}}, \bibinfo {author}
	  {\bibfnamefont {C.}~\bibnamefont {Spezzani}}, \bibinfo {author}
	  {\bibfnamefont {S.}~\bibnamefont {Di~Mitri}}, \bibinfo {author}
	  {\bibfnamefont {G.}~\bibnamefont {Penco}}, \bibinfo {author} {\bibfnamefont
	  {M.}~\bibnamefont {Trov\'o}}, \bibinfo {author} {\bibfnamefont {W.~M.}\
	  \bibnamefont {Fawley}}, \bibinfo {author} {\bibfnamefont {P.~R.}\
	  \bibnamefont {Rebernik}}, \bibinfo {author} {\bibfnamefont {D.}~\bibnamefont
	  {Gauthier}}, \bibinfo {author} {\bibfnamefont {C.}~\bibnamefont {Grazioli}},
	  \bibinfo {author} {\bibfnamefont {M.}~\bibnamefont {Coreno}}, \bibinfo
	  {author} {\bibfnamefont {B.}~\bibnamefont {Ressel}}, \bibinfo {author}
	  {\bibfnamefont {A.}~\bibnamefont {Kivim\"aki}}, \bibinfo {author}
	  {\bibfnamefont {T.}~\bibnamefont {Mazza}}, \bibinfo {author} {\bibfnamefont
	  {L.}~\bibnamefont {Glaser}}, \bibinfo {author} {\bibfnamefont
	  {F.}~\bibnamefont {Scholz}}, \bibinfo {author} {\bibfnamefont
	  {J.}~\bibnamefont {Seltmann}}, \bibinfo {author} {\bibfnamefont
	  {P.}~\bibnamefont {Gessler}}, \bibinfo {author} {\bibfnamefont
	  {J.}~\bibnamefont {Gr\"unert}}, \bibinfo {author} {\bibfnamefont
	  {A.}~\bibnamefont {De~Fanis}}, \bibinfo {author} {\bibfnamefont
	  {M.}~\bibnamefont {Meyer}}, \bibinfo {author} {\bibfnamefont
	  {A.}~\bibnamefont {Knie}}, \bibinfo {author} {\bibfnamefont {S.~P.}\
	  \bibnamefont {Moeller}}, \bibinfo {author} {\bibfnamefont {L.}~\bibnamefont
	  {Raimondi}}, \bibinfo {author} {\bibfnamefont {F.}~\bibnamefont {Capotondi}},
	  \bibinfo {author} {\bibfnamefont {E.}~\bibnamefont {Pedersoli}}, \bibinfo
	  {author} {\bibfnamefont {O.}~\bibnamefont {Plekan}}, \bibinfo {author}
	  {\bibfnamefont {M.~B.}\ \bibnamefont {Danailov}}, \bibinfo {author}
	  {\bibfnamefont {A.}~\bibnamefont {Demidovich}}, \bibinfo {author}
	  {\bibfnamefont {I.}~\bibnamefont {Nikolov}}, \bibinfo {author} {\bibfnamefont
	  {A.}~\bibnamefont {Abrami}}, \bibinfo {author} {\bibfnamefont
	  {J.}~\bibnamefont {Gautier}}, \bibinfo {author} {\bibfnamefont
	  {J.}~\bibnamefont {L\"uning}}, \bibinfo {author} {\bibfnamefont
	  {P.}~\bibnamefont {Zeitoun}},\ and\ \bibinfo {author} {\bibfnamefont
	  {L.}~\bibnamefont {Giannessi}},\ }\href
	  {https://doi.org/10.1103/PhysRevX.4.041040} {\bibfield  {journal} {\bibinfo
	  {journal} {Phys. Rev. X}\ }\textbf {\bibinfo {volume} {4}},\ \bibinfo {pages}
	  {041040} (\bibinfo {year} {2014})}\BibitemShut {NoStop}%
	\bibitem [{\citenamefont {Gutt}\ \emph {et~al.}(2017)\citenamefont {Gutt},
	  \citenamefont {Sant}, \citenamefont {Ksenzov}, \citenamefont {Capotondi},
	  \citenamefont {Pedersoli}, \citenamefont {Raimondi}, \citenamefont {Nikolov},
	  \citenamefont {Kiskinova}, \citenamefont {Jaiswal}, \citenamefont {Jakob},
	  \citenamefont {Kl{\"a}ui}, \citenamefont {Zabel},\ and\ \citenamefont
	  {Pietsch}}]{Gutt2017}%
	  \BibitemOpen
	  \bibfield  {author} {\bibinfo {author} {\bibfnamefont {C.}~\bibnamefont
	  {Gutt}}, \bibinfo {author} {\bibfnamefont {T.}~\bibnamefont {Sant}}, \bibinfo
	  {author} {\bibfnamefont {D.}~\bibnamefont {Ksenzov}}, \bibinfo {author}
	  {\bibfnamefont {F.}~\bibnamefont {Capotondi}}, \bibinfo {author}
	  {\bibfnamefont {E.}~\bibnamefont {Pedersoli}}, \bibinfo {author}
	  {\bibfnamefont {L.}~\bibnamefont {Raimondi}}, \bibinfo {author}
	  {\bibfnamefont {I.~P.}\ \bibnamefont {Nikolov}}, \bibinfo {author}
	  {\bibfnamefont {M.}~\bibnamefont {Kiskinova}}, \bibinfo {author}
	  {\bibfnamefont {S.}~\bibnamefont {Jaiswal}}, \bibinfo {author} {\bibfnamefont
	  {G.}~\bibnamefont {Jakob}}, \bibinfo {author} {\bibfnamefont
	  {M.}~\bibnamefont {Kl{\"a}ui}}, \bibinfo {author} {\bibfnamefont
	  {H.}~\bibnamefont {Zabel}},\ and\ \bibinfo {author} {\bibfnamefont
	  {U.}~\bibnamefont {Pietsch}},\ }\href {https://doi.org/10.1063/1.4990650}
	  {\bibfield  {journal} {\bibinfo  {journal} {Structural Dynamics}\ }\textbf
	  {\bibinfo {volume} {4}},\ \bibinfo {pages} {055101} (\bibinfo {year}
	  {2017})}\BibitemShut {NoStop}%
	\bibitem [{\citenamefont {Ziolek}\ \emph {et~al.}(2001)\citenamefont {Ziolek},
	  \citenamefont {Naskrecki}, \citenamefont {Lorenc}, \citenamefont {Karolczak},
	  \citenamefont {Kubicki},\ and\ \citenamefont {Maciejewski}}]{ZIOLEK2001467}%
	  \BibitemOpen
	  \bibfield  {author} {\bibinfo {author} {\bibfnamefont {M.}~\bibnamefont
	  {Ziolek}}, \bibinfo {author} {\bibfnamefont {R.}~\bibnamefont {Naskrecki}},
	  \bibinfo {author} {\bibfnamefont {M.}~\bibnamefont {Lorenc}}, \bibinfo
	  {author} {\bibfnamefont {J.}~\bibnamefont {Karolczak}}, \bibinfo {author}
	  {\bibfnamefont {J.}~\bibnamefont {Kubicki}},\ and\ \bibinfo {author}
	  {\bibfnamefont {A.}~\bibnamefont {Maciejewski}},\ }\href
	  {https://doi.org/https://doi.org/10.1016/S0030-4018(01)01454-7} {\bibfield
	  {journal} {\bibinfo  {journal} {Optics Communications}\ }\textbf {\bibinfo
	  {volume} {197}},\ \bibinfo {pages} {467} (\bibinfo {year}
	  {2001})}\BibitemShut {NoStop}%
	\bibitem [{\citenamefont {Willems}\ \emph {et~al.}(2019)\citenamefont
	  {Willems}, \citenamefont {Sharma}, \citenamefont {v.~Korff~Schmising},
	  \citenamefont {Dewhurst}, \citenamefont {Salemi}, \citenamefont {Schick},
	  \citenamefont {Hessing}, \citenamefont {Str\"uber}, \citenamefont {Engel},\
	  and\ \citenamefont {Eisebitt}}]{PhysRevLett.122.217202}%
	  \BibitemOpen
	  \bibfield  {author} {\bibinfo {author} {\bibfnamefont {F.}~\bibnamefont
	  {Willems}}, \bibinfo {author} {\bibfnamefont {S.}~\bibnamefont {Sharma}},
	  \bibinfo {author} {\bibfnamefont {C.}~\bibnamefont {v.~Korff~Schmising}},
	  \bibinfo {author} {\bibfnamefont {J.~K.}\ \bibnamefont {Dewhurst}}, \bibinfo
	  {author} {\bibfnamefont {L.}~\bibnamefont {Salemi}}, \bibinfo {author}
	  {\bibfnamefont {D.}~\bibnamefont {Schick}}, \bibinfo {author} {\bibfnamefont
	  {P.}~\bibnamefont {Hessing}}, \bibinfo {author} {\bibfnamefont
	  {C.}~\bibnamefont {Str\"uber}}, \bibinfo {author} {\bibfnamefont {W.~D.}\
	  \bibnamefont {Engel}},\ and\ \bibinfo {author} {\bibfnamefont
	  {S.}~\bibnamefont {Eisebitt}},\ }\href
	  {https://doi.org/10.1103/PhysRevLett.122.217202} {\bibfield  {journal}
	  {\bibinfo  {journal} {Phys. Rev. Lett.}\ }\textbf {\bibinfo {volume} {122}},\
	  \bibinfo {pages} {217202} (\bibinfo {year} {2019})}\BibitemShut {NoStop}%
	\bibitem [{\citenamefont {Abrudan}\ \emph {et~al.}(2015)\citenamefont
	  {Abrudan}, \citenamefont {Br{\"u}ssing}, \citenamefont {Salikhov},
	  \citenamefont {Meermann}, \citenamefont {Radu}, \citenamefont {Ryll},
	  \citenamefont {Radu},\ and\ \citenamefont {Zabel}}]{alice}%
	  \BibitemOpen
	  \bibfield  {author} {\bibinfo {author} {\bibfnamefont {R.}~\bibnamefont
	  {Abrudan}}, \bibinfo {author} {\bibfnamefont {F.}~\bibnamefont
	  {Br{\"u}ssing}}, \bibinfo {author} {\bibfnamefont {R.}~\bibnamefont
	  {Salikhov}}, \bibinfo {author} {\bibfnamefont {J.}~\bibnamefont {Meermann}},
	  \bibinfo {author} {\bibfnamefont {I.}~\bibnamefont {Radu}}, \bibinfo {author}
	  {\bibfnamefont {H.}~\bibnamefont {Ryll}}, \bibinfo {author} {\bibfnamefont
	  {F.}~\bibnamefont {Radu}},\ and\ \bibinfo {author} {\bibfnamefont
	  {H.}~\bibnamefont {Zabel}},\ }\href {https://doi.org/10.1063/1.4921716}
	  {\bibfield  {journal} {\bibinfo  {journal} {Review of Scientific
	  Instruments}\ }\textbf {\bibinfo {volume} {86}},\ \bibinfo {pages} {063902}
	  (\bibinfo {year} {2015})}\BibitemShut {NoStop}%
	\bibitem [{\citenamefont {Kachel}\ \emph {et~al.}(2015)\citenamefont {Kachel},
	  \citenamefont {Eggenstein},\ and\ \citenamefont {Follath}}]{Kachel:ie5135}%
	  \BibitemOpen
	  \bibfield  {author} {\bibinfo {author} {\bibfnamefont {T.}~\bibnamefont
	  {Kachel}}, \bibinfo {author} {\bibfnamefont {F.}~\bibnamefont {Eggenstein}},\
	  and\ \bibinfo {author} {\bibfnamefont {R.}~\bibnamefont {Follath}},\ }\href
	  {https://doi.org/10.1107/S1600577515010826} {\bibfield  {journal} {\bibinfo
	  {journal} {Journal of Synchrotron Radiation}\ }\textbf {\bibinfo {volume}
	  {22}},\ \bibinfo {pages} {1301} (\bibinfo {year} {2015})}\BibitemShut
	  {NoStop}%
	\bibitem [{\citenamefont {von Korff~Schmising}\ \emph
	  {et~al.}(2020)\citenamefont {von Korff~Schmising}, \citenamefont {Willems},
	  \citenamefont {Sharma}, \citenamefont {Yao}, \citenamefont {Borchert},
	  \citenamefont {Hennecke}, \citenamefont {Schick}, \citenamefont {Radu},
	  \citenamefont {Str{\"u}ber}, \citenamefont {Engel}, \citenamefont {Shokeen},
	  \citenamefont {Buck}, \citenamefont {Bagschik}, \citenamefont {Viefhaus},
	  \citenamefont {Hartmann}, \citenamefont {Manschwetus}, \citenamefont
	  {Grunewald}, \citenamefont {D{\"u}sterer}, \citenamefont {Jal}, \citenamefont
	  {Vodungbo}, \citenamefont {L{\"u}ning},\ and\ \citenamefont
	  {Eisebitt}}]{app10217580}%
	  \BibitemOpen
	  \bibfield  {author} {\bibinfo {author} {\bibfnamefont {C.}~\bibnamefont {von
	  Korff~Schmising}}, \bibinfo {author} {\bibfnamefont {F.}~\bibnamefont
	  {Willems}}, \bibinfo {author} {\bibfnamefont {S.}~\bibnamefont {Sharma}},
	  \bibinfo {author} {\bibfnamefont {K.}~\bibnamefont {Yao}}, \bibinfo {author}
	  {\bibfnamefont {M.}~\bibnamefont {Borchert}}, \bibinfo {author}
	  {\bibfnamefont {M.}~\bibnamefont {Hennecke}}, \bibinfo {author}
	  {\bibfnamefont {D.}~\bibnamefont {Schick}}, \bibinfo {author} {\bibfnamefont
	  {I.}~\bibnamefont {Radu}}, \bibinfo {author} {\bibfnamefont {C.}~\bibnamefont
	  {Str{\"u}ber}}, \bibinfo {author} {\bibfnamefont {D.~W.}\ \bibnamefont
	  {Engel}}, \bibinfo {author} {\bibfnamefont {V.}~\bibnamefont {Shokeen}},
	  \bibinfo {author} {\bibfnamefont {J.}~\bibnamefont {Buck}}, \bibinfo {author}
	  {\bibfnamefont {K.}~\bibnamefont {Bagschik}}, \bibinfo {author}
	  {\bibfnamefont {J.}~\bibnamefont {Viefhaus}}, \bibinfo {author}
	  {\bibfnamefont {G.}~\bibnamefont {Hartmann}}, \bibinfo {author}
	  {\bibfnamefont {B.}~\bibnamefont {Manschwetus}}, \bibinfo {author}
	  {\bibfnamefont {S.}~\bibnamefont {Grunewald}}, \bibinfo {author}
	  {\bibfnamefont {S.}~\bibnamefont {D{\"u}sterer}}, \bibinfo {author}
	  {\bibfnamefont {E.}~\bibnamefont {Jal}}, \bibinfo {author} {\bibfnamefont
	  {B.}~\bibnamefont {Vodungbo}}, \bibinfo {author} {\bibfnamefont
	  {J.}~\bibnamefont {L{\"u}ning}},\ and\ \bibinfo {author} {\bibfnamefont
	  {S.}~\bibnamefont {Eisebitt}},\ }\href {https://doi.org/10.3390/app10217580}
	  {\bibfield  {journal} {\bibinfo  {journal} {Applied Sciences}\ }\textbf
	  {\bibinfo {volume} {10}},\ \bibinfo {pages} {7580} (\bibinfo {year}
	  {2020})}\BibitemShut {NoStop}%
	\bibitem [{\citenamefont {Hessing}(2021)}]{hessing_phd_2021}%
	  \BibitemOpen
	  \bibfield  {author} {\bibinfo {author} {\bibfnamefont {P.}~\bibnamefont
	  {Hessing}},\ }\emph {\bibinfo {title} {Interference-based spectroscopy with
	  XUV radiation}},\ \href {https://doi.org/10.14279/depositonce-11864}
	  {\bibinfo {type} {Doctoral thesis}},\ \bibinfo  {school} {Technische
	  Universit{\"a}t Berlin}, \bibinfo {address} {Berlin} (\bibinfo {year}
	  {2021})\BibitemShut {NoStop}%
	\bibitem [{\citenamefont {Radu}\ \emph {et~al.}(2011)\citenamefont {Radu},
	  \citenamefont {Vahaplar}, \citenamefont {Stamm}, \citenamefont {Kachel},
	  \citenamefont {Pontius}, \citenamefont {Durr}, \citenamefont {Ostler},
	  \citenamefont {Barker}, \citenamefont {Evans}, \citenamefont {Chantrell},
	  \citenamefont {Tsukamoto}, \citenamefont {Itoh}, \citenamefont {Kirilyuk},
	  \citenamefont {Rasing},\ and\ \citenamefont {Kimel}}]{Radu2011}%
	  \BibitemOpen
	  \bibfield  {author} {\bibinfo {author} {\bibfnamefont {I.}~\bibnamefont
	  {Radu}}, \bibinfo {author} {\bibfnamefont {K.}~\bibnamefont {Vahaplar}},
	  \bibinfo {author} {\bibfnamefont {C.}~\bibnamefont {Stamm}}, \bibinfo
	  {author} {\bibfnamefont {T.}~\bibnamefont {Kachel}}, \bibinfo {author}
	  {\bibfnamefont {N.}~\bibnamefont {Pontius}}, \bibinfo {author} {\bibfnamefont
	  {H.~A.}\ \bibnamefont {Durr}}, \bibinfo {author} {\bibfnamefont {T.~A.}\
	  \bibnamefont {Ostler}}, \bibinfo {author} {\bibfnamefont {J.}~\bibnamefont
	  {Barker}}, \bibinfo {author} {\bibfnamefont {R.~F.~L.}\ \bibnamefont
	  {Evans}}, \bibinfo {author} {\bibfnamefont {R.~W.}\ \bibnamefont
	  {Chantrell}}, \bibinfo {author} {\bibfnamefont {A.}~\bibnamefont
	  {Tsukamoto}}, \bibinfo {author} {\bibfnamefont {A.}~\bibnamefont {Itoh}},
	  \bibinfo {author} {\bibfnamefont {A.}~\bibnamefont {Kirilyuk}}, \bibinfo
	  {author} {\bibfnamefont {T.}~\bibnamefont {Rasing}},\ and\ \bibinfo {author}
	  {\bibfnamefont {A.~V.}\ \bibnamefont {Kimel}},\ }\href
	  {https://doi.org/10.1038/nature09901} {\bibfield  {journal} {\bibinfo
	  {journal} {Nature}\ }\textbf {\bibinfo {volume} {472}},\ \bibinfo {pages}
	  {205} (\bibinfo {year} {2011})}\BibitemShut {NoStop}%
	\bibitem [{\citenamefont {Evans}\ \emph {et~al.}(2014)\citenamefont {Evans},
	  \citenamefont {Fan}, \citenamefont {Chureemart}, \citenamefont {Ostler},
	  \citenamefont {Ellis},\ and\ \citenamefont {Chantrell}}]{Evans2014}%
	  \BibitemOpen
	  \bibfield  {author} {\bibinfo {author} {\bibfnamefont {R.~F.~L.}\
	  \bibnamefont {Evans}}, \bibinfo {author} {\bibfnamefont {W.~J.}\ \bibnamefont
	  {Fan}}, \bibinfo {author} {\bibfnamefont {P.}~\bibnamefont {Chureemart}},
	  \bibinfo {author} {\bibfnamefont {T.~A.}\ \bibnamefont {Ostler}}, \bibinfo
	  {author} {\bibfnamefont {M.~O.~A.}\ \bibnamefont {Ellis}},\ and\ \bibinfo
	  {author} {\bibfnamefont {R.~W.}\ \bibnamefont {Chantrell}},\ }\href
	  {https://doi.org/10.1088/0953-8984/26/10/103202} {\bibfield  {journal}
	  {\bibinfo  {journal} {Journal of Physics: Condensed Matter}\ }\textbf
	  {\bibinfo {volume} {26}},\ \bibinfo {pages} {103202} (\bibinfo {year}
	  {2014})}\BibitemShut {NoStop}%
	\bibitem [{\citenamefont {\textsc{vampire}~software package}()}]{vampire-url}%
	  \BibitemOpen
	  \bibfield  {author} {\bibinfo {author} {\bibnamefont
	  {\textsc{vampire}~software package}},\ }\href@noop {} {}\bibinfo {note}
	  {Version 5. Available from \url{https://vampire.york.ac.uk/}}\BibitemShut
	  {NoStop}%
	\bibitem [{\citenamefont {Ostler}\ \emph {et~al.}(2012)\citenamefont {Ostler},
	  \citenamefont {Barker}, \citenamefont {Evans}, \citenamefont {Chantrell},
	  \citenamefont {Atxitia}, \citenamefont {Chubykalo-Fesenko}, \citenamefont
	  {El~Moussaoui}, \citenamefont {Le~Guyader}, \citenamefont {Mengotti},
	  \citenamefont {Heyderman}, \citenamefont {Nolting}, \citenamefont
	  {Tsukamoto}, \citenamefont {Itoh}, \citenamefont {Afanasiev}, \citenamefont
	  {Ivanov}, \citenamefont {Kalashnikova}, \citenamefont {Vahaplar},
	  \citenamefont {Mentink}, \citenamefont {Kirilyuk}, \citenamefont {Rasing},\
	  and\ \citenamefont {Kimel}}]{Ostler2012}%
	  \BibitemOpen
	  \bibfield  {author} {\bibinfo {author} {\bibfnamefont {T.~A.}\ \bibnamefont
	  {Ostler}}, \bibinfo {author} {\bibfnamefont {J.}~\bibnamefont {Barker}},
	  \bibinfo {author} {\bibfnamefont {R.~F.~L.}\ \bibnamefont {Evans}}, \bibinfo
	  {author} {\bibfnamefont {R.~W.}\ \bibnamefont {Chantrell}}, \bibinfo {author}
	  {\bibfnamefont {U.}~\bibnamefont {Atxitia}}, \bibinfo {author} {\bibfnamefont
	  {O.}~\bibnamefont {Chubykalo-Fesenko}}, \bibinfo {author} {\bibfnamefont
	  {S.}~\bibnamefont {El~Moussaoui}}, \bibinfo {author} {\bibfnamefont
	  {L.}~\bibnamefont {Le~Guyader}}, \bibinfo {author} {\bibfnamefont
	  {E.}~\bibnamefont {Mengotti}}, \bibinfo {author} {\bibfnamefont {L.~J.}\
	  \bibnamefont {Heyderman}}, \bibinfo {author} {\bibfnamefont {F.}~\bibnamefont
	  {Nolting}}, \bibinfo {author} {\bibfnamefont {A.}~\bibnamefont {Tsukamoto}},
	  \bibinfo {author} {\bibfnamefont {A.}~\bibnamefont {Itoh}}, \bibinfo {author}
	  {\bibfnamefont {D.}~\bibnamefont {Afanasiev}}, \bibinfo {author}
	  {\bibfnamefont {B.~A.}\ \bibnamefont {Ivanov}}, \bibinfo {author}
	  {\bibfnamefont {A.~M.}\ \bibnamefont {Kalashnikova}}, \bibinfo {author}
	  {\bibfnamefont {K.}~\bibnamefont {Vahaplar}}, \bibinfo {author}
	  {\bibfnamefont {J.}~\bibnamefont {Mentink}}, \bibinfo {author} {\bibfnamefont
	  {A.}~\bibnamefont {Kirilyuk}}, \bibinfo {author} {\bibfnamefont
	  {T.}~\bibnamefont {Rasing}},\ and\ \bibinfo {author} {\bibfnamefont {A.~V.}\
	  \bibnamefont {Kimel}},\ }\href {https://doi.org/10.1038/ncomms1666}
	  {\bibfield  {journal} {\bibinfo  {journal} {Nature Communications}\ }\textbf
	  {\bibinfo {volume} {3}},\ \bibinfo {pages} {666} (\bibinfo {year}
	  {2012})}\BibitemShut {NoStop}%
	\bibitem [{\citenamefont {Iacocca}\ \emph {et~al.}(2019)\citenamefont
	  {Iacocca}, \citenamefont {Liu}, \citenamefont {Reid}, \citenamefont {Fu},
	  \citenamefont {Ruta}, \citenamefont {Granitzka}, \citenamefont {Jal},
	  \citenamefont {Bonetti}, \citenamefont {Gray}, \citenamefont {Graves},
	  \citenamefont {Kukreja}, \citenamefont {Chen}, \citenamefont {Higley},
	  \citenamefont {Chase}, \citenamefont {Le~Guyader}, \citenamefont {Hirsch},
	  \citenamefont {Ohldag}, \citenamefont {Schlotter}, \citenamefont {Dakovski},
	  \citenamefont {Coslovich}, \citenamefont {Hoffmann}, \citenamefont {Carron},
	  \citenamefont {Tsukamoto}, \citenamefont {Kirilyuk}, \citenamefont {Kimel},
	  \citenamefont {Rasing}, \citenamefont {St{\"o}hr}, \citenamefont {Evans},
	  \citenamefont {Ostler}, \citenamefont {Chantrell}, \citenamefont {Hoefer},
	  \citenamefont {Silva},\ and\ \citenamefont {D{\"u}rr}}]{Iacocca2019}%
	  \BibitemOpen
	  \bibfield  {author} {\bibinfo {author} {\bibfnamefont {E.}~\bibnamefont
	  {Iacocca}}, \bibinfo {author} {\bibfnamefont {T.-M.}\ \bibnamefont {Liu}},
	  \bibinfo {author} {\bibfnamefont {A.~H.}\ \bibnamefont {Reid}}, \bibinfo
	  {author} {\bibfnamefont {Z.}~\bibnamefont {Fu}}, \bibinfo {author}
	  {\bibfnamefont {S.}~\bibnamefont {Ruta}}, \bibinfo {author} {\bibfnamefont
	  {P.~W.}\ \bibnamefont {Granitzka}}, \bibinfo {author} {\bibfnamefont
	  {E.}~\bibnamefont {Jal}}, \bibinfo {author} {\bibfnamefont {S.}~\bibnamefont
	  {Bonetti}}, \bibinfo {author} {\bibfnamefont {A.~X.}\ \bibnamefont {Gray}},
	  \bibinfo {author} {\bibfnamefont {C.~E.}\ \bibnamefont {Graves}}, \bibinfo
	  {author} {\bibfnamefont {R.}~\bibnamefont {Kukreja}}, \bibinfo {author}
	  {\bibfnamefont {Z.}~\bibnamefont {Chen}}, \bibinfo {author} {\bibfnamefont
	  {D.~J.}\ \bibnamefont {Higley}}, \bibinfo {author} {\bibfnamefont
	  {T.}~\bibnamefont {Chase}}, \bibinfo {author} {\bibfnamefont
	  {L.}~\bibnamefont {Le~Guyader}}, \bibinfo {author} {\bibfnamefont
	  {K.}~\bibnamefont {Hirsch}}, \bibinfo {author} {\bibfnamefont
	  {H.}~\bibnamefont {Ohldag}}, \bibinfo {author} {\bibfnamefont {W.~F.}\
	  \bibnamefont {Schlotter}}, \bibinfo {author} {\bibfnamefont {G.~L.}\
	  \bibnamefont {Dakovski}}, \bibinfo {author} {\bibfnamefont {G.}~\bibnamefont
	  {Coslovich}}, \bibinfo {author} {\bibfnamefont {M.~C.}\ \bibnamefont
	  {Hoffmann}}, \bibinfo {author} {\bibfnamefont {S.}~\bibnamefont {Carron}},
	  \bibinfo {author} {\bibfnamefont {A.}~\bibnamefont {Tsukamoto}}, \bibinfo
	  {author} {\bibfnamefont {A.}~\bibnamefont {Kirilyuk}}, \bibinfo {author}
	  {\bibfnamefont {A.~V.}\ \bibnamefont {Kimel}}, \bibinfo {author}
	  {\bibfnamefont {T.}~\bibnamefont {Rasing}}, \bibinfo {author} {\bibfnamefont
	  {J.}~\bibnamefont {St{\"o}hr}}, \bibinfo {author} {\bibfnamefont {R.~F.~L.}\
	  \bibnamefont {Evans}}, \bibinfo {author} {\bibfnamefont {T.}~\bibnamefont
	  {Ostler}}, \bibinfo {author} {\bibfnamefont {R.~W.}\ \bibnamefont
	  {Chantrell}}, \bibinfo {author} {\bibfnamefont {M.~A.}\ \bibnamefont
	  {Hoefer}}, \bibinfo {author} {\bibfnamefont {T.~J.}\ \bibnamefont {Silva}},\
	  and\ \bibinfo {author} {\bibfnamefont {H.~A.}\ \bibnamefont {D{\"u}rr}},\
	  }\href {https://doi.org/10.1038/s41467-019-09577-0} {\bibfield  {journal}
	  {\bibinfo  {journal} {Nature Communications}\ }\textbf {\bibinfo {volume}
	  {10}},\ \bibinfo {pages} {1756} (\bibinfo {year} {2019})}\BibitemShut
	  {NoStop}%
	\bibitem [{\citenamefont {Jiang}\ and\ \citenamefont {Tsai}(2005)}]{Jiang2005}%
	  \BibitemOpen
	  \bibfield  {author} {\bibinfo {author} {\bibfnamefont {L.}~\bibnamefont
	  {Jiang}}\ and\ \bibinfo {author} {\bibfnamefont {H.-L.}\ \bibnamefont
	  {Tsai}},\ }\href {https://doi.org/10.1115/1.2035113} {\bibfield  {journal}
	  {\bibinfo  {journal} {Journal of Heat Transfer}\ }\textbf {\bibinfo {volume}
	  {127}},\ \bibinfo {pages} {1167} (\bibinfo {year} {2005})}\BibitemShut
	  {NoStop}%
	\bibitem [{\citenamefont {Battiato}\ \emph {et~al.}(2014)\citenamefont
	  {Battiato}, \citenamefont {Barbalinardo},\ and\ \citenamefont
	  {Oppeneer}}]{PhysRevB.89.014413}%
	  \BibitemOpen
	  \bibfield  {author} {\bibinfo {author} {\bibfnamefont {M.}~\bibnamefont
	  {Battiato}}, \bibinfo {author} {\bibfnamefont {G.}~\bibnamefont
	  {Barbalinardo}},\ and\ \bibinfo {author} {\bibfnamefont {P.~M.}\ \bibnamefont
	  {Oppeneer}},\ }\href {https://doi.org/10.1103/PhysRevB.89.014413} {\bibfield
	  {journal} {\bibinfo  {journal} {Phys. Rev. B}\ }\textbf {\bibinfo {volume}
	  {89}},\ \bibinfo {pages} {014413} (\bibinfo {year} {2014})}\BibitemShut
	  {NoStop}%
	\bibitem [{\citenamefont {Berritta}\ \emph {et~al.}(2016)\citenamefont
	  {Berritta}, \citenamefont {Mondal}, \citenamefont {Carva},\ and\
	  \citenamefont {Oppeneer}}]{PhysRevLett.117.137203}%
	  \BibitemOpen
	  \bibfield  {author} {\bibinfo {author} {\bibfnamefont {M.}~\bibnamefont
	  {Berritta}}, \bibinfo {author} {\bibfnamefont {R.}~\bibnamefont {Mondal}},
	  \bibinfo {author} {\bibfnamefont {K.}~\bibnamefont {Carva}},\ and\ \bibinfo
	  {author} {\bibfnamefont {P.~M.}\ \bibnamefont {Oppeneer}},\ }\href
	  {https://doi.org/10.1103/PhysRevLett.117.137203} {\bibfield  {journal}
	  {\bibinfo  {journal} {Phys. Rev. Lett.}\ }\textbf {\bibinfo {volume} {117}},\
	  \bibinfo {pages} {137203} (\bibinfo {year} {2016})}\BibitemShut {NoStop}%
	\bibitem [{\citenamefont {Blaha}\ \emph {et~al.}(2018)\citenamefont {Blaha},
	  \citenamefont {Schwarz}, \citenamefont {Madsen}, \citenamefont {Kvasnicka},
	  \citenamefont {Luitz}, \citenamefont {Laskowski}, \citenamefont {Tran},\ and\
	  \citenamefont {Marks}}]{Blaha2018}%
	  \BibitemOpen
	  \bibfield  {author} {\bibinfo {author} {\bibfnamefont {P.}~\bibnamefont
	  {Blaha}}, \bibinfo {author} {\bibfnamefont {K.}~\bibnamefont {Schwarz}},
	  \bibinfo {author} {\bibfnamefont {G.~K.~H.}\ \bibnamefont {Madsen}}, \bibinfo
	  {author} {\bibfnamefont {D.}~\bibnamefont {Kvasnicka}}, \bibinfo {author}
	  {\bibfnamefont {J.}~\bibnamefont {Luitz}}, \bibinfo {author} {\bibfnamefont
	  {R.}~\bibnamefont {Laskowski}}, \bibinfo {author} {\bibfnamefont
	  {F.}~\bibnamefont {Tran}},\ and\ \bibinfo {author} {\bibfnamefont {L.~D.}\
	  \bibnamefont {Marks}},\ }\href@noop {} {\emph {\bibinfo {title} {{WIEN2k, An
	  Augmented Plane Wave + Local Orbitals Program for Calculating Crystal
	  Properties (Karlheinz Schwarz, Techn. Universit{\"{a}}t Wien, Austria)}}}}\
	  (\bibinfo {year} {2018})\BibitemShut {NoStop}%
	\bibitem [{\citenamefont {Kresse}\ and\ \citenamefont
	  {Furthm{\"u}ller}(1996)}]{Kresse1996}%
	  \BibitemOpen
	  \bibfield  {author} {\bibinfo {author} {\bibfnamefont {G.}~\bibnamefont
	  {Kresse}}\ and\ \bibinfo {author} {\bibfnamefont {J.}~\bibnamefont
	  {Furthm{\"u}ller}},\ }\href
	  {https://www.sciencedirect.com/science/article/pii/0927025696000080}
	  {\bibfield  {journal} {\bibinfo  {journal} {Computational Materials Science}\
	  }\textbf {\bibinfo {volume} {6}},\ \bibinfo {pages} {15} (\bibinfo {year}
	  {1996})}\BibitemShut {NoStop}%
	\bibitem [{\citenamefont {Perdew}\ \emph {et~al.}(1996)\citenamefont {Perdew},
	  \citenamefont {Burke},\ and\ \citenamefont {Ernzerhof}}]{Perdew1996}%
	  \BibitemOpen
	  \bibfield  {author} {\bibinfo {author} {\bibfnamefont {J.~P.}\ \bibnamefont
	  {Perdew}}, \bibinfo {author} {\bibfnamefont {K.}~\bibnamefont {Burke}},\ and\
	  \bibinfo {author} {\bibfnamefont {M.}~\bibnamefont {Ernzerhof}},\ }\href
	  {https://doi.org/10.1103/PhysRevLett.77.3865} {\bibfield  {journal} {\bibinfo
	   {journal} {Phys. Rev. Lett.}\ }\textbf {\bibinfo {volume} {77}},\ \bibinfo
	  {pages} {3865} (\bibinfo {year} {1996})}\BibitemShut {NoStop}%
	\bibitem [{\citenamefont {Valencia}\ \emph {et~al.}(2010)\citenamefont
	  {Valencia}, \citenamefont {Kleibert}, \citenamefont {Gaupp}, \citenamefont
	  {Rusz}, \citenamefont {Legut}, \citenamefont {Bansmann}, \citenamefont
	  {Gudat},\ and\ \citenamefont {Oppeneer}}]{Valencia2010}%
	  \BibitemOpen
	  \bibfield  {author} {\bibinfo {author} {\bibfnamefont {S.}~\bibnamefont
	  {Valencia}}, \bibinfo {author} {\bibfnamefont {A.}~\bibnamefont {Kleibert}},
	  \bibinfo {author} {\bibfnamefont {A.}~\bibnamefont {Gaupp}}, \bibinfo
	  {author} {\bibfnamefont {J.}~\bibnamefont {Rusz}}, \bibinfo {author}
	  {\bibfnamefont {D.}~\bibnamefont {Legut}}, \bibinfo {author} {\bibfnamefont
	  {J.}~\bibnamefont {Bansmann}}, \bibinfo {author} {\bibfnamefont
	  {W.}~\bibnamefont {Gudat}},\ and\ \bibinfo {author} {\bibfnamefont {P.~M.}\
	  \bibnamefont {Oppeneer}},\ }\href
	  {https://doi.org/10.1103/PhysRevLett.104.187401} {\bibfield  {journal}
	  {\bibinfo  {journal} {Phys. Rev. Lett.}\ }\textbf {\bibinfo {volume} {104}},\
	  \bibinfo {pages} {187401} (\bibinfo {year} {2010})}\BibitemShut {NoStop}%
	\bibitem [{\citenamefont {Turgut}\ \emph {et~al.}(2016)\citenamefont {Turgut},
	  \citenamefont {Zusin}, \citenamefont {Legut}, \citenamefont {Carva},
	  \citenamefont {Knut}, \citenamefont {Shaw}, \citenamefont {Chen},
	  \citenamefont {Tao}, \citenamefont {Nembach}, \citenamefont {Silva},
	  \citenamefont {Mathias}, \citenamefont {Aeschlimann}, \citenamefont
	  {Oppeneer}, \citenamefont {Kapteyn}, \citenamefont {Murnane},\ and\
	  \citenamefont {Grychtol}}]{Turgut2016}%
	  \BibitemOpen
	  \bibfield  {author} {\bibinfo {author} {\bibfnamefont {E.}~\bibnamefont
	  {Turgut}}, \bibinfo {author} {\bibfnamefont {D.}~\bibnamefont {Zusin}},
	  \bibinfo {author} {\bibfnamefont {D.}~\bibnamefont {Legut}}, \bibinfo
	  {author} {\bibfnamefont {K.}~\bibnamefont {Carva}}, \bibinfo {author}
	  {\bibfnamefont {R.}~\bibnamefont {Knut}}, \bibinfo {author} {\bibfnamefont
	  {J.~M.}\ \bibnamefont {Shaw}}, \bibinfo {author} {\bibfnamefont
	  {C.}~\bibnamefont {Chen}}, \bibinfo {author} {\bibfnamefont {Z.}~\bibnamefont
	  {Tao}}, \bibinfo {author} {\bibfnamefont {H.~T.}\ \bibnamefont {Nembach}},
	  \bibinfo {author} {\bibfnamefont {T.~J.}\ \bibnamefont {Silva}}, \bibinfo
	  {author} {\bibfnamefont {S.}~\bibnamefont {Mathias}}, \bibinfo {author}
	  {\bibfnamefont {M.}~\bibnamefont {Aeschlimann}}, \bibinfo {author}
	  {\bibfnamefont {P.~M.}\ \bibnamefont {Oppeneer}}, \bibinfo {author}
	  {\bibfnamefont {H.~C.}\ \bibnamefont {Kapteyn}}, \bibinfo {author}
	  {\bibfnamefont {M.~M.}\ \bibnamefont {Murnane}},\ and\ \bibinfo {author}
	  {\bibfnamefont {P.}~\bibnamefont {Grychtol}},\ }\href
	  {https://doi.org/10.1103/PhysRevB.94.220408} {\bibfield  {journal} {\bibinfo
	  {journal} {Phys. Rev. B}\ }\textbf {\bibinfo {volume} {94}},\ \bibinfo
	  {pages} {220408(R)} (\bibinfo {year} {2016})}\BibitemShut {NoStop}%
	\bibitem [{\citenamefont {Vijayakumar}\ and\ \citenamefont
	  {Gopinathan}(1996)}]{Vijayakumar1996}%
	  \BibitemOpen
	  \bibfield  {author} {\bibinfo {author} {\bibfnamefont {M.}~\bibnamefont
	  {Vijayakumar}}\ and\ \bibinfo {author} {\bibfnamefont {M.~S.}\ \bibnamefont
	  {Gopinathan}},\ }\href
	  {https://www.sciencedirect.com/science/article/pii/0166128095042970}
	  {\bibfield  {journal} {\bibinfo  {journal} {Journal of Molecular Structure:
	  THEOCHEM}\ }\textbf {\bibinfo {volume} {361}},\ \bibinfo {pages} {15}
	  (\bibinfo {year} {1996})}\BibitemShut {NoStop}%
	\end{thebibliography}%
	

\end{document}
%
% ****** End of file apstemplate.tex ******

