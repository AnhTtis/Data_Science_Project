\section{Limitations}
During the writing of this paper, OpenAI released GPT-4. Our current study does not examine GPT-4 in the context of GEC, but we may perform such analysis as a future expansion of this research.

While our human raters tended to prefer the extensive rewrites and fluency edits that GPT-3 often output, it can be argued that more constrained corrections are sometimes desirable in certain applications of GEC, such as in language education. Opinions about this matter may vary, including among humans who annotate and edit text. Ultimately, we argue that such extensive fluency edits should not be penalized by GEC metrics in contexts in which human raters may well prefer them. A more clear separation of ``fluency'' and ``error correction'' tasks may be one approach to resolving this.