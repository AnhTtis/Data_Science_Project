\section{Evaluation Experiments} \label{sec:experiments}

\subsection{Data and Benchmarks}
We use two benchmark datasets: the development set\footnote{We use the development set (from the W\&I + LOCNESS dataset~\cite{bryant-etal-2019-bea}) because human-written references  are not publicly available for the test set.} of the BEA-2019 Shared Task~\cite{bryant-etal-2019-bea} and the test set of the JFLEG dataset~\cite{jfleg}.
We selected 100 sentences from each dataset for human evaluation.

\subsection{Human Evaluation}
\label{sec:human_eval}
In our study, we use the method from~\citet{sakaguchi-van-durme-2018-efficient}, which efficiently elicits scalar annotations as a probability distribution by combining two approaches: direct assessment and online pairwise ranking aggregation.

For the human evaluation task, we instruct crowdworkers to compare and score the quality of corrections among the following four versions of each sentence: the source sentence (with no corrections), an output from one of the recent models for each benchmark (\citet{yasunaga-etal-2021-lm} for BEA-2019 and \citet{liu-etal-2021-neural} for JFLEG)\footnote{We appreciate that they shared their model outputs.}, which we refer to as baseline systems, a human-written reference correction (included in the original datasets), and the corrections generated by GPT-3 using our best prompt (as seen in Table \ref{tab:final_prompt}).
For each comparison, we assign three crowdworkers from Amazon Mechanical Turk to score the quality of corrections on a scale of 0 (very poor correction) to 10 (excellent correction).\footnote{From our initial experiment, participants received compensation at a rate of \$1.0 per HIT, which roughly translated to an hourly wage of \$15. The inter-annotator agreement, as indicated by Cohen's kappa, varied between 0.53 (for JFLEG) and 0.66 (for BEA-2019).}

\subsection{Results} \label{section:results}
\section{Results}
\label{results}

\begin{figure*}[ht]
    \centering
    \includegraphics[scale=0.15,trim={0 2.5cm 0 5cm},clip]{images/aoi-single_burst}
    \caption{The time average peak Age of Information with burst and \gls{soa} loss values against the dynamic reliability logic for different network topologies.}
    \label{fig:aoi_burst}\vspace{-0.4cm}
\end{figure*}


This paper focuses on both transport layer and application layer metrics to determine the feasibility of dynamic reliability. For this, we have selected the session packet volume, as transmitted, retransmitted, lost and backlogged packets as \glspl{kpi} for the transport layer; while focusing on the \gls{aoi} for the application layer. The \gls{aoi} was chosen as a crucial indicator for the freshness of packets in real-time applications. More specifically, this work adopts the time average peak \gls{aoi} equation \cite{aoi_equation} depicted in Eq. \ref{aoi}, where $\Delta(r_{i+1})$ is the $i$th update at the time it was received at the server, for a session time period of $\tau$.

\begin{equation}
    \label{aoi}
    \gls{aoi}_\tau = \frac{1}{n-1}\sum_{i=1}^{n-1} \Delta(r_{i+1})
\end{equation}

We include a comparison between the vanilla QUIC implementation which does not enjoy the dynamic reliability extension, with a number of dynamic reliability policies. The tests were run a number of times for statistical significance, with the mean value of vanilla implementation used as a baseline for comparison. The topology utilised both random loss and bursty loss to explore the bounds of dynamic reliability. The \gls{soa} loss in the figures correspond to the loss values presented in Table. \ref{tab:path_char}, for ease of comparison between bursty and random loss scenarios.

\subsection{Transport-Layer KPIs}

To analyse the performance gain at the transport layer due to dynamic reliability, the volume of transmitted and backlogged packets is examined. The figures are in the form of boxplots, which take the vanilla implementation as a benchmark, depicted as the red dashed line.

As seen in Fig. \ref{fig:sent_burst}, the loss plays a crucial role in the performance of the reliability policies. The policies under random loss did incredibly well for the networks with a larger capacity, namely \gls{mmwave} and Sub-6~GHz, whereas for burst loss, the lower network capacities had a larger packet reduction. With the increase in burst loss, the behaviour of the set split reliable policies became unpredictable, if a reliable assignment happened to coincide with a burst loss, the number of transmitted packets increases, and vice versa. On the other hand, in smarter policies, such as Loss-Aware, the performance lightly matched the vanilla baseline, as the reliable assignment dominated the session to compensate for a higher burst loss. Not only that but, the burst loss also impacted the variance of the transmitted packets for the policies.

Unsurprisingly, the unreliable focused policy, 80-20 split, outperformed other policies for all topologies in random and bursty loss scenarios, with an approximate reduction of 80\%. That being said, the majority of the policies reduced the transmitted packets on the link by approximately 70\% for random loss, while the reduction started at $\approx 15\%$ and decreased as the loss increased for the burst loss scenario.

The retransmitted and lost packets, not shown due to space limitations, followed the same trend as the transmitted packets for the random loss scenarios. However, for the burst loss scenarios, the larger capacity networks had a lower reduction in the retransmitted and lost packets. This can be seen as a favorable outcome since the lower capacity networks are scarce on resources. It is important to note that the Loss-Aware policy mimicked the vanilla approach as the burst loss increased, signifying the overwhelming appointment of reliable packets in adapting to the harsh burst loss conditions.
 
Alternatively, Fig. \ref{fig:backlog_burst} clearly shows a stark comparison between the policies and loss scenario in the reduction of the backlogged packets. The Loss-Aware policy for random loss scenario reduced the backlogged packets by up to 50\%, beating all other policies by approximately 30\%. Furthermore, it is clear that the unreliability focused policies resulted in the lowest backlog for the session. In comparison, we notice that the burst loss and the backlogged frequency have a positive correlation, where the maximum reduction of the backlogged packets for the policies is at most 20\%. Much like the transmitted packets, the probability of a burst loss occurrence plays a vital role in the number of retransmissions sent and by extension the number of backlogged packets. Thus, we can conclude that the stress placed on the buffer is a result of the reliable packets which is tightly coupled with the congestion on the session. Whereas, unreliable focused policies did not encounter such a phenomenon regardless if it was experiencing a burst loss.


\subsection{Application-Layer KPIs}

The feasibility of dynamic reliability for real-time applications can be determined by the \gls{aoi}, with comparison across different topologies and policies. If we take a strict approach and consider anything below $10$~ms is real-time \cite{real-time}, then all the reliability policies passed that requirement, which is attractive for real-time applications, as shown in Fig. \ref{fig:aoi_burst}. Utilising the median as an estimate of the runs, the policies in the WLAN and Sub-6~GHz topology with random loss floated around $4-5$~ms with negligible difference, while the \gls{aoi} for \gls{mmwave} was $\approx 2-3$~ms. It is clear that the \gls{aoi} and the network capacity have a negative correlation, as the network capacity decreases, the \gls{aoi} increases. The same correlation is extended to the bursty loss scenarios, where \gls{mmwave} dominated the other topologies. That being said, it is crucial to note that the \gls{aoi} for the reliability policies is often slightly better than or equal to the \gls{aoi} of the vanilla implementation, proving that dynamic reliability reduces the congestion of the session at no cost to the \gls{aoi}.


The results of our investigation are shown in Table~\ref{tab:main_results}. When interpreting results, please note that in the BEA-2019 benchmark, the {F$_{0.5}$} score is essentially 0 for the source. The {F$_{0.5}$} score for human reference is not 100 despite the same single reference because the edits are automatically extracted in the evaluation script~\cite{bryant-etal-2019-bea}. To obtain the score for the ``Human'' corrections in the JFLEG dataset, which has multiple references, we randomly selected one human reference file and compared with it the other three references.

Our results show that GPT-3 achieves high performance on the task of GEC according to human evaluations and the JFLEG GLEU scorer. However, performance seems noticeably lower on the {F$_{0.5}$} metric used for BEA-2019. This is intriguing considering its high scores in human evaluation. 

\subsection{Analysis} \label{Analysis}

We believe that the gap in scores is due to the nature of the BEA-2019 dataset and metric, in which there is a single reference for each sentence, generally with what could be described as minimal edits. Meanwhile, there are often multiple valid ways to correct an erroneous sentence, including fluency edits which involve alterations of multiple words while preserving the writer's intended meaning. We hypothesize that GPT-3 often makes such edits, which are acceptable to humans but penalized by the BEA-2019 scorer. Meanwhile, the JFLEG corpus offsets this issue to an extent by prioritizing fluency corrections and incorporating multiple parallel reference sentences for each source sentence, resulting scores which are more in line with the human evaluation.

To understand the behavior of GPT-3 as a grammatical correction model, we examine its outputs in parallel with the source and reference sentences and those of the baseline error correction models. Consider the following example sentences, presented with associated scores from human raters:

\begin{enumerate}
    \item[] \textbf{Source: (6.67)} \\
    Also, you'll meet friendly people who usually ask to you something to be friends and change your telephone number.	

    \item[] \textbf{Baseline: (6.67)} \\
    Also, you'll meet friendly people who usually ask you to be friends and change your telephone number.

    \item[] \textbf{Human: (8.67)} \\
    Also, you'll meet friendly people who usually ask you to be friends and exchange telephone numbers.

    \item[] \textbf{GPT-3: (7.67)} \\
    Also, you'll meet friendly people who usually ask to be friends and exchange phone numbers.
\end{enumerate}
In this case, the baseline system produces ``ask you to... change your telephone number,'' which is a grammatically valid minimal edit. However, the human editor \emph{hypothesizes} that the writer means ``exchange'' and is making the error of using a similar-sounding word in its place. The human's edit based on this hypothesis produces a better sentence than the baseline system.

Meanwhile, GPT-3 successfully produces such an edit as well. While a human relies on their real-world knowledge for this hypothesis, GPT-3 generates text based on the patterns in its large-scale training data, in which phrases like ``meeting friendly people'' are more associated with ``exchanging'' phone numbers than ``changing'' them. Nevertheless, GPT-3 seems able to ``hypothesize'' edits as part of GEC text completion tasks, allowing it to make more liberal and fluent changes than the minimal edits favored by the baselines.

Another example of GPT-3's tendency to make fluency edits can be seen below:

\begin{enumerate}
    \item[] \textbf{Source: (3.67)} \\
    This reminds me of a trip that I have recently been to and the place is Agra.

    \item[] \textbf{Baseline: (7.33)} \\
    This reminds me of a trip that I have recently been on and the place I visited was Agra.

    \item[] \textbf{Human: (6.67)} \\
    This reminds me of a trip that I have recently been to and the place is Agra.

    \item[] \textbf{GPT-3: (9.66)} \\
    This reminds me of a trip I recently took to Agra.
\end{enumerate}

In this case, the edit made by GPT-3 is the most natural and correct sentence, and it was given the highest score by the raters. Meanwhile, the BEA-2019 scorer gives GPT-3 a score of zero and the baseline a score of 1, and JFLEG's scorer gives scores of 24.8 vs. 25.8. Neither reflects the strong preference humans displayed for GPT-3's edit, indicating the limitation of reference-based metrics~\cite{napoles-etal-2016-theres,asano-etal-2017-reference}.