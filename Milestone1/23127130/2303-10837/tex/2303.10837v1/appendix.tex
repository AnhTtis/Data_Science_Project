\appendix
\section{Key Management}
\label{sec:key_management}
Our general system structure assumes the existence of a potentially-compromised aggregation server, which performs the HE-based secure aggregation. Along side this aggregation server, there also exists a trusted key authority server that generates and distributes HE keys and related crypto context files to authenticated parties (as described previously in Algorithm~\ref{alg:fedml-fhe}). We assume there is no collusion between these two servers.

Moreover, secure computation protocols for more decentralized settings without an aggregation server are also available using cryptographic primitives such as Threshold HE~\cite{aloufi2021computing}, Multi-Key HE~\cite{aloufi2021computing}, and Proxy Re-Encryption~\cite{ateniese2006improved, jin2022secure}. In such settings, secure computation and decryption can be collaboratively performed across multiple parties without the need for a centralized point. We plan to introduce a more decentralized version of FedML-HE in the future. Due to the collaborative nature of such secure computation, the key management will act more as a coordination point instead of a trusted source for key generation. 

\section{FedML-HE Tutorial}
In this section, we provide an easy-to-follow tutorial on how to deploy FedML-HE for real applications\footnote{Detailed Tutorial for MLOps can be found at \url{https://doc.fedml.ai/mlops/user_guide.html}.}.
\label{sec:tutorial}

\begin{figure*}
\includegraphics[width=1\textwidth]{figs/mlops_config.png}
\caption{MLOps Setup Using Server/Client Packages}
\label{fig:mlops_config}
\end{figure*}

\begin{figure*}
\includegraphics[width=1\textwidth]{figs/he_mlops.png}
\caption{System Panel Results of An Example Run Using FedML-HE}
\label{fig:mlops_he}
\end{figure*}

\begin{figure*}
\includegraphics[width=1\textwidth]{figs/plain_mlops.png}
\caption{System Panel Results of An Example Run Using Plaintext FedML}
\label{fig:mlops_plain}
\end{figure*}


