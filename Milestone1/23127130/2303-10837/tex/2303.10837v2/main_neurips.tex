\documentclass{article}


% if you need to pass options to natbib, use, e.g.:
%     \PassOptionsToPackage{numbers, compress}{natbib}
% before loading neurips_2023


% ready for submission
\usepackage{neurips_2023}


% to compile a preprint version, e.g., for submission to arXiv, add add the
% [preprint] option:
%     \usepackage[preprint]{neurips_2023}


% to compile a camera-ready version, add the [final] option, e.g.:
%     \usepackage[final]{neurips_2023}


% to avoid loading the natbib package, add option nonatbib:
%    \usepackage[nonatbib]{neurips_2023}


\usepackage[utf8]{inputenc} % allow utf-8 input
\usepackage[T1]{fontenc}    % use 8-bit T1 fonts
\usepackage{hyperref}       % hyperlinks
\usepackage{url}            % simple URL typesetting
\usepackage{booktabs}       % professional-quality tables
\usepackage{amsfonts}       % blackboard math symbols
\usepackage{nicefrac}       % compact symbols for 1/2, etc.
\usepackage{microtype}      % microtypography
\usepackage{xcolor}         % colors

\usepackage{algorithm}
\usepackage[ruled,algo2e]{algorithm2e}
\usepackage{textcomp}
\def\BibTeX{{\rm B\kern-.05em{\sc i\kern-.025em b}\kern-.08em
    T\kern-.1667em\lower.7ex\hbox{E}\kern-.125emX}}
 % Removes citation information below abstract
%\renewcommand\footnotetextcopyrightpermission[1]{} % removes footnote with conference information in first column
\pagestyle{plain} % removes running headers

\usepackage{algpseudocode}
\usepackage{graphicx}
\usepackage{xspace}
\usepackage{makecell}

\usepackage{enumitem}
\setlist[itemize]{leftmargin=*, noitemsep, nolistsep, topsep=0pt}
\setlist[enumerate]{leftmargin=*, noitemsep, nolistsep, topsep=0pt}

% \usepackage{caption}
% \DeclareCaptionFormat{myformat}{#3}
% \captionsetup[algorithm]{format=myformat}

\setlength{\emergencystretch}{10pt}
\usepackage{hyperref}
\usepackage{enumitem}
\let\proof\relax
\let\endproof\relax
\usepackage{amsmath,amsthm}
%\setlength{\itemsep}{10pt}
%\setlength{\listsep}{0pt}
\usepackage{soul}

\usepackage{pifont}% http://ctan.org/pkg/pifont
\newcommand{\cmark}{\ding{51}}%
\newcommand{\xmark}{\ding{55}}%

\usepackage[compact]{titlesec}
	\titlespacing{\section}{0pt}{2ex}{1ex}
	\titlespacing{\subsection}{0pt}{1ex}{0ex}
	\titlespacing{\subsubsection}{0pt}{0.5ex}{0ex}
%\titlespacing{\section}{0pt}{*0}{*0}
%\titlespacing{\subsection}{0pt}{*0}{*0}
%\titlespacing{\subsubsection}{0pt}{*0}{*0}


\usepackage[
	font=footnotesize,
	labelfont=bf,
	textfont=,
%	justification=centering,
	aboveskip=2pt,
	belowskip=0pt,
]{caption}


%\usepackage{mdwlist}

\usepackage{wrapfig}

\usepackage{subcaption}
\usepackage{geometry}
\geometry{
	left=1in,
	right=1in,
	top=1in, 
	bottom=1in,
% textwidth=15cm,  % llncs has 12.2cm
% textheight=24cm, % llncs has 19.3cm
  heightrounded,   % integer number of lines
  hratio=1:1,      % horizontally centered
  vratio=1:1,      % not vertically centered
}

\setlength{\parskip}{0pt}
\setlength{\parindent}{1em}
\setlength{\parsep}{0pt}
%\setlength{\headsep}{0pt}
\setlength{\topskip}{0pt}
%\setlength{\topmargin}{0pt}
\setlength{\topsep}{0pt}
\setlength{\partopsep}{0pt}

\usepackage[capitalize,noabbrev]{cleveref}

\usepackage{anyfontsize}
\newcommand{\tsref}[1]{\textsection\ref{#1}\xspace}

\usepackage[framemethod=TikZ]{mdframed}
\mdfdefinestyle{myframe}{%
    linecolor=gray!15!white,
    outerlinewidth=0.5pt,
    roundcorner=2pt,
    innertopmargin=2pt,%\baselineskip,
    innerbottommargin=2pt,
    innerrightmargin=2pt,
    innerleftmargin=2pt,
    backgroundcolor=gray!15!white
}

\usepackage[framemethod=TikZ]{mdframed}
\mdfdefinestyle{mydefframe}{%
    linecolor=cyan!10!white,
    outerlinewidth=0.5pt,
    roundcorner=10pt,
    innertopmargin=10pt,%\baselineskip,
    innerbottommargin=\baselineskip,
    innerrightmargin=10pt,
    innerleftmargin=10pt,
    backgroundcolor=cyan!10!white
}

\theoremstyle{plain}
\newtheorem{theorem}{Theorem}[section]
\newtheorem{proposition}[theorem]{Proposition}
\newtheorem{lemma}[theorem]{Lemma}
\newtheorem{corollary}[theorem]{Corollary}
\theoremstyle{definition}
\newtheorem{definition}[theorem]{Definition}
\newtheorem{assumption}[theorem]{Assumption}
\theoremstyle{remark}
\newtheorem{remark}[theorem]{Remark}


\newcommand\review[1]{
{\color{red}
{\textbf{Review:}
{\em#1}}}}
\newcommand\resolve[1]{
{\color{cyan}
{\textbf{Resolve:}
{\em#1}}}}

\newcommand{\weizhao}[1]{{\color{orange} [Weizhao: #1]}}
\newcommand{\chaoyang}[1]{{\color{red} [Chaoyang: #1]}}
\newcommand{\yuhang}[1]{{\color{blue} [Yuhang: #1]}}
\newcommand{\carlee}[1]{{\color{red} [Carlee: #1]}}
\newcommand{\salman}[1]{{\color{red} [Salman: #1]}}
\newcommand{\sri}[1]{{\color{red} [Sri: #1]}}
\newcommand{\ssh}[1]{{\color{blue} [Shanshan: #1]}}

\newcommand{\switch}[1]{%
   \ifthenelse{\equal{#1}{0}}{\renewcommand{\review}[1]{}}{}
   \ifthenelse{\equal{#1}{0}}{\renewcommand{\resolve}[1]{}}{}}
\switch{0}
 \usepackage{comment}
\crefname{section}{\S}{\S\S}

\title{FedHE: An Efficient Homomorphic-Encryption-Based Privacy-Preserving Federated Learning System}


% The \author macro works with any number of authors. There are two commands
% used to separate the names and addresses of multiple authors: \And and \AND.
%
% Using \And between authors leaves it to LaTeX to determine where to break the
% lines. Using \AND forces a line break at that point. So, if LaTeX puts 3 of 4
% authors names on the first line, and the last on the second line, try using
% \AND instead of \And before the third author name.


\author{%
  David S.~Hippocampus\thanks{Use footnote for providing further information
    about author (webpage, alternative address)---\emph{not} for acknowledging
    funding agencies.} \\
  Department of Computer Science\\
  Cranberry-Lemon University\\
  Pittsburgh, PA 15213 \\
  \texttt{hippo@cs.cranberry-lemon.edu} \\
  % examples of more authors
  % \And
  % Coauthor \\
  % Affiliation \\
  % Address \\
  % \texttt{email} \\
  % \AND
  % Coauthor \\
  % Affiliation \\
  % Address \\
  % \texttt{email} \\
  % \And
  % Coauthor \\
  % Affiliation \\
  % Address \\
  % \texttt{email} \\
  % \And
  % Coauthor \\
  % Affiliation \\
  % Address \\
  % \texttt{email} \\
}


\begin{document}
\maketitle


% {
% \color{red}
% Todo:
% \begin{itemize}
%     \item \url{https://docs.google.com/document/d/1YgayY8rShucMbwDQiA7HadP1_F7BjH969lrbH2j7VBM/edit?usp=sharing}
%     \item If multi-key HE is also used, the problem of how to use a model encrypted with a multi-key also becomes non-trivial.
%     \item who does the key management
%     \item security proofs (or any proof sketch) and privacy guarantees discussion 
%     \item effectiveness on Parameter efficiency?
%     \item more model and data poisoning attacks and more datasets
%     \item an anonymous link to code repository
%     \item The Algorithm 1 HE-Based Federated Aggregation must include in the main paper (actually, Figure 1(b) can move to the appendix).
%     \item however, MPC is also used in the form of outsourced MPC for distributed aggregators, and does not suffer from things like drop outs or inter-client synchronisation steps.
% \end{itemize}

% }



Over the past few years, there has been a significant amount of research focused on studying the ReLU activation function, with the aim of achieving neural network convergence through over-parametrization. However, recent developments in the field of Large Language Models (LLMs) have sparked interest in the use of exponential activation functions, specifically in the attention mechanism.

Mathematically, we define the neural function $F: \R^{d \times m} \times  \mathbb{R}^d \rightarrow \mathbb{R}$ using an exponential activation function. Given a set of data points with labels $\{(x_1, y_1), (x_2, y_2), \dots, (x_n, y_n)\} \subset \mathbb{R}^d \times \mathbb{R}$ where $n$ denotes the number of the data. Here $F(W(t),x)$ can be expressed as $F(W(t),x) := \sum_{r=1}^m a_r \exp(\langle w_r, x \rangle)$, where $m$ represents the number of neurons, and $w_r(t)$ are weights at time $t$. It's standard in literature that $a_r$ are the fixed weights and it's never changed during the training. We initialize the weights $W(0) \in \mathbb{R}^{d \times m}$ with random Gaussian distributions, such that $w_r(0) \sim \mathcal{N}(0, I_d)$ and initialize $a_r$ from random sign distribution for each $r \in [m]$.

Using the gradient descent algorithm, we can find a weight $W(T)$ such that $\| F(W(T), X) - y \|_2 \leq \epsilon$ holds with probability $1-\delta$, where $\epsilon \in (0,0.1)$ and $m = \Omega(n^{2+o(1)}\log(n/\delta))$. To optimize the over-parametrization bound $m$, we employ several tight analysis techniques from previous studies [Song and Yang arXiv 2019, Munteanu, Omlor, Song and Woodruff ICML 2022]. 

 



\section{Introduction}

The increasing complexity of source code poses a key challenge to the reliability of large-scale software systems. Software bugs in these systems can lead to safety issues~\cite{bug_safety} for users around the world as well as cause non-negligible financial losses~\cite{bug_loss}. As such, developers have to spend a large amount of time and effort on bug fixing. Consequently, \aprfull (\apr), designed to automatically generate patches to fix software bugs, has attracted wide attention from both academia and industry~\cite{long2016prophet, legoues2012genprog, long2015spr, lou2020can, tufano2018empstudy}. 


To achieve \apr, one popular approach is known as Generate-and-Validate (G\&V)~\cite{qi2015gv, ghanbari2019prapr, lou2020can, le2016hdrepair, legoues2012genprog, wen2018capgen, hua2018sketchfix, martinez2016astor, koyuncu2020fixminder, liu2019tbar, liu2019avatar}, which is typically based on the following pipeline: First, fault localization techniques~\cite{wong2016fl, abreu2007ochiai, zhang2013injecting, papadakis2015metallaxis, li2019deepfl, li2017transforming} are applied to determine the suspicious locations in programs where bugs are likely to exist. Then, the buggy locations are used by the \apr tools to generate a list of patches that replace buggy lines with correct lines. Afterward, each patch is validated against the original test suite to identify any \emph{plausible patches} (i.e., passing all tests in the test suite). Finally, to determine the \emph{correct patches}, developers examine the list of plausible patches to see if any of them can correctly fix the bug. 

Traditional \apr tools can mainly be categorized into heuristic-based~\cite{legoues2012genprog, le2016hdrepair, wen2018capgen}, constraint-based~\cite{mechtaev2016angelix, le2017s3, demacro2014nopol, long2015spr} and \template~\cite{ghanbari2019prapr, hua2018sketchfix, martinez2016astor, liu2019tbar, liu2019avatar}. Among these traditional tools, \template \apr tools~\cite{ghanbari2019prapr, liu2019tbar, benton2020effectiveness} have been able to achieve state-of-the-art results. \Template \apr tools typically leverage pre-defined templates (e.g., adding a nullness check) for bug fixing. However, since these fix templates are typically handcrafted, the number and types of bugs they are able to fix can be limited. 



To address the limitations of traditional \apr, researchers have proposed various \learning \apr tools~\cite{li2020dlfix, chen2018sequencer, jiang2021cure, lutellier2020coconut, zhu2021recoder, ye2022rewardrepair} based on the \nmtfull (\nmt) architecture~\cite{sutskever2014mt} where the input is the buggy code snippets and the goal is to translate the buggy code snippets into a fixed version. To accomplish this, \learning \apr tools require supervised training datasets with pairs of both buggy and fixed code snippets in order to learn how to perform this translation step. These training data are usually obtained by mining historical bug fixes using heuristics/keywords~\cite{dallmeier2007benchmark}, which can be imprecise for identifying bug-fixing commits; even the actual bug-fixing commits can include irrelevant code changes, leading to further pollution in the dataset~\cite{xia2022alpharepair}.
% 
Moreover, it can be hard for such \apr tools to generalize and fix bug types unseen during training. 



To better leverage recent advances in \plmfull{s} (\plm{s}), researchers~\cite{xia2022alpharepair, xia2023repairstudy, kolak2022patch, prenner2021codexws} have directly applied \plm{s} to generate patches without bug-fixing datasets. These \llm-based \apr tools work by either directly generating a complete code function~\cite{prenner2021codexws, xia2023repairstudy} or predict/infill the correct code snippet given its surrounding context~\cite{xia2022alpharepair, xia2023repairstudy}. By directly using \llm{s} that are pre-trained on billions of open-source code snippets, \llm-based \apr tools can achieve state-of-the-art performance on many repair datasets~\cite{xia2022alpharepair}. 


% 
%
%

Traditional \apr tools have long used the insight of the \emph{plastic surgery hypothesis}~\cite{barr2014plastic} where it states that the code ingredients to fix a bug already exist within the same project. Traditional \apr tools have manually designed pattern-~\cite{ghanbari2019prapr, saha2017elixir} or heuristic-based~\cite{jiang2018simfix, legoues2012genprog} approaches to finding and using such relevant code ingredients to generate fixes for bugs. However, the plastic surgery hypothesis has been largely ignored in \llm-based \apr. In fact, \llm provides a unique opportunity to fully automate the plastic surgery hypothesis idea via fine-tuning (learning project-specific information via model updates from the buggy project) and prompting (directly providing relevant code ingredients to the model), and make it directly applicable to different languages (since the \llm{s} are typically multi-lingual).%
Moreover, despite the intensive manual efforts involved, traditional \apr tools still cannot fully leverage project-specific information due to large search space for leveraging/composing existing code ingredients. In contrast, the project-specific information can effectively leveraged by \llm{s} due to their power in code understanding/vectorization, e.g., even partial/imprecise information may still guide \llm{s} in correct patch generation!
 To this end, we ask the question: \emph{How useful is the plastic surgery hypothesis in the era of \plm{s}}?








\mypara{Our Work.} To answer the question, we present \ourtech{\xspace} -- a \llm-based approach that automatically utilizes the plastic surgery hypothesis by systematically combining multiple fine-tuning and prompting strategies for \apr. \ourtech fine-tunes \plm{s} using two novel domain-specific training strategies: \textbf{\epfinetune} -- we fine-tune using the original buggy project by aggressively masking out a high percentage of tokens, which allows \plm to learn project-specific code tokens and programming styles; and \textbf{\rofinetune} -- which only masks out a single continuous code sequence per training sample, allowing the model to get used to the final \csapr task of predicting a single continuous code sequence. Furthermore, we directly leverage the ability for \plm{s} to understand natural language instructions and introduce a novel prompting strategy, \textbf{\idprompting}, which uses information retrieval and static analysis to obtain a list of relevant identifiers for the buggy lines. While such relevant identifiers are critical for fixing some difficult bugs, they may not be seen by the \llm during inference due to limited context window size. Through the use of prompting, we directly tell the model to use these extracted identifiers (relevant code ingredients) to generate the correct code. Finally, to perform repair, we combine all four model variants (including the base model, both fine-tuned models and the base model with prompting) for the final repair.





While our insight of leveraging the plastic surgery hypothesis for \llm-based \apr is generalizable across different types of \plm{s}, to implement \ourtech, we choose a recent \plm{\xspace}, \ctfive~\cite{wang2021codet5}, which is pre-trained on millions of open-source code snippets. \ctfive is an encoder-decoder model trained using \mspfull (\msp) objective where a percentage of tokens are masked out and each continuous masked token sequence is referred to as a masked span. Also, although we only extract relevant identifiers from the current buggy project (since this paper focuses on the plastic surgery hypothesis), our work can be easily extended to obtain other code information (such as relevant statements or functions) from other sources, such as  the massive pre-training corpora~\cite{husain2020codesearchnet} or historical bug-fixing datasets~\cite{jiang2019infer}, which can provide more coding knowledge for \llm{s}. Besides, although we mainly focus on using traditional string comparison algorithms for information retrieval in this paper, these techniques can be easily replaced by other frequency-based retrieval~\cite{robertson2009probabilistic} and neural search (or embedding-based search)~\cite{reimers2019sentence}.
  In summary, this paper makes the following contributions:


%


\begin{itemize}[noitemsep, leftmargin=*, topsep=0pt]
    \item \textbf{Dimension.} This paper is the first to revisit the important plastic surgery hypothesis in the era of \llm{s}. It opens up a new dimension for \llm-based \apr to incorporate previously neglected information from the buggy project itself to boost \apr performance. Furthermore, it demonstrates the promising future of retrieval-based prompting for modern \llm-based \apr.
    \item \textbf{Implementation.} We implement \ourtech based on the recent \ctfive model. We augment the model using two novel fine-tuning strategies: \epfinetune and \rofinetune, along with a novel prompting strategy based on information retrieval and static analysis: \idprompting. We combine the patches generated by all four models together and perform patch ranking to speed up \apr.% 
    \item \textbf{Evaluation Study.} We conduct an extensive evaluation against state-of-the-art \apr tools. On the widely studied \dfj 1.2 and 2.0 datasets~\cite{just2014dfj}, \ourtech is able to achieve the new state-of-the-art results of 89 and 44 correct bug fixes (15 and 8 more than best baseline) respectively.  Furthermore, we perform a broad ablation study to justify our design. \ourtech demonstrates for the first time that the plastic surgery hypothesis can substantially boost \llm-based \apr and advance state-of-the-art \apr, while being fully automated and general. Moreover, even partial/imprecise code ingredients may still effectively guide \llm{s} for \apr!
\end{itemize}


\section{Background on Network Calculus}
\label{sec: background}


\begin{figure*}[tbh]
\centering
\begin{subfigure}[b]{0.3\textwidth}
    \centering
    \includegraphics[width=\linewidth]{images/in-out.png}
    \caption{Arrival and departure data and their relation with delay $d(t)$ and backlog $b(t)$. For a FIFO system, the delay is the horizontal distance between $R(t)$ and $R^*(t)$ but some other multiplexing techniques may shift the data to a later priority, causing a longer delay.}
    \label{fig: data in-out}
\end{subfigure}
\hfill
\begin{subfigure}[b]{0.35\textwidth}
    \centering
    \includegraphics[width=\linewidth]{images/arrival-service.png}
    \caption{Characteristics of an arrival curve and a service curve. From any point of observation, the arriving data never exceeds its arrival curve; the departure data is also never less than the service curve with respect to the data arrival.}
    \label{fig: arrival-service curves}
\end{subfigure}
\hfill
\begin{subfigure}[b]{0.33\textwidth}
    \centering
    \includegraphics[width=\linewidth]{images/bound.png}
    \caption{Delay and backlog bounds of a system. Backlog is the maximum vertical distance between $\alpha(t)$ and $\beta(t)$; FIFO delay is their maximum horizontal distance; but for arbitrary multiplexing, the delay guarantee is when the system clears its buffer, thus it's the intersection of $\alpha(t)$ and $\beta(t)$.}
    \label{fig: system bounds}
\end{subfigure}
\caption{Network calculus framework. We let $R(t)$ and $R^*(t)$ be the arrival and departure data flow of a system; $\alpha(t)$ be the piecewise linear concave arrival curve and $\beta(t)$ be the piecewise linear convex service curve of a system.}
% \hossein{Better to show piece-wise linear concave arrival curve and piece-wise linear convex service curve instead of token-bucket and rate-latency.}}
\end{figure*}

We recall some of the network calculus essentials for a better understanding of the framework used in Saihu. In the following context, we use the following notation: $\mbb{R}^+$ is the set of non-negative real numbers; $[x]_+$ denotes $\max(0, x)$

The data flow is by convention modeled as a left-continuous wide-sense increasing function $R(t): \mbb{R}^+ \mapsto \mbb{R}^+$ with respect to time $t$~\cite{ncbook2001leboudec}. 

A system $\mcal{S}$ receives arrival data described as a cumulative function $R(t)$ and delivers departure data as another cumulative function $R^*(t)$. Figure~\ref{fig: data in-out} illustrates such a system $\mcal{S}$. The benefit of representing a system like this is that we can observe system backlog and delay with such a model. 

\begin{definition}[Backlog and Delay~\cite{ncbook2001leboudec}]
    The backlog of a system at time~$t$ is
    \begin{equation}
        b(t) = R(t) - R^*(t)
    \end{equation}
    
    The virtual delay of a FIFO system at time $t$ is
    \begin{equation}
        d_{FIFO}(t) = \inf \lbp \tau \geq 0 : R(t) \leq R^*(t+\tau) \rbp
    \end{equation}
\end{definition}



The backlog of a system can be viewed as the vertical distance between $R$ and $R^*$. The FIFO (\textit{First-in First-out}) delay is the horizontal distance between $R$ and $R^*$. One may obtain other delay values if the multiplexing technique is not FIFO.

% \begin{figure}
%     \centering
%     \includegraphics[width=0.9\linewidth]{images/in-out.png}
%     \caption{In/out data flow; delay and backlog}
%     \label{fig: data in-out}
% \end{figure}

Since we are interested in the system guarantee instead of a single instance of data flow, we would like to have general bounds to the arrival and departure data flows. Therefore, we define \textit{arrival curve} and \textit{service curve} as the bounds of arrival and departure data flows.

\begin{definition}[Arrival Curve~\cite{ncbook2001leboudec}]
    Given a wide-sense increasing function $\alpha: \mbb{R}^+ \mapsto \mbb{R}^+$, we say that a flow $R(t)$ is $\alpha$-constrained if and only if for all $s \leq t$:
    \begin{equation}
        R(t) - R(s) \leq \alpha(t-s)
    \end{equation}
    We say $R(t)$ has $\alpha$ as an arrival curve.
\end{definition}

\begin{definition}[Service Curve~\cite{ncbook2001leboudec}]
    Given a wide-sense increasing function $\beta: \mbb{R}^+ \mapsto \mbb{R}^+$ and $\beta(0) = 0$. A system $\mcal{S}$ having $R(t)$ and $R^*(t)$ as its arrival and departure flows. We say $\mcal{S}$ offers a service curve $\beta$ if and only if
    \begin{equation}
        R^*(t) \geq (R \otimes \beta)(t) =: \inf_{s \leq t} \lbp R(s) + \beta(t-s) \rbp
    \end{equation}
    where $\otimes$ denotes the min-plus convolution
\end{definition}

Figure~\ref{fig: arrival-service curves} illustrates the arrival and service curves. Any segment of arrival flow $R(t)$ is constrained by arrival curve $\alpha$ and the output curve $R^*(t)$ is always no less than the curve $R\otimes\beta$. As a result, an arrival curve upper bounds the incoming traffic, and a service curve lower bounds the outgoing traffic.

% \begin{figure}
%     \centering
%     \includegraphics[width=\linewidth]{images/arrival-service.png}
%     \caption{Arrival/Service curve}
%     \label{fig: arrival-service curves}
% \end{figure}

We consider 2 special types of curves throughout this paper, \textit{token-bucket} (or sometimes called \textit{leaky-bucket}) curve and \textit{rate-Latency} curve.

\begin{definition}[Token-bucket and Rate-latency~\cite{ncbook2001leboudec}]
    A token-bucket curve $\gamma_{r,b}$ with arrival rate $r$ and burst $b$ is defined as
    \begin{equation}
        \gamma_{r,b}(t) = b + rt
    \end{equation}

    A rate-latency curve $\beta_{R,T}$ with service rate $R$ and latency $T$ is defined as
    \begin{equation}
        \beta_{R,T}(t) = R \lb t - T \rb_+
    \end{equation}
\end{definition}

A token-bucket curve is determined by a burst $b$ and an arrival rate~$r$. Burst represents the maximum possible data volume that can arrive simultaneously, and arrival rate represents the maximum long-term data rate~\cite{bouillard2022tradeoff}.
A rate-latency curve is determined by a latency~$T$ and a service rate~$R$. Latency represents the time a server needs before starting to process the incoming data, and service rate represents the minimum rate to process data after the initial latency.

With the help of arrival and service curves, we can derive delay and backlog bounds for a system $\mcal{S}$ illustrated in Figure~\ref{fig: system bounds}. Suppose a system $\mcal{S}$ has arrival curve $\alpha$ and service curve~$\beta$, its worst-case backlog $b^*$ is the maximum vertical distance between~$\alpha$ and~$\beta$. Similarly, depending on the multiplexing technique applied to the system, its worst-case delay bound $d^*$ is the maximum horizontal distance between $\alpha$ and $\beta$ if $\mcal{S}$ is a FIFO system. If we don't have any information about its multiplexing technique, referred to as arbitrary multiplexing, the best we can say is that when $\alpha$ and $\beta$ intersect each other, where all data has been delivered out of the system. Consequently, the worst-case delay bound for arbitrary multiplexing is the time required for $\mcal{S}$ to clear its buffer.

% \begin{figure}
%     \centering
%     \includegraphics[width=\linewidth]{images/bound.png}
%     \caption{System delay/backlog bounds}
%     \label{fig: system bounds}
% \end{figure}

While a service curve captures the slowest possible output speed of a system, a link's transmission capacity limits the speed as well. Hence, we model this phenomenon using a \textit{greedy shaper} with a sub-additive function $\sigma: \mbb{R}^+ \mapsto \mbb{R}^+$ concatenated with a server. We consider a concatenation as shown in Figure \ref{fig: system}. By convention we assume $\sigma(0) = 0$ and $\beta(t) \leq \sigma(t), \forall t \in \mbb{R}^+$, meaning that the buffer is cleared at the beginning and the service never exceed its physical limitation. With the above definition, such greedy shaper conserves the service provided by the system due to theorem \ref{thm: shaping}.

\begin{figure}[thb]
    \centering
    \includegraphics[width=0.7\linewidth]{images/system.png}
    \caption{Shaping of departure data. A flow that has an arrival curve $\alpha$ feeds into a server with an arrival data flow $R(t)$. The server having service curve $\beta$ takes $R(t)$ and gives a departure data flow $R^*(t)$ to a shaper with shaping function $\sigma$. The shaper takes $R^*(t)$ and shape the data flow as another departure $D(t)$.}
    \label{fig: system}
\end{figure}


\begin{theorem}[Shaping conserves service \cite{ncbook2001leboudec}]
\label{thm: shaping}
Following the system shown in Figure \ref{fig: system}, we have
\begin{equation}
     D = R^* \otimes \sigma \geq \lp R \otimes \beta \rp \otimes \sigma = R \otimes \lp \beta \otimes \sigma \rp = R \otimes \beta
\end{equation}
\end{theorem}

In the following context, we model the shaping function $\sigma$ as a token-bucket curve $\gamma_{C,L}$ with transmission capacity $C$ and the packet size $L$ to capture the link capacity and packetization~\cite{bouillard2022tradeoff}.

%We consider an outsourcing scheme between a client-side data owner $\mathcal{C}$ and a server $\mathcal{S}$, where $\mathcal{S}$ executes a function $f(x): X \rightarrow Y$ on data provided by $\mathcal{C}$. The function $f$ can either belong to $\mathcal{C}$ (e.g., in IaaS/PaaS~\cite{Armbrust2010}), or to the server (e.g., in SaaS/API~\cite{Armbrust2010}). We adopt a more realistic threat model in which the server $\mathcal{S}$ is not fully trustworthy and may be malicious or vulnerable to tampering with the ciphertext $f(x)$. This departs from the traditional semi-honest setting, in which $\mathcal{S}$ is assumed to be honest but curious (HBC) about inferring $\mathcal{C}$'s data privacy. Our scheme should  satisfy the following security properties: {\bf Integrity}: $\mathcal{C}$ could detect an integrity attack when interacting with $\mathcal{S}$ for any input $x$ and ensure the correctness of $y=f(x)$.  {\bf Data Privacy}: $\mathcal{S}$ cannot learn any information about input $x$. 
While the function $f()$ can be made public to both the client and server in many applications~\cite{Occlumency:TEE-deep-learning,VISE-TEE-FHE}, it is still important to note that in certain applications and scenarios~\cite{tramer2018slalom}, preserving the privacy of the function may be a critical concern. {\bf Function Privacy}: If $f()$ is provided by $\mathcal{C}$, $\mathcal{S}$ cannot learn information about $f()$ beyond its size. If $f()$ belongs to $\mathcal{S}$, $\mathcal{C}$ cannot learn more about $f()$ than what is revealed by $y = f(x)$. 

\section{Design}
\label{s:design}
In this section, we will first present the core of our system. Then we present some analysis of the system along with some extensions to address a few practical concerns. We will present details of our cloud implementation separately in the next section.

\subsection{Delivery Based Ordering}
Our solution is composed of three parts. 
\subsubsection{Delivery Clock\\}
\noindent\textbf{What we do.}
Each RB maintains a delivery clock. This delivery clock essentially tracks time relative to when market data was delivered to the participant. We use $DC(i,a)$ to represent delivery clock of participant $i$ at time when trade $(i,a)$ is submitted. Delivery clock is a lexicographical tuple.
\begin{align}
    DC(i,a) = \langle ld(i,a), S(i,a)-D(i, ld(i,a))\rangle.
\end{align}
where $ld(i,a)$ is the latest data point that was delivered to MP$_i$ by time S(i,a), i.e., $D(i,ld(i,a)) \leq S(i,a) < D(i,ld(i,a)+1)$). 
Interval, $S(i,a)-D(i, ld(i,a))$, corresponds to the time that has elapsed since the last delivery and can be measured locally at the RB without requiring any clock synchronization (challenge 1). 

\noindent
\textit{Monotonicity:} Delivery clocks advance monotonically with submission time. As a result, DBO trivially satisfies the causality condition (Equation~\ref{eq:causality}). Further, it incentivizes the participants to submit trades as early as possible. Therefore, \emph{a participant cannot gain any advantage by delaying trades.} %\pg{should this point have a heading of its own}
Finally, we also leverage the monotonic property to overcome challenge 3 (\S\ref{ss:enforcing_ordering}). Figure~\ref{fig:delivery_clock} shows how delivery clock advances with time.

%\pg{I tried to reduce the notation here. I defined delivery clock slightly differently.}

\begin{figure}[t]
\centering
    \includegraphics[width=0.8\columnwidth]{figures/delivery_clock.pdf}
    \caption{\small{\bf Delivery Clock.}}% \pg{Redraw}}% \pg{Eashan see Ranveer's comment}}% \pg{Eashan can you redraw this figure in powerpoint or something.}}}
    \label{fig:delivery_clock}
    \vspace{-2.5mm}
\end{figure}

All incoming trades are marked with the delivery clock at the trade submission time. The ordering buffer uses this delivery clock time to order trades. Formally, the ordering in DBO is given by,  

\vspace{-2mm}
\begin{align}
    O(i,a) = DC(i, a). 
    \label{eq:ordering_with_dc}
\end{align}


\begin{figure}[t]
\centering
    \includegraphics[trim={0 0 0 2mm},clip,width=0.8\columnwidth]{figures/dbo_correct.pdf}
    \vspace{-4mm}
    \caption{\small{{\bf DBO can help correct for late delivery of data.} Delivery of market data to MP$_i$ is lagging behind MP$_j$. There are two trades $(i,a)$ and $(j,b)$ generated in response to the same market data $x$. $(j,b)$ was submitted before $(i,a)$ but
    %, i.e., $S_j(l) < A_i(k)$. 
    response time of $(i,a)$ is less than $(j,b)$.
    %, i.e., $rt_i(k) < rt_j(l)$. 
    In this example, $DC(i,a) (= \langle x, RT(i,a)\rangle) < DC(j,b) (= \langle x, RT(j,b)\rangle)$ and trade $(i,a)$ is correctly ordered ahead of $(j,b)$.}} %Ordering based on the submission time leads to incorrect ordering.}
    %\pg{Correct figure}}
    \label{fig:dbo_correction}
    \vspace{-3mm}
\end{figure}


\noindent\textbf{Why it works.}
When the trigger point of trade $(i,a)$ is indeed the last data point (i.e., $x = TP(i,a) = ld(i, a)$), then, DBO respects condition C2 for LRTF. Figure~\ref{fig:dbo_correction} shows an illustrative example of this.
This is because, the delivery clock directly tracks the response time of $i,a$ in this case and $O(i,a) = DC(i, a) = \langle x, RT(i,a)\rangle$. For a competing trade $(j,b)$ with higher response time, the delivery clock at time of submission will either read $O(j,b) = DC(j, b) = \langle x, RT(j,b)\rangle$ (if S(j,b)<D(j,x+1)) or $DC(j, b) = \langle y, S(j,b)-D(j,y)\rangle$ with $y>x$. In both cases, $O(i,a) < O(j,b)$.


At a high level, in our ordering we are correcting for latency differences in data delivery by using the delivery time of the last data point. When the last data point is not the trigger point for trade $(i,a)$, DBO satisfies the LRTF condition C2, if the following condition holds, 
\begin{align}
    D(i,ld(i,a))-D(i,x) = D(j,ld(i,a))-D(j,x),
    \label{eq:cond_delivery_lrtf}
\end{align}
where $x = TP(i,a)$.  
While it is impossible to ensure that inter-delivery times remain the same for all participants for all points, by pacing data at the RB it is indeed possible to ensure that the above condition is always met.% \radhika{you meant C2 or the above condition?}. \pg{the above condition only}
The main reason why we can meet the above condition is that condition C2 limits that the trigger point $x$ cannot be any arbitrary data point in the past, and that the trigger point must have been delivered recently  $S(i,a)-D(i,x) < \delta$.
%and we only need to ensure same inter-delivery times for. 
In the next subsection, we will show how we can achieve this and solve challenge 2. %\pg{Is this easy to follow?}



%\pg{FIX: say delivery clocks helps correct has static differences in latency. Why are delivery clocks so good on their own, give more intuition and experimentation. Potential things to include, see 6.1. Maybe make a section of.delivery clock on its own. correct the equation here in terms of response time as well.}
%\pg{Should we include results on necessary conditions on delivery times for achieving LRTF. Maybe its a bit of an overkill.}

\noindent
\textit{Remark:} In our cloud experiments, we find that DBO achieves fairness with very high probability. This is because network latency (from CES to any given participant) exhibits temporal correlation in latency especially over  short periods of time. When temporal correlation is high, inter-delivery time at any participant is close to the inter-generation time at the CES. In such cases, condition given by Equation~\ref{eq:cond_delivery_lrtf} is satisfied with high probability.

\noindent
\textbf{Difference with traditional logical clocks:} Logical clocks are commonly used in distributed systems. The most famous ones are lamport clocks~\cite{lamportSeminalPaper} and vector clocks. These clocks can be used for achieving total causal ordering of events. While these clocks can track causality of events, they cannot be used to achieve response time fairness. In particular, these clocks don't say anything about how two competing trades generated using the same market data should be ordered as these two trades have no direct causality relation. Unlike delivery clocks, such logical clocks also have no notion of measuring time between occurrences of two events. Time difference between events is critical to achieve fairnesss for exchanges. 

\noindent\textit{Note:} Several major financial exchanges already rely on heartbeats~\cite{nyse-client} for liveness when traffic is low.


\begin{figure}[t]
\centering
    \includegraphics[width=0.8\columnwidth]{figures/batching_pacing.pdf}
    \vspace{-2mm}
    \caption{\small{\bf Batching and Pacing. Inter-delivery time for consecutive batches is equal to or more than $\delta$.}}% \pg{Redraw}}% \pg{Eashan see Ranveer's comment}}% \pg{Eashan can you redraw this figure in powerpoint or something.}}}
    \label{fig:batching_pacing}
    \vspace{-4.5mm}
\end{figure}

\subsubsection{Batching and Pacing\\}
\noindent
\textbf{What we do.}
In DBO, the CES breaks data into batches. Each new batch contains all data points in the duration $(1+\kappa) \cdot \delta$ after the previous batch. Here $\kappa > 0$. Each release buffer delivers all data points in a batch at the same time. %Two points $x,y$ belonging to the same batch are delivered simultaneously to each participant, i.e., $D(j,y)=D(j,x), \forall j$.
The release buffer delivers batches as quickly as possible while ensuring that the time between delivery of two consecutive batches is atleast $\delta$. Figure~\ref{fig:batching_pacing} shows an illustration of batching. Both batching and pacing increase the delivery time of data points. In the next subsection we will analyze the impact of the two on latency. Note that in the event of queue build up at the RB, since the batch generation rate ($\frac{1}{(1+\kappa) \cdot \delta}$) is slower than the batch dequeue rate($\frac{1}{\delta}$), the queue at the RB eventually gets drained(\S\ref{ss:understanding_latency}).


\noindent
\textbf{Why it works.} With batching and pacing, DBO achieves LRTF. In particular, 
consider a trade $(i,a)$ with response time less than $\delta$. Because of pacing, consecutive batches are separated atleast by $\delta$. This means that the trigger point ($x=TP(i,a)$) must be within the last received batch. The point $ld(i,a)$ is also the last point in this batch and $D(i,ld(i,a)) = D(i,x)$. \emph{With batching and pacing, the delivery clock again directly tracks the response time of $(i,a)$} and $O(i,a) = DC(i,a) = <ld(i,a), RT(i,a)>$.
With batching, for participant $j$, $x$ and $ld(i,a)$ also belong to the same batch $D(j,ld(i,a)) = D(j,x)$.
For a competing trade $(j,b)$ with higher response time, the delivery clock at the time of submission will either read $O(j,b) = DC(j,b)) = \langle ld(i,a)), RT(j,b)\rangle$ (if $(j,b)$ was submitted before the next batch, i.e., $S(j,b) < D(j,ld(i,a)+1)$) or $DC(j, b) = \langle y, S(j,b)-D(j,y)\rangle$ with $y>ld(i,a)$. In both cases, $O(i,a) < O(j,b)$.

\if 0
\begin{figure}[t]
\centering
    \includegraphics[width=0.8\columnwidth, angle = -90]{images/pq_hb.jpg}
    \vspace{-2.5mm}
    \caption{\small{\bf Enforcing the ordering.} \pg{Redraw}}% \pg{Eashan see Ranveer's comment}}% \pg{Eashan can you redraw this figure in powerpoint or something.}}}
    \label{fig:pq_hb}
    \vspace{-2.5mm}
\end{figure}
\fi

\subsubsection{Enforcing the ordering\\}
\label{ss:enforcing_ordering}
OB contains a priority queue where all incoming trades are sorted based on the delivery clock timestamp (Equation~\ref{eq:ordering_with_dc}). A trade $(i,a)$ at the head of the priority queue should be forwarded to the CES only when the OB has received all trades $(j,b)$ with lower ordering $DC(j,b) < DC(i,a)$. 

\noindent
\textit{OB's Heartbeat Handler:} In DBO, each RB sends a heartbeat periodically every $\tau$ seconds to the CES. The heartbeat $(i,h)$, from participant $i$ contains the delivery clock timestamp at the time the heartbeat was generated ($DC(i,h)$). Since data in delivered in order and because delivery clock advances monotonically with time, heartbeat $(i,h)$ tells the OB that it has received all trades from participant $i$ with delivery clock less than $DC(i,h)$. The ordering buffer forwards trade $(i,a)$ if it has received heartbeats from all the participants with delivery clock timestamp higher than $DC(i,a)$. 


\subsection{Understanding DBO}

\subsubsection{Latency, parameter setting and straggler mitigation\\}
\label{ss:understanding_latency}

We will first derive the optimal latency for any ordering system that achieves response time fairness. We will then discuss how DBO compares to  optimal latency. We will also present guidelines for setting parameters and how to mitigate stragglers that can impact latency.

We define latency for trade $(i,a)$, $L(i,a)$, as the sum of latency in delivering data (from generation time) and latency in trade forwarding to the CES (from trade submission time). Formally,
\begin{align}
    L(i,a) = (D(i, x) - G(x)) + (F(i,a) - S(i,a)),\nonumber\\
    L(i,a) = F(i,a) - G(x) - RT(i,a),
    \label{eq:latency_def}
\end{align}
where $x=TP(i,a)$.

\noindent
\textbf{Optimal Latency:} Formally trade $(i,a)$ can only be forwarded to the CES's ME only when the CES has received all potential competing trades $(j,b)$ with lower response times ($RT(j,b) < RT(i,a)$). Let $R(i, x, RT)$ represent the time when the CES receives trade $(i,a)$ whose whose trigger point is x and response time is RT. Formally, 
\begin{align}
    F(i,a) = \max_{j}(R(j, x=TP(i,a), RT=RT(i,a))). 
\end{align}
A subtle point to note here is that even if participant $j$ does not produce any trades, we still need to wait for that participant till $R(j, x=TP(i,a), RT(i,a))$. Before this time, fundamentally the CES cannot be sure that it will not receive a trade from participant $j$ with a lower response time. 

We use $RTT(i, x, RT)$ to represent the sum of raw network latency for point x from CES to MP $_i$ and latency of trade from MP$_i$ to the CES (whose trigger point is x and response time RT).  In the best case scenario for latency (no buffering at any point in the path) we get
\begin{align}
    R(i, x, RT) = G(x) + RTT(i, x, RT) + RT.
\end{align}


Using the above two equations, we can write the following theorem.
\begin{theorem}
For any ordering system that achieves response time fairness, the minimum latency for trade $(i,a)$ is given by,
\begin{align}
    L(i,a) = \max_{j}(RTT(j, x=TP(i,a), RT=RT(i,a))).
\end{align}
\vspace{-2mm}
\label{thm:latency}
\end{theorem}

Put it simply, the above theorem states for achieving response time fairness, the minimum latency is bounded by the maximum round trip time across all participants. This means that fundamentally bad latency for a participant affects the latency of all trades. To achieve low latency consistently, we would like to ensure that latency of all the participants is well behaved majority of the times. How to better achieve this goal is left as a subject for future work.

%This theorem implies that even in cloud settings exchanges should ask for  network latency  

%With a very large number of participants thus pose a 
%\pg{fundamental issue with scalability}

\noindent
\textbf{How does DBO compare with the optimal?} DBO achieves close to optimal latency.  Compared to the optimal, batching and pacing introduce additional delay in delivery of market data points.  Since heartbeats are  generated only periodically they can  introduce an additional delay of $\tau$ at the ordering buffer. We now discuss the delay due to each of these components and how do the parameters $\kappa$, $\delta$ and $\tau$ affect latency. %\pg{Include a table here for parameters?}

\noindent
\textbf{Impact of batching:} Batching can introduce an additional delay of $(1+\kappa)\cdot \delta$ in the worst case. 

\noindent
\textit{Setting $\delta$:} $\delta$ thus presents a trade-off between latency and fairness (how large of a horizon can we pick). The right trade-off really depends on the needs of the exchange. Ideally, the exchange should pick the minimum value of $\delta$ that accommodates the response time of the fastest participants in a race. Our conversations reveal that fastest participants typically respond within a few microseconds and majority of the speed races last 5-10 $\mu s$. For our cloud experiments we  use $\delta = 20 \mu s$.

\begin{figure}[t]
    \centering
    \includegraphics[trim={0 0 0 0mm},clip,width=0.8\linewidth]{images/latency_b+p.pdf}
    \vspace{-5mm}
    \caption{\small{\textbf{Latency in data delivery:} x-axis shows the generation time of the market data. y-axis shows the latency from generation time to data delivery. $\kappa$  governs the average slope of the orange line immediately after latency spike (slope = $\frac{\kappa}{1+\kappa}$}).} %\pg{Include orange line and the base latency. Change labels to DBO and direct-delivery. Slope is $\kappa/(1+kappa)$}}
    %\pg{Eashan: Include the drain rate, make the colored lines thicker and use different linestyles for the three schemes..}}% \pg{Maybe label the drain rate in the figure for S1 and S2.}}
    \label{fig:latency_b+p}
    \vspace{-5mm}
\end{figure}

\noindent
\textbf{Impact of pacing.} Pacing restricts the batch dequeue rate at the RB. When network latency to a participant is not varying, the batch arrival/enqueue rate at the RB ($\frac{1}{(1+\kappa) \cdot \delta}$) is higher than the batch dequeue rate limit ($\frac{1}{\delta}$) and there is no queue build up. However, when network latency to a participant is decreasing (e.g., after a latency spike), batch arrival rate at the RB can exceed the dequeue rate limit leading to a queue build up. The overall queue - dequeue rate can be given by $\text{batch size} \cdot \text{batch rate limit} = 1+\kappa$. Figure~\ref{fig:latency_b+p} shows the impact of batching and pacing on latency in delivery of data in the event of a queue build up. The figure also shows the latency when data is delivered directly (raw network latency). The smaller sawtooths in the batching + pacing are because of batching. The deviation in direct delivery and batching + pacing is because of the rate limit imposed by pacing.

\noindent
\textit{Setting $\kappa$:} Increasing $\kappa$ increases batching delay but also increases the queue drain rate in the event of queue build up due to tail latency spikes. Increasing $\kappa$ thus presents a trade-off between reducing tail latency and increasing average latency. In our experiments we use $\kappa = 0.25$.
 
\noindent\textbf{Impact of heartbeats:} Heartbeats present a trade-off. Too frequent heartbeats can overwhelm the network, the ordering buffer or the release buffer. 
Infrequent heartbeats can increase the time OB has to wait of the participants. In particular, hearbeats can introduce an additional wait time of $\tau$. Note that the number of heartbeats, the OB needs to process increases linearly with the number of participants. In the next section we show how the heartbeat handler can be sharded for scalability.

\noindent\textit{Setting $\tau$:} Ideally we want to pick as low of a value as possible for the heartbeats without overwhelming the system. This number is very much dependent on the capabilities of the network and the processing power of the RB and the OB. In our cloud implementation we use $\tau = 20 \mu s$.

\noindent\textit{A note on latency:} When the network latency to participants is not varying with time, there is no queue build up at the release buffers. In such cases, DBO adds maximum of $((1+\kappa)\cdot \delta) + \tau$ additional latency over the optimal.

\noindent\textbf{Straggler Mitigation and RB/MP failure} In the event a  participant or release buffer crashes, DBO can stall processing trades. Further, the overall system latency also gets impacted when a certain participant is experiencing unusually high network latency (see Theorem~\ref{thm:latency}). Here we have the option to wait for the delayed participant and take a latency hit but not let the fairness be impacted. Ideally, we want to let the system continue with low latency with only the affected participant incurring unfairness. In DBO, we use a simple strategy to mitigate this. Using the heartbeats and the generation time of data points, the OB tracks the round trip latency to each participant. If this latency goes beyond a certain threshold for a participant, then the OB does not wait for heartbeats from such straggler participant before forwarding trades. When the round trip latency goes down, OB again starts waiting for heartbeats from the straggler. In the event of crashes, OB might not hear any heartbeats. If the OB does not hear a heartbeat from a particular participant for the above threshold, then it concludes that round trip latency exceeds the threshold and the OB deems the participant a straggler. 
 
\noindent\textit{OB failure:} In the event, the OB crashes all trades in the priority queue will be lost. System will incur unfairness in such cases. 

%The above strategy is also helpful in controlling overall system latency when a certain participant is experiencing unusually high network latency.


\subsubsection{Is Batching and Pacing necessary?\\}
\textbf{Batching and pacing contribute delays; are they necessary?} The answer is yes. Similar to Lemma~\ref{lemma:inter_delivery_imp}, we can derive the necessary conditions for achieving LRTF. 
\begin{corollary}
When trigger points are unknown, the \textit{necessary} conditions on the delivery processes for achieving response time fairness with any ordering system is given by,
\vspace{-1mm}
\begin{align*}
    \text{If }  D(i,y) - D(i,x) &< \delta, \text{ then},\nonumber\\
    D(i, y) - D(i,x) &= D(i,y) - D(i,x), & \forall i,j.
\end{align*}
\label{cor:inter_delivery_lrtf}
\vspace{-6mm}
\end{corollary}

\begin{proof}
Please see Appendix~\ref{app:cor_inter_delivery_lrtf}.
\end{proof}
\vspace{-1mm}
In contrast to Lemma~\ref{lemma:inter_delivery_imp}, the above condition states that the inter-delivery time of two points should be same across all participants only if they are separated by less than $\delta$ for some participant. Batching and pacing indeed satisfies this, for two points x and y in a batch, the inter-delivery times across all participants is indeed zero and hence equal. For point $x$ and $y$ belonging to different batches, since the inter-delivery time is greater than $\delta$ across all participants, there is no additional contraint on inter-delivery times being equal.
 
\subsubsection{Impact of RB to MP latency\\}
In scenarios where RB and the participant cannot be colocated, DBO can incur unfairness. If this latency is unbounded, then, it might be impossible to achieve fairness. If latency is bounded, however, then DBO provides the following fairness guarantees.

\begin{theorem}
    If round trip network latency from release buffer $i$ to it's corresponding participant is bounded between $B_l(i)$ and $B_h(i)$, then, DBO achieves the following guarantee for ordering trades.
    \begin{align*}
    C3: &\text{ if } TP(i,a)= TP(j,b) = x\\ 
    &\land RT(i,a) < RT(j,b) - (B_h(i)-B_l(j)), \\
    & \land RT(i,a) < \delta - B_h(i),\\
    &\text{ then, }O(i,a) < O(j,b).
\end{align*}
    \label{thm:rb_to_mp_latency}
    \vspace{-5mm}
\end{theorem}

\vspace{-1mm}
\begin{proof}
See Appendix~\ref{app:rb_to_mp_latency}.
\end{proof}
\vspace{-1mm}

Compared to LRTF, the above condition reduces the bound on response time for the faster trade $(i,a)$ to $\delta - B_h(i)$.
Additionally, the above condition states that trades are ordered fairly only when the response time of the faster trade is lower than the response time of the competing trade by atleast the variability in latency ($B_h(i)-B_l(j)$). This theorem essentially states that when RB and MP cannot be colocated, for better fairness we should ensure that latency between them is both consistent (across participants) and the upper bound is small.



\subsubsection{Impact of Losses\\}

Although infrequent, packet losses can occur in cloud environments. Such losses can impact fairness in DBO. However, only the fairness for trades that are lost and trades  whose trigger point is lost is impacted (see Appendix~\ref{app:impact_losses}).



\if 0
\subsubsection{Excessive queing at RB and OB\\}
\pg{This can be cut?}

Even though DBO employs straggler mitigation to limit the latency at the OB, it can build up a large queue if it receives a very large number of trades (little's law). The RB can also overflow in scenarios where the network latency is decreasing (Figure~\ref{fig:latency_b+p}) for a large period of time. 

\noindent
\textbf{RB:} In the event a release buffer's queue fills up (exceeds a certain threshold), to avoid overflow the release buffer forgoes pacing and starts releasing data as fast as possible to reduce the queue. In such cases, the delivery clock advances faster than as dictated by pacing. As a result, trades from such a participant might unfairly get ordered behind. The fairness for trades from other participants remains unaffected. When the queue goes down the RB resumes normal operation.

\noindent
\textbf{OB overflow:} In the event the order buffer's queue fills up, the OB starts releasing trades as fast as possible without waiting for heartbeats from participants. Once the queue goes down, OB resumes normal operation. In such cases, fairness of all trades are impacted. 
\fi

\subsubsection{Thwarting front-running attacks\\}

%Monotonicity of delivery clocks ensures that participants are incentivized to submit trades as early as possible and delaying trades does not offer any competitive advantage.% and participants are incentivized to be honest.
There is a front-running attack possible in our system. In particular, if a participant receives a market data point $x$ through some other way before RB delivers the data point $x$ to the participant then the participant has a competitive advantage. This scenario (though unlikely) is still possible. 

A simple to avoid this is to limit that a participant cannot talk to anyone beyond the CES. 
%\pg{External participants}
However, we would like the participant machine to use other  ``helper'' machines in the cloud, e.g.,  to aid computation. We also want to allow the participants to be able to talk to machines outside the cloud, e.g., to get a news stream. %stream.%\footnote{Participants use external news streams update trading strategies and make trading decisions.} 

In Appendix~\ref{app:front_running}, we show how we can prevent such front running attacks. In our solution, the participant and its helpers cannot communicate with any other participants or their helpers using the cloud network. 
To prevent scenarios where a participant uses a proxy machine outside the cloud to send market data to other  participants (faster than the network), we precisely add additional latency for data being sent outside the cloud.
While our solution introduces latency for data going out, the latency of speed trades remains unaffected.

\if 0

While monotonicity of delivery clocks ensure that participants are incentivized to submit trades as early as possible an delaying trades does offer any competitive advantage, there is still a potential front-running attack possible in our system. In particular, if a participant receives a market data point $x$ through some other way before RB delivers the data point $x$ to the participant then it has a competitive advantage. This scenario though unlikely is still possible.
A simple to avoid this is to limit that participant cannot talk to anyone beyond the CES. 

However, we would like the participant machine to use other  ``helper'' machines in the cloud to aid computation. We also want to allow the participants to be able to talk to machines outside the cloud. Participants do use external news streams and feeds from other exchanges to update trading strategies and make trading decisions. We will discuss fairness with respect to such streams shortly.  

Allowing such communication naively can lead to attacks.
By restricting communication, it is possible to ensure that no participant gets early access to market data %(at the cost of introducing latency in messages from the front-end to helpers outside the cloud)
and thwart such front-running attacks. 

%
%\pg{Which of two alternatives is better?}
%
To this end, we impose two simple constraints on communication. \begin{enumerate*}[label=(\arabic*)]\item A participant machine and its helper machines can communicate with each other freely but they cannot communicate with any other machines in the cloud. This restriction can be imposed easily by cloud providers today using security groups. This restriction ensures that a participant machine cannot get market data from other participant machines in the cloud directly. Next, we will ensure that a participant machine cannot get an earlier market data feed from outside the cloud. 
We will do so by restricting that a participant can only send data point x out of the cloud, when x has been delivered to all participants in the cloud. This way, market data points can only be available outside the cloud once they have been delivered to all the participants.
\item The helper machines cannot send data outside the cloud. Any data (excluding the trade orders) from a participant being sent outside the cloud is tagged by the delivery clock at the RB and buffered at a gateway. The data sent by the participant could potentially be a market data point with id less than or equal to the last point id (first tuple) of the delivery clock time stamp. The gateway thus buffers this data until it is sure that the all data points with id less than the last data point id in the delivery clock time stamp have been delivered. For this purpose, RB's periodically communicate their delivery clock to the gateway. 
%
%A simple way to achieve this is for each RB to send other RBs periodic beacons communicating the status of its delivery clock. This way each RB can maintain a lower bound on the delivery clocks at other RBs. 
\end{enumerate*}
\pg{include this? a bit hand-wavy and not clean. There is one challenge to be solved though. If data delivery to a particular participant is straggling then the gateway buffer can get bloated. It is not necessary for the gateway to wait for such straggler if we disable the incoming data to the straggler. The gateway can identify such stragglers and then disable any data coming from outside the cloud.}

Note that the above solution adds additionaly latency for data being sent outside the cloud. However, the latency of speed trades remains unaffected.
%There are other ways to thwart front-running that impose weaker restrictions on communication or are easier to implement. We chose to present this one for its simplicity.


\fi



\subsubsection{Limtations of DBO: Fairness beyond LRTF\\}
\label{ss:beyond_fairness}

With DBO, it is not guaranteed that trades that do not directly follow the LRTF model (Theorem~\ref{thm:1} and Equation~\ref{eq:cm})are ordered fairly. However, DBO still ensures that fairness for the most latency-sensitive speed trades. While ensuring guaranteed fairness for trades that do not follow the might be impossible, we will discuss potential some solutions.


%This will impose some system challenges. Another challenge is that different participants might be requesting different external streams. 
%


\noindent\textbf{Trades with response time > $\delta$:} DBO does not provide any guarantees for trades with response time greater than $\delta$. %If the inter-delivery times for batches across participants are same then DBO provides response time fairness for such trades. Again achieving the same inter-delivery times for all the batches is impossible. 
In case we have access to synchronized clocks, we can try and ensure (to the extent possible) that batches are indeed delivered at the same time across participants. 
When batches are delivered simultaneously, delivery clocks also get synchronized and DBO simply orders trades in the order of submission time. DBO thus ensures better fairness for such trades (when data is delivered simultaneous) while always guaranteeing LRTF. %\pg{Is this clear?}


%Regardless of whether using clocksync or not for deliverying the data, the performance of DBO for such trades is comparable to 


\noindent\textbf{Generalized compute model for trades:} A trade's submission time might be governed by delivery times of multiple data points. Again in such cases if we have access to synchronized clocks, we can try and ensure simultaneous delivery to the extent possible and achieve better fairness for such trades.


\noindent\textbf{External data streams:} In theory, external data streams like news events or market data from a competing exchange can trigger speed races. While DBO does not delay delivery of such streams to the participants (Appendix~\ref{app:front_running}), as described it does not guarantee fairness with respect to such streams. Existing exchanges do not provide any simultaneous delivery guarantees with respect to such external streams. Such streams typically traverse the internet, and the variability is network latency is substantially higher (order of milliseconds) than the market data stream (order of microseconds). Potentially, the exchange can serialize such external streams with the market data stream and ensure LRTF with respect to such a super stream. Such a serialization might not be trivial. Participants are requesting different data streams. We need to think carefully about what constitutes a fair serialization.
%\pg{Talk about how  further system challenges.}


%\subsubsection{\pg{Miscellaneous, do if time:}}
%\pg {Radhika advidce here would be helpful}

%\pg{1. Impact of clock drift rate, 3. Is batching and pacing necessary 4. Discussion, sharding for scalability, a separate RB for each asset class}













\if 0

\subsubsection{Delivery Clock\\}
Each RB maintains a delivery clock. This delivery clock essentially tracks time relative to when market data was delivered to the participant. We use $DC(i,t)$ to represent deliver clock of participant $i$ at time $t$. Delivery clock is a lexicographical tuple.
\begin{align}
    DC(i,t) = \langle ld(i,t), t-D(i, ld(i,t))\rangle.
\end{align}
where $ld(i,t)$ is the latest data point that was delivered to MP$_i$ at time t.% (i.e., $D_i(x_l(t)) \leq t < D_i(x_l(t)+1)$). 
Interval, $t-D(i, ld(i,t))$, corresponds to the time that has elapsed since the last delivery and can be measured locally at the RB without requiring any clock synchronization (challenge 1). Delivery clock advance monotonically with time. This property will help us overcome challenge 3 and also guard us against certain attack. (\pg{forward pointers}). Figure~\ref{fig:delivery_clock} shows how delivery clock advances with time.

\begin{figure}[t]
\centering
    \includegraphics[width=0.8\columnwidth]{images/delivery_clock.jpg}
    \vspace{-2.5mm}
    \caption{\small{\bf Delivery Clock.} \pg{Redraw}}% \pg{Eashan see Ranveer's comment}}% \pg{Eashan can you redraw this figure in powerpoint or something.}}}
    \label{fig:delivery_clock}
    \vspace{-2.5mm}
\end{figure}

All incoming trades are market with the delivery clock at the trade submission time. The ordering buffer uses this delivery clock time to order trades. Formally, the ordering in DBO is given by,  

\begin{align}
    O(i,a) = DC(i, S(i,a)). 
    \label{eq:ordering_with_dc}
\end{align}


\begin{figure}[t]
\centering
    \includegraphics[trim={0 0 0 2mm},clip,width=0.9\columnwidth]{hotnets-images/time series visualization (3).pdf}
    \vspace{-3mm}
    \caption{\small{{\bf DBO can help correct for late delivery of data.} Delivery of market data to MP$_i$ is lagging behind MP$_j$. There are two trades $(i,a)$ and $(j,b)$ generated in response to the same market data $x$. $(j,l)$ was submitted before $(i,k)$ but
    %, i.e., $S_j(l) < A_i(k)$. 
    response time of $(i,k)$ is less than $(j,l)$.
    %, i.e., $rt_i(k) < rt_j(l)$. 
    With DBO, $O(i,a) (= \langle x, RT(i,a)\rangle) < O(j,b) (= \langle x, RT(j,b)\rangle)$ and trade $(i,a)$ is correctly ordered ahead of $(j,b)$.} %Ordering based on the submission time leads to incorrect ordering.}
    \pg{Correct figure}}
    \label{fig:dbo_correction}
    \vspace{-4mm}
\end{figure}


When the trigger point of trade $(i,a)$ is indeed the last data point (i.e., $x = TP(i,a) = ld(i, S(i,a))$), then, DBO respects condition C2 for LRTF. Figure~\ref{fig:dbo_correction} shows an illustrative example of this.
This is because $O(i,a) = DC(i, S(i,a)) = \langle x, RT(i,a)\rangle$. For, a competing trade $(j,b)$ with higher response time, the delivery clock at time of submission will either read $O(j,b) = DC(j, S(j,b)) = \langle x, RT(j,b)\rangle$ (if D(j,x+1)>S(j,b)) or $DC(j, S(j,b) = \langle y, S(j,b)-D(j,y)\rangle$ with $y>x$. In both cases, $O(i,a) < O(j,b)$.


\noindent
\t
At a high level, in our ordering we are correcting for latency differences in data delivery by using the delivery time of the last data point. When the last data point is not the trigger point for trade $(i,a)$, DBO satisfies the LRTF condition C2, if the following condition holds, 
\begin{align}
    D(i,ld(i,t))-D(i,x) = D(j,ld(i,t))-D(j,x),
    \label{eq:cond_delivery_lrtf}
\end{align}
where $x = TP(i,a)$.  
While it is impossible to ensure that inter-delivery times remain the same for all participants for all points, by pacing data at the RB it is indeed possible to ensure that the above condition is always met. 
The main reason why we can do so is thaat condition C2 limits that the trigger point $x$ cannot be any arbitrary data point in the past ($S(i,a)-D(i,x) < \delta$).
%and we only need to ensure same inter-delivery times for. 
In the next subsection, we will show how we can achieve this and solve challenge 2. \pg{Is this easy to follow?}

\pg{Should we include results on necessary conditions on delivery times for achieving LRTF}

\noindent
\textit{Remark:} In our cloud experiments, we find that DBO achieves fairness with very high probability. This is because network latency (from CES to any given participant) exhibits temporal correlation in latency especially over  short periods of time. When temporal correlation is high, inter-delivery time at any participant is close to the inter-generation time at the CES. In such cases, condition given by Equation~\ref{eq:cond_delivery_lrtf} is satisfied with high probability.

\begin{figure}[t]
\centering
    \includegraphics[width=0.8\columnwidth]{images/batching_pacing.jpg}
    \vspace{-2.5mm}
    \caption{\small{\bf Batching and Pacing.} \pg{Redraw}}% \pg{Eashan see Ranveer's comment}}% \pg{Eashan can you redraw this figure in powerpoint or something.}}}
    \label{fig:batching_pacing}
    \vspace{-2.5mm}
\end{figure}

\subsubsection{Batching and Pacing\\}
In DBO, the CES breaks data into batches. Each new batch contains all data points in the duration $(1+\kappa) \cdot \delta$ after the previous batch. Here $\kappa > 0$. Each release buffer delivers all data points in a batch at the same time. %Two points $x,y$ belonging to the same batch are delivered simultaneously to each participant, i.e., $D(j,y)=D(j,x), \forall j$.
The release buffer delivers batches as quickly as possible while ensuring that the time between delivery of two consecutive batches is atleast $\delta$. Figure~\ref{fig:batching_pacing} shows an illustration of batching. Both batching and pacing increase the delivery time of data points. In the next subsection we will analyze the impact of the two on latency. Note that since $\kappa > 0$ batch generation rate is slower than batch drain rate and build up queue because of pacing will eventually get drained. 



With batching and pacing, DBO achieves LRTF. In particular, 
consider a trade $(i,a)$ with response time less than $\delta$. Because of pacing, batches are separated by $\delta$. This means that the trigger point ($x=TP(i,a)$) must be within the last received batch. The point $ld(i,S(i,a))$ is also the last point in this batch and $D(i,ld(i,S(i,a)) = D(i,x)$. $O(i,a) = DC(i,S(i,a)) = <ld(i,S(i,a)), RT(i,a)>$.
With batching, for participant $j$, $x$ and $ld(i,S(i,a))$ also belong to the same batch $D(j,ld(i,S(i,a)) = D(j,x)$.
For, a competing trade $(j,b)$ with higher response time, the delivery clock at the time of submission will either read $O(j,b) = DC(j, S(j,b)) = \langle ld(i,S(i,a)), RT(j,b)\rangle$ (if $(j,b)$ was submitted before the next batch, i.e., $D(j,ld(i,S(i,a))+1) > S(j,b)$,) or $DC(j, S(j,b) = \langle y, S(j,b)-D(j,y)\rangle$ with $y>ld(i,S(i,a))$. In both cases, $O(i,a) < O(j,b)$.

\fi

\if 0
\subsection{Compute Model of the HFT Trader and Definition of Fairness}

\begin{enumerate}
    \item $MD_R(i, x):$ Receive time of market data at the gateway/RBi
    \item $TO_G(i, a):$ Generation time of trade order a by trader i
    \item $TP(i,a):$ Trigger/stimuli for trade (i,a)
    \item $RT(i,a):$ Response time of for trade (i,a) 
\end{enumerate}


\textbf{Compute Model:}
Time of generation of trade= time participant received the market point that triggered the trade + response time (or time it took to generate the trade)
\begin{equation}
    TO_G(i,a) = MD_R(i,TP(i,a)) + RT(i,a)
\end{equation}


\textbf{Perceived Fairness with respect to participant i}
If all other participants received the market data at the same time as i, then how should the trades be ordered
\begin{align*}
    \text{Trade (i,a) should be ordered ahead if}\\
    TO_G(i,a) &< MD_R(i,y) + RT(j,b)\\
    TO_G(i,a) - MD_R(i,y) &< TO_G(j,b) - MD_R(j,y)
\end{align*}
This definition states for two orders trades we need to measure time relative to event y

alternatively what if i goes into j's time domain
\begin{align*}
    &\text{Trade (i,a) should be ordered ahead iff O(i,a)<O(j,b)}\\
    MD_R(j,x) + RT(i,a) &< TO_G(j,b)\\
    TO_G(i,a)-MD_R(i,x) &< TO_G(j,b) - MD_R(j,x)
\end{align*}

Correction, relative ordering




\textbf{Achieving fairness}
There are two challenges,
\begin{outline}
    \1 How do you decide how to order these trades when TP y is unknown. \pg{Three options 1) Delivery Clocks 2) Equal RTT 3) Directly to limited fairness} \pg{Time domain: two options a) I's domain b) zero latency time doman. Fairness for trades using different data points.}
        \2 Don't know which x, recency \pg{equivalence between equal inter-delivery and correcting one way latency}
        \2 Clocks are not synced
        \2 Monotonic ordering with time
    \1 How do you enforce the ordering process. In particular, trades may take an arbitrary amount of time to reach the OB.
\end{outline}

What is the lowest RTT possible with this system?\\
Say you knew the trigger points x,y what then, \\
Say you didn't know the trigger points\\
Enforcing the ordering: key insight Enforcing an ordering at a single point is easier than controlling things at multiple RBs\\
What about trades with response time greater than delta\\


Question: Fairness wrt to external data stream

\textbf{Practical Considerations}

\begin{enumerate}
    \item Collusion attacks: Ensure that any market data point is delivered only after all participants have received it
    \item external participants: Have all participants submit trade via a dummy MP machine (we dont support fairness for such particpants)
    \item External data streams:
    \item Stragglers: 
\end{enumerate}


\textit{Correction by latency pitch}
\begin{align*}
    TO_G(i,a) - MD_R(i,y) &< TO_G(j,b) - MD_R(j,y)\\
    TO_G(i,a) - (G(y) - MD_R(i,y))) &< TO_G(j,b) +(G(y)- MD_R(j,y))
\end{align*}

\pg{Alternatively fairness in the same or equal or zero latency time domain?}
\begin{align*}
    &\text{Trade (i,a) should be ordered ahead iff O(i,a)<O(j,b)}\\
    G(x) + RT(i,a) &< G(y) + RT(j,b))\\
    TO_G(i,a) + (G(x)-MD_R(i,x)) &< TO_G(j,b) + (G(y) - MD_R(j,y))
\end{align*}


\textbf{Final Pitch Attempt}
\begin{enumerate}
    \item Introduce generalized compute model
    \item Talk about zero latency model for fairness. Three problems clocksync, which x to use, how to enforce ordering. \pg{Introduce C1 from strong fairness here?}
    \item clocksync: We are interested in competing trades that are generated using the same data point \pg{is clocksync really necessary to force this}
    \item which x to use: the last x since trades are fast. What about latency for trades with response time greater than delta
    \item how to enforce ordering: monotonic ordering process \pg{unclear if monotonic is time property is even needed (if )} 
    \item part of above? No fooling: C1 property of strong fairness
    \item \pg{Limitations: Our solution doesn't work with this model for trades generated using different data points. What about approx fairness? This is kind of nice because it talks about latency/}
\end{enumerate}
\fi
\section{Implementation}
\label{sec:impl}

At \company, we have deployed \sysname in our internal clusters to serve daily DL workloads.
The internal clusters consist of heterogeneous GPUs, including NVIDIA T4 GPU and NVIDIA A10 GPU.
Integrated with Kubernetes~\cite{k8s}, \sysname manages thousands of GPUs in each cluster and more than 20,000 GPUs in all.

\parabf{Service manager.}
For online workloads, we use the existing service manager at \company which deploys containers, discovers service, and autoscales horizontal pods.

\parabf{Global manager.}
We modify the Kubernetes scheduler to schedule offline workloads.
The workload profiler takes several dedicated GPUs, whose number is negligible to the total number of GPUs.
When a new offline workload comes, the workload profiler performs a few dry runs of the workload and utilizes the NVIDIA Data Center GPU Manager (DCGM) tools~\cite{dcgm} and NVIDIA Management Library (NVML)~\cite{nvml} libraries to collect GPU metrics.
We collect about 2,000 data for each GPU type to train the speed predictor.
The MLPs of the speed predictor have four layers with hidden size $64\times 64$.
The MLPs are trained with momentum SGD optimizer~\cite{ruder2016overview} in PyTorch v1.8.0~\cite{paszke2019pytorch} until they converge.
\sysname invokes the scheduler periodically to schedule all offline workloads.
When moving workloads, we record checkpoints of offline workloads and restart the workloads after transmitting the models and checkpoints.
As the datasets are usually colossal, we store the datasets in a remote file system and fetch data during the execution.
We implement the scheduler as a third-party plugin to the Kubernetes scheduler.


\parabf{Local executor.}
Each local executor executes online workloads according to the service manager and offline workloads according to the global manager.
DL workloads are executed in Docker containers with our customized components.
We add Best-Effort GPU DevicePlugin in Kubernetes and relevant control paths with Kubelet and \sysprobe for offline workloads.
To control SM percentage, we leverage the environment variable $CUDA\_MPS\_ACTIVE\_THREAD\_PERCENTAGE$ provided by MPS.
The GPU monitor collects resource metrics through DCGM~\cite{dcgm} and NVML~\cite{nvml} for NVIDIA GPU.
The \sysprobe updates the state machine with the collected resource metrics and empirically-set thresholds.
When the state is unhealthy, the \sysprobe will ask the NodeManager in Kubernetes to evict offline workloads.
\bytecuda intercepts nearly 800 CUDA driver APIs for GPU memory allocation and kernel launch.
The GPU memory quota of offline workloads is fixed to $40\%$ as Figure~\ref{fig:motiv_gpu_resource} reports that most online workloads use less than $60\%$ GPU memory.
We adopt the cpuset of Cgroup for CPU isolation.
For memory, \sysname will evict offline workloads if memory usage is higher than a threshold or the kernel swap daemon is busy for a long time.
The parameters to calculate GPU load in Equation~\ref{equ:gpu_load}$\&$\ref{equ:clock_factor} are empirically selected through trial-and-error.

\section{Optimizations}
\label{sec:opt}

It is widely acknowledged that HE inevitably introduces large overheads regarding both computation and communication~\cite{gouert2022new}.

\begin{figure}
\includegraphics[width=0.47\textwidth]{figs/model_comp.pdf}
\includegraphics[width=0.47\textwidth]{figs/model_comm.pdf}
\caption{(w/o Optimization) Computational (up) and Computation (bottom) Overhead Comparison For Models of Different Sizes: FedML-HE vs. Nvidia FLARE  vs. Plaintext Aggregation. Due to TenSeal's larger file sizes, FLARE did not finish the run on BERT on our 32GB memory machine (exceeding memory).}
\label{fig:comp}
\end{figure}

%yuhang: go on kk

% \begin{figure}
% \includegraphics[width=0.47\textwidth]{figs/model_comm.pdf}
% \caption{(w/o Optimization) Communicational Overhead For Models of Different Sizes: FedML-HE vs. Nvidia FLARE vs. Plaintext Aggregation}
% \label{fig:comm}
% \end{figure}

\textit{Observation}: The computational and communicational (package size) overheads introduced by HE are $O(n)$, both growing linearly with the input size $n$, which in our case the sizes of the models for aggregation. This is also shown by the evaluation results in Figure~\ref{fig:comp} %and Figure~\ref{fig:comm} 
(also Table~\ref{tab:models} in \tsref{sec:models}). Empirically, it typically has $10\times$ more computation overhead and $15\times$ more communication overhead than the plaintext FL.
%\yuhang{The computation cost and the communication cost (file size) are $O(n)$, both grows linearly with the model size $n$.} 
%\carlee{is there existing theoretical analysis on HE overheads to support this observation? E.g., big-O runtime bounds?} 

With this observation, we focus on reducing the input model sizes for HE functions as the primary optimization direction, which reduces the overheads of computation and communication simultaneously without loss of general usability and privacy. The input model sizes are determined in two stages: the resulting model parameters from the ML pipeline and the chosen model parameters used in HE functions.

\begin{figure}
\includegraphics[width=0.45\textwidth]{figs/opt_scheme_fixed.pdf}
\caption{A Universal Overhead Optimization Scheme: Parameter Efficiency x Parameter Selection. Reducing the number of model parameters and selectively encrypting them significantly reduces FedML-HE's overhead while preserving privacy.}
\vspace{-0.4cm}
\label{fig:opt_scheme}
\end{figure}

Thus, we propose a universal overhead optimization scheme. As shown in Figure~\ref{fig:opt_scheme}, input model sizes of HE can be optimized in two steps: (1) \textbf{Parameter Efficiency} to reduce the model sizes from an ML perspective and (2) \textbf{Parameter Selection} to selectively choose a certain portion of the model as inputs for HE functions while the rest of the model undergoes a plaintext aggregation at a certain privacy level. The optimization scheme can be defined as $r_{total} = r_{PE} \times r_{PS}$, where $r_{total}$ is the total optimization rate, $r_{PE}$ is the optimization rate from \textbf{Parameter Efficiency} and $r_{PS}$ is the optimization rate from \textbf{Parameter Selection}.

In the rest of \tsref{sec:opt}, we will discuss in detail how we can optimize overheads in these two steps. 

\begin{figure*}[ht]
\includegraphics[width=1\textwidth]{figs/mask_fixed.pdf}
\caption{Parameter Selection: in the initialization stage, clients first calculate privacy sensitivities on the model using its own dataset and local sensitivities will be securely aggregated to a global model privacy map. The encryption mask will be then determined by the privacy map and a set selection value $s$ per overhead requirements and privacy guarantee. Only the masked parameters will be aggregated in the encrypted form.}
\label{fig:mask}
\end{figure*}

\subsection{Parameter Efficiency}
In general, there are two directions to downsize the communicated models from an ML perspective: \textbf{Model Compression} to minimize the parameters from a local model upon communication and \textbf{Parameter-Efficient Tuning} to find a small set of parameters to efficiently tune large models. Although we here introduce several efficiency optimization techniques, further efficiency techniques can also be easily integrated into our scheme.

\noindent\textbf{Model Compression} is the typical method to reduce the number of communication parameters when training from scratch, which compresses the model to have a smaller size for communication~\cite{tang2019doublesqueeze, alistarh2017qsgd, wang2018atomo, aji2017sparse}. It may include sparsification (only communicate parameters with relatively higher weights), quantization (reduce the number of bits required to store the weights), and low-rank decomposition. The compression technique can largely reduce the number of parameters or the number of bits for each parameter to be encrypted and homomorphically aggregated. %\carlee{how does quantization do this? does it make HE more efficient?}

\noindent\textbf{Parameter-Efficient Tuning} is widely used for fine-tuning large models. For example, LoRA~\cite{hu2021lora} adjusts lightweight trainable parameters while keeping most pre-trained parameters frozen. Under this tuning scheme, only a small amount of trainable parameters will be shared in federated learning for HE.



\subsection{Parameter Selection}

Fully encrypted models can guarantee no access to plaintext local models from the adversary. However, previous work on privacy leakage analysis shows that ``partial transparency'', e.g. hiding parts of the models~\cite{hatamizadeh2022gradvit, mo2020layer}, can grant the adversary a limited chance to successfully perform attacks like gradient inversion attacks~\cite{lu2022april}. Although FLARE~\cite{roth2022nvidia} code provides a configuration to only perform HE operations on certain layers, its interaction and tradeoffs regarding privacy leakage are unknown, leaving users a ``Pandora's privacy box''. On the contrary, to clarify, our solution can support parameter-wise selection for more fine-grained overhead control with privacy leakage analysis (Figure~\ref{fig:mask}).

%We observe that transformers with (i) a smaller patch size, (ii) more parameters, and (iii) stronger training recipe with distillation, reveal more original information ", for transformer specifically, Component-wise, ", reconstructions from MLP gradients lack important details, whereas utilizing
%the gradients of MSA layers alone can already yield high-quality reconstructions
\subsubsection{Parameter Privacy Sensitivity Map} 
\hfill\\
Privacy leakage analysis can be done by directly performing gradient inversion attack~\cite{wei2020framework} and evaluating the success rate of the attack, which can take much more time than the model training. Differently, Mutual Information (MI)~\cite{xutheory} is proposed to calculate the usable information of model parameters while is practically similar to computing the area under the curve of attack models with even higher computation cost~\cite{mo2020layer}.

Sensitivity is then adopted for measuring the general privacy risk on gradients w.r.t. input, high-level features, or output~\cite{novaksensitivity, sokolic2017robust, mo2020layer}. Theoretically, if the privacy sensitivity of the plaintext parameters obtained by $\mathcal{A}$ is lower, the success rate of the attack can be also expected to be lower. Given model $\mathbf{W}$ and $K$ data samples with input matrix $\mathbf{X}$ and ground truth label vector $\mathbf{y}$, we compute the sensitivity of each parameter $w_m$ by 
$$\frac{1}{K} \sum_{k=1}^K \left\|J_m\left(y_k\right) \right\|,$$ 
where $J_m\left(y_k\right) =\frac{\partial}{\partial y_k}\left(\frac{\partial \ell\left(\mathbf{X}, \mathbf{y}, \mathbf{W}\right)}{\partial w_m}\right) \in R$, $\ell(\cdot)$ is the loss function given $\mathbf{X}$, $\mathbf{y}$ and $\mathbf{W}$, and $\left\|\cdot \right\|$ calculates the absolute value. The intuition is to calculate how large the gradient of the parameter will change with the true output $y_k$ for each data point $k$.


% block $\mathbf{w}_m$ (a 
%  group of parameters) w.r.t the ground truth label $\hat{\mathrm{Y}}$ with 

% $$
% \frac{1}{K \sqrt{|\mathrm{g}_m (\hat{\mathrm{Y}}_k )| }} \sum_{k=1}^K\left\|\frac{\mathbf{J}_m^{(\mathrm{g})}\left(\hat{\mathrm{Y}}_k\right)}{\operatorname{range}\left(\mathrm{g}_m\left(\hat{\mathrm{Y}}_k\right)\right)}\right\|_F,
% $$

% where $\mathbf{J}_m^{(\mathrm{g})}(\hat{\mathrm{Y}})=\frac{\partial \mathrm{g}_m(\hat{\mathrm{Y}})}{\partial \hat{\mathrm{Y}}}=\frac{\partial}{\partial \hat{\mathrm{Y}}}\left(\frac{\partial \ell\left(\hat{\mathrm{Y}}, \boldsymbol{\theta}_m\right)}{\partial \boldsymbol{\theta}_m}\right)$ and $\mathrm{g}_m(.) = \frac{\partial \ell\left(\hat{\mathrm{Y}}, \boldsymbol{\theta}_m\right)}{\partial \boldsymbol{\theta}_m}$ represents the function that produces block $m$ 's gradients $\mathrm{g}_m$ with true output $\hat{\mathrm{Y}}$. $\ell(.)$ is the loss function over $\hat{\mathrm{Y}}$ and $\boldsymbol{\theta}$ (parameters of the complete model). %so $\mathrm{g}_m($.$)$ can be regarded as the partial derivative of $\ell(.)$ w.r.t. block $m$'s parameters $\boldsymbol{\theta}_m$ (i.e. backward propagation). 
% Function range(.) returns the range of values in one vector (i.e. $\max ()-\min ())$. 




\subsubsection{Encryption Mask}
\hfill\\
As the privacy sensitivity analysis above as well as our results show in \tsref{sec:opt_results}, different parts of a model contribute to attacks by revealing uneven amounts of information. Using this insight, we propose to only select and encrypt parts of the model that are more important and susceptible to attacks to reduce HE overheads while maintaining the privacy guarantee. We use encryption masks (EM) to effectively select encrypted parameters. EM works by first pre-calculating the model parameter privacy map via the privacy sensitivity and applying the mask with a set privacy threshold to pick parameters to encrypt that fit certain overhead expectations. 

\subsubsection{Mask Agreement} 
\hfill\\
The encryption mask is dependent on the data it is processed on. With potentially different data distributions at each client, it is challenging to have all clients agree on a certain global encryption mask without revealing local datasets since the local privacy map carries information about the client data. To solve this problem, we encrypted aggregate local privacy maps at the server such that no further private information about local data is revealed. During the initialization, clients first calculate the local privacy sensitivities using their own local data and then the server aggregates local sensitivity maps to a global privacy map (as shown in the first part of Figure~\ref{fig:mask}). The encryption mask is configured using a privacy-overhead threshold $s$ and the global privacy map is then shared among clients as part of the federated learning configuration. 

% In practice, privacy maps are usually sliced into around a hundred parts (i.e. a couple of hundred numbers to aggregate), the HE overhead from this mask agreement process is relatively small (Figure~\ref{fig:pie}).

% \subsubsection{Empirical Selection Recipes} 
% \hfill\\
% We then provide parameter selection strategies: \carlee{explain how these are related}
% \begin{itemize}
%     \item \noindent\textbf{Layer Importance}: Layer selection is a simpler variant of parameter selection. We rank the sensitivity of each layer, then encrypt the most important layers. Empirically, layers in the beginning and end stages tend to be more important for information leakage~\cite{hatamizadeh2022gradvit}.
%     \item \noindent\textbf{Training Stages}: \weizhao{whats the conclusion about this in the literature?}
% \end{itemize}


% \weizhao{Do we still want the parts below? Seems like a jump from selection}

% \begin{itemize}
%     \item \noindent\textbf{Multiple Local Steps}: the gradient inversion attack requires the gradient and the model, while in FL, we usually perform gradient descents with multiple local steps. The gradient of each step is unavailable by communicating the models, which increases the search space of attack. Based on ~\cite{wei2020framework}, when local step $>9$, the attack success rate is low. 
%     \item \noindent\textbf{Larger Batch Size}: Increasing the batch size of gradient descent will make the gradient inversion attack try to reconstruct all data in a batch at the same time, which will also increase the search space. Based on~\cite{huang2021evaluating}, when batch size $>32$, it makes the reconstruction almost unrecognizable.
%     %\item Batch Size~\cite{wei2020framework, huang2021evaluating}
%     %\item \noindent\textbf{Gradient Alternate Hiding}: to perform most gradient inversion attacks, the adversary has to calculate the gradients between each FL round's model updates. Alternately encrypting models in certain rounds can make it harder for the adversary to compute the fine-grained gradients at the first stage of attacks.
%     %\item \noindent\textbf{Training Round Guarding}: beginning rounds are more important to reduce attack effect since gradients change at bigger steps per training data.
%     %\item Larger Batch Size
% \end{itemize}
% We can then provide a simple empirical recipe, by \emph{encrypting 10\% model parameters and performing gradient descent with batch size $= 64$ and local step $=10$}, every single component can make attacks infeasible and all these components together can provide a good solution for preserving the privacy with low communication overhead.

%\weizhao{mathematical proofs here: 1. different layer importance 2. different training stage}

%\yuhang{1. Attack by masking a few layers or parameters is feasible? 2. Attacks on different layers 3. Attacks on different parameters 4. Attack on different training round. 5. Attack with accumulated training information? 6. Attack on multiple local training steps}




% \subsection{Robustness Mechanism}
% maybe?
 \section{Benchmarks and Evaluation}
\label{sec:eval}

We evaluate \krakenSpace to answer the following set of questions:
\begin{itemize}
\item How much improvement does partial evaluation and our implemented compiler optimizations give \kraken? %(\S \ref{sec:eval2})
\item How much faster is our purely functional f-expr language, \krakenSpace, compared to other implementations of fexprs? %(\S \ref{sec:eval1} - \ref{sec:eval2})
\item How does \kraken's performance, with its fexprs, compare to macros? %(\S \ref{sec:eval1}, \S \ref{sec:eval3})
\item How do the different partial evaluation mechanisms/optimizations in \krakenSpace contribute towards reduction in overall runtime?
%\item What does \krakenSpace do internally when we create a data structure and evaluate it for some function? (\S \ref{sec:casestudy})
\end{itemize}

\textbf{Experimental Setup}: 
We ran these experiments in a reproducible Nix environment on a NixOS install \cite{10.1145/1411203.1411255} (Kernel 6.0.0) on a laptop with 8 cores / 16 threads and 64 GB of RAM.
Our code contains the scripts and Nix Flakes needed to reproduce the exact set of dependencies to run our tests.
%The code can be found at \url{https://github.com/limvot/kraken}.

The Kraken benchmarks were run using both the Wasmtime and WAVM WebAssembly engines for most benchmarks.
The Wasmtime WebAssembly engine is one of the most popular, developed by the Bytecode Alliance itself, and uses the CraneLift code generation backend.
The WAVM WebAssembly engine is interesting for its use of LLVM, and it often produces the fastest code on benchmarks but has a higher startup time.
We eliminated the Cfold Wasmtime benchmark due to problems running out of stack space (a known property of the Cfold benchmark).

\textbf{Benchmarks}: 
To showcase the capability of Kraken, we created benchmarks that are commonly implemented in functional languages and have been used as benchmarks in other papers \cite{reinking2021perceus, 10.1145/3547646}.
The benchmarks are
\begin{itemize}
\item Fib - Calculating the nth Fibonacci number
\item RB-Tree - Inserting n items into a red-black tree, then traversing the tree to sum its values
\item Deriv - Computing a symbolic derivative of a large expression
\item Cfold - Constant-folding a large expression
\item NQueens - Placing n number of queens on the board such that no two queens are diagonal, vertical, or horizontal from each other
\end{itemize}
All benchmarks besides Fibonacci use the fexpr version of match for pattern matching in \kraken, which is equivalent to the macro version in NewLisp. We also RB-Tree using NewLisp's~\cite{mueller2018newlisp} version of fexpr match. We modified the sizes of the problems presented to the benchmark to account for the longer running times of some of the less-optimized implementations.
The code for Kraken and NewLisp is very similar, and we should note that it is very unidiomatic NewLisp.
Our goal was not to compare Kraken and NewLisp as implementation languages for Red-Black Trees, but to stress test a single reasonably complex fexpr/macro, namely pattern matching.
% \textbf{Comparison with other languages}: We evaluated \krakenSpace against a language that contains f-exprs, as well as against itself with various optimizations disabled. The only other language we could find which contains a real f-expr mechanism is NewLisp~\cite{mueller2018newlisp} and so we ported \kraken's benchmark implementation to NewLisp.

%The six state-of-the-art languages are Java 17.0.1, Swift 5.4.2, Koka 2.3.2, C++, Haskell 8.10.7, and OCaml 4.12.
%The language choices were taken directly from Perceus reference-counting paper \cite{reinking2021perceus}.
%The Fibonacci benchmark additionally tests Python 3.9.11 and Chez Scheme 9.5.4.
%Koka, Ocaml and Haskell are good comparison points as statically-typed, compiled, functional programming languages, while Chez Scheme is a good comparison point as a mature and industrial strength dynamically-typed Scheme implementation known for its performance. 
%\subsection{Basic Level Comparison}
\subsection{The Effect of Partial Evaluation on Eval Calls}

\begin{table}[h]
\caption{Number of eval calls with no partial evaluation for Fexprs}
	\begin{tabular}{||c | c c c c c ||} 
		\hline
		&Evals & Eval w1 Calls & Eval w0 Calls & Comp Dyn & Comp Dyn\\ 
        & & & & w1 Calls & w0 Calls\\ [0.5ex] 
		\hline\hline
		Cfold 5 & 10897376 & 2784275 & 879066  & 1 & 0 \\ 
		\hline
		  Deriv 2  & 11708558 & 2990090 & 946500 & 1 & 0 \\ 
        \hline
		  NQueens 7 & 13530241 & 3429161 & 1108393 & 1 & 0 \\ 
    \hline
		  Fib 30 & 119107888 & 30450112 & 10770217 & 1 & 0 \\ 
    \hline
		  RB-Tree 10 & 5032297 & 1291489 & 398104 & 1 & 0 \\ 
		\hline
	\end{tabular}
    \label{npe:calls}
 \end{table}

As mentioned before, using fexprs without partial evaluation will prelude optimization and cause a massive amount of repeated work. Table \ref{npe:calls} and Table \ref{pe:calls} show the number of calls to the \krakenSpace runtime's eval function, the number of times the runtime's eval function executed a call to an applicative with wrap\_level=1, the number of times the runtime's eval function executed a call to an operative with wrap\_level=0, the number of compiled dynamic calls to applicatives with wrap\_level=1, and the number of compiled dynamic calls to operatives with wrap\_level=0.
These are shown for \krakenSpace test cases with partial evaluation turned off and turned on. 
\begin{table}[h]
\caption{Number of eval calls in Partially Evaluated Fexprs}
	\begin{tabular}{||c | c c c c c ||} 
		\hline
		&Evals & Eval w1 Calls & Eval w0 Calls & Comp Dyn & Comp Dyn\\ 
        & & & & w1 Calls & w0 Calls\\ [0.5ex] 
		\hline\hline
		Cfold 5 & 0 & 0 & 0  & 0 & 0 \\ 
		\hline
		  Deriv 2  & 0 & 0 & 0 & 2 & 0 \\ 
        \hline
		  NQueens 7 & 0 & 0 & 0 & 0 & 0 \\ 
    \hline
		  Fib 30 & 0 & 0 & 0 & 0 & 0 \\ 
    \hline
		  RB-Tree 10 & 0 & 0 & 0 & 10 & 0 \\ 
		\hline
	\end{tabular}
    \label{pe:calls}
 \end{table}

\begin{table}[h]
\caption{Number of calls to the runtime's eval function for RB-Tree. The table shows the non-partial evaluation numbers -> partial evaluation numbers.}
	\begin{tabular}{||c | c c c c c ||} 
		\hline
		&Evals & Eval w1 Calls & Eval w0 Calls & Comp Dyn & Comp Dyn\\ 
        & & & & w1 Calls & w0 Calls\\ [0.5ex] 
		\hline\hline
		  RB-Tree 7 & 2952848 -> 0 & 757932 -> 0 & 233513 -> 0 & 1 -> 7 & 0 -> 0\\ 
        \hline
		  RB-Tree 8 & 3532131 -> 0 & 906548 -> 0 & 279379 -> 0 & 1 -> 8 & 0 -> 0\\ 
        \hline
		  RB-Tree 9 & 4278001 -> 0 & 1097965 -> 0 & 3383831 -> 0 & 1 -> 9 & 0 -> 0\\ 
		\hline
	\end{tabular}
    \label{pe:rb}
    \vspace{-4mm}
 \end{table}

Without partial evaluation, no compilation can be done because it is impossible to tell if arguments to calls will be evaluated. In all benchmarks, partial evaluation removed all calls to the runtime's eval function, resulting in a completely compiled program. Looking at RB-Tree, there are over a million calls to combiners with wrap level 1 (normal functions), and 398,000 calls to combiners with wrap level 0 (operatives replacing macros). This massive blowup in the number of calls is due to the repeated and exponential re-execution of macro-like-combiners in the definition of other macro-like-combiners, as discussed in the Introduction.

The non-partially-evaluated benchmarks show 1 compiled dynamic call to an applicative (its the first call into eval) and 0 compiled dynamic calls to operatives, because there is no compilation at all. For the partially evaluated benchmarks, there are a few compiled dynamic calls to applicatives due to higher-order function use in the benchmarks, and there are no compiled dynamic calls to operatives, as all operative use has been eliminated.
We also varied the inputs for RB-Tree shown in Table \ref{pe:rb} to give a sense for how the number scale with respect to input size.

The incredible slowdown implied by these tables comes to full fruition in our RB-Tree test in Fig.~\ref{fig:kraken_nqueens_rbtree}.
We kept this run shorter because Kraken's non-partial-evaluating interpreter takes an incredibly long time even for 100 insertions (40 minutes).
The compounding layers of repeated macro-like operative calls in the non-partially-evaluated Kraken version cause a ~70,000x slowdown relative to the partial evaluated, optimized, and compiled version.
For the remaining benchmarks, we remove the naive interpreted \krakenSpace version, as in each case its performance is so bad as to blow out the graph and make it impossible to do any comparison.
In our optimized Kraken, our partial evaluation algorithm is able to fully collapse these levels of inefficiency, evaluate and inline the results, and give the backend more specialized code to optimize, emitting a compiled version that handily beats not only the NewLisp-fexpr implementation but even the NewLisp-macro implementation, as can be seen in Fig.~\ref{fig:kraken_vs_world_fib}.
We kept the benchmark sizes small in this test because the stack limits of NewLisp prevent sizes larger then ~880, while the Tail Call Elimination performed by the \krakenSpace compiler allows us to run much larger benchmarks, including the run of 4,800,000 inserts to the RB-Tree.
This result shows the dramatic effect of partial evaluation and compiler optimizations on runtime for \kraken. Our technique takes the performance of a fully fexpr based language from being completely infeasible to being faster than a macro-based dynamic scripting language currently in use.
% \begin{center}
% \begin{table}[ht]
% \caption{Number of call to the runtime's eval function for Fib. The table shows the non-partial evaluation numbers -> partial evaluation numbers}
% 	\begin{tabular}{||c | c c c c c ||} 
% 		\hline
% 		&Evals & Eval w1 Calls & Eval w0 Calls & Comp Dyn w1 Calls & Comp Dyn w0 Calls\\ [0.5ex] 
% 		\hline\hline
% 		Fib 10 & 8468 -> 0 & 2167 -> 0  & 777 -> 0 & 1 -> 0 & 0 -> 0 \\ 
% 		\hline
% 		  Fib 15  & 87916 -> 0 & 22478 -> 0 & 7961 -> 0 & 1 -> 0 & 0 -> 0 \\ 
%         \hline
% 		  Fib 20 & 969010 -> 0 & 247731 -> 0 & 87633 -> 0 & 1 -> 0 & 0 -> 0 \\ 
%     \hline
% 		  Fib 25 & 10740492 -> 0 & 2745825 -> 0  & 971209 -> 0 & 1 -> 0 & 0 -> 0 \\ 
% 		\hline
% 	\end{tabular}
%     \label{pe:fib}
%  \end{table}
% \end{center}

\begin{figure}[h]
\caption{Constant Fold and Deriv}
\includegraphics[width=0.45\textwidth]{cfold_table.csv_}
\includegraphics[width=0.45\textwidth]{deriv_table.csv_}
\label{fig:kraken_const_deriv}
\vspace{-6mm}
\end{figure}
\subsection{Comparison between Kraken Versions}
Beyond the massive speedup from partial-evaluation, Fig. \ref{fig:kraken_const_deriv} and \ref{fig:kraken_nqueens_rbtree} show the effect of the various compiler optimizations we described by disabling them one by one.
 Our main four optimizations have a strong positive effect on runtime, with the exception of lazy environment instantiation. Lazy environment instantiation helps massively on fib, and some on Deriv, but generally hurts the rest slightly.


\begin{figure}[h]
\caption{N-Queens}
\includegraphics[width=0.45\textwidth]{nqueens_table.csv_}
\includegraphics[width=0.45\textwidth]{slow_rbtree_table.csv_}
\label{fig:kraken_nqueens_rbtree}
\vspace{-4mm}
\end{figure}


\subsection{Comparison against Others}


To give a general idea of our current performance, we also show a Fibonacci benchmark that mostly exercises pure function-call speed and inlining as seen in Fig. ~\ref{fig:kraken_vs_world_fib}.
We include Python and Chez Scheme to give a general idea for where an exemplar slow and an exemplar fast dynamic language would fall.
With the benefit of our partial evaluation, compilation, and leaning upon mature WebAssembly implementations, we beat both, but this should be taken with a grain of salt, as this is a very limited micro-benchmark only meant to give a general sense of the order of magnitude of our performance.



\label{sec:eval1}
\begin{figure}[h]
\caption{Kraken vs. Others. Ordered by fastest to slowest}
\includegraphics[width=0.45\textwidth]{fib_table.csv_}
\includegraphics[width=0.45\textwidth]{rbtree_table.csv_}
\label{fig:kraken_vs_world_fib}
\end{figure}

%\label{sec:eval_nqueens}
%\begin{figure}[h]
%\caption{N-Queens}
%\includegraphics[width=0.45\textwidth]{nqueens_table.csv_}
%\includegraphics[width=0.45\textwidth]{slow_nqueens_table.csv_}
%\label{fig:kraken_nqueens}
%\end{figure}

%\label{sec:eval_nqueens}
%\begin{figure}[h]
%\caption{Kraken, N-Queens, absolute value and log-scale}
%\includegraphics[width=0.45\textwidth]{nqueens_table.csv_}
%\includegraphics[width=0.45\textwidth]{nqueens_table.csv_log}
%\label{fig:kraken_nqueens}
%\end{figure}
%\label{sec:eval_nqueensp}
%\begin{figure}[h]
%\caption{Kraken, N-Queens, absolute value and log-scale}
%\includegraphics[width=0.45\textwidth]{slow_nqueens_table.csv_}
%\includegraphics[width=0.45\textwidth]{slow_nqueens_table.csv_log}
%\label{fig:kraken_nqueensp}
%\end{figure}

%\label{sec:eval_cfold}
%\begin{figure}[h]
%\caption{C-Fold}
%\includegraphics[width=0.45\textwidth]{cfold_table.csv_}
%\includegraphics[width=0.45\textwidth]{slow_cfold_table.csv_}
%\label{fig:kraken_cfold}
%\end{figure}
%\label{sec:eval_cfold}
%\begin{figure}[h]
%\caption{Kraken, C-Fold, absolute value and log-scale}
%\includegraphics[width=0.45\textwidth]{cfold_table.csv_}
%\includegraphics[width=0.45\textwidth]{cfold_table.csv_log}
%\label{fig:kraken_cfold}
%\end{figure}
%\label{sec:eval_cfoldp}
%\begin{figure}[h]
%\caption{Kraken, C-Fold, absolute value and log-scale}
%\includegraphics[width=0.45\textwidth]{slow_cfold_table.csv_}
%\includegraphics[width=0.45\textwidth]{slow_cfold_table.csv_log}
%\label{fig:kraken_cfoldp}
%\end{figure}

%\label{sec:eval_deriv}
%\begin{figure}[h]
%\caption{Deriv}
%\includegraphics[width=0.45\textwidth]{deriv_table.csv_}
%\includegraphics[width=0.45\textwidth]{slow_deriv_table.csv_}
%\label{fig:kraken_deriv}
%\end{figure}
%\label{sec:eval_deriv}
%\begin{figure}[h]
%\caption{Kraken, Deriv, absolute value and log-scale}
%\includegraphics[width=0.45\textwidth]{deriv_table.csv_}
%\includegraphics[width=0.45\textwidth]{deriv_table.csv_log}
%\label{fig:kraken_deriv}
%\end{figure}
%\label{sec:eval_derivp}
%\begin{figure}[h]
%\caption{Kraken, Deriv, absolute value and log-scale}
%\includegraphics[width=0.45\textwidth]{slow_deriv_table.csv_}
%\includegraphics[width=0.45\textwidth]{slow_deriv_table.csv_log}
%\label{fig:kraken_derivp}
%\end{figure}

%\subsection{Comparison against state-of-the-art languages}
%\label{sec:eval3}

%\begin{figure}[h]
%\caption{Kraken vs. S.o.t.A.}
%\includegraphics[width=0.45\textwidth]{cfold_table.csv_}
%\includegraphics[width=0.45\textwidth]{rbtree_table.csv_}
%\label{fig:kraken_vs_world1}
%\end{figure}

%\begin{figure}[h]
%\caption{Kraken vs. S.o.t.A.}
%\includegraphics[width=0.45\textwidth]{deriv_table.csv_}
%\includegraphics[width=0.45\textwidth]{nqueens_table.csv_}
%\label{fig:kraken_vs_world2}
%\end{figure}

% \begin{figure}[h]
% \caption{Kraken vs. S.o.t.A. (Log)}
% \includegraphics[width=0.45\textwidth]{cfold_table.csv_log}
% \includegraphics[width=0.45\textwidth]{rbtree_table.csv_log}
% \label{fig:kraken_vs_world_log_1}
% \end{figure}
% \begin{figure}[h]
% \caption{Kraken vs. S.o.t.A. (Log)}
% \includegraphics[width=0.45\textwidth]{deriv_table.csv_log}
% \includegraphics[width=0.45\textwidth]{nqueens_table.csv_log}
% \label{fig:kraken_vs_world_log_2}
% \end{figure}

%As we noted before with the Fib(30) microbenchmark in Section \ref{sec:eval1}, we remain significantly slower than state-of-the-art compiled languages.
%This is particularly true for memory-intensive benchmarks due to our naive reference-counting and malloc/free implementations.
%However, our results are of a similar order of magnitude to the difference between the state-of-the-art compiled languages and dynamic scripting languages, like Python's results in the Fib(30) microbenchmark.
%We assert that is not a fundamental limitation because the classic f-expr slowness is being eliminated, as shown by Fig. \ref{fig:kraken_vs_newlisp1} and Fig. \ref{fig:kraken_vs_newlisp2}.
%In future work, we plan to expand our compile-time analysis and optimization to implement a modified, dynamic-language version of Perceus reference counting.
%With this change, we belive \krakenSpace can be competitive with these state-of-the-art languages.

%\subsection{Case Study: Red-Black Tree}
%\label{sec:casestudy}

%\begin{figure}[h]
%\caption{Kraken vs. S.o.t.A. - RB-Tree Focus}
%\includegraphics[width=0.4\textwidth]{rbtree_table.csv_}
%\includegraphics[width=0.4\textwidth]{rbtree_table.csv_log}
%\label{fig:kraken_vs_world_rbtree}
%\end{figure}


%To evaluate our partial evaluation algorithm and compiler, we extracted the benchmarks used by the Koka language project from their code repository and added Kraken versions, as well as implementing a naive Fibonacci microbenchmark ourselves to evaluate pure function call speed.\\
%With partial evaluation and the compiler optimizations listed above, we get fairly strong performance on purely numerical computations, such as the naive Fibonacci microbenchmark.
%Unfortunately, the overhead of our unsophisticated reference counting, dynamic type checking, and bounds checking causes poor performance on benchmarks involving data structures relative to mainstream programming language implementations.
%This is not a fundamental limitation, and will be addressed in future work, as recounted in the next section.
%It should be noted, however, that while the performance relative to established language implementations is very poor for the memory-intensive benchmarks (600-900x slower), we still realize a massive speedup compared to an unoptimized and non-partial-evaluated f-expr implementation (100,000x faster)!

%\section{Related work}
\noindent \textbf{Video foundation models.}
With sufficient computational power and an abundant source of data, there have been attempts to build a single large-scale foundation model that can be adapted to diverse downstream tasks.
Along with the success of foundations models in the natural language processing domain~\cite{brown2020language,chen2021evaluating,devlin2019bert} and in computer vision~\cite{bertasius2021space,jia2021scaling,radford2021learning}, video data has become another data type of interest, as it has grown in scale due to numerous internet video-sharing platforms.
Accordingly, several methods to train a video foundation model have been proposed.
Due to the innate multi-modality of video data, \textit{i.e.}, a combination of visual $\cdot$ vocal $\cdot$ textual context, most works have centered around the variations of the cross-modal attention mechanism \cite{akbari2021vatt,bertasius2021space,gabeur2020multi,luo2020univl,neimark2021video,tan2021look,wei2020multi,yang2021taco}.
In addition, as most video data lack proper labels or descriptions, contrastive learning methods were studied to learn meaningful feature representations or enhance video-text alignment in a self-supervised manner \cite{akbari2021vatt,kuang2021video,luo2020univl,yang2021taco}.

More specifically, MERLOT \cite{zellers2021merlot} proposed a multi-modal representation learning method for visual commonsense reasoning, which also performed well in twelve video reasoning tasks.
VATT \cite{akbari2021vatt} introduced a multi-modal learning method via contrastive learning. 
The pre-trained model performed well in a variety of vision tasks from image classification to video action recognition and zero-shot video retrieval.
Another representative work, UniVL \cite{luo2020univl} proposed a straightforward pre-training method with auxiliary loss functions. 
After fine-tuning on a specific task, the pre-trained model performed outstandingly in a wide range of tasks of text-to-video retrieval, action segmentation, action step localization, video sentiment analysis, and video captioning.
Other foundation models for multiple video tasks include \cite{li2020hero,sun2019learning,sun2019videobert,zhu2020actbert,fu2021violet,wang2022all}. 

\noindent \textbf{Auxiliary learning.}
In order to enhance the performance of one or a multitude of primary tasks, auxiliary learning methods can be incorporated.
\cite{ruder2017overview} introduced Multi-task learning (MTL) to the deep neural networks by training a single model with multiple task losses to assist learning on the main task.
Such a method is generally adapted to pre-train the foundation models in the self-supervised manner~\cite{li2020hero,sun2019learning,sun2019videobert,zhu2020actbert,fu2021violet,wang2022all}.
However, these various pretext task losses used in the pre-training phase are ignored in the fine-tuning phase, and only the primary task loss is minimized.

Recently, meta-learning methods have been introduced for auxiliary learning.
\cite{liu2019self,navon2020auxiliary,shu2019meta} proposed a meta-learning method in which the model learns auxiliary tasks to generalize well to unseen data. 
In these settings, a separate subset of data is held out as the primary task, while the others are used as auxiliary tasks that aid the primary task's performance.
Similar methods were adopted for computer vision tasks such as semantic segmentation \cite{xu2021leveraging}.
Other domain applications include navigation tasks with reinforcement learning \cite{ye2021auxiliary}, or self-supervised learning methods on graph data \cite{hwang2020self}.
\section{Conclusion}\label{sec:conclusion}
In this work, we focus on addressing the fundamental challenge of OOD detection tasks, which is how to fully understand the semantic discrepancy between the ID/OOD samples. We reveal that the key to success in the realistic SCOOD task is to allocate as many ID samples in the unlabeled set correctly as possible. To this end, we propose a novel uncertainty-aware optimal transport scheme that introduces class-specific energy scores as guidance for effective label assignment. Experimental results show that our method achieves better performance than previous state-of-the-art methods on SCOOD benchmarks.

\textbf{Limitations.} In addition to temperature scaling, other techniques such as feature clipping applied in ReAct~\cite{sun2021react} also enhance the performance of energy score, so how to obtain an OOD score that best fits the SCOOD task can be further explored. Moreover, a setting highly related to SCOOD has been proposed in \cite{katz2022training} and formulated as a constrained optimization problem. We will also theoretically analyze these practical OOD settings in our feature work.

% \section*{Acknowledgments}
\textbf{Acknowledgments.} 
This work is supported by National Key R\&D Program of China under Grant 2020AAA0105701, National Natural Science Foundation of China (NSFC) under Grants 61872327, Major Special Science and Technology Project of Anhui, National Natural Science Foundation of China (62033012) and Ant Group through Ant Research Intern Program.



% % \begin{ack}

% % \end{ack}



% \newpage
% \bibliography{refs}
% \bibliographystyle{ACM-Reference-Format}

%%%%%%%%%%%%%%%%%%%%%%%%%%%%%%%%%%%%%%%%%%%%%%%%%%%%%%%%%%%%
Below we first briefly describe the selected models and then their implementation details during pre-training.

% Traditional convolutional action recognition networks before 2017 are mostly built to process single frame or multiple consecutive frames; however, such simple structures overlook the importance of long-range temporal context in action recognition, which somehow underestimates the intrinsic temporal information within videos. 
Temporal segment networks (TSN) proposes segment-based sampling to learn temporal information across frames. 
Specifically, in TSN, a video is evenly divided into several temporal segments, which one random frame is sampled from. 
Then the output from each segment will be aggregated via pooling to obtain the final prediction. 
Temporal Shift Module (TSM) shifts feature channels along the temporal axis, which facilitates information exchanged among neighboring frames. 
It can be plug-and-played in 2D networks to enable stronger temporal modeling at zero computation and zero parameters.
Thus, TSM can achieve the performance of heavy 3D CNNs while maintaining the efficiency of 2D CNNs.
% TSM introduces stronger temporal learning capacity to 2D networks while maintaining light-weight. 

Inflated 3D ConvNet (I3D) is designed to bootstrap from the corresponding 2D network since (1) the architecture of 2D network is well designed and (2) the  weights of 2D network is well pre-trained, e.g., Inception~\cite{inception} $\rightarrow$ Inception-I3D~\cite{carreira2017quo}. 
% utilize pre-trained weights from the corresponding 2D network since these 2D weights have been well-designed and trained to perceive visual concepts.
I3D initializes its 3D kernels by duplicating the 2D ones along the temporal dimension, which helps the convergence of 3D CNNs. 
Inspired by~\cite{vaswani2017attention}, non-local networks (NL) adapts the non-local operation (i.e., self-attention~\cite{vaswani2017attention}) in its building block to model long-range dependency.
For video action recognition, its goal is to relate the same object, or person-object interaction within a distant time interval in videos.
Similar to TSM, non-local block is compatible to most convolutional networks.


TimeSformer is a pure transformer-based model, which is an extension of ViT~\cite{dosovitskiy2020image} to the spatiotemporal space. 
Given the quadratic complexity of self-attention, TimeSformer compares several attention strategies when considering temporal dimention in videos.
Finally, TimeSformer introduces the divided space-time attention to greatly reduce the computation burden but achieves promising results.
% on most video action recognition datasets. 
% This structure shows both effectiveness and efficiency in their reported results. 
Continuing this modeling shift from CNNs to Transformers, VideoSwin extends Swin Transformer~\cite{liu2021swin} by adding the inductive bias of locality in video transformers. 
Simply speaking, it adapts the idea of 2D shifted window self-attention to 3D space, which results in better speed-accuracy trade-off compared to previous approaches~\cite{bertasius2021space,arnab2021vivit}.
% Similarly, VideoSwin is an extension of Swin Transformer~\cite{liu2021swin}, by adapting the 2D shifted window self-attention to 3D.
% And shifted window ensure the connection across distant regions in the spatiotemporal tensors.


\begin{figure}[t]
\centering
    \includegraphics[width=8cm]{figures/radar_new.pdf}
    \caption{The rank of the averaged performance within different data domains for the 6 models in different settings. The most outside in these radar images means the highest performance. For each domain, we average the top-1 accuracy as the scores in finetuning and average the top-1 accuracy of 16-shot results in few-shot learning. Complete results are shown in Table~\ref{tab:finetune} and Figure~\ref{fewshot}.}
    \label{radar}
\end{figure}
\bibliography{refs}
\bibliographystyle{ACM-Reference-Format}

\end{document}