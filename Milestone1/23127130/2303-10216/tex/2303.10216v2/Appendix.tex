\begin{appendices}

\section*{Appendix}
\section{Monte Carlo integration}
The foundation of Monte Carlo approximation is the ability to draw independent samples from a distribution given by some measure $P$. Details on Monte Carlo theory can be found in \cite{MCbook} and \cite[\S 24]{Wasserman}. Assuming that $F(Z)$ has bounded variance, where $Z\sim P$, the weak law of large numbers guarantees that the estimator
\[
    I_K = \frac{1}{K}\sum_{k=1}^K F\left(Z^{(k)}\right)
\]
is a consistent estimator of $\E[F(Z)]=\int F(Z)dP$, where $\{Z^{(k)}\}$ is an i.i.d. sequence of random variables distributed according to $P$. The estimation of the error of $I_K$ is given by $\sqrt{Var[I_K]}=\sqrt{Var[F(Z)]}\cdot K^{-1/2}$. Furthermore, $I_K$ is unbiased since $\E[I_K] = \E[F(Z)]$; see Ch. I, $\S 5$ of \cite{Shiryaev}.

\begin{remark}\rm
    The weak law of large numbers does not require $Var[F(Z)]< \infty$ for convergence to occur. It will simply be the case that the convergence may be slower than $O(K^{-1/2})$ by only having the integrability of $F(Z)$.
\end{remark}

\end{appendices}