\section{Additional regression measures}
\label{app:fullRegression}
The regression method to obtain the cost factors is explained in \Sec{regression}, illustrated by the simulations of the flexible tube with the FV-FE. The accuracy of the various regressions for the other parameter studies is summarized in \Tab{regressionAccuracy}.

Besides the RRMSE, the absolute variant is also given, namely the root mean square error (RMSE)
\begin{equation}
    \mathrm{RMSE} = \sqrt{\frac{\sum^\numCalc\abs{
    \timing^f-(\totalProblemIter^c \cdot \costFix + \totalProblemIter^f \cdot \costIter)
    %\sum^{\totalProblemIter^c}_{k=1}\costSolver^f(k)
    }^2}{m}}.
    \label{equ:RMSE}
\end{equation}

\begin{table}[ph]
	\caption{Quality measures for the coupling, flow, and solid regression for all parameters studies, as well as deviation between the actual and equivalent run times determined with the new measure $\costSimulation$ and with the measure used in literature, i.e., only regarding the number of coupling iterations.}
    \label{tab:regressionAccuracy}
    \begin{center}
        \smallIfElsevier % \small % change template 
        \label{tab:RRMSE_Measures}
		\begin{tabular}{l|*{6}{l}|ll|ll}
			%\hline
			& \multicolumn{2}{c}{Coupling} & \multicolumn{2}{c}{Flow} & \multicolumn{2}{c|}{Solid} & \multicolumn{2}{c|}{New equivalent time} & \multicolumn{2}{c}{Measure in literature} \\
            & RMSE & RRMSE & RMSE & RRMSE & RMSE & RRMSE & MAPE & maxAPE & MAPE & maxAPE \\
            \hline
			\textbf{Lid-driven cavity} & & & & & & & & & & \\
			$\quad$ FE-FE     & \qty{0.2}{\second} & 7.88 \% 
                              & \qty{1.7}{\second} & 0.44 \%  
                              & \qty{0.3}{\second} & 1.26 \%  
                              & 0.33 \% & 0.89 \% 
                              & 22.77 \% & 34.32 \% \\
            $\quad$ FV-FE     & \qty{13.8}{\second} & 11.95 \% 
                              & \qty{468.7}{\second} & 5.60 \%  
                              & \qty{6.3}{\second} & 3.74 \%  
                              & 3.86 \% & 15.76 \%
                              & 12.29 \% & 36.20 \% \\
			%\hline
            \textbf{Flexible tube} & & & & & & & & & & \\
			$\quad$ FE-FE     & \qty{2.8}{\second} & 1.86 \% 
                              & \qty{7.3}{\second} & 0.32 \%  
                              & \qty{0.8}{\second} & 0.26 \%  
                              &  0.22 \% &  0.58 \% 
                              & 12.70 \% & 33.32 \% \\
			$\quad$ FV-FE     & \qty{2.7}{\second} & 2.65 \% 
                              & \qty{233.2}{\second} & 7.43 \%  
                              & \qty{27.7}{\second} & 3.44 \%  
                              & 3.80 \% & 14.62 \%
                              & 13.58 \% & 52.69 \% \\
			%\hline
		\end{tabular}
	\end{center}
\end{table}

\section{Additional results}
\label{app:fullResults}
To increase readability, the discussion, tables, and figures of \Sec{results} focused on one trend at a time.
% 
For the sake of completeness, this appendix groups all findings in a single table for each parameter study, listing the equivalent time measure, the number of coupling iterations, and the number of \subproblemIter s for every run within a study.
\Tab{FECavity} and \Tab{FVCavity} list the results of the lid-driven cavity case for the FE-FE and FV-FE framework, respectively.
% 
Similarly, the flexible tube case with the FE-FE setup is covered by \Tab{FETube}, and its FV-FE analogue by \Tab{FVTube}.

\renewcommand{\minval}{0.777}
\begin{table}
\centering
\caption{Lid-driven cavity case with the FE-FE framework.
The row and column header contain the maximal number of \subproblemIter s for the flow and solid solver, $n_{max}^{f}$ and $n_{max}^{s}$, respectively. For each run, the normalized \textbf{equivalent time} is given, as well as the number of \cIt{coupling iterations}, \fIt{flow solver iterations}, and \sIt{solid solver iterations}.}
\label{tab:FECavity}
\begin{tabular}{cc|cc|cc|cc|cc}
\multicolumn{2}{c}{}    & \multicolumn{8}{c}{Newton iterations per coupling iteration - Structural solver} \\
\multicolumn{2}{c}{}    & \multicolumn{2}{c|}{1} & \multicolumn{2}{c|}{2} & \multicolumn{2}{c|}{3} & \multicolumn{2}{c}{$\infty$} \\ \cline{3-10}
\multirow{8}{*}{\rotatebox[origin=c]{90}{\makecell{Newton iterations \\ per coupling iteration \\ - Flow solver}}}
&                       & \eqTime{1.07} & \cIt{8554}& \eqTime{1.15} & \cIt{8876}& \eqTime{1.12} & \cIt{8542}& \eqTime{1.12} & \cIt{8499}\\
& \multirow{-2}{*}{1}   & \fIt{8554} & \sIt{8554}& \fIt{8876} & \sIt{17043}& \fIt{8542} & \sIt{18894}& \fIt{8499} & \sIt{19662}\\ \cline{2-10}
&                       & \eqTime{0.91} & \cIt{5066}& \eqTime{0.78} & \cIt{4304}& \eqTime{0.78} & \cIt{4293}& \eqTime{0.79} & \cIt{4299}\\
& \multirow{-2}{*}{2}   & \fIt{9014} & \sIt{5066}& \fIt{7470} & \sIt{7908}& \fIt{7452} & \sIt{9294}& \fIt{7473} & \sIt{10183}\\ \cline{2-10}
&                       & \eqTime{1.05} & \cIt{4985}& \eqTime{0.83} & \cIt{3989}& \eqTime{0.84} & \cIt{4014}& \eqTime{0.84} & \cIt{4002}\\
& \multirow{-2}{*}{3}   & \fIt{11098} & \sIt{4985}& \fIt{8489} & \sIt{7278}& \fIt{8515} & \sIt{8729}& \fIt{8458} & \sIt{9579}\\ \cline{2-10}
&                       & \eqTime{1.21} & \cIt{5004}& \eqTime{0.99} & \cIt{4007}& \eqTime{0.99} & \cIt{3982}& \eqTime{1.00} & \cIt{3974}\\
& \multirow{-2}{*}{$\infty$}   & \fIt{13487} & \sIt{5004}& \fIt{10853} & \sIt{7314}& \fIt{10761} & \sIt{8667}& \fIt{10786} & \sIt{9523}
\end{tabular}
\end{table}

\renewcommand{\minval}{0.759}
\begin{table}
\centering
\caption{Lid-driven cavity case with the FV-FE framework.
The row and column header contain the maximal number of \subproblemIter s for the flow and solid solver, $n_{max}^{f}$ and $n_{max}^{s}$, respectively. For each run, the normalized \textbf{equivalent time} is given, as well as the number of \cIt{coupling iterations}, \fIt{flow solver iterations}, and \sIt{solid solver iterations}. A missing value indicates that the coupling did not converge.}
\label{tab:FVCavity}
\begin{tabular}{cc|cc|cc|cc|cc}
\multicolumn{2}{c}{}    & \multicolumn{8}{c}{Newton iterations per coupling iteration - Structural solver} \\
\multicolumn{2}{c}{}    & \multicolumn{2}{c|}{1} & \multicolumn{2}{c|}{2} & \multicolumn{2}{c|}{3} & \multicolumn{2}{c}{$\infty$} \\ \cline{3-10}
\multirow{30}{*}{\rotatebox[origin=c]{90}{Fixed-point iterations per coupling iteration - Flow solver}}
&                       & \eqTime{-} & \cIt{-}& \eqTime{-} & \cIt{-}& \eqTime{-} & \cIt{-}& \eqTime{-} & \cIt{-}\\
& \multirow{-2}{*}{5}   & \fIt{-} & \sIt{-}& \fIt{-} & \sIt{-}& \fIt{-} & \sIt{-}& \fIt{-} & \sIt{-}\\ \cline{2-10}
&                       & \eqTime{0.96} & \cIt{7209}& \eqTime{0.97} & \cIt{7204}& \eqTime{0.95} & \cIt{7111}& \eqTime{0.96} & \cIt{7176}\\
& \multirow{-2}{*}{6}   & \fIt{37477} & \sIt{7209}& \fIt{37581} & \sIt{12900}& \fIt{36929} & \sIt{16148}& \fIt{37399} & \sIt{17802}\\ \cline{2-10}
&                       & \eqTime{0.86} & \cIt{6218}& \eqTime{0.85} & \cIt{6126}& \eqTime{0.85} & \cIt{6132}& \eqTime{0.85} & \cIt{6138}\\
& \multirow{-2}{*}{7}   & \fIt{36473} & \sIt{6218}& \fIt{35965} & \sIt{10865}& \fIt{36001} & \sIt{13791}& \fIt{36068} & \sIt{15110}\\ \cline{2-10}
&                       & \eqTime{0.84} & \cIt{5882}& \eqTime{0.81} & \cIt{5613}& \eqTime{0.81} & \cIt{5618}& \eqTime{0.81} & \cIt{5608}\\
& \multirow{-2}{*}{8}   & \fIt{38449} & \sIt{5882}& \fIt{36629} & \sIt{9926}& \fIt{36706} & \sIt{12648}& \fIt{36590} & \sIt{13896}\\ \cline{2-10}
&                       & \eqTime{0.84} & \cIt{5692}& \eqTime{0.79} & \cIt{5328}& \eqTime{0.79} & \cIt{5318}& \eqTime{0.79} & \cIt{5310}\\
& \multirow{-2}{*}{9}   & \fIt{40841} & \sIt{5692}& \fIt{38382} & \sIt{9443}& \fIt{38163} & \sIt{11991}& \fIt{38127} & \sIt{13234}\\ \cline{2-10}
&                       & \eqTime{0.85} & \cIt{5581}& \eqTime{0.78} & \cIt{5091}& \eqTime{0.78} & \cIt{5075}& \eqTime{0.78} & \cIt{5103}\\
& \multirow{-2}{*}{10}   & \fIt{43394} & \sIt{5581}& \fIt{39559} & \sIt{8943}& \fIt{39430} & \sIt{11354}& \fIt{39631} & \sIt{12683}\\ \cline{2-10}
&                       & \eqTime{0.86} & \cIt{5485}& \eqTime{0.77} & \cIt{4906}& \eqTime{0.77} & \cIt{4885}& \eqTime{0.77} & \cIt{4888}\\
& \multirow{-2}{*}{11}   & \fIt{45919} & \sIt{5485}& \fIt{40814} & \sIt{8613}& \fIt{40662} & \sIt{10865}& \fIt{40715} & \sIt{12124}\\ \cline{2-10}
&                       & \eqTime{0.87} & \cIt{5398}& \eqTime{0.76} & \cIt{4706}& \eqTime{0.76} & \cIt{4718}& \eqTime{0.76} & \cIt{4720}\\
& \multirow{-2}{*}{12}   & \fIt{48329} & \sIt{5398}& \fIt{41953} & \sIt{8279}& \fIt{42020} & \sIt{10519}& \fIt{41978} & \sIt{11790}\\ \cline{2-10}
&                       & \eqTime{0.89} & \cIt{5387}& \eqTime{0.76} & \cIt{4607}& \eqTime{0.76} & \cIt{4602}& \eqTime{0.76} & \cIt{4588}\\
& \multirow{-2}{*}{13}   & \fIt{51042} & \sIt{5387}& \fIt{43364} & \sIt{8056}& \fIt{43320} & \sIt{10201}& \fIt{43319} & \sIt{11447}\\ \cline{2-10}
&                       & \eqTime{0.91} & \cIt{5373}& \eqTime{0.77} & \cIt{4566}& \eqTime{0.77} & \cIt{4545}& \eqTime{0.77} & \cIt{4546}\\
& \multirow{-2}{*}{14}   & \fIt{53842} & \sIt{5373}& \fIt{45288} & \sIt{7982}& \fIt{45168} & \sIt{10091}& \fIt{45386} & \sIt{11354}\\ \cline{2-10}
&                       & \eqTime{0.92} & \cIt{5352}& \eqTime{0.79} & \cIt{4558}& \eqTime{0.78} & \cIt{4534}& \eqTime{0.78} & \cIt{4534}\\
& \multirow{-2}{*}{15}   & \fIt{56336} & \sIt{5352}& \fIt{47478} & \sIt{7966}& \fIt{47241} & \sIt{10068}& \fIt{47344} & \sIt{11351}\\ \cline{2-10}
&                       & \eqTime{1.01} & \cIt{5349}& \eqTime{0.85} & \cIt{4519}& \eqTime{0.85} & \cIt{4487}& \eqTime{0.85} & \cIt{4500}\\
& \multirow{-2}{*}{20}   & \fIt{67992} & \sIt{5349}& \fIt{56902} & \sIt{7903}& \fIt{56645} & \sIt{9958}& \fIt{56739} & \sIt{11219}\\ \cline{2-10}
&                       & \eqTime{1.07} & \cIt{5357}& \eqTime{0.91} & \cIt{4511}& \eqTime{0.91} & \cIt{4498}& \eqTime{0.90} & \cIt{4483}\\
& \multirow{-2}{*}{25}   & \fIt{76627} & \sIt{5357}& \fIt{64779} & \sIt{7896}& \fIt{64833} & \sIt{9994}& \fIt{64530} & \sIt{11213}\\ \cline{2-10}
&                       & \eqTime{1.12} & \cIt{5345}& \eqTime{0.95} & \cIt{4499}& \eqTime{0.96} & \cIt{4503}& \eqTime{0.95} & \cIt{4491}\\
& \multirow{-2}{*}{30}   & \fIt{83527} & \sIt{5345}& \fIt{71215} & \sIt{7886}& \fIt{71336} & \sIt{9997}& \fIt{71008} & \sIt{11211}\\ \cline{2-10}
&                       & \eqTime{1.17} & \cIt{5348}& \eqTime{1.00} & \cIt{4523}& \eqTime{1.00} & \cIt{4492}& \eqTime{1.00} & \cIt{4509}\\
& \multirow{-2}{*}{$\infty$}   & \fIt{89966} & \sIt{5348}& \fIt{77618} & \sIt{7922}& \fIt{77344} & \sIt{9983}& \fIt{77272} & \sIt{11221}
\end{tabular}
\end{table}


\begin{comment}
\renewcommand{\minval}{0.736}
\begin{table}[]
\caption{Lid-driven cavity with the FV-FE framework.
The column and row header contain the limit number $n_{max}^p$ for the flow and solid solver, respectively. For each case, the \textbf{equivalent time} is given, as well as the number of \cIt{coupling iterations}, \fIt{flow solver iterations} and \sIt{solid solver iterations}}
\label{tab:FVCavity}
\begin{tabular}{cc|cc|cc|cc|cc}
\multicolumn{2}{c}{}    & \multicolumn{8}{c}{Newton iterations per coupling iteration - Structural solver} \\
\multicolumn{2}{c}{}    & \multicolumn{2}{c|}{1} & \multicolumn{2}{c|}{2} & \multicolumn{2}{c|}{3} & \multicolumn{2}{c}{$\infty$} \\ \cline{3-10}
\multirow{30}{*}{\rotatebox[origin=c]{90}{Fixed-point iterations per coupling iteration - Flow solver}}
&                       & \eqTime{-} & \cIt{-}& \eqTime{-} & \cIt{-}& \eqTime{-} & \cIt{-}& \eqTime{-} & \cIt{-}\\
& \multirow{-2}{*}{5}   & \fIt{-} & \sIt{-}& \fIt{-} & \sIt{-}& \fIt{-} & \sIt{-}& \fIt{-} & \sIt{-}\\ \cline{2-10}
&                       & \eqTime{0.92} & \cIt{7260}& \eqTime{0.90} & \cIt{7110}& \eqTime{0.91} & \cIt{7176}& \eqTime{0.94} & \cIt{7404}\\
& \multirow{-2}{*}{6}   & \fIt{37808} & \sIt{7260}& \fIt{36974} & \sIt{12665}& \fIt{37327} & \sIt{16320}& \fIt{38632} & \sIt{18368}\\ \cline{2-10}
&                       & \eqTime{0.82} & \cIt{6236}& \eqTime{0.81} & \cIt{6096}& \eqTime{0.81} & \cIt{6095}& \eqTime{0.81} & \cIt{6137}\\
& \multirow{-2}{*}{7}   & \fIt{36479} & \sIt{6236}& \fIt{35712} & \sIt{10761}& \fIt{35747} & \sIt{13706}& \fIt{36006} & \sIt{15089}\\ \cline{2-10}
&                       & \eqTime{0.80} & \cIt{5853}& \eqTime{0.77} & \cIt{5628}& \eqTime{0.77} & \cIt{5598}& \eqTime{0.77} & \cIt{5595}\\
& \multirow{-2}{*}{8}   & \fIt{38204} & \sIt{5853}& \fIt{36783} & \sIt{9939}& \fIt{36648} & \sIt{12627}& \fIt{36620} & \sIt{13884}\\ \cline{2-10}
&                       & \eqTime{0.81} & \cIt{5696}& \eqTime{0.76} & \cIt{5314}& \eqTime{0.76} & \cIt{5319}& \eqTime{0.76} & \cIt{5318}\\
& \multirow{-2}{*}{9}   & \fIt{40816} & \sIt{5696}& \fIt{38193} & \sIt{9398}& \fIt{38151} & \sIt{11957}& \fIt{38218} & \sIt{13256}\\ \cline{2-10}
&                       & \eqTime{0.82} & \cIt{5561}& \eqTime{0.75} & \cIt{5076}& \eqTime{0.75} & \cIt{5098}& \eqTime{0.75} & \cIt{5078}\\
& \multirow{-2}{*}{10}   & \fIt{43317} & \sIt{5561}& \fIt{39472} & \sIt{8934}& \fIt{39496} & \sIt{11366}& \fIt{39512} & \sIt{12647}\\ \cline{2-10}
&                       & \eqTime{0.83} & \cIt{5477}& \eqTime{0.74} & \cIt{4861}& \eqTime{0.74} & \cIt{4881}& \eqTime{0.74} & \cIt{4881}\\
& \multirow{-2}{*}{11}   & \fIt{45844} & \sIt{5477}& \fIt{40729} & \sIt{8551}& \fIt{40573} & \sIt{10846}& \fIt{40659} & \sIt{12127}\\ \cline{2-10}
&                       & \eqTime{0.85} & \cIt{5423}& \eqTime{0.74} & \cIt{4700}& \eqTime{0.74} & \cIt{4706}& \eqTime{0.74} & \cIt{4715}\\
& \multirow{-2}{*}{12}   & \fIt{48604} & \sIt{5423}& \fIt{41937} & \sIt{8275}& \fIt{41927} & \sIt{10494}& \fIt{42051} & \sIt{11790}\\ \cline{2-10}
&                       & \eqTime{0.86} & \cIt{5355}& \eqTime{0.74} & \cIt{4606}& \eqTime{0.74} & \cIt{4603}& \eqTime{0.74} & \cIt{4598}\\
& \multirow{-2}{*}{13}   & \fIt{51000} & \sIt{5355}& \fIt{43450} & \sIt{8093}& \fIt{43295} & \sIt{10211}& \fIt{43345} & \sIt{11458}\\ \cline{2-10}
&                       & \eqTime{0.89} & \cIt{5386}& \eqTime{0.75} & \cIt{4565}& \eqTime{0.75} & \cIt{4555}& \eqTime{0.75} & \cIt{4565}\\
& \multirow{-2}{*}{14}   & \fIt{53728} & \sIt{5386}& \fIt{45344} & \sIt{7988}& \fIt{45300} & \sIt{10124}& \fIt{45356} & \sIt{11363}\\ \cline{2-10}
&                       & \eqTime{0.91} & \cIt{5375}& \eqTime{0.77} & \cIt{4545}& \eqTime{0.77} & \cIt{4539}& \eqTime{0.76} & \cIt{4524}\\
& \multirow{-2}{*}{15}   & \fIt{56326} & \sIt{5375}& \fIt{47373} & \sIt{7952}& \fIt{47376} & \sIt{10070}& \fIt{47211} & \sIt{11302}\\ \cline{2-10}
&                       & \eqTime{0.99} & \cIt{5323}& \eqTime{0.84} & \cIt{4513}& \eqTime{0.84} & \cIt{4488}& \eqTime{0.84} & \cIt{4495}\\
& \multirow{-2}{*}{20}   & \fIt{67765} & \sIt{5323}& \fIt{57015} & \sIt{7911}& \fIt{56766} & \sIt{9973}& \fIt{56895} & \sIt{11261}\\ \cline{2-10}
&                       & \eqTime{1.07} & \cIt{5355}& \eqTime{0.90} & \cIt{4520}& \eqTime{0.90} & \cIt{4491}& \eqTime{0.90} & \cIt{4486}\\
& \multirow{-2}{*}{25}   & \fIt{76752} & \sIt{5355}& \fIt{64872} & \sIt{7894}& \fIt{64599} & \sIt{9975}& \fIt{64460} & \sIt{11224}\\ \cline{2-10}
&                       & \eqTime{1.12} & \cIt{5349}& \eqTime{0.95} & \cIt{4513}& \eqTime{0.95} & \cIt{4508}& \eqTime{0.95} & \cIt{4484}\\
& \multirow{-2}{*}{30}   & \fIt{83413} & \sIt{5349}& \fIt{71235} & \sIt{7899}& \fIt{71170} & \sIt{9984}& \fIt{71179} & \sIt{11234}\\ \cline{2-10}
&                       & \eqTime{1.17} & \cIt{5365}& \eqTime{1.00} & \cIt{4511}& \eqTime{1.00} & \cIt{4490}& \eqTime{1.00} & \cIt{4486}\\
& \multirow{-2}{*}{$\infty$}   & \fIt{89941} & \sIt{5365}& \fIt{77498} & \sIt{7896}& \fIt{77368} & \sIt{9993}& \fIt{77302} & \sIt{11225}
\end{tabular}
\end{table}
\end{comment}

\begin{comment}
\renewcommand{\minval}{0.815}
\begin{table}[]
\caption{Lid-driven cavity with the FV-FE framework.
The column and row header contain the limit number $n_{max}^p$ for the flow and solid solver, respectively.
For each case, the \textbf{equivalent time} is given, as well as the number of \cIt{coupling iterations}, \fIt{flow solver iterations} and \sIt{solid solver iterations}.}
\label{tab:FVCavity}
\begin{tabular}{cc|cc|cc|cc|cc}
\multicolumn{2}{c}{}    & \multicolumn{8}{c}{Newton iterations per coupling iteration - Structural solver} \\
\multicolumn{2}{c}{}    & \multicolumn{2}{c|}{1} & \multicolumn{2}{c|}{2} & \multicolumn{2}{c|}{3} & \multicolumn{2}{c}{$\infty$} \\ \cline{3-10}
\multirow{26}{*}{\rotatebox[origin=c]{90}{Fixed-point iterations per coupling iteration - Flow solver}}
&                       & \eqTime{0.86} & \cIt{9230}& \eqTime{0.86} & \cIt{9161}& \eqTime{-} & \cIt{-}& \eqTime{0.86} & \cIt{8925}\\
& \multirow{-2}{*}{5}   & \fIt{41768} & \sIt{9230}& \fIt{41331} & \sIt{12648}& \fIt{-} & \sIt{-}& \fIt{40263} & \sIt{12745}\\ \cline{2-10}
&                       & \eqTime{0.83} & \cIt{7043}& \eqTime{0.83} & \cIt{6870}& \eqTime{0.83} & \cIt{6950}& \eqTime{0.83} & \cIt{6904}\\
& \multirow{-2}{*}{6}   & \fIt{36636} & \sIt{7043}& \fIt{35769} & \sIt{9372}& \fIt{36119} & \sIt{9834}& \fIt{35821} & \sIt{9740}\\ \cline{2-10}
&                       & \eqTime{0.82} & \cIt{6046}& \eqTime{0.82} & \cIt{5915}& \eqTime{0.82} & \cIt{5962}& \eqTime{0.82} & \cIt{5931}\\
& \multirow{-2}{*}{7}   & \fIt{35391} & \sIt{6046}& \fIt{34493} & \sIt{7962}& \fIt{34693} & \sIt{8379}& \fIt{34583} & \sIt{8298}\\ \cline{2-10}
&                       & \eqTime{0.83} & \cIt{5732}& \eqTime{0.82} & \cIt{5537}& \eqTime{0.82} & \cIt{5532}& \eqTime{0.82} & \cIt{5532}\\
& \multirow{-2}{*}{8}   & \fIt{37298} & \sIt{5732}& \fIt{35942} & \sIt{7499}& \fIt{35847} & \sIt{7806}& \fIt{35888} & \sIt{7818}\\ \cline{2-10}
&                       & \eqTime{0.84} & \cIt{5520}& \eqTime{0.83} & \cIt{5238}& \eqTime{0.83} & \cIt{5277}& \eqTime{0.83} & \cIt{5260}\\
& \multirow{-2}{*}{9}   & \fIt{39476} & \sIt{5520}& \fIt{37397} & \sIt{7089}& \fIt{37563} & \sIt{7469}& \fIt{37400} & \sIt{7451}\\ \cline{2-10}
&                       & \eqTime{0.85} & \cIt{5415}& \eqTime{0.84} & \cIt{5068}& \eqTime{0.84} & \cIt{5073}& \eqTime{0.84} & \cIt{5064}\\
& \multirow{-2}{*}{10}   & \fIt{42025} & \sIt{5415}& \fIt{38964} & \sIt{6849}& \fIt{38875} & \sIt{7188}& \fIt{38801} & \sIt{7169}\\ \cline{2-10}
&                       & \eqTime{0.86} & \cIt{5321}& \eqTime{0.84} & \cIt{4902}& \eqTime{0.84} & \cIt{4914}& \eqTime{0.84} & \cIt{4898}\\
& \multirow{-2}{*}{11}   & \fIt{44199} & \sIt{5321}& \fIt{40305} & \sIt{6586}& \fIt{40344} & \sIt{6939}& \fIt{40218} & \sIt{6925}\\ \cline{2-10}
&                       & \eqTime{0.88} & \cIt{5274}& \eqTime{0.85} & \cIt{4724}& \eqTime{0.85} & \cIt{4723}& \eqTime{0.85} & \cIt{4734}\\
& \multirow{-2}{*}{12}   & \fIt{46864} & \sIt{5274}& \fIt{41452} & \sIt{6346}& \fIt{41516} & \sIt{6692}& \fIt{41540} & \sIt{6708}\\ \cline{2-10}
&                       & \eqTime{0.89} & \cIt{5235}& \eqTime{0.86} & \cIt{4633}& \eqTime{0.86} & \cIt{4649}& \eqTime{0.86} & \cIt{4628}\\
& \multirow{-2}{*}{13}   & \fIt{49252} & \sIt{5235}& \fIt{42899} & \sIt{6235}& \fIt{43022} & \sIt{6591}& \fIt{42911} & \sIt{6563}\\ \cline{2-10}
&                       & \eqTime{0.90} & \cIt{5214}& \eqTime{0.87} & \cIt{4597}& \eqTime{0.87} & \cIt{4594}& \eqTime{0.87} & \cIt{4606}\\
& \multirow{-2}{*}{14}   & \fIt{51729} & \sIt{5214}& \fIt{44826} & \sIt{6197}& \fIt{44727} & \sIt{6535}& \fIt{44823} & \sIt{6543}\\ \cline{2-10}
&                       & \eqTime{0.91} & \cIt{5226}& \eqTime{0.88} & \cIt{4597}& \eqTime{0.88} & \cIt{4569}& \eqTime{0.88} & \cIt{4586}\\
& \multirow{-2}{*}{15}   & \fIt{54210} & \sIt{5226}& \fIt{46922} & \sIt{6205}& \fIt{46783} & \sIt{6500}& \fIt{46931} & \sIt{6524}\\ \cline{2-10}
&                       & \eqTime{0.97} & \cIt{5199}& \eqTime{0.93} & \cIt{4568}& \eqTime{0.93} & \cIt{4543}& \eqTime{0.93} & \cIt{4546}\\
& \multirow{-2}{*}{20}   & \fIt{64912} & \sIt{5199}& \fIt{56188} & \sIt{6173}& \fIt{55750} & \sIt{6491}& \fIt{55814} & \sIt{6489}\\ \cline{2-10}
&                       & \eqTime{1.05} & \cIt{5211}& \eqTime{1.00} & \cIt{4560}& \eqTime{1.00} & \cIt{4582}& \eqTime{1.00} & \cIt{4573}\\
& \multirow{-2}{*}{$\infty$}   & \fIt{79258} & \sIt{5211}& \fIt{69550} & \sIt{6160}& \fIt{69574} & \sIt{6518}& \fIt{69644} & \sIt{6518}
\end{tabular}
\end{table}
\end{comment}
\renewcommand{\minval}{0.794}
\begin{table}
\centering
\caption{Flexible tube case with the FE-FE framework.
The row and column header contain the maximal number of \subproblemIter s for the flow and solid solver, $n_{max}^{f}$ and $n_{max}^{s}$, respectively. For each run, the normalized \textbf{equivalent time} is given, as well as the number of \cIt{coupling iterations}, \fIt{flow solver iterations}, and \sIt{solid solver iterations}.}
\label{tab:FETube}
\begin{tabular}{cc|cc|cc|cc|cc}
\multicolumn{2}{c}{}    & \multicolumn{8}{c}{Newton iterations per coupling iteration - Structural solver} \\
\multicolumn{2}{c}{}    & \multicolumn{2}{c|}{1} & \multicolumn{2}{c|}{2} & \multicolumn{2}{c|}{3} & \multicolumn{2}{c}{$\infty$} \\ \cline{3-10}
\multirow{8}{*}{\rotatebox[origin=c]{90}{\makecell{Newton iterations \\ per coupling iteration \\ - Flow solver}}}
&                       & \eqTime{0.79} & \cIt{920}& \eqTime{0.80} & \cIt{863}& \eqTime{0.82} & \cIt{866}& \eqTime{0.82} & \cIt{870}\\
& \multirow{-2}{*}{1}   & \fIt{920} & \sIt{920}& \fIt{863} & \sIt{1626}& \fIt{866} & \sIt{1840}& \fIt{870} & \sIt{1879}\\ \cline{2-10}
&                       & \eqTime{0.91} & \cIt{815}& \eqTime{0.88} & \cIt{757}& \eqTime{0.89} & \cIt{758}& \eqTime{0.89} & \cIt{752}\\
& \multirow{-2}{*}{2}   & \fIt{1218} & \sIt{815}& \fIt{1109} & \sIt{1414}& \fIt{1112} & \sIt{1623}& \fIt{1107} & \sIt{1642}\\ \cline{2-10}
&                       & \eqTime{1.03} & \cIt{811}& \eqTime{0.99} & \cIt{763}& \eqTime{1.00} & \cIt{761}& \eqTime{1.00} & \cIt{755}\\
& \multirow{-2}{*}{3}   & \fIt{1461} & \sIt{811}& \fIt{1328} & \sIt{1426}& \fIt{1326} & \sIt{1628}& \fIt{1321} & \sIt{1647}\\ \cline{2-10}
&                       & \eqTime{1.03} & \cIt{811}& \eqTime{0.99} & \cIt{763}& \eqTime{1.00} & \cIt{761}& \eqTime{1.00} & \cIt{755}\\
& \multirow{-2}{*}{$\infty$}   & \fIt{1461} & \sIt{811}& \fIt{1328} & \sIt{1426}& \fIt{1326} & \sIt{1628}& \fIt{1321} & \sIt{1647}
\end{tabular}
\end{table}

\renewcommand{\minval}{0.772}
\begin{table}
\centering
\caption{Flexible tube case with the FV-FE framework.
The row and column header contain the maximal number of \subproblemIter s for the flow and solid solver, $n_{max}^{f}$ and $n_{max}^{s}$, respectively. For each run, the normalized \textbf{equivalent time} is given, as well as the number of \cIt{coupling iterations}, \fIt{flow solver iterations}, and \sIt{solid solver iterations}. A missing value indicates that the coupling did not converge.}
\label{tab:FVTube}
\begin{tabular}{cc|cc|cc|cc|cc}
\multicolumn{2}{c}{}    & \multicolumn{8}{c}{Newton iterations per coupling iteration - Structural solver} \\
\multicolumn{2}{c}{}    & \multicolumn{2}{c|}{1} & \multicolumn{2}{c|}{2} & \multicolumn{2}{c|}{3} & \multicolumn{2}{c}{$\infty$} \\ \cline{3-10}
\multirow{30}{*}{\rotatebox[origin=c]{90}{Fixed-point iterations per coupling iteration - Flow solver}}
&                       & \eqTime{1.14} & \cIt{2135}& \eqTime{-} & \cIt{-}& \eqTime{-} & \cIt{-}& \eqTime{-} & \cIt{-}\\
& \multirow{-2}{*}{7}   & \fIt{13567} & \sIt{2135}& \fIt{-} & \sIt{-}& \fIt{-} & \sIt{-}& \fIt{-} & \sIt{-}\\ \cline{2-10}
&                       & \eqTime{0.95} & \cIt{1716}& \eqTime{1.10} & \cIt{1814}& \eqTime{-} & \cIt{-}& \eqTime{1.16} & \cIt{1749}\\
& \multirow{-2}{*}{8}   & \fIt{12120} & \sIt{1716}& \fIt{13006} & \sIt{3471}& \fIt{-} & \sIt{-}& \fIt{12438} & \sIt{5006}\\ \cline{2-10}
&                       & \eqTime{0.88} & \cIt{1544}& \eqTime{0.94} & \cIt{1517}& \eqTime{0.96} & \cIt{1481}& \eqTime{1.06} & \cIt{1586}\\
& \multirow{-2}{*}{9}   & \fIt{11896} & \sIt{1544}& \fIt{11838} & \sIt{2847}& \fIt{11581} & \sIt{3492}& \fIt{12381} & \sIt{4358}\\ \cline{2-10}
&                       & \eqTime{0.82} & \cIt{1400}& \eqTime{0.86} & \cIt{1348}& \eqTime{0.92} & \cIt{1387}& \eqTime{0.94} & \cIt{1385}\\
& \multirow{-2}{*}{10}   & \fIt{11758} & \sIt{1400}& \fIt{11351} & \sIt{2488}& \fIt{11722} & \sIt{3225}& \fIt{11667} & \sIt{3646}\\ \cline{2-10}
&                       & \eqTime{0.81} & \cIt{1355}& \eqTime{0.81} & \cIt{1245}& \eqTime{0.85} & \cIt{1254}& \eqTime{0.88} & \cIt{1279}\\
& \multirow{-2}{*}{11}   & \fIt{12242} & \sIt{1355}& \fIt{11281} & \sIt{2291}& \fIt{11430} & \sIt{2865}& \fIt{11542} & \sIt{3275}\\ \cline{2-10}
&                       & \eqTime{0.80} & \cIt{1299}& \eqTime{0.79} & \cIt{1189}& \eqTime{0.83} & \cIt{1201}& \eqTime{0.84} & \cIt{1196}\\
& \multirow{-2}{*}{12}   & \fIt{12701} & \sIt{1299}& \fIt{11457} & \sIt{2175}& \fIt{11614} & \sIt{2695}& \fIt{11584} & \sIt{3055}\\ \cline{2-10}
&                       & \eqTime{0.82} & \cIt{1299}& \eqTime{0.78} & \cIt{1154}& \eqTime{0.80} & \cIt{1143}& \eqTime{0.81} & \cIt{1133}\\
& \multirow{-2}{*}{13}   & \fIt{13455} & \sIt{1299}& \fIt{11767} & \sIt{2093}& \fIt{11717} & \sIt{2552}& \fIt{11621} & \sIt{2835}\\ \cline{2-10}
&                       & \eqTime{0.84} & \cIt{1289}& \eqTime{0.77} & \cIt{1114}& \eqTime{0.80} & \cIt{1114}& \eqTime{0.83} & \cIt{1137}\\
& \multirow{-2}{*}{14}   & \fIt{14155} & \sIt{1289}& \fIt{12101} & \sIt{2036}& \fIt{12085} & \sIt{2462}& \fIt{12278} & \sIt{2852}\\ \cline{2-10}
&                       & \eqTime{0.85} & \cIt{1278}& \eqTime{0.78} & \cIt{1104}& \eqTime{0.80} & \cIt{1101}& \eqTime{0.80} & \cIt{1076}\\
& \multirow{-2}{*}{15}   & \fIt{14872} & \sIt{1278}& \fIt{12604} & \sIt{2013}& \fIt{12540} & \sIt{2430}& \fIt{12156} & \sIt{2658}\\ \cline{2-10}
&                       & \eqTime{0.93} & \cIt{1264}& \eqTime{0.82} & \cIt{1062}& \eqTime{0.83} & \cIt{1049}& \eqTime{0.84} & \cIt{1046}\\
& \multirow{-2}{*}{20}   & \fIt{18399} & \sIt{1264}& \fIt{14898} & \sIt{1914}& \fIt{14672} & \sIt{2280}& \fIt{14534} & \sIt{2570}\\ \cline{2-10}
&                       & \eqTime{1.05} & \cIt{1255}& \eqTime{0.91} & \cIt{1050}& \eqTime{0.91} & \cIt{1031}& \eqTime{0.92} & \cIt{1033}\\
& \multirow{-2}{*}{30}   & \fIt{23462} & \sIt{1255}& \fIt{18756} & \sIt{1873}& \fIt{18217} & \sIt{2233}& \fIt{18154} & \sIt{2518}\\ \cline{2-10}
&                       & \eqTime{1.12} & \cIt{1264}& \eqTime{0.98} & \cIt{1069}& \eqTime{0.98} & \cIt{1043}& \eqTime{0.97} & \cIt{1010}\\
& \multirow{-2}{*}{40}   & \fIt{26140} & \sIt{1264}& \fIt{21584} & \sIt{1922}& \fIt{21056} & \sIt{2265}& \fIt{20296} & \sIt{2486}\\ \cline{2-10}
&                       & \eqTime{1.14} & \cIt{1273}& \eqTime{1.00} & \cIt{1064}& \eqTime{0.98} & \cIt{1021}& \eqTime{1.01} & \cIt{1037}\\
& \multirow{-2}{*}{50}   & \fIt{26768} & \sIt{1273}& \fIt{22083} & \sIt{1932}& \fIt{21210} & \sIt{2206}& \fIt{21368} & \sIt{2564}\\ \cline{2-10}
&                       & \eqTime{1.14} & \cIt{1263}& \eqTime{0.99} & \cIt{1052}& \eqTime{0.98} & \cIt{1030}& \eqTime{1.00} & \cIt{1028}\\
& \multirow{-2}{*}{$\infty$}   & \fIt{27089} & \sIt{1263}& \fIt{21898} & \sIt{1892}& \fIt{21292} & \sIt{2206}& \fIt{21282} & \sIt{2543}
\end{tabular}
\end{table}


\section{Additional results without reuse}
\label{app:noReuse}

\Tab{FETubeNoReuse} and \Tab{FVTubeNoReuse} present two additional parameter studies for the FE-FE and FV-FE framework, respectively,
simulating the flexible tube case with identical settings, but without reuse of data from past time steps in the IQN Jacobian approximation ($q$=0).

\renewcommand{\minval}{0.526}
\begin{table}
\centering
\caption{Flexible tube case simulated with the FE-FE framework, but without reusing past data in the IQN Jacobian approximation. 
The formatting is the same as in the previous tables.
The cost factors determined by the regression model for this parameter study are, in seconds,
$\costIter^f=0.2177$, $\costIter^s=0.2253$ and $\costCoupleSum=0.1620$.}
%$\costIter^f=0.217656119$, $\costIter^s=0.225342559$ and $\costCoupleSum=0.162$.}
\label{tab:FETubeNoReuse}
\begin{tabular}{cc|cc|cc|cc|cc}
\multicolumn{2}{c}{}    & \multicolumn{8}{c}{Newton iterations per coupling iteration - Structural solver} \\
\multicolumn{2}{c}{}    & \multicolumn{2}{c|}{1} & \multicolumn{2}{c|}{2} & \multicolumn{2}{c|}{3} & \multicolumn{2}{c}{$\infty$} \\ \cline{3-10}
\multirow{8}{*}{\rotatebox[origin=c]{90}{\makecell{Newton iterations \\ per coupling iteration \\ - Flow solver}}}
&                       & \eqTime{0.53} & \cIt{1465}& \eqTime{0.69} & \cIt{1417}& \eqTime{0.76} & \cIt{1417}& \eqTime{0.77} & \cIt{1417}\\
& \multirow{-2}{*}{1}   & \fIt{1465} & \sIt{1465}& \fIt{1417} & \sIt{2734}& \fIt{1417} & \sIt{3330}& \fIt{1417} & \sIt{3362}\\ \cline{2-10}
&                       & \eqTime{0.67} & \cIt{1463}& \eqTime{0.82} & \cIt{1416}& \eqTime{0.90} & \cIt{1416}& \eqTime{0.91} & \cIt{1416}\\
& \multirow{-2}{*}{2}   & \fIt{2573} & \sIt{1463}& \fIt{2479} & \sIt{2732}& \fIt{2479} & \sIt{3328}& \fIt{2479} & \sIt{3360}\\ \cline{2-10}
&                       & \eqTime{0.77} & \cIt{1463}& \eqTime{0.92} & \cIt{1416}& \eqTime{1.00} & \cIt{1416}& \eqTime{1.00} & \cIt{1416}\\
& \multirow{-2}{*}{3}   & \fIt{3318} & \sIt{1463}& \fIt{3206} & \sIt{2732}& \fIt{3206} & \sIt{3328}& \fIt{3206} & \sIt{3360}\\ \cline{2-10}
&                       & \eqTime{0.77} & \cIt{1463}& \eqTime{0.92} & \cIt{1416}& \eqTime{1.00} & \cIt{1416}& \eqTime{1.00} & \cIt{1416}\\
& \multirow{-2}{*}{$\infty$}   & \fIt{3318} & \sIt{1463}& \fIt{3206} & \sIt{2732}& \fIt{3206} & \sIt{3328}& \fIt{3206} & \sIt{3360}
\end{tabular}
\end{table}

\renewcommand{\minval}{0.497}
\begin{table}
\centering
\caption{Flexible tube case simulated with the FV-FE framework, but without reusing past data in the IQN Jacobian approximation. 
The formatting is the same as in the previous tables.
The cost factors determined by the regression model for this parameter study are, in seconds,
$\costIter^f=0.1139$, $\costIter^s=0.2410$, and $\costCoupleSum=1.2825$.}
\label{tab:FVTubeNoReuse}
\begin{tabular}{cc|cc|cc|cc|cc}
\multicolumn{2}{c}{}    & \multicolumn{8}{c}{Newton iterations per coupling iteration - Structural solver} \\
\multicolumn{2}{c}{}    & \multicolumn{2}{c|}{1} & \multicolumn{2}{c|}{2} & \multicolumn{2}{c|}{3} & \multicolumn{2}{c}{$\infty$} \\ \cline{3-10}
\multirow{30}{*}{\rotatebox[origin=c]{90}{Fixed-point iterations per coupling iteration - Flow solver}}
&                       & \eqTime{0.54} & \cIt{2077}& \eqTime{0.61} & \cIt{2120}& \eqTime{0.61} & \cIt{2044}& \eqTime{0.64} & \cIt{2060}\\
& \multirow{-2}{*}{5}   & \fIt{9641} & \sIt{2077}& \fIt{9815} & \sIt{3970}& \fIt{9444} & \sIt{4721}& \fIt{9511} & \sIt{5286}\\ \cline{2-10}
&                       & \eqTime{0.50} & \cIt{1834}& \eqTime{0.55} & \cIt{1827}& \eqTime{0.57} & \cIt{1827}& \eqTime{0.59} & \cIt{1828}\\
& \multirow{-2}{*}{6}   & \fIt{10097} & \sIt{1834}& \fIt{10052} & \sIt{3427}& \fIt{10050} & \sIt{4299}& \fIt{10045} & \sIt{4892}\\ \cline{2-10}
&                       & \eqTime{0.50} & \cIt{1746}& \eqTime{0.54} & \cIt{1757}& \eqTime{0.57} & \cIt{1763}& \eqTime{0.59} & \cIt{1756}\\
& \multirow{-2}{*}{7}   & \fIt{10960} & \sIt{1746}& \fIt{10973} & \sIt{3261}& \fIt{10986} & \sIt{4101}& \fIt{10965} & \sIt{4715}\\ \cline{2-10}
&                       & \eqTime{0.50} & \cIt{1691}& \eqTime{0.54} & \cIt{1681}& \eqTime{0.57} & \cIt{1685}& \eqTime{0.59} & \cIt{1685}\\
& \multirow{-2}{*}{8}   & \fIt{11881} & \sIt{1691}& \fIt{11844} & \sIt{3101}& \fIt{11880} & \sIt{3928}& \fIt{11899} & \sIt{4582}\\ \cline{2-10}
&                       & \eqTime{0.50} & \cIt{1638}& \eqTime{0.54} & \cIt{1610}& \eqTime{0.57} & \cIt{1620}& \eqTime{0.59} & \cIt{1633}\\
& \multirow{-2}{*}{9}   & \fIt{12889} & \sIt{1638}& \fIt{12767} & \sIt{2994}& \fIt{12811} & \sIt{3807}& \fIt{12841} & \sIt{4490}\\ \cline{2-10}
&                       & \eqTime{0.51} & \cIt{1595}& \eqTime{0.55} & \cIt{1578}& \eqTime{0.57} & \cIt{1583}& \eqTime{0.59} & \cIt{1593}\\
& \multirow{-2}{*}{10}   & \fIt{13941} & \sIt{1595}& \fIt{13729} & \sIt{2923}& \fIt{13758} & \sIt{3727}& \fIt{13772} & \sIt{4418}\\ \cline{2-10}
&                       & \eqTime{0.52} & \cIt{1579}& \eqTime{0.55} & \cIt{1555}& \eqTime{0.58} & \cIt{1559}& \eqTime{0.60} & \cIt{1559}\\
& \multirow{-2}{*}{11}   & \fIt{15046} & \sIt{1579}& \fIt{14713} & \sIt{2865}& \fIt{14741} & \sIt{3672}& \fIt{14741} & \sIt{4360}\\ \cline{2-10}
&                       & \eqTime{0.54} & \cIt{1572}& \eqTime{0.56} & \cIt{1531}& \eqTime{0.59} & \cIt{1523}& \eqTime{0.61} & \cIt{1520}\\
& \multirow{-2}{*}{12}   & \fIt{16166} & \sIt{1572}& \fIt{15756} & \sIt{2820}& \fIt{15747} & \sIt{3608}& \fIt{15745} & \sIt{4302}\\ \cline{2-10}
&                       & \eqTime{0.55} & \cIt{1564}& \eqTime{0.58} & \cIt{1520}& \eqTime{0.60} & \cIt{1515}& \eqTime{0.62} & \cIt{1515}\\
& \multirow{-2}{*}{13}   & \fIt{17309} & \sIt{1564}& \fIt{16836} & \sIt{2787}& \fIt{16814} & \sIt{3577}& \fIt{16815} & \sIt{4280}\\ \cline{2-10}
&                       & \eqTime{0.57} & \cIt{1562}& \eqTime{0.59} & \cIt{1511}& \eqTime{0.61} & \cIt{1509}& \eqTime{0.63} & \cIt{1508}\\
& \multirow{-2}{*}{14}   & \fIt{18463} & \sIt{1562}& \fIt{17905} & \sIt{2767}& \fIt{17872} & \sIt{3565}& \fIt{17863} & \sIt{4267}\\ \cline{2-10}
&                       & \eqTime{0.59} & \cIt{1563}& \eqTime{0.60} & \cIt{1502}& \eqTime{0.63} & \cIt{1500}& \eqTime{0.65} & \cIt{1500}\\
& \multirow{-2}{*}{15}   & \fIt{19615} & \sIt{1563}& \fIt{18969} & \sIt{2746}& \fIt{18915} & \sIt{3541}& \fIt{18914} & \sIt{4246}\\ \cline{2-10}
&                       & \eqTime{0.67} & \cIt{1564}& \eqTime{0.67} & \cIt{1483}& \eqTime{0.70} & \cIt{1474}& \eqTime{0.72} & \cIt{1470}\\
& \multirow{-2}{*}{20}   & \fIt{25033} & \sIt{1564}& \fIt{24210} & \sIt{2684}& \fIt{24124} & \sIt{3471}& \fIt{24107} & \sIt{4178}\\ \cline{2-10}
&                       & \eqTime{0.80} & \cIt{1564}& \eqTime{0.81} & \cIt{1479}& \eqTime{0.83} & \cIt{1472}& \eqTime{0.85} & \cIt{1472}\\
& \multirow{-2}{*}{30}   & \fIt{34388} & \sIt{1564}& \fIt{33540} & \sIt{2676}& \fIt{33450} & \sIt{3461}& \fIt{33440} & \sIt{4170}\\ \cline{2-10}
&                       & \eqTime{0.90} & \cIt{1564}& \eqTime{0.91} & \cIt{1479}& \eqTime{0.93} & \cIt{1472}& \eqTime{0.95} & \cIt{1472}\\
& \multirow{-2}{*}{40}   & \fIt{41504} & \sIt{1564}& \fIt{40657} & \sIt{2675}& \fIt{40565} & \sIt{3461}& \fIt{40556} & \sIt{4170}\\ \cline{2-10}
&                       & \eqTime{0.93} & \cIt{1564}& \eqTime{0.94} & \cIt{1479}& \eqTime{0.96} & \cIt{1472}& \eqTime{0.99} & \cIt{1472}\\
& \multirow{-2}{*}{50}   & \fIt{43672} & \sIt{1564}& \fIt{42838} & \sIt{2675}& \fIt{42749} & \sIt{3461}& \fIt{42741} & \sIt{4170}\\ \cline{2-10}
&                       & \eqTime{0.93} & \cIt{1564}& \eqTime{0.96} & \cIt{1479}& \eqTime{0.98} & \cIt{1472}& \eqTime{1.00} & \cIt{1472}\\
& \multirow{-2}{*}{$\infty$}   & \fIt{43677} & \sIt{1564}& \fIt{43793} & \sIt{2675}& \fIt{43704} & \sIt{3461}& \fIt{43697} & \sIt{4170}
\end{tabular}
\end{table}


% \newpage
% \newpage
