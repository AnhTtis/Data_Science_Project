% Commands

% If siunitx is older than version 3
\newcommand{\qty}[1]{\SI{#1}}

% Custom commands and quantitites
\newcommand{\vekt}[1]{\mathbf{#1}}
\newcommand{\fvel}[0]{\vekt{v}^f}
\newcommand{\CauchyStress}[0]{\vekt{T}}
\newcommand{\nsd}[0]{{n_{sd}}}
\newcommand{\foralltime}[0]{\forall t \ge 0}

%  Partitioned FSI, IQN, etc.
\newcommand{\fluidSolver}[0]{{\boldsymbol{\mathcal{F}}}}
\newcommand{\structSolver}[0]{{\boldsymbol{\mathcal{S}}}}
\newcommand{\xtil}[0]{\vekt{\tilde{x}}}
\newcommand{\vektil} [1] { \vekt{\tilde{#1}}}
\newcommand{\Rk}[1]{\vect{r}^{#1}}
\newcommand{\Jac}[0] { \widehat{\vekt{J}^{-1}}}

% Reference to an equation, figure, table, chapter, section, appendix, algorithm, line, step, page and correction
\newcommand{\Equ}[1]{Eq.~(\ref{equ:#1})}
\newcommand{\Eqs}[1]{Eqs.~(\ref{eqs:#1})}
\newcommand{\Fig}[1]{Figure~\ref{fig:#1}}
\newcommand{\Tab}[1]{Table~\ref{tab:#1}}
\newcommand{\Cha}[1]{Chapter~\ref{cha:#1}}
\newcommand{\Sec}[1]{Section~\ref{sec:#1}}
%\newcommand{\App}[1]{Appendix~\ref{app:#1}}
\newcommand{\App}[1]{\ref{app:#1}}
\newcommand{\Alg}[1]{Algorithm~\ref{alg:#1}}
\newcommand{\Lin}[1]{line~\ref{lin:#1}}
\newcommand{\Ste}[1]{step~\ref{ste:#1}}
\newcommand{\Pag}[1]{page~\pageref{pag:#1}}
\newcommand{\Rem}[1]{Remark~\ref{Remark:#1}}
\newcommand{\Fnt}[1]{footnote~\ref{fnt:#1}}

% Often used symbols
\newcommand{\uv}[0]{\vect{u}}
\newcommand{\bv}[0]{\vect{b}}
\newcommand{\Am}[0]{\mat{A}}
\newcommand{\Mm}[0]{\mat{M}}
\newcommand{\resSolid}[0]{\vekt{r}_s}
\newcommand{\resFlow}[0]{\vekt{r}_f}
\newcommand{\resProblem}[0]{\vekt{r}_p}

\newcommand{\abs}[1]{\left\lvert#1\right\rvert}
\newcommand{\norm}[1]{\left\lVert#1\right\rVert}
\newcommand{\normE}[1]{\norm{#1}_2}
\newcommand{\normF}[1]{\norm{#1}_F}
\newcommand{\normEsqrt}[1]{\norm{#1}_{2,\sqrt{n}}}

% Flow and solid quantities
\newcommand{\flowDomain}[0]{\Omega^f}
\newcommand{\solidDomain}[0]{\Omega^s}

% Coloring in table
% Gradient macro
\newcommand{\minval}{0.0} % this value is updated for each table
\newcommand{\maxval}{1.0}
\definecolor{high}{HTML}{ff0000}
\definecolor{mid}{HTML}{ffff00}
\definecolor{low}{HTML}{22b20c}
\newcommand{\opacity}{40}
\newcommand{\ApplyGradient}[1]{
    \pgfmathsetmacro{\midval}{(\minval+\maxval)/2}
    \IfStrEq{#1}{-}{
        \cellcolor{high!\opacity} #1
    }{
        \IfDecimal{#1}{
            \ifdim #1 pt > \midval pt
                \pgfmathparse{max(min(100*(#1-\midval)/(\maxval-\midval),100),0)}
                \xdef\tempa{\pgfmathresult}
                \hspace{-0.33em}\cellcolor{high!\tempa!mid!\opacity} #1
            \else
                \pgfmathparse{max(min(100*(\midval-#1)/(\midval-\minval),100),0)}
                \xdef\tempa{\pgfmathresult}
                \hspace{-0.33em}\cellcolor{low!\tempa!mid!\opacity} #1
            \fi
        }{
            #1
        }
    }
}
% Table formatting
\newcommand{\eqTime}[1]{\textbf{\ApplyGradient{#1}}}
\newcommand{\cIt}[1]{\textit{\textcolor{black}{#1}}}
\newcommand{\fIt}[1]{\textcolor{blue}{#1}}
\newcommand{\sIt}[1]{\textcolor{orange}{#1}}

% Names within the text
\newcommand{\subproblemIter}[0]{subproblem iteration}
% \newcommand{\subproblemIters}[0]{\subproblemIter s }

% Mathematical quantities
\newcommand{\mat} [1]{\mathbf{#1}}
\newcommand{\vect}[1]{\mathbf{#1}}
\newcommand{\E}[1]{\num{e#1}}

% Dimensions
\newcommand{\inR}[1]{\in \mathbb{R}^{#1}}
\newcommand{\ofxt}[0]{(\vect{x},t)}
\newcommand{\nInterfaceDofs}[0]{n_\Gamma}

% Cost factors
\newcommand{\costFac}[1]{C_{\text{#1}}}
\newcommand{\costCouple}[0]{\costFac{}^c}
\newcommand{\costSolver}[0]{\costFac{}}
\newcommand{\costIter}[0]{\costFac{iter}}
\newcommand{\costFix}[0]{\costFac{fix}}
\newcommand{\costSimulation}[0]{\costFac{simulation}}
\newcommand{\costCoupleSum}[0]{\overline{\costFac{}}^c}

% Time measures
\newcommand{\timeSimulation}[0]{t_{\text{simulation}}}

% Iteration counts
\newcommand{\totalCoupleIter}[0]{N^{c}}
\newcommand{\totalProblemIter}[0]{N}
\newcommand{\iterPerCall}[0]{n_{max}}

% Iteration index
\newcommand{\kbar}[0]{\bar{k}}

% Timing
\newcommand{\timing}[0]{t}

% Number of calculations
\newcommand{\numCalc}[0]{m}

% Sums
\newcommand{\sumProblems}[0]{\sum_{p=f,s}}
\newcommand{\sumCplIter}[0]{\sum_{\kbar=1}^{\totalCoupleIter}}

% Left indices
\newcommand{\leftIdx}[2] { \mbox{}^{#1} #2}

% These commands include the outline in the text
\newcommand{\sectionOutline}[2]{\textcolor{blue}{\underline{#1}: #2 }}
\newcommand{\outline}[1]{\textcolor{blue}{#1}}

% Use this version to produce document without outline
% \newcommand{\outline}[1]{}
% \newcommand{\sectionOutline}[1]{}

% Remark
\newcounter{para}
\newcommand \mypara{\refstepcounter{para} \par \noindent\textit{Remark \thepara:\space}}

% Formatting
\newcommand{\marktext}[1]{\textit{#1}}

% Blank footnote
\makeatletter
\def\blfootnote{\xdef\@thefnmark{}\@footnotetext}
\makeatother
