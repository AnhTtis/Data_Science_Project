
This work studies the influence of the nonlinear \subproblemIter s on the 
computational cost of a partitioned scheme.
Although the focus of this work is on partitioned algorithms for fluid-structure interaction problems, its findings are more widely applicable to any multi-field problem that is solved by coupling black-box solvers.
% 
While it is common in FSI literature to consider only the total number of coupling iterations,
%
the \subproblemIter s are demonstrated 
% to have a significant influence on the total run time
to be just as crucial for the overall computational cost.
% vital for an accurate cost measure.
% 
% the \subproblemIter s are demonstrated to be
Consequently, 
converging in as few coupling iterations as possible 
does not necessarily minimize the run time of the partitioned simulation.
% 
In order to be efficient, a coupling algorithm instead
should reduce the number of \subproblemIter s in the flow and solid solver too.
% 
The cost measure proposed in this work therefore
% The weighting factors of these 
weights these conflicting objectives with the cost of one coupling, flow or solid iteration, respectively.
% 
% which depend, i.a.,
%
Among other things, 
the cost factors depend on the considered problem, the employed solvers, and the computer hardware,
making them very case-specific.
% 

Nevertheless,
this work sheds light on the relation between \subproblemIter s and computational cost based on typical benchmark problems and common solver types by running multiple parameter studies.
% 
% 
% To make sure the solution fields of each run are solved with the same accuracy,
% a new convergence criterion is proposed
% that relies solely on the convergence bounds of the subproblem residuals, rather than introducing an own epsilon
% for the fixed-point residual as commonly done in literature.
% % 
% 
% The parameter studies systematically investigate how limiting the maximum number of \subproblemIter s per solver call
% influences
% the total number of coupling, flow, and solid iterations as well as the required run time.
% 
These studies systematically investigate how limiting the maximum number of \subproblemIter s per solver call
influences
the total number of coupling, flow, and solid iterations as well as the required run time.
% 
To make sure all parameter sets yield the same solution fields,
% accuracy is not affected by the number of \subproblemIter s run per solver call,
a new convergence criterion is proposed
that relies solely on the convergence tolerances of the subproblem residuals, rather than introducing an own tolerance
for the fixed-point residual as commonly done in literature
This new way of assessing convergence can obviously also be applied in other FSI simulations.

The key finding of the parameter studies is that limiting
the \subproblemIter s per solver call
% The parameter studies
% 
% to investigate the impact of the \subproblemIter s on the computational cost.
% 
% In particular, it demonstrates that 
% limiting the maximum number of \subproblemIter s run per solver call
cannot only reduce the total number of \subproblemIter s,
but also significantly lower the simulation's overall computational cost. 
% 
This raises the question of how many \subproblemIter s per solver call yield the most efficient partitioned algorithm.
% % to minimize computational cost.

Unfortunately, 
this question
is as non-trivial as it is important.
% 
Although a definite answer is yet to be found and is expected to be case-specific,
the results discussed in this work demonstrate that iterating to full convergence in every solver call 
typically causes a computational overhead.
% 
Instead, running only a few \subproblemIter s in each coupling step can be much more efficient. 
For the four parameter studies, the maximum 
reduction in computational cost was above $20\,\%$.
Depending on the coupling technique, even higher savings are possible, as indicated by the studies without reuse of past time step data in the IQN for which the cost was almost halved.
% 
The ideal choice may be difficult or even impossible to determine a priori, 
but the results show that the optimum is rather flat.
% 
For example, limiting the Newton iterations per solver call to $2$ and the fixed-point iterations to $12$
led to a speed up of $12\,\%$ to $24\,\%$ for all test cases and frameworks investigated.
% 
While these performance gains might be no quantum leap, it should be noted that they come without any additional effort, cost, or code changes.

Furthermore, the limits imposed on the number of \subproblemIter s per solver call were 
kept fixed here for the sake of cleaner parameter studies that are easier to interpret. 
Dynamically adapting $\iterPerCall^f$ and $\iterPerCall^s$
in a ``smart'' way,
for example based on some quality measure for the coupling data, 
% like the fixed-point residual,
has the potential to further shorten the run time.\\


% In conclusion, this work demonstrates the importance of the \subproblemIter s run per solver call for the computational cost of a partitioned algorithm. 
% In doing so, 

All in all, the primary goal of this work is to raise awareness
% 
of a gap in current literature on FSI and other coupled problems,
% and 
% 
concerning
the number of \subproblemIter s performed per solver call 
and how it influences the computational cost of a partitioned algorithm.
% Moreover, first groundwork is layed for 
In addition, this work lays the groundwork
% and to lay the groundwork 
for potential future research bridging this gap.
% 
% Lastly, it is important to note that beyond fluid-structure interaction, 
% the investigations and conclusions of this work are expected to be applicable to all multi-field problems solved in a partitioned manner.

% The underlying parameter studies systematically investigate how such a limit influences
% the total number of coupling, flow and solid iterations as well as the required run time,
% % 
% highlighting the importance of asking
% how many \subproblemIter s per solver call yield the most efficient partitioned algorithm.
% % to minimize computational cost.

% To make sure the solution fields of all runs are identical, a new convergence criterion is proposed
% that relies solely on the convergence bounds of the subproblem residuals, rather than introducing an own epsilon
% for the fixed-point residual as commonly done in literature.

% Instead of introducing an own bound for the fixed-point residual as commonly done,
% this criterion relies only on the convergence bounds of the subproblem residuals
% to derive a clear and comparable measure for convergence.

% Unfortunately, a definite to the question of how many \subproblemIter s should be run per solver call to minimize computational cost
% is as non-trivial 

% The question of how many
% This underlines that the question of how many \subproblemIter s should be run per solver call to minimize computational cost
% % is as non-trivial 

% \vspace{2cm}
% ---------------------------------------------------------------------------------------------\\

% as a measure for 

% On the one hand, 
% the total number of \subproblemIter s is demonstrated to be just as vital for an accurate cost measure
% as the coupling iterations.


% % 
% First, the total number of \subproblemIter s is demonstrated to be essential... \\
% % 