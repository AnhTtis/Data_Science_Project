
Partitioned FSI couples two black-box solvers: one for the flow and one for the solid problem.
To increase the scope of this work, two different sets of solution techniques for the subproblems are investigated, both of which are widely used in FSI.
The first one uses finite-element (FE) methods for both the flow and solid problem, whereas the second framework relies on a finite-element method for the solid, but a finite-volume (FV) method for the flow problem.
Both setups are summarized below:

\begin{itemize}
% FE-FE
    \item In the first simulation framework, the in-house solver XNS discretizes the flow problem by stabilized Lagrangian P1-P1 finite elements in space \cite{Pauli2017,donea2003finite} and a backward Euler method in time \cite{forti2015semi}. The ALE mesh is adapted to deforming domains via the linear elastic mesh-update method \cite{johnson1994mesh,behr2002free},
    and its velocity determined by a first-order finite difference scheme \cite{forster2006geometric}. The solid subproblem is solved by the in-house code FEAFA using isogeometric analysis \cite{hughes2005isogeometric,cottrell2009isogeometric}, a spline-based variant of finite elements, in space and a generalized-$\alpha$ scheme in time \cite{chung1993time,erlicher2002analysis}. Non-matching interface discretizations are handled by a spline-enhanced version of finite-interpolation elements \cite{hostersspline, makespline}.
    This setup will in the following be labeled \textbf{FE-FE}.
% FV-FV
    \item The second framework uses a finite-volume method for the flow problem and a backward Euler discretization in time within the commercial solver ANSYS Fluent \cite{Fluent2019R3}.
    The mesh is structured and the discretization scheme for the convection terms of the momentum equations is second-order upwind \cite{Barth1989}.
    For the pressure equation, the second-order and the standard scheme \cite{Rhie1983} are used for the lid-driven cavity and flexible tube case, respectively, see \Sec{results}.
    The deforming fluid domain is included using the arbitrary Lagrangian-Eulerian (ALE) frame of reference and mesh deformation is based on spring-based smoothing \cite{Batina1990}. The solid problem is discretized by piecewise linear finite elements in space and a generalized-$\alpha$ scheme in time, using the Structural Mechanics Application of the Kratos Multiphysiscs code \cite{Ferrandiz2022}. The coupling between the two is performed by the in-house code CoCoNuT \cite{Delaisse2022}. The most recent code can be found in the GitHub repository \href{https://github.com/pyfsi/coconut}{pyfsi/coconut}. Data exchange on the non-matching interface is realized with radial basis mapping \cite{Lombardi2013}. This set of solution techniques will from here on be termed \textbf{FV-FE}.
\end{itemize}

Both finite-element and finite-volume methods are common discretization schemes in modern computational engineering science.
Since they are well-documented in literature, see for example \cite{Mueller2015, Versteeg1995, ferziger2002} for finite volumes and \cite{Bathe1996, zienkiewicz2005finite, zienkiewicz2005femsolids, reddy2019introduction} for finite elements, any in-depth discussion is omitted here for the sake of conciseness.

The aspect most important for this work is that both techniques transform a continuous partial differential equation (PDE) and its boundary conditions into a discrete set of algebraic equations.
In general, this system of equations is nonlinear, meaning that the system matrix $\Am$ depends on the solution $\uv$, yielding the matrix form
\begin{equation}
    \label{equ:nonlinearSystem}
    \Am(\uv) \, \uv = \bv,
\end{equation}
where $\Am \inR{n \times n}$ is the sparse coefficient matrix, $\uv \inR{n}$ holds the dependent variables of interest, 
% $\bv \inR{n}$ contains source terms and boundary conditions,
and $n$ denotes the number of degrees of freedom (DOF).
% 
% Note that 0the RHS $\bv$ 
% 
Note that the right-hand side (RHS) vector $\bv$, containing source terms and boundary conditions,
is considered independent of $\uv$ within each subproblem\footnote{In many algorithms, part of the dependence on $\uv$ is treated explicitly resulting in a lagging contribution to the RHS $\bv$. Since this is not the focus in this work and for the sake of simplicity, it is assumed that all dependence on $\uv$ is treated implicitly, i.e., within the system matrix $\Am(\uv)$.\label{fnt:implicit}}. \Sec{convergence_criterion} will explain, however, that when coupling two subproblems as in FSI, the RHS in fact becomes a function of $\uv$ as well, i.e., $\bv = \bv(\uv)$.
%
For unsteady PDEs, such as \Eqs{INS_Strong} or \Equ{Elasto_Strong}, additionally a time-stepping scheme is applied, so that an algebraic system of the form \Equ{nonlinearSystem} is obtained for every time step.
It can then be solved using suitable numerical methods.

%The nonlinearity of the right-hand side, however, is typically neglected as no explicit function $\mathbf{b(u)}$ is available. %-> keep this for later? or is this something related to FE

The following subsections focus on how the nonlinearity in the discrete system of equations \Equ{nonlinearSystem} is treated.
Although the system of equations can be solved using Newton or fixed-point iterations in both finite-element and finite-volume methods, this work follows the most common approach: 
Newton iterations for the finite-element method and fixed-point iterations for the finite-volume method.
The most important takeaway is that finite-element method solves \Equ{nonlinearSystem} with a few Newton iterations (typically 2 to 5), while the finite-volume method uses much cheaper fixed-point iterations to reach the solution, but requires a larger number of them (typically 50 to 200).

\subsection{Finite-element method} \label{sec:FE_Newton}


Although originally developed for solid mechanics,
% today
finite-element methods are widely used in many fields of computational engineering,
including fluid dynamics \cite{donea2003finite,reddy2010finite}.
% 
After transforming the PDE into its variational form,
the domain is divided into a mesh of \textit{elements}
to approximate the solution by a linear combination of the elements' basis functions.
% 
The unknown coefficients
% representing for example the displacement, velocity or pressure of each mesh node,
% i.e., the unknowns,
are then determined from the resulting algebraic system of the form \Equ{nonlinearSystem}.

% treatment of nonlinear system of equations
In finite-element methods, it is common practice to employ Newton's method to tackle the nonlinearity of $\Am$.
Starting from an initial guess $\mathbf{u}^0$, 
in each iteration $i=1,\cdots$
the linearized system
\begin{equation}
    \label{equ:linearizedFE}
    \underbrace{ \left( \Am^{i-1} + \left.\frac{\partial \Am}{ \partial \uv} \right\vert_{i-1} \, \uv^{i-1} \right)}_{=: \mat{K}^{i-1}} ~\Delta \uv^i 
    = \underbrace{ \bv - \Am^{i-1} \, \uv^{i-1} }_{=: \resProblem^{i-1}}
    \quad \Longleftrightarrow \quad
    \mat{K}^{i-1} \, \Delta \uv^i = \resProblem^{i-1} 
    % \underbrace{ \left( \Am^i + \left.\frac{\partial \Am}{ \partial \uv} \right\vert_i \, \uv^i \right)}_{=: \mat{K}^i} ~\Delta \uv^i 
    % = \underbrace{ \bv - \Am^i \, \uv^i }_{=: \vect{r}^i}
    % \quad \Longleftrightarrow \quad
    % \mat{K}^i ~ \Delta \uv^i = \vect{r}^i 
\end{equation}
 is solved for the solution increment $\Delta \uv^i$, where $\mat{K}^{i-1}$ is the tangent stiffness (or system) matrix,
and
$\Am^{i-1}$ is a shorthand notation for $\Am(\uv^{i-1})$,
Further,
$\resProblem^{i-1}$ is the residual vector of the considered subproblem,
i.e., either the fluid ($p=f$) or the solid ($p=s$) problem.
Subsequently, the solution field is updated by $\uv^{i} = \uv^{i-1} + \Delta \uv^i$.
The Newton iteration is considered converged when the residual norm is lower than some tolerance $\varepsilon$, i.e., if $\frac{\normE{\resProblem^i}}{\sqrt{n}}< \varepsilon$.
% $\normEsqrt{\vect{r}^i}< \varepsilon$.

Note that the computational cost of this procedure is typically dominated by the assembly of the linear system in \Equ{linearizedFE} on the one hand and its numerical solution on the other hand (in this work via a preconditioned GMRES \cite{saad1986} approach). %also for Kratos?
Both these operations are repeated for every Newton iteration.


\subsection{Finite-volume method}
% use of FV
Finite-volume methods are very common for solving flow problems.
% short description of FV
The principle of the finite-volume method is to discretize the spatial domain into \textit{finite volumes} or cells and apply the integral form of the governing equations on each of them.
In these integral forms, the volume integral of the divergence is transformed into a surface integral over its boundaries, using Gauss's theorem, so that this method is conservative by design.
Finally, this results in a large system of algebraic equations in the form of \Equ{nonlinearSystem}.
In pressure-based finite-volume solvers, pressure and velocity are either solved for together, in a so-called coupled approach (as done in this work), or sequentially, in a segregated approach. In the latter case, \Equ{nonlinearSystem} is only a symbolic notation.

% treatment of nonlinear system of equations
The system of equations can also be solved with Newton iterations, but it is more common to linearize the nonlinear coefficient matrix $\Am$ using Picard or fixed-point iterations
\begin{equation}
    \label{equ:linearizedFV}
    \Mm^{i-1} \, (\uv^{i}-\uv^{i-1}) = \bv - \Am^{i-1} \, \uv^{i-1},
\end{equation}
where $\Mm^{i-1}=\Mm(\uv^{i-1})$ is some approximation of $\Am^{i-1}$, e.g., $\mathrm{diag}\,(\Am^{i-1})$.
It should be noted that this notation is only symbolic, as the actual solution techniques are more involved and therefore out of scope for this work, see for example the multigrid methodologies, e.g., the algebraic multigrid method (AMG) with smoother \cite{Stueben2001}.
%It should be noted that within every iteration this linear system is not solved fully, but only up to a certain degree of accuracy.
The iteration is considered converged when the residual norm is lower than some tolerance $\varepsilon$.
In this work, the residual is calculated as the total imbalance scaled by a factor representative for the flow rate of the respective component of $\uv$ through the domain.
%In this work, the iteration is considered converged when the average imbalance is lower than some bound $\varepsilon$, i.e. if $\frac{1}{\sqrt{m}}\sum_{m}\abs{r^j}< \varepsilon$, where $r^j$ ($1<j<m)$ are the components of $\vect{r}$ and $m$ is the number of cells TODO. 
%and solved with an iterative procedure, e.g., algebraic mutligrid method (AMG) with a smoother.
%Newton-like methods such as in finite elements are typically not used, as the generation of the Jacobian and the subsequent solving of the system up to machine tolerance are so expensive that the overall cost is higher, even though a Newton method would converge in only a few iterations \cite{ferziger2002}.
