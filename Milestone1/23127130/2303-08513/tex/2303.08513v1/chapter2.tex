
For the greater part,
the research questions investigated in this work
are expected to affect
% not only fluid-structure interaction, but 
any multi-field problem
solved in a partitioned manner.
% 
Nevertheless, 
we restrict ourselves to 
the interaction of an incompressible fluid in the domain $\flowDomain \inR{\nsd}$ and 
an elasto-dynamic solid $\solidDomain \inR{\nsd}$,
where $\nsd$ is the number of spatial dimensions.


\subsection{Incompressible flow}


The flow velocity $\fvel \ofxt$ and the fluid pressure $p^f \ofxt$
are governed by
the unsteady Navier-Stokes equations
% 
% 
% The flow problem is governed by the unsteady Navier-Stokes equations
\begin{subequations} \label{eqs:INS_Strong}
	\begin{alignat}{2} 
		\frac{\partial \fvel}{\partial t} + \fvel \cdot \boldsymbol{\nabla} \fvel  - \frac{1}{\rho^f} \boldsymbol{\nabla}  \cdot \CauchyStress^f &= \vekt{b}^f	\qquad	&& \text{in} ~\Omega^f ~~\foralltime~,\\
		\boldsymbol{\nabla}  \cdot \fvel \,&= 0 && \text{in} ~\Omega^f ~~\foralltime~,
	\end{alignat}
\end{subequations}
with the constant fluid density $\rho^f$ and the external body force $\vect{b}^f$.
% , and points in space and time $\vect{x}$ and $t$.
% It is solved for the flow velocity $\fvel \ofxt$ and the pressure $p^f$.
% 
For an incompressible Newtonian fluid,
the Cauchy stress tensor is given by $\CauchyStress^f( \fvel, p^f) = - p^f \vekt{I} + \mu^f \left(  \nabla \fvel + (\nabla \fvel)^T \right)$,
where $\mu^f$ is the dynamic viscosity.
% 
The problem is closed by an appropriate set of boundary conditions on $\partial \Omega^f$
%$\Gamma^{f}=\partial \Omega^f$
and a divergence-free initial velocity field.

\subsection{Elastic solid}


% The solid problem is formulated in a Lagrangian manner with respect to the undeformed reference configuration.
% Following a total Lagrangian formulation,
The solid displacement field $\vekt{d}^s(\vekt{x},t)$ is determined from 

% \begin{alignat}{2} \label{equ:Elasto_Strong}
% 	\frac{d^2 \vekt{d}^s}{dt^2} - \frac{1}{\rho^s} \boldsymbol{\nabla} \cdot \CauchyStress^s &= \vekt{b}^s \qquad &&\text{in } \Omega^s ~~\foralltime~,
% \end{alignat}
% with the solid density $\rho^s$, the external body force $\vect{b}^s$,
% and the Cauchy stress tensor $\CauchyStress^s$.
\begin{alignat}{2} \label{equ:Elasto_Strong}
	\frac{d^2 \vekt{d}^s}{dt^2} - \frac{1}{\rho_0^s} \boldsymbol{\nabla}_0 \cdot \left( \mat{S}^s \mat{F}^T \right) &= \vekt{b}^s \qquad &&\text{in } \Omega_0^s ~~\foralltime~,
\end{alignat}
with the solid density $\rho_0^s$ and the external body force $\vect{b}^s$.
% 
% The subscript $0$ indicates that a quantity or operation refers to the undeformed domain $\Omega_0^s$.
% 
Following a total Lagrangian viewpoint, this equation of motion is formulated with respect to the
undeformed reference state $\Omega_0^s$, indicated for all affected quantities and operators by the subscript $0$.
Accordingly, the inner stress is not expressed in terms of the
Cauchy stress tensor $\CauchyStress^s$, but the second Piola-Kirchhoff stress tensor $\mat{S}^s = \det(\mat{F})\, \mat{F}^{-1} \CauchyStress^s \mat{F}^{-T}$
with the deformation gradient $\mat{F}$.
% and the second Piola-Kirchhoff stress tensor $\mat{S}^s$, which is defined based on the
% Cauchy stress tensor $\CauchyStress^s$ and the deformation gradient $\mat{F}$ by
% $\mat{S}$
% 
As constitutive equation, a Hookean or the St. Venant-Kirchhoff material model is used, resulting in a geometrically nonlinear solid problem \cite{Bathe1996}.
It is closed by an initial (zero) displacement field and suitable boundary conditions on $\partial \Omega^s$.

\subsection{Coupling conditions}


To ensure the conservation of mass, momentum, and mechanical energy over the 
shared interface  
$\Gamma^{fs} = \partial \flowDomain \cap \partial \solidDomain$,
the solution fields of the two subproblems have to satisfy
kinematic and dynamic continuity:
\begin{subequations}
\begin{align}
    \vekt{d}^f &= \vekt{d}^s && \text{on } \Gamma^{fs} ~~\foralltime ~, \label{equ:kinCont} \\
    \CauchyStress^f  \cdot \vekt{n}^f  &= - \CauchyStress^s \cdot \vekt{n}^s  && \text{on } \Gamma^{fs} ~~\foralltime ~,
\end{align}
\end{subequations}
where $\vect{d}^f$ is the fluid's displacement, while $\vekt{n}^{f}$ and $\vekt{n}^{s}$ are the interface unit normal vectors pointing outwards from the corresponding domains.
Note that \Equ{kinCont} implies the equality of velocities and accelerations too. 

\subsection{Dirichlet-Neumann partitioning}


In partitioned fluid-structure interaction simulations,
the two subproblems are addressed by two distinct solvers that are coupled in a black-box manner, i.e., solely via the exchange of interface data.
% 
This strategy is very flexible and modular concerning the solvers,
but their communication entails some additional challenges.
On the one hand, the interface discretizations of the two subproblems in general do not match, so that transferring data requires a spatial projection.
% ('\marktext{spatial coupling}').
On the other hand, 
an iterative procedure is needed to find consistent solutions of the
two interdependent subproblems within each time step.
% ('\marktext{temporal coupling}').

The most common partitioned approach is the combination of a \marktext{Dirichlet-Neumann} partitioning with a Gauss-Seidel type iteration 
scheme\footnote{Although all numerical experiments of this work use a Dirichlet-Neumann partitioning, the investigated impact of the \subproblemIter s on computational cost is expected to be essentially the same for other partitionings, such as Robin-Neumann or Robin-Robin schemes \cite{Badia2008,nobile2008effective,spenkeRNQN}.}.
For every coupling iteration, it solves the flow problem with the current interface deformation and passes the interface tractions 
% $(\CauchyStress \vect{n})^k = \mathcal{F}(\vect{d}^k)$ 
as a Neumann boundary condition to the solid.
The solid solver then computes the new deformation state and returns the interface displacement
to the flow solver,
% 
where the resulting interface velocity 
% prescribing the interface velocity as a 
poses a
% for which the resulting interface velocity
Dirichlet condition.
Once this procedure has converged, the next time step is started.

The main drawback of the Dirichlet-Neumann partitioning is its sensitivity to the added-mass effect. 
%
% Roughly speaking, this instability is characterized by an overestimation of the deformation state that is amplified throughout the coupling iterations.
For this work, it is sufficient to note that this instability 
is inherent to partitioned solution schemes and
increases mainly with the density ratio $\rho^f/\rho^s$, but more
detailed investigations can be found in literature \cite{Causin2005, Forster2007, VanBrummelen2009}.


\subsection{Interface quasi-Newton methods}
\label{sec:IQN}

An effective countermeasure against the added-mass effect is
to modify the interface deformation with an \marktext{interface quasi-Newton} (IQN) method
before passing it to the flow solver\footnote{Although it is a lot less common, updating the interface tractions before passing them to the solid solver is possible too \cite{spenkeRNQN}.}.
% 
Identifying the solution of the coupled problem as a fixed-point of the coupling iteration loop,
their basic idea is to employ the Newton-like update step
\begin{align}
    \vect{d}^{k+1} = 
    \vect{\tilde{d}}^k + \Delta\vect{\tilde{d}}^k_{IQN} =
    \tilde{\vect{d}}^k - \left( \frac{\partial \Rk{k}}{\partial \tilde{\vect{d}}^k} \right)^{-1} \Rk{k} ~,
\end{align}
% and pass the resulting interface deformation $$\vect{d}^{k+1}$ rather than the to the flow solver 
where $\tilde{\vect{d}}^k$ is the interface deformation computed by the solid solver 
% in coupling iteration $k$ 
and $\vect{d}^{k+1}$ the one sent to the flow solver, before starting the next coupling iteration $k+1$. 
$\Rk{k} \equiv \tilde{\vect{d}}^k -\vect{d}^k$ denotes the fixed-point residual of the interface deformation.

Since the exact Jacobian $\frac{\partial \Rk{k}}{\partial \tilde{\vect{d}}^k}$ is not available for black-box solvers, however,
a low-rank approximation is used instead.  
To avoid additional solver calls, this inverse Jacobian approximation $\Jac  \approx \left( \frac{\partial \Rk{k}}{\partial \tilde{\vect{d}}^k} \right)^{-1}$ is constructed from the interface deformation states that were computed in previous coupling iterations. 

For details on the concept of interface quasi-Newton methods, different variants, and implementation aspects
the authors recommend the works \cite{Delaisse2023, spenkeRNQN, spenke2020multi, Delaisse2022, Lindner2015, Degroote2009}.