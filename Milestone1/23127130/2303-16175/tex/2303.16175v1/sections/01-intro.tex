\section{Introduction}\label{sec:intro}

Writing assistants (\WA) are changing the way people interact with text. These tools can be incredibly useful in improving the quality and efficiency of writing~\cite{gero2022design}. However, with recent progress in the development of large language models such as GPT-3~\cite{brown2020language} and MT-NLG~\cite{smith2022using}, \WA can create long output~\cite{shi2022effidit} that can sometimes be overwhelming and difficult to manage for anyone and especially for neurodiverse users. Creating tools that aim to reduce cognitive load during writing-related tasks can help address this problem. 

The programming field has long been working on this issue. Code writing is a mentally challenging task that imposes much cognitive load on the user. For this reason, Integrated Development Environments (IDEs), an ecosystem of \WA, have been created to make code writing more efficient and less demanding for mental effort~\cite{habeck2008security,fakhoury2020measuring,hunter2021ten}. Over the last several decades, a lot of work has already been done to support users' performance by incorporating programming language peculiarities into IDEs (\textit{e.g.}, keyword coloring, code writing conventions, and fixed context). However, there is still a long way to go. To embrace inclusivity, thorough research should be carried out on possible interventions to reduce the cognitive load of various groups of users. As an ever-changing industry-level tool, an IDE is an excellent instrument for that: tasks, changes, and measurements can be easily incorporated into the system by plugins to test hypotheses~\cite{denissov2021creating,ramos2022tool}.

We argue that established practices in the development of IDEs, as well as the use of IDEs to research \WA, can be fruitful for next-generation \WA development.