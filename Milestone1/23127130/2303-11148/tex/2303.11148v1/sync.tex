\documentclass[twocolumn,pre,floatfix]{revtex4-2}
\usepackage{psfrag,epsfig,amsfonts,amssymb,amsmath,wasysym,bm}
\usepackage{dcolumn}
%\usepackage{bbold}
\usepackage[normalem]{ulem}
\usepackage{color}
\usepackage{tabularx}
\usepackage{tikz}
\usepackage{ulem}

\usepackage[inline]{enumitem} % enumerated lists


%%%%%%%%%%
%\usepackage[notref,notcite]{showkeys}
\usepackage{hyperref}
%%%%%%%%%%

\newcommand{\ket}[1]{\lvert #1 \rangle} 	% ket
\newcommand{\bra}[1]{\langle #1 \rvert}	% bra
\newcommand{\ketN}[1]{\lvert #1 \rangle_{\! 0}} 	% ket
\newcommand{\braN}[1]{{_0\!}\langle #1 \rvert}	% bra
\newcommand{\ketL}[1]{\lvert #1 \rangle_{\! \lambda}} 	% ket
\newcommand{\braL}[1]{{_\lambda\!}\langle #1 \rvert}	% bra
\newcommand{\ketP}[1]{\lvert #1 \rangle^{\! \prime}} 	% ket
\newcommand{\braP}[1]{{^\prime\!}\langle #1 \rvert}	% bra
\newcommand{\ketV}[1]{\lvert #1 \rangle_{\! V}} 	% ket
\newcommand{\braV}[1]{{_V\!}\langle #1 \rvert}	% bra
\newcommand{\mc}{\mathcal}	% calligraphic
\newcommand{\sUp}{\uparrow}		% spin up
\newcommand{\sDn}{\downarrow}	% spin down

% Mathematics: Miscellaneous
\newcommand{\const}{\operatorname{const}}	% constant

\newcommand{\<}{\left\langle}	% angular brackets
\renewcommand{\>}{\right\rangle}	% "

% Constants
\newcommand{\e}{\mathrm{e}}		% Euler number
\newcommand{\I}{\mathrm{i}}		% imaginary unit
\newcommand{\kB}{k_{\mathrm B}} % Boltzmann constant

% Abbreviations for this document

\newcommand{\G}{\Lambda}
\newcommand{\At}{\langle A\rangle_{\! t}}
%\newcommand{\At}{\langle A\rangle_{\!\rho(t)}}
%\newcommand{\Ao}{\langle A\rangle_{\!\rho(0)}}

\newcommand{\dof}{\kappa}
\newcommand{\gap}{\gamma}
\newcommand{\Att}{{\cal A}_{t}}
%\newcommand{\Att}{{\cal A} (t)}
\newcommand{\TT}{{\cal T}}
\newcommand{\pmax}{p_{\rm max}}
\newcommand{\rhomax}{\rho_{\rm max}^0}
\newcommand{\amax}{a_{\rm max}}
\newcommand{\amin}{a_{\rm min}}
\newcommand{\Da}{\Delta_{\!A}}

\newcommand{\lvsp}{\varepsilon}
\newcommand{\De}{\Delta}
\newcommand{\Ath}{\langle A\rangle_{\!\!\;\rm{mc}}}
\newcommand{\varV}{\sigma^2}
\newcommand{\cmagV}{\bar\sigma}
\newcommand{\cvarV}{\cmagV^2}
\newcommand{\bwV}{\Delta_v}

\newcommand{\esh}{\Delta\mathcal{E}}
\newcommand{\E}{\mathcal{E}}
\newcommand{\ptbw}{\Delta V}
\newcommand{\ptst}{\lambda}
\newcommand{\rateExp}{\Gamma}
\newcommand{\rateBessel}{\gamma}

\newcommand{\rhoDia}{\rho_\infty}
\newcommand{\rhomc}{\rho_{\mathrm{mc}}}
\newcommand{\pmc}{p_{\mathrm{mc}}}
\newcommand{\rhoeq}{\rho_{\mathrm{eq}}}
\newcommand{\rhoForward}{\rho_{\mathrm{f}}}
\newcommand{\rhoScrambling}{\rho_{\mathrm{s}}}
\newcommand{\rhoBackward}{\rho_{\mathrm{b}}}
\newcommand{\rhoPert}{\rho^\prime}
\newcommand{\rhoTarget}{\rho_{\mathrm{T}}}
\newcommand{\rhoTargetPert}{\rho^\prime_{\mathrm{T}}}
\newcommand{\rhoRev}{\rho_{\mathrm{R}}}
\newcommand{\rhoRevPert}{\rho^\prime_{\mathrm{R}}}

\newcommand{\Aeq}{A_{\mathrm{eq}}}
\newcommand{\Amc}{A_{\mathrm{mc}}}

\newcommand{\dosH}{D_H}
\newcommand{\dosV}{D_V}
\newcommand{\dosVft}{\hat{D}_V}
\newcommand{\ldos}{u}
\newcommand{\ldosTilde}{q}
\newcommand{\deff}{d_{\mathrm{eff}}}
\newcommand{\bos}{\mathrm{B}}
\newcommand{\fer}{\mathrm{F}}
\newcommand{\Ei}{\mathrm{Ei}}
\newcommand{\dA}{\mathcal{A}}
\newcommand{\dAp}{{\mathcal{A}}^\prime}

\newcommand{\KK}{\mathcal{K}}
\newcommand{\LL}{\mathcal{L}}

\newcommand{\Imc}{I_{\mathrm{mc}}}
\newcommand{\pr}{{\rm{Prob}}}

\newcommand{\id}{\mathbb 1}
\newcommand{\RR}{{\mathbb R}}
\newcommand{\CC}{{\mathbb C}}
\newcommand{\NN}{{\mathbb N}}
\newcommand{\ZZ}{{\mathbb Z}}

\newcommand{\tr}{\mbox{Tr}}
\newcommand{\hr}{{\cal H}}
\newcommand{\ord}{{\cal O}}
\newcommand{\II}{{\cal I}}
\newcommand{\mic}{\mathrm{mc}}
\newcommand{\rhomic}{\rho^0_{\mathrm{mc}}}
\newcommand{\tto}{\rightsquigarrow}
\providecommand{\norm}[1]{\|#1\|}
\newcommand{\da}{\Delta_{\! A}}
\newcommand{\dda}{\delta \! A}

\providecommand{\av}[1]{[#1]_V}

%\providecommand{\avv}[1]{\left[#1\right]_{\!V}}
\providecommand{\avx}[1]{E[#1]}
%\providecommand{\avvx}[1]{E\left[#1\right]}

%\providecommand{\av}[1]{\mathbb{E}[#1]}
%\providecommand{\avv}[1]{\mathbb{E}\!\left[#1\right]}
%\providecommand{\avu}[1]{\mathbb{E}[#1]}
%\providecommand{\avvu}[1]{\mathbb{E}\!\left[#1\right]}

%%% LALC
\renewcommand{\d}{\mathrm{d}} % differential
\newcommand{\trans}{\mathsf{T}}     % transpose
\newcommand{\lmat}{\left( \begin{matrix}}	% matrix begin
\newcommand{\rmat}{\end{matrix} \right)}	% matrix end
\newcommand{\str}{\mathrm{str}} % super trace
\newcommand{\sdet}{\mathrm{sdet}} % super determinant
%%%
\newcommand{\figref}[1]{Fig.~\protect\ref{#1}}
\newcommand{\fmref}[1]{(\protect\ref{#1})}
\newcommand{\xref}[1]{\protect\ref{#1}}
%%%
\newcommand{\LD}[1]{\textcolor{blue}{#1}}
\newcommand{\PR}[1]{\textcolor{red}{#1}}
\newcommand{\TODO}[1]{\textcolor{red}{[#1]}}
\newcommand{\TMP}[1]{\textcolor{green}{#1}}
\newcommand{\JS}[1]{\textcolor{magenta}{#1}}



\begin{document}

\title{Non-equilibration, synchronization, and time crystals in isotropic Heisenberg models}
%\title{Far-from-equilibrium properties of the isotropic Heisenberg model}
%\title{Far-from-equilibrium behavior of the Heisenberg model}

\author{Peter Reimann}
\author{Patrick Vorndamme}
\author{J\"urgen Schnack}
\affiliation{Faculty of Physics, 
Bielefeld University, 
33615 Bielefeld, Germany}
\date{\today}

\begin{abstract}
%We consider isotropic, but otherwise largely arbitrary %many-body 
%Heisenberg models in the presence of a homogeneous magnetic field.
%In particular, integrable, non-integrable, as well as disordered examples are covered.
Isotropic, but otherwise largely arbitrary 
%many-body 
Heisenberg models in the presence of a homogeneous magnetic field
are considered, including various integrable, non-integrable, as 
well as disordered examples, and not necessarily restricted to one 
dimension 
%and
or
short-range interactions.
%with short- or long-range interactions. 
%in any dimension.
Taking for granted that the non-equilibrium initial condition 
and the spectrum of the field-free model satisfy some very weak requirements,
expectation values of generic observables are analytically shown to
exhibit permanent long-time oscillations, thus ruling out equilibration.
%Moreover, 
If the model (but not necessarily the initial condition) is translationally invariant, 
%those everlasting 
the long-time
oscillations are moreover shown to 
%synchronize in the sense of being invariant under translations of the observable. 
exhibit synchronization in the long run, meaning that they are invariant under arbitrary translations of the observable.
%Finally, 
Analogous long-time oscillations are also recovered for 
%generic 
temporal correlation functions when the system is 
already at thermal
equilibrium from the outset, thus realizing a so-called time crystal.
\end{abstract}


\maketitle

%%%%%%%%%%%%%%%%%%%%%%%%%%%%%%%%%%%%%%%%%%%%%%%%%%%%%%%
\section{Introduction}
\label{s1}

A macroscopic system without 
%permanent 
external perturbations
%or %additional 
%heat baths
%tend towards 
approaches
a steady equilibrium 
state 
%behavior
after sufficiently long times,
%after a sufficiently long time evolution
no matter how far from equilibrium it started out. 
On the phenomenological level, this is an extremely 
well-established fact both in everyday life and under 
controlled laboratory conditions.
More precisely speaking, in every single run of an experiment,
one may still encounter certain statistical or quantum mechanical 
fluctuations, especially for microscopic observables, but 
on the average over many repetitions of the experiment, 
%
the expectation value will
%are generally 
%expected and observed to 
closely approach some constant equilibrium value 
%after sufficiently long times.
in the long run.
%to equilibrate towards some constant long-time value.
On the other hand, a satisfactory theoretical understanding 
of these 
%issues 
empirical observations
in terms of the underlying fundamental 
laws of quantum mechanics still remains a challenging 
open question, both qualitatively and quantitatively,
to which a considerable amount of experimental, numerical,
and analytical efforts have been devoted in recent years 
\cite{mor18,dal16,gog16,lan16,ued20,nan15}.

Obviously, a particularly fascinating 
%direction 
endeavor in this context is 
to identify cases which 
%exhibit 
%entail
give rise to
certain deviations from the 
above-mentioned standard 
%behavior. 
scenario.
For instance,
it has been discovered that models exhibiting integrability or 
many-body localization may fail to thermalize
%so-called integrable models and %disordered systems %exhibiting%with 
%many-body localization  %are known to defy
%have been discovered not to always exhibit 
%thermalization 
%in the long run
\cite{mor18,dal16,gog16,lan16,ued20,nan15},
in contradiction to what equilibrium (textbook) 
statistical mechanics predicts.
%for suitably chosen non-equilibrium initial conditions,
%in disagreement with the predictions 
%of equilibrium (texbook) statistical mechanics.
Nevertheless, %all these examples 
generically they
still exhibit equilibration in the sense
\cite{rei08,lin09,sho11,sho12,rei12,bal16}
that the time-dependent expectation values 
stay extremely close to a constant value for the vast majority
of all sufficiently late times, i.e., apart from
%the usual ``trivial'' exceptions of 
the transient relaxation processes during some initial 
time-interval, and apart from the well-known, exceedingly 
rare but unavoidable quantum revival effects.
%and certain fine-tuned (unrealistic) initial conditions.

At the focus of our present work are many-body
systems 
whose expectation values
%which 
do not even equilibrate in the above sense, but rather exhibit
permanent 
%long-time oscillations of observable quantities.
long-time oscillations.
%ad infinitum.
Leaving aside trivial cases like non-interacting (separable)
models or perfect harmonic oscillators, 
related previously proposed examples 
%of this type 
that may come to one's mind are the  ``quantum Newton's cradle'' 
experiment by Kinoshita, Wenger, and Weiss \cite{kin06},
the exploration of Rydberg-atom quantum simulators 
by Bernien et al. \cite{ber18}, or
the numerical study by Banuls, Cirac, and Hastings in 
Ref.~\cite{ban11}.
However, it was later discovered that in fact
all those examples
ultimately still must exhibit equilibration when monitoring
the dynamical evolution over sufficiently long times
\cite{li20,mbs,kim15,lin17,far17}.
On the other hand, analytically provable absence of equilibration 
in the context of many-body quantum scars
has been recently 
%demonstrated 
established
for various abstract models
in combination with special initial conditions 
\cite{mbs}, yet their significance with regard to real-world 
systems still remains to be explored.

In our present work, we 
%scrutinize the quantum dynamics of 
focus on
one of the simplest and best-established
many-body quantum systems, 
namely the isotropic Heisenberg model
with a homogeneous magnetic field.
%which is exposed to a spatially homogeneous magnetic field.
Besides the original and most common version of the model, 
also various generalizations and modifications 
%of the model 
will be covered, including non-intergable systems 
(e.g., in more than one dimension) 
and disorder in the form of randomized interactions.
The only indispensable prerequisites are that the field-free model 
%must exhibit a SU(2) symmetry
must be SU(2) symmetric 
(isotropic),
the external field must be spatially homogeneous,
%and for the rest 
and
%, apart from the %concomitant 
%hence induced special features,
%the model must exhibit a generic energy spectrum.
the energy levels must
%exhibit some rather weak and generic features.}
satisfy some rather weak and generic assumptions.
%that, apart from this symmetry, the energy spectrum 
%does not exhibit a certain non-generic special feature.

Our first main objective is to analytically demonstrate and 
numerically illustrate the typical 
%emergence 
occurrence
of 
%everlasting oscillations (and thus the absence of equilibration)
non-equilibration in the form of everlasting oscillations 
in such systems.
In particular,
%we will show that
this behavior is neither restricted to 
special initial conditions nor to integrable models.

Furthermore, 
%those oscillations are shown to
we analytically show that those oscillations entail synchronization 
under the additional condition that the model -- but not 
necessarily the initial condition -- is translationally invariant.
%thus complementing and substantially extending previous
%related investigations, e.g., in Refs.~\cite{med20,vor21,buc22}.}

Turning to systems at thermal equilibrium, we finally establish
the generic 
%emergence 
occurrence
of analogous long-time oscillations for 
dynamic (time-dependent) correlation functions, and we discuss their implications 
with respect to the topic of time crystals
\cite{han22,ven19,med20,wat15,wat20,hua19}.

In terms of these main findings, but also methodologically,
our present paper is closely related in a variety of
different respects to a considerable number of previous
works, 
including Refs.~\cite{med20,vor21,rei08,lin09,sho11,sho12,rei12,bal16,alh20,wat15,wat20,hua19}.
Since an adequate comparison is only possible
on the basis of a minimal amount of formal definitions,
such a more detailed discussion of pertinent previous 
works will be provided at various places throughout the 
paper.

%\LD{Fehlt noch: relevante Referenzen.}


%%%%%%%%%%%%%%%%%%%%%%%%%%%%%%%%%%%%%%%%%%%%%%%%%%%%%%%
\section{General framework}
\label{s2}

We consider a Heisenberg model on an arbitrary 
(not necessarily one-dimensional) lattice, 
whose sites are labeled by $i$. We denote by $\G $ 
the set of all possible lattice sites, and by $\dof$ their 
total number.
(Alternatively, $\dof$ may thus be viewed as the
system size or as the number of degrees of freedom.)
%Indicating the corresponding single-site spin operators by
The single-site spin operators are indicated by
vectors $\vec s_i$ with three components 
$s_i^{a}$, $a\in\{x,y,z\}$, while the single-site
spin quantum number is given by the same
integer or half-integer $s$ on every site.
%[Generalizations to site-dependent spin quantum numbers 
%are immediate.]}

Denoting the components of the total spin by 
\begin{eqnarray}
S^a:=\sum_{i\in \G }  s_i^a
\ ,
\label{1}
\end{eqnarray}
the considered Hamiltonians must be of the general form
\begin{eqnarray}
H & := & H_0 + h\, S^z \, ,
\label{2}
\\
H_0 & := & \sum_{i,j\in \G } J_{ij} \, \vec s_i\cdot\vec s_j \ ,
\label{3}
\end{eqnarray}
where the magnetic field $h$ and the coupling constants 
$J_{ij}$ are, for the time being, still largely arbitrary model parameters.

%Indicating the operator norm (largest eigenvalue in modulus) 
%of a self-adjoint operator $A$ 
%by $\norm{A}$, it is (as usual) assumed that 
%$\norm{s_i^a}$ is an integer or half-integer, which is
%independent of $i$ and $a$, and which is named the
%single-site spin quantum number $s$.

By means of well-established standard arguments 
one finds that $H_0$ commutes with $S^a$
for all $a\in\{x,y,z\}$  (SU(2) symmetry).
As a consequence, the eigenvectors of $H_0$ can be chosen 
so that they are simultaneously eigenvectors of $S^z$
as well as of  $\vec S^2:=(S^x)^2+(S^y)^2+(S^z)^2$,
and thus can be written as $|n,l\rangle$ 
with the properties
\begin{eqnarray}
H_0 |n, l\rangle & = & E_n^0 \, |n, l\rangle
\ ,
\label{4}
\\
S^z |n,l\rangle & = & l \, |n,l\rangle
\ ,
\label{5}
\\
\vec S^2 |n,l\rangle & = & L_n(L_n+1)\, |n,l\rangle
\ .
\label{6}
\end{eqnarray}
%where $n\in\{1,...,N\}$, $l \in \{-L_n,...,L_n\}$,
%and where the $L_n$ are positive integers or half-integers.
Here, the indices $n\in\{1,...,N\}$ label the energy eigenvalues,
the $l \in \{-L_n,...,L_n\}$ are the total magnetic quantum 
numbers, 
while the $L_n$ are 
positive integers or half-integers, often denoted as 
total spin quantum numbers.
%whose values may be positive integers or half-integers.
In other words, for any given $n$, the energies
$E_n^0$ are $(2L_n\!+\!1)$-fold degenerate 
with
%total spin quantum numbers $L_n$ and
spin multiplets $\{|n,l\rangle\}_{l=-L_n}^{L_n}$.
Traditionally, those simultaneous eigenvectors of 
$H_0$, $\vec S^2$, and $S^z$ are often 
denoted as $|n,L_n,l\rangle$, but since the
$L_n$'s are unique functions of the $n$'s,
we employ the shorter notation $|n,l\rangle$.
One readily verifies that $0\leq L_n \leq \dof s$, 
and one can evaluate how many 
%states 
eigenvectors
belong to a certain $l$ or $L_n$ \cite{BSS:JMMM00}, 
but for the rest, the actual quantitative
%values of all the $L_n$'s belonging to a level $n$ are 
value of $L_n$ belonging to any given 
%energy $E_n^0$ 
$n$ (or $E_n^0$) is
in general quite difficult to tell (see also Appendix \ref{app1}).
We finally remark that the energies $E_n^0$ are 
generically expected to be pairwise different, 
but that this property is not
actually required in most of our subsequent
explorations.
%[Ist das alles so korrekt und die Namen und Notation hinreichend g\"angig?]

Exploiting (\ref{2}), (\ref{4}), (\ref{5}) it follows that
\begin{eqnarray}
H\, |n,l\rangle & = & E_{nl} \, |n,l\rangle
\ ,
\label{7}
\\
E_{nl} & := & E_n^0+l\, h
\ .
\label{8}
\end{eqnarray}
The eigenvectors $|n,l\rangle$ are thus independent
of $h$, while the above-mentioned degeneracies of 
the eigenvalues for $h=0$ are expected to be 
generically lifted for $h\not =0$
(Zeeman splitting).

Given any pure or mixed initial state $\rho(0)$,
its time evolution is governed by the 
%Schr\"odinger or 
von Neumann equation, resulting at time $t$ in the state
$\rho(t)=e^{-iHt}\rho(0)e^{iHt}$ ($\hbar=1$).
Accordingly, the expectation value of any observable 
(Hermitian operator) $A$ at time $t$ is given by
\begin{eqnarray}
\At:=\tr\{\rho(t) A\} \ .
\label{8a}
\end{eqnarray}
By employing the eigenvalues and eigenvectors of $H$ 
from (\ref{7}) and (\ref{8}) it 
follows that
\begin{eqnarray}
\At = \sum_{mnkl} \rho_{mn}^{k,l} A_{nm}^{l,k}\, e^{i(E_n^0-E_m^0+[l-k]h)t}
\ ,
\label{9}
\end{eqnarray}
where the sum is tacitly restricted to indices $m,n,k,l$ within their 
admitted range as specified below (\ref{6}), and
where the matrix elements $\rho_{mn}^{k,l}$ and $A_{nm}^{l,k}$ are defined as
\begin{eqnarray}
\rho_{mn}^{k,l} & := & \langle m,k|\rho(0)|n,l\rangle
\ , 
\label{10}
\\
A_{nm}^{l,k} & := & \langle n,l | A|m,k\rangle
\  .
\label{11}
\end{eqnarray}
%for indices $m,n\in\{1,...,N\}$, $k\in\{-L_m,...,L_m\}$, $l\in\{-L_n,...,L_n\}$ (see below (\ref{6})), and as zero otherwise.
%In doing so, the sum over the indices $m,n,k,l$ in (\ref{9}) is tacitly restricted
%to pairs $n,l$ for which $|n,l\rangle$ are well-defined eigenvectors in (\ref{7}),
%i.e. $n\in\{1,...,N\}$ and $l\in\{-L_n,...,L_n\}$ (see below (\ref{5})),
%and likewise for the pairs $m,k$.
%
%Alternatively, for indices $n,l$ so that $|n,l\rangle$ is not a well-defined 
%eigenvector, we may define those (so far undefined) vectors $|n,l\rangle$
%as being equal to the null vector (hence $\rho_{mn}^{k,l}=0$, $A_{nm}^{l,k}=0$).
%As a consequence, we may now consider all four indices
%$m,n,k,l$ in the sum in (\ref{9}) to run over all integer values
%(but without ever entailing any problem of convergence, 
%since the number of non-vanishing summands always remains finite).
%Since this convention turns out to be more convenient, it is
%henceforth taken for granted.
%
%In particular, we now can replace the 
%summation index $l$ in (\ref{9}) by $\nu:=l-k$, yielding
Going over from the summation index $l$ in (\ref{9}) to $\nu:=l-k$
then yields
\begin{eqnarray}
\At & = & \sum_\nu f_{\nu}(t) \, e^{i\nu h t}
\ ,
\label{12}
\\
f_{\nu}(t) & := & \sum_{mn} e^{i(E_n^0-E_m^0)t}\sum_k  \rho_{mn}^{k,k+\nu} A_{nm}^{k+\nu,k}
\ .
\label{13}
\end{eqnarray}

One readily verifies that $f_{-\nu}(t)=f^\ast_\nu(t)$, hence (\ref{12}) 
could also be rewritten as a purely real Fourier series.
%(see also Eq.~(\ref{25}) below).
Since the eigenvectors $|n,l\rangle$ in (\ref{4})
and thus in (\ref{7}) are independent of $h$, the same property is inherited by the
matrix elements in (\ref{10}) and (\ref{11}), and finally by the functions 
$f_\nu(t)$ in (\ref{13}).
In other words, the only $h$-dependence in (\ref{12}) arises via the exponential 
factors on the right-hand side.

%%%%%%%%%%%%%%%%%%%%%%%%%%%%%%%%%%%%%%%%%%%%%
%
\subsection{Model classification}
\label{s21}

The general structure in (\ref{1})-(\ref{3}) 
still covers a wide variety of models in one or more 
dimensions, whose interactions may 
be of short- or long-range character, 
and may even exhibit various kinds of disorder
(quenched randomness) with
%as it is commonly considered 
concomitant many-body localization effects \cite{nan15}.
Moreover, also our assumption that all lattice sites exhibit the same 
spin quantum number $s$  (see above Eq.~(\ref{1})) 
can be readily relaxed.

We emphasize that these models (\ref{1})-(\ref{3})
%, in particular,
include many examples which are commonly 
considered 
%either 
as being either integrable or 
%as being 
non-integrable,
even though the precise meaning of ``integrability'' 
is still not entirely clear \cite{dal16,gog16}.
%\JS{Vorschlag=Weglassen: 
%For instance, a variety of one-dimensional cases with 
%nearest-neighbor interactions are integrable in the sense of 
%being solvable by a Bethe ansatz, while their
%counterparts in higher dimensions or with 
%next-nearest-neighbor interactions are classified as
%non-integrable in terms of their energy level statistics
% [\PR{J\"urgen: ``habe Refs.''}].}
%
%\LD{[Ich tendiere jetzt dazu, alles zum Thema Integrabilit\"at
%(inkl. relevante Referenzen) hier zu sammeln, und sp\"ater
%nicht mehr genauer darauf einzugehen. Bitte gegebenenfalls erg\"anzen.]}
%
%As mentioned below (\ref{3}) the notion of integrability 
%is still to some extent a matter of debate.
%However, in our present case it 
%\PR{Apart from that,
%Even without pursuing this controversial issue any further,
Independently of such still unsettled subtleties,
for our present purposes it seems reasonable to require
that whether a given model in (\ref{1})-(\ref{3}) is considered 
as (non-)integrable
%or not, 
should {\em not} depend on the value of 
the external field $h$.
%, since also 
The reason is that since the eigenvectors in (\ref{7}) 
are independent of $h$, and the dependence of 
the eigenvalues in (\ref{8}) on $h$ is rather trivial,
%Hence 
it would not be satisfying if a transition from integrable 
to non-integrable would be achievable by simply changing the 
value of $h$.



%%%%%%%%%%%%%%%%%%%%%%%%%%%%%%%%%%%%%%%%%%%%%
%
%\section{Long-time behavior}
\section{Main results}
\label{s3}

%In the following, 
%In this section,
%our main objective is to show that
Our first main result consists in the prediction that,
for sufficiently large systems,
the expectation values in (\ref{12}) 
%stay very close to
can be approximated very well by
\begin{eqnarray}
\Att :=\sum_{\nu} \bar f_{\nu} \, e^{i\nu ht}
\label{14}
\end{eqnarray}
for the vast majority of all sufficiently late times $t$,
%and provided the system size $\dof$ is sufficiently large,
where $\bar f_{\nu}$ essentially amounts to the long-time 
average of $f_\nu(t)$ from (\ref{13}).
More precisely speaking,
\begin{eqnarray}
%\bar f_{\nu}:= \sum_{nk} \rho_{nn}^{k,k+\nu} A_{nn}^{k+\nu,k}
\bar f_{\nu}:= {\sum_{mnk}}' \rho_{mn}^{k,k+\nu} A_{nm}^{k+\nu,k}
\ ,
\label{15}
\end{eqnarray}
where the prime symbol indicates that the summation is
restricted to indices $m$ and $n$ with the property $E_m^0=E_n^0$.
In the generic case that all energies $E_n^0$ are pairwise different
(see below Eq.~(\ref{6})), this boils down to
\begin{eqnarray}
\bar f_{\nu} = \sum_{nk} \rho_{nn}^{k,k+\nu} A_{nn}^{k+\nu,k}
\ .
\label{16}
\end{eqnarray}
More generally, the same simplification (\ref{16}) of (\ref{15})
also applies to cases where either $\rho_{mn}^{k,l}$ or 
$A_{nm}^{l,k}$ vanishes whenever $m\not=n$ and 
$E_m^0=E_n^0$ (degeneracies).
We also recall that similar restrictions as below (\ref{9}) are 
understood to apply to the sums in (\ref{15}) and (\ref{16}).

Before providing the quantitative analytical details of the above prediction,
%Before specifying more precisely the meaning of ``very close'',
%``the vast majority of all sufficiently large times $t$'',
%and the preconditions, under which such a result will
%then be derived,
let us briefly motivate it
by means of a somewhat oversimplified 
heuristic argument.

Indicating the average over all times $t\geq 0$ 
by $\langle \,\cdot\,\rangle_{\!\infty}$, 
%and assuming for the sake of simplicity
%that $E_m^0\not=E_n^0$ whenever 
%$m\not=n$ (non-degeneracy)
we can conclude that
$\langle e^{i(E_n^0-E_m^0)t} \rangle_{\!\infty}$ 
equals unity if $E_m^0=E_n^0$ and zero otherwise.
%= \delta_{mn}$ (Kronecker delta).
Together with (\ref{13}) and (\ref{15}) it follows that
$\langle f_{\nu}(t)\rangle_{\!\infty} = \bar f_\nu$.
Moreover, for sufficiently large systems, the number of summands 
on the right-hand side of (\ref{13}) may be expected to become
very large. (In view of the restrictions mentioned below Eq.~(\ref{9}),
this is in fact not really obvious, and sometimes actually wrong).
The key point now consists in the heuristic conjecture 
that this large number of summands in (\ref{13})
entails some kind of 
``dephasing effect'', at least for generic time points $t$
(after initial transients have died out, and apart from the well-known,
very rare but unavoidable quantum revivals),
with the result that all the summands with 
%$m\not=n$
$E_m^0\not=E_n^0$
effectively cancel each other 
in sufficiently good approximation.
As a consequence,
every $f_\nu(t)$ in (\ref{13}) is conjectured to stay near its time 
average (\ref{15}), and  hence the expectation values (\ref{12})
to stay near $\Att$ from (\ref{14}).
%[Irgendwie hatte ich den Eindruck, dass diese Voraussage
%f\"ur euch ziemlich offensichtlich ist. Warum eigentlich?]

Next we turn to a more rigorous foundation
of this oversimplified argument.
%, also including its generalization to cases with degenerate energies $E_n^0$.
In doing so, we proceed in three steps.
First, the two most important quantities appearing 
in our main analytical result are introduced.
Next, the analytical result itself is presented
and discussed.
Finally, the actual derivation of the result is
provided in Appendix \ref{app1}.

For an instructive numerical illustration of those
general predictions, we refer to Sec.~\ref{s35}.


%%%%%%%%%%%%%%%%%%%%%%%%%%%%%%%%%%%%%%%%%%%%%
%
\vspace*{0.8cm}
\subsection{Level populations and energy gaps}
\label{s31}

According to the first remark below Eq.~(\ref{8}), 
the quantity $\langle n,l |\rho(0)| n,l\rangle$
is independent of the magnetic field $h$.
%in (\ref{3}).
Moreover, it can be identified with the population of the energy 
eigenstate $|n,l\rangle$ by the initial state $\rho(0)$.
%Accordingly, $\pmax$ is the largest among all those 
%populations and must be 
Likewise,
\begin{eqnarray}
\pmax & :=& \max_{n,l} \langle n,l |\rho(0)| n,l\rangle
\label{17}
\end{eqnarray}
thus amounts to  the {\em maximal level population}
and is $h$-independent.

Next we focus on an arbitrary but fixed pair of indices 
$(m,n)$ with 
%non-degenerate energies,i.e., 
the property $E_m^0\not =E_n^0$,
and we count all possible index pairs
$(m',n')$ whose energy gaps $E_{m'}^0-E_{n'}^0$
are equal to the given reference gap $E_m^0-E_n^0$.
The number of those pairs $(m',n')$
is denoted as
$\gap_{mn}$.
For obvious reasons, this number $\gap_{mn}$ is called
the degeneracy of the energy gap $E_m^0-E_n^0$,
and it has the properties that $\gap_{mn}\geq 1$ and
$\gap_{nm}=\gap_{mn}$.
Specifically, if $\gap_{mn}=1$ then $E_m^0-E_n^0$
is called a non-degenerate energy gap.
%In the remaining cases that $E_m$ and $E_n$
%are degenerate (vanishing gaps)
%we adopt the definition $g_{mn}:=1$.
Finally, the {\em maximal energy gap degeneracy}
is defined as
\begin{eqnarray}
\gap:=\max_{m,n} \gap_{mn}
\ ,
\label{18}
\end{eqnarray}
where the maximum is taken over all pairs
$(m,n)$ with non-vanishing energy gaps $E_{m}^0-E_{n}^0$ 
\cite{sho12}.

We close with two side remarks: 
(i) %Note that 
Since the unperturbed energies $E_n^0$ are obviously
independent of the magnetic field $h$, 
%hence 
the same applies to $\gap$ in (\ref{18}).
(ii) As already mentioned below (\ref{6}), we do not require 
that all $E_n^0$ are pairwise different,
%It should also be emphasized that the word ``degeneracy''
%here refers to the property $E_m^0=E_n^0$,
%{\em not} to the $(2L_n\!+\!1)$-fold  degeneracy of
%every single $E_n^0$, see also discussion below Eq.~(\ref{6}).
%In particular, 
%%the $E_n^0$'s are admitted to exhibit such degeneracies,
%we do not require that all $E_n^0$ are pairwise different,
%, and that 
%this issue in fact plays no role with respect to the 
%definitions (\ref{17}) and (\ref{18}).
%However, in case of degeneracies this will have 
%the following indirect consequences with 
with the following implication 
with regard to $\gap$:
Denoting for any given $n$ the number
of indices $k$ with the property
$E_k^0=E_n^0$ by $\mu(n)$ (``multiplicity of $E_n^0$'')
it readily follows that $\gap_{mn}\geq \mu(m)\mu(n)$,
and hence that $\gap\geq \mu_{\max}^2$,
where $\mu_{\max}:=\max_n\mu(n)$ is 
the maximal number of pairwise identical
%degeneracy within the given set of 
energies $E_n^0$.
On the other hand, even if all $E_n^0$ are pairwise different
and thus $\mu_{\max}=1$, it is still possible that $\gap>1$.




%%%%%%%%%%%%%%%%%%%%%%%%%%%%%%%%%%%%%%%%%%%%%
%
\subsection{Main analytical prediction}
\label{s32}

Employing the definitions (\ref{17}) and (\ref{18}),
and indicating the temporal average over an interval $[0,T]$ 
by the symbol $\left\langle \,\cdot\,\right\rangle_{T}$, 
%and defining the temporal variance
%\begin{eqnarray}
%\sigma^2:= \overline{[\At - \Att ]^2}
%\label{19}
%%19
%\end{eqnarray}
%it is shown that the above condition implies
it is shown in Appendix \ref{app1} that the
% temporal 
mean square deviation of the ``true'' expectation 
values (\ref{12}) from the auxiliary function (\ref{14})
obeys for all sufficiently large $T$ the inequality
\begin{eqnarray}
\left\langle [\At - \Att ]^2 \right\rangle_{T} 
& \leq & \gap\, (2s \dof\!+\!1)^2\, \Da^2\, \pmax
\ ,
\label{19}
\end{eqnarray}
where $s$ is the single-spin quantum number and $\dof$ the system 
size 
%(number of spins) from Sec.~\ref{s2}.
(see above Eq.~(\ref{1})).
Furthermore, $\Da$ is the measurement range of the observable $A$,
i.e., the difference between the largest and smallest possible 
measurement outcomes (eigenvalues of $A$).

Our first remark is that the right-hand side of (\ref{19}) is 
independent of the magnetic field $h$ in (\ref{2}).

Our second remark is that for systems with many degrees of 
freedom $\dof$, the number $N$ of energies $E_n^0$ grows 
exponentially with $\dof$ 
(see also Eq.~(\ref{a3})),
%according to (\ref{a3}),
while their range (difference between the largest and smallest energies)
is generically expected to grow subexponentially (usually linearly) with $\dof$.
Hence, the level density will become unimaginably large for truly
macroscopic systems, and it will be virtually impossible to
notably populate  only a small number of 
%those levels $E_n^0$ energy 
eigenstates $|n,l\rangle$
in a real experiment.
Rather, one expects that the number of  non-negligibly 
populated 
levels 
%eigenstates
will still be exponentially large in $\dof$. 
Recalling Eq.~(\ref{17}), 
%and the discussion of its physical meaning, and since 
and that the sum of all level 
%eigenstate 
populations must be unity, 
%we thus expect 
one thus expects \cite{rei08,rei12,bal16}
that a very rough order of magnitude estimate of the form
\begin{eqnarray}
\pmax \approx \exp\{{-\ord(\dof)}\}
\label{20}
\end{eqnarray}
will be generically fulfilled under all experimentally 
realistic circumstances.
A more detailed, explicit example will be worked out in Sec.~\ref{s33}.

Our third remark is that, obviously, no significant conclusion 
about the expectation values in (\ref{9}) can be drawn 
without any knowledge whatsoever regarding the energies 
$E_n^0$ appearing on the right-hand side.
On the other hand, these energies are in general not explicitly 
known in sufficient quantitative detail.
%[In the special case of 
%a one-dimensional spin-1/2 Heisenberg model,
%the energies are in principle buried in the analytic results 
%of the Bethe ansatz, but in practice this is 
%%often 
%of  little use 
%to draw further conclusions. 
%for our present purposes.]
[An exception is given by models that are analytically solvable by 
means of the Bethe ansatz, but in practice this is of little use
for our present purposes.]
%More general versions of the model are no longer analytically solvable.]
For instance, already one of the simplest and most important 
%spectral properties, 
features of the energies $E_n^0$,
namely the so-called level statistics (probability 
distribution of the distances between neighboring energy 
levels), is not analytically available for practically any
quantum many-body system of physical interest, 
including our present Heisenberg 
models of the general form (\ref{2}).
%Nevertheless, 
However,
it is commonly taken for granted (on the basis
of heuristic arguments and ample numerical evidence)
that the level statistics tends to some well-defined 
and reasonably smooth asymptotics in the thermodynamic limit.
[Moreover, this asymptotics is often expected to be 
close to, for instance, a Wigner-Dyson or a Poisson
distribution, but such ``details''  do not matter here.]

Our present assumption regarding the 
%spectrum of $H$
energies $E_n^0$ is 
%in fact 
in essence
quite similar in spirit 
%but not exactly identical 
to these
%the above mentioned 
common assumptions regarding the level statistics.
Namely, we assume that the maximal energy gap
degeneracy in (\ref{18}) grows at most subexponentially 
with the system size $\dof$.
Indeed, this is 
%essentially tantamount 
closely related to requiring that
the level statistics does not develop delta-peaks
in the thermodynamic limit.
In particular, this also means that the maximal 
%degeneracy of the energies 
number of pairwise identical energies
$E_n^0$ must grow at most 
subexponentially with $\dof$, see 
remark (ii) at the end of Sec.~\ref{s31}.

%From NJoP2019/FGAntrag2017/see also Farrelly report
%For similar reasons, not too many of the ``energy gaps'' $E_n-E_m$ in (\ref{10}) may coincide, or if they coincide,
%they must contribute with sufficiently small weights. In view of the usually very dense and irregular energy spectra, 
% he above (or some equivalent) requirements are commonly taken for granted under all experimentally relevant conditions.
%%[R1-R4],\cite{lin09,lin10,sho11,sho12,mal14}.
%%\cite{lin09,lin10,sho11,sho12,mal14,laz14,het15,kie17}.
%In particular, the differences between neighboring energies $E_n$ are expected to be distributed 
%under quite generic conditions according to some well-defined probability density, for instance a so-called Wigner-Dyson 
%or a Poisson statistics \cite{bro81}, which is furthermore bounded from above.
%If so, the probability 
%%that two energies $E_n$ are strictly equal, or 
%that two energy differences $E_n-E_m$ (with $n\not=m$) 
%happen to exactly coincide, is zero.
%%(For instance, a non-negligible fraction 
%%of degenerate energies would imply that 
%%the above mentioned probability density 
%%must exhibit a delta peak at zero.)

Though not rigorously provable, it is intuitively evident that this assumption
will be generically fulfilled in the sense that 
%when considering some suitably defined set of models,
% then the assumption will be violated at most by a subset of relative measure zero.
the set of model parameters $J_{ij}$ 
%and $h$ 
in (\ref{3}) which violates the 
assumption is of negligible measure compared to the set of all 
the {\em a priori} possible (and physically sensible) 
values of those parameters.
Since there is usually no {\em a priori} reason why some specific 
model of actual interest must belong to this exceptional subset, 
it is physically reasonable to take for granted that the model
indeed is a member of the overwhelming majority.

Finally, it is also noteworthy that our above assumptions 
regarding $\pmax$ and $\gap$ are by now very well-established 
in the context of equilibration and thermalization in many-body 
quantum systems, and that there 
%is 
exists
essentially no rigorous 
analytical result in this context which is valid without taking for granted 
the same or some very similar assumptions
\cite{mor18,gog16,rei08,lin09,sho11,sho12,rei12,bal16,far17,tas98,sre99,mul15,imb16,gal18,wil19}.

%In fact, we will work out two slightly different predictions,
%based on two slightly different assumptions about the 
%%spectrum of $H$.
%energies $E_n^0$.

Altogether, we thus can and will take our above assumptions
regarding $\pmax$ and $\gap$ for granted.
For large $\dof$, the small factor $\pmax$ in (\ref{20}) then overrules
by far the factors $\gap$ and $\dof^2$ on the right-hand side of 
(\ref{19}), implying that the time-averaged variance  
on the left-hand side of (\ref{19}) will be exponentially 
small compared to the (squared) measurement range 
$\Da$ of the observable.
In turn, this is only possible if the difference
$\At - \Att$ is unmeasurably small
(below the resolution limit of the measurement device $A$)
for the overwhelming majority of all time points $t\in[0,T]$.
%provided $T$ is sufficiently large.
As already said in the Introduction,
time points $t$ belonging to the complementary, exceedingly
small minority are generically expected to occur during the
initial transient relaxations processes, and on the occasion
of the well-known, exceedingly rare, but unavoidable 
quantum-revivals events.
(The initial relaxation may in fact be viewed as one of them.
Moreover, the origin of those revivals is closely related to
the fact that the sum in (\ref{9}) is a quasi-periodic function of
$t$.)
%, hence the expectation value inevitably must return close
%to its initial value after sufficiently long times.)
All these complications are effectively taken into account by 
our requirement above (\ref{19}) that $T$ must be sufficiently
large.

In summary, our main finding is that the deviations between $\At$
and $\Att$ will be negligibly small for the overwhelming majority
of all sufficiently late times $t$, symbolically indicated as
\begin{eqnarray}
\At \rightsquigarrow \Att
%=\frac{a}{f}\, \cos (2ht) + \frac{b}{f} \sin(2ht)
\ .
\label{21}
\end{eqnarray}


%%From this viewpoint, 
%Taking all this into account, our above conclusion that 
%$\At$ will be practically 
%undistinguishable from $\Att$ for practically
%all sufficiently late times is therefore as good as 
%%one possibly can expect.
%it possibly can be.

Incidentally,
similar methods as in the derivation of our present result in
%in Sec.~\ref{s2} and 
Appendix \ref{app1}
have been previously adopted, e.g., in Ref.~\cite{rei08,lin09,sho11,sho12,rei12,bal16}
in the context of {\em equilibration}, i.e., for the purpose
to show that the expectation 
values $\At$ remain -- under suitable conditions on the 
Hamiltonian $H$, the initial state $\rho(0)$, and the observable $A$ --
very close to some {\em constant} value for the vast 
majority of all sufficiently late times $t$.
%for suitably chosen Hamiltonian $H$, initial state $\rho(0)$, and observable $A$.
Obviously, such a prediction of equilibration
%previously established results 
cannot apply to our present models (\ref{2}) with $h\not =0$ 
since they generically give rise to everlasting oscillations of  $\At$.
The main reason is that the energies $E_{nl}$ in (\ref{8})
violate (for $h\not=0$) the corresponding requirements
in Refs.~\cite{rei08,lin09,sho11,sho12,rei12,bal16}
regarding the maximally admissible degeneracy
of the pertinent energy gaps.
Indeed, one finds that our present models entail some
exponentially large sets of degenerate energy gaps:
For instance, considering two arbitrary but fixed indices $l$ and $l'$
we can conclude from Eq.~(\ref{8}) that the energy gaps $E_{nl}-E_{nl'}$ 
are equal for all possible values of $n$, while the total 
number of all those $n$ values is often 
expected to be exponential in the system size.
Likewise, for any given set of indices $n,l,n',l'$ the energy
gaps $E_{n(l+l'')}-E_{n'(l'+l'')}$ are equal for all possible
values of $l''$.
%\JS{Indeed, one finds that our present models entail some
%exponentially large sets of degenerate energy eigenstate pairs,
%e.g.\ between $E_{nl}-E_{n'l'}$ and $E_{n(l+l'')}-E_{n'(l'+l'')}$
%or $E_{nl}-E_{nl'}$ and $E_{n'l}-E_{n'l'}$
%since the difference of their Zeeman energies can only assume 
%(a few) multiples of $h$.}
As a consequence, for $h\neq 0$ our models violate one of the central 
preconditions for equilibration established in Refs.~\cite{rei08,lin09,sho11,sho12,rei12,bal16}.

In contrast, the maximal degeneracy of energy gaps employed in
(\ref{18}) is a property of the {\em unperturbed} ($h=0$)
energies $E_n^0$, {\em not} of the energies
$E_{nl}$ pertaining to
the actually considered model Hamiltonian $H$ in (\ref{2}).
In passing, we also remark that,  according to Refs.~\cite{rei08,lin09,sho11,sho12,rei12,bal16}, 
it is the degeneracy of these gaps 
%$E_{nl}-E_{n'l'}$ 
which prohibits equilibration, 
{\em not} their commensurability, as 
%erroneously 
speculated, e.g., in \cite{boo20}.



%%%%%%%%%%%%%%%%%%%%%%%%%%%%%%%%%%%%%%%%%%%%%%%%%%%%%%%
\subsection{Canonical quenches}
\label{s33}

In view of (\ref{17}) we can conclude that
$(\pmax)^2$ is upper bound by 
$\sum_{nl}  \langle n,l |\rho(0)| n,l\rangle^2$ and hence by 
$\sum_{nlmk}  |\langle n,l |\rho(0)| m,k\rangle|^2= \tr\{[\rho(0)]^2\}$,
implying
\begin{eqnarray}
\pmax \leq \sqrt{ \tr\{[\rho(0)]^2\} }
\ .
\label{22}
\end{eqnarray}

As a particularly simple and interesting example, 
let us assume that the initial state is given by a thermal 
Gibbs state (canonical ensemble) of the form 
\begin{eqnarray}
\rho(0) = \tilde Z^{-1} e^{-\beta \tilde H}\ ,\ \ \tilde Z:=\tr\{ e^{-\beta \tilde H} \}
%\rho :=e^{-\beta H_0}\!/\tr\{ e^{-\beta H_0} \}
\ ,
\label{23}
\end{eqnarray}
where $\tilde H$ is in general different from the Hamiltonian 
$H$ in (\ref{2}) which governs the subsequent 
temporal evolution of $\rho(0)$.

For instance, one may choose $\tilde H$ to be of the general form
\begin{eqnarray}
\tilde H & := & H_0 + \sum_{i\in \G } \vec h_i\cdot \vec s_i
\ ,
\label{24}
\end{eqnarray}
thus differing from $H$ in (\ref{2}) with respect to the
direction and possibly also the magnitude of the externally 
applied magnetic field at any of the lattice sites $i$.
Further examples of how to choose physically reasonable 
$\tilde H$'s are rather obvious,
see also Sec.~\ref{s35} below.
%More generally speaking, $\tilde H$ may actually
%be chosen largely arbitrarily.
%%we always tacitly restrict ourselves to cases with non-vanishing tempetature $\beta^{-1}$.

%In other words, 
From a different viewpoint, the system may thus
be considered as being
at thermal equilibrium for $t<0$ and experiencing an instantaneous
``quantum quench'' at $t=0$, with pre-quench 
Hamiltonian $\tilde H$ and post-quench Hamiltonian $H$.

Recalling that the textbook free energy $F_\beta$
associated with the canonical ensemble (\ref{23}) 
obeys the relation $e^{-\beta F_\beta}=\tr\{e^{-\beta \tilde H}\}$,
one can conclude that $\tr\{[\rho(0)]^2\}=e^{-2\beta G_{\! \beta}}$
with $G{\! _\beta}:=F_{2\beta}-F_{\beta}$.
Taking for granted that the pre-quench system exhibits
generic thermodynamic properties, it follows that
$G{\! _\beta}$ is an extensive quantity.
[In particular, temperatures $\beta^{-1}$ extremely close to zero
are always tacitly excluded.]
Hence, $\tr\{[\rho(0)]^2\}$ decreases exponentially 
with the system size $\dof$, and likewise for $\pmax$
in (\ref{22}).
%Furthermore, we can conclude from (\ref{17}) that
%$(\pmax)^2$ is upper bound by 
%$\sum_{nl}  \langle n,l |\rho(0)| n,l\rangle^2$ and hence by 
%$\sum_{nlmk}  |\langle n,l |\rho(0)| m,k\rangle|^2= \tr\{[\rho(0)]^2\}$.
%More generally, it is in fact sufficient that the $G{\! _\beta}$
%grows faster than logarithmically with the system size
%to guarantee the smallness of (\ref{19}).

Altogether, we thus have rigorously verified (\ref{20})
for initial conditions of the  canonical form (\ref{24}).
%under the sufficient conditions that the 
%partition function $\tilde Z$ entails
%an extensive free energy.
%and that the temperature $\beta^{-1}$ is non-zero.
The same conclusion can also be readily recovered
for microcanonical instead of a canonical initial states
$\rho(0)$.

%%%%%%%%%%%%%%%%%%%%%%%%%%%%%%%%%%%%%%%%%%%%%%%%%%%%%%%
\subsection{Permanent oscillations}
%\subsection{Further conclusions}
\label{s34}

%Before doing so, we point out some interesting
%general features of the function $\Att$ in (\ref{a7}) 
%itself:
%Our first observation is that 
%all summands on the right hand side of (\ref{15}) must be zero if $|\nu|> 2fs$
%as a consequence of the discussion below (\ref{11})
To begin with, we note that
$\bar f_\nu$ in (\ref{15}) must be zero if $|\nu|> 2\dof s$
as a consequence of the restrictions on the summation indices below (\ref{9})
(the detailed reasoning is worked out 
%in Appendix \ref{app1}
below Eq.~(\ref{a14})).
%Furthermore, we observe that $\Att$ in 
%(\ref{14}) is a real valued function of $t$.
Furthermore, one can infer from (\ref{10}), (\ref{11}),
and (\ref{15}) that $\bar f_{-\nu}=\bar f_{\nu}^\ast$.
Representing the complex numbers $\bar f_\nu$ 
%by means of the corresponding complex phases $\varphi_\nu$ 
in the polar form $|\bar f_\nu|e^{i\varphi_\nu}$, 
we thus can rewrite (\ref{14}) as
\begin{eqnarray}
\Att =\bar f_0+ 2 \sum_{\nu=1}^{2\dof s} |\bar f_{\nu}| \, \cos(\nu ht+\varphi_\nu)
\ .
\label{25}
\end{eqnarray}

Generically, the quantities $\bar f_\nu$ in (\ref{15}) are 
not expected to identically vanish for all 
$\nu\not=0$,
%$\nu\in\{1,...,2fs\}$,
hence (\ref{25}) together with (\ref{21}) implies 
the occurrence of permanent oscillations for all
sufficiently late times $t$.
%about the mean value $\bar f_0$.

As an aside, we 
%finally 
remark that
the quantity $\bar f_0$ in (\ref{25}) obviously represents the long-time average of
$\Att$. In the generic case that all energies $E_n^0$ are pairwise different
(see also below Eq.~(\ref{6})), $\bar f_0$
can be further rewritten by means of 
(\ref{16}) and the so-called 
{\em diagonal ensemble}
\begin{eqnarray}
\rho_{\rm{dia}}:=\sum_{nl} \rho_{nn}^{l,l} \, |n,l\rangle\langle n,l |
\label{26}
\end{eqnarray}
in the form
\begin{eqnarray}
\bar f_0 = \tr\{\rho_{\rm{dia}} A\}
\ .
\label{27}
\end{eqnarray}

We also remark that our present oscillatory long-time effects are 
%also somewhat reminiscent 
similar to those recently 
%reported 
discovered in the ground-breaking work \cite{med20}. 
A first important difference is that Ref.~\cite{med20} is mainly focused on the
one-dimensional spin-1/2 XXZ-model (which is integrable), while our present model
class also covers, for instance, various non-integrable and disordered systems
(cf. Sec.~\ref{s21}).
The second important difference is that 
the findings reported in Ref.~\cite{med20} are
mainly based on non-rigorous arguments and numerical 
evidence,
adopting 
%and utilize
%for 
some rather special initial states and observables.
Finally, the 
%relevant effects 
prediction of permanent oscillations in Ref.~\cite{med20}
only applies to a quite restricted (discrete) subset of the XXZ spin chain's 
anisotropy parameter values.
%Spaeter benutzt: Altogether, our present work thus complements
%and substantially extends those previous findings.}
%[Heuristic estimates based on unproven conjectures 
%have been obtained in similar cases in Ref.~\cite{med20}).]



%%%%%%%%%%%%%%%%%%%%%%%%%%%%%%%%%%%%%%%%%%%%%%%%%%%%%%%
\subsection{Numerical examples}
\label{s35}

The subsequent numerical examples are chosen to illustrate 
our two main analytical findings for Hamiltonians of the
general form (\ref{1})-(\ref{3}): 
(I) permanent oscillations, and (II) 
synchronization of these oscillations 
in case of translationally invariant Hamiltonians
(see also Sec.~\xref{s4} below).
To this end, we numerically explore the behavior 
of
the following 
four specific models:
(i) A spin ring (periodic boundary conditions)
with unperturbed Hamiltonian
\begin{eqnarray}
%H_0^R 
H_0
& := & \sum_{i = 1}^{\dof} J_i \vec s_i\cdot \vec s_{i+1}
\ ,
\label{H0R}
\end{eqnarray}
exhibiting quenched disorder by choosing the interactions 
$J_i$ as independent, identically distributed random numbers.
(ii) The same spin ring model as in (\ref{H0R}), but now with
identical couplings $J_i$ for all $i$ (no disorder).
(iii) A two-dimensional (2D) 
$5\times 5$ square lattice model
with identical nearest-neighbor interactions,
open boundary conditions in both directions, 
and unperturbed Hamiltonian
\begin{eqnarray}
\label{H02D}
%H_0^{2D} 
H_0
& := & J \sum_{<i,j>} \vec s_i\cdot \vec s_{j}
\ . 
\end{eqnarray}
(iv) The same square lattice model as in (\ref{H02D}), 
but now with periodic boundary conditions in both 
directions.
Similar to (\ref{1})-(\ref{3}),
an additional homogeneous magnetic field in $z$-direction 
%$\sum_{i = 1}^{\dof} h s_i^z $ 
is applied during time evolution 
in all four cases (i)-(iv).
Accordingly, the spin ring without disorder represents an 
integrable model, whereas all the other examples (i), (iii), and (iv)
are commonly considered as non-integrable, see also Sec.~\ref{s21}.
Moreover, (ii) and (iv) are so-called translationally
invariant models (see also Sec. \ref{s4} below),
while (i) and (iii) are not.


%%%%%%%%%%%%%%%%%%%%%%%%%%%%%%%%%%%%%%%%%%%%%%%%%%%%%%%%%%%%%%%%%%%%%
\begin{figure}
\includegraphics[scale=0.20]{fig1.png}
\caption{
Red arrows: Visualization of the projections to the $(x,y)$-plane 
of the expectation values of the local spin vector operators 
$\vec s_i$ with respect to the initial state \eqref{psi0}.
(a): Spin ring models (i) and (ii) as specified around 
Eq.~(\ref{H0R}) and below (\ref{HtildeR}).
(b): Square lattice models (iii) and (iv) as specified around 
Eq.~(\ref{H02D}) and below (\ref{HtildeR}).
The grey and white regions indicate our choice of the
sublattices $\Lambda_1$ and $\Lambda_2$ in (\ref{HtildeR}).
All the remaining model parameter values in (\ref{psi0}) and
(\ref{HtildeR})
have been chosen in (a) as detailed in Fig.~\ref{fig2},
and in (b) as detailed in Fig.~\ref{fig3}.
%for the spin ring (a) and the square lattice (b). 
%It applies $\kappa=24$ and random $J_i \in \lbrack-3,1\rbrack$ for 
%the ring and $\kappa=25$, $J=-2$ for the square lattice.
%For the initial state, see \eqref{HtildeR}, $\beta = 1$ is applied as 
%well as $h_x=1$ for the spins in $\Lambda_1$ (gray background) and
%$h_y=1$ for the spins in $\Lambda_2$ (white background).
}
\label{fig1}
\end{figure}
%%%%%%%%%%%%%%%%%%%%%%%%%%%%%%%%%%%%%%%%%%%%%%%%%%%%%%%%%%%%%%%%%%%%%

As our initial condition $\rho(0)$ (see above Eq.~(\ref{8a}))
we choose a pure state of the form 
$\rho(0)=|\psi\rangle\langle\psi|$ with
\begin{eqnarray}
|\psi \rangle \propto e^{-\frac{\beta}{2} \tilde H}| \phi\rangle
\ ,
\label{psi0}
\end{eqnarray}
where $|\phi\rangle$ is a normalized random vector, 
which may be viewed as point on the unit 
sphere in $\CC^{(2s+1)^{\kappa}}$,
randomly sampled according to a uniform 
distribution.
It is well-known that such an initial condition 
exhibits a so-called dynamical typicality property,
meaning that it imitates very accurately the behavior 
of the canonical ensemble from \eqref{23}, see, e.g. 
Ref. \cite{rei} and further references therein.
More precisely speaking, for the vast majority of
all those randomly sample initial states
$\rho(0)=|\psi\rangle\langle\psi|$,
the time-dependent expectation values in 
(\ref{8a}) become, for sufficiently large 
system sizes $\kappa$, practically indistinguishable 
from those which one would obtain by choosing 
$\rho(0)$ according to (\ref{23}).
A more precise analytical quantification of 
%the above notion 
%``sufficiently large'' 
the remaining deviations
is in general
quite difficult, but we numerically verified 
that our results for different random initial states
were indeed nearly indistinguishable
on the scale of the subsequent plots.
%so we do not need to average over different initial states. 
Apart from this connection to the canonical ensemble in (\ref{23}),
our initial state (\ref{psi0}) represents, of course, already
in itself a perfectly legitimate, generally far from 
equilibrium initial condition.

%%%%%%%%%%%%%%%%%%%%%%%%%%%%%%%%%%%%%%%%%%%%%%%%%%%%%%%%%%%%%%%%%%%%%
\begin{figure}
\includegraphics[scale=1.3]{fig2.pdf}
\caption{
%\JS{Spin ring  \eqref{H0R}, $\kappa=24$, random $J_i \in \lbrack-3,1\rbrack$, $h = 1$, 
%and inverse temperature $\beta = 1$ in \fmref{psi0}: 
%expectation values of local operators $s_i^x$ for early times (a) as well as 
%for late %times (b) and nearest-neighbor correlations $s_i^x s_{i+1}^x$ for early (c) as well as 
%late times (d).}}
(a) and (b): Expectation values \eqref{8a} of the local 
observables $A=s_i^x$ for early times (a) as well as for late times (b)
by numerically solving the spin ring model from \eqref{H0R} with 
$\kappa=24$ spins, periodic boundary conditions,
random couplings $J_i \in \lbrack-3,1\rbrack$,
and magnetic field $h = 1$, see also Eqs.~\eqref{1}-\eqref{3}. 
The different colors correspond to the 24 different 
observables $A=s_i^x$.
The 
%far from equilibrium 
initial condition $\rho(0)$
is given by a 
canonical ensemble of the form \eqref{23}, \eqref{24} with 
$\beta=1$, choosing the Hamiltonian $\tilde H$ according to 
\eqref{HtildeR} with $\tilde H_0=H_0$,
%$\tilde h=1$, 
$h_x=h_y=1$,
and sublattices $\Lambda_{1,2}$ as indicated in Fig.~\ref{fig1}(a),
see also main text for more details.
%Ev. nur im Haupttext: 
In the actual numerics, the behavior of the corresponding time evolved 
$\rho(t)$ was imitated by numerically evolving a random initial state 
as explained around Eq.~\eqref{psi0}.
(c) and (d): Same, but for the observables $A=s_i^xs_{i+1}^x$ with $i=1,...,23$.
}
\label{fig2}
\end{figure}
%%%%%%%%%%%%%%%%%%%%%%%%%%%%%%%%%%%%%%%%%%%%%%%%%%%%%%%%%%%%%%%%%%%%%

%%%%%%%%%%%%%%%%%%%%%%%%%%%%%%%%%%%%%%%%%%%%%%%%%%%%%%%%%%%%%%%%%%%%%
\begin{figure}
\includegraphics[scale=1.3]{fig3.pdf}
\caption{
Same as in Fig.~\ref{fig2}, but now for a $5\times 5$ square lattice model 
of the form \eqref{H02D} with $\kappa=25$ spins, open boundary conditions,
and couplings $J=-2$.
In particular, the initial condition is again of the form
\eqref{23}, \eqref{24}, \eqref{HtildeR}
with $\beta=1$, $\tilde H_0=H_0$,
%$\tilde h=1$, 
$h_x=h_y=1$, 
and 
%two-dimensional 
sublattices $\Lambda_{1,2}$ as indicated in Fig.~\ref{fig1}(b).
%\Lambda_{1,2}$ as specified in the main text and Fig.~\ref{fig1}.
%, and magnetic field $h = 1$.}
%Two-dimensional square lattice \eqref{H02D}, $\kappa=5\times 5$, $J=-2$, open boundary condition,
%$h = 1$, and inverse temperature $\beta = 1$ in \fmref{psi0}: 
%expectation values of local operators $s_i^x$ for early times (a) as well as for late times (b)
%and nearest-neighbor correlations $s_i^x s_{i+1}^x$ for early (c) and late times (d).}
}
\label{fig3}
\end{figure}
%%%%%%%%%%%%%%%%%%%%%%%%%%%%%%%%%%%%%%%%%%%%%%%%%%%%%%%%%%%%%%%%%%%%%

Once the initial state has been chosen,
we numerically evolved it in time 
by means of Suzuki-Trotter product expansion techniques,
as detailed, for instance, in Ref. \cite{michi}.

While this temporal evolution is governed by the
above specified, so-called post-quench Hamiltonian $H$
(see also Sec. \ref{s33}), 
the so-called pre-quench Hamiltonian $\tilde H$, 
governing the initial condition via
(\ref{23}) and (\ref{psi0}), is 
chosen as
\begin{eqnarray}
\tilde H & := & \tilde H_0  
+ 
%\PR{\tilde h} 
h_x
\sum_{i\in \Lambda_1} s_i^x 
+ 
%\PR{\tilde h}  
h_y
\sum_{i\in \Lambda_2} s_i^y
\ ,
\label{HtildeR}
\end{eqnarray}
where $\tilde H_0$ is 
%assumed to be 
of the same general 
structure as in (\ref{H0R}) in our one-dimensional examples 
(i) and (ii), and as in (\ref{H02D}) in our two-dimensional 
examples (iii) and (iv).
More precisely speaking, $\tilde H_0$ was chosen identical 
to $H_0$ from (\ref{H0R}) and (\ref{H02D}) in (i) and (iii), 
respectively,
while the same $\tilde H_0$'s as in (i) and (iii) were 
then also 
employed in (ii) and (iv), respectively.
Furthermore, $\Lambda_1$ and $\Lambda_2:=\Lambda\setminus\Lambda_1$ 
in (\ref{HtildeR}) 
denote two complementary subsets of the respective
total lattices $\Lambda$ (see above Eq.~(\ref{1})).
Their specific choice for the examples
(i) and (ii) is visualized by the grey and white regions
in \figref{fig1}(a), and for the examples (iii) and (iv) 
in \figref{fig1}(b).
According to (\ref{HtildeR}), the spins
in those two sublattices (grey and white)
are thus polarized by the external magnetic 
fields $h_x$ and $h_y$ along orthogonal directions, 
resulting via (\ref{psi0}) in initial conditions
for the individual spins as cartooned by the red 
arrows in \figref{fig1}.

Such inhomogeneous initial states with 
%basically 
two extended domains of macroscopic magnetization 
appeared to us as particularly interesting and 
non-trivial examples.
%to provide a non-trivial initial dynamics and to 
%indicate that \JS{the initial state} is not relevant for later behavior.
For instance, they clearly are {\em not} translationally 
invariant (see also Sec.~\ref{s4} below).
Moreover, they are far from thermal equilibrium with respect
to the post-quench Hamiltonian $H$.

As a first example, \figref{fig2} displays 
the dynamics of various local observables for a spin 
ring model of type (i) with parameters given in the caption. 
With these parameters, the energy is not close 
to the edges of the spectrum. 
At early times, panels (a) and (c), 
the different observables behave
%mostly individually 
rather irregularly, starting from their various initial values, 
whereas at later times, panels (b) and (d), all observables 
exhibit quite regular oscillations with angular frequency
$h$ in (b) and $2h$ in (d),
thus confirming and illustrating our main
analytical prediction from the previous subsections.
%for $s_i^x$ and $2h$ for $s_i^x s_{i+1}^x$.
The phases between these long-time oscillations 
seem to be astonishingly small, but the amplitudes 
differ quite notably (and in (d) also the 
time-averaged values).
We will briefly return to this observation
%an observation that will be addressed 
at the end of Sec.~\xref{s4}.

For the two-dimensional square lattice model of type (iii),
%\eqref{H02D} 
qualitatively quite similar results are observed in
\figref{fig3}. 
The main difference is that some of the long-time oscillations,
especially in (d), still exhibit notable deviations from a 
strictly periodic behavior,
which can be naturally understood as finite-size 
corrections to our analytical predictions.
%Also here, the expectation values of local observables 
%oscillate permanently. For different spins the
%expectation values differ somewhat more strongly 
%compared to the ring, in particular for 
%$s_i^x s_{i+1}^x$.
%\JS{\textbf{Streichen:} For this system spins are not equivalent due to %open boundary conditions.}

%%%%%%%%%%%%%%%%%%%%%%%%%%%%%%%%%%%%%%%%%%%%%%%%%%%%%%%%%%%%%%%%%%%%%
\begin{figure}
\includegraphics[scale=1.3]{fig4.pdf}
\caption{
Same as in Fig.~\ref{fig2}, but now for non-random 
couplings $J_i=-1$ in the spin ring model (\ref{H0R}).
In particular, exactly the same the initial condition 
%$|\psi(t=0) \rangle$
as in Fig.~\ref{fig2} was utilized.
%i.e., with random couplings $J_i$ appearing in (\ref{HtildeR}) via the same 
%unperturbed Hamiltonian $\tilde H_0$ as in Fig.~\ref{fig2}.
%\JS{Maybe here a too detailed explanation is confusing. Exactly the same should be sufficient.}
%\JS{Spin ring \eqref{H0R}, $\kappa=24$, \LD{$J_i = -1 \; \forall i$}, $h = 1$, 
%initial state $|\psi(t=0) \rangle$ identical to \figref{fig2}:
%and inverse temperature $\beta = 1$ in \fmref{psi0}: 
%expectation values of local operators $s_i^x$ for early times (a) as well 
%as for late times (b)and nearest-neighbor correlations $s_i^x s_{i+1}^x$ 
%for early (c) and late times (d).}
}
\label{fig4}
\end{figure}
%%%%%%%%%%%%%%%%%%%%%%%%%%%%%%%%%%%%%%%%%%%%%%%%%%%%%%%%%%%%%%%%%%%%%

Turning to the two remaining, translationally invariant models (ii) and (iv),
%For the more symmetric cases, i.e.\ translationally invariant ring and lattice, 
%where all spins of the system are equivalent, 
we encounter almost perfect synchronization of the individual local 
observables after initial transients have died out.
Figure~\xref{fig4} displays this behavior for
the spin ring model (ii), where local spin operators are related 
to each other by a translation along the ring, 
a symmetry operation under which the Hamiltonian is invariant
(see also Sec. \ref{s4} below).
For late times, panels (b) and (d), the various oscillations 
superimpose perfectly, although the initial state is exactly the same
as 
the one in \figref{fig2}, i.e.\ adapted to the Hamiltonian
with disorder. We have verified that this synchronization 
behavior is practically independent of the initial conditions.

%%%%%%%%%%%%%%%%%%%%%%%%%%%%%%%%%%%%%%%%%%%%%%%%%%%%%%%%%%%%%%%%%%%%%
\begin{figure}
\includegraphics[scale=1.3]{fig5.pdf}
\caption{
Same as in Fig.~\ref{fig3}, but now for periodic 
boundary conditions in the $5\times 5$
square lattice model (\ref{H02D}).
In particular, exactly the same the initial condition 
%$|\psi(t=0) \rangle$
as in Fig.~\ref{fig3} was utilized.
%, i.e., with open boundary conditions appearing in (\ref{HtildeR}) 
%via the same unperturbed Hamiltonian $\tilde H_0$ as in Fig.~\ref{fig3}.
%\JS{Two-dimensional square lattice \eqref{H02D}, $\kappa=5\times 5$, $J= -2$, 
%periodic boundary condition, $h = 1$, initial state $|\psi(t=0) \rangle$ 
%identical to \figref{fig4}: expectation values of local operators $s_i^x$ 
%for early times (a) as well as for late times (b) and nearest-neighbor 
%correlations $s_i^x s_{i+1}^x$ for early (c) as well as late times (d).}
}
\label{fig5}
\end{figure}
%%%%%%%%%%%%%%%%%%%%%%%%%%%%%%%%%%%%%%%%%%%%%%%%%%%%%%%%%%%%%%%%%%%%%

A qualitative similar behavior is also observed for
our $5\times 5$ square lattice model (iv)
%with periodic boundary conditions and 
in \figref{fig5}.
As will be explained in more detail in Sec.~\ref{s4},
the observed synchronization at large times
has its origin in the model's translational 
invariance.
%Also for our $5\times 5$ square lattice model (iii)
%%with periodic boundary conditions and 
%the translational symmetry both along $x$- and $y$-direction 
%we observe synchronization as expected for equivalent spins, 
%\LD{see \figref{fig5} and compare with} sec.~\xref{s4}. 
Similarly as in Fig.~\ref{fig3}, the remnant deviations
from prefect synchronization, especially in Fig.~\ref{fig5}(d),
can be explained in terms of finite-size effects.
Apparently, the fact that our square lattice models (iii) and (iv)
only exhibit a relatively short period of $5$ along each spatial 
direction is responsible for the stronger finite-size
corrections in comparison to the spin ring models (i) and (ii).
We also confirmed this expectation by directly comparing 
the numerical results for different system sizes with each 
other (not shown).
%Again, the late-time imperfections are
%larger than for the ring, a phenomenon that we take as an effect of the finite size
%ascribe to the rather short period of 5 in every direction of the lattice.

Further numerical examples for a variety of other model Hamiltonians 
and, more importantly, other initial conditions can also be 
found in Ref.~\cite{vor21}.

%%%%%%%%%%%%%%%%%%%%%%%%%%%%%%%%%%%%%%%%%%%%%%%%%%%%%%%
\vspace*{1cm}
\section{Synchronization}
\label{s4}

A particularly remarkable feature of the numerical results in 
Figs.~\ref{fig4} and \ref{fig5} is the close agreement of all 
the differently colored graphs for sufficiently late times (right panels),
while Figs.~\ref{fig2} and \ref{fig3} do {\em not} exhibit 
such a behavior. In the following, our main objective is a 
better understanding of this numerical observation.

%\JS{In the following we are going to explain, how synchronization as seen in the 
%previous section arises from translational invariance. For this purpose and without 
%loss of generality,
For the sake of simplicity, we mainly focus
on one-dimensional spin models (\ref{2}). 
Moreover, we require that the model is
translationally invariant in the sense that
site $i=\dof+1$ is identified with $i=1$ (periodic boundary conditions)
and the couplings $J_{ij}$ only depend on the  difference $i-j$ modulo $\dof$.
Finally, we restrict ourselves to
%systems without any degeneracies of the energies $E_n^0$, i.e., $E_m^0=E_n^0$ implies
%$m=n$ (see also end of Sec.~\ref{s31}).
the generic case that all energies $E_n^0$ 
are pairwise different,
%Accordingly, (\ref{15}) can be rewritten as
%\begin{eqnarray}
%\bar f_{\nu}:= \sum_{nk} \rho_{nn}^{k,k+\nu} A_{nn}^{k+\nu,k}
%\ .
%\label{16}
%\end{eqnarray}
hence the quantities $\bar f_\nu$ are given by (\ref{16}),
see also the remarks below Eqs.~(\ref{6}) and (\ref{16}).

%Note, that any such model must exhibit periodic boundary conditions.
For the rest, short- as well as long-range interactions are still admitted.
Moreover, various generalizations, e.g., to higher-dimensional 
hypercubic lattices (with periodic boundary conditions) are straightforward
(see also Sec.~\ref{s21}), but will not be explicitly worked out.

Denoting by $\TT$ the so-called translation operator,
it is shown in Appendix \ref{app2} that
\begin{eqnarray}
\langle n, l |\TT^\dagger\! B \TT |n,l'\rangle=\langle n, l | B | n,l' \rangle
\label{28}
\end{eqnarray}
for arbitrary Hermitian operators $B$ and 
%integers 
indices
$n,l,l'$.
Physically, $\TT^\dagger\! B \TT$ represents the same observable
as $B$, except that ``everything is shifted'' by one unit along the 
periodic spin chain.
For instance, for a single-site spin operator $s_i^a$ 
(with $a\in\{x,y,z\}$)
one finds that $\TT^\dagger\! s_i^a \TT=s_{i+1}^a$, 
and analogously for arbitrary sums and products of such operators.
In particular, the total spin components from (\ref{1}) 
and the Hamiltonian from (\ref{2}) are found to be
translationally invariant  in the sense that they commute with $\TT$.

Taking into account (\ref{11}) and (\ref{28}),
one can conclude that the quantities $\bar f_\nu$ in (\ref{16}) 
and thus the function $\Att$ in (\ref{14}) and (\ref{25})
remain unchanged if we replace the observable $A$ by 
its shifted counterpart $\TT^\dagger\! A \TT$.
For instance, the expectation values of the single-site 
spin operators $s_i^a$ are thus
predicted to synchronize (look the same for all $i$)
in the long run, 
and likewise for arbitrary sums and products of 
such operators.
These findings are illustrated by 
Figs. \ref{fig4} and \ref{fig5},
see also Sec.~\xref{s35}.
The small remnant deviations from strict synchronizations
in these numerical examples can be naturally understood as
finite-size effects.
%(see also Secs.~\ref{s32} and \ref{s35}).
%This is also confirmed by comparing those deviations for
%different system sizes with each other (not shown).

It readily follows that so-called local operators $A_i$ with 
the property $\TT^\dagger\! A_i \TT=A_{i+1}$
will synchronize in the above sense not only with each other
but also with their ``intensive'' counterpart
$A:=\sum_{i\in \G } A_i/f$, as exemplified by Eq.~(\ref{29}) below.
We also remark that all these conclusions apply
to arbitrary initial states $\rho(0)$ 
%(provided (\ref{16}) and (\ref{20}) are satisfied).
(as long as they satisfy (\ref{20})).
In particular, $\rho(0)$ is {\em not} 
required to be translationally invariant.

An analogous line of reasoning implies that
the initial condition $\rho(0)$ and its shifted counterpart
$\TT^\dagger\! \rho(0) \TT$ exhibit in the long run
(nearly) identical expectation values for arbitrary 
observables $A$ and any initial state $\rho(0)$ 
which satisfies (\ref{20}).

Altogether, the generically occurring, permanent long-time oscillations 
from Sec.~\ref{s34} are thus found to synchronize in the sense of being 
invariant under arbitrary translations of the considered observable, 
provided the system Hamiltonian 
(but not necessarily the initial condition) is translationally invariant.

Closely related numerical findings have been recently reported in Ref.~\cite{vor21}.
Our present work 
%thus 
amounts to a rigorous analytical validation and 
generalization of this numerical discovery of synchronization
in closed (isolated) systems of the form (\ref{1})-(\ref{3}).
Similarities and differences with respect to related 
(transient and permanent) synchronization phenomena 
in open (dissipative) systems have also
been addressed already in Ref.~\cite{vor21} (see also \cite{buc22}),
and are therefore not repeated here.
The salient point is that while the observable phenomena are
similar, the basic physical mechanisms 
as well as the analytical methods are entirely different 
%in the two cases.
for closed and open systems.

Various slightly different 
%(mostly weaker) 
notions of synchronization 
%have been collected and discussed 
are reviewed, for instance, in Ref.~\cite{buc22}.
Our present notion appears to us particularly simple and natural.

Intuitively, and also on the basis of our above calculations, it
seems reasonable to suspect that translational invariance
(equivalence of all spin sites)
is not only sufficient but that it generically is
even necessary for the occurrence of synchronization 
in our present sense.
This expectation is further corroborated by the numerical examples 
in Figs.~\xref{fig2} and \xref{fig3}.

%\JS{We thus conclude, that there is no rigorous argument to assume 
%synchronization for spins that are not equivalent.}
%
%Furthermore, an intuitive argument in support of this
%conjecture is as follows:
%Starting out from a translationally invariant spin chain with nearest-neighbor 
%interactions,
%we remove the interaction between two arbitrarily chosen neighbors,
%resulting in an ``almost translationally invariant'' model with open 
%boundary conditions.
%Next, we compare the expectation values at thermal equilibrium 
%(for instance by employing the canonical ensemble) of a 
%local observable at one of the chain's ends with the corresponding 
%shifted (but otherwise identical) local observable in the middle 
%of the chain.
%Though these expectation values may be quite similar in many
%cases, there is no reason to expect that they generically 
%assume exactly equal values.
%In other words the translational invariance of those thermal
%expectation values in the original, translationally invariant 
%model has disappeared. 
%Analogous consequences are expected for any other 
%way of breaking the original translational invariance 
%as well as in cases
%where the long-time property of interest consist in 
%permanent non-equilibrium oscillations instead of 
%steady (thermal equilibrium) expectation values.

%%%%%%%%%%%%%%%%%%%%%%%%%%%%%%%%%%%%%%%%%%%%%%%%%%%%%%%
%\vspace*{1cm}
\section{Simple Analytical examples}
\label{s5}

Of foremost interest are cases where
$\bar f_\nu$ in (\ref{15}) is non-zero at least for one $\nu\not=0$, 
giving rise to non-equilibration in the form of permanent 
oscillations in %(\ref{14}) and 
(\ref{25}).
In general, the explicit evaluation of
$\bar f_\nu$ in (\ref{15}) is a quite demanding task.
In the following, we focus on some particularly
simple examples.



%%%%%%%%%%%%%%%%%%%%%%%%%%%%%%%%%%%%%%%%%%%%%%%%%%%%%%%
\subsection{Single spins}
\label{s51}

To begin with, we illustrate the general idea by means 
of the observables
\begin{eqnarray}
M_a:=\frac{1}{\dof}\, S^a=\frac{1}{\dof}\sum_{i\in \G }s_i^a
\ ,
\label{29}
\end{eqnarray}
see also Eq.~(\ref{1}),
i.e., the $M_a$ are essentially the magnetizations
along the spatial direction $a\in\{x,y,z\}$.
Employing the usual 
%definitions
raising and lowering operators
\begin{eqnarray}
S^\pm & := & S^x\pm i S^y
\label{30}
\end{eqnarray}
one readily recovers the textbook relations (see also around
Eqs.~(\ref{4})-(\ref{6}))
\begin{eqnarray}
S^\pm |n,l\rangle & = & c^\pm_{n,l} |n,l\pm 1\rangle
\label{31}
\\
c^\pm_{n,l} & := & \sqrt{L_n(L_n+1) - l (l\pm 1)}
\ .
\label{32}
\end{eqnarray}

Observing that Eqs.~(\ref{12}) and (\ref{13}) are linear in $A$, 
the same equations must also apply to the non-Hermitian operator 
$A:=S^+$ from (\ref{30}).
Exploiting (\ref{11}) and (\ref{31}), it follows that
\begin{eqnarray}
A_{nm}^{k+\nu,k} = \delta_{n,m}\, \delta_{\nu,1} \, c_{n,k}^+
\ ,
\label{33}
\end{eqnarray}
where $\delta_{n,m}$ and $\delta_{\nu,1}$ are Kronecker deltas.
Hence, we can conclude with Eqs.~(\ref{13}) and (\ref{15}) that
\begin{eqnarray}
f_{\nu}(t) = \bar f_\nu = \delta_{\nu,1}\, f_1(0)
\label{34}
\end{eqnarray}
and with Eq.~(\ref{12}) and (\ref{14}) that $\At = \Att= \langle A\rangle_{\! 0}e^{iht}$.
By means of a similar line of reasoning for $A:=S^-$ one thus
arrives at
\begin{eqnarray}
\langle S^\pm\rangle_{\! t}  
= {\cal S}_t^\pm
= \langle S^\pm\rangle_{\! 0} e^{\pm iht}
\ .
\label{35}
\end{eqnarray}

Since $S^x=(S^+ + S^-)/2$ and $S^y=(S^+ - S^-)/2i$ according to
(\ref{30}), we finally obtain for the magnetizations $M_{x,y}$ from (\ref{29})
the result
\begin{eqnarray}
\langle M_x\rangle_{\! t} 
& = & 
a_1 \cos (ht) - b_1 \sin(ht)
\ ,
\label{36}
\\
\langle M_y\rangle_{\! t} 
& = & 
b_1 \cos (ht) + a_1 \sin(ht)
\label{37}
\\
a_1
& := & 
\langle M_x\rangle_{\! 0}
\ , 
\label{38}
\\
b_1 
& := & 
\langle M_y\rangle_{\! 0}
\ . 
\label{39}
\end{eqnarray}
i.e., these particular observables 
exhibit perfect harmonic oscillations 
right from the beginning (i.e., for all $t$)
for any initial state
$\rho(0)$ with a non-vanishing expectation 
value of $M_x$ or of $M_y$.
In the same way one finds that
\begin{eqnarray}
\langle M_z\rangle_{\! t} 
& = & 
\langle M_z\rangle_{\! 0} 
\ ,
\label{40}
\end{eqnarray}
i.e., this particular observable is, as expected, always
a conserved quantity.

Similarly as in the first equality in (\ref{35}), one readily sees 
that for the three specific observables $A:=M_a$ from above,
the auxiliary functions $\Att$ 
%in (\ref{25}) 
happen to be exactly identical to the
true expectation values $\At$ 
%in (\ref{36})
for all $t$.
In other words, none of further preconditions on the
energies $E_n^0$, the system size $\dof$, 
and the initial condition $\rho(0)$
from Sec.~\ref{s32} are actually
needed in these specific examples.
%(see also beginning of this section).

In a next step, let us focus on systems which
satisfy the preconditions for our main result
in Sec.~\ref{s32} as well as the preconditions 
for synchronization as detailed at the beginning 
of Sec.~\ref{s4}.
According to the discussion at the end of
Sec.~\ref{s4}, we thus can conclude that
the single-spin expectation values
$\langle s_i^a\rangle_{\! t}$ 
behave for most sufficiently large times $t$ very similarly
to each other and thus to 
$\langle M_a\rangle_{\! t}$ (see (\ref{29})),
symbolically indicated as
\begin{eqnarray}
\langle s_i^a\rangle_{\! t}
\rightsquigarrow
\langle M_a\rangle_{\! t}
%=\frac{a}{f}\, \cos (2ht) + \frac{b}{f} \sin(2ht)
\ ,
\label{41}
\end{eqnarray}
where $a\in\{x,y,z\}$.
In particular, for any given observable $A:=s_i^x$,
the corresponding auxiliary function $\Att$
takes the $i$-independent explicit form (\ref{36}),
and similarly for $s_i^y$ and $s_y^z$.
On the other hand, for short times $t$ the expectation values
$\langle s_i^a\rangle_{\! t}$ are in general
no longer close to $\langle M_a\rangle_{\! t}$.
%In particular, the quantity in (\ref{38}) is in general {\em not} 
%equal (or close to) $\langle s_i^x\rangle_{\! 0}$.
Rather, and as can be seen in Figs. \ref{fig4} and \ref{fig5},
any given $\langle s_i^a\rangle_{\! t}$ 
%is expected to 
generically exhibits a non-trivial initial
relaxation process of its own, whose details depend 
in a complicated manner on the initial state $\rho(0)$ 
and on the Hamiltonian $H$.
Moreover, even for large times $t$ there will 
generically remain
fluctuations of $\langle s_i^a\rangle_{\! t}$ about
$\langle M_a\rangle_{\! t}$, which are negligibly small
for most $t$ but may become large for some
very rare $t$'s (quantum revivals).

%%%%%%%%%%%%%%%%%%%%%%%%%%%%%%%%%%%%%%%%%%%%%%%%%%%%%%%
\subsection{Higher harmonics}
\label{s52}

Our next examples are observables of the form $A:=M_a^2$.
By means of similar calculations as before 
one finds  that
\begin{eqnarray}
\langle M_x^2\rangle_{\! t} 
& = & 
a_2\, \cos (2ht) - b_2 \sin(2ht) + c_2
\ ,
\label{42}
\\
\langle M_y^2\rangle_{\! t} 
& = & 
-a_2\, \cos (2ht) + b_2 \sin(2ht) + c_2
\ ,
\label{43}
\\
\langle M_z^2\rangle_{\! t} 
& = & 
\langle M_z^2\rangle_{\! 0} 
\ ,
\label{44}
\end{eqnarray}
where we introduced the abbreviations
\begin{eqnarray}
a_2
& := & 
\langle M_x^2-M_y^2\rangle_{\! 0}/2
\ , 
\label{45}
\\
b_2 
& := & 
\langle M_xM_y+M_yM_x\rangle_{\! 0}/2
\ . 
\label{46}
\\
c_2
& := & 
\langle M_x^2+M_y^2\rangle_{\! 0}/2
\ , 
\label{47}
\end{eqnarray}
As expected,  $M_x^2+M_y^2$ and $M_z^2$ are 
thus conserved quantities.
Moreover, the observables $M_{x,y}^2$ exhibit perfect harmonic 
oscillations for all $t$ and for all initial conditions $\rho(0)$
with a non-vanishing expectation value in (\ref{44}) or in (\ref{46}).
Last but not least, the oscillation frequency is now twice as large 
as in  (\ref{36}) and (\ref{37}) (higher harmonics).

Combining (\ref{36}) and (\ref{42}) one can conclude that
\begin{eqnarray}
\langle M_x^2\rangle_{\! t} - \langle M_x\rangle^2_{\! t}
\!\!
& = & 
\!\!
a_2' \cos (2ht) 
- 
b_2' \sin(2ht) 
+ c_2'
\, ,
\label{48}
\\
a_2'
& := &
(\sigma^2_{xx}-\sigma^2_{yy})/2
\ ,
\label{49}
\\
b_2'
& := &
(\sigma^2_{xy}+\sigma^2_{yx})/2
\ ,
\label{50}
\\
c_2'
& := &
(\sigma^2_{xx}+\sigma^2_{yy})/2
\ ,
\label{51}
\end{eqnarray}
where
\begin{eqnarray}
\sigma^2_{ab}
& := & 
\langle M_a M_b\rangle_{\! 0} 
- 
\langle M_a \rangle_{\! 0}
\langle M_b \rangle_{\! 0} 
%=
%\langle
%(S^a-\langle S^a\rangle_{\! 0}) 
%(S^b-\langle S^b\rangle_{\! 0}) 
%\rangle_{\! 0} 
\label{52}
\end{eqnarray}
for arbitrary $a,b\in\{x,y,z\}$.
Analogous results as for $M_x$ in (\ref{48})
apply to $M_y$ and $M_z$.

Incidentally, for observables of the form
$s_i^xs_j^x$ one still can deduce from
(\ref{5}) and (\ref{15}) 
that the long-time asymptotics must be of 
the general structure
\begin{eqnarray}
\langle s_i^xs_j^x\rangle_{\! t}
\rightsquigarrow
a_{ij} \cos (2ht) 
+ b_{ij} \sin(2ht) 
+ c_{ij}
\ .
\label{53}
\end{eqnarray}
Most importantly, the oscillation frequency is 
again twice as large as in (\ref{36}), (\ref{41}),
in accordance with the numerical examples in Figs. 
\ref{fig2}-\ref{fig5}.
Similarly as in Eqs.~(\ref{36})-(\ref{41}), 
the coefficients $a_{ij}, b_{ij}, c_{ij}$ in (\ref{53})
are once more independent of $h$,
but now their quantitative dependence on the 
initial state $\rho(0)$ and on the Hamiltonian $H_0$ is 
%\PR{in general} 
very difficult to specify in more detail.
%\PR{(a trivial exception is $i=j$ since $(s_i^x)^2=1/4)$).}
Analogous statements apply to observables of the form 
$s_i^a s_j^b$ with $a,b\in\{x,y\}$ and to products of more 
than two such factors.

%%%%%%%%%%%%%%%%%%%%%%%%%%%%%%%%%%%%%%%%%%%%%%%%%%%%%%%
\subsection{Thermodynamic limit}
\label{s53}

Next we turn to the issue of how the above findings 
depend on the system size $\dof$, and, in particular,
how they behave for asymptotically large $\dof$
(thermodynamic limit).
As usual in this context, we focus on systems whose 
size can be ``upscaled'' in a physically natural way. 
Particularly simple examples are translationally
invariant Hamiltonians (see Sec.~\ref{s4})
with short-range interactions, i.e., the couplings
$J_{ij}$ in (\ref{3}) decay sufficiently fast 
(and independent of $\dof$) with increasing distance 
between the two sites $i$ and $j$.
Similarly, the initial states $\rho(0)$ must be chosen so
that they amount to ``physically similar situations''
for different system sizes $\dof$.
For example, the system energy $\tr\{\rho(0)H\}$
is often expected to grow linearly with the 
system size $\dof$,
i.e., the energy density (energy per degree of freedom) 
is kept constant.
Simple examples are canonical ensembles
of the form (\ref{23}), (\ref{24}) with fixed
parameters $\beta$ and $\vec h_i$
(independent of $\dof$ and $i$).

Rather than trying to formally define this class of
``extensive'' Hamiltonians $H$ and initial states 
$\rho(0)$ more precisely, we assume as
a ``minimal requirement'' that the concomitant
expectation values of ``intensive observables'',
such as the magnetization $M_a$ in
(\ref{29}), can be considered as asymptotically 
independent of the system size $\dof$,
and that their statistical fluctuations and/or quantum 
uncertainties, as exemplified by (\ref{52}),
decay to zero with increasing system size $\dof$
(usually as $1/\dof$).
Moreover, we assume that correlations between local
observables in the initial state $\rho(0)$,
such as
\begin{eqnarray}
c_{ij}^{ab}:=\langle s_i^a s_j^b \rangle_{\! 0}
- \langle s_i^a\rangle_{\! 0}\langle s_j^b \rangle_{\! 0}
\ ,
\label{54}
\end{eqnarray}
decay to zero with the distance between the sites $i$ and $j$
sufficiently fast and asymptotically independently
of the system size $\dof$.
Essentially, this is tantamount to the so-called
cluster decomposition property \cite{wic63,wei97,ess16,mur19,glu19},
and as such is commonly expected to be obeyed 
by any ``physically realistic'' $\rho(0)$ 
(at least outside the realm where phase transitions may occur),
though this property has until now only be rigorously 
established for a quite restricted set of examples
\cite{ara69,par82,par95,kli14,fro15} .

In particular, for systems that
possibly may exhibit large thermal fluctuations
(as a precursor of spontaneous symmetry breaking
in the thermodynamic limit), the 
%temperature $\beta^{-1}$
energy density must be chosen 
outside the range where such effects 
occur.
%beyond the corresponding critical values.
[The opposite situation will be further explored in Sec.~\ref{s6}.]

Given the initial magnetizations $\langle M_a \rangle_{\! 0}$
are asymptotically independent of the system size $\dof$,
the same follows for any later time $t$ 
according to (\ref{36})-(\ref{40}), and thus
for the late-time behavior of any single spin 
according to (\ref{41}).

In the same vein, the initial expectation values in (\ref{45})-(\ref{47})
are expected to be asymptotically independent of the system size
$\dof$ for physically realistic initial states $\rho(0)$, hence the same
applies to the time-dependent expectation values in (\ref{42})-(\ref{44}).
On the other hand, the initial variances $\sigma^2_{aa}$ 
%in (\ref{52})
(see (\ref{52}))
generically decay to zero for large $\dof$.
The same follows for the correlations $\sigma^2_{ab}$ 
in (\ref{52})
upon observing that
$[\sigma^2_{ab}]^2\leq \sigma^2_{aa}\sigma^2_{bb}$ 
(Cauchy-Schwarz inequality), and hence for the
variance of $M_x$ in (\ref{48}) (and similarly for $M_y$ and $M_z$).
Essentially, this reflects the common fact that quantum and statistical
fluctuations become negligible for macroscopic observables.
The main conclusion is that $\langle M_x^2\rangle_{\! t}$
can often be very well approximated by $\langle M_x\rangle^2_{\! t}$.

Finally it is reasonable to expect that a large-$\dof$ asymptotics
qualitatively similar to (\ref{41}) will also apply to local 
observables of the form $s_i^a s_j^b$. 
However, more rigorous and/or quantitative 
statements along these lines are difficult to obtain,
see also discussion below Eq.~(\ref{53}).
%[Heuristic estimates based on unproven conjectures 
%have been obtained in similar cases in Ref.~\cite{med20}).]

On the other hand, quantum and statistical
fluctuations of microscopic (local) observables 
are well-known to generically remain non-negligible.
Accordingly, initial correlations (at $t=0$), as exemplified by (\ref{54})
(with not too large distances between the sites $i$ and $j$),
are not expected to approach zero for large $\dof$, and likewise 
for the analogous correlations at any later time point $t$.
Numerical examples in support of this expectation are provided 
by Figs. \ref{fig2}-\ref{fig5}.

%%%%%%%%%%%%%%%%%%%%%%%%%%%%%%%%%%%%%%%%%%%%%%%%%%%%%%%
\subsection{Final remarks}
\label{s54}

Our first remark is that in case of the macroscopic
observables (\ref{29}), the exact time-dependencies 
(\ref{36})-(\ref{40})
can also be obtained ``directly'', i.e., without 
exploiting our main results from Sec.~\ref{s3},
and likewise for (\ref{42})-(\ref{47}).
Namely, by exploiting the specific symmetries of 
the Hamiltonian $H$ in (\ref{2}), the Heisenberg 
equations of motion
which govern the expectation values of 
those observables can be readily solved, as 
detailed, e.g., in Ref \cite{vor21}.
From this viewpoint, the absence of equilibration 
and thermalization in such models may thus be considered
as a relatively obvious consequence of their special
symmetry properties.

For most other observables, the generic occurrence of 
permanent long-time oscillations is a far from obvious
key finding of our present work.
The fact that this finding is indeed non-trivial is
already quite evident by recalling that usually 
an (approximately) time-periodic behavior only appears 
after sufficiently long times (see Figs. \ref{fig2}-\ref{fig5}), 
and even then the actual expectation values still 
exhibit certain deviations from strict periodicity
(for systems of finite size).
Moreover, the oscillations are asynchronous 
unless the system happens to translationally 
invariant (Sec.~\ref{s4}).

Our second remark is that ``single spin observables''
$s_i^a$ with $a\in\{x,y\}$ and their intensive 
counterparts $M_a$ from (\ref{29})
%(and sums thereof) 
were found in Sec.~\ref{s51}
to exhibit 
%(approximately) 
harmonic long-time oscillations 
with angular frequency $h$.
%[In some special cases the amplitude may happen 
%to vanish, as exemplified by (\ref{40}).]
In the same vein, ``two-spin observables''
$s_{i_1}^{a_1}s_{i_2}^{a_2}$ with $a_{1,2}\in\{x,y\}$
%\PR{and $s_{i_1}^{a_1}\not = s_{i_2}^{a_2}$}
were found to harmonically oscillate with angular 
frequency $2h$ in Sec.~\ref{s52},
while $\langle M_a^2\rangle_{\! t}$ turned out to 
be often close to $\langle M_a\rangle^2_{\! t}$ 
in Sec.~\ref{s53}.
Analogously, it is quite evident that harmonic 
oscillations with angular frequency $\nu h$ will 
arise for $\nu$-spin observables
$s_{i_1}^{a_1}\cdots s_{i_\nu}^{a_\nu}$,
%\PR{with pairwise different $s_{i_1}^{a_1},...,s_{i_\nu}^{a_\nu}$}, 
while
$\langle M_a^\nu\rangle_{\! t}$ will be close to
$\langle M_a\rangle^\nu_{\! t}$ in many cases.
The latter example implies that the long-time 
oscillations are in general {\em not} of a 
purely harmonic character.


%%%%%%%%%%%%%%%%%%%%%%%%%%%%%%%%%%%%%%%%%%%%%%%%%%%%%%%
\vspace*{1cm}
\section{Equilibrium correlations and time crystals}
\label{s6}

Throughout this section we restrict ourselves to
%We consider 
system states of the specific form
\begin{eqnarray}
\rho = \sum_{nl} p_{nl}\, |n,l\rangle\langle n,l |
%\ ,
\label{55}
\end{eqnarray}
%where the $p_{nl}$ are largely arbitrary ``statistical weights'' 
with $p_{nl}\geq 0$ and $\sum_{nl} p_{nl}=1$.
It follows from (\ref{7}) that $[H,\rho]=0$, i.e. the 
state $\rho$ remains unchanged in the course of time
(steady or equilibrium state).
Particularly important examples are 
thermal equilibrium 
%states 
ensembles
of the canonical form
\begin{eqnarray}
\rho = e^{-\beta H}\!/\tr\{e^{-\beta H}\}
\ .
\label{56}
\end{eqnarray}
Other examples are microcanonical ensembles, or, more generally,
largely arbitrary diagonal ensembles of low purity
(see also Eqs.~(\ref{22}), (\ref{26}), and below Eq.~(\ref{61})).

At the focus of the present section are temporal correlations  
(also called, among others, dynamic or two-point correlation functions)
%of the general form
\begin{eqnarray}
C_{\! AB}(t) :=\tr\{\rho A B(t)\}
\label{57}
\end{eqnarray}
for any given pair of observables $A$ and $B$,
where $B(t):=e^{iHt}Be^{-iHt}$ (Heisenberg picture, $\hbar=1$).

Similarly as in (\ref{9})-(\ref{13}) one finds that
\begin{eqnarray}
C_{\! AB}(t) 
& \!\! = \!\! &  
\sum_\nu g_{\nu}(t) \, e^{i\nu h t}
\ ,
\label{58}
\\
g_{\nu}(t) 
& \!\! := \!\! & 
\sum_{mn} e^{i(E_n^0-E_m^0)t}\sum_k  p_{mk}\, A_{mn}^{k,k+\nu} B_{nm}^{k+\nu,k}
\, , \ \ 
\label{59}
\end{eqnarray}
and similarly as in (\ref{14}), (\ref{15}), (\ref{21}) that
\begin{eqnarray}
C_{\! AB}(t) 
&  \rightsquigarrow &
\sum_\nu \bar g_{\nu} \, e^{i\nu h t}
\ ,
\label{60}
\\
\bar g_{\nu}
& := &
{\sum_{mnk}}' p_{mk}\, A_{mn}^{k,k+\nu} B_{nm}^{k+\nu,k}
\label{61}
\end{eqnarray}
under the very same preconditions as those discussed 
in Secs.~\ref{s32} and \ref{s33}.
The detailed derivation is quite similar to Appendix \ref{app1}
(see also Supplemental Material of Ref.~\cite{alh20}) 
and therefore omitted here.

As a consequence, the generic 
appearance 
%occurrence
of permanent oscillations is predicted similarly as
in Sec.~\ref{s34}, and of synchronization effects 
similarly as in Sec.~\ref{s4} in case of translationally 
invariant systems.
In particular, correlations of local observables
$A_i$ and $B_i$ with the property $\TT^\dagger\! A_i \TT=A_{i+1}$
and $\TT^\dagger\! B_i \TT=B_{i+1}$
are predicted to synchronize with each other,
and also with the correlations of their intensive
counterparts
$A:=\sum_{i\in \G } A_i/\dof$ and $B:=\sum_{i\in \G } B_i/\dof$,
respectively.

Note that the correlation in (\ref{57}) is, in general,
a complex valued function
of $t$, and as such not an immediately 
observable quantity. However, analogous predictions
readily carry over to its
%purely real, symmetrized counterparts
real (symmetrized) part
\begin{eqnarray}
C^s_{\! AB}(t) := 
%\frac{C_{\! AB}(t) +C^\ast_{\! AB}(t)}{2}=
[\tr\{\rho A B(t)\}+\tr\{\rho B(t) A\}]/2
\ ,
\label{62}
\end{eqnarray}
and analogously for its imaginary part.

Focusing on the specific observables $A=B=M_x$ from
(\ref{29}), one finally finds, similarly as in Sec.~\ref{s5}, 
%that
%that the symmetrized correlations
%\begin{eqnarray}
%\%tilde C_{xx}(t):=\tr\{\rho (M_x M_x(t) + M_x(t) M_x)\}/2
%\label{62}
%\end{eqnarray}
%can be written 
for arbitrary $t$ and without any further approximation 
%in the form
that
\begin{eqnarray}
C^s_{M_x\!M_x}(t) & = &
%\langle M_x^2\rangle_{\! 0} 
\tilde a_2
\, \cos(ht)
\ ,
\label{63}
\\
\tilde a_2 & := & \tr\{\rho M_x^2 \}
\ ,
\label{64}
\end{eqnarray}
and likewise for $A=B=M_y$.
In case of a translationally invariant system, 
we furthermore can conclude under similar conditions as 
above (\ref{41}) that
\begin{eqnarray}
C^s_{s_i^x\! s_i^x}(t)
\rightsquigarrow
C^s_{M_x\!M_x}(t) 
%\tilde a_2 \, \cos(ht)
%=\frac{a}{f}\, \cos (2ht) + \frac{b}{f} \sin(2ht)
\ .
\label{65}
\end{eqnarray}

These findings imply interesting conclusions
with respect to the topic of time crystals.
At the focus of the latter issue are, generally speaking, 
various conceivable forms and disguises 
of a spontaneously broken time-translation symmetry
(see, e.g., Refs.~\cite{han22,ven19} for recent reviews).
Here, we specifically address the possible occurrence 
of such fascinating phenomena 
%in their arguably most basic form, namely 
in isolated many-body quantum systems 
{\em at thermal equilibrium}
(no periodic driving, no external bath(s) or other
sources of dissipation, not considering 
ground states (zero temperature limit),
not taking the thermodynamic 
limit before the long-time limit 
etc. \cite{han22,ven19}).
Under theses circumstances, a particularly well-established 
definition of a time crystal explicitly refers to the behavior of
temporal correlations at thermal equilibrium, 
requiring that they must exhibit permanent oscillations 
in time as well as long-range order in space \cite{wat15}.
In our present context, this is largely equivalent \cite{wat15,wat20,hua19}
to the requirement that there must exist intensive observables
$A,B$ (exemplified by (\ref{29}) and more generally 
defined below Eq.~(\ref{61})),
whose correlation function in (\ref{57})
exhibits permanent oscillations that do not
tend to zero for asymptotically large system
size $\dof$ (see below (\ref{54})).

Combining this definition and Eq.~(\ref{63}), a time crystal will thus be 
realized by focusing on the example $A=B=M_x$ and showing
that $\tilde a_2$ in (\ref{64}) approaches a positive limiting value
for asymptotically large $\dof$ in the canonical ensemble from
(\ref{56}).
%(see below (\ref{54})) then the same applies to
%the temporal correlations in (\ref{63}).
%If this is furthermore the case for canonical system states
%(\ref{56}) then the behavior in (\ref{63}) amounts
%to a so-called time-crystal according to the
%commonly accepted definition from Ref.~\cite{wat15}.
Observing (\ref{64}) and that $\tr\{\rho M_x\}=0$
such a behavior of $\tilde a_2$ is tantamount
to the appearance of macroscopic thermal fluctuations
of $M_x$ and is thus expected to 
%occur 
arise
if the Heisenberg model in (\ref{1})-(\ref{3}) exhibits 
in the thermodynamic limit a spontaneous symmetry 
breaking (phase transition) with respect to $M_x$.
In this context it may be worth to recall that, 
as always, we tacitly focus on cases with a
%{\em in the presence of a 
non-vanishing external field $h$ in (\ref{2}).

Remarkably, we thus established a direct
connection between a spontaneously broken time-translation
invariance in the context of time crystals, and a
spontaneously broken spatial symmetry in the context
of phase transitions at thermal equilibrium.

As demonstrated analytically in Refs.~\cite{wat15,wat20,hua19},
this kind of time crystals is in fact impossible, at least for all 
many-body systems with short-range interactions. Accordingly, 
also the above-mentioned phase transition can be ruled out.

%Whether or not such phase transitions exist is not obvious.
An alternative, weaker definition of a time crystal has recently been
proposed and explored in Ref.~\cite{med20}, requiring that the ratio
between the temporal correlation in (\ref{57}) an its initial value
$C_{AB}(0)$ must exhibit permanent long-time oscillations.
According to (\ref{63}), this condition is always fulfilled for the
specific choice $A=B=M_x$.
In other words, according to this definition, a time crystal
is expected to generically arise for any model of the general 
form (\ref{1})-(\ref{3}) with non-vanishing field $h$.
Similarly to the discussion at the end of Sec.~\ref{s34},
our present findings thus complement
and substantially extend those obtained in 
the seminal previous Ref.~\cite{med20}.


%\PR{Spin-flip ($Z_2$) symmetry considerations as an appendix?}


%%%%%%%%%%%%%%%%%%%%%%%%%%%%%%%%%%%%%%%%%%%%%%%%%%%%%%%
\section{Summary and Conclusions}
\label{s7}

Our first main prediction (see Sec.~\ref{s3}) is that 
any Heisenberg model of the general 
form (\ref{1})-(\ref{3}) gives rise to 
time-dependent expectation values
%(\ref{12}),
(\ref{8a}), which become
%which are
practically indistinguishable from the 
auxiliary function (\ref{25}) for 
practically all sufficiently large times $t$.
%and system sizes.
The very weak
%sufficient 
preconditions for this prediction are that 
the system size $\dof$ must be large, the maximal 
gap degeneracy $\gap$ must not be exceedingly large
(see below Eq.~\eqref{20}), 
and the maximal level population $\pmax$ must 
be small (see Eq.~(\ref{20})). 
For instance, the latter condition is known to be
%generically 
fulfilled if the initial state arises as 
the result of a canonical quench (see Sec.~\ref{s33}).

In turn, this auxiliary function (\ref{25})
generically exhibits time-periodic 
(but not necessarily harmonic)
oscillations,
hence the same must be (approximately) the case
for the long-time behavior of the corresponding
expectation values in (\ref{8a}),
as exemplified by Figs.~\ref{fig2}-\ref{fig5}.
The main requirements for such permanent  
long-time oscillations are a non-vanishing magnetic field 
$h$ in (\ref{2}), and a non-equilibrium initial condition.
More precisely speaking, the initial state must not be
a diagonal ensemble of the form (\ref{26}) or (\ref{59}).
In all these cases, the system thus exhibits
neither equilibration nor thermalization.

We remark that the considered models 
(\ref{1})-(\ref{3}) themselves are not
subject to any time-dependent external driving.
Moreover, all the above findings are
%, in particular, 
independent of whether the system is integrable or not,
features disorder (and possibly many-body localization) 
or not, is extensive (short-range interactions) or not,
nor does the dimensionality of the system play any 
significant role.

Put differently, approximately periodic long-time 
oscillations are predicted to occur for (almost)
any observable.
%generically (for the vast majority of observables and 
%initial conditions).
Moreover, during some initial time interval,
the expectation values are generically far
for from being periodic, and exhibit some small 
deviations from strict periodicity even for large times.
Finally, those oscillations are in general 
not of a purely harmonic character, including
as special cases oscillations with arbitrary 
multiples of the reference frequency 
$h$ 
%(magnetic field, 
(cf. Eq.~(\ref{25})).
Accordingly (see also Sec.~\ref{s54}), for such 
observables we are unable to complement 
our analytical theory by some simple ``physical 
explanation'' of what is essentially going on.
%\PR{In many instances, the angular frequency of those 
%long-time oscillations is given by the inverse magnetic  field $h^{-1}$. 
%(in natural units).
%However, we also unravelled in Figs.~\ref{fig2}-\ref{fig5}
%and Sec.~\ref{s5} many interesting examples 
%with angular frequency $2h^{-1}$ (higher harmonics).
%Given that these examples essentially consist of products 
%of two spins (and sums thereof), it is reasonable to expect
%that analogous oscillations with angular frequency $nh^{-1}$
%will arise for observables consisting of some suitable
%products of $n$ spins.}

Another challenging open problem is to explain 
all observable properties for a finite magnetic field 
$h$ in (\ref{2}) in terms of the field-free properties.
More precisely speaking, the eigenvalues and eigenvectors
are of course trivially related via (\ref{7}), (\ref{8}), 
but does the behavior of all (physically relevant)
%expectation values 
observables for $h=0$ 
already determine their behavior for $h\not =0$\,?

Our second main result (see Sec.~\ref{s4})
is the prediction of synchronization under the 
additional requirement that the 
system is translationally invariant (and thus obeys 
periodic boundary conditions) in all spatial directions.
Here, the term synchronization means that the above discussed
long-time oscillations become approximately invariant under arbitrary 
translations of any given observable,
as exemplified by Figs.~\ref{fig4} and \ref{fig5}.
Once again, this approximate invariance is furthermore predicted 
to become asymptotically exact for large times and large 
system sizes.
Even more generally speaking (without any reference 
to some underlying lattice geometry), it seems in fact
sufficient to require that all spins of the considered
model are equivalent (in some suitably defined sense), 
and likewise for the synchronizing observables.
%\JS{Our second main prediction is that for models consisting 
%of equivalent spins as e.g.\ in translationally invariant
%lattices the expectation values of operators that are
%related by the respective symmetry operations as e.g.\ translations
%show one and the same time-dependent expectation values 
%for practically all sufficiently late times. This gives
%rise to perfect synchronization as demonstrated 
%in sec.~\xref{s4}.}

We emphasize that our present synchronization 
phenomenon does not depend
on whether the interactions $J_{ij}$ in (\ref{3}) 
are negative (i.e. of ferromagnetic character)
or not \cite{vor21}, contrary to what one might have naively 
expected to be necessary for the ``alignment'' of 
all the spins in such a system.
In the same vein, the system's dimensionality once again plays no role,
nor is it necessary that the initial condition is translationally
invariant.
%\JS{In view of possible misconceptions, we would like to stress that 
%the observed phenomena, in particular the synchronization, has nothing to 
%do with ferromagnetic interaction; on the contrary it does not at all depend 
%on the sign of the exchange interactions, see also \cite{vor21}.}
More generally speaking, ordering and phase transition phenomena
at thermal equilibrium are apparently of little help to better 
understand our present synchronization effects,
nor are we able to provide any other kind of simple 
intuitive explanation of the basic underlying physics.

Obviously, 
the above predicted long-time oscillations
of any given observable $A$
%obviously 
in general still depend in a very complicated 
way (via the phases and amplitudes in (\ref{25})) 
on the choice of the initial condition $\rho(0)$.
However, for translationally invariant 
Hamiltonians those long-time oscillations were 
shown in Sec. \ref{s4} to be invariant under 
arbitrary translations of the initial 
condition $\rho(0)$ (even if $\rho(0)$ 
itself is not translationally invariant).
This quite remarkable finding is in fact 
equivalent to the prediction of synchronization,
and therefore seems again not to admit a 
simple physical explanation.

Our third main result concerns the issue of time crystals.
Unfortunately, even the precise definition of 
%what is 
a time crystal still appears to be somewhat ambiguous.
For instance, already our permanent oscillations 
from Sec.~\ref{s3} can be considered as the characteristic
signature of a time crystal according to one of the definitions
provided in Ref.~\cite{ven19} (see Figure 8, second column,
last row therein):
Indeed, since the time-translation invariance of the model 
Hamiltonian is spontaneously broken (reduced to a 
time-discrete invariance) for arbitrarily long times, 
which in turn may be viewed as a thermodynamic limit in the 
time domain, we may speak of a ``crystal'' in the time 
domain.
In our present explorations in Sec.~\ref{s6}, we mainly
focused on the somewhat more generally established definition
of a time crystal from Ref.~\cite{wat15}.
We also may recall that the no-go theorem for this type of 
time  crystals from Ref.~\cite{wat15} 
has been shown in Ref.~\cite{ven19}
to still contain a loophole, which in turn 
has been subsequently closed in \cite{wat20} 
(see also \cite{hua19}).
Our present explorations are (or course) compatible
with this latter no-go theorem, i.e., we do not
find a time crystal in the sense of Ref.~\cite{wat15}.
Finally, yet another, somewhat weaker definition of a
time crystal has been proposed in Ref.~\cite{med20},
according to which our findings in Sec.~\ref{s6}
lead to the conclusion that models of the general
form (\ref{1})-(\ref{3}) generically do exhibit the
characteristic signature of a time crystal.
The question of what 
%additional physical insight
we actually gained by knowing whether or not
some given model system qualifies as a time crystal 
in one or the other 
sense remains unclear to the present authors.

Finally, it seems reasonable to expect that our 
main findings will also be recovered in a broad 
class of alternative models such as the Hubbard model,
as long as their general symmetry properties 
are similar as in our present model.
%as long as they possess SU(2) symmetry.}

%\JS{Whether or not one wants to term our observed and rationalized 
%persistent oscillations a \emph{time crystal} is a matter of taste. 
%The authors of [Roderich] offer a rather wide definition where we see 
%our findings included in Table (Fig.~?), second column, third row. \dots}

%%%%%%%%%%%%%%%%%%%%%%%%%%%%%%%%%%%%%%%%%%%%%%%%%%%%
\begin{acknowledgments}

This work was funded by the Deutsche Forschungsgemeinschaft (DFG,
German Research Foundation) -- 
355031190 (FOR~2692), 397303734, and 397300368.
\end{acknowledgments}
%%%%%%%%%%%%%%%%%%%%%%%%%%%%%%%%%%%%%%%%%%%%%%%%%%%%



%%%%%%%%%%%%%%%%%%%%%%%%%%%%%%%%%%%%%%%%%%%%%
%
\appendix
\section{Derivation 
%and generalization 
of Eq.~(\ref{19})}
\label{app1}

As usual, the unperturbed energies are denoted 
by $E_n^0$ with $n\in\{1,...,N\}$ (see below (\ref{6})),
%The $L_n$ are integers or half-integers, whose detailed dependence on $n$ 
%For the rest, the detailed dependence of $L_n$ on $n$
%is quite complicated.
%Yet, upon
and the operator norm (largest eigenvalue in modulus) 
of any Hermitian operator $A$ 
is denoted by $\norm{A}$.

%Observing (\ref{5}), $| l |\leq L_n$ (see below (\ref{6})), and
Choosing $l = L_n$ in (\ref{5}), and exploiting that
$\norm{S^z}\leq \sum_{i\in \G }\norm{s_i^z}=\dof s$,
where $s$ is the 
%above defined 
single-site spin quantum number and $\dof$ the system size
(see above (\ref{1})),
we 
%still can conclude in full generality that
%obtain
can conclude that
\begin{eqnarray}
L_n\leq \dof s
\label{a1}
\end{eqnarray}
for any $n\in\{1,...,N\}$.

%Recalling 
Given that a single spin at any given site $i$ 
spans a Hilbert space of 
dimension $2s\! +\! 1 $, the dimensionality of the full Hilbert 
space will be $(2s\! +\! 1 )^{\dof}$.
Hence, 
%the number $N$ of different possible indices $n$ 
the total number $N$ of all energy eigenvalues
$E_n^0$ can be {\em upper bounded} by $(2s\! +\! 1 )^{\dof}$,
\begin{eqnarray}
N\leq (2s + 1 )^{\dof}
\ .
\label{a2}
\end{eqnarray}
Conversely, for any given $n$, the total number 
$2L_n\! +\! 1$ of all possible labels $l$ 
(see below (\ref{6}))
%can be upper bounded as 
is upper bounded by $2\dof s\! +\! 1 $ according to (\ref{a1}).
We thus obtain the {\em  lower bound}
%the number $N$ of different possible indices $n$ 
%(see below (\ref{5})) 
%can be lower bounded as
\begin{eqnarray}
N\geq\frac{(2s + 1 )^{\dof}}{2\dof s + 1 }
\ .
\label{a3}
\end{eqnarray}
Altogether, (\ref{a2}) and (\ref{a3}) imply
that the number $N$ of energy eigenvalues 
$E_n^0$ must grow exponentially with the system size $\dof$
% where $N$ is exponentially large in the system size $\dof$ 
%but finite (see above (\ref{a3})).

%Accordingly, 
The set of all possible (ordered)
pairs of indices $m$ and $n$ is defined as
\begin{equation}
{\cal G}_{\rm tot}:=\bigl\{(m,n)\, |\, m ,n\in \{1,\dots,N\} \bigr\}
\ .
\label{a4}
\end{equation}
For any given pair $\alpha=(m,n)\in{\cal G}_{\rm tot}$
%with $\alpha=(m,n)$, 
we furthermore define
\begin{eqnarray}
G_{\!\alpha} & := & E_n^0-E_m^0
\ ,
\label{a5}
\\
\eta_{\alpha}^\nu
& := & 
\sum_k  \rho_{mn}^{k,k+\nu} A_{nm}^{k+\nu,k}
\ .
\label{a6}
\end{eqnarray}
Hence, (\ref{13}) can be rewritten as
\begin{eqnarray}
f_{\nu}(t) & := & \sum_{\alpha\in{\cal G}_{\rm tot}} e^{iG_{\!\alpha} t} \, \eta_{\alpha}^\nu
\ .
\label{a7}
\end{eqnarray}
%where the sum is understood to run over all $\alpha\in{\cal G}_{\rm tot}$

Next, we introduce the subset ${\cal G}\subset {\cal G}_{\rm tot}$ of all
pairs $(m,n)$ with the property that $E_m^0\not=E_n^0$,
i.e.,
\begin{equation}
{\cal G}  :=  \bigl\{\alpha \in {\cal G}_{\rm tot}\, |\, G_{\!\alpha}\not=0 \bigr\}
\ .
\label{a8}
\end{equation}
%and we denote its complement by
Accordingly, its complement satisfies
\begin{equation}
{\bar{\cal G}} :=  {\cal G}_{\rm tot}\!\setminus 
{\cal G} = \bigl\{\alpha \in {\cal G}_{\rm tot}\, |\, G_{\!\alpha}=0 \bigr\}
\ .
\label{a9}
\end{equation}

It readily follows that the maximal gap degeneracy from
(\ref{18}) can be rewritten in the form
\begin{eqnarray}
\gap & = & \max_{\beta\in{\cal G}} \left|\{ \alpha\in{\cal G} | \, G_\alpha=G_\beta\}\right|
\ ,
\label{a10}
\end{eqnarray}
where $|S|$ denotes the number of elements contained in the set $S$.
Similarly, the long-time average of $f_{\nu}(t)$ from (\ref{13}) or (\ref{a7})
can be rewritten in  the form (\ref{15}) or 
\begin{eqnarray}
\bar f_{\nu} = \sum_{\alpha\in {\bar{\cal G}}} \eta_\alpha^\nu
\ ,
\label{a11}
\end{eqnarray}
respectively.

%Let us assume that there exists a pair 
%$(m,n) \in {\cal G}_{\rm tot}$
%with the properties $E_m^0-E_n^0=0$ and $m\not = n$.
%Choosing $m'=n$ and $n'=m$ then implies that 
%%the relation (\ref{16}) is fulfilled for a pair $(m',n')$ which 
%%does {\em not} obey the requirement below (\ref{16}).
%the condition (\ref{16}) from the main text is violated.
%In turn, taking for granted condition (\ref{16}) thus implies that:
%(i) The unperturbed energies $E_n^0$
%do not exhibit degeneracies.
%(ii) The set ${\bar{\cal G}}$ from (\ref{a9})
%only contains pairs $\alpha=(m,n)$ with the
%property $m=n$.
%(iii) The quantity  $\bar f_{\nu}$ from (\ref{a11}) 
%can be rewritten in the form (\ref{15}).

%Importantly, in what follows we do {\em not\,} take for granted
%the condition (\ref{16}) from the main text, and we will 
%work with the corresponding non-trivial generalization 
%(\ref{a11}) of (\ref{15}).

As announced in the main text, our objective is 
to show  that the difference 
\begin{eqnarray}
\Delta(t) := \At - \Att
\label{a12}
\end{eqnarray}
between the time-dependent expectation values from
(\ref{12}) and the auxiliary function from (\ref{14})
is small for most sufficiently late times $t$.
Employing (\ref{12}), (\ref{14}), (\ref{a7}), (\ref{a9}), 
and (\ref{a11}), we therefore rewrite (\ref{a12}) as
\begin{eqnarray}
\Delta(t) & = & \sum_{\nu} \delta_{\nu}(t)\, e^{i\nu ht}
\ ,
\label{a13}
\\
\delta_{\nu}(t) & := & f_{\nu}(t)-\bar f_{\nu}=
\sum_{\alpha\in{\cal G}} e^{iG_{\!\alpha} t} \, \eta_{\alpha}^\nu
\ .
\label{a14}
\end{eqnarray}
Next we recall that the sum over the indices $m,n,k,l$ in (\ref{9}) 
is tacitly restricted to pairs $n,l$ for which $|n,l\rangle$ are 
well-defined eigenvectors in (\ref{7}), i.e. $n\in\{1,...,N\}$ and 
$l\in\{-L_n,...,L_n\}$,
%(see below (\ref{5})), 
and likewise for the pairs $m,k$.
Alternatively, for indices $n,l$ so that $|n,l\rangle$ is not a well-defined 
eigenvector, we may define those (so far undefined) vectors $|n,l\rangle$
as being equal to the null vector (hence $\rho_{mn}^{k,l}=0$, $A_{nm}^{l,k}=0$).
As a consequence, we may now consider all four indices
$m,n,k,l$ in the sum in (\ref{9}) to run over all integer values,
and likewise for the summation indices in (\ref{13}), (\ref{15}), 
and (\ref{a6}).
Furthermore, it follows that
%As explained below (\ref{11}), 
the matrix elements $\rho_{mn}^{k,k+\nu}$ are zero if
%may only be non-zero under the 
%(necessary but not sufficient) conditions that 
%$k\in\{-L_m,...,L_m\}$ and simultaneously $k+\nu\in\{-L_n,...,L_n\}$.
$k\not\in\{-L_m,...,L_m\}$ or $k+\nu\not\in\{-L_n,...,L_n\}$.
Hence, it is sufficient to keep 
on the right-hand side in (\ref{a6}) 
only those 
%$k$-values 
summands
which satisfy $|k|\leq L_m$ 
and $|\nu+k|\leq L_n$.
Observing (\ref{a1}) and $|\nu+k|\geq |\nu|-|k|$
(triangle inequality) it follows that 
%$\dofs\geq |\nu|-|k|\geq |\nu|-\dof s$ 
$|\nu|-\dof s\leq  |\nu|-|k| \leq |\nu+k|\leq \dof s$ 
must be fulfilled.
As a consequence, it is necessary that $|\nu|\leq 2\dof s$
in order that $\eta_\alpha^\nu$ in (\ref{a6})
is non-zero.
Therefore, it is sufficient to keep in (\ref{a13})
only those $\nu$ which are contained in 
$I_\nu:=\{-2\dof s,...,2\dof s\}$, and by employing the 
Cauchy-Schwarz inequality we obtain
\begin{eqnarray}
|\Delta(t)|^2 \leq \sum_{\nu\in I_\nu} |\delta_{\nu}(t)|^2
\sum_{\nu\in I_\nu} |e^{i\nu ht}|^2
\ .
\label{a15}
\end{eqnarray}
The last sum can be identified with $4\dof s\! +\! 1 $,
yielding
\begin{eqnarray}
|\Delta(t)|^2 & \leq & (4\dof s\! +\! 1 )\,  \sum_\nu |\delta_{\nu}(t)|^2
\ ,
\label{a16}
\end{eqnarray}
where, without loss of generality, the sum has 
again been extended to all integer indices $\nu$.

Denoting, as in the main text, 
the temporal average of an arbitrary function $f(t)$ over
the time interval $[0,T]$ by 
\begin{eqnarray}
\left\langle f(t)\right\rangle_{\! T}:=\frac{1}{T}\int_0^T \!\!dt\, f(t)
\ ,
\label{a17}
\end{eqnarray}
%Considering $\nu$ as arbitrary but fixed, and temporarily
%omitting the index $\nu$ in $\eta_{\alpha}^\nu$,
we can conclude from (\ref{a14}) that
\begin{eqnarray}
\left\langle |\delta_\nu(t)|^2 \right\rangle_{\! T}  & = &  \sum_{\alpha,\beta\in{\cal G}} (\eta^\nu_\alpha)^\ast
M^{\alpha\beta}_T \eta^\nu_{\beta}
\ ,
\label{a18}
\\
M^{\alpha\beta}_T & := & 
\left\langle e^{-i(G_\alpha-G_\beta)t}\right\rangle_{\! T} 
\ .
\label{a19}
\end{eqnarray}
Viewing $M^{\alpha\beta}_T$ as the matrix elements of some operator $M_T$,
one can infer from (\ref{a19}) that $M_T$ is Hermitian and non-negative,
and therefore
\begin{eqnarray}
\sum_{\alpha,\beta\in{\cal G}} (\eta^\nu_\alpha)^\ast M^{\alpha\beta}_T \eta^\nu_{\beta}
\leq
\norm{M_T}\sum_{\alpha\in{\cal G}} |\eta^\nu_\alpha|^2 
\ .
\label{a20}
\end{eqnarray}
%where $\norm{M_T}$ is the operator norm (largest eigenvalue in modulus) of $M_T$.
As detailed, e.g., in Refs.~\cite{sho12,bal16},
one can furthermore show that
\begin{eqnarray}
\norm{M_T} & \leq & 2\gap
\label{a21}
%\\
%\gap & := & \max_{\beta\in{\cal G}} \left|\{ \alpha\in{\cal G} | \, G_\alpha=G_\beta\}\right|
%\ ,
%\label{a10}
\end{eqnarray}
 for all sufficiently large $T$, where $\gap$ is given in (\ref{a10}).
%where $|S|$ denotes the number of elements contained in the set $S$.

Altogether, (\ref{a16}), (\ref{a18}), (\ref{a20}), and (\ref{a21}) thus imply
\begin{eqnarray}
\left\langle |\Delta(t)|^2 \right\rangle_{\! T} & \leq &2g\, (4\dof s\! +\! 1 )\, \sigma^2
\ ,
\label{a22}
\\
\sigma^2 & := & \sum_\nu \sum_{\alpha\in{\cal G}} |\eta^\nu_\alpha|^2 
\ ,
\label{a23}
\end{eqnarray}
for all sufficiently large $T$.
Extending the sum in (\ref{a23}) over all index pairs $\alpha\in {\cal G}_{\rm tot}$
and exploiting (\ref{a6}), we find
\begin{eqnarray}
\sigma^2 & \leq &  \sum_\nu \sum_{mn} \sum_{kl} 
\rho_{mn}^{k,k+\nu} A_{nm}^{k+\nu,k}
(\rho_{mn}^{l,l+\nu} A_{nm}^{l+\nu,l})^\ast
\ ,
\nonumber
\\
& = & 
\sum_{\nu mn} Q_{\nu mn}
\ ,
\label{a24}
\\
Q_{\nu mn} & := & \sum_{kl} V_{\nu mn}^{k,l} (V_{\nu mn}^{l,k})^\ast 
\ ,
\label{a25}
\\
V_{\nu mn}^{k,l} & := & \rho_{mn}^{k,k+\nu} (A_{nm}^{l+\nu,l})^\ast
\ .
\label{a26}
\end{eqnarray}
Utilizing the Cauchy-Schwarz inequality in (\ref{a25}) implies
\begin{eqnarray}
|Q_{\nu mn}|^2\leq \sum_{kl} |V_{\nu mn}^{k,l}|^2 \sum_{kl} |V_{\nu mn}^{l,k}|^2
\ .
\label{a27}
\end{eqnarray}
Observing that the two sums on the right-hand side are in fact identical,
we can infer with (\ref{a24}) that
\begin{eqnarray}
\sigma^2 \leq \sum_{\nu mn} |Q_{\nu mn}| \leq \sum_{\nu mnkl} |V_{\nu mn}^{k,l}|^2 
\label{a28}
\end{eqnarray}
and with (\ref{a26}) that
\begin{eqnarray}
\sigma^2 \leq  \sum_{\nu mnkl} |\rho_{mn}^{k,k+\nu}|^2\, |A_{nm}^{l+\nu,l}|^2
\ .
\label{a29}
\end{eqnarray}

Exploiting (\ref{10}) and the Cauchy-Schwarz inequality,
one can conclude that 
$|\rho_{mn}^{k,l}|^2\leq \rho_{mm}^{k,k}\rho_{nn}^{l,l}$.
Since the density operator $\rho(0)$ must be semi-positive,
it follows with (\ref{10}) that $\rho_{mm}^{k,k}$ and 
$\rho_{nn}^{l,l}$ are non-negative, real numbers.
Altogether, $|\rho_{mn}^{k,k+\nu}|^2$ in (\ref{a29}) can thus be upper bounded by
$\rho_{mm}^{k,k}\pmax$, where $\pmax$ is defined in (\ref{17}),
yielding
\begin{eqnarray}
\sigma^2 & \leq & \pmax  \sum_{mk} \rho_{mm}^{k,k}\, W_{m}
\ ,
\label{a30}
\\
W_{m} & := & \sum_l w_{ml}
\ ,
\label{a31}
\\
w_{ml} & := & \sum_{n\nu} |A_{nm}^{l+\nu,l}|^2
\ .
\label{a32}
\end{eqnarray}
Replacing in (\ref{a32}) the summation index $\nu$ by $j:=l+\nu$ and exploiting
(\ref{11}) thus yields
\begin{eqnarray}
w_{ml}  =  \sum_{nj} |A_{nm}^{j,l}|^2 
=  \sum_{nj}\langle m,l | A | n,j \rangle \langle n,j |A| m,l \rangle
\, . \ \ \ 
\label{a33}
\end{eqnarray}
Since $\sum_{nj} | n,j\rangle \langle n,j|$ is the unit operator, we 
see that $w_{ml}$ equals $\langle m,l | A ^2 | m,l\rangle$
and thus 
\begin{eqnarray}
W_{m} = \sum_l \langle m,l | A ^2 | m,l\rangle
\ .
\label{a34}
\end{eqnarray}
As discussed below (\ref{a14}), the summands on the right-hand 
side of (\ref{a34})
are zero for $l\not\in\{-L_m,...,L_m\}$.
In other words, there are at most $2L_m\!+\!1$ 
non-vanishing summands. Furthermore, each of those
summands can be upper bounded by $\norm{A^2}=\norm{A}^2$.
Due to (\ref{a1}) we thus arrive at
\begin{eqnarray}
W_{m} \leq (2\dof s\! +\! 1) \,\norm{A}^2
\ .
\label{a35}
\end{eqnarray}
Observing (\ref{10}), the remaining sum in (\ref{a30}) can be 
identified with $\tr\{\rho(0)\}=1$, yielding
\begin{eqnarray}
\sigma^2 & \leq &   (2\dof s\! +\! 1 ) \,\norm{A}^2\,\pmax
\ .
\label{a36}
\end{eqnarray}
Together with (\ref{a22}), we finally can conclude that
\begin{eqnarray}
\left\langle |\Delta(t)|^2 \right\rangle_{\! T} & \leq &4\gap\, (2\dof s\! +\! 1 )^2\, \norm{A}^2\, \pmax
\label{a37}
\end{eqnarray}
for all sufficiently large $T$.

Note that  if we replace $A$ by $A+c$ then both terms on the right-hand side
in (\ref{a12}) are shifted by the same constant $c$, thus the
left-hand side is independent of $c$.
Accordingly, 
the left-hand side in (\ref{a37}) is independent of $c$, 
while the right-hand side yields in general a different
upper bound for different choices of $c$.
Denoting by $\amax$ and $\amin$ the largest and smallest eigenvalues of $A$,
respectively, one finds that the tightest upper bound is achieved 
for the choice $c=-(\amax+\amin)/2$.
Altogether, (\ref{a37}) and (\ref{a12}) thus yield 
%as our final result
\begin{eqnarray}
\left\langle [\At - \Att ]^2 \right\rangle_{T} 
& \leq &
\gap\, (2\dof s\! +\! 1 )^2\, \Da^2\, \pmax
\ \ \ 
\label{a38}
\end{eqnarray}
for all sufficiently large $T$,
where $\Da:=\amax-\amin$ is the measurement range of $A$
(difference between largest and smallest possible measurement 
outcomes).
In other words, we recover Eq.~(\ref{19}).

%Recalling that the condition (\ref{16}) implies $g=1$ 
%(see also below (\ref{a11}) and below (\ref{a10})),
%one readily recovers 
%%from (\ref{a12}) and (\ref{a38}) 
%the relation (\ref{19}) from the main text.

%%%%%%%%%%%%%%%%%%%%%%%%%%%%%%%%%%%%%%%%%%%%%
%
\section{Derivation of Eq.~(\ref{28})}
%\section{Translation invariance}
%\section{Properties of the translation operator}
\label{app2}

We focus on spin models (\ref{2})
%with sites 
on a {\em one-dimensional} lattice $\G =\{1,...,\dof \}$ 
with { periodic boundary conditions}. 
[Generalizations to hypercubic lattices in arbitrary 
dimensions are straightforward.]
In other words, we are dealing with
$\dof$ { identical} ``units'' (spins) on a ring (chain with 
periodic boundary conditions) which are 
labeled by $i\in \{1,...,\dof \}$.

In the absence of interactions, each unit ``lives'' on 
%some $K$-dimensional
a
Hilbert space $\hr_i$ with orthonormal basis $|k\rangle_i$, 
%where $k=1,...,K$ and $K:=2s\!+\!1$.
where $k=1,...,2s\!+\!1$.
%$k=-s,...,s$.
Apart from ``belonging'' to different units $i$, all those
Hilbert spaces are identical copies of each other.

The pertinent Hilbert space $\hr$ of the total system is 
the tensor product of all the $\hr_i$.
Abbreviating $\dof$-tuples $(k_1,...,k_{\dof} )$ as
$\vec k$, the vectors
$|\vec k\rangle:=|k_1\rangle_1\cdots |k_{\dof} \rangle_{\dof} $
then amount to an orthonormal basis  of $\hr$.
%An arbitrary vector $|\psi\rangle\in\hr$ can thus be written
%as $\sum c_{\vec k} |\vec k\rangle$, where the sum
%runs over all possible $\vec k$'s and
%$c_{\vec k}:=\langle\vec k|\psi\rangle$.

%Next we define a ``shift'' or ``translation'' operator $\TT$ on $\hr$ 
%according to $\TT |\vec k\rangle :=|k_2\rangle_1|k_3\rangle_2\cdots |k_f\rangle_{\!f-1}|k_1\rangle_{\!f}$.
%Furthermore, $\TT$ is required to be a linear operator,
%i.e., for an arbitrary $|\psi\rangle=\sum c_{\vec k} |\vec k\rangle$
%we have $\TT |\psi\rangle=\sum c_{\vec k} \TT |\vec k\rangle$.
%One readily concludes that 
%%$\TT ^f=\id$ (identity on $\hr$) and 
%$\TT ^\dagger=\TT ^{-1}$ (unitary operator).

Next, a ``shift'' or ``translation'' operator $\TT:\hr\to\hr$
is defined via its action on any basis vector: 
$\TT |\vec k\rangle :=|k_2\rangle_1|k_3\rangle_2\cdots |k_{\dof} \rangle_{\!\dof -1}|k_1\rangle_{\!\dof }$.
One readily concludes that $\TT$ is norm-preserving.
It follows that $\TT$ must be a unitary operator, i.e.,
 $\TT ^\dagger=\TT ^{-1}$.

Our main assumption is that the unperturbed Hamiltonian
$H_0$ from (\ref{3}) is {\em translationally invariant} in the
sense that the couplings $J_{ij}$ do not depend separately
on $i$ and $j$, but only on the 
%natural distance (accounting for the periodic boundary conditions) between the sites $i$ and $j$.
difference $i-j$ (modulo $\dof$).
It follows that $H_0$ is also translationally invariant
in the alternative sense that 
\begin{eqnarray}
\TT^\dagger\! H_0\TT=H_0
\ ,
\label{b1}
\end{eqnarray}
or, equivalently, $[H_0,\TT]=0$ (commutator).
Likewise, one sees that each component of the
total spin $S^a$ from (\ref{1}) is translationally 
invariant.
It follows that all four operators 
$H_0$, $S^z$ , $\vec S^2$, and $\TT$
commute with each other. Without loss of generality,
we thus can 
%and will 
assume that the eigenvectors $|n,l\rangle$
of $H_0$ are at the same time not only 
eigenvectors of $S_z$, and $\vec S^2$,
see (\ref{4})-(\ref{6}), but also eigenvectors of $\TT$.
Since $\TT$ is unitary, the corresponding eigenvalues
must be of unit modulus, i.e.,
\begin{eqnarray}
\TT|n,l\rangle = e^{i \theta_{n,l}}|n,l\rangle
\label{b2}
\end{eqnarray}
with certain ``phases'' $\theta_{n,l}\in[0,2\pi)$.
%[As in the main text, $n\in\{1,...,N\}$ and 
%$l\in\{-L_n,...,L_n\}$ is tacitly understood.]

%The salient point is that the phases $\theta_{n,l}$
%actually only depend on $n$, but {\em not} on $l$.

%%Essentially, the reason is as follow: 
%Defining as usual $S^+:=S^x+iS^y$ (raising operator), 
%one recovers the common relation $S^+|n,l\rangle = c_{n,l} |n,l+1\rangle$, 
%where $c_{n,l}$ is a normalization constant.
%%and similarly for the lowering operator.
Since $S^x$ and $S^y$ commute with $\TT$
(see above),  the same applies to the raising operator
$S^+$ from (\ref{30}). Together with (\ref{31})
It follows that
\begin{eqnarray}
S^+\TT|n,l\rangle & = & e^{i \theta_{n,l}}S^+|n,l\rangle=e^{i \theta_{n,l}} c^+_{n,l} |n,l+1\rangle=
\nonumber
\\
\TT S^+|n,l\rangle & = &c^+_{n,l}  \TT  |n,l+1\rangle=c^+_{n,l}e^{i \theta_{n,l+1}}|n,l+1\rangle
\, . \ \ \ \ \ \
\label{b3}
\end{eqnarray}
%Together with (\ref{b2}), one 
We
thus can conclude that 
$\theta_{n,l+1}=\theta_{n,l}$, and finally that $\theta_{n,l}$  
only depends on $n$, but not on $l$.

Combining (\ref{b2}) with the $l$-independence of
$\theta_{n,l}$ one recovers (\ref{28})
%\begin{eqnarray}
%\langle n, l |\TT^\dagger\! B \TT |n,l'\rangle=\langle n, l | B | n,l' \rangle
%\label{b3}
%\end{eqnarray}
for arbitrary Hermitian operators $B$.
As in the main text, it is {\em a priori} understood in 
(\ref{28}) that $n\in\{1,...,N\}$ and $l,l'\in \{-L_n,...,L_n\}$,
but with the convention adopted below (\ref{a14}),
one readily can extend the same relation 
to arbitrary 
%integers 
$n,l,l'$.

We finally mention that the choice of the basis
as specified below (\ref{b1}) may in principle not be 
unique, but that such ambiguities can be excluded 
if all energies $E_n^0$ are pairwise different,
as it is assumed at the beginning of Sec.~\ref{s4}.
%Moreover, such an ambiguity still does not
%undermine the main conclusions at the end of 
%Sec.~\ref{s4}, since the existence of one basis 
%with the property (\ref{28}) is sufficient for their derivation.}



%%%%%%%%%%%%%%%%%%%%%%%%%%%%%%%%%%%%%%%%%%%%%%%%%%%%
\begin{thebibliography}{70}

\bibitem{ued20}
M. Ueda,
Quantum equilibration, thermalization and prethermalization in ultracold atoms, 
Nat. Rev. Phys. {\bf 2}, 669 (2020).

\bibitem{mor18}
T. Mori, T. N. Ikeda, E. Kaminishi, and M. Ueda,
Thermalization and prethermalization 
in isolated quantum systems: a theoretical overview,
J. Phys. B  {\bf 51}, 112001 (2018).

\bibitem{gog16}
C. Gogolin and J. Eisert,
Equilibration, thermalization, and the emergence
of statistical mechanics in closed quantum systems,
Rep. Prog. Phys. {\bf 79}, 056001 (2016).

\bibitem{dal16}
L. D'Alessio, Y. Kafri, A. Polkovnikov, and M. Rigol,
From Quantum Chaos and Eigenstate Thermalization
to Statistical Mechanics and Thermodynamics,
Adv. Phys.  {\bf 65}, 239 (2016).

\bibitem{lan16}
T. Langen,  T. Gasenzer, and  J. Schmiedmayer,
Prethermalization and universal dynamics in 
near-integrable quantum systems,
J. Stat. Mech. 064009 (2016).

\bibitem{nan15}
R. Nandkishore and D. A. Huse,
Many-body localization and thermalization 
in quantum statistical mechanics,
Annu. Rev. Condens. Matter Phys. {\bf 6}, 15 (2015).

\bibitem{rei08}
P. Reimann,
Foundation of statistical mechanics under 
experimentally realistic conditions,
Phys. Rev. Lett. {\bf 101}, 190403  (2008).

\bibitem{lin09}
N. Linden, S. Popescu, A. J. Short, and A. Winter,
Quantum mechanical evolution towards equilibrium,
Phys. Rev.  E {\bf 79}, 061103 (2009).

\bibitem{sho11}
A. J. Short, 
Equilibration of quantum systems and subsystems,
New J. Phys. {\bf 13}, 053009  (2011).

\bibitem{rei12}
P. Reimann and M.  Kastner,
Equilibration of macroscopic quantum systems,
New J. Phys. {\bf 14}, 043020 (2012).

\bibitem{sho12}
A. J. Short and T. C.  Farrelly,
Quantum equilibration in finite time,
New J. Phys. {\bf 14}, 013063 (2012).

\bibitem{bal16}
B. N. Balz and P. Reimann, 
Equilibration of isolated many-body quantum systems with 
respect to general distinguishability measures, 
Phys. Rev. E {\bf 93}, 062107 (2016).

\bibitem{kin06}
T.  Kinoshita, T. Wenger, and D. S. Weiss,
A quantum Newton's cradle,
Nature {\bf  440}, 900 (2006).

\bibitem{ber18}
H. Bernien et al,.
Probing many-body dynamics on a 51-atom quantum simulator,
Nature {\bf 551}, 579 (2017).

\bibitem{ban11}
M. C. Banuls, J. I. Cirac, and M. B. Hastings,
Strong and weak thermalization of infinite nonintegrable quantum systems,
Phys. Rev. Lett. {\bf 106}, 050405 (2011).

\bibitem{li20}
C. Li et al.,
Relaxation of bosons in one dimension and the onset of dimensional crossover,
SciPost Phys. {\bf 9}, 058 (2020).

\bibitem{mbs}
S. Moudgalya, B. A. Bernevig,  and N. Regnault,
Quantum Many-Body Scars and Hilbert Space Fragmentation: A Review of Exact Results,
arXiv:2109.00548;
M. Serbyn, D. A. Abanin, and Z. Papic,
Quantum many-body scars and weak breaking of ergodicity,
Nat. Phys. {\bf 17}, 675 (2021).

\bibitem{kim15}
H. Kim, M. C. Banuls, J. I. Cirac, M. B. Hastings, and D. A. Huse,
Slowest local operators in quantum spin chains,
Phys. Rev. E {\bf 92}, 012128 (2015).

\bibitem{lin17}
C.-J. Lin and O. I. Motrunich,
Quasiparticle explanation of weak-thermalization regime under quench 
in a nonintegrable quantum spin chain,
Phys. Rev. A {\bf 95}, 023621 (2017).

\bibitem{far17}
T. Farrelly, F. G. S. L.  Brand\~ao, and M. Cramer,
Thermalization and return to equilibrium on finite quantum lattice systems,
Phys. Rev. Lett. {\bf 118}, 140601 (2017).

\bibitem{han22}
P. Hannaford and K. Sacha, 
A decade of time crystals: quo vadis?,
EPL {\bf 139}, 10001 (2022).

\bibitem{ven19}
V. Khemani, R. Moessner, and S. L. Sondhi,
A brief history of time crystals,
arXiv:1910.10745.

\bibitem{med20}
M. Medenjak, B. Bu\ifmmode \check{c}\else \v{c}\fi{}a, and D. Jaksch,
Isolated Heisenberg magnet as a quantum time crystal,
Phys. Rev. B {\bf 102}, 041117(R) (2020).

\bibitem{wat15}
H. Watanabe and M. Oshikawa,
Absence of quantum time crystals,
Phys. Rev. Lett. {\bf 114}, 251603 (2015).

\bibitem{wat20}
H. Watanabe, M. Oshikawa, and T. Koma,
Proof of absence of long-range temporal orders in Gibbs states,
J. Stat. Phys. {\bf 178}, 926 (2020).

\bibitem{hua19}
Y. Huang, 
Absence of temporal order in states with spatial correlation decay,
arXiv:1912.01210.

\bibitem{vor21}
P. Vorndamme, H.-J. Schmidt, C. Schr\"oder, and J. Schnack,
Observation of phase synchronization and alignment during free induction 
decay of quantum spins with Heisenberg interactions,
New J. Phys. {\bf 23}, 083038 (2021).

\bibitem{alh20}
A. M. Alhambra, J. Riddell, and L. P. Garcia-Pintos,
Time evolution of correlation functions in quantum many-body systems,
Phys. Rev. Lett. {\bf 124}, 110605 (2020).

\bibitem{BSS:JMMM00}
K. B\"arwinkel, H.-J. Schmidt, and J. Schnack, 
Structure and relevant dimension of the Heisenberg model and applications to spin rings, 
J. Magn. Magn. Mater. {\bf 212}, 240 (2000).

\bibitem{tas98}
H. Tasaki,
From quantum dynamics to the canonical distribution:
general picture and rigorous example,
Phys. Rev. Lett. {\bf 80}, 1373 (1998).

\bibitem{sre99}
M. Srednicki, 
The approach to thermal equilibrium in quantized chaotic systems,
J. Phys. A: Math. Gen {\bf 32}, 1163 (1999).

\bibitem{mul15}
M. P. M\"uller, E. Adlam, L. Masanes, and N. Wiebe,
Thermalization and canonical typicality in translation-invariant quantum lattice systems,
Commun. Math. Phys. {\bf 340}, 499 (2015).

\bibitem{imb16}
J. Z. Imbrie,
On Many-Body Localization for Quantum Spin Chains,
J. Stat. Phys. {\bf 163}, 998  (2016).

\bibitem{gal18}
R. Gallego, H. Wilming, J. Eisert, and C. Gogolin,
What it takes to avoid equilibration,
Phys. Rev. E {\bf 98}, 022135 (2018).

\bibitem{wil19}
H. Wilming, M. Goihl, I. Roth, and J. Eisert,
Entanglement-ergodic quantum systems equilibrate exponentially well,
Phys. Rev. Lett. {\bf 123}, 200604 (2019).

\bibitem{boo20}
C. Booker, B. Bu\ifmmode \check{c}\else \v{c}\fi{}a, and D. Jaksch,
Non-stationarity and dissipative time crystals: spectral properties and finite-size effects,
New J. Phys. {\bf 22}, 085007 (2020).

\bibitem{rei}
P. Reimann and J. Gemmer,
Why are macroscopic experiments reproducible? 
Imitating the behavior of an ensemble by single pure states,
Phys. A (Amsterdam) {\bf 552}, 121840 (2020).

\bibitem{michi}
H. De Raedt and K. Michielsen, Computational methods for
simulating quantum computers, arXiv:quant-ph/0406210.

\bibitem{buc22}
B. Bu\ifmmode \check{c}\else \v{c}\fi{}a, C. Booker, and D. Jaksch,
Algebraic theory of quantum synchronization and limit cycles under dissipation,
SciPost Phys. {\bf 12}, 097 (2022).

\bibitem{wic63}
E. H. Wichmann and J. H. Crichton,
Cluster decomposition properties of the $S$ matrix,
Phys. Rev. {\bf 132}, 2788 (1963).

\bibitem{wei97}
S. Weinberg, What is quantum field theory, and what did we think it is?,
arXiv:hep-th/9702027

\bibitem{ess16}
F. H. L. Essler and M. Fagotti,
Quench dynamics and relaxation in isolated integrable quantum spin chains,
J. Stat. Mech. {\bf 6}, 064002 (2016).

\bibitem{mur19}
C. Murthy and M.  Srednicki,
Relaxation to Gaussian and generalized Gibbs states in systems of particles with quadratic Hamiltonians,
Phys. Rev. E {\bf 100}, 012146 (2019).

\bibitem{glu19}
M. Gluza, J. Eisert, and T.  Farrelly,
Equilibration towards generalized Gibbs ensembles for non-interacting systems,
SciPost {\bf 7}, 038 (2019).

\bibitem{ara69}
H. Araki,
Gibbs states of a one dimensional quantum lattice,
Commun. Math. Phys. {\bf 14}, 120 (1969).

\bibitem{par82}
Y. M. Park,
The cluster expansion for classical and quantum lattice systems,
J. Stat. Phys. {\bf 27}, 553 (1982)

\bibitem{par95}
Y. M. Park, and H. J. Yoo,
Uniqueness and clustering properties of Gibbs states 
for classical and quantum unbounded spin systems,
J. Stat. Phys. {\bf 80}, 223 (1995).

\bibitem{kli14}
M. Kliesch, C. Gogolin, M. J.  Kastoryano, A. Riera,  and J.  Eisert,
Locality of temperature,
Phys. Rev. X {\bf 4}, 031019 (2014).

\bibitem{fro15}
J. Fr\"ohlich and D. Ueltschi,
Some properties of correlations of quantum lattice 
systems in thermal equilibrium,
J. Math. Phys. (N.Y.) {\bf 56}, 053302 (2015).

\end{thebibliography}


\end{document}
