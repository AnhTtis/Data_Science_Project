\appendix
\label{sec:appendix}
\renewcommand\thetable{\Alph{section}\arabic{table}}   
\renewcommand\thefigure{\Alph{section}\arabic{figure}}  
% \section{Appendix}

 
\section{Implementation Details}
\label{sec:implementation_details}
We give more implementation details in the main paper of the "comparison with the task-specific methods" and "comparison with the efficient tuning methods". 

\paragraph{Basic Setting.} Our method contains a backbone for feature extraction and a decoder for segmentation prediction. We initialize the weight of the backbone via ImageNet classification pre-training, and the weight of the decoder is randomly initialized. Below, we give the details of each variant.


\paragraph{Full-tuning.} We follow the basic setting above, and then, fine-tune all the parameters of the encoder and decoder.

\paragraph{Only Decoder.} We follow the basic setting above, and then, fine-tune the parameters in the decoder only.

\paragraph{VPT~\cite{vpt}.} We first initialize the model following the basic setting. Then, we concatenate the prompt embeddings in each transformer block of the backbone only. Notice that, their prompt embeddings are implicitly shared across the whole dataset. We follow their  original paper and optimize the parameters in the prompt embeddings and the decoder. 

\paragraph{AdaptFormer~\cite{chen2022adaptformer}.} We first initialize the model following the basic setting above. Then, the AdaptMLP is added to each transformer block of the backbone for feature adaptation. We fine-tune the parameters in the decoder and the newly introduced AdaptMLP. 

\paragraph{EVP~(Ours).} We also initialize the weight following the basic setting. Then, we add the explicit prompting as described in the main paper of Figure~\ref{fig:arch}. 

\paragraph{Metric.}
AUC calculates the area of the ROC curve. ROC curve is a function of true positive rate~($\frac{tp}{tp + fn}$) in terms of false positive rate~($\frac{fp}{fp + tn}$), where $tp$, $tn$, $fp$, $fn$ represent the number of pixels which are classified as true positive, true negative, false positive, and false negative, respectively. $F_{1}$ score is defined as $F_{1} = \frac{2 \times precision \times recall}{precision + recall}$, where $precision = \frac{tp}{tp + fp}$ and $recall = \frac{tp}{tp + fn}$. 
The balance error rate~(BER) $ = \left(1-\frac{1}{2}\left(\frac{tp}{tp+fn}+\frac{tn}{tn+fp}\right)\right) \times 100$.
F-measure is calculated as $F_{\beta} = \frac{\left(1+\beta^2\right) \times precision \times recall}{\beta^2 \times precision + recall}$, where $\beta^2=0.3$. MAE computes pixel-wise average distance.
Weighted F-measure ($F_\beta^w$) weighting the quantities TP, TN, FP, and FN according to the errors at their location and their neighborhood information:
$F_\beta^w = \frac{\left(1+\beta^2\right) \times precision^w \times recall^w}{\beta^2 \times precision^w + recall^w}$.
E-measure~($E_\phi$) jointly considers image statistics and local pixel matching:
$E_\phi=\frac{1}{W \times H} \sum_{i=1}^W \sum_{j=1}^H \phi_S(i, j)$,
where $\phi_S$ is the alignment matrix depending on the similarity of the prediction and ground truth.


\paragraph{Training Data.}
Note that most forgery detection methods~(ManTraNet~\cite{wu2019mantra}, SPAN~\cite{hu2020span}, PSCCNet~\cite{liu2022pscc}, and ObjectFormer~\cite{wang2022objectformer} in Table~\ref{tab:sota_forgery}) and one shadow detection method~(MTMT~\cite{mtmt} in Table~\ref{tab:sota_shadow}) use extra training data to get better performance. We only use the training data from the standard datasets and obtain SOTA performance.



\section{More Results}
\label{sec:more_results}
We provide more experimental results in addition to the main paper.


\subsection{High-Frequency Prompting}
Our method gets the knowledge from the explicit content of the image itself, hence we also discuss other similar explicit clues of images as the prompts. Specifically, we choose the common-used Gaussian filter, the noise-filter~\cite{fridrich2012rich}, the all-zero image, and the original image as experiments. From Table~\ref{tab:filter}, we find the Gaussian filter shows a better performance in defocus blur since 
it is also a kind of blur. Also, the noise filter~\cite{fridrich2012rich} from forgery detection also boosts the performance. Interestingly, we find that simply replacing the original image with an all-zero image also boosts the performance, since it can also be considered as a kind of implicitly learned embeddings across the full dataset as in VPT~\cite{vpt}. Differently, the high-frequency components of the image achieve consistent performance improvement to other methods on these several benchmarks.


\subsection{HFC \textit{v.s.} LFC} We conduct the ablation study on choosing of high-frequency features or the low-frequency features in Table~\ref{tab:hfc}. From the table, using the low-frequency components as the prompting just show some trivial improvement on these datasets. Differently, the high-frequency components are more general solutions and show a much better performance in shadow detection, forgery detection, and camouflaged detection. Similar to the Gaussian filter as we discussed above, the LFC is also a kind of blur, which makes the advantage of LFC in the defocus blur detection.

\subsection{Mask Ratio $\tau$} We further evaluate the hyper-parameter mask ratio~$\tau$ introduced in Section~\ref{Preliminary}. From Table~\ref{tab:hfc}, when we mask out 25\% of the central pixels in the spectrum, it achieves consistently better performance in all the tasks. We also find that the performance may drop when the increasing of mask ratio~(all 0 images), especially in shadow detection, forgery detection, and camouflaged object detection. 


Below we first briefly describe the selected models and then their implementation details during pre-training.

% Traditional convolutional action recognition networks before 2017 are mostly built to process single frame or multiple consecutive frames; however, such simple structures overlook the importance of long-range temporal context in action recognition, which somehow underestimates the intrinsic temporal information within videos. 
Temporal segment networks (TSN) proposes segment-based sampling to learn temporal information across frames. 
Specifically, in TSN, a video is evenly divided into several temporal segments, which one random frame is sampled from. 
Then the output from each segment will be aggregated via pooling to obtain the final prediction. 
Temporal Shift Module (TSM) shifts feature channels along the temporal axis, which facilitates information exchanged among neighboring frames. 
It can be plug-and-played in 2D networks to enable stronger temporal modeling at zero computation and zero parameters.
Thus, TSM can achieve the performance of heavy 3D CNNs while maintaining the efficiency of 2D CNNs.
% TSM introduces stronger temporal learning capacity to 2D networks while maintaining light-weight. 

Inflated 3D ConvNet (I3D) is designed to bootstrap from the corresponding 2D network since (1) the architecture of 2D network is well designed and (2) the  weights of 2D network is well pre-trained, e.g., Inception~\cite{inception} $\rightarrow$ Inception-I3D~\cite{carreira2017quo}. 
% utilize pre-trained weights from the corresponding 2D network since these 2D weights have been well-designed and trained to perceive visual concepts.
I3D initializes its 3D kernels by duplicating the 2D ones along the temporal dimension, which helps the convergence of 3D CNNs. 
Inspired by~\cite{vaswani2017attention}, non-local networks (NL) adapts the non-local operation (i.e., self-attention~\cite{vaswani2017attention}) in its building block to model long-range dependency.
For video action recognition, its goal is to relate the same object, or person-object interaction within a distant time interval in videos.
Similar to TSM, non-local block is compatible to most convolutional networks.


TimeSformer is a pure transformer-based model, which is an extension of ViT~\cite{dosovitskiy2020image} to the spatiotemporal space. 
Given the quadratic complexity of self-attention, TimeSformer compares several attention strategies when considering temporal dimention in videos.
Finally, TimeSformer introduces the divided space-time attention to greatly reduce the computation burden but achieves promising results.
% on most video action recognition datasets. 
% This structure shows both effectiveness and efficiency in their reported results. 
Continuing this modeling shift from CNNs to Transformers, VideoSwin extends Swin Transformer~\cite{liu2021swin} by adding the inductive bias of locality in video transformers. 
Simply speaking, it adapts the idea of 2D shifted window self-attention to 3D space, which results in better speed-accuracy trade-off compared to previous approaches~\cite{bertasius2021space,arnab2021vivit}.
% Similarly, VideoSwin is an extension of Swin Transformer~\cite{liu2021swin}, by adapting the 2D shifted window self-attention to 3D.
% And shifted window ensure the connection across distant regions in the spatiotemporal tensors.


\begin{figure}[t]
\centering
    \includegraphics[width=8cm]{figures/radar_new.pdf}
    \caption{The rank of the averaged performance within different data domains for the 6 models in different settings. The most outside in these radar images means the highest performance. For each domain, we average the top-1 accuracy as the scores in finetuning and average the top-1 accuracy of 16-shot results in few-shot learning. Complete results are shown in Table~\ref{tab:finetune} and Figure~\ref{fewshot}.}
    \label{radar}
\end{figure}

% \section{Visualization}
\section{Additional Visual Results}
\label{sec:visulation}

We give more visual results of EVP and other task-specific methods on the four tasks in Figure~\ref{fig:supp_forgery},~\ref{fig:supp_shadow},~\ref{fig:supp_defocus}, and~\ref{fig:supp_cod} as supplementary to the visual results in the main paper.

\begin{figure*}[t]
  \captionsetup[subfigure]{position=b}
  \centering

  \setlength{\tabcolsep}{0pt}% Remove column gap in tabular
  \subcaptionbox{\scriptsize{Input}}{%
    \begin{tabular}{c}
      \includegraphics[width=0.19\textwidth, height=0.15\textwidth]{images/forgery/input/Sp_D_CNN_A_sec0041_pla0037_0098.jpg} \\[0.1em]
      \includegraphics[width=0.19\textwidth, height=0.15\textwidth]{images/forgery/input/Sp_D_CNN_R_sec0018_sec0004_0101.jpg} \\[0.1em]
      \includegraphics[width=0.19\textwidth, height=0.15\textwidth]{images/forgery/input/Sp_D_NND_A_nat0075_arc0069_0612.jpg}\\[0.1em]
      \includegraphics[width=0.19\textwidth, height=0.15\textwidth]{images/forgery/input/Sp_D_NND_A_sec0014_ani0016_0617.jpg}\\[0.1em]
      \includegraphics[width=0.19\textwidth, height=0.15\textwidth]{images/forgery/input/Sp_D_NNN_A_arc0026_arc0008_0290.jpg}
    \end{tabular}}%
  \hspace{0.1em}%
  \subcaptionbox{\scriptsize{GT}}{%
    \begin{tabular}{c}
      \includegraphics[width=0.19\textwidth, height=0.15\textwidth]{images/forgery/gt/Sp_D_CNN_A_sec0041_pla0037_0098.jpg}\\[0.1em]
      \includegraphics[width=0.19\textwidth, height=0.15\textwidth]{images/forgery/gt/Sp_D_CNN_R_sec0018_sec0004_0101.jpg} \\[0.1em]
      \includegraphics[width=0.19\textwidth, height=0.15\textwidth]{images/forgery/gt/Sp_D_NND_A_nat0075_arc0069_0612.jpg}\\[0.1em]
      \includegraphics[width=0.19\textwidth, height=0.15\textwidth]{images/forgery/gt/Sp_D_NND_A_sec0014_ani0016_0617.jpg}\\[0.1em]
      \includegraphics[width=0.19\textwidth, height=0.15\textwidth]{images/forgery/gt/Sp_D_NNN_A_arc0026_arc0008_0290.jpg}
    \end{tabular}}%
  \hspace{0.1em}%
  \subcaptionbox{\scriptsize{Ours}}{%
    \begin{tabular}{c}
      \includegraphics[width=0.19\textwidth, height=0.15\textwidth]{images/forgery/ours/Sp_D_CNN_A_sec0041_pla0037_0098.jpg} \\[0.1em]
      \includegraphics[width=0.19\textwidth, height=0.15\textwidth]{images/forgery/ours/Sp_D_CNN_R_sec0018_sec0004_0101.jpg}\\[0.1em]
      \includegraphics[width=0.19\textwidth, height=0.15\textwidth]{images/forgery/ours/Sp_D_NND_A_nat0075_arc0069_0612.jpg}\\[0.1em]
      \includegraphics[width=0.19\textwidth, height=0.15\textwidth]{images/forgery/ours/Sp_D_NND_A_sec0014_ani0016_0617.jpg}\\[0.1em]
      \includegraphics[width=0.19\textwidth, height=0.15\textwidth]{images/forgery/ours/Sp_D_NNN_A_arc0026_arc0008_0290.jpg}
    \end{tabular}}%
  \hspace{0.1em}%
  \subcaptionbox{\scriptsize{ManTraNet}}{%
    \begin{tabular}{c}
      \includegraphics[width=0.19\textwidth, height=0.15\textwidth]{images/forgery/mantranet/Sp_D_CNN_A_sec0041_pla0037_0098.jpg} \\[0.1em]
      \includegraphics[width=0.19\textwidth, height=0.15\textwidth]{images/forgery/mantranet/Sp_D_CNN_R_sec0018_sec0004_0101.jpg} \\[0.1em]
      \includegraphics[width=0.19\textwidth, height=0.15\textwidth]{images/forgery/mantranet/Sp_D_NND_A_nat0075_arc0069_0612.jpg}\\[0.1em]
      \includegraphics[width=0.19\textwidth, height=0.15\textwidth]{images/forgery/mantranet/Sp_D_NND_A_sec0014_ani0016_0617.jpg}\\[0.1em]
      \includegraphics[width=0.19\textwidth, height=0.15\textwidth]{images/forgery/mantranet/Sp_D_NNN_A_arc0026_arc0008_0290.jpg}
    \end{tabular}}%
  \hspace{0.1em}%
  \subcaptionbox{\scriptsize{SPAN}}{%
    \begin{tabular}{c}
      \includegraphics[width=0.19\textwidth, height=0.15\textwidth]{images/forgery/span/Sp_D_CNN_A_sec0041_pla0037_0098.jpg}\\[0.1em]
      \includegraphics[width=0.19\textwidth, height=0.15\textwidth]{images/forgery/span/Sp_D_CNN_R_sec0018_sec0004_0101.jpg} \\[0.1em]
      \includegraphics[width=0.19\textwidth, height=0.15\textwidth]{images/forgery/span/Sp_D_NND_A_nat0075_arc0069_0612.jpg}\\[0.1em]
      \includegraphics[width=0.19\textwidth, height=0.15\textwidth]{images/forgery/span/Sp_D_NND_A_sec0014_ani0016_0617.jpg}\\[0.1em]
      \includegraphics[width=0.19\textwidth, height=0.15\textwidth]{images/forgery/span/Sp_D_NNN_A_arc0026_arc0008_0290.jpg}
    \end{tabular}}%
  \hspace{0.0em}%
  % \vspace{-1em}
  
\caption{More results on CAISA~\cite{dong2013casia} dataset for forgery detection. We compare to ManTraNet~\cite{wu2019mantra} and SPAN~\cite{hu2020span}.}
\label{fig:supp_forgery}
\end{figure*}




\begin{figure*}[t]
  \captionsetup[subfigure]{position=b}
  \centering
  \setlength{\tabcolsep}{0pt}% Remove column gap in tabular
  \subcaptionbox{\scriptsize{Input}}{%
    \begin{tabular}{c}
      \includegraphics[width=0.15\textwidth, height=0.15\textwidth]{images/shadow/input/103-5.png} \\[0.1em]
      \includegraphics[width=0.15\textwidth, height=0.15\textwidth]{images/shadow/input/114-1.png} \\[0.1em]
      \includegraphics[width=0.15\textwidth, height=0.15\textwidth]{images/shadow/input/115-7.png}\\[0.1em]
      \includegraphics[width=0.15\textwidth, height=0.15\textwidth]{images/shadow/input/125-10.png}\\[0.1em]
      \includegraphics[width=0.15\textwidth, height=0.15\textwidth]{images/shadow/input/127-1.png}
    \end{tabular}}%
  \hspace{0.1em}%
  \subcaptionbox{\scriptsize{GT}}{%
    \begin{tabular}{c}
      \includegraphics[width=0.15\textwidth, height=0.15\textwidth]{images/shadow/gt/103-5.png} \\[0.1em]
      \includegraphics[width=0.15\textwidth, height=0.15\textwidth]{images/shadow/gt/114-1.png} \\[0.1em]
      \includegraphics[width=0.15\textwidth, height=0.15\textwidth]{images/shadow/gt/115-7.png}\\[0.1em]
      \includegraphics[width=0.15\textwidth, height=0.15\textwidth]{images/shadow/gt/125-10.png}\\[0.1em]
      \includegraphics[width=0.15\textwidth, height=0.15\textwidth]{images/shadow/gt/127-1.png}
    \end{tabular}}%
  \hspace{0.1em}%
  \subcaptionbox{\scriptsize{Ours}}{%
    \begin{tabular}{c}
      \includegraphics[width=0.15\textwidth, height=0.15\textwidth]{images/shadow/ours/103-5.png} \\[0.1em]
      \includegraphics[width=0.15\textwidth, height=0.15\textwidth]{images/shadow/ours/114-1.png} \\[0.1em]
      \includegraphics[width=0.15\textwidth, height=0.15\textwidth]{images/shadow/ours/115-7.png}\\[0.1em]
      \includegraphics[width=0.15\textwidth, height=0.15\textwidth]{images/shadow/ours/125-10.png}\\[0.1em]
      \includegraphics[width=0.15\textwidth, height=0.15\textwidth]{images/shadow/ours/127-1.png}
    \end{tabular}}%
  \hspace{0.1em}%
  \subcaptionbox{\scriptsize{DSD}}{%
    \begin{tabular}{c}
      \includegraphics[width=0.15\textwidth, height=0.15\textwidth]{images/shadow/dsd/103-5.png} \\[0.1em]
      \includegraphics[width=0.15\textwidth, height=0.15\textwidth]{images/shadow/dsd/114-1.png} \\[0.1em]
      \includegraphics[width=0.15\textwidth, height=0.15\textwidth]{images/shadow/dsd/115-7.png}\\[0.1em]
      \includegraphics[width=0.15\textwidth, height=0.15\textwidth]{images/shadow/dsd/125-10.png}\\[0.1em]
      \includegraphics[width=0.15\textwidth, height=0.15\textwidth]{images/shadow/dsd/127-1.png}
    \end{tabular}}%
  \hspace{0.1em}%
  \subcaptionbox{\scriptsize{MTMT}}{%
    \begin{tabular}{c}
      \includegraphics[width=0.15\textwidth, height=0.15\textwidth]{images/shadow/mtmt/103-5.png} \\[0.1em]
      \includegraphics[width=0.15\textwidth, height=0.15\textwidth]{images/shadow/mtmt/114-1.png} \\[0.1em]
      \includegraphics[width=0.15\textwidth, height=0.15\textwidth]{images/shadow/mtmt/115-7.png}\\[0.1em]
      \includegraphics[width=0.15\textwidth, height=0.15\textwidth]{images/shadow/mtmt/125-10.png}\\[0.1em]
      \includegraphics[width=0.15\textwidth, height=0.15\textwidth]{images/shadow/mtmt/127-1.png}
    \end{tabular}}%
  \hspace{0.1em}%
  \subcaptionbox{\scriptsize{FDRNet}}{%
    \begin{tabular}{c}
      \includegraphics[width=0.15\textwidth, height=0.15\textwidth]{images/shadow/fdrnet/103-5.png} \\[0.1em]
      \includegraphics[width=0.15\textwidth, height=0.15\textwidth]{images/shadow/fdrnet/114-1.png} \\[0.1em]
      \includegraphics[width=0.15\textwidth, height=0.15\textwidth]{images/shadow/fdrnet/115-7.png}\\[0.1em]
      \includegraphics[width=0.15\textwidth, height=0.15\textwidth]{images/shadow/fdrnet/125-10.png}\\[0.1em]
      \includegraphics[width=0.15\textwidth, height=0.15\textwidth]{images/shadow/fdrnet/127-1.png}
    \end{tabular}}%
  \hspace{0.0em}%
  % \vspace{-1em}
  
\caption{More results on ISTD~\cite{wang2018stacked} dataset for shadow detection. We compare to DSD~\cite{zhao2021self}, MTMT~\cite{mtmt}, FDRNet~\cite{zhu2021mitigating}.}
\label{fig:supp_shadow}
\end{figure*}




\begin{figure*}[t]
  \captionsetup[subfigure]{position=b}
  \centering
  \setlength{\tabcolsep}{0pt}% Remove column gap in tabular
  \subcaptionbox{\scriptsize{Input}}{%
    \begin{tabular}{c}
      \includegraphics[width=0.15\textwidth, height=0.15\textwidth]{images/defocus/input/20.jpg} \\[0.1em]
      \includegraphics[width=0.15\textwidth, height=0.15\textwidth]{images/defocus/input/31.jpg} \\[0.1em]
      \includegraphics[width=0.15\textwidth, height=0.15\textwidth]{images/defocus/input/43.jpg} \\[0.1em]
      \includegraphics[width=0.15\textwidth, height=0.15\textwidth]{images/defocus/input/78.jpg} \\[0.1em]
      \includegraphics[width=0.15\textwidth, height=0.15\textwidth]{images/defocus/input/93.jpg}
    \end{tabular}}%
  \hspace{0.1em}%
  \subcaptionbox{\scriptsize{GT}}{%
    \begin{tabular}{c}
      \includegraphics[width=0.15\textwidth, height=0.15\textwidth]{images/defocus/gt/20.jpg} \\[0.1em]
      \includegraphics[width=0.15\textwidth, height=0.15\textwidth]{images/defocus/gt/31.jpg} \\[0.1em]
      \includegraphics[width=0.15\textwidth, height=0.15\textwidth]{images/defocus/gt/43.jpg} \\[0.1em]
      \includegraphics[width=0.15\textwidth, height=0.15\textwidth]{images/defocus/gt/78.jpg} \\[0.1em]
      \includegraphics[width=0.15\textwidth, height=0.15\textwidth]{images/defocus/gt/93.jpg}
    \end{tabular}}%
  \hspace{0.1em}%
  \subcaptionbox{\scriptsize{Ours}}{%
    \begin{tabular}{c}
      \includegraphics[width=0.15\textwidth, height=0.15\textwidth]{images/defocus/ours/20.jpg} \\[0.1em]
      \includegraphics[width=0.15\textwidth, height=0.15\textwidth]{images/defocus/ours/31.jpg} \\[0.1em]
      \includegraphics[width=0.15\textwidth, height=0.15\textwidth]{images/defocus/ours/43.jpg} \\[0.1em]
      \includegraphics[width=0.15\textwidth, height=0.15\textwidth]{images/defocus/ours/78.jpg} \\[0.1em]
      \includegraphics[width=0.15\textwidth, height=0.15\textwidth]{images/defocus/ours/93.jpg}
    \end{tabular}}%
  \hspace{0.1em}%
  \subcaptionbox{\scriptsize{BTBNet}}{%
    \begin{tabular}{c}
      \includegraphics[width=0.15\textwidth, height=0.15\textwidth]{images/defocus/btbnet/20.jpg} \\[0.1em]
      \includegraphics[width=0.15\textwidth, height=0.15\textwidth]{images/defocus/btbnet/31.jpg} \\[0.1em]
      \includegraphics[width=0.15\textwidth, height=0.15\textwidth]{images/defocus/btbnet/43.jpg} \\[0.1em]
      \includegraphics[width=0.15\textwidth, height=0.15\textwidth]{images/defocus/btbnet/78.jpg} \\[0.1em]
      \includegraphics[width=0.15\textwidth, height=0.15\textwidth]{images/defocus/btbnet/93.jpg}
    \end{tabular}}%
  \hspace{0.1em}%
  \subcaptionbox{\scriptsize{CENet}}{%
    \begin{tabular}{c}
      \includegraphics[width=0.15\textwidth, height=0.15\textwidth]{images/defocus/cenet/20.jpg} \\[0.1em]
      \includegraphics[width=0.15\textwidth, height=0.15\textwidth]{images/defocus/cenet/31.jpg} \\[0.1em]
      \includegraphics[width=0.15\textwidth, height=0.15\textwidth]{images/defocus/cenet/43.jpg} \\[0.1em]
      \includegraphics[width=0.15\textwidth, height=0.15\textwidth]{images/defocus/cenet/78.jpg} \\[0.1em]
      \includegraphics[width=0.15\textwidth, height=0.15\textwidth]{images/defocus/cenet/93.jpg}
    \end{tabular}}%
  \hspace{0.1em}%
  \subcaptionbox{\scriptsize{EFENet}}{%
    \begin{tabular}{c}
      \includegraphics[width=0.15\textwidth, height=0.15\textwidth]{images/defocus/efenet/20.jpg} \\[0.1em]
      \includegraphics[width=0.15\textwidth, height=0.15\textwidth]{images/defocus/efenet/31.jpg} \\[0.1em]
      \includegraphics[width=0.15\textwidth, height=0.15\textwidth]{images/defocus/efenet/43.jpg} \\[0.1em]
      \includegraphics[width=0.15\textwidth, height=0.15\textwidth]{images/defocus/efenet/78.jpg} \\[0.1em]
      \includegraphics[width=0.15\textwidth, height=0.15\textwidth]{images/defocus/efenet/93.jpg}
    \end{tabular}}%
  \hspace{0.0em}%
  % \vspace{-1em}
  
\caption{More results on CUHK~\cite{shi2014discriminative} dataset for defocus blur detection.We compare to BTBNet~\cite{zhao2019btbnet}, CENet~\cite{zhao2019cenet} and EFENet~\cite{zhao2021defocus}.}
\label{fig:supp_defocus}
\end{figure*}



\begin{figure*}[t]
  \captionsetup[subfigure]{position=b}
  \centering
  \setlength{\tabcolsep}{0pt}% Remove column gap in tabular
  \subcaptionbox{\scriptsize{Input}}{%
    \begin{tabular}{c}
      \includegraphics[width=0.13\textwidth, height=0.13\textwidth]{images/cod/input/camourflage_00631.jpg} \\[0.1em]
      \includegraphics[width=0.13\textwidth, height=0.13\textwidth]{images/cod/input/camourflage_00758.jpg} \\[0.1em]
      \includegraphics[width=0.13\textwidth, height=0.13\textwidth]{images/cod/input/camourflage_00869.jpg}\\[0.1em]
      \includegraphics[width=0.13\textwidth, height=0.13\textwidth]{images/cod/input/camourflage_01001.jpg}\\[0.1em]
      \includegraphics[width=0.13\textwidth, height=0.13\textwidth]{images/cod/input/camourflage_00510.jpg}
    \end{tabular}}%
  \hspace{0.1em}%
  \subcaptionbox{\scriptsize{GT}}{%
    \begin{tabular}{c}
      \includegraphics[width=0.13\textwidth, height=0.13\textwidth]{images/cod/gt/camourflage_00631.jpg} \\[0.1em]
      \includegraphics[width=0.13\textwidth, height=0.13\textwidth]{images/cod/gt/camourflage_00758.jpg} \\[0.1em]
      \includegraphics[width=0.13\textwidth, height=0.13\textwidth]{images/cod/gt/camourflage_00869.jpg}\\[0.1em]
      \includegraphics[width=0.13\textwidth, height=0.13\textwidth]{images/cod/gt/camourflage_01001.jpg}\\[0.1em]
      \includegraphics[width=0.13\textwidth, height=0.13\textwidth]{images/cod/gt/camourflage_00510.jpg}
    \end{tabular}}%
  \hspace{0.1em}%
  \subcaptionbox{\scriptsize{Ours}}{%
    \begin{tabular}{c}
      \includegraphics[width=0.13\textwidth, height=0.13\textwidth]{images/cod/ours/camourflage_00631.jpg} \\[0.1em]
      \includegraphics[width=0.13\textwidth, height=0.13\textwidth]{images/cod/ours/camourflage_00758.jpg} \\[0.1em]
      \includegraphics[width=0.13\textwidth, height=0.13\textwidth]{images/cod/ours/camourflage_00869.jpg}\\[0.1em]
      \includegraphics[width=0.13\textwidth, height=0.13\textwidth]{images/cod/ours/camourflage_01001.jpg}\\[0.1em]
      \includegraphics[width=0.13\textwidth, height=0.13\textwidth]{images/cod/ours/camourflage_00510.jpg}
    \end{tabular}}%
  \hspace{0.1em}%
  \subcaptionbox{\scriptsize{SINet}}{%
    \begin{tabular}{c}
      \includegraphics[width=0.13\textwidth, height=0.13\textwidth]{images/cod/sinet/camourflage_00631.jpg} \\[0.1em]
      \includegraphics[width=0.13\textwidth, height=0.13\textwidth]{images/cod/sinet/camourflage_00758.jpg} \\[0.1em]
      \includegraphics[width=0.13\textwidth, height=0.13\textwidth]{images/cod/sinet/camourflage_00869.jpg}\\[0.1em]
      \includegraphics[width=0.13\textwidth, height=0.13\textwidth]{images/cod/sinet/camourflage_01001.jpg}\\[0.1em]
      \includegraphics[width=0.13\textwidth, height=0.13\textwidth]{images/cod/sinet/camourflage_00510.png}
    \end{tabular}}%
  \hspace{0.1em}%
  \subcaptionbox{\scriptsize{PFNet}}{%
    \begin{tabular}{c}
      \includegraphics[width=0.13\textwidth, height=0.13\textwidth]{images/cod/pfnet/camourflage_00631.jpg} \\[0.1em]
      \includegraphics[width=0.13\textwidth, height=0.13\textwidth]{images/cod/pfnet/camourflage_00758.jpg} \\[0.1em]
      \includegraphics[width=0.13\textwidth, height=0.13\textwidth]{images/cod/pfnet/camourflage_00869.jpg}\\[0.1em]
      \includegraphics[width=0.13\textwidth, height=0.13\textwidth]{images/cod/pfnet/camourflage_01001.jpg}\\[0.1em]
      \includegraphics[width=0.13\textwidth, height=0.13\textwidth]{images/cod/pfnet/camourflage_00510.png}
    \end{tabular}}%
  \hspace{0.1em}%
  \subcaptionbox{\scriptsize{JCOD}}{%
    \begin{tabular}{c}
      \includegraphics[width=0.13\textwidth, height=0.13\textwidth]{images/cod/jcod/camourflage_00631.png} \\[0.1em]
      \includegraphics[width=0.13\textwidth, height=0.13\textwidth]{images/cod/jcod/camourflage_00758.png} \\[0.1em]
      \includegraphics[width=0.13\textwidth, height=0.13\textwidth]{images/cod/jcod/camourflage_00869.png}\\[0.1em]
      \includegraphics[width=0.13\textwidth, height=0.13\textwidth]{images/cod/jcod/camourflage_01001.png}\\[0.1em]
      \includegraphics[width=0.13\textwidth, height=0.13\textwidth]{images/cod/jcod/camourflage_00510.png}
    \end{tabular}}%
  \hspace{0.1em}%
  \subcaptionbox{\scriptsize{RankNet}}{%
    \begin{tabular}{c}
      \includegraphics[width=0.13\textwidth, height=0.13\textwidth]{images/cod/ranknet/camourflage_00631.png} \\[0.1em]
      \includegraphics[width=0.13\textwidth, height=0.13\textwidth]{images/cod/ranknet/camourflage_00758.png} \\[0.1em]
      \includegraphics[width=0.13\textwidth, height=0.13\textwidth]{images/cod/ranknet/camourflage_00869.png}\\[0.1em]
      \includegraphics[width=0.13\textwidth, height=0.13\textwidth]{images/cod/ranknet/camourflage_01001.png}\\[0.1em]
      \includegraphics[width=0.13\textwidth, height=0.13\textwidth]{images/cod/ranknet/camourflage_00510.png}
    \end{tabular}}%
  \hspace{0.0em}%
  
  % \vspace{-1em}
\caption{More results on CAMO~\cite{le2019anabranch} dataset for camouflaged object detection. We compare to SINet~\cite{fan2020camouflaged}, PFNEt~\cite{mei2021camouflaged}, JCOD~\cite{li2021uncertainty} and RankNet~\cite{lv2021simultaneously}.}
\label{fig:supp_cod}
\end{figure*}
