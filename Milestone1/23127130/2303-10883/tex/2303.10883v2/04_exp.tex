\section{Experiment}
\label{sec:exp}

The ARMBench dataset presents: 1) a collection of sensor data acquired by a robotic manipulation workcell performing pick-and-place operation, 2) metadata and reference images for objects in containers, 3) a set of annotations acquired either automatically, by virtue of the system design, or via manual labeling, and 4) tasks and metrics to benchmark perception algorithms for robotic manipulation. Fig.\ \ref{fig:contributions} illustrates the benchmark tasks and variety of objects captured in the dataset. The dataset captures diversity in objects with respect to Amazon product categories as well as physical characteristics such as size, shape, material, deformability, appearance, fragility, etc. 

The data collection platform is a robotic manipulation workcell performing pick-and-place operation in a warehouse \cite{Sparrow2022}. The workcell contains a robotic arm mounted with a vacuum-based end-effector. It is presented with a heterogeneous collection of objects placed in unstructured configurations within a container (storage tote). The robotic arm is tasked with picking one object at a time (singulation) and place it on moving trays until the container is empty. The empty container ejects the workcell and is replaced by a new container. While the operation is completely autonomous, it includes a human-in-the-loop to monitor the status of each pick-and-place activity, annotate, and resolve any defects during manipulation. Multiple imaging sensors are placed in the workcell to facilitate and validate the pick-and-place operation. Following is a list of sensor data (Fig.\ \ref{fig:intro}) associated with each pick activity:
\begin{itemize}
\item Pick-image: A 5\,MP camera is used to capture a top-down image of the container.
% \item Pick-3D: Two Ensenso sensors capture the 3D point cloud of the source container.
\item Transfer-images: Multiple 5\,MP cameras are placed on different sides in the workcell to capture the moving object from different viewpoints.
% \item Transfer-Barcode: Multiple Cognex barcode sensors are used to scan the barcode of the object during transfer.
\item Place-image: A top-down view of the object is captured once it is placed on the tray.
\item Video: A camera is mounted to capture 720p videos of pick-and-place manipulation processes at 30\,FPS
\end{itemize}
Additionally, the following metadata (Fig.\ \ref{fig:contributions} (b)) is available by virtue of a warehouse tracking system:
\begin{itemize}
\item Container-manifest: A list of objects present in the container along with data such as product description, coarse dimensions, and weight.
\item Reference images: One or more images of objects from previous operations within the warehouse.
\end{itemize}
The sensor data and metadata were consumed by perception algorithms required to autonomously operate the robotic workcell. Benchmarking against these algorithms would not only optimize a manipulation task such as the one used for data collection but also enable more complex and intentional manipulation. This work considers a subset of such perception tasks namely object segmentation, object identification, and defect detection. These are critical not only to make informed grasping and motion decisions but also to track the state of the objects and containers within the warehouse. The following sections will describe these tasks and present the challenges using annotations, baseline algorithms, and evaluation metrics.
\begin{figure*}[tp]
    \centering
    \includegraphics[width=0.9\linewidth]{figs/images/sota_result.pdf}
    \caption{Comparisons with other task-specific methods. We show the results  of: 
    SINet~\cite{fan2020camouflaged} and PFNet~\cite{mei2021camouflaged} on CAMO~\cite{le2019anabranch} dataset for camouflaged object detection (Top-left),
    ManTra~\cite{wu2019mantra} and SPAN~\cite{hu2020span} on CAISA~\cite{dong2013casia} dataset for forgery detection (Top-right),
    MTMT~\cite{mtmt} and FDRNet~\cite{zhu2021mitigating} on ISTD~\cite{wang2018stacked} dataset for shadow detection (Bottom-left), CENet~\cite{zhao2019cenet} and EFENet~\cite{zhao2021defocus} on CUHK~\cite{shi2014discriminative} dataset for defocus blur detection (Bottom-right).}
    \label{fig:sota_result}
\end{figure*}


\subsection{Datasets}
We evaluate our model on a variety of datasets for four tasks: forgery detection, shadow detection, defocus blur detection, and camouflaged object detection. A summary of the basic information of these datasets is illustrated in Table~\ref{tab:dataset}.


\paragraph{Forgery Detection.} CASIA~\cite{dong2013casia} is a large dataset for forgery detection, which is composed of 5,123 training and 921 testing spliced and copy-moved images. 
IMD20~\cite{novozamsky2020imd2020} is a real-life forgery image dataset that consists of 2, 010 samples for testing.
We follow the protocol of previous works~\cite{liu2022pscc, hao2021transforensics,wang2022objectformer} to conduct the training and evaluation at the resolution of $256 \times 256$. We use pixel-level Area Under the Receiver Operating Characteristic Curve~(AUC) and $F_{1}$ score to evaluate the performance. 

\paragraph{Shadow Detection.} SBU~\cite{sbu} is the largest annotated shadow dataset which contains 4,089 training and 638 testing samples, respectively.
ISTD~\cite{wang2018stacked} contains triple samples for shadow detection and removal, we only use the shadowed image and shadow mask to train our method.
Following \cite{mtmt,zhu2018bidirectional,zhu2021mitigating}, we train and test both datasets with the size of $400 \times 400$. 
As for the evaluation metrics, We report the balance error rate~(BER).

\paragraph{Defocus Blur Detection.} 
Following previous work~\cite{zhao2018defocus, cun2020defocus}, we train the defocus blur detection model in the CUHK dataset~\cite{shi2014discriminative}, which contains a total of 704 partial defocus samples. We train the network on the 604 images split from the CUHK dataset and test in DUT~\cite{zhao2018defocus} and the rest of the CUHK dataset. 
The images are resized into $320 \times 320$, following~\cite{cun2020defocus}. We report performances with commonly used metrics: F-measure~($F_{\beta}$) and mean absolute error~(MAE). 



\paragraph{Camouflaged Object Detection.} 
COD10K~\cite{fan2020camouflaged} is the largest dataset for camouflaged object detection, which contains 3,040 training and 2,026 testing samples. CHAMELEON~\cite{skurowski2018animal} includes 76 images collected from the Internet for testing. CAMO~\cite{le2019anabranch} provides diverse images with naturally camouflaged objects and artificially camouflaged objects. Following~\cite{fan2020camouflaged,mei2021camouflaged}, we train on the combined dataset and test on the three datasets. We employ commonly used metrics: S-measure~($S_{m}$), mean E-measure~($E_\phi$), weighted F-measure~($F_\beta^w$), and MAE for evaluation. 




\subsection{Implementation Details}
All the experiments are performed on a single NVIDIA Titan V GPU with 12G memory. AdamW~\cite{adam} optimizer is used for all the experiments. The initial learning rate is set to $2e^{-4}$ for defocus blur detection and camouflaged object detection, and $5e^{-4}$ for others. Cosine decay is applied to the learning rate. The models are trained for 20 epochs for the SBU~\cite{sbu} dataset and camouflaged combined dataset~\cite{fan2020camouflaged,skurowski2018animal}, and 50 epochs for others. Random horizontal flipping is applied during training for data augmentation. The mini-batch is equal to 4. Binary cross-entropy (BCE) loss is used for defocus blur detection and forgery detection, balanced BCE loss is used for shadow detection, and BCE loss and IOU loss are used for camouflaged object detection. All the experiments are conducted with SegFormer-B4~\cite{xie2021segformer} pre-trained on the ImageNet-1k~\cite{imagenet} dataset.
 

\begin{figure*}[tp]
    \centering
    \includegraphics[width=0.9\linewidth]{figs/images/sota_result.pdf}
    \caption{Comparisons with other task-specific methods. We show the results  of: 
    SINet~\cite{fan2020camouflaged} and PFNet~\cite{mei2021camouflaged} on CAMO~\cite{le2019anabranch} dataset for camouflaged object detection (Top-left),
    ManTra~\cite{wu2019mantra} and SPAN~\cite{hu2020span} on CAISA~\cite{dong2013casia} dataset for forgery detection (Top-right),
    MTMT~\cite{mtmt} and FDRNet~\cite{zhu2021mitigating} on ISTD~\cite{wang2018stacked} dataset for shadow detection (Bottom-left), CENet~\cite{zhao2019cenet} and EFENet~\cite{zhao2021defocus} on CUHK~\cite{shi2014discriminative} dataset for defocus blur detection (Bottom-right).}
    \label{fig:sota_result}
\end{figure*}
\begin{figure}
       \centering
        \setlength{\tabcolsep}{1pt}
        {\scriptsize
        \begin{tabular}{c c c c c c c }
            { Original } &
            \multicolumn{2}{c}{  } &
            \multicolumn{4}{c}{$\longleftarrow$ Object level variations $\longrightarrow$} \\
            \includegraphics[width=0.185\linewidth]{images/ablation/chair.jpg} &
            \multicolumn{2}{c}{  } &
            \includegraphics[width=0.185\linewidth]{images/ablation/1_only_prompt_mixing/bench.jpg} &
            \includegraphics[width=0.185\linewidth]{images/ablation/1_only_prompt_mixing/stool.jpg} &
            \includegraphics[width=0.185\linewidth]{images/ablation/1_only_prompt_mixing/armchair.jpg} &
            \includegraphics[width=0.185\linewidth]{images/ablation/1_only_prompt_mixing/saddle.jpg} \\
            \multicolumn{3}{c}{  } &
            \multicolumn{4}{c}{ Only Prompt Mixing } \\
            \multicolumn{3}{c}{ } &
            \includegraphics[width=0.185\linewidth]{images/ablation/2_with_self_attn_injection/bench.jpg} &
            \includegraphics[width=0.185\linewidth]{images/ablation/2_with_self_attn_injection/stool.jpg} &
            \includegraphics[width=0.185\linewidth]{images/ablation/2_with_self_attn_injection/armchair.jpg} &
            \includegraphics[width=0.185\linewidth]{images/ablation/2_with_self_attn_injection/saddle.jpg} \\
            \multicolumn{3}{c}{  } &
            \multicolumn{4}{c}{ + Attention-Based Shape Localization } \\
            \multicolumn{3}{c}{ } &
            \includegraphics[width=0.185\linewidth]{images/ablation/3_background_blending/bench.jpg} &
            \includegraphics[width=0.185\linewidth]{images/ablation/3_background_blending/stool.jpg} &
            \includegraphics[width=0.185\linewidth]{images/ablation/3_background_blending/armchair.jpg} &
            \includegraphics[width=0.185\linewidth]{images/ablation/3_background_blending/saddle.jpg} \\
            \multicolumn{3}{c}{  } &
            \multicolumn{4}{c}{ + Controllable Background Preservation } \\
        \end{tabular}
        }
    \vspace{1mm}
    \captionof{figure}{
    Ablating our full object variations pipeline. Original image was crated using the prompt ``A \emph{chair} with a dog on it''. 
    }
    \vspace{-10pt}
    \label{fig:ablation}
\end{figure}


\subsection{Main Results}
\label{sec:main_results}


\paragraph{Comparison with the task-specific methods.}
EVP performs well when compared with task-specific methods. We report the comparison of our methods and other task-specific methods in Table~\ref{tab:sota_defocus}, Table~\ref{tab:sota_shadow}, Table~\ref{tab:sota_forgery}, and Table~\ref{tab:sota_cod}. Thanks to our stronger backbone and prompting strategy, EVP achieves the best performance in 5 datasets across 4 different tasks. However, compared with other well-designed domain-specific methods, EVP only introduces a small number of tunable parameters with the frozen backbone and obtains non-trivial performance. We also show some visual comparisons with other methods for each task individually in Figure~\ref{fig:sota_result}. We can see the proposed method predicts more accurate masks compared to other approaches.



\paragraph{Comparison with the efficient tuning methods.}
We evaluate our method with full finetuning and only tuning the decoder, which are the widely-used strategies for down-streaming task adaption. And similar methods from image classification, \ie, VPT~\cite{vpt} and AdaptFormer~\cite{chen2022adaptformer}. 
The number of prompt tokens is set to 10 for VPT and the middle dimension of AdaptMLP is set to 2 for a fair comparison in terms of the tunable parameters.
It can be seen from Table~\ref{tab:sota_finetune} that when only tuning the decoder, the performance drops largely. Compared with similar methods, introducing extra learnable tokens~\cite{vpt} or MLPs in Transformer block~\cite{chen2022adaptformer} also benefits the performance. We introduce a hyper-parameter~($r$) which is used to control the number of parameters of the Adaptor as described in equation~\ref{eqn:fpe}. We first compare EVP~($r$=16) with similar parameters as other methods. From the table, our method achieves much better performance. We also report EVP~($r$=4), with more parameters, the performance can be further improved and outperforms full-tuning on 3 of 4 datasets. 


% % Figure SOTA
\begin{figure*}[!t]
  \captionsetup[subfigure]{position=b}
  \centering
  \setlength{\tabcolsep}{0pt}% Remove column gap in tabular
  \subcaptionbox{}{%
    \begin{tabular}{c}
      \includegraphics[width=0.1\textwidth, height=0.1\textwidth]{images/arch/input/91-14.png} \\[0.1em]
      \includegraphics[width=0.1\textwidth, height=0.1\textwidth]{images/arch/input/94-3.png} \\[0.1em]
      \includegraphics[width=0.1\textwidth, height=0.1\textwidth]{images/arch/input/126-3.png} \\
    \end{tabular}}%
  \hspace{0.0em}%
%   \vspace{0.0em}%
  \subcaptionbox{}{%
    \begin{tabular}{c}
      \includegraphics[width=0.1\textwidth, height=0.1\textwidth]{images/arch/gt/91-14.png} \\[0.1em]
      \includegraphics[width=0.1\textwidth, height=0.1\textwidth]{images/arch/gt/94-3.png} \\[0.1em]
      \includegraphics[width=0.1\textwidth, height=0.1\textwidth]{images/arch/gt/126-3.png} \\
    \end{tabular}}%
  \hspace{0.0em}%
  \subcaptionbox{}{%
    \begin{tabular}{c}
      \includegraphics[width=0.1\textwidth, height=0.1\textwidth]{images/arch/fulltune/91-14.png} \\[0.1em]
      \includegraphics[width=0.1\textwidth, height=0.1\textwidth]{images/arch/fulltune/94-3.png} \\[0.1em]
      \includegraphics[width=0.1\textwidth, height=0.1\textwidth]{images/arch/fulltune/126-3.png} \\
    \end{tabular}}%
  \hspace{0.0em}%
  \subcaptionbox{}{%
    \begin{tabular}{c}
      \includegraphics[width=0.1\textwidth, height=0.1\textwidth]{images/arch/decoder/91-14.png} \\[0.1em]
      \includegraphics[width=0.1\textwidth, height=0.1\textwidth]{images/arch/decoder/94-3.png} \\[0.1em]
      \includegraphics[width=0.1\textwidth, height=0.1\textwidth]{images/arch/decoder/126-3.png} \\
    \end{tabular}}%
  \hspace{0.0em}%
  \subcaptionbox{}{%
    \begin{tabular}{c}
      \includegraphics[width=0.1\textwidth, height=0.1\textwidth]{images/arch/pred_without_embedding_tune/91-14.png} \\[0.1em]
      \includegraphics[width=0.1\textwidth, height=0.1\textwidth]{images/arch/pred_without_embedding_tune/94-3.png} \\[0.1em]
      \includegraphics[width=0.1\textwidth, height=0.1\textwidth]{images/arch/pred_without_embedding_tune/126-3.png} \\
    \end{tabular}}%
  \hspace{0.0em}%
  \subcaptionbox{}{%
    \begin{tabular}{c}
      \includegraphics[width=0.1\textwidth, height=0.1\textwidth]{images/arch/pred_without_hfc_tune/91-14.png} \\[0.1em]
      \includegraphics[width=0.1\textwidth, height=0.1\textwidth]{images/arch/pred_without_hfc_tune/94-3.png} \\[0.1em]
      \includegraphics[width=0.1\textwidth, height=0.1\textwidth]{images/arch/pred_without_hfc_tune/126-3.png} \\
    \end{tabular}}%
  \hspace{0.0em}%
    \subcaptionbox{}{%
    \begin{tabular}{c}
      \includegraphics[width=0.1\textwidth, height=0.1\textwidth]{images/arch/pred_with_shared_tunemlp/91-14.png} \\[0.1em]
      \includegraphics[width=0.1\textwidth, height=0.1\textwidth]{images/arch/pred_with_shared_tunemlp/94-3.png} \\[0.1em]
      \includegraphics[width=0.1\textwidth, height=0.1\textwidth]{images/arch/pred_with_shared_tunemlp/126-3.png} \\
    \end{tabular}}%
  \hspace{0.0em}%
  \subcaptionbox{}{%
    \begin{tabular}{c}
      \includegraphics[width=0.1\textwidth, height=0.1\textwidth]{images/arch/pred_with_unshared_upmlp/91-14.png} \\[0.1em]
      \includegraphics[width=0.1\textwidth, height=0.1\textwidth]{images/arch/pred_with_unshared_upmlp/94-3.png} \\[0.1em]
      \includegraphics[width=0.1\textwidth, height=0.1\textwidth]{images/arch/pred_with_unshared_upmlp/126-3.png} \\
    \end{tabular}}%
  \hspace{0.0em}%
  \subcaptionbox{}{%
    \begin{tabular}{c}
      \includegraphics[width=0.1\textwidth, height=0.1\textwidth]{images/arch/pred/91-14.png} \\[0.1em]
      \includegraphics[width=0.1\textwidth, height=0.1\textwidth]{images/arch/pred/94-3.png} \\[0.1em]
      \includegraphics[width=0.1\textwidth, height=0.1\textwidth]{images/arch/pred/126-3.png} \\
    \end{tabular}}%
  \hspace{0.0em}%
\caption{Quantitative comparison using full-tuning and different prompting designs on ISTD~\cite{wang2018stacked} dataset for shadow detection. From the left to right is: (a)~Input, (b)~GT, (c)~Full-tuning, (d)~Decoder~(No prompting), (e)~Ours w/o $F_{pe}$, (f)~Ours w/o $F_{hfc}$, (g)~Ours w/ Shared $\mathtt{MLP^i_{tune}}$, (h)~Ours w/ Unshared $\mathtt{MLP_{up}}$, (i)~Ours Full.}
\label{fig:ablation_arch}
\end{figure*}


\subsection{Ablation Study}
\label{sec:ablation_study}
We conduct the ablation to show the effectiveness of each component. The experiments are performed with the scaling factor $r=4$ except specified.

\paragraph{Architecture Design.}
To verify the effectiveness of the proposed visual prompting architecture, we modify it into different variants. As shown in Table~\ref{tab:arch} and Figure~\ref{fig:ablation_arch}, sharing $\mathtt{MLP^i_{tune}}$ in different Adaptors only saves a small number of parameters~(0.55M \textit{v.s.} 0.34M) but leads to a significant performance drop. It cannot obtain consistent performance improvement when using different $\mathtt{MLP_{up}}$ in different Adaptors, moreover introducing a large number of parameters~(0.55M \textit{v.s.} 1.39M). On the other hand, the performance will drop when we remove $F_{pe}$ or $F_{hfc}$, which  means that they are both effective visual prompts.


\paragraph{Tuning Stage.}
We try to answer the question: which stage contributes mostly to prompting tuning? Thus, we show the variants of our tuning method by changing the tunable stages in the SegFormer backbone. SegFormer contains 4 stages for multi-scale feature extraction. We mark the Stage$_{x}$ where the tunable prompting is added in Stage $x$. Table~\ref{tab:tuning_stage} shows that better performance can be obtained via the tunable stages increasing. Besides, the maximum improvement occurs in Stage$_{1,2}$ to Stage$_{1,2,3}$. Note that the number of transformer blocks of each stage in SegFormer-B4 is 3, 8, 27, and 3, respectively. Thus, the effect of EVP is positively correlated to the number of the prompted transformer blocks.


\paragraph{Scale Factor $r$~(equation \ref{eqn:fpe}).}
We introduce $r$ in Sec~\ref{sec:explicit_visual_prompting} of the main paper to control the number of learnable parameters. A larger $r$ will use fewer parameters for tuning. As shown in Table~\ref{tab:model_size}, the performance improves on several tasks when $r$ decreases from 64 to 4; when $r$ continues to decrease to 2 or 1, it can not gain better performance consistently even if the model becomes larger. It indicates that $r=4$ is a reasonable choice to make a trade-off between the performance and model size. 



\paragraph{EVP in Plain ViT.}
We experiment on SETR~\cite{zheng2021rethinking} to confirm the generalizability of EVP. SETR employs plain ViT as the backbone and a progressive upsampling ConvNet as the decoder, while SegFormer has a hierarchical backbone with 4 stages. Therefore, the only distinction between the experiments using SegFormer is that all modifications are limited to the single stage in plain ViT. The experiments are conducted with ViT-Base~\cite{dosovitskiy2020image} pretrained on the ImageNet-21k~\cite{imagenet} dataset. The number of prompt tokens is set to 10 for VPT, the middle dimension of AdaptMLP is set to 4 for AdaptFormer, and $r$ is set to 32 for our EVP. As shown in Table~\ref{tab:setr}, EVP also outperforms other tuning methods when using plain ViT as the backbone.

