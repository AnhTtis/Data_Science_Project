\documentclass[preprint,12pt]{elsarticle}
%% Use the option review to obtain double line spacing
% \documentclass[preprint,review,12pt]{elsarticle}
\usepackage{amsmath,amssymb,amsthm}
\usepackage{hyperref}
\usepackage{enumitem}
\usepackage{theoremref}
\usepackage{float}
\usepackage{physics}
\usepackage{lineno}                                          % line numbers

% % Macros
% \DeclareMathOperator{\co}{Co}                               
% \DeclareMathOperator{\diag}{diag}

% Theorem-like environments
\newtheorem{theorem}{Theorem}[section]
\newtheorem{proposition}[theorem]{Proposition}
\newtheorem{lemma}[theorem]{Lemma}
\newtheorem{corollary}[theorem]{Corollary}

\theoremstyle{definition}           
\newtheorem{mydef}{Definition}
\newtheorem{remark}[theorem]{Remark}
\newtheorem{example}[theorem]{Example}

\journal{Linear Algebra and its Applications}

\begin{document}

\begin{frontmatter}

%% Title, authors and addresses

%% use the tnoteref command within \title for footnotes;
%% use the tnotetext command for theassociated footnote;
%% use the fnref command within \author or \address for footnotes;
%% use the fntext command for theassociated footnote;
%% use the corref command within \author for corresponding author footnotes;
%% use the cortext command for theassociated footnote;
%% use the ead command for the email address,
%% and the form \ead[url] for the home page:
\title{Kellogg's eigenvalue inequality for $P$ and $P_0$ matrices}
% \tnotetext[label1]{Supported by NSF Award \href{https://www.nsf.gov/awardsearch/showAward?AWD_ID=2150511}{DMS-2150511}.}

\author[uwb1]{Devon N.~Munger}
\ead{mungerd@uw.edu}
\affiliation[uwb1]{
    organization={University of Washington Bothell},
    addressline={18115 Campus Way NE},
    city={Bothell},
    postcode={98011-8246},
    state={WA},
    country={U.S.A.}}

\author[uwb2]{Pietro Paparella\corref{cor1}}
\ead{pietrop@uw.edu}
\ead[url]{http://faculty.washington.edu/pietrop/}
\cortext[cor1]{Corresponding author.}
\affiliation[uwb2]{
    organization={Division of Engineering \& Mathematics},
    addressline={University of Washington Bothell},
    city={Bothell},
    postcode={98011-8246},
    state={WA},
    country={U.S.A.}}

\begin{abstract}
In this work, the converse of the Cowling--Thron--Obrechkoff theorem is established. In addition to its obvious theoretical interest, the result fills a gap in the proof of Kellogg's celebrated eigenvalue inequality for matrices whose principal minors are positive or nonnegative.
\end{abstract}

\begin{keyword}
%% keywords here, in the form: keyword \sep keyword
eigenvalue inequality \sep $P$ matrix \sep $P_0$ matrix

%% MSC codes here, in the form: \MSC code \sep code
%% or \MSC[2008] code \sep code (2000 is the default)
\MSC[2020] 15A42 \sep 26C10

\end{keyword}
\end{frontmatter}

% \linenumbers % switch for line numbers

%---------------------
\section{Introduction}

An $n$-by-$n$ matrix $A$ with complex entries is called a \emph{$P$ matrix} (respectively, \emph{$P_0$ matrix}) if each of its principal minors is positive (respectively, nonnegative). This class of matrices was introduced by Fiedler and Pt\'{a}k \cite{fp1966} as a common generalization of the class of \emph{$M$ matrices} and the class of \emph{positive definite matrices}.  

In 1972, Kellogg \cite[Corollary 1]{k1972} offered the following result. 

%--------------
\begin{theorem}
[Kellogg]
\thlabel{kell}
If $\lambda = r(\cos\theta + \sin\theta i)\in \mathbb{C}$, with $\theta \in (-\pi,\pi]$, then $\lambda$ is an eigenvalue of a $P$ matrix if and only if 
\begin{equation}
    \label{eigineq}
        \vert \theta - \pi \vert > {\pi}/{n}.     
\end{equation}
A nonzero $\lambda$ is an eigenvalue of an $n$-by-$n$ $P_0$ matrix if and only if \eqref{eigineq} holds with equality allowed.
\end{theorem}

However, as will be explained in the sequel, Kellogg's proof is incomplete (see Remark following Theorem \ref{kelltheorem4}) and requires the converse of the following theorem, which was established by Cowling and Thron in 1954 \cite[Theorem 4.1]{ct1954}; is a consequence of a more general theorem established by Obrechkoff in 1923 \cite{o1923}; and was recently rediscovered by Melman \cite{m2021,m2022}.

%--------------
\begin{theorem}
[Cowling--Thron--Obrechkoff]
    \label{cto}
    If $q$ is a polynomial of degree $n$ with nonnegative coefficients such that $q(0) \ne 0$ and $\lambda = r(\cos\theta + \sin\theta i)$ is a zero of $q$, then $\vert \theta \vert > {\pi}/{n}$, unless $q$ is of the form $q(t) = a_0 + a_n t^n$, in which case it has a zero satisfying $\vert \theta \vert = {\pi}/{n}$ (if $n>1$, then $q(t) = a_0 + a_n t^n$ has a conjugate pair of zeros satisfying $\vert \theta \vert = {\pi}/{n}$).
\end{theorem}

If $\lambda = r(\cos\theta + \sin\theta i)$ is a nonzero eigenvalue of a $P$ or $P_0$ matrix, with $\theta \in (-\pi,\pi]$, then it can be shown that $-\lambda = r(\cos(\theta-\pi) + \sin(\theta-\pi) i)$ is a zero of a polynomial with nonnegative coefficients (see Section \ref{necexsec}). Hence, by Theorem \ref{cto}, $\vert \theta - \pi \vert \ge \pi/n$. 

Reversing the steps in the preceding argument requires the converse of Theorem \ref{cto}, which is noticeably absent in Kellogg's proof of \thref{kell}.     

%--------------
\begin{theorem}
    \thlabel{convcto}
    If $\lambda = r(\cos\theta + \sin\theta i) \in \mathbb{C}$ and $\vert \theta \vert \ge \pi/n$, then there is a polynomial $q$ of degree $n$ with nonnegative coefficients such that $q(\lambda) = 0$ and $q(0) \ne 0$.
\end{theorem}

The purpose of this work is to establish Theorem \ref{convcto} and to simplify other demonstrations given by Kellogg \cite{k1972}. To the best of our knowledge, Theorem \ref{convcto}, which is of interest to a general mathematical audience, is novel and has not appeared appeared in the literature. 

%-----------------------------
\section{Preliminary Results}
\label{necexsec}

Suppose that $A$ is an $n$-by-$n$ matrix with complex entries and eigenvalues $\lambda_1,\ldots,\lambda_n$ (repetitions included). Recall that if $E_k(A)$ denotes the sum of the $\binom{n}{k}$ \emph{principal minors of size $k$} \cite[p.~17]{hj2013} and $p_A$ denotes the characteristic polynomial of $A$, then 
\[ p_A(t) := \det(tI-A) = \prod_{k=1}^n (t - \lambda_k) = \sum_{k=0}^{n-1} (-1)^{n-k} E_{n-k}(A) t^k + t^n\]
(see, e.g., Horn and Johnson \cite[p.~53]{hj2013}). If
\begin{equation}
    \label{qpoly}
    q_A (t) := (-1)^n p_A(-t) = \prod_{k=1}^n (t + \lambda_k) = \sum_{k=0}^{n-1} E_{n-k}(A) t^k + t^n,
\end{equation} 
then $p_A(\lambda) = 0$ if and only if $q_A(-\lambda)=0$. 

The following result is immediate. 

%--------------
\begin{theorem}
    \label{thm:pmatrix}
    If $A$ is an $n$-by-$n$ matrix with complex entries, then $A$ is a $P$ matrix (respectively, $P_0$ matrix) if and only if the polynomial $q_A$ as defined in \eqref{qpoly} has positive (respectively, nonnegative) coefficients.
\end{theorem}  

Kellogg \cite[Theorem 4]{k1972} gave a necessary and sufficient condition on a multiset of complex numbers to be the spectrum of a $P$ matrix or $P_0$ matrix. We utilize the companion matrix to give a simpler proof.  

%--------------
\begin{theorem}
    \label{kelltheorem4}
    If $\Lambda = \{ \lambda_1,\ldots,\lambda_n \}$ is a multiset of complex numbers, then $\Lambda$ is the spectrum of a $P$ matrix (respectively, $P_0$ matrix) if and only if the polynomial
        \begin{equation}
            \label{qpoly2}
            q(t) := \prod_{k=1}^n (t+\lambda_k) = \sum_{k=0}^n c_k t^k    
        \end{equation}
    has positive (respectively, nonnegative) coefficients.
\end{theorem}

\begin{proof}
Necessity is established above.

For sufficiency, suppose that $\Lambda = \{ \lambda_1,\ldots,\lambda_n \}$ is a multiset of complex numbers and that the polynomial $q$ defined in \eqref{qpoly2} has positive (respectively, nonnegative) coefficients. If $p(t) := (-1)^n q(-t)$, then 
\begin{align*}
p(t) 
&= (-1)^n \sum_{k=0}^n c_k (-t)^k           \\
&= (-1)^n \sum_{k=0}^n c_k (-1)^k t^k       \\
&= (-1)^n \sum_{k=0}^n c_k (-1)^{-k} t^k    \\
&= \sum_{k=0}^n c_k (-1)^{n-k} t^k,
\end{align*}
i.e., $p$ is a monic polynomial. If $C$ is the \emph{companion matrix} \cite[pp.~194--195]{hj2013} of $p$, then $p_C = p$ and $q_C = q$. By Theorem \ref{thm:pmatrix}, $C$ is a $P$ matrix (respectively, $P_0$ matrix) with the desired spectrum.
\end{proof}

%-------------
\begin{remark}
Kellogg \cite[p.~174]{k1972} asserts that the converse of \thref{kell} follows from Theorem \ref{kelltheorem4} alone. However, if $\lambda$ satisfies \eqref{eigineq}, it is not clear how to generate a multiset $\Lambda$ such that \eqref{qpoly2} holds. This observation motivates and necessitates \thref{convcto}.  
\end{remark}

%----------------------
\section{Main Results}

In this section, we establish Theorems \ref{kell} and \ref{convcto}. To this end, we require several ancillary results.  

% %------------
% \begin{lemma}
% \label{trigid}
%     If $n \ge 2$, then 
%     \[\sin n \theta = 2 \cos \theta \sin (n-1) \theta - \sin (n-2) \theta. \] 
% \end{lemma}

% \begin{proof}
% Since 
% \[ 2 \cos x \sin y = \sin(x + y) - \sin(x-y), \]
% it follows that
% \begin{align*}
%     &2 \cos \theta \sin (n-1) \theta - \sin (n-2) \theta        \\
%     &= \sin n \theta - \sin(- (n-2) \theta) - \sin (n-2) \theta \\
%     &= \sin n \theta. \qedhere 
% \end{align*}
% \end{proof}

%------------
\begin{lemma}
\label{sinntheta}
    If $n \ge 2$ and \(\theta \in \left[\frac{\pi}{n},\frac{\pi}{n-1} \right) \), then 
    \(\sin n \theta \le 0\)   
    and 
    \( \sin (n-1) \theta > 0. \) 
\end{lemma}

\begin{proof}
If \(\theta \in \left[\frac{\pi}{n},\frac{\pi}{n-1} \right) \), then \( n \theta \in \left[\pi, \frac{n \pi}{n-1} \right) \) and because \(\frac{n \pi}{n - 1}\leq 2\pi\), it follows that $n \theta \in [\pi, 2\pi)$. Thus, $\sin n\theta \leq 0$. 

Similarly, \( (n-1)\theta \in \left[\frac{(n-1)\pi}{n},\pi\right) \) and because \( \frac{(n-1)\pi}{n} > 0\), it follows that $(n-1)\theta \in (0, \pi)$. Thus, $\sin(n-1) \theta > 0$.
\end{proof}

% %------------
% \begin{lemma}
%     \label{lem:qpoly}
%     If $n \geq 2$, then
%     \[ q(t) \sum_{j=1}^{n-1} \sin(n-j) \theta t^{j-1} = \sin \theta t^n - \sin n \theta t + \sin(n-1) \theta, \]
%     where $q(t) := t^2 - 2 \cos \theta t + 1$.
% \end{lemma}

% \begin{proof}
% Proceed via induction on $n$. If $n=2$, then
% \begin{align*}
%     q(t) \sum_{j=1}^{1} \sin (2-j) \theta t^{j-1} 
%     &= q(t) \sin\theta                                              \\
%     &= \sin \theta t^2 - 2 \cos \theta \sin \theta t + \sin\theta   \\
%     &= \sin \theta t^2 - \sin 2 \theta t + \sin\theta,
% \end{align*}
% which establishes the base case. 

% For the induction-step, if
% \begin{align*}
%     q(t) \sum_{\ell=1}^{k - 1} \sin(k-\ell) \theta t^{\ell-1} = \sin \theta t^{k} - \sin k\theta t + \sin(k-1)\theta, \tag{IH} \label{IH}
% \end{align*}
% where $k \ge 2$, then 
% \begin{align*}
% &q(t) \sum_{j=1}^{k} \sin(k+1-j) \theta t^{j-1}                                                                     \\
% &= q(t) \sin(k \theta) + q(t) \sum_{j=2}^{k} \sin(k+1-j) \theta t^{j-1}                                             \\
% &= q(t)\sin(k \theta) + q(t)\sum_{\ell=1}^{k-1} \sin(k-\ell) \theta t^\ell              \tag{$\ell := j - 1$}       \\
% &= q(t)\sin(k \theta) + t q(t) \sum_{\ell=1}^{k-1} \sin(k-\ell) \theta t^{\ell-1}                                   \\
% &= \sin k \theta t^2 - 2\cos\theta\sin k\theta t + \sin k\theta                                                     \\
% & \quad + t(\sin \theta t^{k} - \sin k\theta t + \sin(k-1)\theta) \tag{by \eqref{IH}}                               \\
% &= \sin \theta t^{k+1} - (2\cos\theta\sin k\theta - \sin(k-1)\theta)t + \sin k\theta                                \\
% &= \sin \theta t^n - \sin(k+1)\theta t + \sin k\theta,                                  \tag{by Lemma \ref{trigid}}
% \end{align*}
% i.e., the result holds when $n= k + 1$. The entire result follows by the principle of mathematical induction. 
% \end{proof}

In order to motivate the next result, if $\lambda = r(\cos\theta + \sin \theta i)$, with $\theta \in [\frac{\pi}{2},\pi)$, then $\sin\theta \ne 0$ and 
\begin{align*}
    (t - \lambda)(t - \bar{\lambda})
    &= t^2 - 2 \Re\lambda t + \lambda\bar{\lambda}                                  \\ 
    &= t^2 - 2r \cos\theta t + r^2                                                  \\
    &= t^2 - \frac{\sin2\theta}{\sin\theta} r t + r^2 \frac{\sin\theta}{\sin\theta}.
\end{align*}
Since $\sin\theta > 0$ and $\cos\theta \le 0$, the polynomial $(t - \lambda)(t - \bar{\lambda})$ has nonnegative coefficients. This observation generalizes as follows.

%--------------
\begin{theorem}
\label{thm:qnpoly}
    Suppose that $\lambda = r(\cos \theta + i \sin \theta)$, with $\theta \in \left[\frac{\pi}{n},\frac{\pi}{n-1} \right)$ and $n\geq2$. If 
        \begin{align}
            \label{qmonic}
            q_n(t) := t^n - \frac{\sin n \theta}{\sin \theta} r^{n-1} t + \frac{\sin (n-1) \theta}{\sin \theta} r^n,     
        \end{align}
    then $q_n$ has nonnegative coefficients, $q_n (0) \ne 0$, and $q_n(\lambda) = 0$.   
\end{theorem}

\begin{proof}
    First, notice that $\sin \theta > 0$ because $\theta \in [\frac{\pi}{n},\frac{\pi}{n-1})$ and $n \ge 2$. By Lemma \ref{sinntheta}, $\sin (n \theta) \le 0$ and $\sin(n-1) \theta > 0$. Thus, $q_n$ has nonnegative coefficients and $q_n (0) \ne 0$.
    
    Finally, notice that 
    \begin{align*}
        & ~~~~~q_n(\lambda)/r^n \\
        &= \cos n\theta + \sin n\theta i - \frac{\sin n\theta(\cos\theta + \sin\theta i)}{\sin\theta} + \frac{\sin (n-1)\theta}{\sin\theta}             \\
        &= \frac{\sin\theta\cos n\theta + \sin\theta\sin n\theta i - \cos\theta\sin n\theta - \sin\theta\sin n\theta i + \sin (n-1)\theta}{\sin\theta}  \\
        &= \frac{\sin\theta\cos n\theta - \cos\theta\sin n\theta + \sin (n-1)\theta}{\sin\theta}                                                        \\
        &= \frac{\sin(\theta - n\theta) + \sin (n-1)\theta}{\sin\theta}                                                                                 \\
        &= \frac{\sin(-(n-1)\theta) + \sin (n-1)\theta}{\sin\theta}                                                                                     \\
        &=0,
    \end{align*}
i.e., $q_n(\lambda) = 0$.
\end{proof}

% %--------------
% \begin{theorem}
% \label{thm:qnpoly}
%     Suppose that $\lambda = r(\cos \theta + i \sin \theta)$, with $\theta \in \left[\frac{\pi}{n},\frac{\pi}{n-1} \right)$, and $n\geq2$. If 
%     \begin{equation}
%         q_n(t) := \sin \theta t^n - \sin n\theta r^{n-1}t + \sin (n-1)\theta r^{n}, \label{qnpoly}
%     \end{equation} 
%     then $q_n$ is a polynomial of degree $n$ with nonnegative coefficients such that $q_n(\lambda) = 0$ and $q_n (0) \ne 0$.   
% \end{theorem}

% \begin{proof}
% First, notice that $\sin \theta > 0$ because $n \ge 2$ and $\theta \in [\frac{\pi}{n},\frac{\pi}{n-1})$. By Lemma \ref{sinntheta}, $\sin (n \theta) \le 0$ and $\sin(n-1) \theta > 0$.

% If $r = 1$, then $\lambda = \cos \theta + i \sin \theta$ and
% \begin{align*}
%     (t-\lambda)(t - \bar{\lambda}) = t^2 - 2 \Re \lambda t + |\lambda|^2 = t^2 - 2\cos\theta t + 1. 
% \end{align*}
% Combining this with Lemma \ref{lem:qpoly}, we see that $\lambda$ is a zero the polynomial  
% \begin{equation}
%     \sin\theta t^n - \sin n \theta t + \sin(n-1) \theta, \label{qmod1}
% \end{equation}
% which corresponds to \eqref{qnpoly} with $r=1$. Thus, the result follows for complex numbers having unit modulus. 

% If $r > 0$, then $\lambda / r$ is a zero of the polynomial given in \eqref{qmod1}, i.e.,
% \[ \frac{\sin\theta\lambda^n}{r^n} - \frac{\sin n \theta\lambda}{r} + \sin(n-1) \theta = 0. \]
% Multiplying by $r^n$ yields
% \begin{equation}
% \label{qnpoly2}
%     \sin\theta\lambda^n - \sin n \theta r^{n-1} \lambda + \sin (n-1) \theta r^n = 0,    
% \end{equation}
% i.e., $q_n(\lambda) = 0$. 
% \end{proof}

%-------------
\begin{remark}
    It is worth noting that if $q_n$ is the polynomial defined in \eqref{qmonic}, then 
\begin{equation*}
    {q}_n(t) = t^n - U_{n-1}(\cos\theta) r^{n-1} t + U_{n-2}(\cos\theta) r^n,
\end{equation*}
where $U_k$ denotes the \textit{Chebyshev polynomial of the second kind of degree $k$}.
\end{remark}

We are now ready to prove Theorems \ref{convcto} and \thref{kell}.

\begin{proof}
    [Proof of Theorem \ref{convcto}]
        It is sufficient to show that $\lambda$ is a zero of a polynomial of degree $k$, $1 \le k \le n$, with the desired properties, since we can multiply it by $t^{n-k} + 1$ to obtain a polynomial of degree $n$ having the desired properties.
        
        If $n=1$ and $\vert \theta \vert \ge \pi$, then $\theta = \pi$, $\lambda < 0$, and the desired polynomial is $t - \lambda$. Otherwise, assume that $n > 1$ and $\vert \theta \vert \ge \pi/n$. Again, if $\theta = \pi$, then $t - \lambda$ is the desired polynomial. Otherwise, it may be assumed, without loss of generality, that $\pi/n \le \theta < \pi$ since zeros of polynomials with real coefficients appear in conjugate pairs. Notice that $\exists k \in \mathbb{N}$, with $2 \leq k \leq n$, such that $\theta \in [\frac{\pi}{k},\frac{\pi}{k-1})$. By Theorem \ref{thm:qnpoly}, the polynomial $q_k$ defined in \eqref{qmonic} has the desired properties.
\end{proof}

\begin{proof}
    [Proof of \thref{kell}]
        Suppose that $\lambda$ is an eigenvalue of an $n$-by-$n$ $P$ matrix ($P_0$ matrix) $A$. For contradiction, if $\vert \theta - \pi \vert < \pi/n$, then $q_A(-\lambda) = 0$. By Theorem \ref{thm:pmatrix}, $q_A$ has nonnegative coefficients. By Theorem \ref{cto}, $\vert \theta - \pi \vert \ge \pi/n$, a contradiction. 
        
        The converse follows from Theorems \ref{convcto} and \ref{kelltheorem4}. 
\end{proof}

% \begin{acknowledgment}{Acknowledgment.}
% The authors wish to thank the Greek polymath Anonymous, whose prolific works are an endless source of inspiration.
% \end{acknowledgment}

\bibliographystyle{abbrv}
\bibliography{refs}

\end{document}