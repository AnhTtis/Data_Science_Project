%%% ---- PACKAGES ---- %%%

%% Sets page size and margins
    % \usepackage[a4paper,top=3cm,bottom=2cm,left=3cm,right=3cm,marginparwidth=1.75cm]{geometry}

%% Language and font encodings
    % \usepackage[english]{babel}
    % \usepackage[utf8x]{inputenc}
    % \usepackage[T1]{fontenc}
    % \usepackage{CJKutf8} % Japanese text
    
%% Graphics
	\usepackage{graphicx}
	\usepackage{adjustbox}

	\graphicspath{{../../tikz/}{figures/}}
    % \usepackage{graphicx}
    % \usepackage{xcolor}     % for colour
    % \usepackage[colorinlistoftodos]{todonotes}
    %\usepackage[colorlinks=true, allcolors=blue]{hyperref} % For colored clickable hyperlinks
    % \usepackage[colorlinks=true, linkcolor=black]{hyperref}
    % \usepackage{tikz}
    % \usepackage{graphics}   % For resizebox
    % \usepackage{adjustbox}  % For \adjincludegraphics
    % \usepackage{subcaption}
    \usepackage{orcidlink}
    \usepackage[font=scriptsize, textfont=sf]{subcaption}
	%\usepackage[hidelinks]{hyperref}
	\usepackage{cleveref}

    \usepackage{xcolor}
    \usepackage{hhline}
    \usepackage{tabularray}

%% Symbols
	% \usepackage{amsmath} 
    % \usepackage{amssymb}
    % \usepackage{pifont}  %for \xmark and \cmark
    % \newcommand{\cmark}{\ding{51}}%
    % \newcommand{\xmark}{\ding{55}}%
    % \usepackage{ulem}
    \usepackage{mathrsfs}   % For using mathscr{F}
    \usepackage{bm}         % For bold math symbols (eg: Greek). Use: \bm{}
    % \usepackage{nomencl}    % For nomenclatures
    % \usepackage{etoolbox}   % For nomenclatures
    % \usepackage{textcomp}   % For degree symbol \textdegree
    \usepackage{gensymb} % For degree symbol \degree
    \usepackage{steinmetz}  % For \phase notation
    % \usepackage{yfonts}     % For \textgoth
    % \usepackage{stmaryrd}   % For  \varoslash
    % \usepackage{eurosym}    % For euro symbol
    \usepackage{siunitx}    % for \si unit measures; also angles \ang
    % e.g.: \si{kg.m.s^{-1}} \ang{10}
    % \usepackage{accents}  % for \accentset{[m]}{H}
    \usepackage{rotate}
    
%% Equations
	%\usepackage{amsmath}
	\interdisplaylinepenalty=2500
	\usepackage{empheq}     % For equations with multiline labels (eg: (1a)-(1b)). Use: \begin{subequations}\begin{empheq}[left]{align}
	% \usepackage{stfloats}  % Double float fixes
	% \usepackage{array} % Enhanced arrays, tables, matrices
	%\usepackage{breqn}      % Breaks equation in lines (\begin{dmath} instead of equation). This shit never works
	% \numberwithin{equation}{section} % Equations are indexed by section
	% \usepackage{exscale,relsize}    % For \mathlarger{}

%% Tables
    \usepackage{makecell}   %Vertically aligned multilines (e.g. \makecell[l]{Line1\\Line2})
    \usepackage{multirow}   % For tables spanning more than one row/column. 
    \usepackage{tabu}
    \usepackage{arydshln}   % for cdashline
    \usepackage{threeparttablex}
    

%% Algorithms
    \usepackage[ruled,vlined]{algorithm2e}
    %\usepackage{algorithm}
    %\usepackage{algpseudocode}

%% Logic (and afterpage)
    \usepackage{xifthen}    % For \ifthenelse{\equal{#1}{}}{}{}
    \usepackage{url}
    % \usepackage{afterpage}
    % \afterpage{\clearpage}
    % \usepackage[section]{placeins}
    % \usepackage{float}
    %\restylefloat{table}
    %\newfloat{algorithm}{t}{lop}
    %\floatname{algorithm}{Algorithm} 
    % \usepackage{appendixnumberbeamer} % Put \appendix not to count backup slides

%% Glossary package
	\usepackage{cite}
    \usepackage[nopostdot,symbols,nogroupskip,nonumberlist]{glossaries-extra} % For glossaries
    % \usepackage{glossary-mcols}
    % \usepackage{morewrites} % Not enough writing space
    % \newglossary*{symbol}{Symbols}
    % \newglossarystyle{mcolalttreenoskip}{%
    %     \setglossarystyle{alttree}%
    %     \renewenvironment{theglossary}%
    %     {%
    %         \begin{multicols}{2}%
    %             \def\@gls@prevlevel{-1}%
    % %           \mbox{}\par
    %         }%
    %         {\par\end{multicols}}%
    % }
    % \makeglossaries % Makes glossaries
    % \renewcommand*{\glspostdescription}{}
    % \renewcommand{\glossarypreamble}{%
    %     \glsfindwidesttoplevelname[\currentglossary]%
    % }

%% Image Source
    % \newcommand{\sourceimg}[2]{
    %     \begin{tikzpicture}
    %         \node[] (fig) {
    %           #1
    %         };
    %         \node [anchor=north east,color=black!40!,inner sep=0,xshift=-5pt,yshift=0pt]
    %           at (fig.south east) {\tiny Source: #2};
    %     \end{tikzpicture}
    % }
    % \usepackage{stackengine}
    % \def\stackalignment{r}
    % \newcommand{\sourceimg}[2]{
    %     \stackunder{#1}%
    %               {\tiny%
    %                 \textcolor{black!40!}{Source: #2}}
    % }
     \newcommand{\sourceimg}[2]{
         #1\\%
         {\tiny \textcolor{black!40!}{Source: #2}}
     }

%% Gray Box
    % Makes a Gray box. Use: \begin{myframing}
    % \usepackage{ntheorem}   % for theorem-like environments
    % \usepackage{mdframed}   % for framing
    % \theoremstyle{break}
    % \theoremheaderfont{\bfseries}
    % \newmdtheoremenv[%
    % linecolor=gray,leftmargin=0,%
    % rightmargin=0,
    % backgroundcolor=gray!40,%
    % innertopmargin=0pt,%
    % ntheorem]{myframing}{Summary}[section]
    %\newtheorem{theorem}{Theorem}[section]
    
%%% ---- VARIABLES AND OPERATORS ---- %%%

%% Sets, Spaces, Varieties
    % \newcommand{\R}[1]{\in\mathbb{R}^{#1}} %short for R^{a} domains
    % \newcommand{\RR}[2]{\in\mathbb{R}^{#1\times#2}} % Short for R^{axb}
    % \newcommand{\I}{\mathcal{I}}
    %\newcommand{\I}{\mathbf{i}}
    % \newcommand{\J}{\mathcal{J}}
    % % \newcommand{\K}{\mathcal{K}}
    % \newcommand{\set}[1]{\mathscr{#1}}
    % \newcommand{\CC}{\mathbb{C}}
    % \newcommand{\RE}{\mathbb{R}}

%% Variables' formatting
    % \newcommand{\M}[1]{\mathbf{#1}}
    % \newcommand{\vect}[1]{\boldsymbol{#1}} % vectors
    % \newcommand{\matr}[1]{\boldsymbol{#1}} % matrices
    \newcommand{\tens}[1]{\boldsymbol{\mathcal{#1}}} %tensors
    % \newcommand{\tenselem}[1]{\mathcal{#1}}
    \newcommand\chem[1]{\ensuremath{\mathrm{#1}}} %Chemical symbols


%% Operators
    % \newcommand{\Su}[2]{\sum_{#1 =1}^{#2}} % Short for Sum operators starting from 1
    % \newcommand{\argmin}[1]{\mathop{\mathrm{argmin}}_{#1}}
    % \newcommand{\argmina}[1]{\arg\min_{#1}}
    %\newcommand{\argminb}[1]{\arg\!\min_{#1}}
    % \newcommand{\conc}{\mathbin{+\mkern-10mu+}}
    % \newcommand{\trace}{\mathop{\mathrm{trace}}}
    % \newcommand{\rank}{\mathop{\mathrm{rank}}}

    \newcommand{\prox}{\mathop{\mathrm{prox}}}
    % \newcommand{\proj}{\mathop{\mathrm{proj}}}
    % \newcommand{\prm}{\mathop{\mathrm{perm}}}
    % \newcommand{\sgn}{\mathop{\mathrm{sgn}}}
    % \newcommand{\Id}{\mathop{\mathrm{Id}}}
    % \newcommand{\tr}{\mathop{\mathrm{tr}}}

    \newcommand{\lexi}{\mathop{\mathrm{matr}}}
    \newcommand{\vct}{\mathop{\mathrm{vec}}}
    % \newcommand{\thres}{\mathop{\mathrm{thres}}}
    \newcommand{\rshp}{\mathop{\mathrm{reshape}}}
    \newcommand{\stck}{\mathop{\mathrm{stack}}}
    \newcommand{\std}{\mathop{\mathrm{std}}}
    % \newcommand{\cov}{\mathop{\mathrm{cov}}}
    % \newcommand{\var}{\mathop{\mathrm{var}}}
    \newcommand{\diag}{\mathop{\mathrm{diag}}}
    \newcommand{\lino}[1]{\mathop{\mathbb{#1}}}   % For linear operators
    % \newcommand{\shft}[1]{\check{#1}} % The shift operator for coordinates
    \newcommand{\nmlz}[1]{\widetilde{#1}} % The normalization operator
    
    % \renewcommand{\Re}{\operatorname{Re}}
    % \renewcommand{\Im}{\operatorname{Im}}
    
    \newcommand{\hadam}{\odot}       % Hadamard product
    %\newcommand{\hadam}{\boxdot}    % Hadamard product
    % \newcommand{\kron}{\otimes}      % Kronecker product
    %\newcommand{\kron}{\boxtimes}   % Kronecker product
    %\newcommand{\out}{\otimes}      % outer (tensor) product
    %\newcommand{\khatri}{\odot}     % Khatri-Rao product
    %\newcommand{\con}{\mathop{\bullet}}   % Contraction
    \newcommand{\cnv}{\;*\mkern-3mu *\;}        % Convolution Product
    \newcommand{\cnvo}{\;*\;}        % Lexicographic convolution product
    \newcommand{\T}{^{\textsf{T}}}     % Transposition
    % \newcommand{\fenc}{^\star}       % Fenchel conjugate
    % \newcommand{\masked}{^{\boxdot}} % Masked operator
    % \newcommand{\E}{\mathbb{E}}      % math expected value
    % \newcommand{\F}{\mathcal{F}}     % Fourier transform
    %\usepackage{mathrsfs}\newcommand{\F}{\mathscr{F}}    % Fourier transform symbol
    % \newcommand{\paral}{\mathbin{\!\backslash\mkern-5mu\backslash\!}}

    % \newcommand{\ratio}{\rho}       % Scale Ratio (Variable to define)
    \newcommand{\range}[3]
    {%
         \ifthenelse{\equal{#1}{}}{\left[#2,\,...\,,\,#3\right]}{#1\in\left[#2,\,...\,,\,#3\right]}%
    }
    %\newcommand{\range}[3]{#1\in[#2,...,#3]}   % For ranges           


%% Other macrocommands
    % \newcommand{\eqdef}{\stackrel{\rm def}{=}}     % equal by definition
    % \newcommand\numberthis{\addtocounter{equation}{1}\tag{\theequation}}

%% Colors
    % \newcommand{\dan}[1]{\textcolor{red}{#1}}
    % \newcommand{\tb}[1]{\textcolor{blue}{#1}}
    % \newcommand{\tg}[1]{\textcolor{green}{#1}}
    % \newcommand{\fin}[1]{\textcolor{black}{#1}}
    % \definecolor{purple}{rgb}{0.7,0,0.5}
    % \newcommand{\mj}[1]{\textcolor{purple}{#1}}
    % \newcommand{\mdm}[1]{\textcolor{orange}{#1}}
    % \newcommand{\an}[1]{\textcolor{brown}{#1}}


%% Switch-Case algorithm
    % \newcommand{\SWITCH}[1]{\STATE \textbf{switch} (#1)}
    % \newcommand{\ENDSWITCH}{\STATE \textbf{end switch}}
    % \newcommand{\CASE}[1]{\STATE \textbf{case} #1\textbf{:} \begin{ALC@g}}
    % \newcommand{\ENDCASE}{\end{ALC@g}}
    % \newcommand{\CASELINE}[1]{\STATE \textbf{case} #1\textbf{:} }
    % \newcommand{\DEFAULT}{\STATE \textbf{default:} \begin{ALC@g}}
    % \newcommand{\ENDDEFAULT}{\end{ALC@g}}
    % \newcommand{\DEFAULTLINE}[1]{\STATE \textbf{default:} }