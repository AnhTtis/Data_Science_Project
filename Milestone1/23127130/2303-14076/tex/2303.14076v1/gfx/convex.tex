\begin{figure}[!htbp]
    \captionsetup[subfigure]{justification=centering}
    \centering
     \begin{subfigure}[t]{0.49\linewidth}
        \includegraphics[trim={0 0 0 0},width=\linewidth]{lenses/biconvex/main.pdf}
        \caption{Biconvex lens}
        \label{fig:Convex_Convex}
    \end{subfigure}
    \begin{subfigure}[t]{0.49\linewidth}
        \includegraphics[trim={0 0 0 0},width=\linewidth]{lenses/biconcave/main.pdf}
        \caption{Biconcave lens}
        \label{fig:Convex_Concave}
    \end{subfigure}
    \caption[Concave and convex lenses]{Visual representation of convex and concave lenses. The focus $d_F$ represents the point of convergence of incident rays parallel to the optical axis coming from the left side. Thin lenses are characterized by only one principal plane.}
    \label{fig:Convex}
\end{figure}