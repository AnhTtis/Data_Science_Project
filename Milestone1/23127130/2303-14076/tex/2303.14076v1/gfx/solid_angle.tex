\begin{figure}
    \captionsetup[subfigure]{justification=centering}
    \centering
    
    \begin{subfigure}[t]{0.47\linewidth}
        \includegraphics[trim={10 10 10 10},width=0.99\linewidth]{figures/optics/Tikz__Solid_Angle.pdf}
        \caption{Solid angle in spherical coordinates}
        \label{fig:SolidAngle}
    \end{subfigure}
    \hfil
    \begin{subfigure}[t]{0.47\linewidth}
        \includegraphics[trim={10 10 10 10},width=0.99\linewidth]{figures/optics/Tikz__Radiance.pdf}
        \caption{Radiance over a detector surface}
        \label{fig:Radiance}
    \end{subfigure}
    \caption[Solid angle and radiance]{The figure on the left shows a representation of the elementary solid angle $d\Omega$ over an unitary sphere (bounded by the dashed lines). This visual representation shows how it's link with the variation $d\theta$ of the polar angle and $d\phi$ of the azimuth angle. On the right, a representation of the radiance incident vector over an elementary detection surface $d\mathcal{S}$ centered at $\mathbf{r}^{[0]}$.}
    \label{fig:OpticMeasures}
\end{figure}

% \begin{figure}[!htbp]
%     \centering
%     \includegraphics[width=0.6\linewidth]{figures/imspoc/Tikz__Solid_Angle.pdf}
%     \caption[Solid Angle Variation]{Representation of the elementary solid angle $d\Omega$ over an unitary sphere (bounded by the dashed lines). This visual representation shows how it's link with the variation $d\theta$ of the polar angle and $d\phi$ of the azimuth angle.}
%     \label{fig:SolidAngle}
% \end{figure}