\begin{figure}[!htbp]
    \captionsetup[subfigure]{justification=centering}
    \centering
	
	\begin{subfigure}[t]{0.49\linewidth}
		\includegraphics[trim={0 0 0 0}, clip, width=\linewidth]{figures/results/comparison_50.pdf}
		\caption{Reconstruction comparison}
		\label{fig:Spectral_IMSPOCUV1_GN2i150}
	\end{subfigure}
	\hfil
	\begin{subfigure}[t]{0.49\linewidth}
		\includegraphics[trim={0 0 0 0}, clip, width=\linewidth]{figures/results/gain_3.pdf}
		\caption{Gain estimation}
		\label{fig:curves_gain}
	\end{subfigure}
	
	\smallskip
	
	\begin{subfigure}[t]{0.49\linewidth}
		\includegraphics[trim={0 0 0 0}, clip, width=\linewidth]{figures/results/opd_3.pdf}
		\caption{\glsentryshort{opd} estimation}
		\label{fig:curves_opd}
	\end{subfigure}
	\hfil
	\begin{subfigure}[t]{0.49\linewidth}
		\includegraphics[trim={0 0 0 0}, clip, width=\linewidth]{figures/results/reflectivity_3.pdf}
		\caption{Reflectivity estimation}
		\label{fig:curves_reflectivity}
	\end{subfigure}
	
    %\includegraphics[width=0.3\linewidth]{figures/delta.png}
    \caption[Spectral calibration comparison]{Comparison between the observed interferogram (in blue) and its parametric reconstruction (in red) with various configurations for the \gls{p1} and \gls{p3}.}
    \label{fig:curves}
\end{figure}
