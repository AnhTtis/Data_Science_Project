\documentclass{optica-article}
\journal{oe}
\articletype{Research Article}


% \usepackage[latin1]{inputenc}
\usepackage[british]{babel}
\usepackage[all]{xy}
\usepackage{amscd}
\usepackage{amssymb}
\usepackage{amsthm}
\usepackage{enumitem}
\usepackage{mathrsfs,bbm}
\usepackage{xcolor,graphicx}
\usepackage{graphics}
\usepackage{soul}
\usepackage{comment}
\usepackage[all]{xy}
\usepackage{amscd}
\usepackage{amssymb,amsmath,latexsym}
\usepackage{amsthm}
\usepackage{enumitem}
\usepackage{mathrsfs,bbm}
\usepackage{dsfont}
\usepackage{tikz-cd}
\usepackage[T1]{fontenc}
\usepackage[utf8]{inputenc}  
 %
%%%%%%%%%%%%%%%%%%%%%%%%%%%%%%%%%%
%pagestyle
%%%%%%%%%%%%%%%%%%%%%%%%%%%%%%%%%%
%\pagestyle{plain}
\textwidth=430pt
\headsep=.7cm
\evensidemargin=15pt
\oddsidemargin=15pt
\leftmargin=0cm
\rightmargin=0cm
%%
%%%%%%%%%%%%%%%%%%%%%%%
\newcommand*\fixitem {\item[]%
  \refstepcounter{enumi}\hskip-\leftmargin\labelenumi\hskip\labelsep}
\newtheorem*{mainthm}{Main Theorem}
\newtheorem*{mainthm1}{Theorem}
\newtheorem*{maincor}{Corollary}
\usepackage[colorlinks=true]{hyperref}
\DeclareMathOperator{\Forall}{\forall}
\DeclareMathOperator{\Exists}{\exists}
\DeclareMathOperator{\ord}{ord}
\newcommand{\phiD}{\varphi_D}
\newcommand{\phiDI}{\varphi_{\mathbf{D}_I}}
\newcommand{\phiDIj}{\varphi_{\mathbf{D}_I (j)}}
\newcommand{\phiH}{\varphi_H}
\newcommand{\phiTimes}{\phiD \otimes \phiH}
\newcommand{\phiTimesDI}{\varphi_{\mathbf{D}_I} \otimes \phiH}
\newcommand{\R}{\mathscr{A}}
\newcommand{\X}{\mathscr{X}}
\newcommand{\Xf}{\mathscr{X}_{(k_0 ,i)}[r_0]}
\newcommand{\Xfr}{\mathscr{X}_{(k_0,i)}[r]}
\newcommand{\hotimes}{\widehat{\otimes}}
\newcommand{\C}{\mathbb{C}_p}
\newcommand{\V}{\mathscr{V}}
\newcommand{\B}{\mathscr{B}}
\newcommand{\dualD}{\mathfrak{D}}
\newcommand{\Dg}{\mathbf{D}}
\newcommand{\DD}{\mathcal{D}^0}
\newcommand{\DDg}{\mathcal{D}}
\newcommand{\DV}{\mathcal{D}}
\newcommand{\W}{\mathscr{W}_N}
\newcommand{\Ao}{\mathbf{A}^\circ}
\newcommand{\AoK}{\mathbf{A}^\circ_{\K}}
\newcommand{\AK}{\mathbf{A}_{/\K}}
\newcommand{\OOO}{\mathscr{A}^\circ}
\newcommand{\K}{\mathcal{K}} 
\newcommand{\OK}{\mathcal{O}_{\K}}
\newcommand{\varprojlog}[1]{\underleftarrow{\log\!^{#1}}}
\newcommand{\T}{\mathscr{T}}
\newcommand{\TT}{\mathbf{T}}
\newcommand{\VV}{\mathbf{V}}
\newcommand{\HH}{\mathcal{H}}
\newcommand{\hh}{\mathcal{H}^+}
\newcommand{\HG}[2]{\mathcal{H}_{#1}(#2)}
\newcommand{\hhl}{\mathcal{H}^{+,[l]}}
\newcommand{\hhj}{\mathcal{H}^{+,[j]}}
\newcommand{\hhjj}{\mathcal{H}^{+,[l,l']}}
\newcommand{\GS}{G_{\mathbb{Q},S}}
\newcommand{\Rf}{R_{(k_0 ,i)}[r_0]}
\newcommand{\Rfr}{R_{(k_0 ,i)}[r]}
\newcommand{\parT}{\langle T\rangle}
\newcommand{\Zf}{Z_{(k_0 ,i)}[r_0]}
\newcommand{\Zfr}{\mathscr{Z}_{(k_0 ,i)}[r]}
\newcommand{\ZFf}{\mathscr{Z}_{(k_0 ,i)}[r_0]}
\newcommand{\ZFfr}{\mathscr{Z}_{(k_0 ,i)}[r]}
\newcommand{\ZF}{\mathscr{Z}}
\glssetcategoryattribute{acronym}{indexonlyfirst}{true}
\newignoredglossary{ignored}
% \newglossary*{symbolsopt}{Optics Symbols}
% \newglossary*{symbolsinv}{Inversion Symbols}
% \newglossary*{symbolsop}{Operators}
% \newglossary*{symbolsfus}{Multimodal Variables}
% \newglossary*{symbolsint}{Interferometric Quantities}
% \newglossary*{symbolsspa}{Spaces and Dimensions}


\newabbreviation{hri}{HRI}{high resolution image}
\newabbreviation{lri}{LRI}{low resolution image}
\newabbreviation{gt}{GT}{ground truth}
\newabbreviation{pan}{PAN}{panchromatic}
\newabbreviation{ms}{MS}{multispectral}
\newabbreviation{hs}{HS}{hyperspectral}
\newabbreviation{hsi}{HSI}{hyperspectral imaging}
\newabbreviation{rgb}{RGB}{red-green-blue}
\newabbreviation{rgbw}{RGBW}{red-green-blue-white}
\newabbreviation{cym}{CYM}{cyan yellow magenta}
\newabbreviation{swir}{SWIR}{short wave infrared}
\newabbreviation{nir}{NIR}{near infrared response}
\newabbreviation{vis}{VIS}{visible}
\newabbreviation{uv}{UV}{ultraviolet}
\newabbreviation{ir}{IR}{infrared}
\newabbreviation{gsd}{GSD}{ground sample distance}

\newabbreviation{exp}{EXP}{interpolated image}
\newabbreviation{tps}{TPS}{thin plate spline}
\newabbreviation{idw}{IDW}{inverse distance weighting}
\newabbreviation{rbf}{RBF}{radial basis function}
\newabbreviation{tps-rbf}{TPS-RBF}{\glsentrylong{tps} \glsentrylong{rbf}}
\newabbreviation{gcv}{GCV}{generalized cross validation}
\newabbreviation{sure}{SURE}{Stein's unbiased risk estimate}
\newabbreviation{pchip}{PCHIP}{piecewise cubic Hermite interpolating polynomial}
\newabbreviation{tin}{TIN}{triangulated irregular network}

\newabbreviation{snr}{SNR}{signal to noise ratio}
\newabbreviation{mtf}{MTF}{modulation transfer function}

\newabbreviation{nn}{NN}{nearest neighbour}
\newabbreviation{cs}{CS}{component substitution}
\newabbreviation{mra}{MRA}{multiresolution analysis}
\newabbreviation{glp}{GLP}{generalized Laplacian pyramid}
\newabbreviation{wb}{WB}{weighted bilinear}
\newabbreviation{id}{ID}{intensity difference}
\newabbreviation{sd}{SD}{spectral difference}
\newabbreviation{iid2}{ItID}{iterative intensity difference}
\newabbreviation{isd}{ItSD}{iterative spectral difference}
\newabbreviation{ap}{AP}{alternating projections}
\newabbreviation{msg}{MSG}{multiscale gradients}
\newabbreviation{ri}{RI}{residual interpolation}
\newabbreviation{ari}{ARI}{adaptative residual interpolation}
\newabbreviation{mlri}{MLRI}{minimized-Laplacian residual interpolation}
\newabbreviation{btes}{BTES}{binary tree-based edge-Sensing}

\newabbreviation{bt2}{BrT}{Brovey transform}
\newabbreviation{gihs}{GIHS}{generalized intensity hue saturation}
\newabbreviation{gs}{GSA}{Gram-Schmidt}
\newabbreviation{gsa}{GSA}{Gram-Schmidt adaptive}
\newabbreviation{atwt}{ATWT}{à trous wavelet transform}
\newabbreviation{hm}{HM}{histogram matching}
\newabbreviation{cbd}{CBD}{context based decision}
\newabbreviation{hpm}{HPM}{high pass modulation}
\newabbreviation{mtf-glp}{MTF-GLP}{\glsentryshort{mtf}-matched \glsentrylong{glp}}
\newabbreviation{mtf-glp-cbd}{MTF-GLP-CBD}{\glsentryshort{mtf}-matched \glsentrylong{glp} with context based decision injection}
\newabbreviation{mtf-glp-hpm}{MTF-GLP-HPM}{\glsentryshort{mtf}-matched \glsentrylong{glp} with high pass modulation injection}
\newabbreviation{bdsd}{BDSD}{band-dependent spatial detail}
\newabbreviation{bayn}{BayesNaive}{Bayesian with naive regularization}
\newabbreviation{hysure}{HySure}{hyperspectral superresolution}
\newabbreviation{cnmf}{CNMF}{coupled nonnegative matrix factorization}
\newabbreviation{pca}{PCA}{principal component analysis}

\newabbreviation{psnr}{PSNR}{peak \glsentrylong{snr}}
\newabbreviation{scc}{sCC}{spatial cross-covariance coefficient}
\newabbreviation{ergas}{ERGAS}{relative dimensionless global error in synthesis}
%\newabbreviation{ergas}{ERGAS}{erreur relative globale adimensionnelle de synthèse}
\newabbreviation{sam}{SAM}{spectral angle mapper}
\newabbreviation{q2n}{\ensuremath{Q^2n}}{\ensuremath{Q2^n} index}
\newabbreviation{uiqi}{UIQI}{universal image quality index}
\newabbreviation{ssim}{SSIM}{structural similarity}
\newabbreviation{med}{MED}{mean Euclidean distance}

\newabbreviation{bin}{BIN}{radiometric binning}
\newabbreviation{cassi}{CASSI}{compressive coded aperture spectral imaging}
\newabbreviation{mrca}{MRCA}{multiresolution compressed acquisition}
\newabbreviation{jodefu}{JoDeFu}{joint demosaicing and fusion}
\newabbreviation{irca}{IRCA}{interferometer response characterization algorithm}


\newabbreviation{dmd}{DMD}{digital micromirror device}
\newabbreviation{doas}{DOAS}{differential optical absorption spectroscopy}

\newabbreviation{bt}{BT}{binary tree}
\newabbreviation{ubt}{UBT}{uniform \glsentrylong{bt}}
\newabbreviation{dbt}{DBT}{dominant \glsentrylong{bt}}
\newabbreviation{cove}{COVE}{\glsentrylong{pan} coverage}
\newabbreviation{peri}{PERI}{periodic}
\newabbreviation{diag}{DIAG}{diagonal}
\newabbreviation{vert}{VERT}{vertical}
\newabbreviation{btree}{BT}{binary tree}
\newabbreviation{rand}{RAND}{random pick}
\newabbreviation{mxdis}{MAXDIS}{maximum distance}
\newabbreviation{diri}{DIRI}{Dirichlet distribution based}
\newabbreviation[shortplural={s.v.}]{sv}{s.v.}{singular value}
\newabbreviation{svd}{SVD}{singular value decomposition}
\newabbreviation{pmd}{PMD}{penalized matrix decomposition}
\newabbreviation{l0}{L0}{level 0}
\newabbreviation{l1}{L1}{level 1}
\newabbreviation{l2}{L2}{level 2}
\newabbreviation{l3}{L3}{level 3}

\newabbreviation{pmf}{pmf}{probability mass function}
\newabbreviation{pdf}{pdf}{probability density function}
\newabbreviation{iid}{i.i.d.}{independent and identically distributed}
\newabbreviation{psd}{PSD}{power spectral density}
\newabbreviation{mse}{MSE}{mean square error}
\newabbreviation{rmse}{RMSE}{root \glsentrylong{mse}}
\newabbreviation{afn}{AFN}{average Fourier norm}
\newabbreviation{maue}{MAUE}{mean absolute unbiased error}
\newabbreviation{mae}{MAE}{mean absolute error}
\newabbreviation{awgn}{AWGN}{additive white Gaussian noise}
\newabbreviation{std}{STD}{standard deviation}

\newabbreviation{ml}{ML}{maximum likelihood}
\newabbreviation{es}{ES}{exhaustive search}
\newabbreviation{bf}{BF}{brute force}
\newabbreviation{gd}{GD}{gradient descent}
\newabbreviation{gn}{TRR}{trust region refinement}
\newabbreviation{lm}{LM}{Levenberg-Marquardt}
\newabbreviation{pinv}{PINV}{pseudo-inversion}
\newabbreviation{tsvd}{TSVD}{truncated \glsentrylong{svd}}
\newacronym{tsvdl}{TSVDL}{\glsentryshort{tsvd} with L-curve based parametric estimation}
\newabbreviation{gsvd}{GSVD}{generalized \glsentrylong{svd}}
\newabbreviation{tik}{RR}{ridge regression}
\newacronym{tikl}{RRL}{\glsentryshort{tik} with L-curve based parametric estimation}
\newabbreviation{wav}{WAV}{wavelet regularization}
\newabbreviation{ldct}{LDCT}{\glsentryshort{lasso}-\glsentryshort{dct}}
\newabbreviation{ldwt}{LDWT}{\glsentryshort{lasso}-\glsentryshort{dwt}}
\newabbreviation{ltv}{LTV}{\glsentryshort{lasso}-\glsentryshort{tv}}

\newabbreviation{gce}{GCE}{Gaussian-fit center estimation}
\newabbreviation{cce}{CCE}{centroid center estimation}
\newabbreviation{sce}{SCE}{scanline center estimation}
\newabbreviation{mle}{MLE}{maximum likelihood estimation}
\newabbreviation{rv}{r.v.}{random variable}
\newabbreviation{fft}{FFT}{fast Fourier transform}
\newabbreviation{dft}{DFT}{discrete Fourier transform}
\newabbreviation{dct}{DCT}{discrete cosine transform}
\newabbreviation{dwt}{DWT}{discrete wavelet transform}
\newabbreviation{swt}{SWT}{stationary wavelet transform}
\newabbreviation{qmf}{QMF}{quadrature mirror filter}
\newabbreviation{lpf}{LPF}{low pass filter}
\newabbreviation{fir}{FIR}{finite impulse response}
\newabbreviation{iir}{IIR}{infinite impulse response}
\newabbreviation{idct}{IDCT}{inverse \glsentrylong{dct}}
\newabbreviation{idwt}{IDWT}{inverse \glsentrylong{dwt}}
\newabbreviation{rip}{RIP}{restricted isometry property}
\newabbreviation{lasso}{LASSO}{least absolute shrinkage and selection operator}
\newabbreviation{omp}{OMP}{orthogonal matching pursuit}
\newabbreviation{admm}{ADMM}{alternating direction method of multipliers}

\newabbreviation{tv}{TV}{total variation}
\newabbreviation{rof}{ROF}{Rudin-Osher-Fatemi}
\newabbreviation{atv}{ATV}{anisotropic \glsentrylong{tv}}
\newabbreviation{itv}{ITV}{isotropic \glsentrylong{tv}}
\newabbreviation{vtv}{VTV}{vector \glsentrylong{tv}}
\newabbreviation{ctv}{CTV}{collaborative \glsentrylong{tv}}
\newabbreviation{stv}{STV}{Shannon \glsentrylong{tv}}
\newabbreviation{utv}{UTV}{upwind \glsentrylong{tv}}
\newabbreviation{tgv}{TGV}{total generalized variation}
\newabbreviation{als}{ALS}{alternating least squares}
\newabbreviation{mm}{MM}{majorization-minimization}

\newabbreviation{cfa}{CFA}{color filter array}
\newabbreviation{sfa}{SFA}{spectral filter array}
\newabbreviation{msfa}{MSFA}{\glsentrylong{ms} filter array}

\newabbreviation{cnn}{CNN}{convolutional neural network}
\newabbreviation{fts}{FTS}{Fourier transform spectrometer}
\newabbreviation{ftir}{FTIR}{Fourier transform infrared spectroscopy}
\newabbreviation{em}{EM}{electro-magnetic}
\newabbreviation{tem}{TEM}{transverse electro-magnetic}
\newabbreviation{te}{TE}{transverse electric}
\newabbreviation{tm}{TM}{transverse magnetic}
\newabbreviation{fsr}{FSR}{free spectral range}
\newabbreviation{fwhm}{FWHM}{full width at half maximum}
\newabbreviation{fpa}{FPA}{focal plane array}
\newabbreviation{fov}{FoV}{field of view}
\newabbreviation{aoa}{AoA}{angle of acceptance}
\newabbreviation{fp}{FP}{Fabry-Perot}
\newabbreviation{opd}{OPD}{optical path difference}
\newabbreviation{opl}{OPL}{optical path length}
\newabbreviation{ccd}{CCD}{charged coupled device}
\newabbreviation{cmos}{CMOS}{complementary metal-oxide-semiconductor}
\newabbreviation{sar}{SAR}{synthetic aperture radar}
\newabbreviation{lidar}{LIDAR}{light detection and ranging}
\newabbreviation{uav}{UAV}{unmanned aerial vehicle}
\newacronym{csens}{CoSe}{compressed sensing}
\newabbreviation{cp}{CP}{compressed pansharpening}
\newabbreviation{ft}{FT}{Fourier transform}
\newabbreviation{psf}{PSF}{point spread function}
\newabbreviation{ffc}{FFC}{flat field correction}
\newabbreviation[longplural={regions of interest}]{roi}{RoI}{region of interest}

\newabbreviation[longplural={greenhouse gases}]{ghg}{GHG}{greenhouse gas}
\newabbreviation{ehs}{EHS}{environment, health and safety}
\newabbreviation{fpga}{FPGA}{field programmable gate array}

\newabbreviation{qb}{QB}{QuickBird}
\newabbreviation{wv2}{WV2}{WorldView-2}
\newabbreviation{wv3}{WV3}{WorldView-3}
\newabbreviation{iko}{IKO}{IKONOS}
\newabbreviation{ge1}{GE-1}{GeoEye-1}
\newabbreviation{prisma}{PRISMA}{Hyperspectral Precursor of the Application Mission}
%\newabbreviation{prisma}{PRISMA}{PRecursore IperSpettrale della Missione Applicativa}
\newabbreviation{jpeg}{JPEG}{Joint Photographic Experts Group}
\newabbreviation{jp2k}{JP2K}{\glsentryshort{jpeg} 2000}
\newabbreviation{ccsds}{CCSDS}{Consultative Committee for Space Data Systems}

\newabbreviation{fui}{FUI}{Fonds Unique Interministériel}
\newabbreviation{anr}{ANR}{Agence Nationale de Recherche}
\newabbreviation{ipag}{IPAG}{Institut de Planétologie et d'Astrophysique de Grenoble}
\newabbreviation{onera}{ONERA}{Office National d'Etudes et de Recherches Aérospatiales}
\newabbreviation{imspoc}{ImSPOC}{Image SPectrometer On Chip}
\newabbreviation{tralfic}{TRALFIC}{trust region algorithm for low finesse interferometer characterization}
%\newabbreviation{imspocuv}{ImSPOC-UV}{\glsentrylong{imspoc}-\glsentrylong{uv}}
\newabbreviation[type=ignored]{imagaz}{ImaGAZ}{ImaGAZ}
\newabbreviation{scarbo}{SCARBO}{Space CARBon Observatory}
\newabbreviation[type=ignored]{imspoc-uv}{ImSPOC-UV}{\glsentryshort{imspoc}-\glsentrylong{uv}}
\newabbreviation{presto}{PRESTO}{Precursory Research for Embryonic Science and Technology}
%\newabbreviation{acro}{Astrid}{ASTRID project}
\newabbreviation{nasa}{NASA}{National Aeronautics and Space Administration}
\newabbreviation{h2020}{H2020}{Horizon 2020}
\newabbreviation{fumultispoc}{FuMultiSPOC}{FUsion MULTIspectral-\glsentryshort{imspoc}}
% \newabbreviation{imagaz}{imaGAZ}{Image Spectrometer on Chip}


\newabbreviation{p1}{IMSPOC-UV v1}{prototype \glsentryshort{imspoc-uv}/\glsentryshort{vis}}
\newabbreviation{p2}{IMSPOC-UV v2}{prototype \glsentryshort{imspoc-uv}-drone}
\newabbreviation{p3}{IMAGAZ}{prototype \glsentryshort{imagaz}-1}
\newabbreviation{p4}{PROTO-4}{prototype NanoCarb-1}
\newabbreviation{p5}{WFAI}{prototype WFAI}

%\newabbreviation{ord2}{ORD2}{permuted lexicographic representation}

\newacronym{phd}{PhD}{doctor of philosophy}

%% SYMBOLS

%\glsxtrnewsymbol

% \newabbreviation[type=symbolsopt]{lambda}{\ensuremath{\lambda}}{wavelength}
% \glsxtrnewsymbol[type=symbolsopt]{sigma}{\ensuremath{\sigma}}{wavenumber in the vacuum}
% \newabbreviation[type=symbolsopt]{k}{\ensuremath{\mathbf{k}}}{wavevector}
% \newabbreviation[type=symbolsopt]{nu}{\ensuremath{\nu}}{optical frequency}
% \newabbreviation[type=symbolsopt]{theta}{\ensuremath{\theta}}{polar angle}
% \newabbreviation[type=symbolsopt]{phi}{\ensuremath{\phi}}{azimuth angle}
% \newabbreviation[type=symbolsopt]{theta}{\ensuremath{\bm{\theta}}}{spherical coordinate angle}
% \newabbreviation[type=symbolsopt]{refl}{\ensuremath{\mathcal{R}}}{reflectivity}
% \newabbreviation[type=symbolsopt]{trans}{\ensuremath{\mathcal{T}}}{transmissivity}
% \newabbreviation[type=symbolsopt]{n}{\ensuremath{n}}{refraction index}
% \newabbreviation[type=symbolsopt]{n0}{\ensuremath{n_0}}{refraction index in the air/vacuum}
% \newabbreviation[type=symbolsopt]{eps}{\ensuremath{\varepsilon}}{absolute permittivity}
% \newabbreviation[type=symbolsopt]{mu}{\ensuremath{\mu}}{absolute permeability}
% \newabbreviation[type=symbolsopt]{c}{\ensuremath{c}}{phase velocity}
% \newabbreviation[type=symbolsopt]{c0}{\ensuremath{c_0}}{speed of light in the vacuum}

% \newabbreviation[type=symbolsinv]{Al}{\ensuremath{\lino{A}}}{direct model (linear operator)}
% \newabbreviation[type=symbolsinv]{L}{\ensuremath{\lino{L}}}{regularizer (linear operator)}
% \newabbreviation[type=symbolsinv]{Ltv}{\ensuremath{\lino{L}_{tv}}}{\glsentrylong{tv} operator}
% \newabbreviation[type=symbolsinv]{Lutv}{\ensuremath{\lino{L}_{utv}}}{\glsentrylong{utv} operator}
% \newabbreviation[type=symbolsinv]{J}{\ensuremath{J(\cdot)}}{objective function}
% \newabbreviation[type=symbolsinv]{f}{\ensuremath{f(\cdot)}}{data fidelity metric}
% \newabbreviation[type=symbolsinv]{g}{\ensuremath{g(\cdot)}}{regularization metric}
% \newabbreviation[type=symbolsinv]{lambdac}{\ensuremath{\check{\lambda}}}{regularization parameter}
% \newabbreviation[type=symbolsinv]{zetac}{\ensuremath{\check{\zeta}}}{\glsentrylong{sv}}
% \newabbreviation[type=symbolsinv]{rhoc}{\ensuremath{\check{\rho}}}{over-relaxation parameter}
% \newabbreviation[type=symbolsinv]{tauc}{\ensuremath{\check{\tau}}}{main convergence parameter}
% \newabbreviation[type=symbolsinv]{sigmac}{\ensuremath{\check{\sigma}}}{secondary convergence parameter}

% \newabbreviation[type=symbolsfus]{P}{\ensuremath{\mathbf{P},\,\mathbf{p},\,\mathbf{U}^{[p]}}}{\glsentrylong{hri}}
% \newabbreviation[type=symbolsfus]{M}{\ensuremath{\mathbf{M},\,\mathbf{U}^{[m]}}}{\glsentrylong{lri}}
% \newabbreviation[type=symbolsfus]{Mup}{\ensuremath{\widetilde{\mathbf{M}}^{\uparrow}}}{upsampled \glsentryshort{lri}}
% \newabbreviation[type=symbolsfus]{H}{\ensuremath{\mathbf{H},\,\mathbf{U}^{[h]}}}{mask}
% \newabbreviation[type=symbolsfus]{B}{\ensuremath{\mathbf{B},\,\mathbf{U}^{[b]}}}{blurring kernel}
% \newabbreviation[type=symbolsfus]{ratio}{\ensuremath{\ratio}}{scale ratio}
% \newabbreviation[type=symbolsfus]{ratio}{\ensuremath{\rho_c}}{compression ratio}
% \newabbreviation[type=symbolsfus]{Ab}{\ensuremath{\lino{A}_b}}{\glsentryshort{hri} blur operator}
% \newabbreviation[type=symbolsfus]{Ac}{\ensuremath{\lino{A}_c}}{compression operator}
% \newabbreviation[type=symbolsfus]{Ad}{\ensuremath{\lino{A}_d}}{degradation operator}
% \newabbreviation[type=symbolsfus]{Am}{\ensuremath{\lino{A}_m}}{spatial degradation operator}
% \newabbreviation[type=symbolsfus]{Amd}{\ensuremath{\lino{A}_m^{\downarrow}}}{downsampling operator}
% \newabbreviation[type=symbolsfus]{Ap}{\ensuremath{\lino{A}_p}}{spectral degradation operator}

% \newabbreviation[type=symbolsspa]{Ex}{\ensuremath{\mathtt{E}_x}}{space of inputs}
% \newabbreviation[type=symbolsspa]{Ey}{\ensuremath{\mathtt{E}_y}}{space of acquisitions}
% \newabbreviation[type=symbolsspa]{Eb}{\ensuremath{\mathtt{E}_b}}{space of parameters}
% \newabbreviation[type=symbolsspa]{Na}{\ensuremath{N_a}}{number of acquisitions}
% \newabbreviation[type=symbolsspa, description={number of bands}]{Nb}{\ensuremath{N_b}}{number of bands}
% \newabbreviation[type=symbolsspa]{Nbp}{\ensuremath{N_{b_p}}}{number of \glsentryshort{hri} bands}
% \newabbreviation[type=symbolsspa]{Ni}{\ensuremath{N_i}}{number of incidence angle}
% \newabbreviation[type=symbolsspa]{Nk}{\ensuremath{N_k}}{number of interferometers/subimages}
% \newabbreviation[type=symbolsspa]{Nm}{\ensuremath{N_m}}{number of modes or order}
% \newabbreviation[type=symbolsspa]{No}{\ensuremath{N_o}}{number of photodetectors}
% \newabbreviation[type=symbolsspa]{Nx}{\ensuremath{N_x}}{number of input samples}
% \newabbreviation[type=symbolsspa]{Ny}{\ensuremath{N_y}}{number of output samples}

% \newabbreviation[type=symbols]{X}{\ensuremath{\tens{X},\,\mathbf{X},\,\mathbf{U}^{[x]}}}{ideal input}
% \newabbreviation[type=symbols]{Y}{\ensuremath{\tens{Y},\,\mathbf{Y},\,\mathbf{U}^{[y]}}}{acquisition}
% \newabbreviation[type=symbols]{A}{\ensuremath{\tens{A},\,\mathbf{A}}}{optical transfer matrix}
% \newabbreviation[type=symbols]{X}{\ensuremath{\widehat{\tens{X}},\,\widehat{\mathbf{X}}}}{reconstructed input}
% \newabbreviation[type=symbols]{A}{\ensuremath{\tens{B},\,\bm{\beta}}}{parameters}
% \newabbreviation[type=symbols]{R0}{\ensuremath{\mathbf{R}^{[0]}}}{subimage centers}
% \newabbreviation[type=symbols]{Rt}{\ensuremath{\mathbf{R}^{[t]}},\,\check{\mathbf{R}}^{[t]}}{photodetector centers}
% \newabbreviation[type=symbols]{Rf}{\ensuremath{\check{\mathbf{R}}^{[f]}}}{feature positions}
% \newabbreviation[type=symbols]{Rd}{\ensuremath{\check{\mathbf{R}}^{[d]}}}{misalignments}
% %\newabbreviation[description={misalignments}]{Rs}{\ensuremath{\check{\mathbf{R}}^{[s]}}}

% %\newabbreviation[description={time}]{t}{\ensuremath{t}}
% %\newabbreviation[description={force}]{F}{\ensuremath{F}}

% \newabbreviation[type=symbolsop]{vct}{\ensuremath{\vct(\cdot)}}{column concatenation}
% \newabbreviation[type=symbolsop]{lexi}{\ensuremath{\lexi(\cdot)}}{lexicographic order reshaping}
% \newabbreviation[type=symbolsop]{rshp}{\ensuremath{\rshp(\cdot)}}{permuted lexicographic order reshaping}
% \newabbreviation[type=symbolsop]{stck}{\ensuremath{\stck(\cdot)}}{datacube stacking}
% \newabbreviation[type=symbolsop]{cnv}{\cnv,\,\cnvo}{convolution product}
% \newabbreviation[type=symbolsop]{cnv}{\hadam}{Hadamard product}
% \newabbreviation[type=symbolsop]{max}{\ensuremath{\max(\cdot),\,\max(\cdot,\cdot)}}{maximum operator}
% \newabbreviation[type=symbolsop]{min}{\ensuremath{\min(\cdot),\,\min(\cdot,\cdot)}}{minimum operator}
% \newabbreviation[type=symbolsop]{adj}{\ensuremath{(\cdot)^*}}{adjoint operator}
% \newabbreviation[type=symbolsop]{trans}{\ensuremath{(\cdot)\T}}{adjoint operator}
% \newabbreviation[type=symbolsop]{pinv}{\ensuremath{(\cdot)^\dagger}}{Moore-Penrose pseudo-inverse}
% \newabbreviation[type=symbolsop]{fenc}{\ensuremath{(\cdot)\fenc}}{Fenchel conjugate}
% \newabbreviation[type=symbolsop]{l1}{\ensuremath{\|\cdot\|_1}}{$\ell_1$ norm}
% \newabbreviation[type=symbolsop]{l2}{\ensuremath{\|\cdot\|_2}}{$\ell_2$ norm}
% \newabbreviation[type=symbolsop]{linf}{\ensuremath{\|\cdot\|_\infty}}{$\ell_\infty$ norm}
% \newabbreviation[type=symbolsop]{scalar}{\ensuremath{\langle \cdot,\cdot \rangle}}{scalar product}

% \glsxtrnewsymbol[type=symbol, description={wavelength}]{lambda}{\ensuremath{\lambda}}
% \glsxtrnewsymbol[type=symbol, description={wavenumber in the vacuum}]{sigma}{\ensuremath{\sigma}}
% \glsxtrnewsymbol[type=symbol, description={wavevector}]{k}{\ensuremath{\mathbf{k}}}
% \glsxtrnewsymbol[type=symbol, description={optical frequency}]{nu}{\ensuremath{\nu}}
% \glsxtrnewsymbol[type=symbol, description={polar angle}]{theta}{\ensuremath{\theta}}
% \glsxtrnewsymbol[type=symbol, description={azimuth angle}]{phi}{\ensuremath{\phi}}
% \glsxtrnewsymbol[type=symbol, description={spherical coordinate angle}]{thetam}{\ensuremath{\bm{\theta}}}
% \glsxtrnewsymbol[type=symbol, description={reflectivity}]{refl}{\ensuremath{\mathcal{R}}}
% \glsxtrnewsymbol[type=symbol, description={transmissivity}]{trans}{\ensuremath{\mathcal{T}}}
% \glsxtrnewsymbol[type=symbol, description={refractive index}]{n}{\ensuremath{n}}
% \glsxtrnewsymbol[type=symbol, description={refractive index in the air/vacuum}]{n0}{\ensuremath{n_0}}
% \glsxtrnewsymbol[type=symbol, description={absolute permittivity}]{eps}{\ensuremath{\varepsilon}}
% \glsxtrnewsymbol[type=symbol, description={absolute permeability}]{mu}{\ensuremath{\mu}}
% \glsxtrnewsymbol[type=symbol, description={phase velocity}]{c}{\ensuremath{c}}
% \glsxtrnewsymbol[type=symbol, description={speed of light in the vacuum}]{c0}{\ensuremath{c_0}}
% \glsxtrnewsymbol[type=symbol, description={radiance}]{La}{\ensuremath{\mathcal{L}}}
% \glsxtrnewsymbol[type=symbol, description={spectral radiance}]{Ls}{\ensuremath{\mathcal{L}_\sigma}}
% \glsxtrnewsymbol[type=symbol, description={irradiance}]{E}{\ensuremath{\mathcal{E}}}
% \glsxtrnewsymbol[type=symbol, description={spectral irradiance}]{Es}{\ensuremath{\mathcal{E}_\sigma}}


% \glsxtrnewsymbol[type=symbol, description={direct model (linear operator)}]{Al}{\ensuremath{\lino{A}}}
% \glsxtrnewsymbol[type=symbol, description={regularizer (linear operator)}]{L}{\ensuremath{\lino{L}}}
% \glsxtrnewsymbol[type=symbol, description={\glsentrylong{tv} operator}]{Ltv}{\ensuremath{\lino{L}_{tv}}}
% \glsxtrnewsymbol[type=symbol, description={\glsentrylong{utv} operator}]{Lutv}{\ensuremath{\lino{L}_{utv}}}
% \glsxtrnewsymbol[type=symbol, description={objective function}]{J}{\ensuremath{J}}
% \glsxtrnewsymbol[type=symbol, description={data fidelity metric}]{f}{\ensuremath{f}}
% \glsxtrnewsymbol[type=symbol, description={regularization metric}]{g}{\ensuremath{g}}
% \glsxtrnewsymbol[type=symbol, description={regularization parameter}]{lambdac}{\ensuremath{\check{\lambda}}}
% \glsxtrnewsymbol[type=symbol, description={\glsentrylong{sv}}]{zetac}{\ensuremath{\check{\zeta}}}
% \glsxtrnewsymbol[type=symbol, description={over-relaxation parameter}]{rhoc}{\ensuremath{\check{\rho}}}
% \glsxtrnewsymbol[type=symbol, description={main convergence parameter}]{tauc}{\ensuremath{\check{\tau}}}
% \glsxtrnewsymbol[type=symbol, description={secondary convergence parameter}]{sigmac}{\ensuremath{\check{\sigma}}}

% \glsxtrnewsymbol[type=symbol, description={\glsentrylong{hri}}]{P}{\ensuremath{\mathbf{P},\,\mathbf{p},\,\mathbf{U}^{[p]}}}
% \glsxtrnewsymbol[type=symbol, description={\glsentrylong{lri}}]{M}{\ensuremath{\mathbf{M},\,\mathbf{U}^{[m]}}}
% \glsxtrnewsymbol[type=symbol, description={upsampled \glsentryshort{lri}}]{Mup}{\ensuremath{\widetilde{\mathbf{M}}^{\uparrow}}}
% \glsxtrnewsymbol[type=symbol, description={mask}]{H}{\ensuremath{\mathbf{H},\,\mathbf{U}^{[h]}}}
% \glsxtrnewsymbol[type=symbol, description={blurring kernel}]{B}{\ensuremath{\mathbf{B},\,\mathbf{U}^{[b]}}}
% \glsxtrnewsymbol[type=symbol, description={scale ratio}]{ratio}{\ensuremath{\ratio}}
% \glsxtrnewsymbol[type=symbol, description={compression ratio}]{comp}{\ensuremath{\rho_c}}
% \glsxtrnewsymbol[type=symbol, description={\glsentryshort{hri} blur operator}]{Ab}{\ensuremath{\lino{A}_b}}
% \glsxtrnewsymbol[type=symbol, description={compression operator}]{Ac}{\ensuremath{\lino{A}_c}}
% \glsxtrnewsymbol[type=symbol, description={degradation operator}]{Ad}{\ensuremath{\lino{A}_d}}
% \glsxtrnewsymbol[type=symbol, description={spatial degradation operator}]{Am}{\ensuremath{\lino{A}_m}}
% \glsxtrnewsymbol[type=symbol, description={downsampling operator}]{Amd}{\ensuremath{\lino{A}_{m^\downarrow}}}
% \glsxtrnewsymbol[type=symbol, description={spectral degradation operator}]{Ap}{\ensuremath{\lino{A}_p}}

% \glsxtrnewsymbol[type=symbol, description={space of inputs}]{Ex}{\ensuremath{\mathtt{E}_x}}
% \glsxtrnewsymbol[type=symbol, description={space of acquisitions}]{Ey}{\ensuremath{\mathtt{E}_y}}
% \glsxtrnewsymbol[type=symbol, description={space of parameters}]{Eb}{\ensuremath{\mathtt{E}_b}}
% \glsxtrnewsymbol[type=symbol, description={number of acquisitions}]{Na}{\ensuremath{N_a}}
% \glsxtrnewsymbol[type=symbol, description={number of bands}]{Nb}{\ensuremath{N_b}}
% \glsxtrnewsymbol[type=symbol, description={number of \glsentryshort{hri} bands}]{Nbp}{\ensuremath{N_{b_p}}}
% \glsxtrnewsymbol[type=symbol, description={number of incidence angle}]{Ni}{\ensuremath{N_i}}
% \glsxtrnewsymbol[type=symbol, description={number of interferometers/subimages}]{Nk}{\ensuremath{N_k}}
% \glsxtrnewsymbol[type=symbol, description={number of modes or order}]{Nm}{\ensuremath{N_m}}
% \glsxtrnewsymbol[type=symbol, description={number of photodetectors}]{No}{\ensuremath{N_o}}
% \glsxtrnewsymbol[type=symbol, description={number of input samples}]{Nx}{\ensuremath{N_x}}
% \glsxtrnewsymbol[type=symbol, description={number of output samples}]{Ny}{\ensuremath{N_y}}

% \glsxtrnewsymbol[type=symbol, description={ideal input}]{X}{\ensuremath{\tens{X},\,\mathbf{X},\,\mathbf{U}^{[x]}}}
% \glsxtrnewsymbol[type=symbol, description={acquisition}]{Y}{\ensuremath{\tens{Y},\,\mathbf{Y},\,\mathbf{U}^{[y]}}}
% \glsxtrnewsymbol[type=symbol, description={direct model}]{A}{\ensuremath{\tens{A},\,\mathbf{A}}}
% \glsxtrnewsymbol[type=symbol, description={reconstructed input}]{Xhat}{\ensuremath{\widehat{\tens{X}},\,\widehat{\mathbf{X}}}}
% \glsxtrnewsymbol[type=symbol, description={parameters}]{beta}{\ensuremath{\tens{B},\,\bm{\beta}}}
% \glsxtrnewsymbol[type=symbol, description={subimage centers}]{R0}{\ensuremath{\mathbf{R}^{[0]}}}
% \glsxtrnewsymbol[type=symbol, description={photodetector centers}]{Rt}{\ensuremath{\mathbf{R}^{[t]},\,\check{\mathbf{R}}^{[t]}}}
% \glsxtrnewsymbol[type=symbol, description={feature positions}]{Rf}{\ensuremath{\check{\mathbf{R}}^{[f]}}}
% \glsxtrnewsymbol[type=symbol, description={misalignments}]{Rd}{\ensuremath{\check{\mathbf{R}}^{[d]}}}

% \glsxtrnewsymbol[type=symbol, description={column concatenation}]{vct}{\ensuremath{\vct}}
% \glsxtrnewsymbol[type=symbol, description={lexicographic order reshaping}]{lexi}{\ensuremath{\lexi}}
% \glsxtrnewsymbol[type=symbol, description={permuted lexicographic order reshaping}]{rshp}{\ensuremath{\rshp}}
% \glsxtrnewsymbol[type=symbol, description={datacube stacking}]{stck}{\ensuremath{\stck}}
% \glsxtrnewsymbol[type=symbol, description={convolution product}]{cnv}{\ensuremath{*,\,*\mkern-1.5mu *}}
% \glsxtrnewsymbol[type=symbol, description={Hadamard product}]{hadam}{\ensuremath{\hadam}}
% \glsxtrnewsymbol[type=symbol, description={maximum operator}]{max}{\ensuremath{\max}}
% \glsxtrnewsymbol[type=symbol, description={minimum operator}]{min}{\ensuremath{\min}}
% \glsxtrnewsymbol[type=symbol, description={adjoint operator}]{adj}{\ensuremath{(\cdot)^*}}
% \glsxtrnewsymbol[type=symbol, description={transpose operator}]{transp}{\ensuremath{(\cdot)\T}}
% \glsxtrnewsymbol[type=symbol, description={extension}]{exten}{\ensuremath{(\cdot)^{\uparrow}}}
% \glsxtrnewsymbol[type=symbol, description={decimation}]{decim}{\ensuremath{(\cdot)^{\downarrow}}}
% \glsxtrnewsymbol[type=symbol, description={Moore-Penrose pseudo-inverse}]{pinverse}{\ensuremath{(\cdot)^\dagger}}
% \glsxtrnewsymbol[type=symbol, description={sparse channel}]{masked}{\ensuremath{(\cdot)\masked}}
% \glsxtrnewsymbol[type=symbol, description={Fenchel conjugate}]{fenc}{\ensuremath{(\cdot)\fenc}}
% \glsxtrnewsymbol[type=symbol, description={$\ell_1$ norm}]{ell1}{\ensuremath{\|\cdot\|_1}}
% \glsxtrnewsymbol[type=symbol, description={$\ell_2$ norm}]{ell2}{\ensuremath{\|\cdot\|_2}}
% \glsxtrnewsymbol[type=symbol, description={$\ell_\infty$ norm}]{ellinf}{\ensuremath{\|\cdot\|_\infty}}
% \glsxtrnewsymbol[type=symbol, description={scalar product}]{scalar}{\ensuremath{\langle \cdot,\cdot \rangle}}

% correct bad hyphenation here
\hyphenation{op-tical net-works semi-conduc-tor}

%\usepackage{lineno}
%\linenumbers
%\pagenumbering{roman}
\pagenumbering{arabic}

\begin{document}

\title{The ImSPOC snapshot imaging spectrometer: image formation model and device characterization}
	
	%Physical Modeling of the ImSPOC Concept, a Snapshot Fabry-Perot Interferometry-based Spectro Imaging System}

\author{
	Daniele Picone,\authormark{1,2}\,\orcidlink{0000-0002-0226-8399}
	Silv\`{e}re Gousset,\authormark{1}\,\orcidlink{0000-0003-3106-8262}
	Mauro Dalla Mura,\authormark{2,3,*}\,\orcidlink{0000-0002-9656-9087}
	Yann Ferrec,\authormark{4}\,\orcidlink{0000-0002-8361-5398}
	and
	\'{E}tienne le Coarer\authormark{1}\,\orcidlink{0000-0001-6571-5494}
}

\address{
	\authormark{1} Univ. Grenoble Alpes, CNRS, Grenoble INP, IPAG,
	38000 Grenoble, France\\
	\authormark{2} Univ. Grenoble Alpes, CNRS, Inria, Grenoble INP, GIPSA-lab,
	38000 Grenoble, France\\
	\authormark{3} Institut Universitaire de France (IUF)\\
	\authormark{4} ONERA/DOTA, BP 80100, chemin de la Huni\`{e}re, FR-91123 Palaiseau
}
\email{\authormark{*}mauro.dalla-mura@grenoble-inp.fr}

\begin{abstract}
In recent years, the demand for capturing spectral information with finer detail has increased, requiring hyperspectral imaging devices capable of acquiring the required information with increased temporal, spatial and spectral resolution. In this work, we present the image acquisition model of the \gls{imspoc}, a novel compact snapshot image spectrometer based on the interferometry of Fabry-Perot. Additionally, we propose the \gls{irca}, a robust three-step procedure to characterize the \gls{imspoc} device that estimates the optical parameters of the composing interferometers' transfer function. The proposed algorithm processes the image output from a set of monochromatic light sources, refining the results through nonlinear regression after an ad-hoc initialization. Experimental analysis confirms the performances of the proposed approach for the characterization of 4 different \gls{imspoc} prototypes. The source code associated to this paper is available at \href{https://github.com/danaroth83/irca}{https://github.com/danaroth83/irca}.
\end{abstract}

\glsresetall

\section{Introduction}
\label{sec:intro} 
    
    In recent years, \gls{hs} image spectrometers are gathering increasing interest to capture the spectral information of the scene under target with finer spectral resolutions~\cite{Eism12}. This characteristic can be exploited for various user end applications in many fields, such as astronomy, precision agriculture, molecular biology, biomedical imaging, geosciences, physics, and surveillance~\cite{Till07, BenD09, Adam10, Kim17}. In particular, the need for accurate measurements of gases in the atmosphere is ever increasing for tasks as monitoring climate change and air quality study and regulation issues~\cite{Gous19}.
   	
   	%Following this trend, novel technologies are being explored in scientific and commercial venues for the manufacture of novel \gls{hs} imaging systems in order to overcome the limitations of currently available commercial devices. Such improvements involve increasing the spectral or spatial resolution of the user-end image datacube, the richness of the information for specific application, the user friendliness, or reducing the cost, the acquisition time, the dimensions and weight of the camera.  
   	
   	In this paper, we present and characterize the optical devices based on the so-called \textit{\gls{imspoc}} concept~\cite{Gous17,Gous18,Gous19,Ferr18}. \Gls{imspoc} refers to a family of miniaturized and low-cost snapshot image spectrometers based on the interferometry of \gls{fp}. Its spectral resolution aims to be comparable to more traditional devices, such as those based on temporal scanning through diffraction grating~\cite{Bouy18} or dispersion based hyperspectral cameras ($\sim$ 5--10 \si{nm}). 
   	%
   	In this work, the image formation model of the \gls{imspoc} prototypes is formalized rigorously: we briefly recall the derivation of the spectral response of a Fabry-Perot resonator, defining the parameters of interest in the generic case of an integer amount of emerging waves of decreasing optical intensity.
   	%
   	We furthermore propose a general procedure for the parametric characterization of the instrument, divided into a measurement session, where the device is illuminated with a set of flat field monochromatic sources, and the \gls{irca}, a computational procedure to infer the parameters of the system.
   	
   	The proposed \gls{irca} is originally designed to be applied to \gls{imspoc} devices, as their response can be well modeled as the Airy distribution~\cite{Isma16} and described in terms of the \gls{opd}, reflectivity, gain, and phase shift.
   	
   	However, the \gls{irca} can be also potentially applied to characterize and regularly update the calibration of any device whose response can be modeled as that of a \gls{fp} interferometer. For example, it could be potentially applied to compressive imagers~\cite{Oikn18, Oikn18a}, or hyperspectral imaging systems with dielectric mirrors~\cite{Pisa09, Zucc14}. Additional conditions of robustness apply if complementary information is available (e.g., flat field, spatial/temporal redundancy).
   	
   	The \gls{irca} is defined by a three step procedure: the overall optical gain is firstly addressed discarding any interferometric effect, then a first rough assessment of the remaining parameters is performed by casting the problem as a \gls{ml} estimation of the characteristics of a sinusoidal signal. The estimation is finally refined by casting the problem as a nonlinear regression and solving it with the \gls{lm} algorithm~\cite{More78}. 
   	The nonlinear regression approach was also employed in other works~\cite{Hasa18}, but we focus our attention here on providing a robust solution for optical devices whose sensors are particularly sensitive to noise, as the different parameters are made separable by imposing that their polynomial expression in terms of wavelength has a limited degree.
   	
   	To summarize, the novel contributions of this work are:
   	\begin{enumerate}
   		\item we rigorously formalize the image formation principle of an  \gls{imspoc} system (interferometers, lenslet, etc.), formulating the different regimes of finesse within one single framework;
   		\item we develop the \gls{irca}, a procedure for the estimation of parameters for transfer responses of devices operating as \glsentrylong{fp} interferometers;
   		\item we propose an experimental procedure of \gls{imspoc} characterization, using monochromatic sources. We test the effectiveness of the proposed method on real acquisitions from four \gls{imspoc} prototypes with different characteristics.
   	\end{enumerate}
   
    \section{Concept}\label{sec:concept}

To facilitate the presentation of the methodology in \cref{sec:methodolgy}, we want to first describe the conceptual idea behind PINNSim. It originates from the problem that we face when solving \glspl{DAE}, a form of differential equations where a set of algebraic equations constraints the differential equations. It formulates as
\begin{subequations}\label{eq:DAE_general}%
\begin{align}
    \ddt \xstate &= \fupdateof{\xstate, \ystate}\\
    \bm{0} &= \gupdateof{\xstate, \ystate}
\end{align}
\end{subequations}
and we refer to $\xstate(t)$ as the differential variables and to $\ystate(t)$ as the algebraic variables. The update function \fupdateof{\xstate, \ystate} and the algebraic relationship  \gupdateof{\xstate, \ystate} govern the dynamics of the system\footnote{For notational clarity, we formulate an autonomous, unforced system. The conceptual idea can also accommodate non-autonomous and forced systems.}. Our interest focuses on the particular form of index-1 \glspl{DAE} or semi-analytical \glspl{DAE} \cite{brenan_numerical_1995}. This form implies that \gupdate{} can be differentiated once with respect to time $t$ which is possible when $\frac{\partial \gupdate{}}{\partial \ystate}$ is non-singular. Based on \cref{eq:DAE_general}, we could describe the temporal evolution of \xstate{} and \ystate{} as 
\begin{subequations}\label{eq:DAE_general_integral}
\begin{align}
    \xstate(t) &= \xinitial + \int_{t_0}^t \fupdateof{\xstate, \ystate} d\tau\\
    \ystate(t) &= \gupdate'(\xstate)
\end{align}
\end{subequations}
where $\xinitial = \xstate(t_0)$ represents the initial condition and $\gupdate'(\xstate)$ describes the solution to the algebraic equations given \xstate{}. However, usually no analytical expression describes the evolution $\xstate(t)$ and $\ystate(t)$ for a given \xinitial{}. Hence, we revert to numerical integration methods to obtain an approximate solution. 

To resolve the non-trivial integration operation and the implicit relationship \gupdate{}, numerical schemes often restrict the functional form of \xstate{} to a certain approximation \xstatehat{}. For instance, \gls{RK} methods assume a polynomial form of $\xstatehat{}(t) = \xinitial + a_1 (t-t_0) +  a_2 (t-t_0)^2 + \dots$ as they match the Taylor expansion up to a certain degree by construction. The different \gls{RK} schemes prescribe the order of the scheme and the computation of the coefficients. When algebraic variables are present, they have to be interfaced with the approximation of the differential variable to incorporate their interaction with each other. Simultaneous and partitioned integration methods describe such routines \cite{stott_power_1979} and are used in many variations. The accuracy of these constructions is dependent on the order of the used integration scheme and potential \say{interface} errors. These considerations and aspects of numerical stability limit the usable time step size. 

% We query this approximation at sufficiently many points to render the problem fully determined. For example, the explicit 4th-order \gls{RK} scheme uses the initial condition \xinitial{} and 4 additional conditions to fit a fourth-order polynomial, while the implicit trapezoidal rule defines a second-order polynomial based on \xinitial{} and two additional conditions. For \glspl{DAE} the algebraic variables \ystatehat{} will, depending on the approach, fulfil (simultaneous-implicit) or approximately fulfil (partitioned-explicit) the algebraic relationship at the points for which the above-mentioned conditions are evaluated \cite{stott_power_1979}. The accuracy of this construction is dependent on the order of the scheme and sets a practical limit on the time step size. 

By choosing a different functional form for \xstatehat{}, like Fourier-series based or around the Adomian decomposition \cite{adomian_review_1988}, the practical time step size might be increased. Due to their high flexibility of their functional form, \glspl{PINN}, and \glspl{NN} in general as suggested in \cite{lagaris_artificial_1998}, can allow significantly larger time steps. In fact, we can even choose a functional form of $\ystatehat{}(t)$ and approximate $\xstatehat{} = \PINN(t, \ystatehat{})$ in dependence of it. However, this great approximation flexibility of \glspl{PINN} comes with the challenge of generalising well across the entire domain of interest, i.e., being accurate over the entire domain and not only on the training dataset. When the dimensionality of \xstate{} and \ystate{} in \cref{eq:DAE_general} increases, the training of a \gls{PINN} to approximate a wide range of solutions becomes increasingly difficult and eventually intractable. With PINNSim, we avoid this problem by exploiting the structure of the power system specific \glspl{DAE}. The structure allows a decomposition of \cref{eq:DAE_general} into multiple smaller sub-problems which remain tractable from a learning perspective. At the same time, the use of \glspl{PINN} allows for large and accurate time steps of these sub-problems. To simulate the entire system, we need to align the sub-problems' solutions by enforcing the algebraic relationship \gupdateof{\xstate{}, \ystate{}}.
   	
   	The article is organized as follows: in Section~\ref{sec:imspoc} we describe the operating principle of \gls{imspoc} and its image formation model in Section~\ref{sec:formation}. Section~\ref{sec:characterization} describes the proposed spectral characterization setup and estimation algorithm, and in Section~\ref{sec:experiments} we evaluate its performances and discuss its results in comparison with the physics of the devices.
   	
\section{The \texorpdfstring{\glsfirst{imspoc}}{image spectrometer on chip (ImSPOC)} concept}
\label{sec:imspoc}

	\subsection{Operating principle}
   	\label{ssec:imspoc_intro}
   	
   		\begin{figure*}
   	\centering
   	\includegraphics[width=0.8\linewidth]{figures/tikz/imspoc_example.pdf}
    %\sourceimg{}{~\cite{}}
    \caption[\Glsentryshort{imspoc} example acquisition]{Visual representation of an \gls{imspoc} acquisition. The targeted scene is shown on the left. On the top right, filtering effect of specific interferometers in the staircase matrix; on the bottom, interferogram samples and associated reconstructed spectrum. The pictured spectra are only for illustrative purposes.}
    \label{fig:example}
\end{figure*}
   		
   		\Gls{imspoc} describes a concept for a novel image spectrometer (see \figurename~\ref{fig:concept}), whose patent was deposited in 2018~\cite{Guer18}. The concept defines a miniaturized snapshot acquisition system for \gls{hs} imagery, whose optical design consists of a matrix of micro-lenses and a staircase-shaped optical plate, superposed to a focal plane.
   		Each step of the staircase pattern is made of a \gls{fp} interferometer and of an associated micro-lens, whose \textit{division of aperture} allows to capture multiple subimages over their own assigned surfaces on the focal plane. \figurename~\ref{fig:concept_imspoc} shows the cross-section view of the optical design for such a system without a front afocal lens. In this design, the interferometers are cavities carved within a glass optical plate and coated with a reflective surface in titanium dioxide \chem{TiO_2}. An external spectral filter can be also installed to limit the input wavenumber range of the device for certain applications (e.g. for \chem{NO_2} gas detection~\cite{Dole21}).
	   	
	   	An example of acquisition is shown in \figurename~\ref{fig:example}, which also showcases the division of aperture principle. We artificially highlighted the separation between different subimages associated to interferometers on the staircase matrix and assigned them indices in increasing order of nominal thickness. The subimages can be interpreted as a modulation of the input spectrum of the scene by the specific spectral responses of each interferometer, capturing replicas of the scene with different characteristics.
	   	
	   	The set of readings across the different subimages (i.e., the orange dots in the acquisition) represents a sampled version of a continuous interferogram (portrayed with a black background), associated to the spectrum of the portion of the scene at a specific solid angle~\footnote{Following the literature of \glsentrylongpl{fts}, in this work the spectra are expressed in terms of wavenumbers $\sigma$, that is as the reciprocal of the wavelengths (e.g., a wavelength of 500 \si{nm} corresponds to $\sigma=2\times 10^4$ \si{cm^{-1}})}.
	   	
	   	\begin{figure*}
    \captionsetup[subfigure]{justification=centering}
    \centering
    %\hspace{0.1\linewidth}
    \begin{subfigure}[t]{0.49\linewidth}
        \includegraphics[trim={0 0 0 0}, clip, width=0.99\linewidth]{{figures/tikz/sampling_regular.pdf}}
        \caption{Nominal \gls{opd} values}
        \label{fig:opd_shift_regular}
    \end{subfigure}
    \hfil
    \begin{subfigure}[t]{0.49\linewidth}
        \includegraphics[trim={0 0 0 0}, clip, width=0.99\linewidth]{figures/tikz/sampling_irregular.pdf}
        \caption{Exact \gls{opd} values}
        \label{fig:opd_shift_irregular}
    \end{subfigure}
    %\includegraphics[width=0.3\linewidth]{figures/imspoc/delta.png}
    \caption[\Glsentryshort{opd} deviation effect on interferogram sampling]{Comparison between interferograms' samples with nominal and exact values for the \glspl{opd}.}
    \label{fig:opd_shift}
\end{figure*}
	    
	    \subsection{Motivation of the work}
	    \label{ssec:imspoc_motivation}
	    
	    The manufacture of an \gls{imspoc} device follows a series of design choices, which aim to satisfy the requirements of the user application. It is however necessary, especially for initial prototypes, to verify that the optical transformations performed by the real device follow, within a certain degree of accuracy, the desired behavior.
	    Furthermore, it is often necessary to calibrate the instrument at regular intervals, to ensure that this behavior is kept even if some of its physical characteristics have deviated in time. 
	    Both case scenarios highly benefits from a standardized procedure, which we aim to detail in this work.
	    
	    
	    %the importance of the characterization is not only limited to mere quality control, but it also has important consequences if viewed as a typical \textit{computational imaging} problem~\cite{Mait18}. As the observation is captured in a domain other than the one requested by the final user, i.e. the \gls{opd}, the acquisition has to be processed with dedicated algorithms to recover the desired product, i.e. the incident light field spectrum and the associated image.
	    The characterization is also often a necessary step for several techniques of spectrum reconstruction. For example, it is often required to accurately know where the interferogram samples are located with respect of the \gls{opd} axis. 
	    The interferometers' thicknesses are typically designed with a constant step size, so that the interferogram may be regularly sampled (\figurename~\ref{fig:opd_shift_regular}). However, due to the limited accuracy in either manufacturing or assembling the various device parts, this is not the case in most practical scenarios. 
	    If this information is not taken into account, the interferogram samples are then placed incorrectly in the \gls{opd} domain (\figurename~\ref{fig:opd_shift_irregular}) and the quality of the reconstructed spectra strongly degrades. 
	    
	    \begin{table}[t]
    \footnotesize
    \NineColors{saturation=high}
    \caption[Variables list]{Selection of variables used in this paper, grouped in their respective categories.}
    \begin{adjustbox}{width=\linewidth}
    \begin{tblr}
    	{
    		%caption = {Selection of variables used in this paper, grouped in categories.},
    		colspec = {cll|ll},
    		vlines = {white},
    		hlines = {white},
    		vline{4} = {2}{-}{white},
    		cell{1}{2-5} = {red3, fg=yellow9, font=\bfseries},
    		cell{2-14}{1} = {red3, fg=yellow9, font=\bfseries},
    		%cell{1-10}{2-5} = {red3, fg=yellow9, font=\bfseries},
    		cell{3,5,7,9,11,13,15}{2-5} = {red9},
    		%cell{3-6}{1}={red3, fg=yellow9},
    		%cell{3-6}{1} = {font=\bfseries},
    	}
    	&Symbol&Description&Symbol&Description\\
    	\multirow{4}{*}{\rotatebox{90}{Acq. model}}&
    	$\sigma$ & Wavenumbers &$\mathbf{s}$&Focal plane coordinates\\
    	&$\bm{\omega}=(\theta^{[i]},\phi^{[i]})$& Incident angle&$\{\Omega_j\}_{\range{j}{1}{N_p}}$& Solid angle of incidence\\
   		&$\mathcal{L}(\sigma,\bm{\omega})$ & Input spectral radiance&$\{\Phi_{jk}\}_{\range{j}{1}{N_p},\range{k}{1}{N_i}}$&Received flux\\
   		&$\{S_k\}_{\range{k}{1}{N_i}}$&Entrance pupil surface&$\{d_k\}_{\range{k}{1}{N_i}}$&Interferometer thickness\\
    	\multirow{4}{*}{\rotatebox{90}{Parameters}}&
    	$\bm{\delta}=\{\delta_i\}_{\range{i}{1}{N_i}}$& \glsentryshortpl{opd} & $\varphi^{ }_0$&Phase shift\\
    	&$\mathcal{A}(\sigma)$&Gain &$\mathbf{a}=\{a_m\}_{\range{m}{0}{N_d}}$& Gain coefficients\\
    	&$\mathcal{R}(\sigma)$&Surface reflectivity &$\mathbf{r}=\{r_m\}_{\range{m}{0}{N_d}}$& Reflectivity coefficients\\
    	&$\bm{\beta}=\{\beta_m\}_{\range{m}{1}{N_m}}$& Vector of parameters &$\hat{\bm{\beta}}=\{\hat{\beta}_m\}_{\range{m}{1}{N_m}}$& Estimated parameters \\
    	\multirow{3}{*}{\rotatebox{90}{Acq. vectors}} &$\bm{\sigma}=\{\sigma_i\}_{\range{i}{1}{N_a}}$& Central wavenumbers& $T_{\bm{\beta}}(\sigma_i)=\{t_i\}_{\range{i}{1}{N_a}}$ & Transfer function\\
    	&$\mathbf{y}=\{y_i\}_{\range{i}{1}{N_a}}$ & Single pixel acquisition& $\mathbf{w}=\{w_i\}_{\range{i}{1}{N_a}}$ & Flat field pixel statistic \\
    	& $\mathbf{u}=\{u_i\}_{\range{i}{1}{N_a}}$ & Neighborhood mean&$\mathbf{v}=\{v_i\}_{\range{i}{1}{N_a}}$ & Scaled neighborhood mean\\
    	\multirow{3}{*}{\rotatebox{90}{Amount}} &$N_a$ & Acquisitions & $N_i$ & Interferometers\\
    	& $N_p$ & Pixels per interferometer & $W$ & Waves\\
    	& $N_d$ & Degree & $N_m$ & Parameters\\
%    	
 	\end{tblr}
 	\end{adjustbox}
    \label{tab:variables}
\end{table}


\section{\Glsentryshort{imspoc} image formation model}
\label{sec:formation}

	In this section, we describe the image formation model of \gls{imspoc}; we first provide an overall analysis of the optical transfer of the components at play in Section~\ref{ssec:formation_optical}, detailing the one for \gls{fp} interferometers in Section~\ref{ssec:formation_fabry_perot}.
	
	For the reader convenience, the variables used in this paper are shown in \tablename~\ref{tab:variables}, separated into variables for the continuous image formation model, for its parameters, for the acquisition vectors and the vector sizes. These variables will be formerly introduced when relevant to the discussion. 
	
	
	\subsection{Optical transfer model}
	\label{ssec:formation_optical}
	
		In order to define the transfer function from incident light to the sensors' readout, we refer to a single couple interferometer/lens, as shown in \figurename~\ref{fig:formation_focusing}.
		
		If the target source is at the far field, the $k$-th \gls{fp} cavity acts as a spectral filter, whose transfer function $\mathcal{T}_k(\sigma, \bm{\omega})$ varies only with the angle of incidence $\bm{\omega}$ and the wavenumbers $\sigma$. At the focal plane, assuming no crosstalk in the formation of each subimage, the spectral flux $\Phi_{jk}(\sigma)$ received by the $j$-th sensor (i.e. a photodetector) is only due to incident light within a given $k$-th interferometer. Its expression is given by:
		\begin{equation}
			\Phi_{jk}(\sigma) = \int \mathcal{T}_k\left(\sigma,\,\mathbf{s}\right)\mathcal{L}\left(\sigma, \mathbf{s}\right)\;d\mathcal{G}\,,
			\label{eq:flux}
		\end{equation}
		where $\mathcal{L}(\sigma, \omega)$ denotes the spectral radiance of the incident light, $d\mathcal{G}$ the geometric etendue subtended by the surface of the $j$-th photodetector and the exit pupil associated to the $k$-th microlens, and $\mathbf{s}$ are the coordinates that span over the surface of the $j$-th photodetector. In the above expression, the dependency of $\mathcal{L}$ and $\mathcal{T}$ by $\omega$ is shown as a dependency by $\mathbf{s}$, as those are equivalent if the microlens behaves as an ideal thin lens. 
		
		If we then assume that this etendue is constant across the given pixel, the expression of the flux becomes:
		\begin{equation}
			\Phi_{jk}(\sigma) = {S_k} \iint\limits_{\Omega_j}\mathcal{T}_k\left(\sigma,\,\bm{\omega}\right) \mathcal{L}\left(\sigma,\,\bm{\omega}\right) n_0\cos\theta^{[i]} \,d\bm{\omega}\,,
			\label{eq:flux_etendue}
		\end{equation}
		where $\Omega_j$ is the solid angle of incident rays that focus over the $j$-th sensor, $S_k$ is the surface of the entrance pupil associated to the $k$-th interferometer, while $\theta^{[i]}$ is the polar component of the incident angle $\bm{\omega}$. The term $\mathcal{T}_k(\sigma,\bm{\omega})$ is the attenuation of the radiant flux within the $k$-th interferometer and can be modeled as a generic Airy distribution as detailed in the next section.

		Finally, we model the intensity level $x_{jk}$ captured by the photodetector as:
		\begin{equation}
			x_{jk}=\Delta t\int\limits_{\sigma_{min}}^{\sigma_{max}}\Phi_{jk}(\sigma)\,\xi(\sigma)\,\eta_j(\sigma)\,d\sigma\,,
			\label{eq:intensity}
		\end{equation} 
		where $[\sigma_{min}, \sigma_{max}]$ is the bandwidth of the instrument, $\eta_j(\sigma)$ denotes the quantum efficiency of the $j$-th sensor, $\xi(\sigma)$ denotes the spectral response of the accessory elements of the optical system (entry filter, leading optics, etc.), and $\Delta t$ denotes the integration time.
	
		\begin{figure*}
	
	\begin{subfigure}[b]{.6\linewidth}
		\centering
		\raisebox{5mm}{
			\includegraphics[width=\linewidth]{figures/tikz/imspoc_focusing.pdf}
		}
		\caption{Focusing effect on the focal plane}
		\label{fig:formation_focusing}
	\end{subfigure}
	\hfil
	\begin{minipage}[b]{.34\textwidth}
		\begin{subfigure}[b]{\linewidth}
			\centering
			\includegraphics[width=\linewidth,trim={50 0 0 0},clip]{figures/tikz/fabry_perot.pdf}
			\caption{\Glsentrylong{fp} interferometry light rays}
			\label{fig:formation_fabry_perot}
		\end{subfigure}
	
		\vspace{2mm}
		
		\begin{subfigure}[b]{\linewidth}
			\centering
			\includegraphics[trim={0cm 0cm 0cm 0cm},clip,width=\linewidth]{figures/python/airy.pdf}
			\caption{Airy distribution}
			\label{fig:formation_finesse}
		\end{subfigure}
	\end{minipage}
	
    \caption[\Glsentryshort{imspoc} image formation]{\Glsentryshort{imspoc} image formation model visualization of the light ray path within the instrument and corresponding interferometry transfer function.}
    \label{fig:formation}
\end{figure*}

	        
	\subsection{\texorpdfstring{\Glsentrylong{fp}}{Fabry-Perot} wave models}
	\label{ssec:formation_fabry_perot}
	
        We briefly recall here the expression of the Airy distribution, in order to explicit the term $\mathcal{T}_k(\sigma,\bm{\omega})$ from eq.~\ref{eq:flux_etendue}.
        Let us consider a monochromatic plane wave with complex amplitude $\mathbf{E}^{[i]}(\sigma)$ incident to the \gls{fp} interferometer, forming an angle $\theta^{[i]}$ with the normal to the incident plane.
        The complex amplitude $\mathbf{E}^{[o]}$  of the transmitted light can be seen as a sum of the set of $W\rightarrow\infty$ subsequent emerging waves $\{\mathbf{E}_m\}_{\range{m}{0}{W-1}}$. Each emerging wave introduces a fixed round trip phase difference: 
        \begin{equation}
        	\varphi=2\pi\delta\sigma-\varphi_0^{}\,,
        \end{equation}
    	where $\varphi_0^{}$ defines a constant phase shift and $\delta$ defines the \gls{opd} between two consecutive emerging wave.
        A visual representation of the \gls{opd} is shown in \figurename~\ref{fig:formation_fabry_perot}, where it is identifiable as the difference in the round trip inner reflection (in green) and the direct transmission path. By making use of Snell's law, simple geometry manipulations yield:
        \begin{equation}
        	\delta=
        	n\frac{2d_k}{\cos\theta}-n^{}_0(2d_k\tan\theta\sin\theta^{[o]})=
        	n\left(\frac{2d_k}{\cos\theta}-2d_k\tan\theta\sin\theta\right)=
        	2nd_k\cos\theta\,,
        	\label{eq:opd_fabry_perot}
        \end{equation}
        where $d_k$ denotes the thickness of the $k$-th \gls{fp} cavity, while $n$ and $\theta$ are the refractive index and the reflection angle within the cavity, respectively.
        
       	We then define the transfer function $\mathcal{T}_k(\sigma, \bm{\omega})$ due to the \gls{fp} interferometer as the ratio between the output and input irradiance, obtaining, for $W$ emerging waves:
        
        \begin{equation}
        		\mathcal{T}^{[W]}_k:=\left\lvert\frac{\mathbf{E}^{[o]}(\sigma)}{\mathbf{E}^{[i]}(\sigma)}\right\rvert^2=
        		(1-\mathcal{R})^2\left\lvert\sum_{m=0}^{W-1}\left(\mathcal{R}\;e^{-j\varphi}\right)^m\right\rvert^2 =
        		(1-\mathcal{R})^2\left\lvert\frac{1-\mathcal{R}^{W}e^{-jW\varphi}}{1-\mathcal{R}e^{-j\varphi}}\right\rvert^2\,,
        	\label{eq:airy_definition}
        \end{equation}
        where $\mathcal{R}$ is the surface reflectivity, and the resulting term $(1-\mathcal{R})^2$ is due to the direct transmission through the cavity.
        
       	We specify the above expression for $2$, $W\ge0$ and $\infty$ waves, in order to describe the interferometer under different regimes of finesse:
       	\begin{subequations}
       		\begin{align}
       			\mathcal{T}_k^{[2]}(\sigma, \bm{\omega})&=\left(1+\mathcal{R}^2+2\mathcal{R}\cos\varphi\right)(1-\mathcal{R})^2\,,&\textrm{2 waves}\,,	\label{eq:airy_1}\\
       			\mathcal{T}_k^{[W]}(\sigma, \bm{\omega})&=\frac{1+\mathcal{R}^{2W}-2\mathcal{R}^{W}\cos(W\varphi)}{1+\mathcal{R}^2-2\mathcal{R}\cos\varphi}(1-\mathcal{R})^2\,,&\textrm{W waves}\,,\label{eq:airy_2}\\
       			\mathcal{T}_k^{[\infty]}(\sigma, \bm{\omega})&=\frac{(1-\mathcal{R})^2}{(1-\mathcal{R})^2+4\mathcal{R}\sin^2(\varphi/2)}\,,&\textrm{$\infty$ waves}\,,\label{eq:airy_3}
       		\end{align}
       		\label{eq:airy}
       	\end{subequations}
       	and $\mathcal{T}_k^{[\infty]}(\sigma)=\lim_{W\rightarrow\infty}\mathcal{T}^{[W]}_k(\sigma)$ is known as the \textit{Airy distribution}~\cite{Isma16}. 
       	For our purposes, it is also convenient to scale the expression $\mathcal{T}_k^{[W]}$ of  eq.~\eqref{eq:airy} such that the mean of the scaled expression $\overline{\mathcal{T}}_k^{[W]}$ is equal to one:
       	\begin{equation}
       		\overline{\mathcal{T}}_k^{[W]}(\sigma,\bm{\omega})=\frac{1+\mathcal{R}}{(1-\mathcal{R}^{2W})(1-\mathcal{R})}\mathcal{T}_k^{[W]}(\sigma, \bm{\omega})\;.
       		\label{eq:norm}
       	\end{equation}
       	
       	Its expression from eq.~\eqref{eq:airy_3} varies only slightly with the angle of incidence $\theta^{[i]}$, as the \gls{fov} is designed to be relatively limited (e.g., below $0.2\pi$ \si{rad}).
       	However, it strongly varies with the wavenumber $\sigma$; assuming all the other parameters  are constant, the explicit dependency from $\sigma$ is given through $\phi$ and shown in \figurename~\ref{fig:formation_finesse} with all the remaining ones set as constants.
       	The peaks are more spaced out if the thicknesses $d_k$ of the interferometer are large, which causes the different filtering effect of \figurename~\ref{fig:example}. For low finesse devices (that is with low surface reflectivity), this spectral response resembles a pure sinusoid, allowing for higher throughput and consequently higher \gls{snr} captured by the sensors.
       	Eq.~\eqref{eq:airy_3} also has an implied dependency from $\sigma$ within the reflectivity $\mathcal{R}(\sigma)$ and through the refractive index $n(\sigma)$, to be taken in account for the spectral characterization of the device.
       	
       	\subsection{Proposed formulation of the model}
       	\label{ssec:characterization_problem}
       	
       	In order to characterize the overall spectral response of the instrument at a given pixel, the physical acquisition model employed from eq.~\ref{eq:flux_etendue} may be simplified, assuming the optical transfer function is roughly constant within the targeted solid angle $\Omega_j$ and $\cos(\theta)=1$, resulting in:
       	\begin{equation}
       		x_{jk}=\int\limits_{\sigma_{min}}^{\sigma_{max}}T_{\bm{\beta}}\left(\sigma\right)\left(\iint\limits_{\Omega_{j}} \mathcal{L}(\sigma,\,\bm{\omega})\,d\bm{\omega}\right)\,d\sigma\,.
       		\label{eq:intensity_simple}
       	\end{equation}
       	Here, $T_{\bm{\beta}}(\sigma)$ models the spectral response of the instrument associated to a given pixel on the \gls{fpa}. For convenience, it is useful to describe it in terms of the expression of eq.~\eqref{eq:norm}:
       	\begin{equation}
       		T_{\bm{\beta}}(\sigma) = \mathcal{A}(\sigma)\overline{\mathcal{T}}^{[W]}(\sigma,\bm{\omega}_j)
       		\label{eq:transfer_function}
       	\end{equation}
       	where we defined a gain variable:
       	\begin{equation} 					 	\mathcal{A}(\sigma):=\xi(\sigma)\eta_j(\sigma)S_k\Omega_j\frac{1+\mathcal{R}(\sigma)}{(1-\mathcal{R}^{2W}(\sigma))(1-\mathcal{R}(\sigma))}\;,
       	\end{equation}
       	which incorporates all the multiplicative terms from eq.s~\eqref{eq:flux_etendue} and \eqref{eq:intensity}, while $\bm{\omega}_j$ is the centroid of the solid angle $\Omega_j$. The transfer function is written in its scalar form so that the the mean value with respect to $\sigma$ of $T_\beta(\sigma)$ is equal to that of $\mathcal{A}(\sigma)$.		
       	By substituting the value of $W$, equivalent expressions can be derived from eq.~\eqref{eq:airy} for $2$ and $\infty$ waves.
       	
       	The set of values $[\mathcal{A}(\sigma),\,\mathcal{R}(\sigma), \delta,\,\varphi_0]$ allow for a full description of $T_{\bm{\beta}}$.
       	$\mathcal{A}(\sigma)$ and $\mathcal{R}(\sigma)$ are strongly coupled in the expression~\eqref{eq:transfer_function}; in order for their contribution to be separable, we assume that they are slowly varying with $\sigma$ and limit their models to polynomials of limited degree $N_d$. 
       	In other terms, we consider that $\mathcal{A}(\sigma)$ and $\mathcal{R}(\sigma)$ are in the form:
       	\begin{equation}
       		\mathcal{A}(\sigma)=\sum_{m=0}^{N_d} a_m\sigma^m\,,\;\;\;\;\;\;\;\;\;\;\;\;\mathcal{R}(\sigma)=\sum_{m=0}^{N_d}r_m\sigma^m\,.\label{eq:polynomial}
       	\end{equation}
       	
       	The \gls{opd} value $\delta$ is assumed to be constant with the wavelength $\sigma$ as the rays interfer within the air in the prototypes under test (\figurename~\ref{fig:concept_imspoc}). This assumption is extended to the phase shift $\varphi^{}_0$ in order to simplify the computation. Our goal then summarizes to find an estimation $\hat{\bm{\beta}}$ of the $2N_d+4$ elements of  $\bm{\beta}=\left[a_0\,,...,\,a_{N_d},\,r_0\,,...,\,r_{N_d},\,\delta,\,\varphi_0^{}\right]$ allowing to approximate the transfer function $T_{\bm{\beta}}$ as accurately as possible.
       	
       	In the next section, we describe the calibration setup that was employed in this study to acquire an observation vector $\mathbf{y}$ of  $N_a$ noisy acquisitions $\mathbf{y}=\{y_i\}_{\range{i}{1}{N_a}}$ at given wavenumber values $\bm{\sigma}=\{\sigma_i\}_{\range{i}{1}{N_a}}$, that we use to formalize our problem as a \gls{mse} minimization.
       	
       	% \begin{figure}
    \captionsetup[subfigure]{justification=centering}
    \centering
    	
	\begin{subfigure}[t]{0.45\linewidth}
		\centering
		\includegraphics[width=\linewidth,trim={50 0 0 0},clip]{figures/tikz/fabry_perot.pdf}
		\caption{\Glsentrylong{fp} interferometer}
		\label{fig:interferometers_fabry_perot}
	\end{subfigure}
	\hfil
    \begin{subfigure}[t]{0.54\linewidth}
    	\centering
        \includegraphics[trim={3cm 0 3cm 3cm},clip,width=\linewidth]{figures/extra/finesse.pdf}
        \caption{Airy distribution}
        \label{fig:finesse_finesse}
    \end{subfigure}
    \caption[Interferogram fringes]{On the left, light reflection within a \gls{fp} resonator. On the right, Airy distribution normalized intensity, for different values of reflectivity.}
    \label{fig:finesse}
\end{figure}


\section{Proposed characterization procedure}
\label{sec:characterization}

	In this section, we present the proposed procedure for the spectral characterization of \gls{fp} interferometers. Specifically, we describe the problem in more formal terms in Section~\ref{ssec:characterization_problem} the calibration setup in Section~\ref{ssec:characterization_setup}. Finally, we provide an overview of the proposed algorithm in Section~\ref{ssec:characterization_overview}, detailing each step in the subsequent sections.

	\subsection{Measurement setup for \glsentryshort{imspoc} characterization}
	\label{ssec:characterization_setup}
	
		
\subsection{Datasets and training details}

\begin{table}
    \centering
    \resizebox{\linewidth}{!}{
    \setlength{\tabcolsep}{2pt}
    \begin{tabular}{l|cccc}
        \bf Dataset  & \bf Task & \bf \#cls & \bf \#train & \bf \#val \\
        \midrule
        {\color{ImageDark}{\inetOneK}} (\inetOneKShort)~\cite{ILSVRC15} & Image cls. & 1000 & 1M & 50K \\
        {\color{ImageDark}{\inat}} (\inatShort)~\cite{iNaturalist} & Fine-grained cls. & 8142 & 437K & 24K \\
        {\color{ImageDark}{\inetvTwo}} (\inetvTwoShort)~\cite{recht2019imagenet} & Image cls. & 1000 & -- & 10K \\
        {\color{ImageDark}{\inetReal}} (\inetRealShort)~\cite{beyer2020we} & Image cls. & 1000 & -- & 50K \\
        {\color{ImageDark}{\objectNet}} (\objectNetShort)~\cite{barbu2019objectnet} & Image cls. & 113 & -- & 19K \\
        {\color{ImageDark}{\food}} (\foodShort)~\cite{bossard2014food} & Image cls. & 101 & N\slash{}A & 25K \\
        {\color{DetDark}{\cocoShort}}~\cite{lin2014microsoft} & Obj.\ det. & 80 & 118K & 5K \\
        {\color{DetDark}{\lvisShort}}~\cite{gupta2019lvis} & Obj.\ det. & 1K & 100K & 20K \\
        {\color{VideoDark}{\kinetics}} (\kineticsShort)~\cite{kay2017kinetics} & Action cls. & 400 & 220K & 20K \\
        {\color{VideoDark}{\sthsth}} (\sthsthShort)~\cite{goyal2017something} & Action cls. & 174 & 169K & 25K \\
    \end{tabular}
    }
    \caption{\textbf{Evaluation datasets} used to evaluate \ours on {\color{ImageDark} image classification}, {\color{DetDark} object detection}, and {\color{VideoDark} video action recognition} tasks.
    The table reports the task, number of classes (\#cls), number of training samples (\#train), and number of validation samples (\#val) for each dataset.
    }
    \label{tab:eval_datasets}
\end{table}


\noindent \textbf{Pretraining dataset.} We use \igSize(\igSizeShort) a billion-scale multi-label dataset
sourced from \ig (\igShort). 
This multi-label dataset contains 28K classes and 3B unique images, resampled to 5B total images,
and was produced by running the dataset generation pipeline from \swag~\cite{singh2022revisiting} without modification.
Compared to~\cite{singh2022revisiting}, our
version of the dataset has 16\% fewer images (3.0B \vs 3.6B), but we were able to reproduce the results
from~\cite{singh2022revisiting} with our version.
We obtain labels using an automated process wherein we first obtain hashtags from the associated image captions, and
then map the hashtags to WordNet synsets following~\cite{singh2022revisiting}.
After this processing, we get the weakly labeled \igShort dataset that contains images and their associated labels.



\par \noindent \textbf{Evaluation datasets.}
We evaluate \ours on a variety of different downstream visual recognition tasks.
To evaluate our model on image classification, we use the standard \inetOneK~\cite{ILSVRC15} (\inetOneKShort) dataset,
and also the long-tailed and fine-grained \inat~\cite{iNaturalist} (\inatShort) dataset.
For object detection and segmentation, we use the popular \cocoShort~\cite{lin2014microsoft} dataset, and also \lvisShort
\cite{gupta2019lvis}, a large vocabulary dataset for long tailed object recognition.
We evaluate video classification performance using two popular action recognition datasets, \kinetics~\cite{kay2017kinetics}
(\kineticsShort) and \sthsth~\cite{goyal2017something} (\sthsthShort). For zero-shot transfer, we evaluate on \inetOneKShort
and \food~\cite{bossard2014food} (\foodShort).
We also evaluate the robustness of our models on test sets which overlap with \inetOneKShort classes, specifically
\inetvTwo~\cite{recht2019imagenet} (\inetvTwoShort), \inetReal~\cite{beyer2020we} (\inetRealShort),
and \objectNet~\cite{barbu2019objectnet} (\objectNetShort). 
Please see~\cref{tab:eval_datasets} for more details.


\par \noindent \textbf{\mae pretraining details.}
We follow \cite{he2021masked} to train \mae models on \igSizeShort without using any labels.
We mask 75\% of the image for this training and train the model for 1 epoch over the dataset. We follow the same hyperparameters
used in \cite{he2021masked} for pretraining on \inetOneKShort.

\par \noindent \textbf{Supervised pretraining details.}
We train with a supervised cross-entropy loss on \igSizeShort using the hashtags as labels.
This model is trained by default with random weight initialization and we use the training hyperparameters from~\cite{singh2022revisiting}.
\par \noindent \textbf{Using \prept.}
When using \prept, we first train a model from scratch using \mae on the \igShort dataset.
We then use the weights of the \mae encoder and perform supervised pretraining using the cross-entropy loss as described above.
We reuse the same hyperparameters and training details as
\cite{singh2022revisiting}, \ie \emph{there is no hyperparameter search needed} for \ours, and we train for 1 epoch on
\igSizeShort. 


\par \noindent \textbf{Zero-shot training and evaluation details.}
To impart zero shot understanding capabilities to our models, we use the \lit approach from~\cite{zhai2022lit}.
For \lit, we use the original \emph{(image, caption)} pairs from the \igSizeShort dataset.
We \emph{freeze} the image encoder, and train a text encoder to encode the image captions and match the text embeddings to the associated image embedding using a \clip loss~\cite{radford2021learning}.
We train the text encoder for 1 epoch.
For evaluation, we follow~\cite{radford2021learning} -- we use the text encoder to compute embeddings from the templated text descriptions of classes and use the cosine similarity of the image and text embeddings as the classification score.

For full training details and hyperparameters, please refer to \cref{app:pretraining_details}.



		
		 In order to characterize a given device under test, a certain set of observations from reference sources have to be taken under controlled conditions, so that the transfer function can be inferred.
		 Perhaps the most straightforward approach consists in a flat field illumination of the device with a set of monochromatic incident spectra with given central wavenumbers.
		 In fact, according to eq.~\eqref{eq:intensity_simple}, this is equivalent to take $N_a$ acquisitions $\mathbf{y}\in\mathbb{R}^{N_a}$, whose expected values are samples of $T_{\bm{\beta}}(\sigma)$ evaluated for the set of wavenumbers $\bm{\sigma}\in\mathbb{R}^{N_a}$.
		 The measurement setup is shown in \figurename~\ref{fig:experimental_setup}; it is composed by a wideband lamp, acting as light source, whose incident light is filtered by a monochromator (i.e., based on diffraction grating). The sharply impulsive output spectrum, whose bandwidth is lower than the spectral resolution of the filter, is diffused uniformly through an integrating sphere. The monochromator is tunable to select a sequence of central wavenumbers and the device under test captures an image for each of these illuminations. An external spectrometer or a probe is used to measure the incident power of the instrument; its value is used to equalize the energy of all the acquired images across different wavenumbers, after its background level is set to zero.
		 The vector $\mathbf{y}$ is finally extracted by slicing the datacube of acquisition at the spatial position that corresponds to the pixel to characterize. Roughly, the problem is then equivalent to find the estimation $\hat{\bm{\beta}}$ of the vector of parameters minimizing a cost function such as:
		 \begin{equation}
		 	\hat{\bm{\beta}}=\arg\min_{\bm{\beta}}\sum_{i=1}^{N_a} \left(T_{\bm{\beta}}(\sigma_i)-y_i\right)^2\;.
		 	\label{eq:cost}
		 \end{equation}
	
	\subsection{Overview of the \glsreset{irca}\gls{irca}}
	\label{ssec:characterization_overview}
		
		% \textcolor{red}{Refer to Tarantola's book to call it as model identification~\cite{Tara05}}
	
		Solving eq.~\eqref{eq:cost} is a particularly challenging problem, due to the nonlinear dependency of $T_{\bm{\beta}}$ from the parameters $\bm{\beta}$. The available tools for solving nonlinear regression methods are particularly sensitive to converging to non-local maxima~\cite{Rusz06}, so that a proper initialization is critical to produce an accurate parametrization of the optical system.
		
		Therefore, the proposed \glsreset{irca}\gls{irca} consists of three steps, which are depicted in \figurename~\ref{fig:algorithm}. The algorithm exploits different sufficient statistics of the set of $N_a$ acquisitions captured by the measurement setup to improve the overall robustness of the final result:
		\begin{itemize}
			\item The \textbf{gain estimation} step provides an initial approximation of the gain coefficients  $\{\hat{a}_i\}_{\range{i}{0}{N_d}}$ of $\mathcal{A}(\sigma)$. This result is obtained by processing the vector $\mathbf{w}\in{\mathbb{R}^{N_a}}$, which represents a \textit{flat field statistic}. The vector $\mathbf{w}$ is supposed to represent the response of the pixel under test if the matrix of interferometers was stripped away from the device. However, as this operation is not always possible, we assume it to represent a given percentile from the raw acquisition evaluated over the whole focal plane. In fact, the global response of the image naturally dampens the oscillations of the interferometric fringes;
			\item The \textbf{\glsreset{ml}\gls{ml} initialization} returns an approximate estimation $\left[\hat{\delta},\,\hat{r}_0,\,\hat{\varphi}_0\right]$ of the remaining parameters (\gls{opd}, reflectivity and phase shift, respectively); at this stage, those parameters are assumed to be constant with the wavenumbers. This step processes the vector $\mathbf{u}\in\mathbb{R}^{N_a}$, which we define as the \textit{mean over neighbors}. This is simply obtained by taking the spatial average of each acquisition within a window around the target pixel. The intrinsic effect of noise dampening is not strictly necessary, but improves the robustness of the algorithm.
			\item In the \textbf{\glsreset{gn}\gls{gn}} step, the final estimation $\hat{\bm{\beta}}$ of the desired parameters is obtained with the  \gls{lm} algorithm~\cite{More78}. In this stage, we process the raw acquisition vector $\textbf{y}$ in order to solve eq.~\eqref{eq:cost}. The estimation is refined with a sequence of iterations, after it is initialized with the vector $\bm{\beta}^{[0]}$ of the parameters inferred in the previous steps.
		\end{itemize}
	
	    \begin{algorithm}
\footnotesize
    \caption{$k$-SALSA}
    \label{alg:overall}
    \textbf{Input:} Private dataset $X=(x_1,\dots,x_n)$, auxiliary dataset $X_0$ for GAN model training, integer $k>1$ (assume $n = mk$ for integer $m$ without loss of generality), number of iterations $T$, loss ratio parameter $\lambda$ \\
    \textbf{Output:} Synthetic dataset $\tilde{X}$ of size $m$ with $k$-anonymity

    \begin{algorithmic}[1]
        \State Train a GAN generator $G$ and a GAN inversion encoder $E$ on $X_0$
        \State Obtain latent code $w_i = E(x_i)$ for each $i\in [n]$ and let $W=\{w_i\}_{i=1}^n$
        \State $(C_1,\dots,C_m)= \textsf{SameSizeClustering}(W, k)$  \Comment{$C_j\subset W$, $|C_j|=k$, $|C_j\cap C_{j'\neq j}|=0$, $\forall j$}
      
        \State Initialize $\tilde{X} = \emptyset$
        \For {each cluster $j\in [m]$}
        \State Let $C_j = (w_1',\dots,w_k')$, and $x_i'$ the original image of $w_i'$ for each $i$
        \State Compute $w_0 = \frac{1}{k} \sum_{i=1}^{k} w'_i$ and generate $x_0 = G(w_0)$
        \State Initialize $w_\text{avg}^{(0)} = w_0$
        \For {each iteration $t \in [T]$}
            \State Generate $x_\text{avg}^{(t-1)} = G(w_\text{avg}^{(t-1)})$
            \State Compute content loss $\mathcal{L}_\text{content}(x_0, x_\text{avg}^{(t-1)})$ using Eq.~\ref{eq:loss-content} 
            \State Compute local style alignment loss $\mathcal{L}_\text{style}((x'_1,\dots,x'_k), x_\text{avg}^{(t-1)})$ using Eq.~\ref{eq:loss-style}
            \State Compute total loss $\mathcal{L}_\text{total}=\lambda \mathcal{L}_\text{content}+(1-\lambda)\mathcal{L}_\text{style}$
            \State Update $w_{\text{avg}}^{(t)}$ using $w_{\text{avg}}^{(t-1)}$ and the gradient $\nabla_{w_\text{avg}^{(t-1)}} \mathcal{L}_\text{total}$
        \EndFor
        \State Add $G(w_{\text{avg}}^{(T)})$ to $\tilde{X}$
        \EndFor
    \State    \Return $\tilde{X}$

    \end{algorithmic}
\end{algorithm}

		
		The following sections describe each of these algorithm in further detail.
	
	\subsection{Step 1: Gain estimation}
	\label{ssec:characterization_gain}
	
		The gain estimation step aims at a preliminary estimation of the vector $\hat{\mathbf{a}}=\{\hat{a}_m\}_{\range{m}{0}{N_d}}$, whose elements describe the coefficients of the polynomial representation $\hat{\mathcal{A}}(\sigma)=\sum_{m=0}^{N_d}\hat{a}_m\sigma^m$ of the gain $\mathcal{A}(\sigma)$. The operation is modeled as a nonlinear regression, where the coefficients are updated iteratively to fit the given vector $\mathbf{w}=\{w_i\}_{\range{i}{1}{N_a}}$ representing the overall flat field intensity of the set of acquisitions.
		In mathematical terms, the problem is formalized as follows:
		\begin{equation}
			\hat{\mathbf{a}}=\arg\min_{\mathbf{a}}\sum_{i=1}^{N_a}(\mathcal{A}(\sigma_i)-w_i)^2\,.
			\label{eq:cost_gain}
		\end{equation}
		We propose to solve the problem above through the \gls{lm} algorithm, following the implementation of~\cite{More78}, which is briefly described in Appendix~\ref{ssec:lm}.
		
		For the gain estimation step, a straightforward but robust initialization $\mathbf{a}^{[0]}$ is to set the bias element $a_0^{[0]}$ to the mean value of $\mathbf{w}$, while the remaining elements are set to zero.
		
	\subsection{Step 2: \texorpdfstring{\Glsentrylong{ml}}{Maximum likelihood} initialization}
	\label{ssec:characterization_ml}
	
		The \glsentryfull{ml} initialization described in this section is an extension of the method that we proposed in~\cite{Dole19} and roughly defines a curve fitting procedure for sinusoidal signals. The algorithm exploits the low finesse nature of the device under test, for which the transfer function $T_{\bm{\beta}}(\sigma)$ behaves like the 2-waves model of eq.~\eqref{eq:airy_1}: 
		\begin{equation}
			T_{\bm{\beta}}(\sigma)=\left(1+\frac{2r_0^{}}{1+r_0^2}\cos(2\pi\delta\sigma-\varphi^{}_0)\right)\mathcal{A}(\sigma)\,.
		\end{equation}
		In the above, we assume for simplicity that the reflectivity $\mathcal{R}$ is uniform and equal to $r_0$ over the whole wavenumber range.
		By normalizing both terms of the minimization problem of eq.~\eqref{eq:cost} and applying it to the mean over neighborhood vector $\mathbf{u}$, the problem can be rewritten as:
		
		\begin{subequations}
			\begin{align}
				\hat{\bm{\beta}}&\approx\arg\min_{\bm{\beta}}\sum_{i=1}^{N_a}\left(\frac{T_{\bm{\beta}}(\sigma_i)}{\mathcal{A}(\sigma_i)}-\frac{u_i}{\hat{\mathcal{A}}(\sigma_i)}\right)^2=\label{eq:cost_ml_1}\\
				&\approx\arg\min_{\bm{\beta}}\sum_{i=1}^{N_a}\left(\alpha\cos\left(2\pi\delta\sigma_i-\varphi_0^{}\right)-v_i\right)^2\,,\label{eq:cost_ml_2}
			\end{align}
			\label{eq:cost_ml}
		\end{subequations} 
		where we defined $\alpha:=2r_0^{}/(1+r_0^2)$ and $v_i:=\left(u_i-\hat{\mathcal{A}}(\sigma_i)\right)/\hat{\mathcal{A}}(\sigma_i)$, assuming $\mathcal{A}(\sigma_i)\approx\hat{\mathcal{A}}(\sigma)$.
		
		Eq.~\ref{eq:cost_ml_2} is in the form of the classical problem of the inference of the parameters in a sinusoid affected by Gaussian noise, whose \gls{mle} is a well known result in the literature (e.g., example 7.16 in~\cite{Kay93}).
		Specifically, if $\delta$ is reasonably far from the extremes of its interval of confidence $[0, 1/(2\Delta\sigma)]$, where $\Delta\sigma=(\sigma_{max}-\sigma_{min})/N_a$ is the average central wavenumber step, then the \gls{mle} $\hat{\delta}$ is equivalent to the \gls{opd} which maximizes the generalized \gls{dft} of $\mathbf{v}$, commonly known as \textit{periodogram}:
		\begin{equation}
			\hat{\delta}=\arg\max_{\delta\in\left[0,\frac{1}{2\Delta\sigma}\right]}\left\lvert\sum_{i=1}^{N_a}v_ie^{-j2\pi\delta\sigma_i}\right\rvert\,.
			\label{eq:delta_zero}
		\end{equation}
		The condition above can be verified heuristically over a sampled version of the interval, but the accuracy of the estimation has a limited resolution $1/(2N_a\Delta\sigma)$.
		If some prior information is known on the \gls{opd}, e.g. if its value is known to be around a certain nominal value with a given accuracy, the sampling region can be reduced further accordingly. 
		Given the above result, the estimation $\hat{r}_0$ of $r_0$ is then obtained in terms of the \gls{mle} $\hat{\alpha}$ of the amplitude $\alpha$ of the sinusoid:
		\begin{equation}
			\hat{r}_0=1-\sqrt{1-\hat{\alpha}^2}\,,\;\;\;\;\;\;\textrm{where }\;\;\;\hat{\alpha}=\frac{2}{N_a}\left\lvert\sum_{i=1}^{N_a}v_i e^{-j2\pi\hat{\delta}\sigma_i}\right\rvert\,,\label{eq:reflectivity_hat}
		\end{equation}		
		 and the \gls{mle} $\hat{\varphi}^{}_0$ of $\varphi^{}_0$ is:
		\begin{equation}
			\hat{\varphi}_0^{}=\arctan\frac{\sum_{i=1}^{N_a}v_i\sin(2\pi\hat{\delta}\sigma_i)}{\sum_{i=1}^{N_a}v_i\cos(2\pi\hat{\delta}\sigma_i)}\,,
			\label{eq:phase_shift_hat}
		\end{equation}
		where $\arctan$ denotes the four-quadrant arctangent version that allows for $\hat{\varphi}_0^{}$ to assume any value in the range $[-\pi,\pi)$. The \gls{ml} method requires very low computational power, but its applicability is limited by the validity of its assumptions. Some other possible initialization strategies, like the \gls{es} developed in~\cite{Pico20a}, are based on a grid search in the sample space of the parameters, have the advantage to work with a wider variety of models, but are vastly slower and may even produce less accurate results, as the estimations for $r_0^{}$ and $\varphi_0^{}$ are limited to the finite amount of values of the discrete sample space. 
	
	\subsection{Step 3: Trust region refinement (TRR)}
	\label{ssec:characterization_gna}
	
		The final refinement $\hat{\bm{\beta}}$ of the parameters is also performed through non linear regression, and formalized to solve the problem of eq.~\eqref{eq:cost}.
		The problem is approached with an analogous procedure of Section~\ref{ssec:characterization_gain}, which makes use once again of the \gls{lm} algorithm, this time solving eq.~\eqref{eq:cost}. The parameter vector $\bm{\beta}$ can be initialized with the vector $\bm{\beta}^{[0]}=[\hat{a}_0,...,\,\hat{a}_{N_d},\, \hat{r}_0^{},\, 0, ...,\,0,\, \hat{\delta},\hat{\varphi}_0^{}]$ which contains the coefficients estimated in the previous steps.


\section{Experimental results}
\label{sec:experiments}

	In this section, we discuss the experimental results of the characterization of \gls{imspoc} prototypes with different characteristics. In Section~\ref{ssec:experiments_setup} we describe the experimental setup, in Section~\ref{ssec:experiments_robustness} we test various configurations for the proposed algorithm and compare with previous works. Finally, in Section~\ref{ssec:experiments_discussion} we discuss the physical interpretation of the parameters analyzing a specific case study. A Python implementation of the proposed algorithms, together with a simulator of the image formation model of \gls{imspoc} is available at the first author's repository~\footnote{Code repository available at: \href{https://github.com/danaroth83/irca}{https://github.com/danaroth83/irca}}.
	
	
	\subsection{Experimental setup}
	\label{ssec:experiments_setup}
	
		\begin{table}[t]
    \centering
    % est-ce que cela veut dire que OPD_max = 216*2*100nm? si oui, cela donne un intervalle spectral libre (=période de la fonction d'Airy) de 23mm-1, alors que la résolution de monochromateur est de 10mm-1 => la fonction d'appareil du monochromateur fait, pour les marches les plus hautes, pratiquement une demi-pédiode de la donction d'Airy => comment peut-on négliger l'influence de la résolution du monochromateur? c'est aussi le cas (un peu moins critique tout de même) pour WFAI
    %\textcolor{red}{Does the value for ImSPOC-UV v2 mean that $OPD_{max}=216*2*100$ \si{nm}? If so, that means that the free spectral range (= period of the Airy's function) is 23 \si{mm^{-1}}, where the resolution of the monochromator is 10 \si{mm^{-1}} => That means that the transfer function (fonction d'appareil) of the monochromator practically does half the period of the Airy's function for the highest thicknesses. How can we ignore the influence of the resolution of the monochromator? This is also the case for WFAI (although less critical)}
    
    \sisetup{detect-weight=true,detect-inline-weight=math}
    \footnotesize
    \NineColors{saturation=high}
    \begin{talltblr}[
    	caption = {Characteristics of the available \gls{imspoc} prototypes used in this work and of their spectral characterization experimental acquisitions.
    	\label{tab:proto}},
    	note{*} = {In this prototype, two interferometers are at the optical contact for testing purposes.},
    	note{**} = {This cell gives the mean and standard deviation of the step size, which is irregularly spaced for this experiment.}]
    	{
    		colspec = {l|ccccccc},
    		vlines = {white},
    		hlines = {white},
    		cell{2}{1} = {red3, fg=yellow9, font=\bfseries},
    		cell{1-2}{2-8} = {red3, fg=yellow9, font=\bfseries},
    		cell{4,6}{2-8} = {red9},
    		cell{3-6}{1}={red3, fg=yellow9},
    		cell{3-6}{1} = {font=\bfseries},
    	}
    	&\multicolumn{4}{c}{Device specifications}&\multicolumn{3}{c}{Acquisition specifications}\\
    	Prototype label& \makecell{Interf.s\\$N_i$} & \makecell[c] {$\Delta d$\\$[\si{nm}]$} & \makecell[c]{Focal plane\\size [\si{px}]} & \makecell[c]{Subimage\\size [\si{px}]} & \makecell[c]{Wavenumber\\range [\si{mm^{-1}}]} & \makecell{Acq.s\\$N_a$} & \makecell[c]{$\Delta\sigma$\\$[\si{{cm}^{-1}}]$}
    	\\
    	\glsentryshort{p1} &$216$ &$100$ &$2808\times 1096$ &$100\times 100$ &$1000-2000$ &101  &100
    	\\
    	\glsentryshort{p2} &$319$ &$87.5$ &$2808\times 1096$ &$96\times 96$   &$1000-2850$ &721 &25
    	\\
    	\glsentryshort{p5} &$672$ &$87.5$ &$2808\times 1096$ &$66\times66$  &$1230-2880$ &551 &30
    	\\
    	\glsentryshort{p3}  &$79\,(+1)$\TblrNote{*} &$200$ &$640\times 512$   &$64\times 64$   &$625-1000$  &343 &$11 \pm 12$\TblrNote{**}
 	\end{talltblr}
\end{table}
	
		For this work, the characterization datacubes were captured with the setup shown in \figurename~\ref{fig:experimental_setup}, using a tunable monochromatic light source from Zolix Instruments Co., Ltd, with a 500 \si{W} Xenon light source model Gloria-X500A and the monochromator model Omni-300$\lambda$i. The setup also employed a Specralon coated 5.3 inches diameter integrating sphere model 4P-GPS-053-SF from Labsphere, Inc. The incident optical power was measured either with the fiber optic gated spectrometers model USB2000+ from Ocean Optics, Inc. or with the photodiode power sensor model S120VC from Thorlabs~\footnote{Specifications for the products are available at: \href{https://www.idil-fibres-optiques.com/product/tunable-light-source/}{https://www.idil-fibres-optiques.com/product/tunable-light-source/},  \href{https://www.labsphere.com/product/general-purpose-sphere/}{https://www.labsphere.com/product/general-purpose-sphere/}, \href{https://www.oceaninsight.com/products/spectrometers/}{https://www.oceaninsight.com/products/spectrometers/}, and \href{https://www.thorlabs.com/thorproduct.cfm?partnumber=S120VC}{https://www.thorlabs.com/}}.
		
		The devices under test are 4 different \gls{imspoc} prototypes, whose characteristics are described in \tablename~\ref{tab:proto}. Each prototype features an array of interferometers disposed over a bidimensional matrix in a staircase pattern, whose thicknesses linearly increasing with a nominally constant step size $\Delta d$. For each device, the characterization datacube was captured by tuning the central wavenumber of the monochromator such that the wavenumber range is sampled regularly with a step size $\Delta\sigma$; the specifications for these acquisitions are also reported in~\tablename~\ref{tab:proto}.
		
		In each acquisition, the central pixel of each subimage was extracted in order to construct $N_p$ vectors. For any vector $\mathbf{y}$, the characterization methods described in Section~\ref{sec:characterization} are applied to obtain a characterization vector $\hat{\bm{\beta}}$; the associated neighborhood mean $\mathbf{u}$ uses an $11 \times 11$ kernel window, while we employ the 90-percentile flat field statistic for $\mathbf{w}$. This operation was preferred for \gls{snr} dampened acquisition over taking the pixel mean over multiple acquisitions, as that would lead to unfeasibly long sessions of measurement.
		Additionally, we assume that the gain estimation step is applied to all methods we test and with $N_d=5$ as polynomial degree.
		To verify the estimation quality, we use the \gls{rmse} metric, defined as follows:
		\begin{equation}
			RMSE=\sqrt{\frac{1}{N_a}\sum_{i=1}^{N_a}\left(\frac{T_{\hat{\bm{\beta}}}(\sigma_i)-y_i}{\overline{y}}\right)^2}\,.
			\label{eq:rmse}
		\end{equation}
		where $\overline{y}=(\sum_{i=1}^{N_a}y_i)/N_a$ denotes the mean value of $\mathbf{y}$ with $T_{\hat{\bm{\beta}}}(\sigma_i)$ from eq.~\eqref{eq:transfer_function} evaluated with the estimated vector of parameters $\hat{\bm{\beta}}$.
	
	
	\subsection{Algorithm and model comparisons}
	\label{ssec:experiments_robustness}
	
		\begin{table}[t]
    \centering
    \caption[Model characterization results]{Model characterization \gls{rmse} comparison. Best results are in bold.}
    %\setlength{\tabcolsep}{3pt}
    %\newcolumntype{g}{>{\columncolor{gray!10}}c}
    %\rowcolors{1}{gray!10}{white}
    \footnotesize
    \NineColors{saturation=high}
%	\begin{tblr}
%		{
%			colspec = {lcccccc},
%			vline{2-7} = {white},
%			hline{2,3,6,8,10} = {white},
%			cell{1}{2-7} = {red3, fg=yellow9},
%			cell{3,5,7,9}{3-7} = {red9},
%			cell{2-10}{2} = {red9},
%			cell{2-10}{1}={red3, fg=yellow9},
%			cell{2-10}{2} = {font=\bfseries},
%		}
%		&\textbf{Method}&$M$&\textbf{\glsentryshort{p1}}&\textbf{\glsentryshort{p2}}&\textbf{\glsentryshort{p5}}&\textbf{\glsentryshort{p3}}\\
%		\multirow{6}{*}{\rotatebox[origin=c]{90}{\textbf{Fixed} $\mathcal{A}$}}
%		&\glsentryshort{ml}~\cite{Dole19}&$2$& $0.3072 \pm 0.0702$ &$0.2291 \pm 0.0993$ & $0.4395 \pm 0.2437$ & $0.2314 \pm 0.0770$ \\
%		&\multirow{3}{*}{\glsentryshort{es}~\cite{Pico20}}& $2$& $0.3117 \pm 0.0763$ & $0.2558 \pm 0.1063$ & $0.4708 \pm 0.2524$ & $0.2502 \pm 0.0820$  \\
%		&& $3$ & $0.3016 \pm 0.0777$ &$0.2533 \pm 0.1548$ & $0.4730 \pm 0.2540$ & $0.2503 \pm 0.0827$ \\
%		&& $\infty$  & $0.3009 \pm 0.0772$ & $0.2613 \pm 0.1949$ & $0.4732 \pm 0.2541$ & $0.2504 \pm 0.0827$ \\
%		&\multirow{2}{*}{\glsentryshort{ml}+\glsentryshort{gn}}& $2$& $0.2716 \pm 0.0708$  & $0.2125 \pm 0.0426$ & $0.4181 \pm 0.2360$ & $0.1814 \pm 0.0787$ \\
%		&& $\infty$ & $0.2559 \pm 0.0732$ & $0.1836 \pm 0.0386$ & $0.4180 \pm 0.2369$ & $0.1811 \pm 0.0830$ \\
%		\multirow{3}{*}{\rotatebox[origin=c]{90}{\textbf{Free} $\mathcal{A}$}} 
%		&\multirow{2}{*}{\glsentryshort{ml}+\glsentryshort{gn}}& $2$ & $0.2135 \pm 0.0384$& $0.1684 \pm 0.0365$ & $0.2186 \pm 0.0526$ & $0.0724 \pm 0.0209$\\
%		&& $\infty$ & $0.1917 \pm 0.0413$ & $\mathbf{0.1329 \pm 0.0343}$ & $0.2169 \pm 0.0517$ & $\mathbf{0.0672 \pm 0.0222}$\\
%		&\glsentryshort{es}+\glsentryshort{gn}& $\infty$& $\mathbf{0.1915 \pm 0.0417}$  & $0.1332 \pm 0.0355$ & $\mathbf{0.2128 \pm 0.0518}$ & $0.0679 \pm 0.0233$ \\
%	\end{tblr}


	\begin{tblr}
	{
		colspec = {lcccccc},
		vline{2-7} = {white},
		hline{2,3,6,8,10} = {white},
		cell{1}{2-7} = {red3, fg=yellow9},
		cell{3,5,7,9}{3-7} = {red9},
		cell{2-10}{2} = {red9},
		cell{2-10}{1}={red3, fg=yellow9},
		cell{2-10}{2} = {font=\bfseries},
	}
		&\textbf{Method}&$W$&\textbf{\glsentryshort{p1}}&\textbf{\glsentryshort{p2}}&\textbf{\glsentryshort{p5}}&\textbf{\glsentryshort{p3}}\\
		\multirow{6}{*}{\rotatebox[origin=c]{90}{\textbf{Fixed} $\mathcal{A}$}}
		&\glsentryshort{ml}~\cite{Dole19}&$2$& $0.3022 \pm 0.0605$ &$0.1948 \pm 0.0720$ & $0.4363 \pm 0.2338$ & $0.2291 \pm 0.0756$ \\
		&\multirow{3}{*}{\glsentryshort{es}~\cite{Pico20}}& $2$& $0.3052 \pm 0.0638$ &$0.2319 \pm 0.0792$ & $0.4654 \pm 0.2393$ & $0.2472 \pm 0.0795$ \\
		&& $3$ & $0.2941 \pm 0.0638$ &$0.2153 \pm 0.0991$ & $0.4672 \pm 0.2403$ & $0.2472 \pm 0.0800$ \\
		&& $\infty$ & $0.2934 \pm 0.0640$ &$0.2204 \pm 0.1247$ & $0.4673 \pm 0.2404$ & $0.2472 \pm 0.0800$ \\
		&\multirow{2}{*}{\glsentryshort{ml}+\glsentryshort{gn}}& $2$& $0.2721 \pm 0.0637$ &$0.2698 \pm 0.2166$ & $0.4153 \pm 0.2338$ & $0.1856 \pm 0.0852$ \\
		&& $\infty$ & $0.2544 \pm 0.0631$ &$0.2445 \pm 02217$ & $0.4126 \pm 0.2342$ & $0.1836 \pm 0.0873$ \\
		\multirow{3}{*}{\rotatebox[origin=c]{90}{\textbf{Free} $\mathcal{A}$}} 
		&\multirow{2}{*}{\glsentryshort{ml}+\glsentryshort{gn}}& $2$ & $0.2169 \pm 0.0392$ &$0.1691 \pm 0.0366$ & $0.2186 \pm 0.0528$ & $0.0724 \pm 0.0210$ \\
		&& $\infty$ & $\mathbf{0.1937 \pm 0.0432}$ &$0.1336 \pm 0.0343$ & $0.2170 \pm 0.0519$ & $\mathbf{0.0669 \pm 0.0222}$ \\
		&\glsentryshort{es}+\glsentryshort{gn}& $\infty$& $\mathbf{0.1937 \pm 0.0432}$ &$\mathbf{0.1335 \pm 0.0343}$ & $\mathbf{0.2130 \pm 0.0520}$ & $0.0676 \pm 0.0233$ \\
	\end{tblr}

    \label{tab:fitting}
    \end{table}


		\begin{figure*}
	\captionsetup[subfigure]{justification=centering}
	\centering
	\begin{subfigure}[b]{0.49\linewidth}
		\centering
		\includegraphics[width=\linewidth]{figures/python/imagaz_1_estimation_20.pdf}
		\caption{\Glsentryshort{p3} prototype; 20-th interferometer}
		\label{fig:curves_p3}
	\end{subfigure}
	\hfil
	\begin{subfigure}[b]{0.49\linewidth}
		\centering
		\includegraphics[width=\linewidth]{figures/python/imspoc_uv_1_estimation_50.pdf}
		\caption{\Glsentryshort{p1} prototype; 50-th interferometer}
		\label{fig:curves_p1}
	\end{subfigure}
	\caption[Spectral response fitting]{Two examples of spectral response fitting. The dashed curves refer to the \gls{ml} and to the proposed method.}
	\label{fig:curves}
\end{figure*}
	
		The interferometer characterization with the setup discussed before is tested here with different configurations of our algorithm. Specifically, the algorithms can be tested employing different wave models for the optical transfer function $T_{\bm{\beta}}(\sigma)$, according to the definitions of eq.s~\eqref{eq:airy} and~\eqref{eq:transfer_function}, which we specialize to the case of $W=2$, $3$ or $\infty$ emerging light rays.
		
		\begin{figure*}
	\captionsetup[subfigure]{justification=centering}
	\centering
	\begin{subfigure}[b]{0.49\linewidth}
		\centering
		\includegraphics[width=\linewidth]{figures/python/gain.pdf}
		\caption{Gain comparison}
		\label{fig:gain}
	\end{subfigure}
	\hfil
	\begin{subfigure}[b]{0.49\linewidth}
		\centering
		\includegraphics[width=\linewidth]{figures/python/reflectivity.pdf}
		\caption{Estimated eflectivity}
		\label{fig:reflectivity}
	\end{subfigure}
	\caption[Gain and reflectivity comparison]{Estimated gain and reflectivity plot for the \glsentryshort{p2} prototype.}
	\label{fig:parameters}
\end{figure*}

		
		Additionally, the proposed procedure was tested both with and without the trust region refinement, in order to showcase the advantage of the iterative curve fitting procedure. In our computations, we either let the gain coefficients evolve or be fixed after the first step.
		The results are showcased in \figurename~\ref{fig:gain}, where we compare the result for the prototype \glsentryshort{p2}. For the blue curve, the gain can be only adjusted by a multiplicative factor; in the orange curve, the shape of the estimated gain can vary for each interferometer, allowing to dynamically adjust for the local intensity attenuations introduced by the device.
		For the same prototype, we also provide the plot of the reflectivity estimated with the proposed method in \figurename~\ref{fig:reflectivity}, which shows the reduced sensitivity of the instrument for the extreme values of wavenumber range.

		The \gls{rmse} results, given in \tablename~\ref{tab:fitting}, shows that the proposed method is consistently the best performing, regardless of the different characteristics of the prototypes.
		It also highlights how the $\infty$-wave model, which is a better representation of the generalized Airy distribution, provides a more accurate fit for the spectral response. Letting the \gls{lm} algorithm control the gain parameter evolution has also a considerable impact on the refinement of those obtained through a flat field estimation. The proposed options for the initialization both reach comparable results, suggesting that the \gls{ml} method should be preferred, as it is faster by a factor of 10-20 over the \gls{es} methodology. In general terms, the proposed method exploits the advantage of exploring a continuous space of parameters, with respect, i.e. of the \gls{ml} where the \gls{opd} space is explored in discrete steps.
		A visual comparison between their reconstructed spectral responses is shown in \figurename~\ref{fig:curves}; while the \gls{ml} algorithm  infers the \gls{opd} with a remarkable accuracy, introducing a polynomial description of the reflectivity represents more properly the nature of amplitudes of the oscillations, which varies with the wavelength.
		
		\subsection{Physical interpretation of the results}
		\label{ssec:experiments_discussion}
		
		\begin{figure*}
    %\captionsetup[subfigure]{justification=centering}
    \centering
    \begin{minipage}[b]{.29\linewidth}
    	\begin{subfigure}[b]{\linewidth}
    		\centering
    		\includegraphics[width=\linewidth]{figures/python/imspoc_uv_2_oldest_fit_150.pdf}
    		\caption{Date: 2021-10-20}
    		\label{fig:opd_old}
    	\end{subfigure}
    	
    	\smallskip
    	
    	\begin{subfigure}[b]{\linewidth}
    		\centering
    		\includegraphics[width=\linewidth]{figures/python/imspoc_uv_2_fit_150.pdf}
    		\caption{Date: 2021-12-13}
			\label{fig:opd_new}
    	\end{subfigure}
    \end{minipage}
	\hfil
    \begin{subfigure}[b]{0.7\linewidth}
        \includegraphics[trim={0 0 0 0}, clip, width=0.99\linewidth]{figures/python/opd_comparison.pdf}
        \caption{Comparison of estimated \glspl{opd} (with zoomed regions)}
        \label{fig:opd_comparison}
    \end{subfigure}
    \caption[\Glsentryshort{opd} estimation]{On the right, \glspl{opd} of the \glsentryshort{p2} prototype estimated from acquisitions taken at different times. On the left, fitted spectral responses for the $150$-th interferometer.}
    \label{fig:opd}
\end{figure*}

		\begin{figure*}
	\captionsetup[subfigure]{justification=centering}
	\centering
	\begin{minipage}[b]{.29\linewidth}
		\begin{subfigure}[b]{\linewidth}
			\centering
			\includegraphics[width=\linewidth]{figures/tikz/plates_default.pdf}
			\caption{Ideal optical plate installation}
			\label{fig:plate_ideal}
		\end{subfigure}
		
		\medskip
		
		\begin{subfigure}[b]{\linewidth}
			\centering
			\includegraphics[width=\linewidth]{figures/tikz/plates_tilted.pdf}
			\caption{Tilted plate}
			\label{fig:plate_tilted}
		\end{subfigure}
	\end{minipage}
	\hfil
	\begin{subfigure}[b]{0.70\linewidth}
		\includegraphics[trim={0 0 0 0}, clip, width=0.99\linewidth]{figures/python/imspoc_uv_1_opd_difference_step.pdf}
		\caption{\Gls{opd} step difference deviation from nominal value.}
		\label{fig:plate_data}
	\end{subfigure}
	\caption[\Glsentryshort{opd} plates]{On the right, heatmap of the deviation of the \glspl{opd} increase between successive interferometers. The interferometers' indexes, referring to the \gls{p1} prototype, are arranged in increasing nominal thickness order. On the left, effect of tilting the optical plate on the thickness of the \gls{fp} cavity.}
	\label{fig:plate}
\end{figure*}
		
		For the \glsentryshort{p2} prototype, the session of measurements for the spectral characterization was repeated at three different dates, using progressively refined setups. 
		The proposed method was applied to each of those datasets in order both to analyze the robustness of the algorithm and to detect eventual drifting in the parameters.
		
		For a nadir illumination, such as in the case of the central pixels of the subimages, the \glspl{opd} are roughly expected to be double the thickness of interferometers, as a consequence of substituting $\theta^{[i]}\approx0$ in eq.~\ref{eq:opd_fabry_perot}.
		
		A comparison of the estimated \glspl{opd} is shown in \figurename~\ref{fig:opd_comparison}, with the yellow halo denoting the region that was not explored for the \gls{ml} computation in eq.~\eqref{eq:delta_zero}, as that strays too far from the nominal values known from the design of the instrument. 
		The samples that were processed, two of which for the same interferometer are shown in \figurename~\ref{fig:opd_old} and~\ref{fig:opd_new}, look quite different; that most recent datacube were compensated by the incident optical power of the light source, while the other ones were not. Nevertheless, the estimation of the \glspl{opd} stays vastly consistent, with only very sparse examples where the results does not align.
		
		The plot of the estimated \glspl{opd}, is arranged in increasing order of nominal thickness, shows a pattern of alternating slopes. This effect is strongly correlated to the disposition of interferometers on the matrix.
		
		This effect is particularly evident for the \gls{p1} prototype, which is manufactured with a particular staircase design pattern. The thicknesses increases with constant steps in both the left and right direction starting from a central vertical axis. The heatmap of \figurename~\ref{fig:plate_data} shows the deviation between the estimated \gls{opd} difference across successive interferometers compared to this constant increase. The different behavior of the left and right side of the instrument can be straightforwardly interpreted if we assume a slight tilt on the optical plates that limit the upper surface of the \gls{fp} cavities, as shown in \figurename\ref{fig:plate_tilted}. On the left side, the thickness is inferior with respect to the nominal value, as the tilt compresses the free space within the cavities, and viceversa for the right side. % \textcolor{red}{With a regression analysis this tilt can be measured to be around equal to ...}  
		

	
\section{Conclusion}

	In this paper, we presented the spectral characterization procedure associated to the \gls{imspoc} device, an image spectrometer based on the interferometry of \glsentrylong{fp}. We described the image formation model and we expressed its spectral response in terms of a limited amount of parameters, following the formulation of Airy distribution. The proposed characterization algorithm exploits the two emerging wave model to cast the problem in the Fourier domain, where the \gls{ml} estimator for the \glspl{opd} is equivalent to a maximization of the periodogram and refines the results with nonlinear regression; separating the responsabilities in the algorithm allows for both robustness and precision in the final results. The estimated parameters can highlight manufacturing issues in an easily interpretable format. A proper characterization is extremely important for a proper recovery of the spectrum, which, in the future could be optimized jointly with the spectral response of the devices in architectures where such parameters can be learned dynamically~\cite{Sun16, Yang20}.
	
\section*{Funding}
	% This article is funded by ImSPOC-UV, convention FEDER n° RA0022348
	This work is partly supported by the AuRA region and FEDER under the project ImSPOC-UV (convention FEDER n. RA0022348) and by grant ANR FuMultiSPOC (ANR-20-ASTR-0006).
	
\appendix

\section{Levenberg-Marqardt algorithm}
\label{ssec:lm}

We aim to provide here an approachable explanation of the \gls{lm} algorithm, to build an intuition of what are the operations involved in the process.
The algorithm can be seen as a trust region based approach for nonlinear regression. 
The aim is to find an estimation $\hat{\bm{\beta}}$ of the parameters $\bm{\beta}=\{\beta_m\}_{\range{i}{1}{N_m}}$, in order for the samples $\{t_i(\bm{\beta})\}_{\range{i}{1}{N_a}}$ of an analytical function to fit a set of observation $\{y_i\}_{\range{i}{1}{N_a}}$.

The algorithm addresses the problem by finding a sequence of iteratively more accurate solutions $\{\bm{\beta}^{[q]}\}_{q\ge0}$ from a given initialization $\bm{\beta}^{[0]}$, using the following update rule:
\begin{equation}
	\bm{\beta}^{[q]}=\arg\min_{\bm{\beta}}\sum_{i=1}^{N_a}\left(t_i(\bm{\beta}^{[q-1]})+\sum_{m=1}^{N_m}j_{im}\left(\beta_m-\beta_m^{[q-1]}\right)-w_i\right)^2+\lambda\sum_{m=1}^{N_m}\beta_m^2\,.
	\label{eq:cost_lm}
\end{equation}
where $\lambda\ge0$ denotes an user defined dampening parameter.
In the above function, $t_i(\bm{\beta})\approx t_i(\bm{\beta}^{[q-1]})+\sum_{m=0}^{N_m-1}j^{[q-1]}_{im}\left(\beta_m-\beta_m^{[q-1]}\right)$ represents a truncated Taylor expansion of $t(\bm{\beta})$ around the current estimation $\bm{\beta}^{[q-1]}$. In this representation, the terms $j_{im}$ denote the coefficients of the Jacobian matrix $\mathbf{J}\in\mathbb{R}^{N_p\times N_m}$, which are defined as:
\begin{equation}
	j_{im}= \left.\frac{\partial t_i(\bm{\beta})}{\partial \beta_m}\right\rvert_{\bm{\beta}=\bm{\beta}^{^{[q-1]}}}\,,\;\;\;\;\;\;\forall\range{i}{1}{N_a},\,\forall\range{m}{1}{N_m}\,.
\end{equation}

Eq.~\ref{eq:cost_lm} admits as analytical solution:
\begin{equation}
	\bm{\beta}^{[q]}=\bm{\beta}^{[q-1]}+\left(\mathbf{J}\T\mathbf{J}+\lambda\mathbf{I}\right)^{-1}\mathbf{J}\T\mathbf{e}\,,
\end{equation}
where $\mathbf{I}$ denotes an identity matrix and the vector $\mathbf{e}$, whose $i$-th coefficient is defined as $e_i:=t_i(\bm{\beta}^{[q-1]})-w_i$, denotes the current estimation residual. When a certain convergence condition is verified (e.g. after a given number of iterations), the result of the last update is chosen as the desired estimation $\hat{\bm{\beta}}$. Additional implementation details, e.g. to define a criterion to assign the value of $\lambda$, to simplify the evaluation of $\mathbf{J}$, and to evaluate the stopping conditions on the iterations are given in the related paper~\cite{More78}.

\bibliography{biblio.bib}


\end{document}

