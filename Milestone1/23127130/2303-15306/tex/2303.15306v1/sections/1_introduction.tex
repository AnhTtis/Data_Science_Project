\section{Introduction}
\label{sec:introduction}

One of the main factors that influence music perception is the hierarchical structure of music compositions. 
Regardless of their level of musical knowledge and harmonic sensitivity \cite{tan1981harmonic} or their cultural origins \cite{stevens2012music}, listeners are able to use intuitive knowledge to organize their perception of musical structures \cite{chiappe1997phrasing}. 
Indeed, there is empirical evidence that neural activity correlates with musical structure in listeners' perception \cite{knosche2005perception}. The structuring and predictability of musical compositions is also recognised as a viable therapy in the treatment and assessment of children and adolescents with autistic spectrum disorder \cite{wigram2006music}.

Music structuring is one of the tools used by composers to tell a story. According to \cite{burns1987typology} ``Music-making is, to a large degree, the manipulation of structural elements through the use of repetition and change.''. The repetition of harmonic progressions (sequences of chords), in particular in the context of western tonal music, gives to artists the ability to guide listeners through a journey that creates dramatic narratives, conveying a sense of conflict that demands a solution \cite{temperley2004cognition}.
%It is by means of repetition of harmonic progressions (sequences of chords) that an artist is able to guide listeners through a journey that creates dramatic narratives, conveying a sense of conflict that demands a solution \cite{temperley2004cognition}.

\medskip
\begin{figure}[ht]
    \centering
    \includegraphics[width=\textwidth]{images/helterskelter.png}
    \caption{Structure of Helter Skelter by The Beatles. Chords are presented in Harte format \cite{harte2005symbolic}.}
    \label{fig:helterskelter-structure}
\end{figure}

Figure \ref{fig:helterskelter-structure} shows the structure of \textit{Helter Skelter} by The Beatles, highlighting the musical chords of the song.
In the example, by means of the alternation between \textit{verse} and \textit{refrain} the artist establishes a common repetitive pattern. The addition of an instrumental section after the second \textit{refrain} and the repetition of the \textit{intro} reinforces the repetitive aspect of the composition. The upcoming \textit{outro} section denies the expectation of a new \textit{verse}, right before the song ends. 
Expectation and the way it is fulfilled or denied is an essential part in musical enjoyment \cite{temperley2004cognition}.
In fact, it has been shown empirically that the emotional response to a musical composition varies as the degree of repetition changes \cite{livingstone2012emotional}. 
Understanding musical structures is hence fundamental in music analysis and composition. 
Artists can benefit from the feedback provided by a system able to highlight possible hierarchical structures in their compositions.


Music structure segmentation is a broad term related to the study of musical form, which describes how musical pieces are structured. In particular it can be divided in two main categories: phrase-structure segmentation and global segmentation \cite{giraud2016computational}. Phrase-structure consists in detecting sections from the melodic information of a piece. While the aim of phrase-structure is not to obtain a global segmentation, the detected sections provide valuable insights in the task of global segmentation.
In the following, we will refer to music structure segmentation as the task of global segmentation. Music structure segmentation is a music information retrieval (MIR) task that consists in identifying and labelling key music segments (e.g. \textit{chorus}, \textit{verse}, \textit{bridge}) of a music piece \cite{McCallum19unsupervised}. 
Given a musical composition, its musical segmentation consists in the identification of non-overlapping segments, which we will refer to as sections. Each section is characterized by a label that classifies its function such as \textit{intro} or \textit{verse} in figure \ref{fig:helterskelter-structure}. 
A correct segmentation does not necessarily assign the correct labels to each section of the composition, but rather focuses on the correct estimation of the boundaries of each section. 
Once boundaries has been accurately predicted, a labeling process is performed to obtain the final annotation \cite{nieto2020audio}.

Most of the recent methods and research approaches are based on audio analysis techniques \cite{nieto2020audio}, nonetheless harmonic information, isolated from tempo and rhythm, have been successfully used in several tasks in the field of music information retrieval (MIR) (e.g. \cite{de2013geometrical, de2013structural}).

In this paper, we focus on the music structure segmentation task by only taking into account harmonic information extracted from symbolic notations (music chord annotations). 
The assumption behind this approach is that the identification of harmonic sub-sequences (harmonic patterns) can be influential in defining the structure of a song and the sections of which it is composed.
For instance, by taking a closer look at Figure \ref{fig:helterskelter-structure} it is easy to notice how harmonic information can provide valuable information in the structure segmentation task: all \textit{verse}s are roughly based on the same harmonic progression (\textit{E}, \textit{G}, \textit{A}, \textit{E}) while \textit{refrain}s are based on a different harmonic progression (\textit{A}, \textit{E}, \textit{A}, \textit{E}, \textit{E}). 
A segmentation strongly based on those recurrent patterns is likely to be coherent with the way the composer shaped the progression in the first place.

\medskip
%In this paper, we propose a novel chord embedding method, based on the pitch class of a chord (the notes that compose a chord), and a novel structure segmentation method that achieves new state-of-the-art results on musical composition segmentation.
The objective of this paper is threefold: 
\begin{enumerate*}[label=(\roman*)]
    \item we propose \textit{pitchclass2vec}, a novel chord embedding method;
    \item we use this encoding with a recurrent neural network on a corpus of musical chords; and
    \item we compare the performance of the encoding with the state-of-the art methods in the field.
\end{enumerate*}

The chord embedding method proposed, \textit{pitchclass2vec}, encodes a chord using a one-hot encoding of the notes that compose it by making use of word embedding techniques. 
Each embedded chord is defined to be similar to the embedding of its neighbouring chords in an harmonic progression. This formalization is supposed to approximate the semantic meaning of a chord \cite{sahlgren2008distributional} and has been widely used in the natural language processing field \cite{mikolov2013word2vec, bojanowski2018fasttext}.
We use \textit{pitchclass2vec} embeddings to train an LSTM neural network that predicts the section of each chord. 
Through its recurrent layers the neural network is able to learn relationships between the elements of a sequence. 
This allows the model to detect repetitive patterns of the harmonic progression and predict the a segmentation of the whole composition.  The model provides a baseline to test the efficacy of the proposed chord embedding method.
% State-of-the-art results are achieved in the task of music structure segmentation on symbolic harmonic data when compared to related works. This provides evidences that \textit{pitchclass2vec} is more suited to accurately represents harmonic progressions and chords when compared to related embedding methods and can be beneficial to all the other MIR tasks that involve symbolic harmonic information.
State-of-the-art results are achieved in the task of music structure segmentation on symbolic harmonic data, providing evidences that \textit{pitchclass2vec} is able to provide accurate chord representations.
However, the embedding method employed here for the music segmentation task, can be used in a variety of applications in the field of Music Information Retrieval, such as retrieving harmonically related pieces \cite{de2013geometrical}, automatic chord recognition \cite{ohanlon2021fifthnet} and music genre classification \cite{liang2020pirhdy}.

%%%%%%%%%%%%%%%%%%%%%%
% **before the next paragraph, summarise the contribution of the paper: the method for chord embeddings and the method for segmentation***
%%%%%%%%%%%%%%%%%%%%%

\medskip
The paper is organised as follows: Section \ref{sec:related} introduces the related works, Section \ref{sec:model} describes the novel chord embedding method and the recurrent neural network used for the segmentation task. Section \ref{sec:experiments} presents the experiments performed and Section \ref{sec:results} gives an overview of the obtained results. Finally in Section \ref{sec:conclusion} we discuss the results and new research directions to be explored.
