\section{Conclusion and Future Work}\label{sec:conclusion}

In this article, we presented a new embedding method for musical chords, \textit{pitchclass2vec}, that considers the component notes of the chord (also called \textit{pitchclass}), instead of the chord label, as used in embedding methods in the natural language processing field.  
In addition, we proposed hybrid embedding forms, which combine embedding on the chord label and the novel \textit{pitchclass2vec}. 
We compared different embedding models, including \textit{pitchclass2vec} with the state-of-the-art approach in the field of music structural segmentation. 
We used ChoCo, a dataset of chord annotations, for training the embeddings and Billboard, a dataset of structurally annotated tracks, for the music segmentation task. We used the different types of embeddings on a recurrent neural network (LSTM).
The results obtained by using our embeddings outperform the state of the art in every case, with the best result obtained by \textit{pitchclass2vec}, achieving a pairwise F1 score of $0.548$ and an over-under-segmentation F1 score of $0.538$. 

\textit{Pitchclass2vec} is effectively able to learn the harmonic relationships that ties different chords together. Even though the experiments based on \textit{fasttext} and \textit{word2vec} proves to be effective as well, the musical theoretical approach upon which we base \textit{pitchclass2vec} is an essential factor that needs to be taken into account. The presented embedding model proves to be a promising method to improve results in MIR tasks that can be complemented with harmonic information. Moreover, it provides a valuable tool to better understand and analyse harmonic progressions, since it allows a richer comparison between chords and chord sequences when compared to string labels.
%\textit{Pitchclass2vec} is effectively able to learn the harmonic relationships that ties different chords together. Even though the experiments based on \textit{fasttext} and \textit{word2vec} proves to be effective as well, the musical theoretical approach upon which we base \textit{pitchclass2vec} is an essential factor that needs to be taken into account when music information retrieval tasks are tackled using musical chords as well.

%The results obtained with the presented LSTM model and the flexibility of \textit{pitchclass2vec} embeddings seem to be a promising research path in music structure segmentation.


There are additional information that we plan to integrate on \textit{pitchclass2vec} to obtain a richer and more accurate representation.
For instance, one of the main limitations of our approach stems from the fact that we do not take temporal information into account. 
We plan to test this possibility by using the temporal information directly in the embedding process and further modify the LSTM model to condition the classification of the section of a chord based on its duration.
As discussed in Section \ref{sec:model}, chord labels have their own semantic as well. 
Since the hybrid models proposed did not directly result in more accurate results, we plan on expanding the \textit{pitchclass2vec} method to take into account the label of a chords as well directly in its embedding model.
To obtain a semantically richer representations we plan on enhancing \textit{pitchclass2vec} by using deep contextual word embeddings \cite{peters2018elmo} along with knowledge enhancement techniques \cite{peters2019knowledge} that combines domain-specific ontologies, such as \cite{kantarelis2023functional}, with deep contextual word embeddings.

Moreover, we plan an in-depth analysis of \textit{pitchclass2vec} training parameters, described in section \ref{sec:implementation-details}, since in \cite{caselles2018word2vec} the authors showed that, on a product recommendation task, carefully optimized hyper-parameters nearly double the final accuracy on all the experiments.

It is worth mentioning that the LSTM model that we implemented for the structure segmentation task does not take advantage of two fundamental aspect of the task itself: the segmentation of a musical piece should be conditioned by its musical genre. 
In fact, the annotation guidelines provided in \cite{smith2011salami} defines some genre-specific labels and encourage their use whenever applicable. 
Even though a relabeling process can partially solve this issue, the need of genre-specific labels proves that the use-cases of specific structure labels, for instance \textit{theme}, might be different in different musical genres.
%It is important to remark that, in the formalization of the structure segmentation task, label assignment is not a fundamental aspect, being boundary prediction the more challenging aspect. 
%We will accommodate this aspect by re-framing our LSTM model objective as a supervised clustering problem instead of a classification problem.
Finally, as already discussed in section \ref{sec:experiments}, many recent techniques has shown to be effective in increasing the accuracy of different tasks in the NLP field \cite{huang2015crfbilstm, shi2015convlstm, wang2016attentionlstm} when using continuous word representations. We plan to address these issues in future works.

% \newpage
   
