\section{Preliminaries}

\paragraph{A Büchi Automaton}
   is a tuple $\aut{A} = (\Sigma, Q, \delta, q_0, F)$ where $\Sigma$ is a finite alphabet; $Q$ is a set of states; $\delta \subseteq Q \times \Sigma \times Q$ is the transition relation; $q_0\in Q$ is the initial state; and $F\subseteq Q$ is the set of accepting states.
    A \emph{run} of the Büchi automaton $\aut{A}$ on a word $\sigma \in \Sigma^\omega$ is an infinite sequence $q_0~q_1~q_2 \dots \in Q^\omega$ of states such that for all $i \in \mathbb{N}, (q_i, \sigma_i, q_{i+1}) \in \delta$. An infinite word $\sigma$ is \emph{accepted} by $\aut{A}$ if there is a run on $\sigma$ with infinitely many $i \in \mathbb{N}$ such that $q_i \in F$. The language of $\aut{A}$, $\mathcal{L}(\aut{A})$, is the set of words accepted by $\aut{A}$.
