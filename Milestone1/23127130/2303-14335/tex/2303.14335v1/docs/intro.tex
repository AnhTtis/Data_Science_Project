\section{Introduction}
\label{sec:intro}  % \label{} allows reference to this section


The general process of MPLD is to assign the layout features close to each other into the different masks to enhance the lithography resolution
since the features are far away enough to be printed with the existing lithography techniques.
Previous literature generally models the MPLD to graph coloring problems and solves them using linear programming (LP) methods or variants.
Unlike the classical coloring problems, the MPLD problems have their features.
1) The graph node representing the layout polygon can be split into multiple polygon segments, called stitch.
2) There are other rules beyond the wildly adopted spacing constraint for the same color.
Those constraints impose different challenges to the MPLD problems.
Typically, MPLDs are formulated into mathematical optimization models, which can be roughly categorized into three types~\cite{openmpl}:
1) integer linear programming (ILP) and its relaxation, 2) graph-based methods, 3) search-based approaches.
Specifically, ILP for double patterning layout decomposition (DPLD)~\cite{DPL-ICCAD2009-Xu,DPL-TCAD2010-Kahng,DPL-TCAD2010-Yuan,SADP-ASPDAC2014-Gao}
or triple patterning layout decomposition (TPLD)~\cite{TPL-TCAD2015-Yu,TPL-ICCAD2013-Yu,TPLEC-JM3-2015-Yu, TPL-JM3-2017-Lin,TPL-SPIE2016-Lin,TPL-ICCAD2011-Yu,TPLEC-SPIE2013-Yu,TPL-DAC2014-Yu,TPL-SPIE2012-Lucas,TPL-SPIE2014-Yu}.
The relaxation techniques for ILP methods are well-researched due to the $\mathcal{NP}$-hardness of TPLD and QPLD~\cite{TPL-TCAD2015-Yu,TPL-JM3-2017-Lin,TPL-TC2017-Li,MPL-DAC2015-Pan}.
The graph feature of the input layout makes it natural to handle the MPLD with graph-theoretical algorithms,
\eg, the maximal-independent set (MIS) \cite{TPL-DAC2012-Fang}, the shortest path \cite{TPL-ICCAD2012-Tian}.
Another category is to use search-based algorithms following the divide-and-conquer principle,
performing the search procedure on the sub-graphs~\cite{TPL-DAC2013-Kuang,TPL-DAC2016-Chang}.
Nevertheless, graph coloring-based models encounter substantial obstacles when sophisticated rules are required.
To address the intricate rules and density balance, exact cover(EC)-based MPLD models have been suggested~\cite{TPL-TCAD2017-Jiang}.


% Add more of the previous works.
% Add more citation of prof Bei.
% Add more citation of Haoyu Yang, Tinghuan.
% Add more citation for liwei.


% ILP
% EC

Donald Knuth invented Algorithm $\text{X}^{\ast}$ to solve the EC problem
and further suggested an efficient implementation technique called dancing links (DLX)~\cite{knuth2000dancing},
using doubly-linked circular lists to represent the matrix of the problem.
Jiang and Chang~\cite{TPL-TCAD2017-Jiang} designed a general and flexible MPLD framework based on augmenting DLX with LD task-related treatments,
which can concurrently consider complex coloring rules and maintain density balancing.
Li \etal presented OpenMPL~\cite{openmpl}, an open-source LD framework,
which introduced an improved flexible EC-based algorithm achieving better quality with a sacrifice in the runtime.
However, this sacrifice in runtime is even more evident in the larger industrial designs, making the EC-based methods less practical when compared with ILP-based methods.
With the increasing power of modern graphics processing units (GPUs), numerous successful applications use GPU to
accelerate the design automation tasks, \eg, mask optimization\cite{OPC-ICCAD2021-DevelSet,OPC-DATE2021-Yu,OPC-ICCAD2020-DAMO,MCH-DATE2022-Yang,AdaOPC-iccad-zhao,OPC-TCAD2020-Geng}, design space exploration \cite{PTPT-TCAD22-Geng,HS-TCAD22-Geng,BSL-TNNLS22-CHEN,HSD-TCAD2022-Geng}, and layout generation~\cite{LayouTransformer-iccad22-wen,deepattern-dac19-yang,pc-ispd21-li}.
Recent work pioneers using GPU and deep learning on pre or-post process of MPLD~\cite{OPC-ICCAD2017-Ma,OPC-DAC2020-Zhong,TPL-DAC2020-Li,MPL-VLSI-SOC2017-Ma}.
However, the acceleration of the Layout decomposition algorithm itself has not been explored.
It's necessary to explore the GPU-accelerated EC-based methods for large industrial designs.
In this paper, we propose a GPU-accelerated matrix cover algorithm for MPLD.
Our main contributions are:
\begin{itemize}
  \item We replace the CPU-based dancing link algorithms by leveraging the CUDA indexing model to solve the EC-based MPLD problems.
  \item We apply task parallelism to decompose the layout graph and improve the computation efficiency by parallel execution of CUDA kernel functions.
  \item We develop our GPU acceleration MPLD algorithms on top of the open-sourced decomposer OpenMPL~\cite{openmpl} to ensure usability and scalability.
\end{itemize}

% However openmpl not using GPU and not good in the large VLSI cases.
% GPU has many successful cases in EDA, my OPC cite myself.
% leverage the GPU to accelarate the MPLD is necessary.
% but it is not easy to directly using the GPU


\begin{figure}[tb!]
  \centering
    \subfloat {\includegraphics[width=.24\linewidth]{stitch-1}} \hspace{5em}
    \subfloat {\includegraphics[width=.24\linewidth]{stitch-2}} \\ \vspace{.5em}
    \subfloat {\includegraphics[width=.24\linewidth]{stitch-3}} \hspace{5em}
    \subfloat {\includegraphics[width=.24\linewidth]{stitch-4}} \vspace{.4em}
    \caption{
      An example of TPLD with stitches.
      (a) The input layout consists of multiple polygons.
      (b) The layout graph (LG) construction process. The LG is 4-clique therefore not 3-colorable.
      (c) Stitches insertion process. Insert two stitches into the original LG to eliminate the 4-clique.
      (d) Decomposition process. The LG will be decomposed using a coloring solver with stitch candidate generation.
      The final layout will be decomposed into three masks colored with different colors.
    }
  \label{fig:stitch}
\end{figure}

