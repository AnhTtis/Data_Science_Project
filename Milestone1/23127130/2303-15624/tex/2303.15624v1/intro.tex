Evaluation of Feynman integrals is both a classic problem and an expanding frontier of research in quantum field theory, with many areas of applications such as collider physics, statistical mechanics, and gravitational physics. A widely used method for numerical evaluations of Feynman integrals is Monte Carlo integration with sector decomposition \cite{Binoth:2000ps, Bogner:2007cr, Kaneko:2009qx, Borowka:2015mxa, Borowka:2017idc, Smirnov:2013eza, Smirnov:2015mct}. Another method, numerical series solutions of differential equations \cite{Moriello:2019yhu, Hidding:2020ytt, Liu:2017jxz, Liu:2021wks, Liu:2022chg, Armadillo:2022ugh, Hidding:2022ycg}, enjoyed intense developments in the last few years. Some other methods, which are either inherently numerical or can be used numerically, include Mellin-Barnes representations \cite{Usyukina:1992jd, Usyukina:1993ch, Smirnov:1999gc, Tausk:1999vh, Czakon:2005rk, Smirnov:2009up, Gluza:2007rt, Belitsky:2022gba}, difference equations from dimensional recursion \cite{Laporta:2000dsw, Lee:2012te, Lee:2022art}, Loop-tree duality \cite{Catani:2008xa, Runkel:2019yrs, Capatti:2019ypt}, and tropical Monte Carlo integration \cite{Borinsky:2020rqs, Borinsky:2023jdv}. See also books \cite{Weinzierl:2022eaz, Smirnov:2012gma} which comprehensively review both numerical and analytic methods. Despite these developments, numerical evaluation of Feynman integrals still presents challenges in the ongoing quest for higher precisions in perturbative calculations, and new explorations are warranted.

A fruitful recent idea in theoretical physics is the use of positivity constraints, e.g.\ arising from the unitarity of a Hilbert space, to constrain unknown parameters from first principles, sometimes reaching great accuracy. Prominent examples include the conformal bootstrap \cite{Rattazzi:2008pe}, the non-perturbative S-matrix bootstrap (see review \cite{Kruczenski:2022lot} and references within), and EFT positivity bounds \cite{Adams:2006sv, Bellazzini:2020cot, Caron-Huot:2020cmc, Tolley:2020gtv, Arkani-Hamed:2020blm, Caron-Huot:2021rmr, Caron-Huot:2022ugt}. Some of the predictions in the S-matrix and EFT contexts have been checked against explicit perturbative calculations involving Feynman integrals \cite{Bern:2021ppb, Chen:2022nym, Bellazzini:2022wzv}, so it is natural to explore positivity properties for Feynman integrals themselves. What directly inspired this paper, though, is recent applications of positivity constraints to bootstrapping simple quantum mechanical systems \cite{Lin:2020mme, Han:2020bkb, Berenstein:2021dyf, Berenstein:2022unr} and lattice models \cite{Anderson:2016rcw, Kazakov:2022xuh, Cho:2022lcj}, which crucially use various linear identities that are reminiscent of integration-by-parts identities for Feynman integrals \cite{Chetyrkin:1981qh} as well as a numerical technique known as semidefinite programming \cite{vandenberghe1996semidefinite} which will be adapted to our calculations. Semidefinite programming was introduced to theoretical physics by Ref.~\cite{Poland:2011ey} in the conformal bootstrap and subsequently applied to wider contexts.

In this work, we will formulate positivity constraints for \emph{Euclidean Feynman integrals}, which can be either (1) integrals in Euclidean spacetime or can be rewritten in Euclidean spacetime after a trivial Wick rotation, or (2) integrals in Minkowskian spacetime but with kinematics in the so called \emph{Euclidean region}, i.e.\ with center-of-mass energy of incoming momenta below the particle production threshold. In case (1), the loop integrand in Euclidean momentum space has non-negative propagator denominators, and the integrand is non-negative as long as the numerator is non-negative. In case (2), the integral is real due to the absence of Cutkosky cuts. After Feynman parametrization, the parametric integrand involves non-negative graph polynomials and stays non-negative when multiplied by a positive function of the Feynman parameters. In both cases, a non-negative integrand implies a non-negative integral as long as the integral is convergent, i.e.\ has no ultraviolet or infrared divergences.
The initial restriction to convergent integrals is not a fundamental limitation, as divergent integrals can be rewritten as linear sums of convergent integrals multiplied by divergent coefficients \cite{vonManteuffel:2014qoa}, and then it suffices to evaluate the convergent integrals as a Taylor series in the dimensional regularization parameter $\epsilon$.

We will see that positivity constraints, expressed in the language of semidefinite programming, are strong enough to precisely determine the values of Feynman integrals. Moreover, rigorous error bounds can be obtained. Our machinery relies on linear relations between Feynman integrals, including integration-by-parts identities \cite{Chetyrkin:1981qh} and dimension-shifting identities \cite{Tarasov:1996br}, to express all positivity constraints as linear constraints on the values of a set of master integrals.

The outline of the paper is as follows. In Section \ref{sec:bubble}, we introduce the main methods using the simple example of massive one-loop bubble integrals. Specifically, Subsection \ref{subsec:bubbleIBP} gives a short review of two kinds of linear identities for Feynman integrals, integration-by-parts identities and dimension shifting identities. Subsection \ref{subsec:posMomSpace} discusses positivity constraints in Euclidean momentum space, starting from ad hoc constraints that narrow down the allowed value of the bubble master integral to $\sim 50\%$ accuracy at an example kinematic point, then developing the machinery of semidefinite programming to reach an accuracy of $10^{-14}$. We switch to Feynman parameter space in Subsection \ref{subsec:posFeynSpace}, which allows calculating the integrals in a wider region of kinematic parameters below the two-particle cut threshold, while many steps of the calculations are unchanged from the momentum-space case. Subsection \ref{subsec:epExpansion} formulates positivity constraints for $\epsilon$ expansions of Feynman integrals in dimensional regularization. Subsection \ref{subsec:numericalDiff} presents an alternative method for calculating the $\epsilon$ expansion based on numerical differentiation of results w.r.t.\ the spacetime dimension. Having laid out most of the methods, in Section \ref{sec:banana}, we present a nontrivial application to three-loop banana integrals with four unequal masses. Due to a large number of undetermined master integrals, semidefinite programming becomes essential in efficiently solving the positivity constraints. We calculate all master integrals at an example kinematic point up to the second order in $\epsilon$ expansion in $d=2-2\epsilon$, and a detailed account of the numerical accuracies is given. We end with some discussions in Section \ref{sec:discussions}.
