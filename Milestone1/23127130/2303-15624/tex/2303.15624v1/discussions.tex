We have demonstrated a new method for evaluating Feynman integrals which, to our best knowledge, is the first method based on inequality constraints, while previously exploited consistency conditions for Feynman integrals are based on equality constraints (such as the vanishing of the coefficient of a certain spurious singularity). Our calculation strategy is writing down an infinite class of convergent integrals with non-negative integrands and reducing them to linear sums of a set of master integrals. This constraints an infinite number of linear sums of master integrals to be non-negative. A truncated set of the constraints can be solved as a semidefinite programming problem in mathematical optimization. Surprisingly, the constraints appear strong enough to determine the integrals to any desired precision since the bounds appear to converge exponentially as the truncation cutoff is increased. Like the method of differential equations \cite{Kotikov:1990kg,Bern:1993kr,Remiddi:1997ny, Gehrmann:1999as,Henn:2013pwa}, our method relies on integration-by-parts (IBP) identities, but instead of using differential equations to transport the values of the integrals across kinematic space, we only use IBP information at a single point in kinematic space.

Though our study is preliminary, the numerical results are promising. We have demonstrated the applicability of our methods to a nontrivial example, namely three-loop banana integrals with four unequal internal masses in $d= 2 - 2\epsilon$ dimensions. With modest computational resources, we evaluated the $\mathcal O(\epsilon^0)$ part of all the 11 nontrivial master integrals to a relative accuracy of at least $10^{-9}$. The accuracies for $\mathcal O(\epsilon^1)$ and $\mathcal O(\epsilon^2)$ terms are lower, though only slightly so for $\mathcal O(\epsilon^1)$ terms when the numerical differentiation method is used. For all but the smallest problems, extended-precision floating point arithmetic is needed to ensure numerical stability in the semidefinite programming solver, similar to what was encountered in the conformal bootstrap \cite{Simmons-Duffin:2015qma} and the quantum mechanics bootstrap \cite{Berenstein:2022unr}. We note that extended precision is also generally need in the evaluation of Feynman integrals by series solutions of differential equations as observed in e.g.\ Refs.~\cite{Moriello:2019yhu, Hidding:2020ytt}.

We have also revealed hidden consistency relations that link different terms in the $\epsilon$ expansions of Feynman integrals. As explained in Section \ref{subsubsec:eps_generic}, for any (quasi-) finite Feynman integral without numerators, the $\epsilon$ expansion terms (appropriately normalized) must give rise to a positive-semidefinite Hankel matrix. This is an extremely general statement which can be checked against a huge number of Feynman integral computations in the literature, because many Feynman integrals have Euclidean regions and it is believed that a quasi-finite basis exist for any family of integrals \cite{vonManteuffel:2014qoa}. This result is an elementary consequence of our analysis but has not been previously exposed in the literature. Such constraints have been solved numerically in our paper to predict the $\epsilon$ expansion terms to high accuracy. We have also formulated an alternative method to obtain the $\epsilon$ expansion terms by numerical differentiation of semidefinite programming solutions with respect to the spacetime dimension. The above two methods for calculating $\epsilon$ expansion terms are complementary and we have found cases in which either of them outperforms the other in accuracy.

Our new method for calculating Feynman integrals is analogous to recent developments in bootstrapping quantum mechanics systems and lattice models \cite{Lin:2020mme, Han:2020bkb, Berenstein:2021dyf, Berenstein:2022unr, Anderson:2016rcw, Kazakov:2022xuh, Cho:2022lcj}. For example, the role of IBP and dimensional-shifting identities in our work is analogous to the role of moment recursion relations in the quantum mechanics bootstrap. Analogous identities also appear in EFT bounds as ``null constraints'' from crossing symmetry \cite{Caron-Huot:2020cmc}.\footnote{We thank Francesco Riva for pointing out this connection.} While our work has imported techniques developed in non-perturbative contexts to perturbative physics, in the reverse direction, the differential equation method of perturbative calculations has been applied to non-perturbative lattice correlation functions in Refs.~\cite{Gasparotto:2022mmp, Weinzierl:2020nhw, Cacciatori:2022mbi}, also exploiting identities similar to those from IBP. Therefore, we expect a fruitful exchange of techniques between perturbative and non-perturbative calculations.

Finally, we speculate on possible future work. Except for the generic constraints on the $\epsilon$ expansion, this paper has mainly treated massive Feynman integrals, and for integral families involving massless internal lines, it would be necessary to identify non-negative integrals free of not only ultraviolet but also infrared divergences to generate the positivity constraints. To extend our method to integrals outside the Euclidean region, it remains to be seen how positivity constrains can be formulated, possibly for real and imaginary parts separately after a suitable deformation of the integration contour. Connections with other notions of positivity relevant for Feynman integrals \cite{Abreu:2019wzk, Yang:2022gko} remain to be explored. Another interesting question is whether positivity constraints can be used to understand complete scattering amplitudes (rather than individual Feynman integrals) at a fixed order in perturbation theory, in light of numerical observations in $\mathcal N=4$ super-Yang-Mills theory amplitudes in Ref.~\cite{Dixon:2016apl}.
