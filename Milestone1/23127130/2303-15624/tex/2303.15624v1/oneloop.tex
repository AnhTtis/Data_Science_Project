We consider the following family of Feynman integrals in Minkowski spacetime parametrized by two integers $a_1$ and $a_2$,
\begin{equation}
    I_{a_1,a_2}^d \equiv \int \frac {d^d l \, e^{\gamma_E \epsilon}}{i \pi^{d/2}} \frac{1} {(- l^2 + m^2)^{a_1} [-(p + l)^2 + m^2]^{a_2}} , \label{eq:selfEnergyInt}
\end{equation}
which correspond to the diagram in Fig.~\label{fig:bubble} but with the propagator denominators raised to general integer powers.
%
\begin{figure}
  \centering
  \includegraphics[width=0.3\textwidth]{figs/bubble.pdf}
  \caption{The one-loop bubble integral with external legs of virtuality $p^2$ and two internal massive line with the same squared mass $m^2$.}
  \label{fig:boundPlot}
\end{figure}
%
The external mass is $\sqrt{p^2}$ and the internal mass for the two propagators is $m$, with the two propagators raised to powers $a_1$ and $a_2$. If either $a_1$ or $a_2$ is non-positive, Eq.~\eqref{eq:selfEnergyInt} becomes a massive tadpole integral possibly with a numerator. If $a_1$ and $a_2$ are both non-positive, the integral vanishes in dimensional regularization. The spacetime dimension $d$ is equal to $4- 2 \epsilon$, with $\epsilon$ being the dimensional regularization parameter.
Eq.~\eqref{eq:selfEnergyInt} can be re-written, by Wick rotation of the integration contour, as the following integrals in \emph{Euclidean} spacetime,
\begin{equation}
    I_{a_1,a_2}^d \equiv \int \frac {d^d \bm l \, e^{\gamma_E \epsilon}}{\pi^{d/2}} \frac{1} {(\bm l^2 + m^2)^{a_1} [(\bm p + \bm l)^2 + m^2]^{a_2}} , \label{eq:selfEnergyIntEuc} \, .
\end{equation}
where $\bm p^2 \equiv -p^2$.

We will use positivity constraints to numerically evaluate bubble integrals Eq.~\eqref{eq:selfEnergyInt} for $p^2 < 4m^2$, i.e.~below the two-particle production threshold. We first consider the case $p^2 < 0$, i.e.~$\bm p^2 > 0$. In this case, $\bm p$ can be literally embedded in Euclidean spacetime as a vector with real-valued components, and we will derive positivity constraints starting from the loop momentum integral Eq.~\eqref{eq:selfEnergyIntEuc}. We will subsequently present a treatment applicable to all $p^2 < 4m^2$ by using Feynman parameter representations of the integrals. Before actual calculations, we first briefly review linear identities for the Feynman integrals involved, arising from \emph{integration by parts} and \emph{dimension shifting}. Efficiently solving IBP identities for more complicated Feynman integrals is a major research problem, and we refer readers to Refs.~\cite{Laporta:2000dsw, Laporta:1996mq, Anastasiou:2004vj, vonManteuffel:2012np, Smirnov:2008iw, Smirnov:2019qkx, Lee:2012cn, Lee:2013mka, Maierhofer:2017gsa} for relevant computational algorithms and software.

\subsection{Review: integration-by-parts (IBP) and dimensional-shifting identities}
\label{subsec:bubbleIBP}
$I^d_{a_1,a_2}$ at different values of $a_1$ and $a_2$, as defined in Eq.~\eqref{eq:selfEnergyInt}, are related through integration-by-parts (IBP) identities \cite{Chetyrkin:1981qh}, as total derivatives integrate to zero without boundary terms in dimensional regularization. To derive the identities, we will write dot products as linear combinations of denominators and constants,
\begin{equation}
  l^2 = -(-l^2 + m^2) + m^2, \qquad 2 p \cdot l = -[-(p + l)^2 + m^2] + [- l^2 + m^2] - p^2 \, .
\end{equation}
The actual identities are
\begin{align}
  & \int \frac {d^d l \, e^{\gamma_E \epsilon}}{i \pi^{d/2}} \frac{\partial}{\partial l^\mu} \frac{p^\mu} {(- l^2 + m^2)^{a_1} [-(p + l)^2 + m^2]^{a_2}} = 0 \nonumber \\
  & \implies -a_1 p^2 I^d_{a_1+1,a_2} + a_2 p^2 I^d_{a_1,a_2+1} - a_1 I^d_{a_1+1,a_2-1} + (a_1-a_2) I^d_{a_1,a_2} + a_2 I^d_{a_1-1,a_2+1} = 0 \, , \label{eq:bubbleIBP1}
\end{align}
and
\begin{align}
  & \int \frac {d^d l \, e^{\gamma_E \epsilon}}{\pi^{d/2}} \frac{\partial}{\partial l^\mu} \frac{l^\mu} {(-l^2 + m^2)^{a_1} [-(p + l)^2 + m^2]^{a_2}} = 0 \implies \nonumber \\
  & 2a_1 m^2 I^d_{a_1+1,a_2} + a_2 (-p^2 + 2m^2) I^d_{a_1,a_2+1} + (d- 2a_1 - a_2) I^d_{a_1,a_2} - a_2 I^d_{a_1-1,a_2+1} = 0 \, , \label{eq:bubbleIBP2}
\end{align}
We also need the diagram reflection symmetry relations from relabeling $l \rightarrow -(p + l)$,
\begin{equation}
  I^d_{a_1, a_2} = I^d_{a_2, a_1} \, , \label{eq:bubbleSym}
\end{equation}
and the boundary condition in dimensional regularization,
\begin{equation}
  I^d_{a_1, a_2} = 0, \ \text{when both } a_1 \leq 0 \ \text{and }  a_2 \leq 0 \, . \label{eq:bubbleRestriction}
\end{equation}
Solving eqs.~\eqref{eq:bubbleIBP1}-\eqref{eq:bubbleRestriction}, e.g.\ by iteratively eliminating $I^d_{a_1,a_2}$ with the largest values of $|a_1| + |a_2|$, allow us to rewrite any $I^d_{a_1, a_2}$ with fixed spacetime dimension $d$ in terms of two \emph{master integrals},
\begin{equation}
  I^d_{1,1}, \quad I^d_{1,0}, \label{eq:bubbleNaiveBasis}
\end{equation}
i.e.\ a bubble integral and a tadpole integral. We will then change to an alternative basis $I_1$ and $I_2$, which is ultraviolet finite as $d$ approaches $4$ and in fact for any $d<6$,
\begin{equation}
  \begin{aligned}
    I^d_{2,1} &= \frac {d-3} {p^2 - 4 m^2} I^d_{1,1} + \frac {d-2} {2 m^2 (p^2 - 4 m^2)} I^d_{1,0} \, , \\
    I^d_{3,0} &= \frac {(d-2)(d-4)}{8 m^4} I^d_{1,0} \, .
  \end{aligned}
  \label{eq:bubbleFiniteBasis}
\end{equation}
\emph{For the rest of this paper, we will impose $d<6$ in the treatment of the bubble integrals and pay special attention to values of $d$ near $4$.}

The process of carrying out the above calculation and rewriting any integral as linear sums of master integrals is referred to as \emph{IBP reduction}. Here are two examples showing how other integrals are expressed as linear combinations of the basis Eq.~\eqref{eq:bubbleFiniteBasis},
\begin{align}
  I^d_{2,2} &= \frac{1}{p^2 (4m^2-p^2)} \Big( [(6-d)p^2 - 4m^2] I^d_{2,1} + 4 m^2 I^d_{3,0} \Big) \label{eq:I22result} \\
  I^d_{3,1} &= I^d_{1,3} = \frac{1}{2p^2 (4m^2-p^2)} \Big( [(4-d)p^2 + 4m^2] I^d_{2,1} + 2(p^2-2m^2) I^d_{3,0} \Big) \, . \label{eq:I31result}
\end{align}
The tadpole integral $I^d_{3,0}$ is well known in textbooks. The bubble integral with one of the propagator raised to a higher power, $I^d_{2,1}$, is also known analytically, but we will use positivity constraints to numerically evaluate it for the purpose of demonstrating our methods. We will write
\begin{align}
  I^d_{2,1} = \hat I^d_{2,1} \cdot I^d_{3,0} \, ,
\end{align}
and treat the normalized bubble integral
\begin{equation}
  \hat I^d_{2,1} = I^d_{2,1} / I^d_{3,0} \label{eq:bubbleMasterParam}
\end{equation}
as an unknown quantity to be bounded by positivity constraints. More generally, we define a ``hatted'' notation for integrals normalized against the finite tadpole integral,
\begin{equation}
  \hat I^d_{a_1, a_2} = I^d_{a_1, a_2} / I^d_{3,0} , \, \label{eq:bubbleNormalizedA1A2}
\end{equation}
under which Eqs.~\eqref{eq:I22result} and \eqref{eq:I31result} become
\begin{align}
  \hat I^d_{2,2} &= \frac{1}{p^2 (4m^2-p^2)} \Big( [(6-d)p^2 - 4m^2] \hat I^d_{2,1} + 4 m^2 \Big) \label{eq:I22resultNormalized} \\
  \hat I^d_{3,1} &= I^d_{1,3} = \frac{1}{2p^2 (4m^2-p^2)} \Big( [(4-d)p^2 + 4m^2] \hat I^d_{2,1} + 2(p^2-2m^2) \Big) \, . \label{eq:I31resultNormalized}
\end{align}
We also need dimensional-shifting identities.
Using the Schwinger parametrization, Eq.~\eqref{eq:selfEnergyInt} is rewritten as
\begin{equation}
  I^d_{a_1, a_2} = \frac {e^{\gamma_E \epsilon}} {\Gamma(a_1) \Gamma(a_2)} \int_0^\infty d x_1 \int_0^\infty d x_2 \, x_1^{a_1-1} x_2^{a_2-1} (x_1 + x_2)^{-d/2} \exp (- i \mathcal U / \mathcal F) \, ,
  \label{eq:bubbleSchwinger}
\end{equation}
where $\mathcal U$ and $\mathcal F$ are graph polynomials depending on $x_1$ and $x_2$,
\begin{equation}
  \mathcal U(x_1, x_2) = x_1+x_2, \quad \mathcal F(x_1, x_2) = m^2(x_1+x_2)^2 - p^2 x_1 x_2 - i 0^+ \, .
  \label{eq:bubbleFpolySchwinger}
\end{equation}
Therefore, $(d-2)$ dimensional integrals can be written as
\begin{align}
  I^{d-2}_{a_1, a_2} &= \frac {e^{\gamma_E \epsilon}} {\Gamma(a_1) \Gamma(a_2)} \int_0^\infty d x_1 \int_0^\infty d x_2 \, x_1^{a_1-1} x_2^{a_2-1} (x_1 + x_2)^{-(d-2)/2} \exp (- i U / F) \nonumber \\
  &= \frac {e^{\gamma_E \epsilon}} {\Gamma(a_1) \Gamma(a_2)} \int_0^\infty d x_1 \int_0^\infty d x_2 \, x_1^{a_1-1} x_2^{a_2-1} (x_1 + x_2) (x_1 + x_2)^{-d/2} \exp (- i U / F) \nonumber \\
  &= \frac {e^{\gamma_E \epsilon}} {\Gamma(a_1) \Gamma(a_2)} \int_0^\infty d x_1 \int_0^\infty d x_2 \, \big( x_1^{a_1} x_2^{a_2-1} + x_1^{a_1-1} x_2^{a_2} \big)
  (x_1 + x_2)^{-d/2} \exp (- i U / F) \nonumber \\
  &= a_1 I^d_{a_1+1, a_2} + a_2 I^d_{a_1, a_2+1} \, ,
  \label{eq:bubbleDimShift}
\end{align}
where the last line used Eq.~\eqref{eq:bubbleFpolySchwinger} to map different monomials in $x_1$ and $x_2$ to bubble integrals with different propagator powers.
Applying Eq.~\eqref{eq:bubbleDimShift} to the finite bubble master integral $I^d_{2,1}$ on the LHS of Eq.~\eqref{eq:bubbleFiniteBasis} in $(d-2)$ spacetime dimensions and then performing IBP reduction, we obtain
\begin{equation}
  I^{d-2}_{2,1} = \frac {2 (d-5)} {p^2 - 4m^2} I^d_{2,1} - \frac 2 {p^2 - 4m^2} I^d_{3,0} \, . \label{eq:dimshift1}
\end{equation}
The dimension-shifting formula for the tadpole integral can be obtained easily, e.g.\ by using closed-form results for tadpole integrals. The result is
\begin{equation}
  I^{d-2}_{3,0} = \frac{6-d}{2 m^2} I^d_{3,0} \, . \label{eq:dimshift2}
\end{equation}
Eqs.~\eqref{eq:dimshift1} and \eqref{eq:dimshift2} are the dimension-shifting identities that express the two master integrals in Eq.~\eqref{eq:bubbleFiniteBasis} in lower spacetime dimensions to the same master integrals in higher spacetime dimensions. By inverting Eqs.~\eqref{eq:dimshift1} and \eqref{eq:dimshift2}, we obtain dimension shifting identities in the reverse direction, i.e.\ expressing master integrals in higher spacetime dimensions in terms of master integrals in lower spacetime dimensions,
\begin{align}
  I^{d+2}_{2,1} &= \frac {p^2 - 4m^2} {2(d-3)} I^d_{2,1} - \frac {2m^2} {(d-3)(d-4)} I^d_{3,0} \, , \label{eq:dimshift1prime} \\
  I^{d+2}_{3,0} &= \frac{2 m^2} {4-d} I^d_{3,0} \, . \label{eq:dimshift2prime}
\end{align}

\subsection{Positivity constraints in loop momentum space}
\label{subsec:posMomSpace}
Here we focus on the case $p^2 <0$ and use the Wick-rotated expression for the integrals, Eq.~\eqref{eq:selfEnergyIntEuc}, with
\begin{equation}
  \bm p^2 = M^2 = -p^2 \, .
\end{equation}
We will later present a Feynman parameter-space treatment that seems more powerful and in particular works for all $p^2 < 4m^2$, but the loop momentum space treatment here will help build up intuitions.
\subsubsection{A crude first attempt}
\label{subsubsec:crude}
We first consider the following two convergent integrals with non-negative integrands in loop momentum space,
\begin{equation}
  \begin{aligned}
    I^d_{2,2} &= \int \frac {d^d \bm l \, e^{\gamma_E \epsilon}}{\pi^{d/2}} \frac{1} {(\bm l^2 + m^2)^2 [(\bm p + \bm l)^2 + m^2]^2} \, , \\
    I^d_{3,1} + I^d_{1,3} - 2 I^d_{2,2}  &= \int \frac {d^d \bm l \, e^{\gamma_E \epsilon}}{\pi^{d/2}} \frac{1} {(\bm l^2 + m^2) [(\bm p + \bm l)^2 + m^2]} \\
    &\quad \times \left(  \frac 1 {\bm l^2 + m^2} - \frac 1 {(\bm p + \bm l)^2 + m^2 } \right)^2 \, .
  \end{aligned}
\end{equation}
We have used the notation Eq.~\eqref{eq:selfEnergyInt} and the subsequent Wick rotation Eq.~\eqref{eq:selfEnergyIntEuc}. Since the integrals are massive and have no infrared divergence, a simple ultraviolet power-counting shows that the above integrals are convergent near $4$ dimensions. These integrals therefore have finite non-negative values,
\begin{equation}
  I^d_{2,2} \geq 0, \qquad I^d_{3,1} + I^d_{1,3} - 2 I^d_{2,2} \geq 0 \, .
\end{equation}
By IBP reduction as described in Section \ref{subsec:bubbleIBP}, the above inequalities translate into constraints on the finite master integrals on the LHS of Eq.~\eqref{eq:bubbleFiniteBasis}. The needed IBP reduction results are shown in Eqs.~\eqref{eq:I22result} and \eqref{eq:I31result}, in which we will rewrite $p^2=-M^2$. We use the parametrization Eq.~\eqref{eq:bubbleMasterParam} to factor out the positive tadpole integral $I^d_{3,0}$, finally arriving at
\begin{align}
  & \frac {I^d_{3,0}} {M^2 (M^2 + 4m^2)} \big[ \left((6-d)M^2 + 4 m^2\right) \hat I^d_{2,1} - 4 m^2 \big] \geq 0 \, , \nonumber \\
  & \frac {I^d_{3,0}} {M^2 (M^2 + 4m^2)} \big[ -\left((8-d)M^2 + 12 m^2\right) \hat I^d_{2,1} + (2 M^2 + 12 m^2) \big] \geq 0 \, , \nonumber \\
  & \text{for any } d < 6 \, ,
\end{align}
i.e.
\begin{equation}
  \frac{4m^2}{(6-d)M^2 + 4 m^2} \leq \hat I^d_{2,1} = I^d_{2,1} / I^d_{3,0} \leq \frac{2 M^2 + 12 m^2}{(8-d)M^2 + 12 m^2} \, , \quad \text{for any } d < 6 \, . \label{eq:bubbleCrudeResult}
\end{equation}
It is easily shown that the LHS of the inequality above is always less than the RHS when $d<6$, so the normalized bubble integral $\hat I^d_{2,1} = I^d_{2,1} / I^d_{3,0}$ is bounded in a finite range. When $d$ tends to $6$ from below, the LHS and RHS of the inequality both approach $1$, leading to the prediction that $I^d_{2,1} / I^d_{3,0} \to 1$ as $d \to 6$; this is exactly as expected since both $I^d_{2,1}$ and $I^d_{3,0}$ have an ultraviolet pole $1/(d-6)$ with the same coefficient. Specializing to the case $d=4$, Eq.~\eqref{eq:bubbleCrudeResult} becomes
\begin{equation}
  \frac{2m^2}{M^2 + 2 m^2} \leq \hat I^{d=4}_{2,1} \leq \frac{M^2 + 6 m^2}{2M^2 + 6 m^2} \, , \quad \text{for } d = 4 \, . \label{eq:bubbleCrudeResult4D}
\end{equation}
Let us arbitrarily choose an example numerical point,
\begin{equation}
  M^2 = 2, \quad m^2 = 1 \, , \label{eq:bubbleEucNumPoint}
\end{equation}
At this point, Eq.~\eqref{eq:bubbleCrudeResult4D} becomes
\begin{equation}
  0.5 \leq \hat I^{d=4}_{2,1} \leq 0.8 \, , \quad \text{for } M^2=2, m^2=1 \, . \label{eq:bubbleCrudeResultNum}
\end{equation}
This is consistent with the analytic result given in Appendix \ref{app:bubbleAnalytic} with $p^2 = -M^2$,
\begin{align}
  \hat I^{d=4}_{2,1}  &= I^{d=4}_{2,1} / I^{d=4}_{3,0} = \frac {2 m^2} {\beta M^2} \log \frac {\beta+1}{\beta-1} \, , \label{eq:bubbleAnalyticFinite} \\
  \beta & \equiv \sqrt{ 1 + \frac {4m^2}{M^2}} \ ,
\end{align}
which evaluates to
\begin{equation}
  \hat I^{d=4}_{2,1} \approx 0.7603459963 \, , \quad \text{at } M^2=2, m^2=1 \, . \label{eq:bubEucAnalytic}
\end{equation}

Since our crude positivity bound Eq.~\eqref{eq:bubbleCrudeResultNum} and the analytic result Eq.~\eqref{eq:bubbleAnalyticFinite} only depend on the dimensionless ratio $M^2/m^2$, we plot them in Fig.~\ref{fig:boundPlot}. It is not surprising that all three curves in the plot tend to $1$ as $M^2/m^2 \to 0$, since in this case we can set the external momenta to 0, and the bubble and tadpole integrals in Eq.~\eqref{eq:bubbleFiniteBasis} then become identical. The general observation is that our positivity bounds can become exact in special limits of kinematics or the spacetime dimension.
%
\begin{figure}
  \centering
  \includegraphics[width=0.6\textwidth]{figs/boundPlot.png}
  \caption{Comparison between ad hoc positivity bounds Eq.~\eqref{eq:bubbleCrudeResult4D} and the analytic result for the bubble integral $\hat I^{d=4}_{2,1}$ normalized according to Eq.~\eqref{eq:bubbleNormalizedA1A2}.}
  \label{fig:boundPlot}
\end{figure}
%
\subsubsection{Formulation in terms of matrix eigenvalues}
\label{subsubsec:eig}
In preparation for the introduction of semidefinite programming, we will first recast positivity constraints in terms of eigenvalues of an appropriate matrix. To simplify the notation of Eq.~\eqref{eq:selfEnergyInt}, let us define the following shorthand notations for the denominators,
\begin{equation}
  \rho_1 = -l^2 + m^2, \qquad \rho_2 = -(p+l)^2 + m^2 \, .
\end{equation}
Let us consider a class of positive-semidefinite integrals of the bubble family, parametrized by three real numbers $\alpha_1, \alpha_2, \alpha_3$,
\begin{equation}
  \int \frac {d^d \bm l \, e^{\gamma_E \epsilon}}{i \pi^{d/2}} \frac{1} {\rho_1^2 \rho_2}
  \left( \alpha_1 + \frac{\alpha_2}{\rho_1} + \frac{\alpha_3}{\rho_2} \right)^2
  = \sum\limits_{i,j} \alpha_i M_{ij} \alpha_j \\
  = {\vec \alpha}^T \, \mathbb M \, \vec \alpha \, , \label{eq:posAnsatzSize3}
\end{equation}
where the last line switched to a notation involving a length-3 column vector
\begin{equation}
  \vec \alpha = \begin{pmatrix} \alpha_1 \\ \alpha_2 \\ \alpha_3 \end{pmatrix}
\end{equation}
and a $3\times 3$ symmetric matrix $\mathbb M$, given by
\begin{equation}
  \mathbb M = \begin{pmatrix}
    I^d_{2,1} & I^d_{3,1} & I^d_{2,2} \\
    I^d_{3,1} & I^d_{4,1} & I^d_{3,2} \\
    I^d_{2,2} & I^d_{3,2} & I^d_{2,3}
  \end{pmatrix} \, , \label{eq:posMatrix3by3}
\end{equation}
using the index notation Eq.~\eqref{eq:selfEnergyInt}.
The expression Eq.~\eqref{eq:posAnsatzSize3} is non-negative for any choice of $(\alpha_1, \alpha_2, \alpha_3)$ because after Wick rotation, $\rho_1$ and $\rho_2$ are non-negative and the squared expression is also non-negative. Therefore, the symmetric matrix $\mathbb M$ must be positive-semidefinite, represented by the shorthand notation
\begin{equation}
  \mathbb M \succcurlyeq 0 \, , \label{eq:posGeq} \, .
\end{equation}

By IBP reduction, the matrix entries of $\mathbb M$ are integrals which can be re-expressed as linear sums of the two finite master integrals on the LHS of Eq.~\eqref{eq:bubbleFiniteBasis}. As an example,
\begin{align}
  \mathbb M_{23} &= \mathbb M_{32} = \int \frac {d^d \bm l \, e^{\gamma_E \epsilon}}{i \pi^{d/2}} \frac{1} {\rho_1^2 \rho_2^2} = I^d_{2,2},
\end{align}
which is then reduced to the finite master integrals according to Eq.~\eqref{eq:I22result}. Therefore $\mathbb M$ can be written as the sum of two individual master integral contributions,
\begin{equation}
  \mathbb M = I_{3,0}^d \mathbb M_1 + I_{2,1}^d \mathbb M_2 = I_{3,0}^d \left( \mathbb M_1 + \hat I^d_{2,1} \mathbb M_2 \right) \, , \label{eq:M1plusM2}
\end{equation}
using the notation $\hat I^d_{2,1} = I^d_{2,1} / I^d_{3,0}$ introduced in Eq.~\eqref{eq:bubbleMasterParam}. Since $I_{3,0}$ is itself positive, the positive-semidefiniteness of $\mathbb M$ implies
\begin{equation}
  \widetilde {\mathbb M} \equiv \mathbb M / I^d_{3,0} = \mathbb M_1 + \hat I^d_{2,1} \mathbb M_2 \succcurlyeq 0 \, , \label{eq:tildeMpos}
\end{equation}
again employing the shorthand notation introduced in Eq.~\eqref{eq:posGeq} to indicate positive-semidefiniteness. Equivalently, all the eigenvalues of $\mathbb M_1 + \hat I^d_{2,1} \mathbb M_2$ must be non-negative.
In Eq.~\eqref{eq:M1plusM2}, the matrices $\mathbb M_1$ and $\mathbb M_2$ contain entries that are rational functions of the spacetime $d$ and kinematic variables $p^2, m^2$, since IBP reduction always produces rational coefficients for master integrals. Although the general $d$ dependence is not complicated, we will present $\mathbb M_1$ and $\mathbb M_2$ in the $d=4$ case for brevity of presentation,
\begin{align}
  \mathbb M_1 \big|_{d=4} &=
  \begin{pmatrix}
    0 & \frac{2 m^2+M^2}{M^2 \left(4 m^2+M^2\right)} & -\frac{4}{M^2 \left(4 m^2+M^2\right)} \\
    \frac{2 m^2+M^2}{M^2 \left(4 m^2+M^2\right)} & \frac{\left(m^2+M^2\right) \left(6 m^2+M^2\right)}{3 m^2 M^2 \left(4 m^2+M^2\right)^2} & -\frac{2 m^2-M^2}{m^2 M^2 \left(4 m^2+M^2\right)^2} \\
    -\frac{4 m^2}{M^2 \left(4 m^2+M^2\right)} & \frac{M^2-2 m^2}{M^2 \left(4 m^2+M^2\right)^2} & -\frac{2 m^2-M^2}{m^2 M^2 \left(4 m^2+M^2\right)^2} \\
  \end{pmatrix} \, , \\
  \mathbb M_2 \big|_{d=4}  &=
  \begin{pmatrix}
    1 & -\frac{2 m^2}{M^2 \left(4 m^2+M^2\right)} & \frac{2 \left(2 m^2+M^2\right)}{m^2 M^2 \left(4 m^2+M^2\right)} \\
    -\frac{2 m^2}{M^2 \left(4 m^2+M^2\right)} & -\frac{2 m^2}{M^2 \left(4 m^2+M^2\right)^2} & \frac{2 \left(m^2+M^2\right)}{m^2 M^2 \left(4 m^2+M^2\right)^2} \\
    \frac{2 \left(2 m^2+M^2\right)}{M^2 \left(4 m^2+M^2\right)} & \frac{2 \left(m^2+M^2\right)}{M^2 \left(4 m^2+M^2\right)^2} & \frac{2 \left(m^2+M^2\right)}{m^2 M^2 \left(4 m^2+M^2\right)^2} \\
  \end{pmatrix} \, .
\end{align}

Now we treat $\hat I^d_{2,1}$ as an undetermined parameter to be constrained by Eq.~\eqref{eq:tildeMpos}. Again we look at the example numerical point as in Eq.~\eqref{eq:bubbleEucNumPoint},
\begin{equation}
  M^2 = 2, \quad m^2 = 1 \, ,
\end{equation}
and plot the three eigenvalues of the $3\times 3$ matrix $\widetilde {\mathbb M} = \mathbb M_1 + \hat I^d_{2,1} \mathbb M_2$ in Fig.~\ref{fig:momSpaceThreeEigs}.
%
\begin{figure}
  \centering
  \includegraphics[width=0.5\textwidth]{figs/momSpaceThreeEigs.png}
  \caption{Three eigenvalues of $\widetilde {\mathbb M}$ defined in Eq.~\eqref{eq:tildeMpos} as a function of $\hat I^d_{2,1}$, at the kinematic point Eq.~\eqref{eq:bubbleEucNumPoint}. The lowest eigenvalue corresponds to the bottom orange curve and the remaining two eigenvalues correspond to the upper black curves.}
  \label{fig:momSpaceThreeEigs}
\end{figure}
%
It can be seen in the figure that most of the parameter range shown is ruled out due to the presence of a negative eigenvalue indicated by the lowest orange curve. In Fig.~\ref{fig:momSpaceThreeEigs1}, we zoom in to a smaller parameter region and only plot the smallest eigenvalue, since the matrix is positive semidefinite as long as the smallest eigenvalue is non-negative.
%
\begin{figure}
  \centering
  \includegraphics[width=0.5\textwidth]{figs/momSpaceThreeEigs1.png}
  \caption{Magnified version of the vicinity of a small region of Fig.~\ref{fig:momSpaceThreeEigs} in which the lowest eigenvalue of $\widetilde {\mathbb M}$, shown in the curve, is non-negative.}
  \label{fig:momSpaceThreeEigs1}
\end{figure}
%
The allowed parameter region shown in the plot is
\begin{equation}
  0.630 \leq \hat I^{d=4}_{2,1} \leq 0.847 \, ,
\end{equation}
which provides a more stringent lower bound than the previous result Eq.~\eqref{eq:bubbleCrudeResultNum}. In fact, far better lower and upper bounds can be achieved in this approach by using an ansatz larger than the one in Eq.~\eqref{eq:posAnsatzSize3}. For example, let us replace the squared term in Eq.~\eqref{eq:posAnsatzSize3} by an arbitrary degree-3 polynomial in $1/\rho_1$ and $1/\rho_2$, parametrized by 10 undetermined free coefficients. We obtain the constraint,
\begin{equation}
  0 <\int \frac {d^d \bm l \, e^{\gamma_E \epsilon}}{i \pi^{d/2}} \frac{1} {\rho_1^2 \rho_2}
  \left(\alpha_1 + \frac{\alpha_2}{\rho_1} + \frac{\alpha_3}{\rho_2}
  + \frac{\alpha_4}{\rho_1^2} + \frac{\alpha_5}{\rho_1 \rho_2} + \frac{\alpha_6}{\rho_2^2}
  + \frac{\alpha_7}{\rho_1^3} + \frac{\alpha_8}{\rho_1^2 \rho_2} + \frac{\alpha_9}{\rho_1 \rho_2^2} + \frac{\alpha_{10}}{\rho_2^3}
  \right)^2 \, . \label{eq:posAnsatzSize10}
\end{equation}
Repeating the above analysis, we obtain a constraint for $\hat I^d_{2,1}$ similar to Eq.~\eqref{eq:tildeMpos}, except that the matrices involved have $10 \times 10$ sizes. The ten eigenvalues as functions of $\hat I^d_{2,1}$ are plotted in Fig.~\ref{fig:momSpaceTenEigs}, as an ``upgraded'' version of Fig.~\ref{fig:momSpaceThreeEigs} with a larger ansatz size. Some of the ten eigenvalues are too close to each other on the plot to be seen individually, but only the smallest eigenvalue (represented by the lowest curve in orange color) matters as it determines whether we can satisfy the constraint that all eigenvalues are positive.
%
\begin{figure}
  \centering
  \includegraphics[width=0.5\textwidth]{figs/momSpaceTenEigs.png}
  \caption{Eigenvalues as a function $\hat I^d_{2,1}$, for the $10\times 10$ symmetric matrix that represent the quadratic dependence of the RHS of Eq.~\eqref{eq:posAnsatzSize10} on the $\alpha_i$ parameters after factoring out $I^d_{3,0}$.}
  \label{fig:momSpaceTenEigs}
\end{figure}
%
In Fig.~\eqref{fig:momSpaceTenEigs1}, we zoom in to the small allowed range for the parameter $\hat I^d_{2,1}$.
%
\begin{figure}
  \centering
  \includegraphics[width=0.5\textwidth]{figs/momSpaceTenEigs1.png}
  \caption{A magnified version of Fig.~\ref{fig:momSpaceTenEigs} around the small region of $\hat I^d_{2,1}$ where all eigenvalues are positive, showing only the smallest eigenvalue.}
  \label{fig:momSpaceTenEigs1}
\end{figure}
%
The allowed region, as shown in the plot, is
\begin{equation}
  0.7598 \leq \hat I^{d=4}_{2,1} \leq 0.7610 \, ,
\end{equation}
which tightly constrains $\hat I^{d=4}_{2,1}$ around its true value $\hat I^{d=4}_{2,1} \approx 0.7603$ from evaluating the known analytic result at $d=4, M^2 = -p^2 = 2$, with a relative error of around $10^{-3}$.

The above two-sided bounds are rigorous, but we will also explore a prescription to assign a ``central value'', or ``best estimate'', of the value of the integrals. The prescription described below, though not justified from first principles, empirically achieve a closer agreement with true values of the integrals than the rigorous bounds in the examples in this paper. The prescription is simply finding the value of $\hat I^d_{2,1}$ which maximizes the smallest eigenvalue of the matrix that is required to be positive semidefinite, e.g.\ the matrix $\widetilde M$ of Eq.~\eqref{eq:M1plusM2}. For the positivity constraint Eq.~\eqref{eq:posAnsatzSize10} with 10 free parameters, the prescription picks the value of $\hat I^d_{2,1}$ corresponding to the maximum of the curve in Fig.~\ref{fig:momSpaceTenEigs1}, which deviates from the exact result, again at the example point $d=4, m=1, M^2=-p^2=2$, by only a relative error of about $10^{-6}$. In this case, the prescription happens to produce a value that is very close to the exact result, but typically we observe the prescription to give a ``central value'' that is one to two orders of magnitude better than the accuracy indicated by rigorous bounds.

\subsubsection{High-precision evaluation using semidefinite programming}
\label{subsubsec:sdp}
We have formulated positivity constraints in terms of eigenvalues of a matrix which, in the toy example of the one-loop bubble integrals, depends linearly on only one undetermined parameter, as shown in Eq.~\eqref{eq:tildeMpos}. For more complicated Feynman integrals, there will be more than one master integrals to be evaluated and all of them will be considered as undetermined parameters. So a search in higher-dimensional space is needed to locate the region in which all eigenvalues of the matrix are non-negative, and this can become computationally challenging. Fortunately, very efficient algorithms exist to solve \emph{semidefinite programming problems} in mathematical optimization \cite{vandenberghe1996semidefinite}. Loosely speaking, semidefinite programs are generalizations of linear programs allowing not only linear constraints but also positive semidefiniteness constraints on matrices that have linear dependence on the optimization variables. Here we show how our problem of constraining unknown master integrals can be stated as semidefinite programming problems. Following the treatment of Section \ref{subsubsec:eig} above, finding the minimum allowed value of $\hat I^d_{2,1}$ can be formulated as
\begin{equation}
\begin{aligned}
\text{minimize } \quad & \hat I^d_{2,1} \, , \\
\text{subject to }\quad & \mathbb M_1 + \hat I^d_{2,1} \cdot \mathbb M_2 \succcurlyeq 0 \, ,
\end{aligned}
\label{eq:sdpMin}
\end{equation}
which is in the form of a semidefinite program. We used the $\succcurlyeq 0$ notation, already introduced in Eq.~\eqref{eq:posGeq}, to indicate that the matrix on the LHS must be positive semidefinite.
Similarly, finding the minimum allowed value of $\hat I^d_{2,1}$ can be formulated as
\begin{equation}
\begin{aligned}
\text{maximize } \quad & \hat I^d_{2,1} \, , \\
\text{subject to }\quad & \mathbb M_1 + \hat I^d_{2,1} \cdot \mathbb M_2 \succcurlyeq 0 \, .
\end{aligned}
\label{eq:sdpMax}
\end{equation}
Finally, to implement our prescription of maximizing the smallest eigenvalue to find the ``central value'' of the undetermined master integrals, we introduce an additional undetermined parameter $\lambda$ and formulate the problem as
\begin{equation}
\begin{aligned}
\text{maximize } \quad & \lambda \, , \\
\text{subject to }\quad & \mathbb M_1 + \hat I^d_{2,1}\cdot \mathbb M_2 - \lambda \mathbb I \succcurlyeq 0 \, .
\end{aligned}
\label{eq:sdpCentral}
\end{equation}
This is again in the form of a semidefinite program, where both $\hat I^d_{2,1}$ and $\lambda$ are undetermined parameters whose values will be fixed to satisfy the optimization objective, namely to maximize $\lambda$. Note that in this case, finding a optimal solution to the semidefinite program does not guarantee $\mathbb M_1 + \hat I^d_{2,1} \cdot \mathbb M_2 \succcurlyeq 0$, \emph{unless} the value of $\lambda$ in the solution is non-negative. The value of $\hat I^d_{2,1}$ in the solution is then taken as the central value for this undetermined free parameter.

There exist many computer codes that specialize in solving semidefinite programs. Wolfram Mathematica has supported semidefinite programming since version 12 with the {\tt SemidefiniteOptimization} function, and the default backend (which can be changed by the {\tt Method} option) is the open source library CSDP \cite{borchers1999csdp} at least for the problems we deal with, working with double-precision floating numbers, i.e.\ the standard machine precision on current hardware. The SDPA family \cite{yamashita2012latest} of computer programs support computation at double precision as well as a variety of extended precisions, e.g.\ double-double precision with SDPA-DD, quadruple-double precision with SDPA-QD, and arbitrary precision with SDPA-GMP. The SDPB solver by Simmons-Duffin \cite{Simmons-Duffin:2015qma} specializes in polynomial programming problems in the conformal bootstrap and works in arbitrary precision. Most of the work in this paper will make use of the SDPA family, while some results from Mathematica / CSDP will also be shown for the purpose of comparison.

We go on to discuss how to achieve higher numerical precision for the one-loop bubble integral. We enlarge the ansatz for positive integrals in Eqs.~\eqref{eq:posAnsatzSize3} and \eqref{eq:posAnsatzSize10} to have more parameters, as
\begin{equation}
  0 < \int \frac {d^d \bm l \, e^{\gamma_E \epsilon}}{i \pi^{d/2}} \frac{1} {\rho_1^2 \rho_2}
  P(1/\rho_1, 1/\rho_2)^2 \, , \label{eq:posAnsatzSizeGeneral}
\end{equation}
where P is an arbitrary polynomial with a maximum degree $N$, i.e.\ an arbitrary linear sum of all monomials in $1/\rho_1$ and $1/\rho_2$, with each monomial multiplied by a free parameter $\alpha_i$. The $N=1$ and $N=3$ cases are shown previously in Eqs.~\eqref{eq:posAnsatzSize3} and \eqref{eq:posAnsatzSize10}, respectively.  Generally, the number of possible monomials in two variables with maximum degree $N$ is equal to $(N+1)(N+2)/2$. For each value of $N$ from 1 to 10, we again set $\hat I^d_{2,1} = I_{3,0} / I_{2,1}$ at $d=4, m=1, M^2=-p^2=2$ and solve the semidefinite programs in Eqs.~\eqref{eq:sdpMin} to \eqref{eq:sdpCentral} to obtain the lower bound $(\hat I^d_{2,1})_{\rm min}$, upper bound $(\hat I^d_{2,1})_{\rm max}$, and central value $(\hat I^d_{2,1})_{\rm central}$ for the undetermined parameter $\hat I^d_{2,1}$. To ensure numerical stability, we use the SDPA-DD solver working at double-double precision. Then we compare with the exact result $(\hat I^d_{2,1})_{\rm exact}$ to find the relative error in the best estimate value, defined as
\begin{equation}
  \left| \frac {(\hat I^d_{2,1})_{\rm central}} {(\hat I^d_{2,1})_{\rm exact}} - 1 \right|
  \label{eq:defErrorCentral}
\end{equation}
as well as the relative error of the rigorous bounds, defined as
\begin{equation}
  \left| \frac{ (\hat I^d_{2,1})_{\rm max} - (\hat I^d_{2,1})_{\rm min} }{ 2 (\hat I^d_{2,1})_{\rm exact}} \right|
  \label{eq:defErrorRigorous}
\end{equation}
In Fig.~\ref{fig:error_bounds_central}, we plot the relative errors against the cutoff degree $N$ of the polynomial $P$ in Eq.~\eqref{eq:posAnsatzSizeGeneral}.
%
\begin{figure}
  \centering
  \includegraphics[width=0.8\textwidth]{figs/error_bounds_central.png}
  \caption{Relative errors in numerical results for $\hat I^{d=4}_{2,1}$ at the kinematic point Eq.~\eqref{eq:bubbleEucNumPoint} from solving positivity constraints Eq.~\eqref{eq:posAnsatzSizeGeneral} using semidefinite programming. The horizontal axis is the cutoff degree of the polynomial $P$. The relative errors are defined by Eq.~\eqref{eq:defErrorRigorous} for the rigorous bounds and Eq.~\eqref{eq:defErrorCentral} for the central values.}
  \label{fig:error_bounds_central}
\end{figure}
%
The plot is on a log scale, and we can see that the numerical results appear to converge exponentially to the true value, reaching a precision as high as $10^{-14}$ with a cutoff degree of 10. The central value from our somewhat arbitrary prescription is seen to be consistently more precise than the rigorous bounds.

We revisit the issue of numerical stability. In Fig.~\ref{fig:error_CSDP_SDPA_DD}, we compare the accuracy obtained for the central values obtained by SDPA-DD with double-double precision, which was used above, and Mathematica / CSDP with double precision. The two results visibly deviate from each other once the cutoff degree is 5 or above, and only the double-double precision computation continues to exhibit exponential reduction in errors as the cutoff degree is increased. This signals that numerical instability has occurred if one computes at double precision only. We have checked that the accuracies do not improve further when computing with quadruple-double precision using SDPA-QD.
%
\begin{figure}
  \centering
  \includegraphics[width=0.8\textwidth]{figs/error_CSDP_SDPA_DD.png}
  \caption{Relative errors in numerical results for the central values of $\hat I^{d=4}_{2,1}$ at the kinematic point Eq.~\eqref{eq:bubbleEucNumPoint}, obtained with semidefinite programming solvers working at two different numerical precisions, namely Mathematica / CSDP working at double precision and SDPA-DD working at double-double precision. As the plot shows, double-double precision is needed for the relative errors to improve exponentially beyond a cutoff degree of 5.}
  \label{fig:error_CSDP_SDPA_DD}
\end{figure}
%

\subsection{Positivity constraints in Feynman parameter space}
\label{subsec:posFeynSpace}
In Section \ref{subsec:posMomSpace}, we used positivity constraints in loop momentum space to evaluate bubble integrals defined in Eq.~\eqref{eq:selfEnergyInt} in the case $p^2 < 0$ when it is possible to Wick-rotate the integrals into Euclidean spacetime with real-valued external momenta. We now use positivity constrains in Feynman parameter space instead to evaluate bubble integrals for any value of $p^2$ less than $4m^2$, which is what is commonly referred to as the ``Euclidean region'' where the bubble integrals have no imaginary parts.
We write down the Feynman parameter representation of bubble integrals defined in Eq.~\eqref{eq:selfEnergyInt},
\begin{align}
  I_{a_1,a_2}^d & \equiv \int \frac {d^d l \, e^{\gamma_E \epsilon}}{i \pi^{d/2}} \frac{1} {(- l^2 + m^2)^{a_1} [-(p + l)^2 + m^2]^{a_2}} \nonumber \\
  &= \frac {\Gamma(a_1 + a_2 - d / 2) e^{\gamma_E \epsilon}}
  {\Gamma(a_1) \Gamma(a_2)} \int_0^\infty dx_1 \int_0^\infty dx_2 \, \delta (1-x_1-x_2) \nonumber \\
  & \quad \times x_1^{a_1 - 1} x_2^{a_2 - 1}
  \frac {\mathcal U(x_1, x_2)^{a_1 + a_2 -d}} {\mathcal F(x_1, x_2)^{a_1 + a_2 - d/2}} \label{eq:bubbleFeynParam0} \\
  &= \frac {\Gamma(a_1 + a_2 - d / 2) e^{\gamma_E \epsilon}}
  {\Gamma(a_1) \Gamma(a_2)} \int_0^1 dx \, x^{a_1 - 1} (1-x)^{a_2 - 1}
  \frac 1 {\mathcal F(x)^{a_1 + a_2 - d/2}} \, ,
  \label{eq:bubbleFeynParam}
\end{align}
where the graph polynomials were already given in Eq.~\eqref{eq:bubbleFpoly0} for the Schwinger parametrization, printed again here:
\begin{equation}
  \mathcal U(x_1, x_2) = x_1+x_2, \quad \mathcal F(x_1, x_2) = m^2(x_1+x_2)^2 - p^2 x_1 x_2 - i 0^+ \, .
  \label{eq:bubbleFpoly0}
\end{equation}
In the last line of Eq.~\eqref{eq:bubbleFeynParam}, we integrated $x_2$ over the delta function in Eq.~\eqref{eq:bubbleFeynParam0} to arrive at Eq.~\eqref{eq:bubbleFeynParam} with
\begin{equation}
  \mathcal F(x) \equiv \mathcal F(x, 1-x) = m^2 - p^2 x(1-x) - i 0^+ \, ,
  \label{eq:bubbleFpoly}
\end{equation}
and
\begin{equation}
  \mathcal U(x) \equiv \mathcal U(x, 1-x) \equiv 1 \, ,
\end{equation}
which appears as a unit numerator in Eq.~\eqref{eq:bubbleFeynParam}.

\emph{Aside:} we note that Eq.~\eqref{eq:bubbleFeynParam0} has the property that when ignoring the Dirac delta function $\delta(1-\sum_i x_i)$, the rest of the expression (with the integration measure taken into account) is invariant under the rescaling
\begin{equation}
  x_i \to \lambda x_i,
  \label{eq:projective}
\end{equation}
where $\lambda$ is an arbitrary nonzero real number. This is called \emph{projective invariance} and holds for the Feynman parameter form of arbitrary Feynman integrals written down in Eq.~\eqref{eq:generalFeynParam}. The deeper reason is that Feynman parameter integrals can generally be written as integrals in real projective space $\mathbb R \mathbb P^{N-1}$ (see e.g.\ Section 2.5.3 of Ref.~\cite{Weinzierl:2022eaz}). $\mathbb R \mathbb P^{N-1}$ is the space of $N$ real coordinates $x_i$, excluding the origin, where any ray, i.e.\ a set of points related to each other by a rescaling Eq.~\eqref{eq:projective}, is identified as the same point.
Abusing the language of gauge theory, Eq.~\eqref{eq:projective} is a gauge symmetry and $\delta(1- \sum_i x_i)$ in Eq.~\eqref{eq:bubbleFeynParam0} is a gauge-fixing term that restricts the integration to one of the infinitely many gauge-equivalent slices. The Fadeev-Popov Jacobian associated with this gauge-fixing term is unity and therefore does not appear explicitly. Projective invariance is not a prerequisite for following the rest of the paper, though it helps motivate some of the developments.

Now we specialize to the following kinematic region for bubble integrals,
\begin{equation}
  0 < p^2 < 4m^2,
\end{equation}
i.e. with the value of $p^2$ below the Cutkosky cut threshold but cannot be trivially Wick-rotated into Euclidean spacetime. We have
\begin{equation}
  m^2 - p^2 x (1-x) \geq m^2 - p^2/4 > 0,
\end{equation}
so the $-i0^+$ prescription in Eq.~\eqref{eq:bubbleFpoly} is negligible and can be dropped, and the integral is real.
For the rest of the paper, we will adopt the common terminology of the Euclidean region to be the kinematic region in which all graph polynomials are non-negative and the Feynman integral is real-valued due to the lack of Cutkosky cuts. Generally such integrals cannot be embedded into Euclidean spacetime. This is e.g.\ the working definition when the literature refers to the Euclidean region of Feynman integrals with massless external legs, since nonzero massless momenta cannot be literally embedded into Euclidean spacetime.


If we set $a_2 = 1$, we can invert Eq.~\eqref{eq:bubbleFeynParam} to obtain
\begin{align}
  &\quad \int_0^1 dx \, x^{a_1-1} \frac 1 {\mathcal F(x)^{1 + a_1 - d/2}} \nonumber \\
  &=
  \int_0^1 dx \, x^{a_1-1} \frac 1 {\left[ m^2 - p^2 x (1-x) \right]^{1 + a_1 - d/2}} \nonumber \\
  &= \frac{\Gamma(a_1)} {e^{\gamma_E \epsilon} \, \Gamma(a_1 + 1  - d/2)} I^d_{a_1, 1} \, ,
  \label{eq:bubbleFeynParamInvert1}
\end{align}

It will be more useful to have a version of the above equation with a fixed exponent for $\mathcal F(x)$ on the LHS, even when the value of $a_1$ changes. Below is a version with a fixed exponent $d/2 - 3$ for $\mathcal F(x)$, obtained by replacing $d \rightarrow d + 2(a_1-2)$ in Eq.~\eqref{eq:bubbleFeynParamInvert1} and multiplying by a constant prefactor,
\begin{align}
  & \quad 2(m^2)^{3-d/2} \int_0^1 dx \, x^{a_1-1} \frac 1 {\mathcal F(x)^{3-d/2}} \nonumber \\
  &=
  2 \int_0^1 dx \, x^{a_1-1} \frac 1 {\left[ 1 - p^2 x (1-x) / m^2 \right]^{3-d/2}} \nonumber \\
  &= \frac {2 \Gamma(a_1) (m^2)^{1+\epsilon}} {e^{\gamma_E \epsilon} \, \Gamma(3  - d/2)} I^{d+2a_1-4}_{a_1, 1} \nonumber \\
  &= I^{d+2a_1-4}_{a_1, 1} / I^d_{3, 0}
  \, , \label{eq:bubbleFeynParamInvert0}
\end{align}
where the last line used the explicit result for $I^d_{3,0}$ in Eq.~\eqref{eq:tadpoleAnalytic}. We define
\begin{equation}
  \hat F(x) = \mathcal F(x) / m^2 = 1 - p^2 x (1-x) / m^2 \, , \label{eq:bubbleHatFDef}
\end{equation}
and rewrite Eq.~\eqref{eq:bubbleFeynParamInvert0} as
\begin{equation}
  2 \int_0^1 dx \, x^{a_1-1} \frac 1 {\hat F(x)^{3-d/2}} = I^{d+2a_1-4}_{a_1, 1} / I^d_{3, 0} \, .
  \label{eq:bubbleFeynParamInvert}
\end{equation}
The RHS of Eq.~\eqref{eq:bubbleFeynParamInvert} can be simplified further, as IBP identities and dimension-shifting identities can be applied to reduce $I^{d+2a_1-4}_{a_1, 1}$ to a linear combination of the two master integrals, $I^d_{2,1}$ and $I^d_{3,0}$.
We will only use Eq.~\eqref{eq:bubbleFeynParamInvert} in the case $a_1 \geq 2$, when the RHS involves a bubble integral in spacetime dimension greater than or equal to $d= 4 - 2\epsilon$. As an example, consider the case $a_1=3$, and Eq.~\eqref{eq:bubbleFeynParamInvert} becomes
\begin{equation}
  2 \int_0^1 dx \, x^2 \frac 1 {\hat F(x)^{3-d/2}} = I^{d+2}_{3,1} / I^d_{3, 0} \, .
\end{equation}
Then we simplify the expression using the IBP reduction result Eq.~\eqref{eq:I31result} with $d$ replaced by $d+2$, obtaining
\begin{equation}
  2 \int_0^1 dx \, x^2 \frac 1 {\hat F(x)^{3-d/2}} = \frac{1}{2p^2 (4m^2-p^2)} \Big( [(2-d)p^2 + 4m^2] I^{d+2}_{2,1} + 2(p^2-2m^2) I^{d+2}_{3,0} \Big) / I^d_{3,0} \, .
\end{equation}
Finally, applying dimension-shifting identities Eqs.~\eqref{eq:dimshift1prime} and \eqref{eq:dimshift2prime}, the above equation becomes
\begin{align}
  2 \int_0^1 dx \, x^2 \frac 1 {\hat F(x)^{3 - d/2}} &= \left( \frac {(d-2)p^2 - 4m^2} {4(d-3)p^2} I^d_{2,1} + \frac{m^2}{(d-3) p^2} I^d_{3,0} \right) / I^d_{3,0} \nonumber \\
  &= \frac {(d-2)p^2 - 4m^2} {4(d-3)p^2} \hat I^d_{2,1} + \frac{m^2}{(d-3) p^2} \, .
  \label{eq:bubbleFeynParamInvertExample}
\end{align}
This concludes our example for simplifying the RHS of Eq.~\eqref{eq:bubbleFeynParamInvert} in the case of $a_1=3$.
We now formulate a first version of positivity constraints for any $d<6$, to be improved upon later, as,
\begin{equation}
  0 \leq 2 \int_0^1 dx \, x P(x)^2 \frac 1 {\hat F(x)^{3 - d/2}} \, ,
  \label{eq:posAnsatzFeynParam}
\end{equation}
where $P(x)$ is an arbitrary polynomial in $x$, which is analogous to the arbitrary polynomial $P(1/\rho_1, 1/\rho_2)$ in Eq.~\eqref{eq:posAnsatzSizeGeneral} used to construct positive integrals in momentum space. In the special case $P(x)=1$, the RHS of Eq.~\eqref{eq:posAnsatzFeynParam} is simply proportional to $I^d_{2,1}$ in unshifted spacetime dimension $d$, according to Eq.~\eqref{eq:bubbleFeynParamInvert}. Since $x$ is non-negative in the range of integration $0 \leq x \leq 1$, $x P(x)^2$ is non-negative.
We have chosen to use $x P(x)^2$ instead of just $P(x)^2$ to ensure that each monomial in the expanded expression contains at least one power of $x$ and is related to an ultraviolet convergent integral in Eq.~\eqref{eq:bubbleFeynParamInvert} as discussed above. Another valid choice is $(1-x) P(x)^2$, but this will not give more constraints for the bubble integral, because in Eq.~\eqref{eq:bubbleFeynParam}, $\mathcal F(x) = p^2 - m^2 x(1-x)$ is invariant under the exchange $x \leftrightarrow 1-x$, owing to a reflection symmetry of the bubble diagram.

It is possible to slightly refine the inequality Eq.~\eqref{eq:posAnsatzFeynParam} to make the constraint stronger. As we assume $p^2>0$, we have
\begin{equation}
  \mathcal F(x) = m^2 - p^2 x(1-x) < m^2, \quad
  \hat F(x) = \mathcal F(x) / m^2 < 1
  \, .
\end{equation}
So for any $d<6$, we can modify Eq.~\eqref{eq:posAnsatzFeynParam} with an extra term,
\begin{equation}
  0 \leq 2 \int_0^1 dx \, x P(x)^2 \left( \frac 1 {\hat F(x)^{3-d/2}} - 1 \right) \, . \label{eq:bubbleFeynRefinedPositivity}
\end{equation}
Eq.~\eqref{eq:bubbleFeynRefinedPositivity} can also be written in a form that manifests the projective invariance discussed around Eq.~\eqref{eq:projective},
\begin{align}
  0 &\leq 2 \int_0^\infty dx_1 \int_0^\infty dx_2 \, \delta(1-x_1-x_2) \nonumber \\
  &\quad \times \frac {x_1}{U(x_1, x_2)}  P\left( \frac {x_1}{\mathcal U(x_1, x_2)} \right)^2 \left( \frac {\mathcal U(x_1, x_2)^{3-d}} {\hat F(x)^{3-d/2}} - \frac 1 {\mathcal U(x_1, x_2)^3} \right) \, ,
\end{align}
but we will use the form Eq.~\eqref{eq:bubbleFeynRefinedPositivity} below.

To be more concrete, in Eq.~\eqref{eq:bubbleFeynRefinedPositivity}, we use
\begin{equation}
  P(x) = \alpha_1 + \alpha_2 x^2 + \dots + \alpha_N x^N \, , \label{eq:PxParametrization}
\end{equation}
where $N$ is the cutoff degree of the polynomial and $\alpha_i$ with $1 \leq i \leq N$ are free parameters, and Eq.~\eqref{eq:bubbleFeynRefinedPositivity} must hold for any values of the $\alpha_i$ parameters. For each monomial from expanding $x P(x)^2$, the first term in the curly bracket of Eq.~\eqref{eq:bubbleFeynRefinedPositivity} gives a bubble integral in a shifted dimension, normalized against the tadpole integral $I^d_{3,0}$, according to the formula Eq.~\eqref{eq:bubbleFeynParamInvert}, while the second term in the curly bracket of Eq.~\eqref{eq:bubbleFeynRefinedPositivity} contributes to an integral of a monomial in $x$ over $0\leq x \leq 1$ which can be evaluated trivially. Using both dimension shifting identities and IBP identities, the bubble integrals produced above are rewritten as linear combinations of finite master integrals Eq.~\eqref{eq:bubbleFiniteBasis}. Therefore Eq.~\eqref{eq:bubbleFeynRefinedPositivity} is turned into the form
\begin{equation}
  {\vec \alpha}^T \, \mathbb M \, \vec \alpha \geq 0 \, , \label{eq:aMa_is_positive}
\end{equation}
similar to the momentum-space version Eq.~\eqref{eq:posAnsatzSize3}, with
\begin{equation}
  \mathbb M = \mathbb M_1 + \hat I^d_{2,1} \mathbb M_2 \, ,
\end{equation}
where we use the definition $\hat I^d_{2,1} = I^d_{2,1} / I^d_{3,0}$ as before, and the ``inhomogeneous'' term $\mathbb M_1$ receives contribution from both constant terms in Eq.~\eqref{eq:bubbleFeynRefinedPositivity} and tadpole integrals coming from dimension-shifting and IBP identities. Following analogous developments in Sections \ref{subsubsec:eig} and \ref{subsubsec:sdp}, we solve the constraint
\begin{equation}
  \mathbb M_1 + \hat I^d_{2,1} \mathbb M_2 \succcurlyeq 0
\end{equation}
while minimizing or maximizing $\hat I^d_{2,1}$ to find rigorous bounds for $\hat I^d_{2,1}$, or alternatively maximizing the smallest eigenvalue of $\mathbb M_1 + \hat I^d_{2,1} \mathbb M_2$ to find a central value for $\hat I^d_{2,1}$ using the same prescription as described before.

As an example, we pick numerical values
\begin{equation}
  p^2 = 2, m=1 \, ,
  \label{eq:bubbleNumPoint}
\end{equation}
for bubble integrals in $d=4$, and compare our numerical results against the exact result. As $p^2$ is positive, though below the Cutkosky cut threshold $4m^2$, the Euclidean momentum space treatment in Section \ref{subsec:posMomSpace} is not applicable. SDPA-QD working at quadruple-double precision is used to compute central values and SDPA-GMP working at 8 times the double precision is used to compute rigorous bounds.
In Fig.~\ref{fig:error_bounds_central_feyn}, we plot the relative error of the central value for $\hat I^d_{2,1}$ as well as the relative error of the rigorous bounds for $\hat I^d_{2,1}$.
%
\begin{figure}
  \centering
  \includegraphics[width=0.8\textwidth]{figs/error_bounds_central_feyn.png}
  \caption{Relative errors in numerical results for $\hat I^{d=4}_{2,1}$ at the kinematic point Eq.~\eqref{eq:bubbleNumPoint} from solving positivity constraints Eq.~\eqref{eq:bubbleFeynRefinedPositivity} in Feynman-parameter space using semidefinite programming. The horizontal axis is the cutoff degree of the polynomial $P$. The relative errors are defined by Eq.~\eqref{eq:defErrorRigorous} for the rigorous bounds and Eq.~\eqref{eq:defErrorCentral} for the central values.}
  \label{fig:error_bounds_central_feyn}
\end{figure}
%
We can see on the log-scale plot that the numerical result again converge exponentially to the exact result as the cutoff degree is increased. In particular, with cutoff degree $N=14$, the numerical result is
\begin{equation}
  \hat I_{2,1}^{d=4} \Big|_{p^2=2, \, m=1} \approx 1.57079632679413 \, , \label{eq:bubNumericalDeg14}
\end{equation}
which is slightly smaller than the exact result, with a relative error of $4.9 \times 10^{-13}$.
\subsection{Constraints for expansions in dimensional regularization parameter}
\label{subsec:epExpansion}
The methods described in Sections \ref{subsec:posMomSpace} and \ref{subsec:posFeynSpace} are applicable to fixed spacetime dimensions, i.e.\ $d=4-2\epsilon$ with fixed values of $\epsilon$, which can be 0 if we target the 4-dimensional case, or any value larger than $(-1)$ which will preserve the ultraviolet convergence properties of the integrals involved in the text above. However, for practical applications, Feynman integrals typically need to be evaluated as a Laurent expansion in $\epsilon$. In the examples given in this paper, we can choose master integrals which are finite as $\epsilon \to 0$, so the task is to calculation their Taylor expansions in $\epsilon$. Any divergent integral can be reduced to rational-linear combinations of the master integrals, with all divergences absorbed into $\epsilon$ poles of the coefficients.\footnote{In fact, it is believed that in general, one can choose ``quasi-finite'' master integrals \cite{vonManteuffel:2014qoa} which are convergent as $\epsilon \to 0$ except for a possible $1/\epsilon$ pole that appears as an overall prefactor in the Feynman parameter representation.}

We now two strategies for calculating the $\epsilon$ expansion, using the one-loop bubble integral example. The first strategy presented below is directly formulating positivity constraints for the $\epsilon$ expansion terms, and the second strategy presented is numerical differentiation of the results with respect to $\epsilon$ around $\epsilon = 0$.

\subsubsection{Generic constraints}
\label{subsubsec:eps_generic}
We will take a break from the bubble integrals and write down the general form of the Feynman parametrization for an $L$-loop integral with $n$ propagators,
\begin{align}
  &\quad I^d_{a_1,a_2, \dots , a_n} \equiv \left( \prod_{i=1}^L \int \frac {d^d l_i \, e^{\gamma_E \epsilon}}{i \pi^{d/2}}  \right)
  \frac{1} {\rho_1^{a_1} \rho_2^{a_2} \dots \rho_n^{a_n}} \nonumber \\
  &= \frac {\Gamma(a - L d / 2) e^{\gamma_E \epsilon}}
       {\Gamma(a_1) \Gamma(a_2) \dots \Gamma(a_n)}
  \int_{x_i \geq 0} d^n x_i \, \delta \left( 1 - \sum x_i \right) \, \left( \prod_i x_i^{a_i - 1} \right)
  \frac{\mathcal U(x_i)^{a - (L+1)d/2}} {\mathcal F(x_i)^{a - Ld/2}} \, ,
  \label{eq:generalFeynParam}
\end{align}
where $a \equiv \sum a_i$, and $\mathcal U$ and $\mathcal F$ are graph polynomials. In the Euclidean region, i.e.\ when external kinematics do not allow any Cutkosky cuts, $\mathcal U$ and $\mathcal F$ are non-negative in the range of integration. To make it easier to formulate positivity constraints, we adjust constant prefactors and slightly rewrite Eq.~\eqref{eq:generalFeynParam} as
\begin{align}
  &\quad \tilde I^d_{a_1,a_2, \dots , a_n} \equiv \frac {\Gamma(a_1) \Gamma(a_2) \dots \Gamma(a_n)} {\Gamma(a - L d / 2) e^{\gamma_E \epsilon}} I^d_{a_1,a_2, \dots , a_n} \nonumber \\
  &= \frac {\Gamma(a_1) \Gamma(a_2) \dots \Gamma(a_n)} {\Gamma(a - L d / 2) e^{\gamma_E \epsilon}} 
  \left( \prod_{i=1}^L \int \frac {d^d l_i \, e^{\gamma_E \epsilon}}{i \pi^{d/2}}  \right)
  \frac{1} {\rho_1^{a_1} \rho_2^{a_2} \dots \rho_n^{a_n}} \nonumber \\
  &= 
  \int_{x_i \geq 0} d^n x_i \, \delta \left( 1 - \sum x_i \right) \, \left( \prod_i x_i^{a_i - 1} \right)
  \frac{\mathcal U(x_i)^{a - (L+1)d/2}} {\mathcal F(x_i)^{a - Ld/2}} \, .
  \label{eq:generalFeynParam1}
\end{align}
We will restrict our attentions to values of $d$ and $a_i$ under which the RHS of Eq.~\eqref{eq:generalFeynParam1} is convergent. Since there are otherwise no restrictions on $a_i$, generally the integrals are not master integrals which are usually chosen to have small values of $a_i$.
We set
\begin{equation}
  d = d_0 - 2 \epsilon \, ,
\end{equation}
where $d_0$ is usually an integer spacetime dimension such as 4. The Taylor expansion of the LHS of Eq.~\eqref{eq:generalFeynParam1} is written as
\begin{equation}
  \tilde I_{a_1,a_2, \dots , a_n} = \tilde I_{a_1,a_2, \dots , a_n} \big|_{\epsilon^0} + \epsilon \cdot \tilde I_{a_1,a_2, \dots , a_n} \big|_{\epsilon^1} +
  \epsilon^2 \cdot \tilde I_{a_1,a_2, \dots , a_n} \big|_{\epsilon^2} \, \dots
\end{equation}
The only $\epsilon$ dependence of the RHS of Eq.~\eqref{eq:generalFeynParam1} is in the exponents on the graph polynomials, so the $\mathcal O(\epsilon^k)$ term in the Taylor expansion is
\begin{equation}
  \tilde I_{a_1,a_2, \dots , a_n} \Big|_{\epsilon^k} = 
  \int_{x_i \geq 0} d^n x_i \left( 1 - \sum x_i \right) \, \left( \prod_i x_i^{a_i - 1} \right)
  \frac{\mathcal U(x_i)^{a - (L+1)d_0/2}} {\mathcal F(x_i)^{a - Ld_0/2}}
  \frac 1 {k!} \log^k \frac {\mathcal U^{L+1}} {\mathcal F^L}
  \, .
  \label{eq:epsilonExp}
\end{equation}
Note that $\mathcal U^{L+1} / \mathcal F^l$ in the equation above is a quantity that is invariant under the rescaling symmetry Eq.~\eqref{eq:projective}, since $\mathcal U$ is a homogeneous polynomial of degree $l$ and $\mathcal F$ is a homogeneous polynomial of degree $l+1$. For the Euclidean region, as $\mathcal U$ and $\mathcal F$ are positive in the range of integration, no branch-cut singularities from the logarithm are encountered. Now we write down a positivity constraint using our usual trick of constructing non-negative integrands from squares of polynomials,
\begin{equation}
  0 \leq \int_{x_i \geq 0} d^n x_i \left( 1 - \sum x_i \right) \, \left( \prod_i x_i^{a_i - 1} \right)
  \frac{\mathcal U(x_i)^{a - (L+1)d_0/2}} {\mathcal F(x_i)^{a - Ld_0/2}}
  P^2\left( \log \frac {\mathcal U^{L+1}} {\mathcal F^L} \right)
  \, ,
  \label{eq:feynPositivity}
\end{equation}
where $P$ is an arbitrary polynomial (of the argument in the bracket) under a cutoff degree $N$, as a sum of monomials each multiplied by a free parameter. After expanding the square of the polynomial, each monomial term is identified with a term in the Taylor expansion over $\epsilon$ using Eq.~\eqref{eq:epsilonExp}. Using the same manipulations as in Sections \ref{subsec:posMomSpace} and \ref{subsec:posFeynSpace}, Eq.~\eqref{eq:feynPositivity} implies that a certain symmetric matrix is positive semidefinite. We first define an auxiliary notation
\begin{equation}
  H_k \equiv (k!) \tilde I^d_{a_1,a_2, \dots , a_n} \Big|_{\epsilon^k} \, .
\end{equation}
Then we have
\begin{align}
  \begin{pmatrix}
    H_0 & H_1 & H_2 & \dots & H_N \\
    H_1 & H_2 & H_3 & \dots & H_{N+1} \\
    H_2 & H_3 & H_4 & \dots & H_{N+2} \\
    \vdots & \vdots & \vdots & \ddots & \vdots \\
    H_N & H_{N+1} & H_{N+2} & \dots & H_{2N}
  \end{pmatrix}
  \succcurlyeq 0 \, , \label{eq:epsExpHankel}
\end{align}
where we again used the notation $\succcurlyeq 0$ to indicate that a matrix is positive semidefinite. The matrix is in a special form called a Hankel matrix, where all matrix entries are defined through a sequence $H_0$, $H_1$, \dots, $H_N$. Hankel matrices have also appeared in the context of EFT positivity bounds, e.g.\ in Ref.~\cite{Arkani-Hamed:2020blm}.

Eq.~\eqref{eq:epsExpHankel} is extremely general and applies to any convergent Feynman integral (or quasi-finite Feynman integral \cite{vonManteuffel:2014qoa} after dropping an overall divergent prefactor) in the Euclidean region with arbitrary powers of propagators and no numerators.
This tells us that the $\epsilon$ expansion of such Feynman integrals are not arbitrary but are constrained by positivity constraints which, to our best knowledge, have not been previously revealed in the literature.

\subsubsection{Taylored constraints for specific Feynman integrals}
\label{subsubsec:eps_taylored}
From Eq.~\eqref{eq:epsilonExp}, it is not hard to anticipate that more specialized positivity constraints exist if we focus on a particular family of Feynman integrals if $\log (\mathcal U^{L+1} / \mathcal F^l)$ has either an upper bound or lower bound, or both, in the range of integration. For example, for one-loop bubble integrals, the $\mathcal U$ polynomial is equal to $x_1+x_2$ and is set to 1 by the Dirac delta function in Eq.~\eqref{eq:generalFeynParam}. So we recover the Feynman parametrization for the one-loop bubble integral, Eq.~\eqref{eq:bubbleFeynParam}, with $x_1 = x, \, x_2 = 1-x, \, \mathcal U(x) = 1, \, \mathcal F(x) = m^2 - p^2x(1-x)$. We have, for $0<p^2 < 4m^2$ under consideration, in the integration range $0 \leq x \leq 1$,
\begin{equation}
  \log \frac{\mathcal U^{L+1}} {\mathcal F^L} = \log \frac 1 {\mathcal F} = \log \frac{1} {m^2 - p^2x(1-x)} \leq \log \frac{1} {m^2 - p^2 / 4} \equiv \log \max \frac{\mathcal U^{L+1}} {\mathcal F^L} \, . \label{eq:bubbleFpolyBound}
\end{equation}
Therefore
\begin{equation}
  \log \max \frac{\mathcal U^{L+1}} {\mathcal F^L} - \log \frac{\mathcal U^{L+1}} {\mathcal F^L} \label{eq:logMaxMinusLog}
\end{equation}
is a positive quantity. Similarly, if a minimum of $\log \mathcal U^{L+1} / \mathcal F^L$ exists over the range of integration, then
\begin{equation}
  \log \frac{\mathcal U^{L+1}} {\mathcal F^L} - \log \min \frac{\mathcal U^{L+1}} {\mathcal F^L} \label{eq:logMinusLogMin}
\end{equation}
is a positive quantity. In the bubble integral example, again under $0<p^2 < 4m^2$ and $0 \leq x \leq 1$,
\begin{equation}
  \log \frac{\mathcal U^{L+1}} {\mathcal F^L} = \log \frac 1 {\mathcal F} = \log \frac{1} {m^2 - p^2x(1-x)} \geq \log \frac{1} {m^2} \equiv \log \min \frac{\mathcal U^{L+1}} {\mathcal F^L} \, . \label{eq:bubbleFpolyBoundAlt}
\end{equation}

Now we show an example of using Eq.~\eqref{eq:bubbleFpolyBound} to contrain the $O(\epsilon)$ term in the expansion of the finite bubble integral $I_{2,1}$, in the style of an ad hoc constraint as was done for the $\mathcal O(\epsilon^0)$ part in Section \ref{subsubsec:crude}.\footnote{It is also possible to use Eq.~\eqref{eq:bubbleFpolyBoundAlt} instead, or in combination.} Using the definition $\hat F(x) = \mathcal F(x) / m^2$ in Eq.~\eqref{eq:bubbleHatFDef}, Eq.~\eqref{eq:bubbleFpolyBound} is rewritten as
\begin{equation}
  \log \frac 1 {\hat F(x)} \leq \log \max \frac 1 {\mathcal F} = \log \frac{1} {1 - p^2 / (4 m^2)} \, ,
\end{equation}
i.e.,
\begin{equation}
  \log \frac{1} {1 - p^2 / (4 m^2)} - \log \frac 1 {\hat F(x)} \geq 0 \, ,
  \label{eq:bubbleFpolyBoundHat}
\end{equation}
We expand both the LHS and RHS of Eq.~\eqref{eq:bubbleFeynParamInvertExample}, with $d=4 - 2\epsilon$, as a Taylor series in $\epsilon$. Equating the $\epsilon^0$ terms gives
\begin{equation}
  2 \int_0^1 dx \, x^2 \frac 1 {\hat F(x)} = \frac {p^2 - 2m^2} {2p^2} \left( \hat I_{2,1} \big|_{\epsilon^0} \right) + \frac{m^2}{p^2} \, ,
  \label{eq:bubbleFeynParamInvertExampleEp0}
\end{equation}
while equating the $\epsilon^1$ terms gives
\begin{equation}
  2 \int_0^1 dx \, x^2 \frac 1 {\hat F(x)} \cdot \log \left( \frac 1 {\hat F(x)} \right) =
  \frac {p^2 - 4m^2} {2p^2} \left( \hat I_{2,1} \big|_{\epsilon^0} \right)
  + \frac {p^2 - 2m^2} {2p^2} \left( \hat I_{2,1} \big|_{\epsilon^1} \right)
  + \frac{2m^2}{p^2} \, .
  \label{eq:bubbleFeynParamInvertExampleEp1}
\end{equation}
Now we use Eq.~\eqref{eq:bubbleFpolyBoundHat} to write down the positivity constraint
\begin{equation}
  2 \int_0^1 dx \, x^2 \frac 1 {\hat F(x)} \left( \log \frac{1} {1 - p^2 / (4 m^2)} - \log \frac 1 {\hat F(x)} \right) \geq 0 \, .
\end{equation}
Then applying Eqs.~\eqref{eq:bubbleFeynParamInvertExampleEp0} and \eqref{eq:bubbleFeynParamInvertExampleEp1} leads to a constraint on $\hat I_{2,1} \big|_{\epsilon^1}$,
\begin{align}
  0 & \leq \left( \frac {p^2-2} {2p^2} \log \frac{1} {1 - p^2 / (4 m^2)} + \frac {4m^2-p^2}{2p^2} \right) \left( \hat I_{2,1} \big|_{\epsilon^0} \right) + \frac{2m^2 - p^2}{2p^2} \left( \hat I_{2,1} \big|_{\epsilon^1} \right) \nonumber \\
  & + \left( \log \frac{1} {1 - p^2 / (4 m^2)} - 2 \right) \frac {m^2} {p^2} \, .
  \label{eq:ep1_constraint_example}
\end{align}
We plot the RHS of Eq.~\eqref{eq:ep1_constraint_example} versus $p^2/m^2$ in Fig.~\ref{fig:ep1_constraint_example}. We can see that it is indeed non-negative in the range $0 < p^2/m^2 < 4$.
%
\begin{figure}
  \centering
  \includegraphics[width=0.5\textwidth]{figs/ep1_constraint_example.png}
  \caption{The RHS of Eq.~\eqref{eq:ep1_constraint_example} versus $p^2/m^2$ in the range of validity $0 < p^2 < 4$.}
  \label{fig:ep1_constraint_example}
\end{figure}
%
\subsubsection{Bubble integral up to second order in $\epsilon$}
\label{sec:bubbleResultsEps}
We proceed to combine positivity constraints for $\epsilon$ expansion coefficients covered in Sections \ref{subsubsec:eps_generic} and \ref{subsubsec:eps_taylored} with the semidefinite programming technique already used in Sections \ref{subsec:posMomSpace} and \ref{subsec:posFeynSpace}, in order to obtain high-precision results for the $\epsilon$ expansion of the bubble integral.

Recall that we evaluated the $\mathcal O(\epsilon^0)$ part, i.e.\ the $d=4$ result, for the bubble integral starting from Eq.~\eqref{eq:posAnsatzFeynParam} and the refined version Eq.~\eqref{eq:bubbleFeynRefinedPositivity}. For simplicity, we will build upon the first version, Eq.~\eqref{eq:posAnsatzFeynParam}, and use the additional positive building block Eq.~\eqref{eq:bubbleFpolyBoundHat} to write down the following positivity constraint at $d=d_0-2 \epsilon = 4-2 \epsilon$,
\begin{align}
  0 & \leq \int_0^1 dx \, x P(x)^2 \frac 1 {\hat F(x)^{3-d_0/2}}
  \left( \log \max \frac{1} {\hat F} - \log \frac{1} {\hat F(x)} \right) \\
  &=
  \int_0^1 dx \, x P(x)^2 \frac 1 {[1 - x(1-x) p^2/m^2]}
  \left( \log \frac{1} {1 - p^2/(4m^2)} - \log \frac{1} {1 - x(1-x) p^2/m^2} \right)
  \, ,
  \label{eq:posAnsatzFeynParamEp1}
\end{align}
where $P(x)$ is again an arbitrary polynomial in $x$, and we are free to choose the maximum degree of monomials that are included, depending on the accuracy we would like to attain. After expanding $P(x)^2$ into a sum of monomials, the contribution from each monomial can be evaluated following the same procedure as used in the example in Section \ref{subsubsec:eps_taylored}.
In particular, the result will be a sum of terms proportional to $\hat I_{2,1} \big|_{\epsilon^0}$, terms proportional to $\hat I_{2,1} \big|_{\epsilon^1}$, and constant terms. The remaining calculation steps are very similar to those of Section \ref{subsec:posFeynSpace}. Re-using the parametrization Eq.~\eqref{eq:PxParametrization} for the polynomial $P$ with a cutoff degree $N=14$, we again arrive at
\begin{equation}
  {\vec \alpha}^T \, \mathbb M \, \vec \alpha \geq 0 \, ,
\end{equation}
stating that an appropriate matrix $\mathbb M$ is positive semidefinite, i.e.
\begin{equation}
  \mathbb M \succcurlyeq 0 \, .
\end{equation}
In this case, $\mathbb M$ is a sum of three terms,
\begin{equation}
  \mathbb M = \mathbb M_1 + I_{2,1}\big|_{\epsilon^0} \cdot \mathbb M_2 + I_{2,1}\big|_{\epsilon^1} \cdot \mathbb M_3 \, . \label{eq:matM_sum_three_parts}
\end{equation}
where the three matrices $\mathbb M_{1,2,3}$, with rational dependence on $p^2$ and $m$, are obtained from dimension shifting, IBP, and finally $\epsilon$-expansion as in the example of Section \ref{subsubsec:eps_taylored}. We approximate $I_{2,1}\big|_{\epsilon^0}$ to be the central value Eq.~\eqref{eq:bubNumericalDeg14} obtained in the previous $d=4$ calculation with the same cutoff degree $N=14$. At this point, $I_{2,1}\big|_{\epsilon^1}$ remains the only unknown parameter on the RHS of Eq.~\eqref{eq:matM_sum_three_parts}, and its allowed range as well as the central value can be determined by semidefinite programming solvers as covered in previous sections. We again use SDPA-QD to produce the ``central value'' for the numerical result, defined by the same prescription as before, obtaining
\begin{equation}
  I_{2,1}\big|_{\epsilon^1} \approx 0.74313814320586 \, , \ \text{at } {p^2=2, m=1} \, ,
\end{equation}
which is slightly larger than the exact result with a relative error of $4.3 \times 10^{-12}$.

We continue to present how the $\mathcal O(\epsilon^2)$ term of the bubble integral is calculated. We use the positivity constraint,
\begin{equation}
  0 \leq \int_0^1 dx \, x P \left( x, \log \frac{1} {\hat F} \right)^2 \left[ m^2 - p^2 x (1-x) \right]^{d_0/2 - 3}  \, ,
  \label{eq:posAnsatzFeynParamEp2}
\end{equation}
where $P$ is a polynomial in $x$ and $\log 1 / \hat F(x)$ with maximum degrees $N_1$ and $N_2$ in the two variables, parametrized as
\begin{equation}
  P\left( x, \log \frac{1} {\hat F(x)} \right) = \sum\limits_{0\leq i_1 \leq N_1} \sum\limits_{0 \leq i_2 \leq N_2} \alpha_{i_1, i_2} \, x^{i_1} \left( \log \frac{1} {\hat F(x)} \right)^{i_2} \, .
\end{equation}
We use $N_1=14$ and $N_2 = 1$, so that there are at most 14 power of $x$ in $P(x)$ and at most one power of $\log [1 / \hat F(x)]$. After expanding the $P^2$, there are at most two powers of the aforementioned logarithm, therefore there are $\epsilon$ expansions coefficients at orders $\epsilon^0$, $\epsilon^1$, and $\epsilon^2$. We obtain an expression of the form
\begin{equation}
  {\vec \alpha}^T \, \mathbb M \, \vec \alpha \geq 0 \, ,
\end{equation}
where the column vector $\vec \alpha$ groups together all the $\alpha_{i_1, i_2}$ parameters. The matrix $\mathbb M$ is the sum of a constant term, a term proportional to $I_{2,1}\big|_{\epsilon^0}$, a term proportional to $I_{2,1}\big|_{\epsilon^1}$, and finally a term proportional to $I_{2,1}\big|_{\epsilon^2}$. We use previous numerical results for $I_{2,1}\big|_{\epsilon^0}$ and $I_{2,1}\big|_{\epsilon^1}$, and solve a semidefinite programming problem to find the central value of $I_{2,1}\big|_{\epsilon^2}$ to be
\begin{equation}
  I_{2,1}\big|_{\epsilon^2} \approx 0.208108744452 \, , \ \text{at } {p^2=2, m=1} \, ,
\end{equation}
which is slightly larger than the exact result with a relative error of $1.9 \times 10^{-11}$.

There is no obstruction to obtaining results at even higher orders in the $\epsilon$ expansion for the bubble integral example. Generally, we calculate iteratively to higher and higher orders in $\epsilon$, at each step taking previous numerical results as known input. The positivity constraint is Eq.~\eqref{eq:posAnsatzFeynParamEp2} with an appropriate cutoff degree for $\log [1 / \hat F(x)]$ depending on the desired order in the $\epsilon$ expansion. The RHS of Eq.~\eqref{eq:posAnsatzFeynParamEp2} can be optionally multiplied by either Eq.~\eqref{eq:logMaxMinusLog} or Eq.~\eqref{eq:logMinusLogMin}, with $\mathcal U=1$ and $L=1$ in the case of bubble integrals, to give more constraints. This will produce constraints for bubble integrals to any desired order in the $\epsilon$ expansion.
\subsection{$\epsilon$ expansion from numerical differentiation w.r.t.\ spacetime dimension}
\label{subsec:numericalDiff}
Here we present an alternative method for numerically evaluating the $\epsilon$-expansion of the normalized bubble integral $\hat I^d_{2,1}$ defined in Eq.~\eqref{eq:bubbleMasterParam}. Instead of formulating constraints for the $\epsilon$ expansion coefficients, we calculate the $\hat I^d_{2,1}$ at numerical values of the spacetime dimension near 4, and use finite-difference approximations to obtain derivatives w.r.t.\ $\epsilon$. The derivatives are related to the terms in the $\epsilon$ expansion via
\begin{equation}
  \hat I_{2,1} \big|_{\epsilon^k} = \frac{1} {k!} \frac{d^k} {d \epsilon^k} \hat I^{d=4-2\epsilon}_{2,1} \Big|_{\epsilon=0} \, .
\end{equation}
For an arbitrary function $f(\epsilon)$, we use 4th-order finite-difference approximations,
\begin{align}
  \frac{d} {d \epsilon} f(\epsilon) \bigg|_{\epsilon = \epsilon^0} & \approx
  \frac 1 {\Delta \epsilon} \bigg[ \frac 1 {12} f(\epsilon_0 - 2 \Delta \epsilon) - \frac 2 {3} f(\epsilon_0 - \Delta \epsilon) \nonumber \\
    & \qquad + \frac 2 {3} f(\epsilon_0 + \Delta \epsilon) - \frac 1 {12} f(\epsilon_0 + 2 \Delta \epsilon) \bigg] \, , \label{eq:finiteDiff1} \\
  \frac {d^2} {d \epsilon^2} f(\epsilon) \bigg|_{\epsilon = \epsilon^0} & \approx
  \frac 1 {\Delta \epsilon^2} \bigg[ -\frac 1 {12} f(\epsilon_0 - 2 \Delta \epsilon) + \frac 4 {3} f(\epsilon_0 - \Delta \epsilon) - \frac 5 2 f(\epsilon_0) \nonumber \\
    &\qquad + \frac 4 {3} f(\epsilon_0 + \Delta \epsilon) - \frac 1 {12} f(\epsilon_0 + 2 \Delta \epsilon) \bigg] \, , \label{eq:finiteDiff2}
\end{align}
where $\Delta \epsilon$ is the step size. As the name suggests, these formulas are exact when $f$ is a polynomial with a degree up to 4.
Note that the method of Section \ref{subsec:posFeynSpace} can be used to evaluate the normalized bubble integral $\hat I^d_{2,1}$ in any spacetime dimension $d<6$, i.e. $\epsilon > -1$, so Eqs.~\eqref{eq:finiteDiff1} and \eqref{eq:finiteDiff2} can be readily used with $\epsilon_0 = 0$ and a small $\Delta \epsilon$, chosen to be
\begin{equation}
  \Delta \epsilon = 10^{-3} \, .
\end{equation}
We again choose kinematic parameter values $p^2=2$ and $m=1$. The final results from numerical differentiation, up to the second order in $\epsilon$, are
\begin{equation}
  \hat I_{2,1} \big|_{\epsilon^1} \approx 0.7431381432049 \, ,
\end{equation}
which is slightly larger than the exact result with a relative error of $3.2 \times 10^{-12}$, and
\begin{equation}
  \hat I_{2,1} \big|_{\epsilon^2} \approx 0.20810874450 \, ,
\end{equation}
which is slightly larger than the exact result with a relative error of $2.8 \times 10^{-10}$.
