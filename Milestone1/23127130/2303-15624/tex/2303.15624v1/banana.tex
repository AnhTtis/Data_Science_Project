\subsection{Definitions and conventions}
\label{subsec:bananaDefinitions}
Here we present a three-loop example, the so called banana integrals. Banana integrals are various loop orders have received intense interest from an analytic perspective due to connections with Calabi-Yau manifold \cite{Groote:2005ay, Klemm:2019dbm, Bonisch:2020qmm, Bonisch:2021yfw, Kreimer:2022fxm, Forum:2022lpz, Pogel:2022ken}. We apply our numerical method developed in the previous section \ref{sec:bubble} on one-loop bubble integrals, with some minor adaptations, to evaluate $11$ nontrivial master integrals for the banana diagram. We assume the readers to be familiar with the previous section as many shared techniques will not be introduced again.
The diagram for the integrals is shown in Fig.~\ref{fig:banana}.
%
\begin{figure}
  \centering
  \includegraphics[width=0.3\textwidth]{figs/banana.pdf}
  \caption{Three-loop banana family of integrals, with internal and external squared masses labeled.}
  \label{fig:banana}
\end{figure}
%
Due to dimension-shifting identities reviewed in Section \ref{subsec:bubbleIBP}, the $\epsilon$ expansions of integrals in $d=4-2\epsilon$ and $d=2-2\epsilon$ can be related to each other. We will always use
\begin{equation}
  d=2-2\epsilon \, ,
\end{equation}
for three-loop banana integrals, which is convenient as the scalar integral has no ultraviolet divergence in this spacetime dimension.
The banana family of integrals is defined as
\begin{align}
  I_{a_1, a_2, a_3, a_4} &\equiv \left( \prod_{i=1}^3 \int \frac {d^d l_i \, e^{\gamma_E \epsilon}} {i \pi^{d/2}} \right)
  \frac 1 {(-l_1^2 + m_1^2)^{a_1}}
  \frac 1 {(-l_2^2 + m_2^2)^{a_2}}
  \frac 1 {(-l_3^2 + m_3^2)^{a_3}} \nonumber \\
  &\qquad \times 
  \frac 1 {[-(p+l_1+l_2+l_3)^2 + m_4^2]^{a_4}}\, , \qquad \text{with } d = 2 - 2\epsilon \, . \label{eq:banana}
\end{align}
We have suppressed the $d$ dependence on the LHS of the above equation, unlike the one-loop case Eq.~\eqref{eq:selfEnergyInt}, since we will not make use of dimension-shifting in the treatment of three-loop banana integrals and will exclusively work with $d= 2 - 2\epsilon$.
If any one of the four indices $a_i$ is non-positive in Eq.~\eqref{eq:banana}, the remaining propagators have the structure of three one-loop massive tadpole integrals. For example, if $a_4=0$, Eq.~\eqref{eq:banana} clearly factorizes into the product of three scalar tadpole integrals. For facilitating the discussion of positivity constraints, it will be convenient to define a variant of Eq.~\eqref{eq:banana} with slightly adjusted constant factors,
\begin{align}
  \hat I_{a_1, a_2, a_3, a_4} &\equiv \left( \prod_{i=1}^3 \int \frac {d^d l_i} {i \pi^{d/2}} \right) \frac {1} {\Gamma(4 - 3d/2)}
  \frac 1 {(-l_1^2 + m_1^2)^{a_1}}
  \frac 1 {(-l_2^2 + m_2^2)^{a_2}}
  \frac 1 {(-l_3^2 + m_3^2)^{a_3}} \nonumber \\
  &\qquad \times 
  \frac 1 {[-(p+l_1+l_2+l_3)^2 + m_4^2]^{a_4}} \, . \label{eq:bananaHat}
\end{align}

By IBP reduction, all integrals of the banana family, with integer values of $a_i$ in Eq.~\eqref{eq:bananaHat}, can be expressed as linear sums of 15 master integrals. Publicly available software, such as those presented Refs.~\cite{Anastasiou:2004vj, vonManteuffel:2012np, Smirnov:2008iw, Smirnov:2019qkx, Lee:2012cn, Lee:2013mka, Maierhofer:2017gsa}, can be used to give a list of master integrals as well as performing the actual IBP reduction of integrals. There are 11 nontrivial ``top-level'' master integrals that are not products of tadpole integrals, shown in three groups below according to the total number of the four indices,
\begin{equation}
  \begin{aligned}
    &\hat I_{1,1,2,2}, \, \hat I_{1,2,1,2}, \, \hat I_{1,2,2,1}, \, \hat I_{2,1,1,2}, \, \hat I_{2,1,2,1}, \, \hat I_{2,2,1,1}, \\
    &\hat I_{1,1,1,2}, \, \hat I_{1,1,2,1}, \, \hat I_{1,2,1,1}, \, \hat I_{2,1,1,1}, \\
    &\hat I_{1,1,1,1} \, .
  \end{aligned}
  \label{eq:mastersBananaTop}
\end{equation}
In addition, there are 4 master integrals that are trivial products of tadpole integrals. These master integrals are chosen as
\begin{equation}
  \hat I_{0,2,2,2}, \, \hat I_{2,0,2,2}, \, \hat I_{2,2,0,2}, \, \hat I_{2,2,2,0} \, , \label{eq:mastersTadCubed}
\end{equation}
where we raised every propagator to a 2nd power to make the integral UV finite in $d=2-2\epsilon$. Each of these four master integrals is a product of three one-loop tadpole integrals given in Eq.~\eqref{eq:tadpoleGeneral} with $n=2$, with some adjustment of the overall factor according to Eq.~\eqref{eq:bananaHat},
\begin{align}
  & \hat I_{0,2,2,2} = \frac {\Gamma^3(1+\epsilon)}{\Gamma(1+3\epsilon)} \left( \frac 1 {m_2^2 m_3^2 m_4^2} \right)^{1+\epsilon} , \quad
  \hat I_{2,0,2,2} = \frac {\Gamma^3(1+\epsilon)}{\Gamma(1+3\epsilon)} \left( \frac 1 {m_1^2 m_3^2 m_4^2} \right)^{1+\epsilon} , \nonumber \\
  & \hat I_{2,2,0,2} = \frac {\Gamma^3(1+\epsilon)}{\Gamma(1+3\epsilon)} \left( \frac 1 {m_1^2 m_2^2 m_4^2} \right)^{1+\epsilon} , \quad
  \hat I_{2,2,2,0} = \frac {\Gamma^3(1+\epsilon)}{\Gamma(1+3\epsilon)} \left( \frac 1 {m_1^2 m_2^2 m_3^2} \right)^{1+\epsilon} \, .
  \label{eq:tadpoleCubedValues}
\end{align}
The values of the 11 remaining master integrals in Eq.~\eqref{eq:mastersBananaTop} will be calculated numerically from positivity constraints. We will work with kinematic variables in the range
\begin{equation}
  p^2 < (m_1+m_2+m_3+m_4)^2 \, ,
  \label{eq:bananaEucRegion}
\end{equation}
i.e.\ below the particle production threshold, which ensures that the integrals are real-valued.
Similar to the case of one-loop bubble integrals, when the chosen value of $p^2$ is non-negative, we cannot embed the integrals into Euclidean momentum space and need to use Feynman-parameter space to formulate positivity constraints. The Feynman parametrization follows from the general formula Eq.~\eqref{eq:generalFeynParam} with adjustment of constant factors according to Eq.~\eqref{eq:bananaHat},
\begin{align}
  \hat I_{a_1, a_2, a_3, a_4} &= \frac {\Gamma(a-3d/2) / \Gamma(4-3d/2)} {\Gamma(a_1) \Gamma(a_2) \Gamma(a_3) \Gamma(a_4)} \int_{x_i \geq 0} dx_1 dx_2 dx_3 dx_4 \, \delta(1-x_1-x_2-x_3-x_4) \nonumber \\
  &\quad \times \left( \prod_{i=1}^4 x_i^{a_i-1} \right) \frac {\mathcal U(x_i)^{a-2d}} {\mathcal F(x_i)^{a-3d/2}} \, ,
  \label{eq:feynParamBanana}
\end{align}
where we used the definition $a \equiv a_1+a_2+a_3+a_4$. The two graph polynomials $\mathcal U$ and $\mathcal F$ are, for the banana family of integrals,
\begin{align}
  \mathcal U(x_1, x_2, x_3, x_4) &= x_2 x_3 x_4 + x_1 x_3 x_4 + x_1 x_2 x_4 + x_1 x_2 x_3 , \nonumber \\
  \mathcal F(x_1, x_2, x_3, x_4) &= p^2 x_1 x_2 x_3 x_4 + (m_1^2 x_1 + m_2^2 x_2 + m_3^2 x_3 + m_4^2 x_4) \mathcal U(x_1, x_2, x_3, x_4) \, .
\end{align}
Note that we have
\begin{equation}
  \mathcal U(x_1, x_2, x_3, x_4) \geq 0, \quad \mathcal F(x_1, x_2, x_3, x_4) \geq 0 \, ,
\end{equation}
in the integration region of Eq.~\eqref{eq:feynParamBanana}, i.e.\ $x_i \geq 0, \, \sum_i x_i=1$. This will help us formulate positivity constraints. As in the one-loop bubble case, except for the ``gauge fixing'' Dirac delta function, the rest of Eq.~\eqref{eq:feynParamBanana} has the projective invariance Eq.~\eqref{eq:projective}, as the $\mathcal U$ and $\mathcal F$ polynomials are homogeneously of degree 3 and 4, respectively.

\subsection{Positivity constraints}
We define $\tilde x_i$ variables 
\begin{equation}
  \tilde x_i = \frac{\mathcal U(x_1, x_2, x_3, x_4)} {\mathcal F(x_1, x_2, x_3, x_4)} x_i, \quad i=1,2,3,4 \, ,
  \label{eq:rescaleXi}
\end{equation}
which are invariant under the scaling Eq.~\eqref{eq:projective}. We rewrite Eq.~\eqref{eq:feynParamBanana} as
\begin{align}
  &\quad \frac {\Gamma(4-3d/2)} {\Gamma(a-3d/2)} \Gamma(a_1) \Gamma(a_2) \Gamma(a_3) \Gamma(a_4) \, \hat I_{a_1, a_2, a_3, a_4} \nonumber \\
  &= \int_{x_i \geq 0} dx_1 dx_2 dx_3 dx_4 \, \delta(1-x_1-x_2-x_3-x_4) \left( \prod_{i=1}^4 \tilde x_i^{a_i-1} \right) \frac {\mathcal U(x_i)^{4-2d}} {\mathcal F(x_i)^{4-3d/2}} \, ,
  \label{eq:feynParamRescaledXi}
\end{align}
again with $a \equiv a_1+a_2+a_3+a_4$. Note that if $a_i \geq 1$, $a \geq 4$,
\begin{equation}
  \frac {\Gamma(4-3d/2)} {\Gamma(a-3d/2)} = \frac 1 {(4-3d/2) (5-3d/2) \dots (a-1-3d/2)}
\end{equation}
is a rational function in $d$.
With any non-negative polynomial $Q(\tilde x_i)$, we formulate a positivity constraint,
\begin{align}
  0 &\leq \int_{x_i \geq 0} dx_1 dx_2 dx_3 dx_4 \, \delta(1-x_1-x_2-x_3-x_4) Q(\tilde x_i) \frac {\mathcal U(x_i)^{4-2d}} {\mathcal F(x_i)^{4-3d/2}} \, ,
  \label{eq:bananaPos}
\end{align}
which is compatible with the projective invariance Eq.~\eqref{eq:projective}.
After expanding the polynomial $Q(\tilde x_i)$ into a sum of monomials, the contribution of each monomial $\prod_i \tilde x_i^{a_i-1}$ can be written as some $\hat I_{a_1, a_2, a_3, a_4}$ multiplied by a prefactor that is rational in $d$, according to Eq.~\eqref{eq:feynParamRescaledXi}. All such integrals are UV convergent by power counting and also IR convergent due to internal masses. No change of spacetime dimensions is involved, unlike the treatment of one-loop bubble integrals in Section \ref{subsec:posFeynSpace}.

We consider the following choices of $Q(\tilde x_i)$, with the help of an arbitrary polynomial $P(\tilde x_i)$ under a chosen maximum degree,
\begin{align}
\text{choice 1:} \quad Q(\tilde x_i) &= P(\tilde x_i)^2 , \label{eq:Qchoice1} \\
\text{choice 2:} \quad Q(\tilde x_i) &= \tilde x_1 P(\tilde x_i)^2 , \label{eq:Qchoice2} \\
\text{choice 3:} \quad Q(\tilde x_i) &= \tilde x_2 P(\tilde x_i)^2 , \\
\text{choice 4:} \quad Q(\tilde x_i) &= \tilde x_3 P(\tilde x_i)^2 , \\
\text{choice 5:} \quad Q(\tilde x_i) &= \tilde x_4 P(\tilde x_i)^2 \, . \label{eq:Qchoice5}
\end{align}
With any of the above five choices for $Q(\tilde x_i)$ and with any choice of $P(\tilde x_i)$, the inequality Eq.~\eqref{eq:bananaPos} must hold. The general form of $P(\tilde x_i)$ is a sum of all monomials under a chosen cutoff degree, each multiplied by an arbitrary coefficient. For example, if the cutoff degree is 1, then $P(\tilde x_i)$ is parametrized as
\begin{equation}
  \text{cutoff degree 1:} \quad P(\tilde x_i) = \alpha_{0,0,0,0} + \alpha_{1,0,0,0} \tilde x_1 + \alpha_{0,1,0,0} \tilde x_2 + \alpha_{0,0,1,0} \tilde x_3 + \alpha_{0,0,0,1} \tilde x_4 \, .
\end{equation}
With cutoff degree $N$, the parametrization is
\begin{equation}
  \text{cutoff degree $N$:} \quad P(\tilde x_i) = \sum_{i_1, i_2, i_3, i_4 \geq 0}^{i_1+i_2+i_3+i_4 \leq N} \alpha_{i_1, i_2, i_3, i_4} \tilde x_1^{i_1} \tilde x_2^{i_2} \tilde x_3^{i_3} \tilde x_4^{i_4} \, ,
  \label{eq:bananaGeneralP}
\end{equation}
where the number of free parameters $\alpha_{i_1, i_2, i_3, i_4}$ is equal to ${N+4 \choose 4}$ by combinatorics.

Since we have already discussed how to set up semidefinite optimization programs in the context of one-loop bubble integrals, we will be brief in covering the analogous steps here. Grouping the $\alpha_{i_1, i_2, i_3, i_4}$ parameters into a column vector $\vec \alpha$ of length ${N+4 \choose 4}$, the five choices of $Q$, Eqs.~\eqref{eq:Qchoice1} to \eqref{eq:Qchoice5}, lead to
\begin{align}
  & (\vec \alpha)^T \mathbb M^{(1)} \vec \alpha \geq 0 , \quad
  (\vec \alpha)^T \mathbb M^{(2)} \vec \alpha \geq 0 , \quad 
  (\vec \alpha)^T \mathbb M^{(3)} \vec \alpha \geq 0 , \nonumber \\
  & (\vec \alpha)^T \mathbb M^{(4)} \vec \alpha \geq 0 , \quad
  (\vec \alpha)^T \mathbb M^{(5)} \vec \alpha \geq 0 ,
\end{align}
respectively, for any values of the vector $\alpha$. For reasons we do not fully understand, the first constraint $(\vec \alpha)^T \mathbb M^{(1)} \vec \alpha \geq 0$ leads to poor numerical convergence and is discarded. The remaining four constraints are rewritten as requiring the matrices to be positive semidefinite using the notation Eq.\eqref{eq:posGeq},
\begin{equation}
  \mathbb M^{(2)} \succcurlyeq 0, \quad \mathbb M^{(3)} \succcurlyeq 0, \quad \mathbb M^{(4)} \succcurlyeq 0, \quad \mathbb M^{(5)} \succcurlyeq 0 \, .
\end{equation}
For convenience, this can be rephrased as the positive semidefiniteness of a single matrix which contains the above four matrices as diagonal blocks,
\begin{equation}
  \mathbb M = 
  \begin{pmatrix}
    \mathbb M_2 & 0 & 0 & 0 \\
    0 & \mathbb M_3 & 0 & 0 \\
    0 & 0 & \mathbb M_4 & 0 \\
    0 & 0 & 0 & \mathbb M_5 \\
  \end{pmatrix}
  \succcurlyeq 0 \, . \label{eq:bananaBlockMatPos}
\end{equation}

Analogous to the case of one-loop bubble integrals, IBP reduction expresses $\mathbb M$ as a linear combination of the 15 master integrals in Eq.~\eqref{eq:mastersBananaTop} and \eqref{eq:mastersTadCubed}, each multiplied by a matrix of rational functions in $p^2, m_1^2, m_2^2, m_3^2, m_4^2$. It is necessary to perform IBP reduction for banana integrals with up to 13 ``dots'', i.e.\ additional powers of propagators beyond the standard first power, since the positive polynomial $Q$ in Eqs.~\eqref{eq:Qchoice2}-\eqref{eq:Qchoice5} bring 13 powers of $x_i$ when $P$ has degree 6. The four master integrals in Eq.~\eqref{eq:mastersTadCubed} are known analytically in Eq.~\eqref{eq:tadpoleCubedValues}, and the values of the remaining 11 master integrals are unknown parameters to be constrained by Eq.~\eqref{eq:bananaBlockMatPos}.

Before presenting numerical results, we also formulate positivity constraints for the $\epsilon$ expansion of banana integrals. Recall that banana integrals in $d=2-2\epsilon$, normalized according to Eq.~\eqref{eq:bananaHat}, has a Feynman-parameter representation Eq.~\eqref{eq:feynParamRescaledXi} using redefined Feynman parameters in Eq.~\eqref{eq:rescaleXi}. For any integers $a_i \geq 1$, Taylor-expanding both sides of Eq.~\eqref{eq:feynParamRescaledXi} and equating the coefficients of the $\epsilon^k$ term for any integer $k$, we have
\begin{equation}
  \begin{aligned}
    &\quad \left[ \frac {\Gamma(4-3d/2)} {\Gamma(a-3d/2)} \Gamma(a_1) \Gamma(a_2) \Gamma(a_3) \Gamma(a_4) \, \hat I_{a_1, a_2, a_3, a_4} \right] \Bigg |_{\epsilon^k} \\
    &= \int_{x_i \geq 0} dx_1 dx_2 dx_3 dx_4 \, \delta(1-x_1-x_2-x_3-x_4) \left( \prod_{i=1}^4 \tilde x_i^{a_i-1} \right) \frac {1} {\mathcal F(x_i)} \\
    & \quad \times \frac 1 {k!}  \log^k \frac {\mathcal U^4(x_i)} {\mathcal F^3(x_i)} \, .
  \end{aligned}
  \label{eq:feynParamRescaledXiEps}
\end{equation}
By integration-by-parts reduction, the LHS of the above equation can be written as linear combinations of the $\epsilon$ expansions of the 15 master integrals up to the $\epsilon^k$ order, assuming that the coefficients of the master integrals (from IBP reduction) are finite as $\epsilon \to 0$, which is the case here. Now we are ready to write down positivity constraints for the $\epsilon$ expansion by extending Eq.~\eqref{eq:bananaPos},
\begin{align}
  0 &\leq \int_{x_i \geq 0} dx_1 dx_2 dx_3 dx_4 \, \delta(1-x_1-x_2-x_3-x_4) Q\left( \tilde x_i, \log \frac {\mathcal U^4(x_i)} {\mathcal F^3(x_i)} \right) \frac {\mathcal U(x_i)^{4-2d}} {\mathcal F(x_i)^{4-3d/2}} \, ,
  \label{eq:bananaPosEps}
\end{align}
where $Q$ is now a positive polynomial in its two arguments above.
To build the most general form of $Q$, we follow Section \ref{subsubsec:eps_taylored} and use the building block
\begin{equation}
  \log \max \frac {\mathcal U^4} {\mathcal F^3} - \log \frac {\mathcal U^4(x_i)} {\mathcal F^3(x_i)} \geq 0 \, .
  \label{eq:bananaMaxUFBlock}
\end{equation}
The value of $\max (\mathcal U^4 / \mathcal F^3)$ will be found numerically once $p^2$ and $m_i^2$ parameters are specified in Eq.~\eqref{eq:bananaNumericPoint}, in the next subsection on numerical results. However, we find no minimum of $\log (\mathcal U^4/\mathcal F^3)$ at the same parameter values, as $\mathcal U^4 / \mathcal F^3$ can become arbitrarily close to zero (from above) in the range of integration.
We will use the following choices of $Q$,
\begin{align}
  & Q \left( \tilde x_i, \log \frac {\mathcal U^4(x_i)} {\mathcal F^3(x_i)} \right) = \tilde x_k P^2 \left( \tilde x_i, \log \frac {\mathcal U^4(x_i)} {\mathcal F^3(x_i)} \right) , \label{eq:QepsChoice1} \\
  \text{or } \ & Q \left( \tilde x_i, \log \frac {\mathcal U^4(x_i)} {\mathcal F^3(x_i)} \right) = \tilde x_k \left( \log \max \frac {\mathcal U^4} {\mathcal F^3} - \log \frac {\mathcal U^4(x_i)} {\mathcal F^3(x_i)} \right) P^2 \left( \tilde x_i, \log \frac {\mathcal U^4(x_i)} {\mathcal F^3(x_i)} \right) \, , \label{eq:QepsChoice2}
\end{align}
where $k$ can be 1, 2, 3, or 4, and $P$ is an arbitrary polynomial with a maximum total degree $N_1$ for the four $\tilde x_i$ variables and maximum degree $N_2$ for $\log (\mathcal U^4/\mathcal F^3)$.
Similar to the bubble integral case in Section \ref{subsec:epExpansion}, to constrain the $\mathcal O(\epsilon)$ part of the master integrals, we will use Eq.~\eqref{eq:QepsChoice2} with $N_2=0$, and to constrain the $\mathcal O(\epsilon^2)$ part, we will use Eq.~\eqref{eq:QepsChoice1} $N_2=1$. For both $\mathcal O(\epsilon^1)$ and $\mathcal O(\epsilon^2)$ parts, $N_1$ will be chosen to be the same as the cutoff degree used for the $\mathcal O(\epsilon^0)$ calculation.

\subsection{Numerical results}
We present numerical results for the 11 nontrivial master integrals of the banana family in Eq.~\eqref{eq:mastersBananaTop} at the following numerical values for kinematic variables,
\begin{equation}
  p^2 = 2, \quad m_1^2 = 2, \quad m_2^2 = 3/2, \quad m_3^2 = 4/3, \quad m_4^2 = 1 \, .
  \label{eq:bananaNumericPoint}
\end{equation}
We remind readers that the spacetime dimension is set to
\begin{equation}
  d = 2 - 2\epsilon \, .
\end{equation}
As this paper is aimed at illustrating a new method, we have chosen example integrals that are known to high precision in the existing literature. For three-loop banana integrals, high precision results from series solutions to differential equations are available from the DiffExp package \cite{Hidding:2020ytt}. In fact, we have chosen the same values for the masses in Eq.~\eqref{eq:bananaNumericPoint} as the example in the aforementioned paper, though we chose a different value of $p^2$ as we restrict to the Euclidean region Eq.~\eqref{eq:bananaEucRegion}. DiffExp is used to compute the master integrals to a precision of about $10^{-54}$, which can be taken as exact values for the purpose of validating our numerical results.

For the values of the master integrals at $\mathcal O(\epsilon^0)$, i.e.\ in exactly $d=2$, we will use cutoff degrees of up to 6 in Eq.~\eqref{eq:bananaGeneralP}. Unlike the one-loop case in Section \ref{sec:bubble}, we will not fully characterize the allowed parameter region which is a sub-region of an 11-dimensional parameter space and cannot be described by just a lower bound and an upper bound. Instead, we will only compute the central values for the 11 undetermined master integrals. Following the prescription laid out in Section \ref{sec:bubble}, the central values are defined to maximize the lowest eigenvalue of the matrix $\mathbb M$ in Eq.~\eqref{eq:bananaBlockMatPos}. With the largest cutoff degree 6, there are ${{6+4} \choose 4} = 210$ free parameters in $\vec \alpha$. Therefore, each of the four diagonal blocks in Eq.~\eqref{eq:bananaBlockMatPos} has size $210 \times 210$, and the full matrix $\mathbb M$ has size $840 \times 840$. We use SDPA-QD as the semidefinite programming solver working at quadruple-double precision. The solver is able to take advantage of the block diagonal structure of the matrix $\mathbb M$ to improve efficiency.

In Fig.~\ref{fig:bananaErrorVsDeg}, we plot the relative errors of the central values of three representative master integrals against the cutoff degree, for the $\mathcal O(\epsilon^0)$ order only. The actual results are given later in Eq.~\eqref{eq:bananaSampleResults} together with further terms in the $\epsilon$ expansion.
%
\begin{figure}
  \centering
  \includegraphics[width=\textwidth]{figs/bananaErrorVsDeg.png}
  \caption{Relative errors of the central values of three representative master integrals of the banana family, versus the cutoff degree in the calculation.}
  \label{fig:bananaErrorVsDeg}
\end{figure}
%
The vertical axis of the plot is on a logarithmic scale, and we can see that the results converge rapidly, in a apparently exponential fashion, as the cutoff degree is raised. With the largest cutoff degree 6, each of the 11 master integrals is evaluated to an accuracy of at least $10^{-9}$.

For the values of master integrals at $\mathcal O(\epsilon^1)$. We will need the result for the parameter values Eq.~\eqref{eq:bananaNumericPoint},
\begin{equation}
  \max \left( \mathcal U^4 / \mathcal F^3 \right) \approx 5000/229059 \, ,
  \label{eq:bananaUFratioMax}
\end{equation}
which we found by numerical maximization in Mathematica. To be conservative, we have slighted rounded up the numerical result to a larger nearby rational number to guarantee that the inequality Eq.~\eqref{eq:bananaMaxUFBlock} is true when using Eq.~\eqref{eq:bananaUFratioMax}. This maximum value occurs at
\begin{equation}
  x_1 \approx 0.12222, \quad x_2 \approx 0.22592, \quad x_3 \approx 0.26701, \quad x_4 \approx 0.38485 \, .
  \label{eq:bananaUFratioMaxPoint}
\end{equation}
While Eq.~\eqref{eq:bananaUFratioMax} will be used directly in calculations, Eq.~\eqref{eq:bananaUFratioMaxPoint} is only included for completeness. The normalization in Eq.~\eqref{eq:bananaUFratioMaxPoint} does not matter since $\mathcal U^4 / \mathcal F^3$ is invariant under the scaling transformation Eq.~\eqref{eq:projective}.

Then the calculation is similar to the calculation of one-loop bubble interals to $\mathcal O(\epsilon^1)$ in Section \ref{sec:bubbleResultsEps}. We use the positivity constraint Eq.~\eqref{eq:bananaPosEps} with Eq.~\eqref{eq:QepsChoice2} for the positive polynomial $Q$, taking the values $k=1,2,3,4$ and combining the constraints from the four different choices. For the $P$ polynomial in Eq.~\eqref{eq:QepsChoice2}, we use a maximum total degree $N_1=6$ for the $\tilde x_i$ variables and a maximum degree of $0$ for the logarithm, i.e. dropping any terms involving the logarithm. The logarithm still appears in the bracket preceding $P^2$ in Eq.~\eqref{eq:QepsChoice2} and contributes to $O(\epsilon)$ parts of integrals by Eq.~\eqref{eq:feynParamRescaledXiEps}. Using the $\mathcal O(\epsilon^0)$ results as known inputs, we again solve a semidefinite programming problem involving a $840 \times 840$ matrix with four diagonal blocks, each of size $210\ \times 210$, to obtain values for the $\mathcal O(\epsilon^1)$ terms of the master integrals.

For the values of master integrals at $\mathcal O(\epsilon^2)$, we use the positivity constraint Eq.~\eqref{eq:bananaPosEps} with Eq.~\eqref{eq:QepsChoice1} for the positive polynomial $Q$. We again use a maximum total degree $N_1=6$ for the $\tilde x_i$ variables but now uses a maximum degree of $1$ for the logarithm. Since the logarithm can appear in a monomial in $P$ with either power 0 or power 1, the size of the matrix in the semidefinite programming problem is doubled to $1680 \times 1680$, with four diagonal blocks each of size $420 \times 420$. Taking both $\mathcal O(\epsilon^0)$ and $\mathcal O(\epsilon^1)$ results as known inputs, we run SDPA-QD to find the central values for the $\mathcal O(\epsilon^2)$ results. For brevity, we show results for 3 representative master integrals out of the 11 top-level master integrals, with kinematic variables taking values of Eq.~\eqref{eq:bananaNumericPoint},
\begin{equation}
  \begin{aligned}
    \hat I_{1122} & \approx 0.31328353052153 -0.12137516161264 \epsilon -1.5577062442336 \epsilon^2, \\
    \hat I_{1112} & \approx 1.3758733318476 -3.5451169250640 \epsilon +0.61363537070259 \epsilon^2, \\
    \hat I_{1111} & \approx 5.9437542439912 -33.914772364319 \epsilon +106.87640125797 \epsilon^2\, .
  \end{aligned}
    \label{eq:bananaSampleResults}  
\end{equation}
For documenting the computational outputs, we have kept each number to 14 significant figures, even their actual accuracies are lower as shown in plots in this section.

We have also calculated both $\mathcal O(\epsilon^1)$ and $\mathcal O(\epsilon^2)$ results using numerical differentiation of integrals evaluated at fixed values of dimensions, following the same strategy of Section \ref{subsec:numericalDiff} for one-loop bubble integrals. The calculations are identical to the $d=2$, i.e.\ $\mathcal O(\epsilon^0)$ case and are based on Eq.~\eqref{eq:bananaPos} without any Taylor expansion in $\epsilon$, with the only change being that $\epsilon$ is set to small numerical values different from 0, i.e.\ $d$ is set to numerical values that slightly deviate from $2$. The 4th-order numerical differentiation formulas, Eqs.~\eqref{eq:finiteDiff1} and Eq.~\eqref{eq:finiteDiff2} are applied with $\epsilon_0=0$ and $\Delta \epsilon = 10^{-3}$, with the spacetime dimension $d=2-2\epsilon$. Example results from this alternative method are, again keeping each number of 14 significant figures,
\begin{equation}
  \begin{aligned}
    \hat I_{1122} & \approx 0.31328353052153 - 0.12137519105424 \epsilon -1.5576503067221 \epsilon^2, \\
    \hat I_{1112} & \approx 1.3758733318476 -3.5451170400199 \epsilon + 0.61369255775305 \epsilon^2, \\
    \hat I_{1111} & \approx 5.9437542439912 -33.914771261794 \epsilon + 106.87318272740 \epsilon^2\, .
  \end{aligned}
    \label{eq:bananaFiniteDiffResults}  
\end{equation}
Note that the $\mathcal O(\epsilon^0)$ results are copied from Eq.~\eqref{eq:bananaSampleResults} as they are not re-calculated.

%
\begin{figure}[h]
  \centering
  \includegraphics[width=\textwidth]{figs/bananaCombinedChart.png}
  \caption{Log-scale plot of relative errors of numerical results for 11 nontrivial master integrals of the three-loop banana family, Eq.~\eqref{eq:mastersBananaTop}, with the normalization Eq.~\eqref{eq:bananaHat}, up to second order in the dimension regularization parameter $\epsilon$. The four numbers under each bar indicates the subscript indices of $\hat I_{a_1, a_2, a_3, a_4}$, with commas omitted as all four indices are single-digit numbers (either 1 or 2). For each master integral, the five vertical bars, from left to right, show the relative errors for the $\epsilon^0$ term, the $\epsilon^1$ term calculated from direct positivity constraints (abbreviated as {\tt cons} in the legend), the $\epsilon^1$ term calculated from numerical differentiation (abbreviated as {\tt diff} in the legend), the $\epsilon^2$ term calculated from direct positivity constraints, and the $\epsilon^2$ term calculated from numerical differentiation.}
  \label{fig:bananaCombinedChart}
\end{figure}
%
These numerical results for the $\epsilon$ expansion of master integrals Eq.~\eqref{eq:mastersBananaTop} are obtained with the normalization of Eq.~\eqref{eq:bananaHat}. The reference results from DiffExp are have the normalization of Eq.~\eqref{eq:banana} and additional factors for individual master integrals. The reference results have been converted to use our normalizations for comparison. DiffExp results for the three sample integrals, truncated to 14 significant digits, are
\begin{equation}
    \begin{aligned}
      \hat I_{1122} & \approx 0.31328353056677-0.121375191032390 \epsilon-1.5577067713048 \epsilon^2 , \\
      \hat I_{1112} & \approx 1.37587333189510-3.5451170391547 \epsilon+0.61363351857945 \epsilon^2, \\
      \hat I_{1111} & \approx 5.9437542414259-33.914771263107 \epsilon+106.876390717227 \epsilon^2 \, .
    \end{aligned}
    \label{eq:bananaDiffExpResults}  
\end{equation}
The final numerical accuracy for the 11 master integrals, with values of kinematic parameters chosen in Eq.~\eqref{eq:bananaNumericPoint}, is shown in Fig.~\ref{fig:bananaCombinedChart}.
The $\epsilon$ expansion results from ``direct positivity constraints'' are labeled {\tt cons} and results from numerical differentiation are labeled {\tt diff} in the plot legend. We can see that numerical differentiation gives very good accuracy for $\mathcal O(\epsilon^1)$ terms, comparable with the accuracy of $\mathcal O(\epsilon^0)$ terms, while for the $\mathcal (\epsilon^2)$ terms, direct positivity constraints yield more accurate results. In any case, both methods for the $\epsilon$ expansion have demonstrated their potentials in this initial investigation, as all results for $\mathcal O(\epsilon^1)$ terms have relative errors below $10^{-6}$ and all results for $\mathcal O(\epsilon^2)$ terms have relative errors below $10^{-3}$.

We briefly comment on computational resources used. IBP reduction takes a few CPU-hours with FIRE6 \cite{Smirnov:2019qkx} with numerical kinematics Eq.~\eqref{eq:bananaNumericPoint}. The IBP reduction results are obtained with analytic dependence on $d$ and can be subsequently expanded in $\epsilon$, so no extra IBP reduction is needed for obtaining the $\epsilon$ expansions of master integrals beyond the zeroth order. Running the semidefinite programming solver SDPA-QD takes a few CPU-hours for every run, including one run for solving positivity constraints for the $\mathcal O(\epsilon^i)$ part for each $i=0, 1, 2$, and for the alternative method based on numerical differentiation, several runs at different numerical values of $\epsilon$ to generate the data needed to feed into finite-difference approximations.
