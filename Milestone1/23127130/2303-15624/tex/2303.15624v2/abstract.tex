We explore inequality constraints as a new tool for numerically evaluating Feynman integrals. A convergent Feynman integral is non-negative if the integrand is non-negative in either loop momentum space or Feynman parameter space. Applying various identities, all such integrals can be reduced to linear sums of a small set of master integrals, leading to infinitely many linear constraints on the values of the master integrals. The constraints can be solved as a semidefinite programming problem in mathematical optimization, producing rigorous two-sided bounds for the integrals which are observed to converge rapidly as more constraints are included, enabling high-precision determination of the integrals. Positivity constraints can also be formulated for the $\epsilon$ expansion terms in dimensional regularization and reveal hidden consistency relations between terms at different orders in $\epsilon$. We introduce the main methods using one-loop bubble integrals, then present a nontrivial example of three-loop banana integrals with unequal masses, where 11 top-level master integrals are evaluated to high precision.
