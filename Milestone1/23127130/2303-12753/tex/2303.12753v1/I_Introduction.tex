
\section{Introduction} \label{sec:Intro}


Segmentation is a fundamental task in a lot of applications, such as 
% autonomous vehicles, 
% satellite imagery, 
shadow detection
and medical imaging \cite{liao2021shadow, wu2021federated, yang2020co-exploration}. 
With the  fast development of artificial intelligence (AI), targeting the segmentation tasks, there emerge both manually developed deep neural network architectures (e.g., encoder and decoder \cite{noh2015learning, badrinarayanan2017segnet} and U-Net \cite{ronneberger2015u}) as well as automated generated architectures (e.g., Auto-DeepLab \cite{liu2019auto}).
These deep learning models have demonstrated superior performance on benchmarking datasets against segmentation in traditional imaging processing.
However, when it comes to real-world applications, the lack of annotated data becomes a critical issue: on the one hand, the annotation demands human labor to label data which is costly, not mention to the segmentation tasks that require labels for each pixel;
on the other hand, the lack of training data will drastically degrade segmentation performance.
In consequence, how to perform segmentation without plenty of annotated data is highly desired.

With such a need, unsupervised learning seems to be a natural answer, since it can discover useful patterns in data without annotation \cite{celebi2016unsupervised}. 
It seems straightforward to apply unsupervised learning to segmentation, however, the segmentation task itself and the demands from the application bring new challenges.
First, unlike unsupervised classification (a.k.a, clustering), segmentation requires the process of features for every pixel to figure out which ones are compact enough to form distinct clusters.
As such, unsupervised segmentation can require a deeper neural architecture, which makes the model much larger.
Second, real-world segmentation applications commonly have data privacy demands (e.g., medical imaging) and real-time requirements (e.g., autonomous vehicles).
The ideal solution to address data privacy is to process data on-device~\cite{yang2020co, jiang2019accuracy, jiang2020hardware, jiang2019achieving, zhan2021accelerating, zhan2018energy}; however, the edge devices have limited computation resources and they are facing a large model size, both of which conflict with the real-time requirement.

To address the above challenges, we are rethinking what is the best computing model to extract features for segmentation tasks.
For a long while, due to the superior performance of neural networks, they are typically adopted in unsupervised segmentation \cite{xia2017w, kim2020unsupervised}.
However, as stated, it requires a large model size, which may easily exceed the capacity of edge devices.
In segmentation, the pixel position and color are the most important information, and we find that it can achieve high performance if we smartly map the pixels into a high-dimensional space, according to the position and color information.

With such motivation, in this paper, we for the first time bring a recent emerging computing model, i.e., brain-inspired hyperdimensional computing (HDC) \cite{ge2020classification}, into image segmentation.
HDC has shown its superiority in robustness, scalability, and high energy efficiency for classification tasks \cite{zhang2021assessing, zhang2022scalehd, yangautomated, yang2022hardware, zhang2022energy}.
Fundamentally, HDC encodes data into a high-dimensional space using a hypervector (HV), and learns features in that space \cite{sheng2023toward}.
It seems that HDC is naturally suitable for segmentation; however, this characteristic of HDC has not been well developed and utilized. 
To bring HDC to segmentation, the first task is to figure out how to encode pixels such that the encoded pixels can 
% figure out how to design HDC at the pixel level.
% Then, it requires a new encoding method that can
precisely describe the disparity.
What's more, since the feature extracted by HDC is represented in high-dimension; it is challenging to efficiently perform clustering on high-dimension vectors.



% Furthermore, straightforwardly applying HDC to perform image segmentation, it is not clear how to apply HDC in pixel-level work and how to effectively perform segmentation utilizing HDC.  


% Image segmentation is typically used to locate objects and boundaries (lines, curves, etc.) in images




% To implement on-device ML tasks, we may consider other types of algorithms.
% Recently, brain-inspired hyperdimensional computing (HDC)\cite{ge2020classification}, 
% % another type of machine learning, 
% is becoming an active research area due to its robustness, scalability, and high energy efficiency.
% HDC has shown its superiority in supervised image classification and text processing tasks \cite{liang2022distrihd, imani2021revisiting}. 
% In classification tasks to which HDC has been mostly applied, HDC encodes the data into a high dimensional space using a hypervector (HV), and learns features in such space. 
% % For example, HDC can map the information of images, like color and position, to the high dimensional space and classify the data using the mapped HV. 
% % with features in high dimensionality
% % called encoded HV, and \todo{accumulates the HV in the same classes to gain new HVs}, called class HVs, to represent the classes. The class HVs contain the features of the classes and thus HDC checks the similarity between the given test HV and the class HVs to predict which class the test data should be classified. 
% % HDC can map the features of color and position to a very high dimensional space and classify the data with features in high dimensionality.
% It seems that HDC is naturally suitable for segmentation; however, this characteristic of HDC has not been well developed and utilized. Further than straightforwardly applying HDC to perform image segmentation, it is not clear how to apply HDC in pixel-level work and how to effectively perform segmentation utilizing HDC.  



% % As an example, it can be used in the process of multimedia content as clustering or partitioning of data in the absence of class labels \cite{cunningham2008supervised}. 

% \clearpage


% However, supervised learning is the availability of annotated training data \cite{cunningham2008supervised}. 


% a lot of challenges can be solved with AI techniques, e.g, image classification, language processing, and image segmentation. 
% Machine learning (ML), the main domain in artificial intelligence (AI), mainly consists of supervised learning and unsupervised learning. The characteristic of supervised learning is the availability of annotated training data \cite{cunningham2008supervised}. 
% Data is one of the most significant parts of supervised learning because supervised learning algorithms induce models from the training data. 
% However, supervised learning algorithms can not perform well when there is no enough data and 
% % a lot of data or 
% even worse when data is not available. 
% In contrast, unsupervised learning does not require labeled data. 
% % during the learning process. 
% Unsupervised learning can automatically discover useful patterns in unavailable data \cite{celebi2016unsupervised}. 
% Specifically, it can be used in the process of multimedia content as clustering or partitioning of data in the absence of class labels \cite{cunningham2008supervised}. 
% % Besides, it is also important for tasks of data exploration, recommender systems and etc.


% % The defining characteristic of supervised learning is the availability of annotated training data.  Supervised learning algorithms induce models from these training data and these models can be used to classify other unlabelled data.
% % particularly machine learning and deep learning,

% % 1. supervised learning and unsupervised learning
% % (1) AI development.
% % (2) supervised learning good perfromance.
% % (3) data problem.
% % (4) unsupervised learning can be used in a lot of places.
% % (5) unsupervised learning important.

% % % 2. image segmentation
% % (1) the meaning of segmentation
% % (2) difficulty
% % (1) machine learning can be used to seg。
% % (2) most supervised are used.
% % (3) recently unsupervised 也有
% % (4) 问题所在

% Image segmentation is one of the hottest topics in ML. It can be widely applied in applications of scene understanding, medical image analysis, 
% % robotic perception, video surveillance, augmented reality, and image compression, 
% among many others \cite{minaeeimage}. 
% % Image segmentation is not an easy task due to plenty of reasons, like the variety of images. 
% In recent years, researchers utilize deep learning (DL) algorithms to perform image segmentation and gain good results. 
% Most of these algorithms are supervised learning algorithms \cite{minaeeimage}. 
% % since they greatly rely on annotated data \todo{\cite{}}. 
% % \todo{Unsupervised image segmentation may not be able to perform as well as supervised image segmentation algorithms, }
% % But it is still valuable as collecting labeled samples for segmentation problems is problematic in many application domains, particularly so in medical image analysis \cite{minaeeimage}.
% Nevertheless, with the difficulty of data acquisition, researchers start to consider unsupervised image segmentation recently \cite{kim2020unsupervised}. 
% % In addition to accuracy, 
% However, new challenges (e.g., the time-consuming and the big size) of the convolutional neural network (CNN) are raised by unsupervised image segmentation algorithms. 


% Data pravicy and security for ML tasks is also an issue when deploying model on devices. On-device learning \cite{zhou2021device} has been proposed as one of the desirable solutions.
% % have privacy
% % requirements, therefore, it further calls for on-device learning
% % The emerging edge intelligence is highly desired on-device learning due to the problems of privacy and security, 
% % personalized needs,
% % and data transmission \cite{zhou2021device}.
% % On-device learning is important. 
% The primary constraint to train models on-device in a reasonable time frame is the lack of computing and memory on the device \cite{dhar2021survey}. 
% % Specifically, on-device image segmentation is extremely meaningful cause a sea of applications takes it as a base, e.g., object extraction and background editing \cite{zhang2019portraitnet}. 
% % However, as the resource constraints mentioned above, seldom algorithms could be applied on the edge device. 
% Thus, seldom algorithms can be applied on edge devices, particularly for image segmentation algorithms.
% What's worse, all these applicable ones are supervised learning, which means they will not work when training data is available. Thus, an on-device unsupervised image segmentation approach is highly required.

% % 4.On-device learning
% % (1)difficulty 
% % (2) on-device segmentation meaningful
% % (3) but has not been achieved
% % (4) highly required.

% % And thus so far the on-device unsupervised image segmentation has not been solved. 


% % as collecting labeled samples for segmentation problems is problematic in many application domains, particularly so in medical image analysis

% % With the boosting development of machine learning (ML), deep learning 

% % % 3. HDC
% % (1) basic concept
% % (2) can be used to image classification and text processing
% % (3) advantages, 除了常规的,还写map到high level
% % Besides DL, 
% To implement on-device ML tasks, we may consider other types of algorithms.
% Recently, brain-inspired hyperdimensional computing (HDC)\cite{ge2020classification}, 
% % another type of machine learning, 
% is becoming an active research area due to its robustness, scalability, and high energy efficiency.
% HDC has shown its superiority in supervised image classification and text processing tasks \cite{liang2022distrihd, imani2021revisiting}. 
% In classification tasks to which HDC has been mostly applied, HDC encodes the data into a high dimensional space using a hypervector (HV), and learns features in such space. 
% % For example, HDC can map the information of images, like color and position, to the high dimensional space and classify the data using the mapped HV. 
% % with features in high dimensionality
% % called encoded HV, and \todo{accumulates the HV in the same classes to gain new HVs}, called class HVs, to represent the classes. The class HVs contain the features of the classes and thus HDC checks the similarity between the given test HV and the class HVs to predict which class the test data should be classified. 
% % HDC can map the features of color and position to a very high dimensional space and classify the data with features in high dimensionality.
% It seems that HDC is naturally suitable for segmentation; however, this characteristic of HDC has not been well developed and utilized. Further than straightforwardly applying HDC to perform image segmentation, it is not clear how to apply HDC in pixel-level work and how to effectively perform segmentation utilizing HDC.  

To address these challenges, we propose a novel framework, namely SegHDC.
%To the best of our knowledge, this is the first work to implement on-device unsupervised image segmentation. 
%We novelly utilize HDC to perform image segmentation. 
Through a fundamental analysis of the characteristics of HDC and image segmentation, we first propose a brand-new HDC encoding approach to encode both position and color information into high dimensional space. Then, a revised K-Means algorithm has been devised to cluster encoded HVs and label corresponding pixels.


The main contributions of this paper are as follows.
\begin{itemize}
    \item To the best of our knowledge, this is the first work to perform on-device unsupervised image segmentation.
    \item We innovatively apply hyperdimensional computing for image segmentation tasks with a brand-new design for encoding images. 
    % And we first propose the new HDC encoding method for images.
    \item Evaluation results verify the  effectiveness of SegHDC, which surpasses  state-of-the-art unsupervised CNN-based algorithm.
\end{itemize} 

We have carried out a set of experiments on 3 commonly used segmentation datasets to evaluate the effectiveness of our proposed SegHDC. 
% Our approach deals with all images without using any annotation and produces masks for each image.
Experiment results on the DSB2018 dataset
show the efficiency of SegHDC, specifically by outperforming the CNN-based baseline with 28.0\% improvement in Intersection over Union (IoU) score; meanwhile, achieving over 300$\times$ speedup on the edge device.
On the BBBC005 dataset with a larger size of images, SegHDC achieves an IoU score of 0.9414 which is 25.7\% higher than the score obtained by the baseline. SegHDC also obtains a 0.9587 IoU score for a sample image in the BBBC005 dataset with a latency of only around 178 seconds, while the existing CNN-based unsupervised segmentation approach cannot predict this image due to the computing resources limitation. On the other dataset, MoNuSeg, SegHDC also gains an improvement of 8.27\% compared with the baseline method.
% larger size of the image, 

The remainder of the paper is as follows: Section~\ref{sec:rel} presents
the related work and motivation. Section~\ref{sec:method} demonstrates our proposed SegHDC framework. Experimental results and conclusion are in Section~\ref{sec:exp} and Section~\ref{sec:conclusion} respectively.
% so, we do this work.


% \clearpage





% \clearpage


% contribution.

% organization





% \clearpage