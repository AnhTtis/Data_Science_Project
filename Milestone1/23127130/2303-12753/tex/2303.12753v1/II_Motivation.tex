
% 
% \subsection{Related Work}

% \textbf{1. Unsupervised image segmentation}

% \textbf{2. On-device learning}

% \textbf{3. Hyperdimensional Computing}


% \clearpage
\vspace{3pt}
\section{Related Work and Motivation}\label{sec:rel}
\vspace{3pt}
% \subsection{Motivation and Challenges}

This section will first discuss the need for unsupervised segmentation and related work. 
Then, we provide our observation which motivates to use hyperdimensional computing for segmentation.

%\noindent\textbf{Need and challenge:
\noindent\textbf{Need and challenge: Unsupervised and on-device learning for image segmentation is highly demanded.} 

Image segmentation is a typical task in machine learning, and supervised learning has a high cost of labeling.
Unlike the classification task that one image needs only one label \cite{sheng2022larger, hu2022design, jiang2020standing, zhan2021improving}, the labeling of the segmentation dataset requires the assignment of a class to each pixel in an image.
What's worse, since segmentation is largely required in domain-specific applications, like medical imaging, the labeling task commonly calls for domain expertise.
For example, only good Computed Tomography (CT) doctors can distinguish if there is accurate lesion exists and where the accurate lesion is.
The involvement of doctors to do the labeling work is obviously too costly.
% too expensive to hire a lot of good  CT doctors to do the heavy work. 
To overcome the high cost of labeling for segmentation tasks, unsupervised learning is highly demanded, which does not need labels of data to perform the segmentation tasks.
What's more, unsupervised segmentation can be applied to perform the automated annotation.


Although promising, unsupervised segmentation commonly requires a longer time over the inference of supervised learning.
What's more, when unsupervised segmentation is applied to real-world applications, with the consideration of data security, in-situ and real-time processing on edge devices is typically required.
Besides, the high cost of unsupervised learning and the limited computing resources on edge devices make the problem more challenging.

% is really expensive. The limited computing resources make it hard to come true.

% Edge devices are deployed in various environments. Supervised models may be deployed on edge devices to do inference. However,  
% \todo{the classical CNN models of segmentation may be too large to deploy on edge devices.} Another issue is that the weight of the model can not be tuned according to the change in environment. This may lead to a dramatic drop in performance. Learning on the device seems to be a solution to this issue. However, the cost of on-device supervised learning is really expensive. The limited computing resources make it hard to come true.
% Thus, unsupervised learning is another promising solution that may solve the previous issues. Unsupervised learning of image segmentation 
% also brings a new issue, time consumption. Unsupervised segmentation may require a longer time to do the prediction than supervised learning inference. Thus, how to solve all of these issues is a big challenge.

% \textbf{Motivation 3: On-device efficient image segmentation is highly desired.} 

% There are a lot of limitations and issues that have not been solved for on-device image segmentation. But it is really valuable to implement segmentation on edge devices. To solve these problems, we wish to apply the HDC on edge devices. Since it seems that HDC with its characteristics is naturally suitable for edge devices and the segmentation task. Thus we are motivated to design an on-device efficient image segmentation tool.




% Data is one most significant parts of machine learning. 
% Most classification tasks in machine learning have one label for one item (e.g., labeling "cat" for an image with a cat).
% Different from the classification tasks, labeling segmentation tasks are much more complex. Labeling image segmentation tasks  require labeling all the pixels of the figure.
% This means the workload may be hundreds, thousands, or even more than the classification tasks \todo{\cite{}}. 
% \todo{cite to show the cost of segmentation task}. Particularly, in some domains, segmentation is highly required. However, normal people labeling for the specific domain images may lead to coarse label results, and thus the data can not be used for learning tasks. 



% labeling task 
% % So the experts in these domains are needed to do labeling. 
% % the people without experience may be not able to do the labeling, especially in some specific domain. 
% But the experts in these domains are too expensive to do the labeling. 
% For example, only good Computed Tomography (CT) doctors can distinguish if there is accurate lesion exists and where the accurate lesion is.  
% But the number of good CT doctors is much less than required.
% So it is too expensive to hire a lot of good  CT doctors to do the heavy work. 
% Therefore, segmentation with little data is a highly required task.




% Supervised learning may provide good performance, while unsupervised learning totally does not need labels of data. Thus, unsupervised learning may be a better choice when the  data or labeled data are not available.
% From the aspect of segmentation labeling challenges, unsupervised learning on segmentation is much more meaningful than it is on the classification tasks. While the previous segmentation works are mainly using convolutional neural networks (CNNs) to extract the feature of the pixels. There is no previous work working on applying HDC to segmentation. So, we are highly willing to apply the HDC to do the unsupervised segmentation.

% \textbf{Challenge 2: Segmentation is highly required, but labeling is too expensive, in particular for medical images.} 


%\textbf{Related work: Supervised segmentation has been widely studied, but unsupervised segmentation is still in its infancy.} 
\vspace{3pt}
\noindent\textbf{Related work: Supervised segmentation has been widely studied, but unsupervised segmentation is still in its infancy.} \vspace{3pt}

%\textbf{Supervised segmentataion}: 
The very first deep learning-based image segmentation was proposed in \cite{long2015fully}, which used a fully convolutional network (FCN) to perform segmentation. Another popular method used encoder-decoder
architecture segment the images \cite{noh2015learning, badrinarayanan2017segnet}. Inspired by the FCN and encoder-decoder architecture, U-Net \cite{ronneberger2015u} and V-Net \cite{milletari2016v} were proposed, which largely prompt the segmentation performance. 
%\textbf{Unsupervised segmenatation:} 
Unsupervised segmentation becomes active recently.
% and it can be applied for medical image. 
Authors in \cite{moriya2019unsupervised} presented a novel unsupervised segmentation method for the 3-D
% three-dimensional 
microstructure of lung cancer specimens in micro-computed tomography 
% (micro-CT) 
images. A CNN-based unsupervised image segmentation method was proposed in \cite{kim2020unsupervised}, and a clustering function was used after the CNN to cluster the pixels. Generative adversarial networks (GANs) were also used for unsupervised image segmentation \cite{abdal2021labels4free}. The authors utilized the features generated by a GAN and trained the segmentation networks.

Recently, there emerge research efforts in on-device learning for better deploying ML models on edge devices.
Some traditional ML 
models, like SVMs, can be directly accommodated on edge devices \cite{dhar2021survey}. In \cite{zhou2021octo}, the authors 
employ the ``int8'' quantization in both forward and backward passes over a deep model to enable on-device learning. Neural architecture search (NAS) is another solution to find tiny ML models and on-device training, like MCUNets \cite{lin2022device}.  

However, there is still a missing link between unsupervised segmentation and on-device learning.
Instead of repeatedly designing algorithms based on convolutional neural networks, we believe new innovations are needed to fill such a gap.


\vspace{3pt}
\noindent\textbf{Observation: Vectorizing pixels in a high-dimension space can map pixels in a similar area.} 
\vspace{3pt}

\begin{figure}[t]
%\vskip 0.2in
\begin{center}
% \centerline{\includegraphics[width=\columnwidth]{figure/framework_overview.eps}}
% \includegraphics[width=\columnwidth]{figures/motivation_new.pdf}
\includegraphics[width=3in]{figures/motivation_new.pdf}
\vskip -0.1in
% \vspace{-15pt}
\caption{Vectorized pixels with the same color have a short distance in space.}
\label{fig:mot1}
\end{center}
\vskip -0.1in
\end{figure}

The previous segmentation works mainly employ convolutional neural networks (CNNs) to extract the feature of the pixels, which brings high computation costs and can easily become the performance bottleneck on edge devices.
In this paper, we aim to simplify the feature extraction process for segmentation.
Specifically, each pixel has two fundamental information: (1) position, and (2) color. The question is whether we can perform segmentation by only exploring the spatial and color correlation of pixels.

To this end, we perform vectorization of pixel position and color and map each pixel in a high-dimension space.
% by using vector operators.
To enable the visualization, we limit the vector dimension to 3, as the example shown in Figure \ref{fig:mot1}.
In this example, we generate a $3\times 3$ binary image. 
We randomly assign a binary vector to each row and column, and use $XOR$ to associate the vectors between a row and a column to generate the vector of a position. 
We also randomly generate a vector for each color.
For example, the first pixel has the vector of $(1,0,1)$ for position and $(0,1,1)$ for the white color.
The second pixel has a different position vector $(1,1,1)$ and the same vector for color.
After we obtain these vectors, we sum up the pair of vectors at the same position to map the pixel to the 3-dimension space.
The right-hand in Figure \ref{fig:mot1} shows the distribution of all these 9 pixels in the Cartesian coordinate system. 
% We then use the vectors to plot the 9 pixels in the Cartesian coordinate system shown in the right part of Figure \ref{fig:mot1}. 
We can easily observe that all the white pixels are mapped to a small area while the black pixels are mapped to another distinct area.


% And we also assigned 2 binary vectors  with dimensionality 3, where the value of each dimension in one is opposite to the  value of each dimension in the other. For example, the vector (1, 0, 0) is used to represent the color ``black'', while the other vector (0, 1, 1) is used to represent the color ``white''.

% (e.g., (1, 0, 0), (1, 1, 0), (1, 0, 1)) and column (e.g., (0, 0, 1), (0, 1, 1), (0, 0, 0)) and used the $XOR$ to associate the vectors of rows and columns. 
% Thus we got 9 vectors with a dimensionality of 3 to represent the 9 pixel points.
% And we also assigned 2 binary vectors  with dimensionality 3, where the value of each dimension in one is opposite to the  value of each dimension in the other. For example, the vector (1, 0, 0) is used to represent the color ``black'', while the other vector (0, 1, 1) is used to represent the color ``white''. 
% We will get 9 new vectors to represent the 9 pixels if we use the 9 pixel points vectors to add the relative color vectors. 

% We then use the vectors to plot the 9 pixels in the Cartesian coordinate system shown in the right part of Figure \ref{fig:mot1}. It is obvious that all the white pixels are mapped to a small area while the black pixels are mapped to another distinct area.
\vspace{3pt}
\noindent\textbf{Motivation: Using hyperdimensional computing (HDC) for unsupervised segmentation.} 
\vspace{3pt}

The above example gives us the hint that mapping image pixels to high-dimension space can be an effective way to perform the segmentation.
We are inspired and novelly involve hyperdimensional computing for segmentation process and apply it to more complicated images, to better represent pixels using vectors with much higher dimensions (e.g., 10,000) than the 3-dimension vectors used in the above example.

%\textbf{Related work for HDC:} 
Unlike the traditional ML algorithms to directly process data, 
%Different from traditional ML, 
HDC will encode data into a high-dimensional space, where the data are represented by high-dimensional and pseudo-orthogonal hypervectors (HVs) \cite{ge2020classification}.
% ,kanerva2009hyperdimensional}.
For an HV with dimension $d$, we can denote it as $
    \overrightarrow{H} \  = \ (\overrightarrow{e_1}, \ \overrightarrow{e_2}, \ ..., \ \overrightarrow{e_i}, \ ..., \ \overrightarrow{e_d})$,
% as follows, 
% \begin{equation}
% \small
%     \centering
%     \overrightarrow{H} \  = \ (\overrightarrow{e_1}, \ \overrightarrow{e_2}, \ ..., \ \overrightarrow{e_i}, \ ..., \ \overrightarrow{e_d}) \label{equ:hv}
% \end{equation}
where $\overrightarrow{e_i}$ represents the $i-th$ element in HV $\overrightarrow{H}$. 
In this way, data can be encoded and features can be extracted in the high-dimensional space.
Existing works have shown that HDC can work well on image classification \cite{liang2022distrihd}.

To the best of our knowledge, there is no work to use HDC for segmentation tasks. It seems straightforward to extend HDC designed for classification to perform segmentation tasks.
However, most of the existing classification approaches rely on the randomly generated vectors to represent pixels, while the relative relationship between different positions and between different colors is a key to extracting features in segmentation, which is not studied.
As explored in our work, together with experimental results, directly applying the existing approach cannot perform a good segmentation.
%Directly applying the existing approach cannot perform a good segmentation, which will be demonstrated in our experimental results.
Therefore, a more dedicated design of position and color embedding is needed. With such vision, we have proposed a holistic framework to complete these tasks with details in Section \ref{sec:method}.


 


% Based on the observation, HDC may be really suitable for segmentation. But it seems this characteristic has not been developed. HDC can extract the position information and color information and map them to very high dimensionality using hypervector. From the high dimensionality, it is easier to distinguish the hypervector with different features. According to the nature of HDC, we need to explore how to utilize the characteristics to perform segmentation.
% there is no previous research on this.

% \textbf{Challenge 1: How to leverage such property to perform segmentation.}  

% Observation 1 gives us some insight. The pixels with position and color information can be mapped to 3 dimensions. However, the real images are much more complex than the example. And vectors with dimension 3 may not have such capacity to contain a lot of features. It seems to be straightforward to apply the HDC to do the image segmentation. Nevertheless, to the best of our knowledge, there is no previous work to apply the HDC at the pixel level. Thus, how to apply the HDC to segmentation is a big challenge.

% \clearpage

% \textbf{Challenge 3: On-device Segmentation is not easy to be done.} 

% Edge devices are deployed in various environments. Supervised models may be deployed on edge devices to do inference. However,  
% \todo{the classical CNN models of segmentation may be too large to deploy on edge devices.} Another issue is that the weight of the model can not be tuned according to the change in environment. This may lead to a dramatic drop in performance. Learning on the device seems to be a solution to this issue. However, the cost of on-device supervised learning is really expensive. The limited computing resources make it hard to come true.
% Thus, unsupervised learning is another promising solution that may solve the previous issues. Unsupervised learning of image segmentation 
% also brings a new issue, time consumption. Unsupervised segmentation may require a longer time to do the prediction than supervised learning inference. Thus, how to solve all of these issues is a big challenge.

% \textbf{Motivation 3: On-device efficient image segmentation is highly desired.} 

% There are a lot of limitations and issues that have not been solved for on-device image segmentation. But it is really valuable to implement segmentation on edge devices. To solve these problems, we wish to apply the HDC on edge devices. Since it seems that HDC with its characteristics is naturally suitable for edge devices and the segmentation task. Thus we are motivated to design an on-device efficient image segmentation tool.


% % \clearpage