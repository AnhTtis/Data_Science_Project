\section{Background and Related Works}
\label{sec:background}

\textbf{Privacy Nutrition Labels.}
Originally introduced by Kelley et al.~\cite{kelley_labels, kelley_study_label}, privacy nutrition labels aim to summarize the privacy practices of websites in a nutrition label format for better visual comprehension. They later designed the ``Privacy Facts'' display to allow the users to consider privacy while installing apps~\cite{kelley2013privacy}. More recently, researchers proposed an Internet of Things (IoT) security and privacy label~\cite{emami2020ask, emami2021privacy} to surface privacy and security information related to IoT devices to the users. Researchers have also studied the design and evaluation of privacy notices and labels~\cite{balebako2015impact, kelley_labels, kelley_privacy_app, kelley_study_label, kelley2013privacy, cranor2012necessary,schaub2015design, mcdonald2009comparative, fox2018communicating, cranor2022mobile}.
\smallskip

In December 2020, Apple adopted the privacy nutrition labels for the app store and mandated that app developers provide their apps' privacy information in the form of the Apple Privacy Label (APL). More recently, Google also required app developers to add a Data Safety Section (DSS) on the Google Play Store.


  


\begin{figure}[t]
  \centering
  \includegraphics[width=\columnwidth]{figures/apple_apl.pdf}
  \caption{The hierarchy of Apple Privacy Labels}
  \label{fig:apple_label}
\end{figure}

\medskip
\noindent
\textbf{Apple Privacy Label.}
The Apple Privacy Label (APL) is a four-level hierarchy (as shown in \cref{fig:apple_label}). The top level consists of four high-level privacy practices, known as \textit{Privacy Types}. The second level of the label discusses the purpose for data usage, while the third and fourth level describes high-level \textit{Data Categories} and fine-grained \textit{Datatypes}, respectively. In the top level, \textit{No Data Collected} denotes that the app does not collect any data from the users. 


Among the other three categories, \textit{Data used to Track you} covers the practices when user data is linked with third-party data for targeted advertising, Ad measurement, or sharing with a data broker. Notably, tracking does not apply when the data is never sent off the device in \textit{a way that can identify the user or device}, or if the data is used for fraud detection. \textit{Data linked to you} covers the personal information and data that is linked to the user's identity as opposed to \textit{Data not linked to you}.

The next level describes the purposes for which data collected in \textit{Data linked to you} and \textit{Data not linked to you} may be used. Apple defines five main purposes: \textit{Third party advertising and marketing}, \textit{Developers' advertising and marketing}, \textit{Analytics}, \textit{Product Personalization}, \textit{App Functionality} and \textit{Other Purposes}. It is important to note that \textit{Data Used to Track you} does not get a purpose level as its purpose is to track the users. In the \textit{Data Categories} level, Apple defines 14 categories of data such as \textit{Contact Info} (consisting of personal information), \textit{Health and Fitness}, \textit{Financial Info} etc. \textit{Data Categories} consists of the final level - \textit{DataTypes} which consists of 32 fine-grained datatypes that the developers can use, such as \textit{App Interactions, Precise Location, Contacts, Phone} etc. An illustrative example of APL is shown in \cref{fig:pl_example}. \smallskip


\noindent
\textbf{Google Data Safety Section}
\label{sec:data_safety}
\begin{figure}[t]
  \centering
  \includegraphics[width=\columnwidth]{figures/google_dss.pdf}
  \caption{Google Data Safety Section}
  \label{fig:google_dss}
\end{figure}
The Data Safety Section (DSS) also consists of four levels, where the first is high level \textit{Privacy Practices}. The second and third levels consist of \textit{Data Categories} and \textit{Data Types}, and the fourth level consists of \textit{Purpose}.

The first level includes three practices: \textit{Data Collection}, which covers the details about the data that is collected and its intended use; \textit{Data Sharing}, where the developers disclose what data is shared with third parties; and \textit{Security Practices} that covers the data practices related to user choice and data security. \textit{Security Practices} include three tags: \textit{Encrypted in Transit}, \textit{Data Deletion Option}, and \textit{Review against Global Security Standards}.


In the second level, \textit{Data Categories} includes 14 categories such as App Info and Performance and App Activity. Each \textit{Data Category} can also have \textit{Data Types}, which provide fine-grained information about the data used by the app. For example, \textit{App Activity} includes \textit{App Interactions and Installed App}, as shown in \cref{fig:google_dss}. The final level of the Data Safety Section consists of \textit{Purposes} that describe the reasons for collecting or sharing the data.

We note that even though the two privacy labels (APL and DSS) have some overlap at the lowest level, they cover different high-level practices. For instance, APL focuses on surfacing tracking practices and the linkability of the data. DSS focuses on data-centric practices, including collection, sharing, encryption, and deletion. In the rest of the paper, we will use APL and DSS to denote privacy labels for iOS apps and android apps, respectively. Further, we use the term \textit{Privacy Labels} to refer to both APL and DSS collectively.\smallskip

\noindent
\textbf{Usability of Privacy Labels.} Researchers have studied the usability of APLs from both users'~\cite{zhang2022usable} and developers'~\cite{li2022understanding} perspectives. Zhang et al.~\cite{zhang2022usable} studied 24 iPhone users to understand their experiences, understanding, and perceptions of privacy labels on the app store. They uncovered that users find the labels confusing with unfamiliar terms. From the developers' perspective, Li et al.~\cite{li2022understanding} interviewed 12 iOS developers and reported that the sources of errors by developers in privacy labels included both under-reporting and over-reporting data collection. They further concluded that the label design is generally confusing for the developers either due to known factors (lack of resources, improper documentation) or unknown factors (preconceptions, knowledge gaps). More recently, researchers also built and evaluated a tool~\cite{gardner2022helping} that helps iOS developers generate privacy labels by identifying data flows through code analysis. While these works focus on the usability evaluation of APL, our work compares the privacy practices present in privacy policies and labels.\smallskip

\noindent
\textbf{Studies on Privacy Labels.} 
Similar to our work, Xiao et al.~\cite{xiao2022lalaine} characterize non-compliance of apple privacy labels by studying data flow to label consistency of 5K iOS apps. They also provide insights for improving label design. This work is complementary to ours as we measure the consistency of privacy labels with the data practices mentioned in the apps' privacy policies.

The works most similar to ours perform longitudinal measurement of privacy labels to understand the adoption and evolution of apple privacy labels over time~\cite{balash2022longitudinal, li2022understanding, scoccia2022empirical}. In particular, Scoccia et al.~\cite{scoccia2022empirical} conducted an empirical study of 17K apps to characterize how sensitive data is collected and shared for iOS apps. They found that free apps collect more sensitive data for tracking purposes. Li et al.~\cite{li2022understanding} and Balash et al.~\cite{balash2022longitudinal} collected weekly snapshots of apple privacy labels and characterized the privacy practices mentioned in privacy labels for \nnumber{573k} apps. Balash et al.~\cite{balash2022longitudinal} also perform additional correlation analysis with app meta-data like user rating, content rating, and app size.

Our work is different in two ways. First, we provide complimentary analysis by analyzing privacy labels from Apple and Google to provide a comprehensive understanding of practices mentioned in APL and DSS. In doing so, we also verify their findings on how sensitive data is being collected and used. Second, we perform a consistency analysis of privacy labels with privacy policies. We also create a dataset with cross-listed apps on both platforms to understand how developers disclose their practices on different platforms. To the best of our knowledge, ours is the first work performing this analysis. \smallskip

\noindent
\textbf{Automated Privacy Policy Analysis.}
In 2016, Wilson et al.~\cite{wilson2016creation} introduced a privacy policy taxonomy along with an annotated dataset (OPP-115). The taxonomy covers privacy practices mentioned in the privacy policies of the websites. In the past few years, several works have trained classifiers using the taxonomy for automated policy analysis~\cite{harkous2018polisis, srinath2020privacy, wilson2016creation, mousavi2020establishing, wagner2022privacy}. Researchers have also used automated policy analysis to check for consistency within the policy~\cite{andow2019policylint, andow2020actions}, as well as consistency with the code~\cite{zimmeck2016automated, zimmeck2019maps}. Finally, automated analysis has also been used to study the impact of law and regulations on privacy policy~\cite{linden2018privacy, zaeem2020effect}. In this work, we extend the OPP-115 taxonomy to cover practices from nutrition labels and develop new classifiers to extract relevant privacy practices from the policies.




% Privacy nutrition labels are a way of showing how an organization’s data collection and use practices affect its users’ privacy. Similar to the nutrition labels which provide users with detailed information about the food products and their constituents, privacy nutrition labels are designed to give users a more nuanced understanding of how their data is being used than the existing notice and choice model, which has been shown to be ineffective~\cite{cranor2012necessary}.


%In an attempt to bring more transparency to data handling practices and promote user privacy, apple introduced Apple Privacy Label for the app store. Starting from December 8, 2020, the app developers are required to show data practices of the apps on an easy to read privacy label. 

% The privacy label will either have this category, or a combination of the other three. The apps having \textit{No Data Collected} is considered to have the best practices as it is not collecting any data. 



% On the app store, the APL has two views - the summary view contains the high level summary describes the \textit{Privacy Type} and the \textit{Data Categories} used whereas the detailed view contains the complete details including the \textit{Purposes} and the \textit{DataTypes}. 

% Specifically, it consists of three tags \textit{Encrypted in Transit} describing whether the datat collected is being encrypted during transit, \textit{Data Deletion Option} which conveys whether the users can request for their data to be deleted or not, and finally - \textit{Review against Global Security Standard} where the developers can indicate whether they have gone through an independent review to validate the apps' security practices against a global standard. 

%Researchers have also  analyzed the impact of apple privacy label on consumers~\cite{sarker2022consumer, kesler2022impact, bian2021supply}. In particular, Bian et al.~\cite{bian2021supply} studied the impact of addition of privacy labels by comparing weekly downloads for iOS apps and their android counterparts. They found a 14\% drop in downloads after the privacy labels were added. Kollnig et al.~\cite{kollnig2022goodbye} analyzed 1.7k iOS apps before and after the addition of privacy label. They focused on cross-device tracking apps and attempted to understand the impact of privacy labels on these apps. 

%These works build upon the OPP-115 taxonomy and the dataset. In contrast, in this work, we present a new privacy label centric dataset, and use it to  study inconsistencies between privacy label and the privacy policy for an app.

% \subsection{Related Works}

% \noindent
% \textbf{Privacy nutrition labels}
% ~\cite{kelley_labels} - Kelley et al, introducing privacy labels for websites to say what they are collecting etc
% ~\cite{kelley_privacy_app} - build on prev work to include mobile apps
% ~\cite{li2022understanding} looked into how app developers make errors in making labels.

% \noindent
% \textbf{Automated Privacy Policy Analysis}

% \noindent
% \textbf{App behavior analysis}


