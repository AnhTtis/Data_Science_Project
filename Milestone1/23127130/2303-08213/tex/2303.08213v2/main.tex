% \documentclass[letterpaper,twocolumn,10pt]{article}
% \documentclass[conference]{IEEEtran}
% \pagestyle{plain}
% \usepackage{usenix-2020-09}


% \documentclass[letterpaper,twocolumn,10pt]{article}
\documentclass[11pt]{article}
\usepackage[margin=1in]{geometry}
\usepackage{authblk}
\usepackage{times}
\usepackage[noadjust]{cite}
\usepackage[normalem]{ulem}
\usepackage[english]{babel}
\usepackage{blindtext}
\newcommand{\nnumber}[1]{\textcolor{black}{#1}}


\usepackage{lmodern}
\usepackage{booktabs}
\usepackage{url}
\usepackage{graphicx}
\usepackage{tikz}
\usepackage{amsmath}
\usepackage{epsfig,endnotes, graphicx,booktabs, multirow, array}
\usepackage{xspace}
\usepackage{colortbl}
% \usepackage{enumerate} 

\usepackage{tabularx}
% \usepackage[title]{appendix}
\usepackage[utf8]{inputenc}
% \usepackage[colorinlistoftodos,prependcaption,textsize=small]{todonotes}
\usepackage{regexpatch}
% \usepackage[]{caption} 
\usepackage[font=footnotesize]{caption} % abovecaptionskip=1pt
% \captionsetup[table]{font=footnotesize,labelfont=sc,textfont=sc}
\setlength{\belowcaptionskip}{-5pt}

\makeatletter
\newcommand{\linebreakand}{%
  \end{@IEEEauthorhalign}
  \hfill\mbox{}\par
  \mbox{}\hfill\begin{@IEEEauthorhalign}
}
\makeatother

\usepackage{algorithm}
\usepackage[noend]{algpseudocode}

% \usepackage{subcaption}
\usepackage{xspace}
\usepackage{hyperref}
\usepackage[textsize=small,backgroundcolor=orange]{todonotes}
\usepackage{booktabs}
% \usepackage[table,xcdraw]{xcolor}
\usepackage{graphicx}
\usepackage{enumitem}
\usepackage{multirow,tabularx}
\newcommand{\html}{{HTML}\xspace}
\newcommand{\dom}{{DOM}\xspace}
\newcommand{\css}{{CSS}\xspace}
\newcommand{\webpage}{{webpage}}
\renewcommand{\UrlFont}{\ttfamily\small}

\newcommand{\crawler}{\textit{Crawler}\xspace}
\newcommand{\result}[1]{\textcolor{black}{#1}}
\newcommand{\notes}[1]{\textcolor{blue}{#1}}
\newcommand{\generator}{\textit{Recipe-Generator}\xspace}
\newcommand{\uicomp}{\textit{UI-component}\xspace}
\newcommand{\eval}{{evaluation set}}
\newcommand{\inpagelink}{\textit{In-page} links\xspace}
\newcommand{\iscontrol}{\textit{Is-Control}\xspace}
% Some very useful LaTeX packages include:
% (uncomment the ones you want to load)
\definecolor{aliceblue}{rgb}{0.94, 0.97, 1.0}
\newcommand{\name}{\mbox{\textit{CookieEnforcer}}\xspace}
\newcommand{\bertbasecased}{BERT\textsubscript{Base-Cased}\xspace}

\newcommand{\insertref}[1]{\todo[color=green!40]{#1}}
\newcommand{\explainindetail}[1]{\todo[inline, color=red!40]{#1}}
\definecolor{aliceblue}{rgb}{0.94, 0.97, 1.0}
\newcommand{\red}[1]{\textcolor{red}{#1}}

\newcommand{\set}{5k\xspace}

\usepackage{tcolorbox}
\newtcolorbox{mybox}[1]{colback=aliceblue,colframe=black,fonttitle=\bfseries,title=#1}

\hypersetup{
    colorlinks=true,
    linkcolor=blue,
    filecolor=magenta,      
    urlcolor=cyan
    }
\RequirePackage[capitalise]{cleveref}
\usepackage{cleveref} 
%\setlist{nosep} % or 
% \setlist[itemize]{align=parleft,left=1pt..1em}
% \setlist[enumerate]{align=parleft,left=1pt..1em}
% *** MISC UTILITY PACKAGES ***
%
%\usepackage{ifpdf}
% Heiko Oberdiek's ifpdf.sty is very useful if you need conditional
% compilation based on whether the output is pdf or dvi.
% usage:
% \ifpdf
%   % pdf code
% \else
%   % dvi code
% \fi
% The latest version of ifpdf.sty can be obtained from:
% http://www.ctan.org/tex-archive/macros/latex/contrib/oberdiek/
% Also, note that IEEEtran.cls V1.7 and later provides a builtin
% \ifCLASSINFOpdf conditional that works the same way.
% When switching from latex to pdflatex and vice-versa, the compiler may
% have to be run twice to clear warning/error messages.






% *** CITATION PACKAGES ***
%
\usepackage{cite}
% \usepackage{breakurl}           % break too-long urls in refs
\usepackage{url}                % allow \url in bibtex for clickable links
\usepackage{xcolor}             % color definitions, to be use for...
% \usepackage[]{hyperref}         % ...clickable refs within pdf...
\hypersetup{                    % ...like so
  colorlinks,
  linkcolor={green!80!black},
  citecolor={red!70!black},
  urlcolor={blue!70!black}
}


\usepackage{mwe}
\usepackage{tikz}
\usetikzlibrary{arrows}
\usepackage{verbatim}

\usepackage{booktabs}
% \pagestyle{fancy}
% \fancyhf{}
% \rhead{Face Fairness Tentative Timeline}
% \lhead{Rosenberg and Tang}
\usepackage{dirtree}
\usepackage{wrapfig}
\hyphenation{op-tical net-works semi-conduc-tor}

\usepackage{caption}
% \usepackage{subcaption}

\title{\Large \bf The Overview of Privacy Labels and their Compatibility with Privacy Policies}

% \author{
%   Rishabh Khandelwal\\
%     University of Wisconsin--Madison\\
%     \texttt{rkhandelwal3@wisc.edu}
%   \and
%   Asmit Nayak\\
%   University of Wisconsin--Madison \\
%   \texttt{anayak6@wisc.edu} 
%     \and
%   Hamza Harkous \\ 
%   Google \\
%   \texttt{harkous@google.com}
%   \and
%   Kassem Fawaz \\
%   University of Wisconsin -- Madison \\
%   \texttt{kfawaz@wisc.edu}

% }

\renewcommand*{\Authsep}{, }
\renewcommand*{\Authand}{, }
\renewcommand*{\Authands}{, }
\renewcommand*{\Affilfont}{\normalsize\normalfont}
\renewcommand*{\Authfont}{\bfseries}    % make author names boldface    
\setlength{\affilsep}{2em}   % set the space between author and affiliation

\newsavebox\affbox

% \title{Aa Article Title}  

\usepackage{enumitem}
\usepackage{csquotes}
\usepackage{subfig}
\usepackage{tabularx}
\usepackage{amsmath}
\usepackage{makecell}
\usepackage{multirow}
% \usepackage{subfig}
% \colorlet{red}{darkgray}
\definecolor{aliceblue}{rgb}{0.94, 0.97, 1.0}
\usepackage{tikz}
\usepackage{amsmath}
\usepackage{xspace}
\usepackage{colortbl}
\usepackage{authblk}
\usepackage{tabularx}
\usepackage[title]{appendix}
\usepackage[utf8]{inputenc}
% \usepackage{hyperref}
\begin{document}
\newcommand*\samethanks[1][\value{footnote}]{\footnotemark[#1]}
\author{Rishabh Khandelwal\thanks{Equal Contribution}}
\author{Asmit Nayak\samethanks}
\author{Paul Chung}
\author{Kassem Fawaz}
\affil[]{%
  \savebox\affbox{\Affilfont Department of Chemical Engineering, University of AAAAA BBBBBB, CCCCC road,}%
  \parbox[t]{\wd\affbox}{\protect\centering} University of Wisconsin -- Madison} 
\date{} 








% \documentclass[11pt]{article}
% \usepackage[margin=1in]{geometry}
% \pagestyle{plain}
% \usepackage{times}
% \usepackage{tikz}
% \usepackage{amsmath}
% \usepackage{url}
% % \usepackage{filecontents}
% % \usepackage{citesort}
% \usepackage[noadjust]{cite}
% % \usepackage[dvipsnames]{xcolor}

% \definecolor{purple}{rgb}{1, 0, 1}

\newcommand{\ie}{\emph{i.e.,}\xspace}
\newcommand{\eg}{\emph{e.g.,}\xspace}
\newcommand{\abr}{\emph{abbr.}\xspace}
\newcommand{\ea}{\emph{et al.}\xspace}
\newcommand{\gensync}{\emph{GenSync}\xspace}
\newcommand{\colosseum}{\emph{Colosseum}\xspace}
\newcommand{\srep}{\emph{SREP}\xspace} % Set Reconciliation Enhances
\newcommand{\srepsim}{\emph{SREPSim}\xspace}
% Propagation
\newcommand{\esrep}{\emph{E-SREP}\xspace}
\newcommand{\epsrep}{\emph{EP-SREP}\xspace}
\newcommand{\mesrep}{\emph{ME-SREP}\xspace}
\newcommand{\mempoolsync}{\emph{MempoolSync}}

\newcommand{\fref}[1]{Fig.~\ref{#1}}
\newcommand{\tref}[1]{Table~\ref{#1}}
\newcommand{\aref}[1]{Algorithm~\ref{#1}}
\newcommand{\procref}[1]{Procedure~\ref{#1}}
\newcommand{\sref}[1]{Section~\ref{#1}}
\newcommand{\lineref}[1]{line~\ref{#1}}
\newcommand{\appref}[1]{Appendix~\ref{#1}}

% Change \eqref
\LetLtxMacro{\originaleqref}{\eqref}
\renewcommand{\eqref}{Eq.~\originaleqref}

% Theorems and corollaries
\newcounter{theoremcount}
\setcounter{theoremcount}{0}
\DeclareRobustCommand{\theorem}[1]{%
  \refstepcounter{theoremcount}%
  \noindent\textit{\textbf{Theorem \thetheoremcount\label{theorem:#1}: }}%
}
\DeclareRobustCommand{\theoremref}[1]{Theorem~\ref{theorem:#1}}

\DeclareRobustCommand{\proof}{\emph{Proof:}\xspace}
\DeclareRobustCommand{\qqed}{\hfill$\blacksquare$}

\newcounter{corollcount}
\setcounter{corollcount}{0}
\DeclareRobustCommand{\coroll}[1]{%
  \refstepcounter{corollcount}%
  \noindent\textit{\textbf{Corollary \thecorollcount\label{coroll:#1}: }}%
}
\DeclareRobustCommand{\corollref}[1]{Corollary~\ref{coroll:#1}}

\newcounter{lemmacount}
\setcounter{lemmacount}{0}
\DeclareRobustCommand{\lemma}[1]{%
  \refstepcounter{lemmacount}%
  \noindent\textit{\textbf{Lemma \thelemmacount\label{lemma:#1}: }}%
}
\DeclareRobustCommand{\lemmaref}[1]{Lemma~\ref{lemma:#1}}

\newcounter{definitioncount}
\setcounter{definitioncount}{0}
\DeclareRobustCommand{\definition}[1]{%
  \refstepcounter{definitioncount}%
  \noindent\textit{\textbf{Definition \thedefinitioncount\label{definition:#1}: }}%
}
\DeclareRobustCommand{\defref}[1]{Definition~\ref{definition:#1}}

%notes of different authors
\newif\ifnotes
\notestrue
\notesfalse

\newif\ifdiff
\difftrue
\difffalse

\newcommand{\anote}[1]{\ifnotes $\ll$\textsf{\textcolor{purple}{Ari: {#1}}}$\gg$ \fi}
\newcommand{\nnote}[1]{\ifnotes $\ll$\textsf{\textcolor{orange}{Novak: {#1}}}$\gg$ \fi}
\newcommand{\diff}[1]{\ifdiff\textcolor{orange}{#1}\else#1\fi}

%%% Local Variables:
%%% mode: latex
%%% TeX-master: "main"
%%% End:

% % \usepackage[]{apacite}
% % \usepackage{biblatex}
% \usepackage{amsthm}
% \usepackage[normalem]{ulem}
% % \theoremstyle{definition}
% \usepackage{enumitem}
% % \theoremstyle{plain}
% \usepackage[super]{nth}
% % \usepackage{breqn}
% \theoremstyle{remark}
% \newtheorem{definition}{Definition}[section]
% \newtheorem{theorem}[definition]{Theorem}



% % \newenvironment{statisticalTest}{\begin{proof}[Statistical Test]}{\end{proof}}
% \newtheorem{proposition}[definition]{Proposition}
% \newtheorem{lemma}[definition]{Lemma}
% \newtheorem{fact}{Fact}[subsection]

% % \theoremstyle{definition}
% \newtheorem{nullHyp}[definition]{Null Hypothesis}


% \newtheorem{remark}[definition]{Remark}
% \newtheorem*{statisticalTest}{Statistical Test}


% \usepackage[english]{babel}
% \usepackage{blindtext}


% % \bibliography{references}
% \usepackage{fancyhdr}
% \newcommand{\brian}[1]{{\color{cyan}#1}}
% \newcommand{\harrison}[1]{{\color{magenta}#1}}
% \newcommand{\kassem}[1]{{\color{violet}#1}}
% \newcommand{\somesh}[1]{{\color{olive}#1}}
% \newcommand{\varun}[1]{{\color{red}#1}}

% \makeatletter
% \newcommand{\linebreakand}{%
%   \end{@IEEEauthorhalign}
%   \hfill\mbox{}\par
%   \mbox{}\hfill\begin{@IEEEauthorhalign}
% }
% \makeatother

% \usepackage{algorithm}
% \usepackage[noend]{algpseudocode}
% % \usepackage[margin=1.25in]{geometry}
% \usepackage{mwe}
% \usepackage{tikz}
% \usetikzlibrary{arrows}
% \usepackage{verbatim}

% \usepackage{booktabs}
% % \pagestyle{fancy}
% % \fancyhf{}
% % \rhead{Face Fairness Tentative Timeline}
% % \lhead{Rosenberg and Tang}
% % \rfoot{Page \thepage}
% \usepackage{dirtree}
% \usepackage{wrapfig}

% \hyphenation{op-tical net-works semi-conduc-tor}

% \usepackage{caption}
% \usepackage{subcaption}

% % \renewcommand{\qedsymbol}{}

% \title{\Large \bf Fairness Properties of Face Recognition and Obfuscation Systems}



% \author{
% Harrison Rosenberg \\
% University of Wisconsin--Madison \\
% \texttt{hrosenberg@ece.wisc.edu}
% \and
% Brian Tang \\
% University of Michigan \\
% \texttt{bjaytang@umich.edu}
% \and
% Kassem Fawaz \\
% University of Wisconsin--Madison \\
% \texttt{kfawaz@wisc.edu}
% \and
% Somesh Jha \\ 
% University of Wisconsin--Madison \\
% \texttt{jha@cs.wisc.edu}
% }


% make title bold and 14 pt font (Latex default is non-bold, 16 pt)
% \author{
% {\rm Harrison Rosenberg}\\
% University of Wisconsin--Madison
% \and
% {\rm Brian Tang}\\
% University of Wisconsin--Madison
% \and
% {\rm Kassem Fawaz}\\
% University of Wisconsin--Madison
% \and
% {\rm Somesh Jha}\\
% University of Wisconsin--Madison
% copy the following lines to add more authors
% \and
% {\rm Name}\\
%Name Institution
% } % end author

% \author{\IEEEauthorblockN{Harrison Rosenberg}
% \IEEEauthorblockA{University of Wisconsin - Madison\\
% hrosenberg@ece.wisc.edu}
% \and
% \IEEEauthorblockN{Brian Tang}
% \IEEEauthorblockA{University of Wisconsin - Madison\\
% bjtang2@wisc.edu}
% \and
% \IEEEauthorblockN{Somesh Jha}
% \IEEEauthorblockA{University of Wisconsin - Madison\\
% jha@cs.wisc.edu}
% \linebreakand
% \IEEEauthorblockN{Kassem Fawaz}
% \IEEEauthorblockA{University of Wisconsin - Madison\\
% kfawaz@wisc.edu}}

% \IEEEoverridecommandlockouts
% \makeatletter\def\@IEEEpubidpullup{6.5\baselineskip}\makeatother
% \IEEEpubid{\parbox{\columnwidth}{
%     Network and Distributed Systems Security (NDSS) Symposium 2022\\
%     February 27 – March 3 2022\\
%     ISBN 1-891562-66-5\\
%     https://dx.doi.org/10.14722/ndss.2021.23xxx\\
%     www.ndss-symposium.org
% }
% \hspace{\columnsep}\makebox[\columnwidth]{}}


% make the title area
\maketitle

\section{Introduction}

The ability to reason about plans is critical for performing long-horizon tasks \citep{erol1996hierarchical, sohn2018hierarchical, sharma-etal-2022-skill}, compositional generalization \citep{corona-etal-2021-modular} and generalization to unseen tasks and environments \citep{shridhar2020alfred}.
Consider a simple long-horizon planning scenario where a robot is tasked with preparing a meal and serving it on the table. 
This presents a non-trivial planning problem since the agent needs to understand the sequence of operations required to perform the task and search for the relevant objects in the unfamiliar environment by interacting with various objects. %



Large language models have been recently shown to possess commonsense knowledge about the world such as object affordances and physical dynamics \citep{ouyang2022training,chowdhery2022palm}.
Early approaches considered text based environments and fine-tuned PLMs to predict actions given the history of past observations and actions \citep{jansen-2020-visually,micheli-fleuret-2021-language,yao-etal-2020-keep}.
Recent work has used this ability to reason about plans from text instructions in simulated household environments with simplifying assumptions such as text-only environment observations or feedback \citep{huang2022language,ahn2022can,li2022pre,logeswaran-etal-2022-shot}.


We focus on \emph{visually grounded planning} with PLMs --- the ability to adapt plans based on interaction and visual feedback from the environment.
While PLMs have strong planning commonsense priors, predictions from a PLM may not be directly realizable in the environment since the observation and action spaces are unknown.
This requires \emph{grounding} the PLM in the environment and adapting it to observe visual feedback, which is highly non-trivial.
Some prior works assume the availability of a pre-trained affordance function \citep{ahn2022can} or a success detector \citep{mirchandani2021ella}.
Notably, SayCan \citep{ahn2022can} completely decouples the PLM from observation information by selecting actions that have both high affordability (through a pre-trained affordance model) and high PLM likelihood.
Although this partially addresses the grounding problem, the use of visual feedback for action affordance alone is limited.
Often an agent must choose one of many affordable actions using information from observations.
For example, a driving agent should re-navigate and possibly turn around when encountering a ``road closed'' sign, but both turning around and driving forward are indistinguishable to SayCan because they are both affordable and the PLM is blind to observations.

Another workaround explored in prior work is translating the information in the visual observations to text using a pre-trained captioning system \citep{shridhar2021alfworld,huang2022language}.
However, it can be difficult to faithfully describe an image in words and information is lost in this inherently noisy process, which limits the information available to the planner.



Recent work shows that PLMs can be adapted for various natural language tasks by inserting tunable embeddings or soft prompts at the input of the PLM (also called prompt tuning or prefix tuning)~\citep{li-liang-2021-prefix,lester-etal-2021-power}.
This approach also extends to multi-modal understanding tasks such as image captioning \citep{mokady2021clipcap} and VQA \citep{tsimpoukelli2021multimodal} where images are encoded as soft prompts and finetuned for the target task.
Transformer based architectures have also been successfully applied to offline Reinforcement Learning in recent work \citep{chen2021decision,janner2021offline,li2022pre,reid2022can}.

Taking inspiration from these works, we propose the simple approach of embedding visual observations (`visual prompts') and \textit{directly inserting them as PLM input embeddings}.
The visual encoder and PLM are jointly trained for the target task, an approach we call \textbf{\oursfull}~(\ours).
By teaching the PLM to use observations for planning in an end to end manner, we remove the dependency on external data such as captions and affordability information that was used in prior work.
We show that this simple approach performs better than prior PLM-based planning approaches on two embodied planning benchmarks based on ALFWorld~\citep{shridhar2021alfworld} and Virtualhome~\cite{puig2018virtualhome}.



\section{Related work}
\noindent \textbf{Implict Neural Representation}.
Implicit neural representations (also known as coordinate-based representations) are a popular way to parameterize content of all kinds, such as audio, images, video, or 3D scenes~\cite{FFL, siren, srn, NeRF}.
Recent works \cite{NeRF, DeepSDF, occnet, srn} build neural implicit fields for geometric reconstruction and novel view synthesis achieving outstanding performance.
The implicit neural representation is continuous, resolution-independent, and expressive, and is capable of reconstructing geometric surface details and rendering photo-realistic images. 
%
While explicit representations like point clouds\cite{points1, NHR}, meshes\cite{NT}, and voxel grids\cite{deepvoxels, occnet, NeuralVolume, voxel1} are usually limited in resolution due to memory and topology restrictions.
%
One of the most popular implicit representations - Neural Radiance Field (NeRF) \cite{NeRF} -  proposes to combine the neural radiance field with differentiable volume for photo-realistic novel views rendering of static scenes. However, NeRF requires optimizing the 5D neural radiance field for each scene individually, which usually takes hours to converge. Recent works\cite{PixelNeRF, ibrnet, MVSNeRF} try to extend NeRF to generalization with sparse input views.
%
In this work, we extend the neural radiance field to a general human reconstruction scenario by introducing conditional geometric code and appearance code. 


\noindent \textbf{3D Model-based Human Reconstruction}
With the emergence of human parametric models like SMPL\cite{SMPL,SMPLX} and SCAPE\cite{SCAPE}, many model-based 3D human reconstruction works have attracted wide attention from academics. Benefiting from the statistical human prior, some works\cite{tex2shape, Multi-Garment, expose, VIBE} can reconstruct the rough geometry from a single image or video. 
However, limited by the low resolution and fixed topology of statistical models, these methods cannot represent arbitrary body geometry, such as clothing, hair, and other details well. 
To address this problem, some works\cite{PIFu, pifuhd} propose to use pixel-aligned features together with neural implicit fields to represent the 3D human body, but still have poor generalization for unseen poses. To alleviate such generalization issues, \cite{pamir, arch, doublefield} incorporate the human statistical model SMPL\cite{SMPL, SMPLX} into the implicit neural field as a geometric prior, which improves the performance on unseen poses. 
Although these methods have achieved stunning performance on human reconstruction, high-quality 3D scanned meshes are required as supervision, which is expensive to acquire in real scenarios. Therefore, prior works\cite{PIFu, pifuhd, pamir, arch} are usually trained on synthetic datasets and have poor generalizability to real scenarios due to domain gaps. To alleviate this limitation, 
some works\cite{neuralbody, Anim-NeRF, animnerf_zju, humannerf, arah, a-nerf}  combine neural radiance fields\cite{NeRF} with SMPL\cite{SMPL} to represent the human body, which can be rendered to 2D images by differentiable rendering. 
Currently, some works\cite{gpnerf, genebody, NHP, keypointNeRF, doublediffuse, doublefield} can quickly create neural human radiance fields from sparse multi-view images without optimization from scratch.
While these methods usually rely on accurate SMPL estimation which is not always applicable in practical applications. 
% We introduce a xxx

% identity-specific models, like NeuralBody\cite{neuralbody}
% generalizable models, 
% SMPL\cite{SMPL}, SMPLX\cite{SMPLX}, SCAPE\cite{SCAPE}, Tex2Shape\cite{tex2shape}, Multi-Garment Net\cite{Multi-Garment}, VIBE\cite{VIBE}, Expose\cite{expose}, NeuralBody\cite{neuralbody}, Anim-NeRF\cite{animnerf_zju, animnerf}, Neural Actor\cite{neuralactor}, SelfRecon\cite{selfrecon}, HumanNeRF\cite{humannerf}, PIFu\cite{PIFu}, PIFuHD\cite{pifuhd}, Pamir\cite{pamir}, Arch\cite{arch}, Double Field\cite{doublefield}, GNR\cite{genebody}, NHP\cite{NHP}, GPNeRF\cite{gpnerf}, KeypointNeRF\cite{keypointNeRF}, DoubleDiffuse\cite{doublediffuse}











\begin{figure*}[ht]
    \centering
    \vspace{-1em}
    \includegraphics[width=1.0\linewidth]{figures/method/pipeline.png}
    \vspace{-1.5em}
    \caption{\textbf{The architecture of our method}. Given $m$ calibrated multi-view images and registered SMPL, we build the generalizable model-based neural human radiance field. First, we utilize the image encoder to extract multi-view image features, which are used to provide geometric and appearance information, respectively. In order to adequately exploit the geometric prior, we propose the visibility-based attention mechanism to construct a structured geometric body embedding, which is further diffused to form a geometric feature volume. For any spatial point $\mathbf{x}$, we trilinearly interpolate the feature volume $\mathcal{G}$ to obtain the geometric code $\mathbf{g}(\mathbf{x})$. In addition, we also propose geometry-guided attention to obtain the appearance code $\mathbf{a}(\mathbf{x}, \mathbf{d})$ directly from the multi-view image features. We then feed the geometric code $\mathbf{g}(\mathbf{x})$ and appearance code $\mathbf{a}(\mathbf{x}, \mathbf{d})$ into the MLP network to build the neural feature field $(\mathbf{f}, \sigma) = F(\mathbf{g}(\mathbf{x}), \mathbf{a}(\mathbf{x}, \mathbf{d}))$. Finally, we employ volume rendering and neural rendering to generate the novel view image.
    % \Liqian{1) Add section ref. 2) add detailed caption. 3) Modulate the fig, each module corresponds to a sub-section. 4) keep fig text  consistent with method text}
    }
    \vspace{-1em}
    \label{fig:architecture}
\end{figure*}

% \input{src/03_nutrition_labels.tex}
\begin{figure*}[t!]
\includegraphics[width=1.0\linewidth, trim={0 0.3cm 0 0.1cm}, clip]{figures/architecture/architecture.pdf}
\vspace{-15pt}
\caption{
\textbf{Point2Vec pre-training.}
Our model divides the input point cloud into %
point patches using farthest point sampling (FPS) and $k$-NN aggregation.
We obtain patch embeddings by applying a mini-PointNet\,\colorsquare{m_pointnet} to each point patch (\emph{right}).
The teacher Transformer encoder\,\colorsquare{m_green} infers a contextualized %
representation for all patch embeddings which, after normalization and averaging over the last $K$ Transformer layers, serve as training targets.
The student's input is a masked view on the input data, \ie we randomly mask out a ratio of patch embeddings and only pass the remaining embeddings into the student Transformer encoder\,\colorsquare{m_blue}.
After applying a shallow decoder\,\colorsquare{m_red} on the outputs of the student, padded with learned mask embeddings\,\protect\maskembedding{}, we train the student and decoder to predict the latent teacher representation of the patch embeddings.
\vspace{-10pt}
}
\label{fig:model}
\end{figure*}
\section{Method}

The aim of this work is to unlock the full potential of data2vec-like\,\cite{baevski2022data2vec} pre-training on point clouds by addressing point cloud specific challenges.
To achieve this, we first summarize the technical concepts of data2vec (\refsec{method_d2v}) and show how to learn rich representations on point clouds using data2vec pre-training (\refsec{method_d2v_pcl}).
Finally, we propose \name{}, which accounts for the point cloud specific limitations of data2vec (\refsec{method_p2v}).

\subsection{Data2vec}\label{sec:method_d2v}
Data2vec\,\cite{baevski2022data2vec} is designed to pre-train Transformer-based models, which involve a feature encoder that maps the input data to a sequence of embeddings.
These embeddings are subsequently passed to a standard Transformer encoder to generate the final latent representations.
During pre-training, two versions of the Transformer encoder are kept: a \emph{student} and a \emph{teacher}.
The teacher is a momentum encoder, \ie its parameters $\Delta$ track the student's parameters $\theta$ by being updated after each training step according to an exponential moving average (EMA) rule\,\cite{caron2021dino, baevski2022data2vec, grill2020BYOL, he2020moco}: $\Delta \leftarrow \tau \Delta + (1-\tau)\theta$,
where $\tau \in [0,1]$ is the EMA decay rate.
The teacher provides the training targets, which the student predicts given a corrupted version of the same input.

In a first step, the teacher encodes the uncorrupted input sequence.
The training targets are then constructed by averaging the outputs of the last $K$ blocks of the teacher, which are normalized beforehand to prevent a single block from dominating the sum.
Due to the self-attention layers, these targets are \emph{contextualized}, \ie they incorporate global information from the whole input sequence.
This is an important difference to other masked-prediction methods such as BERT\,\cite{devlin2018bert} and MAE\,\cite{he2022mae}, where the targets only comprise local information, \eg a word or an image patch. %

The student is given a masked version of the same input, where some of the embeddings in the input sequence are substituted by a special learned \emph{mask embedding}. %
The student's task is to predict the targets corresponding to the masked parts of the input.
The model is trained by optimizing a Smooth L1 loss on the regressed targets. %







\subsection{Data2vec for Point Clouds}\label{sec:method_d2v_pcl}

To apply data2vec to point clouds, we utilize the same underlying model as Point-BERT\,\cite{yu2021pointbert} and Point-MAE\,\cite{pang2022pointmae}.
This model is well suited for data2vec pre-training: it extracts a sequence of patch embeddings from the input point cloud and feeds it to a standard Transformer encoder.
For downstream tasks, we append a task-specific head to the Transformer encoder (\refsec{experiments}).
Next, we describe the point cloud embedding and the Transformer in detail and conclude with a summary of data2vec for point clouds.


\parag{Point Cloud Embedding.}
First, we sample $n$ center points from the input point cloud using farthest point sampling (FPS)\,\cite{qi2017pointnetplusplus}.
Grouping the center points' $k$-nearest neighbors ($k$-NN) in the point cloud yields $n$ contiguous \emph{point patches}, \ie sub-clouds of $k$ elements.
Next, we normalize the point patches by subtracting the corresponding center point from the patch's points.
This untangles the positional and the structural information.
To account for the permutation-invariant property of point clouds, we employ a mini-PointNet\,\cite{qi2016pointnet} (\reffig{model}, \emph{right}) that maps each normalized point patch to a \emph{patch embedding}.

The mini-PointNet involves the following steps:
First, we map each point of a patch to a feature vector using a shared MLP.
Then, we concatenate max-pooled features to each feature vector.
The resulting feature vectors are then passed through a second shared MLP and a final max-pooling layer to obtain the patch embedding.

\paragraph{Transformer Encoder.}
The central component of the model is a standard Transformer encoder.
The patch embeddings form the input sequence to the Transformer encoder.
Since the point patches are normalized, the patch embeddings carry no positional information;
therefore, a two-layer MLP maps each center point to a position embedding, which is then added to the corresponding patch embedding.
Due to the special importance of positional information in point clouds, the position embeddings are added again before each subsequent Transformer block to ensure that the positional information is incorporated at every step of the encoding process.

\paragraph{\emakefirstuc{\datavec{}}.}

To establish a baseline, we apply the unmodified data2vec approach to the previously described underlying model of Point-BERT and Point-MAE.
Going forward, we will refer to this approach as \datavec{}.


\subsection{\emakefirstuc{\name{}}}\label{sec:method_p2v}
In \reffig{model}, we present the complete pipeline of our \name{} model.
Directly applying data2vec to point cloud data without modifications is not optimal, as the position embeddings are also added to the mask embeddings, revealing the overall shape of the point cloud to the student.
As positions are the only features for point clouds, this makes the masking far less effective, as noted by Pang \etal \cite{pang2022pointmae} in the context of masked autoencoders.

To solve this issue, we adopt an approach inspired by MAE\,\cite{he2022mae}, where we only feed the non-masked embeddings to the student\,\colorsquare{m_blue}.
A separate decoder\,\colorsquare{m_red}, implemented as a shallow Transformer encoder, takes the output of the student and the previously held-back masked embeddings\,\maskembedding{} as input and predicts the training targets.
In contrast to \datavec{}, this approach does not suffer from leaking positional information from the masked-out point patches to the student.
Moreover, utilizing an MAE-inspired setup provides additional benefits:
First, the student is more computationally efficient, as it only needs to process the non-masked embeddings.
Second, the model's inputs during fine-tuning are more similar to those during pre-training because the inputs during pre-training are no longer dominated by masked embeddings which are absent during fine-tuning.
This likely makes the learned representations more transferable to downstream tasks.

% \section{Dataset Creation}\label{sec:datasets}
We curated three large scale datasets to understand the  landscape of privacy nutrition labels and their consistency with privacy policies: (1) Privacy label dataset for apps on play store(n=2.6M), (2) Privacy label dataset for apps on apple store (n=1.6M), (3) Privacy policies for the apps present on both the platforms (n=1.5M). From hereon, we will refer to (1) as Data Safety Section (DSS) Dataset and (2) as Apple Privacy Label (APL) dataset. The overview of the data collection pipelines is shown in Figure~\ref{fig:methodology}. The policy dataset size is lower than the number of apps because (a) multiple apps can have the same developer and list the same privacy policy, (b) some apps may use generic policy templates hosted by third party websites, and (c) not all apps have a working privacy policy URL. Further, we note that we only work with policies and privacy labels that are present in the English Language. 


\begin{figure}[t]
  \centering
  \includegraphics[width=\columnwidth]{figures/methodology.pdf}
  \caption{Overview of methodology}
  \label{fig:methodology}
\end{figure}


\subsection{Google Data Safety Section}
Google required app developers to complete the data safety section by July 20, 2022. Starting from Jun 20, 2022, we collected periodic snapshots of  the data safety sections of all the apps present on the play store. The final snapshot was collected on Sept 9, 2022.
To collect the data safety section, we start with the APK list provided by Androzoo~\cite{androzoo}. This APK list consists of android app ids from various sources such as google play store, amazon fire store etc. This list is updated on a nightly basis, that ensures that we get the most up to date list of apps present on the play store. We use the list to extract the package names for the apps. The package names are unique and can be used to construct the apps' page on the play store. For example, the URL for any app can be constructed by appending the package name to ``\url{https://play.google.com/store/apps/details?id=}''. Additionally, to filter out the apps that not available on play store, we perform a validation step by checking the existence of the URL constructed using the package name.

After obtaining the valid package names, we use a customized version of publicly available google play store scraper library,\newline~\texttt{google-play-scraper} ~\cite{}. The customization\footnote{Note that the library was later updated in August to include the data safety section} was required to capture the data safety section, along with the app meta data. Specifically, the response object contains the practices mentioned in the data safety section as well as app name, number of installs, privacy policy URL, app developers' website etc.  We store the privacy policy URL for each app separately which is later used to create the privacy policy dataset(discussed below). Finally, we store the parsed data in a database for further analysis. 

 The total time to retrieve data for 2.3M apps is between 24 to 48 hours using 4 local machines. We note here that Google required app developers to complete Data Safety Section by July 20, 2022. By collecting data before and after the deadline, we are able to capture developers' practice before- and after- the required date which allows us to study the addition and evolution of data safety section at scale.

% \subsubsection{Methodology}
% % Google Play Store doesn't have an API that can be used to get the list of all the apps in the Play Store. So, we used the Androzoo[] android apk list to extract the package names of the API. All apps on the Google Play website are referred to as https://play.google.com/store/apps/details?id=<APK Package>. This helped us to scrape the Play Store using a modified google-play-scraper to parse both the app details, app permissions and the data safety section of an app.

% % Using these package names we could navigate to the app listing on the Google Play website. 
% \paragraph*{\textbf{App Selection}}
% We collected an initial list of 2.2 million app ids from the Androzoo APK list~\cite{}. We then used a custom modified version of google-play-scraper~\cite{} to parse the metadata along with the Data Safety section of the apps from the Google Play website. For every apps in our list, we collected metadata, permissions list, and a parsed version of the data safety section. 

% We started collecting data on June 20th and each week we updated our apk list and re-downloaded the app details and the corresponding data safety portions. We continued to do this until August when all Play Store apps were expected to have data safety sections. Lastly, we scraped again in September for to get a final set of apps.

% % \begin{figure}[t]
% %   \centering
% %   \includegraphics[width=\columnwidth]{figures/apple_methodology.pdf}
% %   \caption{Apple Methodology}
% %   \label{fig:apple_methodology}
% % \end{figure}

% While we started with 2.2M apps by September we had over 2.5M. 

% We would parse the data collected at the end of the weekly scraping. The data was parsed into two data files. One corresponds to the apps that have a Data Safety Section and the other corresponded to the latest data of all apps scraped till then. We parsed data into the following categories app name; developer name; developer website; downloads; privacy policy; genre type; data collected; the purpose for each type of data being collected; data shared; the purpose of data being shared.


\subsection{Apple Privacy Label} \label{apple_label_methodology}
For the Apple privacy label, we collected our data on November 13, 2022. We first start by parsing the XML site map for the app store using the library \texttt{ultimate-sitemap-parser}~\cite{}. The sitemap consists of the \textit{appIds} and unique URLs for the apps present in the app store. In our parsing, we found that the unique number of apps on the App store was 1.6M. Using the unique appIds, we call the Apple Store Catalogue API~\cite{} to extract the metadata and the privacy nutrition label for the app. The API response is a JSON string that contains the privacy nutrition label, including the privacy types, data categories, data types, and purpose. The JSON response also contains the associated metadata of the app such as the developer website, app name, privacy policy URL, genre, etc. In addition to the API calls, we used the unique URLs to perform headless crawls to extract the privacy policy link shown on the app store page. The privacy policy is then extracted (as described below) for further analysis. 

We leverage cloud functions to perform the API calls to extract the privacy nutrition labels. The API call can occasionally result in errors if too many requests are sent. To avoid overwhelming the servers, we wait for 2 hours before retrying the failed instances. Using 11 instances of google cloud functions, and the entire scraping process finished in 14-16 hours. From these apps, we were able to analyze 1.38M apps; the rest were filtered out due to non-English content. At the end of the scrape, we had 955K apps with Apple Privacy Label out of 1.38M apps in our set (69.2\%). We also obtained 750k Unique policy links from our the apps. Note that the developers can use generic privacy policy templates hosted by third party websites to fulfil the requirements. 

\subsection{Privacy Policy}
We build the privacy policy dataset using the privacy policy URLs extracted during privacy labels scraping, as described above. Using the URLs, we retrieve the raw HTML of the policies and perform text extraction  and text cleaning. We then pass it to the is-policy classifier which determines whether the webpage is a privacy policy. We note here that although 
both the platform require the app developers to add privacy policy link with the app, in many cases the privacy policy URL is either not available or is not valid. 

\subsubsection*{\textbf{Text Extraction and Cleaning}} Starting with the privacy policy URLs, we use the \texttt{requests} library to obtain the raw HTML of the page. Next, we clean the HTML using the Boilerpipe~\cite{} library which removes the headers and footers from the web page. We then extract the text using the BeautifulSoup~\cite{} library. Following prior works~\cite{}, we filter out the instances where the text is smaller than 100 words as they are more likely to be error messages than privacy policies. We note here that some websites may store policy in PDF files. We use the PyPDF2~\cite{} library to extract text from the PDF files. We filter out the instances where the policy is stored in other non-standard formats such as google documents. The non-standard nature of these documents make it challenging to extract text from these documents; for example, in some cases, we are required to request access to view the documents. In our dataset, we find that roughly 3\% of the apps have privacy policies as google documents. While filtering these policies affects our coverage, it doesn't affect the overall results. Finally, as our analysis pipeline relies on the English language, we also filter out non-English policies using the language detection library langid.py~\cite{}. 

\subsubsection*{\textbf{Is-Policy Classification}} After extracting text from the privacy policies, we pass it through a binary classifier which determines whether the text corresponds to a privacy policy. The classifier is a 1-D  Convolutional Neural Network(CNN), based on classifiers mentioned in the prior works~\cite{TOPS, GDPR}, achieving an accuracy of 99\% on the testing set. It was trained with the same corpus of 1000 privacy policies as in ~\cite{}. More details about the training are present in Appendix~\ref{}. 


On the app store, we found that 1.01M apps out of 1.38M apps had 713K unique privacy policy URLs. 
% ZZZ apps had non-unique privacy policy.
On the play store, we obtained 1.12M unique privacy policy URLs. In total, we had 1.3M unique privacy policy URLs. 
\todo[inline]{Add in the numbers from the pipeline here}


\subsection{Ethical Considerations}
For our datasets, we collected data only from publicly available web pages and APIs. While our data collection scripts add additional load to Google and Apple's servers, we were careful to not abuse these resources. In particular, we added back-off strategies in case of errors and waited for sufficient time before retrying for the failed cases. Furthermore, for privacy policy extraction, we were respectful of robots.txt and only extracted HTML when the website allowed us to.
% \section{Method}
% \label{sec:method}

\begin{figure*}[t]
\centering
\includegraphics[width=\linewidth]{figures/framework_v3.pdf}
\vspace{-5mm}
\caption{An overview of our 3D-CLR framework. First, we learn a 3D compact scene representation from multi-view images using neural fields (I). Second, we use CLIP-LSeg model to get per-pixel 2D features (II). We utilize a 3D-2D alignment loss to assign features to the 3D compact representation (III). By calculating the dot-product attention between the 3D per-point features and CLIP language embeddings, we could get the concept grounding in 3D (IV). Finally, the reasoning process is performed via a set of neural reasoning operators, such as \textsc{Filter}, \textsc{Get\_Instance} and \textsc{Count\_Relation} (V). Relation operators are learned via relation networks.}
\vspace{-5mm}
\label{fig:framework}
\end{figure*}

Fig.~\ref{fig:framework} illustrates an overview of our framework. Specifically, our framework consists of three steps.  First, we learn a 3D compact representation from multi-view images using neural field. And then we propose to leverage pre-trained 2D vision-and-language model to ground concepts on 3D space. This is achieved by 1) generating 2D pixel features using CLIP-LSeg; 2) aligning the features of 3D voxel grid and 2D pixel features from CLIP- LSeg~\cite{li2022language}; 3) dot-product attention between the 3D features and CLIP language features~\cite{li2022language}. Finally, to perform visual reasoning, we propose neural reasoning operators, which execute the question step by step on the 3D compact representation and outputs a final answer. For example, we use \textsc{Filter} operators to ground semantic concepts on the 3D representation, \textsc{Get\_Instance} to get all instances of a semantic class, and \textsc{Count\_Relation} to count how many pairs of the two semantic classes have the queried relation.
% \gc{metion all the neural operators.}
% Works from linguistic and cognitive science suggest that semantic concepts are diverse and open-vocabulary, while relational concepts describing 3D objects' relationships can be very limited and thus can be considered a close-class vocabulary \cite{Landau1993WhatA, Hayward1995SpatialLA}. Therefore, it's unrealistic to learn the embeddings of all the concepts in the question-answering pairs, while it's more natural to learn the relation embeddings. Inspired by this, we propose to leverage 2D pretrained vision-language model (\textit{i.e.,} CLIP) for open-vocabulary semantic concept learning, while proposing a neural relation module network for relational reasoning. 

\subsection{Learning 3D Compact Scene Representations}

% Since 3D-related reasoning works on 3D compact representations rather than 2D images, we first propose to use a neural field to extract 3D representations from multi-view images. The next step is to learn the 3D features for visual reasoning. However, 3D assets are limited in diversity and scale, posing challenges for training large-scale 3D foundation models, while there's much progress on large-scale 2D pretrained models which provide decent features\cite{Radford2021LearningTV, Ramesh2021ZeroShotTG}. Since neural field maps a 2D pixel to several 3D points along the ray, it's natural to get 3D features for 2D per-pixel features. We apply CLIP-LSeg\cite{Li2022LanguagedrivenSS} to learn per-xel 2D features, and use an alignment loss to align 3D features with 2D features.

% \paragraph{3D Compact Representation from neural field.} 
Neural radiance fields  \cite{mildenhall2020nerf} are capable of learning a 3D representation that can reconstruct a volumetric 3D scene representation from a set of images. Voxel-based methods \cite{Garbin2021FastNeRFHN, Hedman2021BakingNR, Yu2021PlenOctreesFR, Sun2022DirectVG} speed up the learning process by explicitly storing the scene properties (\textit{e.g.}, density, color and feature) in its voxel grids. We leverage Direct Voxel Grid Optimization (DVGO) \cite{Sun2022DirectVG} as our backbone for 3D compact representation for its fast speed. DVGO stores the learned density and color properties in its grid cells. The rendering of multi-view images is by interpolating through the voxel grids to get the density and color for each sampled point along each sampled ray, and integrating the colors based on the rendering alpha weights calculated from densities according to quadrature rule \cite{Max1995OpticalMF}. The model is trained by minimizing the L2 loss between the rendered multi-view images and the ground-truth multi-view images. By extracting the density voxel grid, we can get the 3D compact representation (\textit{e.g.,} By visualizing points with density greater than 0.5, we can get the 3D representation as shown in Fig. \ref{fig:framework} I. ) 

\subsection{3D Semantic Concept Grounding}
Once we extract the 3D compact representation of the scene, we need to ground the semantic concepts for reasoning from language. 
Recent work from \cite{hong20223d} has proposed to ground concepts from paired 3D assets and question-answers. Though promising results have been achieved on synthetic data, it is not feasible for open-vocabulary 3D reasoning in real-world data, since it is hard to collect large-scale 3D vision-and-language paired data.  To address this challenge, our idea is to leverage  pre-trained 2D vision and language model \cite{Radford2021LearningTV, Ramesh2021ZeroShotTG} for 3D concept grounding in real-world scenes.  But how can we map 2D concepts into 3D neural field representations? Note that 3D compact representations can be learned from 2D multi-view images and that each 2D pixel actually corresponds to several 3D points along the ray. Therefore, it's possible to get 3D features from 2D per-pixel features. Inspired by this, we first add a feature voxel grid representation to DVGO, in addition to density and color, to represent 3D features. 
% it's natural to utilize 2D VLMs to ground semantic concepts on the 3D representations. 
 We then apply CLIP-LSeg\cite{li2022language} to learn per-pixel 2D features, which can be attended to by CLIP concept embeddings. We use an alignment loss to align 3D features with 2D features so that we can perform concept grounding on the 3D representations.
% Since 3D voxel grids and 2D pixels are aligned via alpha compositing, we add one L1 loss to force the features of 3D voxel grids to align with the 2D LSeg pixels based on the alpha values. 

\noindent\textbf{2D Feature Extraction.}
To get per-pixel features that can be attended by concept embeddings, we use the features from language-driven semantic segmentation (CLIP-LSeg) \cite{li2022language}, which learns 2D per-pixel features from a pre-trained vision-language model (\textit{i.e.,} \cite{Radford2021LearningTV}). Specifically, it
uses the text encoder from CLIP, trains an image encoder to produce an embedding vector for each pixel, and calculates the scores of word-pixel correlation by dot-product. By outputting the semantic class with the maximum score of each pixel, CLIP-LSeg is able to perform zero-shot 2D semantic segmentation.

\noindent\textbf{3D-2D Alignment.}
In addition to density and color, we also store a 512-dim feature in each grid cell in the compact representation. To align the 3D per-point features with 2D per-pixel features, we calculate an L1 loss between each pixel and each 3D point sampled on the ray of the pixel. The overall L1 loss along a ray is the weighted sum of all the pixel-point alignment losses, with weights same as the rendering weights: $\mathcal{L}_{\text {feature}}=\sum_{i=1}^K w_i(\|\boldsymbol{f_i}-F(\boldsymbol{r})\|),$
where $\boldsymbol{r}$ is a ray corresponding to a 2D pixel, $F(\boldsymbol{r})$ is the 2D feature from CLIP-LSeg, $K$ is the total number of sampled points along the ray and $\boldsymbol{f_i}$ is the feature of point $i$ by interpolating through the feature voxel grid, $w_i$ is the rendering weight.
% \gc{add equations.} 

\noindent\textbf{Concept Grounding through Attention.}  Since our feature voxel grid representation is learnt from CLIP-LSeg, by calculating the dot-product attention $<\boldsymbol{f}, \boldsymbol{v}> $ between per-point 3D feature $\boldsymbol{f}$ and the CLIP concept embeddings $\boldsymbol{v}$, we can get zero-shot view-independent concept grounding and semantic segmentations in the 3D representation, as is presented in Fig. \ref{fig:framework} IV. 
% \gc{add equations.}

\subsection{Neural Reasoning Operators}
Finally, we use the grounded semantic concepts for 3D reasoning from language. We first transform questions into a sequence of operators that can be executed on the 3D representation for reasoning. We adopt a LSTM-based semantic parser   \cite{Yi2018NeuralSymbolicVD} for that. As \cite{Mao2019TheNC, hong20223d}, we further devise a set of operators which can be executed on the 3D representation.  Please refer to \textbf{Appendix} for a full list of operators.

\noindent\textbf{Filter Operators.}  We filter all the grid cells with a certain semantic concept.

\noindent\textbf{Get\_Instance Operators.} We implement this by utilizing DBSCAN \cite{Ester1996ADA}, an unsupervised algorithm which assigns clusters to a set of points. Specifically, given a set of points in the 3D space, it can group together the points that are closely packed together for instance segmentation.

\noindent\textbf{Relation Operators.} We cannot directly execute the relation on the 3D representation as we have not grounded relations. Thus, we represent each relation using a distinct neural module (which is practical as the vocabulary of relations is limited \cite{Landau1993WhatA}). We first concatenate the voxel grid representations of all the referred objects and feed them into the relation network.
% \yd{Do we do something afterwards -- we first concatenate, then what?} 
The relation network consists of three 3D convolutional layers and then three 3D deconvolutional layers. A score is output by the relation network indicating whether the objects have the relationship or not. Since vanilla 3D CNNs are very slow, we use Sparse Convolution \cite{spconv2022} instead. Based on the relations asked in the questions, different relation modules are chosen. 

% \subsection{Learning 3D Compact Representation}
% In recent years, neural field models(\textit{e.g.,} \cite{mildenhall2020nerf}) have gained much popularity since they can reconstruct a volumetric 3D scene representation from a set of images. Recent works \cite{Garbin2021FastNeRFHN, Hedman2021BakingNR, Yu2021PlenOctreesFR, Sun2022DirectVG} have pushed it further by using classic voxel-grids to explicitly store the scene properties (\textit{e.g.}, density, color and feature) for rendering, which allows for real-time rendering. Since concept grounding and relation learning are expected to work on the per-point features in the 3D space \cite{hong20223d} of thousands of scenes, it's more suitable to use voxel-grid-based methods since they store explicit properties in each point which can be directly used for reasoning, and super-fast convergence makes it feasible to train thousands of scenes. Specifically, we use the fine reconstruction process of Direct Voxel Grid Optimization \cite{Sun2022DirectVG} as our backbone for 3D compact representation for its fast speed. 

% A compact voxel-grid representation models the modalities of interest (\textit{e.g.,} density, color or feature) explicitly in its grid cells. To query the properties at any given 3D point, interpolation is used:
% \begin{equation}
% \operatorname{interp}(\boldsymbol{x}, \boldsymbol{V}):\left(\mathbb{R}^3, \mathbb{R}^{C \times N_x \times N_y \times N_z}\right) \rightarrow \mathbb{R}^C
% \end{equation}
% where $\boldsymbol{x}$ is the queried 3D point,  $\boldsymbol{V}$ is the voxel grid, and $C$ is
% the dimension of one of the modalities, and $N_x, N_Y, N_z$ is the number of voxels. We first predict the density of a specified point by interpolating the density grid. This is crucial for the geometric reconstruction of the scene.  
% \begin{equation}
% \sigma=\operatorname{interp}\left(\boldsymbol{x}, \boldsymbol{V}^{(\text {density })}\right)
% \end{equation}
% where $\sigma$ is the volume density at position $\boldsymbol{x}$. For the modeling of color emission, we use an explicit-implicit hybrid representation where  a shallow MLP is placed after the color voxel grid interpolation process:
% \begin{equation}
% \boldsymbol{c}=\operatorname{MLP}^{(\mathrm{rgb})}\left(\operatorname{interp}\left(\boldsymbol{x}, \boldsymbol{V}^{(\mathrm{color})}\right), \boldsymbol{x}, \boldsymbol{d}\right)
% \end{equation}
% where $\boldsymbol{c}$ is the view-dependent color emission at position $\boldsymbol{x}$ viewing from direction $\boldsymbol{d}$.

% To render the color $\hat{C}(\boldsymbol{r})$ of ray $r$, K points are sampled on ray $r$ with densities and colors $\left\{\left(\sigma_i, \boldsymbol{c}_i\right)\right\}_{i=1}^K$. The K results are accumulated by the quadrature rule by Max \cite{Max1995OpticalMF}:
% \begin{align}
% \hat{C}(\mathbf{r})=\sum_{i=1}^K T_i\left(1-\exp \left(-\sigma_i \delta_i\right)\right) \mathbf{c}_i, 
% &\\
% T_i=\exp \left(-\sum_{j=1}^{i-1} \sigma_j \delta_j\right)
% \end{align}
% where $\delta_i=t_{i+1}-t_i$ is the distance between adjacent points along a ray, and $\alpha_i=1-\exp \left(-\sigma_i \delta_i\right)$ is the alpha value for traditional alpha compositing.

% The backbone is trained by minimizing the mean
% square error between the rendered and observed color. 

% \begin{equation}
% \mathcal{L}_{\text {color }}=\|\hat{C}(\boldsymbol{r})-C(\boldsymbol{r})\|_2^2
% \end{equation}

% By extracting the density values of the voxel grid $\boldsymbol{V}^{(\text {density })} \in \mathbb{R}^{1 \times N_x \times N_y \times N_z}$, we can get the compact 3D representation of the scene, as shown in the middle of Figure 2.

% We refer the readers to \cite{Sun2022DirectVG} for more details about the Direct Voxel Grid Optimization.


% \subsection{3D Semantic Concept Grounding}
% In \cite{hong20223d}, a Neural Descriptor Field (NDF) \cite{simeonov2021neural} which gives a feature vector for each 3D coordinate a feature vector is used for concept grounding by aligning the feature vector with the learned concept embeddings. Drawing inspiration from this, we also propose to use a feature voxel-grid  (in addition to density voxel grid and color voxel grid) used for concept grounding. The compact 3D feature representation is composed of one feature voxel-grid representation plus one view-independent shallow MLP: 

% \begin{equation}
% \boldsymbol{f}=\operatorname{MLP}^{(\mathrm{feature})}\left(\operatorname{interp}\left(\boldsymbol{x}, \boldsymbol{V}^{(\mathrm{feature})}\right), \boldsymbol{x}, \boldsymbol{d}\right)
% \end{equation}

% However, the drawback of \cite{hong20223d} is that the embeddings of concepts are learnt from sratch, which is unrealistic in the open-vocabulary reasoning in real-world data. Furthermore, compared to 2D data, 3D assets are limited in diversity and scale, posing challenges for training large vision-language models (VLMs) on 3D-and-language data. Therefore, there's no large-scale 3D VLMs that can be directly used for concept grounding. On the contrary, there's much progress on large-scale 2D VLMs \cite{Radford2021LearningTV, Ramesh2021ZeroShotTG} thanks to the countless image-caption data on the internet. Since we obtain 3D compact representations from 2D multi-view images, it's natural to utilize 2D VLMs to ground semantic concepts on the 3D representations. Based on the CLIP model \cite{Radford2021LearningTV}, LSeg\cite{Li2022LanguagedrivenSS} manages to ground semantic concepts on each 2D pixel (and thus each ray $r$). We denote the feature of ray $r$ as $F(\boldsymbol{r})$.
% Since 3D voxel grids and 2D pixels are aligned via alpha compositing, we add one L1 loss to force the features of 3D voxel grids to align with the 2D LSeg pixels based on the alpha values. Specifically,

% \begin{equation}
% \mathcal{L}_{\text {feature}}=\sum_{i=1}^K T_i\left(1-\exp \left(-\sigma_i \delta_i\right)\right)(\|\boldsymbol{f}-F(\boldsymbol{r})\|)
% \end{equation}

% Assuming we have a set of concepts $P$, the similarities between a concept $\boldsymbol{p} \in P$ and a feature $\boldsymbol{f}$ is calculated as $\langle \boldsymbol{f}, \boldsymbol{p} \rangle$. We define a \textsc{Filter} operator. Specifically, the 3D compact representation for a semantic class $p$ after filtering out that class is:

% \begin{equation}
% \boldsymbol{V}_{\boldsymbol{p}} =  min(\langle\boldsymbol{V}^{(\mathrm{feature})}, \boldsymbol{p}\rangle, \boldsymbol{V}^{(\mathrm{density})}) 
% \end{equation}

% In practice, we only set the values of voxel grids with densities < 0.5 to 0, since we find that those points are irrelevant to the 3D geometry of the scene.

% To get each instance of the objects of the same category, we use DBSCAN \cite{Ester1996ADA} to implement the \textsc{Get\_Instance} operator which assigns clusters to all true values of $\boldsymbol{V}_{\boldsymbol{p}}$. The DBSCAN takes the 3D coordinates as input.





\section{Data Practices in Privacy Labels}
% In this section, we study how the app developers report their apps' privacy practices in the privacy labels. Specifically, we use the APL and DSS datasets curated in \Cref{sec:measurement} to answer the question: \textit{How do applications collect and use data?}. Further, we analyze the privacy labels for apps along three dimension: age rating, price and popularity. Finally, we identify sensitive data flows and perform a case study to show how apps can misuse the data.
\subsection{Google Data Safety Section}
\label{google-data-safety-sec4}
In this section, we analyze the DSS dataset (\cref{sec:google_dsc}) comprising \nnumber{573K} apps. We first discuss the practices present in DSS and then examine how these practices vary with an age rating, price, and popularity.
% In this section, we begin by discussing the high level practices. We then discuss the practices at \textit{Data Category} and \textit{Purpose} level. In our  we find that \nnumber{50}\% of the apps on the play store have Data Safety Sections.\smallskip

\noindent
\textbf{Data Collection and Sharing:} 
Among the apps having DSS, we saw \nnumber{42.3}\% collecting at least one type of data, and \nnumber{35.8}\% sharing at least one data type (purple bars in the top plot for \cref{fig:free_paid}). This suggests that the majority of the apps on the play store report do not collect or share data. This is in contrast with the findings from prior work~\cite{wang2015wukong} that found that the majority of the apps use at least one third-party application, which has been shown to collect sensitive information~\cite{book2013longitudinal, lin2013understanding}. One possible explanation for this is that developers find it hard to understand the collection and sharing practices of third-party libraries. This is also supported by prior research~\cite{balebako2014privacy, li2022understanding}. As such, when inquired about change in DSS, one developer also alluded to lack of transparency by third parties:\newline
\textit{``We don't collect or share any user data. But we use Meta (former Facebook) audience network for monetizing non-paying users with ads. Unfortunately, the details provided by Meta are very vague..''}

We also note that among the apps not collecting any data around \nnumber{23}\% are sharing data. This is because \textit{Data Collection} is defined as the instance when the developers retrieve the data from the device using the app~\cite{googledocumentation}, whereas \textit{Data Sharing} is defined as when the data is transferred from the device to a third party. This way, the developers can share data without collecting it if the application uses third-party libraries which directly send data to their servers.\smallskip

\begin{figure}
  \centering
  \includegraphics[width=\columnwidth]{figures/good_pdfs/google_apple_high_level_privacy_labels_distribution_cropped.pdf}
  \caption{Distribution of privacy types in Google Data Safety Sections and Apple Privacy Labels. The normalization is done by the total number of apps with privacy labels.}
  \label{fig:free_paid}
\end{figure}

\noindent\textbf{Security Practices:}
We find that \nnumber{23}\% of the apps do not provide any details of their security practices. \nnumber{65}\% of the apps encrypt data that they collect or share while it's in transit, and \nnumber{42}\% allow the users to request that their data be deleted or automatically anonymize data within 90 days. Notably, we find that \nnumber{17.4\%} of the apps state that they do not collect or share data, but encrypt the data in transit. We explore this behavior further in \cref{sec:developer}. As apps need network permissions to transmit data, we cross-verified encryption practices with apps' network permission requests and find that \nnumber{10.5\%} apps do request network permission but do not encrypt data, potentially exposing user data in plain text. Additionally, \nnumber{2.2\%} of apps do not request network permissions, yet state that they encrypt data in transit, suggesting that some developers might be over-reporting their practices, consistent with prior research~\cite{li2022understanding}.\smallskip


\begin{figure}
  \centering
  \includegraphics[width=\columnwidth]{figures/good_pdfs/google_apple_highlevel_category_cropped.pdf}
  \caption{Distribution of Top-5 data categories for high-level practices for apps in Play Store (top) and App store (bottom). The normalization is done by the total number of apps with privacy labels. For plots with data categories, see \cref{fig:inconsistency_app_no_thresh} in the Appendix.}
  \label{fig:google_collected_sharing}
\end{figure}

%and among the apps not sharing any data, 30\% are collecting data. 

\noindent\textbf{Category and Purpose Level Practices:}
In \cref{fig:google_collected_sharing}, we present the top-5 data categories for \textit{Data Collection} and \textit{Data Sharing} by apps in play store. A full plot including all data categories can be found in \cref{fig:inconsistency_app_no_thresh} (Appendix).
Our findings indicate that the data categories \textit{Personal Information} and \textit{App activity} are among the most frequently collected, and are primarily used for \textit{App functionality} and \textit{Analytics}. However, \textit{Location} and \textit{Device Ids} are more commonly shared for the purpose of \textit{Advertising or Marketing}. We emphasize that this flow poses serious privacy risks and allows for tracking by third parties.
We also observe that sensitive data types such as \textit{Audio}, \textit{Files and Docs}, and \textit{Health and Fitness} are collected less frequently, with the most common purpose being \textit{App functionality}. Furthermore, we note that out of the 7 possible purposes for collecting data there are over \nnumber{4K} apps that list 6 or more purposes for the data they collect, which may indicate that app developers list all purposes out of convenience. For example, \textit{Workplace from Meta} with over 15M+ downloads, lists the same 6 purposes for all the data they collect like access to \textit{Installed Apps}, \textit{SMS or MMS}, \textit{Music Files}. This is consistent with the findings of Li et al.~\cite{li2022understanding}, who suggest that developers may over-report in cases of ambiguity.\smallskip

\noindent\textbf{Variation of Practices with Popularity:}  We first investigate the relationship between privacy practices and app popularity. We classify apps into three categories based on their number of downloads: extremely popular (greater than 1M download, n=56K), semi-popular (more than 10K downloads, n=524K), and low-popular (less than 10K downloads, n=621K).
Our findings reveal that 1) the fraction of apps displaying Data Safety Sections (DSS) increases with the popularity of the apps (42\% for low-popular, 51\% for semi-popular and 76\% for extremely popular) and 2) the fraction of apps collecting and sharing data is less for popular apps (41\% for low-popular, 46\% for semi-popular and 12\% for extremely popular). These results suggest that developers from popular apps tend to report more privacy-friendly practices.\smallskip

\begin{figure}
  \centering
  \includegraphics[width=\columnwidth]{figures/good_pdfs/google_apple_high_level_age_rating_distribution_cropped.pdf}
  \caption{Distribution of privacy types based on age rating for DSS and APLs. The normalization is done by the total number of apps with privacy labels.}
  \label{fig:age_rating}
\end{figure}

\noindent\textbf{Variation of Practices with Age Rating:}
Next, we examine how the practices of apps differ based on their age rating as determined by the Google Play Store. The Play Store assigns five different age ratings: Everyone, Teen, Mature 17+, and Everyone 10+\footnote{Google also has Adults 18+ rating, but we found less than 200 apps in this category and decided to filter it out for this analysis}. We acknowledge the importance of this distinction, as apps that are accessible to children and teens (falling in the Everyone and Teen categories) are expected to have higher transparency and collect less data. However, our analysis of the dataset reveals that 59\% of apps with the \textit{Mature 17+} rating have a Data Safety Section (DSS), while the fraction of apps with a DSS in the other age ratings ranges from 47\% (Everyone) to 55\% (Everyone 10+). The data practices for different age ratings are shown in \cref{fig:age_rating}.  We find that the fraction of apps having \textit{Data Collection} and \textit{Data Sharing} is lowest for apps rated for \textit{Everyone}, whereas apps targeting \textit{Mature 17+} have the highest encryption rate.\smallskip

\noindent\textbf{Variation of Practices with Price:}
Finally, we study the difference in practices based on whether the app is available for free, free with in-app purchases, or paid. We find that 68\% of the paid apps have DSS whereas, for free apps, only 46\% have DSS. \cref{fig:free_paid} shows the distribution of high-level practices with free and paid apps. We note that for paid apps, a fraction of apps collecting and sharing data is lower. Furthermore, apps with \textit{Data Encryption} and \textit{Data Deletion} are lower because the apps are collecting and sharing fewer data. This suggests that paid apps tend to have better data practices. 

% \rishabh{@Asmit: Add the detailed analysis for encryption here}.

%=======================================================================================================================

\subsection{Apple Privacy Labels}
Next, we examine the Apple Privacy Label (APL) dataset (\cref{sec:apple_label_methodology}) consisting of privacy labels from \nnumber{955K} apps. We first discuss the practices present in APL and then dive into variations of practices with an age rating and price. Finally, we conclude by comparing the low-level practices mentioned in APL and DSS \smallskip

\noindent
\textbf{High-Level Practices:} In our dataset, 42\% of apps collected data from users that were not linked back to the user (Data Not Linked to You), whereas 37\% of apps did collect data that is linked to the user (\cref{fig:free_paid}. Note that apps could collect multiple types of data some of which may be linked to the users while others may not. Furthermore, around 18\% of the apps reported collecting data that was used to track the users. Note that this reflects the status of the APLs after the \textit{Apple Tracking Transparency} policy was implemented, which requires developers to obtain consent from users before tracking. We also find that 42\% of apps report that they do not collect any data from users. Recent works~\cite{kollnig2021iphones, kollnig2022goodbye} analyzing iOS apps have found that at least 80\% of the apps still use tracking libraries in the apps. Further, these libraries have been shown to collect user data~\cite{book2013longitudinal, lin2013understanding}. Similar to the case of android developers, this discrepancy can be explained by the lack of transparency of privacy practices by the third-party libraries, resulting in confusion for the developers. \smallskip
% \rishabh{Should we go one level deeper and talk about data types as well? }

For \textit{Data Used to Track You}, we find that \textit{Usage Data} and \textit{Identifiers} are most commonly used. We note here that Apple defines \textit{Tracking} as when data collected is linked with third-party data for targeted advertising, as well as when the data is shared with a data broker. Additionally, we observe that 25\% of the apps collecting \textit{Location} information also use it for tracking. This poses severe privacy risks to the users as entities can track the physical location of the users which can reveal sensitive details about users' habits and routines. \smallskip

\noindent\textbf{Data Category and Purpose Level Practices:} In \cref{fig:google_collected_sharing} (bottom), we show the top-6 data categories mentioned in the high-level practices in the APL dataset. We find that for \textit{Data Linked to You}, \textit{Contact Information} and \textit{Identifiers} are collected most frequently, whereas for \textit{Data Not Linked to You}, \textit{Diagnostics} and \textit{Usage Data} are collected most frequently. Apple defines \textit{Contact Information} as name, email, phone number, and physical address, whereas \textit{Usage Data} refers to product interactions and advertising data such as information about the ads that the user has viewed. Analyzing purposes for these data categories, we find that nearly 60\% of the apps use these data categories for \textit{App functionality} and \textit{Analytics}. It is also worthwhile to note that \textit{Contact Information} is used for \textit{Advertisements} in only 8\% of the apps that collect this information, indicating that apps generally do not use personal information for advertisements. We also note that \textit{Identifiers}, commonly used for tracking users for targeted advertising is used for \textit{Advertisement or Marketing} in more than 20\% of the apps that collect \textit{Identifiers}. Interestingly, \textit{Location}, under \textit{Data Linked to You} is also used for \textit{Advertisement or Marketing} by 20\% of the apps that collect \textit{Location}.\smallskip

\noindent
\textbf{Variation of Practices with Age Rating} Next, we investigate the correlation between the privacy practices described in the Android Permission List (APL) and the age rating and price of apps. The App Store assigns four different age ratings: 4+, 9+, 12+, and 17+ (which roughly align with the rating system used by the Google Play Store). Our analysis reveals that the fraction of apps with an age rating of 17+ is highest at 76\%. However, we note that the high-level data practices, shown in Figure 1, are consistently more privacy-friendly for apps with lower age ratings. For instance, only 13\% of the apps with an age rating of 4+ track users. Similarly, data collection for these apps is also consistently lower than that of other categories.\smallskip

\noindent
\textbf{Variation of Practices with Price:} Finally, we categorize the dataset into free and paid apps and examine the differences in privacy labels. Recall that for the play store, we observed that paid apps contained more DSS than free apps. For APL, we find the reverse trend with 70\% of the free apps having APL as compared to 52\% paid apps. On the other hand, the high-level practices are decidedly better for the paid apps, as shown in \cref{fig:free_paid} (bottom chart). For instance, 82\% of the paid apps reported not collecting any data, while only 3\% of paid apps mentioned using data to track the user. This indicates that the paid apps on iOS platforms are more friendly than the free apps.\smallskip

\noindent
\textbf{Comparison Between DSS and APL:} As discussed in \cref{sec:background}, DSS and APL provide different information to the users, and cannot be directly compared based on high-level practices. However, since the underlying data collected is the same, we can compare the practices shown in \cref{fig:google_collected_sharing}. We observe that the fraction of apps requesting similar datatypes is much smaller for apps on the play store than that of the app store (with a notable exception of Location). This can be attributed to the fact that developers have had a longer to work with the APL framework, while the DSS framework is still relatively new. In our communication with app developers, one app developer mentioned that they try different answers on the data safety form. We also received communication indicating that some developers updated their DSS based on the questions that we had asked. This indicates that the developers are unclear on the process involved in the data safety forms, which might result in some inaccuracies in the DSSs. This is also supported by the study conducted by Li. et al ~\cite{lin2013understanding} where they find that app developers find it difficult and challenging to fill out the privacy labels, especially, because the frameworks for Apple and Google are starkly different and can create confusion.

\subsection{Developer Study}
\label{sec:developer}
In \cref{sec:privacy_labels} and \cref{google-data-safety-sec4}, we identified three trends in Data Safety Sections: (A) apps stating that they encrypt data without collecting or sharing data, (B) apps changing their practice from not collecting/sharing data to collecting/sharing data, and (C) apps changing their practices from collecting/sharing data to not collecting/sharing data. To gain a deeper understanding of these trends, we reached out to developers via email and asked them one general question about their Data Safety Sections and one specific question about the type of trend we observed in their app. We contacted 30K developers from the Play Store. It is worth noting that, since Apple does not provide email addresses for developers, we only conducted this study with Android developers.

In our initial email, we clearly identified ourselves as researchers and stated that we were studying their application and wanted information regarding their data safety section (\cref{app:dev-study}). Additionally, we do not collect any personally identifiable information from the developers and only use their publicly available contact information from the Play Store to contact them. As such, the study has been approved by the IRB at our institute.\smallskip

% One of the common themes in all the findings so far has been inconsistency in the privacy labels. With this, a natural question arises: \textit{What are the factors that contribute to this inconsistency?} To understand this further, we reached out to 50K android app developers via email and requested more information surrounding the inconsistencies within the Data Safety Cards. In our initial email, we clearly identified ourselves as researchers and then stated that we were studying their application and wanted information regarding some inconsistency in their data safety section. 
\noindent
\textbf{Findings:} Based on our initial emails, we received 2500 responses. After filtering out the automated replies using keyword filtering, we were left with 889 responses where the app developers describe the challenges they face while working with the privacy labels, as well as provided information about their data safety section. We further manually examined each response and curated a set of 307 replies. This manual filtering removed replies that included non-relevant replies. 
% In this section, we analyze these responses from developers by manually coding the responses. 
Next, one of the authors manually coded the responses to identify the factors for the trends, as well as general challenges described by the developers. Another author independently verified the findings by coding a subset of 50 responses independently. Specifically, we first present the major contributing factors for the different types of trends mentioned above. We then discuss the top challenges that developers face while working with the Data Safety Section.\smallskip
% Our approach has the natural advantage of extracting information under natural working environment, as opposed to survey based studies or interview based studies where the participants have the notion \rishabh{Complete this section once the analysis is complete.}

\noindent
\textbf{Type A: Apps Stating that they encrypt data without collecting or sharing data}: For this trend, we obtained responses from 165 developers. Of these, 56\% mentioned data is collected by third-party services like ads or Google Firebase but were not sure if it should be added to DSS while another 36\% were not sure what data was collected, 3\% of the developers were confused regarding encryption and added the option thinking of SSL encryption for communications between the server and the app, without collecting/sharing data. For example, one of the developers said the following: \textit{``I use Google's own libraries for this. In the Google Play Console, Policies section, I had to guess that Google is sending data and I rely on Google to encrypt that data. Because Google says that the developer is responsible for the libraries they use. That's why you find the contradictory result.''}\smallskip

\noindent
\textbf{Type B: Apps changing their practice from not collecting/ sharing to collecting/sharing:} For this trend, we received responses from 130 developers, 12\% of whom did not understand the process and selected any option that was accepted whereas 74\% changed DSS after realizing that third party libraries are collecting data. 12\% of them had an app update while 2\% changed DSS to ensure that they were up to date with the regulations like GDPR. For example, one developer said, \textit{``... Admob SDK I am integrating with the app might collect information [...] And According to Google policy, if I am using the latest version of their Admob SDK, I have to specify that the app is collecting or sharing data ...''}\smallskip

\noindent
\textbf{Type C: Apps changing their practice from collecting/sharing to not collecting/sharing:} 
We only obtained 12 responses for this trend, 58\% of which stated that their app was updated, but the DSS reflects an older version, 25\% mentioned that data was collected by ad libraries that have since been removed, and 9\% mentioned regulations as a factor for the change in DSS. For example, one developer said, \textit{``...we have changed the data safety section of our application because we [...] removed any data collecting libraries such as Firebase [...] Admob for monetization...''} \smallskip

\noindent
\textbf{Challenges for Developers: } We find that the developers are generally confused about how to fill the Data Safety Section. The source of confusion varies from \textit{Not understanding the Process} to \textit{Not understanding if data collected by the third party should be reflected in DSS}. For example, one developer stated 
\textit{``What [...] keeps changing every few months is Google's privacy policies.  They are difficult to understand and they shift like sand... I don't really understand half of them and so we just keep submitting answers in hopes it's what they are looking for...''}
indicating that the process is very unclear, while another mentioned \textit{``the reason for the change was because google play forced me to put that information''}. These confusions are problematic as they may result in inaccurate privacy labels. They can also under-represent the privacy practices in the privacy labels which can give a false sense of security to the users, increasing their privacy risks. 

We note that in an earlier qualitative study, Li et al.~\cite{li2022understanding} found that ``Developers felt unconcerned about privacy and that it was not their responsibility''. In our study, we found that developers cared about user privacy, but did not have enough means (either lack of resources or lack of transparency with third-party libraries' privacy practices) to create accurate labels, causing some frustration on their part. For example, one developer said \textit{``... we use Meta (former Facebook) audience network for monetizing non-paying users with ads. Unfortunately, the details provided by Meta are very vague, but definitely are considered as collecting and sharing data. If possible we would love to switch to an ad provider that offers proper non-personalized ads with zero/minimal data collection, but it seems impossible to find such a provider.''}. 

% We note that our findings are consistent with the qualitative study performed by Li et al.~\cite{li2022understanding}. It also worth noting that even though our sampling was done based on apps showing particular trends, the general challenges (non-exhaustive) mentioned by developers give some insights on developers' view on DSS. These also highlight the need for further research and clarity about the privacy labels for the developers. 



% \rishabh{We might want to remove the case studies?}
% \subsection{Case Studies}
% For the case studies, we looked into how apps use sensitive data like sexual orientation, race and ethnicity, and political ideations. From the dataset of google apps, we find about \nnumber{7.6K} apps that collect sensitive data types, and \nnumber{4K} apps that share these sensitive data.

% Of the apps that share sensitive data, there are \nnumber{398} apps with more than 500K downloads. Of these \nnumber{398} apps, there are \nnumber{51} apps that share sensitive data for \textit{Developer Communications}, \nnumber{132} apps sharing data for \textit{Advertising or Marketing} purposes, and \nnumber{74} apps sharing data for \textit{Fraud Prevention, Security, and Compliance}. We further found that \nnumber{31} apps under the Parenting genre collecting and/or sharing sensitive data, with \nnumber{11} apps having more than 100K downloads and \nnumber{3} apps having more than 1M downloads. These apps are targeted toward children and it's interesting to observe that they are collecting and/or sharing sensitive data. For example, the app ~\textit{Find My Kids Family tracker} \footnote{~\url{https://play.google.com/store/apps/details?id=org.findmykids.app}} collects the Sexual Orientation of the users for App functionality and personalization, which in this case are clearly kids. Another app, ~\textit{Preggers | Pregnancy \& Baby} \footnote{~\url{https://play.google.com/store/apps/details?id=life.stroller.preggers}} collects Sexual Orientation data for developer communication, and Race and Ethnicity data for App Functionality and Analytics. This app is aimed at expectant parents and collects significant sensitive data from their users as stated in their app descriptions.

% Similarly, we analyzed the apps in the App Store to observe the trends in apps handling sensitive data. We found that there are 14K apps that handled sensitive data. 11K apps collected data that was linked to the user, 2.4K apps collected data that was not linked to the user, and 500 apps that use sensitive data to track users. Furthermore, there were 600 Education apps, and 1200 Financial apps collecting sensitive data. Moreover, among the 14K apps, 2.3K apps are rated 12+, 128 apps are rated 9+ and 7.6K apps are rated 4+. Upon further analyzing the apps rated 4+ and how they handle sensitive data we find that 239 apps use it to track users, 365 apps use it for advertising. For example, \textit{Nike Training Club: Fitness}\footnote{\url{https://apps.apple.com/us/app/nike-training-club-fitness/id301521403}}, rated 4+, in the app store collects sensitive data for tracking. Same app on Google Play Store\footnote{\url{https://play.google.com/store/apps/details?id=com.nike.ntc}} has over 23M downloads.
\subsection{Takeaways}
The analysis presented here results in three main takeaways: 1) Privacy practices reported in the privacy nutrition labels differ from the privacy practices derived using app analysis by prior works~\cite{wang2015wukong}. Specifically, prior works have shown that third-party libraries are used in the majority of the apps and that these libraries collect sensitive information from the users. This is inconsistent with what we find in the privacy labels. This inconsistency can be explained by the fact that privacy practices of third-party libraries are often vague and create confusion among the developers (consistent with findings from literature~\cite{li2022understanding}). 2) We also show that paid apps, and apps that are open to all age groups, including children, are more privacy-friendly. As shown in \cref{fig:free_paid} and \cref{fig:age_rating}, these apps are less likely to engage in tracking, data collection, and data sharing. 3) \cref{fig:google_collected_sharing} also shows that location data is often used for advertising, marketing, and tracking. This poses severe privacy risks, as location data can reveal sensitive information about an individual's habits and routines. Our research suggests that further attention should be paid to the use of location data in mobile apps, and the potential risks it poses to user privacy.












%We first perform longitudinal study was conducted to analyze the periodic snapshots of data safety cards across all apps on the Google Play Store. The study also analyzed the practices reported by app developers in the Data Safety Card.

%Google imposed a hard deadline for app developers to add the data safety cards on July 20, 2022, which falls within the data collection window. This allowed for the study of the adoption rate as well as the evolution of data safety cards.

%During the period of June 20, 2022 to November 23, 2022, it was found that the fraction of apps with data safety cards increased from 27.9% to 43.3%. The largest change was observed between July 13, 2022 and August 1, 2022, which coincides with the deadline provided by Google.

% We first perform a longitudinal study of the data safety cards by analyzing periodic snapshots of data safety cards across all apps on the play store. We then analyze the practices as reported by the app developers in the Data Safety Card.
% \vspace{-1.5mm}
% \noindent
% For example, ~\textit{Shake-it Alarm - Alarm Clock
% }\footnote{\url{https://play.google.com/store/apps/details?id=com.ingyomate.shakeit}} with over 7M downloads state that they share `Health and Fitness' data with third parties but earlier in July\footnote{\url{https://web.archive.org/web/20220715114242/https://play.google.com/store/apps/details?id=com.ingyomate.shakeit}} their DSS didn't have that information. In fact, Google's own apps ~\textit{Clock}\footnote{\url{https://play.google.com/store/apps/details?id=com.google.android.deskclock}} added the collection of `Personal Info' after June.

% \rishabh{Add one/two examples here, possibly showing a sensitive datatype being added/removed}.

% \red{In order to understand the nature of change in privacy labels, we contacted developers of these apps via email asking them to provide information about the change in the privacy labels. In particular, we sent emails to developers of top 10K apps where the data safety card changed. Based on the responses of \nnumber{n} developers, we find that filling the data safety card is confusing for the developers, which can results in errors, requiring updates.}\smallskip

%\Cref{fig:trend_with_time} shows %the rate of adoption trend of high level data practices with time. 
%how the data safety cards were added during our collection window. 

% In particular, we find that the fraction of apps with data safety card increases over time from \nnumber{27.9\%} to \nnumber{43.3\%}. 
% % However, it is worth noting that we found \nnumber{7\%} of the apps removing their data safety card after adding it at an earlier date. 
% This change over time can be attributed to 4 cases:
% \begin{enumerate}
%   \item \nnumber{x} new apps with DSS are added. This number is affected by 2 conditions:
%   \begin{enumerate}
%     \item New apps are added to Google Play Store which are also listed in the androzoo list
%     \item Earlier apps are now added to the androzoo list
%   \end{enumerate}
%   \item \nnumber{x} old apps without DSS update and add DSS 
%   \item \nnumber{x} apps removed their DSS
%   \item \nnumber{x} apps are removed from Play Store
% \end{enumerate}
 
% \todo[inline]{We can say that such a significant turnout suggests that apps developers sometimes just put down random data to meet deadline then remove it}
% \rishabh{This feels very incomplete. We are not providing any insights here, just a plot and how to read the plot.}

%From the plot, it shows that the number of new apps being added is more than the number of net apps that have DSS. Although the number of apps whose DSS is removed without the app itself being deleted from the play store is low but not insignificant. 

% <TALK ABOUT FINDINGS FROM THE PLOT, INTERSTING EXAMPLE WHERE DATA SAFETY CARD CHANGED FOR THE BETTER AND ONE FOR THE WORSE>
% \begin{figure}
%   \centering
%   \includegraphics[width=\columnwidth]{figures/good_pdfs/collected_shared_vs_purpose.pdf}
%   \caption{Purpose distribution for apps for data sharing and data collected. The denominator here is the number of apps mentioning collection or sharing in the Data safety card.}
%   \label{fig:methodology}
% \end{figure}
% \subsubsection{Data Practices in Data Safety Cards}
% Next, we analyze the data practices as reported by the app developers on the android platform. We study the self-reported practices of the developers along three dimensions: age rating, number of downloads and price.\smallskip


% \rishabh{Look into transition from landscape to consistency - we can leverage the fact that the practices reported here are not in line with prior works, so in the remaining paper, we investigate the consistency}


% We have the following key takeaways: \rishabh{TODO: Form a coherent paragraph from these}
% \begin{itemize}
%     \item Privacy practices described in the PLs differ from what research has found - main reason being third party libraries, supported by previous research
%     \item Paid apps, and apps that are open to all age groups, including children are more privacy friendly - less tracking, collection and sharing.
%     \item Location is being used for ads, marketing and tracking - poses severe privacy risks. 
% \end{itemize}
% One of the points:

% 60\% of the apps do not disclose their data collection practices, even when they have a DSS. Of these apps, over \red{95\%} of the apps mention collecting data in their privacy policy. This indicates that developers might be uploading inaccurate data safety cards, just to comply with the regulations. Without a consistency check, the data safety cards are not usable as the users do not know whether the practices mentioned in the DSS are accurate.
% \subsection{Developer Study}
\label{sec:developer}
In \cref{sec:privacy_labels} and \cref{google-data-safety-sec4}, we identified three trends in Data Safety Sections: (A) apps stating that they encrypt data without collecting or sharing data, (B) apps changing their practice from not collecting/sharing data to collecting/sharing data, and (C) apps changing their practices from collecting/sharing data to not collecting/sharing data. To gain a deeper understanding of these trends, we reached out to developers via email and asked them one general question about their Data Safety Sections and one specific question about the type of trend we observed in their app. We contacted 30K developers from the Play Store. It is worth noting that, since Apple does not provide email addresses for developers, we only conducted this study with Android developers.

In our initial email, we clearly identified ourselves as researchers and stated that we were studying their application and wanted information regarding their data safety section (\cref{app:dev-study}). Additionally, we do not collect any personally identifiable information from the developers and only use their publicly available contact information from the Play Store to contact them. As such, the study has been approved by the IRB at our institute.\smallskip

% One of the common themes in all the findings so far has been inconsistency in the privacy labels. With this, a natural question arises: \textit{What are the factors that contribute to this inconsistency?} To understand this further, we reached out to 50K android app developers via email and requested more information surrounding the inconsistencies within the Data Safety Cards. In our initial email, we clearly identified ourselves as researchers and then stated that we were studying their application and wanted information regarding some inconsistency in their data safety section. 
\noindent
\textbf{Findings:} Based on our initial emails, we received 2500 responses. After filtering out the automated replies using keyword filtering, we were left with 889 responses where the app developers describe the challenges they face while working with the privacy labels, as well as provided information about their data safety section. We further manually examined each response and curated a set of 307 replies. This manual filtering removed replies that included non-relevant replies. 
% In this section, we analyze these responses from developers by manually coding the responses. 
Next, one of the authors manually coded the responses to identify the factors for the trends, as well as general challenges described by the developers. Another author independently verified the findings by coding a subset of 50 responses independently. Specifically, we first present the major contributing factors for the different types of trends mentioned above. We then discuss the top challenges that developers face while working with the Data Safety Section.\smallskip
% Our approach has the natural advantage of extracting information under natural working environment, as opposed to survey based studies or interview based studies where the participants have the notion \rishabh{Complete this section once the analysis is complete.}

\noindent
\textbf{Type A: Apps Stating that they encrypt data without collecting or sharing data}: For this trend, we obtained responses from 165 developers. Of these, 56\% mentioned data is collected by third-party services like ads or Google Firebase but were not sure if it should be added to DSS while another 36\% were not sure what data was collected, 3\% of the developers were confused regarding encryption and added the option thinking of SSL encryption for communications between the server and the app, without collecting/sharing data. For example, one of the developers said the following: \textit{``I use Google's own libraries for this. In the Google Play Console, Policies section, I had to guess that Google is sending data and I rely on Google to encrypt that data. Because Google says that the developer is responsible for the libraries they use. That's why you find the contradictory result.''}\smallskip

\noindent
\textbf{Type B: Apps changing their practice from not collecting/ sharing to collecting/sharing:} For this trend, we received responses from 130 developers, 12\% of whom did not understand the process and selected any option that was accepted whereas 74\% changed DSS after realizing that third party libraries are collecting data. 12\% of them had an app update while 2\% changed DSS to ensure that they were up to date with the regulations like GDPR. For example, one developer said, \textit{``... Admob SDK I am integrating with the app might collect information [...] And According to Google policy, if I am using the latest version of their Admob SDK, I have to specify that the app is collecting or sharing data ...''}\smallskip

\noindent
\textbf{Type C: Apps changing their practice from collecting/sharing to not collecting/sharing:} 
We only obtained 12 responses for this trend, 58\% of which stated that their app was updated, but the DSS reflects an older version, 25\% mentioned that data was collected by ad libraries that have since been removed, and 9\% mentioned regulations as a factor for the change in DSS. For example, one developer said, \textit{``...we have changed the data safety section of our application because we [...] removed any data collecting libraries such as Firebase [...] Admob for monetization...''} \smallskip

\noindent
\textbf{Challenges for Developers: } We find that the developers are generally confused about how to fill the Data Safety Section. The source of confusion varies from \textit{Not understanding the Process} to \textit{Not understanding if data collected by the third party should be reflected in DSS}. For example, one developer stated 
\textit{``What [...] keeps changing every few months is Google's privacy policies.  They are difficult to understand and they shift like sand... I don't really understand half of them and so we just keep submitting answers in hopes it's what they are looking for...''}
indicating that the process is very unclear, while another mentioned \textit{``the reason for the change was because google play forced me to put that information''}. These confusions are problematic as they may result in inaccurate privacy labels. They can also under-represent the privacy practices in the privacy labels which can give a false sense of security to the users, increasing their privacy risks. 

We note that in an earlier qualitative study, Li et al.~\cite{li2022understanding} found that ``Developers felt unconcerned about privacy and that it was not their responsibility''. In our study, we found that developers cared about user privacy, but did not have enough means (either lack of resources or lack of transparency with third-party libraries' privacy practices) to create accurate labels, causing some frustration on their part. For example, one developer said \textit{``... we use Meta (former Facebook) audience network for monetizing non-paying users with ads. Unfortunately, the details provided by Meta are very vague, but definitely are considered as collecting and sharing data. If possible we would love to switch to an ad provider that offers proper non-personalized ads with zero/minimal data collection, but it seems impossible to find such a provider.''}. 

% We note that our findings are consistent with the qualitative study performed by Li et al.~\cite{li2022understanding}. It also worth noting that even though our sampling was done based on apps showing particular trends, the general challenges (non-exhaustive) mentioned by developers give some insights on developers' view on DSS. These also highlight the need for further research and clarity about the privacy labels for the developers. 


\section{Practices Present in Privacy Policies}
\label{sec:policy_inconsistency}
The next research question that we answer is: \textit{How do the privacy practices mentioned in the privacy labels of the apps compare with the privacy practices described in their privacy policies?} We perform this comparison by training machine learning classifiers to automatically extract privacy practices mentioned in the privacy labels, as described in \Cref{sec:privacy_policy}. For Google's DSS we have \nnumber{346K} apps with valid policies, whereas for Apple, we have \nnumber{343K} apps. As described in \cref{sec:privacy_policy}, we filter out the policies which are not in English. 
% \rishabh{@Rishabh: Make sure that this is well described in earlier section and then adjust text accordingly here.} 
Note that to obtain presence of a particular practice in privacy policy, we require that there exists at least one segment which classifies the segment for that practice. For example, for a policy to have \textit{Data Encryption} practice, we require presence of at least one segment where our classifier tags it as positive for \textit{Data Encryption}. A complete mapping from classifiers to practices in APL/DSS is described in \cref{tab:policy_to_label}.

% It is important to note here that we do not rely on high level classifiers such as \textit{First Party Collection} or \textit{Third Party Data Sharing} for extracting high level privacy practices such as \textit{Data Collection} and \textit{Data Sharing}. This is done to avoid generic collection or sharing statements contaminating the 

There are two types of inconsistencies that can arise: 1) \textit{In Label}, where a given privacy practice is mentioned in the privacy label but is absent from the privacy policy, and 2) \textit{In Policy}, where a practice is found in privacy policy but is missing from privacy label. As privacy policies can potentially cover multiple applications, websites and products, \textit{In Policy} inconsistency does not necessarily mean that policy is inconsistent with the privacy label. For example, the Google app \textit{Clock} reports that it does not collect or share any \textit{Location} information. However, since Google has one policy to cover all the products, the policy states that they can collect \textit{Location} (applicable in Google Maps). In such cases, it is inaccurate to say that privacy label are inconsistent with the privacy policy without further analyses. 
However, if an app mentions collection of data and the policy does not mention it, then we can conclusively say that the privacy policy and the app are inconsistent. Thus, in this work, we will focus primarily on \textit{In Label} inconsistencies, except when there is a negative practice is involved (\textit{Data Not Collected} or \textit{Data Not Linked to You}). This is because if the policy says that data is not collected, then no app corresponding to that policy should collect any data, and in this case we focus on \textit{In Policy} consistency.


% \red{The discussion above also highlights a major issue with using privacy policy as the source to understand privacy practices of the app. Specifically, we highlight the notion that privacy policies usually cover practices for websites (and potentially other products) and hence, may not accurately inform users about privacy practices of the application. In fact, they can mis-inform the user about the privacy practices in certain cases. \rishabh{Maybe move this to discussion and add that while privacy labels address this issue, without consistency checks, they do not contribute much.}}


% \subsection{Comparison with Privacy Labels}
% In this section, we look at how the privacy policies of apps compared with their respective privacy labels as stated in DSS and APL. For Google's DSS we had \nnumber{346K} apps and for Apple, we analyzed \nnumber{458K} apps. \rishabh{The total number of apps with DSS/APL is way higher, right? So these are the apps which had both DSS/APL and privacy labels? If so - are we saying that 350K apps on google had DSS but not a valid privacy policy? This would needs more explanation}\red{These numbers represent the number of apps that have both: privacy labels and a classified privacy policy. Here a classified privacy policy is defined if (1) it exists, (2) is in English.}\smallskip
\noindent
\subsection{Google Data Safety Sections}
\cref{fig:policy_comp} shows the \textit{In Policy} and \textit{In Label} inconsistency for high level practices for apps on the play store. We note that only 5\% (6\%) of the apps with DSS that collect (share) data have \textit{In Label} inconsistency with their privacy policies. 
% \rishabh{@Asmit: Add one example here, possibly from one of the emails that we got.} 
We also find that \textit{In Policy} inconsistencies for these categories are more than 55\%, but as discussed above, these could be due to privacy policy covering multiple apps and websites. To understand the extent to which this happens due to multiple apps, we analyze DSS for apps from the same developers. There are 15,380 developers who have 3 or more apps. These developers have an average of 13 apps and a median of 7 apps per developer. We find that 68\%(10,420) of these developers have duplicate data safety section for their apps. 
%The mean percentage of duplicate data safety section among all of these developers is 68.2\%, with a standard deviation of 22\% and a median of 75\%. 
For example, the app developer \textit{Premium Software} has over 9 apps across 6 genres but with only 2 unique DSS. 
% This indicates that \textit{In Policy} inconsistencies for developers having multiple apps are either coming from practices of websites or  are legitimate inconsistencies. 
This also highlights that developers might be duplicating their DSS across their apps, even though the apps can span multiple genres and have different features. 
% \rishabh{Maybe we need an example here? Or remove this part altogether?}

Analyzing the inconsistencies for \textit{Data Encryption} and \textit{Data Deletion}, we find that the majority of apps declare them in their privacy labels but there is no mention of such practices in their privacy policies. For example, \textit{Snapchat} mentions in their DSS that data is encrypted in transit but no corresponding practice is present in their privacy policy. Similarly, \textit{Kik — Messaging \& Chat App} state in their DSS that \textit{Data can’t be deleted} yet their privacy policy states that users can ask them to delete their information. It is worth noting that from a privacy and regulation standpoint, these two practices are extremely important. \textit{Data Deletion} option gives the users the right to either delete their data or ensure that it stays in anonymized form, which has roots in several regulations such as the GDPR~\cite{linden2018privacy} and the CCPA~\cite{ccpa}. \textit{Data Encryption} on the other hand, is crucial to prevent data snooping attacks which aim to get unlawful access to the data while the data is in transit.

% From analyzing the data gathered on apps from the Google Play Store, as shown in the top plot in \cref{fig:policy_comp}, we can see that for \textit{Data Collection} and \textit{Data Sharing}, over 50\% of apps have some type of inconsistency between their privacy policies and labels. Moreover, we note that for a majority of these apps their privacy policy mentions \textit{First Party} or \textit{Third Party Data Collection}, yet their privacy labels do not reflect that. For example, according to the DSS of the app, \textit{ShareMe: File sharing}\footnote{\url{https://play.google.com/store/apps/details?id=com.xiaomi.midrop}}, no data is being collected but their privacy policy does state that they can collect data. Interestingly this app also shares data with third parties but doesn't collect any data, which is non-intuitive but can happen when third party libraries are used. 

\begin{figure}
  \centering
  \includegraphics[scale=0.25]{figures/good_pdfs/high_level_inconsistency4_cropped.pdf}
  \caption{Inconsistencies between privacy policies and DSS and APL. The normalization is based on the total number of apps with privacy labels and classified policies.}
  \label{fig:policy_comp}
\end{figure}

Analyzing practices at the category level, we find that there are significant inconsistencies between Privacy Labels and policies for data sharing and data collection. Specifically, we find that for data sharing, 89\% of the apps have inconsistent (\textit{In Label}) Privacy Labels for \textit{Location}, 82\% of the apps for\textit{Device IDs}, and 74\% of the apps for \textit{Health and Fitness}. For example, \textit{Myntra - Fashion Shopping App} states that they collect location, health info, contact list, and much more, yet its privacy policy doesn't mention the collection or sharing of such data types. Similarly, \textit{Tripadvisor: Plan \& Book Trips} states that they collect and share location data yet there is no mention of such practices in their privacy policy. This suggests that developers report more precise data-sharing practices in Privacy Labels, and can inform users allowing them to make better choices. 

% \rishabh{TODO: Talk about datatypes, purposes - some insightful examples}
% \chatgpt{In analyzing purposes, we have noticed that there is often a discrepancy between what is stated in app policies and what is present in labels. However, we also observed that a significant percentage of apps have more information in their privacy labels than in their policies. For instance, \textit{YouTube}\footnote{\url{https://play.google.com/store/apps/details?id=com.google.android.youtube}} does not mention developer communication in its labels, but it is mentioned in their privacy policy\footnote{\url{https://policies.google.com/privacy\#:~:text=Communicate\%20with\%20you}}}.

% \chatgpt{When analyzing datatypes and categories, we found that most inconsistencies in datatypes occur with App Activity, App Info, Performance, and Personal Info, with these datatypes being collected according to policies but absent or inconsistent from labels. Similarly, for datatypes such as Location, Device IDs, and Other IDs, the inconsistency arises from these being present in labels but absent or claimed to not be collected in policies. For example, \textit{Paytm: Secure UPI Payments}\footnote{\url{https://play.google.com/store/apps/details?id=net.one97.paytm}} does not mention the collection of the user's Financial Info in its labels, but the app's privacy policy states that Financial Info is collected.}


% We note here that the distinction of the type inconsistency between label and privacy policy is important. For instance, if a data practice is present in the privacy label, it should be present in the corresponding privacy policy (e.g. \textit{Data Encryption} or \textit{Data Deletion}). However, if a privacy policy contains \textit{Data Collection} but the label doesn't, it is possible, though unlikely, that they are still consistent. This is because privacy policy can describe practices of the entity as a whole, rather than a single application. For example, multiple apps from the same developer will have the same privacy policy, but can have different privacy labels for different apps. 

% \todo[inline]{TALKED TO RISHABH ABOUT THIS}
% To understand this further, we analyze the data safety section for applications with the same developer. We identified \red{x} developers with an average of \red{y} applications. For \red{z\%} of the developers with more than 3 apps, we find that the data safety card does/does not change with apps. For example, \rishabh{give an example here where even the genre is different}. 
% \chatgpt{There are 15,380 developers who have 3 or more apps with over 10,000 downloads each, regardless of the app genre. Similarly, there are 3,790 developers who have 10 or more apps with over 10,000 downloads each. Additionally, there are 10,993 developers who have an app count between 3 and 10 and have over 10,000 downloads for each app. These developers have an average of 13 apps and a median of 7 apps per developer. Of these developers, 10,420 have duplicate data safety section for their apps. The mean percentage of duplicate data safety section among all of these developers is 68.2\%, with a standard deviation of 22\% and a median of 75\%.} For example, the app developer \textit{Premium Software}\footnote{\url{https://play.google.com/store/apps/dev?id=4843900089693546737}} has over 9 apps across 6 genres but with only 2 unique DSS.
% We do note that in some cases, developers do specify privacy practices for each app differently, however, for majority of the developers, the privacy label did not change, indicating that the inconsistencies with the privacy policies are legitimate. \smallskip
%\todo[inline]{Asmit: Insight: Well because developer follow the same PP maybe it makes sense that the DSS are same} 

\subsection{Apple Privacy Label}
Next, we analyze the apps on the App store and compare privacy practices mentioned in the apps' privacy policies and their Apple Privacy Labels. The bottom plot in \cref{fig:policy_comp} shows inconsistencies for the high-level categories present in APL. We find that for \textit{Data Linked to You} and \textit{Data Used to Track You}, the \textit{In Label} inconsistency is 5\% and 4\% respectively. As the other two of the high-level practices, \textit{Data Not Linked To You} and \textit{Data Not Collected} are negations, we consider the \textit{In Policy} inconsistency (see \cref{sec:policy_inconsistency}). We find that 42\% of the apps have policies that state that they do not link data whereas the privacy label indicates otherwise. Furthermore, 13\% of the apps have policies that do not have data collection or sharing, but the privacy label indicates otherwise. For example. ~\textit{Superior Vision} app on App Store states that they collect \textit{Health \& Fitness} data yet their policy doesn't state that.

In \cref{fig:policy_comp}, we observe that \textit{In Policy} inconsistency for \textit{Data Used to Track you} is very high. This implies that privacy policies include tracking practices while privacy labels do not. This can potentially be due to the presence of segments related to cookies in the privacy policy, for example. ~\textit{Netflix}'s privacy policy talks about using cookies to track users on their site but not about tracking via their app.

We next examine the consistency at the data category level for \textit{Data Linked to You} and find that, similar to DSS, \textit{Location} (39\%), \textit{Identifiers (51\%)} and \textit{Health and Fitness} (60\%) had the largest \textit{In Label} inconsistencies. For \textit{Data Not Linked to You}, we find inconsistencies primarily in the same data categories. For example, ~\textit{Jetpack Joyride} states that they collect Location data but their privacy policy states that they collect Location based on IP Address and not the GPS location.

% A similar analysis of Apple App Store apps, show that on average 30\% of apps have inconsistencies between what is reported by the APL and what their privacy policy states. Similar to what we observed in google apps, a considerable percentage of apps's policy mention that they collect data, anonymously or not, and use the data for tracking but their APLs fail to reflect these practices. \red{A notable example would be \textit{Netflix}\footnote{\url{https://apps.apple.com/us/app/netflix/id363590051}} which do not mention of any data being shared, their policy does share data}
% \todo[inline]{Netflix has Third Party Ads but not Data Used to Track You ---!?!??!}


\subsection{Takeaways}
In this section, we find that at least 40\% of the apps with DSS, and APL are inconsistent with their privacy policy. Additionally, we note that DSSs contain more information about \textit{Security practices} than privacy policies, and thus can provide useful information to the users. We also note that sensitive datatypes such as \textit{Location}, \textit{Identifiers} and \textit{Health and Fitness} had the largest \textit{In Label} inconsistencies, indicating that the developers disclose collection/sharing of these fine-grained datatypes in the privacy labels. 
% The main takeaways: 
% \begin{itemize}
%     \item For security practices (Encryption and data deletion), DSS contain more fine-grained information than the policy. 
%     \item DSS also contain more precise data sharing practices at the category level, for example for location, device id etc.
%     \item For APL - high inconsistency in Data Not Linked to you. 
% \end{itemize}

% We find that on Google Play Store, \red{50\%} apps have at least one inconsistency between privacy labels and privacy policies, whereas on the App store, \red{40\%} of the apps have inconsistencies. These inconsistencies can have three implications: a) Privacy Labels are accurate: In this case, 
% it is encouraging to see that more app developers take measures to ensure data security and choice. However, the inconsistency here highlights the shortcomings of the privacy policies and how it may fail to capture some important privacy practices. b) Privacy Labels are inaccurate: This brings the usefulness of privacy labels into question and can be even more dangerous than having no privacy labels as incorrect privacy labels might induce a false sense of security in users.

% \rishabh{From inconsistency between data types - we learn that in policy inconsistency is zero, indicating that privacy labels provide more fine grained details about the data types, and can be a very useful resource for users to make informed decision}
% \chatgpt{In our analysis of the inconsistency with respect to purposes, we have found that most apps state in their privacy policies that they are collecting or sharing data for purposes that are not reflected in their DSSs (Data Sharing and Collection). The most significant inconsistencies occur with regard to Developer Communication and Account Management. For example, \textit{My Talking Tom}\footnote{\url{https://play.google.com/store/apps/details?id=com.outfit7.mytalkingtomfree}} states in its DSS that it does not collect any information for Account Management, but its privacy policy\footnote{\url{https://outfit7.com/privacy/en/\#:~:text=If\%20you\%20use\%20the\%20\%22Log\%20in\%20with\%22\%20or\%20\%22Connect\%20to\%22\%20feature\%20we\%20may\%20access\%20and\%20store\%20some\%20or\%20all\%20of\%20the\%20following\%20information}} states otherwise.}

% \chatgpt{Upon examining the inconsistencies of data types, we have found that for most datatypes, the inconsistency is due to the privacy policy stating that the app collects the data while the DSS does not mention it. However, Location and Device or Other IDs have the most inconsistency, with apps stating in their DSSs that they collect these datatypes but their privacy policies do not indicate so. For example, the privacy policy\footnote{\url{https://privacy.microsoft.com/en-ca/privacystatement}} for \textit{Microsoft Word: Edit Documents}\footnote{\url{https://play.google.com/store/apps/details?id=com.microsoft.office.word}} does not mention the collection of location data, yet the DSS for the app states that location is collected.}
% \rishabh{Write more about data which dataypes were most common, what purpose was most common etc. AN: Why here?}
\section{Data Practices Across Platforms}
\label{sec:consistency_cross}
Next, we compare the self-reported privacy practices in Privacy Labels of cross-listed apps across the Google play store and Apple app store using the cross-listed dataset described in \cref{sec:cross_apps}. 
% As DSS and APL use different terminology for privacy practices, we first create a common mapping for privacy practices. We then compare the practices reported in privacy labels.
%In particular, we first identify the apps that are present in both the platforms, extract their privacy nutrition labels and investigate whether the same app (and developers) report different privacy practices across different platforms. We first begin by detailing the methodology for identifying cross-listed apps and then perform the comparison between self reported privacy practices at scale. 

\begin{figure*}
\hspace*{-2.2cm} 
  \centering
  \includegraphics[width=1.3\columnwidth]{figures/good_pdfs/heatmap_normalized4.pdf}
  \caption{Normalized Heatmap showing the inconsistencies in the datatype-purpose pair. Normalization is done for each cell block in the heatmap, \textit{i.e.}, for each datatype-purpose pair, we normalize with the total number of apps which have that datatype-purpose pair.}
  \label{fig:heatmap}
\end{figure*}



% \subsection{Identifying Cross Listed Apps}
% In the app market, the two major platforms are Apple store and Google play store. Developers can increase their reach by building and serving their app across both the platforms. However, identifying two versions of the same app across platforms is a challenging problem due to lack of unique identifiers. There are several pseudo identifies such as app name, developer name, privacy policy and developer website, but none of them can be used as unique identifiers. For example, a same developer can build multiple apps across both platforms, each of the app having their own privacy nutrition label. However, these apps might have the same privacy policy, developer name and developer website. Further, a developer might also use different URLs to show privacy policies for different platforms, such as \url{APP NAME}. 

% \begin{figure}[t]
%   \centering
%   \includegraphics[width=\columnwidth]{figures/cross_listing.pdf}
%   \caption{Pipeline for identifying cross listed apps}
%   \label{fig:cross_listed}
% \end{figure}


% To address this problem and uniquely map apps across platforms, we use combination of pseudo identifies and relied on a set of heuristic pipeline, shown in Fig.~\ref{fig:cross_listed}. Specifically, we start with the apps which have the same name across both the platforms (n=500k). Next, if the privacy policy of the apps match, then we treat them as a unique match (n=190K). In some cases, <ENTER EXAMPLE HERE>, the app developers can include platform specific identifiers in the URLs for privacy policies. To capture these instances, we match the first level domain of the privacy policy URLs, and identify them as unique matches (n=104K). Finally, while providing privacy policy links is highly encouraged in both platforms, some apps do not contain the link to privacy policy. To further increase the coverage, we also match the first level domain of the developer website, which is present in both the platforms. Using this criteria, we are further able to get 45K matches. This way, we obtain a total of 400K apps which have instances in both Apple play store and Google play store. 

% Another potential scenario is that apps can have different names but the same privacy policy and the same developer name. In such cases, we can do a fuzzy string matching and find apps with similar names. While capturing these instances will increase our coverage, it might also result in higher number of false positives as different apps developed by the same developer might get flagged with this approach. For example, \texttt{app1} and \texttt{app2} have the same developer and the same privacy policy but have different functionality. Our objective with the analysis of privacy labels for cross-listed apps is to understand whether developers disclose their privacy practices similarly on the two platforms. For our purposes, having an accurately mapped set for is more important than captures all instances cross-listed apps. Therefore, we rely on the strong set of heuristics to curate this set.

% \subsubsection*{\textbf{Manual Verification}} We note that the set of cross-listed apps identified above are based on a set of heuristic rules, with the objective to minimize false positives. To further ensure that we do not compare practices of different apps, two of the authors manually verified 150 app pairs identified using each of the three steps shown in Fig.~\ref{fig:cross_listed} and found that no app from Google Play Store was matched to an incorrect app from App Store. 

% \subsubsection*{\textbf{Cross-listed Apps Dataset}} Using the methodology described above, we find a total of 400K cross listed apps. Among these apps, we find that 15\% have privacy nutrition label only on the Google Play Store, 31\% of the apps have the label only on the Apple App store, 43\% of the apps have labels on both the platforms while 11\% of the apps do not have privacy nutrition label on either of the platform. The higher rate of privacy labels for App store can be understood as Apple enforced nutrition label on their platform earlier than Google, giving more time for developers to add the details in the privacy label. For the 13\% of the apps which have privacy label for play store but not for app store - majority of the apps have not been updated in the last 1.5 years, suggesting that a driving factor for adding the nutrition labels is when the developers are forced to add it. Examining the apps which have privacy label for app store but not for play store also reveals a similar pattern. 
%  <add some examples>


\subsection{Mapping DSS categories to APL categories}
%Privacy Nutrition labels are designed to provide information about the data collection, sharing and usage by a given app. 
As previously discussed in Sec.~\ref{sec:background}, the privacy labels for android and iOS platforms cover different aspects of data practices. APL emphasizes on tracking and linkability of the collected data without distinguishing between collected and shared data, while DSS focuses on security practices and whether the data is collected or shared with a third party. 
%For example, in Apple privacy label, data used for tracking is highlighted separately whereas in data safety sections, security practices such as data encryption in transit and option to delete data are mentioned. 
However, despite covering disjoint high-level practices, the lower-level attributes in the privacy labels - namely the \texttt{datatype} and \texttt{purpose} - have a large overlap. Thus, to compare the disclosure practices of app developers across platforms, we first find the common datatypes and purposes in the two labels, and then compare 1) the datatypes, and 2) datatype-purpose pairs. 
% For the latter, we take a union of  all the common datatype-purpose pairs present in the APL and compare them with the common datatype-purpose pairs present in DSS. 

% \begin{table}[t]
% \footnotesize
% \centering
% \begin{tabular}{>{\arraybackslash}l>{\arraybackslash}l>{\arraybackslash}l}
% \toprule
% \multicolumn{1}{c}{\textbf{Google Purposes}} & $\rightarrow$      & \multicolumn{1}{c}{\textbf{Apple Purposes}} \\ \midrule
% Approximate Location  &$\rightarrow$  & Coarse Location  \\ \midrule
% Race and Ethnicity   & $\rightarrow$ & Sensitive Info\\ \midrule
% Sexual Orientation  & $\rightarrow$ & Sensitive Info \\ \midrule
% d &  &d \\ \midrule
% Personalization  & $\rightarrow$ & Personalization \\ \midrule
% Account Management       & $\rightarrow$ & N/A  \\  \midrule
% Developer Communication  &  & \\
% \bottomrule
% \end{tabular}}
% \caption{Table showing the common mapping from Data Safety Card and Apple Privacy Label}
% \end{table}


It is worth noting that the datatype and purpose tags used in the two labels can be used to denote different concepts. For example, in APL, \texttt{App functionality} also includes fraud prevention and implementing security measures, whereas the data safety section has separate tags for app functionality and fraud prevention and security measures. For the purposes of this analysis, we combine these two purposes into \textit{App functionality} to create a common map. Since APL does not have any tags for \textit{Account Management} and \textit{Dev. Communicaitons}, we removed them from DSS for this comparison. After taking the intersection of the available datatypes and purposes, we end up with 4 purpose categories and 26 datatypes. A complete mapping for classes from DSS to APL is shown in \cref{appendix:mapping}.
%There are also data types which are not common in the labels such as Files and Docs, Calendar etc. For our comparison, we remove these data types, along with the ambiguous data types such as \textit{Other info} in personal information and \textit{Other actions} (in App interactions). ~\cref{table:common_map1} and ~\cref{table:common_map2} provides a complete mapping that we have used to compare.

\subsection{Findings}
We compare the self-reported privacy practices of the 100K apps that are cross-listed on both the platforms and have privacy labels. Specifically, we ask the following questions: a) How does the high-level practice of data collection compare between the two labels? and b) Is the purpose for using datatypes consistent between the two labels?

% \begin{itemize}[leftmargin=*, align=parleft, labelsep=4mm]
%     \item[\textbf{Q1}] How does the high level practice of data collection compare between the two labels?
%     \item[\textbf{Q2}] When the apps uses datatypes, is the purpose for using datatypes consistent between the two labels?
% \end{itemize}

\subsubsection{Comparison of Data Collection}
 To compare how many apps do not collect data, we rely on the \textit{Data Not Collected} tag for the iOS platform and \textit{Data Shared} and \textit{Data Collected} tags for the android platform. 
 % Note that here we use both data shared and data collected tags for android platform because iOS platform denotes both the practices with 
 % does not make a distinction between the two, and the \textit{Data Not Collected} refers to both collection and sharing. 
 % We  emphasize here that due to the reason mentioned above, it is not possible to compare the data sharing practices between the two labels. 
We find that a total of \nnumber{22K(~22\%)} apps report different data collection practices on the two platforms. Of these apps, \nnumber{42\%} of the apps report collecting data on android while \nnumber{58\%} of the apps report collecting data on the iOS platform. Examining these apps further, we find that \nnumber{18\%} of these apps have more than 100k downloads, and \nnumber{5\%} has over 1M downloads indicating that even popular apps have this inconsistency. For example, ~\textit{KineMaster - Video Editor} a video editing app with over 400M+ downloads on Google Play Store states that they do not collect any data in the Play store but states in App Store that they do collect sensitive data such as \textit{Location} and \textit{Identifiers}.

The inconsistency in self-reported data collection practices as indicated by the inaccuracies in privacy labels undermines the credibility of the Privacy Label framework. This poses a significant concern for users, as they may base their decisions on inaccurate information, thereby increasing their privacy and security risks.

% The inconsistency suggests that at least one of the privacy labels is inaccurate which is concerning as this indicates that developers are not consistent in self reporting their data collection practices, which in turn erodes on the credibility of the Privacy Label framework. If the privacy labels are inaccurate, then privacy-concious users might be at higher privacy and security risk as they may make decisions about which app to use based on false information.

\subsubsection{Comparison of Fine-grained Practice}
% Next, we investigate whether the apps mention the same purpose for using datatypes in the privacy labels across the platforms. To perform this analysis, we only consider the apps where both the labels report collecting data, resulting in a set of \nnumber{yy} apps.
%removed the 50K apps identified above from this analysis as our objective is to analyze fine-grained inconsistencies across the two privacy labels when both the apps are collecting data. 

%Our assumption here is that same app will have the same features across the platforms and therefore will request same datatypes on both the platforms for same purpose. We note that neither of the platforms disallow developers from collecting or using a particular data type. The underlying operating systems might provide more controls to the user, but the developers are not prohibited from accessing any resources. Further, we note that we removed the 50K apps identified above from this analysis as our objective is to understand 

\begin{figure}[t]
  \centering
  \includegraphics[width=\columnwidth]{figures/good_pdfs/datatype_inconsistency2_cropped.pdf}
  \caption{Distribution of inconsistent apps with datatypes. Each datatype is normalized with the number of apps using that particular datatype on either platform. Note that we have omitted some datatypes here for brevity. The full distribution can be found in the ~\cref{fig:inconsistency_app_no_thresh} in the Appendix}.
  \label{fig:inconsistency}
\end{figure}

We compare fine-grained practices along two dimensions: 1) DataType where we check whether the privacy labels report collecting/sharing the same datatypes; and 2) DataType-Purpose pairs where we compare the common datatype-purpose pair in the two labels. If there is at least one datatype-purpose pair that is not present in both sets, we treat the app as inconsistent for that datatype-purpose pair. We also tag instances as inconsistent where the datatype-purpose pair is present in one of the labels and is missing from the other. For example, if the DSS of an app has \textit{(Location - Personalization)}, while the APL has \textit{(Location - App Functionality)}, then we treat the app as inconsistent. Similarly, if the DSS of an app has \textit{(Location - Personalization)}, while the APL has \textit{(Location - App Functionality)} and \textit{(Location - Personalization)}, we still treat the app as inconsistent as the tags \textit{(Location - App Functionality)} is not common in both. 

Across the cross-listed apps that have privacy labels, we find that at least \nnumber{60\%} of the apps have at least one inconsistency. For example, in \textit{Tiktok}, DSS states that they collect the contact list of the users for `Advertising and Marketing' purposes, but APL states that the app does not collect a contact list. 
\cref{fig:inconsistency} shows the inconsistency in datatypes across the two platforms. We find that \textit{Sensitive Information}, \textit{Browsing History}, and \textit{Emails or Text Messages} have the highest inconsistencies across the two platforms.  From Fig.~\ref{fig:inconsistency}, we observe that DeviceID and Product Interactions are the data counts with the highest inconsistencies. We also observe that \textit{Precise Location} and \textit{Coarse Location} is inconsistent with \textit{Advertising} implying that at least in one of the labels, location is used for advertising, raising privacy concerns for the users.

% Analyzing inconsistencies in purpose \rishabh{Add this when the plot is ready}. 

% We find that at least \red{70\%} of the apps have at least one inconsistency between the two privacy labels. Fig.~\ref{fig:inconsistency} shows the distribution of inconsistent apps with data types and with purpose. We find that \textit{App functionality} is the most common purpose for which inconsistencies arise. For Datatypes \textit{DeviceID} is the most common data type for which there are inconsistencies. For example, <AAP A>, a popular app with XX downloads has ...
% Only looking at the popular apps, we see a similar pattern with App Functionality and DeviceID being the most common purpose and datatype respectively, with inconsistencies. 

To analyze datatype-purpose inconsistencies, we show the normalized heatmap of inconsistent apps in \cref{fig:heatmap}. Normalization is done for each cell block in the heatmap, \textit{i.e.}, for each datatype-purpose pair, we normalize with the total number of apps that have that datatype-purpose pair. 
%Thus, each cell block in \cref{fig:heatmap} denotes the fraction of apps in which the corresponding datatype-purpose pair is consistent. 
We find that \textit{Fitness} and \textit{Sensitive Information} when used for \textit{Advertising or Marketing} are frequently inconsistent. The plot shows that even though \textit{Sensitive Information} and \textit{Fitness} data are not collected very often (Fig.~\ref{fig:inconsistency}), when they are collected, they are often inconsistent in privacy labels across two platforms. On the other hand, \textit{Credit Information} and \textit{Financial Information} have the least number of inconsistencies, which is encouraging considering the sensitive nature of this information.

% Fig.~\ref{fig:heatmap} shows the distribution of apps that are inconsistent across datatype and purposes, without distinguishing between where the mismatch occurs. To understand how purposes are being matched incorrectly, we show the confusion matrix for a following datatypes - \textit{Device ID}, \textit{Location} (Coarse and Precise combined), \textit{Performance Data}, \textit{Product Interaction}, \textit{Sensitive Information} and \textit{Purchase History}. 

% \todo[inline]{Add the figures, describe them and give some examples}

% \paragraph*{\textbf{Case Studies}} Here, we perform a qualitative case study with two genres that normally are associated with sensitive information. Our objective is to understand the impact and reach of these inconsistencies. Specifically, we focus on the following two genres: Medical, and Finance. We observed that for both these genres the most confusing data type was Location. Moreover, we 

% In fact, when taking a large scale view across genres, we observe that apps systematically collect more data in google than in apple as shown in the confusion matrices in Appendix. ~\todo[inline]{add cnf mats to the appendix and refer here}


% \paragraph*{\textbf{Discussion}} Earlier, we made an assumption that same apps with same features will access same datatypes on both the platforms.  We note that in the event that our assumption were wrong, there would be a systematic skew in the data in favor of one platform. For example, if it was difficult for developers to access location for the users in iOS, then we would observe that for apps that are accessing location on the android platform, the corresponding labels for iOS platform would have them as missing. However, as shown in Fig.~\ref{fig:location_confusion}, apps accessing location on the android platform consistently get the purpose confused when reporting on the iOS platform. We observed similar behavior across all common data types



% \begin{figure}
%   \centering
%   \includegraphics[width=\columnwidth]{figures/good_pdfs/credit_info.png}
%   \caption{Confusion metric for Credit Information. Each row shows what the purpose in apple for credit information was mislabeled in Google whereas each column shows what the purpose in Google was mislabeled in Apple. Takeaway: When the developers mentioned app functionality in google, a lot of times, it was confused with advertising or analytics or personalization, indicating that the frameworks are not working as they are supposed to, i.e. the users might not be getting accurate information about privacy practices of the apps.}
%   \label{fig:methodology}
% \end{figure}

% \begin{figure}
%   \centering
%   \includegraphics[width=\columnwidth]{figures/good_pdfs/payment_info.png}
%   \caption{Confusion metric for Payment Information. Each row shows what the purpose in apple for payment information was mislabeled in Google whereas each column shows what the purpose in Google was mislabeled in Apple.}
%   \label{fig:methodology}
% \end{figure}

% \begin{figure}
%   \centering
%   \includegraphics[width=\columnwidth]{figures/good_pdfs/precise_location.png}
%   \caption{Confusion metric for precise location}
%   \label{fig:methodology}
% \end{figure}

% \begin{figure}
%   \centering
%   \includegraphics[width=\columnwidth]{figures/good_pdfs/purchase_history.png}
%   \caption{confusion metric for purchase history}
%   \label{fig:methodology}
% \end{figure}



\subsection{Takeaway}
In this section, we analyzed the consistency of privacy labels for the same apps across the two platforms. We find that 60\% of the cross-listed apps had at least one inconsistency between APL and DSS. We further find that inconsistencies are highest for \textit{Sensitive Information}, \textit{Browsing History}, and  \textit{Emails or Text Messages} datatypes. Through a detailed analysis of datatype-purpose inconsistencies, we find that \textit{Emails and Text Messages} when used for Advertising results in inconsistencies 96\% of the time, indicating a concerning problem with disclosure of practices in privacy labels.

\section{Discussion}
In this paper, we investigated the consistency of privacy labels with privacy policies and labels on other platforms. Our findings suggest that there is a significant degree of inconsistency in privacy labels. Overall, there is a need for greater consistency in the way that privacy practices are disclosed to users, both within and between platforms. In this section, we discuss the implications of our findings and suggest potential solutions for improving the transparency and consistency of privacy practices. We also discuss the limitations of our study.\smallskip

\noindent\textbf{Comparison between the two labels.} We analyzed both the Data Safety Sections and the Apple Privacy Labels and find that the two labels cover different aspects of data practices. While both labels provide information about the types of data that apps collect, Apple's privacy label does not distinguish between data collection and data sharing. Apple's privacy label is more explicit about certain aspects of data practices, such as linkability, third-party advertising, and tracking, whereas data safety sections lack these details, but does inform the users about the safety of their data (\textit{Data Encryption}) and the choices that they have with developers (\textit{Data Deletion Option}). These practices may be of particular interest to the users in light of the GDPR~\cite{linden2018privacy}, which requires companies to provide a clear and explicit purpose for the collection and use of personal data. The regulations like the GDPR and the CCPA also provide the right to delete the data to the users, which is covered in Data safety forms but not in the Apple privacy labels.

The comparison between the two labels highlights the importance of considering multiple sources of information when evaluating the data practices of apps. By combining the information provided by both labels, users can make more informed decisions about their privacy and the apps they choose to use.\smallskip

\noindent\textbf{Usability of Privacy Policies For Apps.} Privacy policies have been used as a default framework for notice and choice to users. Our analysis reveals that many developers have several products, including websites, Internet of Things (IoT) devices, and applications. However, it is common for these products from the same developer to have the same privacy policy, even if they collect data in different ways. This provides inaccurate information, as the privacy policy itself may not accurately convey the privacy practices of a specific product. This can be addressed by having separate privacy policies for each product or by clearly identifying the specific practices that apply to each product within a single privacy policy. Failing to do so may lead to misunderstandings and mistrust among users, and may also violate privacy regulations.\smallskip

\noindent\textbf{Inconsistencies in disclosed practices across platforms.} Our findings indicate that there are inconsistencies between the privacy labels in the Apple Privacy Labels and the Google Data Safety Sections for the same apps. One possible reason for these inconsistencies is the confusing framework for privacy labels. While previous research~\cite{li2022understanding} has shown that privacy labels are useful for both developers and users, it also highlighted that filling privacy labels is perceived as challenging extra work. On top of that, developers are also unclear about definitions which can result in confusion and ultimately, inaccurate privacy labels. This confusion can be compounded by the fact that different platforms may use different terminology to describe similar practices. For example, in Apple's privacy label, the term \textit{tracking} is used when data collected is linked with third-party data for advertising purposes or when data is shared with a third party, which can be confusing to the developers, even when they are asked to pay close attention~\cite{li2022understanding}.

Another possible reason for the inconsistencies we observed is the casual attitude of some developers toward disclosing their data practices. Some developers may not fully understand the data practices of their own apps, or may not prioritize accurately disclosing this information to users. Finally, the platforms lack consistency checks to ensure that the information provided in the privacy labels is accurate. Without these checks, it is possible for developers to provide misleading or incomplete information about their data practices, just to meet the requirements.

We note that these inconsistencies can have serious consequences for users, as they may be confused about the privacy practices of the apps they use. If the practices disclosed in the privacy labels are inaccurate, it can reduce the efficacy of these labels as a tool for helping users make informed decisions about their privacy. Even worse, it could induce a false sense of security in users, who may assume that their data is being handled in a certain way when it is not.\smallskip

\noindent\textbf{Usability of Privacy Labels.} Even though our analysis finds inconsistencies between privacy labels and privacy practices, evidence suggests that privacy labels generally carry more specific information about the practices. They include information about the types of data that an app collects, how the data is used, and whether it is shared with third parties. This information can be very useful for users who are concerned about their privacy and want to ensure that they are only using apps that respect their personal data.

However, the accuracy of privacy labels is not guaranteed. While developers are required to disclose their data practices in order to obtain a privacy label, there is no guarantee that the information they provide is accurate or complete. As such, it is important for platforms to recognize that developers may not always be honest about their data practices. Therefore, it is necessary to have systems in place to verify the accuracy of privacy labels and to hold developers accountable for any discrepancies. This is particularly important because the false labels can create a false sense of security among the users. 

One potential model for regulating privacy labels is a system similar to the one used for food nutrition labels, which are regulated by the Food and Drug Administration (FDA). A regulatory body could be established to oversee privacy labels and ensure that they are accurate and consistent. This could help to build trust among users and encourage developers to be more transparent about their data practices.\smallskip

\noindent\textbf{Limitations.} Extracting privacy practices using automated analysis comes with several limitations. First, the framework used here treats privacy policies as segmented text, missing out of relations between different segments. This can potentially result in internal contradictions, as shown in ~\cite{andow2019policylint}. Second, the classifiers used to extract privacy practices can introduce errors, which can then propagate through the pipeline and induce uncertainty in the inconsistency rates. We do however note that the error rate of our classifiers is significantly less than the inconsistency rate obtained, indicating that the results presented in the paper are valid. 

% \begin{itemize}
%     \item Inconsistency via data flow analysis
%     \item Limitations of Automated Policy analysis, specially for practices which can long range relations
% \end{itemize}

% \begin{itemize}
%     \item Comparison between the two labels: Talk about how the two labels combined can more information.\\
%     Subpoint: The two labels are covering very different aspects data practices. Apple does not distinguish between colelction and sharing, whereas google is not explicit about linkability, third party advertising, tracking etc. Connect these to the GDPR
%     \item possible reasons for the discrepancy, both for privacy policies and for labels
%     \item Talk about privacy policies not being adequate for conveying information about the practices and they cover vastly different informations. This is sort of a motivation for using privacy labels, but the information provided there is not reliable as well. 
%     \item Limitations: talk about automated privacy policy analysis.
%     \item Future direction: Use privacy labels to automatically edit privacy policies and generate a more comprehensive/readable privacy policies
% \end{itemize}
% \newpage
% page 7
% \newpage
% page 8

% \newpage
% \subsection{Limitation}
% % \xiaodong{ablation study of the temporal-conv, failed case when changing bird to the flying dinosaur.}
% % \chenyang{Move this section to supp? It does not present our contribution. }
% % page 8.5
% % \newpage

% While our method achieves impressive results in video editing, it still has some limitations. During shape editing, since the motion is leaned by the one-shot video diffusion model~\cite{tuneavideo}, it is difficult to generate totally new motion~(\eg, `swimming' $\xrightarrow{}$ `fly' ) or very different shape~(\eg, `swan' $\xrightarrow{}$ 'pterosaur'). We believe a stronger video diffusion model might solve these problems.
% (b) Our editing capacity is bounded by the performance of pretrained-model, which bring obstacles to generate diverse movie styles as gen1~\cite{gen1} (\eg, claymation)


\section{Conclusion}
In this paper, we propose a new text-driven video editing framework \texttt{FateZero} that performs temporal consistent zero-shot editing of attribute, style, and shape. 
We make the first attempt to study and utilize the cross-attention and spatial-temporal self-attention during DDIM inversion, which provides fine-grained motion and structure guidance at each denoising step.
A new Attention Blending Block is further proposed to enhance the shape editing performance of our framework.
Our framework benefits \textbf{video} editing using widely existing \textbf{image} diffusion models, which we believe will contribute to a lot of new video applications. 

\noindent \textbf{Limitation \& Future Work.}
While our method achieves impressive results,
% in video editing, 
it still has some limitations. During shape editing, since the motion is produced by the one-shot video diffusion model~\cite{tuneavideo}, it is difficult to generate totally new motion~(\eg,`swim'$\xrightarrow{}$`fly' ) or very different shape~(\eg,`swan' $\xrightarrow{}$`pterosaur'). We will test our method on the generic pretrained video diffusion model for better editing abilities.

% We leave the application of our techniques to other pretrained image diffusion models~\cite{controlnet} as future work.
\label{sec:conclusion}


    
% \newpage
% \bibliographystyle{IEEEtranS}
% argument is your BibTeX string definitions and bibliography database(s)
\bibliographystyle{abbrv}
\bibliography{ref.bib}

% \input{src/A200_Extra_Experiments}
\appendix

\section{Privacy Policy Analysis}
\subsection{Privacy Policy Taxonomy}
\label{appendix:taxonomy}

\textbf{Limitations of the OPP-115 Taxonomy} Figure.~\ref{fig:opp_taxonomy} shows the privacy taxonomy proposed by Wilson et al.~\cite{wilson2016creation}. The top-level defined high-level privacy categories whereas the lower level defined a set of privacy attributes that can take a particular set of values. Additionally, some examples of attribute-value pairs are shown such as Information Type and Purpose. Note that several lower-level attributes are shared across the high-level categories. 
\begin{figure*}
\hspace*{-2.2cm} 
  \centering
  \includegraphics[width=1.25\columnwidth]{figures/good_pdfs/opp_dataset-crop.pdf}
  \caption{The privacy policy taxonomy by Wilson et al. ~\cite{wilson2016creation}}
  \label{fig:opp_taxonomy}
\end{figure*}

Prior works~\cite{harkous2018polisis, srinath2020privacy} have used the OPP-115 taxonomy and the associated dataset to build machine-learning classifiers that tag segments of the policy with the labels from the taxonomy. However, there are two limitations to directly using the taxonomy (and existing frameworks such as Polisis~\cite{harkous2018polisis}) to compare privacy practices between privacy labels and privacy policies. First, the OPP-115 taxonomy was developed for privacy policies of websites, which is vastly different than the ecosystem of applications (both Android and iOS). In particular, the applications have access to sensitive data types, which are present in the privacy labels. This taxonomy, while having some overlap with the APL and DSS privacy labels, does not cover such app-specific data types. For example, \texttt{app activity}, a data category covering users' interactions within the application, is not covered in the taxonomy. Second, the OPP-115 dataset has limited annotations for the lower-level attributes that overlap with the private labels. For example, \textit{Encryption in Transit}, which is a separate practice covered in Data Safety Sections, only has less than 100 labeled instances in the OPP-115 dataset. 

\begin{figure}
  \centering
  \includegraphics[scale=0.6]{figures/good_pdfs/privacy_label_taxonomy.pdf}
  \caption{Privacy Label Taxonomy}
  \label{fig:privacy_taxonomy}
\end{figure}

We address these limitations by incorporating the missing labels to the existing OPP-115 taxonomy. We derive a \textit{Privacy Label Taxonomy} (\cref{fig:privacy_taxonomy}) as a union of a subset of the original OPP-115 taxonomy with the new labels from APL and DSS. To build the taxonomy, we first identify the categories from the taxonomy that are relevant to privacy labels, thus creating a subset of the original taxonomy. We then add the missing categories to get the new taxonomy. 

\medskip
\noindent \textbf{Identifying relevant categories from OPP-115} 
As discussed in Sec.~\ref{sec:background}, privacy labels consist of high-level privacy practices, data categories, and purposes for the use of data. The high-level categories \textit{First-party-data-collection} and \textit{Third-party-sharing-collection} from the OPP-115 taxonomy are relevant as they map directly to Data Safety Sections' \textit{Data Collection} and \textit{Data Sharing} privacy types. Further, APL covers the first-party collection and sharing practices implicitly through \textit{Data Linked to You} and \textit{Data Not Linked to You}. Similarly, the attribute level categories \textit{Purpose}, \textit{Data Type}, and \textit{Identifiable} are relevant. 


For example, in Apple Privacy Label (APL), the \textit{Data Used to Track You} privacy type includes the data that is linked with third-party data for targeted advertising. It also includes cases when the data is shared with a data broker. Note here that linking can be done by both the app developers (by using data obtained from a third party) or by sharing the data with a third party. Thus, this privacy practice can be represented with \textit{Advertising or Marketing} purpose of \textit{First-party-collection-use} and \textit{Third-party-sharing}. At this stage, we drop the categories absent in the privacy labels. For example, \textit{Policy Change} is a high-level category in OPP-115 which is not present in the \textit{Privacy Label Taxonomy}. 

\medskip
\noindent \textbf{Adding New Categories} As indicated earlier, the OPP-115 taxonomy misses some of the lower-level data categories and purposes. We add these missing elements and adapt the OPP-115 taxonomy to \textit{Privacy Label Taxonomy}. 

Apart from the high level categories from the taxonomy, we also add two high level categories: \textit{Data Deletion Option} and \textit{Encryption in Transit}. Both the categories are part of \textit{Security Practices} privacy type from DSS. \textit{Data Deletion} corresponds to when the app ``Provides a way for you to request that your data be deleted, or automatically deletes or anonymizes your data within 90 days''. As there is no specific way to get this information from the taxonomy, we create a separate high level practice for Data Deletion. For \textit{Secure Data Transfer}, there is low level element in the taxonomy that covers the practice, however, since the other categories from the taxonomy in the hierarchy are not related, we add \textit{Secure Data Transfer} as a high level category. Also note that since there were less than 100 annotations for this category, we also perform additional annotations and increase the dataset size.

\begin{figure}[htbp]
  \centering
  \includegraphics[width=\columnwidth]{figures/good_pdfs/datatype_inconsistency2_no_thresh_cropped.pdf}
  \caption{}
  \label{fig:inconsistency_app_no_thresh}
\end{figure}

% \begin{figure*}[h]
%   \centering
%   \includegraphics[width=2\columnwidth]{figures/good_pdfs/heatmap2_cropped.pdf}
%   \caption{}
%   \label{fig:inconsistency_app}
% \end{figure*}

\subsection{Annotation Setup}
\label{appendix:annotation_setup}

\noindent \textbf{Creating Annotation Set} For our \textit{Privacy Label Taxonomy}, we were able to have the data missing from OPP-115 for \red{13} elements. Curating the candidate set for missing categories is a major challenge due to label imbalance. To address this issue, we follow the approach used by Harkous et al.,~\cite{harkous2022hark} and use the task of \textit{Natural Language Inference} (NLI) to curate the candidate set. The NLI tasks consist of a hypothesis and a premise, and the objective is to determine if the hypothesis is true (\texttt{entailment}), false (\texttt{contradiction}) or undetermined (\texttt{neutral}) given the premise~\cite{maccartney2009natural}. For example, if the premise is: \textit{``Your data is safely and completely removed from our servers or retained only in anonymized form.''} and the hypothesis is \textit{``Data deletion is being discussed''}, then this instance will receive an entailment. On the other hand, if the hypothesis were \textit{``Policy change is being discussed''}, then the label would be neutral. This method of using NLI-based sampling to reduce the annotation effort has been shown to be effective by Harkous et al.~\cite{harkous2022hark}.

We start by creating a hypothesis for each of the missing categories that we have. For example, for \textit{Data Deletion Option}, we created two hypotheses: ``Data deletion is being discussed'' and ``Data Anonymization is being discussed''. For the NLI task, we used the T5-Large model checkpoint from \texttt{Huggingface}. This model is already trained on MultiNLI task~\cite{2020t5} which consists of a multi-genre dataset covering a large variety of domains.  Next, we run the NLI model and get weak labels for all the missing categories. Note that these are weak labels that are later manually annotated to create the training set.

\medskip
\noindent \textbf{Annotation Details} Using the NLI sampling approach, we curated a candidate set with 2000 segments for each of the missing categories. These segments are roughly balanced based on the weak labels assigned by the NLI model. For each class, we then randomly sample 500 segments to annotate. Two of the authors annotated the segments and created the training set.

The annotation was performed using the label studio framework~\cite{labstud}. The framework supports not only simple natural language processing tasks but also sophisticated labels such as taxonomies and sentence highlightings. The framework also supports active learning with the capability of integrating a backend machine learning classifier of one's taste in order to facilitate annotation.

The label studio server was deployed in an internal network, where the two authors simultaneously worked on annotating and creating the training set. Figure \ref{fig:anno_setup} shows the annotation setup used by the authors.

\begin{figure}[htbp]
  \centering
  \includegraphics[width=\columnwidth]{figures/good_pdfs/anno_setup.png}
  \caption{Label Studio Annotation Setup}
  \label{fig:anno_setup}
\end{figure}

After the annotation step, the dataset was converted into a data frame consisting of the text column and a binary indicator column for each of the categories, to prepare for the training.


\subsection{Training Setup}
\label{appendix:training_setup}
Large language models like BERT~\cite{devlin-etal-2019-bert}, T5~\cite{2020t5} etc have shown remarkable performance using small training sets. Thus, for our purposes, we use the \texttt{distilbert-base-uncased}~\cite{sanh2019distilbert} model consisting of 67 million parameters. This model is the distilled version of BERT~\cite{sanh2019distilbert}. It has 40\% fewer parameters, can run 60\% faster and performs only slightly worse (\textasciitilde 5\%) than the original \texttt{bert-base-uncased} model on several natural language tasks. Additionally, we also perform domain adaptation by pre-training the DistilBert model on privacy policy text with the Masked Language Model (MLM) task. In particular, we pre-trained the model with the default hyperparameters, with a batch size of 256 for 24800 steps on a single NVIDIA A100 GPU.

We then use the new pre-trained model to train the category classifiers for the Privacy Label Taxonomy. We use a classification head on the model after adding a linear classification. For classification, the data annotated by the authors is split into two parts: testing (20\%) and training sets (80\%).


\section{Data Practices in Privacy Labels}
\label{appendix:landscape_details}
In this section, we look into the distribution of all the data types as shown in \cref{fig:inconsistency_app_no_thresh}. From this figure, we observe that the six categories--App Activity, App Info \& Performance, Device IDs, Financial Info, Location, Personal Info--are reported to be the most collected in Google, while the six categories--Contact Info, Diagnostics, Identifiers, Location, Purchases, Usage Data and User Content--are reported to be the most collected in Apple.

\begin{figure*}
\hspace*{-2.2cm} 
  \centering
  \includegraphics[width=1.25\columnwidth]{figures/good_pdfs/google_apple_highlevel_category_all_cropped.pdf}
  \caption{Distribution of data categories for high level
practices for apps in Play Store (top) and App store (bottom).}
  \label{fig:high_level_data_dist}
\end{figure*}



 
\section{Developer Study}
\label{app:dev-study}
For the Developer Study (\cref{sec:developer}) we sent emails to developers in 3 different categories: (A) apps stating that they encrypt data without collecting or sharing data, (B) apps changing their practice from not collecting/sharing data to collecting/sharing data, and (C) apps changing their practices from collecting/sharing data to not collecting/sharing data.

For category (A) we used the following template:
\begin{enumerate}
\item[] \textit{We hope this email finds you well. We are researchers at <LAB\_NAME> and have been using your app, <APP\_NAME>, in our recent studies. We have noticed that in the data safety section of your app, it states that you encrypt data. However, we have also noticed that your app does not collect or share data.}

\textit{We are reaching out to ask if you could clarify this for us. We are trying to better understand the data safety section implemented in your app. We appreciate any information you can provide.}

\textit{Thank you for your time and we look forward to your response.}
\end{enumerate}
\section{Comparing Privacy Policies with Privacy Labels}
In this section, we provide details about how to obtain privacy practices present in the policies. We first note that \textit{Data Category} and \textit{Purpose} are the lowest levels in the taxonomy such that there are classifiers for each data category and purpose. So, extracting lower-level practices from policies is straightforward. Furthermore, we note that we have added classifiers for \textit{Encryption in Transit} and \textit{Data Deletion Option} separately. Additionally, we have two high-level classifiers, namely \textit{First-Party-Collection} and \textit{Third-party-collection} that capture segments that refer to data collection by first parties and data sharing to third parties, respectively.

To extract the remaining high-level practices, we follow the mapping shown in  \cref{tab:policy_to_label}. Specifically, as shown in \cref{tab:policy_to_label} we use multiple data category classifiers in conjunction with high-level classifier (First-party-collection-share/Third-party-collection-share) to indicate if the policy mentions data collection or data sharing. This result can be directly compared with the categories present in Google's Data Safety Section.

To match the policy to the Apple Privacy Label, we use an additional classifier: \textit{Identifiability}. This represents if the data being collected is anonymous or not. If the data is anonymous then we equate it to the \textit{Data Not Linked to You} label else the \textit{Data Linked to You} label. For example, To obtain whether a policy is collecting \textit{Location} under \textit{Data Linked To You}, we check whether there are any segments where the lower level \textit{Data Category} classifier tags the segments to contain \textit{Location} information. Then we check  whether these segments also have either \textit{First-party-collection} or \textit{Third-party-collection-share} tags in combination with \textit{Identifiability-identifiable} tag. 
\begin{figure}[ht]
    \centering
    \includegraphics[scale=0.17]{figures/good_pdfs/purpose_label_policy_cropped.pdf}
    \caption{Inconsistencies between privacy policies and DSS, purpose based}
    \label{fig:purpose_inconst_google}
\end{figure}


\section{Mapping from DSS to APL}
\label{appendix:mapping}

In \cref{table:common_map1} and \cref{table:common_map2} we show how we convert the datatypes of DSS to those of APL. We do these conversions based on the definitions provided by Google and Apple respectively. 

\begin{figure}[ht]
    \centering
    \includegraphics[scale=0.17]{figures/good_pdfs/apple_datatypes_inconsistency_cropped.pdf}
    \caption{Inconsistencies between privacy policies and APL, datatypes based}
    \label{fig:datatype_inconst_apple}
\end{figure}
% \clearpage
\begin{figure}[ht]
    \centering
    \includegraphics[scale=0.17]{figures/good_pdfs/datatypes_label_policy_cropped.pdf}
    \caption{Inconsistencies between privacy policies and DSS, datatypes based}
    \label{fig:datatype_inconst_google}
\end{figure}
% \begin{figure}[htbp]
%   \centering
%   \includegraphics[width=\columnwidth]{figures/good_pdfs/data_safety_card_trends2_cropped.pdf}
%   \caption{Data Safety Trends}
%   \label{fig:trend_with_time}
% \end{figure}

\begin{figure}[p]
  \centering
  \includegraphics[scale=0.17]{figures/good_pdfs/collected_shared_vs_category2_cropped.pdf}
  \caption{Plot showing what data category is collected or shared in Google Data Safety cards}
  \label{fig:google_datatype_distribution}
\end{figure}

\begin{figure}[ht]
  \centering
  \includegraphics[scale=0.8]{figures/good_pdfs/privacyTypes_vs_category_cropped.pdf}
  \caption{Plot showing the data categories used for high-level privacy practices in APL. }
  \label{fig:methodology}
\end{figure}



\begin{figure}[ht]
  \centering
  \includegraphics[width=\columnwidth]{figures/good_pdfs/purposes_google_apple_cropped.pdf}
  \caption{Plot showing the distribution of purpose for high-level privacy practices in APL}
  \label{fig:purpose_methodology}
\end{figure}


% \begin{figure*}
%   \centering
%   \includegraphics[width=2\columnwidth]{figures/good_pdfs/purpose_datatype_paircount_google_cropped.pdf}
%   \caption{Plot showing what data category is collected or shared in Google Data Safety cards}
%   \label{fig:google_datatype_purpose_distribution}
% \end{figure*}


\begin{table}[htbp]
\footnotesize
    \centering
\begin{tabularx}{\columnwidth}{|p{3.3cm}|>{\centering\arraybackslash}X|>{\centering\arraybackslash}X|>{\centering\arraybackslash}X|>{\centering\arraybackslash}X|}
\hline
\textbf{Data Category Classifiers}  & \textbf{High Level Classifier}  & \textbf{Google Data Safety Section} & \textbf{Classifier} & \textbf{Apple Privacy Labels} \\ \hline
\multirow{2}{3.3cm}{App Activity, App Info and Performance, Sensitive Info, Location, Health and Fitness, ...} & \multirow{2}{3cm}{First-party-collection-share} & \multirow{2}{3cm}{Data Collection}  & Identifiability (Identifiable) & Data Linked to You \\ \cline{4-5} 
 & & & Identifiability (Anonymous) & Data Not Linked to You \\ \hline
\multirow{2}{3.3cm}{App Activity, App Info and Performance, Sensitive Info, Location, Health and Fitness, ...} & \multirow{2}{3cm}{Third-party-collection-share} & \multirow{2}{3cm}{Data Sharing} & Identifiability (Identifiable) & Data Linked to You \\ \cline{4-5} 
& & & Identifiability (Anonymous) & Data Not Linked to You        \\ \hline
\end{tabularx}
\caption{This table shows the bottoms-up approach to get the high-level classification for Google's Data Safety Section and Apple Privacy Labels}
\label{tab:policy_to_label}
\end{table}

\begin{table}[htbp]
\centering
\begin{tabular}{lccc}
\toprule
\textbf{Category} & \textbf{CNN ~\cite{o2015introduction}} & \textbf{BERT ~\cite{devlin-etal-2019-bert}} & \textbf{Here}\\
\midrule
\rowcolor{aliceblue}First-party-collection-share  & 82 & 91 & 98 \\
Third-party-sharing-collection & 82 & 90 & 96 \\
\rowcolor{aliceblue}Identifiability & 77 & 91 & 97\\
Does-does-not & 86 & 93 & 96 \\
\rowcolor{aliceblue}Encryption-in-transit  & N/A & N/A & 99 \\
Data Deletion Option  & N/A & N/A & 91 \\
\rowcolor{aliceblue} (DC) App Activity & N/A & N/A & 93 \\
(DC) App Info and Performance & N/A & N/A & 93 \\
\rowcolor{aliceblue} (DC) Sensitive Info  & N/A & N/A & 97 \\
(DC) Location  & N/A & N/A & 99 \\
\rowcolor{aliceblue} (DC) Health and Fitness  & N/A & N/A & 97 \\
(DC) Device or Other ID  & N/A & N/A & 94 \\
\rowcolor{aliceblue} (DC) Photos and Videos  & N/A & N/A & 96 \\
(DC) Web Browsing  & N/A & N/A & 96 \\
\rowcolor{aliceblue} (DC) Contacts  & N/A & N/A & 87 \\
(DC) Calendar  & N/A & N/A & 94 \\
\rowcolor{aliceblue} (Purp) Account Management  & N/A & N/A & 92 \\
(Purp) Developer Communication  & N/A & N/A & 93 \\
\rowcolor{aliceblue} (Purp) Personalization & N/A & N/A & 98 \\
\bottomrule
\end{tabular}
\caption{Classifiers's performance on the test set. Training was done on one GPU with early stopping.}
\label{tab:privacy_label_classifier_stats}
\end{table}

\begin{table}[htbp]
\footnotesize
\centering
\begin{tabular}{>{\arraybackslash}l>{\arraybackslash}l>{\arraybackslash}l}
\toprule
\multicolumn{1}{c}{\textbf{Google Purposes}} & $\rightarrow$      & \multicolumn{1}{c}{\textbf{Apple Purposes}} \\ \midrule
Advertising or Marketing  &$\rightarrow$  & Advertising or Marketing  \\ \midrule
\rowcolor{aliceblue}Analytics   & $\rightarrow$ & Analytics\\ \midrule
App Functionality   & $\rightarrow$ &App Functionality \\ \midrule
\rowcolor{aliceblue}\begin{tabular}[l]{@{}l@{}}Fraud prevention, Security, \\ and Compliance\end{tabular} & $\rightarrow$ & App Functionality\\ \midrule
Personalization  & $\rightarrow$ & Personalization \\ \midrule
\rowcolor{aliceblue}Account Management       & $\rightarrow$ & N/A  \\  \midrule
Developer Communication  & $\rightarrow$ & N/A  \\
\bottomrule
\end{tabular}
\caption{Table showing the common mapping from Data Safety Card to Apple Privacy Label}
\label{table:common_map1}
\end{table}

\begin{table}[p]
    \footnotesize
    \centering
    \begin{tabular}{>{\arraybackslash}l>{\arraybackslash}l>{\arraybackslash}l}
    \toprule
    \multicolumn{1}{c}{\textbf{Google DataType}} & $\rightarrow$ & \multicolumn{1}{c}{\textbf{Apple DataType}} \\ \midrule
    Approximate Location & $\rightarrow$ & Coarse Location\\ \midrule
\rowcolor{aliceblue}Precise Location & $\rightarrow$ & Precise Location\\ \midrule
Name & $\rightarrow$ & Name\\ \midrule
\rowcolor{aliceblue}Email Address & $\rightarrow$ & Email Address\\ \midrule
Address & $\rightarrow$ & Physical Address\\ \midrule
\rowcolor{aliceblue}Phone Number & $\rightarrow$ & Phone Number\\ \midrule
Race And Ethnicity & $\rightarrow$ & Sensitive Info\\ \midrule
\rowcolor{aliceblue}Political Or Religious Belief & $\rightarrow$ & Sensitive Info\\ \midrule
Sexual Orientation & $\rightarrow$ & Sensitive Info\\ \midrule
\rowcolor{aliceblue}User Ids & $\rightarrow$ & User Id\\ \midrule
User Payment Info & $\rightarrow$ & Payment Info\\ \midrule
\rowcolor{aliceblue}Credit Score & $\rightarrow$ & Credit Info\\ \midrule
Other Financial Info & $\rightarrow$ & Other Financial Info\\ \midrule
\rowcolor{aliceblue}Purchase History & $\rightarrow$ & Purchase History\\ \midrule
Health Info & $\rightarrow$ & Health\\ \midrule
\rowcolor{aliceblue}Fitness Info & $\rightarrow$ & Fitness\\ \midrule
Emails & $\rightarrow$ & Emails Or Text Messages\\ \midrule
\rowcolor{aliceblue}Sms Or Mms & $\rightarrow$ & Emails Or Text Messages\\ \midrule
Other In-App Messages & $\rightarrow$ & N/A\\ \midrule
\rowcolor{aliceblue}Photos & $\rightarrow$ & Photos Or Videos\\ \midrule
Videos & $\rightarrow$ & Photos Or Videos\\ \midrule
\rowcolor{aliceblue}Voice Or Sound Recordings & $\rightarrow$ & Audio Data\\ \midrule
Music Files & $\rightarrow$ & N/A\\ \midrule
\rowcolor{aliceblue}Other Audio Files & $\rightarrow$ & N/A\\ \midrule
Files And Docs & $\rightarrow$ & N/A\\ \midrule
\rowcolor{aliceblue}Calendar & $\rightarrow$ & N/A\\ \midrule
Contacts & $\rightarrow$ & Contacts\\ \midrule
\rowcolor{aliceblue}App Interactions & $\rightarrow$ & Product Interaction\\ \midrule
Other User-Generated Content & $\rightarrow$ & Other User Content\\ \midrule
\rowcolor{aliceblue}In-App Search History & $\rightarrow$ & Search History\\ \midrule
Other Actions & $\rightarrow$ & N/A\\ \midrule
\rowcolor{aliceblue}Web Browsing History & $\rightarrow$ & Browsing History\\ \midrule
Crash Logs & $\rightarrow$ & Crash Data\\ \midrule
\rowcolor{aliceblue}Diagnostics & $\rightarrow$ & Performance Data\\ \midrule
Other App Performance Data & $\rightarrow$ & Other Diagnostic Data\\ \midrule
\rowcolor{aliceblue}Device Or Other Ids & $\rightarrow$ & Device Id\\ \midrule
Other Info & $\rightarrow$ & N/A\\ \midrule
    \bottomrule
    \end{tabular}
    % \caption{Caption}
    % \label{tab:my_label}    
    \caption{Table showing the common mapping from Data Safety Card to Apple Privacy Label}
    \label{table:common_map2}
\end{table}


\end{document}

