\section{Introduction}
\label{sec:intro}
Privacy policies have traditionally been the primary means of conveying the privacy practices of a service to users. However, studies have shown that privacy policies are often ineffective due to readability and reachability issues, as users tend to avoid reading them due to their length and vagueness ~\cite{cate2010limits, gluck2016short}. Introduced by Kelly et al~\cite{kelley_labels}, the concept of privacy labels has gained traction in the tech industry, with Google introducing Data Safety Sections (DSS) and Apple introducing Apple Privacy Labels (APL) for all new and updated apps on the App Store.



% Prior research has examined the reactiveness of developers in implementing Apple Privacy Labels and analyzed the data collection practices of apps according to these labels ~\cite{balash2022longitudinal, li2022understandingios}, however, they have not provided a comprehensive analysis of the current status of privacy labels for the play store. 

Researchers have shown the benefit of privacy labels for users, making privacy practices more accessible~\cite{zhang2022usable}. However, prior work has also shown that inaccurate labels can exist due to the developer's knowledge gaps or resource limitations~\cite{li2022understanding}. 
Incorrect privacy labels can cause confusion and harm users by creating a false sense of security. Furthermore, inaccurate privacy labels can mislead users into downloading and using insecure apps, increasing their privacy risks. Therefore, it is crucial to investigate the accuracy and compliance of privacy labels in real-world scenarios, in order to determine how well they align with the actual data practices of apps.

Xiao et al~\cite{xiao2022lalaine} proposed a methodology to check for consistency of privacy labels by comparing practices in labels with privacy practices inferred by analyzing the dataflow using dynamic analysis.  One major limitation of flow-to-consistency analysis is that it requires dynamic analysis of apps which is hard to scale, as pointed out by Xiao et al~\cite{xiao2022lalaine}. 

In this work, we check the consistency of privacy labels using different approaches -- by comparing privacy practices reported in privacy labels with those present in privacy policies. We also compare privacy labels of the same apps across different platforms to gain an understanding of how developers report their apps' privacy practices. A major advantage of our approach is that using automated analysis, it can scale to a large number of apps. Thus, this paper aims to provide a comprehensive analysis of the current state of privacy labels and identify areas for improvement by asking the following research questions:

\begin{itemize}
    \item What practices are developers reporting in privacy labels? How do these practices evolve over time?
    \item How do the privacy practices present in privacy labels compare with the privacy policies? 
    \item Do apps have different practices across platforms?
    % \item Does an app’s privacy policy change since the introduction of the privacy nutrition labels;
\end{itemize}


To answer the above questions, we conduct a large-scale analysis of the privacy labels of apps listed on the Google Play Store and Apple App Store. We also conduct a developer study with android developers to understand their data safety section and highlight the challenges faced by them while working with privacy labels. Our analysis includes a comparison of the privacy practices mentioned in the privacy labels with those present in the privacy policies, as well as a comparison of the privacy practices across apps cross-listed on both platforms. 

We first start by developing a scraper for the Google Play Store and Apple App Store to collect metadata for over 2.5M apps on the Play Store and 1.3M apps on the App Store. We also periodically collected metadata for apps on the play store to track any changes made to an app's description, privacy policies, and data safety section. In addition to collecting each app's metadata, we also scraped privacy policies for apps on both platforms. Next, we automatically analyzed the privacy policies to extract privacy practices by developing a privacy label-centric taxonomy by adapting an existing privacy policy taxonomy. Specifically, we added missing elements and added more annotations for the new taxonomy. 
% We recognize that annotation cost can increase significantly for privacy policies, thus we proposed a bootstrapped approach that leverages advances in Natural Language Processing to minimize the annotation cost. 
We then compare these extracted practices with those present in the privacy label to perform a consistency analysis. Finally, we curate a dataset with apps cross-listed on both platforms and compare the privacy labels to understand how consistent developers are in disclosing their practices via privacy labels.



With this work, we make the following contributions:
% \todo[inline]{Mention the date of data? as in 2.2 as of the last}
\begin{itemize}
    \item We perform large-scale measurements of privacy practices reported in privacy labels across two major platforms - App store (n=\nnumber{1.38M}) and Google Play Store(n=\nnumber{2.4M}). We filter out apps with less than 1000 downloads for Google Play Store. This limits the number of apps on the Google Play Store to \nnumber{2.6M}.  We find that only \nnumber{50.2\%} of the apps provide privacy labels on the Google Play Store, whereas on the App Store, only \nnumber{69.2\%} of the apps contain privacy labels.
    \item We perform a longitudinal analysis for privacy labels on the Google play store and study the evolution of Data Safety Forms before and after the hard deadline imposed by Google. We find that app developers have changed data safety forms frequently. 
    \item We compare the data practices mentioned in the privacy policy with privacy labels for apps in app store and google play store and find that on play store, at least 40\% of the apps have inconsistencies. 
    \item We also identify \nnumber{165K} apps cross listed on both the platforms and compare how the practices are reported. Surprisingly, we find that privacy labels for \nnumber{51.5\%} of the apps are not consistent across the different platforms.
    \item We provide first large scale datasets for privacy labels for Android (n=\nnumber{1.14M}) and iOS (n=\nnumber{1.3M}). Further, we also release the new dataset for the newly formed privacy centric taxonomy. Finally, we release a large policy dataset annotated with the privacy centric taxonomy. The datasets will be available after publication. 
\end{itemize}


\begin{figure}[t]
  \centering
  \includegraphics[width=\columnwidth]{figures/good_pdfs/illustration_privacy_label_new_cropped.pdf}
  \caption{Illustrative Example of nutrition labels}
  \label{fig:pl_example}
\end{figure}





% dataflow-to-label consistency for iOS apps and showed that app developers fail to disclose all datatypes in the Apple Privacy Labels. With a developer study, Li et al~\cite{li2022understanding} show that this can be a result of knowledge gaps, lack of resources or bad preconceptions of the developers. 

% We observe that for a given app on a given platform, privacy practices described in its privacy labels can also be found in a) apps' privacy policy, and b) privacy labels of the same apps on different platform. Using different  

% due to the concise nature of privacy nutrition labels, there may be relevant and important details that are omitted. Additionally, there may be a mismatch between the privacy practices outlined in the privacy policy and in the privacy nutrition labels.  This paper aims to fill this gap by providing a comprehensive analysis of the current state of privacy labels and identifying areas for improvement by asking the following research questions:


% Our study contributes to the field by providing insights on the implementation and effectiveness of privacy nutrition labels, and highlights the need for a more comprehensive and fine-grained dataset for mobile app privacy. We aim to provide a thorough examination of the privacy practices of mobile applications, and to help users make more informed decisions when using these apps.

% Previous research has shown the benefit of having privacy labels for users; making privacy practices of an application more accessible. They present an effective platform to inform users about privacy practices in comparison to reading cumbersome privacy policies ~\cite{}. However, due to the concise nature of privacy nutrition labels, there could be relevant and important details that are omitted. Additionally, there could be a mismatch between the privacy practices outlined in the privacy policy and in the privacy nutrition labels. Li et al ~\cite{} showed that such errors can occur due to the developer’s knowledge limitations. Other research in this area focuses on the reactiveness of developers in implementing these privacy nutrition labels and analyzing the data collection practices of apps according to these labels ~\cite{}. These works do provide us with a good understanding of how developers react to the privacy nutrition labels and the issues in creating such labels. However, they fall short of providing a comprehensive analysis of the current status of the privacy labels found across the app store and play store. In this work, we aim to bridge this gap by answering the following research questions: 

% We answer the above questions by conducting a large-scale analysis on privacy labels of the apps listed in the two platforms - Google play store and Apple app store. In particular, we analyze privacy labels for all the apps across the two platform. We also compare the apps' privacy practices mentioned in the privacy labels with those present in the privacy policies. Finally, we compare the privacy practices across apps cross-listed in both the App store and Google Play store. 
% %We build a pipeline to analyze the app’s privacy policies and the privacy practices outlined in the data safety section.

% In order to achieve this, there are two main challenges: (1) There is no readily available play store or app store APIs that provide privacy labels, (2) extracting privacy labels (Data Safety section or Apple Privacy labels) is challenging as existing privacy policy taxonomies are not catered towards mobile applications and hence do not cover the categories mentioned in Privacy Labels. 

% We overcome these challenges by first developing a Google Play Store, and Apple App Store scraper to gather the metadata of over \nnumber{2.4M} apps in the Play Store and \nnumber{1.3M} apps on the App Store. We also periodically collected the metadata to track any changes that happen to an app’s description, privacy policies, and data safety section. Along with collecting each app’s play store metadata, we also scraped their privacy policies. To analyze privacy policies, we develop a privacy label centric taxonomy by adapting an existing privacy policy taxonomy. Specifically, we added the misisng elements and added more annotations for the new taxonomy. As the annotation cost can increase significantly for privacy policies, we described a bootstrapped approach that leverages advances in Natural Language Process to minimize the annotation cost. 

% Unfortunately, there aren’t any readily available Play Store or App Store APIs that can provide us with the metadata of the apps. To remedy this we first developed a Google Play Store, and Apple App Store scraper to gather the metadata of over 2.4M apps in the Play Store and 1.3M apps on the App Store. Next, we periodically collected the metadata to track any changes that happen to an app’s description, privacy policies, and data safety section. Lastly, we matched apps across platforms by developing a novel pipeline that utilized the existing apple iTunes API and cosine similarity matching. We also created a database of privacy policies of apps in the Play Store at various points in time.

% Along with collecting each app’s play store metadata, we also scraped their privacy policies. Then we retrain the Polisis ~\cite{} framework on a new dataset of privacy policies and use these new classifiers to analyze the privacy policies and then compared the data usage practices outlined in the privacy policies with that in the app’s data safety section. Furthermore, we note how an app’s privacy policies were changed given the introduction of the data safety section.


% Besides answering questions outlined above, the temporal data we collected helps us answer additional questions:
% The adoption rate among developers
% Collecting data over a period of time gave us a snapshot of the adoption rate among developers to set up their data safety section. Additionally, it gave us an idea of how often developers updated their data safety sections and in what manner. We also get a glimpse of the number of apps that still lacked a data safety section after the hard deadline of July 20th. 


% Privacy policies are a traditional way to convey the privacy practices of an application to the users. However, privacy policies are often ineffective due to readability and reachability issues. In particular, users tend to avoid reading them as they are lengthy, and vague~\cite{}. In an effort to easily convey the data practices of the applications, researchers introduced the concept of “Privacy Nutrition Labels” ~\cite{}. These “nutrition labels” were inspired by the nutrition information labels on food products, which provide a quick snapshot of the nutritional value of the item. Similarly, the privacy nutrition labels are designed to provide a quick glance into the privacy practices of an application. Recently, the concept of privacy labels has gained traction in the tech industry. In particular, Google introduced Data Safety section (DSS) requiring the developers to make information about their privacy practices easily accessible to the users. A similar system was introduced on December 08, 2020, by Apple for all new and updated apps on the App Store,  called Apple Privacy Label (APL).

% \todo[inline]{Write about contributions and then give a short outline for the paper}

% \begin{itemize}
%     \item RQ1: How do apps collect and use data at scale?
%     \item RQ1: How similar are the privacy practices being conveyed by Apple APL and Google’s DSS?
%     \item RQ2: How do the practices described in Data Safety section compare with practices described in privacy policies? 
%     \item RQ3: For cross-listed apps are the privacy practices same across platforms, how do the privacy practices outlined in the APL and DSS differ from that in their privacy policies?
% \end{itemize}