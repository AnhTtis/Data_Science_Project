\subsection{Developer Study}
\label{sec:developer}
In \cref{sec:privacy_labels} and \cref{google-data-safety-sec4}, we identified three trends in Data Safety Sections: (A) apps stating that they encrypt data without collecting or sharing data, (B) apps changing their practice from not collecting/sharing data to collecting/sharing data, and (C) apps changing their practices from collecting/sharing data to not collecting/sharing data. To gain a deeper understanding of these trends, we reached out to developers via email and asked them one general question about their Data Safety Sections and one specific question about the type of trend we observed in their app. We contacted 30K developers from the Play Store. It is worth noting that, since Apple does not provide email addresses for developers, we only conducted this study with Android developers.

In our initial email, we clearly identified ourselves as researchers and stated that we were studying their application and wanted information regarding their data safety section (\cref{app:dev-study}). Additionally, we do not collect any personally identifiable information from the developers and only use their publicly available contact information from the Play Store to contact them. As such, the study has been approved by the IRB at our institute.\smallskip

% One of the common themes in all the findings so far has been inconsistency in the privacy labels. With this, a natural question arises: \textit{What are the factors that contribute to this inconsistency?} To understand this further, we reached out to 50K android app developers via email and requested more information surrounding the inconsistencies within the Data Safety Cards. In our initial email, we clearly identified ourselves as researchers and then stated that we were studying their application and wanted information regarding some inconsistency in their data safety section. 
\noindent
\textbf{Findings:} Based on our initial emails, we received 2500 responses. After filtering out the automated replies using keyword filtering, we were left with 889 responses where the app developers describe the challenges they face while working with the privacy labels, as well as provided information about their data safety section. We further manually examined each response and curated a set of 307 replies. This manual filtering removed replies that included non-relevant replies. 
% In this section, we analyze these responses from developers by manually coding the responses. 
Next, one of the authors manually coded the responses to identify the factors for the trends, as well as general challenges described by the developers. Another author independently verified the findings by coding a subset of 50 responses independently. Specifically, we first present the major contributing factors for the different types of trends mentioned above. We then discuss the top challenges that developers face while working with the Data Safety Section.\smallskip
% Our approach has the natural advantage of extracting information under natural working environment, as opposed to survey based studies or interview based studies where the participants have the notion \rishabh{Complete this section once the analysis is complete.}

\noindent
\textbf{Type A: Apps Stating that they encrypt data without collecting or sharing data}: For this trend, we obtained responses from 165 developers. Of these, 56\% mentioned data is collected by third-party services like ads or Google Firebase but were not sure if it should be added to DSS while another 36\% were not sure what data was collected, 3\% of the developers were confused regarding encryption and added the option thinking of SSL encryption for communications between the server and the app, without collecting/sharing data. For example, one of the developers said the following: \textit{``I use Google's own libraries for this. In the Google Play Console, Policies section, I had to guess that Google is sending data and I rely on Google to encrypt that data. Because Google says that the developer is responsible for the libraries they use. That's why you find the contradictory result.''}\smallskip

\noindent
\textbf{Type B: Apps changing their practice from not collecting/ sharing to collecting/sharing:} For this trend, we received responses from 130 developers, 12\% of whom did not understand the process and selected any option that was accepted whereas 74\% changed DSS after realizing that third party libraries are collecting data. 12\% of them had an app update while 2\% changed DSS to ensure that they were up to date with the regulations like GDPR. For example, one developer said, \textit{``... Admob SDK I am integrating with the app might collect information [...] And According to Google policy, if I am using the latest version of their Admob SDK, I have to specify that the app is collecting or sharing data ...''}\smallskip

\noindent
\textbf{Type C: Apps changing their practice from collecting/sharing to not collecting/sharing:} 
We only obtained 12 responses for this trend, 58\% of which stated that their app was updated, but the DSS reflects an older version, 25\% mentioned that data was collected by ad libraries that have since been removed, and 9\% mentioned regulations as a factor for the change in DSS. For example, one developer said, \textit{``...we have changed the data safety section of our application because we [...] removed any data collecting libraries such as Firebase [...] Admob for monetization...''} \smallskip

\noindent
\textbf{Challenges for Developers: } We find that the developers are generally confused about how to fill the Data Safety Section. The source of confusion varies from \textit{Not understanding the Process} to \textit{Not understanding if data collected by the third party should be reflected in DSS}. For example, one developer stated 
\textit{``What [...] keeps changing every few months is Google's privacy policies.  They are difficult to understand and they shift like sand... I don't really understand half of them and so we just keep submitting answers in hopes it's what they are looking for...''}
indicating that the process is very unclear, while another mentioned \textit{``the reason for the change was because google play forced me to put that information''}. These confusions are problematic as they may result in inaccurate privacy labels. They can also under-represent the privacy practices in the privacy labels which can give a false sense of security to the users, increasing their privacy risks. 

We note that in an earlier qualitative study, Li et al.~\cite{li2022understanding} found that ``Developers felt unconcerned about privacy and that it was not their responsibility''. In our study, we found that developers cared about user privacy, but did not have enough means (either lack of resources or lack of transparency with third-party libraries' privacy practices) to create accurate labels, causing some frustration on their part. For example, one developer said \textit{``... we use Meta (former Facebook) audience network for monetizing non-paying users with ads. Unfortunately, the details provided by Meta are very vague, but definitely are considered as collecting and sharing data. If possible we would love to switch to an ad provider that offers proper non-personalized ads with zero/minimal data collection, but it seems impossible to find such a provider.''}. 

% We note that our findings are consistent with the qualitative study performed by Li et al.~\cite{li2022understanding}. It also worth noting that even though our sampling was done based on apps showing particular trends, the general challenges (non-exhaustive) mentioned by developers give some insights on developers' view on DSS. These also highlight the need for further research and clarity about the privacy labels for the developers. 

