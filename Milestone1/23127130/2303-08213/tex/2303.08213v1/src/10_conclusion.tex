\section{Conclusion}

In conclusion, our large-scale measurements of Privacy Labels have provided valuable insights into the privacy practices of apps. By analyzing Data Safety Sections for 2.5M apps and Apple Privacy Labels for 1.38M apps, we provided a comprehensive picture of the privacy practices of the applications. On one hand, privacy labels provide users with more specific information about the data practices of apps than traditional privacy policies. However, our analysis showed that there is often a discrepancy between the information disclosed in privacy labels and the information contained in privacy policies. This can be confusing for users and may make it difficult for them to make informed decisions about which apps to use based on their privacy concerns. Furthermore, our comparison of Privacy Labels for cross-listed apps in the Play store and Apple store showed differences in the practices disclosed, indicating that developers are not consistently disclosing the same information on different platforms. Overall, these findings highlight the importance of carefully reviewing Privacy Labels and other sources of information when evaluating the privacy practices of apps. They also suggest that there is a need for improved transparency and accountability in the app industry, as developers may not always be accurately disclosing their data collection and use practices. Having a more transparent system will allow the consumers to be aware of the data collection and use practices of the apps and make informed decisions about their privacy. 