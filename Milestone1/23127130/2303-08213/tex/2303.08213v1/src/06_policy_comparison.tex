\section{Practices Present in Privacy Policies}
\label{sec:policy_inconsistency}
The next research question that we answer is: \textit{How do the privacy practices mentioned in the privacy labels of the apps compare with the privacy practices described in their privacy policies?} We perform this comparison by training machine learning classifiers to automatically extract privacy practices mentioned in the privacy labels, as described in \Cref{sec:privacy_policy}. For Google's DSS we have \nnumber{346K} apps with valid policies, whereas for Apple, we have \nnumber{343K} apps. As described in \cref{sec:privacy_policy}, we filter out the policies which are not in English. 
% \rishabh{@Rishabh: Make sure that this is well described in earlier section and then adjust text accordingly here.} 
Note that to obtain presence of a particular practice in privacy policy, we require that there exists at least one segment which classifies the segment for that practice. For example, for a policy to have \textit{Data Encryption} practice, we require presence of at least one segment where our classifier tags it as positive for \textit{Data Encryption}. A complete mapping from classifiers to practices in APL/DSS is described in \cref{tab:policy_to_label}.

% It is important to note here that we do not rely on high level classifiers such as \textit{First Party Collection} or \textit{Third Party Data Sharing} for extracting high level privacy practices such as \textit{Data Collection} and \textit{Data Sharing}. This is done to avoid generic collection or sharing statements contaminating the 

There are two types of inconsistencies that can arise: 1) \textit{In Label}, where a given privacy practice is mentioned in the privacy label but is absent from the privacy policy, and 2) \textit{In Policy}, where a practice is found in privacy policy but is missing from privacy label. As privacy policies can potentially cover multiple applications, websites and products, \textit{In Policy} inconsistency does not necessarily mean that policy is inconsistent with the privacy label. For example, the Google app \textit{Clock} reports that it does not collect or share any \textit{Location} information. However, since Google has one policy to cover all the products, the policy states that they can collect \textit{Location} (applicable in Google Maps). In such cases, it is inaccurate to say that privacy label are inconsistent with the privacy policy without further analyses. 
However, if an app mentions collection of data and the policy does not mention it, then we can conclusively say that the privacy policy and the app are inconsistent. Thus, in this work, we will focus primarily on \textit{In Label} inconsistencies, except when there is a negative practice is involved (\textit{Data Not Collected} or \textit{Data Not Linked to You}). This is because if the policy says that data is not collected, then no app corresponding to that policy should collect any data, and in this case we focus on \textit{In Policy} consistency.


% \red{The discussion above also highlights a major issue with using privacy policy as the source to understand privacy practices of the app. Specifically, we highlight the notion that privacy policies usually cover practices for websites (and potentially other products) and hence, may not accurately inform users about privacy practices of the application. In fact, they can mis-inform the user about the privacy practices in certain cases. \rishabh{Maybe move this to discussion and add that while privacy labels address this issue, without consistency checks, they do not contribute much.}}


% \subsection{Comparison with Privacy Labels}
% In this section, we look at how the privacy policies of apps compared with their respective privacy labels as stated in DSS and APL. For Google's DSS we had \nnumber{346K} apps and for Apple, we analyzed \nnumber{458K} apps. \rishabh{The total number of apps with DSS/APL is way higher, right? So these are the apps which had both DSS/APL and privacy labels? If so - are we saying that 350K apps on google had DSS but not a valid privacy policy? This would needs more explanation}\red{These numbers represent the number of apps that have both: privacy labels and a classified privacy policy. Here a classified privacy policy is defined if (1) it exists, (2) is in English.}\smallskip
\noindent
\subsection{Google Data Safety Sections}
\cref{fig:policy_comp} shows the \textit{In Policy} and \textit{In Label} inconsistency for high level practices for apps on the play store. We note that only 5\% (6\%) of the apps with DSS that collect (share) data have \textit{In Label} inconsistency with their privacy policies. 
% \rishabh{@Asmit: Add one example here, possibly from one of the emails that we got.} 
We also find that \textit{In Policy} inconsistencies for these categories are more than 55\%, but as discussed above, these could be due to privacy policy covering multiple apps and websites. To understand the extent to which this happens due to multiple apps, we analyze DSS for apps from the same developers. There are 15,380 developers who have 3 or more apps. These developers have an average of 13 apps and a median of 7 apps per developer. We find that 68\%(10,420) of these developers have duplicate data safety section for their apps. 
%The mean percentage of duplicate data safety section among all of these developers is 68.2\%, with a standard deviation of 22\% and a median of 75\%. 
For example, the app developer \textit{Premium Software} has over 9 apps across 6 genres but with only 2 unique DSS. 
% This indicates that \textit{In Policy} inconsistencies for developers having multiple apps are either coming from practices of websites or  are legitimate inconsistencies. 
This also highlights that developers might be duplicating their DSS across their apps, even though the apps can span multiple genres and have different features. 
% \rishabh{Maybe we need an example here? Or remove this part altogether?}

Analyzing the inconsistencies for \textit{Data Encryption} and \textit{Data Deletion}, we find that the majority of apps declare them in their privacy labels but there is no mention of such practices in their privacy policies. For example, \textit{Snapchat} mentions in their DSS that data is encrypted in transit but no corresponding practice is present in their privacy policy. Similarly, \textit{Kik — Messaging \& Chat App} state in their DSS that \textit{Data can’t be deleted} yet their privacy policy states that users can ask them to delete their information. It is worth noting that from a privacy and regulation standpoint, these two practices are extremely important. \textit{Data Deletion} option gives the users the right to either delete their data or ensure that it stays in anonymized form, which has roots in several regulations such as the GDPR~\cite{linden2018privacy} and the CCPA~\cite{ccpa}. \textit{Data Encryption} on the other hand, is crucial to prevent data snooping attacks which aim to get unlawful access to the data while the data is in transit.

% From analyzing the data gathered on apps from the Google Play Store, as shown in the top plot in \cref{fig:policy_comp}, we can see that for \textit{Data Collection} and \textit{Data Sharing}, over 50\% of apps have some type of inconsistency between their privacy policies and labels. Moreover, we note that for a majority of these apps their privacy policy mentions \textit{First Party} or \textit{Third Party Data Collection}, yet their privacy labels do not reflect that. For example, according to the DSS of the app, \textit{ShareMe: File sharing}\footnote{\url{https://play.google.com/store/apps/details?id=com.xiaomi.midrop}}, no data is being collected but their privacy policy does state that they can collect data. Interestingly this app also shares data with third parties but doesn't collect any data, which is non-intuitive but can happen when third party libraries are used. 

\begin{figure}
  \centering
  \includegraphics[scale=0.25]{figures/good_pdfs/high_level_inconsistency4_cropped.pdf}
  \caption{Inconsistencies between privacy policies and DSS and APL. The normalization is based on the total number of apps with privacy labels and classified policies.}
  \label{fig:policy_comp}
\end{figure}

Analyzing practices at the category level, we find that there are significant inconsistencies between Privacy Labels and policies for data sharing and data collection. Specifically, we find that for data sharing, 89\% of the apps have inconsistent (\textit{In Label}) Privacy Labels for \textit{Location}, 82\% of the apps for\textit{Device IDs}, and 74\% of the apps for \textit{Health and Fitness}. For example, \textit{Myntra - Fashion Shopping App} states that they collect location, health info, contact list, and much more, yet its privacy policy doesn't mention the collection or sharing of such data types. Similarly, \textit{Tripadvisor: Plan \& Book Trips} states that they collect and share location data yet there is no mention of such practices in their privacy policy. This suggests that developers report more precise data-sharing practices in Privacy Labels, and can inform users allowing them to make better choices. 

% \rishabh{TODO: Talk about datatypes, purposes - some insightful examples}
% \chatgpt{In analyzing purposes, we have noticed that there is often a discrepancy between what is stated in app policies and what is present in labels. However, we also observed that a significant percentage of apps have more information in their privacy labels than in their policies. For instance, \textit{YouTube}\footnote{\url{https://play.google.com/store/apps/details?id=com.google.android.youtube}} does not mention developer communication in its labels, but it is mentioned in their privacy policy\footnote{\url{https://policies.google.com/privacy\#:~:text=Communicate\%20with\%20you}}}.

% \chatgpt{When analyzing datatypes and categories, we found that most inconsistencies in datatypes occur with App Activity, App Info, Performance, and Personal Info, with these datatypes being collected according to policies but absent or inconsistent from labels. Similarly, for datatypes such as Location, Device IDs, and Other IDs, the inconsistency arises from these being present in labels but absent or claimed to not be collected in policies. For example, \textit{Paytm: Secure UPI Payments}\footnote{\url{https://play.google.com/store/apps/details?id=net.one97.paytm}} does not mention the collection of the user's Financial Info in its labels, but the app's privacy policy states that Financial Info is collected.}


% We note here that the distinction of the type inconsistency between label and privacy policy is important. For instance, if a data practice is present in the privacy label, it should be present in the corresponding privacy policy (e.g. \textit{Data Encryption} or \textit{Data Deletion}). However, if a privacy policy contains \textit{Data Collection} but the label doesn't, it is possible, though unlikely, that they are still consistent. This is because privacy policy can describe practices of the entity as a whole, rather than a single application. For example, multiple apps from the same developer will have the same privacy policy, but can have different privacy labels for different apps. 

% \todo[inline]{TALKED TO RISHABH ABOUT THIS}
% To understand this further, we analyze the data safety section for applications with the same developer. We identified \red{x} developers with an average of \red{y} applications. For \red{z\%} of the developers with more than 3 apps, we find that the data safety card does/does not change with apps. For example, \rishabh{give an example here where even the genre is different}. 
% \chatgpt{There are 15,380 developers who have 3 or more apps with over 10,000 downloads each, regardless of the app genre. Similarly, there are 3,790 developers who have 10 or more apps with over 10,000 downloads each. Additionally, there are 10,993 developers who have an app count between 3 and 10 and have over 10,000 downloads for each app. These developers have an average of 13 apps and a median of 7 apps per developer. Of these developers, 10,420 have duplicate data safety section for their apps. The mean percentage of duplicate data safety section among all of these developers is 68.2\%, with a standard deviation of 22\% and a median of 75\%.} For example, the app developer \textit{Premium Software}\footnote{\url{https://play.google.com/store/apps/dev?id=4843900089693546737}} has over 9 apps across 6 genres but with only 2 unique DSS.
% We do note that in some cases, developers do specify privacy practices for each app differently, however, for majority of the developers, the privacy label did not change, indicating that the inconsistencies with the privacy policies are legitimate. \smallskip
%\todo[inline]{Asmit: Insight: Well because developer follow the same PP maybe it makes sense that the DSS are same} 

\subsection{Apple Privacy Label}
Next, we analyze the apps on the App store and compare privacy practices mentioned in the apps' privacy policies and their Apple Privacy Labels. The bottom plot in \cref{fig:policy_comp} shows inconsistencies for the high-level categories present in APL. We find that for \textit{Data Linked to You} and \textit{Data Used to Track You}, the \textit{In Label} inconsistency is 5\% and 4\% respectively. As the other two of the high-level practices, \textit{Data Not Linked To You} and \textit{Data Not Collected} are negations, we consider the \textit{In Policy} inconsistency (see \cref{sec:policy_inconsistency}). We find that 42\% of the apps have policies that state that they do not link data whereas the privacy label indicates otherwise. Furthermore, 13\% of the apps have policies that do not have data collection or sharing, but the privacy label indicates otherwise. For example. ~\textit{Superior Vision} app on App Store states that they collect \textit{Health \& Fitness} data yet their policy doesn't state that.

In \cref{fig:policy_comp}, we observe that \textit{In Policy} inconsistency for \textit{Data Used to Track you} is very high. This implies that privacy policies include tracking practices while privacy labels do not. This can potentially be due to the presence of segments related to cookies in the privacy policy, for example. ~\textit{Netflix}'s privacy policy talks about using cookies to track users on their site but not about tracking via their app.

We next examine the consistency at the data category level for \textit{Data Linked to You} and find that, similar to DSS, \textit{Location} (39\%), \textit{Identifiers (51\%)} and \textit{Health and Fitness} (60\%) had the largest \textit{In Label} inconsistencies. For \textit{Data Not Linked to You}, we find inconsistencies primarily in the same data categories. For example, ~\textit{Jetpack Joyride} states that they collect Location data but their privacy policy states that they collect Location based on IP Address and not the GPS location.

% A similar analysis of Apple App Store apps, show that on average 30\% of apps have inconsistencies between what is reported by the APL and what their privacy policy states. Similar to what we observed in google apps, a considerable percentage of apps's policy mention that they collect data, anonymously or not, and use the data for tracking but their APLs fail to reflect these practices. \red{A notable example would be \textit{Netflix}\footnote{\url{https://apps.apple.com/us/app/netflix/id363590051}} which do not mention of any data being shared, their policy does share data}
% \todo[inline]{Netflix has Third Party Ads but not Data Used to Track You ---!?!??!}


\subsection{Takeaways}
In this section, we find that at least 40\% of the apps with DSS, and APL are inconsistent with their privacy policy. Additionally, we note that DSSs contain more information about \textit{Security practices} than privacy policies, and thus can provide useful information to the users. We also note that sensitive datatypes such as \textit{Location}, \textit{Identifiers} and \textit{Health and Fitness} had the largest \textit{In Label} inconsistencies, indicating that the developers disclose collection/sharing of these fine-grained datatypes in the privacy labels. 
% The main takeaways: 
% \begin{itemize}
%     \item For security practices (Encryption and data deletion), DSS contain more fine-grained information than the policy. 
%     \item DSS also contain more precise data sharing practices at the category level, for example for location, device id etc.
%     \item For APL - high inconsistency in Data Not Linked to you. 
% \end{itemize}

% We find that on Google Play Store, \red{50\%} apps have at least one inconsistency between privacy labels and privacy policies, whereas on the App store, \red{40\%} of the apps have inconsistencies. These inconsistencies can have three implications: a) Privacy Labels are accurate: In this case, 
% it is encouraging to see that more app developers take measures to ensure data security and choice. However, the inconsistency here highlights the shortcomings of the privacy policies and how it may fail to capture some important privacy practices. b) Privacy Labels are inaccurate: This brings the usefulness of privacy labels into question and can be even more dangerous than having no privacy labels as incorrect privacy labels might induce a false sense of security in users.

% \rishabh{From inconsistency between data types - we learn that in policy inconsistency is zero, indicating that privacy labels provide more fine grained details about the data types, and can be a very useful resource for users to make informed decision}
% \chatgpt{In our analysis of the inconsistency with respect to purposes, we have found that most apps state in their privacy policies that they are collecting or sharing data for purposes that are not reflected in their DSSs (Data Sharing and Collection). The most significant inconsistencies occur with regard to Developer Communication and Account Management. For example, \textit{My Talking Tom}\footnote{\url{https://play.google.com/store/apps/details?id=com.outfit7.mytalkingtomfree}} states in its DSS that it does not collect any information for Account Management, but its privacy policy\footnote{\url{https://outfit7.com/privacy/en/\#:~:text=If\%20you\%20use\%20the\%20\%22Log\%20in\%20with\%22\%20or\%20\%22Connect\%20to\%22\%20feature\%20we\%20may\%20access\%20and\%20store\%20some\%20or\%20all\%20of\%20the\%20following\%20information}} states otherwise.}

% \chatgpt{Upon examining the inconsistencies of data types, we have found that for most datatypes, the inconsistency is due to the privacy policy stating that the app collects the data while the DSS does not mention it. However, Location and Device or Other IDs have the most inconsistency, with apps stating in their DSSs that they collect these datatypes but their privacy policies do not indicate so. For example, the privacy policy\footnote{\url{https://privacy.microsoft.com/en-ca/privacystatement}} for \textit{Microsoft Word: Edit Documents}\footnote{\url{https://play.google.com/store/apps/details?id=com.microsoft.office.word}} does not mention the collection of location data, yet the DSS for the app states that location is collected.}
% \rishabh{Write more about data which dataypes were most common, what purpose was most common etc. AN: Why here?}