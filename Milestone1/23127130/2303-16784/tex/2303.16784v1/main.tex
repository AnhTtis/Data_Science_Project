\documentclass[superscriptaddress,prl,floatfix,twocolumn,amsmath,amssymb,aps,showkeys]{revtex4-1}

\usepackage{graphicx}
\usepackage{amsmath}
\usepackage{amssymb}
\usepackage{dcolumn}
\usepackage{color}
\usepackage{multirow}
\usepackage{bm}
\usepackage{subfigure}
\usepackage{tabularx}
\usepackage{booktabs}
% \newcommand{\mycenter}[1]{\begin{tabular}{c} #1 \end{tabular}}

\begin{document}
\title{Complex phase diagram and supercritical matter}
\author{Xiao-Yu Ouyang}
\author{Qi-Jun Ye}
\email{qjye@pku.edu.cn}
\affiliation{State Key Laboratory for Artificial Microstructure and Mesoscopic Physics, Frontier Science Center for Nano-optoelectronics and School of Physics, Peking University, Beijing 100871, P. R. China}
\author{Xin-Zheng Li}
\email{xzli@pku.edu.cn}
\affiliation{State Key Laboratory for Artificial Microstructure and Mesoscopic Physics, Frontier Science Center for Nano-optoelectronics and School of Physics, Peking University, Beijing 100871, P. R. China}
\affiliation{Interdisciplinary Institute of Light-Element Quantum Materials, Research Center for Light-Element Advanced Materials, and Collaborative Innovation Center of Quantum Matter, Peking University, Beijing 100871, P. R. China}
\affiliation{Peking University Yangtze Delta Institute of Optoelectronics, Nantong, Jiangsu 226010, P. R. China}
\date{\today}

\begin{abstract}
    Supercritical region is often described as uniform with no definite transitions.
    %
    The distinct behaviors of the matter therein, e.g., as liquid-like and gas-like, however, indicate their should-be different belongings.
    %
    Here, we provide a mathematical description of these phenomena by revisiting the Lee-Yang (LY) theory and using a complex phase diagram, e.g. a 4-D one with complex $T$ and $p$.
    %
    Beyond the critical point, the 2-D phase diagram with real $T$ and $p$, i.e. the physical plane, is free of LY zeros and hence no criticality emerges.
    %
    But off-plane zeros in this 4-D scenario still come into play by inducing critical anomalies for different physical properties.
    %
    This is evidenced by the correlation between the Widom lines and LY edges in van der Waals model and water.
    %
    The present distinct criteria to distinguish the supercritical matter manifest the high-dimensional feature of the phase diagram:
    e.g. when the LY zeros of complex $T$ or $p$ are projected onto the physical plane, a boundary defined by isobaric heat
    capacity $C_p$ or adiabatic compression coefficient $K_T$ emanates.
    %
    These results demonstrate the incipient phase transition nature of the supercritical matter.
\end{abstract}

% \keywords{Lee-Yang zero, Lee-Yang edge, Liquid-liquid phase transition, Widom line,  Crossover, Supercritical region}
% \pacs{05.70.Fh, 05.70.Ce, 02.70.-c}

\maketitle

%
Phase transition, by definition, must be accompanied by non-analytical changes in the thermodynamic state functions as one tunes a thermodynamic system across its transition point.
%
In two seminal articles of Lee and Yang in 1952~\cite{yangStatisticalTheoryEquations1952,leeStatisticalTheoryEquations1952}, the mathematical structure of these phenomena had
been established using zeros of the grand canonical partition function, i.e. Lee-Yang (LY) zeros.
%
For condensation of a monatomic gas and magnetization of the Ising model, it was shown that the edge points of the LY zeros on the complex plane of the fugacity
and magnetic field approach the real axis in the thermodynamic limit when phase transition happens, locating phase boundaries and critical points in the phase diagram.
%
In 1965, Fisher generalized LY's theory to the canonical ensemble and defined Fisher zeros for complex temperature ($T$)~\cite{fisherLecturesTheoreticalPhysics1965}.
%
Now, it is customary to analyze the phase transition phenomena using such LY or Fisher zeros, with applications extending to the studies of
their experimental measurements~\cite{pengExperimentalObservationLeeYang2015,weiLeeYangZerosCritical2012}, nonequilibrium problems~\cite{brandnerExperimentalDeterminationDynamical2017,flindtTrajectoryPhaseTransitions2013}, protocols of quantum
simulators~\cite{francisManybodyThermodynamicsQuantum2021,krishnanMeasuringComplexpartitionfunctionZeros2019,xuProbingFullDistribution2019,gnatenkoTwotimeCorrelationFunctions2017},
and dynamical quantum phase transitions~\cite{heylDynamicalQuantumPhase2018,heylDynamicalQuantumPhase2013}.
%
Unlike these phase transition problems within the critical point, the mathematical structure for theoretical descriptions of the supercritical phenomena is weak.
%

%
The supercritical behavior in real systems plays a crucial role in both fundamental research and emerging applications.
%
For example, supercritical water not only subtly shapes our planet but is also deployed as an ecologically benign solvent in chemical reactions and waste management~\cite{WeingartnerFranck2005,Savage1999,Keppler1996}.
%
Starting from Cagniard De La Tour, studies on the supercritical regions of the phase diagram had greatly extended our understanding of phases in
the last $\sim$200 years~\cite{Cagniard1822,Postorino_Tromp_Ricci_Soper_Neilson_1993,Smith_Kay_1999, McMillan_Stanley_2010,Galli_Pan_2013,DSouza_Nagler_2015,Cheng_Mazzola_Pickard_Ceriotti_2020,Cockrell_Trachenko_2022}.
%
Lack of transition means that the supercritical matter was usually introduced as a single phase~\cite{williamsSupercriticalFluidMethods2000,kiranSupercriticalFluids2000,Proctor2020}.
%
Looking through the thermal and dynamic properties such as the heat capacity, however, the lines of maxima as functions of $T$ or pressure ($p$), i.e. the Widom lines,
provide no criticality but obvious criteria to draw a distinction \cite{xuRelationWidomLine2005,abascalWidomLineLiquid2010,galloWidomLineDynamical2014,liSupercriticalPhenomenonHydrogen2015,luoBehaviorWidomLine2014,lupiDynamicalCrossoverIts2021},
e.g. as liquid-like and gas-like beyond the vaporization critical point.
%
The challenge is to obtain a picture unifying these alternatives and hence to interpret the macroscopically invisible boundaries and dynamical crossover~\cite{gartnerManifestationsMetastableCriticality2021,palmerMetastableLiquidLiquid2014,poolePhaseBehaviourMetastable1992}.
%
Recent studies of quantum chromodynamics (QCD) models demonstrate a correlation between the LY zeros and the crossover behavior~\cite{basarUniversalityLeeYangSingularities2021,connellyUniversalLocationYangLee2020}.
%
The establishment of theoretical and numerical connections between the supercritical behaviors and the complex LY zeros in real condensed matter systems might pave the way.
%

%
In this work, we investigate the supercritical phenomena using a complex phase diagram.
%
For an ordinary $T$-$p$ phase diagram, the physical plane consists of the real axes of $T$ and $p$.
%
We extend the framework of LY theory and emphasize that the LY zeros exist in a 4-D space with extra imaginary axes of $T$ and $p$.
%
In this scenario, the so-called Widom line for each response function exactly corresponds to the closest LY zeros to the physical plane, i.e. the LY edges.
%
For the simplest cases, the maxima of the isobaric heat capacity $C_p$ (adiabatic compression coefficient $K_T$) as the response function
to changed $T$ ($p$), i.e. its Widom line, show close coincidence with the $T$- ($p$-) edges.
%
Through a complex phase diagram with one of the imaginary axes contracted, we provide insights into the supercritical behavior.
%
Distinct zeros dominate on both sides of the edge, resulting in diverged properties of the two regions, especially for those far from the edge.
%
To unfold these ideas concretely, we implement theoretical calculations and molecular simulations in both van der Waals (vdW) gas and water.
%
These zeros present good illustration of the phases and crossovers, which offers us an intuitive tool to unify the phase behavior.
%

%
We start by recalling how the LY theory describes the phase transitions below the critical points.
%
Taking a grand canonical system at a given thermal and chemical potential $(T,\mu)$ as an example, its partition function can be viewed as a polynomial in fugacity $y=e^{-\beta \mu}$, as $Z(T,\mu) = Z_T(y) = \sum_k c_k y^k$ by summing over all possibility of having a certain number of particles $k$.
%
Typically when the interatomic potential is of a repulsion core, the system of volume $V$ can contain particles up to $N_{\text{max}}$ ($\sim V/v_0$, with $v_0$ the volume of the hard core) and algebraically $Z_T(y)$ is of the same degree with $N_{\text{max}}$ roots $y^*$, i.e. LY zeros, satisfying $Z_T(y^*)=0$~\cite{leeStatisticalTheoryEquations1952,yangStatisticalTheoryEquations1952}.
%
Fisher noted that such treatment can be applied for other ensembles, e.g. the canonical partition function is an entire function of $T$ and its zeros as $T^*$ will
also locate a critical temperature $T_{\text{c}}$ when it approaches the real axis in the thermodynamic limit~\cite{fisherLecturesTheoreticalPhysics1965,SI}.
%
Being complex-valued~\footnote{The coefficients $c_k$ are real and positive since they are simply the integrals of the Boltzmann-Gibbs factors. Consequently, $Z_T(y)$ are always positive for real-valued $y$, i.e., algebra equation $Z_T(y) = 0$ are free of real roots for finite $V$ and hence finite $N_{\text{max}}$.}, zeros of fugacity or $T$ can only be physically accessed in the thermodynamic limit as the critical field $\mu_{\text{c}}$ or $T_{\text{c}}$, where the energetic state functions proportional to $-\ln Z_T(y^*)$ and thermal properties as their derivatives manifest singularities when phase transition occurs~\cite{leeStatisticalTheoryEquations1952}.
%

%
Heuristically, one can apply this picture to descriptions of supercritical behaviors.
%
Without losing generality, we use two typical thermodynamic fields $T$ and $p$ and consider the $T$-$p$ phase diagram.
%
A complex space $\widetilde{x}$ of $T\to \widetilde{T} = T + i\tau$ and $p\to \widetilde{p}= p + i\zeta$ is employed.
%
By analytic continuation, the partition function can be represented in terms of complex zeros corresponding to $\widetilde{T}$- or $\widetilde{p}$- perspectives\footnotetext{The prerequisite for such treatment is that the partition function is an entire function according to Weierstrass factorization theorem. We note that this is true for most physical cases. Please see the SI for discussions. }, as
\begin{equation}
    \begin{aligned}
        Z(\widetilde{T},\widetilde{p}) & = Z_{\widetilde{p}}(\widetilde{T}) = e^{g_{\widetilde{p}}(\widetilde{T})} \prod_{k=1}^{\infty} \left( 1-\widetilde{T}/{{\widetilde{T}}^*_{\widetilde{p},k}} \right)   \\
                                       & =  Z_{\widetilde{T}}(\widetilde{p}) = e^{g_{\widetilde{T}}(\widetilde{p})} \prod_{k=1}^{\infty} \left( 1-\widetilde{p}/{{\widetilde{p}}^*_{\widetilde{T},k}} \right),
    \end{aligned}
    \label{partition function}
\end{equation}
where $\widetilde{T}^*_{\widetilde{p},k}$ is the $k$-th non-zero root for $Z_{\widetilde{p}}(\widetilde{T})=0$ at given $\widetilde{p}$, and $\widetilde{p}^*_{\widetilde{T},k}$ is defined similarly.
%
Due to the dependency on each other, $\widetilde{T}^*_{\widetilde{p}}$ and $\widetilde{p}^*_{\widetilde{T}}$ manifest a unified cluster of zeros
in a 4-D complex space $\mathcal{C}^2$.
%
When taking physical values $T$ ($p$), $\widetilde{p}^*_{\widetilde{T}=T}$ ($\widetilde{T}^*_{\widetilde{p}=p}$) returns to LY (Fisher) zeros.
%
Acknowledging that zeros in the physical plane (real $T$ and $p$) locate phase boundaries and critical points, we emphasize here that zeros outside the physical plane are
of crucial importance and responsible for the anomalies in the supercritical region.
%

\begin{figure}[b]
    \centering
    \includegraphics[width=0.9\linewidth]{fig1.pdf}
    \caption{
    The distribution of LY zeros by making the analogy to charges in the Ising model.
    %
    (a) Below the critical point, LY zeros (grey solid line) distribute uniformly on the unit circle and (b) in the supercritical region, LY zeros terminate at the edges (red points).
    %
    The associated energetic state functions, order parameters, and susceptibilities (analogously electric potential, field, and field gradient in Ref.~[\onlinecite{leeStatisticalTheoryEquations1952}]) are in red, yellow, and purple dashed lines, respectively.
    %
    }
    \label{Scheme}
\end{figure}

\begin{figure*}[t]
    \centering
    \includegraphics[width=0.95\linewidth]{fig2.pdf}
    \caption{Lee-Yang zeros and edges of vdW liquid in the supercritical region.
        %
        The derivatives of free energy $G$ as (a) $|\partial G/\partial T|$ at $p=1.1$ and (b) $|\partial G/\partial p|$ at $T=1.02$.
        %
        The density of zeros $\rho(x)$ at $\mathcal{C}^2$ as 2-D slices (c) at $p=1.1$ and (d) at $T=1.02$, with edges in black circles.
        %
        The associate thermal properties are also shown: (c) enthalpy $H$ and isobaric heat capacity $C_p$ and (d) volume $V$ and isothermal compression coefficient $K_T$ in purple dashed and blue solid lines, and its crossover (cross) and maxima (circle) in corresponding colors, respectively.
        %
        The LY edge, maximum of response function A crossover of $H$ (the purple cross) and a maximum of $C_p$ (the blue circle) are shown at the $T$-edge (the black circle).
        %
        (e)(f) (Sharing the same DOZ colorbar with (c)(d)) The 3-D plot of DOZ and the line of edges, with the Widom line of (e) $C_p$ and (f) $K_T$.
        %
        The real plane of the phase diagram is painted blue.
    }
    \label{Fig1}
\end{figure*}

%
To illustrate this, we use the electrostatic analogy proposed by Lee and Yang.
%
Taking the logarithm of Eq.~(\ref{partition function}) and replacing the summation over the discrete zeros with the integral of the density of zeros (DOZ) $\rho(\widetilde{x})$, the energetic state function can be written as~\cite{SI},
\begin{equation}
    {F}(x)\sim -\ln Z(x)\sim -\int_{\mathcal{C}^2} \rho(\widetilde{x}) \ln (x-\widetilde{x})|\mathrm{d}\widetilde{x}|.
\end{equation}
%
This expression is exactly the form of a 2-D Coulomb potential $\phi$ produced by a circular cylinder with surface charge density $\rho(x)$ per unit area.
%
It means the behavior of order parameter $\Omega$ and susceptibility $\chi$ of the response function can be perceived equivalently from electric field $\epsilon$ and its gradient $\epsilon'$.
%
For example, the uniformly charged shell keeps $\phi$ a constant inside it and drives $\phi$ decrease outside it, accompanied by a sharp change in $\epsilon$ and $\epsilon'$ at the intersection point (Fig.~\ref{Scheme}(a)).
%
This corresponds to a phase transition and the latter accounts for the singularity.
%
When the shell is cut, $\phi$, $\epsilon$, and $\epsilon'$ manifest continuous changes and finite maxima in the vicinity of the edges.
%
Considering that the interactions are inverse to the distance, the predominant zeros are those closest to the physical plane, i.e. the LY edges.
%
The different tendencies on both sides of the edges mean a crossover of the original two phases.
%




%
To put this analysis on a firm ground, we consider two prototypical systems, i.e. the vdW model and a realistic water system with TIP4P
intermolecular interactions.
%
The vdW model is the simplest real gas model where particles interact and occupy finite volumes.
%
Taking its critical point in a reduced unit, as $(p_c,V_c,T_c)\to(1,1,1)$, the equation of state writes $\left(p+{3}/{V^2}\right)(3V-1)=8T$.
%
The free energy $G$ reads
\begin{equation}
    G(p,T)  =-\left[\frac{3}{2}\ln T+\ln \left(V-\frac{1}{3}\right)+\frac{9-3pV^2}{8VT}\right],
\end{equation}
%
from which other quantities can be obtained using $H=-T^2(\partial G/\partial T)_p$, $V=T(\partial G/\partial p)_p$, $C_p=(\partial H/\partial T)_p$,
and $K_T=-(\partial V/\partial p)_T$.
%
These quantities will experience sudden changes across the LY singularities.
%
On the two sides of the trajectory of the LY zeros, the derivative of the free energy at $\widetilde{x}_0$ drops by
\begin{equation}
    \left| \left(\frac{\partial G}{\partial \widetilde{x}}\right)_{\widetilde{x}_{0+}}-\left(\frac{\partial G}{\partial \widetilde{x}}\right)_{\widetilde{x}_{0-}} \right|=2\pi \rho(\widetilde{x}_0),
    \label{DOZ}
\end{equation}
where $\widetilde{x}_{0\pm}$ represent the different sides with respect to the interface.
%
Therefore, we can locate the LY zeros by the non-analytical positions of $G$, as shown in Fig.~\ref{Fig1}(a)(b).
%
The points with nonzero DOZ converge to a line, with sharp LY edges (Fig.~\ref{Fig1}(c)(d)).
%
The $T$-edges are intimately related to the crossover of $H$ and the maxima (not singularities) of $C_p$ (Fig.~\ref{Fig1}c),
so do those of the $p$-edges and the crossover of $V$ and the maxima of $K_T$ (Fig.~\ref{Fig1}d).
%
For ease of presentation, we show 3D-plots with real $T$ or $p$ on the $xy$ plane and imaginary $T$ or $p$ as $z$ axis respectively in Fig.~\ref{Fig1}(e)\&(f).
%
The Widom line as maxima of $C_p$ or $K_T$ shows apparent correspondence to $T$- or $p$- edges, respectively.
%
These results are consistent with the theoretical analysis based on electrostatic analogy.
%

%
One fascinating but intricate fact about supercritical matter is that there are distinct criteria to distinguish them.
%
This is intrinsic to the high-dimensional feature of the LY zeros.
%
With alternative degrees of freedom to describe a real system, the above treatments on $T$ and $p$ can be generalized and a consequent complex phase diagram should be used.
%
Beyond the conventional one, this phase diagram embodies the full statistical information by retaining complex LY zeros rather than only the real ones.
%
When the closest zeros are on the physical plane, this point overrides the others and all properties diverge at the same place.
%
When zeros are away from the physical plane, there can be different edges emanating from the same zeros corresponding to different physical properties.
%
As a result, the Widom lines appear at different places.
%
These are exactly the case of phase transition and supercriticality, respectively.
%
An example of complex $\widetilde{T}$-$\widetilde{p}$ phase diagram of the vdW model is shown in Fig.~\ref{cpd}.
%
The supercritical region is no longer a ``no transition's land" since we can see how the complex LY edges determine the phase diagram.
%
The Widom lines largely overlap with the corresponding trajectories of the LY edges, with small deviation due to contributions from non-edge zeros with large
density (inset of Fig.~\ref{cpd}).
%
In this view, one can interpret supercriticality as a phase transition in the complex phase diagram and an incipient one in the physical phase diagram.
%

\begin{figure}[t]
    \centering
    \includegraphics[width=1.0\linewidth]{fig3.pdf}
    \caption{The complex $\widetilde{T}$-$\widetilde{p}$ phase diagram of the van der Waals model around the critical point.
        %
        The Lee-Yang edges corresponding to $T$ and $p$ are plotted in the 3D complex phase diagram, with an imaginary z-axis of both $T$ and $p$.
        %
        The edges converge to the same coexistence line and terminate at the critical point in the physical plane.
        %
        While in the supercritical region, one witnesses different edges in the complex plane and hence different Widom lines in the physical plane.
        %
        The inset shows a close connection between the projection of edges and Widom lines.
    }
    \label{cpd}
\end{figure}

%
Not unique to vdW fluids, these phenomena are also observed in water systems.
%
To investigate this, we performed molecular dynamic (MD) simulations within the $NpT$ ensemble and located the phase boundary between gas and liquid phases of water, its critical point, and $C_p$/$K_T$ Widom lines in its $T$-$p$ phase diagram.
%
We developed an approximate method for calculating LY zeros based on MD results at $(T_0, p_0)$ by estimating the density of states using the probability distribution of enthalpy $H$ and volume $V$~\cite{SI}.
%
By discretizing the partition function into polynomials with arguments $e^{-\beta \Delta E}$ or $e^{-\beta p \Delta V}$, we derived $T$- or $p$-zeros (Fig.~\ref{water}(a)(b)).
%
The complex phase diagram of water is given in Fig.~\ref{water}(c).
%
Beyond the critical point, the $T$-/$p$-edges deviate from the coincident transition points and gradually move away from the physical plane (Fig.~\ref{water}(d)), and manifest similar correspondences to Widom lines with vdW results (Fig.~\ref{water}(e)).
%
The consistent findings within different models/treatments indicate the power of the complex diagram.
%

\begin{figure}[t]
    \centering
    \includegraphics[width=1.0\linewidth]{fig4.pdf}
    \caption{LY zeros, edges, and the complex phase diagram of water using the TIP4P model.
        From left to right: (a) complex $p$-zeros at $T=675$ to 775~K, and (b) complex $T$-zeros at $p=50$ to 250~atm.
        (c) The complex $\widetilde{T}$-$\widetilde{p}$ phase diagram, where the critical point is determined to be approximately at $T_c\sim720$~K and $p_C\sim 150$~atm.
        (d) The side view and (e) the top view of this complex phase diagram.
        Due to the finite size effect, LY zeros below the critical point approach real axes instead of being exactly onto them.
        Throughout, the Lee-Yang edges corresponding to $T$ and $p$ are labeled with solid marks in blue and red, while the maximum of $C_p$ and $K_T$ are labeled with hollow marks in red and blue, respectively.
        %
    }
    \label{water}
\end{figure}

%
Considering the purely mathematical origin of zeros, one might wonder if the complex fields corresponding to LY zeros and hence the complex diagram are physically accessible.
%
We answer this by noting that the LY zeros can be obtained either numerically or experimentally.
%
The closest few zeros to the real axis can be extracted via the high-order cumulant method, which is viable in both molecular simulations and experiments~\cite{flindtTrajectoryPhaseTransitions2013a,Brandner2017,brangeLeeYangTheoryBoseEinstein2023}.
%
Furthermore, direct experimental observation of LY zeros was reported by measuring the quantum coherence of a probe spin coupled to an Ising-type bath, where the evolution of the former relates to complex LY zeros~\cite{weiLeeYangZerosCritical2012,pengExperimentalObservationLeeYang2015}.
%
Besides this, complex fields can also reveal extra degrees of freedom within the scope of several emerging phenomena, such as the dynamical quantum phase
transition (DQPT), non-Hermitian physics, and non-equilibrium phase diagram~\cite{heylDynamicalQuantumPhase2013,heylDynamicalQuantumPhase2018,yamamotoTheoryNonHermitianFermionic2019,ashidaNonHermitianPhysics2020,liYangLeeSingularityBCS2022,matsumotoEmbeddingYangLeeQuantum2022}.
%
For example, M.~Heyl \textit{et al.} suggested a connection between the thermodynamic phase transition and real-time evolution problems by introducing a complex effective temperature
as $\beta ~\sim \text{i}t$, revealing DQPT as the nonanalytical behavior at temporal zeros $t^*$ after quench~\cite{heylDynamicalQuantumPhase2013,heylDynamicalQuantumPhase2018}.
%
The complex interactions as the coupling of the complex intensive field and real extensive quantities also indicate the non-Hermitian nature of open quantum systems~\cite{yamamotoTheoryNonHermitianFermionic2019,ashidaNonHermitianPhysics2020, liYangLeeSingularityBCS2022,matsumotoEmbeddingYangLeeQuantum2022}.
%
The complex phase diagram might be manifested as ``hidden phases'', where the typically forbidden ferroelectric phase in strontium titanate can be transiently induced by infrared pulses or terahertz fields~\cite{novaMetastableFerroelectricityOptically2019,liTerahertzFieldInduced2019}.
%

%
Last but not least, the complex phase diagram employed should evoke a revisit for the definition of ``phase''.
%
Conventionally, the phases are defined in a viewpoint of phase transition: finding a physical path of transition from one to another, i.e. two phases are distinguished only when abrupt changes occur with varying real thermal fields.
%
While in the viewpoint of LY zeros, phase means unique analytic behaviors within a potential produced by zeros, where the geometric relationship between the location of the system's state and the cluster of zeros dominates by making the electrostatic analogy.
%
These two perspectives merge when there are real zeros.
%
However, the former is inadequate since crossover replaces phase transition in the supercritical region, leading to critical anomalies and inconsistent Widom lines instead of singularities and consistent phase boundaries.
%
The latter survives by providing zero determined complex phase diagram as a unified picture underlying phase transition and crossover.
%
The two cases are intuitively the analogies of fully screening potential with a closed shell of zeros and a flux leakage with cuts in this shell, respectively (Fig.~\ref{Scheme}(b)).
%
Crossover is just the consequence that the field produced by zeros leaks from one phase to another, with strength determined by the cut size (the closest distance of zeros to the physical plane) and the distance to the cut.
%
With these, we conclude by saying that the complex zeros stand firmly in physics which merits further experimental explorations with the state of the art of measuring techniques,
and a revisit of the definition of ``phase'' is needed, which shed light on a systematic mathematical description of phases including the complicated supercritical region.
%


\section{acknowledge}
\begin{acknowledgments}
    The authors acknowledge very insightful discussions with Prof. H. T. Quan. They are supported by the National Natural Science Foundation of China (Grant Nos. 12204015, 12234001, and  11934003), the National Basic Research Programs of China (Grant Nos. 2021YFA1400503 and 2022YFA1403500), Beijing Natural Science Foundation (Grant No. Z200004), and the Strategic Priority Research Program of the Chinese Academy of Sciences (Grant No. XDB33010400). The computational resources were provided by the supercomputer center at Peking University, China.
\end{acknowledgments}


% \bibliography{ref}

%merlin.mbs apsrev4-1.bst 2010-07-25 4.21a (PWD, AO, DPC) hacked
%Control: key (0)
%Control: author (8) initials jnrlst
%Control: editor formatted (1) identically to author
%Control: production of article title (-1) disabled
%Control: page (0) single
%Control: year (1) truncated
%Control: production of eprint (0) enabled
\begin{thebibliography}{49}%
    \makeatletter
    \providecommand \@ifxundefined [1]{%
        \@ifx{#1\undefined}
    }%
    \providecommand \@ifnum [1]{%
        \ifnum #1\expandafter \@firstoftwo
        \else \expandafter \@secondoftwo
        \fi
    }%
    \providecommand \@ifx [1]{%
        \ifx #1\expandafter \@firstoftwo
        \else \expandafter \@secondoftwo
        \fi
    }%
    \providecommand \natexlab [1]{#1}%
    \providecommand \enquote  [1]{``#1''}%
    \providecommand \bibnamefont  [1]{#1}%
    \providecommand \bibfnamefont [1]{#1}%
    \providecommand \citenamefont [1]{#1}%
    \providecommand \href@noop [0]{\@secondoftwo}%
    \providecommand \href [0]{\begingroup \@sanitize@url \@href}%
    \providecommand \@href[1]{\@@startlink{#1}\@@href}%
    \providecommand \@@href[1]{\endgroup#1\@@endlink}%
    \providecommand \@sanitize@url [0]{\catcode `\\12\catcode `\$12\catcode
        `\&12\catcode `\#12\catcode `\^12\catcode `\_12\catcode `\%12\relax}%
    \providecommand \@@startlink[1]{}%
    \providecommand \@@endlink[0]{}%
    \providecommand \url  [0]{\begingroup\@sanitize@url \@url }%
    \providecommand \@url [1]{\endgroup\@href {#1}{\urlprefix }}%
    \providecommand \urlprefix  [0]{URL }%
    \providecommand \Eprint [0]{\href }%
    \providecommand \doibase [0]{http://dx.doi.org/}%
    \providecommand \selectlanguage [0]{\@gobble}%
    \providecommand \bibinfo  [0]{\@secondoftwo}%
    \providecommand \bibfield  [0]{\@secondoftwo}%
    \providecommand \translation [1]{[#1]}%
    \providecommand \BibitemOpen [0]{}%
    \providecommand \bibitemStop [0]{}%
    \providecommand \bibitemNoStop [0]{.\EOS\space}%
    \providecommand \EOS [0]{\spacefactor3000\relax}%
    \providecommand \BibitemShut  [1]{\csname bibitem#1\endcsname}%
    \let\auto@bib@innerbib\@empty
    %</preamble>
    \bibitem [{\citenamefont {Yang}\ and\ \citenamefont
                {Lee}(1952)}]{yangStatisticalTheoryEquations1952}%
    \BibitemOpen
    \bibfield  {author} {\bibinfo {author} {\bibfnamefont {C.~N.}\ \bibnamefont
            {Yang}}\ and\ \bibinfo {author} {\bibfnamefont {T.~D.}\ \bibnamefont {Lee}},\
    }\href {\doibase 10.1103/PhysRev.87.404} {\bibfield  {journal} {\bibinfo
            {journal} {Phys. Rev.}\ }\textbf {\bibinfo {volume} {87}},\ \bibinfo {pages}
        {404} (\bibinfo {year} {1952})}\BibitemShut {NoStop}%
    \bibitem [{\citenamefont {Lee}\ and\ \citenamefont
                {Yang}(1952)}]{leeStatisticalTheoryEquations1952}%
    \BibitemOpen
    \bibfield  {author} {\bibinfo {author} {\bibfnamefont {T.~D.}\ \bibnamefont
            {Lee}}\ and\ \bibinfo {author} {\bibfnamefont {C.~N.}\ \bibnamefont {Yang}},\
    }\href {\doibase 10.1103/PhysRev.87.410} {\bibfield  {journal} {\bibinfo
            {journal} {Phys. Rev.}\ }\textbf {\bibinfo {volume} {87}},\ \bibinfo {pages}
        {410} (\bibinfo {year} {1952})}\BibitemShut {NoStop}%
    \bibitem [{\citenamefont
                {Fisher}(1965)}]{fisherLecturesTheoreticalPhysics1965}%
    \BibitemOpen
    \bibfield  {author} {\bibinfo {author} {\bibnamefont {Fisher}},\ }\href@noop
    {} {\emph {\bibinfo {title} {Lectures in Theoretical Physics}}},\
    Vol.~\bibinfo {volume} {7C}\ (\bibinfo  {publisher} {University of Colorado
        Press},\ \bibinfo {year} {1965})\BibitemShut {NoStop}%
    \bibitem [{\citenamefont {Peng}\ \emph {et~al.}(2015)\citenamefont {Peng},
                \citenamefont {Zhou}, \citenamefont {Wei}, \citenamefont {Cui}, \citenamefont
                {Du},\ and\ \citenamefont {Liu}}]{pengExperimentalObservationLeeYang2015}%
    \BibitemOpen
    \bibfield  {author} {\bibinfo {author} {\bibfnamefont {X.}~\bibnamefont
            {Peng}}, \bibinfo {author} {\bibfnamefont {H.}~\bibnamefont {Zhou}}, \bibinfo
        {author} {\bibfnamefont {B.-B.}\ \bibnamefont {Wei}}, \bibinfo {author}
        {\bibfnamefont {J.}~\bibnamefont {Cui}}, \bibinfo {author} {\bibfnamefont
            {J.}~\bibnamefont {Du}}, \ and\ \bibinfo {author} {\bibfnamefont {R.-B.}\
            \bibnamefont {Liu}},\ }\href {\doibase 10.1103/PhysRevLett.114.010601}
    {\bibfield  {journal} {\bibinfo  {journal} {Phys. Rev. Lett.}\ }\textbf
        {\bibinfo {volume} {114}},\ \bibinfo {pages} {010601} (\bibinfo {year}
        {2015})}\BibitemShut {NoStop}%
    \bibitem [{\citenamefont {Wei}\ and\ \citenamefont
                {Liu}(2012)}]{weiLeeYangZerosCritical2012}%
    \BibitemOpen
    \bibfield  {author} {\bibinfo {author} {\bibfnamefont {B.-B.}\ \bibnamefont
            {Wei}}\ and\ \bibinfo {author} {\bibfnamefont {R.-B.}\ \bibnamefont {Liu}},\
    }\href {\doibase 10.1103/PhysRevLett.109.185701} {\bibfield  {journal}
        {\bibinfo  {journal} {Phys. Rev. Lett.}\ }\textbf {\bibinfo {volume} {109}},\
        \bibinfo {pages} {185701} (\bibinfo {year} {2012})}\BibitemShut {NoStop}%
    \bibitem [{\citenamefont {Brandner}\ \emph
                {et~al.}(2017{\natexlab{a}})\citenamefont {Brandner}, \citenamefont {Maisi},
                \citenamefont {Pekola}, \citenamefont {Garrahan},\ and\ \citenamefont
                {Flindt}}]{brandnerExperimentalDeterminationDynamical2017}%
    \BibitemOpen
    \bibfield  {author} {\bibinfo {author} {\bibfnamefont {K.}~\bibnamefont
            {Brandner}}, \bibinfo {author} {\bibfnamefont {V.~F.}\ \bibnamefont {Maisi}},
        \bibinfo {author} {\bibfnamefont {J.~P.}\ \bibnamefont {Pekola}}, \bibinfo
        {author} {\bibfnamefont {J.~P.}\ \bibnamefont {Garrahan}}, \ and\ \bibinfo
        {author} {\bibfnamefont {C.}~\bibnamefont {Flindt}},\ }\href {\doibase
        10.1103/PhysRevLett.118.180601} {\bibfield  {journal} {\bibinfo  {journal}
            {Phys. Rev. Lett.}\ }\textbf {\bibinfo {volume} {118}},\ \bibinfo {pages}
        {180601} (\bibinfo {year} {2017}{\natexlab{a}})}\BibitemShut {NoStop}%
    \bibitem [{\citenamefont {Flindt}\ and\ \citenamefont
                {Garrahan}(2013{\natexlab{a}})}]{flindtTrajectoryPhaseTransitions2013}%
    \BibitemOpen
    \bibfield  {author} {\bibinfo {author} {\bibfnamefont {C.}~\bibnamefont
            {Flindt}}\ and\ \bibinfo {author} {\bibfnamefont {J.~P.}\ \bibnamefont
            {Garrahan}},\ }\href {\doibase 10.1103/PhysRevLett.110.050601} {\bibfield
        {journal} {\bibinfo  {journal} {Phys. Rev. Lett.}\ }\textbf {\bibinfo
            {volume} {110}},\ \bibinfo {pages} {050601} (\bibinfo {year}
        {2013}{\natexlab{a}})}\BibitemShut {NoStop}%
    \bibitem [{\citenamefont {Francis}\ \emph {et~al.}(2021)\citenamefont
                {Francis}, \citenamefont {Zhu}, \citenamefont {Huerta~Alderete},
                \citenamefont {Johri}, \citenamefont {Xiao}, \citenamefont {Freericks},
                \citenamefont {Monroe}, \citenamefont {Linke},\ and\ \citenamefont
                {Kemper}}]{francisManybodyThermodynamicsQuantum2021}%
    \BibitemOpen
    \bibfield  {author} {\bibinfo {author} {\bibfnamefont {A.}~\bibnamefont
            {Francis}}, \bibinfo {author} {\bibfnamefont {D.}~\bibnamefont {Zhu}},
        \bibinfo {author} {\bibfnamefont {C.}~\bibnamefont {Huerta~Alderete}},
        \bibinfo {author} {\bibfnamefont {S.}~\bibnamefont {Johri}}, \bibinfo
        {author} {\bibfnamefont {X.}~\bibnamefont {Xiao}}, \bibinfo {author}
        {\bibfnamefont {J.~K.}\ \bibnamefont {Freericks}}, \bibinfo {author}
        {\bibfnamefont {C.}~\bibnamefont {Monroe}}, \bibinfo {author} {\bibfnamefont
            {N.~M.}\ \bibnamefont {Linke}}, \ and\ \bibinfo {author} {\bibfnamefont
            {A.~F.}\ \bibnamefont {Kemper}},\ }\href {\doibase 10.1126/sciadv.abf2447}
    {\bibfield  {journal} {\bibinfo  {journal} {Sci. Adv.}\ }\textbf {\bibinfo
            {volume} {7}},\ \bibinfo {pages} {eabf2447} (\bibinfo {year}
        {2021})}\BibitemShut {NoStop}%
    \bibitem [{\citenamefont {Krishnan}\ \emph {et~al.}(2019)\citenamefont
                {Krishnan}, \citenamefont {Schmitt}, \citenamefont {Moessner},\ and\
                \citenamefont {Heyl}}]{krishnanMeasuringComplexpartitionfunctionZeros2019}%
    \BibitemOpen
    \bibfield  {author} {\bibinfo {author} {\bibfnamefont {A.}~\bibnamefont
            {Krishnan}}, \bibinfo {author} {\bibfnamefont {M.}~\bibnamefont {Schmitt}},
        \bibinfo {author} {\bibfnamefont {R.}~\bibnamefont {Moessner}}, \ and\
        \bibinfo {author} {\bibfnamefont {M.}~\bibnamefont {Heyl}},\ }\href {\doibase
        10.1103/PhysRevA.100.022125} {\bibfield  {journal} {\bibinfo  {journal}
            {Phys. Rev. A}\ }\textbf {\bibinfo {volume} {100}},\ \bibinfo {pages}
        {022125} (\bibinfo {year} {2019})}\BibitemShut {NoStop}%
    \bibitem [{\citenamefont {Xu}\ and\ \citenamefont {{del
                            Campo}}(2019)}]{xuProbingFullDistribution2019}%
    \BibitemOpen
    \bibfield  {author} {\bibinfo {author} {\bibfnamefont {Z.}~\bibnamefont
            {Xu}}\ and\ \bibinfo {author} {\bibfnamefont {A.}~\bibnamefont {{del
                        Campo}}},\ }\href {\doibase 10.1103/PhysRevLett.122.160602} {\bibfield
        {journal} {\bibinfo  {journal} {Phys. Rev. Lett.}\ }\textbf {\bibinfo
            {volume} {122}},\ \bibinfo {pages} {160602} (\bibinfo {year}
        {2019})}\BibitemShut {NoStop}%
    \bibitem [{\citenamefont {Gnatenko}\ \emph {et~al.}(2017)\citenamefont
                {Gnatenko}, \citenamefont {Kargol},\ and\ \citenamefont
                {Tkachuk}}]{gnatenkoTwotimeCorrelationFunctions2017}%
    \BibitemOpen
    \bibfield  {author} {\bibinfo {author} {\bibfnamefont {K.~P.}\ \bibnamefont
            {Gnatenko}}, \bibinfo {author} {\bibfnamefont {A.}~\bibnamefont {Kargol}}, \
        and\ \bibinfo {author} {\bibfnamefont {V.~M.}\ \bibnamefont {Tkachuk}},\
    }\href {\doibase 10.1103/PhysRevE.96.032116} {\bibfield  {journal} {\bibinfo
            {journal} {Phys. Rev. E}\ }\textbf {\bibinfo {volume} {96}},\ \bibinfo
        {pages} {032116} (\bibinfo {year} {2017})}\BibitemShut {NoStop}%
    \bibitem [{\citenamefont {Heyl}(2018)}]{heylDynamicalQuantumPhase2018}%
    \BibitemOpen
    \bibfield  {author} {\bibinfo {author} {\bibfnamefont {M.}~\bibnamefont
            {Heyl}},\ }\href {\doibase 10.1088/1361-6633/aaaf9a} {\bibfield  {journal}
        {\bibinfo  {journal} {Rep. Prog. Phys.}\ }\textbf {\bibinfo {volume} {81}},\
        \bibinfo {pages} {054001} (\bibinfo {year} {2018})}\BibitemShut {NoStop}%
    \bibitem [{\citenamefont {Heyl}\ \emph {et~al.}(2013)\citenamefont {Heyl},
                \citenamefont {Polkovnikov},\ and\ \citenamefont
                {Kehrein}}]{heylDynamicalQuantumPhase2013}%
    \BibitemOpen
    \bibfield  {author} {\bibinfo {author} {\bibfnamefont {M.}~\bibnamefont
            {Heyl}}, \bibinfo {author} {\bibfnamefont {A.}~\bibnamefont {Polkovnikov}}, \
        and\ \bibinfo {author} {\bibfnamefont {S.}~\bibnamefont {Kehrein}},\ }\href
    {\doibase 10.1103/PhysRevLett.110.135704} {\bibfield  {journal} {\bibinfo
            {journal} {Phys. Rev. Lett.}\ }\textbf {\bibinfo {volume} {110}},\ \bibinfo
        {pages} {135704} (\bibinfo {year} {2013})}\BibitemShut {NoStop}%
    \bibitem [{\citenamefont {Weingärtner}\ and\ \citenamefont
                {Franck}(2005)}]{WeingartnerFranck2005}%
    \BibitemOpen
    \bibfield  {author} {\bibinfo {author} {\bibfnamefont {H.}~\bibnamefont
            {Weingärtner}}\ and\ \bibinfo {author} {\bibfnamefont {E.~U.}\ \bibnamefont
            {Franck}},\ }\href {\doibase 10.1002/anie.200462468} {\bibfield  {journal}
        {\bibinfo  {journal} {Angew. Chem., Int. Ed.}\ }\textbf {\bibinfo {volume}
            {44}},\ \bibinfo {pages} {2672–2692} (\bibinfo {year} {2005})}\BibitemShut
    {NoStop}%
    \bibitem [{\citenamefont {Savage}(1999)}]{Savage1999}%
    \BibitemOpen
    \bibfield  {author} {\bibinfo {author} {\bibfnamefont {P.~E.}\ \bibnamefont
            {Savage}},\ }\href {\doibase 10.1021/cr9700989} {\bibfield  {journal}
        {\bibinfo  {journal} {Chem. Rev.}\ }\textbf {\bibinfo {volume} {99}},\
        \bibinfo {pages} {603–622} (\bibinfo {year} {1999})}\BibitemShut {NoStop}%
    \bibitem [{\citenamefont {Keppler}(1996)}]{Keppler1996}%
    \BibitemOpen
    \bibfield  {author} {\bibinfo {author} {\bibfnamefont {H.}~\bibnamefont
            {Keppler}},\ }\href {\doibase 10.1038/380237a0} {\bibfield  {journal}
        {\bibinfo  {journal} {Nature}\ }\textbf {\bibinfo {volume} {380}},\ \bibinfo
        {pages} {237–240} (\bibinfo {year} {1996})}\BibitemShut {NoStop}%
    \bibitem [{\citenamefont {Cagniard de~la Tour}(1822)}]{Cagniard1822}%
    \BibitemOpen
    \bibfield  {author} {\bibinfo {author} {\bibfnamefont {C.}~\bibnamefont
            {Cagniard de~la Tour}},\ }\href@noop {} {\bibfield  {journal} {\bibinfo
            {journal} {Ann. Chim. Phys.}\ }\textbf {\bibinfo {volume} {21}} (\bibinfo
        {year} {1822})}\BibitemShut {NoStop}%
    \bibitem [{\citenamefont {Postorino}\ \emph {et~al.}(1993)\citenamefont
                {Postorino}, \citenamefont {Tromp}, \citenamefont {Ricci}, \citenamefont
                {Soper},\ and\ \citenamefont
                {Neilson}}]{Postorino_Tromp_Ricci_Soper_Neilson_1993}%
    \BibitemOpen
    \bibfield  {author} {\bibinfo {author} {\bibfnamefont {P.}~\bibnamefont
            {Postorino}}, \bibinfo {author} {\bibfnamefont {R.~H.}\ \bibnamefont
            {Tromp}}, \bibinfo {author} {\bibfnamefont {M.-A.}\ \bibnamefont {Ricci}},
        \bibinfo {author} {\bibfnamefont {A.~K.}\ \bibnamefont {Soper}}, \ and\
        \bibinfo {author} {\bibfnamefont {G.~W.}\ \bibnamefont {Neilson}},\ }\href
    {\doibase 10.1038/366668a0} {\bibfield  {journal} {\bibinfo  {journal}
            {Nature}\ }\textbf {\bibinfo {volume} {366}},\ \bibinfo {pages} {668–670}
        (\bibinfo {year} {1993})}\BibitemShut {NoStop}%
    \bibitem [{\citenamefont {Smith}\ and\ \citenamefont
                {Kay}(1999)}]{Smith_Kay_1999}%
    \BibitemOpen
    \bibfield  {author} {\bibinfo {author} {\bibfnamefont {R.~S.}\ \bibnamefont
            {Smith}}\ and\ \bibinfo {author} {\bibfnamefont {B.~D.}\ \bibnamefont
            {Kay}},\ }\href {\doibase 10.1038/19725} {\bibfield  {journal} {\bibinfo
            {journal} {Nature}\ }\textbf {\bibinfo {volume} {398}},\ \bibinfo {pages}
        {788–791} (\bibinfo {year} {1999})}\BibitemShut {NoStop}%
    \bibitem [{\citenamefont {McMillan}\ and\ \citenamefont
                {Stanley}(2010)}]{McMillan_Stanley_2010}%
    \BibitemOpen
    \bibfield  {author} {\bibinfo {author} {\bibfnamefont {P.}~\bibnamefont
            {McMillan}}\ and\ \bibinfo {author} {\bibfnamefont {H.}~\bibnamefont
            {Stanley}},\ }\href {\doibase 10.1038/nphys1711} {\bibfield  {journal}
        {\bibinfo  {journal} {Nat. Phys.}\ }\textbf {\bibinfo {volume} {6}},\
        \bibinfo {pages} {479–480} (\bibinfo {year} {2010})}\BibitemShut {NoStop}%
    \bibitem [{\citenamefont {Galli}\ and\ \citenamefont
                {Pan}(2013)}]{Galli_Pan_2013}%
    \BibitemOpen
    \bibfield  {author} {\bibinfo {author} {\bibfnamefont {G.}~\bibnamefont
            {Galli}}\ and\ \bibinfo {author} {\bibfnamefont {D.}~\bibnamefont {Pan}},\
    }\href {\doibase 10.1073/pnas.1303740110} {\bibfield  {journal} {\bibinfo
            {journal} {Proc. Natl. Acad. Sci. U.S.A.}\ }\textbf {\bibinfo {volume}
            {110}},\ \bibinfo {pages} {6250–6251} (\bibinfo {year} {2013})}\BibitemShut
    {NoStop}%
    \bibitem [{\citenamefont {D’Souza}\ and\ \citenamefont
                {Nagler}(2015)}]{DSouza_Nagler_2015}%
    \BibitemOpen
    \bibfield  {author} {\bibinfo {author} {\bibfnamefont {R.~M.}\ \bibnamefont
            {D’Souza}}\ and\ \bibinfo {author} {\bibfnamefont {J.}~\bibnamefont
            {Nagler}},\ }\href {\doibase 10.1038/nphys3378} {\bibfield  {journal}
        {\bibinfo  {journal} {Nat. Phys.}\ }\textbf {\bibinfo {volume} {11}},\
        \bibinfo {pages} {531–538} (\bibinfo {year} {2015})}\BibitemShut {NoStop}%
    \bibitem [{\citenamefont {Cheng}\ \emph {et~al.}(2020)\citenamefont {Cheng},
                \citenamefont {Mazzola}, \citenamefont {Pickard},\ and\ \citenamefont
                {Ceriotti}}]{Cheng_Mazzola_Pickard_Ceriotti_2020}%
    \BibitemOpen
    \bibfield  {author} {\bibinfo {author} {\bibfnamefont {B.}~\bibnamefont
            {Cheng}}, \bibinfo {author} {\bibfnamefont {G.}~\bibnamefont {Mazzola}},
        \bibinfo {author} {\bibfnamefont {C.~J.}\ \bibnamefont {Pickard}}, \ and\
        \bibinfo {author} {\bibfnamefont {M.}~\bibnamefont {Ceriotti}},\ }\href
    {\doibase 10.1038/s41586-020-2677-y} {\bibfield  {journal} {\bibinfo
            {journal} {Nature}\ }\textbf {\bibinfo {volume} {585}},\ \bibinfo {pages}
        {217–220} (\bibinfo {year} {2020})}\BibitemShut {NoStop}%
    \bibitem [{\citenamefont {Cockrell}\ and\ \citenamefont
                {Trachenko}(2022)}]{Cockrell_Trachenko_2022}%
    \BibitemOpen
    \bibfield  {author} {\bibinfo {author} {\bibfnamefont {C.}~\bibnamefont
            {Cockrell}}\ and\ \bibinfo {author} {\bibfnamefont {K.}~\bibnamefont
            {Trachenko}},\ }\href {\doibase 10.1126/sciadv.abq5183} {\bibfield  {journal}
        {\bibinfo  {journal} {Sci. Adv.}\ }\textbf {\bibinfo {volume} {8}},\ \bibinfo
        {pages} {eabq5183} (\bibinfo {year} {2022})}\BibitemShut {NoStop}%
    \bibitem [{\citenamefont {Williams}\ and\ \citenamefont
                {Clifford}()}]{williamsSupercriticalFluidMethods2000}%
    \BibitemOpen
    \bibfield  {author} {\bibinfo {author} {\bibfnamefont {J.~R.}\ \bibnamefont
            {Williams}}\ and\ \bibinfo {author} {\bibfnamefont {A.~A.}\ \bibnamefont
            {Clifford}},\ }\href {\doibase 10.1385/1592590306} {\emph {\bibinfo {title}
            {Supercritical {{Fluid Methods}} and {{Protocols}}}}},\ Vol.~\bibinfo
    {volume} {13}\ (\bibinfo  {publisher} {{Humana Press}})\BibitemShut {NoStop}%
    \bibitem [{\citenamefont {Kiran}\ \emph {et~al.}()\citenamefont {Kiran},
                \citenamefont {Debenedetti},\ and\ \citenamefont
                {Peters}}]{kiranSupercriticalFluids2000}%
    \BibitemOpen
    \bibinfo {editor} {\bibfnamefont {E.}~\bibnamefont {Kiran}}, \bibinfo
    {editor} {\bibfnamefont {P.~G.}\ \bibnamefont {Debenedetti}}, \ and\ \bibinfo
    {editor} {\bibfnamefont {C.~J.}\ \bibnamefont {Peters}},\ eds.,\ \href
    {\doibase 10.1007/978-94-011-3929-8} {\emph {\bibinfo {title} {Supercritical
                    {{Fluids}}}}}\ (\bibinfo  {publisher} {{Springer Netherlands}})\BibitemShut
    {NoStop}%
    \bibitem [{\citenamefont {Proctor}(2020)}]{Proctor2020}%
    \BibitemOpen
    \bibfield  {author} {\bibinfo {author} {\bibfnamefont {J.~E.}\ \bibnamefont
            {Proctor}},\ }\href
    {http://gen.lib.rus.ec/book/index.php?md5=C53AC5493825D49FAA6028B62FDF1737}
    {\emph {\bibinfo {title} {The liquid and supercritical fluid states of
                matter}}},\ \bibinfo {edition} {first edition}\ ed.\ (\bibinfo  {publisher}
    {CRC Press},\ \bibinfo {year} {2020})\BibitemShut {NoStop}%
    \bibitem [{\citenamefont {Xu}\ \emph {et~al.}(2005)\citenamefont {Xu},
                \citenamefont {Kumar}, \citenamefont {Buldyrev}, \citenamefont {Chen},
                \citenamefont {Poole}, \citenamefont {Sciortino},\ and\ \citenamefont
                {Stanley}}]{xuRelationWidomLine2005}%
    \BibitemOpen
    \bibfield  {author} {\bibinfo {author} {\bibfnamefont {L.}~\bibnamefont
            {Xu}}, \bibinfo {author} {\bibfnamefont {P.}~\bibnamefont {Kumar}}, \bibinfo
        {author} {\bibfnamefont {S.~V.}\ \bibnamefont {Buldyrev}}, \bibinfo {author}
        {\bibfnamefont {S.-H.}\ \bibnamefont {Chen}}, \bibinfo {author}
        {\bibfnamefont {P.~H.}\ \bibnamefont {Poole}}, \bibinfo {author}
        {\bibfnamefont {F.}~\bibnamefont {Sciortino}}, \ and\ \bibinfo {author}
        {\bibfnamefont {H.~E.}\ \bibnamefont {Stanley}},\ }\href {\doibase
        10.1073/pnas.0507870102} {\bibfield  {journal} {\bibinfo  {journal} {Proc.
                Natl. Acad. Sci. U.S.A.}\ }\textbf {\bibinfo {volume} {102}},\ \bibinfo
        {pages} {16558} (\bibinfo {year} {2005})}\BibitemShut {NoStop}%
    \bibitem [{\citenamefont {Abascal}\ and\ \citenamefont
                {Vega}(2010)}]{abascalWidomLineLiquid2010}%
    \BibitemOpen
    \bibfield  {author} {\bibinfo {author} {\bibfnamefont {J.~L.~F.}\
            \bibnamefont {Abascal}}\ and\ \bibinfo {author} {\bibfnamefont
            {C.}~\bibnamefont {Vega}},\ }\href {\doibase 10.1063/1.3506860} {\bibfield
        {journal} {\bibinfo  {journal} {J. Chem. Phys.}\ }\textbf {\bibinfo {volume}
            {133}},\ \bibinfo {pages} {234502} (\bibinfo {year} {2010})}\BibitemShut
    {NoStop}%
    \bibitem [{\citenamefont {Gallo}\ \emph {et~al.}(2014)\citenamefont {Gallo},
                \citenamefont {Corradini},\ and\ \citenamefont
                {Rovere}}]{galloWidomLineDynamical2014}%
    \BibitemOpen
    \bibfield  {author} {\bibinfo {author} {\bibfnamefont {P.}~\bibnamefont
            {Gallo}}, \bibinfo {author} {\bibfnamefont {D.}~\bibnamefont {Corradini}}, \
        and\ \bibinfo {author} {\bibfnamefont {M.}~\bibnamefont {Rovere}},\ }\href
    {\doibase 10.1038/ncomms6806} {\bibfield  {journal} {\bibinfo  {journal}
            {Nat. Commun.}\ }\textbf {\bibinfo {volume} {5}},\ \bibinfo {pages} {5806}
        (\bibinfo {year} {2014})}\BibitemShut {NoStop}%
    \bibitem [{\citenamefont {Li}\ \emph {et~al.}(2015)\citenamefont {Li},
                \citenamefont {Chen}, \citenamefont {Li}, \citenamefont {Wang},\ and\
                \citenamefont {Xu}}]{liSupercriticalPhenomenonHydrogen2015}%
    \BibitemOpen
    \bibfield  {author} {\bibinfo {author} {\bibfnamefont {R.}~\bibnamefont
            {Li}}, \bibinfo {author} {\bibfnamefont {J.}~\bibnamefont {Chen}}, \bibinfo
        {author} {\bibfnamefont {X.}~\bibnamefont {Li}}, \bibinfo {author}
        {\bibfnamefont {E.}~\bibnamefont {Wang}}, \ and\ \bibinfo {author}
        {\bibfnamefont {L.}~\bibnamefont {Xu}},\ }\href {\doibase
        10.1088/1367-2630/17/6/063023} {\bibfield  {journal} {\bibinfo  {journal}
            {New J. Phys.}\ }\textbf {\bibinfo {volume} {17}},\ \bibinfo {pages} {063023}
        (\bibinfo {year} {2015})}\BibitemShut {NoStop}%
    \bibitem [{\citenamefont {Luo}\ \emph {et~al.}(2014)\citenamefont {Luo},
                \citenamefont {Xu}, \citenamefont {Lascaris}, \citenamefont {Stanley},\ and\
                \citenamefont {Buldyrev}}]{luoBehaviorWidomLine2014}%
    \BibitemOpen
    \bibfield  {author} {\bibinfo {author} {\bibfnamefont {J.}~\bibnamefont
            {Luo}}, \bibinfo {author} {\bibfnamefont {L.}~\bibnamefont {Xu}}, \bibinfo
        {author} {\bibfnamefont {E.}~\bibnamefont {Lascaris}}, \bibinfo {author}
        {\bibfnamefont {H.~E.}\ \bibnamefont {Stanley}}, \ and\ \bibinfo {author}
        {\bibfnamefont {S.~V.}\ \bibnamefont {Buldyrev}},\ }\href {\doibase
        10.1103/PhysRevLett.112.135701} {\bibfield  {journal} {\bibinfo  {journal}
            {Phys. Rev. Lett.}\ }\textbf {\bibinfo {volume} {112}},\ \bibinfo {pages}
        {135701} (\bibinfo {year} {2014})}\BibitemShut {NoStop}%
    \bibitem [{\citenamefont {Lupi}\ \emph {et~al.}(2021)\citenamefont {Lupi},
    \citenamefont {V{\'a}zquez~Ram{\'i}rez},\ and\ \citenamefont
    {Gallo}}]{lupiDynamicalCrossoverIts2021}%
    \BibitemOpen
    \bibfield  {author} {\bibinfo {author} {\bibfnamefont {L.}~\bibnamefont
        {Lupi}}, \bibinfo {author} {\bibfnamefont {B.}~\bibnamefont
    {V{\'a}zquez~Ram{\'i}rez}}, \ and\ \bibinfo {author} {\bibfnamefont
        {P.}~\bibnamefont {Gallo}},\ }\href {\doibase 10.1063/5.0059190} {\bibfield
        {journal} {\bibinfo  {journal} {J. Chem. Phys.}\ }\textbf {\bibinfo {volume}
            {155}},\ \bibinfo {pages} {054502} (\bibinfo {year} {2021})}\BibitemShut
    {NoStop}%
    \bibitem [{\citenamefont {Gartner}\ \emph {et~al.}(2021)\citenamefont
                {Gartner}, \citenamefont {Torquato}, \citenamefont {Car},\ and\ \citenamefont
                {Debenedetti}}]{gartnerManifestationsMetastableCriticality2021}%
    \BibitemOpen
    \bibfield  {author} {\bibinfo {author} {\bibfnamefont {T.~E.}\ \bibnamefont
            {Gartner}}, \bibinfo {author} {\bibfnamefont {S.}~\bibnamefont {Torquato}},
        \bibinfo {author} {\bibfnamefont {R.}~\bibnamefont {Car}}, \ and\ \bibinfo
        {author} {\bibfnamefont {P.~G.}\ \bibnamefont {Debenedetti}},\ }\href
    {\doibase 10.1038/s41467-021-23639-2} {\bibfield  {journal} {\bibinfo
            {journal} {Nat. Commun.}\ }\textbf {\bibinfo {volume} {12}},\ \bibinfo
        {pages} {3398} (\bibinfo {year} {2021})}\BibitemShut {NoStop}%
    \bibitem [{\citenamefont {Palmer}\ \emph {et~al.}(2014)\citenamefont {Palmer},
                \citenamefont {Martelli}, \citenamefont {Liu}, \citenamefont {Car},
                \citenamefont {Panagiotopoulos},\ and\ \citenamefont
                {Debenedetti}}]{palmerMetastableLiquidLiquid2014}%
    \BibitemOpen
    \bibfield  {author} {\bibinfo {author} {\bibfnamefont {J.~C.}\ \bibnamefont
            {Palmer}}, \bibinfo {author} {\bibfnamefont {F.}~\bibnamefont {Martelli}},
        \bibinfo {author} {\bibfnamefont {Y.}~\bibnamefont {Liu}}, \bibinfo {author}
        {\bibfnamefont {R.}~\bibnamefont {Car}}, \bibinfo {author} {\bibfnamefont
            {A.~Z.}\ \bibnamefont {Panagiotopoulos}}, \ and\ \bibinfo {author}
        {\bibfnamefont {P.~G.}\ \bibnamefont {Debenedetti}},\ }\href {\doibase
        10.1038/nature13405} {\bibfield  {journal} {\bibinfo  {journal} {Nature}\
        }\textbf {\bibinfo {volume} {510}},\ \bibinfo {pages} {385} (\bibinfo {year}
        {2014})}\BibitemShut {NoStop}%
    \bibitem [{\citenamefont {Poole}\ \emph {et~al.}(1992)\citenamefont {Poole},
                \citenamefont {Sciortino}, \citenamefont {Essmann},\ and\ \citenamefont
                {Stanley}}]{poolePhaseBehaviourMetastable1992}%
    \BibitemOpen
    \bibfield  {author} {\bibinfo {author} {\bibfnamefont {P.~H.}\ \bibnamefont
            {Poole}}, \bibinfo {author} {\bibfnamefont {F.}~\bibnamefont {Sciortino}},
        \bibinfo {author} {\bibfnamefont {U.}~\bibnamefont {Essmann}}, \ and\
        \bibinfo {author} {\bibfnamefont {H.~E.}\ \bibnamefont {Stanley}},\ }\href
    {\doibase 10.1038/360324a0} {\bibfield  {journal} {\bibinfo  {journal}
            {Nature}\ }\textbf {\bibinfo {volume} {360}},\ \bibinfo {pages} {324}
        (\bibinfo {year} {1992})}\BibitemShut {NoStop}%
    \bibitem [{\citenamefont {Ba{\c
                            s}ar}(2021)}]{basarUniversalityLeeYangSingularities2021}%
    \BibitemOpen
    \bibfield  {author} {\bibinfo {author} {\bibfnamefont {G.}~\bibnamefont
            {Ba{\c s}ar}},\ }\href {\doibase 10.1103/PhysRevLett.127.171603} {\bibfield
        {journal} {\bibinfo  {journal} {Phys. Rev. Lett.}\ }\textbf {\bibinfo
            {volume} {127}},\ \bibinfo {pages} {171603} (\bibinfo {year}
        {2021})}\BibitemShut {NoStop}%
    \bibitem [{\citenamefont {Connelly}\ \emph {et~al.}(2020)\citenamefont
                {Connelly}, \citenamefont {Johnson}, \citenamefont {Rennecke},\ and\
                \citenamefont {Skokov}}]{connellyUniversalLocationYangLee2020}%
    \BibitemOpen
    \bibfield  {author} {\bibinfo {author} {\bibfnamefont {A.}~\bibnamefont
            {Connelly}}, \bibinfo {author} {\bibfnamefont {G.}~\bibnamefont {Johnson}},
        \bibinfo {author} {\bibfnamefont {F.}~\bibnamefont {Rennecke}}, \ and\
        \bibinfo {author} {\bibfnamefont {V.~V.}\ \bibnamefont {Skokov}},\ }\href
    {\doibase 10.1103/PhysRevLett.125.191602} {\bibfield  {journal} {\bibinfo
            {journal} {Phys. Rev. Lett.}\ }\textbf {\bibinfo {volume} {125}},\ \bibinfo
        {pages} {191602} (\bibinfo {year} {2020})}\BibitemShut {NoStop}%
    \bibitem [{SI()}]{SI}%
    \BibitemOpen
    \href@noop {} {}\bibinfo {note} {See supplemental material at http://xxx for
        details of the method and computational setups, as well as additional
        discussions.}\BibitemShut {Stop}%
    \bibitem [{Note1()}]{Note1}%
    \BibitemOpen
    \bibinfo {note} {The coefficients $c_k$ are real and positive since they are
        simply the integrals of the Boltzmann-Gibbs factors. Consequently, $Z_T(y)$
        are always positive for real-valued $y$, i.e., algebra equation $Z_T(y) = 0$
        are free of real roots for finite $V$ and hence finite $N_{\protect \text
                    {max}}$.}\BibitemShut {Stop}%
    \bibitem [{\citenamefont {Flindt}\ and\ \citenamefont
                {Garrahan}(2013{\natexlab{b}})}]{flindtTrajectoryPhaseTransitions2013a}%
    \BibitemOpen
    \bibfield  {author} {\bibinfo {author} {\bibfnamefont {C.}~\bibnamefont
            {Flindt}}\ and\ \bibinfo {author} {\bibfnamefont {J.~P.}\ \bibnamefont
            {Garrahan}},\ }\href {\doibase 10.1103/PhysRevLett.110.050601} {\bibfield
        {journal} {\bibinfo  {journal} {Phys. Rev. Lett.}\ }\textbf {\bibinfo
            {volume} {110}},\ \bibinfo {pages} {050601} (\bibinfo {year}
        {2013}{\natexlab{b}})}\BibitemShut {NoStop}%
    \bibitem [{\citenamefont {Brandner}\ \emph
                {et~al.}(2017{\natexlab{b}})\citenamefont {Brandner}, \citenamefont {Maisi},
                \citenamefont {Pekola}, \citenamefont {Garrahan},\ and\ \citenamefont
                {Flindt}}]{Brandner2017}%
    \BibitemOpen
    \bibfield  {author} {\bibinfo {author} {\bibfnamefont {K.}~\bibnamefont
            {Brandner}}, \bibinfo {author} {\bibfnamefont {V.~F.}\ \bibnamefont {Maisi}},
        \bibinfo {author} {\bibfnamefont {J.~P.}\ \bibnamefont {Pekola}}, \bibinfo
        {author} {\bibfnamefont {J.~P.}\ \bibnamefont {Garrahan}}, \ and\ \bibinfo
        {author} {\bibfnamefont {C.}~\bibnamefont {Flindt}},\ }\href {\doibase
        10.1103/PhysRevLett.118.180601} {\bibfield  {journal} {\bibinfo  {journal}
            {Phys. Rev. Lett.}\ }\textbf {\bibinfo {volume} {118}},\ \bibinfo {pages}
        {180601} (\bibinfo {year} {2017}{\natexlab{b}})}\BibitemShut {NoStop}%
    \bibitem [{\citenamefont {Brange}\ \emph {et~al.}(2023)\citenamefont {Brange},
                \citenamefont {Pyh{\"a}ranta}, \citenamefont {Heinonen}, \citenamefont
                {Brandner},\ and\ \citenamefont
                {Flindt}}]{brangeLeeYangTheoryBoseEinstein2023}%
    \BibitemOpen
    \bibfield  {author} {\bibinfo {author} {\bibfnamefont {F.}~\bibnamefont
            {Brange}}, \bibinfo {author} {\bibfnamefont {T.}~\bibnamefont
            {Pyh{\"a}ranta}}, \bibinfo {author} {\bibfnamefont {E.}~\bibnamefont
            {Heinonen}}, \bibinfo {author} {\bibfnamefont {K.}~\bibnamefont {Brandner}},
        \ and\ \bibinfo {author} {\bibfnamefont {C.}~\bibnamefont {Flindt}},\ }\href
    {\doibase 10.48550/arXiv.2301.10997} {\enquote {\bibinfo {title}
    {Lee-{{Yang}} theory of {{Bose-Einstein}} condensation},}\ } (\bibinfo {year}
    {2023}),\ \Eprint {http://arxiv.org/abs/arXiv:2301.10997} {arXiv:2301.10997}
    \BibitemShut {NoStop}%
    \bibitem [{\citenamefont {Yamamoto}\ \emph {et~al.}(2019)\citenamefont
                {Yamamoto}, \citenamefont {Nakagawa}, \citenamefont {Adachi}, \citenamefont
                {Takasan}, \citenamefont {Ueda},\ and\ \citenamefont
                {Kawakami}}]{yamamotoTheoryNonHermitianFermionic2019}%
    \BibitemOpen
    \bibfield  {author} {\bibinfo {author} {\bibfnamefont {K.}~\bibnamefont
            {Yamamoto}}, \bibinfo {author} {\bibfnamefont {M.}~\bibnamefont {Nakagawa}},
        \bibinfo {author} {\bibfnamefont {K.}~\bibnamefont {Adachi}}, \bibinfo
        {author} {\bibfnamefont {K.}~\bibnamefont {Takasan}}, \bibinfo {author}
        {\bibfnamefont {M.}~\bibnamefont {Ueda}}, \ and\ \bibinfo {author}
        {\bibfnamefont {N.}~\bibnamefont {Kawakami}},\ }\href {\doibase
        10.1103/PhysRevLett.123.123601} {\bibfield  {journal} {\bibinfo  {journal}
            {Phys. Rev. Lett.}\ }\textbf {\bibinfo {volume} {123}},\ \bibinfo {pages}
        {123601} (\bibinfo {year} {2019})}\BibitemShut {NoStop}%
    \bibitem [{\citenamefont {Ashida}\ \emph {et~al.}(2020)\citenamefont {Ashida},
                \citenamefont {Gong},\ and\ \citenamefont
                {Ueda}}]{ashidaNonHermitianPhysics2020}%
    \BibitemOpen
    \bibfield  {author} {\bibinfo {author} {\bibfnamefont {Y.}~\bibnamefont
            {Ashida}}, \bibinfo {author} {\bibfnamefont {Z.}~\bibnamefont {Gong}}, \ and\
        \bibinfo {author} {\bibfnamefont {M.}~\bibnamefont {Ueda}},\ }\href {\doibase
        10.1080/00018732.2021.1876991} {\bibfield  {journal} {\bibinfo  {journal}
            {Adv. Phys.}\ }\textbf {\bibinfo {volume} {69}},\ \bibinfo {pages} {249}
        (\bibinfo {year} {2020})}\BibitemShut {NoStop}%
    \bibitem [{\citenamefont {Li}\ \emph {et~al.}(2022)\citenamefont {Li},
                \citenamefont {Yu}, \citenamefont {Nakagawa},\ and\ \citenamefont
                {Ueda}}]{liYangLeeSingularityBCS2022}%
    \BibitemOpen
    \bibfield  {author} {\bibinfo {author} {\bibfnamefont {H.}~\bibnamefont
            {Li}}, \bibinfo {author} {\bibfnamefont {X.-H.}\ \bibnamefont {Yu}}, \bibinfo
        {author} {\bibfnamefont {M.}~\bibnamefont {Nakagawa}}, \ and\ \bibinfo
        {author} {\bibfnamefont {M.}~\bibnamefont {Ueda}},\ }\href@noop {} {\enquote
    {\bibinfo {title} {Yang-{{Lee Singularity}} in {{BCS Superconductivity}}},}\
    } (\bibinfo {year} {2022}),\ \Eprint {http://arxiv.org/abs/2211.10975}
    {arXiv:2211.10975} \BibitemShut {NoStop}%
    \bibitem [{\citenamefont {Matsumoto}\ \emph {et~al.}(2022)\citenamefont
                {Matsumoto}, \citenamefont {Nakagawa},\ and\ \citenamefont
                {Ueda}}]{matsumotoEmbeddingYangLeeQuantum2022}%
    \BibitemOpen
    \bibfield  {author} {\bibinfo {author} {\bibfnamefont {N.}~\bibnamefont
            {Matsumoto}}, \bibinfo {author} {\bibfnamefont {M.}~\bibnamefont {Nakagawa}},
        \ and\ \bibinfo {author} {\bibfnamefont {M.}~\bibnamefont {Ueda}},\ }\href
    {\doibase 10.1103/PhysRevResearch.4.033250} {\bibfield  {journal} {\bibinfo
            {journal} {Phys. Rev. Research}\ }\textbf {\bibinfo {volume} {4}},\ \bibinfo
        {pages} {033250} (\bibinfo {year} {2022})}\BibitemShut {NoStop}%
    \bibitem [{\citenamefont {Nova}\ \emph {et~al.}(2019)\citenamefont {Nova},
                \citenamefont {Disa}, \citenamefont {Fechner},\ and\ \citenamefont
                {Cavalleri}}]{novaMetastableFerroelectricityOptically2019}%
    \BibitemOpen
    \bibfield  {author} {\bibinfo {author} {\bibfnamefont {T.~F.}\ \bibnamefont
            {Nova}}, \bibinfo {author} {\bibfnamefont {A.~S.}\ \bibnamefont {Disa}},
        \bibinfo {author} {\bibfnamefont {M.}~\bibnamefont {Fechner}}, \ and\
        \bibinfo {author} {\bibfnamefont {A.}~\bibnamefont {Cavalleri}},\ }\href
    {\doibase 10.1126/science.aaw4911} {\bibfield  {journal} {\bibinfo  {journal}
            {Science}\ }\textbf {\bibinfo {volume} {364}},\ \bibinfo {pages} {1075}
        (\bibinfo {year} {2019})}\BibitemShut {NoStop}%
    \bibitem [{\citenamefont {Li}\ \emph {et~al.}(2019)\citenamefont {Li},
                \citenamefont {Qiu}, \citenamefont {Zhang}, \citenamefont {Baldini},
                \citenamefont {Lu}, \citenamefont {Rappe},\ and\ \citenamefont
                {Nelson}}]{liTerahertzFieldInduced2019}%
    \BibitemOpen
    \bibfield  {author} {\bibinfo {author} {\bibfnamefont {X.}~\bibnamefont
            {Li}}, \bibinfo {author} {\bibfnamefont {T.}~\bibnamefont {Qiu}}, \bibinfo
        {author} {\bibfnamefont {J.}~\bibnamefont {Zhang}}, \bibinfo {author}
        {\bibfnamefont {E.}~\bibnamefont {Baldini}}, \bibinfo {author} {\bibfnamefont
            {J.}~\bibnamefont {Lu}}, \bibinfo {author} {\bibfnamefont {A.~M.}\
            \bibnamefont {Rappe}}, \ and\ \bibinfo {author} {\bibfnamefont {K.~A.}\
            \bibnamefont {Nelson}},\ }\href {\doibase 10.1126/science.aaw4913} {\bibfield
        {journal} {\bibinfo  {journal} {Science}\ }\textbf {\bibinfo {volume}
            {364}},\ \bibinfo {pages} {1079} (\bibinfo {year} {2019})}\BibitemShut
    {NoStop}%
\end{thebibliography}%


\end{document}
