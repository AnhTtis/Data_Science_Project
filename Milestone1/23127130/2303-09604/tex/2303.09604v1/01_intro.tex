\section{Introduction}
\label{sec:intro}

%\noindent
%``Art is a form of exploration, of sailing off into the unknown alone, heading for those unmarked places on the map.''
%\vspace{-5pt}
%\begin{flushright}
%--- Michael Chabon
%\end{flushright}

\begin{figure}[htbp]
\centering
\includegraphics[width=\textwidth]{Figures/Figs/categories.png}        
\caption{Overview of this survey. This survey categorizes approaches based on four labeling scenarios: No label (Section \ref{sec:ssl}), insufficient label (Section \ref{sec:semi}), inexact label (Section \ref{sec:mil}), and label refinement (Section \ref{sec:al}). This figure illustrates the disparity between data growth and annotator scarcity, the core techniques employed in each scenario, and trends in label-efficient learning applications. Detailed survey scope can be referred to Appendix \ref{appendix1}.}
	\label{fig_class}
\end{figure}

\rz{We explore the artistic-creative potential of automated generative processes modeled by modern neural networks. Specifically, we
are interested in the generation of {\em artistic typography\/}. According to Wikipedia, typography is the art and technique of arranging 
type to make written language legible, readable, and appealing when displayed. Artistic typography represents a style of 
typography that goes beyond the basic function of conveying information through text and seeks to create a visual impact on the reader. 
It involves using typography as a form of artistic expression and allows designers to create eye-catching typographic designs that 
express a message visually and creatively.

In this paper, we aim to automatically generate an artistic typography by {\em stylizing\/} one or more letter fonts, to visually convey 
the semantics of an input word, while ensuring that the output typography is readable; see Figure~\ref{fig:teaser} for such 
examples referred to as ``word-as-image''~\cite{berio2022strokestyles,tendulkar2019trick,zhang2017synthesizing,iluz2023word2img}.
%
It is an arduous task to combine semantics and text in a legible and artistic manner for several reasons. First, the goal of 
incorporating the aesthetics of a style in an abstract and creative way into a letter or word can conflict with the desire to 
maintain readability of the original word/letter. Second, what a good artistic typography is can be a subjective matter. Without
a universally accepted ``ground truth", %and a lack of large volumes of artistic creations, 
a viable learning approach will have to be {\em unsupervised\/}. Last but not least, semantics can be depicted in numerous ways. 
For instance, to indicate the presence of a lion, one can use the entire face, the tail, or the whole animal. There is a vast 
range of lion images, icons, and shapes that are accessible, making it nearly impossible to manually search, deform, and substitute them. 
While experienced artists and designers are capable of producing beautiful semantic typography, obtaining reasonable results for 
ordinary users and hobbyists without proper assistive tools are out of reach.

To address all the above challenges, we resort to recently popularized large language 
models~\cite{rombach2022high,radford2021learning} to bridge texts and visual images for stylization and build our unsupervised
generative model for artistic typography on Latent Diffusion~\cite{radford2021learning}. Specifically, we employ the {\em denoising 
generator\/} in Latent Diffusion Model,
%and their encoder, 
with the key addition of a CNN-based {\em discriminator\/} to adapt the input style 
onto the input text (Figure~\ref{fig:method_pipeline}). The discriminator uses rasterized images of a given letter/word font as real samples and output of the denoising 
generator as fake samples. To obtain images for the denoising generator to guide the letter stylization, we generate 25 style images 
from the input word, again, with Latent Diffusion. The selection of this number aims to ensure an adequate number of instances for the diffusion model to extract underlying features and attain diversity in the outputs. We fine-tune the denoising generator on 
these images using the diffusion loss based on the style images and the discriminator loss.}

%Typography is the art of arranging letters and text in a way that makes the copy legible, clear, and visually appealing to the reader

%Semantic typography is the practice of using typography to visually reinforce the meaning of text.

%iluz2023word2img

%Typography is an essential element of visual communication and design.

%Great efforts have been devoted to artistic typography for making written texts and written language legible, readable, and appealing.
%It also allows designers to create eye-catching typographic designs that express a message creatively.
%Among such efforts, semantic typography tries to convey meaning and message through typography in a subtle way by deforming a letter into a semantic icon, fusing a semantic accent to a text, arranging letters in a form that resembles a semantic, etc.

%It is an arduous task to combine semantics and text in a legible and artistic manner, and this is a difficult problem for several reasons. Firstly, there are various approaches that one can use to add a semantic aspect to a word. Secondly, a semantic can be depicted in numerous ways. For instance, to indicate the presence of a lion as a semantic, one can use the entire face, the tail, or the whole animal. Additionally, there is a vast range of lion images and icons that are accessible, making it nearly impossible to manually search, deform, and substitute them. While crafted artists and designers are capable of producing beautiful semantic
%typography, obtaining reasonable results for ordinary users and hobbyists without proper assistive tools is out of reach.

%In this paper, we attempt to incorporate the aesthetics of a style in an abstract and creative way onto a letter or word, such that the readability of the original word/letter is maintained. For this purpose diffusion models are an attractive choice due to their []. D-Fusion uses the stable diffusion model's denoising generator and their encoder, and adds a CNN based discriminator to adapt the style onto a specific text. The discriminator use s rasterized images of given text as real samples and output of denoising generator as fake samples. To generate images from the denoising generator we generate a set of ~\mt{twenty-five} style images based on our chosen style using stable diffusion. ~\mt{This number is chosen so as to provide enough examples for the diffusion model to extract underlying features and have diversity in outputs.} We then fine-tune the denoising generator on these images by using both diffusion loss based on the style images and discriminator loss. 

%\am{Our method can automatically produce artistic typography for a given letter or word. This is a highly challenging task since merely distorting a letter to suit a target semantic is insufficient. It requires prior knowledge of the semantic's visual domain. To overcome this obstacle, our approach incorporates language-based generative models.}

Our model is coined {\em DS-Fusion\/} for discriminated and stylized diffusion. While the core idea is quite simple, it is among the first to 
integrate adversarial learning and diffusion in a single framework. By utilizing the powerful generation capabilities of diffusion models and 
employing a discriminator as a critic, we ensure that the produced artistic typography remains true to the input font. Figure~\ref{fig:teaser}
shows some results generated by DS-Fusion {\em fully automatically\/}.

We showcase the effectiveness of DS-Fusion for generating artistic typography through numerous experiments. Our approach produces visual results that demonstrate its generality and versatility in accommodating different semantics, letters, and artistic styles. We report quantitative evaluations and ablation studies to assess the contribution of individual components. Additionally, we have conducted user studies to assess the quality of our automatically generated typography results. These studies reveal that in about 42\% of the cases, our results were favored over or considered equally good as results produced by professional artists, and in close to 50\% of cases, our method outperforms the state-of-the-art alternatives, including DALL-E 2~\cite{ramesh2022hierarchical}, a strong baseline that had been trained on significantly more images than LDM. It is worth noting that we did not choose specific inputs catering to DS-Fusion for these comparisons. Instead, we performed a Google image search on ``artistic typography" and extracted a suitable subset of artist-generated results to come up with inputs for both user studies.
