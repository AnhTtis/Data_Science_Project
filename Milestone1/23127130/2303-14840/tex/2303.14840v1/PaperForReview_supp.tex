% CVPR 2022 Paper Template
% based on the CVPR template provided by Ming-Ming Cheng (https://github.com/MCG-NKU/CVPR_Template)
% modified and extended by Stefan Roth (stefan.roth@NOSPAMtu-darmstadt.de)

\documentclass[10pt,twocolumn,letterpaper]{article}

%%%%%%%%% PAPER TYPE  - PLEASE UPDATE FOR FINAL VERSION
%\usepackage[review]{cvpr}      % To produce the REVIEW version
\usepackage{cvpr}              % To produce the CAMERA-READY version
%\usepackage[pagenumbers]{cvpr} % To force page numbers, e.g. for an arXiv version
\makeatletter
\@namedef{ver@everyshi.sty}{}
\makeatother
\usepackage{tikz}

% Include other packages here, before hyperref.
\usepackage{graphicx}
\usepackage{amsmath}
\usepackage{amssymb}
\usepackage{booktabs}
\usepackage[accsupp]{axessibility}



% It is strongly recommended to use hyperref, especially for the review version.
% hyperref with option pagebackref eases the reviewers' job.
% Please disable hyperref *only* if you encounter grave issues, e.g. with the
% file validation for the camera-ready version.
%
% If you comment hyperref and then uncomment it, you should delete
% ReviewTempalte.aux before re-running LaTeX.
% (Or just hit 'q' on the first LaTeX run, let it finish, and you
%  should be clear).
\usepackage[pagebackref,breaklinks,colorlinks]{hyperref}


% Support for easy cross-referencing
\usepackage[capitalize]{cleveref}
\crefname{section}{Sec.}{Secs.}
\Crefname{section}{Section}{Sections}
\Crefname{table}{Table}{Tables}
\crefname{table}{Tab.}{Tabs.}


%\usepackage{multicol}
\usepackage{multirow}
\usepackage{booktabs}
\usepackage{makecell}
\usepackage{adjustbox}

\usepackage{color}

\usepackage[normalem]{ulem}

\usepackage{todonotes}

\definecolor{ben}{rgb}{0.9,0.,0.5}
\newcommand{\ben}[1]{\textcolor{ben}{\emph{Ben:~{#1}}}}

\definecolor{pat}{rgb}{0.6,0.2,0.1}
\newcommand{\pat}[1]{\textcolor{pat}{\emph{Pat:~{#1}}}}

\definecolor{GY}{rgb}{0.4,0.1,0.9}
\newcommand{\gy}[1]{\textcolor{GY}{\emph{GY:~{#1}}}}

\newcommand{\mycomment}[1]{}

% custom commands
\newlength\savewidth\newcommand\shline{\noalign{\global\savewidth\arrayrulewidth
  \global\arrayrulewidth 1pt}\hline\noalign{\global\arrayrulewidth\savewidth}}


%%%%%%%%% PAPER ID  - PLEASE UPDATE
\def\cvprPaperID{9363} % *** Enter the CVPR Paper ID here
\def\confName{CVPR}
\def\confYear{2023}


\begin{document}

%%%%%%%%% TITLE - PLEASE UPDATE
\title{On the Importance of Accurate Geometry Data for Dense 3D Vision Tasks \\ -- \\ Supplementary Material}

% \author{First Author\\
% Institution1\\
% Institution1 address\\
% {\tt\small firstauthor@i1.org}
% % For a paper whose authors are all at the same institution,
% % omit the following lines up until the closing ``}''.
% % Additional authors and addresses can be added with ``\and'',
% % just like the second author.
% % To save space, use either the email address or home page, not both
% \and
% Second Author\\
% Institution2\\
% First line of institution2 address\\
% {\tt\small secondauthor@i2.org}
% }
\author{
\hspace{-20pt}
HyunJun Jung$^{\ast 1}$,
Patrick Ruhkamp$^{\ast 1,2}$,
Guangyao Zhai$^{1}$,
Nikolas Brasch$^{1}$,
Yitong Li$^{1}$,\\
Yannick Verdie$^{1,3}$,
Jifei Song$^{3}$,
Yiren Zhou$^{3}$,
Anil Armagan$^{3}$,
Slobodan Ilic$^{1,4}$,\\
Ales Leonardis$^{3}$,
Nassir Navab$^{1}$,
Benjamin Busam$^{1,2}$
% HyunJun Jung$^{\ast 1}$\and
% Patrick Ruhkamp$^{\ast 1,2}$\and
% Guangyao Zhai$^{1}$\and
% Nikolas Brasch$^{1}$\and
% Yitong Li$^{1}$\and
% Yannick Verdie$^{1,3}$\and
% Jifei Song$^{3}$\and
% Yiren Zhou$^{3}$\and
% Anil Armagan$^{3}$\and
% Slobodan Ilic$^{1,4}$\and
% Ales Leonardis$^{3}$\and
% Nassir Navab$^{1}$\and
% Benjamin Busam$^{1,2}$
\\
\\
\small
$^1$ Technical University of Munich,
$^2$ 3Dwe.ai,
$^3$  Huawei Noah's Ark Lab,
$^4$  Siemens AG,
$^*$ Equal Contribution
\\ \footnotesize{\fontfamily{qcr}\selectfont
hyunjun.jung@tum.de, p.ruhkamp@tum.de, guangyao.zhai@tum.de, b.busam@tum.de
}
}
\maketitle

% \twocolumn[{%
% \renewcommand\twocolumn[1][]{#1}%
% \maketitle
% \begin{center}
%     \centering
%     \captionsetup{type=figure}
%     \includegraphics[width=\textwidth]{figures/hammer_teaser_quali56.pdf}
%     \captionof{figure}{\textbf{Sensor Influence on Dense 3D Vision Tasks.} Monocular depth estimation, 3D reconstruction, and novel view synthesis are all influenced by inherent sensor artefacts. Time-of-Flight (ToF) sensors suffer from Multi-Path-Inference (MPI) and fail to measure correctly reflective or translucent objects. Active Stereo (AS) recovers such material but struggles on diffuse texture-less parts and at depth discontinuities. Our novel multi-modal dataset allows for the first time to analyse systematically such sensor characteristics quantitatively and qualitatively and fosters the way for novel learning-based dense 3D computer vision methods in photometrically challenging environments.}
%     \label{fig:teaser}
% \end{center}%
% }]



%%%%%%%%% ABSTRACT
% \begin{abstract}
% ...
% \end{abstract}

% Learning-based methods to solve dense 3D vision problems typically train on 3D sensor data. The respectively used principle of measuring distances provides advantages and drawbacks. These are typically not compared nor discussed in the literature due to a lack of multi-modal datasets. Texture-less regions are problematic for structure from motion and stereo, reflective material poses issues for active sensing, and distances for translucent objects are intricate to measure with existing hardware. Training on inaccurate or corrupt data induces model bias and hampers generalisation capabilities. These effects remain unnoticed if the sensor measurement is considered as ground truth during the evaluation. This paper investigates the effect of sensor errors for the dense 3D vision tasks of depth estimation and reconstruction. We rigorously show the significant impact of sensor characteristics on the learned predictions and notice generalisation issues arising from various technologies in everyday household environments. For evaluation, we introduce a carefully designed dataset\footnote{Our dataset will be made publicly available upon acceptance.} comprising measurements from commodity sensors, namely D-ToF, I-ToF, passive/active stereo, and monocular RGB+P. Our study quantifies the considerable sensor noise impact and paves the way to improved dense vision estimates and targeted data fusion.



% %Geometry estimation is a core element in 3D computer vision pipelines.
% %It is an inherent task to predict reliable depth and to reconstruct scenes.
% %Learning based methods to solve dense 3D vision problems are typically provided with depth data from various sources during training.

% %, and sensor capabilities are typically not discussed in the literature.
% %Everyday objects in indoor environments, however, pose severe challenges for some devices.

% %with challenging but everyday scene content.


% % ECCV
% %Depth estimation is a core task in 3D computer vision. Recent methods investigate the task of monocular depth trained with various depth sensor modalities. Every sensor has its advantages and drawbacks caused by the nature of estimates. In the literature, mostly mean average error of the depth is investigated and sensor capabilities are typically not discussed. Especially indoor environments, however, pose challenges for some devices. Textureless regions pose challenges for structure from motion, reflective materials are problematic for active sensing, and distances for translucent material are intricate to measure with existing sensors. This paper proposes HAMMER, a dataset comprising depth estimates from multiple commonly used sensors for indoor depth estimation, namely ToF, stereo, active stereo together with monocular RGB+P data. We construct highly reliable ground truth depth maps with the help of 3D scanners and aligned renderings. A popular depth estimators is trained on this data and typical depth sensors. The estimates are extensively analyze on different scene structures. We notice generalization issues arising from various sensor technologies in household environments with challenging but everyday scene content. HAMMER, which we make publicly available (https://github.com/Junggy/HAMMER-dataset), provides a reliable base to pave the way to targeted depth improvements and sensor fusion approaches.
% %\keywords{Depth estimation, indoor scenes, monocular depth, ToF, stereo, active stereo, sensor fusion}
% \end{abstract}

%%%%%%%%% BODY TEXT

% % \begin{figure}[t]
%     % \begin{subfigure}{1\linewidth}
%     %   \centering
%     % %   \includegraphics[width=1\linewidth]{figs/fig_1_moti_textattn.pdf}  
%     % %   \includegraphics[width=1\linewidth]{figs/fig_1_moti_textattn_v2.pdf}  
%     %   \includegraphics[width=1\linewidth]{figs/fig_1_moti_textattn_v5.pdf}  
%     %   \vspace{-0.5cm}
%     %     \caption{Amount of attention added to each video clip from the source video and query text in the self-attention layers of Moment-DETR encoder.}
%     %     % \caption{Distribution of attention for source and query in Moment-DETR encoder}
%     %     % Visualization of video clip's self-attention score in Moment-DETR encoder.
%     %   \label{fig:fig1_text_attn_ex}
%     % \end{subfigure}%\hfill% or  or \hspace{0.3\textwidth}
%     \vspace{0.2cm}
%     % \begin{subfigure}{1\linewidth}
%       \centering
%     %   \includegraphics[width=1\linewidth]{figs/fig1_moti_negattn.pdf}  
%       \includegraphics[width=1\linewidth]{figs/fig1_moti_negattn_v3.pdf}  
%       \vspace{-0.4cm}
%     %   \caption{Correspondence of saliency scores on the relevance between video clips and the text query.}
%     % \caption{Predicted saliency scores against the video relevant positive query and video irrelevant negative query}
%       \label{fig:fig1_neg_attn_ex}
%     % \end{subfigure}%\hfill% or  or \hspace{0.3\textwidth}
%     \caption{
%     % 원준 원본
%     % (a) Comparison between attention scores of source and query for each video clip~(We sum the attention scores from video and text). 
%     % We observe that the attention scores are dominated by other clips in the source video. 
%     % Text queries do not account for much attention regardless of the relevance to the video clips.
%     % \textbf{(a)} Inspection of the query dependency in Moment-DETR encoder.
%     % % We visualize the attention score of video tokens in the transformer encoder and observe that text query accounts for only a low portion of attention.
%     % % This tendency occurs regardless of the relevance between the text query and video clips. 
%     % We visualize the attention score of video tokens in the transformer encoder and observe 1) text query only accounts for a low portion of attention, and 2) relevance between video-query pair does not affect the attention scores ratio of text.
%     \textbf{(b)} Comparison of highlight-ness when relevant and non-relevant queries are input.
%     As observed in , existing work only uses queries to play an insignificant role, thereby may not be capable of detecting false queries and considering the video-query relevance even when the problem in (a) is resolved. 
%     % \SE{} % 이 부분이 "not capable of" 란 용어가 세다는 피드백이 있는 듯 합니다. 이러한 능력이 없다는 것은 굉장히 강한 어조인거 같기는 하고, 이러한 경우들이 종종 있다거나 좀 약화시킬 필요가 있어보이긴 하네요.
%     On the other hand, our QD-DETR yields a query-dependent representation that the relevance between the source video and query text is updated in the saliency scores.
%     There is a large gap between positive and negative saliency scores, and scores are consistent since the clips are all highly correlated to others.
%     }
%     \label{fig:motivation_ex}
%     % \captionsetup{belowskip=13pt}
%     % \setlength{\belowcaptionskip}{-10pt}
% \end{figure}
\begin{figure}
    \centering
    \includegraphics[width=1\linewidth]{figs/fig1_moti_negattn_1111.pdf}
    % \includegraphics[width=1\linewidth]{figs/fig1_moti_negattn_1109.pdf}
    % \includegraphics[width=1\linewidth]{figs/fig1_moti_negattn_stat.pdf}
    \vspace{-0.6cm}
    \caption{
        % \SE{} % 수정 필요
        Comparison of highlight-ness~(saliency score) when relevant and non-relevant queries are given.
        We found that the existing work only uses queries to play an insignificant role, thereby may not be capable of detecting negative queries and video-query relevance; saliency scores for clips in ground-truth~(GT) moments are low and equivalent for positive and negative queries.
        % This also results in mispredicted moments when ground-truth~(GT) moment is dominated by clips unrelated to GT since their prediction is highly focused on the video.
        % \SE{} % 여기 한번 더 보면 좋을 듯 합니다. GT moment에 unrelated한 clip이 많으면? label이 틀렷을 경우를 말씀하시는건지?
        % As observed in saliency graph, existing work only uses queries to play an insignificant role, thereby may not be capable of detecting false queries and considering the video-query relevance.
        On the other hand, query-dependent representations of QD-DETR result in corresponding saliency scores to the video-query relevance and precisely localized moments.
        % On the other hand, our QD-DETR yields a query-dependent representation that the
        % saliency scores are in accordance with the relevance between the video and query.
        % text is in accordance with the saliency scores.
        % There is a large gap between positive and negative saliency scores, and scores are consistent since the clips are all highly correlated to others.
}
    \label{fig:motivation_ex}
\end{figure}


\section{Introduction}
% 원준 원본
% Along with the advance of digital devices and platforms, video is now one of the most desired data type for consumers. However, although the large information capacity of videos may be beneficial in many aspects, e.g., informative and entertaining, on the contrary perspective, videos are time-consuming, and hard to search for desirable moments. 
% This has led many creators to use extra manpower to crop and edit the video to generate highlight clips to gain the consumer’s attention.
Along with the advance of digital devices and platforms, video is now one of the most desired data types for consumers~\cite{apostolidis2021video,wu2017deep}.
% SE: Video aware deep learning application & survey papers?
Although the large information capacity of videos might be beneficial in many aspects, e.g., informative and entertaining, inspecting the videos is time-consuming, so that it is hard to capture the desired moments~\cite{anne2017localizing,apostolidis2021video}. 
% This has led many creators to use extra manpower to crop and edit the video to generate highlight clips to gain the consumer’s attention.


% On the other side, 
Indeed, the need to retrieve user-requested or highlight moments within videos is greatly raised.
Numerous research efforts were put into the search for the requested moments in the video~\cite{anne2017localizing, gao2017tall, liu2015multi, escorcia2019temporal} and summarizing the video highlights~\cite{zhang2016video, mahasseni2017unsupervised, badamdorj2022contrastive, wei2022learning}.
% Numerous research efforts were put into the search for the requested moments in the video~\cite{anne2017localizing, gao2017tall, liu2015multi, escorcia2019temporal}, summarizing the video to generate highlights was another popular topic~\cite{zhang2016video, mahasseni2017unsupervised, badamdorj2022contrastive, wei2022learning}.
Recently, Moment-DETR~\cite{momentdetr} further spotlighted the topic by proposing a QVHighlights dataset that enables the model to perform both tasks, retrieving the moments with their highlight-ness, simultaneously.

% 원준 원본
% To detect the desired moments, previous works employed transformer encoder-decoder architectural designs to fuse the text query into the video representations. Moment-DETR~\cite{mDETR} modified detection transformer to process capture the moment as a set, and UMT~\cite{umt} implemented transformer decoder as to output clip-wise saliency. 
% Yet to their outstanding breakthroughs in the literature of moment retrieval with the seminal architectures, their limitation is that the role of the given text query is insignificant in representing the query-conditioned video representation; the attention mechanism of moment DETR is not explicitly conditioned on the text query, and the text query is conditioned on multi-modal clips where the differences between the clips are smoothed after encoding process in UMT.



% \begin{figure}[t]
% \centering
%     \begin{subfigure}[l]{0.37\linewidth}
%       \centering
%       \vspace{0.20cm}
%     %   \includegraphics[width=1\linewidth]{figs/fig_1_moti_textattn.pdf}  
%     %   \includegraphics[width=1\linewidth]{figs/fig_1_moti_textattn_v2.pdf}  
%       \includegraphics[width=1\linewidth]{figs/fig1_moti_violin_a.pdf}  
%       \vspace{-0.60cm}
%     %   \caption{text attention}
%         \caption{Importance of queries in video representation}
%       \label{fig:fig1_text_attn}
%     \end{subfigure}%\hfill% or  or \hspace{0.3\textwidth}
%     \vspace{0.2cm}
%     \begin{subfigure}[r]{0.61\linewidth}
%       \centering
%     %   \includegraphics[width=1\linewidth]{figs/fig1_moti_negattn.pdf}  
%       \includegraphics[width=1\linewidth]{figs/fig1_moti_violin_b.pdf}  
%     %   \caption{neg attention}
%         % \caption{Relation between the highlight-ness and the relevance between videos and query texts.}
%         \caption{Highlight-ness~(saliency) histogram of positive and negative video-query pairs\SE{}}
%       \label{fig:fig1_neg_attn}
%     \end{subfigure}%\hfill% or  or \hspace{0.3\textwidth}
%     % \vspace{-0.2cm}
%     \caption{Overall statistics for attention scores in Fig.~\ref{fig:motivation_ex} in QVHighlights dataset. 
%     (a) For the attention scores that measure how much the text query is generally involved in video representation, we use violin plots to show the probability density. We plot the score for each layer in the encoder.
%     % (b) Using the histogram, we compare how the baseline and QD-DETR yield different salient scores given the positive and negative video-text pairs.
%     (b) Saliency histogram shows the distributional gap between positive and negative video-text query pairs of baseline~(Moment-DETR) and proposed QD-DETR.\SE{}
%     }
%     \label{fig:motivation}
%     % \captionsetup{belowskip=13pt}
%     % \setlength{\belowcaptionskip}{-10pt}
% \end{figure}

% \begin{figure}[t]
% \centering

%     \begin{subfigure}[r]{1\linewidth}
%       \centering
%       \hspace{-0.2cm}
%     %   \includegraphics[width=1\linewidth]{figs/fig1_moti_negattn.pdf}  
%       \includegraphics[width=1.1\linewidth]{figs/fig1_moti_violin_a_v2.pdf}  
%     %   \caption{neg attention}
%         % \caption{Relation between the highlight-ness and the relevance between videos and query texts.}
%         \vspace{-0.5cm}
%         % \caption{Saliency histogram of positive and negative video-query pairs}
%         \caption{We plot the histograms and its average value~(dotted line) to compare saliency scores when true and false text queries are given for each method. (left) Since the video representations do not include much textual information, both the true and false queries yield similar saliency scores. (Middle) Even when the video representation is enforced to be updated with the textual information, the issue is not much resolved. (Right) By extracting discriminative features in the text query, distributions are differentiated.
%         % \SE{} % R1@0.5 설명
%         Also, R1@0.5 indicates evaluation metric, Recall at 1 with IoU 0.5 threshold on QVhighlight \textit{val} set.
%         }
%       \label{fig:fig1_neg_attn}
%     \end{subfigure}%\hfill% or  or \hspace{0.3\textwidth}
%     \\
%     \begin{tabular}{cc}
%     \hspace{-0.2cm}
%         \begin{minipage}{.4\linewidth}
%             \begin{subfigure}[l]{1\linewidth}
%               \centering
%             %   \vspace{0.20cm}
%             %   \includegraphics[width=1\linewidth]{figs/fig_1_moti_textattn.pdf}  
%             %   \includegraphics[width=1\linewidth]{figs/fig_1_moti_textattn_v2.pdf}  
%               \includegraphics[width=1\linewidth]{figs/fig1_moti_violin_a.pdf}  
%               \vspace{-0.60cm}
%             %   \caption{text attention}
%                 \caption{Importance of queries in video representation}
%               \label{fig:fig1_text_attn}
%             \end{subfigure}%\hfill% or  or \hspace{0.3\textwidth}
%         \end{minipage}
        
%         \begin{minipage}{.6\linewidth}
%             \vspace{-0.2cm}
%             \caption{Overall statistics of Fig.~\ref{fig:motivation_ex} in QVHighlights dataset. 
%             (a) Saliency histogram shows the distributional gap between positive and negative video-text query pairs.
%             % (a) For the attention scores that measure how much the text query is generally involved in video representation, we use violin plots to show the probability density. We plot the score for each layer in the encoder.
%             % (b) Using the histogram, we compare how the baseline and QD-DETR yield different salient scores given the positive and negative video-text pairs.
%             % (b) Text ratio in self-attention layer to  of Moment-DETR
%             % (b) Ratio of text when representing video tokens in self-attention of Moment-DETR.
%             % (b) Magnitude of attention text query involved.
%             % (b) Attention score of video tokens
%             % (b) Magnitude of text query to refine the video tokens in self-attention layer of Moment-DETR.
%             (b) Probability density depicting the weight of the text query in attention score for video clips. Scores are from the self-attention layers in Moment-DETR encoder.
%             % (b) The text query ratio in attention score of video clips (Self-attention layer in Moment-DETR encoder). We use violin plots to show probability density.
%             % 텍스트 쿼리가, 비디오 피쳐에 얼만큼 attend 하는지
%             }
%         \end{minipage}
    
%     \end{tabular}
%     \vspace{-0.5cm}
%     \label{fig:moti}
%     % \captionsetup{belowskip=13pt}
%     % \setlength{\belowcaptionskip}{-10pt}
% \end{figure}


% \begin{figure}
%     \centering
%     % \includegraphics[width=1\linewidth]{figs/fig1_moti_negattn_1109.pdf}
%     \includegraphics[width=1\linewidth]{figs/fig1_moti_negattn_stat_v2.pdf}
%     \vspace{-0.8cm}
%     \caption{
%         Histogram of saliency when the positive and negative queries are given. We plot the histograms and its average value~(dotted line) to compare saliency scores when relevant~(positive) and irrelevant~(negative) text queries are given for each method. (Left) Since the video representations do not properly reflect textual information, both the positive and negative queries yield similar saliency scores. 
%         % (Middle) Even when the video representation is enforced to be updated with the textual information, the issue is not much resolved. 
%         (Right) By representing video clips in query-dependent manner, distributions are differentiated.
%     }
%     \vspace{-0.6cm}
%     \label{fig:motivation}
% \end{figure}


% One of the demanding task is moment retrieval task, which is detecting the desired moments from the given query, typically the text query.
When describing the moment, one of the most favored types of query is the natural language sentence~(text)\cite{anne2017localizing}. 
While early methods utilized convolution networks~\cite{zhang2020learning, gao2021fast, wang2020temporally}, recent approaches have shown that deploying the attention mechanism of transformer architecture is more effective to fuse the text query into the video representation.
% To handle these modalities, previous works simply employed the attention mechanism of transformer architecture to fuse the text query into the video representation.
For example, Moment-DETR~\cite{momentdetr} introduced the transformer architecture which processes both text and video tokens as input by modifying the detection transformer~(DETR), and UMT~\cite{umt} proposed transformer architectures to take multi-modal sources, e.g., video and audio. 
Also, they utilized the text queries in the transformer decoder.
Although they brought breakthroughs in the field of MR/HD with seminal architectures, they overlooked the role of the text query.
To validate our claim, we investigate the Moment-DETR~\cite{momentdetr} in terms of the impact of text query in MR/HD~(Fig.\ref{fig:motivation_ex}).
Given the video clips with a relevant positive query and an irrelevant negative query, we observe that the baseline often neglects the given text query when estimating the query-relevance scores, i.e., saliency scores, for each video clip.
% the output saliency score, i.e. query-relevance scores.
% Based on the observation, we traced the actual saliency prediction of the model against both the video-relevant query and the irrelevant dummy one where we find that the baseline often neglects the given text query when estimating the query-relevance scores of video clips.
% For example, in Fig.~\ref{fig:motivation_ex}, saliency scores are not affected even when the query is substituted with the dummy.
% % General statistics for Fig.~\ref{fig:motivation_ex} is shown in Fig.~\ref{fig:motivation}. 
% General statistics corresponding to Fig.~\ref{fig:motivation_ex} are also shown in Fig.~\ref{fig:motivation}.



% The limitation of the concrete baseline~\cite{momentdetr} is inspected in two different aspects; 1) Utilization of text-query in the encoding process and 2) the output saliency score, i.e. query-relevance scores.
% Firstly, we visualize the attention score when video clips are given as a query in self-attention. 
% We observe that the text queries have relatively small impacts compared to other video features, as shown in Fig.~\ref{fig:fig1_text_attn_ex}.
% That is, the text does not account for much in representing every video clip, although the goal of MR/HD is to detect query-relevant moments.
% Based on the observation, we traced the actual saliency prediction of the model against both the video-relevant query and the irrelevant dummy one where we find that the baseline often neglects the given text query when estimating the query-relevance scores of video clips.
% For example, in Fig.~\ref{fig:motivation_ex}, saliency scores are not affected even when the query is substituted with the dummy.
% % General statistics for Fig.~\ref{fig:motivation_ex} is shown in Fig.~\ref{fig:motivation}. 
% General statistics are also shown in Fig.~\ref{fig:motivation}.

% Consequently, in Fig.~\ref{fig:fig1_neg_attn_ex}~(b), we found that the baseline often neglects the given text query when estimating the query-relevance scores of video clips; 
% For example, 


% We validate the previous work sometimes neglects the given query when estimating the saliency of video clips.
% For example, there is an example that the saliency scores from positive and negative queries cannot be distinguishable, as shown in Fig.~\ref{fig:fig1_neg_attn_ex}.
% % 우리는 추가로 text attention을 추가도 해봤지만, 효과가 있긴 했으나, still 이슈가 있는 것을 확인하였다?
% % Still, we observe that assuring the high attendance of text queries does not resolve the overlap which motivates us to question the quality of the naive use of task-agnostic text representation~\cite{momentdetr, umt}.
% We found that introducing the text-attention for ensuring the high attendance of text queries relieve the overlap, but there still be a severe overlap.


% To validate their limitations, we inspect the impacts of text queries in the concrete baseline~\cite{momentdetr} with the two different aspects, 1) tendency of attention in self-attention layer and 2) saliency score, i.e. query-relevance scores. \SE{} % attention 이 갑자기 등장하는가?
% Firstly, we visualize the attention score when video clips are given as a query in self-attention. We observe the text queries have relatively low attention scores compared to the video features, as shown in Fig.~\ref{fig:fig1_text_attn_ex}.
% That is, the text does not account for much in representing every video clip, although the goal of MR/HD is to detect query-relevant moments.
% Based on this observation, we trace the actual saliency prediction of the model against both positive and negative text queries.
% We validate the previous work sometimes neglects the given query when estimating the saliency of video clips.
% For example, there is an example that the saliency scores from positive and negative queries cannot be distinguishable, as shown in Fig.~\ref{fig:fig1_neg_attn_ex}.
% % 우리는 추가로 text attention을 추가도 해봤지만, 효과가 있긴 했으나, still 이슈가 있는 것을 확인하였다?
% % Still, we observe that assuring the high attendance of text queries does not resolve the overlap which motivates us to question the quality of the naive use of task-agnostic text representation~\cite{momentdetr, umt}.
% We found that introducing the text-attention for ensuring the high attendance of text queries relieve the overlap, but there still be a severe overlap.



% Thus, we 
% query dependency를 높이기 위해 
% Cross-attention? text-attention? detailed explanation on text-attention should be needed?
% By handling these two issues, we find that more precise retrieval can be achieved.
% 
% 
%
% By projecting video-discriminative text features with high text attendance to source video, we f 
% We also find the need to improve the quality of query features since assuring high text attendance also results in...
% pairs are not finetuned to be discriminative that even the similarity within the pairs does not reflect the relevance between the query and the video clips.
% General statistics for Fig.~\ref{fig:motivation_ex} is shown in Fig.~\ref{fig:motivation}. 
% \SE{} % 이거 ??로 뜨는데, 위처럼 figure 그리면 label이 안되는걸까요
% \SE{}
% 형님 아래 사항 생각 좀 해보는게 좋을 거 같아요.
% fig 1. (a) 그림만 봤을 때 모든 clip에 대해 text attention이 일정이상 존재하긴 하니까, 뭔가 not assured to be conditioned가 와닿지 않는거 같아요.
% + 왜 text가 항상 attend 해야하나?
% not assured to be conditioned --> text shows relatively low affects compared to video 같이 실제 나타난 현상까지 같이 적으면 어떨까 싶어요.
% fig 1. (b) 덜 반영한다?

% \SU{}
% 일단 text가 attend 잘 되어야 한다는 것에 좀 궁금점이 생깁니다. 결국에는 text와 관련있는 frame들을 attend해서 higlight를 찾아야 하는게 아닐까요? 그리고, 현제 저희의 모델 구조상 text query가 Key와 Value로 거의 활용되고 있는데 그렇다면 결국에는 해당 모델은 text에 대한 attention이 전혀 없다고 봐도 무방하지 않을까요? 그런 면에서 text attention을 강조하는게 좀 걸리긴 합니다.

% Specifically, the text query is not assured to be explicitly conditioned on every clip of the video, and as the query texts are evenly treated, discriminative keywords may not be spotlighted.
% attention mechanism of Moment-DETR is not explicitly conditioned on the text query as shown in Fig~\ref{}(d), and in UMT, the text are only used for conditioning the queries while the video representation are refined itself by self-attention.

% \begin{figure}[t]
%     \begin{subfigure}{1\linewidth}
%       \centering
%     %   \includegraphics[width=1\linewidth]{figs/fig_1_moti_textattn.pdf}  
%     %   \includegraphics[width=1\linewidth]{figs/fig_1_moti_textattn_v2.pdf}  
%       \includegraphics[width=1\linewidth]{figs/fig_1_moti_textattn_v4.pdf}  
%       \vspace{-0.5cm}
%     %   \caption{text attention}
%         \caption{Distribution of attention scores in Moment-DETR encoder}
%       \label{fig:fig1_text_attn}
%     \end{subfigure}%\hfill% or  or \hspace{0.3\textwidth}
%     \vspace{0.2cm}
%     \begin{subfigure}{1\linewidth}
%       \centering
%     %   \includegraphics[width=1\linewidth]{figs/fig1_moti_negattn.pdf}  
%       \includegraphics[width=1\linewidth]{figs/fig1_moti_negattn_v2.pdf}  
%       \vspace{-0.5cm}
%     %   \caption{neg attention}
%         \caption{Saliency score against positive and negative text queries}
%       \label{fig:fig1_neg_attn}
%     \end{subfigure}%\hfill% or  or \hspace{0.3\textwidth}
%     \vspace{0.2cm}
%     \begin{subfigure}{1\linewidth}
%       \centering
%     %   \includegraphics[width=1\linewidth]{figs/fig1_moti_violin.pdf}  
%       \includegraphics[width=1\linewidth]{figs/fig1_moti_violin_v2.pdf}  
%       \vspace{-0.5cm}
%       \caption{violin}
%       \label{fig:fig1_violin}
%     \end{subfigure}%\hfill% or  or \hspace{0.3\textwidth}
%     \vspace{-0.2cm}
%     \caption{(a) 1. portion of text attention vs. video attention 2. relation with text query and content (e.g. fg, bg) of clip seems not to affect the attention score
%     (b) 1. high variability even though entire clips are highly correlated with the given text query 2. positive and negative query makes overlaps on saliency score distribution
%     (3) actual distribution on validation dataset.}
%     \label{fig:motivation}
%     % \captionsetup{belowskip=13pt}
%     % \setlength{\belowcaptionskip}{-10pt}
% \end{figure}

To this end, we propose Query-Dependent DETR~(QD-DETR) that produces query-dependent video representation.
% Our key focus is to ensure each clip in predicted moments is explicitly conditioned by the query, particularly on the video-descriptive portion of the text query.
% Our key focus is to ensure that query-relevant clips are predicted by enforcing each clip to be explicitly conditioned by the query.
%Our key focus is to ensure that the model prediction for each clip is highly relevant to the query.
Our key focus is to ensure that the model's prediction for each clip is highly dependent on the query.
% by enforcing each clip to be explicitly conditioned by the query. :)
% hmm...
% \SE {} % "query-relevant clips are predicted" 이 문장이 좀 애매한거 같습니다. relevant 클립을 놓지지 않고 찾는 것을 보장한다? 이런 느낌인지 아니면 높은 saliency 를 주는게 목적이다? model prediction이 query-relevance를 반영하는 것을 보장한다?
% Our key focus is to ensure that the model prediction reflects query-relevance of clips by enforcing each clip to be explicitly conditioned by the query.
First, to fully utilize the contextual information in the query, we revise the transformer encoder to be equipped with cross-attention layers at the very first layers.
% 상익's thought :  single video - query간의 관계만 고려 - 같은 word가 더 많이 쓰이는 것을 보고 
% 교수님's thought : neg pair 를 쓰면 쿼리를 보지 않고서는 video clip간만 고려하는 것이 사라짐. 왜냐면 0으로 내보내야 하기 때문. --> SE: relative difference 만 고려하다가, 
By inserting a video as the query and a text as the key and value of the cross-attention layers, our encoder enforces the engagement of the text query in extracting video representation.
% 원준 교수님 코멘트 반영해서 다시
Then, in order to not only inject a lot of textual information into the video feature but also make it fully exploited, we leverage the negative video-query pairs generated by mixing the original pairs.
Specifically, the model is learned to suppress the saliency scores of such  negative~(irrelevant) pairs.
Our expectation is the increased contribution of the text query in prediction since the videos will be sometimes required to yield high saliency scores and sometimes low ones depending on whether the text query is relevant or not.
% \SE{}
% learns to?
% By suppressing the saliency scores of the irrelevant video-query pairs, the model learns to spotlight only the video-specific discriminative words in the query.
% % \SE{} % ====================== 상익 수정 ========================
% However, this architectural design still lacks the capability of identifying the video-descriptive keywords in the query.
% % However, this architectural design still lacks in identifying proper query relevance.
% This is because the current training scheme only focuses on the interactions of video and clips within a single video while neglecting information shared throughout the entire video.
% % We argue the problem of the current training scheme that only focuses on distinguishing the clips in a single video while neglecting information shared throughout the entire video.
% Therefore, we leverage the negative video-query relationships to enhance the capability of identifying the contextual similarity of query and video clips.
% 
% 원준 원본 
% However, this architectural design heavily relies on the quality of the text query.
% Therefore, we leverage the negative video-query relationships to enable the model to emphasize key corresponding query features.
% By suppressing the saliency scores of the irrelevant video-query pairs, the model learns to spotlight only the video-specific discriminative words in the query.
% =========================================================
Lastly, to apply the dynamic criterion to mark highlights for each instance, we deploy a saliency token to represent the entire video and utilize it as an input-adaptive saliency criterion. 
With all components combined, our QD-DETR produces query-dependent video representation by integrating source and query modalities.
This further allows the use of positional queries~\cite{dabdetr} in the transformer decoder.
% Furthermore, we can exploit the advanced DETR decoder architectures using the positional information, e.g., DAB-DETR, since our encoded tokens consist of identical position representations from a single modality.
% \SE{} % ====================== 상익 수정 ========================
% Furthermore, we can exploit the advanced DETR decoder architectures using the positional information, e.g., DAB-DETR, since our video clip tokens consist of identical position representations from a single modality.
% 원준 원본
% It also enables the use of advanced DETR decoder architectures, e.g., DAB-DETR, for the first time, as these works exploit the position information within a single modality.
% =========================================================
Overall, our superior performances over the existing approaches validate the significance of the role of text query for MR/HD.
% Our extensive experiments on QVHighlights, TVSum, and Charades-STA datasets validate the significance of considering the role and the quality of text query.

% All components combined with dynamic anchor moments for the query of decoder, our FOQUE fosters the query-dependent video representation, thereby making the 
% All components combined, our modified transformer encoding process fosters the query-dependent video representation thereby achieving the state-of-the-art results on various benchmarks of moment-retrieval and highlight detection.
	
% -	Video Platform & Streamer & Consumer의 증가. 
% Video는 다른 데이터 타입보다 정보가 많아 유용하지만, 이는 다른 말로 해석하면 video를 보는 것은 time-consuming 하고, 원하는 것을 찾아보기에는 힘들 수 있음.
% 따라서, 많은 매체에서는 사람들의 더 많은 이목을 끌기 위해 highlight 비디오라는 것을 편집하여 공유도 함.
% 하지만, highlight video를 만들기 위해 사람의 노력이 필요한 현 시점에서, This spotlights the need to retrieve the user-requested / Highlight moments in the video.

% -	이전에도 이러한 문제를 해결하기 위해 (asdfasdf) for moment retrieval, (asdfasdf) for highlight detection 등이 제안 되었지만, 이들은 비디오의 특정 영역을 찾는다는 공통된 목적을 가지고 있으면서도, 데이터 셋의 한계로 인해 따로 연구되었음. 이를 문제 삼으며, 최근에는 두 task를 동시에 학습할 수 있는 dataset이 소개 되었는데, 컴퓨터비전에서 최근 각광을 받고 있는 Transformer 모델 도입과 함께 큰 발전을 거듭하고 있음.

% -	구체적으로, 이 두가지 task를 수행하기 위해서는 transformer를 두가지 방법으로 이용할 수 있는데, moment-DETR 처럼 moment 를 clip의 set 단위로 예측할 수 있고, UMT 처럼 clip-wise prediction을 할 수 있음. 하지만, 이들은 query를 condition이 아닌 video와 동등한 레벨로 취급하거나 [mDETR], 매 클립이 self-attention으로 mixing 된 후에 condition을 걸어주어 clip간의 차이를 확실하지 이용하지 못하였고, 또한, 확실하게 condition으로 주지 못하였고, video와 query 사이의 관계를 한정적으로만 이용하였다.

% -	따라서, we explore three different ways to fully exploit query information. First, we design one-way cross-attention layer to condition every clip with the query features. Then, we utilized the negative video-text pairs to better model the relationships between the video and the text embeddings. Lastly, we define the saliency token to be the video-query dependent saliency estimator.


















% ===================== neg pair 부분 ===========================
% Nevertheless, the current training scheme, only considering the given video-query pair, still disturbs the model from identifying proper query-relevance prediction.
% In detail, the model focus on learning the fine-grained discrepancy between video clips, while neglecting the information they share, which contains significant clues to understand the context of video.
% Therefore, we leverage the negative video-query relationships to enhance the capability of identifying the contextual similarity of query and video clips.
% Therefore, we leverage the negative video-query relationships by suppressing those pairs, so that enhance the capability of identifying the contextual similarity of query and video clips.
% We hypothsize the diversity in query-video pairs are insufficient to learn the general relationship between text query and video.
% Therefore, we leverage the negative video-query relationships by suppressing the saliency scores of the irrelevant video-query pairs.
% However, this architectural design still lacks in identifying proper query relevance.
% We argue that the current training scheme only focuses on learning the fine-grained discrepancy between clips in a single video, while neglecting the information they share, which contains significant clues to understand the context of the video.
% Therefore, we leverage the negative video-query relationships to enhance the capability of identifying the contextual similarity of query and video clips.
% However, this architectural design still lacks in identifying proper query relevance.
% We argue the problem of the current training scheme that only focuses on learning the fine-grained discrepancy between clips in a single video.
% That is, the current design neglects the information shared throughout the video, although it contains significant clues to understand the context of the video.
% \section{Related Work} \label{sec:relatedwork}

The section introduces the research related to the paper, which can be divided into three parts: (1) high-utility pattern mining; (2) top-$k$ utility itemset mining; and (3) targeted pattern mining.

\subsection{High-utility pattern mining}

Frequent itemset mining (FIM) \cite{aggarwal2014frequent,agrawal1994fast,han2000mining} has been extensively studied for decades. However, relying only on frequency cannot bring enough benefits to users. Factors such as quantity and profit should also be considered. For this reason, Chen \textit{et al.} \cite{chan2003mining} put forward a new task called high-utility itemset mining (HUIM). Since then, utility mining research has developed rapidly \cite{gan2021survey,lin2016efficient,song2016high,wu2021haop}. For the convenience of discussion, high-utility itemset mining algorithms are grouped into the following three categories:

\textbf{Apriori-based algorithms}: Since Agrawal \textit{et al.} \cite{agrawal1993mining} proposed the Apriori property in 1994, lots of algorithms based on Apriori have been published. For example, Liu \textit{et al.} \cite{liu2005two} introduced the prominent Two-Phase algorithm to handle the difficulty that the utility, unlike frequency, is neither monotone nor anti-monotone. That algorithm uses an overestimation of the utility called \textit{TWU} (Transaction Weighted Utilization) to find candidate itemsets in a first phase. Thereafter, in a second phase, the database is searched again to determine the exact utility value of each candidate itemset. The IIDS algorithm \cite{li2008isolated} is an improved version of Two-Phase that discards isolated items to shrink the search space. However, the common disadvantage of Apriori-like algorithms is that plenty of candidate patterns are generated, resulting in considerable computational costs and memory consumption.

\textbf{Tree-based algorithms}: Tseng \textit{et al.} \cite{tseng2010up} designed the UP-tree structure, a utility-pattern tree, and introduced the UP-Growth algorithm inspired by FP-Growth. Subsequently, other versions of tree-based algorithms \cite{song2014mining,tseng2012efficient} have been presented. In general, utilizing the UP-tree can prevent many meaningless database scans. When working with large-scale databases, however, this structure grows increasingly complex and occupies a massive amount of memory.

\textbf{Other structure-based algorithms}: HUI-Miner \cite{liu2012mining} utilizes a novel data structure known as a utility-list, which avoids the difficulty of generating numerous candidates. Moreover, the FHM algorithm \cite{fournier2014fhm} reduces the cost of join operations by using a tighter upper bound, which results in outperforming HUI-Miner. However, the join operation on lists of these algorithms takes time and memory. Thus, Zida \textit{et al.} \cite{zida2015efim} proposed the EFIM algorithm with high-utility database projection (HDP) and high-utility transaction merging (HTM) techniques to lower the expensive cost of database passes. The utility-list-based CoUPM algorithm for correlated utility-based pattern mining \cite{gan2019correlated}. In summary, these algorithms integrate various strategies to discover HUIs as efficiently as possible.

\subsection{Top-$k$ utility itemset mining}

Although the above algorithms are effective in finding the desired set of itemsets, the efficiency of mining is strongly related to the selection of the minimum utility threshold. However, it is not easy to identify an appropriate threshold. Many top-$k$ pattern mining algorithms were thus designed to directly discover the set of top-$k$ HUIs, rather than asking users to specify a utility threshold. Top-$k$ HUIM algorithms mainly consist of two types: the first is the two-phase algorithms, and the other is the one-phase algorithms.

\textbf{Two-phase algorithms}: The task of discovering the top-$k$ HUIs was proposed by Wu \textit{et al.} \cite{wu2012mining} with the TKU algorithm, which outperformed HUIM algorithms in terms of speed. The TKU algorithm is a two-phase algorithm. In the first phase, a UP-Tree is built, and promising top-$k$ HUIs are generated. Then, in the second phase, the desired top-$k$ HUIs are selected among them. TKU applies several strategies to filter unpromising candidates during the search \cite{tseng2015efficient} and achieve higher efficiency. Subsequently, REPT \cite{ryang2015top} was introduced with optimizations to record and pre-calculate the utility of items to prune the search space effectively and raise the minimum utility threshold. REPT uses a tree structure and pre-evaluation matrixes as tools to store utility information. However, these two-phase algorithms still generate large sets of candidates, which causes unreasonably long runtimes and high memory usage.

\textbf{One-phase algorithms}: For top-$k$ HUIM, the one-phase TKO algorithm \cite{tseng2015efficient} was developed to solve the shortcomings of two-phase algorithms. TKO takes advantage of the utility-list structure of HUI-Miner, and outperforms the TKU and REPT algorithms according to experiments \cite{tseng2015efficient}. Similarly, another one-phase algorithm called KHMC \cite{duong2016efficient} also discovers the top-$k$ HUIs by using the utility-list structure. In KHMC, an estimated utility co-occurrence pruning (EUCP) technique is applied, which is based on precalculating the TWU of 2-itemsets. Moreover, the algorithm also adds another pruning strategy named early abandoning to avoid completely constructing the lists of unpromising itemsets. Three threshold-raising strategies are able to significantly shrink the search space and enhance the algorithm's efficiency. The THUI algorithm \cite{krishnamoorthy2019mining} has better performance thanks to introducing the concept of Leaf Itemset Utility (LIU), a triangular matrix, which can be implemented with only a small amount of memory to store utility information. Besides, the LIU-E and LIU-LB threshold raising strategies also accelerate the mining speed of the algorithm. THUI greatly outperforms TKO and KHMC, especially for dense or large datasets.

In addition, there are various other top-$k$ pattern mining problems and variations, such as mining top-$k$ sequential patterns \cite{zhang2021tkus}, mining top-$k$ HUIs in data streams \cite{cheng2021etkds}, discover top-$k$ high-utility sequential patterns \cite{zhang2021tkus}, and mining top-$k$ HUIs with negative utility values \cite{sun2021mining}.


\subsection{Targeted pattern mining}

Those algorithms listed above are designed to find all itemsets that meet a single predetermined criterion. Target-oriented query algorithms give an alternative solution to this problem by filtering out unnecessary information. Rather than searching for numerous but mostly insignificant items, the user can enter any target and then discover patterns containing the desired items. Several target-oriented query algorithms based on frequency have been developed in earlier studies. These interactive methods are capable of returning results containing a target. Kubat \textit{et al.} \cite{kubat2003itemset} were among the first to address the issue of processing target queries in a transactional database. They implemented target query processing algorithms for association mining by creating itemset trees that can be progressively updated. Fournier-Viger \textit{et al.} \cite{fournier2013meit} developed the Memory Efficient Itemset Tree (MEIT) to further reduce memory requirements. The tree is optimized to perform incremental modifications when new transactions are inserted, and it employs a node-compression method. For multi-objective mining of big data, the guided FP-growth (GFP-growth) algorithm based on FP-Growth was proposed by Shabtay \textit{et al.} \cite{shabtay2018guided}. In particular, many experiments have illustrated the excellent performance of the algorithm on imbalanced data. Target-oriented mining has also been studied and applied to discover sequential patterns. The targeted mining algorithm for sequential patterns proposed by Chueh \textit{et al.} \cite{chueh2010mining} speeds up the search for the target itemsets by using the reversion of the original sequence and comparing the reversed sequence with the related itemsets. Furthermore, clustering analysis is applied to automatically set time partition values for the task of time-interval sequential pattern mining. A novel target-oriented sequential pattern mining approach was presented by Chand \textit{et al.} \cite{chand2012target}, which uses RFM (recency, frequency, and monetary) constraints. As a result, fewer database projections are done, and the space complexity is reduced. To remove some useless or irrelevant patterns in high utility sequential pattern mining \cite{zhang2021shelf,gan2021explainable}, the TUSQ algorithm \cite{zhang2021tusq} first introduced the concept of utility into target sequence queries. The algorithm does not focus on frequency like previous algorithms, but rather on utility. Recently, the TargetUM algorithm \cite{miao2021targeted} has been proposed to fill the gap and perform target-oriented mining in HUIM.

In general, the TargetUM algorithm provides an integrated approach for high-utility mining with a target query, which serves as the foundation for this research. However, there are no studies combining top-$k$ high-utility methods with target pattern queries. This paper introduces the problem of targeted utility mining with the concept of top-$k$ patterns to prevent the generation of large sets of HUIs and to accurately and quickly process target queries.

% \section{Data Acquisition \& Sensor Modalities}
\label{sec:dataset}


% \begin{figure}[!b]
%  \centering
%     \includegraphics[width=\linewidth]{figures/hammer_other_datasets_comp.pdf}
%     \caption{\textbf{Issues of Other Datasets.} Multiple sources of error, e.g. missing glass surfaces (table in left and in right scenes), noisy background, missing data, corrupted thin objects, holes, etc. exist in Replica~\cite{replica19arxiv}, ScanNet~\cite{dai2017scannet}, or MatterPort3D~\cite{Matterport3D}.}
%     \label{fig:comparison_datasets}
% \end{figure}


We set up scenes composed of multiple objects with different shapes and materials to analyse sensor characteristics. 3D models of photometrically challenging objects with reflective or transparent surfaces are recorded with high quality a priori and aligned to the scenes. Images are captured from a synchronised multi-modal custom sensor mounted at a robot end-effector to allow for precise pose camera measurements~\cite{PhoCal}. High-quality rendered depth can be extracted a posteriori from the fully annotated scenes for the viewpoint of each sensor. The acquisition pipeline is depicted in Fig.~\ref{fig:annotation_overview}.
 
Previous 3D and depth acquisition setups~\cite{replica19arxiv,Matterport3D,dai2017scannet} scan the scene as a whole which limits the quality by the used sensor. We instead separately scan every single object, including chairs and background, as well as small household objects a priori with two high-quality structured light object scanners. 
This process significantly pushes the annotation quality for the scenes as the robotic 3D labelling process only has a point RMSE error of $0.80$~mm~\cite{PhoCal}. 
For comparison, a Kinect Azure camera induces a standard deviation of $17$~mm in its working range~\cite{liu2021stereobj}. 
The accuracy allows us to investigate depth errors arising from sensor noise objectively, as shown in Fig.~\ref{fig:dataset_quality}, while resolving common issues of imperfect meshes in available datasets (cf. Fig.~\ref{fig:teaser}, left).


\subsection{Sensor Setup \& Hardware Description}
The table-top scanner (EinScan-SP, SHINING 3D Tech. Co., Ltd., Hangzhou, China) uses a rotating table and is designed for small objects. The other is a hand-held scanner (Artec Eva, Artec 3D, Luxembourg) which we use for larger objects and the background. 
For objects and areas with challenging material, self-vanishing 3D scanning spray (AESUB Blue) is used. For larger texture-less areas such as tables and walls we temporarily attach small markers~\cite{garrido2014automatic} to the surface to allow for relocalization of the 3D scanner. 
The robotic manipulator is a KUKA LBR iiwa 7 R800 (KUKA Roboter GmbH, Germany) with a position accuracy of $\pm0.1$~mm. 
We validated this during our pivot calibration stage (Fig.~\ref{fig:annotation_overview} b) by calculating the 3D location of the tool tip (using forward kinematics and hand-tip calibration) while varying robot poses. The position varied in $\left[-0.158,0.125\right]$~mm in line with this.
Our dataset features a unique multi-modal setup with four different cameras, which provide four types of input images (RGB, polarization, stereo, Indirect ToF (I-ToF) correlation) and three different depth images modalities (Direct ToF (D-ToF), I-ToF, Active Stereo). RGB and polarization images are acquired with a Phoenix 5.0 MP Polarization camera (PHX050S1-QC, LUCID Vision Labs, Canada) equipped with a Sony Polarsens sensor (IMX264MYR CMOS, Sony, Japan). 
To acquire stereo images, we use an Intel RealSense D435 (Intel, USA) with switched off infrared projector. Depth is acquired from an Intel RealSense L515 D-ToF sensor, an Intel Realsense D435 active stereo sensor with infrared pattern projection, and a Lucid Helios (HLS003S-001, LUCID Vision Labs, Canada) I-ToF sensor. 
A Raspberry Pi triggers each camera separately to remove interference effects between infrared signals of depth sensors. 
The hardware is rigidly mounted at the robot end-effector (see Fig.~\ref{fig:hardware}) which allows to stop frame-by-frame for the synchronized acquisition of a pre-recorded trajectory.


\begin{figure}[!t]
 \centering
    \includegraphics[width=0.9\linewidth]{figures/sensor_rig_one_column.png}
    \vspace{-2mm}
    \caption{\textbf{Camera Rig and 3D Sensor Data.} The custom multi-modal sensor rig comprises depth sensors for I-ToF (top left), Stereo (lower left), D-ToF (lower right), and RGB-P (Polarization, top right). It is fixed to a robot end-effector (top) and a Raspberry Pi (right) triggers acquisition.}
    \label{fig:hardware}
\end{figure}

\begin{table*}[!ht]
\setlength{\tabcolsep}{7pt}
\centering
\caption{\textbf{Comparison of Datasets}. Shown are differences between our dataset and previous multi-modal depth datasets for indoor environments. Our dataset is the only one that provides highly accurate GT (Depth, Surface Normals, 6D Object Poses, Instance Masks, Camera Poses, Dense Scene Mesh) together with varying sensor data for real scenes.}
\label{tab:dataset_comparison}
\vspace{-2mm}
\begin{tabular}{r | c c c c c c c c c c c c } 
\toprule
  \small Dataset &
  {\small Acc.GT}          &
  {\small RGB}        &
  {\small D-ToF}          &
  {\small I-ToF}        &
  {\small Stereo} &
  {\small Act.Stereo} &
  {\small Polar.} &
  {\small Indoor}          &
  {\small Real}         &
  {\small Video}        &
  {\small Frames}       \\ 
\midrule
Agresti~\cite{Agresti_2019_CVPR} & -          &  - & - & \checkmark & -          & -& -          & \checkmark & \checkmark & -          & $113$      \\
CroMo~\cite{CroMo} & - &  -  & - & \checkmark & \checkmark & \checkmark & \checkmark & (\checkmark) & \checkmark & \checkmark & ${>}10$k    \\
Zhu~\cite{zhu2019depth}      & -            & (\checkmark) & -          & - & - & - & \checkmark & \checkmark & \checkmark & -        & $1$   \\
Sturm~\cite{sturm2012benchmark}         & - & \checkmark & - & - & - & -          & -          & \checkmark & \checkmark & \checkmark & ${>}10$k   \\
\cite{kadambi2017depth}/\cite{qiu:2019a}/\cite{ba2020deep}            & - & \checkmark   & -          & - & - & - & \checkmark & \checkmark & \checkmark & -          & $1/40/300$ \\
Guo~\cite{Guo_2018_ECCV}       & \checkmark   & -          & - & \checkmark & -          & - & -          & \checkmark & - & - & $2000$\\
\textbf{Ours} & \checkmark & \checkmark & \checkmark & \checkmark & \checkmark & \checkmark & \checkmark & \checkmark & \checkmark & \checkmark & ${>}10$k\\
\bottomrule
\end{tabular}
\end{table*}


\subsection{Scene Statistics \& Data Comparison}
We scanned 7 indoor areas, 6 tables, and 4 chairs, with the handheld scanner as background and large objects. 64 household objects from 9 categories (bottle, can, cup, cutlery, glass, remote, teapot, tube, shoe) are scanned with the tabletop structured light scanner. The data comprises 13 scenes split into 10 scenes for training and 3 scenes for testing. Each scene is recorded with 2 trajectories of 200-300 frames with and without the objects. This sums up to 800-1200 frames per scene, with a total of 10k frames for training and 3k frames for our test set. The 3 test scenes have different background setups: 1) Seen background, 2) Seen background with different lighting conditions and 3) Unseen background and table, with three different object setups: 1) Seen objects 2) Unseen objects from the seen category 3) Unseen objects from unseen categories (shoe and tube).
Table~\ref{tab:dataset_comparison} compares our dataset with various existing setups. To the best of our knowledge, our dataset is the only multi-modal dataset comprising RGB, ToF, Stereo, Active Stereo, and Polarisation modalities simultaneously with reliable ground truth depth maps. 
% \section{Motion Imitation}
The foundation of our algorithm is an imitation learning framework. In the following sections, we describe the observations, actions, reward functions, and the training procedure. We explain these for a single hand, but our method can trivially incorporate both hands by increasing the state and observation dimensions.

\subsection{Problem Formulation}
Given motion capture trajectories for a hand (or both hands) and an object, we aim to learn an effective policy that can manipulate the object by following a reference trajectory. 
% We apply deep reinforcement learning (DRL) to learn such a control policy.

We formulate the control problem as a partially observable Markov Decision Process (PoMDP) with tuple $(\mathcal{S}, \mathcal{O},o, \mathcal{A},\mathcal{T},r, o, \gamma)$, where $\mathcal{S}$ is the state space, $\mathcal{O}$ is the observation space for the hand and the object, $\mathcal{A}$ is the action space for actuating the hand, $\mathcal{T}$ is the transition dynamics (physics simulation), $r$ is the reward function, $o$ is the observation emission function, and $\gamma$ is the discount factor. At a high level, we want to find a parameterized control policy $\pi_{\theta}$ so that it will maximize the expected sum of rewards over a distribution of trajectories
\begin{equation}
    \pi_{\theta^*} = \argmax_{\theta} \mathbb{E}_{(s_0,s_1,\dots,s_T)}\left[\sum_{t=0}^{T}\gamma^{t}r(s_t, \pi_{\theta}(s_t))\right].
\end{equation}

\subsection{Observation representation}
%\sloppy The observation of the agent is encoded in the following vector form $\mathbf{o} = (\mathbf{x}_{hand},\mathbf{x}_{obj},\mathbf{v}_{hand},\mathbf{v}_{obj},\bar{\mathbf{x}}_{hand},\bar{\mathbf{x}}_{obj},\bar{\mathbf{x}}_{hand}\ominus \mathbf{x}_{hand},\bar{\mathbf{x}}_{obj}\ominus \mathbf{x}_{obj},\mathcal{C})$. 
The observation space can be divided into four components: the states of the simulated hand and the object $\{\mathbf{x}_{hand},\mathbf{x}_{obj}\}$, reference states of hand and object from mocap $\{\bar{\mathbf{x}}_{hand},\bar{\mathbf{x}}_{obj}\}$, their differences from simulation $\{\bar{x}_{hand}\ominus x_{hand} ,\bar{x}_{obj}\ominus x_{obj}\}$, and contact information $\{\mathcal{C}\}$. As studied in Bergamin et al.~\shortcite{bergamin2019drecon}, using the difference between simulation and reference as observation improves both training speed and quality. 

\emph{Simulated states:} The positional state of the hand $\mathbf{x}_{hand} =\linebreak[1]  (\mathbf{x}_{h\_root},\linebreak[1]  \mathbf{R}_{h\_root},\linebreak[1]  \cos(q),\linebreak[1]  \sin(q))$ is a $46D$ vector containing the pose of the root and all joint angles of the hand, where $\mathbf{x}_{h\_root} \in \mathbb{R}^3$ is the $3D$ position of the wrist, $\mathbf{R}_{h\_root} \in \mathbb{R}^3$ orientation of the wrist represented as axis-angle, and $q \in \mathbb{R}^{20}$ are all the joint angles in a hand. $\mathbf{x}_{obj} = (\mathbf{x}_{o\_root},\mathbf{R}_{o\_root})$ is a $6D$ vector containing the position and axis-angle orientation of the simulated object. $\mathbf{v}_{hand}$ and $\mathbf{v}_{obj}$ are $26D$ and $6D$ vectors containing the velocities of the hand and the object. Note that we actuate the hand's orientation directly, and we found the use of axis-angle for hand orientation to be especially important for success.

\emph{Reference states:} $\bar{\mathbf{x}}_{hand}$ and $\bar{\mathbf{x}}_{obj}$ are the reference poses of the hand and object at the current frame expressed in the same format as the pose of simulated hand and object. 

\emph{State differences:} $\bar{x}_{hand}\ominus x_{hand}$ and $\bar{x}_{obj}\ominus x_{obj}$ are the differences between the simulated pose and the reference pose of both the hand and object. For all the rotational information, we evaluate the rotation differences in $SO(3)$ and express the difference into the corresponding format \revised{of axis angle or sine and cosine of Euler angles} used in the observation.

\emph{Contact information:} We include an additional $19D$ vector $\mathcal{C}$ to capture the contact forces between the hand and the object. We surround each finger capsule with a contact sensor that is a slightly larger capsule with a 10\% larger radius. These contact sensors register contact forces from the object at each control step, and we record the sum of all contact forces exerted by the object on each rigid segments of the hand through the contact sensors. Because the sensors correspond to finger segments, their readings indicate how much contact forces are being exerted, and implicitly inform where the contacts are. 
%In the last part of the observation, we want to implicitly inform the policy about the geometry of the object, so that the policy will be able to adjust it's behaviour based on the shape changes of the object. To do so, 
% \yuting{not clear how contact force magnitude encodes geometry info}

Combining all the described observation components, we have a $207D$ vector that describes the state of the simulation when we are considering a manipulating task involving a single hand. When we train the policy to track a two-hand manipulation sequence, we double the observations for the hand and end up with a $390D$ vector.

\subsection{Action Representation}
Similar to the approach in \cite{bergamin2019drecon}, at each time step $t$, the policy outputs an action $\mathbf{a}_t = \{\Delta{\mathbf{x}},\Delta{\mathbf{R}},\Delta{\mathbf{q}}\}$, a $26D$ vector specifying the spatial displacement to the hand's reference pose, where $\Delta{\mathbf{x}} \in \mathbb{R}^3$ is the hand root linear displacement, $\Delta{\mathbf{R}} \in \mathbb{R}^3$ is the root orientation displacement expressed in axis-angle, and $\Delta{\mathbf{q}} \in \mathbb{R}^{20}$ is the joint angle displacement for each joint. We apply an exponential action filter with $\alpha=0.3$ to generate smoother motions. Once the PD target is computed, we compute joint torques using the stable-PD controller~\cite{tan2011stable}. \revised{The full control loop is shown in Figure \ref{fig:control_loop}}
\begin{figure}
\centering
\includegraphics[width=0.48\textwidth]{figures/control_loop.png}
\caption{Overview of the control loop}
\label{fig:control_loop}
\end{figure}
%For more details, please refer to the original Stable PD paper.

% After an action is generated, we compute the target $q_{targ}=(1-\alpha) q_{prev} + \alpha (q_{ref}+a_{t})$ as blending between hand pose from previous frame and new pose generated by the action so that the generated target across simluation is smooth.

% Instead of directly using proportional derivative (PD) controllers, we adapt the stable-PD controller proposed by Tan et al.~\shortcite{tan2011stable} in simulation. This change allows us to more easily set the proportional and damping coefficients in the controller to make sure the simulated hand can always accurately reach the desired pose in the next few simulation steps without overshooting. %\sehoon{Provide a one-line justification of Stable PD}
%  Since our hands have full translation and rotation freedom, the learned policy needs to directly control both the roots as well as all the finger joints. A torque for each degree of freedom is computed using the stable-PD formulation describe as:
% \begin{equation}
%     \tau = -K_p (q_t + \dot{q}_t\Delta{t}-q_{targ})-K_d(\dot{q}_t+\ddot{q}_t\Delta{t}).
% \end{equation}
% For more details, please refer to the original Stable PD paper.

\subsection{Reward Function}
Our goal is to track the reference motions of both the hands and the object as closely as possible. Inspired by the original DeepMimic paper~\cite{peng2018deepmimic}, we design our reward function as follows:
\begin{equation}
    r = w_{od}r_{od}+w_{or}r_{or}+w_{hd}r_{hd}+w_{hr}r_{hr}+w_{hj}r_{hj}
\end{equation}
which consists of the object position term $r_{od}$, the object rotation term $r_{or}$, the hand position term $r_{hd}$, the hand orientation term $r_{hr}$, and the hand joint term $r_{hj}$.
To enforce a match between the simulated object and the reference object's position and orientation, we define the terms $r_{od}$ and $r_{or}$:
\begin{equation}
    r_{od} = \exp\left(-k_{od}\|\hat{x}_{obj}-x_{obj}\|^2\right),
\end{equation} and
\begin{equation}
    r_{or}=\exp\left(-k_{or}\|\hat{q}_{obj}^{-1}q_{obj}\|^2\right),
\end{equation}
which compares the object's position $x_{obj}$ and orientation $q_{obj}$ to their desired values.
For all $N$ rigid segment of the hand with the index i, we define the reward terms $r_{hd}$ and $r_{hr}$:
\begin{equation}
    r_{hd}=\exp\left(-k_{hd}\sum_{i =1}^N\|\hat{x}_{i}-x_{i}\|^2\right),
\end{equation} and 
\begin{equation}
    r_{hr}=\exp\left(-k_{hr}\sum_{i =1}^N\|\hat{q}_{i}^{-1}q_{i}\|^2\right),
\end{equation}
where $x_{i}$ and $q_{i}$ represent the position and orientation of the $i$th body segment.
In addition enforcing the hand rigid segment's tracking, we also define the reward term $r_{hj}$
\begin{equation}
    r_{hj}=\exp\left(-k_{hj}\sum_{i =1}^{M}\|\hat{\theta}_{i}-\theta_{i}\|^2\right),
\end{equation}
by comparing all the current and desired joint angles, $\theta$ and $\hat{\theta}$. For all experiments, we set the weights as $w_{od}=4$, $w_{or}=4$, $w_{hd}=0.05$, $w_{hr}=0.05$, and $w_{hj}=0.1$. 
% These terms minimize positional differences of each rigid finger segment between simulation and the reference as well as joint angle differences between simulated and reference hands.

\subsection{Terminal Condition}
As studied in DeepMimic~\cite{peng2018deepmimic}, early termination of a rollout when the simulation enters an unrecoverable state can save computation on low value trajectories. We design the early termination criteria to restrict how much the object's state is allowed to deviate from the reference: either $d_{thr}=10cm$ in translation or $\phi_{thr}=60^\circ$ in rotation. We choose these thresholds to allow the hands to explore its action space more freely, but this still eliminates irredeemable failures.  

% \subsection{Policy Training}
% We use Proximal Policy Optimization (PPO)~\cite{schulman2017proximal}, a common on-policy reinforcement learning algorithm in all our training. \sehoon{Yunbo, please say a sentence or two about the policy architecture}
% PPO aims to minimized two optimization losses $L_{surrogate}$ and $L_{KL}$ described as:
% \begin{align}
%     L_{PPO}(\theta) &=L_{surrogate}+L_{KL}\nonumber\\
%     &=-\mathbb{E}_{t}\left[min(r(\theta)A_t,clip(r(\theta),1-\epsilon,1+\epsilon)\right]\\
%     &\ \ \ - \beta\mathbb{E}_{t}\left[KL\left[\pi_{\theta}(s_t)|\pi_{\theta_{old}}(s_t)\right]\right]\nonumber .
% \end{align}
% A dynamically adjusted coefficient $\beta$ is used to make sure policies from two consecutive iterations do not deviate too much in their KL divergence. 



%We terminate the rollout when the simulated object is drifted away more than distance $d_{thr}=10cm$ from the reference or its orientation is off by $\phi_{thr}=60^\circ$ from the reference. This criteria gives some tolerance for the hand to adjust its motion to capture the object back in place, and it will be triggered most likely when an object is out of control and about to fall out of the hand. 

 


\section{Greedy Shape Curriculum for Novel Objects}


% Motivation for reusing the mocap
It would be undesirable to record a new motion capture sequence for each new object that we want to manipulate. Instead, we would like to generalize an existing motion example to different objects in simulation. For example, we may want to manipulate a teapot or a toy train using the same reference motion for a cube. However, it is not straightforward to learn an effective policy for a new shape because it often requires significant changes in the control strategy.

% Intuition. Why would a naive curriculum not work?
Our key intuition is that we can co-train policies on a set of intermediate shapes morphing between the original object and the target object as a curriculum. A naive method would be to use a training curriculum that starts the learning from the source object and gradually morph the shape to the target in a linear progression. In practice, however, this linear curriculum is often unsuccessful because the morphing progression may not exactly correlate to the task's difficulty. \revised{A better tuned morphing algorithm might be able to give stronger correlation between morphing progression and training difficulty, but such algorithm requires additional human effort, and may not be able to generalize across different source target pairs.}


\begin{figure}
\centering
\includegraphics[width=0.48\textwidth]{figures/system_diagram.png}
\caption{Illustration of our greedy shape curriculum. Each iteration of the algorithm (1) selects and trains the most promising (policy, shape) pair, (2) evaluates the updated policy on all shapes, and (3) overwrites a shape’s policy pairing if the new policy is better than the cached policy. Example policy performance metrics are displayed numerically below each policy shape.}
\label{fig:morph_training}
\end{figure}
% Our approach: insight and summary.
Instead, we design a novel training schedule that allows greedily switching between any intermediate shape morphs for a more flexible curriculum. Our algorithm maintains a collection of best policies for each shape. For every $K=20$ policy iteration, it selects the best performing \emph{unsuccessful} morph and its paired policy for the next round of training. Once the policy is further trained, the newly updated policy's performance is evaluated across the entire collection of shapes, and overrides existing policies if it performs better (Figure \ref{fig:morph_training}). Despite its greedy nature, we found this automated curriculum more effective in policy transfer than naive fining-tuning or a fix curriculum. The full procedure is described in Algorithm~\ref{alg:shape_morph_training}.
\begin{figure*}[h!]
    \centering
    \includegraphics[height=\low]{figures/sequence_still_frames/cube_bunny_0.0.png}
    \hfill
    \includegraphics[height=\low]{figures/sequence_still_frames/cube_bunny_0.2.png}
    \hfill
    \includegraphics[height=\low]{figures/sequence_still_frames/cube_bunny_0.4.png}
    \hfill
    \includegraphics[height=\low]{figures/sequence_still_frames/cube_bunny_0.6.png}
    \hfill
    \includegraphics[height=\low]{figures/sequence_still_frames/cube_bunny_0.8.png}
    \hfill
    \includegraphics[height=\low]{figures/sequence_still_frames/cube_bunny_1.0.png}
    \hfill
    \caption{Morph stages of the collision shapes for transferring the cube motion to a bunny after applying V-HACD.}
    \label{fig:collision_shape_morphs}
\end{figure*} 
\begin{figure*}[h!]
    \centering
    \includegraphics[height=\high]{figures/sequence_still_frames/wineglass_1.png}
    \hfill
    \includegraphics[height=\high]{figures/sequence_still_frames/wineglass_2.png}
    \hfill
    \includegraphics[height=\high]{figures/sequence_still_frames/wineglass_3.png}
    \hfill
    \includegraphics[height=\high]{figures/sequence_still_frames/wineglass_4.png}
    \hfill
    \includegraphics[height=\high]{figures/sequence_still_frames/wineglass_5.png}
    \hfill
    \label{fig:torus_large1_single}
\end{figure*}

\begin{figure*}[h!]
    \centering
    \includegraphics[height=\high]{figures/sequence_still_frames/hemi_0.png}
    \hfill
    \includegraphics[height=\high]{figures/sequence_still_frames/hemi_1.png}
    \hfill
    \includegraphics[height=\high]{figures/sequence_still_frames/hemi_2.png}
    \hfill
    \includegraphics[height=\high]{figures/sequence_still_frames/hemi_3.png}
    \hfill
    \includegraphics[height=\high]{figures/sequence_still_frames/hemi_4.png}
    \hfill
    \caption{Still frames from manipulation sequences involving a wineglass (\textbf{top}) and a hemisphere (\textbf{bottom}).}
    \label{fig:hemisphere_large1_single}
    \vspace{-0.1in}
\end{figure*}
\begin{algorithm}[tb]
\caption{Greedy Shape Curriculum}
\label{alg:shape_morph_training}
\begin{algorithmic}[1]
    \STATE Initialize score list $S$
    \STATE Initialize policy list $\Pi$
    \STATE Initialize current shape $s=0$ and current policy $\pi = \Pi[s]$
    \FOR{$i = 0, 1, 2 ,\dots$}
    \IF{$i \mod k ==0$}
    \FOR{Every shape $j$}
    \STATE score = rollout on $j$ using policy $\pi$
    \IF{score > $S[j]$}
    \STATE $S[j]$ = score
    \STATE $\Pi[j] = \pi$
    \ENDIF
    \ENDFOR
    \STATE $s$ = Get best unsuccessful shape
    \STATE $\pi = \Pi[s]$
    \ENDIF
    \STATE PPO using $s$ and $\pi$
    \ENDFOR
\end{algorithmic}
\end{algorithm}

A key component of our greedy shape curriculum is a \emph{goodness score} that describes how likely a policy will succeed on a given shape. This metric will be used to update the best policy for a shape, and for selecting the next (shape, policy) pair for training. An obvious choice would be the average episodic reward. However, the consideration here is slightly different. A high episodic reward imposes a more strict constraint to the quality of object pose matching, making it hard to achieve when the target shape is too different from the source shape. A low episodic reward, on the other hand, cannot guarantee the completion of a rollout. On the other hand, the rollout length alone is too simple and fails to reflect the quality of the motion. We want a criteria with high tolerance to object deviation but with low tolerance to failure of completion. To this end, we design our criteria as a combination of the rollout duration and tracking accuracy, and this works robustly in practice. As described in \revised{Equation} \ref{equ:eval_score}, we use the product between the normalized episode length and the sum of hand joint reward of the rollout as the goodness score of a policy for a given shape. This encourages the resulting policy to use a similar manipulation strategy to the input. We consider a score higher than $d=0.55$ as \emph{successful}, and we only pick from the unsuccessful shape morphs for policy training to make progress.
%we compare this evaluation score against a threshold $d$, and mark all shapes with score below that as unsuccessful. This makes sure that we are always training policies greedily from all the unsuccessful candidates.
\begin{equation}
    \label{equ:eval_score}
    f = \frac{L}{T} \cdot \frac{\sum_{0}^{L}{r_{joint}}}{T}
\end{equation}

Our greedy schedule is effectively an exploitation strategy, and we still need to balance it with some exploration to avoid local minima. If a particular shape is repeatedly picked for training and starving other shapes, we instead randomly select another shape and its paired policy in the next iteration. This is especially helpful when a policy gets ``stuck'' on a challenging frame towards the end of a sequence while other shapes have not had much training yet. Training progress on other shapes can then help improve such challenging cases later. Similarly, if we are successful on all shapes before the compute budget has been reached, we randomly pick a policy to continue training for further improvement. 
%This would make sure all shapes can continue to improve. In the meantime, if a particular morph is being too challenging to be solved, this design gives the learning a chance to attend to some other morphs and potentially helping solve the challenging morph. 



% It would be undesirable to need to record a new motion capture sequence for each new object that we want to manipulate.  Instead, we would like to be able to learn how to manipulate a new object using the mocap from another object. As an example, we may want to manipulate a teapot following the reference motion from a cube. \sehoon{challenge?} 

% To train such a policy, we start by creating a set of intermediate shape morphs between the original object and target object.  We then use a learning process to train a collection of policies, one policy for each individual morph, including the original and target object. The intermediate morphs act as stepping stones between the original shape from the mocap sequence and the new object that we want to manipulate. 

% The most straightforward way to train the collection of morphs would be to use a training curriculum.  We would start by training a policy on the source object. Once this policy is successful, this would be used as a starting policy that is then further trained using the next of the morph shapes. The policy from the second shape would then be use for the third shape, and so on. Unfortunately, we found that this strict curriculum over object shapes is often unsuccessful. \sehoon{provide more insights} We hypothesized that it may be beneficial to jump between shapes out of sequence, and indeed this turned out to be the case.

% \sehoon{Two key aspects: how to maintain a list, and when to transfer.}

% \sehoon{Little more focus on greedy aspects..}
% Instead of training a policy based on a sequential curriculum, we maintain a list of the best performing policies for each individual shape in the list as well as their performances. For every $k=20$ training iteration, the system selects the best performing unsuccessful morph and its paired policy for the next round of training. Once the policy is further trained, the new policy's performance is evaluated for each of the shapes. When it is better than the shape's current policy, it becomes the new policy for that shape. This full procedure is described in Algorithm \ref{alg:shape_morph_training}.


%We have found our training approach to be more successful than a naive training curriculum across shapes. Figure~\ref{fig:policy_tree} shows our approach for the source and target shapes of a cube and an elephant. Each arrow indicates when an updated policy performs better than the current policy for another shape.

% \greg{Yunbo, please add a description of how you calculate a given policy's score.}
% To properly measure how well a policy performs on a shape morph, we take consideration of both the duration of a rollout.
% \begin{algorithm}[tb]
% \caption{Shape Morphing Training}
% \label{alg:shape_morph_training}
% \begin{algorithmic}[1]
%     \STATE Initialize score list $S$
%     \STATE Initialize policy list $\Pi$
%     \STATE Initialize current shape $s=0$ and current policy $\pi = \Pi[s]$
%     \FOR{$i = 0, 1, 2 ,\dots$}
%     \IF{$i \mod k ==0$}
%     \FOR{Every shape $j$}
%     \STATE score = rollout on $j$ using policy $\pi$
%     \IF{score > $S[j]$}
%     \STATE $S[j]$ = score
%     \STATE $\Pi[j] = \pi$
%     \ENDIF
%     \ENDFOR
%     \STATE $s$ = Get best unsuccessful shape
%     \STATE $\pi = \Pi[s]$
%     \ENDIF
%     \STATE PPO using $s$ and $\pi$
%     \ENDFOR
% \end{algorithmic}
% \end{algorithm}


% 
% \begin{table}[t]
%     \tablestyle{2pt}{1.05}
    
%     \centering
%     %\resizebox{1\columnwidth}{!}{
%     \begin{tabular}{@{}l|ccccccc}
%     	\toprule
%             \multicolumn{8}{c}{Untrimmed Spatial-Temporal Grounding}
%     	\toprule
%     	\multicolumn{1}{c}{} & \multicolumn{7}{c}{GroundingYouTube}  \\ 
%     	\cmidrule(lr){2-8} 
%     	\multirow{2}{*}{\textbf{Method}}    & \multirow{2}{*}{IoU+Point} &\multicolumn{6}{c}{mAP}  \\ 
%     	                                    &  & 0.1 & 0.2 & 0.3 & 0.4 & 0.5  & 0.1:0.5 \\ 
%     	\midrule
%     	MIL-NCE \citep{miech2020end} & 4.67 & 33.94 & 25.16 & 12.65 & 3.42 & 0.41  & 15.11 \\
%          CoMMA* \citep{tan2021look}   & 1.02 & 2.18  & 1.72 & 1.11 & 0.93 & 0.37 & 1.26\\
%              %Ours S3D                         & 7.78 & 39.43 & 31.47 & 19.38 & 9.14 & 3.79  & 20.64  \\
%              Ours S3D                      & 9.12 & 42.70  & 35.49 & 25.16 & 16.22 & 10.05  & 25.92 \\
%              \midrule
%             CLIP \citep{radford2021learning}  & 3.59 & 29.54  & 22.15 & 9.16 & 2.48 & 0.39 & 12.74 \\
%             CoMMA$\dagger$              & 1.68 & 3.51 & 2.32 & 1.88 & 0.99 & 0.40 & 1.82 \\
%     	   Ours                        & 10.09 & 42.81  & 36.05 & 25.84 & 17.10 & 11.35  & 26.63 \\
%             \midrule
%             GLIP \citep{li2022grounded}      &  1.24 & 2.83 & 2.10 & 1.52 & 0.96 & 0.37 & 1.56 \\
%     	\bottomrule
%     \end{tabular}
%     %\vspace{+0.3cm}
%     \caption{\textbf{Spatio-temporal localization on full videos}. Since our model learned global representations encoding temporal information and spatial correspondences across modalities, it achieves the best performance in spatio-temporal evaluation.
%     % \caption{\textbf{Spatial-temporal localization on full videos}. Our model learned both global representation which encodes temporal information. It also learned spatial correspondence across modalities, which ends up with the best performance in spatial temporal evaluation.
%     \label{tab:st_long}
%     %\vspace{-0.7cm}
%     }
%     %}
% \end{table}
\begin{table*}[t]
    \tablestyle{4pt}{1.05}
    \tiny
    \centering
    \resizebox{2\columnwidth}{!}{
    \begin{tabular}{@{}l|ccccccccccc}
    	\toprule
    	\multicolumn{5}{c}{} &\multicolumn{7}{c}{GroundingYoutube}  \\ 
    	\cmidrule(lr){6-12} 
    	\multirow{2}{*}{\textbf{Method}}  & \multirow{2}{*}{\textbf{Backbone}} & \multirow{2}{*}{\textbf{DataSet}} & \multirow{2}{*}{\textbf{Supervision}} & \multirow{2}{*}{\textbf{Modality}}  & \multirow{2}{*}{IoU+Point} &\multicolumn{6}{c}{mAP}  \\ 
    	  & & & & & & 0.1 & 0.2 & 0.3 & 0.4 & 0.5  & 0.1:0.5 \\ 
    	\midrule
    	
         CoMMA$\dagger$ \citep{tan2021look}  & S3D &HT250K & Self &VT& 1.02 & 2.18  & 1.72 & 1.11 & 0.93 & 0.37 & 1.26\\
         MIL-NCE \citep{miech2020end} & S3D* &HT100M & Self &VT& 4.67 & 33.94 & 25.16 & 12.65 & 3.42 & 0.41  & 15.11 \\
             %Ours S3D                         & 7.78 & 39.43 & 31.47 & 19.38 & 9.14 & 3.79  & 20.64  \\
             %\midrule
             Ours                   & S3D &HT100M & Self &VT  & \textbf{9.12} & \textbf{42.70}  & \textbf{35.49} & \textbf{25.16} & \textbf{16.22} & \textbf{10.05}  & \textbf{25.92} \\
             \midrule
            GLIP \citep{li2022grounded}   & Swin-L*  & Cap24M & Weak & IT &  1.24 & 2.83 & 2.10 & 1.52 & 0.96 & 0.37 & 1.56 \\
            CoMMA$\ddagger$   \citep{tan2021look} & CLIP &HT100M& Self & VT & 1.68 & 3.51 & 2.32 & 1.88 & 0.99 & 0.40 & 1.82 \\
            CLIP \citep{radford2021learning}& CLIP &HT100M & Self & IT& 3.59 & 29.54  & 22.15 & 9.16 & 2.48 & 0.39 & 12.74 \\
            RegionCLIP \citep{zhong2022regionclip}   & ResNet-101*  & CC3M & Weak & IT &  5.65 & 35.65 & 27.43 & 15.69 & 4.31 & 0.86 &  16.78 \\
            %\midrule
    	   Ours       & CLIP &HT100M & Self &VT  &10.09 & 42.81  & 36.05 & 25.84 & 17.10 & 11.35  & 26.63 \\
               Ours                    & CLIP* &HT100M & Self &VT  & \textbf{11.53} & \textbf{43.64}  & \textbf{36.94} & \textbf{26.78} & \textbf{19.45} & \textbf{14.61}  & \textbf{28.26} \\
               \midrule
               MIL-NCE(temp.)+RegionCLIP(spa.)   &  -  & - & - & VT  & 9.21  & 40.54  & 34.97  & 22.38  &  13.79 & 9.18  &  22.33  \\
    	\bottomrule
    \end{tabular}}
    %\vspace{+0.3cm}
    \caption{\textbf{Spatio-temporal grounding on GroundingYouTube full videos}.   
The proposed model learns global representations encoding global information and spatial correspondences across modalities, achieving a better performance in spatio-temporal evaluation compared to models trained on only spatial or temporal grounding. 
(V: video, I: image, T: text.) $^*$ indicates finetuned backbone.
    % \caption{\textbf{Spatial-temporal localization on full videos}. Our model learned both global representation which encodes temporal information. It also learned spatial correspondence across modalities, which ends up with the best performance in spatial temporal evaluation.
    \label{tab:st_long}
    \vspace{-0.3cm}
    }
    %}
\end{table*}

\section{Experiments}

\subsection{Datasets} \label{dataset}
\noindent \textbf{Training Data:} \textbf{HowTo100M dataset} contains 1.2M instructional videos along with their corresponding automatically generated speech (ASR).
The narrations may be inaccurate and do not always accurately depict the video scene.
%We randomly selected 200K  video clips from the \textit{Food and Entertaining} category for training. %, and thus, we mainly focus on instructional videos in the area of cooking and kitchen tasks. 
%
%\begin{table*}%[htpb]
    \centering
    \small
    \setlength{\tabcolsep}{4pt}
    \resizebox{\textwidth}{!}{
    \begin{tabu}{lr|ccccccccc|ccccccccc}
        \toprule
        \multirow{2}{*}{\bf Method} & \multirow{2}{*}{\bf \#Pairs} & \multicolumn{9}{c|}{\bf FT Retrieval \ \ R@1 / R@5 / R@10} & \multicolumn{9}{c}{\bf ZS Retrieval \ \ R@1 / R@5 / R@10} \\
        & & \multicolumn{3}{c}{MSRVTT} & \multicolumn{3}{c}{DiDeMo} & \multicolumn{3}{c|}{ActivityNet} & \multicolumn{3}{c}{MSRVTT} & \multicolumn{3}{c}{DiDeMo} & \multicolumn{3}{c}{ActivityNet}  \\
        \midrule
        ClipBERT~\cite{lei2021less}  & 5.4M & 22.0 & 46.8 & 59.9 & 20.4 & 48.0 & 60.8 & 21.3 & 49.0 & 63.5 & -& -& -& -& -& -& -& -& -\\
        VideoCLIP~\cite{xu2021videoclip}  & 136M & 30.9 & 55.4 & 66.8 & -& -& -& -& -& -& 10.4 & 22.2 & 30.0 & 16.6 & 46.9 & -& -& -& -\\
        Frozen~\cite{bain2021frozen}  & 5M & 31.0 & 59.5 & 70.5 & 34.6 & 65.0 & 74.7  & -& -& -& 18.7 & 39.5 & 51.6 & 20.2 & 46.4 & 58.5 & -& -& -\\
        ALPRO~\cite{li2022align}  & 5M & 33.9 & 60.7 & 73.2 & 35.9 & 67.5 & 78.8 & -& -& -& 24.1 & 44.7 & 55.4 & 23.8 & 47.3 & 57.9 & -& -& -\\
        VIOLET~\cite{fu2021violet}  & 138M & 34.5 & 63.0 & 73.4 & 32.6 & 62.8 & 74.7 & -& -& -& 25.9 & 49.5 & 59.7 & 23.5 & 49.8 & 59.8 & -& -& - \\
        All-in-one~\cite{wang2022all} & 138M & 37.9 & 68.1 & 77.1 & 32.7 & 61.4 & 73.5 & 22.4 & 53.7 & 67.7 & -& -& -& -& -& -& -& -& -\\
        LAVENDER~\cite{li2022lavender} & 30M & 40.7 & 66.9 & 77.6 & 53.4 & 78.6 & 85.3 &  - & -& -& -& -& -& -& -& -& -& -& -\\
        Singularity~\cite{lei2022revealing} & 17M & 42.7 & 69.5 & 78.1 & 53.1 & 79.9 & 88.1 & 48.9 & 77.0 & 86.3 & 34.0 & 56.7 & 66.7 & 37.1 & 61.7 & 69.9 & 30.6 & 55.6 & 66.9 \\
        OmniVL~\cite{wang2022omnivl} & 17M & 47.8 & 74.2 & 83.8 & 52.4 & 79.5 & 85.4 & -& -& -& 34.6 & 58.4 & 66.6 & 33.3 & 58.7 & 68.5 & -& -& -\\ 
        VINDLU~\cite{Cheng2022VindLUAR} & 25M & 46.5 & 71.5 & 80.4 & 61.2 & 85.8 & 91.0 & 55.0 & 81.4 & 89.7 & 32.0 & 54.6 & 62.0 & 36.9 & 61.7 & 70.5 & 30.9 & 57.0 & 68.2 \\
        \rowfont{\color{Gray}}
        CLIP4Clip~\cite{luo2022clip4clip} & 400M & 44.5 & 71.4 & 81.6 & 42.8 & 68.5 & 79.2 & 40.5 & 72.4 & 83.4 & 31.2 & 53.7 & 64.2 & -& -& -& -& -& -\\
        % \rowfont{\color{Gray}}
        % CLIP-Hhiker~\cite{bain2022clip} & 400M & 47.7 & 74.1 & 82.9 & -& -& -& 44.0 & 74.9 & 86.1 & -& -& -& -& -& -& -& -& -\\
        \rowfont{\color{Gray}}
        CLIP-ViP~\cite{xue2022clip} & 500M & 54.2 & 77.2 & 84.8 & 50.5 & 78.4 & 87.1 & 53.4 & 81.4 & 90.0 & -& -& -& -& -& -& -& -& -\\
        \rowfont{\color{Gray}}
        InternVideo~\cite{Wang2022InternVideoGV} & 646M & 55.2 & 79.6 & 87.5 & 57.9 & 82.4 & 88.9 & 62.2  & 85.9 & 93.2 & 40.7 & 65.3 & 74.1 & 31.5 & 57.6 & 68.2 & 30.7 & 57.4 & 70.2 \\
        \midrule
        \multirow{3}{*}{\Modelname-Base} & 5M & 46.3 & 72.7 & 82.0 & 54.8 & 83.0 & 89.0 & 52.1 & 80.5 & 89.6 & 29.6 & 52.8 & 61.9 & 33.4 & 58.3 & 67.0 & 28.3 & 53.0 & 64.2 \\
        & 17M & 50.6 & 75.4 & 83.5 & 60.8 & 85.1 & 91.0 & 56.1 & 82.5 & 91.2 & 35.5 & 59.3 & 68.6 & 41.9 & 66.7 & 75.0 & 33.8 & 59.1 & 70.4 \\
        & 25M & 51.0 & 76.5 & 84.2 & 61.6 & 86.8 & 91.5 & 58.3 & 83.9 & 91.5 & 35.2 & 57.8 & 66.0 & 41.2 & 65.4 & 74.9 & 35.5 & 60.6 & 71.8 \\
        \hline
        \multirow{3}{*}{\Modelname-Large} & 5M & 53.3 & 76.6 & 83.9 & 59.7 & 84.9 & 90.8 & 58.1 & 85.5 & 92.9 & 33.3 & 58.1 & 66.7 & 34.0 & 60.4 & 68.7 & 31.9 & 60.2 & 72.0 \\
        & 17M & \underline{56.5} & \underline{80.1} & \underline{87.4} & \underline{66.6} & \underline{89.9} & \textbf{93.7} & \underline{66.6} & \underline{88.6} & \underline{94.7} & \textbf{42.6} & \textbf{64.4} & \textbf{73.1} & \underline{46.4} & \underline{70.0} & \underline{78.8} & \textbf{42.8} & \textbf{69.6} & \textbf{79.8} \\
        & 25M & \textbf{58.8} & \textbf{81.0} & \textbf{87.1} & \textbf{70.4} & \textbf{90.1} & \underline{93.5} & \textbf{66.8} & \textbf{89.1} & \textbf{94.9} & \underline{40.7} & \underline{63.4} & \underline{71.8} & \textbf{48.6} & \textbf{72.9} & \textbf{80.0} & \underline{41.9} & \underline{68.9} & \underline{80.3} \\
        \bottomrule
    \end{tabu}
    }
    \vspace{-0.3cm}
    \caption{Comparison to the state-of-the-art text-to-video retrieval methods on MSRVTT, DiDeMo and AcitivityNet.
    \#Pairs denotes the number of pre-training pairs.
    ``FT'' and ``ZS'' refer to the fine-tuning and zero-shot results.
    }
    \label{tab:retrieval}
\end{table*}
%\input{tables/spatio_vhico}

%\vspace{-0.3cm}
\noindent\textbf{Downstream Datasets:} %\textbf{YouCook2}: For the text-to-video retrieval downstream task, we use the common YouCook2 dataset , which provides a human-generated caption for 3.5K video clips for cooking instruction. 
%blah blah ... \hkc{add some details here?}
\textbf{GroundingYoutube (GYT)} is used to evaluate the task of multi-action spatio-temporal grounding as described in Section \ref{sec:dataset:annotation}.
% , we annotated the dense spatio-temporal location information as described in Section \ref{sec:dataset:annotation}.
%for 512 verb-noun phrases. All occurrences of the specific phrase in the test video are hence annotated, allowing us to evaluate spatio-temporal grounding in full untrimmed videos.
\noindent\textbf{MiningYoutube (MYT)} \citep{kuehne2019mining} %: To evaluate the temporal grounding abilities, we leverage the MiningYoutube \citep{kuehne2019mining} dataset, as it 
provides temporal annotation and is limited to the domain of cooking instruction videos. %The dataset features 250 full instructional videos, which are annotated with 512 action classes and temporal boundary information. 
%We use it to evaluate the temporal grounding abilities.
%Here, temporal alignment, the task of finding the right temporal boundaries given the sequences of actions, is used during evaluation to relax the task of temporal detection. 
\noindent\textbf{YouCook-Interaction (YC-Inter)} \citep{tan2021look} is an extension of the YouCook2 dataset \citep{zhou2018towards} for cooking instruction providing bounding boxes for 6K selected frames. The bounding boxes usually comprise the hand and the tool mentioned in the respective sentence-wise annotation. %We evaluate the spatial grounding abilities of models on this dataset.
% \noindent\textbf{YouCook2-Interaction (YC-Inter)}  To evaluate the spatial grounding abilities of our system, we use the YouCook2-Interaction dataset \citep{tan2021look}, an extension of a subset of the YouCook2 dataset \citep{zhou2018towards} for cooking instruction, which provides bounding boxes for 6K selected frames. The bounding boxes usually comprise the hand and the tool mentioned in the respective sentence-wise annotation.    
To further benchmark on general video domains on the \textbf{V-HICO} dataset~\citep{li2021weakly} with 6.5k videos with human-object interaction bounding boxes annotations, 
% that have been semi-automatically curated from sentence captions, 
and \textbf{Daly} action dataset~\citep{weinzaepfel2016human}, featuring videos consisting of daily actions such as ``brushing teeth''.% and ``cleaning windows''.



\subsection{Baseline methods}

%The proposed system is compared to various multimodal methods based on self- and weak supervision: 
\textbf{Temporal}: MIL-NCE~\citep{miech2020end} utilizes S3D~\citep{xie2018rethinking} and word2vec~\citep{mikolov2013efficient}. CLIP~\citep{radford2021learning}, an image-text model with transformer. 
\textbf{Spatial}:
CoMMA~\citep{tan2021look}, SSL model ($\dagger$ for weights shared by the author\footnote{We thank the authors for providing code and weights.} $\ddagger$ trained with CLIP);  
GLIP~\citep{li2022grounded}, RegionCLIP~\citep{zhong2022regionclip}, SOTA weakly supervised grounding model. % trained with image-text pairs.
\textbf{Spatio-temporal}: We construct MIL-NCE+RegionCLIP following the inference pipeline in Figure \ref{fig:inference}. 
TubeDETR~\citep{yang2022tubedetr} and STCAT \citep{jin2022embracing} are supervised. 
More descriptions of the baselines are given in the Appendix \ref{sup:baseline}.
Details of the implementation and experimental settings can be found in the appendix \ref{backbone_and_training}. Inference setups for each baseline are described in Section \ref{inference_sup}.

%% \begin{table}[t]
%     \tablestyle{2pt}{1.05}
    
%     \centering
%     %\resizebox{1\columnwidth}{!}{
%     \begin{tabular}{@{}l|ccccccc}
%     	\toprule
%             \multicolumn{8}{c}{Untrimmed Spatial-Temporal Grounding}
%     	\toprule
%     	\multicolumn{1}{c}{} & \multicolumn{7}{c}{GroundingYouTube}  \\ 
%     	\cmidrule(lr){2-8} 
%     	\multirow{2}{*}{\textbf{Method}}    & \multirow{2}{*}{IoU+Point} &\multicolumn{6}{c}{mAP}  \\ 
%     	                                    &  & 0.1 & 0.2 & 0.3 & 0.4 & 0.5  & 0.1:0.5 \\ 
%     	\midrule
%     	MIL-NCE \citep{miech2020end} & 4.67 & 33.94 & 25.16 & 12.65 & 3.42 & 0.41  & 15.11 \\
%          CoMMA* \citep{tan2021look}   & 1.02 & 2.18  & 1.72 & 1.11 & 0.93 & 0.37 & 1.26\\
%              %Ours S3D                         & 7.78 & 39.43 & 31.47 & 19.38 & 9.14 & 3.79  & 20.64  \\
%              Ours S3D                      & 9.12 & 42.70  & 35.49 & 25.16 & 16.22 & 10.05  & 25.92 \\
%              \midrule
%             CLIP \citep{radford2021learning}  & 3.59 & 29.54  & 22.15 & 9.16 & 2.48 & 0.39 & 12.74 \\
%             CoMMA$\dagger$              & 1.68 & 3.51 & 2.32 & 1.88 & 0.99 & 0.40 & 1.82 \\
%     	   Ours                        & 10.09 & 42.81  & 36.05 & 25.84 & 17.10 & 11.35  & 26.63 \\
%             \midrule
%             GLIP \citep{li2022grounded}      &  1.24 & 2.83 & 2.10 & 1.52 & 0.96 & 0.37 & 1.56 \\
%     	\bottomrule
%     \end{tabular}
%     %\vspace{+0.3cm}
%     \caption{\textbf{Spatio-temporal localization on full videos}. Since our model learned global representations encoding temporal information and spatial correspondences across modalities, it achieves the best performance in spatio-temporal evaluation.
%     % \caption{\textbf{Spatial-temporal localization on full videos}. Our model learned both global representation which encodes temporal information. It also learned spatial correspondence across modalities, which ends up with the best performance in spatial temporal evaluation.
%     \label{tab:st_long}
%     %\vspace{-0.7cm}
%     }
%     %}
% \end{table}
\begin{table*}[t]
    \tablestyle{4pt}{1.05}
    \tiny
    \centering
    \resizebox{2\columnwidth}{!}{
    \begin{tabular}{@{}l|ccccccccccc}
    	\toprule
    	\multicolumn{5}{c}{} &\multicolumn{7}{c}{GroundingYoutube}  \\ 
    	\cmidrule(lr){6-12} 
    	\multirow{2}{*}{\textbf{Method}}  & \multirow{2}{*}{\textbf{Backbone}} & \multirow{2}{*}{\textbf{DataSet}} & \multirow{2}{*}{\textbf{Supervision}} & \multirow{2}{*}{\textbf{Modality}}  & \multirow{2}{*}{IoU+Point} &\multicolumn{6}{c}{mAP}  \\ 
    	  & & & & & & 0.1 & 0.2 & 0.3 & 0.4 & 0.5  & 0.1:0.5 \\ 
    	\midrule
    	
         CoMMA$\dagger$ \citep{tan2021look}  & S3D &HT250K & Self &VT& 1.02 & 2.18  & 1.72 & 1.11 & 0.93 & 0.37 & 1.26\\
         MIL-NCE \citep{miech2020end} & S3D* &HT100M & Self &VT& 4.67 & 33.94 & 25.16 & 12.65 & 3.42 & 0.41  & 15.11 \\
             %Ours S3D                         & 7.78 & 39.43 & 31.47 & 19.38 & 9.14 & 3.79  & 20.64  \\
             %\midrule
             Ours                   & S3D &HT100M & Self &VT  & \textbf{9.12} & \textbf{42.70}  & \textbf{35.49} & \textbf{25.16} & \textbf{16.22} & \textbf{10.05}  & \textbf{25.92} \\
             \midrule
            GLIP \citep{li2022grounded}   & Swin-L*  & Cap24M & Weak & IT &  1.24 & 2.83 & 2.10 & 1.52 & 0.96 & 0.37 & 1.56 \\
            CoMMA$\ddagger$   \citep{tan2021look} & CLIP &HT100M& Self & VT & 1.68 & 3.51 & 2.32 & 1.88 & 0.99 & 0.40 & 1.82 \\
            CLIP \citep{radford2021learning}& CLIP &HT100M & Self & IT& 3.59 & 29.54  & 22.15 & 9.16 & 2.48 & 0.39 & 12.74 \\
            RegionCLIP \citep{zhong2022regionclip}   & ResNet-101*  & CC3M & Weak & IT &  5.65 & 35.65 & 27.43 & 15.69 & 4.31 & 0.86 &  16.78 \\
            %\midrule
    	   Ours       & CLIP &HT100M & Self &VT  &10.09 & 42.81  & 36.05 & 25.84 & 17.10 & 11.35  & 26.63 \\
               Ours                    & CLIP* &HT100M & Self &VT  & \textbf{11.53} & \textbf{43.64}  & \textbf{36.94} & \textbf{26.78} & \textbf{19.45} & \textbf{14.61}  & \textbf{28.26} \\
               \midrule
               MIL-NCE(temp.)+RegionCLIP(spa.)   &  -  & - & - & VT  & 9.21  & 40.54  & 34.97  & 22.38  &  13.79 & 9.18  &  22.33  \\
    	\bottomrule
    \end{tabular}}
    %\vspace{+0.3cm}
    \caption{\textbf{Spatio-temporal grounding on GroundingYouTube full videos}.   
The proposed model learns global representations encoding global information and spatial correspondences across modalities, achieving a better performance in spatio-temporal evaluation compared to models trained on only spatial or temporal grounding. 
(V: video, I: image, T: text.) $^*$ indicates finetuned backbone.
    % \caption{\textbf{Spatial-temporal localization on full videos}. Our model learned both global representation which encodes temporal information. It also learned spatial correspondence across modalities, which ends up with the best performance in spatial temporal evaluation.
    \label{tab:st_long}
    \vspace{-0.3cm}
    }
    %}
\end{table*}

\begin{table*}[h]
    \tablestyle{7pt}{1.05}
    \tiny
    \centering
    \resizebox{2\columnwidth}{!}{
    \begin{tabular}{@{}l| cccc | c |c c| c c | c c }
    	\toprule
    	\multicolumn{4}{c}{} & \multicolumn{1}{c}{ } & \multicolumn{1}{c}{YC-Inter} & \multicolumn{2}{c}{GroundingYT}  & \multicolumn{2}{c}{V-HICO}   & \multicolumn{2}{c}{Daly}\\ 
    	\cmidrule(lr){6-6} \cmidrule(lr){7-8}  \cmidrule(lr){9-10} \cmidrule(lr){11-12}  
    	Method  & Backbone &Data&Super.&Mod.& Acc &  Acc & mAP &  Acc & mAP  &  Acc & mAP \\ 
    	\midrule
        MIL-NCE \citep{miech2020end} & S3D* &HT100M & Self &VT& 23.67  & 27.45  & 8.21 & 12.65 & 11.23 & 13.84 & 24.23 \\
    	CoMMA$\dagger$ \citep{tan2021look} & S3D &HT250K & Self &VT& 48.63   & 47.68 & 23.38 & 40.97 & 21.45 & 54.48 & 33.39 \\
        %\midrule
        Ours                       & S3D &HT100M & Self &VT & \textbf{53.98}   & \textbf{60.62} & \textbf{44.93} & \textbf{44.32} & \textbf{24.31} & \textbf{66.35} & \textbf{45.93} \\
         \midrule
         CLIP   \citep{radford2021learning}            & CLIP&HT100M & Self &IT &    14.10    & 12.50  & 3.49 &  29.23 & 12.51  & 18.02 & 27.28  \\
         CoMMA$\ddagger$  \citep{tan2021look}            & CLIP  &HT100M & Self &VT&   52.65     & 47.56 & 36.42 & 55.20 &  34.54& 61.06 & 44.37  \\
             RegionCLIP   \citep{zhong2022regionclip}            & RN50x4* & CC3M & Weak &IT &   51.56     &   52.84 &  23.42 & 57.92 & 37.82 & 67.12 & 48.62 \\
            GLIP   \citep{li2022grounded}            & Swin-L*&Cap24M & Weak &IT &   52.84      &   53.62 & 24.73 & \textbf{66.05} & 41.17 & - & - \\
            %\midrule
            Ours         & CLIP &HT100M & Self &VT& 57.10    &   55.49 & 43.12 & 60.71& 39.28 & 70.08 & 50.56 \\
            Ours                       & CLIP* &HT100M & Self &VT& \textbf{58.35}    &   \textbf{56.98} & \textbf{45.32} & 62.34& \textbf{41.56} & \textbf{71.35} & \textbf{52.78} \\
            %V-HICO   \citep{}            &  Faster R-CNN &  -      &  - & - & & 67.21 & - & - \\
            \midrule
            {\color{gray}TubeDETR \citep{yang2022tubedetr}}    &  {\color{gray}MDETR} & {\color{gray}Vid-STG} & {\color{gray} Full} & {\color{gray}VT} & {\color{gray}51.63}    &   {\color{gray}53.24} & {\color{gray} 41.76} & {\color{gray}63.23} & {\color{gray}40.87 } & {\color{gray}84.21} & {\color{gray} 62.98} \\
            {\color{gray}STCAT \citep{jin2022embracing}}    &  {\color{gray}ResNet-101} & {\color{gray}Vid-STG} & {\color{gray} Full} & {\color{gray}VT} & {\color{gray}54.47}    &   {\color{gray} 55.90} & {\color{gray}44.21 } & {\color{gray}65.34} & {\color{gray} 41.10 } & {\color{gray}85.42} & {\color{gray} 63.94} \\
    	\bottomrule
    \end{tabular}
    }
    \vspace{-0.2cm}
    \caption{\textbf{Video spatial grounding}. We evaluate the accuracy of the pointing game and the mean average precision. 
    We listed CNN-based methods on top and transformer-based methods in the middle. 
    Models learning global representations (MIL-NCE, CLIP) don't perform well on localization tasks, while our model outperforms other grounding methods. $^*$ indicates finetuned backbone.
    %Models learning global representations (MIL-NCE, CLIP) don't perform well on localization tasks, while our model outperforms other grounding methods. %We listed CNN-based methods on top and transfomer-based methods at the bottom. 
    %(Mod. indicates the modality used, where V: video, I: image, T: text. Super. indicates supervision.)
    %Our method generalized well on both video and image architectures. 
    % Daly GLIP is not workable since every class is action. OOV. V-HICO dataset the CLIP  generalized better to OOV, while word2vec getting worse performance. \bc{maybe we can add supervision: weakly, SSL} \bc{add pretraining data}
    \label{tab:spatial}
    \vspace{-0.5cm}
    }
   
    
\end{table*}

% \begin{table}[t]
%     % \tablestyle{2pt}{1.05}
    
%     \centering
%     %\resizebox{1\columnwidth}{!}{
%     \begin{tabular}{@{}l|cc|cc}
%     	\toprule
%     	\multicolumn{1}{c}{} & \multicolumn{2}{c}{YouCook-Interaction} & \multicolumn{2}{c}{MiningYoutube Grounding}  \\ 
%     	\cmidrule(lr){2-3} \cmidrule(lr){4-5} 
%     	Method  & Acc & IoU   & Acc & IoU \\ 
%     	\midrule
%     	CoMMA* \citep{tan2021look}   & 48.63 & -  & 47.68 & -  \\
%     	MIL-NCE \citep{miech2020end} & 23.67 & -  & 27.45 & -  \\
%     	Ours                        & 48.03 & -  & 47.35 & -  \\
%     	\bottomrule
%     \end{tabular}
%     \vspace{+0.3cm}
%     \caption{Evaluation on spatial-only evaluation using pointing game accuracy and attention heatmap IoU with GT bounding box. Models learning global representation doesn't perform well on localization tasks, while our model maintain comparable performance.
%     \label{tab:spatial}
%     %\vspace{-0.2cm}
%     }
%     %}
    
% \end{table}

\subsection{Downstream Tasks}


%We compare to the SOTA self-supervised method evaluated on spatial \citep{tan2021look} and temporal \citep{kuehne2019mining} grounding.

We considered the following downstream tasks to evaluate spatio-temporal grounding abilities of various models (detailed description is included in the appendix \ref{eval_metric}):

\noindent (i) \textbf{Spatio-temporal grounding in untrimmed video} is evaluated on the proposed Grounding Youtube dataset. The entire video and the respective pool of action instructions were provided. The model needs to localize each action step in time (start-time/end-time) and space (location in the video) as described in Figure \ref{fig:inference}. 
% We evaluate in two metrics: \textbf{IoU+Pointing game} combines spatial grounding~\citep{akbari2019multi} and temporal grounding~\citep{kuehne2019mining} metrics. %For each video frame, the prediction is correct when the model predicts the correct action for the frame. Also, given the predicted action as a query, the maximum point of the heatmap aims to lie within the desired bounding box. We then compute the Intersection over Union (IoU) over all the predictions with the GT to acquire the final score. 
% We also compute \textbf{video mAP} following previous evaluation~\citep{gu2018ava}, where we set IoU threshold between GT and predicted spatio-temporal tubes. A prediction is correct when it surpasses the IoU threshold. We compute the mAP over all classes. %We form a 3D prediction mask following Figure \ref{fig:inference} and compute IoU between our 3D heatmap and 3D tube.
We evaluate in two metrics: \textbf{IoU+Pointing game} combines the evaluation setting from the spatial grounding \citep{akbari2019multi} and temporal grounding \citep{kuehne2019mining} metrics. For each video frame, the prediction is correct when the model predicts the correct action for the frame. Also, given the predicted action as a query, the maximum point of the heatmap aims to lie within the desired bounding box. We then compute the Intersection over Union (IoU) over all the predictions with the GT to acquire the final score. 
We also compute \textbf{video mAP} following previous evaluation \citep{gu2018ava}, where we set IoU threshold between GT and predicted spatio-temporal tubes. A prediction is correct when it surpasses the IoU threshold. We then compute the mAP over all classes. We form a 3D prediction mask following Figure \ref{fig:inference} and compute IoU between our 3D heatmap and 3D tube.

\noindent (ii) \textbf{Spatial grounding} is given a text description to localize the region in the trimmed video. %We use GroundingYoutube, Youcook-Interaction, V-HICO, and Daly for evaluation. %Note that the evaluation is spatial only. It evaluates the results for each frame separately without considering the temporal information. 
It is evaluated using the \textbf{pointing game accuracy}. %Given the query text and video, we compute the attention heatmap on the video as described in Figure \ref{fig:inference}(b). 
If the predicted point lies in the ground truth bounding box, the result counts as a ``hit" and counts as ``miss" otherwise. The final accuracy is calculated as a ratio between hits to the total number of predictions $\frac{\text{\# hits}}{\text{\# hits} + \text{\# misses}}$. 
We also report the mean average precision \textbf{(mAP)} following the settings from V-HICO~\citep{li2021weakly}. %Given a human-object category as the text query, we aim to localize the spatial location in the video frame.
%The predicted location is correct if their Intersection over-Union (IoU) with ground truth bounding boxes is larger than 0.3. 
%Since we do not use any bounding box proposal tools or supervision, we create an attention heatmap as described in Figure \ref{fig:inference}(b) to create a mask for IoU computation. 
%We follow \citep{li2021weakly} and compute the mAP over all verb-object classes.


\noindent (iii) \textbf{Temporal grounding} \label{temporal_grounding}
provides videos with the respective actions and their ordering, including the background. The goal is to find the correct frame-wise segmentation of the video. We follow the inference procedure in \citep{kuehne2019mining} to compute the alignment given the similarity input matrix. The task is evaluated by intersection over detection (IoD), defined as $\frac{G \cap D}{D}$ the ratio between the intersection of ground-truth action $G$ and prediction $D$ to prediction $D$, and the Jaccard index, which is an (IoU) given as $\frac{G \cap D}{G \cup D}$.



\subsection{Comparison with state-of-the-art methods}\label{sota}
\noindent (i) \textbf{Spatio-temporal grounding in untrimmed video:}
We first compare the proposed method with other approaches designed for spatial or temporal grounding in Table \ref{tab:st_long}.
It shows that models without specific loss designs for spatial grounding (MIL-NCE~\citep{miech2020end}, CLIP~\citep{radford2021learning}) show good mAP scores but lower pointing game accuracy. Out of the two weakly supervised methods, GLIP~\citep{li2022grounded} and RegionCLIP~\citep{zhong2022regionclip}), trained with aligned image-text, RegionCLIP show significantly better performance in this setting, while both perform in a similar range in the spatial grounding scenario (see Table~\ref{tab:spatial}). We attribute this behavior to the fact that RegionCLIP distinguishes frames with relevant queries better from background than GLIP, leading to better temporal localization. 
We finally compare the strong baseline MIL-NCE+RegionCLIP, which combines two approaches specialized in temporal and spatial aspects, to our task. 
It shows that the proposed method improves over all other baselines underlining the need to incorporate global (temporal) and local (spatial) representations. 
%Experiments showed that combining a joint objective that learns spatial and temporal information jointly results in better performance than simply applying the best temporal and spatial model. 
% Also, such a combined objective also benefits more when the visual backbone is finetued as well. 
% We construct a split with single action shown in appendix \ref{single_action_stg}.
%Models designed for trimmed videos (CoMMA\citep{tan2021look}) or trained with aligned image-text (GLIP\citep{li2022grounded}, RegionCLIP\citep{zhong2022regionclip}) failed to capture the temporal dynamics, while models without specific loss designs for spatial grounding (MIL-NCE\citep{miech2020end}, CLIP\citep{radford2021learning}) were not able to ground the action in the correct region.
%Note that supervised spatio-temporal grounding approaches~\citep{yang2022tubedetr,jin2022embracing} are not directly applicable in this evaluation since such methods assume the given text query to be ground-truth. %The model must distinguish the correct text query from a pool of action lists. 
%We include an evaluation setting in the supplement where the GT-text queries were provided. \hkc{Do we? If not, we can probably comment the last 2 sentences}
%More experiment setting is in the supplement.

\begin{table}[h]
     \tablestyle{2pt}{1.05}
    
    \centering
    %\resizebox{1\columnwidth}{!}{
    \begin{tabular}{@{}l|ccccc}
    	\toprule
    	%\multicolumn{4}{c}{} &\multicolumn{2}{c}{MiningYoutube}  \\ 
    	%\cmidrule(lr){5-6} 
    	Method   & Backbone &Data & Super. & IoU & IoD \\ 
    	\midrule
    	Mining: MLP \cite{miech2020end} & TSM & MiningYT & Weak & 9.80 & 19.20    \\
             CoMMA* \cite{tan2021look} & S3D-word2vec & HT250K & Self & 2.05 & 5.63    \\
    	MIL-NCE \cite{miech2020end} & S3D-word2vec & HT100M & Self & 18.69 & 26.74    \\
    	Ours                       & S3D-word2vec & HT200K & Self  & 19.18 & 27.65   \\
    	%Ours                       & VAT& S3D-g  & 19.40 & 28.48   \\
            Ours                       & CLIP & HT200K & Self &  \textbf{19.88} & \textbf{28.50}   \\
             %\midrule
            % MCN \cite{chen2021multimodal}      &VAT& R152+RX101   & 23.10 & 32.04    \\
    	\bottomrule
    \end{tabular}
    \vspace{-0.3cm}
    \caption{\textbf{Temporal Grounding on MiningYoutube.} %Spatial-focused model CoMMA is not trained for temporal detection, which results in lower performance, while the proposed model combines global and local representation resulting in better temporal localization than one alone. %\bc{we should include setting without knowing the order}
    %\vspace{-0.5cm}
    \label{tab:temporal}
%    \vspace{-0.4cm}
    }
    %}
\end{table}

\noindent (ii)~\textbf{Spatial grounding: } 
 %We do not report mAP on Youcook interaction since the input is sentence descriptions instead of class.
Second, we compare the performance of the proposed framework to other methods on the task of spatial grounding, including models with weak supervision, as well as models trained in a fully supervised setting in Table \ref{tab:spatial}.
%As shown in Table \ref{tab:spatial}, models trained with global representations such as MIL-NCE and CLIP were not able to localize the text description compared to models learning local representations such as CoMMA, GLIP, RegionCLIP and our approach. 
In the instruction video domain (GYT and YC-Inter), the proposed approach achieves the best result among all weakly and self-supervised trained methods. In the general domain (V-HICO and Daly), the method also achieves competitive results, showing the generalizability of the model to other domains. 
%We attribute this to the transformer architecture in the text branch inheriting knowledge from the open domain during large-scale training, while in contrast the model's performance using word2vec dropped in these datasets. 
Note that in the Daly dataset, the classes are verbs, which are not detectable by the object-focused model GLIP. 
Compared to their weakly trained counterparts, fully-supervised model (TubeDETER~\citep{yang2022tubedetr}, STCAT~\citep{jin2022embracing}) achieve competitive performance in the general domain (V-HICO, Daly) and slightly lower performance in instruction domain (GYT, YC-Inter) due to the domain gap with respect to the training data.
\begin{figure}
       \centering
        \setlength{\tabcolsep}{1pt}
        {\scriptsize
        \begin{tabular}{c c c c c c c }
            { Original } &
            \multicolumn{2}{c}{  } &
            \multicolumn{4}{c}{$\longleftarrow$ Object level variations $\longrightarrow$} \\
            \includegraphics[width=0.185\linewidth]{images/ablation/chair.jpg} &
            \multicolumn{2}{c}{  } &
            \includegraphics[width=0.185\linewidth]{images/ablation/1_only_prompt_mixing/bench.jpg} &
            \includegraphics[width=0.185\linewidth]{images/ablation/1_only_prompt_mixing/stool.jpg} &
            \includegraphics[width=0.185\linewidth]{images/ablation/1_only_prompt_mixing/armchair.jpg} &
            \includegraphics[width=0.185\linewidth]{images/ablation/1_only_prompt_mixing/saddle.jpg} \\
            \multicolumn{3}{c}{  } &
            \multicolumn{4}{c}{ Only Prompt Mixing } \\
            \multicolumn{3}{c}{ } &
            \includegraphics[width=0.185\linewidth]{images/ablation/2_with_self_attn_injection/bench.jpg} &
            \includegraphics[width=0.185\linewidth]{images/ablation/2_with_self_attn_injection/stool.jpg} &
            \includegraphics[width=0.185\linewidth]{images/ablation/2_with_self_attn_injection/armchair.jpg} &
            \includegraphics[width=0.185\linewidth]{images/ablation/2_with_self_attn_injection/saddle.jpg} \\
            \multicolumn{3}{c}{  } &
            \multicolumn{4}{c}{ + Attention-Based Shape Localization } \\
            \multicolumn{3}{c}{ } &
            \includegraphics[width=0.185\linewidth]{images/ablation/3_background_blending/bench.jpg} &
            \includegraphics[width=0.185\linewidth]{images/ablation/3_background_blending/stool.jpg} &
            \includegraphics[width=0.185\linewidth]{images/ablation/3_background_blending/armchair.jpg} &
            \includegraphics[width=0.185\linewidth]{images/ablation/3_background_blending/saddle.jpg} \\
            \multicolumn{3}{c}{  } &
            \multicolumn{4}{c}{ + Controllable Background Preservation } \\
        \end{tabular}
        }
    \vspace{1mm}
    \captionof{figure}{
    Ablating our full object variations pipeline. Original image was crated using the prompt ``A \emph{chair} with a dog on it''. 
    }
    \vspace{-10pt}
    \label{fig:ablation}
\end{figure}

\section{Visualization On Demand} %Visualization Elements
\label{sec:visrisk}
Based on environment data and trajectory evaluation, we now present ways of communicating the situation and risks on a visual display to achieve an ADAS.
In this context, we employ a renderer that visualizes all the information in a joint Cartesian coordinate system (see section \ref{subsec:sim}). 
Once driving risks are detected, design elements are overlayed on the display with section \ref{subsec:active} and section \ref{subsec:warning}. 

\subsection{Simulator Environment}
\label{subsec:sim}
Nodes of the R-LDM have a range of potential attributes, such as the 3D position or geometrical shape of objects. 
% For instance, the road centerline is a polyline with bounderies to the left and right. Crosswalks have a defined width and buildings a polygonal outline description. 
In the renderer, we always visualize static and quasi-static data that lie in the field of view from the ego vehicle. 
For this, a local 3D model is generated by converting geographic points with (lat, lon, alt) into Cartesian coordinates of (x, y, z). 
% and project the positonal relations from a view perspective with a transformation matrix. 
Fig. \ref{fig:3Dsimulator} depicts an exemplary map section having several intersections in bird's-eye view.
% with several intersections, stop lines and crosswalks. 
On the top right, the first person view of a vehicle approaching a crosswalk is shown. 

The dynamic data is then added to this static view. A zoomed-in excerpt from the map is given at the bottom of Fig. \ref{fig:3Dsimulator} that includes a recorded GNSS trace (red).
We project the trace onto the connected lane center, which is pictured in green. 
% Because we project the ego position on the closest lane segment, on the bottom right the measured trace is changed in red and the aligned trace is marked in green.
Consequently, the virtual horizon and its possible paths are retrieved as described in section \ref{subsec:ldm}. 
We can lastly update and move the excerpt with the current position from the GNSS to obtain a live simulation.

\subsection{Proactive Support}
\label{subsec:active}
Communication of spatial as well as spatio-temporal relations is crucial for risk-averse driver support. 
% This has the reason that humans can estimate the time better than positions (especially for risks). 
% Velocity contains implicitly the time as well. 
Further sources of information are cause, likelihood and severity of a potential risks.  
% if a collision happens. 
The next step for RNS is the choice of suitable design elements. 
In this process, we suppose that we know where the ego vehicle is driving (i.e., the ego path) from its navigation route. 
Yet, for surrounding vehicles, all paths are considered.

\subsubsection{Hazard Route Element}
The so-called hazard route in Fig. \ref{fig:charts} is a concept that consists of a scale portraying distances to an upcoming risk element.
Furthermore, the geometrical area or length of risks is considered.
Risk is thus measured with respect to the ego path, ranging from the current position  $\Delta l \hspace{-0.03cm}=\hspace{-0.03cm} \unit[0]{m}$ to the end of the path $\Delta l_{h}$.
Here, the length $\Delta l_{h}$ can be chosen according to own preferences. 

At an upcoming intersection, risk is defined by the section of the path that lies within the junction.
Since risk corresponds to exposition time, we encode the path part from the intersection $I_z$ with a color, ranging from green for short intersections to red for long ones. 
%allgemein risiko entlang des pfades zu intersection zone
%share of junction segment to navigation route + 
%one case with large intersection far and one case with small intersection close
Fig. \ref{fig:charts}~a) gives two examples of the hazard route.
The left bar shows a large intersection (e.g. multi-lane four-way stop) in vicinity and the right bar has a small and consecutive medium junction. 
% In the case of collision risk, the intersection zone $I_z$ can be used.
% Depending on the value of $I_z$ (low, medium and large), the area is marked from green, to yellow until red for conveying the criticality. 
This emphasizes that we may include more than one intersection in our warnings.

\begin{figure}[t]
  \centering
  \includegraphics[width=0.95\linewidth]{./img/simulator.png}
  \caption{Rendered road network from two perspectives with the ego position being projected on the navigation route. \vspace{0.45cm}}
  \label{fig:3Dsimulator}
\end{figure}

\begin{figure}[t]
  \centering
  \resizebox{\linewidth}{!}{
  \import{img/}{velocity_scale_new.pdf_tex}}  
  \caption{Chart elements for proactive support. Hazard route (left) and velocity scale (right).} %\vspace{-0.3cm}}
  \label{fig:charts} 
\end{figure} 

\subsubsection{Velocity Scale Element}
The velocity scale, Fig. \ref{fig:charts}~b), is a second chart element which qualifies the difference between the current velocity of the vehicle $v_0$ and the target velocity $v_{\text{tar}}$ from the trajectory evaluation of section \ref{subsec:trajeval}. 
The scale shows possible velocity values, from standstill $v\hspace{-0.05cm}=\hspace{-0.05cm}\unit[0]{m/s}$ to a maximal velocity $v_{\text{max}}$. Depending on the difference $|v_0 \hspace{0.05cm} - \hspace{0.05cm} v_{\text{tar}}|$, the situation is rated as safe with $v_0 \hspace{-0.042cm} \approx \hspace{-0.042cm} v_{\text{tar}}$ (green, left), as dangerous with e.g. $v_0 \hspace{-0.05cm} < \hspace{-0.05cm} v_{\text{tar}}$ (yellow, middle) to critical with $v_0 \hspace{-0.07cm} \ll \hspace{-0.07cm} v_{\text{tar}}$ (red, right). The same cases hold true for the opposite circumstances, i.e., $v_0 \hspace{-0.032cm} > \hspace{-0.032cm} v_{\text{tar}}$. 
This velocity scale can be employed for curve or regulatory risks. 
Moreover, we may set an enforced speed limit as the target velocity $v_{\text{tar}}$ for proactive behavior, once there is no risk ahead. 
%\noindent -Warning vs behavior support \\
%-Ghost vehicle as in game \\

\subsection{Short-Term Warning Elements}
\label{subsec:warning}
In order to emphasize the criticality of the situation, we propose to add further intuitive warning elements as e.g. pop-up signs and lane colorings. 
The following elements augment the proactive elements.

\subsubsection{Pop-up Signs}
Explicit symbols indicate the risk cause accompanied with the event time for collisions ($s_E$), distances to the risk spot for turns (i.e., right curve with $d_r$ and left curve with $d_l$) or stopping distance for crosswalks ($d_c$). In Fig. \ref{fig:popups}~a), the pop-up signs are pictured. 
% Besides the velocity difference, the risk type is an indication for the severity of the situation.
%Examples for collision risk are car-to-car crash., curve risk can be  as a single-car accident and regulatory risks will be a car-to-object collision. 
We want to stress that this is just a selection and more risk causes can be added. 
The purpose is also to clarify the reason for the warning and give more human-understandable information.

\subsubsection{Colored Events}
Finally, we highlight lane parts or positions according to the corresponding risks.  
% the determined color rating from the hazard route and velocity scale and relate the risks to the simulator environment. 
In the instance of curve and regulatory risk, the lane is colored from the ego position up to the point of maximal risk. 
For collision risk, we mark the point of the closest encounter as a red cube.
An illustration for regulatory risk induced from a stop line is depicted in Fig. \ref{fig:popups}~b). Again, the color is defined by the deviation $|v_0-v_{\text{tar}}|$. It also shows the therein considered navigation route with length $\Delta l_h$ and another unlikely path. 

It should be noted that the visualization of warnings only occurs if the risks are actually present. 
%\textcolor{red}{improve language, repeat intersection zone and navigation route}
%eingrauen unlikely paths and navigation path and describe in text, maybe delete Iz -> put line from unlikely path to green arrow
Altogether, the RNS provides a variety of tools to analyze and circumvent critical situations in intersection scenarios, while not overloading the driver's awareness.

\begin{figure}[t]
  \centering
  \resizebox{\linewidth}{!}{
  \import{img/}{colored_lane_new.pdf_tex}}  
  \vspace{-0.53cm}
  \caption{Short-term warning elements. Selected pop-up warnings (left) and colored lane (right).}
  \label{fig:popups} 
\end{figure} 



\noindent (iii)~\textbf{Temporal grounding:}
We evaluate temporal grounding in Table \ref{tab:temporal}. Here, it shows that global representations also profit from local representation learning.%, achieving state-of-the-art results in temporally localizing actions in untrimmed videos. 
This hypothesis is further validated in the ablation studies in Table~\ref{tab:train_ablations}, where we ablate both losses for all three settings and show a consistent improvement in the joint loss formulation. 

%Our model achieved comparable results with 
%Called action step localization. Evaluated on Mining Youtube. 




%\input{tables/spatial_temporal_clip}






% \noindent (iv) \textbf{Spatio-temporal Clip :}
% \label{ST_clip}
% Following the current spatio-temporal datasets \citep{jiang2014thumos,gu2018ava} which aim to discriminate the action class from the background class in a short clip, we construct a clip level evaluation where the clip varies from 9 sec to 60 Section  Given an action step, we append the video segments before and after the steps with the same time length of the action step to form the final video clip. This results in 2,895 clips for the spatio-temporal clip grounding evaluation.
% For each clip, the  temporal action intervals occupy 33\% of corresponding videos, which demonstrates the difficulty of the setting. As shown in Table \ref{tab:st_clip}, we observe a similar trend as the full video evaluation where our model outperforms all the baselines. 




\subsection{Ablation study} 
%\vspace{-1mm}
We perform ablation studies with respect to all three settings, spatio-temporal grounding, as well as spatial and temporal grounding alone, reporting performance for spatio-temporal grounding on GroundingYT using mAP with IoU@0.4, on temporal grounding using MiningYT IoU, and on spatial grounding using YC-Inter. pointing game. Additional ablation are in appendix \ref{ablation_sup}. %For each setting, we use the same feature extractor for three modalities as described in Sec 4.1 for a fair comparison. 

% add summary here?
% as they are the less computational evaluation tasks.
%This subset of downstream tasks has been chosen for their simplicity of evaluation and because they cover a wide range of tasks.

\noindent\textbf{Frame selection strategy.} 
We perform an ablation on the possible frame selection strategies for our method (Figure \ref{fig:pipeline}(b) and Section \ref{sinkhorn_main}). In Table \ref{tab:frame_ablations}, \textit{None} uses all frames within the ASR boundary ($U=T$) as our video training data. 
\textit{Global} represents the [CLS] token in text and video. \textit{Local} uses the words and spatio-temporal tokens. In the setting Sinkhorn was not applied, the top $T$ frames with the highest similarity score were selected. When we set spatio-temporal tokens as the selection target, we sum over the scores with respect to each frame to acquire the frame similarity score.
%\textit{Global} utilizes the global sentence resp. frame [CLS] token as the query to rank the top $T$ similar frames as the selected frames for training. \textit{Local} uses the words resp spatial-temporal tokens instead of the CLS token as a query and selects the frames with the closest feature distance. 
It shows that selecting frames based on Sinkhorn selection leads to consistently better results as it enforces more variety of visual concepts but also captures frames with possible groundable objects. It further shows that word tokens are more suitable than the global text CLS token for frame selection. Finally, we see that depending on the task (spatial vs. temporal), a local resp. global representation is better, and a combination of both works best for spatio-temporal grounding. 
%, which improves overall performance.%, leading to better supervision.
We provide runtime analysis of such frame selection strategy in the appendix \ref{runtime}.
% \noindent\textbf{Number of frames for training.} We tested different video lengths $T$ used for training. As shown in Table \ref{subtab:ablations2}, selecting less frames for training significantly causes the performance to drop. We hypothesize that not only does the model fail to capture the temporal dynamics with less frames, but loses some frames with groundable objects in the sentence while training. We also found that when the number of frames increases, more irrelevant frames might be selected during training, which decreases the performance.
\begin{table}[!t]
  \centering\small
  \caption{%
    Ablation study on dual-form approximate rank loss.
  }
  \vspace{-3pt}
  % \renewcommand{\arraystretch}{0.8}
  \setlength{\tabcolsep}{2.4mm}{
    \begin{tabular}{l|cccccc}
    \toprule
    \multirow{2}{*}{Loss} & \multicolumn{2}{c}{\textbf{IoU = 0.1}} & \multicolumn{2}{c}{\textbf{IoU = 0.3}} & \multicolumn{2}{c}{\textbf{IoU = 0.5}} \\
    & R@1 & R@5 & R@1 & R@5 & R@1 & R@5  \\
    \midrule
    $\mathcal{L}_{bce}$  & 0.05  & 0.51 & 0.01 & 0.10 & 0.00 & 0.01 \\
    $\mathcal{L}_{nce}$  & 5.26  & 13.65 & 4.09 & 10.90 & 2.32 & 6.73 \\
    $\mathcal{L}_{ar}$   & 10.08  & 22.02 & 8.15 & 18.47 & 4.80 & 12.04 \\
    \midrule
    $\mathcal{L}_{dar}$  & \textbf{11.03}  & \textbf{22.99} & \textbf{8.83} & \textbf{19.48} & \textbf{5.23} & \textbf{13.18} \\
    \bottomrule
    \end{tabular}
  }
  \vspace{-8pt}
  \label{tab:ablation_loss}
\end{table}

%\vspace{-0.1cm}
\noindent\textbf{Global and local loss.} As mentioned in the spatio-temporal evaluation, both features contribute to the final grounding result. We test the model by ablating out each loss. 
Table \ref{tab:train_ablations} shows that each loss not only contributes to the spatio-temporal grounding on the GYT, but also that the whole is more than the sum of its parts (losses) since this task requires both spatial and temporal detection. The reduced impact of the global loss in the case of YC-Inter is that this is a pure spatial grounding dataset (no background frames) without temporal detection, and the local loss plays a more critical role. We observe the same patterns in the temporal grounding result for MYT, where spatial localization is not directly contributing to the final performance. We tried out the same ablation using in the S3D backbone in supplement.
%We provide runtime analysis of different losses in the appendix \ref{runtime}.
%By comparing the results for spatio-temporal grounding in untrimmed videos (Table 1) vs. spatial grounding in trimmed videos (Table 3),  we can further see the impact of the proposed joint representation.

%\bc{to appendix}




% adding the global loss improves the ground performance. This results also shows that spatial grounding benefits from global representation learning. In the spatio-temporal setting, the performance without a global or local loss outperforms other baselines.

% \noindent\textbf{Dataset for training.} As mentioned in Section \ref{dataset}, we trained models with data with food categories. In Table \ref{subtab:ablations4}, we also tested our model trained with a larger set of food and entertaining called HowTo370K used in \citep{han2022temporal}. The full set of HowTo100M contains a total of 1M long videos, which is five times the size of our dataset. We found training with our 200K videos reaches similar performance with much less training hours.

% \noindent\textbf{Affect of audio in training and testing.} Unlike text which describes a discrete concept as a target to ground, audio serves as a continuous representation that is highly relevant to the temporal information. For example, we can determine an action started when we hear a ``cracking'' sound. In Table \ref{subtab:ablations5}, we tested our model using the additional audio modality by expanding our architecture and loss from VT to VAT. We found when training and testing with audio, the spatio-temporal result increases while the spatial-only result remains the same. This validates our assumption that audio contributes more to temporal understanding. When we trained on audio and tested without audio, the performance increases over the VT model, showing that the audio serves as useful supervision for better video/text representations. More details are presented in the supplement. 

\subsection{Qualitative results}
\vspace{-1mm}
We visualize our spatio-temporal result in Figure \ref{fig:visualization}. For the GLIP model, we output the bounding box with the highest confidence score and visualize its center point. We found GLIP model focuses on the salient object while our model focuses more on human-object interaction.


% \section{Discussion}
In this section, we summarise the lessons learnt from our MMLA in-the-wild deployment; then discuss the implications of these findings for practice, identify various limitations of our in-the-wild study, and suggest some potential directions for future research and development.

\subsection{Summary of lessons learnt}

This paper provides a summary of some of the key logistical, privacy and ethical challenges that emerged from our complex MMLA, in-the-wild study. These can be listed as follows: \hfill \break


\emph{Space and place}
\begin{itemize}
\item \textbf{Intrusiveness} -- While students did not report discomfort in wearing sensors, teachers can still get concerned about their potential \textit{distracting factor} and some students can feel \textit{stressed} about being monitored. 

\item \textbf{MMLA Technology readiness} -- The lack of MMLA technology readiness can severely impact the lesson plan. Teachers need to play an active role to create \textit{strategies to moderate} the sensing/analytics technologies, and minimise potential disruptions and setup time.  

\item \textbf{Unexpected issues during the MMLA deployment} -- While several technical issues that can emerge during the MMLA deployment are beyond the control of the research team, reducing the number of devices used can minimise potential technical failures. Some high-end sensors may need to be replaced with less expensive sensors, that may capture coarser data, if the change increases \textit{reliability}. 

\item \textbf{Multimodal data quality, portability of sensors and affordability} -- At least currently, a trade-off may exist between capturing \textit{high quality} data and the portability and affordability of the sensing technology.
\end{itemize}

\emph{Technology: data and analytics}
\begin{itemize}
\item \textbf{Purpose of capturing multimodal data} -- If communicated clearly, students are willing to participate in a complex MMLA study and contribute their data for the purpose of helping their teachers or future students. Teachers can and need to develop strategies to optimise the use of multimodal data to support students.  

\item \textbf{Multimodal data incompleteness and trustworthiness} -- Although multimodal data is required to build analytical representations of an embodied learning experience, multimodal sensor data are intrinsically incomplete and subject to bias. Thus, mechanisms to ensure MMLA systems are \textit{trustworthy} and designing for data incompleteness are required. 

\item \textbf{Emerging issues related to visualising multimodal data} -- Teachers need to be supported to develop relevant \textit{data literacy skills} to understand the basic inner-workings of specific MMLA systems and for them to develop pedagogical \textit{strategies around the effective use} of the intrinsically complex MMLA visual interfaces. Students may also require visualisation guidance or explanatory features for them to the meaning of the data in educational terms.
\end{itemize}

\emph{Design: human-centredness}
\begin{itemize}
\item \textbf{Human-centred MMLA and students' learning} -- Teachers' appreciation of partnering with researchers in the design process can lead to creating MMLA systems aligned with teaching practices and learning goals. 

\item \textbf{Human-centred MMLA and research innovation} -- Involving teachers and students in the design process contributes to the validation of the MMLA interfaces according to the learning design and to the improvement of the logistics of the MMLA research study. 
\end{itemize}

\emph{Social factors}
\begin{itemize}
\item \textbf{Consenting and participation strategies} -- It is challenging to explain to students what a complex MMLA study entails. Providing too many technical details about the sensors and the analytics in advance does not necessarily contribute to clarity. Explaining the complexity of the MMLA deployment \textit{in person} can enable students to ask clarification questions and then provide informed consent.  

\item \textbf{Data privacy and sharing} -- Students were willing to share their multimodal data with others if their privacy is preserved and the purpose is limited to supporting learning. While most students see their multimodal data as only beneficial to themselves, some students can see the potential benefit to make their data available to other students to learn from their experiences or for teachers to improve the design of the learning tasks. 
\end{itemize}

\emph{Sustainability}
\begin{itemize}
\item \textbf{Technological sustainability} -- A potential strategy to maximise long-term technical sustainability is a lightweight \textit{microservices-based architecture} that can enable attaching and detaching heterogeneous sensors as required.

\item \textbf{MMLA appropriation in the classroom} -- A potential strategy to maximise adoption and technology appropriation includes embedding sensing capabilities into the classroom, providing a high degree of user control, providing training to teachers on system usage and data interpretation, and keeping the need for support from a technical actor to a minimum extent.
\end{itemize}

\subsection{Implications for practice}
The lessons learnt from our in-the-wild MMLA study have several implications. We summarise these into the following three recommendations to provide guidance for researchers, developers and designers to make informed decisions about the effective deployment of MMLA in-the-wild. 

\textbf{\textit{Forging design partnerships with teachers and students}.} The more sensors are used to capture activity in complex educational scenarios that involve non-computer mediated interactions, or ill-defined, open tasks such as in teamwork, the more complex the meaning-making process becomes to move from data to insights  \citep{echeverria19towards}. Thus, as rich data infrastructures become more commonplace in educational contexts \citep{guzman2021learning}, it is also becoming critical to forge strong partnership relationships among teachers, students, educational decision-makers, researchers and developers. This has the potential to ensure that algorithmic outputs and data representations are meaningful and aligned to local learning objectives and pedagogical values \citep{Ahn2019}. Indeed, some educational researchers have started to utilise the body of knowledge and practice from design communities, such as participatory design and co-design, in data-intensive educational contexts \cite{BuckinghamShum2019}. However, following human-centred design approaches is yet to be seen in MMLA according to the most recent review \citep{yan2022scalability}. 

In our study, several practical challenges in the MMLA deployment demanded expertise from a wide range of areas (such as learning analytics, interaction design, and information visualisation), plus knowledge from stakeholders contributing insights and evidence from their lived experiences. By giving an active voice to students and involving teachers in the design process we were able to identify the key practical challenges that can easily undermine adoption if they are not addressed in a timely manner. Teacher/student involvement was also critical to give meaning to the complex multimodal data streams both for research purposes, and to design the MMLA dashboard aimed at end-users. An indicator of the success of the teachers' partnering experience, is that once they reflected on the value of the MMLA deployment, they wanted to move the deployment to happen as a part of their regular classes, potentially making the transition from research to practice an immediate possibility. 

Yet, much work is still required to develop specific guidelines to create human-centred MMLA systems. For example, the rapidly growing human-centred AI \citep{shneiderman2021human} movement within and beyond HCI has much to offer to the design and development of MMLA systems to ensure that novel AI tools are effectively in service of students and teachers. Moreover, researchers and developers may want to address the complexity of visual interfaces of multimodal data by grounding their designs in key Information Visualisation principles aimed at scaffolding the interpretation of large amounts of data by non-technical users (e.g., by applying data visualisation guidance \citep{ceneda2016characterizing} or data storytelling \citep{martinez20} principles).  


\textbf{\textit{Designing MMLA considering data imperfection and teacher control}.} 
% Depending on the context, empatica may work when students do not move a lot, otherwise, we need more feasible sensors/devices
In Jeffrey Heer's view \citep{heer2019agency}, \textit{"AI methods can be applied to helpfully reshape, rather than replace, human labor"}. In our study, the ultimate aim is not to replace the teacher but augment their repertoire of tools they can use to support students' reflective thinking through data interfaces. Yet, the data captured from the physical world through sensing devices are often incomplete, noisy, and unreliable \citep{bamgboye2018towards}. Moreover, beyond the use of multimodal data in education, it has been  reported that there is commonly a disconnection between logged data and higher-order educational constructs \citep{echeverria19towards, mangaroska2018learning}. This means that the design of effective MMLA interfaces needs to deal with data incompleteness and partial models of the actual learning activity. Creating MMLA systems that perform fully automated actions based on these incomplete data can thus be risky, and cannot be recommended at this level of MMLA maturity.

A primary finding from our MMLA in-the-wild study is that teachers see that a key requirement to maximise the sustainability of the complex computational system is to provide a high degree of user control. The debate around the balance between human agency and AI automation is not new in HCI \citep[e.g.][]{shneiderman1997direct}, yet, it is nascent in the context of MMLA. Nonetheless, \citet{Ogan19} suggested that once sensing technologies mature to the extent that they enable capturing a variety of behaviours in the classroom, we should let teachers empower themselves to use data for making informed decisions and improving their own classroom practices. 

Moreover, we learnt that if the MMLA interface does not provide any visual cue about potential data incompleteness, both teachers and students can attempt to make potentially misleading inferences from the data. More problematically, decisions can be made and actions can be taken without sufficient recognition that logged student data is, by definition, imperfect \citep{Kitto18Imperfection}. In the long term, this can damage their trust in the system. 

Future work can consider at least two potential ways to address these challenges. First, as suggested by some of the teachers in our study, it may be possible to identify gaps in teachers' knowledge around the use of data in their practice such as whether they are aware of how the multimodal data are collected, what educational constructs are being modelled, the limitations of algorithmic outputs, and the kinds of insights that can be derived from them. Professional development programs can be created to increase teachers' AI literacy \citep{long2020ai} and visualisation literacy \citep{pozd2023} for them to understand, to some extent, how they can integrate the MMLA interfaces into their existing practices or how they can adapt their current practices to the new possibilities enabled by the use of such multimodal data. Alternatively or in parallel, the teachers in our study also suggested that the MMLA user interface can be designed to provide visual cues that alert teachers about the reliability of the data so they can make informed data interpretations or decide not to use the MMLA system for a session with uncertain data. To address this, researchers and developers of this kind of innovations may want to consider elements from the emerging literature on the human aspects of AI explainability \citep{JIANG2022102839,khosravi2022explainable} to design MMLA systems that, for example, reveal their assumptions and biases in ways that make sense to non-specialist users so they can keep in control of the potential pedagogical actions that can be taken \citep{selwyn2019s}. 



\textbf{\textit{Ensuring teachers' and students' safety}. }
Enhancing physical learning spaces with rich sensing capabilities unavoidably raises critical questions about the potentially harmful effects of excessive surveillance and potential threats to students' and teachers' privacy rather than supporting learning. Preserving human safety in increasingly autonomous smart environments has been identified as one of the main HCI grand challenges \citep{stephanidis2019seven}. \citet{selwyn2019s} explains that even learning analytics systems intended to only support students' learning run the risk of being utilised for broader purposes: \textit{"the concern here lies with the secondary (re)uses of learning analytics data by institutions and other `third parties'”} (p.3). Multimodal learning data can raise particular concerns since analysing a combination of on-skin and under-skin sensor data can lead to richer user models that could be used for student profiling or for performance measurement of teachers which may have negative consequences for the individuals concerned \citep{selwyn2018doing}. Unfortunately, the ethical implications of using MMLA systems have been seldom mentioned in the literature, as has been flagged in recent scoping works \citep{cukurova2020promise,worsley2021new} and reviews \citep{Alwahaby2022,crescenzi2020multimodal}. 

 Our findings flagged some further concerns. Teachers and students may not easily grasp all the potential ways in which their data can be exploited. Yet, they had sufficient awareness to confirm that their data should only be used by themselves or by other educational stakeholders to support other students. Strict guidelines about data privacy and data ownership should be established for systems that use students' multimodal data since some of these data can be highly sensitive. For example, designers could explore ways in which end-users can indicate to the MMLA system to forget their multimodal data totally or partially after it has been used for educational purposes \citep{Muller2022}. Visualising multimodal data also raised another set of potential concerns. In our second iteration, students' inclinations to participate in a MMLA study changed as they seemed to be more willing to participate in a study that only involved data collection but were not sure about all the implications related to having a user interface showing their data in front of their classmates. In this regard, future MMLA work aimed at closing the learning analytics loop by providing end-user data interfaces would benefit from building upon the long-standing HCI research focused on designing for sharing personal data through group interfaces \citep{greenberg1999pdas}. Moreover, although some preliminary work has attempted to discuss ways to effectively write up consent forms for MMLA studies \citep{beardsley2020enhancing}, further work is needed to understand how students can make informed decisions regarding their participation in MMLA studies or in terms of data sharing based on the types of data used in a particular MMLA innovation. 

%One suggestion along these lines is to give students ownership of their own data — what could be termed “personal data sovereignty” (Jarchow & Estermann, 2015). 

\subsection{Limitations}
Our study has various limitations. First, the lessons learnt are not generalisable as MMLA studies cannot be treated as a generic type of analytics. Our study involved the use of video, physiological wristbands, audio, and indoor positioning sensing. Although these cover most types of sensors used in  MMLA studies \citep{yan2022scalability}, students' and teachers' perceptions towards sensing technologies can vary across learning situations and technical setups. For example, in other studies where laboratory-grade EEG headsets have been worn by students, their perceptions towards potential negative effects related to sensor intrusiveness have been more prominent compared to those of the students in our study \citep{mangaroska2021challenges}. 

A second limitation is that the teachers and students in our study were, to some extent, accustomed to technology-equipped learning spaces, such as the simulation rooms. Thus, our MMLA sensors were added to an existing ecology of devices and educational practices that involve the use of technologies of various kinds. Nonetheless, most of the existing technologies are not used for the purpose of monitoring and data-intensive reflection thus the lived experiences of the educational stakeholders were novel in relation to the MMLA innovation. 

A third limitation is that the students who participated in the study and the interviews were those who were more willing to participate and often highly motivated as participation was optional. We could not interview participants who were less inclined to experience the MMLA study which prevented us from gaining a deeper understanding of the factors considered by non-consenting students or potential further concerns about the deployment. 

Besides the comments from students and teachers, we also reported some of the lessons learnt from a researcher's perspective with the aim of sharing the particular experiences and insights we gained from this in-the-wild experience. Readers are encouraged to interpret these as such rather than as generalisable claims. 

Finally, evidence was captured heterogeneously from iterations 1 and 2 of our study (e.g., students were interviewed about intrusiveness in iteration 1 but not in iteration 2). This was a consequence of conducting the study under authentic conditions in which the research aims adapted to the needs and availability of the teachers, the students and the planned educational activities. Yet, we did not want to challenge these to preserve the in-the-wild nature of the study.


\noindent
\section{Dense 3D Vision Tasks}
\subsection{Monocular Depth Estimation}
\label{sec:extra_evaluation}
Following the results on monocular depth estimation in the main paper, we describe the implementation details of the training, show additional results on different scenes and provide additional metrics on different test scenes. 

\paragraph{Implementation Details}
For all our depth estimation experiments, we use PyTorch~\cite{paszke2017automatic} and train for 20 epochs for comparability using Adam~\cite{kingma2014adam}.
Monocular approaches are trained with a batch size of 12 on one NVIDIA RTX-3090 GPU.
We chose $\lambda_{\text{s}}=10^{-3}$ and sample $S$ with $T=10$ frames offset due to small relative camera movement between frames and the high frame rate.
The RGB inputs are scaled to $480 \times 320$ for supervised training and to $320 \times 160$ for self-supervised training, respectively. 
The depth network regresses dense depth predictions on four pyramid levels, each with half the resolution of the previous.
Pose network and augmentations follow~\cite{monodepth2}.
We choose an initial learning rate of $1 \times 10^{-4}$ for 15 epochs, which we decrease to $1 \times 10^{-5}$ after 15 epochs in the self-supervised setting.
For the supervised case, we start with a learning rate of $1 \times 10^{-3}$, which we decrease every five epochs by a factor of ten.


\begin{table*}[!ht]
\centering
\caption{\textbf{Depth prediction comparison when training with different modalities and tested on different unseen scenes and seen scenes. }(Top) Evaluation against GT of depth predictions on the test set with dense supervision from different depth modalities. (Bottom) Predictions evaluated on respective modality. 
Error is reported as Sq.Rel. and RMSE in mm.
}
\begin{adjustbox}{}
\footnotesize
\resizebox{1.0\textwidth}{!}{
\begin{tabular}{ll|cc|cccc|cccccc}
\shline
& Mask&\multicolumn{2}{c|}{Full Scene} & \multicolumn{2}{c}{Background} & \multicolumn{2}{c|}{All Objects}  & \multicolumn{2}{c}{Textured} & \multicolumn{2}{c}{Reflective} & \multicolumn{2}{c}{Transparent} \\
\hline
&Metric&Sq.Rel. & RMSE&Sq.Rel. & RMSE&Sq.Rel. & RMSE&Sq.Rel. & RMSE&Sq.Rel. & RMSE&Sq.Rel. & RMSE \\
\shline
\multirow{3}{*}{\rotatebox[origin=c]{-90}{Test 1}}     & I-ToF & 
24.78  &  148.09
 & 
22.25  &  151.07
 & 
29.62  &  123.19
 & 
16.47  &  99.08
 & 
102.79  &  214.60
 & 
44.29  &  134.44
\\
                                & D-ToF & 
24.23  &  151.72
 & 
23.74  &  159.28
 & 
22.85  &  110.88
 & 
16.22  &  101.12
 & 
57.14  &  148.61
 & 
30.23  &  107.23
\\
                                & Active Stereo & 
32.15  &  173.72
 & 
33.84  &  184.16
 & 
22.23  &  116.57
 & 
19.55 &  114.07
 & 
64.27  &  167.71
 & 
12.92  &  69.49
\\
\shline\\
\shline

\multirow{3}{*}{\rotatebox[origin=c]{-90}{Test 2}}     & I-ToF & 
27.42  &  123.79
 & 
22.66  &  116.86
 & 
39.85  &  139.67
 & 
48.66  &  144.92
 & 
16.15  &  99.44
 & 
25.15  &  122.25
\\
                                & D-ToF & 
23.00  &  115.40
 & 
21.18  &  113.27
 & 
27.89  &  119.59
 & 
30.00  &  112.92
 & 
15.81  &  90.89
 & 
23.73  &  117.72
\\
                                & Active Stereo & 
25.94  &  124.17
 & 
25.50  &  126.28
 & 
27.18  &  117.04
 & 
32.81  &  121.24
 & 
16.40  &  101.86
 & 
15.73  &  95.27
\\              
\shline\\
\shline

\multirow{3}{*}{\rotatebox[origin=c]{-90}{Test 3}}     & I-ToF & 
36.82  &  152.51
 & 
35.92  &  153.26
 & 
38.75  &  147.14
 & 
34.09  &  127.51
 & 
20.21  &  110.85
 & 
55.09  &  183.14
\\
                                & D-ToF & 

32.99  &  145.50
 & 
35.64  &  153.07
 & 
25.90  &  120.35
 & 
19.92  &  96.01
 & 
21.59  &  105.41
 & 
37.26  &  149.66
\\
                                & Active Stereo & 
31.63  &  141.77
 & 
35.24  &  151.37
 & 
22.44  &  110.42
 & 
23.47  &  106.63
 & 
14.49  &  94.51
 & 
21.21  &  109.53
\\                              
\shline\\
\shline

\multirow{3}{*}{\rotatebox[origin=c]{-90}{T. Seen}}     & I-ToF & 
9.87  &  77.99
 & 
4.62  &  57.10
 & 
33.91  &  133.46
 & 
6.18  &  60.48
 & 
35.65  &  119.76
 & 
91.30  &  224.27
\\
                                & D-ToF & 

15.43  &  93.31
 & 
11.62  &  79.89
 & 
31.12  &  123.97
 & 
4.40  &  51.91
 & 
17.42  &  82.29
 & 
89.19  &  212.55
\\
                                & Active Stereo & 
9.43  &  88.30
 & 
9.28  &  88.24
 & 
9.11  &  75.21
 & 
6.32  &  65.54
 & 
12.98  &  65.73
 & 
16.62  &  98.75
\\
\shline\\
\\
\\
\multicolumn{6}{l}{Tested on Modality:}
\\
\shline
\multirow{4}{*}{\rotatebox[origin=c]{-90}{Test Seen}}     & I-ToF & 
8.34  &  52.29
 & 
8.57  &  50.00
 & 

7.01  &  58.85
&
3.80  &  43.44
 & 
23.28  &  95.38
 & 
13.69  &  65.41
\\
                                & D-ToF & 

8.05  &  50.43
 & 
6.82  &  45.50
 & 
 
13.52  &  66.34
&
9.00  &  54.15
 & 
30.91  &  87.71
 & 
27.92  &  87.32
\\
                                & Active Stereo & 
39.25  &  101.76
 & 
40.87  &  102.29
 & 
 
30.32  &  90.00
&
32.24  &  90.49
 & 
23.36  &  72.21
 & 
37.25  &  101.23
\\
                                & GT & 
1.12  &  28.81
 & 
0.71  &  24.41
 & 
 
2.65  &  40.41
&
1.83  &  34.89
 & 
2.16  &  29.55
 & 
5.02  &  52.43
\\
\shline

\end{tabular}
}
\end{adjustbox}
\label{tab:depth_supervision_results_additional}
\end{table*}


\subsubsection{Quantitative evaluation}
\paragraph{Test scenes. }
Table~\ref{tab:depth_supervision_results_additional} summarizes the extensive quantitative evaluation of the supervised training with different depth modalities as supervision signal for different test scenes. Test scene 1 has a similar background compared to the training scenes and includes additional unseen objects. The scene is also observed from viewing angles that differ significantly from the training data. The background in test scene 2 is only partly observed in the training data and it includes mostly unseen objects. Test scene 3 is similar to test scene 2, but with a modified object layout and difficult lighting in the background from an additional bright light source above the scene. The additional test set with (partly) seen scenes is an additional test split which includes the first 10 frames of each training sequence. Please note that these frames have not been used during training. Here, we first test all predictions against the rendered ground truth (Top) and additionally on each individual respective modality (Bottom) to highlight the overfitting issue of invalid ground truth from each modality.
The results suggest that overall the supervision with accurate rendered ground truth achieves to generalize best for (mostly) unknown scenes. It is noticeable, that the active stereo achieves to produce good predictions for transparent objects and also performs well for reflective ones. The I-ToF and D-ToF predictions suffer from incorrect ground truth values for such objects.


\paragraph{Overfitting on (partly) seen scenes. }
The (partly) seen scene shows generally lower overall errors for all modalities as compared to the (mostly) unseen test scenes 1,2, and 3. Again, the active stereo can provide decent depth supervision for reflective and transparent objects, where the ToF sensors cannot provide valid depth. The prediction of the background of the scene performs worst for the active stereo, as the textureless wall is still problematic for the sensor.

When testing on the respective modality itself, the overfitting issue due to incorrect depth values of the sensor becomes apparent. It can be noticed, that for objects where the respective sensor cannot yield accurate depth values (e.g. transparent objects for I-ToF or reflective objects for D-ToF), the errors are significantly lower, indicating overfitting to the specific sensor modality.


\subsubsection{Qualitative predictions}
Figures~\ref{fig:qual_scene12_1},~\ref{fig:qual_scene13_1} and~\ref{fig:qual_scene14_1} show predictions on exemplary frames of the test scenes 1, 2 and 3, together with the different sensor modalities and the error plot of the prediction compared against the ground truth. The training with rendered ground truth generally performs best. Both ToF sensors show incorrect depth values for reflective or transparent objects which also translates to incorrect predictions in these areas (compare Fig.~\ref{fig:qual_scene12_1}. The predictions when training with active stereo as supervision are more blurry and show less distinct edges at depth boundaries when compared to other modalities, which may arise from many depth pixels being invalidated by the sensors around such boundaries (compare Fig.~\ref{fig:qual_scene13_1}). The very challenging test scene 3 with bright lighting and many unseen objects is difficult to predict for all training setups (compare Fig.~\ref{fig:qual_scene14_1}. We can see similar artifacts as described above. Additionally, the unseen trophy object with partly reflective and partly transparent material shows large errors for the sensor inputs as well as for its predictions. The desk surface is also incorrectly captured by the D-ToF sensors due to large reflections and MPI from the background.




\begin{figure*}[!p]
 \centering
    \includegraphics[width=\linewidth]{figures/qual_scene12_1.png}
    \caption{Qualitative evaluation on test scene 1. Each depth modality, the network prediction when trained with supervision of each modality, and the error, are shown as qualitative evaluation.}
    \label{fig:qual_scene12_1}
\end{figure*}

\begin{figure*}[!p]
 \centering
    \includegraphics[width=\linewidth]{figures/qual_scene13_1.png}
    \caption{Qualitative evaluation on test scene 2. Each depth modality, the network prediction when trained with supervision of each modality, and the error, are shown as qualitative evaluation.}
    \label{fig:qual_scene13_1}
\end{figure*}

\begin{figure*}[!p]
 \centering
    \includegraphics[width=\linewidth]{figures/qual_scene14_1.png}
    \caption{Qualitative evaluation on test scene 3. Each depth modality, the network prediction when trained with supervision of each modality, and the error, are shown as qualitative evaluation.}
    \label{fig:qual_scene14_1}
\end{figure*}

\subsection{Implicit Reconstruction}
\paragraph{Implementation Details}
As mentioned in the main paper, we follow NeRF~\cite{mildenhall2021nerf} and build upon the work of~\cite{roessle2022dense} without a depth completion network, but leverage the respective sensor depth with a scale-invariant depth loss $\mathcal{L}_{\text{D}}$. We use images with a resolution of $640 \times 480$ and process batches of 1024 rays. We set $\lambda_{\text{D}}$ to 0.1 and the learning rate to 0.0005 and optimize for 100k iterations with Adam optimizer~\cite{kingma2014adam}. 

\subsection{Camera Pose Estimation}
The analysis above focuses on dense monocular depth estimation and novel view synthesis as recent and important approaches - for which pixelwise prediction and evaluation are crucial. We add results for direct SLAM (DSO)~\cite{engel2017direct}, KinectFusion~\cite{kinectfusion} with different depth modalities, and COLMAP SfM~\cite{schoenberger2016sfm} in Fig.~\ref{fig:recon_comp}.
\begin{figure*}[h!]
 \centering
    \includegraphics[width=1.0\textwidth]{figures/recon_comp.png}
    \caption{Qualitative reconstruction results from SLAM and SfM.}
    \label{fig:recon_comp}
\end{figure*}

Tab.~\ref{tab:pose} summarizes the relative pose error for different approaches (cf. Fig~\ref{fig:recon_comp}). Note the pose accuracy results for KinectFusion~\cite{kinectfusion} align with the depth results from Tab.2 in the main paper.

\begin{table}[!t]
\centering
\caption{Relative Pose Error of SLAM and SfM.}
\resizebox{1.0\columnwidth}{!}{
    \centering 
    \begin{tabular}{l|c|ccc|c} 
    \shline
    Error & Direct (DSO) & Dense dToF & Dense iToF & Dense AS & SfM \\ 
    \shline
    rot [deg] & 0.22 & 0.18 & 0.51 & 0.56 & 10.76 \\
    trans [cm] & 0.27 & 0.31 & 0.68 & 0.62 & 2.86 \\
    \shline
    \end{tabular}}
    \label{tab:pose}
\end{table}

\clearpage

\section{Dataset}
\subsection{Detailed Dataset Description}
\label{sec:detailed_dataset_description}

Sec.~3 of the main paper mentioned that our dataset uses multiple images/depth sensors to collect the dataset with highly accurate annotations of the scene using the robotic arm in a synchronized manner. This section shows the detailed description of data we include in our dataset. 


\subsubsection{Polarization Camera}
\label{subsec:polarization_camera}

Fig.~\ref{fig:polarization_included} shows examples of images included for the polarization camera. As mentioned in Sec.~2 of the main paper, a polarization camera provides images with different polarization angles, which can extract cues like the surface normal by using the physical property of object material in the scene. The polarization camera we used in our dataset (See Sec.~3 in the main paper) provides polarized images at 4 different angles (0, 90, 180 270 degrees) which are saved in a single 2x2 image (Fig.~\ref{fig:polarization_included}, (a)). A regular RGB image is obtained by averaging the 4 images (Fig.~\ref{fig:polarization_included}, (b)). To showcase the results of the depth map trained with different depth cameras, we include warped depth images from each depth camera into the polarization camera coordinates using the extrinsic between the two cameras and its depth image (Fig.~\ref{fig:polarization_included}, (d-g)). These can be additionally used for RGBD-based depth completion research. On top of that, we include extra information, such as instance map (Fig.~\ref{fig:polarization_included}, (c)) to help train or validate pipelines for categorical level tasks, accurate 6d pose of the camera as the 4x4 matrix obtained from the robotic arm, extrinsic transformation between cameras as 4x4 matrices and camera intrinsics as 3x3 matrix.

\begin{figure*}[!htbp]
 \centering
    \includegraphics[width=\linewidth]{figures/polarization_included.PNG}
    \caption{Example of the images included for the polarization camera input (top) together with instance label map and depth estimates warped onto the same coordinate reference frame.}
    \label{fig:polarization_included}
\end{figure*}

\subsubsection{D-ToF Camera}
\label{subsec:dtof_camera}


Fig.~\ref{fig:dtof_included} shows an example of images included for the D-ToF camera. Direct ToF (D-ToF) camera senses the depth information of its surrounding by emitting an infrared signal and measuring the difference in time between the emitted and received signal. The quality of this modality highly depends on the reflection of the signal. It often suffers from specific physical noise such as Multi-Path-Interference (MPI) or strong material dependent artefacts (Fig.~\ref{fig:dtof_mpi}). For the D-ToF camera, we provide the depth map from the camera (Fig.~\ref{fig:dtof_included}, (a)) as well as its rendered ground truth depth map (Fig.~\ref{fig:dtof_included}, (b))  such that one can also research on D-ToF refinement pipelines to reduce such errors. As in the polarization camera, we include extra information such as instance label map (Fig.~\ref{fig:dtof_included}, (c)), camera pose, intrinsic and extrinsics of the camera as well.

\begin{figure*}[!htbp]
 \centering
    \includegraphics[width=\linewidth]{figures/dtof_included.PNG}
    \caption{Example of the images included for the D-ToF camera: its depth map (left), ground truth depth (centre) and an object instance label map (right).}
    \label{fig:dtof_included}
\end{figure*}


\subsubsection{I-ToF Camera}
\label{subsec:itof_camera}


Fig.~\ref{fig:itof_included} shows image examples for the I-ToF camera. Indirect ToF (I-ToF) cameras sense the depth information of their surrounding by emitting a frequency modulated signal and measuring the return signal. Unlike Direct ToF (D-ToF), I-ToF cameras do not calculate the time difference to infer the depth. Instead, the camera correlates the returning signal with phase-shifted emitting signals to generate 4 different measurements, called correlation images. These are measured as sinus functions of distance ($\left( \sin(d),\cos(d),-\sin(d),-\cos(d) \right) = \left( c_{1}, c_{2}, c_{3}, c_{4} \right)$ in Fig.~\ref{fig:itof_included}, (a)). Either arc-tangent formula or convolutional neural networks can be used to extract depth information from the correlation images. As I-ToF modality also relies on the reflection of the signal like in D-ToF, it suffers from similar artefacts, such as MPI and material dependent artefacts (compare qualitative results of the test scenes in Figs.~\ref{fig:qual_scene12_1}, ~\ref{fig:qual_scene13_1} and ~\ref{fig:qual_scene14_1}). Here, we provide raw correlation images and depth map from the camera (see Fig.~\ref{fig:itof_included}, (a,b)) as well as its rendered ground truth depth (Fig.~\ref{fig:itof_included}, (c)) such that one can train I-ToF depth improvement pipelines either from raw signal or from I-ToF depth itself. As the other cameras, extras such as instance map (Fig.~\ref{fig:itof_included}, (d)), camera pose, intrinsic and extrinsics are included.

\begin{figure*}[!htbp]
 \centering
    \includegraphics[width=\linewidth]{figures/itof_included.PNG}
    \caption{Example of the images included for the I-ToF camera.}
    \label{fig:itof_included}
\end{figure*}

\subsubsection{Active Stereo Camera}
\label{subsec:active_stereo_camera}

Fig.~\ref{fig:d435_included} shows the examples of images included for the Active Stereo camera. Stereo depth estimation infers depth using and photometric consistency and geometrical constraints from epipolar geometry and triangulates the depth map from the disparity between left and right cameras. As the disparity is calculated via matching on the image itself, the stereo based depth estimation methods suffers less from the specific material, but they suffer from other aspects such as stereo occlusion and large texture-less regions. Active projection (Active Stereo) is used to overcome this issue. We provide both, active and passive stereo left / right images (Fig.~\ref{fig:d435_included}, (a),(b)) and raw depth from the camera (active,  Fig.~\ref{fig:d435_included}, (c)) as well as the rendered ground truth (Fig.~\ref{fig:d435_included}, (d)). This allows to use our dataset to improve stereo methods from either passive or active stereo and also depth refinement pipelines. Similar to the other cameras, extras such as instance map (Fig.~\ref{fig:d435_included}, (e)), camera pose, intrinsic and extrinsics are included.

\begin{figure*}[!htbp]
 \centering
    \includegraphics[width=\linewidth]{figures/d435_included.PNG}
    \caption{Example of the images included for the Active Stereo camera.}
    \label{fig:d435_included}
\end{figure*}

\subsection{Error Analysis on Different Modality}
\label{sec:Error_analysis}
In this section, we show specific errors on each depth modality to illustrate the implication of the depth quality when the given modality is used as the ground truth, as well as advantage of using our rendered depth as the ground truth.

\subsubsection{D-ToF Camera}
\label{subsec:dtof_camera_error}
As mentioned in Subsec.~\ref{subsec:dtof_camera}, D-ToF modality suffers from its own reflection-based nature, such as MPI and material dependent artefacts. When the angle of the surface normal of the scene is close to the incident angle of the infrared signal, the strength of the reflected signal becomes weak due to scattering effects (Fig.~\ref{fig:dtof_mpi}, (a) blue arrow) while multiple scattered signals from the other surfaces which has more traveling distance are received and with stronger strength (Fig.~\ref{fig:dtof_mpi}, (a) red arrow) and interfere with the original signal (MPI), producing a wrong measurement of the depth on the area with further distance which looks like a reflection or shadow of the object to the surface (Fig.~\ref{fig:dtof_mpi}, (b) red marker). This effect can be intensified when the surface material is reflective, which gives even stronger artefact as its reflective surface bounces even weaker and noisier signal with less attenuation (Fig.~\ref{fig:dtof_mpi}, (a,b) yellow arrow\&marker). On the other hands, when the surface material is transparent, the emitted infrared signal rather goes through the object in the both ways  (Fig.~\ref{fig:dtof_mpi}, (a) green arrow) which at the end ignores the object and the sensor produce the depth value as similar level as its background (Fig.~\ref{fig:dtof_mpi}, (b) green marker - material dependent artefact). Quality of the depth map degrades slightly around some boundaries after warping into the RGB frame (Fig.~\ref{fig:dtof_aligned}, (b), red), while the invalid regions actually helps to invalidate more area on wrong depth especially on the reflective objects  (Fig.~\ref{fig:dtof_aligned}, (b), green) , which might become beneficial when it is used in the training.

\begin{figure*}[!htbp]
 \centering
    \includegraphics[width=\linewidth]{figures/dtof_not_aligned.PNG}
    \caption{Detailed ray paths with MPI and surface material induced error on D-ToF modality. While D-ToF produces dense and sharp depth, its quality is highly dependent on the surface material and the incident angle.}
    \label{fig:dtof_mpi}
\end{figure*}

\begin{figure*}[!htbp]
 \centering
    \includegraphics[width=\linewidth]{figures/dtof_aligned.PNG}
    \caption{Error after warping D-ToF into RGB view. Slight errors are introduced on some edges (red) while expansion of the invalid area helps to invalidate on the reflective objects (green).}
    \label{fig:dtof_aligned}
\end{figure*}


\subsubsection{I-ToF Camera}
\label{subsec:itof_camera_error}
As mentioned in Subsec.~\ref{subsec:itof_camera}, I-ToF modality suffers by its own reflection based nature as well similar to D-ToF, such as MPI and material dependent artefact (Fig.~\ref{fig:itof_mpi}. Although the quality of depth itself seems better as the depth itself is more dense (with less invalid region) and amount of the artefacts are less, it is hard to say I-ToF modality is better than D-ToF as these two camera are in different price range and power level. Also less invalid area but rather with wrong depth didn't help invalidating depth (Fig.~\ref{fig:itof_aligned}) not like in D-ToF case, which could result in artefact in the prediction when it is used as GT during the training.

\begin{figure*}[!tbp]
 \centering
    \includegraphics[width=\linewidth]{figures/itof_not_aligned.PNG}
    \caption{Depth quality from I-ToF camera. I-ToF modality suffers from same type of artefect as D-ToF. While depth map itself is more sense and suffers less from MPI artefact on the table.}
    \label{fig:itof_mpi}
\end{figure*}

\begin{figure*}[!tbp]
 \centering
    \includegraphics[width=\linewidth]{figures/itof_aligned.PNG}
    \caption{Error after warping I-ToF into RGB view. Not like D-ToF, most of depth error exists without being invalidated, which might introduce more error when it used as GT during the training.}
    \label{fig:itof_aligned}
\end{figure*}


\subsubsection{Active Stereo Camera}
\label{subsec:active_stereo_camera_error}
As the stereo camera uses left and right matching with photoelectric cue, depth map suffers less on the challenging material as the projection can be visible on the surface as well as left-right check can be performed to invalidate region with the wrong depth. For this reason, depth on glass or the reflective object is significantly more accurate compared to either of ToF modality (Fig.~\ref{fig:d435_not_aligned}, green arrow). On the other hands, due to its nature of pattern projection far distance that depth quality gets worsen as the scene gets further (Fig.~\ref{fig:d435_not_aligned}, red arrow) the projection pattern gets attenuated and spread in the far distance. Moreover, the depth map in general is more blurry, jittery, sparse and has wrong values on some regions without being invalidated (Fig.~\ref{fig:d435_not_aligned}, orange arrow) which can introduce negative influence when it is used as GT, such as blurriness and depth jittering. Error introduced by warping is trivial (Fig.~\ref{fig:d435_aligned}) as the original depth map is already blurry and sparse.

\begin{figure*}[!htbp]
 \centering
    \includegraphics[width=\linewidth]{figures/d435_not_aligned.PNG}
    \caption{Depth quality from Active Stereo camera. While depth map suffers less on the challenging material, quality of depth itself is far behind either of ToF modality in multiple aspects, such as sharpness, variance, sparsity.}
    \label{fig:d435_not_aligned}
\end{figure*}

\begin{figure*}[!htbp]
 \centering
    \includegraphics[width=\linewidth]{figures/d435_aligned.PNG}
    \caption{Error after warping Active Stereo into RGB view. Note that there isn't significant change in the depth quality after the warping.}
    \label{fig:d435_aligned}
\end{figure*}




\clearpage
\newpage

\subsection{Detailed Background and Objects Description}
\label{sec:bckgr_obj_description}


\begin{figure*}[!b]
 \centering
    \includegraphics[width=\linewidth]{figures/Chair.PNG}
    \caption{Chairs used in the dataset. Chairs in group (a) are used for the training set and the chair in (b) is used for the test set.}
    \label{fig:chair}
\end{figure*} 

\begin{figure*}[!p]
 \centering
    \includegraphics[width=\linewidth]{figures/background_full_new.PNG}
    \caption{Backgrounds used in the dataset. Note that one of the background in the group (b) is also included in the training set, but we varied the lighting condition to provide different various factors for evaluation.}
    \label{fig:wall}
\end{figure*}

As described in Sec.~4 in the main paper, our dataset comprises a total of 13 scenes divided into 10 scenes for training and 3 testing scenes composed of a mixture of 4 different chairs, 6 different tables, 64 household objects from 8 plus 4 different categories (i.e. cup, teapot, bottle, remote, boxes, can, glass, cutlery and tube, shoe, plastic kitchenware, trophy) and and 7 different indoor areas. Test sets have 1 unseen background and 2 seen backgrounds with and without different lighting and contain a mixture of seen/unseen objects from seen/unseen categories. In this section, we show detailed images of backgrounds, chairs, tables, and other objects. Fig.~\ref{fig:chair} and~\ref{fig:table} respectively show images of 3 chairs and 6 tables used in the dataset and their corresponding meshes. Fig.~\ref{fig:obj_train} and~\ref{fig:obj_test} show a collection of household objects used in training and test set. Fig.~\ref{fig:wall} shows 9 backgrounds used in the dataset and their corresponding meshes.


\begin{figure*}[!t]
 \centering
    \includegraphics[width=\linewidth]{figures/table.PNG}
    \caption{Tables used in the dataset. Tables in group (a) are used for the training set and the table in (b) is used for the test set. Note that, unlike small objects or chairs, we decide not to scan some parts of the large tables (e.g. end of their legs) as the cameras cannot see the part in their trajectories.}
    \label{fig:table}
\end{figure*}

\begin{figure*}[!b]
 \centering
    \includegraphics[width=\linewidth]{figures/household_train.PNG}
    \caption{Collection of small household objects used in the training set. Objects from 8 household categories are used in the training set, 3 of which have photometrically challenging surface material - partially reflective (can), transparent (glass/plastic), reflective (cutlery).}
    \label{fig:obj_train}
\end{figure*}

\begin{figure*}[!t]
 \centering
    \includegraphics[width=\linewidth]{figures/household_test.PNG}
    \caption{Collection of small household objects used in the test set. The test set comprises a mixture of seen (left column) and unseen (mid column) objects from 8 seen categories and a few objects from unseen categories (right column - tube, slipper, plastic kitchenware, trophy) are used.}
    \label{fig:obj_test}
\end{figure*}




\subsubsection{Detailed Scene Description}
\label{sec:scene_description}

As described, our training set is composed of 10 scenes, and the test set is composed of 3 scenes. For each scene, we include 2 different trajectories. Each trajectory covers 2 setups with and without objects (naked scene). This sums up to 800-1200 frames per scene and a total of ca.~10k frames. In this section, we show several sample images of the scenes in Fig.~\ref{fig:wall1},~\ref{fig:wall2}, and~\ref{fig:wall3}, ~\ref{fig:wall4}. Each of them consists of an annotated mesh and RGB images with different types of rendering, which show the diversity and quality of our dataset.

\begin{figure*}[!htbp]
 \centering
    \includegraphics[width=\linewidth]{figures/scene1_caption.PNG}
    \caption{Example images from Training Scene 1. The annotated mesh is shown on the left together with an RGB view from the scene (second from left) with and without objects. The overlayed masks (second from right) and the rendered depth (right) illustrate the annotation quality of our data.}
    \label{fig:wall1}
\end{figure*}

\begin{figure*}[!htbp]
 \centering
    \includegraphics[width=\linewidth]{figures/scene2-5_caption.PNG}
    \caption{Example images from Training Scene 2-5. The annotated mesh for 4 different scenes is shown on the left together with an RGB view from the scene (second from left) with and without objects. The overlayed masks (second from right) and the rendered depth (right) illustrate the annotation quality of our data.}
    \label{fig:wall2}
\end{figure*}

\begin{figure*}[!htbp]
 \centering
    \includegraphics[width=\linewidth]{figures/scene6-9.PNG}
    \caption{Example images from Training Scene 6-9. The annotated mesh for four different scenes is shown on the left together with an RGB view from the scene (second from left) with and without objects. The overlayed masks (second from right) and the rendered depth (right) illustrate the annotation quality of our data.}
    \label{fig:wall3}
\end{figure*}

\begin{figure*}[!htbp]
 \centering
    \includegraphics[width=\linewidth]{figures/scene10-13.PNG}
    \caption{Example images from Training Scene 10 and Test scene 1-3. The annotated mesh is shown on the left together with an RGB view from the scene (second from left) with and without objects. The overlayed masks (second from right) and the rendered depth (right) illustrate the annotation quality of our data. Note that the test scene 2,3 are recorded in the exactly same pose and trajectory but with the different lighting.}
    \label{fig:wall4}
\end{figure*}


\subsubsection{Partial Scanning of the Scene and Mesh Fitting}
\label{sec:partial_scanning}

As mentioned in Sec.~3 in the main paper, we use partial scanning and mesh fitting to annotate background, large objects, and objects outside the robotic workspace. This section shows images of partial scanning and the mesh fitting from one of the scenes as an example. The green box in Fig.~\ref{fig:annotated_objects}, (a) shows annotated meshes of the objects by the robotic arm. Once the objects are annotated, the scene is partially scanned with multiple viewpoints to make the scanning dense and cover multiple facets of the background. Note that the center of the scanning is not yet in the robot base coordinates (Fig.~\ref{fig:annotated_objects}, (a) blue box). Once the partial scanning is done, the scanned mesh is then fit onto the annotated objects, such that the partially scanned mesh origin concides with the robot base (Fig.~\ref{fig:annotated_objects}, (b)). Once the scanned mesh is put to robot base coordinates, we fit background, large objects, and distant objects meshes also in robot base coordinates to annotate them (Fig.~\ref{fig:fitting_final}, (a)). Fig.~\ref{fig:fitting_final}, (b-c) shows the result of the annotated mesh. %As this method may contain errors propagated from noise and error of the mesh, we manually refined the pose of the fitted mesh in few mm in translation and sub degree in rotation such that the objects can be well globally well aligned into all of the cameras' viewpoints. 
For the mesh fitting, we used Artec Studio 10 Professional (Artec 3D, Luxembourg) which runs a point correspondence and ICP-based method to fit the meshes.

\begin{figure*}[!htbp]
 \centering
    \includegraphics[width=\linewidth]{figures/scanning_before_after_fitting.PNG}
    \caption{Example of partial scanning of the scene before and after the fitting on scene 13. Note that the center of the partial scanned mesh is aligned to robot base (xyz coordinate marker) after fitting it onto the mesh of the annotated objects.}
    \label{fig:annotated_objects}
\end{figure*}

\begin{figure*}[!htbp]
 \centering
    \includegraphics[width=\linewidth]{figures/object_fitting_cut.PNG}
    \caption{Example of far objects and background fitting onto partially scanned mesh. Left: Background and objects are fit to partial scans. Centre: All annotated meshes are shown without partial scans. Right: Corresponding scene from the camera viewpoint with augmented object masks. Note that the annotation quality of meshes with partial scans and robot arm is similar. The annotated meshes via partial scanning are marked with red arrows.}
    \label{fig:fitting_final}
\end{figure*}






\clearpage
\newpage
%%%%%%%%% REFERENCES
{\small
\bibliographystyle{ieee_fullname}
\bibliography{PaperForReview_supp}
}

\end{document}
