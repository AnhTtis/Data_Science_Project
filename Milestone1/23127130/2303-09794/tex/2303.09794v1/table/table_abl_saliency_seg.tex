% \begin{table}[t]%\footnotesize
%     \centering
%     \caption{Ablation study on the background selection in our approach for Cityscapes dataset. 
%     }
%     \label{tab:abl_bg_selection}
%     \begin{tabular}{  c | c c c c }
%          \toprule
%          & 1/16 (186) & 1/16 (372) & 1/4 (744) & 1/2 (1488)\\
%          \midrule
%           void &\textbf{72.42}  &\textbf{75.76}  &77.65  &\textbf{79.18}\\
%           road & 71.38 &75.38 & \textbf{77.85} &78.92\\
%          \bottomrule
%     \end{tabular}
% \end{table}



\begin{table}[h]\small
\vspace{-0.4cm}
\centering
\begin{tabular}{l|ccccc}
\toprule
 Method  &92  &183  &366  &732 \\ 
 \midrule
 %Baseline &67.8  &71.2  &75.6  &77.4 \\ 
 fore/back-ground seg. & 67.7 & 71.9   & 76.1  & 78.1  \\ 
 %Baseline+rec &69.0 &72.4   &76.3  &78.1  \\ 
 \textbf{FOrec (ours)} &\textbf{71.0}  &\textbf{74.7}  &\textbf{77.5}  &\textbf{78.7}    \\ 
 \bottomrule
\end{tabular}
\vspace{-0.3cm}
\caption{Foreground-background (saliency estimation) segmentation on PASCAL VOC 2012 (classic setting).}
\label{tab:abl_fbg_seg}
\vspace{-0.4cm}
\end{table}