%%
%% This is file `sample-manuscript.tex',
%% generated with the docstrip utility.
%%
%% The original source files were:
%%
%% samples.dtx  (with options: `manuscript')
%% 
%% IMPORTANT NOTICE:
%% 
%% For the copyright see the source file.
%% 
%% Any modified versions of this file must be renamed
%% with new filenames distinct from sample-manuscript.tex.
%% 
%% For distribution of the original source see the terms
%% for copying and modification in the file samples.dtx.
%% 
%% This generated file may be distributed as long as the
%% original source files, as listed above, are part of the
%% same distribution. (The sources need not necessarily be
%% in the same archive or directory.)
%%
%% Commands for TeXCount
%TC:macro \cite [option:text,text]
%TC:macro \citep [option:text,text]
%TC:macro \citet [option:text,text]
%TC:envir table 0 1
%TC:envir table* 0 1
%TC:envir tabular [ignore] word
%TC:envir displaymath 0 word
%TC:envir math 0 word
%TC:envir comment 0 0
%%
%%
%% The first command in your LaTeX source must be the \documentclass command.
%%%% Small single column format, used for CIE, CSUR, DTRAP, JACM, JDIQ, JEA, JERIC, JETC, PACMCGIT, TAAS, TACCESS, TACO, TALG, TALLIP (formerly TALIP), TCPS, TDSCI, TEAC, TECS, TELO, THRI, TIIS, TIOT, TISSEC, TIST, TKDD, TMIS, TOCE, TOCHI, TOCL, TOCS, TOCT, TODAES, TODS, TOIS, TOIT, TOMACS, TOMM (formerly TOMCCAP), TOMPECS, TOMS, TOPC, TOPLAS, TOPS, TOS, TOSEM, TOSN, TQC, TRETS, TSAS, TSC, TSLP, TWEB.
% \documentclass[acmsmall]{acmart}

%%%% Large single column format, used for IMWUT, JOCCH, PACMPL, POMACS, TAP, PACMHCI
% \documentclass[acmlarge,screen]{acmart}

%%%% Large double column format, used for TOG
% \documentclass[acmtog, authorversion]{acmart}

%%%% Generic manuscript mode, required for submission
%%%% and peer review
\documentclass[sigconf]{acmart}

\usepackage{xspace}
\usepackage{soul}
\setstcolor{red}
%% Fonts used in the template cannot be substituted; margin 
%% adjustments are not allowed.
%%
%% \BibTeX command to typeset BibTeX logo in the docs
\AtBeginDocument{%
  \providecommand\BibTeX{{%
    \normalfont B\kern-0.5em{\scshape i\kern-0.25em b}\kern-0.8em\TeX}}}

%% Rights management information.  This information is sent to you
%% when you complete the rights form.  These commands have SAMPLE
%% values in them; it is your responsibility as an author to replace
%% the commands and values with those provided to you when you
%% complete the rights form.
\copyrightyear{2023}
\acmYear{2023}
\setcopyright{rightsretained}
\acmConference[CHI '23]{Proceedings of the 2023 CHI Conference on Human Factors in Computing Systems}{April 23--28, 2023}{Hamburg, Germany}
\acmBooktitle{Proceedings of the 2023 CHI Conference on Human Factors in Computing Systems (CHI '23), April 23--28, 2023, Hamburg, Germany}\acmDOI{10.1145/3544548.3581347}
\acmISBN{978-1-4503-9421-5/23/04}

%% These commands are for a PROCEEDINGS abstract or paper.
\acmConference[CHI '23]{CHI Conference on Human Factors in Computing Systems}{April 23-April 28, 2023}{Hamburg, Germany}
%
%  Uncomment \acmBooktitle if th title of the proceedings is different
%  from ``Proceedings of ...''!
%
\acmBooktitle{CHI Conference on Human Factors in Computing Systems (CHI '23), April 23-April 28, 2023, Hamburg, Germany} 
\acmPrice{15.00}
\acmISBN{978-1-4503-XXXX-X/18/06}

% \renewcommand{\UrlFont}{\ttfamily\small}
% \usepackage[usenames,dvipsnames]{xcolor}
% \usepackage[colorinlistoftodos]{todonotes}
% \usepackage{verbatim}
% \usepackage{changes}
\newcommand{\kr}[1]{\textcolor{cyan}{\textit{KR}: #1}}
\newcommand{\rock}[1]{\textcolor{red}{\textit{RP}: #1}}
\newcommand{\rr}[1]{\textcolor{black}{#1}}
\newcommand{\rrr}[1]{\textcolor{black}{#1}}
\newcommand{\kd}[1]{\textcolor{red}{#1}}
% \newcommand{\deleted}[1]{\textcolor{black}{#1}}

\def\participants{participants\xspace} %
\def\researchers{researchers\xspace} %
\def\impactstatements{broader impact statements\xspace} %
%%
%% Submission ID.
%% Use this when submitting an article to a sponsored event. You'll
%% receive a unique submission ID from the organizers
%% of the event, and this ID should be used as the parameter to this command.
\acmSubmissionID{2552}

%%
%% For managing citations, it is recommended to use bibliography
%% files in BibTeX format.
%%
%% You can then either use BibTeX with the ACM-Reference-Format style,
%% or BibLaTeX with the acmnumeric or acmauthoryear sytles, that include
%% support for advanced citation of software artefact from the
%% biblatex-software package, also separately available on CTAN.
%%
%% Look at the sample-*-biblatex.tex files for templates showcasing
%% the biblatex styles.
%%

%%
%% The majority of ACM publications use numbered citations and
%% references.  The command \citestyle{authoryear} switches to the
%% "author year" style.
%%
%% If you are preparing content for an event
%% sponsored by ACM SIGGRAPH, you must use the "author year" style of
%% citations and references.
%% Uncommenting
%% the next command will enable that style.
%%\citestyle{acmauthoryear}

\author{Kimberly Do*\texorpdfstring{$\dagger$}{}}
\affiliation{%
 \institution{Khoury College of Computer Sciences, \\ Northeastern University}
 \streetaddress{360 Huntington Ave}
 \city{Boston}
 \state{Massachusetts}
 \country{USA}
 }

\author{Rock Yuren Pang*}
\thanks{* Listed in alphabetical order. Both authors contributed equally to this research. Corresponding Author: Rock Yuren Pang, ypang2@cs.washington.edu}
\thanks{$\dagger$ Authored during REU at the University of Washington while affilitated with the Georgia Institute of Technology.} 
\affiliation{%
 \institution{Paul G. Allen School of Computer Science, \\ University of Washington}
 \streetaddress{185 E Stevens Way NE}
 \city{Seattle}
 \state{Washington}
 \country{USA}}

\author{Jiachen Jiang}
\affiliation{%
 \institution{Microsoft}
 \streetaddress{}
 \city{Redmond}
 \state{Washington}
 \country{USA}}

\author{Katharina Reinecke}
\affiliation{%
 \institution{Paul G. Allen School of Computer Science, \\ University of Washington}
 \streetaddress{185 E Stevens Way NE}
 \city{Seattle}
 \state{Washington}
 \country{USA}}


%%
%% end of the preamble, start of the body of the document source.
\begin{document}

%%
%% The "title" command has an optional parameter,
%% allowing the author to define a "short title" to be used in page headers.
\title[``That's important, but...'': How Computer Science Researchers Anticipate Unintended Consequences \\ of Their Research Innovations]{``That's important, but...'': How Computer Science Researchers Anticipate Unintended Consequences of Their Research Innovations}


%%
%% The "author" command and its associated commands are used to define
%% the authors and their affiliations.
%% Of note is the shared affiliation of the first two authors, and the
%% "authornote" and "authornotemark" commands
%% used to denote shared contribution to the research.

%%
%% By default, the full list of authors will be used in the page
%% headers. Often, this list is too long, and will overlap
%% other information printed in the page headers. This command allows
%% the author to define a more concise list
%% of authors' names for this purpose.
\renewcommand{\shortauthors}{Do*, Pang*, Jiang, and Reinecke}

%%
%% The abstract is a short summary of the work to be presented in the
%% article.
\begin{abstract}
Computer science research has led to many breakthrough innovations but has also been scrutinized for enabling technology that has negative, unintended consequences for society. Given the increasing discussions of ethics in the news and among researchers, we interviewed 20 researchers in various CS sub-disciplines to identify whether and how they consider potential unintended consequences of their research innovations. We show that considering unintended consequences is generally seen as important but rarely practiced. Principal barriers are a lack of formal process and strategy as well as the academic practice that prioritizes fast progress and publications. Drawing on these findings, we discuss approaches to support researchers in routinely considering unintended consequences, from bringing diverse perspectives through community participation to increasing incentives to investigate potential consequences. We intend for our work to pave the way for routine explorations of the societal implications of technological innovations before, during, and after the research process.
\end{abstract}

%%
%% The code below is generated by the tool at http://dl.acm.org/ccs.cfm.
%% Please copy and paste the code instead of the example below.
%%
\begin{CCSXML}
<ccs2012>
<concept>
<concept_id>10003120.10003121.10003126</concept_id>
<concept_desc>Human-centered computing~HCI theory, concepts and models</concept_desc>
<concept_significance>500</concept_significance>
</concept>
</ccs2012>
\end{CCSXML}

\ccsdesc[500]{Human-centered computing~HCI theory, concepts and models}

%%
%% Keywords. The author(s) should pick words that accurately describe
%% the work being presented. Separate the keywords with commas.
\keywords{Unintended Consequences, Computer Ethics}

%%
%% This command processes the author and affiliation and title
%% information and builds the first part of the formatted document.
\maketitle

\section{Introduction}

The ability to reason about plans is critical for performing long-horizon tasks \citep{erol1996hierarchical, sohn2018hierarchical, sharma-etal-2022-skill}, compositional generalization \citep{corona-etal-2021-modular} and generalization to unseen tasks and environments \citep{shridhar2020alfred}.
Consider a simple long-horizon planning scenario where a robot is tasked with preparing a meal and serving it on the table. 
This presents a non-trivial planning problem since the agent needs to understand the sequence of operations required to perform the task and search for the relevant objects in the unfamiliar environment by interacting with various objects. %



Large language models have been recently shown to possess commonsense knowledge about the world such as object affordances and physical dynamics \citep{ouyang2022training,chowdhery2022palm}.
Early approaches considered text based environments and fine-tuned PLMs to predict actions given the history of past observations and actions \citep{jansen-2020-visually,micheli-fleuret-2021-language,yao-etal-2020-keep}.
Recent work has used this ability to reason about plans from text instructions in simulated household environments with simplifying assumptions such as text-only environment observations or feedback \citep{huang2022language,ahn2022can,li2022pre,logeswaran-etal-2022-shot}.


We focus on \emph{visually grounded planning} with PLMs --- the ability to adapt plans based on interaction and visual feedback from the environment.
While PLMs have strong planning commonsense priors, predictions from a PLM may not be directly realizable in the environment since the observation and action spaces are unknown.
This requires \emph{grounding} the PLM in the environment and adapting it to observe visual feedback, which is highly non-trivial.
Some prior works assume the availability of a pre-trained affordance function \citep{ahn2022can} or a success detector \citep{mirchandani2021ella}.
Notably, SayCan \citep{ahn2022can} completely decouples the PLM from observation information by selecting actions that have both high affordability (through a pre-trained affordance model) and high PLM likelihood.
Although this partially addresses the grounding problem, the use of visual feedback for action affordance alone is limited.
Often an agent must choose one of many affordable actions using information from observations.
For example, a driving agent should re-navigate and possibly turn around when encountering a ``road closed'' sign, but both turning around and driving forward are indistinguishable to SayCan because they are both affordable and the PLM is blind to observations.

Another workaround explored in prior work is translating the information in the visual observations to text using a pre-trained captioning system \citep{shridhar2021alfworld,huang2022language}.
However, it can be difficult to faithfully describe an image in words and information is lost in this inherently noisy process, which limits the information available to the planner.



Recent work shows that PLMs can be adapted for various natural language tasks by inserting tunable embeddings or soft prompts at the input of the PLM (also called prompt tuning or prefix tuning)~\citep{li-liang-2021-prefix,lester-etal-2021-power}.
This approach also extends to multi-modal understanding tasks such as image captioning \citep{mokady2021clipcap} and VQA \citep{tsimpoukelli2021multimodal} where images are encoded as soft prompts and finetuned for the target task.
Transformer based architectures have also been successfully applied to offline Reinforcement Learning in recent work \citep{chen2021decision,janner2021offline,li2022pre,reid2022can}.

Taking inspiration from these works, we propose the simple approach of embedding visual observations (`visual prompts') and \textit{directly inserting them as PLM input embeddings}.
The visual encoder and PLM are jointly trained for the target task, an approach we call \textbf{\oursfull}~(\ours).
By teaching the PLM to use observations for planning in an end to end manner, we remove the dependency on external data such as captions and affordability information that was used in prior work.
We show that this simple approach performs better than prior PLM-based planning approaches on two embodied planning benchmarks based on ALFWorld~\citep{shridhar2021alfworld} and Virtualhome~\cite{puig2018virtualhome}.



\section{Terminology}
 Already in 1936, Merton~\cite{merton1936unanticipated} coined the term \emph{unanticipated  consequences} to describe unforeseen, desirable or undesirable outcomes of one's action. Today, most researchers refer to unforeseen outcomes (the results of policies, technologies, or other ``purposive social actions''~\cite{merton1936unanticipated}) as \emph{unintended consequences}, though some have suggested that this term conflation has caused a loss of nuance~\cite{huntington1971change, parvin2020unintended}. As we show in ~\autoref{fig:term}, Merton's original term of \emph{unanticipated consequences} suggests that such consequences are always unintended. In contrast, \emph{unintended consequences} can be either unanticipated or anticipated. Parvin and Pollock have therefore argued that this lack of precision in terminology may lead people ``to abdicate responsibility for the perfectly foreseeable consequences of particular decisions''~\cite[p.323]{parvin2020unintended}. Hence, the use of the term \emph{unintended consequences} may reduce accountability: ``Phenomena described as unintended consequences are deemed too difficult, too out of scope, too out of reach, or too messy to have been dealt with at any point in time before they created problems for someone else. The descriptive approach works as a defensive and dismissive strategy''~\cite[p.322]{parvin2020unintended}. 
 

 \label{terminology}
 \begin{figure}[h!]
    \centering
    \includegraphics[width=0.4\textwidth]{figures/terminology.pdf}
        \caption{Terminology of anticipated, intended, unintended, and unanticipated consequences. \rr{Unintended consequences can be either unanticipated or anticipated. }}
    \label{fig:term}
    \Description{A Venn diagram with two intersecting circles labeled "unanticipated" and "intended." The intersection is labeled "anticipated". The area including the circle labeled "unanticipated" but excluding the area of the "anticipated" intersection is labeled "unintended."}
\end{figure}
 
 In this paper, we use the term \emph{unintended consequences (UCs)} to purposefully broaden the discussion to include both anticipated and unanticipated, positive or negative unintended side effects of technology on society. Our definition includes consequences that the instigators of an action (i.e., researchers and/or technology innovators) may not have addressed but could have foreseen. While these consequences can be positive or negative in nature (and oftentimes have different effects on a population), our work is inherently oriented towards considering negative UCs more than positive ones. In the remainder of this paper, we use ``technology'' for digital technology, such as hardware devices or software systems. We broadly refer to ``society''  at a regional, national, or international level.
 

\section{Related Work}
 
 Our work draws upon prior work studying values and ethics in digital technology~\cite{Shilton2018ValuesAE} and is informed by discussions about the effects of digital technology on society in fields such as philosophy~\cite{moor1985computer, johnson1985computer}, STS~\cite{winner2017artifacts, winner2010whale, Klein2002TheSC}, social informatics~\cite{kling1996computerization}, feminism~\cite{Bardzell2010FeministHT,haimson2016constructing} and postcolonial theories~\cite{irani2010postcolonial}. We start by showing how researchers in these fields have long discussed various societal effects of technology before outlining the methods and approaches researchers and practitioners have developed for mitigating unintended consequences. 
 
 \paragraph{Critiques of Technology}
 
 Prior work in STS, and later in HCI, has provided critical analyses of the risks and benefits of technology in society since at least the 1960s~\cite{sveiby2009unintended, Shilton2018ValuesAE}. According to STS scholar Winner, technology ``embodies specific forms of power and authority''~\cite{winner2017artifacts} and technologists should ``pay attention not only to the making of physical instruments and processes [...], but also the production of psychological, social, and political conditions as a part of any significant technical change''~\cite{winner2010whale}. 

 \rr{Work on the risks of technology has been published on a broad range of topics, including the Internet ~\cite{kraut1998internet}, health care information technologies~\cite{harrison2007unintended,ash2007categorizing}, mobile phones~\cite{reyns2013unintended,Moser:2016}, smart technologies~\cite{machidon2018societal}, machine learning~\cite{cabitza2017unintended}, and social media ~\cite{del2016spreading, starbird2019disinformation,starbird2017examining}.} 
 

 The examples above provide a critical lens of the role of technology in society and caution about the unknown and differential effects on societies. Historically, however, most innovation research has focused on desirable and intended consequences~\cite{Rogers1976NewPA, sveiby2009unintended}. In 2009, Sveiby and colleagues suggested that this focus could potentially be due to a ``pro-innovation bias among researchers and vested interests of funding agencies''~\cite{sveiby2009unintended}. Our literature review did not reveal whether this bias has changed in the years since. 

 However, we found many recent calls for more accountability for research innovations~\cite{metcalf2019owning,friedman2019value,Hecht2021ItsTT}. Several prominent computing conferences --- such as the Conference on Neural Information Processing System (NeurIPS ~\cite{NeurIps2021}), Annual Meetings of the Association for Computational Linguistics (ACL) ~\cite{ACL2022}, and the ACM Conference on Intelligent User Interfaces (IUI)~\cite{IUI2022} --- have begun to experiment with ways to encourage or even require researchers to state both the positive and negative potential implications of their work in all paper submissions. \rr{Recent work has made several suggestions for such broader impact statements based on an analysis of these statements in NeurIPS conference proceedings~\cite{nanayakkara2021unpacking, Liu2022ExaminingRA, Ashurst2022AIES}}. After requiring all submissions to contain a section describing the impact of the work, NeurIPS has since transitioned towards a checklist system that offers additional guidance and adaptability ~\cite{NeurIps2021}. 
%
% In 2018, the ACM Code of Ethics and Professional Conduct was revised for the first time since 1992 to address the significant advances in computing technology and the growing pervasiveness of computing in all aspects of society~\cite{Gotterbarn2018ACMCO}. There are also calls for researchers within different computing communities to accurately report the design considerations of their datasets and models~\cite{gebru2021datasheets,mitchell2019model,bender2018data,rogers2021changing} as well as the tasks that they are working on~\cite{Lindley2017ImplicationsFA,mohammad2021ethics}. 

 The HCI community has been raising awareness of UCs of computing research through dedicated publication tracks (e.g., Critical Computing at CHI) and workshops~\cite{conseuqences, unethically}. In particular, a CHI 2021 workshop explored how HCI researchers might think about and report potential negative consequences stemming from their research~\cite{conseuqences}. HCI researchers have also advocated for changes to the peer review process to reduce negative impacts of research innovations, suggesting that reviewers should routinely require that papers and proposals discuss potential adverse effects~\cite{Hecht2021ItsTT}. 
%
 Given these calls for examining the societal impacts of technology, our work explores whether researchers adopt any methods for anticipating the UCs of their own work. 

\paragraph{Anticipating and Mitigating Unintended Consequences of Technology}
 UCs are often dismissed as unavoidable because anticipating what may happen in the future can be hard~\cite{parvin2020unintended} \rr{and uncertain~\cite{Nanayakkara2020AnticipatoryEA}.} However, HCI researchers have developed ethics-focused design methods to ensure the inclusion of various stakeholders in the design process (for an overview  see~\cite{chivukula2021surveying}). One prominent example is the value-sensitive design (VSD) approach by Friedman, Kahn, and Borning~\cite{friedman2008value}, which can aid in understanding technology, its human value, and its context of use. The process aims to help product teams and researchers identify alternative approaches that better uphold their chosen values while accommodating the same constraints. 
 A number of recent proposals have sought to bridge the gap between theory and implementation by creating toolkits meant for brainstorming about a product's potential societal impacts. For example, the Envisioning Cards~\cite{EnvisioningCards} present the VSD concepts in a clear and modular fashion~\cite{Nathan2008}. In addition, stakeholder tokens also support a VSD stakeholder analysis~\cite{Yoo:2018}. Prior work in HCI has also recommended the use of Tarot Cards of Tech~\cite{Menking:2019} and the Value Cards~\cite{shen2021value} for anticipating potential UCs of specific design choices. 
 
 \rrr{Another approach for considering possible societal impacts is through design fiction~\cite{bleecker2022design, Baumer2018WhatWY}. As a form of speculative design~\cite{Lindley2016PushingTL, Dunne2013SpeculativeED},  design fiction creates a fictional future world to think through sociotechnical issues that have relevance and implications for the present~\cite{wong2017eliciting, lindley2017implications}. This practice has been used to reflect on potential downsides of public data~\cite{fiesler2019ethical}, technology design~\cite{Harrington2022AllTY}, and research prototypes~\cite{soden2019chi4evil}. More recent work developed the design fiction memos method to explore how UX practitioners engage with ethical issues and social impact in their work~\cite{wong2021using}.} 
 
 \rrr{The existing approaches to consider societal implications, however, were often assumed to be effective in practice~\cite{gray2019ethical} and might be difficult to evaluate ~\cite{Baumer2018WhatWY}. While the toolkits often target designers and practitioners as users~\cite{gray2019ethical}, applying them for research projects may pose additional complexities. We extend this line of work by inquiring into whether computer science researchers are aware of and proactively incorporate these tools in their research process. Our work also explores future design implications to support researchers to consider UCs in their research process.
 }
 %These existing approaches are not without shortcomings.} For one, they are primarily designed for practitioners; applying them to research projects \rr{in practice} may be challenging~\cite{gray2019ethical}. Nevertheless, they could serve as important tools for considering UCs. In the present work, we explore whether computer science researchers are aware of and incorporate any of these tools in their research process.}  
 
\paragraph{Reacting to Unintended Consequences of Technology}
 Not much work has investigated how practitioners and researchers react to UCs in practice. Kling's book on ``Computerization and Controversy'' shows how the power dynamics between programmers and their employers can prevent discussions of potential ethical issues in the products they work on~\cite{kling1996computerization}. As a result, computer science professionals may feel discouraged when reacting to potential or known UCs. 
 \rr{Recently, an interview study showed that the Deepfake open source contributors felt unable to control downstream uses of their software, given the core principle of open source~\cite{widder}.}
 %More recently, some researchers were found to be ``nonchalant'' toward the broader impact statement and perceive it as ``burden''~\cite{Abuhamad2020LikeAR}. 
%
 %\rr{It could well be that similar power dynamics prevent anticipating and reacting to UCs in academic settings.} 
%
 Researchers have occasionally written public posts in response to public backlash or negative press after deploying a research project~\cite{jiang_2021, openai_2022}, but it is unclear whether they also do so when an incident is less public or when it has only been anticipated (but has not materialized). We fill this gap in prior work by studying whether and how academic computer scientists react if they discover that their work may have UCs. 

\section{Methods}

 In this work, we conducted 20 semi-structured interviews to identify and understand whether and how computer science researchers from diverse sub-disciplines currently approach the potential UCs of their research innovations, what barriers they may encounter, and what design opportunities may exist to support this process. 
 
 \paragraph{Sampling and Participants.} We used purposive sampling to select researchers in computer science who were affiliated with \rr{institutions in North America with very high research activities (R1).} \rr{We focused on North American R1 institutions to reduce potential confounds related to the difference in academic culture and structures, such as funding applications and opportunities, requirements for promotion, and the structure of Ph.D. programs.} Participants were required to work on applied research that has led, or could lead, to systems used by the general public. 
 
 We recruited participants via email after reading their webpages and publications to determine whether their research met our criteria and to ensure diversity in levels of research experience. The email briefly described the research goal of finding ways to support researchers in anticipating UCs. We recruited participants until we reached a sample that satisfied our goal of interviewing computer science researchers \rr{from diverse disciplines and seniority levels} and until the interviews reached saturation. Although we attempted to sample researchers from a variety of applied sub-disciplines, our findings may not encapsulate the thoughts and actions of all computer science researchers. 

\begin{table}[t]
\footnotesize
\centering
\caption{Overview of CS researchers in the study.}
\label{table:reserachers}
\centering
\begin{tabular}{p{0.01\columnwidth}p{0.115\columnwidth}p{0.24\columnwidth}p{0.21\columnwidth}p{0.116\columnwidth}p{0.06\columnwidth}}
    \textbf{\#} & \textbf{Institution} & \textbf{Position}& \textbf{Research Area} & \textbf{Released Public Products} & \textbf{Gender}\\ 
    \toprule
     P1 &  Public & PhD Student & NLP, HCI & No & Male\\
     P2 & Private & Assistant Professor & Social Computing &Yes & Male\\
     P3 & Public & Assistant Professor & ML & No & Male\\
     P4 & Private & Associate Professor & Security, Smartphones, AI & Yes & Male\\
     P5 & Public & PhD Student & NLP & No & Male \\
     P6 & Public & Full Professor & NLP & Yes & Female\\
     P7 & Public & Full Professor & Robotics & Yes & Male\\
     P8 & Public & PhD Student & Computer Vision, ML & No & Male\\
     P9 & Private & PhD Student & AI & No & Male\\
     P10 &Private & PhD Student & AR, VR & No & Female \\
     P11 & Private & Assistant Professor & Brain-Computer Interfaces & No & Female\\
     P12 & Private & Postdoc & Accessibility & Yes & Female \\
     P13 & Private & PhD Student & Fabrication, Sensing & Yes & Male\\
     P14 & Private & Associate Professor &  HCI & Yes & Female \\
     P15 & Public & Assistant Professor & AR, Accessibility & Yes & Male\\
     P16 & Public & PhD Student & CS Education, HCI & No & Female \\
     P17 & Public & PhD Student & CS Education & Yes & Male\\
     P18 & Public & Assistant Professor & AI, Robotics & No & Male\\
     P19 & Public & Assistant Professor & Security & Yes & Male\\
     P20 & Public & PhD Student & NLP & Yes & Female\\
    \bottomrule
\end{tabular}
\end{table}

 Our final sample included 20 computer science researchers (7 female, 13 male) from 10 different academic institutions across North America. All participants had built systems as part of their research, and 9 had released one or more systems as (part of) a public product. Participants held various academic positions in their respective computer science departments (see Table~\ref{table:reserachers}):  10 participants were Ph.D. students or postdocs, and the remaining 10 were assistant, associate, or full professors. Our participants worked in a variety of research areas: 9 participants described their work as being mainly in AI or related areas (e.g. CV, ML, NLP). The remaining participants worked on accessibility, AR/VR, CS education, hardware, social computing, robotics, and security, or a combination of the above. All participants had industry experience, meaning that they have either collaborated, interned, or obtained full-time positions in industry while working toward their research projects.

\paragraph{Interview Protocol.} 

\rr{We prefaced our interviews by loosely defining "unintended consequences" as both desirable and undesirable outcomes of one's research, allowing participants to further elaborate on the term's meaning.} 
%\rr{We prefaced our interviews with a definition of ``unintended consequences of technology,'' which we loosely defined as unforeseen, desirable and undesirable outcomes of digital technology innovations. }%one's research to allow participants to further elaborate on the meaning of the word over the course of the interview.}
 We then divided our interviews into five sections: (1) participant research experience (e.g., research areas, educational and professional background); (2) prior experiences with UCs (e.g., from community norms in their sub-disciplines about considering UCs and/or their own research products have resulted in UCs); (3) understanding whether, when, and how they consider UCs in the research process; (4) understanding barriers to considering UCs in the research process; (5) understanding where researchers perceive opportunities to augment and improve the process to consider UCs. 
 %
 To avoid response bias, we started by explaining that the topic of UCs is relatively new to computer science. In addition to their own experience, we asked questions about habits and norms in researchers' labs and communities to better understand the context for their opinions and avoid participants feeling accused or put on the spot. We used additional questions to probe three topics that came up repeatedly: understanding whether researchers actively anticipated UCs, understanding what may hinder them from doing so, and understanding the design opportunities to support them in anticipating UCs. See Supplementary Materials for the complete list of interview questions. 
 
 Nineteen interviews were conducted remotely over Zoom, and one was conducted in person. All participants consented to being audio-recorded. Each interview was between 30-45 minutes long. We made a financial donation of \$15 per participant to a COVID-19 relief fund to compensate each participant for their time. The study was determined as exempt by our Institutional Review Board (IRB). 

\paragraph{Analysis.} 
% development code book, coding process, affinity diagramming
Our research team used an inductive thematic analysis process~\cite{braun2006using} where two researchers individually reviewed and conducted open coding on two interviews. Next, three researchers met to create and discuss the first draft of the codebook. Two members of the research team then independently coded several more interviews and refined or added them to the codebook, which was discussed with the full research team. Once consensus was reached, all interviews were re-coded using the revised codebook. 

The final codebook contained 12 top-level codes relevant to how researchers have thought about, experienced, and responded to UCs; it also included codes related to attitudes towards UCs and support researchers needed to think about UCs. Finally, three researchers used affinity diagramming to develop themes based on our codes. %these codes to organize our findings into three open questions. 
\rr{Although we discussed both positive and negative UCs in our interviews, the nature of our research led us to focus on participants' reports of negative UCs.} We slightly edited some of the quotes in Section \ref{Results} for readability. 

\paragraph{Positionality.}

 We acknowledge that our academic and professional backgrounds shape our perspectives on this topic. One author teaches computer ethics at an R1 institution. Collectively, we are US-based researchers at two R1 universities and a large US-based multinational corporation. Our academic backgrounds are in Computer Science, primarily as HCI researchers. 
\section{Results}\label{sec:Background}

And, so, since 2016, researchers have been probing the submitted methods, and in 2022 NIST published the final 10: ASCON, Elephant, GIFT-COFB, Grain128-AEAD, ISAP, Photon-Beetle, Romulus, Sparkle, TinyJambu, and Xoodyak. A particular focus is on the security of the methods, along with their performance on low-cost FPGAs/embedded processes and their robustness against side-channel attacks.

The current set of benchmarks includes running on an Arduino Uno R3 (AVR ARmega 328P), Arduino Nano Every (AVR ARmega 4809), Arduino MKR Zero (ARM Cortex M10+) and Arduino Nano 33 BLE (ARM Cortex M4F). These are just 8-bit processors and fit into an Arduino board. Along with their processing limitations, they are also limited in their memory footprint (to run code and also to store it). The lightweight cryptography method must thus overcome these limitations, and still, be secure and provide a good performance level. Running AES in block modes on these devices is often not possible, as there is not enough resources. Overall we use a benchmark for encryption — with AEAD (Authenticated Encryption with Additional Data) and for hashing. With AEAD we add extra information — such as the session ID — into the encryption process. This type of method can bind the encryption to a specific stream.



\subsection{ARM Cortex M3}

In Table \ref{tab:table01} [1], we see a sample run using an Arduino Due with an ARM Cortex M3 running at 84MHz. The tests are taken in comparison with the ChaCha20 stream cipher and defined for AEAD, and where the higher the value the better the performance. We can see that Sparkle, Xoodyak and ASCON are the fastest of all. Sparkle has a 100\% improvement, and Xoodyak gives a 60\% increase in speed over ChaCha20. Elephant, ISAP and PHOTON-Beetle have the worst performance for encryption (with around 1/20th of the speed of ChaCha20).

\begin{table*}
\caption{\label{tab:table01} Arduino Due with an ARM Cortex M3 running at 84MHz for encryption against ChaCha20 \cite{light01}}
\centering
\begin{tabular}{|l|l|l|l|l|l|l|l|l|}
\hline
Algorithm&Key Bits&Nonce Bits&Tag Bits&Encrypt 128~B&Decrypt 128~B&Encrypt 16~B &Decrypt 16~B&Aver
\\ \hline \hline
Schwaemm128-128 (SPARKLE)	&128	&128&	128	&1.6	&1.58	&2.84	&2.39	&2.01\\
Xoodyak 	&128	&128	&128	&1.66	&1.51	&1.73	&1.6	&1.62\\
ASCON-128	&128	&128&	128	&1.54	&1.44	&1.78	&1.68	&1.61\\
TinyJAMBU-128 	&128	&96	&64	&0.93	&0.95	&1.63	&1.61	&1.21\\
GIFT-COFB	&128	&128	&128	&1.01	&1.01	&1.16	&1.15	&1.08\\
Grain-128AEAD	&128	&96	&64	&0.26	&0.26	&0.56	&0.56	&0.37\\
Romulus-M1	&128	&128	&128	&0.1	&0.11	&0.15	&0.16	&0.13\\
PHOTON-Beetle-AEAD-ENC-128	&128	&128	&128	&0.06	&0.07	&0.11	&0.12	&0.08\\
ISAP-A-128	&128	&128	&128	&0.08	&0.08	&0.03	&0.04	&0.05\\
Delirium (Elephant)	&128	&96	&128	&0.04	&0.05	&0.06	&0.07	&0.05\\
\hline
\end{tabular}
\end{table*}

Not all of the finalists can do hash functions. Table \ref{tab:table02} outlines these.

\begin{table*}
\caption{\label{tab:table02} Arduino Due with an ARM Cortex M3 running at 84MHz for hashing against BLAKE2s \cite{NISTgov}}
\centering
\begin{tabular}{|l|l|l|l|l|l|}
\hline
Algorithm	& Hash Bits	& 1024 bytes	& 128 bytes	& 16 bytes	& Average\\
\hline\hline
Esch256 (SPARKLE) 	&256	&0.89	&0.78	&1.5	&1.06\\
Xoodyak 	&256	&0.71	&0.65	&1.43	&0.93\\
GIMLI-24-HASH	&256	&0.54	&0.47	&0.86	&0.62\\
ASCON-HASH 	&256	&0.51	&0.41	&0.63	&0.52\\
PHOTON-Beetle-HASH	&256	&0.01	&0.01	&0.05	&0.02\\
\hline
\end{tabular}
\end{table*}


Again, we see Sparkle and Xoodyak in the lead, with Sparkle actually faster in the test than BLAKE2s, and Xoodyak just a little bit slower. ASCON has a weaker performance, and PHOTON-Beetle is relatively slow. For all the tests, the ranking for authenticated encryption is (and where the higher the rank, the better):

14 SPARKLE
12 Xoodyak
12 ASCON
10 TinyJAMBU
9 GIFT-COFB, Gimli
4 Grain-128AEAD,KNOT
0 Elephant, ISAP, PHOTON-Beetle

and for hashing SPARKLE and Xoodyak are ranked the same:

7 SPARKLE, Xoodyak 5 Gimli 3 ASCON 0 PHOTON-Beetle

\subsection{Uno Nano performance}

For AEAD on Uno Nano Every [2], the benchmark is against AES GCM. We can see in \ref{tab:table03} , that SPARKLE is 4.7 times faster than AES GCM for 128-bit data sizes, and Xoodyak comes in second with a 3.3 times improvement over AES GCM. When it comes to 8-bit data sizes TinyJambu actually is the fastest, but where Sparkle and Xoodyak still perform well. PHOTON-Beetle, Grain128 and ISAP do not do well, and only slightly improve on AES GCM. In fact, Grain128 and ISAP are actually slower than AES GCM.




\begin{table*}
\caption{\label{tab:table03} Uno Nano for AEAD against AES GCM and showing cycles (showing fastest of the method)}
\centering
\begin{tabular}{|l|l|l|l|l|l|l|l|l|l|l|}
\hline
Algorithm&Impl.&Primary&Flag&Size&Enc(0:8)&Dec(0:8)&Enc(128:129)&Dec(128:128)&Bench.(128)&Bench.(8)
\\ \hline \hline
sparkle       &rhys	          &yes&	   O3	&12290	&1276	&1316	&4648    &5072  &4.7  &3.3\\
Xoodyak       &XKCP-AVR8	  &yes&    O3	&4560	&2596	&2608	&7184    &7128  &3.3  &1.6\\
knot	      &$avr8_speed$   &no&	   Os	&1664	&2124	&2140	&8144    &8160  &2.9  &2\\
ascon 	      &rhys	          &no&     O3	&5180	&1240	&1284	&8056    &8488  &2.8  &3.3\\
GIFT-COFB     &rhys	          &yes&    O1	&23312	&1852	&1892	&8220    &8776  &2.7  &2.2\\
saeaes	      &ref	          &no&     O3	&17062	&1208	&1212	&8992    &9004  &2.6  &3.4\\
hyena	      &rhys           &yes&    O3	&293860	&1912	&1964	&8960    &9396  &2.5  &2.2\\
elephant      &rhys           &no&     O3	&13106	&1924	&1948	&9260    &9796  &2.4  &2.2\\
estate	      &ref            &yes&    O3	&9434 	&1424	&1448	&10276   &10292 &2.3  &2.9\\
romulus	      &rhys           &no&     O3	&19346 	&1632	&1676	&10152   &10568 &2.2  &2.5\\
spook	      &rhys           &no&     O3	&12942 	&2984	&2968	&10272   &10708 &2.2  &1.4\\
tinyjambu     &rhys           &yes&    O3	&9174 	&1232	&1288	&10364   &10888 &2.2  &3.4\\
subterranean  &rhys           &yes&    Os	&6042 	&3372	&3460	&10288   &10944 &2.2  &1.2\\
orange        &rhys           &yes&    O3	&12140 	&2500	&2536	&11200   &11620 &2    &1.7\\
gimli         &rhys           &yes&    O3	&21272 	&1920	&1956	&11944   &12360 &1.9  &2.2\\
skinny        &rhys           &no&     O1	&12452 	&1604	&1644	&12960   &14372 &1.7  &2.6\\
photon-beetle & $avr8_speed$  &yes&    Os	&3536 	&2444	&2472	&20076   &20092 &1.2  &1.7\\
{\bf reference}&rhys          &yes&    O2	&7874 	&4152	&4156	&23812   &23764 &1    &1\\
grain128aead  &rhys           &yes&    O2	&9532 	&3992	&3980	&30396   &30124 &0.8  &1\\
isap          &rhys           &no&     O2	&3824 	&20212	&20256	&42936   &43372 &0.5  &0.2\\
\hline
\end{tabular}
\end{table*}

And so for AEAD  (performance) the ordering is

1. Sparkle
2. Xoodyak
3. Ascon
4. GIFT-COFB.
5. Elephant.
6. Romulus.
7. Tiny Jambu.
8. PHOTON-Beetle.
9. Grain128
10. ISAP.

For hashing on an Uno Nano Every, Table \ref{tab:table04} shows a similar performance level as to the ARM Cortex M3 assessment. In this case, the benchmark hash is SHA-256, and we can see that it takes Sparkle twice as many cycles for a 128-bit hash, and 2.9 times for Xoodyak. PHOTON-Beetle is way behind with a 128-bit hash and which is 17.4 times slower than SHA-256. That said, though, PHOTON-Beetle could be more focused on reducing power consumption rather than speed. GIMLI and SKINNY are included to show a comparison with well-designed methods in lightweight hashing. It can be seen that every method beats SKINNY, but only SPARKLE and Xoodyak beat GIMLI.


\begin{table*}
\caption{\label{tab:table04}  Uno Nano for hashing against SHA-256 and showing cycles (showing fastest of the method for hashing)}
\centering
\begin{tabular}{|l|l|l|l|l|l|l|l|l|l|l|}
\hline
Algorithm&Impl.&Primary&Flag&Size&h(8)&h(16)&h(32)&h(64)&h(128)&Benchmark
\\ \hline \hline
{\bf reference}&$nacl_ref$    &yes&    O3	&18774 	&768	&768	&772     &1364  &1968  &1\\
sparkle       &rhys	          &yes&	   O1	&7912	&1036	&1036	&1468    &2272  &3884  &2\\
Xoodyak       &XKCP-AVR8	  &yes&    O3	&2604	&1284	&1288	&1924    &3192  &5732  &2.9\\
gimli         &rhys           &yes&    O3	&19554 	&1284	&1920	&2544    &3804  &6312  &3.2\\
ascon 	      &rhys	          &yes&    O3	&2178	&2972	&3552	&4736    &7088  &11784 &6\\
drygascon     &rhys           &no&	   O3	&15500	&4604	&4600	&6540    &10360 &17912 &9.1\\
photon-beetle & $avr8_speed$  &yes&    O3	&2948 	&2372	&2364	&6940    &16084 &34172 &17.4\\
skinny        &rhys           &yes&    O2	&9784 	&7048	&10556	&13976   &20952 &34896 &17.7\\
\hline
\end{tabular}
\end{table*}


And so for hashing (performance) the ordering is:
\begin{enumerate}
    \item Sparkle.
    \item Xoodyak.
    \item Ascon
    \item PHOTON-Beetle. 
\end{enumerate}
    
\section{Discussion and Conclusion}
To conclude, we propose \textbf{\nickname{}}, the first generalizable human NeRF model that recovers animatable 3D humans from single human image inputs.
To render high-fidelity 3D humans, \nickname{} proposes to learn both global and local details from the bank of 3D-aware hierarchical features comprising global features, point-level features, and pixel-aligned features. 
By using a feature fusion transformer, \nickname{} successfully enhances the information from the 2D observation and complements the information missing from the input image. 
% In addition, by modelling the neural radiance field in the canonical space, \nickname{} can animate 3D humans with free poses.
On four large-scale human datasets, \nickname{} achieves state-of-the-art performance and renders high-fidelity images in both novel views and poses.

\vspace{1.75mm}
\noindent \textbf{Limitations:}
1) There still exists visible artifacts in target renderings when some body parts are occluded in the observation space. 
A better feature presentation like occlusion-aware features may be explored to solve this issue. 
2) How to complement the information missing from single image input remains a challenging problem.
\nickname{} starts from the reconstruction view and can only render deterministic results when predicting novel views.
One potential direction is to investigate the use of conditional generative models to diversely generate higher quality novel views.

\vspace{1.75mm}
\noindent \textbf{Potential Negative Societal Impacts:} 
\nickname{} can be misused to create fake images or videos of real humans and cause negative social impacts.


\section{Limitations and Future Work}
 Our work provides insight into the barriers that prevent computer science researchers from considering and addressing UCs of their work. One limitation of our work is the relatively limited sample size and diversity of participants in this study. We selectively spoke to North American computer science researchers from R1 research universities. 
 \rr{Because academia can be structured differently across countries, our findings may not reflect either the resources or challenges that academics in other parts of the world encounter when considering societal impacts of their work. The focus on R1 institutions additionally means that we cannot conclude the habits and challenges in considering UCs are similar to R2 or special focus institutions.}  Moreover, because we employed purposive sampling, our findings may not generalize to all computer science researchers; in particular, we suspect there may be differences if researchers come from marginalized or minority backgrounds, if their research addresses ethics, or if their research is further removed from applications. Future work is needed to investigate whether our findings and suggestions for supporting computer science researchers in anticipating UCs generalize to other research institutions within the US and to other countries.

 An additional limitation is our focus on academic computer science researchers whose applied research products have led to systems used by the general public. \rr{This might have overlooked opinions by researchers in industry who may be subject to different policies and organizational structures within their companies.} We might have missed opinions by researchers in other fields that are deeply affected by UCs in technology. Future work should explore how computer science researchers in their specialized sub-disciplines with varying demographic backgrounds, work experiences, and research experiences enrich our findings.

 Our results are also impacted by the possibility of response bias, a common issue in interview studies. Given the heightened awareness of societal implications and the blame associated with it, the risk is that participants may have appeared more concerned about UCs than they actually are and may have downplayed any UCs that they have experienced themselves. We did not perceive this as an issue in their responses, but our findings on the perceived importance of proactively considering UCs should still be taken with a grain of salt. Future work could build on these findings with an anonymous survey or with longitudinal studies on people's motivations and actions for considering UCs.
 
 Additionally, our work mainly focuses on how researchers anticipate unintended consequences. Future work should explore the opportunities and challenges that researchers encounter when reacting to unintended consequences in greater detail. Based on our initial findings from this work, we assume that researchers face similar structural and knowledge barriers when responding to unintended consequences during the research process.
 % Additionally, only a minority of our participants had uncovered or experienced UCs in their prior work. Based on our initial findings from this work, we assume that researchers face similar structural and knowledge barriers when responding to UCs during the research process. However, future work should explore the opportunities and challenges that researchers encounter when reacting to UCs in greater detail. 
This work presented a simulation approach that centers around finding iteratively an approximation of the evolution of the algebraic variables in the power system \glspl{DAE}. The approximation of the dynamic state evolutions by NNs, instead of classical explicit numerical integration schemes, allows larger time-steps to be realized while being fast to execute. This work aimed at providing a proof of concept, it is foreseeable that future work on this method shares many typical questions with established \gls{DAE} solvers, hence, by applying various existing techniques the computational performance and scalability of the approach should improve significantly.

\begin{acks}
We thank our participants and the anonymous reviewers for their valuable feedback and suggestions. We also thank Sandy Kaplan and Mary Peng for their help revising this paper. This work was funded by the National Science Foundation under award IIS-2006104. Any opinions, findings, conclusions, or recommendations expressed in our work are those of the authors and do not necessarily reflect those of the supporter.
\end{acks}

\bibliographystyle{ACM-Reference-Format}
\bibliography{bibliography}

\appendix

\end{document}
\endinput
%%
%% End of file `sample-authordraft.tex'.
