\section{Limitations and Future Work}
 Our work provides insight into the barriers that prevent computer science researchers from considering and addressing UCs of their work. One limitation of our work is the relatively limited sample size and diversity of participants in this study. We selectively spoke to North American computer science researchers from R1 research universities. 
 \rr{Because academia can be structured differently across countries, our findings may not reflect either the resources or challenges that academics in other parts of the world encounter when considering societal impacts of their work. The focus on R1 institutions additionally means that we cannot conclude the habits and challenges in considering UCs are similar to R2 or special focus institutions.}  Moreover, because we employed purposive sampling, our findings may not generalize to all computer science researchers; in particular, we suspect there may be differences if researchers come from marginalized or minority backgrounds, if their research addresses ethics, or if their research is further removed from applications. Future work is needed to investigate whether our findings and suggestions for supporting computer science researchers in anticipating UCs generalize to other research institutions within the US and to other countries.

 An additional limitation is our focus on academic computer science researchers whose applied research products have led to systems used by the general public. \rr{This might have overlooked opinions by researchers in industry who may be subject to different policies and organizational structures within their companies.} We might have missed opinions by researchers in other fields that are deeply affected by UCs in technology. Future work should explore how computer science researchers in their specialized sub-disciplines with varying demographic backgrounds, work experiences, and research experiences enrich our findings.

 Our results are also impacted by the possibility of response bias, a common issue in interview studies. Given the heightened awareness of societal implications and the blame associated with it, the risk is that participants may have appeared more concerned about UCs than they actually are and may have downplayed any UCs that they have experienced themselves. We did not perceive this as an issue in their responses, but our findings on the perceived importance of proactively considering UCs should still be taken with a grain of salt. Future work could build on these findings with an anonymous survey or with longitudinal studies on people's motivations and actions for considering UCs.
 
 Additionally, our work mainly focuses on how researchers anticipate unintended consequences. Future work should explore the opportunities and challenges that researchers encounter when reacting to unintended consequences in greater detail. Based on our initial findings from this work, we assume that researchers face similar structural and knowledge barriers when responding to unintended consequences during the research process.
 % Additionally, only a minority of our participants had uncovered or experienced UCs in their prior work. Based on our initial findings from this work, we assume that researchers face similar structural and knowledge barriers when responding to UCs during the research process. However, future work should explore the opportunities and challenges that researchers encounter when reacting to UCs in greater detail. 