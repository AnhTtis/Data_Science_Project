\section{Methods}

 In this work, we conducted 20 semi-structured interviews to identify and understand whether and how computer science researchers from diverse sub-disciplines currently approach the potential UCs of their research innovations, what barriers they may encounter, and what design opportunities may exist to support this process. 
 
 \paragraph{Sampling and Participants.} We used purposive sampling to select researchers in computer science who were affiliated with \rr{institutions in North America with very high research activities (R1).} \rr{We focused on North American R1 institutions to reduce potential confounds related to the difference in academic culture and structures, such as funding applications and opportunities, requirements for promotion, and the structure of Ph.D. programs.} Participants were required to work on applied research that has led, or could lead, to systems used by the general public. 
 
 We recruited participants via email after reading their webpages and publications to determine whether their research met our criteria and to ensure diversity in levels of research experience. The email briefly described the research goal of finding ways to support researchers in anticipating UCs. We recruited participants until we reached a sample that satisfied our goal of interviewing computer science researchers \rr{from diverse disciplines and seniority levels} and until the interviews reached saturation. Although we attempted to sample researchers from a variety of applied sub-disciplines, our findings may not encapsulate the thoughts and actions of all computer science researchers. 

\begin{table}[t]
\footnotesize
\centering
\caption{Overview of CS researchers in the study.}
\label{table:reserachers}
\centering
\begin{tabular}{p{0.01\columnwidth}p{0.115\columnwidth}p{0.24\columnwidth}p{0.21\columnwidth}p{0.116\columnwidth}p{0.06\columnwidth}}
    \textbf{\#} & \textbf{Institution} & \textbf{Position}& \textbf{Research Area} & \textbf{Released Public Products} & \textbf{Gender}\\ 
    \toprule
     P1 &  Public & PhD Student & NLP, HCI & No & Male\\
     P2 & Private & Assistant Professor & Social Computing &Yes & Male\\
     P3 & Public & Assistant Professor & ML & No & Male\\
     P4 & Private & Associate Professor & Security, Smartphones, AI & Yes & Male\\
     P5 & Public & PhD Student & NLP & No & Male \\
     P6 & Public & Full Professor & NLP & Yes & Female\\
     P7 & Public & Full Professor & Robotics & Yes & Male\\
     P8 & Public & PhD Student & Computer Vision, ML & No & Male\\
     P9 & Private & PhD Student & AI & No & Male\\
     P10 &Private & PhD Student & AR, VR & No & Female \\
     P11 & Private & Assistant Professor & Brain-Computer Interfaces & No & Female\\
     P12 & Private & Postdoc & Accessibility & Yes & Female \\
     P13 & Private & PhD Student & Fabrication, Sensing & Yes & Male\\
     P14 & Private & Associate Professor &  HCI & Yes & Female \\
     P15 & Public & Assistant Professor & AR, Accessibility & Yes & Male\\
     P16 & Public & PhD Student & CS Education, HCI & No & Female \\
     P17 & Public & PhD Student & CS Education & Yes & Male\\
     P18 & Public & Assistant Professor & AI, Robotics & No & Male\\
     P19 & Public & Assistant Professor & Security & Yes & Male\\
     P20 & Public & PhD Student & NLP & Yes & Female\\
    \bottomrule
\end{tabular}
\end{table}

 Our final sample included 20 computer science researchers (7 female, 13 male) from 10 different academic institutions across North America. All participants had built systems as part of their research, and 9 had released one or more systems as (part of) a public product. Participants held various academic positions in their respective computer science departments (see Table~\ref{table:reserachers}):  10 participants were Ph.D. students or postdocs, and the remaining 10 were assistant, associate, or full professors. Our participants worked in a variety of research areas: 9 participants described their work as being mainly in AI or related areas (e.g. CV, ML, NLP). The remaining participants worked on accessibility, AR/VR, CS education, hardware, social computing, robotics, and security, or a combination of the above. All participants had industry experience, meaning that they have either collaborated, interned, or obtained full-time positions in industry while working toward their research projects.

\paragraph{Interview Protocol.} 

\rr{We prefaced our interviews by loosely defining "unintended consequences" as both desirable and undesirable outcomes of one's research, allowing participants to further elaborate on the term's meaning.} 
%\rr{We prefaced our interviews with a definition of ``unintended consequences of technology,'' which we loosely defined as unforeseen, desirable and undesirable outcomes of digital technology innovations. }%one's research to allow participants to further elaborate on the meaning of the word over the course of the interview.}
 We then divided our interviews into five sections: (1) participant research experience (e.g., research areas, educational and professional background); (2) prior experiences with UCs (e.g., from community norms in their sub-disciplines about considering UCs and/or their own research products have resulted in UCs); (3) understanding whether, when, and how they consider UCs in the research process; (4) understanding barriers to considering UCs in the research process; (5) understanding where researchers perceive opportunities to augment and improve the process to consider UCs. 
 %
 To avoid response bias, we started by explaining that the topic of UCs is relatively new to computer science. In addition to their own experience, we asked questions about habits and norms in researchers' labs and communities to better understand the context for their opinions and avoid participants feeling accused or put on the spot. We used additional questions to probe three topics that came up repeatedly: understanding whether researchers actively anticipated UCs, understanding what may hinder them from doing so, and understanding the design opportunities to support them in anticipating UCs. See Supplementary Materials for the complete list of interview questions. 
 
 Nineteen interviews were conducted remotely over Zoom, and one was conducted in person. All participants consented to being audio-recorded. Each interview was between 30-45 minutes long. We made a financial donation of \$15 per participant to a COVID-19 relief fund to compensate each participant for their time. The study was determined as exempt by our Institutional Review Board (IRB). 

\paragraph{Analysis.} 
% development code book, coding process, affinity diagramming
Our research team used an inductive thematic analysis process~\cite{braun2006using} where two researchers individually reviewed and conducted open coding on two interviews. Next, three researchers met to create and discuss the first draft of the codebook. Two members of the research team then independently coded several more interviews and refined or added them to the codebook, which was discussed with the full research team. Once consensus was reached, all interviews were re-coded using the revised codebook. 

The final codebook contained 12 top-level codes relevant to how researchers have thought about, experienced, and responded to UCs; it also included codes related to attitudes towards UCs and support researchers needed to think about UCs. Finally, three researchers used affinity diagramming to develop themes based on our codes. %these codes to organize our findings into three open questions. 
\rr{Although we discussed both positive and negative UCs in our interviews, the nature of our research led us to focus on participants' reports of negative UCs.} We slightly edited some of the quotes in Section \ref{Results} for readability. 

\paragraph{Positionality.}

 We acknowledge that our academic and professional backgrounds shape our perspectives on this topic. One author teaches computer ethics at an R1 institution. Collectively, we are US-based researchers at two R1 universities and a large US-based multinational corporation. Our academic backgrounds are in Computer Science, primarily as HCI researchers. 