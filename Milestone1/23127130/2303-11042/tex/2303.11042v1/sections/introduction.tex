\section{Introduction}
%\begin{itemize}
%    \item Should inform the audience about my efforts
%    \item Tell do I want to tell the world about this work
%    \item Tell why this specific work is worth printing in the archival literature
%    \item Tell what the reader has to look forward to if continuing reading
%    \item Should answer the questions "Why, and What For (Four)?"
%\end{itemize}

% Starting with a teaser, a growing problem
%There is a steady growth in the number of people requiring acute hospital care. This development, combined with 
Increasingly scarce hospital resources challenge (often oversaturated) hospital wards, with a negative impact on the quality of health care at the hospitals~\cite{af2020association}
Models for predicting the remaining time of patient admissions, i.e., patient length of stay (LOS), could be invaluable for healthcare facilities to plan the availability of beds, staff, and other essential resources. For instance, automatic prediction of discharge time could be used in administrative planning systems for preemptively freeing in-hospital resources to alleviate hospital ward oversaturation~\cite{stone2022systematic}. However, LOS prediction is a challenging problem, requiring methods for handling missing data~\cite{sha2017interpretable} and integration of temporal event dependencies.

%previous work - tabular
Previous work on LOS prediction models patient hospitalizations using tabular data with imputation techniques for replacing missing values~\cite{bacchi2022machine}. 
While tabular data is the most common data representation in Machine Learning (ML) models, it has several drawbacks. Among others, it does not provide immediate support for integrating the temporal dependencies between observations, such as the order of treatments, or the time of conducted procedures. Moreover, standard ML techniques for tabular data, such as artificial neural networks (ANN), gradient boosting (GB), and support vector machines (SVMs), require complete data, hence often relying on imputation techniques when data is incomplete. However, missing data observations in healthcare data are often not missing at random (NMAR), meaning that the mere fact that an observation is missing is in itself important information~\cite{li2021imputation}.

%Our contribution -sequential
To alleviate the problem of temporal dependencies and missing data, attention models, also known as transformers, have recently been investigated for Electronic Health Record (EHR) data formatted as sequences of medical events~\cite{lequertier2021hospital}. However, embedding-based transformer approaches have, to the best of our knowledge, not previously been applied to LOS prediction. This work examines how attention models, can be utilized for this task.

%Previous transformer models  - the gap
Attention models using self-attention alleviate the inefficiency of recurrence networks for long sequences~\cite{song2018attend}. However, they still capture significant sequential information by learning from the order of tokens in the sequence. In medical data, multiple observations may be given the same timestamp, with no meaning assigned to their individual order within the corresponding event. For example, a blood panel drawn from a patient contains several individual measurements whose internal order is insignificant. 
%Our contribution - modified transformer model
Based on layers of transformer encoders, we propose a revised attention model, henceforth called Medic-BERT (M-BERT), based on the Natural Language Processing (NLP) model BERT~\cite{bert19} and its revised semi-supervised training method. We employ the model for LOS prediction based on sequences of patient specific medical events happening during hospitalization and which exhibit the event concurrences common in patient data. We evaluate our method on a cohort of more than $45k$ patient admissions from a large Danish hospital with diverse medical events, such as vital measurements, medication administration, laboratory tests, and conducted procedures.

The rest of this paper is structured as follows. Section \ref{sec:related} reviews related work on LOS prediction. Section \ref{sec:method} presents our model for representing hospitalizations as event sequences and our proposed M-BERT model for this unique data. Section \ref{sec:data} describes the large dataset used to evaluate this work and Section \ref{sec:eval} the evaluation and its results. We conclude with Section \ref{sec:conclusion}. This work is an extended version of Hansen et al.~\cite{aime2023hansen} published at AIME 2023. 

% Integrate the following
%There are several unanswered problems in hospital length of stay prediction. One of the main challenges is the complex and dynamic nature of the healthcare system, which can make it difficult to accurately predict LOS. Other factors that can affect LOS include the availability of beds, staffing levels, and the availability of medical resources.

%Additionally, LOS prediction is often affected by the quality and completeness of the data available. For example, data on patient characteristics, medical conditions, and treatment procedures may be incomplete or inconsistent, which can make it difficult to accurately predict LOS.
