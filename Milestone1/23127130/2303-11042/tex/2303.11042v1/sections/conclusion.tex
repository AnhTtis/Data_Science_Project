\section{Conclusion}\label{sec:conclusion}
% Make sure to 

We have presented a novel approach for predicting LOS by modeling patient information as event sequences. Our approach adapts the transformer machine-learning approach for sequence prediction, which is able to handle the unique features of medical event sequences, namely grouped events and a variety of data types. Our empirical evaluation on a large cohort of emergency care patients from a Danish hospital demonstrates that our model can achieve high accuracy on various LOS problems, 
%predicting categorical and continuous LOS and 
while outperforming traditional non-sequence machine learning approaches.
Future work could include pre-training of the transformer-based model on a medical task to further improve its performance. Overall, the proposed approach has the potential to improve resource allocation and support decision making in healthcare organizations by providing accurate predictions of LOS. All experimental code is available online\footnoteref{footnote:online_appendix}.
% Here we describe our conclusion on the findings and comment on future work.

% Future work. Use Pre-training to learn general embeddings for solving tasks besides LOS.

%  are able to better learn from the  could help pave the way forward in  

% History is big part of the sequence, embed this into its' own space as a single token in the sequence

% 

%To investigate the difference in performance for different times of prediction, we fine-tune a pre-trained model on patient sequences for four different settings. We investigate the performance at the time of admission, where only historic patient information is available, after 6 hours, after 12 hours and after 24 hours of admission.   

%Furthermore, we investigate the performance of patient sequences for the individual modalities as summarized in Table \missing, to investigate what modalities are most important for predicting hospitalization length of stay. 

%Furthermore, to investigate our idea of integrating patient medical history by pre-pending a patient medical history string as illustrated in \missing, to the patients hospitalization, we investigate settings where this information is added to the event string, and where it is omitted. 

%We also investigate 30 day mortality prediction using full patient information.

%What is the effect of actually using the measurements from vital measurements and laboratory tests instead of only the knowledge that some event happened. 
