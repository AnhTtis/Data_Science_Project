\section{Data}
\label{sec:data}
% Explain the Danish dataset we are using for experimentation, We are working with acute patients
We compiled a Danish dataset of patients admitted to a large hospital in northern Jutland from the period 2018-2021 to investigate transformer models for LOS prediction. The dataset consists of $48,177$ emergency care patients with admissions longer than one day. Figure~\ref{fig:los_histograms} illustrates the distribution of the remaining length of stay for the patient cohort. In a clinical setting of resource allocation for emergency care patients, we are mostly interested in the planning of patient care for a couple of weeks in advance. Hence, we clip long patient stays to 30 days of admission to better fit and optimize for the clinical setting, as we are not interested in precisely predicting the long-tail distribution of the data. 

% $188,169$
In Denmark, each person can be uniquely identified by an identification number from the central person register (CPR) henceforth referred to as a CPR-number. As medical events are each associated with a unique CPR-number, it is possible to connect the medical data pertaining to a patient from disparate databases. As mentioned in Section~\ref{subsec:event_sequences}, we divide patient medical data into historic information and admission-specific information. Historic information includes prescriptions, comorbidities, and mode and time of hospitalization, whereas admission information includes laboratory tests, vital measurements, hospital-administered medicine, and procedure codes.

\begin{figure}[tb]
  \centering
  \subfloat[Histogram of length of stay.]{\includegraphics[width=0.47\textwidth]{figures/los_bins_24.pdf}\label{fig:los_hist}}
  \hfill
  \subfloat[Categorical length of stay.]{\includegraphics[width=0.47\textwidth]{figures/los_cat_24_hours.pdf}\label{fig:los_cat}}
  \caption{Illustrations for length of stay distribution over patient hospitalizations.}
  \label{fig:los_histograms}
\end{figure}

\begin{wraptable}{r}{6.6cm}
    \caption{Concept types with their occurrences in the dataset.}
    \label{tab:concept_types}
	\begin{tabular}{p{2.5cm}p{1.7cm}l}
        \textbf{Event Type}  & \textbf{Tokens} & \textbf{Occurrences}         \\ \hline
            Lab Tests      & $748$                         & $2,774,790$ \\
            Vital Measures & $22$                          & $837,931$ \\
            Medication     & $1,441$                       & $376,591$ \\
            Procedures     & $2049$                        & $247,924$ \\
            History        & $81$                          & $1,880,580$	
        \end{tabular}
\end{wraptable}

Danish laboratory tests are coded using the Nomenclature for Properties and Units (NPU) terminology~\cite{npu2009}. NPU ensures that laboratory tests are standardized and patient examinations can be used and understood by all clinicians. 
%The Danish version of NPU contains more than $29$k distinct concepts; however, in practice, most codes are only rarely used. 
As detailed in Section~\ref{subsec:event_measurements}, we integrate the semantics of laboratory test results as part of the event token by appending either high, low, or normal with respect to the patient demographics to the laboratory event token, thus encoding the meaning of the result. Vital measurements are extracted from a system called Clinical Suite and consist of the 7 most common vital observations, including temperature, oxygen saturation, BMI, pulse, respiration rate, blood pressure (systolic and diastolic), and oxygen supplement. As with laboratory tests, patient-specific thresholds are used to encode the meaning of the result into event tokens. In-hospital administered medication is coded using the ATC taxonomy~\cite{ronning2002historical} and consists of more than $5$k chemical substances. Procedure codes specify in-hospital procedures performed on patients. While diagnosis events are an important medical modality, the dataset does not contain the time of such events. Hence, we omit diagnosis events from patient sequences. The event types with distinct tokens and total occurrence in the data are summarized in Table~\ref{tab:concept_types}.
