\section{Conclusion and Limitations}
\label{sec:conclusion}
In conclusion, we have shown that our method of pre-training local features on rigid 3D scenes can generalize well to new and unseen classes of deformable organic shapes, enabling effective performance in various shape analysis tasks. Our study has highlighted the importance of selecting the right receptive field size to ensure feature transferability, which has led to the \textit{first general-purpose local feature pre-training}  for deformable shape analysis tasks. This research also sheds light on the relationship between rigid and non-rigid processing tasks, providing a link between two fields that have traditionally used different tools.

One limitation of our method is its reliance on differentiable voxelization, which can be memory and time-consuming, particularly during pre-training. Nonetheless, our results outperform PointContrast \cite{xie2020pointcontrast}, a point-based method that requires \textit{more training data} and has limited generalizability. Another limitation is that our features rely on LRF estimation, which might lack robustness to thin structures or boundaries of partial shapes. Exploring alternative scalable and robust local feature pre-training strategies is an fascinating direction for future work.

\mypara{Acknowledgements}
The authors would like to thank the anonymous reviewers for their valuable suggestions. 
Parts of this work were supported by the ERC Starting Grant No. 758800 (EXPROTEA) and the ANR AI Chair AIGRETTE.









% In this work, we demonstrated that local features trained for rigid alignment of 3D scenes can generalize remarkably well to new unseen classes and especially deformable organic shapes in a wide range of shape analysis tasks. For this, we first showed the critical role that the receptive field size plays in the transferability of local features and proposed an optimization strategy to enable feature transfer across significantly different shape classes. 

% Our approach leads to the \textit{first general-purpose local feature pre-training} method that is applicable in deformable shape analysis tasks. Remarkably, our learned features enable tasks such as generalizable unsupervised shape matching without relying on shape pre-alignment. Our work also sheds light on the utility of low-level features in 3D transfer learning and creates an interesting link between rigid (3D scene or man-made object) shape analysis and non-rigid processing tasks -- two fields that have traditionally been considering very different tools.

% Perhaps the biggest limitation of our work is that it relies on differentiable voxelization and is thus relatively memory and time-consuming, especially during pre-training. Nevertheless, we obtain better results than PointContrast \cite{xie2020pointcontrast} that, despite being point-based, requires more training data and
% has limited generalizability across domains.

% \mypara{Acknowledgements}
% The authors would like to acknowledge the anonymous reviewers for their valuable suggestions. 
% Parts of this work were supported by the ERC Starting Grant No. 758800 (EXPROTEA) and the ANR AI Chair AIGRETTE.

% \paragraph{Societal impact}
% Efficient methods for non-rigid shape analysis have immediate impact in many
% areas of science and engineering from medical imaging
% (for instance for detecting anomalies, and performing follow-up analysis) to shape recognition and classification in areas such as computational biology, archaeology and paleontology to name a few. Our approach can immediately be adapted and tested in such diverse scenarios, due to the strong generalization power of the proposed descriptors. Our work also opens major avenues for future research as it can facilitate geometric deep learning methods without training, thus potentially enabling small labs to do research in this field without having big clusters or access to large-scale datasets. Finally we note that avoiding extensive training for each application also reduces the environmental impact of geometric deep learning, by significantly reducing the computation requirements for each independent application.

