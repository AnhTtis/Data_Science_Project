\section{Related Work}
Extracting robust local features is a key problem in 3D shape analysis, and a full review is beyond the scope of this paper. Below we discuss methods most closely related to ours and refer the readers to both early \cite{heider2011local,bronstein2012feature,guo20143d} and more recent surveys including \cite{guo2016comprehensive,guo2020deep,cao2020comprehensive,bronstein2017geometric} for a comprehensive overview of local feature extraction and geometric deep learning more broadly.

\mypara{Hand-crafted features.}
Early efforts in designing informative features for 3D geometry have focused primarily on either ensuring invariance to rigid motion \cite{johnson1999using,kortgen20033d,tombari2010unique,zaharescu2009surface,knopp2010hough} or intrinsic features invariant to isometries.
Intrinsic descriptors are typically based on either analysis of geodesic distances \cite{hilaga2001topology,gal2007pose} or derived quantities such as spectral properties arising from the eigenbases of the Laplace-Beltrami operator \cite{sun2009concise,aubry2011wave,bronstein2010scale}.
Intrinsic descriptors, such as the HKS \cite{sun2009concise} and WKS \cite{aubry2011wave}, are
a very popular choice in deformable shape correspondence methods, especially based on the functional maps framework, e.g., \cite{ovsjanikov2012functional,aflalo2013spectral,huang2014functional,eynard2016coupled,burghard2017embedding,rodola2017partial,ren2018continuous}, among many others.

\mypara{Learning for deformable shape matching.}
To overcome the limitations of hand-crafted features, more recent approaches have tried to learn descriptors directly from deformable shape data \cite{litman2013learning,masci2015geodesic,attaiki2023clover,boscaini2016learning,poulenard2018multi,wiersma2020cnns}.
Remarkably, however, many learning-based works still use hand-crafted features (most commonly, SHOT, HKS, WKS, or similar) as input to their learning pipelines \cite{litany2017deep,roufosse2019unsupervised,lim2018simple,maron2017convolutional,wang2020mgcn,eisenberger2020deep,sharp2020diffusionnet}.
Several recent works \cite{groueix20183d,marin22_why,fey2018splinecnn,li2020shape,donati2020deep,attaiki2022ncp,attaiki2021dpfm} investigate learning robust deformable shape correspondence directly from raw geometry, either by exploiting extensive training sets or combining spatial and spectral regularization \cite{donati2020deep,eisenberger2020deep}. Nevertheless, the features learned in these works are typically application and dataset-specific and fail to generalize to new shape classes and shape processing tasks.

\mypara{Learning for man-made shape matching.}
A parallel line of studies has focused on learning local geometric features for man-made object or scene alignment.
Many efforts have been made to explore different representations for local 3D geometry \cite{zeng20173dmatch,gojcic2019perfect,khoury2017learning,deng2018ppfnet,huang2017learning,li2020end}.
With the advancement of 3D deep learning techniques, networks based on PointNet \cite{wang2019deep,wang2019prnet,yew20183dfeat,aoki2019pointnetlk,yew2020rpm}, sparse convolution \cite{choy2019fully,choy20194d}, or kernel point convolution \cite{bai2020d3feat} have been applied to dense feature extraction.
However, these networks have very limited generalization ability even across 3D scene datasets, such as from indoor scenes to outdoor scans \cite{bai2020d3feat}, due to local features being coupled with global scene structures.

\mypara{Feature pretraining.} 
Most closely related to ours are the recent PointContrast \cite{xie2020pointcontrast} and its follow-up works \cite{hou2020exploring,chen20214dcontrast,wang2021unsupervised} that explore the potential of learning informative representations for 3D data, which can then be leveraged in downstream tasks. In contrast to these works, which focus on man-made objects or scenes, we consider generalizable feature pre-training for \textit{deformable shape analysis}, and study generalizability across significantly different 3D shape categories. Most importantly, our work highlights the impact of feature locality on transferability, which is lacking in prior works.
