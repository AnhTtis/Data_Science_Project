% CVPR 2023 Paper Template
% based on the CVPR template provided by Ming-Ming Cheng (https://github.com/MCG-NKU/CVPR_Template)
% modified and extended by Stefan Roth (stefan.roth@NOSPAMtu-darmstadt.de)

\documentclass[10pt,twocolumn,letterpaper]{article}
\pdfoutput=1

%%%%%%%%% PAPER TYPE  - PLEASE UPDATE FOR FINAL VERSION
%\usepackage[review]{cvpr}      % To produce the REVIEW version
\usepackage{cvpr}              % To produce the CAMERA-READY version
%\usepackage[pagenumbers]{cvpr} % To force page numbers, e.g. for an arXiv version

% Include other packages here, before hyperref.
\usepackage{graphicx}
\usepackage[accsupp]{axessibility}
\usepackage{amsmath}
\usepackage{amssymb}
\usepackage{booktabs}
\usepackage{enumitem}
%\usepackage{comment}
\usepackage{color}
\usepackage{xcolor}
\usepackage{textcomp}
\usepackage{epsfig}
\usepackage{multirow}
\usepackage{bigdelim}
\usepackage{wrapfig}
\usepackage{caption}
\usepackage{subcaption}
%\usepackage{url}
%\usepackage{nicefrac}
%\usepackage{tabularx}
%\usepackage{adjustbox}



% It is strongly recommended to use hyperref, especially for the review version.
% hyperref with option pagebackref eases the reviewers' job.
% Please disable hyperref *only* if you encounter grave issues, e.g. with the
% file validation for the camera-ready version.
%
% If you comment hyperref and then uncomment it, you should delete
% ReviewTempalte.aux before re-running LaTeX.
% (Or just hit 'q' on the first LaTeX run, let it finish, and you
%  should be clear).
\usepackage[pagebackref,breaklinks,colorlinks]{hyperref}

\newcommand{\maks}[1]{{\bfseries \small \color{blue} MO: #1}}
\newcommand{\lei}[1]{{\bfseries \small \color{orange} LL: #1}}
\newcommand{\souhaib}[1]{{\bfseries \small \color{green} SA: #1}}
\definecolor{ColorLightBlue}{RGB}{18, 137, 255}
\newcommand{\rev}[1]{{\color{ColorLightBlue}#1}}

\newcommand{\ra}[1]{\renewcommand{\arraystretch}{#1}}


\newlength\secmargin
\newlength\paramargin
\newlength\figmargin

\setlength{\secmargin}{-1.0mm}
\setlength{\paramargin}{-2mm}
\setlength{\figmargin}{-3.0mm}

%\newcommand{\mypara}[1]{\vspace{-2.5mm}\paragraph*{#1}}
%\newcommand{\mypara}[1]{\noindent\textbf{#1}~}
\newcommand{\mypara}[1]{\vspace{2mm}\noindent\textbf{#1}~}

\DeclareMathOperator*{\Minimize}{min}
\DeclareMathOperator*{\ArgMinimize}{argmin}
\newcommand{\accuchange}[1]{$_{\color{teal} (#1)}$}

\def\OurMethodName{VADER}

% Support for easy cross-referencing
\usepackage[capitalize]{cleveref}
\crefname{section}{Sec.}{Secs.}
\Crefname{section}{Section}{Sections}
\Crefname{table}{Table}{Tables}
\crefname{table}{Tab.}{Tabs.}


%%%%%%%%% PAPER ID  - PLEASE UPDATE
\def\cvprPaperID{1769} % *** Enter the CVPR Paper ID here
\def\confName{CVPR}
\def\confYear{2023}


\usepackage{textcomp}
\usepackage{float}
\usepackage{dblfloatfix} 
\usepackage{multicol}


\begin{document}

%%%%%%%%% TITLE - PLEASE UPDATE
\title{Generalizable Local Feature Pre-training for Deformable Shape Analysis}


\author{ Souhaib Attaiki \hspace{1.5cm} Lei Li \hspace{1.5cm} Maks Ovsjanikov\\
LIX, \'Ecole Polytechnique, IP Paris}

\maketitle



%%%%%%%%% ABSTRACT
Answering first-order logical (FOL) queries over knowledge graphs (KG) remains a challenging task mainly due to KG incompleteness. 
Query embedding approaches this problem by computing the low-dimensional vector representations of entities, relations, and logical queries. 
KGs exhibit relational patterns such as symmetry and composition and modeling the patterns can further enhance the performance of query embedding models.
However, the role of such patterns in answering FOL queries by query embedding models has not been yet studied in the literature.
In this paper, we fill in this research gap and empower FOL queries reasoning with pattern inference by introducing an inductive bias that allows for learning relation patterns. 
To this end, we develop a novel query embedding method, RoConE, that defines query regions as geometric cones and algebraic query operators by rotations in complex space. RoConE combines the advantages of Cone as a well-specified geometric representation for query embedding, and also the rotation operator as a powerful algebraic operation for pattern inference. 
%Therefore, RoConE enables inferring patterns during the multi-hop reasoning process.
Our experimental results on several benchmark datasets confirm the advantage of relational patterns for enhancing logical query answering task.

%%%%%%%%% BODY TEXT
\section{Introduction}\label{sec:intro}

Making co-speech gestures is an innate human behavior in daily conversations, which helps the speakers to express their thoughts and the listeners to comprehend the meanings~\cite{cassell1999speech, mcneill2011hand, 2014Gesture}. 
%
Previous linguistic studies verify that such non-verbal behaviors could liven up the atmosphere and improve mutual intimacy~\cite{burgoon1990nonverbal, 1989Gesture, huang2012robot}.
%
Therefore, animating virtual avatars to gesticulate co-speech movements is crucial in embodied AI. 
To this end, recent researches focus on the problem of audio-driven co-speech gesture generation~\cite{ginosar2019learning, yoon2020speech, liu2022learning, li2021audio2gestures}, which synthesizes human upper body gesture sequences that are aligned to the speech audio.

Early attempts downgrade this task as a searching-and-connecting problem, where they predefine the corresponding gestures of each speech unit and stitch them together by optimizing the transitions between consecutive motions for coherent results~\cite{cassell1994animated, huang2012robot, marsella2013virtual}. 
%
In recent years, the compelling performance of deep neural networks has prompted data-driven approaches. 
%
Previous studies establish large-scale speech-gesture corpus to learn the mapping from speech audio to human skeletons in an end-to-end manner~\cite{alexanderson2020style, liu2022beat, xu2022freeform, qian2021speech, liu2022learning, li2021audio2gestures, ao2022rhythmic}. 
%
To attain more expressive results, Ginosar \textit{et al.}~\cite{ginosar2019learning} and Yoon \textit{et al.}~\cite{yoon2020speech} propose GAN-based methods to guarantee realism by adversarial mechanism, where the discriminator is trained to distinguish real gestures from the synthetic ones while the generator's objective is to fool the discriminator. 
%
However, such pipelines suffer from the inherent mode collapse and unstable training, making them difficult to capture the \textit{high-fidelity audio-conditioned} gesture distribution, resulting in dull or unreasonable poses.

\begin{figure}[t]
\centering
\includegraphics[width=1.00\columnwidth]{figure/diffusion_process.pdf}
\caption{\textbf{Illustration of Conditional Generation Process in Co-Speech Gesture Generation.} The diffusion process $q$ gradually adds Gaussian noise to the gesture sequence (\textit{i.e.}, $\bm{x}_0$ sampled from the real data distribution). The generation process $p_{\theta}$ learns to denoise the white noise (\textit{i.e.}, $\bm{x}_T$ sampled from the normal distribution) conditioned on context information $\bm{c}$. Note that $\bm{x}_t$ denotes the corrupted gesture sequence at the $t$-th diffusion step.}
\label{overview}
\vspace{-0.2cm}
\end{figure}

The recent paradigm of diffusion probabilistic models provides a new perspective for realistic generation~\cite{ho2020denoising, song2021scorebased}, facilitating high-fidelity synthesis with desirable properties such as good distribution coverage and stable training compared to GANs.
%
However, it is non-trivial to adapt existing diffusion models for co-speech gesture generation. 
%
Most existing conditional diffusion models deal with \textit{static} data and conditions~\cite{Saharia2022Photorealistic, ramesh2022hierarchical} (\textit{e.g.}, the image-text pairs without temporal dimension), while co-speech gesture generation requires generating \textit{temporally coherent} gesture sequences conditioned on continual audio clips.
%
Further, the commonly used denoising strategy in existing diffusion models samples independently and identically distributed (\textit{i.i.d.}) noises in latent space to increase diversity. However, this strategy tends to introduce variation for each gesture frame and lead to temporal inconsistency in skeleton sequences. 
%
Therefore, how to generate high-fidelity co-speech gestures with strong audio correlations and temporal consistency is quite challenging within the diffusion paradigm.

To address the above challenges, we propose a tailored Diffusion Co-Speech Gesture framework to \textit{capture the cross-modal audio-gesture associations while maintaining temporal coherence} for high-fidelity audio-driven co-speech gesture generation, named \textbf{DiffGesture}.
%
As shown in Figure~\ref{overview}, we formulate our task as a diffusion-conditional generation process on clips of skeleton and audio, where the diffusion phase is defined by gradually adding noise to gesture sequence, and the generation phase is referred as a parameterized Markov chain with conditional context features of audio clips to denoise the corrupted gestures. 
%
As we treat the multi-frame gesture clip as the diffusion latent space, the skeletons can be efficiently synthesized in a non-autoregressive manner to bypass error accumulation.
%
To better attend to the sequential conditions from multiple modalities and enhance the temporal coherence, we then devise a novel \textit{Diffusion Audio-Gesture Transformer} architecture to model audio-gesture long-term temporal dependency.
%
Particularly, the per-frame skeleton and contextual features are concatenated in the aligned temporal dimension and embedded as individual input tokens to a Transformer block.
%
Further, to eliminate the temporal inconsistency caused by the naive denoising strategy in the inference stage, we thus propose a new \textit{Diffusion Gesture Stabilizer} module to gradually anneal down the noise discrepancy in the temporal dimension.
%
Finally, we incorporate implicit classifier-free guidance by jointly training the conditional and unconditional models, which allows us to trade off between the diversity and sample quality during inference.

Extensive experiments on two benchmark datasets show that our synthesized results are coherent with stronger audio correlations and outperform the state-of-the-arts with superior performance on co-speech gesture generation.
%
To summarize, our main contributions are three-fold: \textbf{1)} As an early attempt at taming diffusion models for co-speech gesture generation, we formally define the diffusion and denoising process in gesture space, which synthesizes audio-aligned gestures of high-fidelity. \textbf{2)} We devise the \textit{Diffusion Audio-Gesture Transformer} with implicit classifier-free diffusion guidance to better deal with the input conditional information from multiple sequential modalities. \textbf{3)} We propose the \textit{Diffusion Gesture Stabilizer} to eliminate temporal inconsistency with an annealed noise sampling strategy. 
\section{Related Work}\label{sec:related-work}

\subsection{Short-form Video Temporal Grounding}
Existing methods mainly focus on short-form video temporal grounding and can be categorized into \textit{proposal-based} and \textit{proposal-free} methods. Methods in proposal-based category adopt a two-stage pipeline, which first generate proposal candidates by various proposal generation methods, such as sliding window and proposal generation network, then they rank these candidates and output the proposal with the highest matching score as final prediction. \cite{gao2017tall} propose CTRL, a pioneer work in video grounding. CTRL produces various-length proposal candidates via sliding window and uses the visual-textual fusion modules combined with three operators, \textit{i.e.}, add, multiply and fully-connected layer, to obtain multi-modal fused representation. MAN \cite{zhang2019man} and SCDM~\cite{yuan2019semantic} leverage multiple cascaded temporal convolution layers to generate proposal candidates hierarchically. TGN~\cite{chen2018temporally} temporally captures the evolving fine-grained frame-by-word interactions and uses pre-set anchors to produce multi-scale proposal candidates ending at each time step.
Subsequently \cite{zhang2019cross,wang2020temporally,qu2020fine,zhang2021multi} follow the anchor-based framework and propose various multi-modal reasoning strategies to achieve precise moment localization. In addition, 2D-TAN~\cite{zhang2020learning} enumerate all possible segments as proposal candidates and convert them into 2D feature map, then a temporal adjacent network is proposed to obtain multi-modal representation and encode the video context information. Following this, \cite{zheng2021progressive,wang2021structured,soldan2021vlg} design more complicated cross-modal reasoning strategies to learn the video-language semantic alignment from both coarse and fine-grained granularities. 

Methods in proposal-free category predict the start and end boundaries by computing the time pair directly, or output the confidence scores of being the start and end positions of target moment for each snippet in video. \cite{yuan2019find} propose ABLR, which performs cross-modal reasoning with a multi-modal co-attention interaction modules and outputs target moments by feeding the multi-modal features to regressor. Attention weight-based regression and attention feature-based regression are considered together to achieve precise boundary regression. Concurrently, DRN~\cite{zeng2020dense} considers the data imbalance issue and only uses the frame in ground-truth moment to mitigate the sparsity issue. LGI~\cite{mun2020local} aligns the video and language from phrase-level and propose a local-global interaction network that models the cross-modal relationship considering local and global context information simultaneously.

However, directly applying these methods on long-form videos results in drastic performance degradation, as temporally downsampling a long video to so few frames causes severe temporal information loss.


\subsection{Long-form Video Temporal Grounding}
Recently MAD~\cite{soldan2022mad} and Ego4d~\cite{grauman2022ego4d} pose the challenge of long-form video temporal grounding, and give some baselines that integrate sliding window and temporal downsampling into some short video-fit methods, such as 2D-TAN~\cite{zhang2020learning}, VLG-Net~\cite{soldan2021vlg} and VSLNet~\cite{zhang2020span}. However, all these methods achieve inferior performance, considering both accuracy and efficiency. 
Recently~\cite{hou2022cone} propose CONE, which pre-filters the candidate windows to address the inference inefficiency and learns the cross-modal alignment from proposal-level and frame-level. Nevertheless, it adopts sparse sampling strategy at training stage, which does not explore the potential of long-form video adequately. 
\section{Motivation}
\label{sec:motivation}

Our main objective is to build a general-purpose feature extractor that can be applied to highly deformable shape analysis tasks, such as matching human shapes or segmentation of molecular surfaces, among others.
We first examine existing designs of geometric feature pre-training and perform a pilot study to understand their generalization power in such  tasks.
Our key insight is that the \textit{locality} of geometric features plays a crucial role in their transferability across different categories, which has so far been overlooked in prior works. To this end, we first perform an in-depth analysis of feature locality versus transferability in a representative deformable shape matching task. 

\mypara{Revisiting PointContrast.}
PointContrast \cite{xie2020pointcontrast} is a recent feature pre-training framework, in which a geometric feature extractor is pre-trained by a pretext task involving correspondences on 3D scenes \cite{dai2017scannet} related by rigid motion.
Specifically, PointContrast uses a fully-convolutional sparse U-Net \cite{choy2019fully} to output a feature vector for every point in the input.
During pre-training, given two scene fragments, a contrastive PointInfoNCE loss is used to minimize the feature distance for corresponding points and maximize it for non-corresponding ones:
\begin{equation}
    \label{eq:NCELoss}
    \mathcal{L}_{\text{nce}} = - \sum_{(i,j) \in \Omega} \log \dfrac{\exp(\mathbf{f}_i \cdot \mathbf{f}_j / \tau)}{\sum_{(\cdot, k) \in \Omega} \exp(\mathbf{f}_i \cdot \mathbf{f}_k / \tau)},
\end{equation}
where $\Omega$ is the set of corresponding point pairs in the overlap region, $\mathbf{f}_{\square}$ denotes the learned point-wise feature vector, and $\tau$ is a temperature hyper-parameter.

\mypara{Pilot study on deformable shapes.}
We study the link between the locality of geometric features and their transferability through the lens of a downstream deformable shape matching task \cite{roufosse2019unsupervised,halimi2019unsupervised,eisenberger2020deep,donati2020deep,sharma2020weakly}.
Specifically, we leverage a widely-used human shape dataset, FAUST-Remeshed (FR) \cite{Ren2019,Bogo2014}, consisting of 100 humans in diverse poses.
Given a pair of 3D shapes as input, we first use pre-trained feature extractors to compute a feature vector for every point in the input.
We then find correspondences via nearest neighbor search between the extracted features and apply a lightweight refinement with ZoomOut \cite{Melzi_2019}, a common practice in prior works.
We use the standard mean geodesic error \cite{Kim2011} as the evaluation metric.
Note that there is no fine-tuning of network weights, and thus this study provides a good indication of how informative and transferable the pre-trained features are in downstream applications.

\begin{figure}
    \centering
    \includegraphics[width=1.0\columnwidth]{figures/feat_loc_vs_trans_faust.pdf}
    \caption{Feature locality vs. transferability in a downstream task of non-rigid shape matching on the FAUST-Remeshed dataset. \emph{Local} in the legend denotes local patch-based input at each point.}
    \label{fig:locality_vs_transferability}
\end{figure}


\mypara{Feature locality vs. transferability.}
We evaluated the features produced by PointContrast in this context and obtained a matching error of \textbf{28.7}, compared to \textbf{6.1} achieved by a recent axiomatic method \cite{ren2018continuous}.  
%We first test PointContrast features on FR and obtain a matching error of \textbf{28.7} \maks{add some number for reference? compared e.g., to 2.7 reported in X (citation). or maybe compared to 30 obtained simply via random matching. we can't just use a number without any comparison.}, which shows their lack of utility for deformable shapes.
%
We attribute this limited utility of PointContrast features for deformable shapes to the global structure of its network, which employs a fully-convolutional U-Net design. Furthermore, this network is trained on entire 3D scenes with a global receptive field, making it significantly less likely to generalize to unseen shape categories.

To address this issue, we propose to limit the receptive field size and pre-train feature extractors \textit{that take as input only a local patch centered at each point}, and output a feature vector for the center point (\cref{fig:pipeline}). Intuitively, the space of local patches is significantly smaller than the space of shapes, thus potentially enabling generalization across different shape categories \cite{surfacerec07,patchnet2020,chabra2020,ao2021spinnet,pcpnet2018,fujiwara2011locally}.

To evaluate this general approach, we select three different architectures for pre-training a local feature extractor:
a) SparseConv \cite{choy20194dminkski,xie2020pointcontrast}, a sparse tensor-based network, which also constitutes the backbone of PointContrast;
b) PCPNet \cite{pcpnet2018,qi2017pointnet}, a PointNet-based architecture; 
and c) 3D CNN \cite{gojcic2019perfect,li2021updesc}, operating on voxel grids.
%We pre-train these feature extractors with the PointInfoNCE loss on 3D scene data \cite{zeng20173dmatch} in a similar manner as PointContrast.
%For completeness, we also include SHOT \cite{tombari2010unique}, a hand-crafted feature on 3D shapes.
We then pre-train these local feature extractors for \textit{the rigid alignment task} on 3D man-made scenes \cite{zeng20173dmatch}, following a similar strategy as in PointContrast. We follow the standard design choices and optimal pre-training patch size as used in the existing literature. Please see the exact architectures and pre-training details in the supplementary.

Given these pre-trained local feature extractors, a natural question would be how to adapt the receptive field (patch) size between pre-training and downstream 3D data, which may consist of significantly different shape classes, to make the local feature extractors generalize well. For this, we test a wide range of receptive field sizes (as ratios of the pre-training one) and plot their corresponding matching performance on FR in \cref{fig:locality_vs_transferability}.

When comparing PointContrast and Local SparseConv in \cref{fig:locality_vs_transferability}, we observe that making the network \textit{local} and operating on patches significantly improves feature transferability, especially for some specific receptive field size in this downstream task. Moreover, we observe that Local 3D CNN has the best generalization performance on FR, compared to Local SparseConv and Local PCPNet.

Most importantly, this pilot study highlights the importance of feature locality and the crucial role that the optimal receptive field size plays in the downstream task for successful transfer learning.
In practice, performing an exhaustive search of receptive field sizes is typically infeasible. To address this issue, we propose a differentiable approach to feature locality optimization for downstream 3D geometric data (\cref{subsec:local_feature_transfer}). 

\section{Proposed Modification}
\label{sec:method}

In the previous section, we provided a theoretical analysis of the conditions under which computed ``probe'' functions within the deep functional map pipeline can be used as pointwise descriptors directly, and lead to the same point-to-point maps as computed by the functional maps. In \cref{sec:application}, we provide an extensive evaluation of using learned feature functions for pointwise map computation and thus affirm the validity of \cref{thm:equivalence} in practice.

% \maks{is this paragraph necessary?} 
Our main observation is that the two approaches for point-to-point map computation are indeed often equivalent in practice, especially in ``easy'' cases, where existing state-of-the-art approaches lead to highly accurate maps. In contrast, we found that in more challenging cases, where existing methods fail, the two approaches are not equivalent and can lead to significantly different results. 

% Inspired by our analysis above, we thus propose to use the structural properties suggested in \cref{thm:equivalence} as a way to bridge this gap and, possibly improve the overall accuracy. As we demonstrate in \cref{sec:application}, our proposed modifications while being relatively simple, significantly improve the quality of the computed correspondences, especially in ``difficult'' matching scenarios.

Motivated by our analysis, we propose to use the structural properties suggested in \cref{thm:equivalence} as a way to bridge this gap and improve the overall accuracy. Our proposed modifications are relatively simple, but they significantly improve the quality of computed correspondences, especially in ``difficult'' matching scenarios, as we demonstrate in \cref{sec:application}.

% The two key assumptions in \cref{thm:equivalence} are \textit{basis-aligning} functional maps and \textit{complete} feature extractors. We thus propose to modify the functional map pipeline so that the conditions of the theorem are satisfied. As mentioned above, the basis-aligning property is closely related to \textit{properness} and thus we propose to approach it by imposing that the predicted functional map arises from some pointwise correspondence. For feature completeness, we propose a simple modification of the feature extractor so that it produces \textit{smooth features}. In what follows, we will use the same notation as \cref{sec:notation}.

The two key assumptions in \cref{thm:equivalence} are \textit{basis-aligning} functional maps and \textit{complete} feature extractors. We propose modifying the functional map pipeline to satisfy the conditions of the theorem. Since the basis-aligning property is closely related to \textit{properness}, we propose to impose that the predicted functional map to be proper, \ie arises from some pointwise correspondence. For feature completeness, we suggest modifying the feature extractor to produce \textit{smooth features}. We use the same notation as in \cref{sec:notation}.


\subsection{Enforcing Properness}
\label{sec:proper_fmap}

In this section, we propose two ways to enforce functional map properness and associated losses  for both supervised and unsupervised training.

\paragraph{The adjoint method}
Given feature functions $F_1, F_2$, produced by a feature extractor, we compute the functional map $\C_{12-pred}$ as explained in \cref{sec:background}. To compute a proper functional from it, we first convert $\C_{12-pred}$ into a p2p map $\Pi_{21-pred}$ in a differentiable way and then compute the ``differentiable'' proper functional map $\C_{21-proper} = \Phi_{2}^{\dagger} \Pi_{21-pred} \Phi_{1}$. 

To compute $\Pi_{21-pred}$, denoting $G_1 = \Phi_{1}$ and $G_2 = \Phi_{2}\C_{21-pred}$, we use:
\begin{align}
& \Pi_{21-pred}^{i, j} = \dfrac{\exp\big(\langle G_2^{i}, G_1^{j}\rangle / \tau\big)}{\sum_{k=1}^{n_1}\exp\big(\langle G_2^{i}, G_1^{k} \rangle / \tau\big)}.\label{eq:diff_p2p}%\\
%&s(\mathbf{x}, \mathbf{y}) = \mathbf{x} \cdot \mathbf{y}.\label{eq:FeatureDistanceFunc}
\end{align}
%
Here $\langle \cdot,\cdot \rangle$ is the scalar product measuring the similarity between $G_1$ and $G_2$, and $\tau$ is a temperature hyper-parameter. $\Pi_{21-pred}$ can be seen as a soft point-to-point map, formulated based on the adjoint conversion method described in \cite{Pai_2021_CVPR}, and computed in a differentiable manner, hence it can be used inside a neural network.

\paragraph{The feature-based method}
The feature-based method is similar to the adjoint method in spirit, the only difference being that $\Pi_{21-pred}$ is computed using the predicted features instead of the fmap. For this, we use \cref{eq:diff_p2p}, with $G_1 = F_1$ and $G_2 = F_2$. The modified deep functional map pipeline is illustrated in \cref{fig:fmap-pipeline}.

\begin{figure}
    \centering
    \includegraphics[width=\columnwidth]{figures/fmap_pipeline.pdf}
    \caption{An overview of our revised deep functional map pipeline. The extracted features are used to compute the functional map and the proper functional map, as explained in \cref{sec:proper_fmap}}
    \label{fig:fmap-pipeline}
    \vspace{-1em}
\end{figure}

In addition to $C_{12-pred}$, the two previous methods allow to calculate $C_{21-proper}$. We adapt the functional map losses to take into account this modification.

In the supervised case, we modify the supervised loss (see \cref{eq:sup_loss}) by simply introducing an additional term into the loss: 
\begin{align}
\mathcal{L}_{proper} = \| \C_{12-pred} - \C_{12-proper} \|_F^2 \label{eq:sup_loss_proper}.
\end{align}

The motivation behind this loss is that we want the predicted functional map to be as close as possible to the ground truth and stay within the space of proper functional maps.% a proper one , the gradient is strong enough to force the features to produce a  using \cref{eq:fmap_basic}, that is close to a proper one.

In the unsupervised setting, we simply impose the standard unsupervised losses on the differentiable proper functional map $\C_{12-proper}$ instead of $\C_{12-pred}$. Specifically, in our experiments below, we use the following unsupervised losses: 
%
\begin{align}
\nonumber
\mathcal{L}_{unsup}(\C_{12}, \C_{21}) &= \| \C_{12} \C_{21} - \mathbb{I} \|_F^2 + \| \C_{21} \C_{12} - \mathbb{I} \|_F^2 \\
& + \| \C_{12}^{\top} \C_{12} - \mathbb{I} \|_F^2 + \| \C_{21}^{\top} \C_{21} - \mathbb{I} \|_F^2
\label{eq:unsup_loss_proper}
\end{align}

%\maks{add the losses here.}


% \souhaib{how to justify that the results obtained with NN are better than fmap}












%%%%%%%%%%%%%%%%%%%%%%%%%%%%%%%%%%%%%%%%%%%%%%%%%%%%
%%%%%%%%%%%%%%%%%%%%%%%%%%%%%%%%%%%%%%%%%%%%%%%%%%%%%%%%%%%%%%%%%%%%%%%%%%%%%%%%%%%%%%%%%%%%%%%%%%%%%%%%
%%%%%%%%%%%%%%%%%%%%%%%%%%%%%%%%%%%%%%%%%%%%%%%%%%%%%%%%%%%%%%%%%%%%%%%%%%%%%%%%%%%%%%%%%%%%%%%%%%%%%%%%
%%%%%%%%%%%%%%%%%%%%%%%%%%%%%%%%%%%%%%%%%%%%%%%%%%%%
%%%%%%%%%%%%%%%%%%%%%%%%%%%%%%%%%%%%%%%%%%%%%%%%%%%%

\subsection{As Smooth As Possible Feature Extractor}
Another fundamental assumption of \cref{thm:equivalence} is the completeness of the features produced by a neural network. 

We have experimented with several ways to impose it and have found that it is not easy to satisfy it exactly in general because it would require the network to always produce features in some target  subspace, which is not explicitly specified in advance. Moreover, we have found that explicitly projecting feature functions to a small reduced subspace can also hinder learning. 

To circumvent this, we propose instead to \textit{encourage} this property by promoting the feature extractor to produce smooth features. 

The motivation for this is as follows. If $F_i$ is complete, then there exist coefficients $a_1 ... a_k$ such that $F_i = \sum_{j=1}^k a_j \Phi_i^j$, where $k$ is the size of the functional map used in \cref{eq:fmap_basic}.
However, it's known that Fourier coefficients for smooth functions decay rapidly (faster than any polynomial, if $f$ is of class $C^l$, the coefficients are $o(n^{-l})$), which means that the smoother the function is, the closer it will be to being complete for some index $k$.

Inspired by this, we propose the following simple modification to feature extractors used for deep functional maps. Since feature extractors are made of multiple layers, we propose to project the output of each layer into the Laplacian basis, diffuse it over the surface following \cite{sharp2021diffusion}, and then project it back to the ambient space before feeding it to the next layer, see \cref{fig:feat-extract-modif}. Concretely, for shape $S$, if $f_i$ is the output of layer $i$, we feed to layer $i+1$ the function $f^{'}_i$, such that $f^{'}_i = \Phi_j e^{-t \Delta} \Phi_j^{\dagger} f_i$, where $\Phi_j$ denotes the first $j$ eigenfunctions of the Laplacian, $\Delta$ is a diagonal matrix containing the first j eigenvalues, and $t$ is a learnable parameter. Please note there is no need to do this operation for the final layer, since the features will be projected into the Laplacian basis anyway, for computing the functional map. In practice, we observed that it is beneficial to set $j$ to \textit{be larger} than the size of the functional maps in \cref{eq:fmap_basic}. This allows the network to impose smoothness, while still allowing degrees of freedom to enable optimization.

%Also note, that DiffusionNet \cite{sharp2021diffusion} does this operation by construction for each layer, which can in part explain its success.
%
%\souhaib{what about the receiptive field}


%
%
% - smooth the input before feeding them to the network (doesn't work practically)
% 
% - smooth the features at the end of each layer
% 
% - for better results, increase the receiptive field of the features using diffusion 
%
%
\vspace{-1em}

\paragraph{Implementation details} we provide implementation details, for all our experiments, in the supplementary. Our code and data will be released after publication.

\begin{figure}
    \centering
    \includegraphics[width=\columnwidth]{figures/feature_extractor.pdf}
    \caption{An overview of the feature extractor modification is shown here. The features are made smooth by projecting them into the Laplacian basis at the end of each layer.}
    \label{fig:feat-extract-modif}
    % \vspace{-1.5em}
\end{figure}
\section{Experiments}
\label{sec:experiments}

In this section, we provide extensive experiments to highlight the generalization power of our local feature pre-training and receptive field size optimization methods in a suite of downstream shape analysis tasks especially involving highly deformable, organic shapes.
We consider diverse benchmarks including human and partial animal shape matching as well as molecular surface segmentation.
We reuse the same pre-trained local feature extractor $\mathcal{F}_{s, \Theta}$ and then perform our receptive field size optimization once individually on each deformable shape dataset (\cref{subsec:local_feature_transfer}).

\mypara{Implementation.}
We denote our features as \OurMethodName{} for Voxelized Alignment-based DEscriptoR.
%We implemented our \OurMethodName{} pre-training and transferring pipeline with PyTorch \cite{NEURIPS2019_9015}.
In the pre-training stage, we use the 3DMatch dataset \cite{zeng20173dmatch} and train the local feature extractor $\mathcal{F}_{s, \Theta}$ for 16K steps.
We use the Adam optimizer with a learning rate of $10^{-3}$  for network weight update.
For receptive field size optimization, we use the same learning rate for Adam, and we take $n_s = 10^4$  extracted patches from the pre-training dataset. 
%Our code and data will be released after publication.

\mypara{Baselines.}
We compare a wide spectrum of hand-crafted and pre-trained features in deformable shape tasks.
For the hand-crafted features, we consider the Heat Kernel Signature (HKS)  \cite{sun2009concise}, Wave Kernel Signature (WKS) \cite{aubry2011wave}, and SHOT descriptors \cite{tombari2010unique}, as well as the straightforward vertex positions (XYZ).
For the pre-trained features, we use the PointContrast features learned with the PointInfoNCE loss (PCN) or a hardest-contrastive loss (PCH) \cite{xie2020pointcontrast}.

%\rev{TODO: comparison of Vader with NCE vs cycle-consistency loss in some downstream task?}

\subsection{Human Shape Matching}
\label{subsec:human_matching}

%\setlength{\tabcolsep}{4pt}
\begin{table}[!t]
    \begin{center}
    \ra{1.0}
        \resizebox{0.81\columnwidth}{!}{%
            \begin{tabular}{@{} lrr @{} }
                \toprule
                \textbf{Method / Dataset}                & \textbf{FR}-\textbf{SH} & \textbf{SR}-\textbf{SH} \\
                \midrule
                SURFMNET \cite{roufosse2019unsupervised} & 30.1                    & 28.6                    \\
                Cyclic FMaps \cite{ginzburg2019cyclic}   & 36.5                    & 38.6                    \\
                WSupFMNet \cite{sharma2020weakly}        & 26.3                    & 30.2                    \\
                Deep Shells \cite{eisenberger2020deep}   & 26.3                    & 22.8                    \\
                \midrule
                % \addlinespace
                DiffusionNet - XYZ                       & 22.4                    & 23.3                    \\ % 8.1 & 23.1 &
                DiffusionNet - HKS                       & 10.4                    & 15.4                    \\ % 7.1 & 15.4 &
                DiffusionNet - WKS                       & 9.3                     & 24.0                    \\ % 3.8 & 4.4 &
                DiffusionNet - SHOT                      & 10.8                    & 21.5                    \\ % 3.8 & \textbf{4.2} &
                DiffusionNet - PCH                       & 25.8                    & 33.2                    \\ % 6.3 & 4.3 &
                DiffusionNet - PCN                       & 22.6                    & 36.2                    \\ % 8.7 & 4.8 &
                DiffusionNet - \OurMethodName{} (ours)   & \textbf{6.4}            & \textbf{6.9}            \\ % % 8.7 & 4.8 &
                %DiffusionNet - \OurMethodName{} (ours)  & \textbf{8.6}            & \textbf{8.8}            \\ % \textbf{3.8} & 4.4 &
                %\hdashline
                %\quad + ZoomOut & -- & -- & -- & -- \\
                \bottomrule
            \end{tabular}
        }
        \caption{Performance of various features for unsupervised deformable shape matching on un-aligned data. X-Y means training on X and testing on Y. Values are mean geodesic error $\times 100$ on unit-area shapes.}
        \label{tab:unaligned_unsup}
    \end{center}
    \vspace{\figmargin}
\end{table}
%\setlength{\tabcolsep}{1.4pt}


\mypara{Unsupervised matching.}
We perform unsupervised shape matching \cite{roufosse2019unsupervised,eisenberger2020deep} on the FAUST-Remeshed (FR), SCAPE-Remeshed (SR), and SHREC'19 datasets (SH) \cite{Ren2019,Bogo2014,Anguelov2005,shrec19}, consisting of 100, 71, and 44 \emph{unaligned} human shapes in different poses, respectively. 
The same train/test splits in prior works \cite{sharma2020weakly,eisenberger2020deep} are adopted.
We feed the above baseline features and our \OurMethodName{} respectively as input to a surface learning backbone DiffusionNet \cite{sharp2020diffusionnet}, which produces a functional map \cite{ovsjanikov2012functional,litany2017deep} as output for a given pair of shapes.
We leverage the unsupervised functional map losses \cite{sharma2020weakly} for training the backbone.
The evaluation metric is the mean geodesic error of predicted maps with respect to the ground truth on unit-area shapes \cite{Kim2011}. We use X-Y to denote training on dataset X and testing on dataset Y.

%\maks{we don't mention what  FR-SH stands for. Also, we should highlight that this is a very difficult problem and existing unsuprevised methods have only been used on aligned data. this explains why most of the numbers in Table 1 are so bad.}

As shown in \cref{tab:unaligned_unsup}, our approach yields the best results for unsupervised shape correspondence in both FR-SH and SR-SH settings, while the other tested features fail to achieve reasonable matching performance.
We stress that this is a challenging test case as most existing unsupervised methods rely on aligned shapes, e.g.,  \cite{sharma2020weakly,eisenberger2021neuromorph}.
Note that the PointContrast features perform worse than the hand-crafted features, indicating the overfitting to the pre-training data distribution, as discussed in \cref{sec:motivation}, and thus the limited transferability.
Our approach also outperforms several recent unsupervised approaches in \cref{tab:unaligned_unsup}, including SURFMNET \cite{roufosse2019unsupervised}, Cyclic FMaps \cite{ginzburg2019cyclic}, WSupFMNet \cite{sharma2020weakly}, and Deep Shells \cite{eisenberger2020deep}. 
The comparisons clearly demonstrate the utility of incorporating generalizable pre-trained features, which is missing in prior works. 
We provide qualitative comparisons in \cref{fig:qual_all_human} (Top), showing that only our approach leads to visually plausible results.

\begin{figure}
    \centering
    \includegraphics[width=\columnwidth]{figures/all_humans.pdf}
    \caption{Qualitative comparisons of human shape matching by texture transfer. Top: results of unsupervised matching on SH. Bottom: results of supervised matching on FQ. The best three performing competitors are shown.}
    \label{fig:qual_all_human}
    \vspace{\figmargin}
\end{figure}



\mypara{Robustness to meshing.}
We evaluate the performance of supervised shape matching \cite{donati2020deep} on the original FAUST (FO) dataset \cite{Bogo2014} and its remeshed version by quadratic error simplification (FQ) \cite{sharp2020diffusionnet} to demonstrate the robustness and generalization of our approach against significant mesh connectivity changes across datasets. 
We build a point-wise MLP network \cite{litany2017deep} (to reduce the dependence on the backbone architecture) on top of the baseline features and our \OurMethodName{} respectively, and then predict functional maps for shape pairs.
The MLP backbones are trained on FO with predictions supervised by the ground-truth maps with a simple $L_2$ loss.
We report the mean geodesic error metric on FQ.

\cref{fig:super_fo_fq} shows the correspondence quality with a varying error threshold. 
It can be seen that hand-crafted features such as SHOT degrade rapidly under remeshing. 
Although WKS and HKS are intrinsic features and do not depend on the meshing connectivity, they are not expressive enough by themselves and need to be combined with a powerful network backbone such as DiffusionNet, instead of the point-wise MLPs used in this experiment.
Differently, our approach achieves superior generalization performance in this challenging setting, showing that \OurMethodName{} is highly robust to remeshing and resampling, and effectively captures local geometric structures in deformable shapes.
\cref{fig:qual_all_human} (bottom) presents a qualitative comparison of the computed maps with the three best-performing competitors.

\begin{figure}
    \centering
    \includegraphics[width=\columnwidth]{figures/supervised_fo_fq.pdf}
    \caption{Accuracy of various features for supervised shape matching when the connectivity changes from training to test (mean errors $\times 100$ are reported in the legend).}
    \label{fig:super_fo_fq}
\end{figure}


\subsection{Molecular Surface Segmentation}
\label{subsec:molecular_segmentation}

%\setlength{\tabcolsep}{4pt}
\begin{table}[!t]
     \begin{center}
    \ra{1.0}
          \resizebox{\columnwidth}{!}{%
               \begin{tabular}{@{}lrrr@{}}
                    \cmidrule[\heavyrulewidth]{1-4}
                    \textbf{Method}                                 & \multicolumn{3}{c}{\textbf{Accuracy $\pm$ s.d}}                                                           \\
                    \cmidrule{2-4}
                                                                    & Full Dataset                                    & 50 Shapes                  & 100 Shapes                 \\
                    \cmidrule[\heavyrulewidth]{1-4}
                    PointNet++ \cite{qi2017pointnet}                & 74.4\%                                          & --                         & --                         \\
                    PCNN \cite{atzmon2018pcnn}                      & 78.0\%                                          & --                         & --                         \\
                    SPHNet \cite{poulenard2019effective}            & 80.1\%                                          & --                         & --                         \\
                    SplineCNN \cite{fey2018splinecnn}               & 53.6\%                                          & --                         & --                         \\
                    SurfaceNetworks \cite{kostrikov2018surface}     & 88.5\%                                          & --                         & --                         \\

                    \cmidrule[\heavyrulewidth]{1-4}
                    DiffusionNet - XYZ \cite{sharp2020diffusionnet} & 90.5 $\pm$ 0.6\%                                & 82.7 $\pm$ 0.63\%          & 83.4 $\pm$ 0.67\%          \\
                    DiffusionNet - HKS                              & 90.6 $\pm$ 0.15\%                               & 82.7 $\pm$ 0.16\%          & 84.5 $\pm$ 0.09\%          \\
                    DiffusionNet - WKS                              & 88.7 $\pm$ 0.26\%                               & 77.6 $\pm$ 0.16\%          & 81.2 $\pm$ 0.20\%          \\
                    DiffusionNet - SHOT                             & 92.1 $\pm$ 0.08\%                               & 81.6 $\pm$ 0.31\%          & 85.7 $\pm$ 0.11\%          \\
                    DiffusionNet - PCH                              & 90.3 $\pm$ 0.1\%                                & 79.9 $\pm$ 0.59\%          & 83.6 $\pm$ 0.08\%          \\
                    DiffusionNet - PCN                              & 90.1 $\pm$ 0.09\%                               & 80.1 $\pm$ 0.29\%          & 83.4 $\pm$ 0.28\%          \\
                    DiffusionNet - \OurMethodName{} (ours)          & \textbf{92.6 $\pm$ 0.02\%}                      & \textbf{83.2 $\pm$ 0.20\%} & \textbf{86.8 $\pm$ 0.09\%} \\ % ~\accuchange{+3.4}   ~\accuchange{+2.7}         \\
                    % DiffusionNet - \OurMethodName{} (ours)          & \textbf{93.1 $\pm$ 0.04\%}                      & \textbf{86.1 $\pm$ 0.13\%} & \textbf{88.4 $\pm$ 0.05\%} \\ % ~\accuchange{+3.4}   ~\accuchange{+2.7}
                    \cmidrule[\heavyrulewidth]{1-4}
               \end{tabular}
          }
          \caption{Accuracy of various mesh and point cloud based methods for RNA segmentation. The reported numbers are mean accuracy over 5 runs randomly initialized. $\pm$ denotes standard deviation.}%\maks{shall we remove the 100 shapes setting?}
          \label{tab:rna_seg}
     \end{center}
    \vspace{\figmargin}
\end{table}
%\setlength{\tabcolsep}{1.4pt}


Next, we conduct experiments in the molecular surface segmentation task, which aims to segment RNA molecules into functional components.
We use the dataset introduced in \cite{poulenard2019effective}, consisting of 640 RNA triangle meshes, where
each vertex is labeled into one of 259 atomic categories.
The dataset has an 80/20\% split for training and test sets.
We feed the baseline features and our \OurMethodName{} respectively as input to DiffusionNet and train it to predict a label at each vertex as output.
% Each experiment is run five times with different random initialization.
% Mean accuracy and its standard deviation are reported.

As shown in \cref{tab:rna_seg}, our approach achieves state-of-the-art segmentation performance when used with the full training set, outperforming both hand-crafted and pre-trained PointContrast features as well as several recent shape segmentation networks, such as \cite{kostrikov2018surface}.
We also perform experiments in more challenging settings, where only a fraction of the training set, with respectively 50 and 100 shapes (corresponding to 9\% and 18\% of the training set), is used.
We observe in \cref{tab:rna_seg} that our method consistently outperforms the competitors by a significant margin when given limited training data.
The results highlight that our pre-training and receptive field size optimization strategies bring significant improvement to downstream organic shape analysis tasks.

\subsection{Partial Animal Matching}
\label{subsec:animal_matching}
We also evaluate how well different geometric features perform on deformable shapes in the presence of significant partiality.
For this, we test on the challenging SHREC16' Cuts dataset \cite{cosmo2016shrec}, where the animal classes (cat, centaur, dog, horse, and wolf) are used for partial shape matching.
We follow the setup of DiffusionNet described in \cref{subsec:human_matching} for correspondence prediction. 

We compare our approach to XYZ and SHOT as they are widely used in partial matching pipelines, in addition to full-fledged methods PFM \cite{Rodol2016} and FSP \cite{Litany2017}, specifically tailored toward partial shape matching. 
The results are summarized in \cref{tab:sup_animal}. Qualitative results are visualized in \cref{fig:qual_sup_animal}. Observe that our approach outperforms the competitors in this setting by a significant margin, including specially-tailored partial matching methods PFM and FSP.

\begin{table}[!t]
    \begin{center}
    \ra{1.0}
        \resizebox{0.9\columnwidth}{!}{%
        \begin{tabular}{@{} lr @{}}
            \toprule
            \textbf{Method / Dataset}              & SHREC'16 CUTS Animals \\
            \midrule
            PFM \cite{Rodol2016}                   & 8.8                   \\
            FSP \cite{Litany2017}                  & 12.2                  \\
            DiffusionNet - XYZ                     & 4.9                   \\
            DiffusionNet - SHOT                    & 4.6                   \\ %19.3
            DiffusionNet - \OurMethodName{} (ours) & \textbf{3.7}          \\
            \bottomrule
        \end{tabular}
        }
        \caption{Performance (mean geodesic error $\times 100$) of various features on the SHREC'16 CUTS Animals benchmark.\vspace{-2mm}}
        \label{tab:sup_animal}
    \end{center}
    \vspace{\figmargin}
\end{table}


\begin{figure}[!t]
    \centering
    \includegraphics[width=\columnwidth]{figures/supervised_animals.pdf}
    \caption{Qualitative comparisons of partial animal matching by texture transfer on the cat class of the SHREC'16 CUTS Animals benchmark.}
    \label{fig:qual_sup_animal}
    \vspace{\figmargin}
\end{figure}


\subsection{Shape Classification}
\label{subsec:shape_classification}

% Finally, to show the utility of our pre-trained feature extractor on man-made objects, we adopt the ShapeNet \cite{chang2015shapenet} classification setup from PointContrast \cite{xie2020pointcontrast}, where pre-trained weights are used as initialization for fine-tuning a classification network.
% For comparison with PointContrast, we conduct experiments on the ShapeNetCore v2 dataset with the same train/test split, and our feature extractor pre-trained with the PointInfoNCE loss is used.
% In addition to fine-tuning on the full training set, a limited training data setup (i.e., 1\% or 10\%) is also considered.

% \cref{table-shapenet_classification_accuracy_retrain} shows the classification accuracy comparison.
% We observe that feature pre-training generally improves the performance across different training setups, compared to training from scratch for the classification networks.
% In particular, our network has better classification accuracy than PointContrast when using 1\%, 10\%, or 100\% training data.

We use the ShapeNet dataset \cite{chang2015shapenet} to demonstrate the effectiveness of our pre-trained feature extractor on man-made objects. We follow the classification setup from PointContrast \cite{xie2020pointcontrast}, using pre-trained weights as initialization for fine-tuning a classification network. Here our feature extractor is pre-trained with the PointInfoNCE loss. We conduct experiments on the ShapeNetCore v2 dataset with the same train/test split as PointContrast. We also consider a limited fine-tuning data setup, using a fraction of the fine-tuning data (1\% or 10\%). The classification accuracy comparison is summarized in \cref{table-shapenet_classification_accuracy_retrain}. It can be seen that feature pre-training improves performance compared to training from scratch. Also, our network achieves higher classification accuracy than PointContrast in all training setups.

% \begin{table}[!t]
%     \begin{center}
%         \resizebox{0.8\columnwidth}{!}{%
%         \ra{1.0}
%         \begin{tabular}{@{} lrrr @{}}
%             \toprule
%                                  & 1\% data      & 10\% data     & 100\% data    \\
%             \midrule
%             \multicolumn{4}{c}{\em{PointContrast}}                               \\
%             % \midrule
%             \addlinespace
%             From scratch         & 53.2          & 74.4          & 76.9          \\
%            %Hardest-Contrastive  & 61.3          & 75.3          & 76.8          \\
%             PointInfoNCE         & 60.7          & 73.7          & 77.2          \\
%             \midrule
%             \multicolumn{4}{c}{\em{\OurMethodName{} (ours)}}                                \\
%             % \midrule
%             \addlinespace
%             From scratch         & 59.5          & 72.2          & 79.0          \\
%             PointInfoNCE         & \textbf{66.5} & \textbf{77.2} & \textbf{81.2} \\
%             \bottomrule
%         \end{tabular}
%         }
%         \caption{ShapeNet classification accuracy with limited labeled training data for fine-tuning.\vspace{-1mm}}
%         \label{table-shapenet_classification_accuracy_retrain}
%     \end{center}
%     \vspace{-3mm}
% \end{table}



\begin{table}[!t]
    \begin{center}
        \resizebox{0.99\columnwidth}{!}{%
        \ra{1.0}
        \begin{tabular}{@{} lrrrr @{}}
            \toprule
                \% train data    &  \multicolumn{2}{c}{\em{PointContrast}} & \multicolumn{2}{c}{\em{\OurMethodName{} (ours)}}   \\
                & From scratch  & PointInfoNCE & From scratch  & PointInfoNCE \\
                \midrule
                1\% data & 53.2 & 60.7 & 59.5 & \textbf{66.5}\\
                10\% data & 74.4  & 73.7 & 72.2 & \textbf{77.2}\\
                100\% data & 76.9 & 77.2 & 79.0 & \textbf{81.2}\\
            
            \bottomrule
        \end{tabular}
        }
        \caption{ShapeNet classification accuracy with limited labeled training data for fine-tuning.\vspace{-1mm}}
        \label{table-shapenet_classification_accuracy_retrain}
    \end{center}
    \vspace{-3mm}
\end{table}





\subsection{Ablation Study}
\label{subsec:ablation_study}

\begin{table}[t]
    \begin{center}
    \ra{1.0}
        \resizebox{0.45\columnwidth}{!}{%
        \begin{tabular}{@{} lrr @{}}
            \toprule
            \textbf{Dataset}                       & \textbf{FR-SR}        & \textbf{SR-FR}     \\
            \midrule
            DFAUST                                 & 27.7                  & 4.4                \\
            3DMatch                                & \textbf{4.1}          & \textbf{3.8}       \\
            \bottomrule
        \end{tabular}
        }
        \caption{Results of using features pre-trained on different datasets in the downstream task of unsupervised non-rigid shape matching. 
        %X-Y means training on X and testing on Y. 
        Values are mean geodesic error $\times 100$ on unit-area shapes.}
        \label{tab:dataset_ablation}
    \end{center}
    \vspace{-4.5mm}
\end{table}


We also evaluate the role of the pretraining dataset for local feature learning. 
For this, we compare 3DMatch used in our experiments to DFAUST \cite{dfaust:CVPR:2017}, a large-scale dataset of human subjects in motion.
We use the unsupervised shape matching task and the evaluation protocol introduced in \cref{subsec:human_matching} to test the generalization of a pre-trained feature extractor to the FR and SR datasets.



% \begin{wrapfigure}[8]{r}{0.4\columnwidth}
%     \centering
%     \includegraphics[width=0.4\columnwidth]{figures/pca_projection.png}
%     \vspace{-10pt}
% \end{wrapfigure}

The comparisons in \cref{tab:dataset_ablation} show that pre-training on 3DMatch leads to more generalizable features and consistent matching performance, even though DFAUST has greater similarity to the downstream human shape datasets FR and SR.
We attribute this to the fact that \textit{local geometries }in 3DMatch, which consists of real-world scans, are richer and more complex than those in template-fitted DFAUST, leading to a more universally useful pre-training signal.
% To validate this, we perform PCA \cite{pca01} on the local patches in 
% % here inset
% each dataset and visualize the projections to principal 
% components in the inset figure, showing that the local patches in DFAUST (red dots) are included in 3DMatch (blue dots).
% Please see the supplementary for more analysis and experiments.

To validate this, we perform PCA analysis \cite{pca01} on the local patches of 3DMatch and DFAUST.
For each dataset, we first randomly extract 200K local patches. We then encode each patch as a high dimensional vector by first orienting it using a local reference frame and then voxelizing it to a small 3D grid of resolution = $16^3$ using the method of \cite{gojcic2019perfect}. The resulting vectors are 4096-dimensional and are fed as input to PCA. In \cref{fig:ds_abla} (a), we report the unexplained variance as a function of the number of principal components. It can be seen that 3DMatch is significantly more diverse than DFAUST since more principal components are needed to explain its full variance. In \cref{fig:ds_abla} (b), we visualize the projection of patches in the first two principal components and observe that local patches in DFAUST (red dots) are included in 3DMatch (blue dots), demonstrating the diversity and richness of 3DMatch once more.

\begin{figure}[!t]
    \centering
    \includegraphics[width=\columnwidth]{figures/pretrain_ds_ablation.pdf}
    \caption{Comparing richness of local geometries in 3DMatch and DFAUST via PCA. (a) We perform PCA on sampled 3D local patches and plot the unexplained variance \wrt the number of principal components. (b) We project the local geometries onto the first two principal components.}
    \label{fig:ds_abla}
    \vspace{\figmargin}
\end{figure}






%%%%%%%%%%%%%%%%%%%%%%%%%%%%%%%
%%%%%%%%%%%%%%%%%%%%%%%%%%%%%%
%%%%%%%%%%%%%%%%%%%%%%%%%%%%%%
%%%%%%%%%%%%%%%%%%%%%%%%%%%%%
% We performed PCA on the local patches of 3DMatch and DFAUST.
% For each dataset, we first randomly extracted 200K local patches. 
% We then encode each patch as a high dimensional vector by first orienting it using a local reference frame and then voxelizing it to a small 3D grid of resolution = $16^3$ using the method of \cite{gojcic2019perfect}. 
% The resulting vectors are 4096-dimensional and fed as input to PCA to analyze the internal richness and complexity of each dataset.
% In \cref{fig:pca_unexplained_supp}, we report the unexplained variance as a function of the number of principal components. It further confirms that 3DMatch is significantly more diverse than DFAUST, since more principal components are needed to explain its full variance. 

\section{Conclusion and Limitations}
\label{sec:conclusion}
In conclusion, we have shown that our method of pre-training local features on rigid 3D scenes can generalize well to new and unseen classes of deformable organic shapes, enabling effective performance in various shape analysis tasks. Our study has highlighted the importance of selecting the right receptive field size to ensure feature transferability, which has led to the \textit{first general-purpose local feature pre-training}  for deformable shape analysis tasks. This research also sheds light on the relationship between rigid and non-rigid processing tasks, providing a link between two fields that have traditionally used different tools.

One limitation of our method is its reliance on differentiable voxelization, which can be memory and time-consuming, particularly during pre-training. Nonetheless, our results outperform PointContrast \cite{xie2020pointcontrast}, a point-based method that requires \textit{more training data} and has limited generalizability. Another limitation is that our features rely on LRF estimation, which might lack robustness to thin structures or boundaries of partial shapes. Exploring alternative scalable and robust local feature pre-training strategies is an fascinating direction for future work.

\mypara{Acknowledgements}
The authors would like to thank the anonymous reviewers for their valuable suggestions. 
Parts of this work were supported by the ERC Starting Grant No. 758800 (EXPROTEA) and the ANR AI Chair AIGRETTE.









% In this work, we demonstrated that local features trained for rigid alignment of 3D scenes can generalize remarkably well to new unseen classes and especially deformable organic shapes in a wide range of shape analysis tasks. For this, we first showed the critical role that the receptive field size plays in the transferability of local features and proposed an optimization strategy to enable feature transfer across significantly different shape classes. 

% Our approach leads to the \textit{first general-purpose local feature pre-training} method that is applicable in deformable shape analysis tasks. Remarkably, our learned features enable tasks such as generalizable unsupervised shape matching without relying on shape pre-alignment. Our work also sheds light on the utility of low-level features in 3D transfer learning and creates an interesting link between rigid (3D scene or man-made object) shape analysis and non-rigid processing tasks -- two fields that have traditionally been considering very different tools.

% Perhaps the biggest limitation of our work is that it relies on differentiable voxelization and is thus relatively memory and time-consuming, especially during pre-training. Nevertheless, we obtain better results than PointContrast \cite{xie2020pointcontrast} that, despite being point-based, requires more training data and
% has limited generalizability across domains.

% \mypara{Acknowledgements}
% The authors would like to acknowledge the anonymous reviewers for their valuable suggestions. 
% Parts of this work were supported by the ERC Starting Grant No. 758800 (EXPROTEA) and the ANR AI Chair AIGRETTE.

% \paragraph{Societal impact}
% Efficient methods for non-rigid shape analysis have immediate impact in many
% areas of science and engineering from medical imaging
% (for instance for detecting anomalies, and performing follow-up analysis) to shape recognition and classification in areas such as computational biology, archaeology and paleontology to name a few. Our approach can immediately be adapted and tested in such diverse scenarios, due to the strong generalization power of the proposed descriptors. Our work also opens major avenues for future research as it can facilitate geometric deep learning methods without training, thus potentially enabling small labs to do research in this field without having big clusters or access to large-scale datasets. Finally we note that avoiding extensive training for each application also reduces the environmental impact of geometric deep learning, by significantly reducing the computation requirements for each independent application.



 %%%%% supp, for the arxiv version
\newpage
% \onecolumn
% \null

% \title{Supplementary Materials for:\\Generalizable Local Feature Pre-training for Deformable Shape Analysis}
% \author{ Souhaib Attaiki \hspace{1.5cm} Lei Li \hspace{1.5cm} Maks Ovsjanikov\\
% LIX, \'Ecole Polytechnique, IP Paris}

% \begin{center}
%       % smaller title font only for rebuttal
%       {\Large \bf \title \par}
%       % additional two empty lines at the end of the title
%       {\vspace*{24pt}}
%       {
%       \large
%       \lineskip .5em
%       \begin{tabular}[t]{c}
%         \author
%       \end{tabular}
%       \par
%       }
%       % additional small space at the end of the author name
%       \vskip .5em
%       % additional empty line at the end of the title block
%       \vspace*{12pt}
%    \end{center}

% \begin{multicols}{2}

\twocolumn[{%
 \centering
 {\Large \bf Supplementary Materials for:\\Generalizable Local Feature Pre-training for Deformable Shape Analysis \par}
 {\vspace*{24pt}}
      {
      \large
      \lineskip .5em
      \begin{tabular}[t]{c}
        Souhaib Attaiki \hspace{1.5cm} Lei Li \hspace{1.5cm} Maks Ovsjanikov\\
LIX, \'Ecole Polytechnique, IP Paris
      \end{tabular}
      \par
      }
      % additional small space at the end of the author name
      \vskip .5em
      % additional empty line at the end of the title block
      \vspace*{12pt}
}]
\appendix

In this document, we collect all the results and discussions, which, due to the page limit, could not find space in the main manuscript.
This supplementary material consists of two parts.
First, in \cref{suppsec:implementation_details}, we describe more implementation details mainly regarding our pilot study, local feature pre-training, and experiments on downstream deformable shape data.
Next, in \cref{suppsec:additional_results}, we present additional experimental results and analysis of our local feature pre-training strategy and its generalization in downstream tasks, including deformable shape matching and segmentation. 


\section{Implementation Details}
\label{suppsec:implementation_details}


\subsection{Feature Locality vs. Transferability}
\label{suppsubsec:feature_locality_vs_transferability}
In Sec.~3 of the main text, we conducted a pilot study on feature locality vs. transferability on deformable shapes.
We tested three different architectures for pre-training a \textit{local} feature extractor, and their details are as follows.

\mypara{SparseConv.}
We used the \texttt{ResNet14} architecture introduced in \cite{choy20194d}.
During pre-training, given a 3D point cloud $P$, a fixed-size local patch with a radius of 0.15 is cropped at point $\mathbf{p} \in P$ and then reoriented with a local reference frame (LRF) computed by the method in \cite{gojcic2019perfect} for rotation invariance.
The resulting local patch is fed to the sparse convolution network, which extracts a 32-dimensional feature vector for point $\mathbf{p}$.

\mypara{PCPNet.}
It is a variant of PointNet \cite{qi2017pointnet} endowed with a quaternion spatial transformer.
We used the single-scale architecture proposed by \cite{pcpnet2018}.
PCPNet is designed to be a local network requiring input patches to have a fixed number of points.
Thus during pre-training, a fixed-size local patch (radius = 0.15) is cropped at point $\mathbf{p}$ and reoriented by an LRF. 
The local patch is then resampled to 1,024 points and fed to the network, resulting in a 32-dimensional feature vector for point $\mathbf{p}$.

\mypara{3DCNN.}
We used the architecture from \cite{li2021updesc} with a learnable receptive field size and differentiable voxelization, the same as our \OurMethodName{} in Sec.~4.1 of the main text.
More details can be found in \cref{suppsubsec:local_feature_pretraining}.

\mypara{Dataset.}
We pre-trained the above local networks on the 3DMatch dataset, which is a collection of RGB-D scan datasets with 62 indoor scenes and 4,142 point cloud fragments. 
There are 13K points on average in a fragment after downsampling.

\mypara{Loss.}
We used the PointInfoNCE loss, in which 300 point correspondences were randomly sampled for a pair of point clouds for faster training and the temperature parameter $\tau$ was set to 0.07. 

We also used the cycle consistency loss $\mathcal{L}_c$. During pre-training, we use the extracted features to build correspondences for rigid alignment between shapes $P$ and $Q$.
The intuition for $\mathcal{L}_c$ is that the estimated transformation $(\mathbf{R}, \mathbf{t})$ aligning $P$ to $Q$ should be the inverse of the transformation $(\mathbf{R}', \mathbf{t}')$ aligning $Q$ to $P$.
Mathematically, this can be expressed as:

\begin{equation}
\begin{bmatrix}
\mathbf{R} & \mathbf{t}\\
\mathbf{0} & 1
\end{bmatrix}
\begin{bmatrix}
\mathbf{R'} & \mathbf{t'}\\
\mathbf{0} & 1 
\end{bmatrix}
=
\begin{bmatrix}
\mathbf{R}\mathbf{R'} & \mathbf{R}\mathbf{t'} + \mathbf{t}\\
\mathbf{0} & 1
\end{bmatrix}
= \mathbf{I}
\end{equation}

\mypara{Application to deformable shape matching.}
In Fig. 3 of the main text, we have shown the results of shape matching on the Faust Remeshed dataset, directly using the pre-trained feature extractors. Given two shapes $S_1$, and $S_2$, we compute their respective point-wise features $F_1$ and $F_2$ using a specific pre-trained model. We first produce an estimate of the point-to-point maps $T_{21}^{nn}$ and $T_{12}^{nn}$ using nearest neighbor search between $F_1$ and $F_2$. We then filter the correspondences by mutual check: a pair of points $x \in S_1, y \in S_2$ is considered to be in correspondence, if and only if in the feature space, $x$ is the nearest neighbor of $y$, and $y$ is the nearest neighbor of $x$. This results in two filtered maps $T_{21}^{mf}$ and $T_{12}^{mf}$. Finally, we further refine these two maps using the ZoomOut method \cite{Melzi_2019}, which is based on navigating between the spectral and spatial domains while progressively increasing the number of spectral basis functions. We emphasize that if the initial point-to-point map is noisy or contains strong ambiguities like symmetry ambiguities, ZoomOut is not able to remedy these errors, thus leading to final correspondences of bad quality. We perform 10 iterations of ZoomOut, starting from 30 eigenfunctions up to 100 eigenfunctions.



\subsection{Local Feature Pre-training}
\label{suppsubsec:local_feature_pretraining}
In Sec.~4.1 of the main text, we introduced our local feature pre-training strategy.

\mypara{Feature extraction.} We use $r_{\text{LRF}}=0.3$ and $\sigma=10^{-3}$ for differentiable voxelization \cite{li2021updesc}, and the voxel grid resolution is set to $16^3$.
We pre-trained on the 3DMatch dataset introduced in \cref{suppsubsec:feature_locality_vs_transferability}.

\mypara{Pre-training loss.} 
For the PointInfoNCE loss $\mathcal{L}_{\text{nce}}$, its settings are described in \cref{suppsubsec:feature_locality_vs_transferability}.
For the cycle consistency loss $\mathcal{L}_{\text{c}}$, 300 points were randomly sampled on each point cloud for feature extraction and alignment estimation.
A relaxation-based solver is used in $\mathcal{L}_{\text{c}}$ for estimating a 3D transformation between two point clouds, and its details can be found in \cite{li2021updesc}.
 
In the main text, we investigated the performance difference between the cycle consistency loss and PointInfoNCE loss w.r.t learned feature smoothness.
Suppose that $F \in \mathbb{R}^{m \times n}$ is the matrix of extracted $n$-dimensional point-wise features for a shape of $m$ vertices, we measure the Dirichlet energy as follows: 
\begin{equation}
    % E_{Dirichlet}(F) = \frac{1}{n} \sum_{i=1}^n \frac{F_i^{\top} W F_i}{F_i^{\top} A F_i},
    E_{Dirichlet}(F) = \frac{1}{n} \sum_{i=1}^n F_i^{\top} W F_i,
\end{equation}
where $F_i$ is the $i^{\text{th}}$ column of $F$, and $W$ is the standard stiffness matrix computed using the classical cotangent discretization scheme of the Laplace-Beltrami operator \cite{Pinkall1993}.



\subsection{Baselines}
In Sec.~5 of the main text, we tested our proposed \OurMethodName{} features against a wide spectrum of competitors, including both hand-crafted and learned features.

Specifically, the Heat Kernel Signature (HKS) and Wave Kernel Signature (WKS) features are both sampled at 100 values of energy \textit{t}, logarithmically spaced in the range proposed in their respective original papers.
SHOT descriptors are 352-dimensional, and we used the implementation from the PCL library \cite{Rusu_ICRA2011_PCL}.
PointContrast features are 32-dimensional, and we used the publicly available implementation and the pre-trained weights released by the authors\footnote{\url{https://github.com/facebookresearch/PointContrast}}.


\subsection{Downstream Shape Analysis Training}
In Sec.~5 of the main text, we used DiffusionNet on top of the baselines features and our \OurMethodName{} respectively, in both the shape matching and segmentation tasks. We employed the publicly available implementation of DiffusionNet released by the authors\footnote{\url{https://github.com/nmwsharp/diffusion-net}}.
Unless specified otherwise, in our experiments, we used four DiffusionNet blocks of width = 128. 
The DiffusionNet is trained by an ADAM optimizer \cite{kingma2017adam} with an initial learning rate of $10^{-3}$.

In Sec.~5.1 of the main text, we also used a point-wise MLP network on top of the baselines features and our \OurMethodName{} respectively for supervised shape matching.
For this, we use the same MLP architecture as in FMNet \cite{litany2017deep}. 
After computing the point features with the MLP, we use them to compute the predicted functional map $C_{pred}$ as in \cite{donati2020deep} and penalize its deviation from the ground-truth map $C_{gt}$ using the L2 loss: $L = \|C_{pred} - C_{gt}\|_2^2$.


\subsection{Computational Specifications}
All our experiments were executed using Pytorch \cite{NEURIPS2019_9015}, on a 64-bit machine, equipped with an Intel(R) Xeon(R) CPU E5-2630 v4 @ 2.20GHz and an RTX 2080 Ti Graphics Card.

In terms of computational time, pertaining our method takes about 12 hours on a single RTX 2080 Ti Graphics Card, in contrast to the 64 hours required for PointContrast. The receptive field optimization takes about 20 minutes per dataset. For feature extraction, our method takes 3 seconds to extract local features for a 5000-vertex shape, which is on par with other local features like SHOT~\cite{tombari2010unique}, but slower than PointContrast (0.1s). Finally, the forward pass using \OurMethodName{} takes the same time as for all baseline features, e.g., 0.2 seconds per iteration for the unsupervised shape-matching experiment in Sec 5.1 of the main text.


\section{Additional Results and Analysis}
\label{suppsec:additional_results}

\subsection{Size of the learned receptive field}
Fig.~5 of our paper provides an illustration of the optimized receptive field in downstream tasks. In \cref{fig:receptive_field}, we include more visualizations for shape \textit{pairs} for both humans and animals.
Observe that the optimized receptive field indeed corresponds to interpretable concepts, such as the head or foot of a human, and is consistent across shape pairs.

\begin{figure}[t]
  \centering
  \includegraphics[width=0.99\linewidth]{figures/patch_viz.pdf}
   \caption{Visualizing the optimized receptive field for shape pairs.}
   % \vspace{-0.6cm}
   \label{fig:receptive_field}
\end{figure}


\subsection{Human Shape Matching} 
\label{suppsubsec:human_matching}
In Sec.~5.1 of the main text, we performed unsupervised shape matching on the FAUST-Remeshed (FR), SCAPE-Remeshed (SR), and SHREC’19 datasets (SH) and reported the matching performance in Tab.~1.
We provide additional quantitative results of the FR-SR and SR-FR settings in \cref{tab:unaligned_unsup_supp}.
Compared with the baseline features, our \OurMethodName{} has the best and most consistent performance in both settings.

\begin{table}[t]
    \begin{center}
    \ra{1.0}
        \resizebox{0.8\columnwidth}{!}{%
            \begin{tabular}{@{} lrr @{}}
                \toprule
                \textbf{Method / Dataset}                & \textbf{FR}-\textbf{SR} & \textbf{SR}-\textbf{FR} \\
                \midrule
                SURFMNET                                 & 15.2                    & 9.5                     \\
                Cyclic FMaps                             & 23                      & 23.2                    \\
                WSupFMNet                                & 27.1                    & 14.2                    \\
                Deep Shells                              & 6.0                     & \textbf{3.4}            \\
                \midrule
                DiffusionNet - XYZ                       & 25.7                    & 8.4                     \\ % 8.1 & 23.1 &
                DiffusionNet - HKS                       & 7.9                     & 23                      \\ % 7.1 & 15.4 &
                DiffusionNet - WKS                       & 4.2                     & 24.1                    \\ % 3.8 & 4.4 &
                DiffusionNet - SHOT                      & 7.2                     & 4.1                     \\ % 3.8 & \textbf{4.2} &
                DiffusionNet - PCH                       & 11.4                    & 8.7                     \\ % 6.3 & 4.3 &
                DiffusionNet - PCN                       & 20.4                    & 9.1                     \\ % 8.7 & 4.8 &
                DiffusionNet - \OurMethodName{} (ours)   & \textbf{4.1}            & 3.9                     \\ % % 8.7 & 4.8 &
                \bottomrule
            \end{tabular}
        }
        \caption{Accuracy of various features for unsupervised shape matching on un-aligned data.  X-Y means train on X and test on Y. Values are mean geodesic error $\times 100$ on unit-area shapes.}
        \label{tab:unaligned_unsup_supp}
    \end{center}
\end{table}


% %\setlength{\tabcolsep}{4pt}
\begin{table}
    \begin{center}
        \begin{tabular}{lccc}
            \toprule
            \textbf{Method / Dataset}     & \textbf{FO}-\textbf{FR} & \textbf{FO}-\textbf{FQ} &
            \textbf{FO}-\textbf{SR}                                                                           \\
            \midrule
            MLP - XYZ                     & 13.8                    & 12.8                    & 26.9          \\
            MLP - HKS                     & 12.4                    & 21.5                    & 23.4          \\
            MLP - WKS                     & 12.7                     & 34.0                    & 27.2          \\
            MLP - SHOT                    & 10.0                    & 17.6                    & 10.6          \\
            MLP - PCH                     & 17.7                    & 41.5                    & 32.8          \\
            MLP - PCN                     & 15.7                    & 41.4                    & 28.4          \\
            MLP - \OurMethodName{} (ours) & \textbf{4.2}            & \textbf{4.9}            & \textbf{9.4} \\
            \bottomrule
        \end{tabular}
        \caption{Accuracy of various features for supervised shape matching when the connectivity changes from train to test.  X-Y means train on X and test on Y. Values are mean geodesic error $\times 100$ on unit-area shapes. \rev{Are the \OurMethodName{} results in this table up-to-date? If not, please remove the table. Corresponding discussions have been commented out.}}
        \label{tab:robust_connectivity}
    \end{center}
\end{table}
%\setlength{\tabcolsep}{1.4pt}



% \rev{In the main text, we also tested the robustness of our \OurMethodName{} features to the change of remeshing.
% To complement Fig.~6 of the main text, we provide additional experiments using the same training setup, by training on the FO dataset, and testing on the FR and SR datasets, respectively.
% \cref{tab:robust_connectivity} shows the evaluation results.
% We observe that our method consistently outperforms the competing features.
% This indicates that our \OurMethodName{} features are robust and descriptive under change of sampling and can generalize well across datasets (FO-SR setting).}

\subsection{Molecular Surface Segmentation} 
\label{suppsubsec:qualitative_evaluation}
In \cref{fig:rna_seg_qual}, we show qualitative results of RNA segmentation using DiffusionNet + \OurMethodName{}.
It can be seen that the challenging RNA molecules can be robustly segmented into functional components with our pre-trained features.

\begin{figure}[t]
    \centering
    \includegraphics[width=\linewidth]{figures/mol_seg.pdf}
    \caption{Qualitative evaluation of RNA segmentation on the dataset of \cite{poulenard2019effective}. Left: ground truth. Right: prediction by DiffusionNet + \OurMethodName{}.}
    \label{fig:rna_seg_qual}
\end{figure}



\subsection{Human Shape Segmentation}
\label{suppsubsec:human_segmentation}

We performed an additional experiment on the human shape segmentation task.
We used the dataset introduced in \cite{maron2017convolutional}, which combines segmented human models taken from a variety of existing datasets.
We used the same train/test split of 380 training and 18 test shapes as in prior works.
We compared our \OurMethodName{} only with methods that used the original evaluation protocol as in \cite{maron2017convolutional}, i.e., without using post-processing and evaluating the results on the full shape resolution (techniques such as Mesh Walker \cite{lahav2020meshwalker} are thus excluded). 

We ran each experiment five times and report the mean and standard deviation of the accuracy in \cref{tab:human-segmentation}.  
Our \OurMethodName{} features achieve an accuracy of $92.4 \pm 0.25\%$, the state-of-the-art result on this dataset.
In \cref{fig:human_seg_qual}, we present qualitative results of human segmentation using DiffusionNet + \OurMethodName{}.
Note that the segmentation results are simply the network predictions, and we do not perform any complex post-processing to the segmentation.

%\setlength{\tabcolsep}{4pt}
\begin{table}[t]
    \begin{center}
    \ra{1.0}
        \begin{tabular}{@{}lr@{}}
            % \cmidrule[\heavyrulewidth]{1-2}
            \toprule
            \textbf{Method}                           & \textbf{Accuracy $\pm$ s.d}                   \\
            % \cmidrule{1-2}
            \midrule
            GCNN \cite{masci2015geodesic}             & 86.4\%                                        \\
            ACNN \cite{boscaini2016learning}          & 83.7\%                                        \\
            Toric Cover \cite{maron2017convolutional} & 88.0\%                                        \\
            PointNet++ \cite{qi2017pointnet}          & 90.8\%                                        \\
            MDGCNN \cite{poulenard2019effective}      & 88.6\%                                        \\
            DGCNN \cite{wang2019dgcnn}                & 89.7\%                                        \\
            SNGC \cite{haim2019surface}               & 91.0\%                                        \\
            CGConv \cite{yang2021continuous}          & 89.9\%                                        \\
            \cmidrule{1-2}
            DiffusionNet - XYZ                        & 91.9 $\pm$ 0.27\%                             \\
            DiffusionNet - HKS                        & 91.5 $\pm$ 0.21\%                             \\
            DiffusionNet - WKS                        & 91.8 $\pm$ 0.33\%                             \\
            DiffusionNet - SHOT                       & 91.5 $\pm$ 0.77\%                             \\
            DiffusionNet - PCH                        & 85.6 $\pm$ 0.75\%                             \\
            DiffusionNet - PCN                        & 87.3 $\pm$ 0.57\%                             \\
            DiffusionNet - \OurMethodName{} (ours)    & \textbf{92.4 $\pm$ 0.25\%} \accuchange{+0.9}  \\
            % \cmidrule[\heavyrulewidth]{1-2}
            \bottomrule
        \end{tabular}
        \caption{Human shape segmentation on the dataset of \cite{maron2017convolutional}. Our \OurMethodName{} achieves the state-of-the-art performance among methods that do not perform post-processing and evaluate on the full shape resolution. The reported numbers are the mean and standard deviation of the accuracy over five runs initialized randomly.}
        \label{tab:human-segmentation}
    \end{center}
\end{table}
%\setlength{\tabcolsep}{1.4pt}


\begin{figure}[t]
    \centering
    \includegraphics[width=\linewidth]{figures/human_seg.pdf}
    \caption{Qualitative evaluation of human shape segmentation on the dataset of \cite{maron2017convolutional}. Left: ground truth. Right: prediction by DiffusionNet + \OurMethodName{}.}
    \label{fig:human_seg_qual}
\end{figure}



% \subsection{Training with Limited Data}
% Train on small dataset to show how informative the features are

% we can obtain better results by training on a small dataset
% see table

% not worth it, experiment dropped

\subsection{Robustness to Noise}
We performed an additional experiment to evaluate the robustness of our features to noise. For this, we followed the same setup as in Sec.~5.1 of the main text and in \cref{suppsubsec:human_matching}, by performing unsupervised learning on FR and testing on SR with an increasing amount of noise as input.
We compared our method to the best three competing features. The results are shown in \cref{fig:noise_robust} - left. It can be seen that our features are more robust to noise, i.e., the performance does not vary much with different noise levels (the intensity of the noise can be seen in \cref{fig:noise_robust} - right), which is not the case with other features, such as SHOT, whose performance degrades very quickly.


\begin{figure}[t]
    \centering
    %\includegraphics[width=0.9\columnwidth]{figures/noise_robust.pdf}
    \includegraphics[width=1.0\columnwidth]{figures/noise_levels.pdf}
    \caption{Left: Evolution of the geodetic error as a function of different
input noise levels. Right: Qualitative visualization of noise levels.}
    \label{fig:noise_robust}
\end{figure}



\subsection{Convergence Speed}

In our experiments, we observed that our \OurMethodName{} descriptors take less time to train and facilitate learning. To demonstrate this, we show in \cref{fig:convergence_speed} the evolution of validation accuracy during learning of the RNA segmentation task (Sec. 5.2 of the main text). It can be seen that compared to the other features, VADER requires far fewer training iterations to achieve similar performance. This clearly indicates the better descriptiveness and generalizability of our features.

\begin{figure}[t]
    \centering
    \includegraphics[width=0.9\columnwidth]{figures/eval_acc_seg_new.pdf}
    \caption{Evolution of the RNA segmentation accuracy on the validation set, during the training of DiffusionNet with different features.}
    \label{fig:convergence_speed}
\end{figure}





% \subsection{Ablation Study}
% \label{suppsubsec:ablation_study}

% In Sec.~5.5 of the main text, we investigated different 3D datasets for local feature pre-training and showed that local geometries in 3DMatch are richer than those in DFAUST.
% We performed PCA on the local patches of 3DMatch and DFAUST.
% For each dataset, we first randomly extracted 200K local patches. 
% We then encode each patch as a high dimensional vector by first orienting it using a local reference frame and then voxelizing it to a small 3D grid of resolution = $16^3$ using the method of \cite{gojcic2019perfect}. 
% The resulting vectors are 4096-dimensional and fed as input to PCA to analyze the internal richness and complexity of each dataset.
% In \cref{fig:pca_unexplained_supp}, we report the unexplained variance as a function of the number of principal components. It further confirms that 3DMatch is significantly more diverse than DFAUST, since more principal components are needed to explain its full variance. 

% \begin{figure}[t]
    \centering
    \includegraphics[width=0.9\columnwidth]{figures/unexplained_variance.pdf}
    \caption{Unexplained variance in PCA of local patches from DFAUST and 3DMatch.}
    \label{fig:pca_unexplained_supp}
\end{figure}







%%%%%%%%% REFERENCES
{\small
    \bibliographystyle{ieee_fullname}
    \bibliography{references}
}

\end{document}
