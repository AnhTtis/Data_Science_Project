% ****** Start of file apssamp.tex ******
%
%   This file is part of the APS files in the REVTeX 4.1 distribution.
%   Version 4.1r of REVTeX, August 2010
%
%   Copyright (c) 2009, 2010 The American Physical Society.
%
%   See the REVTeX 4 README file for restrictions and more information.
%
% TeX'ing this file requires that you have AMS-LaTeX 2.0 installed
% as well as the rest of the prerequisites for REVTeX 4.1
%
% See the REVTeX 4 README file
% It also requires running BibTeX. The commands are as follows:
%
%  1)  latex apssamp.tex
%  2)  bibtex apssamp
%  3)  latex apssamp.tex
%  4)  latex apssamp.tex
%
\documentclass[%
preprint,
superscriptaddress,
%groupedaddress,
%unsortedaddress,
%runinaddress,
%frontmatterverbose, 
%preprint,
%showpacs,preprintnumbers,
%nofootinbib,
%nobibnotes,
%bibnotes,
 amsmath,amssymb,
 aps,
%pra,
%prb,
%rmp,
%prstab,
%prstper,
%floatfix,
longbibliography,
]{revtex4-2}

\usepackage{graphicx}% Include figure files
\usepackage{dcolumn}% Align table columns on decimal point
\usepackage{bm}% bold math

\usepackage{physics}%Dirac notation for states
\usepackage{siunitx} % Typesetting of values and units
\usepackage{booktabs} % For tables

\usepackage{xcolor}
\usepackage[normalem]{ulem}


\usepackage[pagebackref=false]{hyperref}
\renewcommand{\figurename}{Supplementary Fig.}

\begin{document}

\author{K.~Hecker$^*$}
\affiliation{JARA-FIT and 2nd Institute of Physics, RWTH Aachen University, 52074 Aachen, Germany,~EU}%
\affiliation{Peter Gr\"unberg Institute  (PGI-9), Forschungszentrum J\"ulich, 52425 J\"ulich,~Germany,~EU}

\author{L.~Banszerus$^*$}
\affiliation{JARA-FIT and 2nd Institute of Physics, RWTH Aachen University, 52074 Aachen, Germany,~EU}%
\affiliation{Peter Gr\"unberg Institute  (PGI-9), Forschungszentrum J\"ulich, 52425 J\"ulich,~Germany,~EU}

\author{A.~Sch\"apers}
\affiliation{JARA-FIT and 2nd Institute of Physics, RWTH Aachen University, 52074 Aachen, Germany,~EU}%


\author{S.~M\"oller}
\affiliation{JARA-FIT and 2nd Institute of Physics, RWTH Aachen University, 52074 Aachen, Germany,~EU}%
\affiliation{Peter Gr\"unberg Institute  (PGI-9), Forschungszentrum J\"ulich, 52425 J\"ulich,~Germany,~EU}


\author{A.~Peters}
\affiliation{JARA-FIT and 2nd Institute of Physics, RWTH Aachen University, 52074 Aachen, Germany,~EU}%

\author{E. Icking}
\affiliation{JARA-FIT and 2nd Institute of Physics, RWTH Aachen University, 52074 Aachen, Germany,~EU}%
\affiliation{Peter Gr\"unberg Institute  (PGI-9), Forschungszentrum J\"ulich, 52425 J\"ulich,~Germany,~EU}

\author{K.~Watanabe}
\affiliation{Research Center for Functional Materials, 
National Institute for Materials Science, 1-1 Namiki, Tsukuba 305-0044, Japan
}
\author{T.~Taniguchi}
\affiliation{ 
International Center for Materials Nanoarchitectonics, 
National Institute for Materials Science,  1-1 Namiki, Tsukuba 305-0044, Japan
}

\author{C.~Volk}
\author{C.~Stampfer}
\affiliation{JARA-FIT and 2nd Institute of Physics, RWTH Aachen University, 52074 Aachen, Germany,~EU}%
\affiliation{Peter Gr\"unberg Institute  (PGI-9), Forschungszentrum J\"ulich, 52425 J\"ulich,~Germany,~EU}%


\title{Supplementary Information\\
Coherent Charge Oscillations\\
in a Bilayer Graphene Double Quantum Dot}
\date{\today}% It is always \today, today,
             %  but any date may be explicitly specified
             
\keywords{}
\maketitle
\newpage



\section{Gate lever arm}
The lever arm $\alpha$ converting the voltage applied to the left FG, $V_\mathrm{L}$, into the detuning energy $\varepsilon$ can be determined by fitting Eq.~(1) in the main manuscript to the photon assisted tunneling (PAT) data (see Fig.~2h of the main manuscript). 
Supplementary Fig.~\ref{fs1} shows a finite bias charge stability diagram of a triple point where the dashed black lines mark a gate voltage range $V_\mathrm{SD}/\alpha$. The outline of the triple point matches very well the independently evaluated lever arm from the PAT measurements.
%%%%%%%%%%%%%%%%%%%%% Figure S1 %%%%%%%%%%%%%%%%%%%%%%%
\begin{figure*}[h]
\centering
\includegraphics[draft=false,keepaspectratio=true,clip,width=0.65\linewidth]{FigS1.pdf}
\caption[Fig00]{\textbf{Charge stability diagram and lever arm.} Charge stability diagram as shown in Fig.~2a in the main manuscript with a finite bias voltage of $V_\mathrm{SD} = 0.5~$mV applied. The black dashed lines indicate the extension of the bias window in detuning, using the lever arm $\alpha$ extracted from the PAT data shown in Fig.~2h in the manuscript. The gray lines are guides to the eye, highlighting the outline of the tripe point pair and regions of co-tunneling.
}
\label{fs1}
\end{figure*}
%%%%%%%%%%%%%%%%%%%%%%%%%%%%%%%%%%%%%%%%%%%%%%%%%%%%%




\newpage
\section{Complementary data on photon-assisted tunneling}
%%%%%%%%%%%%%%%%%%%%% Figure S2 %%%%%%%%%%%%%%%%%%%%%%%
\begin{figure*}[h]
\centering
\includegraphics[draft=false,keepaspectratio=true,clip,width=\linewidth]{FigS2.pdf}
\caption[Fig01]{\textbf{Photon assisted tunneling for varying frequencies.} Complementary data of Fig.~2d-f in the main manuscript measured in a frequency range of $f=10-\SI{30}{GHz}$. Black arrows are highlighting the positions of the PAT resonances shifting on the detuning axis with the applied frequency, $f$. 
}
\label{fs2}
\end{figure*}
%%%%%%%%%%%%%%%%%%%%%%%%%%%%%%%%%%%%%%%%%%%%%%%%%%%%%


Photon-assisted tunneling (PAT) spectroscopy can be used to determine the interdot tunnel coupling $\Delta/(2h)$ and the ensemble charge decoherence time $T^*_{2,\mathrm{PAT}}$.
For that purpose, the power of the microwave excitation has been optimized to achieve a sufficient signal-to-noise ratio while avoiding multi-photon absorption processes~\cite{mavalankar2016photon}. 
%
Supplementary Fig.~\ref{fs2} shows a set of charge stability diagrams of a triple point with an applied microwave excitation ranging from $f = \SI{10}{GHz}$ to $f = \SI{30}{GHz}$ (c.f. Figs.~2d-f in the main manuscript). The black arrows highlight the splitting of the PAT peaks increasing with the applied microwave excitation frequency.
%

Higher order PAT resonances that emerge as the microwave power is increased are shown in Supplementary Fig.~\ref{fs2_}a. Here, a line cut as function of finger gate voltage, $V_\mathrm{L}$, is shown as a function of applied power. The elevated power level renormalizes the tunnel coupling by the squared Bessel function and can be described by the function $f = \sqrt{(\alpha \delta V_\mathrm{L})^2+J_0(a)^2\Delta^2}/h$, where $a=e\beta V_\mathrm{rms}/hf$. In the case of low applied power ($a\ll 1$), $J^2_0\approx1$, the tunnel coupling can be estimated directly by fitting Eq.~1 in the main manuscript \cite{vanderWiel2002Dec}. Supplementary Fig.~\ref{fs2_}b displays the averaged current across the PAT resonances of orders $n=1,2,3$. A comparison with the inset, presenting the squared Bessel functions, reveals the anticipated  proportionality of higher order PAT \cite{vanderWiel2002Dec}. The power for the evaluation of $\Delta$ is set to a minimum, where only the first order PAT process is visible.

%%%%%%%%%%%%%%%%%%%%% Figure S2_ %%%%%%%%%%%%%%%%%%%%%%%
\begin{figure*}[h]
\centering
\includegraphics[draft=false,keepaspectratio=true,clip,width=0.8\linewidth]{FigS2X.pdf}
\caption[Fig00]{\textbf{Line trace as function of the applied microwave power} \textbf{a}~Line cut through a triple point at $V_\mathrm{R}=\SI{3.3601}{V}$, $V_\mathrm{SD}=\SI{5}{\mu V}$ as function of the power of the applied sine pulse with a frequency of $f=\SI{23}{GHz}$. \textbf{b} Averaged current through the DQD system along the PAT peaks of the first, second and third order (n=1,2,3) as function of the effective power $V_\mathrm{rms}$. Inset: Squared Bessel function of the first to third order as function of the parameter $a=e\beta V_\mathrm{rms}/hf$.
}
\label{fs2_}
\end{figure*}
%%%%%%%%%%%%%%%%%%%%%%%%%%%%%%%%%%%%%%%%%%%%%%%%%%%%%


%%%%%%%%%%%%%%%%%%%%% Figure S3 %%%%%%%%%%%%%%%%%%%%%%%
\begin{figure*}[h]
\centering
\includegraphics[draft=false,keepaspectratio=true,clip,width=0.58\linewidth]{FigS3.pdf}
\caption[Fig00]{\textbf{Line traces with fits.} Line traces of the data shown in Supplementary Fig.~2 at a frequency of $f=\SI{20}{GHz}$. The dashed lines show fits to the negative (blue) and positive (red) PAT peaks. The left y-axis marks position of the line trace on the $V_\mathrm{R}$-axis, while the right y-axis shows the current, $I + nI_0$, with $I_0=\SI{20}{pA}$ and the index of the trace, $n$.
}
\label{fs3}
\end{figure*}
%%%%%%%%%%%%%%%%%%%%%%%%%%%%%%%%%%%%%%%%%%%%%%%%%%%%%


Supplementary Fig.~\ref{fs3} shows line cuts through the triple point measured at $f= \SI{20}{GHz}$. Lorentzian line shapes are fitted to the positive (red dashed line) and negative (blue dashed line) PAT peaks (c.f. Fig.~2g in the main manuscript). The shown fits are used to evaluate the peak separation at different $V_\mathrm{R}$. For each frequency, $f$, $2\delta V_\mathrm{L}$ is found by averaging the peak separation of several line traces. The tunnel coupling $\Delta/(2h)$ is determined as described in the main manuscript (see Eq. (1) and Fig.~2h).
%
An estimate of the ensemble charge decoherence time can be extacted from the FWHM, $\gamma$, of the Lorentzian peaks according to $T^*_{2,\mathrm{PAT}} = 2h/(\alpha\gamma)$~\cite{petta2004manipulation,petersson2010quantum}. The decoherence times evaluated from several line traces of the data in Supplementary Fig.~\ref{fs2} and Fig.~2d-f are shown as a histogram in Fig.~5f in the main manuscript.


\clearpage
\section{Complementary data on coherent oscillations}
This section presents further data on the Landau-Zener-St\"uckelberg (LZSM) interference pattern to elucidate the effect of different measurement parameters. 

%%%%%%%%%%%%%%%%%%%%% Figure S4 %%%%%%%%%%%%%%%%%%%%%%%
\begin{figure*}[h]
\centering
\includegraphics[draft=false,keepaspectratio=true,clip,width=\linewidth]{FigS4.pdf}
\caption[Fig01]{\textbf{Charge stability diagrams with applied square pulse.}
%
\textbf{a} Charge stability diagrams of a triple point with a pulse of amplitude $V_\mathrm{p} = \SI{200}{\milli\volt}$ as set at the arbitrary waveform generator (AWG) for different pulse durations $t_\mathrm{p}$. 
%
\textbf{b} Charge stability diagrams of a triple point with a pulse of duration $t_\mathrm{p} = \SI{231}{\pico\second}$ and different pulse amplitudes $V_\mathrm{p}$. 
}
\label{fs4}
\end{figure*}
%%%%%%%%%%%%%%%%%%%%%%%%%%%%%%%%%%%%%%%%%%%%%%%%%%%%%



Supplementary Fig.~\ref{fs4} shows the evolution of interference fringes in the charge stability diagram of a triple point, as a function of (a) the pulse duration $t_\mathrm{p}$ and (b) the pulse amplitude $V_\mathrm{p}$.
%
As explained in the main manuscript, these two parameters change the relative phase that the two parts of a wave function, split in an LZSM experiment, acquire before interfering with each other.
%
Supplementary Fig.~\ref{fs4}a shows a series of triple points where $t_\mathrm{p}$ is increased from \SI{92}{\pico\second} to \SI{277}{\pico\second}.
%
Due to the finite rise time of the pulse of $t_\mathrm{r} \approx 140~$ps, the effective pulse amplitude $A_\mathrm{p}$ applied to the sample is reduced if $t_\mathrm{p} \lesssim t_\mathrm{r}$. 
%
Thus, less interference fringes can be observed in the triple points recorded at $t_\mathrm{p} = 92~$ps and $t_\mathrm{p} = 138~$ps.
%
Caused by dephasing, the fringes become less clearly defined for increasing $t_\mathrm{p}$.
%
In Supplementary Fig.~\ref{fs4}b, a set of measurements is shown, recorded at constant $t_\mathrm{p} = \SI{231}{\pico\second}$ while varying $V_\mathrm{p}$. 
A growing number of interference fringes can be observed for increased pulse amplitudes. 



%%%%%%%%%%%%%%%%%%%%% Figure S5 %%%%%%%%%%%%%%%%%%%%%%%
\begin{figure*}[]
\centering
\includegraphics[draft=false,keepaspectratio=true,clip,width=\linewidth]{FigS5.pdf}
\caption[Fig02]{\textbf{Coherent oscillations in the time domain.}
%
The panels show the effect of a pulse with amplitude $V_\mathrm{p} = \SI{100}{\milli\volt}$ (\textbf{a}), \SI{150}{\milli\volt} (\textbf{b}) and \SI{200}{\milli\volt} (\textbf{c}). From the position of the first interference maximum, marked by black bars, the resulting effective amplitudes are determined to be $A_\mathrm{p} = \SI{148}{\micro\electronvolt}$, \SI{228}{\micro\electronvolt} and \SI{305}{\micro\electronvolt}, respectively. The data in panel \textbf{b} is the same as presented in Fig.~4b of the main manuscript.
}
\label{fs5}
\end{figure*}
%%%%%%%%%%%%%%%%%%%%%%%%%%%%%%%%%%%%%%%%%%%%%%%%%%%%%

The effect of changing the pulse amplitude can also be studied in the time domain. Supplementary Fig.~\ref{fs5} shows the results of LZSM interference experiments with the pulse duration ranging from \SI{0}{\pico\second} to \SI{300}{\pico\second}, for different values of $V_\mathrm{p}$.
%
The data sets have been acquired at the triple point shown in Fig.~4a of the main manuscript at a value of $V_\mathrm{R} = \SI{3.368}{\volt}$.
%
The effective pulse amplitude experienced by the sample is given by the position of the first fringe at a time $t_\mathrm{p} \gg t_\mathrm{r}$. As indicated by the black bars in Supplementary Fig. \ref{fs5}, one finds values of $A_\mathrm{p} = \SI{148}{\micro\electronvolt}$, \SI{228}{\micro\electronvolt} and \SI{305}{\micro\electronvolt}, respectively. This shows that $A_\mathrm{p}$ scales linearly with $V_\mathrm{p}$, as would be expected, with a conversion factor of $\beta \approx 1.50 \pm 0.02~\mu\mathrm{eV / mV}$.



%%%%%%%%%%%%%%%%%%%%% Figure S6 %%%%%%%%%%%%%%%%%%%%%%%
\begin{figure*}[]
\centering
\includegraphics[draft=false,keepaspectratio=true,clip,width=\linewidth]{FigS6.pdf}
\caption[Fig02]{\textbf{Coherent oscillations in the amplitude domain.}
%
The panels show the effect of a pulse with an duration of $t_\mathrm{p} = \SI{169}{\pico\second}$ (\textbf{a}), \SI{200}{\pico\second} (\textbf{b}) and \SI{246}{\pico\second} (\textbf{c}). Fig.~5a in the main manuscript shows a close-up of the data in panel \textbf{b}. The clarity of the interference fringes diminishes for increasing $t_\mathrm{p}$ due to dephasing.
}
\label{fs6}
\end{figure*}
%%%%%%%%%%%%%%%%%%%%%%%%%%%%%%%%%%%%%%%%%%%%%%%%%%%%%

The influence of the pulse duration, in turn, can be demonstrated in the amplitude domain. Supplementary Fig.~\ref{fs6} shows three amplitude-dependent measurements at different $t_\mathrm{p}$.
The clarity of the fringes diminishes as the pulse duration increases, indicating a progressive loss of coherence as the time between successive LZ transitions increases. 



%%%%%%%%%%%%%%%%%%%%% Figure S7 %%%%%%%%%%%%%%%%%%%%%%%
\begin{figure*}[]
\centering
\includegraphics[draft=false,keepaspectratio=true,clip,width=0.8\linewidth]{FigS7.pdf}
\caption[Fig02]{\textbf{Coherent oscillations at a different charge carrier occupation.} 
%
\textbf{a} Coherent oscillations in the amplitude domain complementary to the data shown in Fig.~5a of the main manuscript. A pulse with a width of $t_\mathrm{p} = \SI{200}{\pico\second}$ was applied. 
%
\textbf{b} The triple point at which the data set in \textbf{a} was recorded. Here, a pulse of $V_\mathrm{p} = \SI{175}{\milli\volt}$ and $t_\mathrm{p} = \SI{231}{\pico\second}$ was applied, while $V_\mathrm{R} = \SI{3.5513}{\volt}$.
}
\label{fs7}
\end{figure*}
%%%%%%%%%%%%%%%%%%%%%%%%%%%%%%%%%%%%%%%%%%%%%%%%%%%%%

\begin{figure*}%[H]
\centering
\includegraphics[draft=false,keepaspectratio=true,clip,width=0.8\linewidth]{FigS8.pdf}
\caption[FigR02]
{\textbf{Additional data set for lower tunnel coupling.} \textbf{a} Triple point with applied square pulse $t_\mathrm{p}=\SI{160}{ps}$, $t_\mathrm{i}=\SI{5}{ns}$ and $V_\mathrm{p}=\SI{100}{mV}$. \textbf{b} Line cut along the x-axis in a at $V_\mathrm{R}=\SI{3.334}{V}$ as function of the pulse duration $t_\mathrm{p}$.}\label{fR2}

\end{figure*}



The data sets can strongly vary from triple point to triple point, as changing the charge occupation in the DQD influences both the tunnel rates to the leads~\cite{ihn2009semiconductor} and the level spectrum~\cite{moller2021probing, eich2018spin}.
Supplementary Fig.~\ref{fs7}a shows a data set of amplitude-dependent LZSM interference measured at the triple point shown in Supplementary Fig.~\ref{fs7}b. Supplementary Fig.~\ref{fR2} depicts LZSM measurements at a different charge occupation where the overall tunneling rates have decreased. This is evident from the reduced current as well as the reduced LZSM oscillation period in detuning (a) and pulse duration $t_\mathrm{p}$ (b).

Supplementary Fig.~\ref{fS10} shows a LZSM dataset obtained from a second device fabricated with a similar device geometry.


\begin{figure}
\centering
\includegraphics[draft=false,keepaspectratio=true,clip,width=0.8\linewidth]{FigS10.pdf}
\caption[FigS10]
{\textbf{Additional data set from a second device.}
\textbf{a} False color scanning electron microscope image of the gate structure of the device. The DQD is formed below the two FGs (blue) by applying the voltages $V_\mathrm{L}$ and $V_\mathrm{R}$. Additionally, a sine pulse can be applied to the left FG ($V_\mathrm{AC}$).
\textbf{b} Charge stability diagram showing the triple point of the charge transition $(1,0)-(0,1)$ with a bias voltage of $V_\mathrm{SD}=\SI{1}{mV}$ and $V_\mathrm{b}=\SI{-7.4}{V}$. 
\textbf{c} Cut through the triple point in panel b as a function of the amplitude of the sine pulse with a frequency of $f=\SI{10}{GHz}$. 
\label{fS10}}
\end{figure}









\clearpage
\section{Remark on Fourier transform of the coherent oscillations}
Note that in Ref.~\cite{rudner2008quantum}, the decoherence time is deduced from the Fourier transform of the excitation probability $P$, whereas in our experiment, the measured observable is the current $I$.  
Due to the applied readout scheme, the current is proportional to the excitation probability, i.e. $I = b P$ with the proportionality factor $b$. This implies $I_\mathrm{FT} = b P_\mathrm{FT}$ and hence $\mathrm{ln}|I_\mathrm{FT}| = \mathrm{ln}|b| + \mathrm{ln}|P_\mathrm{FT}|$.
This validates that $T^*_\mathrm{2,FT}$ can be determined from $\mathrm{ln}|I_\mathrm{FT}|$ according to Eq.~(5) in the main manuscript. 

%\bibliography{literature}
\begin{thebibliography}{8}%
\makeatletter
\providecommand \@ifxundefined [1]{%
 \@ifx{#1\undefined}
}%
\providecommand \@ifnum [1]{%
 \ifnum #1\expandafter \@firstoftwo
 \else \expandafter \@secondoftwo
 \fi
}%
\providecommand \@ifx [1]{%
 \ifx #1\expandafter \@firstoftwo
 \else \expandafter \@secondoftwo
 \fi
}%
\providecommand \natexlab [1]{#1}%
\providecommand \enquote  [1]{``#1''}%
\providecommand \bibnamefont  [1]{#1}%
\providecommand \bibfnamefont [1]{#1}%
\providecommand \citenamefont [1]{#1}%
\providecommand \href@noop [0]{\@secondoftwo}%
\providecommand \href [0]{\begingroup \@sanitize@url \@href}%
\providecommand \@href[1]{\@@startlink{#1}\@@href}%
\providecommand \@@href[1]{\endgroup#1\@@endlink}%
\providecommand \@sanitize@url [0]{\catcode `\\12\catcode `\$12\catcode
  `\&12\catcode `\#12\catcode `\^12\catcode `\_12\catcode `\%12\relax}%
\providecommand \@@startlink[1]{}%
\providecommand \@@endlink[0]{}%
\providecommand \url  [0]{\begingroup\@sanitize@url \@url }%
\providecommand \@url [1]{\endgroup\@href {#1}{\urlprefix }}%
\providecommand \urlprefix  [0]{URL }%
\providecommand \Eprint [0]{\href }%
\providecommand \doibase [0]{https://doi.org/}%
\providecommand \selectlanguage [0]{\@gobble}%
\providecommand \bibinfo  [0]{\@secondoftwo}%
\providecommand \bibfield  [0]{\@secondoftwo}%
\providecommand \translation [1]{[#1]}%
\providecommand \BibitemOpen [0]{}%
\providecommand \bibitemStop [0]{}%
\providecommand \bibitemNoStop [0]{.\EOS\space}%
\providecommand \EOS [0]{\spacefactor3000\relax}%
\providecommand \BibitemShut  [1]{\csname bibitem#1\endcsname}%
\let\auto@bib@innerbib\@empty
%</preamble>
\bibitem [{\citenamefont {Mavalankar}\ \emph {et~al.}(2016)\citenamefont
  {Mavalankar}, \citenamefont {Pei}, \citenamefont {Gauger}, \citenamefont
  {Warner}, \citenamefont {Briggs},\ and\ \citenamefont
  {Laird}}]{mavalankar2016photon}%
  \BibitemOpen
  \bibfield  {author} {\bibinfo {author} {\bibfnamefont {A.}~\bibnamefont
  {Mavalankar}}, \bibinfo {author} {\bibfnamefont {T.}~\bibnamefont {Pei}},
  \bibinfo {author} {\bibfnamefont {E.~M.}\ \bibnamefont {Gauger}}, \bibinfo
  {author} {\bibfnamefont {J.~H.}\ \bibnamefont {Warner}}, \bibinfo {author}
  {\bibfnamefont {G.~A.~D.}\ \bibnamefont {Briggs}},\ and\ \bibinfo {author}
  {\bibfnamefont {E.~A.}\ \bibnamefont {Laird}},\ }\bibfield  {title} {\bibinfo
  {title} {{Photon-assisted tunneling and charge dephasing in a carbon nanotube
  double quantum dot}},\ }\href {https://doi.org/10.1103/PhysRevB.93.235428}
  {\bibfield  {journal} {\bibinfo  {journal} {Phys. Rev. B}\ }\textbf {\bibinfo
  {volume} {93}},\ \bibinfo {pages} {235428} (\bibinfo {year}
  {2016})}\BibitemShut {NoStop}%
\bibitem [{\citenamefont {van~der Wiel}\ \emph {et~al.}(2002)\citenamefont
  {van~der Wiel}, \citenamefont {De~Franceschi}, \citenamefont {Elzerman},
  \citenamefont {Fujisawa}, \citenamefont {Tarucha},\ and\ \citenamefont
  {Kouwenhoven}}]{vanderWiel2002Dec}%
  \BibitemOpen
  \bibfield  {author} {\bibinfo {author} {\bibfnamefont {W.~G.}\ \bibnamefont
  {van~der Wiel}}, \bibinfo {author} {\bibfnamefont {S.}~\bibnamefont
  {De~Franceschi}}, \bibinfo {author} {\bibfnamefont {J.~M.}\ \bibnamefont
  {Elzerman}}, \bibinfo {author} {\bibfnamefont {T.}~\bibnamefont {Fujisawa}},
  \bibinfo {author} {\bibfnamefont {S.}~\bibnamefont {Tarucha}},\ and\ \bibinfo
  {author} {\bibfnamefont {L.~P.}\ \bibnamefont {Kouwenhoven}},\ }\bibfield
  {title} {\bibinfo {title} {{Electron transport through double quantum
  dots}},\ }\href {https://doi.org/10.1103/RevModPhys.75.1} {\bibfield
  {journal} {\bibinfo  {journal} {Rev. Mod. Phys.}\ }\textbf {\bibinfo {volume}
  {75}},\ \bibinfo {pages} {1} (\bibinfo {year} {2002})}\BibitemShut {NoStop}%
\bibitem [{\citenamefont {Petta}\ \emph {et~al.}(2004)\citenamefont {Petta},
  \citenamefont {Johnson}, \citenamefont {Marcus}, \citenamefont {Hanson},\
  and\ \citenamefont {Gossard}}]{petta2004manipulation}%
  \BibitemOpen
  \bibfield  {author} {\bibinfo {author} {\bibfnamefont {J.~R.}\ \bibnamefont
  {Petta}}, \bibinfo {author} {\bibfnamefont {A.~C.}\ \bibnamefont {Johnson}},
  \bibinfo {author} {\bibfnamefont {C.~M.}\ \bibnamefont {Marcus}}, \bibinfo
  {author} {\bibfnamefont {M.~P.}\ \bibnamefont {Hanson}},\ and\ \bibinfo
  {author} {\bibfnamefont {A.~C.}\ \bibnamefont {Gossard}},\ }\bibfield
  {title} {\bibinfo {title} {{Manipulation of a Single Charge in a Double
  Quantum Dot}},\ }\href {https://doi.org/10.1103/PhysRevLett.93.186802}
  {\bibfield  {journal} {\bibinfo  {journal} {Phys. Rev. Lett.}\ }\textbf
  {\bibinfo {volume} {93}},\ \bibinfo {pages} {186802} (\bibinfo {year}
  {2004})}\BibitemShut {NoStop}%
\bibitem [{\citenamefont {Petersson}\ \emph {et~al.}(2010)\citenamefont
  {Petersson}, \citenamefont {Petta}, \citenamefont {Lu},\ and\ \citenamefont
  {Gossard}}]{petersson2010quantum}%
  \BibitemOpen
  \bibfield  {author} {\bibinfo {author} {\bibfnamefont {K.~D.}\ \bibnamefont
  {Petersson}}, \bibinfo {author} {\bibfnamefont {J.~R.}\ \bibnamefont
  {Petta}}, \bibinfo {author} {\bibfnamefont {H.}~\bibnamefont {Lu}},\ and\
  \bibinfo {author} {\bibfnamefont {A.~C.}\ \bibnamefont {Gossard}},\
  }\bibfield  {title} {\bibinfo {title} {{Quantum Coherence in a One-Electron
  Semiconductor Charge Qubit}},\ }\href
  {https://doi.org/10.1103/PhysRevLett.105.246804} {\bibfield  {journal}
  {\bibinfo  {journal} {Phys. Rev. Lett.}\ }\textbf {\bibinfo {volume} {105}},\
  \bibinfo {pages} {246804} (\bibinfo {year} {2010})}\BibitemShut {NoStop}%
\bibitem [{\citenamefont {Ihn}(2009)}]{ihn2009semiconductor}%
  \BibitemOpen
  \bibfield  {author} {\bibinfo {author} {\bibfnamefont {T.}~\bibnamefont
  {Ihn}},\ }\href@noop {} {\emph {\bibinfo {title} {Semiconductor
  Nanostructures: Quantum states and electronic transport}}}\ (\bibinfo
  {publisher} {OUP Oxford},\ \bibinfo {year} {2009})\BibitemShut {NoStop}%
\bibitem [{\citenamefont {M{\ifmmode\ddot{o}\else\"{o}\fi}ller}\ \emph
  {et~al.}(2021)\citenamefont {M{\ifmmode\ddot{o}\else\"{o}\fi}ller},
  \citenamefont {Banszerus}, \citenamefont {Knothe}, \citenamefont {Steiner},
  \citenamefont {Icking}, \citenamefont {Trellenkamp}, \citenamefont {Lentz},
  \citenamefont {Watanabe}, \citenamefont {Taniguchi}, \citenamefont {Glazman},
  \citenamefont {Fal{'}ko}, \citenamefont {Volk},\ and\ \citenamefont
  {Stampfer}}]{moller2021probing}%
  \BibitemOpen
  \bibfield  {author} {\bibinfo {author} {\bibfnamefont {S.}~\bibnamefont
  {M{\ifmmode\ddot{o}\else\"{o}\fi}ller}}, \bibinfo {author} {\bibfnamefont
  {L.}~\bibnamefont {Banszerus}}, \bibinfo {author} {\bibfnamefont
  {A.}~\bibnamefont {Knothe}}, \bibinfo {author} {\bibfnamefont
  {C.}~\bibnamefont {Steiner}}, \bibinfo {author} {\bibfnamefont
  {E.}~\bibnamefont {Icking}}, \bibinfo {author} {\bibfnamefont
  {S.}~\bibnamefont {Trellenkamp}}, \bibinfo {author} {\bibfnamefont
  {F.}~\bibnamefont {Lentz}}, \bibinfo {author} {\bibfnamefont
  {K.}~\bibnamefont {Watanabe}}, \bibinfo {author} {\bibfnamefont
  {T.}~\bibnamefont {Taniguchi}}, \bibinfo {author} {\bibfnamefont {L.~I.}\
  \bibnamefont {Glazman}}, \bibinfo {author} {\bibfnamefont {V.~I.}\
  \bibnamefont {Fal{'}ko}}, \bibinfo {author} {\bibfnamefont {C.}~\bibnamefont
  {Volk}},\ and\ \bibinfo {author} {\bibfnamefont {C.}~\bibnamefont
  {Stampfer}},\ }\bibfield  {title} {\bibinfo {title} {{Probing Two-Electron
  Multiplets in Bilayer Graphene Quantum Dots}},\ }\href
  {https://doi.org/10.1103/PhysRevLett.127.256802} {\bibfield  {journal}
  {\bibinfo  {journal} {Phys. Rev. Lett.}\ }\textbf {\bibinfo {volume} {127}},\
  \bibinfo {pages} {256802} (\bibinfo {year} {2021})}\BibitemShut {NoStop}%
\bibitem [{\citenamefont {Eich}\ \emph {et~al.}(2018)\citenamefont {Eich},
  \citenamefont {Herman}, \citenamefont {Pisoni}, \citenamefont {Overweg},
  \citenamefont {Kurzmann}, \citenamefont {Lee}, \citenamefont {Rickhaus},
  \citenamefont {Watanabe}, \citenamefont {Taniguchi}, \citenamefont {Sigrist},
  \citenamefont {Ihn},\ and\ \citenamefont {Ensslin}}]{eich2018spin}%
  \BibitemOpen
  \bibfield  {author} {\bibinfo {author} {\bibfnamefont {M.}~\bibnamefont
  {Eich}}, \bibinfo {author} {\bibfnamefont {F.}~\bibnamefont {Herman}},
  \bibinfo {author} {\bibfnamefont {R.}~\bibnamefont {Pisoni}}, \bibinfo
  {author} {\bibfnamefont {H.}~\bibnamefont {Overweg}}, \bibinfo {author}
  {\bibfnamefont {A.}~\bibnamefont {Kurzmann}}, \bibinfo {author}
  {\bibfnamefont {Y.}~\bibnamefont {Lee}}, \bibinfo {author} {\bibfnamefont
  {P.}~\bibnamefont {Rickhaus}}, \bibinfo {author} {\bibfnamefont
  {K.}~\bibnamefont {Watanabe}}, \bibinfo {author} {\bibfnamefont
  {T.}~\bibnamefont {Taniguchi}}, \bibinfo {author} {\bibfnamefont
  {M.}~\bibnamefont {Sigrist}}, \bibinfo {author} {\bibfnamefont
  {T.}~\bibnamefont {Ihn}},\ and\ \bibinfo {author} {\bibfnamefont
  {K.}~\bibnamefont {Ensslin}},\ }\bibfield  {title} {\bibinfo {title} {{Spin
  and Valley States in Gate-Defined Bilayer Graphene Quantum Dots}},\ }\href
  {https://doi.org/10.1103/PhysRevX.8.031023} {\bibfield  {journal} {\bibinfo
  {journal} {Phys. Rev. X}\ }\textbf {\bibinfo {volume} {8}},\ \bibinfo {pages}
  {031023} (\bibinfo {year} {2018})}\BibitemShut {NoStop}%
\bibitem [{\citenamefont {Rudner}\ \emph {et~al.}(2008)\citenamefont {Rudner},
  \citenamefont {Shytov}, \citenamefont {Levitov}, \citenamefont {Berns},
  \citenamefont {Oliver}, \citenamefont {Valenzuela},\ and\ \citenamefont
  {Orlando}}]{rudner2008quantum}%
  \BibitemOpen
  \bibfield  {author} {\bibinfo {author} {\bibfnamefont {M.~S.}\ \bibnamefont
  {Rudner}}, \bibinfo {author} {\bibfnamefont {A.~V.}\ \bibnamefont {Shytov}},
  \bibinfo {author} {\bibfnamefont {L.~S.}\ \bibnamefont {Levitov}}, \bibinfo
  {author} {\bibfnamefont {D.~M.}\ \bibnamefont {Berns}}, \bibinfo {author}
  {\bibfnamefont {W.~D.}\ \bibnamefont {Oliver}}, \bibinfo {author}
  {\bibfnamefont {S.~O.}\ \bibnamefont {Valenzuela}},\ and\ \bibinfo {author}
  {\bibfnamefont {T.~P.}\ \bibnamefont {Orlando}},\ }\bibfield  {title}
  {\bibinfo {title} {{Quantum Phase Tomography of a Strongly Driven Qubit}},\
  }\href {https://doi.org/10.1103/PhysRevLett.101.190502} {\bibfield  {journal}
  {\bibinfo  {journal} {Phys. Rev. Lett.}\ }\textbf {\bibinfo {volume} {101}},\
  \bibinfo {pages} {190502} (\bibinfo {year} {2008})}\BibitemShut {NoStop}%
\end{thebibliography}%




\end{document}
