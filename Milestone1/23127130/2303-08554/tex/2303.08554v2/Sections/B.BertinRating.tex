\begin{table}[th]
    \centering
    \caption{Bertin's rating of six visual channels. The cells with ``(can be)'' were rated as ``no'' by Bertin initially. A fair amount of evidence found in the literature has indicated that they ``can be'' meeting the requirements of the criteria concerned.}
    \begin{tabular}{@{}rcccc@{}}
        \textbf{Channel$\;$} & \textbf{Associative} & \textbf{Selective}
        & \textbf{Ordered} & \textbf{Quantitative}\\
        \hline
        Size        & no  & limited & yes & yes \\
        Brightness  & no  & yes & yes & yes \\
        Texture     & yes & yes & yes & no \\
        Color       & yes & yes & (can be) & (can be) \\
        Shape       & yes & (can be) & no & no \\
        Orientation & yes & yes & (can be) & (can be)\\
        \hline
    \end{tabular}
    \label{tab:BertinRating}
\end{table}

\begin{table}[th]
    \centering
    \caption{Tentative ratings for the channels listed in Chen et al. \cite{Chen:2014:book}. We used the term ``maybe'' to indicate our own uncertainty.}
    \begin{tabular}{@{}r@{\hspace{2mm}}c@{\hspace{2mm}}c@{\hspace{2mm}}c@{\hspace{2mm}}c@{}}
        \textbf{Channel$\quad$} & \textbf{Associative} & \textbf{Selective}
        & \textbf{Ordered} & \textbf{Quantitative}\\
        \hline
        \multicolumn{5}{c}{\emph{Geometric Channels:}}\\
        Size        & no  & limited & yes & yes \\
        Orientation & yes & yes & (can be) & (can be)\\
        Shape       & yes & (can be) & no & no \\
        Curvature   & no  & limited & yes & maybe \\
        Smoothness  & limited & limited & yes & maybe \\
        \hline
        \multicolumn{5}{c}{\emph{Optical Channels:}}\\
        Brightness  & no  & yes & yes & yes \\
        Color       & yes & yes & (can be) & (can be) \\
        Opacity     & no  & limited & yes & maybe \\
        Texture     & yes & yes & yes & no \\
        Shading     & limited & limited & yes & limited \\
        Halos       & limited & limited & yes & yes \\
        Shadow      & yes & yes & maybe & maybe \\
        Photo effects & limited & limited & maybe & maybe \\
        Implicit motion & limited & limited & maybe & maybe \\
        Explicit motion & yes & yes & yes & yes \\
        \hline
        \multicolumn{5}{c}{\emph{Relational Channels:}}\\
        Connection/edge & limited & limited & no & no \\
        Node & limited & limited & no & no \\
        Inside/outside & limited & limited & no & no \\
        Enclosure/Boundary & limited & limited & no & no \\
        Distance & no & limited & yes & yes \\
        Closure/opening & limited & limited & no & no \\
        Connectivity & yes & yes & maybe & maybe \\
        Partition & yes & yes & maybe & maybe \\
        Intersection/overlap & limited & limited & maybe & maybe \\
        Depth ordering & limited & limited & yes & maybe \\
        Hierarchy/level & limited & limited & yes & maybe \\
        Density/distribution & yes & yes & yes & no \\
        Convexity & limited & limited & no & no \\
        Continuity & limited & limited & maybe & no \\
        Genera & limited & limited & maybe & no \\
        Similarity & limited & limited & maybe & no \\
        Deformation & limited & limited & no & no \\
        \hline
        \multicolumn{5}{c}{\emph{Semantic Channels:}}\\
        Number & yes & yes & yes & yes \\
        Text & yes & yes & maybe & no \\
        Symbol/ideogram & yes & yes & maybe & no \\
        Sign/pictogram & yes & yes & maybe & no \\
        Isotype & yes & yes & maybe & no \\
        \hline
    \end{tabular}
    \label{tab:ChannelRating}
\end{table}

\section{Typedness and Bertin's Criteria}
\label{apx:Bertin}
%
Bertin's original criteria can be considered rateable using a ternary score (yes, limited, or no) based on Bertin's description as shown in Table \ref{tab:BertinRating}. Some cells in the table were rated as ``no'' by Bertin \cite{Bertin:2011:book}. However, a fair amount of evidence found in the literature has indicated that they ``can be'' considered to meet the requirements of the criteria concerned. For example, colors are widely used in heatmap visualization, and orientation can be used for encoding angles and data values suitable for a clock metaphor. 

In the context of Section \ref{sec:Typedness}, we can consider a ``yes'' to be appropriate for a kind of perception (KOP), and a ``no'' to be inappropriate. The terms ``limited'' and ``(can be)'' indicate that the assessment needs to take application-specific requirements into account.

There are many visual channels that are yet to be assessed using Bertin's four criteria. For example, Chen et al. listed more than 30 visual channels \cite{Chen:2014:book}. Assessing all these channels is beyond the scope of this paper. In Table \ref{tab:ChannelRating}, we provide tentative ratings for these channels to aid the users of our MCDA scheme.


