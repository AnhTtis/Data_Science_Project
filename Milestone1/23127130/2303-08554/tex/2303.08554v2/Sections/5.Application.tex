\begin{table*}[t]
\centering
\caption{A summary of the assessments of five optional glyph designs using the MCDA-aided scheme.
$J_1$ uses six lines to represent the min-max ranges of the six angles.
$J_2$ combines each pair of lines for the two related angles into a single line. $J_3$ is similar to $J_2$ but uses 2D arcs instead of lines.
$J_4$ is similar to $J_3$ but uses 3D arcs instead of 2D arcs. 
$J_5$ is organizes the six arcs around a circle, and indicates three pairs using colored wedges. \reviseTVCG{The scores shown are the mean values of two assessors' scores.}}
\label{tab:CaseStudy2}
%
% \vspace{-1mm}
\begin{tabular}{@{\hspace{23mm}}c@{\hspace{8mm}}c@{\hspace{15mm}}c@{\hspace{13mm}}c@{\hspace{12mm}}c@{\hspace{12mm}}c@{}}
    &
    \includegraphics[height=10mm]{Figures/DesignJ1} &
    \includegraphics[height=10mm]{Figures/DesignJ2} &
    \includegraphics[height=10mm]{Figures/DesignJ3} &
    \includegraphics[height=10mm]{Figures/DesignJ4} &
    \includegraphics[height=10mm]{Figures/DesignJ5} \\
    \textbf{Design Options:} &
    \textbf{$J_1$: 6-line} &
    \textbf{$J_2$: 3-line} &
    \textbf{$J_3$ 3-arc 2D}&
    \textbf{$J_4$: 3-arc 3D} &
    \textbf{$J_5$: Pie-arc} \\
\end{tabular}
\begin{tabular}{@{}p{46mm}@{\hspace{4mm}}%
        c@{\hspace{3mm}}c@{\hspace{9mm}}c@{\hspace{3mm}}c@{\hspace{9mm}}%
        c@{\hspace{3mm}}c@{\hspace{9mm}}c@{\hspace{3mm}}c@{\hspace{9mm}}c@{\hspace{3mm}}c@{}}
    \textbf{Criterion} &
        \textbf{weight} & \textbf{score} & \textbf{weight} & \textbf{score} &
        \textbf{weight} & \textbf{score} & \textbf{weight} & \textbf{score} &
        \textbf{weight} & \textbf{score}\\
    \hline
    Typedness                  &   1 & 4.84 &   1 & 4.84 &   1 & 4.84 &   1 & 4.84 &   1 & 4.92\\
    Discernability             &   1 & 4.60 &   1 & 3.14 &   1 & 3.46 &   1 & 3.46 &   1 & 3.68\\
    Intuitiveness              &   1 & 3.43 &   1 & 3.03 &   1 & 4.81 &   1 & 4.97 &   1 & 2.27\\
    Invariance: Geometry       & 0.5 &  4.5 & 0.5 &  4.5 & 0.5 &    5 & 0.5 &    4 & 0.5 &    5\\
    Invariance: Colorimetry    & 0.5 &    5 & 0.5 &    5 & 0.5 &    5 & 0.5 &    5 & 0.5 &    5\\
    Composition: Separability  & 0.5 &    5 & 0.5 &    3 & 0.5 &    3 & 0.5 &    3 & 0.5 &    3\\
    Composition: Comparability & 0.5 &    5 & 0.5 &    4 & 0.5 &    5 & 0.5 &    3 & 0.5 &    3\\
    Attention: Importance      & 0.5 &    5 & 0.5 &    5 & 0.5 &    5 & 0.5 &    5 & 0.5 &    5\\
    Attention: Balance         & 0.5 &    5 & 0.5 &    5 & 0.5 &    5 & 0.5 &    5 & 0.5 &    5\\
    Searchability              & 0.5 &  4.5 & 0.5 &  4.5 & 0.5 &  4.5 & 0.5 &    5 & 0.5 &    5\\
    Learnability               & 0.5 &  3.5 & 0.5 &  3.5 & 0.5 &    4 & 0.5 &    5 & 0.5 &    2\\
    Memorability               & 0.5 &    4 & 0.5 &  3.5 & 0.5 &    4 & 0.5 &    5 & 0.5 &    3\\
    \hline
    \textbf{Total Weight \& Weighted Average}
                               & 7.5 & 4.48 & 7.5 & 4.00 & 7.5 & 4.45 & 7.5 & 4.44 & 7.5 & 3.85\\ 
\end{tabular}
% \vspace{-4mm}
\end{table*}

% ====================
\section{MCDA-aided Evaluation: \reviseTVCG{Scoring Examples}}
\label{sec:Evaluation}
%
We applied the MCDA scheme described in Section \ref{sec:Scheme} to a number of visual designs in the literature and their ``parodies'' (alternative designs) and to some new designs for a biomechanical application. These \reviseTVCG{scoring examples} allowed us to test the scheme, identify ambiguous definitions, unbalanced categorization, and inappropriate thresholds, facilitating the improvement of the scheme. We have included three files (two design documents and one spreadsheet workbook) in the supplementary materials.

\vspace{1mm}
\noindent\textbf{Existing Designs and Parodies}.
Table \ref{tab:CaseStudy1} shows the summary of five \reviseTVCG{examples}. 
Among these, \textbf{A} is the original design by Maguire et al. \cite{Maguire:2012:TVCG}, who also reported a number of alternative encoding methods for individual data variables. It is not difficult for us to configure ``parody'' designs based on these alternative encoding methods. Design \textbf{B} is one of such parody designs. In \textbf{B}, variable S6 is encoded using three colors instead of three shapes (e.g., dark gray instead white circle in Table \ref{tab:CaseStudy1}). Variable S2 is encoded using five colors for an outline instead of five metaphoric shapes (e.g., cyan square instead of an icon for material combination). S5 is encoded using seven basic shapes instead of seven countable metaphoric shapes as shown in the first row of Fig.~\ref{fig:Intuitiveness}.     
From Table \ref{tab:CaseStudy1}, we can observe that the parody design has over-used colors and basic shapes, having a negative impact on several criteria, especially in terms of separability, attention balance, searchability, learnability, and memorability.

\textbf{C} is the original design by Legg et al. for supporting real-time event analysis during a rugby match. Design \textbf{D} is a parody design of \textbf{C} with two modifications: (i) replacing the silhouette pictogram at the center of the glyph with abstract shapes, and (ii) swapping the locations of the outcome circle (orange for unsuccessful) with the territory box (location A) in the glyphs representing Design \textbf{C} and Design \textbf{D}. As the data variable for event types has 16 key values, in comparison with pictograms in \textbf{C}, the abstract shapes in \textbf{D} reduces intuitiveness, learnability, and memorability. Meanwhile, the swapping changes the order between variables outcome and territory. The outcome circle is not as attentive as in \textbf{C} and becomes less separable from the abstract shapes in the center.

\textbf{E} is a design by Chung et al. \cite{chung2015glyph}, where a similar set of pictograms was used for event types. Unlike \textbf{C} that was designed for rugby coaches and sports analysts, \textbf{E} was designed for analysts only, and it contains several numerical variables resulting from video analysis, such as gain, tortuosity, and net lateral movement. The discernability of these variables are affected by size and color degeneration. Although visual designs for these variables were introduced for this application, the analysts who had the technical background could learn and memorize these with a bit of extra effort.

We have included the documentation about the parody designs \textbf{B} and \textbf{D} in the supplementary materials.
\reviseTVCG{There were many scoring exercises that authors have conducted throughout this work. This exercise was among several exercises conducted after the MCDA scheme in Section \ref{sec:Scheme} was more or less finalized. These exercises were designed to debug the text of various definitions in the scheme, and to synchronize the interpretation among the authors. In this particular exercise, two authors first scored the five designs independently. Three authors then met and discussed the two sets of scores. They reached an agreement on every score. The final scores are presented in Table \ref{tab:CaseStudy1}. The detailed Type A scores are provided in the supplementary material. We recommend that ``discussion meeting'' is used as the default approach when involving multiple assessors, especially when some assessors are less familiar with the MCDA scheme.}

% ----------

\vspace{1mm}
\noindent\textbf{New Designs for a Biomechanical Application}. This \reviseTVCG{example} is a design exercise for studying the feasibility of using glyph-based visualization in a medical application. This design exercise considered a subset of data variables, i.e., those related to rotations.    
In musculoskeletal biomechanics and related fields, such as physiotherapy, sports science, and orthopaedics, it is common to measure the angular motion of the large joints of the body (e.g shoulder or knee). The rotations can be quantified in three dimensions using Euler angles: typically called \emph{flexion-extension}, \emph{abduction-adduction}, and \emph{internal-external} rotation. Each of these angles has both maximum and minimum possible values and a ``normal'' range in each rotation direction, giving 24 \emph{populational variables} in total for each joint. When an individual is examined, one or more measures are usually obtained for each angle, giving at least 6 \emph{individual variables} for each joint. When the values of these variables are recorded for an individual in a data table, there are additional labels indicating a few categorical values, such as joint type (neck, hip, knee, ankle, etc.), body side (left, right, and center), and rotation axes and directions. To the best of our knowledge, the total number of variables is much higher than those in the existing glyph designs in the literature (e.g., $\sim$20 variables in Duffy et al. \cite{Duffy:2015:TVCG}).

\revise{While data tables are customarily used in musculoskeletal biomechanics and related fields, using glyphs to encode these data tables is helpful when domain experts need to observe and analyze a number of data tables, e.g., monitoring treatment progress, comparing different clusters, and identifying commonality and anomalies in a cohort.}

The \reviseTVCG{example} is part of a project, which involves an expert of biomechanics, an expert of visualization, and a postgraduate student working on the intersection of the two subjects. During a period of a few years, numerous glyph designs were created for different sets of biomechanical variables. This particular \reviseTVCG{example} focuses on the rotational measures. Several dozens of designs were created for encoding the aforementioned individual and populational variables. These designs were evaluated qualitatively in several meetings, and five designs were selected to be examined in detail using the MCDA scheme described in Section \ref{sec:Scheme}. Table \ref{tab:CaseStudy2} shows the summary of five designs in this \reviseTVCG{example}. The scores for ``Typedness'', ``Discernability'', and ``Intuitiveness'' are Type D scores, each was aggregated from 37 Type A scores.

The scores were first produced by the two VIS researchers independently, and then discussed in several project meetings, where the biomechanics expert offered advisory comments on the scores. \reviseTVCG{The authors noticed that their initial scores were largely in agreement and considered that using mean values would be a valid approach.} A few original scores were adjusted independently by the assessors concerned, and the mean values were calculated as presented in Table \ref{tab:CaseStudy2}. \reviseTVCG{For most criteria, they arrived the same score for each design. There was no difference of more than 1 level. In other words, every integer score resulted from two independent scores that were the same. Each 4.5 score resulted from two close scores 4 and 5, while each 3.5 score resulted from two close scores 3 and 4. The two sets of independent scores (including the detailed Type A scores) are provided in the supplementary material.}

\reviseTVCG{This scoring exercise was conducted many months after the exercise reported in Table \ref{tab:CaseStudy1}. By then, the wordings of the definitions in Section \ref{sec:Scheme} had gone through several iterations of refinement and the authors' interpretation of the text had become fairly consistent. ``Averaging'' is thus the second approach when involving multiple experienced assessors.}

\begin{figure}
    \centering
    \begin{tabular}{@{}c@{\hspace{2mm}}c@{\hspace{2mm}}c@{}}
         \includegraphics[width=26mm]{Figures/DesignJ6} &
         \includegraphics[width=26mm]{Figures/DesignJ7a} &
         \includegraphics[width=26mm]{Figures/DesignJ7b} \\[-4mm]
         \small{(a) $J_6$ 6-arc 2D Design} &
         \small{(b) $J_{6}$ with labels} &
         \small{(c) $J_{6}$ with legend}
    \end{tabular}
    \caption{Two further designs that were created after the assessment of the five designs in Table \ref{tab:CaseStudy2}. (a) Design $J_6$ was based on $J_1$ and $J_3$ and received an overall score 4.73. (b) Each arc is associated an abbreviated label for the corresponding rotation angle. (c) A mini-legend shows six color-coded directions. The overall scores of (b) and (c) are about 4.80.}
    \label{fig:FurtherDesigns}
    % \vspace{-4mm}
\end{figure}

The overall scores ordered the five designs as $J_1 (4.48)$, $J_3 (4.45)$, $J_4 (4.44)$, $J_2 (4.00)$, $J_5 (3.85)$. All team members agreed with the ordering.
While the scores of MCDA assessment should not be treated as a ground truth, the scores can help identify the shortcomings in the existing designs, including the best-ranked design.
\reviseTVCG{For example, although Designs $J_1$, $J_3$, and $J_4$ received similar high overall scores, each has its own strengthens and weaknesses. $J_1$ is strong on its ``Discernability'' but weak on ``Intuitiveness''. $J_4$ is noticeably strong on ``Intuitiveness'', ``Learnability'', and ``Memorability'' because the 3D layout depicts the physical meanings of the three rotation angles, but is noticeably weak on ``Separability'' and ``Comparability'' also because of the 3D layout.}

The better ``Discernability'' of $J_1$ directed us to a common shortcoming of the 3-line and 3-arc designs ($J_2, J_3$), where the two related angles may have overlapped ranges for some joints. When overlapping occurs, the depiction of some values can be confusing. We therefore introduced a 6-arc 2D design $J_6$ as shown in Fig. \ref{fig:FurtherDesigns}(a). It was based on $J_3$ by incorporating the 6-line idea in $J_1$ and bringing together the merits of $J_1$ in terms of ``Discernability'' and ``Separability'' and that of $J_3$ in terms of ``Intuitiveness''. $J_6$ received an overall score 4.73, indicating an improvement over $J_1$ (4.48) and $J_3$ (4.45).  

\reviseTVCG{The strength of Design $J_4$ in ``Intuitiveness'',  ``Learnability'' and ``Memorability'' directed us to consider labeling as an alternative to 3D layout.}
We considered several ways for labeling the six arcs, including using text labels and a mini-legend as shown in Figs. \ref{fig:FurtherDesigns}(b,c) respectively. Our MCDA scores indicated that both designs in Figs. \ref{fig:FurtherDesigns}(b,c) received approximately the same overall score $\approx 4.80$. Our analysis showed that text labels are easier to learn and demand less memorization than a mini-legend, resulting in a slightly higher score in terms of ``Learnability'' and ``Memorability'' than the latter. However, labels are affected by geometric scaling more than the arcs, arrows, and normal bands, resulting in a lower score in terms of ``invariance: Geometry''. In different biomechanical applications, users have different levels of expertise about the motion of joints and may use glyph-based visualization at different levels of frequency. We decide to keep all three design options in Fig. \ref{fig:FurtherDesigns}, and to finalize the designs for individual applications by assessing the requirements for ``Learnability'' and ``Memorability'' based on users' expertise and frequency of user tasks.


