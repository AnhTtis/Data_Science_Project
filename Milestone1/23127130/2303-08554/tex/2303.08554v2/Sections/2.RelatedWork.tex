\section{Related Work}
\label{sec:RelatedWork}
%
Glyph visualization has been applied in many fields, such as
biology and medicine \cite{Maguire:2012:TVCG,somarakis2019imacyte,meuschke2017glyph,raidou2018bladder,lichtenberg2017concentric,oeltze2008glyph,meyer2008glyph},
%kalamaras2022graph,khawatmi2021shapography,muller2014analysis,
meteorology and environmental studies \cite{martin2008results,pilar2013representing,pfeiffer2021glyph,drocourt2011temporal,sanyal2010noodles},
human behavioral analysis \cite{el2016contovi,kovacevic2020glyph},
web and database searching \cite{chau2011visualizing,siva2014evaluation},
sports \cite{legg2016glyph,polk2014tennivis,wang2021tac,wu2021tacticflow,cava2013glyphs,parry2011hierarchical},
music and multimedia \cite{chan2009visualizing,lind2022visualizing,janicke2010soundriver,botchen2008action},
and business and industrial applications \cite{rees2020agentvis,surtola2005effect,suntinger2008event}.
A glyph object conveys a multivariate data record in a concise way, which significantly reduces the perceptual and cognitive load for information comprehension. It has been widely adopted in applications that require a simultaneous view of multiple variables and multiple data records, including
the facilitation of visual search \cite{healey1999large,cai2015applying},
data comparison \cite{verma2004comparative,zhang2015glyph,meuschke2017glyph,koc2022peaglyph},
data ordering \cite{Chung:2016:CGF,miller2019evaluating},
and feature extraction \cite{keck2017towards}.
Glyphs have also been used in combination with spatial or temporal visualization \cite{tominski20053d,ropinski2007surface,drocourt2011temporal,bleisch2017exploring,Legg:2012:CGF}
and other visualization techniques \cite{lichtenberg2017concentric,kammer2020glyphboard,fernstad2020explore,borgo2012empirical}.
%
Algorithms for glyph placement \cite{ward2002taxonomy,lie2009critical,streeb2018design,rees2020agentvis,tong2016glyphlens,mcnabb2019multivariate,hlawitschka2007interactive}
and 3D visualization \cite{lie2009critical,tong2016glyphlens,stevens2016hairy} were developed for presenting glyph objects. Techniques for specialized scenarios or requirements include
temporal summarization \cite{Duffy:2015:TVCG,el2016contovi,gerrits2017glyphs,tominski20053d,botchen2008action},
uncertainty depiction \cite{aigner2005planninglines,sanyal2010noodles,hlawatsch2011flow,wittenbrink1996glyphs},
%ribicic2012sketching,
visualizing data with special structures or inter-relations \cite{rees2020agentvis,dunne2013motif,cayli2013glyphlink,lee2021cluster,soares2020depicting,cao2011dicon,reda2019dynamic},
%,kalamaras2022graph
and depiction of directional and high-dimensional information in vector and tensor fields (e.g., \cite{schultz2010superquadric,gerrits2016glyphs,tong2016crystal,meuschke2017glyph,hergl2019visualization})
%\cite{dovey1995vector,wittenbrink1996glyphs,schultz2010superquadric,tong2016crystal,meuschke2017glyph,zhang2015glyph,gerrits2016glyphs,gerrits2017glyphs,hergl2019visualization,peeters2009fast,peng2011mesh,hashash2003glyph,hlawatsch2011flow}.
Attempts were made to develop tools for glyph generation \cite{ribarsky1994glyphmaker,xia2018dataink,brehmer2021generative,ying2022metaglyph,cunha2018many}.

With a huge design space and the wide variety of applications, it is highly desirable to provide glyph designers with guidance and methods for evaluating different designs.
% The development of a comprehensive list of criteria that allows systematic evaluations of glyph designs could provide designers with a framework to create designs of quality and to compare design options.
Empirical studies have been used to evaluate glyph designs \cite{aigner2005planninglines,surtola2005effect,weigle2005visualizing,chan2009visualizing,chau2011visualizing,siva2014evaluation,dunne2013motif}.
%
While empirical studies are useful for gauging design qualities from user experience and feedback, they incur arduous time and effort and usually cannot be conducted frequently during a design process.   
A survey conducted by Fuchs et al. \cite{fuchs2016systematic} revealed that two task performance measures (accuracy and completion times) were widely adopted for evaluating the effectiveness of glyph designs.


Normative rating is another way to evaluate visualizations, in which design qualities can be individually quantified. User-centered subjective rating is commonly used in empirical studies \cite{lee2003empirical,aigner2005planninglines,surtola2005effect,fuchs2013evaluation}. The rating schemes used in empirical studies normally do not require design expertise. Normative rating led by visualization experts has not been widely reported in the field of visualization. In psychology, McDougall et al. \cite{mcdougall2000exploring} adopted subjective ratings to characterize cognitive features of icon designs and used the ratings to investigate the correlations among different design qualities identified by experts of icon designers. 


Attempts have been made to develop standardized measures for qualities of glyph visualization. Garcia et al. \cite{garcia1994development} proposed a metric to evaluate glyph complexity.
Forsythe et al. \cite{forsythe2003measuring} proposed an automatic complexity measuring algorithm, aiming to remove subjective elements from the evaluation of complexity and to  prevent judgment bias. Up to now, however, the number of standardized measures of visual design qualities remains very limited.


A survey conducted by Borgo et al. \cite{Borgo:2013:STAR} collected a good number of glyph design guidelines and criteria in the literature. They cover different levels of glyph design (e.g., variable encoding, inter-channel interaction, and holistic glyph design) and different aspects of glyph-based visualization (e.g., 2D and 3D designs, interaction, and placement).
At the variable encoding level, Bertin \cite{bertin1983semiology} proposed four basic perceptual criteria for channel encoding. Cleveland and McGill \cite{cleveland1984graphical}, Mackinlay \cite{Mackinlay:1986:TOG}, and Munzner \cite{Munzner:2014:book} recommended the ordering of visual channels. Other considerations include the capacity, orderability, semantic closeness, visual pre-attentiveness, robustness, and normalizability of the channels \cite{lie2009critical,Legg:2012:CGF,ropinski2011survey,ward2008multivariate,chung2015glyph}.
%
At the levels of inter-channel interactions and holistic glyph design, guidelines were proposed for the integration and separability of channels, the balance of attention, searchability, visual hierarchy, and the learnability of glyph designs \cite{karve2007glyph,Maguire:2012:TVCG,lie2009critical,chung2015glyph}. Beyond the design of the glyph object \textit{per se}, advice was provided for other aspects of data mapping and glyph rendering \cite{ropinski2007surface,ward2008multivariate,meyer2008glyph,ropinski2008taxonomy,lie2009critical,ropinski2011survey}.
All these proposed guidelines and evaluation criteria prepared us for developing normative rating methods to be used by visualization experts. Such methods could be employed frequently in a design process, complementing user-centered empirical studies.
