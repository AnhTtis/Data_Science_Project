% ---------------------------------------------------------------------------
% Author guideline and sample document for EG publication using LaTeX2e input
% D.Fellner, v1.15, Dec 14, 2018

\documentclass{egpubl}
\usepackage{eurovis2023}

% --- for  Annual CONFERENCE
% \ConferenceSubmission   % uncomment for Conference submission
% \ConferencePaper        % uncomment for (final) Conference Paper
% \STAR                   % uncomment for STAR contribution
% \Tutorial               % uncomment for Tutorial contribution
% \ShortPresentation      % uncomment for (final) Short Conference Presentation
% \Areas                  % uncomment for Areas contribution
% \MedicalPrize           % uncomment for Medical Prize contribution
% \Education              % uncomment for Education contribution
% \Poster                 % uncomment for Poster contribution
% \DC                     % uncomment for Doctoral Consortium
%
% --- for  CGF Journal
% \JournalSubmission    % uncomment for submission to Computer Graphics Forum
% \JournalPaper         % uncomment for final version of Journal Paper
%
% --- for  CGF Journal: special issue
% \SpecialIssueSubmission    % uncomment for submission to , special issue
\SpecialIssuePaper         % uncomment for final version of Computer Graphics Forum, special issue
%                          % EuroVis, SGP, Rendering, PG
% --- for  EG Workshop Proceedings
% \WsSubmission      % uncomment for submission to EG Workshop
% \WsPaper           % uncomment for final version of EG Workshop contribution
% \WsSubmissionJoint % for joint events, for example ICAT-EGVE
% \WsPaperJoint      % for joint events, for example ICAT-EGVE
% \Expressive        % for SBIM, CAe, NPAR
% \DigitalHeritagePaper
% \PaperL2P          % for events EG only asks for License to Publish

% --- for EuroVis 
% for full papers use \SpecialIssuePaper
% \STAREurovis   % for EuroVis additional material 
% \EuroVisPoster % for EuroVis additional material 
% \EuroVisShort  % for EuroVis additional material

% !! *please* don't change anything above
% !! unless you REALLY know what you are doing
% ------------------------------------------------------------------------
\usepackage[T1]{fontenc}
\usepackage{dfadobe}  

\usepackage{cite}  % comment out for biblatex with backend=biber
% ---------------------------
%\biberVersion
\BibtexOrBiblatex
%\usepackage[backend=biber,bibstyle=EG,citestyle=alphabetic,backref=true]{biblatex} 
%\addbibresource{egbibsample.bib}
% ---------------------------  
\electronicVersion
\PrintedOrElectronic
% for including postscript figures
% mind: package option 'draft' will replace PS figure by a filename within a frame
\ifpdf \usepackage[pdftex]{graphicx} \pdfcompresslevel=9
\else \usepackage[dvips]{graphicx} \fi
\usepackage{egweblnk}
% end of prologue

% Authors' Prologue
\usepackage{amsmath}
\usepackage{xcolor}
\usepackage{xfrac}

\graphicspath{{Figures/}{./}}

\usepackage{color, colortbl}
\definecolor{LightGray}{gray}{0.85}
\definecolor{DarkGray}{gray}{0.65}
\definecolor{Black}{gray}{0}
\usepackage[normalem]{ulem}

% ---------------------------------------------------------------------
% EG author guidelines plus sample file for EG publication using LaTeX2e input
% D.Fellner, v2.03, Dec 14, 2018


\title%[short title]
      {Multi-Criteria Decision Analysis for Aiding Glyph Design}

% for anonymous conference submission please enter your SUBMISSION ID
% instead of the author's name (and leave the affiliation blank) !!
% for final version: please provide your *own* ORCID in the brackets following \orcid; see https://orcid.org/ for more details.
\author[Hsieh et al.]
{\parbox{\textwidth}{\centering%
Hong-Po Hsieh\orcid{0000-0001-5304-8583}\thanks{Email: \url{hong-po.hsieh@eng.ox.ac.uk}},
Amy Zavatsky\orcid{0000-0002-9618-286X},
and Min Chen\orcid{0000-0001-5320-5729}
\\
University of Oxford, UK
}
}
% ------------------------------------------------------------------------

% if the Editors-in-Chief have given you the data, you may uncomment
% the following five lines and insert it here
%
% \volume{36}   % the volume in which the issue will be published;
% \issue{1}     % the issue number of the publication
% \pStartPage{1}      % set starting page


%-------------------------------------------------------------------------
\begin{document}

% uncomment for using teaser
% \teaser{
%  \includegraphics[width=\linewidth]{eg_new}
%  \centering
%   \caption{New EG Logo}
% \label{fig:teaser}
%}

\maketitle
%-------------------------------------------------------------------------
\begin{abstract}
  Glyph-based visualization is one of the main techniques for visualizing complex multivariate data. With small glyphs, data variables are typically encoded with relatively low visual and perceptual precision. Glyph designers have to contemplate the trade-offs in allocating visual channels when there is a large number of data variables. While there are many successful glyph designs in the literature, there is not yet a systematic method for assisting visualization designers to evaluate different design options that feature different types of trade-offs. In this paper, we present an evaluation scheme based on the multi-criteria decision analysis (MCDA) methodology. The scheme provides designers with a structured way to consider their glyph designs from a range of perspectives, while rendering a semi-quantitative template for evaluating different design options. In addition, this work provides guideposts for future empirical research to obtain more quantitative measurements that can be used in MCDA-aided glyph design processes.  

%-------------------------------------------------------------------------
%  ACM CCS 1998
%  (see https://www.acm.org/publications/computing-classification-system/1998)
% \begin{classification} % according to https://www.acm.org/publications/computing-classification-system/1998
% \CCScat{Computer Graphics}{I.3.3}{Picture/Image Generation}{Line and curve generation}
% \end{classification}
%-------------------------------------------------------------------------
%  ACM CCS 2012
%%  (see https://www.acm.org/publications/class-2012)
%The tool at \url{http://dl.acm.org/ccs.cfm} can be used to generate
% CCS codes.
%Example:
% \begin{CCSXML}
% <ccs2012>
% <concept>
% <concept_id>10010147.10010371.10010352.10010381</concept_id>
% <concept_desc>Computing methodologies~Collision detection</concept_desc>
% <concept_significance>300</concept_significance>
% </concept>
% <concept>
% <concept_id>10010583.10010588.10010559</concept_id>
% <concept_desc>Hardware~Sensors and actuators</concept_desc>
% <concept_significance>300</concept_significance>
% </concept>
% <concept>
% <concept_id>10010583.10010584.10010587</concept_id>
% <concept_desc>Hardware~PCB design and layout</concept_desc>
% <concept_significance>100</concept_significance>
% </concept>
% </ccs2012>
% \end{CCSXML}

% \ccsdesc[300]{Computing methodologies~Collision detection}
% \ccsdesc[300]{Hardware~Sensors and actuators}
% \ccsdesc[100]{Hardware~PCB design and layout}

% \printccsdesc   
\end{abstract}  

%-------------------------------------------------------------------------
%% the only exception to this rule is the \firstsection command
% \firstsection{Introduction}

\section{Introduction}
\label{sec:Intro}

Glyph-based visualization is a family of widely used techniques, which are often integrated with other families of visualization techniques, such as small multiples in geo-spatial visualization, multivariate vertices in network visualization, directional and multivariate feature depiction in volume, vector, and tensor field visualization, and dynamic characteristics of objects in event and video visualization. While there have been proposals and discourses on desirable properties of glyph designs in the literature (e.g., \cite{Bertin:2011:book,Maguire:2012:TVCG,Borgo:2013:STAR,chung2015glyph}), there is not yet a coherent methodology that visualization designers can use, consistently and methodically, in evaluating different design options in a process for designing and developing a visualization solution.

In glyph design processes, a designer may face many challenges (e.g., knowledge about the data, users, tasks, cognitive theories and experimental findings related to glyph-based visualization, and so on). This work focuses on one particular challenge, that is, there are many desirable properties of glyph designs, and likely a good design does not necessarily meet all criteria as one might desire, but embodies a relatively optimized set of trade-offs among the visual representations of different variables. In particular, we propose a methodology for evaluating different design options based on multiple-criteria decision analysis (MCDA) \cite{Ishizaka:2013:book,Azzabi:2020:book}, which is an established and widely adopted methodology in management science for evaluating multiple complementary and conflicting criteria explicitly in decision making. Its applications include business, governance, medicine, and engineering. 

We aim to introduce MCDA to glyph design as a systematic and cost-effective methodology and to bring together different desirable properties proposed in the literature into a typology of rateable criteria.
Attention has been paid to (1) providing a good coverage of all proposed criteria for static glyph designs, (2) defining each criterion to facilitate clear interpretation and unambiguous ratings, (3) reducing the overlapping among criteria, (4) enabling distinct considerations of conflicting criteria, and (5) recommending a weighted scoring mechanism that balances between overview vs. detail and precision vs. cost.

\revise{The proposed methodology is built on a comprehensive study of the literature on glyph-based visualization as summarized in Section \ref{sec:RelatedWork}. In particular, the 12 criteria in the proposed MCDA scheme are carefully selected following an in-depth analysis of four existing sets of criteria in the literature \cite{Bertin:2011:book,Maguire:2012:TVCG,Borgo:2013:STAR,chung2015glyph}, which is detailed in Appendix \ref{apx:FrameworkDesign} in the supplementary materials. The analysis enabled us to identify the overlapping or partially overlapping criteria as well as a few missing criteria, while allowing us to gain an appreciation about the scientific reasons behind their categorical arrangement of different criteria. The main contributions of this work include:
%
\begin{itemize}
    %\vspace{-1mm} 
    \item The introduction of the MCDA methodology in glyph design processes for evaluating glyph designs systematically;
    % \vspace{-1mm} 
    \item The integration of four existing sets of criteria for evaluating glyph designs, and the identification and addition of a few criteria absent in the existing sets;%
    \vspace{-1mm}
    \item The definition of five scales for rating each of the 12 criteria, featuring quantitative and meticulous specifications;%
    % \vspace{-1mm}
    \item The testing and demonstration of the proposed MCDA scheme through \reviseTVCG{scoring examples} where the scheme was applied to several existing glyph designs in the literature and their parody versions as well as several new designs in a biomechanical application.% 
\end{itemize}}

The proposed methodology represents a major step forward from the four existing sets of criteria. The definitions in the scheme are expected to be improved through the experience gained in its uses in practice as well as new findings in theoretical and empirical research.
Furthermore, the proposed methodology is not a replacement for user-centered requirements analysis and evaluation, but it can complement, strengthen, and reduce the frequencies of user-centered studies in individual design processes.
We will discuss these further in Section \ref{sec:Conclusions}.
% In particular, we anticipate that the rating schemes for a number of criteria will be enriched when more measurements and models in visualization psychology are becoming available.   
%




\section{Related Work}
\label{sec:RalatedWork}
%
Glyph visualization has been applied in many fields, such as
biology, biomedical and medical research \cite{Maguire:2012:TVCG,somarakis2019imacyte,muller2014analysis,meuschke2017glyph,raidou2018bladder,lichtenberg2017concentric,kalamaras2022graph,khawatmi2021shapography,oeltze2008glyph,meyer2008glyph},
meteorology and environmental studies \cite{martin2008results,pilar2013representing,pfeiffer2021glyph,drocourt2011temporal,sanyal2010noodles},
human behavioral analysis \cite{el2016contovi,kovacevic2020glyph},
web and database searching \cite{chau2011visualizing,siva2014evaluation},
sports \cite{legg2016glyph,polk2014tennivis,wang2021tac,wu2021tacticflow,cava2013glyphs,parry2011hierarchical},
music and multimedia \cite{chan2009visualizing,lind2022visualizing,janicke2010soundriver,botchen2008action},
and business and industrial applications \cite{rees2020agentvis,surtola2005effect,suntinger2008event}.
A glyph object conveys multi-dimensional data in a concise way, which significantly reduces the perceptual and cognitive load for information comprehension. It has been widely adopted in applications that require a simultaneous view of multiple variables, including
the facilitation of visual search \cite{healey1999large,cai2015applying},
data comparison \cite{verma2004comparative,zhang2015glyph,meuschke2017glyph,koc2022peaglyph},
data ordering \cite{Chung:2016:CGF,miller2019evaluating},
and feature extraction \cite{keck2017towards}.
Glyphs have also been used in combination with spatial or temporal displays \cite{tominski20053d,ropinski2007surface,drocourt2011temporal,bleisch2017exploring,Legg:2012:CGF}
and other visualization methods \cite{lichtenberg2017concentric,kammer2020glyphboard,fernstad2020explore,borgo2012empirical}.
%
Algorithms for glyph placement \cite{ward2002taxonomy,lie2009critical,streeb2018design,rees2020agentvis,tong2016glyphlens,mcnabb2019multivariate,hlawitschka2007interactive}
and three-dimensional display \cite{lie2009critical,tong2016glyphlens,stevens2016hairy} were created to ensure effective presentation of glyph objects. Some studies developed techniques for specialized scenarios or requirements, such as
temporal summarization \cite{Duffy:2015:TVCG,el2016contovi,gerrits2017glyphs,tominski20053d,botchen2008action},
uncertainty depiction \cite{aigner2005planninglines,sanyal2010noodles,ribicic2012sketching,hlawatsch2011flow,wittenbrink1996glyphs},
visualizing data with special structure or inter-relations \cite{rees2020agentvis,dunne2013motif,cayli2013glyphlink,lee2021cluster,soares2020depicting,cao2011dicon,reda2019dynamic,kalamaras2022graph},
and display of trends and gradients of vectors and tensor fields \cite{dovey1995vector,wittenbrink1996glyphs,schultz2010superquadric,tong2016crystal,meuschke2017glyph,zhang2015glyph,gerrits2016glyphs,gerrits2017glyphs,hergl2019visualization,peeters2009fast,peng2011mesh,hashash2003glyph,hlawatsch2011flow}.
The recent appearance of accessible tools for glyph generation \cite{ribarsky1994glyphmaker,xia2018dataink,brehmer2021generative,ying2022metaglyph,cunha2018many} should lead to a broader application of this visualization technique.

With the seemingly endless design space and the wide variety of applications, guidance is required for glyph designers to evaluate and make selections among different designs. The development of a comprehensive list of criteria that allows systematic evaluations of glyph designs could provide designers with a framework to create designs of quality and to compare design options. The most common and straightforward evaluation is done via user studies \cite{aigner2005planninglines,surtola2005effect,weigle2005visualizing,chan2009visualizing,chau2011visualizing,siva2014evaluation,dunne2013motif}. A systematic review conducted by Fuchs et al.\cite{fuchs2016systematic} revealed that users' accuracy scores and task completion times are the most widely adopted measures to evaluate the effectiveness of visualisations. These measurements provide reliable statistical evidence for the evaluation; however, there are limitations. Firstly, it takes significant time and effort to design and carry out studies of users. This is especially true when multiple design options and objectives exist. Secondly, the performance outcomes are often the collective effect of various design factors, and they provide few indications of specific changes which could be made to improve the design. %\sout{The effects of individual factors can be finely evaluated by setting up experiments for each design factor while others are well controlled, which, however, further increases the costs of the evaluations.}

Normative rating is another way to evaluate visualizations, in which design qualities can be individually quantified. Subjective rating is commonly used in user studies to gauge design qualities from user experiences \cite{lee2003empirical,aigner2005planninglines,surtola2005effect,fuchs2013evaluation}. McDougall et al. \cite{mcdougall2000exploring} adopted subjective ratings to characterize cognitive features of icon designs and used the ratings in further analyses to investigate the correlations between the qualities. However, in most cases, the subjective rating is performed without a clear definition of each rating level. The only instruction given was how the two ends of the scale map to the extremities of the target quality, e.g. 1=very unfamiliar; 5=very familiar. The boundaries between the rating levels remained flexible to the raters. This casts doubts about the consistency of the scoring standards. It has also been shown that the users' preferences are not always consistent with the results from the statistical evidence \cite{fuchs2016systematic}.

Attempts have been made to develop standardized measures for qualities of glyph visualization. Garcia et al. \cite{garcia1994development} proposed a metric to evaluate glyph complexity.
Forsythe et al. \cite{forsythe2003measuring} proposed an automatic complexity measuring algorithm, aiming to remove subjective elements from the evaluation of complexity and to  prevent judgment bias. Up to now, however, the number of standardized measures of visual design qualities remains very limited.
%The subjective rating is still considered the method with more comprehensive coverage in quantifying visual design features \cite{ng2008visual}. 
%To extend the coverage, there is a need to identify more qualities of visual designs that the users may consider and provide clear criteria for the rating.

Fortunately, there have been abundant glyph design guidelines proposed in the literature. The guidelines concern various levels of glyph design: variable encoding, inter-channel interaction, and holistic glyph design. At the variable encoding level, Bertin \cite{bertin1983semiology} proposed basic semantic criteria for determining the suitability of channel encoding. Cleveland and McGill \cite{cleveland1984graphical} identified the accuracy of human perception in various visual variables and gave recommendations on the choices of visual channels for different tasks and purposes of visualization. Other considerations include the capacity, orderability, semantic closeness, visual pre-attentiveness, robustness, and normalizability of the channels \cite{lie2009critical,Legg:2012:CGF,ropinski2011survey,ward2008multivariate,chung2015glyph,yousef2001assessment}. At the levels of inter-channel interactions and holistic glyph design, guidelines were proposed for the integration and separability of channels, the balance of attention, searchability, visual hierarchy, and the learnability of glyph designs \cite{karve2007glyph,Maguire:2012:TVCG,lie2009critical,chung2015glyph}. Beyond the design for the glyph object \textit{per se}, advice was provided for the stages of data-mapping and glyph rendering \cite{ropinski2007surface,ward2008multivariate,meyer2008glyph,ropinski2008taxonomy,lie2009critical,ropinski2011survey}. A survey conducted by Borgo et al. \cite{Borgo:2013:STAR} collected glyph design guidelines and criteria from a wide range of previous studies. Consulting the three-stage glyph design framework by Lie et al. \cite{lie2009critical} and the perceptual-based glyph taxonomy by Ropinski et al. \cite{ropinski2008taxonomy,ropinski2011survey}, a set of fourteen design guidelines that aimed to provide comprehensive coverage of glyph-based visualization was proposed.

%It is manifested that although design criteria taken into account vary amongst studies, consensus on some universal criteria started to form. 
More recent studies\cite{Maguire:2012:TVCG,chung2015glyph,Borgo:2013:STAR} have integrated guidelines from various previous studies, and consensus on some universal design criteria has started to form. 
%As the outline of universal criteria becomes clearer and the coverage becomes comprehensive, 
With further development, a normative set of evaluation criteria might be expected to emerge through a process of collecting and comparing existing guidelines and selecting, modifying, or creating definitions of criteria to make them generally applicable to a vast range of glyph applications and data types. To make the criteria set succinct and equally weighted amongst different design aspects, higher orthogonality must be considered. Ideally, to make the evaluations consistent amongst raters, standards should be defined for each and every level of the evaluation criteria.



% \vfill\eject
% -\\
% \newpage

% ====================
\section{Overview, Terminology, and Design Considerations}
\label{sec:Overview}
%
In this work, we adopt the narrow definition of \emph{glyph} given by Borgo et al. \cite{Borgo:2013:STAR}, i.e., ``a glyph is a small independent visual object that depicts
attributes of a data record; glyphs are discretely placed in a display space; and
glyphs are a type of visual sign but differ from other types of signs such as icons, indices and symbols.''
We focus only on 2D glyphs.
Consider a multivariate data record for storing the values of $m$ \emph{variables}, $D = \{d_1, d_2, \ldots, d_m\}$, and a glyph for encoding such a record using $n$ \emph{visual channels} $\Gamma = \{\gamma_1, \gamma_2, \ldots, \gamma_n\}$. Since each data variable can be encoded using multiple visual channels, we have $n \geq m$.
% Note that in the literature, a visual channel was sometimes referred to as a ``visual variable'', ``retinal variable'', or ``graphical code''. Here we use the term ``visual channel'' to make it more distinguishable from the term ``data variable''.

Multiple-criteria decision analysis (MCDA) \cite{Ishizaka:2013:book,Azzabi:2020:book} can be viewed as a tree-based scoring system, where the score of each node is a weighted average of the scores at its child-nodes. Here an unweighted average is considered as a special case. There are a few high-level design considerations, e.g., do we decompose the holistic assessment into the assessments of data variables or visual channels and how detailed should we carry out the assessment?     

\noindent\textbf{Design Consideration 1. Data Variables vs Visual Channels.} 
%
As mentioned in Section \ref{sec:RalatedWork}, a number of criteria for glyph design have been proposed in the literature \cite{Bertin:2011:book,Maguire:2012:TVCG,Borgo:2013:STAR,Chung:2016:CGF}. Some criteria are defined for accessing visual channels (e.g., channel capacity \cite{Chung:2016:CGF}, while others are applicable to data variable (e.g., metaphoric representation \cite{Maguire:2012:TVCG}).
When one assumes a 1-to-1 mapping between a data variable and a visual channel, evaluating a data variable implicitly implies the evaluation of the corresponding visual channel, and vice versa. However, when a data variable is encoded using multiple visual channels, assessing individual channels independently may not inform their combined effect on perception and interpretation of the data variable concerned.
Hence the propose the scheme places an emphasis on data variables. 

\noindent\textbf{Design Consideration 2. Hierarchy of MCDA Evaluation.}
%
There are potentially more, if such criteria are combined or finely decomposed in evaluation.
Some criteria may be applicable to each data variable or visual channel (e.g., Bertin's criteria \cite{Bertin:2011:book}), while others may involve assessing a group of data variables (e.g., separability \cite{Chung:2016:CGF}), or a whole glyph (e.g., attention balance \cite{Chung:2016:CGF}). There are four types of basic assessment modes:
%
\begin{description}
    \item[Type A.] A criterion is for evaluating the visual encoding of a data variable $v_i$ independently. Given $m$ data variables, there are $m$ assessments for each Type A criterion.%
    \item[Type B.] A criterion is for evaluating how the encoding of a data variable $d_i$ is affected by the encoding of another $d_j (j \neq i)$. Given $m$ variables, there are potentially $m \times (m-1)$ assessments as the impact of $d_i$ on $d_j$ may not be the same as the impact of $d_j$ on $d_i$.%
    \item[Type C.] A criterion is for evaluating how the encoding of a data variable $d_i$ is affected by that of all others, i.e., $\forall d_j \in \{d_1, d_2, \ldots, d_m\} (j \neq i)$. There are $m$ assessments for $m$ variables.%
    \item[Type D.] A criterion is for evaluating the whole glyph holistically. There is one assessment per glyph for each Type D criterion.
\end{description}

\begin{figure}[ht]
    \centering
    \includegraphics[width=\linewidth]{Figures/Hierachy.pdf}
    \caption{Weighted aggregation is commonly used in hierarchical MCDA. In terms of four basic assessment modes, a Type D criterion may be scored directly, or assessed by aggregating the scores of Type A, Type B, and Type C criteria.}
    \label{fig:Hierarchy}
    \vspace{-6mm}
\end{figure}

As illustrated in Figure \ref{fig:Hierarchy}, the score of a high-level criterion (e.g., Type D) can often be obtained from scores of low-level criteria (e.g., Type A and Type B) through aggregation.
If a MCDA process consists of many low-level criteria, the designer will need to determine scores for individual variables or pairs of variables. In comparison with produce a high-level score directly (i.e., not through aggregation), the low-level scores are usually more precise, easier to judge, but more time-consuming to obtain.
%
Hence, there is a trade-off between precision and time cost.
In our proposed MCDA scheme, we indicate the possible modes, in which each criterion could be assessed, and recommend a specific mode of assessment for optimizing the trade-off concerned.   

\noindent\textbf{Design Consideration 3. Multiple complementary or conflicting facets.}
In the literature, a suggested criterion may consist two or more facets. For example, attention balance \cite{Chung:2016:CGF} encourages glyph designers to allow the encoding of important data variables to attract more attention, while making sure other data variables are not seriously disadvantaged. To ease the assessment, we intentionally divide such a multi-facet criterion into two criteria at the same level, allowing the designers to score the two potentially conflicting criteria independently.


\section{An MCDA Scheme for Glyph Design}
\label{sec:Scheme}
%
Our proposed scheme consists of 12 criteria described in the following subsections (see also Appendices \ref{apx:FrameworkDesign} and \ref{apx:Workflow}). For each criterion, we provide a definition, a specification of five-level scores [1$\sim$5], a recommended weight, and a recommended mode of assessment (in black) and other possible modes (in grey).  

% --------------------
\subsection{Typedness (Criterion 1)}
\label{sec:Typedness}
%
\begin{itemize}
\item \textbf{Definition:} This criterion assesses whether or not the visual channel (channels) of a data variable is (are) appropriately selected to match the data type of the variable to be encoded. Such data types may include, but are not limited to: \emph{nominal}, \emph{ordinal}, \emph{interval}, \emph{ratio}, and \emph{directional}.%
% \vspace{-1.5mm}
\item \textbf{Recommended Modes:} Type A (direct -- \emph{directly assessed}), Type D (aggregated -- \emph{using Type A scores}).%
% \vspace{-1.5mm}
\item \textbf{Recommended Weight:} Type A (unweighted -- \emph{all have weight 1 when being aggregated}), Type D [1.0].
\end{itemize}

The above definition was proposed by Chung et al. \cite{chung2015glyph,Borgo:2013:STAR} for assessing each visual channel in a glyph based on Bertin's four kinds of perception (KOP), i.e., \emph{associative}, \emph{selective}, \emph{ordered}, and \emph{quantitative} perception.
Not all KOPs are applicable to all data types. For example, \emph{ordered} and \emph{quantitative} perception are not both required for a nominal variable. Hence, we consider only applicable KOPs (i.e., AKOPs in short). See also Appendices \ref{apx:Bertin} and \ref{apx:Typedness} in the supplementary materials.%

We consider three levels of typedness, \emph{appropriate}, \emph{usable}, and \emph{inappropriate}. 
When a data variable $d_i$ is encoded using  just one visual channel $\gamma_i$, we may derive a score $s_i$ as follows:    
%
\begin{enumerate}
    \item[5.] The visual channel is appropriate for all AKOPs.
    % \vspace{-1.5mm}
    \item[4.] The visual channel is appropriate for some AKOPs and is usable for the other AKOPs.
    % \vspace{-1.5mm}
    \item[3.] The visual channel is usable for all AKOPs.
    % \vspace{-1.5mm}
    \item[2.] The visual channel is inappropriate for some AKOPs.
    % \vspace{-1.5mm}
    \item[1.] The visual channel is inappropriate for all AKOPs.
\end{enumerate}

When a data variable $d_i$ is encoded using two or more visual channels $\gamma_{i,1}, \gamma_{i,2}, \ldots$, we consider the best visual channel for each KOP. Hence the above scores are redefined as:

\begin{enumerate}
    \item[5.] For each AKOP, at least one visual channel is appropriate.
    % \vspace{-1.5mm}
    \item[4.] For some AKOPs, the best visual channel is appropriate, while for the other AKOPs, the best is usable.
    % \vspace{-1.5mm}
    \item[3.] For each AKOP, the best visual channel is usable.
    % \vspace{-1.5mm}
    \item[2.] For some AKOPs, all visual channels are inappropriate.
    % \vspace{-1.5mm}
    \item[1.] For each AKOP, all visual channels are inappropriate.
\end{enumerate}

% --------------------
\subsection{Discernability (Criterion 2)}
\label{sec:Discernability}
%
\begin{itemize}
\item \textbf{Definition:} This criterion assesses whether the encoding of a data variable allows viewers to differentiate key values or major value ranges. The encoding may use one or more visual channels.%
% \vspace{-1.5mm}
\item \textbf{Recommended Modes:} Type A (direct), Type D (aggregated).%
% \vspace{-1.5mm}
\item \textbf{Recommended Weight:} Type A (unweighted), Type D [1.0].
\end{itemize}


A data variable may have a number of valid values in an application context. For some numerical variables, the number of values can be huge. Glyph-based visualization is normally not intended for users to observe such data variables at a high-resolution (see Appendix \ref{apx:Discernability}).
It is therefore helpful to define a set of key values (or key data ranges) for each data variable in an application context.  
%
Given $k$ key values (or key value ranges), viewers potentially need to differentiate $n = k(k-1)/2$ pairs of values (ranges). We define five levels based on $n$.

\begin{enumerate}
    \item[5.] All $n$ pairs of values (ranges) can be differentiated at ease.%
    % \vspace{-1.5mm}
    \item[4.] Most of the $n$ pairs of values (ranges) can be differentiated at ease, and the rest are perceptually differentiable. Numerically, ``most'' is defined as $[75, 100)$, i.e., $\geq 75\%$ and $< 100\%$.%
    % \vspace{-1.5mm}
    \item[3.] A large portion of the $n$ pairs of values (ranges) can be differentiated at ease, and the rest are perceptually differentiable. Numerically, ``large portion'' is defined as $[50, 75)$, i.e., $\geq 50\%, < 75\%$.%
    % \vspace{-1.5mm}
    \item[2.] Most of the $n$ pairs of values (ranges) are perceptually differentiable. The rest are not. As for [4.], ``most'' is defined as $[75, 100)$.%
    % \vspace{-1.5mm}
    \item[1.] A significant portion of the $n$ pairs of values (ranges) are not perceptually differentiable. Numerically, ``significant portion'' is defined as [25, 100], i.e., $\geq 25\%$.
\end{enumerate}


% --------------------
\begin{figure}[t!]
  \centering
  % \includegraphics[width=85mm]{Figures/Intuitiveness.pdf}
  \includegraphics[width=\linewidth]{Figures/Intuitiveness.pdf}
  \caption{Maguire et al. \cite{Maguire:2012:TVCG} discussed the encoding of seven levels of material granularity in a biological application. The intuitiveness of six encoding methods is assessed here.}
  \label{fig:Intuitiveness}
  % \vspace{-4mm}
\end{figure}

% --------------------
\subsection{Intuitiveness (Criterion 3)}
\label{sec:Intuitiveness}
%
\begin{itemize}
\item \textbf{Definition:} This criterion assesses how the encoding of a data variable is semantically-related to the knowledge of viewers, and how such a relation makes the encoding knowable to the viewers by intuition.%
% \vspace{-1.5mm}
\item \textbf{Recommended Modes:} Type A (direct), Type D (aggregated).%
% \vspace{-1.5mm}
\item \textbf{Recommended Weight:} Type A (unweighted), Type D [1.0].
\end{itemize}

Given a data variable $d_i$ in an application domain, there may be a \emph{Domain-specific Convention} (DC) for encoding the data variable visually. Such encoding captures the semantic knowledge of the discipline, hence facilitating intuitiveness. Given a design option for encoding a data variable $d_i$, there are three basic scenarios of DC: no existing DC (noDC), consistent with the existing DC (cnDC), and inconsistent with the existing DC (inDC).

The encoding may also introduce one or more \emph{Additional Visual Metaphors} (AM) for enhancing the existing DC or filling the gap when there is no existing DC. An appropriate metaphor can make intuitive connection between the semantics of $d_i$ and viewers' knowledge. Hence, we consider four basic scenarios of AM: no additional metaphor (noAM), appropriate metaphor (apAM), adequate metaphor (okAM), and inappropriate metaphor (inAM). 
%
The two sets of scenarios result in 12 combinations. We define five levels as:
%
\begin{enumerate}
    \item[5.] cnDC-apAM or noDC-apAM.
    % \vspace{-1.5mm}
    \item[4.] cnDC-noAM, cnDC-okAM, or noDC-okAM.
    % \vspace{-1.5mm}
    \item[3.] noDC-noAM.
    % \vspace{-1.5mm}
    \item[2.] cnDC-inAM, inDC-okAM, inDC-apAM.
    % \vspace{-1.5mm}
    \item[1.] noDC-inAM, inDC-noAM, inDC-inAM.
\end{enumerate}
%
\noindent Here we do not assume that DC is always better than AM, or vice versa. Nevertheless, the criterion awards the creativity of visual design by rating noDC-apAM slightly higher than cnDC-noAM. Examples of how this criterion can be applied are shown in Fig.~\ref{fig:Intuitiveness}. See also Appendix \ref{apx:Intuitiveness}.

% --------------------
\subsection{Invariance: Geometry (Criterion 4)}
\label{sec:Geometry}
%
\begin{itemize}
\item \textbf{Definition:} This criterion assesses the undesirable impact of the geometrical variations in displaying a glyph upon its visual quality. The primary geometrical variations are size variations. Other considerations include minor variations of aspect radio, projection angle, and rotation.%
% \vspace{-1.5mm}
\item \textbf{Recommended Modes:} \textcolor{gray}{Type A,} Type D (direct).%
% \vspace{-1.5mm}
\item \textbf{Recommended Weight:} Type D [0.5].
\end{itemize}

Maguire et al. \cite{Maguire:2012:TVCG} conducted scalability tests in their glyph design process. Legg et al. \cite{legg2016glyph} studied the geometric and color degeneration of glyphs. To reduce the complexity of assessing these two types of degeneration (i.e., Design Consideration 2), we have two sub-criteria, geometry (this section) and colorimetry (next section), under the common quality term ``Invariance''.

\begin{figure}[t!]
    \centering
    % \includegraphics[width=81mm]{Figures/Invariance.pdf}
    \includegraphics[width=\linewidth]{Figures/Invariance.pdf}
    \caption{The view area depends on two main factors, visual field and viewing distance. Five scaling factors are applied to a circular glyph \cite{Duffy:2015:TVCG} and a rectangular glyph \cite{Maguire:2012:TVCG}.}
    \label{fig:Invariance}
    % \vspace{-5mm}
\end{figure}

In vision science and human-computer interaction, there is a wealth of research on viewing distances and icon sizes. We can build on these findings to consider the geometric scalability of glyphs. As illustrated in Fig.~\ref{fig:Invariance}, the viewing area can be approximated by a cone that has a visual field VF degrees and a viewing distance VD. The central VF of human eyes is $1.5^\circ \sim 5^\circ$ \cite{wandell1995foundations}. Typical viewing distances for monitors, laptops, and phones are  $50\sim100$~cm, $40\sim76$~cm, and $41\sim46$~cm, respectively. We can estimate the viewing area as a circle with diameter $L_\text{diam}$ or its inner square with edge length $L_\text{edge}$ as follows:
%
% \vspace{-1mm}
\[
    % L_\text{diam} = 2 \cdot \text{\text{VD}} \cdot \tan(\text{VF}^\circ), \qquad L_\text{edge} = \frac{L_\text{diam}}{\sqrt{2}}
    L_\text{diam} = 2 \cdot \text{\text{VD}} \cdot \tan(\text{VF}^\circ), \qquad L_\text{edge} = L_\text{diam} / \sqrt{2}
% \vspace{-1mm}
\]
%
\noindent In this work, we use $\text{VF}=5^\circ$ \cite{wandell1995foundations,strasburger2011peripheral} and $\text{VD}=50$~cm (widely-recommended viewing distance) as the baseline measures. The baseline of a circular viewing area has a diameter of 4.37~cm and a square with an edge length of 3.09~cm as illustrated in Fig.~\ref{fig:Invariance}.
We can define the five levels based on scaling factors $1/5,\,2/5,\, 3/5,\, 4/5,\, 5/5$ and the absence of any degradation of discernability of any visual channel (i.e., invariance) at each scale.  
%
\begin{enumerate}
    \item[5.] Discernability is invariant at the $1/5$ scale.%
    % \vspace{-1.5mm}
    \item[4.] Discernability is invariant at the $2/5$ scale but variant at $1/5$.%
    % \vspace{-1.5mm}
    \item[3.] Discernability is invariant at the $3/5$ scale but variant at $2/5$.%
    % \vspace{-1.5mm}
    \item[2.] Discernability is invariant at the $4/5$ scale but variant at $3/5$.%
    % \vspace{-1.5mm}
    \item[1.] Discernability is variant at the $4/5$ scale.%.
\end{enumerate}
%
As exemplified by Maguire et al. \cite{Maguire:2012:TVCG}, such invariance tests can easily be carried out by glyph designers. Since one will likely apply a scaling factor to a whole glyph, we recommend to assess this criterion with a Type D score directly. See also Appendix \ref{apx:Geometry}. 

% --------------------
\subsection{Invariance: Colorimetry (Criterion 5)}
\label{sec:Colorimetry}
%
\begin{itemize}
\item \textbf{Definition:} This criterion assesses the undesirable impact of the non-geometrical appearance variations in displaying a glyph upon its visual quality. The main considerations are chromatic and achromatic variations of colors, which may be caused by limitations of a display device and/or environmental lighting conditions.%
% \vspace{-1.5mm}
\item \textbf{Recommended Modes:} \textcolor{gray}{Type A,} Type D (direct).%
% \vspace{-1.5mm}
\item \textbf{Recommended Weight:} Type D [0.5].
\end{itemize}

Chromatic and achromatic variations of colors are commonly caused by undesirable environmental lighting (e.g., reflection) and occasionally by display devices (e.g., in an energy saving mode). Such variations can be approximated by using a function for varying the contrast and brightness of the imagery representation of a glyph. Consider the RGB representation of a pixel, such that $r, g, b \in [0, 255]$. One commonly-used function \cite{Loch:2021:web} is:
%
% \vspace{-1mm}
\[
    x' = \min\biggl(255, \max\bigl(0, \frac{259(\kappa_\text{ctr} + 255)}{255(259-\kappa_\text{ctr})} (x - 128) + 128 + \kappa_\text{brt} \bigr) \biggr)
% \vspace{-1mm}
\]
%
\noindent where $\kappa_\text{ctr} \in [-255, 255]$ specifies the increment (positive) or decrement (negative) of contrast, while $\kappa_\text{brt} \in [-255, 255]$ specifies the increment or decrement of brightness. We recommend to assess this criterion with a Type D score. We define the five levels based on 10\%, 20\%, 30\%, and 40\% of variations, which are translated to $\pm 25.5, \pm 51, \pm 76.5,$ and $\pm 102$ for $\kappa_\text{ctr}$ and $\kappa_\text{brt}$. See also Appendix \ref{apx:Colorimetry}.

\begin{enumerate}
    \item[5.] Discernability is invariant with $\kappa_\text{ctr} = \pm 102 \land \kappa_\text{brt} = \pm 102$.%
    % \vspace{-1.5mm}
    \item[4.] Discernability is invariant when $\kappa_\text{ctr} = \pm 76.5 \land \kappa_\text{brt} = \pm 76.5$, but variant when $\kappa_\text{ctr} = \pm 102 \lor \kappa_\text{brt} = \pm 102$.%
    % \vspace{-1.5mm}
    \item[3.] Discernability is invariant when $\kappa_\text{ctr} = \pm 51 \land \kappa_\text{brt} = \pm 51$, but variant when $\kappa_\text{ctr} = \pm 76.5 \lor \kappa_\text{brt} = \pm 76.5$.%
    % \vspace{-1.5mm}
    \item[2.] Discernability is invariant when $\kappa_\text{ctr} = \pm 25.5 \land \kappa_\text{brt} = \pm 25.5$, but variant when $\kappa_\text{ctr} = \pm 51 \lor \kappa_\text{brt} = \pm 51$.%
    % \vspace{-1.5mm}
    \item[1.] Discernability is variant when $\kappa_\text{ctr} = \pm 25.5 \lor \kappa_\text{brt} = \pm 25.5$.%
\end{enumerate}

\revise{Similar to the criterion of geometry invariance in Section \ref{sec:Geometry}, we recommend to assess this criterion with a Type D score directly because one will likely test the invariance of colorimetry by manipulating a whole glyph.}

% --------------------
\subsection{Composition: Separability ((Criterion 6)}
\label{sec:Separability}
%
\begin{itemize}
\item \textbf{Definition:} This criterion assesses the undesirable interference among visual channels in a glyph, which would affect the perception of some visual channels.%
% \vspace{-1.5mm}
\item \textbf{Recommended Modes:} \textcolor{gray}{Type B, Type C,} Type D (direct).%
% \vspace{-1.5mm}
\item \textbf{Recommended Weight:} Type D [0.5].
\end{itemize}

\begin{figure*}[ht]
    \centering
    \includegraphics[width=180mm]{Figures/Separability.pdf}
    \caption{Example designs of bi-variate glyphs in four severity categories. We can observe that introducing boundaries to separate different visual channels (b1-d1) or reference lines (c2, d2) can reduce the severity of the interference between two visual channels.}
    \label{fig:Separability}
    % \vspace{-4mm}
\end{figure*}

This criterion is based on the discourse by Maguire et al. \cite{Maguire:2012:TVCG} and Chung et al. \cite{chung2015glyph} on the interference among visual channels in glyph design. Such interference is often caused by visual channels that are integrated into the same visual object or hosted by closely-placed visual objects. Visual channels of similar types (e.g., brightness, length) are more likely to suffer from interference as exemplified in Fig.~\ref{fig:Separability}. 
%
In general, a glyph occupies a small display space, and interference is often unavoidable.
%
Such minor interference would normally be considered to be acceptable in glyph design. In addition, for some applications, it is desirable to make some visual channels comparable, which conflicts with the desire for separating visual channels by using different types of visual channels or moving them away from each other. To balance these two sides of the same ``Composition'' coin, we have introduced ``comparability'' as a distinct criterion.

Although many empirical studies have evaluated the interference among different visual channels, there is not yet a standard metric for measuring the severity of such interference. While such a standard metric will hopefully be defined in future research, we hereby use a relatively subjective measure to categorize interference in glyph design as \emph{major}, \emph{medium}, \emph{minor}, and \emph{none}, which correspond to interference scores 1, 0.1, 0.01, and 0 respectively. Fig.~\ref{fig:Separability} shows two examples for each severity category.  

Given a total of $n$ visual channels $\lambda_1, \lambda_2, \ldots \lambda_n$ in a glyph, each channel $\lambda_i$ can potentially be influenced by $n-1$ other channels. Let $s_\text{int}(\lambda_i, \lambda_j)$ be the score of the interference received by $\lambda_i$ from $\lambda_j$, and $S_\text{int}(\lambda_i)$ be the aggregated score for $\lambda_i$, which is defined as:
%
% \vspace{-1mm}
\[
    S_\text{int}(\lambda_i) = \max \bigl\{
    s_\text{int}(\lambda_i, \lambda_j) \; | \; j=1,2, \ldots, n \;\land\; j \neq i \bigr\}
% \vspace{-1mm}
\]
%
\noindent $s_\text{int}(\lambda_i, \lambda_j)$ is a pairwise score for Type B assessment, and $S_\text{int}(\lambda_i)$ is a score for Type C assessment.
From these scores, we can obtain two Type D scores, a mean score $\text{avg}_\text{int}$ and a maximum score $\max_\text{int}$:
%
% \vspace{-1mm}
\[
    \text{avg}_\text{int} = \frac{1}{n} \sum_{i=1}^n S_\text{int}(\lambda_i) \qquad
    \text{max}_\text{int} = \max\limits_{i=1}^n S_\text{int}(\lambda_i)
% \vspace{-1mm}
\]
%
\noindent We can therefore define the five levels as:
%
\begin{enumerate}
    \item[5.] $0.0 \leq \max_\text{int} < 0.1$: only minor interference.%
    % \vspace{-1.5mm}
    \item[4.] $0.1 \leq \max_\text{int} < 1.0$: some medium but no major interference.%
    % \vspace{-1.5mm}
    \item[3.] $\max_\text{int} = 1.0 \;\land\; \text{avg}_\text{int} < 1/8$: some major interference, and less than $1/8$ of visual channels are affected.
    % \vspace{-1.5mm}
    \item[2.] $\max_\text{int} = 1.0 \;\land\; 1/8 \leq \text{avg}_\text{int} < 1/4$: some major interference, and between $1/8$ and $1/4$ of visual channels are affected.
    % \vspace{-1.5mm}
    \item[1.] $\max_\text{int} = 1.0 \;\land\; \text{avg}_\text{int} \geq 1/4$: some major interference, and more than $1/4$ of visual channels are affected.
\end{enumerate}
%
\noindent As obtaining $n(n-1)$ pairwise scores $s_\text{int}(\lambda_i, \lambda_j)$ will be time-consuming, we recommend to evaluate each glyph design holistically. The above specification of the five levels facilities the option of obtaining a Type D score directly. See also Appendix \ref{apx:Separability}.

% --------------------
\subsection{Composition: Comparability (Criterion 7)}
\label{sec:Comparability}
%
\begin{itemize}
\item \textbf{Definition:} This criterion assesses the desirable level of support featured in a glyph design for enabling required comparative tasks such as determining the order of two related data variables ($d_i$ vs. $d_j$), estimating their addition $(d_i+d_j)$, their difference $|d_i - d_j|$, or their ratio $d_i / d_j$.%
% \vspace{-1.5mm}
\item \textbf{Recommended Modes:} \textcolor{gray}{Type B, Type C,} Type D (direct).%
% \vspace{-1.5mm}
\item \textbf{Recommended Weight:} Type D [0.5].
\end{itemize}

As mentioned in Section \ref{sec:Separability}, this criterion complements ``Separability''. It was not discussed explicitly in the existing surveys (e.g., \cite{Borgo:2013:STAR}), possibly because one only considers this when there is a need to compare some data variables within a glyph. Nevertheless, several glyph designs in the literature addressed the need to compare some data variables within a glyph representation. For example, Duffy et al. \cite{Duffy:2015:TVCG} presented a glyph design representing some 20 data variables, among which three related distance variables were to be compared in terms of their lengths and relative ratios. Duffy et al. encoded these variables using three nested arcs, facilitating easy comparison.
%
When one needs to compare two visual channels that encode two different data variables, the following obstacles will likely hinder comparative tasks:
%
\begin{itemize}
    \item \emph{Major obstacle} -- Two visual channels are of very different types, e.g., area vs. brightness as shown in Fig.~\ref{fig:Separability} (d1).%
    % \vspace{-1.5mm}
    \item \emph{Major obstacle} -- Two visual channels are of the same type but with inconsistent encoding schemes, e.g., two color channels, one with a divergence colormap and another with a sequential colormap, or two length channels, one uses 20 pixels to encode the range [0, 10] and another uses 40 pixels for the same range.%
    % \vspace{-1.5mm}
    \item \emph{Medium obstacle} -- Two visual channels are of the same type but with features that affect consistent perception, e.g., same brightness range but with different hues in Fig.~\ref{fig:Separability} (c1), or same length encoding but in different orientation in Fig.~\ref{fig:Separability} (b1).%
    % \vspace{-1.5mm}
    \item \emph{Medium obstacle} -- Two visual channels do not have any common reference point, e.g., two length channels without any reference lines, unlike Fig.~\ref{fig:Separability} (c2, d2).%
    % \vspace{-1.5mm}
    \item \emph{Minor obstacle} -- Two visual channels are placed far away from each other, where the word ``far'' is in the context of a glyph. 
\end{itemize}
 
It is necessary to note that, when a data variable is encoded using multiple visual channels, as long as one of the channels is comparable, one may omit the consideration of other channels. For example, if data variable $d_1$ is encoded using both length and a continuous colormap, and data variable $d_2$ is encoded using length only, we only need to consider $d_1$-length vs. $d_2$-length. We can define the five levels as:
%
\begin{enumerate}
    \item[5.] \emph{Major}: none; \emph{Medium}: none; \emph{Minor}: none.%
    % \vspace{-1.5mm}
    \item[4.] \emph{Major}: none; \emph{Medium}: none; \emph{Minor}: one or a few.%
    % \vspace{-1.5mm}
    \item[3.] \emph{Major}: none; \emph{Medium}: one; \emph{Minor}: more than a few.%
    % \vspace{-1.5mm}
    \item[2.] \emph{Major}: none; \emph{Medium}: more than one; \emph{Minor}: any.%
    % \vspace{-1.5mm}
    \item[1.] \emph{Major}: at least one. \emph{Medium}: any; \emph{Minor}: any.
\end{enumerate}
%
\noindent where we recommend that the term ``a few'' is defined as less than 10\% of all pairwise comparisons, and ``more than a few'' is 10\% or more but less than 50\%.
We also recommend to evaluate this criterion holistically by obtaining a Type D score directly. When there is no need for comparing any data variables within a glyph, we recommend to set the weight for this criterion to zero. See also Appendix \ref{apx:Comparability}.

% --------------------
\subsection{Attention: Importance (Criterion 8)}
\label{sec:Importance}
%
\begin{itemize}
\item \textbf{Definition:} This criterion assesses the desirable level of support in a glyph design for encoding data variables according to their importance, e.g., by allocating more pre-attentive visual channels or higher encoding bandwidth to more important data variables.%
% \vspace{-1.5mm}
\item \textbf{Recommended Modes:} \textcolor{gray}{Type B, Type C,} Type D (direct).%
% \vspace{-1.5mm}
\item \textbf{Recommended Weight:} Type D [0.5].
\end{itemize}

Ropinski and Preim \cite{ropinski2011survey} instigated the benefit of encoding data variables according to their importance in the context of an application. Maguire et al. \cite{Maguire:2012:TVCG} identified several factors that may influence importance ranking: the level of a variable in a taxonomy, its usage in users' tasks, and so on.
The factors that help a visual channel receive more attention include the pop-out effect, the hierarchy effect, the size of the visual objects hosting the visual channel, and so on. Furthermore, when a data variable is encoded using multiple visual channels, it will likely receive more attention.
% 
Maguire et al. presented a method to bring the rankings of variables and visual channels together in glyph design.    

Chung et al. \cite{chung2015glyph} defined ``attention balance'' as a criterion for matching the levels of attention that visual channels may receive with the importance levels of the variables. The term ``attention balance'' implicitly indicates two sides of the same coin.
%
Following the third design consideration, we make these two sides as two sub-criteria. This criterion focuses on ``importance'', and the next criterion on ``balance''. When there is no importance ranking of the data variables within a glyph, we recommend to set the weight for the importance criterion to zero. Nevertheless, the balance criterion will always be assessed. 

\begin{figure}[t]
    \centering
    % \includegraphics[width=85mm]{Figures/Importance.pdf}
    \includegraphics[width=\linewidth]{Figures/Importance.pdf}
    \caption{Five levels of the ``Attention: Importance'' criterion are defined based on different amounts of correlation between the importance ranks of data variables (encoded using $y$-position) and their attention ranks (encoded using number, color, edge thickness, and edge darkness). Four examples are shown at each level.}
    \label{fig:Importance}
    % \vspace{-4mm}
\end{figure}

Consider a list of data variables, $d_1, d_2, \ldots, d_n$. Each variable $d_i$ is associated with two ranking values: $\iota_i$ for the importance of $d_i$ and $\alpha_i$ for the attention of $d_i$ through its visual encoding.
We have $\iota_i, \alpha_i \in [1, n]$.
The highest ranks of importance and attention are represented by 1 and the lowest by $n$. If two data variables are ranked same for importance (or attention), their $\iota$ (or $\alpha$) values are the same. We can compute the Pearson correlation coefficient (see also Appendix \ref{apx:Importance}):
%
% \vspace{-1mm}
\[
    C = \frac{\sum (\iota_i - \overline{\iota})(\alpha_i - \overline{\alpha}) }{\sqrt{\sum (\iota_i - \overline{\iota})^2 \sum (\alpha_i - \overline{\alpha})^2}}
% \vspace{-1mm}
\]
%
{\noindent where $\overline{\iota}$ and $\overline{\alpha}$ are mean ranking values of importance and attention respectively. The five levels are defined as:
%
\begin{enumerate}
    \item[5.] The correlation coefficient $C > 0.95$.%
    % \vspace{-1.5mm}
    \item[4.] The correlation coefficient $0.85 < C \leq 0.95$.%
    % \vspace{-1.5mm}
    \item[3.] The correlation coefficient $0.5 < C \leq 0.85$.%
    % \vspace{-1.5mm}
    \item[2.] The correlation coefficient $0 < C \leq 0.5$.%
    % \vspace{-1.5mm}
    \item[1.] The correlation coefficient $C \leq 0$.
\end{enumerate}

Fig.~\ref{fig:Importance} shows four examples at each level. We can observe minor misalignment of the ordering has limited impact on the correlation coefficient, which is suitable for the uncertainty in ranking importance and attention, because of the subjective nature of importance ranking and the lack of experimental measures in attention ranking.
In theory, one may first obtain Type B or Type C scores. In practice, it is more efficient to compute a Type D score directly.  

% --------------------
\begin{figure*}[ht]
    \centering
    \includegraphics[width=180mm]{Figures/Imbalance.pdf}
    \caption{Examples of data variables that are ``overshadowed'' by other data variables, and may easily be overlooked (i.e., inattentional blindness). When one is asked to point out which variables have been changed between the two glyphs, the arrow directions in (a) and the triangular marker and the length of the dotted arc in (b) may receive significantly less attention than other variables. }
    \label{fig:Imbalance}
    % \vspace{-4mm}
\end{figure*}
% --------------------
\subsection{Attention: Balance (Criterion 9)}
\label{sec:Balance}
%
\begin{itemize}
\item \textbf{Definition:} This criterion assesses the undesirable disadvantages that some data variables may suffer, which may make such data variables easily overlooked or difficult to perceive. 
%
\item \textbf{Recommended Modes:} \textcolor{gray}{Type B, Type C,} Type D (direct).%
% \vspace{-1.5mm}
\item \textbf{Recommended Weight:} Type D [0.5].%
% \vspace{-1.5mm}
\end{itemize}
%
As described in Section \ref{sec:Importance}, this criterion is assessing the opposite side of the same coin of ``Attention''. When attention is prioritized for the importance of data variables, it is necessary to ensure that no data variable may be seriously disadvantaged or suffer from inattentional blindness, which is a phenomenon studied extensively in psychology. In the context of glyph design, inattentional blindness primarily occurs when some data variables attract significantly more attention and thus limit cognitive resource, causing the variations of some other data variables to go unnoticed.

Here we assume that (i) the variables concerned are discernable (see Section \ref{sec:Discernability}), and (ii) the importance-based ordering is correct (see Section \ref{sec:Importance}). The blindness is caused by imbalanced allocation of cognitive resource for noticing variations. The factors of imbalance may include (a) peripheral location, (b) unsaturated color, (c) minor shape variation, (d) small object, (e) variation demanding high cognitive load, and so on. As illustrated in Fig.~\ref{fig:Imbalance}, the blindness is usually due to the co-existence of two or more such factors.

Ideally, empirical research in the future will provide us with methods for identifying visual encoding that may suffer from inattentional blindness. Until then, one may identify such a variable by juxtaposing two glyphs (of the same design) where all data variables have some variations. As illustrated in Fig.~\ref{fig:Imbalance}, one can observe those variables receiving weak attention, i.e., their variations are easily overshadowed by other variables.
%
We recommend evaluating this criterion holistically by obtaining a Type D score directly. We define the five levels based on the number of data variables that receive weak attention and may cause inattentional blindness:
%
\begin{enumerate}
    \item[5.] No data variable receives weak attention.%
    % \vspace{-1.5mm}
    \item[4.] One data variable receives weak attention.%
    % \vspace{-1.5mm}
    \item[3.] Two data variables receive weak attention.%
    % \vspace{-1.5mm}
    \item[2.] Three data variables receive weak attention.%
    % \vspace{-1.5mm}
    \item[1.] More than three data variables receive weak attention.
\end{enumerate}

% --------------------
\subsection{Searchability (Criterion 10)}
\label{sec:Searchability}
%
\begin{itemize}
\item \textbf{Definition:} This criterion assesses the desirable property that the visual channel(s) for each data variable can be recognized easily among others after a viewer has learned and remembered the encoding scheme. 
%
\item \textbf{Recommended Modes:} \textcolor{gray}{Type C,} Type D (direct).%
% \vspace{-1.5mm}
\item \textbf{Recommended Weight:} Type D [0.5].%
% \vspace{-1.5mm}
\end{itemize}
%
Chung et al. \cite{chung2015glyph} defined ``Searchability'' as the level of ease when one needs to identify a visual channel associated with a specific data variable. Here we assume that the user has already learned and remembered such an association semantically. This allows us to consider searchability independently of whether the encoding is easy to learn or remember. As illustrated in Fig.~\ref{fig:Searchability}, when many variables have similar visual encoding except their positions in a glyph, they can be difficult to find, despite their encoding following Bertin's rules and receiving adequate attention. See also Appendices \ref{apx:Orthogonality} and \ref{apx:Searchability}.

\begin{figure}[t]
    \centering
    \includegraphics[width=\linewidth]{Figures/Searchability.pdf}
    \caption{Three visual designs can encode (a) 10 numerical variables, (b) 9 numerical variables, and (c) 36 Boolean variables respectively. For some visual channels in these three glyphs, it is not easy to relate a visual channel to a specific variable, though they work fine in normal large plots. Problems can be alleviated if the number of similar visual channels are reduced as in (d) and (e).}
    \label{fig:Searchability}
    % \vspace{-4mm}
\end{figure}

Let us consider a meta-variable $v_\text{meta} = r(d, \lambda)$ for representing the association between a data variable $d$ and a visual channel $\lambda$, such that $v_\text{meta} = \textbf{true}$ if an association exists and \textbf{false} otherwise. In each of the three examples in Fig.~\ref{fig:Searchability}, $v_\text{meta}$ is encoded primarily using a spacial location, which may also be searched through a related visual cue such as angle or count. In psychology, previous experiments have shown that the accuracy and response time of visual search and counting are affected by the number of objects and some other factors.
%
For example, in Fig.~\ref{fig:Searchability}(a), the first and last bars are easier to search than other eight. 
If there were a smaller number of bars, e.g., a group of five bars in Fig.~\ref{fig:Searchability}(d,e), symmetry can aid the visual search.
In Fig.~\ref{fig:Searchability}(b), for some numbers, the bars can be placed along lines from the center to the vertices of a square, hexagon, octagon, or even dodecagon, with one vertex at the 12 o'clock direction. These placements can also aid visual search. One can easily use multiple encoding by adding additional visual cues (e.g., colors and symbols) to improve the searchability. For example, in Fig.~\ref{fig:Searchability}(c), one could replace each cross symbol with an object that has the shape defining the column and the color defining the row. In Fig.~\ref{fig:Searchability}(e), the 10 bars are divided into two groups using two colors. 
Ideally, the assessment of this criterion could be based on the measurement of accuracy, response time, and/or cognitive load in visual search. The previous empirical research in psychology has not provided standardized measurements that can be used for assessing glyph designs. We anticipate that this will be obtained in future empirical studies in visualization. For the time being, we coarsely define three levels of cognitive effort in visual search as:

\begin{itemize}
    \item \emph{Low cognitive load} -- It requires almost no effort to find a specific variable, e.g., the first or last bar in Fig.~\ref{fig:Searchability}(a).%
    % \vspace{-1.5mm}
    \item \emph{Medium cognitive load} -- It requires a small and undemanding amount of counting or reasoning effort. One normally feels such an effort, but is fairly sure about the search results. For example, the 2nd and 3rd bars in Fig.~\ref{fig:Searchability}(a) fall into this category.%
    % \vspace{-1.5mm}
    \item \emph{High cognitive load} -- It requires an amount of searching effort that one feels bothersome or burdensome, while one may hesitate about the correctness of the search. For example, the 4th through the 7th bar in Fig.~\ref{fig:Searchability}(a) fall into this category.
\end{itemize}
%
Based on these three categories, we can define the five levels as:

\begin{enumerate}
    \item[5.] \emph{High}: none; \emph{Medium}: none; \emph{Low} all.%
    % \vspace{-1.5mm}
    \item[4.] \emph{High}: none; \emph{Medium}: one or a few; \emph{Low}: most.%
    % \vspace{-1.5mm}
    \item[3.] \emph{High}: none; \emph{Medium}: more than a few; \emph{Low}: more than half.%
    % \vspace{-1.5mm}
    \item[2.] \emph{High}: one or a few; \emph{Medium}: any; \emph{Low}: any.%
    % \vspace{-1.5mm}
    \item[1.] \emph{High}: more than a few; \emph{Medium}: any; \emph{Low}: any.
\end{enumerate}
%
\noindent where we recommend that the term “a few” is defined as fewer than 10\% of all data variables. We also recommend to evaluate this criterion holistically by obtaining a Type D score directly.

% --------------------
\subsection{Learnability (Criterion 11)}
\label{sec:Learnability}
%
\begin{itemize}
\item \textbf{Definition:} This criterion assesses the desirable property that the whole encoding scheme of a glyph is easy to explain and learn.%
% \vspace{-1.5mm}
\item \textbf{Recommended Modes:} Type D (direct).%
% \vspace{-1.5mm}
\item \textbf{Recommended Weight:} Type D [0.5].
\end{itemize}

Chung et al. \cite{chung2015glyph} defined learnability as the level of ease in learning and remembering a visual encoding scheme. As learning and memorizing are often studied separately in psychology, we split ``learnability'' and ``memorability'' into two related sub-criteria. For example, if the glyph in Fig.~\ref{fig:Searchability}(a) encodes the average marks of 10 courses (unnumbered), the scheme is easy to learn but difficult to remember. If it encodes the attendance of 10 weeks in an academic term, it is both easy to learn and remember.   
%
While both learnability and memorability can benefit from the intuitive encoding of individual visual channels (see Section \ref{sec:Intuitiveness}), there are many holistic factors, such as the total number of data variables, their relative positions, their semantic similarity and difference, and so on.
In order not to overload a criterion (Design Consideration 2), we let intuitiveness focus on individual visual channels through Type A scores, while focusing learnability and the memorability on two holistic sub-criteria through Type D scores.

Learnability is user-dependent. In order to focus on glyph designs, we assume that the target users have already had the knowledge about the data variables to be encoded, e.g.,
% the categorization of material perturbation, separation, amplification, combination, and collection in the context of \emph{in silico} biological data manipulation \cite{Maguire:2012:TVCG}
terms such as scrum, ruck, lineout, maul, and try in the context of rugby sports \cite{Legg:2012:CGF}.
In many applications, glyph-based visualization is designed for domain experts. Because of the assumption of domain knowledge, controlled or semi-controlled empirical studies are typically unsuitable for assessing this criterion since the lack of domain knowledge of the experiment participants would invalidate the experiment results.    
Meanwhile, learnability should be assessed in relation to the baseline that the target users have little knowledge about the visual design concerned. For example, domain experts often contribute directly ideas of visual encoding in a design process. Naturally, these domain experts have already ``learned'' the designs to be evaluated and their knowledge would bias the assessment.      

For a typical target user with adequate domain knowledge but little knowledge about the visual design to be evaluated, we define the five levels based on \emph{learning time}, \emph{learning mode} (i.e., levels of training engagement), and the effort required for \emph{repeated learning} after a short period of not using the glyphs. The five levels are:
%
\begin{enumerate}
    \item[5.] \emph{Learning time}: $<0.5$ hours; \emph{Learning mode}: self-learning only; \emph{Repeated learning}: effortless.%
    % \vspace{-1.5mm}
    \item[4.] \emph{Learning time}: $\geq 0.5,\,< 1.0$ hour; \emph{Learning mode}: self-learning + Q\&A; \emph{Repeated learning}: effortless.%
    % \vspace{-1.5mm}
    \item[3.] \emph{Learning time}: $\geq 1.0,\,< 1.5$ hours; \emph{Learning mode}: tutorial; \emph{Repeated learning}: minor effort.%
    % \vspace{-1.5mm}
    \item[2.] \emph{Learning time}: $\geq 1.5,\,< 2.0$ hours; \emph{Learning mode}: tutorial; \emph{Repeated learning}: noticeable effort.%
    % \vspace{-1.5mm}
    \item[1.] \emph{Learning time}: $\geq 3$ hours; \emph{Learning mode}: tutorial; \emph{Repeated learning}: serious effort.%
\end{enumerate}
%
\noindent Here we define three learning modes: self-learning only, self-learning + Q\&A, and tutorial. We define the effort for repeated learning as effortless (e.g., a quick glance at the encoding scheme), minor effort (e.g., reading the encoding scheme again for 5--10 minutes), noticeable (e.g., reading the encoding scheme again for 10--30 minutes and/or requiring Q\&A), and serious effort (e.g., requiring another tutorial and/or more than 30 minutes).  
Note that the frequency of repeated learning relates to memorability. See also Appendix \ref{apx:Learnability}.

% --------------------
\subsection{Memorability (Criterion 12)}
\label{sec:Memorability}
%
\begin{itemize}
\item \textbf{Definition:} This criterion assesses the desirable property that the whole encoding scheme of a glyph is easy to remember once a viewer has learned the scheme.%
% \vspace{-1.5mm}
\item \textbf{Recommended Modes:} Type D (direct).%
% \vspace{-1.5mm}
\item \textbf{Recommended Weight:} Type D [0.5].
\end{itemize}
%
As already discussed in Section \ref{sec:Learnability}, this complementary criterion assesses the easiness of memorizing an encoding scheme. Similar to learnability, it is a holistic criterion and is assessed through a Type D score. The assessment assumes that the users have already learned the encoding scheme, and the effort for repeated learning and memory refreshing is considered as part of learnability.

\begin{table*}[t]
\centering
\caption{A summary of the assessments of five glyph designs using the MCDA-aided scheme. \textbf{A} is is the original design by Maguire et al. \cite{Maguire:2012:TVCG}.
\textbf{B} is a variant of \textbf{A} configured based on several design options discussed in \cite{Maguire:2012:TVCG}.   
\textbf{C} is the original design by Legg et al. \cite{Legg:2012:CGF}.
\textbf{D} is a variant of \textbf{C} where two visual objects swap their positions and the central pictograms are replaced with abstract shapes discussed in \cite{Legg:2012:CGF}. 
\textbf{E} is a design by Chung et al. \cite{chung2015glyph}, which was partly based on \textbf{C}.
Only Type D scores are shown. Further details, including Type A scores, can be found in a spreadsheet file in the supplementary material.}
\label{tab:CaseStudy1}
%
% \vspace{-3mm}
\begin{tabular}{@{\hspace{48mm}}c@{\hspace{6mm}}c@{\hspace{12mm}}c@{\hspace{8mm}}c@{\hspace{11mm}}c@{}}
    \includegraphics[height=10mm]{Figures/MaguireOriginal} &
    \includegraphics[height=10mm]{Figures/MaguireParody} &
    \includegraphics[height=10mm]{Figures/LeggOriginal} &
    \includegraphics[height=10mm]{Figures/LeggParody} &
    \includegraphics[height=10mm]{Figures/ChungOriginal} \\
    \textbf{A:} Maguire et al. &
    \textbf{B:} Parody of \textbf{A} &
    \textbf{C:} Legg et al. &
    \textbf{D:} Parody of \textbf{C} &
    \textbf{E:} Chung et al.\\
\end{tabular}
\begin{tabular}{@{}p{46mm}@{\hspace{4mm}}%
        c@{\hspace{3mm}}c@{\hspace{8mm}}c@{\hspace{3mm}}c@{\hspace{12mm}}%
        c@{\hspace{3mm}}c@{\hspace{8mm}}c@{\hspace{3mm}}c@{\hspace{12mm}}c@{\hspace{3mm}}c@{}}
    \textbf{Criterion} &
        \textbf{weight} & \textbf{score} & \textbf{weight} & \textbf{score} &
        \textbf{weight} & \textbf{score} & \textbf{weight} & \textbf{score} &
        \textbf{weight} & \textbf{score}\\
    \hline
    Typedness                  &   1 & 5.00 &   1 & 4.71 &   1 & 5.00 &    1 & 5.00 &   1 & 5.00\\
    Discernability             &   1 & 5.00 &   1 & 5.00 &   1 & 5.00 &    1 & 5.00 &   1 & 5.00\\
    Intuitiveness              &   1 & 4.14 &   1 & 3.29 &   1 & 4.13 &    1 & 3.63 &   1 & 4.10\\
    Invariance: Geometry       & 0.5 &    5 & 0.5 &    4 & 0.5 &    5 &  0.5 &    5 & 0.5 &    3\\
    Invariance: Colorimetry    & 0.5 &    3 & 0.5 &    3 & 0.5 &    5 &  0.5 &    5 & 0.5 &    4\\
    Composition: Separability  & 0.5 &    5 & 0.5 &    1 & 0.5 &    5 &  0.5 &    3 & 0.5 &    5\\
    Composition: Comparability &     &      &     &      &     &      &      &      &     &\\
    Attention: Importance      & 0.5 &    5 & 0.5 &    5 & 0.5 &    5 &  0.5 &    4 & 0.5 &    5\\
    Attention: Balance         & 0.5 &    5 & 0.5 &    2 & 0.5 &    5 &  0.5 &    5 & 0.5 &    5\\
    Searchability              & 0.5 &    5 & 0.5 &    1 & 0.5 &    5 &  0.5 &    5 & 0.5 &    5\\
    Learnability               & 0.5 &    5 & 0.5 &    2 & 0.5 &    5 &  0.5 &    3 & 0.5 &    4\\
    Memorability               & 0.5 &    4 & 0.5 &    1 & 0.5 &    5 &  0.5 &    1 & 0.5 &    3\\
    \hline
    \textbf{Total Weight \& Weighted Average}
                               &   7 & 4.66 & 7   & 3.21 & 7   & 4.80 & 7    & 4.16 & 7    & 4.44\\ 
\end{tabular}
% \vspace{-4mm}
\end{table*}

Because of the effort to learn a glyph design, the learned encoding scheme must be stored in long-term memory. While one could assess how long the target users can remember an encoding scheme, the overall intention of this MCDA method is to evaluate different glyph designs without too much delay. Therefore, we recommend to base the assessment of this criterion on the memorability after 1 hour and 24 hours following learning. Both time periods meet the requirement for testing long-term memory \cite{Glanzer:1966:JVLVB,Baddeley:2020:book}. See also Appendix \ref{apx:Memorability}.

For a typical target user, we define the five levels according to how much the user can remember about an encoding scheme:

\begin{enumerate}
    \item[5.] \emph{after 1 hour}: 100\%, and \emph{after 24 hours}: 100\%.%
    % \vspace{-1.5mm}
    \item[4.] \emph{after 1 hour}: $<100\%$, $\geq 90\%$, or\\ \emph{after 24 hours}: $<100\%$, $\geq 75\%$.%
    % \vspace{-1.5mm}
    \item[3.] \emph{after 1 hour}: $<90\%$, $\geq 75\%$, or\\ \emph{after 24 hours}: $<75\%$, $\geq 50\%$.%
    % \vspace{-1.5mm}
    \item[2.] \emph{after 1 hour}: $<75\%$, $\geq 50\%$, or\\ \emph{after 24 hours}: $<50\%$, $\geq 25\%$.%
    % \vspace{-1.5mm}
    \item[1.] \emph{after 1 hour}: $<50\%$, or\\ \emph{after 24 hours}: $<25\%$.
\end{enumerate}









\begin{table*}[t]
\centering
\begin{tabular}{@{\hspace{46mm}}c@{\hspace{4mm}}c@{\hspace{8mm}}c@{\hspace{6mm}}c@{\hspace{6mm}}c@{}}
    \includegraphics[height=10mm]{Figures/MaguireOriginal} &
    \includegraphics[height=10mm]{Figures/MaguireParody} &
    \includegraphics[height=10mm]{Figures/LeggOriginal} &
    \includegraphics[height=10mm]{Figures/LeggParody} &
    \includegraphics[height=10mm]{Figures/ChungOriginal} \\
    \textbf{A:} Maguire et al. &
    \textbf{B:} Parody of \textbf{A} &
    \textbf{C:} Legg et al. &
    \textbf{D:} Parody of \textbf{C} &
    \textbf{E:} Chung et al.\\
\end{tabular}
\begin{tabular}{@{}p{46mm}@{\hspace{4mm}}%
        c@{\hspace{3mm}}c@{\hspace{6mm}}c@{\hspace{3mm}}c@{\hspace{12mm}}%
        c@{\hspace{3mm}}c@{\hspace{6mm}}c@{\hspace{3mm}}c@{\hspace{6mm}}c@{\hspace{3mm}}c@{}}
    \textbf{Criterion} &
        \textbf{weight} & \textbf{rank} & \textbf{weight} & \textbf{rank} &
        \textbf{weight} & \textbf{rank} & \textbf{weight} & \textbf{rank} &
        \textbf{weight} & \textbf{rank}\\
    \hline
    Typedness                  &   1 & 5.00 &   1 & 4.71 &   1 & 5.00 &    1 & 5.00 &   1 & 5.00\\
    Discernability             &   1 & 5.00 &   1 & 5.00 &   1 & 5.00 &    1 & 5.00 &   1 & 5.00\\
    Intuitiveness              &   1 & 4.14 &   1 & 3.29 &   1 & 4.13 &    1 & 3.63 &   1 & 4.10\\
    Invariance: Geometry       & 0.5 &    5 & 0.5 &    4 & 0.5 &    5 &  0.5 &    5 & 0.5 &    3\\
    Invariance: Colorimetry    & 0.5 &    3 & 0.5 &    3 & 0.5 &    5 &  0.5 &    5 & 0.5 &    4\\
    Composition: Separability  & 0.5 &    5 & 0.5 &    1 & 0.5 &    5 &  0.5 &    3 & 0.5 &    5\\
    Composition: Comparability &     &      &     &      &     &      &      &      &     &\\
    Attention: Importance      & 0.5 &    5 & 0.5 &    5 & 0.5 &    5 &  0.5 &    4 & 0.5 &    5\\
    Attention: Balance         & 0.5 &    5 & 0.5 &    2 & 0.5 &    5 &  0.5 &    5 & 0.5 &    5\\
    Searchability              & 0.5 &    5 & 0.5 &    1 & 0.5 &    5 &  0.5 &    5 & 0.5 &    5\\
    Learnability               & 0.5 &    5 & 0.5 &    2 & 0.5 &    5 &  0.5 &    3 & 0.5 &    4\\
    Memorability               & 0.5 &    4 & 0.5 &    1 & 0.5 &    5 &  0.5 &    1 & 0.5 &    3\\
    \hline
    \textbf{Total Weight \& Weighted Average}
                               &   7 & 4.66 & 7   & 3.21 & 7   & 4.80 & 7    & 4.16 & 7    & 4.44\\ 
\end{tabular}
\caption{A summary of the assessments of five glyph designs using the MCDA-aided scheme. \textbf{A} is is the original design by Maguire et al. \cite{Maguire:2012:TVCG}.
\textbf{B} is a variant of \textbf{A} configured based on several design options discussed in \cite{Maguire:2012:TVCG}.   
\textbf{C} is the original design by Legg et al. \cite{Legg:2012:CGF}.
\textbf{D} is a variant of \textbf{C} where two visual objects swap their positions and the central pictograms are replaced with abstract shapes discussed in \cite{Legg:2012:CGF}. 
\textbf{E} is a design by Chung et al. \cite{chung2015glyph}, which was partly based on \textbf{C}.}
\label{tab:RankingTypeD}
\vspace{-6mm}
\end{table*}

% ====================
\section{MCDA-aided Evaluation: Case Studies}
\label{sec:Evaluation}
%
We applied the MCDA scheme described in Section \ref{sec:Scheme} to a number of visual designs in the literature and their ``parodies'' (alternative designs). These case studies allowed us to test the scheme, identify ambiguous definitions, unbalanced categorization, and inappropriate thresholds, facilitating the improvement of the scheme.

Table \ref{tab:RankingTypeD} shows the summary of five case studies.  We have included a spreadsheet for recording the MCDA scores in the supplementary materials. 
Among these, \textbf{A} is the original design by Maguire et al. \cite{Maguire:2012:TVCG}, who also reported a number of alternative encoding methods for individual data variables. It is not difficult for us to configure ``parody'' designs based on these alternative encoding methods. Design \textbf{B} is one of such parody designs. In \textbf{B}, variable S6 is encoded using three colors instead of three shapes (e.g., dark gray instead white circle in Table \ref{tab:RankingTypeD}). Variable S2 is encoded using five colors for an outline instead of five metaphoric shapes (e.g., cyan square instead of an icon for material combination). S5 is encoded using seven basic shapes instead of seven countable metaphoric shapes as shown in the first row of Figure \ref{fig:Intuitiveness}. The     
From Table \ref{tab:RankingTypeD}, we can observe that the parody design has over-used colors and basic shapes, having a negative impact on several criteria, especially in terms of separability, attention balance, searchability, learnability, and memorability.

\textbf{C} is the original design by Legg et al. for supporting real-time event analysis during a rugby match. Design \textbf{D} is a parody design of \textbf{C} with two modifications: (i) replacing the silhouette pictogram at the center of the glyph with abstract shapes, and (ii) swapping the locations of the outcome circle (orange for unsuccessful) with the territory box (location A) in the glyphs representing Design \textbf{C} and Design \textbf{D}. As the data variable for event types has 16 key values, in comparison with pictograms in \textbf{C}, the abstract shapes in \textbf{D} reduces intuitiveness, learnability, and memorability. Meanwhile, the swapping changes the order between variables outcome and territory. The outcome circle is not as attentive as in \textbf{C} and becomes less separable from the abstract shapes in the center.

\textbf{E} is a design by Chung et al. \cite{chung2015glyph}, where a similar set of pictograms was used for event types. Unlike \textbf{C} that was designed for rugby coaches and sports analysts, \textbf{E} was designed for analysts only, and it contains several numerical variables resulting from video analysis, such as gain, tortuosity, and net lateral movement. The discernability of these variables are affected by size and color degeneration. Although visual designs for these variables were introduced for this application, the analysts who had the technical background could learn and memorize these with a bit extra effort.

We have included the documentation about the parody designs \textbf{B} and \textbf{D} in a zip file as part of the supplementary materials.

% Overall score
% flexibility existed in the proposed scheme, consistency should be maintained among evaluation for the same purpose.
% channel-wise or global evaluation


\section{Discussions and Conclusions}
\label{sec:Conclusions}
%
\textbf{Summary.} In this work, we have formulated a MCDA-aided assessment scheme for supporting glyph designs, and have tested the scheme on a range of glyph designs in the literature as well as their variants configured in the testing process. The scheme is built on the existing qualitative criteria in the literature, and enables a major step forward towards a more systematic, consistent, and semi-quantitative approach. Even when a glyph designer does not rank each criterion quantitatively, the scheme can serve as a reminder of the major considerations in encoding data variables and integrating different visual channels into a glyph representation.

\noindent \textbf{Limitations.}
(a) The scheme does not encode any domain-specific information. It is not in anyway a replacement for user-centered design and evaluation, especially when the target users are domain experts. Nevertheless, the scheme can help glyph designers phrase questions in seeking advice from domain experts and speed up the design process.
(b) The weights recommended in this work are specified based on our tests and analysis. On the one hand, there is a need for a set of weights that can be consistently applied to most (if not all) applications. On the other hands, weights are not ground truth, and their optimization needs the participation of the VIS community through many iterations. Thus the current recommendation is expected to be improved in the future. Meanwhile, it is important for glyph designers to ensure that the weights used in the assessment are transparent.
(c) For several criteria, e.g., separability, comparability, attention balance, and searchability, the specification of the five levels can be improved in the future based on new empirical research designed to obtain more precise measurements. We hope that the proposed scheme will stimulate such research.
(d) This proposed scheme does not cover the designs of 3D glyphs, glyph layout, interaction with glyphs, multi-scale glyphs, and so on. We hope that future research will extend the scheme.  

\noindent \textbf{Future Work.} In addition to aforementioned future research, the field of VIS can benefit from publicly-available glyph editing tools. The proposed scheme can potentially be integrated into such tools, facilitating human-centered and semi-automatic assessment of glyph designs during a design process.  


% Weights depending on the need ... the three multi-variate glyph purposes
% Individual channel or global rating
% I feel like maybe the two-phase usage of glyph can be used to decide the weighting of each criteria (purpose-based)
% Channel Independence \& Interference has influences on both channel reading and overall glyph effectiveness
% The weighting of different channels for a criterion item. Some are more important and some occur more frequently.

% channel searchability affect the cognitive load in the value comprehension phase
% Many criteria proposed in previous studies covers both intra- and inter-channel properties. Learnability, capacity, attention balance



% 
\section{Additions}

\paragraph{Asymptotic estimates}
%Asymptotic estimates 
of $\chi(d)$ for $d\rightarrow\infty$ can be obtained on the basis of the most efficient tiling of the plane and the most dense packing of points.

The densest packing of points on a disk of diameter $d$ with an unlimited increase in the number of vertices $q$ is ensured by their placement at the nodes of a unit hexagonal lattice. The number of points that fit on the disk and give an upper bound of $\chi$ is estimated from the ratio of the areas of the disk and the unit equilateral triangle: $\chi_{ub}\approx\frac{\pi}{\sqrt3}d^2$.

The most efficient tile shape is a regular hexagon with a side of $1/2$, and the best placement of the tile centers is a hexagonal lattice with a step $d+\sqrt3/2$. The required number of colors $\chi$ is obtained by the ratio of the area of the rhombus with side $d$ formed by the lattice to the area of the tile: $\chi_{lb}\approx\frac43d^2$.

Thus, %we get 
$\chi/d^2\in (4/3, \pi/\sqrt3)+o(1)\approx(1.3333, 1.8138)$.
In other words, for large $d$, %in the limit, 
the bounds on $d$ for a fixed $\chi$ differ by a factor of about $7/6$.

\paragraph{Hardware.}
Two computers worked non-stop for a year, checking more than 10\,000 graphs (approximately equal numbers of e- and w-graphs). Some of them required CNF files around 2 GB in size, some took more than a month (unsuccessful with one exception). 

Parameters of computers: i) Intel Core i5-9400F, 2.9 GHz, 6 cores/6 threads, 16 GB RAM; ii) AMD Ryzen5 4600H, 3 GHz, 6 cores/12 threads, 24 GB RAM. Their total performance turned out to be the same. % (it was compared on several graphs). 
Each graph was checked in a separate thread. The computation time is given in terms of the first computer (2 times faster per thread) without taking into account the assembly time of the CNF file.

\paragraph{Observations.}
The \texttt{glucose} and \texttt{kissat} solvers did not differ much in computation time, only on complex graphs requiring many hours, \texttt{kissat} was noticeably more efficient.

In our studies, e-graphs performed better than w-graphs. Apparently, the vertices outside the circle of radius $d$, which we call the \textit{border}, play an important role here. The width of the border (for a fixed $d$) definitely affects both the $k$-colorability and the computational rate, and some optimum is observed.

Only in two cases did the w-graphs prove to be better: i) when estimating $d_{ub}$ for $\chi=9$; ii) when searching for a minimal graph inside the island of stability. In \cite{pmag}, the 18-vertex w-graph allowed us to give a human verifiable proof for $\chi(d)=7$.

The most efficient tiling giving the maximum ratios $d^2/k$ is observed for $k=u^2$, $u\in\mathbb{Z}_{>0}$. In this case, hexagonal tiles of the same color are oriented towards each other with sides (see Fig. 3). A side effect is the inefficiency of subsequent $k$: in the range $k\in[1, 200]$ all $k= u^2+1$ and almost all $k= u^2+2$ give smaller values of $d(k)$, except for $k=6$ and 27. It can be predicted that on all $k= u^2+1$ there will be problems with obtaining tilings that provide $d(k)>d(k-1)$.

\paragraph{Open questions.}
The most obvious (and difficult) goal of subsequent attacks is to break the threshold $d_{lb}=\sqrt3$ for $\chi=10$ and 11, or to prove that this is impossible.
Here are some other questions:

Is it possible to get more efficient tilings by increasing the size of the repeating pattern, for example by using several different tile shapes for each color?
Is it possible to beat the radial coloring of the annulus for $k=6$?
How to explain the frequent repetition of the same values of $a$ and $b$ (for example, 91) in the table of record e-graphs?
What is the main reason for the noticeable difference in the slopes of extrapolation lines for different $k$? (Maybe it's the relative orientation of pseudo-tiles that are assembled from vertices of the same color?)
Is there a bias in the estimates used, and how can it be eliminated?

\paragraph{Conjectures.} 
We have made some progress in confirming Conjecture~\ref{cexoo} by reducing %the lower bound of the island of certainty 
$d_{min}$ from 1.285 to 1.085 for $\chi=7$.
Consistent with Conjecture~\ref{cwes}, we have discovered several new islands of certainty, and have optimistic forecasts for several others.
However, we propose the opposite
\begin{conj}
\label{cpar}
%For some $\chi\ge 7$ the island of certainty does not exist.
For some integers $k\ge 7$, there is no $d$ such that $\chi(d)=k$.
\end{conj}

It means that, for some $\chi\ge 7$, the island of certainty does not exist. In other words, as $d$ grows, the value of $\chi$ can change in non-unit steps. %This conjecture can also be reformulated in terms of redundant colors: some colors are redundant in the sense that their addition does not lead to an increase in $d$. 
Perhaps $k=10$ and 11 are the closest examples of redundant colors for which there is no island of certainty.

\paragraph{Thanks} to Aubrey de Grey for the corrections. Special thanks to Tom Sirgedas for his wonderful program.

In the endless ocean of chromatic numbers it is difficult to find an island of exact knowledge. But if you see several islands at once, this is an occasion to wonder if there is a mainland nearby.




%-------------------------------------------------------------------------
% bibtex
\bibliographystyle{eg-alpha-doi}
\bibliography{glyphs}

% biblatex with biber
% \printbibliography                

%-------------------------------------------------------------------------

%\section{Discussion}

In this section, we first summarize the lessons we learned during the development of \toolName, 
then we discuss the limitations of \toolName.

\vspace{-0.3cm}

\subsection{Lessons}
% We learned many valuable lessons from the design and evaluation of our visual design.

% \textbf{Fitting visualization into quantum computing.}
During the above evaluation processes, all participants gave highly positive feedback for \toolName.
Among all the responses, participants emphasized a strong need for visualization to fit into quantum computing regarding the complex quantum physics theory, non-transparency of quantum program process, and non-intuitive quantum computing properties (\textit{i.e.}, quantum entanglement and superposition).
The above challenges make it hard for novices and the general public to have a strong sense of quantum computing.
Thus, the quantum computing community urgently needs visualization to aid the transparency and interpretability of quantum computing with its scientific educational capability.


% \textbf{Lowering the learning curve of visual designs for quantum computing users.}
% During the co-design process, five domain experts strongly emphasized the difficulty for quantum computing users to learn and use cross-disciplinary visualizations. 
% All domain experts preferred concise and straightforward visual designs, which can really help them instead of those sophisticated visual solutions.
% Thus, \toolName\ received highly positive feedback concerning the simplicity and intuitiveness, giving it the potential to be widely spread by general users in quantum computing.
% Furthermore, we learned from the interviews that it is crucial to correlate the visual elements using quantum computing characteristics other than using many different and individual visual channels, such as the visual solution provided by \textit{Quirk}.


\vspace{-0.4cm}

\subsection{Limitations}

Our evaluation shows that \toolName\ can effectively facilitate quantum state observation. However, there are still some limitations.



\textbf{Limited support for quantum noise visualization.}
\toolName\ can effectively visualize various quantum states in situations where noise analysis is not required,
such as the design and debugging of quantum algorithms. 
% \yong{why? No noise in these applications??}
\modify{We do not consider the noise analysis of  \toolName\ because the design is built upon quantum simulators where the execution of quantum circuits is completely noise-free.}
% \modify{We do not consider the noise model of the simulators since too complex functionality might confuse the general users and learners under the education scenarios.}
% It would be better to support the comparison with noisy and noise-free quantum state results.
% \yong{Pls check my comments.}






\textbf{Scalability.} 
% As the current focus of \toolName\ is for 1-qubit and 2-qubit problems,
\toolName{} currently targets visualizing the quantum states of one or two qubits. Compared with Block Sphere, it can effectively visualize the quantum entanglement, a significant step towards effective qubit state visualization confirmed by the participants.
% Meanwhile, there are requests to prepare for the next generation of quantum computers with a scale difference of more than an order of magnitude. 
Also, E8 suggested enabling representation for more qubits by adding more triangles on top of \toolName. In the future, we will endeavor to extend the current design for more qubits.



\textbf{Time-consuming input for state vectors.}
The update of \toolName\ is driven by the inputted number of amplitudes, which requires users to input the real and imaginary parts of amplitudes manually.
Participants reported that it is inconvenient to type in the amplitude values.
However, due to our contribution to a design study, we plan to address this limitation in the future.
For example, as hinted by E4, converting from the popular visualization, Bloch Sphere, will also be helpful for users.




\end{document}

