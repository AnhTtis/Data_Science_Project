\begin{table*}[t]
\centering
\begin{tabular}{@{\hspace{46mm}}c@{\hspace{4mm}}c@{\hspace{8mm}}c@{\hspace{6mm}}c@{\hspace{6mm}}c@{}}
    \includegraphics[height=10mm]{Figures/MaguireOriginal} &
    \includegraphics[height=10mm]{Figures/MaguireParody} &
    \includegraphics[height=10mm]{Figures/LeggOriginal} &
    \includegraphics[height=10mm]{Figures/LeggParody} &
    \includegraphics[height=10mm]{Figures/ChungOriginal} \\
    \textbf{A:} Maguire et al. &
    \textbf{B:} Parody of \textbf{A} &
    \textbf{C:} Legg et al. &
    \textbf{D:} Parody of \textbf{C} &
    \textbf{E:} Chung et al.\\
\end{tabular}
\begin{tabular}{@{}p{46mm}@{\hspace{4mm}}%
        c@{\hspace{3mm}}c@{\hspace{6mm}}c@{\hspace{3mm}}c@{\hspace{12mm}}%
        c@{\hspace{3mm}}c@{\hspace{6mm}}c@{\hspace{3mm}}c@{\hspace{6mm}}c@{\hspace{3mm}}c@{}}
    \textbf{Criterion} &
        \textbf{weight} & \textbf{rank} & \textbf{weight} & \textbf{rank} &
        \textbf{weight} & \textbf{rank} & \textbf{weight} & \textbf{rank} &
        \textbf{weight} & \textbf{rank}\\
    \hline
    Typedness                  &   1 & 5.00 &   1 & 4.71 &   1 & 5.00 &    1 & 5.00 &   1 & 5.00\\
    Discernability             &   1 & 5.00 &   1 & 5.00 &   1 & 5.00 &    1 & 5.00 &   1 & 5.00\\
    Intuitiveness              &   1 & 4.14 &   1 & 3.29 &   1 & 4.13 &    1 & 3.63 &   1 & 4.10\\
    Invariance: Geometry       & 0.5 &    5 & 0.5 &    4 & 0.5 &    5 &  0.5 &    5 & 0.5 &    3\\
    Invariance: Colorimetry    & 0.5 &    3 & 0.5 &    3 & 0.5 &    5 &  0.5 &    5 & 0.5 &    4\\
    Composition: Separability  & 0.5 &    5 & 0.5 &    1 & 0.5 &    5 &  0.5 &    3 & 0.5 &    5\\
    Composition: Comparability &     &      &     &      &     &      &      &      &     &\\
    Attention: Importance      & 0.5 &    5 & 0.5 &    5 & 0.5 &    5 &  0.5 &    4 & 0.5 &    5\\
    Attention: Balance         & 0.5 &    5 & 0.5 &    2 & 0.5 &    5 &  0.5 &    5 & 0.5 &    5\\
    Searchability              & 0.5 &    5 & 0.5 &    1 & 0.5 &    5 &  0.5 &    5 & 0.5 &    5\\
    Learnability               & 0.5 &    5 & 0.5 &    2 & 0.5 &    5 &  0.5 &    3 & 0.5 &    4\\
    Memorability               & 0.5 &    4 & 0.5 &    1 & 0.5 &    5 &  0.5 &    1 & 0.5 &    3\\
    \hline
    \textbf{Total Weight \& Weighted Average}
                               &   7 & 4.66 & 7   & 3.21 & 7   & 4.80 & 7    & 4.16 & 7    & 4.44\\ 
\end{tabular}
\caption{A summary of the assessments of five glyph designs using the MCDA-aided scheme. \textbf{A} is is the original design by Maguire et al. \cite{Maguire:2012:TVCG}.
\textbf{B} is a variant of \textbf{A} configured based on several design options discussed in \cite{Maguire:2012:TVCG}.   
\textbf{C} is the original design by Legg et al. \cite{Legg:2012:CGF}.
\textbf{D} is a variant of \textbf{C} where two visual objects swap their positions and the central pictograms are replaced with abstract shapes discussed in \cite{Legg:2012:CGF}. 
\textbf{E} is a design by Chung et al. \cite{chung2015glyph}, which was partly based on \textbf{C}.}
\label{tab:RankingTypeD}
\vspace{-6mm}
\end{table*}

% ====================
\section{MCDA-aided Evaluation: Case Studies}
\label{sec:Evaluation}
%
We applied the MCDA scheme described in Section \ref{sec:Scheme} to a number of visual designs in the literature and their ``parodies'' (alternative designs). These case studies allowed us to test the scheme, identify ambiguous definitions, unbalanced categorization, and inappropriate thresholds, facilitating the improvement of the scheme.

Table \ref{tab:RankingTypeD} shows the summary of five case studies.  We have included a spreadsheet for recording the MCDA scores in the supplementary materials. 
Among these, \textbf{A} is the original design by Maguire et al. \cite{Maguire:2012:TVCG}, who also reported a number of alternative encoding methods for individual data variables. It is not difficult for us to configure ``parody'' designs based on these alternative encoding methods. Design \textbf{B} is one of such parody designs. In \textbf{B}, variable S6 is encoded using three colors instead of three shapes (e.g., dark gray instead white circle in Table \ref{tab:RankingTypeD}). Variable S2 is encoded using five colors for an outline instead of five metaphoric shapes (e.g., cyan square instead of an icon for material combination). S5 is encoded using seven basic shapes instead of seven countable metaphoric shapes as shown in the first row of Figure \ref{fig:Intuitiveness}. The     
From Table \ref{tab:RankingTypeD}, we can observe that the parody design has over-used colors and basic shapes, having a negative impact on several criteria, especially in terms of separability, attention balance, searchability, learnability, and memorability.

\textbf{C} is the original design by Legg et al. for supporting real-time event analysis during a rugby match. Design \textbf{D} is a parody design of \textbf{C} with two modifications: (i) replacing the silhouette pictogram at the center of the glyph with abstract shapes, and (ii) swapping the locations of the outcome circle (orange for unsuccessful) with the territory box (location A) in the glyphs representing Design \textbf{C} and Design \textbf{D}. As the data variable for event types has 16 key values, in comparison with pictograms in \textbf{C}, the abstract shapes in \textbf{D} reduces intuitiveness, learnability, and memorability. Meanwhile, the swapping changes the order between variables outcome and territory. The outcome circle is not as attentive as in \textbf{C} and becomes less separable from the abstract shapes in the center.

\textbf{E} is a design by Chung et al. \cite{chung2015glyph}, where a similar set of pictograms was used for event types. Unlike \textbf{C} that was designed for rugby coaches and sports analysts, \textbf{E} was designed for analysts only, and it contains several numerical variables resulting from video analysis, such as gain, tortuosity, and net lateral movement. The discernability of these variables are affected by size and color degeneration. Although visual designs for these variables were introduced for this application, the analysts who had the technical background could learn and memorize these with a bit extra effort.

We have included the documentation about the parody designs \textbf{B} and \textbf{D} in a zip file as part of the supplementary materials.

% Overall score
% flexibility existed in the proposed scheme, consistency should be maintained among evaluation for the same purpose.
% channel-wise or global evaluation

