\section{Related Work}
\label{sec:RalatedWork}
%
Glyph visualization has been applied in many fields, such as
biology, biomedical and medical research \cite{Maguire:2012:TVCG,somarakis2019imacyte,muller2014analysis,meuschke2017glyph,raidou2018bladder,lichtenberg2017concentric,kalamaras2022graph,khawatmi2021shapography,oeltze2008glyph,meyer2008glyph},
meteorology and environmental studies \cite{martin2008results,pilar2013representing,pfeiffer2021glyph,drocourt2011temporal,sanyal2010noodles},
human behavioral analysis \cite{el2016contovi,kovacevic2020glyph},
web and database searching \cite{chau2011visualizing,siva2014evaluation},
sports \cite{legg2016glyph,polk2014tennivis,wang2021tac,wu2021tacticflow,cava2013glyphs,parry2011hierarchical},
music and multimedia \cite{chan2009visualizing,lind2022visualizing,janicke2010soundriver,botchen2008action},
and business and industrial applications \cite{rees2020agentvis,surtola2005effect,suntinger2008event}.
A glyph object conveys multi-dimensional data in a concise way, which significantly reduces the perceptual and cognitive load for information comprehension. It has been widely adopted in applications that require a simultaneous view of multiple variables, including
the facilitation of visual search \cite{healey1999large,cai2015applying},
data comparison \cite{verma2004comparative,zhang2015glyph,meuschke2017glyph,koc2022peaglyph},
data ordering \cite{Chung:2016:CGF,miller2019evaluating},
and feature extraction \cite{keck2017towards}.
Glyphs have also been used in combination with spatial or temporal displays \cite{tominski20053d,ropinski2007surface,drocourt2011temporal,bleisch2017exploring,Legg:2012:CGF}
and other visualization methods \cite{lichtenberg2017concentric,kammer2020glyphboard,fernstad2020explore,borgo2012empirical}.
%
Algorithms for glyph placement \cite{ward2002taxonomy,lie2009critical,streeb2018design,rees2020agentvis,tong2016glyphlens,mcnabb2019multivariate,hlawitschka2007interactive}
and three-dimensional display \cite{lie2009critical,tong2016glyphlens,stevens2016hairy} were created to ensure effective presentation of glyph objects. Some studies developed techniques for specialized scenarios or requirements, such as
temporal summarization \cite{Duffy:2015:TVCG,el2016contovi,gerrits2017glyphs,tominski20053d,botchen2008action},
uncertainty depiction \cite{aigner2005planninglines,sanyal2010noodles,ribicic2012sketching,hlawatsch2011flow,wittenbrink1996glyphs},
visualizing data with special structure or inter-relations \cite{rees2020agentvis,dunne2013motif,cayli2013glyphlink,lee2021cluster,soares2020depicting,cao2011dicon,reda2019dynamic,kalamaras2022graph},
and display of trends and gradients of vectors and tensor fields \cite{dovey1995vector,wittenbrink1996glyphs,schultz2010superquadric,tong2016crystal,meuschke2017glyph,zhang2015glyph,gerrits2016glyphs,gerrits2017glyphs,hergl2019visualization,peeters2009fast,peng2011mesh,hashash2003glyph,hlawatsch2011flow}.
The recent appearance of accessible tools for glyph generation \cite{ribarsky1994glyphmaker,xia2018dataink,brehmer2021generative,ying2022metaglyph,cunha2018many} should lead to a broader application of this visualization technique.

With the seemingly endless design space and the wide variety of applications, guidance is required for glyph designers to evaluate and make selections among different designs. The development of a comprehensive list of criteria that allows systematic evaluations of glyph designs could provide designers with a framework to create designs of quality and to compare design options. The most common and straightforward evaluation is done via user studies \cite{aigner2005planninglines,surtola2005effect,weigle2005visualizing,chan2009visualizing,chau2011visualizing,siva2014evaluation,dunne2013motif}. A systematic review conducted by Fuchs et al.\cite{fuchs2016systematic} revealed that users' accuracy scores and task completion times are the most widely adopted measures to evaluate the effectiveness of visualisations. These measurements provide reliable statistical evidence for the evaluation; however, there are limitations. Firstly, it takes significant time and effort to design and carry out studies of users. This is especially true when multiple design options and objectives exist. Secondly, the performance outcomes are often the collective effect of various design factors, and they provide few indications of specific changes which could be made to improve the design. %\sout{The effects of individual factors can be finely evaluated by setting up experiments for each design factor while others are well controlled, which, however, further increases the costs of the evaluations.}

Normative rating is another way to evaluate visualizations, in which design qualities can be individually quantified. Subjective rating is commonly used in user studies to gauge design qualities from user experiences \cite{lee2003empirical,aigner2005planninglines,surtola2005effect,fuchs2013evaluation}. McDougall et al. \cite{mcdougall2000exploring} adopted subjective ratings to characterize cognitive features of icon designs and used the ratings in further analyses to investigate the correlations between the qualities. However, in most cases, the subjective rating is performed without a clear definition of each rating level. The only instruction given was how the two ends of the scale map to the extremities of the target quality, e.g. 1=very unfamiliar; 5=very familiar. The boundaries between the rating levels remained flexible to the raters. This casts doubts about the consistency of the scoring standards. It has also been shown that the users' preferences are not always consistent with the results from the statistical evidence \cite{fuchs2016systematic}.

Attempts have been made to develop standardized measures for qualities of glyph visualization. Garcia et al. \cite{garcia1994development} proposed a metric to evaluate glyph complexity.
Forsythe et al. \cite{forsythe2003measuring} proposed an automatic complexity measuring algorithm, aiming to remove subjective elements from the evaluation of complexity and to  prevent judgment bias. Up to now, however, the number of standardized measures of visual design qualities remains very limited.
%The subjective rating is still considered the method with more comprehensive coverage in quantifying visual design features \cite{ng2008visual}. 
%To extend the coverage, there is a need to identify more qualities of visual designs that the users may consider and provide clear criteria for the rating.

Fortunately, there have been abundant glyph design guidelines proposed in the literature. The guidelines concern various levels of glyph design: variable encoding, inter-channel interaction, and holistic glyph design. At the variable encoding level, Bertin \cite{bertin1983semiology} proposed basic semantic criteria for determining the suitability of channel encoding. Cleveland and McGill \cite{cleveland1984graphical} identified the accuracy of human perception in various visual variables and gave recommendations on the choices of visual channels for different tasks and purposes of visualization. Other considerations include the capacity, orderability, semantic closeness, visual pre-attentiveness, robustness, and normalizability of the channels \cite{lie2009critical,Legg:2012:CGF,ropinski2011survey,ward2008multivariate,chung2015glyph,yousef2001assessment}. At the levels of inter-channel interactions and holistic glyph design, guidelines were proposed for the integration and separability of channels, the balance of attention, searchability, visual hierarchy, and the learnability of glyph designs \cite{karve2007glyph,Maguire:2012:TVCG,lie2009critical,chung2015glyph}. Beyond the design for the glyph object \textit{per se}, advice was provided for the stages of data-mapping and glyph rendering \cite{ropinski2007surface,ward2008multivariate,meyer2008glyph,ropinski2008taxonomy,lie2009critical,ropinski2011survey}. A survey conducted by Borgo et al. \cite{Borgo:2013:STAR} collected glyph design guidelines and criteria from a wide range of previous studies. Consulting the three-stage glyph design framework by Lie et al. \cite{lie2009critical} and the perceptual-based glyph taxonomy by Ropinski et al. \cite{ropinski2008taxonomy,ropinski2011survey}, a set of fourteen design guidelines that aimed to provide comprehensive coverage of glyph-based visualization was proposed.

%It is manifested that although design criteria taken into account vary amongst studies, consensus on some universal criteria started to form. 
More recent studies\cite{Maguire:2012:TVCG,chung2015glyph,Borgo:2013:STAR} have integrated guidelines from various previous studies, and consensus on some universal design criteria has started to form. 
%As the outline of universal criteria becomes clearer and the coverage becomes comprehensive, 
With further development, a normative set of evaluation criteria might be expected to emerge through a process of collecting and comparing existing guidelines and selecting, modifying, or creating definitions of criteria to make them generally applicable to a vast range of glyph applications and data types. To make the criteria set succinct and equally weighted amongst different design aspects, higher orthogonality must be considered. Ideally, to make the evaluations consistent amongst raters, standards should be defined for each and every level of the evaluation criteria.



% \vfill\eject
% -\\
% \newpage
