\section{An MCDA Scheme for Glyph Design}
\label{sec:Scheme}
%
Our proposed scheme consists of 12 criteria, which are described in the following subsections. For each criterion, we provide a definition, a specification of five-level scores [1$\sim$5], a recommended weight, and a recommended mode of assessment (in black) and other possible modes (in grey).  

% --------------------
\vspace{-3mm}
\subsection{Typedness}
%
\begin{description}
%
\item[Definition:] This criterion assesses whether or not the visual channel (channels) of a data variable is (are) appropriated selected to match with the data type of the variable to be encoded. Such data types may include, but are not limited to: \emph{nominal}, \emph{ordinal}, \emph{interval}, \emph{ratio}, and \emph{directional}.
%
\item[Recommended Modes:] Type A (direct), Type D (aggregated).
%
\item[Recommended Weight:] Type A (unweighted), Type D [1.0].
\end{description}

The above definition was proposed by Chung et al. \cite{Chung:2016:CGF,Borgo:2013:STAR} for assessing each visual channel in a glyph based on Bertin's four kinds of perception (KOPs), namely, \emph{associative}, \emph{selective}, \emph{ordered}, and \emph{quantitative} perception.
Not all KOPs are applicable to all data types. For example, \emph{ordered}, and \emph{quantitative} perception is not required for a nominal variable. Hence, we consider only applicable KOPs (i.e., AKOPs in short).

% We reworded the original definition slightly to ensure that for Type A assessment, the criterion is applied to each data variable that may be encoded using one or more visual channels.
% While Bertin's judgments about the appropriateness are widely adopted, we recognize that these judgments do not cover all data types and all visual channels, while some judgments have been improved by subsequent research.
% In Appendix A, we discuss and exemplify the needs for designers to interpret Bertin's judgments in conjunction with the latest research and best practice.

We consider three levels of typedness, \emph{appropriate}, \emph{usable}, and \emph{inappropriate}. 
When a data variable $v_i$ is encoded using  just one visual channel $\gamma_i$, we may derive a score $s_i$ as follows:    

\begin{enumerate}
    \item[5.] The visual channel is appropriate for all AKOPs.
    \item[4.] The visual channel is appropriate for some AKOPs and is usable for the other AKOPs.
    \item[3.] The visual channel is usable for all AKOPs.
    \item[2.] The visual channel is inappropriate for some AKOPs.
    \item[1.] The visual channel is inappropriate for all AKOPs.
\end{enumerate}

When a data variable $v_i$ is encoded using two or more visual channels $\gamma_{i,1}, \gamma_{i,2}, \ldots$, we consider the best visual channel for each KOP. Hence the above scores are redefined as:

\begin{enumerate}
    \item[5.] For each AKOP, at least one visual channel is appropriate.
    \item[4.] For some AKOPs, the best visual channel is appropriate, while for the other AKOPs, the best is usable.
    \item[3.] For each AKOP, the best visual channel is usable.
    \item[2.] For some AKOPs, all visual channels are inappropriate.
    \item[1.] For each AKOP, all visual channels are inappropriate.
\end{enumerate}

% --------------------
\vspace{-3mm}
\subsection{Discernability}
\label{sec:Discernability}
%
\begin{description}
\item[Definition:] This criterion assesses whether the encoding of a data variable allows viewers to differentiate key values or major value ranges. The encoding may use one or more visual channels.
%
\item[Recommended Modes:] Type A (direct), Type D (aggregated).
%
\item[Recommended Weight:] Type A (unweighted), Type D [1.0].
\end{description}

% Information theory informs us that visualization in general provides a mean to observe more data variables at a lower resolution than statistics and database retrieval \cite{Chen:2016:TVCG}. Glyph-based visualization is intended for users to observe multivariate data in a small space and at low-resolution.

A data variable may has a number of valid values in an application context. For some numerical variables, the number can be huge. Glyph-based visualization is normally not intended for users to observe such data variables at a high-resolution.
It therefore helpful to define a set of key values (or key data ranges) for each data variable in an application context.  
%
Given $k$ key values (or key value ranges), viewers potentially need to differentiate $n = k(k-1)/2$ pairs of values (ranges). We define five levels based on $n$.

\begin{enumerate}
    \item[5.] All $n$ pairs of values (ranges) can be differentiated at ease.
    \item[4.] Most of the $n$ pairs of values (ranges) can be differentiated at ease, and the rest are perceptually differentiable. Numerically, ``most'' is defined as $[75, 100)$, i.e., $\geq 75\%, < 100\%$ 
    \item[3.] A large portion of the $n$ pairs of values (ranges) can be differentiated at ease, and the rest are perceptually differentiable. Numerically, ``large portion'' is defined as $[50, 75)$, i.e., $\geq 50\%, < 75\%$.
    \item[2.] Most of the $n$ pairs of values (ranges) are perceptually differentiable. The rest are not.
    As [4.], ``most'' is defined as $[75, 100)$.
    \item[1.] A significant portion of the $n$ pairs of values (ranges) are not perceptually differentiable. Numerically, ``significant portion'' is defined as [25, 100], i.e., $\geq 25\%$.
\end{enumerate}


% --------------------
\begin{figure}[ht]
  \centering
  \includegraphics[width=\linewidth]{Figures/Intuitiveness.pdf}
  \caption{Maguire et al. discussed the encoding of seven levels of material granularity in a biological application \cite{Maguire:2012:TVCG}. The intuitiveness of six encoding methods are assessed.}
  \label{fig:Intuitiveness}
  \vspace{-6mm}
\end{figure}

% --------------------
\subsection{Intuitiveness}
\label{sec:Intuitiveness}
%
\begin{description}
\item[Definition:] This criterion assesses how the encoding of a data variable is semantically-related to the knowledge of viewers, and how such a relation makes the encoding knowable to the viewers by intuition.
%
\item[Recommended Modes:] Type A (direct), Type D (aggregated).
%
\item[Recommended Weight:] Type A (unweighted), Type D [1.0].
\end{description}

Given a data variable $d_i$ in an application domain, there may be a \emph{Domain-specific Convention} (DC) for encoding the data variable visually. Such encoding captures the semantic knowledge of the discipline, hence facilitating intuitiveness. Given a design option for encoding a data variable $d_i$, there are three basic scenarios of DC: no existing DC (noDC), consistent with the existing DC (cnDC), and inconsistent with the existing DC (inDC).

The encoding may also introduce one or more \emph{Additional Visual Metaphors} (AM) for enhancing the existing DC or filling the gap when there is no existing DC. An appropriate metaphor can make effective connection between the semantics of $d_i$ and viewers' knowledge to facilitate intuitiveness. Hence, we consider four basic scenarios of AM: no additional metaphor (noAM), appropriate metaphor (apAM), adequate metaphor (okAM), and inappropriate metaphor (inAM). 
%
Combining the two sets of scenarios results in 12 combinations. We define five levels as:
%
\begin{enumerate}
    \item[5.] cnDC-apAM or noDC-apAM.
    \item[4.] cnDC-noAM, cnDC-okAM, or noDC-okAM. 
    \item[3.] noDC-noAM.
    \item[2.] cnDC-inAM, inDC-okAM, inDC-apAM.
    \item[1.] noDC-inAM, inDC-noAM, inDC-inAM.
\end{enumerate}
%
\noindent Here we do not assume that DC is always better than AM, or vice versa. Nevertheless, the criterion awards the creativity of visual design by rating noDC-apAM slightly higher than cnDC-noAM.

% --------------------
\vspace{-3mm}
\subsection{Invariance: Geometry}
%
\begin{description}
\item[Definition:] This criterion assesses the undesirable impact of the geometrical variations in displaying a glyph upon its visual quality. The primary geometrical variations are size variations and other considerations include minor variations of aspect radio, projection angle, and rotation.  
%
\item[Recommended Modes:] \textcolor{gray}{Type A,} Type D (direct).
%
\item[Recommended Weight:] Type D [0.5].
\end{description}

Maguire et al. \cite{Maguire:2012:TVCG} conducted scalability tests in their glyph design process. Legg et al. \cite{legg2016glyph} studied the geometric and color degeneration of glyphs. To reduce the complexity of assessing these two types of degeneration (i.e., Design Consideration 2), we have two sub-criteria, geometry (this section) and colorimetry (next section), under the common quality term ``Invariance''.

\begin{figure}[ht]
    \centering
    \includegraphics[width=\linewidth]{Figures/Invariance.pdf}
    \caption{The view area depends on two main factors, visual field and viewing distance. Five scaling factors are applied to a circular glyph \cite{Duffy:2015:TVCG} and a rectangular glyph \cite{Maguire:2012:TVCG}.}
    \label{fig:Invariance}
    \vspace{-6mm}
\end{figure}

In the fields of vision science and human-computer interaction, there is a wealth of research on viewing distances and icon sizes. We can build on these findings to consider the geometric scalability of glyphs. As illustrated in Figure \ref{fig:Invariance}, the viewing area can be approximated by a cone that has a visual field VF degrees and a viewing distance VD. The central VF of human eyes is $1.5^\circ \sim 5^\circ$. Typical viewing distances for monitors, laptops, and phones are  $50\sim100$~cm, $40\sim76$~cm, and $41\sim46$~cm respectively. We can estimate the viewing area as a circle with diameter $L_\text{diam}$ or its inner square with edge length $L_\text{edge}$ as follows:
\[
    L_\text{diam} = 2 \cdot \text{\text{VD}} \cdot \tan(\text{VF}^\circ), \qquad L_\text{edge} = \frac{L_\text{diam}}{\sqrt{2}}
\]
\noindent In this work, we use $\text{VF}=5^\circ$ \cite{wandell1995foundations,strasburger2011peripheral} and $\text{VD}=50$~cm (widely-recommended viewing distance) as the baseline measures. The baseline of a circular viewing area has a diameter of 4.37~cm and a square with an edge length of 3.09~cm as illustrated in Figure \ref{fig:Invariance}.
We can define the five levels based on scaling factors $1/5,\,2/5,\, 3/5,\, 4/5,\, 5/5$ and the absence of any degradation of discernability of any visual channel (i.e., invariance) at each scale.  
%
\begin{enumerate}
    \item[5.] Discernability is invariant at the $1/5$ scale.
    \item[4.] Discernability is invariant at the $2/5$ scale but variant at $1/5$.% 
    \item[3.] Discernability is invariant at the $3/5$ scale but variant at $2/5$.%
    \item[2.] Discernability is invariant at the $4/5$ scale but variant at $3/5$.%.
    \item[1.] Discernability is variant at the $4/5$ scale.%.
\end{enumerate}
%
As exemplified by Maguire et al. \cite{Maguire:2012:TVCG}, such invariance tests can easily be carried out by glyph designers. Since one will likely apply a scaling factor to a whole glyph, we recommend to assess this criterion with a Type D score directly. 

% --------------------
\vspace{-3mm}
\subsection{Invariance: Colorimetry}
%
\begin{description}
\item[Definition:] This criterion assesses the undesirable impact of the non-geometrical appearance variations in displaying a glyph upon its visual quality. The main considerations are chromatic and achromatic variations of colors, which may be caused by limitations of a display device and/or environmental lighting conditions.
%
\item[Recommended Modes:] \textcolor{gray}{Type A,} Type D (direct).
%
\item[Recommended Weight:] Type D [0.5].
\end{description}

Chromatic and achromatic variations of colors are commonly caused by undesirable environmental lighting (e.g., reflection) and occasionally by display devices (e.g., in an energy saving mode). Such variations can be approximated by using a function for varying the contrast and brightness of the imagery representation of a glyph. Consider the RGB representation of a pixel, such that $r, g, b \in [0, 255]$. One commonly-used function is:
\[
    x' = \max\biggl(255, \min\bigl(0, \frac{259(\kappa_\text{ctr} + 255)}{255(259-\kappa_\text{ctr})} (x - 128) + 128 + \kappa_\text{brt} \bigr) \biggr)
\]
\noindent where $\kappa_\text{ctr} \in [-255, 255]$ specifies the increment (positive) or decrement (negative) of contrast, while $\kappa_\text{brt} \in [-255, 255]$ specifies the increment or decrement of brightness. We recommend to assess this criterion with a Type D score. We define the five levels based on 10\%, 20\%, 30\%, and 40\% of variations, which are translated to $\pm 25.5, \pm 51, \pm 76.5,$ and $\pm 102$ for $\kappa_\text{ctr}$ and $\kappa_\text{brt}$. 

\begin{enumerate}
    \item[5.] Discernability is invariant with $\kappa_\text{ctr} = \pm 102 \land \kappa_\text{brt} = \pm 102$.
    \item[4.] Discernability is invariant when $\kappa_\text{ctr} = \pm 76.5 \land \kappa_\text{brt} = \pm 76.5$, but variant when $\kappa_\text{ctr} = \pm 102 \lor \kappa_\text{brt} = \pm 102$.% 
    \item[3.] Discernability is invariant when $\kappa_\text{ctr} = \pm 51 \land \kappa_\text{brt} = \pm 51$, but variant when $\kappa_\text{ctr} = \pm 76.5 \lor \kappa_\text{brt} = \pm 76.5$.%
    \item[2.] Discernability is invariant when $\kappa_\text{ctr} = \pm 25.5 \land \kappa_\text{brt} = \pm 25.5$, but variant when $\kappa_\text{ctr} = \pm 51 \lor \kappa_\text{brt} = \pm 51$.%
    \item[1.] Discernability is variant when $\kappa_\text{ctr} = \pm 25.5 \lor \kappa_\text{brt} = \pm 25.5$.%
\end{enumerate}

% --------------------
\vspace{-3mm}
\subsection{Composition: Separability}
\label{sec:Separability}
%
\begin{description}
\item[Definition:] This criterion assesses the undesirable interference among visual channels in a glyph, which would affect the perception of some visual channels.
%
\item[Recommended Modes:] \textcolor{gray}{Type B, Type C,} Type D (direct).
%
\item[Recommended Weight:] Type D [0.5].
\end{description}

\begin{figure*}[ht]
    \centering
    \includegraphics[width=\linewidth]{Figures/Separability.pdf}
    \caption{Example designs of bi-variate glyphs in four severity categories. We can observe that introducing boundaries to separate different visual channels (b1-d1) or reference lines (c2, d2) can reduce the severity of the interference between two visual channels.}
    \label{fig:Separability}
    \vspace{-6mm}
\end{figure*}

This criterion is based on the discourse by Maguire et al. \cite{Maguire:2012:TVCG} and hung et al. \cite{Chung:2016:CGF} on the interference among visual channels in glyph design. Such interference is often caused by visual channels that are integrated into the same visual object or hosted by closely-placed visual objects. Visual channels of similar types (e.g., brightness, length) are more likely to suffer from interference as exemplified in Figure \ref{fig:Separability}. 
%
% Maguire et al. \cite{Maguire:2012:TVCG} discussed the need to consider the interference among visual channels in glyph design and highlighted the relevant experimental findings and theoretical discourse in psychology
% (e.g., \cite{Shepard:1964:JMP,Handelt:1972,Krantz:1975:JMP,APP,Burns:1978:PLM}).
% Chung et al. \cite{Chung:2016:CGF} defined ``separability'' as a design criterion for minimizing such interference, which is often caused by visual channels that are integrated into the same visual object or hosted by closely-placed visual objects. Visual channels of similar types (e.g., brightness, length) are more likely to suffer from interference as exemplified in Figure \ref{fig:Separability}.
%
In general, a glyph occupies a small display space, and interference is often unavoidable.
% For example, Albers Szafir \cite{Szafir:2018:TVCG} showed that the variation of line thickness would affect the consistency of color perception.
Such minor interference would normally be considered to be acceptable in glyph design. In addition, for some applications, it is desirable to make some visual channels comparable, which conflicts with the desire for separating visual channels by using different types of visual channels or moving them away from each other. To balance these two sides of the same ``Composition'' coin, we have introduced ``comparability'' as a different criterion.

% \textcolor{blue}{TEMP: "When two or more visual channels are integrated into a compound channel, such as combining intensity, hue and saturation into a colour channel, the interference between different primitive channels should be minimised."}

Although many empirical studies have evaluated the interference among different visual channels, there is not yet a standard metric for measuring the severity of such interference. While such a standard metric will hopefully be defined in future research, we hereby use a relatively subjective measure to categorize interference in glyph design as \emph{major}, \emph{medium}, \emph{minor}, and \emph{none}, which correspond to interference scores 1, 0.1, 0.01, and 0 respectively. Figure \ref{fig:Separability} shows two examples for each severity category.  

Given a total of $n$ visual channels $\lambda_1, \lambda_2, \ldots \lambda_n$ in a glyph, each channel $\lambda_i$ can potentially be influenced by $n-1$ other channels. Let $s_\text{int}(\lambda_i, \lambda_j)$ be the score of the interference received by $\lambda_i$ from $\lambda_j$, and $S_\text{int}(\lambda_i)$ be the aggregated score for $\lambda_i$, which is defined as:
\[
    S_\text{int}(\lambda_i) = \max \bigl\{
    s_\text{int}(\lambda_i, \lambda_j) \; | \; j=1,2, \ldots, n \;\land\; j \neq i \bigr\}
\]
\noindent We can observe that $s_\text{int}(\lambda_i, \lambda_j)$ is a pairwise score for Type B assessment, and $S_\text{int}(\lambda_i)$ is a score for Type C assessment.
From these scores, we can obtain two Type D scores, a mean score $\text{avg}_\text{int}$ and a maximum score $\max_\text{int}$:
%
\[
    \text{avg}_\text{int} = \frac{1}{n} \sum_{i=1}^n S_\text{int}(\lambda_i) \qquad
    \text{max}_\text{int} = \max_{i=1}^n S_\text{int}(\lambda_i)
\]
%
\noindent We can therefore define the five levels as:
%
\begin{enumerate}
    \item[5.] $0.0 \leq \max_\text{int} < 0.1$: only minor interference.%
    \item[4.] $0.1 \leq \max_\text{int} < 1.0$: some medium but no major interference.% 
    \item[3.] $\max_\text{int} = 1.0 \;\land\; \text{avg}_\text{int} < 1/8$: some major interference, and less than $1/8$ of visual channels are affected. 
    \item[2.] $\max_\text{int} = 1.0 \;\land\; 1/8 \leq \text{avg}_\text{int} < 1/4$: some major interference, and between $1/8$ and $1/4$ of visual channels are affected.
    \item[1.] $\max_\text{int} = 1.0 \;\land\; \text{avg}_\text{int} \geq 1/4$: some major interference, and more than $1/4$ of visual channels are affected.
\end{enumerate}
%
\noindent As obtaining $n(n-1)$ pairwise scores $s_\text{int}(\lambda_i, \lambda_j)$ will be time-consuming, we recommend to evaluate each glyph design holistically. The above specification of the five levels facilities the option of obtaining a Type D score directly.

% --------------------
\vspace{-3mm}
\subsection{Composition: Comparability}
\label{sec:Comparability}
%
\begin{description}
\item[Definition:] This criterion assesses the desirable level of support featured in a glyph design for enabling required comparative tasks such as determining the order of two related data variables ($v_i$ vs. $v_j$), estimating their addition $(v_i+v_j)$, their difference $|v_i - v_j|$, or their ratio $v_i / v_j$. 
%
\item[Recommended Modes:] \textcolor{gray}{Type B, Type C,} Type D (direct).
%
\item[Recommended Weight:] Type D [0.5].
\end{description}

As mentioned in Section \ref{sec:Separability}, we introduce this criterion to complement ``Separability''. This criterion was not discussed explicitly in the existing surveys (e.g., \cite{Borgo:2013:STAR}), possibly because one only considers this when there is a need to compare some data variables within a glyph. Nevertheless, several glyph designs in the literature addressed the need to compare some data variables within a glyph representation. For example, Duffy et al. \cite{Duffy:2015:TVCG} presented a glyph design representing some 20 data variables, among which three related distance variables were to be compared in terms of their lengths and relative ratios. Duffy et al. encoded these variables using three nested arcs, facilitating easy comparison.

When one needs to compare two visual channels that encode two different data variables, the following obstacles will likely hinder comparative tasks:
%
\begin{itemize}
    \item \emph{Major obstacle} -- Two visual channels are of very different types, e.g., area vs. brightness as shown in Figure \ref{fig:Separability} (d1).%
    \item \emph{Major obstacle} -- Two visual channels are of the same type but with inconsistent encoding schemes, e.g., two color channels, one with a divergence colormap and another with a sequential colormap, or two length channels, one uses 20 pixels to encode the range [0, 10] and another uses 40 pixels for the same range.%
    \item \emph{Medium obstacle} -- Two visual channels are of the same type but with features that affect consistent perception, e.g., same brightness range but with different hues in Figure \ref{fig:Separability} (c1), or same length encoding but in different orientation in Figure \ref{fig:Separability} (b1).%
    \item \emph{Medium obstacle} -- Two visual channels do not have any common reference point, e.g., two length channels without any reference lines, unlike Figure \ref{fig:Separability} (c2, d2).%
    \item \emph{Minor obstacle} -- Two visual channels are placed far away from each other. where the word ``far'' is in the context of a glyph. 
\end{itemize}
 
It is necessary to note, when a data variable is encoded using multiple visual channels, as long as one of the visual channels is comparable, one can omit the consideration of other visual channels. For example, if data variable $d_1$ is encoded using both length and a continuous colormap, and data variable $d_2$ is encoded using length only, we only need to consider $d_1$-length vs. $d_2$-length. We can define the five levels as:
%
\begin{enumerate}
    \item[5.] \emph{Major}: none; \emph{Medium}: none; \emph{Minor}: none.%
    \item[4.] \emph{Major}: none; \emph{Medium}: none; \emph{Minor}: one or a few.% 
    \item[3.] \emph{Major}: none; \emph{Medium}: one; \emph{Minor}: more than a few.% 
    \item[2.] \emph{Major}: none; \emph{Medium}: more than one; \emph{Minor}: any.%
    \item[1.] \emph{Major}: at least one. \emph{Medium}: any; \emph{Minor}: any.
\end{enumerate}
%
\noindent where we recommend that the term ``a few'' is defined as less than 10\% of all pairwise comparisons, and ``more than a few'' is 10\% or more but less than 50\%.
We also recommend to evaluate this criterion holistically by obtaining a Type D score directly. When there is no need for comparing any data variables within a glyph, we recommend to set the weight for this criterion to zero.

% --------------------
\vspace{-3mm}
\subsection{Attention: Importance}
\label{sec:Importance}
%
\begin{description}
\item[Definition:] This criterion assesses the desirable level of support featured in a glyph design for encoding data variables according to their importance, e.g., by allocating more pre-attentive visual channels or higher encoding bandwidth to more important data variables. 
%
\item[Recommended Modes:] \textcolor{gray}{Type B, Type C,} Type D (direct).
%
\item[Recommended Weight:] Type D [0.5].
\end{description}

Ropinski and Preim \cite{ropinski2011survey} instigated the benefit of encoding data variables according to their importance in the context of an application. Maguire et al. \cite{Maguire:2012:TVCG} identified several factors that may influence an important ranking: the level of a variable in a taxonomy, the usage of a variable in users' tasks, and so on.
The factors that help a visual channel receive more attention include the pop-out effect, the hierarchy effect, the size of the visual objects hosting the visual channel, and so on. Furthermore, when a data variable is encoded using multiple visual channels, it will likely receive more attention.
% The factors that help a visual channel receive more attention include the pop-out effect studied extensively in psychology \cite{}, the hierarchy effect studied by Navon \cite{} and Love \cite{}, size of the visual object hosting the visual channel, and so on. Furthermore, when a data variable is encoded using multiple visual channels, it will likely receive more attention.  
Maguire et al. presented a method to bring the rankings of variables and visual channels together in glyph design.    

Chung et al. \cite{Chung:2016:CGF} defined ``attention balance'' as a criterion for matching the levels of attention that visual channels may receive with the importance levels of the variables. The term ``attention balance'' implicitly indicates two sides of the same coin.
% i.e., a good design needs to enable more important variables to receive more attention while making sure that no variable is disadvantaged disproportionately.
Following the third design consideration, we make the two sides of the same coin as two sub-criteria. This criterion focuses on ``importance'', and the next criterion on ``balance''. When there is no importance ranking of the data variables within a glyph, we recommend to set the weight for the importance criterion to zero. Nevertheless, the balance criterion will always be assessed. 

\begin{figure}[t]
    \centering
    \includegraphics[width=\linewidth]{Figures/Importance.pdf}
    \caption{Five levels of the ``Attention: Importance'' criterion are defined based on different amount of correlation between the importance ranks of data variables (encoded using $y$-position) and their attention ranks (encoded using number, color, edge thickness, and edge darkness). Four examples are shown at each level.}
    \label{fig:Importance}
    \vspace{-6mm}
\end{figure}

Consider a list of data variables, $v_1, v_2, \ldots, v_n$. Each variable $v_i$ is associated with two ranking values: $\iota_i$ for the importance of $v_i$ and $\alpha_i$ for the attention of $v_i$ through its visual encoding.
We have $\iota_i, \alpha_i \in [1, n]$.
The highest ranks of importance and attention are represented by 1 and the lowest by $n$. If two data variables are ranked same for importance (or attention), their $\iota$ (or $\alpha$) values are the same. We can compute the Pearson correlation coefficient:
\[
    C = \frac{\sum (\iota_i - \overline{\iota})(\alpha_i - \overline{\alpha}) }{\sqrt{\sum (\iota_i - \overline{\iota})^2 \sum (\alpha_i - \overline{\alpha})^2}}
\]
\noindent where $\overline{\iota}$ and $\overline{\alpha}$ are mean ranking values of importance and attention respectively. The five levels are defined as:
%
\begin{enumerate}
    \item[5.] The correlation coefficient $C > 0.95$.%
    \item[4.] The correlation coefficient $0.85 < C \leq 0.95$.% 
    \item[3.] The correlation coefficient $0.5 < C \leq 0.85$.% 
    \item[2.] The correlation coefficient $0 < C \leq 0.5$.%
    \item[1.] The correlation coefficient $C \leq 0$.
\end{enumerate}

Figure \ref{fig:Importance} shows four examples at each level. We can observe minor misalignment of the ordering has limited impact on the correlation coefficient, which is suitable for the uncertainty in ranking importance and attention, because of the subjective nature of importance ranking and the lack of experimental measures in attention ranking.
In theory, one may first obtain Type B or Type C scores. In practice, it is more efficient to compute a Type D score directly.  

% --------------------
\begin{figure*}[ht]
    \centering
    \includegraphics[width=160mm]{Figures/Imbalance.pdf}
    \caption{Examples of data variables that are ``overshadowed'' by other data variables, and may easily overlooked (i.e., inattentional blindness). When one is asked to point out which variables have been changed between the two glyphs, the arrow directions in (a), and the triangular marker and the length of the dotted arc may receive significant less attention than other variables. }
    \label{fig:Imbalance}
    \vspace{-6mm}
\end{figure*}
% --------------------
\vspace{-3mm}
\subsection{Attention: Balance}
%
\begin{description}
\item[Definition:] This criterion assesses the undesirable disadvantages that some data variables may suffer, which may make such data variables easily overlooked or difficult to perceive. 
%
\item[Recommended Modes:] \textcolor{gray}{Type B, Type C,} Type D (direct).
%
\item[Recommended Weight:] Type D [0.5].
\end{description}
%
As described in Section \ref{sec:Importance}, this criterion is assessing the opposite side of the same coin of ``Attention''. When attention is prioritized for the importance of data variables, it is necessary to ensure that no data variable may be seriously disadvantaged or suffer from inattentional blindness, which is a phenomenon studied extensively in psychology. In the context of glyph design, inattentional blindness primarily occurs when some data variables attract significantly more attention and thus limit cognitive resource, causing the variations of some other data variables to go unnoticed.

Here we assume that (i) the variables concerned are discernable (see Section \ref{sec:Discernability}), and (ii) the importance-based ordering is correct (see Section \ref{sec:Importance}). The blindness is caused by imbalanced allocation of cognitive resource for noticing variations. The factors of imbalance may include (a) peripheral location, (b) unsaturated color, (c) minor shape variation, (d) small object, (e) variation demanding high cognitive load, and so on. As illustrated in Figure \ref{fig:Imbalance}, the blindness is usually due to the co-existence of two or more such factors.

Ideally, empirical research in the future will provide us with methods for identifying visual encoding that may suffer from inattentional blindness. Until then, one may identify such a variable for juxtaposing two glyphs (of the same design) where all data variables have some variations. As illustrated in Figure \ref{fig:Imbalance}, one can observe those variables receiving weak attention, i.e., their variations are easily overshadowed by other variables.

We recommend evaluating this criterion holistically by obtaining a Type D score directly. We define the five levels based on the number of data variables that receive weak attention and may cause inattentional blindness:
%
\begin{enumerate}
    \item[5.] No data variable receives weak attention.%
    \item[4.] One data variable receives weak attention.% 
    \item[3.] Two data variables receive weak attention.% 
    \item[2.] Three data variables receive weak attention.%
    \item[1.] More than three data variables receive weak attention.
\end{enumerate}

% --------------------
\vspace{-3mm}
\subsection{Searchability}
%
\begin{description}
\item[Definition:] This criterion assesses the desirable properties that the visual channel(s) for each data variable can be recognized easily among others after a viewer has learned and remembered the encoding scheme. 
%
\item[Recommended Modes:] \textcolor{gray}{Type C,} Type D (direct).
%
\item[Recommended Weight:] Type D [0.5].
\end{description}
%
Chung et al. \cite{Chung:2016:CGF} defined ``Searchability'' as the level of ease when one needs to identify a visual channel associated to a specific data variable. Here we assume that the user has already learned and remembered such an association semantically. This allows us to consider searchability independently of whether the encoding is easy to learn or remember. As illustrated in Figure \ref{fig:Searchability}, when many variables have similar visual encoding except their positions in a glyph, they can be difficult to find, despite that their encoding following Bertin's rules and receiving adequate attention.

\begin{figure}[t]
    \centering
    \includegraphics[width=\linewidth]{Figures/Searchability.pdf}
    \caption{Three visual designs can encode (a) 10 numerical variables, (b) 9 numerical variables, and (b) 36 Boolean variables respectively. For some visual channels in these three glyphs, it is not easy to relate a visual channel to a specific variable, though they work fine in normal large plots. Problems can be alleviated if the number of similar visual channels are reduced as in (d) and (e).}
    \label{fig:Searchability}
    \vspace{-6mm}
\end{figure}

Let us consider a meta-variable $v_\text{meta} = r(d, \lambda)$ for representing the association between a data variable $d$ and a visual channel $\lambda$, such that $v_\text{meta} = \textbf{true}$ if an association exists and \textbf{false} otherwise. In each of the three examples in Figure \ref{fig:Searchability}, $v_\text{meta}$ is encoded primarily using a spacial location, which may also be searched through a related visual cue such as angle or count. In psychology, previous experiments have shown that the accuracy and response time of visual search and counting are affected by the number of objects and some other factors.
%
For example, In Figure \ref{fig:Searchability}(a), the first and last bars are easier to search than other eight. 
If there were a smaller number of bars, e.g., a group of five bars in Figure \ref{fig:Searchability}(d,e), symmetry can aid the visual search.
In Figure \ref{fig:Searchability}(b), for some numbers, the bars can be placed along lines from the center to the vertices of a square, hexagon, octagon, or even dodecagon, with one vertex at the 12'o clock direction. These placements can also aid visual search. One can easily use multiple encoding by adding additional visual cues (e.g., colors and symbols) to improve the searchability. For example, in Figure \ref{fig:Searchability}(c), one could replace each cross symbol with an object that has the shape defining the column and the color defining the row. In Figure \ref{fig:Searchability}(e), the 10 bars are divided into two groups using two colors. 
Ideally, the assessment of this criterion could be based on the measurement of accuracy, response time, and/or cognitive load in visual search. The previous empirical research in psychology has not provided standardized measurements that can be used for assessing glyph designs. We anticipate that this will be obtained in future empirical studies in visualization. For the time being, we coarsely define three levels of cognitive effort in visual search as:

\begin{itemize}
    \item \emph{Low cognitive load} -- It requires almost no effort to find a specific variable, e.g., the first or last bar in Figure \ref{fig:Searchability}(a).
    \item \emph{Medium cognitive load} -- It requires a small and undemanding amount of counting or reasoning effort. One normally feels such an effort, but is fairly sure about the search results. For example, the 2nd and 3rd bars in Figure \ref{fig:Searchability}(a) fall into this category.
    \item \emph{High cognitive load} -- It requires an amount of searching effort that one feels bothersome or burdensome, while one may hesitate about the correctness of the search. For example, the 4th through the 7th bar in Figure \ref{fig:Searchability}(a) fall into this category.
\end{itemize}
%
Based on these three categories, we can define the five levels as:

\begin{enumerate}
    \item[5.] \emph{High}: none; \emph{Medium}: none; \emph{Low} all.%
    \item[4.] \emph{High}: none; \emph{Medium}: one or a few; \emph{Low}: most.% 
    \item[3.] \emph{High}: none; \emph{Medium}: more than a few; \emph{Low}: more than half.% 
    \item[2.] \emph{High}: one or a few; \emph{Medium}: any; \emph{Low}: any.%
    \item[1.] \emph{High}: more than a few; \emph{Medium}: any; \emph{Low}: any.
\end{enumerate}
%
\noindent where we recommend that the term “a few” is defined as fewer than 10\% of all data variables. We also recommend to evaluate this criterion holistically by obtaining a Type D score directly.

% --------------------
\vspace{-3mm}
\subsection{Learnability}
\label{sec:Learnability}
%
\begin{description}
\item[Definition:] This criterion assesses the desirable property that the whole encoding scheme of a glyph is easy to explain and to learn. 
%
\item[Recommended Modes:] Type D (direct).
%
\item[Recommended Weight:] Type D [0.5].
\end{description}

Chung et al. \cite{Chung:2016:CGF} defined learnability as the level of ease in learning and remembering a visual encoding scheme. As learning and memorizing are often studied separately in psychology, we split ``learnability'' and ``memorability'' into two related sub-criteria. For example, if the glyph in Figure \ref{fig:Searchability}(a) encodes the average marks of 10 courses (unnumbered), the scheme is easy to learn but difficult to remember. If it encodes the attendance of 10 weeks in an academic term, it is both easy to learn and remember.   
%
While both learnability and memorability can benefit from the intuitive encoding of individual visual channels (see Section \ref{sec:Intuitiveness}), there are many holistic factors, such as the total number of data variables, their relative positions, their semantic similarity and difference, and so on.
In order not to overload a criterion (Design Consideration 2), we let intuitiveness focus on individual visual channels through Type A scores, while focusing learnability and the memorability on two holistic sub-criteria through Type D scores.

Learnability is user-dependent. In order to focus on glyph designs, we assume that that the target users have already had the knowledge about the data variables to be encoded, e.g.,
% the categorization of material perturbation, separation, amplification, combination, and collection in the context of \emph{in silico} biological data manipulation \cite{Maguire:2012:TVCG}
terms such as scrum, ruck, lineout, maul, and try in the context of rugby sports \cite{Legg:2012:CGF}.
In many applications, glyph-based visualization is designed for domain experts. Because of the assumption of domain knowledge, controlled or semi-controlled empirical studies are typically unsuitable for assessing this criterion since the lack of domain knowledge of the experiment participants would invalidate the experiment results.    
Meanwhile, learnability should be assessed in relation to the baseline that the target users have little knowledge about the visual design concerned. For example, domain experts often contribute directly ideas of visual encoding in a design process. Naturally, these domain experts have already ``learned'' the designs to be evaluated and their knowledge would bias the assessment.      

For a typical target user with adequate domain knowledge but little knowledge about the visual design to be evaluated, we defined the five levels based on \emph{learning time}, \emph{learning mode} (i.e., levels of training engagement), and the effort required for \emph{repeated learning} after a short period of not using the glyphs. The five levels are:
%
\begin{enumerate}
    \item[5.] \emph{Learning time}: $<0.5$ hours; \emph{Learning mode}: self-learning only; \emph{Repeated learning}: effortless.%
    \item[4.] \emph{Learning time}: $\geq 0.5,\,< 1.0$ hour; \emph{Learning mode}: self-learning + Q\&A; \emph{Repeated learning}: effortless.%
    \item[3.] \emph{Learning time}: $\geq 1.0,\,< 1.5$ hours; \emph{Learning mode}: tutorial; \emph{Repeated learning}: minor effort.%
    \item[2.] \emph{Learning time}: $\geq 1.5,\,< 2.0$ hours; \emph{Learning mode}: tutorial; \emph{Repeated learning}: noticeable effort.%
    \item[1.] \emph{Learning time}: $\geq 3$ hours; \emph{Learning mode}: tutorial; \emph{Repeated learning}: serious effort.%
\end{enumerate}
%
\noindent Here we define three learning modes: self-learning only, self-learning + Q\&A, and tutorial. We define the effort for repeated learning as effortless (e.g., a quick glance at the encoding scheme), minor effort (e.g., reading the encoding scheme again for 5--10 minutes), noticeable (e.g., reading the encoding scheme again for 10--30 minutes and/or requiring Q\&A), and serious effort (e.g., requiring another tutorial and/or more than 30 minutes).  
Note that the frequency of repeated learning relates to memorability.

% --------------------
\vspace{-3mm}
\subsection{Memorability}
\label{sec:Memorability}
%
\begin{description}
\item[Definition:] This criterion assesses the desirable property that the whole encoding scheme of a glyph is easy to remember once a viewer has learned the scheme. %
\item[Recommended Modes:] Type D (direct).
%
\item[Recommended Weight:] Type D [0.5].
\end{description}
%
As already discussed in Section \ref{sec:Learnability}, this complementary criterion assesses the easiness of memorizing an encoding scheme. Similar to learnability, it is a holistic criterion and is assessed through a Type D score. The assessment assumes that the users have already learned the encoding scheme, and the effort for repeated learning and memory refreshing is considered as part of learnability.

Because of the effort to learn a glyph design, the learned encoding scheme must be stored in long-term memory. While one could assess how long the target users can remember an encoding scheme, the overall intention of this MCDA method is to evaluate different glyph designs without too much delay. Therefore, we recommend to base the assessment of this criterion on the memorability after 1 hour and 24 hours following learning. Both time periods meet the requirement for testing long-term memory \cite{Glanzer:1966:JVLVB,Baddeley:2020:book}.

For a typical target user, we define the five levels according to how much the user can remember about an encoding scheme:

\begin{enumerate}
    \item[5.] \emph{after 1 hour}: 100\%, and \emph{after 24 hours}: 100\%.%
    \item[4.] \emph{after 1 hour}: <100\%, $\geq 90\%$, or \emph{after 24 hours}: <100\%, $\geq 75\%$.% 
    \item[3.] \emph{after 1 hour}: <90\%, $\geq 75\%$, or \emph{after 24 hours}: <75\%, $\geq 50\%$.% 
    \item[2.] \emph{after 1 hour}: <75\%, $\geq 50\%$, or \emph{after 24 hours}: <50\%, $\geq 25\%$.%
    \item[1.] \emph{after 1 hour}: <50\%, or \emph{after 24 hours}: <25\%.
\end{enumerate}








