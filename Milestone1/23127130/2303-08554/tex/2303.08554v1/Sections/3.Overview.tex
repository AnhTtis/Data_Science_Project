% ====================
\section{Overview, Terminology, and Design Considerations}
\label{sec:Overview}
%
In this work, we adopt the narrow definition of \emph{glyph} given by Borgo et al. \cite{Borgo:2013:STAR}, i.e., ``a glyph is a small independent visual object that depicts
attributes of a data record; glyphs are discretely placed in a display space; and
glyphs are a type of visual sign but differ from other types of signs such as icons, indices and symbols.''
We focus only on 2D glyphs.
Consider a multivariate data record for storing the values of $m$ \emph{variables}, $D = \{d_1, d_2, \ldots, d_m\}$, and a glyph for encoding such a record using $n$ \emph{visual channels} $\Gamma = \{\gamma_1, \gamma_2, \ldots, \gamma_n\}$. Since each data variable can be encoded using multiple visual channels, we have $n \geq m$.
% Note that in the literature, a visual channel was sometimes referred to as a ``visual variable'', ``retinal variable'', or ``graphical code''. Here we use the term ``visual channel'' to make it more distinguishable from the term ``data variable''.

Multiple-criteria decision analysis (MCDA) \cite{Ishizaka:2013:book,Azzabi:2020:book} can be viewed as a tree-based scoring system, where the score of each node is a weighted average of the scores at its child-nodes. Here an unweighted average is considered as a special case. There are a few high-level design considerations, e.g., do we decompose the holistic assessment into the assessments of data variables or visual channels and how detailed should we carry out the assessment?     

\noindent\textbf{Design Consideration 1. Data Variables vs Visual Channels.} 
%
As mentioned in Section \ref{sec:RalatedWork}, a number of criteria for glyph design have been proposed in the literature \cite{Bertin:2011:book,Maguire:2012:TVCG,Borgo:2013:STAR,Chung:2016:CGF}. Some criteria are defined for accessing visual channels (e.g., channel capacity \cite{Chung:2016:CGF}, while others are applicable to data variable (e.g., metaphoric representation \cite{Maguire:2012:TVCG}).
When one assumes a 1-to-1 mapping between a data variable and a visual channel, evaluating a data variable implicitly implies the evaluation of the corresponding visual channel, and vice versa. However, when a data variable is encoded using multiple visual channels, assessing individual channels independently may not inform their combined effect on perception and interpretation of the data variable concerned.
Hence the propose the scheme places an emphasis on data variables. 

\noindent\textbf{Design Consideration 2. Hierarchy of MCDA Evaluation.}
%
There are potentially more, if such criteria are combined or finely decomposed in evaluation.
Some criteria may be applicable to each data variable or visual channel (e.g., Bertin's criteria \cite{Bertin:2011:book}), while others may involve assessing a group of data variables (e.g., separability \cite{Chung:2016:CGF}), or a whole glyph (e.g., attention balance \cite{Chung:2016:CGF}). There are four types of basic assessment modes:
%
\begin{description}
    \item[Type A.] A criterion is for evaluating the visual encoding of a data variable $v_i$ independently. Given $m$ data variables, there are $m$ assessments for each Type A criterion.%
    \item[Type B.] A criterion is for evaluating how the encoding of a data variable $d_i$ is affected by the encoding of another $d_j (j \neq i)$. Given $m$ variables, there are potentially $m \times (m-1)$ assessments as the impact of $d_i$ on $d_j$ may not be the same as the impact of $d_j$ on $d_i$.%
    \item[Type C.] A criterion is for evaluating how the encoding of a data variable $d_i$ is affected by that of all others, i.e., $\forall d_j \in \{d_1, d_2, \ldots, d_m\} (j \neq i)$. There are $m$ assessments for $m$ variables.%
    \item[Type D.] A criterion is for evaluating the whole glyph holistically. There is one assessment per glyph for each Type D criterion.
\end{description}

\begin{figure}[ht]
    \centering
    \includegraphics[width=\linewidth]{Figures/Hierachy.pdf}
    \caption{Weighted aggregation is commonly used in hierarchical MCDA. In terms of four basic assessment modes, a Type D criterion may be scored directly, or assessed by aggregating the scores of Type A, Type B, and Type C criteria.}
    \label{fig:Hierarchy}
    \vspace{-6mm}
\end{figure}

As illustrated in Figure \ref{fig:Hierarchy}, the score of a high-level criterion (e.g., Type D) can often be obtained from scores of low-level criteria (e.g., Type A and Type B) through aggregation.
If a MCDA process consists of many low-level criteria, the designer will need to determine scores for individual variables or pairs of variables. In comparison with produce a high-level score directly (i.e., not through aggregation), the low-level scores are usually more precise, easier to judge, but more time-consuming to obtain.
%
Hence, there is a trade-off between precision and time cost.
In our proposed MCDA scheme, we indicate the possible modes, in which each criterion could be assessed, and recommend a specific mode of assessment for optimizing the trade-off concerned.   

\noindent\textbf{Design Consideration 3. Multiple complementary or conflicting facets.}
In the literature, a suggested criterion may consist two or more facets. For example, attention balance \cite{Chung:2016:CGF} encourages glyph designers to allow the encoding of important data variables to attract more attention, while making sure other data variables are not seriously disadvantaged. To ease the assessment, we intentionally divide such a multi-facet criterion into two criteria at the same level, allowing the designers to score the two potentially conflicting criteria independently.

