\section{Discussions and Conclusions}
\label{sec:Conclusions}
%
\textbf{Summary.} In this work, we have formulated a MCDA-aided assessment scheme for supporting glyph designs, and have tested the scheme on a range of glyph designs in the literature as well as their variants configured in the testing process. The scheme is built on the existing qualitative criteria in the literature, and enables a major step forward towards a more systematic, consistent, and semi-quantitative approach. Even when a glyph designer does not rank each criterion quantitatively, the scheme can serve as a reminder of the major considerations in encoding data variables and integrating different visual channels into a glyph representation.

\noindent \textbf{Limitations.}
(a) The scheme does not encode any domain-specific information. It is not in anyway a replacement for user-centered design and evaluation, especially when the target users are domain experts. Nevertheless, the scheme can help glyph designers phrase questions in seeking advice from domain experts and speed up the design process.
(b) The weights recommended in this work are specified based on our tests and analysis. On the one hand, there is a need for a set of weights that can be consistently applied to most (if not all) applications. On the other hands, weights are not ground truth, and their optimization needs the participation of the VIS community through many iterations. Thus the current recommendation is expected to be improved in the future. Meanwhile, it is important for glyph designers to ensure that the weights used in the assessment are transparent.
(c) For several criteria, e.g., separability, comparability, attention balance, and searchability, the specification of the five levels can be improved in the future based on new empirical research designed to obtain more precise measurements. We hope that the proposed scheme will stimulate such research.
(d) This proposed scheme does not cover the designs of 3D glyphs, glyph layout, interaction with glyphs, multi-scale glyphs, and so on. We hope that future research will extend the scheme.  

\noindent \textbf{Future Work.} In addition to aforementioned future research, the field of VIS can benefit from publicly-available glyph editing tools. The proposed scheme can potentially be integrated into such tools, facilitating human-centered and semi-automatic assessment of glyph designs during a design process.  


% Weights depending on the need ... the three multi-variate glyph purposes
% Individual channel or global rating
% I feel like maybe the two-phase usage of glyph can be used to decide the weighting of each criteria (purpose-based)
% Channel Independence \& Interference has influences on both channel reading and overall glyph effectiveness
% The weighting of different channels for a criterion item. Some are more important and some occur more frequently.

% channel searchability affect the cognitive load in the value comprehension phase
% Many criteria proposed in previous studies covers both intra- and inter-channel properties. Learnability, capacity, attention balance


