%-------------------------------------------------------------------------
\section{Introduction}
%
Glyph-based visualization is a family of widely used techniques, which are often integrated with other families of visualization techniques, such as small multiples in geo-spatial visualization, multivariate vertices in network visualization, directional and multivariate feature depiction in volume, vector, and tensor field visualization, and dynamic characteristics of objects in event and video visualization. While there have been proposals and discourses on desirable properties of glyph designs in the literature (e.g., \cite{Bertin:2011:book,Maguire:2012:TVCG,Borgo:2013:STAR,Chung:2016:CGF}), it is not yet a coherent methodology that visualization designers can use, consistently and methodically, in evaluating different design options in a process for designing and developing a visualization solution.

In glyph design processes, a designer may face many challenges (e.g., knowledge about the data, users, tasks, cognitive theories and experimental findings related to glyph-based visualization, and so on). This work focuses on one particular challenge, that is, there are many desirable properties of glyph designs, and likely a good design does not necessarily meet all criteria as one might desire, but embodies a relatively optimized set of trade-offs among the visual representations of different variables. In particular, we propose a methodology for evaluating different design options based on multiple-criteria decision analysis (MCDA) \cite{Ishizaka:2013:book,Azzabi:2020:book}, which is an established and widely adopted methodology in management science for evaluating multiple complementary and conflicting criteria explicitly in decision making. Its applications include business, governance, medicine, and engineering. 

The proposed methodology aims to introduce MCDA to glyph design as a systematic and cost-effective methodology and to bring together different desirable properties proposed in the literature into a typology of rateable criteria.
Much attention has been paid to (1) providing a good coverage of all proposed criteria for static glyph designs, (2) defining each criterion to facilitate clear interpretation and unambiguous ratings, (3) minimizing the overlapping among different criteria, (4) enabling distinct considerations of conflicting criteria, and (5) recommending a weighted scoring mechanism that balances between overview vs. detail and precision vs. cost.

The proposed methodology is expected to be improved through the experience gained in its uses in practice as well as new findings in theoretical and empirical research.
Furthermore, the proposed methodology is not a replacement for user-centered requirements analysis and evaluation, but it can complement, strength, and reduce the frequencies of user-centered studies in individual design processes.
We will discuss these further in Section \ref{sec:Conclusions}.
% In particular, we anticipate that the rating schemes for a number of criteria will be enriched when more measurements and models in visualization psychology are becoming available.   
%



