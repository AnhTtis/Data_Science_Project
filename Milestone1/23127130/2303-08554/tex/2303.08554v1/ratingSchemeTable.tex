

\begin{table*}[t]
  \centering
  \renewcommand{\arraystretch}{1.2}
  \begin{tabular}{|m{1cm}|p{15.9cm}|}
  \hline
         & \textbf{\large Value mapping appropriateness} \\ %[20pt] 
  \hline
    Lv 5 & The value mappings fit the needs for the main purpose and all the information from the original values are kept \\
    Lv 4 & The value mappings fit the needs for the main purpose but some details of the original value are truncated \\
    Lv 3 & The value mappings requires some straightforward transformation to map the value to the need \\
    Lv 2 & The require value details are preserved but extra effort are needed to transform the value to meet the main purpose \\
    Lv 1 & The main information required mismatch/is unclear from the mapping \\
  \hline
  \end{tabular}
  \renewcommand{\arraystretch}{1.5}
  \begin{tabular}{|m{1cm}|p{7.75cm}|p{7.7cm}|}
  \hline
  \rowcolor{LightGray}
         & \textbf{\large Visual Feature Readability} \sout{Value difference discernability} &
        \textbf{\large Value Comprehension} \sout{Feature mapping intuitiveness} \\ 
  \hline
    Lv 5 & Direct reading of value is easy, with extra feature to make robust value differences, minimal possibility of value confusion
        & The variable value can be understood with its direct look with minimal learning \\
    Lv 4 & The direct read of values is not difficult but require some attention. There is some possibility of misjudgement if the condition of the display (resolution, size, sharpness…) or the user (vision issues) is not ideal. 
        & The value can be understood with the learning of very straightforward logic \\
    Lv 3 & The direct read of values is possible but requires higher demand on the users and the display, having a medium risk for value confusion.
        & The logic to understand the value is not very straightforward but is not too hard to memorise. Once the logic is being memorised, the value is can be judged quite easily \\
    Lv 2 & Direct reading of value is difficult but value differences can be seen (easily) by direct comparison
        & Very complicated or no clear logic for feature mapping or the value is difficult to be judged directly from the look \\
    Lv 1 & Difficult to link visual encoding with channel value
        & The feature allocation is misleading \\
  \hline
  \end{tabular}
  \begin{tabular}{|m{1cm}|p{7.75cm}|p{7.7cm}|}
  \hline
  \rowcolor{DarkGray}
  & \textbf{\large \color{white} Channel independence and interference} & \textbf{\large \color{white} Channel Semantic Linkage} \sout{Channel searchability} \\
  \hline
    Lv 5 & The reading of values is independent of or bears no/minimum influences from the value of other channels and does not interfere with other channels
      & Identification of he channels can be achieved by intuition with minimum learning  \\
    Lv 4 & The look of the channel varies slightly with value changes of other channels or interfere with others, but the influence on comprehension is low
      & Identification of he channels can be achieved by the learning of basic construction rules  \\
    Lv 3 & The reading of channel value could be interfered by other channels but correct reading of values can be achieved with some training. Extra care is required to avoid misinterpretation.
      & Constructive rules and channel hints help significantly on the identification of channels, but some other memorisation by rote is required  \\
    Lv 2 & The channel value showed varied visibility or could be easily misinterpreted at certain values of other channels.
      & Constructive rules and channel hints only indicate rough categories of the channels \\
    Lv 1 & The channel values might be severely or completely blocked/hidden/indistinguishable at certain values of other channels.
      & Few help from constructive rules and channel hints for the identification of channels \\
  \hline
  \hline
  \rowcolor{Black}
  & \textbf{\large \color{white} Channel impact visibility} & \textbf{\large \color{white} Cognitive loading for comprehension} \\
  \hline
    Lv 5 & All channel features are clearly visible at the macro view and with good visibility balance
      & The channels can be comprehend almost effortlessly as one single object  \\
    Lv 4 & All channel features are clearly visible at the macro view; the visibility balance is acceptable but can be further improved
      & The channels can be comprehend as one single object but take a bit of efforts to read and memorise  \\
    Lv 3 & Most channel features are clearly visible at the macro view, only a few require further improvement and the balance is acceptable
      & The channels need to be comprehend as a small number of different parts, which is not too difficult to read and memorised  \\
    Lv 2 & The visibility of more channels require further improvement or the balance is not ideal
      & The channels can still be comprehend as different parts and perhaps some loose channels. The number of segments is cognitively overloading. The partitioning is still helpful for glyph comparison \\
    Lv 1 & Poor visibility design for most channels or visibility balance
      & The channels are too much scattered and too heavily cognitively overloading \\
  \hline
  \end{tabular}
  \caption{Blabla}
  \label{tab:1}
\end{table*}

