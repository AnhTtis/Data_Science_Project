
\begin{table}[t!]
    \begin{centering}
    \begin{tabular}{l|cc}
     Instrument & Filter & Flux (nJy) \\
     \hline
     \hline
     \hst/ACS  & F606W & 8.95 $\pm$ 7.67 \\
               & F814W & 0.96 $\pm$ 8.07 \\
     \hst/WFC3 & F098M & -11.00 $\pm$ 18.40 \\
     %          & F105W & \\
               & F125W & 200.78 $\pm$ 11.30 \\
               & F140W & 272.05 $\pm$ 19.18 \\
               & F160W & 306.22 $\pm$ 9.69 \\
   \hline
    \jwst/NIRCam & F115W & 117.07 $\pm$ 3.81 \\
                 & F150W & 334.15 $\pm$ 7.11 \\
                 & F200W & 313.72 $\pm$ 5.63 \\
                 & F277W & 371.35 $\pm$ 4.51 \\
                 & F356W & 449.55 $\pm$ 3.51 \\
                 & F410M & 450.11 $\pm$ 8.81 \\
                 & F444W & 1039.16 $\pm$ 7.65 \\
     \jwst/MIRI  & F560W & 426.80 $\pm$ 21.60 \\
                 & F770W & 404.00 $\pm$ 16.90 \\
    \end{tabular}
    \end{centering}
    \caption{Photometric measurements for CEERS\_1019, with fluxes in nJy. The \hst\ photometry is updated from \citet{finkelstein22} using the NIRCam-selected apertures as described in \citet{finkelstein22c}. \jwst/NIRCam photometry from CEERS Epoch 2 imaging measured in a similar way as \citet{finkelstein22c}, described in \S\ref{sec:nircam-imaging}. \jwst/MIRI photometry from \citet{papovich22} using CEERS Epoch 1 imaging, described in \S\ref{sec:miri-imaging}. }
    \label{tab:photometry}
\end{table}