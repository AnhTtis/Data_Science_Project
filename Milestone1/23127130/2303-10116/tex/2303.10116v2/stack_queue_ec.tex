
\documentclass[cleveref,a4paper,UKenglish]{lipics-v2021}
\bibliographystyle{plainurl}
\sloppy
%\pdfsuppresswarningpagegroup=1
\usepackage{cite} % sort cite numbers
\usepackage{amsmath,amsthm}
\usepackage{amssymb}  
\usepackage{xspace}
% \usepackage{enumitem}
% \usepackage{enumerate}
\usepackage{adjustbox,standalone}


% TiKZ
\usepackage{tikz}
\usepackage{todonotes}
\tikzstyle{every picture} = [>=latex]
\usetikzlibrary{calc,arrows,decorations.markings,quotes}
\usetikzlibrary{backgrounds,shapes,arrows}


%% moving statements and proofs to the appendix automatically - we shall use it everywhere!
% \usepackage{apxproof}
 \usepackage[appendix=inline]{apxproof}
\ifx\proof\inlineproof
  \def\apxmark{}
\else
  \def\apxmark{\,*$\!$}
\fi
\newtheoremrep{propositionx2}[theorem]{Proposition}
\newtheoremrep{proposition2}[theorem]{Proposition\apxmark}
\newtheoremrep{theorem2}[theorem]{Theorem\apxmark}
\newtheoremrep{lemma2}[theorem]{Lemma\apxmark}
\newtheoremrep{claim2}[theorem]{Claim\apxmark}
% can further use the following:
% \ifx\proof\inlineproof
% \else
% \fi
% \noproofinappendix

\newtheorem{case}{Case}

% \newcommand\FO{FO\xspace}
% \newcommand\FPT{{\sf FPT}\xspace}
\def\mx#1{\mbox{\boldmath{$#1$}}}
\def\ca#1{{\cal#1}}
\def\cf#1{{\EuScript#1}}
\def\sn{\mathop{\text{sn}}}
\def\qn{\mathop{\text{qn}}}
\def\levll#1#2{\lambda[#1](#2)}

\hideLIPIcs
\nolinenumbers





\title{Stack and Queue Numbers of Graphs Revisited}
%%%%%%%%%%%%%%%%%%%%%%%%%%%%%%%%%%%%%%%%%%%%%%%%%%%%%%%%%%%%%%%%%%%%%%%
% \titlerunning{}

\author{Petr Hlin{\v e}n\'y}{Masaryk University, Brno,
  Czech republic}{hlineny@fi.muni.cz}{https://orcid.org/0000-0003-2125-1514}{}
\author{Adam Straka}{Masaryk University, Brno,
  Czech republic}{493034@mail.muni.cz}{}{}
  
\authorrunning{P.\ Hlin\v{e}n\'y and A.~Straka}
\Copyright{Petr Hlin\v{e}n\'y}

\keywords{stack number, queue number, Cartesian product}
% \gdef\keywordsHeading{}\keywords{\vspace*{-4ex}}

% \funding{Supported by the {Czech Science Foundation}, project no.~{20-04567S}.}

\ccsdesc[500]{Mathematics of computing~Graph theory}


\begin{document}
\maketitle

\begin{abstract}
A long-standing question of the mutual relation between the stack and queue numbers of a graph,
explicitly emphasized by Dujmovi\'c and Wood in 2005, was ``half-answered''
by Dujmovi\'c, Eppstein, Hickingbotham, Morin and Wood in 2022;
they proved the existence of a graph family with the queue number at most $4$ but unbounded stack number.
We give an alternative very short, and still elementary, proof of the same fact.
\end{abstract}






\section{Introduction}
%%%%%%%%%%%%%%%%%%%%%%%%%%%%%%%%%%%%%%%%%%%%%%%%%%%%%%%%%%%%%%%%%%%%%%%

The graph parameters called stack and queue numbers relate to {\em linear layouts} (i.e, linear vertex orderings, usually of additional ``nice'' properties) of graphs,
and have found numerous applications in theoretical computer science since then.
The parameters were formally introduced by Heath, Leighton, and Rosenberg in \cite{comparing,laying_queues}, and their implicit question of
whether the stack number of a graph is bounded in terms of its queue number, or vice versa, was subsequently emphasized
by Dujmovi\'c and Wood in~\cite{dujmovic2005stacks}.
Quite recently, in 2022, Dujmovi\'c, Eppstein, Hickingbotham, Morin and Wood gave in \cite{dujmovic2021stack} a negative
answer to one half of the question; they proved the existence of a graph family with the queue number at most $4$ but unbounded stack number
(while it remains an open problem whether there exists a family of bounded stack number and unbounded queue number).

We give the basic definitions.
Consider a graph $G$ and a strict linear order $\prec$ on its vertex set $V(G)$.
Two edges $xx',yy'\in E(G)$ with $x\prec x'$ and $y\prec y'$ are said to {\em$\prec$-cross} if $x\prec y\prec x'\prec y'$ or $y\prec x\prec y'\prec x'$,
and to  {\em$\prec$-nest} if $x\prec y\prec y'\prec x'$ or $y\prec x\prec x'\prec y'$.
See \Cref{fig:crossnest}.
The {\em stack number} $\sn(G)$ ({\em queue number} $\qn(G)$) of a graph $G$ is the minimum integer $k$ such that there exist a linear order $\prec$ of $V(G)$
and a colouring of the edges of $G$ by $k$ colours such that no two edges of the same colour $\prec$-cross ($\prec$-nest, resp.).
The corresponding order $\prec$ together with the colouring is called a $k$-stack ($k$-queue) layout of~$G$.

\begin{figure}[h]
\centering
~\raise2ex\hbox{a)~}
\begin{tikzpicture}
\tikzstyle{every node}=[draw, shape=circle, minimum size=3pt,inner sep=0pt, fill=black]
\draw[dotted] (0,0)--(5,0);
\node[label=below:$\prec:$,draw=none,fill=none] at (0,0) (0) {};
\node[label=below:$~x\,~$] at (1,0) (x) {}; \node[label=below:$x'$] at (3,0) (xx) {};
\node[label=below:$~\,y\,~$] at (2,0) (y) {}; \node[label=below:$y'$] at (4,0) (yy) {};
\draw (x) to[bend left] (xx) (y) to[bend left] (yy);
\end{tikzpicture}
\qquad
\raise2ex\hbox{b)~}
\begin{tikzpicture}
\tikzstyle{every node}=[draw, shape=circle, minimum size=3pt,inner sep=0pt, fill=black]
\draw[dotted] (0,0)--(5,0);
\node[label=below:$\prec:$,draw=none,fill=none] at (0,0) (0) {};
\node[label=below:$~x\,~$] at (1,0) (x) {}; \node[label=below:$x'$] at (4,0) (xx) {};
\node[label=below:$~\,y\,~$] at (2,0) (y) {}; \node[label=below:$y'$] at (3,0) (yy) {};
\draw (x) to[bend left] (xx) (y) to[bend left] (yy);
\end{tikzpicture}
\caption{Edges $xx'$ and $yy'$ that (a) $\prec$-cross, and (b) $\prec$-nest.}
\label{fig:crossnest}
\end{figure}

In fact, a notion equal (modulo a negligible technical detail) to the stack number was known long before as the {\em book thickness}
(or page number), see Persinger~\cite{persinger1966subsets} and Atneosen~\cite{atneosen1968embeddability}.

To state the main result of \cite{dujmovic2021stack}, we define the following special graph $H_n$:
the vertex set is $V(H_n)=\{1,\ldots,n\}^2$, and $uv\in E(H_n)$ where $u=[a,b]\in V(H_n)$ and $v=[c,d]\in V(H_n)$, if and only if
$|a-c|+|b-d|=1$ or $a-c=b-d\in\{-1,1\}$.
Note that $H_n$ is the plane dual of the hexagonal (``honeycomb'') grid, and see an illustration in \Cref{fig:SnHn}.

\begin{figure}[ht]
    \centering%\vspace*{-3.1ex}
~\raise2ex\hbox{a)~}
\begin{tikzpicture}[scale=0.6, every node/.style={circle,fill,inner sep=0pt, minimum size=0.12cm}]
    \node (S) at (2,3) { };
    \node (A) at (0,0) { };
    \node (B) at (1,0) { };
    \node (C) at (2,0) { };
    \node (D) at (3,0) { };
    \node (E) at (4,0) { };    
    \path (S) edge (A); 
    \path (S) edge (B); 
    \path (S) edge (C);
    \path (S) edge (D);
    \path (S) edge (E);
\end{tikzpicture}
\qquad\raise2ex\hbox{b)~}
\begin{tikzpicture}[scale=0.75, every node/.style={circle,fill,inner sep=0pt, minimum size=0.12cm}]   
    \node (1) at (0,0) { };
    \node (2) at (1.5,0) { };
    \node (3) at (3,0) { };
    \node (4) at (0,1.5) { };
    \node (5) at (1.5,1.5) { };
    \node (6) at (3,1.5) { }; 
    \node (7) at (0,3) { };
    \node (8) at (1.5,3) { };
    \node (9) at (3,3) { };
    
    \path (1) edge (2); 
    \path (2) edge (3); 
    \path (4) edge (5); 
    \path (5) edge (6); 
    \path (7) edge (8); 
    \path (8) edge (9); 
    \path (1) edge (5); 
    \path (2) edge (6); 
    \path (4) edge (8); 
    \path (5) edge (9); 
    \path (1) edge (4); 
    \path (4) edge (7);
    \path (2) edge (5); 
    \path (5) edge (8); 
    \path (3) edge (6); 
    \path (6) edge (9); 
\end{tikzpicture}
\qquad\raise2ex\hbox{c)~}
\begin{tikzpicture}[scale=0.33, every node/.style={circle,fill=black, inner sep=0.4mm}]
    \node (S1) at (2,3) { };    \node (A1) at (0,0) { };    \node (B1) at (1,0) { };    \node (C1) at (2,0) { };    \node (D1) at (3,0) { };    \node (E1) at (4,0) { }; 
    \path (S1) edge (A1);     \path (S1) edge (B1);     \path (S1) edge (C1);    \path (S1) edge (D1);    \path (S1) edge (E1);
    \node (S2) at (8,3) { };    \node (A2) at (6,0) { };    \node (B2) at (7,0) { };    \node (C2) at (8,0) { };    \node (D2) at (9,0) { };    \node (E2) at (10,0) { }; 
    \path (S2) edge (A2);     \path (S2) edge (B2);     \path (S2) edge (C2);    \path (S2) edge (D2);    \path (S2) edge (E2);
    \node (S3) at (14,3) { };    \node (A3) at (12,0) { };    \node (B3) at (13,0) { };    \node (C3) at (14,0) { };    \node (D3) at (15,0) { };    \node (E3) at (16,0) { }; 
    \path (S3) edge (A3);     \path (S3) edge (B3);     \path (S3) edge (C3);    \path (S3) edge (D3);    \path (S3) edge (E3);
    \node (S4) at (2,7) { };    \node (A4) at (0,4) { };    \node (B4) at (1,4) { };    \node (C4) at (2,4) { };    \node (D4) at (3,4) { };    \node (E4) at (4,4) { }; 
    \path (S4) edge (A4);     \path (S4) edge (B4);     \path (S4) edge (C4);    \path (S4) edge (D4);    \path (S4) edge (E4);
    \node (S5) at (8,7) { };    \node (A5) at (6,4) { };    \node (B5) at (7,4) { };    \node (C5) at (8,4) { };    \node (D5) at (9,4) { };    \node (E5) at (10,4) { }; 
    \path (S5) edge (A5);     \path (S5) edge (B5);     \path (S5) edge (C5);    \path (S5) edge (D5);    \path (S5) edge (E5);
    \node (S6) at (14,7) { };    \node (A6) at (12,4) { };    \node (B6) at (13,4) { };    \node (C6) at (14,4) { };    \node (D6) at (15,4) { };    \node (E6) at (16,4) { }; 
    \path (S6) edge (A6);     \path (S6) edge (B6);     \path (S6) edge (C6);    \path (S6) edge (D6);    \path (S6) edge (E6);
    \node (S7) at (2,11) { };    \node (A7) at (0,8) { };    \node (B7) at (1,8) { };    \node (C7) at (2,8) { };    \node (D7) at (3,8) { };    \node (E7) at (4,8) { }; 
    \path (S7) edge (A7);     \path (S7) edge (B7);     \path (S7) edge (C7);    \path (S7) edge (D7);    \path (S7) edge (E7);
    \node (S8) at (8,11) { };    \node (A8) at (6,8) { };    \node (B8) at (7,8) { };    \node (C8) at (8,8) { };    \node (D8) at (9,8) { };    \node (E8) at (10,8) { }; 
    \path (S8) edge (A8);     \path (S8) edge (B8);     \path (S8) edge (C8);    \path (S8) edge (D8);    \path (S8) edge (E8);
    \node (S9) at (14,11) { };    \node (A9) at (12,8) { };    \node (B9) at (13,8) { };    \node (C9) at (14,8) { };    \node (D9) at (15,8) { };    \node (E9) at (16,8) { }; 
    \path (S9) edge (A9);     \path (S9) edge (B9);     \path (S9) edge (C9);    \path (S9) edge (D9);    \path (S9) edge (E9);
    \path (S1) edge[color=blue, opacity=0.4, bend left=20] (S2); 
    \path (S2) edge[color=blue, opacity=0.4, bend left=20] (S3); 
    \path (S4) edge[color=blue, opacity=0.4, bend left=20] (S5); 
    \path (S5) edge[color=blue, opacity=0.4, bend left=20] (S6); 
    \path (S7) edge[color=blue, opacity=0.4, bend left=20] (S8); 
    \path (S8) edge[color=blue, opacity=0.4, bend left=20] (S9); 
    \path (A1) edge[color=blue, opacity=0.4, bend left=20] (A2); 
    \path (A2) edge[color=blue, opacity=0.4, bend left=20] (A3); 
    \path (A4) edge[color=blue, opacity=0.4, bend left=20] (A5); 
    \path (A5) edge[color=blue, opacity=0.4, bend left=20] (A6); 
    \path (A7) edge[color=blue, opacity=0.4, bend left=20] (A8); 
    \path (A8) edge[color=blue, opacity=0.4, bend left=20] (A9); 
    \path (B1) edge[color=blue, opacity=0.4, bend left=20] (B2); 
    \path (B2) edge[color=blue, opacity=0.4, bend left=20] (B3); 
    \path (B4) edge[color=blue, opacity=0.4, bend left=20] (B5); 
    \path (B5) edge[color=blue, opacity=0.4, bend left=20] (B6); 
    \path (B7) edge[color=blue, opacity=0.4, bend left=20] (B8); 
    \path (B8) edge[color=blue, opacity=0.4, bend left=20] (B9); 
    \path (C1) edge[color=blue, opacity=0.4, bend left=20] (C2); 
    \path (C2) edge[color=blue, opacity=0.4, bend left=20] (C3); 
    \path (C4) edge[color=blue, opacity=0.4, bend left=20] (C5); 
    \path (C5) edge[color=blue, opacity=0.4, bend left=20] (C6); 
    \path (C7) edge[color=blue, opacity=0.4, bend left=20] (C8); 
    \path (C8) edge[color=blue, opacity=0.4, bend left=20] (C9); 
    \path (D1) edge[color=blue, opacity=0.4, bend left=20] (D2); 
    \path (D2) edge[color=blue, opacity=0.4, bend left=20] (D3); 
    \path (D4) edge[color=blue, opacity=0.4, bend left=20] (D5); 
    \path (D5) edge[color=blue, opacity=0.4, bend left=20] (D6); 
    \path (D7) edge[color=blue, opacity=0.4, bend left=20] (D8); 
    \path (D8) edge[color=blue, opacity=0.4, bend left=20] (D9); 
    \path (E1) edge[color=blue, opacity=0.4, bend left=20] (E2); 
    \path (E2) edge[color=blue, opacity=0.4, bend left=20] (E3); 
    \path (E4) edge[color=blue, opacity=0.4, bend left=20] (E5); 
    \path (E5) edge[color=blue, opacity=0.4, bend left=20] (E6); 
    \path (E7) edge[color=blue, opacity=0.4, bend left=20] (E8); 
    \path (E8) edge[color=blue, opacity=0.4, bend left=20] (E9);
    \path (A1) edge[color=green,bend left=20] (A5); 
    \path (A2) edge[color=green,bend left=20] (A6); 
    \path (A4) edge[color=green,bend left=20] (A8); 
    \path (A5) edge[color=green,bend left=20] (A9); 
    \path (B1) edge[color=green,bend left=20] (B5); 
    \path (B2) edge[color=green,bend left=20] (B6); 
    \path (B4) edge[color=green,bend left=20] (B8); 
    \path (B5) edge[color=green,bend left=20] (B9); 
    \path (C1) edge[color=green,bend left=20] (C5); 
    \path (C2) edge[color=green,bend left=20] (C6); 
    \path (C4) edge[color=green,bend left=20] (C8); 
    \path (C5) edge[color=green,bend left=20] (C9); 
    \path (D1) edge[color=green,bend left=20] (D5); 
    \path (D2) edge[color=green,bend left=20] (D6); 
    \path (D4) edge[color=green,bend left=20] (D8); 
    \path (D5) edge[color=green,bend left=20] (D9); 
    \path (E1) edge[color=green,bend left=20] (E5); 
    \path (E2) edge[color=green,bend left=20] (E6); 
    \path (E4) edge[color=green,bend left=20] (E8); 
    \path (E5) edge[color=green,bend left=20] (E9); 
    \path (S1) edge[color=green,bend left=20] (S5); 
    \path (S2) edge[color=green,bend left=20] (S6); 
    \path (S4) edge[color=green,bend left=20] (S8); 
    \path (S5) edge[color=green,bend left=20] (S9);
    \path (A1) edge[color=red, opacity=0.4, bend left=20] (A4); 
    \path (A4) edge[color=red, opacity=0.4, bend left=20] (A7);
    \path (A2) edge[color=red, opacity=0.4, bend left=20] (A5); 
    \path (A5) edge[color=red, opacity=0.4, bend left=20] (A8); 
    \path (A3) edge[color=red, opacity=0.4, bend left=20] (A6); 
    \path (A6) edge[color=red, opacity=0.4, bend left=20] (A9); 
    \path (B1) edge[color=red, opacity=0.4, bend left=20] (B4); 
    \path (B4) edge[color=red, opacity=0.4, bend left=20] (B7);
    \path (B2) edge[color=red, opacity=0.4, bend left=20] (B5); 
    \path (B5) edge[color=red, opacity=0.4, bend left=20] (B8); 
    \path (B3) edge[color=red, opacity=0.4, bend left=20] (B6); 
    \path (B6) edge[color=red, opacity=0.4, bend left=20] (B9); 
    \path (C1) edge[color=red, opacity=0.4, bend left=20] (C4); 
    \path (C4) edge[color=red, opacity=0.4, bend left=20] (C7);
    \path (C2) edge[color=red, opacity=0.4, bend left=20] (C5); 
    \path (C5) edge[color=red, opacity=0.4, bend left=20] (C8); 
    \path (C3) edge[color=red, opacity=0.4, bend left=20] (C6); 
    \path (C6) edge[color=red, opacity=0.4, bend left=20] (C9); 
    \path (D1) edge[color=red, opacity=0.4, bend left=20] (D4); 
    \path (D4) edge[color=red, opacity=0.4, bend left=20] (D7);
    \path (D2) edge[color=red, opacity=0.4, bend left=20] (D5); 
    \path (D5) edge[color=red, opacity=0.4, bend left=20] (D8); 
    \path (D3) edge[color=red, opacity=0.4, bend left=20] (D6); 
    \path (D6) edge[color=red, opacity=0.4, bend left=20] (D9); 
    \path (E1) edge[color=red, opacity=0.4, bend left=20] (E4); 
    \path (E4) edge[color=red, opacity=0.4, bend left=20] (E7);
    \path (E2) edge[color=red, opacity=0.4, bend left=20] (E5); 
    \path (E5) edge[color=red, opacity=0.4, bend left=20] (E8); 
    \path (E3) edge[color=red, opacity=0.4, bend left=20] (E6); 
    \path (E6) edge[color=red, opacity=0.4, bend left=20] (E9); 
    \path (S1) edge[color=red, opacity=0.4, bend left=20] (S4); 
    \path (S4) edge[color=red, opacity=0.4, bend left=20] (S7);
    \path (S2) edge[color=red, opacity=0.4, bend left=20] (S5); 
    \path (S5) edge[color=red, opacity=0.4, bend left=20] (S8); 
    \path (S3) edge[color=red, opacity=0.4, bend left=20] (S6); 
    \path (S6) edge[color=red, opacity=0.4, bend left=20] (S9); 
\end{tikzpicture}
    \caption{(a) The star $S_5$, (b) the graph $H_3$, and (c) their Cartesian product $S_5\square H_3$.
	The four edge colours illustrate a queue layout for $S_5\square H_3$.}
	\label{fig:SnHn}
\end{figure}

Recall that $S_n$ is the star with $n$ leaves, and that $G_1\square G_2$ denotes the Cartesian product of two graphs $G_1$ and~$G_2$.
Dujmovi\'c et al.~\cite{dujmovic2021stack} showed that, for every integers $b,n>0$ and the Cartesian product $G=S_a\square H_n$,
we have $\qn(G)\leq4$.
In fact, they noted that every $H_n$ admits a so-called {\em strict} $3$-queue layout, 
which ``adds up'' with a trivial $1$-queue layout of $S_a$ over Cartesian product by Wood~\cite{wood2005queue}.
Their main result reads:

\begin{theorem}[Dujmovi\'c et al.~\cite{dujmovic2021stack}]\label{thm:main}
For every integer $s$, and for $a,n>0$ which are sufficiently large with respect to~$s$,
the Cartesian product $G:= S_a \square H_n$ is of stack number at least~$s$.
\end{theorem}

Our contribution is to give a very short simplified proof of \Cref{thm:main} (based in parts on the ideas from \cite{dujmovic2021stack},
but also eliminating some rather long fragments of the former proof).




\section{Proof of \Cref{thm:main}}
%%%%%%%%%%%%%%%%%%%%%%%%%%%%%%%%%%%%%%%%%%%%%%%%%%%%%%%%%%%%%%%%%%%%%%%
% \section{A queue number bound on the stack number}
\label{chapter_proof}

We will use some classical results, the first two of which are truly folklore.
\begin{proposition}[Ramsey~\cite{ramsey}]\label{ramsey}
For every integers $r,s>0$ there exists $R=R(r,s)$ such that for any assignment of two colours read and blue to the edges of 
the complete graph $K_R$, there is a red clique on $r$ vertices or a blue clique on $s$ vertices in it.
\end{proposition}
\begin{proposition}[Erd\H{o}s--Szekeres~\cite{erdos}]\label{erdos-szekeres}
For given integers $r,s>0$, any sequence of distinct elements of a linearly ordered set of length more than $r\cdot s$ 
contains an increasing subsequence of length $s+1$ or a decreasing subsequence of length~$r+1$.
\end{proposition}
\begin{proposition}[Gale~\cite{hex}]\label{hex_lemma}
Consider a dual hexagonal grid $H_n$ as above. For any assignment of two colours to the vertices of $H_n$,
there exists a monochromatic path on $n$ vertices.
\end{proposition}

Consider for the rest any fixed stack layout of the graph $G$ of \Cref{thm:main}, with the linear order~$\prec$ on the vertex set~$V(G)$.
Recall that $V(G)=\{(u,p): u\in V(S_a), p\in V(H_n)\}$.

\begin{lemma}
Let $L$ be the set of leaves of $S_a$, and let $b=a^{-m}$ where $m=2^{n^2-1}$.
There is a subsequence $(u_1,\ldots,u_b)$ in the set $L$ of length $b$ such that for each $p \in V(H_n)$, 
either $(u_1, p) \prec (u_2, p) \prec \ldots \prec  (u_b, p)$, or $(u_1, p) \succ (u_2, p) \succ \ldots \succ  (u_b, p)$.
\label{erdos_lemma}
\end{lemma}
\begin{proof}
Let $V(H_n)=\{p_1,\ldots,p_{n^2}\}$ be the vertices of $H_n$.
Start with the permutation $\sigma_1=(u_{i[1,1]},\ldots,u_{i[1,a_1=a]})$ of $L$ such that $(u_{i[1,1]}, p_1) \prec\ldots\prec (u_{i[1,a_1]}, p_1)$.
By \Cref{erdos-szekeres}, for each $j \in \{2,\ldots,n^2\}$, the sequence $\sigma_{j-1}$
contains a subsequence $\sigma_j = (u_{i[j,1]}, \ldots, u_{i[j,a_j]})$ such that $a_j\geq \sqrt{a_{j-1}}$,
and $(u_{i[j,1]}, p_j) \prec\ldots \prec (u_{i[j,a_j]},p_j)$ or $(u_{i[j,1]}, p_j) \succ\ldots \succ (u_{i[j,a_j]},p_j)$.
By simple calculus, we get $a_{n^2}\geq a_1^m=b$ which is the desired outcome.
\end{proof}

Let $S_b\subseteq S_a$ be the (specific) substar of $S_a$ defined by the subset of leaves $\{u_1,\ldots,u_b\}$ (of \Cref{erdos_lemma}).
Colour every vertex $p \in V(H_n)$ red if $(u_1, p) \prec \ldots \prec (u_b, p)$, and colour $p$ blue otherwise.
From this and \Cref{hex_lemma}, we immediately obtain:
\begin{corollary}\label{cor:sameord}
There is a subgraph $Q\subseteq H_n$, being a path on $n$ vertices, such that, without loss of generality, 
$(u_1, q) \prec \ldots \prec (u_b, q)$ holds for every vertex $q\in V(Q)$.
\qed\end{corollary}

Define $X\subseteq G$ to be the subgraph induced on the vertex set $V(S_b)\times V(Q)$, i.e., $X=S_b\square Q$,
and denote by $\ca R$ the set of paths $R_i\subseteq X$ induced on $\{u_i\}\times V(Q)$ for $i=1,\ldots,b$.
We extend $\prec$ to a partial order on $\ca R$ as follows; for $R_i,R_j\in\ca R$, we have $R_i\prec R_j$,
if and only if $u\prec w$ for all $u\in V(R_i)$ and $w\in V(R_j)$.
We say that $R_i$ and $R_j$ are {\em$\prec$-separated} if $R_i\prec R_j$ or $R_i\succ R_j$, and that
$R_i$ and $R_j$ are {\em$\prec$-crossing} if there exist edges $e\in E(R_i)$ and $f\in E(R_j)$ such that $e,f$ $\prec$-cross.
The following is simple but crucial:

\begin{lemma}\label{lem:crossepar}
    Every two distinct paths $R_i, R_j \in \mathcal{R}$ are either $\prec$-crossing, or $\prec$-separated.
\end{lemma}
\begin{proof}
    Assume the contrary; up to symmetry meaning that all edges of $R_i$ are nested in some edge $e_2 = \{(u_j,q),(u_j,q')\}\in E(R_j)$.
Then, in particular, $e_1 = \{(u_i,q),(u_i,q')\}\in E(R_i)$ is nested in $e_2$, and so $(u_j, q) \prec (u_i, q)$ and $(u_j, q') \succ (u_i, q')$.
This contradicts \Cref{cor:sameord}.
\end{proof}

\begin{corollary}\label{cor:crossepar}
For every integers $c,d$ and $n$, and for $b=|\ca R|$ sufficiently large with respect to $c,d$, we have that $\ca R$ contains
at least $c$ pairwise $\prec$-separated or $d$ pairwise $\prec$-crossing paths.
\end{corollary}
\begin{proof}
Imagine a pair of paths $\{R_i,R_j\}\subseteq\ca R$ coloured red if $R_i,R_j$ are $\prec$-crossing, and blue if they are $\prec$-separated.
With respect to \Cref{lem:crossepar}, we apply \Cref{ramsey} with~$b\geq R(c,d)$.
\end{proof}

We finish as follows.
\begin{proof}[Proof of~\Cref{thm:main}]
Respecting the above definition of the set of paths $\ca R$ in $G$, we branch into the two cases determined by \Cref{cor:crossepar}.
\smallskip
\begin{description}
\item[Case~I.]
 There are $c$ pairwise $\prec$-separated paths in $\mathcal{R}$. 

Without loss of generality, let these paths be $R_1 \prec \ldots \prec R_c$.
For the root $t$ of $S_b$, label the $n$ vertices of the set $\{t\}\times V(Q)\subseteq V(X)$ by $t_1\prec\ldots\prec t_n$. 
There are two subcases.
\begin{itemize}
     \item  $R_{\lfloor c/2 \rfloor} \prec t_{\lceil n/2 \rceil}$. 
For each $i=1,\ldots,\min(\lfloor c/2\rfloor,\lceil n/2\rceil)$, pick an edge of $X$ from $t_{\lceil n/2\rceil+i-1}$ to $V(R_i)$ 
(which exist since $R_i$ hits every copy of $S_b$ in~$X$ by the definition).
We have got $\min(\lfloor c/2\rfloor,\lceil n/2\rceil)$ edges in $X$ that pairwise $\prec$-cross, as in \Cref{fig:separat}.
\begin{figure}[h!]
    \centering
\begin{tikzpicture}[every node/.style={thick}]\small
    \node [style=ellipse, minimum width=40pt, label=below:$R_1$, draw](A) at (0,0) {};
    \node [style=ellipse, minimum width=40pt, label=below:$R_2$, draw](B) at (1.5,0) {};
    \node (C) [ellipse, minimum width=100pt, color=white, text = black]at (3,0) {\LARGE . . .};
    \node [ellipse, minimum width=40pt, label=below:$R_{\lfloor c/2 \rfloor}$, draw](D) at (4.5,0) {};
    \node [style={circle,fill,inner sep=0pt, minimum size=1.2mm}, label=below:$t_{\lceil n/2 \rceil}$](A') at (6,0) { };
    \node [style={circle,fill,inner sep=0pt, minimum size=1.2mm}, label=below:$t_{\lceil n/2 \rceil + 1}$](B') at (7.5,0) { };
    \node [ style={circle,fill,inner sep=0pt, minimum size=1.2mm}, color = white, text = black](C') at (9,0) {\LARGE . . .};
    \node [style={circle,fill,inner sep=0pt, minimum size=1.2mm}, label=below:$t_{\lceil n/2 \rceil+\lfloor c/2 \rfloor-1}$](D') at (10.5,0) { };
    \node [style={circle,fill,inner sep=0pt, minimum size=1.2mm}, label=below:$t_n$](D'') at (12,0) { };
    \path (A) edge[bend left=30] (A'); 
    \path (B) edge[bend left=30, color=red] (B'); 
    \path (D) edge[bend left=30, color=blue](D'); 
\end{tikzpicture}
    \caption{Case~I, where $R_{\lfloor c/2 \rfloor} \prec t_{\lceil n/2 \rceil}$ and $\lceil n/2\rceil>\lfloor c/2\rfloor$.}
\label{fig:separat}
\end{figure} 
     \item  $t_{\lceil n/2 \rceil} \prec R_{\lfloor c/2 \rfloor + 1}$ (note that $t_{\lceil n/2\rceil}$ may be ``$\prec$-nested'' in  $R_{\lfloor c/2 \rfloor}$). 
This is symmetric to the previous, and we get $\min(\lceil c/2 \rceil, \lceil n/2 \rceil)$ pairwise $\prec$-crossing edges in $X$ 
between vertices of $R_{\lfloor c/2 \rfloor + 1},\ldots, R_c$ and $s_1,\ldots, s_{\lceil n/2 \rceil}$.
% \begin{figure}[h!]
%     \centering
%     \trimbox{0pt, 10pt, 0pt, 15pt}{
%     \input{pictures/separated_b}
%     }
%     \caption{Case \ref{case_1}, where $s_{\lceil n/2 \rceil} \prec R_{\lceil c/2 \rceil + 1}$}
% \end{figure} 
\end{itemize}
\smallskip
\item[Case~II.]
    There are $d$ pairwise $\prec$-crossing paths in $\mathcal{R}$.

Pick any path $R_0$ out of these $d$ paths. In $Z:=\bigcup_{R\in\ca R, R\not=R_0}E(R)$ there are at least $d-1$ edges 
which $\prec$-cross some edge of $R_0$, and so at least $(d-1)/n$ of them cross the same edge $e\in E(R_0)$.
Having $e=u_1u_2$, $u_1\prec u_2$, and $f=v_1v_2\in E(X)$ such that $e$ and $f$ $\prec$-cross, we say that $v_1$ is the {\em inside}
vertex of $f$ if $u_1\prec v_1\prec u_2$, and then $v_2$ is the {\em outside} vertex.
By the pigeonhole principle, there is a set $Z'\subseteq Z$ of $d'=|Z'|\geq(d-1)/n^2$ edges $\prec$-crossing $e$ such that their inside vertices
belong to the same copy of $S_b$ in~$X$.

The outside vertices of the edges of $Z'$ belong to at most two other copies of $S_b$ in~$X$ (determined by a neighbourhood in the path $Q$),
and each is before of after $e$ in~$\prec$.
By the pigeonhole principle again, and without loss of generality, there is a set $Z''\subseteq Z'$ of size $|Z''|\geq\frac12\cdot\frac12d'=d'/4$,
such that also the outside vertices of the edges of $Z''$ belong to the same copy of $S_b$ in~$X$, and they all lie after $e$ in~$\prec$.
See \Cref{fig:crossi}.
Moreover, by \Cref{cor:sameord} (the ordering claimed therein), the edges in $Z''$ must pairwise $\prec$-cross.

\begin{figure}[th]
    \centering
\begin{tikzpicture}[scale=0.85]\small
    \node (w) [style={circle,fill,inner sep=0pt, minimum size=0.12cm}, label=below:$u_1$] at (0, 0) { };
    \node (ww)[style={circle,fill,inner sep=0pt, minimum size=0.12cm}, label=below:$u_2$] at (6, 0) { };
    \node (x)[style={opacity=0, label}, label=below:$e$] at (0.5, 1) {};
    \path (w) edge[thick, bend left = 40, color=red] (ww);
    
    \node (S1)[style={circle,fill,inner sep=0pt, minimum size=0.12cm}, label=left:$t_1$] at (1.5,1.7) { };
    \node (A1)[style={circle,fill,inner sep=0pt, minimum size=0.12cm}] at (2,0) { };
    \node (B1)[style={circle,fill,inner sep=0pt, minimum size=0.12cm}] at (2.5,0) { };
    \node (C1)[style={circle,fill,inner sep=0pt, minimum size=0.12cm}] at (3,0) { };
    \node (D1)[style={circle,fill,inner sep=0pt, minimum size=0.12cm}] at (3.5,0) { };
    \node (E1)[style={circle,fill,inner sep=0pt, minimum size=0.12cm}] at (4,0) { };    
    \node (x)[style={opacity=0, label}, label=below:$Z''$] at (6.5, 1) {};
    \path (S1) edge[opacity=0.5] (A1); 
    \path (S1) edge[opacity=0.5] (B1); 
    \path (S1) edge[opacity=0.5] (C1);
    \path (S1) edge[opacity=0.5] (D1);
    \path (S1) edge[opacity=0.5] (E1);

    \node (S2)[style={circle,fill,inner sep=0pt, minimum size=0.12cm}, label=right:$t_2$] at (10.5,1.7) { };
    \draw[opacity=0.5] (S1)--(S2);
    \node (A2)[style={circle,fill,inner sep=0pt, minimum size=0.12cm}] at (8,0) { };
    \node (B2)[style={circle,fill,inner sep=0pt, minimum size=0.12cm}] at (8.5,0) { };
    \node (C2)[style={circle,fill,inner sep=0pt, minimum size=0.12cm}] at (9,0) { };
    \node (D2)[style={circle,fill,inner sep=0pt, minimum size=0.12cm}] at (9.5,0) { };
    \node (E2)[style={circle,fill,inner sep=0pt, minimum size=0.12cm}] at (10,0) { };    
    \path (S2) edge[opacity=0.5] (A2); 
    \path (S2) edge[opacity=0.5] (B2); 
    \path (S2) edge[opacity=0.5] (C2);
    \path (S2) edge[opacity=0.5] (D2);
    \path (S2) edge[opacity=0.5] (E2);

    \path (A1) edge[color=blue, bend left = 40, opacity=60] (A2);
    \path (B1) edge[color=blue, bend left = 40, opacity=60] (B2);
    \path (C1) edge[color=blue, bend left = 40, opacity=60] (C2);
    \path (D1) edge[color=blue, bend left = 40, opacity=60] (D2);
    \path (E1) edge[color=blue, bend left = 40, opacity=60] (E2);
\end{tikzpicture}
    \caption{Case II, with emphasized edge $e$, blue parwise-crossing edges of $Z''$, and $t_1,t_2$ being two copies of the root of~$S_b$.}
    \label{fig:crossi}
\end{figure}
\end{description}

To finish the proof, we set $n=2s$ and $a = R(2s, 4n^2s+1)^{m}$ where $m=2^{n^2-1}$.
Then in \Cref{erdos_lemma} we get $b=R(2s, 4n^2s+1)$, and in \Cref{cor:crossepar} we have $c=2s$ and $d=4n^2s+1$.
In Case I, we then obtain at least $\min(\lfloor c/2\rfloor,\lceil n/2\rceil)=s$ edges of $X\subseteq G$ that pairwise $\prec$-cross.
In Case II, it is at least $d'/4=(d-1)/(4n^2)=s$ such pairwise $\prec$-crossing edges, too.
Edges that pairwise $\prec$-cross obviously must receive distinct colours.
A valid stack layout based on $\prec$ hence needs at least $s$ colours, and since $\prec$ has been arbitrary for the graph~$G$, 
we finally conclude that $\sn(G)\geq s$.
\end{proof}






\section{Bi-colourings of dual hexagonal grids}
%%%%%%%%%%%%%%%%%%%%%%%%%%%%%%%%%%%%%%%%%%%%%%%%%%%%%%%%%%%%%%%%%%%%%%%

A crucial starting step of the previous proof is \Cref{hex_lemma} (in combination with \Cref{erdos_lemma}).
While \Cref{hex_lemma} \cite{hex} is known for long time, it does not seem as widely known as the other two tools,
Propositions~\ref{ramsey} and~\ref{erdos-szekeres}.
In order to keep our exposition elementary and self-contained, we add also a short proof of \Cref{hex_lemma}.
This proof implicitly follows kind of a ``fixed-point'' argument, quite similarly to original \cite{hex}, 
but stays on purely combinatorial side of the problem.

\begin{proof}[Alternative proof of \Cref{hex_lemma} \cite{hex}]
Recall that $V(H_n)=\{1,\ldots,n\}^2$.
We define the following subpaths of $H_n$: the path $P_L$ induced on $\{[1,j]:j=1,\ldots,n\}$ (the {\em left path}),
$P_R$ induced on $\{[n,j]:j=1,\ldots,n\}$ (the {\em right path}), $P_B$ induced on $\{[i,1]:i=1,\ldots,n\}$ (the {\em bottom path}),
and $P_T$ induced on $\{[i,n]:i=1,\ldots,n\}$ (the {\em top path}).
Let $P_R\cup P_T$ be called the {\em far path} of~$H_n$.
See \Cref{fig:hexaalt}.

Assume a connected subgraph $X\subseteq H_n$ such that $[n,n]\not\in V(X)$.
% intersecting both the left and the bottom path of~$H_n$, and $[n,n]\not\in V(X)$.
Let $N(X)$ denote the set of neighbours of $X$ in $H_n-V(X)$, and $Z$ the connected component of $H_n-V(X)$ containing $[n,n]$.
Then we call the set $D:=N(X)\cap Z$ the {\em far boundary} of $X$ in $H_n$, and we observe:
\begin{enumerate}[i)]
\item The far boundary $D$ of $X$ induces a connected subgraph of~$H_n$.
This follows immediately from the fact that $H_n$ is a plane quasi-triangulation (all its bounded faces are triangles).
\item If $X$ intersects both the left and the bottom path, and $X$ is disjoint from the far path of~$H_n$,
then the far boundary $D$ of $X$ intersects both the left and the bottom path, too.
For this, let $a$ be the largest index such that $[1,a]\in V(X)\cap V(P_L)$. 
Then $a<n$ and $[1,a+1]$ is connected with $[n,n]$ by a path in $(P_L-V(X))\cup P_T$, and so $[1,a+1]\in D\cap V(P_L)$.
The argument is symmetric for~$D\cap V(P_B)$.
\end{enumerate}
%
\begin{figure}[hbt]
    \centering
~\raise3ex\hbox{a)~}
\begin{tikzpicture}[scale=0.8]\small
    \tikzstyle{every node}=[draw, shape=circle, minimum size=3pt,inner sep=0pt, fill=black]
\foreach \i in {1,2,3,4,5,6} \foreach \j in {1,2,3,4,5,6} \node at (\i,\j) (v\i\j) {};
\foreach \i in {1,2,3,4,5,6} \draw (v\i1) -- (v\i6);
\foreach \j in {1,2,3,4,5,6} \draw (v1\j) -- (v6\j);
\draw (v15)--(v26) (v14)--(v36) (v13)--(v46) (v12)--(v56) (v11)--(v66) (v21)--(v65) (v31)--(v64) (v41)--(v63) (v51)--(v62);
\draw[thick] (v11)--(v16)--(v66)--(v61)--(v11);
\node[label=left:{$[1,1]~$}] at (v11) {}; \node[label=left:{$[1,n]~$}] at (v16) {};
\node[label=right:{$~[n,1]$}] at (v61) {}; \node[label=right:{$~[n,n]$}] at (v66) {};
\draw[blue, thick, dashed] plot [smooth cycle] coordinates {(0.6,2.8) (2.9,2.7) (3.8,0.7) (4.4,0.7) (5.5,2.5) (4,4.3) (2.9,3.5) (2,5.5) (1.4,4)};
\node[label=left:{\color{blue}$X\quad$}] at (v13) {};
\node[label=left:{\color{red}$D~\,$}] at (v15) {};
    \tikzstyle{every node}=[draw, shape=circle, thick, red, inner sep=2.5pt, fill=none]
\node at (v14) {}; \node at (v15) {}; \node at (v26) {}; \node at (v36) {}; \node at (v35) {}; \node at (v34) {};
\node at (v45) {}; \node at (v55) {}; \node at (v54) {}; \node at (v64) {}; \node at (v63) {}; \node at (v62) {}; \node at (v51) {};
\end{tikzpicture}
\qquad\raise3ex\hbox{b)~}
\begin{tikzpicture}[scale=0.8]\small
    \tikzstyle{every node}=[draw, shape=circle, minimum size=3pt,inner sep=0pt, fill=black]
\foreach \i in {1,2,3,4,5,6} \foreach \j in {1,2,3,4,5,6} \node at (\i,\j) (v\i\j) {};
\foreach \i in {1,2,3,4,5,6} \draw (v\i1) -- (v\i6);
\foreach \j in {1,2,3,4,5,6} \draw (v1\j) -- (v6\j);
\draw (v15)--(v26) (v14)--(v36) (v13)--(v46) (v12)--(v56) (v11)--(v66) (v21)--(v65) (v31)--(v64) (v41)--(v63) (v51)--(v62);
\draw[thick] (v11)--(v16)--(v66)--(v61)--(v11);
% \node[label=left:{$[1,1]~$}] at (v11) {}; 
\node[label=left:{$[1,n]~$}] at (v16) {}; \node[label=right:{$~[n,1]$}] at (v61) {}; \node[label=right:{$~[n,n]$}] at (v66) {};
\draw[red, thick, dashed] plot [smooth cycle] coordinates {(0.6,0.6) (2.4,0.6) (2.4,2.4) (0.6,2.4)};
\draw[blue, thick, dashed] plot [smooth cycle] coordinates {(0.6,2.8) (2.6,2.6) (2.7,0.7) (3.2,0.7) (3.4,3.2) (2.5,3.4) (2.3,5.3) (1.9,5.3) (0.6,3.9)};
\draw[red, thick, dashed] plot [smooth cycle] coordinates {(0.6,4.5) (2.4,5.6) (2.6,3.6) (3.5,3.5) (3.7,0.7) (4.3,0.7)
	(4.5,2.3) (5.3,2.6) (5.3,4.2) (3.3,4.9) (4.5,6.2) (1.9,6.3) (0.6,5)};
    \tikzstyle{every node}=[draw, shape=circle, thick, red, inner sep=2.5pt, fill=none]
\node at (v11) {}; \node at (v12) {}; \node at (v21) {}; \node at (v22) {};
\node at (v15) {}; \node at (v26) {}; \node at (v36) {}; \node at (v46) {};
\node at (v35) {}; \node at (v34) {}; \node at (v44) {}; \node at (v54) {};
\node at (v43) {}; \node at (v53) {}; \node at (v42) {}; \node at (v41) {};
    \tikzstyle{every node}=[draw, shape=circle, thick, blue, inner sep=2.5pt, fill=none]
\node at (v31) {}; \node at (v32) {}; \node at (v33) {};
\node at (v13) {}; \node at (v23) {}; \node at (v14) {}; \node at (v24) {}; \node at (v25) {};
\node[label=left:{\color{blue}$Y_2\quad$}] at (v13) {};
\node[label=left:{\color{red}$Y_1\quad$}] at (v11) {};
\node[label=left:{\color{red}$Y_3\quad$}] at (v15) {};
\end{tikzpicture}

    \caption{(a) The dual hexagonal grid $H_n$ with $n=6$. A connected subgraph $X\subseteq H_n$ is delimited by the dashed blue line,
	and the (connected) far boundary $B$ of $X$ is emphasized with red circles.
	(b)~A~sequence of monochromatic components $Y_1,Y_2,Y_3$, s.t.\ $Y_3$ must contain a red path of length~$n$.}
    \label{fig:hexaalt}
\end{figure}

Consider a colouring of $V(H_n)$ by red and blue colours.
Let a connected component of the subgraph of $H_n$ induced by the red (blue) vertices be called a {\em red (blue) component},
and a {\em monochromatic component} be either red or blue.
We define a sequence of monochromatic components $Y_1,\ldots,Y_p\subseteq H_n$ such that $[1,1]\in V(Y_1)$, 
each $Y_i$ for $i\in\{1,\ldots,p\}$ intersects both the left and the bottom path of~$H_n$, and $Y_p$ intersects the far path of~$H_n$:
simply, $Y_{i+1}$ for $i\geq1$ is the monochromatic component of $H_n$ containing the far boundary $D_i$ of~$Y_i$ (Fig.~\ref{fig:hexaalt}b).
This is consistent; $D_i$ indeed is contained in one component by (i) and, inductively, $Y_{i+1}$ for $i<p$ intersects the left and the bottom path by (ii).
Finally, since $Y_p$ also intersects the top or the right path of $H_n$, there is a monochromatic path of length at least $n$ in $Y_p$.
\end{proof}





\section{Conclusion}
%%%%%%%%%%%%%%%%%%%%%%%%%%%%%%%%%%%%%%%%%%%%%%%%%%%%%%%%%%%%%%%%%%%%%%%

We have provided a short elementary proof of \Cref{thm:main}.
Although the original proof in \cite{dujmovic2021stack} is not very long or difficult, by carefully rearranging the arguments
we have succeeded in eliminating some technical steps of the proof in \cite{dujmovic2021stack}
and, in particular, resolved the case of pairwise crossing paths in a direct short way.
Briefly explaining, our proof skips initial technical parts of \cite{dujmovic2021stack} preceding the use of
\Cref{erdos-szekeres} (Erd\H{o}s--Szekeres) and readily applies \Cref{erdos-szekeres} and \Cref{hex_lemma} in a way similar to \cite{dujmovic2021stack},
and then it concludes by \Cref{ramsey} (Ramsey) in which both outcomes straightforwardly lead to a large set of pairwise crossing edges,
thus avoiding other technical steps needed in \cite{dujmovic2021stack} mainly at the end of the arguments.

The presented proof is based on the Bachelor's thesis of the second author \cite{Straka2023thesis,straka2023stack}.

% We have provided a short self-contained proof of \Cref{thm:twwplanar}.
% While the proved bound is not the best currently possible, % cf.~\cite{DBLP:journals/corr/abs-2210-08620},
% the proof given here is way much simpler than those in \cite{DBLP:journals/corr/abs-2205.05378,DBLP:journals/corr/abs-2210-08620}.
% While sacrificing a bit of simplicity of the given proof, we can also give a better upper bound
% of~$9$ (thus matching \cite{DBLP:journals/corr/abs-2205.05378}), but
% we are so far not sure whether a similarly simplified proof can be given for the upper bound of $8$ as in~\cite{DBLP:journals/corr/abs-2210-08620}.





\bibliography{stackqueue}

\end{document}
%%%%%%%%%%%%%%%%%%%%%%%%%%%%%%%%%%%%%%%%%%%%%%%%%%%%%%%%%%%%%%%%%%%%%%%%%%%%%%%%%%%%%%%%%%%%%%%%%%%%%%%%%%%%%%%%%%%%%%%%%%%%%
%%%%%%%%%%%%%%%%%%%%%%%%%%%%%%%%%%%%%%%%%%%%%%%%%%%%%%%%%%%%%%%%%%%%%%%%%%%%%%%%%%%%%%%%%%%%%%%%%%%%%%%%%%%%%%%%%%%%%%%%%%%%%









