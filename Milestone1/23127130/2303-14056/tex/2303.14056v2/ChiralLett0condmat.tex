%\documentclass[aps,pre,onecolumn,superscriptaddress,showpacs,amsmath,amssymb]{revtex4}
%\documentclass[aps,pre,onecolumn,superscriptaddress,amsmath,amssymb]{revtex4-2}
\documentclass[aps,prl,twocolumn,superscriptaddress,amsmath,amssymb]{revtex4-2}


\usepackage[utf8]{inputenc}
\usepackage{amsmath,amsthm}
\usepackage{graphicx}
\usepackage[colorlinks = true,
            linkcolor = blue,
            urlcolor  = blue,
            citecolor = blue,
            anchorcolor = blue]{hyperref}
%\usepackage{bbold}
\usepackage{braket}
\usepackage{xcolor}
%\usepackage{easyReview}
%TMP
%\usepackage[title]{appendix}
\raggedbottom


\newcommand{\U}{\ensuremath{\mathcal{U}}}
\newcommand{\J}{\ensuremath{\mathcal{J}}}
\newcommand{\x}{{x}}
\newcommand{\y}{{y}}
\newcommand{\z}{{z}}
\renewcommand{\S}{\mathbf{S}}
\renewcommand{\L}{\mathbf{L}}
\newcommand{\I}{\mathbf{I}}
\newcommand{\T}{\mathbf{T}}

\newcommand{\V}{\mathcal{V}}
\newcommand{\lhalf}{{\textstyle\frac{\lambda}{2}}}
\newcommand{\mhalf}{{\textstyle\frac{\mu}{2}}}
\renewcommand{\d}[1]{\ensuremath{\operatorname{d}\!{#1}}}
\newcommand{\real}{{\rm Re}}
\newcommand{\imag}{{\rm Im}}
%\newcommand{\be}{\begin{equation}}
%\newcommand{\ee}{\end{equation}}
\newcommand{\bea}{\begin{eqnarray}}
\newcommand{\eea}{\end{eqnarray}}
\newcommand{\ZZ}{\mathbb{Z}}
\newcommand{\RR}{\mathbb{R}}
\newcommand{\CC}{\mathbb{C}}

\newcommand{\tOmega}{\widetilde\Omega}
\global\long\def\l{\la}

\newcommand{\bfvec}[1]{\mathbf{#1}}

\newcommand{\norm}[1]{\left\lVert#1\right\rVert}
\newcommand{\trb}{\mathop{\mathrm{tr}_{1,N}}\limits}
\newcommand{\trzero}{\mathop{\mathrm{tr}_{\mathcal{H}_0}}\limits}
\newcommand{\truno}{\mathop{\mathrm{tr}_{\mathcal{H}_1}}\limits}
\global\long\def\ga{\gamma} \global\long\def\de{\delta}
\global\long\def\De{\Delta} \global\long\def\Ga{\Gamma}
\global\long\def\th{\theta}
\global\long\def\th{\theta}

\global\long\def\ra{\rightarrow}

\global\long\def\dL{\mathbb{L}}
\global\long\def\dF{\mathbb{F}}

\global\long\def\ell#1{\theta_{#1}}
\global\long\def\bell#1{\tilde\theta_{#1}}

\global\long\def\la{\lambda} \global\long\def\ka{\kappa}
\global\long\def\si{\sigma}
\global\long\def\vfi{\varphi}
\global\long\def\s{\sigma}

\global\long\def\ro{\rho} \global\long\def\Om{\Omega}
\global\long\def\eps{\epsilon}
\global\long\def\al{\alpha}
\global\long\def\be{\beta}
\global\long\def\ga{\gamma} \global\long\def\de{\delta}


\global\long\def\no{\nonumber}

\theoremstyle{thm@}
\newtheorem*{pro}{Proposition}
\newtheorem{lem}{Lemma}
\newtheorem{cor}{Corollary}

\theoremstyle{remark}
\newtheorem*{rem}{Remark}

\newcommand{\rev}[1]{{\color{red}#1}}
%\def\multiplet{single particle multiplet}
\def\multiplet{single particle subspace}

\newcommand{\ak}[1]{{\color{blue}#1}}




\global\long\def\braket#1#2{\left\langle #1|#2\right\rangle }


%\renewcommand{\vec}[1]{\mathbf{#1}}

\def\pp{\mathbf{p}}
\def\qq{\mathbf{q}}

\newcommand{\kind}{l}


\newcommand{\eE}{\mathsf{e}}

% from macro.lib
\let\Ph=\phi
\let\PH=\Phi
\def\i{{\rm i}}
\DeclareMathOperator{\re}{e}
\let\e=\varepsilon
\theoremstyle{plain}
\newtheorem{theorem}{Theorem}
\newtheorem*{theorem*}{Theorem}
\def\pv{\mathbf{p}}
\def\qv{\mathbf{q}}
\def\Fv{\mathbf{F}}
\def\Hv{\mathbf{H}}
\def\epc{\, ,}
\def\epp{\, .}
\DeclareMathOperator{\sign}{sign}
\let\p=\pi
\def\rd{{\rm d}}



\usepackage{amsfonts}
\usepackage{amsmath}
\usepackage{amsthm}
\usepackage{amssymb}
\usepackage{stmaryrd}
\usepackage{amscd}
\usepackage{eucal}
\usepackage{dsfont}
\usepackage{bm}
\usepackage{graphicx}


%\renewcommand{\nu}{\ensuremath{\mathbf{n}(\mathbf{u})}\xspace}  % the normal vector at pixel location \V{u}
\newcommand{\pu}{\ensuremath{\mathbf{p}(\mathbf{u})}\xspace}   % the 3d point correspoinding the pixel \V{u}
\newcommand{\du}{\ensuremath{d(\mathbf{u})}\xspace}  
\newcommand{\zu}{\ensuremath{z(\mathbf{u})}\xspace}
\newcommand{\eu}{\ensuremath{\mathbf{e}(\mathbf{u})}\xspace}
\newcommand{\up}{\ensuremath{\V{u}_{\V{p}}}\xspace}
\newcommand{\tup}{\ensuremath{\tilde{\V{u}}_{\V{p}}}\xspace}

\newcommand{\oz}{\ensuremath{\Omega_z}\xspace}  
\newcommand{\on}{\ensuremath{\Omega_n}\xspace}
\newcommand{\Nu}{\ensuremath{\mathcal{N}(\V{u})}\xspace}

\renewcommand{\ni}{normal integration\xspace}
\newcommand{\NI}{Normal Integration\xspace}
\newcommand{\dpe}{discrete Poisson's equation\xspace}
\newcommand{\Dpe}{Discrete Poisson's equation\xspace}


\newcommand{\z}{\ensuremath{\V{z}}\xspace}
\newcommand{\zs}{\ensuremath{\V{z}^*}\xspace}
\newcommand{\rz}{\ensuremath{\red{\V{z}}}\xspace}
\newcommand{\zt}{\ensuremath{\V{z}_{t}}\xspace}
\newcommand{\zto}{\ensuremath{\V{z}_{t+1}}\xspace}
\newcommand{\R}{\ensuremath{\mathbb{R}}\xspace}
\newcommand{\fz}{\ensuremath{f(\V{z})}\xspace}

\newcommand{\rt}{\ensuremath{\V{r}_{t}}\xspace}
\newcommand{\rto}{\ensuremath{\V{r}_{t+1}}\xspace}


\newcommand{\dup}{\ensuremath{\V{D}_u^{+}}\xspace}
\newcommand{\dun}{\ensuremath{\V{D}_u^{-}}\xspace}
\newcommand{\dvp}{\ensuremath{\V{D}_v^{+}}\xspace}
\newcommand{\dvn}{\ensuremath{\V{D}_v^{-}}\xspace}
\newcommand{\nx}{\ensuremath{\V{n}_x}\xspace}
\newcommand{\ny}{\ensuremath{\V{n}_y}\xspace}
\newcommand{\nz}{\ensuremath{\V{n}_z}\xspace}
\newcommand{\Nz}{\ensuremath{\V{N}_z}\xspace}

\newcommand{\ft}{\ensuremath{F(\red{\V{z}};\V{z}_t)}\xspace}
\newcommand{\ftt}{\ensuremath{F(\V{z}_t;\V{z}_t)}\xspace}
\newcommand{\fto}{\ensuremath{F(\V{z}_{t+1};\V{z}_t)}\xspace}

\newcommand{\dpu}{\ensuremath{\partial_u \V{p}}\xspace}
\newcommand{\dpv}{\ensuremath{\partial_v \V{p}}\xspace}

\renewcommand{\u}{\ensuremath{\V{u}}\xspace}
\newcommand{\dzdu}{\ensuremath{\partial_u z}\xspace}
\newcommand{\dzdv}{\ensuremath{\partial_v z}\xspace}
\newcommand{\dztdu}{\ensuremath{\partial_u \tilde{z}}\xspace}
\newcommand{\dztdv}{\ensuremath{\partial_v \tilde{z}}\xspace}
\newcommand{\dzpdu}{\ensuremath{\partial_{u}^{+} z}\xspace}
\newcommand{\dzpdv}{\ensuremath{\partial_{v}^{+} z}\xspace}
\newcommand{\dzndu}{\ensuremath{\partial_{u}^{-} z}\xspace}
\newcommand{\dzndv}{\ensuremath{\partial_{v}^{-} z}\xspace}

\newcommand{\dzpduv}{\ensuremath{\partial_{\{u,v\}}^{+} z}\xspace}
\newcommand{\dznduv}{\ensuremath{\partial_{\{u,v\}}^{-} z}\xspace}
\newcommand{\dzduv}{\ensuremath{\partial_{\{u,v\}} z}\xspace}

\newcommand{\dupz}{\ensuremath{\Delta_{u}^{+} z}\xspace}
\newcommand{\dunz}{\ensuremath{\Delta_{u}^{-} z}\xspace}
\newcommand{\dvpz}{\ensuremath{\Delta_{v}^{+} z}\xspace}
\newcommand{\dvnz}{\ensuremath{\Delta_{v}^{-} z}\xspace}

\newcommand{\nuv}{\ensuremath{\V{n}(u,v)}\xspace}
\newcommand{\zuv}{\ensuremath{z(u,v)}\xspace}
\newcommand{\puv}{\ensuremath{\V{p}(u,v)}\xspace}

\newcommand{\halfpi}{\ensuremath{\pm {\pi \over 2}}\xspace}


\newcommand{\curve}{\ensuremath{\mathbb{S}}\xspace}
\newcommand{\zenith}{zenith\xspace}
\newcommand{\surface}{\ensuremath{\mathcal{M}}\xspace}
\newcommand{\visibility}{\ensuremath{\Phi_{i}}\xspace}
\newcommand{\point}{\ensuremath{\V{x}}\xspace}
\newcommand{\normal}{\ensuremath{\V{n}}\xspace}
\newcommand{\tangent}{\ensuremath{\V{t}}\xspace}
\newcommand{\cameraNum}{\ensuremath{C}\xspace}
\newcommand{\cameraCenter}{\ensuremath{\V{o}_{i}}\xspace}
\newcommand{\viewDirection}{\ensuremath{\V{v}}\xspace}
\newcommand{\batchsize}{\ensuremath{P}\xspace}
\newcommand{\mask}{\ensuremath{O}\xspace}
\newcommand{\projectedTangentVector}{projected tangent vector\xspace}
\newcommand{\projectedTangentVectors}{projected tangent vectors\xspace}
\newcommand{\stackedTangentVectors}{\ensuremath{\V{T}(\point)}\xspace}
\newcommand{\diligentmv}{\mbox{DiLiGenT-MV}\xspace}
\newcommand{\diligent}{DiLiGenT}
\newcommand{\loss}{\mathcal{L}\xspace}
\newcommand{\opticalAxis}{\ensuremath{\V{e}_{z}\xspace}}
\newcommand{\opticalAxisViewI}{\ensuremath{\V{e}_{z_{i}}}\xspace}
\newcommand{\opticalAxisMatrix}{\ensuremath{\V{C}}\xspace}
\newcommand{\ms}{Mumford-Shah integrator\xspace}
\newcommand{\made}{MADE\xspace}

\newcommand{\pandora}{\mbox{PANDORA}\xspace}
\newcommand{\psnerf}{\mbox{PS-NeRF}\xspace}
\newcommand{\sdps}{\mbox{SDPS}\xspace}
\newcommand{\uanet}{\mbox{UA-MVPS}\xspace}
\newcommand{\rmvps}{\mbox{R-MVPS}\xspace}
\newcommand{\bmvps}{\mbox{B-MVPS}\xspace}
\newcommand{\volsdf}{\mbox{VolSDF}\xspace}
\newcommand{\unisurf}{\mbox{UNISURF}\xspace}


\newcommand{\mvas}{MVAS\xspace}

\newcommand{\tsc}{\mbox{TSC}\xspace}

\newcommand{\pointOne}{\ensuremath{\point_1}\xspace}
\newcommand{\pointTwo}{\ensuremath{\point_2}\xspace}
\newcommand{\pointsetOne}{\ensuremath{\chi_{1}}\xspace}
\newcommand{\pointsetTwo}{\ensuremath{\chi_{2}}\xspace}
\newcommand{\fscoreThreshold}{\ensuremath{\tau}\xspace}
\newcommand{\chamferDist}{\ensuremath{d(\pointsetOne, \pointsetTwo)}\xspace}
\newcommand{\precision}{\ensuremath{\mathcal{P}}\xspace}
\newcommand{\recall}{\ensuremath{\mathcal{R}}\xspace}
\newcommand{\fscore}{\ensuremath{\mathcal{F}}\xspace}

\newcommand{\phaseangle}{\ensuremath{\hat{\phi}}\xspace}
\newcommand{\azimuthangle}{\ensuremath{\phi}\xspace}

\newcommand{\colorbar}[3]{
\begin{tabular}[t]{@{}l@{}l@{}}
	\includegraphics[height=#1\linewidth,width=0.5em]{colorbar.pdf} & 
	\begin{tabular}[b]{@{}l}
		#2 \\ [#3pt]
		$0$
	\end{tabular}
\end{tabular}
}



 \allowdisplaybreaks

\begin{document}

\title{Chiral basis for qubits}
\author{Vladislav Popkov}
 \affiliation{Department of Physics,
  University of Wuppertal, Gaussstra\ss e 20, 42119 Wuppertal,
  Germany}
\affiliation{Faculty of Mathematics and Physics, University of Ljubljana, Jadranska 19, SI-1000 Ljubljana, Slovenia}
\author{Xin  Zhang}
\affiliation{Beijing National Laboratory for Condensed Matter Physics, Institute of Physics, Chinese Academy of Sciences, Beijing 100190, China}
\author{  Frank G\"ohmann} 
\affiliation{Department of Physics,
 University of Wuppertal, Gaussstra\ss e 20, 42119 Wuppertal,
  Germany}
\author{  Andreas Kl\"umper} 
\affiliation{Department of Physics,
 University of Wuppertal, Gaussstra\ss e 20, 42119 Wuppertal,
  Germany}


\begin{abstract}
%\rev{possibly to amend..}
We propose a  qubit basis composed of transverse spin helices with
kinks. This chiral basis, in contrast to the usual computational basis,
possesses distinct topological properties and is particularly suited for describing
quantum states with nontrivial topology. By choosing appropriate parameters,
operators containing transverse spin components, such as $\sigma_n^x$ or
$\sigma_n^y$, become diagonal in the chiral basis, facilitating the study of
problems focused on transverse spin components. As an application, we study
the decay of  the transverse polarization  of a spin helix in the XX model, which has
been measured in recent cold atom experiments. We obtain an explicit universal 
function describing the relaxation of helices of arbitrary wavelength.
\end{abstract}
\maketitle



\textbf{ Introduction.--} 
A proper choice of basis often is the crucial first step
towards success. For example,  the   modes of the harmonic oscillator 
 are best described by the  coherent state basis. The use of wavelets is well suited for describing signals
confined in space or time \cite{Wavelets}, while the Fourier basis is natural
for solving linear differential equations with translational invariance in
space and time.

For qubits,  i.e.,  quantum systems with spin-$1/2$ local degrees of freedom, the
most widely used basis is the computational basis, which is composed of the
eigenstates of the $\sigma^z$ operator, i.e., $\sigma^z \binom{1}{0} =
\binom{1}{0}$ and $\sigma^z \binom{0}{1}= -\binom{0}{1}$, on every site. The
advantages of the computational basis are its factorized structure,
orthonormality, and $U(1)$-symmetry ``friendliness". 
%In particular, the latter
%property allows the grouping of basis vectors with an equal number of
%spin-down arrows $m$, forming an invariant subset for any $m$ under
%$U(1)$-invariant dynamics, such as the XXZ model dynamics. 
The computational
basis is well-suited for multi-qubit states that  are eigenstates of the total spin-$z$ magnetization, e.g., the
ground state of $XXZ $ model.  It is also most appropriate for studying correlation functions that do not change the total 
magnetization, such as $\langle \sigma_{n}^{+} \sigma_{m}^{-}\rangle$ or  $\langle \sigma_{n_1}^z \sigma_{n_2}^z \ldots \sigma_{n_k}^z \rangle$.

However, the  computational basis appears poorly equipped to
describe states with nontrivial topology, such as chiral states, current-carrying
states, or states with windings. One prominent example is the spin-helix
state in a 1-dimensional spin chain,
\begin{align}
&\ket{\Psi(\al_0,\eta)} = \bigotimes_n \ket{\phi(\al_0 + n\eta)},  \label{def:SHS}
\end{align}
where $\ket{\phi(\al)}$ describes the state of a qubit, while $+n\eta$
represents a linear increase in the qubit phase along the chain, under a
proper model-dependent parametrization. Thanks to their factorizability, spin
helices (\ref{def:SHS}) are straightforward to prepare in experimental setups
that allow for adjustable spin exchange, such as those involving cold
atoms \cite{SHS-Ketterle,2020NatureSpinHelix,2021KetterleTransverse}. These helices  possess nontrivial properties as
evidenced by both experimental
\cite{SHS-Ketterle,2020NatureSpinHelix,2021KetterleTransverse}
%\cite{2014-SHS-Experimental}
and theoretical \cite{SHS-Phantom,Phantom-Long,ChiralBA,SHS-Hydro}
studies.  It was suggested that quantum states with helicity are
  even better protected from noise than the ground state, and that the
  helical protection extends over intermediate timescales \cite{2023Posske}.

Since the spin helix state (\ref{def:SHS}) is not an eigenstate of the operator of the total magnetization,  it is not confined to a single $U(1)$ block, but is given by a sum over all the blocks, with fine-tuned coefficients, as shown in (\ref{eq:SHSinU(1)}) and (\ref{eq:PBRstates}), even for spatially homogeneous spin helices ($\eta=0$ in (\ref{def:SHS})). A simple shift of a helix phase $\al_0 \rightarrow \al_0 + const$ in (\ref{def:SHS}) gives a linearly independent state with the same qualitative properties (winding, current, etc.). However, to represent such a shift in the standard computational basis,  all the fine-tuned expansion coefficients must be changed in a different manner.

Our goal is to introduce an alternative basis, all components of which are chiral themselves and thus ideally tailored for the description of chiral states.  This basis consists of helices and helices with kinks (phase dislocations), and it provides another block hierarchy based on the number of kinks (instead of the number of spins down in the computational basis). Unlike the usual basis, the chiral basis is intrinsically topological.

In the following we introduce the chiral  basis and demonstrate its application to a problem of spin-helix state decay under XX dynamics.



\textbf{Chiral multi-qubit basis. --}
Our starting point is the operator
\begin{align}
&V = \sum_{k=1}^{N/2} \left(  \si_{2k-1}^x   \si_{2k}^y
- \si_{2k}^y   \si_{2k+1}^x
\right),
\label{eq:Uoper}
\end{align}
defined for an even number of qubits $N$.  It has remarkably simple factorized eigenstates,  namely,
a state 
\begin{align}
\Psi =2^{-N/2}  \vfi_1 \otimes \zeta_1 \otimes  \vfi_2 \otimes \zeta_2 \otimes \ldots  \otimes \vfi_{N/2} \otimes \zeta_{ N/2}  
\label{eq:Psi}
\end{align}
is an eigenstate of  $V$,  provided that all odd (even) qubits are polarized in  positive or negative $x-$ ($y-$) direction,  amounting to
\begin{align}
&\bra{\vfi_j} = (1,\pm 1), \quad \bra{\zeta_j} = (1,\pm \i). 
\label{eq:Psi1}
\end{align}
Thus,  at each link $n,n+1$ the qubit polarization changes by an angle $+\pi/2$ or $-\pi/2$ in the XY-plane.  Each link with clockwise (anticlockwise) rotation $+\pi/2$  $(-\pi/2)$ adds $+1$ $(-1)$ to the eigenvalue of $V$,  so that  
$V \Psi = (N-2M) \Psi$,
where $M$ is the number of ``anticlockwise" links,  further referred to as  \textit{kinks}.  Each $\Psi$ in (\ref{eq:Psi}) is  
fully characterized by the kink positions $n_1,\ldots ,n_M$ (the ``anticlockwise" links between $n_k,n_k+1$),
and the polarization of the first qubit
$\vfi_1$(in  $\pm x$  direction,   distinguished by the bimodal parameter  $\ka = \pm$).
We denote this state as  $\ket{\ka; n_1,n_2,  \ldots n_M}$.

The set of $V$ eigenstates 
\begin{align}
\{\Psi \} \equiv  \{ (-\i)^{\sum_{k} n_k} ) \ket{\ka; n_1,n_2,  \ldots n_M} \}_{\ka=\pm,M}, 
%\quad 1\leq n_1 <n_2 < \ldots < n_M\leq N 
\label{eq:PsiBasis}
\end{align}
(the phase factor is introduced for  convenience) is complete and forms an orthonormal basis of the Hilbert space by construction.  
Finally,  due to periodic boundary conditions in a system with  $N=4N'$ ($N'$  integer)  qubits the number of kinks must be 
 even, while it must be odd if   $N=4N'+2$,
\begin{align}
&V \Psi = (N-2M) \Psi, 
\left[
\begin{array}{cc}
&M=0,2,\ldots,  N, \  \mbox{if} \ N=4N'\\ 
&M=1,3,\ldots ,N-1, \, \mbox{if} \, N=4N'+2
\end{array}
\right.
\label{eq:Psi4}
\end{align}


\textit{Remark 1.} The chiral basis vectors have a topological nature; namely a single kink cannot be removed from (or added to)  a periodic chain.  In an open chain a single kink can only be removed or added at a boundary.  

\textit{Remark 2.} Applying $\sigma_n^z$ in a kink-free zone creates a kink
pair at the neighbouring positions $n-1$ and $n$.  Applying a
string of operators $\sigma_n^z \sigma_{n+1}^z \ldots \sigma_{n+k}^z$ in a
kink-free zone creates two kinks at a distance of $k+1$,  e.g.,
\begin{align}
&\ket{+;1,k+2} = \si_2^z \ \si_3^z \ldots  \si_{k+2}^z \ket{+},\no
\end{align}
where $\ket{+}$ is a perfect spin helix of type (\ref{def:SHS}),
 \begin{align}
&\ket{+}  =\ket{ \rightarrow \downarrow \leftarrow  \uparrow \rightarrow \downarrow \leftarrow \uparrow \ldots}. \label{def:vac}
\end{align}
Here the arrows depict the polarization of the qubits in the XY-plane, e.g., 
$\uparrow,\downarrow $ depicts a qubit $\zeta_j$ (\ref{eq:Psi1}) with polarization along the $y$ axis.

\textit{Remark 3.} The connection between the chiral basis and the standard
computational basis is highly nontrivial. For example, the chiral vacuum state (\ref{def:vac})
is expanded in terms of the computational basis as 
\begin{align}
&\ket{+}=2^{-\frac{N}{2}}\sum_{n=0}^N (-\i)^n\xi_n,\label{eq:SHSinU(1)}\\
&\xi_n\!=\!\frac{1}{n!}\, \sum_{\kind_1,\ldots ,\kind_n=1}^{N}\,
\i^{\kind_1+\ldots +\kind_n}\,
\si_{\kind_1}^{-}\ldots  \si_{\kind_n}^{-}\, \binom{1}{0}^{\otimes_N},\label{eq:PBRstates}
\end{align}
see \cite{SHS-Phantom} for a proof.


 Next we will  explore two applications of the chiral basis. Firstly, we will use it to classify the eigenstates of the XX model according 
to the number of kinks. 
 Secondly, we will apply it to describe the temporal decay of the transverse magnetization  of a spin-helix  in the XX model.

\textbf{Eigenstates of the XX model within the chiral sectors--}
The crucial observation is that  $V$ commutes with the Hamiltonian of the XX model,
\begin{align}
&[V,H]=0, \label{eq:commVH}\\
&H=\sum_{n=1}^{2N'} \si_n^x \si_{n+1}^x+ \si_n^y \si_{n+1}^y, \quad \vec{\sigma}_{2N'+1}\equiv \vec{\sigma}_1. \label{def:XX0}
\end{align}
Consequently $H$ 
can be block-diagonalized within 
distinct topological sectors,  each   containing states with a fixed number of kinks $M$. 
For one-kink states $M=1$,  we obtain
\begin{align*}
&H \ket{\ka;n} = 2 \ket{\ka;n-1}+ 2 \ket{\ka;n+1}, \quad n \neq 1,N, \\
&H \ket{\ka;1} = -2 \ket{-\ka;N}+ 2 \ket{\ka;2}, \\
&H \ket{\ka;N} = -2 \ket{-\ka;1}+ 2 \ket{\ka;N-1}. 
\end{align*}
The $2N$ eigenstates of $H$ belonging to the one-kink 
subspace are given by the ansatz
\begin{align}
& \ket{\mu_1(p)}= \frac{1}{\sqrt{2N}}\sum_{n=1}^N  \re^{\i p n} \left(\ket{+;n}-\re^{\i p N} \ket{-;n}\right),  \label{eq:kink1Eigen}\\
&\re^{\i p N} = \pm 1,
\end{align}
where $p$ is a chiral analogue of a  quasi-momentum.  
The diagonalization of $H$  within a subspace with an arbitrary number of kinks is performed using the coordinate Bethe Ansatz.
The complete set of XX eigenvectors in the chiral basis is given by the
following Theorem:

\begin{theorem*}
%\label{theorem}
 The eigenstates of the XX Hamiltonian are $\ket{\mu_M(\bfvec{p})}$ with even kink number $M=0,2,4,\ldots,N$
 for $N/2$ even, and odd $M=1,3,5,\ldots,N-1$ for $N/2$ odd. Each eigenstate
  $\ket{\mu_M(\bfvec{p})}$ is characterized by an $M$-tuple of chiral quasi-momenta
 $\bfvec{p}=(p_1, p_2, \ldots,p_M)$, where the $p_j$ all satisfy
 either $\re^{\i p_j N}=1$ or $\re^{\i p_j N}=-1$.  The eigenvalues are given by $E_{\bfvec{p}} = 4\sum_{j=1}^{M} \cos p_j$,
and the eigenstates are
 \begin{align}
& \ket{\mu_M(\bfvec{p})}=\sum_{1\leq n_1 < \ldots< n_M \leq N} \chi_{n_1 n_2 \ldots n_M} (\bfvec{p}) \left(
\ket{u;n_1,n_2,\ldots, n_M} \right. \no \\
&\left.  - \re^{\i p_1 N}   \ket{u+2;n_1,n_2,\ldots, n_M} \right),  \label{eq:mu}\\
&\chi_{n_1 n_2 \ldots n_M} = \frac{1}{\sqrt{2 N^M} }
\sum_{Q} (-1)^Q \,\re^{\i \sum_{j=1}^M n_j  p_{Q_j} }, \label{eq:chi}\\
&\ket{u; n_1, \ldots , n_M} = (-\i)^{\sum_{j=1}^M n_j}
\bigotimes_{k=1}^{n_1}   
\psi_k(u)\bigotimes_{k=n_1+1}^{n_2}   
\psi_k(u+2)  \no \\
&\cdots  \bigotimes_{k=n_M+1}^{N}   \psi_k(u+2M),  
 \label{eq:Mshock}\\
& \psi_k(u)=\frac{1}{\sqrt{2}}\binom {1}{\re^{\frac{\i  \pi}{2}(k- u)}}, 
\quad  \psi_k(u+4) = \psi_k(u), \label{eq:0shock}
\end{align}
where $Q$ in (\ref{eq:chi}) is a permutations of numbers $1,2,\ldots M$.
The states $\ket{\mu_M(\pp)}$ are orthonormal, 
 $\braket  {\mu_M(\pp)}  {\mu_{M'}(\pp')} =\de_{\bfvec{p},\bfvec{p'}}  \de_{M,M'}$.
\end{theorem*}
The proof of the Theorem is given in \cite{Supp}.

Some clarifications are necessary here.  First, the states (\ref{eq:Mshock})
are a generalization of those in (\ref{eq:Psi}) by an additional rotation of
all qubits by the same angle $\pi (1 -u)/2$ in the XY-plane. Setting $u=1$ yields
(\ref{eq:Psi}).  The extra degree of freedom originates from the $U(1)$
symmetry of the XX model.

Second,  the XX eigenstates in the chiral topological
basis  closely resemble those for the usual
computational basis \cite{1993ColomoXX0basis}, where the number of spins up plays
the role of the number of kinks.  In particular, the wave function's amplitudes
(\ref{eq:chi}) have the familiar form of Slater determinants.  





\textbf{Spin-helix  decay in the XX model--}
Now we apply our chiral basis to  study the time evolution of a transverse 
spin-helix state magnetization profile,  experimentally measured in \cite{SHS-Ketterle}.
Namely,  we  are interested in the time evolution of the
one-point correlation functions of a spin-helix state,  generated by the XX Hamiltonian (\ref{def:XX0})
\begin{align}
&  \langle \si_{n}^\al(t)\rangle_Q = \bra{\Psi_Q} \re^{\i H t}  \si_n^\al \re^{-\i H t} \ket{\Psi_Q},\label{eq:Sn-alpha}\\
&\ket{\Psi_Q}= \frac{1}{\sqrt{2^N}}\bigotimes_{n=1}^{N}\binom {1}{\re^{\i nQ}},
\label{eq:SHSQ}
\end{align}
where the wavevector $Q$ satisfies the commensurability condition
\begin{align}
&Q N = 0 \ \mod \ 2\pi. \label{def:commensurate}
\end{align}

It can be shown that the  magnetization profile satisfies $\langle
\si_{n+1}^\pm(t) \rangle_Q= \re^{\pm \i Q } \langle \si_n^\pm(t) \rangle_Q$ as well as a self-similarity 
property \cite{2023SHSdecay,Supp},
entailing that
\begin{align}
& \langle \si_{n}^x(t)\rangle_Q= S_N(t \cos Q) \cos (Qn),    \label{eq:Sx}\\
& \langle \si_{n}^y(t)\rangle_Q= S_N(t \cos Q) \sin (Qn),   \label{eq:Sy} \\
& \langle \si_{n}^z(t)\rangle_Q=0,\qquad  \forall n. \label{eq:Sz}
\end{align}
Here $S_N(t)= \langle \si_{1}^x(t) \rangle_0$. 
This means,  that the complete information about the one-point correlation functions
$\langle \si_{n}^\al(t) \rangle_Q$  is given by the single
real-valued function $S_N(t)$, calculated from
(\ref{eq:Sn-alpha}) for the \textit{homogeneous} initial state $Q=0$, i.e., the factorized state with all spins polarized in 
positive $x$ direction,
\begin{align}
&S_N(t)=\bra{\Omega}\re^{\i H t}\si_1^x \re^{-\i H t} \ket{\Omega}, \label{eq:SN(t)}\\
&\ket{\Omega}\equiv \ket{\Psi_0}=\frac{1}{\sqrt{2^N}}\bigotimes_{n=1}^{N}  \binom {1}{1}.\label{def;Omega}
\end{align}
 Note that the quantity $S_N(t)$ cannot be easily computed
via free fermion techniques involving a Jordan-Wigner transformation and the use
of Wick's theorem, because the  density matrix stemming from
$\ket{\Omega}$ is not a Gaussian operator (exponential of 
bilinear expressions)  in terms of  Fermi operators
$c_j,c_j^\dagger$, see \cite{2022AresCalabrese}.

The key simplification in calculating $S_N(t)$ using the chiral basis consists
in the fact that under a proper choice of the overall phase $u=1$ in (\ref{eq:Mshock}),
the operator $\si_1^x$ becomes diagonal in the chiral basis
\begin{align}
&\si_1^x \ket{\pm ; n_1, \ldots,n_M} = \pm \ket{\pm ;n_1, \ldots, n_M}, \quad \forall M,\label{eq:SigmaxAction}
\end{align}
leading to 
\begin{align}
&\bra{\mu_{M'}(\bfvec{q}) }  \si_1^x  \ket{\mu_M(\bfvec{p})}=0, \quad \mbox{if $M \neq M'$.} 
\label{eq:BlockDiagonalizationSigmaX}
\end{align}
The choice $u=1$ will  be kept for the remainder of the main part of the manuscript.


\begin{figure}[tbp]
%\centerline{
\begin{tabular}{c}
\includegraphics[width=0.98\columnwidth]{FigSN_FG_new.pdf}\\[12.pt]
\includegraphics[width=0.98\columnwidth]{FigLogSN-Lim_FG_new.pdf}
\end{tabular}
%}
\caption{ Universal relaxation  function of the spin-helix amplitude (\ref{eq:SN(t)})  for different system sizes, in
  usual scale (top panel) and in logarithmic scale (bottom panel).  \textbf{
    Top Panel:} Green, red and black dots correspond to  $S_6(t),S_8(t),S_{10}(t)$
 respectively, while the
  continuous curve shows $S(t)$ (\ref{S(t)}).  The blue line is to be compared with Fig.~2a in
  \cite{SHS-Ketterle}.  \textbf{Bottom Panel:} $S_N(t)$ for $N=10, 20,\ldots ,50$,  from (\ref{rep2phimn}),
  shows the exponential decay for large times,  
   given by the black dashed
   line, (\ref{S(t)asymptotic}).
%\rev{I can not find the black dashed line. And there are no explanation of different lines. On the bottom panel, the $x$-axis label should be $\ln t$.}  
Coloured dashed curves show
$S(r,t)$ with $r=[N/4]$ from (\ref{EqDetProduct}).  
Curves with the same  colour code correspond to the same $N$. 
Deviations from the straight line at large $t$ are due to finite size effects. 
}
\label{Fig-SN(t)}
\end{figure}




\begin{figure}[tbp]
\centerline{
\includegraphics[width=0.98\columnwidth]{DecayDe0_FG_new.pdf}
}
\caption{ Asympotic decay rate $\ga$ of the spin-helix state  versus rescaled wavevector $Q/\pi$,
given by $\frac{8}{\pi}  |\cos (\pi x )|$.
This Figure is to be compared with the Fig.~3c of \cite{SHS-Ketterle}.
%\rev{Why not replace the axis label $\gamma_{decay}$ with $\gamma$ or $\gamma(Q)$, see Eq. (41).}
}
\label{Fig-gammaDecay}
\end{figure}

Inserting  $I= \sum_{\pp,M}\ket{\mu_M(\pp)}
\bra{\mu_M(\pp)}$ in (\ref{eq:SN(t)}) and using
(\ref{eq:BlockDiagonalizationSigmaX}) we obtain
\begin{multline}
S_N(t) = \sum_{
\pp,\qq, M} \re^{\i (E_\pp-E_\qq) t} \\
\times \braket{\Omega}{\mu_M(\pp)} \bra{\mu_M(\pp)} \si_1^x \ket{\mu_M(\qq)} 
\braket{\mu_M(\qq)}{\Omega}.\label{eq:S(t)}
\end{multline}
Next, we find that
$\braket{\Omega} {\mu_M(\pp)}= 0$ unless  $M = N/2$.  After a lengthy calculation (see \cite{Supp}) we eventually obtain 
  
\begin{align}
&S_N(t) = \real [\det_{m,n = 1, \dots, N/2} \Ph^{(N)}_{m,n} (t)], \label{DetPhiN}\\
&\Ph^{(N)}_{m,n} = \frac{1}{N^2} 
\sum_{\substack{p \in B_+\\q \in B_-}}
                     \frac{(1 + \re^{- \i p})(1 + \re^{\i q})
		           \re^{\i [2(mp - nq) + t(\e_p - \e_q))]}}
		          {\re^{\i (p - q)} - 1},\label{rep2phimn}
\end{align}
where $\e_p = 4 \cos p$ and $B_\pm$ are sets of  $p \in [- \pi, \pi)$ satisfying $\re^{ \i p N}=\pm 1$.  
Eq.~(\ref{rep2phimn}) describes the relaxation of the helix amplitude for finite periodic systems.  
Explicit expressions of $S_N(t)$ for  $N=4,6$ are given in \cite{Supp}.

Interestingly,  analyzing the Taylor expansion of $S_N(t)= \sum_n
C_N(n) t^n$ at $t=0$ we observe that the Taylor coefficients   for different $N$ follow 
the same stable pattern  growing linearly with $N$.  Namely, we find  $C_{N+2}(n)=C_{N}(n)$ for $n=0,1,\ldots, 2N-4$.
Consequently, the stable pattern gives the exact Taylor
expansion about $t=0$ of the decay of spin-helix amplitude  in the thermodynamic limit
\begin{align}
&S(t) = \lim_{N \rightarrow \infty}S_N(t),  \label{S(t)}\\
&S(t) = \,1-4 t^2 + \frac{2^5}{3}t^4 - \frac{2^6}{3}t^6 + \frac{2^9}{15}t^8 -
\frac{2^{11}}{45}t^{10}+
\frac{ \ 2^{12}\, 179}{14175}   t^{12}  \no\\
&-\frac{ 2^{16}\, 11}{14175} t^{14} + \frac{2^{16} 2987}{4465125}
t^{16} -\frac{2^{18}572}{
4465125} t^{18} +\ldots,
\label{eq:Taylor}
\end{align}
obtainable also by direct operatorial methods.

\textbf{Reduction to Bessel functions--}
For large $N$ the sums in  (\ref{rep2phimn}) can be replaced by integrals.  Then, after some algebra,
we find that the  entries  $\Ph^{(N)}_{m,n} (t) \rightarrow \Ph_{m,n} (t)$ converge to
(see  \cite{Supp})
\begin{align}
& (-1)^{m-n} \Ph_{m,n} (t) = \de_{m,n} + K_{m,n} (t), \label{phirelk} \\ 
     &K_{m, n} (t) =
        \frac{t}{m - n} \bigl(J_{2m}(4t) J_{2n-1}(4t) - J_{2n}(4t) J_{2m-1}(4t)\bigr) \nonumber \\
        &+ \frac{t}{m - n} \bigl(J_{2m-1}(4t) J_{2n-2}(4t) - J_{2n-1}(4t) J_{2m-2}(4t)\bigr) \nonumber \\
        &+ \frac{\i t}{m - n - 1/2}
	  \bigl(J_{2m - 2} (4t) J_{2n} (4t) - J_{2n - 1} (4t) J_{2m - 1} (4t)\bigr)\nonumber  \\
        &- \frac{\i t}{m - n + 1/2}
	  \bigl(J_{2m - 1} (4t) J_{2n - 1} (4t) - J_{2n - 2} (4t) J_{2m} (4t)\bigr), \no\\
&K_{n,n}(t)=- (J_0(4t))^2 + (J_{2n - 1 } (4t))^2 +2 \sum_{j=0}^{2n-2}(J_j(4t))^2,
\label{eq:Kfunc}
\end{align}
where the $J_k(x)$ are Bessel functions.  
After further manipulations (see  \cite{Supp}) and taking into account the symmetries of 
$K_{n,m}$ we finally obtain
\begin{align} \label{finaldiscretebessel}
     S(t) &= \lim_{r \rightarrow \infty} S(r,t)\\
S(r,t)&=  |\det_{m,n = 1, \dots, r}  A_{m,n}(t)|^2, \,\, \label{EqDetProduct}\\ 
A_{m,n}(t)&= \de_{m,n} +K_{m,n} (t)+ K_{m,1-n}(t).
\end{align}
These formulae represent $S(t)$ as a product of two infinite
determinants. Infinite determinants may define functions in
very much the same way as infinite series. As in the present
case, they may be extremely efficient in computations 
\cite{Bornemann10}. With a few lines of Mathematica code we
obtain, e.g., $S(t = 50) = 7.64483 \times 10^{- 56}$ within a
few seconds on a laptop computer. As opposed to the Taylor
series (\ref{eq:Taylor}) the determinant representation determines
$S(t)$ for \textit{all} times. The function $S(t)$ shown in
Fig.~\ref{Fig-SN(t)} is directly  comparable  with the experimental
data Fig.~2a in
\cite{SHS-Ketterle}. 

Even though the true thermodynamic limit is given by the limit $r\rightarrow \infty$ in (\ref{finaldiscretebessel}),
already for $r=1$ when the matrix $A$ is a scalar, the  function 
\begin{align} 
     S(1,t) &=g_0^2 + 4 t^2\left(g_0 + \frac{g_1}{3} \right)^2, \quad
g_n = J_n^2(4t) + J_{n+1}^2(4t)
\end{align}
approximates $S(t)$ for $0\leq t \leq 0.5$ (data not shown), and also  reproduces the 
asymptotic Taylor expansion (\ref{eq:Taylor}) up to the order $t^7$.

  Choosing $r=4$ in (\ref{EqDetProduct}) reproduces 
 $S(t)$ with  accuracy $|S(4,t)-S(t)|<10^{-5}$,  for $t<2$ which is enough for any practical 
purpose.  Indeed at $t=t_{max}=2$,  the amplitude $S(t)$ drops by two orders of magnitude with respect to the initial value, $S(t_{max})\approx 0.0093 <S(0)/100$.
For larger $t$, $S(t)$ is well approximated by the asymptotics (\ref{S(t)asymptotic}).

In addition,    our numerics suggests a simple asymptotics for $\det A(t)$,  namely 
\begin{align} 
 & \det A(t) \rightarrow  a_0   \re^{2 \i t} \re^{-\frac{4}{\pi}t },  \quad t\gg 1,  \label{Asymp} \\
&a_0 = 1.2295 \pm 2\times 10^{-5}. \no
\end{align}
The data were obtained by analyzing $ \det A(t)$ for $r\leq 170$, 
and for times $t< t_{m}(r) = r/2.2 -0.19$, data shown in \cite{Supp}.  
Eq.(\ref{Asymp})  corresponds to the $S(t)$ asymptotics  
\begin{align} 
 &\lim_{t \rightarrow \infty} S(t) \approx 1.5117 \re^{-\frac{8}{\pi}t }. 
\label{S(t)asymptotic}
\end{align}

From the asymptotics  (\ref{S(t)asymptotic}) and the self-similarity (\ref{eq:Sx})
we readily get the spin-helix state decay rate 
\begin{align}
&\ga(Q)= -\lim\limits_{t\rightarrow
    \infty} (t^{-1}{\log \langle \si_{n}^x(t)\rangle_Q}) = \frac{8}{\pi}  |\cos (Q )|,
\end{align}
shown in Fig.~\ref{Fig-gammaDecay} and directly comparable with the
experimental result, Fig.~3c of \cite{SHS-Ketterle}.



 

\textbf{Conclusions--}
In this work, we propose a chiral qubit basis that possesses topological properties while retaining a simple factorized structure and orthonormality. The chiral basis at every site is represented by a  pair of mutually orthogonal qubit states,  and  can be implemented with usual  binary code registers.
We demonstrate the effectiveness of the chiral basis by applying it to an experimentally relevant physical problem. 
Our results  in Figs.~\ref{Fig-SN(t)} and \ref{Fig-gammaDecay} are compared to the experimental data.

As a byproduct of our study, we  discovered the existence of a universal function $S(t)$ that governs the relaxation of transversal spin helices with arbitrary wavelengths in an infinite system under XX dynamics.  We gave the explicit determinantal form of $S(t)$ (\ref{EqDetProduct}) and 
 calculated  its Taylor expansion (\ref{eq:Taylor}) and its large $t$ asymptotics (\ref{S(t)asymptotic}).
The possibility to express  correlation functions in 
determinantal form is  typical of integrable systems, see e.g.\
\cite{Lenard64,1993ColomoXX0basis,KoSl90,GKS20a,2021Goehmann,2021Suzuki}. 
We also obtained explicit expressions for the spin-helix state relaxation of
finite systems of qubits (\ref{DetPhiN}) that can be useful for future
experiments, such as those with ring-shaped atom arrays \cite{RingShapedArrays},
where periodic boundary conditions can be realized.

Finally, in addition to applications to not yet solved problems it would
also be interesting to check the performance of our chiral qubit basis
towards problems already treated within traditional
approaches  e.g.~\cite{1970McCoy,2019Sasamoto,2022Essler}.
\begin{acknowledgments}
 Financial support from Deutsche Forschungsgemeinschaft through DFG project KL
 645/20-2 is gratefully acknowledged. V.~P.~acknowledges support by the
 European Research Council (ERC) through the advanced Grant
 No. 694544—OMNES. X. Z. acknowledges financial support from the National
 Natural Science Foundation of China (No. 12204519).
\end{acknowledgments}


%\bibliographystyle{apsrev4-1}
%\bibliography{ChiralBasis}
%apsrev4-2.bst 2019-01-14 (MD) hand-edited version of apsrev4-1.bst
%Control: key (0)
%Control: author (72) initials jnrlst
%Control: editor formatted (1) identically to author
%Control: production of article title (-1) disabled
%Control: page (0) single
%Control: year (1) truncated
%Control: production of eprint (0) enabled
\begin{thebibliography}{24}%
\makeatletter
\providecommand \@ifxundefined [1]{%
 \@ifx{#1\undefined}
}%
\providecommand \@ifnum [1]{%
 \ifnum #1\expandafter \@firstoftwo
 \else \expandafter \@secondoftwo
 \fi
}%
\providecommand \@ifx [1]{%
 \ifx #1\expandafter \@firstoftwo
 \else \expandafter \@secondoftwo
 \fi
}%
\providecommand \natexlab [1]{#1}%
\providecommand \enquote  [1]{``#1''}%
\providecommand \bibnamefont  [1]{#1}%
\providecommand \bibfnamefont [1]{#1}%
\providecommand \citenamefont [1]{#1}%
\providecommand \href@noop [0]{\@secondoftwo}%
\providecommand \href [0]{\begingroup \@sanitize@url \@href}%
\providecommand \@href[1]{\@@startlink{#1}\@@href}%
\providecommand \@@href[1]{\endgroup#1\@@endlink}%
\providecommand \@sanitize@url [0]{\catcode `\\12\catcode `\$12\catcode
  `\&12\catcode `\#12\catcode `\^12\catcode `\_12\catcode `\%12\relax}%
\providecommand \@@startlink[1]{}%
\providecommand \@@endlink[0]{}%
\providecommand \url  [0]{\begingroup\@sanitize@url \@url }%
\providecommand \@url [1]{\endgroup\@href {#1}{\urlprefix }}%
\providecommand \urlprefix  [0]{URL }%
\providecommand \Eprint [0]{\href }%
\providecommand \doibase [0]{https://doi.org/}%
\providecommand \selectlanguage [0]{\@gobble}%
\providecommand \bibinfo  [0]{\@secondoftwo}%
\providecommand \bibfield  [0]{\@secondoftwo}%
\providecommand \translation [1]{[#1]}%
\providecommand \BibitemOpen [0]{}%
\providecommand \bibitemStop [0]{}%
\providecommand \bibitemNoStop [0]{.\EOS\space}%
\providecommand \EOS [0]{\spacefactor3000\relax}%
\providecommand \BibitemShut  [1]{\csname bibitem#1\endcsname}%
\let\auto@bib@innerbib\@empty
%</preamble>
\bibitem [{\citenamefont {Mallat}(2009)}]{Wavelets}%
  \BibitemOpen
  \bibfield  {author} {\bibinfo {author} {\bibfnamefont {S.}~\bibnamefont
  {Mallat}},\ }\href@noop {} {\emph {\bibinfo {title} {A Wavelet Tour of Signal
  Processing: the Sparse Way, 3rd Revised edition}}}\ (\bibinfo  {publisher}
  {Academic Press, London},\ \bibinfo {year} {2009})\BibitemShut {NoStop}%
\bibitem [{\citenamefont {Jepsen}\ \emph {et~al.}(2022)\citenamefont {Jepsen},
  \citenamefont {Lee}, \citenamefont {Lin}, \citenamefont {Dimitrova},
  \citenamefont {Margalit}, \citenamefont {Ho},\ and\ \citenamefont
  {Ketterle}}]{SHS-Ketterle}%
  \BibitemOpen
  \bibfield  {author} {\bibinfo {author} {\bibfnamefont {P.~N.}\ \bibnamefont
  {Jepsen}}, \bibinfo {author} {\bibfnamefont {Y.~K.}\ \bibnamefont {Lee}},
  \bibinfo {author} {\bibfnamefont {H.}~\bibnamefont {Lin}}, \bibinfo {author}
  {\bibfnamefont {I.}~\bibnamefont {Dimitrova}}, \bibinfo {author}
  {\bibfnamefont {Y.}~\bibnamefont {Margalit}}, \bibinfo {author}
  {\bibfnamefont {W.~W.}\ \bibnamefont {Ho}},\ and\ \bibinfo {author}
  {\bibfnamefont {W.}~\bibnamefont {Ketterle}},\ }\href
  {https://doi.org/10.1038/s41567-022-01651-7} {\bibfield  {journal} {\bibinfo
  {journal} {Nature Physics}\ }\textbf {\bibinfo {volume} {18}},\ \bibinfo
  {pages} {899} (\bibinfo {year} {2022})}\BibitemShut {NoStop}%
\bibitem [{\citenamefont {Jepsen}\ \emph {et~al.}(2020)\citenamefont {Jepsen},
  \citenamefont {Amato-Grill}, \citenamefont {Dimitrova}, \citenamefont {Ho},
  \citenamefont {Demler},\ and\ \citenamefont
  {Ketterle}}]{2020NatureSpinHelix}%
  \BibitemOpen
  \bibfield  {author} {\bibinfo {author} {\bibfnamefont {P.~N.}\ \bibnamefont
  {Jepsen}}, \bibinfo {author} {\bibfnamefont {J.}~\bibnamefont {Amato-Grill}},
  \bibinfo {author} {\bibfnamefont {I.}~\bibnamefont {Dimitrova}}, \bibinfo
  {author} {\bibfnamefont {W.~W.}\ \bibnamefont {Ho}}, \bibinfo {author}
  {\bibfnamefont {E.}~\bibnamefont {Demler}},\ and\ \bibinfo {author}
  {\bibfnamefont {W.}~\bibnamefont {Ketterle}},\ }\href
  {https://doi.org/10.1038/s41586-020-3033-y} {\bibfield  {journal} {\bibinfo
  {journal} {NATURE}\ }\textbf {\bibinfo {volume} {588}},\ \bibinfo {pages}
  {403+} (\bibinfo {year} {2020})}\BibitemShut {NoStop}%
\bibitem [{\citenamefont {Jepsen}\ \emph {et~al.}(2021)\citenamefont {Jepsen},
  \citenamefont {Ho}, \citenamefont {Amato-Grill}, \citenamefont {Dimitrova},
  \citenamefont {Demler},\ and\ \citenamefont
  {Ketterle}}]{2021KetterleTransverse}%
  \BibitemOpen
  \bibfield  {author} {\bibinfo {author} {\bibfnamefont {P.~N.}\ \bibnamefont
  {Jepsen}}, \bibinfo {author} {\bibfnamefont {W.~W.}\ \bibnamefont {Ho}},
  \bibinfo {author} {\bibfnamefont {J.}~\bibnamefont {Amato-Grill}}, \bibinfo
  {author} {\bibfnamefont {I.}~\bibnamefont {Dimitrova}}, \bibinfo {author}
  {\bibfnamefont {E.}~\bibnamefont {Demler}},\ and\ \bibinfo {author}
  {\bibfnamefont {W.}~\bibnamefont {Ketterle}},\ }\href
  {https://doi.org/10.1103/PhysRevX.11.041054} {\bibfield  {journal} {\bibinfo
  {journal} {Phys. Rev. X}\ }\textbf {\bibinfo {volume} {11}},\ \bibinfo
  {pages} {041054} (\bibinfo {year} {2021})}\BibitemShut {NoStop}%
\bibitem [{\citenamefont {Popkov}\ \emph {et~al.}(2021)\citenamefont {Popkov},
  \citenamefont {Zhang},\ and\ \citenamefont {Kl\"umper}}]{SHS-Phantom}%
  \BibitemOpen
  \bibfield  {author} {\bibinfo {author} {\bibfnamefont {V.}~\bibnamefont
  {Popkov}}, \bibinfo {author} {\bibfnamefont {X.}~\bibnamefont {Zhang}},\ and\
  \bibinfo {author} {\bibfnamefont {A.}~\bibnamefont {Kl\"umper}},\ }\href
  {https://link.aps.org/doi/10.1103/PhysRevB.104.L081410} {\bibfield  {journal}
  {\bibinfo  {journal} {Phys. Rev. B}\ }\textbf {\bibinfo {volume} {104}},\
  \bibinfo {pages} {L081410} (\bibinfo {year} {2021})}\BibitemShut {NoStop}%
\bibitem [{\citenamefont {Zhang}\ \emph
  {et~al.}(2021{\natexlab{a}})\citenamefont {Zhang}, \citenamefont
  {Kl{\"u}mper},\ and\ \citenamefont {Popkov}}]{Phantom-Long}%
  \BibitemOpen
  \bibfield  {author} {\bibinfo {author} {\bibfnamefont {X.}~\bibnamefont
  {Zhang}}, \bibinfo {author} {\bibfnamefont {A.}~\bibnamefont {Kl{\"u}mper}},\
  and\ \bibinfo {author} {\bibfnamefont {V.}~\bibnamefont {Popkov}},\ }\href
  {https://link.aps.org/doi/10.1103/PhysRevB.103.115435} {\bibfield  {journal}
  {\bibinfo  {journal} {Phys. Rev. B}\ }\textbf {\bibinfo {volume} {103}},\
  \bibinfo {pages} {115435} (\bibinfo {year} {2021}{\natexlab{a}})}\BibitemShut
  {NoStop}%
\bibitem [{\citenamefont {Zhang}\ \emph
  {et~al.}(2021{\natexlab{b}})\citenamefont {Zhang}, \citenamefont
  {Kl{\"u}mper},\ and\ \citenamefont {Popkov}}]{ChiralBA}%
  \BibitemOpen
  \bibfield  {author} {\bibinfo {author} {\bibfnamefont {X.}~\bibnamefont
  {Zhang}}, \bibinfo {author} {\bibfnamefont {A.}~\bibnamefont {Kl{\"u}mper}},\
  and\ \bibinfo {author} {\bibfnamefont {V.}~\bibnamefont {Popkov}},\ }\href
  {https://doi.org/10.1103/PhysRevB.104.195409} {\bibfield  {journal} {\bibinfo
   {journal} {Phys. Rev. B}\ }\textbf {\bibinfo {volume} {104}},\ \bibinfo
  {pages} {195409} (\bibinfo {year} {2021}{\natexlab{b}})}\BibitemShut
  {NoStop}%
\bibitem [{\citenamefont {Cecile}\ \emph {et~al.}(2023)\citenamefont {Cecile},
  \citenamefont {Gopalakrishnan}, \citenamefont {Vasseur},\ and\ \citenamefont
  {De~Nardis}}]{SHS-Hydro}%
  \BibitemOpen
  \bibfield  {author} {\bibinfo {author} {\bibfnamefont {G.}~\bibnamefont
  {Cecile}}, \bibinfo {author} {\bibfnamefont {S.}~\bibnamefont
  {Gopalakrishnan}}, \bibinfo {author} {\bibfnamefont {R.}~\bibnamefont
  {Vasseur}},\ and\ \bibinfo {author} {\bibfnamefont {J.}~\bibnamefont
  {De~Nardis}},\ }\href {https://doi.org/10.1103/PhysRevB.108.075135}
  {\bibfield  {journal} {\bibinfo  {journal} {Phys. Rev. B}\ }\textbf {\bibinfo
  {volume} {108}},\ \bibinfo {pages} {075135} (\bibinfo {year}
  {2023})}\BibitemShut {NoStop}%
\bibitem [{\citenamefont {K\"uhn}\ \emph {et~al.}(2023)\citenamefont {K\"uhn},
  \citenamefont {Gerken}, \citenamefont {Funcke}, \citenamefont {Hartung},
  \citenamefont {Stornati}, \citenamefont {Jansen},\ and\ \citenamefont
  {Posske}}]{2023Posske}%
  \BibitemOpen
  \bibfield  {author} {\bibinfo {author} {\bibfnamefont {S.}~\bibnamefont
  {K\"uhn}}, \bibinfo {author} {\bibfnamefont {F.}~\bibnamefont {Gerken}},
  \bibinfo {author} {\bibfnamefont {L.}~\bibnamefont {Funcke}}, \bibinfo
  {author} {\bibfnamefont {T.}~\bibnamefont {Hartung}}, \bibinfo {author}
  {\bibfnamefont {P.}~\bibnamefont {Stornati}}, \bibinfo {author}
  {\bibfnamefont {K.}~\bibnamefont {Jansen}},\ and\ \bibinfo {author}
  {\bibfnamefont {T.}~\bibnamefont {Posske}},\ }\href
  {https://doi.org/10.1103/PhysRevB.107.214422} {\bibfield  {journal} {\bibinfo
   {journal} {Phys. Rev. B}\ }\textbf {\bibinfo {volume} {107}},\ \bibinfo
  {pages} {214422} (\bibinfo {year} {2023})}\BibitemShut {NoStop}%
\bibitem [{Sup()}]{Supp}%
  \BibitemOpen
  \href {http://link.aps.org/supplemental} {\bibinfo  {journal} {See
  Supplemental Material at ...}\ }\BibitemShut {NoStop}%
\bibitem [{\citenamefont {Colomo}\ \emph {et~al.}(1993)\citenamefont {Colomo},
  \citenamefont {Izergin}, \citenamefont {Korepin},\ and\ \citenamefont
  {Tognetti}}]{1993ColomoXX0basis}%
  \BibitemOpen
\bibfield  {journal} {  }\bibfield  {author} {\bibinfo {author} {\bibfnamefont
  {F.}~\bibnamefont {Colomo}}, \bibinfo {author} {\bibfnamefont {A.~G.}\
  \bibnamefont {Izergin}}, \bibinfo {author} {\bibfnamefont {V.~E.}\
  \bibnamefont {Korepin}},\ and\ \bibinfo {author} {\bibfnamefont
  {V.}~\bibnamefont {Tognetti}},\ }\href {https://doi.org/10.1007/BF01016992}
  {\bibfield  {journal} {\bibinfo  {journal} {Theor. Math. Phys.}\ }\textbf
  {\bibinfo {volume} {94}},\ \bibinfo {pages} {11} (\bibinfo {year}
  {1993})}\BibitemShut {NoStop}%
\bibitem [{\citenamefont {Popkov}\ \emph {et~al.}(2023)\citenamefont {Popkov},
  \citenamefont {\ifmmode \check{Z}\else
  \v{Z}\fi{}nidari\ifmmode~\check{c}\else \v{c}\fi{}},\ and\ \citenamefont
  {Zhang}}]{2023SHSdecay}%
  \BibitemOpen
  \bibfield  {author} {\bibinfo {author} {\bibfnamefont {V.}~\bibnamefont
  {Popkov}}, \bibinfo {author} {\bibfnamefont {M.}~\bibnamefont {\ifmmode
  \check{Z}\else \v{Z}\fi{}nidari\ifmmode~\check{c}\else \v{c}\fi{}}},\ and\
  \bibinfo {author} {\bibfnamefont {X.}~\bibnamefont {Zhang}},\ }\href
  {https://doi.org/10.1103/PhysRevB.107.235408} {\bibfield  {journal} {\bibinfo
   {journal} {Phys. Rev. B}\ }\textbf {\bibinfo {volume} {107}},\ \bibinfo
  {pages} {235408} (\bibinfo {year} {2023})}\BibitemShut {NoStop}%
\bibitem [{\citenamefont {Ares}\ \emph {et~al.}(2023)\citenamefont {Ares},
  \citenamefont {Murciano},\ and\ \citenamefont
  {Calabrese}}]{2022AresCalabrese}%
  \BibitemOpen
  \bibfield  {author} {\bibinfo {author} {\bibfnamefont {F.}~\bibnamefont
  {Ares}}, \bibinfo {author} {\bibfnamefont {S.}~\bibnamefont {Murciano}},\
  and\ \bibinfo {author} {\bibfnamefont {P.}~\bibnamefont {Calabrese}},\ }\href
  {https://doi.org/10.1038/s41467-023-37747-8} {\bibfield  {journal} {\bibinfo
  {journal} {Nature Communications}\ }\textbf {\bibinfo {volume} {14}},\
  \bibinfo {pages} {2036} (\bibinfo {year} {2023})}\BibitemShut {NoStop}%
\bibitem [{\citenamefont {Bornemann}(2010)}]{Bornemann10}%
  \BibitemOpen
  \bibfield  {author} {\bibinfo {author} {\bibfnamefont {F.}~\bibnamefont
  {Bornemann}},\ }\href {https://doi.org/10.1090/s0025-5718-09-02280-7}
  {\bibfield  {journal} {\bibinfo  {journal} {Mathematics of Computation}\
  }\textbf {\bibinfo {volume} {79}},\ \bibinfo {pages} {871} (\bibinfo {year}
  {2010})}\BibitemShut {NoStop}%
\bibitem [{\citenamefont {Lenard}(1964)}]{Lenard64}%
  \BibitemOpen
  \bibfield  {author} {\bibinfo {author} {\bibfnamefont {A.}~\bibnamefont
  {Lenard}},\ }\href {https://doi.org/10.1063/1.1704196} {\bibfield  {journal}
  {\bibinfo  {journal} {J. Math. Phys.}\ }\textbf {\bibinfo {volume} {5}},\
  \bibinfo {pages} {930} (\bibinfo {year} {1964})}\BibitemShut {NoStop}%
\bibitem [{\citenamefont {Korepin}\ and\ \citenamefont
  {Slavnov}(1990)}]{KoSl90}%
  \BibitemOpen
  \bibfield  {author} {\bibinfo {author} {\bibfnamefont {V.~E.}\ \bibnamefont
  {Korepin}}\ and\ \bibinfo {author} {\bibfnamefont {N.~A.}\ \bibnamefont
  {Slavnov}},\ }\href {https://doi.org/10.1007/BF02096781} {\bibfield
  {journal} {\bibinfo  {journal} {Commun. Math. Phys.}\ }\textbf {\bibinfo
  {volume} {129}},\ \bibinfo {pages} {103} (\bibinfo {year}
  {1990})}\BibitemShut {NoStop}%
\bibitem [{\citenamefont {G\"ohmann}\ \emph {et~al.}(2020)\citenamefont
  {G\"ohmann}, \citenamefont {Kozlowski},\ and\ \citenamefont
  {Suzuki}}]{GKS20a}%
  \BibitemOpen
  \bibfield  {author} {\bibinfo {author} {\bibfnamefont {F.}~\bibnamefont
  {G\"ohmann}}, \bibinfo {author} {\bibfnamefont {K.~K.}\ \bibnamefont
  {Kozlowski}},\ and\ \bibinfo {author} {\bibfnamefont {J.}~\bibnamefont
  {Suzuki}},\ }\href {https://doi.org/10.1063/1.5111039} {\bibfield  {journal}
  {\bibinfo  {journal} {J. Math. Phys.}\ }\textbf {\bibinfo {volume} {61}},\
  \bibinfo {pages} {013301} (\bibinfo {year} {2020})}\BibitemShut {NoStop}%
\bibitem [{\citenamefont {Göhmann}\ \emph {et~al.}(2021)\citenamefont
  {Göhmann}, \citenamefont {Kleinemühl},\ and\ \citenamefont
  {Weiße}}]{2021Goehmann}%
  \BibitemOpen
  \bibfield  {author} {\bibinfo {author} {\bibfnamefont {F.}~\bibnamefont
  {Göhmann}}, \bibinfo {author} {\bibfnamefont {R.}~\bibnamefont
  {Kleinemühl}},\ and\ \bibinfo {author} {\bibfnamefont {A.}~\bibnamefont
  {Weiße}},\ }\href {https://doi.org/10.1088/1751-8121/ac200a} {\bibfield
  {journal} {\bibinfo  {journal} {Journal of Physics A: Mathematical and
  Theoretical}\ }\textbf {\bibinfo {volume} {54}},\ \bibinfo {pages} {414001}
  (\bibinfo {year} {2021})}\BibitemShut {NoStop}%
\bibitem [{\citenamefont {Babenko}\ \emph {et~al.}(2021)\citenamefont
  {Babenko}, \citenamefont {G\"ohmann}, \citenamefont {Kozlowski},
  \citenamefont {Sirker},\ and\ \citenamefont {Suzuki}}]{2021Suzuki}%
  \BibitemOpen
  \bibfield  {author} {\bibinfo {author} {\bibfnamefont {C.}~\bibnamefont
  {Babenko}}, \bibinfo {author} {\bibfnamefont {F.}~\bibnamefont {G\"ohmann}},
  \bibinfo {author} {\bibfnamefont {K.~K.}\ \bibnamefont {Kozlowski}}, \bibinfo
  {author} {\bibfnamefont {J.}~\bibnamefont {Sirker}},\ and\ \bibinfo {author}
  {\bibfnamefont {J.}~\bibnamefont {Suzuki}},\ }\href
  {https://doi.org/10.1103/PhysRevLett.126.210602} {\bibfield  {journal}
  {\bibinfo  {journal} {Phys. Rev. Lett.}\ }\textbf {\bibinfo {volume} {126}},\
  \bibinfo {pages} {210602} (\bibinfo {year} {2021})}\BibitemShut {NoStop}%
\bibitem [{\citenamefont {Scholl}\ \emph {et~al.}(2022)\citenamefont {Scholl},
  \citenamefont {Williams}, \citenamefont {Bornet}, \citenamefont {Wallner},
  \citenamefont {Barredo}, \citenamefont {Henriet}, \citenamefont {Signoles},
  \citenamefont {Hainaut}, \citenamefont {Franz}, \citenamefont {Geier} \emph
  {et~al.}}]{RingShapedArrays}%
  \BibitemOpen
  \bibfield  {author} {\bibinfo {author} {\bibfnamefont {P.}~\bibnamefont
  {Scholl}}, \bibinfo {author} {\bibfnamefont {H.~J.}\ \bibnamefont
  {Williams}}, \bibinfo {author} {\bibfnamefont {G.}~\bibnamefont {Bornet}},
  \bibinfo {author} {\bibfnamefont {F.}~\bibnamefont {Wallner}}, \bibinfo
  {author} {\bibfnamefont {D.}~\bibnamefont {Barredo}}, \bibinfo {author}
  {\bibfnamefont {L.}~\bibnamefont {Henriet}}, \bibinfo {author} {\bibfnamefont
  {A.}~\bibnamefont {Signoles}}, \bibinfo {author} {\bibfnamefont
  {C.}~\bibnamefont {Hainaut}}, \bibinfo {author} {\bibfnamefont
  {T.}~\bibnamefont {Franz}}, \bibinfo {author} {\bibfnamefont
  {S.}~\bibnamefont {Geier}}, \emph {et~al.},\ }\href
  {https://link.aps.org/doi/10.1103/PRXQuantum.3.020303} {\bibfield  {journal}
  {\bibinfo  {journal} {PRX Quantum}\ }\textbf {\bibinfo {volume} {3}},\
  \bibinfo {pages} {020303} (\bibinfo {year} {2022})}\BibitemShut {NoStop}%
\bibitem [{\citenamefont {Barouch}\ and\ \citenamefont
  {McCoy}(1970)}]{1970McCoy}%
  \BibitemOpen
  \bibfield  {author} {\bibinfo {author} {\bibfnamefont {E.}~\bibnamefont
  {Barouch}}\ and\ \bibinfo {author} {\bibfnamefont {B.~M.}\ \bibnamefont
  {McCoy}},\ }\href {https://doi.org/10.1103/PhysRevA.2.1075} {\bibfield
  {journal} {\bibinfo  {journal} {Phys. Rev. A}\ }\textbf {\bibinfo {volume}
  {2}},\ \bibinfo {pages} {1075} (\bibinfo {year} {1970})}\BibitemShut
  {NoStop}%
\bibitem [{\citenamefont {Moriya}\ \emph {et~al.}(2019)\citenamefont {Moriya},
  \citenamefont {Nagao},\ and\ \citenamefont {Sasamoto}}]{2019Sasamoto}%
  \BibitemOpen
  \bibfield  {author} {\bibinfo {author} {\bibfnamefont {H.}~\bibnamefont
  {Moriya}}, \bibinfo {author} {\bibfnamefont {R.}~\bibnamefont {Nagao}},\ and\
  \bibinfo {author} {\bibfnamefont {T.}~\bibnamefont {Sasamoto}},\ }\href
  {https://doi.org/10.1088/1742-5468/ab1dd6} {\bibfield  {journal} {\bibinfo
  {journal} {J. Stat. Mech.: Theor. Exp.}\ }\textbf {\bibinfo {volume}
  {2019}},\ \bibinfo {pages} {063105} (\bibinfo {year} {2019})}\BibitemShut
  {NoStop}%
\bibitem [{\citenamefont {Granet}\ \emph {et~al.}(2022)\citenamefont {Granet},
  \citenamefont {Dreyer},\ and\ \citenamefont {Essler}}]{2022Essler}%
  \BibitemOpen
  \bibfield  {author} {\bibinfo {author} {\bibfnamefont {E.}~\bibnamefont
  {Granet}}, \bibinfo {author} {\bibfnamefont {H.}~\bibnamefont {Dreyer}},\
  and\ \bibinfo {author} {\bibfnamefont {F.~H.~L.}\ \bibnamefont {Essler}},\
  }\href {https://doi.org/10.21468/SciPostPhys.12.1.019} {\bibfield  {journal}
  {\bibinfo  {journal} {SciPost Phys.}\ }\textbf {\bibinfo {volume} {12}},\
  \bibinfo {pages} {019} (\bibinfo {year} {2022})}\BibitemShut {NoStop}%
\bibitem [{\citenamefont {Borodin}\ \emph {et~al.}(2000)\citenamefont
  {Borodin}, \citenamefont {Okounkov},\ and\ \citenamefont
  {Olshanski}}]{2000-Borodin}%
  \BibitemOpen
  \bibfield  {author} {\bibinfo {author} {\bibfnamefont {A.}~\bibnamefont
  {Borodin}}, \bibinfo {author} {\bibfnamefont {A.}~\bibnamefont {Okounkov}},\
  and\ \bibinfo {author} {\bibfnamefont {G.}~\bibnamefont {Olshanski}},\ }\href
  {https://doi.org/10.1090/S0894-0347-00-00337-4} {\bibfield  {journal}
  {\bibinfo  {journal} {JOURNAL OF THE AMERICAN MATHEMATICAL SOCIETY}\ }\textbf
  {\bibinfo {volume} {13}},\ \bibinfo {pages} {481} (\bibinfo {year}
  {2000})}\BibitemShut {NoStop}%
\end{thebibliography}%



\end{document}



