%\documentclass[aps,pre,onecolumn,superscriptaddress,showpacs,amsmath,amssymb]{revtex4}
\documentclass[aps,pre,onecolumn,superscriptaddress,amsmath,amssymb]{revtex4-2}


\usepackage[utf8]{inputenc}
\usepackage{amsmath,amsthm}
\usepackage{graphicx}
\usepackage[colorlinks = true,
            linkcolor = blue,
            urlcolor  = blue,
            citecolor = blue,
            anchorcolor = blue]{hyperref}
%\usepackage{bbold}
\usepackage{braket}
\usepackage{xcolor}
\usepackage{easyReview}
%TMP
%\usepackage[title]{appendix}
\raggedbottom


\newcommand{\U}{\ensuremath{\mathcal{U}}}
\newcommand{\J}{\ensuremath{\mathcal{J}}}
\newcommand{\x}{{x}}
\newcommand{\y}{{y}}
\newcommand{\z}{{z}}
\renewcommand{\S}{\mathbf{S}}
\renewcommand{\L}{\mathbf{L}}
\newcommand{\I}{\mathbf{I}}
\newcommand{\T}{\mathbf{T}}

\newcommand{\V}{\mathcal{V}}
\newcommand{\lhalf}{{\textstyle\frac{\lambda}{2}}}
\newcommand{\mhalf}{{\textstyle\frac{\mu}{2}}}
\newcommand{\tr}[1]{\ensuremath{\operatorname{tr}\!{#1}}}
\renewcommand{\d}[1]{\ensuremath{\operatorname{d}\!{#1}}}
\newcommand{\real}{{\rm Re}}
\newcommand{\imag}{{\rm Im}}
%\newcommand{\be}{\begin{equation}}
%\newcommand{\ee}{\end{equation}}
\newcommand{\bea}{\begin{eqnarray}}
\newcommand{\eea}{\end{eqnarray}}
\newcommand{\ZZ}{\mathbb{Z}}
\newcommand{\RR}{\mathbb{R}}
\newcommand{\CC}{\mathbb{C}}
\newcommand{\one}{\mathbb{I}}

\global\long\def\l{\la}

\newcommand{\bfvec}[1]{\mathbf{#1}}

\newcommand{\norm}[1]{\left\lVert#1\right\rVert}
\newcommand{\trb}{\mathop{\mathrm{tr}_{1,N}}\limits}
\newcommand{\trzero}{\mathop{\mathrm{tr}_{\mathcal{H}_0}}\limits}
\newcommand{\truno}{\mathop{\mathrm{tr}_{\mathcal{H}_1}}\limits}
\global\long\def\ga{\gamma} \global\long\def\de{\delta}
\global\long\def\De{\Delta} \global\long\def\Ga{\Gamma}
\global\long\def\th{\theta}
\global\long\def\th{\theta}

\global\long\def\ra{\rightarrow}

\global\long\def\dL{\mathbb{L}}
\global\long\def\dF{\mathbb{F}}

\global\long\def\ell#1{\theta_{#1}}
\global\long\def\bell#1{\tilde\theta_{#1}}

\global\long\def\la{\lambda} \global\long\def\ka{\kappa}
\global\long\def\si{\sigma}
\global\long\def\vfi{\varphi}
\global\long\def\s{\sigma}

\global\long\def\ro{\rho} \global\long\def\Om{\Omega}
\global\long\def\eps{\epsilon}
\global\long\def\al{\alpha}
\global\long\def\be{\beta}
\global\long\def\ga{\gamma} \global\long\def\de{\delta}

\def\XYZ{XY\! Z}
\def\XXZ{XXZ}

\global\long\def\no{\nonumber}

\theoremstyle{thm@}
\newtheorem*{pro}{Proposition}
\newtheorem{lem}{Lemma}
\newtheorem{cor}{Corollary}

\theoremstyle{remark}
\newtheorem*{rem}{Remark}

\newcommand{\rev}[1]{{\color{red}#1}}
%\def\multiplet{single particle multiplet}
\def\multiplet{single particle subspace}

\newcommand{\ak}[1]{{\color{blue}#1}}

\def\ir{{\mathrm i}}


\global\long\def\braket#1#2{\left\langle #1|#2\right\rangle }


%\renewcommand{\vec}[1]{\mathbf{#1}}

\def\pp{\mathbf{p}}
\def\qq{\mathbf{q}}


\newcommand{\eE}{\mathsf{e}}

\newcommand{\kind}{l}

\begin{document}

\title{Chiral bases for qubits and their applications to integrable spin chains}
%$XX0$  spin-$\frac12$ chain }
\author{Vladislav Popkov}
\affiliation{Faculty of Mathematics and Physics, University of Ljubljana, Jadranska 19, SI-1000 Ljubljana, Slovenia}
 \affiliation{Department of Physics,
  University of Wuppertal, Gaussstra\ss e 20, 42119 Wuppertal,
  Germany}
\author{Xin  Zhang}
\affiliation{Beijing National Laboratory for Condensed Matter Physics, Institute of Physics, Chinese Academy of Sciences, Beijing 100190, China}
\author{  Andreas Kl\"umper} 
\affiliation{Department of Physics,
 University of Wuppertal, Gaussstra\ss e 20, 42119 Wuppertal,
  Germany}


\begin{abstract}
%\rev{possibly to amend..}
We propose a novel qubit basis composed of transverse spin helices with
kinks. This chiral basis, in contrast to the usual computational basis,
possesses distinct topological properties and is convenient for describing
quantum states with nontrivial topology. By choosing appropriate parameters,
operators containing transverse spin components, such as $\sigma_n^x$ or
$\sigma_n^y$, become diagonal in the chiral basis, facilitating the study of
problems focused on transverse spin components. As an application, we describe
the transverse spin dynamics of a spin helix in the $XX$ model, which has
been studied in recent cold atom experiments. We derive a determinantal
formula for the temporal dependence of the transverse magnetization.
\end{abstract}
\maketitle

\section{Introduction}

A proper choice of basis for a scientific problem is the crucial first step
towards success. For example, the choice of a coherent state basis is crucial
for discussing the modes of the harmonic oscillator and related uncertainty
relation descriptions. The use of wavelets is natural for describing signals
confined in space or time \cite{Wavelets}, while the Fourier basis is natural
for solving linear differential equations with translational invariance in
space and time.

For qubits, i.e., quantum systems with spin-1/2 local degrees of freedom, the
most widely used basis is the computational basis, which is composed of
eigenstates of the $\sigma^z$ operator, i.e., $\sigma^z \binom{1}{0} =
\binom{1}{0}$ and $\sigma^z \binom{0}{1}= -\binom{0}{1}$, on every site. The
advantages of the computational basis are its factorized structure,
orthonormality, and $U(1)$-symmetry ``friendliness". In particular, the latter
property allows the grouping of basis vectors with an equal number of
spin-down arrows $m$, forming an invariant subset for any $m$ under
$U(1)$-invariant dynamics, such as the $XXZ$ model dynamics. The computational
basis is well-suited for multi-qubit states that belong to such a subset
(i.e., they are eigenstates of the total spin-$z$ magnetization, e.g., the
ground state) and for operators that do not change the total spin
magnetization, such as $\sigma_{n_1}^{+} \sigma_{n_2}^{-}$ or diagonal
operators $\sigma_{n_1}^z \sigma_{n_2}^z \ldots \sigma_{n_k}^z$.

However, the traditional computational basis appears poorly equipped to
describe states with nontrivial topology, such as chiral, current-carrying
states, or states with windings. One prominent example is the spin-helix
eigenstate (SHS) in a 1-dimensional spin chain.  
\begin{align}
&\ket{\Psi(\al_0,\eta)} = \bigotimes_n \ket{\phi(\al_0 + n\eta)},  \label{def:SHS}
\end{align}
where $\ket{\phi(\al)}$ describes the state of a qubit, while $+n\eta$
represents a linear increase in the qubit phase along the chain, under a proper
model-dependent parametrization. Thanks to their factorizability, spin
helices (\ref{def:SHS}) are straightforward to prepare in experimental setups
that allow for adjustable spin exchange, such as those involving cold
atoms. These helices have been found to possess nontrivial properties, as
evidenced by both experimental \cite{SHS-Ketterle2020,SHS-Ketterle} and
theoretical \cite{SHS-Phantom,Phantom-Long,ChiralBA,SHS-Hydro} investigations.

Since the spin helix state is not an eigenstate of the total $S^z$ operator, it is not confined to a single $U(1)$ block but is given by a sum over all the blocks, with fine-tuned coefficients, as shown in (\ref{eq:SHSinU(1)}) and (\ref{eq:PBRstates}), even for spatially homogeneous spin helices ($\eta=0$ in (\ref{def:SHS})). A simple shift of a helix phase $\al_0 \rightarrow \al_0 + const$ in (\ref{def:SHS}) gives a linearly independent state with the same qualitative properties (winding, current, etc.). However, to provide such a shift via the standard computational basis description, all the fine-tuned weights in the sum (\ref{eq:SHSinU(1)}) must be changed in a different manner.

Our goal is to introduce an alternative basis, all components of which are chiral themselves and thus ideally tailored for chiral states description. This basis consists of helices and helices with kinks (phase slides at certain links), and it provides another block hierarchy based on the number of kinks (instead of the number of spins down in the computational basis $U(1)$-blocks). Unlike the usual basis, the chiral basis is intrinsically topological.

For pedagogical purposes, we restrict ourselves to the simplest (and in fact, unique) case when the chiral basis can be made orthonormal without any additional orthogonalization procedure. In the following sections, we describe the basis and demonstrate its application to a problem of spin-helix state decay under $XX$ dynamics.

\section{Chiral multiqubit basis. }

The chiral basis vectors are constructed as site-factorized products of qubit
states, where the polarization of all qubits lies in the $XY$ plane and the
phases of consecutive qubits differ by a $\pi/2$ rotation in the $XY$ plane,
either clockwise or anticlockwise. A kink at position $n$ corresponds to a
$\pi/2$ anticlockwise rotation from site $n$ to site $n+1$. Thus, every chiral
basis vector can be fully characterized by a sequence of kink positions $n_1,
n_2, \dots, n_M$, and an initial phase parameterized by a parameter $u$ as
 \begin{align}
&\ket{u; n_1, \ldots , n_M} =
\eE^{-i\frac{\pi}{2} \sum_{j=1}^M n_j }\bigotimes_{k=1}^{n_1}   
\psi_k(u)   \bigotimes_{k=n_1+1}^{n_2}   
\psi_k(u+1)  \cdots  \bigotimes_{k=n_M+1}^{N}   \psi_k(u+M),  \label{eq:Mshock}\\
&1\leq n_1<n_2 < \ldots < n_M \leq N,\no\\
& \psi_k(u)=\frac{1}{\sqrt{2}}\binom {1}{\eE^{i\pi( \frac{k}{2}-u)}}. \label{eq:0shock}
\end{align}
The exponential phase-factor at the left-hand side of (\ref{eq:Mshock}) is introduced for further convenience.
It is worth noting that the property $\psi_k(u+2) =\psi_k(u) $ leads to periodicity of the chiral basis vectors
 \begin{align}
&\ket{u+2; n_1, \ldots , n_M} =\ket{u; n_1, \ldots , n_M}. \label{eq:periodicity}
\end{align}
For any given value of $u$, each local component of equation (\ref{eq:Mshock})
can be expressed as one of four local qubit states: $\psi_1(u), \psi_2(u),
\psi_3(u),$ or $\psi_4(u)$. These qubit states can be encoded using just two
binary registers, which allows for a convenient binary number-oriented
programming code representation of the chiral basis. If we choose $u =
\frac{1}{2}$, the first qubit in equation (\ref{eq:Mshock}) is polarized in
the positive $x$ direction. A state with clockwise rotations in all links can
be referred to as a chiral vacuum since it does not contain any kinks. The
polarizations of the qubits in the $XY-$plane of a chiral vacuum are
($\rightarrow$, $\uparrow$ correspond to qubits polarized in the positive $x$
and $y$ directions, respectively, etc.):
 \begin{align}
&\ket{+} \equiv \ket{\tfrac12} =\ket{ \rightarrow \downarrow \leftarrow  \uparrow \rightarrow \downarrow \leftarrow \uparrow \ldots}.  \label{def:vac}
\end{align}
The state described above has a spin helix form (\ref{def:SHS}) and is a chiral topological state with a pattern repeating every $4$ sites. The current of this state is given by $j^z=2\langle \si_n^x \si_{n+1}^y - \si_n^y \si_{n+1}^x\rangle= 2$.

A one-kink state $\ket{u;k}$ is a state where at the link $k,k+1$, there is an \textit{anticlockwise} rotation instead of a clockwise one, and the respective links are marked in red:
 \begin{align}
& \ket{\tfrac12; 1} =\ket{ \rev{\rightarrow \uparrow}  \rightarrow \downarrow \leftarrow  \uparrow \rightarrow \downarrow \leftarrow \uparrow \ldots}. \label{def:kink:1}
\end{align}
An example of a state with two kinks is
\begin{align}
& \ket{\tfrac12; 1,4} =\ket{ \rev{\rightarrow \uparrow} \rightarrow \rev{\downarrow \rightarrow} \downarrow \leftarrow \uparrow \rightarrow \downarrow \leftarrow \uparrow \ldots},\label{def:kink:2}
\end{align}
with kink positions marked in red. Finally, by inserting kinks at all
positions, we obtain an ideal spin helix state (SHS) again, but with a
chirality opposite to that of the state (\ref{def:vac}),
\begin{align}
&\ket{-} \equiv \ket{\tfrac12; 1,2,\ldots,N}=\ket{ \rightarrow \uparrow \leftarrow  \downarrow \rightarrow \uparrow \leftarrow \downarrow \ldots}. \label{def:vacNegative}
\end{align}
It is clear that states with an even number of kinks ($0,2,4,\dots$) such as
(\ref{def:vac}),(\ref{def:vacNegative}), (\ref{def:kink:2}) can be fitted into
periodic chains of size $N=4N'$. On the other hand, states with an odd number
of kinks fit into periodic chains of size $N=4N'+2$. Additionally, we can
define an operation $U^{z}=\eE^{\ir \pi
  \sum_{n}\sigma_{n}^{z}}=\prod_{n=1}^{N}\sigma_{n}^{z}$, which performs an
overall $180$-degree rotation of all qubits in the $XY$-plane. This operation
generates a linearly independent state with the same number of kinks. The
action of $U^{z}$ on a state is equivalent to shifting $u\rightarrow u+1$ in
Eq.~(\ref{eq:Mshock}), i.e.,
\begin{align}
U^{z}\ket{u;n_{1},\ldots,n_{M}}=\ket{u+1;n_{1},\ldots,n_{M}}.
\end{align}

We can now state the following theorem, which is proved in Appendix \ref{Appx;1}:

\textbf{Theorem I.} The set of states with an even number of kinks, in all possible positions $\ket{u}$, $\ket{u;k_{1},k_{2}},\dots,\ket{u;k_{1},k_{2},\dots,k_{2m-1},k_{2m}}$, where $m=1,2,\dots,2N'$, together with $U^{z}$-transformed vectors $\ket{u+1}$, $\ket{u+1;k_{1},k_{2}},\dots,\ket{u+1;k_{1},k_{2},\dots,k_{2m-1},k_{2m}}$, where $m=1,2,\dots,2N'$, provides an orthonormal basis in systems with $N=4N'$ qubits for arbitrary real $u$.

Similarly, the set of states with an odd number of kinks, $\ket{u;k_{1},\dots,k_{2m},k_{2m+1}}$, where $m=0,1,2,\dots,2N'$, together with $\ket{u+1;k_{1},\dots,k_{2m},k_{2m+1}}$, provides an orthonormal basis in systems with $N=4N'+2$ qubits, for arbitrary real $u$.

Completeness of the basis follows from easily verified relations
\begin{align*}
&2\sum_{m=0}^{2N'} \binom{N}{2m}= 2^N, \quad \mbox{if} \quad N=4N', \\
& 2\sum_{m=0}^{2N'} \binom{N}{2m+1}=2^N, \quad \mbox{if} \quad N=4N' +2.
\end{align*}
for even $N/2$ and odd $N/2$ respectively. Several remarks are in order.

\textit{Remark 1.} The chiral basis vectors have a topological nature, meaning that a kink cannot be removed from a periodic system through any sequence of local operations. In an open chain, a kink can only be removed at a boundary by first shifting it to the boundary. Similarly, a single kink cannot be added to a periodic chain.

\textit{Remark 2.} Applying $\sigma_n^z$ in a kink-free zone creates a kink
pair at neighboring positions $n-1$ and $n$. On the other hand, applying a
string of operators $\sigma_n^z \sigma_{n+1}^z \ldots \sigma_{n+k}^z$ in a
kink-free zone creates two kinks at a distance of $k+1$.  For instance,
\begin{align}
&\ket{\tfrac12;1,4} = \si_2^z \ \si_3^z \ \si_4^z \ket{\tfrac12}.
\label{def:creation2kinks}
\end{align}

\textit{Remark 3.} The connection between the chiral basis and the standard
computational basis is highly nontrivial. For example, the chiral vacuum state
is expanded in terms of the computational basis as 
\begin{align}
&\ket{u}=2^{-\frac{N}{2}}{\binom{N}{n}}^{\frac12}\sum_{n=0}^N \eE^{-i\pi n u}\ket{+,n},\label{eq:SHSinU(1)}\\
&\ket{+,n}\!=\!\frac{1}{n! \,{\binom{N}{n}}^{\frac12}} \sum_{\kind_1,\ldots ,\kind_n=1}^{N}\,
\eE^{i \frac{\pi}{2} (\kind_1+\ldots +\kind_n)}
\si_{\kind_1}^{-}\ldots  \si_{\kind_n}^{-}\, \binom{1}{0}^{\otimes_N},\label{eq:PBRstates}
\end{align}
see \cite{SHS-Phantom} for a proof.

Moving forward, we will explore two applications of the chiral basis. Firstly, we will use it to describe the eigenstates of the free fermion model. Secondly, we will apply it to describe the time evolution of the magnetization profile of a spin-helix state under free-fermion dynamics.

\section{Eigenstates of the $XX$ model expressed via the chiral basis}

The chiral basis introduced in this work is particularly well-suited for
describing the eigenstates of the free-fermion spin chain known as the $XX$
model. This model is defined on a chain of (even) size $N$,
\begin{align}
&H=\sum_{n=1}^N h_{n,n+1}, \quad h_{n,n+1}=\si_n^x \si_{n+1}^x+ \si_n^y \si_{n+1}^y, \quad \vec{\sigma}_{N+1}\equiv \vec{\sigma}_1, \label{def:XX0}
\end{align}
where the action of the Hamiltonian (\ref{def:XX0}) on any chiral basis vector
does not change the number of kinks. Correspondingly, a subset of chiral
states with fixed number of kinks is an invariant subspace of the $XX0$
Hamiltonian, provided that the chain size $N$ is even, $N=4 N'+2$ or $N=4
N'$. For example, when considering the $1$-kink state in a chain of size
$N=4N'+2$, we obtain the following result:
\begin{align}
&H \ket{u;n} = 2 \ket{u;n-1}+ 2 \ket{u;n+1}, \quad n \neq 1,N \label{def:HamActionKink1}\\
&H \ket{u;1} = -2 \ket{u+1;N}+ 2 \ket{u;2}, \label{def:HamActionKink1-1} \\
&H \ket{u;N} = -2 \ket{u+1;1}+ 2 \ket{u;N-1}. \label{def:HamActionKink1-N}  
\end{align}
It follows from Eqs.(\ref{def:HamActionKink1})-(\ref{def:HamActionKink1-N})
and the identity $\ket{u;n}\equiv \ket{u+2;n}$ that the set of 1-kink chiral
states $\{ \ket{u;n} \}_{n=1}^N \cup \{ \ket{u+1;n} \}_{n=1}^N $ forms an
invariant subspace of $H$.  The $2N$ eigenstates of $H$ corresponding to this
subspace can be found with the ansatz
\begin{align}
& \ket{\mu_1(p)}= \frac{1}{\sqrt{2N}}\sum_{n=1}^N  \eE^{i p n} \left(\ket{u;n}-\eE^{i p N} \ket{u+1;n}\right),  \label{eq:kink1Eigen}\\
&\eE^{i p N} = \pm 1,
\end{align}
where $p$ is a chiral analogue of a  quasimomentum,
related to the conservation of the kink number under the free fermion dynamics. 

Note that for brevity, we shall omit the explicit dependence of the wavefunction on the parameter $u$ in the following discussion. It can be easily verified that the states given by Equation (\ref{eq:kink1Eigen}) are eigenstates of the Hamiltonian $H$ with an eigenvalue of $E_p=4\cos p$.

For a chain of size $N=4N'$, where the chiral basis with an even number of kinks applies, we observe that all chiral vacua are eigenstates of $H$ with an eigenvalue of $0$,
\begin{align}
H \ket{u}=H \ket{u; 1,2, \ldots ,N} =0, \quad u\in\mathbb{C},   
\end{align}
and in the 2-kink sector some key relations are  
\begin{align}
\begin{aligned}\label{def:HamActionKink2}
&H \ket{u;n,m} = 2 \ket{u;n-1,m}+ 2 \ket{u;n+1,m} + 2 \ket{u;n,m+1}+ 2 \ket{u;n,m-1},\,  \ 1<n<m<N,
 \\
&H \ket{u;n,n+1} = 2 \ket{u;n-1,n+1}+ 2 \ket{u;n,n+2}, \quad n \neq 1, N-1,\\% \label{def:HamActionKink2near}\\
&H \ket{u;1,n} = 2 \ket{u+1; n,N} + 2 \ket{u;2,n} +  2 \ket{u; 1,n-1} + 2 \ket{u;1,n+1} , \quad 2<n< N, \\
&H \ket{u;1,N} = 2 \ket{u; 2,N} + 2 \ket{u;1,N-1}, \quad etc. 
\end{aligned}
\end{align}
The complete set of 2-kink relations is provided in Appendix \ref{Appx;2}, which demonstrates the invariance of the 2-kink chiral state subspace ${\ket{u;n,m}}{n<m}\cup{\ket{u+1;n,m}}{n<m}$ under the action of $H$.

It can be verified that the full set of $2$-kink eigenstates for the $XX0$ model can be expressed in terms of a chiral version of the coordinate Bethe Ansatz
\begin{align}
& \ket{\mu_2 (p_1,p_2)}= \frac{1}{\sqrt{2}\  N}\sum_{n<m} \left(\eE^{i p_1 n+i p_2 m }-\eE^{i p_2 n+i p_1 m }
\right) (\ket{u;n,m}  -  \eE^{i p_1 N} \ket{u+1;n,m}), 
\label{eq:kink2Eigen}\\
&H \ket{ \mu_2(p_1,p_2)}  = 4(\cos p_1 + \cos p_2)\ket{ \mu_2(p_1,p_2)}, 
\end{align}
where \textit{both} $p_1,p_2$ satisfy either $\eE^{i p_k N} = 1$ or $\eE^{i p_k N} = -1$, $k=1,2$, $p_1\neq p_2$. 

It is interesting to note an analogy of the above eigenstates in Eq.~(\ref{eq:kink2Eigen}) with the standard coordinate Bethe Ansatz for the $XX0$ wave function in the invariant block with two spins down, $\ket{\vfi_{n,m}}=\si_n^-\si_m^-\bigotimes_{k=1}^N\binom{1}{0}$, which is 
$$ \sum_{n<m}\left(\eE^{i p_1 n+i p_2 m } - \eE^{i p_2 n+i p_1 m }\right)
\ket{\vfi_{n,m}},\quad \eE^{i p_1 N}=\eE^{i p_2 N}=-1,\quad p_1\neq p_2.$$
%comparing to (\ref{eq:kink2Eigen}). 

The complete set of $XX0$ eigenvectors in the chiral basis is given by the
following Theorem:

\bigskip 

 \textbf{Theorem II.~} Eigenstates of the $XX0$ Hamiltonian are given by
 states $\ket{\mu_M(\bfvec{p})}$, with even kink number $M=0,2,4,\ldots,N$
 for $N/2$ even, and odd $M=1,3,5,\ldots,N-1$ for $N/2$ odd. Each eigenstate
  $\ket{\mu_M(\bfvec{p})}$ is characterized by an $M$-tuple of chiral quasimomenta
 $\bfvec{p}={p_1, p_2, \ldots,p_M}$, where all $p_j$ in the $M$-tuple satisfy
 either $\eE^{ip_j N}=1$ or $\eE^{ip_j N}=-1$. The explicit form of the
 eigenstates is
  \begin{align}
& \ket{\mu_M(\bfvec{p})}= \frac{1}{M!}\sum_{n_1 n_2 \ldots n_M=1}^{N} T_{\bfvec{n}} \ \chi_{{n}_1 {n}_2 \ldots {n}_M} (\bfvec{p}) \left(
\ket{u;\tilde{n}_1,\tilde{n}_2,\ldots,\tilde{n}_M}-\eE^{ip_1 N}   \ket{u+1;\tilde{n}_1,\tilde{n}_2,\ldots,\tilde{n}_M}\right) \no \\
&\hspace{1.35cm}=\sum_{n_1 <n_2 \ldots< n_M} \chi_{n_1 n_2 \ldots n_M} (\bfvec{p}) \left(
\ket{u;n_1,n_2,\ldots, n_M} - \eE^{ip_1 N}   \ket{u+1;n_1,n_2,\ldots, n_M} \right),  \label{eq:mu}\\
&\chi_{n_1 n_2 \ldots n_M} = \frac{1}{\sqrt{2\  N^M} }
\sum_{Q} (-1)^Q \,\eE^{i \sum_{j=1}^M n_j  p_{Q_j} }, \label{eq:chi}
&%\ket{u;\tilde n_1, \ldots , \tilde n_M} =e^{-i\frac{\pi}{2}(n_1+ \ldots +n_M)}\bigotimes_{k=1}^{n_1}   \psi_k(x) \ldots  \bigotimes_{k=n_M+1}^{N}   \psi_k(x+M), \quad   \psi_n(x) = \frac{1}{\sqrt{2}}\binom {1}{e^{ n \eta - 2 x \eta  }} \label{eq:Mshock}\\& 1\leq n_1<n_2 < \ldots < n_M \leq N.\no
\end{align}
where in the first line of  (\ref{eq:mu}), the ordered set  $\{\tilde n_1\leq \ldots \leq \tilde n_M\}$ is 
obtained from the  set $\{n_1, n_2,\ldots,n_M\}$ by reordering.
The corresponding  eigenvalue is  
\begin{align}
E_{\bfvec{p}} = 4\sum_{j=1}^{M} \cos p_j.
\end{align}
The factor $T_{\bfvec{n}} $ is equal to $1$ or $-1$, depending on whether the
set ${\bfvec{n}}=\{n_1,n_2 ,\ldots,n_M\}$ is obtained from the ordered set by
an even or odd number of transpositions, $Q$ is a permutation of indices
$Q(1,2,\ldots,M ) = (Q_1 , Q_2,\ldots ,Q_M )$ and $(-1)^Q$ denotes the sign of
the permutation.  Note, that the wave function changes sign if the kink
positions $n_i,n_j$ or ``chiral quasimomenta'' $p_i,p_j$ are exchanged,
meaning that the value $T_{\mathbf n} $ for a set $n_1 n_2 \ldots n_M$ where some
$n_k,n_j$ (or some $p_k,p_j$) coincide is irrelevant and could be set to
zero. For real $u$ the wavevectors $\mu_M(\bfvec{p})$ are orthonormal, $\bra
{\mu_M(\bfvec{p})} {\mu_{M'}(\bfvec{p'})} \rangle=\de_{\bfvec{p},\bfvec{p'}}
\de_{M,M'}$.


The proof of Theorem II is given in Appendix \ref{Appx;2}. 

\textit{Remark.} Strikingly, the $XX0$ eigenstates in the chiral topological
basis given in Theorem II, closely resemble those for the usual
computational basis \cite{1993ColomoXX0basis}, where the number of spins up plays
the role of the number of kinks. In particular, the wave function's amplitudes
(\ref{eq:chi}) have a familiar form of Slater determinants.

\section{Spin-helix state time evolution under free fermion dynamics}

In the last section we apply our chiral basis to the study of time evolution of a transverse 
spin-helix state magnetization profile under free fermion dynamics as experimentally investigated in \cite{SHS-Ketterle}.
From the theoretical viewpoint,  we  are interested in the time evolution of
one-point correlations of a spin-helix state (SHS)
%$\ket{SHS_Q}$ 

\begin{align}
\ket{\Psi_Q}= \frac{1}{\sqrt{2^N}}\bigotimes_{n=1}^{N}\binom {1}{\eE^{i(n-1)Q}},
\label{eq:SHSQ}
\end{align}
with arbitrary wavevector $Q$, satisfying the commensurability condition
\begin{align}
&Q N = 0 \ \mod \ 2\pi,  \label{def:commensurate}
\end{align}
under the XX0  Hamiltonian
\begin{align}
&  \langle \si_{n}^\al(t)\rangle_Q = \bra{\Psi_Q} \eE^{i H t}  \  \si_n^\al \  \eE^{-i H t} \ket{\Psi_Q}. \label{eq:Sn-alpha}
\end{align}
Under this condition the initial magnetization profile satisfies $\langle
\si_{n+1}^\pm(0) \rangle_Q= e^{\pm i Q } \langle \si_n^\pm(0) \rangle_Q$ for
$\forall n$.  Then, the following properties hold at all times \cite{elsewhere}

%\footnote{The
 % properties (\ref{eq:Sx}), (\ref{eq:Sy}), apart from self-similarity of $S_N$
 % for different $Q$, follow from the translational symmetry and the mirror
 % symmetry of the $XX0$ Hamiltonian, while (\ref{eq:Sz}) is a consequence of
 % the $U(1)$ symmetry, see the Appendix. On the other hand, the
 % self-similarity property $S_N(Q,t) = S_N(0,t\cos Q)$ is more intricate, see
%elsewhere for details.}
\begin{align}
& \langle \si_{n}^x(t)\rangle_Q= S_N(t \cos Q) \cos (Q(n-1)),    \label{eq:Sx}\\
& \langle \si_{n}^y(t)\rangle_Q= S_N(t \cos Q) \sin (Q(n-1)),   \label{eq:Sy} \\
& \langle \si_{n}^z(t)\rangle_Q=0,\qquad  n=1,2,\ldots, N. \label{eq:Sz}
\end{align}

Consequently, the full information about the one-point correlation function
$\langle \si_{n}^\al(t) \rangle_Q$ at all times is given by a single
real-valued function $S_N(t)= \langle \si_{1}^x(t) \rangle_0$, calculated from
(\ref{eq:Sn-alpha}) for the homogeneous case $Q=0$,
\begin{align}
&S_N(t)=\bra{\Omega}\eE^{i H t}\si_1^x \eE^{-i H t} \ket{\Omega}, \label{eq:SN(t)}\\
&\ket{\Omega}\equiv \ket{\Psi_0}=\frac{1}{\sqrt{2^N}}\bigotimes_{n=1}^{N}  \binom {1}{1}.\label{def;Omega}
\end{align}
The state $\ket{\Omega}$ is a factorized state with all spins polarized in the
positive $x$ direction.  Note that the quantity $S_N(t)$ cannot be computed
via free fermion techniques involving a Jordan-Wigner diagonalization and a use
of Wick's theorem, because the reduced density matrices stemming from
$\ket{\Omega}$ are not Gaussian (exponential of 
bilinear expressions) operators in terms of the fermionic operators
$c_j$, see \cite{2022AresCalabrese}.

The key simplification in calculating $S_N(t)$ using the chiral basis consists
in the fact that under an appropriate choice of the phase $u$ in (\ref{eq:Mshock}),
the operator $\si_1^x$ becomes diagonal when written in this basis:
\begin{align}
&\si_1^x \ket{\pm \tfrac12; n_1, \ldots,n_M} = \pm \ket{\pm \tfrac12; n_1, \ldots, n_M}, \quad \forall M,\label{eq:SigmaxAction}
\end{align}
leading to 
\begin{align}
&\bra{\mu_{M'}(\bfvec{q}) }  \si_1^x  \ket{\mu_M(\bfvec{p})}=0, \quad \mbox{if $M \neq M'$,} \label{eq:BlockDiagonalizationSigmaX}
\end{align}
if we choose $u=\frac12$ for the $XX0$ basis in Theorem II. The choice $u=
\frac12$ is kept for the remaining part of the manuscript.

For convenience, we first define 
\begin{align}
M_0=\frac{N}{2}.
\end{align}
By inserting two representations of unity $I= \sum_{\pp,M}\ket{\mu_M(\pp)}
\bra{\mu_M(\pp)}$ in (\ref{eq:SN(t)}) and using
(\ref{eq:BlockDiagonalizationSigmaX}) we obtain
\begin{align}
&S_N(t) = \sum_{\pp,\qq,M} \braket{\Omega}{\mu_M(\pp)} \eE^{i (E_\pp-E_\qq) t} \bra{\mu_M(\pp)} \si_1^x \ket{\mu_M(\qq)} \braket{\mu_M(\qq)}{\Omega}, \label{eq:S(t)}
\end{align}
valid for even system sizes $N$, where for $N/2$ being even (odd) the sum over
even (odd) $M$ values is taken. In the next step we find that overlaps
$\braket{\Omega} {\mu_M(\pp)} = 0$ for all $M \neq M_0$, so that the nonzero
contribution to $S_N$ is given by one block with $M_0$ kinks in
(\ref{eq:S(t)}). 
  
Furthermore, we find that the term $\bra{\mu_{M}(\pp)} \si_1^x \ket{\mu_{M}(\qq)} \neq 0$ if and only if the sets $\pp,\qq$ have different
parities, i.e. $\eE^{i p_k N}=1$ (even parity quasimomenta) and $\eE^{i q_k N}=-1$ (odd parity quasimomenta) for all $k=1,2, \ldots, M_0$, or vice versa. 
The contributions of $\pp,\qq$ being of even/odd parity, and being of odd/even parity in (\ref{eq:S(t)}) turn out to be identical, so 
 (\ref{eq:S(t)}) reduces to 
\begin{align}
&S_N(t)= 2\sum_{\pp  \, \mbox{even}} \sum_{\qq  \, \mbox{odd}} \cos ((E_\pp-E_\qq) t) 
\braket{\Omega} {\mu(\pp)} \bra{\mu(\pp)} \si_1^x \ket{\mu(\qq)} \braket{\mu(\qq)} {\Omega} \label{eq:S(t)-1}.
\end{align}
where we used shorthand notation $\ket{\mu(\pp)} \equiv \ket{\mu_{M}(\pp)}|_{M=M_0}$.
Remarkably,  both nonzero overlaps $\braket{\Omega} {\mu(\pp)}$ and $\bra{\mu(\pp)} \si_1^x \ket{\mu(\qq)} $
can be expressed as a 
$M_0 \times M_0 $ determinant (see Appendices \ref{Appx;3}, \ref{Appx;4}):
\begin{align}
&\bra{\mu(\pp)} \si_1^x \ket{\mu(\qq)} =M_0^{-M_0} \det F(\pp,\qq), 
\quad 
F_{nm}(\pp,\qq)=  \frac {1}{\eE^{i(p_m-q_n)}-1}, \quad \quad n,m=1,2,\ldots M_0,    \label{res:DetF} \\
&  \braket{\Omega} {\mu(\pp)} =\frac{K_N}{  \sqrt{2^{N+1} N^{M_0}} }\det G(\pp),
\quad  G_{nm}(\pp)= 
  \eE^{2i n p_m} 
\left(1+ \eE^{-i p_m} \right),     \label{res:DetOverlapG} \\
&K_N = 
\left\{
\begin{array}{ccc}
&(-2 i)^{\frac{N}{4}}, &\quad \frac{N}{2} \mbox {even},\\[2pt]
&- (1+i)\,(-2 i)^{\frac{N-2}{4}},&  \quad \frac{N}{2} \ \mbox {odd}.
\end{array}
\right.
\end{align}
Substituting into (\ref{eq:S(t)-1}) we finally obtain
\begin{align}
&S_N(t) = \frac{1}{N^N}   \sum_{\pp  \, \mbox{even}} \sum_{\qq  \, \mbox{odd}} \cos ((E_\pp-E_\qq) t) \ 
\det G(\pp) \det G(-\qq) \det F(\pp,\qq),
\label{eq:S(t)-2}
\end{align}
where we used $\det G(\qq)^*= \det G(-\qq)$. The size dependence in (\ref{eq:S(t)-2}) enters through the prefactor and the dimension of the determinants, which is $M_0\times M_0$.
The full even parity set $\pp$ consists of  strictly different  $\{p_1, p_2, \ldots , p_{M_0} \}$, given by $p_k= 2 \pi m_k/N$, $m_k$ being integer and  $p_k $ being all different and 
analogously the components $q_k $ of the odd parity set are     $q_k= \pi/N+ 2 \pi m_k/N$. 


The  analytic expressions for the determinants  $\det F,\, \det G$ can be obtained, namely:
\begin{align}
&\det G(\pp) =
\eE^{i \,\frac{N+1}{2}  \sum_{k} p_k} \ 2^{\frac{M_0(M_0+1)}{2}} \ (-i)^{\frac{M_0(M_0-1)}{2}}
\prod_{1\leq n_1<n_2\leq M_0} \sin (p_{n_1}-p_{n_2}) \,
 \prod_{n=1}^{M_0}  \cos\frac{p_{n}}{2}, 
\label{res:SHSoverlap}\\
&\det F(\pp,\qq) =2^{-M_0} A_N \  \eE^{-\frac{i}{2} \sum_{k=1}^{M_0} (p_k-q_k)} \prod_{1\leq n_1<n_2\leq M_0} \sin\frac{p_{n_1}-p_{n_2}}{2} \,   \sin\frac{q_{n_1}-q_{n_2}}{2} 
\prod_{n,m=1}^{M_0} \csc\frac{p_{n}-q_{m}}{2},
\label{res:SigmaXoverlap}\\
&A_N =
\left\{
\begin{array}{ccc}
&1 , &\quad M_0=\frac{N}{2} \ \mbox {even},\\[4pt]
&-i, & \quad  M_0=\frac{N}{2} \ \mbox {odd}.
\end{array}
\right.
 \label{res:A(N)}
\end{align}

Substituting the above into (\ref{eq:S(t)-2}), we get explicit expressions for
the decay rate $S_N(t)$  for different system sizes, e.g.~for $N=4, 6$ we have 
\begin{align}
&S_{4}(t) = \frac{1}{8} \left(2 \cos (4 t)+\left(3+2 \sqrt{2}\right) \cos \left(4 \left(\sqrt{2}-1\right)
   t\right)+\left(3-2 \sqrt{2}\right) \cos \left(4 \left(1+\sqrt{2}\right) t\right)\right) \\
&S_{6}(t) =\frac{1}{96} \left(8 \cos (4 t)+2 \cos (8 t)+4 \cos \left(4 \sqrt{3} t\right)+\left(26+15
   \sqrt{3}\right) \cos \left(4 \left(\sqrt{3}-2\right) t\right)+\right.  \nonumber   \\
& +\left.  2 \left(7+4 \sqrt{3}\right) \cos
   \left(4 \left(\sqrt{3}-1\right) t\right)+2 \left(7-4 \sqrt{3}\right) \cos \left(4
   \left(1+\sqrt{3}\right) t\right)+\left(26-15 \sqrt{3}\right) \cos \left(4 \left(2+\sqrt{3}\right)
   t\right)+2\right),
\end{align}
while for larger $N$ the expressions for $S_N(t)$ get more bulky.
Fig.~\ref{Fig-SN(t)} shows numerically exact functions $S_N(t)$ for different
system sizes computed with the help of Mathematica.  The obvious tendency of
relaxation to zero of the off-diagonal components of the one-site reduced
density matrix can be explained as a result of dynamical restoration of an
initially broken global $U(1)$ symmetry (present in the Hamiltonian, but
broken by the choice of the initial state $\ket{\Omega}$) in the infinite
system \cite{2022AresCalabrese}.


Interestingly, analysing the expressions $S_N(t)$ for finite sizes gives exact
information about $S_N$ in the thermodynamic limit $N \rightarrow \infty$.
Namely, analyzing the coefficients of the Taylor expansion of $S_N(t)= \sum_n
C_N(n) t^N$ at $t=0$ we observe that the sets of coefficients $C_N(n)$ follow
stable patterns as $N$ increases. So, the Taylor expansions for $N=4$ and $N=6$
coincide up to $t^4$ terms, the Taylor expansions for $N=6$ and $N=8$ coincide
up to $t^8$ terms, the Taylor expansions for $N=8$ and $N=10$ coincide up to
$t^{12}$ terms, etc. Consequently, the stable pattern gives the exact Taylor
expansion of $S_\infty(t)$ at $t=0$, which we were able to find up to 
order $t^{24}$:

\begin{align}
S_\infty(t) =& \,1-4 t^2 + \frac{2^5}{3}t^4 - \frac{2^6}{3}t^6 + \frac{2^9}{15}t^8 -
\frac{2048}{45}t^{10}+
\frac{733184}{14175}   t^{12}  -\frac{720896}{14175} t^{14} + \frac{195756032}{4465125} t^{16} \nonumber \\
&-\frac{149946368}{4465125} t^{18} +\frac{7311720448
   }{315748125} t^{20}-\frac{32153534464}{2210236875} t^{22}+
\frac{167498959814656}{19912024006875} t^{24}+
O(t^{26}).  \label{eq:Taylor}
\end{align}


\begin{figure}[tbp]
\centerline{
\includegraphics[width=0.5\textwidth]{FigSN.pdf}
\includegraphics[width=0.5\textwidth]{FigLogSN.pdf}
}
\caption{ Relaxation of the amplitude $S_N(t)$ of a state fully polarized in
  positive $x$ direction,  under $XX0$ dynamics for different system sizes, in
  usual scale (left panel) and in logarithmic scale (right panel).  \textbf{
    Left Panel:} Green, red and black dots correspond to the exact $S_N(t)$
  for periodic systems of $N=6, 8, 10$ sites respectively, while the
  continuous curve shows $S_{20}(t)$.  This figure is an analog of Fig.~2a in
  \cite{SHS-Ketterle}.  \textbf{Right Panel:} $S_N(t)$ for $N=6, 8,\ldots ,22$
  shows exponential decay for times $t>1$, approximately given via
  $S_\infty(t) \approx 1.5\ \eE^{-2.54 t}$ for large times.  The dashed
  straight line corresponding to the function $A \eE^{-2.54 t}$ is a guide to
  the eye.  }
\label{Fig-SN(t)}
\end{figure}
 


\begin{figure}[tbp]
\centerline{
\includegraphics[width=0.5\textwidth]{DecayDe0.pdf}
}
\caption{ Asympotic decay rate $\ga$ of the SHS relaxation in the
  infinite system under $XX0$ dynamics $\ga(Q)= -\lim\limits_{t\rightarrow
    \infty} \frac{\log S_\infty(Q,t)}{t} $ versus rescaled wavevector $Q/\pi$,
  from Eq.~(\ref{eq:Sx}). The line is given by $\ga_0 \ |\cos (\pi x )|$,
  where $\ga_0 \approx 2.54$ is the estimate from $S_{22}(t)$ data, see the
  right panel of Fig.~\ref{Fig-SN(t)}.  This Figure is to be compared with the
  analogous Fig.~3c of \cite{SHS-Ketterle}.  }
\label{Fig-gammaDecay}
\end{figure}

\section{Conclusions}

In this work, we propose a chiral qubit basis that possesses topological properties while retaining a simple factorized structure and orthonormality. This basis is represented by just four local qubit states, organized into two pairs of mutually orthogonal qubit states, making it practical for implementation with just two registers in binary code. We demonstrate the effectiveness of this basis by applying it to the evolution of a special spin-helix state under $XX0$ dynamics, showing how it can be used to solve physical problems, as seen in the determinantal representation (\ref{eq:S(t)-2}). Our results are compared to an experiment in Figs.~\ref{Fig-SN(t)} and \ref{Fig-gammaDecay}.

As a byproduct of our study, we also discover the existence of a universal function $S_\infty(t)$ that governs the relaxation of transversal spin helices with arbitrary wavelengths in an infinite system. Although we do not determine its explicit form, we calculate some of its characteristics, such as the first 24 coefficients of its Taylor expansion (\ref{eq:Taylor}), and provide its explicit determinantal form in (\ref{eq:S(t)-2}) for arbitrary system sizes, which can be analyzed further in the $N \rightarrow \infty$ limit. Furthermore, we obtain explicit expressions for spin-helix state relaxation for finite systems of qubits from (\ref{eq:S(t)-2}), which could be useful for future experiments, such as those with ring-shaped atom arrays \cite{RingShapedArrays}, where periodic boundary conditions are realized.
\begin{acknowledgments}
 Financial support from the Deutsche Forschungsgemeinschaft through DFG
 project KL 645/20-2, is gratefully acknowledged. V.~P. acknowledges support by
 European Research Council (ERC) through the advanced Grant
 No. 694544—OMNES. X. Z. acknowledges the financial support from the National Natural Science
Foundation of China (No. 12204519).
\end{acknowledgments}

%\bibliographystyle{apsrev4-1}
%\bibliography{ChiralBasis}

\begin{thebibliography}{11}%
\makeatletter
\providecommand \@ifxundefined [1]{%
 \@ifx{#1\undefined}
}%
\providecommand \@ifnum [1]{%
 \ifnum #1\expandafter \@firstoftwo
 \else \expandafter \@secondoftwo
 \fi
}%
\providecommand \@ifx [1]{%
 \ifx #1\expandafter \@firstoftwo
 \else \expandafter \@secondoftwo
 \fi
}%
\providecommand \natexlab [1]{#1}%
\providecommand \enquote  [1]{``#1''}%
\providecommand \bibnamefont  [1]{#1}%
\providecommand \bibfnamefont [1]{#1}%
\providecommand \citenamefont [1]{#1}%
\providecommand \href@noop [0]{\@secondoftwo}%
\providecommand \href [0]{\begingroup \@sanitize@url \@href}%
\providecommand \@href[1]{\@@startlink{#1}\@@href}%
\providecommand \@@href[1]{\endgroup#1\@@endlink}%
\providecommand \@sanitize@url [0]{\catcode `\\12\catcode `\$12\catcode
  `\&12\catcode `\#12\catcode `\^12\catcode `\_12\catcode `\%12\relax}%
\providecommand \@@startlink[1]{}%
\providecommand \@@endlink[0]{}%
\providecommand \url  [0]{\begingroup\@sanitize@url \@url }%
\providecommand \@url [1]{\endgroup\@href {#1}{\urlprefix }}%
\providecommand \urlprefix  [0]{URL }%
\providecommand \Eprint [0]{\href }%
\providecommand \doibase [0]{https://doi.org/}%
\providecommand \selectlanguage [0]{\@gobble}%
\providecommand \bibinfo  [0]{\@secondoftwo}%
\providecommand \bibfield  [0]{\@secondoftwo}%
\providecommand \translation [1]{[#1]}%
\providecommand \BibitemOpen [0]{}%
\providecommand \bibitemStop [0]{}%
\providecommand \bibitemNoStop [0]{.\EOS\space}%
\providecommand \EOS [0]{\spacefactor3000\relax}%
\providecommand \BibitemShut  [1]{\csname bibitem#1\endcsname}%
\let\auto@bib@innerbib\@empty
%</preamble>
\bibitem [{\citenamefont {Mallat}(2009)}]{Wavelets}%
  \BibitemOpen
  \bibfield  {author} {\bibinfo {author} {\bibfnamefont {S.}~\bibnamefont
  {Mallat}},\ }\href@noop {} {\emph {\bibinfo {title} {A Wavelet Tour of Signal
  Processing: the Sparse Way, 3rd Revised edition}}}\ (\bibinfo  {publisher}
  {Academic Press, London},\ \bibinfo {year} {2009})\BibitemShut {NoStop}%
\bibitem [{\citenamefont {Jepsen}\ \emph {et~al.}(2020)\citenamefont {Jepsen},
  \citenamefont {Amato-Grill}, \citenamefont {Dimitrova}, \citenamefont {Ho},
  \citenamefont {Demler},\ and\ \citenamefont {Ketterle}}]{SHS-Ketterle2020}%
  \BibitemOpen
  \bibfield  {author} {\bibinfo {author} {\bibfnamefont {P.~N.}\ \bibnamefont
  {Jepsen}}, \bibinfo {author} {\bibfnamefont {J.}~\bibnamefont {Amato-Grill}},
  \bibinfo {author} {\bibfnamefont {I.}~\bibnamefont {Dimitrova}}, \bibinfo
  {author} {\bibfnamefont {W.~W.}\ \bibnamefont {Ho}}, \bibinfo {author}
  {\bibfnamefont {E.}~\bibnamefont {Demler}},\ and\ \bibinfo {author}
  {\bibfnamefont {W.}~\bibnamefont {Ketterle}},\ }\href
  {https://doi.org/10.1038/s41586-020-3033-y} {\bibfield  {journal} {\bibinfo
  {journal} {Nature}\ }\textbf {\bibinfo {volume} {588}},\ \bibinfo {pages}
  {403} (\bibinfo {year} {2020})}\BibitemShut {NoStop}%
\bibitem [{\citenamefont {Jepsen}\ \emph {et~al.}(2022)\citenamefont {Jepsen},
  \citenamefont {Lee}, \citenamefont {Lin}, \citenamefont {Dimitrova},
  \citenamefont {Margalit}, \citenamefont {Ho},\ and\ \citenamefont
  {Ketterle}}]{SHS-Ketterle}%
  \BibitemOpen
  \bibfield  {author} {\bibinfo {author} {\bibfnamefont {P.~N.}\ \bibnamefont
  {Jepsen}}, \bibinfo {author} {\bibfnamefont {Y.~K.}\ \bibnamefont {Lee}},
  \bibinfo {author} {\bibfnamefont {H.}~\bibnamefont {Lin}}, \bibinfo {author}
  {\bibfnamefont {I.}~\bibnamefont {Dimitrova}}, \bibinfo {author}
  {\bibfnamefont {Y.}~\bibnamefont {Margalit}}, \bibinfo {author}
  {\bibfnamefont {W.~W.}\ \bibnamefont {Ho}},\ and\ \bibinfo {author}
  {\bibfnamefont {W.}~\bibnamefont {Ketterle}},\ }\href
  {https://doi.org/10.1038/s41567-022-01651-7} {\bibfield  {journal} {\bibinfo
  {journal} {Nature Physics}\ }\textbf {\bibinfo {volume} {18}},\ \bibinfo
  {pages} {899} (\bibinfo {year} {2022})}\BibitemShut {NoStop}%
\bibitem [{\citenamefont {Popkov}\ \emph {et~al.}(2021)\citenamefont {Popkov},
  \citenamefont {Zhang},\ and\ \citenamefont {Kl\"umper}}]{SHS-Phantom}%
  \BibitemOpen
  \bibfield  {author} {\bibinfo {author} {\bibfnamefont {V.}~\bibnamefont
  {Popkov}}, \bibinfo {author} {\bibfnamefont {X.}~\bibnamefont {Zhang}},\ and\
  \bibinfo {author} {\bibfnamefont {A.}~\bibnamefont {Kl\"umper}},\ }\href
  {https://link.aps.org/doi/10.1103/PhysRevB.104.L081410} {\bibfield  {journal}
  {\bibinfo  {journal} {Phys. Rev. B}\ }\textbf {\bibinfo {volume} {104}},\
  \bibinfo {pages} {L081410} (\bibinfo {year} {2021})}\BibitemShut {NoStop}%
\bibitem [{\citenamefont {Zhang}\ \emph
  {et~al.}(2021{\natexlab{a}})\citenamefont {Zhang}, \citenamefont
  {Kl{\"u}mper},\ and\ \citenamefont {Popkov}}]{Phantom-Long}%
  \BibitemOpen
  \bibfield  {author} {\bibinfo {author} {\bibfnamefont {X.}~\bibnamefont
  {Zhang}}, \bibinfo {author} {\bibfnamefont {A.}~\bibnamefont {Kl{\"u}mper}},\
  and\ \bibinfo {author} {\bibfnamefont {V.}~\bibnamefont {Popkov}},\ }\href
  {https://link.aps.org/doi/10.1103/PhysRevB.103.115435} {\bibfield  {journal}
  {\bibinfo  {journal} {Phys. Rev. B}\ }\textbf {\bibinfo {volume} {103}},\
  \bibinfo {pages} {115435} (\bibinfo {year} {2021}{\natexlab{a}})}\BibitemShut
  {NoStop}%
\bibitem [{\citenamefont {Zhang}\ \emph
  {et~al.}(2021{\natexlab{b}})\citenamefont {Zhang}, \citenamefont
  {Kl{\"u}mper},\ and\ \citenamefont {Popkov}}]{ChiralBA}%
  \BibitemOpen
  \bibfield  {author} {\bibinfo {author} {\bibfnamefont {X.}~\bibnamefont
  {Zhang}}, \bibinfo {author} {\bibfnamefont {A.}~\bibnamefont {Kl{\"u}mper}},\
  and\ \bibinfo {author} {\bibfnamefont {V.}~\bibnamefont {Popkov}},\ }\href
  {https://doi.org/10.1103/PhysRevB.104.195409} {\bibfield  {journal} {\bibinfo
   {journal} {Phys. Rev. B}\ }\textbf {\bibinfo {volume} {104}},\ \bibinfo
  {pages} {195409} (\bibinfo {year} {2021}{\natexlab{b}})}\BibitemShut
  {NoStop}%
\bibitem [{\citenamefont {Cecile}\ \emph {et~al.}()\citenamefont {Cecile},
  \citenamefont {Gopalakrishnan}, \citenamefont {Vasseur},\ and\ \citenamefont
  {De~Nardis}}]{SHS-Hydro}%
  \BibitemOpen
  \bibfield  {author} {\bibinfo {author} {\bibfnamefont {G.}~\bibnamefont
  {Cecile}}, \bibinfo {author} {\bibfnamefont {S.}~\bibnamefont
  {Gopalakrishnan}}, \bibinfo {author} {\bibfnamefont {R.}~\bibnamefont
  {Vasseur}},\ and\ \bibinfo {author} {\bibfnamefont {J.}~\bibnamefont
  {De~Nardis}},\ }\href {https://arxiv.org/abs/2211.03725} {\bibinfo  {journal}
  {arXiv:2211.03725}\ }\BibitemShut {NoStop}%
\bibitem [{\citenamefont {Colomo}\ \emph {et~al.}(1993)\citenamefont {Colomo},
  \citenamefont {Izergin}, \citenamefont {Korepin},\ and\ \citenamefont
  {Tognetti}}]{1993ColomoXX0basis}%
  \BibitemOpen
\bibfield  {journal} {  }\bibfield  {author} {\bibinfo {author} {\bibfnamefont
  {F.}~\bibnamefont {Colomo}}, \bibinfo {author} {\bibfnamefont {A.~G.}\
  \bibnamefont {Izergin}}, \bibinfo {author} {\bibfnamefont {V.~E.}\
  \bibnamefont {Korepin}},\ and\ \bibinfo {author} {\bibfnamefont
  {V.}~\bibnamefont {Tognetti}},\ }\href {https://doi.org/10.1007/BF01016992}
  {\bibfield  {journal} {\bibinfo  {journal} {Theor. Math. Phys.}\ }\textbf
  {\bibinfo {volume} {94}},\ \bibinfo {pages} {11} (\bibinfo {year}
  {1993})}\BibitemShut {NoStop}%
\bibitem [{els()}]{elsewhere}%
  \BibitemOpen
  \href@noop {} {\bibinfo  {journal} {The properties (\ref{eq:Sx}),
  (\ref{eq:Sy}), apart from self-similarity of $S_N$ for different $Q$, follow
  from the translational symmetry and the mirror symmetry of the $XX$
  Hamiltonian, while (\ref{eq:Sz}) is a consequence of the $U(1)$ symmetry. On
  the other hand, the self-similar scaling $S_N(Q,t) = S_N(t \cos Q)$ is more
  intricate, see elsewhere for details.}\ }\BibitemShut {NoStop}%
\bibitem [{\citenamefont {Ares}\ \emph {et~al.}()\citenamefont {Ares},
  \citenamefont {Murciano},\ and\ \citenamefont
  {Calabrese}}]{2022AresCalabrese}%
  \BibitemOpen
\bibfield  {journal} {  }\bibfield  {author} {\bibinfo {author} {\bibfnamefont
  {F.}~\bibnamefont {Ares}}, \bibinfo {author} {\bibfnamefont {S.}~\bibnamefont
  {Murciano}},\ and\ \bibinfo {author} {\bibfnamefont {P.}~\bibnamefont
  {Calabrese}},\ }\href {https://arxiv.org/abs/2207.14693} {\bibinfo  {journal}
  {arXiv:2207.14693}\ }\BibitemShut {NoStop}%
\bibitem [{\citenamefont {Scholl}\ \emph {et~al.}(2022)\citenamefont {Scholl},
  \citenamefont {Williams}, \citenamefont {Bornet}, \citenamefont {Wallner},
  \citenamefont {Barredo}, \citenamefont {Henriet}, \citenamefont {Signoles},
  \citenamefont {Hainaut}, \citenamefont {Franz}, \citenamefont {Geier} \emph
  {et~al.}}]{RingShapedArrays}%
  \BibitemOpen
\bibfield  {journal} {  }\bibfield  {author} {\bibinfo {author} {\bibfnamefont
  {P.}~\bibnamefont {Scholl}}, \bibinfo {author} {\bibfnamefont {H.~J.}\
  \bibnamefont {Williams}}, \bibinfo {author} {\bibfnamefont {G.}~\bibnamefont
  {Bornet}}, \bibinfo {author} {\bibfnamefont {F.}~\bibnamefont {Wallner}},
  \bibinfo {author} {\bibfnamefont {D.}~\bibnamefont {Barredo}}, \bibinfo
  {author} {\bibfnamefont {L.}~\bibnamefont {Henriet}}, \bibinfo {author}
  {\bibfnamefont {A.}~\bibnamefont {Signoles}}, \bibinfo {author}
  {\bibfnamefont {C.}~\bibnamefont {Hainaut}}, \bibinfo {author} {\bibfnamefont
  {T.}~\bibnamefont {Franz}}, \bibinfo {author} {\bibfnamefont
  {S.}~\bibnamefont {Geier}}, \emph {et~al.},\ }\href
  {https://link.aps.org/doi/10.1103/PRXQuantum.3.020303} {\bibfield  {journal}
  {\bibinfo  {journal} {PRX Quantum}\ }\textbf {\bibinfo {volume} {3}},\
  \bibinfo {pages} {020303} (\bibinfo {year} {2022})}\BibitemShut {NoStop}%
\end{thebibliography}%


\appendix

\section{Proof of Theorem I}\label{Appx;1}
Recall the vectors in Theorem I
\begin{align}
&\ket{u;n_1,\ldots,n_k},\quad\ket{u+1;n_1,\ldots,n_k}, \quad 1\leq n_1<n_2\cdots<n_k\leq N,\label{BasisVector}\\
&k=0,2,\ldots,N,\qquad\mbox{if}\,\,N=4m,\quad m\in\mathbb{N^+},\no\\
&k=1,3,\ldots,N-1,\qquad\mbox{if}\,\,N=4m+2,\quad m\in\mathbb{N}.\no
\end{align}
For real valued $u$ in $\psi(u)$, the following identities hold
\begin{align}
\psi_n^\dagger(u)\psi_n(u+1)=0,\quad \psi_n(u+2l)=\psi_n(u).
\end{align}
Hence, for a system with an even number of sites, one can prove that the vectors in
(\ref{BasisVector}) are mutually orthogonal
\begin{align}
\braket{u;n_1,\ldots,n_j}{u';m_1,\ldots,m_k}=\delta_{u,u'}\delta_{j,k}\prod_{r=1}^k\delta_{n_r,m_r},\quad u'=u,u+1,\quad j-k=2l,\quad l\in\mathbb{Z}.\label{Orthogonality}
\end{align}
With the help of the identities $$2\sum_{m=0}^{2k} \binom{4k}{2m}= 2^{4k},\qquad 2\sum_{m=0}^{2k}\binom{4k+2}{2m+1}=2^{4k+2},$$
the completeness of our basis (\ref{BasisVector}) is then proved.


\section{Proof of Theorem II}\label{Appx;2}

Let us first introduce two divergence conditions
\begin{align}
&h_{n,n+1}\psi_n(u)\psi_{n+1}(u)=\eE^{i\frac{\pi}{2}}\psi_n(u)\psi_{n+1} (u+1)+\eE^{-i\frac{\pi}{2}}\psi_n(u-1)\psi_{n+1}(u),\label{DivCondition;1}\\
&h_{n,n+1}\psi_n(u)\psi_{n+1}(u+1)=\eE^{i\frac{\pi}{2}}\psi_n (u+1)\psi_{n+1}(u+1)+\eE^{-i\frac{\pi}{2}}\psi_n(u)\psi_{n+1} (u).\label{DivCondition;2}
\end{align}
Since $\vec\sigma_{N+1}\equiv\vec\sigma_{1}$, we have 
\begin{align}
\psi_{N+1}(u)&\equiv \psi_{1}(u\pm 2l),\quad \mbox{if}\,\,N=4m,\quad m\in\mathbb{N^+},\quad l\in\mathbb{Z},\no\\
\psi_{N+1}(u)&\equiv \psi_{1}(u\pm 2l-1),\quad\mbox{if}\,\, N=4m+2,\quad m\in\mathbb{N},\quad l\in\mathbb{Z}.\label{eq;1}
\end{align}
For even system length $N$, by using Eqs. (\ref{DivCondition;1})-(\ref{eq;1}) and (\ref{eq:periodicity}) repeatedly, one can prove that 
\begin{align}
&H\ket{u;n_1,\ldots,n_{k}}=2\sum_{\sigma=\pm 1}\sum_{l=1}^{k}\ket{u;n_1,\ldots,n_{l-1},n_l+\sigma,n_{l+1},\ldots,n_{k}}.\label{eq;2}
\end{align}
For convenience, in Eq.~(\ref{eq;2}), we denote that 
\begin{align}
&\ket{u;\ldots,n_{j},n_{j+1}=n_j,\ldots}=0,\no\\
&\ket{u;n_1,\ldots,n_{k-1},N+1}=(-i)^{N}\ket{u+1;1,n_1,\ldots,n_{k-1}},\no\\
&\ket{u;0,n_2\ldots,n_k}=i^{N}\ket{u+1;n_1,\ldots,n_{k},N}.
\end{align}
Obviously, the states in (\ref{BasisVector}) with a certain $k$ form a closed system. Next, we show how to construct the eigenstate in terms of our chiral basis.

\subsection{$M=1$ case}

When $N=4m+2,\, m\in\mathbb{N}$,
we obtain the following relations
\begin{align}
&H\ket{u;n}=2\ket{u;n-1}+2\ket{u;n+1},\quad 2 \leq n\leq N-1,\\
&H\ket{u;1}=-2\ket{u+1;N}+2\ket{u;2},\\%\ket{x;0}=\eE^{-2\eta}\ket{x+1,N}%
&H\ket{u;N}=-2\ket{u-1;1}+2\ket{u;N-1}.
\end{align}
The $2N$ eigenstates of the Hamiltonian can be expanded as follows  
\begin{align}
&&\ket{\mu_1(p)}=\sum_{m=0,1}\sum_{n=1}^Nf_{u+m;n}\ket{u+m;n}\quad 
\mbox{with}\quad H\ket{\mu_1(p)}=4\cos p\ket{\mu_1(p)}.
\end{align}
Then the eigen equation gives the following relations
\begin{align}
\begin{aligned}\label{FR;r1}
&2\cos p\,f_{u,n}=f_{u,n+1}+f_{u,n-1},\,\, n=2,\dots,N-1,\\
&2\cos p\,f_{u,1}=f_{u,2}-f_{u+1,N},\\
&2\cos p\,f_{u,N}=f_{u,N-1}-f_{u-1,1}.
\end{aligned}
\end{align}
Introduce the following ansatz 
\begin{align}
f_{u,n}=\eE^{i np-i\pi\alpha u}.
\end{align}
To satisfy the above functional relations (\ref{FR;r1}), we need 
\begin{align}
-f_{u+1,N+n}=f_{u,n},\quad f_{u+2,n}=f_{u,n}.
\end{align}
which gives the twisted Bethe ansatz equations
\begin{align}
&\eE^{i Np}=-\eE^{i\al\pi},\qquad 
\eE^{2i\alpha\pi}=1.\label{BAE;r1}
\end{align}
As we have $\alpha=0,1$, the exact solutions of the BAE (\ref{BAE;r1}) are thus 
\begin{align}
&\alpha=1,\quad p=\frac{2m\pi}{N},\quad m=1,\ldots,N,\\
&\alpha=0,\quad p=\frac{(2m-1)\pi}{N},\quad m=1,\ldots,N.
\end{align}

\subsection{$M=2$ case}
When $N=4m,\,m\in\mathbb{N^+}$, we get 
\begin{align}
&H\ket{u;n_1,n_2}=2(1-\delta_{n_1+1,n_2})\ket{u;n_1+1,n_2}+2(1-\delta_{n_1+1,n_2})\ket{u;n_1,n_2-1}\no\\
&\hspace{2.2cm}+2\ket{u;n_1-1,n_2}+2\ket{u;n_1,n_2+1},\quad 2\leq n_1<n_2\leq N-1,\\
%%%%%%%%%%%%%%%%%%%%%%%%%%%%%%%%%%%%%%%%%%%%%%%%%%%%%
&H\ket{u;1,n_2}=2(1-\delta_{2,n_2})\ket{u;2,n_2}+2(1-\delta_{2,n_2})\ket{u;1,n_2-1}\no\\
&\hspace{2.2cm}+2\ket{u+1;n_2,N}+2\ket{u;1,n_2+1},\qquad 2\leq n_2\leq N-1,\\
%%%%%%%%%%%%%%%%%%%%%%%%%%%%%%%%%%%%%%%%%%%%%%%%%%%%
&H\ket{u;n_1,N}=2\ket{u-1;1,n_1}+2\ket{u;n_1-1,N}\no\\
&\hspace{2.2cm}+2(1-\delta_{n_1+1,N})\ket{u;n_1,N-1}+2(1-\delta_{n_1+1,N})\ket{u;n_1+1,N},\quad 2\leq n_1\leq N-1,\\
%%%%%%%%%%%%%%%%%%%%%%%%%%%%%%%%%%%%%%%%%%%%%%%%%%%%
&H\ket{u;1,N}=2\ket{u;2,N}+2\ket{u;1,N-1}.
\end{align}
The $2\binom{N}{2}$ eigenstates of $H$ can be expanded as
\begin{align*}
&\ket{\mu_2(p_1,p_2)}=\sum_{m=0,1}\sum_{n_1<n_2}f_{u+m,n_1,n_2}\ket{u+m;n_1,n_2},\\
&H\ket{\mu_2(p_1,p_2)}=4\sum_{j=1}^2\cos(p_j)\ket{\mu_2(p_1,p_2)}.
\end{align*}
The eigen equation of $H$ gives the following functional relations 
\begin{align}
&[2\cos(p_1)+2\cos(p_2)]f_{u,n_1,n_2}\no\\
&=f_{u,n_1-1,n_2}+f_{u,n_1,n_2+1}+(1-\delta_{n_1+1,n_2})f_{u,n_1+1,n_2}+(1-\delta_{n_1+1,n_2})f_{u,n_1,n_2-1},\quad 1< n_1<n_2< N, \label{FR;r2;1}\\
%%%%%%%%%%%%%%%%%%%%%%%%%%%%%%%%%%%%%%%%%%%%%%%%%%%%%
&[2\cos(p_1)+2\cos(p_2)]f_{u,1,n_2}\no\\
&=f_{u,1,n_2+1}+f_{u+1,n_2,N}+(1-\delta_{2,n_2})f_{u,2,n_2}+(1-\delta_{2,n_2})f_{u,1,n_2-1},\quad 1< n_2<N,\label{FR;r2;2}\\
%%%%%%%%%%%%%%%%%%%%%%%%%%%%%%%%%%%%%%%%%%%%%%%%%%%%
&[2\cos(p_1)+2\cos(p_2)]f_{u,n_1,N}\no\\
&=f_{u,n_1,N-1}+f_{u-1,1,n_1}+(1-\delta_{n_1+1,N})f_{u,n_1+1,N}+(1-\delta_{n_1+1,N})f_{u,n_1,N-1},\quad 1<n_1<N,\label{FR;r2;3}\\
%%%%%%%%%%%%%%%%%%%%%%%%%%%%%%%%%%%%%%%%%%%%%%%%%
&[2\cos(p_1)+2\cos(p_2)]f_{u,1,N}=f_{u,1,N-1}+f_{u,2,N}.\label{FR;r2;4}
\end{align}
Employ the ansatz 
\begin{align}
f_{u,n_1,n_2}=\eE^{-i\al u\pi}\sum_{a,b}A_{a,b}\,\eE^{i (n_ap_a+ n_bp_b)}.
\end{align}
where $\{a,b\}=\{1,2\}$.
To satisfy Eq.~(\ref{FR;r2;1}), we get the ``two-body scattering matrix"
\begin{align}
\frac{A_{2,1}}{A_{1,2}}\equiv -1.\label{S;r2}
\end{align}
The compatibility of Eqs. (\ref{FR;r2;2}), (\ref{FR;r2;3}) requires 
\begin{align}
f_{u,m,n}=f_{u+1,n,m+N},\quad f_{u,m,n}=f_{u+2,m,n},
\end{align}
which leads to the Bethe ansatz equations
\begin{align}
\begin{aligned}\label{BAE;r2}
&\eE^{i Np_j}= -\eE^{i\alpha\pi},\quad j=1,2,\\
&\eE^{2i\alpha\pi}=1.
\end{aligned}
\end{align}
Solutions of BAE (\ref{BAE;r2}) are 
\begin{align}
&\alpha=1,\quad p_1=\frac{2j_1\pi}{N},\quad p_2=\frac{2j_2\pi}{N}\quad 1\leq j_1<j_2\leq N,\\
&\alpha=0,\quad p_1=\frac{(2j_1-1)\pi}{N},\quad p_2=\frac{(2j_2-1)\pi}{N} \quad 1\leq j_1<j_2\leq N.
\end{align}


\subsection{Arbitrary $M$ case}

The eigenstates $\ket{\mu_M(\pp)}$ can be expanded as
\begin{align}
\ket{\mu_M(\pp)}=\sum_{m=0,1}\sum_{\substack{n_1<n_2<\ldots\\\cdots<n_M}}f_{u+m,n_1,\ldots,n_M}\ket{u+m;n_1,\ldots,n_M}.\label{BA;M}
\end{align}
Employ the ansatz 
\begin{align}
f_{u,n_1,\ldots,n_M}=\eE^{-i\al \pi
  u}\sum_{r_1,\ldots,r_M}A_{r_1,\ldots,r_M}\eE^{i\sum_{k=1}^M n_{r_{k_1}}p_{r_{k_1}}}.\label{Amplitude;M}
\end{align}
where $\{r_1,\ldots,r_M\}=\{1,2,\ldots,M\}$.
The amplitudes $A_{\ldots}$ satisfy 
\begin{align}
\frac{A_{\ldots,r_k,r_{k+1},\ldots}}{A_{\ldots,r_{k+1},r_k,\ldots}}= -1.\label{S;rM}
\end{align}
The Bethe ansatz equations read
\begin{align}
\begin{aligned}\label{BAE;rM}
	&\eE^{i Np_j}=-\eE^{i\alpha\pi},\quad j=1,\ldots,M,\\
	&\eE^{2\alpha\pi}=1.
\end{aligned}
\end{align}
The solutions of BAE (\ref{BAE;rM}) are
\begin{align}
	&\alpha=1,\quad p_k=\frac{2j_k\pi}{N},\quad k=1,\ldots,M,\quad 1\leq j_1<j_2<\ldots<j_M\leq N,\\
	&\alpha=0,\quad p_k=\frac{(2j_k-1)\pi}{N},\quad k=1,\ldots,M, \quad 1\leq j_1<j_2<\ldots<j_M\leq N.
\end{align}
The energy in terms of the Bethe roots $\pp=\{p_1,\ldots,p_M\}$ is 
\begin{align}
E_{\pp}(M)=4\sum_{j=1}^M\cos(p_j).
\end{align}

\textit{Remark}: The eigenstates constructed in (\ref{BA;M}), (\ref{Amplitude;M}) and (\ref{S;rM}) are consistent with the ones in the main text. By normalizing $\ket{\mu_M(\pp)}$ in (\ref{BA;M}) we arrive at Eq.~(\ref{eq:mu}).

\subsection{Proof of normalization of $\ket{\mu_M(\pp)}$ in (\ref{eq:mu})}

For real $u$ the wavevectors $\ket{\mu_M(\pp)}$ are orthogonal, $\braket  {\mu_M(\pp)}  {\mu_{M'}(\pp')} =\de_{\bfvec{p},\bfvec{p'}}  \de_{M,M'}$.
Indeed, for $\bfvec{p}=\bfvec{p'}, M=M'$ we have
\begin{align}
&\braket  {\mu_M(\pp)}  {\mu_{M}(\pp)}=\no\\
&= \frac{1}{2N^{M}(M!)^2}\ \sum_{\mathbf{n},\bfvec{m}} T_{\bfvec{n}}\, T_{\bfvec{m}}  \sum_{Q,Q'} (-1)^{Q+Q'} \eE^{-i \sum_{j=1}^M n_j  p_{Q_j} } \eE^{i \sum_{j=1}^M m_j  p_{Q'_j} } \nonumber\\
&\quad \times \left(
\bra{x;n_1,n_2,\ldots,n_M}-\eE^{-ip_1 N}   \bra{x+1;n_1,n_2,\ldots,n_M}\right)
\left(\ket{x;m_1,m_2,\ldots,m_M}-\eE^{ip_1 N}   \ket{x+1;m_1,m_2,\ldots,m_M}\right)\no\\
&=\frac{1}{N^MM!}\sum_{n_1,n_2,\ldots} T_{\bfvec{n}}\, T_{\bfvec{n}} \,\sum_{Q,Q'} (-1)^{Q+Q'}
\eE^{-i \sum_{j=1}^M n_j  p_{Q_j} } \eE^{i \sum_{j=1}^M n_j  p_{Q'_j}}.
\end{align}
In the last passage, we used the orthogonality of the vectors $\ket{u;m_1,m_2,
  \ldots, m_M}$, $\ket{u+1;m_1,m_2,\ldots ,m_M}$, and assume \textit{real}
$u$, so that $\braket{u;m_1,m_2,\ldots,m_M} {u;m_1,m_2,\ldots, m_M}=1$.  In
the last sum, only $Q'=Q$ permutations survive, there are $M!$ such permutations,
so we continue:
\begin{align}
\braket  {\mu_M(\pp)}  {\mu_{M}(\pp)}=
\frac{1}{N^M}\sum_{n_,n_2,\ldots,n_M} 1  = 1.
\end{align}
%leading to $ \braket{\mu_{\mathbf{p}}(M)}{\mu_{\mathbf{p}}(M)}=1$.



\section{Proof of Eq.~(\ref{res:DetF})}\label{Appx;3}


We aim at the calculation of $\bra{\mu_M(\pp)}\si_1^x \ket{\mu_M(\qq)}$ for
arbitrary $\pp,\qq$.  Firstly, we observe a remarkable fact: choosing
$u_0=\frac12$, all basis states (\ref{eq:Mshock}) become eigenstates
of $\si_1^x$, with eigenvalues $+1$ or $-1$,
\begin{align}
& \si_1^x \ket{u_0; n_1,n_2, \ldots} =  \ket{u_0; n_1,n_2, \ldots},  \label{eq:sxEigen}\\   
& \si_1^x \ket{u_0+1; n_1,n_2, \ldots} = -\ket{u_0+1; n_1,n_2, \ldots}.
\end{align}
Consequently, the matrix elements of $\si_1^x$ between blocks with different
$M$ are then automatically zero, $\bra{\mu_M(\pp')} \si_1^x\ket{\mu_{M'}(\pp)}=0$ if $M \neq M'$.


For $M=1$  the eigenfunctions in the block with $M=1$ are $\ket{\mu_1(p)} = \frac{1}{\sqrt{2N}} \sum_{n=1}^N \eE^{i p n } (\ket{u_0;n}- \eE^{i p N} \ket{u_0+1;n})$ and  we get
\begin{align}
&s_{p',p}=\bra{\mu_1(p')} \si_1^x \ket{\mu_1(p)} = \frac{1}{2N} \sum_{n=1}^N \eE^{i (p-p')n} (1 - \eE^{i (p-p')N} ).
\end{align}
The factor $(1 - \eE^{i (p-p')N})=0$ if $p$ and $p'$ have the same parity, $\eE^{i (p-p')N}=1$. If the parities 
of $p,p'$ are different, i.e. $\eE^{i (p-p') N} =-1$, we obtain
\begin{align}
&s_{p',p}=\frac{1}{N} \sum_{n=1}^N \eE^{i (p-p')n} =\frac{2}{N}\frac{1}{\eE^{i (p'-p)}-1}  \label{eq:spp}
\end{align}

For arbitrary $M$ after some algebra we obtain 
\begin{align}
&\bra{\mu_M(\pp)} \si_1^x \ket{\mu_M(\qq)} =0, \quad  \mbox{if} \ \     \eE^{i (p_1-q_1)N}=1,\\
&\bra{\mu_M(\pp)}  \si_1^x \ket{\mu_M(\qq)}  =\sum_{Q} (-1)^Q \prod_{j=1}^M  s_{p_j, q_{Q_j}}=
M_0^{-M}\det[F(\bfvec{p},\bfvec{q})], \quad \mbox{if} \ \ \eE^{i (p_1 - q_1)N}=-1,\\
&F_{nm}(\bfvec{p},\bfvec{q})= \frac{1}{\eE^{i (p_n-q_m)}-1}.
\end{align}
For $M=M_0$ we obtain (\ref{res:DetF}).



\section{Proof of Eq.~(\ref{res:DetOverlapG}) }\label{Appx;4}
We are interested in the overlap of $\ket{\Omega}$ with the eigenstates of the
free fermion Hamiltonian $H$ in the chiral basis for arbitrary even system
size $N\geq 2$. It can be proved straighforwardly that choosing the shift
parameter $u_0 = \frac12$,
%\label{eq:x0}
all nonzero overlaps belong to the block where the eigenstate
$\ket{\mu_{M}(\pp)}$ is composed of basis states with $M_0$ kinks.  In this
block, we have
\begin{align}
\braket{\Omega}{\mu_{M_0}(\pp)}&=\frac{1}{\sqrt{2 N^{M_0}}} \bra{\Omega} \sum_{Q}\sum_{n_j<n_{j+1}} (-1)^Q 
\eE^{i \sum_{k} p_{Q_k} n_k} 
\left(
\ket{u_0;n_1,n_2,\ldots n_{M_0}} - \eE^{i p_1 N} \ket{u_0+1;n_1,n_2,\ldots n_{M_0}}
 \right)\no\\%\label{G-1}\\
&=\frac{1}{\sqrt{2 N^{M_0}}}\sum_{Q} \sum_{n_j<n_{j+1}}(-1)^Q 
\eE^{i \sum_{k} p_{Q_k} n_k} 
\braket {\Omega} {u_0;n_1,n_2,\ldots n_{M_0}}, \label{eq:overlapSHS-MuP}
\end{align}
%where $A_M$ is a normalization factor\begin{align}&A_M= \sqrt{2^{N+1} N^{M_0}},\label{res:AM}\end{align}
where in the passage from the first to the second line we used
$\braket{\Omega}{u_0+1;n_1,n_2,\ldots n_{M_0}}=0$ for all values of $n_j$.
Next, we can establish that for the allowed values of $n_j$, i.e.  $1\leq
n_1<n_2< \ldots n_{M_0-1}<n_{M_0}\leq N$ and even $N$ we have
\begin{align}
&\braket {\Omega} {u_0;n_1,n_2,\ldots n_{M_0}}= \frac{K_N}{\sqrt{2^N}}
(\de_{n_1,1} + \de_{n_1,2}) (\de_{n_2,3} + \de_{n_2,4}) \cdots(\de_{n_{M_0},N-1} + \de_{n_{M_0},N}),
\label{eq:overlapSHS-basis}
\end{align}
where $K_N$ depends on the system size only. The calculation of $K_N=\sqrt{2^{N}}
\braket{\Omega} {u_0;1,3,\ldots,N-1}$ gives
\begin{align}
K_N=(1-i)^{\frac{N}{2}} (-i)^{\frac{N^2}{4}} = \left\{
\begin{array}{ccc}
&(-2 i)^{\frac{N}{4}}, &\quad \frac{N}{2} \mbox {even},\\[2pt]
&- (1+i)\,(-2 i)^{\frac{N-2}{4}},&  \quad \frac{N}{2} \ \mbox {odd}.\label{res:OM}
\end{array}
\right.
\end{align}
%In particular, for \textit{even} $N/2$ the above expression simplifies to  $K_M =(1-i)^{N/2} =(-2 i )^{N/4} $.
Substituting (\ref{eq:overlapSHS-basis}) into 
(\ref{eq:overlapSHS-MuP}) we obtain
\begin{align}
&\braket{\Omega}{\mu_{M_0}(\pp)} =\frac{K_N}{\sqrt{2^{N+1} N^{M_0}}}\sum_{Q} (-1)^Q 
\eE^{i \sum_{k=1}^{M_0} p_{Q_k} n_k} \prod_{j=1}^{M_0} (\de_{n_j,2j} + \de_{n_j,2j-1})\no\\
&\hspace{1.75cm}=\frac{K_N}{\sqrt{2^{N+1} N^{M_0}}}\sum_{Q} (-1)^Q g_1(p_{Q_1})  g_2(p_{Q_2}) \ldots  g_M(p_{Q_{M_0}})\no\\
&\hspace{1.75cm}=\frac{K_N}{\sqrt{2^{N+1} N^{M_0}}}\det[G(\pp)],\\
&G_{km}(\pp) = g_k(p_m)  = \sum_{n_k=2k-1}^{2k} \eE^{i  p_m n_k}= 
\eE^{2i k p_m} \left( 1+ \eE^{-i  p_m}  \right),
\label{res:overlapSHS-MuP}\\
&k,m = 1,2 \ldots M_0, \nonumber
\end{align}
i.e. Eq.(\ref{res:DetOverlapG}).

%\input{AppTheta}
\end{document}




