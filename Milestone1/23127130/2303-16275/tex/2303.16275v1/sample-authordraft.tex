%%
%% This is file `sample-manuscript.tex',
%% generated with the docstrip utility.
%%
%% The original source files were:
%%
%% samples.dtx  (with options: `manuscript')
%% 
%% IMPORTANT NOTICE:
%% 
%% For the copyright see the source file.
%% 
%% Any modified versions of this file must be renamed
%% with new filenames distinct from sample-manuscript.tex.
%% 
%% For distribution of the original source see the terms
%% for copying and modification in the file samples.dtx.
%% 
%% This generated file may be distributed as long as the
%% original source files, as listed above, are part of the
%% same distribution. (The sources need not necessarily be
%% in the same archive or directory.)
%%
%% Commands for TeXCount
%TC:macro \cite [option:text,text]
%TC:macro \citep [option:text,text]
%TC:macro \citet [option:text,text]
%TC:envir table 0 1
%TC:envir table* 0 1
%TC:envir tabular [ignore] word
%TC:envir displaymath 0 word
%TC:envir math 0 word
%TC:envir comment 0 0
%%
%%
%% The first command in your LaTeX source must be the \documentclass command.
%%%% Small single column format, used for CIE, CSUR, DTRAP, JACM, JDIQ, JEA, JERIC, JETC, PACMCGIT, TAAS, TACCESS, TACO, TALG, TALLIP (formerly TALIP), TCPS, TDSCI, TEAC, TECS, TELO, THRI, TIIS, TIOT, TISSEC, TIST, TKDD, TMIS, TOCE, TOCHI, TOCL, TOCS, TOCT, TODAES, TODS, TOIS, TOIT, TOMACS, TOMM (formerly TOMCCAP), TOMPECS, TOMS, TOPC, TOPLAS, TOPS, TOS, TOSEM, TOSN, TQC, TRETS, TSAS, TSC, TSLP, TWEB.
% \documentclass[acmsmall]{acmart}

%%%% Large single column format, used for IMWUT, JOCCH, PACMPL, POMACS, TAP, PACMHCI
% \documentclass[acmlarge,screen]{acmart}

%%%% Large double column format, used for TOG
\documentclass[sigconf]{acmart}
% \documentclass[acmtog, authorversion]{acmart}

%%%% Generic manuscript mode, required for submission
%%%% and peer review
% \documentclass[manuscript,screen,review]{acmart}
%% Fonts used in the template cannot be substituted; margin 
%% adjustments are not allowed.
%%
%% \BibTeX command to typeset BibTeX logo in the docs
\AtBeginDocument{%
  \providecommand\BibTeX{{%
    \normalfont B\kern-0.5em{\scshape i\kern-0.25em b}\kern-0.8em\TeX}}}

%% Rights management information.  This information is sent to you
%% when you complete the rights form.  These commands have SAMPLE
%% values in them; it is your responsibility as an author to replace
%% the commands and values with those provided to you when you
%% complete the rights form.
% \setcopyright{acmcopyright}
\copyrightyear{2023}
\acmYear{2023}
\acmDOI{XXXXXXX.XXXXXXX}

%% These commands are for a PROCEEDINGS abstract or paper.
\acmConference[The Second In2Writing Workshop @ CHI '23]{April 23, 2023}{Hamburg, Germany}
%
%  Uncomment \acmBooktitle if th title of the proceedings is different
%  from ``Proceedings of ...''!
%
% \acmBooktitle{} 
% \acmPrice{15.00}
% \acmISBN{978-1-4503-XXXX-X/18/06}


\usepackage{todonotes}

\settopmatter{printacmref=false}
\setcopyright{none}
\renewcommand\footnotetextcopyrightpermission[1]{}
\pagestyle{plain}


%%
%% Submission ID.
%% Use this when submitting an article to a sponsored event. You'll
%% receive a unique submission ID from the organizers
%% of the event, and this ID should be used as the parameter to this command.
%%\acmSubmissionID{123-A56-BU3}

%%
%% For managing citations, it is recommended to use bibliography
%% files in BibTeX format.
%%
%% You can then either use BibTeX with the ACM-Reference-Format style,
%% or BibLaTeX with the acmnumeric or acmauthoryear sytles, that include
%% support for advanced citation of software artefact from the
%% biblatex-software package, also separately available on CTAN.
%%
%% Look at the sample-*-biblatex.tex files for templates showcasing
%% the biblatex styles.
%%

%%
%% The majority of ACM publications use numbered citations and
%% references.  The command \citestyle{authoryear} switches to the
%% "author year" style.
%%
%% If you are preparing content for an event
%% sponsored by ACM SIGGRAPH, you must use the "author year" style of
%% citations and references.
%% Uncommenting
%% the next command will enable that style.
%%\citestyle{acmauthoryear}

%%
%% end of the preamble, start of the body of the document source.
\begin{document}

%%
%% The "title" command has an optional parameter,
%% allowing the author to define a "short title" to be used in page headers.
\title{Writing Assistants Should Model Social Factors of Language}

%%
%% The "author" command and its associated commands are used to define
%% the authors and their affiliations.
%% Of note is the shared affiliation of the first two authors, and the
%% "authornote" and "authornotemark" commands
%% used to denote shared contribution to the research.

\author{Vivek Kulkarni}
\affiliation{
  \institution{Grammarly}
  \city{San Francisco}
  \state{CA}
  \country{USA}
}
\email{vivek.kulkarni@grammarly.com}

\author{Vipul Raheja}
\affiliation{
  \institution{Grammarly}
  \city{San Francisco}
  \state{CA}
  \country{USA}
}
\email{vipul.raheja@grammarly.com}

%%
%% By default, the full list of authors will be used in the page
%% headers. Often, this list is too long, and will overlap
%% other information printed in the page headers. This command allows
%% the author to define a more concise list
%% of authors' names for this purpose.
\renewcommand{\shortauthors}{Kulkarni \& Raheja}

%%
%% The abstract is a short summary of the work to be presented in the
%% article.
\begin{abstract}
Intelligent writing assistants powered by large language models (LLMs) are more popular today than ever before, but their further
widespread adoption is precluded by sub-optimal performance. In this position paper, we argue that a major reason for this sub-optimal
performance and adoption is a singular focus on the information content of language while ignoring its social aspects. We analyze the
different dimensions of these social factors in the context of writing assistants and propose their incorporation into building smarter,
more effective, and truly personalized writing assistants that would enrich the user experience and contribute to increased user adoption.
\end{abstract}

%%
%% The code below is generated by the tool at http://dl.acm.org/ccs.cfm.
%% Please copy and paste the code instead of the example below.
%%
\begin{CCSXML}
<ccs2012>
<concept>
<concept_id>10010147.10010178.10010179.10010182</concept_id>
<concept_desc>Computing methodologies~Natural language generation</concept_desc>
<concept_significance>500</concept_significance>
</concept>
</ccs2012>
\end{CCSXML}

\ccsdesc[500]{Computing methodologies~Natural language generation}

%%
%% Keywords. The author(s) should pick words that accurately describe
%% the work being presented. Separate the keywords with commas.
\keywords{writing assistants, large language models, social factors}

% \received{23 February 2023}
% \received[accepted]{6 March 2023}

%%
%% This command processes the author and affiliation and title
%% information and builds the first part of the formatted document.
\maketitle

\section{Introduction}
Advancements in large language models (LLMs) have accelerated their use in many writing assistants \cite{10.1145/3491102.3502030, schick2023peer, du-etal-2022-read, in2writing-2022-intelligent, kim-etal-2022-improving, 10.1145/3490099.3511105}. Millions of people now use AI-driven writing assistants to correct grammar, seek recommendations on word choice, set the right tone, and improve the conciseness and clarity of written content.
However, despite such popularity, it is evident that their sub-optimal performance inhibits further widespread adoption. We argue that an important reason for this sub-optimal performance is due to a limiting modeling assumption -- namely, viewing language as a sequence of tokens with information content. However, as noted in socio-linguistics research and reinforced recently by \citet{hovy-yang-2021-importance}, language is also a social construct used to achieve communicative goals, is grounded in the real world, and is influenced by social aspects. Because underlying social factors heavily influence our understanding of language, they argue that NLP applications should account for social aspects of language to unlock their full potential. They propose a taxonomy of social factors to help researchers reason about these aspects in specific applications. Here, we leverage their proposed taxonomy to comprehensively reason about the various social factors that would specifically benefit intelligent writing assistants, which we outline next.

\section{Applicability of Social Factors of Language in Writing Assistants}
\begin{enumerate}
    \item \textsc{\textbf{Demographics}} \textbf{(Speaker and Receiver Context)}: 
    Prior research has noted age \cite{eckert2017age, tagliamonte2011variationist,barbieri2008patterns, johannsen2015cross}, gender \cite{Holmes1997WomenLA, rickford2013girlz}, and race \cite{blodgett-etal-2016-demographic, blodgett-etal-2018-twitter} influence language use. 
    For example, \citet{tagliamonte2011variationist} observes that word choice is influenced by the age of interlocutors and noted that older people prefer to use ``ha-ha'' over ``lol'' on an instant messenger platform. 
    Further, \citet{van2021sentence} show that there is a gradual deterioration of the ability to interpret long and complex sentences as people age. 
    Similarly, \citet{Johannsen2015CrosslingualSV} report several age and gender-specific variations in word choice and syntactic dependency structures. 
    Finally, \citet{green_2002, jones2015toward} note that African American Vernacular English (AAVE) is a socio-linguistic variety of Standard American English, with distinct syntactic, semantic, and lexical patterns. 
    By modeling these demographic factors, writing assistants can thus improve their recommendations on word choice and sentence phrasing, while seeking to ensure that no biases or stereotypes are perpetuated. \\
    \item \textsc{\textbf{Personality}} \textbf{(Speaker and Receiver Context)}: Personality traits are yet another socio-linguistic variable that significantly influences language use. \citet{schwartz2013personality} reveals significant variation in word use based on latent personality factors. 
    For example, extroverts were more likely to mention social words such as ‘party’, etc. 
    Capturing linguistic variation due to personality factors can make writing assistants truly personalized and account for individual preferences while providing word-choice and sentence phrasing recommendations. \\
    
    \item \textsc{\textbf{Social Relations and Norms}} \textbf{(Social Relation)}: The social relationship between %participants of a communication 
    interlocutors is a very important factor that influences language use. Word-choice, tone, sentence structure of communication between two close friends differs significantly from those between colleagues or acquaintances. Many socio-linguistic phenomena might thus manifest based on the social relation. Examples of such socio-linguistic phenomena include the usage of honorifics, slang, code-switching, code-mixing, and avoidance speech. As a use-case, email communications with a close friend might skip all greetings and use slang, while on the other hand, emails to an executive would typically have greetings, appropriate honorifics, and avoid slang. Thus, writing assistants must account for social relations and societal norms. \\
    \item \textsc{\textbf{Time, Geography, and Domain}} 
    Language also demonstrates variation (both syntactic and semantic) across time, geography, domains, and the broader situational context. \cite{Kulkarni2014StatisticallySD, Kulkarni2016FreshmanOF}  Meanings of words can change across all of these dimensions. For example, the word \emph{awesome} had a negative sentiment (inspiring fear) in the 16th century but has taken on its positive sense over time. Similarly, different tokens may be used to refer to the same real-world concept (\textit{zucchini} in the US vs \textit{courgette} in the UK) \cite{Kulkarni2016FreshmanOF}. Writing assistants not accounting for such linguistic variation may lead to poor user experience (e.g., incorrect sentiment or tone detection, or word recommendations). \\
    \item \textsc{\textbf{Intent}} \textbf{(Communicative Goal)}: Writing assistants need to have an intimate knowledge of the communicative intent of the user to be effective. Recommendations on word choice, sentence and paragraph restructuring, and feedback on sentiment and tone depend on the user's specific communicative goal (which might be to inform, entertain, persuade, or narrate) and targeted setting (academic, creative writing,  or conversational). For example, in content targeted for an academic publication, writing assistants might assist users by recommending templates and phrases that seek to achieve specific communicative goals like (a) introducing standard views, quotations, and an ongoing debate, (b) contrasting with prior work, and (c) motivating claims. % All in all, we emphasize the role communicative goals take in language use and outline how writing assistant may benefit by modeling such goals.
\end{enumerate}


\section{Closing Remarks}
In this paper,  we discuss clear use cases of intelligent writing assistants that would benefit by adopting a richer view of language, which accounts for its social aspects. Building writing assistants that adopt this richer view of language opens up exciting research directions. First, a majority of the current evaluation benchmarks used for evaluating writing assistants today ignore these social factors. Therefore, there is a critical need to construct comprehensive evaluation benchmarks grounded in social factors. Second, note that many of these social factors are extra-linguistic and may involve modeling multiple modalities. Research needs to be undertaken around exploring approaches to modeling these social factors in a manner that is best suited toward their incorporation in writing assistants. Finally,  one needs to work within appropriate considerations around data/user privacy and ethics to ensure models benefit end users and not perpetuate negative biases. We thus conclude by urging the community to advance further research on the social aspects of language and how these aspects can relate to building smarter, more effective, highly personalized, and inclusive writing assistants. 

%%
%% The next two lines define the bibliography style to be used, and
%% the bibliography file.
\bibliographystyle{ACM-Reference-Format}
\bibliography{sample-base}

%%
%% If your work has an appendix, this is the place to put it.
\appendix


\end{document}
\endinput
%%
%% End of file `sample-authordraft.tex'.
