\documentclass[10pt,twocolumn,letterpaper]{article}

\usepackage{iccv}
\usepackage{times}
\usepackage{epsfig}
\usepackage{graphicx}
\usepackage{pifont}% http://ctan.org/pkg/pifont
\usepackage{amsmath}
\usepackage{amssymb}
\usepackage{subcaption}

\newcommand{\cmark}{\textcolor{darkgreen}{\ding{51}}}
\newcommand{\xmark}{\textcolor{red}{\ding{55}}}

% Include other packages here, before hyperref.
\usepackage{booktabs}
\usepackage{listings} 
\usepackage[dvipsnames]{xcolor}
\newcommand{\mc}[2]{\multicolumn{#1}{c}{#2}}
\definecolor{Gray}{gray}{0.85}
\definecolor{LightCyan}{rgb}{0.88,1,1}
\renewcommand{\eg}{\textit{e.g.}\xspace}
\renewcommand{\ie}{\textit{i.e.}\xspace}
\renewcommand{\paragraph}[1]{{\vspace{1mm}\noindent \bf #1}.}
\newcommand*\circled[1]{\tikz[baseline=(char.base)]{
            \node[shape=circle,draw,inner sep=2pt] (char) {#1};}}

% If you comment hyperref and then uncomment it, you should delete
% egpaper.aux before re-running latex.  (Or just hit 'q' on the first latex
% run, let it finish, and you should be clear).
\usepackage[pagebackref=true,breaklinks=true,letterpaper=true,colorlinks,bookmarks=false]{hyperref}

\usepackage[capitalize]{cleveref}
\crefname{section}{Sec.}{Secs.}
\Crefname{section}{Section}{Sections}
\Crefname{table}{Table}{Tables}
\crefname{table}{Tab.}{Tabs.}

\newcommand{\todo}[1]{{\textcolor{red}{\textbf{#1}}}}
\newcommand{\colton}[1]{{\textcolor{red}{\textbf{Colton}: #1}}}
\newcommand{\davis}[1]{{\textcolor{blue}{\textbf{Davis}: #1}}}
\newcommand{\leo}[1]{{\textcolor{magenta}{\textbf{Leo}: #1}}}
\newcommand{\jj}[1]{{\textcolor{green}{\textbf{JJ}: #1}}}

\newcommand{\lidar}{LiDAR\xspace}
\newcommand{\kortx}{Kortx\xspace}
\newcommand{\arch}{CurveCloudNet\xspace}

\newcommand{\rotbox}[2]{\rotatebox[origin=l]{#1}{#2}}

%!TEX root = main.tex
% \usepackage{tikz}
% \usetikzlibrary{shapes}
% \newcommand{\mysymbol}[1][]{
% \begin{tikzpicture}[#1]
% \node[draw,ellipse,minimum height=5pt,minimum width=10pt](e){};
% \end{tikzpicture}
% }


\newcommand{\Sn}{\mathbb{S}^n}
\newcommand{\R}{\mathbb{R}}
\newcommand{\cA}{\mathcal{A}}
\newcommand{\cB}{\mathcal{B}}
\newcommand{\cL}{\mathcal{L}}
\renewcommand{\norm}[1]{\Vert #1 \Vert}
\newcommand{\inprod}[2]{\left\langle #1, #2 \right\rangle}
\newcommand{\vectorize}[1]{\mathrm{vec}\parentheses{#1}}
\newcommand{\Fnorm}[1]{\Vert #1 \Vert_{\mathrm{F}}}

\newcommand{\nnReal}[1]{\mathbb{R}_{+}^{#1}}
\newcommand{\vcat}{\ ;\ }
\newcommand{\mymid}{\ \middle\vert\ }
\newcommand{\cbrace}[1]{\left\{#1\right\}}
\newcommand{\sym}[1]{\mathbb{S}^{#1}}
\newcommand{\calAadj}{\calA^{*}}
\newcommand{\barq}{\bar{q}}
\newcommand{\barW}{\bar{W}}
\newcommand{\bary}{\bar{y}}
\newcommand{\barC}{\bar{C}}
\newcommand{\barcalA}{\bar{\calA}}
\newcommand{\barcalAadj}{\bar{\calA}^{*}}
\newcommand{\bmat}{\left[ \begin{array}}
\newcommand{\emat}{\end{array}\right]}
\newcommand{\psd}[1]{\sym{#1}_{+}}
\newcommand{\parentheses}[1]{\left(#1\right)}
\newcommand{\half}{\frac{1}{2}}

\newcommand{\usphere}[1]{\calS^{#1}}
\newcommand{\hpartial}{\hat{\partial}}
\newcommand{\baralpha}{\bar{\alpha}}
\newcommand{\tldW}{\widetilde{\MW}}
\newcommand{\bareta}{\bar{\eta}}
\newcommand{\tWnu}{\tilde{W}_{\nu}}
\newcommand{\tWzero}{\tilde{W}_{0}}
\newcommand{\tWone}{\tilde{W}_{1}}
\newcommand{\tWhalf}{\tilde{W}_{1/2}}
\newcommand{\Qa}{Q_{\alpha}}
\newcommand{\Qb}{Q_{\baralpha}}
\newcommand{\tV}{\tilde{V}}
\newcommand{\tVzero}{\tV_{0}}
\newcommand{\tVhalf}{\tV_{1/2}}
\newcommand{\tVone}{\tV_{1}}
\newcommand{\tVnu}{\tV_{\nu}}
\newcommand{\Mnu}{M_{\nu}}
\newcommand{\tMnu}{\tilde{M}_{\nu}}
\newcommand{\tM}{\tilde{M}}
\newcommand{\hV}{\hat{V}}
\newcommand{\hM}{\hat{M}}
\newcommand{\hMnu}{\hat{M}_{\nu}}
\newcommand{\Omegahalf}{\Omega^{.5}}
\newcommand{\bracket}[1]{\left[#1\right]}
\newcommand{\abs}[1]{\left|#1\right|}
\newcommand{\tldr}{\tilde{r}}
\newcommand{\tldQ}{\tilde{Q}}
\newcommand{\haty}{\hat{y}}

\renewcommand{\prob}{\mathbb{P}}
\newcommand{\probof}[1]{\mathbb{P}\left[#1\right]}
\newcommand{\Feps}{F^{\epsilon}}
\newcommand{\Fepsball}{\Feps_{\mathrm{ball}}}
\newcommand{\Fepsellipse}{\Feps_{\mathrm{ellipse}}}
\newcommand{\tcalY}{\tilde{\calY}}
\newcommand{\floor}[1]{\lfloor #1 \rfloor}
\newcommand{\ceil}[1]{\lceil #1 \rceil}

\newcommand{\lmo}{\scenario{LM-O}}
\newcommand{\heatmapball}{\scenario{heatmap-ball}}
\newcommand{\heatmapellipse}{\scenario{heatmap-ellipse}}
\newcommand{\pvnetball}{\scenario{PVNet-ball}}
\newcommand{\pvnetellipse}{\scenario{PVNet-ellipse}}
\newcommand{\gtball}{\scenario{gt-ball}}
\newcommand{\gtellipse}{\scenario{gt-ellipse}}
\newcommand{\frcnnball}{\scenario{frcnn-ball}}
\newcommand{\frcnnellipse}{\scenario{frcnn-ellipse}}
\newcommand{\purse}{\scenario{PURSE}}
\newcommand{\ransag}{\scenario{RANSAG}}
\newcommand{\pthreep}{\scenario{P3P}}
\newcommand{\phipeak}{\phi_{\mathrm{peak}}}
\newcommand{\phicov}{\phi_{\mathrm{cov}}}
\newcommand{\pnp}{\scenario{PnP}}
\newcommand{\Seps}{S^{\epsilon}}
\newcommand{\ransac}{\scenario{RANSAC}}
\newcommand{\Rgt}{R_{\mathrm{gt}}}
\newcommand{\tgt}{t_{\mathrm{gt}}}
\newcommand{\sgt}{s_{\mathrm{gt}}}

\newcommand{\gray}[1]{{\color{gray}#1}}

\iccvfinalcopy % *** Uncomment this line for the final submission

\def\iccvPaperID{3328} % *** Enter the ICCV Paper ID here
\def\httilde{\mbox{\tt\raisebox{-.5ex}{\symbol{126}}}}

% Pages are numbered in submission mode, and unnumbered in camera-ready
\ificcvfinal\pagestyle{empty}\fi

\begin{document}

%%%%%%%%% TITLE
\title{CurveCloudNet: Processing Point Clouds with 1D Structure}

\author{
Colton Stearns\\
Stanford University\\
% {\tt\small coltongs@stanford.edu}
%
\and
Jiateng Liu\\
Zhejiang University\\
%
\and
Davis Rempe\\
Stanford University\\
% {\tt\small drempe@stanford.edu}
%
\and
Despoina Paschalidou\\
Stanford University\\
% {\tt\small paschald@stanford.edu}
%
\and
Jeong Joon Park\\
Stanford University\\
% {\tt\small jjpark3d@stanford.edu}
%
% {\tt\small jjpark3d@stanford.edu}
\and
Sébastien Mascha\\
Summer Robotics\\
% {\tt\small sebastien@summerrobotics.ai}
%
\and
Leonidas J. Guibas\\
Stanford University\\
% {\tt\small guibas@cs.stanford.edu}
% \and
% \textbf{\textsuperscript{1}Stanford University}
% \and
% \textbf{\textsuperscript{2}Zhejiang University}
% \and
% \textbf{\textsuperscript{3}Summer Robotics}
}
% \institute{Stanford University \and Toyota Research Institute}

\maketitle
% Remove page # from the first page of camera-ready.
\ificcvfinal\thispagestyle{empty}\fi

%%%%%%%%% ABSTRACT



\begin{abstract}

% Version 1:
% Modern depth sensors such as LiDAR operate by sweeping laser-beams across the scene, resulting in a point cloud with notable 1D curve-like structures. However, most existing point cloud backbones  discard the rich, 1D traversal patterns and rely mainly on Euclidean operations.
% In this work, we present a novel point cloud processing scheme and backbone, \textbf{CurveCloudNet}, that exploits the curve-like structure of modern depth sensors. Concretely, %instead of treating each point independently, 
% we parameterize the point cloud as a collection of polylines and thus establish a local surface-level ordering on the points. 
% We then devise curve-specific operations to process the ``curve clouds:'' (1) a \textit{symmetrical 1D convolution}, 2) a \textit{ball grouping} operation for merging points along curves, and (3) an efficient \textit{1D furthest-point-sampling} algorithm on curves. \textbf{CurveCloudNet} combines these curve operations with existing point-based operations, resulting in an efficient, scalable, and expressive backbone that uses little GPU memory. We evaluate \textbf{CurveCloudNet} on the ShapeNet, Kortx, Audi Driving, and nuScenes datasets, showcasing state-of-the-art segmentation and classification performance across {\em both} object-level and large outdoor scene datasets, the first reported 3D point backbones to do so. 

% Version 2:
% In this work we introduce a new point cloud processing scheme and backbone, called CurveCloudNet, which takes advantage of the curve-like structure inherent in modern depth sensors such as LiDAR. While traditional point cloud backbones discard the rich, 1D laser-traversal patterns and rely on Euclidean operations, CurveCloudNet parameterizes the point cloud as a collection of polylines. This parameterization establishes a local surface-level ordering on the points. Our method applies curve-specific operations to process the ``curve clouds," including symmetrical 1D convolution, ball grouping for merging points along curves, and an efficient 1D furthest-point-sampling algorithm on curves. Combining these curve operations with existing point-based operations results in an efficient, scalable, and expressive backbone that uses little GPU memory. We evaluate CurveCloudNet on several datasets, including ShapeNet, Kortx, Audi Driving, and nuScenes, and report state-of-the-art segmentation and classification performance across \textbf{both} object-level and large outdoor scene datasets, making CurveCloudNet the first 3D point backbone to achieve such results.
% \vspace{-1em}

% Version 3
Modern depth sensors such as LiDAR operate by sweeping laser-beams across the scene, resulting in a point cloud with notable 1D curve-like structures. In this work, we introduce a new point cloud processing scheme and backbone, called \arch, which takes advantage of the curve-like structure inherent to these sensors. While existing backbones discard the rich 1D traversal patterns and rely on generic 3D operations, \arch parameterizes the point cloud as a collection of polylines (dubbed a ``curve cloud”), establishing a local surface-aware ordering on the points. By reasoning along curves, \arch captures lightweight curve-aware priors to efficiently and accurately reason in several \textbf{diverse} 3D environments. 
% , including a symmetric 1D convolution, a ball grouping for merging points along curves, and an efficient 1D farthest point sampling algorithm on curves.
We evaluate \arch on multiple synthetic and real datasets that exhibit distinct 3D size and structure.
%, including: ShapeNet, Audi Driving, nuScenes, Kitti, and a new dataset we name KortX.
We demonstrate that \arch outperforms both point-based and sparse-voxel backbones in various segmentation settings, notably scaling to large scenes better than point-based alternatives while exhibiting improved single-object performance over sparse-voxel alternatives.
In all, \arch is an efficient and accurate backbone that can handle a larger variety of 3D environments than past works. 
%In all, \arch is an off-the-shelf trainable and performant backbone that is ready for the diverse environments faced in open-world applications such as robotics. 

% in various segmentation settings, notably scaling better to large scenes than point-based alternatives while exhibiting better single object performance than sparse-voxel alternatives. 

% CurveCloudNet applies a mix of curve-specific operations and Euclidean point-based operations, resulting in an efficient and accurate backbone that can flexibly reason on \textit{many} different types of 3D scenes. 
% , including a symmetric 1D convolution, a ball grouping for merging points along curves, and an efficient 1D farthest point sampling algorithm on curves.
% By combining these curve operations with existing point-based operations, CurveCloudNet is an efficient and accurate backbone that can flexibly reason on \textit{many} different types of 3D scenes. 
% CurveCloudNet achieves state-of-the-art segmentation performance on the ShapeNet, Kortx, Audi Autonomous Driving, and nuScenes datsets, which include both individual objects and large outdoor scenes captured with various sensor scanning patterns. These evaluations demonstrate that \arch scales to large scenes better than existing point-based backbones while improving object-level semantic segmentation compared to sparse-voxel backbones.
% We evaluate semantic segmentation on four datasets - two common (ShapeNet and nuScenes) and two less common (KortX and Audi Driving). Taken together, these datasets patterns -
% We evaluate semantic segmentation the ShapeNet, Kortx, Audi Driving, and nuScenes datasets. 

% demonstrate that \arch outperforms both point-based and sparse-voxel backbones in various segmentation settings, notably scaling better to large scenes than point-based alternatives while exhibiting better single object performance than sparse-voxel alternatives. 

% Evaluations on ShapeNet, Kortx, Audi Driving, and nuScenes demonstrate that \arch outperforms point-based methods on both individual objects and large-scale scenes, outperforms sparse-voxel backbones on individual objects, and closes the gap between point-based and sparse-voxel backbones on large-scale scenes while requiring significantly less GPU memory.

% CurveCloudNet is evaluated on several datasets that include both individual objects and large
% outdoor scenes captured with various sensor scanning patterns. These evaluations demonstrate that our model can
% outperform point-based and sparse-voxel backbones at both
% object and scene level, achieving state-of-the-art performance on segmentation tasks.
\vspace{-1em}
\end{abstract}

\begin{figure}[tp]
    \centering
    \includegraphics[width=\linewidth]{figs/images/teaser_new.pdf}
    \caption{We propose a unified method for four low-level structure segmentation tasks: camouflaged object, forgery, shadow and defocus blur detection~(Top). Our approach relies on a pre-trained frozen transformer backbone that leverages explicit extracted features, \eg, the frozen embedded features and high-frequency components, to prompt knowledge. } 
    \label{fig:teaser}
\end{figure}

%%%%%%%%%%%%%%%%%%%%%%%%%%%%%%%%%%%%%%%%%%%%%%%%%%%%%%%%%%
% REMEMBER STORYTELLING
%%%%%%%%%%%%%%%%%%%%%%%%%%%%%%%%%%%%%%%%%%%%%%%%%%%%%%%%%%

\section{Introduction}

Integrating \gls{ntn}, e.g. satellites and \glspl{uav}, into current terrestrial infrastructure is one of the important pillars in the development of the sixth-generation standard of mobile communication networks \cite{3GPP.TR.38.863}. So-called holistic 3D networks will enable ubiquitous global coverage, provide capacity for temporally and locally varying traffic demands and enhance the robustness of terrestrial network infrastructure \cite{Leyva-Mayorga2020, Qu}. However, a host of challenges is introduced due to the high dynamism of \gls{ntn} devices.
\gls{leo} satellites have especially fast-changing \gls{los} channels, which are mainly characterized by the relative positions between satellites and users.
To increase spectral efficiency through frequency reuse, precoding based \gls{sdma} is used in satellite communications~\cite{vazquez2018precoding}.
%satellite communications commonly combine \gls{sdma} and precoding \cite{vazquez2018precoding}. 
However, the performance suffers due to imperfect positional information~\cite{MaikBeamspace,liu2022robust}.
%imperfect positional information degrades the precoding performance~\cite{MaikBeamspace,liu2022robust}.

Finding a precoding algorithm that maximizes performance metrics such as the sum rate for multiple geometric constellations while also showing robustness against imperfect positioning and channel estimates can prove challenging. In light of this, \gls{ml} methods present themselves as an attractive choice. \gls{ml} can be used to approximate a viable algorithm where the optimum is either infeasible to determine or wholly unavailable. \gls{dl} in particular has demonstrated tremendous potential on such problems in the past decade, \cite{mnih2013playing, dahrouj_overview_2021}.
Applying \gls{dl} to precoding has recently started to gather more attention, primarily in terrestrial communications, \eg \cite{zhang_data_2022} use supervised learning to approximate a lower complexity \gls{mmse} precoder, \cite{lee_deep_2020} show the ability of \gls{rl} precoders to optimally learn on toy scenarios without interference and\cite{sohrabi_robust_2020} use an autoencoder structure to learn robust precoding and decoding under imperfect channel knowledge.
In the \gls{leo} satellite context, \cite{liu2022robust} have extended their work on robust precoding pertaining imperfect positional knowledge for a single satellite scenario by a supervised low-complexity approximation.
In this paper, we will %opt to 
use model-free deep \gls{rl}. In \gls{rl}, an agent probes the environment (\ie selecting a precoding matrix and observing the result), thereby generating data to learn from, to gain understanding of the system dynamics and adjust their behavior to maximize an objective.
%While deep \gls{rl} is typically applied to sequential decision problems such as {\color{red}example}, it is also suitable for time invariant decision processes, in fact, many {\color{red} value based} \gls{rl} algorithms gain convergence stability in the absence of time dependence.
By using this approach, no assumptions about the error modeling need to be made; it is instead discovered and inferred from the data.
However, data-driven learning requires data samples containing high information content to learn efficiently. For this reason, we use the \gls{sac} learning algorithm~\cite{haarnoja_soft_2019}. \gls{sac} encourages exploring new data samples where the algorithm's understanding is low, generating the necessary high quality data faster than random exploration.

In the next section, we introduce the system model of cooperative multibeam satellite communication, the applied \gls{csit} error models, typical precoding approaches, and the sum rate maximization problem. Following that, we explain the \gls{sac} method as it is used in this paper. We then apply \gls{sac} to maximize the sum rate in the presence of \gls{csit} error, and discuss \& contrast the performance against common \gls{mmse} and \gls{oma} precoding. Finally, we briefly discuss the scalability of the model case presented in this paper. For brevity, we assume prior knowledge of deep \glspl{nn}~\cite{goodfellow_deep_2020}.
 
\textit{Notations}: Lower and upper boldface letters denote vectors $\mathbf{x}$ and matrices $\mathbf{X}$ with $\mathbf{I}_N$ being an identity matrix of size $N \times N$. Transpose and Hermitian operators are indicated by $\{\cdot\}^\text{T}$ and $\{\cdot\}^\text{H}$, whereas $\circ$ is the Hadamard product. $|\cdot |$ and $\| \cdot \|$ signify the absolute value and Euclidean norm, respectively.
 
 
 
 
%%%%%%%%%%%%%%%%%%%%%%%%%%%%%%%%%%%%%%%%%%%%%%%%%%%%%%%%%%
%% EXAMPLE TABLE
%\begin{table}[!t]
%	\renewcommand{\arraystretch}{1.3}
%	\caption{A Simple Example Table}
%	\label{tab:table_example}
%	\centering
%	\rowcolors{2}{white}{gray!10} 
%	\begin{tabular}{cc}
%		\hline
%		\bfseries First & \bfseries Next\\
%		\hline\hline
%		1.0 & 2.0\\
%		1.0 & 2.0\\
%		\hline
%	\end{tabular}
%\end{table}

%% EXAMPLE FIGURE
%\begin{figure}[!t]
%	\centering
%	\includegraphics[width=2.5in]{myfigure}
%	\input{figures/myfigure.pgf}
%	\caption{text}
%	\label{fig:figure_example}
%\end{figure}

\begin{figure*}
    \centering
    \includegraphics[width= 0.92\textwidth]{content/main/images/overview.png}
    \caption{\textit{Overview of Curve Cloud Reasoning.} Starting from laser-scanned input data, we \textcircled{1} link points into polylines to \textcircled{2} parameterize the point cloud as a curve cloud (see \cref{sec:prelims}). We develop operations for learned architectures to specifically exploit the curve structure, including \textcircled{3} 1D farthest-point-sampling along a curve, \textcircled{4} curve grouping, and \textcircled{5} symmetric curve convolutions (see \cref{sec:curve-ops}).}
    \label{fig:overview}
    \vspace{-3mm}
\end{figure*}
\section{Related Work}

Existing point cloud methods can be roughly characterized as
point-based and voxel-based approaches. As our work addresses trade-offs between them, we discuss related works from each category.

\paragraph{Point-Based Networks}
% \subsection{Point-Based Networks}
% \boldparagraph{Point-based Networks}%
Prior work has extensively studied backbones that map a 3D point cloud to a high-dimensional feature space used for downstream applications, such as 3D reconstruction, shape classification, part segmentation, semantic segmentation, and more \cite{Fan2017CVPR, Qi2017CVPR,
Qi2017NIPS, Wang2019SIGGRAPHb, Thomas2019ICCV}. 
%
PointNet \cite{Qi2017CVPR} was a seminal work that combined a series of MLPs with a max pooling layer to learn point-wise features. Following PointNet, several works proposed to aggregate local neighborhood information using hierarchical grouping at multiple geometric scales \cite{Qi2017NIPS, Li2018CVPR, Qian2022PointNeXtRP}. Recently, Ma \etal \cite{Ma2022ICLR} introduced a compelling MLP-based architecture that combines residual multi-scale reasoning and affine transformations. 
% 
Nevertheless, most hierarchical and MLP point networks are inefficient on large-scale point clouds, and although several backbones \cite{Hu2020CVPR,Zhang2022CVPR,Yang20203DSSDP3} have addressed this, they trade off scalability with task-specific frameworks or lower accuracy. 
In contrast, CurveCloudNet uses an efficient curve cloud representation to achieve
superior performance on both objects and large-scale scenes.

Another line of research introduced \emph{kernel-based
convolutions} for learning per-point local features
\cite{Su2018CVPR, Hua2018CVPR, Wang2018CVPRb, Xu2018ECCV, Esteves2018ECCV,
Wu2019CVPR, Lei2019CVPR, Komarichev2019CVPR, Lan2019CVPR, Thomas2019ICCV, Wiersma2022SIGGRAPH}. Kernels are defined using a family of polynomial
functions~\cite{Xu2018ECCV} or can be estimated using MLPs~\cite{Wang2018CVPRb,
Liu2019CVPR}. Likewise, \cite{Atzmon2018SIGGRAPH, Thomas2019ICCV, Wu2019CVPR, Xu2021CVPR,
Boulch2020ACCV} defined the kernel weights directly using the local 3D point coordinates.
% More sophisticated models such as Spherical CNN~\cite{Esteves2018ECCV} addressed 3D rotation equivariance by implementing convolutions in the spherical harmonic domain.
% InterpConv~\cite{Mao2019ICCV} utilized the point coordinates to interpolates
% point features to the neighboring kernel weights and
% DeltaConv~\cite{Wiersma2022SIGGRAPH} proposed to construct anisotropic convolutions by learning combinations of differential operators.
More related to our work is CurveNet~\cite{Xiang2021ICCV}, which proposed to perform random walks on point clouds and aggregate features along these ``curves" to capture local geometric details. In constrast, our work defines efficient point cloud operations for existing 1D curve structures.

An alternative line of research proposed to construct a
graph from the input point cloud and apply \emph{graph-based convolutions}
\cite{Te2018SIGGRAPH, Liu2019ICCVb, Chen2020CVPRa, Eliasof2020NeurIPS} to capture local geometric structure. For example,
\cite{Shen2018CVPR} introduced a graph-based network that replaced
convolutions with correlations computed between points and their $k$-nearest neighbors. Similarly, \cite{Verma2018CVPR,
Wang2019SIGGRAPHb} proposed to construct a local neighborhood graph and apply convolutional operations on the edges connecting neighboring pairs of points.
% DiffGCN~\cite{Eliasof2020NeurIPS} performed message passing based on approximate surface gradients and an approximate Laplacian matrix, whereas \cite{Wang2018ECCVc} investigated using local spectral graph convolution.
Recent works \cite{Xie2018CVPR, Liu2019AAI, Yang2019CVPR, Zhang2019CVPR, Hehe2021CVPR, Zhao2021ICCV, Ishan2021ICCV} have also explored
applying \emph{attention-based aggregation} using
transformer architectures with self-attention \cite{Vaswani2017NIPS}.  Unlike previous graph-based and attention-based networks, CurveCloudNet reasons over local 1D ``curve" neighborhoods.
% Applying neural networks to process 3D point cloud data is an extremely well
% studied problem.
% Beginning with PointNet \cite{pointnet}, a series of follow-up
% works improved learnable point-cloud operations for capturing both local and
% global structure \cite{pointnet2, dgcnn, kpconv, votenet, pa_conv, fa_conv}.
% PointNet++ \cite{pointnet2} applied PointNet hierarchically to extract features
% at many geometric scales.
% DGCNN \cite{dgcnn} constructs a nearest-neighbors
% graph and performs message passing and aggregation between points. KpConv
% \cite{kpconv} develops a convolutional kernel that is defined on a set of
% deformable 3D key points, and other methods extend this idea of point cloud
% convolutions \cite{pa_conv, fa_conv, 3detr}.
% In recent years, researchers have
% also applied self-attention and transformers to point clouds
% \cite{pointtransformer, pct, 3detr, point4dtransformer}.
% Additionally, some
% recent methods attempt to incorporate intrinsic priors into extrinsic point
% cloud backbones \cite{curvenet, diffgcn, deltaconv}.
% CurveNet \cite{curvenet}
% performs random walks on point clouds and aggregates features along these
% ``curves". DeltaConv defines both per-point scalar and gradient features and
% applies feature updates inspired by discrete exterior calculus
% \cite{deltaconv}.
% DiffGCN performs graph message passing based on approximate
% surface gradients and and approximate Laplacian matrix \cite{diffgcn}.
%We differ from previous point cloud methods in that we are interested in the 1D sampling pattern that arises from many 3D scanners.

\paragraph{Voxel-Based Networks}
% \subsection{Voxel-Based Networks}
% \boldparagraph{Voxel-based Networks}%
Although point-based backbones can successfully process individual objects or
small indoor scenes, they struggle to scale to large point clouds due to inefficiency in processing large unstructured point sets. To address this,
several works \cite{Su2018CVPR, Graham2018CVPR, Choy2019CVPR,
Liu2019NeurIPS, Zhang2020CVPR, Zhou2020ARXIV, Zhang2022ARXIV, Yan2018Sensors, Lang2018CVPR, Liu2021TPAMI, Xu2021ICCV, Hou2022CVPR} proposed to convert a point cloud into a 3D
voxel grid and use this volumetric representation. Early works converted a point cloud into a dense voxel grid and applied dense 3D convolutions \cite{Maturana2015IROS, Qi2016VolumetricAMCVPR}, however the cubic size of the dense grid proved to be computationally prohibitive. PVCNN~\cite{Liu2019NeurIPS}
was a seminal work in combining point operations with low-resolution dense voxel convolutions to efficiently
process smaller-scale point clouds. % Nevertheless, their dense voxel representation still struggled to scale to large scenes.

% sparse voxel architectures
To scale to large scenes, several works \cite{Choy2019CVPR, Lang2018CVPR, Zhou2020ARXIV, Liu2021TPAMI, Hou2022CVPR} employed the sparse-voxel data structure from \cite{Graham2018CVPR}. 
% MinkowskiNet~\cite{Choy2019CVPR} extended sparse voxel convolutions to the time domain, showing efficient and accurate reasoning on 4D scenes. 
PVNAS~\cite{Liu2021TPAMI} incorporated a network architecture search, demonstrating the importance of the architecture channels, network depth, kernel sizes, and training schedule. PVT~\cite{Zhang2022ARXIV} introduced a sparse attention module to efficiently process per-voxel local features using a transformer encoder~\cite{Vaswani2017NIPS}.
% Notably, although our model only processes 3D points, we showcase that it can be efficiently employed on larger point clouds.
%
% better descritization
Other methods seek a better voxel discretization of point clouds captured with \lidar scans. For example, PolarNet~\cite{Zhang2020CVPR}
proposed to partition input points using grid cells defined in a polar
coordinate system, while Cylinder3D~\cite{Zhou2020ARXIV} employed a cylindrical
partitioning scheme based on a cylindrical coordinate system. In an alternative line of research, many methods \cite{Wu2019ICRA, Xu2020ECCV, Wu2018ICRA, Wu2019CVPR}
employed spherical or bird's-eye view projections to represent point clouds as images that are passed to a convolutional neural network. 

Unlike these works, our model directly operates on points and curves, scaling to larger scenes without 2D or 3D convolutions. Furthermore, our model does not rely on the point cloud to exhibit cylindrical or planar properties. 
% individual objects, where the point clouds do not necessarily follow
% cylindrical patterns (see \cref{fig:teaser}).

% Unlike these works, our model directly operates on points and curves, and hence does not
% rely on either 2D or 3D convolutions. Notably, although our model only processes 3D points,
% we showcase that it can be efficiently employed on larger point clouds.

% 
% to take advantage of the 3D topology of point clouds captured with LIDAR scans and
% instead of converting the input to a voxel grid, they rely on on cylindrical
% partitioning schemes. For example, PolarNet~\cite{Zhang2020CVPR}
% proposed to partition input points using grid cells, defined in a polar
% coordinate system, while Cylinder3D~\cite{Zhou2020ARXIV} employed a cylinder
% partitioning scheme based on a cylinder coordinate system. Although, both
% \cite{Zhang2020CVPR, Zhou2020ARXIV} can be successfully applied to point
% clouds captured with LIDAR scans, they perform poorly for the case of
% individual objects, where the point clouds do not necessarily follow
% cylindrical patterns (see \cref{fig:teaser}). In an alternative line of research, \cite{Wu2018ICRA, Wu2019CVPR}
% employ spherical projections to represent point clouds as images that are then
% passed to a convolutional neural network to extract per-point features. Unlike
% these works, our model directly operates on points and curves, and hence does not
% rely on either 2D or 3D convolutions. Notably, although our model only processes 3D points,
% we showcase that it can be efficiently employed on larger point clouds.

% Sparse voxel backbones convert a point cloud into a discrete voxel grid and
% apply a series of sparse 3D convolutions or 2D birds-eye-view convolutions
% \cite{octnet, second, splatnet, mincowskinet}. The sparse-voxel backbone is a
% popular alternative to point cloud backbones due to efficient grid
% representations and the ability to scale to large scenes \cite{randla-net,
% performance-lidar}. MinkowskiNet \cite{mincowskinet} is a seminal work in 3D
% and 4D sparse convolutions, offering a highly efficient and expressive
% implementation of the sparse-voxel infrastructure. Following suit, many works
% extend MinkowskiNet \cite{cylinder3d, other-kitti-one, polarnet, spvcnn}.
% Additionally, newer works fuse spare voxel operations and point operations to
% improve overall expressivity \cite{pvcnn, kitti-iclr1, kitti-iclr1-followup,
% pointvoxeltransformer}.

% \subsection{LiDAR Range View Representation}
% In LiDAR processing, it is not uncommon to use the a spherical projection into a 2D plane, called the range view \cite{rangenet++, rangenet, rpvnet}. The range view offers a compact 2D approximation of the 3D point cloud structure, and allows the use of standard 2D image processing techniques. 

% In classic ``parallel-sweeping" LiDAR, the spherical projection will result in a reformatting similar to our 1D polylines. However, our 1D polylines are far more general as they are not contrained to the parallel direction (necessary for Audi dataset), they operate directly in 3D space (integrate more easily into existing backbones), and the range view additionally uses a ``vertical" relationship that our polylines do not.


%\boldparagraph{Sensor-Specific Networks}%
%Many previous works modify 3D backbones to better accommodate sensor sampling
%patterns. A number of works address the outputs of long-range LiDARs. To
%explain, most long-range LiDAR sensors vertically stack 32 to 128 laser beams
%and rapidly ``sweep" the beams in a circular fashion. Many works offer
%long-range LiDAR improvements by polar and cylindrical geometric priors.
%PolarNet \cite{polarnet} introduces polar coordinates in a sparse-voxel
%representation. Range-view methods \cite{rangenet++, rangenet, rpvnet,
%solomon_pillar} perform a 2D polar projection of LiDAR measurements the
%range-view. Cylinder3D \cite{cylinder3d} performs cylindrical convolutions on a
%sparse-voxel representation. 

% In addition to sensor-specific architectures, many works modify 3D backbones for specific tasks. In mesh correspondence and protein classification, recent works apply 2D convolutions along approximated surfaces to better understand intrinsic geometry \cite{mesh_cnn, diffusionnet, others}. Siren \cite{siren} includes sinusoidal activation functions to better capture high-frequency information, while rotation-equivalent backbones modify the neural operators to preserve rotation information. Finally, in medical imaging \cite{something}, \textcolor{yellow}{something else here}.

%In contrast to previous methods, CurveCloudNet targets a more general 1D sampling pattern that includes but is not limited to the parallel ``sweeping" pattern of long-range LiDAR sensors.
% formulate the point cloud as a set of local 3D polylines. In addition to being more fine-grained, our 3D polylines are not tailored for a specific type of sensor motion (i.e. parallel sweeps). 

\section{Method}
% \label{sec:method}

\begin{figure*}[t]
\centering
\includegraphics[width=\linewidth]{figures/framework_v3.pdf}
\vspace{-5mm}
\caption{An overview of our 3D-CLR framework. First, we learn a 3D compact scene representation from multi-view images using neural fields (I). Second, we use CLIP-LSeg model to get per-pixel 2D features (II). We utilize a 3D-2D alignment loss to assign features to the 3D compact representation (III). By calculating the dot-product attention between the 3D per-point features and CLIP language embeddings, we could get the concept grounding in 3D (IV). Finally, the reasoning process is performed via a set of neural reasoning operators, such as \textsc{Filter}, \textsc{Get\_Instance} and \textsc{Count\_Relation} (V). Relation operators are learned via relation networks.}
\vspace{-5mm}
\label{fig:framework}
\end{figure*}

Fig.~\ref{fig:framework} illustrates an overview of our framework. Specifically, our framework consists of three steps.  First, we learn a 3D compact representation from multi-view images using neural field. And then we propose to leverage pre-trained 2D vision-and-language model to ground concepts on 3D space. This is achieved by 1) generating 2D pixel features using CLIP-LSeg; 2) aligning the features of 3D voxel grid and 2D pixel features from CLIP- LSeg~\cite{li2022language}; 3) dot-product attention between the 3D features and CLIP language features~\cite{li2022language}. Finally, to perform visual reasoning, we propose neural reasoning operators, which execute the question step by step on the 3D compact representation and outputs a final answer. For example, we use \textsc{Filter} operators to ground semantic concepts on the 3D representation, \textsc{Get\_Instance} to get all instances of a semantic class, and \textsc{Count\_Relation} to count how many pairs of the two semantic classes have the queried relation.
% \gc{metion all the neural operators.}
% Works from linguistic and cognitive science suggest that semantic concepts are diverse and open-vocabulary, while relational concepts describing 3D objects' relationships can be very limited and thus can be considered a close-class vocabulary \cite{Landau1993WhatA, Hayward1995SpatialLA}. Therefore, it's unrealistic to learn the embeddings of all the concepts in the question-answering pairs, while it's more natural to learn the relation embeddings. Inspired by this, we propose to leverage 2D pretrained vision-language model (\textit{i.e.,} CLIP) for open-vocabulary semantic concept learning, while proposing a neural relation module network for relational reasoning. 

\subsection{Learning 3D Compact Scene Representations}

% Since 3D-related reasoning works on 3D compact representations rather than 2D images, we first propose to use a neural field to extract 3D representations from multi-view images. The next step is to learn the 3D features for visual reasoning. However, 3D assets are limited in diversity and scale, posing challenges for training large-scale 3D foundation models, while there's much progress on large-scale 2D pretrained models which provide decent features\cite{Radford2021LearningTV, Ramesh2021ZeroShotTG}. Since neural field maps a 2D pixel to several 3D points along the ray, it's natural to get 3D features for 2D per-pixel features. We apply CLIP-LSeg\cite{Li2022LanguagedrivenSS} to learn per-xel 2D features, and use an alignment loss to align 3D features with 2D features.

% \paragraph{3D Compact Representation from neural field.} 
Neural radiance fields  \cite{mildenhall2020nerf} are capable of learning a 3D representation that can reconstruct a volumetric 3D scene representation from a set of images. Voxel-based methods \cite{Garbin2021FastNeRFHN, Hedman2021BakingNR, Yu2021PlenOctreesFR, Sun2022DirectVG} speed up the learning process by explicitly storing the scene properties (\textit{e.g.}, density, color and feature) in its voxel grids. We leverage Direct Voxel Grid Optimization (DVGO) \cite{Sun2022DirectVG} as our backbone for 3D compact representation for its fast speed. DVGO stores the learned density and color properties in its grid cells. The rendering of multi-view images is by interpolating through the voxel grids to get the density and color for each sampled point along each sampled ray, and integrating the colors based on the rendering alpha weights calculated from densities according to quadrature rule \cite{Max1995OpticalMF}. The model is trained by minimizing the L2 loss between the rendered multi-view images and the ground-truth multi-view images. By extracting the density voxel grid, we can get the 3D compact representation (\textit{e.g.,} By visualizing points with density greater than 0.5, we can get the 3D representation as shown in Fig. \ref{fig:framework} I. ) 

\subsection{3D Semantic Concept Grounding}
Once we extract the 3D compact representation of the scene, we need to ground the semantic concepts for reasoning from language. 
Recent work from \cite{hong20223d} has proposed to ground concepts from paired 3D assets and question-answers. Though promising results have been achieved on synthetic data, it is not feasible for open-vocabulary 3D reasoning in real-world data, since it is hard to collect large-scale 3D vision-and-language paired data.  To address this challenge, our idea is to leverage  pre-trained 2D vision and language model \cite{Radford2021LearningTV, Ramesh2021ZeroShotTG} for 3D concept grounding in real-world scenes.  But how can we map 2D concepts into 3D neural field representations? Note that 3D compact representations can be learned from 2D multi-view images and that each 2D pixel actually corresponds to several 3D points along the ray. Therefore, it's possible to get 3D features from 2D per-pixel features. Inspired by this, we first add a feature voxel grid representation to DVGO, in addition to density and color, to represent 3D features. 
% it's natural to utilize 2D VLMs to ground semantic concepts on the 3D representations. 
 We then apply CLIP-LSeg\cite{li2022language} to learn per-pixel 2D features, which can be attended to by CLIP concept embeddings. We use an alignment loss to align 3D features with 2D features so that we can perform concept grounding on the 3D representations.
% Since 3D voxel grids and 2D pixels are aligned via alpha compositing, we add one L1 loss to force the features of 3D voxel grids to align with the 2D LSeg pixels based on the alpha values. 

\noindent\textbf{2D Feature Extraction.}
To get per-pixel features that can be attended by concept embeddings, we use the features from language-driven semantic segmentation (CLIP-LSeg) \cite{li2022language}, which learns 2D per-pixel features from a pre-trained vision-language model (\textit{i.e.,} \cite{Radford2021LearningTV}). Specifically, it
uses the text encoder from CLIP, trains an image encoder to produce an embedding vector for each pixel, and calculates the scores of word-pixel correlation by dot-product. By outputting the semantic class with the maximum score of each pixel, CLIP-LSeg is able to perform zero-shot 2D semantic segmentation.

\noindent\textbf{3D-2D Alignment.}
In addition to density and color, we also store a 512-dim feature in each grid cell in the compact representation. To align the 3D per-point features with 2D per-pixel features, we calculate an L1 loss between each pixel and each 3D point sampled on the ray of the pixel. The overall L1 loss along a ray is the weighted sum of all the pixel-point alignment losses, with weights same as the rendering weights: $\mathcal{L}_{\text {feature}}=\sum_{i=1}^K w_i(\|\boldsymbol{f_i}-F(\boldsymbol{r})\|),$
where $\boldsymbol{r}$ is a ray corresponding to a 2D pixel, $F(\boldsymbol{r})$ is the 2D feature from CLIP-LSeg, $K$ is the total number of sampled points along the ray and $\boldsymbol{f_i}$ is the feature of point $i$ by interpolating through the feature voxel grid, $w_i$ is the rendering weight.
% \gc{add equations.} 

\noindent\textbf{Concept Grounding through Attention.}  Since our feature voxel grid representation is learnt from CLIP-LSeg, by calculating the dot-product attention $<\boldsymbol{f}, \boldsymbol{v}> $ between per-point 3D feature $\boldsymbol{f}$ and the CLIP concept embeddings $\boldsymbol{v}$, we can get zero-shot view-independent concept grounding and semantic segmentations in the 3D representation, as is presented in Fig. \ref{fig:framework} IV. 
% \gc{add equations.}

\subsection{Neural Reasoning Operators}
Finally, we use the grounded semantic concepts for 3D reasoning from language. We first transform questions into a sequence of operators that can be executed on the 3D representation for reasoning. We adopt a LSTM-based semantic parser   \cite{Yi2018NeuralSymbolicVD} for that. As \cite{Mao2019TheNC, hong20223d}, we further devise a set of operators which can be executed on the 3D representation.  Please refer to \textbf{Appendix} for a full list of operators.

\noindent\textbf{Filter Operators.}  We filter all the grid cells with a certain semantic concept.

\noindent\textbf{Get\_Instance Operators.} We implement this by utilizing DBSCAN \cite{Ester1996ADA}, an unsupervised algorithm which assigns clusters to a set of points. Specifically, given a set of points in the 3D space, it can group together the points that are closely packed together for instance segmentation.

\noindent\textbf{Relation Operators.} We cannot directly execute the relation on the 3D representation as we have not grounded relations. Thus, we represent each relation using a distinct neural module (which is practical as the vocabulary of relations is limited \cite{Landau1993WhatA}). We first concatenate the voxel grid representations of all the referred objects and feed them into the relation network.
% \yd{Do we do something afterwards -- we first concatenate, then what?} 
The relation network consists of three 3D convolutional layers and then three 3D deconvolutional layers. A score is output by the relation network indicating whether the objects have the relationship or not. Since vanilla 3D CNNs are very slow, we use Sparse Convolution \cite{spconv2022} instead. Based on the relations asked in the questions, different relation modules are chosen. 

% \subsection{Learning 3D Compact Representation}
% In recent years, neural field models(\textit{e.g.,} \cite{mildenhall2020nerf}) have gained much popularity since they can reconstruct a volumetric 3D scene representation from a set of images. Recent works \cite{Garbin2021FastNeRFHN, Hedman2021BakingNR, Yu2021PlenOctreesFR, Sun2022DirectVG} have pushed it further by using classic voxel-grids to explicitly store the scene properties (\textit{e.g.}, density, color and feature) for rendering, which allows for real-time rendering. Since concept grounding and relation learning are expected to work on the per-point features in the 3D space \cite{hong20223d} of thousands of scenes, it's more suitable to use voxel-grid-based methods since they store explicit properties in each point which can be directly used for reasoning, and super-fast convergence makes it feasible to train thousands of scenes. Specifically, we use the fine reconstruction process of Direct Voxel Grid Optimization \cite{Sun2022DirectVG} as our backbone for 3D compact representation for its fast speed. 

% A compact voxel-grid representation models the modalities of interest (\textit{e.g.,} density, color or feature) explicitly in its grid cells. To query the properties at any given 3D point, interpolation is used:
% \begin{equation}
% \operatorname{interp}(\boldsymbol{x}, \boldsymbol{V}):\left(\mathbb{R}^3, \mathbb{R}^{C \times N_x \times N_y \times N_z}\right) \rightarrow \mathbb{R}^C
% \end{equation}
% where $\boldsymbol{x}$ is the queried 3D point,  $\boldsymbol{V}$ is the voxel grid, and $C$ is
% the dimension of one of the modalities, and $N_x, N_Y, N_z$ is the number of voxels. We first predict the density of a specified point by interpolating the density grid. This is crucial for the geometric reconstruction of the scene.  
% \begin{equation}
% \sigma=\operatorname{interp}\left(\boldsymbol{x}, \boldsymbol{V}^{(\text {density })}\right)
% \end{equation}
% where $\sigma$ is the volume density at position $\boldsymbol{x}$. For the modeling of color emission, we use an explicit-implicit hybrid representation where  a shallow MLP is placed after the color voxel grid interpolation process:
% \begin{equation}
% \boldsymbol{c}=\operatorname{MLP}^{(\mathrm{rgb})}\left(\operatorname{interp}\left(\boldsymbol{x}, \boldsymbol{V}^{(\mathrm{color})}\right), \boldsymbol{x}, \boldsymbol{d}\right)
% \end{equation}
% where $\boldsymbol{c}$ is the view-dependent color emission at position $\boldsymbol{x}$ viewing from direction $\boldsymbol{d}$.

% To render the color $\hat{C}(\boldsymbol{r})$ of ray $r$, K points are sampled on ray $r$ with densities and colors $\left\{\left(\sigma_i, \boldsymbol{c}_i\right)\right\}_{i=1}^K$. The K results are accumulated by the quadrature rule by Max \cite{Max1995OpticalMF}:
% \begin{align}
% \hat{C}(\mathbf{r})=\sum_{i=1}^K T_i\left(1-\exp \left(-\sigma_i \delta_i\right)\right) \mathbf{c}_i, 
% &\\
% T_i=\exp \left(-\sum_{j=1}^{i-1} \sigma_j \delta_j\right)
% \end{align}
% where $\delta_i=t_{i+1}-t_i$ is the distance between adjacent points along a ray, and $\alpha_i=1-\exp \left(-\sigma_i \delta_i\right)$ is the alpha value for traditional alpha compositing.

% The backbone is trained by minimizing the mean
% square error between the rendered and observed color. 

% \begin{equation}
% \mathcal{L}_{\text {color }}=\|\hat{C}(\boldsymbol{r})-C(\boldsymbol{r})\|_2^2
% \end{equation}

% By extracting the density values of the voxel grid $\boldsymbol{V}^{(\text {density })} \in \mathbb{R}^{1 \times N_x \times N_y \times N_z}$, we can get the compact 3D representation of the scene, as shown in the middle of Figure 2.

% We refer the readers to \cite{Sun2022DirectVG} for more details about the Direct Voxel Grid Optimization.


% \subsection{3D Semantic Concept Grounding}
% In \cite{hong20223d}, a Neural Descriptor Field (NDF) \cite{simeonov2021neural} which gives a feature vector for each 3D coordinate a feature vector is used for concept grounding by aligning the feature vector with the learned concept embeddings. Drawing inspiration from this, we also propose to use a feature voxel-grid  (in addition to density voxel grid and color voxel grid) used for concept grounding. The compact 3D feature representation is composed of one feature voxel-grid representation plus one view-independent shallow MLP: 

% \begin{equation}
% \boldsymbol{f}=\operatorname{MLP}^{(\mathrm{feature})}\left(\operatorname{interp}\left(\boldsymbol{x}, \boldsymbol{V}^{(\mathrm{feature})}\right), \boldsymbol{x}, \boldsymbol{d}\right)
% \end{equation}

% However, the drawback of \cite{hong20223d} is that the embeddings of concepts are learnt from sratch, which is unrealistic in the open-vocabulary reasoning in real-world data. Furthermore, compared to 2D data, 3D assets are limited in diversity and scale, posing challenges for training large vision-language models (VLMs) on 3D-and-language data. Therefore, there's no large-scale 3D VLMs that can be directly used for concept grounding. On the contrary, there's much progress on large-scale 2D VLMs \cite{Radford2021LearningTV, Ramesh2021ZeroShotTG} thanks to the countless image-caption data on the internet. Since we obtain 3D compact representations from 2D multi-view images, it's natural to utilize 2D VLMs to ground semantic concepts on the 3D representations. Based on the CLIP model \cite{Radford2021LearningTV}, LSeg\cite{Li2022LanguagedrivenSS} manages to ground semantic concepts on each 2D pixel (and thus each ray $r$). We denote the feature of ray $r$ as $F(\boldsymbol{r})$.
% Since 3D voxel grids and 2D pixels are aligned via alpha compositing, we add one L1 loss to force the features of 3D voxel grids to align with the 2D LSeg pixels based on the alpha values. Specifically,

% \begin{equation}
% \mathcal{L}_{\text {feature}}=\sum_{i=1}^K T_i\left(1-\exp \left(-\sigma_i \delta_i\right)\right)(\|\boldsymbol{f}-F(\boldsymbol{r})\|)
% \end{equation}

% Assuming we have a set of concepts $P$, the similarities between a concept $\boldsymbol{p} \in P$ and a feature $\boldsymbol{f}$ is calculated as $\langle \boldsymbol{f}, \boldsymbol{p} \rangle$. We define a \textsc{Filter} operator. Specifically, the 3D compact representation for a semantic class $p$ after filtering out that class is:

% \begin{equation}
% \boldsymbol{V}_{\boldsymbol{p}} =  min(\langle\boldsymbol{V}^{(\mathrm{feature})}, \boldsymbol{p}\rangle, \boldsymbol{V}^{(\mathrm{density})}) 
% \end{equation}

% In practice, we only set the values of voxel grids with densities < 0.5 to 0, since we find that those points are irrelevant to the 3D geometry of the scene.

% To get each instance of the objects of the same category, we use DBSCAN \cite{Ester1996ADA} to implement the \textsc{Get\_Instance} operator which assigns clusters to all true values of $\boldsymbol{V}_{\boldsymbol{p}}$. The DBSCAN takes the 3D coordinates as input.





We begin by briefly comparing the performance of the three predictors for $\hat{\bm{x}}_i$ (FE, RK4, NN) before testing \cref{algo:simulator}.

\subsection{Predictor comparison}\label{subsec:results_predictors}
The approximator $\hat{\bm{x}}_i$ should have two properties: being fast and being accurate for large time-steps $\Delta t$. \Cref{fig:predictor_characteristics} shows the two properties. The left panel displays the error in the differential variable $\Delta \omega$ across 200 points of random initial conditions $\bm{x}_{0}$ and voltage parameterizations $\yparams{}_i$. In terms of accuracy, the \gls{NN} performs well and only for smaller time-steps ($\Delta t < 0.05 \si{\second}$), the RK4 approximation becomes more accurate. The RK-schemes exhibit the expected dependency of the time-step - the local truncation error should follow $\mathcal{O}(\Delta t)$ and $\mathcal{O}(\Delta t^4)$ for FE and RK4. The error of the \gls{NN} in contrast is near independent of $\Delta t$. At the same time, the \gls{NN} is the fastest approximator. While the FE approximator is similarly fast, its poor accuracy makes it undesirable and while more accurate, the RK4 scheme has the drawback of comparably long run-time.     

\begin{figure}[!th]
    \centering
    \includegraphics[width=0.95\linewidth]{figures_pdf/predictor_error.pdf}
    \caption{Predictor characteristics: (left) prediction error of $\Delta \omega$ for 200 predictions with random $\bm{x}_0$ and \yparams{}, (right) run-time per point. These results show that \gls{NN} constitute an accurate and fast predictor.}
    \label{fig:predictor_characteristics}
\end{figure}

\subsection{Simulator results}\label{subsec:results_simulators}

We now focus on the performance of the simulator in \cref{algo:simulator}. \Cref{fig:ieee9_prediction} displays the prediction for $\Delta t = \SI{0.05}{\second}$ using \gls{NN}-based approximators $\hat{\bm{x}}_i^{NN}$. 
\begin{figure}[!ht]
    \centering
    \includegraphics[width=0.95\linewidth]{figures_pdf/simulation_results.pdf}
    \caption{Prediction of a state trajectory ($\Delta \omega_i$) and an algebraic variable $V_i$ for time-step size $\Delta t = \SI{0.05}{\second}$ with a \gls{NN}-based simulator. The predictions (dashed lines) coincide with the ground truth (gray lines).}
    \label{fig:ieee9_prediction}
\end{figure}
The results correspond to the first row in \cref{tbl:simulator_results} where we report the maximum absolute errors of $V$ and $\Delta \omega$ along the trajectory and the run-time. \Cref{tbl:simulator_results} shows further results for the \gls{RK4}-based simulator, for time-steps of $\Delta t = \SI{0.1}{\second}$ and $\Delta t = \SI{0.15}{\second}$ and for collocation points at $\bm{T}=[0.3, 0.7]\Delta t$ and $\bm{T}=[0.1, 0.3, 0.5, 0.7, 0.9]\Delta t$, i.e., $s=2$ and $s=5$. The \gls{NN}-based simulator is consistently faster and more accurate than the \gls{RK}-based simulator, except for the case of $\Delta t = \SI{0.05}{\second}$. These results confirm the predictor characteristics observed in \cref{subsec:results_predictors}. The simulation run-time scales approximately inversely proportional with $\Delta t$. Deviations can arise due to varying numbers of iteration in \cref{algo:parameter_update}, however, in the reported cases, we observe usually 5-7 iterations. Increasing the number of collocation points $s$ results in a small increase in run-time in all cases. In terms of accuracy, we observe that more collocation points can lead to better accuracy, when the overall solution quality is good. However, for too large time-steps, here $\Delta t = \SI{0.15}{\second}$, the effect might reverse. The choice of the location of the collocation points, i.e., $\bm{T}$, also matters.%\bz{Can we bold the best numbers in a box? This would help to highlight what people should look at. There are a lot of numbers.}
\begin{table}[!ht]
\renewcommand{\arraystretch}{1.2}
\caption{Comparison of simulators with different predictor schemes}
\label{tbl:simulator_results}
\centering
\begin{tabular}{ccc|ccc}
\toprule
$\Delta t$ & Predictor & s & run-time & $\max |V_i - \hat{V}_i|$ & $\max | \omega_i - \hat{\omega}_i|$ \\
$[\si{\second}]$ & & & [\si{\second}] & $\times 10^{-2} [\si{\pu}]$ & $\times 10^{-3} [\si{\pu}]$ \\ \midrule
\multirow{2}{*}{$0.05$} & NN & $5$ & $\bm{1.85}$ & $0.82$ & $0.35$\\
 & RK4 & $5$ & $3.88$ & $\bm{0.31}$ & $\bm{0.29}$\\ \midrule
\multirow{4}{*}{$0.10$} & \multirow{2}{*}{NN} & $2$ & $\bm{0.98}$ & $2.71$ & $1.29$ \\
& & $5$ & $1.05$ & $\bm{1.32}$ & $\bm{0.62}$ \\
 & \multirow{2}{*}{RK4} & $2$ & $2.21$ & $5.07$ & $2.40$ \\
& & $5$ & $2.27$ & $3.88$ & $2.01$ \\ \midrule
\multirow{4}{*}{$0.15$} & \multirow{2}{*}{NN} & $2$ & $\bm{0.60}$ & $\bm{6.28}$ & $2.93$ \\
& & $5$ & $0.77$ & $6.36$ & $\bm{2.90}$ \\
 & \multirow{2}{*}{RK4} & $2$ & $1.19$ & $11.8$ & $5.06$ \\
& & $5$ & $1.46$ & $19.0$ & $7.75$ \\
% $\approx$ 0.020 & BDF & - & 0.83 & 0.07 & 0.26\\
% $\approx$ 0.045 & BDF & - & 0.67 & 7.73 & 3.36\\
\bottomrule
\end{tabular}%
\end{table}



%The last two rows stem from the \gls{DAE}-solver in Assimulo based on a variable order backward differentiation formula (BDF) with different tolerance levels. We observe that the required time-step size is significantly smaller than for the proposed simulator. The resulting run-times are comparable, however, neither method was optimized for run-tme in this study. \bz{I would delete the the last two rows and skip this discussion. Not sure how this helps the paper.}
This paper presented a comprehensive analysis of the use of \acrfull{PINN} for power system dynamic simulations. We show that \glspl{PINN} (i) are 10 to 1'000 times faster than conventional solvers, (ii) do not face issues of numerical instability unlike conventional solvers, and, (iii) achieve a decoupling between the power system size and the required solution time. However, \glspl{PINN} are less flexible (i.e. they do not easily handle parameter changes), and require an up-front training cost. Overall, this makes \gls{PINN}-based solutions well-suited for repetitive tasks as well as task where run-time speed is crucial, such as for screening.

Besides the comparison between conventional and \gls{NN}-based methods, this paper conducts a deeper analysis on the parameters that affect the performance of the \gls{NN} solutions. In that respect, we introduce a new \gls{NN} regularisation, called dtNN, as a intermediate step between \glspl{NN} and \glspl{PINN}. We show that \glspl{PINN} achieve overall higher levels of accuracy, and more balanced error distributions thanks to the evaluation of the collocation points.

{\small
\bibliographystyle{ieee_fullname}
\bibliography{bibliography_long,bibliography,bibliography_custom}
}

\end{document}
