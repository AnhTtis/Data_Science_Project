\documentclass[10pt,twocolumn,letterpaper]{article}

%%%%%%%%% PAPER TYPE  - PLEASE UPDATE FOR FINAL VERSION
% \usepackage{cvpr}              % To produce the CAMERA-READY version
% \usepackage[review]{cvpr}      % To produce the REVIEW version
\usepackage[pagenumbers]{cvpr} % To force page numbers, e.g. for an arXiv version

% Import additional packages in the preamble file, before hyperref
\usepackage{xcolor}
% \usepackage[latin1]{inputenc}
\usepackage[british]{babel}
\usepackage[all]{xy}
\usepackage{amscd}
\usepackage{amssymb}
\usepackage{amsthm}
\usepackage{enumitem}
\usepackage{mathrsfs,bbm}
\usepackage{xcolor,graphicx}
\usepackage{graphics}
\usepackage{soul}
\usepackage{comment}
\usepackage[all]{xy}
\usepackage{amscd}
\usepackage{amssymb,amsmath,latexsym}
\usepackage{amsthm}
\usepackage{enumitem}
\usepackage{mathrsfs,bbm}
\usepackage{dsfont}
\usepackage{tikz-cd}
\usepackage[T1]{fontenc}
\usepackage[utf8]{inputenc}  
 %
%%%%%%%%%%%%%%%%%%%%%%%%%%%%%%%%%%
%pagestyle
%%%%%%%%%%%%%%%%%%%%%%%%%%%%%%%%%%
%\pagestyle{plain}
\textwidth=430pt
\headsep=.7cm
\evensidemargin=15pt
\oddsidemargin=15pt
\leftmargin=0cm
\rightmargin=0cm
%%
%%%%%%%%%%%%%%%%%%%%%%%
\newcommand*\fixitem {\item[]%
  \refstepcounter{enumi}\hskip-\leftmargin\labelenumi\hskip\labelsep}
\newtheorem*{mainthm}{Main Theorem}
\newtheorem*{mainthm1}{Theorem}
\newtheorem*{maincor}{Corollary}
\usepackage[colorlinks=true]{hyperref}
\DeclareMathOperator{\Forall}{\forall}
\DeclareMathOperator{\Exists}{\exists}
\DeclareMathOperator{\ord}{ord}
\newcommand{\phiD}{\varphi_D}
\newcommand{\phiDI}{\varphi_{\mathbf{D}_I}}
\newcommand{\phiDIj}{\varphi_{\mathbf{D}_I (j)}}
\newcommand{\phiH}{\varphi_H}
\newcommand{\phiTimes}{\phiD \otimes \phiH}
\newcommand{\phiTimesDI}{\varphi_{\mathbf{D}_I} \otimes \phiH}
\newcommand{\R}{\mathscr{A}}
\newcommand{\X}{\mathscr{X}}
\newcommand{\Xf}{\mathscr{X}_{(k_0 ,i)}[r_0]}
\newcommand{\Xfr}{\mathscr{X}_{(k_0,i)}[r]}
\newcommand{\hotimes}{\widehat{\otimes}}
\newcommand{\C}{\mathbb{C}_p}
\newcommand{\V}{\mathscr{V}}
\newcommand{\B}{\mathscr{B}}
\newcommand{\dualD}{\mathfrak{D}}
\newcommand{\Dg}{\mathbf{D}}
\newcommand{\DD}{\mathcal{D}^0}
\newcommand{\DDg}{\mathcal{D}}
\newcommand{\DV}{\mathcal{D}}
\newcommand{\W}{\mathscr{W}_N}
\newcommand{\Ao}{\mathbf{A}^\circ}
\newcommand{\AoK}{\mathbf{A}^\circ_{\K}}
\newcommand{\AK}{\mathbf{A}_{/\K}}
\newcommand{\OOO}{\mathscr{A}^\circ}
\newcommand{\K}{\mathcal{K}} 
\newcommand{\OK}{\mathcal{O}_{\K}}
\newcommand{\varprojlog}[1]{\underleftarrow{\log\!^{#1}}}
\newcommand{\T}{\mathscr{T}}
\newcommand{\TT}{\mathbf{T}}
\newcommand{\VV}{\mathbf{V}}
\newcommand{\HH}{\mathcal{H}}
\newcommand{\hh}{\mathcal{H}^+}
\newcommand{\HG}[2]{\mathcal{H}_{#1}(#2)}
\newcommand{\hhl}{\mathcal{H}^{+,[l]}}
\newcommand{\hhj}{\mathcal{H}^{+,[j]}}
\newcommand{\hhjj}{\mathcal{H}^{+,[l,l']}}
\newcommand{\GS}{G_{\mathbb{Q},S}}
\newcommand{\Rf}{R_{(k_0 ,i)}[r_0]}
\newcommand{\Rfr}{R_{(k_0 ,i)}[r]}
\newcommand{\parT}{\langle T\rangle}
\newcommand{\Zf}{Z_{(k_0 ,i)}[r_0]}
\newcommand{\Zfr}{\mathscr{Z}_{(k_0 ,i)}[r]}
\newcommand{\ZFf}{\mathscr{Z}_{(k_0 ,i)}[r_0]}
\newcommand{\ZFfr}{\mathscr{Z}_{(k_0 ,i)}[r]}
\newcommand{\ZF}{\mathscr{Z}}

% include hyperref
\definecolor{cvprblue}{rgb}{0.21,0.49,0.74}
\usepackage[pagebackref,breaklinks,colorlinks,citecolor=cvprblue]{hyperref}


% Add in paper ID
\def\paperID{2555} % *** Enter the Paper ID here
\def\confName{CVPR}
\def\confYear{2024}

\title{CurveCloudNet: Processing Point Clouds with 1D Structure}


\author{
Colton Stearns\\
Stanford University\\
\and
Davis Rempe\\
Stanford University\\
\and
Alex Fu\\
Stanford University\\
\and
Jiateng Liu\\
Zhejiang University\\
\and
Sébastien Mascha\\
Summer Robotics\\
\and
Jeong Joon Park\\
Stanford University\\
\and
Despoina Paschalidou\\
Stanford University\\
\and
Leonidas J. Guibas\\
Stanford University\\
}

%%%%%%%%% Begin Document
\begin{document}
\twocolumn[{%
\renewcommand\twocolumn[1][]{#1}%
\maketitle
\begin{center}
    \centering
    \vspace{-5mm}
    \captionsetup{type=figure}
    \includegraphics[width=1.0\textwidth,height=4.5cm]{content/main/images/banner.jpg}
    \vspace{-5mm}
    \captionof{figure}{Visualizations of the input curve cloud (left) and CurveCloudNet's segmentation prediction (right) for each of our five evaluation datasets. Each evaluation dataset exhibits distinct size, structure, and laser scanning pattern, as shown in \cref{tab:summary-results-and-datasets}}
    % \vspace{-2mm}
    \label{fig:teaser}
\end{center}%
}]

% \maketitle
% \begin{figure}[tp]
    \centering
    \includegraphics[width=\linewidth]{figs/images/teaser_new.pdf}
    \caption{We propose a unified method for four low-level structure segmentation tasks: camouflaged object, forgery, shadow and defocus blur detection~(Top). Our approach relies on a pre-trained frozen transformer backbone that leverages explicit extracted features, \eg, the frozen embedded features and high-frequency components, to prompt knowledge. } 
    \label{fig:teaser}
\end{figure}
%%%%%%%%% ABSTRACT



\begin{abstract}

% Version 1:
% Modern depth sensors such as LiDAR operate by sweeping laser-beams across the scene, resulting in a point cloud with notable 1D curve-like structures. However, most existing point cloud backbones  discard the rich, 1D traversal patterns and rely mainly on Euclidean operations.
% In this work, we present a novel point cloud processing scheme and backbone, \textbf{CurveCloudNet}, that exploits the curve-like structure of modern depth sensors. Concretely, %instead of treating each point independently, 
% we parameterize the point cloud as a collection of polylines and thus establish a local surface-level ordering on the points. 
% We then devise curve-specific operations to process the ``curve clouds:'' (1) a \textit{symmetrical 1D convolution}, 2) a \textit{ball grouping} operation for merging points along curves, and (3) an efficient \textit{1D furthest-point-sampling} algorithm on curves. \textbf{CurveCloudNet} combines these curve operations with existing point-based operations, resulting in an efficient, scalable, and expressive backbone that uses little GPU memory. We evaluate \textbf{CurveCloudNet} on the ShapeNet, Kortx, Audi Driving, and nuScenes datasets, showcasing state-of-the-art segmentation and classification performance across {\em both} object-level and large outdoor scene datasets, the first reported 3D point backbones to do so. 

% Version 2:
% In this work we introduce a new point cloud processing scheme and backbone, called CurveCloudNet, which takes advantage of the curve-like structure inherent in modern depth sensors such as LiDAR. While traditional point cloud backbones discard the rich, 1D laser-traversal patterns and rely on Euclidean operations, CurveCloudNet parameterizes the point cloud as a collection of polylines. This parameterization establishes a local surface-level ordering on the points. Our method applies curve-specific operations to process the ``curve clouds," including symmetrical 1D convolution, ball grouping for merging points along curves, and an efficient 1D furthest-point-sampling algorithm on curves. Combining these curve operations with existing point-based operations results in an efficient, scalable, and expressive backbone that uses little GPU memory. We evaluate CurveCloudNet on several datasets, including ShapeNet, Kortx, Audi Driving, and nuScenes, and report state-of-the-art segmentation and classification performance across \textbf{both} object-level and large outdoor scene datasets, making CurveCloudNet the first 3D point backbone to achieve such results.
% \vspace{-1em}

% Version 3
Modern depth sensors such as LiDAR operate by sweeping laser-beams across the scene, resulting in a point cloud with notable 1D curve-like structures. In this work, we introduce a new point cloud processing scheme and backbone, called \arch, which takes advantage of the curve-like structure inherent to these sensors. While existing backbones discard the rich 1D traversal patterns and rely on generic 3D operations, \arch parameterizes the point cloud as a collection of polylines (dubbed a ``curve cloud”), establishing a local surface-aware ordering on the points. By reasoning along curves, \arch captures lightweight curve-aware priors to efficiently and accurately reason in several \textbf{diverse} 3D environments. 
% , including a symmetric 1D convolution, a ball grouping for merging points along curves, and an efficient 1D farthest point sampling algorithm on curves.
We evaluate \arch on multiple synthetic and real datasets that exhibit distinct 3D size and structure.
%, including: ShapeNet, Audi Driving, nuScenes, Kitti, and a new dataset we name KortX.
We demonstrate that \arch outperforms both point-based and sparse-voxel backbones in various segmentation settings, notably scaling to large scenes better than point-based alternatives while exhibiting improved single-object performance over sparse-voxel alternatives.
In all, \arch is an efficient and accurate backbone that can handle a larger variety of 3D environments than past works. 
%In all, \arch is an off-the-shelf trainable and performant backbone that is ready for the diverse environments faced in open-world applications such as robotics. 

% in various segmentation settings, notably scaling better to large scenes than point-based alternatives while exhibiting better single object performance than sparse-voxel alternatives. 

% CurveCloudNet applies a mix of curve-specific operations and Euclidean point-based operations, resulting in an efficient and accurate backbone that can flexibly reason on \textit{many} different types of 3D scenes. 
% , including a symmetric 1D convolution, a ball grouping for merging points along curves, and an efficient 1D farthest point sampling algorithm on curves.
% By combining these curve operations with existing point-based operations, CurveCloudNet is an efficient and accurate backbone that can flexibly reason on \textit{many} different types of 3D scenes. 
% CurveCloudNet achieves state-of-the-art segmentation performance on the ShapeNet, Kortx, Audi Autonomous Driving, and nuScenes datsets, which include both individual objects and large outdoor scenes captured with various sensor scanning patterns. These evaluations demonstrate that \arch scales to large scenes better than existing point-based backbones while improving object-level semantic segmentation compared to sparse-voxel backbones.
% We evaluate semantic segmentation on four datasets - two common (ShapeNet and nuScenes) and two less common (KortX and Audi Driving). Taken together, these datasets patterns -
% We evaluate semantic segmentation the ShapeNet, Kortx, Audi Driving, and nuScenes datasets. 

% demonstrate that \arch outperforms both point-based and sparse-voxel backbones in various segmentation settings, notably scaling better to large scenes than point-based alternatives while exhibiting better single object performance than sparse-voxel alternatives. 

% Evaluations on ShapeNet, Kortx, Audi Driving, and nuScenes demonstrate that \arch outperforms point-based methods on both individual objects and large-scale scenes, outperforms sparse-voxel backbones on individual objects, and closes the gap between point-based and sparse-voxel backbones on large-scale scenes while requiring significantly less GPU memory.

% CurveCloudNet is evaluated on several datasets that include both individual objects and large
% outdoor scenes captured with various sensor scanning patterns. These evaluations demonstrate that our model can
% outperform point-based and sparse-voxel backbones at both
% object and scene level, achieving state-of-the-art performance on segmentation tasks.
\vspace{-1em}
\end{abstract}


\section{Introduction}
\label{sec:intro}
% \begin{figure}[tp]
    \centering
    \includegraphics[width=\linewidth]{figs/images/teaser_new.pdf}
    \caption{We propose a unified method for four low-level structure segmentation tasks: camouflaged object, forgery, shadow and defocus blur detection~(Top). Our approach relies on a pre-trained frozen transformer backbone that leverages explicit extracted features, \eg, the frozen embedded features and high-frequency components, to prompt knowledge. } 
    \label{fig:teaser}
\end{figure}
\begin{table*}[t]
\centering
    \begin{subtable}[t]{0.45\textwidth}
    \centering
    \scalebox{0.75}{
    \setlength{\tabcolsep}{4pt}
    \begin{tabular}{ l l | c | c c c c c}
    \toprule
      % & & & \multicolumn{4}{c}{\textit{Per-Dataset mIoU} ($\uparrow$)} \\
      \textit{\textbf{Method / mIOU ($\uparrow$)}} & \textit{Type} & \textit{AVG} & \rotbox{0}{KortX} & \rotbox{0}{ShapeNet} & \rotbox{0}{A2D2} & \rotbox{0}{nuScenes} & \rotbox{0}{Kitti}\\
      \midrule
      % PointNet~\cite{Qi2017CVPR} & Point- & 50.8 & 68.5 & 78.9 & 34.2 & 21.5 \\
      PointNet++~\cite{Qi2017NIPS} & Point & 62.2 & 71.0 & 80.1 & 46.5 & 51.1 & --\\
      % RandLANet~\cite{Hu2020CVPR} &  &  \\
      CurveNet~\cite{Xiang2021ICCV} & & 52.9 & 71.5 & \underline{82.8} & 4.4 & -- & -- \\
      PointMLP~\cite{Ma2022ICLR} & & 62.3 & \underline{75.4} & 80.9 & 47.6 & 67.9 & 39.5 \\ 
      PointNext~\cite{Qian2022PointNeXtRP} & & 66.6 & 73.7 & \underline{82.8} & 45.0 & 65.0 & -- \\ \hline
      MinkowskiNet~\cite{Choy2019CVPR} & Voxel & 67.6 & 60.1 & 81.1 & 53.8 & 76.2 & 66.8 \\
      Cylinder3D~\cite{Zhou2020ARXIV} & & 67.6 & 63.5 & 79.6 & 53.0 & 76.1 & 65.9\\
      SphereFormer~\cite{lai2023spherical} & & 70.6 & 69.7 & 79.5 & \textbf{55.1} & \textbf{79.5} & \underline{69.0}\\
      \hline
      CurveCloudNet (ours) & Curve & \textbf{72.7} & \textbf{78.9} & \textbf{83.1} & \underline{54.1} & \underline{78.0} & \textbf{69.5} \\
      \bottomrule
    \end{tabular}
    }
    \label{tab:summary-results}
    \end{subtable}
    %
    %
    \hfill
    %
    %
    \begin{subtable}[h]{0.45\textwidth}
    \centering
    \scalebox{0.75}{
    \setlength{\tabcolsep}{4pt}
    \begin{tabular}{ l | c c c c c }
      \toprule
      \textit{\textbf{Dataset Statistic}} & ShapeNet & KortX & A2D2 & nuScenes & Kitti \\
      \midrule
      % \multicolumn{1}{l}{\textit{Dataset Statistic}} & & & \\
      location & Synthetic & Indoor & Outdoor & Outdoor & Outdoor \\
      scale & $\pm$ 1 & $\pm$ 2m & $+$ 70m  & $\pm$ 50m & $\pm$ 50m\\
      laser pattern & ALL & Random & Grid & Parallel & Parallel \\
      \# points & 2048 & 2048 & $\sim$ 8K & $\sim$ 35K & $\sim$ 100K \\
      % \# classes & 50 & 15 & 13 & 17 \\
      \# train & 12K & 6K & 18K & 28K & 19K\\
      train/val gap & \xmark & \cmark & \xmark & \xmark & \xmark \\
      \bottomrule
    \end{tabular}
    }
    \label{tab:dataset-overview}
    \end{subtable}
    \caption{\textit{Dataset and Performance Overview} \textbf{(Left)} CurveCloudNet achieves the best mIOU on average and is best or second-best for every dataset. Empty entries indicate excessive training time that exceeds 20 days. Validation splits are reported because not all baselines were submitted to test servers. \textbf{(Right)} We evaluate on five segmentation benchmarks that exhibit diverse size, structure, and training settings. Refer to \cref{fig:teaser} for illustrations of the parallel, random, and grid laser patterns.}
    \label{tab:summary-results-and-datasets}
    \vspace{-4mm}
\end{table*}


Over the past decade, the computer vision community has proposed many backbones for processing 3D point clouds for fundamental tasks such as semantic segmentation \cite{Qi2017CVPR, Qi2017NIPS, Wang2019SIGGRAPHb, Thomas2019ICCV, Hu2020CVPR}
and object detection \cite{Wu2022GeosciRemoteSens, Wang2021NeurIPS, Wu2022ARXIV, Wu2022CVPR, Yang2022ECCV}.
%
Existing 3D backbones can be generally characterized as point-based or discretization-based.
%
Backbones that directly operate on 3D points \cite{Qi2017NIPS, Su2018CVPR, Hua2018CVPR, Wang2018CVPRb, Xu2018ECCV, Esteves2018ECCV,
Wu2019CVPR, Thomas2019ICCV} typically exchange and aggregate point features in Euclidean space, and have shown success for individual objects or relatively small indoor scenes. These methods, however, do not scale well to large scenes (\eg in outdoor settings) due to inefficiencies in processing large unstructured point sets.
%
On the other hand, popular discretization approaches such as sparse voxel methods \cite{Wu2018ICRA, Su2018CVPR, Graham2018CVPR, Choy2019CVPR,
Liu2019NeurIPS, Zhou2020ARXIV, Hu2020CVPR, Zhang2022ARXIV} rely on efficient sparse data structures that scale better to large scenes. However, for small or irregularly-distributed point sets, they often incur discretization errors. %, especially when the scene is not axis-aligned structure.

% While for the past decade this trend was fine, 
In recent years, this trade off between point and voxel backbones has been less explored due to the distinct environments in most 3D applications - autonomous vehicles do not leave roads, manufacturing robots do not leave warehouses, and quality-assurance systems do not look beyond a tabletop. However, as the community moves to dynamic and unregulated settings such as open-world robotics (\eg embodied agents), it is essential to have architectures that consistently perform well in diverse settings.

To this end, we present a novel point cloud processing scheme that achieves both performance and flexibility across diverse 3D environments. We achieve this by tailoring 
%
%Our key idea is to tailor 
our approach to the popular family of laser-scanning 3D sensors (such as LiDARs), which gather 3D measurements by sweeping laser-beams across the scene.
%
While previous works ignore the innate curve-like structures of the scanner outputs, we parameterize the point cloud as a collection of polylines, which we refer to as a ``curve cloud”. Our formulation establishes a local structure on the points, allowing for efficient and cache-local communication between points along a curve. This enables scaling to large scenes without incurring discretization errors and/or computational overhead. Furthermore, we hypothesize that the local curve ordering injects a lightweight and flexible surface-aware prior into the network (see \cref{subsec:why-curves}).

% To this end, we present a novel point cloud processing scheme that achieves both performance and flexibility across many 3D environments captured with a laser-scanning 3D sensor (such as LiDAR).
% Laser scanning 3D sensors operate by sweeping laser-beams across the scene. As the laser traverses object surfaces, it returns dense measurements along the scanning direction, resulting in a 3D point cloud with explicit curve-like structures (see \cref{fig:overview}).
% While previous works ignore the innate curve-like structures of the scanner outputs and treat points independently, we parameterize the point cloud as a collection of polylines, which we refer to as a ``curve cloud”. Our formulation establishes a local structure on the points, allowing for efficient and cache-local communication between points along a curve. This enables scaling to large scenes without incurring discretization errors and/or computational overhead. Furthermore, we hypothesize that the local curve ordering injects a lightweight and flexible surface-aware prior into the network (see \cref{subsec:why-curves}).

We propose a new backbone, \arch, that applies 1D operations along curves and combines curve operations with state-of-the-art point-based operations \cite{Qian2022PointNeXtRP, Ma2022ICLR, Wang2019SIGGRAPHb, Qi2017NIPS}. \arch uses curve operations at higher resolutions when there is clearer curve structure and uses point operations at downsampled resolutions. 
Put together, \arch is an efficient, scalable, and accurate backbone that can outperform segmentation and classification pipelines in a variety of settings (see \cref{tab:summary-results-and-datasets}a).

% : this includes indoor, outdoor, object-centric, scene-centric, sparsely scanned, and densely scanned scenes, where each setting exhibits unique laser scanner patterns (see \todo{Supp. tab 3} for example scanning patterns).

% achieves the best of both points and voxels. uses little GPU memory (see \cref{fig:teaser}). and combines 1D curve reasoning with established point-based operations to achieve the best of both points and voxels. That is, curve reasoning makes CurveCloudNet efficient, scalable, and accurate, and
% Our proposed \arch combines 1D curve reasoning with established point-based operations to achieve the best of both points and voxels. That is, curve reasoning makes our backbone scalable, accurate, and efficient for both objects and large-scale scenes and across many scanning patterns.

%
% \todo{is this next sentence too negative?}
% Intuitively, \arch echos a story of strategic specialization - by leveraging the intrinsic curve structure of laser-scanned data, we can improve perception for this common medium. 


% The essence of 1D curve reasoning is a novel adaptation of three standard point cloud operations for 1D curve structures: (1) a \textit{symmetrical 1D convolution} that operates along curves, (2) a \textit{ball grouping} similar to PointNet++ \cite{Qi2017NIPS} that groups points along curves, and (3) an efficient \textit{1D farthest-point-sampling} algorithm on curves.
% Carefully integrated together with state-of-the-art point-based operations \cite{Qian2022PointNeXtRP, Ma2022ICLR, Wang2019SIGGRAPHb, Qi2017NIPS}, CurveCloudNet is a versatile, efficient, and high-performing backbone that uses little GPU memory (see \cref{fig:teaser}).

We evaluate CurveCloudNet on a variety of object-level and outdoor scene-level datasets that exhibit distinct 3D size, structure, and unique laser scanning patterns (see \cref{tab:summary-results-and-datasets}b and \cref{fig:teaser}): this includes indoor, outdoor, object-centric, scene-centric, sparsely scanned, and densely scanned scenes. We evaluate \arch on the object part segmentation task using the ShapeNet \cite{Chang2015ARXIV, Yi2016ToG} dataset along with a new real-world object-level dataset captured with the \kortx scanning system \cite{summer-robotics}. For the outdoor semantic segmentation task, we use the nuScenes~\cite{Caesar2020CVPR}, Audi Autonomous Driving (A2D2)~\cite{Geyer2020ARXIV}, and Semantic Kitti ~\cite{behley2019iccv, Geiger2012CVPR} datasets.
%The A2D2 dataset offers a unique sensor configuration resulting in a grid-like scanning pattern, which is distinct from previous LiDAR datasets \cite{Geiger2012CVPR, Caesar2020CVPR, Chang2019CVPR, Sun2020CVPR}.
Supplementary experiments on object classification demonstrate flexibility to other perception tasks. Our evaluations demonstrate that using curve structures leads to improved or competitive performance on \textit{all} experiments, with the best performance on average (see \cref{tab:summary-results-and-datasets}a).
%We believe our diverse set of evaluations is more indicative of the unpredictable 3D environments faced in open-world robotics.
% As 3D applications such as open-world robotics become more dynamic and diverse, we beleive it is important to have 3D architectures exhibit flexibility and great ``on-average" performance across various environments.
% To the best of our knowledge, our work is the first to experimentally showcase accurate and efficient predictions on \emph{both} object-level and outdoor datasets. 


% contributions
In summary, we make the following \textbf{contributions}: 
\textbf{(1)} we propose operating on laser-scanned point cloud data using a \emph{curve cloud} representation, \textbf{(2)} we design efficient operations that run on polyline curves, \textbf{(3)} we design a novel backbone, CurveCloudNet, that strategically combines both curve and point operations, and \textbf{(4)} we show accurate and efficient segmentation results on real-world data captured for both objects and large-scale scenes in multiple environments and with various scanning patterns.

\begin{figure*}
    \centering
    \includegraphics[width= 0.92\textwidth]{content/main/images/overview.png}
    \caption{\textit{Overview of Curve Cloud Reasoning.} Starting from laser-scanned input data, we \textcircled{1} link points into polylines to \textcircled{2} parameterize the point cloud as a curve cloud (see \cref{sec:prelims}). We develop operations for learned architectures to specifically exploit the curve structure, including \textcircled{3} 1D farthest-point-sampling along a curve, \textcircled{4} curve grouping, and \textcircled{5} symmetric curve convolutions (see \cref{sec:curve-ops}).}
    \label{fig:overview}
    \vspace{-3mm}
\end{figure*}
\section{Related Work}

Existing point cloud methods can be roughly characterized as
point-based and voxel-based approaches. As our work addresses trade-offs between them, we discuss related works from each category.

\paragraph{Point-Based Networks}
% \subsection{Point-Based Networks}
% \boldparagraph{Point-based Networks}%
Prior work has extensively studied backbones that map a 3D point cloud to a high-dimensional feature space used for downstream applications, such as 3D reconstruction, shape classification, part segmentation, semantic segmentation, and more \cite{Fan2017CVPR, Qi2017CVPR,
Qi2017NIPS, Wang2019SIGGRAPHb, Thomas2019ICCV}. 
%
PointNet \cite{Qi2017CVPR} was a seminal work that combined a series of MLPs with a max pooling layer to learn point-wise features. Following PointNet, several works proposed to aggregate local neighborhood information using hierarchical grouping at multiple geometric scales \cite{Qi2017NIPS, Li2018CVPR, Qian2022PointNeXtRP}. Recently, Ma \etal \cite{Ma2022ICLR} introduced a compelling MLP-based architecture that combines residual multi-scale reasoning and affine transformations. 
% 
Nevertheless, most hierarchical and MLP point networks are inefficient on large-scale point clouds, and although several backbones \cite{Hu2020CVPR,Zhang2022CVPR,Yang20203DSSDP3} have addressed this, they trade off scalability with task-specific frameworks or lower accuracy. 
In contrast, CurveCloudNet uses an efficient curve cloud representation to achieve
superior performance on both objects and large-scale scenes.

Another line of research introduced \emph{kernel-based
convolutions} for learning per-point local features
\cite{Su2018CVPR, Hua2018CVPR, Wang2018CVPRb, Xu2018ECCV, Esteves2018ECCV,
Wu2019CVPR, Lei2019CVPR, Komarichev2019CVPR, Lan2019CVPR, Thomas2019ICCV, Wiersma2022SIGGRAPH}. Kernels are defined using a family of polynomial
functions~\cite{Xu2018ECCV} or can be estimated using MLPs~\cite{Wang2018CVPRb,
Liu2019CVPR}. Likewise, \cite{Atzmon2018SIGGRAPH, Thomas2019ICCV, Wu2019CVPR, Xu2021CVPR,
Boulch2020ACCV} defined the kernel weights directly using the local 3D point coordinates.
% More sophisticated models such as Spherical CNN~\cite{Esteves2018ECCV} addressed 3D rotation equivariance by implementing convolutions in the spherical harmonic domain.
% InterpConv~\cite{Mao2019ICCV} utilized the point coordinates to interpolates
% point features to the neighboring kernel weights and
% DeltaConv~\cite{Wiersma2022SIGGRAPH} proposed to construct anisotropic convolutions by learning combinations of differential operators.
More related to our work is CurveNet~\cite{Xiang2021ICCV}, which proposed to perform random walks on point clouds and aggregate features along these ``curves" to capture local geometric details. In constrast, our work defines efficient point cloud operations for existing 1D curve structures.

An alternative line of research proposed to construct a
graph from the input point cloud and apply \emph{graph-based convolutions}
\cite{Te2018SIGGRAPH, Liu2019ICCVb, Chen2020CVPRa, Eliasof2020NeurIPS} to capture local geometric structure. For example,
\cite{Shen2018CVPR} introduced a graph-based network that replaced
convolutions with correlations computed between points and their $k$-nearest neighbors. Similarly, \cite{Verma2018CVPR,
Wang2019SIGGRAPHb} proposed to construct a local neighborhood graph and apply convolutional operations on the edges connecting neighboring pairs of points.
% DiffGCN~\cite{Eliasof2020NeurIPS} performed message passing based on approximate surface gradients and an approximate Laplacian matrix, whereas \cite{Wang2018ECCVc} investigated using local spectral graph convolution.
Recent works \cite{Xie2018CVPR, Liu2019AAI, Yang2019CVPR, Zhang2019CVPR, Hehe2021CVPR, Zhao2021ICCV, Ishan2021ICCV} have also explored
applying \emph{attention-based aggregation} using
transformer architectures with self-attention \cite{Vaswani2017NIPS}.  Unlike previous graph-based and attention-based networks, CurveCloudNet reasons over local 1D ``curve" neighborhoods.
% Applying neural networks to process 3D point cloud data is an extremely well
% studied problem.
% Beginning with PointNet \cite{pointnet}, a series of follow-up
% works improved learnable point-cloud operations for capturing both local and
% global structure \cite{pointnet2, dgcnn, kpconv, votenet, pa_conv, fa_conv}.
% PointNet++ \cite{pointnet2} applied PointNet hierarchically to extract features
% at many geometric scales.
% DGCNN \cite{dgcnn} constructs a nearest-neighbors
% graph and performs message passing and aggregation between points. KpConv
% \cite{kpconv} develops a convolutional kernel that is defined on a set of
% deformable 3D key points, and other methods extend this idea of point cloud
% convolutions \cite{pa_conv, fa_conv, 3detr}.
% In recent years, researchers have
% also applied self-attention and transformers to point clouds
% \cite{pointtransformer, pct, 3detr, point4dtransformer}.
% Additionally, some
% recent methods attempt to incorporate intrinsic priors into extrinsic point
% cloud backbones \cite{curvenet, diffgcn, deltaconv}.
% CurveNet \cite{curvenet}
% performs random walks on point clouds and aggregates features along these
% ``curves". DeltaConv defines both per-point scalar and gradient features and
% applies feature updates inspired by discrete exterior calculus
% \cite{deltaconv}.
% DiffGCN performs graph message passing based on approximate
% surface gradients and and approximate Laplacian matrix \cite{diffgcn}.
%We differ from previous point cloud methods in that we are interested in the 1D sampling pattern that arises from many 3D scanners.

\paragraph{Voxel-Based Networks}
% \subsection{Voxel-Based Networks}
% \boldparagraph{Voxel-based Networks}%
Although point-based backbones can successfully process individual objects or
small indoor scenes, they struggle to scale to large point clouds due to inefficiency in processing large unstructured point sets. To address this,
several works \cite{Su2018CVPR, Graham2018CVPR, Choy2019CVPR,
Liu2019NeurIPS, Zhang2020CVPR, Zhou2020ARXIV, Zhang2022ARXIV, Yan2018Sensors, Lang2018CVPR, Liu2021TPAMI, Xu2021ICCV, Hou2022CVPR} proposed to convert a point cloud into a 3D
voxel grid and use this volumetric representation. Early works converted a point cloud into a dense voxel grid and applied dense 3D convolutions \cite{Maturana2015IROS, Qi2016VolumetricAMCVPR}, however the cubic size of the dense grid proved to be computationally prohibitive. PVCNN~\cite{Liu2019NeurIPS}
was a seminal work in combining point operations with low-resolution dense voxel convolutions to efficiently
process smaller-scale point clouds. % Nevertheless, their dense voxel representation still struggled to scale to large scenes.

% sparse voxel architectures
To scale to large scenes, several works \cite{Choy2019CVPR, Lang2018CVPR, Zhou2020ARXIV, Liu2021TPAMI, Hou2022CVPR} employed the sparse-voxel data structure from \cite{Graham2018CVPR}. 
% MinkowskiNet~\cite{Choy2019CVPR} extended sparse voxel convolutions to the time domain, showing efficient and accurate reasoning on 4D scenes. 
PVNAS~\cite{Liu2021TPAMI} incorporated a network architecture search, demonstrating the importance of the architecture channels, network depth, kernel sizes, and training schedule. PVT~\cite{Zhang2022ARXIV} introduced a sparse attention module to efficiently process per-voxel local features using a transformer encoder~\cite{Vaswani2017NIPS}.
% Notably, although our model only processes 3D points, we showcase that it can be efficiently employed on larger point clouds.
%
% better descritization
Other methods seek a better voxel discretization of point clouds captured with \lidar scans. For example, PolarNet~\cite{Zhang2020CVPR}
proposed to partition input points using grid cells defined in a polar
coordinate system, while Cylinder3D~\cite{Zhou2020ARXIV} employed a cylindrical
partitioning scheme based on a cylindrical coordinate system. In an alternative line of research, many methods \cite{Wu2019ICRA, Xu2020ECCV, Wu2018ICRA, Wu2019CVPR}
employed spherical or bird's-eye view projections to represent point clouds as images that are passed to a convolutional neural network. 

Unlike these works, our model directly operates on points and curves, scaling to larger scenes without 2D or 3D convolutions. Furthermore, our model does not rely on the point cloud to exhibit cylindrical or planar properties. 
% individual objects, where the point clouds do not necessarily follow
% cylindrical patterns (see \cref{fig:teaser}).

% Unlike these works, our model directly operates on points and curves, and hence does not
% rely on either 2D or 3D convolutions. Notably, although our model only processes 3D points,
% we showcase that it can be efficiently employed on larger point clouds.

% 
% to take advantage of the 3D topology of point clouds captured with LIDAR scans and
% instead of converting the input to a voxel grid, they rely on on cylindrical
% partitioning schemes. For example, PolarNet~\cite{Zhang2020CVPR}
% proposed to partition input points using grid cells, defined in a polar
% coordinate system, while Cylinder3D~\cite{Zhou2020ARXIV} employed a cylinder
% partitioning scheme based on a cylinder coordinate system. Although, both
% \cite{Zhang2020CVPR, Zhou2020ARXIV} can be successfully applied to point
% clouds captured with LIDAR scans, they perform poorly for the case of
% individual objects, where the point clouds do not necessarily follow
% cylindrical patterns (see \cref{fig:teaser}). In an alternative line of research, \cite{Wu2018ICRA, Wu2019CVPR}
% employ spherical projections to represent point clouds as images that are then
% passed to a convolutional neural network to extract per-point features. Unlike
% these works, our model directly operates on points and curves, and hence does not
% rely on either 2D or 3D convolutions. Notably, although our model only processes 3D points,
% we showcase that it can be efficiently employed on larger point clouds.

% Sparse voxel backbones convert a point cloud into a discrete voxel grid and
% apply a series of sparse 3D convolutions or 2D birds-eye-view convolutions
% \cite{octnet, second, splatnet, mincowskinet}. The sparse-voxel backbone is a
% popular alternative to point cloud backbones due to efficient grid
% representations and the ability to scale to large scenes \cite{randla-net,
% performance-lidar}. MinkowskiNet \cite{mincowskinet} is a seminal work in 3D
% and 4D sparse convolutions, offering a highly efficient and expressive
% implementation of the sparse-voxel infrastructure. Following suit, many works
% extend MinkowskiNet \cite{cylinder3d, other-kitti-one, polarnet, spvcnn}.
% Additionally, newer works fuse spare voxel operations and point operations to
% improve overall expressivity \cite{pvcnn, kitti-iclr1, kitti-iclr1-followup,
% pointvoxeltransformer}.

% \subsection{LiDAR Range View Representation}
% In LiDAR processing, it is not uncommon to use the a spherical projection into a 2D plane, called the range view \cite{rangenet++, rangenet, rpvnet}. The range view offers a compact 2D approximation of the 3D point cloud structure, and allows the use of standard 2D image processing techniques. 

% In classic ``parallel-sweeping" LiDAR, the spherical projection will result in a reformatting similar to our 1D polylines. However, our 1D polylines are far more general as they are not contrained to the parallel direction (necessary for Audi dataset), they operate directly in 3D space (integrate more easily into existing backbones), and the range view additionally uses a ``vertical" relationship that our polylines do not.


%\boldparagraph{Sensor-Specific Networks}%
%Many previous works modify 3D backbones to better accommodate sensor sampling
%patterns. A number of works address the outputs of long-range LiDARs. To
%explain, most long-range LiDAR sensors vertically stack 32 to 128 laser beams
%and rapidly ``sweep" the beams in a circular fashion. Many works offer
%long-range LiDAR improvements by polar and cylindrical geometric priors.
%PolarNet \cite{polarnet} introduces polar coordinates in a sparse-voxel
%representation. Range-view methods \cite{rangenet++, rangenet, rpvnet,
%solomon_pillar} perform a 2D polar projection of LiDAR measurements the
%range-view. Cylinder3D \cite{cylinder3d} performs cylindrical convolutions on a
%sparse-voxel representation. 

% In addition to sensor-specific architectures, many works modify 3D backbones for specific tasks. In mesh correspondence and protein classification, recent works apply 2D convolutions along approximated surfaces to better understand intrinsic geometry \cite{mesh_cnn, diffusionnet, others}. Siren \cite{siren} includes sinusoidal activation functions to better capture high-frequency information, while rotation-equivalent backbones modify the neural operators to preserve rotation information. Finally, in medical imaging \cite{something}, \textcolor{yellow}{something else here}.

%In contrast to previous methods, CurveCloudNet targets a more general 1D sampling pattern that includes but is not limited to the parallel ``sweeping" pattern of long-range LiDAR sensors.
% formulate the point cloud as a set of local 3D polylines. In addition to being more fine-grained, our 3D polylines are not tailored for a specific type of sensor motion (i.e. parallel sweeps). 

\input{content/main/text/04_method_v2}
We begin by briefly comparing the performance of the three predictors for $\hat{\bm{x}}_i$ (FE, RK4, NN) before testing \cref{algo:simulator}.

\subsection{Predictor comparison}\label{subsec:results_predictors}
The approximator $\hat{\bm{x}}_i$ should have two properties: being fast and being accurate for large time-steps $\Delta t$. \Cref{fig:predictor_characteristics} shows the two properties. The left panel displays the error in the differential variable $\Delta \omega$ across 200 points of random initial conditions $\bm{x}_{0}$ and voltage parameterizations $\yparams{}_i$. In terms of accuracy, the \gls{NN} performs well and only for smaller time-steps ($\Delta t < 0.05 \si{\second}$), the RK4 approximation becomes more accurate. The RK-schemes exhibit the expected dependency of the time-step - the local truncation error should follow $\mathcal{O}(\Delta t)$ and $\mathcal{O}(\Delta t^4)$ for FE and RK4. The error of the \gls{NN} in contrast is near independent of $\Delta t$. At the same time, the \gls{NN} is the fastest approximator. While the FE approximator is similarly fast, its poor accuracy makes it undesirable and while more accurate, the RK4 scheme has the drawback of comparably long run-time.     

\begin{figure}[!th]
    \centering
    \includegraphics[width=0.95\linewidth]{figures_pdf/predictor_error.pdf}
    \caption{Predictor characteristics: (left) prediction error of $\Delta \omega$ for 200 predictions with random $\bm{x}_0$ and \yparams{}, (right) run-time per point. These results show that \gls{NN} constitute an accurate and fast predictor.}
    \label{fig:predictor_characteristics}
\end{figure}

\subsection{Simulator results}\label{subsec:results_simulators}

We now focus on the performance of the simulator in \cref{algo:simulator}. \Cref{fig:ieee9_prediction} displays the prediction for $\Delta t = \SI{0.05}{\second}$ using \gls{NN}-based approximators $\hat{\bm{x}}_i^{NN}$. 
\begin{figure}[!ht]
    \centering
    \includegraphics[width=0.95\linewidth]{figures_pdf/simulation_results.pdf}
    \caption{Prediction of a state trajectory ($\Delta \omega_i$) and an algebraic variable $V_i$ for time-step size $\Delta t = \SI{0.05}{\second}$ with a \gls{NN}-based simulator. The predictions (dashed lines) coincide with the ground truth (gray lines).}
    \label{fig:ieee9_prediction}
\end{figure}
The results correspond to the first row in \cref{tbl:simulator_results} where we report the maximum absolute errors of $V$ and $\Delta \omega$ along the trajectory and the run-time. \Cref{tbl:simulator_results} shows further results for the \gls{RK4}-based simulator, for time-steps of $\Delta t = \SI{0.1}{\second}$ and $\Delta t = \SI{0.15}{\second}$ and for collocation points at $\bm{T}=[0.3, 0.7]\Delta t$ and $\bm{T}=[0.1, 0.3, 0.5, 0.7, 0.9]\Delta t$, i.e., $s=2$ and $s=5$. The \gls{NN}-based simulator is consistently faster and more accurate than the \gls{RK}-based simulator, except for the case of $\Delta t = \SI{0.05}{\second}$. These results confirm the predictor characteristics observed in \cref{subsec:results_predictors}. The simulation run-time scales approximately inversely proportional with $\Delta t$. Deviations can arise due to varying numbers of iteration in \cref{algo:parameter_update}, however, in the reported cases, we observe usually 5-7 iterations. Increasing the number of collocation points $s$ results in a small increase in run-time in all cases. In terms of accuracy, we observe that more collocation points can lead to better accuracy, when the overall solution quality is good. However, for too large time-steps, here $\Delta t = \SI{0.15}{\second}$, the effect might reverse. The choice of the location of the collocation points, i.e., $\bm{T}$, also matters.%\bz{Can we bold the best numbers in a box? This would help to highlight what people should look at. There are a lot of numbers.}
\begin{table}[!ht]
\renewcommand{\arraystretch}{1.2}
\caption{Comparison of simulators with different predictor schemes}
\label{tbl:simulator_results}
\centering
\begin{tabular}{ccc|ccc}
\toprule
$\Delta t$ & Predictor & s & run-time & $\max |V_i - \hat{V}_i|$ & $\max | \omega_i - \hat{\omega}_i|$ \\
$[\si{\second}]$ & & & [\si{\second}] & $\times 10^{-2} [\si{\pu}]$ & $\times 10^{-3} [\si{\pu}]$ \\ \midrule
\multirow{2}{*}{$0.05$} & NN & $5$ & $\bm{1.85}$ & $0.82$ & $0.35$\\
 & RK4 & $5$ & $3.88$ & $\bm{0.31}$ & $\bm{0.29}$\\ \midrule
\multirow{4}{*}{$0.10$} & \multirow{2}{*}{NN} & $2$ & $\bm{0.98}$ & $2.71$ & $1.29$ \\
& & $5$ & $1.05$ & $\bm{1.32}$ & $\bm{0.62}$ \\
 & \multirow{2}{*}{RK4} & $2$ & $2.21$ & $5.07$ & $2.40$ \\
& & $5$ & $2.27$ & $3.88$ & $2.01$ \\ \midrule
\multirow{4}{*}{$0.15$} & \multirow{2}{*}{NN} & $2$ & $\bm{0.60}$ & $\bm{6.28}$ & $2.93$ \\
& & $5$ & $0.77$ & $6.36$ & $\bm{2.90}$ \\
 & \multirow{2}{*}{RK4} & $2$ & $1.19$ & $11.8$ & $5.06$ \\
& & $5$ & $1.46$ & $19.0$ & $7.75$ \\
% $\approx$ 0.020 & BDF & - & 0.83 & 0.07 & 0.26\\
% $\approx$ 0.045 & BDF & - & 0.67 & 7.73 & 3.36\\
\bottomrule
\end{tabular}%
\end{table}



%The last two rows stem from the \gls{DAE}-solver in Assimulo based on a variable order backward differentiation formula (BDF) with different tolerance levels. We observe that the required time-step size is significantly smaller than for the proposed simulator. The resulting run-times are comparable, however, neither method was optimized for run-tme in this study. \bz{I would delete the the last two rows and skip this discussion. Not sure how this helps the paper.}
This paper presented a comprehensive analysis of the use of \acrfull{PINN} for power system dynamic simulations. We show that \glspl{PINN} (i) are 10 to 1'000 times faster than conventional solvers, (ii) do not face issues of numerical instability unlike conventional solvers, and, (iii) achieve a decoupling between the power system size and the required solution time. However, \glspl{PINN} are less flexible (i.e. they do not easily handle parameter changes), and require an up-front training cost. Overall, this makes \gls{PINN}-based solutions well-suited for repetitive tasks as well as task where run-time speed is crucial, such as for screening.

Besides the comparison between conventional and \gls{NN}-based methods, this paper conducts a deeper analysis on the parameters that affect the performance of the \gls{NN} solutions. In that respect, we introduce a new \gls{NN} regularisation, called dtNN, as a intermediate step between \glspl{NN} and \glspl{PINN}. We show that \glspl{PINN} achieve overall higher levels of accuracy, and more balanced error distributions thanks to the evaluation of the collocation points.

% ==============================================
% ========== FOR ADDING IN SUPP! ===============
% \title{\bf \Large Supplementary Material for ``CurveCloudNet: Processing Point Clouds with 1D Structure"}
% \maketitle

% \renewcommand{\thesection}{A}
% \renewcommand{\thetable}{A\arabic{table}}
% \renewcommand{\thefigure}{A\arabic{figure}}


\renewcommand{\thesection}{S~\arabic{section}}
% \renewcommand\cftsecnumwidth{2.5em}
% \renewcommand\cftsubsecnumwidth{3em}
\renewcommand{\thetable}{S\arabic{table}}
\renewcommand{\thefigure}{S\arabic{figure}}
\setcounter{table}{0}
\setcounter{figure}{0}
\setcounter{section}{0}


\twocolumn
\newpage
\section{Overview}
In this document, we provide additional method details, dataset details, implementation details, experimental analysis, and qualitative results. In \cref{additional-method}, we concretely outline how we convert a point cloud into a curve cloud and how we implement our 1D farthest point sampling algorithm. In \cref{additional-dataset}, we provide a detailed overview of the Kortx software system and dataset, our ShapeNet simulator, and the A2D2 dataset. In \cref{additional-implementation}, we discuss implementation details of
\arch not covered in the main paper. Finally, in \cref{additional-experiments}, we report results covering GPU memory analysis, object classification, ShapeNet segmentation, and the nuScenes test split.


\section{Additional Method Details} \label{additional-method}

\subsection{Constructing Curve Clouds}
\paragraph{Curve Cloud Conversion}
\begin{figure}[b!]
    \footnotesize
    \centering
    \vspace{-1mm}
    \hrulefill
    \vspace{-1mm}
    % \begin{minted}[fontsize={\fontsize{7.0}{8.0}\selectfont}]{python}
    \begin{lstlisting}[language=Python]

'''
Inputs: P, T, B, delta
    P: array of size (N, 3) with xyz coordinates
    T: array of size (N,) with timesteps
    B: array of size (N,) with beam IDs
Outputs: curves
    curves: list of arrays, array j is size (N_j, 3)
'''
curves = []
for b in unique(B):
    # filter to a single laser beam's measurments
    beam_P, beam_T = P[beams==b], T[beams==b]

    # order points by laser's traversal
    sequential_ordering = argsort(beam_T)
    beam_P = beam_P[sequential_ordering]

    # split laser's traversal into cont. curves
    edge_lens = norm(beam_P[1:] - beam_P[:-1])
    split_locations = edge_lens > delta

    # convert into polylines
    beam_C = split_seq(beam_P, split_locations)
    curves += beam_C
    \end{lstlisting}
    % \end{minted}
    \hrulefill
    % \vspace{2mm}
    \caption{\textit{Point to Curve Cloud Conversion.} Algorithm (in Python) to convert an input point cloud into a set of polylines.}
    \label{fig:curve-cloud-conversion}
\end{figure}

% Outputs: C
%     C: list<array> containing (N', 3) arrays;
%        each array is a variable-length polyline

\paragraph{Constructing Curve Cloud}
We refer the reader to Sec. 3.1 of the main paper for an overview of constructing curve clouds.
As input, we assume that a laser-based 3D sensor outputs a point cloud $P = \{p_1, ..., p_N\}$ where $p_i = [x_i, y_i, z_i] \in \mathbb{R}^3$, an acquisition timestamp $t_i \in \mathbb{R}$ for each point, and an integer laser-beam ID $b_i \in [1, B]$ for each point. We wish to convert the input into a curve cloud $C = \{c_1, ..., c_M\}$, where a curve $c = [p_{i}, ..., p_{i+K}]$ is defined as a sequence of $K$ points where consecutive point pairs are connected by a line segment, \ie, a \textit{polyline}. As outlined in \cref{fig:curve-cloud-conversion}, we first group points by their laser beam ID and sort points based on their acquisition timesteps, resulting in an ordered sequence of points that reflects a single beam's traversal through the scene. Next, for each sequence, we compute the distances between pairs of consecutive points (denoted as polyline ``edge lengths"). Finally, we split the sequence whenever an edge length is greater than a threshold $\delta$, resulting in many variable-length polyline ``curves".  In practice, we parallelize the conversion across all points, and on the large-scale nuScenes dataset, the algorithm runs at 1500Hz.

We select a threshold $\delta$ that reflects the sensor specifications and scanning environment. In particular, the threshold is conservatively set to approximately $10\times$ the \textit{median distance} between consecutively scanned points one meter away from the sensor. On the A2D2 dataset~\cite{Geyer2020ARXIV}, we set $\delta = [0.1, 0.17, 0.1, 0.12, 0.1]$ for the five LiDARs, and on the nuScenes dataset~\cite{Caesar2020CVPR} and KITTI dataset~\cite{behley2019iccv, Geiger2012CVPR} we set $\delta = 0.08$. Additionally, on the A2D2, nuScenes, and KITTI datasets, we scale $\delta$ proportional to the square root of the distance from the sensor, since point samples becomes sparser at greater distance. On the object-level ShapeNet dataset~\cite{Chang2015}, we set $\delta = 0.01$. Experimentally, we observed that \arch is flexible across different $\delta$ values.

\paragraph{Kortx Curve Representation}
The Kortx vision system directly generates and operates on 3D curves sampled from a triangulated system of event-based sensors and laser scanners.
As the detected laser reflection traverses the scene, it produces a frameless 4D data stream that enables low latency, low processing requirements, and high angular resolution. 3D curves are an intrinsic component of the Kortx perception system, and the system directly outputs a curve cloud without the need for additional data processing. 

\subsection{Additional Details on Curve Operations}
\paragraph{1D Farthest Point Sampling}
We refer the reader to Sec. 3.2.2 of the main paper for an overview of our 1D farthest point sampling (FPS) algorithm. The goal of this algorithm is to \textit{efficiently} sample a subset of points on each curve such that consecutive points will be approximately $\epsilon$ apart along the downsampled curve. Concretely, for a curve $c$ with $K$ points, $c = [p_{i}, ..., p_{i+K}]$, we first compute the $K{-}1$ ``edge lengths", $e = [d_{i}, ..., d_{i+K-1}]$, where $d_{i}$ is the distance between consecutive points $(p_{i}, p_{i+1})$. Next, we estimate the geodesic distance along the curve via a cumulative sum operation on the edge lengths: $g = \texttt{CUMSUM}(e)$. Then, we divide the geodesic distances by our desired spacing, $\epsilon$, and take the floor, resulting in $\epsilon$-spaced intervals $I = \texttt{FLOOR}(g / \epsilon)$. We output the first point in each interval, resulting in $L$ $\epsilon$-spaced subsampled points $\{q_1, ..., q_{L}\}$.

For each curve, the computational complexity of the 1D FPS algorithm is $O(K)$. Extending this to all curves, the computational complexity is $O(N)$. When parallelized on a GPU, for each curve, the 1D FPS algorithm has a has \textit{parallel} complexity of $O(\log K)$, where the $\texttt{CUMSUM}$ operation is the parallelization bottleneck (see \textbf{prefix sum} algorithms for more information~\cite{Blelloch1990PrefixSA}). The algorithm trivially parallelizes across curves, leading to a total parallel complexity of $O(\log K)$. In contrast, Euclidean farthest point sampling has a comptational complexity of $O(N^2)$ or $O(N log^2 N)$ (depending on whether a KD-tree is used), and a parallel complexity of $O(L)$. When $L$ is large (\ie we are subsampling a large number of points), Euclidean FPS is a significant performance bottleneck.

\subsection{Additional Details on Point Operations}

\paragraph{Set Abstraction (SA)}
\arch uses a series of set abstraction layers from PointNet++~\cite{Qi2017NIPS}, and we follow previous works to improve the set abstraction layer. First, we perform relative position normalization~\cite{Qian2022PointNeXtRP} -- given a centroid point $p_i$ with local neighborhood points $\mathcal{N}_i$, we center the neighborhood about $p_i$ \textbf{and} we divide by $r$ to normalize the relative positions. Additionally, we opt for the attentive pooling from RandLANet~\cite{Hu2020CVPR} instead of max pooling. 
% Finally, on large datasets nuScenes and KITTI, we find farthest point sampling to be slow (even after our efficient curve downsampling), and we opt for voxel downsampling   

\paragraph{Graph Convolution}
\arch uses a series of graph convolutions, which are modeled after the edge convolution from DGCNN~\cite{Wang2019SIGGRAPHb}. Unlike DGCNN however, we construct the K-Nearest-Neighbor graph based on 3D point distances instead of feature distances -- this permits more efficient neighborhood construction, irrespective of feature size. Furthermore, we use attentive pooling from RandLANet~\cite{Hu2020CVPR} instead of max pooling.

% \paragraph{Set Abstraction (SA)} 
% Set abstraction is the downsampling layer introduced in \cite{Qi2017NIPS}. 
% It proceeds by (1) \textit{sampling} a subset of ``centroid'' points from the point cloud at the current resolution, (2) \textit{grouping} the points around these centroids into local neighborhoods, (3) translating points into the local frame of their centroid and processing all points with a shared MLP, and (4) pooling over each local neighborhood to get a downsampled point cloud with associated features.
% \textit{Sampling} uses iterative farthest point sampling (FPS) \cite{Qi2017NIPS}, and
% % : at each iteration $k$, a point $p_k$ is added to the sampled subset such that $p_k$ is the farthest (in the Euclidean sense) from all previously sampled points $\{p_1, p_2, ..., p_{k-1}\}$.
% \textit{grouping} commonly uses a ball query, which groups together all points within a specified radius from a centroid.
% % Lastly, pooling can be max-pooling, mean-pooling, or an attention pooling~\cite{Hu2020CVPR}.

% \paragraph{Feature Propagation (FP)} 
% Feature propagation is the upsampling layer introduced in \cite{Qi2017NIPS}.
% Its goal is to propagate features from a low-resolution point cloud $\{q_1, ..., q_L\}$ with $L$ points to a higher-resolution point cloud $\{p_1, ..., p_H\}$ with $H$ points where $L{<}H$.
% This is achieved with \textit{feature interpolation}: given the low-res point features $\{g_1, ..., g_L\}$, the high-res point features  $F{=}\{f_1, ..., f_H\}$ are determined by a distance-weighted feature interpolation of the $k$ nearest low-res neighbors for each high-res point (based on the spatial coordinates of the points). 
% % To get the final upsampled point features, the high-res interpolated features $F$ are concatenated with skip-linked features from a corresponding SA layer and processed by a shared MLP.

% \paragraph{Graph Convolution}
% DGCNN~\cite{Wang2019SIGGRAPHb} leverages convolution operations on a dynamically constructed graph to process points.
% At each layer, the graph is constructed by adding edges between each point and its $k$ nearest neighbors in the learned feature space. 
% The convolution centered at a point uses a learned edge function for each edge followed by aggregation. Constructing a dynamic graph in feature space is computationally prohibitive on larger scenes, so we opt to construct the graph based on 3D point distances (\cref{sec:curvecloudnet}). 

\section{Additional Dataset Details} \label{additional-dataset}

\subsection{Kortx Perception System and Dataset}
\paragraph{Kortx Perception System}
\kortx is a perception software system developed by Summer Robotics~\cite{summer-robotics}.
It is an active-light, multi-view stereoscopic system using one or more scanning lasers. It can be configured to use two or more event-based vision sensors to build up arbitrary capture volumes.
Event-based vision sensors are used to detect the scanning laser reflection from target surfaces. Event-based sensors are well suited to this setup as their readout electronics are event triggered instead of time triggered.
Furthermore, the Kortx System supports arbitrary continuous scan patterns, allowing a user to create their own patterns and use their own scan hardware.
For more information, please visit the \href{https://www.summerrobotics.ai/}{Summer Robotics Website}.

\begin{figure}
    \centering
    \includegraphics[width=0.5\textwidth]{content/supp/images/kortx_dataset_qual.jpg}
    \caption{\textit{Kortx Dataset Objects.} Our Kortx dataset contains scans of 7 real-world objects. We visualize one aggregated ``scan" per object from a single viewpoint.}
    \label{fig:kortx-objects-qual}
    % \vspace{-3mm}
\end{figure}
\begin{table}
\setlength{\tabcolsep}{3pt}
\centering
\scalebox{0.93}{
    \begin{tabular}{ l c | c c c c c c}
    \toprule
    \textbf{Object} & \textbf{Total} & Cap & Chair & Earphone & Knife & Mug & Rocket\\
    \midrule
    \textbf{Instances} & 7 & 1 & 1 & 1 & 1 & 2 & 1 \\
    \textbf{Scans} & 39 & 6 & 6 & 4 & 6 & 12 & 5 \\
    \textbf{Frames} & 195 & 30 & 30 & 20 & 30 & 60 & 25 \\
    \bottomrule
    \end{tabular}}
    \vspace{1mm}
    \caption{\textit{\kortx Dataset Statistics.} ``Instance" is a unique 3D object. ``Scan" is a dense object scan from a single viewpoint. ``Frame" is a single frame within the 20Hz stream of the dense scan.}
    \label{tab:kortx-dataset}
\end{table}

\paragraph{Kortx Dataset}
Using Kortx, we scanned $7$ real-world objects: \emph{cap, chair, earphone, knife, mug-1, mug-2} and \emph{rocket} (see \cref{fig:kortx-objects-qual}). Each object was scanned multiple times in different poses, resulting in $39$ total scans (summarized in \cref{tab:kortx-dataset}).
Because the Kortx platform provides a continuous event-based 3D scan output (points are sampled every 5µs), we defined a “frame” as a batch of 2048 consecutive point measurements, corresponding to roughly a 20Hz frame rate. Because each frame differs in its dynamic scanning pattern, we evaluate on 5 consecutive frames per scan in our Kortx dataset, hence resulting in 195 point clouds in total.
We manually labeled scanned points with the semantic part categories defined in the ShapeNet Part Segmentation Benchmark \cite{Chang2015}.  Each Kortx scan is mean-centered, however it is \textit{not} aligned into a canonical pose, resulting in an object's orientation depending on the sensor's reference frame. 
% Because each 2048 point ``frame" exhibits a different dynamic scanning pattern, we select 5 frames from each scan for our Kortx dataset, resulting in 195 point clouds.

\subsection{ShapeNet Simulator} \label{subsec:shapenet-sim}
\begin{figure}[b]
    \centering
    \includegraphics[width=0.48\textwidth]{content/supp/images/shapenet-laser-motions.png}
    \caption{\textit{ShapeNet Simulator.} Our ShapeNet laser-based 3D capture simulator can produce different types of sampling patterns.}
    \label{fig:shapenet-sampling-patterns}
    % \vspace{-3mm}
\end{figure}

We simulate laser-based 3D capture on the ShapeNet Dataset \cite{Chang2015}. For each mesh, we randomly sample a sensor pose on the unit sphere and render the mesh's depth values into a $2048\times2048$ image. Next, we sample 2D lines on the depth image that correspond to a laser's traversal. For the \textit{random} sampling pattern used in the Kortx evaluation (see Sec 4.1 of the main paper) and the ShapeNet Classification evaluation (see \cref{subsec:shapenet-class}), we select random linear traversals in the image plane, with each traversal parameterized by a pixel coordinate $(i, j)$ and direction $\theta \in [0, \pi)$. For the \textit{grid} and \textit{parallel} sampling patterns used in our ShapeNet Segmentation evaluation (see Sec 4.1 of the main paper), we sample evenly-spaced vertical and horizontal lines. To reduce descritization artifacts introduced from the rasterization, we query every 6th pixel along each line for the \kortx segmentation task and every 4th pixel for the ShapeNet segmentation and classification tasks. We repeatedly generate synthetic laser traversals until we have sampled 2048 points from the mesh. Fig. 1 of the main paper shows an example of the \textit{random}, \textit{grid}, and \textit{parallel} sampling patterns used in our ShapeNet Segmentation evaluation. \cref{fig:shapenet-sampling-patterns} provides an additional qualitative illustration of the three sampling patterns, but showing 4096 points per scan for greater visual clarity.

\subsection{A2D2 LiDAR Segmentation}
The Audi Autonomous Driving Dataset (A2D2) \cite{Geyer2020ARXIV} contains 41,280 frames of labeled outdoor driving
scenes captured in three cities. The vehicle is equipped with five LiDAR sensors, each mounted on a different part of the vehicle and with a different orientation, resulting in a unique grid-like scanning pattern. The A2D2 data was captured in urban, highway, and rural environments as well as in different weather conditions. At the time of writing, the A2D2 dataset only contains semantic labels for the front-facing camera. Thus, we evaluate on LiDAR observations within the front-facing camera's field of view, and we map camera categories to LiDAR categories. We will release the code detailing the exact mapping.

\subsection{Discussion on 3D Datasets}
As \lidar and other 3D scanning technologies continue to develop, they are being applied to new and diverse applications, including open-world robotics (\ie embedded agents), city planning, agriculture, mining, and more. Additionally, there is an increasing variety of sensors and sensor configurations, spanning hardware that scans at different point densities, different ranges, and with unique (or controllable) scanning patterns. The A2D2 and Kortx datasets are two recent examples of such a trend. We believe an important future direction will be to develop a 3D backbone that is performant in \textit{all} these settings. Furthermore, we believe it is important to understand \textit{which} settings ``break" previous assumptions such as the range-view projection, the birds-eye-view projection, and spherical attention. While \arch is a first step towards this goal, we believe it will be important to capture and compile new 3D datasets, and to evaluate on a greater diversity of environments. 
% Additionally, because systems such as \kortx allow in-the-loop feedback to dynamically focus on regions of interest, we believe a valuable future direction will be to design 3D architectures that can support multi-resolution in-the-loop feedback during data capture.


% While \arch is largely agnostic to the scanning pattern, we believe an important future direction will be to explicitly understand how new scanning patterns might ``break" previous assumptions in 3D data processing, such as the polar range-view projection, the birds-eye-view projection, and the cylindrical convolution. Additionally, because systems such as \kortx allow in-the-loop feedback to dynamically focus on regions of interest, we believe a valuable future direction will be to design 3D architectures that can support multi-resolution in-the-loop feedback during data capture.

% \todo{I should tie this into our NEW story!}
% As LiDAR and other 3D scanning technologies develop, there exists a clear trend towards scanning more points, performing higher-quality 3D measurements, and generating more complex and controllable scanning patterns \cite{summer-robotics, wang2022microsnanoeng, moussy2022website}. The A2D2 and Kortx datasets are two recent examples of such a trend, exhibiting grid scanning patterns and arbitrary scanning patterns. While \arch is largely agnostic to the scanning pattern, we believe an important future direction will be to explicitly understand how new scanning patterns might ``break" previous assumptions in 3D data processing, such as the polar range-view projection, the birds-eye-view projection, and the cylindrical convolution. Additionally, because systems such as \kortx allow in-the-loop feedback to dynamically focus on regions of interest, we believe a valuable future direction will be to design 3D architectures that can support multi-resolution in-the-loop feedback during data capture.




\section{Additional Implementation Details} \label{additional-implementation}

\subsection{Baselines}

\paragraph{PointNet++ and DGCNN}
We train and evaluate PointNet++~\cite{Qi2017NIPS} and DGCNN~\cite{Wang2019CVPR} using the reproduced implemenations from Pytorch Geometric~\cite{FeyLenssen2019}\footnote{\href{https://github.com/pyg-team/pytorch_geometric/tree/master/examples}{https://github.com/pyg-team/pytorch\_geometric/}}. For PointNet++, we run hyperparameter sweeps to tune the radius and downsampling ratio on each dataset. For DGCNN, we use the authors' reported hyperparameters. 

\paragraph{RandLANet}
We train and evaluate RandLANet~\cite{Hu2020CVPR} using the reproduced implementation from Open3D-ML~\cite{Zhou2018open3d}\footnote{\href{https://github.com/isl-org/Open3D-ML}{https://github.com/isl-org/Open3D-ML}}. We additionally improve the latency by incorporating GPU-implementations for point grouping and sampling from PyTorch3D~\cite{Ravi2020pytorch3d}\footnote{\href{https://github.com/facebookresearch/pytorch3d}{https://github.com/facebookresearch/pytorch3d}}. We use the authors' reported hyperparameters.

\paragraph{CurveNet}
We train and evaluate CurveNet~\cite{Xiang2021ICCV} using the authors' official implementation\footnote{\href{https://github.com/tiangexiang/CurveNet}{https://github.com/tiangexiang/CurveNet}}. We use the authors' reported hyperparameters for all datasets.

\paragraph{PointMLP}
We train and evaluate PointMLP~\cite{Ma2022ICLR} using the authors' official implementation\footnote{\href{https://github.com/ma-xu/pointMLP-pytorch}{https://github.com/ma-xu/pointMLP-pytorch}}. We use the reported hyperparameters for all datasets.

\paragraph{PointNext}
We train and evaluate PointNext~\cite{Qian2022PointNeXtRP} using the authors' official implementation\footnote{\href{https://github.com/guochengqian/PointNeXt}{https://github.com/guochengqian/PointNeXt}}. As outlined by the authors, we use PointNext-Small for the ShapeNet and KortX datasets. On the A2D2, nuScenes, and KITTI datasets, we use the larger PointNext-XL. Because the authors indicate the importance of the network ``radius", we additionally performed a hyperparameter sweep to find the best radius of 0.05 for the A2D2, nuScenes, and KITTI datasets.  

\paragraph{MinkowskiNet}
We train and evaluate MinkowskiNet~\cite{Choy2019CVPR} using the authors' official implementation\footnote{\href{https://github.com/NVIDIA/MinkowskiEngine}{https://github.com/NVIDIA/MinkowskiEngine}}. We use the larger MinkUNet-34A for all experiments. We use an initial voxel size of $0.05$ on outdoor datasets and $0.015$ on object-level datasets. 

% 
\paragraph{Cylinder3D}
We train and evaluate Cylinder3D~\cite{Zhou2020ARXIV} using the authors' official implementation\footnote{\href{https://github.com/xinge008/Cylinder3D}{https://github.com/xinge008/Cylinder3D}}. We use the reported hyperparameters on the nuScenes and KITTI datasets. On the A2D2 dataset, we set the cylindrical voxel grid to cover a $\pm 31 ^{\circ}$ forward-facing azimuth with a maximum radius of 80 meters and a height covering $[-5, 20]$ meters; we define the initial grid to have 360 radial partitions, 120 angular partitions, and 120 height partitions. On the ShapeNet and KortX datasets, we set the voxel grid to cover all $360^{\circ}$ with a radius of $1.0$ and height of $1.0$; to address latency and memory constraints, we define the initial grid to have 96 radial partitions, 96 angular partitions, and 96 height partitions.

\paragraph{SphereFormer}
We train and evaluate SphereFormer~\cite{lai2023spherical} using the authors' official implementation\footnote{\href{https://github.com/dvlab-research/SphereFormer}{https://github.com/dvlab-research/SphereFormer}}. We use the reported hyperparameters for the nuScenes and KITTI datasets, and we use the reported nuScenes hyperparameters for the A2D2 dataset. For the KortX and ShapeNet datasets, we also use the reported hyperparameters, and we reduce the voxel size from $0.1$ to $0.015$ to account for the dataset's smaller 3D scale. On the KortX and ShapeNet datasets, we additionally ran a sweep on different voxel sizes and spherical window sizes, but observed limited differences.

\subsection{Training Strategy}
We train \arch and baselines on segmentation tasks with a standard cross-entropy loss. Following previous works, we also supplement the loss with a Lovasz loss \cite{Berman2018CVPR, Zhou2020ARXIV} for the nuScenes, A2D2, and KITTI datasets. At training, we apply random scaling and translation augmentations, as well as random flips on the nuScenes, A2D2, and KITTI datasets. Importantly, we use an \textbf{\textit{identical}} training strategy for \arch and each baseline. We experimentally observe convergence in all models' validation accuracies by the end of training.

% \paragraph{Baseline Implementations}
% For PVCNN~\cite{Liu2019NeurIPS}\footnote{\href{https://github.com/mit-han-lab/pvcnn}{https://github.com/mit-han-lab/pvcnn}}, PointMLP~\cite{Ma2022ICLR}\footnote{\href{https://github.com/ma-xu/pointMLP-pytorch}{https://github.com/ma-xu/pointMLP-pytorch}}, and Cylinder3D~\cite{Zhou2020ARXIV}\footnote{\href{https://github.com/xinge008/Cylinder3D}{https://github.com/xinge008/Cylinder3D}}, we use the official GitHub implementation of each paper.
% \todo{provide update for each baseline}

\paragraph{Object Part Segmentation}
We train CurveCloudNet and all baselines for 60 epochs in the KortX experiment and 120 epochs in the ShapeNet experiment with the Adam optimizer \cite{Kingma2015ICLR}, a learning rate of $1e^{-4}$, batch momentum decay of $0.97$, and exponential learning rate decay of $0.97$. For
all models, except for Cylinder3D, we use a batch size of 24. For Cylinder3D, we use a batch size of 12 because 24 exceeds our GPU memory capacity. 
% Additionally, for Cylinder3D we follow \cite{iek20163DUL} and choose a voxel-grid size of $96\times96\times96$. We do not use the original size of $180 \times 180 \times 180$ because it results in excessive memory usage at training (a batch size of 12 requires $\sim$70GB of GPU memory). We experimentally observe convergence in all models' validation accuracy after $60$ epochs.

\paragraph{A2D2 LiDAR Segmentation}
We train CurveCloudNet and all baselines for 140 epochs with the Adam optimizer, a batch size of 7, a learning rate of $1e^{-3}$, and an exponential learning rate decay of $0.97$. 
% We experimentally observe convergence in all models' validation accuracy after $140$ epochs.

\paragraph{nuScenes and KITTI LiDAR Segmentation}
We train CurveCloudNet and all baselines for 100 epochs with the Adam optimizer, a batch size of 4 on nuScenes and 2 on KITTI, a learning rate of $1e^{-3}$, and an exponential learning rate decay of $0.97$. At test time, we follow previous works \cite{Zhou2020ARXIV,Liu2021TPAMI} and average model predictions over axis-flipping and scaling augmentations.
% average model predictions over $x$-axis, $y$-axis, and $xy$-axis flip augmentations.

% \paragraph{nuScenes LiDAR Segmentation}
% \todo{convert into KITTI lidarseg!}
% We train CurveCloudNet and all baselines for 100 epochs with the Adam optimizer, a batch size of 2, a learning rate of $1e^{-3}$, and an exponential learning rate decay of $0.97$. At test time, we follow previous works \cite{Zhou2020ARXIV,Liu2021TPAMI} and average model predictions over axis-flipping and scaling augmentations.

\section{Experiments}
\label{sec:experiments}

\subsection{Setup}
\textbf{Datasets.} We evaluate RFFR with four challenging datasets specifically designed for deepfake detection. We adopt the high quality (HQ) version of Faceforensics++ (FF)~\cite{ff} for training our deepfake detector. Faceforensics++ includes videos of real faces as well as four subsets of fake faces, each manipulated with a different algorithm, namely Deepfakes (DF), Face2Face (F2F), FaceSwap (FSW) and NeuralTextures (NT). We also utilize the test set of Celeb-DF~\cite{celeb-df} and DFDC~\cite{dfdc} for evaluating the cross-dataset performance of our model. Finally, in addition to real faces of Faceforensics++, we adopt the real face images from ForgeryNet (FN)~\cite{forgerynet} for learning RFFR, which helps improve representation learning with additional data.

\textbf{Implementation Details.} We extract the frames from all video datasets and use RetinaFace~\cite{retinaface} to detect and align the faces. All images are scaled to the size of $224 \times 224$. For our RFFR model, we adopt a base version of Masked Autoencoder (MAE)~\cite{mae} and train it on real faces with a batch size of $128$. Following MAE, we set the learning rate at $7.5 \times 10^{-5}$ and adjust it with a schedule with warmup and cosine decay. By default, we train this model with the real faces from both FF~\cite{ff} and FN~\cite{forgerynet}. 

For training the deepfake detector, we divide each image with $k = 4$ (Refer to Appendix for the motivation of choosing $k$). Each block enters the classifier with a probability of $p = 0.25$, and the residual images are amplified by $\alpha=4$. No data augmentation is applied to the images. We initialize both branches of Vision Transformer with ImageNet-pretrained weights and train them with a learning rate of $2 \times 10^{-5}$. During testing, we iteratively mask and restore all blocks to obtain a full residual image for the detector to process. We evaluate the testing results with AUC (Area Under Curve). 

\subsection{Cross-domain performance evaluation}
In this section, we test the performance of our RFFR-based deepfake detector with cross-manipulation and cross-dataset evaluations. 

\textbf{Cross-manipulation evaluations.} We train our deepfake detector on each subset of Faceforensics++ and test on all four subsets to demonstrate our model's ability to identify different manipulations, including those not seen during training. \emph{We adopt the HQ version of FF for both training and testing, and only use one frame every video for testing.} We compare our results with state-of-the-art image-based methods Multi-Attention~\cite{multiatt}, DCL~\cite{dcl}, RECCE~\cite{recce} and UIA-ViT~\cite{uia}. We ran the public code of RECCE and UIA-ViT to produce results under the same setting.

In~\cref{tab:cross-manipulation}, we show that our method outperforms the state-of-the-art methods under most settings, with a maximum improvement of $10.25\%$ (F2F $\rightarrow$FSW). Meanwhile, our model remains effective under the four intra-domain settings, which are shown in gray. The method tends to slightly underperform when trained on NeuralTextures, likely because its manipulation patterns only exist in certain small regions, and may be neglected during our block sampling. Nevertheless, compared to existing methods, our deepfake detector yields much better overall performances. 

\begin{table}[t]
\setlength\tabcolsep{4.5pt} 
\caption{Cross-manipulation performances in terms of AUC(\%) compared with previous methods. Classifiers are trained on one subset of FF and tested on all four subsets. Intra-domain results are marked in gray. We ran the public code of methods marked with "*" to produce results under identical settings \emph{(HQ for training and single frames for testing).}}
\vspace{-1.5em}
\label{tab:cross-manipulation}
\begin{center}  
\scalebox{0.80}{
\begin{tabular}{c|l|cccc|c}
\toprule
Training &\multirow{2}*{Method} & \multicolumn{4}{c|}{Test data} & \multirow{2}*{Avg} \\
\cmidrule(lr){3-6}
     data  &            ~                   & DF    & F2F   & FSW   & NT    & ~   \\
     
\midrule
\multirow{5}*{DF}
& MultiAtt~\cite{multiatt} & \cellcolor{Gray}99.92 & 75.23 & 40.61 & 71.08 & 71.71                \\ 
& DCL~\cite{dcl}       & \cellcolor{Gray}\textbf{99.98} & \textbf{77.13} & 61.01 & 75.01 & 78.28              \\
& RECCE*~\cite{recce}     & \cellcolor{Gray}99.19 & 74.39 & 57.42 & \textbf{85.04} & 79.01                \\ 
& UIA-ViT*~\cite{uia}  & \cellcolor{Gray}99.39      &   74.44    &   53.89    &   70.92    & 74.66 \\ 
& Ours  & \cellcolor{Gray}99.19 & 76.61 & \textbf{68.96} & 74.83 & \textbf{79.90}            \\ 
       
\midrule
\multirow{5}*{F2F}
        & MultiAtt~\cite{multiatt}       & 86.15 & \cellcolor{Gray}99.13 & 60.14 & 64.59 & 77.50 \\
        & DCL~\cite{dcl}       & 91.91 & \cellcolor{Gray}99.21 & 59.58 & 66.67 & 79.34 \\
       & RECCE*~\cite{recce}       & 88.04 & \cellcolor{Gray}98.93 & 67.35 & 74.16 & 82.12 \\
       & UIA-ViT*~\cite{uia}       & 83.39 & \cellcolor{Gray}98.32 & 68.37 & 67.17 & 79.31 \\
       & Ours                                  & \textbf{93.75} & \cellcolor{Gray}\textbf{99.61} & \textbf{78.62} & \textbf{79.56} & \textbf{87.81} \\

\midrule
\multirow{5}*{FSW}
& MultiAtt~\cite{multiatt} & 64.13 & 66.39 & \cellcolor{Gray}99.67 & 50.10 & 70.07              \\
& DCL~\cite{dcl}           & 74.80 & 69.75 & \cellcolor{Gray}99.90 & 52.60 & 74.26              \\
& RECCE*~\cite{recce}       & 66.66 & 73.66 & \cellcolor{Gray}\textbf{99.76} & \textbf{57.46} & 74.39               \\

& UIA-ViT*~\cite{uia}       &   81.02    &   66.30    & \cellcolor{Gray}99.04      &   49.26    & 73.91 \\ 
& Ours                                           & \textbf{87.46} & \textbf{75.96} & \cellcolor{Gray}99.42 & 55.87 & \textbf{79.68}            \\ 

\midrule
\multirow{5}*{NT}
& MultiAtt~\cite{multiatt} & 87.23 & 75.33 & 48.22 & \cellcolor{Gray}98.66 & 77.36                \\
& DCL~\cite{dcl}      & 91.23 & 79.31 & 52.13 & \cellcolor{Gray}\textbf{98.97} & 80.41                \\
& RECCE*~\cite{recce}    & \textbf{90.20}  & 76.65 & \textbf{58.06} & \cellcolor{Gray}97.17 & \textbf{80.52}                \\
 & UIA-ViT*~\cite{uia}  &    79.37   &   67.98    &   45.94    &\cellcolor{Gray}94.59       & 71.97 \\
 & Ours     & 84.31 & \textbf{81.04} & 54.67 & \cellcolor{Gray}96.19 & 79.05          \\
       
\bottomrule
\end{tabular}}
\vspace{-2em}
\end{center}
\end{table}

\textbf{Cross-dataset evaluations.} We train our model on the Faceforensics++ dataset and evaluate its performance on the test sets of Celeb-DF\cite{celeb-df} and DFDC~\cite{dfdc}. Specifically, following the previous practice in~\cite{lip}, we validate the model on Celeb-DF and use the selected model to test on DFDC.  \emph{We adopt the HQ version of FF for training, and only use one frame every video for testing.} Under the same setting, we ran the public code of RECCE~\cite{recce}, UIA-ViT~\cite{uia} and SBI~\cite{sbi} to produce corresponding results. In Table~\ref{tab:cross-dataset}, we show a competitive performance with existing image-based methods, signaling satisfying adaptability of RFFR to different datasets, especially high quality datasets like Celeb-DF. 
  
SBI~\cite{sbi} is a recent powerful deepfake detection method. By utilizing a hand-crafted blending algorithm to produce diverse fake samples, it achieves highly competitive performances on datasets including Celeb-DF. We show that by training on fake samples generated by SBI, our approach can further improve upon their state-of-the-art result. 

\begin{table}[]
\setlength\tabcolsep{4.5pt} 
\caption{Cross-dataset performances in terms of AUC(\%) compared with previous methods. Classifiers are trained on FF and tested on Celeb-DF and DFDC. We ran the public code of methods marked with "*" to produce results under identical settings \emph{(HQ for training and single frames for testing).}}
\vspace{-1em}
\label{tab:cross-dataset}
\begin{center}  
\scalebox{0.90}{
\begin{tabular}{l|cc}
\toprule
\multirow{2}*{Method} & \multicolumn{2}{c}{Test data}\\
\cmidrule{2-3}
        ~                           &     Celeb-DF         &  DFDC \\
\midrule
      Xception~\cite{xception}  &     65.30       &    -  \\
      Face X-ray~\cite{xray}          &     74.20       &     70.00 \\
      MultiAtt~\cite{multiatt}        &     67.44       &     67.34 \\
      SPSL~\cite{SPSL}                &     76.88        &   -  \\
      SOLA~\cite{sola}                &       76.02         &  -    \\
      SLADD~\cite{sladd}              &    79.70       &  -  \\
      RECCE*~\cite{recce}             &     68.94       &   68.34   \\
      UIA-ViT*~\cite{uia}             &     80.31      &   67.93   \\
      SBI*~\cite{sbi}                       &       86.46     &   66.60     \\
\midrule
 	Ours                                      &   81.97  & \textbf{72.08}  \\
    Ours + SBI~\cite{sbi}                  &  \textbf{88.98}           &    67.84   \\
\bottomrule
\end{tabular}}
\vspace{-2.5em}
\end{center}
\end{table}

\subsection{Ablation Study}
\label{ablation}

In this section, we analyze the effect of our implementations for RFFR learning and deepfake detection. 

\textbf{Effect of the training data for RFFR.} The effectiveness of deepfake detection with RFFR depends on the quality of representation learning, where the real faces plays an important role. In this experiment, we examine the effect of scaling the real face dataset for representation learning. As a baseline, we learn RFFR with only real faces from Faceforensics++ (FF), the same data we use for the downstream classification tasks. Meanwhile, another model is supplemented with real faces from both FF and ForgeryNet (FN), a significantly larger and more diverse dataset. We train deepfake detectors on the F2F subset of FF with residual images produced by these two models. In Table~\ref{tab:data}, we demonstrate that including the extra dataset of ForgeryNet for learning RFFR consistently improves the performances of the deepfake detector in all tests, creating a maximum performance gain of $9.57\%$  in terms of AUC (F2F $\rightarrow$ NT).

We note that learning RFFR with FF already allows our deepfake detector to outperform the state-of-the-arts. Nevertheless, learning with extra data enhances the efficacy of our real face foundation representations, and further improves the downstream task of deepfake detection. Therefore, refining the representation learning of real faces, especially with large-scale datasets, could be a viable path for further improving generalized deepfake detection. 

In addition, we examine the scalability of RECCE under the same setting, considering that RECCE~\cite{recce} also involves learning to reconstruct real samples for deepfake detection. However, their performance gain is less significant than ours. Although the reconstruction branch of RECCE~\cite{recce} is able to highlight forgery cues with residual images, they tend to involve more background noise caused by imperfect reconstructions, as depicted in~\cref{fig:unet_comparison},. This undermines the ability of residual images to expose artifacts for deepfake detection. 

\begin{table}[t]
\setlength\tabcolsep{4.5pt} 
\caption{Deepfake detection performances of RECCE~\cite{recce} and our method with different real face dataset, namely the real faces from Faceforensics++ (FF) alone, and FF combined with ForgeryNet (FF + FN). Classifiers are trained on F2F and tested on four subsets of FF. We present the results in AUC (\%).  }
\vspace{-1.5em}
\label{tab:data}
\begin{center}  
\scalebox{0.90}{
\begin{tabular}{c|c|cccc|c}
\toprule
\multirow{2}*{Method} & Real face  & \multicolumn{4}{c|}{Test data} & \multirow{2}*{Avg} \\
\cmidrule(lr){3-6}
&dataset  &      DF    & F2F   & FSW   & NT    & ~   \\
    \midrule
\multirow{2}*{RECCE~\cite{recce}}&FF           & 88.04          & 98.93          & 67.35          & 74.16          & 82.12          \\
&FN + FF &  90.12       & 99.24       & 69.89    & 79.59     & 84.71		\\
    \midrule
\multirow{2}*{Ours}&FF           & 90.16          & 98.56          & 74.10          & 69.99          & 83.20          \\
&FN + FF & \textbf{93.44}       & \textbf{99.61}        & \textbf{78.62}       & \textbf{79.56}        & \textbf{87.81}		\\
\bottomrule
\end{tabular}}
\vspace{-1em}
\end{center}
\end{table}

\textbf{Effect of masked image modeling for RFFR.} We analyze the effect of using MIM-based residual images for deepfake detection. We train a UNet-based autoencoder (AE) to learn the reconstruction of real faces and obtain residual images. Our MIM-trained inpainting model and the AE are compared on the quality of reconstruction in~\cref{fig:unet_comparison}. Note that despite being trained with real faces, the AE "generalizes" well to fake images, preserving delicate details, including the artifacts caused by manipulations. Such generalization leaves the residual images empty with little information. 

\begin{figure}
\centering
  \includegraphics[width=0.9\columnwidth]{figs/compare_ICCV_Final.pdf}
  \vspace{-1em}
   \caption{Reconstruction results and residual images of the autoencoder (AE), RECCE~\cite{recce} and our inpainting model. AE reconstructs both images perfectly, leaving no information in residual images. RECCE~\cite{recce} suffers from insufficient training. Our model successfully highlights potential artifacts in the residual image of only the fake face, and therefore can best facilitate deepfake detection. }
\vspace{-1em}
\label{fig:unet_comparison}
\end{figure}

Masked image modeling enables our model to learn better real face representations and inpaint fake faces with real textures instead of artifacts. In the downstream task of deepfake detection,  our classifier generalizes significantly better than the AE-based classifier, which performs only marginally better than learning with no residuals (detailed in Appendix). Both the reconstruction results and the downstream performance confirm the validity of our choice to learn RFFR with MIM instead of direct reconstruction. 


\textbf{Effect of classifier backbone.} In Table~\ref{tab:backbone}, we present the deepfake detection results of vanilla Xception~\cite{xception} and Vision Transformer (ViT)~\cite{vit}, both trained with full original images. The models are trained with the F2F subset of FF and tested on all four subsets. While a larger backbone increases a deepfake detector's generalization performance in some cases, it is not the primary factor of our performance improvement. Instead, it is the residual input aided by RFFR that leads the performance gain.

\begin{table}[t]
\setlength\tabcolsep{4.5pt} 
\caption{Comparing ours results with vanilla backbones. We present the results in AUC (\%).  }
\label{tab:backbone}
\vspace{-1.5em}
\begin{center}  
\scalebox{0.90}{
\begin{tabular}{c|c|cccc|c}
\toprule
Training  &  \multirow{2}*{Method}    &   \multicolumn{4}{c|}{Test Data} & \multirow{2}*{Avg} \\
\cmidrule(lr){3-6}
 data  &   ~  &   DF    & F2F   & FSW   & NT    & ~   \\
    \midrule
\multirow{3}*{F2F} & Xception~\cite{xception} & 84.94          & 99.26          & 58.82          & 71.19          & 78.55          \\
                                   & ViT~\cite{vit}      & 84.25          & 97.89          & 65.53          & 65.18          & 78.21          \\
                                   & Ours     & \textbf{93.44} & \textbf{99.61} & \textbf{78.62} & \textbf{79.56} & \textbf{87.81} \\
\bottomrule
\end{tabular}}
\vspace{-1.5em}
\end{center}
\end{table}

\textbf{Effect of classifier design.} We compare different variants of our classifier design. Specifically, we analyze the performance gains brought by the introduction of two branches and the random input mechanism. We test six variants of our classifier by training them with the F2F subset of FF and testing with the FSW subset. The settings of these variants are specified by the input data they accept, as shown in~\cref{tab:classifier}. 

\begin{table}[t]
\caption{Deepfake detection performances with classifiers of different inputs in terms of AUC (\%). We train the classifiers on F2F and test on FSW.}
\label{tab:classifier}
\vspace{-1.5em}
\begin{center}
\begin{tabular}{c|c|c|c|c}
\toprule
\multicolumn{2}{c|}{Original Image} & \multicolumn{2}{c|}{Residual Image} & \multirow{2}*{AUC (\%)} \\
\cline{1-4}
               Full        &             Random           &          Full          &          Random          &   ~\\
 \hline
\checkmark        &                                       &                            &                                   &  65.53\\
% \hline
                              &                                      &   \checkmark    &                                   &  66.30  \\
 %\hline
\checkmark        &                                      &   \checkmark    &                                   &  71.48  \\
 %\hline
                             &       \checkmark          &                             &                                   &  70.76  \\
%\hline
                             &                                       &                             &      \checkmark       &  68.10  \\
 %\hline
                             &        \checkmark         &                             &      \checkmark       &  \textbf{78.62}  \\
\bottomrule
\end{tabular}
\vspace{-2em}
\end{center}
\end{table}

\begin{table*}[t]
\setlength\tabcolsep{4.5pt} 
\caption{Deepfake detection performances of validated and non-validated models. Classifiers are trained on F2F and tested on four subsets of FF. We present the results and the performance gaps in AUC (\%). Second best results are underlined. }
\label{tab:validation}
\vspace{-1em}
\begin{center}  
\scalebox{0.90}{
\begin{tabular}{c|c|llll|l}
\toprule
\multirow{2}*{Method}  & \multirow{2}*{Validated} & \multicolumn{4}{c|}{Test Data} & \multirow{2}*{Avg} \\
\cmidrule(lr){3-6}
~                   &                      ~                   &      DF               & F2F                    & FSW                 & NT                    & ~   \\
    \midrule
\multirow{2}*{Xception\cite{xception}} &   \checkmark    & 84.94                 & 99.26                & 58.82                 & 71.19                & 78.55            \\
~ &                                             -                              & 83.08   (- 1.86) & 99.12   (- 0.14) & 46.63   (- 12.19) & 64.93   (- 6.26)  & 73.44   (- 5.11)  \\
 \hline
 \multirow{2}*{RECCE\cite{recce}} &\checkmark               & 88.04                & 98.93                 & 67.35                & 74.16                & 82.12            \\
 ~&                                                -                  & 74.51   (- 8.57) & 99.22   (+ 0.29)  & 50.17   (- 17.18) & 59.46   (- 14.70)  & 70.84   (- 11.28) \\
 \hline
\multirow{2}*{Ours} &    \checkmark  & \textbf{93.44}            & \textbf{99.61}            & \textbf{78.62}            & \textbf{79.56}            & \textbf{87.81}            \\
 ~&  - & \underline{91.56} (- 1.88) & \underline{99.39}   (- 0.22) & \underline{76.00}   (- 2.62)  & \underline{76.41} (- 3.15) & \underline{85.84}   ( - 1.97)    \\
\hline
\end{tabular}}
\vspace{-2em}
\end{center}
\end{table*}

We treat the vanilla ViT with full original image input as a baseline, which achieves an AUC of $65.53\%$. By switching to accept the full residual images, we obtain a $0.77\%$ performance gain. Combining the two modalities to form a dual-branch classifier further increases our result to $71.48\%$. This demonstrates that the artifacts are better exploited when both the original and the residual images enter the classifier, and are used as references to each other. Therefore, both modalities should be considered for classification. 

In addition, we improve on the test by merely modifying the baseline ViT to accept randomly selected original image blocks. This results in a $5.23\%$ increase in performance. Similarly, changing full residual input to random residual blocks also results in a $1.8\%$ improvement. These observations confirm our hypothesis in \cref{sec:method_deepfake_detection} that models benefit from learning with random inputs, which prevents the model from only focusing on the most prominent features in an image, and forces it to learn from subtle artifacts. 

Finally, bringing in the random input mechanism for the dual-branch classifier completes our full implementation, which maximally exploits the artifacts exposed by RFFR and achieves the best performance of $78.62\%$. 



\subsection{Validation-free Model Selection}
\label{sec:validation-free}

\begin{figure}
\centering
  \includegraphics[width=0.5\textwidth]{figs/validation-free_ICCV_Final.png}
  \vspace{-1.5em}
   \caption{Comparing the validation curves of RFFR-based deepfake detector and previous methods. Detectors are trained on the F2F subset of FF for $15k$ iterations and validated on four different subsets. (a) to (d) correspond to experiments on DF, F2F, FSW and NT.  Results are reported in AUC (\%). All three methods perform well when validated on F2F. However, under cross-manipulation settings, only our method avoids overfitting during training. The curves are smoothed for better visibility.}
\label{fig:validation-free}
\vspace{-1em}
\end{figure}

Models expected to generalize to other domains benefit from target domain validations~\cite{domainbed}. By frequently performing model validation, we can select the model  that best suits the detection of target manipulation, resulting in high performance on the test set. While using such an \textit{oracle} could be acceptable for the early development of cross-domain algorithms~\cite{domainbed}, it is not ideal for applications, as labeled data of unseen manipulation is usually not available. 

In this section, we demonstrate the potential of our deepfake detector to circumvent this practice and therefore avoid the need for extra validation data. As shown in \cref{tab:validation}, we train our classifier on F2F for 15k iterations and directly use the final model for testing. Simultaneously, we employ four validation sets to select the models with the best validation performances on target data. All validated and non-validated models are tested under the same conditions. We report all results on the target test sets in Table~\ref{tab:validation}. The performance gaps between validated and non-validated models are reported along with the test results. Although our non-validated models are not performing as well as those selected with a validation set, we show that our model remains effective on target data, with a maximum performance drop of $3.15\%$ and an average drop of $1.97\%$. However, previous methods~\cite{xception, recce} suffer from significantly larger performance drops when evaluated under the same procedure. 

To take a closer look at how the cross-manipulation performances vary during training, we train the deepfake detectors again with F2F. We test the AUC performances on all target subsets every 50 iterations to produce validation curves in \cref{fig:validation-free}. Our RFFR-based deepfake detector consistently maintains a high performance long after its peaks without serious overfitting. On the contrary, both previous methods compared here overfit quickly after reaching their highest target domain performances. In addition, compared methods exhibit large fluctuations across different evaluations, while our model remains stable. This suggests that with RFFR, our model focuses exclusively on generalizable features which fall outside the distribution of RFFR. Such resistance to overfitting guarantees our model a satisfying performance even when labeled validation sets are not available, which is generally expected in practice. We present more results on validation-free evaluations in Appendix.


{
    \small
    \bibliographystyle{ieeenat_fullname}   \bibliography{bibliography_long,bibliography,bibliography_custom}
}

\end{document}
