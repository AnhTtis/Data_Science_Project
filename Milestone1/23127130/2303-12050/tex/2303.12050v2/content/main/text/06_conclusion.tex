%
\section{Discussion and Limitations}

% \todo{sum up experiments again, we do good on everything, flexible and efficient, refer to fig 1}
% \todo{conclusion and limitations}

We have described a point cloud processing scheme and backbone, \arch, which introduces curve-level operations to achieve accurate, efficient, and flexible performance on point cloud segmentation. CurveCloudNet outperforms or is competitive with previous methods on the ShapeNet, \kortx, A2D2, nuScenes, and KITTI datasets, and on average achieves the best performance. Put together, \arch is a unified solution to \emph{both} small and large-scale scenes with various scanning patterns.

Nevertheless, \arch has limitations. First, \arch is only designed for laser-scanned data, \ie point clouds with \textit{explicit} curve structure due to 1D laser traversals. We believe a promising future direction is to investigate \textit{virtual curves} that could extend \arch to uniformly sampled point clouds.
% Furthermore, although all laser scanners capture a 1D trajectory of scanned points, it may require additional hardware configuration to preserves this ordering when streaming data between devices.      
Furthermore, we believe that future research can continue to improve curve operations. While our proposed curve operations yield significant improvements, an exciting future direction will be to investigate explicit curve-to-curve communication, curve self-attention and cross-attention, and curve intersections.
% In future research, we intend to incorporate explicit curve-to-curve communication that can further improve accuracy and efficiency on large scenes. Additionally, we believe an exciting direction will be to explicitly incorporate geometric properties of polylines, such as surface tangents and normals along with their connectivity via intersections.




