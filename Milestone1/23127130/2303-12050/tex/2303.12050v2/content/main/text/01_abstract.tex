%%%%%%%%% ABSTRACT



\begin{abstract}

% Version 1:
% Modern depth sensors such as LiDAR operate by sweeping laser-beams across the scene, resulting in a point cloud with notable 1D curve-like structures. However, most existing point cloud backbones  discard the rich, 1D traversal patterns and rely mainly on Euclidean operations.
% In this work, we present a novel point cloud processing scheme and backbone, \textbf{CurveCloudNet}, that exploits the curve-like structure of modern depth sensors. Concretely, %instead of treating each point independently, 
% we parameterize the point cloud as a collection of polylines and thus establish a local surface-level ordering on the points. 
% We then devise curve-specific operations to process the ``curve clouds:'' (1) a \textit{symmetrical 1D convolution}, 2) a \textit{ball grouping} operation for merging points along curves, and (3) an efficient \textit{1D furthest-point-sampling} algorithm on curves. \textbf{CurveCloudNet} combines these curve operations with existing point-based operations, resulting in an efficient, scalable, and expressive backbone that uses little GPU memory. We evaluate \textbf{CurveCloudNet} on the ShapeNet, Kortx, Audi Driving, and nuScenes datasets, showcasing state-of-the-art segmentation and classification performance across {\em both} object-level and large outdoor scene datasets, the first reported 3D point backbones to do so. 

% Version 2:
% In this work we introduce a new point cloud processing scheme and backbone, called CurveCloudNet, which takes advantage of the curve-like structure inherent in modern depth sensors such as LiDAR. While traditional point cloud backbones discard the rich, 1D laser-traversal patterns and rely on Euclidean operations, CurveCloudNet parameterizes the point cloud as a collection of polylines. This parameterization establishes a local surface-level ordering on the points. Our method applies curve-specific operations to process the ``curve clouds," including symmetrical 1D convolution, ball grouping for merging points along curves, and an efficient 1D furthest-point-sampling algorithm on curves. Combining these curve operations with existing point-based operations results in an efficient, scalable, and expressive backbone that uses little GPU memory. We evaluate CurveCloudNet on several datasets, including ShapeNet, Kortx, Audi Driving, and nuScenes, and report state-of-the-art segmentation and classification performance across \textbf{both} object-level and large outdoor scene datasets, making CurveCloudNet the first 3D point backbone to achieve such results.
% \vspace{-1em}

% Version 3
Modern depth sensors such as LiDAR operate by sweeping laser-beams across the scene, resulting in a point cloud with notable 1D curve-like structures. In this work, we introduce a new point cloud processing scheme and backbone, called \arch, which takes advantage of the curve-like structure inherent to these sensors. While existing backbones discard the rich 1D traversal patterns and rely on generic 3D operations, \arch parameterizes the point cloud as a collection of polylines (dubbed a ``curve cloud”), establishing a local surface-aware ordering on the points. By reasoning along curves, \arch captures lightweight curve-aware priors to efficiently and accurately reason in several \textbf{diverse} 3D environments. 
% , including a symmetric 1D convolution, a ball grouping for merging points along curves, and an efficient 1D farthest point sampling algorithm on curves.
We evaluate \arch on multiple synthetic and real datasets that exhibit distinct 3D size and structure.
%, including: ShapeNet, Audi Driving, nuScenes, Kitti, and a new dataset we name KortX.
We demonstrate that \arch outperforms both point-based and sparse-voxel backbones in various segmentation settings, notably scaling to large scenes better than point-based alternatives while exhibiting improved single-object performance over sparse-voxel alternatives.
In all, \arch is an efficient and accurate backbone that can handle a larger variety of 3D environments than past works. 
%In all, \arch is an off-the-shelf trainable and performant backbone that is ready for the diverse environments faced in open-world applications such as robotics. 

% in various segmentation settings, notably scaling better to large scenes than point-based alternatives while exhibiting better single object performance than sparse-voxel alternatives. 

% CurveCloudNet applies a mix of curve-specific operations and Euclidean point-based operations, resulting in an efficient and accurate backbone that can flexibly reason on \textit{many} different types of 3D scenes. 
% , including a symmetric 1D convolution, a ball grouping for merging points along curves, and an efficient 1D farthest point sampling algorithm on curves.
% By combining these curve operations with existing point-based operations, CurveCloudNet is an efficient and accurate backbone that can flexibly reason on \textit{many} different types of 3D scenes. 
% CurveCloudNet achieves state-of-the-art segmentation performance on the ShapeNet, Kortx, Audi Autonomous Driving, and nuScenes datsets, which include both individual objects and large outdoor scenes captured with various sensor scanning patterns. These evaluations demonstrate that \arch scales to large scenes better than existing point-based backbones while improving object-level semantic segmentation compared to sparse-voxel backbones.
% We evaluate semantic segmentation on four datasets - two common (ShapeNet and nuScenes) and two less common (KortX and Audi Driving). Taken together, these datasets patterns -
% We evaluate semantic segmentation the ShapeNet, Kortx, Audi Driving, and nuScenes datasets. 

% demonstrate that \arch outperforms both point-based and sparse-voxel backbones in various segmentation settings, notably scaling better to large scenes than point-based alternatives while exhibiting better single object performance than sparse-voxel alternatives. 

% Evaluations on ShapeNet, Kortx, Audi Driving, and nuScenes demonstrate that \arch outperforms point-based methods on both individual objects and large-scale scenes, outperforms sparse-voxel backbones on individual objects, and closes the gap between point-based and sparse-voxel backbones on large-scale scenes while requiring significantly less GPU memory.

% CurveCloudNet is evaluated on several datasets that include both individual objects and large
% outdoor scenes captured with various sensor scanning patterns. These evaluations demonstrate that our model can
% outperform point-based and sparse-voxel backbones at both
% object and scene level, achieving state-of-the-art performance on segmentation tasks.
\vspace{-1em}
\end{abstract}
