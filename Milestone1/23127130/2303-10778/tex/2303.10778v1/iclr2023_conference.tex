

\documentclass{article} % For LaTeX2e
\usepackage{iclr2023_conference,times}

% Optional math commands from https://github.com/goodfeli/dlbook_notation.
\newcommand{\bbox}{\text{bbox}}
\newcommand{\alphapck}{\alpha_\bbox}
\newcommand{\kcycle}{\text{k-CyPCK}}
\newcommand{\cycle}{\text{-CyPCK}}

\newcommand{\I}{\mathbf{I}}
\newcommand{\Ia}{\I^\text{a}}
\newcommand{\Ib}{\I^\text{b}}
\newcommand{\Iatob}{\I^\text{a $\rightarrow$ b}}
\newcommand{\F}{\mathbf{F}}
\newcommand{\Fa}{\F^\text{a}}
\newcommand{\Fb}{\F^\text{b}}
\newcommand{\f}{\mathbf{f}}
\newcommand{\fa}{\f^\text{a}}
\newcommand{\fb}{\f^\text{b}}
\newcommand{\p}{\mathbf{p}}
\newcommand{\pa}{\p^\text{a}}
\newcommand{\pb}{\p^\text{b}}
\newcommand{\A}{\boldsymbol{\Phi}_\text{align}}
\newcommand{\G}{\mathbf{G}}
\newcommand{\C}{\mathbf{C}}
\newcommand{\Ca}{\C^\text{a}}
\newcommand{\Cb}{\C^\text{b}}
\newcommand{\cc}{\mathbf{c}}
\newcommand{\cca}{\cc^\text{a}}
\newcommand{\ccb}{\cc^\text{b}}
\newcommand{\Irec}{\I_\text{Recon}}
\newcommand{\M}{\mathbf{M}}
\newcommand{\Mrec}{\M_\text{Recon}}
\newcommand{\loss}{\mathcal{L}}
\newcommand{\T}{\mathcal{T}}
\newcommand{\W}{\mathcal{W}}
\newcommand{\Id}{\mathcal{I}}


\usepackage{hyperref}
\usepackage{url}
\usepackage{amsmath}
\usepackage{amssymb}
\usepackage{bm}
\usepackage{mathtools}
\usepackage{booktabs}       % professional-quality tables
\usepackage{amsfonts}       % blackboard math symbols
\usepackage{nicefrac}       % compact symbols for 1/2, etc.
\usepackage{microtype}      % microtypography
\usepackage{xcolor}         % colors
\usepackage{multirow}
\usepackage{wrapfig}
\usepackage{subcaption}
\usepackage{float}
% \usepackage[moderate]{savetrees}


\title{Deep Declarative Dynamic Time Warping for End-to-End Learning of Alignment Paths}

% Authors must not appear in the submitted version. They should be hidden
% as long as the \iclrfinalcopy macro remains commented out below.
% Non-anonymous submissions will be rejected without review.

\author{Ming Xu$^{1, 2}$ \quad Sourav Garg\textsuperscript{1} \quad Michael Milford$^{1}$ \quad Stephen Gould$^{2}$ \\
  $^{1}$Queensland University of Technology \quad $^{2}$Australian National University \\
  \texttt{\{mingda.xu, stephen.gould\}@anu.edu.au} \\
  \texttt{\{s.garg, michael.milford\}@qut.edu.au}
}

% The \author macro works with any number of authors. There are two commands
% used to separate the names and addresses of multiple authors: \And and \AND.
%
% Using \And between authors leaves it to \LaTeX{} to determine where to break
% the lines. Using \AND forces a linebreak at that point. So, if \LaTeX{}
% puts 3 of 4 authors names on the first line, and the last on the second
% line, try using \AND instead of \And before the third author name.

\newcommand{\fix}{\marginpar{FIX}}
\newcommand{\new}{\marginpar{NEW}}

\iclrfinalcopy % Uncomment for camera-ready version, but NOT for submission.
\begin{document}


\maketitle

\begin{abstract}
As models continue to grow in size, the development of memory optimization methods (MOMs) has emerged as a solution to address the memory bottleneck encountered when training large models. To comprehensively examine the practical value of various MOMs, we have conducted a thorough analysis of existing literature from a systems perspective. 
% Furthermore, we have evaluated the most widely adopted MOMs employed in mainstream frameworks for both vision and language models.
Our analysis has revealed a notable challenge within the research community: the absence of standardized metrics for effectively evaluating the efficacy of MOMs. The scarcity of informative evaluation metrics hinders the ability of researchers and practitioners to compare and benchmark different approaches reliably. Consequently, drawing definitive conclusions and making informed decisions regarding the selection and application of MOMs becomes a challenging endeavor.
To address the challenge, this paper summarizes the scenarios in which MOMs prove advantageous for model training. We propose the use of distinct evaluation metrics under different scenarios. By employing these metrics, we evaluate the prevailing MOMs and find that their benefits are not universal. We present insights derived from experiments and discuss the circumstances in which they can be advantageous.

\end{abstract}
\section{Introduction}


Recent years have witnessed the rise of human digitization~\cite{habermannDeepCapMonocularHuman2020,alexanderCREATINGPHOTOREALDIGITAL,pengNeuralBodyImplicit2021,alldieckDetailedHumanAvatars2018, rajANRArticulatedNeural2020}. This technology greatly impacts the entertainment, education, design, and engineering industry.
There is a well-developed industry solution for this task.
High-fidelity reconstruction of humans can be achieved either with full-body laser scans~\cite{saitoSCANimateWeaklySupervised2021}, dense synchronized multi-view cameras~\cite{xiangModelingClothingSeparate2021a,xiangDressingAvatarsDeep2022a}, or light stages~\cite{alexanderCREATINGPHOTOREALDIGITAL}.
However, these settings are expensive and tedious to deploy and consist of a complex processing pipeline, preventing the technology's democratization.

Another solution is to view the problem as inverse rendering and learn digital humans directly from custom-collected data.
Traditional approaches directly optimize explicit mesh representation~\cite{loperSMPLSkinnedMultiperson2015, fangRMPERegionalMultiperson2018, pavlakosExpressiveBodyCapture2019} which suffers from the problems of smooth geometry and coarse textures~\cite{prokudinSMPLpixNeuralAvatars2020,alldieckVideoBasedReconstruction2018}. Besides, they require professional artists to design human templates, rigging, and unwrapped UV coordinates.
Recently, with the help of volumetric-based implicit representations~\cite{mildenhallNeRFRepresentingScenes2020, parkDeepSDFLearningContinuous2019, meschederOccupancyNetworksLearning2019} and neural rendering~\cite{laineModularPrimitivesHighPerformance2020, liuSoftRasterizerDifferentiable2019, thiesDeferredNeuralRendering2019}, 
one can easily digitize a quality-plausible human avatar from video footage~\cite{jiangNeuManNeuralHuman2022,wengHumanNeRFFreeviewpointRendering}.
Particularly, volumetric-based implicit representations~\cite{mildenhallNeRFRepresentingScenes2020, pengNeuralBodyImplicit2021} can reconstruct scenes or objects with much higher fidelity against previous neural renderer~\cite{thiesDeferredNeuralRendering2019,prokudinSMPLpixNeuralAvatars2020}, and is more user-friendly as it does not need any human templates, pre-set rigging, or UV coordinates.
Captured visual footage and corresponding skeleton tracking are enough for training.
However, better reconstructions and more friendly usability are at the expense of the following factors.
1) \textbf{Inefficiency:}
They require longer optimization times (typically tens of hours or days) and inference slowly.
Volume rendering~\cite{mildenhallNeRFRepresentingScenes2020,lombardiNeuralVolumesLearning2019} formulates images by querying the densities and colors of millions of spatial coordinates. 
In the training stage, due to memory constraints, only a small fraction of points are sampled which leads to slow convergence speed.
2) \textbf{Entangled representations}:
The geometry, materials, and motion dynamics are entangled in the neural networks. 
Due to the implicit nature of neural nets, one can hardly edit one property without touching the others~\cite{yuanNeRFEditingGeometryEditing2022a,liuEditingConditionalRadiance2021}.
3) \textbf{Graphics incompatibility}:
Volume rendering is incompatible with the current popular graphic pipeline,
which renders triangular/quadrilateral meshes efficiently with the rasterization technique.
Many downstream applications require mesh rasterization in their workflow (\eg, editing~\cite{foundationBlenderOrgHome}, simulation~\cite{benderPositionBasedSimulationMethods2015}, real-time rendering~\cite{akenine2019real}, ray-tracing~\cite{waldRTXRayTracing}).
Although there are approaches~\cite{lorensenMarchingCubesHigh,labelleIsosurfaceStuffingFast2007} can convert volumetric fields into meshes, the gaps from discrete sampling degrade the output quality in terms of both meshes and textures.


To address these issues, we present \textbf{EMA}, a method based on \textbf{E}fficient \textbf{M}eshy neural fields to reconstruct animatable human \textbf{A}vatars.
Our method enjoys flexibility from implicit representations and efficiency from explicit meshes, yet still maintains high-fidelity reconstruction quality.
Given video sequences and the corresponding pose tracking, our method digitizes humans in terms of canonical triangular meshes, physically-based rendering (PBR) materials, and skinning weights \textit{w.r.t.} skeletons.
We jointly learn the above components via inverse rendering~\cite{laineModularPrimitivesHighPerformance2020,chenDIBRLearningPredict2021,chenLearningPredict3D2019} in an end-to-end manner.
Each of them is derived from a separate neural field, which relaxes the requirements of a preset human template, rigging, or UV coordinates.
Specifically, we predict a canonical mesh out of a signed distance field (SDF) by differentiable marching tetrahedra~\cite{shenDeepMarchingTetrahedra2021,gaoGET3DGenerativeModel,gaoLearningDeformableTetrahedral2020,munkbergExtractingTriangular3D2022}, then we extend the marching tetrahedra~\cite{shenDeepMarchingTetrahedra2021} for spatial-varying materials by utilizing a neural field to predict PBR materials \textit{on the mesh surfaces} after rasterization~\cite{munkbergExtractingTriangular3D2022,hasselgrenShapeLightMaterial2022,laineModularPrimitivesHighPerformance2020}.
To make the canonical mesh animatable, we take another neural field to model the forward linear blend skinning for the meshes. 
Given a posed skeleton, the canonical mesh is then transformed into the corresponding poses.
Finally, we shade the mesh with a rasterization-based differentiable renderer~\cite{laineModularPrimitivesHighPerformance2020} and train our models with a photo-metric loss.
After training, we export the mesh with materials and discard the neural fields.

\looseness=-1
There are several merits of our method design.
1) \textbf{Efficiency}:
Powered by efficient mesh rendering, our method can render in real-time.
Besides, the training speed is boosted as well, 
since we compute loss holistically on the whole image and the gradients only flow on the mesh surface. In contrast, volume rendering takes limited pixels for loss computation and back-propagates the gradients in the whole space.
Our method only needs about an hour of training and minutes of optimization are enough for plausible avatar reconstruction.
2) \textbf{Disentangled representations}:
Our shape, materials, and motion modules are disentangled naturally by design, which facilitates editing. 
Besides, Canonical meshes with forward skinning modeling handle the out-of-distribution poses better.
3) \textbf{Graphics compatibility}:
Our derived mesh representation is compatible with 
the prominent graphic pipeline, which leads to instant downstream applications (\eg, the shape and materials can be edited directly in design software~\cite{foundationBlenderOrgHome}).
To further improve reconstruction quality, we additionally optimize image-based environment lights and non-rigid motions.


We conduct extensive experiments on standards benchmarks H36M~\cite{ionescuHuman36MLarge2014b} and ZJU-MoCap~\cite{pengNeuralBodyImplicit2021}.
Our method achieves very competitive performance for novel view synthesis, generalizes better for novel poses, 
and significantly improves both training time and inference speed against previous arts.
Our research-oriented code reaches real-time inference speed (100+ FPS for rendering $512\times512$ images).
We in addition showcase applications including novel pose synthesis, material editing, and relighting.
\section{Related Work}
\mypara{Roadside Perception.}
Concurrent perception efforts for autonomous driving are mainly limited to the ego vehicle~\cite{caesar2020nuscenes, sun2020waymo}. While the roadside perception, which comparatively has a longer perceptual range and more robustness to occlusion and long-time event prediction, is mainly under-explored. Recently, some pioneers have present roadside datasets~\cite{ye2022rope3d, yu2022dair}, hoping to facilitate the 3D perception tasks in roadside scenarios. Compared with the vehicle perceptual system, which only observes surroundings in a short distance, the roadside cameras, mounted on poles a few meters above the ground, can provide long-range perception. However, the cameras mounted on roadside units have ambiguous mounting positions and 
%inconsistent intrinsic 
variable extrinsic parameters, which bring critical challenges to current perception models. In this paper, we take the advances and challenges of roadside cameras into account, and design an efficient and robust roadside perception framework, \name{}.


% \mypara{Vision-based Multi-View BEV perception.}
%  single-camera setting and multi-camera setting
\mypara{Vision Centric BEV Perception.}
Recent vision-centric works predict objects in 3D space, which is very suitable for applying multi-view feature aggregation under BEV for autonomous driving. Popular methods can be divided into transformer-based and depth-based schema. 
Following DETR3D~\cite{wang2022detr3d}, transformer-based detectors design a set of object queries~\cite{liu2022petr, liu2022petrv2, jiang2022polarformer, Chen2022PolarPF, Saha2022TranslatingII} or BEV grid queries\cite{li2022bevformer}, then perform the view transformation through cross-attention between queries and image features. 
Following LSS~\cite{philion2020lift}, depth-based methods ~\cite{reading2021cadnn, huang2021bevdet, huang2022bevdet4d} explicitly predict the depth distribution and use it to construct the 3D volumetric feature. Followup works introduce depth supervision from the LiDAR sensors ~\cite{li2022bevdepth} or multi-view stereo techniques ~\cite{wang2022sts, li2022bevstereo, park2022solofusion} to improve the depth estimation accuracy and achieve state-of-the-art performance. 
Additionally, transformer-based detectors' implicit 3D location information also can benefit from accurate depth cues. Inspired by~\cite{zhang2022monodetr,huang2022monodtr}, CrossDTR~\cite{tseng2022crossdtr} proposed depth-guided transformers, which compose depth-aware embedding from depth maps and are supervised by ground truth depth maps to enhance performance.
However, when applying these methods to roadside perception, the bonus of accurate depth information fades. As the complex mounting positions and variable extrinsic parameters of the roadside cameras, predicting depth from them is difficult. In this work, our \name{} utilizes the height estimation to achieve state-of-the-art performance and best robustness of roadside 3D object detection.


\vspace{-0.3em}
\section{Method}
\vspace{-0.3em}

Our sensitivity-aware visual parameter-efficient fine-tuning consists of two stages. In the first stage, SPT measures the task-specific sensitivity for the pre-trained parameters (Section~\ref{subsec:sensitivity}). Based on the parameter sensitivity and a given parameter budget, SPT then adaptively allocates trainable parameters to task-specific important positions (Section~\ref{subsec:SPT}).

\vspace{-0.3em}
\subsection{Task-specific Parameter Sensitivity}
\label{subsec:sensitivity}
\vspace{-0.3em}

Recent research has observed that pre-trained backbone parameters exhibit varying feature patterns~\cite{raghu2021vision,naseer2021intriguing} and criticality~\cite{zhang2019all,chatterji2019intriguing} at distinct positions. 
Moreover, when transferred to downstream tasks, their efficacy varies depending on how much pre-trained features are reused and how well they adapt to the specific domain gap~\cite{yosinski2014transferable,kumar2022finetuning,neyshabur2020being}. Motivated by these observations, we argue that not all parameters contribute equally to the performance across different tasks in PEFT and propose a new criterion to measure the sensitivity of the parameters in the pre-trained backbone for a given task.

Specifically, given the training dataset $\gD_t$ for the $t$-th task and the pre-trained model weights $\vw=\left\{w_1, w_2, \ldots, w_N\right\}\in \sR^N$ where $N$ is the total number of parameters, the objective for the task is to minimize the empirical risk: $\min_{\vw} E(\gD_t, \vw)$.
We denote the parameter sensitivity \bohan{set} as $\gS=\{s_1, \ldots, s_N\}$ and the sensitivity $s_n$ for parameter $w_n$ is measured by the empirical risk difference when tuning it:
\begin{equation}
\vspace{-0.3em}
    \begin{aligned}
        s_n = E(\gD_t, \vw)-E(\gD_t, \vw\mid w_n=w_n^*),
    \end{aligned}
\label{eq:sensitivity}
\end{equation}
where $w_n^*=\underset{w_n}{\rm argmin}(E(\gD_t, \vw))$. We can reparameterize the tuned parameters as  $w_n^*=w_n+\Delta_{w_n}$, where $\Delta_{w_n}$ denotes the update for $w_n$ after tuning. Here we individually measure the sensitivity of each parameter, which is reasonable given that most of the parameters are frozen during fine-tuning in PEFT. However, it is still computationally intensive to compute Eq.~(\ref{eq:sensitivity}) for two reasons. Firstly, getting the empirical risk for $N$ parameters requires forwarding the entire network $N$ times, which is time-consuming. Secondly, it is challenging to derive $\Delta_{w_n}$, as we have to tune each individual $w_n$ until convergence.

{\begin{algorithm}[t!]
\caption{\label{alg:tps} Computing task-specific parameter sensitivities}
\begin{algorithmic}
    \STATE \textbf{Input:} Pre-trained model with network parameters $\vw$, training set $\gD_t$ for the $t$-th task, and number of training samples $C$ used to calculate the parameter sensitivities
    \STATE \textbf{Output:} Sensitivity set $\gS=\{s_1, \ldots, s_N\}$
    \STATE Initialize $\gS=\{0\}^N$
    \FOR{$i\in\{1,\ldots,C\}$}
        \STATE Get the $i$-th training sample of $\gD_t$
	    \STATE Compute loss $E$
		\STATE Compute gradients $\vg$
		\FOR{$n\in\{1,\ldots,N\}$}
                \STATE Update sensitivity for the $n$-th parameter: $s_{n} = s_{n} + g_n^2$
		    \ENDFOR
    \ENDFOR
\end{algorithmic}
\end{algorithm}}


\begin{figure*}[t]
\begin{center}
    \includegraphics[width=\linewidth]{main_figure.pdf}
\end{center}\vspace{-2em}
\caption{Overview of our trainable parameter allocation strategy. With the parameter sensitivity \bohan{set} $\gS$, we first get the top-$\tau$ sensitive parameters. Instead of directly tuning these sensitive parameters, we also boost the representational capability by replacing unstructured tuning with structured tuning at sensitive weight matrices that have a large number of sensitive parameters, which can be implemented by an existing structured tuning method, \eg, LoRA~\cite{hu2022lora} and Adapter~\cite{houlsby2019parameter}. Red lines and blocks represent trainable parameters and modules, while blue lines represent frozen parameters.}
\label{fig:main}
\vspace{-1.5em}
\end{figure*}


To overcome the first barrier, we simplify the empirical loss by approximating $s_n$ in the vicinity of $\vw$ by its first-order Taylor expansion
\vspace{-0.3em}
\begin{equation}
\vspace{-0.5em}
    \begin{aligned}
        s_n^{(1)} = -g_n\Delta_{w_n},
    \end{aligned}
\label{eq:first-order}
\end{equation}
where the gradients $\vg=\partial E/\partial\vw$, and $g_n$ is the gradient of the $n$-th element of $\vg$. 
To address the second barrier, following~\cite{liu2018darts,cai2018proxylessnas}, we take the one-step unrolled weight as the surrogate for $w_n^*$ and approximate $\Delta_{w_n}$ in Eq.~(\ref{eq:first-order}) with a single step of gradient descent. We can accordingly get $s_n^{(1)} \approx g_n^2\epsilon$,
where $\epsilon$ is the learning rate. Since $\epsilon$ is the same for all parameters, we can eliminate it when comparing the sensitivity with the other parameters and finally get 
\vspace{-0.5em}
\begin{equation}
\vspace{-0.3em}
    \begin{aligned}
        s_n^{(1)} \approx g_n^2.
    \end{aligned}
\label{eq:first-order-simp}
\end{equation}
Therefore, the sensitivity of a parameter can be efficiently measured by its potential to reduce the loss on the target domain. Note that although our criterion draws inspiration from pruning work~\cite{molchanov2019importance}, it is distinct from it. \cite{molchanov2019importance} measures the parameter importance by the squared change in loss when removing them, \ie, $\left( E(\gD_t, \vw)-E(\gD_t, \vw\mid w_n=0) \right)^2$ and finally derives the parameter importance by $\left( g_n w_n \right)^2$, which is different from our formulations in Eqs.~(\ref{eq:sensitivity}) and~(\ref{eq:first-order-simp}).

In practice, we accumulate $\gS$ from a total number of $C$ training samples ahead of fine-tuning to generate accurate sensitivity as shown in Algorithm~\ref{alg:tps}, where $C$ is a pre-defined hyper-parameter. In Section~\ref{subsec:abl}, we show that employing only 400 training samples is sufficient for getting reasonable parameter sensitivity, which requires only 5.5 seconds with a single GPU for any VTAB-1k dataset with ViT-B/16 backbone~\cite{vit}.

\vspace{-0.3em}
\subsection{Adaptive Trainable Parameters Allocation}
\label{subsec:SPT}
\vspace{-0.2em}

Our next step is to allocate trainable parameters based on the obtained parameter sensitivity set $\gS$ and a desired parameter budget $\tau$. A straightforward solution is to directly tune the top-$\tau$ most sensitive unstructured connections (parameters) \rev{while keeping the rest frozen}, which we name unstructured tuning. Specifically, we select the top-$\tau$ most sensitive weight connections in $\gS$ to form the sensitive weight connection set $\gT$. Then, for \rev{a} weight matrix $\mW\in \sR^{d_{\rm in}\times d_{\rm out}}$, we can get a binary mask $\mM\in \sR^{d_{\rm in}\times d_{\rm out}}$ computed by
\vspace{-0.5em}
\begin{equation}
\vspace{-0.5em}
    {\begin{array}{ll}
    \small
    \begin{aligned}
    \mM^j =
    \left\{\begin{array}{ll} 
    1 ~~~~~ \mW^j \in \gT \\
    0 ~~~~~ \mW^j \notin \gT
    \end{array}\right.
    \end{aligned},
    \small
    \end{array}}
\label{eq:mask}
\end{equation}
where $\mW^j$ and $\mM^j$ are the $j$-th element in $\mW$ and $\mM$, respectively. Accordingly, we can train the sensitive parameters by gradient descent and the updated weight matrix can be formulated as $\mW'\leftarrow \mW - \epsilon\vg_{\mW}\odot\mM$, where $\vg_{\mW}$ is the gradient for $\mW$.

However, considering PEFT approaches generally limit the proportion of trainable parameters to less than 1\%, tuning only a small number of unstructured weight connections might not have enough representational capability to handle the downstream datasets with large domain gaps from the source pre-training data. Therefore, to improve the representational capability, we propose to replace unstructured tuning with structured tuning at the sensitive weight matrices that have a high number of sensitive parameters. To preserve the parameter budget, we can implement structured tuning with an existing efficient structured tuning PEFT method~\cite{hu2022lora,chen2022adaptformer,houlsby2019parameter,jie2022convolutional} that learns to directly adjust \rev{all hidden dimensions at once}. We depict an overview of our trainable parameter allocation strategy in Figure~\ref{fig:main}. For example, we can employ the low-rank reparameterization trick LoRA~\cite{hu2022lora} to the sensitive weight matrices \rev{and the one-step update for $\mW$ can be formulated as}
\vspace{-0.4em}
\begin{equation}
\vspace{-0.4em}
    {\begin{array}{ll}
    \small
    \begin{aligned}
    \mW' = \left\{\begin{array}{ll} 
    \mW + \mW_{\rm down}\mW_{\rm up} & ~~ \text { if } ~~ \sum_{j=0}^{d_{\rm in}\times d_{\rm out}} \mM^j \geq \sigma_{\rm opt} \\
    \mW - \epsilon\vg_{\mW}\odot\mM & ~~ {\rm otherwise}
    \end{array}\right.
    \end{aligned},
    \small
    \end{array}}
\label{eq:weight_updat}
\end{equation}
where $\mW_{\rm down}\in \sR^{d_{\rm in}\times r}$ and $\mW_{\rm up}\in \sR^{r\times d_{\rm out}}$ are two learnable low-rank matrices to approximate the update of $\mW$ and rank $r$ is a hyper-parameter where $r \ll {\rm min}(d_{\rm in},d_{\rm out})$. In this way, we perform structured tuning on $\mW$ when its number of sensitive parameters exceeds $\sigma_{\rm opt}$, whose value depends on the pre-defined type of structured tuning method. For example, since implementing structured tuning with LoRA requires $2\times d_{\rm in} \times d_{\rm out} \times r$ trainable parameters for each sensitive weight matrix, we set $\sigma_{\rm LoRA} \leftarrow 2\times d_{\rm in} \times d_{\rm out} \times r$ to ensure that the number of trainable parameters introduced by structured tuning is always equal to or lower than the number of sensitive parameters.

In this way, our SPT adaptively incorporates both structured and unstructured tuning granularities to enable higher flexibility and stronger representational power, simultaneously. In Section~\ref{subsec:abl}, we show that structured tuning is important for the downstream tasks with larger domain gaps and both unstructured and structured tuning contribute clearly to the superior performance of our SPT.
\begin{table*}[t!]
\begin{minipage}{0.175\linewidth}
\centering
% \hspace{1.8mm}
\captionof{table}{\small Datasets statistics \label{graph_datasets}}
\begin{tiny}
\begin{tabular}{c||c|c}
      {\bf Graph} & {\bf \#nodes} & {\bf \#edges} \\ \hline
      {\em FL} & 80\,513     & 5\,899\,882 \\
      {\em YT} & 1\,138\,499 & 2\,990\,443 \\
      {\em LJ} & 2\,238\,731 &14\,608\,137 \\
      {\em OR} & 3\,072\,441 & 117\,185\,083 \\
      {\em TW} & 41\,652\,230 & 1\,468\,365\,182 \\
\end{tabular}
\end{tiny}
\end{minipage}%
 \quad
 \begin{minipage}{.265\linewidth}
\centering
\tabcolsep=0.05cm

\captionof{table}{\small Avg. memory footprint (GB) of {\sf DistGER} and {\sf KnightKing} on each machine, where $\sigma$ is the standard deviation}
\label{Memory_usage}
\begin{tiny}
\newcommand{\tabincell}[2]{\begin{tabular}{@{}#1@{}}#2\end{tabular}}
  % \caption{\small {\color{blue} Avg. memory footprint (GB) of {\sf DistGER} and {\sf KnightKing} on each machine, where $\sigma$ is the standard deviation.}}
  \begin{tabular}{c|cc|cc}
    %\hline
    { }&\multicolumn{2}{c|}{\bfseries{ Sampling}}&\multicolumn{2}{c}{\bfseries{Training}}\\
    \hline
    {\bf{Graph}} &{\sf KnightKing} &{\sf DistGER} &{\sf KnightKing} &{\sf DistGER} \\
    \hline
     {\em FL} & 0.66($\pm$0.06)	&{\bf 0.41($\pm$0.02)}	&1.31($\pm$0.17) 	&{\bf 0.86($\pm$0.06)} 	\\

     {\em YT} &4.11($\pm$0.55)	&{\bf 1.36($\pm$0.23)} 	&4.73($\pm$0.72) 	&{\bf 4.26($\pm$0.63)} \\

     {\em LJ} & 7.65($\pm$0.82)	&{\bf 1.95($\pm$0.16)}	&6.38($\pm$0.97) 	&{\bf 5.49($\pm$0.85)} 	\\

     {\em CO} &10.98($\pm$1.03)	&{\bf 3.27($\pm$0.79)} 	&8.52($\pm$1.01) 	&{\bf 6.86($\pm$0.69)} 	\\

     {\em TW} & out-of-memory	&{\bf 20.18($\pm$3.62)} 	&out-of-memory 	& {\bf 67.16($\pm$5.18)} 	\\
  %\hline
\end{tabular}
\end{tiny}

\end{minipage}
\quad
\begin{minipage}{.25\linewidth}
    \centering
    \includegraphics[width= 1.85in, height = 1.2in]{./Figures/Dist_total_time_partition.eps}%
    \captionof{figure}
      {\small Efficiency: {\sf PBG} \cite{PBG_2019}, {\sf DistDGL} \cite{DistDGL_2020}, {\sf KnightKing} \cite{KnighKing_2019}, {\sf HuGE-D} (baseline), {\sf DistGER} (ours)
        \label{overall_performance}
      }
\end{minipage}%\hfill
\quad
\begin{minipage}{.25\linewidth}
    \centering
    \includegraphics[width= 1.85in, height = 1.2in]{./Figures/Dist_scalability_partition.eps}%
    \captionof{figure}
      {\small Scalability: {\sf PBG} \cite{PBG_2019}, {\sf DistDGL} \cite{DistDGL_2020}, {\sf KnightKing} \cite{KnighKing_2019}, {\sf HuGE-D} (baseline), {\sf DistGER} (ours)
        \label{Dist_scalability}
      }
\end{minipage}
\end{table*}


\section{Experimental Results}
\label{sec:experiments}
We evaluate the efficiency (\S \ref{sec:overall}) and scalability (\S \ref{sec:scalability}) of our proposed method, {\sf DistGER}
by comparing with {\sf HuGE-D} (baseline),
{\sf KnightKing} \cite{KnighKing_2019}, {\sf PyTorch-BigGraph} ({\sf PBG}) \cite{PBG_2019}, and {\sf Distributed DGL}
({\sf DistDGL}) \cite{DistDGL_2020}. We also compare the effectiveness (\S \ref{sec:effectiveness}) of generated embeddings
on link prediction.
% and multi-label classification tasks. 
Finally, we analyze efficiency due to individual
parts of {\sf DistGER} (\S \ref{sec:individual})
and the generality of {\sf DistGER} for other random walk-based embeddings (\S \ref{sec:generality}).
Our codes and datasets are at \cite{code}.
%
\subsection{Experimental Setup}
\label{sec:setup}
%
\spara{Environment.} We conduct experiments on a cluster of 8 machines with 2.60GHz Intel $^\circledR$ Xeon $^\circledR$ Gold 6240 CPU with 72 cores (hyper-threading)
in a dual-socket system, and each machine is equipped with 192GB DDR4 memory and connected by a 100Gbps network.
The machines run Ubuntu 16.04 with Linux kernel 4.15.0. We use GCC v9.4.0 for compiling {\sf DistGER}, {\sf KnightKing}, and {\sf HuGE-D},
and use Python v3.6.15 and torch v1.10.2 as the backend deep learning framework for {\sf Pytorch-BigGraph} and {\sf DistDGL}.

\spara{Datasets. } We employ five widely-used, real-world graphs
(Table~\ref{graph_datasets}): {\em Flickr} (FL) \cite{Flickr_Youtube_Graph},
{\em Youtube} (YT) \cite{Flickr_Youtube_Graph},
{\em LiveJournal} (LJ) \cite{BlogCatalog_Twitter_LiveJournal_Graph},
{\em Com-Orkut} (OR) \cite{com-orkut_2012}, and {\em Twitter} (TW) \cite{twitter_2010}.
The first two graphs are selected for multi-label node classification with distinct number of node labels 195 and 47, respectively, %in {\em Flickr} and {\em Youtube},
where labels in {\em Flickr} represent interest groups of users, and {\em Youtube}'s labels represent groups of viewers that enjoy common video genres. The last four graphs are used in link prediction. We also use synthetic graphs \cite{RMAT_2004} (up to 1 billion nodes, 10 billion edges) and a real-world {\em UK graph} \cite{BSVLTAG} (100M nodes, 3.7B edges) to assess the scalability of {\sf DistGER}.
Considering the default settings of popular random walk-based methods (e.g., Deepwalk, node2vec, HuGE), we use their undirected version.

\spara{Competitors.} We compare {\sf DistGER} against three state-of-the-art distributed graph embedding frameworks: the distributed random walk engine, {\sf KnightKing} {\scriptsize\url{https://github.com/KnightKingWalk/KnightKing}}
\cite{KnighKing_2019}; the distributed multi-relations based graph embedding system, {\sf PyTorch-BigGraph} ({\sf PBG})
{\scriptsize\url{https://github.com/facebookresearch/PyTorch-BigGraph}} \cite{PBG_2019} -- designed by Facebook; and
the distributed graph neural networks-based system, {\sf DistDGL} {\scriptsize\url{https://github.com/dmlc/dgl}}
\cite{DistDGL_2020} -- recently proposed by Amazon. We also implement {\sf HuGE-D}, a distributed version of
information-centric random walk-based graph embedding ({\sf HuGE} \cite{HuGE_2021}), on top of {\sf KnightKing},
served as our baseline. Since {\sf KnightKing} and {\sf HuGE-D} provide distributed support only for
random walk without that for embedding learning, we generate their node embeddings using
{\sf Pword2vec} {\scriptsize\url{https://github.com/IntelLabs/pWord2Vec}} \cite{Pword2vec_2019},
the most popular distributed {\sf Skip-Gram} system released by Intel.
%We find that {\sf pSGNScc} \cite{pSGNSCC_2017} (\S \ref{sec:learning})
%only provides a single-machine implementation, thus we do not include it in our distributed experiments.

\spara{Parameters.} For {\sf DistGER} and {\sf HuGE-D} random walks, we set
parameters $\mu$=0.995, $\delta$=0.001 based on information measurements (\S \ref{sec:preliminaries}),
while {\sf KnightKing} uses $L$=80 and $r$=10 that are routine configurations in the traditional
random walk-based graph embedding \cite{node2vec_2016, DeepWalk_2014, KnighKing_2019}. For {\sf DistGER}, {\sf KnightKing}, and {\sf HuGE-D} training,
we set the sliding window size $w$=10, number of negative samples $K$=5, and synchronization period=0.1 sec \cite{Pword2vec_2019},
and additionally, multi-windows number=2, $\gamma$=2 for {\sf DisrGER}.
%For {\sf Pytorch-BigGraph} ({\sf PBG}), we set the number of partitions to 16 following \cite{PBG_2019}, that is, using $2m$ partitions for the number of machines $m$ = 8
%in our case. %For {\sf DistDGL}, the deployed {\sf GaphSAGE} model uses three graph convolutional layers.
For fair comparison across all systems, %the efficiency performance of all systems involved in the experiments,
we set the embedding dimension $d$=128 that is commonly used \cite{HuGE_2021,node2vec_2016,DeepWalk_2014,Line_2015,Verse_2018,ProNE_2019},
and report the average running time for each epoch. For task effectiveness evaluations,
we find the best results from a grid search over learning rates from 0.001-0.1, \# epochs from 1-30,
and \# dimensions from 128-512.


%
\eat{
\begin{table}
\newcommand{\tabincell}[2]{\begin{tabular}{@{}#1@{}}#2\end{tabular}}
  \caption{\small Avg. memory footprint (GB) of {\sf DistGER} and {\sf KnightKing} on each machine, where $\sigma$ is the standard deviation.}
  \label{Memory_usage}
  \begin{center}
   \footnotesize
  \begin{tabular}{c|cc|cc}
    %\hline
    { }&\multicolumn{2}{c|}{\bfseries{ Sampling}}&\multicolumn{2}{c}{\bfseries{Training}}\\
    \hline
    {\bf{Graph}} &{\sf KnightKing} &{\sf DistGER} &{\sf KnightKing} &{\sf DistGER} \\
    \hline
     {\em Flickr} & 0.66($\pm$0.06)	&{\bf 0.41($\pm$0.02)}	&1.31($\pm$0.17) 	&{\bf 0.86($\pm$0.06)} 	\\

     {\em Youtube} &4.11($\pm$0.55)	&{\bf 1.36($\pm$0.23)} 	&4.73($\pm$0.72) 	&{\bf 4.26($\pm$0.63)} \\

     {\em LiveJournal} & 7.65($\pm$0.82)	&{\bf 1.95($\pm$0.16)}	&6.387($\pm$0.97) 	&{\bf 5.49($\pm$0.85)} 	\\

     {\em Com-Orkut} &10.98($\pm$1.03)	&{\bf 3.27($\pm$0.79)} 	&8.52($\pm$1.01) 	&{\bf 6.86($\pm$0.69)} 	\\

     {\em Twitter} & out-of-memory	&{\bf 37.1($\pm$5.28)} 	&out-of-memory 	& {\bf 79.5($\pm$7.27)} 	\\
  %\hline
\end{tabular}
\end{center}
\end{table}
%
}
%


%
\subsection{Efficiency and Memory Use w.r.t. Competitors}
\label{sec:overall}
%\begin{figure}
%  \centering
%  \includegraphics[width= 3 in]{Dist_total_time.eps}
%  \caption{\small Overall performance of PBG, DistDGL, KnightKing, HuGE-D and DistGER for generating embeddings on different read-word graphs, {\color{blue}for Twitter graph, DistDGL cannot finish in one day, and KnightKing fails to perform due to memory issue, where the y axis is in log-scale.}}
%  \label{overall_performance}
%\end{figure}
%
We report the end-to-end running times of {\sf PBG}, {\sf DistDGL}, {\sf KnightKing}, {\sf HuGE-D}, and {\sf DistGER}
on five real-world graphs with the cluster of 8 machines in Figure~\ref{overall_performance}.
The reported end-to-end time includes the running time of partitioning, random walks (for random walk-based frameworks), and training procedures.
%{\color{blue} Noted that the reported end-to-end time in our experiments excludes the partition time for all evaluated frameworks due to the all used partition schemes are executed as a preprocessing component, and we separately evaluate the partition efficiency in Section 6.5, thus the end-to-end time refers to the running time of random walk (only for random walk-based framework) and training procedure.}
{\sf DistGER} significantly outperforms the competitors
on all these graphs, achieving a speedup ranging from 2.33$\times$ to 129$\times$. %, by an average acceleration of $39.78 \times$.
Recall that {\sf DistGER} is a similar type of system as {\sf KnightKing} and {\sf HuGE-D},
and our key improvements are discussed in \S \ref{sec:DistGER} and in \S \ref{sec:learning}.
Analogously, Figure~\ref{overall_performance} exhibits that our system, %designs are more effective (see more evaluation details in \S 6.3),
{\sf DistGER} achieves an average speedup of 9.25$\times$ and 6.56$\times$ compared with {\sf KnightKing} and {\sf HuGE-D}.
Notice that we fail to run {\sf KnightKing} on the largest {\em Twitter} dataset
because its routine random walk strategy requires more main memory space.
%Although Huge-D achieves comparable performance,
The advantage of information-centric random walk in {\sf HuGE} is almost wiped out in {\sf HuGE-D}
due to on-the-fly information measurements and the higher communication costs in a distributed setting.
The multi-relation-based {\sf PBG} leverages a parameter server to synchronize embeddings between clients,
resulting in more load on the communication network. As a result, {\sf PBG} is on average
26.22$\times$ slower than {\sf DistGER}. For graph neural network-based system {\sf DistDGL},
due to the long running time of graph sampling (e.g., taking 80\% of the overhead for the {\sf GraphSAGE}),
it is highly inefficient than other systems. For the billion-edge {\em Twitter} graph, it does not terminate in 1 day.
%
%Considering the resource consumption that affects scalability,
{Table ~\ref{Memory_usage}} shows {\sf DistGER}'s average memory footprint on each machine of the 8-machine cluster. %from a cluster of 8 machines.
%and the standard deviation %($\sigma$) %of the results
%in 
%Table ~\ref{Memory_usage}. 
Compared
to %other methods, %with the 
same type of system
{\sf KnightKing}, 
% that is of the same system type, 
{\sf DistGER} requires less memory for sampling and training.


\subsection{Scalability w.r.t. Competitors}
\label{sec:scalability}
%
%\begin{figure}
%  \centering
%  \includegraphics[width= 3.2 in]{Dist_scalability.eps}
%  \caption{\small Scalability comparison on LiveJournal graph, where the y axis is in log-scale.}
% \label{Dist_scalability}
%\end{figure}
%
Figure~\ref{Dist_scalability} shows end-to-end running times of all competing
systems on the {\em LiveJournal} graph, as we increase \# machines
from 1 to 8 to evaluate scalability. {\sf DistGER} achieves better scalability than the other
four distributed systems.
%Due to space limitation, we omit results on other graph datasets,
%which exhibit similar trends.
{\sf PBG} leverages a parameter server and a shared network filesystem
to synchronize the parameters in the distributed model. %The edges are partitioned into $m^2$ buckets
%and training can be performed in parallel using up to $m/2$ machines. After one bucket completes
%the training, it needs to communicate with the parameter server.
When the number of machines increases, {\sf PBG} puts more load
on the communications network, resulting in poor scalability. Likewise, {\sf DistDGL}
is bounded by the synchronization overhead for gradient updates,
limiting its scalability.
%Since {\sf DistDGL} uses mini-batches for sampling, %features %for GraphSAGE,
%if the mini-batch samples cannot be generated on time, the trainer will be delayed on the forward pass, and all other
%machines need to wait before starting the backward pass. Thus, increasing the number of
%machines also affects the efficiency of backward pass. %Being the random walk-based distributed systems,
Both {\sf KnightKing} and {\sf HuGE-D} suffer from higher communication costs during random walks,
due to their only workload-balancing partitioning scheme (\S \ref{sec:dRand}, \S \ref{sec:individual}).
%Their scalability is relatively poor as the number of machines increases.
%{\sf KnightKing} partitions the graph by a workload-balancing scheme, inevitably introducing higher
%cross-machine communications due to the randomness inherent in the random walking procedure (\S \ref{sec:dRand}, \S \ref{sec:individual}).
%With more machines, the inefficiency of the partitioning scheme is further magnified.
Since {\sf HuGE-D} is implemented on top of {\sf KnigtKing},
it exhibits worse scalability due to high communication costs and on-the-fly information measurements in a distributed setting (\S \ref{sec:HUGED}).
%In contrast, its performance is much better than all the competitors in a single machine.
In comparison, {\sf DistGER} incorporates multi-proximity-aware streaming graph partitioning and incremental computations
to reduce both communication and computation costs, it also employs hotness-block based parameters synchronization
during training to dramatically reduce the pressure on network bandwidth. Hence, {\sf DistGER} achieves better scalability than other systems.
Due to space limitations, we omit {\sf DistGER}'s scalability results on other graphs, which exhibit similar trends. On {\em Twitter}, the end-to-end running times {\sf DistGER} on 1, 2, 4, and 8 machines are 3090s, 1739s, 1197s, and 746s, respectively,
while on {\em Com-Orkut}, the results are 304s, 204s, 149s, and 89s, respectively. 
The results show a good linear relationship.
% The results demonstrate a desired scalability with the increase of the machines.

\begin{table}
\quad
\begin{minipage}{0.46\linewidth}
    \centering
    \includegraphics[width= 1.6 in]{./Figures/Dist_scalability_datasize.eps}%
    \captionof{figure}
      {\small {Scalability of {\sf DistGER} on synthetic graphs, where Y-axis is in log-scale}}
      %The lines depict the running time required for random walk (blue line) and training (red line), respectively. Pentagrams show the time cost of six real-world graphs,
        \label{Dist_scalability_data}
      
\end{minipage}\hfill
\quad
\begin{minipage}{.46\linewidth}
    \centering
    \includegraphics[width= 1.6 in]{./Figures/Dist_time_auc.eps}%
    \captionof{figure}
      {\small {The influence of running time on embedding quality for {\sf DistGER} and competitors}}
        \label{Dist_time_auc}
\end{minipage}
\end{table}


To further assess the scalability of {\sf DistGER}, we generate synthetic graphs \cite{RMAT_2004} with a fixed node degree of 10 and the number of nodes from $10^5$ to $10^9$. Figure~\ref{Dist_scalability_data} presents the running times for random walks and training on these synthetic graphs using a cluster of 8 machines, suggesting that the running time increases linearly with the size of a graph, and {\sf DistGER} has the capability to handle even billion-node graphs. Moreover, the running times for six real-world graphs (including the {\em UK graph} with $|E|=3.7B$, $|V|=100M$, for which the competing systems do not terminate in 1 day or crash due to hardware and memory limitation) are inserted into the plot, which is consistent with the trend on synthetic data.

%
%
\subsection{Effectiveness w.r.t. Competitors}
\label{sec:effectiveness}
%
\spara{Link prediction.} To perform link prediction on a given graph $G$, following \cite{HuGE_2021,node2vec_2016,Verse_2018,NRP_2020},
we first uniformly at random remove 50\% edges as positive test edges, and the rest are used as positive training edges.
We also provide negative training and test edges by considering those node pairs between which no edge exists in $G$.
We ensure that the positive and negative set sizes are similar. %For a pair of nodes $(u, v)$, let $\varphi(u)$ and
%$\varphi(v)$ be the vectors learned by embedding methods.
The link prediction is conducted as a classification task
based on the similarity of $u$ and $v$, i.e., $\varphi(u)\cdot\varphi(v)$.
The effectiveness of link prediction is measured via the $AUC$ (Area Under Curve) score \cite{AUC_kdd} -- the higher the better.
We repeat this procedure 50 times to offset the randomness of edge removal and report the average $AUC$ in
Table~\ref{AUC_results}.
%shows $AUC$ for all the methods on five real-world graphs.
%, respectively, where a ``$-$'' indicates that the method fails due to the limitation of computing resources or because its running time exceeds 1 day.
{\sf DistGER} outperforms all competitors on these graphs, except for {\sf PBG} on {\em Com-Orkut}, where {\sf DistGER} ranks second.
On average, {\sf DistGER} has an 11.7\% higher $AUC$ score compared with the other three systems, thanks to our
information-centric random walks. {\sf PBG} is the best on {\em Com-Orkut} because this graph is much denser
and is friendly to the multi-relationship-based model in {\sf PBG}.
Figure~\ref{Dist_time_auc} exhibits accuracy-efficiency tradeoffs of {\sf DistGER} and competitors, i.e., their $AUC$ convergence curves w.r.t. increasing running times of random walks and training, over {\em LiveJournal}, further indicating
that {\sf DistGER} has better efficiency and effectiveness than the competitors.
%As a system of the same type, DistGER achieves better accuracies on all graphs than KnightKing which leverages the routine random walk configuration, thanks to its information-centric random walk strategies. We do not report the effectiveness of HuGE-D here because it uses the same random walk model as DistGER.
%
\begin{table}[h!]
\newcommand{\tabincell}[2]{\begin{tabular}{@{}#1@{}}#2\end{tabular}}
  \caption{\small $AUC$ scores of {\sf DistGER} and competitors for link prediction}
  \label{AUC_results}
  \begin{center}
  \footnotesize
  \begin{tabular}{cccccc}
%    \hline
    {Method}&\tabincell{c}{Youtube}&{LiveJournal}&\tabincell{c}{Com-Orkut}&{ Twitter}\\
    \hline
    {\sf PBG}        & 0.753           &0.882            &\bfseries{0.955} &0.912\\

    {\sf DistDGL}    &0.894            &0.718            &0.815            & running time $>$ 1 day \\

    {\sf KnightKing} &0.904            &0.963            & $0.918$         & out-of-memory\\

    {\sf DistGER}    &\bfseries{0.966} &\bfseries{0.976} &0.921            &\bfseries{0.919}\\
%  \hline
\end{tabular}
\end{center}
\end{table}

% \eat{
\spara{Multi-label node classification.}
This task predicts one or more labels for each graph node and has applications in %modern applications ranging from
text categorization \cite{zhang2006multilabel} and bioinformatics \cite{zhang2018ontological}.
We use embedding vectors and a one-vs-rest logistic regression classifier
with L2 regularization \cite{MLC_LIBLINEAR_2008}, %(using the LIBLINEAR library),
then evaluate the effectiveness by micro-averaged F1 ($Micro-F1$) and macro-averaged F1 ($Macro-F1$) \cite{WangC016}
scores, where $Micro-F1$ gives equal weight to each test instance and $Macro-F1$ assigns equal weight to each label category \cite{keikha2018community}.
%To train a classifier, nodes are uniformly at random split into training and test sets.
Following \cite{HuGE_2021,node2vec_2016,DeepWalk_2014,Line_2015,Verse_2018},
we select 10\% to 90\% training data ratio on {\em Flickr}, and 1\% to 9\% training ratio on {\em Youtube}.
%and the remaining nodes for testing.
We report the averaged $Macro-F1$ and $Micro-F1$ scores from 50 trials in Figure~\ref{Dist_MLC_mac_mic_F1}.
% shows the $Macro-F1$ and $Micro-F1$ scores achieved by each system as a
%function of the training ratio variation, respectively.
We find that {\sf DistGER} has better $Macro-F1$ and $Micro-F1$ scores
than existing frameworks, %on these graphs, %. In particular, compared with the KnightKing,
%DistGER consistently outperforms the other random walk-based systems on all graphs in $Macro-F1$ and $Micro-F1$ scores,
gaining 9.2\% and 3.3\% average improvements, respectively, due to its more effective information-centric random walks.
%Definition of $Macro-F1$ and $Micro-F1$ are as the following:

\begin{figure}[h!]
  \centering
  \includegraphics[width= 3.45 in]{./Figures/Dist_MLC_mac_mic_F1_1.eps}
  \caption{\small $Macro-F1$ (a1, b1) and $Micro-F1$ (a2, b2) scores for multi-label node classification. $X$-axis: training data ratio}
  \label{Dist_MLC_mac_mic_F1}
\end{figure}

% }
%\begin{equation}
%Precision = \frac{\sum\nolimits_{i}^{K}TP(i)}{\sum\nolimits_{i}^{K}(TP(i)+FP(i))}
%\end{equation}
%
%\begin{equation}
%Recall = \frac{\sum\nolimits_{i}^{K}TP(i)}{\sum\nolimits_{i}^{K}(TP(i)+FN(i))}
%\end{equation}
%
%\begin{equation}
%Micro-F1 = \frac{2\times Precision\times Recall}{Precision+Recall}
%\end{equation}
%
%\begin{equation}
%Macro-F1 = \frac{\sum\nolimits_{i}^{K}Micro-F1(i)}{|K|}
%\end{equation}
%
%where $TP(i)$, $FP(i)$ and $FN(i)$ are the number of true positives, false positives and false negatives in the instances which are predicted as $i$, respectively. Suppose $K$ is the overall label set, $Micro-F1$($i$) and $Macro-F1$ are the measure of $Micro-F1$ and $Macro-F1$ for the label $i$, respectively.
%\begin{table}
%\setlength{\abovecaptionskip}{0.cm}
%\setlength{\belowcaptionskip}{-0.cm}
%\newcommand{\tabincell}[2]{\begin{tabular}{@{}#1@{}}#2\end{tabular}}
%  \caption{$Macro-F1$ and $Micro-F1$ for multi-label classification on Flickr and Youtube graph, where train ratio is 0.5.}
%  \label{cluster_results}
%  \begin{center}
%  \small
%  \begin{tabular}{ccccc}
%    \hline
%    { }&\multicolumn{2}{c}{\bfseries{ \scriptsize Flickr}}&\multicolumn{2}{c}{\bfseries{\scriptsize Youtube}}\\
%    \hline
%    { }&Macro-F1 &Micro-F1&Macro-F1 &Micro-F1\\
%
%    \hline
%    \small PBG & 0.225	&0.387 	&0.295 	&0.406 	\\
%
%    \small DistDGL &0.205 	&0.378 	&0.283 	&0.403 	\\
%
%    \small KnightKing &0.239    &0.386 &0.285 	&0.402 	\\
%
%    \small DistGER  &\bfseries{0.277} &\bfseries{0.409}&\bfseries{0.298} &\bfseries{0.417}\\
%
%  \hline
%\end{tabular}
%\end{center}
%\end{table}
%

\begin{figure}
  \centering
  \includegraphics[width= 3.2 in]{./Figures/Dist_sampling_training_Mpad_efficiency.eps}
  \caption{\small {(a) Random walk efficiency, (b) training efficiency, (c) \# cross-machine messages, (d) random walk efficiency for {\sf MPGP} (ours) and workload-balancing scheme ({\sf KnightKing})}}
  \label{Dist_efficiency_sampling_training_MPGP}
\end{figure}

\begin{table*}[t!]
\begin{minipage}{0.275\linewidth}
\centering
\renewcommand\arraystretch{1.2}
\captionof{table}{\small Performance evaluation of partitioning for {\sf DistGER} and Competitors } %{\sf PBG} and {\sf DistDGL}
\label{Partition_sechme_overhead}
\begin{scriptsize}
\begin{tabular}{ccccc}
    \multicolumn{5}{c}{{\bfseries (a) Partitioning time for {\sf DistGER} and competitors }} \\
    \hline
    {\bf graph} & {\sf PBG} & {\sf DistDGL} & {\sf DistGER}\\
                &           & ({\sf METIS}) &  ({\sf MPGP}) \\
    \hline
    {\sf FL} & 383.28 s & 127.72 s & \bfseries{15.96 s} \\
    {\sf YT} & 349.15 s & 116.30 s & \bfseries{13.56 s} \\
    {\sf LJ} & 458.52 s & 425.19 s & \bfseries{36.42 s} \\
    {\sf OR} & 2662.62 s & 2761.25 s &\bfseries{294.68 s}\\
    {\sf TW} & 22 hour s & $>$ 1 day &\bfseries{9 hours}\\
    \hline
%    \multicolumn{5}{c}{} \\
    \multicolumn{5}{c}{{\bfseries (b) Evaluation of {\sf Parallel MPGP} }} \\
    \hline
    {\bf graph} & {\sf Streaming} & {\sf Partitioning} & {\sf Walking}\\
    \hline
  %  {\sf MPGP}   &DFS+deg  & 9 hours & \bfseries{575.22 s} \\
    \multirow{2}{*}{\sf LJ} &DFS+deg & 21.86 s & \bfseries{23.78 s} \\
           & BFS+deg & \bfseries{21.25 s} & 24.79 s \\
    \multirow{2}{*}{\sf OR} &DFS+deg & \bfseries{151.29 s} & 77.12 s \\
           & BFS+deg & 156.37 s & \bfseries{46.55 s} \\
    \multirow{2}{*}{\sf TW} &DFS+deg & \bfseries{1940.65} s & 683.81 s \\
           & BFS+deg & 2034.21 s & \bfseries{590.36 s}
\end{tabular}
\end{scriptsize}
\end{minipage}%\hfill
\quad
\begin{minipage}{.3\linewidth}
    \centering
    \includegraphics[width= 2.5in, height = 1.45 in]{./Figures/Dist_Mpad_streaming_vertex_time.eps}%
    \captionof{figure}
      {\small The distribution of local computations and cross-machine communications for different streaming orders on {\em LiveJournal}. The top table reports their running times for partitioning and random walks
        \label{Dist_MPaD_streaming}
      }
\end{minipage}%\hfill
\qquad
\begin{minipage}{.37\linewidth}
    \centering
    \includegraphics[width= 2.5 in, height = 1.45 in]{./Figures/Dist_generality_table_HuGE+.eps}%
    \captionof{figure}
      {\small Generality of {\sf DistGER} vs. {\sf KnightKing}. The bars show random walk efficiency ($-R$) and training efficiency ($-T$) for {\sf Deepwalk} ({\sf DW}), {\sf node2vec} ({\sf n2v}) and {\sf HuGE+}. The top table shows the ratio $\frac{\text{{\em AUC} for {\sf DistGER}}}{\text{{\em AUC} of {\sf KnightKing}}}$, with {\sf DW} and {\sf n2v}, task: link prediction
        \label{Dist_generality}
      }
\end{minipage}
\end{table*}


\subsection{Efficiency due to Individual Parts of DistGER}
\label{sec:individual}
\spara{Random walk and training efficiency.}
To evaluate the system design of {\sf DistGER} (\S \ref{sec:DistGER}, \S \ref{sec:learning}),
we first compare the efficiency of random walks and training with those of {\sf KnighKing} and {\sf HuGE-D}.
%For fair comparison, the running times that we reported for {\sf KnightKing} and {\sf HuGE-D} exclude the
%time of vocabulary table construction, since it is a serial process in {\sf Pwode2vec}, while {\sf DistGER}
%pipelines the construction during random walks.
For random walks (Figure~\ref{Dist_efficiency_sampling_training_MPGP}(a)),
{\sf DistGER} significantly outperforms {\sf KnightKing} and {\sf HuGE-D} on all our graph
datasets, achieving an average speedup of $3.32\times$ and $3.88\times$, respectively.
Although {\sf HuGE-D} implements information-oriented random walks on {\sf KnightKing},
due to additional computation and communication overheads during on-the-fly information
measurements (\S \ref{sec:HUGED}), its efficiency can be lower than that of {\sf KnightKing}.
We also notice that the random walk lengths ($L$) and the number of random walks ($r$) reduce (on average)
63.2\% and 18\%, respectively, in our information-oriented random walks, compared to {\sf KnightKing}'s
routine random walk configuration.
%which supports the traditional
%random walk methods. %To provide a straightforward adaptation for the information-oriented approach, DistGER leverages the incremental information-centric computation mechanism to mitigate the redundant computation and high communication cost in HuGE-D, then it achieves an average speedup of $3.32\times$ and $3.88\times$ in random walk procedure compared to KnightKing and HuGE-D.

Another benefit of information-centric random walks is that it generates concise and effective corpus to improve 
training efficiency. Compared to {\sf KnightKing}, {\sf DistGER} achieves $17.37\times$-$27.95\times$ acceleration
in training over all our graphs. Next, considering the same corpus size, we compare the training efficiency of {\sf Pword2vec} and {\sf DSGL}
(trainer in {\sf DistGER}). Figure~\ref{Dist_efficiency_sampling_training_MPGP}(b) shows that {\sf DSGL} achieves $4.31\times$ average speedup
compared to {\sf Pword2vec}. We also notice that the average throughput (number of nodes processed per second) for {\sf DSGL} is up to 49.5 million/s,
while that of {\sf Pword2vec} is only up to 16.1 million/s. These results indicate that our distributed {\sf Skip-Gram} learning model (\S \ref{sec:learning})
is more efficient than {\sf Pword2vec}.
%
%\begin{figure}
%  \centering
%  \includegraphics[width= 2.5 in]{Dist_Mpad_efficiency.eps}
%  \caption{\small (a) exhibits the number of cross-machine computation for DistGER on workload-balancing and MPGP partition scheme, respectively, and (b) shows the random walk time of DistGER on the two schemes, where y axis is in log-scale.}
%  \label{Dist_efficiency_MPaD}
%\end{figure}

\spara{Partitioning efficiency.} Considering the %large number cross-machine computing introduced by the
randomness inherent in random walks, the partitioning scheme is critical to overall efficiency. %of the distributed framework.
%To validate the efficiency of our multi-proximity-aware streaming graph partitioning (MPGP),
%we deploy the workload balancing scheme used in KnightKing and MPGP on DistGER,
%respectively,
%and report the number of cross-machine computations during the random walk procedure for the two schemes.
%We also present the efficiency performance of MPGP compared with the workload-balancing scheme.
For {\sf DistGER},
Figure~\ref{Dist_efficiency_sampling_training_MPGP}(c) exhibits that our multi-proximity-aware streaming graph partitioning ({\sf MPGP})
significantly reduces (avg. reduction $45\%$) the number of cross-machine messages than the workload-balancing partition of {\sf KnightKing}
on five graphs. Moreover, it improves the efficiency by 38.9\% for the random walking procedure
(Figure~\ref{Dist_efficiency_sampling_training_MPGP}(d)) over the same set of walks.
We report in Table~\ref{Partition_sechme_overhead}(a) the time required for graph partitioning in competing systems,
where {\sf DistDGL} uses the {\sf METIS} algorithm \cite{METIS_1998} for partitioning.
The results show that {\sf MPGP} performs partitioning with very little overhead in most cases, and
the partitioning efficiency is on average $25.1\times$ faster than competitors.
In Figure~\ref{Dist_MPaD_streaming}, we exhibit the distribution of local computations and cross-machine communications
on four machines for different streaming orders, and the top table reports their running times for partitioning and random walks.
For sequential {\sf MPGP}, we find that the {\sf DFS+degree}-based streaming order (\S \ref{sec:partition}) is more efficient than other streaming orders,
and it also strikes the best balance between cross-machine communications reduction and workload balancing.
Table~\ref{Partition_sechme_overhead}(b) exhibits the performance evaluation of {\sf parallel MPGP} on the small- ({\em LiveJournal}), medium- ({\em Com-Orkut}) and large-scale ({\em Twitter}) graphs. The results show that {\sf DFS+Degree} in {\sf parallel MPGP} is still the best or comparable in terms of partition time, due to the same reason as stated in our third optimization scheme (\S \ref{sec:partition}). On the other hand, {\sf BFS+Degree} in {\sf parallel MPGP} works the best in terms of random walk time due to preserving the locality of the graph structure (our fourth optimization scheme in \S \ref{sec:partition}).
%as using its streaming order to parallel partitioning can reduce the influence of relevance between each segment.
We ultimately recommend {\sf BFS+Degree} for {\sf parallel MPGP}, since it reduces the partition time greatly, while the random walk time is comparable to that obtained from sequential {\sf MPGP}.
%
%\begin{figure}
%  \centering
%  \includegraphics[width= 3 in]{Dist_Mpad_streaming_vertex_time.eps}
%  \caption{\small The distribution of local computations and cross-machine communications for different streaming orders on {\em LiveJournal}. The top table reports their running times for partitioning and random walks.}
%  \label{Dist_MPaD_streaming}
%\end{figure}
%
%\begin{table}
%\setlength{\abovecaptionskip}{0.cm}
%\setlength{\belowcaptionskip}{-0.cm}
%\newcommand{\tabincell}[2]{\begin{tabular}{@{}#1@{}}#2\end{tabular}}
%  \caption{\small Time execution time (seconds) of the partition scheme in PBG, DistDGL, and DistGER, ``$-$'' means the scheme fails under constrains of computation resource.}
%  \label{Partition_sechme_overhead}
%  \begin{center}
%  \small
%  \begin{tabular}{ccccc}
%    \hline
%    {Graph}&{PBG}&{DistDGL(METIS)}&{DistGER}\\
%    \hline
%    Flickr& 383.28 &127.72 &\bfseries{15.96} \\
%
%    Youtube& 349.15 &116.30&\bfseries{13.56} \\
%
%    LiveJournal& 458.52 &425.19 &\bfseries{36.42} \\
%
%    Com-Orkut& 2662.62 &2761.25 &\bfseries{294.68}\\
%
%    Twitter&78986.85 &$-$&\bfseries{35500.41}\\
    %\hline
%    \multicolumn{5}{l}{* HuGE+ generates the smallest corpus size for training among all methods tested.} \\
%
%  \hline
%\end{tabular}
%\end{center}
%\end{table}
%


\subsection{Generality of DistGER}
\label{sec:generality}
%\begin{figure}
%  \centering
%  \includegraphics[width= 3 in, height= 1.65 in]{Dist_generality_table.eps}
%  \caption{\small Generality comparison for DistGER and KnightKing, %on real-word graphs,
%  The bars display random walk (denoted as $-R$) and training efficiency (denoted as $-T$) for {\sf Deepwalk} (DW) and {\sf node2vec} (n2v), respectively.%, and the y axis is in log-scale.
%  Top table shows the ratio $\frac{\text{{\em AUC} for {\sf DistGER}}}{\text{{\em AUC} of {\sf KnightKing}}}$, both with Deepwalk and node2vec, respectively, considering link prediction.}
%  \label{Dist_generality}
%\end{figure}
%
%Since our proposed information-oriented random walk framework DistGER aims to address the redundant computations and high communication cost introduced by the effectiveness measurement of the generated walking information in distributed setting, it provides a good systematic support for the information-centric approach HuGE as shown by the previous experimental results. A natural question arises: can DistGER also support the traditional random-walk-based methods?
To demonstrate the generality of {\sf DistGER}, we deploy {\sf Deepwalk} \cite{DeepWalk_2014}, {\sf node2vec} \cite{node2vec_2016} and {
\sf HuGE+} \cite{HuGE+_2022}
on {\sf DistGER}. While the original {\sf Deepwalk} and {\sf node2vec} follow
traditional random walks, in {\sf DistGER} the walk length and the number of walks are decided via information-centric measurements.
Next, we also deploy both {\sf Deepwalk} and {\sf node2vec} on {\sf KnightKing} which supports the routine configuration random walk.
Figure~\ref{Dist_generality} illustrates that {\sf DistGER} reduces the random walks time by 41.1\% and 51.6\% on average for
{\sf Deepwalk} and {\sf node2vec}, respectively. For training, {\sf DistGER} is on average $17.7\times$ and $21.3\times$ faster than {\sf KnightKing}+{\sf Pword2vec}
for {\sf Deepwalk} and {\sf node2vec}, respectively.
Moreover, we also show the {\em AUC} ratio of {\sf DistGER} and {\sf KnightKing}, considering {\sf Deepwalk} and {\sf node2vec}, for link prediction.
% tasks, where performing multi-label classification on Flickr graph  and link prediction on other graphs, the accuracy metric for the two task are $Miro-F1$ and $AUC$ score, respectively, it can be found from
Our results depict that {\sf DistGER} has comparable (in most cases, higher) {\em AUC} scores, while it improves the efficiency significantly
even for traditional random walk-based graph embedding methods.
{\sf HuGE+} is an extension of {\sf HuGE}, and it uses the same {\sf HuGE} information-centric method to determine the walk length and the number of walks per node. Figure~\ref{Dist_generality} exhibits the compatibility of {\sf HuGE+} on {\sf DistGER} via its general API.
\section{Conclusion and Future Works}

We present a novel formulation for a differentiable DTW layer that can be embedded into a deep network, based on continuous time warping and implicit deep layers. Unlike existing approaches, DecDTW yields the \textit{optimal} alignment path as a differentiable function of layer inputs. We demonstrate the effectiveness of DecDTW in utilising path information in challenging, real-world alignment tasks compared to existing methods. One promising direction for future work is in exploring more compact, alternative parameterisations of the warp function to improve scalability, inspired by~\cite{zhou12gtw}. Another direction would be to integrate more sophisticated DTW variants such as jumpDTW~\citep{jumpdtw}, which allows for repeats and discontinuities in the warp, into the GDTW (and DecDTW) formulation. Finally, we presented methodology for allowing regularisation and constraints to be learnable parameters in a deep network but did not explore this in our experiments. Future work will also explore this capability in more detail.


\subsubsection*{Acknowledgments}

S. Garg and M. Milford are with the QUT Centre for Robotics, School of Electrical Engineering and Robotics at the Queensland University of Technology. M. Xu and S. Gould are with the School of Computing, College of Engineering, Computing and Cybernetics at the Australian National University. The majority of the work was completed while M. Xu was at the QUT Centre for Robotics. M. Xu was supported by an Australian Government Research Training Program (RTP) Scholarship as well as by the QUT Centre for Robotics. M. Milford is supported by funding from ARC Laureate Fellowship (FL210100156) funded by the Australian Government. S. Gould is the recipient of an ARC Future Fellowship (FT200100421) funded by the Australian Government. Finally, we thank the reviewers and the area chair for their insightful and constructive comments during the discussion period.

\subsubsection*{Reproducibility Statement}

To facilitate reproducibility, we provide a reference implementation for our method, including code to reproduce all results across both of the experiments presented in the paper. Our code provides the full training and evaluation pipeline for all methods. In addition, we provide detailed instructions for setting up the datasets for training and evaluation, again for both experiments, as per the descriptions provided in the Appendix. Furthermore, we make publicly available the model checkpoint for the pre-trained single-image baseline network used for sequence fine-tuning in the VPR experiments.

\bibliography{iclr2023_conference}
\bibliographystyle{iclr2023_conference}

\appendix
\section{Training and Evaluation Details}

\begin{figure*}[t]
\begin{center}
\includegraphics[width=16.7cm]{figures/image_condition.pdf}
\end{center}
\vspace*{-0.3cm}
\hspace{2cm} (a) Grilled salmon \hspace{1.8cm} (b) Crepe \hspace{2.2cm} (c) Omelette \hspace{2.1cm} (d) Burrito
\vspace*{-0.4em}
\caption{Image-conditioned open-vocabulary detection. Image queries are from novel classes and the sub-captions represent text corresponding to the image queries.}
\label{fig:image_conditioned}
\vspace*{-0.4cm}
\end{figure*}


We assess our method by conducting experiments on two widely used open-vocabulary object detection datasets, namely OV-COCO and OV-LVIS. Following the literature\,\cite{guopen2022vild, zang2022open}, we split the classes in MS-COCO into 48 base categories and 17 novel categories, where the remaining categories are not included since they do not belong to a synset in the WordNet hierarhiy\,\cite{zareian2021open}. Regarding OV-LVIS, we follow the setup of \cite{zareian2021open}, selecting 337 rare classes as novel categories and the remaining classes as common and frequent categories. Because of the difference in the number of object categories, OV-LVIS has a much larger number of objects to detect. Additionally, the number of object queries should be larger than the number of actual objects in an image. Therefore, the number of object queries is set to 300 and 1,500 for OV-COCO and OV-LVIS. 

\smallskip\smallskip
\noindent\textbf{Training Loss Function.}
\algname{} utilize the loss function from (Deformable) DETR\,\cite{carion2020end, zhu2020deformable}, which includes a detection head that generates a set of bounding boxes. We modify this for OVD such that the detection head only generates the boxes and class labels given by the class prompts. During the training phase, the number of class prompts can vary according to the number of visible base classes in the current mini-batch. 

Hungarian matching is employed to identify a bipartite matching between the predicted boxes ${\rm \hat{B}}$ and the ground-truth boxes ${\rm {B}}$. The training process basically involves three types of losses: a classification loss $\ell_{\rm cls}$, which is the focal loss\footnote{Multi-label classification is performed only for the classes given by the class prompts.}, a box distance loss $\ell_{\rm l1}$ and a generalized IoU loss $\ell_{\rm iou}$ for box regression as \,\cite{songvidt, zhu2020deformable}:
 \begin{equation}
\begin{gathered}
\ell_{\rm cls}(i) = -\text{log}~\hat{\rm {P}}_{\sigma(i),c_i},~~~\ell_{\rm  l1}(i) =  ||{\rm B}_{i}-\hat{\rm B}_{\sigma(i)}||_{1},\\
\ell_{\rm iou}(i) = 1 \!-\! \big( \frac{|{\rm B}_{i} \cap  \hat{{\rm B}}_{\sigma(i)}|}{|{\rm B}_{i} \cup  \hat{{\rm B}}_{\sigma(i)}|} - \frac{|{\sf {\rm B}}({\rm B}_{i}, \hat{{\rm B}}_{\sigma(i)}) \symbol{92} {\rm B}_{i} \cup  \hat{{\rm B}}_{\sigma(i)}| }{|{\sf {\rm B}}({\rm B}_{i}, \hat{{\rm B}}_{\sigma(i)})|} \big),
\end{gathered}
\end{equation}
where $c_i$ is the target base class and $\sigma(i)$ is the bipartite assignment of the $i$-th ground-truth box. Also, the embedding loss $\ell_{\rm embed}$ proposed by \cite{zang2022open} is used to distil the CLIP's knowledge for open vocabulary object detection. Therefore, the final objective of \algname{} is formulated as:
\begin{equation}
\ell = \lambda_{\rm cls}\ell_{\rm cls} + \lambda_{\rm l1}\ell_{\rm l1} + \lambda_{\rm iou}\ell_{\rm iou} + \lambda_{\rm embed}\ell_{\rm embed}, 
\label{eq:naive_roi}
\end{equation}
where $\lambda$s are the balancing parameters, where $\lambda_{\rm cls}=3, \lambda_{\rm l1}=5, \lambda_{\rm iou}=2$, and $\lambda_{\rm embed}=2$. 

\smallskip\smallskip
\noindent\textbf{Evaluation.}
In the testing phase, the class prompts are inclusive of all base and novel classes following the recent literature\,\cite{guopen2022vild, zang2022open, zhong2022regionclip, zhou2022detic, rasheedbridging}. For producing the final results, we apply RoI-based pruning to identify the RoIs with high object scores. Next, the classification results of the selected bounding boxes are ensembled with those from the ViT-based CLIP by Eq.\,\eqref{eq:ensemble}. 
\section{Potential Enhancement}

We discuss three possible ways to further enhance the potential of our framework. 

\smallskip\smallskip
\noindent\textbf{Prediction Ensemble.} In our ensemble method, we adopt an arithmetic mean approach as shown in Eq. \eqref{eq:ensemble}. However, this requires hyperparameter tuning to achieve the best performance, and the results are sensitive to the selection of $\alpha$ values, as indicated in Table \ref{tab:alpha}. The variation in performance highlights the importance of carefully choosing the hyperparameters for \algname{}, as suboptimal selections can have a significant impact on the overall performance. Although the simple ensemble approach provides a satisfactory detection accuracy for OV-COCO and OV-LVIS, we are of the belief that a more effective prediction ensemble strategy can be studied as future work to leverage both CLIP and a detection model in synergy.



\smallskip\smallskip
\noindent\textbf{Unrestricted Setup.} Our primary focus is on developing an end-to-end Transformer-based framework that is both efficient and effective. Therefore, we adopt a restricted setup in which external data is not permitted, as our aim is to investigate the potential of the framework itself in achieving optimal performance. Despite our restricted setup, we acknowledge that in real-world scenarios, large-scale external data are often available and their utilization can significantly enhance the zero-shot detection performance. We are confident that our framework can be extended in this direction, owing to its simple end-to-end encoding and decoding pipeline. However, we leave this as a potential avenue for future work, as our current focus is on investigating the framework's performance under the restricted setup.


\smallskip\smallskip
\noindent\textbf{Instance Segmentation.} A drawback of \algname{} is the considerable discrepancy in the performance between object detection and instance segmentation. While this is partly due to inheriting the limitations of using SOLQ\,\cite{dong2021solq} for end-to-end joint learning, the development of a more effective segmentation approach based on DETR will enhance the performance of our framework significantly. We leave this as an avenue for future research.


\section{Image Conditioned Detection}

One advantage of our framework is that it can detect objects using an {image query} instead of relying solely on the class name. This feature has the potential to identify the target object by using a cropped image of the object itself, even when we may not know the object's name. 

Figure \ref{fig:image_conditioned} shows the results of \algname{} when applying this image-conditioned object detection using four different image queries. Although their class names are not provided, \algname{} is capable of accurately localizing target objects that were not encountered during the training phase.  Additionally, \algname{} can recognize new classes even when the image query has a dissimilar shape of target objects, as depicted in Figure \ref{fig:image_conditioned}(a). These findings suggest that our algorithm is resilient in detecting objects using image queries for open-set scenarios.
%


%We conducted further analysis on the qualitative results depicted in Figure~\ref{fig:image_conditioned} when applying image queries. The findings demonstrate that \algname{} is capable of accurately localizing target objects that were not encountered during the training phase. Moreover, we observe that \algname{} can identify new classes even when the image query has a dissimilar shape of target objects, as illustrated in Figure~\ref{fig:image_conditioned}(a). These results suggest that our algorithm is robust in detecting objects using image queries.

% [TBD] [TBD] [TBD] [TBD] [TBD] [TBD] [TBD] [TBD] [TBD] [TBD] [TBD] [TBD] [TBD] [TBD] [TBD] [TBD] [TBD] [TBD] [TBD] [TBD] [TBD] [TBD] [TBD] [TBD] [TBD] [TBD] [TBD] [TBD] [TBD] [TBD] [TBD] [TBD] [TBD] [TBD] [TBD] [TBD] [TBD] [TBD] [TBD] [TBD] [TBD] [TBD] [TBD] [TBD] [TBD] [TBD] [TBD] [TBD] [TBD] [TBD] [TBD] [TBD] [TBD] [TBD] [TBD] [TBD] [TBD] [TBD] [TBD] [TBD] [TBD] [TBD] [TBD] [TBD] [TBD] [TBD] [TBD] [TBD] [TBD] [TBD] [TBD] [TBD] [TBD] [TBD] [TBD] [TBD] [TBD] [TBD] [TBD] [TBD] [TBD] [TBD] [TBD] [TBD] [TBD] [TBD] [TBD] [TBD] [TBD] [TBD] [TBD] [TBD] [TBD] [TBD] [TBD] [TBD] [TBD] [TBD] [TBD] [TBD] [TBD] [TBD] [TBD] [TBD] [TBD] [TBD] [TBD] [TBD] [TBD] [TBD] [TBD] [TBD] [TBD] [TBD] [TBD] [TBD] [TBD] [TBD] [TBD] [TBD] [TBD] [TBD] [TBD] [TBD] [TBD] [TBD] [TBD] [TBD] [TBD] [TBD] [TBD] [TBD] [TBD] [TBD] [TBD] [TBD] [TBD] [TBD] [TBD] [TBD] [TBD] [TBD] [TBD] [TBD] [TBD] [TBD] [TBD] [TBD] [TBD] [TBD] [TBD] [TBD] [TBD] [TBD] [TBD] [TBD] [TBD] [TBD] [TBD] [TBD] [TBD] [TBD] [TBD] [TBD] [TBD] [TBD] [TBD] [TBD] [TBD] [TBD] [TBD] [TBD] [TBD] [TBD] [TBD] [TBD] [TBD] [TBD] [TBD] [TBD] [TBD] [TBD] [TBD] [TBD] [TBD] [TBD] [TBD] [TBD] [TBD] [TBD] [TBD] [TBD] [TBD] [TBD] [TBD] [TBD] [TBD] [TBD] [TBD] [TBD] [TBD] [TBD] [TBD] [TBD] [TBD] [TBD].

\begin{table}[t!]
\centering
\caption{Performance when using different attention layers for RoI-based masked attention on OV-LVIS }
\vspace*{-0.25cm}
\label{tab:blk_num}
\resizebox{1.0\linewidth}{!}{%
\begin{tabular}{@{}lrrrr@{}}
\toprule 
{Attn. Layer Num.}& $\mathrm{mAP^{box}_{novel}}$ & $\mathrm{mAP^{box}}$ & $\mathrm{mAP^{mask}_{novel}}$ & $\mathrm{mAP^{mask}}$ \\
\midrule
% & %(memory update / robust learning) &
% \multicolumn{1}{r}{\footnotesize {\sffamily RCNN-based}} \\
% ViLD-text & 10.1 & 24.9 & 5.9 & 49.3 \\
$\mathrm{15^{th}}$ & 13.6 & 28.3 & 10.5 & 20.4 \\ 
$\mathrm{20^{th}}$ & 20.4 & 30.1 & 16.2 & 21.9 \\ 
\textbf{$\mathrm{24^{th}}$ (last layer)} & \textbf{29.4} & \textbf{33.0} & \textbf{23.1} & \textbf{24.2} \\
\bottomrule
\end{tabular}%
}
\label{tab:diff_layer}
\vspace*{-0.4cm}
\end{table}

\section{Supplementary Analysis}

We provide supplementary analysis on applying RoI-based masked attention to different attention layers in Appendix \ref{sec:diff_layer} and using different pre-trained ViT backbones with \algname{} in Appendix \ref{sec:diff_pretrain}.




\subsection{Pre-trained Weights.} 
\label{sec:diff_pretrain}

We analyze the impact of using different initial weights for the backbone, ImageNet and MAE, on the overall performance. As outlined in Table~\ref{tab:pretrained_model}, training the backbone using the MAE pre-trained weights yields better performance in dense prediction tasks such as object detection and instance segmentation, even under an open-vocabulary setup, compared to starting from the ImageNet pre-trained weights. This observation is consistent with previous research, such as \cite{li2022exploring}.
It is also worth noting that even when using the ImageNet pre-trained weights, \algname{} still outperforms the baseline\,(OV-DETR) with a much faster inference speed.
In conclusion, the choice of initial weight for the backbone plays a crucial role in the overall performance of the model, and using the MAE pre-trained weights as the starting point results in better performance.





\subsection{Different Layers for RoI-based Attention} 
\label{sec:diff_layer}

We investigate the impact of applying our RoI-based masked attention to different attention layers in the ViT-L/14 image encoder of CLIP. The results, as summarized in Table \ref{tab:diff_layer}, show that detection accuracy improves as the layer number increases, from the 15th to the 24th. Therefore, applying the technique to the last layer is the best design choice for open vocabulary detection with \algname{}.

\begin{table}[t!]
\centering
\caption{Performance when using different pre-trained ViT backbones on OV-LVIS.}
\vspace*{-0.25cm}
\label{tab:pretrained_model}
\resizebox{1.0\linewidth}{!}{%
\begin{tabular}{@{}lrrrr@{}}
\toprule 
{Pretrained Model}& $\mathrm{mAP^{box}_{novel}}$ & $\mathrm{mAP^{box}}$ & $\mathrm{mAP^{mask}_{novel}}$ & $\mathrm{mAP^{mask}}$ \\
\midrule
% & %(memory update / robust learning) &
% \multicolumn{1}{r}{\footnotesize {\sffamily RCNN-based}} \\
% ViLD-text & 10.1 & 24.9 & 5.9 & 49.3 \\
ImageNet~\cite{deng2009imagenet} & 26.4 & 28.7 & 20.7 & 20.9\\
\textbf{MAE}~\cite{he2022masked} & \textbf{29.4} & \textbf{33.0} & \textbf{23.1} & \textbf{24.2}\\
\bottomrule
\end{tabular}%
}
\end{table}

\end{document}
