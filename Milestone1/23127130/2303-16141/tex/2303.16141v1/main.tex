%
% IEEE Transactions on Microwave Theory and Techniques example
% Tibault Reveyrand - http://www.microwave.fr
%
% http://www.microwave.fr/LaTeX.html
% ---------------------------------------



% ================================================
% Please HIGHLIGHT the new inputs such like this :
% Text :
%  \hl{comment}
% Aligned Eq. 
% \begin{shaded}
% \end{shaded}
% ================================================



\documentclass[journal]{IEEEtran}

% \documentclass[12pt, letterpaper,
% twoside]{article}

% \usepackage{emoji}
\usepackage[graphicx]{realboxes}
%\usepackage[retainorgcmds]{IEEEtrantools}
%\usepackage{bibentry}  
\usepackage{xcolor,soul,framed} %,caption
\usepackage{cite}
\usepackage{multirow}
\usepackage{fontawesome}
\colorlet{shadecolor}{yellow}
% \usepackage{color,soul}
% \usepackage[pdftex]{graphicx}
% \graphicspath{{../pdf/}{../jpeg/}}
\DeclareGraphicsExtensions{.pdf,.jpeg,.png}
\usepackage{caption}
\usepackage{subcaption}
\usepackage[cmex10]{amsmath}
\usepackage[colorinlistoftodos]{todonotes}

%Mathabx do not work on ScribTex => Removed
%\usepackage{mathabx}
% \usepackage{algorithm}
% \usepackage{algpseudocode}
\usepackage{comment}
% \let\hl[1]\includecomment{#1}
\usepackage{enumerate}% http://ctan.org/pkg/enumerate
% \newcommand{\hlcyan}[1]{{\sethlcolor{cyan}\hl{#1}}}
\newcommand\textblue[1]{\textcolor{blue}{#1}}
\newcommand\textred[1]{\textcolor{red}{#1}}

% \newcommand\myworries[1]{\textcolor{red}{#1}}

\newcommand\quickthings[1]{\textblue{\\\faQuestion #1}}
\newcommand\maybelater[1]{\textred{\\\faClockO#1}}
% \newcommand\quickthings[1]{\textblue{\\\faQuestion #1}}
\newcommand\peter[1]{\textcolor{red}{\faComment #1}}
\newcommand\abs[1]{\\\hl{#1}}



%Comment to keep todo parts, Uncomment to remove TODO parts!
\renewcommand\quickthings[1]{}
% \renewcommand\absolutelynecessary[1]{}
\renewcommand\peter[1]{}
\renewcommand\maybelater[1]{}
\renewcommand\abs[1]{}
\renewcommand\hl[1]{#1}

\usepackage[linesnumbered,ruled,vlined]{algorithm2e}



% % \usepackage
% \usepackage[ruled,vlined]{algorithm2e}
% \DeclareMathOperator{\ind}{ind}
% \DeclareMathOperator{\val}{val}
% \DeclareMathOperator{\ptr}{ptr}
% \DeclareMathOperator{\row}{row}
% \usepackage{algpseudocode}
\hyphenation{op-tical net-works semi-conduc-tor}

%\bstctlcite{IEEE:BSTcontrol}


%=== TITLE & AUTHORS ====================================================================
\begin{document}
\bstctlcite{IEEEexample:BSTcontrol}
    % \title{Adversarial attacks on federarted learning for medical image analysis}
    \title{ A Comparative Study of Federated Learning Models for COVID-19 Detection  }
  \author{Erfan~Darzidehkalani$^{1}$,
  Nanna M. Sijtsema$^{1}$,
      P.M.A van Ooijen$^{1}$
    %   and~Zoya~Popovi\'c,~\IEEEmembership{Fellow,~IEEE}% <-this % stops a space
\\\textit{$^{1}$Machine learning lab, Data Science Center in Health (DASH) \\University of Groningen, Hanzeplein 1, Groningen, The Netherlands}
\\ e.darzidehkalani@rug.nl
  \thanks{ This paper is being prepared with IEEE standards This work was funded in part by NWO under project AMICUS}
  \thanks{Erfan Darzidehkalani is with UMCG (e-mail: e.darzidehkalani@umcg.nl).}% <-this % stops a space
  }
% The paper headers
% \markboth{IEEE TRANSACTIONS ON MICROWAVE THEORY AND TECHNIQUES, VOL.~60, NO.~12, DECEMBER~2012
% }{Roberg \MakeLowercase{\textit{et al.}}: Cross modality image transformation using cyclic residual
% architecture and evolutionary algorithm}
% ===================================================================
\maketitle
% === ABSTRACT ====================================================================
% =================================================================================
\begin{abstract}

Deep learning is effective in diagnosing COVID-19 and requires a large amount of data to be effectively trained. Due to data and privacy regulations,  hospitals generally have no access to data from other hospitals. Federated learning (FL) has been used to solve this problem, where it utilizes a distributed setting to train models in hospitals in a privacy-preserving manner. Deploying FL is not always feasible as it requires high computation and
network communication resources. This paper evaluates five FL algorithms' performance and resource efficiency for Covid-19 detection. 
A decentralized setting with CNN networks is set up, and the performance of FL algorithms is compared with a centralized environment. We examined the algorithms with varying numbers of participants, federated rounds, and selection algorithms.

Our results show that cyclic weight transfer can have better overall performance, and results are better with fewer participating hospitals. Our results demonstrate good performance for detecting COVID-19 patients and might be useful in deploying FL algorithms for covid-19
detection and medical image analysis in general.


\end{abstract}



% === KEYWORDS
% ====================================================================
% =================================================================================
\begin{IEEEkeywords}
federated learning, medical image analysis, COVID-19, privacy preserving machine learning
\end{IEEEkeywords}






% For peer review papers, you can put extra information on the cover
% page as needed:
% \ifCLASSOPTIONpeerreview
% \begin{center} \bfseries EDICS Category: 3-BBND \end{center}
% \fi
%
% For peerreview papers, this IEEEtran command inserts a page break and
% creates the second title. It will be ignored for other modes.
\IEEEpeerreviewmaketitle


% ====================================================================
% ====================================================================
% ====================================================================










% \tableofcontents
% === I. INTRODUCTION =============================================================
% =================================================================================
\section{Introduction}

\IEEEPARstart{ C}oronaviruses are a family of viruses that cause respiratory and intestinal illnesses in humans and animals. The best-known variants are those responsible for the COVID-19, SARS, and MERS epidemics. Some people with COVID-19 will develop serious complications including COVID pneumonia, which can be recognized in lung CT scans. Research has shown the effectiveness of chest imaging in diagnosing COVID-19-infected people. Deep learning methods, such as Convolutional neural networks (CNN), can help radiologists diagnose COVID-19 with severe symptoms in various image analysis tasks\cite{kogilavani2022covid}.
For Covid-19 deep learning, models have shown great promise in spotting infected areas in CT scans and X-ray images. 

The training of deep learning models requires sufficient and diverse medical datasets gathered from multiple data holders. And most of the existing solutions rely on a central entity in charge of collecting data from different hospitals. However, medical images may contain confidential and sensitive information about patients that often cannot be shared outside the institutions of their origin. One potential solution to this problem is federated deep learning. FL aims to decentralize the whole process of training by keeping the data locally. In FL, the algorithm training is performed in a decentralized manner by different nodes, or clients, that use local data. In this scenario, each decentralized node
trains an individual model using its data and shares the model parameters (instead of the data) with the rest.
% the distributed computation and develpmoent of deep learning models ensure the accumulation of knowledge and reaching the full capacity of deep learning models.

FL can differ from centralized data sharing in a number of ways. While both approaches aim to optimize their learning objective, FL algorithms have to account for the fact that communication with clients takes place over unreliable networks with very limited upload speeds. So unlike 
the centralized setting in which computation is generally a bottleneck, in FL communication might be the bottleneck. 

In this paper, we developed a framework that enables collaboration between hospitals and uses multiple data sources to detect COVID-19 infection using FL
The decentralized way of distributing data among different centers guarantees privacy and data is kept locally\cite{darzidehkalanifederatedII}.




% There are several ways of doing federated learning since federated learning is performed in a higher level compared to conventional deep learning. There are more parameters of complexity since the model should take into consideration the general and local methods of optimization, data accumulation for all the sites, and local training models. 
% There are several parameters playing a role in the final performance of a federated learning setting.

 





% It can be spread by human-human interaction. There are several waus tp diagnose COVID-19. Transcription-polymerase chain reaction (RT-PCR) test are one of the most common ways of diagnosing COVID-19. However, there are other ways of diagnosing COVID-19 \cite{rubin2020role}
% However, due to large number of people having the symptoms and being tested for COVID-19, it might not always be feasible for radiologists to examine large number of patients during the outbreak. Scans of CT and Xray of infected patients have sorts of deformation, misalignment between certain parts, or have pixel intensities different than normal people in infected parts. Observing this misalignment in not an easy task and requires effort and focus from radiologist to correctly distringuish betweeen healthy and infected people. 
% Instead, 
% Various architectures of CNN have been tested  and shown a great performance on the imaging datasets. Models such as AlexNet \cite{krizhevsky2012imagenet}
%  ResNet \cite{he2016deep}  and MobileNet \cite{howard2017mobilenets} could classify thousands of objects belonging to hundreds of classes with the same accuracy as human.
 



% It has also been used to detect COVID-19 from imaging data. Most of the usecases were either for Xray like the researches in 
% While CT scans can help screen COVID-19 potential infected people, CT scans of other diseases could also have overlapping features with COVID ingected patients. Hence, it is tricky to distinguish COVID infected patients from other types of lung inflammation and requires radiologists level experrtise. The most literature focuses on the ways to better distirnguish COVID infected patients from other diseases, as can be seen in [8], [9], [10], [11], [12]. 

\textbf{Image captioning metrics}
As with other natural language generation tasks, image captioning can be evaluated using various proposed metrics.
%As with other natural language generation tasks, various metrics have been proposed to evaluate the image captioning model. 
BLEU~\cite{papineni2002bleu}, ROUGE~\cite{lin2004rouge}, and METEOR~\cite{banerjee2005meteor} are representative image captioning metrics based on n-gram similarity with reference captions.
%BLEU, ROUGE, and METEOR, based on N-gram similarity with reference caption, are representative metrics of the image captioning task. 
Other widely used reference-based metrics include CIDEr~\cite{vedantam2015cider}, which weights n-gram similarity~\cite{kondrak2005n} through TF-IDF~\cite{aizawa2003information}, and SPICE~\cite{anderson2016spice}, which evaluates captioning based on scene graphs.
%Afterward, CIDEr, which weights N-gram similarity through TF-IDF, and SPICE, which evaluates captioning based on scene graphs, are proposed and widely used. 
%\textcolor{blue}{Recently, reference-based metrics such as BERTScore~\cite{zhang2019bertscore}, BERT-TBR~\cite{yi2020improving}, and VilBERTScore~\cite{lee2020vilbertscore}, which use embedding similarity with reference captions based on a model, have been introduced.}\\
Recently, reference-based metrics using embedding similarity with reference captions based on a model, such as BERTScore~\cite{zhang2019bertscore}, BERT-TBR~\cite{yi2020improving}, and VilBERTScore~\cite{lee2020vilbertscore}, have been introduced.

%Recently, metrics such as BERTScore, BERT-TBR, and VilBERTScore using the embedding similarity with reference based on the model have been introduced.\\
\indent Researchers have also proposed unreferenced image captioning metrics that evaluate generated captions by comparing them with original images.
%\indent Unlike these reference-based metrics, researchers have proposed reference-free metrics that sorely evaluate generated captions by comparing them with original images without reference caption. 
For instance, VIFIDEL~\cite{madhyastha2019vifidel} uses the word mover distance~\cite{kusner2015word} between the image and candidate caption, and UMIC~\cite{lee2021umic}, which fine-tunes UNITER~\cite{chen2020uniter} using contrastive loss from augmented captions, directly evaluates captions generated from vision-and-language embedding spaces.
%For instance, VIFIDEL uses the word mover distance between the image and candidate caption, and UMIC, fine-tunes UNITER using contrastive loss from augmented captions, directly evaluates generated caption from the vision-and-language embedding spaces. 
\\
\\
\noindent \textbf{CLIPScore} 
CLIPScore~\cite{hessel2021clipscore} is a reference-free metric that does not require ground-truth captions.
%CLIPScore is a reference-free metric that does not require a ground-truth caption. 
CLIPScore relies heavily on the CLIP~\cite{radford2021learning} model, trained with 400 million image caption pairs using a contrastive objective function that distinguishes original image–caption pairs from unmatched captions.
%CLIPScore heavily relies on the CLIP model trained with 400M image caption pairs using a contrastive objective function that distinguishes original image-caption pairs from unmatched captions. 
The calculated CLIPScore is the weighted value of cosine similarity between image embedding and text embedding encoded by the CLIP model.
%CLIPScore is the weighted value of cosine similarity between image embedding and text embedding encoded by the CLIP model. 
Thanks to the power of the massive training dataset and CLIP’s objective function, CLIPScore exhibits a high correlation with human evaluation.
%Thanks to the power of the massive training dataset and CLIP’s objective function, CLIPScore shows a high correlation with human evaluation. 
However, CLIPScore is limited in that it is an image captioning metric that applies only to English.
%However, CLIPScore has a limitation in that it is an image captioning metric that applies only to English. 
In this study, we propose a new multilingual image captioning metric developed by extending CLIPScore to a multilingual setting.\\
%In this study, we propose a new multilingual image captioning metric by extending CLIPScore to multi-lingual setting.\\

\noindent \textbf{Perturbation Robustness}
In a recent study,  \citet{sai2021perturbation} selected various criteria for use in assessing how various NLG evaluation metrics perform.
%In a recent study, Sai-et-al. selected various criteria for NLG evaluation metrics to investigate whether each metric reflects the criteria well. 
In addition, perturbation was applied to multiple image captioning factors to assess the perturbation robustness of the image captioning metrics.
%Also, for the image captioning, perturbation was given to multiple factors constituting the criteria of correctness and thoroughness to measure the degree of perturbation robustness of image captioning metrics. 
 \citet{sai2021perturbation} provided a perturbation checklist of metrics for NLG tasks; we go further and present a novel metric that overcomes the limitations of other metrics.
%Sai-et-al. has only provided a perturbation checklist of metrics for the NLG tasks, but we go beyond that and present a novel metric that overcomes that limitations. 
We select some perturbation criteria from among those suggested by  \citet{sai2021perturbation}, designate them as target perturbations, and show that the CLIPScore cannot detect these perturbations in multiple languages.
%We select some perturbation criteria among them and designate them as target perturbation, and show that these perturbations cannot be detected by the CLIPScore in multiple languages. 
Even if the generated captions are corrupted, CLIPScore outputs similar results for the original and corrupted sentences.
%Even if the generated captions are corrupted, CLIPScore outputs similar results for the original and corrupted sentences. 
This study proposes a novel metric with perturbation robustness based on CLIPScore to address its weaknesses in multiple languages.
%In this study, we propose a novel metric with perturbation robustness based on CLIPScore to solve weaknesses for multiple languages.



\iffalse
\textbf{Reference-based Metric.}
% As with other NLG tasks, various metrics have been proposed to evaluate Image Captioning. 
% BLEU\cite{papineni2002bleu}, ROUGE\cite{lin2004rouge}, and METEOR\cite{banerjee2005meteor}, which are based on n-gram similarity with reference caption, are representative metrics of the Image Captioning task. 
% Afterwards, CIDEr\cite{vedantam2015cider}, which weights n-gram similarity through TF-IDF, and SPICE\cite{anderson2016spice}, which evaluates captioning based on scene graph, are also well-known metrics.
% After the study of BERTScore\cite{zhang2019bertscore}, many studies on metrics based on the model were conducted.
% BERTScore uses BERT\cite{devlin2018bert} to evaluate caption by comparing contextualized embedding of generated caption and reference. The BERT-TBR\cite{yi2020improving} study based on variance in multiple hypotheses and the VilBERTScore\cite{lee2020vilbertscore} based on the VilBERT\cite{lu2019vilbert} model are also well-known model-based Image Captioning Metrics. \\
As with other NLG tasks, various metrics have been proposed to evaluate the Image Captioning model. 
BLEU~\cite{papineni2002bleu}, ROUGE~\cite{lin2004rouge}, and METEOR~\cite{banerjee2005meteor}, which are based on n-gram similarity with reference caption, are representative metrics of the Image Captioning task. 
Afterward, CIDEr~\cite{vedantam2015cider}, which weights n-gram similarity through TF-IDF, and SPICE~\cite{anderson2016spice}, which evaluates captioning based on scene graph, are proposed and widely used.
Recently, embedding-based metrics, which compare generated captions with ground captions on the vector space, show higher human correlation than previous n-gram-based metrics. One of the main advantages of these metrics lies in their robustness, free from lexical matching.
BERTScore~\cite{zhang2019bertscore} uses BERT~\cite{devlin2018bert} to evaluate caption by comparing contextualized embedding of generated caption and reference. The BERT-TBR~\cite{yi2020improving} consider variance in multiple hypotheses and the VilBERTScore~\cite{lee2020vilbertscore} uses multimodal information (i.e., image and caption) to compute its corresponding vector representation for comparision. \\


\noindent \textbf{Reference-free Metric.}
% All of the Image Captioning Metrics above require reference captions for performance evaluation.
% Unlike these, in the case of VIFIDEL\cite{madhyastha2019vifidel}, a metric using the word mover distance\cite{kusner2015word} of image and candidate caption, reference caption is not required. UMIC, based on the UNITER\cite{chen2020uniter} model, trained by contrastive loss presentes a model-based reference-free metric.
% Similar to UMIC\cite{lee2021umic}, our study proposes a metric with good performance that meets the purpose through fine-tuning of the Vision-and-Language transformer model. 
% However, unlike UMIC, our study proposes a new methodology to create a perturbation robust metric by giving perturbation robustness to the existing metric.
Researchers have proposed unreferenced Image Captioning metrics sorely evaluate generated captions by comparing them with original images.
For instance, VIFIDEL~\cite{madhyastha2019vifidel} uses the word mover distance~\cite{kusner2015word} between image and candidate caption and UMIC~\cite{lee2021umic}, finetunes UNITER~\cite{chen2020uniter} using contrastive loss from augmented captions, directly evaluate generated caption from the vision-and-language embedding spaces.
Similar to UMIC, our study proposes a metric with good performance that meets the purpose through fine-tuning the Vision-and-Language transformer model. 
However, unlike UMIC, our study proposes a new methodology to create a perturbation robust metric by training a model to distinguish differences in key information. We further extend its applicabilities to a multilingual domain.

\subsection{CLIPScore}
% CLIPScore is a reference-free metric, like UMIC, which does not require a reference and measures the score using the given image and caption. UMIC is a metric based on UNITER, and CLIPScore is a metric based on the CLIP model. For the CLIP model trained with a 400M image-caption pairs as a contrastive loss, the cosine similarity for the contrastive loss used during training is itself used as an evaluation metric for the similarity between the image and generated caption. CLIPScore is the weighted value for this cosine similarity between image embedding and text embedding, and in itself, it is a great image captioning metric that has a very high correlation with human evaluation.
% However, even if the generated captions are corrupted (for example, repetitive or attribute replacement occurs), CLIPScore outputs similar results for the original and corrupted sentences, showing weaknesses. In this study, we propose a novel method to solve this weakness by using the CLIPscore as the baseline but giving perturbation robustness. Additionally, by extending CLIPScore to multilingual, we propose a new multiLingual image captioning metric, and present a metric with robustness for perturbation in multilingual settings. 
CLIPScore~\cite{hessel2021clipscore} is a reference-free metric that does not require a ground-truth caption. CLIPScore heavily relies on CLIP~\cite{radford2021learning} model trained with 400M image-caption pairs using a contrastive objective function that distinguishes original image-caption pairs from unmatched captions.
CLIPScore is the weighted value of cosine similarity between image embedding and text embedding encoded by CLIP model. Thanks to the power of the massive training dataset and CLIP's objective function, CLIPScore shows a high correlation with human evaluation.\\
However, even if the generated captions are corrupted (for example, repetitive or attribute replacement occurs), CLIPScore outputs similar results for the original and corrupted sentences, showing weaknesses. In this study, we propose a novel method to solve this weakness by using the CLIPscore as the baseline but giving perturbation robustness. Additionally, by extending CLIPScore to multilingual, we propose a new multiLingual image captioning metric, and present a metric with robustness for perturbation in multilingual settings. 

\subsection{Perturbation Robustness}
In recent study,  \citet{sai2021perturbation} select various criteria for NLG Evaluation Metrics to investigate whether each metric reflects the criteria well. Also for Image Captioning, perturbation was given to various factors constituting the Criteria of Correctness and Thoroughness, and the degree of perturbation robustness of Image Captioning Metrics was measured. We select some perturbation methods among them and designate them as target perturbation, and show that this perturbation cannot be reflected in the existing metrics. In addition, we propose a new learning methodology and a metric with robustness for the selected perturbations.
\fi


\section{Algorithms}

% \label{sec:algorithms}
\textbf{Centralized data sharing}
In Centralized data sharing (CDS), data is stored in a central location and can be accessed by all clients. This is in contrast to federated and decentralized data sharing methods, where data is stored in multiple locations and accessed by a single user or limited numbers of users. We use CDS as a  baseline  for comparison with other algorithms. 

\textbf{Federated averaging}: The learning procedure for federated averaging is an iterative process containing local and global steps. Each data owner trains a model received from a global server on its local dataset in local iterations \cite{darzidehkalanifederatedI}. The global server updates the global model by aggregating the updated local models. Then it sends it back to clients for the next round.  The optimization problem for federated averaging can be formulated as
\begin{equation}
w^{t+1} = \sum\limits_{i=1}^{N}{p_{i} w_{i}^{t}} , w_{i}^{t}=\arg\min\limits_{w_{i}}{\left(\mathcal{L}(\mathcal{D}_{i};w^{t})\right)}
\end{equation}
where $N$ is the number of data owners, $\mathcal{L}(\mathcal{D}_{i};w^{t})$ is a loss function indicating global model parameters  $w^{t}$ of local datasets, and $p_{i}$ is the probability of selecting client $i$. 
Local optimization can be formulated as $w_{i}^{t+1} \leftarrow w^{t}-\eta\cdot \nabla \mathcal{L}(w^{t};\mathcal{D}_{i})$, where 
 $\eta$ is the learning rate. The global model can be updated based on the local models $w_{i}$ and is shared for aggregation: 
\begin{equation}
w^{t+1} = \sum\limits_{i=1}^{N}{p_{i} w_{i}^{t+1}}
\end{equation}

\textbf{Federated stochastic gradient descent}:Federated Stochastic Gradient Descent (FedSGD) is a variation of Federated Averaging (FedAvg) that uses a large-batch synchronous approach to multi-client learning. FedSGD utilizes a subset of clients from the total number of clients, where $C$ defines the subset of selected clients. This subset of clients is selected at each global round, and the global server sends the most recent global model to them. Each client then performs local training over its dataset for a select number of epochs. The global model is updated based on the local models received from each client and is shared for aggregation, similar to FedAvg. However, in FedSGD, the gradient is computed over the selected batch of clients and therefore, $C<1$, for $C=1$ the training would be non-stochastic (full batch) since all the clients are involved. This allows for training with large batches, as the gradient is computed over the selected subset of clients.   The optimization problem for FedSGD can be formulated as
\begin{equation}
w^{t+1} = w^{t} - \eta \cdot \sum\limits_{i=1}^{C}p_i \cdot \nabla \mathcal{L} (w^t;\mathcal{D}_i)
\end{equation}
where $\eta$ is the learning rate, $p_i$ is the probability of selecting client $i$ and $\mathcal{L}$ is the loss function.
The key difference between FedAvg and FedSGD lies in the use of large-batch synchronous approach in FedSGD. This approach has been shown to outperform the naive asynchronous SGD training due to the increased accuracy and efficiency, as compared to the local training approach used in FedAvg \cite{chai2020fedeval}\cite{charles2021large}. Additionally, FedSGD has been shown to be more robust to non-IID data distributions, compared to FedAvg \cite{chai2020fedeval}. 
\quickthings{This part is unclear to me.}
% \absolutelynecessary{}
\maybelater{paraphrised}

\textbf{Cyclic weight transfer}: Federated learning techniques have been widely used in medical image processing tasks using a method known as cyclic weight transfer (CWT)\cite{balachandar2020accounting}. This method involves training models on individual clients for a number of iterations and then cyclically sharing the updated weights with the following client. However, the existing CWT algorithm faces a notable challenge, as it lacks the ability to effectively manage inter-client variability in training data or labels.  To ensure the practical application of CWT, it is crucial to develop a version that can handle the common variations observed in a majority of real-world medical imaging datasets.\cite{darzidehkalanifederatedII}

\textbf{Single weight transfer}: Single weight transfer (SWT) is another FL model widely used in the medical imaging domain. In Single weight transfer, models are trained in each client with its local data, and then the updated model is transferred to the next client. The difference between this method and CWT is that here the model passes each client only once. 

\textbf{Stochastic weight transfer}:
In stochastic weight transfer (STWT), we select a subsample of clients and train them in a cyclic manner. Similar to FedSGD, a ratio defines the number of selected clients to the total number of clients in each federated round.



% سوال / مسیله چیه
% Importance: Why your research matters in the context of an industry or the world
% اهمیت موضوع
% 	کجاها کاربرد داره
% مشکلات موضوعات قبلی و اینکه مسایل دگ کار نمیکنن
% برای اینکه این مسایل کار کنن نیازی به چه چیزی بود. ک اونا کم داشتن
% چرا این روش تو داره کار میکنه
% چه سوالی رو جواب میده
% توضیح اینکه چجوری کار میکنه
% تعریف و تمجید از کار کردنش

% \maybelater{chanta akse tamiz ba coreldraw bekesh}

% Their
% We are comparing various implementations of FL and comparing their performance level. The performance is done by metrics.
% 2. Although FL has been introduced to tackle the problem of privacy, its distributed nature requires too much commuincation between clients. Knowsing can determine its deployability and scailibitlty. Hence the question is how to these model compare it terms of communication between clients.
% 3. Computation time. Each FL model is investigated based on the required time for computation.
% 4. Effect of rounds: Number of Federated rounds 


% \quickthings{Maybe you can add a benchmarking section similar to "A Performance Evaluation of FL Algorithms"}
% \maybelater{Some ideas are herer https://arxiv.org/pdf/1709.05929.pdf
% }



\section{Experiment}

\subsection{Experiment Setup}

\subsubsection{Train Data-Set Preparation}
%The data set used for training the PV-RNN model was time series data of Torobo joint angles while performing four different cyclic movements \emph{A, B, C} and \emph{D}, where each movement pattern consists of 20 time-step long.
The proposed PV-RNN model was trained in a supervised manner by preparing 4-dimensional joint angle teaching trajectories.
In preparation for teaching trajectories, we considered four types of cyclic movement patterns (20 time-steps for each cycle) using Torobo's shoulder and elbow joint angles in both arms.
% Each cyclic movement is a simple movement that uses Torobo's shoulder and elbow joints, such as raising both hands and lowering them back.
Then, it was assumed that cycling movement patterns transit from one to another following a probabilistic finite state machine (Fig.\ref{fig:PFSM}).
For example, after a movement pattern \emph{A} is generated for one cycle with the state at \emph{S1}, \emph{A} can be generated for one more cycle with $90\%$ probability staying at the same state, or \emph{B} or \emph{C} can transit to \emph{S2} or \emph{S3} with a probability of $3\%$ and $7\%$, respectively. 
% When each cyclic movement ends, the transition occurs with a probability of around $10$ \%.
%For instance, while Torobo is performing \emph{A}, after each \emph{A}, it keeps generating \emph{A} with a probability of $90$ \%.
%On the other hand, with $3$ \% and $7$ \%, Torobo switches its movement into \emph{B} and \emph{C}, respectively.
\begin{figure}[htbp]
  \centering
  \includegraphics[width=0.5\textwidth]{figures/Probabilistic_FSM.pdf}
  \caption{Schematic of the probabilistic finite state machine from which training data was generated.}
  \label{fig:PFSM}
\end{figure}
%The data set was four-dimensional, where each dimension corresponds to the angles of Torobo's shoulders and elbows.
We prepared 10 sequences that each consisted of 200 cycles of movement patterns, extending 4000 time-steps.

\subsubsection{The Network Configuration and Training}
To conduct human-robot dyadic interaction experiments, PV-RNN was trained 3 times with identical parameters (Table.\ref{tab:parameters}).
$\#\mathbf{d}, \#\mathbf{z}, \tau, \mathbf{w}^t, \mathbf{w}^i$ indicates the number of d neurons and z neurons, time constant, and meta-prior during the training phase and interaction phase, respectively.
%Among each PV-RNN model, parameters remained the same except for the random seed used for generating random numbers to structure a different model from the same given data set.
The network was trained for 50,000 epochs to minimize the evidence-free energy shown in Eq.\ref{eq:evidenceF}, starting with random weights generated with different seeds for each training.
%
\begin{table}[htbp]\centering
\caption{PV-RNN Parameters}
\label{tab:parameters}
\begin{tabular}{cccccc}
\toprule\footnotesize
     & $\#\mathbf{d}$ & $\#\mathbf{z}$ & $\tau$ & $\mathbf{w}^{t}$ & $\mathbf{w}^{i}$\\
\cmidrule(lr){2-6}
\textbf{Layer 1 (Top)} & 60 & 6 & $3$  & 0.01 & $[0.01, 0.05, 0.1]$\\
\textbf{Layer 2 (Bottom)} & 30 & 3 & $9$ & 0.01 & $[0.01, 0.05, 0.1]$ \\
\bottomrule
\end{tabular}
\end{table}
%
% Fig. 4
Fig.\ref{fig:train_result} shows one of the resultant training processes.
\begin{figure}[htbp]
  \centering
  \includegraphics[width=0.7\textwidth]{figures/train_result.png}
  \caption{
    Resultant time-development from the training phase of one of the PV-RNN models.
  Development of KLD with respect to the number of training epochs is shown for (a) the Top layer (layer 2) and (b) the Bottom layer (layer 1).
  Development of the average prediction error over time steps in each teaching sequence is shown in (c).}
  %All three values clearly decrease with respect to epoch, which indicates that the PV-RNN is successfully learning the train data set.}
  \label{fig:train_result}
\end{figure}
It can be seen that prediction error and the KL divergence in both the top and the bottom layers decreased throughout training.
All three training processes converged in a similar way, achieving prediction errors and KL-divergences shown in Table.\ref{tab:TrainingResult}.


\begin{table}[htbp]
\centering
\caption{Training processes of all three PV-RNN model}
\label{tab:TrainingResult}
\begin{tabular}{ ccc  c c }
\cmidrule(lr){1-5}
 && \multicolumn{3}{c}{Epoch} \\
\cmidrule(lr){3-5}
\cmidrule(lr){3-5}
     & Model &    2,000     &       5,000       &    50,000\\
\cmidrule(lr){2-5}
\multirow{3}{*}{KL divergence} & 1 & $2.0 \times 10^{-3}$ & $5.4 \times 10^{-4}$ & $4.4 \times 10^{-5}$\\
& 2     &  $2.4 \times 10^{-3}$ &  $9.1 \times 10^{-4}$  & $5.6 \times 10^{-5}$\\
& 3    & $2.5 \times 10^{-3}$ &  $7.2 \times 10^{-4}$  &  $6.9 \times 10^{-5}$\\
\cmidrule(lr){1-5}
\multirow{3}{*}{Pred. Err.} &  1       &$1.8 \times 10^{-2}$ &  $7.2 \times 10^{-4}$  & $1.4 \times 10^{-4}$\\
& 2     & $1.4 \times 10^{-2}$ &  $6.6 \times 10^{-3}$  & $4.6 \times 10^{-5}$\\
& 3    & $1.3 \times 10^{-2}$  &  $1.6 \times 10^{-3}$  &  $1.4 \times 10^{-4}$\\
 \hline
\end{tabular}
\end{table}
%
% One of the training results over the number of epochs is shown in Fig.\ref{fig:train_result}. top and the middle row of Fig.\ref{fig:train_result} are the KL-divergence calculated in layer 1 (top layer) and layer 2 (bottom layer), respectively. The bottom row is the mean squared error calculated following Fig.\ref{eq:MSE} from the $target$ value, which in this case is the data set and the output of the PV-RNN model.

\subsubsection{Evaluation of the trained networks}
Trained PV-RNNs were evaluated on the basis of how closely the state transition probability wasreconstructed compared with the target (Fig.\ref{fig:PFSM}). 
For this purpose, we conducted \emph{Prior Generation} in which the forward computation by following Eq.\ref{eq:hiddenlayer}-\ref{eq:output} was performed without any external observation for 40,000 time-steps for each trained network.
Then, probabilities for all possible movement pattern transitions were measured during prior generation and the resulting state transition probability was inferred (Table.\ref{tab:TrainEvaluation}).
The state transition probability computed for all trained networks is quite similar to the target one. 
Output and network dynamics of movement pattern transitions during prior generation are shown in Appendix, Fig.\ref{fig:PriorGeneration}.
%
\begin{table}[htbp]
    \centering
    \caption{Transition probabilities of the \emph{Prior Generation} and data set}
    \begin{tabular}{cccccc}
    \cmidrule(lr){1-6}
    Model & \emph{S1} $\rightarrow$ \emph{S2} & \emph{S1} $\rightarrow$  \emph{S3} & \emph{S2} $\rightarrow$  \emph{S4} & \emph{S3} $\rightarrow$  \emph{S4} & \emph{S4} $\rightarrow$  \emph{S1} \\
    \cmidrule(lr){2-6}
    1 & 2.8 \% & 3.5 \% & 15.6 \% & 12.5 \% & 3.1 \% \\
    2 & 2.7 \% & 9.3 \% & 14.1 \% & 11.5 \% & 7.2 \% \\
    3 & 3.1 \% & 4.1 \% & 9.8 \%  & 12.3 \% & 5.1 \% \\
    Target & 3 \% & 7 \% & 10 \% & 15 \% & 5 \% \\
    \cmidrule(lr){1-6}
    \end{tabular}
    \label{tab:TrainEvaluation}
\end{table}

\subsubsection{Human-Robot Interaction}
After the training phase, we conducted two types of human-robot interaction experiments using trained PV-RNNs.
In these experiments, while Torobo was generating movement pattern transitions successively based on the training, the human experimenter attempted to induce various movement pattern transitions by grasping both arms of Torobo and exerting force on them. 
These movement pattern transitions included trained transitions (\emph{AB, AC, BD, CD, DA}) and untrained transitions (\emph{AD, DB, DC, BA, CA}) in which \emph{AB}, for example, dictates that \emph{A} pattern is forced to transit to \emph{B} pattern.
Each transition from one pattern to another requires some guiding force by a human experimenter, even for trained transitions, since ongoing patterns tend to repeat another cycle with high probability, more than $85\%$ for all patterns.
This means that there exist some conflicts between movement trajectories intended by the robot and those by the human experimenter during both trained and untrained transitions.

Experiment-1 examined the effect of $w^i$ settings on the interactions.
While Torobo was generating movement pattern transitions successively for 2,200 time-steps, the human experimenter attempted to induce one of the trained transition every 200 time-steps starting from $t = 200$ which resulted in 10 trained transitions.
The duration of each attempt lasted 100 time-steps at the most.
This experiment was conducted three times for all three trained PV-RNNs by changing $w^i$ with $0.01$, $0.05$, and $0.1$.

Experiment-2 examined the difference between trained and untrained transitions by setting $w^i$ with a fixed value.
The experimental procedure was the same as in Experiment-1, although this time, $w^i$ was fixed at $0.01$ for both layers.
This experiment, using all three trained PV-RNNs, resulted in 30 untrained transition attempts.
In these experiments, we recorded the time development of essential values including latent variables, predicted and observed joint angles, prediction error, and the KL-divergence in both layers for later analysis of experiment results.

\subsection{Experiment Results}
In this section, we show the results of the aforementioned experiments.

\subsubsection{Experiment-1}
First, we examined time-development of essential values during movement pattern transitions induced by the experimenter for each case with a different meta-prior setting.
Fig.\ref{fig:TimeDevelopmentDifferentW} shows an example snapshot of future prediction and past reflection, which shifted every 7 time-steps of the current time during the transition \emph{AB} performed under different settings of meta-prior, $w^i = 0.01, 0.05, 0.1$.
% Fig.5
\begin{figure*}[h] % 
\begin{center}
\includegraphics[width=0.95\linewidth]{figures/TimeDevelopment.png}
\caption{Time-development of excess torque, prediction error, and KL-divergence between the approximate posterior and the prior in trained movement transitions
in cases with three different meta-prior $w^i$ settings with $0.01$, $0.05$, and $0.1$. The grey area represents the past window where the head of the window is the current time. The window is shifted 5 times during the movement transition \emph{AB}.
}
\label{fig:TimeDevelopmentDifferentW}
\end{center}
\end{figure*}
%
Each snapshot shows one of the observed joint angles $\theta$ (dotted blue) and its prediction $\Bar{\theta}$ (blue) in the top row, the KL-divergences between the approximate posterior and the prior in the layer 1 (orange) and in the layer 2(dark orange) in the second row, the prediction error (green) in the third row, and the excess torque (black) in the bottom row.
The grey area represents the past window where the approximate posterior in terms of adaptive variables $\mathbf{A}_t^{\mu}, \mathbf{A}_t^{\sigma}$ is updated.
We provide two supplementary videos for experiment-1 showing the interaction between the experimenter and Torobo, as well as network dynamics in the case with the meta-prior $w^i$ set to 0.01 (\href{https://youtu.be/jQGnfPMAWes}{video-link1}) and 0.1 (\href{https://youtu.be/th1-5Ay603Y}{video-link2}).

The sequences of time-shifted snapshots in Fig.\ref{fig:TimeDevelopmentDifferentW}, show that excess torque appears first followed by rises in the prediction error (negative log-likelihood) and the KL-divergence. 
Later, the predicted joint angle pattern shifts from pattern \emph{A} to \emph{B} while the gap between the observed joint angle and the reconstructed joint angle remains in the past window.
This is the same for all three cases with different $w^i$ settings.

However, we can see some qualitative differences in the transition process depending on the $w^i$ setting.
The prediction error and the excess torque in the case with a small $w^i$ ($w^i=0.01$) are smaller than those in the case with large $w^i$ ($w^i=0.1$).
Also, the error and torque with a small $w^i$ are less persistent than those with a larger $w^i$.
However, the KL-divergence in the case with a small $w^i$ is larger and persists longer than that in the case with large $w^i$.
In order to confirm these observations, we conducted statistical analysis on the excess torque, KL-divergence, and prediction error time-averaged for each transition period (100 steps). 
This computation was repeated 10 times for each of three different trained networks set with three different $w^i$ values.
%In order to confirm these observations, we conducted statistical analysis on the time-averaged excess torque, KL-divergence, and prediction error in each transition repeated 10 times for each of three different trained networks set with three different $w^i$ values.

The results are shown in Fig.\ref{fig:TransitionForDifferentW}.
%
% Fig. 6
\begin{figure}[h] % 
\begin{center}
\includegraphics[width=0.95\linewidth]{figures/Exc_KL_MSE_per_attempt.png}
\caption{Time-averaged excess torque, prediction error, and the KL-divergence between the approximate posterior and the prior in the trained movement transition in cases with three different meta-prior $w^i$ settings with $0.01$, $0.05$, and $0.1$.
}
\label{fig:TransitionForDifferentW}
\end{center}
\end{figure}
%
Both the time-averaged prediction error and the excess torque measured in the cases with small $w^i$ are significantly smaller than those with large $w^i$.
On the other hand, the time-averaged KL-divergence with small $w^i$ is significantly larger than that with large $w^i$.

By considering these statistical results, movement pattern transitions exerted by the experimenter require greater force when $w^i$ is set larger, since the approximate posterior distribution strongly follows the prior distribution representing the current movement intention of PV-RNN, minimizing the KL-divergence between them while the error $\Delta \theta_t$ between the predicted joint angle $\bar{\theta_t}$ and the observed joint angle $\theta_t$ becomes larger.
If the experimenter attempts to move the trajectory of the robot's joint angles in a direction different from that predicted by the PV-RNN, this requires a large excess torque $\tilde{e}_t$ to counteract the large error $\Delta \theta_t$, as derived from Eq. \ref{eq:target joint angle}.

On the other hand, when $w^i$ is set smaller, the approximate posterior follows the prior only weakly, allowing larger KL-divergence between them while the prediction error becomes smaller.
In this case, only a small amount of excess torque is necessary to counteract the small error.
The top-down actional intention of the robot became stronger in the case of larger $w^i$ settings; therefore, the human experimenter was required to exert more force on the robot arms to induce a transition, whereas less force was required with smaller $w^i$ settings, since the top-down actional intention of the robot became weaker.

In the current experiment, the value of $w^i$ was set between $0.01$ and $0.1$.
This is because our preliminary experiments showed that the robot's behaviour became noisy when $w^i$ was set smaller than $0.01$, since the approximate posterior could easily deviate from the prior by noise sampling.
On the other hand, it became difficult for the experimenter to initiate a transition when $w^i$ was set larger than $0.1$, because of substantially increased resistance. 

\subsubsection{Experiment-2}
Next, we looked at the difference in the excess torque, prediction error, and KL-divergence between trained and untrained transitions while $w^i$ was fixed at a given value.
Both trained and untrained transitions were attempted 10 times for each of the three trained networks with $w^i$ set to $0.01$ for both trained and untrained cases.
However, for the untrained case, among 30 attempts, only 24 succeeded in performing a transition.
Our preliminary experiment showed that the untrained transition became more difficult when $w^i$ was set higher than $0.01$ because of the strong resistance when the robot attempted to lead trained movement transitions.
Fig.\ref{fig:TrainedVSUntrained} shows the time-average of the excess torque, prediction error, and KL-divergence for both trained and untrained transitions attempted by the experimenter.
% Fig. 7
\begin{figure}[h] % 
\begin{center}
\includegraphics[width=0.95\linewidth]{figures/TrainedVSUntrained_box.png}
\caption{The amount of excess torque, prediction error exerted and the KL-divergence between the approximate posterior and the priorc during each attempt for trained and untrained transitions. The meta-prior $w^i$ was set to $0.01$.
}
\label{fig:TrainedVSUntrained}
\end{center}
\end{figure}
%
The time-average of the prediction error, the excess torque, and the KL-divergence are larger in untrained than trained transitions.
This means that untrained transitions require the experimenter to exert more force than for trained transitions because of the free energy, which is the sum of the prediction error and the KL-divergence, and which increases more in untrained transitions.
We provide two supplementary videos for experiment-2 showing the interaction between the experimenter and Torobo as well as the network dynamics in the case with meta-prior $w^i$ set to 0.01, where trained transitions (\href{https://youtu.be/jQGnfPMAWes}{video-link1}) and untrained transitions (\href{https://youtu.be/lZEn5Qvun90}{video-link3}) are performed.

%%%%%%%%%%% END
%%%%%%%%%%%






%%%%%%%%%%%%%%%%%%%%%%%%%%%%%%%%%%%%%%%%%%%%%%%
%%%%%%%        4. Results         %%%%%%%
%%%%%%%%%%%%%%%%%%%%%%%%%%%%%%%%%%%%%%%%%%%%%%%

\section{Results}
\label{sec:results}

\subsection{MOS prediction results}
\label{subsec:mos_results}
We first evaluate our MOS-prediction performance in comparison with other approaches. In particular, we compare against NISQA~\cite{mittag2019non}, which we modified to estimate human-accessed MOS. Originally, they estimate perceptual objective listening quality assessment (POLQA)~\cite{beerends2013perceptual} scores using a CNN and BLSTM architecture. We also compare against the PMOS model proposed in~\cite{dong2020pyramid}, which is identical in structure to our PMOS model. Finally, we include our proposed SE+PMOS approach~\cite{nayem2021incorporating} (no joint training), where our PMOS model is held fixed while the SE model is training using the embeddings from the PMOS encoder. 

We use four metrics to evaluate MOS-estimation performance: mean absolute error (MAE), epsilon insensitive root mean squared error (RMSE)~\cite{rec2012p}, Pearson’s correlation coefficient $\gamma$ (PCC), and Spearman’s rank correlation coefficient $\rho$ (SRCC). 

%    Later, both models are jointly-trained for fine tuning. Our proposed PMOS model is similar of \cite{nayem2021incorporating}, however, SE models are different in structure.

%%%%%%%%%%%%%%%%%%%%%%%%%%%%%%%%%%%%%%%%%%%%%%%%%%%%
% Table 1, MOS results
%%%%%%%%%%%%%%%%%%%%%%%%%%%%%%%%%%%%%%%%%%%%%%%%%%%%
\begin{table}[t!]

\centering
\caption{Performance comparison with MOS prediction models {comparing against the ground truth MOS obtained from human subjects}. Best results are shown in \textbf{bold}.}
\label{tab:mos_results}
% \vspace{-0.5em}
\resizebox{\columnwidth}{!}{%
\begin{tabular}{| l | c c c c | }
\cline{2-5}
   \multicolumn{1}{c|}{}         & {MAE}$\downarrow$ & {RMSE}$\downarrow$ & {PCC ($\gamma$)}$\downarrow$ & {SRCC ($\rho$)}$\downarrow$ \\ \hline
   
NISQA~\cite{mittag2019non}    & 0.62 ($\pm$0.18)        & 0.7 ($\pm$0.16)      & 0.71 ($\pm$0.14)           & 0.79 ($\pm$0.15)            \\
PMOS~\cite{dong2020pyramid}                      & 0.51 ($\pm$0.15)         & 0.57 ($\pm$0.12)          & 0.88 ($\pm$0.17)           & 0.88 ($\pm$0.14)           \\
SE+PMOS~\cite{nayem2021incorporating}                     & \textbf{0.45} ($\pm$0.08) & \textbf{0.52} ($\pm$0.09) & \textbf{0.9} ($\pm$0.12) & \textbf{0.91} ($\pm$0.1)           \\
Proposed                     & \textbf{0.45} ($\pm$0.08) & \textbf{0.52} ($\pm$0.09) & \textbf{0.9} ($\pm$0.12) & \textbf{0.91} ($\pm$0.1)         \\
\hline
\end{tabular}
}
% \vspace{-2em}
\end{table}

Table~\ref{tab:mos_results} shows the results, where our proposed approach and SE+PMOS clearly outperform the other MOS prediction models according to all metrics. MAE is minimized by $0.6$ compared to the original PMOS~\cite{dong2020pyramid} approach. There is also a $0.05$ reduction in RMSE. This justifies our proposed approach that combines MOS estimation and speech enhancement tasks. Note, however, that similar results are obtained for our proposed approach and the SE+PMOS approach, which suggests that joint training (e.g., fine tuning) may help speech enhancement more than MOS prediction.  




\subsection{Speech enhancement model}
\label{subsec:se_results}
%%%%%%%%%%%%%%%%%%%%%%%%%%%%%%%%%%%%%%%%%%%%%%%%%%%%
% Table 2, SE comparison results on COSINE & VOiCES
%%%%%%%%%%%%%%%%%%%%%%%%%%%%%%%%%%%%%%%%%%%%%%%%%%%%

% Please add the following required packages to your document preamble:
% \usepackage{multirow}
% \usepackage[table,xcdraw]{xcolor}
% If you use beamer only pass "xcolor=table" option, i.e. \documentclass[xcolor=table]{beamer}
\begin{table*}[t!]
\centering
\caption{Average results of the speech enhancement models in different performance metrics. Best results are shown in \textbf{bold}.}
\label{tab:results_cosineVoices}
\resizebox{\linewidth}{!}{%
\begin{tabular}{ | l | l | c c c c | c c c c | }
\cline{3-10}
\multicolumn{1}{l}{\multirow{2}{*}{}} &                    & \multicolumn{4}{ c |}{{COSINE}}                             & \multicolumn{4}{ c |}{{VOiCES}} 
\\ \hline
\multicolumn{1}{|l|}{{models}}                 & \multicolumn{1}{c|}{{loss func.}} & {PESQ}$\uparrow$ & {SI-SDR}$\uparrow$ & {ESTOI}$\uparrow$ & {MOS-LQO}$\uparrow$ & {PESQ}$\uparrow$ & {SI-SDR}$\uparrow$ & {ESTOI}$\uparrow$ & {MOS-LQO}$\uparrow$ \\ \hline
\multicolumn{1}{|l|}{{Mixture}}                                        & {-}                                                       & {1.46} & {0.53}   & {0.62}  & {4.04}    & {1.26} & {-1.3}   & {0.48}  & {2.74}    \\ \hline
\multicolumn{1}{|l|}{}                                                        & mse                                                              & 2.68          & 2.8             & 0.8            & 3.2              & 2.3           & 1.2             & 0.69           & 3.5              \\ 
\multicolumn{1}{|l|}{}                                                        & mos~\cite{fu2019learning}                                                              & 2.8           & 3.8             & 0.82           & 4.2              & 2.37          & 1.66            & 0.74           & 5.3              \\ 
\multicolumn{1}{|l|}{}                                                        & mse+sa                                                           & 2.72          & 3.1             & 0.82           & 4                & 2.35          & 1.6             & 0.7            & 3.8              \\ 
\multicolumn{1}{|l|}{}                                                        & mos+sa                                                           & 2.89          & 4.1             & 0.85           & 4.4              & 2.42          & 1.72            & 0.77           & 5.7              \\ 
\multicolumn{1}{|l|}{\multirow{-5}{*}{SE}}                                    & sdr~\cite{kawanaka2020stable}                                                              & 2.7           & 4.5             & 0.82           & 3.4                & 2.32          & 2.01            & 0.72           & 3              \\ \hline
\multicolumn{1}{|l|}{{ }}                                 & mse                                                              & 3.1           & 4               & 0.85           & 4.2              & 2.48          & 1.8             & 0.8            & 6                \\ 
\multicolumn{1}{|l|}{{}}                                 & mse+sa                                                           & 3.19          & 4.6             & 0.93           & 4.8              & 2.54          & 2.08            & 0.86           & 6.3              \\  
\multicolumn{1}{|l|}{\multirow{-3}{*}{SE+PMOS~\cite{nayem2021incorporating}}}        & mse+sa+mos                                                       & 3.19          & 4.5             & 0.92           & \textbf{5.1}     & 2.53          & 2.06            & 0.84           & \textbf{6.5}     \\ \hline
\multicolumn{1}{|l|}{}                                                        & pesq                                                             & \textbf{3.28} & 4.4             & 0.9            & 5                & \textbf{2.67} & 2.01            & 0.83           & 6.1              \\ 
\multicolumn{1}{|l|}{\multirow{-2}{*}{MetricGAN~\cite{fu2019metricGAN}} }                             & stoi                                                             & 3.19          & 4.3             & \textbf{0.94}  & 4.8              & 2.5           & 2               & \textbf{0.87}  & 5.8              \\ \hline
\multicolumn{1}{|l|}{SSEMS~\cite{zezario2019specialized}}                                                   & qnet ($\phi=0dB$)                                                       & 2.85          & 2.99            & 0.83           & 3                & 2.4           & 1.8             & 0.7            & 2.8              \\ \hline
\multicolumn{1}{|l|}{{Chi++\textsubscript{fQSM,bS}~\cite{nayem2021towards}}}                     &    dc+cls+sa                                                              & 2.9           & 3.3             & 0.84           & 3.4              & 2.44          & 1.78            & 0.7            & 3                \\ \hline
\multicolumn{1}{|l|}{}                                 & mse+sa                                                           & 3.25          & 4.8             & \textbf{0.94}  & 4.75             & 2.64          & 2.1             & \textbf{0.87}  & 6.2              \\ 
\multicolumn{1}{|l|}{\multirow{-2}{*}{Proposed}} & mse+sa+mos                                                       & 3.25          & \textbf{4.82}   & \textbf{0.94}  & 5.04             & 2.64          & \textbf{2.13}   & \textbf{0.87}  & 6.47             \\ \hline
\end{tabular}
}
\end{table*}
For speech enhancement, we compare against a baseline approach without an attention mechanism \cite{graves2013speech}. We denote this baseline approach as SE. Five separate loss functions are applied to optimize this approach, and they are MSE, MSE plus signal approximation, MOS, signal approximation with MOS, and SDR. To compute the MOS loss function, we utilize the SE loss function from \cite{fu2019learning} which leverages objective-MOS (oMOS) ratings learned from a speech assessment model~\cite{fu2018quality}. SDR~\cite{kawanaka2020stable} loss functions are proposed in literature previously with different enhancement architectures. For the SDR loss function, the SE model is optimized using the following cost function:
\begin{align}
    \mathcal{L}_{SDR} = \sum_{n=1}^N \mathcal{K}_{\theta}  \Big( 10 \log \frac{\Vert s^n\Vert^2}{\Vert s^n-\hat{s}^n\Vert^2} \Big)
\end{align}
where $\mathcal{K}_\theta(a)=\theta\cdot \tanh(\frac{a}{\theta})$, $\theta$ is a clipping parameter, $N$ is the mini-batch size, and $s^n$ and $\hat{s}^n$ are the n\textsuperscript{th} sample of the clean and estimated speech signal in time. We use $\theta=20$ in our training. We also compare against a generative adversarial network (GAN) approach that individually optimizes with PESQ and STOI~\cite{fu2019metricGAN}. We denote this model as MetricGAN. 
% They estimate the IRM conditioned on continuous space of the discriminator label based on either PESQ or STOI target label. 
They estimate the IRM for a speech mixture conditioned on a GAN discriminator that outputs evaluation scores in continuous space (i.e. scores between 0 and 1) based on either normalized PESQ or STOI target metrics. 
We compare our model with the ensemble-based Specialized Speech Enhancement Model Selection (SSEMS) approach~\cite{zezario2019specialized} that uses Quality-Net~\cite{fu2018quality} as its objective function in a black-box manner. Quality-Net is an oMOS approach that estimates the Perceptual Evaluation of Speech Quality (PESQ) score. The SSEMS approach uses an ensemble of enhancement models, each trained on audio at specific SNRs and speaker genders. During inference, it selects the output with the highest PESQ score. SSEMS uses a SNR threshold of $20$ dB, while we use a threshold of $0$ dB for balanced training and better performance. Additionally, we conduct a comparison with our initial approach that integrates MOS embeddings in speech enhancement, as presented in \cite{nayem2021incorporating}. This model is referred to as SE+PMOS, and it does not involve joint training or the QSM language model. We evaluate SE+PMOS with varying combinations of loss functions. %We compare against a quantized speech enhancement model which utilizes a spectral language model~\cite{nayem2021towards}. This model is motivated from chimera++~\cite{wang2018alternative} in structure with BLSTM layers and deep clustering (dc) loss.
%Traditional chimera++ model estimates a phase-sensitive mask which has been applied in the task of speech enhancement in non-speech noisy conditions with multi-talker speech~\cite{wichern2019wham, yang2019improved}. However, in \cite{nayem2021incorporating}, they estimate quantized speech signal, not mask; they use cross-entropy classification (cls) loss, and signal approximation loss altogether. They report best results using per-frequency quantized spectral model (fQSM) as language model for beam search (bS) with beam size $100$. We use this model as our comparison model denoting as Chi++\textsubscript{fQSM,bS}. 
All models are trained using the experimental setup that is previously mentioned. We modify the comparison models using the code provided by the original authors.

We assess speech enhancement performance using PESQ~\cite{rix2001perceptual}, scale-invariant SDR (SI-SDR)~\cite{le2019sdr}, and extended STOI (ESTOI)~\cite{jensen2016algorithm}. In the absence of actual human quality objective, we measure the predicted MOS score of the enhanced speech, using our proposed PMOS model, since we aim to improve human-assessed speech quality. We denote this metric as MOS listener quality objective (MOS-LQO). Table~\ref{tab:results_cosineVoices} shows the average results of the different enhancement models, according to each of the performance metrics on COSINE and VOiCES dataset. As the scores of the unprocessed mixtures show, the VOiCES corpus is  more challenging than the COSINE corpus. 
With the baseline SE model, we experiment with 5 different combination of loss functions. Using the MSE loss only in SE:mse, we see improvements in objective scores, except with MOS-LQO for the COSINE data. Then we apply a MOS loss $\mathcal{L}_{mos}$ as the sole objective criterion, as proposed in \cite{fu2019learning}. Our experimental results show that this approach results in an overall improvement of $1.4$ in MOS-LQO compared to SE:mse. %We apply MOS-LQO scores of enhanced speech to calculate MOS loss $\mathcal{L}_{mos}$ as the only objective criteria as proposed in \cite{fu2019learning}, which gives improves MOS-LQO by $1.4$ overall compared with SE:mse. 
Then we separately combine the signal approximation loss with the mse loss and MOS loss (e.g., mse+sa and mos+sa). In PESQ, we gain an average of $\ge0.05$ and $\ge0.07$ compared to the models that use only the MSE loss and only the MOS loss, respectively. Furthermore, the model trained with the mos+sa loss function achieves the highest MOS-LQO score of $4.4$ and $5.7$ among all five loss functions tested with the SE model in COSINE and VOiCES dataset, respectively. This result is on average $1.15$ MOS-LQO higher than that obtained with the mse+sa loss function. These scores suggest that $\mathcal{L}_{mse}$ and $\mathcal{L}_{sa}$ maximize the overall speech intelligibility, whereas $\mathcal{L}_{mos}$ guides the model towards perceptual speech quality. Note that in all these $\mathcal{L}_{mos}$ calculations, we use a separately trained PMOS model's output without joint learning.
Lastly, we apply the SDR loss function as proposed in \cite{kawanaka2020stable}, which is used as the pre-training stage for model training. We observe an average gain of $0.9$ in SI-SDR, however, it yields a poor score according to other metrics, especially a $0.7$ loss in MOS-LQO compared to SE with mse and sa loss terms. 

SE+PMOS is separately investigated with 3 combinations of loss functions, i.e. mse, mse+sa, and mse+sa+mos. Compared with SE models, SE+PMOS with mse loss achieves $0.9$ SI-SDR and $1.75$ MOS-LQO improvements on average, which shows the benefit of incorporating the PMOS model. The SE+PMOS:mse+sa model improves the performance further with an average of $0.14$ ESTOI gain over the SE:mse+sa model. The inclusion of the mos loss gives the best MOS-LQO scores of $5.1$ and $6.5$ over all the comparison models in noisy and reverberant conditions, respectively.

%%%%%%%%%%%%%%%%%%%%%%%%%%%%%%%%%%%%%%%%%%%%%%%%%%%%
% Table 3, SE comparison test results on CHiME 5+4 
%%%%%%%%%%%%%%%%%%%%%%%%%%%%%%%%%%%%%%%%%%%%%%%%%%%%

% Please add the following required packages to your document preamble:
% \usepackage{multirow}
\begin{table*}[t!]
\centering
\caption{Average testing results of the speech enhancement models on CHiME-5 and CHiME-4 datasets. Best results are shown in \textbf{bold}.}
\label{tab:results_chime}
\resizebox{\linewidth}{!}{%
\begin{tabular}{| l | l | c c c c c | c c c c c |}
\cline{3-12}
\multicolumn{1}{l}{\multirow{2}{*}{}} &                    & \multicolumn{5}{ c |}{{CHiME-5}}                             & \multicolumn{5}{ c |}{{CHiME-4}} 
\\ \hline
\multicolumn{1}{|l|}{models}                          & \multicolumn{1}{c|}{loss func.} & PESQ$\uparrow$          & SI-SDR$\uparrow$       & ESTOI$\uparrow$         & MOS-LQO$\uparrow$      & WER\%$\downarrow$         & PESQ$\uparrow$          & SI-SDR$\uparrow$        & ESTOI$\uparrow$         & MOS-LQO$\uparrow$      & WER\%$\downarrow$         \\ \hline
\multicolumn{1}{|l|}{Mixture}                & -                               & 1.7           & 2.4          & 0.52          & 3.8          & 152.1         & 1.96          & 2.86          & 0.6           & {4.6} & {33.7} \\ \hline
\multicolumn{1}{|l|}{SE}                              & mos+sa                          & {2.25} & {3.9} & {0.62} & {4}   & {96.4} & {2.32} & {5.22} & {0.63} & {5}   & {25.6} \\ \hline
\multicolumn{1}{|l|}{SE+PMOS}                         & mse+sa+mos                      & 2.37          & 6.1          & 0.67          & 4.4          & 84.5          & 2.45          & 7.6           & 0.7           & 5.8          & 22.6          \\ \hline
\multicolumn{1}{|l|}{\multirow{2}{*}{MetricGAN}}      & pesq                            & \textbf{2.44} & {6.3} & {0.65} & {4.1} & {94.8} & \textbf{2.51} & {7}    & {0.68} & {5.3} & {19.7} \\ 
\multicolumn{1}{|l|}{}                                & stoi                            & 2.39          & 6.2          & \textbf{0.71} & 4.1          & 91.3          & 2.45          & {6.45} & \textbf{0.73} & 5.6          & 21.5          \\ \hline
\multicolumn{1}{|l|}{\multirow{2}{*}{Proposed}} & mse+sa                          & 2.41          & 7.1          & {0.68} & 4.7          & \textbf{78.3} & 2.5           & {7.9}  & 0.72          & 5.76         & \textbf{18.1} \\ 
\multicolumn{1}{|l|}{}                                & mse+sa+mos                      & 2.41          & \textbf{7.3} & {0.68} & \textbf{4.9} & 79.4          & {2.5}  & \textbf{8.61} & \textbf{0.73} & \textbf{6}   & 18.9          \\ \hline
\end{tabular}}
\end{table*}
MetricGAN optimizes PESQ or STOI, therefore, it outperforms other comparison models in terms of PESQ and ESTOI, although the scores for the SE+PMOS approaches are higher according to the other evaluation metrics even though these metrics are not leveraged during training. 
SSEMS yields the lowest scores across all metrics compared with SE+PMOS and MetricGAN approaches, though we do parameter tuning for this model.
Chi++\textsubscript{fQSM,bS} estimates quantized speech, and the results show that it affects the traditional objective functions. This performs poorly compared with the SE+PMOS and MetricGAN approaches, however, on average, it outperforms SSEMS in all criteria, and the SE models in terms of PESQ. With the MOS-LQO criteria, it fails to produce good scores. This points out the importance of incorporating perceptual features during enhancement, which Chi++\textsubscript{fQSM,bS} clearly lacks.

We calculate the performance of our proposed model using two combinations of loss functions. 
Using only mse and sa loss terms, we achieve the highest ESTOI scores for both corpora, though these results are nearly identical to the model trained with all three loss terms. Using $\mathcal{L}$ (eq:\ref{eq:loss}) in our proposed model, we obtain the highest SI-SDR scores while maintaining similar PESQ and ESTOI performance as compared to the best-performing model. Specifically, our proposed model achieves the highest ESTOI score and an average PESQ score that is only $0.03$ less than that of the best performing MetricGAN:pesq model.
Contrasting with the Chi++\textsubscript{fQSM,bS} model, which uses spectral language model to estimate quantized speech, our proposed approach outperforms the quantized model according to all metrics, which proves the significance of joint learning.% to direct speech enhancement model towards perceptually better speech using a speech quality assessment model.
When comparing MOS-LQO scores, our proposed:mse+sa+mos model achieves better scores than the other models except the SE+PMOS:mse+sa+mos model with an average of only $0.05$ declination. Thus, the inclusion of a spectral language model helps the model proposed (e.g., mse+sa+mos) to estimate better quality speech according to the overall evaluation criteria. 
It is important to note that our proposed approach performs best according to SI-SDR in both noisy and reverberant environments, where this metric is not used by any of the approaches during optimization.  

We further examine our approaches using completely unseen corpora. We test models with the CHiME-5 and CHiME-4 corpora where the models are trained from the COSINE dataset according to the system setup mentioned in section~\ref{subsec:setup}. Table~\ref{tab:results_chime} shows the performance evaluated according to PESQ, SI-SDR, ESTOI, MOS-LQO, and word error rate (WER). To calculate WER, we use the conventional ASR baseline that is provided with CHiME-5 and CHiME-4 dataset. We investigate WER with both GMM based ASR and end-to-end ASR, however, we find that the end-to-end approach results in a higher error compared to the GMM baseline. This might happen due to larger data requirements of the end-to-end ASR system as mentioned in \cite{barker2018fifth}. Therefore, we use the GMM ASR approach to compare the WER performance of the enhancement models.
From the scores of mixtures, we find that CHiME-5 is more challenging than CHiME-4 with a $118.8\%$ higher WER and a $0.46$ lower SI-SDR. Our proposed approach yields the best MOS-LQO scores with $4.9$ with CHiME-5 and $6$ with CHiME-4 data. The proposed mse+sa model results in the lowest WER of $78.3$ and $18.1$ using CHiME-5 and CHiME-4, respectively. Note that the WER of the GMM baseline ASR for the CHiME-5 challenge is $72.8$ in binaural and $91.7$ in single array conditions. Here our approaches enhance monaural speech, a more challenging condition. Our proposed approach outperforms other comparison models in terms of SI-SDR with a $5.29$ average improvement compared to others. According to PESQ and ESTOI metrics, MetricGAN variants give the best performace, however, proposed model's performance is $0.02$  and $ 0.015$ lower according to PESQ and ESTOI, respectively, for the best performing MetricGAN models. Hence, our proposed approach is effective on out-of-vocabulary scenario trained by a comparable dataset.


% \nayem{*** Possibly add graphs of evaluation metrics vs SNRs.}

%%%%%%%%%%%%%%%%%%%%%%%%%%%%%%%%%%%%%%%%%%%%%%%%%%%%
% Table 3, DNSMOS results
%%%%%%%%%%%%%%%%%%%%%%%%%%%%%%%%%%%%%%%%%%%%%%%%%%%%
% \begin{table}[thb!]

% \centering
% \caption{Average MOS ratings of the speech enhancement modes on CHiME-4 and CHiME-5 datasets using DNSMOS P.835~\cite{reddy2022dnsmos}. Best results are shown in \textbf{bold}.}
% \label{tab:dnsmos_results}
% % \vspace{-0.5em}
% \resizebox{\columnwidth}{!}{%
% \begin{tabular}{| l | c c | }
% \cline{2-3}
%   \multicolumn{1}{c|}{}         & {CHiME-4} & {CHiME-5} \\ \hline
   
% Mixture   & 1.54 ($\pm$0.85)         & 1.3 ($\pm$1.1)                \\
% PMOS+SE                      & 4.28 ($\pm$0.9)       & 3.67 ($\pm$1.3)\\
% MetricGAN                    & 4.26 ($\pm$0.87) & 3.5 ($\pm$1.34)          \\
% Proposed                     & \textbf{4.32} ($\pm$0.8)& \textbf{3.8} ($\pm$1.41)            \\ \hline
% Clean                     & 4.67 ($\pm$1.2) & -      \\
% \hline
% \end{tabular}
% }
% % \vspace{-2em}
% \end{table}

%%%%%%%%%%%%%%%%%%%%%%%%%%%%%%%%%%%%%%%%%%%%%%%%%%%%
% Fig 3, DNSMOS results plot
%%%%%%%%%%%%%%%%%%%%%%%%%%%%%%%%%%%%%%%%%%%%%%%%%%%%

\begin{figure}[b!]
    \centering
\begin{tikzpicture}
	\begin{axis}[
	    cycle list/Dark2-4,
		boxplot/draw direction = y,
		boxplot/box extend=0.8,
% 		x=3em,
% 		x axis line style = {opacity=0.6},
		axis x line* = bottom,
		axis y line = left,
		enlarge y limits,
		ymajorgrids,
		xtick = {1, 2, 3, 4, 5, 6, 7, 8},
		xticklabel style = {align=center, font=\small, rotate=60, alias={xtick-\ticknum}},
		xticklabels = {Mixture, SE+PMOS, MetricGAN, Proposed, Mixture, SE+PMOS, MetricGAN, Proposed},
% 		xtick style = {draw=none}, % Hide tick line
		ylabel = {MOS},
		ytick = {1, 2, 3, 4, 5},
	]
	
	\addplot+[
        boxplot prepared={
        lower whisker=1, lower quartile=1.45,
        median=1.74,
        upper quartile=2.5, upper whisker=4.05, }, fill, draw=black]
        coordinates {}
        node[above, color=black] at
        (boxplot box cs: \boxplotvalue{median},.5)
        {\scriptsize \pgfmathprintnumber{\boxplotvalue{median}}};
    \addplot+[
        boxplot prepared={
        lower whisker=1.38, lower quartile=1.84,
        median=2.28,
        upper quartile=3.1, upper whisker=4.3, }, fill, draw=black]
        coordinates {}
        node[above, color=black] at
        (boxplot box cs: \boxplotvalue{median},.5)
        {\scriptsize \pgfmathprintnumber{\boxplotvalue{median}}};
    \addplot+[
        boxplot prepared={
        lower whisker=1.3, lower quartile=1.75,
        median=2.13,
        upper quartile=3.2, upper whisker=4.1, }, fill, draw=black]
        coordinates {}
        node[above, color=black] at
        (boxplot box cs: \boxplotvalue{median},.5)
        {\scriptsize \pgfmathprintnumber{\boxplotvalue{median}}};
    \addplot+[
        boxplot prepared={
        lower whisker=1.4, lower quartile=1.9,
        median=2.46,
        upper quartile=3.16, upper whisker=4.34, }, fill, draw=black]
        coordinates {}
        node[above, color=black] at
        (boxplot box cs: \boxplotvalue{median},.5)
        {\scriptsize \pgfmathprintnumber{\boxplotvalue{median}}};
        
    \addplot+[
        boxplot prepared={
        lower whisker=1.0, lower quartile=1.35,
        median=1.64,
        upper quartile=2.39, upper whisker=4.18, }, fill, draw=black]
        coordinates {}
        node[above, color=black] at
        (boxplot box cs: \boxplotvalue{median},.5)
        {\scriptsize \pgfmathprintnumber{\boxplotvalue{median}}};
    \addplot+[
        boxplot prepared={
        lower whisker=1.31, lower quartile=1.8,
        median=2.18,
        upper quartile=2.76, upper whisker=4.24, }, fill, draw=black]
        coordinates {}
        node[above, color=black] at
        (boxplot box cs: \boxplotvalue{median},.5)
        {\scriptsize \pgfmathprintnumber{\boxplotvalue{median}}};
    \addplot+[
        boxplot prepared={
        lower whisker=1.26, lower quartile=1.71,
        median=2.06,
        upper quartile=3.17, upper whisker=4.32, }, fill, draw=black]
        coordinates {}
        node[above, color=black] at
        (boxplot box cs: \boxplotvalue{median},.5)
        {\scriptsize \pgfmathprintnumber{\boxplotvalue{median}}};
    \addplot+[
        boxplot prepared={
        lower whisker=1.34, lower quartile=1.85,
        median=2.25,
        upper quartile=3.07, upper whisker=4.48, }, fill, draw=black]
        coordinates {}
        node[above, color=black] at
        (boxplot box cs: \boxplotvalue{median},.5)
        {\scriptsize \pgfmathprintnumber{\boxplotvalue{median}}};
        
	\end{axis}
	
	\path (0,0) coordinate (P);
    \draw [thick,decoration={brace,mirror,raise=5em},decorate] (xtick-0|-P) -- (xtick-3.5|-P) 
        node[midway,yshift=-6em]{CHiME-4};
    \draw [thick,decoration={brace,mirror,raise=5em},decorate] (xtick-4|-P) -- (xtick-7.5|-P) 
        node[midway,yshift=-6em]{CHiME-5};

    % \node[text width=3cm] at (1.54,0.5) 
    % {\scriptsize 1.54};

\end{tikzpicture}

\caption{MOS ratings of the speech enhancement modes on CHiME-4 and CHiME-5 datasets using DNSMOS P.835.}
    % \vspace{-2em}
\label{fig:dnsmos_results}
    % \vspace{-0.4cm}
\end{figure}

\subsection{Perceptual quality evaluation}
\label{subsec:dnsmos}

We finally evaluate our model using P.835 metric~\cite{reddy2022dnsmos} to measure perceptual quality. We calculate the DNSMOS score on a scale of $[1-5]$ ($1$ = worst, $5$ = best) for the mixture, PMOS+SE, MetricGAN, and our proposed models using the CHiME-4~\cite{vincent2017analysis} and CHiME-5~\cite{barker2018fifth} datasets (simulated and real-recording). Figure~\ref{fig:dnsmos_results} shows the scores. With CHiME-4, the original mixture scores range from $1.45$ to $2.5$ with a median of $1.74$. Our proposed model achieves a median MOS of $2.46$, which is higher than the others. Fon CHiME-5, the original mixture scores range from $1.0$ to $4.18$. Our proposed model outperforms the others with a median of $2.25$. Our proposed model and PMOS+SE have smaller standard deviations compared to MetricGAN. Overall, our proposed model improves noisy speech in both the acoustic and perceptual aspects. 




% \subsection{Listening results}
% \label{subsec:listening_results}

% We conduct an IRB-approved listening study using Amazon Mechanical Turk to conceive the perceptual quality of enhanced speech assessed by normal-hearing listeners. 

% This study follows the design structure of \cite{nayem2021towards} and figure~\ref{fig:survey} shows the actual listener study interface of a single question. The study is conducted as follows, the participant will listen to two audio signals, one is enhanced and the other is clean audio as reference.  Then they provide a preference score using a Likert scale. The scale ranges from $-3$ to $+3$, where $-3$ refers to a strong preference towards the first signal, $+3$ refers to a strong preference towards the second signal, and $0$ refers to no preference. Before providing a score, the participant can listen to the signals as many as times they like, where the scores are not limited to integer values. The two signals are randomly selected, and the participant listens to different audio clips in each question. The audio clips are chosen from the CHiME-5 and CHiME-4 corpus spoken by both males and females in equal proportion. Prior to actual survey questions, each participants has to pass eligibility test and make themselves familiar with the upcoming study session by going through a practice session. The structure of this practice session is similar to the actual study, however, speakers' voice and audio clips which participants hear in practice session are not used in the actual study. A tentative feedback is provided in the practice session to give a guideline to the participants, however, to avoid biases and leading answers, the feedback is provided in a form of range where the expected answer should reside.



%  \begin{figure}[thb!]
%     \centering
%     \includegraphics[width = 0.5\linewidth]{IEEEtran/figs/survey.png}
%     % \vspace{-2em}
%     \caption{A question of actual listener study interface conducted on MTurk.}
%     \label{fig:survey}
%     % \vspace{-2em}
% \end{figure}

% \nayem{***One paragraph on the statistics of the conducted study.}
% The study session contains total 30 questions, which is preceded by a practice session of 7 questions. Ten participants (9 male, 1 female) who are native English speakers over the age of 18 participated, where a headset/headphone was required to be worn. On average, participants took 14 minutes to complete the study, they were given $\$3$ monetary incentive.


\section{Discussion}
\label{sec:discuss}

Our proposed model outperforms all comparison models on SI-SDR metrics for both seen and unseen datasets, without optimization of any of the models (Table \ref{tab:results_cosineVoices}, \ref{tab:results_chime}). This means that our approach improves speech quality by minimizing the distortion ratio when separated from the noise component. Additionally, our models yield the best MOS-LQO ratings on real-world captured audios (CHiME datasets, Table \ref{tab:results_chime}). These results are consistent with the findings of \cite{zezario2022deep, nayem2021incorporating} that incorporating embeddings from a speech assessment model improves SE performance, and the results of \cite{braun2022effect} that using MOS loss during model optimization leads to higher MOS-LQO scores. Our proposed approach achieves PESQ and ESTOI scores that are only slightly lower than those of the best-performing model, with a difference of only $0.03$ and $0.01$, respectively. This indicates that speech quality and intelligibility metrics are closely related to the subjective speech quality metric (MOS-LQO), and that these metrics can be improved without explicit optimization. Furthermore, our proposed model achieves the best average DNSMOS scores with low standard deviations on CHiME datasets (Figure \ref{fig:dnsmos_results}), indicating that it is effective in a wide range of real-world noise levels. This is a desirable quality for an effective SE model to be effective not only in high SNRs and limited noisy environments, but also in large SNR ranges and real-world conditions such as those offered by the CHiME dataset.

When comparing our proposed model that uses mse+sa+mos loss to the PMOS+SE model (as shown in Table \ref{tab:results_chime}), we can observe significant improvements in all performance metrics. As both models use the same loss function, the improvements are attributed to the incorporation of LM and the joint learning method. Moreover, we found that these two models exhibit similar performance on the MOS prediction (Table \ref{tab:mos_results}), indicating that the benefits of joint learning mostly impact the enhancement part of the model.

An intriguing finding is that our proposed model shows a decline in WER\% when MOS loss is incorporated, especially for larger real-world recordings such as CHiME-5, with degradation up to $1.1$. Although our study is not primarily concerned with ASR performance, this suggests a potential trade-off between ASR accuracy and subjective speech quality scores. Further investigation is needed to comprehend this relationship.

Our proposed method demonstrates that training a speech enhancement (SE) model and a MOS-based speech assessment model jointly can lead to better speech quality measured by objective metrics such as perceptual quality, intelligibility, and MOS ratings. However, we acknowledge that our study's use of subjective MOS (sMOS) estimation instead of actual human listeners may introduce discrepancies between MOS-LQO and human-rated MOS, which could impact our findings. To address this limitation, we plan to conduct sMOS evaluation by human listeners in future work. Although we used the same MOS prediction model for all comparison models, we believe that incorporating human-rated sMOS evaluations will provide more robust insights into our proposed method's effectiveness.
For computing loss terms, we opt for the MSE loss function along with a bi-gram language model that considers only time-along transitions. Our aim is to keep the model simple and focus on the effectiveness of our approach. However, we acknowledge that using different loss functions for different loss components and employing a more complex language model that considers both temporal and spectral transition levels can be beneficial. We plan to explore these possibilities in our future work.



% \newpage

% !TEX root = root.tex

\section{Conclusions and Future Work}
\label{sec:5_discussion}
Our work unifies the SF-GPI and value composition to the continuous concurrent composition framework and allows reconstructing task policy from a set of primitives. The proposed method was extended to composition at the action component level. We demonstrate in the Pointmass environment that our multi-task agents can reconstruct the task policy from a set of primitives in real time and transfer the skills to solve unseen tasks while the single-task performance is competitive with SAC.
This flexible framework incorporates well with the reward-shaping techniques, such as entropy regularization, curiosity\cite{pmlr-v70-pathak17a}, etc. In addition, the task-agnostic property should benefit the autotelic framework \cite{colas2022autotelic} where agents can set goals and curriculum for themselves \cite{narvekar2020curriculum}. 

However, the primary concern at this stage is whether the proposed approach can scale to higher dimensional problems. Additionally, two important topics are left as future works. First, look for the corresponding value composition for DAC. A good starting point might be thinking of the MSF composition with weights evaluated by GPE. 
Second, the optimality of each composition method. One might start with bounding the loss incurred by the policy and value composition. 





\section*{Acknowledgement}
This research is supported by KWF Kankerbestrijding and the Netherlands Organisation for Scientific Research (NWO)  Domain AES, as part of their joint strategic research programme: Technology for Oncology IL. The collaboration project is co-funded by the PPP allowance made available by Health Holland, Top Sector Life Sciences \& Health, to stimulate public-private partnerships. The authors acknowledge Nikos Sourlos for his comments.



\ifCLASSOPTIONcaptionsoff
  \newpage
\fi



\bibliographystyle{IEEEtran}
\bibliography{IEEEabrv}



\vfill
% \section{To do list}
% \subsection{Quick things}

% \subsection{Absolutely necessary}

% %   \item add pathology image to figure perturbation of MRI
% \subsubsection{Paper}
% \begin{itemize}
%     % \item paraphrise + shorten FL and DP sections
%     % \item baad yebar ba dide Chaoning bekhun
% \end{itemize}
% \subsubsection{Coding}
% \begin{itemize}

% \end{itemize}
% \subsection{Maybe later}
% \subsubsection{coding}
% \maybelater{CWT but with random ordering}
% \maybelater{One shot FL?}
% \maybelater{Vazne kamtari dadan be latest client in CWT ?}
% \maybelater{Testing with other deep networks?}
% \maybelater{At least you can increase depth of network}
% \maybelater{Impact of data heterogeneity? just split the data heterogeneously and see how do they work?}

% \maybelater{Effect of Differential privacy?}
% \maybelater{Effect of pretrained=False?}
% \maybelater{HasSpike model?}
% \maybelater{Maybe small variations to generate new models}
% \maybelater{Maybe doing gradient masked FedAvG?}
% \maybelater{Maybe evaluating different number of local epochs like https://github.com/siddarth-c/FedGMA}
% \maybelater{Adding all these ideas to Obsidian/presentations}
% \maybelater{Going for another paper if an idea is worth pursuing}
% \maybelater{Another paper: CWT/SWT with DP}
% \maybelater{Local GD like Ahmed khaled}

% \subsubsection{Paper}
% \begin{itemize}
%     \maybelater{Adding mathematical expression of each FL model (good to improve math level of paper)}
%     \maybelater{maybe adding detailed graph for each participant as an Appendix?}
%     \maybelater{Adding algorithmic view of each FL alg?}
  
% \end{itemize}





% that's all folks
\end{document}


