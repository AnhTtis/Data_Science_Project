\section{Background and Related works}

 % $\Delta W_n^t$
 
Federated learning has been used for various imaging modalities such as MRI\cite{sheller2020federated} \cite{silva2019federated}, X-ray \cite{balachandar2020accounting} retinal imaging, \cite{balachandar2020accounting}  and for tasks such as brain tumor segmentation \cite{bakas2017advancing} \cite{lee2018privacy} diagnosis \cite{pan2019improving} and treatment 
 selection \cite{lee2018privacy}. FL has shown great promise in developing models to support doctors in making treatment decisions for COVID-19 patients; it was investigated and reported that FL had a clear impact on patient care in a large-scale study on COVID-19 patients across 20 centers on five continents\cite{flores2021federated}. They used chest X-Ray imaging data in addition to clinical data to determine hospital triage for level of care and oxygen requirement in COVID-19 patients. They demonstrated that FL improved model performance for clients with limited datasets, compared to when they were trained on their local data. Another finding was that medical centers with smaller datasets had some classes with only a few patients resulting in underrepresented categories. These clients saw a significant improvement in prediction for those patient categories, which is especially important because, in some clients, less than 5\% of the COVID-19 patients were categorized as having severe symptoms, while more than 95\% had moderate symptoms. However, their care is more critical and requires more attention. 
 
 
\begin{figure*}[t!]
 \centering
 \includegraphics[width=1\textwidth]{photo/FLmodels.png}
 \caption{Schematic view of FL models and algorithms, (a) Federated averaging, clients train on local batch of data (b) FedSGD, a subsample of clients are selected, and each performs single step SGD and sends the model updates to the server (c) Cyclic Weight transfer (CWT), clients train locally, and pass the model to the next clients, and the cycle repeats (d) Single weight transfer (SWT) model passes each client only once. (e) Stochastic weight transfer (STWT), the model is passed sequentially through clients, and participating clients in each round are sampled randomly.}
 \label{fig:samplesCTs}
\end{figure*}
% \subsection{Covid-19 detection}
% \maybelater{LITERATURE RO FAQAT AZ YE NAFAR COPY KON TAHESH INE KE MIGIRANET VA BE IN RAZEE HASTI BA FARZE INKE MIGIRANET BORO}

Several recent studies have been done to classify scan images of COVID-19 infeced patients, and healthy subjects, and to locate the lesion areas. The primary focus of AI tooling in the management of COVID-19 patients is interpreting radiology images, mainly chest CT, which has been widely applied for detecting lung changes to optimize patient management, and guide treatment decisions\cite{yan2020interpretable}\cite{hu2020challenges}\cite{burian2020intensive}. Other studies have investigated 3D classification networks,  \cite{wang2020weakly} or Covid-19 detection with limited training samples. 
Most of the above studies achieved good accuracy and assumed a centralized environment where one data center has access to all the data.Three publications exist that successfully applied distributed learning for COVID-19 detection\cite{zhang2021dynamic}\cite{kumar2021blockchain}\cite{ho2022fedsgdcovid}.

The global aggregation models used in the above studies were limited to model averaging in federated\cite{ho2022fedsgdcovid}, \cite{zhang2021dynamic}, or blockchain setting \cite{kumar2021blockchain}. Other studies showed limitations of the existing algorithms like a large communication overhead \cite{remedios2020federated}, and problems with convergence or catastrophic forgetting after increasing the number of participating hospitals
\cite{sheller2020federated} \cite{chang2018distributed}.
To our knowledge, no study has been performed that compared multiple FL algorithms under standard conditions to evaluate their applicability. Therefore, comparing multiple FL algorithms under standard conditions could be informative in evaluating their applicability in practice.
% In this paper, we perform a comparison of the above algorithms under standard settings through an experiment with different numbers of clients, and under various settings is still missing. 
To evaluate the existing methods from multiple perspectives, we have implemented the most popular models and compared them in terms of performance, communication overhead, and computation burden.

% \maybelater{Works that implemented/compared FL in MIA?}
% \maybelater{Works that did DL for COVID?}
% % \subsection{FL}
% \maybelater{works that compared FL algorithms in general? write if not found for MIA}

% A three dimensional deep neural network was implemented by Wang, by integrating activated regions with unattended connected components in the classification network, attaining 90,1\% accuracy and 95,5 percent ROC. This 3D classification model does, however, have drawbacks such as a long training period. Han \cite{han2020accurate} has proposed a data-instance prediction mechanism based pooling process that achieves 97.9\% accuracy and 99\% AUC in the COVID-19, CP and normal classification tasks, although there are restricted test set size issues, For example, 3D-based prediction.  Wang et al. \cite{wang2020deep} have developed a COVID-19 technique for deep learning on X-rays and model integration, with strong results, 96.1 percent in predictive tri-classification tasks,   although only 140 pictures from COVID-19 exist.