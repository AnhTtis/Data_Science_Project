%
% IEEE Transactions on Microwave Theory and Techniques example
% Tibault Reveyrand - http://www.microwave.fr
%
% http://www.microwave.fr/LaTeX.html
% ---------------------------------------



% ================================================
% Please HIGHLIGHT the new inputs such like this :
% Text :
%  \hl{comment}
% Aligned Eq. 
% \begin{shaded}
% \end{shaded}
% ================================================



\documentclass[journal]{IEEEtran}
\usepackage{emoji}
%\usepackage[retainorgcmds]{IEEEtrantools}
%\usepackage{bibentry}  
\usepackage{xcolor,soul,framed} %,caption
\usepackage{cite}
\usepackage{multirow}
\usepackage{fontawesome}
\colorlet{shadecolor}{yellow}
% \usepackage{color,soul}
\usepackage[pdftex]{graphicx}
\graphicspath{{../pdf/}{../jpeg/}}
\DeclareGraphicsExtensions{.pdf,.jpeg,.png}
\usepackage{caption}
\usepackage{subcaption}
\usepackage[cmex10]{amsmath}
%Mathabx do not work on ScribTex => Removed
%\usepackage{mathabx}

% \usepackage{algorithm}
% \usepackage{algpseudocode}


\usepackage[linesnumbered,ruled,vlined]{algorithm2e}



% % \usepackage
% \usepackage[ruled,vlined]{algorithm2e}
% \DeclareMathOperator{\ind}{ind}
% \DeclareMathOperator{\val}{val}
% \DeclareMathOperator{\ptr}{ptr}
% \DeclareMathOperator{\row}{row}
% \usepackage{algpseudocode}
\hyphenation{op-tical net-works semi-conduc-tor}

%\bstctlcite{IEEE:BSTcontrol}


%=== TITLE & AUTHORS ====================================================================
\begin{document}
\bstctlcite{IEEEexample:BSTcontrol}
    % \title{Adversarial attacks on federarted learning for medical image analysis}
    \title{Adversarial attacks on federated learning networks for medical image analysis}
  \author{Erfan~Darzi,~\IEEEmembership{Student Member,~IEEE,}
      P.M.A van Ooijen,~\IEEEmembership{Senior Member,~IEEE,}
    %   and~Zoya~Popovi\'c,~\IEEEmembership{Fellow,~IEEE}% <-this % stops a space

  \thanks{ \hl{This paper is being prepared with IEEE standards} This work was funded in part by NWO under project AMICUS}
  \thanks{Erfan Darzidehkalani is with UMCG (e-mail: e.darzidehkalani@umcg.nl).}% <-this % stops a space
  }

% The paper headers
% \markboth{IEEE TRANSACTIONS ON MICROWAVE THEORY AND TECHNIQUES, VOL.~60, NO.~12, DECEMBER~2012
% }{Roberg \MakeLowercase{\textit{et al.}}: Cross modality image transformation using cyclic residual
% architecture and evolutionary algorithm}


% ====================================================================
\maketitle



% === ABSTRACT ====================================================================
% =================================================================================
\begin{abstract}
%\boldmath



\end{abstract}


% === KEYWORDS ====================================================================
% =================================================================================
\begin{IEEEkeywords}
\hl{}
\end{IEEEkeywords}






% For peer review papers, you can put extra information on the cover
% page as needed:
% \ifCLASSOPTIONpeerreview
% \begin{center} \bfseries EDICS Category: 3-BBND \end{center}
% \fi
%
% For peerreview papers, this IEEEtran command inserts a page break and
% creates the second title. It will be ignored for other modes.
\IEEEpeerreviewmaketitle


% ====================================================================
% ====================================================================
% ====================================================================










% \tableofcontents
% === I. INTRODUCTION =============================================================
% =================================================================================
\section{Introduction}

\IEEEPARstart{F}



\end{document}
