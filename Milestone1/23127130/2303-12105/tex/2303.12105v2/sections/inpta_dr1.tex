 \section{Brief description of the I\lowercase{n}PTA DR1}
\label{sec:inpta-dr1}



The InPTA DR1 \cite{TarafdarNobleson+2022} consists of observations of 14 MSPs conducted using the uGMRT \cite[][]{GuptaAjithkumar+2017} as part of the InPTA experiment from 2018 to 2021 typically with a bi-weekly cadence. 
These observations were carried out during observing cycles 34$-$35 and 37$-$40 of the uGMRT, where the 30 uGMRT antennae were divided into multiple phased subarrays, simultaneously observing the same source in multiple bands in total intensity mode \cite{JoshiGopu+2022}.
The channelized time series data generated by the uGMRT are recorded using the GMRT Wideband Backend~\cite{ReddyKudale+2017} in a binary raw data format, and RFI-mitigated and partially folded into \psrfits{} archives using the \pinta{} pipeline \cite[][]{SusobhananMaan+2021}.
The narrowband ToAs were measured using the Global Positioning System (GPS) and a local topocentric frequency standard was provided by the hydrogen maser clock at the GMRT.
The ToAs were fitted using \tempotwo{} \cite[][]{HobbsEdwardsManchester2006} to obtain the timing residuals. 
The timing procedure involved epoch-wise DM correction by incorporating the DM time series obtained using \dmcalc{} \cite{KrishnakumarManoharan+2021} from low-frequency simultaneous multi-band uGMRT data. 
Additionally, wideband timing residuals were generated using the wideband likelihood method described in Ref.~\cite[][]{AlamArzoumanian+2020b} and implemented in the \texttt{TEMPO} \cite[][]{NiceDemorest+2015} pulsar timing package.
For our analysis, we use only the narrowband data for all pulsars. More details of the InPTA DR1 can be found in ~\cite{TarafdarNobleson+2022}.