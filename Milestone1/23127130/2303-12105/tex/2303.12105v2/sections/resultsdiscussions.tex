\section{Results and discussion}
\label{sec:discussion}


This section presents the results of custom noise modelling for each pulsar. One thing to note is that as the InPTA DR1 parfiles contain \texttt{DMXs} that absorb the DM variations, we remove them from the parfiles and fit for \texttt{DM1} and \texttt{DM2} using \texttt{TEMPO2}. These final parfiles are used for our noise analysis. We also remove \texttt{T2EFACs} from our parfiles before using them for SPNA work. In the initial step of EQUAD model selection, we find that pulsars J0613$-$1224, J1012+5307, J1744$-$1134 and J2124$-$3358 do not prefer EQUAD, while all the remaining ten pulsars strongly prefer it.
Overall, eight pulsars J1012+5307, J1022+1001, J1643$-$1224, J1713+0747, J1744$-$1134, J1909$-$3744, J1939+2134 and J2145$-$0750 show the presence of DMv in InPTA dataset. For three pulsars J1643$-$1224, J1939+2134, and J2145$-$0750, we can see from Table~\ref{model_select} that \textit{WRDS} is preferred over other models, i.e. all the red noises considered in this paper are present in these pulsars. Except for J0751+1807 and J1600$-$3053, all the other pulsars support the red noises in the InPTA data. Interestingly, we see scattering variations for four pulsars in our sample: J1643$-$1224, J1713+0747, J1939+2134, and J2145$-$0750. Six pulsars J0751+1807, J1012+5307, J1022+1001, J1600$-$3053, J1713+0747 and J1744$-$1134 do not show the presence of achromatic red noise. Pulsars J1909$-$3744, J1939+2134 and J2145$-$0750 prefer the highest number of Fourier modes for RN, denoting the presence of RN at high frequencies. Similarly, Sv is present in high frequencies for PSR J2145$-$0750, while only in low frequencies for J1643$-$1224, J1713+0747 and J1939+2134.



\subsection*{\texorpdfstring{J0437$-$4715}{J0437-4715}}
PSR J0437$-$4715 is one of the brightest pulsars observed by InPTA. In the InPTA DR1, we can see achromatic red noise in the ToA residuals and do not find significant DM variations in the DM time series, whereas both these noises are present in PPTA for this pulsar~\cite{GonchorovReardon+2021}. As the Bayes factor for \textit{WRD} over \textit{WR} was inconclusive, we selected \textit{WR} because it has less number of free parameters. As the baseline for this pulsar is around one year for InPTA, it could be one of the reasons for the differences in models.
To understand this, more work is underway to characterize the jitter across frequencies as we have simultaneous multi-band observations.


\subsection*{\texorpdfstring{J0613$-$0200}{J0613-0200}}
For this pulsar, again, we see slight variations in the ToA residuals, while no such variations are seen in the corresponding DM time series, where the $\Delta\text{DMs}$ have precision up to the fourth decimal place. \textit{WR}, \textit{WRD}, and \textit{WRDS} have nearly the same value of Bayesian Evidence, and \textit{WRD}, \textit{WRDS} have inconclusive Bayes factors over \textit{WR} as seen in Table \ref{model_select}. Hence, we select \textit{WR} based on it having fewer number of free parameters. We also find that the noise model in PPTA~\cite{GonchorovReardon+2021} also contains only achromatic red noise similar to our results, while EPTA~\cite{ChalumeauBabak+2022} contains additional DM variations.


\subsection*{J0751+1807}
For this pulsar, Bayesian analysis suggests no significant Bayes factors of models with any kind of red noise over the white noise only model, i.e. \textit{W} and hence, \textit{W} is chosen based on the simplicity of the model. The DM time series and residuals from the InPTA DR1 also show no significant variation over time, supporting the selected noise model.

\subsection*{J1012+5307}
In this pulsar, the DM variations are evident, especially in the later part of the data, while no such variations are observed in ToA residuals. The Bayes factor for \textit{WRD}, \textit{WDS}, and \textit{WRDS} with respect to \textit{WD} is insignificant as seen in Table \ref{model_select}. Hence we select \textit{WD} based on the model's simplicity. Also, the data before cycle 37 did not have very high precision, causing large error bars on the DM, but the subsequent data has high precision DMs, where a discernible trend is evident.
Hence, one expects the DM variations to be the dominant noise process, and Bayesian analysis favors the same. The prefered noise model for this pulsar in EPTA~\cite{ChalumeauBabak+2022} contains both achromatic red noise and DM variations. The lack of red noise in InPTA DR1 could be due to the short data span and large data gap for this pulsar.



\begin{figure*}[t!]
\includegraphics[keepaspectratio=true,scale=0.40]{figures/J1600-3053_crn.pdf}
  \includegraphics[keepaspectratio=true,scale=0.40]{figures/J1939+2134_crn_arn_dmn_scn.pdf}
   \caption{(Left Panel): J1600$-$3053 posterior distributions of the chromatic index $\chi_{FCN}$. (Right Panel): J1939+2134 posterior distributions of the chromatic index $\chi_{FCN}$.}
  \label{fig:16001939crn}
\end{figure*}  



\subsection*{J1022+1001}
For this pulsar, we again observe that the DM variations are conspicuous from cycle 37 data onward. Hence, it seems to be the dominant process, with no discernible variations in the ToA residuals. 
The Bayesian analysis gives comparable evidences for \textit{WDS} to \textit{WD} with Bayes factors close to one (see Table \ref{model_select}), which implies there is no preferred model among these. Therefore, we chose \textit{WD} as it had the smallest number of free parameters. 
Our noise model also agrees with the PPTA noise model for this pulsar~\cite{GonchorovReardon+2021}.






\subsection*{\texorpdfstring{J1600$-$3053}{J1600-3053}}
This pulsar does not show signatures of any type of red noises based on Bayesian evidence, where all the models show comparable Bayesian evidence values. This is reaffirmed by the analysis based on the free chromatic index, where the chromatic index is a one-sided distribution with large error bars, as seen in Fig.~\ref{fig:16001939crn}. In contrast, the amplitude and spectral index are unconstrained, which implies that it encapsulates only the white noise. The DM series and ToA residuals show no significant variations, supporting the Bayesian analysis result of \textit{W}. PPTA~\cite{GonchorovReardon+2021} and EPTA~\cite{ChalumeauBabak+2022} report achromatic red noise and DM variations (along with scattering variations in the case of PPTA), and we also see scattering variations in DM time series in NANOGrav 12.5-year dataset \cite{AlamArzoumanian+2020b}. {InPTA dataset for this pulsar begins where the NANOGrav 12.5-year dataset terminates \cite{TarafdarNobleson+2022,AlamArzoumanian+2020b}, and comparing the DMs between their last epoch (52.333508 $pc/cm^3$) and our first epoch (with \texttt{DMX}) (52.3326 $pc/cm^3$), we observe the difference of $9\times10^{-4}~pc/cm^3$, which is consistent within errors. Furthermore, we do not see any significant scatter-broadening in the low-frequency observations of this pulsar. In addition, we also find that in the NANOGrav 12.5-yr dataset, the DMs tend to stabilize towards the end. A complete understanding of this inconsistency shall be explored, but we suspect that the DMs are stable across the 3.5 years InPTA dataset and may have been variable before, as seen in NANOGrav dataset along with EPTA \cite{ChalumeauBabak+2022}.

\subsection*{\texorpdfstring{J1643$-$1224}{J1643-1224}}
Bayesian analysis for this pulsar strongly prefers \textit{WRDS}, which has achromatic red noise, DM, and scattering variations. Low-frequency observations of this pulsar confirm significant scatter-broadening, which varies from epoch to epoch. In addition to that, DM variations are also seen. From the DR1 plots, it is difficult to adjudicate between scatter broadening and DM variations as both lead to similar exponential scatter, but DM variations seem very evident.



 \subsection*{J1713+0747}
PSR J1713+0747 strongly supports the \textit{WDS} based on the estimated Bayes factor, i.e. only DM and scattering variations. This is one of the best-timed PTA pulsars, and there seems to be very less achromatic red noise. There are DM and scattering variations, but their amplitudes are very small. Overall, the pulsar seems to be a good timer, provided these noise sources are included. One thing to note is that the data does not include the DM event \cite{lam+2018} and profile change event \cite{singha+2021}. There is very little scattering variation, as seen by other PTAs \cite{ChalumeauBabak+2022}, while our model contains it. This could be because the InPTA dataset contains low-frequency data, which is absent in the EPTA dataset, hence could be more sensitive to detect this weak scattering, which may have been missed in EPTA.

\subsection*{\texorpdfstring{J1744$-$1134}{J1744-1134}}
The data span for this pulsar is only about six months; hence it is difficult to make definitive conclusions about the RN here. On the other hand, the DM variations are evident in the DM time series. The Bayesian analysis supports this and provides \textit{WD} as the most selected model. The selected model for this pulsar in EPTA~\cite{ChalumeauBabak+2022} has RN along with DMv, which is absent in InPTA data due to a small time span, while 
 \citet{GonchorovReardon+2021} obtain a similar model as ours.

\subsection*{J1857+0943}
In this pulsar, the achromatic red noise is quite visible in the ToA residuals and seems to be the dominant source, while there are no visible DM variations in the DM time series. Bayesian model selection prefers \textit{WR}, which is expected based on our observations and supports our claim. \citet{GonchorovReardon+2021} model also contains DM noise, which is absent in our modelling and can be due to our relatively short data span.

\begin{figure*}
  \includegraphics[keepaspectratio=true,scale=0.40]{figures/J1909-3744_crn_arn.pdf}
  \includegraphics[keepaspectratio=true,scale=0.40]{figures/J1909-3744_crn_arn_dmn.pdf}
 \caption{(Left Panel): J1909$-$3744 posterior distributions of the chromatic index $\chi_{FCN}$ \bthis{with the noise model containing WN and RN}. (Right Panel): J1909$-$3744 posterior distributions of the chromatic index $\chi_{FCN}$ \bthis{with the noise model containing WN, RN and DMv}.}
  \label{fig:1909crn}
\end{figure*}

\subsection*{\texorpdfstring{J1909$-$3744}{J1909-3744}}
PSR J1909-3744 exhibits the best stability among the PTA pulsars. Nevertheless, there are significant variations in the DM towards this pulsar in the InPTA DR1, as seen from the DM time series. Our analysis indicates strong evidence for \textit{WRD} and \textit{WRDS}, both of which incorporate achromatic red noise and DM noise. The latter model also includes a GP corresponding to the variations in scatter-broadening and is marginally favored over the former by $\ln(\text{Bayes Factor})$ of 4.4. However, this pulsar has no pulse broadening, even at 300 MHz. To investigate this further, we examined the parameters estimated for the different noise processes in \textit{WR}, \textit{WRD}, and \textit{WRDS}. The amplitude \bthis{(at frequency of $1 yr^{-1}$)}, and high-frequency cut-off for the achromatic noise are consistent for all these models ($\log A_{RN} \sim -12.5$ and $\gamma_{RN} \sim$ 0.68), and so is the case for DM noise ($\log A_{DMv} \sim -13.5$ and $\gamma_{DMv} \sim$ 2) with DM noise process an order of magnitude weaker than the achromatic red-noise with a much larger frequency content than the latter. In contrast, the noise process amplitude (\bthis{at frequency of $1 yr^{-1}$}) representing the scatter broadening is two orders of magnitude smaller than either the achromatic red-noise process or the DM noise process with $\gamma$ essentially consistent with zero (in other words, consistent with no variations). We further investigated this by using a GP with a free chromatic index. A model involving just the achromatic red noise process with a chromatic process having a free index yields hyper-parameters similar to those of chromatic red noise and the DM noise with a chromatic index of $2.4^{+0.19}_{-0.16}$, close to the expected chromatic index of 2 for a DM noise process as seen in Fig.~\ref{fig:1909crn}. Furthermore, parameter estimation with a model consisting of achromatic red noise, DM noise, and chromatic process with a free index again yields a two-order weaker free index process, with significant error bars on the index (see Fig.~\ref{fig:1909crn} again). Thus, there is no strong evidence for a scattering process consistent with the absence of any observed scattering at 300 MHz despite marginally higher evidence for a scattering process. Hence, we have chosen \textit{WRD} for this pulsar which is the simpler model.

\subsection*{J1939+2134}
PSR J1939+2134 is the oldest known MSP. For a long time, it was assumed to be the most stable rotator \citep{krt1992}. Still, high-precision timing campaigns have shown that not only does its rotation rate wander, but it also exhibits significant DM and scatter-broadening variations. Observations near 300 MHz clearly show that the pulse profile is scatter broadened at these frequencies~\citep{joshirama2006}. Thus, we needed to model the  DM and scattering variations apart from the achromatic red noise and white noise for this bright but relatively distant PTA pulsar (\textit{WRDS}). The posterior distributions of the relevant GP are shown in Appendix~\ref{appendix_b}, while those of the full model are presented in Appendix~\ref{appendix_c}. As mentioned before, the GP representing the scatter-broadening variations assumes Kolmogorov turbulence with a chromatic index of 4. Previous studies have shown that this is not true along all lines of sight, and the chromatic index can go as low as -0.7, i.e. shallower than Kolmogorov~\cite{levin2016,kk2019,turner2021}. To investigate this, we used a FCN model apart from the usual white noise, achromatic red noise, DM variations, and scattering variations. While this incorporates an additional free parameter in terms of the chromatic index, we find that the corresponding FCN parameters to be well constrained, as can be seen in Fig.~\ref{fig:16001939crn} with $\chi_{FCN}$ $\sim$ 2.86, and $\log_{10}A_{FCN}$ $\sim$ -12.58, which is much greater than conventional scattering variation model with chromatic index 4, and comparable with other red noise amplitudes (see Table \ref{parameter_values}). This indicates that for this pulsar, the chromatic index does not agree with Kolmogorov turbulence with chromatic index 4. We plan to investigate this further from direct measurements.

\subsection*{\texorpdfstring{J2124$-$3358}{J2124-3358}}
The selected model for this pulsar is \textit{WR}, as \textit{WRD} and \textit{WR} have comparable Bayesian evidence, and hence, inconclusive Bayes factor for \textit{WRD} over \textit{WR}. Therefore, \textit{WR} is chosen based on fewer number of  free parameters. Although, the ToA residuals have large error bars in InPTA DR1, slight achromatic red noise is visible. At the same time, the high precision $\Delta\text{DM}$ estimates vary only in the fourth decimal place, and hence do not showing significant DM variations, which supports the selected model.

\subsection*{\texorpdfstring{J2145$-$0750}{J2145-0750}}
For PSR J2145$-$0750, we obtain \textit{WRDS} as the strongly preferred using Bayesian model selection, suggesting the presence of achromatic red noise, as well as DM and scattering variations. In the DR1 plots, we see in ToA residuals that there is a small amplitude achromatic red noise variation over a large time scale, and we see in the DM series there are short time scale small amplitude DM variations. Scattering noise is not evident as it is difficult to see small scattering in DM series due to smaller amplitude than DM noise in this case.

