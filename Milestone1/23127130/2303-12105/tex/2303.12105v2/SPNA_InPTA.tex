\documentclass[
reprint,
aps,
prd,
twocolumn,
superscriptaddress,
nofootinbib,
preprintnumbers,
floatfix
]{revtex4-1}
 


\newcommand{\discreteVectorField}{\mathcal{V}}
\newcommand{\numberProcesses}{n_p}
\newcommand{\ghostLayer}{\mathcal{G}}
\newcommand{\domain}{\mathcal{M}}
\newcommand{\range}{\mathbb{R}}
\newcommand{\sublevelset}[1]{#1^{-1}_{-\infty}}
\newcommand{\superlevelset}[1]{#1^{-1}_{+\infty}}
\newcommand{\Star}{St}
\newcommand{\Link}{Lk}
\newcommand{\simplex}{\sigma}
\newcommand{\face}{\tau}
\newcommand{\lowerlink}{\Link^{-}}
\newcommand{\upperlink}{\Link^{+}}
\newcommand{\Index}{\mathcal{I}}
\newcommand{\offset}{o}
\newcommand{\Natural}{\mathbb{N}}
\newcommand{\criticalSet}{\mathcal{C}}
\newcommand{\diagram}{\mathcal{D}}
\newcommand{\wasserstein}[1]{W^{\diagram}_#1}
\newcommand{\projection}{\Delta}
\newcommand{\hierarchy}{\mathcal{H}}
\newcommand{\decimation}{D}
\newcommand{\xDimD}{L_x^\decimation}
\newcommand{\yDimD}{L_y^\decimation}
\newcommand{\zDimD}{L_z^\decimation}
\newcommand{\xDim}{L_x}
\newcommand{\yDim}{L_y}
\newcommand{\zDim}{L_z}
\newcommand{\Grid}{\mathcal{G}}
\newcommand{\GridD}{\mathcal{G}^\decimation}
\newcommand{\x}{\phantom{x}}
\newcommand{\Mod}{\;\mathrm{mod}\;}
\newcommand{\NN}{\mathbb{N}}
\newcommand{\forwardIntegralLine}{\mathcal{L}^+}
\newcommand{\backwardIntegralLine}{\mathcal{L}^-}
\newcommand{\triangulationOp}{\phi}
\newcommand{\decimationOp}{\Pi}
\newcommand{\isovalue}{w}
\newcommand{\persistence}{\mathcal{P}}
\newcommand{\pointMetric}{d}
\newcommand{\diagramSet}{\mathcal{S}_\mathcal{D}}
\newcommand{\diagramSpace}{\mathbb{D}}
\newcommand{\jointree}{\mathcal{T}^-}
\newcommand{\splittree}{\mathcal{T}^+}
\newcommand{\mergetree}{\mathcal{T}}
\newcommand{\mergetreeSet}{\mathcal{S}_\mathcal{T}}
\newcommand{\branchset}{\mathcal{S}_\mathcal{B}}
\newcommand{\branchspace}{\mathbb{B}}
\newcommand{\mergetreeSpace}{\mathbb{T}}
\newcommand{\editdistance}{D_E}
\newcommand{\wassersteinTree}{W^{\mergetree}_2}
\newcommand{\distanceSequence}{d_S}
\newcommand{\branchtree}{\mathcal{B}}
\newcommand{\branchtreeSet}{\mathcal{S}_\mathcal{B}}
\newcommand{\branchtreeSpace}{\mathbb{B}}
\newcommand{\forest}{\mathcal{F}}
\newcommand{\sequenceSpace}{\mathbb{S}}
\newcommand{\forestMatrix}{\mathbb{F}}
\newcommand{\treeMatrix}{\mathbb{T}}
\newcommand{\normalizedLocation}{\mathcal{N}}
\newcommand{\normalizedWasserstein}{W^{\normalizedLocation}_2}
\newcommand{\geodesictree}{\mathcal{G}}
\newcommand{\dummyVector}{\mathcal{V}}
\newcommand{\geodesictreeVec}{g}
\newcommand{\geodesicAxis}{\mathcal{A}}
\newcommand{\directionVector}{\mathcal{V}}
\newcommand{\geodesicdiagram}{\mathcal{G}^{\diagram}}
\newcommand{\reconstructionError}{E_{L_2}}
\newcommand{\pcaBasis}{B_{\mathbb{R}^d}}
\newcommand{\mtPgaBasis}{B_{\branchtreeSpace}}
\newcommand{\mtPgaError}{E_{\wassersteinTree}}
\newcommand{\frechetEnergy}{E_F}
\newcommand{\geodesicExtremity}{\mathcal{E}}
\newcommand{\vectorNotation}[1]{\protect\vv{#1}}
\newcommand{\axisNotation}[1]{\protect\overleftrightarrow{#1}}
% \newcommand{\axisNotation}[1]{\vectorNotation{#1}}
\newcommand{\individualEnergy}{E}
\newcommand{\ensembleSize}{N}
\newcommand{\numberBranchinBarycenter}{N_1}
\newcommand{\numberGeodesicSamples}{N_2}
\newcommand{\planarGridX}{N_x}
\newcommand{\planarGridY}{N_y}
\newcommand{\regularGrid}{G}
\newcommand{\distanceMatrix}{\mathbb{D}}
\newcommand{\maxDimensions}{{d_{max}}}
\newcommand{\projectionOperator}{\mathcal{P}}
\newcommand{\reconstructed}[1]{\widehat{#1}}
\newcommand{\gt}{>}
\newcommand{\lt}{<}
% \newcommand{\normalizedTree}{N}
\newcommand{\branch}{b}



%\newcommand{\revision}[1]{\textcolor{blue}{#1}}
% \newcommand{\revision}[1]{\textcolor{black}{#1}}
% \newcommand{\minorRevision}[1]{\textcolor{blue}{#1}}

\renewcommand{\figureautorefname}{Fig.}
\renewcommand{\sectionautorefname}{Sec.}
\renewcommand{\subsectionautorefname}{Sec.}
\renewcommand{\subsubsectionautorefname}{Sec.}
\renewcommand{\equationautorefname}{Eq.}
\renewcommand{\tableautorefname}{Tab.}
\newcommand{\algorithmautorefname}{Alg.}
\newcommand{\lineautorefname}{Alg.}

\newcommand{\eqSpace}{-1.75ex}

\newcommand{\mycaption}[1]{
%\vspace{-3.5ex}
\caption{#1}
%\vspace{-3ex}
}
%

\usepackage[utf8]{inputenc}
\usepackage[T1]{fontenc}
\usepackage{graphicx}
\graphicspath{ {figures/} }
\usepackage[export]{adjustbox}
\usepackage{dcolumn}
\usepackage{bm}
\usepackage{hyperref}
\usepackage{amsmath}	
\usepackage{amssymb}	
\usepackage{slashed}
\usepackage[normalem]{ulem}
\usepackage[dvipsnames]{xcolor}
\usepackage{xspace}
\usepackage{multirow}
\usepackage{booktabs}
\usepackage{aas_macros}
\usepackage{multirow}
\usepackage{caption}
\usepackage{array}
\captionsetup{justification=raggedright,singlelinecheck=false}
\usepackage{subfigure}
\usepackage{rotating}
\usepackage{adjustbox}
%\newcommand{\bthis}[1]{\textcolor{black}{\textbf{#1}}}
\newcommand{\bthis}[1]{\textcolor{black}{#1}}
\hypersetup{colorlinks,linkcolor={red},citecolor={blue},urlcolor={blue}}
\newcommand{\dmunit}{pc~cm$^{-3}$}
\newcommand{\pinta}{\texttt{pinta}}
\newcommand{\gptool}{\texttt{gptool}}
\newcommand{\rficlean}{\texttt{RFIClean}}
\newcommand{\filterbank}{\texttt{filterbank}}
\newcommand{\dspsr}{\texttt{dspsr}} 
\newcommand{\timer}{\texttt{Timer}}
\newcommand{\psrchive}{\texttt{PSRCHIVE}}
\newcommand{\psrfits}{\texttt{PSRFITS}}
\newcommand{\tempo}{\texttt{TEMPO}}
\newcommand{\tempotwo}{\texttt{TEMPO2}}
\newcommand{\dmcalc}{\texttt{DMcalc}}
\newcommand{\pulseportr}{\texttt{PulsePortraiture}}
\newcommand{\enterprise}{\texttt{ENTERPRISE}}
\newcommand{\dynesty}{\texttt{DYNESTY}}
\newcommand{\ptmcmc}{\texttt{PTMCMCSAMPLER}}
\allowdisplaybreaks

\begin{document}

\preprint{APS/123-QED}

\title{Noise analysis of the Indian Pulsar Timing Array data release I}


\author{Aman Srivastava}
 \email{amnsrv1@gmail.com}
 \affiliation{Department of Physics, IIT Hyderabad, Kandi, Telangana 502284, India}


\author{Shantanu Desai}
\affiliation{Department of Physics, IIT Hyderabad, Kandi, Telangana 502284, India}

\author{Neel Kolhe}
\affiliation{Department of Physics, St. Xavier’s College (Autonomous), Mumbai 400001, Maharashtra, India}

\author{Mayuresh Surnis}
\affiliation{Department of Physics, IISER Bhopal, Bhopal Bypass Road, Bhauri, Bhopal 462066, Madhya Pradesh, India}

\author{Bhal Chandra Joshi}
\affiliation{National Centre for Radio Astrophysics, Pune University Campus, Pune 411007, India}

\author{Abhimanyu Susobhanan}
\affiliation{ Center for Gravitation, Cosmology, and Astrophysics, University of Wisconsin-Milwaukee, Milwaukee, WI 53211, USA}

\author{Aur\'elien Chalumeau}
\affiliation{Dipartimento di Fisica “G. Occhialini", Universit\'a degli Studi di Milano-Bicocca, Piazza della Scienza 3, 20126 Milano, Italy}

\author{Shinnosuke Hisano}
\affiliation{Kumamoto University, Graduate School of Science and Technology, Kumamoto, 860-8555, Japan}


\author{Nobleson K.}
\affiliation{Department of Physics, BITS Pilani Hyderabad Campus, Hyderabad 500078, Telangana, India}


\author{Swetha Arumugam}
\affiliation{Department of Electrical Engineering, IIT Hyderabad, Kandi, Telangana 502284, India}

\author{Divyansh Kharbanda}
\affiliation{Department of Physics, IIT Hyderabad, Kandi, Telangana 502284, India}

\author{Jaikhomba Singha}
\affiliation{Department of Physics, Indian Institute of Technology Roorkee, Roorkee-247667, India}

\author{Pratik Tarafdar}
\affiliation{The Institute of Mathematical Sciences, C. I. T. Campus, Taramani, Chennai 600113, India}


\author{P Arumugam}
\affiliation{Department of Physics, Indian Institute of Technology Roorkee, Roorkee-247667, India}

\author{Manjari Bagchi}
\affiliation{The Institute of Mathematical Sciences, C. I. T. Campus, Taramani, Chennai 600113, India}
\affiliation{Homi Bhabha National Institute, Training School Complex, Anushakti Nagar, Mumbai 400094, India}

\author{Adarsh Bathula}
\affiliation{Department of Physical Sciences, Indian Institute of Science Education and Research, Mohali, Punjab, India -140306.}

\author{Subhajit Dandapat}
\affiliation{Department of Astronomy and Astrophysics, Tata Institute of Fundamental Research, Homi Bhabha Road, Navy Nagar, Colaba, Mumbai 400005, India}

\author{Lankeswar Dey}
\affiliation{Department of Physics and Astronomy, West Virginia University, P.O. Box 6315, Morgantown, WV 26505, USA}
\affiliation{Center for Gravitational Waves and Cosmology, West Virginia University, Chestnut Ridge Research Building, Morgantown, WV 26505, USA}

\author{Churchil Dwivedi}
\affiliation{Department of Earth and Space Sciences, Indian Institute of Space Science and Technology, Valiamala, Thiruvananthapuram, Kerala, India (695547)}

\author{Raghav Girgaonkar}
\affiliation{Department of Physics and Astronomy, University of Texas Rio Grande Valley, One West University Blvd., Brownsville, Texas 78520, USA}


\author{A. Gopakumar}
\affiliation{Department of Astronomy and Astrophysics, Tata Institute of Fundamental Research, Homi Bhabha Road, Navy Nagar, Colaba, Mumbai 400005, India}


\author{Yashwant Gupta}
\affiliation{National Centre for Radio Astrophysics, Pune University Campus, Pune 411007, India}

\author{Tomonosuke Kikunaga}
\affiliation{Kumamoto University, Graduate School of Science and Technology, Kumamoto, 860-8555, Japan}

\author{M. A. Krishnakumar}
\affiliation{Max-Planck-Institut f\"ur Radioastronomie, Auf dem H\"ugel 69, 53121 Bonn, Germany}
\affiliation{Fakult\"at f\"ur Physik, Universit\"at Bielefeld, Postfach 100131, 33501 Bielefeld, Germany}

\author{Kuo Liu}
\affiliation{Max-Planck-Institut f\"{u}r Radioastronomie, Auf dem H\"{u}gel 69, 53121, Bonn, Germany}

\author{Yogesh Maan}
\affiliation{National Centre for Radio Astrophysics, Pune University Campus, Pune 411007, India}

\author{P K Manoharan}
\affiliation{Arecibo Observatory, University of Central Florida, PR 00612, USA}



\author{Avinash Kumar Paladi}
\affiliation{Joint Astronomy Programme, Indian Institute of Science, Bengaluru, Karnataka, 560012, India}


\author{Prerna Rana}
\affiliation{Department of Astronomy and Astrophysics, Tata Institute of Fundamental Research, Homi Bhabha Road, Navy Nagar, Colaba, Mumbai 400005, India}

\author{Golam M. Shaifullah}
\affiliation{Dipartimento di Fisica “G. Occhialini", Universit\'a degli Studi di Milano-Bicocca, Piazza della Scienza 3, 20126 Milano, Italy}
\affiliation{INFN, Sezione di Milano-Bicocca, Piazza della Scienza 3, I-20126 Milano, Italy}
\affiliation{INAF, Osservatorio Astronomico di Cagliari, Via della Scienza 5, 09047 Selargius, Italy }

\author{Keitaro Takahashi}
\affiliation{Division of Natural Science, Faculty of Advanced Science and Technology, Kumamoto University,
2-39-1 Kurokami, Kumamoto 860-8555, Japan}
\affiliation{International Research Organization for Advanced Science and Technology, Kumamoto University, 2-39-1 Kurokami, Kumamoto 860-8555, Japan
}



\begin{abstract}
The Indian Pulsar Timing Array (InPTA) collaboration has recently made its first official data release (DR1) for a sample of 14 pulsars using 3.5  years of uGMRT observations.
We present the results of single-pulsar noise analysis for each of these 14 pulsars using the InPTA DR1. 
For this purpose, we consider  white noise, achromatic red noise, dispersion measure (DM) variations, and scattering variations in our analysis. 
We apply  Bayesian model selection to obtain the preferred noise models among these for each pulsar.  For PSR J1600$-$3053, we find no evidence of DM and scattering variations, while for PSR J1909$-$3744, we find no significant scattering variations. Properties vary dramatically among pulsars. For example, we find a strong chromatic noise with chromatic index $\sim$ 2.9 for PSR J1939+2134, indicating the possibility of a scattering index that doesn't agree with that expected for a Kolmogorov scattering medium consistent with similar results for millisecond pulsars in past studies. Despite the relatively short time baseline, the noise models broadly agree with the other PTAs and provide, at the same time, well-constrained DM and scattering variations.

\end{abstract}


\maketitle



\section{Introduction}
\label{sec:introduction}
% \begin{itemize}
%     % Diffusion of FL
%     \item {\st{Diffusion of FL}}
%     % Security threats to FL
%     \item {\st{Security threats to FL with particular focus on model poisoning}}
%     % Limitations of existing countermeasures
%     \item {\st{Current countermeasures (e.g., KRUM) and their limitations}}
%     % Proposed method and its advantages
%     \item {\st{Intuitive description of the proposed method and its difference (i.e., advantages) w.r.t. state of the art}}
%     % Main contributions
%     \item {\st{Summary of the main contributions of this work}}
%     % Paper's structure and organization
%     \item {\st{Paper's structure and organization}}
% \end{itemize}

% Diffusion of FL
Recently, {\em federated learning} (FL) has emerged as the leading paradigm for training distributed, large-scale, and privacy-preserving machine learning (ML) systems~\cite{mcmahan2017googleai,mcmahan2017aistats}. 
The core idea of FL is to allow multiple edge clients to collaboratively train a shared, global model without disclosing their local private training data.
%Specifically, an FL system consists of a central server and many edge clients; 
A typical FL round involves the following steps: {\em(i)} the server randomly picks some clients and sends them the current, global model; {\em(ii)} each selected client locally trains its model with its own private data; then, it sends the resulting local model to the server;\footnote{Whenever we refer to global/local model, we mean global/local model {\em parameters}.} {\em(iii)} the server updates the global model by computing an \emph{aggregation function}, usually the average (FedAvg), on the local models received from clients.
% \begin{enumerate}
%     \item[{\em(i)}] the server sends the current, global model to the clients and appoints some of them for training;
%     \item[{\em(ii)}] each selected client locally trains its copy of the global model with its own private data; then, it sends the resulting local model back to the server;\footnote{Whenever we refer to global/local model, we mean global/local model {\em parameters}.}
%     \item[{\em(iii)}] the server updates the global model by computing an \emph{aggregation function} on the local models received from clients (by default, the average, also referred to as FedAvg~\cite{mcmahan2017aistats}).
% \end{enumerate}
This process goes on until the global model converges. %(e.g., after a certain number of rounds or other similar stopping criteria).
%\\
% The advantages of FL over the traditional, centralized learning paradigm are undoubtedly clear in terms of flexibility/scalability (clients can join/disconnect from the FL network dynamically), network communications (only model weights\footnote{We will use \textit{parameters} and \textit{weights} interchangeably.} are exchanged between clients and server), and privacy (each client's private training data is kept local at the client's end and not uploaded to the server).
\\
% Security threats to FL
%However, the growing adoption of FL also raises security concerns~\cite{costa2022covert}, particularly about its confidentiality, integrity, and availability.
Although its advantages over standard ML, FL also raises security concerns~\cite{costa2022covert}. %, particularly about its confidentiality, integrity, and availability~\cite{costa2022covert}.
% OLD, LONG VERSION
% Indeed, some work deals with privacy leakage that may expose the local data of some clients~\cite{melis2019sp}. 
% A large body of work, instead, investigates attacks that usually aim to detriment the predictive accuracy of the learned global model. For instance, \emph{data poisoning} attacks achieve this goal by letting an adversary pollute the training set of some corrupt FL clients with maliciously crafted examples~\cite{jagielski2018sp}.
% Similarly, in \emph{model poisoning} the attacker attempts to tweak the global model weights~\cite{bhagoji2019pmlr} by directly perturbing the local model's weights of some infected FL clients before these are sent to the central server for aggregation, usually via so-called Byzantine attacks. 
% It turns out that Byzantine model poisoning attacks severely impact standard FedAvg; therefore, more robust aggregation functions must be designed to make FL systems secure.
Here, we focus on \emph{untargeted model poisoning} attacks~\cite{bhagoji2019pmlr}, where an adversary attempts to tweak the global model weights %\footnote{We will use the terms \textit{parameters} and \textit{weights} interchangeably.} 
by directly perturbing the local model's parameters of some infected clients before these are sent to the central server for aggregation.
In doing so, the adversary aims to jeopardize the global model \textit{indiscriminately} at inference time.
Such model poisoning attacks severely impact standard FedAvg; therefore, more robust aggregation functions must be designed to secure FL systems.
\\
% In this paper, we focus on designing a novel robust aggregation scheme at the server's end to contrast the effect of Byzantine model poisoning attacks.
%
% Current countermeasures and their limitations
%Several countermeasures have been proposed in the literature to combat model poisoning attacks on FL systems.
% Some methods use simple statistics more robust than plain average to smooth the impact of malicious updates (e.g., Trimmed Mean and FedMedian~\cite{yin2018icml}). 
% Other defenses implement outlier detection techniques to discard malicious updates from the aggregation performed at the server's end. Those are either based on heuristics (e.g., Krum/Multi-Krum~\cite{blanchard2017nips} and Bulyan~\cite{mhamdi2018pmlr}) or data-driven approaches (e.g., K-means clustering~\cite{shen2016acm} or DnC via spectral analysis~\cite{shejwalkar2021ndss}). 
% Finally, some strategies rely on a centralized ``source of trust'' to spot potential malicious updates (e.g., FLTrust~\cite{cao2020fltrust}).
% Several countermeasures have been proposed in the literature to combat model poisoning attacks on FL systems, i.e., to discard possible malicious local updates from the aggregation performed at the server's end. 
% These techniques range from simple statistics more robust than plain average (e.g., Trimmed Mean and FedMedian~\cite{yin2018icml}) to outlier detection heuristics (e.g., Krum/Multi-Krum~\cite{blanchard2017nips} and Bulyan~\cite{mhamdi2018pmlr}) or data-driven approaches (e.g., spectral analysis via K-means clustering~\cite{shen2016acm} or spectral analysis), or methods based on ``source of trust'' (e.g., FLTrust~\cite{cao2020fltrust}).
% OLD, LONG VERSION
%Several countermeasures have been proposed in the literature to combat Byzantine model poisoning attacks on FL systems.
% Descriptive statistics
% For example, Trimmed Mean and FedMedian aggregate local model updates using more robust statistics than standard average~\cite{yin2018icml}.
%
% % Heuristics for outlier detection
% Many existing Byzantine-resilient strategies implement some outlier detection heuristics to discard the model updates sent by potentially malicious clients from the input of the aggregation function.
% One of the most popular heuristics is Krum~\cite{blanchard2017nips}.
% This strategy tries to mitigate the impact of Byzantine attacks by selecting as a global model the local model with the smallest sum of Euclidean distances to {\em all} the other local models.
% Although powerful, Krum requires the server to know (or, at least, estimate) the number of malicious FL clients upfront, which is generally impossible in a realistic attack scenario. %
% Moreover, Krum may become ineffective for complex, high-dimensional model parameter spaces due to the curse of dimensionality.
% Bulyan~\cite{mhamdi2018pmlr} tries to overcome this issue by combining Krum with a variant of Trimmed Mean.
% % Data-driven outlier detection
% Other strategies use data-driven outlier detection techniques -- e.g., via K-means clustering~\cite{shen2016acm} -- to spot potential malicious local model updates. 
% %For instance, Shen et al. propose to cluster local model updates with K-means and thus identify outliers.
%
% % Other techniques
% As far as the server is concerned, any local model received can be from a potential malicious client. 
% FLTrust~\cite{cao2020fltrust} assumes the server acts as a client, i.e., trains a local model on an additional {\em trustworthy} dataset at the server's end and compares it against all the local models from other clients. 
% This way, the server can rely on some ``source of trust'' when discarding potentially malicious clients.
%\\
% Limitations of existing Byzantine-resilient strategies
Unfortunately, existing defense mechanisms either rely on simple heuristics (e.g., Trimmed Mean and FedMedian by~\cite{yin2018icml}) or need strong and unrealistic assumptions to work effectively (e.g., foreknowledge or estimation of the number of malicious clients in the FL system, as for Krum/Multi-Krum~\cite{blanchard2017nips} and Bulyan~\cite{mhamdi2018pmlr}, which, however, cannot exceed a fixed threshold).
Furthermore, outlier detection methods using K-means clustering~\cite{shen2016acm} or spectral analysis like DnC~\cite{shejwalkar2021ndss} do not directly consider the temporal evolution of local model updates received.
Finally, strategies like FLTrust~\cite{cao2020fltrust} require the server to collect its own dataset and act as a proper client, thereby altering the standard FL protocol.
\\
% OLD, LONG VERSION
% Overall, existing Byzantine-resilient strategies are either simple heuristics (e.g., FedMedian) or, if they are more complex, they rely on strong and unrealistic assumptions to work effectively (e.g., knowing the number of malicious clients in the FL system in advance, as for Krum and alike).
% Furthermore, data-driven outlier detection methods do not consider the temporary evolution of local model updates received (e.g., K-means clustering). 
% Finally, strategies like FLTrust requires the server to collect its own dataset and act as a proper client, thereby altering the standard FL protocol.
%
% Description of the proposed method
This work introduces a novel pre-aggregation \textit{filter} robust to untargeted model poisoning attacks. Notably, this filter $(i)$ operates without requiring prior knowledge or constraints on the number of malicious clients and $(ii)$ inherently integrates temporal dependencies. 
The FL server can employ this filter as a preprocessing step before applying \textit{any} aggregation function, be it standard like FedAvg or robust like Krum or Bulyan.
Specifically, we formulate the problem of identifying corrupted updates as a multidimensional (i.e., matrix-valued) time series anomaly detection task. 
The key idea is that legitimate local updates, resulting from well-calibrated iterative procedures like stochastic gradient descent (SGD) with an appropriate learning rate, show \textit{higher predictability} compared to malicious updates. This hypothesis stems from the fact that the sequence of gradients (thus, model parameters) observed during legitimate training exhibit regular patterns, as validated in Section~\ref{subsec:intuition}. %until convergence. 
%This regularity may be more pronounced for smooth convex loss functions, but it can still be captured within an appropriate time window, even for more complex and convoluted loss surfaces. 
%We provide evidence of this claim in Appendix~B, where we show that the average mutual information (i.e., ``predictability''), calculated over pairs of legitimate model updates sent at different FL rounds, is significantly higher than the corresponding computation for a malicious client.
\\
Inspired by the matrix autoregressive (MAR) framework for multidimensional time series forecasting~\cite{chen2021je}, we propose the FLANDERS ({\em \textbf{F}ederated \textbf{L}earning meets \textbf{AN}omaly \textbf{DE}tection for a \textbf{R}obust and \textbf{S}ecure}) filter.
The main advantages of FLANDERS over existing strategies like FLDetector~\cite{zhao2020multivariate} are its resilience to large-scale attacks, where $50\%$ or more FL participants are hostile, and the capability of working under realistic non-iid scenarios.
We attribute such a capability to two key factors: $(i)$ FLANDERS works without knowing a priori the ratio of corrupted clients, and $(ii)$ it embodies temporal dependencies between intra- and inter-client updates, quickly recognizing local model drifts caused by evil players. Below, we summarize our main contributions:

\begin{itemize}
\item[{\em(i)}]
We provide empirical evidence that the sequence of models sent by legitimate clients is more predictable than those of malicious participants performing untargeted model poisoning attacks.
\\
\item[{\em(ii)}] 
We introduce FLANDERS, the first pre-aggregation filter for FL robust to untargeted model poisoning based on multidimensional time series anomaly detection.
\\
\item[{\em(iii)}] 
We integrate FLANDERS into Flower,\footnote{\scriptsize{\url{https://flower.dev/}}} a popular FL simulation framework for reproducibility.
\\
\item[{\em(iv)}] 
We show that FLANDERS improves the robustness of the existing aggregation methods under multiple settings: different datasets, client's data distribution (non-iid), models, and attack scenarios.
\\
\item[{\em(v)}] 
We publicly release all the implementation code of FLANDERS along with our experiments.\footnote{\scriptsize{\url{https://anonymous.4open.science/r/flanders_exp-7EEB}}}
\end{itemize}

% Paper's structure and organization
The remainder of the paper is structured as follows. %some related work and the current state-of-the-art solutions to security issues that FL entails. 
Section~\ref{sec:background} covers background and preliminaries. 
In Section~\ref{sec:related}, we discuss related work.
Section~\ref{sec:problem} and Section~\ref{sec:method} describe the problem formulation and the method proposed. % to tackle it. 
Section~\ref{sec:experiments} gathers experimental results. %, and Section~\ref{sec:limitations} discusses some limitations of this work.
Finally, we conclude in Section~\ref{sec:conclusion}.
 %discusses the limitations of this work and draws future research directions.
%reports conclusions and draws perspectives for future research directions.

%%%%%%% OLD %%%%%%%
%to overcome the resilience of Byzantine failures in distributed Stochastic Gradient Descent computations. 
% The strength of Krum is its time complexity, which is linear in the gradient dimension. 
% However, the robustness of the approach is guaranteed for gradient-based learning applications only when the majority of the clients are not compromised. 
% Besides, the aggregation mechanism of Krum, as well as that of similar methods, is robust from a coarse-grained perspective and does not provide solutions to errors and perturbations that may occur at inference time.
%A related approach to~\cite{blanchard2017nips} is the work of Su et al.~\cite{su2016dc}. Here, the authors propose an iterated approximate agreement to tackle a multi-layer scenario attacked by Byzantine agents. 
%However, the method works efficiently on the sole discrete context and it is inapplicable to continuous state environments.
%\gabri{Maybe, we should just talk about the main limitations of existing countermeasures without digging into their details (or, we can just mention Krum as this is the most popular one). I will move the description of all these methods to the Related Work section.}


 \section{Brief description of the I\lowercase{n}PTA DR1}
\label{sec:inpta-dr1}



The InPTA DR1 \cite{TarafdarNobleson+2022} consists of observations of 14 MSPs conducted using the uGMRT \cite[][]{GuptaAjithkumar+2017} as part of the InPTA experiment from 2018 to 2021 typically with a bi-weekly cadence. 
These observations were carried out during observing cycles 34$-$35 and 37$-$40 of the uGMRT, where the 30 uGMRT antennae were divided into multiple phased subarrays, simultaneously observing the same source in multiple bands in total intensity mode \cite{JoshiGopu+2022}.
The channelized time series data generated by the uGMRT are recorded using the GMRT Wideband Backend~\cite{ReddyKudale+2017} in a binary raw data format, and RFI-mitigated and partially folded into \psrfits{} archives using the \pinta{} pipeline \cite[][]{SusobhananMaan+2021}.
The narrowband ToAs were measured using the Global Positioning System (GPS) and a local topocentric frequency standard was provided by the hydrogen maser clock at the GMRT.
The ToAs were fitted using \tempotwo{} \cite[][]{HobbsEdwardsManchester2006} to obtain the timing residuals. 
The timing procedure involved epoch-wise DM correction by incorporating the DM time series obtained using \dmcalc{} \cite{KrishnakumarManoharan+2021} from low-frequency simultaneous multi-band uGMRT data. 
Additionally, wideband timing residuals were generated using the wideband likelihood method described in Ref.~\cite[][]{AlamArzoumanian+2020b} and implemented in the \texttt{TEMPO} \cite[][]{NiceDemorest+2015} pulsar timing package.
For our analysis, we use only the narrowband data for all pulsars. More details of the InPTA DR1 can be found in ~\cite{TarafdarNobleson+2022}.

\section{Noise models for PTA data}
\label{sec:noise-models}

In this section, we discuss the various noise components used in our analysis. The noise analysis is critical in the search for gravitational waves to separate noise processes from the correlated GWB signal. \citet{hazboun2020} using simulations showed that improper noise models could cause bias in GWB estimates. Hence, it is critical to robustly model these noise sources to search and characterize any such correlated signals among pulsars. We model the noise processes as a stationary Gaussian processes (GP) \cite{vhaasteren+2014}. The details of the myriad achromatic and chromatic noise processes are described in this section, which will be used to obtain custom noise models for each pulsar.

\subsection{White noise}
White noise refers to the stochastic signal, where the power spectral density  is constant  across the whole frequency range and is uncorrelated across time. In PTA data, white noise dominates at high frequencies. It is modelled by re-scaling the initial ToA uncertainties ($\sigma_{ToA}$) as follows:
\begin{equation}
    \sigma^2 = \rm{EFAC}^2 \times (\sigma^2_{\text{ToA}} + \rm{EQUAD}^2) 
\end{equation}
where the \text{EFAC} accounts for radiometer noise and the \text{EQUAD} denotes the intrinsic scatter related to  the stochastic profile variations \cite{Liu.et.al,shannon2014,lam2016}. Hence, the white noise covariance matrix $C_W$, which is a diagonal matrix with diagonal elements as the re-scaled variances of ToAs, is given by:
\begin{equation}
    C_{W,i,j} = \sigma^2_{ij} \delta_{ij}.
\end{equation}
Note that the re-scaling is based on the {\it ansatz} that this uncorrelated ToA noise is Gaussian. Refs.~\cite{lentati2014,vallisneri2017} have discussed the non-Gaussian character of this noise, and its presence in a few MSPs has also been recently reported~\cite{GonchorovReardon+2021}. Although, we do not investigate the non-Gaussianity aspect in this work, modelling it may provide better ToA precision, which we plan to explore in the future.


\subsection{Red noise}
In pulsar timing, red noise refers to a  time-correlated noise, which is  stronger at lower frequencies compared to higher frequencies. 
As the GWB itself may appear as a correlated red noise signal that is spatially correlated across pulsars \cite{shannon2010}, it is of utmost importance to correctly model the pulsar-specific red noise in the data. The red noise is modelled as a stationary Gaussian process, and we adopt the ``Fourier space'' representation of the Gaussian process \cite{temponest2014}.
The timing residuals $t_i$ at each epoch due to the stochastic red signal (SRS) are approximated as:
\begin{equation}
    \delta t^{\text{SRS}}(t_i) =\sum_{l=1}^{N} X_l \cos (2\pi t_i f_l) + Y_l \sin(2\pi t_i f_l)
\end{equation}
where one can easily notice that $X_l$ and $Y_l$ appear as weights, and the basis functions are:
\begin{align}
    F_{2l-1} (t_i) = \cos (2\pi t_i f_l) \\
    F_{2l}(t_i) = \sin (2\pi t_i f_l)
\end{align}
where $l = 1,2,...,N$. If $f_l = 1/T$ where $T$ is the total observing time span, and if the epochs are evenly spaced, then this would correspond to the discrete Fourier transform. Also, we typically truncate the set at a low frequency instead of using the entire set, using an evenly spaced set of frequencies, truncating at $N/T$, where $N$ is the number of Fourier modes. We use $N$ as a hyper-parameter in our noise model selection. The choice of the optimum number of Fourier modes is discussed in Ref. \cite{ChalumeauBabak+2022}. 
\\
The covariance matrix $\Sigma$ for Fourier coefficients $X_l$, $Y_l$ is defined by power spectral density (PSD), $S$. For our analysis, we will use the power law for fitting the red noises, which can be written as follows:
\begin{equation}
S(A, \gamma) = \frac{A^2}{12\pi^2}\left(\frac{f}{yr^{-1}}\right)^{-\gamma} yr^3   
\end{equation}
where $S(A,\gamma)$ is the power spectral density, $A$ is the amplitude with normalization at a frequency of (1 $\text{yr}^{-1}$), and $\gamma$ is the spectral index. The covariance matrix for red noise in the frequency domain (see \cite{ChalumeauBabak+2022} and references within) is given by
\begin{equation}
    \Sigma _{\kappa \alpha l\beta} = S(f_k:A_\alpha,\gamma_\alpha)\delta_{kl}\delta_{\alpha\beta}/T
\end{equation}
where $l,k = 1,2,...,N$, and $\alpha,\beta$ denote the indices of the pulsar. The Kronecker delta function has been introduced,  since we take into account the spatially uncorrelated red noise.


\subsubsection{Achromatic red noise}
Achromatic red noise (RN), also known as timing noise, is modelled in PTA data to account for the  spin irregularities in pulsars \cite{cordes1985, allessandro1995}. This noise might not be significant in MSPs compared to younger pulsars but it can be detected with data over a long baseline (e.g. \cite{alam2021,GonchorovReardon+2021}). This is the observing frequency-independent noise originating from the pulsar. We model achromatic red noise using the power law described above for our analysis.


\subsubsection{Chromatic red noise}
There are delays in ToAs due to the interaction of pulse signals with matter along the path of propagation, such as the ionized interstellar medium (IISM), the ionosphere of the Earth, and the interplanetary medium. Delays in such signals are observing-frequency dependent in nature. One such dominating effect is due to the dispersion, which causes the frequency-dependent delay in the arrival time of pulses. The delay in ToAs due to the DM is related to the observing frequency according to  $\Delta t^{\text{DM}} \propto \nu^{-2}$, where $\nu$ is the observing frequency, and $DM$ is the dispersion measure \cite{LorimerKramer2004}. The timing model accounts for this effect by considering its value at reference epoch along with its first (\texttt{DM1}) and second derivatives (\texttt{DM2}). However, turbulence and inhomogeneity in the IISM coupled with the relative motion of the earth, pulsar and the IISM, may induce an additional time-correlated red noise due to these DM variations (DMv), which depends on the observing frequency \cite{you2007,keith2013}. Another such effect is the delay due to the scattering variations (Sv) caused by the signal's multi-path propagation in IISM due to refraction and diffraction, which occurs when the radio pulses from a pulsar pass through the interstellar medium (ISM), leading to delay, broadening, and other distortion of the pulses \cite{LorimerKramer2004}. The delay due to the scattering is given by $\Delta t^{\text{SC}} \propto \nu^{-4}$. It is crucial to have multi-band observations to disentangle the chromatic components of the red noise (see, for e.g. Ref.~\cite{caballero2016}). 

For the covariance matrix $F^{\text{chrom}}_i$ of chromatic noise, we use the same formula as that used for  red noise, with the additional dependency of induced ToA delays on the observing frequency, as given below:
\begin{equation}
    F^{\text{chrom}}_i = F_i\times\left(\frac{\nu_i}{1.4GHz}\right)^{-\chi}
\end{equation}
where $F_i$ is the Fourier transform of the time-domain red noise signal and contains the incomplete sine and cosine functions, $\nu_i$ is observing frequency, and $\chi$ is the chromatic index, which is 0, 2, and 4 for RN, DMv, and Sv, respectively.
 Apart from DMv and Sv, we also use the ``Free Chromatic Noise'' model (FCN), which has the chromatic index ($\chi_{FCN}$) as an additional free parameter along with the amplitude and spectral index (see Refs.~\cite{GoncharovShannon+2021,ChalumeauBabak+2022}). This model is used as a diagnostic tool for our selected noise models, where we fit for the ($\chi_{FCN}$) to look for the presence of achromatic and chromatic red noise. 


\begin{table}[!t]\begin{center}
\caption{\textbf{Analysis of offset mechanisms in 360Attention and backbone variants} on 360BEV-Matterport dataset.}
\vskip -1ex
\label{tab:analysis}
\setlength{\tabcolsep}{1mm}
\renewcommand{\arraystretch}{1.2}
\resizebox{\columnwidth}{!}{
    \begin{tabular}{ l l | c | c | l}
    \toprule[1pt]
    \textbf{Methods} & \textbf{Backbone} & \textbf{\#Param} & \textbf{FLOPs} & \textbf{mIoU} \\ \midrule\midrule
    
    \circled{1} Ours (360Attention offset) & MiT-B0 & 04.60M  & 248.57G & 36.98     \\
    \circled{2} Ours (360Attention offset) & MiT-B2 & 26.30M & 283.94G & 44.32 \\ 
    \circled{3} Ours (360Attention offset) & MiT-B4  & 62.91M & 341.34G &  \textbf{45.53}    \\  \midrule
    \circled{4} Ours (Multi-scale offset) & MiT-B2  & 26.43M  &284.17G &43.65~\obf{-0.67}   \\
    \circled{5} Ours (Fixed-range offset) & MiT-B2  & 26.30M & 283.44G &  43.28~\obf{-1.04}\\
    \circled{6} Ours (Separate offset) & MiT-B2 & 26.19M & 279.18G &  42.82~\obf{-1.50}\\\midrule
    \circled{7} Ours (360Attention offset) & MSCA-B  & 27.69M &274.59G & \textbf{46.31}~\gbf{+1.99} \\ 

    \bottomrule
    \end{tabular}
}
\end{center}
\vskip -4ex
\end{table}

\section{Results and discussion}
\label{sec:discussion}


This section presents the results of custom noise modelling for each pulsar. One thing to note is that as the InPTA DR1 parfiles contain \texttt{DMXs} that absorb the DM variations, we remove them from the parfiles and fit for \texttt{DM1} and \texttt{DM2} using \texttt{TEMPO2}. These final parfiles are used for our noise analysis. We also remove \texttt{T2EFACs} from our parfiles before using them for SPNA work. In the initial step of EQUAD model selection, we find that pulsars J0613$-$1224, J1012+5307, J1744$-$1134 and J2124$-$3358 do not prefer EQUAD, while all the remaining ten pulsars strongly prefer it.
Overall, eight pulsars J1012+5307, J1022+1001, J1643$-$1224, J1713+0747, J1744$-$1134, J1909$-$3744, J1939+2134 and J2145$-$0750 show the presence of DMv in InPTA dataset. For three pulsars J1643$-$1224, J1939+2134, and J2145$-$0750, we can see from Table~\ref{model_select} that \textit{WRDS} is preferred over other models, i.e. all the red noises considered in this paper are present in these pulsars. Except for J0751+1807 and J1600$-$3053, all the other pulsars support the red noises in the InPTA data. Interestingly, we see scattering variations for four pulsars in our sample: J1643$-$1224, J1713+0747, J1939+2134, and J2145$-$0750. Six pulsars J0751+1807, J1012+5307, J1022+1001, J1600$-$3053, J1713+0747 and J1744$-$1134 do not show the presence of achromatic red noise. Pulsars J1909$-$3744, J1939+2134 and J2145$-$0750 prefer the highest number of Fourier modes for RN, denoting the presence of RN at high frequencies. Similarly, Sv is present in high frequencies for PSR J2145$-$0750, while only in low frequencies for J1643$-$1224, J1713+0747 and J1939+2134.



\subsection*{\texorpdfstring{J0437$-$4715}{J0437-4715}}
PSR J0437$-$4715 is one of the brightest pulsars observed by InPTA. In the InPTA DR1, we can see achromatic red noise in the ToA residuals and do not find significant DM variations in the DM time series, whereas both these noises are present in PPTA for this pulsar~\cite{GonchorovReardon+2021}. As the Bayes factor for \textit{WRD} over \textit{WR} was inconclusive, we selected \textit{WR} because it has less number of free parameters. As the baseline for this pulsar is around one year for InPTA, it could be one of the reasons for the differences in models.
To understand this, more work is underway to characterize the jitter across frequencies as we have simultaneous multi-band observations.


\subsection*{\texorpdfstring{J0613$-$0200}{J0613-0200}}
For this pulsar, again, we see slight variations in the ToA residuals, while no such variations are seen in the corresponding DM time series, where the $\Delta\text{DMs}$ have precision up to the fourth decimal place. \textit{WR}, \textit{WRD}, and \textit{WRDS} have nearly the same value of Bayesian Evidence, and \textit{WRD}, \textit{WRDS} have inconclusive Bayes factors over \textit{WR} as seen in Table \ref{model_select}. Hence, we select \textit{WR} based on it having fewer number of free parameters. We also find that the noise model in PPTA~\cite{GonchorovReardon+2021} also contains only achromatic red noise similar to our results, while EPTA~\cite{ChalumeauBabak+2022} contains additional DM variations.


\subsection*{J0751+1807}
For this pulsar, Bayesian analysis suggests no significant Bayes factors of models with any kind of red noise over the white noise only model, i.e. \textit{W} and hence, \textit{W} is chosen based on the simplicity of the model. The DM time series and residuals from the InPTA DR1 also show no significant variation over time, supporting the selected noise model.

\subsection*{J1012+5307}
In this pulsar, the DM variations are evident, especially in the later part of the data, while no such variations are observed in ToA residuals. The Bayes factor for \textit{WRD}, \textit{WDS}, and \textit{WRDS} with respect to \textit{WD} is insignificant as seen in Table \ref{model_select}. Hence we select \textit{WD} based on the model's simplicity. Also, the data before cycle 37 did not have very high precision, causing large error bars on the DM, but the subsequent data has high precision DMs, where a discernible trend is evident.
Hence, one expects the DM variations to be the dominant noise process, and Bayesian analysis favors the same. The prefered noise model for this pulsar in EPTA~\cite{ChalumeauBabak+2022} contains both achromatic red noise and DM variations. The lack of red noise in InPTA DR1 could be due to the short data span and large data gap for this pulsar.



\begin{figure*}[t!]
\includegraphics[keepaspectratio=true,scale=0.40]{figures/J1600-3053_crn.pdf}
  \includegraphics[keepaspectratio=true,scale=0.40]{figures/J1939+2134_crn_arn_dmn_scn.pdf}
   \caption{(Left Panel): J1600$-$3053 posterior distributions of the chromatic index $\chi_{FCN}$. (Right Panel): J1939+2134 posterior distributions of the chromatic index $\chi_{FCN}$.}
  \label{fig:16001939crn}
\end{figure*}  



\subsection*{J1022+1001}
For this pulsar, we again observe that the DM variations are conspicuous from cycle 37 data onward. Hence, it seems to be the dominant process, with no discernible variations in the ToA residuals. 
The Bayesian analysis gives comparable evidences for \textit{WDS} to \textit{WD} with Bayes factors close to one (see Table \ref{model_select}), which implies there is no preferred model among these. Therefore, we chose \textit{WD} as it had the smallest number of free parameters. 
Our noise model also agrees with the PPTA noise model for this pulsar~\cite{GonchorovReardon+2021}.






\subsection*{\texorpdfstring{J1600$-$3053}{J1600-3053}}
This pulsar does not show signatures of any type of red noises based on Bayesian evidence, where all the models show comparable Bayesian evidence values. This is reaffirmed by the analysis based on the free chromatic index, where the chromatic index is a one-sided distribution with large error bars, as seen in Fig.~\ref{fig:16001939crn}. In contrast, the amplitude and spectral index are unconstrained, which implies that it encapsulates only the white noise. The DM series and ToA residuals show no significant variations, supporting the Bayesian analysis result of \textit{W}. PPTA~\cite{GonchorovReardon+2021} and EPTA~\cite{ChalumeauBabak+2022} report achromatic red noise and DM variations (along with scattering variations in the case of PPTA), and we also see scattering variations in DM time series in NANOGrav 12.5-year dataset \cite{AlamArzoumanian+2020b}. {InPTA dataset for this pulsar begins where the NANOGrav 12.5-year dataset terminates \cite{TarafdarNobleson+2022,AlamArzoumanian+2020b}, and comparing the DMs between their last epoch (52.333508 $pc/cm^3$) and our first epoch (with \texttt{DMX}) (52.3326 $pc/cm^3$), we observe the difference of $9\times10^{-4}~pc/cm^3$, which is consistent within errors. Furthermore, we do not see any significant scatter-broadening in the low-frequency observations of this pulsar. In addition, we also find that in the NANOGrav 12.5-yr dataset, the DMs tend to stabilize towards the end. A complete understanding of this inconsistency shall be explored, but we suspect that the DMs are stable across the 3.5 years InPTA dataset and may have been variable before, as seen in NANOGrav dataset along with EPTA \cite{ChalumeauBabak+2022}.

\subsection*{\texorpdfstring{J1643$-$1224}{J1643-1224}}
Bayesian analysis for this pulsar strongly prefers \textit{WRDS}, which has achromatic red noise, DM, and scattering variations. Low-frequency observations of this pulsar confirm significant scatter-broadening, which varies from epoch to epoch. In addition to that, DM variations are also seen. From the DR1 plots, it is difficult to adjudicate between scatter broadening and DM variations as both lead to similar exponential scatter, but DM variations seem very evident.



 \subsection*{J1713+0747}
PSR J1713+0747 strongly supports the \textit{WDS} based on the estimated Bayes factor, i.e. only DM and scattering variations. This is one of the best-timed PTA pulsars, and there seems to be very less achromatic red noise. There are DM and scattering variations, but their amplitudes are very small. Overall, the pulsar seems to be a good timer, provided these noise sources are included. One thing to note is that the data does not include the DM event \cite{lam+2018} and profile change event \cite{singha+2021}. There is very little scattering variation, as seen by other PTAs \cite{ChalumeauBabak+2022}, while our model contains it. This could be because the InPTA dataset contains low-frequency data, which is absent in the EPTA dataset, hence could be more sensitive to detect this weak scattering, which may have been missed in EPTA.

\subsection*{\texorpdfstring{J1744$-$1134}{J1744-1134}}
The data span for this pulsar is only about six months; hence it is difficult to make definitive conclusions about the RN here. On the other hand, the DM variations are evident in the DM time series. The Bayesian analysis supports this and provides \textit{WD} as the most selected model. The selected model for this pulsar in EPTA~\cite{ChalumeauBabak+2022} has RN along with DMv, which is absent in InPTA data due to a small time span, while 
 \citet{GonchorovReardon+2021} obtain a similar model as ours.

\subsection*{J1857+0943}
In this pulsar, the achromatic red noise is quite visible in the ToA residuals and seems to be the dominant source, while there are no visible DM variations in the DM time series. Bayesian model selection prefers \textit{WR}, which is expected based on our observations and supports our claim. \citet{GonchorovReardon+2021} model also contains DM noise, which is absent in our modelling and can be due to our relatively short data span.

\begin{figure*}
  \includegraphics[keepaspectratio=true,scale=0.40]{figures/J1909-3744_crn_arn.pdf}
  \includegraphics[keepaspectratio=true,scale=0.40]{figures/J1909-3744_crn_arn_dmn.pdf}
 \caption{(Left Panel): J1909$-$3744 posterior distributions of the chromatic index $\chi_{FCN}$ \bthis{with the noise model containing WN and RN}. (Right Panel): J1909$-$3744 posterior distributions of the chromatic index $\chi_{FCN}$ \bthis{with the noise model containing WN, RN and DMv}.}
  \label{fig:1909crn}
\end{figure*}

\subsection*{\texorpdfstring{J1909$-$3744}{J1909-3744}}
PSR J1909-3744 exhibits the best stability among the PTA pulsars. Nevertheless, there are significant variations in the DM towards this pulsar in the InPTA DR1, as seen from the DM time series. Our analysis indicates strong evidence for \textit{WRD} and \textit{WRDS}, both of which incorporate achromatic red noise and DM noise. The latter model also includes a GP corresponding to the variations in scatter-broadening and is marginally favored over the former by $\ln(\text{Bayes Factor})$ of 4.4. However, this pulsar has no pulse broadening, even at 300 MHz. To investigate this further, we examined the parameters estimated for the different noise processes in \textit{WR}, \textit{WRD}, and \textit{WRDS}. The amplitude \bthis{(at frequency of $1 yr^{-1}$)}, and high-frequency cut-off for the achromatic noise are consistent for all these models ($\log A_{RN} \sim -12.5$ and $\gamma_{RN} \sim$ 0.68), and so is the case for DM noise ($\log A_{DMv} \sim -13.5$ and $\gamma_{DMv} \sim$ 2) with DM noise process an order of magnitude weaker than the achromatic red-noise with a much larger frequency content than the latter. In contrast, the noise process amplitude (\bthis{at frequency of $1 yr^{-1}$}) representing the scatter broadening is two orders of magnitude smaller than either the achromatic red-noise process or the DM noise process with $\gamma$ essentially consistent with zero (in other words, consistent with no variations). We further investigated this by using a GP with a free chromatic index. A model involving just the achromatic red noise process with a chromatic process having a free index yields hyper-parameters similar to those of chromatic red noise and the DM noise with a chromatic index of $2.4^{+0.19}_{-0.16}$, close to the expected chromatic index of 2 for a DM noise process as seen in Fig.~\ref{fig:1909crn}. Furthermore, parameter estimation with a model consisting of achromatic red noise, DM noise, and chromatic process with a free index again yields a two-order weaker free index process, with significant error bars on the index (see Fig.~\ref{fig:1909crn} again). Thus, there is no strong evidence for a scattering process consistent with the absence of any observed scattering at 300 MHz despite marginally higher evidence for a scattering process. Hence, we have chosen \textit{WRD} for this pulsar which is the simpler model.

\subsection*{J1939+2134}
PSR J1939+2134 is the oldest known MSP. For a long time, it was assumed to be the most stable rotator \citep{krt1992}. Still, high-precision timing campaigns have shown that not only does its rotation rate wander, but it also exhibits significant DM and scatter-broadening variations. Observations near 300 MHz clearly show that the pulse profile is scatter broadened at these frequencies~\citep{joshirama2006}. Thus, we needed to model the  DM and scattering variations apart from the achromatic red noise and white noise for this bright but relatively distant PTA pulsar (\textit{WRDS}). The posterior distributions of the relevant GP are shown in Appendix~\ref{appendix_b}, while those of the full model are presented in Appendix~\ref{appendix_c}. As mentioned before, the GP representing the scatter-broadening variations assumes Kolmogorov turbulence with a chromatic index of 4. Previous studies have shown that this is not true along all lines of sight, and the chromatic index can go as low as -0.7, i.e. shallower than Kolmogorov~\cite{levin2016,kk2019,turner2021}. To investigate this, we used a FCN model apart from the usual white noise, achromatic red noise, DM variations, and scattering variations. While this incorporates an additional free parameter in terms of the chromatic index, we find that the corresponding FCN parameters to be well constrained, as can be seen in Fig.~\ref{fig:16001939crn} with $\chi_{FCN}$ $\sim$ 2.86, and $\log_{10}A_{FCN}$ $\sim$ -12.58, which is much greater than conventional scattering variation model with chromatic index 4, and comparable with other red noise amplitudes (see Table \ref{parameter_values}). This indicates that for this pulsar, the chromatic index does not agree with Kolmogorov turbulence with chromatic index 4. We plan to investigate this further from direct measurements.

\subsection*{\texorpdfstring{J2124$-$3358}{J2124-3358}}
The selected model for this pulsar is \textit{WR}, as \textit{WRD} and \textit{WR} have comparable Bayesian evidence, and hence, inconclusive Bayes factor for \textit{WRD} over \textit{WR}. Therefore, \textit{WR} is chosen based on fewer number of  free parameters. Although, the ToA residuals have large error bars in InPTA DR1, slight achromatic red noise is visible. At the same time, the high precision $\Delta\text{DM}$ estimates vary only in the fourth decimal place, and hence do not showing significant DM variations, which supports the selected model.

\subsection*{\texorpdfstring{J2145$-$0750}{J2145-0750}}
For PSR J2145$-$0750, we obtain \textit{WRDS} as the strongly preferred using Bayesian model selection, suggesting the presence of achromatic red noise, as well as DM and scattering variations. In the DR1 plots, we see in ToA residuals that there is a small amplitude achromatic red noise variation over a large time scale, and we see in the DM series there are short time scale small amplitude DM variations. Scattering noise is not evident as it is difficult to see small scattering in DM series due to smaller amplitude than DM noise in this case.



\section{Conclusion}\label{sec:conclusion}
In this work, we focus on addressing the fundamental challenge of OOD detection tasks, which is how to fully understand the semantic discrepancy between the ID/OOD samples. We reveal that the key to success in the realistic SCOOD task is to allocate as many ID samples in the unlabeled set correctly as possible. To this end, we propose a novel uncertainty-aware optimal transport scheme that introduces class-specific energy scores as guidance for effective label assignment. Experimental results show that our method achieves better performance than previous state-of-the-art methods on SCOOD benchmarks.

\textbf{Limitations.} In addition to temperature scaling, other techniques such as feature clipping applied in ReAct~\cite{sun2021react} also enhance the performance of energy score, so how to obtain an OOD score that best fits the SCOOD task can be further explored. Moreover, a setting highly related to SCOOD has been proposed in \cite{katz2022training} and formulated as a constrained optimization problem. We will also theoretically analyze these practical OOD settings in our feature work.

% \section*{Acknowledgments}
\textbf{Acknowledgments.} 
This work is supported by National Key R\&D Program of China under Grant 2020AAA0105701, National Natural Science Foundation of China (NSFC) under Grants 61872327, Major Special Science and Technology Project of Anhui, National Natural Science Foundation of China (62033012) and Ant Group through Ant Research Intern Program.




\begin{acknowledgments}

InPTA acknowledges the support of the GMRT staff in resolving technical difficulties and providing technical solutions for high-precision work. We acknowledge the GMRT telescope operators for the observations. The GMRT is run by the National Centre for Radio Astrophysics of the Tata Institute of Fundamental Research, India.
AmS is supported by CSIR fellowship Grant number 09/1001(12656)/2021-EMR-I
and T-641 (DST-ICPS).
BCJ, YG and YM acknowledge the support of the Department of Atomic Energy, Government of India, under Project Identification \# RTI 4002. BCJ acknowledges the support of the Department of Atomic Energy, Government of India, under project No. 12-R\&D-TFR-5.02-0700. 
AS is supported by the NANOGrav NSF Physics Frontiers Center (awards \#1430284 and 2020265).
KT is partially supported by JSPS KAKENHI Grant Numbers 20H00180, 21H01130, and 21H04467, Bilateral Joint Research Projects of JSPS, and the ISM Cooperative Research Program (2021-ISMCRP-2017).
AKP is supported by CSIR fellowship Grant number 09/0079(15784)/2022-EMR-I.
KN is supported by the Birla Institute of Technology \& Science Institute fellowship.
SH is supported by JSPS KAKENHI Grant Number 20J20509.
TK is partially supported by the JSPS Overseas Challenge Program for Young Researchers.
AC acknowledges financial support provided under the European Union’s H2020 ERC Consolidator Grant “Binary Massive Black Hole Astrophysics” (B Massive, Grant Agreement: 818691, PI - A. Sesana).
GS acknowledges financial support provided under the European Union’s H2020 ERC Consolidator Grant “Binary Massive Black Hole Astrophysics” (B Massive, Grant Agreement: 818691, PI - A. Sesana).
We acknowledge the National Supercomputing Mission (NSM) for providing computing resources of ‘PARAM Ganga’ at the Indian Institute of Technology Roorkee as well as `PARAM Seva' at IIT Hyderabad, which is implemented by C-DAC and supported by the Ministry of Electronics and Information Technology (MeitY) and Department of Science and Technology (DST), Government of India.\\

\end{acknowledgments}
\bibliographystyle{apsrev4-1}
\bibliography{spna}

\appendix

\section{Parameters obtained from noise analysis}
\label{appendix_a}
Here, we provide the final models and the parameters in Table \ref{parameter_values} for all 14 pulsars for each noise component with a 68 \% confidence interval. 






\begin{sidewaystable*}[h!]
	\centering
 \vspace{9cm}
	\caption{Median and 16 $-$ 84 \% credible intervals of the posterior distributions of each single-pulsar noise model parameters for all 14 DR1 pulsars.}
	\setlength{\tabcolsep}{4.5pt}
 {\renewcommand\arraystretch{2.0}
\begin{tabular}{|c|c|c|c|c|c|c|c|c|c|c|c|}
\hline
\multirow{2}{*}{Pulsar} & \multirow{2}{*}{Final Models} &  \multirow{2}{*}{\bthis{Timespan }} & \multicolumn{3}{c|}{Red noise} & 
 \multicolumn{3}{c|}{DM variations} & \multicolumn{3}{c|}{Scattering variations} \\

\cline{4-12}
  & & \bthis{(yr)} & \multicolumn{1}{c|}{$\log_{10}\rm{A}_{RN}$} & \multicolumn{1}{c|}{$\gamma_{RN}$} & \multicolumn{1}{c|}{Fourier modes} & \multicolumn{1}{c|}{$\log_{10}\rm{A}_{DM}$} & \multicolumn{1}{c|}{$\gamma_{DM}$} & \multicolumn{1}{c|}{Fourier modes} & \multicolumn{1}{c|}{$\log_{10}\rm{A}_{Sv}$} & \multicolumn{1}{c|}{$\gamma_{Sv}$} & \multicolumn{1}{c|}{Fourier modes}\\
\hline
J0437$-$4715 & Model2 & 0.85 & $-12.17^{+0.17}_{-0.12}$ & $0.28^{+0.36}_{-0.20}$ & 10 & ------ & ------ & ------ & ------ & ------ & ------ \\  

J0613$-$0200 & Model2 & 3.39 & $-11.85^{+0.16}_{-0.14}$  & $1.10^{+0.59}_{-0.57}$ & 27 & ------ & ------ & ------& ------ & ------ &------ \\

J0751+1807 &  Model1 & 3.39 & ------ & ------ & & ------ & ------ & ------ & ------ & ------ &  ------\\

J1012+5307 &  Model6 & 3.36 & ------ & ------ & ------ & $-13.17^{+0.19}_{-0.20}$ & $0.53^{+0.55}_{-0.37}$ & 16 & ------ & ------ & ------ \\

J1022+1001 &  Model6 & 3.36 & ------ & ------ &------ & $-12.64^{+0.15}_{-0.13}$ & $0.59^{+0.44}_{-0.36}$ & 40 & ------ & ------ &------ \\

 J1600$-$3053 &  Model1 & 3.36 & ------ & ------ & & ------ & ------ &------ & ------ & ------ & ------\\

J1643$-$1224 &  Model5 & 3.39 & $-12.44^{+0.25}_{-0.33}$ & $4.11^{+1.62}_{-1.37}$ & 27 & $-12.42^{+0.09}_{-0.08}$ & $2.46^{+0.29}_{-0.24}$ & 40 & $-13.46^{+0.17}_{-0.16}$ & $3.44^{+1.17}_{-0.99}$ & 6 \\

J1713+0747 &  Model4 & 2.87 & ------ & ------ & ------ & $-13.85^{+0.25}_{-0.17}$ & $0.48^{+0.62}_{-0.34}$ & 34 & $-14.48^{+0.22}_{-0.23}$ & $3.70^{+1.61}_{-1.39}$ & 5\\

J1744$-$1134 &  Model6 & 0.44 & ------ & ------ & ------ & $-11.97^{+0.43}_{-0.36}$ & $3.65^{+1.09}_{-1.02}$ & 4 & ------ & ------ & ------ \\

J1857+0943 & Model2 & 3.39 & $-12.09^{+0.22}_{-0.52}$ & $1.43^{+2.40}_{-0.97}$ & 6 & ------ & ------ & ------ & ------ &  ------ & ------ \\

J1909$-$3744 &  Model5  & 3.38 & $-12.45^{+0.13}_{-0.13}$ & $0.67^{+0.32}_{-0.33}$ & 40 & $-13.43^{+0.12}_{-0.10}$ & $2.06^{+0.69}_{-0.46}$ & 40 & ------ & ------ & ------ \\

J1939+2134 &  Model5 & 3.38 & $-12.52^{+0.10}_{-0.10}$ & $1.09^{+0.26}_{-0.27}$ & 40 & $-12.75^{+0.08}_{-0.07}$ & $1.84^{+0.24}_{-0.23}$ & 40 & $-14.12^{+0.10}_{-0.09}$ & $3.13^{+0.54}_{-0.48}$ & 16 \\

J2124$-$3358 &  Model2 & 3.38 & $-12.23^{+0.15}_{-0.14}$ & $1.04^{+0.96}_{-0.70}$ & 6 & ------ & ------ & ------ & ------ & ------ &------ \\

J2145$-$0750 &  Model5 & 3.38 & $-12.23^{+0.12}_{-0.11}$ & $1.98^{+0.71}_{-0.52}$ & 40 & $-13.43^{+0.30}_{-0.37}$ & $4.50^{+1.55}_{-1.49}$ & 27 & $-14.07^{+0.11}_{-0.10}$ & $1.63^{+0.29}_{-0.27}$ & 40 \\ \hline

\end{tabular}
}
\label{parameter_values}
\end{sidewaystable*}
 



\section{Red noise posterior plots for all pulsars}
\label{appendix_b}
Here, we provide the posteriors plots for each noise component present for all the pulsars, which can be found in Fig \ref{posterior_plots}.

\begin{figure*}[h!]
\caption{1D marginalized posterior distributions with 68\%,90\%,99\% credible intervals for red noise components present in respective pulsars.}
 \begin{subfigure}
		\centering
   \label{posterior_plots}
 \includegraphics[keepaspectratio=true,scale=0.4]{figures/J0437-4715_arn.pdf}
	\end{subfigure}
 \begin{subfigure}
		\centering
 \includegraphics[keepaspectratio=true,scale=0.4]{figures/blank.pdf}
	\end{subfigure}
 \begin{subfigure}
		\centering
 \includegraphics[keepaspectratio=true,scale=0.4]{figures/blank.pdf}
 \captionsetup{labelformat=empty}
  \caption{FIG. 3(A): J0437$-$4715 posterior distributions with 68\%,90\%,99\% credible intervals for achromatic red noise for \textit{WR} model.}
	\end{subfigure}
 
 \vspace{1cm}

 \begin{subfigure}
		\centering
 \includegraphics[keepaspectratio=true,scale=0.4]{figures/J0613-0200_arn.pdf}
	\end{subfigure}
 \begin{subfigure}
		\centering
 \includegraphics[keepaspectratio=true,scale=0.4]{figures/blank.pdf}
	\end{subfigure}
  \begin{subfigure}
		\centering
 \includegraphics[keepaspectratio=true,scale=0.4]{figures/blank.pdf}
  \captionsetup{labelformat=empty}
  \caption{FIG. 3(B): J10613$-$0200 posterior distributions with 68\%,90\%,99\% credible intervals for achromatic red noise for \textit{WR} model.}
	\end{subfigure}
 
 \vspace{1cm}

 	\begin{subfigure}
		\centering	
 \includegraphics[keepaspectratio=true,scale=0.4]{figures/J1012+5307_dm.pdf}
	\end{subfigure}
    \begin{subfigure}
		\centering
 \includegraphics[keepaspectratio=true,scale=0.4]{figures/blank.pdf}
	\end{subfigure}
  \begin{subfigure}
		\centering
 \includegraphics[keepaspectratio=true,scale=0.4]{figures/blank.pdf}
  \captionsetup{labelformat=empty}
  \caption{FIG. 3(C): J1012+5307 posterior distributions with 68\%,90\%,99\% credible intervals for DMv for \textit{WD} model.}
	\end{subfigure}
 \end{figure*}

\begin{figure*}[h!]
 	\begin{subfigure}
	\centering	
 \includegraphics[keepaspectratio=true,scale=0.4]{figures/J1022+1001_dm.pdf}
 	\end{subfigure}
   \begin{subfigure}
		\centering
 \includegraphics[keepaspectratio=true,scale=0.4]{figures/blank.pdf}
	\end{subfigure}
    \begin{subfigure}
		\centering
 \includegraphics[keepaspectratio=true,scale=0.4]{figures/blank.pdf}
  \captionsetup{labelformat=empty}
  \caption{FIG. 3(D): J1022+1001 posterior distributions with 68\%,90\%,99\% credible intervals for DMv for \textit{WD} model.}
	\end{subfigure}
 
\vspace{1.5cm}

\begin{subfigure}
 \centering
 \includegraphics[keepaspectratio=true,scale=0.4]{figures/J1643-1224_arn.pdf}
\end{subfigure} 
 \begin{subfigure}
 \centering
 \includegraphics[keepaspectratio=true,scale=0.4]{figures/J1643-1224_dm.pdf}
\end{subfigure} \begin{subfigure}
 \centering
 \includegraphics[keepaspectratio=true,scale=0.4]{figures/J1643-1224_sc.pdf}
  \captionsetup{labelformat=empty}
  \caption{FIG. 3(E): J1643$-$1224 posterior distributions with 68\%,90\%,99\% credible intervals for achromatic red noise, DMv and Sv for \textit{WRDS} model.}
\end{subfigure} 

\vspace{1.5cm} 

\begin{subfigure}
 \centering
 \includegraphics[keepaspectratio=true,scale=0.4]{figures/J1713+0747_dm.pdf}
\end{subfigure} 
\begin{subfigure}
 \centering
 \includegraphics[keepaspectratio=true,scale=0.4]{figures/J1713+0747_sc.pdf}
\end{subfigure}
    \begin{subfigure}
		\centering
 \includegraphics[keepaspectratio=true,scale=0.4]{figures/blank.pdf}
  \captionsetup{labelformat=empty}
  \caption{FIG. 3(F): J1713+0747 posterior distributions with 68\%,90\%,99\% credible intervals for DMv and Sv for \textit{WDS} model.}
	\end{subfigure}


\end{figure*}


\begin{figure*}[h!]


 \begin{subfigure}
		\centering
 \includegraphics[keepaspectratio=true,scale=0.4]{figures/J1744-1134_dm.pdf}
	\end{subfigure}
  \begin{subfigure}
		\centering
 \includegraphics[keepaspectratio=true,scale=0.4]{figures/blank.pdf}
	\end{subfigure}
  \begin{subfigure}
		\centering
 \includegraphics[keepaspectratio=true,scale=0.4]{figures/blank.pdf}
   \captionsetup{labelformat=empty}
  \caption{FIG. 3(G): J1744$-$1134 posterior distributions with 68\%,90\%,99\% credible intervals for DMv for \textit{WD} model.}
	\end{subfigure}

 \vspace{1.5cm}

	\begin{subfigure}
		\centering
 \includegraphics[keepaspectratio=true,scale=0.4]{figures/J1857+0943_arn.pdf}
	\end{subfigure}
  \begin{subfigure}
		\centering
 \includegraphics[keepaspectratio=true,scale=0.4]{figures/blank.pdf}
	\end{subfigure}
  \begin{subfigure}
		\centering
 \includegraphics[keepaspectratio=true,scale=0.4]{figures/blank.pdf}
   \captionsetup{labelformat=empty}
  \caption{FIG. 3(H): J1857+0943 68\%,90\%,99\% credible intervals for achromatic red noise for \textit{WR} model.}
	\end{subfigure}
 
\vspace{1.5cm}

\begin{subfigure}
		\centering
 \includegraphics[keepaspectratio=true,scale=0.4]{figures/J1909-3744_arn.pdf}
	\end{subfigure}
 	\begin{subfigure}
		\centering	
 \includegraphics[keepaspectratio=true,scale=0.4]{figures/J1909-3744_dm.pdf}
	\end{subfigure}
  \begin{subfigure}
		\centering
 \includegraphics[keepaspectratio=true,scale=0.4]{figures/blank.pdf}
    \captionsetup{labelformat=empty}
  \caption{FIG. 3(I): J1909$-$3744 posterior distributions with 68\%,90\%,99\% credible intervals for achromatic red noise and DMv for \textit{WRD} model.}
	\end{subfigure}

 
 \end{figure*}


\begin{figure*}[h!]

 
 	\begin{subfigure}
		\centering	
 \includegraphics[keepaspectratio=true,scale=0.4]{figures/J1939+2134_arn.pdf}
	\end{subfigure}	
 \begin{subfigure}
		\centering	
  \includegraphics[keepaspectratio=true,scale=0.4]{figures/J1939+2134_dm.pdf}
	\end{subfigure}
 	\begin{subfigure}
		\centering	
 \includegraphics[keepaspectratio=true,scale=0.4]{figures/J1939+2134_sc.pdf}
    \captionsetup{labelformat=empty}
  \caption{FIG. 3(J): J1939+2134 posterior distributions with 68\%,90\%,99\% credible intervals for achromatic red noise, DMv and Sv for \textit{WRDS} model.}
	\end{subfigure}
 
\vspace{1.5cm}

	\begin{subfigure}
		\centering	 
  \includegraphics[keepaspectratio=true,scale=0.4]{figures/J2124-3358_arn.pdf}
	\end{subfigure}
  \begin{subfigure}
		\centering
 \includegraphics[keepaspectratio=true,scale=0.4]{figures/blank.pdf}
	\end{subfigure}
  \begin{subfigure}
		\centering
 \includegraphics[keepaspectratio=true,scale=0.4]{figures/blank.pdf}
    \captionsetup{labelformat=empty}
  \caption{FIG. 3(K): J2124$-$3358 posterior distributions with 68\%,90\%,99\% credible intervals for achromatic red noise for \textit{WR} model.}
	\end{subfigure}


 \vspace{1.5cm}


	\begin{subfigure}
		\centering	
 \includegraphics[keepaspectratio=true,scale=0.4]{figures/J2145-0750_arn.pdf}
	\end{subfigure}
 	\begin{subfigure}
		\centering	
 \includegraphics[keepaspectratio=true,scale=0.4]{figures/J2145-0750_dm.pdf}
 	\end{subfigure}
 \begin{subfigure}
		\centering
 \includegraphics[keepaspectratio=true,scale=0.4]{figures/J2145-0750_sc.pdf}
     \captionsetup{labelformat=empty}
  \caption{FIG. 3(L): J2145$-$0750 posterior distributions with 68\%,90\%,99\% credible intervals for achromatic red noise, DMv and Sv for \textit{WRDS} model.}
	\end{subfigure}
\end{figure*}


\section{Full corner plots for all pulsars}
\label{appendix_c}
The marginalized posterior credible intervals  with white and red noises for each pulsar are shown here in FIG.4 on page \pageref{fullcorner}.

\begin{figure*}[h!]

 \captionsetup{labelformat=empty}
\caption{FIG. 4: Posterior distributions with 68\%,90\%,99\% credible intervals for all noise components present in respective pulsars. For white noises, we used abbreviations such that \textbf{B3} and \textbf{B5} stand for band3 and band5 data, followed by \textbf{A} or \textbf{B}, which denotes pre-cycle36 or post-cycle36 data respectively. \textbf{EFAC} is \texttt{efac} while \textbf{EQ} is \texttt{log10\_t2equad} (for eg: \textbf{B5BEQ} is \texttt{log10\_t2equad} for band5 post-cycle36 data).}
\label{fullcorner}
 \vspace*{1in}
 \centering
 \includegraphics[keepaspectratio=true,scale=0.40]{figures/J0437-4715_final.pdf}
 \captionsetup{labelformat=empty}
  \caption{FIG. 4(A): J0437$-$4715 posterior distributions with 68\%,90\%,99\% credible intervals for white noise and achromatic red noise for \textit{WR} model.}
\end{figure*}


\begin{figure*}[h!]
 \centering
 \vspace*{1.75in}
 \includegraphics[keepaspectratio=true,scale=0.40]{figures/J0613-0200_final.pdf}
 \captionsetup{labelformat=empty}
 \caption{FIG. 4(B): J0613$-$0200 posterior distributions with 68\%,90\%,99\% credible intervals for white noise and achromatic red noise for \textit{WR} model.}
\end{figure*}




\begin{figure*}
 \centering
 \vspace*{2in}
 \includegraphics[keepaspectratio=true,scale=0.35]{figures/J0751+1807_final.pdf}
 \captionsetup{labelformat=empty}
  \caption{FIG. 4(C): J0751+1807 posterior distributions with 68\%,90\%,99\% credible intervals for white noise for \textit{W} model.}

\end{figure*}


\begin{figure*}
\begin{subfigure}
 \centering
 \includegraphics[keepaspectratio=true,scale=0.35]{figures/J1012+5307_final.pdf}
   \captionsetup{labelformat=empty}
  \caption{FIG. 4(D): J1012+5307 posterior distributions with 68\%,90\%,99\% credible intervals for white noise and DMv for \textit{WD} model.}
\end{subfigure}
 \vspace{0.3in}
\begin{subfigure}
 \centering
 \includegraphics[keepaspectratio=true,scale=0.35]{figures/J1744-1134_final.pdf}
  \captionsetup{labelformat=empty}
 \caption{FIG. 4(E): J1744$-$1134 posterior distributions with 68\%,90\%,99\% credible intervals for white noise and DMv for \textit{WD} model.}
\end{subfigure}
\end{figure*}


\begin{figure*}
 \centering
 \vspace*{1.25in}
 \includegraphics[keepaspectratio=true,scale=0.33]{figures/J1022+1001_final.pdf}
   \captionsetup{labelformat=empty}
  \caption{FIG. 4(F): J1022+1001 posterior distributions with 68\%,90\%,99\% credible intervals for white noise and DMv for \textit{WD} model.}
\end{figure*}


\begin{figure*}
 \centering
 \vspace*{1.25in}
 \includegraphics[keepaspectratio=true,scale=0.40]{figures/J1600-3053_final.pdf}
  \captionsetup{labelformat=empty}
  \caption{FIG. 4(G): J1600$-$3053 posterior distributions with 68\%,90\%,99\% credible intervals for white noise for \textit{W} model.}
\end{figure*}

\begin{figure*}
 \centering
 \vspace*{1.25in}
 \includegraphics[keepaspectratio=true,scale=0.23]{figures/J1643-1224_final.pdf}
  \captionsetup{labelformat=empty}
  \caption{FIG. 4(H): J1643$-$1224 posterior distributions with 68\%,90\%,99\% credible intervals for white noise, achromatic red noise, DMv and Sv for \textit{WRDS} model.}
\end{figure*}

\begin{figure*}
 \centering
 \vspace*{1.25in}
 \includegraphics[keepaspectratio=true,scale=0.27]{figures/J1713+0747_final.pdf}
   \captionsetup{labelformat=empty}
  \caption{FIG. 4(I): J1713+0747 posterior distributions with 68\%,90\%,99\% credible intervals for white noise, DMv and Sv for \textit{WDS} model.}
\end{figure*}



\begin{figure*}
 \centering
 \vspace*{1.25in}
 \includegraphics[keepaspectratio=true,scale=0.33]{figures/J1857+0943_final.pdf}
   \captionsetup{labelformat=empty}
  \caption{FIG. 4(J): J1857+0943 posterior distributions with 68\%,90\%,99\% credible intervals for white noise and achromatic red noise for \textit{WR} model.}
\end{figure*}


\begin{figure*}
 \centering
 \vspace*{1.25in}
   \includegraphics[keepaspectratio=true,scale=0.28]{figures/J1909-3744_final.pdf}
  \captionsetup{labelformat=empty}
  \caption{FIG. 4(K): J1909$-$3744 posterior distributions with 68\%,90\%,99\% credible intervals for white noise, achromatic red noise and DMv for \textit{WRD} model.}
\end{figure*}


\begin{figure*}

 \centering
 \vspace*{1.25in}
 \includegraphics[keepaspectratio=true,scale=0.23]{figures/J1939+2134_final.pdf}
   \captionsetup{labelformat=empty}
  \caption{FIG. 4(L): J1939+2134 posterior distributions with 68\%,90\%,99\% credible intervals for white noise, achromatic red noise, DMv and Sv for \textit{WRDS} model.}

\end{figure*}


\begin{figure*}
 \centering
 \vspace*{1in}
 \includegraphics[keepaspectratio=true,scale=0.4]{figures/J2124-3358_final.pdf}
   \captionsetup{labelformat=empty}
  \caption{FIG. 4(M): J2124$-$3358 posterior distributions with 68\%,90\%,99\% credible intervals for white noise and achromatic red noise for \textit{WR} model.}
\end{figure*}

\begin{figure*}
 \centering
 \vspace*{1.25in}
 \includegraphics[keepaspectratio=true,scale=0.23]{figures/J2145-0750_final.pdf}
   \captionsetup{labelformat=empty}
  \caption{FIG. 4(N): J2145$-$0750 posterior distributions with 68\%, 90\%, 99\% credible intervals for white noise, achromatic red noise, DMv and Sv for \textit{WRDS} model.}
\end{figure*}



\end{document}

