\section{Analysis techniques}
\label{sec:anl-tech}

\subsection{Bayesian analysis}

We now provide a brief prelude to the Bayesian model comparison techniques for selecting the best noise model. In this work, Bayesian model comparison is used for selecting the best noise model and for selecting the optimum number of Fourier modes for the selected noise model. Bayesian regression is then used for estimating the optimum parameters of the selected noise model. More details on Bayesian inference and Bayesian model selection can be found in Refs.~\cite{Sanjib,Weller,Krishak} (and references therein). We follow the same notation as in Ref.~\cite{Sanjib}.

The starting point for Bayesian Model comparison is the Bayes Theorem, which states that for a model $M$ parameterized by the parameter vector $\theta$ and given the data $D$:
\begin{equation}
  P(\theta|D,M) = \frac{P(D|\theta,M)P(\theta|M)}{P(D|M)},
  \label{eq:bayesthm}
 \end{equation} 
where $P(\theta|M)$ is the prior on the parameter vector ($\theta$) for that model; $P(D|\theta,M)$ represents the likelihood; $P(\theta|D,M)$ represents the posterior probability; and $P(D|M)$ is the marginal likelihood, also known as the Bayesian Evidence. For Bayesian parameter estimation, one needs to evaluate the posterior $P(\theta|D,M)$. 

For model selection, we need to evaluate the Bayesian evidence, which can be defined as:
\begin{equation}
P(D|M) = \int P(D|\theta,M)P(\theta|M) \, d\theta
\label{eq:evid}
\end{equation}
 
To perform model selection between two models $M_1$ and $M_2$, we calculate the Bayes factor (BF), which is given by the ratio of the Bayesian evidence for the two models:
\begin{equation} 
B_{21} = \frac{\int P(D|M_2, \theta_2)P(\theta_2|M_2) \, d\theta_2}{\int P(D|M_1, \theta_1)P(\theta_1|M_1) \, d\theta_1} 
\end{equation}
The Bayes factor is then used for Bayesian model comparison. The model with the larger value of Bayesian evidence will be considered the favored model. We then use Jeffrey's scale to assess the significance of the favored model~\cite{Weller}. Based on this scale, a Bayes Factor $< 1$ indicates negative support for the model in the numerator ($M_2$), thereby favouring the model $M_1$. A value exceeding 10 points to ``substantial'' evidence for $M_2$, while a value greater than 100 points to decisive evidence. We choose 100 as the threshold Bayes factor above which a more complex model is chosen over another with a smaller number of free parameters.
In case the Bayes factor is greater than one but less than 100, we follow Occam's razor and choose the model with fewer free parameters. In the case where both models have the same number of free parameters, we chose the model based on prior information on the presence of the parameters in that dataset. However, since concerns have been raised about the reliability of Jeffreys scale~\cite{NesserisBellido13,KeeleyShafieloo22}, we also carry out additional consistency and sanity checks on the selected noise model to assess its reliability independent of the values of Bayes factor which we obtain. We assume that the stochastic processes present in our data are Gaussian, and the data is represented by these Gaussian processes, whereas deterministic signals are included in the timing parameters. We apply Gaussian likelihood in our analysis following the previous studies \citep{haasteren2009, haasteren2012, taylor2017}, and \texttt{ENTERPRISE} is used to evaluate the likelihood function.



We now provide details of the model selection procedure for selecting the best noise model, followed by harmonic mode selection. Finally, we provide details of the parameter estimation procedure for the selected noise model.



\begin{table*}[t!]
 \caption{Priors used in our noise analysis. This table gives the distributions for priors used in the Bayesian analysis for model selection. Here $\cal U$ and $\log_{10} $$\cal U$ stand for uniform and log-uniform distributions, respectively.}
	\setlength{\tabcolsep}{10pt}
 {\renewcommand\arraystretch{1.5}
\begin{tabular}{ lcc }
 Noise (abrev.) & Parameters & Priors (or fixed val.) \\ \hline \hline
 White Noise & EFAC & $\cal U$(0.1,5) \\
 (WN) & EQUAD [s] & $\log_{10}$$\cal U$$(10^{-9},10^{-5})$ \\ \hline

 Achromatic red-noise & $A_{RN}$ & $\log_{10} $$\cal U$$(10^{-18},10^{-10})$ \\
 (RN) & $\gamma_{RN}$ & $\cal U$(0,7) \\ \hline

 DM variations & $A_{DM}$ & $\log_{10} $$\cal U$$(10^{-18},10^{-10})$ \\
 (DMv) & $\gamma_{DM}$ & $\cal U$(0,7) \\ \hline

 Scattering variations & $A_{Sv}$ & $\log_{10} $$\cal U$$(10^{-18},10^{-10})$ \\
 (Sv) & $\gamma_{Sv}$ & $\cal U$(0,7) \\ \hline

 Free chromatic noise & $A_{FCN}$ & $\log_{10} $$\cal U$$(10^{-18},10^{-10})$ \\
 (FCN) & $\gamma_{FCN}$ & $\cal U$(0,7) \\
  & $\chi_{FCN}$ & $\cal U$(0,7) \\ \hline
\end{tabular}
}
\label{priors}
\end{table*}



\begin{table}
\centering
\caption{Six pre-defined noise models were used for model selections
\textbf{W}, \textbf{R}, \textbf{D}, and \textbf{S} stand for white noise, achromatic red noise, DM variations and scattering variations, respectively. \textit{Red noise parameters} column gives the number of red noise parameters for each model.}
\label{6models}
\setlength{\tabcolsep}{4.5pt}
 {\renewcommand\arraystretch{1.6}
\begin{tabular}{lll}
\textbf{Model name} & \textbf{Noise model} &\textbf{Red noise}\\ 
 & & \textbf{parameters} \\ \hline \hline
Model1(W)    & WN   &  0       \\ \hline
Model2(WR)    & WN + RN   &      2   \\ \hline
Model3(WRD)    & WN + RN + DMv  &    4   \\ \hline
Model4(WDS)    & WN + DMv + Sv  &    4   \\ \hline
Model5(WRDS)    & WN + RN + DMv + Sv &   6   \\ \hline
Model6(WD)    & WN + DMv    &   2   \\ \hline
\end{tabular}
}
\end{table}






\begin{table*}[!]
\centering
\caption{The table contains the $\ln(\text{BF})$ with respect to the selected model for each model for all 14 pulsars. The zeroes, which are in bold text in each row, represent the selected model based on the Bayes factor and the simplicity of the model. We used the maximum number of Fourier modes while performing the model selection for all pulsars.}
\label{model_select}
\setlength{\tabcolsep}{14pt}
 {\renewcommand\arraystretch{2.0}
\begin{tabular}{ccccccc}
\hline
\textbf{Pulsar} & \textbf{Model1} & \textbf{Model2} & \textbf{Model3} & \textbf{Model4} & \textbf{Model5} & \textbf{Model6} \\ 
& \textbf{(W)} & \textbf{(WR)} & \textbf{(WRD)} & \textbf{(WDS)} & \textbf{(WRDS)} & \textbf{(WD)} \\ \hline \hline
J0437$-$4715  &  -250.8  & \textbf{0}  & 1.6   & -94.8   &  0.7  & -186.5   \\ \hline
J0613$-$0200  & -77.5  & \textbf{0}  &  0   & -8.4   &  -0.7  & -14.1   \\ \hline
J0751+1807  & \textbf{0}  & -0.6   & -0.6   & -0.9   & -1.2   & -0.2     \\ \hline
J1012+5307  & -8.2   & -5.8   & -0.2   & 0.1   & -1.3   & \textbf{0}  \\ \hline
J1022+1001  & -246.8   & -85.6   & 2.0    & 0.2   & 1.1    & \textbf{0}  \\ \hline
J1600$-$3053  & \textbf{0}  & 1.9    & 2.3    & 0.8   & 2.0    & 1.1    \\ \hline
J1643$-$1224  & -164    & -150   & -137    & -16   & \textbf{0}  & -159    \\ \hline
J1713+0747  & -38    & -34    & -31    & \textbf{0}  & 1    & -30    \\ \hline
J1744$-$1134  & -48.4   & -19.8   & 0.7    & -0.6   & 0.2   & \textbf{0}  \\ \hline
J1857+0943  & -6.0    & \textbf{0} & 0.3    & -6.6   & -1     & -5.7    \\ \hline
J1909$-$3744  & -334.0  & -132.2   & \textbf{0}  & -18.8   & 4.4    & -43.5    \\ \hline
J1939+2134  & -1914.4   & -1498.8   & -557.5    & -56.8  & \textbf{0}  & -729.7    \\ \hline
J2124$-$3358  & -15.1   & \textbf{0}  & -0.2    & -4.8   & -2.0   & -4.4     \\ \hline
J2145$-$0750  & -150.4   & -56.0    & -30.8    & -12.0  & \textbf{0}  & -33.7    \\ \hline
\end{tabular}
}
\end{table*}




\subsection{Model selection}
For model selection, we calculate the Bayes factor by applying nested sampling using the \texttt{DYNESTY} package. In the first step, we perform Bayesian model selection to look for EQUAD in white noise for each pulsar. We use two white noise-only models, i.e. the first with only EFAC and the second with both EFAC and EQUAD. Once we finalize the white noise for each pulsar, we perform the Bayesian model selection using six pre-defined noise models listed in Table \ref{6models}. We form these six models with different combinations of RN, DMv and Sv. As the $\Delta\text{DM}$ estimates are very precise, it is highly unlikely to have Sv without a discernible DMv in the data. Hence, we did not use a model with only RN and Sv (without DMv) for our analysis. For this model selection step, we use the highest number of Fourier modes for RN, DMv as well as Sv, i.e. the highest frequency mode equivalent to a frequency of once per month, which roughly corresponds to two observations (using Nyquist sampling theorem) per pulsar, as the cadence of InPTA observations is roughly 14 days. The priors for EQUAD and all types of red noise amplitudes are log-uniform priors ($\log_{10} $$\cal U$), which serve as reliable approximations of non-informative priors for scale-invariant parameters. The priors for EFACs and spectral index for the various red noise models are uniform distributions ($\cal U$). For the same prior sets used for SPNA by \citet{ChalumeauBabak+2022}, our tests show that this prior range is adequate for our dataset. The final selected models, along with the Bayes factors for all the pulsars, can be found in Table \ref{model_select}. \\


\subsection{Selection of number of Fourier modes}
As shown in Ref.~\cite{ChalumeauBabak+2022}, the noise models are sensitive to the number of Fourier modes ($k$) for all types of red noise. Hence, it is imperative to perform a model selection for different numbers of Fourier modes for RN, DMv, and Sv for each pulsar. This is accomplished amongst four values of $k = 2,5,8,12~(\times T_{span})$, where $T_{span}$ is in years. As the data spans differ for a few pulsars, we have created the number of Fourier modes as a function of $T_{span}$. These values are chosen to evenly spread the number from the lowest to highest frequency modes. The lowest frequency corresponds to $2/T_{span}$, while the highest frequency corresponds to once per month. The optimum number of Fourier modes selection is performed in multiple steps, starting from RN, using \textit{WR} model, which contains only WN and RN. We obtain the Bayes factors for four values of modes described above. We set a Bayes factor threshold of 10 for the selection of the number of modes, similar to ~\cite{ChalumeauBabak+2022}. After the optimum number is obtained for RN, and if the noise model of a pulsar contains DMv, we perform mode selection for DMv using \textit{WRD} model, keeping the RN modes at the optimum mode number, as obtained in the previous step. Similarly, if the model also contains scattering variations, we perform modes selection using \textit{WRDS} model, keeping the RN and DMv modes at their optimum value and performing mode selection for only scattering variations. If the selected model only contains DMv, we use \textit{WD} model to obtain the optimum number. If the model contains DMv and Sv and no RN, then we first use \textit{WD} to obtain the optimum number of DMv and then use \textit{WDS} model to perform the modes selection for Sv, keeping DMv modes at the optimum number. Once again, we calculate the Bayesian evidence and corresponding Bayes factors using the \texttt{DYNESTY} package. The number of modes for each pulsar and noise model can be found in Table~\ref{parameter_values}.



\subsection{Parameter estimation}
After the selection of the most robust noise model, followed by choosing the optimum number of Fourier modes for each pulsar, we perform the parameter estimation using Bayesian regression to obtain the final noise parameter values using \texttt{PTMCMCSAMPLER}~\cite{EllisvanHaasteren2017}. We again use the same Gaussian likelihood and the priors for the selected noise models as those which were used for calculating the evidence. The final values are tabulated in Table~\ref{parameter_values}. All the red noise posterior plots and the full corner plots for all the noise models can be found in Appendix \ref{appendix_b} and \ref{appendix_c}, respectively. 

