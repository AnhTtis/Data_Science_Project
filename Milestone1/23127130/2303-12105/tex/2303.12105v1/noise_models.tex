\section{Noise models for PTA data}
\label{sec:noise-models}

In this section, we discuss the various noise components used in our analysis. The noise analysis is critical in the search for gravitational waves to separate noise processes from the correlated GWB signal. \citet{hazboun2020} using simulations showed that improper noise models could cause bias in GWB estimates. Hence, it is critical to robustly model these noise sources to search and characterize any such correlated signals among pulsars. We model the noise processes as a stationary Gaussian processes (GP) \cite{vhaasteren+2014}. The details of the myriad achromatic and chromatic noise processes are described in this section, which will be used to obtain custom noise models for each pulsar.

\subsection{White noise}
White noise refers to the stochastic signal, where the power spectral density  is constant  across the whole frequency range and is uncorrelated across time. In PTA data, white noise dominates at high frequencies. It is modelled by re-scaling the initial ToA uncertainties ($\sigma_{ToA}$) as follows:
\begin{equation}
    \sigma^2 = \rm{EFAC}^2 \times (\sigma^2_{\text{ToA}} + \rm{EQUAD}^2) 
\end{equation}
where the \text{EFAC} accounts for radiometer noise and the \text{EQUAD} denotes the intrinsic scatter related to  the stochastic profile variations \cite{Liu.et.al,shannon2014,lam2016}. Hence, the white noise covariance matrix $C_W$, which is a diagonal matrix with diagonal elements as the re-scaled variances of ToAs, is given by:
\begin{equation}
    C_{W,i,j} = \sigma^2_{ij} \delta_{ij}.
\end{equation}
Note that the re-scaling is based on the {\it ansatz} that this uncorrelated ToA noise is Gaussian. Refs.~\cite{lentati2014,vallisneri2017} have discussed the non-Gaussian character of this noise, and its presence in a few MSPs has also been recently reported~\cite{GonchorovReardon+2021}. Although, we do not investigate the non-Gaussianity aspect in this work, modelling it may provide better ToA precision, which we plan to explore in the future.


\subsection{Red noise}
In pulsar timing, red noise refers to a  time-correlated noise, which is  stronger at lower frequencies compared to higher frequencies. 
As the GWB itself may appear as a correlated red noise signal that is spatially correlated across pulsars \cite{shannon2010}, it is of utmost importance to correctly model the pulsar-specific red noise in the data. The red noise is modelled as a stationary Gaussian process, and we adopt the ``Fourier space'' representation of the Gaussian process \cite{temponest2014}.
The timing residuals $t_i$ at each epoch due to the stochastic red signal (SRS) are approximated as:
\begin{equation}
    \delta t^{\text{SRS}}(t_i) =\sum_{l=1}^{N} X_l \cos (2\pi t_i f_l) + Y_l \sin(2\pi t_i f_l)
\end{equation}
where one can easily notice that $X_l$ and $Y_l$ appear as weights, and the basis functions are:
\begin{align}
    F_{2l-1} (t_i) = \cos (2\pi t_i f_l) \\
    F_{2l}(t_i) = \sin (2\pi t_i f_l)
\end{align}
where $l = 1,2,...,N$. If $f_l = 1/T$ where $T$ is the total observing time span, and if the epochs are evenly spaced, then this would correspond to the discrete Fourier transform. Also, we typically truncate the set at a low frequency instead of using the entire set, using an evenly spaced set of frequencies, truncating at $N/T$, where $N$ is the number of Fourier modes. We use $N$ as a hyper-parameter in our noise model selection. The choice of the optimum number of Fourier modes is discussed in Ref. \cite{ChalumeauBabak+2022}. 
\\
The covariance matrix $\Sigma$ for Fourier coefficients $X_l$, $Y_l$ is defined by power spectral density (PSD), $S$. For our analysis, we will use the power law for fitting the red noises, which can be written as follows:
\begin{equation}
S(A, \gamma) = \frac{A^2}{12\pi^2}\left(\frac{f}{yr^{-1}}\right)^{-\gamma} yr^3   
\end{equation}
where $S(A,\gamma)$ is the power spectral density, $A$ is the amplitude with normalization at a frequency of (1 $\text{yr}^{-1}$), and $\gamma$ is the spectral index. The covariance matrix for red noise in the frequency domain (see \cite{ChalumeauBabak+2022} and references within) is given by
\begin{equation}
    \Sigma _{\kappa \alpha l\beta} = S(f_k:A_\alpha,\gamma_\alpha)\delta_{kl}\delta_{\alpha\beta}/T
\end{equation}
where $l,k = 1,2,...,N$, and $\alpha,\beta$ denote the indices of the pulsar. The Kronecker delta function has been introduced,  since we take into account the spatially uncorrelated red noise.


\subsubsection{Achromatic red noise}
Achromatic red noise (RN), also known as timing noise, is modelled in PTA data to account for the  spin irregularities in pulsars \cite{cordes1985, allessandro1995}. This noise might not be significant in MSPs compared to younger pulsars but it can be detected with data over a long baseline (e.g. \cite{alam2021,GonchorovReardon+2021}). This is the observing frequency-independent noise originating from the pulsar. We model achromatic red noise using the power law described above for our analysis.


\subsubsection{Chromatic red noise}
There are delays in ToAs due to the interaction of pulse signals with matter along the path of propagation, such as the ionized interstellar medium (IISM), the ionosphere of the Earth, and the interplanetary medium. Delays in such signals are observing-frequency dependent in nature. One such dominating effect is due to the dispersion, which causes the frequency-dependent delay in the arrival time of pulses. The delay in ToAs due to the DM is related to the observing frequency according to  $\Delta t^{\text{DM}} \propto \nu^{-2}$, where $\nu$ is the observing frequency, and $DM$ is the dispersion measure \cite{LorimerKramer2004}. The timing model accounts for this effect by considering its value at reference epoch along with its first (\texttt{DM1}) and second derivatives (\texttt{DM2}). However, turbulence and inhomogeneity in the IISM coupled with the relative motion of the earth, pulsar and the IISM, may induce an additional time-correlated red noise due to these DM variations (DMv), which depends on the observing frequency \cite{you2007,keith2013}. Another such effect is the delay due to the scattering variations (Sv) caused by the signal's multi-path propagation in IISM due to refraction and diffraction, which occurs when the radio pulses from a pulsar pass through the interstellar medium (ISM), leading to delay, broadening, and other distortion of the pulses \cite{LorimerKramer2004}. The delay due to the scattering is given by $\Delta t^{\text{SC}} \propto \nu^{-4}$. It is crucial to have multi-band observations to disentangle the chromatic components of the red noise (see, for e.g. Ref.~\cite{caballero2016}). 

For the covariance matrix $F^{\text{chrom}}_i$ of chromatic noise, we use the same formula as that used for  red noise, with the additional dependency of induced ToA delays on the observing frequency, as given below:
\begin{equation}
    F^{\text{chrom}}_i = F_i\times\left(\frac{\nu_i}{1.4GHz}\right)^{-\chi}
\end{equation}
where $F_i$ is the Fourier transform of the time-domain red noise signal and contains the incomplete sine and cosine functions, $\nu_i$ is observing frequency, and $\chi$ is the chromatic index, which is 0, 2, and 4 for RN, DMv, and Sv, respectively.
 Apart from DMv and Sv, we also use the ``Free Chromatic Noise'' model (FCN), which has the chromatic index ($\chi_{FCN}$) as an additional free parameter along with the amplitude and spectral index (see Refs.~\cite{GoncharovShannon+2021,ChalumeauBabak+2022}). This model is used as a diagnostic tool for our selected noise models, where we fit for the ($\chi_{FCN}$) to look for the presence of achromatic and chromatic red noise. 

