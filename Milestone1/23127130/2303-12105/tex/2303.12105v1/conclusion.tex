\section{Conclusions}
\label{sec:conclusion}

We have used the InPTA DR1 data set to carry out noise analysis for individual pulsars. Using Bayesian inference, we have chosen the most optimum noise models for each pulsar in our data set (see Table \ref{model_select}). We have also estimated the optimum number of Fourier modes for the red noise analysis for each pulsar. Even with a relatively modest timing baseline of 3.5 years, 8 out of 14 pulsars show a clear presence of red noise. Finally, given the unique low-frequency coverage of the InPTA data set, we were able to constrain the DM noise for eight pulsars while also detecting scatter-broadening variations in four pulsars (see Table \ref{parameter_values}). While our results are broadly in agreement with the other PTAs, we would like to note the well-constrained DM and scatter-broadening variations, even with a short timing baseline. Our results seem to deviate from those obtained for other PTAs for pulsars with either a short timing baseline or a gap in observations. The most noteworthy result from our analysis was obtained for PSR J1939+2134, where we found the residual chromatic noise with a $\chi \sim$ 2.86, with two orders larger amplitude than scattering variation with $\chi_{Sv}$ = 4, providing an indication towards scattering index that doesn't agree with that expected for a Kolmogorov scattering medium. 
Also for PSR J1909$-$3744, we find no significant scattering variations in profiles that are present in other PTAs noise models and found using the FCN model that residual DMv was masquerading as scattering variations. This exemplifies the pivotal role  of InPTA data towards modelling the noise budget of pulsars in the IPTA data set, especially in the cases of pulsars whose noise budget is dominated by variations in the delays caused by the ISM. 

These results bode well for the planned single pulsar noise and timing analysis (SPNTA) in future, which will enable the search for gravitational bursts in the InPTA data. The SPNTA would answer many questions about the utility of the low-frequency data in modelling and mitigating the ISM contribution in the overall noise budget of the IPTA pulsars. 
