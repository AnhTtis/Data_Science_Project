\section{Introduction}
\label{sec:intro}

Millisecond pulsars (MSPs) are known for their exceptional rotational stability and accuracy comparable to atomic clocks. 
Pulsar Timing Array experiments (PTAs)~\cite{FosterBacker1990} aim to detect ultra-low frequency ($\sim$ 1-100 nHz) gravitational waves (GWs) by monitoring an ensemble of MSPs distributed across the Galaxy. 
This is possible because the GWs travelling across the line of sight to a pulsar perturb the null geodesics along which the pulsar electromagnetic signals propagate, thereby modulating their times of arrival (ToAs) radio pulses.
GW signals in the PTA frequency range are typically expected to originate from orbiting supermassive black hole binaries (SMBHBs) in the inspiral phase, both as a stochastic GW background (GWB) formed by the incoherent addition of GWs from a large number of SMBHBs, and as strong individual sources standing out above this background \cite{Burke-SpolaorTaylor+2019}.
Such a GWB induces spatially correlated ToA modulations in different pulsars, characterized by the Hellings-Downs overlap reduction function \cite{HellingsDowns1983}. 
Other proposed sources of nanohertz GWs include cosmological phase transitions \cite{Hogan1986}, cosmic strings \cite{DamourVilenkin2000}, and relic GWs emanating from cosmic fluctuations in the early universe \cite{Grishchuk2005}. 

PTA experiments working towards the goal of detection of nHz GWs include the European Pulsar Timing Array \cite[EPTA:][]{KramerChampion2013}, the Parkes Pulsar Timing Array \cite[PPTA:][]{Hobbs2013}, the North American Nanohertz Observatory for Gravitational Waves \cite[NANOGrav:][]{McLaughlin2013}, the Indian Pulsar Timing Array \cite[InPTA:][]{JoshiArumugasamy+2018}, and emerging PTAs such as the Chinese Pulsar Timing Array \cite[CPTA:][]{Lee2016}, and the MeerTime Pulsar Timing Array \cite{SpiewakBailes+2022}.
The International Pulsar Timing Array consortium \cite[IPTA:][]{HobbsArchibald+2010} aims to improve the prospects of nanohertz GW detection and post-detection science by combining the data and resources from different PTA experiments.
Over the last decade, the PTA experiments have put increasingly stringent constraints on the stochastic GWB, culminating in the recent detection of a common red noise process in multiple PTA datasets \cite{GonchorovReardon+2021,ChalumeauBabak+2022,ArzoumanianBaker+2020,AntoniadisArzoumanian+2022}.


The InPTA experiment~\cite{JoshiGopu+2022} aims to use the upgraded Giant Metre-wave Radio Telescope~\cite[uGMRT:][]{GuptaAjithkumar+2017} to complement the international PTA efforts via low-frequency observations of PTA pulsars.
The uGMRT observations significantly improve the prospects of characterizing the interstellar medium effects, such as dispersion measure (DM)\footnote{Integrated free electron density along the line of sight to the pulsar.} and scatter-broadening variations, which are the strongest at low frequencies \cite{KrishnakumarManoharan+2021}.
The recently published InPTA Data Release 1 \cite[InPTA DR1:][]{TarafdarNobleson+2022} provided ToA measurements, timing analysis, and the characterization of DM variations for 14 pulsars over a time span of 3.5 years, estimated using both the traditional narrowband method \cite{Taylor1992} and the more recent wideband method \cite{PennucciDemorestRansom2014,Pennucci2019}.

The intrinsic wander of the rotation rate of the constituent pulsars, the variations in DM and scatter-broadening, as well as  the instrumental noise of radio telescopes are often covariant with the slowly varying GW signature in the data and act as sources of  chromatic and achromatic noise. The detection and characterization of GWs are strongly affected by the faithfulness of noise models and can be highly dependent on custom noise modelling for each pulsar \cite{caballero2016,lentati2016,ArzoumanianBaker+2020,chen2021,ChalumeauBabak+2022}.
Characterizing these single pulsar noise processes, which are uncorrelated across the constituent pulsars, is a crucial first step for extracting the weak GW signal, which is otherwise correlated across pulsars \cite{GonchorovReardon+2021,ChalumeauBabak+2022,ArzoumanianBaker+2020,AntoniadisArzoumanian+2022}.


This work presents the single pulsar noise analysis (SPNA) of the 14 pulsars present in the InPTA DR1 using the \enterprise{} package \cite{EllisVallisneri+2019}.
We perform Bayesian model selection among a finite set of noise models for each pulsar based on the Bayes factors estimated using the  \dynesty{} package \cite{Speagle2020}, which implements the dynamic nested sampling algorithm \cite{nestedsampling}. 
 Finally, we perform parameter estimation for the preferred noise model using \ptmcmc{}~\cite{EllisvanHaasteren2017}. We note that corresponding noise analyses have also been carried out with EPTA DR2 for six pulsars~\cite{ChalumeauBabak+2022} and the PPTA DR2 dataset for 26 pulsars~\cite{GonchorovReardon+2021}, and we shall also do a comparison with their results for the same  pulsar as appropriate.

The rest of this article is organized  as follows: Section~\ref{sec:inpta-dr1} briefly describes the InPTA DR1. 
Section~\ref{sec:noise-models} discusses the various noise sources incorporated into our analysis. 
Section~\ref{sec:anl-tech} discusses the Bayesian analysis methodology for noise model selection and parameter estimation.
Section~\ref{sec:discussion} discusses the noise modelling results for each pulsar.
We present our conclusions in Section~\ref{sec:conclusion}.