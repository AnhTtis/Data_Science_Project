
%% bare_conf.tex
%% V1.4b
%% 2015/08/26
%% by Michael Shell
%% See:
%% http://www.michaelshell.org/
%% for current contact information.
%%
%% This is a skeleton file demonstrating the use of IEEEtran.cls
%% (requires IEEEtran.cls version 1.8b or later) with an IEEE
%% conference paper.
%%
%% Support sites:
%% http://www.michaelshell.org/tex/ieeetran/
%% http://www.ctan.org/pkg/ieeetran
%% and
%% http://www.ieee.org/

%%*************************************************************************
%% Legal Notice:
%% This code is offered as-is without any warranty either expressed or
%% implied; without even the implied warranty of MERCHANTABILITY or
%% FITNESS FOR A PARTICULAR PURPOSE! 
%% User assumes all risk.
%% In no event shall the IEEE or any contributor to this code be liable for
%% any damages or losses, including, but not limited to, incidental,
%% consequential, or any other damages, resulting from the use or misuse
%% of any information contained here.
%%
%% All comments are the opinions of their respective authors and are not
%% necessarily endorsed by the IEEE.
%%
%% This work is distributed under the LaTeX Project Public License (LPPL)
%% ( http://www.latex-project.org/ ) version 1.3, and may be freely used,
%% distributed and modified. A copy of the LPPL, version 1.3, is included
%% in the base LaTeX documentation of all distributions of LaTeX released
%% 2003/12/01 or later.
%% Retain all contribution notices and credits.
%% ** Modified files should be clearly indicated as such, including  **
%% ** renaming them and changing author support contact information. **
%%*************************************************************************


% *** Authors should verify (and, if needed, correct) their LaTeX system  ***
% *** with the testflow diagnostic prior to trusting their LaTeX platform ***
% *** with production work. The IEEE's font choices and paper sizes can   ***
% *** trigger bugs that do not appear when using other class files.       ***                          ***
% The testflow support page is at:
% http://www.michaelshell.org/tex/testflow/



\documentclass[conference]{IEEEtran}
% Some Computer Society conferences also require the compsoc mode option,
% but others use the standard conference format.
%
% If IEEEtran.cls has not been installed into the LaTeX system files,
% manually specify the path to it like:
% \documentclass[conference]{../sty/IEEEtran}





% Some very useful LaTeX packages include:
% (uncomment the ones you want to load)


% *** MISC UTILITY PACKAGES ***
%
%\usepackage{ifpdf}
% Heiko Oberdiek's ifpdf.sty is very useful if you need conditional
% compilation based on whether the output is pdf or dvi.
% usage:
% \ifpdf
%   % pdf code
% \else
%   % dvi code
% \fi
% The latest version of ifpdf.sty can be obtained from:
% http://www.ctan.org/pkg/ifpdf
% Also, note that IEEEtran.cls V1.7 and later provides a builtin
% \ifCLASSINFOpdf conditional that works the same way.
% When switching from latex to pdflatex and vice-versa, the compiler may
% have to be run twice to clear warning/error messages.






% *** CITATION PACKAGES ***
%
%\usepackage{cite}
% cite.sty was written by Donald Arseneau
% V1.6 and later of IEEEtran pre-defines the format of the cite.sty package
% \cite{} output to follow that of the IEEE. Loading the cite package will
% result in citation numbers being automatically sorted and properly
% "compressed/ranged". e.g., [1], [9], [2], [7], [5], [6] without using
% cite.sty will become [1], [2], [5]--[7], [9] using cite.sty. cite.sty's
% \cite will automatically add leading space, if needed. Use cite.sty's
% noadjust option (cite.sty V3.8 and later) if you want to turn this off
% such as if a citation ever needs to be enclosed in parenthesis.
% cite.sty is already installed on most LaTeX systems. Be sure and use
% version 5.0 (2009-03-20) and later if using hyperref.sty.
% The latest version can be obtained at:
% http://www.ctan.org/pkg/cite
% The documentation is contained in the cite.sty file itself.






% *** GRAPHICS RELATED PACKAGES ***
%
\ifCLASSINFOpdf
  % \usepackage[pdftex]{graphicx}
  % declare the path(s) where your graphic files are
  % \graphicspath{{../pdf/}{../jpeg/}}
  % and their extensions so you won't have to specify these with
  % every instance of \includegraphics
  % \DeclareGraphicsExtensions{.pdf,.jpeg,.png}
\else
  % or other class option (dvipsone, dvipdf, if not using dvips). graphicx
  % will default to the driver specified in the system graphics.cfg if no
  % driver is specified.
  % \usepackage[dvips]{graphicx}
  % declare the path(s) where your graphic files are
  % \graphicspath{{../eps/}}
  % and their extensions so you won't have to specify these with
  % every instance of \includegraphics
  % \DeclareGraphicsExtensions{.eps}
\fi
% graphicx was written by David Carlisle and Sebastian Rahtz. It is
% required if you want graphics, photos, etc. graphicx.sty is already
% installed on most LaTeX systems. The latest version and documentation
% can be obtained at: 
% http://www.ctan.org/pkg/graphicx
% Another good source of documentation is "Using Imported Graphics in
% LaTeX2e" by Keith Reckdahl which can be found at:
% http://www.ctan.org/pkg/epslatex
%
% latex, and pdflatex in dvi mode, support graphics in encapsulated
% postscript (.eps) format. pdflatex in pdf mode supports graphics
% in .pdf, .jpeg, .png and .mps (metapost) formats. Users should ensure
% that all non-photo figures use a vector format (.eps, .pdf, .mps) and
% not a bitmapped formats (.jpeg, .png). The IEEE frowns on bitmapped formats
% which can result in "jaggedy"/blurry rendering of lines and letters as
% well as large increases in file sizes.
%
% You can find documentation about the pdfTeX application at:
% http://www.tug.org/applications/pdftex





% *** MATH PACKAGES ***
%
%\usepackage{amsmath}
% A popular package from the American Mathematical Society that provides
% many useful and powerful commands for dealing with mathematics.
%
% Note that the amsmath package sets \interdisplaylinepenalty to 10000
% thus preventing page breaks from occurring within multiline equations. Use:
%\interdisplaylinepenalty=2500
% after loading amsmath to restore such page breaks as IEEEtran.cls normally
% does. amsmath.sty is already installed on most LaTeX systems. The latest
% version and documentation can be obtained at:
% http://www.ctan.org/pkg/amsmath





% *** SPECIALIZED LIST PACKAGES ***
%
%\usepackage{algorithmic}
% algorithmic.sty was written by Peter Williams and Rogerio Brito.
% This package provides an algorithmic environment fo describing algorithms.
% You can use the algorithmic environment in-text or within a figure
% environment to provide for a floating algorithm. Do NOT use the algorithm
% floating environment provided by algorithm.sty (by the same authors) or
% algorithm2e.sty (by Christophe Fiorio) as the IEEE does not use dedicated
% algorithm float types and packages that provide these will not provide
% correct IEEE style captions. The latest version and documentation of
% algorithmic.sty can be obtained at:
% http://www.ctan.org/pkg/algorithms
% Also of interest may be the (relatively newer and more customizable)
% algorithmicx.sty package by Szasz Janos:
% http://www.ctan.org/pkg/algorithmicx




% *** ALIGNMENT PACKAGES ***
%
%\usepackage{array}
% Frank Mittelbach's and David Carlisle's array.sty patches and improves
% the standard LaTeX2e array and tabular environments to provide better
% appearance and additional user controls. As the default LaTeX2e table
% generation code is lacking to the point of almost being broken with
% respect to the quality of the end results, all users are strongly
% advised to use an enhanced (at the very least that provided by array.sty)
% set of table tools. array.sty is already installed on most systems. The
% latest version and documentation can be obtained at:
% http://www.ctan.org/pkg/array


% IEEEtran contains the IEEEeqnarray family of commands that can be used to
% generate multiline equations as well as matrices, tables, etc., of high
% quality.




% *** SUBFIGURE PACKAGES ***
%\ifCLASSOPTIONcompsoc
%  \usepackage[caption=false,font=normalsize,labelfont=sf,textfont=sf]{subfig}
%\else
%  \usepackage[caption=false,font=footnotesize]{subfig}
%\fi
% subfig.sty, written by Steven Douglas Cochran, is the modern replacement
% for subfigure.sty, the latter of which is no longer maintained and is
% incompatible with some LaTeX packages including fixltx2e. However,
% subfig.sty requires and automatically loads Axel Sommerfeldt's caption.sty
% which will override IEEEtran.cls' handling of captions and this will result
% in non-IEEE style figure/table captions. To prevent this problem, be sure
% and invoke subfig.sty's "caption=false" package option (available since
% subfig.sty version 1.3, 2005/06/28) as this is will preserve IEEEtran.cls
% handling of captions.
% Note that the Computer Society format requires a larger sans serif font
% than the serif footnote size font used in traditional IEEE formatting
% and thus the need to invoke different subfig.sty package options depending
% on whether compsoc mode has been enabled.
%
% The latest version and documentation of subfig.sty can be obtained at:
% http://www.ctan.org/pkg/subfig




% *** FLOAT PACKAGES ***
%
%\usepackage{fixltx2e}
% fixltx2e, the successor to the earlier fix2col.sty, was written by
% Frank Mittelbach and David Carlisle. This package corrects a few problems
% in the LaTeX2e kernel, the most notable of which is that in current
% LaTeX2e releases, the ordering of single and double column floats is not
% guaranteed to be preserved. Thus, an unpatched LaTeX2e can allow a
% single column figure to be placed prior to an earlier double column
% figure.
% Be aware that LaTeX2e kernels dated 2015 and later have fixltx2e.sty's
% corrections already built into the system in which case a warning will
% be issued if an attempt is made to load fixltx2e.sty as it is no longer
% needed.
% The latest version and documentation can be found at:
% http://www.ctan.org/pkg/fixltx2e


%\usepackage{stfloats}
% stfloats.sty was written by Sigitas Tolusis. This package gives LaTeX2e
% the ability to do double column floats at the bottom of the page as well
% as the top. (e.g., "\begin{figure*}[!b]" is not normally possible in
% LaTeX2e). It also provides a command:
%\fnbelowfloat
% to enable the placement of footnotes below bottom floats (the standard
% LaTeX2e kernel puts them above bottom floats). This is an invasive package
% which rewrites many portions of the LaTeX2e float routines. It may not work
% with other packages that modify the LaTeX2e float routines. The latest
% version and documentation can be obtained at:
% http://www.ctan.org/pkg/stfloats
% Do not use the stfloats baselinefloat ability as the IEEE does not allow
% \baselineskip to stretch. Authors submitting work to the IEEE should note
% that the IEEE rarely uses double column equations and that authors should try
% to avoid such use. Do not be tempted to use the cuted.sty or midfloat.sty
% packages (also by Sigitas Tolusis) as the IEEE does not format its papers in
% such ways.
% Do not attempt to use stfloats with fixltx2e as they are incompatible.
% Instead, use Morten Hogholm'a dblfloatfix which combines the features
% of both fixltx2e and stfloats:
%
% \usepackage{dblfloatfix}
% The latest version can be found at:
% http://www.ctan.org/pkg/dblfloatfix




% *** PDF, URL AND HYPERLINK PACKAGES ***
%
%\usepackage{url}
% url.sty was written by Donald Arseneau. It provides better support for
% handling and breaking URLs. url.sty is already installed on most LaTeX
% systems. The latest version and documentation can be obtained at:
% http://www.ctan.org/pkg/url
% Basically, \url{my_url_here}.




% *** Do not adjust lengths that control margins, column widths, etc. ***
% *** Do not use packages that alter fonts (such as pslatex).         ***
% There should be no need to do such things with IEEEtran.cls V1.6 and later.
% (Unless specifically asked to do so by the journal or conference you plan
% to submit to, of course. )




% correct bad hyphenation here
\hyphenation{op-tical net-works semi-conduc-tor}

\usepackage{xcolor}


\begin{document}

\definecolor{FlosColor}{rgb}{0.1,0.8,0.4}
\newcommand{\florian}[1]{\textsf{\textbf{\textcolor{FlosColor}{[TODO: #1]}}}}
%
% paper title
% Titles are generally capitalized except for words such as a, an, and, as,
% at, but, by, for, in, nor, of, on, or, the, to and up, which are usually
% not capitalized unless they are the first or last word of the title.
% Linebreaks \\ can be used within to get better formatting as desired.
% Do not put math or special symbols in the title.
\title{Self-Reflection as a Tool to Foster Profound Sustainable Consumption Decisions}


% author names and affiliations
% use a multiple column layout for up to three different
% affiliations
\author{\IEEEauthorblockN{Florian Bemmann}
\IEEEauthorblockA{LMU Munich\\
Munich, Germany\\
florian.bemmann@ifi.lmu.de}
\and
\IEEEauthorblockN{Heinrich Hussmann}
\IEEEauthorblockA{LMU Munich\\
Munich, Germany\\
hussmann@ifi.lmu.de}
}

% conference papers do not typically use \thanks and THIS command
% is locked out in conference mode. If really needed, such as for
% the acknowledgment of grants, issue a \IEEEoverridecommandlockouts
% after \documentclass

% for over three affiliations, or if they all won't fit within the width
% of the page, use this alternative format:
% 
%\author{\IEEEauthorblockN{Michael Shell\IEEEauthorrefmark{1},
%Homer Simpson\IEEEauthorrefmark{2},
%James Kirk\IEEEauthorrefmark{3}, 
%Montgomery Scott\IEEEauthorrefmark{3} and
%Eldon Tyrell\IEEEauthorrefmark{4}}
%\IEEEauthorblockA{\IEEEauthorrefmark{1}School of Electrical and Computer Engineering\\
%Georgia Institute of Technology,
%Atlanta, Georgia 30332--0250\\ Email: see http://www.michaelshell.org/contact.html}
%\IEEEauthorblockA{\IEEEauthorrefmark{2}Twentieth Century Fox, Springfield, USA\\
%Email: homer@thesimpsons.com}
%\IEEEauthorblockA{\IEEEauthorrefmark{3}Starfleet Academy, San Francisco, California 96678-2391\\
%Telephone: (800) 555--1212, Fax: (888) 555--1212}
%\IEEEauthorblockA{\IEEEauthorrefmark{4}Tyrell Inc., 123 Replicant Street, Los Angeles, California 90210--4321}}




% use for special paper notices
%\IEEEspecialpapernotice{(Invited Paper)}




% make the title area
\maketitle

% As a general rule, do not put math, special symbols or citations
% in the abstract
\begin{abstract}
 The production of goods we buy on a daily basis accounts for a large portion of greenhouse gas emissions. Although consumers have the power to influence industries' behavior through their demand, making sustainable purchases is challenging.
Current ICT systems supporting sustainable shopping decisions are not established in consumers’ daily lifes. Shopping decisions are made on a complex set of criteria, thus classical persuasive approaches, like recommender-systems, might not be suitable. This work compiles the state of research on ICT supporting sustainable consumption, outlines unsolved challenges, and finally presents a novel concept: a system based on self-reflection instead of classical persuasive approaches, like recommender-systems. Self-reflection provokes revising individual behaviour and decisions, instead of presenting instructions. Combined with additional information on e.g., decision impact, people could learn how to make more sustainable decisions independently.
We envision the deployment of such a system, fostering a change towards more sustainable industries to combat climate change.
%\florian{Die Key Konzept einbringen damit man gleich sieht wie es sich abhebt?}
\end{abstract}

% no keywords




% For peer review papers, you can put extra information on the cover
% page as needed:
% \ifCLASSOPTIONpeerreview
% \begin{center} \bfseries EDICS Category: 3-BBND \end{center}
% \fi
%
% For peerreview papers, this IEEEtran command inserts a page break and
% creates the second title. It will be ignored for other modes.
\IEEEpeerreviewmaketitle

\section{Introduction}
% Production of daily goods makes up a large part of CO2 emissions
% Consumers have the power to force industries into a more sustainable direction by their decisions
The fabrication of goods we consume every day accounts for a large share of our personal, environmental impact \cite{tukker2006environmental,berners2012relative}. Although recently there is a trend towards sustainable production chains, manufacturers strive to keep their production costs low in order to compete. Fortunately, consumers have the power to lead companies to sustainable products and processes, by shifting demand on the market.
% Barriers are lack of time for research, price, lack of information about product, …
This is easier said than done: Among people with an attitude towards sustainable consumption, their behavior often is not \cite{vermeir2004sustainable, hughner2007organic, young2010sustainable}. The barriers to sustainable consumption are mainly informational, despite higher prices being the most relevant barrier:  lack of information about the impact of a product/company, skepticism of certifications, unclear understanding of certifications, a lack of (perceived) availability and insufficient marketing \cite{young2010sustainable,hughner2007organic,aschemann2014elaborating}.
%As barriers to sustainable consumption research identified the lack of time for research, high prices, lack of information about company's / product's impact resp. skepticism of certifications and understanding them, lack of (perceived) availability resp. insufficient marketing, satisfaction with current products, and cosmetic defects \cite{young2010sustainable,hughner2007organic,aschemann2014elaborating}.

% List some current approaches and limitation in one sentence each
There exist various approaches aiming to overcome the attitude-behavior gap in the domain of sustainable consumption: 
In-shop decision support systems provide information on specific products, e.g. food miles \cite{kalnikaite2011nudge} or consumer-generated environmental impact information \cite{tomlinson2008prototyping, montiel2017mobile}. Recommender-systems were developed to recommend sustainable products \cite{tomkins2018recommender}. However, shopping decisions are subject to complex constraints like family dynamics and daily routines \cite{clear2015supporting,clear2016bearing}, e.g. the distance to the shop, preferences of all family members and required cooking effort. Related systems have to fit into the broader context of life in order to be used. Thus, recommender-systems that decide based on a limited set of criteria are no ideal solution \cite{kalnikaite2011nudge}.
% people have to incorporate all criteria relevant for them individually in their decisoin
Instead, people should be enabled to decide in a sustainable manner on their own.

Designing for the provision of information on specific products accompanies the problem that such concrete information are not always available \cite{montiel2017mobile}. Furthermore, systems tailored to one specific shop are limited in their applicability, and with recommender-systems users face the difficulty of integrating their decision into everyday life \cite{kalnikaite2011nudge}. 
%For example when buying food for the family, among others the distance to the shop, preferences of all family members and required cooking effort constrain the shopping decisions.



%  List some work where self-reflection has proven to be appropriate
%Also work with related aims shows that self-reflection is promising:
% TODO darf ich einen prerint zitieren?
%EcoPanel, presented by Katzeff et al., is a dashboard that visualizes organic food purchases \cite{katzeff2019encouraging}. Their study showed that the stimulated reflection that was triggered by their visualization lead to an increased purchase of organic food.
%Ganglbauer et al. study the potential of self-reflection for food waste reduction \cite{ganglbauer2015and}. They triggered self-reflection by manual diary keeping and enrichment of the entries with categories and reasons.

In this work we propose a system supporting sustainable consumption, based on solutions proposed in existing research and identified drawbacks: Instead of recommending concrete products with quantitative impact data, people should be enabled to make profound decisions themselves. We propose a retrospective analysis of shopping behavior, by self-reflecting on the bought goods in the context of sustainability aspects. In an interactive manner, people should become informed about sustainability aspects regarding their bought product types, and thereby learn to incorporate these insights into their future shopping decisions. Related approaches have been presented by Katzeff et al. who evaluated their dashboard on organic food purchases \cite{katzeff2019encouraging}, a study of self-reflection in the domain of food waste by Ganglbauer et al. \cite{ganglbauer2015and}, or in the Ecofriends app where users reflect on seasonality of foods \cite{tholander2012but}. 

%We hypothesize that consumption is a domain where self-reflection is the best way to support behavior change.




\section{Application Concept}
In this section we present a concept for a novel self-reflection tool that supports sustainable consumption. It arose from existing research in the fields of decision support and information presentation for sustainable behavior, and self-reflection. Our concept is based on the assumptions that (1) people are partially unknowing about how to buy sustainably, (2) manufacturers will produce more sustainably if consumers buy more sustainable products and (3) information on the impact of general product types is available. 

In the following, we describe the key concepts of our system.

\subsection{Enable Profound Decisions instead of Recommending Actions}
To fit into daily life and its constraints, we argue against recommender systems in the domain of consumption decisions.
% recommender systems are bad
%Shopping decisions are subject to complex constraints like family dynamics and daily routines \cite{clear2015supporting,clear2016bearing}. Related systems have to fit into the broader context of life in order to be used. Thus, recommender-systems that decide based on a limited set of criteria are no ideal solution \cite{kalnikaite2011nudge}.
% people have to incorporate all criteria relevant for them individually in their decisoin
Instead, people should be enabled to decide in a sustainable manner on their own. Shopping is a repetitive endeavour, that should be influenced by reflection over longer periods of time \cite{clear2016bearing}. The system presented will foster critical reflection of the recent purchase, giving support in the judgement of a product's environmental impact. In the long run people could be enabled to judge products by themselves and might decide more sustainably.

\subsection{Reflection-On-Action}
% Use the tool after, not during the shopping
% e.g. CartScanner showed that Smartphones are unpraktisch in the supermarkt
To allow for a frequent usage, the system should be independent of the supermarket \cite{clear2016bearing}. However, the usage of one's own smartphone is no ideal solution, as it can be overwhelming and lead to poor decisions in the hurry of the supermarket \cite{kalnikaite2011nudge}. Thus we argue for reflection on-action \cite{ploderer2014}: the self-reflection does not take place at the point of time when the decision is made, but afterwards in a more relaxed and flexible environment e.g. on a personal- or tablet computer.

\subsection{Leverage Social- and Group Dynamics}
% Einkäufe betreffen die ganze Familie - also muss diese auch hier einbezogen werden
Families are the main source of constraints for shopping decisions. Hence they should be involved in a sustainability support system \cite{clear2015supporting}. Family members could additionally contribute to the learning process: Relying on an objective information basis about (un-)sustainable characteristics of general product types, interpretation and derivation of consequences concerning the family's cart could be topic of a social process. 
%with their different knowledge on product impact.

\subsection{Gamification to Trigger Learning about Product Impact}
Gamification can be used to engage people \cite{looyestyn2017does}. In our system it will engage people to learn about the sustainability aspects of their bought products. Thereby, they will understand over time why some product is superior according to a certain criteria set to some other, without prescribing a definition of what is good or bad.

\subsection{Semi-Automatic Purchase Logging}
Manual diary keeping is an important source of self-reflection \cite{ganglbauer2015and}. To lower the burden for users we argue for a semi-automatic diary keeping: the bought products will be captured from the receipt, by taking a photo or through electronic receipts \cite{froehlich2009sensing}. The self-reflection then takes place during data enrichment: users will be engaged to add additional information (e.g. origin country or sustainable impact factor) and be able to correct errors.

\section{How this Work will be presented at the Conference}
Depending on the virtual presentation format, we will present a digital prototype mockup alongside the poster. Visitors can empathize in a typical user story and try the prototype. Screens of the future system will be sketched so that participants can navigate through the system for the given user story. We thereby hope to gain valuable feedback about the idea in general, its concepts, and the prototypical implementation.




% conference papers do not normally have an appendix


% trigger a \newpage just before the given reference
% number - used to balance the columns on the last page
% adjust value as needed - may need to be readjusted if
% the document is modified later
%\IEEEtriggeratref{8}
% The "triggered" command can be changed if desired:
%\IEEEtriggercmd{\enlargethispage{-5in}}

% references section

% can use a bibliography generated by BibTeX as a .bbl file
% BibTeX documentation can be easily obtained at:
% http://mirror.ctan.org/biblio/bibtex/contrib/doc/
% The IEEEtran BibTeX style support page is at:
% http://www.michaelshell.org/tex/ieeetran/bibtex/
%\bibliographystyle{IEEEtran}
% argument is your BibTeX string definitions and bibliography database(s)
%\bibliography{IEEEabrv,../bib/paper}
%
% <OR> manually copy in the resultant .bbl file
% set second argument of \begin to the number of references
% (used to reserve space for the reference number labels box)
\bibliographystyle{IEEEtran}
\bibliography{sample-base}
%\begin{thebibliography}{1}

%\bibitem{IEEEhowto:kopka}
%H.~Kopka and P.~W. Daly, \emph{A Guide to \LaTeX}, 3rd~ed.\hskip 1em plus
%  0.5em minus 0.4em\relax Harlow, England: Addison-Wesley, 1999.

%\end{thebibliography}




% that's all folks
\end{document}


