\section{Introduction}
\label{sec:intro}

\begin{figure*}
  \centering
  \includegraphics[scale = 0.65]{fig/general_diagram.png}
  %\setlength{\abovecaptionskip}{0.1cm}
  \caption{\projectname{} collects gaze data and text information and uses a transformer-based machine learning model to detect unknown words for ESL.}
  \label{fig:pipeline} 
\end{figure*}

Unknown words, which we defined as words that the user spends extra time thinking about, can significantly increase the reading difficulty for English as a Second Language (ESL) learners~\cite{rigg1991whole,mokhtar2012guessing}, since vocabulary knowledge is considered an essential feature of reading ability. Automatic detection of unfamiliar words can help users improve reading fluency and reading experience~\cite{hyrskykari2006eyes}. Previous works used eye movement features such as fixation duration and the number of regressions to detect unknown words, considering the strong correlation between eye movement and reading difficulty~\cite{unknown-word_hiraoka_2016, gaze-text_garain_2017, difficulties_lunte_2020}. However, all these methods are based on dedicated eye-tracking devices. Even for portable eye trackers, prices of hundreds or thousands of dollars prevent most people from using these technologies in their daily lives. 

The ubiquity of webcams makes it possible to track eye movement non-obtrusively. Researchers have used webcam-based eye-tracking methods to analyze users' reading behavior~\cite{eval-webcam_hutt_2022}, but the low precision~\cite{searchGazer_papoutsaki_2017} makes it infeasible to extract eye movement features proposed by previous works. Therefore, these reading analysis approaches cannot be directly transferred to webcam-based unknown word detection, which requires fine-grained tracking.

In this work, we detect unknown words using the webcam for ESL learners by integrating gaze and text information with the help of transformer-based language models. Our method tracks eye movement using WebGazer~\cite{papoutsaki2016webgazer} and embeds the positional information of gaze and texts with Long Short-term Memory~\cite{hochreiter1997long} (LSTM) models. In order to improve the model's performance given only noisy gaze data acquired by webcam, we leverage pre-trained language models to encode the context information for assistance. The gaze and context data are then fed to the model's classifier to predict the targeted unknown word. The F1-score of our model is 75.73\% which is higher than the text- and gaze-only model. We also conduct a user study to discover the needs of ESL learners and explore the design scope according to our method. The results show that our future design should focus more on proper nouns, multi-meaning words, and long and complex sentences, as well as exploring ways to reduce disturbance to the user's reading process through interactive design. In summary, the main contributions of this work include the following:

1) We propose \projectname{}, a novel webcam-based unknown word detecting method that leverages gaze and text information for ESL learners. The accuracy is 98.09\%,  the F1-score is 75.73\%, and the cross-user F1-score is 78.26\%.

2) We explore the design scope for ESL reading based on our method, guiding our future design direction towards proper nouns, multi-meaning words, and long and complex sentences.


%Eye movement can reflect users' cognitive state and has a strong correlation with reading\todo{need citation}. Eye movement data is widely used to analyze human reading behavior, such as the mental state of reading\todo{need citation}, reading comprehension\todo{need citation}, and reading difficulty\todo{need citation}. For language learners, eye movement data is also collected to predict difficult words to help them improve their reading proficiency\todo{need citation}. Previous works have fully explored the space of using eye trackers to assist reading. However, even for portable eye trackers, prices of hundreds or even thousands of dollars prevent most people from using these technologies in their daily lives.

%The ubiquity of webcams makes it possible to track eye movement non-obtrusively. Researchers have used webcam-based eye-tracking methods to analyze users' reading behavior\todo{need citation}, but the low precision\todo{need citation} makes it infeasible to deploy the feature-based analysis proposed by most of the previous works. Therefore, these reading analysis approaches cannot be directly transferred to the webcam-based scenario, especially some that require fine-grained tracking such as unfamiliar word detection.

%With the rapid development of artificial intelligence, it has become a trend to combine artificial intelligence and user behavior analysis. At present, there are lots of explorations on how to analyze text data to support various applications. In recent years, some work has assisted NLP tasks by introducing eye movement data\todo{need citation}. Inspired by this, we try to improve the recognition of gaze targets by introducing these NLP technologies to determine unfamiliar words and facilitate the user's reading.

%Unfamiliar words are one of the major causes of reading difficulty for language learners, especially English as a Second Language (ELS) learners\todo{need citation}.\todo{some explanations}

%In this work, we integrate eye movements and test information\todo{language model} to predict unfamiliar words facilitating ESL reading, and explore ESL learners' needs and design scope. \todo{Our method embeds gaze by xxx and extracts text features by xxx. The gaze xxx and text xxx are xxx into a xxx to predict the target word. The accuracy/f1-score of our model is xxx which is higher than the text-only model and gaze-only model.} We also conduct an interview to discover the needs of ELS learners. \todo{xxx(result of study).} In summary, the main contributions of this work are listed below:
%1) We propose a novel method that integrates eye movement and text information to predict ESL learners' unfamiliar words in reading. \todo{The accuracy is xxx and the f1-score is xxx.}
%2) We explore the design scope for ESL reading based on our method. \todo{result xxx.}