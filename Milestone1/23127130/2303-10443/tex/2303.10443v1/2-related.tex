\section{Background and Related work}
\label{sec:related}
It is time-consuming for second language learners to find the meaning of unknown words by looking up dictionaries or using translation tools. Automatically detecting unknown words can enable a variety of ways to facilitate reading. Researchers have implemented automatic unknown word detection by tracking users' clicks on a web document~\cite{web_ehara_2010} or combining text features with motion data on a smartphone~\cite{imu_higa_2022}. In addition, other works used eye movements to detect unknown words. The physiological and psychological reason behind it is that eye movements largely reflect people's tendency to pay attention when reading~\cite{rayner1998eye}. Word length, word frequency, and familiarity with the word largely explain the length of time users spend looking at it, in other words, longer fixation time~\cite{just1980theory,clifton2007eye}. Mostly related to our work, ~\cite{gaze-text_garain_2017} and~\cite{unknown-word_hiraoka_2016} leverage features from the gaze and document context to detect difficult words in reading. However, the above methods are all based on eye movement features, such as the number of fixation and regression, so dedicated eye tracking equipment is required to obtain high-precision eye movement data.

The webcam is a ubiquitous device that can track users' eye movements. Researchers apply calibration-free webcam-based eye tracking to predict reading comprehension and mind wandering in reading and achieve comparable accuracy to infrared-based eye tracking on these tasks~\cite{eval-webcam_hutt_2022}. Previous work also utilizes fixation points computed from webcam data to analyze search behaviors, and the mean error of the real-time eye tracking is 128.9 pixels~\cite{searchGazer_papoutsaki_2017}. Although webcam-based eye tracking shows the potential of modeling user behavior, the tracking precision is not enough for detecting unknown words, which requires precise locations of the gaze. 

Based on the strong connection between eye movement and attention in reading, many works introduce gaze to assist model training in Natural Language Processing (NLP) area~\cite{sentiment_long_21,barrett-etal-2018-sequence,ner_Hollenstein_2019}. Since pre-trained language models~\cite{devlin2019bert,liu2019roberta} contain rich text information for downstream tasks, we leverage them to compensate for the inaccuracy of the webcam-based eye-tracking method. By combining gaze and text information, we can accurately detect unknown words for users based on a webcam, which can enable various interactions assisting ESL reading.


% \subsection{Eye Movement in Reading}
% Eye movements can reflect the location of the user's attention. Although it has been shown that attention can be shifted even without eye movements~\cite{posner1980orienting} , shifting attention through eye movements is necessary when dealing with more complex stimuli, such as reading text. During reading, eye movements and attention are more closely bonded, through which we can assume that eye movements reflect the majority of attention during reading~\cite{rayner1998eye}.

% The key role of eye movements in reading text was summarized in detail in Rayner's 1998 review work. Due to the physiology of the eye and the way the brain processes visual information, there are two main types of eye movements in reading, saccades and fixation. Saccades have a short duration of about 30-50 ms, during each time of saccade the eye moves forward 7-9 characters. This is how people read content line by line. Studies have shown that our brain does not acquire information during saccades~\cite{uttal1968recognition}. Only during fixation does our brain acquire information, and each fixation lasts approximately 200-300ms~\cite{rayner1998eye}.

% The occurrence of saccades during reading is related to the structure of our eyes. While reading, people unconsciously place what they are reading in the center of their visual field, when information is most efficiently acquired. This is due to the fact that our visual field can be divided into: foveal, parafoveal, and peripheral. Our vision acuity is best at the foveal. And if a normal-sized word is presented in the parafoveal vision, it will be recognized more quickly and accurately when a saccade is performed~\cite{jacobs1986eye,jacobs1987spatial,rayner1981masking}.

%  The difficulty of the text affects the duration of fixation. People usually stay shorter when reading simple, shorter-length words and longer when reading long, difficult words. If a word is easy to recognize and understand, the fixation span is shorter~\cite{clifton2007eye}. One of the most studied effects is the effect of word frequency on fixation time. It has been observed that readers tend to look longer at infrequently occurring words or at long words~\cite{rayner1977visual}. According to existing research, nearly 70\% of the variation in mean fixation duration can be explained by word length and word frequency~\cite{just1980theory}.

% Taken together, we believe that eye movements largely reflect people's tendency to pay attention when reading. To a certain extent, we should be able to infer a person's intention when reading from eye movements. Therefore, it is natural to consider the use of eye movements to track attention and assist users in reading.

% \subsection{Eye Tracking Method facilitating Reading}

% Since eye movement is strongly associated with reading behavior, eye movement data has been used to facilitate reading. Gaze has been used to locate the paragraph users are reading and help automatically add notes along with it~\cite{note-taking_khan_2022}. Some researchers also leverage eye movement features such as fixation, saccade and regression to detect reading states, for example, distraction, to improve reading efficiency~\cite{smartglass_yuan_2020,distraction_vidhya_2011}. For language learners,  the fixation which is most correlated with reading difficulty is used to recognize words that trouble users and provide visual hints for them~\cite{difficulties_lunte_2020}. Eye gaze is also used to detect unknown words~\cite{unknown-word_hiraoka_2016} and summarize reading session~\cite{summary_dubey_2020} combing with text features (e.g., the word rarity and word length) to aid ESL reading. However, the above methods are all based on eye movement features, such as the number of fixation and regression, so dedicated eye tracking equipment is required to obtain high-precision eye movement data.

% Webcam is a ubiquitous device that can be used to track users' eye movement. Researchers have evaluated calibration-free webcam-based eye tracking on predicting mind wandering in reading and reading comprehension. Results show that the webcam-based method can achieve comparable accuracy to infrared-based eye tracking on these tasks~\cite{eval-webcam_hutt_2022}. Previous work utilizes fixation points computed from webcam data to analyze search behaviors and the mean error of the real-time eye tracking is 128.9 pixels~\cite{searchGazer_papoutsaki_2017}. Although webcam-based eye tracking shows the potential of modeling user behavior, the tracking precision is not enough for a reading task like detecting unknown words which required accurate locations of the gaze. 

% Based on the close connection between eye movement and reading behavior, there are many works that introduce gaze to assist model training in the natural language processing (NLP) area~\cite{sentiment_long_21,barrett-etal-2018-sequence,ner_Hollenstein_2019}. Correspondingly, we introduce \todo{xxx methods} in natural language processing to compensate for the inaccuracy of the webcam-based eye-tracking method. By combining eye movement data and text information, we can accurately infer unfamiliar words for users based on a webcam, which can enable various interactions facilitating ELS reading.
