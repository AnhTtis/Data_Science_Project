
\documentclass[sigconf]{acmart}
% \documentclass[manuscript,review,anonymous]{acmart}

%% Fonts used in the template cannot be substituted; margin 
%% adjustments are not allowed.
%%
%% \BibTeX command to typeset BibTeX logo in the docs
\AtBeginDocument{%
  \providecommand\BibTeX{{%
    \normalfont B\kern-0.5em{\scshape i\kern-0.25em b}\kern-0.8em\TeX}}}

\usepackage{stfloats}
\usepackage{graphics}
\usepackage{url}
\usepackage{hyperref}
% \usepackage[normalem]{ulem}

\newcommand{\projectname}{GazeReader}

\copyrightyear{2023}
\acmYear{2023}
\setcopyright{rightsretained}
\acmConference[CHI EA '23]{Extended Abstracts of the 2023 CHI Conference on Human Factors in Computing Systems}{April 23--28, 2023}{Hamburg, Germany}
\acmBooktitle{Extended Abstracts of the 2023 CHI Conference on Human Factors in Computing Systems (CHI EA '23), April 23--28, 2023, Hamburg, Germany}\acmDOI{10.1145/3544549.3585790}
\acmISBN{978-1-4503-9422-2/23/04}

% %% Rights management information.  This information is sent to you
% %% when you complete the rights form.  These commands have SAMPLE
% %% values in them; it is your responsibility as an author to replace
% %% the commands and values with those provided to you when you
% %% complete the rights form.
% \setcopyright{acmcopyright}
% \copyrightyear{2023}
% \acmYear{2023}
% \acmDOI{XXXXXXX.XXXXXXX}

% %% These commands are for a PROCEEDINGS abstract or paper.
% \acmConference[CHI '23]{Make sure to enter the correct
%   conference title from your rights confirmation emai}{June 03--05,
%   2018}{Hamburg, Germany}
% %
% %  Uncomment \acmBooktitle if th title of the proceedings is different
% %  from ``Proceedings of ...''!
% %
% \acmBooktitle{CHI '23: ACM CHI Conference on Human Factors in Computing Systems,
%  April 23--28, 2023, Hamburg, Germany}
% \acmPrice{15.00}
% \acmISBN{978-1-4503-XXXX-X/18/06}

% \newcommand\todo[1]{\textit{\textcolor{blue}{[todo] #1}}}
% \newcommand\yuqi[1]{\textit{\textcolor{olive}{[yuqi] #1}}}
% \newcommand\bowen[1]{\textit{\textcolor{purple}{[bowen] #1}}}
% \newcommand\jiexin[1]{\textit{\textcolor{orange}{[jiexin] #1}}}

\begin{document}

%%
%% The "title" command has an optional parameter,
%% allowing the author to define a "short title" to be used in page headers.
\title[\projectname{}: Detecting Unknown Word Using Webcam ...]{\projectname{}: Detecting Unknown Word Using Webcam for English as a Second Language (ESL) Learners}

%%
%% The "author" command and its associated commands are used to define
%% the authors and their affiliations.
%% Of note is the shared affiliation of the first two authors, and the
%% "authornote" and "authornotemark" commands
%% used to denote shared contribution to the research.

\author{Jiexin Ding}
\authornote{The authors contribute equally to this paper.}
\email{jxding17@gmail.com}
\affiliation{%
  \department{Global Innovation Exchange}
  \institution{Tsinghua University}
  \city{Beijing}
  \country{China}
}

\author{Bowen Zhao}
\authornotemark[1]
\email{bowen98@uw.edu}
\affiliation{%
    \department{Global Innovation Exchange}
  \institution{University of Washington}
  \city{Seattle}
  \country{United States}
}
\affiliation{%
  \institution{Tsinghua University}
  \city{Beijing}
  \country{China}
}

\author{Yuqi Huang}
\authornotemark[1]
\email{huangyuqi1999@gmail.com}
\affiliation{%
    \department{Global Innovation Exchange}
  \institution{University of Washington}
  \city{Seattle}
  \country{United States}
}
\affiliation{%
  \institution{Tsinghua University}
  \city{Beijing}
  \country{China}
}

\author{Yuntao Wang}
\email{yuntaowang@tsinghua.edu.cn}
\affiliation{%
    \department{Department of Computer Science and Technology}
  \institution{Tsinghua University}
  \city{Beijing}
  \country{China}
}
\authornote{denotes as the corresponding author.}

\author{Yuanchun Shi}
\email{shiyc@tsinghua.edu.cn}
\affiliation{%
  \department{Department of Computer Science and Technology}
  \institution{Tsinghua University}
  \city{Beijing}
  \postcode{100084}
  \country{China}
}

%%
%% By default, the full list of authors will be used in the page
%% headers. Often, this list is too long, and will overlap
%% other information printed in the page headers. This command allows
%% the author to define a more concise list
%% of authors' names for this purpose.
\renewcommand{\shortauthors}{Jiexin Ding, et al.}

%%
%% The abstract is a short summary of the work to be presented in the
%% article.
\begin{abstract}
%Identifying unknown words is the foundation of many reading applications that assist English as a Second Language (ESL) learners. 
Automatic unknown word detection techniques can enable new applications for assisting English as a Second Language (ESL) learners, thus improving their reading experiences. 
%Eye movement has a strong correlation with cognitive states in reading and the ubiquity of webcams shows the great potential of applying eye tracking in daily life. 
However, most modern unknown word detection methods require dedicated eye-tracking devices with high precision that are not easily accessible to end-users. 
In this work, we propose \projectname{}, an unknown word detection method only using a webcam. 
\projectname{} tracks the learner's gaze and then applies a transformer-based machine learning model that encodes the text information to locate the unknown word. 
We applied knowledge enhancement including term frequency, part of speech, and named entity recognition to improve the performance. 
The user study indicates that the accuracy and F1-score of our method were 98.09\% and 75.73\%, respectively. Lastly, we explored the design scope for ESL reading and discussed the findings. 
%We also explored the design space for ESL learners and proposed several potential applications.

%Eye movement can be used to predict users' reading states and predict unfamiliar words for English as a Second Language (ESL) learners to facilitate their reading. Webcam can track users' eye movement without additional costs. However, the low precision of eye tracking based on the webcam makes it difficult to extract valid eye movement features as proposed in infrared-based tracking. In this paper, we propose \projectname, integrating eye movement and language model to accurately predict unfamiliar words to facilitate ESL reading. \todo{We embed eye movement and text features by xxx and xxx them into a xxx model to predict unfamiliar words.} The accuracy and f1-score of our model are xxx and xxx, respectively. We also explore the design scope for ESL learners based on our model and proposed several potential applications.
  
\end{abstract}

%%
%% The code below is generated by the tool at http://dl.acm.org/ccs.cfm.
%% Please copy and paste the code instead of the example below.
%%

\begin{CCSXML}
<ccs2012>
<concept>
<concept_id>10003120.10003121.10003128</concept_id>
<concept_desc>Human-centered computing~Interaction techniques</concept_desc>
<concept_significance>500</concept_significance>
</concept>
</ccs2012>
\end{CCSXML}

\ccsdesc[500]{Human-centered computing~Interaction techniques}
% \begin{CCSXML}
% <ccs2012>
%  <concept>
%   <concept_id>10010520.10010553.10010562</concept_id>
%   <concept_desc>Computer systems organization~Embedded systems</concept_desc>
%   <concept_significance>500</concept_significance>
%  </concept>
%  <concept>
%   <concept_id>10010520.10010575.10010755</concept_id>
%   <concept_desc>Computer systems organization~Redundancy</concept_desc>
%   <concept_significance>300</concept_significance>
%  </concept>
%  <concept>
%   <concept_id>10010520.10010553.10010554</concept_id>
%   <concept_desc>Computer systems organization~Robotics</concept_desc>
%   <concept_significance>100</concept_significance>
%  </concept>
%  <concept>
%   <concept_id>10003033.10003083.10003095</concept_id>
%   <concept_desc>Networks~Network reliability</concept_desc>
%   <concept_significance>100</concept_significance>
%  </concept>
% </ccs2012>
% \end{CCSXML}

% \ccsdesc[500]{Computer systems organization~Embedded systems}
% \ccsdesc[300]{Computer systems organization~Redundancy}
% \ccsdesc{Computer systems organization~Robotics}
% \ccsdesc[100]{Networks~Network reliability}

%%
%% Keywords. The author(s) should pick words that accurately describe
%% the work being presented. Separate the keywords with commas.
\keywords{unknown words detection, webcam, eye tracking, natural language processing}

%% A "teaser" image appears between the author and affiliation
%% information and the body of the document, and typically spans the
%% page.
% \begin{teaserfigure}
%   \includegraphics[width=\textwidth]{sampleteaser}
%   \caption{Seattle Mariners at Spring Training, 2010.}
%   \Description{Enjoying the baseball game from the third-base
%   seats. Ichiro Suzuki preparing to bat.}
%   \label{fig:teaser}
% \end{teaserfigure}

% \received{20 February 2007}
% \received[revised]{12 March 2009}
% \received[accepted]{5 June 2009}

%%
%% This command processes the author and affiliation and title
%% information and builds the first part of the formatted document.

\maketitle

\section{Introduction}
\IEEEPARstart{T}{he} method Neural Radiance Fields (NeRF)~\cite{mildenhall2020nerf} is proposed for photorealistic novel view synthesis. Given many views of the scene, it creates implicit multi-view geometry and learns for view synthesis. However, it has poor generalizations to new scenes and requires retraining or fine-tuning on each scene. 
 
 Recent work~\cite{Yu_2021_CVPR,Trevithick_2021_ICCV} has explored the ways of using a single image to train NeRF. They introduce a convolutional feature encoder to learn the image representation which gives it some limited generalization abilities to unseen scenes.  But, without fine-tuning, these methods produce many floats and artifacts in rendering novel views. 
 
  Multi-Plane Images (MPI) representation that learns multiple RGB images from a single image is also used in \cite{Wu_2021_ICCV,Tucker_2020_CVPR,wu2022remote} for  novel view synthesis. However, MPI heavily relies on the qualities of the planar images and needs plenty of image planes to avoid blurs. There is no strong 3D geometry constraint and it fails in many complex scenes.
  
  MINE~\cite{Li_2021_ICCV2} introduces the volume rendering of NeRF into the MPI. It runs faster and produces better depth rendering quality compared with single-view NeRFs~\cite{Yu_2021_CVPR,Trevithick_2021_ICCV}. However, the rendering quality heavily relies on the number of image planes. It needs high-resolution 4D volumes to store the 4-channel  (RGB and volume density) image planes that cost a large amount of GPU memory in both training and 
 prediction.  
 

 
 \begin{figure}[t]
\setlength{\abovecaptionskip}{7pt}
\setlength{\belowcaptionskip}{0pt}
	\centering
% 	\subfigure[MINE (PSNR:14.9)]{  % for AAAI
	\subfloat[MINE (PSNR:14.9)]{
%			\centering
			\includegraphics[width=0.23\textwidth]{figure/intro/DJI_20200223_163206_598_0_MINE.png}
%			\label{subfig:pixelnerf}
	}\subfloat[MINE (depth)]{
%			\centering
			\includegraphics[width=0.23\textwidth]{figure/intro/MINE_disp.png}
%			\label{subfig:mpi}
	}
	\\[-3mm]
	\subfloat[Ours (PSNR:17.0)]{
%			\centering
			\includegraphics[width=0.23\textwidth]{figure/intro/DJI_20200223_163206_598_0_ours.png} 
	}\subfloat[Ours (depth)]{
%			\centering
			\includegraphics[width=0.23\textwidth]{figure/intro/ours_disp.png}
	}
	\caption{Comparison with state-of-the-art methods. (a-b) RGB and depth rendering results of  \cite{Li_2021_ICCV2}. It produces many blurs and floats in the occluded regions and at the object/depth edges. 
	(c-d) Our method employs a joint rendering mechanism that preserves more image details and predicts sharp depth edges.}
	\label{fig:performance_illustration}
\end{figure}
 
 In this paper, we propose a joint rendering mechanism that takes the MPI strategy for coarse sampling proposals and the MLP\&volume-based rendering~\cite{mildenhall2020nerf} for fine sampling and rendering. Then, both the coarse point samples and the fine samples are combined according to their geometry distribution to realize a more accurate joint rendering. More importantly, we introduce a depth teacher net that serves as the guidance for the joint rendering. The monocular depth teacher predicts dense pseudo depth maps that assist the consistent 3D geometry learning between the MPI, the fine volume, and the joint rendering. It also boosts the multi-view geometry consistency between the source view and the target novel views that 
helps handle the occlusions, reduce the blurs and floats, and render accurate depths. 
 
In the experiments,  we verify the effectiveness of our method on three challenging real-scene datasets (RealEstate10K~\cite{zhou2018stereo}, NYU~\cite{silberman2012indoor} and  NeRF-LLFF~\cite{mildenhall2020nerf}) for novel view synthesis or depth estimation. Given a single image as input, our method is shown able to produce higher qualities in both the RGB image rendering and depth map prediction. It far outperforms state-of-the-art methods~\cite{Li_2021_ICCV2,Yu_2021_CVPR} with improvements of 5$\sim$20\% in PSNR and SSIM for the RGB rendering and reduces 20$\sim$50\% of the errors for the depth prediction.
\vspace{-4mm}
\section{Related Works}
\noindent\textbf{Sign Language Recognition.} Sign language recognition (SLR) is a fundamental task in the field of sign language understanding.
Feature extraction plays a key role in an SLR model.
% 
Most recent SLR works \cite{jiang2021sign, jiang2021skeleton, hu2021signbert, hu2021hand, li2020transferring, li2020word, joze2019ms, hu2021global, stmc, zuo22_interspeech, vac} adopt CNN-based architectures, \eg, I3D \cite{I3D} and R3D \cite{qiu2017learning}, to extract vision features from RGB videos.
In this work, we adopt S3D \cite{xie2018rethinking} as the backbone of our VKNet due to its excellent accuracy-speed trade-off.

However, RGB-based SLR models may suffer from the large variation of video backgrounds. 
As a complement, some SLR works \cite{jiang2021skeleton, jiang2021sign, hu2021hand, hu2021signbert, chentwo} explore to jointly model RGB videos and keypoints.
For example, SAM-SLR \cite{jiang2021skeleton} uses graph convolutional networks (GCNs) to model pre-extracted keypoints.
HMA \cite{hu2021hand} and SignBERT \cite{hu2021signbert} propose to decode 3D hand keypoints from RGB videos.
A common deficiency of these works is that they need a dedicated network to model keypoints.
In this work, we represent keypoints as a sequence of heatmaps~\cite{duan2022revisiting, chentwo} so that the keypoint encoder of our VKNet can share the identical architecture with the video encoder.

To enable mini-batch training, previous works \cite{jiang2021sign, jiang2021skeleton, hu2021signbert, hu2021hand, li2020transferring, li2020word} crop fixed-length clips from raw videos as model inputs.
However, the model may overfit to the training videos of fixed temporal receptive fields.
In contrast, our VKNet is trained on videos with varied temporal receptive fields to improve its generalization capability.



\noindent\textbf{Word Representation Learning.}
Word2vec \cite{word2vec} and GloVe \cite{glove} are two classical word representation learning frameworks in the field of NLP.
Based on word2vec, fastText \cite{mikolov2018advances} improves word representations with several modifications including the use of sub-word information \cite{bojanowski2017enriching} and position independent features \cite{mnih2013learning}.
Although some advanced language models, \eg, BERT \cite{kenton2019bert}, can also be used to extract word representations, they are computationally intensive and are not dedicated to word representation learning.
In this paper, we adopt the lightweight but effective fastText, which is also used in a recent sign language translation work \cite{yin2021simulslt}, to pre-compute gloss (word) representations.


\noindent\textbf{Vision-Language Models.}
Recently, a majority of vision-language models \cite{clip, align, yao2022filip, gu2022wukong} learn visual representations on large-scale image-text pairs.
Among them, CLIP \cite{clip} is the pioneer to jointly optimize an image encoder and a text encoder through a contrastive loss. 
% 
Besides, the pre-trained CLIP can be generalized to various downstream tasks, \eg, semantic segmentation \cite{xu2022groupvit, li2021language, xu2021simple}, object detection \cite{du2022learning, rao2022denseclip}, image classification~\cite{zhou2022learning,huang2022unsupervised}, and style transfer \cite{patashnik2021styleclip, kwon2022clipstyler}.
In this work, we exploit the implicit knowledge included in glosses (sign labels), which is distinct from previous works on vision-language modeling.


\noindent\textbf{Multi-label Classification.} Real-world objects may have multiple semantic meanings, which motivates research on multi-label classification \cite{ridnik2021asymmetric, ke2022hyperspherical, zhang2013review, rajeswar2022multi, kim2022large} requiring models to map inputs to multiple possible labels.
Although the VISigns may be associated with the multi-label classification problem, most widely-adopted SLR datasets \cite{li2020word, joze2019ms, hu2021global} are singly labeled.
In this work, we deal with the VISigns by incorporating language information included in glosses.

\vspace{-0.3em}
\section{Method}
\vspace{-0.3em}

Our sensitivity-aware visual parameter-efficient fine-tuning consists of two stages. In the first stage, SPT measures the task-specific sensitivity for the pre-trained parameters (Section~\ref{subsec:sensitivity}). Based on the parameter sensitivity and a given parameter budget, SPT then adaptively allocates trainable parameters to task-specific important positions (Section~\ref{subsec:SPT}).

\vspace{-0.3em}
\subsection{Task-specific Parameter Sensitivity}
\label{subsec:sensitivity}
\vspace{-0.3em}

Recent research has observed that pre-trained backbone parameters exhibit varying feature patterns~\cite{raghu2021vision,naseer2021intriguing} and criticality~\cite{zhang2019all,chatterji2019intriguing} at distinct positions. 
Moreover, when transferred to downstream tasks, their efficacy varies depending on how much pre-trained features are reused and how well they adapt to the specific domain gap~\cite{yosinski2014transferable,kumar2022finetuning,neyshabur2020being}. Motivated by these observations, we argue that not all parameters contribute equally to the performance across different tasks in PEFT and propose a new criterion to measure the sensitivity of the parameters in the pre-trained backbone for a given task.

Specifically, given the training dataset $\gD_t$ for the $t$-th task and the pre-trained model weights $\vw=\left\{w_1, w_2, \ldots, w_N\right\}\in \sR^N$ where $N$ is the total number of parameters, the objective for the task is to minimize the empirical risk: $\min_{\vw} E(\gD_t, \vw)$.
We denote the parameter sensitivity \bohan{set} as $\gS=\{s_1, \ldots, s_N\}$ and the sensitivity $s_n$ for parameter $w_n$ is measured by the empirical risk difference when tuning it:
\begin{equation}
\vspace{-0.3em}
    \begin{aligned}
        s_n = E(\gD_t, \vw)-E(\gD_t, \vw\mid w_n=w_n^*),
    \end{aligned}
\label{eq:sensitivity}
\end{equation}
where $w_n^*=\underset{w_n}{\rm argmin}(E(\gD_t, \vw))$. We can reparameterize the tuned parameters as  $w_n^*=w_n+\Delta_{w_n}$, where $\Delta_{w_n}$ denotes the update for $w_n$ after tuning. Here we individually measure the sensitivity of each parameter, which is reasonable given that most of the parameters are frozen during fine-tuning in PEFT. However, it is still computationally intensive to compute Eq.~(\ref{eq:sensitivity}) for two reasons. Firstly, getting the empirical risk for $N$ parameters requires forwarding the entire network $N$ times, which is time-consuming. Secondly, it is challenging to derive $\Delta_{w_n}$, as we have to tune each individual $w_n$ until convergence.

{\begin{algorithm}[t!]
\caption{\label{alg:tps} Computing task-specific parameter sensitivities}
\begin{algorithmic}
    \STATE \textbf{Input:} Pre-trained model with network parameters $\vw$, training set $\gD_t$ for the $t$-th task, and number of training samples $C$ used to calculate the parameter sensitivities
    \STATE \textbf{Output:} Sensitivity set $\gS=\{s_1, \ldots, s_N\}$
    \STATE Initialize $\gS=\{0\}^N$
    \FOR{$i\in\{1,\ldots,C\}$}
        \STATE Get the $i$-th training sample of $\gD_t$
	    \STATE Compute loss $E$
		\STATE Compute gradients $\vg$
		\FOR{$n\in\{1,\ldots,N\}$}
                \STATE Update sensitivity for the $n$-th parameter: $s_{n} = s_{n} + g_n^2$
		    \ENDFOR
    \ENDFOR
\end{algorithmic}
\end{algorithm}}


\begin{figure*}[t]
\begin{center}
    \includegraphics[width=\linewidth]{main_figure.pdf}
\end{center}\vspace{-2em}
\caption{Overview of our trainable parameter allocation strategy. With the parameter sensitivity \bohan{set} $\gS$, we first get the top-$\tau$ sensitive parameters. Instead of directly tuning these sensitive parameters, we also boost the representational capability by replacing unstructured tuning with structured tuning at sensitive weight matrices that have a large number of sensitive parameters, which can be implemented by an existing structured tuning method, \eg, LoRA~\cite{hu2022lora} and Adapter~\cite{houlsby2019parameter}. Red lines and blocks represent trainable parameters and modules, while blue lines represent frozen parameters.}
\label{fig:main}
\vspace{-1.5em}
\end{figure*}


To overcome the first barrier, we simplify the empirical loss by approximating $s_n$ in the vicinity of $\vw$ by its first-order Taylor expansion
\vspace{-0.3em}
\begin{equation}
\vspace{-0.5em}
    \begin{aligned}
        s_n^{(1)} = -g_n\Delta_{w_n},
    \end{aligned}
\label{eq:first-order}
\end{equation}
where the gradients $\vg=\partial E/\partial\vw$, and $g_n$ is the gradient of the $n$-th element of $\vg$. 
To address the second barrier, following~\cite{liu2018darts,cai2018proxylessnas}, we take the one-step unrolled weight as the surrogate for $w_n^*$ and approximate $\Delta_{w_n}$ in Eq.~(\ref{eq:first-order}) with a single step of gradient descent. We can accordingly get $s_n^{(1)} \approx g_n^2\epsilon$,
where $\epsilon$ is the learning rate. Since $\epsilon$ is the same for all parameters, we can eliminate it when comparing the sensitivity with the other parameters and finally get 
\vspace{-0.5em}
\begin{equation}
\vspace{-0.3em}
    \begin{aligned}
        s_n^{(1)} \approx g_n^2.
    \end{aligned}
\label{eq:first-order-simp}
\end{equation}
Therefore, the sensitivity of a parameter can be efficiently measured by its potential to reduce the loss on the target domain. Note that although our criterion draws inspiration from pruning work~\cite{molchanov2019importance}, it is distinct from it. \cite{molchanov2019importance} measures the parameter importance by the squared change in loss when removing them, \ie, $\left( E(\gD_t, \vw)-E(\gD_t, \vw\mid w_n=0) \right)^2$ and finally derives the parameter importance by $\left( g_n w_n \right)^2$, which is different from our formulations in Eqs.~(\ref{eq:sensitivity}) and~(\ref{eq:first-order-simp}).

In practice, we accumulate $\gS$ from a total number of $C$ training samples ahead of fine-tuning to generate accurate sensitivity as shown in Algorithm~\ref{alg:tps}, where $C$ is a pre-defined hyper-parameter. In Section~\ref{subsec:abl}, we show that employing only 400 training samples is sufficient for getting reasonable parameter sensitivity, which requires only 5.5 seconds with a single GPU for any VTAB-1k dataset with ViT-B/16 backbone~\cite{vit}.

\vspace{-0.3em}
\subsection{Adaptive Trainable Parameters Allocation}
\label{subsec:SPT}
\vspace{-0.2em}

Our next step is to allocate trainable parameters based on the obtained parameter sensitivity set $\gS$ and a desired parameter budget $\tau$. A straightforward solution is to directly tune the top-$\tau$ most sensitive unstructured connections (parameters) \rev{while keeping the rest frozen}, which we name unstructured tuning. Specifically, we select the top-$\tau$ most sensitive weight connections in $\gS$ to form the sensitive weight connection set $\gT$. Then, for \rev{a} weight matrix $\mW\in \sR^{d_{\rm in}\times d_{\rm out}}$, we can get a binary mask $\mM\in \sR^{d_{\rm in}\times d_{\rm out}}$ computed by
\vspace{-0.5em}
\begin{equation}
\vspace{-0.5em}
    {\begin{array}{ll}
    \small
    \begin{aligned}
    \mM^j =
    \left\{\begin{array}{ll} 
    1 ~~~~~ \mW^j \in \gT \\
    0 ~~~~~ \mW^j \notin \gT
    \end{array}\right.
    \end{aligned},
    \small
    \end{array}}
\label{eq:mask}
\end{equation}
where $\mW^j$ and $\mM^j$ are the $j$-th element in $\mW$ and $\mM$, respectively. Accordingly, we can train the sensitive parameters by gradient descent and the updated weight matrix can be formulated as $\mW'\leftarrow \mW - \epsilon\vg_{\mW}\odot\mM$, where $\vg_{\mW}$ is the gradient for $\mW$.

However, considering PEFT approaches generally limit the proportion of trainable parameters to less than 1\%, tuning only a small number of unstructured weight connections might not have enough representational capability to handle the downstream datasets with large domain gaps from the source pre-training data. Therefore, to improve the representational capability, we propose to replace unstructured tuning with structured tuning at the sensitive weight matrices that have a high number of sensitive parameters. To preserve the parameter budget, we can implement structured tuning with an existing efficient structured tuning PEFT method~\cite{hu2022lora,chen2022adaptformer,houlsby2019parameter,jie2022convolutional} that learns to directly adjust \rev{all hidden dimensions at once}. We depict an overview of our trainable parameter allocation strategy in Figure~\ref{fig:main}. For example, we can employ the low-rank reparameterization trick LoRA~\cite{hu2022lora} to the sensitive weight matrices \rev{and the one-step update for $\mW$ can be formulated as}
\vspace{-0.4em}
\begin{equation}
\vspace{-0.4em}
    {\begin{array}{ll}
    \small
    \begin{aligned}
    \mW' = \left\{\begin{array}{ll} 
    \mW + \mW_{\rm down}\mW_{\rm up} & ~~ \text { if } ~~ \sum_{j=0}^{d_{\rm in}\times d_{\rm out}} \mM^j \geq \sigma_{\rm opt} \\
    \mW - \epsilon\vg_{\mW}\odot\mM & ~~ {\rm otherwise}
    \end{array}\right.
    \end{aligned},
    \small
    \end{array}}
\label{eq:weight_updat}
\end{equation}
where $\mW_{\rm down}\in \sR^{d_{\rm in}\times r}$ and $\mW_{\rm up}\in \sR^{r\times d_{\rm out}}$ are two learnable low-rank matrices to approximate the update of $\mW$ and rank $r$ is a hyper-parameter where $r \ll {\rm min}(d_{\rm in},d_{\rm out})$. In this way, we perform structured tuning on $\mW$ when its number of sensitive parameters exceeds $\sigma_{\rm opt}$, whose value depends on the pre-defined type of structured tuning method. For example, since implementing structured tuning with LoRA requires $2\times d_{\rm in} \times d_{\rm out} \times r$ trainable parameters for each sensitive weight matrix, we set $\sigma_{\rm LoRA} \leftarrow 2\times d_{\rm in} \times d_{\rm out} \times r$ to ensure that the number of trainable parameters introduced by structured tuning is always equal to or lower than the number of sensitive parameters.

In this way, our SPT adaptively incorporates both structured and unstructured tuning granularities to enable higher flexibility and stronger representational power, simultaneously. In Section~\ref{subsec:abl}, we show that structured tuning is important for the downstream tasks with larger domain gaps and both unstructured and structured tuning contribute clearly to the superior performance of our SPT.
\section{Results and Findings}
\label{sec:results}


\subsection{Prediction Accuracy}

\begin{table}[b]
\begin{tabular}{lrr|rrr}
\hline
PLM Backbone & $n_p$ & $n_k$ & Precision & Recall & F-1 Score \\ \hline
RoBERTa      & 16             & 16           &68.31     &79.20    &73.97           \\
RoBERTa      & 16             & 32                  &\textbf{71.21}           &80.70        &\textbf{75.73}           \\
RoBERTa      & 32             & 16                  &67.29           &84.81        &75.11           \\
RoBERTa      & 32             & 32                  &65.50           &\textbf{86.55}        &74.97           \\ \hline
\end{tabular}
\caption{Unknown word detection results of \projectname.}
\label{tab:main-results}
% \vspace{-8mm}
\end{table}

The main results of our model are shown in Table~\ref{tab:main-results}. We use precision, recall, and F-1 score as metrics for our unknown word detection task. We trained our models for 5 epochs, with the learning rate of 8e-5 for parameters in the RoBERTa and linear layers, and 0.1 for LSTM layers. Our best model can achieve the accuracy of 98.09\% with 75.73\% F-1 score. Moreover, we tried out different hidden dimensions for gaze-text attention $n_p$ and knowledge embedding $n_k$. It shows that by increasing the knowledge dimension $n_k$, the overall performance of our model can be improved while increasing the gaze-document attention dimension $n_p$ helps promote the recall.

\subsection{Transfer Learning Analysis}

\subsubsection{Transfer Learning Setting}
Table~\ref{tab:transfer-learning} indicates our model's transferability between users and reading materials. For the cross-user transfer learning experiment, we trained the model 12 times in total, and for each running, we selected one user's data as the test set and took other users' data as the training set. For the cross-document experiment, we also conducted 12 pieces of training and evaluations. Under each running, we selected the data generated from three documents as the test set, and the rest of the data were combined as the training set. Therefore, our relative data sizes between training and testing data are similar under cross-user and cross-document settings.

\begin{figure}
    \centering
    \includegraphics[width=\columnwidth]{fig/jaccard.png}
    \caption{Jaccard similarity of unknown words between users.}
    \label{fig:jaccard}
\end{figure}

\subsubsection{Cross-User Transferability}
We find out that when the model is applied to new users' data, its performance is still on par with its original setting. Therefore, our model can be effectively adapted to new user groups without specific calibrations or tuning, which successfully fits our goal of ubiquitous unknown word detection. We further analyze the unknown words of different users, and their Jaccard similarity matrix is shown in Fig~\ref{fig:jaccard}. The average of cross-user unknown words Jaccard similarity score is 0.23 and most of their similarities are below 0.3, meaning that different users' unknown words are distinct even given the same article. It further proves that our model does not simply remember the unknown words of users, but predicts the unknown words based on the user's behaviors. Meanwhile, the model's performance can become significantly better on some users, and a possible reason is that these users' behaviors are more general, i.e., their language skills might be the average among all people.

\begin{table*}[]
\begin{tabular}{ll|rrr}
\hline
User  dependency & Document dependency & Precision & Recall & F-1 Score \\ \hline
Dependent        & Dependent           &71.21           &80.70        &75.73           \\
Independent      & Dependent           &$73.65 \pm 5.83$           &$82.77\pm 4.41$       &$78.26\pm 4.53$           \\
Dependent        & Independent         &$46.56\pm 4.26$           &$68.39\pm 5.80$        &$56.31 \pm 3.38$           \\ \hline
\end{tabular}
\caption{Transferability analysis of unknown word detection by \projectname{}.}
\label{tab:transfer-learning}
\end{table*}

\subsubsection{Cross-Document Transferability}
In order to test whether our model only memorizes the difficult words from the text, we conducted the experiment to see how the model would perform if the words in the test data were not presented for training. We find that the model's performance dropped in this setting, while the recall is still acceptable. 

\subsection{Ablation Study}

\begin{table*}[]
\begin{tabular}{l|rrr}
\hline
Model                         & Precision & Recall & F-1 Score \\ \hline
\projectname{}                           &\textbf{71.21}           &80.70        &\textbf{75.73}           \\
\projectname{}~w/o context-awareness     &4.78           &59.03        &10.00           \\
\projectname{}~w/o gaze                  &66.35           &\textbf{86.67}        &75.59           \\
\projectname{}~w/o knowledge enhancement &69.96          &80.32        &74.93           \\ \hline
\end{tabular}
\caption{Ablation study of \projectname{}.}
\label{tab:ablation-study}
% \vspace{-5mm}
\end{table*}

We tried to exclude three different modalities of data presented in our model respectively, which are text information, gazing data, and word-level knowledge. After the text information is removed, the model's performance drops significantly, which proves that the use of a pre-trained language model can effectively promote the model's ability to detect unknown words of users. 
%Meanwhile, after removing the gazing data, the model's performance slightly decreases since, in this case, the model cannot determine which specific span of text the user is reading. Therefore, we can see that the model's precision has dropped because it tend to recognize difficult words that the user is not reading as predicted unknown words. 
Then we removed the gaze data, in which case the model cannot obtain the gaze behavior related to the text. The model's precision drops and the recall increases. It indicates that both the true positive and the false positive increase. In other words, without gaze, the model tends to predict a word as an unknown word. The gaze may help identify whether a difficult word is an unknown word or not.
Finally, the result of ablating knowledge enhancement proves that infusing word-level knowledge can promote the model's performance.

\subsection{Design Guidance}

Our results of the 5000-level VLT show that users can be divided into a high-level group and a low-level group. A Mann-Whitney U test was run to determine if there were differences in the test score between them. Scores for the high-level group (mean rank = 9.5) and low-level group (mean rank = 3.5) were statistically significantly different, \textit{U} = .000, z = -2.887, \textit{p} = .002, using an exact sampling distribution for U. The results demonstrate that our users covered learners of different levels.

As for factors affecting the reading of academic texts, compared to a neutral attitude score of 3, the scores were 3.67 for new words (\textit{SD} = 0.98, \textit{t}(11) = 2.345, \textit{p} < .05), 3.83 for the ambiguous meaning of words (\textit{SD} = 0.72, \textit{t}(11) = 4.022, \textit{p} < .01), 4.25 for long and complex sentences (\textit{SD} = 0.87, \textit{t}(11) = 5.000, \textit{p} < .001), all of which were perceived by users as having a significant effect on comprehension of the text. This result suggests the importance of contextualizing the interpretation of unknown words in reading.

The scores of the two proposed features are 4.35 for eye-tracking reading assistance (\textit{SD} = 0.64, \textit{t}(11) = 7.244, \textit{p} < .001) and 3.93 for vocabulary management (\textit{SD} = 0.99, \textit{t}(11) = 3.260, \textit{p} < .01), respectively. Both are significantly higher than a neutral score of 3, demonstrating that users hold a positive attitude toward our proposed features. The results show that none of the privacy and power consumption is rated significantly, which indicates the possibility of future tool design.

By analyzing words users wrote about expecting features, we found that the three most frequently mentioned features are translating proper nouns, translating long and complex sentences, and accurately identifying multi-meaning words. Two other users also mentioned the need to avoid pop-ups from affecting their reading. We believe these goals can all be achieved using our webcam-based tool and NLP model.

Based on the results mentioned above, applying the webcam to unknown words detecting and reading assistance by our approach of eye tracking and NLP model has theoretical and practical value. Interaction design strategies should also be addressed in developing applications to improve the user experience.
% \input{5-study.tex}
\section{Conclusion and Future Works}

We present a novel formulation for a differentiable DTW layer that can be embedded into a deep network, based on continuous time warping and implicit deep layers. Unlike existing approaches, DecDTW yields the \textit{optimal} alignment path as a differentiable function of layer inputs. We demonstrate the effectiveness of DecDTW in utilising path information in challenging, real-world alignment tasks compared to existing methods. One promising direction for future work is in exploring more compact, alternative parameterisations of the warp function to improve scalability, inspired by~\cite{zhou12gtw}. Another direction would be to integrate more sophisticated DTW variants such as jumpDTW~\citep{jumpdtw}, which allows for repeats and discontinuities in the warp, into the GDTW (and DecDTW) formulation. Finally, we presented methodology for allowing regularisation and constraints to be learnable parameters in a deep network but did not explore this in our experiments. Future work will also explore this capability in more detail.

\section{Conclusion and Future Work}
\label{sec:conclusion}

We have presented a novel neural network that successively learns shape sketch and extrusion without any expensive annotations of shape segmentation and labels as the supervision.
%Without the guidance of sketch labels, 
Our approach is able to learn smooth sketches, followed by the differentiable extrusion to reconstruct CAD models that are close to the ground truth. 
We evaluate SECAD-Net using diverse CAD datasets and demonstrate the advantages of our approach by ablation studies and comparing it to the state-of-the-art methods. 
We further demonstrate our method’s applicability in single-image CAD reconstruction. 
Additionally, the CAD shapes generated by our approach can be directly fed into off-the-shelf CAD software for sketch-level or cylinder primitive-level editing. 

% We tested SE-Net on ABC dataset and Fusion 360 dataset. Quantitative results demonstrate that SE-Net can efficiently reconstruct 3D CAD shapes. Qualitative results show that our model can learn fine 2d sketches without any associated ground-truth.


% We propose SE-Net, a network that successively learns shape sketch and extrusion in an unsupervised manner. The CAD shapes generated by the network can be directly sent to off-the-shelf CAD software for sketch-level or cylinder primitive-level editing. SE-Net can be reconstructed to generate smooth sketches and the reconstruction effect is due to the current state-of-the-art, including supervised methods. Additionally, our method is the first to learn sketches from raw shapes without the guidance of sketch labels.

In future work, we plan to extend our approach to learn more CAD-related operations such as \emph{revolve, bevel, and sweep}. %using neural methods. 
Besides, we find that current deep learning models perform poorly on datasets with large differences in shape geometry and structure. %structural and topological variations
Therefore, another promising direction is to explore how to improve the generalization of neural networks and enhance the realism of the generated shapes by learning structural and topological information.

\begin{acks}
This work is supported by the Natural Science Foundation of China (NSFC) under Grant  No. 62132010 and No. 62002198, Young Elite Scientists Sponsorship Program by CAST 
under Grant No. 2021QNRC001, Tsinghua University Initiative Scientific Research Program, Beijing Key Lab of Networked Multimedia, Institute for Artificial Intelligence, Tsinghua University.
\end{acks}

% Bibliography
\bibliographystyle{ACM-Reference-Format}
\bibliography{ref}

\end{document}
\endinput
%%
%% End of file `sample-authordraft.tex'.
