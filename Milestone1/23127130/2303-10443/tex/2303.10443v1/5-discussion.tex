\section{Discussion and future work}
\label{sec:discussion}

The accuracy of our model is high by combining the information from the gaze, text, and knowledge, but the F1-score is only 75.73\%. The reason could be that unknown words are sparse compared with other words. It causes an unbalance in the dataset. In addition, the size of the dataset is relatively small compared with the parameter in the neural network might be the other reason. More case studies are needed to find out the reason.

The ablation shows that the F1-score doesn't significantly decrease when we remove the gaze from the model. One of the potential reasons is the inaccuracy of the webcam-based eye-tracking algorithm we used. WebGazer~\cite{searchGazer_papoutsaki_2017} uses the coordinates of clicks to calibrate the position of gaze points, so tracking accuracy becomes worse when participants read without using a mouse. The optimization of the eye-tracking algorithm is needed to make gaze contribute more to the performance. Multi-modal data obtained by earphones and smart glasses can be used to infer head orientation and position~\cite{faceori,reflectrack} to facilitate the eye-tracking algorithm. 


%\subsection{Generalizability}
Our data is collected on a commodity laptop using an eye-tracking library that can work on any website~\cite{papoutsaki2016webgazer}. Therefore, our method should be able to generalize to other models of computers, though we only used one type of laptop to collect data in this paper. In addition to the current web-based implementation, the same algorithm should be able to migrate to various applications.

We took TOEFL reading passages as reading materials in the experiment, considering that unknown words in academic text create more difficulty in reading. The TOEFL readings cover a variety of topics, so our method has the potential to work on different types of articles. In the future, we will use more diverse materials to make our approach more generalizable. Participants in our experiment are all graduate students in the United States. We regard them as our target users because they have a more urgent need to read more efficiently. Our participants consisted of learners with different vocabulary levels, and we will also include more users of different learning stages in the future.

Applying our approach to languages other than English is another aspect of future work. We believe that this work may be generalized to other alphabetic languages because alphabetic languages have similar word structures and some of them even have the same alphabet. It remains to be seen whether gaze data can be used to detect unknown words in non-alphabetic languages.

%In addition to unknown words, other barriers that second language learners encounter in reading also deserve attention. 
For applications, future work could explore how to provide proper assistance to second language users during reading according to our exploratory user study. We aim to design a simple webcam-based tool across different levels of users that has high efficiency and usability at the same time. Exploring ways to properly present the information that users need would be a compelling future study.



%\subsection{Contribution of Gaze and Context}

%The ablation shows that the F1-score doesn't significantly decrease when we remove the gaze from the model. The inaccuracy of the webcam-based eye-tracking algorithm we used is one of the potential reasons. WebGazer~\cite{searchGazer_papoutsaki_2017} uses the coordinates of clicks to calibrate the position of gaze points, so tracking accuracy became worse when participants read without using a mouse. The optimization of the eye-tracking algorithm is needed to make gaze contribute more in future work.

%The accuracy of our model is high by combining the information from the gaze, text, and knowledge, but the F1-score is only 75.73\%. The reason could be that unknown words are sparse compared with other words. It causes an unbalance in the dataset. 
%Besides, when we remove the gaze from the model, the F1-score doesn't significantly decrease. It is reasonable because the set of unknown words may have a large overlap with the low-frequency words. However, if we make more improvements to the webcam-based eye-tracking algorithm, the gaze demonstrates more positive influence. 
%In addition, the size of the dataset is relatively small compared with the parameter in the neural network might be the other reason. More case studies are needed to find out the reason.

% \subsection{Future Work}
% According to our exploratory user study, future work could explore how to provide proper assistance to ESL users during reading. We aim to design a simple webcam-based tool across different levels of users that have high efficiency and usability at the same time. Exploring ways to properly present the information that users need would be a compelling future study.







%These findings corroborate with existing study's suggestions \cite{rayner1986lexical, barrett2020sequence}, which emphasized the influence of word frequency, verb complexity, and lexical ambiguity. We believe that these goals can all be achieved using our webcam-based tool and \bowen{xxx model, ex. context-based NLP models}.

%Based on this we believe that applying the webcam to unknown words detecting and reading assistance by our approach of eye tracking and NLP model has theoretical and practical value.