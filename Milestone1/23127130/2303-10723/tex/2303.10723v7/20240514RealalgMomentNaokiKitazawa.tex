\documentclass{amsart}
%\setlength{\textheight}{43pc}
%\setlength{\textwidth}{28pc}
\usepackage{amsfonts}
\usepackage{amsmath,amssymb}
\usepackage{amsthm}
\usepackage{amscd}
\usepackage{graphics}
\usepackage{graphicx}

%\usepackage[pagewise]{lineno}\linenumbers
\theoremstyle{remark}{
\newtheorem{Def}{{\rm Definition}}
\newtheorem{Ex}{{\rm Example}}
\newtheorem{Rem}{{\rm Remark}}
\newtheorem{Prob}{{\rm Problem}}
\newtheorem*{MainProb}{Main Problem}
}
\theoremstyle{plain}
{
\newtheorem{Cor}{Corollary}
\newtheorem{Prop}{Proposition}
\newtheorem{Thm}{Theorem}
\newtheorem{MainThm}{Main Theorem}
\newtheorem*{MainCor}{Main Corollary}
\newtheorem{Lem}{Lemma}
\newtheorem{Fact}{Fact}
}
\renewcommand*{\urladdrname}{\itshape Webpage}
\begin{document}
\title[Reconstructing real algebraic maps locally like moment-maps]{Reconstructing real algebraic maps locally like moment-maps with prescribed images and compositions with the canonical projections to the $1$-dimensional real affine space}
\author{Naoki kitazawa}
\keywords{(Non-singular) real algebraic manifolds and real algebraic maps. Smooth maps. Morse-Bott functions. Moment maps. Graphs. Reeb graphs. \\
\indent {\it \textup{2020} Mathematics Subject Classification}: Primary~14P05, 14P25, 57R45, 58C05. Secondary~57R19.}

\address{Institute of Mathematics for Industry, Kyushu University, 744 Motooka, Nishi-ku Fukuoka 819-0395, Japan\\
 TEL (Office): +81-92-802-4402 \\
 FAX (Office): +81-92-802-4405 \\
}
\email{n-kitazawa@imi.kyushu-u.ac.jp, naokikitazawa.formath@gmail.com}
\urladdr{https://naokikitazawa.github.io/NaokiKitazawa.html}
\maketitle
\begin{abstract}
We present new real algebraic maps of non-positive codimensions with prescribed images whose boundaries consist of explicit non-singular real algebraic hypersurfaces satisfying so-called "transversality". Explicit information on important real polynomials is also given. Preimages of them are one-point sets or products of spheres. They are locally like so-called {\it moment} maps. 

Celebrated theory of Nash and Tognoli says that smooth closed manifolds are {\it non-singular} real algebraic manifolds and the zero sets of some real polynomial maps. In general, we can approximate smooth functions or more generally, maps, by real algebraic ones. It is in general difficult to have explicit examples. We have constructed maps of a specific class of the present class containing the canonical projections of the unit spheres previously where preimages are one-point sets or spheres.  

We also present explicit families of functions represented as compositions of such maps with the canonical projections.

 


\end{abstract}
%【REVISE】 combinatoric ~ is → combinatorial object. It is .
%【REVISE】  such that a point is a vertex if and only if the corresponding connected component of the level set contains some singular points → whose vertex set is the set of all points containing some singular points in the corresponding connected component of the level set .
%【REVISE】 We delete "extending the result before".
\section{Introduction.}
\label{sec:1}

Real algebraic geometry has a long nice history and studies ({\it non-singular}) real algebraic manifolds or more generally, varieties. For related topics, see \cite{bochnakcosteroy, bochnakkucharz, kollar, kucharz, nash, tognoli} for example. Here, we concentrate on non-singular real algebraic manifolds.

Some of Nash and Tognoli's celebrated theory shows that smooth closed manifolds are non-singular real algebraic manifolds and the zero sets of some real polynomial maps. In considerable cases, we can approximate smooth functions on {\it non-singular} real algebraic manifolds or more generally, maps between non-singular real algebraic manifolds, by real algebraic ones. Here, as another problem, we study construction of explicit real algebraic maps of non-positive codimensions and their explicit global information. Some explicit examples of such functions and maps are well-known. The canonical projections of spheres embedded naturally in Euclidean spaces are simplest ones. Some natural functions on so-called projective spaces, Lie groups and their quotient spaces are also well-known. They are represented as real polynomials. 
Knowing explicit global information of the maps and the manifolds such as topological properties is difficult in general.
See \cite{maciasvirgospereirasaez, ramanujam, takeuchi} for example.

We give some new answers to this problem explicitly. 
We construct real algebraic maps whose codimensions are non-positive with prescribed images and preimages. The boundaries of the images consist of non-singular real algebraic hypersurfaces intersecting with the condition on "transversality". In addition, preimages are products of spheres. We can also know real polynomials for the desired zero sets and the manifolds explicitly. Such maps also generalize the canonical projections of spheres and locally like so-called {\it moment maps}.
%We also study global structures of the functions obtained as the compositions with canonical projections.
Our method applies some singularity theory of smooth maps, some theory on differential topology of manifolds, and some elementary real algebraic arguments and is interdisciplinary. Our study also adds to our related pioneering studies \cite{kitazawa3, kitazawa7}. Studies on domains formed by real algebraic curves \cite{bodinpopescupampusorea, kohnpieneranestadrydellshapirosinnsoreatelen, 
	sorea1, sorea2} also motivate us.
We also investigate functions represented as compositions of our new maps with the canonical projections. They are shown to be {\it Morse-Bott}. We mainly study global topological descriptions by the {\it Reeb graphs}, the natural spaces of connected components of preimages.
\subsection{Smooth manifolds and maps.}
Let $X$ be a topological space having the structure of some cell complex whose dimension is finite. We can define the dimension $\dim X$ uniquely. The dimension $\dim X$ is an integer of course. 
A topological manifold is well-known to have the structure of a CW complex. A smooth manifold is well-known to have the structure of a polyhedron. We can define the structure of a certain polyhedron for a smooth manifold canonically. This is a so-called PL manifold. It is also well-known that a topological space having the structure of a polyhedron whose dimension is at most $2$ has the structure of a polyhedron uniquely. For a topological manifold of dimension at most $3$, the same fact holds. This respects celebrated and well-known theory by \cite{moise} for example. Let ${\rm Int}\ X$ denote the interior of  a manifold $X$ and $\partial X$ the boundary $X-{\rm Int}\ X$ of $X$.
  
Let ${\mathbb{R}}^k$ denote the $k$-dimensional Euclidean space. It is a simplest $k$-dimensional smooth manifold and the Riemannian manifold with the standard Euclidean metric. 
Let $\mathbb{R}:={\mathbb{R}}^1$ and $\mathbb{N} \subset \mathbb{Z} \subset \mathbb{R}$ be the set of all positive integers and that of all integers respectively.
For each point $x \in {\mathbb{R}}^k$, we can define $||x|| \geq 0$ as the distance between $x$ and the origin $0$ under this metric.
This is also naturally a simplest real algebraic manifold: the $k$-dimensional real affine space. Let $S^k:=\{x \in {\mathbb{R}}^{k+1} \mid ||x||=1\}$ denote the $k$-dimensional unit sphere, which is a $k$-dimensional smooth compact submanifold of ${\mathbb{R}}^{k+1}$ and has no boundary. It is connected for any positive integer $k \geq 1$. It is a discrete two-point set for $k=0$. It is the zero set of the real polynomial $||x||^2-1={\Sigma}_{j=1}^{k+1} {x_j}^2-1$ with  $x:=(x_1,\cdots,x_{k+1})$ and a real algebraic (sub)manifold (hypersurface). 
%It is also a smooth real algebraic submanifold defined by the zero set of the real polynomial $||x||-1={\Sigma}_{j=1}^{k+1} {x_j}^2 -1$ where $x:=(x_1,\cdots,x_{k+1})$.
Let $D^k:=\{x \in {\mathbb{R}}^{k} \mid ||x|| \leq 1\}$ denote the $k$-dimensional unit disk. It is a $k$-dimensional smooth compact and connected submanifold of ${\mathbb{R}}^{k}$ for any positive integer $k \geq 1$. Of course we have $\partial D^k=S^{k-1}$.

For a smooth manifold $X$, let $T_xX$ denote the tangent vector space at $x \in X$. Let $c:X \rightarrow Y$ be a differentiable map from a differentiable manifold $X$ into another one $Y$. Let ${dc}_x:T_x X \rightarrow T_{c(x)}Y$ denote the differential of $c$ at $x \in X$: this is a linear map between the tangent vector spaces. If the rank of the differential ${dc}_x$ is smaller than the minimum of the (multi)set $\{\dim X, \dim Y\}$, then $x$ is a {\it singular point} of $c$.  For a {\it singular} point $x \in X$ of $c$, the value $c(x)$ is a {\it singular value} of $c$. Let $S(c)$ denote the set of all singular points of $c$. 

In our paper, we consider smooth maps, defined as maps of the class $C^{\infty}$, as differentiable maps, unless otherwise stated. A canonical projection of the Euclidean space ${\mathbb{R}}^k$ is denoted
by ${\pi}_{k,k_1}:{\mathbb{R}}^{k} \rightarrow {\mathbb{R}}^{k_1}$. This is defined as the map mapping 
each point $x=(x_1,x_2) \in {\mathbb{R}}^{k_1} \times {\mathbb{R}}^{k_2}={\mathbb{R}}^k$ to the first component $x_1 \in {\mathbb{R}}^{k_1}$ with the conditions on the dimensions given by $k_1, k_2>0$ and $k=k_1+k_2$. A canonical projection of the unit sphere $S^{k-1}$ is the restriction of it.

Here, a real algebraic manifold is represented as a union of connected components of the intersection of finitely many zero sets of real polynomials. {\it Non-singular} real algebraic manifolds are defined naturally by the implicit function theorem for the polynomials or the maps defined canonically from the polynomials. The real affine space and the unit sphere are simplest non-singular ones. {\it Real algebraic} maps here are represented as the compositions of the canonical embeddings into the real affine spaces with canonical projections.


\subsection{Our main result.}
% \cite{masumotosaeki} generalizes the pioneering work of \cite{sharko}. \cite{sharko} constructs nice smooth functions on closed surfaces and \cite{masumotosaeki} extends this to arbitrary finite graphs.
%Later, for example, \cite{martinezalfaromezasarmientooliveira, michalak} have set explicit problems and solved. Before the study \cite{kitazawa1} of the author, functions are, ones on closed surfaces or ones preimages containing no singular points of which are disjoint unions of spheres essentially.

%However we can apply some important cases. For example \cite{saeki2} considers very general cases and we cannot use analytic functions for construction there.

Hereafter, let ${\mathbb{N}}_a$ be the set of all elements of $\mathbb{N}$ smaller than or equal to a given real number $a$. For example, for a positive integer $i$, ${\mathbb{N}}_i:=\{1,\cdots,i\}$ and the set of all integers from $1$ to $i$.

\begin{MainThm}
	\label{mthm:1}
	Let $l_1$, $l_2$ and $n$ be positive integers.
	Let $\{S_j\}_{j=1}^{l_1}$ be a family of non-singular
	real algebraic hypersurfaces of ${\mathbb{R}}^n$ each $S_j$ of which is connected and the zero set of a real polynomial $f_j$.
Let $D$ be a connected open subset of ${\mathbb{R}}^n$ which is non-empty and whose closure $\overline{D}$ is compact and connected. Suppose also the following.
	\begin{enumerate}
		\item \label{mthm:1.1} %It holds that $S_j \bigcap \overline{D}$ is non-empty for each $1 \leq j \leq l_1$. % It is also connected. %The intersection of $U_D$ and the zero set of $f_j$ is $U_D \bigcap S_j$.
For any sufficiently small open neighborhood $U_D$ of $\overline{D}$, it holds that $D=U_D \bigcap {\bigcap}_{j=1}^{l_1} \{x \mid f_j(x)>0\}$ and that $\overline{D}=U_D \bigcap {\bigcap}_{j=1}^{l_1} \{x \mid f_j(x) \geq 0\}$.
% and $\overline{D}-D \subset {\bigcup}_{j=1}^{l_1} S_j$ hold. 

	%	\item \label{mthm:1.2} ${\bigcap}_{i=1}^{l_3} S_{j_i}$ is either empty or a non-singular real algebraic submanifold of dimension $n-l_3$ where $\{j_i\}_{i=1}^{l_3}$ is an increasing sequence. The real algebraic submanifold is also with no boundary of course.
	%	  \item \label{mthm:1.3}
%The {\rm (}$n-l_3${\rm )}-dimensional manifold is cornered in general.
\item \label{mthm:1.2} For each increasing sequence $\{j_i\}_{i=1}^{l_3} \subset {\mathbb{N}}_{l_1}$ {\rm (}$1 \leq l_3 \leq l_1${\rm )} and each point $p$ in the intersection ${\bigcap}_{i=1}^{l_3} S_{j_i} \bigcap \overline{D}$, the dimension of the intersection ${\bigcap}_{i=1}^{l_3} T_{p} S_{j_i}$ of the tangent vector spaces is $n-l_3$. 
%Furthermore, the set ${\bigcap}_{i=1}^{l_3} S_{j_i} \bigcap \overline{D}$ is either empty or an {\rm (}$n-l_3${\rm )}-dimensional smooth compact submanifold of ${\bigcap}_{i=1}^{l_3} S_{j_i}$. 
\item \label{mthm:1.3} A surjective map $m_{l_1,l_2}:{\mathbb{N}}_{l_1} \rightarrow {\mathbb{N}}_{l_2}$ enjoying the following properties exists: for any increasing sequence $\{j_i\}_{i=1}^{l_3}$ making the set ${\bigcap}_{i=1}^{l_3} S_{j_i} \bigcap \overline{D}$ non-empty,  the restriction of $m_{l_1,l_2}$ to the set $\{j_i\}_{i=1}^{l_3}$ is always injective.
	\end{enumerate}
Let $m_{l_2}:{\mathbb{N}}_{l_2} \rightarrow \mathbb{N} \sqcup \{0\}$ be a map. Let $m:=n+{\Sigma}_{j=1}^{l_2} m_{l_2}(j)$. Then there exist an $m$-dimensional non-singular real algebraic closed and
connected manifold $M$ and a smooth real algebraic map $f:M \rightarrow {\mathbb{R}}^n$ with the image $f(M)$ being the closure $\overline{D}$, the set $f^{-1}(p)$ {\rm (}$p \in \overline{D}${\rm )} being a one-point set or a product of spheres and the property 
$\dim f^{-1}(p) \leq m-n$  {\rm (}$p \in \overline{D}${\rm )}.
	\end{MainThm}

This is an extension or a variant of main results of the author, presented in \cite{kitazawa3, kitazawa7}. There the hypersurfaces do not intersect. See also \cite{kitazawa4, kitazawa9}. Additional Main Theorems, presented in the third section, are on the functions represented as the compositions of the maps with the canonical projections and their global structures. \cite{kitazawa3} also studies this kind of problems. More precisely, these new theorems show explicit functions  represented as the compositions of maps into ${\mathbb{R}}^2$ of Main Theorem \ref{mthm:1} with each $S_j$ being a circle with the canonical projection ${\pi}_{2,1}$.
 %We need so-called {\it Reeb graphs}. They are graphs representing the functions compactly. 
%We give a remark on conditions in Main Theorem \ref{mthm:1}. The conditions (\ref{mthm:1.2}) and (\ref{mthm:1.3}) come from (\ref{mthm:1.4}). However we present them as a kind of additional exposition which may be redundant. 

We add exposition on Main Theorem \ref{mthm:1} (\ref{mthm:1.2}). This is on "transversality". From the condition on the tangent vector spaces, for any open set $U_D$ as in (\ref{mthm:1.1}), the subset $U_D \bigcap {\bigcap}_{i=1}^{l_3} S_{j_i}$ is an {\rm (}$n-l_3${\rm )}-dimensional smooth regular submanifold of $U_D \subset {\mathbb{R}}^n$ with no boundary and we have ${\bigcap}_{i=1}^{l_3} T_{p} S_{j_i}=T_p (U_D \bigcap {\bigcap}_{i=1}^{l_3} S_{j_i})$  {\rm (}if the subset is not empty{\rm )}. This subset is also a closed subset of $U_D \subset {\mathbb{R}}^n$. See \cite{golubitskyguillemin} for transversality.%In addition, some conditions will be dropped. However, we do not omit them here. 
%Related extensions are presented in Remark \ref{rem:2} and Example \ref{ex:1}, in the fourth section, shortly. 


The next section is on Main Theorem \ref{mthm:1}.
Main Theorems \ref{mthm:2} and \ref{mthm:3} are presented as an application in the third section. The fourth section presents additional remarks. \\
\ \\
\noindent {\bf Conflict of interest.} \\
The author was a member of the project JSPS Grant Number JP17H06128.
The author was a member of the project JSPS KAKENHI Grant Number JP22K18267. Principal Investigator for them is all Osamu Saeki.  The author works at Institute of Mathematics for Industry (https://www.jgmi.kyushu-u.ac.jp/en/about/young-mentors/) and this is closely related to our study. Our study thanks them for their supports. The author is also a researcher at Osaka Central
Advanced Mathematical Institute (OCAMI researcher), supported by MEXT Promotion of Distinctive Joint Research Center Program JPMXP0723833165. He is not employed there. This is for our studies
and our study also thanks this. \\
\ \\
{\bf Data availability.} \\
Data essentially supporting our present study are all in the
 paper. We also note that the present version is a revised version of the previous version \cite{kitazawa8}. We have revised based on our consideration. This does not affect our main ingredients. We have added some observations and exposition. We also have changed the title.   
\section{On Main Theorem \ref{mthm:1}.}
\subsection{Our proof of Main Theorem \ref{mthm:1}.}
%\subsection{Additional several terminologies and notions.}
%A {\it diffeomorphism} is a smooth homeomorphism with no singular points. A {\it diffeomorphism on a smooth manifold} is a diffeomorphism from the manifold to itself.
%%The {\it diffeomorphism type} of a smooth manifold is defined as the equivalence class under the natural equivalence relation. 
%This is defined on the family of all smooth manifolds defined by the existence of diffeomorphisms. Two manifolds whose diffeomorphism types are same are said to be {\it diffeomorphic}.

%The {\it diffeomorphism group} of a smooth manifold is the group of all diffeomorphisms there. This is topologized with the so-called {\it Whitney $C^{\infty}$ topology} and a so-called topological group. More generally, Whitney $C^{\infty}$ topologies on the set of all smooth maps between two given smooth manifolds and subspaces of the space are important in the (singularity) theory of smooth maps for example. See \cite{golubitskyguillemin} for example. This is a book for elementary and some advanced theory of singularity theory of differentiable maps.

%A {\it smooth} bundle is a bundle whose fiber is a smooth manifold and whose structure group is regarded as (some subgroup) of the diffeomorphism group of the fiber.

%We introduce {\it fold} maps. For the Whitney $C^{\infty}$ topology and fold maps for example, see also \cite{golubitskyguillemin} as a book for elementary singularity theory of differentiable maps
%for example.

%\begin{Def}
%	Let $X$ and $Y$ be smooth manifolds with no boundaries satisfying $\dim X \geq \dim Y$.
%	A {\it fold} map $c:X \rightarrow Y$ is a smooth map such that at each singular point $p$, we have suitable integer $0 \leq i(p) \leq \frac{\dim X-\dim Y+1}{2}$ and local coordinates around $p$ and $c(p)$ and that there we have a local form $c(x_1,\cdots,x_{\dim X})=(x_1,\cdots,x_{\dim Y-1},{\Sigma}_{j=1}^{\dim X-\dim Y-i(p)+1} {x_{\dim Y-1+j}}^2-{\Sigma}_{j=1}^{i(p)} {x_{\dim X-i(p)+j}}^2)$ around the singular point $p$ and the singular value $c(p)$.
%\end{Def}
%5\begin{Prop}
%	In the previous definition, $i(p)$ is chosen uniquely and defined as the {\rm index} of $p$. The set of all singular points of $c$ of a fixed index is a smooth regular submanifold of $c$ and of dimension $\dim Y-1$. If $X$ is closed, then the submanifold is compact and with no boundary. The restriction to the submanifold is a smooth immersion. 
%\end{Prop}
%5\begin{Def}
%	If in the definition of the fold map, $i(p)=0$ always holds, then this is called a {\it special generic} map.
%\end{Def}
%A Morse function is of course a fold map. 
%In short, fold maps are locally projections or the product map of a Morse function and the identity map on some disk. For special generic maps, the Morse function is chosen as a so-called {\it height function} of a unit disk. A {\it height function} $h$ of a unit sphere is a function of the form $h(x)=\pm ||x||^2+c$ where $c$ is some real number.

%We introduce very fundamental and explicit special generic maps. They are also keys in our main results.

%\begin{Ex}
%\label{ex:1}
%	\begin{enumerate}
	%	\item Canonical projections of unit spheres are special generic. The restrictions to the singular sets, which are also regarded as unit spheres, are embeddings. The images are regarded as the unit disks whose dimensions are same as those of the Euclidean spaces of the targets.
	%	\item Let $m>n$ be positive integers. 
	%	Let $M$ be an $m$-dimensional smooth manifold represented as a connected sum of $l> 0$ manifolds diffeomorphic to $S^{k_j} \times S^{m-k_j}$ for each integer $1 \leq j \leq l$ and some integer $1 \leq k_j \leq n-1$ where the connected sum is taken in the smooth category. We easily have a special generic map $f:M \rightarrow {\mathbb{R}}^n$ such that the restriction to the singular set $S(f)$ is an embedding and that the image is a smoothly embedded submanifold diffeomorphic to one represented as a boundary connected sum of $l> 0$ manifolds diffeomorphic to $S^{k_j} \times D^{n-k_j}$ for each integer $1 \leq j \leq l$. The boundary connected sum is, as before, taken in the smooth category. 
%		\end{enumerate}
%For these maps, we also have the following two trivial smooth bundles. We consider a case where $m$ and $n$ are the dimensions of the manifolds of the domain and the target.
%\begin{itemize}
%	\item We have some small collar neighborhood of the boundary of the image and the composition of the restriction of the map to the preimage with the canonical projection to the boundary gives a trivial smooth bundle whose fiber is diffeomorphic to a unit disk $D^{m-n+1}$. 
%	\item On the complementary set of the interior of the collar neighborhood, the restriction of the map gives a trivial smooth bundle whose fiber is diffeomorphic to a unit sphere $S^{m-n}$.
%\end{itemize}
%Moreover, they are glued by the product map of the diffeomorphism for the natural identification between the base spaces and the identity map on the fiber. Note that fibers are identified in some canonical way. For such special generic maps, see also the preprint \cite{kitazawa4, kitazawa7} of the author.
%\end{Ex}



%A {\it Morse-Bott} function is a smooth function on a manifold with no boundary at each singular point of which it is represented as the composition of some projection with a Morse function for suitable local coordinates. See \cite{bott}.

%As our main work, we construct real algebraic functions such that for each singular point, the singularity is as one of these cases (at least) topologically. We do not investigate these singularities in our main theorems. This presents another fundamental, important and difficult problem on singularities of polynomial maps or more generally, smooth maps.

%To simplify our arguments, let us assume the following where $l \geq 0$ is an integer.
%\begin{itemize}
%\item For each hypersurface $S_j$ in the family $\{S_j\}_{j=1}^l$, a real polynomial $f_{{\rm P}, S_j}$ is given so that the zero set and $S_j$ coincide and that the polynomial function $f_{{\rm P}, S_j}:S_j \rightarrow \mathbb{R}$ defined canonically has no singular points on $S_j$.
%\item $D$ is assumed to be the intersection ${\bigcap}_{j=1}^l  \{x \in {\mathbb{R}}^k \mid  f_{{\rm P}, S_j}(x)>0 \}$.
%\end{itemize}

%For example, the interior ${\rm Int}\ D^k$ of $D^k:=\{x \in {\mathbb{R}}^{k} \mid ||x|| \leq 1 \}$ is a simplest example and $D^k$ is the $k$-dimensional unit disk. This is also a $k$-dimensional smooth, compact and connected submanifold.
%Note that $||x||={\Sigma}_{j=1}^k {x_j}^2$ where $x:=(x_1,\cdots,x_k)$.

%A {\it Poincar\'e-Reeb graph} is defined for a pair of an algebraic domain $D$ of the real affine space of dimension $k>1$ and a canonical projection ${\pi}_{k,1}$ mapping $(x_1,x_2) \in {\mathbb{R}}^{k}$ to $x_1 \in {\mathbb{R}}$. This can be presented in a more general manner.
%Hereafter, we mainly respect the preprint \cite{bodinpopescupampusorea} and there such cases are discussed. Note that terminologies and situations are different in considerable cases and that here we can argue in a self-contained way.
%\begin{Def}
%\label{def:1}
%A {\it Poincar\'e-Reeb graph for the pair $(D,{\pi}_{k,1})$} is a graph in the real affine space embedded by a piecewise smooth embedding with the following conditions.
%\begin{itemize}
%\item Each edge $e$ intersects each preimage of the projection ${\pi}_{k,1}$ in a so-called {\it generic} way or satisfying the "transversality". In other words, each edge is embedded smoothly and for each point $p_e$ in each edge $e$, the image of the differential at the point and the tangent vector space at the value $v(p_e)$ in the preimage ${{\pi}_{k,1}}^{-1}(p)$ of a suitable {\rm (}unique{\rm )} point $p$ by the projection ${\pi}_{k,1}$ generate the tangent vector space at the point $v(p_e) \in {\mathbb{R}}^k$. 
%\item Two points in the closure $\overline{D}$ of $D$ can be defined to be equivalent if and only if they are in a same connected component of the preimage $\overline{D} \bigcap {{\pi}_{k,1}}^{-1}(p)$ for some point $p \in {\mathbb{R}}$ and the map obtained by the restriction of the projection to the closure $\overline{D}$. Let ${\pi}_{D}$ denote the restriction to  the closure $\overline{D}$.
%Our Poincar\'e-Reeb graph for the pair can be also defined as the quotient space obtained by this equivalence relation. This is isomorphic to the Reeb graph of ${\pi}_{D}$. Furthermore, an isomorphism is defined as the canonically obtained correspondence. 
%\%item The vertex set of our Poincar\'e-Reeb graph for the pair is the union of the set of all singular points of the restrictions of the projection ${\pi}_{k,1}$ or ${\pi}_D$ to all connected components of the boundary $\partial \overline{D} \subset \overline{D}$. This set is also finite. 
%\end{itemize}
%%\end{Def}
%See also \cite{sorea1, sorea2} for related theory for example. We present our main result. In the following section we prove this and present related comments as our main content.
%\begin{MainThm}
%5\label{mthm:1}
%Consider a Poincar\'e-Reeb graph $K$ for the pair in Definition \ref{def:1} such that the closure $\overline{D}$ is compact. Take an arbitrary integer $k_0>k+1$. Then we can construct a real algebraic function on some {\rm (}$k_0-1${\rm )}-dimensional smooth closed manifold regarded as a smooth real algebraic manifold whose Reeb graph is isomorphic to the graph $K$ as a graph.
%\end{MainThm}


%\cite{kitazawa3} presents most of our key tools. We review some important ingredients. We apply them. We also apply them in suitably improved ways in several scenes. 
%At present we do not find essential errors in the (accepted) version on which a positive report for publication has been announced to have been sent. However we will not publish the paper revising the accepted version essentially.

%Note that the work is motivated by \cite{bodinpopescupampusorea} with \cite{sorea1, sorea2}. \cite{bodinpopescupampusorea} studies {\it algebraic domains} collapsing to some graphs. An {\it algebraic domain} means an open set in an real affine space the boundary of whose closure is surrounded by smooth real algebraic hypersurfaces. 
%Originally our main theorem of \cite{kitazawa3} respects these studies.
%The work essentially related to such work is regarded as one of important future problems. However, we do not investigate related problems in our paper. For the related studies, it seems that we need more sophisticated or advanced knowledge and arguments on real algebraic geometry.

Hereafter, we use "$\prod$" for the products of numbers, functions or sets. We use the notation in the forms ${\prod}_{j=1}^l$ for a positive integer $l$ and ${\prod}_{x \in X}$ for a (finite) set $X$ for example. For coordinates and points for example, we use the notation like $x=(x_1,\cdots,x_l)$ with a positive integer $l$ for example.

\begin{Thm}\label{thm:1}
In Main Theorem \ref{mthm:1}, we can also construct a suitable map $f:M \rightarrow {\mathbb{R}}^n$ on a suitable manifold $M$ in such a way that each preimage $f^{-1}(p)$ is as follows.
\begin{enumerate}
\setcounter{enumi}{3}
	\item For $p \in D$, it is diffeomorphic to ${\prod}_{j=1}^{l_2} S^{m_{l_2}(j)}$.
	\item Let $\{j_i\}_{i=1}^{l_3} \subset {\mathbb{N}}_l$ be an increasing sequence of integers {\rm (}$1 \leq l_3 \leq l_1${\rm )}. Let $p$ be a point of the set ${\bigcap}_{i=1}^{l_3} S_{j_i} \bigcap \overline{D}$ such that for any increasing sequence $\{{j^{\prime}}_i\}_{i=1}^{l_3+1}$ containing the increasing sequence $\{j_i\}_{i=1}^{l_3}$ as a subsequence, $p \notin {\bigcap}_{i=1}^{l_3+1} S_{{j^{\prime}}_i}$ holds. Then it is diffeomorphic to ${\prod}_{j \in {\mathbb{N}}_{l_2}-\{m_{l_1,l_2}(j_i)\}_{i=1}^{l_3}} S^{m_{l_2}(j)}$ if ${\mathbb{N}}_{l_2}-\{m_{l_1,l_2}(j_i)\}_{i=1}^{l_3}$ is not empty and the one-point set if ${\mathbb{N}}_{l_2}-\{m_{l_1,l_2}(j_i)\}_{i=1}^{l_3}$ is empty.
\end{enumerate}
\end{Thm}
This is an extension of a main result of the author of \cite{kitazawa3}. 
There hypersurfaces do not intersect.
The result is also presented as a part of Theorem \ref{thm:1} of \cite{kitazawa4}. Some essential steps are same as those of their original proofs. 
\begin{proof}[A proof of Main Theorem \ref{mthm:1} with Theorem \ref{thm:1}]


 
We define a subset $S_0:=\{(x,y_1,\cdots,y_{l_2}) \in U_D \times {\prod}_{i=1}^{l_2} {\mathbb{R}}^{m_{l_2}(i)+1} \subset {\mathbb{R}}^n \times {\prod}_{i=1}^{l_2} {\mathbb{R}}^{m_{l_2}(i)+1}={\mathbb{R}}^{m+l_2} \mid {\prod}_{j \in {m_{l_1,l_2}}^{-1}(i)} (f_{j}(x_1,\cdots,x_n))-{||y_i||}^2=0, i \in {\mathbb{N}}_{l_2}\} \subset {\mathbb{R}}^{m+l_2}$. Here $y_i:=(y_{i,1},\cdots,y_{i,m_{l_2}(i)+1})$. 
\\
\ \\
STEP 0 The subset $S_0$ is not empty. \\
The connected open set $D$ is assumed to be non-empty.
We remember the condition (\ref{mthm:1.3}) and the map  $m_{l_1,l_2}:{\mathbb{N}}_{l_1} \rightarrow {\mathbb{N}}_{l_2}$.

The set ${m_{l_1,l_2}}^{-1}(i)$ is not empty for any $i \in {\mathbb{N}}_{l_2}$ since $m_{l_1,l_2}$ is assumed to be surjective. 
%The set $S_j \bigcap \overline{D}$ is assumed to be non-empty. 

They imply that the set $S_0$ is not empty. \\

%Here we may replace $||y||^2$ by a general polynomial, represented in the form ${\Sigma}_{j=1}^{n^{\prime}-n} a_j {y_{j}}^{b_j}$ where $a_j$ and $b_j$ are a positive integer and a positive even integer, respectively.  
\ \\
STEP 1 A proof of the fact by implicit function theorem that the set $S_0$ is a smooth compact submanifold with no boundary. \\
We calculate the partial derivatives of the real polynomial function defined by the real polynomial ${\prod}_{j \in {m_{l_1,l_2}}^{-1}(i)}  (f_{j}(x_1,\cdots,x_n))-||y_i||^2={\prod}_{j \in {m_{l_1,l_2}}^{-1}(i)}  (f_{j}(x_1,\cdots,x_n))-{\Sigma}_{j^{\prime}=1}^{m_{l_2}(i)+1} {y_{i,j^{\prime}}}^2$ for variables $x_j$ and $y_{i,j^{\prime}}$ ($1 \leq i \leq l_2$). In STEP 1-1 and STEP 1-2, we apply implicit function theorem around each point of $S_0$. After that, in STEP 1-3, we give additional arguments to complete our proof of the fact that the set $S_0$ is a smooth compact submanifold with no boundary. \\
\ \\
STEP 1-1 The case for a point $(x_0,y_0) \in S_0 \subset U_D \times {\mathbb{R}}^{m-n+l_2}$ such that for $y_0:=(y_{0,1},\cdots,y_{0,l_2})$, $y_{0,j}$ is not the origin for any $1 \leq j \leq l_2$. \\
%By the assumption on the sets $D$ and $U_D$ we have $x_0 \in D$ and
We have the inequality $f_{j}(x_0)>0$ for any $1 \leq j \leq l_1$. This is thanks to our first condition (\ref{mthm:1.1}). As a result we also have the inequality ${\prod}_{j \in {m_{l_1,l_2}}^{-1}(i)}  f_{j}(x_0)>0$ for each integer $1 \leq i \leq l_2$. 
%We use the notation $x_0:=(x_{0,1},\cdots,x_{0,n})$ and $y_0:=(y_{0,1},\cdots,y_{0,n^{\prime}-n})$ for example as before. 
We have a real number $2y_{i,{j_0}^{\prime}}=2y_{0,i,{j_0}^{\prime}} \neq 0$ as the value of the partial derivative at the point $(x_0,y_0)$ for some variable $y_{i,{j_0}^{\prime}}$: here, $y_{0,i,{j_0}^{\prime}}$ is from $y_{0,i}:=(y_{0,i,1},\cdots,y_{0,i,m_{l_2}(i)+1})$ with $y_0:=(y_{0,1},\cdots,y_{0,l_2})$.
Let us abuse the same notation. We also have $0$ as the value of the partial derivative at the point for any variable $y_{i^{\prime},{j}^{\prime}}$ satisfying $i^{\prime} \neq i$.

The differential of the restriction of the map into ${\mathbb{R}}^{l_2}$ defined canonically from the $l_2>0$ real polynomials ${\prod}_{j \in {m_{l_1,l_2}}^{-1}(i)}  (f_{j}(x))-||y_i||^2$ ($1 \leq i \leq l_2$) at $(x_0,y_0)$ is of rank $l_2$. 
%Remember the inequality ${\prod}_{j \in {m_{l_1,l_2}}^{-1}(i)}  (f_{j}(x_0))>0$.
The point is not a singular point of the map. Moreover, around the point $(x_0,y_0)$, the set $S_0$ is represented as the graph of a smooth map as follows. The variable of the target of the map consists of exactly $l_2$ components $y_{i,{j_0}^{\prime}}$ ($1 \leq i \leq l_2$). 
\\
\ \\
STEP 1-2 The case for a point $(x_{\rm O},y_{\rm O}) \in S_0 \subset U_D \times {\mathbb{R}}^{m-n+l_2}$ such that for $y_{\rm O}:=(y_{{\rm O},1},\cdots,y_{{\rm O},l_2}) \in {\mathbb{R}}^{m-n+l_2}$, $y_{{\rm O},j}$ is the origin for some $1 \leq j \leq l_2$. \\
By the condition (\ref{mthm:1.1}), we can also assume that the fact $x_{\rm O} \in {\bigcap}_{i=1}^{l_3} S_{j_i} \bigcap \overline{D}$ holds for some increasing sequence $\{j_i\}_{i=1}^{l_3}$ and that the fact $x_{\rm O} \notin {\bigcap}_{i=1}^{l_3+1} S_{{j^{\prime}}_i}$ holds for any increasing sequence $\{{j^{\prime}}_i\}_{i=1}^{l_3+1}$ containing the sequence $\{j_i\}_{i=1}^{l_3}$ as a subsequence.

We define a set of all points 
${x_{\rm O}}^{\prime} \in {\bigcap}_{i=1}^{l_3} S_{j_i} $ such that the fact ${x_{\rm O}}^{\prime} \notin {\bigcap}_{i=1}^{l_3+1} S_{{j^{\prime}}_i}$ holds for any increasing sequence $\{{j^{\prime}}_i\}_{i=1}^{l_3+1}$ containing the sequence $\{j_i\}_{i=1}^{l_3}$ as a subsequence.
Let $S_{0,\{j_i\}_{i=1}^{l_3}}$ denote the set. It is a subset of the ($n-l_3$)-dimensional manifold ${\bigcap}_{i=1}^{l_3} S_{j_i}$ of course.

By the condition (\ref{mthm:1.1}),
for each point ${x_{\rm O}}^{\prime} \in S_{0,\{j_i\}_{i=1}^{l_3}} \bigcap \overline{D}$ and a sufficiently small open neighborhood $U_{{x_{\rm O}}^{\prime},{\bigcap}_{i=1}^{l_3} S_{j_i}}$ of it discussed and chosen in the space ${\bigcap}_{i=1}^{l_3} S_{j_i}$, we have the relation $U_{{x_{\rm O}}^{\prime},{\bigcap}_{i=1}^{l_3} S_{j_i}} \subset S_{0,\{j_i\}_{i=1}^{l_3}} \bigcap \overline{D}$. 

The set $S_{0,\{j_i\}_{i=1}^{l_3}}$ and $S_{0,\{j_i\}_{i=1}^{l_3}} \bigcap \overline{D}$
 are open sets in ${\bigcap}_{i=1}^{l_3} S_{j_i}$ and ($n-l_3$)-dimensional manifolds with no boundary. They are also submanifolds of the ($n-l_3$)-dimensional manifold ${\bigcap}_{i=1}^{l_3} S_{j_i}$. Furthermore, the manifold ${\bigcap}_{i=1}^{l_3} S_{j_i}$ is a regular submanifold and a closed subset of ${\mathbb{R}}^n$ and has no boundary.

%there exists a point $p_D \neq p$ in $D$ such that the distance of $p$ and $p_D$ is smaller than $\frac{\epsilon}{2}$. For any point $p_{S}$ in any sufficiently small open neighborhood of $p$ in the ($n-l_3$)-dimensional regular submanifold $U_D \bigcap {\bigcap}_{i=1}^{l_3} S_{j_i} \subset U_D \subset {\mathbb{R}}^n$ of $U_D$ with no boundary, the distance of $p_{S}$ and $p_D$ is smaller than $\epsilon$. We can see the fact $p_{S} \in \overline{D}$. We can also see that  the set ${\bigcap}_{i=1}^{l_3} S_{j_i} \bigcap \overline{D}$
% is also an ($n-l_3$)-dimensional manifold and a submanifold of the manifold $U_D \bigcap {\bigcap}_{i=1}^{l_3} S_{j_i} \subset U_D \subset {\mathbb{R}}^n$ if the set ${\bigcap}_{i=1}^{l_3} S_{j_i} \bigcap \overline{D}$ is non-empty. The set ${\bigcap}_{i=1}^{l_3} S_{j_i}-\overline{D}$ is also an open subset of ${\bigcap}_{i=1}^{l_3} S_{j_i}$ and as a result the set ${\bigcap}_{i=1}^{l_3} S_{j_i} \bigcap \overline{D}$ is also a closed subset of ${\bigcap}_{i=1}^{l_3} S_{j_i}$ and $U_D \bigcap {\bigcap}_{i=1}^{l_3} S_{j_i}$. The set of all points $p_{S,D}$ of ${\bigcap}_{i=1}^{l_3} S_{j_i} \bigcap \overline{D}$ such that for any increasing sequence $\{{j^{\prime}}_i\}_{i=1}^{l_3+1}$ containing the sequence $\{j_i\}_{i=1}^{l_3}$ as a subsequence the fact $p_{S,D} \notin {\bigcap}_{i=1}^{l_3+1} S_{{j^{\prime}}_i}$ holds is an open subset of ${\bigcap}_{i=1}^{l_3} S_{j_i} \bigcap \overline{D}$. % It is also an open subset of ${\bigcap}_{i=1}^{l_3} S_{j_i}$. 
This argument is essential in Theorem \ref{thm:2}, presented later.



%We use the notation like $x_0:=(x_{0,1},\cdots,x_{0,k})$ and $y_0:=(y_{0,1},\cdots,y_{0,k})$ before. 
%By the assumption, for the real polynomials $f_{j}(x)$, $f_{j}(x_{\rm O})> 0$ for $j \neq a$. The real polynomial function defined canonically from the real polynomial $f_{a}$ is assumed to have no singular points on the hypersurface $S_{a}$.

	

Let $i_y \in \mathbb{N}_{l_2}-m_{l_1,l_2}(\{j_i\}_{i=1}^{l_3})$. We investigate partial derivatives of the function defined by the real polynomial ${\prod}_{j \in {m_{l_1,l_2}}^{-1}(i_y)} (f_{j}(x_1,\cdots,x_n))-||y_{i_y}||^2={\prod}_{j \in {m_{l_1,l_2}}^{-1}(i_y)} (f_{j}(x_1,\cdots,x_n))-{\Sigma}_{j^{\prime}=1}^{m_{l_2}(i_y)+1} {y_{i_y,j^{\prime}}}^2$. More precisely, we investigate the partial derivative for each variable $y_{i_1,j^{\prime}}:=y_{i_y,j^{\prime}}$ at $(x_{\rm O},y_{\rm O})$.
We have a real number $2y_{i_{1},{j_0}^{\prime}}=2y_{0,i_{1},{j_0}^{\prime}} \neq 0$ for some variable $y_{i_{1},{j_0}^{\prime}}$ at $(x_{\rm O},y_{\rm O})$ where we abuse the notation in STEP 1-1 for example. Let us abuse the same notation. We also have $0$ as the value of the partial derivative at the point for any variable $y_{i_2,{j}^{\prime}}$ satisfying $i_2 \neq i_1$. This is like an argument in STEP 1-1. This counts $l_2-l_3$ of the rank of the differential of the restriction of the map into ${\mathbb{R}}^{l_2}$ defined canonically from
the $l_2>0$ real polynomials. In other words, we choose the $l_2-l_3$ real polynomials from the $l_2$ real polynomials choosing such $l_2-l_3$ elements $i_y$. Remember also that the restriction of $m_{l_1,l_2}$ to $\{j_i\}_{i=1}^{l_3}$ is assumed to be injective in the condition (\ref{mthm:1.3}). 

We choose an integer $i_x \in m_{l_1,l_2}(\{j_i\}_{i=1}^{l_3})$. We discuss partial derivatives of the function defined canonically from the real polynomial ${\prod}_{j \in {m_{l_1,l_2}}^{-1}(i_x)} (f_{j}(x_1,\cdots,x_n))-||y_{i_x}||^2={\prod}_{j \in {m_{l_1,l_2}}^{-1}(i_x)} (f_{j}(x_1,\cdots,x_n))-{\Sigma}_{j^{\prime}=1}^{m_{l_2}(i_x)+1} {y_{i_x,j^{\prime}}}^2$.
%We explain about partial derivatives of the function defined canonically from the real polynomial ${\prod}_{j \in {m_{l_1,l_2}}^{-1}(i_x)} (f_{j}(x_1,\cdots,x_n))-||y_{i_x}||^2={\prod}_{j \in {m_{l_1,l_2}}^{-1}(i_x)} (f_{j}(x_1,\cdots,x_n))-{\Sigma}_{j^{\prime}=1}^{m_{l_2}(i_x)} {y_{i_x,j^{\prime}}}^2$ for each $i_x \in m_{l_1,l_2}(\{j_i\}_{i=1}^{l_3})$. 
We investigate the partial derivatives for the variables $y_{i_1,{j_0}^{\prime}}$ and $y_{i_2,{j}^{\prime}}$, defined just before. We always have $0$ as the values by our definition. 
More precisely, by our definition, at $(x_{\rm O},y_{\rm O})$, $y_{i_x}$ is the origin. We also see that the value of the partial derivative for any variable $y_{i,j}$ is $0$ at the point $(x_{\rm O},y_{\rm O})$.
We also discuss the partial derivative for each variable $x_{i_3}$. For the integer $i_x$, we have the unique number $j_{i_x} \in \{j_i\}_{i=1}^{l_3}$ satisfying $i_x=m_{l_1,l_2}(j_{i_x})$. This follows from the condition (\ref{mthm:1.3}) that the restriction of $m_{l_1,l_2}$ to $\{j_i\}_{i=1}^{l_3}$ is injective. 
The value of the partial derivative is represented as the product of the following two numbers.
\begin{itemize}
	\item The value of the partial derivative of the function $f_{j_{i_x}}(x_1,\cdots,x_n)$ for the variable $x_{i_3}$ at $x_{\rm O}$. 
	Each given hypersurface $S_j$ is assumed to be non-singular and the zero set of the real polynomial $f_j$. 
	Thus the value of the partial derivative is not $0$ for some variable $x_{i_3}:=x_{i_{3,0}}$ ($1 \leq i_{3,0} \leq n$).
	\item The product of the real numbers obtained as the values of the canonically defined real polynomial functions in the family $\{f_{j}(x_1,\cdots,x_n)\}_{j \in {m_{l_1,l_2}}^{-1}(i_x)-\{j_{i_x}\}}$ at $x_{\rm O}$. By our definitions and assumptions with our construction, we can see that this is not $0$.
	\end{itemize}
In addition, remember the assumption on the transversality on intersections for hypersurfaces $S_j$ or the condition (\ref{mthm:1.2}). 
This counts $l_3$ of the rank of the differential of the restriction of the map into ${\mathbb{R}}^{l_2}$ defined canonically from
the $l_2>0$ real polynomials. 
In other words, we choose the $l_3$ real polynomials from the $l_2$ real polynomials respecting such elements $i_x$ and $j_{i_x}$.
This also counts "$l_3$" of the rank independently from the "$l_2-l_3$" before. This is thanks to the following.

\begin{itemize} 
\item At $(x_{\rm O},y_{\rm O})$, $y_{i_x}$ is the origin.
\item For the $l_3$ polynomials the values of the partial derivatives for the variables $y_{i_1,{j_0}^{\prime}}$, $y_{i_2,{j}^{\prime}}$ and an arbitrary variable $y_{i,j}$ are always $0$.
\end{itemize} 

Integrating these arguments, we can see that at $(x_{\rm O},y_{\rm O})$, the differential of the restriction of the map into ${\mathbb{R}}^{l_2}$ defined canonically from
 the $l_2>0$ real polynomials is of rank $l_2>0$.
%satisfying the inequality ${\prod}_{j \in {m_{l_1,l_2}}^{-1}(i)} (f_{j}(x_{\rm O}))>0$ at $(x_{\rm O},y_{\rm O})$ is of rank $l_2$. This is not a singular point of the map.
We have a result like the one in STEP 1-1. \\
\ \\
STEP 1-3 Additional arguments to show that the set $S_0$ is a smooth compact submanifold with no boundary. \\
By the assumption on $U_D$, which is an arbitrary sufficiently small connected open neighborhood of the closure $\overline{D}$ of a nicely given connected open set $D \subset {\mathbb{R}}^n$, we can see that we cannot take a point $x \in U_D-\overline{D}$ as the component of $(x,y) \in S_0$. We discuss this precisely.
First we choose an arbitrary point $x_{U} \in U_D-\overline{D}$. By the rule of choosing our open neighborhood $U_D$ of the compact and connected set $\overline{D}$,
we can discuss $U_D$ and related important subsets as follows.
\begin{itemize}
\item For each point $p$ in $\overline{D}-D \subset {\bigcup}_{j=1}^l S_j$, we can choose a sufficiently small open disk $D_p$ in ${\mathbb{R}}^n$ and we do. We can also regard $p$ as a point as follows.
\begin{itemize}
\item It holds that $p \in {\bigcap}_{i=1}^{i_{p,0}} S_{j_i} \bigcap \overline{D}$ for some increasing sequence $\{j_i\}_{i=1}^{i_{p,0}}$.
\item It also holds that $p \notin {\bigcap}_{i=1}^{i_{p,0}+1} S_{{j^{\prime}}_{i}}$ for any increasing sequence $\{{j^{\prime}}_i\}_{i=1}^{i_{p,0}+1}$ containing the sequence $\{j_i\}_{i=1}^{i_{p,0}}$ as a subsequence. 
\end{itemize}
Remember our definitions and conditions on the compact set $
\overline{D}$ and the zero sets $S_j$
for example. We can also say that at any point of $D_p$, the value of $f_j$ is positive for any $j \notin \{j_i\}_{i=1}^{i_{p,0}}$.
\item %For the open set $D \bigcup {\bigcup}_{p \in \overline{D}-D} D_p$ and 
For the compact and connected subset $\overline{D}$, we can find the open cover $\{D\} \sqcup \{D_p\}_{p \in \overline{D}-D}$ . We can choose finitely many open sets to cover this compact and connected set $\overline{D}$. We choose such sets.
\item (Different from "the original $U_D$",) we can choose $U_D$ as the open set $U_D:={U_{D}}^{\prime}$ as the union of the finitely many open sets chosen before instead. It can and must be chosen as a connected set by our definitions, conditions and construction. We may also choose our new $U_D:={U_D}^{\prime \prime}$ as an arbitrary open connected subset of ${U_D}^{\prime}$ if it is also an open neighborhood of $\overline{D}$.
\end{itemize}

By the condition that the restriction of $m_{l_1,l_2}$ to $\{j_i\}_{j=1}^{l_3}$ is injective, we can choose an arbitrary point $x_{U} \in U_D-\overline{D}$ in such a way that we also have the following.
\begin{itemize}
\item For some sufficiently small open disk $D_{p_0}$ in the family of the chosen finitely many open disks just before, $x_U \in D_{p_0}$.
\item There exists an integer $1 \leq i_{x_{U,0}} \leq l_2$ with the following properties.
\begin{itemize} 
\item It holds that $f_{j}(x_{U})>0$ for each $j \in  {m_{l_1,l_2}}^{-1}(i_{x_{U,0}})$ except one number $j=j_{x_{U,0}} \in {m_{l_1,l_2}}^{-1}(i_{x_{U,0}})$.
\item It also holds that $f_{j_{x_{U,0}}}(x_{U})<0$.
\end{itemize}
\end{itemize}
Here we also see that the value ${\prod}_{j \in {m_{l_1,l_2}}^{-1}(i_{x_{U,0}})} f_{j}(x_{U})$ is smaller than $0$. 

This means that we cannot take a point $x \in U_D-\overline{D}$ satisfying $(x,y) \in S_0$.

By the argument and STEPs 1-1 and 1-2 with implicit function theorem, the set $S_0$ is an $m$-dimensional smooth compact and connected submanifold in $U_D \times {\mathbb{R}}^{m-n+l_2}$ and has no boundary. This is also regarded as a non-singular real algebraic manifold in the real affine space ${\mathbb{R}}^{m+l_2}$. \\
\ \\
STEP 2 Our desired map $f:M \rightarrow {\mathbb{R}}^n$ on $M:=S_0$. \\
We restrict the canonical projection ${\pi}_{m+l_2,n}$ to $M$. Thus we have a smooth real algebraic map $f:M:=S_0 \rightarrow {\mathbb{R}}^n$. We check that $f$ is our desired map.
 
Thanks to our previous arguments, our non-singular real algebraic manifold $M=S_0$ is a connected component of the intersection of finitely many zero sets of real polynomials (in the real affine space).
STEP 1-3 and our definition of $S_0$ also show that the image $f(M)$ is a subset of $U_D$ containing $\overline{D}$ as a subset. Each point of $U_D-\overline{D}$ is also shown to be outside the image. We can see $f(M)=\overline{D}$. 
On the smooth manifolds of the preimages (in Theorem \ref{thm:1}), we can also easily see our desired fact from our definitions, conditions and construction. 

We can see that $f$ is our desired map. \\
%We investigate $S_0$ and show that this is a smoot
% 
%First we consider a point $p_1 \in D$ and a point $(p_1,q_1) \in \overline{D^{\prime}}$ and take its sufficiently small open neighborhood $U_{p_1,q_1}$ in ${\mathbb{R}}^{n^{\prime}}$. 
%By the definition and the assumption, 
%$U_{p_1,q_1} \bigcap \{(x_1,\cdots,x_n,\cdots,x_{n^{\prime}}) \mid f^{\prime}(x_1,\cdots,x_n,\cdots,x_{n^{\prime}})>0\}=U_{p_1,q_1} \bigcap D^{\prime}$ and $U_{p_1,q_1} \bigcap \{(x_1,\cdots,x_n,\cdots,x_{n^{\prime}}) \mid f^{\prime}(x_1,\cdots,x_n,\cdots,x_{n^{\prime}}) \geq 0\}=U_{p_1,q_1} \bigcap \overline{D^{\prime}}$ hold.

%Second we consider a point $p_2 \in \partial \overline{D}$ in the boundary $\partial \overline{D} \subset \overline{D}$ and a point $(p_2,q_2) \in \overline{D^{\prime}}$. We take
 %its sufficiently small open neighborhood $U_{p_2,q_2}$ in ${\mathbb{R}}^{n^{\prime}}$. By the definition and the assumption on the hypersurfaces $S_j$ and the real polynomials, we have a similar observation.

%This completes the proof of (\ref{thm:1.2}).

%By observing the structures of the maps and the manifolds, we have (\ref{thm:1.4}).

%By the construction, the canonical projection onto $\overline{D}$ gives a desired smooth real algebraic map.
This completes the proof.
	
\end{proof}

We can know the following easily by the construction and local structures of the functions and the maps. We can also say that this is also thanks to an argument presented in STEP 1-2 and announced to be essential there. 

\begin{Thm}
\label{thm:2}
In Main Theorem \ref{mthm:1} and Theorem \ref{thm:1}, we can have our map $f:M \rightarrow {\mathbb{R}}^n$ enjoying the following property.
\begin{enumerate}
\item The image $f(S(f))$ of the singular set of $f$ satisfies the relation $f(S(f))=\overline{D}-D$. 
\item Choose any point $p \in {\bigcap}_{i=1}^{l_3} S_{j_i}$ such that for any increasing sequence $\{{j^{\prime}}_i\}_{i=1}^{l_3+1}$ containing the sequence $\{j_i\}_{i=1}^{l_3}$, the condition $p \notin {\bigcap}_{i=1}^{l_3+1} S_{{j^{\prime}}_i}$ is satisfied. Each point $q \in f^{-1}(p)$ satisfies the relation ${df}_q(T_qM)=T_p({\bigcap}_{i=1}^{l_3} S_{j_i})$.
\end{enumerate}
\end{Thm}

Note again that \cite{kitazawa3} presents a specific case for Main Theorem \ref{mthm:1}. These hypersurfaces $S_j$ do not intersect and $l_2=1$ there.

\subsection{A remark on structures of obtained maps.}

We present exposition on global structures of obtained maps. We do not need to understand this rigorously in the present paper.

In the case where the hypersurfaces do not intersect, the obtained map is a so-called {\it special generic} map. There exist such a nice map $f_0:M \rightarrow {\mathbb{R}}^n$ and a diffeomorphism $\Phi:M \rightarrow M$ enjoying the relation $f_0 \circ \Phi=f$.
It is first presented as a remark which is not essential in the paper.
Discussing this rigorously and carefully, the author has found that this may not be trivial. \cite{kitazawa8} proves this rigorously first, after earlier versions of the present paper have appeared.

In the case where the hypersurfaces may intersect, the map is (, at least topologically,) locally regarded as a so-called {\it moment map}.

Here we adopt methods for some presentations in \cite{kitazawa7} for example.

{\it Special generic} maps are, in short, higher dimensional versions and generalizations of the canonical projections of the unit spheres and Morse functions on spheres with exactly two singular points.
For special generic maps, see \cite{saeki1} and see also recent preprints \cite{kitazawa5, kitazawa10} of the author. Of course, arguments and results in \cite{kitazawa5, kitazawa10} and \cite{kitazawa6}, presented later, are independent of our study and we do not need to understand them precisely. We only give related short remark.

First, see \cite{buchstaberpanov, delzant} for moment maps on so-called {\it symplectic toric} manifolds. \cite{kitazawa6} is a preprint introducing a certain class of smooth maps generalizing the class of special generic maps first. There the class of {\it simply generalized special generic} maps has been introduced. It also investigates their topological properties, especially the cohomology rings of the manifolds. This respects and extends the results of \cite{kitazawa10}. These maps are locally moment maps. We can have maps locally, at least topologically, special generic maps or simply generalized special generic maps being not special generic in cases where the hypersurfaces $S_j$ do not intersect. Our main result of \cite{kitazawa3, kitazawa9} is for an explicit and simplest case and for special generic cases.

For related singularity theory of differentiable maps, see \cite{golubitskyguillemin} for example. As presented in the assumption of Main Theorem \ref{mthm:1}, the notion "transversality" is also from this theory, for example.

For Morse-Bott functions, see \cite{bott} for example. We present its definition in (the second subsection of) the third section as a kind of short review. Some Morse-Bott functions are generalized to the presented maps as higher dimensional versions. 

For here, see also \cite{kohnpieneranestadrydellshapirosinnsoreatelen}. This is on explicit classifications of regions surrounded by non-singular real algebraic hypersurfaces intersecting satisfying the condition on the "transversality" like our case. Related to this, \cite{bodinpopescupampusorea, sorea1, sorea2} are mainly on regions surrounded by non-singular real algebraic hypersurfaces with no intersections. This appears explicitly in Example 1 and FIGURE 1 of \cite{kitazawa3} and some of the figures are also presented in the
next section.


\section{Applications of Main Theorem \ref{mthm:1} and Theorems \ref{thm:1} and \ref{thm:2}: functions represented as the compositions of maps obtained through (arguments in our proof of) them with the canonical projections.}
\subsection{Graphs and Reeb graphs.} 
({\it Reeb}) {\it graphs} are our fundamental tools. 

A {\it graph} can be defined as a $1$-dimensional CW complex with the {\it vertex set} and the {\it edge set}. They are the set of all $0$-dimensional cells and the set of all $1$-dimensional cells, respectively. A {\it vertex} is an element of the vertex set. An {\it edge} is an element of the edge set. 
The closure of an edge homeomorphic to a circle is a {\it loop}. We do not discuss graphs having loops. In other words, a graph is always regarded as a $1$-dimensional simplicial complex and a polyhedron. Furthermore, it is {\it finite}. In a word, its vertex set and edge set are finite. On the other hand, a graph may be a so-called multi-graph. This means that a graph may have an edge connecting two distinct vertices. An {\it isomorphism} from a graph $K_1$ onto another graph $K_2$ means a piecewise smooth homeomorphism mapping the edge set and the vertex set of $K_1$ onto those of $K_2$. This defines a natural equivalence relation on the family of all graphs here. Two graphs $K_1$ and $K_2$ are {\it isomorphic} if an isomorphism from $K_1 $ onto $K_2$ exists. 

For a smooth function $c:X \rightarrow \mathbb{R}$, we can define an equivalence relation by the rule that two points $x_1$ and $x_2$ in $X$ are equivalent if and only if they are in a same connected component of a preimage $c^{-1}(y)$ ($y \in \mathbb{R}$). Let $W_c:=X/ {\sim}_c$ denote the quotient space of $X$ under this relation $\sim c$. Let $q_c:X \rightarrow W_c$ denote the quotient map.

\begin{Def}
If the Reeb space $W_c$ has the structure of a graph by the following rule, then $W_c$ is the {\it Reeb graph} of $c$. A point $p$ is a vertex if and only if ${q_c}^{-1}(p)$ contains some singular points of $c$.
\end{Def}
We introduce Theorem 3.1 of \cite{saeki2}.
For a smooth function on a compact manifold having finitely many singular values of it, the quotient space $W_c$ is homeomorphic to a graph. If the manifold of the domain is closed, then we can define the Reeb graph $W_c$ of $c$. {\rm Morse}({\rm -Bott}) functions and functions of some considerably wide classes satisfy such conditions.

The Reeb graph of a smooth function has been already defined in \cite{reeb} for example. The Reeb graphs of nice smooth functions have been important tools and objects in the theory on singularities and applications to geometry. These graphs have some important information on the manifolds compactly.

\subsection{Functions represented as the compositions of maps obtained through Main Theorem \ref{mthm:1} and Theorems \ref{thm:1} and \ref{thm:2} with the canonical projections to the $1$-dimensional real affine space.}


%\begin{figure}
	
%	\includegraphics[height=75mm, width=100mm]{20230819forMT2.1.eps}

%	\caption{The image of the map $f:M \rightarrow {\mathbb{R}}^2$ constructed by using Main Theorem \ref{mthm:1} and the Reeb graph of the resulting function in Main Theorem \ref{mthm:2} (in the case $G:=G_{{\rm l,r},l}$).}
%	\label{fig:1}
%\end{figure}

%\begin{figure}
	
%	\includegraphics[height=75mm, width=100mm]{20230801forMT2.2.eps}

%	\caption{Local structure around the arrow in FIGURE \ref{fig:1} and the Reeb graph.}
%	\label{fig:2}
%\end{figure}

%Note also that to obtain desired graphs as Reeb graphs, connectedness of preimages is essential. $\cdots$.
%The assumption that the dimensions of the manifolds of the domains are sufficiently high is for this. More precisely, the values of $m_{l_2}$ are chosen as sufficiently large integers, for example.

%This completes our proof.
%\end{proof}

%We give some remarks.

%For example, (the right figure in) FIGURE 1 of \cite{kitazawa3} shows a graph $G:=G_{{\rm l,r},l}$ in Main Theorem \ref{mthm:2}.

%According to (our proof of) Main Theorem \ref{mthm:1} with Theorem \ref{thm:1} and \ref{thm:2}, we can apply the theorems in such a way that the following facts hold. The preimage $f^{-1}(p)$ is diffeomorphic to the product of two (unit) spheres for $p \in D$ where we abuse the notation from Main Theorem \ref{mthm:1}. $l_2=2$ here. $f^{-1}(p)$ is diffeomorphic to a unit sphere for $p \in S_j$ contained in exactly one sphere in the family $\{S_j\}$ where we abuse the notation. $f^{-1}(p)$ is a one-point set for $p \in S_{j_1} \bigcap S_{j_2}$ where the notation is abused with the condition $j_1 \neq j_2$. We can also set the situation that the values of $m_{l_2}$ are always 1 for example for Main Theorem \ref{mthm:2}: in this case $m=n+2$. 


We discuss functions represented as the compositions of maps obtained through (arguments in our proof of) Main Theorem \ref{mthm:1} and Theorems \ref{thm:1} and \ref{thm:2}.
We mainly investigate cases where the hypersurfaces $S_j$ are circles in the plane ${\mathbb{R}}^2$.
 
Hereafter, we need elementary knowledge on Morse(-Bott) functions. We omit the definition of a Morse function. A {\it Morse-Bott} function is a smooth function which is represented as the composition of a submersion with a Morse function around each singular point (for suitable local coordinates).


% Main Theorem \ref{mthm:2} has been presented before our explicit generalization as a kind of examples or exrecises.

Hereafter, a {\it circle} means a circle in ${\mathbb{R}}^2$ centered at a point $p_0:=(p_{0,1},p_{0,2}) \in {\mathbb{R}}^2$ and of a radius $r_0>0$ unless otherwise stated. This is represented as the zero set of the real polynomial ${(x_1-p_{0,1})}^2+{(x_2-p_{0,2})}^2-{r_0}^2$ ($x=(x_1,x_2) \in {\mathbb{R}}^2$). Of course this is non-singular. We call the point $(p_{0,1} \pm r,p_{0,2})$ ($(p_{0,1},p_{0,2} \pm r)$) a {\it vertical} (resp. {\it horizontal}) {\it pole} of the circle. 

\begin{Prop}
\label{prop:1}
In Main Theorem \ref{mthm:1} and Theorems \ref{thm:1} and \ref{thm:2}, let $n=2$ and suppose that the family $\{S_j\}_{j=1}^{l_1} \subset {\mathbb{R}}^2$ is a family of mutually disjoint circles and that the relation $\overline{D}-D={\sqcup}_{j=1}^{l_1} S_j$ holds. We do our construction of $f:M \rightarrow {\mathbb{R}}^2$ as presented in our proof. The function $f_0:={\pi}_{2,1} \circ f:M \rightarrow \mathbb{R}$ enjoys the following.
\begin{enumerate}
\item Let $S_j$ be a circle in ${\mathbb{R}}^2$ centered at a point $p_j:=(p_{j,1},p_{j,2}) \in {\mathbb{R}}^2$ and of a radius $r_j>0$.
The singular set $S(f_0) \subset M$ of the function $f_0$ is represented as $f^{-1}({\sqcup}_{j=1}^{l_1} \{(p_{j,1}-r_j,p_{j,2}),(p_{j,1}+r_j,p_{j,2})\})$.
\item The function $f_0$ is a Morse-Bott function. In the case $l_2=1$, we can have our function $f_0$ as a Morse function.
\end{enumerate}

\end{Prop}

\begin{proof}
Most of our idea for our proof is presented in our preprint \cite{kitazawa9} and the proof of "Main Theorem \ref{mthm:2} there".
We can also argue from "Discussion 14 of \cite{kollar}" as in \cite{kitazawa9}. We can discuss \cite{kollar} independently. We do not need to understand the arguments of the preprint \cite{kitazawa9}.

By the presented arguments, the singular set $S(f_0)$ is represented as a subset of $f^{-1}({\sqcup}_{j=1}^{l_1} \{(p_{j,1}-r_j,p_{j,2}),(p_{j,1}+r_j,p_{j,2})\})$. We can also understand this from our construction here.

Let $1 \leq j_0 \leq l_1$ be an integer. Let us abuse the notation $y=(y_1,\cdots y_{l_2})$, used in our previous proof of Main Theorem \ref{mthm:1} with Theorems \ref{thm:1} and \ref{thm:2}. We also abuse our notation there other than this in our present proof. 
We choose a sufficiently small positive number $a_{j,0}>0$ and an arbitrary number $t_0 \in [0,1]$ of the interval $[0,1]$.
We check the zero set of the polynomial ${\prod}_{j \in {m_{l_1,l_2}}^{-1}(m_{l_1,l_2}(j_0))} (f_{j}(x_1,x_2))-a_{j,0} t_0$. The relation $a_{j,0} t_0={||y_{m_{l_1,l_2}(j_0)}||}^2$ holds at a point $(x,y) \in \overline{D} \times {\mathbb{R}}^{m-n+l_2} \subset {\mathbb{R}}^{m+l_2}$ ($x=(x_1,x_2)$) in this zero set. We apply some of our arguments in the proof of "Main Theorems 1 and 2 of \cite{kitazawa8}" or the arguments from \cite{kollar}. Thus, around the zero set of the polynomial $f_{j}=f_{j_0}$, our map $f:M \rightarrow {\mathbb{R}}^2$ is represented as the product map of a natural Morse function on the disk $D^{m_2 \circ m_{l_1,l_2}(j_0)+1}$ and the identity map on the product of the zero set of the polynomial $f_{j_0}$ and a sphere diffeomorphic to the product ${\prod}_{j \in {\mathbb{N}_{l_2}}-\{m_{l_1,l_2}(j_0)\}} S^{m_2(j)}$ in the case $l_2>1$ and the product map of a natural Morse function on the disk $D^{m_2 \circ m_{l_1,l_2}(j_0)+1}$ and the identity map on the zero set of the polynomial $f_{j_0}$ in the case $l_2=1$. 

We discuss these functions and maps more precisely.

%First, Morse functions for product maps are obtained by the arguments of \cite{kollar}. 

The identity map on ${\prod}_{j \in {\mathbb{N}_{l_2}}-\{m_{l_1,l_2}(j_0)\}} S^{m_2(j)}$ comes from a natural map mapping each point $(x,y) \in M \bigcap (\overline{D} \times {\mathbb{R}}^{m-n+l_2})=M \bigcap (\overline{D} \times {\mathbb{R}}^{m_2 \circ m_{l_1,l_2}(j_0)+1} \times {\mathbb{R}}^{m-n+l_2-m_2 \circ m_{l_1,l_2}(j_0)-1} \subset {\mathbb{R}}^{m+l_2})$ which is also located around the zero set of $f_{j_0}$ to an element of $\overline{D} \times {\mathbb{R}}^{m_2 \circ m_{l_1,l_2}(j_0)+1} \times {\prod}_{j \in {\mathbb{N}_{l_2}}-\{m_{l_1,l_2}(j_0)\}} S^{m_2(j)}$.

We explain this natural map explicitly.
We remember the notation $y=(y_1,\cdots y_{l_2})$. For $i \in {\mathbb{N}_{l_2}}-\{m_{l_1,l_2}(j_0)\}$, the component $y_i$ is mapped onto the component $\frac{1}{\sqrt{{\prod}_{j \in {m_{l_1,l_2}}^{-1}(i)} (f_{j}(x_1,x_2))}} y_i \in S^{m_2(i)}$ at $(x,y)$. The component $x$ is mapped to $x$ itself and the component $y_{m_{l_1,l_2}(j_0)}$ is mapped to $y_{m_{l_1,l_2}(j_0)}$ itself. Around the zero set of the polynomial $f_{j_0}$ and its preimage by the map $f$, this map also gives a diffeomorphism.

Remember that restrictions of the projection ${\pi}_{2,1}$ to circles are Morse functions. We can show this as a kind of elementary exercises on Morse functions.

These arguments complete the proof. 

\end{proof}
We present  Main Theorem \ref{mthm:2}. This gives some explicit families of real algebraic functions enjoying the following properties.
\begin{itemize}
\item These functions are obtained as compositions of real algebraic maps into ${\mathbb{R}}^2$ in Main Theorem \ref{mthm:1} and Theorems \ref{thm:1} and \ref{thm:2} with ${\pi}_{2,1}$.
\item These functions are Morse-Bott.
\item The images of real algebraic maps into ${\mathbb{R}}^2$ before are bounded and connected regions in ${\mathbb{R}}^2$ surrounded by circles intersecting satisfying transversality.
\item The Reeb graphs of the functions collapse to (given) graphs. For example, ones isomorphic to the Reeb graphs of Morse-Bott functions in Proposition \ref{prop:1}.
\item The manifolds $M$ in these theorems are the zero sets of some real polynomial maps.
\end{itemize}
We first define the class of maps constructed in (proving) Main Theorem \ref{mthm:1} and Theorems \ref{thm:1} and \ref{thm:2}.
\begin{Def}
In Main Theorem \ref{mthm:1} and Theorems \ref{thm:1} and \ref{thm:2}, 
the connected component $S_0:=\{(x,y_1,\cdots,y_{l_2}) \in U_D \times {\prod}_{i=1}^{l_2} {\mathbb{R}}^{m_{l_2}(i)+1} \subset {\mathbb{R}}^{m+l_2} \mid {\prod}_{j \in {m_{l_1,l_2}}^{-1}(i)} (f_{j}(x_1,\cdots,x_n))-{||y_i||}^2=0, i \in {\mathbb{N}}_{l_2}\} \subset {\mathbb{R}}^{m+l_2}$ of the zero set of a real algebraic map is defined. We have also defined the $m$-dimensional non-singular real algebraic manifold $M:=S_0$.  The map $f$ is defined as the composition of the canonical embedding into ${\mathbb{R}}^{m+l_2}$ with ${\pi}_{m+l_2,n}$.
We can canonically define $M$ and $f$ from the data $(D, \{f_{j}\}_{j=1}^{l_1},m_{l_1,l_2},m_{l_2})$. Let us use the notation $M:=M_{(D,\{f_{j}\}_{j=1}^{l_1},m_{l_1,l_2},m_{l_2})}$ and $f:=f_{(D,\{f_{j}\}_{j=1}^{l_1},m_{l_1,l_2},m_{l_2})}$.
We also call the map $f$ the {\it moment-like} map reconstructed from $(D, \{f_{j}\}_{j=1}^{l_1},m_{l_1,l_2},m_{l_2})$. 
\end{Def}
Let the composition of such a map with ${\pi}_{n,1}$, which is a smooth function, denoted by $f_{0,(D,\{f_{j}\}_{j=1}^{l_1},m_{l_1,l_2},m_{l_2})}$. 
Hereafter, we respect the case each $S_j$ is a circle unless otherwise stated. We use the notation for "the set $S_j$" instead of that for "the polynomial $f_j$" in such a case. This contains no problems. More rigorously, here the circle $S_j$ is for a polynomial ${(x_1-p_{0,1})}^2+{(x_2-p_{0,2})}^2-{r_0}^2$ or ${r_0}^2-{(x_1-p_{0,1})}^2-{(x_2-p_{0,2})}^2$ ($x=(x_1,x_2) \in {\mathbb{R}}^2$). Each of these two types is chosen in such a way that this is compatible with our open set $D$ and Theorem \ref{mthm:1} (\ref{mthm:1.1}).
\begin{MainThm}
\label{mthm:2}
Given a situation as in Proposition \ref{prop:1}, in Main Theorem \ref{mthm:1} and Theorems \ref{thm:1} and \ref{thm:2} satisfying the following. Let $n=2$, $l_2:=l_{0,2}=1$ and suppose that the family $\{S_j:=S_{0,j}\}_{j=1}^{l_{0,1}} \subset {\mathbb{R}}^2$ {\rm (}$l_1:=l_{0,1}${\rm )} is a family of mutually disjoint circles and that the relation $\overline{D_0}-D_0={\sqcup}_{j=1}^{l_{0,1}} S_j$ {\rm (}$D:=D_0${\rm )} holds. 
We define the moment-like map $f=f_{(D, \{S_{j}\}_{j=1}^{l_1},m_{l_1,l_2},m_{l_2})}:=f_{(D_0, \{S_{0,j}\}_{j=1}^{l_{0,1}},m_{l_{0,1},l_{0,2}},m_{l_{0,2}})}:M_{(D, \{S_{j}\}_{j=1}^{l_1},m_{l_1,l_2},m_{l_2})}:=M_{(D_0, \{S_{0,j}\}_{j=1}^{l_{0,1}},m_{l_{0,1},l_{0,2}},m_{l_{0,2}})} \rightarrow {\mathbb{R}}^2$. For the Reeb graph $W_{f_{0,(D_0, \{S_{0,j}\}_{j=1}^{l_{0,1}},m_{l_{0,1},l_{0,2}},m_{l_{0,2}})}}$ of the function $f_{0,(D_0, \{S_{0,j}\}_{j=1}^{l_{0,1}},m_{l_{0,1},l_{0,2}},m_{l_{0,2}})}$, let $n_e$ denote the number of the edges of $W_{f_{0,(D_0, \{S_{0,j}\}_{j=1}^{l_{0,1}},m_{l_{0,1},l_{0,2}},m_{l_{0,2}})}}$ and $\{e_j\}_{j=1}^{n_e}$ the family of all edges of the graph.
Here let  $m_{l_{0,2}}$ denote not only the function but also the value $m_{l_{0,2}}(j)$. Note that the function $m_{l_{0,2}}$ is constant here. For each map $l_{\{e_j\}_{j=1}^{n_e}}:{\mathbb{N}}_{n_e} \rightarrow {\mathbb{N}} \sqcup \{0\}$ and an arbitrary integer $m_{l_{\{e_j\}_{j=1}^{n_e}}}>m_{l_{0,2}}+2$, there exist a non-singular real algebraic manifold $M_{l_{\{e_j\}_{j=1}^{n_e}}}$ which is also a closed and connected manifold, which is the zero set of a real polynomial map, and whose dimension is $m_{l_{\{e_j\}_{j=1}^{n_e}}}$ and a smooth real algebraic map $f_{l_{\{e_j\}_{j=1}^{n_e}}}:M_{l_{\{e_j\}_{j=1}^{n_e}}} \rightarrow {\mathbb{R}}^2$ with the following properties. 
\begin{enumerate}
\item \label{mthm:2.1} Each map $f:=f_{l_{\{e_j\}_{j=1}^{n_e}}}$ is  the moment-like map reconstructed from some data $(D,\{S_j\}_{j=1}^{l_1},m_{l_1,l_2},m_{l_2}):=(D_{l_{\{e_j\}_{j=1}^{n_e}}},\{S_{l_{\{e_j\}_{j=1}^{n_e}},j}\}_{j=1}^{l_1},m_{l_{\{e_j\}_{j=1}^{n_e}},l_1,l_2},m_{l_{\{e_j\}_{j=1}^{n_e}},l_2})$ by choosing each $S_{l_{\{e_j\}_{j=1}^{n_e}},j}$ suitably satisfying the following conditions.
\begin{enumerate}
\item We define the two integers $l_1:=l_{0,1}+{\Sigma}_{j^{\prime}=1}^{n_e} l_{\{e_j\}_{j=1}^{n_e}}(j^{\prime})$ and $l_2:=l_{0,2}+1=1+1=2$.
\item Each $S_j$ is a circle.  For $1 \leq j \leq l_{0,1}$, $S_j:=S_{j,0}$.
\item Distinct circles in the family $\{S_{l_{0,1}+j}\}_{j=1}^{l_1-l_{0,1}}$ are disjoint. For each circle $S_{l_{0,1}+j_2}$ {\rm (}$1 \leq j_2 \leq l_1-l_{0,1}${\rm )} in this family, there exists a unique circle $S_{j_1,0}$ in the family $\{S_{j,0}\}_{j=1}^{l_{0,1}}$ such that the intersection $S_{j_1,0} \bigcap S_{l_{0,1}+j_2}$ is not empty. Furthermore, the intersection is a discrete two-point set. %Note that the relation $l_2 \geq 2$ must be satisfied here.
\end{enumerate}
Hereafter, let $f_{0,l_{\{e_j\}_{j=1}^{n_e}}}:={\pi}_{2,1} \circ f_{l_{\{e_j\}_{j=1}^{n_e}}}$.

%\item \label{mthm:3.3} The dimensions of the distinct manifolds $M_{l_{\{e_j\}_{j=1}^{n_e},1}}$ and $M_{l_{\{e_j\}_{j=1}^{n_e},2}}$ may be distinct for distinct maps $l_{\{e_j\}_{j=1}^{n_e}}:=l_{\{e_j\}_{j=1}^{n_e},1}$ and $l_{\{e_j\}_{j=1}^{n_e}}:=l_{\{e_j\}_{j=1}^{n_e},2}$. 
\item \label{mthm:2.2} Each function $f_{0,l_{\{e_j\}_{j=1}^{n_e}}}$ is a Morse-Bott function.
\item \label{mthm:2.3} Each Reeb graph $W_{f_{0,l_{\{e_j\}_{j=1}^{n_e}}}}$ collapses to the original Reeb graph $W_{f_{0,(D_0, \{S_{0,j}\}_{j=1}^{l_{0,1}},m_{l_{0,1},l_{0,2}},m_{l_{0,2}})}}$. 
\item \label{mthm:2.4}
For any two distinct cases such that the values of the maps $l_{\{e_j\}_{j=1}^{n_e}}:=l_{\{e_j\}_{j=1}^{n_e},1}$ and $l_{\{e_j\}_{j=1}^{n_e}}:=l_{\{e_j\}_{j=1}^{n_e},2}$ are same except at exactly one number in ${\mathbb{N}}_{n_e}$, the Reeb graphs $W_{f_{0,l_{\{e_j\}_{j=1}^{n_e},1}}}$ and $W_{f_{0,l_{\{e_j\}_{j=1}^{n_e},2}}}$ are not isomorphic.

\end{enumerate}
% We can also choose the dimensions of the manifolds $M_{l_{\{e_j\}_{j=1}^{n_e}}}$ as a fixed dimension greater than or equal to $4$. 
\end{MainThm}
\begin{proof}


We choose circles $S_j$ for our proof.
We can choose our circles enjoying the following properties.
\begin{enumerate}\setcounter{enumi}{4}
\item \label{mthm:2.5} For $1 \leq j \leq l_{0,1}$, $S_j:=S_{j,0}$.
\item \label{mthm:2.6} Circles in the family $\{S_{l_{0,1}+j}\}_{j=1}^{l_1-l_{0,1}}$ are centered at points in some circles in the family $\{S_{j,0}\}_{j=1}^{l_{0,1}}$ which are not vertical or horizontal poles. These circles are sufficiently small and mutually disjoint.
We remember the map $f_{(D_0, \{S_{0,j}\}_{j=1}^{l_{0,1}},m_{l_{0,1},l_{0,2}},m_{l_{0,2}})}$, the function $f_{0,(D_0, \{S_{0,j}\}_{j=1}^{l_{0,1}},m_{l_{0,1},l_{0,2}},m_{l_{0,2}})}$, and the quotient map $q_{f_{0,(D_0, \{S_{0,j}\}_{j=1}^{l_{0,1}},m_{l_{0,1},l_{0,2}},m_{l_{0,2}})}}$. We take the preimage $P_{e_{i}}:={q_{f_{0,(D_0, \{S_{0,j}\}_{j=1}^{l_{0,1}},m_{l_{0,1},l_{0,2}},m_{l_{0,2}})}}}^{-1}(e_{i}) \subset M_{(D_0, \{S_{0,j}\}_{j=1}^{l_{0,1}},m_{l_{0,1},l_{0,2}},m_{l_{0,2}})}$, the image $f_{(D_0, \{S_{0,j}\}_{j=1}^{l_{0,1}},m_{l_{0,1},l_{0,2}},m_{l_{0,2}})}(P_{e_i})$ and the intersection \\ $f_{(D_0, \{S_{0,j}\}_{j=1}^{l_{0,1}},m_{l_{0,1},l_{0,2}},m_{l_{0,2}})}(P_{e_i}) \bigcap {\sqcup}_{j=1}^{l_{0,1}} S_j$. 
Remember also Proposition \ref{prop:1} and that the function $f_{0,(D_0, \{S_{0,j}\}_{j=1}^{l_{0,1}},m_{l_{0,1},l_{0,2}},m_{l_{0,2}})}$ is a Morse function for example.
The intersection $f_{(D_0, \{S_{0,j}\}_{j=1}^{l_{0,1}},m_{l_{0,1},l_{0,2}},m_{l_{0,2}})}(P_{e_i}) \bigcap {\sqcup}_{j=1}^{l_{0,1}} S_j$ is not empty and it is a disjoint union of two curves diffeomorphic to $\mathbb{R}$. We choose one connected curve $L_i$ from these curves.
The circle $S_{l_{0,1}+{\Sigma}_{j^{\prime}=1}^{i-1} l_{\{e_j\}_{j=1}^{n_e}}(j^{\prime})+j^{\prime \prime}}$ is centered at a point in $L_i$ for $1 \leq j^{\prime \prime} \leq l_{\{e_j\}_{j=1}^{n_e}}(i)$. Let ${\mathbb{R}^2}_{S_{l_{0,1}+j}}$ be the unbounded connected component of the complementary set ${\mathbb{R}}^2-S_{l_{0,1}+j}$ for $1 \leq j \leq l_1-l_{0,1}$. We define $D=D_{l_{\{e_j\}_{j=1}^{n_e}}}:=D_0 \bigcap {\bigcap}_{j=1}^{l_1-l_{0,1}} {\mathbb{R}^2}_{S_{l_{0,1}+j}}$.
%We can define $D$ as the connected component of the complementary set of the union ${\bigcup}_{j=1}^{l_1} S_j \subset {\mathbb{R}}^2$ contained in $D_0$.
\item \label{mthm:2.7} Each circle $S_{l_{0,1}+j}$ in $\{S_{l_{0,1}+j}\}_{j=1}^{l_1-l_{0,1}}$ contains exactly one vertical pole $v_{S_{l_{0,1}+j}}$ of the circle $S_{l_{0,1}+j}$ satisfying $v_{S_{l_{0,1}+j}} \in \overline{D}$ and two points $p_{S_{l_{0,1}+j},1}$ and $p_{S_{l_{0,1}+j},2}$ contained in exactly two distinct circles in $\{S_j\}_{j=1}^{l_1}$ and $\overline{D}$. These three points are mutually distinct and the values of the first components are also mutually distinct. For exactly two circles containing the two points here, one is from the family $\{S_j\}_{j=1}^{l_{0,1}}$ and the other is from the family $\{S_{l_{0,1}+j}\}_{j=1}^{l_1-l_{0,1}}$.
For distinct circles $S_{l_{0,1}+j_1}$ and $S_{l_{0,1}+j_2}$ from $\{S_{l_{0,1}+j}\}_{j=1}^{l_1-l_{0,1}}$, either of the relation $$\max \{{\pi}_{2,1}(v_{S_{l_{0,1}+j_1}}),{\pi}_{2,1}(p_{S_{l_{0,1}+j_1},1}), {\pi}_{2,1}(p_{S_{l_{0,1}+j_1},2})\}<\min \{{\pi}_{2,1}(v_{S_{l_{0,1}+j_2}}),{\pi}_{2,1}(p_{S_{l_{0,1}+j_2},1}), {\pi}_{2,1}(p_{S_{l_{0,1}+j_2},2})\}$$ or the relation $$\max \{{\pi}_{2,1}(v_{S_{l_{0,1}+j_2}}),{\pi}_{2,1}(p_{S_{l_{0,1}+j_2},1}), {\pi}_{2,1}(p_{S_{l_{0,1}+j_2},2})\}<\min \{{\pi}_{2,1}(v_{S_{l_{0,1}+j_1}}),{\pi}_{2,1}(p_{S_{l_{0,1}+j_1},1}), {\pi}_{2,1}(p_{S_{l_{0,1}+j_1},2})\}$$ holds. Note that the map ${\pi}_{2,1}$ is used to represent the first components.
\item \label{mthm:2.8} Each circle $S_{j,0}$ in $\{S_{j,0}\}_{j=1}^{l_{0,1}}$ contains exactly two vertical poles $v_{S_{j,0},1}$ and $v_{S_{j,0},2}$ of it satisfying $v_{S_{j,0},1}, v_{S_{j,0},2} \in \overline{D}$. %These vertical poles are different from the points discussed in (\ref{mthm:3.9}). 
The values of the first components of these vertical poles are always distinct from those of points discussed in (\ref{mthm:2.7}): the vertical poles $v_{S_{l_{0,1}+j}}$ and points $p_{S_{l_{0,1}+j},1}$ and $p_{S_{l_{0,1}+j},2}$ contained in exactly two distinct circles in $\{S_j\}_{j=1}^{l_1}$.
\item \label{mthm:2.9} We can apply Main Theorem \ref{mthm:1} with Theorems \ref{thm:1} and \ref{thm:2} and our proof to have our new map $f_{l_{\{e_j\}_{j=1}^{n_e}}}:M_{l_{\{e_j\}_{j=1}^{n_e}}} \rightarrow {\mathbb{R}}^2$. More explicitly we set as follows. We put $l_2:=2$.

We set $m_{l_{\{e_j\}_{j=1}^{n_e}},l_1,l_2}$ as a function onto the set $\{1,2\}$ and $m_{l_{\{e_j\}_{j=1}^{n_e}},l_2}$ as a function with the following conditions.
\begin{enumerate}
	\item The restriction $m_{l_{\{e_j\}_{j=1}^{n_e}},l_1,l_2} {\mid}_{{\mathbb{N}}_{l_{0,1}}}$ and the function $m_{l_{0,1},l_{0,2}}$ agree where the sets of the targets are seen as $\{1,2\}$.
	\item The restriction $m_{l_{\{e_j\}_{j=1}^{n_e}},l_1,l_2} {\mid}_{{\mathbb{N}}_{l_{0,1}}}$ is a constant function whose values are always $1$. The restriction $m_{l_{\{e_j\}_{j=1}^{n_e}},l_1,l_2} {\mid}_{{\mathbb{N}}_{l_1}-{\mathbb{N}}_{l_{0,1}}}$ is a constant function whose values are always $2$.
	\item We put $m_{l_{\{e_j\}_{j=1}^{n_e}},l_2}(1)=m_{l_{0,2}}:=m_{l_{0,2}}(1)$.
	We put $m_{l_{\{e_j\}_{j=1}^{n_e}},l_2}(2):=m_{l_{\{e_j\}_{j=1}^{n_e}}}-(m_{l_{0,2}}+2)>0$.
\end{enumerate}

\end{enumerate}
%the preimage ${f_{l_{\{e_j\}_{j=1}^{n_e}}}}^{-1}(p)$ contains some singular points of the function $f_{0,l_{\{e_j\}_{j=1}^{n_e}}}$ if and only if it is a vertical pole $v_j$ of some circle $S_j$ satisfying $v_j \in \overline{D}$ or a point contained in exactly two disjoint circles in $\{S_j\}_{j=1}^{l_1}$


We construct our map $f_{l_{\{e_j\}_{j=1}^{n_e}}}:M_{l_{\{e_j\}_{j=1}^{n_e}}} \rightarrow {\mathbb{R}}^2$ in (\ref{mthm:2.9}). We show that this map is our desired map. 

We can check the property (\ref{mthm:2.1}) here easily from our situation. 

We show the property (\ref{mthm:2.2}). This is on the singularities of the functions. 
%The values of the first components are distinct for distinct vertical poles and points contained in exactly two disjoint circles in $\{S_j\}_{j=1}^{l_1}$.
We discuss local singularities of the function $f_{0,l_{\{e_j\}_{j=1}^{n_e}}}$. 

For $p \in \overline{D}$, the preimage ${f_{l_{\{e_j\}_{j=1}^{n_e}}}}^{-1}(p)$ contains some singular points of the function $f_{0,l_{\{e_j\}_{j=1}^{n_e}}}$ only if it is a vertical pole $v$ of some circle $S_j$ satisfying $v \in \overline{D}$ or a point contained in exactly two disjoint circles in $\{S_j\}_{j=1}^{l_1}$ and $\overline{D}$. We can see this fact on singularities from the structures of the functions and the maps easily.

We remember our proof of Proposition \ref{prop:1} and related arguments on singularities of the function at vertical poles. These singular points are for singular points of Morse-Bott functions.

We present an argument similar to "our argument for local structures of the map $f$ around the zero set of $f_j$ in the proof of Proposition \ref{prop:1}". This is a higher dimensional version for a non-empty set represented as $S_{j_{0,1}} \bigcap S_{j_{0,2}}$ ($j_{0,1} \neq j_{0,2}$). 

Around each point of the discrete set $S_{j_{0,1}} \bigcap S_{j_{0,2}}$, our map $f:M \rightarrow {\mathbb{R}}^2$ is represented as the product map of a natural Morse function on the disk $D^{m_2 \circ m_{l_1,l_2}(j_{0,1})+1}$, a natural Morse function on the disk $D^{m_2 \circ m_{l_1,l_2}(j_{0,2})+1}$ and the identity map on a sphere diffeomorphic to the product ${\prod}_{j \in {\mathbb{N}_{l_2}}-\{m_{l_1,l_2}(j_{0,1}),m_{l_1,l_2}(j_{0,2})\}} S^{m_2(j)}$ in the case $l_2>2$ and the product map of a natural Morse function on the disk $D^{m_2 \circ m_{l_1,l_2}(j_{0,1})+1}$ and a natural Morse function on the disk $D^{m_2 \circ m_{l_1,l_2}(j_{0,2})+1}$ in the case $l_2=2$. Remember that here $l_2=2$. We have discussed a general case for the singularities.

The identity map on the set ${\prod}_{j \in {\mathbb{N}_{l_2}}-\{m_{l_1,l_2}(j_{0,1}),m_{l_1,l_2}(j_{0,2})\}} S^{m_2(j)}$ comes from a natural map mapping $(x,y) \in M \subset \overline{D} \times {\mathbb{R}}^{m-n+l_2}=\overline{D} \times {\mathbb{R}}^{m_2 \circ m_{l_1,l_2}(j_{0,1})+1} \times {\mathbb{R}}^{m_2 \circ m_{l_1,l_2}(j_{0,2})+1} \times {\mathbb{R}}^{m-n+l_2-m_2 \circ m_{l_1,l_2}(j_{0,1})-m_2 \circ m_{l_1,l_2}(j_{0,2})-2} \subset {\mathbb{R}}^{m+l_2}$ to an element of $\overline{D} \times {\mathbb{R}}^{m_2 \circ m_{l_1,l_2}(j_{0,1})+1} \times {\mathbb{R}}^{m_2 \circ m_{l_1,l_2}(j_{0,2})+1} \times {\prod}_{j \in {\mathbb{N}_{l_2}}-\{m_{l_1,l_2}(j_{0,1}), m_{l_1,l_2}(j_{0,2})\}} S^{m_2(j)}$
around the discrete set.

We present the natural map explicitly.
We remember the notation $y=(y_1,\cdots y_{l_2})$. For $i \in {\mathbb{N}_{l_2}}-\{m_{l_1,l_2}(j_{0,1}),m_{l_1,l_2}(j_{0,2})\}$, the component $y_i$ is mapped onto the component $\frac{1}{\sqrt{{\prod}_{j \in {m_{l_1,l_2}}^{-1}(i)} (f_{j}(x_1,x_2))}} y_i \in S^{m_2(i)}$ at $(x,y)$. The component $x$ is mapped to $x$ itself and the component $y_{m_{l_1,l_2}(j_{0,a})}$ is mapped to $y_{m_{l_1,l_2}(j_{0,a})}$ itself for $a=1,2$. This natural map also gives a diffeomorphism around the discrete set $S_{j_{0,1}} \bigcap S_{j_{0,2}}$ and its preimage by the map $f$.
%Note that we can generalize this to higher dimensional cases of course. However we do not need this here.

We respect these arguments and local coordinates of the functions and the maps. We also remember the properties, mainly, (\ref{mthm:2.6}, \ref{mthm:2.7}, \ref{mthm:2.8}, \ref{mthm:2.9}). Here, we also respect these properties, implying that the location of our circles $S_j$ is sufficiently general.
%We also remember the property on singularities (\ref{mthm:3.8}) with related properties (\ref{mthm:3.9}, \ref{mthm:3.10}).
This completes the proof of the property (\ref{mthm:2.2}).

We show the properties (\ref{mthm:2.3}, \ref{mthm:2.4}). We mainly remember the properties (\ref{mthm:2.6}, \ref{mthm:2.7}, \ref{mthm:2.8}, \ref{mthm:2.9}) again.
We investigate the preimage ${f_{0,l_{\{e_j\}_{j=1}^{n_e}}}}^{-1}(t)$ containing no singular points of the function $f_{0,l_{\{e_j\}_{j=1}^{n_e}}}$. We investigate the preimage ${{\pi}_{2,1}}^{-1}(t)$, which is the straight line $L_t \subset {\mathbb{R}}^2$ represented as the zero set of the real polynomial $x_1-t$ ($(x_1,x_2) \in {\mathbb{R}}^2$). We investigate each connected component $L_{e,t}$ of the intersection of the straight line $L_t$ and the image $f_{l_{\{e_j\}_{j=1}^{n_e}}}(M_{l_{\{e_j\}_{j=1}^{n_e}}})$. The interior ${\rm Int}\ L_{e,t}$ is in the interior ${\rm Int}\ f_{l_{\{e_j\}_{j=1}^{n_e}}}(M_{l_{\{e_j\}_{j=1}^{n_e}}})$ of the image and the boundary consists of exactly two points $p_{e,1}$ and $p_{e,2}$ contained in some circles $S_j$ which are mutually distinct. On this, either of the following holds.


\begin{itemize}
\item These points $p_{e,1}$ and $p_{e,2}$ are in distinct circles represented as $S_{0,j}=S_j$ ($1 \leq j \leq l_{0,1}$). The preimage ${f_{l_{\{e_j\}_{j=1}^{n_e}}}}^{-1}(L_{e,t})$ is diffeomorphic to $S^{m_{l_{\{e_j\}_{j=1}^{n_e}},l_2}(1)+1} \times S^{m_{l_{\{e_j\}_{j=1}^{n_e}},l_2}(2)}$. This is also regarded as a manifold diffeomorphic to one obtained by gluing two copies of $D^{m_{l_{\{e_j\}_{j=1}^{n_e}},l_2}(1)+1} \times S^{m_{l_{\{e_j\}_{j=1}^{n_e}},l_2}(2)}$ along the boundaries  in a canonical way. In other words, we take a double of the copy here.
\item One of the points is in a circle represented as $S_{0,j}=S_j$ ($1 \leq j \leq l_{0,1}$) and the other point is in a circle represented as $S_{j}$ ($j>l_{0,1}$). The preimage ${f_{l_{\{e_j\}_{j=1}^{n_e}}}}^{-1}(L_{e,t})$ is diffeomorphic to $S^{m_{l_{\{e_j\}_{j=1}^{n_e}}}-1}$. This is also regarded as a manifold diffeomorphic to one obtained by gluing $D^{m_{l_{\{e_j\}_{j=1}^{n_e}},l_2}(1)+1} \times \partial D^{m_{l_{\{e_j\}_{j=1}^{n_e}},l_2}(2)+1}$ and $\partial (D^{m_{l_{\{e_j\}_{j=1}^{n_e}},l_2}(1)+1}) \times D^{m_{l_{\{e_j\}_{j=1}^{n_e}},l_2}(2)+1}$ along the boundaries in a canonical way.
 %$D^{m_{l_{\{e_j\}_{j=1}^{n_e}},l_2}(1)+1} \times S^{m_{l_{\{e_j\}_{j=1}^{n_e}},l_2}(2)}$
\end{itemize}
 
 From our situation, we cannot observe the case that these points $p_{e,1}$ and $p_{e,2}$ are in distinct circles represented as $S_{l_{0,1}+j}$ ($1 \leq j \leq l_1-l_{0,1}$).
 
These preimages are also connected components of the preimage ${f_{0,l_{\{e_j\}_{j=1}^{n_e}}}}^{-1}(t)$. See also Remark \ref{rem:1} for a kind of related counterexamples.


We can easily check the property (\ref{mthm:2.3}).


The property (\ref{mthm:2.4}) follows from the difference of numbers of vertices. 

%We can show the last argument as follows. We define $m_{l_1,l_2}:{\mathbb{N}}_{l_1} \rightarrow {\mathbb{N}}_{l_2}={\mathbb{N}}_2$ as a function such that $m_{l_1,l_2}({\mathbb{N}}_{l_{0,1}})=\{1\}$ and $m_{l_1,l_2}({\mathbb{N}}_{l_1}-{\mathbb{N}}_{l_{0,1}})=\{2\}$. We also define $m_{l_2}=m_2$ as a function whose vaues are always $1$. We apply the theorems under this situation to show that we can consider the case $m=4$. 
%For $m>4$, we can define $m_{l_2}=m_2$ as a function whose image consists of one or two numbers such that the sum is $m-2$.
%This kind of arguments is also in the short remark after our proof of Main Theorem \ref{mthm:2}.
Finally, we also show that our $M_{l_{\{e_j\}_{j=1}^{n_e}}}$ is not only a connected component of the zero set of a real polynomial map, but also the zero set. This follows from the fact that the set "$U_D$ is chosen as ${\mathbb{R}}^n$ in Main Theorem \ref{mthm:1}" and the construction in its proof. In Main Theorem \ref{mthm:3}, we can argue this in the same way.
This completes the proof.
\end{proof}

FIGUREs \ref{fig:1} and \ref{fig:2} show an explicit case for Main Theorem \ref{mthm:2} with $n=2$ and $(l_{0,1},l_{0,2},l_1,l_2,n_e)=(3,1,5,2,5)$.
\begin{figure}
	
	\includegraphics[height=75mm, width=100mm]{20240503MT2_1.eps}

	\caption{An example for Main Theorem \ref{mthm:2}. The image of a moment-like map $f_{(D_0,\{S_{0,j}\}_{j=1}^{l_{0,1}},m_{l_{0,1},l_{0,2}},m_{l_{0,2}})}$ into ${\mathbb{R}}^2$ for a region $D_0$ surrounded by circles and colored in gray with $n=2$ and $(l_{0,1}.l_{0,2})=(3,1)$. The Reeb graph of the function $f_{0,(D_0,\{S_{0,j}\}_{j=1}^{l_{0,1}},m_{l_{0,1},l_{0,2}},m_{l_{0,2}})}:={\pi}_{2,1} \circ f_{(D_0,\{S_j\}_{j=1}^{l_{0,1}},m_{l_{0,1},l_{0,2}},m_{l_{0,2}})}$. Black dots are for singular points of the function.}
	\label{fig:1}
\end{figure}
\begin{figure}
	
	\includegraphics[height=75mm, width=100mm]{20240509to10MT2_2.eps}

	\caption{An example for Main Theorem \ref{mthm:2} following FIGURE \ref{fig:2}. 
The image of a moment-like map $f_{(D_{l_{\{e_j\}_{j=1}^{n_e}}},\{S_{l_{\{e_j\}_{j=1}^{n_e}},j}\}_{j=1}^{l_1},m_{l_{\{e_j\}_{j=1}^{n_e}},l_1,l_2},m_{l_{\{e_j\}_{j=1}^{n_e}},l_2})}$ into ${\mathbb{R}}^2$ for a region $D_{l_{\{e_j\}_{j=1}^{n_e}}}$ surrounded by circles and colored in gray: here FIGURE \ref{fig:1} is respected with $(l_{0,1},l_{0,2})=(3,1)$, $l_1=5$, $l_2=2$, $n_e=5$, and $l_{\{e_j\}_{j=1}^{n_e}}$ is a function whose value is $2$ at an edge and whose values are $0$ at the remaining edges. The Reeb graph of the function $f_{0,(D_{l_{\{e_j\}_{j=1}^{n_e}}},\{S_{l_{\{e_j\}_{j=1}^{n_e}},j}\}_{j=1}^{l_1},m_{l_{\{e_j\}_{j=1}^{n_e}},l_1,l_2},m_{l_{\{e_j\}_{j=1}^{n_e}},l_2})}:={\pi}_{2,1} \circ f_{(D_{l_{\{e_j\}_{j=1}^{n_e}}},\{S_{l_{\{e_j\}_{j=1}^{n_e}},j}\}_{j=1}^{l_1},m_{l_{\{e_j\}_{j=1}^{n_e}},l_1,l_2},m_{l_{\{e_j\}_{j=1}^{n_e}},l_2})}$. Black dots are for singular points of the function. For each point in the interior of each black (green) edge, the preimage is diffeomorphic to $S^{m_{l_{\{e_j\}_{j=1}^{n_e}},l_2}(1)+1} \times S^{m_{l_{\{e_j\}_{j=1}^{n_e}},l_2}(2)}$ (resp. $S^{m_{l_{\{e_j\}_{j=1}^{n_e}}}-1}$).}
	\label{fig:2}
\end{figure}

We can also show the following by applying several arguments in our proof of Main Theorem \ref{mthm:2} for example.

\begin{MainThm}
\label{mthm:3}
Suppose that a situation as in Proposition \ref{prop:1}, in Main Theorem \ref{mthm:1} and Theorems \ref{thm:1} and \ref{thm:2} is given as follows. Let $n=2$, $l_2:=l_{1,2} \geq 1$, suppose that the family $\{S_j:=S_{1,j}\}_{j=1}^{l_{1,1}} \subset {\mathbb{R}}^2$ {\rm (}$l_1:=l_{1,1}${\rm )} is a family of circles which may intersect mutually and that the relation $\overline{D_1}-D_1={\bigcup}_{j=1}^{l_{1,1}} S_j$ {\rm (}$D:=D_1${\rm )} holds. In addition the functions $m_{l_{1,1},l_{1,2}}$ and $m_{l_{1,2}}$ are given in such a way that we can define the moment-like map $f=f_{(D, \{S_{j}\}_{j=1}^{l_1},m_{l_1,l_2},m_{l_2})}:=f_{(D_1, \{S_{1,j}\}_{j=1}^{l_{1,1}},m_{l_{1,1},l_{1,2}},m_{l_{1,2}})}:M_{(D, \{S_{j}\}_{j=1}^{l_1},m_{l_1,l_2},m_{l_2})}:=M_{(D_1, \{S_{1,j}\}_{j=1}^{l_{1,1}},m_{l_{1,1},l_{1,2}},m_{l_{1,2}})} \rightarrow {\mathbb{R}}^2$. As an additional condition, we also define the function $m_{l_{1,2}}$ as a function whose values are always positive integers.

Furthermore, for the Reeb graph $W_{f_{0,(D_1, \{S_{1,j}\}_{j=1}^{l_{1,1}},m_{l_{1,1},l_{1,2}},m_{l_{1,2}})}}$ of the function $f_{0,(D_1, \{S_{1,j}\}_{j=1}^{l_{1,1}},m_{l_{1,1},l_{1,2}},m_{l_{1,2}})}$, let $n_e$ denote the number of the edges of $W_{f_{0,(D_1, \{S_{1,j}\}_{j=1}^{l_{1,1}},m_{l_{1,1},l_{1,2}},m_{l_{1,2}})}}$ and $\{e_j\}_{j=1}^{n_e}$ the family of all edges of the graph.

Then, for each edge $e_i$ and an arbitrary integer $m_{e_i}>{\Sigma}_{j=1}^{l_{1,2}} (m_{e_i,l_{1,2}}(j))+2$, there exist a non-singular real algebraic manifold $M_{e_i}$ which is also a closed and connected manifold, which is the zero set of a real polynomial map, and whose dimension is $m_{e_i}$ and a smooth real algebraic map $f_{e_i}:M_{e_i} \rightarrow {\mathbb{R}}^2$ with the following properties. 
\begin{enumerate}
\item \label{mthm:3.1} The map $f:=f_{e_i}$ is  the moment-like map reconstructed from some data $(D,\{S_j\}_{j=1}^{l_1},m_{l_1,l_2},m_{l_2}):=(D_{e_i},\{S_{e_i,j}\}_{j=1}^{l_1},m_{e_i,l_1,l_2},m_{e_i,l_{2}})$ by choosing each $S_{e_i,j}$ suitably satisfying the following conditions.
\begin{enumerate}
\item The circles are defined by $S_j:=S_{1,j}$ for $1 \leq j \leq l_{1,1}$.
\item The integers $l_1$ and $l_2$ are defined by $l_1:=l_{1,1}+1$ and  $l_2:=l_{1,2}+1$. 
\item The set $S_{l_1} \bigcap S_{1,j}$ is a non-empty set for a unique integer $1 \leq j \leq l_{1,1}$. The non-empty set is a discrete two-point set.
\end{enumerate}

\item \label{mthm:3.2} The function $f_{0,e_i}:={\pi}_{2,1} \circ f_{e_i}$ is a Morse-Bott function.
\item \label{mthm:3.3} The Reeb graph $W_{f_{0,e_i}}$ is isomorphic to the graph obtained in the following way.
\begin{enumerate}
\item We choose the edge $e_i$ of $W_{f_{0,(D_1, \{S_{1,j}\}_{j=1}^{l_{1,1}},m_{l_{1,1},l_{1,2}},m_{l_{1,2}})}}$.We add two distinct new vertices $v_{e_i,1}$ and $v_{e_i,2}$ in the interior of $e_i$.
\item We add an edge to the previous new graph which contains $v_{e_i,1}$ or $v_{e_i,2}$ as a vertex and another newly added vertex. This graph is our new graph collapsing to the original Reeb graph $W_{f_{0,(D_1, \{S_{1,j}\}_{j=1}^{l_{1,1}},m_{l_{1,1},l_{1,2}},m_{l_{1,2}})}}$. We remember that the number of edges is greater than that of the original Reeb graph by $3$ and that the number of vertices is greater than that of the original graph by $3$.
\end{enumerate} 
\end{enumerate}
\end{MainThm}
\begin{proof}[A sketch of a proof]
Most of important arguments is discussed in the proof of Main Theorem \ref{mthm:2}. We only present a sketch of a proof.

We can choose circles $S_{1,j}$ and the unique additional circle $S_{l_1}=S_{l_{1,1}+1}$ as in the proof of Main Theorem \ref{mthm:2}. 
We can abuse most of the notation. We can have our map $f=f_{e_i}:M_{e_i} \rightarrow {\mathbb{R}}^2$ like one presented through (\ref{mthm:2.5})--(\ref{mthm:2.9}).

We present some remarks on the definition of our map. The function $m_{e_i,l_{1},l_{2}}$ is defined in such a way that the restriction to ${\mathbb{N}}_{l_{1,1}}$ is same as $m_{l_{1,1},l_{1,2}}$ with the set of the target being ${\mathbb{N}}_{l_2}={\mathbb{N}}_{l_{1,2}+1}$ and that $m_{e_i,l_{1},l_{2}}(l_{1,1}+1)=l_{1,2}+1=l_2$. We define the function $m_{e_i,l_{2}}$  in such a way that the restriction to ${\mathbb{N}}_{l_{1,2}}$ and $m_{l_{1,2}}$ agree and that the value at $l_{1,2}+1$ is $m_{e_i,l_{2}}(l_{1,2}+1):=m_{e_i}-({\Sigma}_{j=1}^{l_{1,2}} (m_{e_i,l_{2}}(j))+2)>0$.
The map $l_{\{e_j\}_{j=1}^{n_e}}:{\mathbb{N}}_{n_e} \rightarrow {\mathbb{N}} \sqcup \{0\}$, discussed in our proof of Main Theorem \ref{mthm:2}, can be abused in a natural way. It is defined as a map whose values are $1$ at $e_i$ and $0$ at the remaining edges, in our new situation.

We investigate the preimage ${f_{0,e_i}}^{-1}(t)$ containing no singular points of the function $f_{0,e_i}$.  We also abuse the notation here.
Similarly, we have $L_{e,t}$ as a connected component of the intersection of the image of $f_{e_i}$ and the zero set of the real polynomial $x_1-t=0$ in ${\mathbb{R}}^2$ ($(x_1,x_2) \in {\mathbb{R}}^2$). We also have two points $p_{e,1}$ and $p_{e,2}$ in the boundary of  $L_{e,t}$. Either of the following two holds.
\begin{itemize}
\item These points $p_{e,1}$ and $p_{e,2}$ are in distinct circles represented as $S_{j_1}$ and $S_{j_2}$ and the relation $m_{e_i,l_1,l_2}(j_1)=m_{e_i,l_1,l_2}(j_2)$ holds.
The preimage ${f_{e_i}}^{-1}(L_{e,t})$ is diffeomorphic to $S^{m_{e_i,l_2}(m_{e_i,l_1,l_2}(j_1))+1} \times {\prod}_{j \in {\mathbb{N}}_{l_2}-\{m_{e_i,l_1,l_2}(j_1)\}} S^{m_{e_i,l_2}(m_{e_i,l_1,l_2}(j))}=S^{m_{e_i,l_2}(m_{e_i,l_1,l_2}(j_2))+1} \times {\prod}_{j \in {\mathbb{N}}_{l_2}-\{m_{e_i,l_1,l_2}(j_2)\}} S^{m_{e_i,l_2}(m_{e_i,l_1,l_2}(j))}$. Here "$S^{m_{e_i,l_2}(m_{e_i,l_1,l_2}(j_1))+1}=S^{m_{e_i,l_2}(m_{e_i,l_1,l_2}(j_2))+1}$" is regarded as a manifold diffeomorphic to a double of two copies of $D^{m_{e_i,l_2}(m_{e_i,l_1,l_2}(j_1))+1}=D^{m_{e_i,l_2}(m_{e_i,l_1,l_2}(j_2))+1}$.
%This is also regarded as a manifold diffeomorphic to one obtained by gluing two copies of $D^{m_{l_{\{e_j\}_{j=1}^{n_e}},l_2}(1)+1} \times S^{m_{l_{\{e_j\}_{j=1}^{n_e}},l_2}(2)}$ in  a canonical way.
\item These points $p_{e,1}$ and $p_{e,2}$ are in distinct circles represented as $S_{j_1}$ and $S_{j_2}$ 
and the relation $m_{e_i,l_1,l_2}(m_{e_i,l_1,l_2}(j_1)) \neq m_{e_i,l_1,l_2}(m_{e_i,l_1,l_2}(j_2))$ holds.
The preimage ${f_{e_i}}^{-1}(L_{e,t})$ is diffeomorphic to $S^{m_{e_i,l_2}(m_{e_i,l_1,l_2}(j_1))+m_{e_i,l_2}(m_{e_i,l_1,l_2}(j_2))+1} \times  {\prod}_{j \in {\mathbb{N}}_{l_2}-\{m_{e_i,l_1,l_2}(j_1),m_{e_i,l_1,l_2}(j_2)\}} S^{m_{e_i,l_2}(m_{e_i,l_1,l_2}(j))}$.  Here "$S^{m_{e_i,l_2}(m_{e_i,l_1,l_2}(j_1))+m_{e_i,l_2}(m_{e_i,l_1,l_2}(j_2))+1}$" is regarded as a manifold diffeomorphic to a manifold obtained by gluing two manifolds $D^{m_{e_i,l_2}(m_{e_i,l_1,l_2}(j_1))+1} \times \partial D^{m_{e_i,l_2}(m_{e_i,l_1,l_2}(j_2))+1}$ and $\partial (D^{m_{e_i,l_2}(m_{e_i,l_1,l_2}(j_1))+1}) \times D^{m_{e_i,l_2}(m_{e_i,l_1,l_2}(j_2))+1}$  along the boundaries in a canonical way.

% One of the points is in a circle represented as $S_{0,j}=S_j$ ($1 \leq j \leq l_{0,1}$) and the other point is in a circle represented as $S_{j}$ ($j>l_{0,1}$). The preimage ${f_{l_{\{e_j\}_{j=1}^{n_e}}}}^{-1}(L_{e,t})$ is diffeomorphic to $S^{m_{l_{\{e_j\}_{j=1}^{n_e}}}-1}$. 
%This is also regarded as a manifold diffeomorphic to one obtained by gluing $D^{m_{l_{\{e_j\}_{j=1}^{n_e}},l_2}(1)+1} \times \partial D^{m_{l_{\{e_j\}_{j=1}^{n_e}},l_2}(2)+1}$ and $\partial D^{m_{l_{\{e_j\}_{j=1}^{n_e}},l_2}(1)+1} \times D^{m_{l_{\{e_j\}_{j=1}^{n_e}},l_2}(2)+1}$ in  a canonical way.
 %$D^{m_{l_{\{e_j\}_{j=1}^{n_e}},l_2}(1)+1} \times S^{m_{l_{\{e_j\}_{j=1}^{n_e}},l_2}(2)}$
\end{itemize}
 
These preimages are also connected components of the preimage ${f_{0,e_i}}^{-1}(t)$.

We can check that all properties are enjoyed for our map. 
\end{proof}
FIGUREs \ref{fig:3} and \ref{fig:4} show an explicit case for Main Theorem \ref{mthm:3}.
\begin{figure}
	
	\includegraphics[height=75mm, width=100mm]{20240503.MT3.0.eps}

	\caption{An example for Main Theorem \ref{mthm:3}. The image of a moment-like map $f_{(D_1,\{S_{1,j}\}_{j=1}^{l_{1,1}},m_{l_{1,1},l_{1,2}},m_{l_{1,2}})}$ into ${\mathbb{R}}^2$ for a region $D_1$ surrounded by circles and colored in gray with $n=2$ and $l_{1,1}=l_{1,2}=2$. The Reeb graph of the function $f_{0,(D_1,\{S_{1,j}\}_{j=1}^{l_{1,1}},m_{l_{1,1},l_{1,2}},m_{l_{1,2}})}:={\pi}_{2,1} \circ f_{(D_1,\{S_{1,j}\}_{j=1}^{l_{1,1}},m_{l_{1,1},l_{1,2}},m_{l_{1,2}})}$. Black dots are for singular points of the function.}
	\label{fig:3}
\end{figure}
\begin{figure}
	
	\includegraphics[height=75mm, width=100mm]{20240503.MT3.eps}

	\caption{An example for Main Theorem \ref{mthm:3} following FIGURE \ref{fig:3}. The image of a moment-like map $f_{e_i}=f_{(D_{e_i},\{S_{e_i,j}\}_{j=1}^{l_1},m_{e_i,l_1,l_2},m_{e_i,l_2})}$ into ${\mathbb{R}}^2$ for a region $D=D_{e_i}$ surrounded by circles and colored in gray: here FIGURE \ref{fig:3} is respected and $(l_{1,1},l_1,l_2)=(2,3,l_{1,2}+1)$ ($l_{1,2}=2$). The Reeb graph of the function $f_{0,e_i}=f_{0,(D_{e_i},\{S_{e_i,j}\}_{j=1}^{l_1},m_{e_i,l_1,l_2},m_{e_i,l_2})}:={\pi}_{2,1} \circ f_{(D,\{S_j\}_{j=1}^{l_1},m_{e_i,l_1,l_2},m_{e_i,l_2})}$. Black dots are for singular points of the function.}
	\label{fig:4}
\end{figure}
%For a smooth function $c$, we can also define the quotient map $q_c:x \rightarrow W_c$ and another continuous map $\bar{c}:W_c \rightarrow \mathbb{R}$ enjoying the relation $c=\bar{c} \circ q_c$ uniquely.
%The {\it degree} of a vertex of a graph means the number of edges containing it.

%FIGURE 1 of \cite{kitazawa3} shows two explicit cases for the paper.
	
%We discuss the lower figure. Let $l>1$ be an integer. The lower figure shows a graph with exactly $2$ vertices of degree $1$, exactly $2$ vertices of degree $l+1$ and exactly $l+2$ edges. Furthermore, we can explain about the graph as follows.

%\begin{itemize}
%\item The first two vertices are denoted by $v_{\rm l}$ and $v_{\rm r}$, respectively. 
%5%\item The other two vertices are denoted by $v_{\rm 1}$ and $v_{\rm 2}$, respectively. 
%\item Two of the edges are denoted by $e_{\rm l}$ and $e_{\rm r}$, respectively. $e_{\rm l}$ connects $v_{\rm l}$ and $v_1$. $e_{\rm r}$ connects $v_{\rm r}$ and $v_2$.
%\item The remaining $l$ edges are denoted by $e_{j}$ where $1 \leq j \leq l$ is an integer. They connect $v_1$ and $v_2$.
%\end{itemize}

%	More precisely, these two graphs show two simplest examples of the so-called {\it Poincar\'e-Reeb graphs}. A {\it Poincar\'e-Reeb} graph is defined for a pair of an algebraic domain in a real affine space and a canonical projection of the real affine space to the $1$-dimensional real affine space. This is defined as a graph to which the algebraic domain naturally collapses.  
\begin{Rem}
\label{rem:1}
FIGURE \ref{fig:5} shows an important example related to Main Theorem \ref{mthm:2}. This is a kind of counterexamples.
This is same as the case of FIGUREs \ref{fig:1} and \ref{fig:2} except the condition $m_{l_{\{e_j\}_{j=1}^{n_e}},l_2}(2)=0$.
This implies that we cannot show Main Theorem \ref{mthm:2} (\ref{mthm:2.3}) or Main Theorem \ref{mthm:3} (\ref{mthm:3.3}) under more general situations. For example, in the case the values of the function $m_{l_2}$ at some integers may be $0$. We omit examples on Main Theorem \ref{mthm:3}. 
\end{Rem}
\begin{figure}
	
	\includegraphics[height=75mm, width=100mm]{20240508R1.eps}

	\caption{An example for Remark \ref{rem:1}. This follows FIGURE \ref{fig:1}. This shows a graph which is a bit different from that of FIGURE \ref{fig:2}. Here we replace a condition on the value $m_{l_{\{e_j\}_{j=1}^{n_e}},l_2}(2)$ by $m_{l_{\{e_j\}_{j=1}^{n_e}},l_2}(2)=0$, dropping the condition on the value of Main Theorem \ref{mthm:2}.}
	\label{fig:5}
\end{figure}
\section{Closing comments on our study.}
We close our paper by giving Remarks. %They are all on Main Theorem \ref{mthm:2} mainly.
\begin{Rem}
\label{rem:2}
Main Theorems \ref{mthm:2} and \ref{mthm:3} are regarded as our additional result to main results of \cite{kitazawa3, kitazawa7}. Moreover, these studies are originally motivated by Sharko's question in \cite{sharko}. It asks whether we can have an explicit nice smooth function whose Reeb graph is isomorphic to a given graph on some closed surface or more generally, some smooth closed manifold. More precisely, these studies are also motivated by a revised question by the author. 
It asks whether we can respect singularities of the functions and preimages in addition. For related studies, see \cite{kitazawa1, kitazawa2} as pioneering studies by the author for example. \cite{saeki2} is also a related study based on our informal discussions on \cite{kitazawa1}. For the case of Morse functions such that connected components of preimages with no singular points are spheres, \cite{michalak1} is a pioneering work. \cite{michalak2} is also a related work studying fundamental deformations of Reeb graphs of Morse functions on fixed manifolds. These studies of Michalak have also motivated the author. 

\cite{kitazawa3} is a pioneering study related to Main Theorems \ref{mthm:2} and \ref{mthm:3}. They study cases where hypersurfaces $S_j$ are mutually disjoint. The study is also regarded as a study for the case $D_G$ in Remark \ref{rem:4}, presented later.

It is also remarkable that we construct smooth functions which are not real algebraic or real analytic functions in several scenes in the study. Important functions in \cite{saeki2} with \cite{kitazawa2} show explicit cases.
\end{Rem}
%\begin{Rem}
%	As presented in "Remark 1 of \cite{kitazawa5}", in several cases such as so-called "generic" cases, we can have a result similar to one of Main Theorem \ref{mthm:2} by considering "approximations". Approximations are presented shortly in our first section or our introduction.
%For example, we can apply such arguments in the case where the Reeb graphs of Morse functions are generic. In other words, we consider Morse functions such that at distinct singular points, the values are always distinct. For example, a graph with exactly one edge and two vertices and "a graph in the left figure in FIGURE 1 of \cite{kitazawa3}" are regarded as generic. In Main Theorem \ref{mthm:2}, the graph in the latter case is seen as generic if and only if $l=2$.
%
%	On the other hand, we have our results and explicit functions by avoiding such approximations. 
%	We can also know real polynomials explicitly for our desired manifolds and zero sets. We can also give such a comment for Main Theorem \ref{mthm:1}. 
%	Our previous studies \cite{kitazawa3, kitazawa5} also do this.
%	\end{Rem}


\begin{Rem}
\label{rem:3}
We may obtain revised versions of our main result under weaker conditions, for example. However, we respect simple or explicit cases. For example, we respect very explicit cases with the hypersurfaces $S_j$ being spheres of fixed radii in Main Theorem \ref{mthm:1} and show Main Theorems \ref{mthm:2} and \ref{mthm:3}. %Moreover, as presented in the end of the previous section, Main Theorem \ref{thm:2} may be extended to cases where graphs are more general.
\end{Rem}

Related to this, we give several small examples of extensions of Main Theorem \ref{mthm:1}.

\begin{Ex}
\label{ex:1}
In Main Theorem \ref{mthm:1}, we can drop conditions as follows.
\begin{enumerate}
\item We can argue similarly in the case $\overline{D}$ may not be compact. In such a case, $M$ may not be compact.
% In the assumption (\ref{mthm:1.2}), the set ${\bigcap}_{i=1}^{l_3} S_{j_i} \bigcap \overline{D}$ is assumed to be either empty or an {\rm (}$n-l_3${\rm )}-dimensional smooth submanifold of ${\bigcap}_{i=1}^{l_3} S_{j_i}$ as before and in addition, we pose a technical assumption that the set ${\bigcap}_{i=1}^{l_3} S_{j_i} \bigcap \overline{D}$ is also a closed subset of ${\bigcap}_{i=1}^{l_3} S_{j_i}$. The subset ${\bigcap}_{i=1}^{l_3} S_{j_i} \subset {\mathbb{R}}^n$ is an {\rm (}$n-l_3${\rm )}-dimensional smooth regular submanifold of ${\mathbb{R}}^n$ by the transversality as in the original case.
We can argue similarly in the case $\overline{D}$ may not be connected. In such a case, $M$ may not be connected.
\item We can argue similarly in the case the hypersurface $S_j$ may not be connected and may be a union of connected components of the zero set of a polynomial $f_j$.
\item We can also argue similarly in the following case.
\begin{itemize}
\item The zero set of the polynomial $f_j$ has connected components other than (connected components of) $S_j$.
\item The connected components of the zero set of the polynomial $f_j$ except (ones of) $S_j$ are disjoint from the closure $\overline{D}$ of $D$.
\end{itemize} 
%\item In the complementary set ${\mathbb{R}}^k-\overline{D}$, we do not need the assumption "(\ref{mthm:1.2}) in Main Theorem \ref{mthm:1}" on the transversality for intersections of the hypersurfaces $S_j$.
\end{enumerate}
\end{Ex}

\begin{Rem}
\label{rem:4}
We review some important arguments from studies \cite{kitazawa3, kitazawa9} of the author. We also respect a study \cite{bodinpopescupampusorea}, having motivated the author to present these studies. Prepare a finite and connected graph $G$ enjoying the following properties.
\begin{itemize}
\item The degree of each vertex of $G$ is $1$ or $3$.
\item The graph $G$ admits a piecewise smooth embedding $e_G:G \rightarrow {\mathbb{R}}^2$ enjoying the following properties.

\begin{itemize}
\item The function ${\pi}_{2,1} \circ e_G:G \rightarrow \mathbb{R}$ is injective at each edge $e$ of $G$. 
\item If the function ${\pi}_{2,1} \circ e_G:G \rightarrow \mathbb{R}$ has a extremum at a point $p \in G$, then $p$ is a vertex of $G$ whose degree is $1$.
\item At distinct vertices, the values of the functions ${\pi}_{2,1} \circ e_G:G \rightarrow \mathbb{R}$ are always distinct.
\end{itemize}

\end{itemize}
Then we have a natural moment-like map $f_{(D_G,\{S_{G,j}\}_{j=1}^{l_1},m_{G,l_1,l_2},m_{G,l_2})}:M_{(D_G,\{S_{G,j}\}_{j=1}^{l_1},m_{G,l_1,l_2},m_{G,l_2})} \rightarrow {\mathbb{R}}^2$ with the following properties.
\begin{enumerate}
\item We can know the existence of the data $(D_G,\{S_{G,j}\}_{j=1}^{l_1},m_{G,l_1,l_2},m_{G,l_2})$ such that the non-singular real algebraic curves $S_{G,j}$ are mutually disjoint and that the graph $e_G(G)$ is regarded as a graph enjoying the following properties. We also note that the curve $S_{G,j}$ is abused in the data as a suitably defined real polynomial $f_{G,j}$. 
The open set $D_G$ and the family $\{S_{G,j}\}_{j=1}^{l_1}$ of non-singular curves in ${\mathbb{R}}^2$ are obtained from theory \cite{bodinpopescupampusorea, sorea1}: the hypersurfaces $S_{G,j}$ are obtained by approximation of Weierstrass type respecting derivatives, for example.
\begin{itemize}
	\item The boundary $\overline{D_G}-D_G$ of the closure $\overline{D_G}$ and the disjoint union ${\sqcup}_{j=1}^{l_1} S_{G,j}$ coincide.
	\item The Reeb space of the restriction of ${\pi}_{2,1}$ to the closure $\overline{D_G}$ is homeomorphic to the graph $e_G(G)$ and $e_G(G)$ is also embedded in $D_G$. In terms of \cite{bodinpopescupampusorea, sorea1}, the graph $e_G(G)$ is a {\it Poincar\'e-Reeb} graph of $D_G$.
	\item The Reeb space of the restriction of ${\pi}_{2,1}$ to the closure $\overline{D_G}$ is also regarded as a graph. The vertex set of it consists of all points containing points in the boundary $\overline{D_G}-D_G$ being also singular points of the restriction to this boundary of ${\pi}_{2,1}$. The graph $e_G(G)$ is also isomorphic to this graph.
\end{itemize} 
\item The Reeb graph $W_{f_{0,(D_G,\{S_{G,j}\}_{j=1}^{l_1},m_{G,l_1,l_2},m_{G,l_2})}}=W_{{\pi}_{2,1} \circ f_{(D_G,\{S_{G,j}\}_{j=1}^{l_1},m_{G,l_1,l_2},m_{G,l_2})}}$ of the function ${\pi}_{2,1} \circ f_{(D_G,\{S_{G,j}\}_{j=1}^{l_1},m_{G,l_1,l_2},m_{G,l_2})}= f_{0,(D_G,\{S_{G,j}\}_{j=1}^{l_1},m_{G,l_1,l_2},m_{G,l_2})}$ is isomorphic to $G$. We can have this function as a Morse-Bott function. We can show that this is a Morse-Bott function by applying (a slight generalization of) Proposition \ref{prop:1}.
As a specific case, in the case $l_2=1$ or equivalently, the case $m_{G,l_1,l_2}$ is a constant function, we can also show that this is a Morse function.
\end{enumerate}

 %Moreover, as presented in the end of the previous section, Main Theorem \ref{thm:2} may be extended to cases where graphs are more general.
\end{Rem}
We give an example related to Remark \ref{rem:4}.
\begin{Ex}
\label{ex:2}
We first give an explicit graph $G$ and its embedding $e_G$ in Remark \ref{rem:4}. This is shown in blue in FIGURE \ref{fig:6}.
FIGURE \ref{fig:6} shows the image of a moment-like map $f_{(D_{G_{\rm S}},\{S_{G_{{\rm S},j}}\}_{j=1}^{l_1},m_{G_{{\rm S},j},l_1,l_2},m_{G_{{\rm S},j},l_2})}$ into ${\mathbb{R}}^2$ reconstructed from some data $(D_{G_{\rm S}},\{S_{G_{{\rm S},j}}\}_{j=1}^{l_1},m_{G_{{\rm S}},l_1,l_2},m_{G_{{\rm S}},l_2})$. Here the region $D_{G_{\rm S}}$ is a connected open set of ${\mathbb{R}}^2$ surrounded by three circles ($l_1=3$). The Reeb graph of the function $f_{0,(D_{G_{\rm S}},\{S_{G_{{\rm S},j}}\}_{j=1}^{l_1},m_{G_{{\rm S},j},l_1,l_2},m_{G_{{\rm S},j},l_2})}$ is isomorphic to the graph $G$.

For a method to find a region $D_G$ as in Remark \ref{rem:4} whose Poincar\'e-Reeb graph is $G$ ($e_G(G)$), see the original article \cite{bodinpopescupampusorea}.

Note also that this shows an example which is not from Main Theorem \ref{mthm:2} or \ref{mthm:3}: see the four points in the center for singular points of the function $f_{0,(D,\{S_j\}_{j=1}^{l_1},m_{l_1,l_2},m_{l_2})}$ and the smallest circle containing the four points for example, and compare this case to cases of Main Theorems \ref{mthm:2} and \ref{mthm:3}. 
\end{Ex}
\begin{figure}
	
	\includegraphics[height=75mm, width=100mm]{Ex2.2.eps}

	\caption{Example \ref{ex:2}. The image of a moment-like map $f_{(D_{G_{\rm S}},\{S_{G_{{\rm S},j}}\}_{j=1}^{l_1},m_{G_{{\rm S}},l_1,l_2},m_{G_{{\rm S}},l_2})}$ into ${\mathbb{R}}^2$ reconstructed from the data $(D_{G_{\rm S}},\{S_{G_{{\rm S},j}}\}_{j=1}^{l_1},m_{G_{{\rm S}},l_1,l_2},m_{G_{{\rm S}},l_2})$ where a region $D_{G_{\rm S}}$ is surrounded by three circles and colored in gray. The Reeb graph of the function $f_{0,(D_{G_{\rm S}},\{S_{G_{{\rm S},j}}\}_{j=1}^{l_1},m_{G_{{\rm S}},l_1,l_2},m_{G_{{\rm S}},l_2})}:={\pi}_{2,1} \circ f_{(D_{G_{\rm S}},\{S_{G_{{\rm S},j}}\}_{j=1}^{l_1},m_{G_{{\rm S}},l_1,l_2},m_{G_{{\rm S}},l_2})}$, isomorphic to the blue colored graph $G$ ($e_G(G)$). Black dots are for singular points of the function. Note that this also shows a case which is not for Main Theorem \ref{mthm:2} or Main Theorem \ref{mthm:3}. See the four dots in the center and the smallest circle containing the four points for example.}
	\label{fig:6}
\end{figure}
We present a related problem.
\begin{Prob}
Given a graph $G$ (with an embedding $e_G$) in Remark \ref{rem:4}. 
\begin{enumerate}

\item Can we define meaningful classes of  natural moment-like maps $f_G$ such that the Reeb graphs of the functions $f_{0,G}:={\pi}_{2,1} \circ  f_G$ are isomorphic to $G$ and $e_G(G)$? Can we classify such maps of certain classes we define?
\item What is the most natural moment-like map $f_G$ such that the Reeb graph of $f_{0,G}={\pi}_{2,1} \circ f_G$ is isomorphic to $G$ and $e_G(G)$? Can we define (or formulate problems of defining) such maps under certain conditions.
\end{enumerate}
\end{Prob}
\begin{thebibliography}{25}
	\bibitem{buchstaberpanov} V. M. Buchstaber and T. E. Panov, \textsl{Toric topology}, Mathematical Surveys and Monographs, Vol. 204, American Mathematical Society, Providence, RI, 2015.
%	\bibitem{burletderham} O. Burlet and G. de Rham, \textsl{Sur certaines applications g\'en\'eriques d'une vari\'et\'e close a $3$ dimensions dans le plan}, Enseign. Math. 20 (1974). 275--292.
	%	\bibitem{calabi} E. Calabi, Quasi-surjective mappings and a generation of Morse theory, Proc. U.S.-Japan Seminar in Differential Geometry, Kyoto, 1965, pp. 13--16.
	%
	%		\bibitem{cavicchioli} A. Cavicchioli, \textsl{Covering numbers of manifolds and critical points of a Morse function}, Israel. J. Math. 70 (1990), 279--304.
	% \bibitem{cerf} J. Cerf, \textsl{La stratification naturelle des espaces de fonctions deff\'erentiables r'eelles et le th'eor`eme de la pseudo-isotopie}, Inst. Hautes Etudes Sci. Publ. Math. 39 (1970), 5--173.
	%		\bibitem{choimasudasuh} S. Choi, M. Masuda and D. Y. Suh, \textsl{Topological classification of generalized Bott towers}, Trans. Amer. Math. Soc. 362 (2010), 1097--1112.
	%		\bibitem{cornealuptonopreatanre} O. Cornea, G. Lupton, J. Oprea and D. Tanr\'e, \textsl{Lusternik-Schnirelmann category}, Mathematical Surveys and Monographs, 103, Amer. Math. Soc., Providence, RI, 2003.
	%\bibitem{crowleyescher} D. Crowley and C. Escher, \textsl{A classification of $S^3$-bundles over $S^4$}, Differential. Geom. Appl. 18 (2003), 363--380, arXiv:0004147.
	%\bibitem{crowleynordstrom} D. Crowley and J. Nordstr\"{o}m, \textsl{The classification of $2$-connected $7$-manifolds}, Proc. London. Math. Soc. 119 (2019), 1--54, arXiv:1406.2226.

	\bibitem{bochnakcosteroy} J. Bochnak, M. Coste and M.-F. Roy, \textsl{Real algebraic geometry}, Ergebnisse der Mathematik und ihrer Grenzgebiete (3) [Results in Mathematics and Related Areas (3)], vol. 36, Springer-Verlag, Berlin, 1998. Translated from the 1987 French original; Revised by the authors.
		\bibitem{bochnakkucharz} J. Bochnak and W. Kucharz, \textsl{Algebraic approximation of mappings into spheres}, Michigan Mathematical Journal, vol. 34, no. 1, 1987.
	\bibitem{bodinpopescupampusorea} A. Bodin, P. Popescu-Pampu and M. S. Sorea, \textsl{Poincar\'e-Reeb graphs of real algebraic domains}, Revista Matem\'atica Complutense, https://link.springer.com/article/10.1007/s13163-023-00469-y, 2023, arXiv:2207.06871v2.
\bibitem{bott} R. Bott, \textsl{Nondegenerate critical manifolds}, Ann. of Math. 60 (1954), 248--261.
%\bibitem{costantino}  F. Costantino, \textsl{A short introduction to shadows of $4$-manifolds}, Fundamenta Mathematicae 251 no. 2 (2005), 427--442.
%\bibitem{costantinothurston} F. Costantino, D. Thurston, \textsl{$3$-manifolds efficiently bound $4$-manifolds}, J. Topol. 1 (2008),
%703--745.
	\bibitem{delzant} T. Delzant, \textsl{Hamiltoniens p\'eriodiques et images convexes de l'application moment}, Bull. Soc. Math. France 116 (1988), No. 3, 315--339.
%\bibitem{ehresmann} C. Ehresmann, \textsl{Les connexions infinitesimales dans un espace fibre differentiable}, Colloque de Topologie, Bruxelles (1950), 29--55.
%\bibitem{fujitakitabeppumitsuishi} H. Fujita, Y Kitabeppu and A. Mitsuishi, \textsl{Distance functions and convex bodies and symplectic toric manifolds}, arXiv:2003.02293.
%\bibitem{gelbukh} I. Gelbukh, \textsl{Loops in Reeb graphs of $n$-manifolds}, diskrete \& Computational Geometry, 59 (4) (2018), 843--863. 
%%\bibitem{gelbukh2} I. Gelbukh, \textsl{Approximation of Metric Spaces by Reeb Graphs: Cycle Rank of a Reeb Graph, the Co-rank of the Fundamental Group, and Large Components of Level Sets on Riemannian Manifolds}, Filomat (in press), arxiv:1903.00777.
%\bibitem{gelbukh} I. Gelbukh, \textsl{A finite graph is homeomorphic to the Reeb graph of a Morse-Bott function}, Mathematica Slovaca, 71 (3), 757--772, 2021; doi: 10.1515/ms-2021-0018. 
%%\bibitem{gelbukh2} I. Gelbukh, \textsl{Morse-Bott functions with two critical values on a surface}, Czechoslovak Mathematical Journal, 71 (3), 865--880, 2021; doi: 10.21136/CMJ.2021.0125-20. 
\bibitem{golubitskyguillemin} M. Golubitsky and V. Guillemin, \textsl{Stable Mappings and Their Singularities}, Graduate Texts in Mathematics (14), Springer-Verlag (1974).
%\bibitem{hempel} J. Hempel, \textsl{3- Manifolds}, AMS Chelsea Publishing, 2004. 
%\bibitem{hiratukasaeki} J. T. Hiratuka and O. Saeki, \textsl{Triangulating Stein factorizations of generic maps and Euler Characteristic formulas}, RIMS Kokyuroku Bessatsu B38 (2013), 61--89. 
%\bibitem{hiratukasaeki2} J. T. Hiratuka and O. Saeki, \textsl{Connected components of regular fibers of differentiable maps}, in "Topics on Real and Complex Singularities", Proceedings of the 4th Japanese-Australian Workshop (JARCS4), Kobe 2011,  World Scientific, 2014, 61--73. 
%\bibitem{ishikawakoda} M. Ishikawa and Y. Koda, \textsl{Stable maps and branched shadows of $3$-manifolds}, Mathematische Annalen 367 (2017), no. 3, 1819--1863, arXiv:1403.0596.
%\bibitem{kitazawa1} N. Kitazawa, \textsl{On round fold maps} (in Japanese), RIMS Kokyuroku Bessatsu B38 (2013), 45--59.
%\bibitem{kitazawa2} N. Kitazawa, \textsl{On manifolds admitting fold maps with singular value sets of concentric spheres}, Doctoral Dissertation, Tokyo Institute of Technology (2014).
%\bibitem{kitazawa3} N. Kitazawa, \textsl{Fold maps with singular value sets of concentric spheres}, Hokkaido Mathematical Journal Vol.43, No.3 (2014), 327--359.
\bibitem{kitazawa1} N. Kitazawa, \textsl{On Reeb graphs induced from smooth functions on $3$-dimensional closed orientable manifolds with finitely many singular values}, Topol. Methods in Nonlinear Anal. Vol. 59 No. 2B, 897--912, arXiv:1902.08841.
\bibitem{kitazawa2} N. Kitazawa, \textsl{On Reeb graphs induced from smooth functions on closed or open surfaces}, Methods of Functional Analysis and Topology Vol. 28 No. 2 (2022), 127--143, arXiv:1908.04340.
\bibitem{kitazawa3} N. Kitazawa, \textsl{Real algebraic functions on closed manifolds whose Reeb graphs are given graphs}, Methods of Functional Analysis and Topology Vol. 28 No. 4 (2022), 302--308, arXiv:2302.02339, 2023.
\bibitem{kitazawa4} N. Kitazawa, \textsl{Explicit construction of explicit real algebraic functions and real algebraic manifolds via Reeb graphs}, Algebraic and geometric methods of analysis 2023 “The book of abstracts”, 49—51, this is the abstract book of the conference "Algebraic and geometric methods of analysis 2023" and published after a short review (https://www.imath.kiev.ua/$\sim$topology/conf/agma2023/), https://imath.kiev.ua/$\sim$topology/conf/agma2023/contents/abstracts/texts/kitazawa/kitazawa.pdf, 2023.
\bibitem{kitazawa5} N. Kitazawa, \textsl{Notes on explicit special generic maps into Euclidean spaces whose dimensions are greater than $4$}, a revised version is submitted based on positive comments (major revision) by referees and editors after the first submission to a refereed journal, arXiv:2010.10078.

%\bibitem{kitazawa6} N. Kitazawa, \textsl{Round fold maps and the topologies and the differentiable structures of manifolds admitting explicit ones}, submitted to a refereed journal, arXiv:1304.0618.
%\bibitem{kitazawa0.5} N. Kitazawa, \textsl{Constructing fold maps by surgery operations and homological information of their Reeb spaces}, submitted to a refereed journal, arxiv:1508.05630.
%\bibitem{kitazawa0.6} N. Kitazawa, \textsl{Notes on fold maps obtained by surgery operations and algebraic information of their Reeb spaces}, arxiv:1811.04080.


%\bibitem{kitazawa6} N. Kitazawa, \textsl{On Reeb graphs induced from smooth functions on $3$-dimensional closed manifolds which may not be orientable}, a revised version is submitted to a refereed journal after based on positive comments by editors and referees after the second submission to a refreed journal, arXiv:2108.01300.
%\bibitem{kitazawa7} N. Kitazawa, \textsl{Realization problems of graphs as Reeb graphs of Morse functions with prescribed preimages}, submitted to a refereed journal, arXiv:2108.06913.
%\bibitem{kitazawa10} N. Kitazawa,\textsl{Round fold maps on $3$-dimensional manifolds and their integral and rational cohomology rings}, arXiv:2301.07008.
\bibitem{kitazawa6} N. Kitazawa, \textsl{A class of naturally generalized special generic maps}, arXiv:2212.03174.
\bibitem{kitazawa7} N. Kitazawa, \textsl{Construction of real algebraic functions with prescribed preimages}, submitted to a refereed journal as the second version based on positive comments by referees and editors, arXiv:2303.00953v3.
\bibitem{kitazawa8} N. Kitazawa, \textsl{Natural real algebraic maps of non-positive codimensions with prescribed images whose boundaries consist of non-singular real algebraic hypersurfaces satisfying transversality}, a previous version of the present paper, arXiv:2303.10723v5.
%\bibitem{kitazawa6} N. Kitazawa, \textsl{A note on real algebraic maps which are topologically special generic maps}, previous version(s) of the present article and the version arXiv:2303.00953v2 is submitted to a refereed journal, arXiv:2312.10646. 
\bibitem{kitazawa9} N. Kitazawa, \textsl{Some remarks on real algebraic maps which are topologically special generic maps}, submitted to a refereed journal, arXiv:2312.10646. 
\bibitem{kitazawa10} N. Kitazawa, \textsl{A note on cohomological structures of special generic maps}, a revised version is submitted based on positive comments by referees and editors after the third submission to a refereed journal.
%\bibitem{kitazawasaeki1} N. Kitazawa and O. Saeki, \textsl{Round fold maps on $3$-manifolds}, accepted for publication after a refereeing process and to appear in Algebraic \& Geometric Topology, arXiv:2105.00974.
		%	\bibitem{kitazawasaeki2} N. Kitazawa and O. Saeki, \textsl{Round fold maps of $n$-dimensional manifolds into ${\mathbb{R}}^{n-1}$}, submitted to a refereed journal, arXiv:2111.13510.
%\bibitem{ishikawakoda} M. Ishikawa, Y. Koda, \textsl{Stable maps and branched shadows of $3$-manifolds}, arXiv:1403.0596.
%\bibitem{kobayashisaeki} M. Kobayashi and O. Saeki, \textsl{Simplifying stable mappings into the plane from a global viewpoint}, Trans. Amer. Math. Soc. 348 (1996), 2607--2636.
\bibitem{kohnpieneranestadrydellshapirosinnsoreatelen} K. Kohn, R. Piene, K. Ranestad, F. Rydell, B. Shapiro, R. Sinn, M-S. Sorea and S. Telen, \textsl{Adjoints and Canonical Forms of Polypols}, arXiv:2108.11747.
\bibitem{kollar} J. Koll\'ar, \textsl{Nash's work in algebraic geometry}, Bulletin (New Series) of the American Matematical Society (2) 54, 2017, 307--324.
\bibitem{kucharz} W. Kucharz, \textsl{Some open questions in real algebraic geometry}, Proyecciones Journal of Mathematics, Vol. 41 No. 2 (2022), Universidad Cat\'olica del Norte Antofagasta, Chile, 437--448.
%\bibitem{martinezalfaromezasarmientooliveira} J. Martinez-Alfaro, I. S. Meza-Sarmiento and R. Oliveira, \textsl{Topological  classification of simple Morse Bott functions on surfaces}, Contemp. Math. 675 (2016), 165--179.%
%\bibitem{masumotosaeki} Y. Masumoto and O. Saeki, \textsl{A smooth function on a manifold with given Reeb graph}, Kyushu J. Math. 65 (2011), 75--84.
\bibitem{maciasvirgospereirasaez} E. Mac\'ias-Virg\'os and M. J. Pereira-S\'aez, Height functions on compact symmetric spaces, Monatshefte f\"ur Mathematik 177 (2015), 119--140. 
\bibitem{michalak1} L. P. Michalak, \textsl{Realization of a graph as the Reeb graph of a Morse function on a manifold}. Topol. Methods in Nonlinear Anal. 52 (2) (2018), 749--762, arXiv:1805.06727.
\bibitem{michalak2} L. P. Michalak, \textsl{Combinatorial modifications of Reeb graphs and the realization problem}, Discrete Comput. Geom. 65 (2021), 1038--1060, arXiv:1811.08031.
%\bibitem{milnor} J. Milnor, \textsl{Singular points of complex hypersurfacs}, Annals of Mathematics Studies, No. 61, Princeton University Press, Princeton, N. J.; University of Tokyo Press, Tokyo, 1968.
%\bibitem{milnor} J. Milnor, \textsl{Lectures on the h-cobordism theorem}, Math. Notes, Princeton Univ. Press, Princeton, N.J. 1965.
\bibitem{moise} E. E. Moise, \textsl{Affine Structures in $3$-Manifold{\rm :} V. The Triangulation Theorem and Hauptvermutung}, Ann. of Math., Second Series, Vol. 56, No. 1 (1952), 96--114.
%\bibitem{morin} B. Morin, \textsl{Formes canoniques des singulariti\'{e}s d\'{}une application diff\'{e}rentiable}, C. E. Acad. Sci. Paris 260 (1965), 5662--5665, 6503--6506.
\bibitem{nash} J. Nash, \textsl{Real algbraic manifolds}, Ann. of Math. (2) 56 (1952), 405--421.
%\bibitem{ranicki} A. Ranicki, \textsl{Algebraic and geometric surgery}, https://www.maths.ed.ac.uk/~v1ranick/books/surgery.pdf, 2002.
\bibitem{ramanujam} S. Ramanujam, \textsl{Morse theory of certain symmetric spaces}, J. Diff. Geom. 3 (1969), 213--229.
\bibitem{reeb} G. Reeb, \textsl{Sur les points singuliers d\'{}une forme de Pfaff compl\'{e}tement int\`{e}grable ou d\'{}une fonction num\'{e}rique}, Comptes Rendus
 Hebdomadaires des S\'{e}ances de I\'{}Acad\'{e}mie des Sciences 222 (1946), 847--849.
%\bibitem{saeki1} O. Saeki, \textsl{Notes on the topology of folds}, J. Math. Soc. Japan Volume 44, Number 3 (1992), 551--566.
\bibitem{saeki1} O. Saeki, \textsl{Topology of special generic maps of manifolds into Euclidean spaces}, Topology Appl. 49 (1993), 265--293.
%\bibitem{saeki0.2} O. Saeki, \textsl{Topology of singular fibers of differentiable maps}, Lecture Notes in Math., Vol. 1854, Springer-Verlag, 2004. 
%\bibitem{saeki4} O. Saeki, \textsl{Morse functions with sphere fibers}, Hiroshima Math. J. Volume 36, Number 1 (2006),  141--170.
\bibitem{saeki2} O. Saeki, \textsl{Reeb spaces of smooth functions on manifolds}, International Mathematics Research Notices, maa301, Volume 2022, Issue 11, June 2022, 3740--3768, https://doi.org/10.1093/imrn/maa301, arXiv:2006.01689.
%\bibitem{saekitakase} O. Saeki and M. Takase, \textsl{Desingularizing special generic maps}, Journal of G\"okova Geometry Topology (2013), 1--24.
%\bibitem{sakurai} S. Sakurai, Master Thesis, Kyushu. Univ..
% \bibitem{saekitakase} O. Saeki and M. Takase, \textsl{Desingularizing special generic maps}, Journal of Gokova Geometry Topology 7 (2013), 1--24.
%\bibitem{saeki2} O. Saeki, \textsl{Topology of special generic maps of manifolds into Euclidean spaces}, Topology Appl. 49 (1993), 265--293.
%\bibitem{saeki4} O. Saeki, \textsl{Singular fibers and $4$-dimensional cobordism group}, Pacific J. Math. 248 (2010), 233--256.
%\bibitem{saekisakuma} O. Saeki and K. Sakuma, \textsl{On special generic maps into ${\mathbb{R}}^3$}, Pacific J. Math. 184 (1998), 175--193.
%\bibitem{saekisuzuoka} O. Saeki and K. Suzuoka, \textsl{Generic smooth maps with sphere fibers} J. Math. Soc. Japan Volume 57, Number 3 (2005), 881--902.
\bibitem{sharko} V. Sharko, \textsl{About Kronrod-Reeb graph of a function on a manifold}, Methods of Functional Analysis and
 Topology 12 (2006), 389--396.
%\bibitem{shiota} M. Shiota, \textsl{Thom's conjecture on triangulations of maps}, Topology 39 (2000), 383--399.
\bibitem{sorea1} M. S. Sorea, \textsl{The shapes of level curves of real polynomials near strict local maxima},  Ph. D. Thesis, Universit\'e de Lille, Laboratoire Paul Painlev\'e, 2018.
\bibitem{sorea2} M. S. Sorea, \textsl{Measuring the local non-convexity of real algebraic curves}, Journal of Symbolic Computation 109 (2022), 482--509.
%\bibitem{sorea1} M. S. Sorea, \textsl{The shapes of level curves of real polynomials near strict local maxima},  Ph. D. Thesis, Universit\'e de %Lille, Laboratoire Paul Painlev\'e, 2018.
%\bibitem{sorea2} M. S. Sorea, \textsl{Measuring the local non-convexity of real algebraic curves}, J. Symbolic Compute. 109 (2022), 482--509.
%\bibitem{stong} R. E. Stong, \textsl{Notes on cobordsm theory}, Princeton Universty Press, 1968.
\bibitem{takeuchi} M. Takeuchi, \textsl{Nice functions on symmetric spaces}, Osaka. J. Mat. (2) Vol. 6 (1969), 283--289.
%\bibitem{thom} R. Thom, \textsl{Les singularites des applications differentiables}, Ann. Inst. Fourier (Grenoble) 6 (1955-56), 43--87.
\bibitem{tognoli} A. Tognoli, \textsl{Su una congettura di Nash}, Ann. Scuola Norm. Sup. Pisa (3) 27 (1973), 167--185.
%\bibitem{turaev} Vladimir G. Turaev, \textsl{Topology of shadows}, Preprint, 1991.
%\bibitem{wall} C. T. C Wall, \textsl{Classification problems in differential topology -- {\rm I:} Classificationon handlebodies}, Topology 2 (1963), 253--261.
%\bibitem{wall2} C. T. C. Wall \textsl{Classification problems in differential topology -- {\rm II:} Diffeomorphismsof handlebodies}, Topology 2 (1963), 263--272.
%\bibitem{wall3} C. T. C. Wall, \textsl{Classification problems in differential topology -- {\rm Q:} Quadratic forms on finite groups and related topics}, Topology 2 (1963), 281--298.
%\bibitem{wall4} C. T. C. Wall, \textsl{Classification problems in differential topology -- {\rm III:} Applications to special cases}, Topology 3 (1965), 291--304.
%%\bibitem{wall5} C. T. C. Wall, \textsl{Classification problems in differential topology -- {\rm IV:} Thickenings}, Topology 5 (1966), 73--94.
%\bibitem{wall6} C. T. C. Wall, \textsl{Classification problems in differential topology -- {\rm VI:} Classification of |{\rm (}$s-1${\rm )}-connected {\rm (}$2s+1${\rm )}-manifolds}, Topology 6 (3) (1967), 273--296.
%\bibitem{whitney} H.  Whitney,  \textsl{On singularities of mappings of Euclidean spaces: I,  mappings of the plane into the plane},  Ann.  of Math.  62 (1955),  374--410. 

	%\bibitem{zhubr1} A. V. Zhubr, Closed simply-connected six-dimensional manifolds: proofs of classification theorems, Algebra i Analiz 12 (2000), no. 4, 126--230.
%\bibitem{zhubr2} A. V. Zhubr (responsible for the page), http://www.map.mpim-bonn.mpg.de/6-manifolds:\_1-connected.
\end{thebibliography}
\end{document} 
%\bibitem{wrazidlo} D. Wrazidlo, \textsl{Bordism of constrained Morse functions}, arxiv:1803.11177.&
      % points 
      %  Thom's 
      % An answer will be presented in a forthcoming paper.
%STEP 1
% We delete the exposition via handle attachments to surfaces and decided to avoidusing the figure ''3hd.eps''. We respect the last version. First we start with Michalak'ws argument and generalize.  
%Step 2 Case 1 A_i → E_i
%We corrected some other minor phrases without changing the argument.
%Case 2 We changed the exposision of fold maps: we construct the fold map by using a defromation of smooth functions on an closed and connected surface of genus $0$ with holes (for the functions ''20201125func.eps'' is added). 
%In Problem 3, we adopt this way to present the answer more clearly and we added an exposition on smooth functions and a deformation of these functionsn we need. For STEP 1 and STEP 2 Case 1 in this scene, we introduce functions ${t^{\prime}}_{i,s_1,s_0,s_2}$ and ${t^{\prime}}_{d,s_1,s_0,s_2}$.
%The exposition of an answer to Problem 2 is revised.  