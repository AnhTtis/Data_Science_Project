\documentclass{amsart}
%\setlength{\textheight}{43pc}
%\setlength{\textwidth}{28pc}
\usepackage{amsfonts}
\usepackage{amsmath,amssymb}
\usepackage{amsthm}
\usepackage{amscd}
\usepackage{graphics}
\usepackage{graphicx}

%\usepackage[pagewise]{lineno}\linenumbers
\theoremstyle{remark}{
\newtheorem{Def}{{\rm Definition}}
\newtheorem{Ex}{{\rm Example}}
\newtheorem{Rem}{{\rm Remark}}
\newtheorem{Prob}{{\rm Problem}}
\newtheorem*{MainProb}{Main Problem}
}
\theoremstyle{plain}
{
\newtheorem{Cor}{Corollary}
\newtheorem{Prop}{Proposition}
\newtheorem{Thm}{Theorem}
\newtheorem{MainThm}{Main Theorem}
\newtheorem*{MainCor}{Main Corollary}
\newtheorem{Lem}{Lemma}
\newtheorem{Fact}{Fact}
}
\renewcommand*{\urladdrname}{\itshape Webpage}
\begin{document}
\title[New explicit real algebraic maps of non-positive codimensions]{Natural real algebraic maps of non-positive codimensions with prescribed images whose boundaries consist of non-singular real algebraic hypersurfaces intersecting with transversality}
\author{Naoki kitazawa}
\keywords{(Non-singular) real algebraic manifolds and real algebraic maps. Smooth maps. Moment maps. Graphs. Reeb graphs. \\
\indent {\it \textup{2020} Mathematics Subject Classification}: Primary~14P05, 14P25, 57R45, 58C05. Secondary~57R19.}

\address{Institute of Mathematics for Industry, Kyushu University, 744 Motooka, Nishi-ku Fukuoka 819-0395, Japan\\
 TEL (Office): +81-92-802-4402 \\
 FAX (Office): +81-92-802-4405 \\
}
\email{n-kitazawa@imi.kyushu-u.ac.jp, naokikitazawa.formath@gmail.com}
\urladdr{https://naokikitazawa.github.io/NaokiKitazawa.html}
\maketitle
\begin{abstract}
We present new real algebraic maps of non-positive codimensions with prescribed images whose boundaries consist of explicit non-singular real algebraic hypersurfaces intersecting with so-called "transversality". We also have the maps with explicit information on important real polynomials.

This gives new construction of explicit real algebraic maps of non-positive codimensions with explicit information on images, preimages and important real polynomials. 
Thanks to some celebrated theory of Nash, smooth closed manifolds are {\it non-singular} real algebraic manifolds and the zero sets of some real polynomials. In considerable cases, we can approximate smooth functions or more generally, maps, by real algebraic ones. It is in general difficult to have explicit examples. We have constructed some previously. We have ones of some new type with explicit information on global structures of the maps.  

 


\end{abstract}
%【REVISE】 combinatoric ~ is → combinatorial object. It is .
%【REVISE】  such that a point is a vertex if and only if the corresponding connected component of the level set contains some singular points → whose vertex set is the set of all points containing some singular points in the corresponding connected component of the level set .
%【REVISE】 We delete "extending the result before".
\section{Introduction.}
\label{sec:1}

Real algebraic geometry has a long nice history and studies ({\it non-singular}) real algebraic manifolds or more generally, varieties. 
For related topics, see \cite{bochnakcosteroy, bochnakkucharz, kollar, kucharz, nash, tognoli} for example. Here, we only consider non-singular real algebraic manifolds essentially.

Some of Nash's celebrated theory shows that smooth closed manifolds are non-singular real algebraic manifolds and the zero sets of some real polynomials. In considerable cases, we can approximate smooth functions on {\it non-singular} real algebraic manifolds or more generally, maps between non-singular real algebraic manifolds, by real algebraic ones. On the other hand, here, we concentrate on construction of explicit real algebraic maps of non-positive codimensions and their explicit global information. Some explicit examples of such functions and maps have been well-known. Canonical projections of spheres embedded naturally in Euclidean spaces are simplest ones. Some natural functions on so-called projective spaces, Lie groups and their quotient spaces are also well-known. They are represented as real polynomials. 
Knowing explicit global information of the maps and the manifolds such as topological properties is difficult in general.
See \cite{maciasvirgospereirasaez, ramanujam, takeuchi} for example.

We give some new answers to this difficult problem explicitly. 
We construct real algebraic maps whose codimensions are non-positive with prescribed images and preimages. The boundaries of the images consist of non-singular real algebraic hypersurfaces intersecting with the condition on "transversality. In addition, preimages are products of spheres. We can also know real polynomials for the desired zero sets and the manifolds explicitly.
%We also study global structures of the functions obtained as the compositions with canonical projections.
Our method applies some singularity theory of smooth maps, some theory on differential topology of manifolds, and some elementary real algebraic arguments and it is interdisciplinary. Our study also adds to our related pioneering studies \cite{kitazawa3, kitazawa5}. \cite{bodinpopescupampusorea, kohnpieneranestadrydellshapirosinnsoreatelen, 
	sorea1, sorea2}, which are studies on domains formed by real algebraic hypersurfaces, also motivates us.
\subsection{Smooth manifolds and maps.}
Let $X$ be a topological space having the structure of some cell complex whose dimension is finite. We can define the dimension $\dim X$ uniquely. $\dim X$ is an integer of course. 
A topological manifold is well-known to have the structure of a CW complex. A smooth manifold is well-known to have the structure of a polyhedron. We can define the structure of a certain polyhedron for a smooth manifold canonically. This is a so-called PL manifold. It is also well-known that a topological space having the structure of a polyhedron whose dimension is at most $2$ has the structure of a polyhedron uniquely. For a topological manifold, the same fact holds in the case where the dimension is at most $3$. This respects celebrated and well-known theory by \cite{moise} for example. 
  
Let ${\mathbb{R}}^k$ denote the $k$-dimensional Euclidean space. It is a simplest $k$-dimensional smooth manifold and the Riemannian manifold with the standard Euclidean metric. 
Let $\mathbb{R}:={\mathbb{R}}^1$ and $\mathbb{N} \subset \mathbb{Z} \subset \mathbb{R}$ be the set of all positive integers and that of all integers respectively.
For each point $x \in {\mathbb{R}}^k$, we can define $||x|| \geq 0$ as the distance between $x$ and the origin $0$ under this metric.
This is also naturally a simplest real algebraic manifold: the $k$-dimensional real affine space. $S^k:=\{x \in {\mathbb{R}}^{k+1} \mid ||x||=1\}$ denotes the $k$-dimensional unit sphere, which is a $k$-dimensional smooth compact submanifold of ${\mathbb{R}}^{k+1}$ and has no boundary. It is connected for any positive integer $k \geq 1$. It is a discrete set with exactly two points for $k=0$. It is the zero set of the real polynomial $||x||^2={\Sigma}_{j=1}^{k+1} {x_j}^2$ with  $x:=(x_1,\cdots,x_{k+1})$ and a real algebraic (sub)manifold (hypersurface). 
%It is also a smooth real algebraic submanifold defined by the zero set of the real polynomial $||x||-1={\Sigma}_{j=1}^{k+1} {x_j}^2 -1$ where $x:=(x_1,\cdots,x_{k+1})$.
$D^k:=\{x \in {\mathbb{R}}^{k} \mid ||x|| \leq 1\}$ is the $k$-dimensional unit disk. It is a $k$-dimensional smooth compact and connected submanifold of ${\mathbb{R}}^{k}$ for any non-negative integer $k \geq 0$.

Let $c:X \rightarrow Y$ be a differentiable map from a differentiable manifold $X$ into another one $Y$. $x \in X$ is a {\it singular} point of $c$ if the rank of the differential at $x$ is smaller than the minimum of the (multi)set $\{\dim X, \dim Y\}$. $c(x)$ is a {\it singular value} of $c$.
Let $S(c)$ denote the set of all singular points of $c$. In our paper, we consider smooth maps, defined as maps of the class $C^{\infty}$, as differentiable maps, unless otherwise stated. A canonical projection of the Euclidean space ${\mathbb{R}}^k$ is denoted
by ${\pi}_{k,k_1}:{\mathbb{R}}^{k} \rightarrow {\mathbb{R}}^{k_1}$. This is defined as the map mapping 
each point $x=(x_1,x_2) \in {\mathbb{R}}^{k_1} \times {\mathbb{R}}^{k_2}={\mathbb{R}}^k$ to the first component $x_1 \in {\mathbb{R}}^{k_1}$ with the conditions on the dimensions given by $k_1, k_2>0$ and $k=k_1+k_2$. A canonical projection of the unit sphere $S^{k-1}$ is the restriction of it.

Here, a real algebraic manifold is represented as some union of connected components of the intersection of finitely many zero sets of real polynomials. {\it Non-singular} real algebraic manifolds are defined naturally by the implicit function theorem for the polynomials or the maps defined canonically from the polynomials. The real affine space and the unit sphere are simplest non-singular ones. {\it Real algebraic} maps here are represented as the canonical embeddings into the real affine spaces with canonical projections.


\subsection{Our main results.}
% \cite{masumotosaeki} generalizes the pioneering work of \cite{sharko}. \cite{sharko} constructs nice smooth functions on closed surfaces and \cite{masumotosaeki} extends this to arbitrary finite graphs.
%Later, for example, \cite{martinezalfaromezasarmientooliveira, michalak} have set explicit problems and solved. Before the study \cite{kitazawa1} of the author, functions are, ones on closed surfaces or ones preimages containing no singular points of which are disjoint unions of spheres essentially.

%However we can apply some important cases. For example \cite{saeki2} considers very general cases and we cannot use analytic functions for construction there.

Hereafter, let ${\mathbb{N}}_a$ be the set of all elements of $\mathbb{N}$ smaller than or equal to a given real number $a$. For example, for a positive integer $i$, ${\mathbb{N}}_i:=\{1,\cdots,i\}$ and the set of all integers from $1$ to $i$.

\begin{MainThm}
	\label{mthm:1}
	Let $l_1$, $l_2$ and $n$ be positive integers.
	Let $\{S_j\}_{j=1}^{l_1}$ be a family of non-singular
	real algebraic hypersurfaces of ${\mathbb{R}}^n$ each $S_j$ of which is connected and the zero set of a real polynomial $f_j$.
	Let $D$ be a connected open subset of ${\mathbb{R}}^n$ whose closure $\overline{D}$ is compact and connected. Suppose also that for an arbitrary sufficiently small connected open neighborhood $U_D$ of $\overline{D}$ the following are enjoyed.
	\begin{itemize}
		\item $S_j \bigcap D$ is empty. $S_j \bigcap \overline{D}$ is connected and non-empty. %The intersection of $U_D$ and the zero set of $f_j$ is $U_D \bigcap S_j$
 $D=U_D \bigcap {\bigcap}_{j=1}^{l_1} \{x \mid f_j(x)>0\}$, $\overline{D}=U_D \bigcap {\bigcap}_{j=1}^{l_1} \{x \mid f_j(x) \geq 0\}$ and $\overline{D}-D \subset {\bigcup}_{j=1}^{l_1} S_j$ hold. 
		\item A surjective map $m_{l_1,l_2}:{\mathbb{N}}_{l_1} \rightarrow {\mathbb{N}}_{l_2}$ enjoying the following properties exists.
		\begin{itemize}
			\item For the intersection ${\bigcap}_{i=1}^{i_0} S_{j_i}$ which is not empty for some positive integer $i_0$ and some increasing sequence $\{j_i\}_{i=1}^{i_0}$, the restriction of $m_{l_1,l_2}$ to the set $\{j_i\}_{i=1}^{i_0}$ is injective and ${\bigcap}_{i=1}^{i_0} S_{j_i}$ is a non-singular real algebraic submanifold of dimension $n-i_0$.
		  \item ${\bigcap}_{i=1}^{i_0} S_{j_i} \bigcap \overline{D}$ is a {\rm (}cornered{\rm )} smooth compact manifold. 
\item A kind of "transversality" holds{\rm :} for each point $p$ in the intersection ${\bigcap}_{i=1}^{i_0} S_{j_i}$, the intersection of the images of the differentials at points ${e_{j_i}}^{-1}(p)$ of the canonical embeddings $e_{j_i}$ for all $S_{j_i}$ and the image of the differential of the canonical embedding $e_{\{j_i\}_{i=1}^{i_0}}$ of ${\bigcap}_{i=1}^{i_0} S_{j_i}$ at one point ${e_{\{j_i\}_{i=1}^{i_0}}}^{-1}(p)$ agree. See \cite{golubitskyguillemin} for related notions.
		  \end{itemize}
	\end{itemize}
Let $m_{l_2}:{\mathbb{N}}_{l_2} \rightarrow \mathbb{N} \sqcup \{0\}$ be a map. Let $m:=n+{\Sigma}_{j=1}^{l_2} m_{l_2}(j)$. Then there exist an $m$-dimensional non-singular real algebraic closed and
connected manifold $M$ and a smooth real algebraic map $f:M \rightarrow {\mathbb{R}}^n$ such that the image $f(M)$ is the closure $\overline{D}$ and that 
the dimension of each preimage $f^{-1}(p)$ {\rm (}$p \in \overline{D}${\rm )} is at most $m-n$.
	\end{MainThm}

This is an extension or a variant of main results of the author, presented in \cite{kitazawa3, kitazawa5}. There the hypersurfaces do not intersect. See also \cite{kitazawa4}. Main Theorem \ref{mthm:2}, presented in the third section, is also a kind of these extensions. This is on the functions represented as the compositions of the maps with canonical projections and their global structures. %We need so-called {\it Reeb graphs}. They are graphs representing the functions compactly.
The next section explains about Main Theorem \ref{mthm:1}.
Main Theorem \ref{mthm:2} is presented as an application of this in the third section. The fourth section presents additional remarks on our study. \\
\ \\
\noindent {\bf Conflict of interest.} \\
The author was a member of the project JSPS Grant Number JP17H06128.
The author was a member of the project JSPS KAKENHI Grant Number JP22K18267. Principal Investigator for them is all Osamu Saeki.  The author works at Institute of Mathematics for Industry (https://www.jgmi.kyushu-u.ac.jp/en/about/young-mentors/) and this is closely related to our study. Our study thanks them for their supports. The author is also a researcher at Osaka Central
Advanced Mathematical Institute (OCAMI researcher). He is not employed there. This is for our studies
and our study also thanks this. \\
\ \\
{\bf Data availability.} \\
Data essentially supporting our present study are all in the
 paper.
\section{On Main Theorem \ref{mthm:1}.}
%\subsection{Additional several terminologies and notions.}
%A {\it diffeomorphism} is a smooth homeomorphism with no singular points. A {\it diffeomorphism on a smooth manifold} is a diffeomorphism from the manifold to itself.
%%The {\it diffeomorphism type} of a smooth manifold is defined as the equivalence class under the natural equivalence relation. 
%This is defined on the family of all smooth manifolds defined by the existence of diffeomorphisms. Two manifolds whose diffeomorphism types are same are said to be {\it diffeomorphic}.

%The {\it diffeomorphism group} of a smooth manifold is the group of all diffeomorphisms there. This is topologized with the so-called {\it Whitney $C^{\infty}$ topology} and a so-called topological group. More generally, Whitney $C^{\infty}$ topologies on the set of all smooth maps between two given smooth manifolds and subspaces of the space are important in the (singularity) theory of smooth maps for example. See \cite{golubitskyguillemin} for example. This is a book for elementary and some advanced theory of singularity theory of differentiable maps.

%A {\it smooth} bundle is a bundle whose fiber is a smooth manifold and whose structure group is regarded as (some subgroup) of the diffeomorphism group of the fiber.

%We introduce {\it fold} maps. For the Whitney $C^{\infty}$ topology and fold maps for example, see also \cite{golubitskyguillemin} as a book for elementary singularity theory of differentiable maps
%for example.

%\begin{Def}
%	Let $X$ and $Y$ be smooth manifolds with no boundaries satisfying $\dim X \geq \dim Y$.
%	A {\it fold} map $c:X \rightarrow Y$ is a smooth map such that at each singular point $p$, we have suitable integer $0 \leq i(p) \leq \frac{\dim X-\dim Y+1}{2}$ and local coordinates around $p$ and $c(p)$ and that there we have a local form $c(x_1,\cdots,x_{\dim X})=(x_1,\cdots,x_{\dim Y-1},{\Sigma}_{j=1}^{\dim X-\dim Y-i(p)+1} {x_{\dim Y-1+j}}^2-{\Sigma}_{j=1}^{i(p)} {x_{\dim X-i(p)+j}}^2)$ around the singular point $p$ and the singular value $c(p)$.
%\end{Def}
%5\begin{Prop}
%	In the previous definition, $i(p)$ is chosen uniquely and defined as the {\rm index} of $p$. The set of all singular points of $c$ of a fixed index is a smooth regular submanifold of $c$ and of dimension $\dim Y-1$. If $X$ is closed, then the submanifold is compact and with no boundary. The restriction to the submanifold is a smooth immersion. 
%\end{Prop}
%5\begin{Def}
%	If in the definition of the fold map, $i(p)=0$ always holds, then this is called a {\it special generic} map.
%\end{Def}
%A Morse function is of course a fold map. 
%In short, fold maps are locally projections or the product map of a Morse function and the identity map on some disk. For special generic maps, the Morse function is chosen as a so-called {\it height function} of a unit disk. A {\it height function} $h$ of a unit sphere is a function of the form $h(x)=\pm ||x||^2+c$ where $c$ is some real number.

%We introduce very fundamental and explicit special generic maps. They are also keys in our main results.

%\begin{Ex}
%\label{ex:1}
%	\begin{enumerate}
	%	\item Canonical projections of unit spheres are special generic. The restrictions to the singular sets, which are also regarded as unit spheres, are embeddings. The images are regarded as the unit disks whose dimensions are same as those of the Euclidean spaces of the targets.
	%	\item Let $m>n$ be positive integers. 
	%	Let $M$ be an $m$-dimensional smooth manifold represented as a connected sum of $l> 0$ manifolds diffeomorphic to $S^{k_j} \times S^{m-k_j}$ for each integer $1 \leq j \leq l$ and some integer $1 \leq k_j \leq n-1$ where the connected sum is taken in the smooth category. We easily have a special generic map $f:M \rightarrow {\mathbb{R}}^n$ such that the restriction to the singular set $S(f)$ is an embedding and that the image is a smoothly embedded submanifold diffeomorphic to one represented as a boundary connected sum of $l> 0$ manifolds diffeomorphic to $S^{k_j} \times D^{n-k_j}$ for each integer $1 \leq j \leq l$. The boundary connected sum is, as before, taken in the smooth category. 
%		\end{enumerate}
%For these maps, we also have the following two trivial smooth bundles. We consider a case where $m$ and $n$ are the dimensions of the manifolds of the domain and the target.
%\begin{itemize}
%	\item We have some small collar neighborhood of the boundary of the image and the composition of the restriction of the map to the preimage with the canonical projection to the boundary gives a trivial smooth bundle whose fiber is diffeomorphic to a unit disk $D^{m-n+1}$. 
%	\item On the complementary set of the interior of the collar neighborhood, the restriction of the map gives a trivial smooth bundle whose fiber is diffeomorphic to a unit sphere $S^{m-n}$.
%\end{itemize}
%Moreover, they are glued by the product map of the diffeomorphism for the natural identification between the base spaces and the identity map on the fiber. Note that fibers are identified in some canonical way. For such special generic maps, see also the preprint \cite{kitazawa4, kitazawa7} of the author.
%\end{Ex}



%A {\it Morse-Bott} function is a smooth function on a manifold with no boundary at each singular point of which it is represented as the composition of some projection with a Morse function for suitable local coordinates. See \cite{bott}.

%As our main work, we construct real algebraic functions such that for each singular point, the singularity is as one of these cases (at least) topologically. We do not investigate these singularities in our main theorems. This presents another fundamental, important and difficult problem on singularities of polynomial maps or more generally, smooth maps.

%To simplify our arguments, let us assume the following where $l \geq 0$ is an integer.
%\begin{itemize}
%\item For each hypersurface $S_j$ in the family $\{S_j\}_{j=1}^l$, a real polynomial $f_{{\rm P}, S_j}$ is given so that the zero set and $S_j$ coincide and that the polynomial function $f_{{\rm P}, S_j}:S_j \rightarrow \mathbb{R}$ defined canonically has no singular points on $S_j$.
%\item $D$ is assumed to be the intersection ${\bigcap}_{j=1}^l  \{x \in {\mathbb{R}}^k \mid  f_{{\rm P}, S_j}(x)>0 \}$.
%\end{itemize}

%For example, the interior ${\rm Int}\ D^k$ of $D^k:=\{x \in {\mathbb{R}}^{k} \mid ||x|| \leq 1 \}$ is a simplest example and $D^k$ is the $k$-dimensional unit disk. This is also a $k$-dimensional smooth, compact and connected submanifold.
%Note that $||x||={\Sigma}_{j=1}^k {x_j}^2$ where $x:=(x_1,\cdots,x_k)$.

%A {\it Poincar\'e-Reeb graph} is defined for a pair of an algebraic domain $D$ of the real affine space of dimension $k>1$ and a canonical projection ${\pi}_{k,1}$ mapping $(x_1,x_2) \in {\mathbb{R}}^{k}$ to $x_1 \in {\mathbb{R}}$. This can be presented in a more general manner.
%Hereafter, we mainly respect the preprint \cite{bodinpopescupampusorea} and there such cases are discussed. Note that terminologies and situations are different in considerable cases and that here we can argue in a self-contained way.
%\begin{Def}
%\label{def:1}
%A {\it Poincar\'e-Reeb graph for the pair $(D,{\pi}_{k,1})$} is a graph in the real affine space embedded by a piecewise smooth embedding with the following conditions.
%\begin{itemize}
%\item Each edge $e$ intersects each preimage of the projection ${\pi}_{k,1}$ in a so-called {\it generic} way or satisfying the "transversality". In other words, each edge is embedded smoothly and for each point $p_e$ in each edge $e$, the image of the differential at the point and the tangent vector space at the value $v(p_e)$ in the preimage ${{\pi}_{k,1}}^{-1}(p)$ of a suitable {\rm (}unique{\rm )} point $p$ by the projection ${\pi}_{k,1}$ generate the tangent vector space at the point $v(p_e) \in {\mathbb{R}}^k$. 
%\item Two points in the closure $\overline{D}$ of $D$ can be defined to be equivalent if and only if they are in a same connected component of the preimage $\overline{D} \bigcap {{\pi}_{k,1}}^{-1}(p)$ for some point $p \in {\mathbb{R}}$ and the map obtained by the restriction of the projection to the closure $\overline{D}$. Let ${\pi}_{D}$ denote the restriction to  the closure $\overline{D}$.
%Our Poincar\'e-Reeb graph for the pair can be also defined as the quotient space obtained by this equivalence relation. This is isomorphic to the Reeb graph of ${\pi}_{D}$. Furthermore, an isomorphism is defined as the canonically obtained correspondence. 
%\%item The vertex set of our Poincar\'e-Reeb graph for the pair is the union of the set of all singular points of the restrictions of the projection ${\pi}_{k,1}$ or ${\pi}_D$ to all connected components of the boundary $\partial \overline{D} \subset \overline{D}$. This set is also finite. 
%\end{itemize}
%%\end{Def}
%See also \cite{sorea1, sorea2} for related theory for example. We present our main result. In the following section we prove this and present related comments as our main content.
%\begin{MainThm}
%5\label{mthm:1}
%Consider a Poincar\'e-Reeb graph $K$ for the pair in Definition \ref{def:1} such that the closure $\overline{D}$ is compact. Take an arbitrary integer $k_0>k+1$. Then we can construct a real algebraic function on some {\rm (}$k_0-1${\rm )}-dimensional smooth closed manifold regarded as a smooth real algebraic manifold whose Reeb graph is isomorphic to the graph $K$ as a graph.
%\end{MainThm}


%\cite{kitazawa3} presents most of our key tools. We review some important ingredients. We apply them. We also apply them in suitably improved ways in several scenes. 
%At present we do not find essential errors in the (accepted) version on which a positive report for publication has been announced to have been sent. However we will not publish the paper revising the accepted version essentially.

%Note that the work is motivated by \cite{bodinpopescupampusorea} with \cite{sorea1, sorea2}. \cite{bodinpopescupampusorea} studies {\it algebraic domains} collapsing to some graphs. An {\it algebraic domain} means an open set in an real affine space the boundary of whose closure is surrounded by smooth real algebraic hypersurfaces. 
%Originally our main theorem of \cite{kitazawa3} respects these studies.
%The work essentially related to such work is regarded as one of important future problems. However, we do not investigate related problems in our paper. For the related studies, it seems that we need more sophisticated or advanced knowledge and arguments on real algebraic geometry.

Hereafter, we use "$\prod$" for the products of numbers, functions or sets. We use the notation in the forms ${\prod}_{j=1}^l$ for a positive integer $l$ and ${\prod}_{x \in X}$ for a finite set $X$ for example. For coordinates and points for example, we use the notation like $x=(x_1,\cdots,x_l)$ with a positive integer $l$ for example.

\begin{Thm}\label{thm:1}
In Main Theorem \ref{mthm:1}, we can also construct a suitable map $f:M \rightarrow {\mathbb{R}}^n$ on a suitable manifold $M$ in such a way that each preimage $f^{-1}(p)$ is as follows.
\begin{enumerate}
	\item For $p \in D$, it is diffeomorphic to ${\prod}_{j=1}^{l_2} S^{m_{l_2}(j)}$.
	\item For $p \in {\bigcap}_{i=1}^{i_0} S_{j_i} \bigcap \overline{D}$ in the increasing sequence $\{j_i\}_{i=1}^{i_0}$ such that $p \notin {\bigcap}_{i=1}^{i_0+1} S_{{j^{\prime}}_i}$ for any increasing sequence $\{{j^{\prime}}_i\}_{i=1}^{i_0+1}$ containing the sequence $\{j_i\}_{i=1}^{i_0}$ as a subsequence, it is diffeomorphic to ${\prod}_{j \in {\mathbb{N}}_{l_2}-\{m_{l_1,l_2}(j_i)\}_{i=1}^{i_0}} S^{m_{l_2}(j)}$.
\end{enumerate}
\end{Thm}
This is an extension of a main result of the author of \cite{kitazawa3}. 
There hypersurfaces do not intersect.
The result is also presented as a part of Theorem \ref{thm:1} of \cite{kitazawa4}. Some essential steps are same as those of their original proofs. 
\begin{proof}[A proof of Main Theorem \ref{mthm:1} with Theorem \ref{thm:1}]


 
We define $S_0:=\{(x,y_1,\cdots,y_{l_2}) \in U_D \times {\prod}_{i=1}^{l_2} {\mathbb{R}}^{m_{l_2}(i)+1} \subset {\mathbb{R}}^n \times {\prod}_{i=1}^{l_2} {\mathbb{R}}^{m_{l_2}(i)+1}={\mathbb{R}}^{m+l_2} \mid {\prod}_{j \in {m_{l_1,l_2}}^{-1}(i)} (f_{j}(x_1,\cdots,x_n))-{||y_i||}^2=0, i \in {\mathbb{N}}_{l_2}\} \subset {\mathbb{R}}^{m+l_2}$. Here $y_i:=(y_{i,1},\cdots,y_{i,m_{l_2}(i)})$. ${m_{l_1,l_2}}^{-1}(i)$ is not empty for any $i \in {\mathbb{N}}_{l_2}$ since $m_{l_1,l_2}:{\mathbb{N}}_{l_1} \rightarrow {\mathbb{N}}_{l_2}$ is assumed to be surjective. \\
%Here we may replace $||y||^2$ by a general polynomial, represented in the form ${\Sigma}_{j=1}^{n^{\prime}-n} a_j {y_{j}}^{b_j}$ where $a_j$ and $b_j$ are a positive integer and a positive even integer, respectively.  
\ \\
STEP 1 A proof of the fact by implicit function theorem that $S_0$ is a smooth compact submanifold with no boundary. \\
As an important tool, we consider the partial derivatives of the real polynomial function defined by the real polynomial ${\prod}_{j \in {m_{l_1,l_2}}^{-1}(i)}  (f_{j}(x_1,\cdots,x_n))-||y_i||^2={\prod}_{j \in {m_{l_1,l_2}}^{-1}(i)}  (f_{j}(x_1,\cdots,x_n))-{\Sigma}_{j^{\prime}=1}^{m_{l_2}(i)} {y_{i,j^{\prime}}}^2$ for variants $x_j$ and $y_{i,j^{\prime}}$ and for each integer $1 \leq i \leq l_2$. In STEP 1-1 and STEP 1-2, we apply implicit function theorem around each point of $S_0$. After that, in STEP 1-3, we give additional arguments to complete our proof of the fact that $S_0$ is a smooth compact submanifold with no boundary. \\
\ \\
STEP 1-1 The case for a point $(x_0,y_0) \in S_0 \subset U_D \times {\mathbb{R}}^{m-n+l_2}$ such that for $y_0:=(y_{0,1},\cdots,y_{0,l_2})$, $y_{0,j}$ is not the origin for any $1 \leq j \leq l_2$. \\
%By the assumption on the sets $D$ and $U_D$ we have $x_0 \in D$ and
We have $f_{j}(x_0)>0$ for any $1 \leq j \leq l_1$. This is thanks to our first condition in the two listed conditions On the sets $S_j$, $D$ and $U_D$. As a result ${\prod}_{j \in {m_{l_1,l_2}}^{-1}(i)}  f_{j}(x_0)>0$ for each integer $1 \leq i \leq l_2$. 
%We use the notation $x_0:=(x_{0,1},\cdots,x_{0,n})$ and $y_0:=(y_{0,1},\cdots,y_{0,n^{\prime}-n})$ for example as before. 
We have $2y_{i,{j_0}^{\prime}}=2y_{0,i,{j_0}^{\prime}} \neq 0$ as the value of the partial derivative at the point $(x_0,y_0)$ for some variant $y_{i,{j_0}^{\prime}}$ where $y_{0,i,{j_0}^{\prime}}$ is from $y_{0,i}:=(y_{0,i,1},\cdots,y_{0,i,m_{l_2}(i)})$ with $y_0:=(y_{0,1},\cdots,y_{0,l_2})$.
Let us abuse the same notation. We also have $0$ as the value of the partial derivative at the point for any variant $y_{i^{\prime},{j}^{\prime}}$ satisfying $i^{\prime} \neq i$.

The differential of the restriction of the map into ${\mathbb{R}}^{l_2}$ defined canonically from the $l_2>0$ real polynomials ${\prod}_{j \in {m_{l_1,l_2}}^{-1}(i)}  (f_{j}(x))-||y_i||^2$ ($1 \leq i \leq l_2$) at $(x_0,y_0)$ is of rank $l_2$. 
%Remember the inequality ${\prod}_{j \in {m_{l_1,l_2}}^{-1}(i)}  (f_{j}(x_0))>0$.
The point is not a singular point of the map. \\
\ \\
STEP 1-2 The case for a point $(x_{\rm O},y_{\rm O}) \in S_0 \subset U_D \times {\mathbb{R}}^{m-n+l_2}$ such that for $y_{\rm O}:=(y_{{\rm O},1},\cdots,y_{{\rm O},l_2}) \in {\mathbb{R}}^{m-n+l_2}$, $y_{{\rm O},j}$ is the origin for some $1 \leq j \leq l_2$. \\
\ We can also assume that $x_{\rm O} \in {\bigcap}_{i=1}^{i_0} S_{j_i} \bigcap \overline{D}$ holds for some increasing sequence $\{j_i\}_{i=1}^{i_0}$ and that $x_{\rm O} \notin {\bigcap}_{i=1}^{i_0+1} S_{{j^{\prime}}_i}$ holds for any increasing sequence $\{{j^{\prime}}_i\}_{i=1}^{i_0+1}$ containing the sequence $\{j_i\}_{i=1}^{i_0}$ as a subsequence.

%We use the notation like $x_0:=(x_{0,1},\cdots,x_{0,k})$ and $y_0:=(y_{0,1},\cdots,y_{0,k})$ before. 
%By the assumption, for the real polynomials $f_{j}(x)$, $f_{j}(x_{\rm O})> 0$ for $j \neq a$. The real polynomial function defined canonically from the real polynomial $f_{a}$ is assumed to have no singular points on the hypersurface $S_{a}$.

	

Let $i_y \in \mathbb{N}_{l_2}-m_{l_1,l_2}(\{j_i\}_{i=1}^{i_0})$. We consider partial derivatives of the function defined by the real polynomial ${\prod}_{j \in {m_{l_1,l_2}}^{-1}(i_y)} (f_{j}(x_1,\cdots,x_n))-||y_{i_y}||^2={\prod}_{j \in {m_{l_1,l_2}}^{-1}(i_y)} (f_{j}(x_1,\cdots,x_n))-{\Sigma}_{j^{\prime}=1}^{m_{l_2}(i_y)} {y_{i_y,j^{\prime}}}^2$. More precisely, we consider the partial derivative for each variant $y_{i_1,j^{\prime}}:=y_{i_y,j^{\prime}}$ at $(x_{\rm O},y_{\rm O})$.
We have $2y_{i_{1},{j_0}^{\prime}}=2y_{0,i_{1},{j_0}^{\prime}} \neq 0$ for some variant $y_{i_{1},{j_0}^{\prime}}$ at $(x_{\rm O},y_{\rm O})$ where we abuse the notation in STEP 1-1 for example. Let us abuse the same notation. We also have $0$ as the value of the partial derivative at the point for any variant $y_{i_2,{j}^{\prime}}$ satisfying $i_2 \neq i_1$. This is like an argument in STEP 1-1. This counts $l_2-i_0$ of the rank of the differential of the restriction of the map into ${\mathbb{R}}^{l_2}$ defined canonically from
the $l_2>0$ real polynomials. In other words, we choose the $l_2-i_0$ real polynomials from the $l_2$ real polynomials considering such $l_2-i_0$ elements $i_y$. Remember also that the restriction of $m_{l_1,l_2}$ to $\{j_i\}_{i=1}^{i_0}$ is assumed to be injective. 

We consider an integer $i_x \in m_{l_1,l_2}(\{j_i\}_{i=1}^{i_0})$. We discuss partial derivatives of the function defined canonically from the real polynomial ${\prod}_{j \in {m_{l_1,l_2}}^{-1}(i_x)} (f_{j}(x_1,\cdots,x_n))-||y_{i_x}||^2={\prod}_{j \in {m_{l_1,l_2}}^{-1}(i_x)} (f_{j}(x_1,\cdots,x_n))-{\Sigma}_{j^{\prime}=1}^{m_{l_2}(i_x)} {y_{i_x,j^{\prime}}}^2$.
%We explain about partial derivatives of the function defined canonically from the real polynomial ${\prod}_{j \in {m_{l_1,l_2}}^{-1}(i_x)} (f_{j}(x_1,\cdots,x_n))-||y_{i_x}||^2={\prod}_{j \in {m_{l_1,l_2}}^{-1}(i_x)} (f_{j}(x_1,\cdots,x_n))-{\Sigma}_{j^{\prime}=1}^{m_{l_2}(i_x)} {y_{i_x,j^{\prime}}}^2$ for each $i_x \in m_{l_1,l_2}(\{j_i\}_{i=1}^{i_0})$. 
We consider the partial derivatives for the variants $y_{i_1,{j_0}^{\prime}}$ and $y_{i_2,{j}^{\prime}}$, defined just before. We always have $0$ as the values by our definition. 
More precisely, by our definition, at $(x_{\rm O},y_{\rm O})$, $y_{i_x}$ is the origin. We also see that the value of the partial derivative for any variant $y_{i,j}$ is $0$ at the point $(x_{\rm O},y_{\rm O})$.
We also discuss the partial derivative for each variant $x_{i_3}$. For the integer $i_x$, we have the unique number $j_{i_x} \in \{j_i\}_{i=1}^{i_0}$ satisfying $i_x=m_{l_1,l_2}(j_{i_x})$. This follows from the assumption that the restriction of $m_{l_1,l_2}$ to $\{j_i\}_{i=1}^{i_0}$ is injective. 
As the value of the partial derivative,
we also have the value
represented as the product of the following two numbers.
\begin{itemize}
	\item The value of the partial derivative of the function $f_{j_{i_x}}(x_1,\cdots,x_n)$ for the variant $x_{i_3}$ at $x_{\rm O}$. 
	Each given hypersurface $S_j$ is assumed to be non-singular and the zero set of the real polynomial $f_j$. 
	Thus the value of the partial derivative is not $0$ for some variant $x_{i_3}:=x_{i_{3,0}}$ where $1 \leq i_{3,0} \leq n$ is a suitable integer.
	\item The product of the real numbers obtained as the values of the canonically defined real polynomial functions in the family $\{f_{j}(x_1,\cdots,x_n)\}_{j \in {m_{l_1,l_2}}^{-1}(i_x)-\{j_{i_x}\}}$ at $x_{\rm O}$. By our definitions and assumptions with our construction, we can see that this is not $0$.
	\end{itemize}
In addition, remember the assumption on the transversality on intersections for hypersurfaces $S_j$. 
This counts $i_0$ of the rank of the differential of the restriction of the map into ${\mathbb{R}}^{l_2}$ defined canonically from
the $l_2>0$ real polynomials. 
In other words, we choose the $i_0$ real polynomials from the $l_2$ real polynomials considering such elements $i_x$ and $j_{i_x}$.
This also counts "$i_0$" of the rank independently from the "$l_2-i_0$" before. This is thanks to the presented fact that at $(x_{\rm O},y_{\rm O})$, $y_{i_x}$ is the origin and that the values of the partial derivatives for the variants $y_{i_1,{j_0}^{\prime}}$, $y_{i_2,{j}^{\prime}}$ and an arbitrary variant $y_{i,j}$ are always $0$.    

Integrating these arguments, we can see that at $(x_{\rm O},y_{\rm O})$, the differential of the restriction of the map into ${\mathbb{R}}^{l_2}$ defined canonically from
 the $l_2>0$ real polynomials is of rank $l_2>0$.
%satisfying the inequality ${\prod}_{j \in {m_{l_1,l_2}}^{-1}(i)} (f_{j}(x_{\rm O}))>0$ at $(x_{\rm O},y_{\rm O})$ is of rank $l_2$. This is not a singular point of the map.
We have a result like the one in STEP 1-1. \\
\ \\
STEP 1-3 Additional arguments to show that $S_0$ is a smooth compact submanifold with no boundary. \\
By the assumption on $U_D$, which is an arbitrary sufficiently small connected open neighborhood of the closure $\overline{D}$ of a nicely given connected open set $D \subset {\mathbb{R}}^n$, we can see that we cannot take a point $x \in U_D-\overline{D}$ as the component of $(x,y) \in S_0$. We explain about this.
First we choose an arbitrary point $x_{U} \in U_D-\overline{D}$. By the rule of choosing our open neighborhood $U_D$ of the compact and connected set $\overline{D}$,
$U_D$ can be seen as a set as follows.
\begin{itemize}
\item For each point $p$ in $\overline{D}-D \subset {\bigcup}_{j=1}^l S_j$, we can choose a sufficiently small open disk $D_p$ in ${\mathbb{R}}^n$ and we do. Note that $p$ is a point of ${\bigcap}_{i=1}^{i_{p,0}} S_{j_i} \bigcap \overline{D}$ for some increasing sequence $\{j_i\}_{i=1}^{i_{p,0}}$ and that $p \notin {\bigcap}_{i=1}^{i_{p,0}+1} S_{{j^{\prime}}_{i}}$ holds for any increasing sequence $\{{j^{\prime}}_i\}_{i=1}^{i_{p,0}+1}$ containing the sequence $\{j_i\}_{i=1}^{i_{p,0}}$ as a subsequence. Remember our definitions and conditions on the compact set $
\overline{D}$ and the zero sets $S_j$
for example. We can also say that at any point of $D_p$, the value of $f_j$ is positive for any $j \notin \{j_i\}_{i=1}^{i_{p,0}}$.
\item Consider the open set $D \bigcup {\bigcup}_{p \in \overline{D}-D} D_p$ and the open cover $\{D\} \sqcup \{D_p\}_{p \in \overline{D}-D}$ of the compact and connected subset $\overline{D}$. We can choose finitely many open sets to cover this compact and connected set. We choose such sets.
\item $U_D$ can be chosen as the open set defined as the union of the finitely many open sets chosen before. It can be chosen as a connected set by our conditions and construction. Instead of this, we may also choose $U_D$ as an arbitrary open connected subset of the original "$U_D$" being also an open neighborhood of $\overline{D}$.
\end{itemize}

By the condition that the restriction of $m_{l_1,l_2}$ to $\{j_i\}_{j=1}^{i_0}$ is injective, we can choose an arbitrary point $x_{U} \in U_D-\overline{D}$ in such a way that we also have the following.
\begin{itemize}
\item $x_U \in D_{p_0}$ for some sufficiently small open disk $D_{p_0}$ in the family of the chosen finitely many open disks, defined before.
\item We have some integer $1 \leq i_{x_{U,0}} \leq l_2$.
\item We have the two relations $f_{j}(x_{U})>0$ for each $j \in  {m_{l_1,l_2}}^{-1}(i_{x_{U,0}})$ except one $j=j_{x_{U,0}} \in {m_{l_1,l_2}}^{-1}(i_{x_{U,0}})$ and $f_{j_{x_{U,0}}}(x_{U})<0$.
\end{itemize}
 Here we also see that the value ${\prod}_{j \in {m_{l_1,l_2}}^{-1}(i_{x_{U,0}})} f_{j}(x_{U})$ is smaller than $0$. 

This means that we cannot take a point $x \in U_D-\overline{D}$ satisfying $(x,y) \in S_0$.

By the argument and STEPs 1-1 and 1-2 with implicit function theorem, $S_0$ is an $m$-dimensional smooth compact and connected submanifold in $U_D \times {\mathbb{R}}^{m-n+l_2}$ and has no boundary. This is also regarded as a non-singular real algebraic manifold in the real affine space ${\mathbb{R}}^{m+l_2}$. \\
\ \\
STEP 2 Our desired map $f:M \rightarrow {\mathbb{R}}^n$ on $M:=S_0$. \\
We consider the canonical projection ${\pi}_{m+l_2,n}$ and its restriction to $M$. Thus we have a smooth real algebraic map $f:M:=S_0 \rightarrow {\mathbb{R}}^n$. We explain about our desired fact that $f$ is a desired map.
 
 Thanks to the previous arguments, our non-singular real algebraic manifold $M=S_0$ is a connected component of the intersection of finitely many zero sets of real polynomials (in the real affine space).
STEP 1-3 and our definition of $S_0$ also show that the image $f(M)$ is a subset of $U_D$ containing $\overline{D}$ as a subset. Each point of $U_D-\overline{D}$ is outside the image. We can see $f(M)=\overline{D}$. 
On the smooth manifolds of the preimages (in Theorem \ref{thm:1}), we can also easily see our desired fact from our definitions and conditions. 

We can see that $f$ is our desired map. \\
%We investigate $S_0$ and show that this is a smoot
% 
%First we consider a point $p_1 \in D$ and a point $(p_1,q_1) \in \overline{D^{\prime}}$ and take its sufficiently small open neighborhood $U_{p_1,q_1}$ in ${\mathbb{R}}^{n^{\prime}}$. 
%By the definition and the assumption, 
%$U_{p_1,q_1} \bigcap \{(x_1,\cdots,x_n,\cdots,x_{n^{\prime}}) \mid f^{\prime}(x_1,\cdots,x_n,\cdots,x_{n^{\prime}})>0\}=U_{p_1,q_1} \bigcap D^{\prime}$ and $U_{p_1,q_1} \bigcap \{(x_1,\cdots,x_n,\cdots,x_{n^{\prime}}) \mid f^{\prime}(x_1,\cdots,x_n,\cdots,x_{n^{\prime}}) \geq 0\}=U_{p_1,q_1} \bigcap \overline{D^{\prime}}$ hold.

%Second we consider a point $p_2 \in \partial \overline{D}$ in the boundary $\partial \overline{D} \subset \overline{D}$ and a point $(p_2,q_2) \in \overline{D^{\prime}}$. We take
 %its sufficiently small open neighborhood $U_{p_2,q_2}$ in ${\mathbb{R}}^{n^{\prime}}$. By the definition and the assumption on the hypersurfaces $S_j$ and the real polynomials, we have a similar observation.

%This completes the proof of (\ref{thm:1.2}).

%By observing the structures of the maps and the manifolds, we have (\ref{thm:1.4}).

%By the construction, the canonical projection onto $\overline{D}$ gives a desired smooth real algebraic map.
This completes the proof.
	
\end{proof}

We can know the following easily by the construction and local structures of the functions and the maps. 

\begin{Thm}
\label{thm:2}
In Main Theorem \ref{mthm:1} and Theorem \ref{thm:1}, consider a point $p \in {\bigcap}_{i=1}^{i_0} S_{j_i}$ such that for any increasing sequence $\{{j^{\prime}}_i\}_{i=1}^{i_0+1}$ containing the sequence $\{j_i\}_{i=1}^{i_0}$, $p \notin {\bigcap}_{i=1}^{i_0+1} S_{{j^{\prime}}_i}$ holds. The image of the differential of the map $f$ at each point of $f^{-1}(p)$ and the tangent vector space of ${\bigcap}_{i=1}^{i_0} S_{j_i}$ at $p$ agree. 
\end{Thm}

More precisely, the obtained map is, {\it at least topologically}, a so-called {\it special generic} map or a smooth map locally regarded as a so-called {\it moment} map. In other words, there exists such a nice map $f_0:M \rightarrow {\mathbb{R}}^n$ and a homeomorphism $\Phi:M \rightarrow M$ enjoying the relation $f_0 \circ \Phi=f$. Here we adopt methods for some presentations in \cite{kitazawa4}.

{\it Special generic} maps are, in short, higher dimensional versions and generalizations of canonical projections of unit spheres and Morse functions on spheres with exactly two singular points.
For special generic maps, see \cite{saeki1} and see also recent preprints \cite{kitazawa6, kitazawa8} of the author. Of course, arguments and results in \cite{kitazawa6, kitazawa8} and \cite{kitazawa7}, presented later, are independent of our study and we do not need to understand them precisely. We only give some short remark related to them.

First, see \cite{buchstaberpanov, delzant} for moment maps on so-called {\it symplectic toric} manifolds. \cite{kitazawa7} is a preprint introducing a certain class of smooth maps generalizing the class of special generic maps first as the class of {\it simply generalized special generic} maps. It also investigates their topological properties, especially the cohomology rings of the manifolds. This respects and extends the results of \cite{kitazawa7}. These maps are locally moment maps. We can have maps at least topologically (simply generalized) special generic maps in cases where hypersurfaces $S_j$ do not intersect. The main result of \cite{kitazawa3} is for an explicit and simplest case.

For related singularity theory of differentiable maps, see \cite{golubitskyguillemin} for example. As presented in the assumption of Main Theorem \ref{mthm:1}, the notion "transversality" is also from this theory, for example.

For here, see also \cite{kohnpieneranestadrydellshapirosinnsoreatelen}. This is on explicit classifications of regions surrounded by non-singular real algebraic hypersurfaces intersecting satisfying the condition on the "transversality" like our case. Related to this, \cite{bodinpopescupampusorea} with \cite{sorea1, sorea2} are mainly on regions surrounded by non-singular real algebraic hypersurfaces with no intersections. This appears explicitly in Example 1 and FIGURE 1 of \cite{kitazawa3}, which are also presented in the
next section.


\section{On Main Theorem \ref{mthm:2}.}
\subsection{Graphs and Reeb graphs.} 
({\it Reeb}) {\it graphs} are our fundamental tools. 

A {\it graph} can be defined as a $1$-dimensional CW complex with the {\it vertex set} and the {\it edge set}. They are the set of all $0$-dimensional cells and the set of all $1$-dimensional cells, respectively. A {\it vertex} is an element of the vertex set. An {\it edge} is an element of the edge set. 
The closure of an edge homeomorphic to a circle is a {\it loop}. We do not consider graphs having loops. In other words, a graph is always regarded as a $1$-dimensional polyhedron. Furthermore, it is {\it finite}. In a word, its vertex set and edge set are finite. On the other hand, a graph may be a so-called multi-graph. This means that a graph has more than one edge connecting some distinct vertices. An {\it isomorphism} from a graph $K_1$ onto another graph $K_2$ means a PL or a piecewise smooth homeomorphism mapping the edge set and the vertex set of $K_1$ onto those of $K_2$. This defines a natural equivalence relation on the family of all graphs here. Two graphs $K_1$ and $K_2$ are {\it isomorphic} if an isomorphism from $K_1 $ onto $K_2$ exists. 

For a smooth function $c:X \rightarrow \mathbb{R}$, we can define an equivalence relation by the rule that two points $x_1$ and $x_2$ in $X$ are equivalent if and only if they are in some same connected component of some preimage $c^{-1}(y)$. Let $W_c$ denote the quotient space. Let $q_c:X \rightarrow W_c$ denote the quotient map.

\begin{Def}
	If $W_c$ has the structure of a graph where a point $p$ is a vertex if and only if ${q_c}^{-1}(p)$ has some singular points of $c$, then $W_c$ is the {\it Reeb graph} of $c$.
\end{Def}
We explain about a main result of \cite{saeki2}.
For a smooth function on a compact manifold having finitely many singular values, the quotient space $W_c$ has been shown to be homeomorphic to a graph. If the manifold of the domain is closed, then we can define the Reeb graph $W_c$ of $c$. {\rm Morse}({\rm -Bott}) functions and functions of some considerably wide classes satisfy such conditions.

The Reeb graph of a smooth function has been already defined in \cite{reeb} for example. The Reeb graphs of nice smooth functions have been important tools and objects in the theory on singularities and applications to geometry. These graphs have some important information on the manifolds compactly.

\subsection{Main Theorem \ref{mthm:2} and its proof.}


\begin{MainThm}
	\label{mthm:2}
	Let $m$ be a sufficiently large integer. Consider a graph $G$ with exactly one edge and two vertices or a graph in the following.
	\begin{itemize}
		\item The vertex set consists of exactly $4$ vertices. Two of them are denoted by $v_{\rm l}$ and $v_{\rm r}$. The remaining two are denoted by $v_{1}$ and $v_{2}$.
		\item The edge connecting $v_{\rm l}$ and $v_{\rm 1}$ is uniquely defined. The edge connecting $v_{\rm r}$ and $v_{\rm 2}$ is uniquely defined.
		\item The number of edges connecting $v_1$ and $v_2$ is $l>1$. 
	\end{itemize}
	
	 We have a family of countably many smooth real algebraic functions on some $m$-dimensional non-singular real algebraic closed and connected manifolds such that the Reeb graphs collapse to a graph isomorphic to $G$ and that the Reeb graphs of distinct functions are mutually not isomorphic. 	\end{MainThm}
\begin{proof}[A proof of Main Theorem \ref{mthm:2}]
%We can take a large circle in ${\mathbb{R}}^2$ or we consider
%the situation of Example 1 and FIGURE 1 in \cite{kitazawa3}, and take a sufficiently large outermost circle. 

We discuss both cases.
We choose a circle centered at the origin $(0,0) \in {\mathbb{R}}^2$ and of a sufficiently large radius $R>0$.
We can choose $l-1$ circles the $j$-th circle in which is centered at some point $(0,t_j)$ ($t_j \in \mathbb{R}$) and whose radius is $1$ in such a way that the circles are the boundaries of disks in ${\mathbb{R}}^2$ of radii $1$ and mutually disjoint. We choose such $l-1$ circles in the latter case.

For any positive integer $k>0$, we can choose $k$ disks of suitably chosen radii enjoying the following properties.
\begin{itemize}
\item These disks are located sufficiently close to the point $(-R,0)$.
\item These disks are centered at points in the given sufficiently large circle centered at the origin $(0,0) \in {\mathbb{R}}^2$ and of radius $R>0$. In other words, these circles are centered at points the distances between which and the origin are $R$ of course.
\item These disks are mutually disjoint.
\item These disks are centered at points $(x_1,x_2)$ satisfying $x_1<0$ and $x_2 \geq 0$. One of the disks is centered at $(-R,0)$. They are also sufficiently small enough to argue as in the following.
\end{itemize}	
We explain about our desired family of functions and their Reeb graphs. We can construct maps into ${\mathbb{R}}^2$ by applying Main Theorem \ref{mthm:1} respecting the situations. After that we compose the canonical projection ${\pi}_{2,1}$ to the first component. 

We explain about singular points of the functions and their Reeb graphs shortly. We abuse the notation of Main Theorem \ref{thm:1} and according to Theorem \ref{thm:2}, singular points of the functions exist in the preimage $f^{-1}(p)$ ($p \in {\mathbb{R}}^2$) in Main Theorem \ref{mthm:1} if and only if $p$ is contained in two distinct circles $S_j$ or regarded as a "pole" of some circle $S_j$.
We discuss such points $p$ such that the preimages $f^{-1}(p)$ contain some singular points of the resulting functions. 
We can see that the number of such points is $3k+1$ in the former case. We can see that the number is $3k+2(l-1)+1$ in the latter case. The number of the vertices of the Reeb graph of the resulting function is $3k+1$ in the former case.
The number is $3k+3$ in the latter case.

Related to this, we explain about the presented property
in the listed four properties just before that each disk is sufficiently small in the latter case.

We consider two distinct disks and the circles of their boundaries here. The three points whose preimages (for $f$)
contain singular points of the resulting function in one circle are all in the left or the right of the three points whose preimages (for $f$) contain singular points of the resulting function in the remaining
circle. Points whose preimages (for $f$)
contain singular points of the resulting function in the $l-1$ circles chosen first and the point $(R,0)$, whose preimage contains singular points of the resulting function, are all in the right of the points before. This is also thanks to the listed properties.

See also FIGURES \ref{fig:1} and \ref{fig:2}, presented here.
\begin{figure}
	
	\includegraphics[height=75mm, width=100mm]{20230731forMT2.1.eps}

	\caption{The image of the map $f:M \rightarrow {\mathbb{R}}^2$ constructed by using Main Theorem \ref{mthm:1} and the Reeb graph of the resulting function in Main Theorem \ref{mthm:2} (in the latter case).}
	\label{fig:1}
\end{figure}

\begin{figure}
	
	\includegraphics[height=75mm, width=100mm]{20230801forMT2.2.eps}

	\caption{Around the arrow in FIGURE \ref{fig:1} and local structure of the Reeb graph.}
	\label{fig:2}
\end{figure}
This completes our proof.
\end{proof}

We give some remarks.

For example, (the right figure in) FIGURE 1 of \cite{kitazawa3} shows a graph $G$ in Main Theorem \ref{mthm:2}.

According to (our proof of) Main Theorem \ref{mthm:1}, we can apply Main Theorem \ref{mthm:1} in such a way that the following facts hold. The preimage $f^{-1}(p)$ is diffeomorphic to the product of two (unit) spheres for $p \in D$ where we abuse the notation from Main Theorem \ref{mthm:1}. $f^{-1}(p)$ is diffeomorphic to a unit sphere for $p \in S_j$ contained in exactly one sphere in the family $\{S_j\}$ where we abuse the notation. $f^{-1}(p)$ is a one-point set for $p \in S_{j_1} \bigcap S_{j_2}$ where the notation is abused with the condition $j_1 \neq j_2$. 


It seems to be not so difficult to generalize Main Theorem \ref{thm:2} to cases for graphs of more general classes. 
%For a smooth function $c$, we can also define the quotient map $q_c:x \rightarrow W_c$ and another continuous map $\bar{c}:W_c \rightarrow \mathbb{R}$ enjoying the relation $c=\bar{c} \circ q_c$ uniquely.
%The {\it degree} of a vertex of a graph means the number of edges containing it.

%FIGURE 1 of \cite{kitazawa3} shows two explicit cases for the paper.
	
%We discuss the lower figure. Let $l>1$ be an integer. The lower figure shows a graph with exactly $2$ vertices of degree $1$, exactly $2$ vertices of degree $l+1$ and exactly $l+2$ edges. Furthermore, we can explain about the graph as follows.

%\begin{itemize}
%\item The first two vertices are denoted by $v_{\rm l}$ and $v_{\rm r}$, respectively. 
%5%\item The other two vertices are denoted by $v_{\rm 1}$ and $v_{\rm 2}$, respectively. 
%\item Two of the edges are denoted by $e_{\rm l}$ and $e_{\rm r}$, respectively. $e_{\rm l}$ connects $v_{\rm l}$ and $v_1$. $e_{\rm r}$ connects $v_{\rm r}$ and $v_2$.
%\item The remaining $l$ edges are denoted by $e_{j}$ where $1 \leq j \leq l$ is an integer. They connect $v_1$ and $v_2$.
%\end{itemize}

%	More precisely, these two graphs show two simplest examples of the so-called {\it Poincar\'e-Reeb graphs}. A {\it Poincar\'e-Reeb} graph is defined for a pair of an algebraic domain in a real affine space and a canonical projection of the real affine space to the $1$-dimensional real affine space. This is defined as a graph to which the algebraic domain naturally collapses.  
\section{Closing comments on our study.}
We close our paper by giving Remarks. %They are all on Main Theorem \ref{mthm:2} mainly.
\begin{Rem}
Main Theorem \ref{mthm:2} is regarded as an additional result to main results of \cite{kitazawa3, kitazawa5}. Moreover, these studies are originally motivated by Sharko's question in \cite{sharko}. It asks whether we can have an explicit nice smooth function whose Reeb graph is isomorphic to a given graph on some closed surface or more generally, some smooth closed manifold. More precisely, these studies are also motivated by a revised question by the author. 
It asks whether we can respect singularities of the functions and preimages in addition. For related studies, see \cite{kitazawa1, kitazawa2} as pioneering studies by the author for example. \cite{saeki2} is also a related study based on our informal discussions on \cite{kitazawa1}. For the case of Morse functions such that connected components of preimages with no singular points are spheres, \cite{michalak} is a pioneering work. \cite{michalak2} is also a related work studying fundamental deformations of Reeb graphs of Morse functions on fixed manifolds. These studies of Michalak have motivated the author. 

It is also remarkable that we construct smooth functions which are not real algebraic or real analytic functions in several scenes in the study. Important functions in \cite{kitazawa2, saeki2} show explicit cases.
\end{Rem}
\begin{Rem}
	As presented in Remark 1 of \cite{kitazawa5}, in several cases such as so-called "generic" cases, we can have a result similar to one of Main Theorem \ref{mthm:2} by considering "approximations". Approximations are presented shortly in our first section or our introduction.
For example, we can apply such arguments in the case where the Reeb graphs of Morse functions are generic. In other words, we consider Morse functions such that at distinct singular points, the values are always distinct. For example, a graph with exactly one edge and two vertices and a graph in the left figure in FIGURE 1 of \cite{kitazawa3} are regarded as generic. In Main Theorem \ref{mthm:2}, the graph in the latter case is seen as generic if and only if $l=2$.

	On the other hand, we have our results and explicit functions by avoiding such approximations. 
	We can also know real polynomials explicitly for our desired manifolds and zero sets. We can also give such a comment for Main Theorem \ref{mthm:1}. 
	Our previous studies \cite{kitazawa3, kitazawa5} also do this.
	\end{Rem}


\begin{Rem}
We may obtain revised versions of our main results under weaker conditions, for example. However, we respect simple or explicit cases. For example, we respect very explicit cases with the hypersurfaces $S_j$ being spheres of fixed radii in Main Theorem \ref{mthm:1} and show Main Theorem \ref{mthm:2}. Moreover, as presented before, Main Theorem \ref{thm:2} may be extended to cases where graphs are more general.
\end{Rem}
  
\begin{thebibliography}{25}
	\bibitem{buchstaberpanov} V. M. Buchstaber and T. E. Panov, \textsl{Toric topology}, Mathematical Surveys and Monographs, Vol. 204, American Mathematical Society, Providence, RI, 2015.
%	\bibitem{burletderham} O. Burlet and G. de Rham, \textsl{Sur certaines applications g\'en\'eriques d'une vari\'et\'e close a $3$ dimensions dans le plan}, Enseign. Math. 20 (1974). 275--292.
	%	\bibitem{calabi} E. Calabi, Quasi-surjective mappings and a generation of Morse theory, Proc. U.S.-Japan Seminar in Differential Geometry, Kyoto, 1965, pp. 13--16.
	%
	%		\bibitem{cavicchioli} A. Cavicchioli, \textsl{Covering numbers of manifolds and critical points of a Morse function}, Israel. J. Math. 70 (1990), 279--304.
	% \bibitem{cerf} J. Cerf, \textsl{La stratification naturelle des espaces de fonctions deff\'erentiables r'eelles et le th'eor`eme de la pseudo-isotopie}, Inst. Hautes Etudes Sci. Publ. Math. 39 (1970), 5--173.
	%		\bibitem{choimasudasuh} S. Choi, M. Masuda and D. Y. Suh, \textsl{Topological classification of generalized Bott towers}, Trans. Amer. Math. Soc. 362 (2010), 1097--1112.
	%		\bibitem{cornealuptonopreatanre} O. Cornea, G. Lupton, J. Oprea and D. Tanr\'e, \textsl{Lusternik-Schnirelmann category}, Mathematical Surveys and Monographs, 103, Amer. Math. Soc., Providence, RI, 2003.
	%\bibitem{crowleyescher} D. Crowley and C. Escher, \textsl{A classification of $S^3$-bundles over $S^4$}, Differential. Geom. Appl. 18 (2003), 363--380, arXiv:0004147.
	%\bibitem{crowleynordstrom} D. Crowley and J. Nordstr\"{o}m, \textsl{The classification of $2$-connected $7$-manifolds}, Proc. London. Math. Soc. 119 (2019), 1--54, arXiv:1406.2226.

	\bibitem{bochnakcosteroy} J. Bochnak, M. Coste and M.-F. Roy, \textsl{Real algebraic geometry}, Ergebnisse der Mathematik und ihrer Grenzgebiete (3) [Results in Mathematics and Related Areas (3)], vol. 36, Springer-Verlag, Berlin, 1998. Translated from the 1987 French original; Revised by the authors.
		\bibitem{bochnakkucharz} J. Bochnak and W. Kucharz, \textsl{Algebraic approximation of mappings into spheres}, Michigan Mathematical Journal, vol. 34, no. 1, 1987.
	\bibitem{bodinpopescupampusorea} A. Bodin, P. Popescu-Pampu and M. S. Sorea, \textsl{Poincar\'e-Reeb graphs of real algebraic domains}, Revista Matem\'atica Complutense, https://link.springer.com/article/10.1007/s13163-023-00469-y, arXiv:2207.06871v2.
\bibitem{bott} R. Bott, \textsl{Nondegenerate critical manifolds}, Ann. of Math. 60 (1954), 248--261.
%\bibitem{costantino}  F. Costantino, \textsl{A short introduction to shadows of $4$-manifolds}, Fundamenta Mathematicae 251 no. 2 (2005), 427--442.
%\bibitem{costantinothurston} F. Costantino, D. Thurston, \textsl{$3$-manifolds efficiently bound $4$-manifolds}, J. Topol. 1 (2008),
%703--745.
	\bibitem{delzant} T. Delzant, \textsl{Hamiltoniens p\'eriodiques et images convexes de l'application moment}, Bull. Soc. Math. France 116 (1988), No. 3, 315--339.
%\bibitem{ehresmann} C. Ehresmann, \textsl{Les connexions infinitesimales dans un espace fibre differentiable}, Colloque de Topologie, Bruxelles (1950), 29--55.
%\bibitem{fujitakitabeppumitsuishi} H. Fujita, Y Kitabeppu and A. Mitsuishi, \textsl{Distance functions and convex bodies and symplectic toric manifolds}, arXiv:2003.02293.
%\bibitem{gelbukh} I. Gelbukh, \textsl{Loops in Reeb graphs of $n$-manifolds}, diskrete \& Computational Geometry, 59 (4) (2018), 843--863. 
%%\bibitem{gelbukh2} I. Gelbukh, \textsl{Approximation of Metric Spaces by Reeb Graphs: Cycle Rank of a Reeb Graph, the Co-rank of the Fundamental Group, and Large Components of Level Sets on Riemannian Manifolds}, Filomat (in press), arxiv:1903.00777.
%\bibitem{gelbukh} I. Gelbukh, \textsl{A finite graph is homeomorphic to the Reeb graph of a Morse-Bott function}, Mathematica Slovaca, 71 (3), 757--772, 2021; doi: 10.1515/ms-2021-0018. 
%%\bibitem{gelbukh2} I. Gelbukh, \textsl{Morse-Bott functions with two critical values on a surface}, Czechoslovak Mathematical Journal, 71 (3), 865--880, 2021; doi: 10.21136/CMJ.2021.0125-20. 
\bibitem{golubitskyguillemin} M. Golubitsky and V. Guillemin, \textsl{Stable Mappings and Their Singularities}, Graduate Texts in Mathematics (14), Springer-Verlag(1974).
%\bibitem{hempel} J. Hempel, \textsl{3- Manifolds}, AMS Chelsea Publishing, 2004. 
%\bibitem{hiratukasaeki} J. T. Hiratuka and O. Saeki, \textsl{Triangulating Stein factorizations of generic maps and Euler Characteristic formulas}, RIMS Kokyuroku Bessatsu B38 (2013), 61--89. 
%\bibitem{hiratukasaeki2} J. T. Hiratuka and O. Saeki, \textsl{Connected components of regular fibers of differentiable maps}, in "Topics on Real and Complex Singularities", Proceedings of the 4th Japanese-Australian Workshop (JARCS4), Kobe 2011,  World Scientific, 2014, 61--73. 
%\bibitem{ishikawakoda} M. Ishikawa and Y. Koda, \textsl{Stable maps and branched shadows of $3$-manifolds}, Mathematische Annalen 367 (2017), no. 3, 1819--1863, arXiv:1403.0596.
%\bibitem{kitazawa1} N. Kitazawa, \textsl{On round fold maps} (in Japanese), RIMS Kokyuroku Bessatsu B38 (2013), 45--59.
%\bibitem{kitazawa2} N. Kitazawa, \textsl{On manifolds admitting fold maps with singular value sets of concentric spheres}, Doctoral Dissertation, Tokyo Institute of Technology (2014).
%\bibitem{kitazawa3} N. Kitazawa, \textsl{Fold maps with singular value sets of concentric spheres}, Hokkaido Mathematical Journal Vol.43, No.3 (2014), 327--359.
\bibitem{kitazawa1} N. Kitazawa, \textsl{On Reeb graphs induced from smooth functions on $3$-dimensional closed orientable manifolds with finitely many singular values}, Topol. Methods in Nonlinear Anal. Vol. 59 No. 2B, 897--912, arXiv:1902.08841.
\bibitem{kitazawa2} N. Kitazawa, \textsl{On Reeb graphs induced from smooth functions on closed or open surfaces}, Methods of Functional Analysis and Topology Vol. 28 No. 2 (2022), 127--143, arXiv:1908.04340.
\bibitem{kitazawa3} N. Kitazawa, \textsl{Real algebraic functions on closed manifolds whose Reeb graphs are given graphs}, a positive report for publication has been announced to have been sent and this will be published in Methods of Functional Analysis and Topology, arXiv:2302.02339v3.
\bibitem{kitazawa4} N. Kitazawa, \textsl{Explicit construction of explicit real algebraic functions and real algebraic manifolds via Reeb graphs}, this is the abstract of our talk in an international conference "Algebraic and geometric methods of analysis 2023" (https://www.imath.kiev.ua/$\sim$topology/conf/agma2023/) and after a review process accepted for publication in the book of abstracts, https://imath.kiev.ua/$\sim$topology/conf/agma2023/contents/abstracts/texts/kitazawa/kitazawa.pdf.
\bibitem{kitazawa5} N. Kitazawa, \textsl{Construction of real algebraic functions with prescribed preimages}, submitted to a refereed journal, arXiv:2303.00953.
%\bibitem{kitazawa6} N. Kitazawa, \textsl{Round fold maps and the topologies and the differentiable structures of manifolds admitting explicit ones}, submitted to a refereed journal, arXiv:1304.0618.
%\bibitem{kitazawa0.5} N. Kitazawa, \textsl{Constructing fold maps by surgery operations and homological information of their Reeb spaces}, submitted to a refereed journal, arxiv:1508.05630.
%\bibitem{kitazawa0.6} N. Kitazawa, \textsl{Notes on fold maps obtained by surgery operations and algebraic information of their Reeb spaces}, arxiv:1811.04080.

\bibitem{kitazawa6} N. Kitazawa, \textsl{Notes on explicit special generic maps into Euclidean spaces whose dimensions are greater than $4$}, a revised version is submitted based on positive comments (major revision) by referees and editors after the first submission to a refereed journal, arxiv:2010.10078.
%\bibitem{kitazawa6} N. Kitazawa, \textsl{On Reeb graphs induced from smooth functions on $3$-dimensional closed manifolds which may not be orientable}, a revised version is submitted to a refereed journal after based on positive comments by editors and referees after the second submission to a refreed journal, arXiv:2108.01300.
%\bibitem{kitazawa7} N. Kitazawa, \textsl{Realization problems of graphs as Reeb graphs of Morse functions with prescribed preimages}, submitted to a refereed journal, arXiv:2108.06913.
%\bibitem{kitazawa10} N. Kitazawa,\textsl{Round fold maps on $3$-dimensional manifolds and their integral and rational cohomology rings}, arXiv:2301.07008.
\bibitem{kitazawa7} N. Kitazawa, \textsl{A class of naturally generalized special generic maps}, submitted to a refereed journal, arXiv:2212.03174.
\bibitem{kitazawa8} N. Kitazawa, \textsl{A note on cohomological structures of special generic maps}, a revised version is submitted based on positive comments by referees and editors after the third submission to a refereed journal.
%\bibitem{kitazawasaeki1} N. Kitazawa and O. Saeki, \textsl{Round fold maps on $3$-manifolds}, accepted for publication after a refereeing process and to appear in Algebraic \& Geometric Topology, arXiv:2105.00974.
		%	\bibitem{kitazawasaeki2} N. Kitazawa and O. Saeki, \textsl{Round fold maps of $n$-dimensional manifolds into ${\mathbb{R}}^{n-1}$}, submitted to a refereed journal, arXiv:2111.13510.
%\bibitem{ishikawakoda} M. Ishikawa, Y. Koda, \textsl{Stable maps and branched shadows of $3$-manifolds}, arXiv:1403.0596.
%\bibitem{kobayashisaeki} M. Kobayashi and O. Saeki, \textsl{Simplifying stable mappings into the plane from a global viewpoint}, Trans. Amer. Math. Soc. 348 (1996), 2607--2636.
\bibitem{kohnpieneranestadrydellshapirosinnsoreatelen} K. Kohn, R. Piene, K. Ranestad, F. Rydell, B. Shapiro, R. Sinn, M-S. Sorea and S. Telen, \textsl{Adjoints and Canonical Forms of Polypols}, arXiv:2108.11747.
\bibitem{kollar} J. Koll\'ar, \textsl{Nash's work in algebraic geometry}, Bulletin (New Series) of the American Matematical Society (2) 54, 2017, 307--324.
\bibitem{kucharz} W. Kucharz, \textsl{Some open questions in real algebraic geometry}, Proyecciones Journal of Mathematics, Vol. 41 No. 2 (2022), Universidad Cat\'olica del Norte Antofagasta, Chile, 437--448.
%\bibitem{martinezalfaromezasarmientooliveira} J. Martinez-Alfaro, I. S. Meza-Sarmiento and R. Oliveira, \textsl{Topological  classification of simple Morse Bott functions on surfaces}, Contemp. Math. 675 (2016), 165--179.%
%\bibitem{masumotosaeki} Y. Masumoto and O. Saeki, \textsl{A smooth function on a manifold with given Reeb graph}, Kyushu J. Math. 65 (2011), 75--84.
\bibitem{maciasvirgospereirasaez} E. Mac\'ias-Virg\'os and M. J. Pereira-S\'aez, Height functions on compact symmetric spaces, Monatshefte f\"ur Mathematik 177 (2015), 119--140. 
\bibitem{michalak} L. P. Michalak, \textsl{Realization of a graph as the Reeb graph of a Morse function on a manifold}. Topol. Methods in Nonlinear Anal. 52 (2) (2018), 749--762, arXiv:1805.06727.
\bibitem{michalak2} L. P. Michalak, \textsl{Combinatorial modifications of Reeb graphs and the realization problem}, arxiv:1811.08031.
%\bibitem{milnor} J. Milnor, \textsl{Singular points of complex hypersurfacs}, Annals of Mathematics Studies, No. 61, Princeton University Press, Princeton, N. J.; University of Tokyo Press, Tokyo, 1968.
%\bibitem{milnor} J. Milnor, \textsl{Lectures on the h-cobordism theorem}, Math. Notes, Princeton Univ. Press, Princeton, N.J. 1965.
\bibitem{moise} E. E. Moise, \textsl{Affine Structures in $3$-Manifold{\rm :} V. The Triangulation Theorem and Hauptvermutung}, Ann. of Math., Second Series, Vol. 56, No. 1 (1952), 96--114.
%\bibitem{morin} B. Morin, \textsl{Formes canoniques des singulariti\'{e}s d\'{}une application diff\'{e}rentiable}, C. E. Acad. Sci. Paris 260 (1965), 5662--5665, 6503--6506.
\bibitem{nash} J. Nash, \textsl{Real algbraic manifolds}, Ann. of Math. (2) 56 (1952), 405--421.
%\bibitem{ranicki} A. Ranicki, \textsl{Algebraic and geometric surgery}, https://www.maths.ed.ac.uk/~v1ranick/books/surgery.pdf, 2002.
\bibitem{ramanujam} S. Ramanujam, \textsl{Morse theory of certain symmetric spaces}, J. Diff. Geom. 3 (1969), 213--229.
\bibitem{reeb} G. Reeb, \textsl{Sur les points singuliers d\'{}une forme de Pfaff compl\'{e}tement int\`{e}grable ou d\'{}une fonction num\'{e}rique}, Comptes Rendus
 Hebdomadaires des S\'{e}ances de I\'{}Acad\'{e}mie des Sciences 222 (1946), 847--849.
%\bibitem{saeki1} O. Saeki, \textsl{Notes on the topology of folds}, J. Math. Soc. Japan Volume 44, Number 3 (1992), 551--566.
\bibitem{saeki1} O. Saeki, \textsl{Topology of special generic maps of manifolds into Euclidean spaces}, Topology Appl. 49 (1993), 265--293.
%\bibitem{saeki0.2} O. Saeki, \textsl{Topology of singular fibers of differentiable maps}, Lecture Notes in Math., Vol. 1854, Springer-Verlag, 2004. 
%\bibitem{saeki4} O. Saeki, \textsl{Morse functions with sphere fibers}, Hiroshima Math. J. Volume 36, Number 1 (2006),  141--170.
\bibitem{saeki2} O. Saeki, \textsl{Reeb spaces of smooth functions on manifolds}, International Mathematics Research Notices, maa301, Volume 2022, Issue 11, June 2022, https://doi.org/10.1093/imrn/maa301, arXiv:2006.01689.
%\bibitem{saekitakase} O. Saeki and M. Takase, \textsl{Desingularizing special generic maps}, Journal of G\"okova Geometry Topology (2013), 1--24.
%\bibitem{sakurai} S. Sakurai, Master Thesis, Kyushu. Univ..
% \bibitem{saekitakase} O. Saeki and M. Takase, \textsl{Desingularizing special generic maps}, Journal of Gokova Geometry Topology 7 (2013), 1--24.
%\bibitem{saeki2} O. Saeki, \textsl{Topology of special generic maps of manifolds into Euclidean spaces}, Topology Appl. 49 (1993), 265--293.
%\bibitem{saeki4} O. Saeki, \textsl{Singular fibers and $4$-dimensional cobordism group}, Pacific J. Math. 248 (2010), 233--256.
%\bibitem{saekisakuma} O. Saeki and K. Sakuma, \textsl{On special generic maps into ${\mathbb{R}}^3$}, Pacific J. Math. 184 (1998), 175--193.
%\bibitem{saekisuzuoka} O. Saeki and K. Suzuoka, \textsl{Generic smooth maps with sphere fibers} J. Math. Soc. Japan Volume 57, Number 3 (2005), 881--902.
\bibitem{sharko} V. Sharko, \textsl{About Kronrod-Reeb graph of a function on a manifold}, Methods of Functional Analysis and
 Topology 12 (2006), 389--396.
%\bibitem{shiota} M. Shiota, \textsl{Thom's conjecture on triangulations of maps}, Topology 39 (2000), 383--399.
\bibitem{sorea1} M. S. Sorea, \textsl{The shapes of level curves of real polynomials near strict local maxima},  Ph. D. Thesis, Universit\'e de Lille, Laboratoire Paul Painlev\'e, 2018.
\bibitem{sorea2} M. S. Sorea, \textsl{Measuring the local non-convexity of real algebraic curves}, Journal of Symbolic Computation 109 (2022), 482--509.
%\bibitem{sorea1} M. S. Sorea, \textsl{The shapes of level curves of real polynomials near strict local maxima},  Ph. D. Thesis, Universit\'e de %Lille, Laboratoire Paul Painlev\'e, 2018.
%\bibitem{sorea2} M. S. Sorea, \textsl{Measuring the local non-convexity of real algebraic curves}, J. Symbolic Compute. 109 (2022), 482--509.
%\bibitem{stong} R. E. Stong, \textsl{Notes on cobordsm theory}, Princeton Universty Press, 1968.
\bibitem{takeuchi} M. Takeuchi, \textsl{Nice functions on symmetric spaces}, Osaka. J. Mat. (2) Vol. 6 (1969), 283--289.
%\bibitem{thom} R. Thom, \textsl{Les singularites des applications differentiables}, Ann. Inst. Fourier (Grenoble) 6 (1955-56), 43--87.
\bibitem{tognoli} A. Tognoli, \textsl{Su una congettura di Nash}, Ann. Scuola Norm. Sup. Pisa (3) 27 (1973), 167--185.
%\bibitem{turaev} Vladimir G. Turaev, \textsl{Topology of shadows}, Preprint, 1991.
%\bibitem{wall} C. T. C Wall, \textsl{Classification problems in differential topology -- {\rm I:} Classificationon handlebodies}, Topology 2 (1963), 253--261.
%\bibitem{wall2} C. T. C. Wall \textsl{Classification problems in differential topology -- {\rm II:} Diffeomorphismsof handlebodies}, Topology 2 (1963), 263--272.
%\bibitem{wall3} C. T. C. Wall, \textsl{Classification problems in differential topology -- {\rm Q:} Quadratic forms on finite groups and related topics}, Topology 2 (1963), 281--298.
%\bibitem{wall4} C. T. C. Wall, \textsl{Classification problems in differential topology -- {\rm III:} Applications to special cases}, Topology 3 (1965), 291--304.
%%\bibitem{wall5} C. T. C. Wall, \textsl{Classification problems in differential topology -- {\rm IV:} Thickenings}, Topology 5 (1966), 73--94.
%\bibitem{wall6} C. T. C. Wall, \textsl{Classification problems in differential topology -- {\rm VI:} Classification of |{\rm (}$s-1${\rm )}-connected {\rm (}$2s+1${\rm )}-manifolds}, Topology 6 (3) (1967), 273--296.
%\bibitem{whitney} H.  Whitney,  \textsl{On singularities of mappings of Euclidean spaces: I,  mappings of the plane into the plane},  Ann.  of Math.  62 (1955),  374--410. 

	
\end{thebibliography}
\end{document} 
%\bibitem{wrazidlo} D. Wrazidlo, \textsl{Bordism of constrained Morse functions}, arxiv:1803.11177.&
      % points 
      %  Thom's 
      % An answer will be presented in a forthcoming paper.
%STEP 1
% We delete the exposition via handle attachments to surfaces and decided to avoidusing the figure ''3hd.eps''. We respect the last version. First we start with Michalak'ws argument and generalize.  
%Step 2 Case 1 A_i → E_i
%We corrected some other minor phrases without changing the argument.
%Case 2 We changed the exposision of fold maps: we construct the fold map by using a defromation of smooth functions on an closed and connected surface of genus $0$ with holes (for the functions ''20201125func.eps'' is added). 
%In Problem 3, we adopt this way to present the answer more clearly and we added an exposition on smooth functions and a deformation of these functionsn we need. For STEP 1 and STEP 2 Case 1 in this scene, we introduce functions ${t^{\prime}}_{i,s_1,s_0,s_2}$ and ${t^{\prime}}_{d,s_1,s_0,s_2}$.
%The exposition of an answer to Problem 2 is revised.  