%% For double-blind review submission, w/o CCS and ACM Reference (max submission space)
%\documentclass[acmsmall,review,anonymous]{acmart}\settopmatter{printfolios=true,printccs=false,printacmref=false}
%% For double-blind review submission, w/ CCS and ACM Reference
%\documentclass[acmsmall,review,anonymous]{acmart}\settopmatter{printfolios=true}
%% For single-blind review submission, w/o CCS and ACM Reference (max submission space)
%\documentclass[acmsmall,review]{acmart}\settopmatter{printfolios=true,printccs=false,printacmref=false}
%% For single-blind review submission, w/ CCS and ACM Reference
%\documentclass[acmsmall,review]{acmart}\settopmatter{printfolios=true}
%% For final camera-ready submission, w/ required CCS and ACM Reference
%\documentclass[acmsmall,review,anonymous]{acmart}\settopmatter{}

%\documentclass[acmsmall]{acmart}\settopmatter{}
%\documentclass[acmsmall]{acmart}\settopmatter{printfolios=true,printccs=false,printacmref=false}
%\documentclass[acmsmall]{acmart}\settopmatter{printfolios=true,printccs=false,printacmref=false}
\documentclass[nonacm=true,format=acmsmall]{acmart}\settopmatter{}
%\documentclass[dvipsnames,nonacm=true,format=sigconf,anonymous=false,review=false]


%% Journal information
%% Supplied to authors by publisher for camera-ready submission;
%% use defaults for review submission.
\acmJournal{PACMPL}
\acmVolume{1}
\acmNumber{ICFP} % CONF = POPL or ICFP or OOPSLA
\acmArticle{1}
\acmYear{2023}
\acmMonth{1}
\acmDOI{} % \acmDOI{10.1145/nnnnnnn.nnnnnnn}
\startPage{1}

%% Copyright information
%% Supplied to authors (based on authors' rights management selection;
%% see authors.acm.org) by publisher for camera-ready submission;
%% use 'none' for review submission.
\setcopyright{none}
%\setcopyright{acmcopyright}
%\setcopyright{acmlicensed}
%\setcopyright{rightsretained}
%\copyrightyear{2018}           %% If different from \acmYear

%% Bibliography style
\bibliographystyle{ACM-Reference-Format}
%% Citation style
%% Note: author/year citations are required for papers published as an
%% issue of PACMPL.
\citestyle{acmauthoryear}   %% For author/year citations


%%%%%%%%%%%%%%%%%%%%%%%%%%%%%%%%%%%%%%%%%%%%%%%%%%%%%%%%%%%%%%%%%%%%%%
%% Note: Authors migrating a paper from PACMPL format to traditional
%% SIGPLAN proceedings format must update the '\documentclass' and
%% topmatter commands above; see 'acmart-sigplanproc-template.tex'.
%%%%%%%%%%%%%%%%%%%%%%%%%%%%%%%%%%%%%%%%%%%%%%%%%%%%%%%%%%%%%%%%%%%%%%


%% Some recommended packages.
\usepackage{booktabs}   %% For formal tables:
                        %% http://ctan.org/pkg/booktabs
\usepackage{subcaption} %% For complex figures with subfigures/subcaptions
                        %% http://ctan.org/pkg/subcaption

%% based on http://cgswords.github.io/latex-semantics/
%% Many thanks for the help typesetting!
%%\usepackage{amsmath, listings, bcprules, amsthm, amssymb, xspace, stmaryd}
\usepackage{amsmath, listings}
\usepackage{csvsimple}
\usepackage{xstring}
\usepackage{graphicx}
\graphicspath{ {./images/} }
\usepackage[ruled,vlined,linesnumbered,commentsnumbered]{algorithm2e}
\usepackage{xcolor}
\usepackage{float}
\usepackage{verbatim}
\usepackage{longtable}
\usepackage{multirow, multicol}
\usepackage{subfiles}

\definecolor{codegreen}{rgb}{0,0.6,0}
\definecolor{codegray}{rgb}{0.5,0.5,0.5}
\definecolor{codepurple}{rgb}{0.58,0,0.82}
\definecolor{backcolour}{rgb}{0.95,0.95,0.92}


\lstdefinestyle{mystyle}{
    backgroundcolor=\color{backcolour},   
    basicstyle=\ttfamily\footnotesize,
    breakatwhitespace=false,         
    breaklines=true,                 
    captionpos=b,                    
    keepspaces=true,                 
    numbers=left,                    
    numbersep=5pt,                  
    showspaces=false,                
    showstringspaces=false,
    showtabs=false,                  
    tabsize=2
}
\lstset{style=mystyle}

\newcommand{\alt}{~~|~~}
\newcommand{\comp}[1]{\llbracket #1 \rrbracket}
\newcommand{\inlineexp}[1]{
  {\footnotesize
  \[\begin{array}{l}
    #1
  \end{array}\]}}
\newcommand{\inlineexpa}[2]{
  {\footnotesize
  \[\begin{array}{#1}
    #2
  \end{array}\]}}

\newcommand{\krakenSpace} {\textit{Kraken} }
\newcommand{\kraken} {\textit{Kraken}}

\newcommand{\lamdefe} [2]     {\lambda #1.~#2}

\newcommand{\falsev}          {\mathsf{false}}
\newcommand{\truev}           {\mathsf{true}}

\newcommand{\Ctxt}            {\mathcal{E}}
\newcommand{\InCtxt}  [1]     {\Ctxt[#1]}


\newcommand{\ssosredex}       {\rightarrow}
\newcommand{\ctxtreduce}      {\mapsto}
\newcommand{\sstep}    [3] [] {#2 &\ssosredex&  #3 &\textsc{#1}}
\newcommand{\ctxtstep} [3] [] {#2 &\ctxtreduce& #3 &\textsc{#1}}

\newcommand{\subst}    [3] {#3 [#2 / #1]}

\newcommand{\kenv}     [3] {\langle \langle #1~|#2,~#3 \rangle \rangle}

\newcommand{\kprim}    [2] {\langle #1~\textbf{#2} \rangle}
%\newcommand{\kcomb}    [6] {\langle \textbf{comb} ~ #1 ~ #2 ~ #3 ~ #4 ~ #5 ~ #6\rangle}
\newcommand{\kcomb}    [5] {\langle \textbf{comb} ~ #1 ~ #2 ~ #3 ~ #4 ~ #5\rangle}
\newcommand{\keval}    [2] {[\text{eval} ~ #1 ~ #2]}
\newcommand{\kcombine} [3] {[\text{combine} ~ #1 ~ #2 ~ #3 ]}


\newcommand{\kmrk}        [5] {#1\{#2~#3~#4~#5\}}
\newcommand{\kmks}        [3] {\kmrk{#1}{#2}{#3}{\emptyset}{\emptyset}}
%\newcommand{\kmkd}        [1] {\kmrk{#1}{h}{PI}{\emptyset}{\emptyset}}
\newcommand{\kpeval}      [5] {[\text{peval} ~ #1 ~ #2 ~ #3 ~ #4 ~ #5]}
\newcommand{\kpevals}     [2] {\kpeval{#1}{#2}{ES}{f}{UHs}}
%\newcommand{\kunval}      [1] {[\text{uneval} ~ #1]}
\newcommand{\kpvthen}     [3] {[\text{vthen}~#1~#2~#3]}

\newcommand{\kpcombines}  [3] {[\text{pcombine}~{#1}~{#2}~{#3}~{ES}~{UHs}]}
\newcommand{\kpcombined}  [3] {[\text{pcombined}~{#1}~{#2}~{#3}~{ES}~{UHs}]}
\newcommand{\kpunders}    [4] {[\text{punder}~{#1}~{#2}~{#3}~{ES}~{UHs}~{#4}]}
\newcommand{\kpdropred}   [2] {\text{drop\_extra\_eval}(#1~#2~ES~UHs)}



\newcommand{\kmark}   [1] {\text{mark}(#1)}
\newcommand{\kmkd}    [2] {/#1/#2}
\newcommand{\kval}        {\mathsf{val}}
\newcommand{\kfresh}      {\mathsf{freshCall}}
\newcommand{\katt}    [2] {\mathsf{atmdCall}~#1~#2}
\newcommand{\kunval}  [1] {\text{unval}(#1)}
\newcommand{\kprog}   [1] {\text{neededIDs}(#1)}
\newcommand{\kuprog}  [1] {\text{upperIDs}(#1)}
\newcommand{\kiprog}  [1] {\text{resumeForms}(#1)}
%                                        T  E envSet currentlyExecuting
\newcommand{\kpval}     [4] {[\text{peval} ~ #1 ~ #2 ~ #3 ~ #4]}
\newcommand{\kpcombine} [5] {[\text{combine} ~ #1 ~ #2 ~ #3 ~ #4 ~ #5]}
\newcommand{\kunder}    [5] {[\text{under} ~ #1 ~ #2 ~ #3 ~ #4 ~ #5]}

\begin{document}
%% Title information
\title[Kraken]{Practical compilation of fexprs using partial evaluation}
                                        %% [Short Title] is optional;
                                        %% when present, will be used in
                                        %% header instead of Full Title.
%\titlenote{with title note}             %% \titlenote is optional;
                                        %% can be repeated if necessary;
                                        %% contents suppressed with 'anonymous'
\subtitle{Fexprs can performantly replace macros in purely-functional Lisp}
                                        %% \subtitle is optional
%\subtitlenote{with subtitle note}       %% \subtitlenote is optional;
                                        %% can be repeated if necessary;
                                        %% contents suppressed with 'anonymous'


%% Author information
%% Contents and number of authors suppressed with 'anonymous'.
%% Each author should be introduced by \author, followed by
%% \authornote (optional), \orcid (optional), \affiliation, and
%% \email.
%% An author may have multiple affiliations and/or emails; repeat the
%% appropriate command.
%% Many elements are not rendered, but should be provided for metadata
%% extraction tools.

%% Author with single affiliation.
\author{Nathan Braswell}
%\authornote{with author1 note}          %% \authornote is optional;
                                        %% can be repeated if necessary
\affiliation{
  \position{PhD Student}
  \department{School of Computer Science}       %% \department is recommended
  \institution{Georgia Institute of Technology} %% \institution is required
  \city{Atlanta}
  \state{GA}
  \country{United States}                    %% \country is recommended
}
\email{nathan.braswell@gtri.gatech.edu}      %% \email is recommended

\author{Sharjeel Khan}
%\authornote{with author1 note}          %% \authornote is optional;
                                        %% can be repeated if necessary
\affiliation{
  \department{School of Computer Science}       %% \department is recommended
  \institution{Georgia Institute of Technology} %% \institution is required                 %% \country is recommended
  \city{Atlanta}
  \state{GA}
  \country{United States}                    %% \country is recommended
}
\email{smkhan@gatech.edu}      %% \email is recommended

\author{Santosh Pande}
%\authornote{with author1 note}          %% \authornote is optional;
                                        %% can be repeated if necessary
\affiliation{
  \department{School of Computer Science}       %% \department is recommended
  \institution{Georgia Institute of Technology} %% \institution is required                 %% \country is recommended
  \city{Atlanta}
  \state{GA}
  \country{United States}                    %% \country is recommended
}
\email{santosh.pande@cc.gatech.edu}      %% \email is recommended


%% Author with two affiliations and emails.
%\author{First2 Last2}
%\authornote{with author2 note}          %% \authornote is optional;
                                        %%% can be repeated if necessary
%\orcid{nnnn-nnnn-nnnn-nnnn}             %% \orcid is optional
%\affiliation{
  %\position{Position2a}
  %\department{Department2a}             %% \department is recommended
  %\institution{Institution2a}           %% \institution is required
  %\streetaddress{Street2a Address2a}
  %\city{City2a}
  %\state{State2a}
  %\postcode{Post-Code2a}
  %\country{Country2a}                   %% \country is recommended
%}
%\email{first2.last2@inst2a.com}         %% \email is recommended
%\affiliation{
  %\position{Position2b}
  %\department{Department2b}             %% \department is recommended
  %\institution{Institution2b}           %% \institution is required
  %\streetaddress{Street3b Address2b}
  %\city{City2b}
  %\state{State2b}
  %\postcode{Post-Code2b}
  %\country{Country2b}                   %% \country is recommended
%}
%\email{first2.last2@inst2b.org}         %% \email is recommended

%Notes from meeting 2022-11-04:
% Bring out 2 key contributions in introduction
%   - new online partial evaluation
%   - optimizations enabled by semantics
% Beef up optimization sections
% benchmark with & without the different ones being on
% Talk about the type inference and how effective it is
% log and show percentage of inlined primitives?
% log and show number of lazy environments avoided
%     - get amount of memory bloat avoided
% log closure inlining, before/after
%
% Add some high-level pseudocode to help with understanding partial evaluation
% and an overview of where it really helps with optimizations
%
% Show when they can be statically removed (when used as macros!)
%
%- (me state) - compare vs macro impl in Chez
%
%Elaborate hygiene + first-class benefits
% macro(x) (x+x) and return intrinsic __arctan examples
%Professor thinks this should really be elaborated - double-check vs POPL reviewer introduction criticism
% this kinda goes with my and macro example
% specify online partial evaluation being online binding time, which we still do statically at compile time

%write based on abstract
% Our new type of partial evaluation which enables partial evalaution of f-exprs and our following optimizations

% Justify benchmarks
% qualitative discussion of f-exprs/macro use in benchmarks
% "well known? benchmarks for functional languages"
% "to demonstrate power of f-eprs have rewritten in Kraken"
%
% Find citations for macro programming being hard to debug/error prone
% "we believe our approach makes debuggability much better based on our personal experience and here is how"


% MY PLAN
%   - Add overview partial eval pseudocode
%   -   Finish explanation text, referring back to pseudocode
%   - Add examples using calculus
%   - Evaluation statistics (number vs time)
%   -   Lazy Environments
%   -   Y-Combinator elimination
%   -   Primitive Inlining
%   -   Closure Inlining
%   - Introduction - hygine example to go with and example
%   -   specify "online" partial evaluation and explain


%% Abstract
%% Note: \begin{abstract}...\end{abstract} environment must come
%% before \maketitle command
\begin{abstract}
%In this work we devise an online partial evaluator and compiler backend supporting first-class, partially symbolic environments specifically focused on evaluating away f-exprs that behave like macros in order to show that a functional Lisp based on F-expressions (f-exprs) can be approximately as efficient and at least as expressive as one based on macros.
%  Macros are a common part of Lisp languages but it is difficult to create a powerful, safe, and intuitive macro system due to the complexities of controlling hygiene as well as macro's second-class nature.
%The resulting macro systems tend to be complex and resulting programs quite difficult to debug.

%% COMMENTS - Santosh : Can you simplify the writing in the following paragraph - break down into smaller sentences and easy flow
%% Made a run at it, thank you!
Macros are a common part of Lisp languages, and one of their most lauded features.
Much research has gone into making macros both safer and more powerful resulting in developments in multiple areas, including maintaining hygiene, and typed program staging \cite{rompf2013optimizing}.
However, macros do suffer from various downsides, including being second-class. Particularly egregious for eager functional programming, they are unable to be passed to higher-order functions or freely composed.

  Fexprs, as reformulated by John Shutt \cite{shutt2010fexprs}, provide a first-class and more powerful alternative to macros that meshes well with pure functional programming.
 Unfortunately, naive execution of fexprs is much slower than macros due to re-executing unoptimized operative combiner code at runtime that, in a macro-based language, would have been expanded and then optimized at compile time.
To show that fexprs can be practical replacements for macros, we formulate a small purely functional fexpr based Lisp, \kraken, with an online partial evaluation and compilation framework that supports first-class, partially-static-data environments and can completely optimize away fexprs that are used and written in the style of macros.
We show our partial evaluation and compilation framework produces code that is more than 70,000 times faster than naive interpretation due to the elimination of repeated work and exposure of static information enabling additional optimization.
  In addition, our \krakenSpace compiler performs better compared to existing interpreted languages that support fexprs, including improving on NewLisp's~\cite{mueller2018newlisp} fexpr performance by 233x on one benchmark.

  % All code available at \url{https://github.com/limvot/kraken}
  %(Code publically available but URL redacted for blind review)
\end{abstract}


%% 2012 ACM Computing Classification System (CSS) concepts
%% Generate at 'http://dl.acm.org/ccs/ccs.cfm'.
\begin{CCSXML}
<ccs2012>
<concept>
<concept_id>10011007.10011006.10011008.10011009.10011012</concept_id>
<concept_desc>Software and its engineering~Functional languages</concept_desc>
<concept_significance>500</concept_significance>
</concept>
<concept>
<concept_id>10011007.10011006.10011041</concept_id>
<concept_desc>Software and its engineering~Compilers</concept_desc>
<concept_significance>500</concept_significance>
</concept>
</ccs2012>
\end{CCSXML}

\ccsdesc[500]{Software and its engineering~Functional languages}
\ccsdesc[500]{Software and its engineering~Compilers}
%% End of generated code


%% Keywords
%% comma separated list
\keywords{partial evaluation, Vau, F-exprs, fexprs, WebAssembly} %% \keywords are mandatory in final camera-ready submission


%% \maketitle
%% Note: \maketitle command must come after title commands, author
%% commands, abstract environment, Computing Classification System
%% environment and commands, and keywords command.
\maketitle

\section{Introduction}

The increasing complexity of source code poses a key challenge to the reliability of large-scale software systems. Software bugs in these systems can lead to safety issues~\cite{bug_safety} for users around the world as well as cause non-negligible financial losses~\cite{bug_loss}. As such, developers have to spend a large amount of time and effort on bug fixing. Consequently, \aprfull (\apr), designed to automatically generate patches to fix software bugs, has attracted wide attention from both academia and industry~\cite{long2016prophet, legoues2012genprog, long2015spr, lou2020can, tufano2018empstudy}. 


To achieve \apr, one popular approach is known as Generate-and-Validate (G\&V)~\cite{qi2015gv, ghanbari2019prapr, lou2020can, le2016hdrepair, legoues2012genprog, wen2018capgen, hua2018sketchfix, martinez2016astor, koyuncu2020fixminder, liu2019tbar, liu2019avatar}, which is typically based on the following pipeline: First, fault localization techniques~\cite{wong2016fl, abreu2007ochiai, zhang2013injecting, papadakis2015metallaxis, li2019deepfl, li2017transforming} are applied to determine the suspicious locations in programs where bugs are likely to exist. Then, the buggy locations are used by the \apr tools to generate a list of patches that replace buggy lines with correct lines. Afterward, each patch is validated against the original test suite to identify any \emph{plausible patches} (i.e., passing all tests in the test suite). Finally, to determine the \emph{correct patches}, developers examine the list of plausible patches to see if any of them can correctly fix the bug. 

Traditional \apr tools can mainly be categorized into heuristic-based~\cite{legoues2012genprog, le2016hdrepair, wen2018capgen}, constraint-based~\cite{mechtaev2016angelix, le2017s3, demacro2014nopol, long2015spr} and \template~\cite{ghanbari2019prapr, hua2018sketchfix, martinez2016astor, liu2019tbar, liu2019avatar}. Among these traditional tools, \template \apr tools~\cite{ghanbari2019prapr, liu2019tbar, benton2020effectiveness} have been able to achieve state-of-the-art results. \Template \apr tools typically leverage pre-defined templates (e.g., adding a nullness check) for bug fixing. However, since these fix templates are typically handcrafted, the number and types of bugs they are able to fix can be limited. 



To address the limitations of traditional \apr, researchers have proposed various \learning \apr tools~\cite{li2020dlfix, chen2018sequencer, jiang2021cure, lutellier2020coconut, zhu2021recoder, ye2022rewardrepair} based on the \nmtfull (\nmt) architecture~\cite{sutskever2014mt} where the input is the buggy code snippets and the goal is to translate the buggy code snippets into a fixed version. To accomplish this, \learning \apr tools require supervised training datasets with pairs of both buggy and fixed code snippets in order to learn how to perform this translation step. These training data are usually obtained by mining historical bug fixes using heuristics/keywords~\cite{dallmeier2007benchmark}, which can be imprecise for identifying bug-fixing commits; even the actual bug-fixing commits can include irrelevant code changes, leading to further pollution in the dataset~\cite{xia2022alpharepair}.
% 
Moreover, it can be hard for such \apr tools to generalize and fix bug types unseen during training. 



To better leverage recent advances in \plmfull{s} (\plm{s}), researchers~\cite{xia2022alpharepair, xia2023repairstudy, kolak2022patch, prenner2021codexws} have directly applied \plm{s} to generate patches without bug-fixing datasets. These \llm-based \apr tools work by either directly generating a complete code function~\cite{prenner2021codexws, xia2023repairstudy} or predict/infill the correct code snippet given its surrounding context~\cite{xia2022alpharepair, xia2023repairstudy}. By directly using \llm{s} that are pre-trained on billions of open-source code snippets, \llm-based \apr tools can achieve state-of-the-art performance on many repair datasets~\cite{xia2022alpharepair}. 


% 
%
%

Traditional \apr tools have long used the insight of the \emph{plastic surgery hypothesis}~\cite{barr2014plastic} where it states that the code ingredients to fix a bug already exist within the same project. Traditional \apr tools have manually designed pattern-~\cite{ghanbari2019prapr, saha2017elixir} or heuristic-based~\cite{jiang2018simfix, legoues2012genprog} approaches to finding and using such relevant code ingredients to generate fixes for bugs. However, the plastic surgery hypothesis has been largely ignored in \llm-based \apr. In fact, \llm provides a unique opportunity to fully automate the plastic surgery hypothesis idea via fine-tuning (learning project-specific information via model updates from the buggy project) and prompting (directly providing relevant code ingredients to the model), and make it directly applicable to different languages (since the \llm{s} are typically multi-lingual).%
Moreover, despite the intensive manual efforts involved, traditional \apr tools still cannot fully leverage project-specific information due to large search space for leveraging/composing existing code ingredients. In contrast, the project-specific information can effectively leveraged by \llm{s} due to their power in code understanding/vectorization, e.g., even partial/imprecise information may still guide \llm{s} in correct patch generation!
 To this end, we ask the question: \emph{How useful is the plastic surgery hypothesis in the era of \plm{s}}?








\mypara{Our Work.} To answer the question, we present \ourtech{\xspace} -- a \llm-based approach that automatically utilizes the plastic surgery hypothesis by systematically combining multiple fine-tuning and prompting strategies for \apr. \ourtech fine-tunes \plm{s} using two novel domain-specific training strategies: \textbf{\epfinetune} -- we fine-tune using the original buggy project by aggressively masking out a high percentage of tokens, which allows \plm to learn project-specific code tokens and programming styles; and \textbf{\rofinetune} -- which only masks out a single continuous code sequence per training sample, allowing the model to get used to the final \csapr task of predicting a single continuous code sequence. Furthermore, we directly leverage the ability for \plm{s} to understand natural language instructions and introduce a novel prompting strategy, \textbf{\idprompting}, which uses information retrieval and static analysis to obtain a list of relevant identifiers for the buggy lines. While such relevant identifiers are critical for fixing some difficult bugs, they may not be seen by the \llm during inference due to limited context window size. Through the use of prompting, we directly tell the model to use these extracted identifiers (relevant code ingredients) to generate the correct code. Finally, to perform repair, we combine all four model variants (including the base model, both fine-tuned models and the base model with prompting) for the final repair.





While our insight of leveraging the plastic surgery hypothesis for \llm-based \apr is generalizable across different types of \plm{s}, to implement \ourtech, we choose a recent \plm{\xspace}, \ctfive~\cite{wang2021codet5}, which is pre-trained on millions of open-source code snippets. \ctfive is an encoder-decoder model trained using \mspfull (\msp) objective where a percentage of tokens are masked out and each continuous masked token sequence is referred to as a masked span. Also, although we only extract relevant identifiers from the current buggy project (since this paper focuses on the plastic surgery hypothesis), our work can be easily extended to obtain other code information (such as relevant statements or functions) from other sources, such as  the massive pre-training corpora~\cite{husain2020codesearchnet} or historical bug-fixing datasets~\cite{jiang2019infer}, which can provide more coding knowledge for \llm{s}. Besides, although we mainly focus on using traditional string comparison algorithms for information retrieval in this paper, these techniques can be easily replaced by other frequency-based retrieval~\cite{robertson2009probabilistic} and neural search (or embedding-based search)~\cite{reimers2019sentence}.
  In summary, this paper makes the following contributions:


%


\begin{itemize}[noitemsep, leftmargin=*, topsep=0pt]
    \item \textbf{Dimension.} This paper is the first to revisit the important plastic surgery hypothesis in the era of \llm{s}. It opens up a new dimension for \llm-based \apr to incorporate previously neglected information from the buggy project itself to boost \apr performance. Furthermore, it demonstrates the promising future of retrieval-based prompting for modern \llm-based \apr.
    \item \textbf{Implementation.} We implement \ourtech based on the recent \ctfive model. We augment the model using two novel fine-tuning strategies: \epfinetune and \rofinetune, along with a novel prompting strategy based on information retrieval and static analysis: \idprompting. We combine the patches generated by all four models together and perform patch ranking to speed up \apr.% 
    \item \textbf{Evaluation Study.} We conduct an extensive evaluation against state-of-the-art \apr tools. On the widely studied \dfj 1.2 and 2.0 datasets~\cite{just2014dfj}, \ourtech is able to achieve the new state-of-the-art results of 89 and 44 correct bug fixes (15 and 8 more than best baseline) respectively.  Furthermore, we perform a broad ablation study to justify our design. \ourtech demonstrates for the first time that the plastic surgery hypothesis can substantially boost \llm-based \apr and advance state-of-the-art \apr, while being fully automated and general. Moreover, even partial/imprecise code ingredients may still effectively guide \llm{s} for \apr!
\end{itemize}


\section{Applications}
\label{sec:apps}
To demonstrate the wide range of usagages of our model, we implement a series of applications:
\begin{enumerate}
	\item Incremental surface \& color reconstruction
	\item 3D saliency detection
	\item Open vocabulary scene understanding
	\item Surface infrared field
	\item 3D style transfer
\end{enumerate}
Originating from our motivation in inspection and service robotics, we implement 1) Incremental surface \& color reconstruction for visualization of robot surroundings.
For robot exploration, we implement 2) 3D saliency detection to indicate the salient regions in maps.
For recovering object-level semantic information in environments, we implement 3) open vocabulary scene understanding to yield the regions containing the objects..
Furthermore, to demonstrate the flexibility, we implement 4) surface infrared fields and 5) 3D style transfer for artistic purposes. 

In~\cref{fig:latent_diff}, we classify those 3 applications into 3 categories: (a) directly obtaining the properties from sensor observation, such as application 1) and 4). (b) processing on sensor data and predict properties, such as application 2), 5). (c) extending (b) to operating beyond latent features, such as application 3).
%Thus, in the following, we discuss about those categories of applications.
% we mainly describe the application 1) (\cref{sec:incremental_reconstruction}) and 3) (\cref{sec:openvoc}).

Application 1) and 4) are in the first one category. Thus, we mainly describe 1) incremental surface \& color reconstruction (\cref{sec:incremental_reconstruction}), while for 4) we can easily exchange color with infrared.
%
For the second with 2) and 5) in~\cref{sec:fabircated_prop}, we mainly describe the usage of fabricated properties.
As the mapping part is redundant to previous category, it will not be detailed.
%
The third category is the application 3) that maps a LIM for high dimensional latent fields.
We demonstrate that this application provides a flexible inference in \cref{sec:openvoc}.


%Afterwards, we evaluate application 1) and 3) in~\cref{sec:exp} and extensively show demonstration for all application in~\cref{sec:exp:extensive_app}.

\documentclass[12pt]{article}
\usepackage{article_base}
\usepackage{datetime2}
\usepackage[misc]{ifsym}
\title{A Hodge-Type Filtration on Rigid Cohomology}
\date{}
\author{
Bruno Chiarellotto\footnote{\Letter \, chiarbru@math.unipd.it}, Yukihide Nakada\footnote{\Letter \, yukihide.nakada@unipd.it \newline \indent $\ast, \dag$ Dipartimento di Matematica ``Tullio Levi-Civita", Universit\'a degli Studi di Padova, Via Trieste 63, 35121 Padova, Italy
\newline
\newline 
}
}
%\date{Compiled \DTMnow}
\begin{document}
\maketitle
\begin{abstract}
    Given a scheme $X$ over a complete discrete valuation ring $\O_K$ of mixed characteristic, Gros, in his study of syntomic cohomology in the smooth and proper case, introduced a filtration on the rigid cohomology of the special fiber $X_k$.
Gros claimed that this construction was independent of the choice of immersions involved, citing an unpublished paper of Berthelot.
    However this paper and Berthelot's argument, in preparation at the time, was never published and no manuscripts are known exist.
    Here, we provide a complete proof of this result.
\end{abstract}

\textbf{Keywords}\, Rigid Cohomology $\cdot$ $p$-adic cohomology

\vspace{12pt}

\noindent \textbf{Mathematics Subject Classification}\, 14F30; Secondary 14G22, 11G25
\section{Introduction}
Fix a prime number $p$.
Let $K/\Q_p$ be a finite extension with ring of integers $\O_K$ and residue field $k$.
In \cite{Gros1994}, Gros studies the syntomic cohomology of a scheme defined over $\O_K$.
This cohomology theory has several, overlapping definitions (see as a sampler \cite[Definition 5.3.2]{ChiarellottoCiccioniMazzari2013}, \cite[Definition 2.1]{Gros1994}, and \cite[Definition 8.4]{Besser2000}) but the idea is that for a scheme $X \to \op{Spec}(\O_K)$, one searches for a filtration that entwines the Hodge filtration on the cohomology of the generic fiber $H^*_{\op{dR}}(X_K)$ with the Frobenius action (and its filtration) on the $p$-adic cohomology of the special fiber.
From a broader point of view, it is the $p$-adic analogue of Deligne-Beilinson cohomology.

Instead of a traditional Hodge filtration, Gros's definition of syntomic cohomology in the proper and smooth case used a distinct filtration \cite[(3.2)]{Gros1994} on the rigid cohomology arising from the characteristic $p$ special fiber.
This filtration has also been contextualized by Besser \cite[\S 9]{Besser2000}, who showed that it is related to his definition of syntomic cohomology.
This filtration turns out to coincide with the na\"ive filtration of the dagger de Rham complex in the smooth and affine case (hence in the  Monsky-Washnitzer cohomology), but there are potential links to prismatic cohomology and the Nygaard filtration.

Just as the construction of rigid cohomology involves the choice of a compactification $X_k \subseteq Y_k$ followed by a closed immersion $Y_k \subseteq P$ into a formal $\O_K$-scheme smooth around $X_k$, the definition of this filtration involves some choices.
An essential element of the construction of rigid cohomology is that the resulting complex $R\Gamma_{\op{rig}}(X_k)$ is independent of the choices.
The same independence for the filtration has been suggested as proven by both Gros and Besser but no such proof in fact exists, leaving a hole in the literature: our goal in this paper is to prove that the filtration is independent of the choice of frame.

To be more precise, we recall that for any algebraic $k$-variety $X_k$,  Berthelot \cite{Berthelot1986} defines the rigid cohomology groups $H^n_{\op{rig}}(X_k)$ by embedding $X_k$ within a $K$-frame 
\[
X_k \hookrightarrow Y_k \hookrightarrow \fP
\]
where $X_k \hookrightarrow Y_k$ is an open immersion into a proper $k$-scheme $Y_k$ and $Y_k \hookrightarrow \fP$ is a closed immersion of $Y_k$ into a formal $\O_K$-scheme $\fP$, smooth in a neighborhood of $X_k$.
In this context he constructs the complex of overconvergent differential forms $\jdag_{X_k} \Omega^{\bullet}_{]Y[_\fP}$ and defines rigid cohomology as
\[
    H^n_{\op{rig}}(X_k) := H^n(\tube{Y_k}{\fP},\jdag_{X_k}\Omega^{\bullet}_{]Y_k[_\fP}).
\]
This construction is shown to be independent of the choice of frame.

Suppose now that there exists a pair of morphisms
\[
X_k \to Y \hookrightarrow P
\]
where $Y$ and $P$ are $\O_K$-schemes, $X_k \hookrightarrow Y_k$ is an open immersion into the special fiber $Y_k$ of $Y$, and $Y \hookrightarrow P$ is a closed immersion. 
Let $\fP := \widehat{P}$ denote the formal completion of $P$. 
In this setting Gros introduces in \cite[\S3]{Gros1994} a decreasing filtration on $\jdag_{X_k}\Omega^\bullet_{\tube{Y_k}{\fP}}$ of complexes of $\jdag_{X_k}\O_{\tube{Y_k}{\fP}}$-modules, which we denote by $\op{Fil}^s \subseteq \jdag_{X_k}\Omega^\bullet_{\tube{Y}{\fP}}$.
We call it the \emph{Gros filtration}, and it induces a filtration
\[
    F^sH^m_{\op{rig}}(X_k) := \op{Im}(H^n(]Y_k[_P, \op{Fil}^s) \to H^m_{\op{rig}}(X_k)) \subseteq H^m_{\op{rig}}(X_k)
\]
on rigid cohomology.

This filtration is simply rigid cohomology in degree $s = 0$, so it is natural to ask whether the entire filtration is independent on the choice of immersions $(X_k \subseteq Y \subseteq P)$, in a sense that we will make more precise in Section \ref{gpl_section_gros_filtration}.
Gros credits a positive answer to this question to Berthelot as \cite[Proposition 3.3, Proposition 3.5]{Gros1994} but the paper, which was in preparation at the time, remained unpublished with no known existent partial proofs. 
Our main result in this paper is Theorem \ref{gros_independence}, which provides a complete proof.

\section{The Gros Filtration}\label{gpl_section_gros_filtration}
Let $X_k$ be a $k$-scheme. 
We give a name to the lifts of frames over $\O_K$ that we need to define the filtration:
\begin{definition}
    An \emph{algebraic $\O_K$-frame for $X_k$} is a sequence of embeddings
    \[
    X_k \hookrightarrow Y \hookrightarrow P
    \]
    where $Y$ and $P$ are $\O_K$-schemes, $X_k \hookrightarrow Y_k$ is an open immersion, and $Y$ is closed in $P$. 
    We denote an algebraic $\O_K$-frame by $(X_k \subseteq Y \subseteq P)$.
    Morphisms of algebraic $\O_K$-frames are defined in the natural way. 
\end{definition}

% \textbf{Or rather, an algebraic $\O_K$-frame has nothing to do with $X_k$, and the filtration is defined when $X_k$ can be embedded in an algebraic $\O_K$-frame.}

\begin{remark}
    We note the distinction between the notion of an algebraic $\O_K$-frame and that of a $S$-frame when $S$ is a \emph{formal} $\O_K$-scheme, for example $S = \op{Spf}(\O_K)$. 
    This is typically defined (see \cite[Definition 3.1.6]{LeStum2007}) to be a $K$-frame over the trivial frame 
    \[
    S_k = S_k \hookrightarrow S    
    \]
    associated to $S$.
    They are linked by the fact that any algebraic $\O_K$-frame $(X_k \subseteq Y \subseteq P)$ induces a $\op{Spf}(\O_K)$-frame $(X_k \subseteq Y_k \subseteq \widehat{P})$.
\end{remark}

We define properties of algebraic $\O_K$-frames analogously to their standard counterparts:
\begin{definition}\label{definition_morphisms_of_frames}
    Let
    \begin{center}
        \begin{tikzcd}
            X_k \ar[r,hook] \ar[d] & Y \ar[r, hook] \ar[d, "f"] & P \ar[d,"u"]\\
            C_k \ar[r,hook] & D \ar[r,hook] & Q
        \end{tikzcd}
    \end{center}
    be a morphism of algebraic $\O_K$-frames.
    We say that it is
    \begin{enumerate}
        \item \emph{open} when all vertical maps are open immersions, \emph{mixed} when the first two are closed immersions and the last is an open immersion, \emph{Cartesian} if the left-hand square is Cartesian,
        % [\textbf{I need to check if this is the right definition for our new definition of algebraic $\O_K$-frame with $X_k$ instead of $X$}]
        and \emph{strictly Cartesian} or \emph{strict} when both are;
    
        \item \emph{smooth} (resp.\,\emph{\'etale}) if $u$ is smooth (resp.\,\'etale) in a neighborhood of $X$; 

        \item \emph{projective} (resp.\,\emph{proper}) if $f$ is projective (resp.\,proper).
    \end{enumerate}

    An algebraic $\O_K$-frame $(X_k \subseteq Y \subseteq P)$ is called \emph{smooth} (resp.\,\emph{proper}, \emph{projective}, \emph{\'etale}) if the morphism 
    \begin{center}
        \begin{tikzcd}
            X_k \ar[r,hook] \ar[d] & Y \ar[r, hook] \ar[d] & P \ar[d]\\
            \op{Spec}k \ar[r,hook] & \op{Spec}\O_K \ar[r, equal] & \op{Spec}\O_K
        \end{tikzcd}
    \end{center}
    is smooth (resp. proper, projective, \'etale). 
\end{definition}
It is easy to check that if $\mathcal{P}$ is one of the properties of a morphism of algebraic $\O_K$-frames defined above then $\mathcal{P}$ holds for the induced morphism of frames, in the sense of \cite[Definitions 3.1.11, 3.3.5, 3.3.10]{LeStum2007}.

% Fix an algebraic $k$-variety $X_k$. Locally we may embed it inside a $K$-frame $X_k \to Y_k \to \widehat{P}$ arising from an \emph{algebraic $\O_K$-frame}:

For brevity, given an algebraic $\O_K$-frame $(X_k \subseteq Y \subseteq P)$, we'll write $\jdag_X := \jdag_{X_k}$, $]X[_P := ]X_k[_{\widehat{P}}$ and $]Y[_P := ]Y_k[_{\widehat{P}}$ when there's no risk of confusion
Let $I_{X,Y,P}$, or simply $I$ when the context is clear, be the ideal defining the closed subspace $\widehat{Y}_K$ in $]Y[_P$, that is, the kernel 
\[
0 \to I_{X,Y,P} \to \O_{\tube{Y}{P}} \to i_*\O_{\widehat{Y}_K} \to 0.
\]

Recall that the complex $\jdag_X \Omega^\bullet_{\tube{Y}{P}}$ computes rigid cohomology.
Our goal in this paper is to study the following filtration on $\jdag_X\Omega^\bullet_{\tube{Y}{P}}$ and the corresponding filtration in cohomology:
\begin{definition}
    \begin{enumerate}
        \item The \emph{Gros filtration} on $\jdag_X \Omega_{\tube{Y}{P}}^\bullet$ is given in degree $s \in \bN^{\ge 0}$ by 
        \begin{align*}
            \op{Fil}^s = \op{Fil}^s_{X,Y,P} &:= \jdag_X(I^s \to I^{s-1}\Omega^1_{\tube{Y}{P}} \to I^{s - 2}\Omega^2_{\tube{Y}{P}} \to \dots)\\
            &= \jdag_X(I^{s - \bullet} \otimes \Omega^\bullet_{\tube{Y}{P}}).
        \end{align*}
        \item The induced filtration
    \[
        F^sH^n_{\op{rig}}(X_k) := \op{Im}(H^n(\tube{Y}{P}, \op{Fil}^s) \to H^n_{\op{rig}}(X_k)) \subseteq H^n_{\op{rig}}(X_k)    
    \]
    we call the \emph{Hodge-type filtration} on rigid cohomology.
    \end{enumerate}
\end{definition}

The term Hodge-type filtration on rigid cohomology is justified in part by the following special cases.
\begin{example}
Suppose $X_k$ admits a frame of the form $(X_k \subseteq Y_k \subseteq \widehat{Y})$ induced by an algebraic $\O_K$-frame $(X_k \subseteq Y \subseteq Y)$ where $Y$ is proper. 
In this case the filtration collapses into the usual ``naive'' filtration.
Indeed, if such a frame exists then with respect to this frame we have 
\[
    \tube{Y}{P} \cong \widehat{Y}_K
\]
so that $I_{X,Y,P} = 0$.
Then
\begin{align*}
    (\op{Fil}^s_{X,Y,P})^i &= \jdag_X(I^{s-i} \otimes \Omega_{\tube{Y}{P}}^i)\\
    &= 
    \begin{cases}
        0 & \text{ when } i \le s\\
        \jdag_X \Omega^i_{\tube{Y}{P}} &\text{ when }i > s
    \end{cases}
\end{align*}
which is the filtration inducing the ``naive'' filtration.

This condition is satisfied, for instance, when $X_k$ is affine smooth or when it has a proper and smooth lifting $X$ over $\O_K$.
If $X_k \cong \op{Spec}(k[x_1,\dots,x_n]/\mathfrak{a})$ is smooth and affine, in which case rigid cohomology coincides with Monsky-Washnitzer cohomology, $X_k$ has a smooth lifting
\[
X \cong \op{Spec}(\O_K[x_1,\dots,x_n]/\tilde{\mathfrak{a}}).
\]
If $Y = \overline{X} \subseteq \P^n_{\O_K}$ denotes the closure of $X$ in $\P^n_{\O_K}$, then 
\[
    X_k \hookrightarrow Y \hookrightarrow Y
\]
is such an $\O_K$-frame for $X_k$. 
If $X$ has a proper and smooth lifting $X$ over $\O_K$, the trivial $\O_K$-frame 
\[
    X_k \hookrightarrow X \hookrightarrow X
\]
is such an $\O_K$-frame for $X_k$. 
% In this setting, we may write this filtration as 
% \[
%     \op{Fil}^s \cong \Omega^{\ge s}_{A^{\dagger}} \otimes K
% \]
\end{example}

Berthelot shows that $H^n_{\op{rig}}(X_k)$ is independent of the proper smooth frame $(X_k \subseteq Y_k \subseteq \fP)$ used to compute it.
In order to use the Gros filtration to induce a well-defined filtration on rigid cohomology, we need to show that this filtration is also independent of our choices in the derived category.
More precisely, our goal is to show the following:
\begin{theorem}[({\cite[Proposition 3.3, 3.5]{Gros1994}})]\label{gros_independence}
    Suppose 
    \begin{center}
        \begin{tikzcd}
        & Y' \ar[dd,"g"] \ar[r, closed, hook] & P' \ar[dd, "u"]\\
        X_k \ar[ur, open, hook] \ar[dr, open, hook]& \\
        & Y \ar[r, closed, hook]& P
        \end{tikzcd}
    \end{center}
    is a proper smooth morphism of smooth algebraic $\O_K$-frames with $Y$ and $Y'$ reduced.
    Let $u_K: \tube{Y'}{P'} \to \tube{Y}{P}$ denote the induced map of tubes. 
    Then the filtration induced by these frames are isomorphic; that is, the natural base change map
    \[
    \op{Fil}^s_{X,Y,P} \cong Ru_{K*}\op{Fil}^s_{X,Y', P'}   
    \]
    is an isomorphism.
\end{theorem}

This result should be thought of as a variant of the independence of rigid cohomology proven by Berthelot and detailed by Le Stum:
\begin{proposition}[(Berthelot, Le Stum {{\cite[Proposition 6.5.3]{LeStum2007}}})]\label{gros_Berthelot_result}
    Let $S$ be a formal $\O_K$-scheme and let
    \begin{center}
        \begin{tikzcd}
        & Y_k' \ar[dd,"g"] \ar[r, closed, hook] & \fP' \ar[dd, "u"]\\
        X_k \ar[ur, open, hook] \ar[dr, open, hook]& \\
        & Y_k \ar[r, closed, hook]& \mathfrak{P}
        \end{tikzcd}
    \end{center}
    be a proper smooth morphism of smooth $S$-frames.
    Let $E$ be a coherent $\jdag_{X_k}\O_{\tube{Y_k}{\fP}}$-module with an overconvergent integrable conection over $S_K$.
    Then the base change map is an isomorphism 
    \[
    u*: E \otimes_{\tube{Y_k}{\fP}} \Omega^\bullet_{\tube{Y_k}{\fP}/S_K} \cong Ru_{K*} u^{\dagger}E \otimes_{\O_{\tube{Y_k'}{\fP'}}} \Omega_{\tube{Y_k'}{\fP'}/S_K}^\bullet.
    \]
\end{proposition}
% \textbf{[Are there any conditions on $S$?]}

\begin{notation}
    The notation $\jdag_X$ is ambiguous since $\jdag_X := \jdag_{\tube{X}{P}}$ depends on our choice of algebraic $\O_K$-frame $(X_k \subseteq Y \subseteq P)$.
    When there's ambiguity in the choice of embedding we'll write $\jdag_P := \jdag_{\tube{X}{P}}$ for the dagger functor corresponding to an algebraic $\O_K$-frame $(X_k \subseteq Y \subseteq P)$.
\end{notation}

Berthelot's proof of Proposition \ref{gros_Berthelot_result} is a series of reductions concluding in an explicit computation of the filtration; namely, one reduces the general result to the case where (1) the morphism is proper \'etale, and (2) $g = \op{id}_Y$.
Our argument for the independence of the filtration follows this argument closely. 
We begin the proof in \S \ref{gpl_section_basics_and_general_case} by laying out the basic formalism for working with the Gros filtration.
We then prove Theorem \ref{gros_independence} by assuming the following special cases, analogous to the cases to which Berthelot's proof reduced the independence of rigid cohomology:
% Likewise, two special cases to which our result is reduced are the following:

\begin{manuallemma}{\ref{gpl_proper_etale} \textnormal{(Proper Etale (\emph{PE}))}}
    Suppose 
    \begin{center}
        \begin{tikzcd}
        & Y' \ar[dd,"g"] \ar[r, closed, hook] & P' \ar[dd,"u"]\\
        X_k \ar[ur, open, hook] \ar[dr, open, hook]& \\
        & Y \ar[r, closed, hook]& P
        \end{tikzcd}
    \end{center}
    is a proper \'etale morphism of smooth algebraic $\O_K$-frames with $Y$ and $Y'$ reduced.
    Then
    \[
    \op{Fil}^s_{X,Y,P} \xrightarrow{\sim} Ru_{K*}\op{Fil}^s_{X,Y', P'}.
    \]
\end{manuallemma}

\begin{manualtheorem}{\ref{gpl} \textnormal{(Global Filtered Poincar\'e Lemma (\emph{GFPL})})}
    Let
    \begin{center}
        \begin{tikzcd}
            & & P' \ar[dd, "u"]\\
            X_k \ar[r, hook, open] & Y \ar[ur, hook, closed] \ar[dr, hook, closed]\\
            & & P
        \end{tikzcd}
    \end{center}
    be a smooth morphism of smooth algebraic $\O_K$-frames.
    % and let $u_K$ denote the the induced map of tubes $u_K: \tube{Y}{P'} \to \tube{Y}{P}$.
    Then there is a quasi-isomorphism 
    \[
       \jdag_P I_{X,Y,P}^\ell \xrightarrow{\sim} Ru_{K*}\jdag_{P'} (I_{X,Y,P'}^{\ell - \bullet} \otimes_{\O_{\tube{Y}{P'}}} \Omega_{\tube{Y}{P'}/\tube{Y}{P}}^\bullet).
    \]
    Moreover, $\op{Fil}^s_{X,Y,P} \cong Ru_{K*}\op{Fil}^s_{X,Y,P'}$.
\end{manualtheorem}
We prove these special cases in \S \ref{gpl_section_proper_etale} and \S \ref{gpl_section_gpl}, respectively.

\section{Proof of Main Theorem by Reduction to Special Cases}\label{gpl_section_basics_and_general_case}
We begin by providing some basic manipulations for the Gros filtration, analogous to results proved in \cite{LeStum2007} for working with ordinary (relative) rigid cohomology.
% In order to make the parallels more transparent, we define a filtered version of relative rigid cohomology:
% \begin{definition}[(c.f. {{\cite[Definition 6.2.1]{LeStum2007}}})]\label{gpl_filtered_rigid_cohomology}
    % Let 
    % \begin{center}
    %     \begin{tikzcd}
    %         X_k \ar[r,hook] \ar[d] & Y \ar[r, hook] \ar[d] & P \ar[d,"u"]\\
    %         C_k \ar[r,hook] & D \ar[r,hook] & Q
    %     \end{tikzcd}
    % \end{center}
    % be a morphism of algebraic $\O_K$-frames.
    % If $E$ is a $\jdag_X \O_{\tube{Y}{P}}$-module with an integrable connection over $K$, we define
    % \[
    % Ru_{\op{rig-fil}, s}E := Ru_{K*}(E \otimes I^{s-\bullet}_{X,Y,P}\Omega^\bullet_{\tube{Y}{P}})    
    % \]
    % with $u_K: \tube{Y}{P} \to \tube{D}{Q}$.
% \end{definition}
First of all, our arguments require that $\op{Fil}^s_{X,Y,P}$ and $Ru_{K*}\op{Fil}^s_{X,Y',P'}$ be local in some sense.
A prerequisite is that the ideal $I_{X,Y,P}$ is appropriately local:
\begin{lemma}\label{gpl_ideal_restriction}
    % Let
    % \begin{center}
    %     \begin{tikzcd}
    %         X' \ar[r, open, hook] \ar[d, hook] & Y' \ar[r, closed, hook] \ar[d, hook, open] & P' \ar[open, d]\\
    %         X \ar[r, open] & Y \ar[r, closed] & P
    %     \end{tikzcd}
    % \end{center}
    % be an open immersion of algebraic frames.
    % Then we have an isomorphism 
    % \[
    % (\jdag_X I_{X,Y,P})|_{\tube{Y'}{P'}} \cong \jdag_{X'}I_{X',Y',P'}
    % \]
    Let $(X_k \subseteq Y \subseteq P)$ be an algebraic $\O_K$-frame.
    Let $U \subseteq P$ be an open subscheme and consider the induced morphism of algebraic frames 
    \begin{center}
        \begin{tikzcd}
            X_k \cap U_k \ar[r,hook,open] \ar[d, hook, open] & Y \cap U \ar[r,hook,closed] \ar[d,hook,open] & U \ar[d, hook, open]\\
            X_k \ar[r, hook, open] & Y \ar[r,hook, closed] & P.
        \end{tikzcd}
    \end{center}
    Then there is a canonical isomorphism 
    \[
    I_{X,Y,P}|_{\tube{Y \cap U}{U}} \cong I_{X \cap U, Y \cap U, U}.
    \]
\end{lemma}

\begin{proof}
    By the functoriality of the specialization map, we have an identification 
    \[
    \tube{Y \cap U}{U} = \tube{Y}{P} \cap \widehat{U}_K
    \]
    where we implicitly identify everything with its image in $\widehat{P}_K$.
    We also have an isomorphism $\widehat{Y \cap U}_K = \widehat{Y}_K \cap \widehat{U}_K$. 
    It follows that the ideal $I_{X \cap U, Y \cap U}$ is the ideal defining the bottom closed immersion in the restriction 
    \begin{center}
        \begin{tikzcd}
            \widehat{Y}_K \ar[r,hook,closed] & \tube{Y}{P}\\
            \widehat{Y}_K \cap \widehat{U}_K \ar[r, hook, closed] \ar[u,hook,open]& \tube{Y}{P} \cap \widehat{U}_K\ar[u,hook, open].
        \end{tikzcd}
    \end{center}
    But the formation of ideals of definition commutes with restriction, which immediately gives the claim.  
\end{proof}

By an overconvergent $\O_{\tube{Y}{P}}$-module we mean an $\jdag_{X}\O_{\tube{Y}{P}}$-module (c.f. \cite[Proposition 5.3.1]{LeStum2007}). 
In particular, it has nothing to do with the separate concept of the overconvergence of a connection.

Like rigid cohomology, $Ru_{K*}\op{Fil}^s_{X,Y,P}$ is well-behaved with respect to restrictions to strict neighborhoods:
\begin{lemma}[(c.f. {{\cite[Proposition 6.2.2]{LeStum2007}}})]\label{gpl_res_strict_nbhd_lemma} 
    Let   
    \begin{center}
        \begin{tikzcd}
            X_k \ar[r,hook] \ar[d] & Y \ar[r, hook] \ar[d] & P \ar[d,"u"]\\
            C_k \ar[r,hook] & D \ar[r,hook] & Q
        \end{tikzcd}
    \end{center}
    be a morphism of algebraic $\O_K$-frames
    % and $E$ a $\jdag_X\O_{\tube{Y}{P}}$-module with an integrable connection over $K$.
    If $W$ is a strict neighborhood of $\tube{C}{Q}$ in $\tube{D}{Q}$, $V$ is a strict neighborhood of $\tube{X}{P}$ in $u_K^{-1}(W) \cap \tube{Y}{P}$, and $u_K: V \to W$ is the map induced by $u$, then
    % $Ru_{\op{rig-fil},s}E$ is overconvergent for all $s \ge 0$ and 
    we isomorphisms 
    \[
    (Ru_{K*}\op{Fil}^s_{X,Y,P})|_W = Ru_{K*}(\op{Fil}^s_{X,Y,P})|_V
    \]
    and
    \[
    (Ru_{K*}(\jdag_X I^{s-\bullet}_{X,Y,P} \Omega^\bullet_{\tube{Y}{P}/\tube{D}{Q}}))|_W \cong Ru_{K*}(\jdag_X I^{s-\bullet}_{X,Y,P}\Omega^\bullet_{\tube{Y}{P}/\tube{D}{Q}})|_V
    \]
    for all $s \ge 0$.
\end{lemma}
\begin{proof}
    We may assume as in \cite[Proposition 5.1.17]{LeStum2007} that $W = \tube{D}{Q}$.
    If instead we let $u_K: \tube{Y}{P} \to \tube{D}{Q}$ be the map induced by $u$, and we let $j: V \hookrightarrow \tube{Y}{P}$ denote the inclusion map, the first isomorphism is reduced to proving that 
    \[
    Ru_{K*}(\jdag_X I_{X,Y,P}^{s-\bullet} \Omega^\bullet_{\tube{Y}{P}})= R(u_K \circ j)_*j^{-1}(\jdag_X I^{s-\bullet}_{X,Y,P}\Omega^\bullet_{\tube{Y}{P}}).
    \]
    To show this, it suffices to show that for any $\jdag_X\O_{\tube{Y}{P}}$-module $E$ we have
    \[
    Rj_*j^{-1}E \cong E
    \]
    and this was proven in \cite[Proposition 5.3.7]{LeStum2007}.
    The second isomorphism is proven in exactly the same way.
    % \textbf{I need to understand the rest of the argument, but it should be the same. In particular, why do we have}
    % \[
    % \jdag_X \Omega^\bullet_{\tube{Y}{P}} \cong Rj_*j^{-1} \jdag_X \Omega^\bullet_{\tube{Y}{P}} 
    % \]
    % \textbf{?}
\end{proof}

We also have an analogue to the fact that rigid cohomology is `local downstairs'.

\begin{lemma}[(c.f. {{\cite[Proposition 6.2.9]{LeStum2007}}})]\label{gpl_local_downstairs}
    Let
    \begin{center}
        \begin{tikzcd}
            X_k \ar[r,hook] \ar[d] & Y \ar[r, hook] \ar[d] & P \ar[d,"u"]\\
            C_k \ar[r,hook] & D \ar[r,hook] & Q
        \end{tikzcd}
    \end{center}
    be a morphism of algebraic $\O_K$-frames,
    \begin{center}
        \begin{tikzcd}
            C'_k \ar[r,hook] \ar[d,hook] & D' \ar[r, hook] \ar[d,hook] & Q' \ar[d,hook]\\
            C_k \ar[r,hook] & D \ar[r,hook] & Q
        \end{tikzcd}
    \end{center}
    be an open or mixed immersion of frames which is cartesian, and
    \begin{center}
        \begin{tikzcd}
            X'_k \ar[r,hook] \ar[d] & Y' \ar[r, hook] \ar[d] & P' \ar[d,"u'"]\\
            C'_k \ar[r,hook] & D' \ar[r,hook] & Q'
        \end{tikzcd}
    \end{center}
    be the pullback of the first morphism of frames along this immersion.
    
    Then we have isomorphisms
    \[
    (Ru_{K*}\op{Fil}^s_{X,Y,P})|_{\tube{D'}{Q'}} \cong Ru'_{K*}\op{Fil}^s_{X',Y',P'}
    \]
    % \[
    % \op{Fil}^s_{X,Y,P}|_{\tube{D'}{Q'}} \cong Ru'_{K*}(\jdag_{X'}I^{s-\bullet}_{X',Y',P'}\Omega^\bullet_{\tube{Y'}{P'}})  
    % \]
    % \[
        % (Ru_{\op{rig-fil},s}E)|_{\tube{D'}{Q'}} \cong Ru_{\op{rig-fil}, s}E|_{\tube{Y'}{P'}}
    % \]
    and 
    \[
        (Ru_{K*}(\jdag_X I^{s-\bullet}_{X,Y,P}\Omega^\bullet_{\tube{Y}{P}/\tube{D}{Q}}))|_{\tube{D'}{Q'}} \cong Ru'_{K*}(\jdag_{X'} I_{X',Y',P'}^{s-\bullet} \Omega^\bullet_{\tube{Y'}{P'}/\tube{D'}{Q'}})
    \]
    for all $s \ge 0$.
\end{lemma}

\begin{proof}
%    \textbf{Review this proof, do I really get it? And change all the de Rham complexes from relative to absolute, and track what changes!} 
   As noted before, the morphism of frames associated to a cartesian morphism of algebraic frames is cartesian, since base change commutes with pullbacks.
   Similarly, the morphism of frames associated to an open (resp. mixed) immersion of algebraic frames is open (resp. mixed).
   Therefore we may simply follow the proof of \cite[Proposition 6.2.9]{LeStum2007} and similarly exploit the fact that higher direct images commute with open immersions to obtain 
   \begin{align*}
   (Ru_{K*}\op{Fil}^s_{X,Y,P})|_{\tube{D'}{Q'}} &= (Ru_{K*}(\jdag_X I^{s-\bullet}_{X,Y,P}\Omega^\bullet_{\tube{Y}{P}}))|_{\tube{D'}{Q'}}\\
   &\cong Ru'_{K*}(\jdag_X I^{s-\bullet}_{X,Y,P}\Omega^\bullet_{\tube{Y}{P}})|_{\tube{Y'}{P'}}\\
   &\cong Ru'_{K*}(\jdag_{X'} I_{X,Y,P}^{s-\bullet}|_{\tube{Y'}{P'}} \Omega^\bullet_{\tube{Y}{P}}|_{\tube{Y'}{P'}})\\
   &\cong Ru'_{K*}(\jdag_{X'} I_{X',Y',P'}^{s-\bullet}\Omega^\bullet_{\tube{Y'}{P'}})\\
   &= Ru'_{K*}\op{Fil}^s_{X',Y',P'}
   \end{align*}
    where the third line follows from the fact that restriction commutes with tensor products, and the last line follows from Lemma \ref{gpl_ideal_restriction}.
   The restriction of the relative de Rham complex is similarly computed as
   \begin{align*}
   (Ru_{K*}(\jdag_X I^{s-\bullet}_{X,Y,P}\Omega^\bullet_{\tube{Y}{P}/\tube{D}{Q}}))|_{\tube{D'}{Q'}} &\cong Ru'_{K*}(\jdag_X I^{s-\bullet}_{X,Y,P}\Omega^\bullet_{\tube{Y}{P}/\tube{D}{Q}})|_{\tube{Y'}{P'}}\\
%    &\cong Ru'_{K*}(E|_{\tube{Y'}{P'}} \otimes I_{X,Y,P}^{s-\bullet}|_{\tube{Y'}{P'}} \Omega^\bullet_{\tube{Y}{P}/\tube{D}{Q}}|_{\tube{Y'}{P'}})\\
   &\cong Ru'_{K*}(\jdag_{X'} I_{X',Y',P'}^{s-\bullet}\Omega^\bullet_{\tube{Y'}{P'}/\tube{D'}{Q'}})
   \end{align*}
   using the fact that $\Omega^\bullet_{\tube{Y}{P}/\tube{D}{Q}}|_{\tube{Y'}{P'}} \cong \Omega^\bullet_{\tube{Y'}{P'}/\tube{D'}{Q'}}$.
\end{proof}

In particular,
\begin{corollary}\label{gros_lemma_is_local}
    Theorem \ref{gros_independence} is local on $X_k$ and $P$.
\end{corollary}

\begin{proof}
    The theorem is local on $X_k$ by \cite[Proposition 5.2.8]{LeStum2007}, and Lemma \ref{gpl_local_downstairs} says precisely that we can localize the Gros filtration on the base.
\end{proof}

% Finally, in the reduction we will need the flexibility of an \emph{a priori} stronger but in fact equivalent version of Theorem \ref{gros_independence}:
% \begin{lemma}\label{gros_proposition_locally_free_equivalent}
%     Suppose 
%     \begin{center}
%         \begin{tikzcd}
%         & Y' \ar[dd,"g"] \ar[r, closed, hook] & P' \ar[dd, "u"]\\
%         X_k \ar[ur, open, hook] \ar[dr, open, hook]& \\
%         & Y \ar[r, closed, hook]& P
%         \end{tikzcd}
%     \end{center}
%     is a proper smooth morphism of proper smooth algebraic $\O_K$-frames (see Definition \ref{definition_morphisms_of_frames}).
%     Then the following are equivalent:
%     \begin{enumerate}
%         \item The natural base change map 
%         \[
%             \op{Fil}^s_{X,Y,P} \cong Ru_{K*}\op{Fil}^s_{X,Y', P'}   
%         \]
%         is an isomorphism;

%         \item For any (finite) locally free $\O_{\tube{Y}{P}}$-module $\mathcal{M}$, we have 
%         \[
%             \mathcal{M} \otimes \op{Fil}^s_{X,Y,P} \cong Ru_{K*}(u_K^*\mathcal{M} \otimes \op{Fil}^s_{X,Y',P'})    
%         \]
%     \end{enumerate}
% \end{lemma}

% \begin{proof}
%     This is an immediate consequence of the following projection formula in the derived category \cite[\href{https://stacks.math.columbia.edu/tag/0B54}{Lemma 0B54}]{stacks-project}. Let $f: X \to Y$ is a morphism of ringed spaces and let $E \in D(\O_X)$ and $F \in D(\O_Y)$. If $F$ is perfect (see  \cite[\href{https://stacks.math.columbia.edu/tag/08CL}{Section 08CL}]{stacks-project}), then 
%     \[
%     Rf_*E \otimes_{\O_Y}^{\mathbb{L}} F = Rf_*(E \otimes_{\O_X}^{\mathbb{L}} Lf^*F).  
%     \]
%     Note that the functor $u_K^*$ is exact and locally free sheaves are both perfect complexes and their own projective resolutions.
%     Using \cite[Proposition 5.3.2]{LeStum2007} we can infer for any $\mathcal{M}$ that
%     \[
%         Ru_{K*}(u_K^*\mathcal{M} \otimes \op{Fil}^s_{X,Y',P'}) \cong \mathcal{M} \otimes Ru_{K*}\op{Fil}^s_{X,Y',P'}.
%     \]
%     We can thus get (2) by tensoring (1) with $\mathcal{M}$. 
% \end{proof}
We now prove Theorem \ref{gros_independence} given Lemma \hyperref[gpl_proper_etale]{PE} and Theorem \hyperref[gpl]{GFPL}. 

\begin{proof}[Proof of Theorem \ref{gros_independence}]
    By Corollary \ref{gros_lemma_is_local} we can localize on $X_k$ and $P$.
    In particular, we may assume that $X_k$ is quasi-projective. 

    By Chow's lemma \cite[Corollary 5.7.14]{GrusonRaynaud1971} we may blow up a closed subscheme of $Y'$ outside $X_k$ in $P'$ and obtain a diagram
    \begin{center}
        \begin{tikzcd}
            & \widetilde{Y} \ar[r,hook] \ar[d, "f"] & \widetilde{P} \ar[d, "v"]\\
            X_k \ar[ur, hook] \ar[dr, hook] \ar[r,hook] & Y' \ar[r, hook] \ar[d,"g"] & P' \ar[d,"u"]\\
            & Y \ar[r,hook] & P
        \end{tikzcd}
    \end{center}
    where the upper morphism of frames is strict with $v$ a blow-up and where $g \circ f$ is projective.

    Because $g$ is separated (being proper), it follows from  \cite[\href{https://stacks.math.columbia.edu/tag/0C4Q}{Lemma 0C4Q}]{stacks-project} that $f$ is projective.
    Furthermore, since $v$ is a blow-up outside of $X_k$ it follows that $v$ is an isomorphism (and hence smooth) in a neighborhood of $X_k$.
    Thus the morphism of frames corresponding to $v$ and to $u \circ v$ are both projective smooth, and this reduces the theorem to the projective smooth case. 
    Indeed, if the theorem holds in this special case then we can infer isomorphisms
    \[
    \op{Fil}_{X,Y,P}^s \cong R(u \circ v)_{K*}\op{Fil}^s_{X,\widetilde{Y}, \widetilde{P}} \quad \text{ and } \quad \op{Fil}^s_{X,\widetilde{Y}, \widetilde{P}} \cong Rv_{K*}\op{Fil}^s_{X,Y', P'}
    \]
    for their respective frames.
    % \textbf{It's either this or using the fact that a blow-up induces an isomorphism on strict neighborhoods a la \cite[Proposition 3.1.13]{LeStum2007}. In this case we might need to convert our algebraic frame to a usual frame before blowing-up.}
    The result for our original morphism of frames then follows immediately by composing the latter with $Ru_{K*}$.

    Thus we may assume that $g$ is projective.
    Since we can assume that $P$ is affine, there exists an integer $n \ge 0$ such that 
    \begin{center}
        \begin{tikzcd}
        & Y' \ar[dd,"g"] \ar[r, closed, hook] & \P_P^n \ar[dd, "\pi"]\\
        X_k \ar[ur, open, hook] \ar[dr, open, hook]& \\
        & Y \ar[r, closed, hook]& P
        \end{tikzcd}
    \end{center}
    commutes, where $\pi$ is the canonical projection. 
    In this situation, we may further find a closed subscheme $Y'' \subseteq Y'$ and $P'' \subseteq \P^n_P$ such that the composition of frames 
    \begin{center}
        \begin{tikzcd}
            &   Y'' \ar[r] \ar[d, closed, hook, "\iota"] & P'' \ar[d]\\
        X_k \ar[ur, open] \ar[r, open] \ar[dr, open] &  Y' \ar[d, "g"] \ar[r] &   \P_P^n \ar[d, "\pi"]\\
            &   Y   \ar[r,hook, closed] & P
        \end{tikzcd}
    \end{center}
    is proper \'etale.
    Indeed, note that, localizing on $X_k$ if necessary, the closed immersion $X \hookrightarrow \pi^{-1}(X_k) = \P_{X_k}^n$ is a section of the canonical projection which is smooth.
    It follows that there exists a covering of $\P^n_{X_k}$ by open sets $U$ such that $X_k$ is defined in $U$ by a regular sequence $(\tilde{t}_1,\dots,\tilde{t}_d)$, induced by sections $t_1,\dots,t_d \in \Gamma(\P_P^n,\O(m))$.
    We may assume that $U = D^+(s) \cap \pi^{-1}(Y)$ for some $s \in \Gamma(\P_P^n, \O(r))$ for some $r$.
    It then suffices to take $P'' := V(t_1,\dots,t_d)$ and $Y'' := Y' \cap P''$, since by construction $P''$ is the identity (and hence \'etale) in a neighborhood of $X_k$.

    Let us take this embedding $Y'' \xrightarrow{\iota} Y'$ and consider its composition with the original morphism of frames:
    \begin{center}
        \begin{tikzcd}
            & Y'' \ar[d, hook, closed, "\iota"] \ar[dr, hook, closed]\\
            X_k \ar[r, hook, open] \ar[ur, hook, open] \ar[dr, hook, open] & Y' \ar[d, "g"] \ar[r, closed, hook] & P' \ar[d, "u"]\\
            & Y \ar[r, hook, closed] & P 
        \end{tikzcd}
    \end{center}
    The theorem holds for the top morphism of frames by Lemma \ref{gpl_closed_immersion} (to be proven later), so to prove it for the bottom morphism it suffices to prove it for the diagram corresponding to $g \circ \iota$.
    Since $\iota$ is a closed immersion, $g \circ \iota$ remains projective.
    Thus we can assume not only that $g$ is projective, but that there exists a morphism of frames 
    \begin{center}
        \begin{tikzcd}
        & Y' \ar[dd,"g"] \ar[r, closed, hook] & P'' \ar[dd, "w"]\\
        X_k \ar[ur, open, hook] \ar[dr, open, hook]& \\
        & Y \ar[r, closed, hook]& P
        \end{tikzcd}
    \end{center}
    with $w$ \'etale in a neighborhood of $X_k$.
    The theorem holds for this morphism of frames by virtue of Lemma \hyperref[gpl_proper_etale]{PE}.

    Now consider the diagonal embedding $Y' \to P' \times_P P''$ with its respective projections $p$ and $q$.
    These morphisms fit into the commutative diagram 
    \begin{center}
        \begin{tikzcd}
            & & & P' \times_P P'' \ar[dr, swap, "q"] \ar[dl, "p"]\\
        & Y' \ar[r, hook, closed] \ar[rrr, crossing over, bend right=20] \ar[urr, hook, closed] \ar[dd, "g"]&  P' \ar[dd, "v"] & & P'' \ar[ddll, "w"]\\
        X_k \ar[ur,hook,open] \ar[dr, hook, open]\\
        & Y \ar[r,hook, closed] & P
        \end{tikzcd}
    \end{center}
    The projections $p$ and $q$ are both smooth in a neighborhood of $X_k$.
    Thanks to Theorem \hyperref[gpl]{GFPL} and Lemma \ref{gpl_locally_free_sheaf_equivalent}, we thus have 
    \begin{align*}
        \op{Fil}^s_{X,Y',P''} &\cong Rq_{K*}\op{Fil}^s_{X,Y',P' \times P''}\\
        \op{Fil}^s_{X,Y',P'} &\cong Rp_{K*}\op{Fil}^s_{X,Y',P' \times P''}
    \end{align*}
    Since $v \circ p = w \circ q$, we have 
    \begin{align*}
        \op{Fil}^s_{X,Y,P} &\cong Rw_{K*}\op{Fil}^s_{X,Y',P''}\\
        &\cong Rw_{K*}(Rq_{K*}\op{Fil}^s_{X,Y',P' \times P''})\\
        &\cong R(w \circ q)_{K*}(\op{Fil}^s_{X,Y',P' \times P''})\\
        &= R(v \circ p)_{K*}(\op{Fil}^s_{X,Y',P' \times P''})\\
        &\cong Rv_{K*}(Rp_{K*}\op{Fil}^s_{X,Y',P' \times P''})\\
        &\cong Rv_{K*}\op{Fil}^s_{X,Y',P'}
    \end{align*}
    which was what we wanted to show. 
\end{proof}
\section{The Proper \'Etale Case}\label{gpl_section_proper_etale}
    In this section we prove Lemma \hyperref[gpl_proper_etale]{PE}, which validates the isomorphism for proper \'etale morphisms of frames:
\begin{lemma}[(Proper \'Etale (\emph{PE}))]\label{gpl_proper_etale}
    Suppose 
    \begin{center}
        \begin{tikzcd}
        & Y' \ar[dd,"g"] \ar[r, closed, hook] & P' \ar[dd,"u"]\\
        X_k \ar[ur, open, hook] \ar[dr, open, hook]& \\
        & Y \ar[r, closed, hook]& P
        \end{tikzcd}
    \end{center}
    is a proper \'etale morphism of smooth algebraic $\O_K$-frames with $Y$ and $Y'$ reduced. 
    Then
    \[
    \op{Fil}^s_{X,Y,P} \xrightarrow{\sim} Ru_{K*}\op{Fil}^s_{X,Y', P'}.    
    \]
\end{lemma}

\begin{remark}
        The hypothesis that $Y$ and $Y'$ are reduced is necessary for the argument.
        However, the frame $(X_k \subseteq Y_k \subseteq \mathfrak{P})$ used to compute rigid cohomology, which depends only on the tubes of $X_k$ and $Y_k$ in $\mathfrak{P}$, can be assumed to have $X_k$ and $Y_k$ reduced. 
        Indeed, if $Y_k \hookrightarrow \mathfrak{P}$ is a closed immersion of a $k$-scheme $Y_k$ into a formal $\O_K$-scheme $\mathfrak{P}$ then $\tube{Y_k}{\mathfrak{P}} = \tube{Y_k^{\op{red}}}{\mathfrak{P}}$ where $Y_k^{\op{red}}$ is $Y_k$ with the reduced induced closed subscheme structure.
        This is because the tube is insensitive, locally, to the degree of the elements defining $Y$ in $\mathfrak{P}_k$; see \cite[Proposition 2.2.13]{LeStum2007}.
        As such, insofar as we're trying to define a filtration on rigid cohomology, there is no loss of generality in assuming that $Y$ and $Y'$ are reduced. 
\end{remark}

We first prove a special case:
\begin{lemma}\label{gpl_closed_immersion}
    Given a morphism of smooth algebraic $\O_K$-frames 
    \begin{center}
        \begin{tikzcd}
        X_k \ar[r] \ar[dr] & Y' \ar[d,hook, closed,"g"] \ar[r] & P\\
        & Y \ar[ur]
        \end{tikzcd}
    \end{center}
    where $g$ is a closed immersion and $Y$ is reduced, the natural open immersion of tubes $\tube{Y'}{P} \hookrightarrow \tube{Y}{P}$ induces an isomorphism $\jdag_X\O_{\tube{Y'}{P}} \cong \jdag_X \O_{\tube{Y}{P}}|_{\tube{Y'}{P}}$.
    Moreover, it gives an isomorphism of complexes 
    \[
    \op{Fil}^s_{X,Y,P}|_{\tube{Y'}{P}} \xrightarrow{\sim} \op{Fil}^s_{X,Y',P}.
    \]
\end{lemma}

\begin{proof}
    The claim is local on $P$ and $X_k$, so we may suppose that $P$ is affine and $X_k$ is the complement of a hypersurface of $Y_k$ defined by a function $f$ on $Y_k$. 
    Of course since $X_k \subseteq Y'_k$ we have $Y_k = Y'_k \cup (Y_k \setminus X_k) = Y_k' \cup V(f)$.
    Take a lifting $\tilde{f}$ of $f$.
    We then have that
    \[
    \widehat{Y}_K = \widehat{Y}'_K \cup V(\tilde{f})
    \] 
    as a closed analytic subspace of the tube $\tube{Y}{P}$.

    On the other hand, consider the system of admissible open subsets
    \[
        V^\lambda := \tube{Y}{P} \setminus \{|\tilde{f}| < \lambda\}    
    \]
    for $\lambda < 1$, shown to be strict neighborhoods of $\tube{X}{P}$ in $\tube{Y}{P}$ in \cite[Proposition 3.3.1]{LeStum2007}.
    By \cite[Propositions 3.1.9, 3.1.10]{LeStum2007}, we have that $\{V^\lambda \cap \tube{Y'}{P}\}_\lambda$ are strict neighborhoods of $\tube{X}{P}$ in $\tube{Y'}{P}$ as well as in $\tube{Y}{P}$.

    Since $|\tilde{f}| \ge \lambda > 0$ on $V^\lambda$, we have that $V(\tilde{f}) \cap V^\lambda = \varnothing$ and hence 
    \[
    \widehat{Y}_K \cap V^\lambda = \widehat{Y}'_K \cap V^\lambda    
    \]
    for all $\lambda$.
    Intersecting with $\tube{Y'}{P}$, we have 
    \[
    \widehat{Y}_K \cap (V^\lambda \cap \tube{Y'}{P}) = \widehat{Y}_K' \cap   (V^\lambda \cap \tube{Y'}{P}) \hookrightarrow V^\lambda \cap \tube{Y'}{P}.
    \]
    But the ideals of the immersions
    \begin{align*}
        \widehat{Y}_K \cap (V^\lambda \cap \tube{Y'}{P}) &\hookrightarrow V^\lambda \cap \tube{Y'}{P}\\
        \widehat{Y}'_K \cap (V^\lambda \cap \tube{Y'}{P}) &\hookrightarrow V^\lambda \cap \tube{Y'}{P}
    \end{align*}
    are respectively the restrictions of $I_{X,Y,P}$ and $I_{X,Y',P}$ to the strict neighborhood $V^\lambda \cap \tube{Y'}{P}$.
    That these two immersions are identical means that 
    \[
        {I_{X,Y,P}}_{|(V^\lambda \cap \tube{Y'}{P})} = {I_{X,Y',P}}_{|(V^\lambda \cap \tube{Y'}{P})}.
    \]

    By \cite[Proposition 5.1.12]{LeStum2007}, the result follows if we can show that, for any quasi-compact admissible open $W \subseteq \tube{Y'}{P}$, the subsets $W \cap (V^\lambda \cap \tube{Y'}{P})$ are cofinal in the set of all $W \cap V'$ where $V'$ ranges over all strict neighborhoods of $\tube{X}{P}$ in $\tube{Y'}{P}$.  

    To see this, fix such a $V'$.
    Since $g$ is a closed immersion, $\tube{Y'}{P}$ is an admissible open subset of $\tube{Y}{P}$, so $W$ is an admissible open subset of $\tube{Y}{P}$ as well.
    Furthermore, by \cite[Proposition 3.1.10]{LeStum2007}, $V'$ is a strict neighborhood of $\tube{X}{P}$ in $\tube{Y}{P}$.
    So by \cite[Proposition 3.3.2]{LeStum2007} there exists $\lambda < 1$ such that 
    \[
    W \cap V^\lambda \subseteq V'.
    \]
    It follows that
    \[
    (W \cap V^\lambda) \cap \tube{Y'}{P} \subseteq W \cap V'
    \]
    and we're done.
\end{proof}

In addition, we provide for completion a proof of a fact that is well-known:
\begin{fact}\label{fact_overconvergent_restriction}
        Let $(X \subseteq Y \subseteq P)$ be a frame and let $f: \cE_1^\bullet \to \cE_2^\bullet$ be a morphism of complexes of overconvergent sheaves in $D(\tube{Y}{P})$.
        If there exists a strict neighborhood $V$ of $\tube{X}{P}$ in $\tube{Y}{P}$ such that $f|_V$ is a quasi-isomorphism, then $f$ is a quasi-isomorphism. 
    \end{fact}

    \begin{proof}
        The category of overconvergent modules has enough injectives and the construction of injective resolutions is functorial, so there exist injective resolutions $\cE_1^\bullet \xrightarrow{\sim} \cI_1^\bullet$ and $\cE_2^\bullet \xrightarrow{\sim} \cI_2^\bullet$ and a morphism $g: \cI_1^\bullet \to \cI_2^\bullet$ such that
        \begin{center}
            \begin{tikzcd}
                \cE_1^\bullet \ar[r,"f"] \ar[d, "\cong"] & \cE_2^\bullet \ar[d,"\cong"]\\
                \cI_1^\bullet \ar[r, "g"] & \cI_2^\bullet
            \end{tikzcd}
        \end{center}
        commutes.

        If a sheaf $\mathcal{F}$ is overconvergent then it trivially satisfies the universal property of $j^\dagger_X \mathcal{F}$ so $\mathcal{F} = \jdag_X \mathcal{F}$.
        Hence we have $\cI_1^\bullet = \jdag_X \cI_1^\bullet$ and similarly for $\cI_2^\bullet$.
        Furthermore, restriction to $V$ is functorial and we can check that a morphism $\vp$ is an isomorphism in the derived category by checking that $R\Gamma(W,\vp)$ is an isomorphism in the derived category for all affinoid open subsets of $\tube{Y}{P}$.

        Putting this together, let $W$ be an affinoid open, which in particular is a quasi-compact admissible open subset of $\tube{Y}{P}$.
        We have 
        \begin{center}
            \begin{tikzcd}
                R\Gamma(W, \cE_1^\bullet) \ar[r,"R\Gamma(f)"] \ar[d, "\cong"] & R\Gamma(W,\cE_2^\bullet) \ar[d,"\cong"]\\
                \Gamma(W, \cI_1^\bullet) \ar[d, "="] \ar[r, "R\Gamma(g)"] & \Gamma(W, \cI_2^\bullet) \ar[d,"="]\\
                \Gamma(W, \jdag_X \cI_1^\bullet) \ar[r] \ar[d, "="] & \Gamma(W, \jdag_X \cI_2^\bullet) \ar[d,"="] \\
                \underset{V' \subset \tube{Y}{P}}{\varinjlim}\, \Gamma(W \cap V', \cI_1^\bullet) \ar[r] \ar[d,"="]& \underset{V'\subset \tube{Y}{P}}{\varinjlim}\,\Gamma(W \cap V', \cI_2^\bullet) \ar[d,"="]\\
                \underset{V' \subset V}{\varinjlim}\, \Gamma(W \cap V', {\cI_1^\bullet}_{|V}) \ar[r, "\cong"] & \underset{V'\subset V}{\varinjlim}\,\Gamma(W \cap V', {\cI_2^\bullet}_{|V})
            \end{tikzcd}
        \end{center}
        where the last morphism (equal to $R\Gamma(W, g|_V)$) is a quasi-isomorphism since $f|_V$ is a quasi-isomorphism by assumption.
        This gives the result.
    \end{proof}
\begin{proof}[Proof of Lemma {\hyperref[gpl_proper_etale]{PE}}]
        We first reduce Lemma \hyperref[gpl_proper_etale]{PE} to the following: there exist a strict neighborhood $V'$ of $\tube{X}{P'}$ in $\tube{Y'}{P'}$ and a strict neighborhood $V$ of $\tube{X}{P}$ in $\tube{Y}{P}$ such that $u_K$ restricts to a morphism $u_K: V' \to V$ and the vertical morphisms in the diagram 
        \begin{center}
            \begin{tikzcd}
                \widehat{Y}_K' \cap V' \ar[r,hook,closed] \ar[d,"u_K"] & V' \ar[d,"u_K"] \\
                \widehat{Y}_K \cap V \ar[r,hook,closed] & V
            \end{tikzcd}
        \end{center}
        are isomorphisms.
        To see that this is sufficient, note the top (resp. bottom) closed immersion is the restriction to $V'$ (resp. $V$) of the closed immersion
        \begin{align*}
            \widehat{Y}'_K &\hookrightarrow \tube{Y'}{P'}\\
            (\text{resp. } \widehat{Y}_K &\hookrightarrow \tube{Y}{P})
        \end{align*}
        whose ideals of definition is $I_{P'} := I_{X,Y',P'}$ (resp. $I_{P'} := I_{X,Y',P'}$).
        Lemma \ref{gpl_res_strict_nbhd_lemma} tells us that
        \[
            (Ru_{K*}\op{Fil}^s_{X,Y',P'})|_V = Ru_{K*}(\jdag_{P'}I^{s-\bullet}_{P'} \otimes \Omega^\bullet_{\tube{Y'}{P'}})|_{V'}.
        \]
        But $u_K$ is an isomorphism so in particular its own derived functor, and restiction commutes with dagger functors (\cite[Proposition 5.1.5]{LeStum2007}) so
        \[
        (Ru_{K*}\op{Fil}^s_{X,Y',P'})|_V = u_{K*}(\jdag_{P'}(I_{P'}^{s - \bullet}|_{V'}) \otimes \Omega_{V'}^\bullet).
        \]
        But the fact that the above vertical maps are isomorphisms means precisely that $I_P^{s-\bullet}|_V \cong u_{K*}(I_{P'}^{s - \bullet}|_{V'})$, since the closed immersions are the restrictions to $V$ and $V'$ of the closed immersions defining the ideals $I_P$ and $I_{P'}$, respectively.

        This implies that 
        \begin{align*}
        \op{Fil}^s_{X,Y,P}|_V &= \jdag_P(I_P^{s-\bullet}|_V) \otimes \Omega_V^\bullet \\
        &\cong u_{K*}(\jdag_{P'}(I_{P'}^{s - \bullet}|_{V'}) \otimes \Omega_{V'}^\bullet)\\
        &\cong (Ru_{K*}\op{Fil}^s_{X,Y',P'})|_V
        \end{align*}
        and by Fact \ref{fact_overconvergent_restriction} this is enough to conclude the lemma.

        Having reduced Lemma \hyperref[gpl_proper_etale]{PE} to this problem, we now find such strict neighborhoods $V$ and $V'$. 
        Proper maps are in particular closed, so we have a factorization
        \begin{center}
            \begin{tikzcd}
                & Y' \ar[r, hook, closed] \ar[d, two heads, "g"] & P' \ar[d,"u"]\\
            X_k \ar[r, hook, open] \ar[ur, hook, open] \ar[dr, hook, open] & \op{Im}(g) \ar[r, hook, closed] \ar[d, hook, closed] & P\\
                & Y \ar[ur, hook, closed]
            \end{tikzcd}
        \end{center}
        where $\op{Im}(g)$ is the schematic image of $g$. 
        We know from Lemma \ref{gpl_closed_immersion} that 
        \[
        \op{Fil}^\bullet_{X,Y,P} = \op{Fil}^\bullet_{X,\op{Im}(g), P}    
        \]
        so we may replace $Y$ by $\op{Im}(g)$ and thus suppose that $g$ is proper surjective.

        We know from the proof of \cite[Proposition 3.4.12]{LeStum2007} and \cite[Proposition 3.3.11]{LeStum2007} that given a proper \'etale morphism of frames we may find isomorphic strict neighborhoods $V' \xrightarrow{u_K} V$ as follows.
        \cite[Proposition 3.3.11]{LeStum2007} tells us that for a fixed sequence $\eta_n \xrightarrow{<} 1$ we can find $\delta_n \xrightarrow{<} 1$ and $\lambda_n \xrightarrow{<} 1$ such that, for each $n \in \mathbb{N}$, the morphism $u_K$ induces an isomorphism
        \[
            V_n' := u_K^{-1}([Y]_{P\eta_n}) \cap V^{'\lambda_n}_{\eta_n} = [Y']_{P'\delta_n} \cap u_K^{-1}(V^{\lambda_n}_{\eta_n}) \cong V^{\lambda_n}_{\eta_n}
        \]
        and that the induced morphism $u_K: V' := \cup_n V_n' \to \cup_n V^{\lambda_n}_{\eta_n} =: V$ is an isomorphism of strict neighborhoods.
        We show that these strict neighborhoods suffice.

        Since $g: Y' \to Y$ is proper surjective, it follows from base change that $g_K: \widehat{Y}'_K \to \widehat{Y}_K$ is surjective.
        We may identify $\widehat{Y}'_K$ and $\widehat{Y}_K$ with their images in $\tube{Y'}{P'}$ and $\tube{Y}{P}$, respectively, and by commutativity this identification is compatible with $u_K$.
        Hence for each $n$ we have a surjection 
        \[
            u_K: \widehat{Y}'_K \cap u_K^{-1}(V^{\lambda_n}_{\eta_n}) \twoheadrightarrow  \widehat{Y}_K \cap V^{\lambda_n}_{\eta_n}.
        \]
        Since $\widehat{Y}_K' \subset [Y']_{P'\delta}$ for all $\delta$, we have
        \[
        \widehat{Y}'_K \cap [Y']_{P'\delta_n} \cap u_K^{-1}(V^{\lambda_n}_{\eta_n}) = 
        \widehat{Y}'_K \cap u_K^{-1}(V^{\lambda_n}_{\eta_n})
        \]
        so this is identical to the map 
        \[
        u_K: \widehat{Y}'_K \cap [Y']_{P'\delta_n} \cap u_K^{-1}(V^{\lambda_n}_{\eta_n}) \twoheadrightarrow \widehat{Y}_K \cap V^{\lambda_n}_{\eta_n}. 
        \]
        Taking the union over all $n$, we find that 
        \[
        u_K: \widehat{Y}'_K \cap V' \twoheadrightarrow \widehat{Y}_K \cap V    
        \]
        is surjective.

        To show that this is an isomorphism, we first show that it is a closed immersion.
        Identifying $V'$ with $V$ via the isomorphism $u_K$, we have a diagram of rigid-analytic spaces 
        \begin{center}
            \begin{tikzcd}
                \widehat{Y}_K' \cap V'  \ar[dr, hook, closed, "\iota'"] \ar[dd, swap, "u_K"]\\
                & V\\
                \widehat{Y}_K \cap V \ar[ur, swap, hook, closed, "\iota"]
            \end{tikzcd}
        \end{center}
        Let $V = \bigcup V_i$ be an admissible affinoid covering with $V_i \cong \op{Sp}A_i$. 
        Then by \cite[Proposition 9.5.3.2]{BGR}, $\widehat{Y}_K' \cap V' = \bigcup \iota'^{-1}(V_i)$ and $\widehat{Y}_K \cap V = \bigcup \iota^{-1}(V_i)$ are admissible affinoid coverings as well, say $\iota'^{-1}(V_i) \cong \op{Sp}B_i$ and $\iota^{-1}(V_i) \cong \op{Sp}C_i$, and the corresponding homomorphisms $A_i \to B_i$ and $A_i \to C_i$ are surjective.
        Thus the restriction of the above diagram to $V_i$ is a diagram of affinoid spaces, and the corresponding diagram of rings is 
        \begin{center}
            \begin{tikzcd}
                C_i\\
                & A_i \ar[dl, twoheadrightarrow] \ar[ul, twoheadrightarrow]\\
                B_i \ar[uu]
            \end{tikzcd}
        \end{center}
        It follows that each homomorphism $B_i \to C_i$ must also be surjective.
        Thus the covering $\widehat{Y}_K \cap V = \bigcup \iota^{-1}(V_i) \cong \bigcup \op{Sp}B_i$ is an admissible affinoid covering with the property that for each $i$ the preimage $u_K^{-1}(B_i) = \iota'^{-1}(V_i) = \op{Sp}C_i$ is affinoid and the corresponding homomorphism $B_i \to C_i$ is surjective, i.e., $u_K$ is a closed immersion by definition (\cite[\S 9.5.3]{BGR}).

        Being a surjective closed immersion does not guarantee in itself that $u_K$ is an isomorphism.
        It does follow, however, that the ideal corresponding to this closed immersion is locally nilpotent.
        Indeed, locally it is a morphism of the form
        \[
            u_K: \op{Sp}(B/I) \twoheadrightarrow \op{Sp}(B)
        \]
        for $B$ an affinoid $K$-algebra and $I$ an ideal.
        Algebraically this means that every maximal ideal of $B$ contains $I$, so that $I$ is contained in the intersection of all maximal ideals of $B$.
        But all affinoid algebras are Jacobson, so $I$ is nilpotent.

        But we have the additional fact that $Y$ is a reduced scheme - the schematic image of a closed map with reduced source is the topological image endowed with the reduced induced closed subscheme structure.
        It would suffice, then, to show that the rigid-analytic variety $\widehat{Y}_K$ associated to the reduced scheme $Y$ is itself reduced: if $\widehat{Y}_K$ is reduced then the admissible open subset $\widehat{Y}_K \cap V$ is reduced as well, and it would follow that the locally nilpotent ideal sheaf corresponding to the closed immersion $u_K$ is in fact zero.

        To see this, note first that since the question is local on $Y$, and $Y$ is of finite type over $\O_K$, we can assume that $Y$ is affine with an embedding $Y \subseteq \mathbb{A}^d_{\O_K}$.
        The generic fiber $Y_K$ is open in $Y$ and hence reduced, so that $Y_K^{\op{an}}$ is reduced as well
        % [\textbf{do I need to justify this?}]
        But we have 
        \[
        \widehat{Y}_K = Y_K^{\op{an}} \cap B^d(0,1^+)
        \]
        by \cite[Corollary 2.2.12]{LeStum2007} so $\widehat{Y}_K$ must also be reduced, as desired.
    \end{proof}

\section{The Global Filtered Poincar\'e Lemma}\label{gpl_section_gpl}
%     In this situation we know from \cite[Theorem 3.4.12]{LeStum2007} that $u_K$ induces an isomorphism between strict neighborhoods of $\tube{X}{\widetilde{P}}$ in $\tube{\widetilde{Y}}{\widetilde{P}}$ and strict neighborhoods of $\tube{X}{P}$ in $\tube{Y}{P}$. 
    Here we prove the remaining case, which is a variant of the Global Poincar\'e Lemma for rigid cohomology \cite[Lemma 6.5.5]{LeStum2007}:
\begin{theorem}\label{gpl}
    Let
    \begin{center}
        \begin{tikzcd}
            & & P' \ar[dd, "u"]\\
            X_k \ar[r, hook, open] & Y \ar[ur, hook, closed] \ar[dr, hook, closed]\\
            & & P
        \end{tikzcd}
    \end{center}
    be a smooth morphism of smooth algebraic $\O_K$-frames.
    % and let $u_K$ denote the the induced map of tubes $u_K: \tube{Y}{P'} \to \tube{Y}{P}$.
    Then there is a quasi-isomorphism 
    \[
       \jdag_P I_{X,Y,P}^\ell \xrightarrow{\sim} Ru_{K*}\jdag_{P'} (I_{X,Y,P'}^{\ell - \bullet} \otimes_{\O_{\tube{Y}{P'}}} \Omega_{\tube{Y}{P'}/\tube{Y}{P}}^\bullet).
    \]
    Moreover, $\op{Fil}^s_{X,Y,P} \cong Ru_{K*}\op{Fil}^s_{X,Y,P'}$.
    % Let
    % \begin{center}
    %     \begin{tikzcd}
    %         & & P' \ar[dd, "u"]\\
    %         X \ar[r, hook, open] & Y \ar[ur, hook, closed] \ar[dr, hook, closed]\\
    %         & & P
    %     \end{tikzcd}
    % \end{center}
    % be a smooth morphism of frames, and let $u_K$ denote the the induced map of tubes $u_K: \tube{Y}{P'} \to \tube{Y}{P}$.
    % Then there is a quasi-isomorphism 
    % \[
    %    \jdag I_{X,Y,P}^\ell \xrightarrow{\sim} Ru_{K*}\jdag_X (I_{X,Y,P'}^{\ell - \bullet} \otimes_{\O_{\tube{Y}{P'}}} \Omega_{\tube{Y}{P'}/\tube{Y}{P}}^\bullet) =: Ru_{\op{fil-rig, s}}(\jdag_X \O_{\tube{Y}{P'}})
    % \]
    % Moreover, $\op{Fil}^s_{X,Y,P} \cong Ru_{K*}\op{Fil}^s_{X,Y,P'}$.
\end{theorem}

To prove this theorem we will need to localize it not only on $X$ and $P$ as in Corollary \ref{gros_lemma_is_local} but also on $P'$.
The same reduction has been used in \cite[Lemma 6.5.5]{LeStum2007}, but it is not enough to be able to find open affine refinements of $P$ and $P'$ as in \cite[Lemma 2.3.14]{LeStum2007}.
Indeed, to apply \cite[Proposition 6.2.9]{LeStum2007} for example, we need an affine refinement $P = \bigcup P_i$ such that $u^{-1}(P_i)$ is affine, and this is not guaranteed by \cite[Lemma 2.3.14]{LeStum2007}.

\begin{lemma}\label{gpl_local_on_P_and_P'}
   Theorem \hyperref[gpl]{GFPL} is local on $X$, $P$, and $P'$, and we can assume that both $P$ and $P'$ are affine.
\end{lemma}

\begin{proof}
    As before, it is local on $X$ and $P$ by \cite[Proposition 5.2.8]{LeStum2007} and Lemma \ref{gpl_local_downstairs}, respectively. 
    It suffices to prove the following: there exists a collection of affine open subsets $(V_\alpha)_\alpha$ of $P$ such that $\bigcup_\alpha Y \cap V_\alpha = Y$ and, for each $\alpha$, in the pullback of the morphism of frames by the natural inclusion induced by $V_\alpha \hookrightarrow P$
    % \begin{center}
    %     \begin{tikzcd}
    %         & & & U_\alpha \ar[d, hook, open]\\
    %         & & & P' \ar[dd, "u"]\\
    %         & & P' \cap u^{-1}(V_\alpha)\\
    %     & Y \ar[rr, closed] \ar[uurr, closed] & & P\\
    %     Y \cap V_\alpha \ar[ur] \ar[rr] & & V_\alpha \ar[ur]
    %     \end{tikzcd}
    % \end{center}
    \begin{center}
        \begin{tikzcd}
            & P' \cap u^{-1}(V_\alpha) \ar[d, "u"]\\
        Y \cap V_\alpha \ar[r, hook, closed] \ar[ur, hook, closed] & V_\alpha
        \end{tikzcd}
    \end{center}
    there is an affine open $U_\alpha \subseteq P' \cap u^{-1}(V_\alpha)$ admitting a factorization 
    \begin{center}
        \begin{tikzcd}
            & U_\alpha \ar[d, hook, open]\\
            & P' \cap u^{-1}(V_\alpha) \ar[d, "u"]\\
        Y \cap V_\alpha \ar[uur, hook, dotted] \ar[r, hook, closed] \ar[ur, hook, closed] & V_\alpha.
        \end{tikzcd}
    \end{center}
    where the dotted morphism $Y \cap V_\alpha \hookrightarrow U_\alpha$ is a closed immersion. 
    Indeed, Lemma \ref{gpl_local_downstairs} implies that
    % [\textbf{This isn't $Ru_{\op{fil-rig}}$ any more, but the full notation is too long. What to do?}]
    \[
    (Ru_{K*}\jdag_{P'} (I_{X,Y,P'}^{\ell - \bullet} \otimes_{\O_{\tube{Y}{P'}}} \Omega_{\tube{Y}{P'}/\tube{Y}{P}}^\bullet))|_{\tube{Y \cap V_\alpha}{V_\alpha}}
    \]
    coincides with
    \[
    Ru_{K*}\jdag_{P'} (I_{X \cap V_\alpha,Y \cap V_\alpha,U_\alpha}^{\ell - \bullet} \otimes_{\O_{\tube{Y \cap V_\alpha}{U_\alpha}}} \Omega_{\tube{Y \cap V_\alpha}{U_\alpha}/\tube{Y \cap V_\alpha}{V_\alpha}}^\bullet)
    \]
    % \begin{align*}
        % Ru_{K*}\jdag_X (I_{X,Y,P'}^{\ell - \bullet} \otimes_{\O_{\tube{Y}{P'}}} \Omega_{\tube{Y}{P'}/\tube{Y}{P}}^\bullet)|_{\tube{Y \cap V_\alpha}{V_\alpha}} &= Ru_{K*}(\jdag_X (I_{X,Y,P'}^{\ell - \bullet} \otimes_{\O_{\tube{Y}{P'}}} \Omega_{\tube{Y}{P'}/\tube{Y}{P}}^\bullet)|_{\tube{Y \cap V_\alpha}{P' \cap u^{-1}(V_\alpha)}}
        % (Ru_{\op{fil-rig}, s}\O_{\tube{Y}{P'}})|_{\tube{Y \cap V_\alpha}{V_\alpha}} &\cong Ru_{\op{fil-rig},s}\O_{\tube{Y \cap V_\alpha}{P' \cap u^{-1}(V_\alpha)}}\\
    %     % &= Ru_{\op{fil-rig},s}\O_{\tube{Y \cap V_\alpha}{U_\alpha}}
    % \end{align*}
    for each $\alpha$, where the last equality follows from the fact that for a general formal embedding $X \hookrightarrow P$ the tube $\tube{X}{P}$ depends only on an open formal subscheme of $P$ containing $X$.
    Since $\tube{Y}{P} = \bigcup_\alpha \tube{Y \cap V_\alpha}{P} = \bigcup_\alpha \tube{Y \cap V_\alpha}{V_\alpha}$ is an admissible open covering, this shows that we can prove Theorem \hyperref[gpl]{GFPL} by proving it for each morphism of frames
    \begin{center}
        \begin{tikzcd}
            & U_\alpha \ar[d, "u"]\\
            Y \cap V_\alpha \ar[ur, hook, closed] \ar[r, hook, closed] & V_\alpha.
        \end{tikzcd}
    \end{center}

    We now construct such a family $(V_\alpha)$.
    Since we already know that the theorem is local on $P$, let $P = \op{Spec}A$ be affine.
    Closed subschemes of affine schemes are affine, so $Y \cong \op{Spec}(A/I)$ for some ideal $I$.
    To begin, let $P' = \bigcup_i U_i$ be an affine open covering, say $U_i \cong \op{Spec}B_i$ for each $i$, and fix an index $i$.

    Let $\op{Spec}(A/I) \cap U_i = \bigcup_{j} D(\overline{f_{ij}})$ be a covering of the open subscheme $\op{Spec}(A/I) \cap U_i$ of $\op{Spec}(A/I)$ by principal open subsets, where $\overline{f_{ij}} \in A/I$ is the reduction of an element $f_{ij} \in A$.
    Fix an index $j$.
    The pullback of the diagram 
    \begin{center}
        \begin{tikzcd}
            & & U_i \ar[d, hook, open]\\
            & & P' \ar[d, "u"]\\
        \op{Spec}(A/I) \cap U_i \ar[r, hook, open] \ar[uurr, hook, closed] & \op{Spec}(A/I) \ar[ur, hook, closed] \ar[r, hook, closed] & \op{Spec}A
        \end{tikzcd}
    \end{center}
    by the open immersion on the base induced the inclusion $D(f_{ij}) \hookrightarrow P$ is the diagram 
    \begin{center}
        \begin{tikzcd}
            & & U_i \cap u^{-1}(D(f_{ij})) \ar[d, hook, open]\\
            & & P' \cap u^{-1}(D(f_{ij})) \ar[d, "u"]\\
        D(\overline{f_{ij}}) \ar[r, "="] \ar[uurr, hook, closed] & D(\overline{f_{ij}}) \ar[r, hook, closed] \ar[ur, hook, closed] & D(f_{ij}).
        \end{tikzcd}
    \end{center}
    But if we denote by $\vp: A \to B_i$ the morphism of rings corresponding to the composition $v: \op{Spec}B_i \cong U_i \hookrightarrow P' \to \op{Spec}A$, we see that 
    \begin{align*}
      U_i \cap u^{-1}(D(f_{ij})) &= v^{-1}(D(f_{ij})) = D(\vp(f_{ij}))
    \end{align*}
    is affine, and we're done.
\end{proof}

As a technicality, we also need the following lemma:
\begin{lemma}\label{gpl_locally_free_sheaf_equivalent}
    In the setting of Theorem \hyperref[gpl]{GFPL}, suppose the base change map
        \[
            \jdag_P I_{X,Y,P}^\ell \xrightarrow{\sim} Ru_{K*}\jdag_{P'} (I_{X,Y,P'}^{\ell - \bullet} \otimes_{\O_{\tube{Y}{P'}}} \Omega_{\tube{Y}{P'}/\tube{Y}{P}}^\bullet)
        \]
        is an isomorphism.
        Then for any finite locally free $\O_{\tube{Y}{P}}$-module $\mathcal{M}$, the base change map 
        \begin{align*}
               \jdag_P I_{X,Y,P}^\ell \otimes \mathcal{M} &\xrightarrow{\sim} Ru_{K*}\jdag_{P'} (I_{X,Y,P'}^{\ell - \bullet} \otimes_{\O_{\tube{Y}{P'}}}u_K^*\mathcal{M} \otimes_{\O_{\tube{Y}{P'}}} \Omega_{\tube{Y}{P'}/\tube{Y}{P}}^\bullet)
        \end{align*}
        is an isomorphism.
\end{lemma}

\begin{proof}
    % The proof follows the proof of Lemma \ref{gros_proposition_locally_free_equivalent} verbatim. 
    This follows immediately from the following projection formula: let $f: X \to Y$ be a morphism of rigid-analytic spaces, let $\cF^\bullet$ be a complex of $\O_X$-modules with differential operators of order 1, and let $\cE$ be a finite locally free $\O_Y$-module.
    Then there is a functorial isomorphism 
    \[
    Rf_*\blt{\cF} \otimes_{\O_Y} \cE \xrightarrow{\sim} Rf_*(\blt{\cF} \otimes_{\O_X} f^*\cE).
    \]
    
    To see this, note that there is a morphism of spectral sequences
    \begin{center}
        \begin{tikzcd}
        E_1^{p,q} = \cE \otimes_{\O_X} R^pf_*(\cF^q) \ar[r,Rightarrow] \ar[d] & \cE \otimes_{\O_X} R^{p+q}\blt{\cF} \ar[d]\\
        E_1^{p,q} = R^pf_*(f^*\cE \otimes_{\O_Y} \cF^q) \ar[r, Rightarrow] & R^{p+q}f_*(f^*\cE \otimes_{\O_Y} \blt{\cF}).
        \end{tikzcd}
    \end{center}
    where the morphisms are defined as in \cite[\href{https://stacks.math.columbia.edu/tag/01E6}{Section 01E6}]{stacks-project} and \cite[Proposition 5.6]{Hartshorne1966}; here the assumption that $\cE$ is finite locally free is crucial.
    But the morphism at the $E_1$-level is an isomorphism by \cite[\href{https://stacks.math.columbia.edu/tag/01E8}{Lemma 01E8}]{stacks-project}, so it follows that the induced morphism on the limit is an isomorphism as well.
    Since $\cE$ is locally free and hence flat, this implies the claimed projection formula. 
    % \begin{align*}
    %     E_1^{p,q} \  \&Rightarrow
    % \end{align*}
\end{proof}

% \begin{proof}
%     This is an immediate consequence of the following projection formula in the derived category \cite[\href{https://stacks.math.columbia.edu/tag/0B54}{Lemma 0B54}]{stacks-project}. Let $f: X \to Y$ is a morphism of ringed spaces and let $E \in D(\O_X)$ and $K \in D(\O_Y)$. If $K$ is perfect (see  \cite[\href{https://stacks.math.columbia.edu/tag/08CL}{Section 08CL}]{stacks-project}), then 
%     \[
%     Rf_*E \otimes_{\O_Y}^{\mathbb{L}} K = Rf_*(E \otimes^{\mathbb{L}} Lf^*K).  
%     \]
%     Note that the functor $u_K^*$ is exact and locally free sheaves are both perfect complexes and their own projective resolutions.
%     Using \cite[Proposition 5.3.2]{LeStum2007} we can infer that 
%     \begin{align*}
%         Ru_{\op{fil-rig},s}(\jdag_Xu_K^*\mathcal{M}) &= Ru_{K*}\jdag_X (I_{X,Y,P'}^{\ell - \bullet} \otimes_{\O_{\tube{Y}{P'}}}u_K^*\mathcal{M} \otimes_{\O_{\tube{Y}{P'}}} \Omega_{\tube{Y}{P'}/\tube{Y}{P}}^\bullet)\\
%         &\cong Ru_{K*}\jdag_X (I_{X,Y,P'}^{\ell - \bullet} \otimes_{\O_{\tube{Y}{P'}}} \Omega_{\tube{Y}{P'}/\tube{Y}{P}}^\bullet) \otimes \mathcal{M}\\
%         &= Ru_{\op{fil-rig},s}(\jdag_X \O_{\tube{Y}{P'}}) \otimes \mathcal{M}
%     \end{align*}
%     so we get (2) from (1) by tensoring each side with $\mathcal{M}$.
% \end{proof}

For brevity, let $A = \tube{Y}{P'}$, $B = \tube{Y}{P}$, and write $I_A := I_{X,Y,P'}$ and $I_B := I_{X,Y,P}$.
As a prerequisite for proving the Global Filtered Poincar\'e Lemma, we extend the Gauss-Manin construction \cite[pp.82]{LeStum2007} to our context.
Since $u_K$ is smooth, there is a short exact sequence of locally free sheaves 
\[
    0 \to u_K^*\Omega_B^1 \to \Omega_A^1 \to \Omega^1_{A/B} \to 0.
\]
It is described in detail in \cite[pp.82]{LeStum2007} and \cite[\href{https://stacks.math.columbia.edu/tag/0FMK}{Section 0FMK}]{stacks-project} how this naturally provides a filtration by locally free subsheaves 
\[
\op{Fil}^\ell\Omega^r_A := \op{Im}(u_K^*\Omega_B^\ell \otimes \Omega^{r-\ell}_A \to \Omega^r_A)    
\]
with graded parts 
\[
\op{Gr}^\ell \Omega^r_A = \Omega^{r-\ell}_{A/B} \otimes u_K^*\Omega^\ell_B.
\]
Varying over all $r$, we obtain a filtration $\{\op{Fil}^\ell \Omega_A^\bullet\}_{\ell \ge 0}$ on the complex $\Omega^\bullet_A$ with graded parts
\[
\op{Gr}^\ell \Omega^\bullet_A = \Omega_{A/B}^\bullet[-\ell] \otimes u_K^*\Omega^\ell_B.
\]
More generally, if $\cE$ is an $\O_A$-module with an overconvergent integrable connection, then on $M^\bullet = \cE \otimes \Omega^\bullet_A$ we have a filtration 
\[
   \op{Fil}^\ell M^\bullet = \op{Im}(M^\bullet[\ell] \otimes u_K^*\Omega_B^\ell \to M^\bullet)
\]
with graded parts
\[
    \op{Gr}^\ell M^\bullet = (\cE \otimes \Omega^\bullet_{A/B})[-\ell] \otimes u_K^*\Omega_B^\ell.
\]

For any $r,s \ge 0$ the above filtration on $\Omega^r_A$ canonically defines a filtration on the submodule $\jdag_A I_A^{s-r}\Omega^r_A$ in the natural way (detailed in \cite[\href{https://stacks.math.columbia.edu/tag/0120}{Section 0120}]{stacks-project}), with graded parts 
\[
\op{Gr}^\ell(I_A^{s-r}\Omega^r_A) = I_A^{s-r}(\Omega^{r-\ell}_{A/B} \otimes_A u_K^*\Omega_B^\ell),
\]
which induces a filtration on $I_A^{s-\bullet}\Omega_A^r$ with graded parts 
\[
\op{Gr}^\ell(I_A^{s-\bullet}\Omega_A^\bullet) = I_A^{s-\bullet}(\Omega_{A/B}^\bullet[-\ell] \otimes_A u_K^*\Omega_B^\ell).
\]
We can then obtain a filtration on $M^\bullet = \cE \otimes_A I_A^{s-\bullet}\Omega^\bullet_A$ as above; indeed, our situation of an overconvergent module equipped with a submodule of the trivial connection is a special case of a module with an overconvergent integrable connection.
We omit the details for brevity. 

The following connects the Gros filtration to the main quasi-isomorphism of Theorem \hyperref[gpl]{GFPL}:
\begin{lemma}\label{gpl_gros_to_total_complex_lemma}
    For fixed $s \ge 0$, let 
    \[
        C_u^{a,b}(s) := \jdag_A(I_A^{s-a-b} \otimes_A \Omega_{A/B}^a \otimes_A u_K^*\Omega_B^b)
    \]
    defined for $a,b \in \mathbb{Z}^{\ge 0}$, where the vertical (resp. horizontal) differential is induced by that of $\Omega_B^\bullet$ (resp. $\Omega^\bullet_{A/B}$).
    Then there is a natural isomorphism 
    \[
    \op{Tot}(C_u^{\bullet,\bullet}(s)) \cong \op{Fil}^s_{X,Y,P'}    
    \]
\end{lemma}

\begin{proof}
    We prove this by comparing the the natural filtrations on these two complexes. 
    On the former we assign the row filtration 
    \[
    F^i\op{Tot}(C_u^{\bullet,\bullet}(s)) = \op{Tot}C_i^{\bullet,\bullet}    
    \]
    where 
    \[
    C_i^{a,b} = 
    \begin{cases}
        C_u^{a,b} &\text{ if } b \ge i\\
        0 &\text{ if }b < i
    \end{cases}    
    \]
    % [graphically it's $C_u^{i,j}$ with rows 0 through $i-1$ replaced with zeroes.]
    The $\ell^{\op{th}}$ graded part $\op{Gr}^\ell_F(\op{Tot}(C_u^{\bullet,\bullet}(s)))$ is the $\ell^{\op{th}}$-row of $C_u^{\bullet,\bullet}$; more precisely, 
    \begin{align*}
    \op{Gr}^\ell_F(\op{Tot}(C_u^{\bullet,\bullet}(s))) &= C_u^{\bullet - \ell, \ell}\\
    &= \jdag_A(I_A^{s-\bullet} \otimes_A \Omega^{\bullet}_{A/B}[-\ell]\otimes_A u_K^*\Omega_B^\ell).
    \end{align*}

    On the other hand, our discussion of the Gauss-Manin construction says that $\op{Fil}^s_{X,Y,P'}$ has a natural filtration $F$ which also has graded parts 
    \[
        \op{Gr}^\ell \op{Fil}^s_{X,Y,P'} = \jdag_A(I_A^{s-\bullet} \otimes_A \Omega^{\bullet}_{A/B}[-\ell]\otimes_A u_K^*\Omega_B^\ell).
        % \jdag_A(I_A^{s-\bullet} \otimes_A u_K^*\Omega_B^\ell \otimes_A \Omega^\bullet_{A/B}[-\ell])    
    \]
    Thus, since both complexes are bounded and finite, to prove the claim it suffices to find a filtration-preserving morphism of complexes 
    \[
    \op{Tot}(C_u^{\bullet,\bullet}(s)) \to \op{Fil}^s_{X,Y,P'}.
    \]
    For this the natural morphism suffices.
    Namely we choose, for each $i \ge 0$, the map
    \[
        \op{Tot}(C_u^{\bullet,\bullet}(s))^i \twoheadrightarrow \jdag_A(I_A^{s-i} \otimes_A u_K^*\Omega^i_B) \hookrightarrow \jdag_AI_A^{s-i} \otimes \Omega_A^i = (\op{Fil}^s_{X,Y,P'})^i.
    \]
    This map is filtration-preserving since for $\ell \le i$ it restricts to the inclusion 
    \begin{align*}
        F^\ell\op{Tot}(C_u^{\bullet,\bullet}(s))^i &\twoheadrightarrow \jdag_A(I_A^{s-i} \otimes_A u_K^*\Omega^i_B)\\
        &\hookrightarrow \op{Im}(\jdag_A I_A^{s-i} \otimes_A \Omega_A^{i-\ell} \otimes_A u_K^*\Omega_B^\ell  \to \jdag_A I_A^{s-i} \Omega_A^i)\\
        &= F^\ell (\op{Fil}^s_{X,Y,P'})^i.
    \end{align*}
\end{proof}
By virtue of the previous lemma, to prove that $Ru_{K*}\op{Fil}^s_{X,Y,P'} \cong \op{Fil}^s_{X,Y,P}$, it suffices to prove that $Ru_{K*}\op{Tot}(C^{\bullet,\bullet}(s)) \cong \op{Fil}^s_{X,Y,P}$.
To accomplish this we may show that the direct image $Ru_{K*}C^{k,\bullet}(s)$ of each column of the total complex is isomorphic to $(\op{Fil}^s_{X,Y,P})^k$, i.e.,
    \begin{equation}\label{gpl_double_complex_column}
    \jdag_B(I_B^{s-k}\otimes_B \Omega_B^k) \cong Ru_{K*}\jdag_A(I_A^{s-k-\bullet} \otimes u^*\Omega_B^k \otimes \Omega_{A/B}^\bullet).
    \end{equation}
By Lemma \ref{gpl_locally_free_sheaf_equivalent}, it suffices to prove that
    \[
    \jdag_B I_B^{s-k} \cong Ru_{K*}\jdag_A(I_A^{s-k-\bullet} \otimes \Omega^\bullet_{A/B}),
    \]
which justifies our claim in the statement of Theorem \hyperref[gpl]{GFPL}.

In the course of reducing this quasi-isomorphism to an explicitly computable special case, we'll need the following fact:
\begin{lemma}\label{gpl_composition}
    Consider a diagram of smooth morphisms of smooth algebraic $\O_K$-frames 
    \begin{center}
        \begin{tikzcd}
            &   & P' \ar[d,"u"]\\
        X_k \ar[r,hook, open] & Y \ar[ur, hook, closed] \ar[r, hook, closed] \ar[dr, hook, closed] & P \ar[d,"v"]\\
            &   & P''.
        \end{tikzcd}
    \end{center}
    If isomorphism (\ref{gpl_double_complex_column}) holds for the frames corresponding to $v$ and $u$, respectively, then it holds for the frame corresponding to $v \circ u$.
\end{lemma}

\begin{proof}
    Fix $s \ge 0$.
    We keep the notation of the previous proof and further write $C = \tube{Y}{P''}$ and $I_C := I_{X,Y,P''}$ for the ideal corresponding to the algebraic frame $(X \subseteq Y \subseteq P'')$.

    Consider $u$ as a smooth morphism of smooth $C$-frames, via the structure morphisms 
    \begin{center}
        \begin{tikzcd}
            A \ar[rr, "u"] \ar[dr, swap, "v \circ u"] && B \ar[dl, "v"]\\
            & C
        \end{tikzcd}
    \end{center}
    The Gauss-Manin construction with respect to this morphism provides a filtration on $I_A^{s-\bullet}\Omega^\bullet_{A/C}$ with graded parts 
    \[
    \op{Gr}^\ell(I_A^{s-\bullet}\Omega^\bullet_{A/C}) = I_A^{s-\bullet}(\Omega^\bullet_{A/B}[-\ell] \otimes u_K^*\Omega^\ell_{B/C})    
    \]
    for $\ell \ge 0$.

    With this in mind, fix $k \ge s$ and define 
    \[
        K^\bullet := \jdag_A(I_A^{s-k-\bullet}  \otimes \Omega^\bullet_{A/C}\otimes (v \circ u)^*_K\Omega_C^k)
    \]
    which we can rewrite this as 
    \[
    K^\bullet = \jdag_A((v \circ u)_K^*\Omega_C^k) \otimes I_A^{s-k-\bullet}\Omega^\bullet_{A/C}.
    \]
    The Gauss-Manin construction then provides a filtration on $K^\bullet$ with graded parts 
    \begin{align*}
    \op{Gr}^\ell K^\bullet &= \jdag_A ((v \circ u)_K^*\Omega_C^k) \otimes I_A^{s-k-\bullet}(\Omega_{A/B}^{\bullet}[-\ell] \otimes u_K^*\Omega^\ell_{B/C})\\
    &= \jdag_A(I_A^{s-k-\ell-\bullet} \otimes \Omega^\bullet_{A/B} \otimes (v \circ u)_K^*\Omega_C^k )[-\ell] \otimes u_K^*\Omega^\ell_{B/C}.
    \end{align*}
    Applying the functor $Ru_{K*}$ to both sides and using the projection formula, we obtain an isomorphism
    \begin{equation}\label{gros_gauss_manin_RuK}
        Ru_{K*}\op{Gr}^\ell K^\bullet \cong Ru_{K*}\jdag_A(I_A^{s-k-\ell-\bullet}  \otimes \Omega^\bullet_{A/B}\otimes (v \circ u)_K^*\Omega_C^k)[-\ell] \otimes \Omega^\ell_{B/C}.
    \end{equation}
    But by assumption we have an isomorphism 
    \[
    Ru_{K*}\jdag_A(I_A^{s-k-\ell-\bullet} \otimes \Omega^\bullet_{A/B}) \cong \jdag_B I_B^{s-k-\ell} 
    \]
    which provides, in combination with Lemma \ref{gpl_locally_free_sheaf_equivalent}, an isomorphism 
    \begin{align*}
        Ru_{K*}\jdag_A(I_A^{s-k-\ell-\bullet} \otimes \Omega^\bullet_{A/B}\otimes (v \circ u)_K^*\Omega_C^k) &\cong \jdag_B I_B^{s-k-\ell} \otimes v_K^*\Omega_C^k.
    \end{align*}
    Plugging this into Equation (\ref{gros_gauss_manin_RuK}) and using the fact that $Ru_{K*}\op{Gr}^\ell K^\bullet = \op{Gr}^\ell Ru_{K*}K^\bullet$, we finally obtain an isomorphism 
    \[
    \op{Gr}^\ell Ru_{K*}K^\bullet \cong (\jdag_B I_B^{s-k-\ell} \otimes v_K^*\Omega_C^k)[-\ell] \otimes \Omega^\ell_{B/C}.
    \]
    This isomorphism implies that the filtration on $Ru_{K*}K^\bullet$ is canonical in the sense of \cite[pp.79]{FaisceauxPervers}.
    It follows by \cite[Proposition 3.1.6]{FaisceauxPervers} that 
    \[
    Ru_{K*}K^\bullet \cong \jdag_B(I_B^{s-k-\bullet} \otimes \Omega_{B/C}^\bullet \otimes v_K^*\Omega_C^k)
    \]
    and since the desired isomorphism holds for the morphism corresponding to $v$ also by assumption, we see that
    \begin{align*}
        R(v \circ u)_{K*}K^\bullet &= Rv_{K*}(Ru_{K*}K^\bullet)\\
        &= Rv_{K*}(\jdag_B(I_B^{s-k-\bullet} \otimes \Omega_{B/C}^\bullet \otimes v_K^*\Omega_C^k))\\
        &\cong \jdag_C I_C^{s-k} \otimes \Omega_C^k
    \end{align*}
    which was what we wanted to show.
\end{proof}
        


\begin{proof}[Proof of the Global Filtered Poincar\'e Lemma]
    By Corollary \ref{gpl_local_on_P_and_P'} we can assume that $P$ and $P'$ are both affine. 
    Further, since the question is local on $X_k$, we can assume that the complement of $X_k$ in $Y_k$ is a hyperplane.
    The morphism in Lemma \hyperref[gpl]{GFPL} is not a strict Cartesian morphism of frames in general, but note that both morphisms in the composition 
    \[
    Y \to Y \times_P Y \to Y \times_P P'    
    \]
    are closed immersions, the first since $Y \hookrightarrow P$ is closed and hence separated, and second because both $Y \xrightarrow{id} Y$ and $Y \hookrightarrow P'$ are closed immersions.
    This inserts into a morphism of frames 
    \begin{center}
        \begin{tikzcd}
            X_k \ar[r,hook,open] \ar[dr, hook, open] & Y \ar[r,hook, closed] \ar[d,hook, closed]& P'\\
            & Y \times_P P' \ar[ur, hook, closed]
        \end{tikzcd}
    \end{center}
    Lemma \ref{gpl_closed_immersion} says that the theorem is independent of closed immersions, so it is equivalent to prove the theorem for the morphism 
    \begin{center}
        \begin{tikzcd}
            & & P' \ar[dd, "u"]\\
            X_k \ar[r, hook, open] & Y \times_P P' \ar[ur, hook, closed] \ar[dr, hook, closed]\\
            & & P
        \end{tikzcd}
    \end{center}
    In other words, we may even assume that the morphism of algebraic frames is strictly Cartesian. 

    For any $d \ge 1$, let $\mathbb{A}^d_P := \mathbb{A}^d \times_{\Z} P$ denote the affine space of dimension $d$ over $P$.
    By \cite[Proposition 3.3.13]{LeStum2007} we may assume, locally, that there is an \'etale morphism of affine frames 
    \begin{center}
        \begin{tikzcd}
            & & P' \ar[dd,"u'"]\\
            X_k \ar[r,hook, open] & Y \ar[ur, hook, closed] \ar[dr, hook, closed]\\
            & & \mathbb{A}_P^d
        \end{tikzcd}
    \end{center}
    where $Y$ is embedded into $\widehat{\mathbb{A}}^d_P$ using the zero section.
    We thus have a factorization 
    \begin{center}
        \begin{tikzcd}
            &   & P' \ar[d,"u'"] \ar[dd, bend left=50, "u"]\\
        X_k \ar[r,hook, open] & Y \ar[ur, hook, closed] \ar[r, hook, closed] \ar[dr, hook, closed] & \mathbb{A}_P^d \ar[d,"\pi_P"]\\
            &   & P
        \end{tikzcd}
    \end{center}
    As we already have the result for the upper morphism of frames by Lemma \hyperref[gpl_proper_etale]{PE},
    Lemma \ref{gpl_composition} reduces the theorem to the case
    \begin{center}
        \begin{tikzcd}
            & & \mathbb{A}_P^d\ar[dd,"u"]\\
            X_k \ar[r,hook, open] & Y \ar[ur, hook, closed] \ar[dr, hook, closed]\\
            & & P
        \end{tikzcd}
    \end{center}
    with $u$ the canonical projection.
    Chaining together multiple instances of Lemma \ref{gpl_composition}, we may further reduce ourself to the case $d = 1$.
    Since we may prove the result by checking it on $R\Gamma(W,-)$ for all affinoid subsets $W \subseteq \tube{Y}{P}$, it remains to prove the following variation of the Local Poincar\'e Lemma \cite[Lemma 6.5.7]{LeStum2007}.
\end{proof}

\begin{lemma}[(Local Filtered Poincar\'e Lemma)]
    Let $(X_k \subseteq Y \subseteq P)$ be an affine algebraic $\O_K$-frame where the complement of $X_k$ in $Y_k$ is a hypersurface, and let
    \[
        p: \mathbb{A}_P^1 \to P
    \]
    be the projection.
    Consider the morphism of affine frames
    \begin{center}
        \begin{tikzcd}
                &   &  \mathbb{A}^1_P \ar[dd, "p"]\\
            X_k \ar[r, hook, open] & Y \ar[ur, hook, closed] \ar[dr, hook, closed]\\
                &   &   P 
        \end{tikzcd}
    \end{center}
    Let $I_{\mathbb{A}^1}$ denote the ideal corresponding to the frame $(X_k \subseteq Y \subseteq \mathbb{A}^1_P)$ and $I_P$ the ideal corresponding to the frame $(X_k \subseteq Y \subseteq P)$.  
    If $W$ is an affinoid open subset of $\tube{Y}{P}$, there is a canonical isomorphism 
    \[
    \Gamma(W, \jdag_X I_P^\ell) \cong R\Gamma(W \times \mathbb{D}(0,1^{-}), \jdag_X I_{\mathbb{A}^1}^\ell \xrightarrow{\partial/\partial t} \jdag_X I_{\mathbb{A}^1}^{\ell - 1}).
    \]
\end{lemma}

We start by giving an explicit description of $I_{\mathbb{A}^1}^\ell$.

\begin{lemma}\label{lpl_explicit}
    Let $W = \op{Sp}(A) \subseteq \tube{Y}{P}$ be an affinoid open subset and let $0 < \eta < 1$.
    We have 
    \[
    \Gamma(W \times \mathbb{D}(0,\eta^+),I^n_{\mathbb{A}^1}) = \left\{\sum_{i \ge 0} a_it^i \in A\{t\}_{\eta}\,:\, a_i \in \Gamma(W,I_P^{n-i})\right\}
    \]
    for all $n \ge 0$, where
    \[
        A\{t\}_{\eta} = \left\{\sum_{i \ge 0}a_it^i \,:\, a_i \in A, |a_i|\eta^i \to 0\right\}.
    \]
    % Furthermore, the map $I_{\mathbb{A}^1}^n(W) \xrightarrow{\partial/\partial t} I_{\mathbb{A}^1}^{n-1}(W)$ is surjective for all $n \ge 1$.
\end{lemma}
\begin{proof}
    Let $\widehat{P}_K = \op{Sp}(R)$.
    The diagram 
    \begin{center}
        \begin{tikzcd}
            & \tube{Y}{\mathbb{A}^1_P} \ar[dd,"u"]\\
        \widehat{Y}_K \ar[ur, hook, closed, "\iota_{\mathbb{A}^1}"] \ar[dr, hook, closed, swap, "\iota_P"]\\
            & \tube{Y}{P}
        \end{tikzcd}
    \end{center}
    corresponding to the ideals $I_P$ and $I_{\mathbb{A}^1}$ induces on sheaves of sections the diagram 
    \begin{center}
        \begin{tikzcd}
            & \O_{\tube{Y}{\mathbb{A}^1}}(W \times D(0,1^{-})) \cong A \otimes_R R\{t\} \cong A\{t\} \ar[dl, swap, "\iota_{\mathbb{A}^1}^\flat(W)"]\\ 
        \iota_{P*}\O_{\widehat{Y}_K}(W)\\
            & \O_{\tube{Y}{P}}(W) = A \ar[uu,hook] \ar[ul, "\iota_P^\flat(W)"]
        \end{tikzcd}
    \end{center}
    where the vertical arrow is the natural inclusion. Since $Y$ is embedded into $\mathbb{A}^1_P$ by the zero section, $\iota_{\mathbb{A}^1}^\flat(W)$ is the map $t \mapsto 0$.
    By restriction, we get a similar diagram with the uppermost group being replaced by $\O_{\tube{Y}{\mathbb{A}^1}}(W \times D(0,\eta^+)) \cong A\{t\}_{\eta}$; as an abuse of notation we'll use the same notation for any $\eta > 0$.

    Hence if $\Gamma(W,I_P) = \op{ker}(\iota_P^\flat(W))$ is generated as an $A$-module by sections $(h_1,\dots,h_s)$ (such a finite set of generators exists since $I_P$ is coherent), then $\Gamma(W \times D(0,\eta^+), I_{\mathbb{A}^1})$ is generated as an $A\{t\}_{\eta}$-module by $(h_1,\dots,h_s,t)$.

    With this description it is easy to see that 
    \[
    \Gamma(W \times D(0,\eta^+), I_{\mathbb{A}^1}) = \left\{\sum_{i \ge 0}a_it^i \in A\{t\}_\eta \,:\, a_0 \in \Gamma(W, I_P)\right\}
    \]
    which is precisely the lemma when $n = 1$.
    We'll prove the general case by induction.

    For the rest of the proof, we write $I_P^n$ and $I_{\mathbb{A}^1}^n$ for $\Gamma(W,I_P^n)$ and $\Gamma(W \times \mathbb{D}(0,\eta^+), I_{\mathbb{A}^1}^n)$, respectively, with the implicit assumption that we're always working with sections.
    
    Let
    \[
    S^n := \left\{\sum_{i \ge 0} a_it^i \in A\{t\}_{\eta}\,:\, a_i \in I_P^{n-i}\right\}.
    \]
    First note that this is an ideal. Indeed, if $\sum_{j \ge 0}b_jt^j \in A\{t\}_{\eta}$ is arbitrary and $\sum_{i \ge 0}a_it^i \in S^n$ then 
    \[
        \left(\sum_{j \ge 0} b_jt^j\right)\left(\sum_{i \ge 0} a_it^i\right) = \sum_{k \ge 0}\left(\sum_{i + j = k}a_ib_j\right)t^k,
    \]
    but $a_ib_j \in I_P^{n-i} \subseteq I_P^{n-k}$ for all such $i$ and $j$, so this product is in $S^n$.

    As our inductive hypothesis suppose $I^{n-1}_{\mathbb{A}^1} = S^{n-1}$.
\begin{itemize}
        \item $S^n \subseteq I_{\mathbb{A}_P^1}^n$: Let $\sum a_it^i \in S^n$ and view this sum as
        \begin{align*}
            \sum_{i \ge 0}a_it^i &= a_0 + t\sum_{i \ge 0}a_{i+1}t^i
        \end{align*}
        Since $a_{i+1} \in I_P^{n - i - 1}$ for all $i$ by assumption, we have by the inductive hypothesis that $\sum_{i \ge 0}a_{i+1}t^i \in I^{n-1}_{\mathbb{A}_P^1}$, and hence that $t\sum_{i \ge 0}a_{i+1}t^i \in I^n_{\mathbb{A}_P^1}$. Since also
        \[
        a_0 \in I_P^n \subseteq I^n_{\mathbb{A}_P^1}    
        \]
        we have that each summand is in $I^n_{\mathbb{A}_P^1}$, which is all we need.

        \item $I^n_{\mathbb{A}_P^1} \subseteq S^n$: By definition $I^n_{\mathbb{A}_P^1} = I^{n-1}_{\mathbb{A}_P^1}\cdot I_{\mathbb{A}_P^1}$. Applying the inductive hypothesis we see that this ideal consists of sums of elements of the form
        \[
            \left(\sum_{i \ge 0}a_it^i\right) \left(\sum_{j \ge 0}b_jt^j\right)    
        \]
        where $a_i \in I_P^{n - i - 1}$ for all $i$ and $b_0 \in I_P$. We rearrange this as 
        \begin{align*}
            \left(\sum_{i \ge 0}a_it^i\right) \left(\sum_{j \ge 0}b_jt^j\right) &= \left(\sum_{i \ge 0}a_it^i\right)\left(b_0 + t\left(\sum_{j \ge 1}b_jt^{j-1}\right)\right)\\
            &=
            \left(\sum_{i \ge 0}b_0a_it^i\right) + t\left(\sum_{i \ge 0}a_it^i\right)\left(\sum_{j \ge 1}b_jt^{j-1}\right)\\
            &= \left(\sum_{i \ge 0}b_0a_it^i\right) + \left(\sum_{i \ge 1}a_{i-1}t^i\right)\left(\sum_{j \ge 1}b_jt^{j-1}\right)
        \end{align*}
        We have $b_0a_i \in I_P^{n-i}$ for all $i$ so the first summand is in $S^n$. Likewise, $a_{i-1} \in I_P^{n-i}$ for all $i$ so $\sum_{i \ge 1} a_{i-1}t^i \in S^n$, and since $S^n$ is an ideal the second summand is in $S^n$ also.
    \end{itemize}
\end{proof}

\begin{proof}[Proof (of local filtered Poincar\'e lemma)]
    Fix a complement $Z = V(f)$ for $X$ in $Y$.
    Let $\{V^\lambda\}_{\lambda < 1}$ be the usual system of strict neighborhoods
    \[
    V^\lambda = \tube{Y}{P} \setminus \tube{Z}{P\lambda}
    \]
    of $\tube{X}{P}$ in $\tube{Y}{P}$ (see, e.g., \cite[Proposition 3.3.1]{LeStum2007}).
    If $M(\ell) := \Gamma(W, \jdag_P I_P^\ell)$, it follows from \cite[Proposition 5.1.12]{LeStum2007} that 
    \[
        M(\ell) = \varinjlim_\lambda M^\lambda(\ell)
    \]
    where $M^\lambda(\ell) := \Gamma(W \cap V^\lambda, I_P^\ell)$.

    Choose a sequence $\eta_k \xrightarrow{<} 1$ and define 
    \[
        M_k(\ell) := \Gamma(W \times \mathbb{D}(0,\eta_k^+),\jdag_{\mathbb{A}^1} I^\ell_{\mathbb{A}^1}).
    \]
    In order to apply \cite[Proposition 5.1.12]{LeStum2007} to this space of sections, we explicitly describe the $V^\lambda$ construction with respect to the algebraic frame $(X_k \subseteq Y \subseteq \bbA_P^1)$, which we denote by $V^\lambda_{\bbA^1_P}$
    If the complement of $X$ in $Y$ is the hypersurface $V(f)$ then, since $Y$ is defined in $\bbA_P^1$ by the zero section $t \mapsto 0$ by assumption, the complement as a closed subset of $\bbA_P^1$ is given by
    \[
    Z_{\bbA_P^1} = V(f,t) \cap (\bbA_P^1)_k \subseteq \bbA_P^1. 
    \]
    Fix an $\eta_k$.
    Given $\lambda > 0$, we have by \cite[Lemma 3.2.2]{LeStum2007} that
    \[
    V^\lambda_{\bbA_P^1} = (V^\lambda \times \mathbb{D}(0,1^+)) \cup (\tube{Y}{P} \times \mathbb{A}(0,\lambda^+,1^+))
    \]
    where $\bbA(0,\lambda^+,1^+) = \mathbb{D}(0,1^+) \setminus \mathbb{D}(0,\lambda^-)$ is the closed annulus of inner radius $\lambda$ and outer radius 1.

    Since in \cite[Propoition 5.1.12]{LeStum2007} we care only about the limit $\lambda \to 1$, we may suppose that $\lambda > \eta_k$.
    In this case, we have
    \[
    (W \times \mathbb{D}(0,\eta_k^+)) \cap V^\lambda_{\bbA_P^1} = (W \cap V^\lambda) \times \mathbb{D}(0,\eta_k^+)    
    \]
    since $\mathbb{D}(0,\eta_k^+) \cap \bbA(0,\lambda^+,1^+) = \varnothing$.
    Thus by \cite[Proposition 5.1.12]{LeStum2007},
    \[
    % \Gamma(W \times \mathbb{D}(0,\eta_k^+), \jdag_{\bbA^1}I^\ell_{\bbA^1})
    M_k(\ell) \cong \varinjlim_{\lambda} M_k^\lambda(\ell)
    \]
    where
    \[ 
    M_k^\lambda(\ell) := \Gamma((W \cap V^\lambda) \times \mathbb{D}(0,\eta_k^+), I^\ell_{\bbA^1})
    \]

    Lemma \ref{lpl_explicit} does not directly describe $M_k^\lambda(\ell)$ since $W \cap V^\lambda$ is not necessarily affinoid.
    However, $W \cap V^\lambda$ is quasi-Stein in the sense of \cite[pp.\,146]{Chiarellotto1990}.
    Indeed, suppose $Y = V(\overline{g_1},\dots,\overline{g_r}) \cap P_k \subseteq P_k$ and $Y_k \setminus X_k = V(\overline{f}) \cap Y_k \subseteq P_k$.
    If $g_1,\dots,g_r,f \in P$ are liftings of $\overline{g_1},\dots,\overline{g_r},\overline{f}$, respectively, then by \cite[Proposition 3.2.15]{LeStum2007} we have
    \[
    W \cap V^\lambda = W \cap \widetilde{V}^\lambda    
    \]
    where 
    \begin{align*}
    \widetilde{V}^\lambda &:= \{x \in \tube{Y}{P} \,:\, |f(x)| \ge \lambda\}\\
    &= \{x \in P_K \,:\, |g_1(x)|,\dots,|g_r(x)| < 1, |f(x)| \ge \lambda\}
    \end{align*}
    using \cite[Lemma 2.2.10]{LeStum2007} to explicitly describe $\tube{Y}{P}$. 
    This is easily seen to be an increasing limit of Laurent, and hence affinoid, subdomains satisfying the density condition for being quasi-Stein.
    In addition, since the intersection of an affinoid subdomain with a quasi-Stein subspace is again quasi-Stein, we see that $W \cap \widetilde{V}^\lambda = W \cap V^\lambda$ is quasi-Stein, as needed.

    Thus $W \cap V^\lambda$ is also an increasing limit of affinoid subdomains, say 
    \[
    W \cap V^\lambda = \bigcup_n \op{Sp}A^\lambda_n.    
    \]
    We thus have
    \begin{align*}
        \Gamma((W \cap V^\lambda) \times \bbD(0,\eta_k^+), I^\ell_{\bbA^1}) &= \Gamma\left(\bigcup_n (\op{Sp}A_n^\lambda \times \bbD(0,\eta_k^+)), I^\ell_{\bbA^1}\right)\\
        &= \varprojlim_n \Gamma(\op{Sp}A_n^\lambda \times \bbD(0,\eta_k^+), I^\ell_{\bbA^1})
    \end{align*}
    which, using the explicit description from Lemma \ref{lpl_explicit}, can be written as
    \begin{align*}
        \varprojlim_n \left\{\sum_{i \ge 0} a_{n,i}t^i \in A_n^\lambda\{t\}_{\eta}\,:\, a_{n,i} \in \Gamma(\op{Sp}A_n^\lambda,I_P^{n-i})\right\}.
    \end{align*}
    Passing to rings with the inverse limit topology, we can write this as 
    \[
        \left\{\sum_{i \ge 0}a_it^i \in \varprojlim_n A_n^\lambda\{t\}_\eta \,:\, a_n \in \varprojlim_n \Gamma(\op{Sp}A_n^\lambda, I_P^{n-i})\right\}.
    \]
    % \begin{align*}
    %     \Gamma((W \cap V^\lambda) \times \bbD(0,\eta_k^+), I^\ell_{\bbA^1}) &= \Gamma\left(\bigcup_n (\op{Sp}A_n^\lambda \times \bbD(0,\eta_k^+)), I^\ell_{\bbA^1}\right)\\
    %     &= \varprojlim_n \Gamma(\op{Sp}A_n^\lambda \times \bbD(0,\eta_k^+), I^\ell_{\bbA^1})\\
    %     &\cong \varprojlim_n \left\{\sum_{i \ge 0} a_{n,i}t^i \in A_n^\lambda\{t\}_{\eta}\,:\, a_{n,i} \in \Gamma(\op{Sp}A_n^\lambda,I_P^{n-i})\right\}\\
    %     &= \left\{\sum_{i \ge 0}a_it^i \in \varprojlim_n A_n^\lambda\{t\}_\eta \,:\, a_n \in \varprojlim_n \Gamma(\op{Sp}A_n^\lambda, I_P^{n-i})\right\}
    % \end{align*}
    Explicitly, we impose on $\varprojlim_n A_n^\lambda$ the inverse limit topology
    \[
    \varprojlim_n A_n^\lambda\{t\}_\eta = \left\{\sum_{i \ge 0}(a_{n,i})t^i \,:\, |a_{n,i}|_n \eta^i \to 0 \, \forall n  \right\}    
    \]
    where $(a_{n,i}) \in \varprojlim A_n^\lambda$ is a compatible system and $|\cdot|_n$ is the Banach norm on $A_n^\lambda$. 
    This computation allows us to work explicitly with sections of $\Gamma((W \cap V^\lambda) \times \bbD(0,\eta_k^+), I^\ell_{\bbA^1})$ as convergent power series over a ring. 

    Since we assumed that $X_k$ is the complement of a hypersurface in $Y_k$ it follows from \cite[Proposition 5.4.14]{LeStum2007} that the affinoid covering
    \[
    W \times \mathbb{D}(0,1^{-}) = \bigcup_k (W \times \mathbb{D}(0,\eta_k^+))    
    \]
    is acyclic for coherent modules.
    Hence we are in the setting of \cite[Lemma 6.5.10]{LeStum2007}, which tells us that the cohomology 
    \[
    R\Gamma(W \times \mathbb{D}(0,1^{-}), \jdag I_{\mathbb{A}^1}^\ell \xrightarrow{\partial/\partial t} \jdag I_{\mathbb{A}^1}^{\ell - 1})    
    \]
    can be computed as the total complex of the double complex
    \begin{center}
        \begin{tikzcd}
            \prod_k M_k(\ell) \ar[r,"\partial"] \ar[d, "d"]&\prod_k M_k(\ell - 1) \ar[d,"d"]\\
            \prod_k M_k(\ell) \ar[r,"\partial"] &\prod_k M_k(\ell - 1)\\
        \end{tikzcd}
    \end{center}
    where $d(s_k) = (s_{k+1}|_{D(0,\eta_k^+)}  - s_k)$ and $\partial(s_k) = (\partial/\partial t (s_k))$.

    Let $i: M^\lambda(\ell) \hookrightarrow M_k^\lambda(\ell)$ denote the inclusion as the constant coefficient (see Lemma \ref{lpl_explicit}).
    We'll also make use of the integration map 
    \begin{align}
        M_k^\lambda(\ell) &\xrightarrow{I} M_{k-1}^\lambda(\ell + 1)\\
        t^m &\mapsto \frac{t^{m+1}}{m+1}.
    \end{align}
    This is well-defined: $|\frac{a_{i-1}}{m}|\eta^m_{k-1} \to 0$ if $|a_m|\eta_k^m \to 0$ given that $\eta_k \xrightarrow{<} 1$ is strictly increasing, and if $\sum a_mt^m \in M^\lambda_k(\ell)$ then 
    \[
    \sum_{m \ge 1}\frac{a_{m-1}}{m}t^m \in M^\lambda_{k-1}(\ell+1)
    \]
    since $a_{m-1} \in I_P^{\ell+1-m}$ for all $m$.

    The maps $i$ and $I$ are compatible with the transition maps, so taking limits, we can extend $i$ and $I$ to maps $i: M(\ell) \hookrightarrow M_k(\ell)$ and $I: M_k(\ell) \to M_{k-1}(\ell+1)$, respectively, and taking products we obtain maps
    \begin{align}
        i: M(\ell)^{\mathbb{N}} &\hookrightarrow \prod_k M_k(\ell)\\
        I: \prod_k M_k(\ell) &\to \prod_k M_k(\ell+1).
    \end{align}
    Finally, let $\delta: M(\ell) \to M(\ell)^{\mathbb{N}}$ be the diagonal embedding.

    With these notations, the lemma is equivalent to the following claim:
    \begin{claim}
        The sequence 
        \begin{center}
            \footnotesize{
        \[
        0 \to M(\ell) \xrightarrow{i \circ \delta} \prod_k M_k(\ell) \xrightarrow{(d,\partial)} \prod_k M_k(\ell) \oplus \prod_k M_k(\ell-1) \xrightarrow{\partial - d} \prod_k M_k(\ell-1) \to 0
        \]
            }
        \end{center}
        is exact.
    \end{claim}
    \begin{enumerate}
        \item It's immediate that this is a complex and that the first map is injective.
        \item To see that the last map is surjective, note that for all $s = (s_k) \in M_k(\ell-1)$ we have 
        \[
        (s_k)_k \overset{I}{\mapsto} (I(s_{k+1}))_k \overset{\partial}{\mapsto} (s_{k+1})_k.    
        \]
        Hence 
        \[
        (s_k)_k = (s_{k+1})_k - (s_{k+1} - s_k)_k = \partial(I(s_k)) - d(s_k)  
        \]
        that is, $(\partial - d)(I(s), s) = s$.

        \item Next we show exactness in the second term.
        Each $s_k \in M_k(\ell) = \varinjlim_\lambda M_k^\lambda(\ell)$ can be represented as an equivalence class $\overline{(\sum_i a_{\lambda,k,i}t^i)_\lambda} \in \varinjlim_\lambda M_k^\lambda(\ell) = \prod_\lambda M_k^\lambda(\ell)/\sim$ in the usual way.
        It is easy to see that all of our operations are compatible with the equivalence relation, so as an abuse of notation we treat each $s_k$ as a sequence $(\sum_i a_{\lambda,k,i}t^i)_\lambda$.
        % Let $(s_k) \in \prod_k M_k(\ell)$ and suppose that $d(s_k) = \partial(s_k) = 0$.

        Let $(s_k) \in \prod_k M_k(\ell)$ and suppose that $d(s_k) = \partial(s_k) = 0$.
        The fact that $\partial(s_k) = 0$ means that $(\partial/\partial t(\sum_i a_{\lambda,k,i}t^i))_\lambda = 0$ for all $k$. 
        The transition functions are simply restrictions of coefficients of power series so this means, as expected, that all of these power series are constant, i.e. $a_{\lambda,k,i} = 0$ for $i \ge 1$.
        On the other hand, $d(s_k) = 0$ means that, as systems of power series, $s_i = s_j$ for all $i,j \ge 0$.
        Thus 
        \[
            (s_k)_k = (s_0)_k = ((a_{\lambda})_\lambda)_k = (i \circ \delta)((a_\lambda)_\lambda)    
        \]
        as needed.

        \item Finally we show exactness in the third term.
        For brevity, we extend the abuse of notation from the previous step and imagine $(s_k) \in \prod_k M_k(\ell)$ to consist of power series $s_k = \sum a_{k,i}t^i$, with the implicit understanding that, while $s_k$ in equality is represented by a sequence $(\sum_i a_{\lambda,k,i}t^i)_\lambda$, all of our operations are defined componentwise and hence there is no harm in implicitly focusing our calculation on a single component. 

        Let $s = (s_k) = (\sum a_{k,i}t^i) \in \prod_k M_k(\ell)$ and $s' = (s_k') = (\sum b_{k,j}t^j) \in \prod_k M_k(\ell-1)$ with $\partial(s) = d(s')$.
        We're looking for $s'' \in \prod_k M_k(\ell)$ such that $d(s'') = s$ and $\partial(s'') = s'$.
        We have 
        \begin{align*}
            I(s_k')_k &= \left(\sum_{j \ge 1} \frac{b_{j-1, k+1}}{i}t^i\right)_k\\
            \Rightarrow d(I(s_k')) &= \left(\sum_{j \ge 1}\frac{b_{j-1,k+2} - b_{j-1,k+1}}{i}t^i\right)_k\\
            &= I(s_{k+1}' - s_k')_k\\
            &= I(d(s'))\\
            &= I(\partial(s))\\
            &= (s_{k+1} - a_{0,k+1})_k.
        \end{align*} 
        Hence 
        \begin{align*}
            d(I(s') - s + (a_{0,k+1})) &= (s_{k+1} - a_{0,k+1})_k - (s_{k+1} - s_k)_k - (a_{0,k+1})_k\\
            &= s
        \end{align*}
        and 
        \begin{align*}
            \partial (I(s') - s + (a_{0,k+1})_k) &= \partial(I(s') - s)\\
            &= s' + d(s') - \partial(s)\\
            &= s' + d(s') - d(s')\\
            &= s'
        \end{align*}
        (see the computation in (2)), so $s'' = I(s') - s + (a_{0,k+1})_k$ does the trick.
    \end{enumerate}
\end{proof}

{\bf Acknowledgments.} We thank Nicola Mazzari fo some useful suggestions.  Chiarellotto and Nakada were supported by grant MIUR-PRIN2017 ``Geometric, Algebraic, and Analytic Methods in Arithmetic". Nakada was also supported by INdAM grant INdAM-DP-COFUND-2015.

    \bibliographystyle{halpha-abbrv}
    \bibliography{Algebraic_Geometry}
\end{document}
\section{Partial Evaluation}
\label{sec:partial}

Partial Evaluation dates back at least to Lombardi's partial evaluator for Lisp \cite{lombardi1964lisp}.
In this work, we devise a partial evaluator specifically focused on partially evaluating away fexprs that behave like macros that we call \textbf{macro-like operatives}. By partially evaluating and inlining calls to these fexprs in a way analogous to macro expansion, we show a language based on fexprs instead of macros can be approximately as efficient and at least as powerful.
Our partial evaluation algorithm can eliminate all uses of macro-like operatives, where macro-like means the uses are static, and the combiner definition looks something like Listing \ref{code:macrolike}.
\begin{lstlisting}[language=Lisp,caption={A Macro-like Combiner Definition},label=code:macrolike]
(let (helper (lambda (...) <generate code like macro>))
     (vau dynamic_env (parameters) (eval (helper parameters) dynamic_env)))
\end{lstlisting}

In these cases, the combiner takes in its arguments unevaluated (as it is an operative), passes them to a helper function that generates new code based on the passed code parameters, and then immediately evaluates the code generated by the macro-like function.
Any combiner written following this template will be partially evaluated away by our algorithm.
Our partial evaluator and compiler can handle more cases than just this exact "macro-like" template, but we focus on this case.
By showing that it is possible to port anything written as a macro (which will be expanded away) to an analogous operative combiner (that will be partially evaluated away), we establish the practicality of using fexprs instead of macros as the fundamental building blocks of a purely functional Lisp-like language. When creating a new language, one loses neither expressivity nor significant performance by choosing fexprs as the base construct over the combination of functions and macros.
Any use of fexprs that is not partially evaluated away is something that \textit{could not be expressed via macros}, and thus paying a performance penalty in these cases is not onerous.


We implemented an online partial evaluation strategy for \kraken. Based on \cite{10.1145/243439.243447}'s description, offline partial evaluation relies on a Binding Time Analysis to have been done so it can determine which pieces of code are dependent upon dynamic runtime values of the system and which can be computed statically.
On the other hand, online partial evaluation actually executes the program using real values if they are statically known, and symbolic values otherwise.
Computations that use only static values result in more statically known values, whereas the computations involving dynamic values generate residual code to be included in the final program.

In our case, it is impossible to even determine what symbols will be used as variables without performing at least some execution, much less determining if those variables will contain static or dynamic values, so online partial evaluation is the solution for us.
By being an online partial evaluation algorithm, we do not mean it is happening at program run-time rather it runs at compile-time. In other words,
"online" only refers to when the binding analysis is done relative to partial evaluation, not when the partial evaluation itself is done. 

Sophisticated partial evaluation algorithms can handle partially-static data structures, in which only some of the data and structure is known \cite{10.1145/249069.231419}.
A classic example of partially-static data is a Lisp cons cell, one side of which contains a static integer and the other residual code.
Some advanced partial evaluators maintain additional information about values and residual code, such as its type \cite{ruf1993topics}.

The following is the key difficulty in compiling away macro-esque operatives - the call site environment must be reified and passed to the operative, and that reified environment will be a partially-static data structure.
That is, each frame in the environment is either static data, mapping symbols to static values, or it is a static description of dynamic data, containing symbols and the ID of the combiner that created it (by being called).
Either of these types of frames may then chain upwards to a parent frame, which may be of either type as well.
Correctly handling these partially-static-data environments while ensuring that all macro-esque operatives are partially evaluated away while preventing both non-termination and exponential runtime was the key challenge in this work.
A thorny complication is that that a combiner definition may have to be partially evaluated multiple times in environments with different amounts of static data before it has enough information to reduce as far as it should according to our criteria of eliminating all statically known operative calls.
This means if partial evaluation for a form fails, we can't turn it into residual code right away, as it might be evaluated again later.
If this is done naively, it is very easy to incur exponential runtime as at every call first the combiner is evaluated, then the parameters, followed by the combiner again.
Since this compounds at every call inside a combiner (and compounds via environments containing combiners containing environments without memorization or similar), a mitigating technique is needed.
Our solution to this issue is the needed-for-progress-IDs system, which will be explained below.

As bits of the partial evaluation, especially as relating to calls, get complex, we broke down the algorithm into sections to make it easier to understand. In each section, we provide examples along the way, and at the end provide some final examples to show the process. Our partial evaluation roadmap:
\begin{itemize}
    \item \textbf{Section 4.1}: Partial Evaluation Contexts, Marking Syntax, and Unval Relation
    \item \textbf{Section 4.2}: Small Step Semantics
    \item \textbf{Section 4.3}: Helper Relations
    \item \textbf{Section 4.4}: Partial-Eval Versions of Primitives
    \item \textbf{Section 4.5}: Total Effect of Partial Evaluation
    \item \textbf{Section 4.6}: Examples
\end{itemize}

\subsection{Evaluation Contexts, Marking Syntax, and Unval Relation}
For partial evaluation, we added a new active term, \textit{under}, replace \textit{eval} with \textit{peval} (partial eval), and add new values to \textit{combine}, as shown in Fig. \ref{fig:extendedtermsyntax}.
\begin{figure}[h]
\[
  \begin{array}{rccll}
    AT &:=&& \kpval{T}{\kmkd{i_x}{E}}{ES}{FS}\\
       && \alt & \kunder{T}{(T\dots)}{\kmkd{i_x}{E}}{ES}{FS}\\
       && \alt & \kcombine{T}{(T\dots)}{E} & \text{(Active terms)}
  \end{array}
  \]
  \caption{New Terms for Partial Evaluation}
  \label{fig:extendedtermsyntax}
\end{figure}


We track two critical types of information throughout the partial evaluation process: IDs of environments that contain values needed for progress, and a set of the currently evaluating terms.
This allows us to massively prune the execution space and guide the path of partial evaluation.

This bookkeeping information is added via a "mark" pass.
Partial evaluation will operate on this "mark"'d representation, as will the compiler.
The extra information that bookkeeping adds for each type of form is as follows:
\begin{itemize}
    \item \textbf{Combiner}: A unique ID ($i$) that indicates environments created by calling this combiner\\
    Example: $\kmkd{i}{\kcomb{n}{s'}{\kmkd{i'_{f'}}{E'}}{(s\dots)}{Tb}}$
    \item \textbf{Env}: A unique ID ($i_r$ or $i_f$) matching this environment to the combiner whose call created it. The $r$ or $f$ subscript indicating whether the environment is "fake", a static description of the dynamic environment which maps symbols to placeholder values, or "real", a fully static map from symbols to (almost entirely) static values. "Almost entirely" because a "real" environment can map a symbol to a "fake" environment, as happens when partially evaluating a call to an operative that takes in its call site environment. \\
    Example: $\kmkd{i_r}{\kenv{(s \leftarrow V)\dots}{s' \leftarrow \kmkd{i_x}{E}}{\kmkd{i'_{x'}}{E'}}}$\\
    Example: $\kmkd{i_f}{\kenv{(s \leftarrow V)\dots}{s' \leftarrow \kmkd{i_x}{E}}{\kmkd{i'_{x'}}{E'}}}$
    \item \textbf{Symbol}: ID of environment that the symbol will be resolved in, if applicable. True means the symbol has yet to be partially evaluated, so the ID of the environment will resolve to unknown. $\emptyset$ means the symbol is a value instead of a suspended lookup.\\
    Example: $\kmkd{\truev}{s}$\\
    Example: $\kmkd{7}{s}$\\
    Example: $\kmkd{\emptyset}{s}$
    \item \textbf{Array}: Arrays can be marked by one of three options: \textit{val}, \textit{freshCall}, or \textit{attemptedCall}. \textit{val} indicates the array is a value. \textit{freshCall} indicates the array is a call that hasn't been partially evaluated yet.  \textit{attemptedCall} indicates the suspended call was partially evaluated, but couldn't proceed. \textit{attemptedCall} contains two values - the ID of the dynamic calling environment if the called combiner needs it (otherwise $\emptyset$) and/or the (form, environment) pair that caught and prevented infinite recursion.\\
    Example: $\kmkd{\kval}{(V_1~V_2)}$\\
    Example: $\kmkd{\kfresh}{(V_1~V_2)}$\\
    Example: $\kmkd{\katt{\emptyset}{\emptyset}}{(V_1~V_2)}$
\end{itemize}

In our implementation, we store additional data for efficiency, such as all for-progress-IDs found inside an array, but we will define them in Appendix A as stand-alone functions for ease of presentation.
Our implementation essentially pre-computes and stores the results of these functions as additional bookmarking data, reducing exponential lookups to constant factors.
This could also be achieved via memoization.

The mark relation in Fig. \ref{fig:markrelation} only needs to add bookkeeping to the surface syntax, so it is quite simple.
\begin{figure}[h]
\[
  \begin{array}{rcll}
    \kmark{n} &=& n\\
    \kmark{s} &=& \kmkd{\emptyset}{s}\\
    \kmark{(T\dots)} &=& \kmkd{\kval}{(\kmark{T}\dots)}\\
  \end{array}
  \]
  \caption{Mark Relation}
  \label{fig:markrelation}
\end{figure}

We must also split evaluation into two pieces: "unval"-ing and "partial-eval"-ing.
Unvaling takes a value and turns it into a suspended computation. Partial-evaling takes a suspended computation (or a value containing one) and tries to perform at least some evaluation in order to reduce work at runtime.
Now, evaluation is just unvaling composed with partial evaluation.

\begin{figure}[h]
\[
  \begin{array}{rcll}
    \kunval{n} &=& n\\
    \kunval{\kprim{n}{o}} &=& \kprim{n}{o}\\
    \kunval{\kmkd{i''}{\kcomb{n}{s'}{\kmkd{i'_{f'}}{E'}}{(s\dots)}{Tb}}} &=& \kmkd{i''}{\kcomb{n}{s'}{\kmkd{i'_{f'}}{E'}}{(s\dots)}{Tb}}\\
    \kunval{\kmkd{i}{E}} &=& \kmkd{i}{E}\\
    \kunval{\kmkd{\emptyset}{s}} &=& \kmkd{\truev}{s}\\
    \kunval{\kmkd{\kval}{(T_1~T_2\dots~T_n)}} &=& \kmkd{\kfresh}{(\kunval{T_1}~T_2\dots~T_n)}\\
  \end{array}
  \]
  \caption{Unval Relation}
  \label{fig:unvalrelation}
\end{figure}
Unvaling (Fig. \ref{fig:unvalrelation}) a self-evaluating value is just the value itself, as there is no computation to suspend.
The two forms that change when unvaling are symbols (which go from values to suspended lookups of that symbol in an environment) and arrays (which turn into suspended calls).


Contexts in Fig. \ref{fig:pecontexts} are defined like before, but augmented with our new "peval" (partial evaluate), combiner bodies, and multiple positions in the new "under" form.
\begin{figure}[h]
\[
  \begin{array}{rcl}
    \Ctxt &:=& \square \alt \kpval{\Ctxt}{\kmkd{i_x}{E}}{ES}{FS} \alt \kcomb{n}{s'}{\kmkd{i_x}{E}}{(s\dots)}{\Ctxt}\\
    && \alt \kunder{\Ctxt}{(T\dots)}{\kmkd{i_x}{E}}{ES}{FS} \alt \kunder{T}{(\Ctxt~T\dots)}{\kmkd{i_x}{E}}{ES}{FS}\\
    && \alt \kunder{T}{(T~\Ctxt~T\dots)}{\kmkd{i_x}{E}}{ES}{FS} \alt \kunder{T}{(T\dots~\Ctxt)}{\kmkd{i_x}{E}}{ES}{FS}\\
    && \alt \kcombine{\Ctxt}{(T\dots)}{E} \alt \kcombine{T}{(\Ctxt,T\dots)}{E}\\
    && \alt \kcombine{T}{(T\dots,\Ctxt,T\dots)}{E} \alt \kcombine{T}{(T\dots,\Ctxt)}{E}\\
  \end{array}
\]
  \caption{Contexts for Partial Evaluation}
  \label{fig:pecontexts}
\end{figure}

\subsection{Partial-Eval Small-Step Semantics}
In Appendix B we present some simplified pseudocode to give an additional reference and roadmap for the relations presented here.

For simplicity, we split the semantics into two subsections. The first one will talk about partial evaluation of everything but calls. The second one will just be the calls due to their complexity. 

\subsubsection{Partial Evaluation of Non-Calls}: The partial evaluation of non-calls is easier to understand, and it helps to get a feel for the basic semantics of partial evaluation.

\begin{figure}[h]
\[
  \begin{array}{rcl}
    \InCtxt{E}    &\rightarrow& \InCtxt{E'} ~ (\text{if } E \rightarrow E')\\
    \\
    \kpval{x}{\kmkd{i_x}{E}}{ES}{FS}  &\rightarrow& x~~\text{if}~\truev~\notin~\kprog{x}~\land\\
                                                  &&~\kprog{x}~\cap~ES~=~\emptyset~\land\\
                                                  &&~\kiprog{x}~\cap~ES~=~\kiprog{x}\\
                                                  && \textbf{else:}\text{ continue below}\\
                                                  \\
    \kpval{\kmkd{x}{s}}{\kmkd{i_x}{E}}{ES}{FS}  &\rightarrow& lookup(s,E)\\
    \kpval{\kmkd{i'_{x'}}{E'}}{\kmkd{i_{x}}{E}}{ES}{FS}  &\rightarrow& \kmkd{i'_{x'}}{E''}~\textbf{if}~\kmkd{i'_{x''}}{E''} \in ES ~\textbf{else}~\kmkd{i'_{x'}}{E'}\\
    \\

    \kpval{&&\\ \kmkd{i''}{\kcomb{n}{s'}{\kmkd{i'_r}{E'}}{(s\dots)}{Tb}}}{&&\\\kmkd{i_x}{E}}{ES}{FS}  &\rightarrow& \kmkd{i''}{\kcomb{n}{s'}{\kmkd{i'_r}{E'}}{(s\dots)}{Tb}}\\
    \\
    \kpval{&&\\ \kmkd{i''}{\kcomb{n}{s'}{\kmkd{i'_{f'}}{E'}}{(s\dots)}{Tb}}}{&&\\\kmkd{i_x}{E}}{ES}{FS}  &\rightarrow& \text{let}~E''=\kenv{(s \leftarrow \kmkd{i''}{s})\dots}{s' \leftarrow \kmkd{i''}{s'}}{\kmkd{i_x}{E}}~\text{in}\\
    && \kmkd{i''}{\kcomb{n}{s'}{\kmkd{i_x}{E}}{(s\dots)}{\\&&\qquad \kpval{Tb}{\kmkd{i''_f}{E''}}{\{\kmkd{i''_f}{E''}\}\cup ES}{FS}}}\\
  \end{array}
\]
  \caption{Non-Call Partial-Eval Semantics}
  \label{fig:ncpesemantics}
\end{figure}
In Fig. \ref{fig:ncpesemantics}, we see the formal relations for partial evaluation of all forms except calls. 
The components of an actively partially evaluating form $\kpval{x}{\kmkd{i_x}{E}}{ES}{FS}$ is as follows:
\begin{itemize}
    \item The form that needs to be partially evaluated ($x$ in the figure).
    \item The environment to partially evaluate it in ($\kmkd{i_x}{E}$ in the figure)
    \item The "call stack" (a set in this case) of environments for evaluating the forms above this one ($ES$ in the figure)
    \item A set of currently evaluating forms ($FS$ in the figure) used to check for and prevent infinite recursion
\end{itemize}

%We begin with the step-inside-context rule before getting to our first interesting rule, the will-make-progress condition.
%If we cannot make progress, we return the form unchanged.  Otherwise, we fall through the other rules to determine how to proceed. 

Our first relation is the same as from the small-step semantics of the base language - that a larger expression can step if a sub-component of it as indicated by the Context form can step.

The next relation checks to see if partially evaluating the form will make any progress.
There are three conditions which, if true, mean that the form cannot make progress:
\begin{itemize}
    \item the form has already been partially evaluated ($\truev~\notin~\kprog{x}$)
    \item none of the environments the form needs to progress are currently real or in our environment stack ($\kprog{x}~\cap~ES~=~\emptyset$)
    \item the form hadn't previously stopped evaluating to prevent infinite recursion or we are no longer under the form that began the loop ($\kiprog{x}~\cap~ES~=~\kiprog{x}$)
\end{itemize}
If all three of these are true, partially evaluating the form won't have any affect and instead the form can be returned immediately.

On the other hand, if any of these are false the form might be able to make progress and it falls through to the remaining relations.
Note that integers, symbol values, and array values will all already have been returned by the progress-check condition.
We are only left with suspended symbol lookup, environment value, or derived combiner.
For suspended symbol lookup, we just lookup the symbol in the current environment. If this environment is fake, it will return the same symbol marked with the ID of the environment that it would be found in if it were real.
When partially evaluating an environment value we check the current environment\_stack to see if the environment has a newer, real version (identified by ID) - if so we return it, else we return the environment unchanged.


A derived combiner is the most complicated case in this section.
First, we check to see if its static environment is real or not.
If it is, the fifth rule of Fig. \ref{fig:ncpesemantics} (as noted by the r subscript in the static environment $\kmkd{i'_r}{E'}$ in the combiner value), we return it immediately, since this combiner has already been evaluated in a real environment.
As a closure, it captured the environment from its original evaluation during its creation, and since it might have since moved this environment should not be replaced.
On the other hand, a fake static environment (as noted by the f subscript in the static environment $\kmkd{i'_f}{E'}$ in the combiner value) means the closure's initial evaluation has not yet happened and the closure is still at its original defining place.
In this case, we replace the old static environment with the current one, and then create a new fake environment that maps the function's parameters to placeholder symbols marked with the ID of the combiner/fake environment. We further partially evaluate the combiner's body in this fake environment.
The final result is a new derived combiner form that wraps up the new static environment and the partially evaluated body form.

\subsubsection{Partial Evaluation of Calls}
The partial evaluation of calls is more complex due to the high amount of bookkeeping that needs to be maintained. This complexity can be seen in Fig. ~\ref{fig:cpesemantics}.

\begin{figure}[h]
\[
  \begin{array}{rcl}
    \kpval{\kmkd{\kfresh}{(T_1~T_2\dots)}}{\kmkd{i_x}{E}}{ES}{FS}   &\rightarrow& \kpcombine{\\&&\qquad\kpval{T_1}{\kmkd{i_x}{E}}{ES}{FS}}{\\&&\quad(T_2\dots)}{\kmkd{i_x}{E}}{ES}{FS}\\
    \kpval{\kmkd{\katt{x}{y}}{(T_1~T_2\dots)}}{\kmkd{i_x}{E}}{ES}{FS}   &\rightarrow& \kpcombine{\\&&\qquad\kpval{T_1}{\kmkd{i_x}{E}}{ES}{FS}}{\\&&\quad(T_2\dots)}{\kmkd{i_x}{E}}{ES}{FS}\\
    \\
    \kpcombine{\kmkd{x}{s}}{(T_2\dots)}{\kmkd{i_x}{E}}{ES}{FS} &\rightarrow& \kmkd{\katt{\emptyset}{\emptyset}}{(\kmkd{x}{s}~T_2\dots)}\\
    \kpcombine{\kmkd{\katt{x}{y}}{(T_1\dots)}}{&&\\(T_2\dots)}{\kmkd{i_x}{E}}{ES}{FS} &\rightarrow& \kmkd{\katt{\emptyset}{\emptyset}}{\\&&\qquad(\kmkd{\katt{x}{y}}{(T_1\dots)}~T_2\dots)}\\
    \\
    \kpcombine{\kprim{(S~n)}{o}}{(V\dots)}{\kmkd{i_x}{E}}{ES}{FS} &\rightarrow& \kpcombine{\kprim{n}{o}}{\\&&\qquad\kpval{\kunval{\kpval{V}{\kmkd{i_x}{E}}{ES}{FS}}}{\\&&\qquad\qquad\kmkd{i_x}{E}}{ES}{FS}\dots}{\kmkd{i_x}{E}}{ES}{FS}\\
    \\
    \kpcombine{&&\\ \kmkd{i''}{\kcomb{(S~n)}{s'}{\kmkd{i'_{x'}}{E'}}{(s\dots)}{Tb}}}{&&\\ (V\dots)}{\kmkd{i_x}{E}}{ES}{FS} &\rightarrow& \kpcombine{\kmkd{i''}{\kcomb{n}{s'}{\kmkd{i'_{x'}}{E'}}{s}{Tb}}}{\\&&\qquad\kpval{\kunval{\kpval{V}{\kmkd{i_x}{E}}{ES}{FS}}}{\\&&\qquad\qquad\kmkd{i_x}{E}}{ES}{FS}\dots}{\kmkd{i_x}{E}}{ES}{FS}\\
    \\
    \kpcombine{&&\\ \kmkd{i''}{\kcomb{0}{s'}{\kmkd{i'_{x'}}{E'}}{(s\dots)}{Tb}}}{&&\\(V\dots)}{\kmkd{i_x}{E}}{ES}{FS} &\rightarrow&  \textbf{let}~E''=\kenv{(s \leftarrow V)\dots}{s' \leftarrow \kmkd{i_x}{E}}{\kmkd{i'_{x'}}{E'}}~\textbf{in}\\
    &&\textbf{let}~C = \kmkd{i''}{\kcomb{0}{s'}{\kmkd{i'_{x'}}{E'}}{(s\dots)}{Tb}}~\textbf{in}\\
    &&\textbf{let}~F = (C,E'')~\textbf{in}\\
    &&\kmkd{\katt{\emptyset}{F}}{(C~V\dots)}~\textbf{if}~F \in FS\\
    &&\textbf{else}~\kunder{\\
    &&\kpval{Tb}{\kmkd{i''_r}{E''}}{(\kmkd{i''_r}{E''}\cup ES)}{\{F\}\cup FS}}{\\&&(\kmkd{i''}{\kcomb{0}{s'}{\kmkd{i'_{x'}}{E'}}{(s\dots)}{Tb}}~V\dots)}{\\&&\kmkd{i_x}{E}}{ES}{FS}\\
    \\
    \kunder{z}{&&\\(\kmkd{i''}{\kcomb{0}{s'}{\kmkd{i'_{x'}}{E'}}{(s\dots)}{Tb}~V\dots)}}{&&\\\kmkd{i_x}{E}}{ES}{FS} &\rightarrow& \textbf{let}~o = returnOk(z,i')~\textbf{in}\\
    &&dropRV(z,\kmkd{i_x}{E},ES,FS)~\textbf{if}~o=\truev~\textbf{else}\\
    &&\textbf{let}~i_?= i_x~\textbf{if}~s' \neq \emptyset~\textbf{else}~\emptyset~\textbf{in}\\
    &&\kmkd{\katt{i_?}{\emptyset}}{(\kmkd{i''}{\\&&\qquad\kcomb{0}{s'}{\kmkd{i'_{x'}}{E'}}{(s\dots)}{Tb}}V\dots)}\\
  \end{array}
\]
  \caption{Call Partial-Eval Semantics}
  \label{fig:cpesemantics}
\end{figure}
First, we partially evaluate the first item in the array, which will be the combiner, transitioning to a \textit{combine} form to indicate we are in the process of evaluating a call, as seen before in the base language semantics in Section 3.
The next rule handles the case where the thing to be called isn't a combiner but instead a suspended symbol lookup ($\kmkd{x}{s}$). In this case, we simply return the call as-is, marked as attempted, because we can't make progress.
We do the same if the thing to be called is another suspended call ($\kmkd{\katt{x}{y}}{(T_1\dots)}$).
Otherwise, we have a combiner, either primitive or derived, and we extract the wrap\_level from it. We then partial eval, unval, and partial eval again the arguments wrap\_level number of times.
If the combiner is a normal applicative, this would mean 1 time, like function calls in most other languages. The other common option is 0 times when the combiner is an operative, in order to take the code for the parameters in unevaluated. This might be so the operative can behave like a macro, or implement special control flow.
Numbers greater than 1 for wrap\_level are also possible, but rarely useful.
We retain the ability for wrap\_levels greater than 1 for uniformity, but have not used them in practice.

Now, we are ready to execute the actual call due to the parameters being evaluated the proper number of times according to the wrap\_level in the combiner.
The semantics for executing calls to primitive combiners are given as relations in Fig. 9 in Appendix A.
For derived combiners, we create a new inner environment ($E''$) that maps the parameter symbols to the argument values, the special dynamic environment parameter symbol ($s'$) to the current dynamic environment ($\kmkd{i_x}{E}$), and chains up to the indicated static environment ($\kmkd{i'_{x'}}{E'}$) from the combiner value.
We first check to see if this (form, environment) pair is currently executing ($(C,E'') \in FS$), and return early if so to prevent infinite recursion.
If it does return early, the returned call notes the (form, environment) pair that caused it to stop executing in its \textit{atmdCall} form: $\kmkd{\katt{\emptyset}{F}}{(C~V\dots)}$.
Otherwise, the combiner body is evaluated in this new environment, with the call site marked by the \textit{under} marker form.
The \textit{under} form delineates where a currently executing call is. This allows us to check whether the value/code resulting from partially evaluating a call site is legal to return whenever the form stops evaluating.
To do this, it contains both the partial evaluation of the combiner body to execute the call as well as a fallback term.
The fallback term, which is an updated suspended call in this case, is to be used if the call fails.
The fallback term additionally carries the current dynamic environment, environment stack, and executing form set to be used in the next rule.
The result is checked by the \textit{returnOk} auxiliary function to see if this value is safe to return (defined in Appendix A, overview given in Section 4.3). If it is safe, it passes through \textit{dropRV} before being returned (Fig. 8 in Appendix A).

The case where it is not legal to return a value is if there is, or could be, a reference that must be resolved in the inner environment created by this call.
If the result is legal to return, the dropRedundentVeval (or \textit{dropRV}) ensures no useless call to \textit{veval} (a modified \textit{eval} used in the primitive semantics (Fig. 9 in Appendix A)) to wrap any part of the result.
This is important because these unnecessary wrapper calls to \textit{veval} can block partial evaluation progress in some cases.


When the result form is not legal to return, we return a call form with the partially evaluated combiner (with updated wrap\_level) and partially evaluated arguments based on the fallback value in the \textit{under} form.
If the combiner does take in a dynamic environment, we note the current dynamic environment's ID on the suspended call form ($i_?$).



\subsection{Helper Relations}
Next we quickly describe the various helper relations (fully defined in Appendix A) used in our definition of the partial evaluation relations.
\begin{itemize}
    \item \textbf{Needed-For-Progress}: The needed-for-progress relation shows the set of IDs for which at least one needs to correspond to a real environment (static data) with values in order for the form to make progress. $\truev$ means that it can make progress no matter what.
    \item \textbf{Needed-For-Progress-Upper}: An "upper" version of the needed-for-progress relation (in Appendix A) that tracks IDs of environments that are real themselves, but chain upwards to fake environment IDs.
    \item \textbf{Needed-For-Progress-Infinite}: This relation extracts the forms that have previously stopped executing to prevent infinite recursion. This is important when we are not currently executing in the call stack of one of these forms. If we are not, we should keep executing this form, as it has more evaluation to go before it hits another opportunity for infinite recursion. This is part of the mechanism that allows us to use the Y-combinator to implement recursion without either having infinite recursion or un-evaluated recursive calls with finite inputs.
    \item \textbf{Lookup}: As its names signifies, it finds the value associated with a symbol in an environment.
    \item \textbf{returnOk}: It determines if it is legal to return a particular result out of a particular environment. For example, a term can be returned out of a call if the ID of the inner environment/combiner is not present in the term either explicitly or implicitly through a suspended call to a combiner that takes in its dynamic environment.
    \item \textbf{IDin}: It determines if this ID appears in this value without being under a combiner that introduces this ID.
    \item \textbf{takesDE}: It determines if this combiner takes in the dynamic environment.
    \item \textbf{dropRV} / dropRedundentVeval: Our final helper function removes extraneous calls to \textit{veval} that can get in the way of further partial evaluation.
    \textit{veval} is a version of the applicative \textit{eval}, but with both parameters (the term to evaluate and the environment to evaluate it in) already unvaled and partially evaluated.
    It requires special handling (which it receives via the -1 wrap\_level), because its term argument should not be partially evaluated via the normal machinery, which would use the wrong environment (the current dynamic environment, instead of the explicit environment passed to \textit{veval}/\textit{eval}).
Calls to \textit{eval} are common in macro-like operatives, where the normal final call of a macro-like operative is to \textit{eval} the code constructed during the body of the combiner.
This call to \textit{eval} will partially evaluate to a call to \textit{veval}, which will then return successfully to the call site.
In a normal macro-like operative call, this call site's dynamic environment is the explicit environment passed to \textit{veval}, and thus the call to \textit{veval} is extraneous and the term will be inlined directly, completing the partial evaluation dance that fully expands macro-like operative calls.
This removal is the responsibility of \textit{dropRV}.
A "macro-like operative combiner call" example that demonstrates this happening is located below in section 4.6.
\end{itemize}

\subsection{Partial Eval Primitives Small-Step Semantics}
Finally, we come to the (selected) semantics of the primitive combiners.
For space, their formal definition is relegated to Appendix A. In general, they perform the same function as the simpler base primitives, but operate on the marked terms.
The main combiners of interest are \textit{if0} and \textit{vau} which perform additional partial evaluation on their branches and body, respectively.
We have already heard about \textit{eval}'s half-life as \textit{veval} and function in carrying along the proper environment to evaluate a suspended piece of code in until it can be unwrapped by \textit{dropRV} and spliced into its final location.
The formal definition of \textit{eval}, \textit{veval}, and \textit{dropRV} are in Appendix A.
The primitive implementations omitted are the trivial ones - they only evaluate if their parameters are fully evaluated values and they evaluate to the same value that they would under the base semantics.


\subsection{Combined Effect of Partial Evaluation}
The end result of all of this interconnected machinery is the removal of all statically called operatives where the operative was written in a macro-esque style.
%Note that due to the invariants maintained by partial evaluation exactly the calls to derived combiners for which all arguments are values are executed.
Note that the invariants maintained by partial evaluation ensure that derived combiners are only executed whose arguments are all values.
Additionally, the result of a call to a combiner is either a value, a suspended computation where all calls either don't take in the dynamic environment (which would be the combiner's environment that it's being returned from), or an explicit call to veval providing its own environment.
These are the cases that can be returned by a macro-like combiner!
In addition, the partial evaluation process strips redundant calls to veval.
Once the suspended code is returned from the call to the macro-like combiner, its call to veval will become redundant and the inner suspended computation will be inlined into the parent, just like how a macro would be expanded to code spliced into its calling location.

\subsection{Examples}
To illustrate the algorithm, we have some examples of marking, unvaling, and then partial-evaling in stages. The full step-by-step examples are too long for any paper, so these have been somewhat abbreviated.
\subsubsection{Addition Example}
We'll walk through the full partial evaluation steps for $(+~1~2)$ which is 3.
\begin{longtable}{cc}
      $(+~1~2)$ & The initial code\\
      $\kmkd{\kval}{(\kmkd{\emptyset}{+}~1~2)}$ & Marked\\
      $\kmkd{\kfresh}{(\kmkd{\truev}{+}~1~2)}$ & Then unvaled\\
      \\
      $\kpval{\kmkd{\kfresh}{(\kmkd{\truev}{+}~1~2)}}{\kmkd{i_r}{E}}{ES}{FS}$ & We can't show the entire env,\\
      &but for illustration say that E\\
      &maps "+" to the primitive + combiner\\
      \\
      $\kpcombine{\kpeval{\kmkd{\truev}{+}}{\kmkd{i_r}{E}}{ES}{FS}}{(1~2)}{\kmkd{i_r}{E}}{ES}{FS}$ & Begin call, PV combiner\\
      $\kpcombine{\kprim{1}{+}}{(1~2)}{\kmkd{i_r}{E}}{ES}{FS}$ & Lookup replaces the symbol +\\
      & with the primitive \\&combiner\\
      \\
      $\kpcombine{\kprim{0}{+}}{(\kpval{\kunval{1}}{\kmkd{i_r}{E}}{ES}{FS}$ & Unval+PartialEval to evaluate\\
      $\kpval{\kunval{2}}{\kmkd{i_r}{E}}{ES}{FS})}{\kmkd{i_r}{E}}{ES}{FS}$ & parameters, but integers stay the \\&same\\
      \\
      $\kpcombine{\kprim{0}{+}}{(1~2)}{\kmkd{i_r}{E}}{ES}{FS}$ & And then the call\\
      +(1~2) & Primitive does the calculation\\
      3 & result is 3, as expected,\\
        & which is legal to return\\
\end{longtable}
\subsubsection{Constant Combiner Example}
Now that we've done addition, we'll get slightly more complex by introducing the creation of a combiner with a body that can be partially evaluated. We take larger steps because we've gone over the details with the simple addition. In this case, we have $(vau~(x)~(+~1~2~x))$ meaning "1+2+x" for some input x.
We use just a pinch of syntactic sugar to have a 2-argument vau that is equivalent to the 3-argument vau that ignores its special dynamic environment parameter.
\begin{longtable}{cc}
      $(vau~(x)~(+~1~2~x))$ & The initial code\\
      $\kmkd{\kval}{(\kmkd{\emptyset}{vau}~\kmkd{\kval}{(x)}~\kmkd{\kval}{(\kmkd{\emptyset}{+}~1~2~x)})}$ & \text{Parsed and marked syntax}\\
    $\kmkd{\kfresh}{(\kmkd{\truev}{vau}~\kmkd{\kval}{(x)}~\kmkd{\kval}{(\kmkd{\emptyset}{+}~1~2~x)})}$ & \text{Unvaled}\\
      &\\
      $\kpval{\kmkd{\kfresh}{$&\text{We can't show the entire env.}\\$(\kmkd{\truev}{vau}~\kmkd{\kval}{(x)}~\kmkd{\kval}{(\kmkd{\emptyset}{+}~1~2~x)})$&\text{For illustration, E maps "vau"}\\$}}{\kmkd{i_r}{E}}{ES}{FS}$ &\text{to the primitive vau combiner}\\
    \\
      $\kpcombine{\kpeval{\kmkd{\truev}{vau}}{\kmkd{i_r}{E}}{ES}{FS}}{$&\\$(\kmkd{\kval}{(x)}~\kmkd{\kval}{(\kmkd{\emptyset}{+}~1~2~x)}))}{\kmkd{i_r}{E}}{ES}{FS}$ & \text{Begin call, PV combiner}\\
      &\\
      $\kpval{$&\text{The symbol Vau maps to its}\\
      $\kmkd{7}{\kcomb{0}{s'}{\kmkd{i_f}{E}}{(x)}{\kmkd{\kfresh}{(\kmkd{\truev}{+}~1~2~\kmkd{\emptyset}{x})}}}}{$ &\text{combiner value that will now}\\
      $\kmkd{i_r}{E}}{ES}{FS}$ &\text{be partially evaluated}\\
      \\
      $\kmkd{7}{\kcomb{0}{s'}{\kmkd{i_f}{E}}{(x)}{$&\text{Partial evaluating the body}\\$\kpval{\kmkd{\kfresh}{(\kmkd{\truev}{+}~1~2~\kmkd{\emptyset}{x})}}{$& \text{with fake environment. Notice, we}\\$\kmkd{7_f}{\kenv{(x \leftarrow \kmkd{7}{x})}{}{\kmkd{i_r}{E}}}}{ES}{FS}}}$ & \text{are almost back to our first example}\\
      &\\
      $\kmkd{7}{\kcomb{0}{s'}{\kmkd{i_r}{E}}{(x)}{\kmkd{\katt{\emptyset}{\emptyset}}{(\kprim{0}{+}~3~\kmkd{7}{x})}}}$ & \text{We'll fast forward through the}\\
      & \text{process from our first example}\\
\end{longtable}
We moved quickly through the part mostly shared with the previous example. The only difference being the addition of the parameter reference $x$ to the addition.
The only thing to note is that $\kmkd{\emptyset}{x}$ is unvaled to $\kmkd{\truev}{x}$ (not shown) and then was partially evaluated to $\kmkd{7}{x}$ (7 being the ID of the combiner/fake environment). Furthermore, the partial evaluation version of addition had to return a new partially evaluated call, since it could not yet evaluate $x$.
We are left with a partially-evaluated combiner, but haven't yet seen this new technique do anything that previous partial evaluation techniques couldn't.
For that, let's see a high-level view of how a macro-like operative call would be partially evaluated away.
\subsubsection{Macro-like Operative Call Example}
Due to the length of a step-by-step evaluation of this code, we will take larger jumps than before, omitting that which could be inferred from our two previous examples.
\begin{lstlisting}[language=Lisp,caption={},label=code:additionexample]
(let ( (double_parameter (vau de (x) (eval (array + x x) de))) )
     (vau (x) (double_parameter (+ 1 2 x))))
\end{lstlisting}
We'll skip over how "let" works for now (it's an operative combiner too) and focus on the partial evaluation of a macro-like operative part.
This piece of code defines a macro-like operative called "double\_parameter" that takes in a piece of code "x" unevaluated along with the dynamic calling environment "de" and then constructs the code "(+ x x)" as an array and evaluates it in de, the calling environment.
This code is intentionally simplistic and will evaluate its argument twice (though the downsides of doing so are considerably reduced in a purely functional language where evaluating x cannot have side effects).
This should bring to mind the classic macro example from C, "\#define double(x) (x+x)", and indeed we chose it for its familiarity.
Let's start by seeing what the macro-like f-expression itself looks like partially evaluated, and then we'll jump right into its application. This code:
\begin{lstlisting}[language=Lisp,caption={},label=code:additionexample]
(vau de (x) (eval (array + x x) de))
\end{lstlisting}
becomes
\[
      \kmkd{6}{\kcomb{0}{de}{\kmkd{i_r}{E}}{(x)}{\kmkd{\katt{\emptyset}{\emptyset}}{(\kprim{0}{eval}~\kmkd{\katt{\emptyset}{\emptyset}}{(\kprim{0}{array}~\kprim{1}{+}~\kmkd{6}{x}~\kmkd{6}{x})}~\kmkd{6}{de})}}}
\]
in a way very similar to our earlier examples of partially evaluated combiners.
We have a combiner with two suspended calls, nested, with some suspended symbol lookups.\\
Now let's look at its use, when partially evaluating the body:
\begin{lstlisting}[language=Lisp,caption={},label=code:additionexample]
(vau (x) (double_parameter (+ 1 2 x)))
\end{lstlisting}
becomes, skipping forwards to evaluating the body
\begin{longtable}{cc}
      $\kmkd{7}{\kcomb{0}{\emptyset}{\kmkd{i_f}{E}}{(x)}{\kpval{$&\\
      $\kmkd{\kfresh}{(\kmkd{\truev}{double\_parameter}~\kmkd{\kval}{(\kmkd{\emptyset}{+}~1~2~\kmkd{\emptyset}{x})})}}{$&\\
      $\kmkd{i_r}{\kenv{(x \leftarrow \kmkd{7}{x})}{}{\kmkd{i_r}{E}}}}{ES}{FS}}}$ & \text{forwards to call}\\
      \\
      $\kmkd{7}{\kcomb{0}{\emptyset}{\kmkd{i_f}{E}}{(x)}{\kpval{\kmkd{\kfresh}{($
      &\\
      $\kmkd{6}{\kcomb{0}{de}{\kmkd{i_r}{E}}{(x)}{\kmkd{\katt{\emptyset}{\emptyset}}{$
      &\\
      $(\kprim{0}{eval}~\kmkd{\katt{\emptyset}{\emptyset}}{(\kprim{0}{array}~\kprim{1}{+}~\kmkd{6}{x}~\kmkd{6}{x})}~\kmkd{6}{de})}}}$
      &\\
      $\kmkd{\kval}{(\kmkd{\emptyset}{+}~1~2~\kmkd{\emptyset}{x})})}}{$&\\
      $\kmkd{7_r}{\kenv{(x \leftarrow \kmkd{7}{x})}{}{\kmkd{i_r}{E}}}}{ES}{FS}}}$ & \text{substituting in definition}\\
      \\
\end{longtable}
We'll use a symbol for this nested environment, as it is quite unwieldy.
\[
    E'' = \kmkd{6_r}{\kenv{(x \leftarrow \kmkd{\kval}{(\kmkd{\emptyset}{+}~1~2~\kmkd{\emptyset}{x})})}{\kmkd{7_r}{\kenv{(x \leftarrow \kmkd{7}{x})}{}{\kmkd{i_r}{E}}}}{\kmkd{i_r}{E}}}
\]
Let's continue along with the example.
\begin{longtable}{cc}
      \\
      $\kpval{\kmkd{\katt{\emptyset}{\emptyset}}{$&\text{evaluating body}\\
      $(\kprim{0}{eval}~\kmkd{\katt{\emptyset}{\emptyset}}{(\kprim{0}{array}~\kprim{1}{+}~\kmkd{6}{x}~\kmkd{6}{x})}~\kmkd{6}{de})}}{E''}{$
      &\\
      $ES}{FS}$ &\text{with generated env}\\
      \\
      $\kpval{\kmkd{\katt{\emptyset}{\emptyset}}{$&\text{evaluating array call}\\
      $(\kprim{0}{eval}~\kmkd{\kval}{(\kprim{1}{+}~(\kmkd{\emptyset}{+}~1~2~\kmkd{\emptyset}{x})(\kmkd{\emptyset}{+}~1~2~\kmkd{\emptyset}{x}))}$&\\
      $\kmkd{7_f}{\kenv{(x \leftarrow \kmkd{7}{x})}{}{\kmkd{i_r}{E}}})}}{E''}{$
      &\\
      $ES}{FS}$ & \text{and its symbol parameters}\\
      \\
      $\kpcombine{\kmkd{i''}{\kprim{-1}{veval}}}{$
      &\\
      $(\kunval{\kmkd{\kval}{(\kprim{1}{+}~(\kmkd{\emptyset}{+}~1~2~\kmkd{\emptyset}{x})(\kmkd{\emptyset}{+}~1~2~\kmkd{\emptyset}{x}))}}$
      &\\
      $\kmkd{7_f}{\kenv{(x \leftarrow \kmkd{7}{x})}{}{\kmkd{i_r}{E}}})}{$
      &\\
      $E''}{ES}{FS}$ & \text{evaluating eval}\\
 \\
      $\kpcombine{\kmkd{i''}{\kprim{-1}{veval}}}{$
      &\\
      $(\kmkd{\kfresh}{(\kprim{1}{+}~(\kmkd{\emptyset}{+}~1~2~\kmkd{\emptyset}{x})(\kmkd{\emptyset}{+}~1~2~\kmkd{\emptyset}{x}))}$
      &\\
      $\kmkd{7_f}{\kenv{(x \leftarrow \kmkd{7}{x})}{}{\kmkd{i_r}{E}}})}{$
      &\\
      $E''}{ES}{FS}$ & \text{unval}\\
      \\
      $\text{let}~\kmkd{7_f}{E'}~=~\kmkd{7_f}{\kenv{(x \leftarrow \kmkd{7}{x})}{}{\kmkd{i_r}{E}}}~\text{in}$ & \text{we also abbriviate E'}\\
      $\text{let}~V'~=~\kpval{$
      &\\
      $\kmkd{\kfresh}{(\kprim{1}{+}~(\kmkd{\emptyset}{+}~1~2~\kmkd{\emptyset}{x})(\kmkd{\emptyset}{+}~1~2~\kmkd{\emptyset}{x}))}}{$
      &\\
      $\kmkd{7_f}{E'}}{\{\kmkd{7_f}{E'}\}\cup ES}{FS}~\text{in}$ &\\
      $\kunder{V'}{(\kprim{-1}{veval}~V'~\kmkd{7_f}{E'})}{\kmkd{i_x}{E}}{ES}{FS}$ & \text{stepping veval}\\
\end{longtable}
V' will step to:
\[
    \text{let}~V'~=~\kmkd{\katt{\emptyset}{\emptyset}}{(\kprim{0}{+}~\kmkd{\katt{\emptyset}{\emptyset}}{(\kprim{0}{+}~3~\kmkd{7}{x})}~\kmkd{\katt{\emptyset}{\emptyset}}{(\kprim{0}{+}~3~\kmkd{7}{x})})}~\text{in}\\
\]
Now, the under call will fail to complete, as $returnOk$ will return false, finding $7$ in the result.
Thus, the fallback will be returned and will be the final result of the call to $double\_parameter$.
This time, $returnOk$ will be true, as $6$, $double\_parameter$'s ID, is nowhere to be found in the result.
Thus, we will transition to a call to $dropRV$ with the ID $7$.
\[
    dropRV(\kmkd{\katt{\emptyset}{\emptyset}}{(\kprim{-1}{veval}~V'~\kmkd{7_f}{E'})},~\kmkd{7_f}{E'},~ES,~ FS)
\]
Of course, this is a redundant veval, so the final result of partially evaluating the call to \\ $double\_parameter$ is:
\[
    V'
\]
that is,
\[
    \kmkd{\katt{\emptyset}{\emptyset}}{(\kprim{0}{+}~\kmkd{\katt{\emptyset}{\emptyset}}{(\kprim{0}{+}~3~\kmkd{7}{x})}~\kmkd{\katt{\emptyset}{\emptyset}}{(\kprim{0}{+}~3~\kmkd{7}{x})})}
\]

Note that this suspended computation is returned and replaces the call to the macro-like operative call.
This code is the equivalent of:
\begin{lstlisting}[language=Lisp,caption={},label=code:additionexample]
(vau (x) (+ (+ 3 x) (+ 3 x)))
\end{lstlisting}
This is what we would have gotten if we had used a macro and basic constant propagation.

Note that this evaluation of the example did not depend on the exact values, the code generated by the macro-like operative, or the parameters to the call.
In just this way, static calls to macro-like operatives will be partially evaluated away by our algorithm in a way congruent to macro-expansion and then constant propagation in a more standard optimizing Scheme implementation.



\section{Compiler Backend and Optimizations}
\label{sec:opt}

Partial evaluation can eliminate all inefficiencies from macro-like operatives but there are other inefficiencies left that require backend optimizations.
One major remaining inefficiency is dynamic combiner call sites.
In such cases, no local information is available ahead of time to determine if the combiner that will be called is an applicative or an operative - that is, whether the arguments will be evaluated or not and whether the combiner needs to access its calling dynamic environment.
This would normally mean that code in the parameter position of dynamic calls cannot even be partially evaluated.
To overcome this inefficiency, we tag the compiled combiner closure values with bits indicating their wrap level and need of the calling environment.
Each dynamic call site branches on these bits.
Inside the "wrap\_level=0" side of the dynamic branch, a reference to the unevaluated arguments written out in static memory is emitted.
Inside the "wrap\_level=1" side of the dynamic branch, the compiler re-invokes unval and the partial evaluator recursively on each argument, resulting in code that is again as efficient as a language without fexprs (plus the overhead of the dynamic branch).
Note that this requires that both the partial evaluation algorithm above as well as the compilation algorithm be extended to support failure. Failure during partial evaluation does not necessarily mean failure during run-time. For instance, a dynamic combiner might always be an operative with a wrap level of 0, and so some erroring parameter code is actually data, and will never be evaluated (and thus never lead to an error).

Garbage is collected via reference counting for simplicity. Since this is a purely functional language, there are no cycles to worry about.
In order to remove the rest of the inefficiencies after partial evaluation, we have implemented various compiler optimizations.
The following sections cover the other key optimizations implemented in our compiler. 

\subsection{Lazy Environment Instantiation}
We delay the allocation and initialization of dynamic environment values until they are actually needed.  Combiner calls that do take in the dynamic environment check a dedicated register to see if the environment value has already been created. If not, it creates it.
This means the dynamic execution traces of combiner calls where there is no call that takes in the dynamic environment never reifies it and incurs only a single (predictable) branch of overhead.
For static combiner calls, this information (if arguments are evaluated, if it takes in the surrounding environment, etc) is known at compile time, and no runtime branches are generated.

\subsection{Type-Inference-Based Primitive Inlining}
In order to reduce the overhead of every built-in operation being a combiner call with dynamic types, we implemented type-inference guided inlining of primitive operations.
An analysis pass infers types based on branch predicates, which works quite well with the code generated by our match operative.
For instance, the combiner \textit{len} can be inlined to just a few bit-twiddling opcodes by determining that a particular variable must contain an array in \textit{cond}.

For instance, consider the following code:
    \begin{lstlisting}[language=Lisp,caption={Type Inference Example},label=code:typeinfer]
(cond (and (array? a) (= 3 (len a)))    (idx a 2)
      true                              nil)\end{lstlisting}
The call to \textit{idx} can be fully inlined without type or bounds checking because it resides in a block only reachable if the variable 'a' does contain an array of length 3.
No type information is needed to inline type predicates, as they only need to look at the tag bits.
Equality checks can be inlined as a simple word/ptr compare if any of its parameters are of a type that can be word/ptr compared (ints, bools, and symbols).
When type inference and primitive inlining is combined together, it means that every primitive call in most match expressions can be fully inlined into a handful of opcodes apiece.
In the above example, every single primitive listed will be inlined: the \textit{cond} to WebAssembly if blocks, the predicate functions to bit-twiddling and branches, the \text{idx} to bit-twiddling and a load with a constant offset, etc.

\subsection{Immediately-Called Closure Inlining}
Inlining calls to closure values that are allocated and then immediately used helps incur no overhead for implementing some operatives. The main macro-like operative reaping the benefit is "let".
As seen below, Listing \ref{code:letinline} is partially evaluated to the equivalent of Listing \ref{code:letinline2}, then inlined. As a result, the only overhead is the creation of a new environment, which is further made lazy and eliminated in the common case by Lazy Environment Instantiation. In this way, "let" is actually syntactic sugar for the definition and immediate call of a closure, like in many lambda calculi, but no efficiency is lost by doing so.
\begin{lstlisting}[language=Lisp,caption={Let Inlining Example},label=code:letinline]
    (let (a (+ 1 2))
         (+ a 3))
\end{lstlisting}
\begin{lstlisting}[language=Lisp,caption={Let Inlining Example - Expanded},label=code:letinline2]
    ((wrap (vau (a) (+ a 3))) (+ 1 2))
\end{lstlisting}



\subsection{Y-Combinator Elimination}
Continuing the theme of making the classic lambda-calculi implementations of concepts as efficient as standard implementations, the final set of optimizations ensures no overhead from using the Y-Combinator to implement recursion.
In Kraken, the Y-Combinator looks like  Listing \ref{code:ycomb} with a tiny example of its use shown in Listing \ref{code:ycombuse}.
    \begin{lstlisting} [language=Lisp,caption={The Y Combinator, as defined in Kraken},label=code:ycomb]
(let Y (lambda (f)
           ((lambda (x) (x x))
            (lambda (x) (f (wrap (vau app_env (& y) (lapply (x x) y app_env)))))))
)\end{lstlisting}

    \begin{lstlisting}[language=Lisp,caption={A Factorial function explicitly using the Y Combinator},label=code:ycombuse]
(Y (lambda (recurse) (lambda (n) (if (= 0 n) 1
                                             (* n (recurse (- n 1)))))))\end{lstlisting}


Normally, one does not manually use the Y-Combinator. In Kraken, there is a \textit{rec-lambda} derived operative that is easy to use and evaluates Y-Combinator behind the scenes.
Y-Combinator is actually always used to implement recursion in Kraken whether explicitly stated or not.
This allows us to keep our language pure, and in agreement with the calculus.

This optimization actually falls out naturally from our architecture, with just a little bit of care taken while bookkeeping.
When compiling a combiner, the compiler first inserts what the combiner index will be into a memoization dictionary before re-executing partial evaluation on the body of the combiner.
Any static recursive calls will have the exact form of the combiner currently being compiled, and so the compiler can emit a static reference to the correct combiner index.
All of this works because the re-executed partial evaluation of the body before compilation made sure to normalize the form of the combiner of a recursive call to be identical to that of the combiner being compiled before this partial-evaluation.
Since this is an eager language, the definition of the Y-Combinator in our language has an extra closure to prevent infinite recursion inside the Y-Combinator itself.
We, thus, additionally implement eta-conversion in the compiler to remove this extra level of indirection. Since the expression inside is now a constant instead of a call, there is no risk of infinite recursion. Combined with the normalization above, we achieve fully-efficient static recursive calls when using the Y-Combinator to define recursive functions.


Finally, as a purely functional Lisp, we use recursion instead of iteration.
While we wait for the tail\_call instruction in WebAssembly to be merged and implemented, we implemented a more limited form of Tail Call Elimination where auto-recursive calls in tail position are transformed into branches to the head of a loop that encloses the combiner's body.
The combination of Tail Call Elimination with Y-Combinator Elimination above means that a recursive function defined using the Y-Combinator can be as efficient as an imperative loop in other languages.
When WebAssembly finishes implementing the tail\_call instruction, it can easily be emitted to gain full proper tail calls.

 \section{Benchmarks and Evaluation}
\label{sec:eval}

We evaluate \krakenSpace to answer the following set of questions:
\begin{itemize}
\item How much improvement does partial evaluation and our implemented compiler optimizations give \kraken? %(\S \ref{sec:eval2})
\item How much faster is our purely functional f-expr language, \krakenSpace, compared to other implementations of fexprs? %(\S \ref{sec:eval1} - \ref{sec:eval2})
\item How does \kraken's performance, with its fexprs, compare to macros? %(\S \ref{sec:eval1}, \S \ref{sec:eval3})
\item How do the different partial evaluation mechanisms/optimizations in \krakenSpace contribute towards reduction in overall runtime?
%\item What does \krakenSpace do internally when we create a data structure and evaluate it for some function? (\S \ref{sec:casestudy})
\end{itemize}

\textbf{Experimental Setup}: 
We ran these experiments in a reproducible Nix environment on a NixOS install \cite{10.1145/1411203.1411255} (Kernel 6.0.0) on a laptop with 8 cores / 16 threads and 64 GB of RAM.
Our code contains the scripts and Nix Flakes needed to reproduce the exact set of dependencies to run our tests.
%The code can be found at \url{https://github.com/limvot/kraken}.

The Kraken benchmarks were run using both the Wasmtime and WAVM WebAssembly engines for most benchmarks.
The Wasmtime WebAssembly engine is one of the most popular, developed by the Bytecode Alliance itself, and uses the CraneLift code generation backend.
The WAVM WebAssembly engine is interesting for its use of LLVM, and it often produces the fastest code on benchmarks but has a higher startup time.
We eliminated the Cfold Wasmtime benchmark due to problems running out of stack space (a known property of the Cfold benchmark).

\textbf{Benchmarks}: 
To showcase the capability of Kraken, we created benchmarks that are commonly implemented in functional languages and have been used as benchmarks in other papers \cite{reinking2021perceus, 10.1145/3547646}.
The benchmarks are
\begin{itemize}
\item Fib - Calculating the nth Fibonacci number
\item RB-Tree - Inserting n items into a red-black tree, then traversing the tree to sum its values
\item Deriv - Computing a symbolic derivative of a large expression
\item Cfold - Constant-folding a large expression
\item NQueens - Placing n number of queens on the board such that no two queens are diagonal, vertical, or horizontal from each other
\end{itemize}
All benchmarks besides Fibonacci use the fexpr version of match for pattern matching in \kraken, which is equivalent to the macro version in NewLisp. We also RB-Tree using NewLisp's~\cite{mueller2018newlisp} version of fexpr match. We modified the sizes of the problems presented to the benchmark to account for the longer running times of some of the less-optimized implementations.
The code for Kraken and NewLisp is very similar, and we should note that it is very unidiomatic NewLisp.
Our goal was not to compare Kraken and NewLisp as implementation languages for Red-Black Trees, but to stress test a single reasonably complex fexpr/macro, namely pattern matching.
% \textbf{Comparison with other languages}: We evaluated \krakenSpace against a language that contains f-exprs, as well as against itself with various optimizations disabled. The only other language we could find which contains a real f-expr mechanism is NewLisp~\cite{mueller2018newlisp} and so we ported \kraken's benchmark implementation to NewLisp.

%The six state-of-the-art languages are Java 17.0.1, Swift 5.4.2, Koka 2.3.2, C++, Haskell 8.10.7, and OCaml 4.12.
%The language choices were taken directly from Perceus reference-counting paper \cite{reinking2021perceus}.
%The Fibonacci benchmark additionally tests Python 3.9.11 and Chez Scheme 9.5.4.
%Koka, Ocaml and Haskell are good comparison points as statically-typed, compiled, functional programming languages, while Chez Scheme is a good comparison point as a mature and industrial strength dynamically-typed Scheme implementation known for its performance. 
%\subsection{Basic Level Comparison}
\subsection{The Effect of Partial Evaluation on Eval Calls}

\begin{table}[h]
\caption{Number of eval calls with no partial evaluation for Fexprs}
	\begin{tabular}{||c | c c c c c ||} 
		\hline
		&Evals & Eval w1 Calls & Eval w0 Calls & Comp Dyn & Comp Dyn\\ 
        & & & & w1 Calls & w0 Calls\\ [0.5ex] 
		\hline\hline
		Cfold 5 & 10897376 & 2784275 & 879066  & 1 & 0 \\ 
		\hline
		  Deriv 2  & 11708558 & 2990090 & 946500 & 1 & 0 \\ 
        \hline
		  NQueens 7 & 13530241 & 3429161 & 1108393 & 1 & 0 \\ 
    \hline
		  Fib 30 & 119107888 & 30450112 & 10770217 & 1 & 0 \\ 
    \hline
		  RB-Tree 10 & 5032297 & 1291489 & 398104 & 1 & 0 \\ 
		\hline
	\end{tabular}
    \label{npe:calls}
 \end{table}

As mentioned before, using fexprs without partial evaluation will prelude optimization and cause a massive amount of repeated work. Table \ref{npe:calls} and Table \ref{pe:calls} show the number of calls to the \krakenSpace runtime's eval function, the number of times the runtime's eval function executed a call to an applicative with wrap\_level=1, the number of times the runtime's eval function executed a call to an operative with wrap\_level=0, the number of compiled dynamic calls to applicatives with wrap\_level=1, and the number of compiled dynamic calls to operatives with wrap\_level=0.
These are shown for \krakenSpace test cases with partial evaluation turned off and turned on. 
\begin{table}[h]
\caption{Number of eval calls in Partially Evaluated Fexprs}
	\begin{tabular}{||c | c c c c c ||} 
		\hline
		&Evals & Eval w1 Calls & Eval w0 Calls & Comp Dyn & Comp Dyn\\ 
        & & & & w1 Calls & w0 Calls\\ [0.5ex] 
		\hline\hline
		Cfold 5 & 0 & 0 & 0  & 0 & 0 \\ 
		\hline
		  Deriv 2  & 0 & 0 & 0 & 2 & 0 \\ 
        \hline
		  NQueens 7 & 0 & 0 & 0 & 0 & 0 \\ 
    \hline
		  Fib 30 & 0 & 0 & 0 & 0 & 0 \\ 
    \hline
		  RB-Tree 10 & 0 & 0 & 0 & 10 & 0 \\ 
		\hline
	\end{tabular}
    \label{pe:calls}
 \end{table}

\begin{table}[h]
\caption{Number of calls to the runtime's eval function for RB-Tree. The table shows the non-partial evaluation numbers -> partial evaluation numbers.}
	\begin{tabular}{||c | c c c c c ||} 
		\hline
		&Evals & Eval w1 Calls & Eval w0 Calls & Comp Dyn & Comp Dyn\\ 
        & & & & w1 Calls & w0 Calls\\ [0.5ex] 
		\hline\hline
		  RB-Tree 7 & 2952848 -> 0 & 757932 -> 0 & 233513 -> 0 & 1 -> 7 & 0 -> 0\\ 
        \hline
		  RB-Tree 8 & 3532131 -> 0 & 906548 -> 0 & 279379 -> 0 & 1 -> 8 & 0 -> 0\\ 
        \hline
		  RB-Tree 9 & 4278001 -> 0 & 1097965 -> 0 & 3383831 -> 0 & 1 -> 9 & 0 -> 0\\ 
		\hline
	\end{tabular}
    \label{pe:rb}
    \vspace{-4mm}
 \end{table}

Without partial evaluation, no compilation can be done because it is impossible to tell if arguments to calls will be evaluated. In all benchmarks, partial evaluation removed all calls to the runtime's eval function, resulting in a completely compiled program. Looking at RB-Tree, there are over a million calls to combiners with wrap level 1 (normal functions), and 398,000 calls to combiners with wrap level 0 (operatives replacing macros). This massive blowup in the number of calls is due to the repeated and exponential re-execution of macro-like-combiners in the definition of other macro-like-combiners, as discussed in the Introduction.

The non-partially-evaluated benchmarks show 1 compiled dynamic call to an applicative (its the first call into eval) and 0 compiled dynamic calls to operatives, because there is no compilation at all. For the partially evaluated benchmarks, there are a few compiled dynamic calls to applicatives due to higher-order function use in the benchmarks, and there are no compiled dynamic calls to operatives, as all operative use has been eliminated.
We also varied the inputs for RB-Tree shown in Table \ref{pe:rb} to give a sense for how the number scale with respect to input size.

The incredible slowdown implied by these tables comes to full fruition in our RB-Tree test in Fig.~\ref{fig:kraken_nqueens_rbtree}.
We kept this run shorter because Kraken's non-partial-evaluating interpreter takes an incredibly long time even for 100 insertions (40 minutes).
The compounding layers of repeated macro-like operative calls in the non-partially-evaluated Kraken version cause a ~70,000x slowdown relative to the partial evaluated, optimized, and compiled version.
For the remaining benchmarks, we remove the naive interpreted \krakenSpace version, as in each case its performance is so bad as to blow out the graph and make it impossible to do any comparison.
In our optimized Kraken, our partial evaluation algorithm is able to fully collapse these levels of inefficiency, evaluate and inline the results, and give the backend more specialized code to optimize, emitting a compiled version that handily beats not only the NewLisp-fexpr implementation but even the NewLisp-macro implementation, as can be seen in Fig.~\ref{fig:kraken_vs_world_fib}.
We kept the benchmark sizes small in this test because the stack limits of NewLisp prevent sizes larger then ~880, while the Tail Call Elimination performed by the \krakenSpace compiler allows us to run much larger benchmarks, including the run of 4,800,000 inserts to the RB-Tree.
This result shows the dramatic effect of partial evaluation and compiler optimizations on runtime for \kraken. Our technique takes the performance of a fully fexpr based language from being completely infeasible to being faster than a macro-based dynamic scripting language currently in use.
% \begin{center}
% \begin{table}[ht]
% \caption{Number of call to the runtime's eval function for Fib. The table shows the non-partial evaluation numbers -> partial evaluation numbers}
% 	\begin{tabular}{||c | c c c c c ||} 
% 		\hline
% 		&Evals & Eval w1 Calls & Eval w0 Calls & Comp Dyn w1 Calls & Comp Dyn w0 Calls\\ [0.5ex] 
% 		\hline\hline
% 		Fib 10 & 8468 -> 0 & 2167 -> 0  & 777 -> 0 & 1 -> 0 & 0 -> 0 \\ 
% 		\hline
% 		  Fib 15  & 87916 -> 0 & 22478 -> 0 & 7961 -> 0 & 1 -> 0 & 0 -> 0 \\ 
%         \hline
% 		  Fib 20 & 969010 -> 0 & 247731 -> 0 & 87633 -> 0 & 1 -> 0 & 0 -> 0 \\ 
%     \hline
% 		  Fib 25 & 10740492 -> 0 & 2745825 -> 0  & 971209 -> 0 & 1 -> 0 & 0 -> 0 \\ 
% 		\hline
% 	\end{tabular}
%     \label{pe:fib}
%  \end{table}
% \end{center}

\begin{figure}[h]
\caption{Constant Fold and Deriv}
\includegraphics[width=0.45\textwidth]{cfold_table.csv_}
\includegraphics[width=0.45\textwidth]{deriv_table.csv_}
\label{fig:kraken_const_deriv}
\vspace{-6mm}
\end{figure}
\subsection{Comparison between Kraken Versions}
Beyond the massive speedup from partial-evaluation, Fig. \ref{fig:kraken_const_deriv} and \ref{fig:kraken_nqueens_rbtree} show the effect of the various compiler optimizations we described by disabling them one by one.
 Our main four optimizations have a strong positive effect on runtime, with the exception of lazy environment instantiation. Lazy environment instantiation helps massively on fib, and some on Deriv, but generally hurts the rest slightly.


\begin{figure}[h]
\caption{N-Queens}
\includegraphics[width=0.45\textwidth]{nqueens_table.csv_}
\includegraphics[width=0.45\textwidth]{slow_rbtree_table.csv_}
\label{fig:kraken_nqueens_rbtree}
\vspace{-4mm}
\end{figure}


\subsection{Comparison against Others}


To give a general idea of our current performance, we also show a Fibonacci benchmark that mostly exercises pure function-call speed and inlining as seen in Fig. ~\ref{fig:kraken_vs_world_fib}.
We include Python and Chez Scheme to give a general idea for where an exemplar slow and an exemplar fast dynamic language would fall.
With the benefit of our partial evaluation, compilation, and leaning upon mature WebAssembly implementations, we beat both, but this should be taken with a grain of salt, as this is a very limited micro-benchmark only meant to give a general sense of the order of magnitude of our performance.



\label{sec:eval1}
\begin{figure}[h]
\caption{Kraken vs. Others. Ordered by fastest to slowest}
\includegraphics[width=0.45\textwidth]{fib_table.csv_}
\includegraphics[width=0.45\textwidth]{rbtree_table.csv_}
\label{fig:kraken_vs_world_fib}
\end{figure}

%\label{sec:eval_nqueens}
%\begin{figure}[h]
%\caption{N-Queens}
%\includegraphics[width=0.45\textwidth]{nqueens_table.csv_}
%\includegraphics[width=0.45\textwidth]{slow_nqueens_table.csv_}
%\label{fig:kraken_nqueens}
%\end{figure}

%\label{sec:eval_nqueens}
%\begin{figure}[h]
%\caption{Kraken, N-Queens, absolute value and log-scale}
%\includegraphics[width=0.45\textwidth]{nqueens_table.csv_}
%\includegraphics[width=0.45\textwidth]{nqueens_table.csv_log}
%\label{fig:kraken_nqueens}
%\end{figure}
%\label{sec:eval_nqueensp}
%\begin{figure}[h]
%\caption{Kraken, N-Queens, absolute value and log-scale}
%\includegraphics[width=0.45\textwidth]{slow_nqueens_table.csv_}
%\includegraphics[width=0.45\textwidth]{slow_nqueens_table.csv_log}
%\label{fig:kraken_nqueensp}
%\end{figure}

%\label{sec:eval_cfold}
%\begin{figure}[h]
%\caption{C-Fold}
%\includegraphics[width=0.45\textwidth]{cfold_table.csv_}
%\includegraphics[width=0.45\textwidth]{slow_cfold_table.csv_}
%\label{fig:kraken_cfold}
%\end{figure}
%\label{sec:eval_cfold}
%\begin{figure}[h]
%\caption{Kraken, C-Fold, absolute value and log-scale}
%\includegraphics[width=0.45\textwidth]{cfold_table.csv_}
%\includegraphics[width=0.45\textwidth]{cfold_table.csv_log}
%\label{fig:kraken_cfold}
%\end{figure}
%\label{sec:eval_cfoldp}
%\begin{figure}[h]
%\caption{Kraken, C-Fold, absolute value and log-scale}
%\includegraphics[width=0.45\textwidth]{slow_cfold_table.csv_}
%\includegraphics[width=0.45\textwidth]{slow_cfold_table.csv_log}
%\label{fig:kraken_cfoldp}
%\end{figure}

%\label{sec:eval_deriv}
%\begin{figure}[h]
%\caption{Deriv}
%\includegraphics[width=0.45\textwidth]{deriv_table.csv_}
%\includegraphics[width=0.45\textwidth]{slow_deriv_table.csv_}
%\label{fig:kraken_deriv}
%\end{figure}
%\label{sec:eval_deriv}
%\begin{figure}[h]
%\caption{Kraken, Deriv, absolute value and log-scale}
%\includegraphics[width=0.45\textwidth]{deriv_table.csv_}
%\includegraphics[width=0.45\textwidth]{deriv_table.csv_log}
%\label{fig:kraken_deriv}
%\end{figure}
%\label{sec:eval_derivp}
%\begin{figure}[h]
%\caption{Kraken, Deriv, absolute value and log-scale}
%\includegraphics[width=0.45\textwidth]{slow_deriv_table.csv_}
%\includegraphics[width=0.45\textwidth]{slow_deriv_table.csv_log}
%\label{fig:kraken_derivp}
%\end{figure}

%\subsection{Comparison against state-of-the-art languages}
%\label{sec:eval3}

%\begin{figure}[h]
%\caption{Kraken vs. S.o.t.A.}
%\includegraphics[width=0.45\textwidth]{cfold_table.csv_}
%\includegraphics[width=0.45\textwidth]{rbtree_table.csv_}
%\label{fig:kraken_vs_world1}
%\end{figure}

%\begin{figure}[h]
%\caption{Kraken vs. S.o.t.A.}
%\includegraphics[width=0.45\textwidth]{deriv_table.csv_}
%\includegraphics[width=0.45\textwidth]{nqueens_table.csv_}
%\label{fig:kraken_vs_world2}
%\end{figure}

% \begin{figure}[h]
% \caption{Kraken vs. S.o.t.A. (Log)}
% \includegraphics[width=0.45\textwidth]{cfold_table.csv_log}
% \includegraphics[width=0.45\textwidth]{rbtree_table.csv_log}
% \label{fig:kraken_vs_world_log_1}
% \end{figure}
% \begin{figure}[h]
% \caption{Kraken vs. S.o.t.A. (Log)}
% \includegraphics[width=0.45\textwidth]{deriv_table.csv_log}
% \includegraphics[width=0.45\textwidth]{nqueens_table.csv_log}
% \label{fig:kraken_vs_world_log_2}
% \end{figure}

%As we noted before with the Fib(30) microbenchmark in Section \ref{sec:eval1}, we remain significantly slower than state-of-the-art compiled languages.
%This is particularly true for memory-intensive benchmarks due to our naive reference-counting and malloc/free implementations.
%However, our results are of a similar order of magnitude to the difference between the state-of-the-art compiled languages and dynamic scripting languages, like Python's results in the Fib(30) microbenchmark.
%We assert that is not a fundamental limitation because the classic f-expr slowness is being eliminated, as shown by Fig. \ref{fig:kraken_vs_newlisp1} and Fig. \ref{fig:kraken_vs_newlisp2}.
%In future work, we plan to expand our compile-time analysis and optimization to implement a modified, dynamic-language version of Perceus reference counting.
%With this change, we belive \krakenSpace can be competitive with these state-of-the-art languages.

%\subsection{Case Study: Red-Black Tree}
%\label{sec:casestudy}

%\begin{figure}[h]
%\caption{Kraken vs. S.o.t.A. - RB-Tree Focus}
%\includegraphics[width=0.4\textwidth]{rbtree_table.csv_}
%\includegraphics[width=0.4\textwidth]{rbtree_table.csv_log}
%\label{fig:kraken_vs_world_rbtree}
%\end{figure}


%To evaluate our partial evaluation algorithm and compiler, we extracted the benchmarks used by the Koka language project from their code repository and added Kraken versions, as well as implementing a naive Fibonacci microbenchmark ourselves to evaluate pure function call speed.\\
%With partial evaluation and the compiler optimizations listed above, we get fairly strong performance on purely numerical computations, such as the naive Fibonacci microbenchmark.
%Unfortunately, the overhead of our unsophisticated reference counting, dynamic type checking, and bounds checking causes poor performance on benchmarks involving data structures relative to mainstream programming language implementations.
%This is not a fundamental limitation, and will be addressed in future work, as recounted in the next section.
%It should be noted, however, that while the performance relative to established language implementations is very poor for the memory-intensive benchmarks (600-900x slower), we still realize a massive speedup compared to an unoptimized and non-partial-evaluated f-expr implementation (100,000x faster)!

\section{Related Work}
\label{sec:relatedwork}

%%%%%%%%%%%%%%%%%%%%%%%%%% Outline %%%%%%%%%%%%%%%%%%%%%%%%%%%%%%%%%%%%%
%(1) Evasion Attacks
%(1.1) Surveys on evasion attacks and their relation to data properties - Michael
%(1.2) Individual papers that study non-data related reasons behind evasion attacks - Michael
%(1.3) Techniques related to evasion attacks and defenses (new) - Gabby
%(2) Non-Evasion Attacks (new), and - ???
%(3) Effects of training data on standard generalization - done 
%
%
%
%(1) Evasion Attacks
%(1.1) A number of surveys review literature on evasion attacks. - Michael
%Most of them do not focus specifically on properties of data but also discuss attack and defense mechanisms, non-data-related reasons for adversarial vulnarability, and  more. ~\jr{cite 4}.
%Yet, they these surveys mention data and its relation to evasion attacks. Specifically \jr{what they say about data.}
%The most close to ours is concurrent work by XXX + concrete facts that we have and they don't.
%
%(1.2) individual papers that study non-data related reasons behind evasion attacks, - Michael
%Literature identifies multiple reasons for adversarial vulnerability, in particular, for evasion attacks. 
%These include data-related properties extensively discussed in this survey, as well as reasons related to the models 		   themselves, computations resources, and feature representations. We discuss these below. 
%
%\jr{the rest is from the paper (non-data related reasons for adversarial vulnerability), with sections potentially renamed.}
%
%{\bf Model.}
%
%{\bf Computational Resources.}
%
%{\bf Robustness of Features.}
%
%(1.3) Techniques Related to Evasion Attacks and Defenses (new) - Gabby
%A number of works focus on techniques for generating evasion attacks, countermeasures against these attacks, 
%and defining the notion of the attack itself.   
%
%{\bf Attacks and Defense.}
%Here are the 5 remaining surveys + 1 additional paper for the reviewer.
%
%{\bf Adversarial Examples.}
%2 surveys lines 13 and 14 + 1 additional paper for the reviewer.
%
%(2) Non-Evasion Attacks (new) 
%Need to say that there are other type of attacks, define them, cite surveys (Bo's survey, maybe something else). 
%Only one work explicitly focus on effects of data. 
%
%
%(3) Effects of training data on standard generalization (done)

%%%%%%%%%%%%%%%%%%%%%%%%% Outline %%%%%%%%%%%%%%%%%%%%%%%%%%%%%%%%%%%%%


\revreplace{
We divide related work into three categories:
(1) surveys on adversarial robustness and its relation to data properties,
(2) surveys that discuss the influence of data properties on standard generalization, and
(3) individual papers that study non-data-related reasons for adversarial vulnerability.\\
}
{
This survey investigates properties of training data in the context of model robustness under evasion attacks. 
We start the discussion of related work by reviewing other surveys that focus on evasion attacks and 
include some discussion about data (Section~\ref{sec:relatedwork-surveys-data}).  
We then discuss non-data related reasons behind evasion attacks (Section~\ref{sec:relatedwork-not-data}),
as well as techniques related to evasion attacks and defenses (Section~\ref{sec:relatedwork-attacks}). 
Finally, we discuss data-related concerns for non-evasion attacks (Section~\ref{sec:relatedwork-poisoning}) and
the effects of training data on standard generalization (Section~\ref{sec:relatedwork-standard}).
}

%\vspace{-0.1in}
\subsection{Surveys on Evasion Attacks that Discuss Data}
\label{sec:relatedwork-surveys-data}
Numerous existing surveys 
\revreplace{focus on attack and defense techniques for adversarial robustness. 
%~\cite{Biggio:Roli:PR:2018,
%Rosenberg:Shabtai:Elovici:Rokach:CSUR:2021,
%Li:Li:Ye:Xu:CSUR:2021,
%Maiorca:Biggio:Giorgio:CSUR:2019,
%Demetrio:Coull:Biggio:Lagorio:Armando:Roli:ACMTPS:2021,
%Liu:Tantithamthavorn:Li:Liu:CSUR:2022,
%Liu:Nogueria:Fernandes:Kantarci:IEEECST:2022,
%Akhtar:Mian:IEEEAccess:2018,
%Akhtar:Mian:Kardan:Shah:IEEEAccess:2021,
%Serban:Poll:Visser:CSUR:2020,
%Machado:Silva:Goldschmidt:CSUR:2021,
%Zhang:Sheng:Alhazmi:Li:ACMTIST:2020}.
Only a few of these works mention the relationship between adversarial robustness and properties of the underlying data.} 
{review the literature on evasion attacks.
Most of these works do not focus specifically on properties of data but discuss attack and defense mechanisms, non-data-related reasons for adversarial vulnerability, 
and the different threat models. 
Only a few of these works mention data-related reasons for the existence of adversarial examples~\cite{Serban:Poll:Visser:CSUR:2020, Machado:Silva:Goldschmidt:CSUR:2021, Akhtar:Mian:Kardan:Shah:IEEEAccess:2021, Akhtar:Mian:IEEEAccess:2018}.
}
Specifically, Serban et al.~\cite{Serban:Poll:Visser:CSUR:2020} observe that adversarial vulnerability can be caused by an insufficient training sample size %~\cite{Schmidt:Santurkar:Tsipras:Talwar:Madry:NeurIPS:2018}
and high data dimensionality. %~\cite{Gilmer:Metz:Faghri:Schoenholz:Raghu:Wattenberg:Goodfellow:ICLR:2018}.
Similarly, Machado et al.~\cite{Machado:Silva:Goldschmidt:CSUR:2021} mention that the lack of sufficient training data, high dimensionality, 
and high concentration contribute to adversarial vulnerability.
\revadd{
Akhtar et al.~\cite{Akhtar:Mian:IEEEAccess:2018, Akhtar:Mian:Kardan:Shah:IEEEAccess:2021} also mention high dimensionality, along with other non-data-related reasons, 
as a source of adversarial examples.}

\revadd{A concurrent work by Han et al.~\cite{Han:Lin:Shen:Wang:Guan:CSUR:2023} (published at the end of April 2023) 
studies the origins of adversarial vulnerability in deep learning w.r.t. the model, data, and other perspectives.
The authors mention high dimensionality, distributions with high concentration, a small number of output classes, data imbalance, and the perceptual difference in image frequencies as potential sources of adversarial examples.
However, as (a) the focus of that survey is not on data-related properties in particular, 
(b) its paper search was conducted in 2021, and 
(c) it focuses on deep learning models only, 
our work was able to identify more than 50 additional relevant papers which focus on other types of models, 
e.g., non-parametric and linear classifiers, 
and/or discuss additional types of data-related properties, 
such as, types of distribution, class density, separation, and label quality.}
\revreplace{Yet, none of these surveys explicitly collect and analyze work that focuses on the effects of data properties
on adversarial robustness.}
{In summary, by explicitly focusing on the effects of data properties on evasion attacks in our survey, 
we are able to provide a more complete and detailed discussion on this topic, not covered in prior surveys.}

\vspace{-0.05in}
\subsection{Non-data-related Reasons Behind Evasion Attacks}
\label{sec:relatedwork-not-data}

%\vspace{-0.1in}
%\subsection{Non-data Related Reasons for Adversarial Vulnerability}

There has been a variety of hypotheses regarding the reasons behind adversarial vulnerability of ML systems, particularly for evasion attacks.
%\revreplace{
%In addition to the data used for training,  adversarial robustness could also depend on the choice of the model architecture,
%the training procedure, and the interplay between data and the learning algorithm, i.e., correspondence between the complexity of a model to that of the data.
%This section summarizes the key hypotheses regarding these aspects.
%%The hypotheses reviewed in this section are complementary to the potential influence from the data.
%}
These include data-related properties extensively discussed in this survey, as well as reasons related to the models themselves, 
computational resources, and feature learning procedures. We discuss these below.

%\jr{there is a lot of undefined terminology and jargon in this section.}

\vspace{0.02in}
\noindent
\textbf{Model.}
When Szegedy et al.~\cite{Szegedy:Zaremba:Sutskever:Bruna:Erhan:Goodfellow:Fergus:ICLR:2014} first discovered adversarial examples for visual models, they suspected that the high non-linearity of DNNs resulted in low probability `pockets' of adversarial examples in the learned representation manifold.
They hypothesize that while these pockets can be found through attack algorithms, the samples residing in these pockets have different distributions compared to normal samples and are thus subsequently harder to find when randomly sampling from the input space.
Instead, Goodfellow et al.~\cite{Goodfellow:Shlens:Szegedy:ICLR:2015} hypothesize that
the linearity from activation functions, like ReLU and sigmoid found in high-dimensional neural networks, induce vulnerability towards adversarial perturbations.
To support their claim, they present the attack method FGSM that exploits the linearity of the target classifier.
Fawzi et al.~\cite{Fawzi:Fawzi:Frossard:ICMLWorkshop:2015} also argue against the hypothesis of high non-linearity as the cause for adversarial examples.
They show that all classifiers are susceptible to adversarial attacks and claim that it is the low flexibility of the classifier compared to the complexity of the classification task that results in vulnerability.
The lack of consensus on the primary causes of model vulnerability invites more studies on this topic.

Singla et al.~\cite{Singla:Ge:Basri:Jacobs:NeurIPS:2021} show that enforcing invariance to circular shifts (e.g., rotation) in neural networks induces decision boundaries with a smaller margin than normal, fully connected networks,
which, in turn, reduces the adversarial robustness of the model.
Moosavi{-}Dezfooli et al.~\cite{Moosavi-Dezfooli:Fawzi:Fawzi:Frossard:Soatto:ICLR:2018} introduce universal,
input-agnostic perturbations to mislead the classifier and hypothesize that the vulnerability of a multi-class classifier to such perturbations is related to the shape of its decision boundaries, e.g.,
linear classifiers with decision boundaries that are parallel to each other and
nonlinear classifier with decision boundaries that are curved in a similar way
tend to be less robust as
perturbations in one direction can change the prediction label for a different class.

Tanay and Griffin~\cite{Tanay:Griffin:ArXiv:2016} conjecture that the decision boundary learned by the classifier being too close to (or `tilted towards') the data manifold instead of being perpendicular to it,
results in small perturbations being sufficient to move samples across the decision boundary for misclassification.
%data manifold refers to the underlying structure that the data exhibit

\vspace{0.02in}
\noindent
\textbf{Computational Resources.}
Bubeck et al.~\cite{Bubeck:Lee:Price:Razenshteyn:ICML:2019} use computational hardness theory to show that the time complexity for learning a robust model is exponential to the size of input data and thus is computationally intractable.
Hence, they attribute adversarial vulnerability to computational limitations of current learning algorithms.
Degwekar et al.~\cite{Degwekar:Nakkiran:Vaikuntanathan:COLT:2019} further extend this work and also show the impossibility of efficiently training robust classifiers.

%\subsubsection{Ineffective Learning Perspective}
\vspace{0.02in}
\noindent
\textbf{Feature Learning.}
Ilyas et al.~\cite{Ilyas:Santurkar:Tsipras:Engstrom:Tran:Madry:NeurIPS:2019} show that adversarial vulnerability can be a consequence of a model exploiting well-generalizing but non-robust features,
i.e., features that are spurious and sometimes incomprehensible to humans;
when constraining the model to use robust features, the adversarial robustness increases together with the
interpretability of the learned features.
However, Tsipras et al.~\cite{Tsipras:Santurkar:Engstrom:Turner:Madry:ICLR:2019} note that, as the features for achieving high accuracy may be different from the ones for achieving high robustness, robustness may be at odds with standard accuracy.
%
%\jr{why is it called Ineffective learning when it is about features.}\gx{I put it under ineffective learning as in this case, the model learns/decides the features for generalization, and when given the correct objective, the model in fact, can learn more robust features, so I think the underlying reason is objective we gave for the model didn't guide the model to learn the right features}
%
Instead of seeing adversarial vulnerability as a product of classifiers being overly sensitive to changes in spurious features, Jacobsen et al.~\cite{Jacobsen:Behrmann:Zemel:Bethge:ICLR:2019} hypothesize that classifiers can rather be
overly insensitive to relevant semantic information, e.g., images with drastically different content can share similar latent representations.
The authors introduce a new type of adversarial examples that exploit such insensitivity, where the content of images is altered without changing the resulting prediction label.
%As both insensitivity to semantic content and sensitivity to spurious changes can simultaneously exist in models,
%more investigation into how to define proper objectives for models to effectively distinguish the relevant information is needed.

While all these works propose possible reasons for adversarial vulnerabilities, they are orthogonal to our survey, which focuses particularly on the influence of training data.

\vspace{-0.05in}
\revadd{
\subsection{Evasion Attacks and Defenses}
\label{sec:relatedwork-attacks}
A number of works focus on techniques for generating evasion attacks, countermeasures against these attacks, 
and defining the notion of the attack itself.

%\jr{need to include~\cite{Biggio:Roli:PR:2018,
%Rosenberg:Shabtai:Elovici:Rokach:CSUR:2021,
%Li:Li:Ye:Xu:CSUR:2021,
%Maiorca:Biggio:Giorgio:CSUR:2019,
%Demetrio:Coull:Biggio:Lagorio:Armando:Roli:ACMTPS:2021,
%Liu:Tantithamthavorn:Li:Liu:CSUR:2022,
%Liu:Nogueria:Fernandes:Kantarci:IEEECST:2022,
%Zhang:Sheng:Alhazmi:Li:ACMTIST:2020} x and one more survey.}
%\js{\cite{Biggio:Roli:PR:2018, Rosenberg:Shabtai:Elovici:Rokach:CSUR:2021} moved to Adversarial Examples.
%\cite{Rosenberg:Shabtai:Elovici:Rokach:CSUR:2021,
%Li:Li:Ye:Xu:CSUR:2021,
%Maiorca:Biggio:Giorgio:CSUR:2019, Liu:Tantithamthavorn:Li:Liu:CSUR:2022,
%Liu:Nogueria:Fernandes:Kantarci:IEEECST:2022,
%Zhang:Sheng:Alhazmi:Li:ACMTIST:2020, Demetrio:Coull:Biggio:Lagorio:Armando:Roli:ACMTPS:2021} in Attacks and Defense. \cite{Sun:Dou:Yang:Zhang:Wang:Philip:He:Li:TKDE:2022} was the "one more survey" and is also in Attacks and Defenses.}

\vspace{0.02in}
\noindent
{\bf Attacks and Defense.}
Several works~\cite{Liu:Tantithamthavorn:Li:Liu:CSUR:2022,Liu:Nogueria:Fernandes:Kantarci:IEEECST:2022,Sun:Dou:Yang:Zhang:Wang:Philip:He:Li:TKDE:2022, Demetrio:Coull:Biggio:Lagorio:Armando:Roli:ACMTPS:2021} survey adversarial attacks and defenses, observing that most work focuses on computer vision and NLP domains. 
Zhang et al.~\cite{Zhang:Sheng:Alhazmi:Li:ACMTIST:2020}, 
Rosenberg et al.~\cite{Rosenberg:Shabtai:Elovici:Rokach:CSUR:2021},
Li et al.~\cite{Li:Li:Ye:Xu:CSUR:2021}, and 
Maiorca et al.~\cite{Maiorca:Biggio:Giorgio:CSUR:2019}, 
survey attacks and defenses in the NLP domain, cybersecurity domain for networks, Android malware, and PDF malware, respectively. 
These works identify a similar trend of new attacks constantly bypassing defenses, which gives rise to new defenses being proposed, only to be broken again (a.k.a. the `cat and mouse race' or the `arms race'). 
They also observe that research in this field studies attacks / defenses at a feature-level, which restricts 
the practicality of the developed techniques by the feasibility of perturbing the corresponding features in real life. 

%practical attacks are quite difficult and require some basic knowledge about the model or training data such as the feature set or model architecture. 
%Zhang et al.~\cite{Zhang:Sheng:Alhazmi:Li:ACMTIST:2020}, who study adversarial attacks and defenses in the NLP domain,  
%also find that there are obstacles to generating attacks in real-time. 
%For instance, methods that iteratively use gradients to create adversarial examples can be time-consuming, while one-time approaches may fail to produce potent adversarial examples.
%Several works~\cite{Liu:Tantithamthavorn:Li:Liu:CSUR:2022,Liu:Nogueria:Fernandes:Kantarci:IEEECST:2022,Sun:Dou:Yang:Zhang:Wang:Philip:He:Li:TKDE:2022, Demetrio:Coull:Biggio:Lagorio:Armando:Roli:ACMTPS:2021} 
%discuss how most new attacks and defenses are explored in computer vision and NLP, prior to other fields.


%our survey finds the state of the art w.r.t. data properties
%our survey finds that dimensionality is bad ...
%
%%%Here are the 5 remaining surveys + 1 additional paper for the reviewer.
%Numerous surveys have explored the landscape of adversarial evasion attacks and defenses. 
%For instance, Akhtar et al.~\cite{Akhtar:Mian:IEEEAccess:2018, Akhtar:Mian:Kardan:Shah:IEEEAccess:2021} survey the literature on adversarial robustness of deep learning models from Computer Vision field.
%They review popular attacks on visual models, and provided a categorization of existing defense techniques based on the components it modify in the visual model system \gx{Check}.
%
%Rosenberg et al.~\cite{Rosenberg:Shabtai:Elovici:Rokach:ACMComputingSurvey:2021}, Li et al. ~\cite{Li:Li:Ye:Xu:ACMComputingSurvey:2021} and Demetrio et al.~\cite{Demetrio:Coull:Biggio:Lagorio:Armando:Roli:ACMTPS:2021} review the literature on evasion attacks for cyber-security fields. 
%Li et al. proposed a partial order scheme to compare key attacks and defenses techniques for malware detection in Windows, Android, and PDF domains. 
%
%Zhang et al.~\cite{Zhang:Sheng:Alhazmi:Li:ACMTIST:2020} review the literature on adversarial attacks on deep-learning models for textual classification.
%They pointed out the intrinsic differences between Computer Vision and Natural Language Processing fields that pose challenges to directly apply attacks proposed for Visual models to NLP models and identified the strategies proposed that overcomes the barriers.
%The challenges they identified for creating realistic attacks in NLP fields are from a domain characteristics perspective (e.g., definition of imperceptible perturbations, measurement of the semantic changes),  we differ from them by trying to understand the adversarial robustness of machine learning from the characteristics of underlying data. 
%
%Attack and Defenses for wireless and Mobile systems~\cite{Liu:Nogueria:Fernandes:Kantarci:IEEECST:2022}
%
%

More recent research, not included in the surveys above, has also started investigating the 
susceptibility of newer models to adversarial evasion attacks. 
For example, several studies~\cite{Wang:Pan:Hu:Duan:Pan:IJSWIS:2022,Yin:Lin:Sun:Wei:Chen:TIFS:2023, 
Shi:Han:Tan:Kuang:NeurIPS:2022, Wang:Xie:Microsoft:ChatGPT:ArXiv:2023} proposed attack techniques against contemporary models, 
such as Graph Neural Networks, Generative Pre-training Transformers (GPT), and Vision Transformers. 
These studies showed that adversarial examples persist even for the newer models, some of which are 
trained with large volumes of data. 
As all these works focus on attack and defense mechanisms rather than 
the effects of data on adversarial robustness, our work extends and complements this research.
}

\revadd{
\vspace{0.02in}
\noindent
{\bf Adversarial Examples.}
%2 surveys lines 13 and 14 + 1 additional paper for the reviewer.
Adversarial examples are inputs constructed by perturbing a correctly classified sample in a way that makes the change imperceptible to a human. % but causes the model to misclassify the sample.
However, as `imperceptible to a human' is hard to define, existing research on adversarial examples approximates imperceptibility with a small perturbation measured through $L_p$ norms.
A line of research~\cite{Gilmer:Adams:Goodfellow:Anderson:Dahl:ArXiv:2018,Sharif:Bauer:Reiter:CVPRW:2018,Fezza:Bakhti:Hamidouche:Deforges:QoMEX:2019, Mezher:Deng:Karam:EUVIP:2022} 
investigates the validity of this assumption. 
This work shows that perturbations generated by $L_p$ norms do not entirely align with human perceptions, 
i.e., some changes with a small $L_p$ norm can be apparent to humans. 
In addition, adversarial examples with the minimum $L_p$ perturbation may be less effective and transferable than 
higher perturbation~\cite{Biggio:Roli:PR:2018,Rosenberg:Shabtai:Elovici:Rokach:CSUR:2021}. 
Hence, a number of approaches explore metrics for imperceptibility 
in computer vision and NLP domains~\cite{Fezza:Bakhti:Hamidouche:Deforges:QoMEX:2019,Mezher:Deng:Karam:EUVIP:2022, Zhang:Sheng:Alhazmi:Li:ACMTIST:2020}. 
Yet another issue with $L_p$ norms is that they cannot be used reliably in domains other than images. 
For example, in the case of software/malware, simply generating adversarial examples with $L_p$ norms 
may result in feature representations that are not possible in 
the problem space~\cite{Rosenberg:Shabtai:Elovici:Rokach:CSUR:2021,Pierazzi:Pendlebury:Cortellazz:Cavallaro:2020}. 

While all these works focus on the properties of adversarial examples, 
they are orthogonal to the topic of our survey, as we rather focus on how properties of the training data 
affect the success of adversarial examples.
}

%Gilmer et al.~\cite{Gilmer:Adams:Goodfellow:Anderson:Dahl:ArXiv:2018} argue that, while constraining the perturbations by sufficiently small $L_p$ norms can generate indistinguishable samples for most inputs, the actual imperceptibility of the changes depends on the input sample. 
%Several individual studies~\cite{Sharif:Bauer:Reiter:CVPRW:2018,Fezza:Bakhti:Hamidouche:Deforges:QoMEX:2019, Mezher:Deng:Karam:EUVIP:2022} find faults with using $L_p$ norms to generate adversarial examples. They show that the changes measured by $L_p$ norm, does not entirely align with human perceptions, i.e., some changes with a small $L_p$ norm appear apparent to humans. 
%In some domains adversarial examples do not need to be imperceptible but rather semantically preserving. 
%For example, in the case of Android malware~\cite{Rosenberg:Shabtai:Elovici:Rokach:CSUR:2021}, adversarial examples are small perturbations which fool a model while preserving the semantics of the sample, 
%i.e., a malware stays malicious even after the perturbation. 
%This highlights another problem with $L_p$ norm based adversarial examples as Dong et al.~\cite{Dong:Liu:Shang:NeurIPS:2022} show that the semantics of a sample change during adversarial training. 
%Hence, there is a need for metrics to measure the size of perturbations that is imperceptible or semantically preserving.
%Fezza et al.~\cite{Fezza:Bakhti:Hamidouche:Deforges:QoMEX:2019} and Mezher et al.~\cite{Mezher:Deng:Karam:EUVIP:2022} propose to use objective metrics for image quality to approximate the imperceptibility in the computer vision domain.
%Zhang et al.~\cite{Zhang:Sheng:Alhazmi:Li:ACMTIST:2020}, focusing on providing such a metric for Natural Language Processing.
%Vadillo et al.~\cite{Vadillo:Santana:CS:2022} also highlight conducted subject studies to evaluate the noticeability of audio adversarial examples.

%Even in computer vision, adversarial examples are not always imperceptible. For example, Machado et al.~\cite{Machado:Silva:Goldschmidt:CSUR:2021} find that visible perturbations such as adversarial patch~\cite{Brown:Mane:Roy:Abadi:Gilmer:ArXiv:2017}, and graffiti on stop signs~\cite{Eykholt:Evtimov:Fernandes:Li:Rahmati:Xiao:Prakash:Kohno:Song:CVPR:2018} are also considered adversarial examples in research.

%The aforementioned research examines the work on defining and creating adversarial examples, demonstrating the insufficiency of using conventional $L_p$ norms to evaluate the imperceptibility and semantics between clean and adversarial examples. 

\vspace{-0.1in}
\revadd{
\subsection{Non-Evasion Attacks}
\label{sec:relatedwork-poisoning}
Similar to evasion attacks, data poisoning and backdoor attacks aim to compromise model accuracy. 
However, they achieve it by tampering the training data to create deceptive model decision boundaries. 
%Data poisoning attacks involve modifying the training data to create deceptive decision boundaries, either to manipulate the prediction outcomes of a specific input or the entire model.
%Meanwhile, Backdoor attacks are a form of poisoning attacks where the attacker inject tempered training data with triggers 
% and then activates the attack by showing the trigger pattern at inference time.
In addition, backdoor attacks also require perturbing the test instance to result in a misclassification. 
This is achieved by introducing manipulated training data with triggers that can be activated during the testing phase.

Goldblum et al.~\cite{Goldblum:Tsipras:Xie:Chen:Schwarzchild:song:Madry:Li:Goldstein:TPAMI:2022} and Cinà et al.~\cite{Cina:Grosse:Demontis:Sebastiano:Zellinger:Moser:Oprea:Biggio:Pelillo:Roli:CSUR:2023} 
review recent literature on attack methodologies and countermeasures for both poisoning and backdoor attacks.
Both of these surveys found that existing research made overly-optimistic assumptions when designing / validating attack techniques, e.g., assuming the knowledge of a large portion of training data. 
They advocate for researchers to test proposed methods in more realistic situations to better assess the potential threats. 
Furthermore, they encourage exploration of the relationship between poisoning attacks and evasion attacks. 
This could lead to the creation of attacks that produce less noticeable poisoning examples, 
or defensive strategies that can safeguard models against both backdoor and evasion attacks.
%Their survey catalogs and systematizes the threats in the dataset creation process, and discuss the open problems that benefits the understanding of dataset security. 

In addition to undermining model accuracy, 
adversarial attacks also aim at breaching the privacy and confidentiality of training data. 
In particular, membership inference attacks~\cite{Shokri:Stronati:Song:Shmatikov:SP:2017} attempt to determine whether a specific data point was part of the training set used to train the model.
Hu et al.~\cite{Hu:Salcic:Sun:Dobbie:Yu:Zhang:CSUR:2022} present a comprehensive survey of existing research efforts on membership inference attacks. 
They find that, similar to evasion attacks, the membership inference attack success rate decreases as 
%the training data better represents the whole data distribution, i.e., 
the number of training samples increases.
%and model stealing attacks~\cite{Oliynyk:Mayer:Rauber:CSUR:2023} are designed to breach the privacy of training data and machine learning models. 
However, all these attacks are orthogonal to our survey, as we focus on adversarial evasion attacks.

%Li et al. ~\cite{Li:Jiang:Li:Xia:TNNLS:2022} 
%provide the first survey that focuses on backdoor attacks and identified common scenarios in which backdoor attack happen in real life. 
%Furthermore, they proposed a systematic taxonomy for backdoor attacks and defenses for researchers and practitioners to identify the characteristics and limitations of each method. 

%Wang et al.~\cite{Wang:Ma:Wang:Hu:Qin:Ren:CSUR:2022} and Tian et al.~\cite{Tian:Cui:Liang:Yu:CSUR:2022} argue federated learning~\cite{McMahan:Moore:Ramage:Hampson:Arcas:AISTATS:2017} 
%creates new venue for poisoning attack, and survey recent literature on poisoning attacks for both standard and federated learning scenarios. 
%They present a unified framework to categorize both data poisoning and model poisoning attacks, and compared the defense techniques proposed for each of the learning framework, analyzed their advantages and disadvantages.
}

\vspace{-0.1in}
\subsection{Effects of Training Data on Standard Generalization}
\label{sec:relatedwork-standard}
A number of surveys investigate the influence of data properties on standard
rather than robust generalization.
One of the earliest is probably the work of Raudys and Jain~\cite{Raudys:Jain:TPAMI:1991},
who review studies related to the influence of sample size on binary classifiers, showing that
a limited sample size usually leads to sub-optimal generalization.
%With the development of deep learning and the ever-increasing need for larger training datasets,
%a variety of data augmentation techniques have been proposed.
Bansal et al.~\cite{Bansal:Sharma:Kathuria:CSUR:2021} and
Bayer et al.~\cite{Bayer:Kaufhold:Reuter:CSUR:2022} also survey papers addressing the data scarcity problem,
focusing in particular on the recent advancements in data augmentation techniques in the fields of computer vision, security, and text classification.
Their results show that augmentation techniques %exist for various application domain and
can help improve a model's generalization by reducing the problem of model overfitting.
%They evaluate the effectiveness of such techniques in improving the accuracy of machine learning models.

%Limited sample size is also one of the culprit behind poor robust generalization~\cite{Schmidt:Santurkar:Tsipras:Talwar:Madry:NeurIPS:2018}, we collected a number of researches characterize the sample complexity for robust generalization or propose data augmentation techniques to fill in the sample complexity gap.

Label noise is another aspect of data that influences both standard and robust generalization.
Most works on this topic find that the presence of noisy labels increases the need for a greater number of training samples and may result in unnecessarily complex decision boundaries~\cite{Frenay:Verleysen:TNNLS:2014,Song:Kim:Park:Shin:Lee:TNNLS:2022}.
For example, Fr\'{e}nay and Verleysen~\cite{Frenay:Verleysen:TNNLS:2014} show
that overfitting to label noise greatly degrades a model's standard generalization;
the same effect has been observed in the case of robust generalization~\cite{Sanyal:Dokania:Kanade:Torr:ICLR:2021}.
Song et al.~\cite{Song:Kim:Park:Shin:Lee:TNNLS:2022} survey the impact of label noise in deep learning, arguing
that the presence of noisy labels is a more serious concern for deep models as they contain a larger number of parameters which makes them prone to overfitting to the noise in training data.
%They also point out the connection between adversarial poisoning attacks and noisy labels as
%the countermeasures for both share the goal of learning noise-resilient representations.
They mention that adversarial defense techniques, e.g., adversarial training, are effective against label noise~\cite{Zhu:Zhang:Han:Liu:Niu:Yang:Kankanhalli:Sugiyama:ArXiv:2021, Fatras:Damodaran:Lobry:Flamary:Tuia:Courty:TPAMI:2022}
but do not discuss how label noise influences a deep learning model's robustness under attacks.

Lorena et al.~\cite{Lorena:Garcia:Lehmann:Souto:Ho:CSUR:2020} identify a collection of 26 quantitative metrics that measure data complexity with respect to
(1) ambiguity of classes, i.e., whether the classes can be clearly distinguished with the given features,
(2) sparsity and dimensionality of data, 
%i.e., whether enough information are provided to learn confident decision boundaries, and
(3) complexity of boundary separating the classes, i.e., whether more intricate functions are required to describe the decision boundaries.
The authors also discuss how these metrics help estimate the difficulty of performing classification on a given dataset.
Similar to our survey, the authors show that high dimensionality and small separation between classes hinder standard generalization.
However, the relationship of some of the metrics reviewed by these authors, e.g.,
%faction of borderline points (i.e., a measure for the complexity of the required decision boundary) and
%the fraction of hyperspheres covering data (i.e.,
the number of non-intersecting spheres needed to enclose all data points of a class,
to robust generalization is not studied, according to our survey.

%Moreover, the effect of XXX on standard generalization needs future investigation as well (that is if we found something they do not have).

%Knowing the characteristics of a dataset according to these perspectives can assist researchers and practitioners to select optimal learning algorithms~\cite{Ho:Basu:TPAMI:2002}.

He and Garcia~\cite{He:Garcia:TKDE:2009} focus on the imbalance learning problem. %~--
%the disproportion in the number of samples belonging to each class in a given dataset.
The authors found that most standard algorithms %are designed with the assumption of a balanced class distribution.
%These algorithms
fail to reliably represent the characteristics of the imbalanced data and result in unfavorable performance across classes.
Furthermore, L\'{o}pez et al.~\cite{Lopez:Fernandez:Garcia:Palade:Herrera:InfSci:2013} discuss six intrinsic data characteristics that potentially complicate learning from imbalanced data:
low density, sample overlap between classes, noisy data, borderline instances,
dataset shift between training and testing distributions, and
small disjuncts, i.e., disperse small clusters of samples from a single class.
Their analysis concludes that while all these ``unfavorable'' data characteristics further complicate the data imbalance
issues, data overlap between classes is probably one of the most harmful.
To follow up on this point, Santos et al.~\cite{Santos:Henriques:Pedro:Japkowicz:Fernandez:Soares:Wilk:Santos:AIR:2022}
focus on the joint effect of data imbalance and class overlap on model generalization.
The negative impact of data imbalance, low separation, and noisy data on robust generalization was also discussed in our survey.
Yet, the compounding effect of these factors, as well as the effect of other properties,
on robust generalization needs future investigation.

Recently, Yang et al.~\cite{Yang:Jiang:Song:Guo:IJCV:2022} summarized relevant studies focusing on
long-tailed distributions in the field of Computer Vision.
% and categorize the main methods for alleviating the issues caused by long-tailed distribution.
%They present quantitative metrics for measuring data imbalance and .
This survey also includes work on the influence of long-tail distributions on a model's adversarial robustness~\cite{Wu:Liu:Huang:Wang:Lin:CVPR:2021}, which is covered in our survey.
%which is included in our survey,
The authors advocate for more research on adapting long-tailed-based approaches for standard generalization to improve robust generalization.

Finally, Moreno-Torres et al.~\cite{MorenoTorres:Raeder:Rodrigues:Chawla:Herrera:PR:2012} present a unifying framework to categorize existing definitions of dataset shift~-- the case where the joint distribution of inputs and outputs differs between training and testing data.
While ML models are normally trained under the premise that testing data has a similar distribution to the training data,
in reality, the observed data distribution may be different from the historical data that the model is trained on.
Such difference can substantially compromise the quality of model predictions.
The authors analyze the possible causes for dataset shift, e.g., malicious software that evolves over time, and
review the techniques dealing with dataset shift.
They characterize adversarial attacks as one form of dataset shift, where adversaries adaptively
change test instances to create a distribution that differs from training data.
%All works discussed in our survey assumed similar distribution on training and testing data, treating adversarial attacks as the only dataset shift in the problem setup.
%However, in real applications, the underlying data distribution itself can be non-stationary, and the characterize the influence of the dataset shift between training and testing data on the adversarial robustness is yet to be investigated.

\revadd{Overall, despite the similarities with our work, literature discussed in this section focuses on standard generalization while our survey discusses 
the effect of data on robust generalization.}

%More works use the connection between adversarial attacks and distributional shift to analyze the effect of adversaries on generalization performance~\cite{Tu:Zhang:Tao:NeurIPS:2019}.
%However, we do not discuss them in detail, as they focus more on models instead of data.
%\jr{How is that relevant to data properties section?} \gx{This can be removed, as it an individual work we filtered}

\vspace{-0.1in}
\subsection{Summary}
\revadd{
Our survey is the first to explicitly focus on properties of training data in the context of model robustness under evasion attacks.
Numerous other surveys on evasion attacks discuss attack and defense mechanisms, non-data-related reasons for adversarial vulnerability, and the different threat models. 
We identified only five surveys that considered data-related reasons for evasion attacks. 
However, as these surveys are older and do not focus on data in particular, our work provides a more extensive
and comprehensive view on this topic. 
By including more than 50 papers not covered in prior work, we were able to 
identify additional relevant properties, practical suggestions, and future research directions in this area. 

Additional work studies non-data-related reasons for evasion attacks, as well as non-evasion attacks, 
such as poisoning and backdoor. 
Yet another body of literature examines how data properties affect standard generalization. These works show that 
some of the properties discussed in our survey, such as 
the number of samples, dimensionality, and label quality, also affect clean accuracy. 
There are also additional data properties that are covered exclusively by these or by our work. 
Studying the interplay between data properties for clean and robust accuracy is an interesting research direction, 
which could be facilitated by our work. 
However, all these current works are orthogonal and complementary to ours.
}

%\ad{
%The related work of our survey can be categorized into four key topics: 
%The first topic examines data for other adversarial attacks, this include the research that investigates the link between the data characteristics and model's resilience against poisoning attacks as well as the studies that explore data poisoning and backdoor attacks and their countermeasures. \jr{same issues as before: this is meta-summary, we need a concrete summary.}
%These studies complement our survey as they highlight the threats directly aimed at data, thus emphasizing the importance of secure data collection. 
%The second topic focuses on the relationship between various properties of training data and model's standard generalization ability. 
%This body of work suggests that data traits such as number of samples, dimensionality, label quality also influence model's ability to generalize in standard classification. \jr{this looks more concrete!}
%
%The third strand of research concerns adversarial evasion attacks. 
%The work in this area encompasses the research frontier in evasion attacks and the countermeasures. 
%Due to the large volume of work in this area, there are numerous surveys that gives more detail on the advancement. 
%\jr{meta-summary again}
%In addition to attacks and defenses, one relevant line of work investigates the alignment of the conventional similarity metrics used for adversarial examples and human perception, showing the need for supplementary metrics. \jr{why important?}
%These studies \jr{which "these studies"?} collectively present an extensive overview of other types of work conducted on adversarial robustness.
%The last category of work proposes alternative explanations for model vulnerability to adversarial examples.
%These studies presented hypothesis showing the characteristics of machine learning models, e.g., nonlinearity, invariance to rotational shift etc, induces susceptibility to attacks, as well as limited computational resources and non-robust feature representations. \jr{all text based on previous related work looks somewhat concrete; the new additions should be at least at the same level, or better.}
%These studies supplement our work, offering a broader perspective of potential factors affecting model's robust generalization ability. }
%


\section{Conclusion}\label{sec:conclusion}
In this work, we focus on addressing the fundamental challenge of OOD detection tasks, which is how to fully understand the semantic discrepancy between the ID/OOD samples. We reveal that the key to success in the realistic SCOOD task is to allocate as many ID samples in the unlabeled set correctly as possible. To this end, we propose a novel uncertainty-aware optimal transport scheme that introduces class-specific energy scores as guidance for effective label assignment. Experimental results show that our method achieves better performance than previous state-of-the-art methods on SCOOD benchmarks.

\textbf{Limitations.} In addition to temperature scaling, other techniques such as feature clipping applied in ReAct~\cite{sun2021react} also enhance the performance of energy score, so how to obtain an OOD score that best fits the SCOOD task can be further explored. Moreover, a setting highly related to SCOOD has been proposed in \cite{katz2022training} and formulated as a constrained optimization problem. We will also theoretically analyze these practical OOD settings in our feature work.

% \section*{Acknowledgments}
\textbf{Acknowledgments.} 
This work is supported by National Key R\&D Program of China under Grant 2020AAA0105701, National Natural Science Foundation of China (NSFC) under Grants 61872327, Major Special Science and Technology Project of Anhui, National Natural Science Foundation of China (62033012) and Ant Group through Ant Research Intern Program.



% \section{Future Work}
% \begin{itemize}
%     \item{} Reference Counting overhead\\ 
%       Implement Perceus reference counting.
%     \item{} Type-check and bounds-check hoisting and elimination
%     \item{} Implement a more powerful "overlay" language using f-expressions, Racket-style, that implements a strong type system (along the lines of "Type systems as macros" \cite{chang2017type} and "Dependent type systems as macros" \cite{chang2019dependent}) and Algebraic Effects (a la \cite{xie2021generalized}).
% \end{itemize}


%% Acknowledgments
%\begin{acks}                            %% acks environment is optional
                                        %% contents suppressed with 'anonymous'
  %% Commands \grantsponsor{<sponsorID>}{<name>}{<url>} and
  %% \grantnum[<url>]{<sponsorID>}{<number>} should be used to
  %% acknowledge financial support and will be used by metadata
  %% extraction tools.
  %This material is based upon work supported by the
  %\grantsponsor{GS100000001}{National Science
    %Foundation}{http://dx.doi.org/10.13039/100000001} under Grant
  %No.~\grantnum{GS100000001}{nnnnnnn} and Grant
  %No.~\grantnum{GS100000001}{mmmmmmm}.  Any opinions, findings, and
  %conclusions or recommendations expressed in this material are those
  %of the author and do not necessarily reflect the views of the
  %National Science Foundation.
%\end{acks}


%% Bibliography
\bibliography{bibfile}

\section{Appendix for Proofs}

\paragraph{Proof of Theorem \ref{thm:main}.}

\begin{proof}
\label{proof:main}
Our proof has two steps. In Step 1, we will show that SimCLR is equivalent to minimizing the cross entropy loss defined in Eqn.~(\ref{eqn:cross-entropy}). 
In Step 2, we will show  that minimizing the cross-entropy loss 
is equivalent to spectral clustering on $\bfpi$. 
Combining the two steps together, we have proved our theorem. 

\textbf{Step 1: } SimCLR is equivalent to minimizing the cross entropy loss.

The cross-entropy loss takes expectation over 
$\bfW_\bfX\sim \mathbb{P}(\cdot ; \bfpi)$, 
which means $\bfW_\bfX$ has exactly one non-zero entry in each row $i$. By Lemma~\ref{lem:multinomial}, we know every row $i$ of $\bfW_\bfX$ is independent of other rows. Moreover, 
$\bfW_{\bfX,i}\sim \mathcal{M}(1, \bfpi_i/\sum_j \bfpi_{i,j})=\mathcal{M}(1, \bfpi_i)$, because $\bfpi_i$ itself is a probability distribution.
Similarly, we know $\bfW_\bfZ$ also has the row-independent property by sampling over $\mathbb{P}(\cdot;\bfK_\bfZ)$.
Therefore, by Lemma~\ref{lem:cross_split}, we know Eqn.~(\ref{eqn:cross-entropy}) is equivalent to:
\[
 -\sum_{i=1}^n \mathbb{E}_{\bfW_{\bfX,i}}[\log \mathbb{P}(\bfW_{\bfZ,i}=\bfW_{\bfX,i};\bfK_\bfZ)],
\]

This expression takes expectation over $\bfW_{\bfX,i}$ for the given row $i$. Notice that 
$\bfW_{\bfX,i}$ has exactly one non-zero entry, which equals $1$ (same for $\bfW_{\bfZ,i}$). 
As a result
we expand the above expression to be:
\begin{equation}
 -\sum_{i=1}^n \sum_{j\neq i} \Pr(\bfW_{\bfX,i,j}=1)\log \Pr(\bfW_{\bfZ,i,j}=1).
\label{eqn:detailed-expansion}    
\end{equation}


By Lemma~\ref{lem:multinomial}, $\Pr(\bfW_{\bfZ,i,j}=1)=\bfK_{\bfZ,i,j}/\|\bfK_{\bfZ,i}\|_1$ for $j\neq i$. Recall that $\bfK_\bfZ=(k(\bfZ_i-\bfZ_j))_{(i,j)\in[n]^2}$, which means 
$\bfK_{\bfZ,i,j}/\|\bfK_{\bfZ,i}\|_1=\frac{\exp(-\|\bfZ_i-\bfZ_j\|^2/{2\tau})}{\sum_{k\neq i}
\exp(-\|\bfZ_i-\bfZ_k\|^2/{2\tau})
}$ for $j\neq i$, when $k$ is the Gaussian kernel with variance $\tau$. 

Notice that $\bfZ_i=f(\bfX_i)$, so we know
\begin{equation}
-\log \Pr(\bfW_{\bfZ,i,j}=1)=
-\log \frac{\exp(-\|f(\bfX_i)-f(\bfX_j)\|^2/{2\tau})}{\sum_{k\neq i}
\exp(-\|f(\bfX_i)-f(\bfX_k)\|^2/{2\tau}),
}
\label{eqn:infonce-equivalence}    
\end{equation}


The right hand side is exactly the InfoNCE loss defined in Eqn.~(\ref{eqn:infonce}).
Inserting Eqn.~(\ref{eqn:infonce-equivalence}) into Eqn.~(\ref{eqn:detailed-expansion}), we get the SimCLR algorithm, which first samples augmentation pairs $(i,j)$ with $\Pr(\bfW_{\bfX,i,j}=1)$ for each row $i$, and then optimize the InfoNCE loss. 

\textbf{Step 2: } minimizing the cross entropy loss 
is equivalent to spectral clustering on $\bfpi$.


By Lemma~\ref{lem:convert_to_spectral}, we may further convert the loss to 
\begin{equation}
\label{eqn:main-theorem-repul-attr}
\min_{\bfZ}
-\sum_{(i,j)\in [n]^2} \mathbf{P}_{i,j}
\log k (\bfZ_i-\bfZ_j)+\log \mathbf{R}(\bfZ).
\end{equation}
Since $k$ is the Gaussian kernel, this reduces to \[
\min_\bfZ \mathrm{tr}(\bfZ^\top \mathbf{L}(\bfpi) \bfZ)
+\log \mathbf{R}(\bfZ),
\]

where we use the fact that $\mathbb{E}_{\bfW_\bfX\sim \mathbb{P}(\cdot; \bfpi)}[\mathbf{L}(\bfW_\bfX)]
=\mathbf{L}(\bfpi)
$, because the Laplacian operator is linear and $
\mathbb{E}_{\bfW_\bfX\sim \mathbb{P}(\cdot; \bfpi)}(\bfW_\bfX)=\bfpi
$.
\end{proof}

\paragraph{Proof of Theorem \ref{thm:clip}.}
\begin{proof}
Since $\bfW_\bfX\sim \mathbb{P}(\cdot;\bfpi_{\mathbf{A}, \mathbf{B}})$, we know 
$\bfW_\bfX$ has exactly one non-zero entry in each row, denoting the pair that got sampled. 
A notable difference compared to the previous proof is we now have $n_\mathcal{A}+n_\mathcal{B}$ objects in our graph. CLIP deals with this by taking a mini-batch of size $2N$, 
such that $n_\mathcal{A}=n_\mathcal{B}=N$, and adding the $2N$ InfoNCE losses together. We label the objects in $\mathcal{A}$ as $[n_\mathcal{A}]$, and the objects in $\mathcal{B}$ as $\{n_\mathcal{A}+1, \cdots, n_\mathcal{A}+n_\mathcal{B}\}$. 

Notice that $\bfpi_{\mathbf{A}, \mathbf{B}}$ is a bipartite graph, so the edges of objects in $\mathcal{A}$ will only connect to object in $\mathcal{B}$ and vice versa. We can define the similarity matrix in $\cZ$ as $\bfK_\bfZ$, 
where $\bfK_\bfZ(i, j+n_\mathcal{A})=\bfK_\bfZ(j+n_\mathcal{A},i)= k(\bfZ_i-\bfZ_j)$ for $i\in [n_\mathcal{A}], j\in [n_\mathcal{B}]$, and otherwise we set $\bfK_\bfZ(i,j)=0$. 
The rest is same as the previous proof. 
\end{proof}

\paragraph{Proof of Theorem \ref{thm:exponential}.}

\begin{proof}
\label{proof:exponential}
Since the objective function consists of a linear term combined with an entropy regularization, which is a strongly concave function, the maximization problem is a convex optimization problem. Owing to the implicit constraints provided by the entropy function, the problem is equivalent to having only the equality constraint. We then introduce the Lagrangian multiplier $\lambda$ and obtain the following relaxed problem:

$$
\widetilde{E}(\boldsymbol{\alpha})=\psi_{1}-\sum_{i=1}^n \alpha_{i} \psi_{i}+\tau \sum_{i=1}^n \alpha_{i}\log \alpha_{i}+\lambda\left(\boldsymbol{\alpha}^{\top} \mathbf{1}_n-1\right).
$$

As the relaxed problem is unconstrained, taking the derivative with respect to $\alpha_{i}$ yields

$$
\frac{\partial \widetilde{E}(\boldsymbol{\alpha})}{\partial \alpha_{i}}=-\psi_{i}+\tau\left(\log \alpha_{i}+\alpha_{i} \frac{1}{\alpha_{i}}\right)+\lambda=0.
$$

Solving the above equation implies that $\alpha_{i}$ takes the form
$
\alpha_{i}=\exp \left(\frac{1}{\tau} \psi_{i}\right) \exp \left(\frac{-\lambda}{\tau}-1\right).
$ Since $\alpha_{i}$ lies on the probability simplex, the optimal $\alpha_{i}$ is explicitly given by
$
\alpha^{*}_{i}=\frac{\exp \left(\frac{1}{\tau} \psi_{i}\right)}{\sum_{i^{\prime}=1}^n \exp \left(\frac{1}{\tau} \psi_{i^{\prime}}\right)} .
$ Substituting the optimal point into the objective function, we obtain
$$
\begin{aligned}
E\left(\boldsymbol{\alpha}^*\right)  &=\psi_1-\sum_{i=1}^n \frac{\exp \left(\frac{1}{\tau} \psi_{i}\right)}{\sum_{i^{\prime}=1}^n \exp \left(\frac{1}{\tau} \psi_{i^{\prime}}\right)} \psi_{i}+\tau \sum_{i=1}^n \frac{\exp \left(\frac{1}{\tau} \psi_{i}\right)}{\sum_{i^{\prime}=1}^n \exp \left(\frac{1}{\tau} \psi_{i^{\prime}}\right)}\log \frac{\exp \left(\frac{1}{\tau} \psi_{i}\right)}{\sum_{i^{\prime}=1}^n \exp \left(\frac{1}{\tau} \psi_{i^{\prime}}\right)} \\
& =\psi_1 - \tau \log \left(\sum_{i=1}^n \exp \left(\frac{1}{\tau} \psi_{i}\right)\right).
\end{aligned}
$$
Thus, the Lagrangian dual function is given by
\begin{equation*}
-E\left(\boldsymbol{\alpha}^*\right)= -\tau \log \frac{\exp \left(\frac{1}{\tau} \psi_{1}\right)}{\sum_{i=1}^n \exp \left(\frac{1}{\tau} \psi_{i}\right)}.\qedhere
\end{equation*}
\end{proof}



\section{More on Experiments} \label{section: experiment_details}

\paragraph{CIFAR-10 and CIFAR-100} CIFAR-10 ~\citep{krizhevsky2009learning} and CIFAR-100 ~\citep{krizhevsky2009learning} are well-known classic image classification datasets. Both CIFAR-10 and CIFAR-100 contain a total of 60k $32 \times 32$ labeled images of different classes, with 50k for training and 10k for testing. CIFAR-10 is similar to CIFAR-100, except there are 10 different classes in CIFAR-10 and 100 classes in CIFAR-100.

\paragraph{TinyImageNet} TinyImageNet ~\citep{le2015tiny} is a subset of ImageNet ~\citep{deng2009imagenet}. There are 200 different object classes in TinyImageNet, with 500 training images, 50 validation images, and 50 test images for each class. All the images in TinyImageNet are colored and labeled with a size of $64 \times 64$.

\textbf{Pseudo-code.} Algorithm \ref{alg:Training Procedure} presents the pseudo-code for our empirical training procedure.

\begin{algorithm}[!htbp]
\caption{Training Procedure}
\label{alg:Training Procedure}
\begin{algorithmic}[1]
\REQUIRE trainable encoder network $f$, batch size $N$, augmentation strategy \textit{aug}, loss function $L$ with hyperparameters \textit{args}
\FOR {sampled minibatch ${x_i}_{i=1}^N$}
\FORALL{$i \in { 1, ..., N }$}
\STATE draw two augmentations $t_i = \textit{aug}\left(x_i\right) $, $t_i' = \textit{aug}\left(x_i\right) $
\STATE $z_i = f\left(t_i\right)$, $z_i' = f\left(t_i'\right)$
\ENDFOR
\STATE compute loss $\mathcal{L} = L(N, z, z', \textit{args})$
\STATE update encoder network $f$ to minimize $\mathcal{L}$
\ENDFOR
\STATE \textbf{Return} encoder network $f$
\end{algorithmic}
\end{algorithm}

We also provide the pseudo-code for our core loss function used in the training procedure in Algorithm \ref{alg:Core loss}. The pseudo-code is almost identical to SimCLR's loss function, with the exception of an extra parameter $\gamma$.

\begin{algorithm}[!htbp]
\caption{Core loss function $\mathcal{C}$}
\label{alg:Core loss}
\begin{algorithmic}[1]
\REQUIRE batch size $N$, two encoded minibatches $z_1, z_2$, $\gamma$, temperature $\tau$
\STATE $z = \textit{concat}\left(z_1, z_2\right)$
\FOR {$i \in {1, ..., 2N }, j \in {1, ..., 2N}$ }
\STATE $s_{i,j} = \Vert z_i - z_j \Vert_2^{\gamma}$
\ENDFOR
\STATE \textbf{define} $l(i, j)$ \textbf{as} $l(i, j) = - \log \frac{exp\left(s_{i,j}/\tau \right)}{\sum_{k=1}^{2N} \mathbf{1}{[k \ne i]} exp\left(s{i, j} / \tau \right)} $
\STATE \textbf{Return} $\frac{1}{2N} \sum_{k=1}^N\left[l(i, i+N) + l(i+N, i)\right]$
\end{algorithmic}
\end{algorithm}

Utilizing the core loss function $\mathcal{C}$, we can define all kernel loss functions used in our experiments in Table \ref{table: loss definition}. For all $z_i \in z$ with even dimensions $n$, we define $z_{L_i} = z_i\left[0:n/2\right]$ and $z_{R_i} = z_i\left[n/2:n\right]$.

\begin{table}[ht]
\centering
\begin{tabular}{{@{}l|l@{}}}
Kernel  &  Loss function \\ \midrule
Laplacian & $\mathcal{C}\left(N, z, z', \gamma=1, \tau\right)$\\ \midrule
Sum       & $\lambda * \mathcal{C}\left(N, z, z', \gamma=1, \tau_1\right) + (1-\lambda) * \mathcal{C}\left(N, z, z', \gamma=2, \tau_2\right)$  \\ \midrule
Concatenation Sum&$\lambda * \mathcal{C}\left(N, z_L, z'_L, \gamma=1, \tau_1\right) + (1-\lambda) * \mathcal{C}\left(N, z_R, z'_R, \gamma=2, \tau_2\right)$\\ \midrule
$\gamma = 0.5$ & $\mathcal{C}\left(N, z, z', \gamma=0.5, \tau\right)$          \\ 

\end{tabular}

\caption{Definition of kernel loss functions in our experiments}
\label {table: loss definition}
\end{table}

\textbf{Baselines.} We reproduce the SimCLR algorithm using PyTorch Lightning~\citep{PytorchLightning}.

\textbf{Encoder details.}
The encoder $f$ consists of a backbone network and a projection network. We employ ResNet50~\citep{ResNet} as the backbone and a 2-layer MLP (connected by a batch normalization~\citep{ioffe2015batch} layer and a ReLU \cite{nair2010rectified} layer) with hidden dimensions 2048 and output dimensions 128 (or 256 in the concatenation kernel case).

\textbf{Encoder hyperparameter tuning.}
For each encoder training case, we randomly sample 500 hyperparameter groups (sample details are shown in Table \ref{table: Hyperparameter sample}) and train these samples simultaneously using Ray Tune ~\citep{RayTune}, with the ASHA scheduler~\citep{li2018massively}. Ultimately, the hyperparameter group that maximizes the online validation accuracy (integrated in PyTorch Lightning) within 5000 validation steps is chosen for the given encoder training case.

\begin{table}[ht]
\centering

\begin{tabular}{@{}l|l|l@{}}
\midrule
Hyperparameter  & Sample Range & Sample Strategy \\ \midrule
start learning rate & $\left[10^{-2}, 10\right]$ & log uniform \\ \midrule
$\lambda$       & $\left[0, 1\right]$ & uniform \\ \midrule
$\tau$, $\tau_1$, $\tau_2$ & $\left[0, 1\right]$ & log uniform \\ \midrule
\end{tabular}

\caption{Hyperparameters sample strategy}
\label {table: Hyperparameter sample}
\end{table}

\textbf{Encoder training.} 
We train each encoder using the LARS optimizer~\citep{LARSOptimizer}, LambdaLR Scheduler in PyTorch, momentum 0.9, weight decay $10^{-6}$, batch size 256, and the aforementioned hyperparameters for 400 epochs on a single A-100 GPU.

\textbf{Image transformation.} The image transformation strategy, including augmentation, is identical to the default transformation strategy provided by PyTorch Lightning.

\textbf{Linear evaluation.}
The linear head is trained using the SGD optimizer with a cosine learning rate scheduler, batch size 64, and weight decay $10^{-6}$ for 100 epochs. The learning rate starts at $0.3$ and ends at $0$.

\textbf{Moco Experiments.} We also tested our method based on MoCo~\citep{he2019moco}. The results are summarized in Table \ref{tab:results-moco}. Here we choose ResNet18~\citep{ResNet} as the backbone and set a temperature of $0.1$ as default. For our simple sum kernel, we set $\lambda=0.8$. The results show that our method outperforms the original MoCo method.

\begin{table}[thb]
\centering
\caption{MoCo Experiment Results on CIFAR-10 and CIFAR-100.}
\label{tab:results-moco}
\resizebox{\textwidth}{!}{%
\begin{tabular}{@{}c|ccc|ccc@{}}
\toprule
\multirow{3}{*}{Method} & \multicolumn{3}{c|}{CIFAR-10} & \multicolumn{3}{c}{CIFAR-100} \\ \cmidrule(lr){2-4} \cmidrule(lr){5-7} 
                        & 200 epochs & 400 epochs    & 1000 epochs   & 200 epochs & 400 epochs & 1000 epochs         \\ \midrule
MoCo (repro.)         & $76.41 \pm 0.12$    & $80.01 \pm 0.15$          & $84.45 \pm 0.08$    & $\mathbf{47.02 \pm 0.11}$ & $52.50 \pm 0.07$ & $57.62 \pm 0.15$            \\
\midrule
Laplacian Kernel        & ${78.09 \pm 0.10}$    & $\mathbf{83.85 \pm 0.09}$          & $\mathbf{88.34 \pm 0.16}$    & $46.12 \pm 0.22$   & $53.44 \pm 0.17$ & $59.10 \pm 0.14$        \\
Simple Sum Kernel & $\mathbf{78.12 \pm 0.15}$   & $83.23 \pm 0.18$ & $87.50 \pm 0.20$ & $46.65 \pm 0.06$ & $\mathbf{53.62 \pm 0.19}$ & $\mathbf{59.83 \pm 0.12}$\\
\bottomrule
\end{tabular}
}
\end{table}



\section{More Experiments on Synthetic Data}


Consider a scenario with $n$ clusters, each containing $k$ vertices. Let the probability of vertices $u$ and $v$ from the same cluster belonging to $\bfpi$ be $p$. Conversely, for vertices $u$ and $v$ from different clusters, let the probability of belonging to $\pi$ be $q$. We generate the graph $\bfpi$ randomly, based on $p$ and $q$. We experiment with values of $k=100$ and $n=6$ for ease of visualization, embedding all points in a two-dimensional space. Each vertex's initial position originates from a normal distribution. In each iteration, we sample a subgraph of $\bfpi$ uniformly, ensuring each vertex has an out-degree of $1$. We then optimize the corresponding vectors using InfoNCE loss with an SGD optimizer and iterate until convergence. Our experimental setup consists of an SGD learning rate of $1$, an InfoNCE loss temperature of $0.5$, and a batch size of $50$. We evaluate two scenarios with different $p$ and $q$ values: $p=1$, $q=0$, and $p=0.75$, $q=0.2$. The results of these experiments are visualized in Figure \ref{fig:vis-spectral-cluster}. The obtained embeddings exhibit the hallmark pattern of spectral clustering of graph $\bfpi$.

\begin{figure}[!tb]
\centering
\subfigure{
\includegraphics[width=1\textwidth]{Figures/cluster_pi.png}
\label{fig:vis-cluster}
}
\subfigure{
\includegraphics[width=1\textwidth]{Figures/noised_cluster_pi.png}
\label{fig:vis-noised-cluster}
}
\caption{Visualizations of the optimization process using InfoNCE Loss on the vectors corresponding to $\bfpi$. Points of identical color belong to the same cluster within $\bfpi$. To showcase the internal structure of $\bfpi$, we randomly select 10 vertices from each cluster to display the edge distribution of $\bfpi$.}
\label{fig:vis-spectral-cluster}
\end{figure}



\end{document}
