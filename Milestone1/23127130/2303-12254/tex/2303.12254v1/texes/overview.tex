\section{General Compilation Framework}
\label{sec:overview}

\begin{figure}[!ht]
\includegraphics[width=\textwidth]{overview.png}
	\caption{The Compilation Framework (including partial evaluation and compiler optimizations) compiles \krakenSpace code into a WebAssembly binary that (for our benchmarks) executes on either the WAVM WebAssembly engine or the Wasmtime WebAssembly engine}
\label{fig:overview}
\end{figure}

The \krakenSpace language gets compiled into WebAssembly before it is executed on either the Wasmtime WebAssembly engine or WAVM WebAssembly engine as seen in Fig.~\ref{fig:overview}.
Initially, the \krakenSpace code is parsed to create an abstract syntax tree (AST).
The AST gets 'marked', adding bookkeeping information needed for partial evaluation to the AST, and is then 'unvaled', annotating the sections of the AST which will be evaluated as suspended computations.
Once the AST is unvaled, the AST is partially evaluated by checking each suspended computation to see if it can be evaluated based on the current (generally partially-static-data) environment.
While the compiler backend walks the AST and emits WebAssembly bytecode, it performs optimizations like type-inference-based primitive inlining, single-use-closure inlining, lazy-environment-creation, and basic tail call elimination to further improve code quality for better performance.
This bytecode can then be executed on any compliant WebAssembly engine.

