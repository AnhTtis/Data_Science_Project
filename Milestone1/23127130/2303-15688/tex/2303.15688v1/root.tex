\documentclass[letterpaper, 10 pt, conference]{ieeeconf} 

\IEEEoverridecommandlockouts         
\overrideIEEEmargins                

\usepackage{amssymb}
\usepackage{amsmath}
\usepackage{amsfonts}       % blackboard math symbols
\usepackage{booktabs}       % professional-quality tables
\usepackage{balance}
\usepackage[utf8]{inputenc} % allow utf-8 input
\usepackage[T1]{fontenc}    % use 8-bit T1 fonts
%\usepackage{hyperref}       % hyperlinks
\usepackage[pdftex, pdfstartview={FitV}, pdfpagelayout={TwoColumnLeft},bookmarksopen=true,plainpages = false, colorlinks=true, linkcolor=black, citecolor = black, urlcolor = black,filecolor=black , pagebackref=false,hypertexnames=false, plainpages=false, pdfpagelabels,bookmarks=false ]{hyperref}
%\usepackage{balance}
%\usepackage{flushend}
\usepackage{url}            % simple URL typesetting
\usepackage{nicefrac}       % compact symbols for 1/2, etc.
\usepackage{microtype}      % microtypography
%\usepackage[font=scriptsize]{caption}
%\usepackage{caption}
\usepackage{subcaption}
\usepackage{graphicx}
\usepackage{epstopdf} % needs to be AFTER graphicxs
\usepackage{mathtools}
\usepackage{marginnote}
\usepackage[dvipsnames]{xcolor}
\usepackage{float}
\usepackage[capitalize]{cleveref}
\usepackage[]{algorithmic}
\usepackage{makecell}
\usepackage{multirow}
\usepackage{paralist}
\floatstyle{boxed} \floatname{algorithm}{Algorithm}
\newfloat{algorithm}{t}{loa}[section]
\usepackage{xcolor}
\usepackage[sort,compress]{cite}
\usepackage{soul}

\usepackage{color}
\usepackage{soul}
\usepackage{xspace}
\usepackage[capitalize]{cleveref}
\usepackage{dsfont}
\usepackage{bm}
\usepackage{bbm}
\usepackage{nicefrac}  % compact symbols for 1/2, etc.
%\usepackage{amsthm}
\usepackage{amsfonts}
\usepackage{amssymb}
\usepackage{array}
\usepackage{mathtools}
%\usepackage{caption}
%\usepackage{subcaption}
\usepackage{fontawesome}
\usepackage{thmtools}
\usepackage{thm-restate}
\usepackage{lipsum}
\usepackage{pifont}
\usepackage{makecell}
\usepackage{multirow}
\usepackage{afterpage}
\usepackage{tabularx,balance} 
\usepackage{url}
\usepackage[normalem]{ulem}
%\usepackage{enumitem}  % For removing indent in the numbered list
\usepackage{wrapfig}  % For wrapping figure
\usepackage{textcomp}

\usepackage{caption}
\usepackage{subcaption}
%\usepackage{kantlipsum} %<- For dummy text
%\usepackage{mwe}
\DeclareCaptionLabelSeparator{periodspace}{.\quad}
\captionsetup{font=footnotesize,labelsep=periodspace,singlelinecheck=false}
\captionsetup[sub]{font=scriptsize,singlelinecheck=true}

\definecolor{purple}{rgb}{1, 0, 1}

\newcommand{\ie}{\emph{i.e.,}\xspace}
\newcommand{\eg}{\emph{e.g.,}\xspace}
\newcommand{\abr}{\emph{abbr.}\xspace}
\newcommand{\ea}{\emph{et al.}\xspace}
\newcommand{\gensync}{\emph{GenSync}\xspace}
\newcommand{\colosseum}{\emph{Colosseum}\xspace}
\newcommand{\srep}{\emph{SREP}\xspace} % Set Reconciliation Enhances
\newcommand{\srepsim}{\emph{SREPSim}\xspace}
% Propagation
\newcommand{\esrep}{\emph{E-SREP}\xspace}
\newcommand{\epsrep}{\emph{EP-SREP}\xspace}
\newcommand{\mesrep}{\emph{ME-SREP}\xspace}
\newcommand{\mempoolsync}{\emph{MempoolSync}}

\newcommand{\fref}[1]{Fig.~\ref{#1}}
\newcommand{\tref}[1]{Table~\ref{#1}}
\newcommand{\aref}[1]{Algorithm~\ref{#1}}
\newcommand{\procref}[1]{Procedure~\ref{#1}}
\newcommand{\sref}[1]{Section~\ref{#1}}
\newcommand{\lineref}[1]{line~\ref{#1}}
\newcommand{\appref}[1]{Appendix~\ref{#1}}

% Change \eqref
\LetLtxMacro{\originaleqref}{\eqref}
\renewcommand{\eqref}{Eq.~\originaleqref}

% Theorems and corollaries
\newcounter{theoremcount}
\setcounter{theoremcount}{0}
\DeclareRobustCommand{\theorem}[1]{%
  \refstepcounter{theoremcount}%
  \noindent\textit{\textbf{Theorem \thetheoremcount\label{theorem:#1}: }}%
}
\DeclareRobustCommand{\theoremref}[1]{Theorem~\ref{theorem:#1}}

\DeclareRobustCommand{\proof}{\emph{Proof:}\xspace}
\DeclareRobustCommand{\qqed}{\hfill$\blacksquare$}

\newcounter{corollcount}
\setcounter{corollcount}{0}
\DeclareRobustCommand{\coroll}[1]{%
  \refstepcounter{corollcount}%
  \noindent\textit{\textbf{Corollary \thecorollcount\label{coroll:#1}: }}%
}
\DeclareRobustCommand{\corollref}[1]{Corollary~\ref{coroll:#1}}

\newcounter{lemmacount}
\setcounter{lemmacount}{0}
\DeclareRobustCommand{\lemma}[1]{%
  \refstepcounter{lemmacount}%
  \noindent\textit{\textbf{Lemma \thelemmacount\label{lemma:#1}: }}%
}
\DeclareRobustCommand{\lemmaref}[1]{Lemma~\ref{lemma:#1}}

\newcounter{definitioncount}
\setcounter{definitioncount}{0}
\DeclareRobustCommand{\definition}[1]{%
  \refstepcounter{definitioncount}%
  \noindent\textit{\textbf{Definition \thedefinitioncount\label{definition:#1}: }}%
}
\DeclareRobustCommand{\defref}[1]{Definition~\ref{definition:#1}}

%notes of different authors
\newif\ifnotes
\notestrue
\notesfalse

\newif\ifdiff
\difftrue
\difffalse

\newcommand{\anote}[1]{\ifnotes $\ll$\textsf{\textcolor{purple}{Ari: {#1}}}$\gg$ \fi}
\newcommand{\nnote}[1]{\ifnotes $\ll$\textsf{\textcolor{orange}{Novak: {#1}}}$\gg$ \fi}
\newcommand{\diff}[1]{\ifdiff\textcolor{orange}{#1}\else#1\fi}

%%% Local Variables:
%%% mode: latex
%%% TeX-master: "main"
%%% End:


\title{\LARGE \bf
Efficient Deep Learning of Robust, Adaptive Policies \\ using  Tube MPC-Guided Data Augmentation}


\author{Tong Zhao$^{1,*}$, Andrea Tagliabue$^{2,*}$, Jonathan P. How$^{2}$% <-this % stops a space
\thanks{*Equal contribution}% <-this % stops a space
\thanks{$^{1}$ Department of Electrical Engineering and Computer Science, Massachusetts Institute of Technology. \texttt{tzhao@mit.edu}}%
\thanks{$^{2}$ Department of Aeronautics and Astronautics, Massachusetts Institute of Technology. \tt\{atagliab, jhow\}@mit.edu}%
\thanks{This work was funded by the Air Force Office of Scientific Research MURI FA9550-19-1-0386.}
}


\begin{acronym}
    \acro{LDA}{\emph{Latent Dirichlet Allocation}}
    \acro{CMT}{\emph{Conference Management Toolkit}}
    \acro{TPMS}{\emph{The Toronto Paper Matching System}}
    \acro{MCMF}{\emph{MinCost-MaxFlow}}
\end{acronym}

\begin{document}

\maketitle
\thispagestyle{empty}
\pagestyle{empty}


%%%%%%%%%%%%%%%%%%%%%%%%%%%%%%%%%%%%%%%%%%%%%%%%%%%%%%%%%%%%%%%%%%%%%%%%%%%%%%%%
\begin{abstract}
The deployment of agile autonomous systems in challenging, unstructured environments requires adaptation capabilities and robustness to uncertainties. Existing robust and adaptive controllers, such as the ones based on \ac{MPC}, can achieve impressive performance at the cost of heavy online onboard computations. Strategies that efficiently learn robust and onboard-deployable policies from \ac{MPC} have emerged, but they still lack fundamental adaptation capabilities. In this work, we extend an existing efficient \ac{IL} algorithm for robust policy learning from \ac{MPC} with the ability to learn policies that adapt to challenging model/environment uncertainties. The key idea of our approach consists in modifying the \ac{IL} procedure by conditioning the policy on a learned lower-dimensional model/environment representation that can be efficiently estimated online. We tailor our approach to the task of learning an adaptive position and attitude control policy to track trajectories under challenging disturbances on a multirotor. Our evaluation is performed in a high-fidelity simulation environment and shows that a high-quality adaptive policy can be obtained in about $1.3$ hours. We additionally empirically demonstrate rapid adaptation to in- and out-of-training-distribution uncertainties, achieving a $6.1$ cm average position error under a wind disturbance that corresponds to about $50\%$ of the weight of the robot and that is $36\%$ larger than the maximum wind seen during training.
\end{abstract}


%%%%%%%%%%%%%%%%%%%%%%%%%%%%%%%%%%%%%%%%%%%%%%%%%%%%%%%%%%%%%%%%%%%%%%%%%%%%%%%%
\section{Introduction} \label{sec:introduction}
\documentclass[../main.tex]{subfiles}
\begin{document}

Magnetically actuated medical robots (MAMR) have seen significant focus and development in recent decades due to their potential for miniaturization~\cite{hu2018small}, tether-less actuation~\cite{popek2016six} and high number of controllable degrees-of-freedom (DOFs)~\cite{salmanipour2018eight,pittiglio2022collaborative}. In fact, magnetically guided catheters have been used to treat cardiac arrhythmias since 2003~\cite{nelson2022magnetically,carpi2009stereotaxis}.

A key aspect in their actuation is pose estimation~\cite{bianchi2019localization,barducci2019adaptive}, enabling closed loop control and delivery of functionality~\cite{norton2019intelligent}. Imaging techniques have long been used for this purpose but are generally tied to limited resolution, harmful radiation exposure and need for additional hospital equipment~\cite{aziz2020medical,pane2022ultrasound,daguerre2022localization}. As such, methods based on magnetic field measurements have received significant attention, with magnetic tracking systems being widely available on the market. These, however, are not compatible with magnetic actuation systems due to distortions on the localization magnetic fields.

To address this issue, significant research on magnetic localization coupled with magnetic actuation systems has been done~\cite{khalil2019magnetic,popek2016six,son2018simultaneous,shao2019novel,taddese2018enhanced}. Several works have been based on magnetic field sensing arrays external to MAMR~\cite{micheal20222d,son2018simultaneous}. While advantageous from a miniaturization and internal power consumption point of view, these systems require calibration of large sensor arrays and have limited localization workspace dimensions. Internal sensing to the MAMR, on the other hand, does not suffer from workspace dimension restrictions. It requires, however, on-board power and heterogeneous localization magnetic fields, with 6-DOF localization having been shown for systems with a single external permanent magnet (EPM). \textcolor{black}{Internal sensing methods have been shown for endoscopic capsules, as well as magnetically guided catheters}~\cite{popek2016six,sperry2022six,taddese2018enhanced,fischer2022using}.

Over recent years the need for enhanced control and manipulability of MAMRs has led to the advent of actuation platforms based on multiple magnetic field sources (MMFS) such as multiple electromagnetic coils and multiple permanent magnets~\cite{kummer2010octomag,hoang2019independent,hong2020magnetic,pittiglio2022collaborative,ryan2017magnetic,stereotaxis_patent}. Some of these platforms have been cleared for human use such as Stereotaxis Genesis RMN\textsuperscript{\tiny\textregistered} based on two permanent magnets, and Magnetecs and Aeon Scientific based on multiple electromagnetic coils. 

Despite this progress, magnetic localization for such systems is lagging behind, with fluoroscopic imaging being currently used~\cite{nelson2022magnetically}. Unlike single magnetic field source systems where the singularity regions and localization limitations have been thoroughly investigated and solved for~\cite{taddese2018enhanced}, magnetic localization for MMFS systems suffers from additional challenges due to the superposition of the magnetic fields leading to configuration-specific singularity regions. Only recently, a 3D position localization system with internal magnetic field sensing was demonstrated for a multi-coil system, \textcolor{black}{for a 3~mm catheter}~\cite{fischer2022using}.

Furthermore, a common conundrum in 6-DOF magnetic localization with internal sensing is finding the rotation around gravity, due to the absence of the Earth's magnetic field measurement~\cite{mahony2008nonlinear}. This has been solved in the past by accurately initializing this missing rotation angle and tracking it with a gyroscope~\cite{pittiglio2020observability, di2016jacobian}. However, this is prone to errors over time, especially for slow moving systems where gyroscope data is not as sensitive. Additionally, if communication to the MAMR is lost, a new accurate initialization is needed, proving impossible mid medical intervention. More recently, Taddese et al.~\cite{taddese2018enhanced} fitted an auxiliary coil around a single EPM providing a second set of magnetic field measurements. This solves the missing rotation angle and is also able to eliminate the localization singularity plane when it comes to localization with respect to a single EPM. However, when MMFS are present in the workspace, that singularity plane ceases to exist due to the superposition of magnetic fields, and instead singularity regions are present depending on the relative pose of each EPM.

This paper introduces, for the first time, a 6-DOF magnetic localization method for systems with multiple EPMs without the need for any prior pose information. The method relies on measurements from an accelerometer and a single 3D magnetic field Hall effect sensor (HE), both internal to the MAMR. We analyze the effect that the number of EPMs in the workspace has on the full pose estimation; and demonstrate its performance in a two EPM magnetic actuation platform. Since adding an orthogonal coil is not able to solve for the singularity regions, in this work we do not consider it and instead solve for the missing rotation angle by using multiple magnetic field measurements at different EPM configurations. This works for static or quasi-static systems, with maximum MAMR velocity highly dependent on the actuation system and the magnetic field generated. This is the case for non-actuated parts of a larger system, such as the deployment point at the tip of an endoscope, or for MAMRs while the generated magnetic fields are sufficiently weak to induce actuation. Additionally, unlike common works in literature which parameterize the rotation matrix, in this work the full 6-DOF pose is estimated directly in the special euclidean group $SE(3)$. This avoids any singularities or non-unique representations of the orientation when using Euler angles or quaternions~\cite{mathavaraj2021se,mayhew2011quaternion,taddese2018enhanced}.

\end{document}


\section{Related Works} \label{sec:relaed_works}
\setlength{\tabcolsep}{1.6mm}{
\renewcommand\arraystretch{1.1}
\begin{table}[ht]
  \centering
  \scalebox{0.9}{
  \begin{tabular}{llcccc}
    \toprule
    &\multirow{2}*{Methods} & \multirow{2}*{Sal.} &   \multicolumn{2}{c}{VOC} & MS~COCO \\
    \cmidrule(r){4-5}\cmidrule(r){6-6}
    &&&\texttt{val}&\texttt{test}&\texttt{val}\\
    \hline
    \multirow{13}*{\rotatebox{90}{ResNet-50}}
    &IRN~\cite{irn}          \tiny{CVPR'19}     &              & 63.5       & 64.8          & 42.0  \\
    &LayerCAM~\cite{layercam}\tiny{TIP'21}      &              & 63.0       & 64.5          & -     \\
    &AdvCAM~\cite{advcam}    \tiny{CVPR'21}     &              & 68.1       & 68.0          & 44.2  \\
    &RIB~\cite{rib}          \tiny{NeurIPS'21}  &              & 68.3       & 68.6          & 44.2  \\
    &ReCAM~\cite{recam}      \tiny{CVPR'22}     &              & 68.5       & 68.4          & 42.9  \\
    % \rowcolor{Gray}
    &\cellcolor{Gray}IRN+\texttt{LPCAM}    &\cellcolor{Gray} & \cellcolor{Gray}68.6    & \cellcolor{Gray}68.7      & \cellcolor{Gray}44.5  \\
    &SIPE~\cite{sipe}        \tiny{CVPR'22}     &              & 68.8       & 69.7          & 40.6  \\
    &OOD~\cite{ood}+Adv      \tiny{CVPR'22}     &              & 69.8       & 69.9          & -     \\
    &AMN~\cite{amn}          \tiny{CVPR'22}     &              & 69.5       & 69.6          & 44.7  \\
    &\cellcolor{Gray}AMN+\texttt{LPCAM}    &\cellcolor{Gray} & \cellcolor{Gray}70.1    &\cellcolor{Gray} 70.4      & \cellcolor{Gray}45.5  \\ 
    &ESOL~\cite{esol}        \tiny{NeurIPS'22}  &              & 69.9$^*$   & 69.3$^*$      & 42.6  \\
    &CLIMS~\cite{clims}      \tiny{CVPR'22}     &              & 70.4$^*$   & 70.0$^*$      & -     \\
    &EDAM~\cite{edam}        \tiny{CVPR'21}     &\checkmark    & 70.9$^*$   & 71.8$^*$      & -     \\
    &\cellcolor{Gray}EDAM+\texttt{LPCAM}  &\cellcolor{Gray}\checkmark & \cellcolor{Gray}71.8$^*$ &\cellcolor{Gray} 72.1$^*$& \cellcolor{Gray}42.1\\
    \hline
    \multirow{9}*{\rotatebox{90}{WideResNet-38}}
    &Spatial-BCE~\cite{sbce} \tiny{ECCV'22}     &              & 70.0       & 71.3      & 35.2  \\
    &BDM~\cite{bdm}          \tiny{ACMMM'22}    &\checkmark    & 71.0       & 71.0      & 36.7  \\ 
    &RCA~\cite{rca}+OOA      \tiny{CVPR'22}     &\checkmark    & 71.1       & 71.6      & 35.7  \\
    &RCA~\cite{rca}+EPS      \tiny{CVPR'22}     &\checkmark    & 72.2       & 72.8      & 36.8  \\
    &HGNN~\cite{hgnn}        \tiny{ACMMM'22}    &\checkmark         & 70.5$^*$   & 71.0$^*$  & 34.5  \\ 
    &EPS~\cite{eps}          \tiny{CVPR'21}     &\checkmark         & 70.9$^*$   & 70.8$^*$  & -     \\
    &RPIM~\cite{rpim}        \tiny{ACMMM'22}    &\checkmark         & 71.4$^*$   & 71.4$^*$  & -     \\ 
    &L2G~\cite{l2g}          \tiny{CVPR'22}     &\checkmark         & 72.1$^*$   & 71.7$^*$  & 44.2  \\
    \hline
    \multirow{2}*{\rotatebox{90}{\small{DeiT-S}}}
    &MCTformer~\cite{mctformer}    \tiny{CVPR'22}     &                 & 71.9$^{\dag}$  & 71.6$^{\dag}$   & 42.0  \\
    &\cellcolor{Gray}MCTformer+\texttt{LPCAM}      &\cellcolor{Gray} & \cellcolor{Gray}72.6$^{\dag}$  & \cellcolor{Gray}72.4$^{\dag}$  &\cellcolor{Gray} 42.8 \\
    \bottomrule
  \end{tabular}}
  \vspace{-2mm}
  \caption{The mIoU results (\%) based on DeepLabV2 on VOC and MS~COCO. The side column shows three backbones of multi-label classification model. ``Sal.'' denotes using saliency maps. * denotes the segmentation model is pre-trained on MS~COCO. $^\dag$ denotes the segmentation model is pre-trained on VOC.
  }
  \vspace{-6mm}
  \label{table_related}
\end{table}
}



\section{Preliminaries} \label{sec:preliminaries}
\section{Notation and Preliminaries}\label{sec_prel}
Let $\mathbb{Z}_{>0}$ denote the set of positive integers and let $\mathbb{Z}_{[a,b]}$ denote the set of integers in the interval $[a,b]$. The $m\times m$ identity matrix is denoted by $I_m$ and its columns by $e_i$ for $i\in\mathbb{Z}_{[1,m]}$. We use $\mathbf{0}$ to denote a vector or a matrix of zeros of appropriate dimensions. For a sequence $\{z_k\}_{k=0}^{N-1}$ with $z_k\in\mathbb{R}^\eta$, we denote its stacked vector as $z = \begin{bmatrix}z_0^\top &z_1^\top & \dots & z_{N-1}^\top\end{bmatrix}^\top$ and a stacked window of it as $z_{[l,j]} = \begin{bmatrix}z_l^\top &z_{l+1}^\top & \dots & z_{j}^\top\end{bmatrix}^\top$ with $0\leq l<j$.\par
Persistence of excitation of a sequence and its extension to multiple sequences \cite{vanWaarde20} are defined as follows.
\begin{definition} The sequence \(\{z_k\}_{k=0}^{N-1}\), $z_k\in\mathbb{R}^{\eta}$, is said to be persistently exciting of order \(L\) if \(\textup{rank}(\mathscr{H}_{L}(z))=\eta L\), where $\mathscr{H}_L(z) = \begin{bmatrix}
		z_{[0,L-1]} & z_{[1,L]} & \cdots & z_{[N-L,N-1]}
	\end{bmatrix}$.
	\label{def_PE}
\end{definition}
\begin{definition}[\cite{vanWaarde20}]\label{def_cPE}
	The sequences $\{z_k^{(j)}\}_{k=0}^{N_j-1}$, with $z_k^{(j)}\in\mathbb{R}^\eta$ and $j\in\mathbb{Z}_{[1,r]}$, are said to be \textit{collectively persistently exciting} of order $L$ if rank$(\mathcal{H}_L(\mathscr{Z}))=\eta L$, where $\mathscr{Z} = \begin{bmatrix}
		(z^{(1)})^\top & \cdots & (z^{(r)})^\top
	\end{bmatrix}^\top,$ and
	\begin{equation*}
		\mathcal{H}_L(\mathscr{Z}) = \begin{bmatrix}
			\mathscr{H}_L(z^{(1)}) & \cdots & \mathscr{H}_L(z^{(r)})
		\end{bmatrix}.
	\end{equation*}
\end{definition}

\section{Approach} \label{sec:approach}
\begin{figure*}
    % [trim = {left, bottom, right, top}, clip]
    \centering\includegraphics[width=0.9\textwidth,trim = {0, 20, 45, 0}, clip]{figs/mpc_rma_diagram/mpc_rma_diagram_v4.pdf}
    \vspace{-1.5ex}
    \caption{Schematic representation of \acf{SAMA}, our proposed approach for efficient learning of adaptive polices from \ac{MPC}. The key idea of \ac{SAMA} consists in leveraging an efficient Imitation Learning strategy, \acf{SA} \cite{tagliabue2022demonstration}, to collect demonstrations and perform data augmentation using a Robust Tube \ac{MPC}. This efficiently generated data is used to train a student policy conditioned on a latent representation $z_t$ of environment and robot parameters $e_t$. Following the \acf{RMA} \cite{kumar2021rma} procedure, we then train an adaptation module that can produce an estimate $\hat{z}_t$ of these environment parameters from a sequence of past states and actions. This approach enables efficient learning of a robust, adaptive policy from \ac{MPC} without leveraging \ac{RL}, avoiding any reward tuning and making use of available priors on the model of the robot.}
    \label{fig:approach_diagram}
    \vskip-3ex
\end{figure*}

The proposed approach, summarized in \cref{fig:approach_diagram}, consists in a \textit{three phase} policy learning procedure: 

\subsection{Phase 0: Robust Tube MPC Design}
\label{eqn:adaptive-rtmp}
As in \ac{RMA}, we train our policies in a simulation environment implementing the full nonlinear dynamic model of the robot/environment, with parameters (model/environment uncertainties, disturbances...) captured by the environment parameter vector $e$. At each timestep $t$, each entry in $e$ may change with some probability $p$, with entries changing independently of each other (see Table \ref{table:env-params} for more details on the distributions of $e$ in the train and test environments). Whenever $e$ changes, we update the \ac{RTMPC}, described in \cref{sec:rtmpc_design}, as follows. 
First, since the controller uses linear system dynamics, for a given environment parameter vector $e_t$ at time $t$ we compute a discrete-time linear system by discretizing and linearizing the full nonlinear system dynamics, obtaining:  
\begin{equation}
\label{eqn:rtmpc-time-varying-model}
    x_{t+1} = A(e_t) x_t + B(e_t) u_t.
\end{equation}
The linearization is performed by assuming a given desired operating point; for our multirotor-based evaluation, this point corresponds to the hover condition.

Second, the feedback gain $K_t$ for the ancillary controller in \cref{eqn:rtmpc-ancillary} is updated by solving the infinite horizon, discrete-time LQR problem using $(A(e_t), B(e_t), Q, R)$, leaving the tuning weights $Q$, $R$ fixed. Last, we compute the robust control invariant set $\mathbb{Z}_t$ employed by \ac{RTMPC} from the resulting $K_t$, $A(e_t)$, $B(e_t)$, and a given $\mathbb W$. Due to the computational cost of \textit{precisely} computing $\mathbb{Z}_t$ (from $K_t$, $A(e_t)$, $B(e_t)$, and $\mathbb{W}$), we generate an outer-approximation of $\mathbb{Z}_t$ via Monte Carlo simulation. This is done by computing the axis-aligned bounding box of the state deviations obtained by perturbing the closed loop system $A_{K_t}$ with randomly sampled instances of $w \in \mathbb{W}$. The set $\mathbb W$ is designed to capture the effects of linearization and discretization errors, as well as errors that are introduced by the learning/parameter estimation procedure. 

\subsection{Phase 1: Base Policy and Environment Factor Encoder Learning via Efficient Imitation}
We now describe the procedure to efficiently learn a base policy $\pi$ and an environment factor encoder $\mu$ in simulation. Similar to \ac{RMA}, our base policy takes as input the current state $x_t$, an extrinsics vector $z_t$ and, different from RMA, a reference trajectory $\mathbf{x}_t^{\text{ref}}$. It outputs a vector of actuator commands $a_t$. 
As in \ac{RMA}, the extrinsics vector $z_t$ represents a low dimensional encoding of the environment parameters $e_t$, and it is generated in this phase by the environment factor encoder $\mu$:
\begin{equation}
    \label{eqn:adaptive_policy}
    \begin{split}
        z_t &= \mu(e_t) \\ 
        a_t &= \pi(x_t, z_t, \mathbf{x}_t^{\text{ref}}).
    \end{split}
\end{equation}
We jointly train the base policy $\pi$ and environment encoder $\mu$ end-to-end. However, unlike \ac{RMA}, we do not use \ac{RL}, but demonstrations collected from \ac{RTMPC} in combination with \ac{DAgger} \cite{ross2011reduction}, treating the \ac{RTMPC} as a \textit{expert}, and the policy in \cref{eqn:adaptive_policy} as a \textit{student}. More specifically, at every timestep, given the environment parameters vector $e_t$, the current state of the robot $x_t$, and the reference trajectory $\mathbf{x}_t^{\text{ref}}$, the expert generates a control action $u_t$ by first computing a \textit{safe} reference plan $\check{\mathbf{x}}_t^*, \check{\mathbf{u}}_t^*$, and then by using the ancillary controller in \cref{eqn:rtmpc-ancillary}. The obtained control action is applied to the simulation with a probability $\beta$, otherwise the applied control action is queried from the student (\cref{eqn:adaptive_policy}). At every timestep, we store in a dataset $\mathcal{D}$ the (input, output) pairs $(\{\mathbf{x}_t^{\text{ref}}, x_t, e_t\}, u_t)$.

\noindent
\textbf{Tube-guided Data Augmentation.} \label{sec:approach:tube_augmentation} We leverage our previous work \cite{tagliabue2022demonstration} to augment the collected demonstrations with extra data that accounts for the effects of the uncertainties in $\mathbb W$. This procedure leverages the idea that the tube $\mathbb{Z}_t$ centered around $\check{x}_{0,t}^*$, as computed by \ac{RTMPC}, represents a model of the states that the system may visit when subject to the uncertainties captured by the additive disturbances $w \in \mathbb{W}$, while the ancillary controller \cref{eqn:rtmpc-ancillary} represents an efficient way to compute control actions that ensure the system remains inside the tube. Therefore, at each timestep $t$, given the ``safe'' plan computed by the expert $\check{\mathbf{x}}_t^*, \check{\mathbf{u}}_t^*$, we compute extra state-action pairs $(x_t^+, u_t^+)$ by sampling states from inside the tube $x_t^+ \in \check{x}_{0,t}^* \oplus \mathbb{Z}_t$, and computing the corresponding robust control action $u_t^+$ using the ancillary controller:
\begin{equation}
    u_t^+ = \check{u}_{0,t}^* + K(x_t^+ - \check{x}_{0,t}^*).
\end{equation}
In this way, we obtain extra (input, output) samples $(\{\mathbf{x}_t^{\text{ref}}, x_t^+, e_t\}, u_t^+)$ that are added to  the training dataset $\mathcal{D}$. 
Last, the policy in \cref{eqn:adaptive_policy} is trained end-to-end using the dataset $\mathcal{D}$, by finding the parameters of $\pi$ and $\mu$ that minimize the following \ac{MSE} loss: $\| u_i - \pi(x_i, \mu(e_i), \mathbf{x}_i^{\text{ref}})\|_2^2$, where $i$ denotes the $i$-th datapoint in $\mathcal{D}$.  

\subsection{Phase 2: Learning the Adaptation Module}
This step is performed as in \ac{RMA} \cite{kumar2021rma}, and is described in \cref{eqn:rma_adaptation_module} of this work. 


\section{Evaluation} \label{sec:evaluation}
 \section{Benchmarks and Evaluation}
\label{sec:eval}

We evaluate \krakenSpace to answer the following set of questions:
\begin{itemize}
\item How much improvement does partial evaluation and our implemented compiler optimizations give \kraken? %(\S \ref{sec:eval2})
\item How much faster is our purely functional f-expr language, \krakenSpace, compared to other implementations of fexprs? %(\S \ref{sec:eval1} - \ref{sec:eval2})
\item How does \kraken's performance, with its fexprs, compare to macros? %(\S \ref{sec:eval1}, \S \ref{sec:eval3})
\item How do the different partial evaluation mechanisms/optimizations in \krakenSpace contribute towards reduction in overall runtime?
%\item What does \krakenSpace do internally when we create a data structure and evaluate it for some function? (\S \ref{sec:casestudy})
\end{itemize}

\textbf{Experimental Setup}: 
We ran these experiments in a reproducible Nix environment on a NixOS install \cite{10.1145/1411203.1411255} (Kernel 6.0.0) on a laptop with 8 cores / 16 threads and 64 GB of RAM.
Our code contains the scripts and Nix Flakes needed to reproduce the exact set of dependencies to run our tests.
%The code can be found at \url{https://github.com/limvot/kraken}.

The Kraken benchmarks were run using both the Wasmtime and WAVM WebAssembly engines for most benchmarks.
The Wasmtime WebAssembly engine is one of the most popular, developed by the Bytecode Alliance itself, and uses the CraneLift code generation backend.
The WAVM WebAssembly engine is interesting for its use of LLVM, and it often produces the fastest code on benchmarks but has a higher startup time.
We eliminated the Cfold Wasmtime benchmark due to problems running out of stack space (a known property of the Cfold benchmark).

\textbf{Benchmarks}: 
To showcase the capability of Kraken, we created benchmarks that are commonly implemented in functional languages and have been used as benchmarks in other papers \cite{reinking2021perceus, 10.1145/3547646}.
The benchmarks are
\begin{itemize}
\item Fib - Calculating the nth Fibonacci number
\item RB-Tree - Inserting n items into a red-black tree, then traversing the tree to sum its values
\item Deriv - Computing a symbolic derivative of a large expression
\item Cfold - Constant-folding a large expression
\item NQueens - Placing n number of queens on the board such that no two queens are diagonal, vertical, or horizontal from each other
\end{itemize}
All benchmarks besides Fibonacci use the fexpr version of match for pattern matching in \kraken, which is equivalent to the macro version in NewLisp. We also RB-Tree using NewLisp's~\cite{mueller2018newlisp} version of fexpr match. We modified the sizes of the problems presented to the benchmark to account for the longer running times of some of the less-optimized implementations.
The code for Kraken and NewLisp is very similar, and we should note that it is very unidiomatic NewLisp.
Our goal was not to compare Kraken and NewLisp as implementation languages for Red-Black Trees, but to stress test a single reasonably complex fexpr/macro, namely pattern matching.
% \textbf{Comparison with other languages}: We evaluated \krakenSpace against a language that contains f-exprs, as well as against itself with various optimizations disabled. The only other language we could find which contains a real f-expr mechanism is NewLisp~\cite{mueller2018newlisp} and so we ported \kraken's benchmark implementation to NewLisp.

%The six state-of-the-art languages are Java 17.0.1, Swift 5.4.2, Koka 2.3.2, C++, Haskell 8.10.7, and OCaml 4.12.
%The language choices were taken directly from Perceus reference-counting paper \cite{reinking2021perceus}.
%The Fibonacci benchmark additionally tests Python 3.9.11 and Chez Scheme 9.5.4.
%Koka, Ocaml and Haskell are good comparison points as statically-typed, compiled, functional programming languages, while Chez Scheme is a good comparison point as a mature and industrial strength dynamically-typed Scheme implementation known for its performance. 
%\subsection{Basic Level Comparison}
\subsection{The Effect of Partial Evaluation on Eval Calls}

\begin{table}[h]
\caption{Number of eval calls with no partial evaluation for Fexprs}
	\begin{tabular}{||c | c c c c c ||} 
		\hline
		&Evals & Eval w1 Calls & Eval w0 Calls & Comp Dyn & Comp Dyn\\ 
        & & & & w1 Calls & w0 Calls\\ [0.5ex] 
		\hline\hline
		Cfold 5 & 10897376 & 2784275 & 879066  & 1 & 0 \\ 
		\hline
		  Deriv 2  & 11708558 & 2990090 & 946500 & 1 & 0 \\ 
        \hline
		  NQueens 7 & 13530241 & 3429161 & 1108393 & 1 & 0 \\ 
    \hline
		  Fib 30 & 119107888 & 30450112 & 10770217 & 1 & 0 \\ 
    \hline
		  RB-Tree 10 & 5032297 & 1291489 & 398104 & 1 & 0 \\ 
		\hline
	\end{tabular}
    \label{npe:calls}
 \end{table}

As mentioned before, using fexprs without partial evaluation will prelude optimization and cause a massive amount of repeated work. Table \ref{npe:calls} and Table \ref{pe:calls} show the number of calls to the \krakenSpace runtime's eval function, the number of times the runtime's eval function executed a call to an applicative with wrap\_level=1, the number of times the runtime's eval function executed a call to an operative with wrap\_level=0, the number of compiled dynamic calls to applicatives with wrap\_level=1, and the number of compiled dynamic calls to operatives with wrap\_level=0.
These are shown for \krakenSpace test cases with partial evaluation turned off and turned on. 
\begin{table}[h]
\caption{Number of eval calls in Partially Evaluated Fexprs}
	\begin{tabular}{||c | c c c c c ||} 
		\hline
		&Evals & Eval w1 Calls & Eval w0 Calls & Comp Dyn & Comp Dyn\\ 
        & & & & w1 Calls & w0 Calls\\ [0.5ex] 
		\hline\hline
		Cfold 5 & 0 & 0 & 0  & 0 & 0 \\ 
		\hline
		  Deriv 2  & 0 & 0 & 0 & 2 & 0 \\ 
        \hline
		  NQueens 7 & 0 & 0 & 0 & 0 & 0 \\ 
    \hline
		  Fib 30 & 0 & 0 & 0 & 0 & 0 \\ 
    \hline
		  RB-Tree 10 & 0 & 0 & 0 & 10 & 0 \\ 
		\hline
	\end{tabular}
    \label{pe:calls}
 \end{table}

\begin{table}[h]
\caption{Number of calls to the runtime's eval function for RB-Tree. The table shows the non-partial evaluation numbers -> partial evaluation numbers.}
	\begin{tabular}{||c | c c c c c ||} 
		\hline
		&Evals & Eval w1 Calls & Eval w0 Calls & Comp Dyn & Comp Dyn\\ 
        & & & & w1 Calls & w0 Calls\\ [0.5ex] 
		\hline\hline
		  RB-Tree 7 & 2952848 -> 0 & 757932 -> 0 & 233513 -> 0 & 1 -> 7 & 0 -> 0\\ 
        \hline
		  RB-Tree 8 & 3532131 -> 0 & 906548 -> 0 & 279379 -> 0 & 1 -> 8 & 0 -> 0\\ 
        \hline
		  RB-Tree 9 & 4278001 -> 0 & 1097965 -> 0 & 3383831 -> 0 & 1 -> 9 & 0 -> 0\\ 
		\hline
	\end{tabular}
    \label{pe:rb}
    \vspace{-4mm}
 \end{table}

Without partial evaluation, no compilation can be done because it is impossible to tell if arguments to calls will be evaluated. In all benchmarks, partial evaluation removed all calls to the runtime's eval function, resulting in a completely compiled program. Looking at RB-Tree, there are over a million calls to combiners with wrap level 1 (normal functions), and 398,000 calls to combiners with wrap level 0 (operatives replacing macros). This massive blowup in the number of calls is due to the repeated and exponential re-execution of macro-like-combiners in the definition of other macro-like-combiners, as discussed in the Introduction.

The non-partially-evaluated benchmarks show 1 compiled dynamic call to an applicative (its the first call into eval) and 0 compiled dynamic calls to operatives, because there is no compilation at all. For the partially evaluated benchmarks, there are a few compiled dynamic calls to applicatives due to higher-order function use in the benchmarks, and there are no compiled dynamic calls to operatives, as all operative use has been eliminated.
We also varied the inputs for RB-Tree shown in Table \ref{pe:rb} to give a sense for how the number scale with respect to input size.

The incredible slowdown implied by these tables comes to full fruition in our RB-Tree test in Fig.~\ref{fig:kraken_nqueens_rbtree}.
We kept this run shorter because Kraken's non-partial-evaluating interpreter takes an incredibly long time even for 100 insertions (40 minutes).
The compounding layers of repeated macro-like operative calls in the non-partially-evaluated Kraken version cause a ~70,000x slowdown relative to the partial evaluated, optimized, and compiled version.
For the remaining benchmarks, we remove the naive interpreted \krakenSpace version, as in each case its performance is so bad as to blow out the graph and make it impossible to do any comparison.
In our optimized Kraken, our partial evaluation algorithm is able to fully collapse these levels of inefficiency, evaluate and inline the results, and give the backend more specialized code to optimize, emitting a compiled version that handily beats not only the NewLisp-fexpr implementation but even the NewLisp-macro implementation, as can be seen in Fig.~\ref{fig:kraken_vs_world_fib}.
We kept the benchmark sizes small in this test because the stack limits of NewLisp prevent sizes larger then ~880, while the Tail Call Elimination performed by the \krakenSpace compiler allows us to run much larger benchmarks, including the run of 4,800,000 inserts to the RB-Tree.
This result shows the dramatic effect of partial evaluation and compiler optimizations on runtime for \kraken. Our technique takes the performance of a fully fexpr based language from being completely infeasible to being faster than a macro-based dynamic scripting language currently in use.
% \begin{center}
% \begin{table}[ht]
% \caption{Number of call to the runtime's eval function for Fib. The table shows the non-partial evaluation numbers -> partial evaluation numbers}
% 	\begin{tabular}{||c | c c c c c ||} 
% 		\hline
% 		&Evals & Eval w1 Calls & Eval w0 Calls & Comp Dyn w1 Calls & Comp Dyn w0 Calls\\ [0.5ex] 
% 		\hline\hline
% 		Fib 10 & 8468 -> 0 & 2167 -> 0  & 777 -> 0 & 1 -> 0 & 0 -> 0 \\ 
% 		\hline
% 		  Fib 15  & 87916 -> 0 & 22478 -> 0 & 7961 -> 0 & 1 -> 0 & 0 -> 0 \\ 
%         \hline
% 		  Fib 20 & 969010 -> 0 & 247731 -> 0 & 87633 -> 0 & 1 -> 0 & 0 -> 0 \\ 
%     \hline
% 		  Fib 25 & 10740492 -> 0 & 2745825 -> 0  & 971209 -> 0 & 1 -> 0 & 0 -> 0 \\ 
% 		\hline
% 	\end{tabular}
%     \label{pe:fib}
%  \end{table}
% \end{center}

\begin{figure}[h]
\caption{Constant Fold and Deriv}
\includegraphics[width=0.45\textwidth]{cfold_table.csv_}
\includegraphics[width=0.45\textwidth]{deriv_table.csv_}
\label{fig:kraken_const_deriv}
\vspace{-6mm}
\end{figure}
\subsection{Comparison between Kraken Versions}
Beyond the massive speedup from partial-evaluation, Fig. \ref{fig:kraken_const_deriv} and \ref{fig:kraken_nqueens_rbtree} show the effect of the various compiler optimizations we described by disabling them one by one.
 Our main four optimizations have a strong positive effect on runtime, with the exception of lazy environment instantiation. Lazy environment instantiation helps massively on fib, and some on Deriv, but generally hurts the rest slightly.


\begin{figure}[h]
\caption{N-Queens}
\includegraphics[width=0.45\textwidth]{nqueens_table.csv_}
\includegraphics[width=0.45\textwidth]{slow_rbtree_table.csv_}
\label{fig:kraken_nqueens_rbtree}
\vspace{-4mm}
\end{figure}


\subsection{Comparison against Others}


To give a general idea of our current performance, we also show a Fibonacci benchmark that mostly exercises pure function-call speed and inlining as seen in Fig. ~\ref{fig:kraken_vs_world_fib}.
We include Python and Chez Scheme to give a general idea for where an exemplar slow and an exemplar fast dynamic language would fall.
With the benefit of our partial evaluation, compilation, and leaning upon mature WebAssembly implementations, we beat both, but this should be taken with a grain of salt, as this is a very limited micro-benchmark only meant to give a general sense of the order of magnitude of our performance.



\label{sec:eval1}
\begin{figure}[h]
\caption{Kraken vs. Others. Ordered by fastest to slowest}
\includegraphics[width=0.45\textwidth]{fib_table.csv_}
\includegraphics[width=0.45\textwidth]{rbtree_table.csv_}
\label{fig:kraken_vs_world_fib}
\end{figure}

%\label{sec:eval_nqueens}
%\begin{figure}[h]
%\caption{N-Queens}
%\includegraphics[width=0.45\textwidth]{nqueens_table.csv_}
%\includegraphics[width=0.45\textwidth]{slow_nqueens_table.csv_}
%\label{fig:kraken_nqueens}
%\end{figure}

%\label{sec:eval_nqueens}
%\begin{figure}[h]
%\caption{Kraken, N-Queens, absolute value and log-scale}
%\includegraphics[width=0.45\textwidth]{nqueens_table.csv_}
%\includegraphics[width=0.45\textwidth]{nqueens_table.csv_log}
%\label{fig:kraken_nqueens}
%\end{figure}
%\label{sec:eval_nqueensp}
%\begin{figure}[h]
%\caption{Kraken, N-Queens, absolute value and log-scale}
%\includegraphics[width=0.45\textwidth]{slow_nqueens_table.csv_}
%\includegraphics[width=0.45\textwidth]{slow_nqueens_table.csv_log}
%\label{fig:kraken_nqueensp}
%\end{figure}

%\label{sec:eval_cfold}
%\begin{figure}[h]
%\caption{C-Fold}
%\includegraphics[width=0.45\textwidth]{cfold_table.csv_}
%\includegraphics[width=0.45\textwidth]{slow_cfold_table.csv_}
%\label{fig:kraken_cfold}
%\end{figure}
%\label{sec:eval_cfold}
%\begin{figure}[h]
%\caption{Kraken, C-Fold, absolute value and log-scale}
%\includegraphics[width=0.45\textwidth]{cfold_table.csv_}
%\includegraphics[width=0.45\textwidth]{cfold_table.csv_log}
%\label{fig:kraken_cfold}
%\end{figure}
%\label{sec:eval_cfoldp}
%\begin{figure}[h]
%\caption{Kraken, C-Fold, absolute value and log-scale}
%\includegraphics[width=0.45\textwidth]{slow_cfold_table.csv_}
%\includegraphics[width=0.45\textwidth]{slow_cfold_table.csv_log}
%\label{fig:kraken_cfoldp}
%\end{figure}

%\label{sec:eval_deriv}
%\begin{figure}[h]
%\caption{Deriv}
%\includegraphics[width=0.45\textwidth]{deriv_table.csv_}
%\includegraphics[width=0.45\textwidth]{slow_deriv_table.csv_}
%\label{fig:kraken_deriv}
%\end{figure}
%\label{sec:eval_deriv}
%\begin{figure}[h]
%\caption{Kraken, Deriv, absolute value and log-scale}
%\includegraphics[width=0.45\textwidth]{deriv_table.csv_}
%\includegraphics[width=0.45\textwidth]{deriv_table.csv_log}
%\label{fig:kraken_deriv}
%\end{figure}
%\label{sec:eval_derivp}
%\begin{figure}[h]
%\caption{Kraken, Deriv, absolute value and log-scale}
%\includegraphics[width=0.45\textwidth]{slow_deriv_table.csv_}
%\includegraphics[width=0.45\textwidth]{slow_deriv_table.csv_log}
%\label{fig:kraken_derivp}
%\end{figure}

%\subsection{Comparison against state-of-the-art languages}
%\label{sec:eval3}

%\begin{figure}[h]
%\caption{Kraken vs. S.o.t.A.}
%\includegraphics[width=0.45\textwidth]{cfold_table.csv_}
%\includegraphics[width=0.45\textwidth]{rbtree_table.csv_}
%\label{fig:kraken_vs_world1}
%\end{figure}

%\begin{figure}[h]
%\caption{Kraken vs. S.o.t.A.}
%\includegraphics[width=0.45\textwidth]{deriv_table.csv_}
%\includegraphics[width=0.45\textwidth]{nqueens_table.csv_}
%\label{fig:kraken_vs_world2}
%\end{figure}

% \begin{figure}[h]
% \caption{Kraken vs. S.o.t.A. (Log)}
% \includegraphics[width=0.45\textwidth]{cfold_table.csv_log}
% \includegraphics[width=0.45\textwidth]{rbtree_table.csv_log}
% \label{fig:kraken_vs_world_log_1}
% \end{figure}
% \begin{figure}[h]
% \caption{Kraken vs. S.o.t.A. (Log)}
% \includegraphics[width=0.45\textwidth]{deriv_table.csv_log}
% \includegraphics[width=0.45\textwidth]{nqueens_table.csv_log}
% \label{fig:kraken_vs_world_log_2}
% \end{figure}

%As we noted before with the Fib(30) microbenchmark in Section \ref{sec:eval1}, we remain significantly slower than state-of-the-art compiled languages.
%This is particularly true for memory-intensive benchmarks due to our naive reference-counting and malloc/free implementations.
%However, our results are of a similar order of magnitude to the difference between the state-of-the-art compiled languages and dynamic scripting languages, like Python's results in the Fib(30) microbenchmark.
%We assert that is not a fundamental limitation because the classic f-expr slowness is being eliminated, as shown by Fig. \ref{fig:kraken_vs_newlisp1} and Fig. \ref{fig:kraken_vs_newlisp2}.
%In future work, we plan to expand our compile-time analysis and optimization to implement a modified, dynamic-language version of Perceus reference counting.
%With this change, we belive \krakenSpace can be competitive with these state-of-the-art languages.

%\subsection{Case Study: Red-Black Tree}
%\label{sec:casestudy}

%\begin{figure}[h]
%\caption{Kraken vs. S.o.t.A. - RB-Tree Focus}
%\includegraphics[width=0.4\textwidth]{rbtree_table.csv_}
%\includegraphics[width=0.4\textwidth]{rbtree_table.csv_log}
%\label{fig:kraken_vs_world_rbtree}
%\end{figure}


%To evaluate our partial evaluation algorithm and compiler, we extracted the benchmarks used by the Koka language project from their code repository and added Kraken versions, as well as implementing a naive Fibonacci microbenchmark ourselves to evaluate pure function call speed.\\
%With partial evaluation and the compiler optimizations listed above, we get fairly strong performance on purely numerical computations, such as the naive Fibonacci microbenchmark.
%Unfortunately, the overhead of our unsophisticated reference counting, dynamic type checking, and bounds checking causes poor performance on benchmarks involving data structures relative to mainstream programming language implementations.
%This is not a fundamental limitation, and will be addressed in future work, as recounted in the next section.
%It should be noted, however, that while the performance relative to established language implementations is very poor for the memory-intensive benchmarks (600-900x slower), we still realize a massive speedup compared to an unoptimized and non-partial-evaluated f-expr implementation (100,000x faster)!


\section{Conclusions} \label{sec:conclusions}
\section{Conclusion}\label{sec:conclusion}
In this work, we focus on addressing the fundamental challenge of OOD detection tasks, which is how to fully understand the semantic discrepancy between the ID/OOD samples. We reveal that the key to success in the realistic SCOOD task is to allocate as many ID samples in the unlabeled set correctly as possible. To this end, we propose a novel uncertainty-aware optimal transport scheme that introduces class-specific energy scores as guidance for effective label assignment. Experimental results show that our method achieves better performance than previous state-of-the-art methods on SCOOD benchmarks.

\textbf{Limitations.} In addition to temperature scaling, other techniques such as feature clipping applied in ReAct~\cite{sun2021react} also enhance the performance of energy score, so how to obtain an OOD score that best fits the SCOOD task can be further explored. Moreover, a setting highly related to SCOOD has been proposed in \cite{katz2022training} and formulated as a constrained optimization problem. We will also theoretically analyze these practical OOD settings in our feature work.

% \section*{Acknowledgments}
\textbf{Acknowledgments.} 
This work is supported by National Key R\&D Program of China under Grant 2020AAA0105701, National Natural Science Foundation of China (NSFC) under Grants 61872327, Major Special Science and Technology Project of Anhui, National Natural Science Foundation of China (62033012) and Ant Group through Ant Research Intern Program.


%\section*{ACKNOWLEDGMENT}
%This work was funded by the Air Force Office of Scientific Research MURI FA9550-19-1-0386. We thank Lauren Li, Xiaoyi (Jeremy) Cai, Kota Kondo and Yulun Tian  for reviewing the manuscript.  

\balance
\bibliographystyle{IEEEtran}
\bibliography{bibliography}

\end{document}


