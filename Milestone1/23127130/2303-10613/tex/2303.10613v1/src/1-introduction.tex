\section{Introduction}\label{sec:introduction}

% Computer-Aided Design (CAD) have been widely used by the industry to create precise 3D shapes. %, and CAD products are ubiquitous in practically every aspect of human life. %almost all human environments, ranging from vast outdoor landscapes to indoor spaces.
% However, the inherent complexity of CAD procedures makes it unusable by non-expert users. Therefore, reverse engineering have received a great deal of research interest in computer vision and graphics. 
% It empowers users to reproduce CAD models from other representations and support the designer to create new variations to facilitate various engineering and manufacturing applications.

CAD reconstruction is one of the most sought-after geometric modeling technologies, which plays a substantial role in reverse engineering in case of the original design document is missing or the CAD model of a real object is not available.  %the prototype is modified on the shop floor. 
It empowers users to reproduce CAD models from other representations and supports the designer to create new variations to facilitate various engineering and manufacturing applications. %, such as quality control, re-engineering of parts and design of custom-fit parts. 

The advance in 3D scanning technologies has promoted the paradigm shift from time-consuming and laborious manual dimensions to automatic CAD reconstruction.  
A typical line of works~\cite{requicha1992solid,varady1997reverse,beniere2013comprehensive,buonamici2018reverse} first reconstructs a polygon mesh from the scanned point cloud, then followed by mesh segmentation and primitive extraction to obtain a boundary representation (B-rep). Finally, a CAD shape parser is applied to convert the B-rep into a sequence of modeling operations. 
%a constructive solid geometry (CSG) model, which is a widely accepted parametric representation in modern CAD systems. 
Recently, inspired by the substantial success of point set learning~\cite{achlioptas2018learning,qi2017pointnet++,wang2019dynamic} and deep 3D representations~\cite{mescheder2019occupancy,park2019deepsdf,chen2019learning}, a number of methods have exploited neural networks to improve the above pipeline, \eg, detecting and fitting primitives to raw point clouds directly~\cite{li2019supervised,le2021cpfn,sharma2020parsenet}. 
A few works (\eg, CSG-Net~\cite{sharma2018csgnet}, UCSG-Net~\cite{kania2020ucsg}, and CSG-Stump~\cite{ren2021csg}) further parse point cloud inputs into a constructive solid geometry (CSG) tree by predicting a set of primitives that are then combined with Boolean operations. 
Although achieving encouraging compact representation, they only output a set of simple primitives with limited types (\eg, planes, cylinders, spheres), which restricts their representation capability for reconstructing complex and more general 3D shapes. CAPRI-Net~\cite{yu2022capri} introduces quadric surface primitives and the difference operation based on BSP-Net~\cite{chen2020bsp} to produce complicated convex and concave shapes via a CSG tree. However, controlling the implicit equation and parameters of quadric primitives is difficult for designers to edit the reconstructed models. Thus, the editability of those methods is quite limited.
%richer geometric and topological variations 
% Additionally, controlling the parameters and function of parametric primitives is not intuitive for designer to edit the reconstructed CAD models. Thus, the editability of those methods is quit limited.

% \begin{figure*}[!t]
% \centering
%   \includegraphics[width=1.0\linewidth]{teaser_v3.pdf}   
%   \caption{\textbf{SECAD-Net for CAD reconstruction}. Starting from a voxel grid (top), SECAD-Net learns \textit{Sketch-Extrude} operations to reconstruct CAD models (bottom), without any supervision of part segmentation and sketch labels.
%   }
%   \label{fig:teaser}
% \end{figure*}

%Motivated by the above challenges,
In this paper, we develop a novel and versatile deep neural framework, named SECAD-Net, to reconstruct high-quality and editable CAD models. 
Our approach is inspired by the observation that a CAD model is usually designed as a command sequence of operations~\cite{shah1998designing,wu2021deepcad,willis2021fusion,cascaval2022differentiable,yang2022discovering}, \ie, a set of planar 2D sketches are first drawn then extruded into 3D solid shapes for Boolean operations to create the final model. %Sketch-based CAD modeling is also validated by the recent work~\cite{li2020sketch2cad}.
At the heart of our approach is to learn the sketch and extrude modeling operations, rather than CSG with parametric primitives. 
To determine the position and axis of each sketch plane, SECAD-Net first learns multiple extrusion boxes to decompose the entire shape into multiple local regions. Afterward, for the local profile in each box, we utilize a fully connected network to learn the implicit representation of the sketch. An extrusion operator is then designed to calculate the implicit expression of the cylinders according to the predicted sketch and extrusion parameters. We finally apply a union operation to assemble all extrusion cylinders into the final CAD model. 
% Most closely related to our work is Point2Cyl~\cite{uy2022point2cyl}, which also uses sketch and extrude to learn a collection of extrusion cylinders. 
% which also uses sketch and extrude to decompose a 3D point cloud into a collection of extrusion cylinders.
% As a supervised network, Point2Cyl relies heavily on ground truth sequence data, and trains a complex network by inferring instance segmentation, surface normal, and base/barrel segmentation. 
%It is trained utilizing supervision from ground truth sequence data.
%reconstructs a collection of extrusions which can be manually ordered and combined to build CAD shapes. 
% In contrast, our network is self-supervised with a reconstruction loss. %without the guidance of sketch labels.

% \rev{The works most related to ours either rely on supervised learning~\cite{uy2022point2cyl} or have strict constraints on the shape of the sketches~\cite{ren2022extrudenet}. In contrast, our method employs self-supervised learning and enables 2D implicit networks to represent sketches of arbitrary shapes.}

% We draw inspiration from several earlier systems that explored
% the possibility to sketch novel shapes in the context of an existing
% scene,representedasphotographsor3Dmodels

Benefiting from our representation, our approach is flexible and efficient to construct a wide range of 3D shapes.
%With the constraints and parameters in the sketches are available, in various ways
As the predictions of our method are fully interpretable, it allows users to express their ideas to create variations or improve the design by operating on 2D sketches or 3D cylinders intuitively. 
To summarize, our work makes the following contributions: 
\begin{itemize}
\item We present a novel deep neural network for reverse engineering CAD models with self-supervision, leading to faithful reconstructions that closely approximate the target geometry.
% Our method is the first to accomplish extrusion cylinder reconstruction with self-supervision.
\item SECAD-Net is capable of learning implicit sketches and differentiable extrusions from raw 3D shapes without the guidance of ground truth sketch labels. 
% \item SECAD-Net is the first one, to the best of our knowledge, that is capable of learning implicit sketches and differentiable extrusions from raw 3D shapes without the guidance of ground truth sketch labels. %to tractably work with industry-standard CAD operations.
\item Extensive experiments demonstrate the superiority of SECAD-Net through comprehensive comparisons. We also showcase its immediate applications to CAD interpolation, editing, and single-view reconstruction.
%SECAD-Net achieve state-of-the-art results on the 3D reconstruction task comparing to previous supervised network and other primitive-based method.
\end{itemize}

%  tractably work with industry-standard CAD operations
% motivated by the most frequently used CAD operations.
% provides a better approximation to the target than 

% Our method is the first one that is able to predict CSG tree without any supervision and
% achieve state-of-the-art results on the 2D reconstruction task comparing to CSG-N ET trained in a supervised manner. Predictions of our method are fully interpretable and can aid in CAD applications.


% the CAD model
% of an ‘object’ is not available to the designer, does not
% even exist, or no longer corresponds to the real geom-
% etry of the manufactured object itself. 

%  Zone graphs allow us to tractably work with
% industry-standard CAD operations, unlike prior work using
% CSG with parametric primitives. We focus on CAD pro-
% grams consisting of sketch + extrude + Boolean operations,
% which are common in CAD practice

% while having
% the additional benefit that the constraints and parameters in the
% sketches are available to designers for further editing.

% With just
% the sketch and extrude modeling operations, that also incorporate
% Boolean operations, a highly expressive range of 3D designs can be
% created (Figure 1). W

 




% Users of
% our system thus express their ideas using similar sketching steps as
% they would do on paper, yet obtain as output a regular CAD model,
% along with a trace of the sequential operations, ready to be fabri-
% cated or further edited with existing CAD software.


 








%  Computer-Aided Design has long been adopted
% by the industry to create precise and high-quality 3D models suit-
% able for physical simulation, lighting simulation, and downstream
% manufacturing [Autodesk 2019a,b; Robert McNeel & Associates
% 2019; Trimble 2019]. However, the high precision offered by CAD
% comes at the price of complex interfaces to allow users to select
% appropriate geometric operations and tune their parameters. Vari-
% ous approaches have been considered to reduce this user burden,
% from automatic alignment of existing CAD models on scanned point
% clouds [Avetisyan et al . 2019], to educational visualizations of mod-
% eling sequences [Denning et al . 2011]. We contribute to this effort
% by instantiating CAD operations by sequentially interpreting hand-
% drawn sketches.
% Closer to our work are methods aiming at converting raw 3D
% meshes into editable CAD models, which can be formulated as a
% form of program synthesis [Du et al . 2018; Sharma et al . 2018; Tian
% et al . 2019]


%  The typical starting point in these designs is 2D sketches which
% can later be extruded and combined to obtain complex three-dimensional assem-
% blies

% supporting the designer to re-create new products and facilitate various applications such as building the digital twins of products and systems [15]. Particularly,
% reverse engineering (RE) technologies, especially recon-
% structing implicit or parametric (CAD) models from point
% clouds, have been extensively studied in engineering. 

%  The goal of CAD reconstruction is to reverse
% engineer a 3D shape, typically as a series of parametric primitives,
% given approximate input, such as a point cloud or freehand sketch.
% Smirnov et al . [2021] recover manifold 3D shapes from freehand
% sketch input by deforming Coons patch-based templates from set
% object categories. Given point cloud input, traditional approaches
% segment and then fit parametric primitives, such as planes, spheres,
% and cylinders, to the underlying point cloud [Schnabel et al . 2007].
% Recentprogresswithlearning-basedapproacheshasaddressedprim-
% itive segmentation [Yan et al . 2021] reconstruction of parametric
% curves [Wang et al . 2020] and surfaces [Guo et al . 2022; Li et al . 2019;
% Sharma et al . 2020]. 

% Design sketches are then converted to 3D models for downstream
% engineering and manufacturing, using CAD tools that offer high
% precision and editability [Pipes 2007].






% Computer-Aided Design (CAD) has long been adopted by the industry to create precise and high-quality 3D models suitable for physical simulation, lighting simulation, and downstream manufacturing.% [Autodesk 2019a,b; Robert McNeel & Associates 2019; Trimble 2019].

% Shape is a geometric form, which helps us understand
% objects, surrounding environments and even the world.
% Therefore shape modeling and understanding has always
% been a research topic in computer vision and graphics.

% computer aided design (CAD) software have been used for 3D shape creation
% in a myriad of industrial sectors, ranging from automotive and aerospace to manufacturing and architectural design.

% Computer-Aided-Design (CAD) is the industry standard for creating
% 3D shapes, but its inherent complexity has so far restricted its usage
% to a small elite group of expert users.

% Reverse engineering, the creation of Computer Aided Design (CAD)
% models which closely match some target geometry, is one of the
% most sought-after geometric modeling technologies and has been
% extensively studied 

% Computer-Aided Design (CAD) models are ubiquitous
% in engineering and manufacturing to drive decision making and product evolution related to 3D shapes and geometry.


% Reverse engineering empowers human to analyze a physical part and explore how it was originally built to replicate, create variations, or improve on the design. The goal is to ultimately create a new CAD model for use in manufacturing



% With the advance of CAD/CAM technology, creating
% geometric models of existing objects plays a substantial role in reverse engineering, especially when the prototype is created or
% modified on the shop floor and when the CAD model does not exist.
% Therefore, there are increasing demands to achieve 3D model
% reconstruction of existing objects in various industrial applications.



% Reverse engineering (RE) is an important branch of the mechanical design and manufacture application field, and this technique has been widely recognized as a crucial step in the product design cycle. In a normal automated manufacturing environment, the operation sequence usually starts from product design via computer-aided design (CAD) techniques, and ends with generation of machining instructions required to convert raw material into a finished product. In contrast to this conventional manufacturing sequence, reverse engineering represents an approach for the new design of a product that lacks an existing CAD model.