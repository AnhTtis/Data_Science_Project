% CVPR 2023 Paper Template
% based on the CVPR template provided by Ming-Ming Cheng (https://github.com/MCG-NKU/CVPR_Template)
% modified and extended by Stefan Roth (stefan.roth@NOSPAMtu-darmstadt.de)

\documentclass[10pt,twocolumn,letterpaper]{article}

%%%%%%%%% PAPER TYPE  - PLEASE UPDATE FOR FINAL VERSION
%\usepackage[review]{cvpr}      % To produce the REVIEW version
\usepackage{cvpr}              % To produce the CAMERA-READY version
\usepackage[accsupp]{axessibility}  % Improves PDF readability for those with disabilities.

% \usepackage[pagenumbers]{cvpr} % To force page numbers, e.g. for an arXiv version

% Include other packages here, before hyperref.
\usepackage{tabularx}

\usepackage{graphicx}
\usepackage{amsmath}
\usepackage{amssymb}
\usepackage{booktabs}
\newcolumntype{Y}{>{\centering\arraybackslash}X}
\usepackage{pifont}% http://ctan.org/pkg/pifont
\newcommand{\cmark}{\ding{51}}%
\newcommand{\xmark}{\ding{55}}%	
\newcommand{\rev}[1]{{\textcolor{black}{ #1}}}

\definecolor{purple}{rgb}{1, 0, 1}

\newcommand{\ie}{\emph{i.e.,}\xspace}
\newcommand{\eg}{\emph{e.g.,}\xspace}
\newcommand{\abr}{\emph{abbr.}\xspace}
\newcommand{\ea}{\emph{et al.}\xspace}
\newcommand{\gensync}{\emph{GenSync}\xspace}
\newcommand{\colosseum}{\emph{Colosseum}\xspace}
\newcommand{\srep}{\emph{SREP}\xspace} % Set Reconciliation Enhances
\newcommand{\srepsim}{\emph{SREPSim}\xspace}
% Propagation
\newcommand{\esrep}{\emph{E-SREP}\xspace}
\newcommand{\epsrep}{\emph{EP-SREP}\xspace}
\newcommand{\mesrep}{\emph{ME-SREP}\xspace}
\newcommand{\mempoolsync}{\emph{MempoolSync}}

\newcommand{\fref}[1]{Fig.~\ref{#1}}
\newcommand{\tref}[1]{Table~\ref{#1}}
\newcommand{\aref}[1]{Algorithm~\ref{#1}}
\newcommand{\procref}[1]{Procedure~\ref{#1}}
\newcommand{\sref}[1]{Section~\ref{#1}}
\newcommand{\lineref}[1]{line~\ref{#1}}
\newcommand{\appref}[1]{Appendix~\ref{#1}}

% Change \eqref
\LetLtxMacro{\originaleqref}{\eqref}
\renewcommand{\eqref}{Eq.~\originaleqref}

% Theorems and corollaries
\newcounter{theoremcount}
\setcounter{theoremcount}{0}
\DeclareRobustCommand{\theorem}[1]{%
  \refstepcounter{theoremcount}%
  \noindent\textit{\textbf{Theorem \thetheoremcount\label{theorem:#1}: }}%
}
\DeclareRobustCommand{\theoremref}[1]{Theorem~\ref{theorem:#1}}

\DeclareRobustCommand{\proof}{\emph{Proof:}\xspace}
\DeclareRobustCommand{\qqed}{\hfill$\blacksquare$}

\newcounter{corollcount}
\setcounter{corollcount}{0}
\DeclareRobustCommand{\coroll}[1]{%
  \refstepcounter{corollcount}%
  \noindent\textit{\textbf{Corollary \thecorollcount\label{coroll:#1}: }}%
}
\DeclareRobustCommand{\corollref}[1]{Corollary~\ref{coroll:#1}}

\newcounter{lemmacount}
\setcounter{lemmacount}{0}
\DeclareRobustCommand{\lemma}[1]{%
  \refstepcounter{lemmacount}%
  \noindent\textit{\textbf{Lemma \thelemmacount\label{lemma:#1}: }}%
}
\DeclareRobustCommand{\lemmaref}[1]{Lemma~\ref{lemma:#1}}

\newcounter{definitioncount}
\setcounter{definitioncount}{0}
\DeclareRobustCommand{\definition}[1]{%
  \refstepcounter{definitioncount}%
  \noindent\textit{\textbf{Definition \thedefinitioncount\label{definition:#1}: }}%
}
\DeclareRobustCommand{\defref}[1]{Definition~\ref{definition:#1}}

%notes of different authors
\newif\ifnotes
\notestrue
\notesfalse

\newif\ifdiff
\difftrue
\difffalse

\newcommand{\anote}[1]{\ifnotes $\ll$\textsf{\textcolor{purple}{Ari: {#1}}}$\gg$ \fi}
\newcommand{\nnote}[1]{\ifnotes $\ll$\textsf{\textcolor{orange}{Novak: {#1}}}$\gg$ \fi}
\newcommand{\diff}[1]{\ifdiff\textcolor{orange}{#1}\else#1\fi}

%%% Local Variables:
%%% mode: latex
%%% TeX-master: "main"
%%% End:


\graphicspath{{figs/}}

% It is strongly recommended to use hyperref, especially for the review version.
% hyperref with option pagebackref eases the reviewers' job.
% Please disable hyperref *only* if you encounter grave issues, e.g. with the
% file validation for the camera-ready version.
%
% If you comment hyperref and then uncomment it, you should delete
% ReviewTempalte.aux before re-running LaTeX.
% (Or just hit 'q' on the first LaTeX run, let it finish, and you
%  should be clear).
\usepackage[pagebackref,breaklinks,colorlinks]{hyperref}


% Support for easy cross-referencing
\usepackage[capitalize]{cleveref}
\crefname{section}{Sec.}{Secs.}
\Crefname{section}{Section}{Sections}
\Crefname{table}{Table}{Tables}
\crefname{table}{Tab.}{Tabs.}


%%%%%%%%% PAPER ID  - PLEASE UPDATE
\def\cvprPaperID{5700} % *** Enter the CVPR Paper ID here
\def\confName{CVPR}
\def\confYear{2023}


\begin{document}

%%%%%%%%% TITLE - PLEASE UPDATE
\title{SECAD-Net: Self-Supervised CAD Reconstruction by Learning Sketch-Extrude Operations}
%\title{SE-Net: Learning Sketch and Extrusion in CAD by Self-Supervision}

\author{Pu Li$^{1,2}$ \quad Jianwei Guo$^{1,2}$\thanks{Corresponding author: jianwei.guo@nlpr.ia.ac.cn} \quad Xiaopeng Zhang$^{1,2}$ \quad Dong-Ming Yan$^{1,2}$\\
$^1$MAIS, Institute of Automation, Chinese Academy of Sciences\\
$^2$School of Artificial Intelligence, University of Chinese Academy of Sciences
%{\tt\small jianwei.guo@nlpr.ia.ac.cn}
}

% \author{Pu Li\\
% MAIS, Institute of Automation, Chinese Academy of Sciences\\
% Institution1 address\\
% {\tt\small firstauthor@i1.org}
% % For a paper whose authors are all at the same institution,
% % omit the following lines up until the closing ``}''.
% % Additional authors and addresses can be added with ``\and'',
% % just like the second author.
% % To save space, use either the email address or home page, not both
% \and
% Second Author\\
% Institution2\\
% First line of institution2 address\\
% {\tt\small secondauthor@i2.org}
% }

%\maketitle

\twocolumn[
{%
\renewcommand\twocolumn[1][]{#1}%
\maketitle
\begin{center}
    \centering
    \includegraphics[width=1.0\textwidth]{figs/teaser_v3_compress.jpg}
    \captionof{figure}{\textbf{SECAD-Net for CAD reconstruction}. Starting from a voxel grid (top), SECAD-Net learns \textit{Sketch-Extrude} operations to reconstruct CAD models (bottom), without any supervision of part segmentation and sketch labels. 
    }
\end{center}
}
]

\renewcommand{\thefootnote}{\fnsymbol{footnote}}
\footnotetext[1]{Corresponding author: jianwei.guo@nlpr.ia.ac.cn}


%%%%%%%%% ABSTRACT
\begin{abstract}
%\vskip -0.23cm
Reverse engineering CAD models from raw geometry is a classic but strenuous research problem. 
%3D reverse engineering is one of the most sought-after geometric modeling technologies that has been widely recognized as a crucial step in many applications. 
Previous learning-based methods rely heavily on labels due to the supervised design patterns or reconstruct CAD shapes that are not easily editable. 
In this work, we introduce \emph{SECAD-Net}, an end-to-end neural network aimed at reconstructing compact and easy-to-edit CAD models in a self-supervised manner. 
Drawing inspiration from the modeling language that is most commonly used in modern CAD software, we propose to learn 2D sketches and 3D extrusion parameters from raw shapes, from which a set of extrusion cylinders can be generated by extruding each sketch from a 2D plane into a 3D body. %most common CAD modeling operations used  
By incorporating the Boolean operation (\ie, union), these cylinders can be combined to closely approximate the target geometry. 
We advocate the use of implicit fields for sketch representation, which allows for creating CAD variations by interpolating latent codes in the sketch latent space.
Extensive experiments on both ABC and Fusion 360 datasets demonstrate the effectiveness of our method, and show superiority over state-of-the-art alternatives including the closely related method for supervised CAD reconstruction. 
We further apply our approach to CAD editing and single-view CAD reconstruction. 
Code will be released at \url{https://github.com/BunnySoCrazy/SECAD-Net}.
% \url{http://www.overleaf.com}
% The code will be released to facilitate future research. %upon acceptance. 
\end{abstract}

%%%%%%%%% BODY TEXT
\section{Introduction}


Recent years have witnessed the rise of human digitization~\cite{habermannDeepCapMonocularHuman2020,alexanderCREATINGPHOTOREALDIGITAL,pengNeuralBodyImplicit2021,alldieckDetailedHumanAvatars2018, rajANRArticulatedNeural2020}. This technology greatly impacts the entertainment, education, design, and engineering industry.
There is a well-developed industry solution for this task.
High-fidelity reconstruction of humans can be achieved either with full-body laser scans~\cite{saitoSCANimateWeaklySupervised2021}, dense synchronized multi-view cameras~\cite{xiangModelingClothingSeparate2021a,xiangDressingAvatarsDeep2022a}, or light stages~\cite{alexanderCREATINGPHOTOREALDIGITAL}.
However, these settings are expensive and tedious to deploy and consist of a complex processing pipeline, preventing the technology's democratization.

Another solution is to view the problem as inverse rendering and learn digital humans directly from custom-collected data.
Traditional approaches directly optimize explicit mesh representation~\cite{loperSMPLSkinnedMultiperson2015, fangRMPERegionalMultiperson2018, pavlakosExpressiveBodyCapture2019} which suffers from the problems of smooth geometry and coarse textures~\cite{prokudinSMPLpixNeuralAvatars2020,alldieckVideoBasedReconstruction2018}. Besides, they require professional artists to design human templates, rigging, and unwrapped UV coordinates.
Recently, with the help of volumetric-based implicit representations~\cite{mildenhallNeRFRepresentingScenes2020, parkDeepSDFLearningContinuous2019, meschederOccupancyNetworksLearning2019} and neural rendering~\cite{laineModularPrimitivesHighPerformance2020, liuSoftRasterizerDifferentiable2019, thiesDeferredNeuralRendering2019}, 
one can easily digitize a quality-plausible human avatar from video footage~\cite{jiangNeuManNeuralHuman2022,wengHumanNeRFFreeviewpointRendering}.
Particularly, volumetric-based implicit representations~\cite{mildenhallNeRFRepresentingScenes2020, pengNeuralBodyImplicit2021} can reconstruct scenes or objects with much higher fidelity against previous neural renderer~\cite{thiesDeferredNeuralRendering2019,prokudinSMPLpixNeuralAvatars2020}, and is more user-friendly as it does not need any human templates, pre-set rigging, or UV coordinates.
Captured visual footage and corresponding skeleton tracking are enough for training.
However, better reconstructions and more friendly usability are at the expense of the following factors.
1) \textbf{Inefficiency:}
They require longer optimization times (typically tens of hours or days) and inference slowly.
Volume rendering~\cite{mildenhallNeRFRepresentingScenes2020,lombardiNeuralVolumesLearning2019} formulates images by querying the densities and colors of millions of spatial coordinates. 
In the training stage, due to memory constraints, only a small fraction of points are sampled which leads to slow convergence speed.
2) \textbf{Entangled representations}:
The geometry, materials, and motion dynamics are entangled in the neural networks. 
Due to the implicit nature of neural nets, one can hardly edit one property without touching the others~\cite{yuanNeRFEditingGeometryEditing2022a,liuEditingConditionalRadiance2021}.
3) \textbf{Graphics incompatibility}:
Volume rendering is incompatible with the current popular graphic pipeline,
which renders triangular/quadrilateral meshes efficiently with the rasterization technique.
Many downstream applications require mesh rasterization in their workflow (\eg, editing~\cite{foundationBlenderOrgHome}, simulation~\cite{benderPositionBasedSimulationMethods2015}, real-time rendering~\cite{akenine2019real}, ray-tracing~\cite{waldRTXRayTracing}).
Although there are approaches~\cite{lorensenMarchingCubesHigh,labelleIsosurfaceStuffingFast2007} can convert volumetric fields into meshes, the gaps from discrete sampling degrade the output quality in terms of both meshes and textures.


To address these issues, we present \textbf{EMA}, a method based on \textbf{E}fficient \textbf{M}eshy neural fields to reconstruct animatable human \textbf{A}vatars.
Our method enjoys flexibility from implicit representations and efficiency from explicit meshes, yet still maintains high-fidelity reconstruction quality.
Given video sequences and the corresponding pose tracking, our method digitizes humans in terms of canonical triangular meshes, physically-based rendering (PBR) materials, and skinning weights \textit{w.r.t.} skeletons.
We jointly learn the above components via inverse rendering~\cite{laineModularPrimitivesHighPerformance2020,chenDIBRLearningPredict2021,chenLearningPredict3D2019} in an end-to-end manner.
Each of them is derived from a separate neural field, which relaxes the requirements of a preset human template, rigging, or UV coordinates.
Specifically, we predict a canonical mesh out of a signed distance field (SDF) by differentiable marching tetrahedra~\cite{shenDeepMarchingTetrahedra2021,gaoGET3DGenerativeModel,gaoLearningDeformableTetrahedral2020,munkbergExtractingTriangular3D2022}, then we extend the marching tetrahedra~\cite{shenDeepMarchingTetrahedra2021} for spatial-varying materials by utilizing a neural field to predict PBR materials \textit{on the mesh surfaces} after rasterization~\cite{munkbergExtractingTriangular3D2022,hasselgrenShapeLightMaterial2022,laineModularPrimitivesHighPerformance2020}.
To make the canonical mesh animatable, we take another neural field to model the forward linear blend skinning for the meshes. 
Given a posed skeleton, the canonical mesh is then transformed into the corresponding poses.
Finally, we shade the mesh with a rasterization-based differentiable renderer~\cite{laineModularPrimitivesHighPerformance2020} and train our models with a photo-metric loss.
After training, we export the mesh with materials and discard the neural fields.

\looseness=-1
There are several merits of our method design.
1) \textbf{Efficiency}:
Powered by efficient mesh rendering, our method can render in real-time.
Besides, the training speed is boosted as well, 
since we compute loss holistically on the whole image and the gradients only flow on the mesh surface. In contrast, volume rendering takes limited pixels for loss computation and back-propagates the gradients in the whole space.
Our method only needs about an hour of training and minutes of optimization are enough for plausible avatar reconstruction.
2) \textbf{Disentangled representations}:
Our shape, materials, and motion modules are disentangled naturally by design, which facilitates editing. 
Besides, Canonical meshes with forward skinning modeling handle the out-of-distribution poses better.
3) \textbf{Graphics compatibility}:
Our derived mesh representation is compatible with 
the prominent graphic pipeline, which leads to instant downstream applications (\eg, the shape and materials can be edited directly in design software~\cite{foundationBlenderOrgHome}).
To further improve reconstruction quality, we additionally optimize image-based environment lights and non-rigid motions.


We conduct extensive experiments on standards benchmarks H36M~\cite{ionescuHuman36MLarge2014b} and ZJU-MoCap~\cite{pengNeuralBodyImplicit2021}.
Our method achieves very competitive performance for novel view synthesis, generalizes better for novel poses, 
and significantly improves both training time and inference speed against previous arts.
Our research-oriented code reaches real-time inference speed (100+ FPS for rendering $512\times512$ images).
We in addition showcase applications including novel pose synthesis, material editing, and relighting.
\section{Related work}
\label{sec:relatedWork}

\noindent\textbf{Neural implicit representation.} 
% In 3D computer vision, diverse representations are designed and proposed for different applications, each containing its own advantages and draw-
% backs. 
3D shapes can be represented either \textsl{explicitly} (\eg, point sets, voxels, meshes) or \textsl{implicitly} (\eg, signed-distance functions, indicator functions), each of them comes with its own advantages and drawbacks. 
Recently, there is an explosion of neural implicit representations~\cite{mescheder2019occupancy,park2019deepsdf,chen2019learning} that allow for generating detail-rich 3D shapes by predicting the underlying signed distance fields. 
%Neural implicit surface representation have gained immense popularity because of the ability to generate complex, high spatial resolution 3D shapes while using a small memory footprint during training. 
Thanks to the ability to learn priors over shapes, many deep implicit works have been proposed to solve various 3D tasks, such as shape representation and completion~\cite{sitzmann2019scene,Atzmon2020SAL,chibane2020implicit}, image-based 3D reconstruction~\cite{tulsiani2017multi,xu2019disn,yariv2020multiview}, shape abstraction~\cite{tulsiani2017learning,genova2019learning} and novel view synthesis~\cite{mildenhall2020nerf,dellaert2020neural}. 
Theoretically, any of the above shape representations can be used to represent sketches. However, primitive-based methods usually suppress the ability cap of shape representation. In this work, we choose to fit an implicit sketch  representation using a neural network, and show its superiority over other representations (\eg, BSP~\cite{chen2020bsp}) in the ablation study, see Sec.~\ref{sec:ablations}.

% \jw{Theoretically, any of the above shape representations can represent sketches. However, primitive-based methods suppress the ability cap of shape representation. Other lower-level representations adopted by [BSP] and [CAPRI] rely on CSG operations, which makes network training cumbersome. In order to avoid the above problems, we aim to fit the sketch's implicit representation with a neural network, which has been proven effective in [xxx].} 

% we chose to use functional representation of the output shape as a neural occupancy and 3D mapping function over UV
% domains. Functional representation is well-suited for het-
% erogeneous geometry with varied levels of detail, since it
% does not require choosing a fixed sampling rate. 


% Our method is in sharp contrast to previous work in that

\noindent\textbf{Reverse engineering CAD reconstruction.} 
Over the past decades, reverse engineering has been extensively studied; %in computer vision and graphics and computer vision communities; 
it aims at converting measured data (a surface mesh or a point cloud) into solid 3D models that can be further edited and manufactured by industries. %that can be used in CAD software for further operations. 
%Given a 3D shape represented as a surface mesh or a point cloud, 
Traditional approaches addressing this problem consist of the following tasks: (1) segmentation of the point clouds/meshes~\cite{benkHo2004segmentation,zhang2020blending,Shen2022framework}, (2) fitting of parametric primitives to segmented regions~\cite{schnabel2007efficient,cohen2004variational,yan2012variational}, (3) finishing operations for CAD modeling~\cite{benkHo2001algorithms,langbein2004choosing}. 
Important drawbacks of these conventional methods are the time-consuming process and the requirement of a skilled operator to guide the reconstruction~\cite{buonamici2018reverse}. %the imposition of a time-consuming framework and the requirement of a highly skilled user to guide the reconstruction~\cite{buonamici2018reverse}. 

\begin{figure*}[!t]
\centering
  \includegraphics[width=1.0\linewidth]{figs/network_v6_compress.pdf}   
  \caption{\textbf{Network architecture for SECAD-Net}: The embedding $\mathbf{z}$ encoded from the voxel input is first fed to the extrusion box head to predict extrusion boxes. It is also sent to the sketch head network to calculate the sketch SDF $\hat{\mathcal{S}}_{\mathsf{sk}}^{i}$ after concatenating with the linear transformed sampling point. $\hat{\mathcal{S}} _{\mathsf{cyl} }^i$ stands for the SDF of the cylinder, which is acquired by extruding $\hat{\mathcal{S}}_{\mathsf{sk}}^{i}$ with height $h_i$. Then we convert $\hat{\mathcal{S}} _{\mathsf{cyl} }^i$ to occupancy of cylinder $\hat{\mathcal{O}}_i$ and finally obtain the complete shape by union all the occupancies. %We detail our framework in Sec.~\ref{sec:method}.
  }
  \label{fig:overview}
\end{figure*}

%To overcome the overwhelming complexity of previous methods, 
With the release of several large-scale CAD datasets (\eg, ABC~\cite{koch2019abc}, Fusion 360~\cite{willis2021fusion}), SketchGraphs~\cite{seff2020sketchgraphs}), 
numerous approaches have explored deep learning to address primitive segmentation/detection~\cite{yan2021hpnet,le2021cpfn}, parametric curve or surface inference from point clouds~\cite{li2019supervised,sharma2020parsenet,paschalidou2019superquadrics,wang2020pie,guo2022complexgen} or B-rep models~\cite{lambourne2021brepnet,jayaraman2021uv}. However, by only outputting individual curves or surfaces, these methods lack the CAD modeling operations that are needed to build solid models. 
Focusing on CAD generation rather than reconstruction task as ours, some approaches propose deep generative models that predict sequences of CAD modeling operations to produce CAD designs~\cite{li2020sketch2cad,xu2021inferring,wu2021deepcad,willis2021fusion,xu2022skexgen}. 
Aiming at CAD reconstruction involving inverse CSG modeling~\cite{du2018inversecsg}, CSGNet~\cite{sharma2018csgnet} first develops a neural model that parses a shape into a sequence of CSG operations. More recent works follow the line of CSG parsing by advancing the inference without any supervision~\cite{kania2020ucsg}, or improving representation capability with a three-layer reformulation
of the classic CSG-tree~\cite{ren2021csg}, or handling richer geometric and topological variations by introducing quadric surface primitives~\cite{yu2022capri}. While achieving high-quality reconstruction, CSG tends to combine a large number of shape primitives that are not as flexible as the extrusions of 2D sketches and are also not easily user edited to control the final geometry. 

% Motivated by modern design tools, Point2Cyl~\cite{uy2022point2cyl} adopts the sketch-extrude procedural model and learns 2D sketches that can be extruded to obtain 3D shapes. Our work differs from Point2Cyl in two significant ways: (1) Point2Cyl relies on laborious ground truth labels, including segmentation, normal, and sketch. In sharp contrast to that, SECAD-Net is trained in a self-supervised manner; (2) To achieve per-extrusion cylinder fitting, Point2Cyl requires surface-level segmentation. In contrast, SECAD-Net determines sketch planes by adaptively learning extrusion boxes.
\rev{Motivated by modern design tools, supervised methods are proposed~\cite{uy2022point2cyl,lambourne2022reconstructing} utilizing the sketch-extrude procedural models and learning 2D sketches that can be extruded to 3D shapes. In contrast to their reliance on 2D labels, SECAD-Net is trained in a self-supervised manner.}
\rev{Most closely related to our work is ExtrudeNet~\cite{ren2022extrudenet}. 
SECAD-Net distinguishes itself from ExtrudeNet in several significant aspects:
\romannumeral1) Following the traditional reconstruction process, ExtrudeNet first predicts the parameters of Bézier curves and then converts them into SDFs. In contrast, we jumped out of this paradigm and directly used neural networks to predict the 2D implicit fields of the profiles.
\romannumeral2) ExtrudeNet adopts closed Bézier curves to avoid self-intersection in sketches. This makes ExtrudeNet can only predict star-shaped profiles, which limits the expressive power of their CAD shapes. Our method does not impose any restrictions on the shape of the profile, thus having greater flexibility in shape expression.
\romannumeral3) To pursue the reconstruction effect, ExtrudeNet relies on a larger number of primitives, while our method is able to predict more compact CAD shapes.}

%Our method is in sharp contrast to previous work in that

% These are all trained utilizing supervision from ground truth sequence data. As they do not incorporate a loss function which
% directly compares the generated geometry and the target shape,
% their ability to match target geometry is limited.

% One
% early work is CSGNet [Sharma et al. 2017], which trains a neural
% network to infer the sequence of Constructive Solid Geometry (CSG)
% operations based on visual input. More recent works along this line
% of research include [Chen et al. 2020; Ellis et al. 2019; Kania et al.
% 2020; Tian et al. 2019]. 

% A different approach to CAD reconstruction is to solve the inverse
% problems of procedural CAD models. For example, inverse CSG
% [Du et al. 2018; Kania et al. 2020; Ren et al. 2021; Sharma et al.
% 2018] searches for CSG boolean operations and solid primitives
% that combine into the target shape; [Ganin et al. 2021; Para et al.
% 2021; Seff et al. 2021; Willis et al. 2021b; Wu et al. 2021] assume
% a “sketch+extrude” procedural model and study the generation of
% 2D sketches that can be extruded to obtain 3D shapes, by training
% on datasets of such modeling sequences [Seff et al. 2020; Willis
% et al. 2021c]. 

% the CSGNet paper [Sharma
% et al. 2018] trains a neural network that takes as input a 2D or 3D
% shape and outputs a CSG program. Compared to their work, our
% method does not require a training dataset and we demonstrate our
% algorithm on 3D shapes of much higher complexity. Wu et al. [2018]
% reconstruct a CSG tree from raw point clouds by extracting the
% primitives and inferring CSG tree structures. When building the
% CSG tree, they divide the bounding box into voxels and label each
% voxel as inside or outside the point cloud. A CSG tree is then built
% in a bottom-up manner by solving an energy minimization problem
% based on the labels of each voxel. Our work shares a similar pipeline
% but does not require discretizing the inputs into voxels, which allows
% us to handle inputs with details at various levels.

%3. CAD Datasets





 
%4. deep CSG and sketch
% The methods that either di-
% rectly learn the B-Rep structure of a CAD model [13, 18,
% 12, 45, 6] or predict sketches and CAD operations [43, 26,
% 30, 8], are closely related to our work. The works in [26, 8]



% parametric representation is constructive solid geometry (CSG) as it
% is a well understood, widely accepted staple in modern CAD systems
% and is compact in its representation. CSG encodes geometries as
% trees that are constructed by recursively applying boolean operators
% to primitive shapes [Requicha and Rossignac 1992]. Theory for the
% automatic conversion of 3D models to CSG trees has been widely
% studied for the past 20 years.

% Existing methods for representing meshes, such as BSP-N ET [ 5 ] and C VX N ET [ 6 ], achieve remarkable accuracy on a reconstruction tasks. However, the process of generating the mesh from predicted planes requires an additional post-processing step. These methods also assume that any object can be
% decomposed into a union of convex primitives. While holding, it requires many such primitives to represent concave shapes. Consequently, the decoding process is difficult to explain and modified with some external expert knowledge. On the other hand, there are fully interpretable approaches,
% like CSG-N ET [ 7 , 8 ], that utilize CSG parse tree to represent 3D shape construction process. Such solutions require expensive supervision that assumes assigned CSG parse tree for each example given during training.


\section{Problem Statement and Overview}

In this section, we present an overview of the proposed approach. To precisely explain our techniques, we first provide the definition of several related terminologies (Fig.~\ref{fig:definition}).

\begin{figure}[!t]
    \centerline{
    \includegraphics[width=1.0\linewidth]{figs/definition_v4.pdf}
    }
    \caption{Definitions of CAD terminologies used in this paper. Note that the axis of the sketch plane in the figure is the same as the z-axis in the extrusion box.}
    \label{fig:definition}
\end{figure}

\subsection{Preliminaries}

\noindent\textbf{Definition 1} (Loop, Profile and Sketch) 
\noindent\emph{In CAD terminology, a sketch is represented by a collection of geometric primitives. By referring to a closed curve as a loop and an enclosed region composed of one or multiple \rev{inner/outer} loops as a profile, we define a sketch as the collection of one profile and its loops.}
% \noindent\emph{In CAD terminology, a sketch can be a sequence of CAD instructions~\cite{seff2021vitruvion} or a collection of geometric primitives~\cite{seff2021vitruvion}. This paper adopts the latter definition. Referring a closed curve as loop, a closed region composed of one or multiple loops as profiles, we define a sketch as the collection of one profile and its loops.}
\\

\noindent\textbf{Definition 2} (Sketch plane and Extrusion box) 
\noindent\emph{A sketch plane is a finite plane with width $w$ and length $l$, containing one or more sketches with the same extrusion height $h$. Then we define an extrusion box as a cuboid with the sketch plane as the base and $2h$ as the height.}
\\

\noindent\textbf{Definition 3} (Cylinder primitive and Cylinder) 
\noindent\emph{In this work, a cylinder refers to the shape obtained by extruding a sketch, and a cylinder primitive is obtained by performing an extrude operation on a closed area formed by a loop. A cylinder may contain one cylinder primitive or be obtained from several cylinder primitives through the Difference operation used in CSG modeling.}

\subsection{Overview}
We formulate the problem of CAD reconstruction as \emph{sketch} and \emph{extrude} inference: taking an input 3D shape, SECAD-Net aims to reconstruct the CAD model by predicting a set of geometric proxies that are decomposed to sketch-extrude operations. 
The overall pipeline of SECAD-Net is visualized in Fig.~\ref{fig:overview}. 
%Given a 3D shape that can be represented as a point cloud or voxel grids, a voxel representation of a 3D shape
Given a 3D voxel model, we first map it into a latent feature embedding $\mathbf{z}$ by using an encoder based on a 3D convolutional network. 
An extrusion box head network is then applied to predict the parameters of the sketch planes from $\mathbf{z}$. We employ $N$ sketch head network to independently learn $N$ 2D signed distance fields (SDFs) as the implicit representation of a sketch. Next, we design a differentiable extrusion operator to calculate the SDF of the 3D cylinder primitives corresponding to the sketches. Finally, an occupancy transformation operation and a union operation transform the multiple SDFs into the full 3D reconstructed shape as the output of the network. 
% Note that SECAD-Net is an end-to-end self-supervised learning framework that does not require ground-truth labels in the training phase. %except for the voxel representation of the input shape in the training pipeline.



% Concatenate the two-dimensional coordinates and feature codes obtained by projecting the coordinates of the three-dimensional space sampling points along the normal of the $i$-th sketch plane into the MLP, and calculate the signed distance $D_i$ between the sampling points and the $i$-th sketch.

% We design a differentiable extrusion operator that extrudes $D_i$ into a 3D signed distance field $D_i$ according to the height $h_i$ of the $i$-th sketch. Finally, we convert all $D_i$ to primitive $O_i$ of occupancy representation and perform Union operation to get the final reconstructed shape. SECAD-Net, as an end-to-end self-supervised learning network, does not require other ground truths as labels in the training pipeline except the the input voxels.

\section{Methods}
\label{sec:methodology}
We conducted two studies after outlining the problem space, one from end-users' perspectives (Sec.~\ref{sub:methodology:end_user_surveys}), and the other from XAI/design/AR expert stakeholders' perspectives (Sec.~\ref{sub:methodology:expert_workshops}). The findings from the studies are complementary and provided insights that guided the development of the framework.

\subsection{Study 1: Large-Scale End-User Survey}
\label{sub:methodology:end_user_surveys}
In spite of the existing studies on XAI for end-users, it is unclear whether these findings hold for AR scenarios due to the unique features of AR systems.
Thus, we conducted a large-scale survey with end-users to collect their preferences on various aspects of XAI experiences for everyday AR.
% The survey results reveal the need of XAI in AR scenarios and are generally consistent with previous research outside XAI.

\subsubsection{Participants}
\label{subsub:methodology:end_user_surveys:participants}
\review{We recruited 506 participants from a third-party online user study platform (age 18 - 54, average 37$\pm$10), with a balanced gender distribution (Female 260, Male 241, Non-binary 5).}
Participants' digital literacy with AI varied, thus they were split into six groups: 1) unfamiliar with AI (12.2\%, 62), 2) heard of AI but never used AI-based products (23.5\%, 119), 3) used AI products occasionally a few times (23.1\%, 117), 4) used AI products on a regular basis (12.8\%, 65), 5) used AI products frequently (20.0\%, 101), and 6) worked on AI products (8.3\%, 42).
Participants were familiar with the concept of AR.
Among these groups, we further randomly sampled 20 participants (age 18 - 53, average 37$\pm$9, 11 Female, 9 Male) for a semi-structured interview to collect a more in-depth understanding about their preferences for XAI in AR.

\subsubsection{Design and Procedure}
\label{subsub:methodology:end_user_surveys:materials_design}
We prepared five sets of proof-of-concept descriptions and images with intelligent everyday AR services that represented five scenes in a typical weekday (\ie one set per scene). They included 1) music recommendations for the morning when users would be brushing their teeth, 2) podcast recommendations for when users would be driving to work, 3) music recommendations for when users would be working out, 4) recipe recommendations for when users would be making dinner, and 5) additional spice recommendations for when users would be making dinner.
In this study, we chose recommendations as the main AI service category, since it is arguably one of the most common AI applications in everyday AR~\cite{lam_a2w_2021,chatzopoulos2016readme} and users could easily contextualize these scenes in their mind.

% we pre-determined a set of explanations from different categories
For the AI outcome in each scene, participants were asked whether they wanted explanations (\ie yes, no, neutral). If their answer was yes, they would be directed to answer when they wanted it (\ie always/frequent, contextually dependent, rare/never), their preferred length of explanation (\ie concise \vs detailed) and the presenting modality (\eg visual, audio, neutral).
After viewing these scenes, they were asked to choose the explanation content types that they found useful. Participants were compensated \$5 USD for the task.
% In interviews, the experiment host would follow up with more questions based on participants' answers to collect detailed reasons behind their choices.

% including input/output, why/why-not, how, and certainty. The other three categories are not included as they may not compatible 

% \subsubsection{Procedure}
% \label{subsub:methodology:end_user_surveys:procedure}
% Participants were told to imagine themselves as end-users of AR devices. They then went through the five scenes and answered the questions.
% At the end of the survey, they were asked whether they would be interested in joining a follow-up interview study.
We randomly sampled 20 respondents who were willing to participate in a one-hour interview about the detailed reasons behind their survey responses.
These participants were compensated \$10 for the interview.
The interviews were video-recorded and manually transcribed.
\review{Two researchers collectively summarized and coded the data using a thematic analysis~\cite{braun2012thematic}. Specifically, they first met to establish an agreement on the themes and independently coded all the data. Then, they gathered to discuss and refine the coded data to resolve differences. Their inter-rater reliability ($\kappa$) was over 90\% after the refinement.
}

\subsubsection{Results}
\label{subsub:methodology:end_user_surveys:results}
The survey found that respondents had specific preferences for the timing, content, and modality of explanations.

\begin{figure*}[!b]
    \centering
    \vspace{-0.2cm}
    \begin{subfigure}[b]{.33\textwidth}
    \centering
    \includegraphics[width=1\columnwidth]{figures/percentage_whether.png}
    \caption{Need Explanations?}
    \label{subfig:survey_results:whether}
    \end{subfigure}
    \hfill
    \begin{subfigure}[b]{.34\textwidth}
    \centering
    \includegraphics[width=1\columnwidth]{figures/percentage_when.png}
    \caption{When to Have Explanations?}
    \label{subfig:survey_results:when}
    \end{subfigure}
    \hfill
    \begin{subfigure}[b]{.32\textwidth}
    \centering
    \includegraphics[width=1\columnwidth]{figures/percentage_what.png}
    \caption{What Explanations Are Preferred?}
    \label{subfig:survey_results:what}
    \end{subfigure}
    \vspace{-0.4cm}
    \caption{Highlight of Survey Results with 506 End-Users about Their Needs and Preferences of XAI in everyday AR scenarios.}
    \label{fig:survey_results}
    \Description[Survey Results.]{(a) A stacked percentage bar plot with six bars showing whether users need XAI with their AI experience. The six X-axis stick labels are: unfamiliar with AI, never heard of AI, use AI occasionally, use AI regularly, user AI frequently, and work in AI. The Y-axis is about the percentage. The bar chart shows that around 37\% of the "unfamilar with AI" replied "No". All other types of users have less than 10\% replied "No".
(b) A stacked percentage bar plot with six bars showing when users need XAI with their AI experience. The six X-axis stick labels are the same. And it shows that over 60\% of the users prefer the explanations to be "contextually dependent". As users get more experience with AI, the percentage of choosing "always or frequently" increases, and the percentage of choosing "rare or never" decreases.
(c) A group bar plot with six groups showing when users need XAI with their AI experience. The six X-axis stick labels are the same. Each group has four categories: Input/Output, Why/Why-Not, How, and Certainty. As users get more experience with AI, the percentage of choosing these four categories increases (from around 15\% on average to 45\% on average).}
    \vspace{-0.3cm}
\end{figure*}

\review{\textbf{Finding 1: Most users wanted explanations of AI outputs in AR.}} (related to \colorwhen{\textit{when - availability}}).
A large proportion of respondents wanted explanations (89.7\%), motivating the need for XAI in everyday AR scenarios (see Fig.~\ref{subfig:survey_results:whether}).
Our findings were consistent with previous work on end-users' needs for XAI outside AR~\cite{ehsan2021explainable,ttc_labs}.
The results indicated that if respondents had at least heard of AI, they were more likely to express a need for XAI in AR compared to those who were not familiar with AI.
\review{
% An ANOVA with digital literacy as the main factor indicates statistical significance ($F_{5,500} = 10.8, p < 0.001$).
% Generally, the more familiar respondents were with AI, the more they felt there to be a need for explanations.
Our interviews found that respondents with little knowledge of AI didn't realize what explanations could be used for. 
Interestingly, around 10\% of respondents who worked on AI indicated that they didn't want explanations. Our interviews revealed the main reason being that some users were \textit{``familiar enough... with the algorithm''} (P2).
% For example, \pquote{4}{I can understand why... it's not really needed}. These participants already had knowledge of AI and thus didn't need additional explanations.
% This finding is consistent with previous work outside AR~\cite{ehsan2021explainable,ttc_labs}.
}
% 
% Moreover, our interview results also suggest that when participants were unsatisfied 

\textbf{Finding 2: The majority of users wanted explanations to be occasional and contextual, especially when they saw anomalies} (related to \colorwhen{\textit{when - delivery}}).
Although most respondents wanted explanations, only 13.8\% indicated that they needed explanations all the time.
% , mostly from participants who work on AI (see Fig.~\ref{subfig:survey_results:when}) -- an interesting polarization of XAI needs of users with high digital literacy when comparing \textbf{Finding 1} \& \textit{2}.
The majority of respondents (63.4\%) preferred for explanations to be presented contextually only when they have the need.
% For example, P7 didn't think there was a need for explanations in the morning music recommendation scene, \pquote{7}{I wouldn't be surprised if it knew what I wanted to listen to in the morning. I'm a creature of habit.}
The results of the interviews indicated that the need for explanations was mainly in cases where AI outcomes were new or anomalous to respondents. This finding is also in line with previous studies' findings outside AR~\cite{dhanorkar_who_2021,jiang_who_2022}.

\textbf{Finding 3: Users generally preferred specific types of explanations} (related to \colorwhat{\textit{what - content}}).
Four explanation content types stood out as useful: Input/Output (41.5\%), How (37.1\%), Why/Why-Not (31.6\%), and Certainty (30.6\%).
The first three types were highlighted in previous findings about context-aware systems~\cite{lim_assessing_2009,lim_toolkit_2010}, while the last type has been adopted by industrial practitioners~\cite{google_map_match_rate_2018,spotify_blend_taste}.
As shown in Fig.~\ref{subfig:survey_results:what}, respondents with more knowledge of AI would prefer having these explanation types more than those with less AI knowledge.
% (ANOVA $F_{5,2018} = 15.4, p < 0.001$),
% which aligns with \textbf{Finding 1}.

\textbf{Finding 4: Users found detailed and personalized explanations useful} (related to \colorwhat{\textit{what - detail}}).
Although showing more explanation content can introduce additional cognitive costs, 48.3\% of respondents reported that they would find detailed explanations with multiple content types to be useful.
% \pquote{1}{It gives me more context, substance for why I need to take this suggestion.}
Moreover, respondents indicated that explanations that included personal preferences would be more convincing, \eg \pquote{13}{more personable, more upbeat}.
These results suggest that there is a need to provide options to modulate the level of explanation detail (see Sec.~\ref{subsub:problem_space_factors:problem_space:what}) and the \textit{User Profile} factor in the framework).

\textbf{Finding 5: Users' preferences for modalities depended on the cognitive load in an AR scenario} (related to \colorhow{\textit{how - modality}}).
The five scenes introduced different levels of cognitive load, which led respondents' preferences for XAI modality to vary. We found that for scenes with complex visual stimuli such as driving, respondents tended to prefer audio explanations over visual ones by 40\%, as they were \pquote{8}{more easy and convenient}.
This suggests that it is necessary to take modality bandwidths into account when choosing \colorhow{\textit{how}} to present XAI in different AR scenarios~\cite{buchner2022impact}.

Overall, these findings motivated the need for XAI in AR (\textbf{Finding 1}).
% , and were aligned with previous studies on XAI outside of AR (\textbf{Finding 1, 2} \& \textit{3}).
% This verifies that previous findings are transferable to AR scenarios.
Moreover, these results (\textbf{Finding 2-5}) also provided guidance for design XAI for end-users in AR.

\subsection{Study 2: Iteration with Expert Workshops}
\label{sub:methodology:expert_workshops}
Based on the existing literature and the end-user survey results, we created an early draft of the framework. Since XAIR aims to support designers and researchers during their design process, we utilized our draft within three workshops with expert stakeholders to collect their insights and finalize the framework.

% \footnotetext{By expert stakeholders, we refer to experts related to our research problem (XAI, design, UX, HCI, AR), rather than domain experts as XAI users. We will use the word ``experts'' as this meaning for the rest of the paper.}

\subsubsection{Participants}
\label{subsub:methodology:expert_workshops:participants}
Twelve participants (7 Female, 5 Male, Age 35 $\pm$ 6) from a technology company volunteered to participate in the study. They came from four backgrounds, \ie 3 XAI algorithm developers, 3 designers, 3 UX professionals, and 3 HCI/AR researchers. Participants worked in their domains for at least five years. All participants were familiar with the concept of AI and AR. Participants were randomly assigned into three groups, with each group containing one expert from each domain.

\subsubsection{Design and Procedure}
\label{subsub:methodology:expert_workshops:materials_design}
We proposed a draft of the framework combining the summary of literature and the results of end-user study. It was an early version of XAIR that is introduced in Sec.~\ref{sec:framework} and can be found in Appendix~\ref{sec:appendix:earlier_frameworks}.
We also prepared a set of everyday AR scenarios similar to the ones used in the end-user survey (Sec~\ref{sub:methodology:end_user_surveys}) to provide more context and stimulate more insights from experts.
We utilized a Figma board to show images of the framework and experts could add in-place feedback to different areas of the framework.

We adopted an iterative process using three sequential workshops. \review{All workshops lasted about 90 minutes and were video-recorded. After each workshop, two researchers went through a similar coding and refining process as Sec.~\ref{subsub:methodology:end_user_surveys:materials_design}, to make sure the result achieved a inter-rater reliability ($\kappa$) over 90\%. We summarized experts' feedback, iterated on the framework, and presented the new version in the next workshop.}
% \subsubsection{Procedure}
% \label{subsub:methodology:expert_workshops:procedure}
% After participants signed the consent form, we conducted the workshop with a group of participants. 
% In each workshop, we introduced our framework and walked through an example intelligence everyday AR scenario to show the use case of our framework. We then asked participants to apply the framework on another scene to get a deeper understanding of the framework.
% Participants were encouraged to raise questions or give feedback at any time during the whole process.
% We repeated the process with the other two workshops, each time with an iterated version of the framework.
% met to establish an agreement on the themes. They independently coded all the data, and gathered together to refine the coded data to make sure the result achieved a inter-rater reliability $\kappa$ over 90\%.

% We then update the framework based on the final results.

\subsubsection{Results}
\label{subsub:methodology:expert_workshops:results}
Overall, experts found the framework to be \pquote{2, P6, P7}{useful} and that it would \pquote{11}{serve as a very good reference for design}.
Our framework converged as the workshops proceeded, with us receiving rich feedback during the first workshop, and participants in the last workshop only offering small suggestions. We briefly highlight the major comments that were made.

\textbf{Suggestion 1: Add Missing Pieces.}
Participants found a few factors missing in the early version of the framework.
% For example, they suggested that the \textit{System Goal} error management should also be considered for the automatic delivery (\colorwhen{\textit{when}}) of explanations if the model was uncertain.
% and that the 
\review{
For example, they pointed out that \textit{User Goal} and \textit{User Profile} needed to be considered for the \colorwhat{\textit{what}} part, and that the modality of AI output in AR needed to be taken into account for the \colorhow{\textit{how}} part.
}
% as it would indicate the need for different explanations.
They also provided suggestions on appropriate explanation content types with different system/user goals (\colorwhat{\textit{what - content}}).

\textbf{Suggestion 2: Remove Redundancy.}
Participants also found some parts unnecessarily complex. For example, four experts suggested removing the interface location from \colorhow{\textit{how}} part (\ie where to explain, mentioned in Sec.~\ref{subsub:problem_space_factors:problem_space:how}), because the location needed to be optimized with the whole interface including AI outcomes.

\textbf{Suggestion 3: Add Default Options.}
Participants provided advice for default options of different dimensions.
For instance, they recommended using the manual-trigger as the default delivery method (\colorwhen{\textit{when}}) due to users' limited cognitive capacity in AR.
% We worked with participants to summarize the advice into a set of guidelines.

\textbf{Suggestion 4: Connect across Sub-questions.}
Participants came to the consensus that the three sub-questions were interwoven.
For example, the choice of \colorwhat{\textit{what}} to explain would influence the design of \colorhow{\textit{how}} to explain, and the framework should capture and emphasize such connection.
% The interface design of the \colorhow{\textit{how}} part would also need to consider the manual-/auto-trigger mechanism for the \colorwhen{\textit{when}} part.

\textbf{Suggestion 5: Improve Visual Structure.}
Finally, participants also offered several suggestions about the visual simplification, clarification, and color choices. The figures in Appendix~\ref{sec:appendix:earlier_frameworks} show the evolution of the visual structure.

The results of the end-user study and expert workshops are complementary and guided the final version of the framework.
\section{EXPERIMENTS AND ANALYSIS}
\label{results}
\subsection{Experiment Settings}
The ConvS2S model has 512 hidden units for both encoders and decoders. All embeddings, including the output produced by the decoder before the final linear layer, have a dimensionality of 768. This setup allows the encoders to concatenate with patch embeddings from ViT model. To avoid overfitting, dropout is applied on the embeddings, decoder output, and the input of the convolutional blocks with a retaining probability of 0.5.


% We train the convolutional model using Adam optimizer with a fixed learning rate 2.50e-4.
Many experiments are carried out in order to evaluate the proposed approach toward the VLSP-EVJVQA challenge. We begin by initializing the baseline result of ConvS2S without using any image information. This mean that the generated answers are completely based on the answer-question dependencies learned by the model during the training phase. We then sequentially add hint and image features to the input sequence and study their effect on the overall performance. Because of the limitation in computational resources as well as the strict timeline of the competition, we only deploy the fine-tuned ViLT-B/32 with 200K pretraining steps and pre-trained OFA$_{\mathrm{large}}$ with 472M parameters for hints inference given the question and image.
To have the comparative result, we set up the same hyperparameters for all experiments. The models are trained in 30 epochs using Adam optimizer with a fixed learning rate of 2.50e-4 and batch size of 128. After each epoch, the performance loss on the train and development sets is calculated using the Cross-Entropy Loss function.

The proposed architecture and SOTA vision and language models are implemented in PyTorch and trained on the Kaggle platform with hardware specifications: Intel(R) Xeon(R) CPU @ 2.00GHz; GPU Tesla P100 16 GB with CUDA 11.4.

\subsection{Experimental Results}
\begin{table}[H]

    \centering
    \resizebox{\columnwidth}{!}{%
    \setlength{\tabcolsep}{5pt}
    \renewcommand{\arraystretch}{1.2}
    \begin{tabular}{lcccccc}
    \toprule
        \textbf{Model} & \textbf{F1} & \textbf{BLEU-1} & \textbf{BLEU-2} & \textbf{BLEU-3} & \textbf{BLEU-4} & \textbf{BLEU (Avg.)}  \\ \midrule
        ConvS2S (no image features) & 0.3005 &0.2592	&0.2034	&0.1677	&0.1425& 0.1932  \\ \midrule
        ConvS2S + ViLT-B/32 & 0.3294 &0.2692	&0.2109	&0.1723	&0.1446& 0.1993  \\ 
        ConvS2S + OFA$_{\mathrm{large}}$ & 0.3331 &0.2858	&0.2269	&0.1876	&0.1598 & 0.2150  \\ 
        \textbf{ConvS2S + ViLT-B/32 + OFA$_{\mathbf{large}}$}
        % \tablefootnote{This model is not yet evaluated on the private test set \label{note1}}
        & \textbf{0.3442} &0.2797	&0.2205	&0.1808	&0.1529& \textbf{0.2085}  \\ 
        \midrule
                ConvS2S + ViT-B/16 & 0.3109 &0.2683	&0.2119	&0.1747	&0.1480 & 0.2007  \\ %\midrule
        ConvS2S + ViT-B/16 + ViLT-B/32 & 0.3361 &0.2833	&0.2243	&0.1845	&0.1564 & 0.2122  \\ 
        ConvS2S + ViT-B/16 + OFA$_{\mathrm{large}}$ & 0.3390 &0.2877	&0.2276	&0.1877	&0.1593 & 0.2156  \\
        \textbf{ConvS2S + ViT-B/16 + ViLT-B/32 + OFA$_{\mathbf{large}}$}
        % \textsuperscript{\ref{note1}}
        & \textbf{0.3442} & 0.2747	&0.2148	&0.1747	& 0.1465& \textbf{0.2027} \\ \bottomrule
    \end{tabular}}
    \caption{Performance of ConvS2S with different combinations of pre-trained models on the public test set.}
    \label{result_public}
\end{table}

\begin{figure}[ht]
\centering
% \subfloat[ConvS2S training loss per epoch]{%
%   \includegraphics[width=0.495\textwidth]{figure/train_loss1.pdf}%
% }
% \hspace{-0.2em}
% \subfloat[ConvS2S testing loss per epoch]{%
%   \includegraphics[width=0.495\textwidth]{figure/test_loss1.pdf}%
% }
\includegraphics[width=\textwidth]{figure/all_loss.pdf}
\caption{Training loss and public testing loss comparison of ConvS2S model with different combinations of hint and image features}
\label{loss}
\end{figure}


The two metrics: F1 and BLEU, are used in the challenge to evaluate the results. The BLEU score is the average of BLEU-1, BLEU-2, BLEU-3, and BLEU-4. F1 is used for ranking the final results. Table \ref{result_public} presents the performance of the proposed ConvS2S model with different combinations of pre-trained models on the UIT-EVJVQA public test set.

% First, with only question as input, ConvS2
According to Table \ref{result_public}, the original ConvS2S model without image features but using only question obtained 0.3005 by F1 and 0.1932 by BLEU. When integrating hint features from images, the F1 score improved at least 2.89\% and achieve highest result with 0.3442 by F1 and 0.2085 by BLEU when using both ViLT and OFA hints. After adding image feature from ViT-B/16, the performance of previous models tend to improve. However the final ensemble does not surpass the ConvS2S{\tiny~}+{\tiny~}ViLT-B/32{\tiny~}+{\tiny~}OFA$_{\mathrm{large}}$ ensemble on F1 metrics and even give lower BLEU score. Based on F1, these two ensembles are considered as our best models on the public test set. 
Figure \ref{loss} depicts the gradual improvement in both training loss and testing loss as more image features are added to the ConvS2S model. Memory-based ConvS2S does not catch the image context and thus have the highest loss. Though ConvS2S with ViT+VILT features does not obtained a competitive result on F1 and BLEU score, it has the best loss among methods in the public test phase. In general, the optimal testing loss of methods is achieved between 14th and 20th epoch, then the models tend to be overfitting.


% \begin{figure}
%     \centering
%     \includegraphics[width=\textwidth]{figure/train_loss.pdf}
%     \caption{tmp}
%     \label{100score}
% \end{figure}
% \subsubsection{Qualitative analysis}
% \label{quali_analysis}

% \begin{figure}
%     \centering
%     \includegraphics[width=\textwidth]{figure/test_loss.pdf}
%     \caption{tmp}
%     \label{100score}
% \end{figure}
% \subsubsection{Qualitative analysis}
% \label{quali_analysis}

We manage to deploy two ensembles of ConvS2S using features from ViT-B/16 combined with hints from {\tiny~}ViLT-B/32 and {\tiny~}OFA$_{\mathrm{large}}$, respectively, for the final evaluation on private test set. As shown in Table \ref{result_private}, the ConvS2S{\tiny~}+{\tiny~}ViT-B/16{\tiny~}+{\tiny~}OFA$_{\mathrm{large}}$ model obtained the better result, which is 0.4210 by F1 and 0.3482 by BLEU, and ranked $3^{rd}$ in the challenge. Table \ref{ranking} shows the final standing at the EVLSP-EVJVQA competition, in which our best model perform poorer 1.82\% and 1.39\% by F1 compared with the first and second place solutions. Overall, there is a gap between F1 and BLEU scores.



\begin{table}[H]
    \centering
    \small
    %\resizebox{\columnwidth}{!}{%
    \setlength{\tabcolsep}{5pt}
    \renewcommand{\arraystretch}{1.2}
    \begin{tabular}{lcc}
    \toprule
        \textbf{Model} & \textbf{F1} & \textbf{BLEU}  \\ \midrule
        ConvS2S + ViT-B/16 + ViLT-B/32 &0.4053  &0.3228  \\
        \textbf{ConvS2S + ViT-B/16 + OFA$_{\mathbf{large}}$} & \textbf{0.4210}  & \textbf{0.3482}
  \\ \bottomrule
    \end{tabular}
    \caption{Performance on the private test set.}
    \label{result_private}
\end{table}

\begin{table}[!htbp]
\small
%\resizebox{\columnwidth}{!}{%
\centering
\begin{tabular}{clccccc}
\toprule
\multirow{2}{*}{\textbf{No.}} & \multirow{2}{*}{\textbf{Team name}} & \multicolumn{2}{c}{\textbf{Public Test}} && \multicolumn{2}{c}{\textbf{Private Test}} \\\cmidrule{3-4} \cmidrule{6-7}
                             &                                     & \textbf{F1}         & \textbf{BLEU}      && \textbf{F1}         & \textbf{BLEU}       \\\midrule
1                            & CIST AI                             & 0.3491              & 0.2508             && 0.4392              & 0.4009              \\
2                            & OhYeah                              & 0.5755              & 0.4866             && 0.4349              & 0.3868              \\
3                            & \textbf{DS\_STBFL}                  & \textbf{0.3390}     & \textbf{0.2156}    && \textbf{0.4210}     & \textbf{0.3482}     \\
4                            & FCoin                               & 0.3355              & 0.2437             && 0.4103              & 0.3549              \\
5                            & VL-UIT                              & 0.3053              & 0.1878             && 0.3663              & 0.2743              \\
6                            & BDboi                               & 0.3023              & 0.2183             && 0.3164              & 0.2649              \\
7                            & UIT\_squad                          & 0.3224              & 0.2238             && 0.3024              & 0.1667              \\
8                            & VC\_Internship                      & 0.3017              & 0.1639             && 0.3007              & 0.1337
\\\bottomrule       
\end{tabular}
\caption{Our performance compared with other teams at VLSP2022-EVJVQA}
\label{ranking}
\end{table}

\subsection{Performance Analysis}

According to the final result in the private test phase, the generated output from ConvS2S
+ViT-B/16+OFA$_{\mathrm{large}}$ model are chosen for further analysis. Generally, the model manages to generate answers with correct language with the input question.
\subsubsection{Quantitative analysis}
We randomly choose 100 samples from the generated result to perform quantitative analysis. The average length, vocabulary size, and the number of POS tags in the ground truth and generated answers are calculated for each language. Table \ref{quanti} shows the statistics of the ground truth answer compared with the predicted answer by the model.

% \begin{table}[ht]
% \centering
% %\resizebox{\columnwidth}{!}{%
% \begin{tabular}{llrr}
% \toprule
% &Language&Ground Truth&Predicted\\\midrule

% \multirow{ 4}{*}{Avg. length} & English & 3.74 & 6.18 \\
% & Vietnamese & 4.42 & 5.97\\
% & Japanese & 4.67 & 8.43\\
% & All &4.26&6.78\\\midrule

% \multirow{ 4}{*}{Vocab. size} & English & 78 & 72 \\
% & Vietnamese & 97 & 101\\
% & Japanese & 77 & 83\\
% & All &252&256\\\midrule

% \multirow{ 4}{*}{\# POS tag} & English & 12 & 9 \\
% & Vietnamese &10  &9 \\
% & Japanese & 10 & 11\\
% & All &14 &14\\
% \bottomrule
% \end{tabular}
% \caption{The quantitative statistic of 100 generated samples compared with the ground truth}
% \label{quanti}
% \end{table}

\begin{table}[ht]
\centering
%\resizebox{\columnwidth}{!}{%
\begin{tabular}{llrr}
\toprule
Language&Stats.&Ground Truth&Predicted\\\midrule
\multirow{ 3}{*}{English} & Avg.length  & 3.74 & 6.18 \\
& Vocab. size & 78 & 72 \\
& \# POS tag  & 12 & 9 \\\midrule

\multirow{ 3}{*}{Vietnamese} & Avg.length  & 4.42 & 5.97 \\
& Vocab. size  & 97 & 101 \\
& \# POS tag &10  &9 \\\midrule

\multirow{ 3}{*}{Japanese} & Avg.length   & 4.67 & 8.43 \\
& Vocab. size & 77 & 83 \\
& \# POS tag  & 10 & 11 \\\midrule\midrule

\multirow{ 3}{*}{All} & Avg.length  &4.26 &6.78 \\
& Vocab. size &252 &256 \\
& \# POS tag  &14 &14 \\

\bottomrule
\end{tabular}
\caption{The quantitative statistic of 100 generated samples compared with the ground truth}
\label{quanti}
\end{table}

From Table \ref{quanti}, it can be seen that although the model gave the answers longer than the ground truth answers, the semantics is not as much as the ground truth. It can be seen from Table \ref{quanti} that the predicted answers in English have an average length higher than the ground truth answers. Also, the vocabulary in the generated answers is more than the original. In contrast, the number of POS tag components in the predicted answers is lower than the ground truth. This is similar to the answers in Vietnamese. For the Japanese, the characteristics of the predicted answers in average length and vocabulary size are the same as the two remaining languages. However, the number of POS tags in the predicted answers is more than in the ground truth answers. To make it clear, we propose three types of error on our model in Section \ref{quali_analysis}.

In addition, Figure \ref{100score} illustrates the distributions of F1 and BLEU scores for each language. Generally, the histograms skewed to the right and the model  performs inconsistently across languages. The proportion of samples with F1 and BLEU scores less than 0.2 dominates the overall result across all three languages. In Vietnamese, the number of generated samples with F1 and BLEU scores greater than 0.4 is significantly higher than in other languages. Meanwhile, English and Japanese responses rarely score greater than 0.6 on both metrics, furthermore, no Japanese samples scoring greater than 0.8 in BLEU. This illustrates that our model faces numerous challenges in producing the desired responses, with specific limitations on each language.

\begin{figure}[!ht]
    \centering
    \includegraphics[width=\textwidth]{figure/hist.pdf}
    \caption{Distributions of F1 and BLEU scores for each language from 100 generated samples}
    \label{100score}
\end{figure}

\begin{figure}[!htbp]
\centering
\subfloat[]{%
  \includegraphics[width=0.8\textwidth]{figure/attns1.pdf}%
  \label{attn1}
}

\subfloat[]{%
  \includegraphics[width=0.8\textwidth]{figure/attns2.pdf}%
  \label{attn2}
}

\subfloat[]{%
  \includegraphics[width=0.8\textwidth]{figure/attns3.pdf}%
  \label{attn3}
}

\subfloat[]{%
  \includegraphics[width=0.8\textwidth]{figure/attns4.pdf}%
  \label{attn4}
}

\subfloat[]{%
  \includegraphics[width=0.8\textwidth]{figure/attns5.pdf}%
  \label{attn5}
}
\caption{Numerous samples of attention alignment from ConvS2S and the changes in attention when adding features from ViT-B/16 and OFA$_{\mathrm{large}}$. The x-axis and y-axis of each plot correspond to the words in the question and the generated answer, respectively, while each pixel illustrates the weight $w_{ij}$ of the assignment of the j-th question word for the i-th
answer word.}
\label{attn}
\end{figure}

\subsubsection{Qualitative analysis}
\label{quali_analysis}
\paragraph{Attention visualization}

Figure \ref{attn} shows several samples of attention weights between each element from the generated answer with those in the input sequence that contains no image features, OFA hints, and OFA+ViT features, respectively. The visualization provided an intuitive way to discover which positions in the input sequence were considered more important when generating the target answer word. The brighter a pixel's color, the more important the word in the input sequence is in producing the respect answer word. Through this, we study that OFA hint is importance feature to model's attention as it provide the near-correct insight for the question and reduce the reliance on question words when generating the answer. However, in some cases, the model focuses too much on a specific element of the hint, which may lead to bias. ViT features has shown to control the affection of OFA hint, neutralizing it with other elements from question if hint appears to be off-topic. It may enhance the attention, making the model focus stronger on specific parts of the provided hint, for instance, the hint token ``nhà hàng'' (\textit{restaurant}) in Figure \ref{attn3} is given more attention when adding ViT image features. These features can also reduce the attention in one element and distributes concentration on other parts of the sequence. Figures \ref{attn1} and \ref{attn2} depict the reduction in hint attention into question elements, while Figures \ref{attn4} and \ref{attn5} show the changes in attention weight distribution among hint tokens.

\paragraph{Error analysis}
\begin{figure}[!ht]
\centering
\subfloat[Error Case I]{%
  \includegraphics[width=\textwidth]{figure/err1.pdf}%
\label{fig:1a}}
\vspace{1em}
\subfloat[Error Case II]{%
  \includegraphics[width=\textwidth]{figure/err2.pdf}%
    \label{fig:1b}
}
\vspace{1em}
\subfloat[Error Case III]{%
  \includegraphics[width=\textwidth]{figure/err3.pdf}%
\label{fig:1c}}
\caption{Three typical error cases from generated results.}
\label{fig:1}
\end{figure}

% \begin{figure}[H]
% \centering
% \resizebox{\textwidth}{!}{
%     \begin{subfigure}[b]{.3\linewidth}
%     \centering
%     \includegraphics[width=0.99\textwidth]{figure/00000001682.jpg}
%     \raggedright
%     { \scriptsize \textbf{Question}: what hat does the narrator of the 
%     historical site wear?}\\
%     {\scriptsize \textbf{Groundtruth}: non la}\\
%     {\scriptsize \textbf{Predicted}: the boy wears a white shirt and white and white}\\
%     {\scriptsize \textbf{F1:}  0.0000}\\
%     {\scriptsize \textbf{BLEU:} 0.0000
%     ~~~~~~~~~~~~~~~~~~~~~~~~~~~~~~~~~~~~~~~~~~~~~~~~~~~~~~~~~~~~~~~~~~~~~~~~~~~~~~~~~~~~~~~~~~~~~~~~~~ }
%     \caption{Error Type I}
%     \label{fig:1a}
%   \end{subfigure}%
%   \hspace{0.5em}
  
%  %\hspace*{\fill}
%   \begin{subfigure}[b]{.35\linewidth}
%     \centering
%     \includegraphics[width=0.99\textwidth]{figure/00000004737.jpg}
%     \raggedright
    
%     {\scriptsize \textbf{Question}: có bao nhiêu người đứng bên phải chàng trai? (\textit{English: How many people on the right of the man?})}\\
    
%     {\scriptsize \textbf{Groundtruth}: có ba người đứng bên phải chàng trai (\textit{English: There are three people on the right of the man})}\\
    
%     {\scriptsize \textbf{Predicted}: có hai người đứng bên phải chàng trai (\textit{English: There are two people on the right of the man})}\\
    
%     {\scriptsize F1:  0.8750}\\
    
%     {\scriptsize BLEU: 0.7799}
%     \caption{Error Type II}
%     \label{fig:1b}
%   \end{subfigure}%
%   \hspace{0.5em}
%   %\hspace*{\fill}
%   \begin{subfigure}[b]{0.35\linewidth}
%      \centering
%     \includegraphics[width=0.99\textwidth]{figure/00000000111.jpg}
    
%     \raggedright {\scriptsize \textbf{Question}:}
%     {\tiny
%     \begin{CJK*}{UTF8}{min}
%     {\CJKfamily{goth}小船手は何本のオールを使っていますか? (\scriptsize \textit{English: How many paddles does the boatman use?})}
%     \end{CJK*}}\\
%     {\scriptsize \textbf{Groundtruth}: 2}\\
%     {\scriptsize \textbf{Predicted}:}
%     {\tiny
%     \begin{CJK*}{UTF8}{min}
%     {\CJKfamily{goth}2本の船を使っています (\scriptsize \textit{ English: using two boats})}
%     \end{CJK*}}\\
%     {\scriptsize \textbf{F1:} 0.0000}\\
%     {\scriptsize \textbf{BLEU:} 0.0000 ~~~~~~~~~~~~~~~~~~~~~~~~~~~~~~~~~~~~~~~~~~~~~~~~~~~~~~~~~~~~~~~~~~~~~~~~~~~~~~~~~~~~~~~~~~~~~~~~~~ }\\
%     \caption{Error Type III}
%     \label{fig:1c}
%   \end{subfigure}%  
% }
%   \caption{Example of generated answers that contains errors.(b) the keyword 'hai người' (two people) is given  instead of 'ba người' (three people). Coincidentally, the question and groundtruth in this case both share the same phrase "đứng bên phải chàng trai" ("on the right of the man"), }\label{fig:1}
% \end{figure}


For better understand the generation performance on the VQA task, we examine the generated answers of our best ensemble, ConvS2S
+ViT-B/16+OFA$_{\mathrm{large}}$, to identify the limitations and analyze factors that may cause the model to perform poorly.
Through the error analysis process, various errors and mistakes have been pointed out in the outputs of the model. The typical examples of various types of errors are illustrated in Figure \ref{fig:1}. In summary, we divide these errors into three groups:

\begin{itemize}
    \item The generated answer does not match the question and has no correct tokens compared with the ground truth answer, as shown in Figure \ref{fig:1a}. This error case sometimes accompanied by text degeneration.
    \item The generated response gives the wrong answer to the question but share some insignificant tokens with the ground truth answer, as shown in Figure \ref{fig:1b}, which significantly improves the evaluation score. This incorrect scenario exemplifies the limitation of the evaluation measures.
    \item The model managed to generate the correct key answer while also adding unnecessary information compared to the ground truth, which may lead to the response's meaning being distorted. 
    As shown in Figure \ref{fig:1c}, the model correctly predicted quantity but then added unnecessary tokens afterward, resulting in a low score on both evaluation metrics.
\end{itemize}


% \begin{figure*}[h]
% \centering
%   \begin{tabular}{@{}ccc@{}}
%     \includegraphics[width=0.3\textwidth]{figure/5.3_ex/00000000111.jpg}
%     \includegraphics[width=0.3\textwidth]{example-image-b} &
%     \includegraphics[width=0.3\textwidth]{example-image-b} \\
%   \end{tabular}
%   \caption{This is some figure side by side}
% \end{figure*}


\section{Conclusion and Future Work}
\label{sec:conclusion}

We have presented a novel neural network that successively learns shape sketch and extrusion without any expensive annotations of shape segmentation and labels as the supervision.
%Without the guidance of sketch labels, 
Our approach is able to learn smooth sketches, followed by the differentiable extrusion to reconstruct CAD models that are close to the ground truth. 
We evaluate SECAD-Net using diverse CAD datasets and demonstrate the advantages of our approach by ablation studies and comparing it to the state-of-the-art methods. 
We further demonstrate our method’s applicability in single-image CAD reconstruction. 
Additionally, the CAD shapes generated by our approach can be directly fed into off-the-shelf CAD software for sketch-level or cylinder primitive-level editing. 

% We tested SE-Net on ABC dataset and Fusion 360 dataset. Quantitative results demonstrate that SE-Net can efficiently reconstruct 3D CAD shapes. Qualitative results show that our model can learn fine 2d sketches without any associated ground-truth.


% We propose SE-Net, a network that successively learns shape sketch and extrusion in an unsupervised manner. The CAD shapes generated by the network can be directly sent to off-the-shelf CAD software for sketch-level or cylinder primitive-level editing. SE-Net can be reconstructed to generate smooth sketches and the reconstruction effect is due to the current state-of-the-art, including supervised methods. Additionally, our method is the first to learn sketches from raw shapes without the guidance of sketch labels.

In future work, we plan to extend our approach to learn more CAD-related operations such as \emph{revolve, bevel, and sweep}. %using neural methods. 
Besides, we find that current deep learning models perform poorly on datasets with large differences in shape geometry and structure. %structural and topological variations
Therefore, another promising direction is to explore how to improve the generalization of neural networks and enhance the realism of the generated shapes by learning structural and topological information.
\\

\noindent\textbf{Acknowledgments.} 
We thank the anonymous reviewer for their valuable suggestions. 
This work is partially funded by the National Natural Science Foundation of China (U22B2034, 62172416, U21A20515, 62172415), and the Youth Innovation Promotion Association of the Chinese Academy of Sciences (2022131).

%%%%%%%%% REFERENCES
{\small
\bibliographystyle{ieee_fullname}
\bibliography{egbib}
}


\end{document}
