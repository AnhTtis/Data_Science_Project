\documentclass[
12pt, % The default document font size, options: 10pt, 11pt, 12pt
%oneside, % Two side (alternating margins) for binding by default, uncomment to switch to one side
english, % ngerman for German
onehalfspacing, % Single line spacing, alternatives: onehalfspacing or doublespacing
%draft, % Uncomment to enable draft mode (no pictures, no links, overfull hboxes indicated)
%liststotoc, % Uncomment to add the list of figures/tables/etc to the table of contents
%toctotoc, % Uncomment to add the main table of contents to the table of contents
%parskip, % Uncomment to add space between paragraphs
%nohyperref, % Uncomment to not load the hyperref package
headsepline,
openany,
 % Uncomment to get a line under the header
%chapterinoneline, % Uncomment to place the chapter title next to the number on one line
%consistentlayout, % Uncomment to change the layout of the declaration, abstract and acknowledgements pages to match the default layout
]{MastersDoctoralThesis} % The class file specifying the document structure
\usepackage{booktabs}
\usepackage[utf8]{inputenc} % Required for inputting international characters
\usepackage[T1]{fontenc} % Output font encoding for international characters
%\usepackage{algorithm}
%\usepackage{algorithmicx}
%\usepackage{algpseudocode}
\usepackage{amsmath}
\usepackage{xcolor}%定义了一些颜色
\usepackage{colortbl,booktabs}%第二个包定义了几个*rule
\usepackage{amsmath}
\usepackage{caption}
\usepackage{subfigure}
\usepackage{tabularx}
\usepackage{graphicx}
\usepackage{picinpar}
\usepackage{amsmath}
\usepackage{url}
\usepackage{flushend}
%\usepackage{algpseudocode}
\usepackage{colortbl}
\usepackage{soul}
\usepackage{multirow}
\usepackage{pifont}
\usepackage{color}
\usepackage{alltt}
\usepackage{mathtools}
\usepackage[hidelinks]{hyperref}
\usepackage{enumerate}
%\usepackage{siunitx}
\usepackage{breakurl}
\usepackage{epstopdf}
\usepackage{pbox}
\usepackage{crop}
\usepackage{graphicx} %插入图片的宏包
\usepackage{float} %设置图片浮动位置的宏包
\usepackage{subfigure} %插入多图时用子图显示的宏包
\usepackage{CJK}
\usepackage{mathpazo}
\usepackage{rotating}
\usepackage{bbm}
\usepackage{algorithmic}
\usepackage[ruled,vlined]{algorithm2e}
\usepackage[normalem]{ulem}
\usepackage{adjustbox}



 % Use the Palatino font by default
%\usepackage[backend=bibtex,style=authoryear,natbib=true]{biblatex}
\usepackage[backend=bibtex,natbib=true]{biblatex} % Use the bibtex backend with the authoryear citation style (which resembles APA)


\addbibresource{example.bib} % The filename of the bibliography

\usepackage[autostyle=true]{csquotes} % Required to generate language-dependent quotes in the bibliography

%----------------------------------------------------------------------------------------
%	MARGIN SETTINGS
%----------------------------------------------------------------------------------------

\geometry{
	paper=a4paper, % Change to letterpaper for US letter
	left=3.5cm, % Inner margin
	right=2.5cm, % Outer margin
	 % Binding offset
	top=2.5cm, % Top margin
	bottom=2.5cm, % Bottom margin
	%showframe, % Uncomment to show how the type block is set on the page
}

%----------------------------------------------------------------------------------------
%	THESIS INFORMATION
%----------------------------------------------------------------------------------------

\thesistitle{Medical Image Analysis using Deep Relational Learning} % Your thesis title, this is used in the title and abstract, print it elsewhere with \ttitle
\supervisor{Dr. Huiyu \textsc{Zhou} } % Your supervisor's name, this is used in the title page, print it elsewhere with \supname

\examiner{} % Your examiner's name, this is not currently used anywhere in the template, print it elsewhere with \examname
\degree{Master of Philosophy} % Your degree name, this is used in the title page and abstract, print it elsewhere with \degreename
\author{Zhihua \textsc{Liu}} % Your name, this is used in the title page and abstract, print it elsewhere with \authorname
\addresses{} % Your address, this is not currently used anywhere in the template, print it elsewhere with \addressname

\subject{Computer Science} % Your subject area, this is not currently used anywhere in the template, print it elsewhere with \subjectname
\keywords{Machine learning, Computer Vision, Medical Image Analysis} % Keywords for your thesis, this is not currently used anywhere in the template, print it elsewhere with \keywordnames
\university{\href{https://le.ac.uk/}{University of Leicester}} % Your university's name and URL, this is used in the title page and abstract, print it elsewhere with \univname
\department{\href{https://le.ac.uk/informatics}{School of Informatics}} % Your department's name and URL, this is used in the title page and abstract, print it elsewhere with \deptname
\group{\href{https://sites.google.com/site/huiyujoe/}{Biomedical Image Processing Lab}} % Your research group's name and URL, this is used in the title page, print it elsewhere with \groupname
\faculty{\href{http://faculty.university.com}{}} % Your faculty's name and URL, this is used in the title page and abstract, print it elsewhere with \facname

\AtBeginDocument{
\hypersetup{pdftitle=\ttitle} % Set the PDF's title to your title
\hypersetup{pdfauthor=\authorname} % Set the PDF's author to your name
\hypersetup{pdfkeywords=\keywordnames} % Set the PDF's keywords to your keywords
}

\begin{document}
\linespread{1.25}
\frontmatter % Use roman page numbering style (i, ii, iii, iv...) for the pre-content pages

\pagestyle{plain} % Default to the plain heading style until the thesis style is called for the body content

%----------------------------------------------------------------------------------------
%	TITLE PAGE
%----------------------------------------------------------------------------------------

\begin{titlepage}
\begin{center}

%\vspace*{.06\textheight}
%{\scshape\LARGE \univname\par}\vspace{1.5cm} % University name
%\textsc{\Large Master Thesis}\\[0.5cm] % Thesis type

%\HRule \\[0.4cm] % Horizontal line
{\huge \bfseries \ttitle\par}\vspace{2.5cm} % Thesis title
%\HRule \\[1.5cm] % Horizontal line
 
%\begin{minipage}[t]{0.4\textwidth}
%\begin{flushleft} \large
%\emph{Author:}\\
%\href{}{\authorname} % Author name - remove the \href bracket to remove the link
%\end{flushleft}
%\end{minipage}
%\begin{minipage}[t]{0.4\textwidth}
%\begin{flushright} \large
%\emph{Supervisors:} \\
%\href{https://www2.le.ac.uk/departments/informatics/people/huiyu-zhou}{\supname} 
%\\
%\href{https://www2.le.ac.uk/departments/media/people/qian-sarah-gong}
%{Dr. Qian Gong}
%% Supervisor name - remove the \href bracket to remove the link  
%\end{flushright}
%\end{minipage}\\[2cm]
 
\vfill

\LARGE {Thesis submitted for the degree of \\ \degreename\\at the University of Leicester}\\[1cm] % University requirement text
\textit{by}\\[1cm]
{Zhihua Liu}\\ \groupname\\\deptname\\{University of Leicester}\\[2cm] % Research group name and department name
 
\vfill

{\LARGE \today}\\[4cm] % Date
%\includegraphics{Logo} % University/department logo - uncomment to place it
 
\vfill
\end{center}
\end{titlepage}

%----------------------------------------------------------------------------------------
%	DECLARATION PAGE
%----------------------------------------------------------------------------------------

\begin{declaration}
\addchaptertocentry{\authorshipname} % Add the declaration to the table of contents
\noindent I, \authorname, declare that this thesis titled, \enquote{\ttitle} and the work presented in it are my own. I confirm that:

\begin{itemize} 
\item This work was done wholly or mainly while in candidature for a research degree at this University.
\item Where any part of this thesis has previously been submitted for a degree or any other qualification at this University or any other institution, this has been clearly stated.
\item Where I have consulted the published work of others, this is always clearly attributed.
\item Where I have quoted from the work of others, the source is always given. With the exception of such quotations, this thesis is entirely my own work.
\item I have acknowledged all main sources of help.
\item Where the thesis is based on work done by myself jointly with others, I have made clear exactly what was done by others and what I have contributed myself.\\
\end{itemize}
 
\noindent Signed: Zhihua Liu\\
\rule[0.5em]{25em}{0.5pt} % This prints a line for the signature
 
\noindent Date: 23/Oct/2020\\
\rule[0.5em]{25em}{0.5pt} % This prints a line to write the date
\end{declaration}

%\cleardoublepage

%----------------------------------------------------------------------------------------
%	QUOTATION PAGE
%----------------------------------------------------------------------------------------

%\vspace*{0.2\textheight}

%\noindent\enquote{\itshape Thanks to my solid academic training, today I can write hundreds of words on virtually any topic without possessing a shred of information, which is how I got a good job in journalism.}\bigbreak

%\hfill Dave Barry

%----------------------------------------------------------------------------------------
%	ABSTRACT PAGE
%----------------------------------------------------------------------------------------

\begin{abstract}
\addchaptertocentry{\abstractname} % Add the abstract to the table of contents
Benefited from deep learning techniques, remarkable progress has been made within the medical image analysis area in recent years. However, it is very challenging to fully utilize the relational information (the relationship between tissues or organs or images) within the deep neural network architecture. Thus in this thesis, we propose two novel solutions to this problem called implicit and explicit deep relational learning. We generalize these two paradigms of deep relational learning into different solutions and evaluate them on various medical image analysis tasks.

Automated segmentation of brain glioma in 3D magnetic resonance imaging plays an active role in glioma diagnosis, progression monitoring and surgery planning. In this work, we propose a novel Context-Aware Network that effectively models implicit relation information between features to perform accurate 3D glioma segmentation. We evaluate our proposed method on publicly accessible brain tumor segmentation datasets BRATS2017 and BRATS2018 against several state-of-the-art approaches using different segmentation metrics. The experimental results show that the proposed algorithm has better or competitive performance, compared to the standard approaches.

Subsequently, we propose a new hierarchical homography estimation network to achieve accurate medical image mosaicing by learning the explicit spatial relationship between adjacent frames. We use the UCL Fetoscopy Placenta dataset to conduct experiments and our hierarchical homography estimation network outperforms the other state-of-the-art mosaicing methods while generating robust and meaningful mosaicing results on unseen frames.
\end{abstract}

%----------------------------------------------------------------------------------------
%	ACKNOWLEDGEMENTS
%----------------------------------------------------------------------------------------

\begin{acknowledgements}
\addchaptertocentry{\acknowledgementname} % Add the acknowledgements to the table of contents
First of all, I want to thank my father Mr. Yuhua Du, my mother Mrs. Zengxia Xiao and my fiancee Miss. Xinyu Wang. Thank them for their support to me and the family. I cannot repay the support from my family, and it is also my biggest motivation to continue scientific research.

Secondly, I want to thank my first supervisor, Prof. Huiyu Zhou. Prof. Huiyu Zhou has a rigorous attitude towards science. He also dedicates himself to his work. He is both a good teacher and a good friend. I am grateful to him for his continuous criticism and teaching, which have benefited me for life. At the same time, I would also like to thank my second supervisor, Prof. Yudong Zhang, for his professional guidance on my academic work.

Also, I would like to thank all colleagues of Biomedical Image Processing Lab (BIPL), especially Mr. Zheheng Jiang, Mr. Lei Tong, Mr. Long Chen, Mr. Feixiang Zhou, Mr. Jialin Lyu, Mr. Honghui Du. Thank all of you for your continuous support and help.

Finally, I would also like to thank all the collaborators, including, Dr. Qianni Zhang from Queen Mary Univeristy of London, Dr. Yinhai Wang from AstraZeneca R$\&$D Cambridge, Dr. Caifeng Shan from Philips Research Edinhoven, Prof. Ling Li from University of Kent, Prof. Xiangrong Zhang from Xidian University. Prof. Bin Yang from University of Leicester, Prof. Stephen McKenna from University of Dundee, Prof. Jianguo Zhang from Southern University of Science and Technology, Dr. Tianjun Huang from Southern University of Science and Technology. Dr. Sophia Bano from University College London. It is my honor to collaborate with these world-class researchers. Thank all collaborators for their trust, support and communication, which broadened my horizons and enabled me to move forward in the direction of machine learning and medical image processing. At the same time, I would also like to thank all the staff at the School of Informatics at the University of Leicester. Their professionalism has provided logistical support during my research work.

This thesis was written during the COVID epidemic in 2020. I hope you can stay safe and stay healthy if you are reading this dissertation.

\end{acknowledgements}

%----------------------------------------------------------------------------------------
%	LIST OF CONTENTS/FIGURES/TABLES PAGES
%----------------------------------------------------------------------------------------

\tableofcontents % Prints the main table of contents

\listoffigures % Prints the list of figures

\listoftables % Prints the list of tables

%----------------------------------------------------------------------------------------
%	ABBREVIATIONS
%----------------------------------------------------------------------------------------

%\begin{abbreviations}{ll} % Include a list of abbreviations (a table of two columns)
%\textbf{BraTS2017} & Multi Modal Brain Tumor Segmentation 2017\\
%\textbf{BraTS2018} & Multi Modal Brain Tumor Segmentation 2018\\
%\textbf{BIPL} & Biomedical Image Processing Lab\\

%\textbf{AI} & Artificial Intelligence\\

%end{abbreviations}

%----------------------------------------------------------------------------------------
%	PHYSICAL CONSTANTS/OTHER DEFINITIONS
%----------------------------------------------------------------------------------------

%\begin{constants}{lr@{${}={}$}l} % The list of physical constants is a three column table

% The \SI{}{} command is provided by the siunitx package, see its documentation for instructions on how to use it

%Speed of Light & $c_{0}$ & \SI{2.99792458e8}{\meter\per\second} (exact)%\\
%Constant Name & $Symbol$ & $Constant Value$ with units\\

%\end{constants}

%----------------------------------------------------------------------------------------
%	SYMBOLS
%----------------------------------------------------------------------------------------

%\begin{symbols}{lll} % Include a list of Symbols (a three column table)
%
%$a$ & distance & \si{\meter} \\
%$P$ & power & \si{\watt} (\si{\joule\per\second}) \\
%%Symbol & Name & Unit \\
%
%\addlinespace % Gap to separate the Roman symbols from the Greek
%
%$\omega$ & angular frequency & \si{\radian} \\
%
%\end{symbols}

%----------------------------------------------------------------------------------------
%	DEDICATION
%----------------------------------------------------------------------------------------


%----------------------------------------------------------------------------------------
%	THESIS CONTENT - CHAPTERS
%----------------------------------------------------------------------------------------

\mainmatter % Begin numeric (1,2,3...) page numbering

\pagestyle{thesis} % Return the page headers back to the "thesis" style

% Include the chapters of the thesis as separate files from the Chapters folder
% Uncomment the lines as you write the chapters

% Importance and appeal of children's drawings
Children's depictions of the human figure are highly expressive and varied.
As one of the very first subjects children attempt to draw, the representation begins as an almost unintelligible cloud of scribbles. 
As the child grows, their representation of the human figure becomes more developed and is extended to graphically represent many different types of characters: people, animals, and even personified objects (see Figure 1).

Who among us has not wished, either as a child or as an adult, to see such figures come to life and move around on the page?
Sadly, while it is relatively fast to produce a single drawing, creating the sequence of images necessary for animation is a much more tedious endeavor, requiring discipline, skill, patience, and sometimes complicated software.
As a result, most of these figures remain static upon the page.

% We built a system to animate them.
Inspired by the importance and appeal of the drawn human figure, we design and build a system to automatically animate it given an in-the-wild photograph of a child's drawing. 
Our system is fast, intuitive, and robust to much of the variation present in these types of drawings, making it well-suited to allow our target audience--children--to see their own characters coming to life.
The system is comprised of four stages: figure detection, segmentation masking, pose estimation/rigging, and animation. 
We describe each stage and identify common causes of failure in each. 
For object detection and pose estimation, we make use of existing computer vision models designed to detect human figures and joints in photographs; we fine-tune these models for use with children's drawings.
For segmentation, we present a straightforward, image processing-based method that, for animation purposes, is more useful and accurate than segmentation masks obtained from a fine-tuned object detection model.
During the animation step, we take advantage of the \textit{twisted perspective} commonly seen in children’s drawings to retarget motion capture data onto the character in a novel and appealing way.

% We use existing machine learning models. However, given the wide domain gap it's not clear how much fine-tuning data was needed. So we ran some experiments to find out and report it.
While our system leverages existing models and techniques, most are not directly applicable to the task due to the many differences between photographic images and simple pen and paper representations. 
To this end, we couple the presentation of our system with a set of experiments exploring the relationship between fine-tuning training set size and success rates.
We also include a perceptual study validating viewer preference for incorporating \textit{twisted perspective} into the motion retargeting step.

We validate the desirability and appeal of our system by building and publicly releasing a version of it as the \AD Demo \,\cite{animateddrawings}.
Launched in December 2021, this demo has been used by millions of people around the world to animate their children's drawings.
Inspired by this reception, our second contribution is The Amateur Drawings Dataset: \hjs{180,000 drawings and user-accepted annotations collected, with consent, through the demo. See Section \ref{sec:UI} for a description of how the annotations were generated.}
We believe this dataset will be a resource to researchers from various fields seeking to better understand the space of amateur drawings, evaluate new algorithms in this domain, or develop new drawing-based tools in general.

To summarize, our contributions are as follows:
\begin{enumerate}
    \item 
    We explore the problem of automatic sketch-to-animation for children's drawings of human figures and present a framework that achieves this effect. We also present a set of experiments determining the amount of training data necessary to achieve high levels of success and a perceptual study validating the usefulness of our motion retargeting technique.
    \item To encourage additional research in the domain of amateur drawings, we present a first-of-its-kind dataset of 180,000 user-submitted amateur drawings, along with user-accepted bounding box, segmentation mask, and joint location annotations.
\end{enumerate}

Upon acceptance of this paper, we plan to publicly release the Amateur Drawings Dataset, project code, and fine-tuned model weights.

Our work builds on existing methods from several fields but is, to our knowledge, the first work focused specifically on fully automatic animation of children's drawings of human figures. 
To ground the work, we provide a summary of salient observations from the field of children's art analysis.
In addition, we briefly review related work on 2D image-to-animation and object and pose estimation for abstract images. 


\subsection{Analysis of Children's Drawings}

\hjs{
Children's drawings have long been of interest to the scientific community.
For well over a century, researchers from multiple fields have 
collected\,\cite{IndianaS55:online,kellogg1967rhoda,AWebbasedDatabaseforDrawingsofGods,geist2002they}
and analyzed them, seeking to understand and measure children's thought processes\,\cite{sully2021studies,barnes1892study,clark1897child,buhler2013mental}, 
intellectual development\,\cite{goodenough1926measurement},
and perceptions\,\cite{chambers1983stereotypic,doi:10.1080/01443410500344167}.
}
Particular attention has been given to drawings of human figures, one of the first and most frequently drawn subjects throughout childhood\,\cite{cox2013children}.

As the child develops, the schemas they employ to represent the human form become more complete (see Figure \ref{fig:tadpole-transitional-conventional}).
Even within these schemas, there is significant variation.
In addition to asymmetries and variation in color and proportion, many body parts appear optional to include; a study of drawings by 4-6 year old children showed that, while heads, legs, and eyes are almost universally present, other body parts (including torsos, arms, hands, and feet) were frequently absent\,\cite{cox2013children}.
Inversely, non-human body parts are frequently added in order to represent other subject classes\,\cite{kellogg1969analyzing}. With the addition of large ears, the figure may represent a cat or bear (Figures \ref{fig:maskrcnn_before_after}.m and \ref{fig:maskrcnn_before_after}.g); with the addition of a crown, it can represent a pineapple (Figure \ref{fig:maskrcnn_before_after}.n).
All of these sources of character variation make automatic character animation from drawings a non-trivial task.

\begin{figure}
\includegraphics[width=\linewidth]{images/tadpole-transition-conventional.png}
\caption{
As children learn to draw the human figure, the morphologies of the schemas they employ vary and evolve considerably\,\cite{cox2014drawings}.
Children frequently begin by drawing a \textit{tadpole figure}, a circular head region from which arms and legs extend. 
Some will progress to a \textit{transitional figure}, dropping the arms down so they extend from the legs. 
When a line is drawn between the legs, creating the separate torso region, the \textit{conventional figure} is formed.
Though these are small changes from the perspective of the drawer, they result in significantly different character morphologies when viewed through the lens of character animation.
A successful drawing-to-animation system must be robust to these types of variations.}
\label{fig:tadpole-transitional-conventional}
\end{figure}

Many researchers have focused closely on the unique style of children's drawings.
The psychologist and artist John Willats argues that, in order to understand the style of children's drawings, one must understand that the primary picture primitives employed by children are \textit{regions}, or 2D areas\,\cite{willats2006making}.
A squat volume, such as a head or torso, may be represented by a circular or ellipsoid region, whereas an elongated volume, such as a leg, may be represented by a long, thin region or even a single line.
These regions are not depictions of the object from any particular point of view. 
Rather, they are \textit{3D volumetric object-centered descriptions}\,\cite{marr1982vision},
2D areas with attributes perceptually similar to those of 3D object they are meant to represent.
%The regions begin as circles and lines, but later become modified to better reflect the perceptually impactful aspects of the objects they represent; a region representing a sugar cube or die may be given square corners, and a long region representing an arm may be given a bend to depict the elbow or split at the end to represent fingers (CITE Willats, 2005).

There are two stylistic outcomes of these \textit{object-centered descriptions} that bear mention.
First, the use of foreshortening is very rare in children's drawings \,\cite{piaget1956, willats1992representation}. 
This design choice is understandable; foreshortening a long region, such as a limb, results in a short region which does not adequately reflect the \textit{longness} of the object.
Second, the human figure may appear to have been drawn from many different perspectives, so as to make each part of the character maximally recognizable.
For example, the head and torso may face forward while the legs and feet are pointed to the side.
This technique, often referred to as \textit{twisted perspective}, is frequently seen and well-documented\,\cite{dziurawiec1992twisted}.
Both of these stylistic aspects are used to guide the design decisions of our system when applying human motion capture data onto the character.


\subsection{2D Image to Animation}

Previous researchers have proposed methods to animate drawings or photographs, many of which rely upon additional modes of user input.
Hornung et al. present a method to animate a 2D character in a photograph, given user-annotated joint locations\,\cite{Hornung2007anim2Dpicmotion}.
Pan and Zhang demonstrate a method to animate 2D characters with user-annotated joint locations via a variable-length needle model\,\cite{Pan2011}.
Jain et al. present an integrated approach to generate 3D proxies for animation given joint locations, segmentation masks, and per-part bounding boxes\,\cite{jain:2012}. 
Levi and Gotsman provide a method to create an articulated 3D object from a set of annotated 2D images and an initial 3D skeletal pose\,\cite{ArtiSketch}.
\textit{Live Sketch}\,\cite{su2018livesketch}
tracks control points from a video and applies their motion to user-specified control points upon a character.
Other approaches allow the user to specify character motions through a puppeteer interface, using RGB or RGB-D cameras\,\cite{held20123d,barnes2008video}.
\textit{ToonCap}\,\cite{Fan:2018:TAL} focuses on an inverse problem, capturing poses of a known cartoon character, given a previous image of the character annotated with layers, joints, and handles. 


\textit{Toonsynth}\,\cite{Dvoroznak18-SIG} and \textit{Neural Puppet}\,\cite{poursaeed2020neural} both present methods to synthesize animations of hand-drawn characters given a small set of drawings of the character in specified poses.
Hinz et al. train a network to generate new animation frames of a single character given 8-15 training images with user-specified keypoint annotations\,\cite{hinz2022charactergan}.

\textit{Monster Mash}\,\cite{Dvoroznak20-SA} presents an intuitive framework for sketch-based modeling and animation, and \textit{2.5D Cartoon Models}\,\cite{10.1145/1778765.1778796} presents a novel method of constructing 3D-like characters from a small number of 2D representations. 
Both of these are intuitive and well designed animation tools targeted towards amateur users.


\hjs{
Some animation methods are specifically tailored toward particular forms, such as faces\,\cite{elor2017bringingPortraits}, coloring book characters\,\cite{magnenat2015live}, or characters with human-like proportions. 
One notable work that is focused on the human form is \textit{Photo Wake Up}\,\cite{weng2019photo}. 
The authors show a method for creating a rigged and textured 3D mesh from a single image of a human-like figure.
Similar to us, their end goal is to allow users to seamlessly bring 2D characters to life; their work does an impressive job of accomplishing this.
Our method differs in two significant ways. 
First, while their work is focused on creating a 3D model for a mixed reality use case, 
ours is specifically focused on animating twisted perspective figures while staying within a 2D plane.
Second, children's drawings are much more abstract, incorrectly proportioned, and non human-like than the examples demonstrated in the paper.
We test our method upon the more abstract examples demonstrated in their paper and, with minor segmentation cleaning, they were successfully animated by our method.
}












\hjs{While the approaches listed here are wonderful tools to ease the burden of animation, none were perfectly suited to our use case.
Some require additional user input beyond the drawing itself, making the animation process more complex.
Others require the user to consistently draw the same character in multiple poses, which is beyond the skills of young children.
Others are focused on animating specific forms, precluding their use on children's drawings of the human figure.}


%Siarohin and colleagues propose a method for animating arbitrary classes of subjects,
%but require training videos of class members moving\,\cite{Siarohin_2019_NeurIPS}, making it unsuitable children's drawings.


\subsection{Detection, Segmentation, and Pose Estimation on Non-Photorealistic Images}

\hjs{
Aided by the the existence of large annotated datasets\,\cite{lin2014microsoft,6909866,6682899}, researchers have made considerable progress solving the problems of object detection, segmentation, and pose estimation from photographs. See, for example\,\cite{MaskRCNNhe2017mask,openpose19,guler2018densepose,alphapose,toshev2014deeppose}.
We explain the methods in this area that we leverage in Sections \ref{sec:character_detection} and \ref{sec:joint_detection}.

While traditional methods for detection, segmentation, and pose estimation of non-photorealistic images exist\,\cite{choi2012retrieval,bregler2002turning,davis2006sketching,eitz2012humans}, the lack of easily available datasets has resulted in slower adoption of deep learning models.
Some researchers are addressing this problem by developing methods and releasing datasets focused on the domain of anime characters\,\cite{chen2022bizarre,10.1145/3011549.3011552}, professional sketches\,\cite{brodt2022sketch2pose}, and mouse doodles\,\cite{ha2017neural}.
Other researchers have presented a non-deep learning method for inferring character poses from \textit{gesture drawings}\,\cite{Gesture3D}.
}
Because the Amateur Drawings Dataset is comprised of in-the-wild photographs of drawings created by the general public, we believe it will complement the value of existing datasets and allow for new dimensions of exploration and analysis.
 
% Chapter Template

\chapter{Context Aware Network for 3D Glioma Segmentation} % Main chapter title
\label{Chapter3} % Change X to a consecutive number; for referencing this chapter elsewhere, use \ref{ChapterX}
\section{Introduction}
Glioma is one of the most common primary brain tumors with fateful health damage impacts and high mortality. To provide sufficient evidence for early diagnosis, surgery planning and post-surgery observation, Magnetic Resonance Imaging (MRI) is a widely used technique to provide reproducible and non-invasive measurement, including structural, anatomical and functional characteristics. Different 3D MRI modalities, such as T1, T1 with contrast-enhanced (T1ce), T2 and Fluid Attenuation Inversion Recover (FLAIR), can be used to examine different biological tissues.

\begin{figure}[t]
    \centering
    \includegraphics[width=0.6\textwidth]{TaskFig1.png}
    \caption{Examples of multi-modality data slices from BraTS17 with ground-truth and our segmentation result. In this figure, green represents GD-Enhancing Tumor, yellow represents Pertumoral Edema and red represents NCR$\backslash$ECT.}
\label{fig:task}
\end{figure}

Medical image segmentation provides fundamental guidance and quantitative assessment for medical professionals to achieve disease diagnosis, treatment planning and follow-up services. However, manual segmentation requires certain professional expertise and usually tends to be time and labor-consuming. Fig. \ref{fig:task} shows a general view of the brain tumor segmentation task. Early research on automated brain tumor segmentation was based on traditional machine learning algorithms \cite{bauer2012segmentation, subbanna2012probabilistic, shin2012hybrid, festa2013automatic}, which rely on hand-crafted features, such as textures \cite{reza2013multi} and local histograms \cite{goetz2014extremely}. However, finding the best hand-crafted features or optimal feature combinations in a high dimensional feature space is impracticable. In recent years, deep learning techniques, especially deep convolutional neural networks (DCNNs), can be used to effectively learn high dimensional discriminative features from data and have been widely used on various computer vision tasks \cite{long2015fully}.

Inter-class ambiguity is a common issue in brain tumor segmentation. This issue makes it hard to achieve accurate dense voxel-wise segmentation if only considering isolated voxels, as different classes' voxels may share similar intensity values or close feature representations. To address this issue, we propose a context-aware network, namely CANet, to achieve accurate dense voxel-wise brain tumor segmentation in MRI images. The proposed CANet contains a novel Hybrid Context Aware Feature Extractor (HCA-FE) and a novel Context Guided Attentive Conditional Random Field (CG-ACRF). Our contributions in this work are summarised below:

\begin{itemize}
    \item We propose a novel HCA-FE built with a 3D feature interaction graph neural network and a 3D encoder-decoder convolutional neural network. Different from previous works that usually extract features in the convolutional space, HCA-FE learns hybrid context guided features both in a convolutional space and a feature interaction graph space (the relationship between neighboring feature nodes is utilised and continuously updated). To our knowledge, this is the first practice on brain tumor segmentation, which incorporates adaptive contextual information with graph convolution updates.
    \item We further propose a novel CG-ACRF based fusion module that attentively aggregates features from the feature interaction graph and convolutional spaces. Moreover, we formulate the mean-field approximation of the inference in the proposed CG-ACRF as a convolution operation, enabling the CG-ACRF to be embedded within any deep neural network seamlessly to achieve end-to-end training.
    \item We conduct extensive evaluations and demonstrate that our proposed CANet outperforms several state-of-the-art technologies using different measure metrics on the Multimodal Brain Tumor Image Segmentation Challenge (BraTS) datasets, i.e. BraTS2017 and BraTS2018.
\end{itemize}

\begin{figure*}[t]
    \centering
    \includegraphics[width=\textwidth]{totalsystem.png}
    \caption{The architecture of the proposed dual stream network. Best viewed in color.}
    \label{fig:totalsystem}
\end{figure*}

\section{Review of Brain Glioma Segmentation methods}
Early research on brain tumor segmentation was based on traditional machine learning algorithms such as clustering \cite{shin2012hybrid}, random decision forests \cite{festa2013automatic}, Bayesian models \cite{corso2008efficient} and graph-cuts \cite{wels2008discriminative}. Shin \cite{shin2012hybrid} used sparse coding for generating edema features and K-means for clustering the tumor voxels. However, how to optimise the size of the sparse coding dictionary is still an intractable problem. Pereira et al. \cite{pereira2016brain} proposed to classify each voxel’s label by using random decision forests, which relied on hand-crafted features and complicated post-processing. Corso et al. \cite{corso2008efficient} used a Bayesian formulation for incorporating soft model assignments into the affinities calculation. This method brought the weighted aggregation of multi-scale features but ignored the relationship between different scales. Wels et al. \cite{wels2008discriminative} proposed a graph-cut based method to learn optimal graph representation for tumor segmentation, leading to superior performance. However, this method required a long inference time for dense segmentation tasks, as the number of vertices in its graph is proportional to the number of the voxels.

Promising achievements have been made on multi-modal MRI brain tumor segmentation using deep convolutional neural networks. Zikic et al. \cite{zikic2014segmentation} is one of the pioneers applying DCNNs onto brain tumor segmentation. Havaei et al \cite{havaei2017brain} further improved DCNNs with different sizes of convolutional kernels in order to capture local and global information. Zhao et al. \cite{zhao2018deep} proposed a modified FCN connected with conditional random fields for refining brain tumor segmentation using three MRI modalities. Dong et al. \cite{dong2017automatic} proposed a modified U-Net for brain tumor segmentation. These previous works used 2D convolutional kernels on 2D MRI slices made from original 3D volumetric MRI data. Methods using 2D slices do decrease the number of the used parameters and require less memory due to dimensionality reduction. However, this pre-processing procedure also leads to the spatial context missing. To minimise the information loss and capture evidence from adjacent slices, Lyksborg et al. \cite{lyksborg2015ensemble} ensembled three 2D CNNs on three orthogonal 2D patches. 

To fully make use of 3D contextual information, recent works applied 3D convolutional kernels on original volume data. Kamnitsas et al. \cite{kamnitsas2017efficient} proposed two pathway 3D CNN followed with dense CRF called DeepMedic for brain tumor segmentation. Authors of \cite{kamnitsas2017efficient} further extended the work by using model ensembling \cite{kamnitsas2017ensembles}. The proposed system EMMA ensembled models from FCN, U-Net and DeepMedic for processing 3D patches. To avoid over-fitting problems in 3D voxel-level segmentation on limited training datasets, Myronenko \cite{myronenko20183d} proposed a 3D CNN with an additional variational autoencoder to regularise the decoder by reconstructing the input image. The architecture built in \cite{myronenko20183d} is further developed in various recent works. Su et al. \cite{su2020multimodal} extends the architecture built in \cite{myronenko20183d} into two sub-networks to fuse the information learned from different modalities. Jiang et al. \cite{jiang2019two} proposed two-stage networks where each stage adopts a similar network in \cite{myronenko20183d}. The first stage network generates a coarse result and the second stage network refines the segmentation result. The final result in \cite{jiang2019two} reaches state-of-the-art by ensemble 12 model instances, which requires huge computational resources. Other works also try to fuse information brought by images in a different modality. Wang et al. \cite{wang2020modality} paired data from a different modality and designed the consistency loss to learn the relationship between features in different modalities. Dorent et al. \cite{dorent2019hetero} utilize the network in \cite{myronenko20183d} for multi-task learning, e.g. joint modality completion and  segmentation together. However, the aforementioned approaches only consider the relationship that lies within the modality and ignores the spatial relationship among features, which is more important to achieve accurate segmentation. 

Recent research works began to focus on using graph neural networks for object semantic segmentation.  Qi et al. \cite{qi20173d} and Landrieu et al. \cite{landrieu2018large} construct graph networks for point cloud semantic segmentation based on energy minimization. However, the data used in these approaches are point clouds. Point cloud contains the point nodes, which can be directly used for building graphs. Lu et al. \cite{lu2019graph} construct a graph-FCN for object semantic segmentation. However, this approach builds the graph by extracting nodes using convolutional kernel and ignores the information regulation between normal convolution and graph convolution. The same issue lies in \cite{liu2020scg}  and \cite{zhang2019dual} as the proposed model cannot adaptively make a preference between features from normal convolution and features from graph convolution.


Medical image datasets (e.g. BraTS) usually have an imbalance and inter-class interference problems. To address these issues whilst maintaining segmentation performance, Chen et al. \cite{chen2018focus} and Wang et al. \cite{wang2017automatic} both applied cascaded network structures for segmenting brain tumors, where the input of the inner region segmentation network is the output of the outer region segmentation network. However, these cascaded structures force the networks to crop data in the cascading stage and hence cause information loss. The summary of MRI based brain tumor segmentation is shown in Table \ref{table:lr}.

\begin{sidewaystable}
\centering
\caption{Summary of existing brain tumor segmentation methods.}
\begin{adjustbox}{width=1\textwidth}
\begin{tabular}{lllll}
\hline
Authors         & Base Model                                                                            & Data Format           & Highlights                                                                                                                                                                                                          & Limitations                                                                                                                                                                        \\ \hline
Wels et al. \cite{wels2008discriminative}     & Graph Cut                                                                             & 3D Volumes            & \begin{tabular}[c]{@{}l@{}}(1) A statistical formulation of \\ brain tumor segmentation.\end{tabular}                                                                                                               & \begin{tabular}[c]{@{}l@{}}(1) The size of graph vertices can be large.\\ (2) Using hand-crafted features.\end{tabular}                                                           \\ \hline
Corso et al. \cite{corso2008efficient}   & \begin{tabular}[c]{@{}l@{}}Bayesian Classifier + \\ the weighted Aggragation\end{tabular} & 2D Single-view slice  & \begin{tabular}[c]{@{}l@{}}(1) Explicitly learned the hierarchical \\ information of tumor tissue structures.\end{tabular}                                                                                            & \begin{tabular}[c]{@{}l@{}}(1) The final segmentation performace \\ heavily relys on the result of weighted \\ aggragation.\end{tabular}                                           \\ \hline
Shin \cite{shin2012hybrid}            & \begin{tabular}[c]{@{}l@{}}Spase coding + \\ K-means Clustering\end{tabular}          & 2D Single-view Slice  & (1) Fast and easy implementation.                                                                                                                                                                                   & \begin{tabular}[c]{@{}l@{}}(1) Clustering performance relys on \\ the quality of sparse coding features.\end{tabular}                                                                  \\ \hline
Festa et al. \cite{festa2013automatic}  & Random Decision Forest                                                                & 3D Volumes            & \begin{tabular}[c]{@{}l@{}}(1) Good interpretation based on\\  the classifier decisions.\end{tabular}                                                                                                               & (1) Hand-crafted features.                                                                                                                                                         \\ \hline
Zikic et al. \cite{zikic2014segmentation}   & 2D CNN                                                                                & 2D Patches            & (1) Computational efficient.                                                                                                                                                                                        & \begin{tabular}[c]{@{}l@{}}(1) Cannot directly learn information \\ from 3D space.\end{tabular}                                                                                    \\ \hline
Pereira et al. \cite{pereira2016brain}  & 2D CNN                                                                                & 2D Single-view slice  & \begin{tabular}[c]{@{}l@{}}(1) Stack small size kernels to \\ capture larger receptive field.\end{tabular}                                                                                                          & \begin{tabular}[c]{@{}l@{}}(1) Patch-classification based segmentation.\\ (2) Requires complicated post-processing.\end{tabular}                                                   \\ \hline
Havaei et al. \cite{havaei2017brain}   & 2D CNN                                                                                & 2D Patches            & \begin{tabular}[c]{@{}l@{}}(1) Replace final fully connected layer \\ with convolution layer, which leads\\  to a significant speed up.\\ (2) Studied the effectiveness \\ of different connections.\end{tabular}   & (1) Patch-classification based segmentation.                                                                                                                                       \\ \hline
Dong et al. \cite{dong2017automatic}     & 2D FCNN                                                                               & 2D Single-view Slice  & \begin{tabular}[c]{@{}l@{}}(1) Introduced a novel soft \\ dice loss function.\end{tabular}                                                                                                                          & \begin{tabular}[c]{@{}l@{}}(1) UNet based FCNN, which \\ used simple concatenation \\ for feature fusion.\end{tabular}                                                             \\ \hline
Kamnitsas et al. \cite{kamnitsas2017efficient} & 3D FCNN + CRF                                                                         & 3D Patches            & \begin{tabular}[c]{@{}l@{}}(1) 3D kernel to learn information \\ from volumetric space.\end{tabular}                                                                                                                & \begin{tabular}[c]{@{}l@{}}(1) Patch-classification based segmentation\\ (2) Cascaded connection between\\  FCNN and CRF.\end{tabular}                                             \\ \hline
Kamnitsas et al. \cite{kamnitsas2017ensembles} & 3D CNN Ensembling                                                                     & 3D Volumes            & \begin{tabular}[c]{@{}l@{}}(1) High accuracy benefited from \\ multiple segmentation models.\end{tabular}                                                                                                           & (1) Computation resource exhausted.                                                                                                                                                \\ \hline
Wang et al. \cite{wang2017automatic}     & 2D Cascaded CNN                                                                       & 2D Single-view Slices & \begin{tabular}[c]{@{}l@{}}(1) Explicitly model the hierachical \\ information of tumor tissue structures.\end{tabular}                                                                                              & \begin{tabular}[c]{@{}l@{}}(1) Information and receptive field\\  loss during crop operation.\\ (2) Over-parameterization by \\ introducing complicated sub-networks.\end{tabular} \\ \hline
Zhao et al. \cite{zhao20173d}     & 2D FCNN + CRF                                                                         & 2D Patches            & \begin{tabular}[c]{@{}l@{}}(1) Fully convolutional network to \\ generate segmentation map directly.\end{tabular}                                                                                                   & \begin{tabular}[c]{@{}l@{}}(1) Cascaded connection between \\ FCNN and CRF.\end{tabular}                                                                                           \\ \hline
Chen et al. \cite{chen2018focus}    & 2D Cascaded CNN                                                                       & 2D Single-view Slices & \begin{tabular}[c]{@{}l@{}}(1) Explicitly model the hierachical \\ information of tumor tissue structures.\end{tabular}                                                                                              & \begin{tabular}[c]{@{}l@{}}(1) Information and receptive field\\  loss during crop operation.\\ (2) Over-parameterization by\\  introducing complicated sub-networks.\end{tabular} \\ \hline
Myronenko \cite{myronenko20183d}      & 3D FCNN                                                                               & 3D Volumes            & \begin{tabular}[c]{@{}l@{}}(1) Additional autoencoder branch \\ for encoder backbone regularization.\end{tabular}                                                                                                   & \begin{tabular}[c]{@{}l@{}}(1) Cannot explicitly learn the hierachical \\ information of tumor tissue structures.\end{tabular}                                                      \\ \hline
Ours            & 3D FCNN                                                                               & 3D Volumes            & \begin{tabular}[c]{@{}l@{}}(1) Effectively modeling the hierachical \\ information of tumor tissue structures \\ by learning feature \\ interaction information.\\ (2) Built-in CRF for feature fusion.\end{tabular} & \begin{tabular}[c]{@{}l@{}}(1) No specific strategy to handle \\ from the data imbalance issue.\end{tabular}                                                                       \\ \hline
\end{tabular}
\end{adjustbox}
\label{table:lr}
\end{sidewaystable}

\section{Proposed Method}
\label{Proposed Method}
In this section, we describe our proposed CANet for dense voxel-wise segmentation of 3D MRI brain tumor images. We first describe the proposed HCA-FE with the feature interaction graph and convolutional space contexts in detail. Then we introduce the proposed novel fusion module, CG-ACRF, which deals with the features generated from two branches in HCA-FE and learns to output an optimal feature map. Finally, the formulation of mean-field approximation inference in CG-ACRF as convolutional operations is described, enabling the network to achieve end-to-end training. An illustration of the proposed segmentation framework is shown in Fig. \ref{fig:totalsystem}. Fig. \ref{fig:trainingflow} summarises the training steps of our CANet.

\begin{figure*}[!htb]
    \centering
    \includegraphics[width=\textwidth]{trainingflow.png}
    \caption{Training flow of the proposed CANet. Best viewed in colors.}
    \label{fig:trainingflow}
\end{figure*}

Different from previous works, our proposed HCA-FE can capture long-range contextual information in the feature space by learning the feature interaction, which has not been fully studied in the past. Both streams take the feature map $\mathcal{X} \in \mathbb{R}^{N \times C}$ derived from the shared encoder backbone as input, where $N = H\times W \times D$ is the total number of the voxels in a 3D MRI image. $H$, $W$, and $D$ represent the height, width and depth of the 3D MRI image respectively. $C$ is the number of the feature dimension. The graph stream generates representations in the feature interaction graph space $\mathcal{X}_{\mathcal{G}} \in \mathbb{R}^{N \times C}$ and the convolution stream generates a coordinate space representation $\mathcal{X}_{\mathcal{C}}\in \mathbb{R}^{N \times C}$.

The main concept behind the design of CG-ACRF is to estimate a segmentation map $\mathbf{T} \in \mathcal{T}$ associated with an MRI image $\mathbf{I} \in \mathcal{I}$ by exploiting the relationship between the final representation $\mathcal{X}_{\mathcal{F}} \in \mathbb{R}^{N \times C}$ and the intermediate feature representation $\mathcal{X}$ with auxiliary long-range contextual information $\mathcal{X}_{\mathcal{G}}$, generated from the interaction space with its convolution features $\mathcal{X}_{\mathcal{C}}$. Different from the simple concatenation $\mathcal{X}_{\mathcal{F}} = \textit{concat}(\mathcal{X}, \mathcal{X}_{\mathcal{G}}, \mathcal{X}_{\mathcal{C}})$ or element-wise summation $\mathcal{X}_{\mathcal{F}} = \mathcal{X} + \mathcal{X}_{\mathcal{G}} + \mathcal{X}_{\mathcal{C}}$, we aim to learn a set of latent feature representations $\mathcal{X}_{\mathcal{F}}^\mathcal{H} \in \mathbb{R}^{N \times C}$ through a new CRF. Due to the context information from $\mathcal{X}_{\mathcal{C}}$ and $\mathcal{X}_{\mathcal{G}}$ may contribute differently during the learning $\mathcal{X}_{\mathcal{F}}^\mathcal{H}$, we adopt the idea of an attention mechanism and generalise it into an gate node in CRF. The gate node can regulate the information flow and automatically discover the relevance between different contexts and latent features.
\subsection{Hybrid Context Aware Feature Extractor}
\label{HCG-FA}
\subsubsection{Graph Context Branch}
\textbf{Projection with Adaptive Sampling} We first use the collected feature map to create a feature interaction space by constructing an interaction graph $\mathcal{G} = \{\mathcal{V}, \mathcal{E}, A\}$, where $\mathcal{V}$ represents the set of nodes in the interaction graph, $\mathcal{E}$ represents the edges between the interaction nodes and $A$ represents the adjacency matrix. Given a learned high dimensional feature $X = \{x_{i}\}_{i=1}^{N} \in \mathbb{R}^{N \times C}$  with each $x_{i} \in \mathbb{R}^{1 \times C}$ from the back-bone network, we first project the original feature onto the feature interaction space, generating a projected feature $X_{proj} = \{x_{i}^{proj}\}_{i=1}^{N} \in \mathbb{R}^{K \times C'}$. $K$ is the number of the interaction nodes in the interaction graph and $C'$ is the interaction space dimension. A naive method for getting each element $x_{i}^{proj} \in X_{proj}, i=\{1,...,K\}$ is using the linear combination of its neighbor elements:
\begin{equation}
x_{i}^{proj} = \sum_{\forall j \in \mathcal{N}_{i}} w_{ij}x_{j} A[i,j]
\end{equation}
where $\mathcal{N}_{i}$ denotes the neighbors of pixel $i$. The naive approach normally employs a fully-connected graph with redundant connections and parameters between the interaction nodes, which is very difficult to optimise. More importantly, the linear combination method lacks an ability to perform adaptive sampling because different images contain different contextual information of brain tumors (e.g. location, size and shape). We deal with this issue by performing an adaptive sampling strategy:
\begin{equation}
    \begin{split}
        \triangle \textit{j} &= W_{i,j}x_{i} + b_{i,j}\\
        x_{i}^{proj} &= \sum_{\forall j \in \mathcal{N}_{i}} w_{ij}\rho (x_{j} | \mathcal{V}, j, \triangle j) A[i,j]
    \end{split}
\end{equation}

where $W_{i,j} \in \mathcal{R}^{3 \times (K \times C)}$ and $b_{i,j} \in \mathcal{R}^{3 \times 1}$ are the shift distances which are learned individually for each source feature $x_{i}$ through stochastic gradient decent. $\rho(\dot)$ is the trilinear interpolation sampler which can sample a shifted interaction node $x_{j}^{d}$ around feature node $x_{j}$, given the learned deformation $\triangle j$ and the total set of interaction graph nodes $\mathcal{V}$.

\textbf{Interaction Graph Reasoning} After having projected the input features into the interaction graph $\mathcal{G}$ with $K$ interaction nodes $\mathcal{V} = \{v_{1},...,v_{k}\}$ and edges $\mathcal{E}$, we follow the definition of the graph convolution network \cite{DBLP:conf/iclr/KipfW17}. In particular, we define $A_{\mathcal{G}}$ as the adjacency matrix on $K \times K$ nodes and $W_{\mathcal{G}} \in \mathbb{R}^{D \times D}$ as the weight matrix, and the formulation of the graph convolution operation is formulated as follows:
\begin{equation}
\begin{split}
    X_{\mathcal{GC}} &= \sigma(A X_{proj} W_{\mathcal{G}})\\
    &= \sigma((I - \hat{A}) X_{proj} W_{\mathcal{G}})
\end{split}
\end{equation}
where $\sigma()$ is sigmoid activation function. We first apply Laplacian smoothing and update the adjacency matrix to $(I - \hat{A})$ so as to propagate the node feature over the entire graph.
In practice, we implement $\hat{A}$ and $W_{\mathcal{G}}$ using a $1 \times 1$ convolution layer. We also achieve the implementation of $I$ as a residual connection which can maximise the gradient flow with a faster convergence speed.

\textbf{Re-Projection} Once the feature propagation has been finished, we re-project the features back to the original coordinate space with output $\mathcal{X}_{\mathcal{G}} \in \mathbb{R}^{N \times D}$. Similar to the projection step, we use trilinear interpolation here to calculate each elements $x_{\mathcal{G}}^{i} \in \mathcal{X}_{\mathcal{G}}, i \in \{1,...,N\}$ after having transformed the feature from the interaction space to the coordinate space. As a result, we have the feature $\mathcal{X}_{\mathcal{G}}$ with feature dimension $D$ at all $N$ grid coordinates.

\begin{figure*}[t]
    \centering
    \includegraphics[width=\textwidth]{differentCRF4.png}
    \caption{{{A graph model illustration of previous fusion schemes: (a) basic encoder-decoder neural network, (b) multi-scale neural network, (c) multi-scale CRF, and  our proposed (d) context guided attentive fusion CRF. $I$ denotes the input 3D MRI image. $f_s$ denotes the feature map at scale $s$. $a_s$ indicates the attention map generated from the corresponding feature $f$ at scale $s$. $h_c$ and $h_g$ represent the hidden feature generated from convolutional features and graph convolutional features respectively. $L$ means the final segmentation labeling output. Best viewed in color.}}}
    \label{fig:crfcomp}
\end{figure*}

\subsubsection{Convolution Context Branch}
The convolution context branch is composed of a contracting path (encoder) and an expansive path (decoder) with skip connections between these two paths. The contracting path reduces the spatial dimensionality of the pooling layer in a pyramidal scale whilst the expansive path recovers the spatial dimensionality and the details of the object with the corresponding pyramid scale. One of the advantages of using this architecture is that it fully utilises the features with different scales of contextual information, where large scale features can be used to localise objects and small scale but high dimensionality features can provide more detailed and accurate information for classification.

However, 3D volumetric images require more parameters to learn during feature extraction. It is often observed that training such 3D model often fails for various reasons such as over-fitting and gradient vanishing or exploding. Besides, simple or complicated augmentation technologies used to extend the training dataset may result in a slow convergence speed. To address the issues mentioned above, we develop a deep supervised mechanism that inherits the advantages of the convolution context branch. The proposed deep supervision mechanism thus reinforces the gradient flow and improves the discriminative capability during the training procedure.

Specifically, we use additional upsampling layers to reshape the features created at the deep supervised layer to be of the resolution of the final input. For each transformed layer, we apply the softmax function to obtain additional dense segmentation maps. For these additional segmentation results, we calculate the segmentation errors with regards to the ground-truth segmentation maps. The auxiliary losses are integrated with the loss from the output layer of the whole network and we further back-propagate the gradient for parameter updating during each iteration in the training stage.

We denote the set of the parameters in the deep supervised layers as $W_{S} = \{w^{i}\}_{i=1}^{S}$ and $w^{s}$ as the parameters of the upsampling layer correspond to layer $s$. The auxiliary loss for a deep supervision layer $s$ is formulated using cross-entropy:
\begin{equation}
    \mathcal{L}_{s}(\mathcal{X};W_{S}) = \sum_{i=1}^{S}\sum_{j=1}^{N}-\log  \mathbbm{1}(p(y_{j}|x_{j}^{s};w^{s}))
\end{equation}
where $\mathbbm{1}$ is the indicator function which is 1 if the segmentation result is correct, otherwise 0.  $\mathcal{Y} = \{y_{i}\}_{i=1}^{N}$ is the ground-truth of voxel $i$ and $\mathcal{X}_{S} = \{x^{s}_{i}\}_{i=1,s=1}^{N,S}$ is the predicted segmentation label of voxel $i$ generated from the upsampling layer $s$. Finally, the deep supervision loss $\mathcal{L}_{s}$ can be integrated with the loss $\mathcal{L}_{T}$ from the final output layer. The parameters of the deep supervised layers $W_{S}$ can be updated with the rest parameters $W$ from the whole framework simultaneously using back-propagation:
\begin{equation}
\begin{split}
      \mathcal{L} = \mathcal{L}_{T}(\mathcal{Y}| & \mathcal{X};W, W_{S}) + \sum_{s=1}^{S}\delta_{s}\mathcal{L}_{s}(\mathcal{X};w^{s})\\
      &+ \lambda(||W||^{2} + \sum_{s=1}^{S}(||w^{s}||^{2}))
\end{split}
\label{cnnstreamtogetherloss}
\end{equation}
where $\delta_{s}$ represents the weight factor for the supervision loss of each upsampling layer. As the training procedure continues to approach to the optimal parameter sets, $\delta_{s}$ reduces gradually. The final operation of Eq. (\ref{cnnstreamtogetherloss}) is the $L$2-regularisation of the total trainable weights with the weight factor $\lambda$.

%\subsection{Discrete Adaboost}
%myclassification algorithm is built upon the discrete Adaboost algorithm proposed by Freud et al. \cite{freund1996experiments}. 

\subsection{Context Guided Attentive Conditional Random Field}
We further propose a novel context guided attentive CRF module to perform feature fusion, motivated from two perspectives. A graph model of our proposed CG-ACRF is illustrated in Fig. \ref{fig:crfcomp}. There are two reasons to use CG-ACRF for feature fusion. Firstly, assigning segmentation labels by maximising probabilities may result in blurry boundaries due to the neighboring voxels sharing similar spatial contexts. Secondly, previous works fuse information from different sources (e.g. multi-scale or multi-stage) by using simple channel-wise concatenation or element-wise summation mechanism. However, these mechanisms do not take into account the heterogeneity between different feature maps (e.g. shallow layers tend to focus on low-level visual features while deep layers tend to attend abstract features). Simplifying the relationship between different source feature maps (e.g. feature maps of large kernels tend to represent the object outline while feature maps of small kernels tend to encode the details of the object structure) results in information loss. Different from previous related works and using the inference ability of the probabilistic graphical model, we employ the conditional random field model to learn optimised latent fusion features for final segmentation. As information from different contexts may contribute to the final results at different degrees, we integrate the attention gates of the CRF to regulate how much information should flow between features generated from different contexts. We further show the convolution formulation of CG-ACRF mean-field approximation inference, which allows our attentive CRF fusion module to be integrated into neural networks as a layer and trained in an end-to-end fashion. Compared with previous architectures such as an encoder-decoder neural network (Fig. \ref{fig:crfcomp} (a)) and multi-scale neural network (Fig. \ref{fig:crfcomp} (b)), our proposed CG-ACRF (Fig. \ref{fig:crfcomp} (d)) has a strong inference ability and can jointly learn the hidden representation of features encoded by the neural network backbone, improving the generalisation ability of the segmentation model. Compared with previous architectures such as multi-scale CRF (Fig. \ref{fig:crfcomp} (c)), our proposed CG-ACRF model first uses an attention gate by directly modeling the cost energy in the network (Eq. (\ref{EnergyDefinition})). The attention gate thus regulates the information flow from the features encoded by the backbone neural network to the latent representations by minimising the total energy cost. Moreover, our proposed CG-ACRF learns to project the features into two spaces, i.e. a convolutional space and a feature interaction graph. Hidden representations from different spaces can further boost feature fusion performance. We evaluate the effectiveness of each component in the experiment section.

\subsubsection{Definition}
Given the feature map $\mathcal{X}_{\mathcal{C}} = \{x_{\mathcal{C}}^{i}\}_{i=1}^{N}$ from the convolution context branch and the feature map $\mathcal{X}_{\mathcal{G}} = \{x_{\mathcal{G}}^{i}\}_{i=1}^{N}$ from the interaction graph branch, our goal is to estimate the fusion representation $H_{\mathcal{G}} = \{h_{\mathcal{G}}^{i}\}_{i=1}^{N}, H_{\mathcal{C}} = \{h_{\mathcal{C}}^{i}\}_{i=1}^{N}$ and the attention variable $A = \{a_{\mathcal{G}\mathcal{C}}^{i}\}_{i=1}^{N}$. We formalise the problem by designing a Context Guided Attentive Conditional Random Field with a Gibbs distribution as follows:
\begin{equation}
    P(H,A|I,\Theta) = \frac{1}{Z(I, \Theta)}exp\{-E(H,A,I,\Theta)\}
\end{equation}
where $E(H,A,I,\Theta)$ is the associated energy, and
\begin{equation}
     E(H,A,I,\Theta) = \Phi_{\mathcal{G}}(H_{\mathcal{G}}, X_{\mathcal{G}}) + \Phi_{\mathcal{C}}(H_{\mathcal{C}}, X_{\mathcal{C}}) + \Psi_{\mathcal{G}\mathcal{C}}(H_{\mathcal{G}}, H_{\mathcal{C}}, A)
\label{EnergyDefinition}
\end{equation}
where $I$ is the input 3D MRI image and $\Theta$ is the set of parameters. In Eq.(\ref{EnergyDefinition}), $\Phi_{\mathcal{G}}$ is the unary potential between the latent graph representation $h_{\mathcal{G}}^{i}$ and the graph features $x_{\mathcal{G}}^i$. $\Phi_{\mathcal{C}}$ is the unary potential related to latent convolution representation $h_{\mathcal{C}}^i$ and convolution feature $x_{\mathcal{C}}^i$. In order to enable the estimated latent representation $h^i$ to be close to the observation $x^i$, we use the Gaussian function created in previous works \cite{krahenbuhl2011efficient}:

\begin{equation}
    \Phi(H, X) = \sum_{i}\phi(h^i, x^i) = -\sum_{i=1}^{N}\frac{1}{2}||h^{i} - x^{i}||^{2}.
\end{equation}

The final term shown in Eq. (\ref{EnergyDefinition}) is the attention guided pairwise potential between the latent convolution representation $h_{\mathcal{C}}^i$ and the latent graph representation $h_{\mathcal{G}}^i$. The attention term $a_{\mathcal{GC}}^{i}$ controls the information flow between the two latent representations where the graph representation may or may not contribute to the estimated convolution representation. We define:
\begin{equation}
\begin{split}
    \Psi_{\mathcal{G}\mathcal{C}}(H_{\mathcal{G}}, H_{\mathcal{C}}, & A) = \sum_{i=1}^{N}\sum_{j \in \mathcal{N}_{i}}\psi(a_{\mathcal{GC}}^{j}, h_{\mathcal{C}}^{i}, h_{\mathcal{G}}^{j})\\
    & = \sum_{i=1}^{N}\sum_{j \in \mathcal{N}_{i}}a_{\mathcal{GC}}^{j} h_{\mathcal{C}}^{i} \Upsilon_{\mathcal{GC}}^{i,j} h_{\mathcal{G}}^{j}
\end{split}
\end{equation}
where $\Upsilon_{\mathcal{GC}}^{i,j} \in \mathbb{R}^{D_{\mathcal{G}} \times D_{\mathcal{C}}}$ and $D_{\mathcal{G}}, D_{\mathcal{C}}$ represent the dimensionality of the features $X_{\mathcal{G}}$ and $X_{\mathcal{C}}$ respectively.

\subsubsection{Inference}
\label{MFCRF}
By learning latent feature representations to minimise the total segmentation energy, the system can produce an appropriate segmentation map, \textit{e.g.} the maximum a posterior $P(H,A|I,\Theta)$. However, the optimisation of $P(H, A|I,\Theta)$ is intractable due to the computational complexity in normalising constant $Z(I, \Theta)$, which is exponentially proportional to the cardinality of $h \in H$ and $a \in A_{\mathcal{GC}}$. Therefore, in order to derive the maximum a posterior in an efficient way, we adopt mean-field approximation to approximate a complex posterior probability distribution. We have:
\begin{equation}
    P(H,A | I,\Theta) \approx Q(H,A) = \prod_{i=1}^{N} q_{i}(h_{\mathcal{G}}^i)q_{i}(h_{\mathcal{C}}^i)q_{i}(a_{\mathcal{G}\mathcal{C}}^i)
\label{mfdef}
\end{equation}

Here we use the product of independent marginal distributions $q(h_{\mathcal{G}}^i)$, $q(h_{\mathcal{C}}^i)$ and $q(a_{\mathcal{G}\mathcal{C}}^i)$ to approximate the complex distribution $P(H,A,I,\Theta)$. To achieve a satisfactory approximation result, we minimise the Kullback-Leibler (KL) divergence $D_{KL}(Q||P)$ between the two distributions $Q$ and $P$. By replacing the definition of the energy $E(H,A,I,\Theta)$, we formulate the KL divergence in Eq. (\ref{mfdef}) as follows:
\begin{equation}
\begin{split}
    D_{KL}(Q||P) &= \sum_{h}Q(h)\ln (\frac{Q(h)}{P(h)})\\
    &=\sum_{h}Q(h)E(h) + \sum_{h}Q(h)\ln Q(h) + \ln Z
\end{split}
\label{kl1}
\end{equation}
From Eq.(\ref{kl1}), we can minimise the KL divergence by directly minimising the free energy $FE(Q) = \sum_{h}Q(h)E(h) + \sum_{h}Q(h)\ln (Q(h))$. In $FE(Q)$, the first item represents the cross-entropy between two distributions $Q$ and $E$ and the second item represents the entropy of distribution $Q$. We can further expand the expression of $FE(Q)$ by replacing $Q$ and $E$ with Eqs. (\ref{mfdef}) and (\ref{EnergyDefinition}) respectively:

\begin{equation}
\begin{split}
    FE(Q) &= \sum_{i=1}^{N}q_{i}(h_{\mathcal{G}}^i)q_{i}(h_{\mathcal{C}}^i)q_{i}(a_{\mathcal{G}\mathcal{C}}^i)(\Phi_{\mathcal{G}}+\Phi_{\mathcal{C}}+\Psi_{\mathcal{GC}})\\
    &+ \sum_{i=1}^{N}q_{i}(h_{\mathcal{G}}^i)q_{i}(h_{\mathcal{C}}^i)q_{i}(a_{\mathcal{G}\mathcal{C}}^i)(\ln (q_{i}(h_{\mathcal{G}}^i)q_{i}(h_{\mathcal{C}}^i)q_{i}(a_{\mathcal{G}\mathcal{C}}^i)))
\end{split}
\label{FEDef}
\end{equation}

Eq. (\ref{FEDef}) shows that the problem of minimising $FE(Q)$ can be transferred to a constrained optimisation problem with multiple variables, which can be formally formulated below:

\begin{equation}
\begin{split}
    & \min_{q_{i}(h_{\mathcal{G}}^i), q_{i}(h_{\mathcal{C}}^i), q_{i}(a_{\mathcal{G}\mathcal{C}}^i)} FE(Q), \forall i  \in N\\
    & \textrm{s.t.} \sum_{l=1}^{L}q_{i}(h_{\mathcal{G}}^i) = 1, \sum_{l=1}^{L}q_{i}(h_{\mathcal{C}}^i) = 1, \int_{0}^{1} q_{i}(a_{\mathcal{GC}}^i)da^{i}_{\mathcal{GC}} = 1
\end{split}
\label{optDef}
\end{equation}
where $l$ represents the index of the segmentation label. We can calculate the first order partial derivative by differentiating $FE(Q)$ w.r.t each variable. For example, we have:

\begin{equation}
\begin{split}
    \frac{\partial FE}{\partial q_{i}(h^{i}_{\mathcal{C}})} = \phi_{\mathcal{C}}(h_{\mathcal{C}}^{i}, x_{\mathcal{C}}^{i}) +& \sum_{j\in \mathcal{N}_{i}}\mathbb{E}_{q_{j}(a_{\mathcal{GC}}^{j})} \{ a_{\mathcal{GC}}^{j} \} \mathbb{E}_{q_{j}(h_{\mathcal{G}}^j)} \psi (h_{\mathcal{C}}^i, h_{\mathcal{G}}^j)\\
    &- \ln q_{i}(h^{i}_{\mathcal{C}}) + \text{const}
\end{split}
\label{partialDer}
\end{equation}

By assigning 0 to the left of Eq. (\ref{partialDer}), we reach:
\begin{equation}
\begin{split}
     q_{i}(h_{\mathcal{C}}^{i}) \propto & \exp \Big\{ \phi_{\mathcal{C}}(h_{\mathcal{C}}^{i}, x_{\mathcal{C}}^{i}) + \\
     & \sum_{j \in N_{i}} \mathbb{E}_{q_{j}(a_{\mathcal{GC}}^{j})} \{ a_{\mathcal{GC}}^{j} \} \mathbb{E}_{q_{j}(h_{\mathcal{G}}^j)} \psi (h_{\mathcal{C}}^i, h_{\mathcal{G}}^j) \Big \}
\end{split}
\label{qhcdef}
\end{equation}

Eq. (\ref{qhcdef}) shows that, once the other two independent variables $q(h_{\mathcal{G}})$ and  $q(a_{\mathcal{GC}})$ are fixed, how $ q(h_{\mathcal{C}})$ is updated during the mean-field approximation inference. Further more, we follow the above procedure and obtain the updating of the remaining two variable as follows:
\begin{equation}
\begin{split}
     q_{i}(h_{\mathcal{G}}^{i}) \propto & \exp \Big\{ \phi_{\mathcal{G}}(h_{\mathcal{G}}^{i}, x_{\mathcal{C}}^{i}) + \\
     & \mathbb{E}_{q_{j}(a_{\mathcal{GC}}^{j})} \{ a_{\mathcal{GC}}^{j} \} \sum_{j \in N_{i}} \mathbb{E}_{q_{j}(h_{\mathcal{C}}^j)} \psi (h_{\mathcal{C}}^i, h_{\mathcal{G}}^j) \Big \}
\end{split}
\label{qhgdef}
\end{equation}
\begin{equation}
    q_{i}(a_{\mathcal{GC}}^{i}) \propto \exp \Big\{a_{\mathcal{GC}}^{i} \mathbb{E}_{q_{i}(h_{\mathcal{C}}^{i})} \{ \sum_{j \in N_{i}} \mathbb{E}_{q_{j}(h_{\mathcal{G}}^{j})} \{ \psi(h_{\mathcal{C}}^{i}, h_{\mathcal{G}}^{j})\} \} \Big \}
\label{qatdef}
\end{equation}
where $\mathbb{E}_{q()}$ represents the expectation with respect to the distribution $q()$. Eqs. (\ref{qhcdef}-\ref{qatdef}) shown above denote the computational procedure of seeking an optimal posterior distributions of $h_{\mathcal{C}}$, $h_{\mathcal{G}}$ and $a_{\mathcal{GC}}$ during the mean-field approximation. Intuitively, Eq. (\ref{qhcdef}) shows that, the latent convolution feature $h_{\mathcal{C}}^{i}$ for voxel $i$ can be  used to describe the observation, referred to feature $x_{\mathcal{C}}^{i}$. Afterwards, we use the re-weighted messages from the latent features of the neighboring voxels to learn the co-occurrent relationship of the pixels. The attention weight between the latent convolution and the graph features for voxel $i$ allows us to re-weight the pairwise potential message from the neighbours of voxel $i$, and then use the attention variable to re-weight the total value of voxel $i$. By denoting $\Bar{a}_{\mathcal{G}\mathcal{C}}^{i} = \mathbb{E}_{q(a_{\mathcal{G}\mathcal{C}}^{i})}\{a_{\mathcal{G}\mathcal{C}}^{i}\}$ and $\Bar{h}^{i} = \mathbb{E}_{q(h^{i})}\{h^{i}\}$, we have the feature update as follows:
\begin{equation}
    \Bar{h}_{\mathcal{G}}^{i} = x_{\mathcal{G}}^{i} + \Bar{a}_{\mathcal{G}\mathcal{C}}^{i} \sum_{j \in N_{i}} \Upsilon_{\mathcal{G}\mathcal{C}}^{i,j} \Bar{h}_{\mathcal{C}}^{j}
\end{equation}

\begin{equation}
    \Bar{h}_{\mathcal{C}}^{i} = x_{\mathcal{C}}^{i} + \sum_{j \in N_{i}} \Bar{a}_{\mathcal{G}\mathcal{C}}^{j}  \Upsilon_{\mathcal{G}\mathcal{C}}^{i,j} \Bar{h}_{\mathcal{G}}^{j}
\end{equation}

$\Bar{a}_{\mathcal{G}\mathcal{C}}^{i}$ is also derived from the probabilistic distribution, \textit{i.e.} its value lies in $[0,1]$. Here, we choose the Sigmoid function to formulate the updates for $\Bar{a}_{\mathcal{G}\mathcal{C}}^{i}$:
\begin{equation}
    \Bar{a}_{\mathcal{G}\mathcal{C}}^{i} = \sigma(- \sum_{j \in N_{i}}a_{\mathcal{GC}}^{j} h_{\mathcal{C}}^{i} \Upsilon_{\mathcal{GC}}^{i,j} h_{\mathcal{G}}^{j})
\end{equation}
where $\sigma(.)$ denotes the Sigmoid activation function.

\subsubsection{Mean Field Inference as Convolution Operation}
To achieve joint training and end-to-end optimisation of the proposed CRF with the backbone network, we implement the mean-field approximation of the proposed CRF in neural networks. We aim to perform the updating of the latent feature and attention maps according to the derivation described in Section \ref{MFCRF}. The algorithm for implementing mean-field approximation using convolutional operations is described in Algorithm \ref{Alg:convmf}. A graph illustration of Algorithm \ref{Alg:convmf} is shown in Fig. \ref{fig:crfmf}.

\begin{figure}[t]
    \centering
    \includegraphics[width=0.7\textwidth]{CGACRFMFIteration.png}
    \caption{Details of the mean-field updates within CG-ACRF. The circled symbols indicate message-passing operations within the CG-ACRF block. Best viewed in colors.}
    \label{fig:crfmf}
\end{figure}


\begin{algorithm}%[H]
\caption{Algorithm for Mean-Field Approximation.}
\begin{algorithmic}[1]
\renewcommand{\algorithmicrequire}{\textbf{Input:}}
\renewcommand{\algorithmicensure}{\textbf{Output:}}
\REQUIRE Feature interaction graph output $x_{\mathcal{G}}$ and convolution output $x_{\mathcal{C}}$. Initialize hidden graph feature map $h_{\mathcal{G}}$ with $x_{\mathcal{G}}$. Initialize hidden convolutional feature map $h_{\mathcal{C}}$ with $x_{\mathcal{C}}$
\ENSURE Estimated optimised hidden convolution feature map $h$.
\WHILE {in iteration number}
 \STATE $\hat{a}_{\mathcal{G}\mathcal{C}} \leftarrow h_{\mathcal{C}} \odot (\Upsilon_{\mathcal{G}\mathcal{C}} \ast h_{\mathcal{G}})$;
 \STATE $\Bar{a}_{\mathcal{G}\mathcal{C}} \leftarrow \sigma(-(\hat{a}_{\mathcal{G}\mathcal{C}}))$;
 \STATE $h_{\mathcal{G}} \leftarrow \Upsilon_{\mathcal{G}\mathcal{C}} \ast h_{\mathcal{G}}$;
 \STATE $\Bar{h}_{\mathcal{C}} \leftarrow \Bar{a}_{\mathcal{G}\mathcal{C}} \odot h_{\mathcal{G}}$;
 \STATE $h \leftarrow x \oplus \Bar{h}_{\mathcal{C}}$;
\ENDWHILE
\RETURN Optimised hidden feature map $h$.
\end{algorithmic}
\label{Alg:convmf}
\end{algorithm}

where $\ast, \odot, \oplus$ represent the convolution, element-wise dot product, and element-wise summation respectively. First, the latent feature map is initialised with corresponding observation inputs $x_{\mathcal{G}}$ and $x_{\mathcal{C}}$, while the attention map is initialised from the message passing on the two latent feature maps. Then, we activate and normalise the attention map. The latent convolutional feature map is updated from the message passing on the latent graph feature map. Finally, the updated attention map is used to refine the latent convolutional feature map $h$. We output $h$ with the unary term $x$ by establishing residual connections.


\section{Experiment Setup}
\label{Experimental Setup}
To demonstrate the effectiveness of the proposed CANet for brain tumor segmentation, we conduct experiments on two publicly available datasets: the Multimodal Brain Tumor Segmentation Challenge 2017 (BraTS2017) and the Multimodal Brain Tumor Segmentation Challenge 2018 (BraTS2018). 

\subsection{Datasets and Evaluation Metrics}
\textbf{Datasets.} The \textbf{BraTS2017}\footnote{https://www.med.upenn.edu/sbia/brats2017.html} consists of 285 cases of patients in the training set and 44 cases in the validation set. \textbf{BraTS2018}\footnote{https://www.med.upenn.edu/sbia/brats2018.html} shares the same training set with BraTS2017 and includes 66 cases in the validation set. Each case is composed of four MR sequences, namely native T1-weighted (T1), post-contrast T1-weighted (T1ce), T2-weighted (T2) and Fluid Attenuated Inversion Recovery (FLAIR). Each sequence has a 3D MRI volume of 240$\times$240$\times$155. Ground-truth annotation is only provided in the training set, which contains the background and healthy tissues (label 0), necrotic and non-enhancing tumor (label 1), peritumoral edema (label 2) and GD-enhancing tumor (label 4). We first consider the 5-fold cross-validation on the training set where each fold contains (random division) 228 cases for training and 57 cases for validation. We then evaluate the performance of the proposed method on the validation set. The validation result is generated from the official server of the contest to determine the segmentation accuracy of the proposed methods.	

\textbf{Evaluation Metrics.} Following previous works \cite{wang2017automatic}, \cite{kamnitsas2017efficient}, \cite{bakas2017advancing}, the segmentation accuracy is measured by Dice score, Sensitivity, Specificity and Hausdorff95 distance respectively. In particular,
\begin{itemize}
    \item Dice score: $Dice(P,T) = \frac{|P_1 \cap T_1|}{(|P_1|+|T_1|)/2}$
    \item Sensitivity: $Sens(P,T) = \frac{|P_1 \cap T_1|}{|T_1|}$
    \item Specificity: $Spec(P,T) = \frac{|P_0 \cap T_0|}{|T_0|}$
    \item Hausdorff Distance: $Haus(P,T) = \\
    max\{sup_{p\in P_1}inf_{t\in T_1} d(p,t), sup_{t\in T_1}inf_{p\in P_1} d(t,p)\}$
\end{itemize}
where $P$ represents the model prediction and $T$ represents the ground-truth annotation. $T_1$ and $T_0$ are the subset voxels predicted as positives and negatives for the tumor region. Similar set-ups are made for $P_1$ and $P_0$. Furthermore, the Hausdorff95 distance measures how far the model prediction deviates from the ground-truth annotation. $sup$ represents the supremum and $inf$ represents the infimum. For each metric, three regions namely enhancing tumor (ET, label 1), whole tumor (WT, labels 1, 2 and 4) and the tumor core (TC, labels 1 and 4) are evaluated individually.
\subsection{Data Augmentation and Implementation Details}
\textbf{Data Augmentation} For each sequence in each case, we set all the voxels outside the brain to zero and normalise the intensity of the non-background voxels to be of zero mean and unit variance. During the training, we use randomly cropped images of size 128$\times$128$\times$128.
We further set up a common augmentation strategy for each sequence in each case: (i) randomly rotate an image with the angle between [-20$^{\circ}$, +20$^{\circ}$]; (ii) randomly scale an image with a factor of 1.1; (iii) randomly mirror flip an image across the axial coronal and sagittal planes with the probability of 0.5; (iv) random intensity shift between [-0.1, +0.1]; (v) random elastic deformation with $\sigma = 10$.

\textbf{Implementation Details} We implement the proposed CANet and other benchmark experiments using the PyTorch framework and deploy all the experiments on 2 parallel Nvidia Tesla P100 GPUs for 200 epochs with a batch size of 4. We use the Adam optimizer with an initial learning rate $\alpha_0 = 1\mathrm{e}{-4}$. The learning rate is decreased by a factor of 5 after 100, 125 and 150 epochs. We use a $L$2 regulariser with a weight decay of $1\mathrm{e}{-5}$. We store the weights for each epoch and use the weights that lead to the best dice score for inference.

\section{Experimental Result} 
\label{resultdiscussion}
In this section, we present both quantitative and qualitative experimental results of different evaluations. We first conduct an ablation study of our method to show the effective impact of HCA-FE and CG-ACRF on the segmentation performance. We also perform additional analysis of the encoder backbone and different iteration numbers of approximation for CG-ACRF. Afterward, we compare our approach with several state-of-the-art methods on different datasets. Finally, we present the analysis of failure cases.

\subsection{Ablation Studies}
We first evaluate the effect of HCA-FE and CG-ACRF. To this end, we apply a 5-fold cross-evaluation on the BraTS2017 training set and report the mean result. Table \ref{table:ablationstudy} shows the quantitative results, while the qualitative results can be found in Fig. \ref{fig:ablation} as an example of the segmentation outputs. We start from two baselines. The first baseline is in the fully convolution format with deep supervision on the backbone convolution encoder (CC). The second baseline only uses graph convolution in the convolution encoder without deep supervision (GC). We then evaluate the proposed whole HCA-FE (CC+GC) without any feature fusion method, \textit{i.e.} concatenating feature maps from CC and GC together. Finally, we evaluate the proposed feature fusion module CG-ACRF, which takes a feature map with different contexts from HCA-FE and outputs the optimal latent feature map for the final segmentation. For the experiments are shown in Table \ref{table:ablationstudy} and Fig. \ref{fig:ablation}, we use the encoder of UNet as the backbone network with 5 iterations in CG-ACRF. The experiments described later include the analysis of different backbones and iteration numbers.

\begin{table*}[!ht]
\centering
\caption{Quantitative results of the CANet components by five fold cross-validation for the BraTS2017 training set (dice, sensitivity and specificity). All the methods are based on CANet with UNet as the backbone. The best result is shown in bold text and the runner-up result is underlined.}
\begin{adjustbox}{width=1\textwidth}
\begin{tabular}{|c|c|c|c|c|c|c|c|c|c|c|c|c|}
\hline
              & \multicolumn{3}{c|}{DICE}                              & \multicolumn{3}{c|}{Sensitivity}                       & \multicolumn{3}{c|}{Specificity}                       & \multicolumn{3}{c|}{Hausdorff95}                       \\ \hline
Backbone+         & ET               & WT               & TC               & ET               & WT               & TC               & ET               & WT               & TC               & ET               & WT               & TC               \\ \hline
CC            & \textbf{0.68628} & 0.87467          & 0.82068          & 0.85684          & {\underline{0.92495}}    & 0.86324          & {\underline{0.99708}}    & {\underline{0.99094}}    & 0.99562          & \textbf{6.79149} & 6.88633          & 7.93923          \\ \hline
GC            & 0.6373           & {\underline{0.89365}}    & {\underline{0.82246}}    & \textbf{0.97704} & \textbf{0.96964} & \textbf{0.94428} & 0.98723          & 0.98742          & \textbf{0.99665} & 9.89949          & {\underline{6.40312}}    & {\underline{5.81216}}    \\ \hline
CC+GC+Concatenation         & 0.68194          & 0.86073          & 0.80306          & {\underline{0.85725}}    & 0.92243          & 0.86085          & 0.99672          & 0.98913          & 0.99351          & {\underline{7.75539}}    & 9.37745          & 11.43241         \\ \hline
CC+GC+CG-ACRF & {\underline{0.68489}}    & \textbf{0.90338} & \textbf{0.87291} & 0.80651          & 0.92363          & {\underline{0.86989}}    & \textbf{0.99746} & \textbf{0.99307} & {\underline{0.99592}}    & 7.80448          & \textbf{3.56898} & \textbf{4.03629} \\ \hline
\end{tabular}
\end{adjustbox}
\label{table:ablationstudy}
\end{table*}

\begin{figure*}[!ht]
    \centering
    \includegraphics[width=\textwidth]{ablation.png}
    \caption{Qualitative comparison of different baseline models and the proposed CANet by cross-validation on BraTS2017 training set. From left to right, each column represents the input FLAIR data, ground truth annotation, segmentation result of CANet with only the convolution branch, segmentation result of CANet with only the graph convolution branch, segmentation output of CANet with HCA-FE and concatenation fusion scheme, segmentation output of CANet with HCA-FE and CG-ACRF fusion module. Best viewed in colors.}
    \label{fig:ablation}
\end{figure*}

From Table \ref{table:ablationstudy}, we observe that the GC obtains better performance than CC. For the dice score, GC achieves 0.89365 for the entire tumor and 0.82246 for the tumor core. CC only achieves a dice score of 0.87467 on the entire tumor and 0.82068 on the tumor core, which is 2\% and 0.2\% lower than those by GC respectively. For hausdorff95, GC achieves 6.40312 on the entire tumor and 5.81216 on the tumor core. CC achieves 6.88633 and 7.93923, which are 0.49321 and 2.12707 higher than those of GC on the entire tumor and the tumor core, respectively. From Fig. \ref{fig:ablation}, we observe that GC can accurately predict individual regions. For example, the GD-enhanced tumor region normally does not appear at the outside of the tumor region. This superior performance may benefit from the information learned from the feature interactive graph as the feature nodes of different tumor regions have a strong structural association between them. Learning the relationship may help the system to predict correct labels of the tumor regions. However, the sensitivity of GC is much higher than that of CC. In Table \ref{table:ablationstudy}, for example, the sensitivity score of GC is higher than that of CC: 12.02\% higher on the enhancing tumor, 4.469\% higher on the entire tumor, 8.104\% higher on tumor core, respectively. We observe poor segmentation results at the NCR/ECT region by GC, worse than CC and the ground truth shown in Fig. \ref{fig:ablation}.

We then evaluate the complete HCA-FE with the extracted feature maps by CC and GC simultaneously. Here, we fuse the feature maps of CC and GC using a naive concatenation method. The HCA-FE has less over-segmentation results, depicted in Table \ref{table:ablationstudy}, where the sensitivity of CC+GC is much lower than that of GC. The sensitivity of CC+GC is  0.85725 on the enhancing tumor (ET), 0.92243 on the whole tumor (WT) and 0.86085 on the tumor core (TC), respectively. From Fig. \ref{fig:ablation}, we witness that by introducing the complete HCA-FE, the segmentation model can correct some misclassified regions produced by CC. However, the concatenation fusion method does not demonstrate any benefit to the overall segmentation. CC+GC has a dice score of 0.86073 on the whole tumor and 0.80306 on the tumor core, which is 3.292\% and 1.94\% lower than those of GC respectively. We also observe the loss of the boundary information in Fig. \ref{fig:ablation}, especially the boundaries of NCR/ECT and GD-enhancing tumors excessively shrinks compared with those of GC and CC.

We finally evaluate the effectiveness of our proposed CG-ACRF. By introducing the CG-ACRF fusion module, our segmentation model outperforms the other methods. Benefiting from the inference ability of CG-ACRF, it presents a satisfactory segmentation output. For the whole tumor and the tumor core, its Dice scores are 0.90338 and 0.87291 respectively, which are the top scores in the leader-board. Its Hausdorff95 also is the lowest. For the whole tumor and the tumor core, its hausdorff95 values are 3.56898 and 4.03629 respectively. Referring to much lower sensitivity scores reported in Table \ref{table:ablationstudy}, we conclude that the superior performance has been achieved by the complete CANet. The same conclusion can be drawn from Fig. \ref{fig:ablation} where CG-ACRF can detect optimal feature maps that benefit the downstream deconvolution networks and outline small tumor cores and edges, which may be lost when we use a down-sampling operation in the encoder backbone.


\subsection{Iteration Test}
As described in Algorithm 1, we manually set the iteration number in the mean-field approximation of CG-ACRF. Since the mean-field approximation cannot guarantee a convergence point, we examine the effectiveness of different iteration numbers. Table \ref{table:convmf} reports the quantitative result of using different iteration numbers, i.e. 1, 3, 5, 7, and 10. With the increase of iterations, our proposed model performs better. However, no additional benefit is gained when the iteration number becomes over 5. Fig. \ref{fig:ProbMap} presents the probability map during segmentation, where the light color represents the region with a lower probability while the dark color represents the area with a higher probability. We observe that using only one iteration, CANet can outline the region of interest using the fused feature maps. By increasing the iteration number to 3 or 5, CG-ACRF can gradually extract an optimal feature map, leading to accurate segmentation. We further increase the iteration number to 7 and 10 but no further improvement has been made. Therefore, we set the iteration number to 5 as a good trade-off between the segmentation performance and the number of the engaged parameters.

\begin{table*}[!ht]
\centering
\caption{Quantitative results of different iteration numbers by CG-ACRF mean-field approximation on the five fold cross-validation of the BraTS2017 training set with respect to Dice, Sensitivity, Specificity and Hausdorff95. The best result is in bold and the runner-up result is underlined.}
% Please add the following required packages to your document preamble:
% \usepackage[normalem]{ulem}
\begin{adjustbox}{width=1\textwidth}
\begin{tabular}{|c|c|c|c|c|c|c|c|c|c|c|c|c|}
\hline
             & \multicolumn{3}{c|}{Dice}                              & \multicolumn{3}{c|}{Sensitivity}                       & \multicolumn{3}{c|}{Specificity}                       & \multicolumn{3}{c|}{Hausdorff95}                       \\ \hline
Iteration \# & ET               & WT               & TC               & ET               & WT               & TC               & ET               & WT               & TC               & ET               & WT               & TC               \\ \hline
1            & 0.65697          & 0.86066          & 0.79005          & \textbf{0.90075} & 0.92017          & 0.85205          & 0.9953           & 0.98993          & {\underline{0.99426}}    & 7.99666          & 7.74909          & 10.48848         \\ \hline
3            & 0.68131          & {\underline{0.87267}}    & {\underline{0.80679}}    & {\underline{0.87265}}    & 0.92288          & {\underline{0.86948}}    & 0.99638          & {\underline{0.99007}}    & 0.99397          & \textbf{7.61352} & {\underline{6.8011}}     & {\underline{8.94057}}    \\ \hline
7            & 0.6643           & 0.85534          & 0.76902          & 0.85384          & 0.92108          & 0.86033          & 0.99644          & 0.98955          & 0.99336          & 9.84976          & 9.72             & 12.04193         \\ \hline
10           & {\underline{0.68484}}    & 0.85043          & 0.7839           & 0.83708          & \textbf{0.93128} & 0.85847          & {\underline{0.99675}}    & 0.98757          & 0.99383          & 8.06683          & 11.14894         & 11.64947         \\ \hline
Ours(5)    & \textbf{0.68489} & \textbf{0.90338} & \textbf{0.87291} & 0.80651          & {\underline{0.92363}}    & \textbf{0.86989} & \textbf{0.99746} & \textbf{0.99307} & \textbf{0.99592} & {\underline{7.80448}}    & \textbf{3.56898} & \textbf{4.03629} \\ \hline
\end{tabular}
\end{adjustbox}
\label{table:convmf}
\end{table*}

\begin{figure*}[!ht]
    \centering
    \includegraphics[width=\textwidth]{ProbMap_Iteration.png}
    \caption{Examples to illustrate the effectiveness of different iteration numbers by mean-field approximation in CG-ACRF. Columns from top to bottom represent different patient cases. Rows from left to right indicate FLAIR data, ground truth annotation, attentive map generated by CANet with different iteration numbers (from 1 to 10) in CG-ACRF respectively. Best viewed in colors.}
    \label{fig:ProbMap}
\end{figure*}

\subsection{Comparison with State-of-The-Art methods}
We choose several state-of-the-art deep learning model based brain tumor segmentation methods, including 3D UNet \cite{cciccek20163d}, Attention UNet \cite{oktay2018attention}, PRUNet \cite{brugger2019partially}, NoNewNet \cite{isensee2018no} and 3D-ESPNet \cite{mehta20193despnet}. We first consider the 5-fold cross-validation on the BraTS2017 training set. Each fold contains randomly chosen 228 cases for training and 57 cases for validation. In these cross-validation experiments on the training set, we consider CANet with complete HCA-FE and CG-ACRF fusion module with 5-iteration, which leads to the best performance in the ablation tests. As shown in Table \ref{table:SOTACompare}, our CANet outperforms the rest State-of-The-Art methods on several metrics while the results of the other metrics are competitive. The Dice score of CANet is 0.90338 and 0.87291 for the whole tumor and the tumor core respectively. The former is 8\% higher and the latter is 3\% higher than individual runner-up results. The Hausdorff95 values of CANet are 3.56898 and 4.03629 for the whole tumor and the tumor core, which are much lower than the runner up scores, i.e. 4.15649 and 5.77847, respectively.

To further evaluate the segmentation output, we compare the segmentation output of the proposed approach against the ground-truth. Fig. \ref{fig:SOTA} shows that the proposed CANet can effectively predict the correct regions including small tumor cores and complicated edges while the other state of the art methods fail to do so. Fig. \ref{fig:3D} presents the example segmentation result and the ground-truth annotation in 3D visualisation. From Fig. \ref{fig:3D}, we can observe that our proposed CANet effectively captures 3D forms and shape information in all different circumstances.

Fig. \ref{fig:training record} reports the training curve of CANet and the other state-of-the-art methods. Our proposed method converges to a lower training loss using fewer epochs. Taking the advantage of the powerful HCA-FE and the proposed fusion module CG-ACRF, CANet achieves satisfactory outlining for the brain tumors. With the training epoch increasing, CANet can fine-tune the segmentation map and successfully detect small tumor cores and boundaries.\\

\begin{table*}[!ht]
\centering
\caption{Quantitative results of the state-of-the-art models by cross-validation for the BraTS2017 training set with respect to dice, sensitivity, specificity and hausdorff. The best result is shown in bold and the runner-up result is underlined.}
\begin{adjustbox}{width=\textwidth}
\begin{tabular}{|c|c|c|c|c|c|c|c|c|c|c|c|c|}
\hline
               & \multicolumn{3}{c|}{Dice}                              & \multicolumn{3}{c|}{Sensitivity}                       & \multicolumn{3}{c|}{Specificity}                       & \multicolumn{3}{c|}{Hausdorff95}                       \\ \hline
Model          & ET               & WT               & TC               & ET               & WT               & TC               & ET               & WT               & TC               & ET               & WT               & TC               \\ \hline
3D-UNet\cite{cciccek20163d}        & 0.70646          & 0.86492          & 0.81032          & 0.80275          & 0.9064           & 0.82906          & 0.99791          & 0.99005          & 0.99493          & 6.62407          & 8.19351          & 8.95848          \\ \hline
No-New Net\cite{isensee2018no}     & {\underline{0.74108}}    & 0.87083          & 0.8125           & 0.76688          & 0.89296          & 0.83115          & \textbf{0.99853} & {\underline{0.99243}}    & 0.99528          & \textbf{3.93033} & 7.05536          & 7.64098          \\ \hline
Attention UNet\cite{oktay2018attention} & 0.67174          & 0.8634           & 0.77837          & \textbf{0.84741} & 0.9001           & 0.86171          & 0.99591          & 0.98961          & 0.99186          & 9.34711          & 9.67562          & 10.66793         \\ \hline
PRUNet\cite{brugger2019partially}         & 0.71015          & 0.89072          & 0.81447          & 0.78826          & 0.90028          & 0.84056          & {\underline{0.99804}}    & 0.99002          & 0.99586          & 7.20534          & 7.41411          & 9.1874           \\ \hline
3D-ESPNet\cite{mehta20193despnet}      & 0.68949          & {\underline{0.89548}}    & {\underline{0.84397}}    & 0.80535          & \textbf{0.94666} & \textbf{0.88085} & 0.99671          & 0.99026          & {\underline{0.99677}}    & 6.89359          & {\underline{4.15649}}    & {\underline{5.77847}}    \\ \hline
CANet (Ours)   & 0.68489          & \textbf{0.90338} & \textbf{0.87291} & 0.80651          & {\underline{0.92363}}    & {\underline{0.86989}}    & 0.99746          & \textbf{0.99307} & 0.99592          & 7.80448          & \textbf{3.56898} & \textbf{4.03629} \\ \hline
\end{tabular}
\end{adjustbox}
\label{table:SOTACompare}
\end{table*}

\begin{figure*}[!ht]
    \centering
    \includegraphics[width=\textwidth]{SOTAComparision.pdf}
    \caption{Examples of segmentation results by cross validation for the BraTS2017 training set. Qualitative comparisons with other brain tumor segmentation methods are presented. The eight columns from left to right show the frames of the input FLAIR data, the ground truth annotation, the results generated from our CANet (UNet encoder backbone with HCA-FE and 5-iteration CG-ACRF), 3DUNet \cite{cciccek20163d}, NoNewNet \cite{isensee2018no}, Attention UNet \cite{oktay2018attention}, PRUNet \cite{mehta20193despnet}, respectively. Black arrows indicate the failure in these comparison methods. Best viewed in colors.}
    \label{fig:SOTA}
\end{figure*}

\begin{figure*}[!ht]
    \centering
    \includegraphics[width=0.5\textwidth]{3D.png}
    \caption{3D segmentation results of two volume cases by cross-validation on the BraTS2017 training set. The first and the third rows indicate the ground truth annotation. The second and the fourth rows indicate the segmentation result of our proposed CANet with HCA-FE and 5-iteration CG-ACRF. Rows from left to right indicate the qualitative comparison for the whole tumor, NCR/ECT, GD-enhancing tumor and Pertumoral Edema respectively. Best viewed in colors.}
    \label{fig:3D}
\end{figure*}

\begin{figure*}[!ht]
    \centering
    \includegraphics[width=0.5\textwidth]{NewTraingingLog.png}
    \caption{The learning curve of the state of the art methods and our proposed CANet with HCA-FE and 5-iteration CG-ACRF. Best viewed in color.}
    \label{fig:training record}
\end{figure*}

\begin{table*}[!ht]
\centering
\caption{Quantitative results comparison between CANet and other state of the art results on the BraTS2017 validation set for Dice and Hausdorff95. The best results of these methods are underlined. The bold shows the best score of each tumor region by single prediction approaches. '-' depicts that the result of the associated method has not been reported yet.}
% Please add the following required packages to your document preamble:
% \usepackage[normalem]{ulem}
\begin{adjustbox}{width=1\textwidth}
\begin{tabular}{|cc|ccc|ccc|}
\hline
                  &                  & \multicolumn{3}{c|}{\textbf{Dice}}               & \multicolumn{3}{c|}{\textbf{Hausdorff95}}        \\ \hline
\textbf{Approach} & \textbf{Method}  & \textbf{ET}    & \textbf{WT}    & \textbf{TC}    & \textbf{ET}    & \textbf{WT}    & \textbf{TC}    \\ \hline
                  & Kamnitsas et al. \cite{kamnitsas2017ensembles} & 0.738          & 0.901          & 0.797          & 4.500          & 4.230          & 6.560          \\
                  & Wang et al. \cite{wang2017automatic}      & {\underline {0.786}}    & {\underline {0.905}}    & {\underline {0.838}}    & {\underline {3.282}}    & {\underline {3.890}}    & {\underline {6.479}}    \\
Ensemble          & Zhao et al. \cite{zhao20173d}     & 0.754          & 0.887          & 0.794          & -              & -              & -              \\
                  & Isensee et al. \cite{isensee2017brain}  & 0.732          & 0.896          & 0.797          & 4.550          & 6.970          & 9.480          \\
                  & Jungo et al. \cite{jungo2017towards}    & 0.749          & 0.901          & 0.790          & 5.379          & 5.409          & 7.487          \\ \hline
                  & Islam et al. \cite{islam2017multi}     & 0.689          & 0.876          & 0.761          & 12.938         & 9.820          & 12.361         \\
                  & Jesson et al. \cite{jesson2017brain}    & 0.713          & \textbf{0.899} & 0.751          & 6.980          & \textbf{4.160} & \textbf{8.650} \\
Single Prediction & Roy et al. \cite{roy2018recalibrating}      & 0.716          & 0.892          & 0.793          & 6.612          & 6.735          & 9.806          \\
                  & Pereira er al. \cite{pereira2019adaptive}   & 0.719          & 0.889          & 0.758          & 5.738          & 6.581          & 11.100         \\
                  & CANet (Ours)    & \textbf{0.728} & 0.892          & \textbf{0.821} & \textbf{5.496} & 7.392          & 10.122         \\ \hline
\end{tabular}
\end{adjustbox}
\label{table:17val}
\end{table*}

\begin{table*}[!ht]
\centering
\caption{Quantitative results of the BraTS2018 validation set with respect to Dice and Hausdorff95. The best results of these methods are underlined. The bold results show the best score of each tumor region using single prediction approaches. '-' represents the result of the associated method has not been reported yet.}
\begin{adjustbox}{width=1\textwidth}
\begin{tabular}{|cc|ccc|ccc|}
\hline
                                  &                                  & \multicolumn{3}{c|}{\textbf{Dice}}                                                                           & \multicolumn{3}{c|}{\textbf{Hausdorff95}}                                                                    \\ \hline
\textbf{Approach} & \textbf{Method} & \textbf{ET}    & \textbf{WT}    & \textbf{TC}    & \textbf{ET}    & \textbf{WT}    & \textbf{TC}    \\ \hline
                                  & Isensee et al. \cite{isensee2018no}                   & \uline{0.796}  & 0.908                           & 0.843                           & 3.120                           & 4.790                           & 8.020                           \\
                                  & McKinley et al. \cite{mckinley2018ensembles}                  & 0.793                           & 0.901                           & \uline{0.847}  & 3.603                           & 4.062                           & \uline{4.988}  \\
Ensemble                           & Zhou et al. \cite{zhou2018learning}                      & 0.792                           & \uline{0.907}  & 0.836                           & \uline{2.800}  & 4.480                           & 7.070                           \\
                                  & Cabezas et al. \cite{cabezas2018survival}                   & 0.740                           & 0.889                           & 0.726                           & 5.304                           & 6.956                           & 11.924                          \\
                                  & Feng et al. \cite{feng2018brain}                      & 0.787                           & 0.906                           & 0.834                           & 3.964                           & \uline{4.018}  & 5.340                           \\ \hline
                                  & Sun et al. \cite{sun2018tumor}                       & 0.751                           & 0.865                           & 0.720                           & -                               & -                               & -                               \\
                                  & Myronenko \cite{myronenko20183d}                        & \textbf{0.816} & \textbf{0.904} & \textbf{0.860} & \textbf{3.805} & \textbf{4.483} & 8.278                           \\
Single Prediction                  & Weninger et al. \cite{weninger2018segmentation}                  & 0.712                           & 0.889                           & 0.758                           & 8.628                           & 6.970                           & 10.910                          \\
                                  & Gates et al. \cite{gates2018glioma}                    & 0.678                           & 0.806                           & 0.685                           & 14.523                          & 14.415                          & 20.017                          \\
                                  & CANet(Ours)                     & 0.767                           & 0.898                           & 0.834                           & 3.859                           & 6.685                           & \textbf{7.674} \\ \hline
\end{tabular}
\end{adjustbox}
\label{table:18val}
\end{table*}
We further investigate the segmentation results on the BraTS2017 and BraTS2018 validation sets, where the quantitative result of each patient case is generated from the online evaluation server. The mean result is reported in Table \ref{table:17val} and Table \ref{table:18val}. Box plot in Fig. \ref{fig:boxplot} shows the distribution of the segmentation result among all the patient cases in the validation set. For the BraTS2017 validation set, our proposed CANet with complete HCA-FE and 5-iteration CG-ACRF achieves the state-of-the-art results of Dice on ET, Dice on TC and Hasdorff95 on ET among the single model segmentation benchmarks. Our CANet has Dice on ET of 0.728, which is higher than the approach reported in \cite{pereira2019adaptive}. The Dice on TC by CANet is 0.821, which is higher than the runner-up result reported in \cite{roy2018recalibrating}. The Hausdorff95 on ET of CANet is 5.496, which is much lower than the runner-up generated in \cite{pereira2019adaptive}. For the BraTS2018 validation set, our proposed CANet achieves the state-of-the-art result for Hausdorff95, i.e. 7.674, on the tumor core, while the other results are all runner-ups. Note that the method proposed by Myronenko \cite{myronenko20183d} has the best performance using most of the evaluation metrics. In Myronenko's method, they set up an additional branch using autoencoder to regularise the encoder backbone by reconstructing the input 3D MRI image. This autoencoder branch greatly enhances the feature extraction capability of the backbone encoder. In our framework, we regularise the network weights using a L2-regularisation without any additional branch, and the result of our proposed CANet is better than the other single prediction methods. Be reminded that the standard single prediction models generate the segmentation only using one network, and do not need much computational resources and a complicated voting scheme. Compared with the ensemble methods, the result of our proposed CANet is still very competitive.

\begin{figure*}[!ht]
    \centering
    \includegraphics[width=0.6\textwidth]{MetricScore.png}
    \caption{Boxplot of the segmentation results by CANet with HCA-FE and 5-iteration CG-ACRF. Dots within yellow boxes are individual segmentation results generated for the BraTS2017 validation set. Dots within blue boxes are individual segmentation results generated for the BraTS2018 validation set. Best viewed in color.}
    \label{fig:boxplot}
\end{figure*}

Note that even we incorporate our designed CG-ACRF as iterative convolutional blocks within the system, our model still maintains a relatively small space. Our final model, i.e. HCA-FE with five times convolution approximation for CG-ACRF maintains a parameter size at 1.84E7. Compared with baseline methods such as Isensee et al. \cite{isensee2018no} with a parameter size at 1.45E7 and Myronenko et al. \cite{myronenko20183d} with a parameter size at 2.01E7 \cite{zhang2020exploring}, our system maintains the parameter size at an intermediate level, which prevents the occurrence of over-fitting.


\subsection{Failure Case Studies}
Fig. \ref{fig:imbalance} shows the statistical information of the BraTS2017 training set. As an example, we here report two failure segmentation cases by our proposed approach, which are shown in Fig. \ref{fig:FailureCase}. During the whole training process, CANet focuses on extracting feature maps with different contextual information, e.g. convolutional and graph contexts. However, we have not designed specific strategies for handling the imbalanced issue of the training set. The imbalanced issue is presented in two aspects. Firstly, there exists an unbalanced number of voxels in different tumor regions. As the exemplar case named "Brats17\_TCIA\_605\_1" is shown in Fig. \ref{fig:FailureCase}, the NCR/ECT region is much smaller than the other two regions, suggesting the poor performance of segmenting NCR/ECT. Secondly, there exists an unbalanced number of patient cases from different institutions. This imbalance introduces an annotation bias where some annotations tend to connect all the small regions into a large region while the other annotation tends to label the voxels individually. As the exemplar case named "Brats17\_2013\_23\_1" is shown in Fig. \ref{fig:FailureCase}, the ground truth annotation tends to be sparse while the segmentation output tends to be connected. In future work, we will consider an effective training scheme based on active/transfer learning which can effectively handle the imbalance issue in the dataset. Despite the imbalance issue, our segmentation method on the overall cases qualitatively outperforms the other state-of-the-art methods.
% Fig. \ref{fig:imbalance} shows the statistical information of the BraTS2017 training set.

\begin{figure*}[!ht]
    \centering
    \includegraphics[width=\textwidth]{imbalance.png}
    \caption{Statistics of the BraTS2017 training set. The left-hand side figure of (a) shows the FLAIR and T2 intensity projection, and the right-hand side figure shows the T1ce and T1 intensity projection. (b) is the pie chart of the training data with labels, where the top figure shows the HGG data labels while the bottom figure shows the LGG labels. There are a large region and label imbalance cases here. Best viewed in colors.}
    \label{fig:imbalance}
\end{figure*}

\begin{figure*}[!ht]
    \centering
    \includegraphics[width=\textwidth]{FailureCase.png}
    \caption{Qualitative comparisons in the failure cases. Rows from left to right indicate the input data of the FLAIR modality, ground truth annotation, segmentation result from our CANet, segmentation result from the other SOTA methods respectively. Our results look better than the SOTA methods results. Best viewed in colors.}
    \label{fig:FailureCase}
\end{figure*}

\section{Discussion}
In summary, we have proposed a novel 3D MRI brain tumor segmentation approach called CANet. Considering different contextual information in standard and graph convolutions, we proposed a novel hybrid context-aware feature extractor combined with a deep supervised convolution and a graph convolution stream. Different from previous works that used naive feature fusion schemes such as element-wise summation or channel-wise concatenation, we here designed a novel feature fusion model based on the conditional random field called context guided attentive conditional random field (CG-ACRF), which effectively learns the optimal latent features for downstream segmentations. Furthermore, we formulated the mean-field approximation within CG-ACRF as convolutional operations, which incorporate the CG-ACRF in a segmentation network to perform end-to-end training. We conducted extensive experiments to evaluate the effectiveness of the proposed HCA-FE, CG-ACRF and the complete CANet frameworks. The results have shown that our proposed CANet achieved state-of-the-art results in several evaluation metrics. In the future, we consider combining the proposed network with novel training methods that can better handle the imbalance issue in the datasets.
%----------------------------------------------------------------------------------------
%	SECTION 1
%----------------------------------------------------------------------------------------



% Chapter Template

\chapter{Hierarchical Homography Estimation Network for Medical Image Mosaicing} % Main chapter title

\label{Chapter4} % Change X to a consecutive number; for referencing this chapter elsewhere, use \ref{ChapterX}

%----------------------------------------------------------------------------------------
%	SECTION 1
%----------------------------------------------------------------------------------------

\section{Introduction}
\label{C4:S1}

In Chapter \ref{Chapter3}, we introduced an example of implicit deep relational learning paradigm on medical image analysis, i.e. use graph convolution kernel to learn the correlation information between different features on the built feature graph. One of the shortcomings of this implicit relational learning is that it only focuses on the relationship of features on one image while ignoring the relation relevance of multiple images along the time dimension. To this end, we propose a novel Hierarchical Homography Estimation Network (HHEN) for medical image mosaicing, which can explicitly model the spatial relationship between frames in both short and long ranges.

Image mosaicing refers to sequentially transferring and combining a set of narrow field-of-view images to form a new image or video with a wider field-of-view. Traditional image mosaicing has been widely used in various areas such as satellite photographing \cite{bu2016map2dfusion}, augmented reality \cite{azzari2008markerless} and panoramic photo edition \cite{li2018deep}. In the last few years, the interest in medical image mosaicing has grown in the medical image analysis community, especially in fetoscopic laser photography mosaicing \cite{bano2020deep, bano2020deep2}. Clinical physicians use fetoscopic video to detect vascular anastomoses, make treatment plans and control surgical robots. Thus clinicians can be able to treat various diseases such as abnormal vascular anastomoses in twin-to-twin transfusion syndrome (TTTS). However, it is a challenging problem for clinicians to fully use the fetoscope due to its narrow FoV and low visibility. Image mosaicing can compose a set of fetoscopic images with correlation transformation to generate a new image with wider FoV, thus it can offer computer-assisted diagnosis evidence to effectively detect the vascular anastomoses location.

Traditional mosaicing methods commonly use low-level features, i.e. point-based mosaicing and HoG based mosaicing \cite{kanazawa2004image, zagrouba2009efficient, kourogi1999real}. However, traditional mosaicing methods do not perform well due to poor medical image quality, e.g. turbid amniotic fluid in TTTS treatment. Recently, several deep-learning-based mosaicing methods have been proposed \cite{lv2019improved, nguyen2018unsupervised}. These methods learned deep features or homography estimation automatically and achieved superior performance on various datasets. Despite the remarkable progress that has been made, one open challenge is that these methods only focus on learning local information between adjacent frames while ignoring the non-local information in long series, which is important for time serious tasks. To overcome this, we propose a novel hierarchical medical image mosaicing network using neural homography estimation. Our contributions are summarized as follows:

\begin{enumerate}
\item We propose a new deep hierarchical homography estimation network to automatically and hierarchically estimate the 8 degree-of-freedom homography, upon which multi-scale (local level between adjacent frames and non-local level between long-range frames) homography are jointly learned and optimized in a data-driven manner.

\item Inspired by recent works on color homography \cite{finlayson2017color}, we propose a new data generation method called Partially Image Generation (PIG). PIG only perturb the color, rotation, and translation movement between adjacent frames. The generated frames by PIG can be used for evaluating model performance during network training.

\item We conduct extensive experiments on five different video clips from UCL Fetoscopy Placenta dataset. The results show that our method achieves state-of-the-art performance and generalizes well on different clips under different numbers of frames and different acquisition methods.
\end{enumerate}

The rest of this Chapter is organized as follows. In Section \ref{C4:S2}, we briefly review related works on image mosaicing. In Section \ref{C4:S3}, we introduced our proposed method, including HHEN in Section \ref{C4:S3:SS1} and PIG in Section \ref{C4:S3:SS2} respectively. In Section \ref{C4:S4}, we demonstrate the dataset and implementation details. In Section \ref{C4:S5}, our proposed HHEN is evaluated and compared with the state-of-the-art methods. In addition, the results are analysed in detail. This Chapter is finally concluded in Section \ref{C4:S6}.


\section{Review of Recent Image Mosaicing Methods}
\label{C4:S2}
In this chapter, we briefly review related works on image mosaicking, which refers to the alignment of multiple overlapping images into a large combination with larger FoV.

\begin{figure}[t]
    \centering
    \includegraphics[width=\textwidth]{MosaicingFlow.png}
    \caption{A schematic pipeline for image mosaicing.}
\label{fig:mosaicingflow}
\end{figure}

Finding the point-to-point or other feature correlation between two images is the fundamental part of conventional image mosaicking, which can be found in the mosaicing schematic pipeline shown in Fig. \ref{fig:mosaicingflow}. Traditional image mosaicing methods commonly use low-level features such as edge and point to find the correlation between two images. Kanazawa et al. \cite{kanazawa2004image} extract 100 feature points generated from the Harris operator to find the correlation between two images. Zagrouba et al. \cite{zagrouba2009efficient} extract both Harris points and the semantic segmentation as multi-level features for correlation learning. Kourogi et al. \cite{kourogi1999real} first generated the pseudo-motion vector on each pixel, and then estimate the affine motion parameters by using pixel-wise motion vector matching. Heikkila et al. \cite{heikkila2005image} extract the correlation feature using SIFT first and then reject the outlier points by applying RANSAC algorithms. Botterill et al. \cite{botterill2010real} first represent the image with the Bag-of-Words (BoW) method to find the adjoining frames and then use a perspective transformation to register adjoining frames to achieve video mosaicing.

Different from natural images, the quality of medical images is interfered with by many factors, such as fetoscopic photography captured within a turbid amniotic fluid environment. And traditional feature extraction algorithms for points or edges cannot extract effective and meaningful features. Although the aforementioned algorithms are simple and fast with low demands on computing resources. However, the aforementioned algorithms are difficult to be applied to tasks that require higher precision, such as medical image mosaicing.

Recently, various deep-learning-based image mosaicing methods have been presented. Lv et al. \cite{lv2019improved} proposed a CNN based Speed-up Robust Feature (SURF) extraction method for image mosaicing. Nguyen et al. \cite{nguyen2018unsupervised} proposed an unsupervised learning network for homography estimation in image mosaicing. However, these two methods’ input is still hand-crafted features, which cannot be jointly optimized with the mosaicing network. Zhang et al. \cite{zhang2019convolutional} use a CNN for registration based microscopic image mosaicing. Bano et al. \cite{bano2020deep} use a VGG-stylized CNN for homography estimation between two adjacent frames. However, these methods only focus on adjacent frame homography estimation (local information), while ignoring the correlation between long-range frames, which has been proven as an important factor for sequential frame-based tasks \cite{wang2018non, jiang2018modeling, anderson2018bottom}. To address this open challenge, we propose our novel hierarchical neural homography estimation network (HHEN) for fetoscopic photography mosaicing, which can automatically and hierarchically estimate the 6 degree-of-freedom homography. Besides, we propose a new generation method called Partially Image Generation (PIG). PIG only perturbs the color, rotation, and translation movement between adjacent frames. The generated frames by PIG can be used for evaluating model performance during training.

\section{Proposed Method}
\label{C4:S3}
Recently, deep image homography (DIH) estimation \cite{detone2016deep} have been proposed that use a regression neural network to estimate the homography between two local patches in one image. Based on DIH, several image mosaicing methods \cite{bano2020deep, bano2020deep2} have been proposed to stitching a set of images by estimating homography between every adjacent frame. However, DIH and proposed mosaicing work only focused on adjacent frame feature learning (local information) while ignored the correlation between long-range frames (non-local information), which has been proved as an important factor for achieving video and serious tasks. In this chapter, we first propose our novel image homography estimation framework called Hierarchical Homography Estimation Network (HHEN) in Chapter \ref{C4:S3:SS1}. HHEN treat the target frame as a query frame while transformed both local and non-local template frame as key and value frame. Thus the homography can be learned and optimize hierarchically with local and non-local information in a data-driven manner. We further propose a novel generation method called Partially Image Generation (PIG). PIG only controls the rotation, translation movement and color restoration between adjacent frames as the incision point of fetoscopic photography are fixed. Without any external sensors like electromagnetic tracker (EMT) used in \cite{tella2018probabilistic}, our proposed HHEN with PIG minimizes the drift error effectively and achieve the state-of-the-art result on fetoscopic photography mosaicing.

\subsection{Hierarchical Homography Estimation Network}
\label{C4:S3:SS1}

\begin{figure}[t]
    \centering
    \includegraphics[width=\textwidth]{DIH.png}
    \caption{A schematic pipeline for Deep Image Homography (DIH) estimation.}
\label{fig:DIH}
\end{figure}

A prilimary of our proposed HHEN is deep image homography (DIH) shown in Fig. \ref{fig:DIH}, where DIH estimates the homography between two relative image local patches $\left( P_{A} and P_{B} \right)$ from a single image. Similar to DIH, we also use a VGG style backbone network to estimate homography $\textbf{H}$. Instead of using $\textbf{H}$ with 9 degree-of-freedom, we here define the target homography is 3 point related due to the rotation and shear operation in \cite{detone2016deep} has little effect in our scenario (Due to the fetoscopic photography data is obtained through a fixed incision point). By defining arbitrary three corner points coordinate $\left( x_i, y_i \right)$ of $P_{A}$ and $\left( x_{i}^{\prime}, y_{i}^{\prime} \right)$ of $P_{B}$ with $i=1,2,3$, the 3 point related homography $\textbf{H}$ is defined as:

\[
\textbf{H} =
\begin{bmatrix}
\Delta x_1 & \Delta x_2 & \Delta x_3\\
\Delta y_1 & \Delta y_2 & \Delta y_3
\end{bmatrix}^{\top}
\]

where $\Delta x_i = x_{i}^{\prime} - x_{i}$, $\Delta y_i = y_{i}^{\prime} - y_{i}$. The fourth corner can be calculated due to the patch is extracted as a rectangle. Note that DIH generates $P_B$ by randomly displacing $P_A$ as illustrated in Fig. \ref{fig:DIH}, thus the drift error is not acceptable for image-based mosaicing tasks as the drift error will be accumulated. Mosaicing requires relative homography between adjacent frames with respect to the template frame. For minimizing the drift error in relative homography, several related works \cite{bano2020deep2, bano2020deep} have been proposed to let the backbone network learn the homography between patches extracted from adjacent frames. However, the homography estimation networks in \cite{bano2020deep2, bano2020deep} only considered the local information between adjacent frames while ignoring the non-local information between long-range consecutive frames, which has been proven as an important factor for video applications \cite{wang2018non, jiang2018modeling, anderson2018bottom}. Therefore, we propose a hierarchical homography estimation network (HHEN) that learns long-range information between a set of consecutive frames. As shown in Fig. \ref{fig:HHEN}, we extend the definition of a pair of adjacent frames to a hierarchical composition. We set the frame that needs to be moved (target frame) as \textit{Query Frame} while the template frame is assembled with long-range \textit{Key Frame} and short-range \textit{Value Frame}. Intuitively, for the target \textit{Query Frame}, the network estimates the homography with respect how \textit{Value Frame} moves based on previous non-local \textit{Key Frame}.

Inspired by non-local image de-noising operation \cite{buades2005non} and non-local neural network for video classification \cite{wang2018non}, we define the generic non-local frame fusion head in HHEN as:

\begin{equation}
y_{i} = G(x_{i}^{Q})\frac{1}{N(x)}\sum_{\forall j < i} F(x_{i}^{V}, x_{j}^{K})G(x_{j}^{Q})
\label{eqn:nonlocal}
\end{equation}

wherein Eqn. \ref{eqn:nonlocal}, $i$ is the position of the output in the spatial domain and $j$ is the neighborhood position of $i$ (we use the 8-neighbor pixels in this paper). $x^{Q}, x^{K}, x^{V}$ denotes the input image patch from the query frame, key frame and value frame respectively. $y$ is the output fusion feature map with the same size as $x$. $F(\cdot)$ is the pairwise function that takes the pairwise input $(x_{i}^{V} , x_{j}^{K})$. The output of $F(\cdot)$ is a scalar weight between two inputs, which represents the affinity transformation between $x_{i}^{V}$ and $x_{j}^{K}$. $G(\cdot)$ is the unary function that outputs the representation of the input signal. $N(\cdot)$ is the normalization factor. Overall, Eqn. \ref{eqn:nonlocal} denotes that, for a given input $x_{i}^{Q}$ at position $j$ from query frame $x^{Q}$, $F(\cdot)$ learns how the related input from value frame is updated based on a key frame, then the non-local operator uses the learned affinity transformation between value frame and key frame to update the feature computed by $G(\cdot)$ in query frame.

\begin{figure}[t]
    \centering
    \includegraphics[width=\textwidth]{HHEN.png}
    \caption{A schematic pipeline for Hierarchical Homography Estimation Netowork (HHEN) estimation.}
\label{fig:HHEN}
\end{figure}


The natural instantiation choice  of $F(\cdot)$ is the Gaussian function as implemented in the non-local mean operator \cite{buades2005non}, which is formalized as:

\begin{equation}
F(x_{i}^{V}, x_{j}^{K}) = e^{{x_{i}^{V}}^\top x_{j}^{K}}
\label{eqn:gaussian}
\end{equation}

In Eqn. \ref{eqn:gaussian}, the ${x_{i}^{V}}^\top x_{j}^{K}$ is the dot product to measure the similarity between $x_{i}^{V}$ and $x_{j}^{K}$. Correspondingly, the normalization factor is $N(x) = \sum_{\forall j < i} F(x_{i}^{V}, x_{j}^{K})$. Since the entire system is trained in an end-to-end fashion, in order to enable the final regression loss to directly adjust the parameter update of the non-local operator, we expand the Gaussian function (Eqn. \ref{eqn:gaussian}) to an embedding manner:

\begin{equation}
F(F(x_{i}^{V}, x_{j}^{K})) = e^{\theta(x_{i}^{V}) \top \phi(x_{j}^{K})}
\label{eqn:gaussianembedding}
\end{equation}

In Eqn. \ref{eqn:gaussianembedding}, $\theta(\cdot)$ and $\phi(\cdot)$ are two linear embedding functions, i.e. $\theta(x_{i}^{V}) = W_{\theta}x_{i}^{V}$ and $\phi(x_{j}^{K}) = W_{\phi}x_{j}^{K}$. By using the embedding Gaussian function, given the normalization factor is $N(x) = \sum_{\forall j < i} F(x_{i}^{V}, x_{j}^{K})$, the implementation of the non-local fusion head can be regarded as a softmax activation with inputs from two convolution features generated from $1 \times 1$ convolution kernel:

\begin{equation}
\textbf{y} = G(\textbf{x}) \textit{softmax}(\textbf{x}^{\top} W_{\theta}^{\top} W_{\phi} \textbf{x})G(\textbf{x})
\label{eqn:nonlocalimpl}
\end{equation}

By choosing the same implementation using $1 \times 1$ convolution for $G(\cdot)$, the non-local fusion head can be jointly trained with the downstream network. The output of the HHEN is the homography $H^{Q}$ with respect to the target frame, i.e. \textit{Query Frame}. By iterate every frame in an image set as \textit{Query Frame} and compute every frame's $H^{Q}$, we can complete mosaicing all the image frames in the given image set. 

\subsection{Partially Image Generation}
\label{C4:S3:SS2}
Note that in \cite{detone2016deep}, the author used the MS-COCO dataset \cite{lin2014microsoft} for training and testing. Compared with the MS-COCO dataset which contains natural real images, fetoscopic photography contains particular artifacts such as color distortion caused by amniotic fluid particles. Also, the shear and scale contribute little effect when HHEN is trained with fetoscopic images as the fetoscopic image is obtained through a fixed incision point with constrained distance from the placenta. Thus to minimize the drift error during mosaicing, we propose our Partially Image Generation (PIG) generate the training dataset. PIG only assumes that only rotation, translation and color homography are related between the pair of key-value and value-query frames. For a given image $I_{A}$ and its extracts patch $P_{A}$ with three corners coordinates ($x_{i}, y_{i}$), ($i=1,2,3$). We applied:

\begin{itemize}
\item Rotation with angle $\alpha$ and translation with movement distance $\delta x$ on x coordinate and $\delta y$ on y coordinate.\\
\[
\begin{bmatrix}
x_{i}^{\prime} \\
y_{i}^{\prime}
\end{bmatrix} = \begin{bmatrix}
cos(\alpha) & sin(\alpha) \\
-sin(\alpha) & cos(\alpha)
\end{bmatrix} \begin{bmatrix}
x_{i} \\
y_{i}
\end{bmatrix} + \begin{bmatrix}
\Delta x \\
\Delta y
\end{bmatrix}
\]

\item Color homography transformation: we transfer the original color image with every pixel $\textit{R, G, B}$ value to the greyscale image with every pixel's intensity $\textit{Y}$:\\
\[ 
\textit{Y} = \frac{max(\textit{R, G, B}) + min(\textit{R, G, B})}{2}
\label{Eqn:PIAGrey}
\]
\end{itemize}

to obtain the corresponding pair image $I_{B}$ with perturbed patch $P_{B}$. We empirically set the rotation angle $\alpha$ and translation movement distance $\Delta x$, $\Delta y$ (Sec. \ref{C4:S4:SS2}). By implementing the PIG, the HHEN can learn the homography ($\alpha$, $\Delta x$ and $\Delta y$) without interference from color distortion (the color with artifacts are transformed into greyscale images).

\section{Experiment Setup}
\label{C4:S4}
We conduct extensive experiments to evaluate our proposed fetoscopic photography mosaicing method using HHEN with PIG. We first introduce the datasets and evaluation metrics first (Sec. \ref{C4:S4:SS1}). Then we express the details of method implementations (Sec. \ref{C4:S4:SS2}).

\subsection{Datasets and Evaluation Metrics}
\label{C4:S4:SS1}
We use the 5 fetoscopic video clips from UCL Fetoscopy Placenta Data \cite{bano2020vessel} \footnote{https://www.ucl.ac.uk/interventional-surgical-sciences/fetoscopy-placenta-data}, which includes two synthetic video clips (SYN1 and SYN2), one TTTS Phantom in the water video clip (TTTS1) and two in-vivo TTTS procedure video clips (INVT1 and INVT2). We show the details of all 5 datasets in Table \ref{table:mosaicingdata}. We can observe from Table \ref{table:mosaicingdata} that the datasets are varied among different factors such as object texture, ambient reflected light and capture motion, which post the challenges for mosaicing methods.

\begin{sidewaystable}
\begin{tabular}{|l|c|c|c|c|c|}
\hline
Frame Example      &   \raisebox{-\totalheight}{\includegraphics[width=0.15\textwidth, height=30mm]{Figures/SYN1.jpg}}         &\raisebox{-\totalheight}{\includegraphics[width=0.15\textwidth, height=30mm]{Figures/SYN2.jpg}}           &\raisebox{-\totalheight}{\includegraphics[width=0.15\textwidth, height=30mm]{Figures/TTTS1.jpg}}                       & \raisebox{-\totalheight}{\includegraphics[width=0.15\textwidth, height=30mm]{Figures/INVT1.png}}                       & \raisebox{-\totalheight}{\includegraphics[width=0.15\textwidth, height=30mm]{Figures/INVT2.png}}                       \\ \hline
Data Type          & \begin{tabular}[c]{@{}c@{}}Synthetic\\ (SYN1)\end{tabular} & \begin{tabular}[c]{@{}c@{}}Synthetic\\ (SYN2)\end{tabular} & \begin{tabular}[c]{@{}c@{}}TTTS Phantom \\ in water\\ (TTTS1)\end{tabular} & \begin{tabular}[c]{@{}c@{}}In-vivo \\ TTTS Procedure\\ (INVT1)\end{tabular} & \begin{tabular}[c]{@{}c@{}}In-vivo \\ TTTS Procedure\\ (INVT2)\end{tabular} \\ \hline
Total Frame Num & 500       & 200       & 200                   & 400                    & 100                    \\ \hline
Image Resolution   & 500 x 500 & 500 x 500 & 600 x 600             & 448 x 448              & 448 x 448              \\ \hline
Camera View        & Planar    & Planar    & Planar                & Non Planar             & Planar                 \\ \hline
Motion Type        &Spiral & Circular & Circular                                                           & \begin{tabular}[c]{@{}c@{}}Exploratory \\ Freehand\end{tabular}   & \begin{tabular}[c]{@{}c@{}}Exploratory \\ Freehand\end{tabular}   \\ \hline
\end{tabular}
\caption{Details of the selected datasets for experimental analysis}
\label{table:mosaicingdata}
\end{sidewaystable}

We compare our method with state-of-the-art deep-learning-based methods: deep image homography (DIH) \cite{detone2016deep} and deep sequential mosaicking (DSM) \cite{bano2020deep} and a hand-crafted feature-based method (FEAT with SURF feature matching for homography estimation) \cite{brown2007automatic}. For the synthetic datasets SYN1 and SYN2, we evaluate our HHEN with PIG against all baseline methods and report the mean residual error (MSE) against ground truth data. For datasets obtained from TTTS clinical surgery (TTTS1, INVT1, INVT2), we evaluate the performance of our HHEN with PIG against all baseline methods and report the average root mean square error (RMSE) between image and generated image. Following \cite{brasch2018semantic, godard2019digging}, we also evaluate the average photometric error (APE) by using the $L1$ distance between two images generated by the estimated homography.

\subsection{Implementation Details}
\label{C4:S4:SS2}
We first randomly select 435 images from all datasets except the INVT1 to compose a training set. The image frames in the training set are not considered in the testing set as they are not a full video clip. INVT1 thus is unseen for the model during the training. By using PIG, all images are transferred into greyscale images where the intensity in Eqn. \ref{Eqn:PIAGrey} represent the brightness converted from RGB color to avoid color distortion. Our proposed network and other baseline methods are complied in Keras \footnote{https://keras.io/} platform with a Tensorflow \footnote{https://www.tensorflow.org/} backend. We train our proposed network on a single Nvidia Tesla T4 GPU (16GB) for 2000 epochs. We use the Stochastic Gradient Descent (SGD) as the optimizer with a learning rate $lr = 0.0001$ and momentum with $0.9$. Due to the limitation of GPU memory, we set the batch size as 16. We use the Euclidean loss function as the main goal of the homography is to estimate the distance of rotation and transition between adjacent frames. During training with PIG, we manually set the rotation angle $\alpha$ between ($-8^{\circ}, +8^{\circ}$) and the translation distance $\Delta x$, $\Delta y$ between ($-15, +15$) pixels. The manually perturbed rotation angle and transition distance are stored as ground truth. The network takes the original image patch and the perturbed image patch as inputs and outputs the numerical results. The loss can be estimated between the numerical outputs and the manually settled angel and transition distance.



\section{Experiment Results}
\label{C4:S5}

We conduct sufficient experiments on 5 video clips from UCL Fetoscopy Placenta Data against several baseline methods including FEAT, DIH and DSM. We discuss the experimental results from various perspectives in the following sections.

\begin{figure}[t]
    \centering
    \includegraphics[width=0.6\textwidth]{Figures/mrecurve.png}
    \caption{Quantitative result comparison on mean residual error between HHEN and baseline methods.}
\label{fig:MRE}
\end{figure}

\begin{figure}[t]
    \centering
    \includegraphics[width=0.6\textwidth]{Figures/RMSECompare.png}
    \caption{Quantitative result comparison on average RMSE between HHEN and baseline methods.}
\label{fig:RMSE}
\end{figure}

\begin{figure}[t]
    \centering
    \includegraphics[width=0.6\textwidth]{Figures/APECompare.png}
    \caption{Quantitative result comparison on average photometric error between HHEN and baseline methods.}
\label{fig:APE}
\end{figure}


\subsection{Comparison with Hand-Crafted Methods}
\label{C4:S5:SS1}
We first compare the result between our proposed HHEN and FEAT. Note that FEAT  achieves the mosaicing based on the Speed Up Robust Features (SURF). SURF constructs a Hessian matrix to generate the point-of-interests on the edges. However, such kind of traditional-feature-based mosaicing methods fail on fetoscopic photography mosaicing due to the vessels are blurred and surrounded with turbid amniotic fluid. Compared with synthetic data (SYN1 and SYN2) which ignored these kinds of environmental interference (Table \ref{table:mosaicingdata}), the FEAT failed to achieve robust mosaicing on real data (TTTS1, INVT1, INVT2). As shown in Fig. \ref{fig:RMSE}, FEAT got relatively low RMSE on TTTS1, INVT1 and INVT2 with scores 8.13, 7.86, 7.11 respectively. However, the HHEN benefits from deep feature learning while PIG avoids the color restoration during training, which help HHEN outperforms the 
FEAT with a large margin (2.72 on TTTS1, 2.45 on INVT1, 1.61 on INVT2). The same result is also shown on the evaluation of APE (Fig. \ref{fig:APE}), while HHEN outperforms the FEAT on all clips. Besides, the MRE of FEAT generated mosaicing rise up after 150 frames while the HHEN continuously maintains the MRE at a relatively low level due to the non-local information learned from long-range frames. We can also observe the same phenomenon from the qualitative results shown in Fig. \ref{fig:SYNGT} as the FEAT mosaicing starts drifting away at 150 frames. 


\subsection{Comparison with State-of-The-Art Deep Learning Based Methods}
\label{C4:S5:SS2}

We also compared our proposed HHEN with two state-of-the-art image mosaicing methods, i.e. DIH and DSM. For MSE, DIH explodes very quickly as there are random rotation and translation during training data generation. DSM performs well on short frames due to it learns adjacent homography between adjacent frames. However, the error of DSM starts to accumulate after 300 frames and the mosaicing starts drifting (Fig. \ref{fig:SYNGT}) due to the local information is insufficient to achieve a stable mosaicing during long-range videos. In contrast, our proposed HHEN observed a continuous low MRE even for long-range frames that benefited from non-local information. HHEN also outperforms these two methods on both RMSE and APE. On RMSE, HHEN achieves 0.92, 1.33, 2.72, 2.45 and 1.61 on SYN1, SYN2, TTTS1, INVT1 and INVT2 respectively, while the other two methods obtained higher error (DIH: 3.08, 3.27, 3.86, 4.29, 4.41, DSN: 2.53, 2.17, 2.95, 2.08 3.53). On APE, HHEN also outperforms these two baseline methods on SYN1, SYN2, TTTS1, INVT1 and INVT2 (HHEN: 1.64, 1.76, 2.03, 2.28, 1.04, DIH: 3.27, 3.88, 2.91, 4.65, 3.74, DSM: 2.13, 1.86, 1.90, 2.85, 1.53). We also visualize the mosaicing result of testing sequence TTTS1, INVT1 and INVT2 in Fig. \ref{fig:testresult}. We can observe from Fig. \ref{fig:testresult} that our proposed HHEN with PIG generates the most meaningful mosaicing result even for unseen clip INVT1, which shows the high robustness of our method dealing with different fetoscopic photography scenarios.

\begin{figure}[t]
    \centering
    \includegraphics[width=\textwidth]{Figures/mosaicresult.png}
    \caption{Qualitative result visualization comparison on SYN1 and SYN2 dataset between HHEN and baseline methods. We highlight the mosaicing drift error with red ellipse. Best viewed in colors.}
\label{fig:SYNGT}
\end{figure}

\begin{figure}[t]
    \centering
    \includegraphics[width=\textwidth]{Figures/nogtresult.png}
    \caption{Qualitative result visualization on TTTS1, INVT1 and ONVT2 dataset using generated homography from HHEN. We can observe that HHEN generates meaningful and stable mosaicing even for unseen data INVT1. Best viewed in colors.}
\label{fig:testresult}
\end{figure}


\section{Discussion}
\label{C4:S6}
In this project, we utilize explicit deep relation learning on medical image mosaicing. Specifically, we propose a novel hierarchical homography estimation network to effectively learn the homography between adjacent frames that benefited from both local and non-local information. To further generate sufficient training data, we propose a novel image generation method called partially image generation which only perturbs the image with rotation, translation and color transformation. PIG can accelerate HHEN to learn relative homography while ignoring external interference such as color restoration. We conduct extensive experiments on 5 video data clips and the result shows our proposed method performs superior compared with state-of-the-art mosaicing methods. However, there are some blank spaces left for us to improve the mosaicing method. First, the whole network parameter is shared and updated across the data frames, which can be replaced by a recurrent neural network or a long-short term memory network to model the relationship between frames more effectively. Another future direction is to utilize the vessel segmentation mask as auxiliary information, which represents the saliency information, to achieve more accurate image mosaicing.


 
We provide some comments on the growth conditions which constituted the majority of our analysis in sections \ref{sec:Hmixing} and \ref{sec:Hsigma}. In the simplest cases of Lemma \ref{lemma:unstableGrowth}, growth was established in an analogous fashion to the old one-step expansion condition (\ref{eq:oldOneStepExpansion}), finding the relevant Jacobians $M_j$ and checking that their expansion factors $K(M_j)$ satisfy
\begin{equation}
    \label{eq:discussionOneStep}
    \sum_j \frac{1}{K(M_j)} <1.
\end{equation}
For the more complicated cases, the inductive method used to establish growth near the accumulation points in Lemma \ref{lemma:unstableGrowth} and the weakened one-step expansion condition (\ref{eq:oneStep}) both address the same fundamental issue: the splitting of unstable curves by singularities into an unbounded number of small components. They circumvent this obstacle in rather different ways, however. While (\ref{eq:oneStep}) generalises (\ref{eq:discussionOneStep}) to ensure an growth of unstable curves `on average' (see \cite{chernov_statistical_2009} for a precise statement), our inductive method is a more direct adaptation of (\ref{eq:discussionOneStep}), using it to generate contradictory geometric conditions which a hypothetical non-growing unstable curve must satisfy. It may be possible to prove Theorem \ref{sec:Hmixing} using (\ref{eq:oneStep}) as the basis for growth. Since we required (\ref{eq:oneStep}) anyway for proving Theorem \ref{thm:HsigmaExp}, this could potentially condense our analysis, but only to a minor extent. A convenience of the method used in section \ref{sec:Hmixing} is that, by way of the `simple intersection' property, it naturally gives geometric information on the images of manifolds, useful for proving the property \textbf{(M)} of Theorem \ref{thm:katok-strelcyn}.

We expect that essentially analogous analysis can be applied to establish mixing properties in a wide class of piecewise linear non-uniformly hyperbolic maps, including those (like the OTM) which sit on the boundary of ergodicity and beyond. While we have relied on the precise partition structure of $H_\sigma$, its fundamental feature (self-similar sequences of elements $A^k$, sharing boundaries with its neighbours $A^{k-1},A^{k+1}$ and accumulating onto some point $p$) is quite typical to return map systems. See, for example, those of various stadium billiards \cite{chernov_chaotic_2006,chernov_improved_2008,chernov_statistical_2009} and LTMs \cite{springham_polynomial_2014}. Indeed, the same method can be used to prove the Bernoulli property for non-monotonic LTMs \cite{myers_hill_mixing_2022}, where monotonicity of the manifold images cannot be assumed and the classical argument \cite{sturman_mathematical_2006} fails. The OTM is the pointwise limit of these maps as the boundary shrinks to null measure. It further has utility in proving growth conditions for maps which are uniformly hyperbolic but possess regions $A_j$ where the hyperbolicity is very weak, signified by $K(M_j) \approx 1$, so that (\ref{eq:discussionOneStep}) fails. Typically this leads to suboptimal bounds on mixing windows, see e.g. \cite{wojtkowski_model_1981,przytycki_ergodicity_1983,myers_hill_family_2022}. The map $H_{(\eta,\eta)}$ for $\eta \approx 1/2$ is another example, possessing weak hyperbolicity over $A_2, A_3$. Letting $\varepsilon = |\eta-1/2|>0$, there is an upper bound $N = N(\varepsilon)$ on escape times from the intersections $A_2\cap \sigma, A_3 \cap \sigma$. The growth lemma then follows by applying the inductive step roughly $N$ times and can be established for arbitrarily small $\varepsilon$, opening the door to establishing optimal mixing windows.

The above gives two examples of piecewise linear perturbations to $H$ where mixing with respect to Lebesgue is preserved and our methods can be applied. Nonlinear perturbations to the shear profiles complicate the analysis in several ways. Firstly as the map's Jacobians takes on a broader range of values, cone invariance becomes an increasingly harder condition to establish. Cones must be widened, giving looser bounds on expansion factors, which may already be weak due to new regions of weaker stretching. This, together with the change from polygonal to curvilinear return time partition elements and nonlinear local manifolds, adds some complexity to showing growth conditions. This does not rule out certain (small) nonlinear perturbations however. There is some leeway in the inequalities which govern cone invariance and growth of local manifolds, the latter of which is not too dissimilar from the piecewise linear setting (see Lemmas \ref{lemma:piecewiseApprox}, \ref{lemma:componentLength}). Certain small perturbations would not alter the \emph{topological} structure of the return time partition, i.e. which elements share boundaries, the key information needed for setting up the induction. Finally while the partition elements would no longer be polygonal, only coarse geometric information is required for verifying each inductive step. Following the above, a potential perturbation could be to replace the linear portions of each shear by a cubic, perturbing the tent profile
\[  f(t) = \begin{cases} 2t & 0 \leq t \leq 1/2, \\ 2(1-t) & 1/2 \leq t \leq 1 ,\end{cases} \]
of the OTM shears to
\[  f_a(t) = \begin{cases} \frac{1}{8} t \left(16 - a + 6at - 8at^{2} \right) & 0 \leq t \leq 1/2, \\ \frac{1}{8}\left(1-t\right)\left( 16 - a + 6a\left(1-t\right) - 8a\left(1-t\right)^{2}\right)  & 1/2 \leq t \leq 1, \end{cases}   \]
for $a>0$. For small enough $a$ the gradient range $f'(t)$ is restricted to small neighbourhoods of $\{ 2, -2\}$ and the escape time partition retains a similar structure. We illustrate this in Figure \ref{fig:perturbations}, showing escapes from the square $S_3$ under the map $G \circ F$, equivalent to escapes from the perturbed $A_3$ under the $G \circ F$, but with a cleaner geometry for comparison. When $a$ is too large the analogy to the OTM breaks down. At $a=16$ the map is twice differentiable everywhere and features a new source of slowed mixing, the Jacobian is the identity at the corner points $x,y \in \{  0, 1/2 \}$ giving locally parabolic behaviour (visible in the escape time partition). 

\begin{figure}
    \centering
    \includegraphics[width=0.24 \linewidth]{0.png}
    \includegraphics[width=0.24 \linewidth]{4.png}
    \includegraphics[width=0.24 \linewidth]{8.png}
    \includegraphics[width=0.24 \linewidth]{16.png}
    \caption{Partition of escape times from $S_3$ under the mapping $F \circ G$ for $a= 0,4,8,16$. }
    \label{fig:perturbations}
\end{figure}
%\section{Conclusion}\label{sec:conclusion}
In this work, we focus on addressing the fundamental challenge of OOD detection tasks, which is how to fully understand the semantic discrepancy between the ID/OOD samples. We reveal that the key to success in the realistic SCOOD task is to allocate as many ID samples in the unlabeled set correctly as possible. To this end, we propose a novel uncertainty-aware optimal transport scheme that introduces class-specific energy scores as guidance for effective label assignment. Experimental results show that our method achieves better performance than previous state-of-the-art methods on SCOOD benchmarks.

\textbf{Limitations.} In addition to temperature scaling, other techniques such as feature clipping applied in ReAct~\cite{sun2021react} also enhance the performance of energy score, so how to obtain an OOD score that best fits the SCOOD task can be further explored. Moreover, a setting highly related to SCOOD has been proposed in \cite{katz2022training} and formulated as a constrained optimization problem. We will also theoretically analyze these practical OOD settings in our feature work.

% \section*{Acknowledgments}
\textbf{Acknowledgments.} 
This work is supported by National Key R\&D Program of China under Grant 2020AAA0105701, National Natural Science Foundation of China (NSFC) under Grants 61872327, Major Special Science and Technology Project of Anhui, National Natural Science Foundation of China (62033012) and Ant Group through Ant Research Intern Program.
 

%----------------------------------------------------------------------------------------
%	THESIS CONTENT - APPENDICES
%----------------------------------------------------------------------------------------

\appendix % Cue to tell LaTeX that the following "chapters" are Appendices

% Include the appendices of the thesis as separate files from the Appendices folder
% Uncomment the lines as you write the Appendices

%% Appendix A

\chapter{Topological relations between  parameters of a CDT triangulation} % Main appendix title

\label{AppendixA} 


The following list sums up the topological relations valid  for any CDT triangulation. For the definition of the $A,B,C,D$ and $E$ parameters see Chapter \ref{chapter3}. Note, that for simpler notation in the appendix, contrary to the main text, we use a convention of {\it global} numbers which distinguishes between the number of $s_{41}$  and  $s_{14}$ simplices, denoted $N_{41}$ and $N_{14}$, respectively. Similarly, we distinguish between $N_{32}$ and $N_{23}$.

\begin{enumerate}
    \item[$T_{1}$.:] $2A_1 + C_1 + E = 5 \cdot N_{41}$
    \item[$T_{2}$.:] $C_1 + 2B_{1a} + 2B_{2a} + D = 5 \cdot N_{32}$
    \item[$T_{3}$.:] $C_2 + 2B_{2a} + 2B_{2b} + D = 5 \cdot N_{23}$
    \item[$T_{4}$.:] $2A_2 + C_2 + E = 5\cdot N_{14}$
    \item[$T_{5}$.:] $2A_1 + C_1 = 2A_2 + C_2 = 2(N_{41}+N_{14})$
    \item[$T_{6}$.:] $2B_{1b} + D = 3\cdot N_{32}$
    \item[$T_{7}$.:] $2B_{2b} + D = 3\cdot N_{23}$
    \item[$T_{8}$.:] $2B_{1a} + C_1 = 2\cdot N_{32}$
    \item[$T_{9}$.:] $2B_{2a} + C_2 = 2\cdot N_{23}$
    \item[$T_{10}$.:] $(A + B + C + D + E) = N_3 = \frac{5}{2}N_4$ 
\end{enumerate}

A triangulation can be characterized by  the following global parameters, referring to the number of (sub-) simplices of various types,  $N_{10}, N_{20}, N_{11},$ $N_{30}, N_{21},$ $N_{12}, N_{40}, N_{31}, N_{13}, N_{22}, N_{41}, N_{32}, N_{23}, N_{14}, \chi$, where the first number in the subscript denotes the number of vertices in the spatial  slice $t$ and the second one is the number of vertices in $t+1$, and $\chi$ is the Euler characteristics related to the fixed spatial topology. These global numbers can be joined using the seven Dehn-Sommerville relations \cite{nonperturb}:

\begin{itemize}
\item[$DS_{1}$.:] $N_{40} = N_{41} = \frac{1}{2}(N_{41}+ N_{14})$
\item[$DS_{2}$.:] $N_{30} = 2N_{40} = (N_{41}+ N_{14})$
\item[$DS_{3}$.:] $N_4 = \frac{2}{5}(N_{40}+N_{31}+N_{13}+N_{22})$
\item[$DS_{4}$.:] $N_{10}-N_{20}+N_{30}-N_{40} = 0$
\item[$DS_{5}$.:] $N_{22} = \frac{3}{2}(N_{32}+N_{23})$
\item[$DS_{6}$.:] $2N_1 -3N_2 +4N_3 -5N_4 = 0$
\item[$DS_{7}$.:] $N_0 - N_1 + N_2 - N_3 +N_4 = \chi$ 
\end{itemize}
Using the "$T$" relations: 

\begin{equation}
(N_{32} + N_{23}) = \frac{2}{5}B + \frac{2}{5}D + \frac{1}{5}C = \frac{2}{3}B_b + \frac{2}{3}D = B_a + \frac{1}{2}C = \frac{2}{3} N_{22}, 
\end{equation}
and from this it follows, that $D$ can be expressed as:

\begin{equation}
D = \frac{3}{2}B_a - B_b + \frac{3}{4}C.
\end{equation}
Similarly, one can express the other relations for the two 4-dimensional simplices, and using "$DS$" relations one obtains :

\begin{equation}
(N_{41}+N_{14}) = \frac{1}{2}A + \frac{1}{4}C = N_{30} = 2 N_{40}.
\end{equation}
It also follows that:

\begin{equation}
E = \frac{1}{2}A + \frac{1}{4}C.
\end{equation}
Using $DS_3$ one can find the relations fulfilled by the %first type of 
time-like tetrahedra:

\begin{equation}
N_4 = (N_{41}+N_{14})+(N_{32}+N_{23}) = \frac{2}{5}(N_{40} + N_{31} + N_{22} + N_{13}),
\end{equation}
leading to

\begin{equation}
(N_{31}+N_{13}) = 2(N_{41}+N_{14}) + (N_{32}+N_{23}) = A + B_a + C.
\end{equation}
The formula for the spatial links can be expressed with the help of $DS_4$:

\begin{equation}
N_{20} = N_{10} + \frac{1}{2}(N_{41}+N_{14}) = N_{10} + \frac{1}{4}A + \frac{1}{8}C.
\end{equation}
The remaining numbers $N_{11}$ and $(N_{21}+N_{12})$ are calculated in a bit more involved way. Taking $DS_6$ we can express the total number of time-like links as:

\begin{equation}
N_{11} = \frac{3}{2}(N_{30}+N_{21}+N_{12}) -\frac{3}{2}A -\frac{5}{2}B_a -2C -N_0,
\end{equation}
which involves the number of time-like triangles. Using $DS_7$ one can find the following relation:

\begin{equation}
\chi = N_0 - \frac{1}{2}(N_{30}+N_{21}+N_{12})+N_4,
\end{equation}
which leads to the expression for the time-like triangles:

\begin{equation}
(N_{21}+N_{12}) = 2N_0 -2\chi +\frac{1}{2}A + 2B_a +\frac{3}{2}C,
\end{equation}
which now can be used in the previous equation to get the number of the time-like links:

\begin{equation}
N_{11} = 2N_0 -3\chi +\frac{1}{2}B_a +\frac{1}{4}C.
\end{equation}

With the above mentioned relations one can check, that for any CDT triangulation there are 8 independent parameters, which are enough to compute all other global parameters. For example, one can choose the following set of independent parameters

\begin{equation}
Set_R = \{ N_0, \chi, A_1, A_2, B_{1a}, B_{2a},C_1, C_2 \}.
\end{equation}
One can as well use the following  set, including the currently used global numbers $N_0$, $N_{41}$ and $N_{32}$ appearing in the CDT  action:

\begin{equation}
Set_G = \{ N_0, \chi, N_{41}, N_{32}, N_{23}, C_1, C_2, D\}.
\end{equation}

These new parameters can be used not only as order parameters, but also they can be potentially used to extend the CDT action, see eq. (\ref{eq:ation_kappa}), to the following form 

\begin{equation}
    S_{CDT}^{ext} = -(\kappa_0 + 6\Delta) N_0 + \kappa_4 (N_{41} + N_{32}) + \Delta  N_{41} + \kappa_C C + \kappa_D D,
\end{equation}
where $\kappa_C$ and $\kappa_D$ are the new coupling constants related to the $C$ and $D$ parameters, respectively. The physical meaning of these parameters and the related coupling constants is not straightforward and a discussion of it will not be a part of this thesis.






%##########################################################
%##########################################################
%##########################################################
%##########################################################
%\chapter{Appendix of Chapter 4}
\label{AppendixB}
 % Main appendix title
\section{Depression users posting analysis on CLPSych dataset}
\begin{figure*}[htbp]
\centering
\begin{minipage}{\textwidth}
\includegraphics[width=1\textwidth]{Figures/ttdd_tsne_CLP.png}

\label{fig:ttdd_tsne_CLP}
\end{minipage}
\caption{Document representations distribution using TSNE on CLPsych dataset. (a) Top left: Document vectors learned from LDA. (2)Top Right: Document vectors learned from DocNADE. (3) Bottom Left: Document vectors learned from fdp-DocNADE. (4) Bottom Right: Document vectors learned from fdp-DocNADEa.}
\end{figure*}
 
 \begin{figure*}
\centering
\includegraphics[width=0.85\textwidth]{Figures/wordcloud_topics_CLP.png}
\caption{Top probability words in four topics on CLPcych set with manual defined 'theme'}
\label{fig:word_cloud_CLP}
\end{figure*}


Latent Dirichlet Allocation (LDA) is an classical topic model which is shown in  \figurename{\ref{fig:lda_model}. In \figurename{\ref{fig:lda_model}, a dirichlet distribution $\alpha$ represent a particular document and this topic distribution is $\theta$. A particular topic $Z$ can be selected from $\theta$, then the word distribution $\varphi$ of topic $Z$ will be randomly selected from second dirichlet distribution $\beta$.  $N$ is a generated word from $\varphi$ with topic $Z$. The following procedure is to use Gibbs sampling to maximize the loglikehood of $p(W|Z,\varphi)$. In brief, LDA topic model is a clustering process which concentrates on word frequency and word distribution in corpus. Optimized $\theta$ can be represented as document vectors for classification task and word distribution $\varphi$ can be used to describe document content.
\begin{figure*}[htbp]
\centering
\includegraphics[scale=0.5]{Figures/lda_png.png} 
\caption{Latent Dirichlet Allocation (LDA). Image Courtesy of \cite{blei2003latent}} 
\label{fig:lda_model} 
\end{figure*}
%\section{Appendix C - Preparation and positioning of Nanodiamonds using Pick-and-Place}
In preparation of transferring SiV hosting NDs to the bullseye antenna, a dispersion of SiV containing NDs was drop-casted onto a fused silica coverslip.
The NDs were fabricated using high-pressure high temperature (HPHT) treatment of the catalyst metals-free hydrocarbon growth system. The NDs were
treated with \ch{HNO3 + HClO4 + H2SO4} and HF to remove sp2 carbon and were
washed and dried afterwards. FIB milled marker structures on the fused silica coverslip serve as a position reference to later locate specific SiV-containing NDs
with an AFM. To find SiV-hosting NDs the sample is examined using a homebuilt confocal microscope, orchestrated based on the open-source software qudi \cite{Binder2017Qudi:Processing}. A 2D galvo scanning mirror, a 4f-system and a 1.35 NA oil objective form the basis of this optical setup. Narrowband filtering (740 ± 13 nm)
around the ZPL of the SiV increases the signal to noise ratio when exciting offresonantly with 532 nm, as the dominant emission originates from SiV centers
within this filter window. After suitable NDs have been located, AFM imaging
of the area of interest is carried out. Triangulation of the ND position is done
with the help of the FIB markers, as well as positions of other fluorescing NDs
in the confocal image. In order to pick and place \cite{Schell2011ADevices} the ND of interest, a platin-coated AFM cantilever is approached with a constant force until the ND attaches to the tip. The pick-up is indicated by either the disappearing ND in a subsequent non-contact AFM scan or by image artifacts such as ’double-tip’ features. Those features also hint towards a successful placement strategy of the ND in the following step. With the ND attached to the cantilever tip, the sample is exchanged for the target sample, here the bullseye antenna. A careful non-contact imaging reveals the target position on the structure. For the antenna structure considered herein the central hole in the structure eases the placement of the ND due to its topology. A contrast in height is useful to detach the ND from
the cantilever. After successful ND placement (compare Fig. \ref{fig:C}b), the sample is again examined in the confocal microscope, with the objective being exchanged to 0.55 NA for the sake of a longer working distance and to leverage the collection efficiency \cite{Waltrich2021High-purityAntenna}. The verification of a successful SiV hosting ND placement is seen in Fig. \ref{fig:C}c.

\label{AppendixC}

%----------------------------------------------------------------------------------------
%	BIBLIOGRAPHY
%----------------------------------------------------------------------------------------

\printbibliography[heading=bibintoc]

%----------------------------------------------------------------------------------------

\end{document}  
