
%\chapter{Conclusion and Future works} % Main chapter title
%In this thesis, our main research goal is to establish an automatic Online depression detection system which can judge whether a social network user have trend of suffering depression.  We conclude our works of previous chapters below:
%\begin{enumerate}[(1)]
%\item Chapter\ref{Chapter1} - \textbf{Introduction} We present the background of detection depression users via social networks in first section. Then we briefly demonstrate the weakness of previous Social networks depression detection framework and present our research objectives. The main research contributions is summarized in Section 1.3 and outline of thesis is shown in Section 1.4. 
%\item Chapter\ref{Chapter2} - \textbf{Literature Review} We provide some literatures to confirm the value of detecting depression sufferers online. And previous online depression detection methods are divide into two classes according to using different features extraction techniques, statistical method based approaches and NLP based approaches. We introduce the strength and weakness of existing online depression detection frameworks.
%\item Chapter\ref{Chapter3} - \textbf{Inverse boosting pruning trees for depression detection on Twitter} We presented an novel classifier Inverse boosting pruning trees (IBPT) which combine pruning trees with an inverse boosting structure to classify non-depressed and depression users. The IBPT outperforms several baselines against two depression database, such as TTDD and CLPsych. In the meantime, we verified the convergence of our algorithm IBPT through rigorous theoretical analysis with comprehensive experiments. Moreover, we utilise three UCI datasets to evaluate the classification ability of our method comprehensively, which shows our method outperforms the other baselines. We then analysed the feature importance of the IBPT and described the  difference of the online behaviours of the non-depressed and depression classes. Finally, we used different combinations of the feature categories to confirm the effectiveness of the three feature categories for depression detection. 
%
%\item Chapter\ref{Chapter4} - \textbf{Deep Neural Topic Model for Tweet content analysis} Inspired by the result of feature importance in Chapter \ref{Chapter3}, we observed that extracted features from topic model are crucial for depressed users classification. In this chapter, we try to make some contribution on topic model. Here, we introduced Fasttext word priors and attention mechanism into DocNADE to improve the model clustering and generalisation ability. Our proposed model FastText distributional prior DocNADE with attention mechanism (fdp-DocNADEa) outperform other methods on the evaluation of model perplexity and show strong performance on producing coherent topics and representative document vectors. Finally, we employ fdp-DocNADEa to analysis the typical topics of user's posting content on two Twitter datasets and combine fdp-DocNADEa with our proposed classifier IBPT to confirm the effectiveness of presented approaches for Twitter depression users classification.
%\item Chapter\ref{Chapter5} - \textbf{Discussion} we discuss some weakness of our proposed two techniques (e.g. IBPT, fdp-DocNADEa). And we explain the limitations of present research experiment setup.
%\end{enumerate}
%
%
%Lastly, we have some ideas for further working on online depression detection in the future: (1) attempt to mine information from other source social networks, e.g. Facebook, Instagram, and Tumblr. We want to verify the effectiveness of our propose system on different social networks datasets. (2) At present, we remove emojis or pictures when we analysis tweet content. In some way, emojis or pictures also contain some information about users' emotion, we expect to explore these information by developing  methods which combing computer vision and natural language processing techniques.
%
%\label{Chapter6}