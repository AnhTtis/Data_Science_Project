%\RequirePackage[2020-02-02]{latexrelease}
%\documentclass[aps,pra, twocolumn, noshowpacs, floatfix]{revtex4}

\documentclass[aps,pra, twocolumn, noshowpacs, floatfix]{revtex4}

\usepackage{}
\usepackage{amsfonts}
\usepackage{amssymb}
\usepackage{graphicx}
\usepackage{amsmath}
\usepackage{amsmath}
\usepackage{braket}
\usepackage[english]{babel}
\usepackage[mathscr]{euscript}
\usepackage{color}
\usepackage{epstopdf}
\usepackage{footnote}
\usepackage{isotope}
\usepackage{ulem}

\usepackage[export]{adjustbox}

\begin{document}
\title{Dynamics of dissipative Landau-Zener transitions in an anisotropic three-level system}

\author{Lixing Zhang$^{1}$, Lu Wang$^{2}$, Maxim F. Gelin$^{3}$ and Yang Zhao$^{1}$\footnote{Electronic address:~\url{YZhao@ntu.edu.sg}}}

\affiliation{$^{1}$\mbox{School of Materials Science and Engineering, Nanyang Technological University, Singapore 639798, Singapore}\\
$^{2}$\mbox{School of Science, Inner Mongolia University of Science and Technology, Inner Mongolia 014010, China} \\
$^{3}$\mbox{School of Science, Hangzhou Dianzi University, Hangzhou 310018, China}\\
}

\begin{abstract}
We investigate the  dynamics of Landau-Zener transitions in an anisotropic, dissipative three-level model (3-LZM) using the numerically accurate multiple Davydov $\mathrm{D}_{2}$ Ansatz in the framework of time-dependent variation.
It is demonstrated that a non-monotonic relationship exists between the Landau-Zener transition probability and the phonon coupling strength when the 3-LZM is driven by a linear external field. Under the influence of a periodic driving field, phonon coupling may induce peaks in contour plots of the transition probability when the magnitude of the system anisotropy matches the phonon frequency. Dynamics of the 3-LZM have also been probed in the presence of a super-ohmic phonon bath when driven by a periodic driving field. It is found that both the period and the amplitude of the Rabi cycle decay exponentially with the increasing bath coupling strength.
\end{abstract}

\date{\today}
\maketitle

\section{introduction}

The model of a two-level quantum system possessing an avoided crossing  driven by an external field was proposed by Landau \cite{Landau} and Zener  \cite{Zener} in 1932. This extraordinary versatile model is commonly referred to as the Landau-Zener (LZ) model. It is widely used in molecular physics\cite{m_ph1,m_ph2}, quantum optics\cite{LZ-cavity1}, chemical physics\cite{chem_physics1}, and quantum information processing \cite{QIP1, QIP2} to describe such diverse systems as molecular nanomagnets \cite{nanomagnet}, Bose-Einstein condensates (BECs)\cite{BEC}, quantum annealers \cite{quenching}, nitrogen vacancy (NV) centers in diamonds \cite{NV1,NV2}, and quantum dots in semiconductors \cite{QDot1,QDot2} (see Ref.~\cite{LZ_review} for a recent review).
The multilevel variant of the LZ problem has also gained considerable attention \cite{LZ_review}. In multilevel LZ systems,  transitions between several energy levels can occur simultaneously, adding greatly to the complexity of the system dynamics. Nevertheless, Chernyak, Sinitsin and their coworkers developed classification of the exactly solvable multilevel LZ Hamiltonians \cite{LZ-multi1,Volodya21}.  Kolovsky and Maksimov studied diabatic and adiabatic regimes of multilevel LZ transitions in the optically tilted Bose-Hubbard model \cite{Andrey}. Band and Avishai developed an analytical solution to a three-level LZ model with asymmetric tunneling between the states \cite{band}. Multilevel LZ models found applications in a variety of problems in physics and chemistry, such as BECs in multi-well traps \cite{LZ-BEC-well_trap}, trapped atomic gases \cite{at_gas}, triple quantum dots \cite{TQD1, TQD2},  large spin systems (e.g. NV centers  in diamonds) \cite{NV1,NV2} and Fe$_8$ molecular nanomagnets \cite{nanomagnet}.

A notable extension of the LZ model is achieved through the coupling of the bare LZ system to a dissipative environment.
Two-level LZ systems diagonally and off-diagonally coupled to phonon baths have been extensively studied  \cite{LZ_review}. It was shown, for example,  that  phonon coupling may  assist  LZ transitions and create entanglement between qubits \cite{D2-dw3,LZ-periodic2, LZ-entanglement1,LZ-entanglement2}.
Saito, Wurb and their coworkers derived exact analytical formula for the transition probability in the dissipative LZ model \cite{bath1,bath2} which was  confirmed by accurate numerical calculations \cite{YZ19}.
Nalbach and Thorwart gave numerically-exact path-integral simulations of the dissipative LZ dynamics \cite{Thorwart09,Thorwart15}.
As for multilevel LZ models, the studies of  dissipative effects are scarce. Mention the work of
Saito {\it et al.} who investigated dissipative LZ transitions by employing diabatic representation of the LZ Hamiltonian \cite{dibatic_basis}.
In this work, we study the dynamics of a dissipative  three-level LZ model (hereafter, 3-LZM)
by considering an anisotropic 3-LZM, similar to that in Ref.~\cite{Core-kiselev}, in which degeneracy between the three energy levels is lifted (without the  anisotropy term, the 3-LZM is a direct extension of the original two-level LZ model \cite{extention1, extention2}. Anisotropic 3-LZMs are commonly used to simulate NV centers  in diamonds \cite{NV_review1} and triple quantum dots \cite{TQD1, TQD2},  quantum computers  \cite{NV-computing1, NV-computing2, NV-computing3}, stress sensors \cite{NV-stress}, quantum cellular automata \cite{TQD-automata}, and charging rectifiers \cite{TQD-rectifier}.

Evaluation of the  dissipative 3-LZM dynamics is computationally challenging. In this work, we tackle this problem by invoking the numerically accurate method of the multiple Davydov D$_2$ (multi-D$_2$) ansatz, which is a member of the family of Davydov ans\"atze \cite{D2-review}. This method was demonstrated to be an efficient and reliable tool for obtaining time evolutions of photon-assisted LZ models \cite{LZ-periodic2}, the Holstein-Tavis-Cummings (HTC) model \cite{D2-HTC,D2-TC}, the spin-boson model (SBM) and its variants\cite{D2-SBM1,D2-SBM2}, and cavity quantum electrodynamics (QED)  models \cite{D2-SF}.

The remainder of the paper is arranged as follows: In Sec.~II, we introduce the theoretical framework of the multi-D$_2$ ansatz, as well as define the observables used in this paper. In Sec.~III, we present and discuss our simulation results of the dissipative 3-LZM dynamics. Sec.~IV is the Conclusion. Technical derivations can be found in Appendix A.






\section{METHODOLOGY}


\subsection{The model}
The bare 3-LZM Hamiltonian can be written as
\begin{eqnarray}\label{EQ1}
\hat{H}_{{\rm 3} - {\rm LZM}}& =& \mathcal{D}(S_z^2-\frac{2}{3}I)+ \Omega_z(t) S_z+\Omega_x(t) S_x
\end{eqnarray}
($\hbar$=$1$) where  $S_{x}$ and $S_z$ are the spin 1 operators. The zero-field splitting (ZFS) tensor $\mathcal{D}(S_z^2-\frac{2}{3}I)$ is responsible for the system anisotropy and describes the level splitting without  external fields. The  $-2I/3$ term makes $\mathcal{D}$ traceless and restores the SU(3) symmetry of the Hamiltonian. $\Omega_z(t)$ and $\Omega_x(t)$ are  time-dependent  driving fields in the $z$ and $x$ directions, which can change either linearly or periodically with time.

The $\hat{H}_{\rm 3-LZM}$ Hamiltonian is linearly coupled  to a phonon bath which assists transitions between the spin states, create complex interference patterns and drives the system towards certain steady states.  The spin-phonon coupling, together with the phonon bath, are described by the Hamiltonian $\hat{H}_{\rm sp}$:
\begin{eqnarray}\label{EQ2}
\hat{H}_{\rm sp}& =& \sum_{k}\omega_{ k} \hat{b}_{ k}^{\dagger}\hat{b}_{k}+\sum_{k}(\eta_z^k S_z+\eta_x^k S_x)(\hat{b}_{ k}^{\dagger}+\hat{b}_{k})
\end{eqnarray}
Here $\hat{b}_{ k}^{\dagger}$($\hat{b}_{ k}$) is the annihilation (creation) operator of the $k$th phonon mode, while  $\omega_k$, $\eta_z^k$ and $\eta_x^k$ are the frequency, diagonal coupling, and off-diagonal coupling strengths of the $k$th mode, respectively.

If $\hat{H}_{\rm sp}$ contains only one phonon mode, we drop the subscript $k$ and denote $\eta_z^k = \eta_z$, $\eta_x^k= \eta_x$, and $\omega_k = \omega_p$. If multiple modes are present, the  phonon bath is specified by the spectral density
\begin{equation}\label{EQ3}
J(\omega )=\sum_{k}(\eta_z^{k})^{2}\delta(\omega -\omega _{k})=2\alpha \omega_{c}^{1-s}\omega^s e^{-\omega/\omega_c}
\end{equation}
where $\alpha$ is the dimensionless coupling strength and $s$ is the exponent. When $s = 1$, the bath is Ohmic, while $s > 1$ ($s < 1$) correspond to  super-Ohmic (sub-Ohmic) bath.  Following Ref.~\cite{Core-spec_den}, we choose a super-Ohmic bath with $s = 3$ in our simulations. $\omega_{c}$ is the cut-off frequency, the coupling strength of the phonon modes with $\omega > \omega_{c}$ will decrease drastically.

In order to employ the variational Davydov Ansatz method, the continuum spectral density has to be discretized. Following  Ref.~{\cite{WFCZ}}, we employ the density function of the phonon modes ${\mathcal{P}}(\omega)$, which is defined on the interval $\omega \in [0, \omega_m]$, where $\omega_m$ is the upper bound of the phonon frequencies and
\begin{equation}\label{EQ4}
\int_{0}^{\omega_k}{\mathcal{P}}(\omega) d\omega = k, k = 1, 2, ..., N_b,
\end{equation}
where $N_b$ is the total number of discrete modes in the Hamiltonian and  $\omega_{N_b} = \omega_m$. In this case, the coupling strength of the individual phonon mode is given by the expression  $\eta_z^k= \sqrt{J(\omega_k)/\mathcal{P}(\omega_k))}$. In this work, we adopt the efficient
\begin{equation}\label{EQ5}
{\mathcal{P}}(\omega) = \frac{1}{\mathcal{N}}\frac{J(\omega)}{\omega}.
\end{equation}
Here
\begin{equation}\label{EQ6}
	{\mathcal{N}} = \frac{1}{N_b}\int_0^{\omega_m}\frac{J(\omega)}{\omega}d\omega,
\end{equation}
is the normalization factor that ensures $\int_{0}^{\omega_m}{\mathcal{P}}(\omega) d\omega = N_b$.
In the case of super-Ohmic bath with $s=3$, $\omega_k$ can only be obtained implicitly through Eq.~(\ref{EQ4}).


\subsection{The multi-D$_2$ Ansatz}
First proposed in the 1970s in the study of energy transportation in protein molecules as an approximation of the solution to the Schr\"{o}dinger equation~\cite{Davydov1}, the Davydov Ansatz has two variants, the Davydov D$_1$ Ansatz and D$_2$ Ansatz, with the latter being a simplification of the former~\cite{Davydov2, Davydov4}. Inspired by numerically ``exact" solutions to a two-site problem with short-range coupling to Einstein phonons \cite{shor73}, Shore and Sander pioneered
the multiple Davydov ans\"{a}tze when they
experimented with a trial wave function with two Gaussians as its phonon component, an early forerunner of the multiple Davydov ans\"{a}tze.
In a further development of this method, the multiple Davydov ans\"{a}tze have been applied extensively to a variety of many-body problems in physics and chemistry by Zhao and co-workers \cite{jcppersp,Davydov5,Davydov6,Davydov7,Davydov8,QED_Huang,QED_Zheng,TMD}. The multiple Davydov ans\"{a}tze also have two variants, multi-D$_1$ Ansatz and multi-D$_2$ Ansatz. In this work, the multi-D$_2$ Ansatz is used:
\begin{align}\label{EQ7}
\ket{{\rm D}_{2}^{M}(t)}=\sum_{n=1}^{M}{(A_n(t){\ket{-1}}+B_n(t){\ket{0}}+C_n(t){|1\rangle})\otimes}|f\rangle_{\rm ph}^n
\end{align}
where $M$ is the multiplicity of the Ansatz, and ${\ket{-1}}$, ${|0\rangle}$ and ${|1\rangle}$ are the three spin states in the 3-LZM, for which we assign amplitudes $A_n(t)$, $B_n(t)$ and $C_n(t)$, respectively. $|f\rangle_{\rm ph}^n$ is the phonon coherent state, which can be written as:
\begin{eqnarray}\label{EQ8}
|f\rangle_{\rm ph}^n= {\rm exp}\big[{\sum_{k} {f}_{nk}\left(t\right)\hat{b}^{\dagger}-\rm H.c.}\big]|0\rangle_{\rm ph}
\end{eqnarray}
Here, $|0\rangle_{\rm ph}$ is the phonon vacuum state, H.c.~denotes for the Hermitian conjugate, and $f_{nk} (t)$ is the phonon displacement for the $k$th mode of phonon. $A_n(t)$, $B_n(t)$, $C_n(t)$, and $f_{nk}(t)$ are referred to as the time-dependent variational parameters that define the multi-D$_2$ trial state.

\subsection{The time-dependent variational principle}
To arrive at the time-dependent variational parameters, the Euler-Lagrangian equation is solved under the framework of the time-dependent variational principle:
\begin{align}\label{EQ9}
\frac{d}{dt}\frac{\partial L}{\partial{\dot{u}}_n^\ast}-\frac{\partial L}{\partial u_n^\ast}=0
\end{align}
where $u_n$ are the time-dependent variational parameters, $u_n^\ast$ stands for the complex conjugate of $u_n$, and the Lagrangian is given by:
\begin{eqnarray}\label{EQ10}
L&=&\frac{i}{2}\left[\langle {\rm D}_{2}^{M}(t)|\frac{\overrightarrow{\partial}}{\partial t}|{\rm D}_{2}^{M}(t)\rangle
-\langle {\rm D}_{2}^{M}(t)|\frac{\overleftarrow{\partial}}{\partial t}|{\rm D}_{2}^{M}(t)\rangle\right]\nonumber\\&&-\langle{\rm D}_{2}^{M}(t)|\hat{H}_{{\rm 3} - {\rm LZM}}+\hat{H}_{\rm sp}|{\rm D}_{2}^{M}(t)\rangle.
\end{eqnarray}
By solving the Euler-Lagrangian equation, a series of differential equations that govern the time evolution of the variational parameters $\mu_n$ are obtained. These equations are referred to as the equation of motions (EOMs). To solve the EOMs simultaneously, the $4^{\rm th}$-order Runge-Kutta method (RK4) is adopted. The detailed deduction for EOMs can be found in Appendix \ref{Appendix A}. To avoid singularity in RK4 iterations, a random noise is added to the initial variational parameters at $t=0$ within a range of $[{-10}^{-4}, {10}^{-4}]$.

\subsection{Observables}

Following Eq.~(\ref{EQ7}), the normalization factor of the Ansatz state can be calculated:
\begin{eqnarray}\label{EQ11}
N(t) & = & \langle{\rm D}_{2}^{M}(t)|{\rm D}_{2}^{M}(t)\rangle\nonumber\\
& = & \sum_{m,n}^{M}{\left({A_m}^\ast{A_n}+{B_m}^\ast{B_n}+{C_m}^\ast{C_n}\right)\cdot S_{mn}}
\end{eqnarray}
with $S_{mn}$ to be the well-known Debye-Waller factor in the form of  \cite{D2-dw1, D2-dw2, D2-dw4, D2-dw5}
\begin{eqnarray}\label{EQ12}
S_{mn}={\rm exp}\big[\sum_{k}\big({-\frac{1}{2}{f_{mk}^\ast f_{mk}-\frac{1}{2}f_{nk}^\ast f_{nk}+}f_{mk}^\ast f_{nk}}\big)\big]
\end{eqnarray}
For a perfect Ansatz, the normalization factor should be unitary through out the time evolution \cite{D2-norm, D2-SBM2}.

The time-dependent probabilities that the spin in the states $\ket{0}$, $\ket{1}$ and $\ket{-1}$ are denoted as
\begin{equation}\label{EQ13}
  P_{0}(t)=\braket{{\rm D}_{2}^{M}(t)|0} \braket{0|{\rm D}_{2}^{M}(t)}
=\sum_ {m,n}^{M}{\left({B_m}^\ast B_n\right)S_{mn}},
\end{equation}
\begin{equation}\label{EQ14}
P_{-1}(t)=\braket{{\rm D}_{2}^{M}(t)|-1} \braket{-1|{\rm D}_{2}^{M}(t)}
=\sum_ {m,n}^{M}{\left({A_m}^\ast A_n\right)S_{mn}},
\end{equation}
and
\begin{equation}\label{EQ15}
  P_{1}(t)=\braket{{\rm D}_{2}^{M}(t)|1} \braket{1|{\rm D}_{2}^{M}(t)}
=\sum_ {m,n}^{M}{\left({C_m}^\ast C_n\right)S_{mn}}.
\end{equation}
If the initial state is set to $\ket{0}$, $P_{0}(t)$, $P_{1}(t)$ and $P_{-1}(t)$ express the transition probabilities from the initial state to the three states $\ket{0}$, $\ket{1}$ and $\ket{-1}$ at the time $t$, respectively. The three probabilities satisfy
\begin{equation}\label{EQ16}
P_{-1} + P_{0} + P_{1} =1,
\end{equation}
which means the measurement outcomes are only $\ket{-1}$, $\ket{0}$ and $\ket{1}$. In this work, the initial state is set to $\ket{0}$.

% For physical observables, we calculate the time-dependent transition probability between the initial state and the three states.
% $|0\rangle$ and $\ket{-1}$. The transition probability to stay in $|0\rangle$ at time $t$ can be written as:
% \begin{eqnarray}%\label{EQ10}
% P_{0}(t)=\langle {\rm D}_{2}^{M}(t)|0\rangle\langle0|{\rm D}_{2}^{M}(t)\rangle
% =\sum_ {m,n}^{M}{\left({B_m}^\ast B_n\right)S_{mn}}
% \end{eqnarray}
% Similarly, the transition probability to $\ket{-1}$ at time $t$ can be written as:
% \begin{eqnarray}%\label{EQ11}
% P_{\text{-} 1}(t)=\langle {\rm D}_{2}^{M}(t)|0\rangle\langle\text{-} 1|{\rm D}_{2}^{M}(t)\rangle
% =\sum_ {m,n}^{M}{\left({B_m}^\ast A_n\right)S_{mn}}
% \end{eqnarray}
% and for $P_{1}(t)$:
% \begin{eqnarray}\label{EQ11}
% P_{1}(t)=\langle {\rm D}_{2}^{M}(t)|0\rangle\langle1|{\rm D}_{2}^{M}(t)\rangle
% =\sum_ {m,n}^{M}{\left({B_m}^\ast C_n\right)S_{mn}}
% \end{eqnarray}
% Transition probability between three levels is conserved and obey the following relationship due to the normalization of wave function:
% \begin{eqnarray}%\label{EQ12}
% P_{0}(t)+P_{1}(t)+P_{\text{-} 1}(t)=1
% \end{eqnarray}
% For the results obtained this work, the initial population at $t = 0$ is set as: $P_{0} = 1$, and $P_{1} = P_{\text{-} 1} = 0$.

To explore the role of the bath on the dynamics of LZ transitions, one can calculate the population $P_{k,n}$ of the $\ket{k}$ spin states ($k=-1$, $0$, $1$) with phonons in Fock state $\ket{n}$.
Define the projection operators $\hat{P}_{k,n}=\ket{k}\bra{k}\otimes\ket{n}\bra{n}\equiv \ket{k,\ n}\bra{k,\ n}$, and the population is
\begin{equation}
P_{k,n} = \mathrm{Tr} (\rho \hat{P}_{k,n}) = \left|\braket{k,\ n|\mathrm{D}_{2}^{M}} \right|^{2}
\end{equation}

\section{RESULTS AND DISCUSSION}

The 3-LZM without coupling to the phonon bath has been extensively studied \cite{Core-Statis, Core-kiselev, three1, three2}. When the driving field is linear, the  anisotropy lifts the energy difference between the $|\pm 1\rangle$ and $|0\rangle$ states. The isotropic 3-LZM ($\mathcal{D}=0$) was discussed extensively ~\cite{three1}. When $\mathcal{D}$ is zero, transitions between the states  $|0\rangle$ and $|1\rangle$  occur simultaneously with transitions between the states $|0\rangle$ and $\ket{-1}$.  When $\mathcal{D}$ is small, the two LZ transitions are separated by a small amount of time, which allows interferences and yield pulse-shaped pattern in the time evolution of transition probability. When $\mathcal{D}$ is large, time interval between the two LZ transitions are too large for the interferences to occur, and the transition probability dynamic resembles two sequential  LZ transitions.

When the 3-LZM is driven by periodic external field, $P_{(\text{-} 1)}(t)$ is equal to $P_{1}(t)$, which is a predictable outcome in consideration of the symmetry of the Hamiltonian in time. In this case, $P_{(\pm 1)}(t)$ is non-trivial only if the choice of driving field frequencies and anisotropy suffice certain resonance conditions, which is discussed in ref.~\cite{Core-Statis}. The origin of such phenomenon can be attributed to the Coherent Destruction of Tunneling(CDT)\cite{CDT}, which particularly refers to the disappearance of tunneling effect for some parameter regimes.

Although works have been done on 3-LZM, the effect of phonon coupling is still not yet fully investigated. In this section, we extend the study of the 3-LZM to the phonon coupled regime. In III.A, we set the driving field in the 3-LZM linear and analyze the mechanism behind the change of transition probability caused by phonon coupling. In III.B, we plot the contour plot of the contour plot of $P_{\text{-} 1} (\mathcal{D}, A_z)$ and explore various of phenomena related to it. In III.C, we use
multiple phonon modes to mimic the effect of environmental dissipation and quantitatively study the damping effect at different bath coupling strength. We note that the unit frequency $\omega$ is chosen to be $\omega_z$ for section III. The choice of unit frequency has no influence on the resolved dynamics as long as it is in a reasonable regime.





\subsection{Single-mode phonon coupling with linear driving field}

In this subsection, we study a dissipative 3-LZM driven by an external field linearly. The tunneling rate between any two adjacent states $\Omega_{x}(t)$ is set to $\Delta$, and the linearly external driving is $\Omega_{z}(t)=vt$, where $v$ is the scanning velocity and describes the changing speed of the external magnetic field. The ZFS $\mathcal{D}$ is chosen as $10$. The reason we choose the value of $\mathcal{D}$ is based on following consideration. For a 3-LZM which does not interact with a phonon bath, how the energy levels of $\ket{1}$, $\ket{0}$, $\ket{-1}$ cross with each other is determined with $\mathcal{D}$. The intersection points between $\ket{1}$ and $\ket{0}$, $\ket{1}$ and $\ket{-1}$ as well as $\ket{-1}$ and $\ket{0}$ are separated by a value of $2\mathcal{D}$ \cite{Core-kiselev}. If $\mathcal{D}$ is small, $\ket{1}$ and $\ket{0}$ intersect at $vt=-\mathcal{D}$, $\ket{1}$ and $\ket{-1}$ at $t=0$, and $\ket{-1}$ and $\ket{0}$ at $vt=\mathcal{D}$. The avoided crossing at these close intersecting points will ``interfere'' with each other. In order to avoid the complex interfernces at small anisotropy, $\mathcal{D}=10$ is selected.

\begin{figure}[t]
\centerline{\includegraphics[width=80mm]{fig1a.eps}}
\centerline{\includegraphics[width=80mm]{fig1b.eps}}
\caption{Panels (a),(b) and (c): Time evolution of the LZ transition probabilities $P_{1} (t)$, $P_{0} (t)$ and $P_{-1}(t)$, respectively, from $\omega t = -20$ to $\omega t = 40$ with various values of $\eta_z$: $\eta_z=0$ (blue line), $\eta_z/\omega=0.2 $ (orange line), and $\eta_z/\omega=0.4 $ (green line). The rest of the parameters used are: $v/\omega^2 = 1 $, $\mathcal{D} /\omega= 10$, $\Delta /\omega= 0.5$, and $\omega_{p}/\omega = 1$. Panel (d): The corresponding energy diagram of the Hamiltonian with the same parameter, except $\eta_{z} = 0.4\omega$. The Fock state is restricted with an upper limit of $|{2}\rangle$. Details of Fig.~\ref{Fig1} (d) at $\omega t=-10$ and $0$ are amplified in Figs.~\ref{Fig1} (e) and (f), respectively. }
\label{Fig1}
\end{figure}


{We first investigate how the diagonal coupling strength $\eta_{z}$ affects the dynamics of the LZ transition. The time dependent probabilities for $\ket{1}$, $\ket{0}$ and $\ket{-1}$, $P_{1} (t)$, $P_{0} (t)$ and $P_{-1}(t)$, are plotted in Figs.~\ref{Fig1} (a), (b) and (c), respectively, for three values of $\eta_{z}$.
To clarify which energy levels are
involved in the LZ dynamics, the energy levels are shown in Fig.~\ref{Fig1} (d).
Details of Fig.~\ref{Fig1} (d) at $\omega t=-10$ and $0$ are amplified in Figs.~\ref{Fig1} (e) and (f), respectively. As shown
in Fig.~\ref{Fig1} (a), $P_{1} (t)$ rises at $\omega t=-\mathcal{D}\omega/v=-10$ where a LZ transition occurs between $\ket{1}$ and $\ket{0}$. In Fig.~\ref{Fig1}
(d), one can see that $\ket{1}$ increases with time while $\ket{0}$ remains constant under the linearly changing field ($vt$).
It follows that $\ket{1}$ and $\ket{0}$ meet at
$\omega t=-\mathcal{D}\omega/v=-10$. At $\omega t=-\mathcal{D}\omega/v=-10$,
a corresponding decrease in $P_{0}$ can be seen in Fig.~\ref{Fig1} (b). Meanwhile, the gap are opened between
$\ket{1,n}$ and $\ket{0,m}$ Fig.~\ref{Fig1} (e), where $n$ and $m$ denote the indices of the photonic Fock state.
Similarly, $P_{0} (t)$ dips at $\omega t=\mathcal{D}\omega/v=10$ in Fig.~\ref{Fig1} (b), and a corresponding increase in $P_{-1} (t)$ can be seen in Fig.~\ref{Fig1} (c). Both events correspond to the LZ transition between $\ket{0}$ and $\ket{-1}$, which is shown in Fig.~\ref{Fig1} (d).


Also shown in Fig.~\ref{Fig1} (b), the probability $P_{0} (t)$ decreases in the vicinity of $ \omega t=\mathcal{D}\omega/v=10 $ with a simultaneous rise in $P_{-1}(t)$ in Fig.~\ref{Fig1} (c), which corresponds to the avoided crossing between $\ket{-1}$ and $\ket{0}$.
In Fig.~\ref{Fig1}, the curves of $\eta_{z}/\omega=0$ and $0.2$ nearly overlap, while the curves of $\eta_{z}/\omega=0.4$ deviate substantially from those of $\eta_{z}/\omega=0$ and $0.2$, inferring thta $\eta_{z}/\omega=0.4$ is a relative strong coupling strength. One may say that $\eta_{z}/\omega=0.2$ is weak coupling as the transition probabilities with $\eta_{z}/\omega=0.2$ have little deviation from those with $\eta_{z}=0$. How exactly the coupling strength $\eta_{z}$ influence the dynamics is complex. As shown in Fig.~\ref{Fig1} (b), the rate of descent for $P_{0} (t)$ of $\eta_{z}/\omega=0.4$ at $\omega t=10$ ($-10$) is higher (lower) than those of $\eta_{z}/\omega=0$ and 0.2. Correspondingly, $P_{1} (t)$ ($P_{-1} (t)$) of $\eta_{z}/\omega=0.4$ rises faster (slower) at $\omega t=10$ ($-10 $) in Fig.~\ref{Fig1} (a) (Fig.~\ref{Fig1} (c)). From Fig.~\ref{Fig1} (d), one can see that the energy levels at $\omega t=-10$ and $\omega t=10$ are symmetric. However, the rates of descent for $P_{0} (t)$ of $\eta_{z}/\omega=0.4$ at $\omega t=-10$ and $\omega t=10$ are not the same.}

{To understand the asymmetry of the dynamics at $\omega t=-10$ and $10$, population $P_{k,n}$ is calculated and are plotted in Fig.~\ref{Fig2}. The plots are arranged as follows. Curves with the same spin are plotted in the same row. $P_{1,n}$ are plotted in Figs.~\ref{Fig2} (a), (b) and (c), $P_{0,n}$ are plotted in Figs.~\ref{Fig2} (d), (e) and (f), and $P_{-1,n}$ are plotted in Figs.~\ref{Fig2} (g), (h) and (i). Curves with the same coupling strength are plotted in the same column. $P_{k,n}(t)$ of $\eta_z/\omega=0$ are in Figs.~\ref{Fig2} (a), (d) and (g), $P_{k,n}(t)$ of $\eta_z/\omega=0.2$ are in Figs.~\ref{Fig2} (b), (e) and (h), and $P_{k,n}(t)$ of $\eta_z/\omega=0.4$ are in Figs.~\ref{Fig2} (c), (f), (i). The complex dynamic behavior of $P_{z}(t)$ can be understood with the help of the dynamics of $P_{k,n}(t)$. As shown in Figs.~\ref{Fig2} (a)-(i), $P_{k,n}(t)$ for various Fock state $\ket{n}$ are plotted. $P_{k,0}$ are plotted with blue lines, $P_{k,1}$ are plotted with orange lines, and $P_{k,2}$ are plotted with green lines. For $\eta_z/\omega=0$ ($\eta_z/\omega=0.2$), the curves of $P_{k,n}(t)$ with $n\geqslant 1$ ($n\geqslant 2$) are negligibly small and invisible. For $P_{k,n}(t)$ of $\eta_z/\omega=0$ in Figs.~\ref{Fig2} (a), (d) and (g), the curves share similar topology with $P_{z}(t)$ in Figs.~\ref{Fig1} (a), (b) and (c), respectively, which lends support to our analysis. The most prominent difference of $P_{k,n}(t)$ of $\eta_z/\omega=0.4$ with those of $\eta_z/\omega=0$ and 0.2 is that $P_{k,n}(t)$ with $n\geqslant 1$ are much larger. This leads to dynamics of $P_{z}(t)$ not only influenced by the vacuum state $\ket {n = {0}}$ of the bath but also by the Fock states with $n\geqslant 1$. The dynamics of $P_{z}(t)$ reflect the total impact of the ground and the excited states of the bath. As shown in Fig.~\ref{Fig2} (f), the height of $P_{0, 0}(t)$ is the same as those in Fig.~\ref{Fig2} (d) and (e). However, $P_{0,n=1,2}(t)$ are much larger. We plot $\sum_{n=0}^{2}P_{0,n} (t)$ with red dashed lines in Figs.~\ref{Fig2} (c), (f) and (i).
One can easily find that $\sum_{n=0}^{2}P_{0,n}(t)$ Fig.~\ref{Fig2} (f) is larger than $P_{0,0} (t) $ in Figs.~\ref{Fig2} (d) and (e).
The red dashed curves of $\sum_{n=0}^{2}P_{k,n} (t)$ for $k= \pm 1$ in Figs.~\ref{Fig2} (c) and (i) also support our above analysis. Seen alone, the complex spin dynamics $P_{z}(t)$ may be difficult to understand. The detailed dynamics of the phonon bath, as revealed by our variational method, provide great help to decipher the spin dynamics.




\begin{figure}[t]
\centerline{\includegraphics[width=85mm,scale=1]{fig2.eps}}
\caption{Time evolution of the population $P_{k,n}(t)$ (from $\omega t = -20$ to $\omega t = 40 $). $P_{1,n}$ are plotted in Panels (a), (b) and (c), and
$P_{0,n}$ are plotted in Panels (d), (e) and (f). $P_{-1,n}$ are plotted in Panels (g), (h) and (i). Curves at $\eta_{z}/\omega=0$ are plotted in Panels (a), (d) and (g). Curves at $\eta_{z}/\omega=0.2$ are plotted in Panels (b), (e) and (h). Curves at $\eta_{z}/\omega=0.4$ are plotted in Panels (c), (f) and (i). Other parameters are the same as those in Fig.~\ref{Fig1}. $P_{k,0}(t)$ are plotted in blue, $P_{k,1}(t)$ are plotted in orange and $P_{k,2}(t)$ are plotted in green. $\sum_{n=0}^{2}P_{0,n} (t)$ are plotted with red dashed lines in Panels (c), (f) and (i). }
\label{Fig2}
\end{figure}


\subsection{Single-mode phonon coupling with periodic driving field}

We now switch to the study of 3-LZM in the presence of a periodic driving field. In this subsection, we set $\Omega_z(t) = A_{\rm z}\cos(\omega_{\rm z} t)$ and $\Omega_x(t) = A_{\rm x}\cos(\omega_{\rm x} t)$, where $A_{\rm z}$($A_{\rm x}$) and $\omega_{\rm z}$($\omega_{\rm x}$) are the amplitude and the frequency of the z(x)-direction driving field, respectively. As mentioned earlier, $P_{\text{-} 1}(t) = P_{1}(t)$ when the driving field of the 3-LZM is periodic. By considering Eq.~(\ref{EQ16}), if one of the three transition probabilities is known, one can immediately obtain the other two, and we choose to study the effects of periodic driving field on $P_{(\text{-} 1)}(t)$ in Secs.~III B and III C. In the Hamiltonian we consider, following Eq.~(\ref{EQ2}), the phonon can couple to the 3-LZM on both the x and the z direction. However, as only trivial changes in the dynamics are found when the periodically driven 3-LZM is coupled to z-direction phonons, the x-direction phonon coupling will be our focus in this subsection.

\begin{figure}[h]
\adjustimage{width=.35\textwidth,center}{fig3.eps}
\caption{Contour plot of $P_{\text{-} 1}( t = 5/\omega)$ as a function of $A_z/\omega$ and $\mathcal{D}/\omega$. The x-direction driving field amplitude $A_x /\omega = 0.1$, and $\omega_{x} /\omega= 10$. The phonon frequency $\omega_{p} /\omega= 1$, the x-direction phonon coupling strength $\eta_{x} /\omega= 0.1$, and the z-direction phonon coupling strength $\eta_{z} = 0$. A multiplicity of $M = 6$ is used to generate the results in this figure.}
\label{Fig3}
\end{figure}

To visualize the effects of the periodic driving field \cite{Core-contour_plot}, in Fig.~\ref{Fig3}, a contour plot of $P_{\text{-} 1}( t = 5 /\omega)$ is displayed as a function of $(\mathcal{D}, A_z)$, where the x-axis is the anisotropy $\mathcal{D}$ and the y-axis is the z-direction driving field amplitude $A_z$. $\omega_x /\omega= 10$, $A_x/\omega = 0.1$ and the phonon frequency $\omega_{p}/\omega = 1$. The phonon mode is coupled to the 3-LZM in the x-direction with a coupling strength $\eta_{x}/\omega = 0.1$. The x-direction driving field amplitude is chosen to be $A_x /\omega = 0.1$. A multiplicity of $M = 6$ is selected to ensure the convergence of our results.
Three main peaks appear in Fig.~\ref{Fig3} for weak driving amplitudes. Based on their origins, the three peaks can be grouped as: 1) a pair of peaks that are centered at $\mathcal{D}/\omega = \pm 10 $; 2) a single peak centered at $\mathcal{D}/ \omega = - 1$. Not related to the phonon mode, the former has a smaller height than the latter, and is introduced by the x-direction driving field with their centers corresponding to $\mathcal{D} = \pm \omega_x$~\cite{Core-Statis}.
The single peak with a larger height at the center of Fig.~\ref{Fig3} is induced by the x-direction spin-phonon coupling, which is located
at the resonance position of the phonon mode, $\mathcal{D} / \omega = -1$.

If one assumes off-diagonal phonon coupling plays a similar role as the x-direction driving field in the contour plot, there would be another symmetric peak at $\mathcal{D} /\omega = 1$. However, only a single peak is found in Fig.~\ref{Fig3}, which matches the phonon frequency. The absence of the symmetric peak is the result of the choice of the initial state at $ t = 0$: the initial displacement of the phonon coherent state is set to zero in Fig.~\ref{Fig3}.
For verification, in Fig.~\ref{Fig4}(a), we set the initial displacement of the phonon coherent state $f $ to $1$. The peak at $\mathcal{D}/ \omega = 1$ is seen with a height lower than that of the peak at $\mathcal{D}/ \omega = -1$. From the Fock-state expansion of the coherent state, $\ket{\alpha} = e^{-{|\alpha|}^2/2}\sum_{{n} = 0}^{\infty} {\alpha^n} \ket{{n}} /{\sqrt{{n}!}}$, it is obvious that vacuum state takes up $60.65 \%$ of the initial Ansatz $\ket{\alpha = 1}$, leaving only 39.35 \% of the initial Ansatz to be able to survive the application of $\hat{b}_k$, while 100\% of the initial Ansatz is survivable against the application of $\hat{b}^{\dagger}_k$. This leads to the unequal peak heights at $\mathcal{D} / \omega = \pm 1$.
In Fig.~\ref{Fig4}(b), $\omega_p /\omega$ is increased to $5$ to avoid overlap between the phonon-induced peaks and have a better view of the peak shape.
The peak shapes at $\mathcal{D}/\omega  = \pm 5 $ in Fig.~\ref{Fig4}(b) are found to be fragmented as compared with that in Fig.~\ref{Fig3}.
This demonstrates that non-zero initial displacements will promote transitions to the Fock states with higher phonon number, and lead to more complex dynamics in $P_{\text{-} 1} (t)$.
It can also be seen that in Figs.~\ref{Fig3} and \ref{Fig4}, phonon-induced peak heights exceed those induced by the driving fields, despite that $\eta_x = A_x$, implying that tunneling effect induced by off-diagonal phonon coupling is stronger than that due to the x-direction driving field in this anisotropic system.


\begin{figure}[b]
\adjustimage{width=.5\textwidth,right}{fig4.eps}
\caption{ Panel (a): Contour plot of $P_{\text{-} 1} ( t = 5/\omega)$ with the phonon initial displacement $f = 1$. The rest of parameters are the same as in Fig.~\ref{Fig3}. Panel (b): Same parameters as in Panel (a) except the phonon frequency is changed to $\omega_p/\omega = 5 $.}
\label{Fig4}
\end{figure}

\begin{figure}[h]
\vspace{-0.6cm}
\centerline{\includegraphics[width=92mm,scale=1]{fig5.eps}}
\vspace{-0.6cm}
\caption{$P_{\text{-} 1} (t)$ is shown up to $\omega t = 50  $ along the brightest strip of the phonon-induced peak in the contour plot of Fig.~\ref{Fig3}, where $A_z$ and $\mathcal{D}$ obey the relation of $\mathcal{D} = -\omega+A_z$. Other parameters are $A_x/\omega = 0.1$, $\omega_{x}/\omega = 10$, $\omega_{p} /\omega= 1$, $\eta_{x}/\omega = 0.1$ and $\eta_{z}/\omega = 0$. Three most representative points are selected for each of the five panels. Panel (a): $A_z/\omega = 0, 0.2$ and $ 0.4$; Panel (b): $A_z/\omega = 0.8, 0.9$ and $1$; Panel (c): $A_z/\omega = 1.1, 1.2$ and $ 1.5$; plot (d): $A_z/\omega = 1.8, 1.9$ and $2$; Panel (e): $A_z/\omega = 3, 4$ and$ 5$.}
\label{Fig5}
\end{figure}


In addition to the absence of the peak symmetry with respect to anisotropy $\mathcal{D}$, it should be noted that for the peak near $\mathcal{D}/\omega = - 1$, the segment from $A_z = 0$ to $A_z/\omega = 1.5 $ is more pronounced than the rest of the peak. Motivated by this, points on the brightest strip of the phonon-induced peaks in the contour plot, which can be written as $\mathcal{D}/\omega = -1+A_z /\omega$, are chosen for a deeper probe of their dynamics. In Fig.~\ref{Fig3}, the probability is only shown up to $\omega t =5$. We extend the evolution time to $\omega t =50$ to examine the dynamics at longer times in Fig.~\ref{Fig5}. Based on the monotonicity of the evolution pattern, the $(\mathcal{D},A_z)$ points are grouped into 5 panels in Fig.~\ref{Fig5} for better illustration. $A_z$ and $\mathcal{D}$ of the chosen points obey the expression of the aforementioned strip, and the rest of the parameters remain the same as in Fig.~\ref{Fig3}: $A_x/\omega = 0.1$, $\omega_{x}/\omega = 10$, $\omega_{p}/\omega = 1$, $\eta_{x}/\omega = 0.1$ and $\eta_{z}/\omega = 0$.
In Fig.~\ref{Fig5}(a), for $A_z/\omega = 0$, $P_{\text{-} 1} (t)$ oscillates in a sinusoidal manner. As $A_z$ increases, the amplitude and the period of the sinusoidal oscillations quickly decreases, as if the curve collapses. In Fig.~\ref{Fig5}(b), with increasing $A_z/\omega$, the collapsed curve gradually rises and morphs into the shape of $P_{\text{-} 1} (t)$ at $A_z/\omega = 1$ with fast oscillations of increasing amplitudes. In Fig.~\ref{Fig5}(c), $P_{\text{-} 1} (t)$ collapses again with a quick drop in amplitude as $A_z/\omega$ goes above $1$. In Fig.~\ref{Fig5}(d), the collapsed curve revives as $A_z/\omega$ approaches $2$. Comparing the oscillation pattern before (at $A_z/\omega = 1$) and after (at $A_z/\omega = 2$) the collapse, despite the fast oscillation period remaining the same, the oscillation amplitude decreases significantly at $A_z/\omega = 2$. In Fig.~\ref{Fig5}(e), as $A_z/\omega$ continues to increase, the change in $P_{\text{-} 1} (t)$ is less pronounced. Only a decline in the $P_{\text{-} 1} (t)$ maxima is seen, while the general shape of the $P_{\text{-} 1} (t)$ stays the same.

In the absence of a z-direction driving field ($A_z = 0$), one has $\mathcal{D}/\omega= -1$ on the strip, which is largely different with $\omega_x$. As a result, the x-direction driving field has little influence on the 3-LZM, with tunneling provided only by the off-diagonal phonon coupling, which induces Rabi oscillations in $P_{\text{-} 1} (t)$ shown in Fig.~\ref{Fig5}(a).
For $A_z/\omega = 1$, $P_{\text{-} 1} (t)$ maintains an overall a rising trend with expanding oscillations of a period $2\pi / \omega$, consistent with the frequency of the z-direction driving field, as shown in Fig.~\ref{Fig5}(a). For $A_z/\omega = 2$, the amplitude of the fast oscillations in $P_{\text{-} 1} (t)$ is much reduced, as displayed in Fig.~\ref{Fig5}(d). A possible explanation is that, if one approximates the periodic driving field at the avoided crossings as a linear driving field, the scanning velocity of the linear driving field would be higher if the amplitude of the periodic field is larger. High scanning velocities at the avoided crossing would hinder the effect of z-direction driving field and weaken the high-frequency oscillations. It would also enlarge the gaps between the energy levels at the avoided crossing, which would weaken the tunneling effect, and eventually result in the decline of $P_{\text{-} 1} (t)$ maxima, as seen in Fig.~\ref{Fig5}(e).

It is also found that $P_{\text{-} 1} (t)$ collapses twice with the increase of $A_z/\omega$ in Fig.~\ref{Fig5}, the first is between $A_z/\omega = 0$ and $A_z/\omega = 1$ and the second is between $A_z/\omega = 1$ and $A_z/\omega = 2$. Given the parameter settings at the collapses, the tunneling effect of the off-diagonal phonon coupling is absent, and trivial dynamics is found in $P_{\text{-} 1} (t)$. Such phenomena can be attributed to CDT, where the choice of driving frequency under some parameter settings weakens the tunneling effect of a quantum system \cite{CDT, CDT1, CDT2, CDT3}.






\begin{figure}[b]
\centerline{\includegraphics[width=90mm,scale=1]{fig6.eps}}
\caption{Time evolution of the contour plot of $P_{\text{-} 1} (t)$ from ${\omega t = 0}$ to ${\omega t = 5}$. The parameters used are the same as those in Fig.~\ref{Fig3} (A video that captures the detailed evolution process can be found in supplementary materials).}
\label{Fig6}
\end{figure}

\

Apart from the time evolution of points in the contour plot, the time evolution of the entire contour plot is also important in order to investigate collective behavior of $P_{\text{-} 1} (t)$ in time. Fig.~\ref{Fig6} displays six screenshots of the time evolution of the contour plot in Fig.~\ref{Fig3} from $\omega t = 0$ to $\omega t = 5$. At $\omega t = 0.5$, the features are broad and blurry, and the overall transition probability is small. The peak shape gradually becomes sharper and narrower, and the transition probability also gradually increases. In addition, the stripes in the peaks also rotate clockwise as the system evolves. To explain these observations, one need to consider the Rabi cycle of $P_{\text{-} 1}(t)$. For near resonance points in the contour plot, the period and the amplitude of the Rabi cycle are very large. Because of the large period, the population of the near resonance points rise slowly at the start of the evolution. On the other hand, the population of the off resonance points can rise quickly because of their smaller periods, and they forms the broad and blurry peaks at the start of the evolution. However, as the contour plots in Fig.~\ref{Fig6} further evolve in time, $P_{\text{-} 1}(t)$ of the off resonance points start to decline as a result of their smaller amplitude of the Rabi cycle. $P_{\text{-} 1}(t)$ of the near resonance points, on the other hand, continue to increase and eventually becomes the major contributor to the peak shape in the contour plot. It can also be predicted that as the contour plot further evolves after $\omega t = 5$, the resonance pattern would narrow and eventually evolve into separate lines that tally with the exact resonance conditions. To better visualize the time evolution of the contour plot, the reader is referred to video clips in the accompanying supporting materials.

\subsection{Multi-mode phonon coupling with periodic driving field}


In Secs. III A and B, our discussion is focused on the case of a single-mode bath. In this subsection we extend our discussion to the dynamics of the 3-LZM coupled to the multi-mode bath.


We assume the bath is coupled to the 3-LZM in the $z$-direction. The bath is described by the super-Ohmic spectral density with $s = 3$ (see~Eq.~(\ref{EQ3})). We set $\mathcal{D} /\omega= \omega_x /\omega = 10$, assuming that the 3-LZM is in resonance with the x-direction driving field of the amplitude  $A_x /\omega= 0.05$. The z-direction driving field of the amplitude $A_z/\omega = 0.5$ is also included in the calculations. The cut-off frequency $\omega_c/\omega$ is fixed at  $0.5$.  Following Section II A, the continuum spectrum is discretized into 20 phonon modes ($N_b = 20$) with $\omega_k \in [0, \omega_m]$. In this case,  $\omega_c$, $\omega_m$ and $N_b$ are large enough, their specific values have no effect on the evaluated observables, and multiplicity of $M = 16$ ensures converged results.

\begin{figure}[bt]
\centerline{\includegraphics[width=80mm,scale=0.8]{fig7.eps}}
\caption{Panel (a):  Time evolution of the population $P_{\text{-} 1} (t)$ from $\omega t = 0$ to $\omega t = 50$ with five values of $\alpha$. Other parameters of  $\hat{H}_{\rm 3-LZM}$: $\mathcal{D}/\omega = 10$, $A_z/\omega= 0.5$, $A_x/\omega = 0.05$, and $\omega_x/\omega = 10 $. Parameters in $\hat{H}_{\rm sp}$: $s = 3$, $\omega_c /\omega= 0.5$, $\omega_m/\omega = 6$, $N_b = 20$, and $\alpha$ is chosen between 0 and 0.2 with a step of 0.05. A multiplicity $M = 16$ is used to ensure convergence. Panel (b): amplified portion of Panel (a) from $\omega t = 12.5$ to $\omega t = 17.5$. Panel (c): The periods (red) and the amplitudes (blue) of the Rabi cycles, obtained via sinusoidal fitting of curves in Panel (a), are plotted as a function of $\alpha$.}
\label{Fig7}
\end{figure}

In Fig.~\ref{Fig7}(a), $P_{\text{-} 1} (t)$ is plotted with respect to time for five values of the bath coupling strength $\alpha$ ranging from 0 to 0.2. In the absence of the bath ($\alpha = 0$), $P_{\text{-} 1} (t)$ exhibits Rabi oscillations (cf.  Fig.~\ref{Fig5}(a)). As coupling to the bath becomes stronger, Rabi frequencies  increase but amplitudes of Rabi oscillations decrease, as shown in Fig.~\ref{Fig7}(a). Interestingly, growing x-field amplitudes $A_x$ also increase frequencies and decrease amplitudes of the  $P_{\text{-} 1} (t)$ oscillations in the bath-free case, causing, simultaneously, changes in shapes of the oscillatory features (Fig.~\ref{Fig5}(a)). This observation may help to distinguish between the x-field driving and the bath effects.

In Fig.~\ref{Fig7}(b), we zoom in on a portion of Fig.~\ref{Fig7}(a) from $\omega t = 12.5$ to $\omega t = 17.5$. Apart from Rabi beatings, high-frequency low-amplitude oscillations in $P_{\text{-} 1} (t)$ are clearly visible in Fig.~\ref{Fig7}(b). At $\alpha = 0$, the oscillations are quite regular, with a period around 0.3/$\omega $. This indicates that the high-frequency oscillations are likely caused by the x-direction driving field which has a period of ${2\pi}/{\omega_x}=0.63$. However,  such high-amplitude oscillations cannot be found in Fig.~\ref{Fig5}(a), where $A_z/\omega = 0$. This indicates that the oscillations are, most likely, caused by the beatings between the driving fields in the x and z directions. Note also that the difference between the two curves in Fig.~\ref{Fig5}(a) and Fig.~\ref{Fig7}(b) occurs despite the x-direction phonon coupling and the x-direction driving field may produce peaks with similar shapes in the contour plots.
As $\alpha$ increases, the high-frequency low-amplitude oscillations become less regular but do not disappear. This is a signature of the combined effect of the x-direction driving field and the phonon coupling.

To quantify the bath-induced  damping effect, the periods and amplitudes of the $P_{\text{-} 1} (t)$ oscillations obtained via sinusoidal fitting are plotted in Fig.~\ref{Fig7}(c) as a function of the system-bath coupling $\alpha$. It is found that the amplitude of Rabi oscillations drops more than twice from $\alpha = 0$ to $\alpha = 0.05$. At larger $\alpha$, decrease of the amplitude slows down. A similar mitigation of the bath impact on the period of Rabi oscillations is also observed.







\section{Conclusion}

In this paper, we applied the Davydov-ansatz methodology for performing numerically accurate simulations of the 3-LZM coupled to a vibrational reservoir. We investigated how the combined effect of anisotropy,  external (linear and harmonic) driving fields in the x and z directions, and phonon modes modify the  3-LZM  population evolutions. It is found that phonon modes give rise to a plethora of different effects, from opening up additional phonon-assisted channels for LZ transitions in the case of a single phonon mode to reducing coherent effects and initiation of nontrivial modifications of the 3-LZM dynamics in the case of a  multidimensional phonon bath. While it is no surprise that dynamic behaviors of the chosen 3-LZM with anisotropy are extraordinarily rich and complex, we hvae tried to establish, whenever possible, how signatures of individual contributions (e.g., external driving or phonon coupling) can be identified in the ensuing population evolutions.

The numerically accurate Davydov-Ansatz methodology developed in the present work is exceptionally powerful.
On the one hand, it can be used for benchmarking various approximate (e.g., based on master equations) methods of the description of driven 3-LZMs coupled to phonon modes/baths. On the other hand, it is applicable for arbitrary time-dependent driving fields as well as for boson baths with any spectral densities and, if necessary, at final temperatures. This may facilitate future utilization of the developed many-body simulation machinery for studying specific experimental realizations of 3-LZM systems.



\section*{Authors' contributions}
L.Z. and L.W. contributed equally to this work.

\section*{Acknowledgments}
The authors would also thank Dr. Kewei Sun and Xia Yang for useful discussion.
Support from Nanyang Technological University ``URECA" Undergraduate Research Programme and
the Singapore Ministry of Education Academic Research
Fund Tier 1 (Grant No. RG87/20) is gratefully acknowledged.
\section*{Author Declarations}
\subsection*{Conflict of Interest}
The authors have no conflicts to disclose.

\section*{Data Availability}
The data that support the findings of this study are available from the corresponding author upon reasonable request.

\appendix
\onecolumngrid
\section{The equation of motions for the multi-D$_2$ Ansatz}\label{Appendix A}

The Lagrangian in Eq.~(\ref{EQ10}) can be simplified into the following form due to the normalization of the Ansatz state:
\begin{eqnarray}\label{EQA1}
L&=&i\langle {\rm D}_{2}^{M}(t)|\frac{\overrightarrow{\partial}}{\partial t}|{\rm D}_{2}^{M}(t)\rangle -\langle{\rm D}_{2}^{M}(t)|\hat{H}_{\rm sys }+\hat{H}_{\rm ph}|{\rm D}_{2}^{M}(t)\rangle = L_d - L_H
\end{eqnarray}

with:
\begin{eqnarray}\label{EQA2}
L_d &=&i\langle {\rm D}_{2}^{M}(t)|\frac{\overrightarrow{\partial}}{\partial t}|{\rm D}_{2}^{M}(t)\rangle \nonumber\\
& = &i\sum_{m, n=1}^{M}\left(A_{m}^{*}\dot{A}_{n}+B_{m}^{*}\dot{B}_{n}+C_{m}^{*}\dot{C}_{n}\right)S_{mn}\nonumber\\
&&+ i\sum_{m, n=1}^{M}\left(A_{m}^{*}A_{n}+B_{m}^{*}B_{n}+C_{m}^{*}C_{n}\right)\left(f_{mk}^{*}\dot{f}_{nk}-\frac{1}{2}\dot{f}_{nk}f_{nk}^{*}-\frac{1}{2}f_{nk}\dot{f}_{nk}^{*}\right)S_{mn}
\end{eqnarray}


and
\begin{eqnarray}\label{EQA3}
L_H&=&\langle{\rm D}_{2}^{M}(t)|\hat{H}_{\rm sys }+\hat{H}_{\rm ph}|{\rm D}_{2}^{M}(t)\rangle \nonumber\\
& = & \sum_{m, n=1}^{M} \big[ \Omega_z\left(A_{m}^{*}A_{n}-C_{m}^{*}C_{n}\right)+\Omega_x\left(A_{m}^{*}B_{n}+B_{m}^{*}A_{n}+B_{m}^{*}C_{n}+C_{m}^{*}B_{n}\right) \big]S_{mn}\nonumber\\
&&+ \sum_{m, n=1}^{M}\big[D\left(\frac{1}{3}A_{m}^{*}A_{n}-\frac{2}{3}C_{m}^{*}C_{n}+\frac{1}{3}C_{m}^{*}C_{n}\right)+\sum_k \omega_k \left(A_{m}^{*}A_{n}+B_{m}^{*}B_{n}+C_{m}^{*}C_{n}\right)f_{mk}^{*}f_{nk}\big]S_{mn}\nonumber\\
&&+ \sum_{m, n=1}^{M}\sum_k\big[\eta_k^z\left(A_{m}^{*}A_{n}-C_{m}^{*}C_{n}\right)+\frac{\eta_k^x}{\sqrt{2}}\left(A_{m}^{*}B_{n}+B_{m}^{*}A_{n}+B_{m}^{*}C_{n}+C_{m}^{*}B_{n}\right)\big]\left(f_{mk}^{*}+f_{nk}\right)S_{mn}
\end{eqnarray}

Following the Dirac-Frenkel time dependent variational principle, a set of EOMs can be solved by substituting $u_n$ in Eq.~(\ref{EQ10}) with $A_m$:
\begin{eqnarray}\label{EQA4}
&~& \sum_{n=1}^{M}\left[\mathrm{i}\dot{A}_{n}-\frac{\mathrm{i}}{2}A_{n}\sum_{k}\left(f_{nk}^{*}\dot{f}_{nk}+f_{nk}\dot{f}_{nk}^{*}-2f_{mk}^{*}\dot{f}_{mk}\right)\right]S_{mn}\nonumber\\
&= & \sum_{n=1}^{M}\left[ A_{n}\left( \Omega_z(t)+\frac{D}{3}+\sum_{k}[\omega_{k}f_{mk}^{*}f_{nk}+\eta_{k}^{z}\left(f_{mk}^{*}+f_{nk}\right)]\right )+ B_{n}\left(\Omega_x(t)+\sum_{k}\frac{\eta_{k}^{x}}{\sqrt{2}}\left(f_{mk}^{*}+f_{nk}\right)\right)\right]S_{mn}
\end{eqnarray}

Similarly with $B_m$:

\begin{eqnarray}\label{EQA5}
&~& \sum_{n}\left[\mathrm{i}\dot{B}_{n}-\frac{\mathrm{i}}{2}B_{n}\sum_{k}\left(f_{nk}^{*}\dot{f}_{nk}+f_{nk}\dot{f}_{nk}^{*}-2f_{mk}^{*}\dot{f}_{nk}\right)\right]S_{mn}\nonumber\\
&= & \sum_{n=1}^{M}\left[B_{n}\left(-\frac{2}{3}D+\sum_{k}\omega_{k}f_{mk}^{*}f_{nk}\right)+\left(A_{n}+C_{n}\right)\left( \Omega_x(t)+\sum_{k}\frac{\eta_{k}^x}{\sqrt{2}}\left(f_{mk}^{*}+f_{nk}\right)\right)\right]S_{mn}
\end{eqnarray}

and with $C_m$:

\begin{eqnarray}\label{EQA6}
&~& \sum_{n=1}^{M}\left[\mathrm{i}\dot{C}_{n}-\frac{\mathrm{i}}{2}C_{n}\sum_{k}\left(f_{nk}^{*}\dot{f}_{nk}+f_{nk}\dot{f}_{nk}^{*}-2f_{mk}^{*}\dot{f}_{mk}\right)\right]S_{mn}\nonumber\\
&= & \sum_{n=1}^{M}\left[ C_{n}\left( -\Omega_z(t)+\frac{D}{3}+\sum_{k}[\omega_{k}f_{mk}^{*}f_{nk}+\eta_{k}^{z}\left(f_{mk}^{*}+f_{nk}\right)]\right )+ B_{n}\left(\Omega_x(t)+\sum_{k}\frac{\eta_{k}^{x}}{\sqrt{2}}\left(f_{mk}^{*}+f_{nk}\right)\right)\right]S_{mn}
\end{eqnarray}

Lastly with $f_{mk}$:

\begin{eqnarray}\label{EQA7}
&~& \mathrm{i}\sum_{n=1}^{M}\left[\left(A_{m}^{*}\dot{A}_{n}+B_{m}^{*}\dot{B}_{n}+C_{m}^{*}\dot{C}_{n}\right)f_{nk}\right]S_{mn}\nonumber\\
&&+ \mathrm{i}\sum_{n=1}^{M}\left[\left(A_{m}^{*}{A}_{n}+B_{m}^{*}{B}_{n}+C_{m}^{*}{C}_{n}\right)\left(\dot{f}_{nk}-\frac{1}{2}f_{nk}\sum_{k^\prime}\left[\left(f_{nk^\prime}^{*}-2f_{mk^\prime}^{*}\right)\dot{f}_{nk^\prime}+f_{nk^\prime}\dot{f}_{nk^\prime}^{*}\right]\right)\right]S_{mn}\nonumber\\
&= & \sum_{n=1}^{M}\left(A_{m}^{*}A_{n}-C_{m}^{*}C_{n}\right)\left[\left(\Omega_z(t)+\sum_{k^\prime}\eta_{k^\prime}^{z}\left(f_{mk^\prime}^{*}+f_{nk^\prime}\right)\right)f_{nk}+\eta_{k}^{z}\right]S_{mn}\nonumber\\
&& +\sum_{n=1}^{M}\left(A_{m}^{*}A_{n}+B_{m}^{*}B_{n}+C_{m}^{*}C_{n}\right)\left[f_{nk}\left(\sum_{k^\prime}\omega_{k^\prime}f_{mk^\prime}^{*}f_{nk^\prime}\right)+\omega_{k}f_{nk}\right]S_{mn}\nonumber\\
&& +\sum_{n=1}^{M}\left(A_{m}^{*}B_{n}+B_{m}^{*}A_{n}+B_{m}^{*}C_{n}+C_{m}^{*}B_{n}\right)\left[\left(\Omega_x(t)+\sum_{k^\prime}\frac{\eta_{k^\prime}^{x}}{\sqrt{2}}\left(f_{mk^\prime}^{*}+f_{nk^\prime}\right)\right)f_{nk}+\frac{\eta_{k}^{x}}{\sqrt{2}}\right]S_{mn}\nonumber\\
&& +\sum_{n=1}^{M}\left(A_{m}^{*}A_{n}-2B_{m}^{*}B_{n}+C_{m}^{*}C_{n}\right)\frac{D}{3}f_{nk}S_{mn}
\end{eqnarray}



\twocolumngrid
%\bibliographystyle{jcp}
%\bibliography{refs}


\begin{thebibliography}{999}
\bibitem{Landau}L. D. Landau, Zur Theorie der Energieübertragung. II, Phys. Z. Soviet Union {\bf 2}, pp. 46-51 (1932)

\bibitem{Zener} C. Zener. Non-adiabatic crossing of energy levels. Proc. R. Soc. Lond. Ser. A Math. Phys. Eng. Sci. {\bf 137}, 696-702 (1932)

\bibitem{m_ph1}A. Niranjan, W. Li, and R. Nath, Landau-Zener transitions and adiabatic impulse approximation in an array of two Rydberg atoms with time-dependent detuning, Phys. Rev. A {\bf 101},063415 (2020)


\bibitem{m_ph2}S. S. Zhang and W. Gao and H. Cheng and L. You and H. P. Liu, Symmetry-Breaking Assisted Landau-Zener Transitions in Rydberg Atoms, Phys. Rev. Lett. {\bf 120}, 063203 (2018)


\bibitem{LZ-cavity1}N. A. Sinitsyn and F. Li, Solvable multistate model of Landau-Zener transitions in cavity QED, Phys. Rev. A {\bf 93}, 063859 (2016)


\bibitem{chem_physics1} L. Zhu, A. Widom and P. M. Champion, A multidimensional Landau-Zener description of chemical reaction dynamics and vibrational coherence, J. Chem. Phys. {\bf 107}, 2859 (1997)


\bibitem{QIP1}Cao, G., Li, HO., Tu, T. et al, Ultrafast universal quantum control of a quantum-dot charge qubit using Landau–Zener–Stückelberg interference, Nat Commun {\bf 4}, 1401 (2013)


\bibitem{QIP2}Matityahu, S., Schmidt, H., Bilmes, A. et al. Dynamical decoupling of quantum two-level systems by coherent multiple Landau–Zener transitions, npj Quantum Inf {\bf 5}, 114 (2019)



\bibitem{nanomagnet} W. Wernsdorfer,  R. Sessoli,  A. Caneschi,  D. Gatteschi,  A. Cornia,  Nonadiabatic Landau-Zener tunneling in $Fe_8$ molecular nanomagnets, Europhys. Lett. {\bf 50}, 552 (2000)


\bibitem{BEC}Abraham J. Olson, Su-Ju Wang, Robert J. Niffenegger, Chuan-Hsun Li, Chris H. Greene, and Yong P. Chen, Tunable Landau-Zener transitions in a spin-orbit-coupled Bose-Einstein condensate, Phys. Rev. A {\bf 90}, 013616 (2014)


\bibitem{quenching} L. Arceci, S. Barbarino, R. Fazio, and G. E. Santoro, Dissipative Landau-Zener problem and thermally assisted Quantum Annealing, Phys. Rev. B {\bf 96}, 054301 (2017)


\bibitem{NV1}J. Zhou, P. Huang, Q. Zhang et al, Observation of Time-Domain Rabi Oscillations in the Landau-Zener Regime with a Single Electronic Spin, Phys. Rev. Lett. {\bf 112}, 010503 (2014)


\bibitem{NV2}P. Huang, J. Zhou, F. Fang, X. Kong, X. Xu, C. Ju, J.Du, Landau-Zener-Stückelberg Interferometry of a Single Electronic Spin in a Noisy Environment, Phys. Rev. X {\bf 1}, 011003 (2011)


\bibitem{QDot1} X. Mi, S. Kohler, and J. R. Petta, Landau-Zener interferometry of valley-orbit states in Si/SiGe double quantum dots,
Phys. Rev. B {\bf 98}, 161404(R) (2018)


\bibitem{QDot2}P. Nalbach, J. Kn\"orzer, and S. Ludwig, Nonequilibrium Landau-Zener-Stueckelberg spectroscopy in a double quantum dot,
Phys. Rev. B {\bf 87}, 165425 (2013)


\bibitem{LZ_review}O. V. Ivakhenko, S. N. Shevchenko, F. Nori, Nonadiabatic Landau–Zener–St\"{u}ckelberg–Majorana transitions, dynamics, and interference , Phys. Rep. {995}, pp. 1-89 (2023)


\bibitem{LZ-multi1}V. Chernyak, N. Sinitsyn, C. Sun, Multitime Landau–Zener model: classification of solvable Hamiltonians, J. Phys. A: Math. Theor. {\bf 53} ,185203 (2020)


\bibitem{Volodya21}R. K. Malla,  V. Y. Chernyak,  N. A. Sinitsyn. Nonadiabatic transitions in Landau-Zener grids: integrability and semiclassical theory
 Phys. Rev. B {\bf 103}, 144301 (2021).



\bibitem{Andrey}A. R. Kolovsky and D. N. Maksimov, Mott-insulator state of cold atoms in tilted optical lattices: Doublon dynamics and multilevel Landau-Zener tunneling, Phys. Rev. A {\bf 94}, 043630 (2016)


\bibitem{band}Y. B. Band, Y. Avishai, Three-level Landau-Zener dynamics, Phys. Rev. A{\bf 99}, 032112 (2019)


\bibitem{LZ-BEC-well_trap}E. M. Graefe, H. J. Korsch, and D. Witthaut, Mean-field dynamics of a Bose-Einstein condensate in a time-dependent triple-well trap: Nonlinear eigenstates, Landau-Zener models, and stimulated Raman adiabatic passage, Phys. Rev. A {\bf 73}, 013617 (2006)


\bibitem{at_gas}L. Cornelius Fai, M. Tchoffo, and M. Nana Jipdi,Landau Zener scenario in a trapped atomic gas: multi-level multi-particle model, Eur. Phys. J. B {\bf 88}, 181 (2015)


\bibitem{TQD1}G. Granger, L. Gaudreau, A. Kam et al, Three-dimensional transport diagram of a triple quantum dot, Phys. Rev. B {\bf 82}, 075304 (2010)


\bibitem{TQD2}D. Schr\"{o}er, A. D. Greentree, L. Gaudreau, K. Eberl, L. C. L. Hollenberg, J. P. Kotthaus, and S. Ludwig, Electrostatically defined serial triple quantum dot charged with few electrons, Phys. Rev. B {\bf 76}, 075306 (2007)



\bibitem{D2-dw3}Z. Huang and Y. Zhao, Dynamics of dissipative Landau-Zener transitions, Phys. Rev. A {\bf 97}, 013803 (2018)


\bibitem{LZ-periodic2}F. Zheng, Y. Shen, K. Sun, Y. Zhao, Photon-assisted Landau–Zener transitions in a periodically driven Rabi dimer coupled to a dissipative mode, J. Chem. Phys. {\bf 154}, 044102 (2021)


\bibitem{LZ-entanglement1}C. Quintana, K. Petersson, L. McFaul, S. Srinivasaan, A. Houck, J. Petta, Cavity-mediated entanglement generation via Landau-Zener interferometry, Phys. Rev. Lett. {\bf 110}, 173603 (2013)


\bibitem{LZ-entanglement2}M. Wubs, S. Kohler, and P. H\"anggi, Entanglement creation in circuit QED via Landau-Zener sweeps, Physica E Low Dimens. Syst. Nanostruct. {\bf 40}, 187-197 (2007)



\bibitem{bath1}K. Saito, M. Wubs, S. Kohler, P. H\"anggi, Y. Kayanuma. Quantum state preparation in circuit QED
via Landau-Zener tunneling. Europhys. Lett. {\bf 76}, 22-28 (2006).


\bibitem{bath2} M. Wubs, K. Saito, S. Kohler, P. H\"anggi, and Y. Kayanuma, Gauging a Quantum Heat Bath with Dissipative Landau-Zener Transitions, Phys. Rev. Lett.  {\bf 97}, 200404 (2006)


\bibitem{YZ19} M. Werther,  F. Grossmann, Z. Huang,  Y. Zhao. Davydov-Ansatz for Landau-Zener-Stueckelberg-Majorana transitions in an environment: Tuning the survival probability via number state excitation. J. Chem. Phys. {\bf 150}, 234109 (2019).


\bibitem{Thorwart09} P. Nalbach, M. Thorwart. Landau-Zener Transitions in a Dissipative Environment: Numerically Exact Results, Phys. Rev. Lett. {\bf 103}, 220401 (2009).


\bibitem{Thorwart15}S. Javanbakht,  P. Nalbach, M. Thorwart. Dissipative Landau-Zener quantum dynamics with transversal and longitudinal noise, Phys. Rev.  A {\bf 91}, 052103 (2015).



\bibitem{dibatic_basis}K. Saito, M. Wubs, S. Kohler, Y. Kayanuma, P. H\"anggi, Dissipative Landau-Zener transitions of a qubit: Bath-specific and universal behavior, Phys. Rev. B {\bf 75}, 214308 (2007)



\bibitem{Core-kiselev}M. N. Kiselev, K. Kikoin, and M. B. Kenmoe, SU(3) Landau-Zener interferometry, Euro. Phys. Lett. {\bf 104}, 57004 (2013)


\bibitem{extention1}J. Lin, N. A. Sinitsyn, Exact transition probabilities in the three-state Landau–Zener–Coulomb model, J. Phys. A: Math. Theor. {\bf 47}, 015301 (2014)


\bibitem{extention2}C. E. Carroll, F. T. Hioe, Generalization of Landau-Zener calculation to three levels, J. Phys. A: Math. Gen. {\bf 19}, 1151-1161 (1986)


\bibitem{NV_review1}A. Gali, Ab initio theory of the nitrogen-vacancy center in diamond, Nanophotonics {\bf 8}, https://doi.org/10.1515/nanoph-2019-0154 (2019)


\bibitem{NV-computing1}M. V. Gurudev Dutt, L. Childress, L. Jiang et al, Quantum register based on individual electronic and nuclear spin qubits in diamond, Science {\bf 316},
1312-1316 (2007)


\bibitem{NV-computing2}S. Sangtawesin, C. McLellan, B. Myers, A. Jayich, D. Awschalom, J. Petta, Hyperfine-enhanced gyromagnetic ratio of a nuclear spin in diamond, New J. Phys. {\bf 18}, 083016 (2016)


\bibitem{NV-computing3}P. C. Maurer, G. Kucsko, C. Latta et al, Room-temperature quantum bit memory exceeding one second, Science {\bf 336}, 1283-1286 (2012)


\bibitem{NV-stress} M. Doherty, V. Struzhkin, D. Simpson et al, Electronic Properties and Metrology Applications of the Diamond $NV^{-}$ Center under Pressure, Phys. Rev. Lett. {\bf 112}, 047601 (2014)


\bibitem{TQD-automata}C. Lent, D. Tougaw, W. Porod, G. Bernstein, Quantum cellular automata, Nanotechnology {\bf 4}, 49 (1993)


\bibitem{TQD-rectifier}M. Stopa, Rectifying Behavior in Coulomb Blockades: Charging Rectifiers, Phys. Rev. Lett. {\bf 88}, pp.4 (2002)


\bibitem{D2-review}Y. Zhao, K. Sun, L. Chen, M. Gelin, The hierarchy of Davydov's Ans\"atze and its applications, Wiley Interdiscip. Rev. Comput. Mol. Sci. {\bf 12}, https://doi.org/10.1002/wcms.1589 (2022)


\bibitem{D2-HTC}L. Chen, and Y. Zhao, Finite temperature dynamics of a Holstein polaron: The thermo-field dynamics approach, J. Chem. Phys. {\bf 147}, 214102 (2017)

\bibitem{D2-TC}K.Sun, C. Dou, M. F. Gelin, Y. Zhao,  Dynamics of disordered Tavis-Cummings and Holstein-Tavis-Cummings models, J. Chem. Phys. {\bf 156}, 024102 (2022)



\bibitem{D2-SBM1}L. Wang, F. Zheng, J. Wang, F. Grossmann, Y. Zhao, Schr\"{o}dinger-Cat States in Landau-Zener-St\"{u}ckelberg-Majorana Interferometry: A Multiple Davydov Ansatz Approach, J. Phys. Chem. B {\bf 125}, 3184-3196 (2021)


\bibitem{D2-SBM2}L. Wang, L. Chen, N. Zhou, Y. Zhao,  Variational dynamics of the sub-Ohmic spin-boson model on the basis of multiple Davydov D1 states, J. Chem. Phys. {\bf 144}, 024101 (2016)




\bibitem{D2-SF}K.Sun, M. F. Gelin, and Y. Zhao, Accurate Simulation of Spectroscopic Signatures of Cavity-Assisted, Conical-Intersection-Controlled Singlet Fission Processes, J. Phys. Chem. Lett. {\bf 13}, 4280-4288 (2022).



\bibitem{Core-spec_den}M.L.Goldman, M. Doherty, A. Sipahigil et al,  State-selective intersystem crossing in nitrogen-vacancy centers, Phys. Rev. B Condens. Matter {\bf 91}, 165201 (2015)



\bibitem{WFCZ} L.~Wang, Y.~Fujihashi, L.~Chen,and Y.~Zhao, Finite-temperature time-dependent variation with multiple Davydov states, J. Chem. Phys. {\bf 146}, 124127 (2017).




\bibitem{Davydov1}A. S. Davydov, The Theory of Contraction of Proteins under their Excitation, J. theor. Biol {\bf 38}, pp.559-569 (1973)


\bibitem{Davydov2}A. S. Davydov, and N. I. Kislukha, Solitary Excitons in One-Dimensional Molecular Chains, Phys. Status Solidi B {\bf 59}, pp.465-470 (1973)


\bibitem{Davydov4}A. Scott, Davydov's soliton, Phys. Rep. {\bf 217}, pp.1-67 (1992).


\bibitem{shor73} H.B.~Shore and L.M.~Sander, Ground State of the Exciton-Phonon System, {Phys. Rev. B}\textbf{7}, 4537 (1973).

\bibitem{jcppersp} Y.~Zhao, The hierarchy of Davydov's Ansaetze: from guesswork to numerically ``exact" many-body wave functions,  J. Chem. Phys. {\bf 158}, 080901 (2023).


\bibitem{Davydov5}N. Zhou, L. Chen, D. Mozyrsky, V. Chernyak, Y. Zhao,  Ground-state properties of sub-Ohmic spin-boson model with simultaneous diagonal and off-diagonal coupling, Phys. Rev. B Condens. Matter {\bf 90}, 155135 (2014)


\bibitem{Davydov6}N. Zhou, Z. Huang, J. Zhu, V. Chernyak, Y. Zhao, Polaron dynamics with a multitude of Davydov D2 trial states, J. Chem. Phys. {\bf 143}, 014113 (2015)


\bibitem{Davydov7}Y. Zhao, B. Luo, Y. Zhang, J. Ye, Dynamics of a Holstein polaron with off-diagonal coupling, J. Chem. Phys. {\bf 137}, 084113 (2012)


\bibitem{Davydov8}Y.Zhao, D. W. Brown, and K. Lindenberg, Variational energy band theory for polarons: Mapping polaron structure with the Toyozawa method, J. Chem. Phys. {\bf 107}, 3159 (1997)


\bibitem{QED_Huang} Z.~Huang, F.~Zheng, Y.~Zhang, Y.~Wei, and Y.~Zhao, ``Dissipative dynamics in a tunable Rabi dimer with periodic harmonic driving," \textit{J. Chem. Phys.} \textbf{150}, 184116 (2019).


\bibitem{QED_Zheng}  F.~Zheng, Y.~Zhang, L.~Wang, Y.~Wei, and Y.~Zhao, ``Engineering Photon Delocalization in a Rabi Dimer with a Dissipative Bath," \textit{Ann. Phys.} \textbf{530}, 1800351 (2018).

\bibitem{TMD} K.~Sun, K.~Shen, M.F.~Gelin, and Y.~Zhao, ``Exciton dynamics and time-resolved fluorescence in aanocavity-integrated monolayers of transition metal dichalcogenides," \textit{J. Phys. Chem. Lett.} {\bf 14}, 221-229 (2023).


\bibitem{D2-dw1}K. Sun, M. Gelin, Y. Zhao, Engineering Cavity Singlet Fission in Rubrene, J. Phys, Chem. Lett. {\bf 13}, 4090-4097 (2022)


\bibitem{D2-dw2}Z. Huang, M. Hoshina, H. Ishihara, Y. Zhao, Transient dynamics of super Bloch oscillations of a one dimensional Holstein polaron under the influence of an external AC electric field, Ann. Phys. {\bf 531}, 1800303 (2019)




\bibitem{D2-dw4}K. Sun, Q. Xu, L. Chen, M. Gelin, Y. Zhao, Temperature effects on singlet fission dynamics mediated by a conical intersection, J. Chem. Phys. {\bf 153}, 194106 (2020)


\bibitem{D2-dw5}Y. Fujihashi, L. Chen, A. Ishizaki, J. Wang, Y. Zhao, Effect of high-frequency modes on singlet fission dynamics, J. Chem. Phys. {\bf 146}, 044101 (2017)


\bibitem{D2-norm}Y. Yan, L. Chen, J. Luo, Y. Zhao, Variational approach to time-dependent fluorescenece of a driven qubit, Phys. Rev. A {\bf 102}, 023714 (2020)


\bibitem{Core-Statis}A.D. Kammogne, M.B. Kenmoe and L.C. Fai, Statistics of interferograms in three-level systems, Phys. Lett. A {\bf 425}, 127872 (2022)


\bibitem{three1}G. Wang, D. Ye, L. Fu, X. Chen, J. Liu, Landau-Zener tunneling in a nonlinear three-level system, Phys. Rev. A {\bf 74}, 033414 (2006)



\bibitem{three2}M. B. Kenmoe, H. Phien, M. Kiselev, L. Fai, Effects of colored noise on Landau-Zener transitions: Two and Three-level systems, Phys. Rev. B {\bf 87}, 224301 (2013)


\bibitem{CDT}F. Grossmann, Coherent Destruction of Tunneling, Phys. Rev. Lett. {\bf 67}, 516 (1991)


\bibitem{Core-contour_plot}A. A. Boris and V. M. Krasnov, Quantization of the superconducting energy gap in an intense microwave field, Phys. Rev. B {\bf 92}, 174506 (2015)


\bibitem{CDT1}J. Gong, L. Morales-Molina, and P. Hänggi, Many-Body Coherent Destruction of Tunneling, Phys. Rev. B Condens. Matter, {\bf 103}, 133002 (2009)


\bibitem{CDT2}G. Della Valle, M. Ornigotti, E. Cianci et al, Visualization of coherent destruction of tunneling in an optical double well system, Phys. Rev. Lett. {\bf 98}, 263601 (2007)


\bibitem{CDT3}S. Mukherjee, M. Valiente, N. Goldman et al, Observation of pair tunneling and coherent destruction of tunneling in arrays of optical waveguides, Phys. Rev. A {\bf 94}, 053853 (2016)















\end{thebibliography}


\end{document}

%%% Local Variables:
%%% mode: latex
%%% TeX-master: t
%%% End:

