\begin{figure}[t]
\setlength{\abovecaptionskip}{7pt}
\setlength{\belowcaptionskip}{0pt}
	\centering
% 	\subfigure[MINE (PSNR:14.9)]{  % for AAAI
	\subfloat[MINE (PSNR:14.9)]{
%			\centering
			\includegraphics[width=0.23\textwidth]{figure/intro/DJI_20200223_163206_598_0_MINE.png}
%			\label{subfig:pixelnerf}
	}\subfloat[MINE (depth)]{
%			\centering
			\includegraphics[width=0.23\textwidth]{figure/intro/MINE_disp.png}
%			\label{subfig:mpi}
	}
	\\[-3mm]
	\subfloat[Ours (PSNR:17.0)]{
%			\centering
			\includegraphics[width=0.23\textwidth]{figure/intro/DJI_20200223_163206_598_0_ours.png} 
	}\subfloat[Ours (depth)]{
%			\centering
			\includegraphics[width=0.23\textwidth]{figure/intro/ours_disp.png}
	}
	\caption{Comparison with state-of-the-art methods. (a-b) RGB and depth rendering results of  \cite{Li_2021_ICCV2}. It produces many blurs and floats in the occluded regions and at the object/depth edges. 
	(c-d) Our method employs a joint rendering mechanism that preserves more image details and predicts sharp depth edges.}
	\label{fig:performance_illustration}
\end{figure}