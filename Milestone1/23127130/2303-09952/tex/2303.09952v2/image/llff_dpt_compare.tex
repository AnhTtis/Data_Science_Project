\begin{figure*}[th!] \centering
\setlength{\abovecaptionskip}{4pt}
\setlength{\belowcaptionskip}{-8pt}
\small
\setlength{\tabcolsep}{1pt}
\resizebox{1.0\textwidth}{!}{
\begin{tabular}{ccccc}
     RGB & MINE & PixelNeRF & Ours (w/o $\mathcal{L}'_d$) & Ours \\
     \includegraphics[width=0.19\textwidth]{figure/llff dpt compare/fern.png} & 
     \includegraphics[width=0.19\textwidth]{figure/llff dpt compare/fernMINE} &
     \includegraphics[width=0.19\textwidth]{figure/llff dpt compare/fern_pxn.png}& 
     \includegraphics[width=0.19\textwidth]{figure/llff dpt compare/fernnodpt.png} &
     \includegraphics[width=0.19\textwidth]{figure/llff dpt compare/ferndpt.png} 
     \\
     \includegraphics[width=0.19\textwidth]{figure/llff dpt compare/room.png} &
     \includegraphics[width=0.19\textwidth]{figure/llff dpt compare/roomMINE.png} &
     \includegraphics[width=0.19\textwidth]{figure/llff dpt compare/room_pxn.png}& 
     \includegraphics[width=0.19\textwidth]{figure/llff dpt compare/roomnodpt.png} &
     \includegraphics[width=0.19\textwidth]{figure/llff dpt compare/roomdpt.png} \\
\end{tabular}
}
    \caption{Effects of the pseudo depth loss. Even without the depth teacher, our method can achieve better depth maps compared with MINE and PixelNeRF. The depth teacher with pseudo depth loss further improves the quality of the depth rendering.} 
    % \vspace{-8pt}
    \label{fig:llff_depth}
\end{figure*} 