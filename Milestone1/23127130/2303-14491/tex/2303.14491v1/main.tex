\documentclass[sigconf]{acmart}

\AtBeginDocument{%
  \providecommand\BibTeX{{%
    \normalfont B\kern-0.5em{\scshape i\kern-0.25em b}\kern-0.8em\TeX}}}

\setcopyright{acmlicensed}
\copyrightyear{2023} 
\acmYear{2023} 
\acmDOI{10.1145/3544548.3580701}

\acmConference[CHI '23]{Proceedings of the 2023 CHI Conference on Human Factors in Computing Systems}{April 23--28, 2023}{Hamburg, Germany}
\acmBooktitle{Proceedings of the 2023 CHI Conference on Human Factors in Computing Systems (CHI '23), April 23--28, 2023, Hamburg, Germany}

\acmPrice{15.00}
\acmISBN{978-1-4503-9421-5/23/04}

\usepackage{xcolor}
\usepackage{subcaption}
\usepackage{microtype}
\usepackage{array}
\usepackage{enumitem}
\usepackage{colortbl}
\usepackage{graphicx,calc}
\usepackage{balance}
\usepackage[utf8]{inputenc}
\usepackage{multibib}
\newcites{supp}{Case Example List}

\newcolumntype{\$}{>{\global\let\currentrowstyle\relax}}
\newcolumntype{^}{>{\currentrowstyle}}
\newcommand{\rowstyle}[1]{\gdef\currentrowstyle{#1}%
  #1\ignorespaces
}

\newcommand{\ming}[1]{\textcolor{red}{\textbf{MF:} #1}}

\def\markup{0}
\if\markup 1
\usepackage{soul}
\newcommand{\rv}[1]{{\leavevmode\color{blue}#1}}
\else
\newcommand{\rv}[1]{#1}
\newcommand{\st}[1]{}
\fi

\definecolor{symbolc}{HTML}{E60012}
\definecolor{camerac}{HTML}{28A7E1}
\definecolor{soundc}{HTML}{23AC38}
\definecolor{bangc}{HTML}{EA5413}
\definecolor{footagec}{HTML}{F8B62B}
\definecolor{endingc}{HTML}{956134}

% \textcolor{symbolc}{Symbolism and Metaphor}

\newcommand{\symbolandmetaphor}[1]{\textcolor{symbolc}{Symbolism and Metaphor #1}}

\newcommand{\cameraeye}[1]{\textcolor{camerac}{Camera Eye #1}}

\newcommand{\creativesound}[1]{\textcolor{soundc}{Creative Sound #1}}

\newcommand{\bigbang}[1]{\textcolor{bangc}{Big Bang #1}}

\newcommand{\oldfootage}[1]{\textcolor{footagec}{Old Footage #1}}

\newcommand{\myendingfirst}[1]{\textcolor{endingc}{Ending First #1}}

\definecolor{fullc}{HTML}{4081BC}
\definecolor{exclamationc}{HTML}{FF681D}
\definecolor{questionc}{HTML}{FFC91D}
\definecolor{ellipsisc}{HTML}{40916A}


\newcommand{\etal}{et~al.\ }

\newlength\myheight
\newlength\mydepth
\settototalheight\myheight{Xygp}
\settodepth\mydepth{Xygp}
\setlength\fboxsep{0pt}
\newcommand*\inlinegraphics[1]{%
  \settototalheight\myheight{Xygp}%
  \settodepth\mydepth{Xygp}%
  \raisebox{-\mydepth}{\includegraphics[height=\myheight]{#1}}%
}
%%
%% Submission ID.
%% Use this when submitting an article to a sponsored event. You'll
%% receive a unique submission ID from the organizers
%% of the event, and this ID should be used as the parameter to this command.
%%\acmSubmissionID{123-A56-BU3}

%%
%% For managing citations, it is recommended to use bibliography
%% files in BibTeX format.
%%
%% You can then either use BibTeX with the ACM-Reference-Format style,
%% or BibLaTeX with the acmnumeric or acmauthoryear sytles, that include
%% support for advanced citation of software artefact from the
%% biblatex-software package, also separately available on CTAN.
%%
%% Look at the sample-*-biblatex.tex files for templates showcasing
%% the biblatex styles.
%%

%%
%% The majority of ACM publications use numbered citations and
%% references.  The command \citestyle{authoryear} switches to the
%% "author year" style.
%%
%% If you are preparing content for an event
%% sponsored by ACM SIGGRAPH, you must use the "author year" style of
%% citations and references.
%% Uncommenting
%% the next command will enable that style.
%%\citestyle{acmauthoryear}

%%
%% end of the preamble, start of the body of the document source.
\begin{document}

%%
%% The "title" command has an optional parameter,
%% allowing the author to define a "short title" to be used in page headers.
\title{Is It the End? Guidelines for Cinematic Endings in Data Videos}


\author{Xian Xu}
\affiliation{%
%\institution{Computational Media and Arts}
\institution{The Hong Kong University of Science and Technology}
\city{Hong Kong}
\country{China}
}
\email{xxubq@connect.ust.hk}

\author{Aoyu Wu}
\affiliation{
%\institution{Department of Computer Science and Engineering}
\institution{The Hong Kong University of Science and Technology}
\city{Hong Kong}
\country{China}
}
\email{awuac@connect.ust.hk}

\author{Leni Yang}
\affiliation{
%\institution{Department of Computer Science and Engineering}
\institution{The Hong Kong University of Science and Technology}
\city{Hong Kong}
\country{China}
}
\email{lyangbb@connect.ust.hk}

\author{Zheng Wei}
\affiliation{%
%\institution{Computational Media and Arts}
\institution{The Hong Kong University of Science and Technology}
\city{Guangzhou}
\country{China}
}
\email{zwei302@connect.hkust-gz.edu.cn}

\author{Rong Huang}
\affiliation{%
%\institution{Computational Media and Arts}
\institution{The Hong Kong University of Science and Technology}
\city{Guangzhou}
\country{China}
}
\email{hr316811369@gmail.com}

\author{David Yip}
\affiliation{%
%\institution{Computational Media and Arts}
\institution{The Hong Kong University of Science and Technology}
\city{Guangzhou}
\country{China}
}
\email{daveyip@ust.hk}
\authornote{Corresponding author}

\author{Huamin Qu*}
\affiliation{%
%\institution{Department of Computer Science and Engineering}
\institution{The Hong Kong University of Science and Technology}
\city{Hong Kong}
\country{China}
}
\email{huamin@cse.ust.hk}
% \authornotemark{+}
% \authornote{Corresponding author}
\renewcommand{\shortauthors}{Xu et al.}

%%
%% The abstract is a short summary of the work to be presented in the
%% article.
\begin{abstract}
Data videos are becoming increasingly popular in society and academia.
Yet little is known about how to create endings that strengthen a lasting impression and persuasion.
To fulfill the gap,
this work aims to develop guidelines for data video endings by drawing inspiration from cinematic arts.
%, which have many creative and impactful cinematic endings.
To contextualize cinematic endings in data videos,
111 film endings and 105 data video endings are first analyzed to identify four common styles using the framework of ending punctuation marks.
~\rv{We conducted expert interviews (N=11) and formulated 20 guidelines for creating cinematic endings in data videos.
To validate our guidelines, we conducted a user study where 24 participants were invited to design endings with and without our guidelines, which are evaluated by experts and the general public. The participants praise the clarity and usability of the guidelines, and results show that the endings with guidelines are perceived to be more understandable, impressive, and reflective.}
%~\rv{We conducted expert interviews (N=11) and formulated 20 guidelines for creating cinematic endings in data videos.
%To validate our guidelines,
%a user study is conducted where 24 participants are invited to design endings with and without our guidelines, which are evaluated by experts and the general public. The participants praise the clarity and usability of the guidelines, and results show that the endings with guidelines are perceived to be more understandable, impressive, and reflective.}
\end{abstract}
%%
%% The code below is generated by the tool at http://dl.acm.org/ccs.cfm.
%% Please copy and paste the code instead of the example below.
%%
\begin{CCSXML}
<ccs2012>
   <concept>
       <concept_id>10003120.10003145.10011769</concept_id>
       <concept_desc>Human-centered computing~Empirical studies in visualization</concept_desc>
       <concept_significance>500</concept_significance>
       </concept>
 </ccs2012>
\end{CCSXML}

\ccsdesc[500]{Human-centered computing~Empirical studies in visualization}

%%
%% Keywords. The author(s) should pick words that accurately describe
%% the work being presented. Separate the keywords with commas.
\keywords{Visualization; Storytelling; Interview; Lab Study; Data Video; Guideline}
%\keywords{Visualization; Storytelling; Interview; Lab Study; Data Video; Guideline}
\begin{teaserfigure}
  \includegraphics[width=.9
  \textwidth]{punctuations.jpg}
 \caption{\rv{Four cinematic styles are adaptable to data video endings. The form and content of 111 film endings (left: Dataset of Cinematic Endings) and 105 data video endings are analyzed to identify four styles of data videos endings (right: four cinematic ending definitions) using the framework of punctuation marks, namely, \textit{full stop}, \textit{exclamation point}, \textit{question mark}, and \textit{ellipsis}. Each cinematic ending is shown through a film ending example and a data video ending example (middle).}}
 \Description{This figure shows that four cinematic styles are adaptable to data video endings. The form and content of 111 film endings (left: Dataset of Cinematic Endings) and 105 data video endings are analyzed to identify four styles of data videos endings (right: four cinematic ending definitions) using the framework of punctuation marks, namely, \textit{full stop}, \textit{exclamation point}, \textit{question mark}, and \textit{ellipsis}. Each cinematic ending is shown through a film ending example and a data video ending example (middle).}
  %\Description{This figure shows four styles of data video endings using the framework of punctuation marks, namely \textit{Full Stop}, \textit{Exclamation Point}, \textit{Question Mark}, and \textit{Ellipsis}. Those styles allow us to communicate the relevance between film endings and data video endings, thereby drawing inspiration to create cinematic endings in data videos.}
  \label{fig:punctuations}
  \vspace{-10px}
\end{teaserfigure}

%\received{20 February 2007}
%\received[revised]{12 March 2009}
%\received[accepted]{5 June 2009}

%%
%% This command processes the author and affiliation and title
%% information and builds the first part of the formatted document.
\maketitle

\section{Introduction}
Data videos, defined as custom motion graphics combining visual and auditory stimuli to promote a data story~\cite{amini2015understanding,segel2010narrative},
are becoming increasingly popular in society and academia.
They are widely shared and viewed on social platforms to communicate data insights and knowledge.
For example,
the data videos by Neil Halloran~\cite{neil:fallen,neil:simulation} have been watched more than 10M times on YouTube.
Leading media outlets such as ~\textit{The New York Times}, ~\textit{The Guardian}, and ~\textit{Vox} also increasingly craft data videos to broadcast information to broad audiences (e.g.,~\cite{Vox,nyt,guardian}).

The increasing prominence of data videos has sparked research interest in investigating how to design compelling, effective data videos.
Much research has focused on analyzing data videos in the wild to distill the design space of visual effects such as animation types~\cite{amini2016authoring} and animated transition designs~\cite{tang2020narrative,shi2021communicating}.
In addition to visual effects, another line of research studies the narrative structures of data videos, such as Cohn's visual narrative structure~\cite{cohn2013visual}~(i.e.,~\textit{Establisher}, ~\textit{Initial}, ~\textit{Peak}, and~\textit{Release})~\cite{amini2015understanding} and Freytag's pyramid structure~\cite{freytag1908freytag}~(i.e.,~\textit{Setting},~\textit{Rising-Climax}, and~\textit{Resolution})~\cite{pyramid2021}.
%\rv{To the best of our knowledge on the research of narrative structure, how to create effective endings of data videos remains unclear. Nevertheless, many researchers have studied narrative endings in literature~\cite{smith1968poetic, adamo1995beginnings} and film~\cite{neupert1995end}.}
%As the author and literature professor James Plath~\cite{James} says, 
%~\textit {``The best endings resonate because they echo a word, phrase, or image from earlier in the story, and the reader is prompted to think back to that reference and speculate on a deeper meaning.''} Therefore, it is vital for creating understandability, impression, and reflection of a theme at the end of a story. In existing data videos, the author of a data video would appear at the end to address the concluding statement verbally. 
%In addition to emphasizing key data patterns, the author could also make a commentary or express personal views behind the facts as a researcher, forward thinker, policy advocate, or all of the above. 
%\rv{However, ending forms of data videos still lacks systematic studies on effective communication. Gerson and Page~\cite{gershon2001storytelling} noted that,~\textit{"The ancient art of storytelling and its adaptation in film and video can now be used to efficiently convey information in our increasingly computerized world."}} Therefore,
%for more lasting impressions and stronger persuasion, thoughts and feelings should be conveyed powerfully with literal, visual, or even cinematic expressions~\cite{yip2020invisible}. \rv{Specifically,
%data visualization, including data videos, is not neutral. Previous work pointed out that designers have affective goals. For example, designers try to call to action at the end of data videos~\cite{lee2022affective}.
%Most aesthetic narratives engage emotions, especially for narrative endings~\cite{carroll2007narrative}.}
% To build on this narrative approach of studying data videos, Xu et al.~\cite{xu2022fromwow} integrates visual form with creative content by formulating 28 guidelines specifically on cinematic opening styles for data video. 
% Neil Halloran’s cinematic opening styles of all his data video, which were highly regarded as cinematic data video, were extensively analyzed in the previous study. 
% Creative cinematic style attracts attention and curiosity and it applies throughout Halloran’s inspirational data video from opening to ending. 
% This study further extends the previous study specifically to cinematic ending, which is equally important in creating attention, awareness, reflection and lasting memory of the work. 
Nevertheless, how to create effective endings of data videos, vital for understanding, impression, and reflection of a theme at the end of a story, remains unclear.
%Nevertheless, how to create effective endings of data videos,
%which is vital for creating understandability, impression, and reflection of a theme at the end of a story, remains unclear. 
Often, the author of a data video would appear at the end to address the concluding statement verbally. 
Besides emphasizing key data patterns, the author could also make a commentary or express personal views behind the facts as a researcher, forward thinker, policy advocate, or all of the above~\cite{heyer2020pushing}.
%~\rv{%In particular, ending is the last impression of information from film and data videos delivered to audiences.
%A good ending leaves the audience with a more memorable impression and invites the audience to think and reflect on the subject matter.
%Note also that data videos as a popular form of data visualization are not neutral, as designers have affective goals~\cite{lee2022affective}. For example, designers are trying to call to action at the end of data videos.
~\rv{However, data visualization, including data videos, is not neutral. Previous work pointed out that designers have affective goals. For example, designers try to call to action at the end of data videos~\cite{lee2022affective}.
Most aesthetic narratives engage emotions, especially for narrative endings~\cite{carroll2007narrative}.}
~\rv{Gerson and Page~\cite{gershon2001storytelling} noted that,~\textit{"The ancient art of storytelling and its adaptation in film and video can now be used to efficiently convey information in our increasingly computerized world."}}
%In particular, A data video must have either an open or closed Ending~\cite{cao2020examining}. The ending is the last impression of information from film and data videos delivered to audiences.
%Amini et al.~\cite{amini2015understanding} also pointed out the importance of the ending as \textit{Release} in data videos because it takes more time to design, and it contains~\textit{New Fact, Repetition, Take Away, and Question}. 
%Furthermore, %Yang et al.~\cite{pyramid2021} decomposed the \textit{Release} into four narrative patterns (i.e., \textit{Recap}, \textit{Predicting the future}, \textit{Echoing the beginning}, \textit{Next steps}). 
%A good ending leaves the audience with a more memorable impression and invites the audience to think and reflect on the subject matter.}
%The ending of a data video critically influences its impression as the ending is either understandable, impressive, and reflective to evoke the audience's thoughts and feelings or bores them to forget.}
%Most aesthetic narratives engage the emotions, espacaily for the narrative endings~\cite{carroll2007narrative}.
%However, 
Therefore,
for more lasting impressions and stronger persuasion, thoughts and feelings should be conveyed powerfully with literal, visual, or even cinematic expressions~\cite{yip2020invisible}.
% Visual language and expression are always more universal than verbal language~\cite{yip2020invisible}.
\rv{Studying ending} is challenging as~\rv{it} is always related to the beginning and middle parts of a coherent story in content and context. 
%It is a challenging task as the ending is always related to the beginning and middle part of a coherent story in both content and context. 
% It has everything to do with the beginning and 
As the author and literature professor James Plath~\cite{James} says, 
~\textit {``The best endings resonate because they echo a word, phrase, or image from earlier in the story, and the reader is prompted to think back to that reference and speculate on a deeper meaning.''}
Form and content are inseparable in an ending, in which all the narrative and visual elements come together to create the conclusion.
Thus, the ending cannot be summarized by its form of visual presentation alone.
% , as previous research primarily focused on it. 
Data storytellers need guidelines that integrate content and form in designing data video endings.
%Facing these challenges, there lack guidelines that integrate content and form to guide data storytellers in designing data video endings.
% This integrative approach of content and form makes this study challenging and different from previous studies on the related subject that largely focused on form of visual presentation in data videos. 
% That data videos have a relative short history makes it challenging to follow common research methodology,
% that is, 
% to analyze existing data videos to derive design guidelines.

To fill this gap,
one approach is to draw inspiration from films, the most mature audio-visual language technology with more than 100 years of history~\cite{segel2010narrative}.
%Characters to filmmakers are figuratively like data graphics to data storytellers. 
%Filmmakers tell their stories by visualizing the characters and their actions; likewise, data storytellers tell their stories by visualizing their data. 
%In other words, 
%data motion graphics can be viewed as characters in data videos. 
%Character transforms over time in a story, and so does data transform in a data story, but with different forms, shapes, and colors. 
%Besides, 
Visual expression has many common forms, such as camera shot and movement, color, sound and music, and editing that both filmmakers and data storytellers use. 
Thus,
some of the styles and techniques filmmakers do to tell their stories could be a source of inspiration for data storytellers to visualize the form, shape, and movement of the data graphics in relation to their space, time, and other data attributes~\cite{yip2020visual}. 
% Another evidence is the work of Xu~\ea~\cite{xu2022fromwow} that formulated 28 guidelines specifically on cinematic opening styles for data video.
% Their work extensively analyzed Neil Halloran’s cinematic opening styles of all his data videos.
% Creative cinematic style attracts attention, and curiosity and it applies throughout Halloran’s inspirational data video from opening to ending. 
% This proved the adaptation of cinematic styles to data videos.
% This study further extends the previous study specifically to cinematic ending, which is equally important in creating attention, awareness, reflection and lasting memory of the work. 

% For the sake of clarity, we use \textbf{cinematic ending style} to describe these common visual styles in film ending that involves wide range of expressive cinematic techniques. 

\begin{figure*}
  \includegraphics[width=\textwidth]{method.png}
  \caption{\rv{A survey of corpus analysis, an expert interview, and an evaluation are conducted in this work. The main objective of this research is to identify common cinematic styles and produce a set of guidelines for data designers to create endings with cinematic styles. In Study 1, a large corpus of films and data videos is explored to propose common ending styles adaptable to data videos. Four common styles represented by four punctuation marks, as well as the classic ending examples of films and data videos under each punctuation mark, which are used as materials for the next expert interviews, are classified. In Study 2, the guidelines behind each ending style are derived from expert interviews. In Study 3, a user study and a comparative study are conducted to validate the usability and effectiveness of the guidelines and ending styles.}}
  \label{fig:method}
   \Description{This figure shows an overview of the methodology in our work. A survey of corpus analysis, an expert interview, and an evaluation are conducted in this work. The main objective of this research is to identify common cinematic styles and produce a set of guidelines for data designers to create endings with cinematic styles. In Study 1, a large corpus of films and data videos is explored to propose common ending styles adaptable to data videos. Four common styles represented by four punctuation marks, as well as the classic ending examples of films and data videos under each punctuation mark, which are used as materials for the next expert interviews, are classified. In Study 2, the guidelines behind each ending style are derived from expert interviews. In Study 3, a user study and a comparative study are conducted to validate the usability and effectiveness of the guidelines and ending styles.}
   %\vspace{-18px}
\end{figure*}

In this work,
how to create effective endings of data videos with cinematic styles is studied.
To inform our research,~\rv{this work refers to previous work in data video structures~\cite{amini2015understanding, cao2020examining}, and} the ending of a data video is defined as \textit{the content and form of the final scene or moment of the story at the ending of data videos.}
~\rv{\autoref{fig:method} is an overview of the methodology in our work.}
First, the endings of 111 masterpiece films and 105 data videos are analyzed.
Studying the complexity of boundless forms of content and visual styles finds that using four common punctuation marks to end a sentence can visualize the different tones of an ending, which can alter the final meaning of a sentence and help illustrate the complex and dynamic relationships between content and form. 
In ~\autoref{fig:punctuations},
these punctuation marks provide the ending its last tone of voice, which defines the style and meaning of the ending. 
Then, an interview study is conducted with 11 participants from diverse backgrounds, including cinematographers, directors, film experts, and data visualization researchers.
The interview study results in a set of 20 guidelines.
To evaluate our guidelines,
~\rv{24 participants are further recruited to create data video endings with and without the guidelines, which six experts and 52 general audiences then rate}. 
The results show that participants can efficiently and effectively follow the guidelines to create a more understandable, impressive, and reflective ending. 
The endings with guidelines are also rated as more creative and cinematic by the experts. 
All of the participants praise the clarity and inspiration of the guidelines.
In conclusion,
our contributions are as follows:
\begin{itemize}[leftmargin=*]
    \item Four types of punctuation mark endings with cinematic styles adaptable to data videos are presented by analyzing 111 films and 105 data videos. 
    \item Twenty guidelines are derived by interview studies with six film experts and five data visualization experts.
    \item The effectiveness and usefulness of the guidelines for creating cinematic ending are demonstrated through a user study and a comparative study.
\end{itemize}



% Since data videos are also content with audio-visual elements and narratives,
% some of the styles in film arts can be applicable to the art of data videos.
% In film arts,
% the endings are usually not the final acts.
% Thus,
% they are not necessarily part of the climax or final showdown scene but instead can be in some examples depending on how the storyteller chooses to end the story,
% such as raising new questions, ...\aoyu{need more.}
% Thus,
% we use four dominant punctuation marks that are typically used in literal form to end a sentence as initial framework to study and categorize video endings.


% The punctuation marks allow us to further categorize xx data videos,
% thus bridging film arts and data videos.

% Much research has focused on analyzing data videos in the wild to summarize design space such as animated transition designs~\cite{tang2020narrative,amini2016authoring}, narratives~\cite{shi2021communicating},
% and story designs~\cite{yang2021design}.



% Correspondingly,
% there has been growing interest in investigating how to design and create compelling, affecting, and effective data videos.

% Despite the prominence of data videos,
% there exist few guidelines on designing effective data videos.



%The definition:

%An ending is inseparable part of the entire story. It has everything to do with the beginning and the middle part of a coherent story. The length of the ending of a film varies. It is often the final scene or moment in the final act of the film that concludes the story. However, the ending is not the entire final act. Therefore, it is not necessarily part of the climax or final showdown scene but there are always exceptions in some cases depending on how the storyteller chooses to end the story. In short, the ending is where the storyteller choose to end the story and sometimes concludes it with a final visual statement. May this conclusion a close-ended … happy … open-ended with no one fix interpretation … Sometimes the ending is an unexpected surprise and sometimes it is just an predictable or inevitable outcome of events  … sometimes it is also a moment that the storyteller or director ‘speaks directly’ to the audience through cinematic language. The ending of a film always sum up the story with the outcome of events. A good ending can leave audience thinking and discussing long after viewing, sometimes long afterwards … A powerful ending could leave audience speechless … tears … emotionally and profoundly touched by the outcome of event and let audience continue thinking and even talking about it for years with lasting memory.



\section{Related Work}
Our work is related to narrative visualization, data videos, and \rv{narrative} endings.

\subsection{Narrative Visualization}
The concept of narrative visualizations regarding the integration of data graphics into storytelling was formally introduced by Segel and Heer~\cite{segel2010narrative} in 2010. 
They identified seven common genres for narrative visualizations,
including \textit{magazine style, annotated
chart, partitioned poster, flow chart, comic strip, slide show, and film/video/animation.}
Since then,
researchers have explored additional genres such as \textit{data comics}~\cite{bach2018design, zhao2019understanding, kim2019datatoon, wang2021interactive}, \textit{timeline visualization}~\cite{brehmer2016timelines}, and \textit{data GIFs}~\cite{shu2020makes}.
Moreover, many studies deeply explored and studied the connection between some human perception concepts and visualization, such as visualization recognition and recall~\cite{borkin2015beyond}, embellishments in visualization~\cite{borgo2012empirical}, and anthropographics and visualization~\cite{boy2017showing, morais2020showing}.
Recently, Coelho et al.~\cite{coelho2020infomages} ~\rv{proposed ``infomages'' --- that embedded common data charts into thematic images that are related to the subjects of data (e.g., embedding a pie chart into a laptop in an image of a hacker using the laptop to represent the number of hacking incidents by region).}
%Recently, Coelho et al.~\cite{coelho2020infomages} identified ``infomages'' that include context to data visualization, which also presented the context and data visualization have the potential translation.

In addition, %to genres,
researchers seek to provide suggestions and guidelines on effective data storytelling from an interdisciplinary perspective~\cite{gershon2001storytelling}.
Many studies conduct experiments to surface psychological effects in data visualizations, such as the framing effect~\cite{hullman2011visualization}
% color affection~\cite{bartram2017affective},
and the curse of knowledge~\cite{xiong2019curse}.
Another line of research studies narrative visualization from the perspective of literature~\cite{lee2015more, amini2015understanding, pyramid2021}.
For example,
\rv{Lee et al.~\cite{lee2015more} derived the data storytelling process from data journalism literature and pointed out that the visualization community still lacks enough attention to the structure and sequence of compelling story pieces, including a beginning and an ending, which influences the reception of data stories.}
\rv{Beginnings and endings are the necessary terms in a narrative structure~\cite{cao2020examining}. However, existing research had more exploration about the beginnings of narrative visualization~\cite{segel2010narrative, xu2022fromwow, pyramid2021, amini2015understanding}. For instance, Segel and Heer~\cite{segel2010narrative} explored and discussed the design patterns of narrative visualization, which paid more attention to the beginning parts of the interaction of the story.
Nonetheless, the connection between beginnings and endings was noted in these works. For example, Yang et al.~\cite{pyramid2021} identified one of the narrative patterns for data story ending as \textit{Echoing the beginning}. Xu et al.~\cite{xu2022fromwow} derived one of the opening styles as \textit{Ending First}, which presented associating the opening and the ending as a common nonlinear narrative technique. As such, creating the data story ending can inspire its opening.
To the best of our knowledge, the endings in the community of HCI and VIS that are important and worthy of further research remain much underexplored. Little related work can be found: Amini et al.~\cite{amini2015understanding} decomposed the narrative structure of data videos into four categories and identified the importance of the ending as \textit{Release} because it takes more time to design, and concluded that the \textit{Release} contains~\textit{New Fact, Repetition, Take Away, and Question}. Yang et al.~\cite{pyramid2021} further decomposed the \textit{Release} into four~\rv{narrative patterns (namely, \textit{Recap}, \textit{Predicting the future}, \textit{Echoing the beginning}, and \textit{Next steps}).}}
%Amini et al.~\cite{amini2015understanding} explored the narrative structure of data videos
%and decomposed them into four categories,
%including \textit{Establisher}, \textit{Initial}, \textit{Peak}, and \textit{Release}.
%Yang et al.~\cite{pyramid2021} further decomposed the \textit{Release} into four types.
%Yang et al.~\cite{pyramid2021} further decomposed the \textit{Release} into four~\rv{narrative patterns (i.e., \textit{Recap}, \textit{Predicting the future}, \textit{Echoing the beginning}, \textit{Next steps}).}
%\rv{Specifically, existing research had more exploration about the beginnings of narrative visualization~\cite{segel2010narrative, brehmer2013multi, ragan2015characterizing, xu2022fromwow, pyramid2021, amini2015understanding}. For instance, Segel and Heer~\cite{gaut2010philosophy} demonstrated interactive examples to show the beginnings of interactive visualization. Especially studies showed some connection between beginnings and endings. For example, Yang et al.~\cite{pyramid2021} identified one of the narrative patterns for data story ending as \textit{Echoing the beginning}. Xu et al.~\cite{xu2022fromwow} derived one of the opening styles as \textit{Ending First}, which presented associating the opening and the ending as a common nonlinear narrative technique. As such, creating the data story ending can inspire its opening.
%However, the endings remain much underexplored that are important and worthy of further research~\cite{cao2020examining}.}
Compared with their research,
this work aims to study the common styles of data story endings that have been underexplored.
Specially,
a cinematic lens is taken by summarizing styles in film endings and analyzing how they might be applied to data videos.

\subsection{Data Videos}
As a popular genre for storytelling,
data videos have gained intense research interest over the last decade.
Researchers have developed a wide range of tools to ease the creation.
For example,
DataClips~\cite{amini2016authoring} allowed nonexperts to create and assemble data-driven clips.
%Calliope~\cite{shi2020calliope} automatically generated data stories and outputed a data video.
Gemini~\cite{kim2020gemini} and Canis~\cite{ge2020canis} provided declarative grammar to create animations.
\rv{Recently, Shin et al.~\cite{shin2022roslingifier} introduced a new genre of data-driven storytelling, namely, \textit{data presentation}, similar to data videos, and presented an approach of automatically generating a data fact, enabling users to improve the stories and produce animated visualization. Sun et al.~\cite{sun2022erato} demonstrated a method for producing data facts and visuals using the interpolation algorithm, which helps create data stories more efficiently through human-machine cooperation. Although these existing tools help (semi)-automatically create data stories, it still requires more studies to provide diverse visual and content designs to enrich the design templates and appropriate recommendations from authoring tools.}
%Inspired by these works, we found the value of providing visual and content design templates to improve the usability of these authoring tools in specific parts of data videos, such as the endings.}
In addition to authoring tools,
researchers have sought to distill the design space and propose design guidelines.
Much research focuses on low-level visual effects such as animation types~\cite{amini2016authoring} and animated transition designs~\cite{tang2020narrative,shi2021communicating}.
% However,
% visual effects are insufficient in representing the narrative components of data videos.
% As such,
Recent work further links visual effects with audience feedback, such as affection and attractiveness.
For example,
Lan et al.~\cite{lan2021kineticharts, lan2022negative} studied how animation could evoke different emotions in data videos.
Xu et al.~\cite{xu2022fromwow} summarized six common styles of video openings and formulated 28 guidelines to create an attractive opening with cinematic effects. 
%\rv{A data video must have either an open or closed Ending~\cite{cao2020examining}. Much more work has been dedicated to the beginnings~\cite{segel2010narrative, brehmer2013multi, ragan2015characterizing, xu2022fromwow, pyramid2021, amini2015understanding}. Therefore, }
The above study is extended by examining the endings of data videos with cinematic styles.
Critically,
the ending is content-related and thus cannot be studied by examining its visual representation alone.
This characteristic makes this study challenging and different from previous ones that largely focus on visual presentations but pay slight attention to narrative content.


%Recently, Shin et al. presented an approach of automatically generating a data fact, allowing users to refine the stories and create animated visualizations.
%Sun et al. introduced a method of using the interpolation algorithm to generate data facts and visualizations.

%narrative endings 尤其是ending和beginning ending在文学和电影中很重要。

\rv{\subsection{Narrative Endings}
Many researchers have studied narrative endings in literature and film. 
Early in the development of narratives, researchers have \nobreak noticed the important role that endings can play. Endings are considered a remarkable feature of narratives~\cite{smith1968poetic, carroll2007narrative, abbott2021cambridge}. 
To better understand the role of endings in the narrative, prior research has defined and discussed endings in different fields (e.g., poem~\cite{smith1968poetic}, novel~\cite{adamo1995beginnings}, and film~\cite{neupert1995end}) for a long study history. In terms of literature, Aristotle first proposed the ending concept in narrative and aimed to anatomize and conclude the precisely appropriate place for endings representation~\cite{aristotle2006poetics}.
Klauk et al.~\cite{klauk2016empirical} divided narrative endings into seven categories to understand the features of the text in narrative endings, which helped explore reader reaction to the narrative endings.
Noel Carroll further discussed that the ending is the result of all the pressing questions having been answered and a key element to make the story exciting. Specifically, to ensure impression of complete narrative, endings provide necessary conditions for some scenes of the stories such as the establishing scenes of a film or introductory paragraphs for a\break novel~\cite{carroll2007narrative}.}
%Aristotle first identified narratives as a beginning, a middle, and an ending, as well as defined endings in \textit{Poetics}~\, cite{aristotle2006poetics}: \textit{An end is that which itself naturally occurs, whether necessarily or usually, after a preceding event, but need not be followed by anything else.} 
%Researchers have deeply studied and discussed the narrative endings in literature and film.
%Researchers have intensely studied and discussed the narrative endings in many fields (e.g., poem~\cite{smith1968poetic}, novel~\cite{adamo1995beginnings}, and film~\cite{neupert1995end}) for a long study history.
%Aristotle first proposed the concept of narrative endings and aimed to anatomize and conclude the precisely appropriate place for endings representation~\cite{aristotle2006poetics}.
%In literature, the narrative endings of poems~\cite{smith1968poetic} and novels~\cite{adamo1995beginnings} have much exploration for a long study history. 
%In one case of literature, Klauk et al.~\cite{klauk2016empirical} divided narrative endings into seven categories to understand the features of the text in narrative endings, which helped explore reader reaction to the narrative endings.}

\rv{Similar to the popular discussion on the endings of literature, }
film endings are a cinematic art form that~\rv{attracts much} attention in the film industry and filmmaking research~\cite{giannetti1999understanding, kunkle2016cinematic}.
%How a film story ends can be a masterstroke of the film masters that bring their stories to a new level by expression with cinematic arts~\cite{kunkle2016cinematic}.
Related research starts with concluding types of endings in films such as open and closed, happy, and sad~\cite{neupert1995end, preis1990not}. 
%Recently, Yun~\cite{yun2018study} concluded the ending types of 900 films.
The unlimited potential in endings fueled films across the ages.
For example, films such as~\textit{Citizen Kane (1941)}~\cite{citizen}, \textit{Bicycle Thief (1948)}~\cite{bicycle}, and \textit{The 400 Blows (1959)}~\cite{400blows} are considered an artistic cornerstone in the history of the cinema for their open endings~\cite{cutting2016narrative}.
Previous work found that film endings could create and alter the meanings of the stories and the experiences of viewers~\cite{stone2002hope}.
Research in film has also been exploring how to create better film endings.
For example, King~\cite{king2007don} compared the effects of traditional endings and teaser endings on viewers' enjoyment of horror films and found that participants prefer traditional ones.
Berys et al.~\cite{gaut2010philosophy} discussed how narrative techniques in films could affect the audience's interpretations of an ending.
Research that systematically summarizes cinematography or narrative techniques at the ending of films is lacking, but those of discussions in the wild are many. 
Thus, we identified film techniques by analyzing 111 films and expert interviews, and we selected ones adaptable to data videos in this work.
\rv{After comparing film endings with data video endings, we found that they both help to convey insights from two perspectives. First, closed endings could help summarize and recap the essential insights. Second, open endings prompt the audience to think about the next step of the story or interact with the next data visualization, which could be useful in the ending designs of interactive visualization.
Thus, this paper investigates how to adapt narrative ending techniques, especially cinematic ending techniques, to data videos. To the best of our knowledge, our work is the first one that systematically explores the combination of cinematic storytelling and data video\break endings.}
%~\rv{Different punctuation marks can represent different effects of the endings. 
%Our study aims to derive a set of guidelines for creating data videos with cinematic endings.}
%Thus, we identify film techniques by analyzing 111 films and expert interviews, and we select ones adaptable to data videos. %More details are seen in Section 3.1 and Section 4.1.

%\subsection{Cinematic Endings}
%We introduce background knowledge about the endings of films and related film studies.
% ending的重要性,ending引起了电影界的各种讨论
%Film ending is a cinematic arts form that~\rv{attracts much} attention in the film industry and filmmaking research~\cite{giannetti1999understanding, kunkle2016cinematic}.
%How a film story ends can be a masterstroke of the film masters that bring their stories to a new level by expression with cinematic arts~\cite{kunkle2016cinematic}.
%Related research starts with concluding types of endings in films such as open endings and closed endings, happy endings, and sad endings~\cite{neupert1995end, preis1990not}. 
%Recently, Yun~\cite{yun2018study} concluded ending types of 900 films.
%The unlimited potential in endings fueled films across the ages.
%For examples, films like ~\textit{Citizen Kane (1941)}~\cite{citizen}, \textit{Bicycle Thief (1948)}~\cite{bicycle}, and \textit{The 400 Blows (1959)}~\cite{400blows}, are considered an artistic cornerstone in the history of the cinema for their open endings~\cite{cutting2016narrative}.
%Previous literature found that film endings could create and alter the meanings of the stories and viewers' experiences~\cite{stone2002hope}.
%Research in the film has also been exploring how to create better film endings.
%For example, King~\cite{king2007don} compared the effects of traditional endings and teaser endings on viewers' enjoyment of horror films and found that participants prefer traditional ones.
%Berys et al.~\cite{gaut2010philosophy} discussed how narrative techniques in films could affect the audience's interpretations of an ending.
%However, there is a lack of research that systematically summarizes cinematography or narrative techniques in the ending of films, but lots of discussions in the wild.
%~\rv{Different punctuation marks can represent different effects of the endings. 
%Our study aims to derive a set of guidelines for creating data videos with cinematic endings.}
%Thus, we identify film techniques by analyzing 111 films and expert interviews, and we select ones adaptable to data videos. More details are seen in Section 3.1 and Section 4.1.

%Narrative Endings
%\rv{There have been many research efforts to study narrative endings in literature and film. Endings are considered a significant feature of narratives~\cite{smith1968poetic, carroll2007narrative, abbott2021cambridge}. Aristotle first identified narratives as a beginning, a middle, and an ending as well as defined endings in \textit{Poetics}~\cite{aristotle2006poetics}: \textit{An end is that which itself naturally occurs, whether necessarily or usually, after a preceding event, but need not be followed by anything else.} 
%Researchers have deeply studied and discussed the narrative endings in literature and film.
%Researchers have deeply studied and discussed the narrative endings in many fields (e.g., poem~\cite{smith1968poetic}, novel~\cite{adamo1995beginnings}, and film~\cite{neupert1995end}) for a long study history.
%In literature, the narrative endings of poems~\cite{smith1968poetic} and novels~\cite{adamo1995beginnings} have much exploration for a long study history. 
%In one case of literature, Klauk et al.~\cite{klauk2016empirical} divided narrative endings into seven categories to understand the features of the text in narrative endings, which helped explore reader reaction to the narrative endings.}


%For example, Anz~\cite{} mentioned it is difficult for the reader to understand what has to happen in narrative endings. Subsequently, Klauk et al.~\cite{klauk2016empirical} divided narrative endings into seven categories to understand the feature for the text in narrative endings.}


% and films~\cite{neupert1995end}.}

%More on narrative closure
%The fact that a narrative may, but need not exhibit closure has been recognized many times, and closure is generally recognized as an important feature of narratives.
%\rv{Endings is considered as a significant feature od narratives~\cite{}.}

%Aristotle firstly defined endings in \textit{Poetics}: \textit{An end is that which itself naturally occurs, whether necessarily or usually, after a preceding event, but need not be followed by anything else.}

%Poetic closure: A study of how poems end
%Smith B H. Poetic closure: A study of how poems end[M]. University of Chicago Press, 1968.
%


% A film ending sets the final tone of voice and closure of a story. 
% Cinematic film ending is far more than just the typical happy closure ending. 
% For example, films with open endings require a mature, sophisticated, and thoughtful audience who can interpret the layers of meanings in the open endings~\cite{preis1990not}. 
% Famous films with open endings or no obvious endings, like ~\textit{Citizen Kane (1941)}~\cite{citizen}, \textit{Bicycle Thief (1948)}~\cite{bicycle}, and \textit{The 400 Blows (1959)}~\cite{400blows}, are considered an artistic cornerstone in the history of the cinema~\cite{cutting2016narrative}.
% Other open-ended films, such as \textit{The Graduate (1967)}~\cite{graduate} and \textit{Alice's Restaurant (1969)}~\cite{alice} were well received and generated much discussion among film critics and scholars for decades.
% 关于电影技术可以帮助film ending的理论:1.Cinematic Cuts: Theorizing Film Endings 2.Editing as Punctuation in Film
% https://vimeo.com/138829554  
% 3.Film Terms Glossary
% https://www.filmsite.org/filmterms9.html

% Film endings create and alter meanings and viewing experiences~\cite{stone2002hope}.
% Another line of research studies the audience reaction to the ending. 
% Film endings offer various forms of visual spectacle and enjoyment that can make us laugh, scared, cry, or angry. 
% Film endings can reveal messages within messages. 
% They can have hidden meanings or agendas open for discussion and interpretation, making interesting endings more interesting. 
% %Film ending is a cinematic art form. How a film story ends can be a masterstroke of the film masters through their storytelling and expression of cinematic arts~\cite{kunkle2016cinematic}.
% %A powerful ending can leave the audience a lasting impression and thoughts afterward~\cite{preis1990not}. 
% A powerful ending with stylistic visual, sound, and music elements can make a lasting impression and impact for thoughts and reflection for a long time after the story ends~\cite{preis1990not}. 
% To help visualize a large variety of film endings, film endings can be considered the final punctuation mark of cinematic language~\cite{kunkle2016cinematic,stone2002hope} through framing and editing and other cinematic techniques and styles. 
% Stone~\cite{stone2002hope} used four ending punctuation marks, which inspires this research's methodology to study cinematic ending styles. 
% In this work, we combine the punctuation marks to classify the film endings and customize their cinematic styles to provide adaptable guidelines for data video endings.

% In this section,
% we discuss four styles of cinematic endings that are summarized from films and data videos.
% We organize them using the framework of punctuation marks,
% including \textit{full stop}, \textit{question marks}, \textit{exclamation marks}, and \textit{ellipsis marks}.
% This framework is inspired by literature on film endings using ending punctuation marks to describe cinematic endings in films~\cite{schulz, brody2008punctuation, kunkle2016cinematic, stone2002hope, editing}.
% Specifically, we first used three common ending punctuation marks~\cite{14punctuation, guide}, including full stops, question, and exclamation marks.
% In iterative coding, we notice that another common punctuation mark, the ellipsis mark, can also be used at the end of a sentence to imply omission.
% Although ellipsis marks need to be formally written as …. or …? or …!~\cite{nunberg1990linguistics},
% they are considered as a common film style that implicitly expresses a theme or idea~\cite{ryan2015heretical, filmsite, filmnarrative, hollywoodlexicon, art}.
% Thus, we decide to add this punctuation mark into our analysis framework.
% In the following text, we discuss each cinematic ending style.
% For each style, we base our analysis and organization on a structured format, including the definition, the representative examples in films, and a case study examining data videos through this cinematic lens.

% Film ending is a cinematic art form. 
% How a film story ends can be a masterstroke of the film masters that  their stories and expression with cinematic arts~\cite{kunkle2016cinematic}.
% %Ending是电影领域经常被研究和讨论的对象。
% There are many different kinds of endings in films such as open endings and closed endings, happy endings and sad endings, etc\cite{neupert1995end}. 
% A film ending sets the final tone of voice and closure of a story. 
% Cinematic film ending is far more than just the typical happy closure ending. 
% For example, films with open endings require a mature, sophisticated, and thoughtful audience who can interpret the layers of meanings in the open endings~\cite{preis1990not}. 
% Famous films with open endings or no obvious endings, like ~\textit{Citizen Kane (1941)}~\cite{citizen}, \textit{Bicycle Thief (1948)}~\cite{bicycle}, and \textit{The 400 Blows (1959)}~\cite{400blows}, are considered an artistic cornerstone in the history of the cinema~\cite{cutting2016narrative}.
% Other open-ended films, such as \textit{The Graduate (1967)}~\cite{graduate} and \textit{Alice's Restaurant (1969)}~\cite{alice} were well received and generated much discussion among film critics and scholars for decades.

% "Is it over?"是观众看完电影后经常提出的一个问题或者感叹,而往往是ending中的cinematic styles引导着观众提出热烈的讨论和深刻的反思。

% 有一部分研究在讲ending 能够带来什么样的作用,和效果。他们也研究了电影用了什么技术能带来好的效果。
% 还有标点符号

%Compared with previous work, Xu et al.~\cite{xu2022fromwow} focused on the visual and sound form of cinematic opening to create 28 guidelines. However, there is a strong relationship between ending and content, so we combine content and form in this work.
%Specifically, 

%By summarizing and concluding the film endings, including the ~\textit{Best Movie Endings of All Time}~\cite{CineFix}. We customize their cinematic styles to provide adaptable guidelines for data video endings.
%和之前的研究cinematic opening的工作相比,前者主要focus在cinematic styles在visual和sound等形式进行研究和总结对应data videos的guidelines。然而,ending和content的关系很大,因此我们这次工作将content和form结合研究。Specifically, 我们结合了标点符号作为电影的结尾的形式进行分类,并且通过对10 Best Movie Endings of All Time等电影ending归纳和总结,试图将其cinematic styles为数据视频ending定制出适合的guidelines。
%Arguably, how to end a story is as important, if not more, as to how to begin a story? A good opening leaves the audience wanting what more to come in the rest of the story. 

%A powerful ending with stylistic visual, sound, and music elements can make a lasting impression and impact for thoughts and reflection for a long time after the story ends.  

%As designer Ellen Lupton compares design with storytelling in her book ‘Design is Storytelling’ (Lupton 2017), she quoted Don Norman’s art of emotion with reflective emotion as one of the important elements of design that intends to create a lasting impression and experience. Likewise, a powerful ending of a story can create this impact of thought through reflective emotion. 

% Another key source of our inspiration in film endings comes from CineFix - IGN Movies and TV, which is home for informational and entertaining movie and TV show content made by experts for the modern day cinephile Subscribe with over 3.8 million subscribers. IGN produces a very comprehensive review of about a hundred of film endings for our reference … the link has generated 262,069 views and 8K likes and 1,277 Comments since May 1, 2021




\rv{\section{Study 1: Understanding Cinematic Endings Styles}}
%\section{Cinematic Endings}
%Every story ends with a punctuation mark. 
%Different punctuation marks can represent different effects of the endings. 
%Our study aims to derive a set of guidelines for creating data videos with cinematic endings. 
% For this purpose,
% we adapt a framework of four essential punctuation marks used to visualize how films and data videos end and illustrate the rich history of film can also inspire that data stories.
% We start with analyzing a large corpus of films and data videos to identify common styles of cinematic endings.
\rv{To contextualize cinematic endings to data videos, a survey is conducted to explore a large corpus of films and data videos. We identified four types of punctuation marks with cinematic ending styles applicable to data videos. In this section, we first describe our methodology for identifying cinematic endings styles, and then elaborate on these ending styles through cinematic ending examples from films and data videos.}
%In this section,
%we describe our methodology and results - four types of punctuation marks with cinematic ending styles applicable to data videos.

%To contextualize cinematic endings to data videos, we conducted a survey to explore a large corpus of films and data videos and identify four types of punctuation marks with cinematic ending styles applicable to data videos.


%we start with analyzing a large corpus of films and data videos to identify common styles of cinematic endings.

%In this section,
%we describe our methodology and results - four types of cinematic ending styles applicable to data videos.

%To visualize all these variety of different story endings, we use four basic punctation marks to identify different cinematic endings across different visual media including films and data videos. 
 
%We use cinematic language … to visualize the rich and expressive forms of cinematic language ... we adapt a system of punctation marks that used to end a sentence to visualize how movie and data story ends and to illustrate that data story can also be inspired from the rich history of film.


\subsection{Methodology}
An iterative coding approach is adopted to analyze films and data videos.

\textbf{Dataset.}
Our dataset consists of 111 films and 105 data videos.
For films,
this work refers to the YouTube video \textit{Best Movie Endings of All Time} by CineFix - IGN Movies and TV~\cite{CineFix}.
CineFix is home to informational and entertaining movie and TV show content made by experts for the modern-day cinephile, and has about 3.8 million subscribers. 
It produces a comprehensive review of over 100 film endings for our reference and has generated more than 318k views, 9k likes, and 1.4k comments since May 1st, 2021. 
Some films with cinematic endings were professional inputs from our research team member, who has about 25 years of experience in film teaching and producing.
%\aoyu{source of data videos.}
For data videos, high-quality data videos were collected from previous narrative visualization studies~\cite{shi2021communicating,pyramid2021,xu2022fromwow,segel2010narrative,amini2016authoring, amini2015understanding}, some popular video platforms, and well-known news agencies that have produced data videos (e.g., YouTube~\cite{youtube}, Tencent Video~\cite{tecent}, Vox~\cite{Vox}, New York Times~\cite{nyt}, The Economist~\cite{economist}).~\rv{In particular, we used keywords such as ``data storytelling,'' ``data stories,'' and ``data-driven videos'' to search for data videos and subsequently process videos with a high number of views. All analyzed cases can be found in the supplemental materials.}

\textbf{Procedure.}
The authors were divided into two groups to analyze those films and data videos.
The film group consisted of a professor with about 25 years of experience in teaching films and another author with eight years of experience in film writing and producing film.
% Since they were not familiar with all films on the list,
% we invited three external film experts to 
The data video group included three authors with research experience in data visualization and human computer interaction.
%During the coding process,
%two groups held a meeting twice a week to bridge the gap between data videos and films.
~\rv{Two groups adopted thematic analysis~\cite{braun2006using} to code the ending styles of data videos and films separately, and held meetings twice a week during the coding to bridge the gap between data videos and films.
The goal is to find a set of styles for classifying and describing the endings of films and data videos.
To theorize the classification,
the groups discussed and summarized the characteristics of film and data video endings in terms of content and form for each style.}
\rv{Initially, our classification of ending styles was inspired by the literature on film endings that use punctuation marks to describe cinematic endings in films~\cite{schulz, brody2008punctuation, kunkle2016cinematic, stone2002hope, editing}.
First, three common ending punctuation marks~\cite{14punctuation, guide} were used, namely, \textit{full stop}, \textit{exclamation point}, and  \textit{question mark}.
During iterative coding, another common punctuation mark, the \textit{ellipsis}, can also be used at the end of a sentence to imply omission.
Although \textit{ellipses} need to be formally written as …. or …? or …!~\cite{nunberg1990linguistics},
they are considered a common film style that implicitly expresses a theme or an idea~\cite{ryan2015heretical, filmsite, filmnarrative, hollywoodlexicon, art}.
Thus, this punctuation mark was added to our analysis framework.
After rounds of coding and discussion, new ending styles were no longer found toward the end of analyses. This result suggested that our codes almost reached saturation, and the groups stopped analyzing new videos. Differences were resolved by discussions. Finally, the groups reached a consensus on four cinematic endings styles using four punctuation\break marks.}

%To theorize the classification,
%the groups discussed and summarized the characteristics of film and data video endings in terms of content and form for each style.
%Specially,
%we discussed techniques in film endings that were applicable to data videos and notable techniques in data video endings.

%We organized them using the framework of punctuation marks,
%including \textit{full stop}, \textit{exclamation points}, \textit{question marks}, and \textit{ellipsis}.

%Since the film group was not familiar with all listed films,
%We also invited three external film experts to judge our codes of film endings.


%我们作者分成了两个组,一个是由超过25年电影教学经验和8年电影编剧和制作经验的两位作者组成的电影组,和一个由三位具有超过五年 的数据可视化和HCI经验的作者数据视频组。这两个组分别对超过X(200+)部具有优秀ending的电影【来自于Youtube list-Best Movie Endings of All Time 】和超过100部(X)的数据视频。在电影组的coding中,因为寻在list中的电影不一定作者中看过,所以外请了三位电影专家来帮忙进行列表中一百多部的电影进行对标点符号的判断。两组在coding过程中进行每周两次的会议,对每个标点符号对应的电影和数据视频发现的特征进行开会总结,寻找两者在意识和形态都能达到一致的情况,尝试找出电影结尾中能够被数据视频应用的手法,以及数据视频本身现在已经存在的优质结尾手法。直至最后发现每个标点符号下可以对应的电影和数据视频基本已经明确和饱和后停止继续的新增电影和数据视频。

\rv{\subsection{Results and Analysis}}
%\subsection{Cinematic Ending Styles and Punctuation Marks}
%In this section,
%we discuss four styles of cinematic endings that are summarized from films and data videos.
%We organize them using the framework of punctuation marks,
%including \textit{full stop}, \textit{exclamation points}, \textit{question marks}, and \textit{ellipsis}.
%This framework is inspired by the literature on film endings that use punctuation marks to describe cinematic endings in films~\cite{schulz, brody2008punctuation, kunkle2016cinematic, stone2002hope, editing}.
%Specifically, we first used three common ending punctuation marks~\cite{14punctuation, guide}, including \textit{full stop}, \textit{exclamation points}, \textit{question marks}.
%In iterative coding, we notice that another common punctuation mark, the ellipsis, can also be used at the end of a sentence to imply omission.
%Although ellipsis marks need to be formally written as …. or …? or …!~\cite{nunberg1990linguistics},
%they are considered a common film style that implicitly expresses a theme or idea~\cite{ryan2015heretical, filmsite, filmnarrative, hollywoodlexicon, art}.
%Thus, we decide to add this punctuation mark to our analysis framework.
In the following text,~\rv{our results on four cinematic ending styles are reported and elaborated using the framework of punctuation marks, namely, \textit{Full Stop as Closure}, \textit{Exclamation Point as Expression of Commentary}, \textit{Question Mark as Forward Thinking}, and \textit{Ellipsis as Open for Interpretation}.}
For each style, our analysis and organization are based on a structured format, including the definition, the representative examples in films, and a case study examining data videos through this cinematic lens.


\begin{figure}[!ht]
  \centering
      \includegraphics[width=\linewidth]{fullstop.jpg}
    \caption[]{\rv{Endings of \textit{The Shawshank Redemption (1994)}~~\protect\cite{shawshank}, \textit{Babe (1995)}~~\protect\cite{babe}, \textit{Casablanca (1942)}~~\protect\cite{casablanca}, and \textit{Modern Times (1936)}~~\protect\cite{modern} using \textit{full stop} with the film techniques of a zoom out camera movement, wide shot, and the iris effect to achieve closure}}
    \Description{This figure shows the endings of four film endings of \textit{The Shawshank Redemption (1994)}~\cite{shawshank}, \textit{Babe (1995)}~\cite{babe}, \textit{Casablanca (1942)}~\cite{casablanca}, and \textit{Modern Times (1936)}~\cite{modern} using full stop with the film techniques of a zoom out camera movement, wide shot, and the iris effect to achieve closure.}
    \label{fig:fullstop}
   % \Description{This is the opening sequence of Full Metal Jacket (1987) using Symbolism and Metaphor. The famous haircutting opening scene in Stanley Kubrick's Full Metal Jacket (1987), in which a group of young army recruits were shaved all their hair, symbolized the stripping of their individuality.}
  \end{figure}

% The fundamental punctuation marks used as terminal points are commonly full stop, question mark and exclamation mark(cite5,6).  Meantime, we also notice there is another common punctuation mark that can also be used at the end of a sentence to imply or omit. In text, when ellipsis is placed at the end of a sentence, it must be followed by one of the three common terminal punctuation marks. Ellipsis is a punctuation mark of omission and implication and when it is used at the end of a sentence, it should be written respectively as …. or …? or …!.(cite7)

% Therefore, it can be argued that ellipsis is more than just a punctuation mark of omission and it could also carry a soft tone of voice that can suggest a closure, question mark or exclamation when it is used to implicit any one of the tones of voice in different ending context. Therefore, in order to cover and not to omit diverse forms and content of film and data video endings with clearer category, we decide to add this punctuation mark as the 4th punctuation mark into our analysis framework and coding process.


% These punctuation marks provide an intuitive categorization of different ending styles,
% enabling us to connect and group endings in films and data videos.
% A good ending in films not only concludes the subject matter but also gives the final ending punch to the audience for various emotional and even physical responses. 
% In a similar way,
% ending in data videos can be done in a more visual and cinematic style as an addition to the typical stand-and-deliver straightforward approach seen in many existing ones.



% with representative examples in films and a case study about how data videos can draw inspirations from those films.
\subsubsection{\textcolor{fullc}{Full Stop as Closure}}
A full stop or period used at the ending of a sentence signifies completing an action or meaning.
Therefore, it functions as a close-ended punctuation mark. 
In the literary sense, this punctuation mark can be interpreted as things that have been resolved or questions that have been answered with closure.
It can also suggest the closure of a causality loop or complete connection between past and present. 
As expected, full stops appeared to be most common in the corpus of films (45/105) and data videos (37/111).

%1.数据视频中句号的ending最多(45/105),其实和数据视频的ending通常需要进行总结、清晰地传达数据发现和结论有密切联系。
%1.电影中句号的ending最多(37/111),因为一个故事讲完了,主人公任务完成,这是电影中最常见和完整的一个戏剧故事结构。



\textbf{Films.}
Ending with a zoom out camera movement, wide shot, or montage are examples of cinematic styles that can signify and express the sense of closure and full stop of a film story.
For example, ~\textit{The Shawshank Redemption (1994)}~\cite{shawshank}, as shown in~\autoref{fig:fullstop} (a), ends in a vast zoom out to show a wide shot of the two main characters as free men walking by the sea, a new world that is much larger than their prison cell after years of imprisonment. 
Their freedom is symbolized by this great zoom out and wide shot as the closing shot of the film. 
In some cases, the camera's movements are accompanied by the characters' movements.
% Characters of a film are like motion graphics in data video.
% Their movements can change the shot dynamic with or without camera movement.
In the last shot of~\textit{Casablanca (1942)}~\cite{casablanca}, as shown in~\autoref{fig:fullstop} (c), as the two characters are left to walk away from the camera, the shot size changes from medium to wide as the characters are walking away to begin their ``beautiful friendship." 
Likewise, the ending shot in ~\textit{Modern Times (1936)}~\cite{modern}, as shown in~\autoref{fig:fullstop} (d), shows Charles Chapman is walking away with his girl in a static shot that changes from a medium shot to a wide shot, which signifies the sense of completion or closure in a film story.
Another common editing technique in a full stop ending is the iris effect that serves like a zoom in to close up to focus on the subject, such as the ending shot of~\textit{Babe (1995)}~\cite{babe}, as shown in~\autoref{fig:fullstop} (b), that features the small pig as the final hero of the story. 
Finally, montage is often used to close a film to show the causal relation between action and reaction as the conclusion of the plot. 
For example, the ending of the final montage of~\textit{Schindler's List (1993)}~\cite{schindler} ~(\autoref{fig:fullstop2}) shows the transitions from fiction to reality by showing the real people whom Schindler saved in real life.


% The ending montage of famous films such as~\textit{Dr. Strangelove (1964)}~\cite{strangelover} and ~\textit{Cinema Paradiso (1998)}~\cite{cinema} respectively show this causal relation between past and present in the story. 

\begin{figure}[!ht]
  \centering
      \includegraphics[width=\linewidth]{fullstop2.jpg}
    \caption{\rv{Ending of \textit{Schindler's List (1993)}~\protect\cite{schindler} using \textit{full stop} through the film technique of montage as the conclusion of the story}}
    \Description{\rv{This figure shows the ending of \textit{Schindler's List (1993)}~\cite{schindler} using \textit{full stop} by the film technique of montage as the conclusion of the story.}}
    \label{fig:fullstop2}
   % \Description{This is the opening sequence of Full Metal Jacket (1987) using Symbolism and Metaphor. The famous haircutting opening scene in Stanley Kubrick's Full Metal Jacket (1987), in which a group of young army recruits were shaved all their hair, symbolized the stripping of their individuality.}
      %\vspace{-10px}
 \end{figure}

%The example of zoom out cinematic style can appear in film ending with a full stop. 
%The film The Shawshank Redemption (1995) ends in a big zoom out as the two main characters, after years of imprisonment, have become free men and unit in their new free world now much bigger than their prison cell. 
%This concept is visualized through this big zoom out wide shot as the closing shot of the film. 

%On the contrary, a long zoom in to an extreme close up can be seen in film such as War of the Worlds (2005) which shows the true cause of the end of the alien invasion was in fact the bacteria on earth. 
%This continuous long zoom in shot reveals the hidden truth beneath the surface. 
%This ending also echoes the opening in which our civilization on earth is being watched from outer-space by alien invaders through a long zoom in.

%The interplay of sights and sounds as cinematic styles can have many forms and shapes. 
%Characters of a film are like motion graphics in data video.
%Their movement can change the shot dynamic with or without camera movement.

%In the case of full stop as closure, an iris effect can serve as zoom in to close up to focus on the subject such as the ending shot of Babe (1995) that features the little pig as the final hero of the story. 
%In the last shot of Casablanca (1942), as the two characters are left to walk away from the camera, the shot size changes from medium to wide as the characters are walking away to begin their ``beautiful friendship''.
%Likewise, the ending shot in Modern Times (1936) showing Charles Chapman is walking away with his girl in a static shot that changes from medium shot to wide shot, which signifies sense of completion or closure in a film story.

%There are many cinematic style that can draw closure in a film. 
%Another cinematic style to close a film can be through montage showing the casual relation between action and reaction as the final conclusion of the plot. 
%The ending montage of famous films such as Dr. Strangelove (1964) and Cinema Paradiso (1998) respectively show this casual relation between past and present in the story. 
%The ending of Schindler’s List (1993)'s final montage show more by transiting from fiction to reality by showing the real people who were saved by Schindler in real life.
%Ending on zoom in/out camera movement, wide shot or final montage are examples of cinematic styles that can signify and express sense of closure in a film story.


%Most data videos in our corpus \aoyu{(xxx\%)} end with a full stop,
%indicating the end of the narrative.
%Their ends often give clear and direct vital messages such as the most significant data pattern, an ultimate suggestion, or a call to action.
\begin{figure}[!ht]
  \centering
  \begin{minipage}[b]{\linewidth}
      \includegraphics[width=\textwidth]{fullstop-datavideos.jpg}
    \caption{\rv{Data video ending of \textit{Stock Market Crash in 28 years}~~\protect\cite{stock} using \textit{full stop} through providing clear, direct vital messages}}
    \Description{This figure shows the data video ending of \textit{Stock Market Crash in 28 years}~\cite{stock} using \textit{full stop}through providing clear, direct vital messages.}
    \label{fig:fullstop-datavideos}
   % \Description{This is the opening sequence of Full Metal Jacket (1987) using Symbolism and Metaphor. The famous haircutting opening scene in Stanley Kubrick's Full Metal Jacket (1987), in which a group of young army recruits were shaved all their hair, symbolized the stripping of their individuality.}
  \end{minipage}
     \vspace{-20px}
  \end{figure}


\textbf{Data Videos.} The endings of full stop often provide clear, direct vital messages such as the most substantial data pattern, an ultimate suggestion, or a call to action.
However,
they lack a solid, concise summary for emphasizing the main point.
For example,
the data video \textit{Stock Market Crash in 28 years}~\cite{stock}, as shown in~\autoref{fig:fullstop-datavideos} (a), takes a first-person viewpoint for visualizing a line chart about the stock market index,
engaging audiences in experiencing the ups and downs of the stock market on a roller coaster ride.
The video proceeds by switching between different viewpoints to make audiences visually immersed.
Similarly,
it could end with cinematic techniques to render a cogent summary of data and provoke reflections.
For instance, it could apply a smooth transition from the close-up shot at the last time point in the chart to a medium shot to a wide shot to review the overall fluctuating trend of the stock market index.
Instead,
the video only presents a plain subtitle, ``Beware of the risks in the stock market,'' as shown in~\autoref{fig:fullstop-datavideos} (b), so that audiences might overlook the critical data insights.
This case demonstrates that data videos have enormous potential for improvements by learning from film arts,
underscoring the importance of this research.



%反观数据视频,有许多时候在结尾处其实已经将故事讲述完毕,任务完成,但是却恰恰缺乏了一个对ending进行更有力而清晰的总结,以完成句号结尾所本质带有的肯定、完成和结束之意。比如在数据视频《80秒看中国股市28年跌宕起伏 这几次股灾你躲过了秒?》stock market crash in 28 years之中,以老张作为第一人称叙事,讲述了其28年股民生涯像坐过山车一般跌宕起伏,数据视频的过程中运用了主观视角和客观视角的转换让观众能在视觉上沉浸其中。但是结尾处本可以通过cinematic的方法让结论变得更有力和让人反思时,却只是轻描淡写地用字幕打出“股市有风险,投资需谨慎”的insight,很容易让观众错过数据信息的最核心发现。在数据视频的句号的结尾,其实有很多可以学习电影结尾的空间,但是我们现有的数据视频还没能学习到相应的方法,这也体现出了我们研究的重要性。
%对于句号结尾的数据视频stock market crash in 28 years的ending为了凸显出“股市有风险,入市需谨慎”的insight,可以尝试在video的中间部分将股市跌宕起伏的折线图变成一个人在作画的形式,当没有到达某个时间点的时候并不知道股市会涨还是会跌,结尾处可以用zoom out和wide shot来展现出28年股市涨跌的概括曲线,并且高亮出老张在故事中遇到几次股灾损失的点和线,并展现出老张账户的亏损状况,以真正让观众理解到“股市有风险,入市需谨慎”的发现,完成一个有说服力的句号结尾。

\subsubsection{\textcolor{exclamationc}{Exclamation Point as Expression of Commentary}}
Full stop and exclamation points are considered closed ended, which signifies the end of a complete action in a sentence. 
Specifically, an exclamation point encompasses a certain expression of emphasis, emotions, or feelings in the form of surprise or secret revealed at the end. 
This type of ending can be impressionistic and let the audience have the last ``wow'' moment or a new level of emotional reaction.
This work finds that exclamation points appeared to be moderately common in the film corpus (29/111), where the suspense was lifted in the film endings.
However, exclamation points were not common in data video endings (11/105) that might inspire future practices.
%The major difference between these two close-end marks is that the exclamation mark encompasses a certain expression of emphasis or emotion or feelings in the form of surprise or secret.
%The major difference between these two close-end marks is that the exclamation mark encompasses the certain expression of emphasis or emotion or feelings in the form of surprise or secret %being revealed at the very end. 
%May it be a pleasant surprise or a shocking surprise. This exclamation mark also communicates the various tone of voice from the author to the audience.
%Arguably to some extent, an exclamation mark could encompass both the tone of a question mark and full stop.

%Both full stop and exclamation mark are considered close-end punctuation mark which signifies the end of a complete action in a sentence. 
%The major different between these two close-end marks is that exclamation mark encompasses certain expression of emphasis or emotion or feelings in the form of surprise or secret being revealed at the very end. 
%May it be pleasant surprise or shocking surprise, this exclamation mark also communicates and expresses various tone of voice from the author to the audience.
%Arguably to some extent, exclamation mark could encompass both the tone of question mark and full stop.
%This type of ending can be impressionistic and let audience to have the last wow moment or new level of emotional reaction that comes at the end.



%4.电影中感叹号的结尾不算多也不算少(29/111),以感叹号结尾的电影很多时候是在结尾处揭开了全片设制的悬念来让观众叹为观止。但是出现频次也不算多的原因是,很多电影其实不会在影片最后才揭开谜底,往往是揭开谜底后还有一定的延续,所以不都是感叹号结尾。
%4.感叹号出现次数为(11/105),虽然不多不少,但是对于感叹号的ending的数据视频印象还是很深刻,感叹号结尾对给观众留下的深刻的数据印象有帮助。

\textbf{Films.}
The ending of the exclamation point is a strong cinematic expression of the author's thoughts, emotions, and feelings that appeal to the audience's emotional senses in various forms. 
Sometimes it can appear as a sharp irony in life and death such as the tragic endings in ~\textit{Life is Beautiful (1997)}~\cite{life} and ~\textit{Dancer in the Dark (2000)}~\cite{dancer}, as shown in~\autoref{fig:exclamation} (a). 
% Both are examples of tragic endings but with an ironic tone of an exclamation point at the end. 
%Sometimes this expression of commentary is visually stunning spectacles that go beyond logical reasoning or explanation but somehow appeal to our sensory and emotional senses and leave the audience with a `wow' reaction with different interpretations. 
~\rv{Sometimes this expression of commentary appeals to our sensory and emotional senses and leaves the audience with a ``wow'' reaction with different interpretations.} 
~\textit{Persona (1966)}~\cite{persona} and ~\textit{2001: A Space Odyssey (1968)}~\cite{2001}, as shown in~\autoref{fig:exclamation} (b) films open and end on visually stunning montage sequences as the visual expression of commentary with their powerful exclamation points at the end. 
In addition to using a montage, the filmmaker can also explain and sum up the theme visually in one shot in the ending scene. 
For example, the plot secret revealed at the end of~\textit{Pyscho (1960)}~\cite{pyscho}, as shown in~\autoref{fig:exclamation} (c) shocked many audiences. 
Combined with the effects of voice-over, sound, and music, one of the haunting last shots of the ending superimposed three layers of images with the smiling killer who has a split personality, the corpse of the mother, and the sunk car with the victim being pulled up by the police. 
Another expressive shot is in the ending of~\textit{Source Code (2011)}~\cite{soure}, as shown in~\autoref{fig:exclamation} (d) in which the filmmaker showed a shot with reflections of different shapes to conclude its theme about multiverses or parallel universes visually.
%The former examines human psychology internally with a reflexive view of the art of filmmaking, and the latter deals with the mysterious forces of the universe. 

\begin{figure}[!ht]
  \centering
      \includegraphics[width=\linewidth]{exclamation.jpg}
    \caption{\rv{Endings of \textit{Dancer in the Dark (2000)}~~\protect\cite{dancer}, \textit{2001: A Space Odyssey (1968)}~~\protect\cite{2001}, \textit{Pyscho (1960)}~~\protect\cite{pyscho}, and \textit{Source Code (2011)}~~\protect\cite{soure} using \textit{exclamation points} with stunning montage sequences and multiple collages of different visual shapes to sum up the theme visually}}
    \Description{This figure shows the endings of four films using exclamation points. There are endings of \textit{Dancer in the Dark (2000)}~\cite{dancer}, \textit{2001: A Space Odyssey (1968)}~\cite{2001}, \textit{Pyscho (1960)}~\cite{pyscho}, and \textit{Source Code (2011)}~\cite{soure} with stunning montage sequences and multiple collages of different visual shapes to sum up the theme visually.}
    \label{fig:exclamation}
   % \Description{This is the opening sequence of Full Metal Jacket (1987) using Symbolism and Metaphor. The famous haircutting opening scene in Stanley Kubrick's Full Metal Jacket (1987), in which a group of young army recruits were shaved all their hair, symbolized the stripping of their individuality.}
\end{figure}


%A strong cinematic expression of the author's thoughts, emotions and feelings that appeal to the audience's emotional senses in various forms. 
%Sometimes it can appear as a sharp irony in life and death in tragic but ironic endings in Life is Beautiful (1997) and Dancer in the Dark (2000). 
%Both are examples of tragic endings but with an ironic tone of an exclamation mark at the end. 
%The former one turns a tragic end into a heart-warming tale in order to save his child from being discovered by the Nazi.
%The latter one presents a shocking and disturbing end scene in which the female lead upon knowing her son can be healed during her execution, she sings just moment before she is hanged to be followed by the sudden dead silence.

%Sometimes this expression of commentary is visually stunning spectacles that go beyond logical reasoning or explanation but somehow appeal to our sensory and emotional senses and can leave audience with ‘wow’ reaction with different interpretations. 
%Both Persona (1966) and 2001: A Space Odyssey (1968) films open and end on visually stunning montage sequence as visual expression of commentary with their own powerful exclamation points at the end. 
%The former one examines human psychology internally with a reflexive view of the art of filmmaking and the latter one deals with the mysterious forces of the universe. 
%Both use visually stunning montage sequence as the filmmakers’ own visual expression of commentary statement as opening and closing. 

%Rather than using a montage sequence, sometimes filmmaker is able to explain and sum up the theme visually in one shot in the ending scene. 
%The plot secret revealed at the end of Pyscho (1960) shocked many audience to these days. 
%Combined with the effects of voice over and sound and music, one of the haunting last shots of the ending superimposed three layers of images with the smiling killer who has split personality, the dead corpse of the mother and the sunk car with the victim being pulled up by the police. 
%Another expressive shot in the ending scene of Source Code (2011), after mystery is solved and explained. 
%Filmmaker used visual, not in montage sequence but in the last shot of the film to conclude its theme and message … … to express the author’s view of parallel universes across time and space in one shot. These are examples of exclamation marks in film ending.

\textbf{Data Videos.}
Data videos in this category end with data insights that are surprising or unexpected to the audiences. 
A surprise is often made by compassion or contrast,
impressing audiences with a strong affection to strengthen their understanding of data.
Such style is exemplified by~\textit{America Dodge \$660 Billion in Taxes Each Year}~\cite{america}.
Its ending contrasts education funding with taxes, that is, the money lost with four percentage points more cheating is triple the Department of Education's budget for 2018.
This contrast uses the wide public concern on education funding to highlight the severe tax evasion, leaving a strong expression of commentary.
Another kind of contrast happens when a story reveals discordance or incongruity of facts, which uses irony to increase the dramatic effect.
In the video \textit{Weed is not More Dangerous than Alcohol}~\cite{weed}, the narrator questions the Federal government's unfairly stricter restriction on marijuana compared with alcohol despite the evidence that the overdose of alcohol caused the death of over 10000 times that caused by marijuana.  
The ironic mismatch between the level of danger and strength of restrictions challenges the audiences' expectations and motivates them to reflect on current policies. 
However, most of the content was directly delivered by the author talking about them in front of the camera without any visual treatment, indicating the importance of guidelines for cinematic endings.
%对于以Exclamation mark结尾的数据视频通常能给观众带来一些让人惊叹和意料之外的发现,或者是能通过对比和制造矛盾等方式来制造出惊叹的意思,这些手法都能让数据视频在结尾帮助主题进行进一步深入理解,尤其是通过制造惊叹、讽刺等加强印象制造反思。比如在数据视频 America Dodge $660Billion in Taxes Each Year的结尾处,提出了一个新的关于教育经费和税收的比较角度: Taxes lost with 4 PERCENTAGE POINTS MORE CHEATING($185.7B) 是 DEPT. OF EDUCATION FY 2018 BUDGET ($59B)的三倍。借助大众对教育经费的普遍认知来进一步通过比较突出关于美国偷税漏税严重的主题,已完成一个以感叹结尾的ending。
 % 数据视频weed is not more dangerous than alcohol,通过对比weed和alcohol的危害程度,以此来向政府禁止了行为提出疑惑,达到讽刺的效果,以调动观众的情绪。但是在这个data video里面主要是通过主持人旁白的讲述向政府行为提出疑惑来造成讽刺。其实用cinematic的方式是可以通过分屏对比蒙太奇的方式来展现酗酒和抽大麻的对比情况,并且快速切换对比来营造出惊叹。
 
\subsubsection{\textcolor{questionc}{Question Mark as Forward Thinking}}
%A question mark is open-ended as it raises the big question of what is next, posting new and relevant questions based on the content not previously addressed.
A question mark is open ended as it raises the great question of what is next, posting new, relevant questions based on the content not previously addressed. It is also associated with unsolved answers or mysteries, often suitable narrative materials that arouse curiosity.
%A question mark in film and data video functions somewhat differently in a different context. 
It often opens up new questions or mysteries related to the plot in films.
Question marks were the least common in films (10/111) and data videos (6/105).
Data videos often provide answers to the questions, and new questions are raised at the end.

% Data video addresses questions and answers throughout so that new questions are being raised at the end of the video.
%In this way, this ending ends with new thoughts and ideas. 
%It takes the audience to the next level of thinking or reflection at the end. 


%2.数据视频中问号的ending最少(6/105),和我们理解的数据视频为大众传递数据信息、更多时候是在解答问题的特质很一致。数据视频通常在视频的过程中将提出的问题解决,而结尾的提问更多的是换一个新角度提问、或者是基于data fact提问。
%3.电影中以问号结尾的ending是最少的(10/111),电影中常见的提问方式就两种——一种是创作者知道答案但是故意提问来‘耍’观众,从而让观众思考。第二种是创作者也不知道答案,最后结尾提出来和观众一起探讨。


%Question mark is open-ended as it raises the big question what's next,
%posting new and relevant question based on the content that is not previously addressed. 
%Question mark also associates with unsolved answer or mystery, which are often good narrative materials that arouse curiosity.
%Question mark in film and data video function somewhat differently in different context. 
% Question mark in film associates with plot mystery while question mark in data video functions in the form of literal questions related to the subject matter. 
%Film ending in the form of question mark opens up new question or mystery related to the plot while data video ending in the form of question mark opens up new question related to the data.
%Data video addresses questions and answers throughout. 
%For data video that ends on a question mark specifically means new questions being raised that at the end of the video.
%In this way,
%this ending ends on new thought and idea. 
%It takes the audience to the next level of thinking or reflection at the end. 

\textbf{Films.}
In time-based media, timing or tempo is everything. Selecting a precise moment to end a shot can create new meaning or raise the relevant question to the audience.
%Choosing a precise moment in a continuous action as the ending shot to create meaning or raise the relevant question to the audience is one of the techniques used in the ending. 
The ending shot of the famous film \textit{Inception (2010)}~\cite{inception}, as shown in~\autoref{fig:question} (a), has raised discussions for many years. 
The unusually long shot on the spinning top kept spinning endlessly as if it suggested the main character was still in his dream state, but after many seconds before the spinning top appeared to be wobbling, the film ended. 
This ending style of showing the unusual property of a symbolic object has made a strong impression on the audience. The filmmaker has invited his audience to think about what is real or not with this ending shot. 
Another favorite question to ask in drama is what is next in the story. A sudden change or a break in pattern always draws attention. 
%In the film story, a sudden change of facial expression in the last scene has been used for many decades. 
In the final scene of the famous classic film \textit{The Graduate (1967)}~\cite{graduate}, when the main character runs away with the bride, the two young lovers laugh, but as they jumped on the city bus, their facial expressions turned blank. This sudden change of tone and pace through the characters' facial expressions raises a question about their future. 
%The ending scene in \textit{Taxi Driver (1976)}~\cite{taxi} has been the topic of much discussion about whether the last scene should be read literally as part of the plot or separate the main character’s fantasy. Regardless, it is no doubt that it adds new meaning to the ending. and creates curiosity about the subject. This example illustrates that a point of view shift is possible in the ending and can make a lasting impression.

%the meaning of this ending scene.


%The ending shot of the famous film Inception (2010) has been subject for discussion for many years. 
%The unusually long shot on the spinning top which kept spinning endlessly as if it suggested the main character was still in his dream state but after many seconds of the shot on the spinning top just before it appears to be wobbling, the film ends. 
%This ending style on showing the unusual property of an symbolic object has made a strong impression on the audience as the filmmaker has invited his audience to think about what is reality or not with this ending shot. 
%In time-based media, timing or tempo is everything. 
%Choosing at what precise moment in a continuous action as the ending shot can create meaning or raise relevant question to the audience is one of the technique used in ending. 

%The ending scene in Taxi Driver (1976) has been topic of many discussion about whether the last scene should be read literally as part of the plot or just a separate the main character’s fantasy. Regardless, it is no doubt add new meaning to the ending and create curiosity of the subject. This example illustrates that a point of view shift is possible in the ending and can make lasting impression.


%Another favorite question to ask in drama is what’s next in the story? A sudden change or a break in pattern always draw attention. In film story, a sudden change of facial expression in the last scene has become topic of discussion for many decades. In the last scene of the famous classic film The Graduate (1967), when the main character ran away with the bride, the two young lovers were laughing and excited. But as they jump on city bus, their facial expressions changed and turned blank. This sudden change of tone and pace through characters’ facial expression raise question about what’s next to them and the meaning of this ending scene. 
\begin{figure}[!ht]
  \centering
  \begin{minipage}[b]{\linewidth}
      \includegraphics[width=\linewidth]{question.png}
    \caption{\rv{Endings of \textit{Inception (2010)}~~\protect\cite{inception} and \textit{7 Billion: How Did We Get So Big So Fast?}~~\protect\cite{7billion} using \textit{question marks} with a sudden change or a break in pattern}}
    \Description{This figure shows the endings of \textit{Inception (2010)}~\cite{inception} and \textit{7 Billion: How Did We Get So Big So Fast?}~\cite{7billion} using \textit{question marks} with a sudden change or a break in pattern.}
    \label{fig:question}
   % \Description{This is the opening sequence of Full Metal Jacket (1987) using Symbolism and Metaphor. The famous haircutting opening scene in Stanley Kubrick's Full Metal Jacket (1987), in which a group of young army recruits were shaved all their hair, symbolized the stripping of their individuality.}
  \end{minipage}
  \end{figure}
  
\textbf{Data Videos.}
Indeed, the author would appear at the end of a data video to raise new questions directly, which can be found in many data videos.
However, these questions are often expressed verbally and directly without using visual styles.
Perhaps adding a visual scene at the end could leave a stronger impression on the audience.
Data graphics could function like characters in a story.
Change or break in the form, such as color, shape, or movement, can signify a change in meaning at the end. 
The data video \textit{7 Billion: How Did We Get So Big So Fast?}~\cite{7billion} is an exception that has a cinematic ending similar to \textit{Inception (2010)}~\cite{inception}, as shown in~\autoref{fig:question} (a).
In~\autoref{fig:question} (b),
its ending shots symbolize the world's population growth by filling the glass with water, that is, when the water glass becomes full,
our earth reaches the carrying capacity.
The last frame uses a similar ending style of ending abruptly.
%depicts a boundary point where the glass is about to the full.
This visual metaphor immediately raises the question: Will the water spill over? Could the earth accommodate the rapid population growth? 
This case illustrates how animation and symbolism offer a combined power to evoke questions cinematically.



% By filling the water glass with water, it symbolizes the world population that the earth holds. When the water glass becomes more and more full, it symbolizes that the world population is also increasing. Finally, the video ends when the water in the water glass is about to be full and may overflow. Now, a question is thrown to the audience in time: Can our earth still accommodate these rapidly increasing populations? This example is also a classic cinematic questioning method that combines visual change and symbolism that we found when we were studying the ending of the movie.


%我们在congding的过程中发现,在以问号结尾的数据视频7 Billion: How Did We Get So Big So Fast? 的ending和inception的结尾有异曲同工之妙,通过用水杯装满了水象征着地球容纳的世界人口,当水杯越来越满时,象征着世界人口也越来越多,最后在水杯中的水即将装满可能要溢出来的时候视频结束了,向观众及时抛出了一个疑问:我们的地球还嫩容纳下这些快速增加的人口吗?这个例子也是我们在研究电影ending时发现的一种结合了视觉变化和象征意义的经典cinematic提问方式。


\subsubsection{\textcolor{ellipsisc}{Ellipsis as Open for Interpretation}}
An ellipsis indicates an omission, and when placed at the end of a sentence, suggests an open ending for interpretation and suggests the story has not ended and is still ongoing.  
%Ellipsis marks are used to end a sentence that functions as the opposite of a full stop as it 
%suggests an inconclusive or open for interpretation. 
%It can also 
%suggests the story is still in development as unfinished business. 
This category also applies to an ending that is up to the audience to make their meaning out of the intended ambiguity.
%aims to offer one solution or resolution to a story or problem, which is up to the audience to make their meaning out of the intended ambiguity.
Ellipses are the second most common style in the film (35/111) and data video (43/105) corpus.
It emphasizes that the story keeps evolving or the data are still changing,
thus provoking audiences' thoughts.

%2.其次电影中的省略号ending也很多(35/111),电影的结尾一直是以能让观众有深入的思考而著称,即便影片结束也能让观众回味无穷,其中饱含的经验就像阳春白雪让人望尘莫及……
%3.数据视频中省略号出现的次数也很多(43/105),我们在coding过程中发现,很多结尾是省略号时的purpose都是call for action或者是call for reflection,很多数据的故事也许讲完了,但是仍会引起观众的持续更深入的思考……



%Ellipsis mark used in the ending of a sentence functions like the opposite of a full stop as it suggests an inconclusive or open for interpretation. 
%It can also suggest the story is still in development as unfinished business. 
%Therefore, it functions as an open-end punctuation mark.
%Arguably, to some extent, ellipsis could encompass mixed tone of a small question mark and/ or a small exclamation.
%This category also applies to ending that aims to offer one solution or resolution to a story or problem - it is up to the audience to make their own meaning out of the intended ambiguity.

\textbf{Films.}
When an ending is ambiguous and offers no clear solution or resolution, it fits the description of the ellipsis. This open-ended style can mean that the story might end here, but life continues, or the problem has not been solved. The famous freeze-frame of the last shot of \textit{400 Blows (1959)}~\cite{400blows}, as shown in~\autoref{fig:ellipsis} (a) and \textit{Butch Cassidy and the Sundance Kid (1969)}~\cite{butch}, as shown in~\autoref{fig:ellipsis} (b), have been the subject of many film analyses.
% has been the subject of many film analyses. This ending style also influences another famous film, \textit{Butch Cassidy and the Sundance Kid (1969)}~\cite{butch}~\autoref{fig:ellipsis} (b), which also ends on a freeze-frame when the characters attempt to break their way out in a middle of a gunfight. 
This style gives the audience opportunities to draw their conclusions and have different interpretations of the story. 
% It is as if to suggest that the outcome is less important; it is the life journey that counts.
% Instead of using a freeze-frame, the last shots of \textit{The Wrestler (2008)}~\cite{wrestler} and \textit{A Separation (2011)}~\cite{separation} end before the result or action is completed or delivered.
% The latter film ends without showing the conflict's final resolution, but the family will never be the same despite the result. 
Besides a freeze-frame, the last shot could end before the result or action is completed or delivered.
Similarly, in the ending scene of \textit{The Da Vinci Code (2006)}~\cite{davinci}), as shown in~\autoref{fig:ellipsis-datavideos} (a), a long zoom in was used to show that the protagonist found out the Magdalene stone pavilion was hidden under the pyramids of conjecture. 
The film is over, but the conjecture still needs to continue.
Similarly, the ending shot of \textit{Ex Machina (2014)}~\cite{machina} is when the AI is disappearing into the crowd with the major crisis still in progress and unresolved. 
\begin{figure}[!ht]
  \centering
  \begin{minipage}[b]{\linewidth}
      \includegraphics[width=\linewidth]{ellipsis.png}
    \caption{\rv{Endings of \textit{400 Blows (1959)}~~\protect\cite{400blows} and \textit{Butch Cassidy and the Sundance Kid (1969)}~~\protect\cite{butch} using \textit{ellipses} with the freeze-frame shot}}
    \Description{This figure shows the endings of \textit{400 Blows (1959)}~\cite{400blows} and \textit{Butch Cassidy and the Sundance Kid (1969)}~\cite{butch} using \textit{ellipses} with the freeze-frame shot.}
    \label{fig:ellipsis}
   % \Description{This is the opening sequence of Full Metal Jacket (1987) using Symbolism and Metaphor. The famous haircutting opening scene in Stanley Kubrick's Full Metal Jacket (1987), in which a group of young army recruits were shaved all their hair, symbolized the stripping of their individuality.}
  \end{minipage}
\end{figure}
%which are also examples of ellipsis ending.
%Similarly, this punctuation mark can describe the ending scene of The Truman Show (1998), where Truman finally exited the door and entered the next chapter of his life full of unknown. This style suggests that this chapter of life might have ended here, but there are more to come as life goes on. 

%When an ending is ambiguous and offers no clear solution or resolution, it fits the description of ellipsis mark. This kind of open-ended style can mean that the story might end here but life goes on. 
%The famous freeze frame of the last shot of the classic French film 400 Blows (1959) has been subject of many film analysis for many decades.
%This ending style also influences another famous film Butch Cassidy and the Sundance Kid (1969) which also ends on a freeze frame when the characters attempted to break their way out in a middle of a gun fight. 
%This style has given critics and audience to draw their conclusion and to have different interpretations of the story. 
%It is as if to suggest that the outcome is less important; it is the life journey that counts.

%Instead of using freeze frame, the last shot of The Wrestler (2008) and A Separation (2011) end before the result or action is completed or delivered.
%The latter film ends without showing the final resolution of the conflict but despite the result either, the family will never the same. 
%This kind of style does not stress the question and answer relation, it does not end in the form of question mark.
%Similarly, this punctuation mark can describe the ending scene of The Truman Show (1998) where Truman finally exited the door and entered the next chapter of his life full of unknown. 
%This style suggests that this chapter of life might have ended here but there are more to come as life goes on. 

%Unlike Modern Times (1936) and Casablanca (1942), in which the characters walk away with major crisis solved and completed closure, the ending shot of Bicycle Thief (1948) or Ex Machina (2014) when the father and son or the AI are disappearing into the crowd with the major crisis still in progress and unresolved.

%With interesting cinematic effects of repeated jump cut used in the wide ending shot~\cite{yip2020invisible} of Eternal Sunshine of the Spotless Mind (2004), this unique open-ended style implies that no matter how many times the two lovers tried to erase their painful memory each time, they are destined by fate to end up together and separated on and off over and over again.
%Whether this is good or not, it is up to the audience to draw their own conclusion. 
%The outcome is less important; it is the journey that counts as life goes on.

%An unique example of the combined use of punctuation marks in film. 
%The Soviet film Come and See (1985) presents a very emotionally intense eight minute long film ending sequence that combines the tones of exclamation and ellipsis marks.
%The ending features the young man shooting repeatedly at the picture of Hilter intercutting with montage sequence of historical war footage being played in reverse to Hilter’s rise of power and further backwards to the image of enfant Hilter under his mother’s arms. 
%The repeated shooting then stops and the character broke down in tears. 
%Then he reunited with his troop and continued on the journey.
%This unique ending style displays a powerful expression of intense emotional feelings.

\textbf{Data Videos.}
Neil Halloran’s data videos deal with subject matters from war to climate change, but all their endings project the filmmaker’s feelings and concerns about the uncertain future open for interpretation. 
His voiceover expresses that there are lessons to be learned from the data as human development is always continuous. 
This theme is also expressed visually through his cinematic data visualization. One example is the long zoom in and zoom out in the endings of \textit{The Fallen of World War II (2016)}~\cite{neil:fallen}, as shown in~\autoref{fig:ellipsis-datavideos} (b) and \textit{Simulation of a Nuclear Blast in a Major City (2020)}~\cite{neil:simulation}, respectively. 
In the former video, the director not only expresses his view through his voice-over but also visualizes through magnifying the graphical data in a long zoom in shot to suggest that the meanings of peace can be found ``between the lines.''
% and in his case, the meanings are open for interpretation between the data.
Similarly, in the latter example, the long zoom out shows the possible destruction of a nuclear blast as the voiceover states, ``We can continue ensuring that this nightmare simulation never becomes a reality.'' 
Both endings let us interpret for ourselves the uncertainty that lies ahead, and they are graphically visualized through cinematic language and expression.
The filmmaker also expresses his personal feelings behind the data at the end. 
His feelings are not defined and expressed in the form of forceful exclamation or question marks, as his forward-thinking view of uncertainty is softly expressed as an ellipsis statement. 
His theme about human development as a continuity of uncertainty is consistently expressed in his endings. 
Therefore, these endings function as more ellipsis marks than exclamation or question marks that usually carry a stronger tone of voice and certainty. 

\begin{figure}[!t]
  \centering
  \begin{minipage}[b]{\linewidth}
      \includegraphics[width=\linewidth]{ellipsis-datavideos.jpg}
    \caption{The endings of \textit{The Da Vinci Code (2006)}~~\protect\cite{davinci} and \textit{The Fallen of World War II}~~\protect\cite{neil:fallen} using \textit{ellipses} through a long zoom in shot}
    \Description{This figure shows the endings of \textit{The Da Vinci Code (2006)}~\cite{davinci} and \textit{The Fallen of World War II}~\cite{neil:fallen} using \textit{ellipses} through a long zoom in shot.}
    \label{fig:ellipsis-datavideos}
    \vspace{-10px}
  \end{minipage}
  %\vspace{-10px}
  \end{figure}

\subsection{Discussion}
In short, based on previous works~\cite{pyramid2021, amini2015understanding} related to narrative ending styles, our study further emphasizes the contextual relation between the ending and content of the data story. Our work echoes previous narrative elements (e.g., questions, recap, echoing the beginning) and further proposes new cinematic elements (e.g., montage, wide shot, and the iris effect) as discussed in case examples. 

Aside from inspirations drawn from cinematic language for the effective ending, one important feature of our study is to emphasize the design of an ending is constrained and contextualized by its content and how the desired tone can be expressed with a specific cinematic style that aims to create a lasting impression. The chosen ending style does not exist in isolation or independent from the context and content of the work. This study does not just name these elements but builds on and extends them for the cinematic endings. We further incorporate these elements with design objectives and specific guidelines to create a more lasting impression ending through our guidelines.
%In short, based on previous work~\cite{pyramid2021, amini2015understanding} related to narrative ending styles, our study further emphasizes the contextual relation between the ending and content of the data story. We echo previous narrative elements (e.g., questions, recap, echoing the beginning) and further propose new cinematic elements (e.g., montage, wide shot, and the iris effect) as discussed in case examples. 

%Aside from inspirations drawn from cinematic language for the effective ending, one important distinguishable feature of our study is to emphasize the design of an ending is constrained and contextualized by its content and how the desired tone can be expressed with a specific cinematic style that aims to create a lasting impression. The chosen ending style does not exist in isolation or independent from the context and content of the work. This study does not just name the elements but also emphasizes the design objectives of an ending with specific guidelines that match the design intent so that a more lasting impression can be created through our guidelines.
  
  
% (image zooms out and becomes blurry with music).
% However, this is not to suggest that the tone of ellipsis mark and the visual style of long zoom in/ out and ellipsis mark exclusively associate with each other. 
% These visual style can appear in other punctuation marks and examples of these sorts will be discussed in details in the film examples below. These examples are also some reasons why some of the film endings are topics of discussion for many years and even decades after the films were shown. 



%105个数据视频结尾:45个句号,11个感叹号,6个问号,43个省略号。

%111个电影结尾:37个句号,29个感叹号,10个问号,35个省略号。

%1.数据视频中句号的ending最多(45/105),其实和数据视频的ending通常需要进行总结、清晰地传达数据发现和结论有密切联系。
%2.数据视频中问号的ending最少(6/105),和我们理解的数据视频为大众传递数据信息、更多时候是在解答问题的特质很一致。数据视频通常在视频的过程中将提出的问题解决,而结尾的提问更多的是换一个新角度提问、或者是基于data fact提问。
%3.数据视频中省略号出现的次数也很多(43/105),我们在coding过程中发现,很多结尾是省略号时的purpose都是call for action或者是call for reflection,很多数据的故事也许讲完了,但是仍会引起观众的持续更深入的思考……
%4.感叹号出现次数为(11/105),虽然不多不少,但是对于感叹号的ending的数据视频印象还是很深刻,感叹号结尾对给观众留下的深刻的数据印象有帮助。

%1.电影中句号的ending最多(37/111),因为一个故事讲完了,主人公任务完成,这是电影中最常见和完整的一个戏剧故事结构。
%2.其次电影中的省略号ending也很多(35/111),电影的结尾一直是以能让观众有深入的思考而著称,即便影片结束也能让观众回味无穷,其中饱含的经验就像阳春白雪让人望尘莫及……
%3.电影中以问号结尾的ending是最少的(10/111),电影中常见的提问方式就两种——一种是创作者知道答案但是故意提问来‘耍’观众,从而让观众思考。第二种是创作者也不知道答案,最后结尾提出来和观众一起探讨。
%4.电影中感叹号的结尾不算多也不算少(29/111),以感叹号结尾的电影很多时候是在结尾处揭开了全片设制的悬念来让观众叹为观止。但是出现频次也不算多的原因是,很多电影其实不会在影片最后才揭开谜底,往往是揭开谜底后还有一定的延续,所以不都是感叹号结尾。

\rv{\section{Study 2: Formulating Guidelines}}
%\section{Expert Interviews}
%Based on the analysis, we conduct expert interviews to derive guidelines for producing cinematic endings for data videos.
%In this section,
%we detail our methodology and the resulting guidelines.

\rv{Based on the corpus analysis, expert interviews were conducted to derive guidelines for producing cinematic endings for data videos. Specifically, the classic ending examples of films and data videos under four common styles are used as materials for the expert interviews. 
In this section,
our methodology and the resulting guidelines are discussed.}
%Expert Interview

\subsection{Methodology}
Inspired by~\cite{kim2022mobile, xu2022fromwow, lan2022negative}, experts from different backgrounds were interviewed to derive the guidelines.
First, film experts were asked about suggestions for applying cinematic techniques to data videos.
Second, visualization experts were interviewed about their inspirations for enhancing data storytelling from films.


\textbf{Participants.}
Our interviewees consisted of six film experts (E1-6) and five experts in data visualization experts (E7-11).
~\rv{Inspired by~\cite{krueger2014focus, lin2021engaging}, our utmost was done to diversify the backgrounds of the experts. The eleven interviewees were from the U.S. (4), Mainland China (3), Hong Kong SAR, China (3), and the U.K. (1). Seven males and four females between the ages of 25 and 48, with a minimum of two years of industry and academia experience on film or visualization, were recruited from film and visualization industries. Table \ref{tab:my-table} lists the background of the experts.

\begin{table*}
\centering
\begin{tabular}{llll}
\hline
Participant ID & Gender & Occupation(s)                                     & Experience\cr
\hline
E1             & M      & Full-time University Professor & 20 years in academia and industry\cr
\hline
E2             & M      & Senior Director                             & 10 years industry experience in film \cr
\hline
E3             & F      & Documentary Film Director                            & 9 years industry experience in documentary film  \cr
\hline
E4             & M      & Digital Visual Artist              & 5 years in both academia and industry          \cr
\hline
E5             & M      & Cinematographer                    & 5 years industry experience in cinematography \cr
\hline
E6             & M      & Visual Designer                              & 2 years industry experience in visual design \cr
\hline
E7             & M      & Researcher                              & 7 years academia experience in VIS and HCI community\cr
\hline
E8             & F      & Researcher                             & 6 years academia experience in narrative visualization \cr
\hline
E9             & M      & Researcher                              & 5 years academia experience in VIS and HCI community \cr
\hline
E10             & F      & Researcher                              & 6 years academia experience in information visualization design \cr
\hline
E11             & F      & Researcher                              & 9 years teaching experience in narrative visualization \cr
\hline
\end{tabular}
\caption[]{\rv{Background information of the 11 experts who participated in our interviews}}
%\Description{This table presents the background information of the six experts that participated in our interviews, including their Participant ID, Gender, Occupations, and Experience.}
\label{tab:my-table}
\vspace{-10px}
\end{table*}}

%%Participants
%我们一共邀请了11位专家,其中有6位是电影领域的专家(E1-6),5位是数据专家(E7-11)。包括了一位电影专家超过20年的film practice and theory research经验(E1),Two film directors(E2具有10年的电影导演经验,E3具有九年的纪录片电影经验),Two film visual designer(E4有5年经验,E6有2年经验),一位电影摄影师(E5有5年从业经验)。而我们邀请的五位数据专家皆来自数据可视化领域(E7-11),E7和E11具有超过6年的数据可视化和数据讲故事的经验,E8、E9、E10具有超过4年的数据可视化设计和讲故事经验。这次采访是通过face to face或者线上meeting进行。

\textbf{Interview Procedure.}
~\rv{The interviews were conducted through either face-to-face or online meetings in a two-to-one manner (two authors and one expert).}
Each interview session lasted for about 120 minutes.
We first introduced our research goal, 
to help data video makers produce more understandable, impressive, and reflective endings.
We then briefed the four cinematic ending styles together with corresponding example films and data videos.
For each style,
we asked participants to give consideration and suggestions to the content and visual form of data video endings by referring to corresponding films of the same style.

\textbf{Interview Analysis}
First, three authors extracted thematic codes through an open-coding approach~\cite{charmaz2006constructing} and coded the interview transcripts separately.
Next, they iteratively coded the interview data through six rounds of meetings until reaching a consensus.
Furthermore,
an external professor in film theory and practice was requested to validate the guideline from an empirical point of view.
The whole process resulted in a final list of 20 guidelines.


%Interview procedure
%我们首先向每一位专家介绍我们这次研究的目的是想帮助数据设计师可以做出一个更能让观众对数据主题进行理解、记忆和思考的数据视频结尾。对于每一个标点符号,我们都会给到对应的电影和数据视频来展现对应标点符号的特征和可能带有的cinematic风格,我们期待电影专家能够基于经验对不同标点符号提出Cinematic的在数据视频的应用建议;期待数据专家能够对应我们在采访过程中对对应标点符号的电影有所启发并提出在data videos中data visualization和data storytelling的建议。我们在采访中会对专家关于Ending进行的提问包括cinematic style的content和form:这个标点符号的数据视频在content上可以怎么借鉴对应电影的cinematic storytelling?这个标点符号的数据视频/可视化在visual上可以怎样借鉴对应电影的visual design。
%对于每一个标点符号都会有对应的两到六种不等的风格,每一种风格都有对应的电影和数据视频。每一位专家在观看完每一个类型的电影和数据视频会给出相对应风格的建议。
%采访完11位专家后,先是由第一作者进行采访整理,随后四位作者进行了X轮的讨论和迭代,第二和第三作者进行再次确认整理后,外请了一位具有超过15年电影研究和实践的教授进行采访,并让其阅读每一条guidelines,该教授从经验角度出发确保没有新增和遗漏后,最终确定出了20条guidelines。

%\begin{figure*}[!t]
%\centering
%\includegraphics[width=.85\textwidth]{guidelines.jpg}
  %\caption{Overview of 20 guidelines including their punctuation marks, functions, and visualizations. Designers can select and use the corresponding from the left to the right of this figure.}
%\Description{This figure shows the overview of 20 guidelines, including their punctuation marks, functions, and visualizations. Designers can select and use the corresponding from the left to the right of this figure.}
  %\label{fig:guidelines}
%\end{figure*}

\subsection{Guidelines}
%\autoref{fig:guidelines} 
\begin{table*}
\centering
%\begin{tabular}{|l|l|} 
\begin{tabular}{|p{0.05\textwidth} | p{0.90\textwidth}|}
\hline
                      & \multicolumn{1}{c|}{{\textbf{\textcolor{fullc}{Full Stop}}}}                                                                        \\ 
\hline
G1.1                   & Summarize facts in a declarative mood and recall the previous events of the story through montage and/or interview footage.                                            
\\ 
\hline
G1.2                   & Explicitly present the data insights or make a conclusive statement with a wide static shot.                                                        
\\ 
\hline
G1.3                   & Emphasize the connection between the data facts and the world by intercutting with real-world pictures. Engage the audiences in understanding how the real world is reflected by the data and establish emotional resonance.            \\ 
\hline
G1.4                   & Use camera effects (e.g., zoom out) to visualize the overview of data insights. 
\\ 
\hline
G1.5                   & When playing back multiple charts, juxtapose or use montage clips to arrange them.
\\ 
\hline
                      & \multicolumn{1}{c|}{\textbf{\textcolor{exclamationc}{Exclamation Point}}}                                                                                                   \\ 
\hline
G2.1                   & Present intense contradictions or surprising data facts.    
\\ 
\hline
G2.2                   & Leave audiences with a new perspective after the climax of the story.           \\ 
\hline
G2.3                   & Highlight the contradictions using colors, animations, and scenes with strong visual contrasts.     \\ 
\hline
G2.4                   & Emphasize the contradictory or surprising data facts by comparing contrasting images using visualization techniques such as juxtaposition, visual cues, and animations.                  
\\ 
\hline
G2.5                   & Use analogy to make audiences aware of the magnitude of the data to achieve surprising results.                                      \\ 
\hline
\multicolumn{1}{|c|}{} & \multicolumn{1}{c|}{\textbf{\textcolor{questionc}{Question Mark}}}                                                                                                  \\ 
\hline
G3.1                      & Animate the data change by visualizations or metaphors and end the animation abruptly to leave suspense.                \\ 
\hline
G3.2                 & Ask questions about the next movement of data, especially when the data has an incomplete action.                         
\\ 
\hline
G3.3                  & Emphasize questions (e.g., use of repeated shots) about data from new perspectives and call for actions.  
\\ 
\hline
G3.4                    & End with a perfect dream that invites or provokes audiences to question its authenticity.
\\ 
\hline
                      & \multicolumn{1}{c|}{\textbf{\textcolor{ellipsisc}{Ellipsis}}}                                                                                                         \\ 
\hline
G4.1                   & Open-end that expresses central problems unsolved with uncertainty (e.g., with gentle camera movement) and more actions are needed.                                                   
\\ 
\hline
G4.2                   & Echo the opening using similar colors, scenes, or visualizations and highlight the differences from the beginning.           
\\ 
\hline
G4.3                   & Apply the patterns of periodic data to indicate and predict possible future changes.                   \\ 
\hline
G4.4                   & Reveal data continuously using camera movement of push-in (out) and zoom-in (out) to express different possibilities for the future development.  \\ 
\hline
G4.5                   & Foreshadow the beginning of the following data video in the ending, when the topic is too large and complex to cover in a single data video.                                            
\\ 
\hline
G4.6                   & Use action or graphic match cuts to continue the ongoing topic through seamless transitions.                                  
\\ 
\hline
\end{tabular}
\caption{\label{tab:guideline}\rv{Twenty guidelines for applying the four cinematic ending styles to data videos}}
\Description{This table shows twenty guidelines for applying the four cinematic ending styles to data videos.}
\vspace{-10px}
\end{table*}

Table~\ref{tab:guideline}~provides an overview of 20 guidelines organized by each cinematic ending style (punctuation mark).

%4.2 overview of Guidelines
%对应四种标点符号,我们一共生产出了20条guidelines。有的guidelines是来自于电影专家,有的guidelines是来自于数据专家,而更多的guidelines是通过两者给的建议的结合,作者通过专家的建议进行整理和翻译,生产出适合data video ending的guidelines。

\textbf{\textcolor{fullc}{Full Stop.}}
%More than half of the experts (6 out of 11) considered the full stop style useful when ``summarizing previous events at the end''.
%They agreed that a full stop ending should effectively convey the take-home message,
%and thus the core challenge was to make the key insight clearly understood by and resonate with the audiences.
More than half of the experts (6 out of 11) considered the full stop style useful when ``summarizing previous events at the end.''
They agreed that a full stop ending should effectively convey the take-home message. Thus, the core challenge was to make the key insight clearly understood and resonate with the audience.
% They agreed that such endings tended to be insipid,
% and thus the core challenge was to make the key insight clearly understood by and resonate with the audiences.
% 要不删掉insipid改成:他们认为好的句号应该是能帮助更有力量地传达重要信息(take-home message),而做到影响力的第一步就是信息展现需要清晰明了(这可以作为我们整合专家的话,也可以作为E7的话)。

They first provided suggestions on the usage scenario and content design (G1.1-1.3).
For example,
E9 commented that ``visualizations could be interpreted differently. It is important to state the take-home message explicitly and clearly at the end, which otherwise could result in a question or ellipsis mark,''
which motivated the development of G1.2.
Importantly,
experts underscored the need for developing empathy by connecting the audiences from abstract data to the real world (G1.3).
To illustrate this idea,
E2 referred to historical films,
where ``the hero/heroine often gives a monologue to summarize the past story and describe the present.'' 
E2 continued, ``the purpose is to make audiences realize that this is a true event impacting the real world, which could generate empathy and provoke thoughts.''

%句号: 超过一半的专家(6 out of 11)在对于句号结尾的语境使用场景都提到“句号结尾偏向平淡,通常会用于总结和回顾过去的事物以作为结局“,其中E1提到大部分电影在结尾的时候会扣题,E11提到在数据故事组织上句号的结尾其实是用陈述性话语进行事实的描述。从以上的建议中,我们生产出来G1.1-Summarize facts in a declarative mood and recall the previous events of the story. E9提出Data visualization有时候会confusing,不同人看到不同结果,如果想要确定结果建议直接出。最后有一个message要直接带出,如果不直接就会变成省略号或者问号——让观众创作者想说什么?于是我们得出了G1.2-Explicitly present the data insights and take-home messages. E2提出历史题材的电影中,通常会安排事件的真实主人公出现在影片的最后来总结讲述过去的发生的故事和现在,其目的是让观众能被故事打动、吸引后在结尾处通过人物访谈等形式认识到这也是真实的故事,让观众对这个故事有更多的思考。因此,我们生产出了G1.3-Emphasize the connection between the data facts and the world by referring to real-world pictures. Engage the audiences in understanding how the real world is reflected by the data and establish emotional resonance.

Experts also suggested different ways of form to deliver the content.
They (E4, E5, E7, and E8) commented that camera movements and visual cues could attract audiences' attention and emphasize important data facts,
which corresponds to G1.4.
For example, films often end with camera movements to provide a long shot, allowing audiences to take an almost ~\rv{birds-eye view.} %God-like perspective to view the scene.
They recommended different methods for reviewing multiple visualizations at the end (G1.5) because data videos often contained multiple visualizations.
Specifically,
E1 and E3 explained that montage was commonly used in films to display scenes of the past and the present and illustrate causality.
Similarly,
they advocated using montage to display multiple visualizations to illustrate their relationships.

%E4、E5、E7和E8分别提出了在句号结尾时可以借鉴电影的镜头运动或者visual cue等方式来强调ending处要表达的主要信息,通过整理我们得出了G1.4-When playing back the overview of visualizations, highlight important data facts with visual cues, animations, camera effects, etc. E10在采访中有提到在数据视频的ending回顾中通常会涉及到前文的多个图表回顾,可以考虑图表的排列放置,E3和E1补充电影中的常常会在结尾通过用蒙太奇剪辑方式可以将过去和现在进行联系,同时也是对因果的解释,以完成一个完整的句号结尾。通过对于电影专家和数据专家意见的结合,我们生成了G1.5-When playing back multiple charts, juxtapose or use montage clips to arrange them.




\textbf{\textcolor{exclamationc}{Exclamation Point.}}
All experts agreed that endings with exclamation points could leave a strong impression and provided guidelines for content (G2.1-2.2) and form design (G2.3-2.5).

They first gave valuable suggestions on creating an exclamation (G2.1).
E2 and E4 commented that film endings depicted strong conflicts and contradictions to amaze the audience,
whereas E3 and E8 expressed that exclamations resulted from surprising events and data facts.
Contradictions or surprises are often intense,
creating the need for distinguishing exclamation endings from climaxes (G2.2).
``As the climax often provides the most crucial event or data findings,
the ending could show data that is sub-crucial but opens up a surprising perspective to interpret data,''
said E8.

%感叹号:全部专家都同意做一个感叹号的结尾可以帮助故事令人留下深刻的印象。E2:“电影的结尾如果要让观众惊叹,是需要突显矛盾和冲突”,E4:“矛盾和冲突需要从视觉体现出来,结尾要有视觉冲击力,”基于此我们对应数据的特性生成了G2.1-Present intense contradictions or surprising data facts. E3和E8都提出感叹号结尾是让观众惊讶,但是在电影和数据视频中,往往重要的发现会放在climax出就体现出来了,而ending如果要让观众有意料之外的发现,通常可以把一些新的但又不一定是全文中最重要的发现放在ending。E11提出这种惊讶可以是打破观众直观的认知。于是我们生成了G2.2-They might not necessarily be the most crucial part of the story but leave audiences with a new perspective. 

Other experts suggested how exclamation endings could be visualized.
G2.3 was exemplified by E4's comment,
``Contradictions could be visualized following the principle of graphical design, for example, using complementary colors or shapes.''
Furthermore, they proposed using comparative visualization techniques such as animations to highlight data contrasts (G2.4).
Finally, E10 and E11 commented that audiences might not~\rv{have intuition} about the data size or magnitude.
They explained that analogy might surprise audiences by the data magnitude (G2.5).


%E4和E6提出制造exclamation mark也可以是制造反差和对比,E4:“制造反差和夸张可以遵循视觉设计上原则,比如颜色对比用对比色和互补色。比如形状一个是方块一个圆形,等用有反差的视觉形式进行对比来突出差别”。于是我们生产了G2.3-Highlight the contradictions with strong visual contrasts by using colors, animations, and scenes. E3对于我们在采访中提供的数据视频Weed is not more dangerous than alcohol的例子提出了感叹号结尾的修改建议:“采用分屏的形式在屏幕左边现在吸大麻的情况,右边展现酗酒的情况,并且使用对比蒙太奇的手法快速切换分屏的信息,以让观众产生wow的感觉。”经过整理,我们生成了G2.4-Emphasize the contradictory or surprising data facts by comparative visualization techniques such as juxtaposition, visual cues, animations. E10和E11表示数字很多时候让观众很陌生,即便是很大、具有量级的数据也很难让观众能够意识到数量之大,如果要做一个感叹号的结尾,可以尝试使用类比的方式来让观众更能感受到数据的量级,E11: “比如用‘相当于绕地球几周’的通许易懂的话语来表达。”基于此,我们生成了G2.5-Use analogy to make audiences aware of the magnitude of the data to achieve surprising results。

\textbf{\textcolor{questionc}{Question Mark.}}
Most of the experts (7 out of 11) suggested that ending with a question mark was a powerful approach to provoke thoughts.
They thought highly of the film \textit{Inception} ~\cite{inception} and the data video \textit{7 Billion: How Did We Get So Big So Fast?} ~\cite{7billion} (~\autoref{fig:question}).
E6 explained, ``They both use animation to symbolize the evolution of the story and imply different endings - whether the spinning top would stop and whether the water would spout out.''
E1 further commented, ``The question arises as the storytelling is ended abruptly, but the story is still evolving.''
We derived G3.1 from those comments.

%问号:E2:“在电影结尾中提出问题通常有两种情况:一种是创作者知道答案,但是就是不告诉观众答案,在结尾故弄玄虚抛出一个问题;另一种创作者也不知道答案,在结尾处提出了问题供探讨”。E1、2、3、4、5、6都同意在我们使用的expert interview的问号结尾例子中,电影inception则是属于创作者故意留下悬念,让观众对于这个结尾有更多的思考和印象更深刻,这种结尾很值得我们研究和学习。E6:“从视觉设计的角度看,无论是电影inception中的陀螺是否有停止旋转还是数据视频例子7 Billion: How Did We Get So Big So Fast? | SKUNK BEAR中的水是否会从杯子中溢出来都具有同样的特性——第一两者都用了象征,第二两个用于象征的物体都具有两种状态,旋转的陀螺有停止或者继续的状态,水杯中的水有溢出或不溢出的可能。”,E1:“其实造成疑问的方式是状态还在延续但是故事已经结束来产生疑问”。我们通过整理,得出了G3.1-Animate the data change by visualizations or metaphors and end the animation abruptly to leave suspense.

Experts described different scenarios for asking questions in data videos that informed the development of G3.2-3.4.
For example, E10 commented,
``It is common to ask questions about the future trend of the data,'' which corresponds to G3.2.
E2 and E8 considered the ``so what'' questions for calling for actions, for example, ``what should we do in response to the data facts?'' (G3.3).
Finally, E1 referred to the film endings that ``Films sometimes end with fantasy by the creators that are very different from reality, which makes audiences wonder about the reality and provoke reflections,'' which helped us develop G3.4.
 
%E8、9、10、11都谈及了对于数据的提问往往是两大类,E10:“数据视频的提问可以基于原有的data fact,对未来的趋势和可能性进行提问”,我们生成了G3.2-Ask questions about the future trend of data, especially when the data has a temporal attribute or is incomplete. E2和E8都认为问号的结尾往往能用来call for action,基于现状来提出关于How、why等相关的新角度问题,比如,“面对这个情况我们该怎么办?”由此,我们生成了G3.3-Ask questions about data from new perspectives and call for actions. E1在听完我们对问号的解释和看了我们提供的例子后提出在电影中很常见的一种情况是创作者主观想象出来的ending,而这种ending通常和实际发生的不一致,从而让观众产生对现实的疑问和反思。于是,我们生产了Guidelines3.4-End with a fantasy of the future, which is not necessarily the actual situation, so audiences have questions about the reality.


\textbf{\textcolor{ellipsisc}{Ellipsis.}}
Seven out of 11 experts emphasized that ellipsis marks could make audiences be left wanting more,
which was the most challenging of the four styles.
Experts first advised using scenarios for ellipsis from the aspect of narration and data.

In terms of narration, experts advised two possible scenarios (G4.1-4.2).
E3 and E8 commented, ``The ending does not provide a clear answer about whether the problem at the start has been resolved, which makes audiences realize that the event is still developing.''
Their comments inspired us to develop G4.1 about content design.
G4.2 was illustrated in E2's comments, ``In films, the endings can refresh the openings by using a similar mise-en-scène to imply that the story has not ended.''
This comment echoed the feedback from data experts that ``Data videos could end with a visualization that is similar to that at the beginning with notable differences to show that data keep changing ''.
Based on their comments, we derive G4.2.

% ``the ending in Gone Girl displays a scene that is similar to the beginning. It is not a simple repetition but makes audiences feel about a circularity of the entire story.''

%1)G4.1 Problems have not been resolved 到了结尾,问题仍然没被解决,这是无论电影还是数据视频有可能出现的情景(E3,7,8)(2)G4.2 Similar visualizations with the opening专家对于问题还未解决这个使用场景提出了延伸,E2提出电影中会经常出现refresh the opening的情况,通常是开头和结尾主人公在同样的场景出现,做同意的动作,来表达故事还在继续,仿佛和开头一样。而数据专家E9也提出了在省略号中可以出现和开头相似的可视化,同时这个可视化有需要和开头有所不一样表示故事在发展仍在发展中的意思。 

%省略号:大部分专家(7 out of 11)都提出了生产省略号ending会有一定的挑战,因为省略号很多时候是要从content和form上面都展现出结尾时一种未完待续、或者是结尾通过省略让观众有一种意犹未尽的思考·····也是四种标点符号中最需要发挥创意的一种ending方式。E3、7和8都提到省略号的使用可能出现的情景:无论是电影还是数据视频有可能在ending的时候并没有给出开头事件准确的答复,事情还在发展中,因此针对这一种使用场景,我们得出了G4.1-Show that the problem at the story's beginning has not been resolved and that it needs more efforts to make changes and solve problems. E2提出在电影gone girl中的开头和结尾出现了相似的画面会给干中带来一种循环和未完待续的感觉,但是这种结尾像开头的重复出现又不是简单的重复,而是有意义的,能给予观众更多信息的重复。进而我们生产出对应数据视频的G4.2-Echo the opening by using similar colors, scenes, or visualizations and highlight the differences with the beginning.





In terms of data, other experts provided suggestions based on the basis of the usage scenarios of ellipsis marks.
First,
ellipsis marks are often used to ``show the omission of repetitive or similar words'' (E5,6,9,10).
Thus,
they considered the ellipsis marks to be applicable to periodical data or data with similar patterns (G4.3).
Second,
E4 commented,
``Ellipsis marks can represent infinite, for example, using zooming in/out to change the information density.''
This idea echoes that of E11, ``such zooming brings about a feeling of  seeking answers or looking forward.''
After summarizing their comments, we derived G4.4.

% 2.从适合的数据类型的角度,整理出了两种数据适合用于省略号 G4.3 Periodic data 和 G4.4 Infinite and Beyond Data。 3.针对一种叫‘系列’的常见题材,首先是G4.5Large,complex topic很适合做成系列,然后针对系列,电影专家(E5)和数据专家(E8)都提出了专业的transitions建议来foreshadow the new topic-G4.6

Experts also mentioned series as another usage scenario of ellipsis marks in films and data videos.
E5 explained,
``In film series, the ending is often a crisis to be shown in the next film.''
Experts provided suggestions on applying this idea to data video series, especially when the dataset is large and complex. (G4.5-4.6).
Specifically,
they proposed various techniques to foreshadow the next data video, such as seamless transition techniques in films (E5) and morphing in data videos (E8).


%E5、6、9和10根据省略号能达到“帮助举例时省略相同含义的事物”的特征提出了一些具有周期性规律的数据或者有similar pattern的数据适合用于省略号,我们基于此生产出了G4.3-Applicable to periodic data by using the periodic patterns to predict future data and indicate future changes. 与此同时,E11提出“省略号在故事组织上有一种省略、把速度变慢,在往远处看(zoom out)和往里面看(zoom in)寻找答案和表达期盼的感觉,是一种希望一直维持下去,未尽之意······”,而E4受到我们省略号提供的电影和数据视频的例子则提出“省略号从视觉设计上很多时候是一种通过放大或者缩小或者旋转而造成的视觉上的信息密度变化,省略代表的也许是我们之前忽略的无穷无尽的内容。”,借于此,我们生产了G4.4-When the revealing data is still changing, use the camera movement of push-in (out) and zoom-in (out) to express the infinite and boundless possibilities of the future.





%E5和E6在谈及省略号结尾的时候都提到了系列电影中的一些特质,E5:“系列电影的结尾往往会给出一个危机,我们是否能解决这场危机会成为下一个问题。”E6:“对于系列电影在结尾增加新元素,通常是哪些元素是流行卖点就加哪些,比较随意”。进而我们整理出了G4.5-End with exciting problems to be discussed in the following data videos, especially when the topic is too large and complex to cover in a single data video. 对于在省略号的结尾作出一个转场,E2、5、7、8都分别提出了电影和数据可视化中常见的转场模式并且可以在ending处适合运用,电影常见的技巧性转场模式有利用相同动势进行转场,E5举了一个例子:比如我们常见的电影场景从教室到卧室,通常会通过推开教室门的镜头直接接上主人公在卧室的场景。E8也提出在data video结尾会使用seamless transition来过渡到下一个新的主题。于是我们生成了G4.6-Use seamless transitions to foreshadow the new topic。


\rv{\section{Evaluating Ending Styles and Guidelines}}
\begin{figure*}
  \includegraphics[width=\textwidth]{design.png}
  \caption{\rv{An overview of our guidelines includes their content/intention and design/shot recommendation. Users can select the content or intention they tend to express and then use the corresponding design or shot recommendation in the user study.}}
  \label{fig:design}
   \Description{This figure presents an overview of our twenty guidelines, including their content/intention and design/shot recommendation.
   Users can select the content or intention they tend to express and then use the corresponding design 
   or shot recommendation in the user study.}
\end{figure*}

%\section{Evaluation}
% To show the usefulness of our guidelines, 
%To evaluate our guidelines, we conducted a comparative user study where participants were asked to design data video endings with and without our guidelines.
%Furthermore, we presented a gallery containing four data videos that are on the same topic but end with different punctuation marks and cinematic styles. 
% In the following text,
% we discuss our study design and results.
~\rv{In this section, we conducted a user study and a comparative study to evaluate our ending styles and guidelines derived from corpus analysis and expert interviews. Specifically, we evaluated how well our ending styles and guidelines can help users design cinematic data video endings by considering two questions: (1) whether users can understand and apply the guidelines, and (2) whether the guidelines can help achieve understandable, impressive, and reflective data video endings. To answer the first question, we conducted a user study to ask \textit{guideline participants} to use and rate our endings styles and guidelines. To answer the second question, we conducted a comparative study to ask the general public and experts to rate data video endings by all participants with and without using guidelines in our user study.

Furthermore, we also presented a gallery containing four data videos that are on the same topic but end with different cinematic styles by using our guidelines in order to demonstrate the potential application of our ending styles and guidelines.} 
\subsection{User Study}
%We conducted a user study to collect designs for data video endings.

%\textbf{Participants.}
%By advertising on online platforms, we recruited 12 participants (6 female) whose ages were between 24 to 30.
%Our participants included 8 designers, 2 visual effects artists, and 2 data analysts.
%Their job involved data storytelling,
%e.g., delivering presentations, drafting data stories, or making data videos to analyze and communicate data.
%After the user study, each participant receives remuneration of \$30 USD.
%participants
%Participants:12位 (数据设计师和平时需要讲述数据故事的工作者) 其中6名男6女,年龄在24-30岁。有8位是设计师,2位视觉特效师,2位是数据分析师,他们都来自不同的领域,包括工业设计、平面设计、影视特效设计、数据金融等。所有的参与者都有对于数字故事的需求和经验。通常会在工作中展现分析数据、做presentation或者是制作相关的数据故事和视频等等。
\subsubsection{Participants}
\rv{By advertising on online platforms, we recruited 24 participants (12 female) whose ages were between 24 to 31, with an average of 27.
Our participants included 16 designers, four visual effects artists, and four data analysts.
Their job involved data storytelling,
for example, delivering presentations, drafting data stories, or making data videos to analyze and communicate data.
After the user study, each participant received a remuneration of \$30 USD.}

%\textbf{Study Materials.}
\subsubsection{Study Materials}
We provided participants with three candidate datasets of different themes,
including \textit{the COVID-19}, \textit{the 10 causes of death}, and \textit{the global trends in over weight and obesity}.
Those datasets were selected because they were of general interest with suitable complexity.
\rv{Furthermore,
we provided participants with a sheet about our guidelines (Table~\ref{tab:guideline}), and we further explained each guideline from two aspects, their content or intention (``what'' and ``why''), and its design and shot recommendation (``how''), as shown in ~\autoref{fig:design}.
In addition, we provide participants with an interactive website (\url{https://cinematicendings.github.io/}) to browse examples of our ending styles and guidelines.}


%Study and Teaching Materials
%在这个workshop的中,然后给参与者提供了三个主题的datasets: (三个主题新冠-香港第五波疫情的状况、疾病-The top 10 causes of death和肥胖-The global trends in over- weight and obesity)供设计师随意选择能够符合他们想做的ending的题材。我们给参与者们准备了一个guidelines图(cite fig.),这个guidelines图可以帮助设计师对我们的guidelines进行运用的指导。设计师可以从左到右根据需求使用,最左边是设计师需要做的标点符号ending选择,接着是标点符号底下对应设计师有的素材和元素,然后是指导元素运用在标点符号下对应guidelines。最右边是guidelines对应可以呈现出的visualization的视觉效果,同时,我们也给参与者提供了我们的网站(cite网站)进行学习。网站里面有我们每一个标点符号对应的guidelines。每一条guideline都有对应的visualization和一个story case的GIF以及对应的解释。

%\textbf{Study Design.}
%We divided 12 participants into four groups for each ending style.
%The user study took the within-subject protocol,
%i.e., participants were asked to design data video endings before and after using our guidelines.
%Participants could choose one of the three provided datasets according to their own preferences.
%The workshop was hosted in a mixed mode due to the pandemic,
%i.e., eight participants joined online, and four participants were offline.
%They were instructed to use their familiar tools (e.g., laptops, tablets, pens) for design.

%12位设计师会被随机分配到制作不同标点符号结尾的ending story board。within _3位句号/问号/感叹号/省略号。每个标点符号将有三位设计师进行创作。每位设计师将需要创作出使用guidelines前和使用guidelines后的ending storyboard。设计师可以按照他们的需求在三组datasets中选择进行创作。由于疫情,我们的参与者大部分8位通过online meeting,小部分4位参与者通过线下进行。他们作画有分别通过电脑、ipad、或者是铅笔手绘等他们便捷和熟悉的方式。
\subsubsection{Study Design}
\rv{This user study took the between-subject protocol, that is, 12 participants were asked to create data video endings with our guidelines and 12 participants without guidelines. Specifically, we randomly and equally divided 24 participants into two groups to ensure fairness: one group with 12 participants using guidelines as~\textit{guideline participants} and the other group with 12 participants without guidelines as~\textit{non-guideline participants}. Each group had eight designers, two visual effect artists, and two data analysts. We then randomly and evenly assigned two participant groups to all four ending styles. Each ending style had three~\textit{guideline participants} and three~\textit{non-guideline participants}.
Participants could select one of the three provided datasets according to their own preferences.
The user study was hosted in a mixed mode due to the pandemic,
that is, 16 participants joined online, and eight participants were offline. They were instructed to use their familiar tools (e.g., laptops, tablets, and pens) for design. All the storyboards created by participants with or without guidelines are shown in the supplemental materials, and the 12 storyboards with our guidelines are also shown on our website (\url{https://cinematicendings.github.io/\#/Storyboard}).

\subsubsection{Study Procedure}
Our user study lasted for 55 minutes and had three phases: (1)~\textit{introduction phase}, (2)~\textit{design phase}, and (3)~\textit{feedback phase}.

\textbf{Phase 1: Introduction (around 15 minutes)}
This phase started with a brief session to introduce the study procedure, datasets, the concept of data video endings, and each cinematic ending style and guidelines to each group. For~\textit{non-guideline participants}, we introduced the same information except the guidelines.
We asked participants to create data video endings that could
foster audiences' understanding, impression, and reflection about the themes.
That said,
they were instructed to develop the core theme of their videos and subsequently design the data video endings.

\textbf{Phase 2: Design (around 25 minutes)}
Participants of both groups were given 25 minutes to craft storyboards to show the designs of their data story endings.~\textit{Guideline participants} could have our guidelines teaching materials, including guidelines (Table~\ref{tab:guideline}), design recommendation (\autoref{fig:design}), and interactive website (\url{https://cinematicendings.github.io/}).~\textit{Non-guideline participants} could search for and learn from any information from the Internet.

\textbf{Phase 3: Feedback (around 15 minutes)}
The user study ended with a 15-minute post-study interview session. All participants were asked to explain their designs. Afterward, we interviewed them to understand their considerations, challenges, and comments on creating data story endings with and without guidelines. In addition, we invited~\textit{guideline participants} to rate the understandability, impression (to what extent could the endings make a lasting impression or be memorable in the mind of the audience), reflection, clarity, and usability (to what extent could they apply the styles by using our teaching materials) of each ending style, as well as the clarity and usefulness (to what extent could the guidelines help them achieve the style) of each guideline using a 7-point Likert scale.}

%\textbf{Procedure.}
%We started with a brief session of 15 minutes to introduce the four cinematic ending styles and assign groups for each style.
%We asked participants to create data videos that could
%foster audiences' understanding, impression, and reflection about the themes.
%That said,
%they were instructed to develop the core theme of their videos and subsequently design the data video endings.
%In the next 25 minutes, we asked participants to craft storyboards.
%After the first session, we spent 20 minutes explaining our guidelines, where participants could interact with the website to gain a deeper understanding.
%The following 25 minutes were denoted to re-crafting the storyboards with our guidelines.
%This is an example of the storyboard by a participant ~(\autoref{fig:storyboards}).  
%The workshop ended with a 15-minute post-study interview session.
%Participants were asked to explain their storyboard and rate each guideline on a 7-point Likert scale according to three criteria (~\autoref{fig:workshop_result}).



%process:我们的workshop 分成5个阶段,每位参与者全程需要90分钟。1.15min 首先给参与者介绍标点符号,分配一个标点符号,让其从三个datasets中随意选择来做一个和标点符号相关的ending。---说清楚我们ending的主要目的是帮助加强主题的理解、印象和思考。 在画story前需要提供一个故事的标题和故事的主题。2.25min 直接开始画ending storyboard。3.10min 展示和讲解对应的标点符号guidelines。并且让参与者可以通过网站和我们guidelines导向图更多地了解我们的guidelines。4.25min 重画ending storyboard。5.feedback阶段:15min 介绍storyboard+完成questionnaire+interview。在questionnaire中的问题包括了参与者的:年龄,在可视化领域的时间,什么时候会需要用讲述数据中的发现,还让每位参与者对其使用的标点符号和标点符号对应的guidelines进行了7-point Likert scale的打分。对于标点符号的打分包括1.Clarity-根据每一个标点符号的定义和guidelines,我理解了每个标点符号的结尾方式。2.Usability-根据每一个标点符号的定义和对应准则,我可以运用某标点符号来设计数据视频的结尾。3.能帮助加强对主题的理解understandability/印象impression/思考reflection。 对于每条guidelines打分包括 1.Clarity-我认为我可以理解四种标点符号中对应的每一个准则 2.Usefulness-对应每一个准则,我认为他们能帮我实现对应标点符号要实现的目的和效果   最后我们对于每一位采访者进行了以下的提问 1.你在构思结尾设计的时候考虑了哪些因素? 2.在设计数据故事结尾的时候有遇到哪些困难?故事设计的困难?视觉设计的困难? 3.都用了哪些guidelines? 4.如果未来能够开发一些辅助工具,你觉得什么样的工具可以更好地帮助你改进data story 的创作?)最后的访谈我们会进行记录并后续分析
\rv{\subsection{Comparative Study}
To understand whether those data story endings designed with our guidelines were truly more understandable, impressive, and reflective to audiences than those designed without guidelines, we further recruited experts from the film industry and the general public to evaluate the storyboards from our user study.
\subsubsection{Participants}
We recruited six experts and 52 general audiences. The six experts were from the film industries and different education institutions: two film directors with five and six years of experience, a cinematographer with five years of experience, a full-time university professor with 15 years of film teaching experience, one college lecturer with eight years of theater teaching experience, and one college lecturer with four years of cinematic arts teaching experience. The 52 general audiences were recruited through advertising on social media platforms and snowball sampling. They were 16 females and 36 males, and their ages were between 21 to 33, with an average age of 25.}

\subsubsection{Study Design and Procedure}
We invited 52 general audiences to assess storyboards according to three criteria that conformed to our research goals,
i.e., whether the storyboard can enhance the understandability, impression, and reflection of the core theme.
Six experts were asked to provide a professional assessment on two \rv{more} criteria that required expertise, 
including creativeness \rv{(i.e., to what extent these storyboards be indicated as original and unique)} and \rv{cinematic expression (i.e., to what extent these storyboards could be regarded as cinematic).}
The experts and general audiences rated all 24 storyboards on a 7-point Likert scale. 
Storyboards were grouped according to the same ending style to ensure comparability,
whereas the order was randomly shuffled for each judge to reduce biases.
Judges were blind to the control conditions (i.e., the existence of design guidelines). 

%\textbf{Assessment.}
%We assessed the quality of the storyboards designed by participants by inviting six experts and 52 public judges.
%The six experts included a film director with more than 30 years of experience,
%a film producer from a TV station with 25 years of experience,
%a screenwriter with 18 years of experience,
%a cinematographer with 10 years of experience,
%and two faculties in the film and media arts school with 15 and 5 years of teaching experience, respectively.
%The 52 public audiences were recruited through posts on social media.

% Evaluation part 2-comparative study
%为了帮助验证我们的story ending的设计是否真的能帮助观众更好的对主题进行理解、加深印象和深入思考,以及story ending是否真的更加cinematic(具有cinematic expression)和具有创造力creative(creativeness)。我们进一步邀请了来自电影领域的专家和招募了普通观众来对我们workshop的video ending storyboard进行打分和评估。

% participants
% 我们邀请了六位来自films industry and different universities 的专家:一位是来自电影行业的具有超过30年经验的知名电影导演,一位是来自电视台具有25年电影电视经验的制片人,一位是具有18年电影从业经验的screen writer,一位是具有10年从业经验的电影摄影师,两位是来自电影和传媒学院的教师(一位具有15年教学经验,一位具有5年教学经验)。同时,我们通过广告和社交媒体找到了52位普通观众来参与评估。

%We invited public judges to assess storyboards according to three criteria that conformed to our research goals,
%i.e., whether the storyboard can enhance the understandability, impression, and reflection of the core theme.
%Expert judges were asked to provide a professional assessment on two additional criteria that required expertise, 
%including creativeness and theatricality.

%Each judge rated all 24 storyboards on a 7-point Likert scale.
%Storyboards were grouped according to the used datasets and ending styles to ensure comparability,
%whereas the order was randomly shuffled for each judge to reduce biases.
%Judges were blind to the control conditions (i.e., the existence of design guidelines). 

%\begin{figure*}[!ht]
%  \centering
%  \includegraphics[width=\textwidth]{storyboards.png}
%  \caption{An example of the storyboard created after \textit{using guidelines} by a participant in the user study.}
%  \Description{This figure shows an example of the storyboard created after \textit{using guidelines} by a participant in the user study.}
%  \label{fig:storyboards}
%  \vspace{-10px}
%\end{figure*}

% Study Design and Procedure
% 所有的专家和普通观众都给我们24个作品打分。我们把相同主题的两个ending storyboard work放在同一页问卷中,同时也设置了这些work会随机展现,每一位参与者看见问卷的顺序都是随机生成的。我们告诉打分者这些作品是学生作业请他们凭借经验和直觉进行打分,而他们并不知道有guidelines的存在。专家需要打分的指标有5个:每个故事ending分镜呈现手法对帮助主题的理解understandability/印象impression/思考reflection程度进行打分,同时,也需要对每个结尾的创意creative和电影感 cinematic进行打分。而普通观众需要打分的指标只有3个:对帮助主题的理解understandability/印象impression/思考reflection程度进行打分。我们用的是 a 7-point Likert scale. 总分是7分,7分最高分代表非常同意;6分代表同意,5分代表比较同意,4分代表中立,3分代表比较不同意,2分代表不同意,1分是最低分代表非常不同意。Furthermore, we emphasized that they should rate on the basis of design ideas of the storyboards instead of the efectiveness of the story titles and themes.


\begin{figure}[!t]
\centering
\includegraphics[width=\linewidth]{workshop_result.png}
  \caption{\rv{Guideline participants' ratings with standard errors on the punctuation marks and guidelines. The upper part shows their ratings on each punctuation mark. The lower part shows their rations on each guideline.}}
  \Description{This figure shows the guideline participants' ratings on the punctuation marks and guidelines. For each punctuation, the specific average scores include: (1) the understandability of Full Stop ($Mean = 6.33, SD = 0.58$), Exclamation Point ($Mean = 6.00, SD = 1.00$), Question Mark ($Mean = 6.00, SD = 1.00$), Ellipsis ($Mean = 5.00, SD = 0.00$), (2) the impression of Full Stop ($Mean = 6.33, SD = 0.58$), Exclamation Point ($Mean = 5.33, SD = 2.08$), Question Mark ($Mean = 6.33, SD = 1.53$), Ellipsis ($Mean = 5.33, SD = 1.15$), (3) the reflection of Full Stop ($Mean = 6.33, SD = 0.58$), Exclamation Point ($Mean = 6.33, SD = 0.58$), Question Mark ($Mean = 5.67, SD = 0.58$), Ellipsis ($Mean = 6.67, SD = 0.58$), (4) the clarity of Full Stop ($Mean = 6.00, SD = 1.00$), Exclamation Point ($Mean = 5.33, SD = 0.58$), Question Mark ($Mean = 5.67, SD = 1.15$), Ellipsis ($Mean = 6.33, SD = 0.58$), and (5) the usability of Full Stop ($Mean = 6.00, SD = 0.58$), Exclamation Point ($Mean = 6.00, SD = 0.00$), Question Mark ($Mean = 6.33, SD = 1.00$), Ellipsis ($Mean = 6.00, SD = 0.00$).
For each guideline, the specific average scores include: (1) the clarity of G1.1 ($Mean = 6.00, SD = 1.00$), G1.2 ($Mean = 6.67, SD = 0.58$), G1.3 ($Mean = 6.67, SD = 0.58$), G1.4 ($Mean = 6.67, SD = 0.58$), G1.5 ($Mean = 6.33, SD = 0.58$), G2.1 ($Mean = 5.00, SD = 1.00$), G2.2 ($Mean = 5.67, SD = 0.58$), G2.3 ($Mean = 5.67, SD = 0.58$), G2.4 ($Mean = 5.33, SD = 1.15$), G2.5 ($Mean = 6.00, SD = 0.00$), G3.1 ($Mean = 6.00, SD = 1.00$), G3.2 ($Mean = 5.67, SD = 1.15$), G3.3 ($Mean = 5.67, SD = 1.15$), G3.4 ($Mean = 5.67, SD = 1.15$), G4.1 ($Mean = 6.33, SD = 0.58$), G4.2 ($Mean = 6.00, SD = 1.00$), G4.3 ($Mean = 6.67, SD = 0.58$), G4.4 ($Mean = 6.67, SD = 0.58$), G4.5 ($Mean = 6.33, SD = 0.58$), G4.6 ($Mean = 6.33, SD = 1.15$), and (2) the usefulness of G1.1 ($Mean = 6.00, SD = 1.00$), G1.2 ($Mean = 5.33, SD = 0.58$), G1.3 ($Mean = 6.33, SD = 0.58$), G1.4 ($Mean = 6.67, SD = 0.58$), G1.5 ($Mean = 6.00, SD = 1.00$), G2.1 ($Mean = 5.00, SD = 1.00$), G2.2 ($Mean = 5.00, SD = 1.73$), G2.3 ($Mean = 5.67, SD = 0.58$), G2.4 ($Mean = 5.33, SD = 1.15$), G2.5 ($Mean = 6.00, SD = 0.00$), G3.1 ($Mean = 5.67, SD = 1.15$), G3.2 ($Mean = 5.67, SD = 1.15$), G3.3 ($Mean = 5.33, SD = 1.53$), G3.4 ($Mean = 5.67, SD = 1.15$), G4.1 ($Mean = 5.67, SD = 0.58$), G4.2 ($Mean = 6.00, SD = 0.00$), G4.3 ($Mean = 6.33, SD = 0.58$), G4.4 ($Mean = 6.00, SD = 0.00$), G4.5 ($Mean = 6.33, SD = 0.58$), G4.6 ($Mean = 6.33, SD = 0.58$).}
  \label{fig:workshop_result}
  %\vspace{-10px}
\end{figure}

%\Description{This figure shows the participants' ratings on the punctuation marks and guidelines. For each punctuation, the specific average scores include: (1) the understandability of Full Stop ($Mean = 6.67, SD = 0.578$), Exclamation Point ($Mean = 6.67, SD = 0.578$), Question Mark ($Mean = 5.67, SD = 1.53$), Ellipsis ($Mean = 5.33, SD = 1.53$), (2) the impression of Full Stop ($Mean = 6.67, SD = 0.577$), Exclamation Point ($Mean = 7.00, SD = 0.00$), Question Mark ($Mean = 6.00, SD = 1.00$), Ellipsis ($Mean = 5.00, SD = 1.73$), (3) the reflection of Full Stop ($Mean = 5.67, SD = 1.53$), Exclamation Point ($Mean = 6.67, SD = 0.578$), Question Mark ($Mean = 6.67, SD = 0.578$), Ellipsis ($Mean = 5.67, SD = 2.25$), (4) the clarity of Full Stop ($Mean = 6.00, SD = 0.578$), Exclamation Point ($Mean = 6.67, SD = 0.578$), Question Mark ($Mean = 6.33, SD = 0.578$), Ellipsis ($Mean = 5.33, SD = 2.89$), and (5) the usability of Full Stop ($Mean = 6.00, SD = 1.00$), Exclamation Point ($Mean = 6.67, SD = 0.578$), Question Mark ($Mean = 6.33, SD = 0.578$), Ellipsis ($Mean = 5.33, SD = 1.53$).
%  For each guideline, the specific average scores include: (1) the clarity of G1.1 ($Mean = 6.67, SD = 0.58$), G1.2 ($Mean = 6.33, SD = 1.15$), G1.3 ($Mean = 6.67, SD = 0.58$), G1.4 ($Mean = 5.67, SD = 1.15$), G1.5 ($Mean = 6.33, SD = 0.58$), G2.1 ($Mean = 6.00, SD = 1.00$), G2.2 ($Mean = 7.00, SD = 0.00$), G2.3 ($Mean = 7.00, SD = 0.00$), G2.4 ($Mean = 6.33, SD = 1.15$), G2.5 ($Mean = 7.00, SD = 0.00$), G3.1 ($Mean = 6.00, SD = 1.73$), G3.2 ($Mean = 5.67, SD = 1.15$), G3.3 ($Mean = 6.00, SD = 1.00$), G3.4 ($Mean = 6.33, SD = 1.15$), G4.1 ($Mean = 5.00, SD = 1.73$), G4.2 ($Mean = 5.67, SD = 1.52$), G4.3 ($Mean = 5.00, SD = 1.73$), G4.4 ($Mean = 5.33, SD = 1.53$), G4.5 ($Mean = 5.33, SD = 1.53$), G4.6 ($Mean = 5.33, SD = 1.53$), and (2) the usefulness of G1.1 ($Mean = 6.33, SD = 1.15$), G1.2 ($Mean = 6.00, SD = 1.73$), G1.3 ($Mean = 6.33, SD = 1.15$), G1.4 ($Mean = 6.00, SD = 1.73$), G1.5 ($Mean = 6.33, SD = 0.578$), G2.1 ($Mean = 6.33, SD = 0.578$), G2.2 ($Mean = 7.00, SD = 0.00$), G2.3 ($Mean = 5.67, SD = 2.30$), G2.4 ($Mean = 6.67, SD = 0.578$), G2.5 ($Mean = 7.00, SD = 0.00$), G3.1 ($Mean = 6.00, SD = 1.00$), G3.2 ($Mean = 5.67, SD = 1.15$), G3.3 ($Mean = 5.33, SD = 1.53$), G3.4 ($Mean = 6.67, SD = 0.58$), G4.1 ($Mean = 5.00, SD = 1.73$), G4.2 ($Mean = 5.67, SD = 1.53$), G4.3 ($Mean = 5.00, SD = 1.73$), G4.4 ($Mean = 5.00, SD = 1.73$), G4.5 ($Mean = 5.33, SD = 1.53$), G4.6 ($Mean = 5.33, SD = 1.53$).}

\begin{figure*}[!t]
\centering
\includegraphics[width=\textwidth]{comparative_result.png}
  \caption{\rv{Experts and the general audiences' ratings on the storyboards with standard errors. The upper chart shows the scores of understandability, impression, and reflection about the storyboard ratings by the general audiences, and the lower chart shows the scores of understandability, impression, reflection, cinematic expression, and creativeness about the storyboard ratings by experts.}}
  \Description{This figure shows the experts' and the general audiences' ratings on storyboards with and without guidelines. The upper chart shows the scores of understandability, impression, and reflection about the storyboard ratings by the general audiences, including 
  (1)with guidelines: understandability for full stop ($Mean = 5.94, SD = 0.10$), impression for full stop ($Mean = 5.97, SD = 0.08$), reflection for full stop ($Mean = 6.22, SD = 0.06$); understandability for exclamation ($Mean = 6.01, SD = 0.06$), impression for exclamation ($Mean = 5.97, SD = 0.19$), reflection for exclamation ($Mean = 6.26, SD = 0.06$); understandability for question ($Mean = 5.88, SD = 0.06$), impression for question ($Mean = 5.85, SD = 0.05$), reflection for question($Mean = 6.18, SD = 0.03$); understandability for ellipsis ($Mean = 5.90, SD = 0.10$), impression for ellipsis ($Mean = 6.04, SD = 0.07$), reflection for ellipsis ($Mean = 6.21, SD = 0.13$); 
  (2)without guidelines: understandability for full stop ($Mean = 1.94, SD = 0.34$), impression for full stop ($Mean = 1.83, SD = 0.48$), reflection for full stop ($Mean = 1.58, SD = 0.20$); understandability for exclamation ($Mean = 2.52, SD = 0.32$), impression for exclamation ($Mean = 2.47, SD = 0.12$), reflection for exclamation ($Mean = 1.97, SD = 0.19$); understandability for question ($Mean = 2.37, SD = 0.57$), impression for question ($Mean = 2.27, SD = 0.62$), reflection for question($Mean = 1.96, SD = 0.41$); understandability for ellipsis ($Mean = 2.31, SD = 0.36$), impression for ellipsis ($Mean = 2.23, SD = 0.50$), reflection for ellipsis ($Mean = 1.94, SD = 0.45$).
  The lower chart shows the scores of understandability, impression, reflection, cinematic expression, and creativeness about the storyboard ratings by experts, including 
  (1)with guidelines: understandability for full stop ($Mean = 6.17, SD = 0.29$), impression for full stop ($Mean = 6.06, SD = 0.25$), reflection for full stop ($Mean = 6.22, SD = 0.19$), cinematic expression for full stop ($Mean = 6.28, SD = 0.19$), creativeness for full stop ($Mean = 6.17, SD = 0.33$); understandability for exclamation ($Mean = 5.83, SD = 0.29$), impression for exclamation ($Mean = 5.83, SD = 0.17$), reflection for exclamation ($Mean = 6.06, SD = 0.10$), cinematic expression for exclamation ($Mean = 5.50, SD = 0.29$), creativeness for exclamation ($Mean = 5.67, SD = 0.44$); understandability for question ($Mean = 5.94, SD = 0.25$), impression for question ($Mean = 5.61, SD = 0.25$), reflection for question($Mean = 5.61, SD = 0.10$), cinematic expression for question ($Mean = 5.67, SD = 0.44$), creativeness for question ($Mean = 5.83, SD = 0.17$); understandability for ellipsis ($Mean = 6.11, SD = 0.10$), impression for ellipsis ($Mean = 6.28, SD = 0.10$), reflection for ellipsis ($Mean = 6.22, SD = 0.10$), cinematic expression for ellipsis ($Mean = 6.11, SD = 0.19$), creativeness for ellipsis ($Mean = 6.22, SD = 0.19$); 
  (2)without guidelines: understandability for full stop ($Mean = 2.11, SD = 0.96$), impression for full stop ($Mean = 2.17, SD = 0.73$), reflection for full stop ($Mean = 2.00, SD = 0.73$), cinematic expression for full stop ($Mean = 1.61, SD = 0.59$), creativeness for full stop ($Mean = 1.83, SD = 0.58$); understandability for exclamation ($Mean = 2.94, SD = 0.25$), impression for exclamation ($Mean = 2.77, SD = 0.19$), reflection for exclamation ($Mean = 2.67, SD = 0.29$), cinematic expression for exclamation ($Mean = 2.28, SD = 0.67$), creativeness for exclamation ($Mean = 2.50, SD = 0.50$); understandability for question ($Mean = 2.28, SD = 0.35$), impression for question ($Mean = 1.89, SD = 0.35$), reflection for question($Mean = 1.83, SD = 0.33$), cinematic expression for question ($Mean = 1.72, SD = 0.35$), creativeness for question ($Mean = 1.89, SD = 0.48$); understandability for ellipsis ($Mean = 1.89, SD = 0.75$), impression for ellipsis ($Mean = 2.06, SD = 0.67$), reflection for ellipsis ($Mean = 1.83, SD = 0.58$), cinematic expression for ellipsis ($Mean = 1.39, SD = 0.25$), creativeness for ellipsis ($Mean = 1.39, SD = 0.25$).}
  \label{fig:comparative_result}
  \vspace{-10px}
\end{figure*}

%\subsection{Results}
\rv{\subsection{User Study and Comparative Study Results}}
%We reported the quantitative and qualitative feedback from the participants in the user study and the judges in the assessment stage.
\rv{In this section, the quantitative and qualitative feedback from the participants in the user study and comparative study are reported.}
%\subsubsection{Participants' Ratings on Guidelines}
\rv{\subsubsection{User Study Results}
%Generally, participants agreed on their satisfaction \includegraphics[height=6pt]{satisfication.png} ($M = 5.58, SD = 1.38$) with their designs with guidelines, as well as the inspirations \includegraphics[height=6pt]{inspiration.png} ($M = 6.50, SD = 0.67$) our guidelines provided on a 7-point Likert scale.

Generally, participants who used guidelines agreed on their satisfaction ($M = 5.5, SD = 1.31$) with their designs as well as the inspirations ($M = 6, SD = 1.04$) our guidelines provided on a 7-point Likert scale.}

%Overall, participants indicated that \textit{our styles and guidelines inspired their designs} \includegraphics[height=6pt]{inspiration.png} (M = 6.3, SD = 0.67), and \textit{they were satisfied with their final designs} \includegraphics[height=6pt]{satisfication.png} (M = 6, SD = 0.67) on a 7-point Likert scale.
With our teaching materials, they found that they could clearly understand \rv{($M = 5.83, SD = 0.83$)} and use \rv{($M = 6.08, SD = 0.51$)} each punctuation with corresponding guidelines. 
They also provided positive feedback on the overall effects of each punctuation with corresponding guidelines in enhancing the understandability \rv{($M = 5.83, SD = 0.83$)}, impression \rv{($M = 5.58, SD = 1.31$)}, and reflection \rv{($M = 6.5, SD = 0.52$)} of the core theme.
For clarity and usefulness (to what extent
the guideline could help participants realize punctuation), the participants
provided positive scores (\autoref{fig:workshop_result}).

% user study result
%对于Guidelines使用情况统计:
%句号:G1.1-P5 G1.2-P5 G1.3-P6 G1.4-P4,P6 G1.5-P4,P5
%感叹号:G2.1-P3 G2.2-P2,P3,P8 G2.3-P2,P3 G2.4-P2,P3 G2.5-P3
%问号:G3.1-P11 G3.2-P9 G3.3-P9 G3.4-P10,P11
%省略号:G4.1-P1,P12 G4.2-P1,P7,P12 G4.3-P1 G4.4-P1 G4.5-P12 G4.6-P7
%经过统计,20条guidelines都被12位设计师在创作ending的过程中使用了,并且都给予了很高的评价。participants indicated that our styles and guidelines inspired their designs(M = xxx, SD = xxx), and they were satisfied with their final designs with guidelines(M = xx, SD = xxx) on a 7-point Likert scale. With respect to the attractiveness, clarity, and usability (to what extent they could apply the style by using our teaching materials) of each cinematic opening style, as well as the clarity and usefulness (to what extent the guideline could help them achieve the style), the participants also gave positive scores (see Figure xxx).

% We also analyzed the usage of guidelines.
% \aoyu{all is covered?}
%In the post-study interview,
%all participants agreed that the guidelines helped improve the efficiency in designing story endings.
%P1 praised the ideas of punctuation marks that ``are very intuitive''. He continued, ``Endings without punctuation are confusing. The guidelines are well aligned with each mark.''
%P2, the author of the design in~\autoref{fig:storyboards}, appreciated the visualization of guidelines, commenting ``the visualization in G2.4 gave me great inspiration to show multiple visualizations without conflicting information, which created an exclamation.''
%Participants also praised the diversity of guidelines that covered the data and cinematic aspects of visual and content design.
%However, they suggested that our guidelines could add considerations about audio in the future. 

%在采访过程中,我们向参与者提问guidelines对于他们过程中能否帮助他们提高ending的设计效率和达到目的和是否有解决他们在没用使用guidelines前遇到的困难,12位参与者都表示肯定(P1:用标点符号来帮助设计师做结尾这个想法真的很有效和准确。因为没有标点符号的结尾是混乱不清晰的,guidelines里面的做法都很符合标点符号的需求。P2(文中展现作品fig:storyboard):结尾需要有一个总结,但是我又要做出感叹号的结尾,这是让我觉得困难的地方。但是当我看到G2.4的建议尤其是visualization的图告诉我可以使用分屏来展现信息和突出数据,这就满足了既可以展现过程来突出冲突感叹号又可以总结的需求。P3:我发现每一条guidelines之间的思想都一致的,比如G2.4和G2.1都在知道展现冲突,同时有符合感叹号的需求,让我感到这些指南都经过精心设计!P8:对于结尾设计首先我考虑的是可读性,需要表达清晰,尤其是省略号的结尾很多时候是需要延伸出新的信息,省略号的G4.4给了我非常大创作启发,可以通过zoom-in(out)来展现infinite and beyond data是我没有想到的展现方式,我也把他运用到了我的创作之中。P10:对于结尾提出一个问题,其实是能给设计师一个很大的创作和发挥空间的,G3.4提出可以将想象和现实结合来让观众产生疑问的方式,可以说是一个既能给设计师方向,又不会限制创作者的指南,我一看见就迫不及待地很想尝试!),随后也向12位参与者提问他们是否能有一些新的guidelines补充。他们表示我们的guidelines都比较完整,无论是从内容还是视觉上都涵盖了电影上的镜头运动和数据上排列指导,唯一有一位参与者提出我们以后的工作可以考虑加入声音设计的指导。
\rv{In the post-study interview, the main consideration for endings' design was how to express data insight completely and intuitively, which could be facilitated by our cinematic styles and guidelines. P8 commented, ``The idea of four punctuation marks as a classification and framework of different ending contextualize both visual and content, which is very intuitive and impressive.''
All participants with guidelines (P1-P12) agreed that our guidelines were diverse and helpful to inspire their creating data video endings. P12 praised the inspiration of G4.4 ``the shot design recommendation of zoom-in (out) to present the data expansion or development inspires my ellipsis ending design a lot while I was thinking about how to extend the new information more fluently.'' 
However, they suggested that our guidelines could add considerations about audio in the future.}

%result

%\subsubsection{Assessment Results}
\rv{\subsubsection{Comparative Study Results}
For each storyboard, we averaged the score of the five experts on how much it enhanced the understandability, impression, and reflection of the theme as well as its cinematic expression and creativeness (\autoref{fig:comparative_result}).
For each metric, we conducted a t-test on the score of storyboards designed with and without guidelines.
We found that the scores of storyboards using the guidelines were significantly higher than those of storyboards not using guidelines (all $ p < 0.001 $) on all five metrics.
From the rating by the general public, we found similar results.
In terms of the metrics---understandability, impression, and reflection, the scores of storyboards guided by guidelines were significantly higher than those of storyboards without guidelines (all $ p < 0.001 $).}



%\Description{This figure shows the experts' and the general audiences' ratings on storyboards with and without guidelines. The upper chart shows the scores of understandability, impression, and reflection about the storyboard ratings by the general audiences, including 
%  (1)with guidelines: understandability for full stop ($Mean = 5.98, SD = 0.20$), impression for full stop ($Mean = 6.10, SD = 0.19$), reflection for full stop ($Mean = 5.98, SD = 0.14$); understandability for exclamation ($Mean = 5.97, SD = 0.07$), impression for exclamation ($Mean = 6.01, SD = 0.19$), reflection for exclamation ($Mean = 6.06, SD = 0.13$); understandability for question ($Mean = 6.15, SD = 0.14$), impression for question ($Mean = 6.10, SD = 0.16$), reflection for question($Mean = 6.13, SD = 0.13$); understandability for ellipsis ($Mean = 6.13, SD = 0.09$), impression for ellipsis ($Mean = 6.20, SD = 0.11$), reflection for ellipsis ($Mean = 6.22, SD = 0.11$); 
%  (2)without guidelines: understandability for full stop ($Mean = 2.95, SD = 0.28$), impression for full stop ($Mean = 3.02, SD = 0.30$), reflection for full stop ($Mean = 2.88, SD = 0.30$); understandability for exclamation ($Mean = 4.15, SD = 0.68$), impression for exclamation ($Mean = 4.02, SD = 0.70$), reflection for exclamation ($Mean = 3.99, SD = 0.79$); understandability for question ($Mean = 4.11, SD = 0.85$), impression for question ($Mean = 3.96, SD = 0.97$), reflection for question($Mean = 4.02, SD = 1.03$); understandability for ellipsis ($Mean = 2.99, SD = 0.17$), impression for ellipsis ($Mean = 2.97, SD = 0.16$), reflection for ellipsis ($Mean = 2.94, SD = 0.25$).
%  The lower chart shows the scores of understandability, impression, reflection, cinematic expression, and creativeness about the storyboard ratings by experts, including 
%  (1)with guidelines: understandability for full stop ($Mean = 6.22, SD = 0.51$), impression for full stop ($Mean = 6.22, SD = 0.25$), reflection for full stop ($Mean = 6.00, SD = 0.50$), cinematic expression for full stop ($Mean = 5.94, SD = 0.42$), creativeness for full stop ($Mean = 6.06, SD = 0.35$); understandability for exclamation ($Mean = 6.39, SD = 0.19$), impression for exclamation ($Mean = 6.17, SD = 0.29$), reflection for exclamation ($Mean = 6.05, SD = 0.10$), cinematic expression for exclamation ($Mean = 5.89, SD = 0.25$), creativeness for exclamation ($Mean = 6.00, SD = 0.17$); understandability for question ($Mean = 5.50, SD = 0.50$), impression for question ($Mean = 5.44, SD = 0.59$), reflection for question($Mean = 5.56, SD = 0.35$), cinematic expression for question ($Mean = 5.44, SD = 0.48$), creativeness for question ($Mean = 5.61, SD = 0.42$); understandability for ellipsis ($Mean = 6.05, SD = 0.51$), impression for ellipsis ($Mean = 6.11, SD = 0.35$), reflection for ellipsis ($Mean = 6.06, SD = 0.38$), cinematic expression for ellipsis ($Mean = 5.89, SD = 0.59$), creativeness for ellipsis ($Mean = 6.28, SD = 0.25$); 
% (2)without guidelines: understandability for full stop ($Mean = 2.11, SD = 0.69$), impression for full stop ($Mean = 2.00, SD = 0.44$), reflection for full stop ($Mean = 1.94, SD = 0.25$), cinematic expression for full stop ($Mean = 1.39, SD = 0.67$), creativeness for full stop ($Mean = 1.56, SD = 0.54$); understandability for exclamation ($Mean = 2.89, SD = 0.42$), impression for exclamation ($Mean = 2.89, SD = 0.60$), reflection for exclamation ($Mean = 2.50, SD = 0.67$), cinematic expression for exclamation ($Mean = 1.78, SD = 0.51$), creativeness for exclamation ($Mean = 2.33, SD = 0.60$); understandability for question ($Mean = 2.89, SD = 0.35$), impression for question ($Mean = 5.44, SD = 0.59$), reflection for question($Mean = 2.72, SD = 0.63$), cinematic expression for question ($Mean = 2.5, SD = 0.44$), creativeness for question ($Mean = 2.61, SD = 0.68$); understandability for ellipsis ($Mean = 2.03, SD = 0.39$), impression for ellipsis ($Mean = 2.33, SD = 0.93$), reflection for ellipsis ($Mean = 2.00, SD = 0.60$), cinematic expression for ellipsis ($Mean = 1.56, SD = 0.51$), creativeness for ellipsis ($Mean = 1.72, SD = 0.48$).}

\subsection{Gallery}
To demonstrate the application of our cinematic guidelines, three authors of this work (a film scriptwriter, a film visual designer, and a data visualization designer) created four data videos\rv{, each of which ended on one of} the four punctuation marks (\autoref{fig:fourendings}). 
The data videos were about the recent COVID-19 pandemic and demonstrate how each of the four ending punctuation marks can be visualized with different cinematic ending styles. 
For example, based on the ellipsis ending style and corresponding guidelines G4.1, G4.2, and G4.4, we have designed a visualization of a running time clock and the fading images of the elderly, symbolizing and emphasizing the great sense of urgency to improve the vaccination rate. 
These videos can be found in the supplementary materials and on our website (\url{https://cinematicendings.github.io/}).

\begin{figure*}[!t]
\centering
\includegraphics[width=\textwidth]{fourendings.jpg}
  \caption{\rv{Four data video endings that applied our four cinematic ending styles and guidelines. From left to right, four screenshots of four data video endings with \textit{full stop}, \textit{exclamation point}, \textit{question mark}, and \textit{ellipsis} are from the gallery on our website.}}
  \Description{This figure shows four data video endings that used our four cinematic ending styles and guidelines. From left to right, four screenshots of four data video endings with \textit{full stop}, \textit{exclamation point}, \textit{question mark}, and \textit{ellipsis} are from the gallery on our website.}
  \label{fig:fourendings}
  \vspace{-10px}
\end{figure*}

%To demonstrate the application of our cinematic guidelines, three authors of this work (a film scriptwriter, a film visual designer, and a data visualization designer) have created four data videos with the ends of the four punctuation marks.  
%Using the data about the high death rate of the elderly in the most recent 2022 pandemic crisis in Hong Kong, the data videos demonstrate how each of the four ending punctuation marks can be visualized with different cinematic ending styles. 
%For example, based on the ellipsis ending style and corresponding guidelines G4.1, G4.2 and G4.4, we have designed a visualization of a running time clock and the fading images of the elderly, symbolizing and emphasizing the great sense of urgency to improve the vaccination rate. 
%These videos can be found in the supplemental material and and our website (\url{https://cinematicendings.github.io/}).

% To demonstrate the potential application of Kineticharts, one author
% who was a data journalist and two authors who are professional visual
% designers have created ten short data stories using Kineticharts. Unlike
% the stimuli used in the user studies whose visual representations have
% been controlled, these new data stories were created to mimic real-world
% storytelling practice and show the possible utilization of Kineticharts.
% We created two stories for each of the five positive affects using Kineticharts. The stories were adapted from real data stories and are based
% on different data sizes or topics. We also integrated factors such as color,
% embellishment, and music into the stories to show how Kineticharts
% work in combination with other design techniques. The gallery can be
% accessed at https://kineticharts.idvxlab.com/.

%To demonstrate the potential application of our cinematic guidelines, 
%three authors of this work (including a film script writer, a film visual designer, and a data visualization designer) have created four data videos using four punctuation marks endings.
%Those videos all talk about the recent COVID-19 pandemic but end differently.
%For example, the ellipsis style ending follows G4.1-2,4,6 to show a clock visualization that symbolises the vaccination progress, emphasizing that much efforts are needed to improve the vaccination rate.
%We provide them as a gallery in the supplemental material.

\section{Discussion}
%In this section, we discuss the limitations and implications of this work.

\rv{In this section,
we discuss the limitation of our work and the lessons learned for improving data storytelling.}

\textbf{Towards a deeper understanding of data story endings.}
We classified data story endings according to the framework of punctuation marks,
which are intuitive enough to understand and help communicate the relevance between films and data videos.
However, our classification was not without limitations.
First, punctuation marks can be used in combination.
For instance, the interrobang mark\footnote{\url{https://en.wikipedia.org/wiki/Interrobang}} is a common technique in advertisement and comic endings to combine the functions of the question and the exclamation mark.
In the future, we plan to study such combined use to deepen our understanding of data story endings.
% For instance,
% the Soviet film ~\textit{Come and See (1985)}~\cite{come} presents a very emotionally intense eight-minute long film ending sequence that combines the tones of exclamation and ellipsis marks.
% Although we noted such combined use in our analysis, we decided to focus on the individual use for study manageability.
% In the future, we plan to study such combined use of punctuation marks to deepen our understanding of data story endings.
Second, different punctuation marks might generate similar effects on audiences despite their differences in form and content design.
For instance, both ellipsis and question mark could leave audiences in suspense, whereas the former has a smaller, softer effect.
To distinguish them better, we plan to investigate their influences on audiences.
~\rv{In addition, data analytics and data storytelling should be balanced and incorporated in the data story ending. 
Data and analytic thinking might be decoupled from thoughts and feelings. However, it is crucial to follow the data analysis process to conclude data insights~\cite{provost2013data} for data story endings in the first step and then utilize storytelling techniques to help reveal information and data insights effectively and intuitively~\cite{gershon2001storytelling}. 
The importance of balancing data analytics and storytelling is recognized, and we plan to observe and explore their relationship further in future work.}
% The ending features the young man repeatedly shooting at the picture of Hilter intercutting with a montage sequence of historical war footage being played in reverse to Hilter’s rise of power and further backwards to the image of enfant Hilter under his mother’s arms. 
% The repeated shooting then stopped, and the character broke down in tears. 
% Then he reunited with his troop and continued on the journey.
% This unique ending style displays a powerful expression of intense emotional feelings.
% In the future,
% we plan to deepen our understanding of data stories endings by exploring such combined use and investigating their influences on audiences’ emotions.
% 删减了一点因为感觉太长了

%4. 对于punctuations purpose和punctuations combination的探讨 (这节可要可不要)
%每一种标点符号都有其对应的定义,其实也有不少经典的电影结尾其实是标点符号的结合,这也是为什么我们在标记电影和数据视频ending的时候需要各位作者反复推敲和讨论。因为如果考虑将结合的标点符号一起作为common styles分析的话,第一是情况会变复杂,第二是出现的guidelines会和单个标点符号进行重合。因此我们在这次研究中只考虑了单个标点符号。比如问号和感叹号的组合、感叹号和省略号的组合我们是我们做数据中比较常见的情况。举一个电影中的例子:
%An unique example of the combined use of punctuation marks in film. 
%The Soviet film Come and See (1985) presents a very emotionally intense eight minute long film ending sequence that combines the tones of exclamation and ellipsis marks.
%The ending features the young man shooting repeatedly at the picture of Hilter intercutting with montage sequence of historical war footage being played in reverse to Hilter’s rise of power and further backwards to the image of enfant Hilter under his mother’s arms. 
%The repeated shooting then stops and the character broke down in tears. 
%Then he reunited with his troop and continued on the journey.
%This unique ending style displays a powerful expression of intense emotional feelings.
%组合标点符号也能有表达出强烈的情感和做一个让观众印象深刻的结尾。对于标点符号的combination研究也可以是未来的进一步研究


\textbf{Making guidelines visual and accessible.}
A core challenge encountered during this research was to make guidelines more accessible,
that is, simple and intuitive enough for designers to understand and appreciate easily.
Our solutions were the ``visualization'' of guidelines that helps designers browse and select guidelines based on their intended styles (\autoref{fig:design}).
In addition to text descriptions,
we created intuitive diagrams visualizing each guideline.

Furthermore,
an interactive website (\url{https://cinematicendings.github.io/}) was provided for designers to explore examples,
facilitating their comprehension and understanding.
The participants in our experiment highly appreciated those materials.
To benefit future researchers and users,
we created data videos of four different ending styles based on our guidelines.
We also placed those videos together with participants' storyboards in our online gallery to promote research in narrative visualizations by combining visual and storytelling techniques in cinematic arts, and propose practical guidelines for designers to improve their data storytelling ``in the wild.''
Moving forward,
we plan to continue research on integrating film and visualization studies and investigating other factors such as emotions.

% 2. four punctuations and guidelines 在ending该如何更好被理解和使用
%我们除了研发guidelines,也在一直focus如何可以帮助设计师更好更高效地理解与运用我们的guidelines。为此,除了文字版的guidelines以外,我们为设计师创作guidelines使用流程图(cite figxx)。整个流程图示模拟设计师真是创作的过程--首先可供设计师选择想要完成一个什么标点符号的ending,随后,让设计师基于自己手头上有的材料和素材进行选择elements,而我们的guidelines则是对应elements的具体应用指导,而图表的最右边是我们自创的可视化可以呈现出的示例图。同时,我们还为设计师准备了我们的网站(citexxx),网站里还有对于每个guidelines的story case展示,每个case都有单独的GIF和文字来共同解释,这些case的来源包括我们从电影、数据视频和我们团队设计师的改良创作作品。在workshop的实验过程中,我们也能观察到对设计师的帮助。同时,我们团队也根据guidelines原创了和当下covid-19主题相关的系列数据视频,并制作出了以4种不同标点符号结尾的四个版本视频放在了我们网站的gallery。我们也把12位设计师的24个作品中的12个使用guidelines后的作品展现在了我们网站的storyboard中,希望可以给日后设计师一个做data video ending的启发和更直观的展现。我们一直追求的目标是:结合电影的visual and storytelling技巧为可视化领域贡献出能帮助数据视频做出一个有创意、帮助理解、印象和思考的cinematic ending,并且这些guidelines可以让更多的设计师可以真正的使用,并且能解决data storyteller在讲述数据故事上面的困难。未来我们也是会继续借鉴电影的经验,进一步探索电影中对观众情绪的影响,进而尝试应用到数据故事之中。
%\rv{\textbf{The Generalizability of the Cinematic Styles and Guidelines.}
%We studied the endings of narrative visualizations in the form of data videos. We observed that our cinematic styles and guidelines could be extended to other narrative visualization forms~\cite{segel2010narrative} besides data videos. On the one hand, our cinematic styles and guidelines might be applicable to narrative visualization with storylines and consist of a sequence of frames to present their data insights, such as data comics and data GIFs.  
%On the other hand, our cinematic styles and guidelines might also be applied to interactive visualization, such as data presentation~\cite{shin2022roslingifier}, a new genre of storytelling technique to use interactive visualization to support in-person presentations. Our results could recommend diverse designs for different purposes of data story endings. We can systematically investigate design patterns for interactive visualization in future work.}

\rv{\textbf{Creating interactive tools for data storytelling.}
During our user study, participants were asked about their expectations of future tools for creating data stories. Participants praised the usefulness of our interactive website and study materials, which guided them in selecting punctuation marks, shot/design recommendations, and guidelines step by step. Specifically, they hoped for an interactive system that could recommend narration structures given the user-input data tables and constraints (e.g., ``ending with a question mark''). Such intelligent approaches could help facilitate the design process in crafting story structures and improve efficiency and accuracy. 
Future work can further combine these intelligent approaches with clear story structure temples as a fundamental part of interactive data video authoring tools.}
%Meanwhile, inspired by interactive films, data videos can be ended with multiple endings of different meanings and perspectives. We proposed four types of endings adaptable to data videos, which can be chosen by designers to create and selected by audiences to watch. As such, we have provided a gallery with four data videos which end with four types of cinematic endings on our website. This also illustrates a potential form of interactive data visualization in which visual content can be expressed differently at the end to create different meanings and perspectives. 
%Future work can further develop this design concept of multiple cinematic endings as a fundamental part of interactive data video authoring tools.}
%Except for their endings, four data videos have the same visual and content, which is also an example of the potential of interactive data visualization form. In short, we expect our work to build a fundamental design idea with multiple cinematic endings for interactive data video authoring tools.
%content and form, for example.


%can be generalized to 
%the shot and design recommendations can be applied to visual and content-related data visualization endings.

\rv{\textbf{Generalizability of the Cinematic Styles and Guidelines.}
We studied the endings of narrative visualizations in the form of data videos. Our cinematic styles and guidelines could be extended to other narrative visualization forms~\cite{segel2010narrative, ragan2015characterizing}. 
First, our cinematic styles and guidelines can be applicable to non-interactive narrative visualization that consists of a sequence of frames similar to the key property of data videos to present their data insights, such as \textit{data GIFs}. 
For example, our ending style ``exclamation point'' and its guidelines are suitable for \textit{data GIFs} which are expected to convey messages in a short time~\cite{shu2020makes} and leave the audience with an impression and surprise.
%For example, four types of cinematic ending styles can help data GIFs end with conclusions, surprises, questions, or open interpretation through our content and visual recommendations.
Second, our cinematic styles and guidelines can be generalized to interactive visualization endings. 
Inspired by interactive films~\cite{clarke2001film} and interactive data comics~\cite{wang2021interactive}, data videos can be ended with multiple endings to create different effects. 
In our gallery, we have created four data videos that convey the same data insight but end with four types of cinematic endings (\autoref{fig:fourendings}).
Thus, we propose that four types of endings can be chosen by designers for creating interactive data stories which could present different endings based on audiences’ selections.
This interactive structure is similar to the Martini Glass visualization structure~\cite{segel2010narrative}. The audience should follow the author-driven narrative in the opening and climax~\cite{freytag1908freytag} of data videos, and then they can interact with the multiple endings through the reader-driven narrative. Once the audience selects one of the ending styles based on their preferences, this ending may potentially become a jumping off point for the next data visualization with new insights. Therefore, our diverse data video ending designs show a potential form of interactive visualization applications. 
In addition, these ending styles remain unclear on how to interact appropriately with audiences’ preferences. We can systematically investigate the interactive ending designs based on audiences’ preferences in future work.}
%Although these ending styles remain unclear on how to implement appropriate interaction with audiences' preferences, we can systematically investigate the user's emotional preferences when they select different ending styles in future work.}
% This also illustrates a potential form of interactive data visualization in which visual content can be expressed differently at the end to create different meanings and perspectives.
%Second, our cinematic styles and guidelines might also be applied to interactive narrative visualization, such as data presentation~\cite{shin2022roslingifier}, a new genre of storytelling technique to use interactive visualization to support in-person presentations. 
%Our results could recommend diverse designs for different purposes of data story endings. 
%We can systematically investigate the content and visual design patterns for interactive visualization in future work.}

%\textbf{Creating tools for data storytelling.}
%During the user study,
%we asked participants about their consideration factors and problems encountered in creating data stories.
%Based on their feedback,
%we summarized three research questions for future tools to enhance data storytelling.
%First,
%participants suggested a gallery of data stories and templates in terms of the narration content, visualizations, and visual styles.
%Such a gallery could spark inspiration and improve efficiency in designing data stories.
%Second,
%participants envisioned AI techniques for recommending semantic knowledge based on data,
%e.g., given the mortality rate of 0.03\% to recommend that ``the relatively lower rate suggests the effectiveness of vaccine''.
%This will require more research on visualization recommenders to consider semantic information to turn data into knowledge.
%Finally, participants hoped for an interactive system that could recommend narration structures given the user-input data tables and constraints (e.g., ``ending with a question mark''). Such intelligent approaches could help reduce designers' expertise in crafting story structures and improve efficiency.



%Such intelligent approaches could help reduce designers' experts in crafting story structure and therefore improve efficiency.


% 3. 设计ending时遇到的挑战和关于future tool的实现
%我们在user study的采访时,向设计师提问了“在设计ending时会思考什么因素和遇到哪些困难?”“如果未来可以开发一款工具帮助设计师提升设计数据故事的效率,你希望是什么样的?”。我们整理了设计师。 在这轮的回答主要集中在3个方面--(1)第一类是希望未来的工具可以是一个数据库,提供可以有一些典型的标点符号代表的数据故事包括故事的组织方式和图表的呈现作为模版,以启发他们在故事和视觉设计时产生的灵感和提高设计效率。并有参与者P3提出我们现在这个网站具有story case的展现大大启发了她的设计(2)第二类是提出可以用AI学习来自动把数据变成图表、讲数据进行清晰准确的匪类,AI可以节省下数据设计师消化数据的过程,帮助对这个数据背景不熟的设计师进行对概念的理解。可以和社会现状相结合直接给出数字背后的含义--比如“0.03%的死亡率其实代表的疫苗的作用还是很高”的认识。这样可以解决创作者是在数据理解和处理上遇到的困难,以及提升生产数据可视化的效率。(3)第三类是提出可以做一个和data story structure相关的工具,因为有些设计师会在思考数据故事的时候花费很多时间,而往往对于数据故事的结尾更需要精心设计。那么如果这个工具可以让设计师输入数据,则可以推荐出相应标点符号的结尾的结构和文本,设计师只需要对照文本来创作,这也能大大提高设计的效率。(4)第四类是提到我们可以做一个交互工具,用if  else结合,让这个工具可以给不同水平的设计师甚至新手都可以结合他们实际的问题进行使用。




%5. Is it the end? there is no real ending
%Is it over?是我们研究结尾的时候对结尾提出的第一个问题,随后,我们从is it over延伸到了四种标点符号可能成为的结尾类型。
%"Is it over?" is a question that audiences often ask after watching a film, and it is the cinematic styles in the ending lead to lively discussions and profound reflections. Is it over ? There is a Chinese idiom ‘paint the dragon, dot the eyes’, which means giving the final touch of a dragon painting by dotting their eyes last. To put it in the context of data video, the final touch of a data video can be interpreted as giving the data story its creative ending with style. As American sci-fi author Frank Herbert once said, ‘there is no real ending. It's just the place where you stop the story.’ Ending is not just the last part of a story, it’s an important integral of the whole story. It is like dotting the eyes of a dragon painting at last. It could also be like a punchline of a joke. It comes together at the end in order to go for the emotional response from the audience. Therefore, an ending does not mean the storytelling is over; it could just mean the beginning of a new chapter.


%关于标点符号的overlap
%关于overlap of Question marks and Ellipsis marks

%同:两者的含义中都有关于what is next的语境。

%question:In this way, this ending ends with new thoughts and ideas. It takes the audience to the next level of thinking or reflection at the end. 

%ellipsis: This category also applies to an ending that aims to offer one solution or resolution to a story or problem, which is up to the audience to make their meaning out of the intended ambiguity.

%异:question的语气会比ellipsis疑问感更强烈和清楚。
%(这一段话是ellipsis最后一段:也许这里能用上?)The filmmaker also expresses his personal feelings behind the data at the end. His feelings are not defined in the form of forceful exclamation or question marks, as his forward-looking view of uncertainty is softly expressed as an ellipsis statement. 
%His theme about the lesson as continuous development is consistently expressed in his endings. Therefore, these endings function as more ellipsis marks than exclamation or question marks that usually carry a stronger tone of voice. 








\section{Conclusion}
Endings have received wide attention in literature and film studies.
Endings are not the real end of the story but instead emphasize the concluding statement with stronger impressions and persuasion powers.
% However,
% little research has investigated endings in data storytelling.
We studied how to enhance the endings of data videos by drawing inspiration from hundreds of~\rv{famous} film endings and contextualizing them to popular data videos.
Our work highlights an interdisciplinary methodology,
that is,
we involved participants from different backgrounds (e.g.,~film experts, designers, and data analysts) in all stages of our work (e.g., \rv{analyzing films and data videos to derive styles, deriving guidelines through expert interviews, and validating guidelines through a user study and a comparative experiment}).
% Study results provided evidence on the usefulness of our guidelines and research.
% For instance,
% multiple designers got stuck in creating data video endings without guidelines,
% \ie~they tended to simply enumerate charts. 
% In contrast,
% our guidelines brought up new perspectives for them to create endings,
% which was subsequently shown to be more impressive and thought-provoking.
We hope our work could inspire future research to enhance data storytelling by extending not only the disciplinary depth but also the interdisciplinary breadth. 
% remark: 这一段写的更像是 conclusion,而不是 Discussion


% Discussion/limitation/future work
% 1. Making data video ending xxx (这小节讲的是,我们这次研究的出发点-过程-意义-结果-future work)
% 一个故事的ending一直都被文学、电影等领域重视和研究,Ending is not really ending; ending invites viewers to think what will happen next. 基于此,我们也在尝试研究出一套如何讲好数据故事,适用于数据视频的ending指南。我们借鉴了具有超过一百年生产和传播历史的电影和综合了近二十年里优秀的数据视频,由一个7名研究成员的团队分成:data collection、data analysis、styles mining、expert interviews四个阶段来反复推敲生成了20条适用于数据视频的能创造出cinematic ending的guidelines。随后我们邀请了12位数据设计师对我们的guidelines进行审查并试用,和邀请了6位电影领域专家和50位观众对我们设计师创作的24个对照组作品进行了评估。这些effort都为我们的guidelines是有用和有效的提供了有力的证据。在creativeness和cinematic expression上,无论是从user study的观察或是专家的评估中都得到了很大的肯定。许多设计师在没有看我们的指南之前,让他们根据dataset和已有主题theme创作出一个结尾是比较困难的,而且会用数据设计师的固化思维只是对数据图表进行罗列,而当我们给设计师介绍完guidelines之后,他们纷纷表示豁然开朗,有很大的启发,无论是从创作效率还是创作的质量都得到了很大的提高。从原来25分钟只能想到一个分镜头,到学习完guidelines之后能在25分钟创作出7个分镜头。设计师使用前后的作品评分从各项指标都差距较大。使用后明显比使用前更有助于对主题的理解、印象和思考。future work我们可以思考针对数据视频的climax阶段如何进行准则研发。

\begin{acks}
This research was supported in part by Hong Kong Theme-based Research Scheme T41-709/17N. We would like to thank our interview and user study participants for sharing their thoughts with us. Finally, we thank our reviewers for their constructive comments.

\end{acks}

\balance

%%%\bibliographystyle{ACM-Reference-Format}
%%%\bibliography{main}

%%%\bibliographystylesupp{ACM-Reference-Format}
%%%\bibliographysupp{main}


%%% -*-BibTeX-*-
%%% Do NOT edit. File created by BibTeX with style
%%% ACM-Reference-Format-Journals [18-Jan-2012].

%%% -*-BibTeX-*-
%%% Do NOT edit. File created by BibTeX with style
%%% ACM-Reference-Format-Journals [18-Jan-2012].

\providecommand{\noopsort}[1]{}
\begin{thebibliography}{76}

%%% ====================================================================
%%% NOTE TO THE USER: you can override these defaults by providing
%%% customized versions of any of these macros before the \bibliography
%%% command.  Each of them MUST provide its own final punctuation,
%%% except for \shownote{}, \showDOI{}, and \showURL{}.  The latter two
%%% do not use final punctuation, in order to avoid confusing it with
%%% the Web address.
%%%
%%% To suppress output of a particular field, define its macro to expand
%%% to an empty string, or better, \unskip, like this:
%%%
%%% \newcommand{\showDOI}[1]{\unskip}   % LaTeX syntax
%%%
%%% \def \showDOI #1{\unskip}           % plain TeX syntax
%%%
%%% ====================================================================

\ifx \showCODEN    \undefined \def \showCODEN     #1{\unskip}     \fi
\ifx \showDOI      \undefined \def \showDOI       #1{#1}\fi
\ifx \showISBNx    \undefined \def \showISBNx     #1{\unskip}     \fi
\ifx \showISBNxiii \undefined \def \showISBNxiii  #1{\unskip}     \fi
\ifx \showISSN     \undefined \def \showISSN      #1{\unskip}     \fi
\ifx \showLCCN     \undefined \def \showLCCN      #1{\unskip}     \fi
\ifx \shownote     \undefined \def \shownote      #1{#1}          \fi
\ifx \showarticletitle \undefined \def \showarticletitle #1{#1}   \fi
\ifx \showURL      \undefined \def \showURL       {\relax}        \fi
% The following commands are used for tagged output and should be
% invisible to TeX
\providecommand\bibfield[2]{#2}
\providecommand\bibinfo[2]{#2}
\providecommand\natexlab[1]{#1}
\providecommand\showeprint[2][]{arXiv:#2}

\bibitem[Abbott(2021)]%
        {abbott2021cambridge}
\bibfield{author}{\bibinfo{person}{H~Porter Abbott}.}
  \bibinfo{year}{2021}\natexlab{}.
\newblock \bibinfo{booktitle}{\emph{The Cambridge introduction to narrative}}.
\newblock \bibinfo{publisher}{Cambridge University Press},
  \bibinfo{address}{Cambridge, United Kingdom}.
\newblock


\bibitem[Adamo(1995)]%
        {adamo1995beginnings}
\bibfield{author}{\bibinfo{person}{Giuliana Adamo}.}
  \bibinfo{year}{1995}\natexlab{}.
\newblock \showarticletitle{Beginnings and endings in novels}.
\newblock \bibinfo{journal}{\emph{New Readings}}  \bibinfo{volume}{1}
  (\bibinfo{year}{1995}), \bibinfo{pages}{83--104}.
\newblock


\bibitem[Amini et~al\mbox{.}(2015)]%
        {amini2015understanding}
\bibfield{author}{\bibinfo{person}{Fereshteh Amini}, \bibinfo{person}{Nathalie
  Henry~Riche}, \bibinfo{person}{Bongshin Lee}, \bibinfo{person}{Christophe
  Hurter}, {and} \bibinfo{person}{Pourang Irani}.}
  \bibinfo{year}{2015}\natexlab{}.
\newblock \showarticletitle{Understanding data videos: Looking at narrative
  visualization through the cinematography lens}. In
  \bibinfo{booktitle}{\emph{Proceedings of the 33rd Annual ACM Conference on
  Human Factors in Computing Systems}}. \bibinfo{publisher}{ACM CHI},
  \bibinfo{address}{Seoul Republic of Korea}, \bibinfo{pages}{1459--1468}.
\newblock


\bibitem[Amini et~al\mbox{.}(2016)]%
        {amini2016authoring}
\bibfield{author}{\bibinfo{person}{Fereshteh Amini},
  \bibinfo{person}{Nathalie~Henry Riche}, \bibinfo{person}{Bongshin Lee},
  \bibinfo{person}{Andres Monroy-Hernandez}, {and} \bibinfo{person}{Pourang
  Irani}.} \bibinfo{year}{2016}\natexlab{}.
\newblock \showarticletitle{Authoring data-driven videos with dataclips}.
\newblock \bibinfo{journal}{\emph{IEEE transactions on visualization and
  computer graphics}} \bibinfo{volume}{23}, \bibinfo{number}{1}
  (\bibinfo{year}{2016}), \bibinfo{pages}{501--510}.
\newblock


\bibitem[Aristotle(2006)]%
        {aristotle2006poetics}
\bibfield{author}{\bibinfo{person}{Aristotle}.}
  \bibinfo{year}{2006}\natexlab{}.
\newblock \bibinfo{booktitle}{\emph{Poetics}}.
\newblock \bibinfo{publisher}{ReadHowYouWant},
  \bibinfo{howpublished}{\url{http://www.ReadHowYouWant.com}}.
\newblock


\bibitem[Bach et~al\mbox{.}(2018)]%
        {bach2018design}
\bibfield{author}{\bibinfo{person}{Benjamin Bach}, \bibinfo{person}{Zezhong
  Wang}, \bibinfo{person}{Matteo Farinella}, \bibinfo{person}{Dave
  Murray-Rust}, {and} \bibinfo{person}{Nathalie Henry~Riche}.}
  \bibinfo{year}{2018}\natexlab{}.
\newblock \showarticletitle{Design patterns for data comics}. In
  \bibinfo{booktitle}{\emph{Proceedings of the 2018 chi conference on human
  factors in computing systems}}. \bibinfo{publisher}{ACM CHI},
  \bibinfo{address}{Montréal, Canada}, \bibinfo{pages}{1--12}.
\newblock


\bibitem[Borgo et~al\mbox{.}(2012)]%
        {borgo2012empirical}
\bibfield{author}{\bibinfo{person}{Rita Borgo}, \bibinfo{person}{Alfie
  Abdul-Rahman}, \bibinfo{person}{Farhan Mohamed}, \bibinfo{person}{Philip~W
  Grant}, \bibinfo{person}{Irene Reppa}, \bibinfo{person}{Luciano Floridi},
  {and} \bibinfo{person}{Min Chen}.} \bibinfo{year}{2012}\natexlab{}.
\newblock \showarticletitle{An empirical study on using visual embellishments
  in visualization}.
\newblock \bibinfo{journal}{\emph{IEEE Transactions on Visualization and
  Computer Graphics}} \bibinfo{volume}{18}, \bibinfo{number}{12}
  (\bibinfo{year}{2012}), \bibinfo{pages}{2759--2768}.
\newblock


\bibitem[Borkin et~al\mbox{.}(2015)]%
        {borkin2015beyond}
\bibfield{author}{\bibinfo{person}{Michelle~A Borkin}, \bibinfo{person}{Zoya
  Bylinskii}, \bibinfo{person}{Nam~Wook Kim}, \bibinfo{person}{Constance~May
  Bainbridge}, \bibinfo{person}{Chelsea~S Yeh}, \bibinfo{person}{Daniel
  Borkin}, \bibinfo{person}{Hanspeter Pfister}, {and} \bibinfo{person}{Aude
  Oliva}.} \bibinfo{year}{2015}\natexlab{}.
\newblock \showarticletitle{Beyond memorability: Visualization recognition and
  recall}.
\newblock \bibinfo{journal}{\emph{IEEE transactions on visualization and
  computer graphics}} \bibinfo{volume}{22}, \bibinfo{number}{1}
  (\bibinfo{year}{2015}), \bibinfo{pages}{519--528}.
\newblock


\bibitem[Boy et~al\mbox{.}(2017)]%
        {boy2017showing}
\bibfield{author}{\bibinfo{person}{Jeremy Boy}, \bibinfo{person}{Anshul~Vikram
  Pandey}, \bibinfo{person}{John Emerson}, \bibinfo{person}{Margaret
  Satterthwaite}, \bibinfo{person}{Oded Nov}, {and} \bibinfo{person}{Enrico
  Bertini}.} \bibinfo{year}{2017}\natexlab{}.
\newblock \showarticletitle{Showing people behind data: Does anthropomorphizing
  visualizations elicit more empathy for human rights data?}. In
  \bibinfo{booktitle}{\emph{Proceedings of the 2017 CHI conference on human
  factors in computing systems}}. \bibinfo{publisher}{ACM CHI},
  \bibinfo{address}{Colorado, Denver, USA}, \bibinfo{pages}{5462--5474}.
\newblock


\bibitem[Braun and Clarke(2006)]%
        {braun2006using}
\bibfield{author}{\bibinfo{person}{Virginia Braun} {and}
  \bibinfo{person}{Victoria Clarke}.} \bibinfo{year}{2006}\natexlab{}.
\newblock \showarticletitle{Using Thematic Analysis in Psychology}.
\newblock \bibinfo{journal}{\emph{Qualitative Research in Psychology}}
  \bibinfo{volume}{3}, \bibinfo{number}{2} (\bibinfo{year}{2006}),
  \bibinfo{pages}{77--101}.
\newblock


\bibitem[Brehmer et~al\mbox{.}(2016)]%
        {brehmer2016timelines}
\bibfield{author}{\bibinfo{person}{Matthew Brehmer}, \bibinfo{person}{Bongshin
  Lee}, \bibinfo{person}{Benjamin Bach}, \bibinfo{person}{Nathalie~Henry
  Riche}, {and} \bibinfo{person}{Tamara Munzner}.}
  \bibinfo{year}{2016}\natexlab{}.
\newblock \showarticletitle{Timelines Revisited: A Design Space and
  Considerations for Expressive Storytelling}.
\newblock \bibinfo{journal}{\emph{IEEE Transactions on Visualization and
  Computer Graphics}} \bibinfo{volume}{23}, \bibinfo{number}{9}
  (\bibinfo{year}{2016}), \bibinfo{pages}{2151--2164}.
\newblock


\bibitem[Brody(2008)]%
        {brody2008punctuation}
\bibfield{author}{\bibinfo{person}{Jennifer~DeVere Brody}.}
  \bibinfo{year}{2008}\natexlab{}.
\newblock \bibinfo{booktitle}{\emph{Punctuation}}.
\newblock \bibinfo{publisher}{Duke University Press}, \bibinfo{address}{Durham,
  North Carolina, USA}.
\newblock


\bibitem[Cao et~al\mbox{.}(2020)]%
        {cao2020examining}
\bibfield{author}{\bibinfo{person}{Ruochen Cao}, \bibinfo{person}{Subrata Dey},
  \bibinfo{person}{Andrew Cunningham}, \bibinfo{person}{James Walsh},
  \bibinfo{person}{Ross~T Smith}, \bibinfo{person}{Joanne~E Zucco}, {and}
  \bibinfo{person}{Bruce~H Thomas}.} \bibinfo{year}{2020}\natexlab{}.
\newblock \showarticletitle{Examining the use of narrative constructs in data
  videos}.
\newblock \bibinfo{journal}{\emph{Visual Informatics}} \bibinfo{volume}{4},
  \bibinfo{number}{1} (\bibinfo{year}{2020}), \bibinfo{pages}{8--22}.
\newblock


\bibitem[Carroll(2007)]%
        {carroll2007narrative}
\bibfield{author}{\bibinfo{person}{No{\"e}l Carroll}.}
  \bibinfo{year}{2007}\natexlab{}.
\newblock \showarticletitle{Narrative closure}.
\newblock \bibinfo{journal}{\emph{Philosophical studies}}
  \bibinfo{volume}{135}, \bibinfo{number}{1} (\bibinfo{year}{2007}),
  \bibinfo{pages}{1--15}.
\newblock


\bibitem[Charmaz(2006)]%
        {charmaz2006constructing}
\bibfield{author}{\bibinfo{person}{Kathy Charmaz}.}
  \bibinfo{year}{2006}\natexlab{}.
\newblock \bibinfo{booktitle}{\emph{Constructing grounded theory: A practical
  guide through qualitative analysis}}.
\newblock \bibinfo{publisher}{SAGE}, \bibinfo{address}{NY, USA}.
\newblock


\bibitem[{CineFix - IGN Movies and TV}(2013)]%
        {CineFix}
\bibfield{author}{\bibinfo{person}{{CineFix - IGN Movies and TV}}.}
  \bibinfo{year}{2013}\natexlab{}.
\newblock \bibinfo{title}{Best Movie Endings of All Time}.
\newblock
  \bibinfo{howpublished}{\url{https://www.youtube.com/channel/UCVtL1edhT8qqY-j2JIndMzg}}.
\newblock


\bibitem[Clarke and Mitchell(2001)]%
        {clarke2001film}
\bibfield{author}{\bibinfo{person}{Andy Clarke} {and} \bibinfo{person}{Grethe
  Mitchell}.} \bibinfo{year}{2001}\natexlab{}.
\newblock \showarticletitle{Film and the development of interactive narrative}.
  In \bibinfo{booktitle}{\emph{International Conference on Virtual
  Storytelling}}. \bibinfo{publisher}{Springer}, \bibinfo{address}{Heidelberg,
  Berlin, Germany}, \bibinfo{pages}{81--89}.
\newblock


\bibitem[Coelho and Mueller(2020)]%
        {coelho2020infomages}
\bibfield{author}{\bibinfo{person}{Darius Coelho} {and} \bibinfo{person}{Klaus
  Mueller}.} \bibinfo{year}{2020}\natexlab{}.
\newblock \showarticletitle{Infomages: Embedding data into thematic images}. In
  \bibinfo{booktitle}{\emph{Computer Graphics Forum}},
  Vol.~\bibinfo{volume}{39}. \bibinfo{publisher}{Wiley Online Library},
  \bibinfo{address}{Yankee Ferry, USA}, \bibinfo{pages}{593--606}.
\newblock


\bibitem[Cohn(2013)]%
        {cohn2013visual}
\bibfield{author}{\bibinfo{person}{Neil Cohn}.}
  \bibinfo{year}{2013}\natexlab{}.
\newblock \showarticletitle{Visual narrative structure}.
\newblock \bibinfo{journal}{\emph{Cognitive science}} \bibinfo{volume}{37},
  \bibinfo{number}{3} (\bibinfo{year}{2013}), \bibinfo{pages}{413--452}.
\newblock


\bibitem[Cutting(2016)]%
        {cutting2016narrative}
\bibfield{author}{\bibinfo{person}{James~E Cutting}.}
  \bibinfo{year}{2016}\natexlab{}.
\newblock \showarticletitle{Narrative theory and the dynamics of popular
  movies}.
\newblock \bibinfo{journal}{\emph{Psychonomic bulletin \& review}}
  \bibinfo{volume}{23}, \bibinfo{number}{6} (\bibinfo{year}{2016}),
  \bibinfo{pages}{1713--1743}.
\newblock


\bibitem[{Economist.com}(2012)]%
        {economist}
\bibfield{author}{\bibinfo{person}{{Economist.com}}.}
  \bibinfo{year}{2012}\natexlab{}.
\newblock \bibinfo{title}{The Economist}.
\newblock \bibinfo{howpublished}{\url{https://www.economist.com/}}.
\newblock


\bibitem[{Eugene Doyen}(2019)]%
        {filmnarrative}
\bibfield{author}{\bibinfo{person}{{Eugene Doyen}}.}
  \bibinfo{year}{2019}\natexlab{}.
\newblock \bibinfo{title}{Film Narrative}.
\newblock
  \bibinfo{howpublished}{\url{http://www.filmnarrative.com/ellipsis.html}}.
\newblock


\bibitem[{Fandor}(2018)]%
        {art}
\bibfield{author}{\bibinfo{person}{{Fandor}}.} \bibinfo{year}{2018}\natexlab{}.
\newblock \bibinfo{title}{The Art of the Ellipsis}.
\newblock
  \bibinfo{howpublished}{\url{https://www.youtube.com/watch?v=kR6zckPigPA}}.
\newblock


\bibitem[{Filmsite}(2022)]%
        {filmsite}
\bibfield{author}{\bibinfo{person}{{Filmsite}}.}
  \bibinfo{year}{2022}\natexlab{}.
\newblock \bibinfo{title}{Film Terms Glossary}.
\newblock
  \bibinfo{howpublished}{\url{https://www.filmsite.org/filmterms9.html}}.
\newblock


\bibitem[Freytag and MacEwan(1908)]%
        {freytag1908freytag}
\bibfield{author}{\bibinfo{person}{Gustav Freytag} {and}
  \bibinfo{person}{Elias~J MacEwan}.} \bibinfo{year}{1908}\natexlab{}.
\newblock \bibinfo{booktitle}{\emph{Freytag's technique of the drama: an
  exposition of dramatic composition and art}}.
\newblock \bibinfo{publisher}{Scott, Foresman and Company},
  \bibinfo{address}{Brook, Ill., USA}.
\newblock


\bibitem[Gaut(2010)]%
        {gaut2010philosophy}
\bibfield{author}{\bibinfo{person}{Berys Gaut}.}
  \bibinfo{year}{2010}\natexlab{}.
\newblock \bibinfo{booktitle}{\emph{A philosophy of cinematic art}}.
\newblock \bibinfo{publisher}{Cambridge University Press},
  \bibinfo{address}{Cambridge, United Kingdom}.
\newblock


\bibitem[Ge et~al\mbox{.}(2020)]%
        {ge2020canis}
\bibfield{author}{\bibinfo{person}{Tong Ge}, \bibinfo{person}{Yue Zhao},
  \bibinfo{person}{Bongshin Lee}, \bibinfo{person}{Donghao Ren},
  \bibinfo{person}{Baoquan Chen}, {and} \bibinfo{person}{Yunhai Wang}.}
  \bibinfo{year}{2020}\natexlab{}.
\newblock \showarticletitle{Canis: A High-Level Language for Data-Driven Chart
  Animations}. In \bibinfo{booktitle}{\emph{Computer Graphics Forum}},
  Vol.~\bibinfo{volume}{39}. \bibinfo{publisher}{Wiley Online Library},
  \bibinfo{address}{Yankee Ferry, USA}, \bibinfo{pages}{607--617}.
\newblock


\bibitem[Gershon and Page(2001)]%
        {gershon2001storytelling}
\bibfield{author}{\bibinfo{person}{Nahum Gershon} {and} \bibinfo{person}{Ward
  Page}.} \bibinfo{year}{2001}\natexlab{}.
\newblock \showarticletitle{What storytelling can do for information
  visualization}.
\newblock \bibinfo{journal}{\emph{Commun. ACM}} \bibinfo{volume}{44},
  \bibinfo{number}{8} (\bibinfo{year}{2001}), \bibinfo{pages}{31--37}.
\newblock


\bibitem[Giannetti and Leach(1999)]%
        {giannetti1999understanding}
\bibfield{author}{\bibinfo{person}{Louis~D Giannetti} {and}
  \bibinfo{person}{Jim Leach}.} \bibinfo{year}{1999}\natexlab{}.
\newblock \bibinfo{booktitle}{\emph{Understanding movies}}.
  Vol.~\bibinfo{volume}{1}.
\newblock \bibinfo{publisher}{Prentice Hall}, \bibinfo{address}{Hoboken, New
  Jersey, USA}.
\newblock


\bibitem[{Guardian.com}(2012)]%
        {guardian}
\bibfield{author}{\bibinfo{person}{{Guardian.com}}.}
  \bibinfo{year}{2012}\natexlab{}.
\newblock \bibinfo{title}{The Guardian}.
\newblock
  \bibinfo{howpublished}{\url{https://www.theguardian.com/international}}.
\newblock


\bibitem[{Guider}(2022)]%
        {guide}
\bibfield{author}{\bibinfo{person}{{Guider}}.} \bibinfo{year}{2022}\natexlab{}.
\newblock \bibinfo{title}{The Punctuation Guide}.
\newblock \bibinfo{howpublished}{\url{https://www.thepunctuationguide.com/}}.
\newblock


\bibitem[Heyer et~al\mbox{.}(2020)]%
        {heyer2020pushing}
\bibfield{author}{\bibinfo{person}{Jeremy Heyer}, \bibinfo{person}{Nirmal~Kumar
  Raveendranath}, {and} \bibinfo{person}{Khairi Reda}.}
  \bibinfo{year}{2020}\natexlab{}.
\newblock \showarticletitle{Pushing the (visual) narrative: the effects of
  prior knowledge elicitation in provocative topics}. In
  \bibinfo{booktitle}{\emph{Proceedings of the 2020 CHI Conference on Human
  Factors in Computing Systems}}. \bibinfo{publisher}{ACM CHI},
  \bibinfo{address}{Honolulu, Hawaii, USA}, \bibinfo{pages}{1--14}.
\newblock


\bibitem[{Hollywood}(2019)]%
        {hollywoodlexicon}
\bibfield{author}{\bibinfo{person}{{Hollywood}}.}
  \bibinfo{year}{2019}\natexlab{}.
\newblock \bibinfo{title}{Hollywood Lexicon}.
\newblock
  \bibinfo{howpublished}{\url{http://www.hollywoodlexicon.com/ellipsis.html}}.
\newblock


\bibitem[Hullman and Diakopoulos(2011)]%
        {hullman2011visualization}
\bibfield{author}{\bibinfo{person}{Jessica Hullman} {and} \bibinfo{person}{Nick
  Diakopoulos}.} \bibinfo{year}{2011}\natexlab{}.
\newblock \showarticletitle{Visualization rhetoric: Framing effects in
  narrative visualization}.
\newblock \bibinfo{journal}{\emph{IEEE transactions on visualization and
  computer graphics}} \bibinfo{volume}{17}, \bibinfo{number}{12}
  (\bibinfo{year}{2011}), \bibinfo{pages}{2231--2240}.
\newblock


\bibitem[{James Plath}(2015)]%
        {James}
\bibfield{author}{\bibinfo{person}{{James Plath}}.}
  \bibinfo{year}{2015}\natexlab{}.
\newblock \bibinfo{title}{James Plath Quotes}.
\newblock
  \bibinfo{howpublished}{\url{https://www.goodreads.com/author/quotes/7453.James_Plath}}.
\newblock


\bibitem[{Kathryn Schulz}(2014)]%
        {schulz}
\bibfield{author}{\bibinfo{person}{{Kathryn Schulz}}.}
  \bibinfo{year}{2014}\natexlab{}.
\newblock \bibinfo{title}{Schulz: The 5 Best Punctuation Marks in Literature}.
\newblock
  \bibinfo{howpublished}{\url{https://www.vulture.com/2014/01/best-punctuation-marks-literature-nabokov-eliot-dickens-levi.html}}.
\newblock


\bibitem[Kim et~al\mbox{.}(2022)]%
        {kim2022mobile}
\bibfield{author}{\bibinfo{person}{Jeongyeon Kim}, \bibinfo{person}{Yubin
  Choi}, \bibinfo{person}{Meng Xia}, {and} \bibinfo{person}{Juho Kim}.}
  \bibinfo{year}{2022}\natexlab{}.
\newblock \showarticletitle{Mobile-Friendly Content Design for MOOCs:
  Challenges, Requirements, and Design Opportunities}. In
  \bibinfo{booktitle}{\emph{CHI Conference on Human Factors in Computing
  Systems}}. \bibinfo{publisher}{ACM CHI}, \bibinfo{address}{New Orleans, LA,
  USA}, \bibinfo{pages}{1--16}.
\newblock


\bibitem[Kim et~al\mbox{.}(2019)]%
        {kim2019datatoon}
\bibfield{author}{\bibinfo{person}{Nam~Wook Kim}, \bibinfo{person}{Nathalie
  Henry~Riche}, \bibinfo{person}{Benjamin Bach}, \bibinfo{person}{Guanpeng Xu},
  \bibinfo{person}{Matthew Brehmer}, \bibinfo{person}{Ken Hinckley},
  \bibinfo{person}{Michel Pahud}, \bibinfo{person}{Haijun Xia},
  \bibinfo{person}{Michael~J McGuffin}, {and} \bibinfo{person}{Hanspeter
  Pfister}.} \bibinfo{year}{2019}\natexlab{}.
\newblock \showarticletitle{Datatoon: Drawing dynamic network comics with pen+
  touch interaction}. In \bibinfo{booktitle}{\emph{Proceedings of the 2019 CHI
  Conference on Human Factors in Computing Systems}}. \bibinfo{publisher}{ACM
  CHI}, \bibinfo{address}{Glasgow Scotland UK}, \bibinfo{pages}{1--12}.
\newblock


\bibitem[Kim and Heer(2020)]%
        {kim2020gemini}
\bibfield{author}{\bibinfo{person}{Younghoon Kim} {and}
  \bibinfo{person}{Jeffrey Heer}.} \bibinfo{year}{2020}\natexlab{}.
\newblock \showarticletitle{Gemini: A grammar and recommender system for
  animated transitions in statistical graphics}.
\newblock \bibinfo{journal}{\emph{IEEE Transactions on Visualization and
  Computer Graphics}} \bibinfo{volume}{27}, \bibinfo{number}{2}
  (\bibinfo{year}{2020}), \bibinfo{pages}{485--494}.
\newblock


\bibitem[King and Hourani(2007)]%
        {king2007don}
\bibfield{author}{\bibinfo{person}{Cynthia~M King} {and} \bibinfo{person}{Nora
  Hourani}.} \bibinfo{year}{2007}\natexlab{}.
\newblock \showarticletitle{Don't tease me: Effects of ending type on horror
  film enjoyment}.
\newblock \bibinfo{journal}{\emph{Media Psychology}} \bibinfo{volume}{9},
  \bibinfo{number}{3} (\bibinfo{year}{2007}), \bibinfo{pages}{473--492}.
\newblock


\bibitem[Klauk et~al\mbox{.}(2016)]%
        {klauk2016empirical}
\bibfield{author}{\bibinfo{person}{Tobias Klauk}, \bibinfo{person}{Tilmann
  K{\"o}ppe}, {and} \bibinfo{person}{Thomas Weskott}.}
  \bibinfo{year}{2016}\natexlab{}.
\newblock \showarticletitle{Empirical correlates of narrative closure}.
\newblock \bibinfo{journal}{\emph{Diegesis}} \bibinfo{volume}{5},
  \bibinfo{number}{1} (\bibinfo{year}{2016}), \bibinfo{pages}{1}.
\newblock


\bibitem[Krueger(2014)]%
        {krueger2014focus}
\bibfield{author}{\bibinfo{person}{Richard~A Krueger}.}
  \bibinfo{year}{2014}\natexlab{}.
\newblock \bibinfo{booktitle}{\emph{Focus groups: A practical guide for applied
  research}}.
\newblock \bibinfo{publisher}{SAGE}, \bibinfo{address}{NY, USA}.
\newblock


\bibitem[Kunkle(2016)]%
        {kunkle2016cinematic}
\bibfield{author}{\bibinfo{person}{Sheila Kunkle}.}
  \bibinfo{year}{2016}\natexlab{}.
\newblock \bibinfo{booktitle}{\emph{Cinematic Cuts: Theorizing Film Endings}}.
\newblock \bibinfo{publisher}{SUNY Press}, \bibinfo{address}{Albany, New York,
  USA}.
\newblock


\bibitem[Lan et~al\mbox{.}(2021)]%
        {lan2021kineticharts}
\bibfield{author}{\bibinfo{person}{Xingyu Lan}, \bibinfo{person}{Yang Shi},
  \bibinfo{person}{Yanqiu Wu}, \bibinfo{person}{Xiaohan Jiao}, {and}
  \bibinfo{person}{Nan Cao}.} \bibinfo{year}{2021}\natexlab{}.
\newblock \showarticletitle{Kineticharts: Augmenting Affective Expressiveness
  of Charts in Data Stories with Animation Design}.
\newblock \bibinfo{journal}{\emph{IEEE Transactions on Visualization and
  Computer Graphics}} \bibinfo{volume}{28}, \bibinfo{number}{1}
  (\bibinfo{year}{2021}), \bibinfo{pages}{933--943}.
\newblock


\bibitem[Lan et~al\mbox{.}(2022)]%
        {lan2022negative}
\bibfield{author}{\bibinfo{person}{Xingyu Lan}, \bibinfo{person}{Yanqiu Wu},
  \bibinfo{person}{Yang Shi}, \bibinfo{person}{Qing Chen}, {and}
  \bibinfo{person}{Nan Cao}.} \bibinfo{year}{2022}\natexlab{}.
\newblock \showarticletitle{Negative Emotions, Positive Outcomes? Exploring the
  Communication of Negativity in Serious Data Stories}. In
  \bibinfo{booktitle}{\emph{CHI Conference on Human Factors in Computing
  Systems}}. \bibinfo{publisher}{ACM CHI}, \bibinfo{address}{New Orleans, LA},
  \bibinfo{pages}{1--14}.
\newblock


\bibitem[Lee et~al\mbox{.}(2015)]%
        {lee2015more}
\bibfield{author}{\bibinfo{person}{Bongshin Lee},
  \bibinfo{person}{Nathalie~Henry Riche}, \bibinfo{person}{Petra Isenberg},
  {and} \bibinfo{person}{Sheelagh Carpendale}.}
  \bibinfo{year}{2015}\natexlab{}.
\newblock \showarticletitle{More than telling a story: Transforming data into
  visually shared stories}.
\newblock \bibinfo{journal}{\emph{IEEE computer graphics and applications}}
  \bibinfo{volume}{35}, \bibinfo{number}{5} (\bibinfo{year}{2015}),
  \bibinfo{pages}{84--90}.
\newblock


\bibitem[Lee-Robbins and Adar(2022)]%
        {lee2022affective}
\bibfield{author}{\bibinfo{person}{Elsie Lee-Robbins} {and}
  \bibinfo{person}{Eytan Adar}.} \bibinfo{year}{2022}\natexlab{}.
\newblock \showarticletitle{Affective Learning Objectives for Communicative
  Visualizations}.
\newblock \bibinfo{journal}{\emph{IEEE Transactions on Visualization and
  Computer Graphics}} \bibinfo{volume}{1}, \bibinfo{number}{1}
  (\bibinfo{year}{2022}), \bibinfo{pages}{1}.
\newblock


\bibitem[Lin and Van~Brummelen(2021)]%
        {lin2021engaging}
\bibfield{author}{\bibinfo{person}{Phoebe Lin} {and} \bibinfo{person}{Jessica
  Van~Brummelen}.} \bibinfo{year}{2021}\natexlab{}.
\newblock \showarticletitle{Engaging teachers to Co-design integrated AI
  curriculum for K-12 classrooms}. In \bibinfo{booktitle}{\emph{Proceedings of
  the 2021 CHI Conference on Human Factors in Computing Systems}}.
  \bibinfo{publisher}{ACM CHI}, \bibinfo{address}{Yokohama, Japan},
  \bibinfo{pages}{1--12}.
\newblock


\bibitem[{Max Tohline}(2016)]%
        {editing}
\bibfield{author}{\bibinfo{person}{{Max Tohline}}.}
  \bibinfo{year}{2016}\natexlab{}.
\newblock \bibinfo{title}{Editing as Punctuation in Film}.
\newblock \bibinfo{howpublished}{\url{https://vimeo.com/138829554}}.
\newblock


\bibitem[Morais et~al\mbox{.}(2020)]%
        {morais2020showing}
\bibfield{author}{\bibinfo{person}{Luiz Morais}, \bibinfo{person}{Yvonne
  Jansen}, \bibinfo{person}{Nazareno Andrade}, {and} \bibinfo{person}{Pierre
  Dragicevic}.} \bibinfo{year}{2020}\natexlab{}.
\newblock \showarticletitle{Showing data about people: A design space of
  anthropographics}.
\newblock \bibinfo{journal}{\emph{IEEE Transactions on Visualization and
  Computer Graphics}}  \bibinfo{volume}{1} (\bibinfo{year}{2020}),
  \bibinfo{pages}{1}.
\newblock


\bibitem[Neupert(1995)]%
        {neupert1995end}
\bibfield{author}{\bibinfo{person}{Richard~John Neupert}.}
  \bibinfo{year}{1995}\natexlab{}.
\newblock \bibinfo{booktitle}{\emph{The end: narration and closure in the
  cinema}}.
\newblock \bibinfo{publisher}{Wayne State University Press},
  \bibinfo{address}{Michigan, Detroit, USA}.
\newblock


\bibitem[Nunberg(1990)]%
        {nunberg1990linguistics}
\bibfield{author}{\bibinfo{person}{Geoffrey Nunberg}.}
  \bibinfo{year}{1990}\natexlab{}.
\newblock \bibinfo{booktitle}{\emph{The linguistics of punctuation}}.
\newblock \bibinfo{publisher}{Center for the Study of Language (CSLI)},
  \bibinfo{address}{USA}.
\newblock


\bibitem[Preis(1990)]%
        {preis1990not}
\bibfield{author}{\bibinfo{person}{Eran Preis}.}
  \bibinfo{year}{1990}\natexlab{}.
\newblock \showarticletitle{Not such a happy ending: The ideology of the open
  ending}.
\newblock \bibinfo{journal}{\emph{Journal of Film and Video}}
  \bibinfo{volume}{1}, \bibinfo{number}{1} (\bibinfo{year}{1990}),
  \bibinfo{pages}{18--23}.
\newblock


\bibitem[Provost and Fawcett(2013)]%
        {provost2013data}
\bibfield{author}{\bibinfo{person}{Foster Provost} {and} \bibinfo{person}{Tom
  Fawcett}.} \bibinfo{year}{2013}\natexlab{}.
\newblock \bibinfo{booktitle}{\emph{Data Science for Business: What you need to
  know about data mining and data-analytic thinking}}.
\newblock \bibinfo{publisher}{O'Reilly Media}, \bibinfo{address}{USA}.
\newblock


\bibitem[Ragan et~al\mbox{.}(2015)]%
        {ragan2015characterizing}
\bibfield{author}{\bibinfo{person}{Eric~D Ragan}, \bibinfo{person}{Alex
  Endert}, \bibinfo{person}{Jibonananda Sanyal}, {and} \bibinfo{person}{Jian
  Chen}.} \bibinfo{year}{2015}\natexlab{}.
\newblock \showarticletitle{Characterizing provenance in visualization and data
  analysis: an organizational framework of provenance types and purposes}.
\newblock \bibinfo{journal}{\emph{IEEE transactions on visualization and
  computer graphics}} \bibinfo{volume}{22}, \bibinfo{number}{1}
  (\bibinfo{year}{2015}), \bibinfo{pages}{31--40}.
\newblock


\bibitem[Ryan(2015)]%
        {ryan2015heretical}
\bibfield{author}{\bibinfo{person}{Kathleen~M Ryan}.}
  \bibinfo{year}{2015}\natexlab{}.
\newblock \bibinfo{title}{The Heretical Archive: Digital Memory at the End of
  Film by Domietta Torlasco}.
\newblock
\newblock


\bibitem[Segel and Heer(2010)]%
        {segel2010narrative}
\bibfield{author}{\bibinfo{person}{Edward Segel} {and} \bibinfo{person}{Jeffrey
  Heer}.} \bibinfo{year}{2010}\natexlab{}.
\newblock \showarticletitle{Narrative visualization: Telling stories with
  data}.
\newblock \bibinfo{journal}{\emph{IEEE Transactions on Visualization and
  Computer Graphics}} \bibinfo{volume}{16}, \bibinfo{number}{6}
  (\bibinfo{year}{2010}), \bibinfo{pages}{1139--1148}.
\newblock


\bibitem[Shi et~al\mbox{.}(2021)]%
        {shi2021communicating}
\bibfield{author}{\bibinfo{person}{Yang Shi}, \bibinfo{person}{Xingyu Lan},
  \bibinfo{person}{Jingwen Li}, \bibinfo{person}{Zhaorui Li}, {and}
  \bibinfo{person}{Nan Cao}.} \bibinfo{year}{2021}\natexlab{}.
\newblock \showarticletitle{Communicating with motion: A design space for
  animated visual narratives in data videos}. In
  \bibinfo{booktitle}{\emph{Proceedings of the 2021 CHI Conference on Human
  Factors in Computing Systems}}. \bibinfo{publisher}{ACM CHI},
  \bibinfo{address}{Yokohama, Japan}, \bibinfo{pages}{1--13}.
\newblock


\bibitem[Shin et~al\mbox{.}(2022)]%
        {shin2022roslingifier}
\bibfield{author}{\bibinfo{person}{Minjeong Shin}, \bibinfo{person}{Joohee
  Kim}, \bibinfo{person}{Yunha Han}, \bibinfo{person}{Lexing Xie},
  \bibinfo{person}{Mitchell Whitelaw}, \bibinfo{person}{Bum~Chul Kwon},
  \bibinfo{person}{Sungahn Ko}, {and} \bibinfo{person}{Niklas Elmqvist}.}
  \bibinfo{year}{2022}\natexlab{}.
\newblock \showarticletitle{Roslingifier: Semi-Automated Storytelling for
  Animated Scatterplots}.
\newblock \bibinfo{journal}{\emph{IEEE Transactions on Visualization and
  Computer Graphics}} \bibinfo{volume}{1}, \bibinfo{number}{1}
  (\bibinfo{year}{2022}), \bibinfo{pages}{1}.
\newblock


\bibitem[Shu et~al\mbox{.}(2020)]%
        {shu2020makes}
\bibfield{author}{\bibinfo{person}{Xinhuan Shu}, \bibinfo{person}{Aoyu Wu},
  \bibinfo{person}{Junxiu Tang}, \bibinfo{person}{Benjamin Bach},
  \bibinfo{person}{Yingcai Wu}, {and} \bibinfo{person}{Huamin Qu}.}
  \bibinfo{year}{2020}\natexlab{}.
\newblock \showarticletitle{What makes a Data-GIF understandable?}
\newblock \bibinfo{journal}{\emph{IEEE Transactions on Visualization and
  Computer Graphics}} \bibinfo{volume}{27}, \bibinfo{number}{2}
  (\bibinfo{year}{2020}), \bibinfo{pages}{1492--1502}.
\newblock


\bibitem[Smith(1968)]%
        {smith1968poetic}
\bibfield{author}{\bibinfo{person}{Barbara~Herrnstein Smith}.}
  \bibinfo{year}{1968}\natexlab{}.
\newblock \bibinfo{booktitle}{\emph{Poetic closure: A study of how poems end}}.
  Vol.~\bibinfo{volume}{381}.
\newblock \bibinfo{publisher}{University of Chicago Press},
  \bibinfo{address}{Chicago, USA}.
\newblock


\bibitem[Stone(2002)]%
        {stone2002hope}
\bibfield{author}{\bibinfo{person}{Bryan Stone}.}
  \bibinfo{year}{2002}\natexlab{}.
\newblock \showarticletitle{Hope and Happy Endings}.
\newblock \bibinfo{journal}{\emph{Review \& Expositor}} \bibinfo{volume}{99},
  \bibinfo{number}{1} (\bibinfo{year}{2002}), \bibinfo{pages}{37--50}.
\newblock


\bibitem[Sun et~al\mbox{.}(2022)]%
        {sun2022erato}
\bibfield{author}{\bibinfo{person}{Mengdi Sun}, \bibinfo{person}{Ligan Cai},
  \bibinfo{person}{Weiwei Cui}, \bibinfo{person}{Yanqiu Wu},
  \bibinfo{person}{Yang Shi}, {and} \bibinfo{person}{Nan Cao}.}
  \bibinfo{year}{2022}\natexlab{}.
\newblock \showarticletitle{Erato: Cooperative Data Story Editing via Fact
  Interpolation}.
\newblock \bibinfo{journal}{\emph{IEEE Transactions on Visualization and
  Computer Graphics}} \bibinfo{volume}{1}, \bibinfo{number}{1}
  (\bibinfo{year}{2022}), \bibinfo{pages}{1}.
\newblock


\bibitem[Tang et~al\mbox{.}(2020)]%
        {tang2020narrative}
\bibfield{author}{\bibinfo{person}{Junxiu Tang}, \bibinfo{person}{Lingyun Yu},
  \bibinfo{person}{Tan Tang}, \bibinfo{person}{Xinhuan Shu},
  \bibinfo{person}{Lu Ying}, \bibinfo{person}{Yuhua Zhou},
  \bibinfo{person}{Peiran Ren}, {and} \bibinfo{person}{Yingcai Wu}.}
  \bibinfo{year}{2020}\natexlab{}.
\newblock \showarticletitle{Narrative transitions in data videos}. In
  \bibinfo{booktitle}{\emph{2020 IEEE Visualization Conference (VIS)}}.
  \bibinfo{publisher}{IEEE}, \bibinfo{address}{Salt Lake City, UT, USA},
  \bibinfo{pages}{151--155}.
\newblock


\bibitem[{Tecent Video}(2021)]%
        {tecent}
\bibfield{author}{\bibinfo{person}{{Tecent Video}}.}
  \bibinfo{year}{2021}\natexlab{}.
\newblock \bibinfo{title}{Tecent Video}.
\newblock \bibinfo{howpublished}{https://v.qq.com/}.
\newblock
\newblock
\shownote{accessed: Sep. 2020}.


\bibitem[{The New York Times}(2021)]%
        {nyt}
\bibfield{author}{\bibinfo{person}{{The New York Times}}.}
  \bibinfo{year}{2021}\natexlab{}.
\newblock \bibinfo{title}{The New York Times}.
\newblock \bibinfo{howpublished}{https://www.nytimes.com/}.
\newblock
\newblock
\shownote{accessed: Sep. 2022}.


\bibitem[{UoPeople}(2022)]%
        {14punctuation}
\bibfield{author}{\bibinfo{person}{{UoPeople}}.}
  \bibinfo{year}{2022}\natexlab{}.
\newblock \bibinfo{title}{The 14 Punctuation Marks with Examples}.
\newblock
  \bibinfo{howpublished}{\url{https://www.uopeople.edu/blog/punctuation-marks/}}.
\newblock


\bibitem[{Vox.com}(2014)]%
        {Vox}
\bibfield{author}{\bibinfo{person}{{Vox.com}}.}
  \bibinfo{year}{2014}\natexlab{}.
\newblock \bibinfo{title}{Vox}.
\newblock
  \bibinfo{howpublished}{\url{https://www.youtube.com/channel/UCLXo7UDZvByw2ixzpQCufnA}}.
\newblock


\bibitem[Wang et~al\mbox{.}(2021)]%
        {wang2021interactive}
\bibfield{author}{\bibinfo{person}{Zezhong Wang}, \bibinfo{person}{Hugo Romat},
  \bibinfo{person}{Fanny Chevalier}, \bibinfo{person}{Nathalie~Henry Riche},
  \bibinfo{person}{Dave Murray-Rust}, {and} \bibinfo{person}{Benjamin Bach}.}
  \bibinfo{year}{2021}\natexlab{}.
\newblock \showarticletitle{Interactive Data Comics}.
\newblock \bibinfo{journal}{\emph{IEEE Transactions on Visualization and
  Computer Graphics}} \bibinfo{volume}{28}, \bibinfo{number}{1}
  (\bibinfo{year}{2021}), \bibinfo{pages}{944--954}.
\newblock


\bibitem[Xiong et~al\mbox{.}(2019)]%
        {xiong2019curse}
\bibfield{author}{\bibinfo{person}{Cindy Xiong}, \bibinfo{person}{Lisanne
  Van~Weelden}, {and} \bibinfo{person}{Steven Franconeri}.}
  \bibinfo{year}{2019}\natexlab{}.
\newblock \showarticletitle{The curse of knowledge in visual data
  communication}.
\newblock \bibinfo{journal}{\emph{IEEE transactions on visualization and
  computer graphics}} \bibinfo{volume}{26}, \bibinfo{number}{10}
  (\bibinfo{year}{2019}), \bibinfo{pages}{3051--3062}.
\newblock


\bibitem[Xu et~al\mbox{.}(2022)]%
        {xu2022fromwow}
\bibfield{author}{\bibinfo{person}{Xian Xu}, \bibinfo{person}{Leni Yang},
  \bibinfo{person}{David Yip}, \bibinfo{person}{Mingming Fan},
  \bibinfo{person}{Zheng Wei}, {and} \bibinfo{person}{Huamin Qu}.}
  \bibinfo{year}{2022}\natexlab{}.
\newblock \showarticletitle{From ‘Wow’to ‘Why’: Guidelines for Creating
  the Opening of a Data Video with Cinematic Styles}. In
  \bibinfo{booktitle}{\emph{Proceedings of the 2022 chi conference on human
  factors in computing systems}}. \bibinfo{publisher}{ACM CHI},
  \bibinfo{address}{New Orleans, LA}, \bibinfo{pages}{1--20}.
\newblock


\bibitem[Yang et~al\mbox{.}(2021)]%
        {pyramid2021}
\bibfield{author}{\bibinfo{person}{Leni Yang}, \bibinfo{person}{Xian Xu},
  \bibinfo{person}{Xingyu Lan}, \bibinfo{person}{Ziyan Liu},
  \bibinfo{person}{Shunan Guo}, \bibinfo{person}{Yang Shi},
  \bibinfo{person}{Huamin Qu}, {and} \bibinfo{person}{Nan Cao}.}
  \bibinfo{year}{2021}\natexlab{}.
\newblock \showarticletitle{A Design Space for Applying the Freytag's Pyramid
  Structure to Data Stories}.
\newblock \bibinfo{journal}{\emph{IEEE Transactions on Visualization and
  Computer Graphics}}  \bibinfo{volume}{28} (\bibinfo{year}{2021}),
  \bibinfo{pages}{922--932}.
\newblock
Issue 1.
\urldef\tempurl%
\url{https://doi.org/10.1109/TVCG.2021.3114774}
\showDOI{\tempurl}


\bibitem[Yip(2020a)]%
        {yip2020invisible}
\bibfield{author}{\bibinfo{person}{David Kei-man Yip}.}
  \bibinfo{year}{2020}\natexlab{a}.
\newblock \showarticletitle{The Invisible Art of Storytelling and Media
  Production}. In \bibinfo{booktitle}{\emph{International Conference on Applied
  Human Factors and Ergonomics}}. \bibinfo{publisher}{Springer},
  \bibinfo{address}{Heidelberg, London}, \bibinfo{pages}{262--266}.
\newblock


\bibitem[Yip(2020b)]%
        {yip2020visual}
\bibfield{author}{\bibinfo{person}{David Kei-man Yip}.}
  \bibinfo{year}{2020}\natexlab{b}.
\newblock \showarticletitle{Visual Elements and Design Principles in Media
  Production}. In \bibinfo{booktitle}{\emph{International Conference on Applied
  Human Factors and Ergonomics}}. \bibinfo{publisher}{Springer},
  \bibinfo{address}{Heidelberg, London}, \bibinfo{pages}{283--288}.
\newblock


\bibitem[{YouTube}(2021)]%
        {youtube}
\bibfield{author}{\bibinfo{person}{{YouTube}}.}
  \bibinfo{year}{2021}\natexlab{}.
\newblock \bibinfo{title}{YouTube}.
\newblock \bibinfo{howpublished}{\url{https://www.youtube.com/}}.
\newblock


\bibitem[Zhao et~al\mbox{.}(2019)]%
        {zhao2019understanding}
\bibfield{author}{\bibinfo{person}{Zhenpeng Zhao}, \bibinfo{person}{Rachael
  Marr}, \bibinfo{person}{Jason Shaffer}, {and} \bibinfo{person}{Niklas
  Elmqvist}.} \bibinfo{year}{2019}\natexlab{}.
\newblock \showarticletitle{Understanding partitioning and sequence in
  data-driven storytelling}. In \bibinfo{booktitle}{\emph{International
  Conference on Information}}. \bibinfo{publisher}{Springer},
  \bibinfo{address}{Heidelberg, Berlin, Germany}, \bibinfo{pages}{327--338}.
\newblock

%%%%%%%%%%%%%%%% SECOND BIBLIOGRAPHY


\section*{\leftskip-\leftmargin CASE EXAMPLE LIST}

\bibitem[channel: Andrew~Barbin(1942)]%
        {casablanca}
\bibfield{author}{\bibinfo{person}{YouTube channel: Andrew~Barbin}.}
  \bibinfo{year}{1942}\natexlab{}.
\newblock \bibinfo{title}{Casablanca, directed by Michael Curtiz}.
\newblock
  \bibinfo{howpublished}{\url{https://www.youtube.com/watch?v=G62tkd2t7qk}}.
\newblock


\bibitem[channel: Andys~Chest(1967)]%
        {graduate}
\bibfield{author}{\bibinfo{person}{YouTube channel: Andys~Chest}.}
  \bibinfo{year}{1967}\natexlab{}.
\newblock \bibinfo{title}{The Graduate, directed by Mike Nichols}.
\newblock
  \bibinfo{howpublished}{\url{https://www.youtube.com/watch?v=2TP4MuVnS2A}}.
\newblock


\bibitem[channel: asimranibrahimi(1941)]%
        {citizen}
\bibfield{author}{\bibinfo{person}{YouTube channel: asimranibrahimi}.}
  \bibinfo{year}{1941}\natexlab{}.
\newblock \bibinfo{title}{Citizen Kane, directed by Orson Welles}.
\newblock
  \bibinfo{howpublished}{\url{https://www.youtube.com/watch?v=eP0O1BKu3zk}}.
\newblock


\bibitem[channel: azzyclark(2014)]%
        {machina}
\bibfield{author}{\bibinfo{person}{YouTube channel: azzyclark}.}
  \bibinfo{year}{2014}\natexlab{}.
\newblock \bibinfo{title}{Ex Machina, directed by Alexander Medawar}.
\newblock
  \bibinfo{howpublished}{\url{https://www.youtube.com/watch?v=4w7nNC_twUM}}.
\newblock


\bibitem[channel: Bahadır~Mahmut(1959)]%
        {400blows}
\bibfield{author}{\bibinfo{person}{YouTube channel: Bahadır~Mahmut}.}
  \bibinfo{year}{1959}\natexlab{}.
\newblock \bibinfo{title}{The 400 Blows, directed by François Roland
  Truffaut}.
\newblock
  \bibinfo{howpublished}{\url{https://www.youtube.com/watch?v=a4jGNoag_1g}}.
\newblock


\bibitem[channel: Brandon~Longwell(1995)]%
        {babe}
\bibfield{author}{\bibinfo{person}{YouTube channel: Brandon~Longwell}.}
  \bibinfo{year}{1995}\natexlab{}.
\newblock \bibinfo{title}{Babe, directed by Chris Noonan}.
\newblock
  \bibinfo{howpublished}{\url{https://www.youtube.com/watch?v=Lk7D-_t6cmk}}.
\newblock


\bibitem[channel: Brian~Raize(1936)]%
        {modern}
\bibfield{author}{\bibinfo{person}{YouTube channel: Brian~Raize}.}
  \bibinfo{year}{1936}\natexlab{}.
\newblock \bibinfo{title}{Modern Times, directed by Charles Spencer Chaplin}.
\newblock
  \bibinfo{howpublished}{\url{https://www.youtube.com/watch?v=W5tTGfCTf5w}}.
\newblock


\bibitem[channel: Cine~Paradise(2000)]%
        {dancer}
\bibfield{author}{\bibinfo{person}{YouTube channel: Cine~Paradise}.}
  \bibinfo{year}{2000}\natexlab{}.
\newblock \bibinfo{title}{Dancer in the Dark, directed by Lars von Trier}.
\newblock
  \bibinfo{howpublished}{\url{https://www.youtube.com/watch?v=kKL_-dJxqys}}.
\newblock


\bibitem[channel: Conrad~Dimech(1993)]%
        {schindler}
\bibfield{author}{\bibinfo{person}{YouTube channel: Conrad~Dimech}.}
  \bibinfo{year}{1993}\natexlab{}.
\newblock \bibinfo{title}{Schindler's List, directed by Steven Allan
  Spielberg}.
\newblock
  \bibinfo{howpublished}{\url{https://www.youtube.com/watch?v=_whxQua7Lbc}}.
\newblock


\bibitem[channel: Gameswecan(2010)]%
        {inception}
\bibfield{author}{\bibinfo{person}{YouTube channel: Gameswecan}.}
  \bibinfo{year}{2010}\natexlab{}.
\newblock \bibinfo{title}{Inception, directed by Christopher Edward Nolan}.
\newblock
  \bibinfo{howpublished}{\url{https://www.youtube.com/watch?v=pQd1-4tqymo}}.
\newblock


\bibitem[channel: Iochsenna(1968)]%
        {2001}
\bibfield{author}{\bibinfo{person}{YouTube channel: Iochsenna}.}
  \bibinfo{year}{1968}\natexlab{}.
\newblock \bibinfo{title}{2001: A Space Odyssey, directed by Stanley Kubrick}.
\newblock
  \bibinfo{howpublished}{\url{https://www.youtube.com/watch?v=AXS8P0HksQo}}.
\newblock


\bibitem[channel: James~Oxford(2006)]%
        {davinci}
\bibfield{author}{\bibinfo{person}{YouTube channel: James~Oxford}.}
  \bibinfo{year}{2006}\natexlab{}.
\newblock \bibinfo{title}{The Da Vinci Code, directed by Ronald William "Ron"
  Howard}.
\newblock
  \bibinfo{howpublished}{\url{https://www.youtube.com/watch?v=YsFGe6Afx5M}}.
\newblock


\bibitem[channel: Javier~Vargas(1948)]%
        {bicycle}
\bibfield{author}{\bibinfo{person}{YouTube channel: Javier~Vargas}.}
  \bibinfo{year}{1948}\natexlab{}.
\newblock \bibinfo{title}{Bicycle Thief, directed by Vittorio De Sica}.
\newblock
  \bibinfo{howpublished}{\url{https://www.youtube.com/watch?v=vzZXkk3MsLo}}.
\newblock


\bibitem[channel: MorningBlack(1994)]%
        {shawshank}
\bibfield{author}{\bibinfo{person}{YouTube channel: MorningBlack}.}
  \bibinfo{year}{1994}\natexlab{}.
\newblock \bibinfo{title}{Shawshank Redemption, directed by Frank Darabont}.
\newblock
  \bibinfo{howpublished}{\url{https://www.youtube.com/watch?v=Z7MMTmVZcVs}}.
\newblock


\bibitem[channel: Movieclips(1960)]%
        {pyscho}
\bibfield{author}{\bibinfo{person}{YouTube channel: Movieclips}.}
  \bibinfo{year}{1960}\natexlab{}.
\newblock \bibinfo{title}{Pyscho, directed by Alfred Hitchcock}.
\newblock
  \bibinfo{howpublished}{\url{https://www.youtube.com/watch?v=dYDxxHrlmUg}}.
\newblock


\bibitem[channel: Movieclips(1997)]%
        {life}
\bibfield{author}{\bibinfo{person}{YouTube channel: Movieclips}.}
  \bibinfo{year}{1997}\natexlab{}.
\newblock \bibinfo{title}{Life Is Beautiful, directed by Roberto Remigio
  Benigni}.
\newblock
  \bibinfo{howpublished}{\url{https://www.youtube.com/watch?v=-13ScnosXAk}}.
\newblock


\bibitem[channel: Neil~Halloran(2016)]%
        {neil:fallen}
\bibfield{author}{\bibinfo{person}{YouTube channel: Neil~Halloran}.}
  \bibinfo{year}{2016}\natexlab{}.
\newblock \bibinfo{title}{The Fallen of World War II, directed by Neil
  Halloran}.
\newblock
  \bibinfo{howpublished}{\url{https://www.youtube.com/watch?v=DwKPFT-RioU}}.
\newblock


\bibitem[channel: Neil~Halloran(2020)]%
        {neil:simulation}
\bibfield{author}{\bibinfo{person}{YouTube channel: Neil~Halloran}.}
  \bibinfo{year}{2020}\natexlab{}.
\newblock \bibinfo{title}{Simulation of a Nuclear Blast in a Major City,
  directed by Neil Halloran}.
\newblock
  \bibinfo{howpublished}{\url{https://www.youtube.com/watch?v=Z3RzNEzJyzo&t=62s}}.
\newblock


\bibitem[channel: NPR(2011)]%
        {7billion}
\bibfield{author}{\bibinfo{person}{YouTube channel: NPR}.}
  \bibinfo{year}{2011}\natexlab{}.
\newblock \bibinfo{title}{7 Billion: How Did We Get So Big So Fast?}
\newblock
  \bibinfo{howpublished}{\url{https://www.youtube.com/watch?v=VcSX4ytEfcE}}.
\newblock


\bibitem[channel: ProPublica(2018)]%
        {america}
\bibfield{author}{\bibinfo{person}{YouTube channel: ProPublica}.}
  \bibinfo{year}{2018}\natexlab{}.
\newblock \bibinfo{title}{America Dodge \$660 Billion in Taxes Each Year}.
\newblock
  \bibinfo{howpublished}{\url{https://www.youtube.com/watch?v=ZxXOyYSHcfE}}.
\newblock


\bibitem[channel: THESSALONIAN31N(2011)]%
        {soure}
\bibfield{author}{\bibinfo{person}{YouTube channel: THESSALONIAN31N}.}
  \bibinfo{year}{2011}\natexlab{}.
\newblock \bibinfo{title}{Source Code, directed by Duncan Zowie Haywood Jones}.
\newblock
  \bibinfo{howpublished}{\url{https://www.youtube.com/watch?v=P8-GGkc2wGM}}.
\newblock


\bibitem[channel: Vincent~Price(1969)]%
        {butch}
\bibfield{author}{\bibinfo{person}{YouTube channel: Vincent~Price}.}
  \bibinfo{year}{1969}\natexlab{}.
\newblock \bibinfo{title}{Butch Cassidy and the Sundance Kid, directed by
  George Roy Hill}.
\newblock
  \bibinfo{howpublished}{\url{https://www.youtube.com/watch?v=UucXz3ZGmF4}}.
\newblock


\bibitem[channel: Vox(2014)]%
        {weed}
\bibfield{author}{\bibinfo{person}{YouTube channel: Vox}.}
  \bibinfo{year}{2014}\natexlab{}.
\newblock \bibinfo{title}{Weed is not More Dangerous than Alcohol}.
\newblock
  \bibinfo{howpublished}{\url{https://www.youtube.com/watch?v=zGSzj7ztszI}}.
\newblock


\bibitem[channel: wehrwolfa(1966)]%
        {persona}
\bibfield{author}{\bibinfo{person}{YouTube channel: wehrwolfa}.}
  \bibinfo{year}{1966}\natexlab{}.
\newblock \bibinfo{title}{Persona, directed by Ernst Ingmar Bergman}.
\newblock
  \bibinfo{howpublished}{\url{https://www.youtube.com/watch?v=mIKByxTU2nA}}.
\newblock


\bibitem[DATA-VIZ.cn(2018)]%
        {stock}
\bibfield{author}{\bibinfo{person}{DATA-VIZ.cn}.}
  \bibinfo{year}{2018}\natexlab{}.
\newblock \bibinfo{title}{Stock Market Crash in 28 years}.
\newblock
  \bibinfo{howpublished}{\url{https://v.qq.com/x/page/l0818ao7dtb.html}}.
\newblock


\end{thebibliography}



%%
%% If your work has an appendix, this is the place to put it.
% \appendix

% \section{Web Browser}

% \begin{figure*}
%       \includegraphics[width=\textwidth]{website.png}
%     \caption{Interactive study materials (including cases of videos, GIFs, and images) for twenty guidelines on our website.}
%     %\Descprition{This figure shows the interactive study materials for twenty guidelines on our website.}
%     \label{fig:website}
%  \end{figure*}

\end{document}
