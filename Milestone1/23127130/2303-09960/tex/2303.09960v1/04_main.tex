% This paper uses the template
% LLNCS macro package for Springer Computer Science proceedings;
% Version 2.20 of 2017/10/04
%
\documentclass[runningheads]{llncs}
%
\usepackage{graphicx}
% Used for displaying a sample figure. If possible, figure files should
% be included in EPS format.
%
\usepackage[letterpaper]{geometry}
\usepackage{amssymb}
\usepackage[T1]{fontenc}
% T1 fonts will be used to generate the final print and online PDFs,
% so please use T1 fonts in your manuscript whenever possible.
% Other font encondings may result in incorrect characters.
%
\usepackage[utf8]{inputenc}
\usepackage{amsmath,mathtools}
\let\proof\relax\let\endproof\relax
\usepackage{amsthm}
\usepackage{color}
\usepackage{xcolor}
\usepackage{soul}
\usepackage{algpseudocode}
\usepackage{algorithm}
\usepackage{cite}
\usepackage{url}
\usepackage{hyperref}
\usepackage{enumitem}
\usepackage{makecell}
\usepackage{multicol}
\usepackage{subfigure}
\usepackage{wrapfig}
\usepackage{lipsum,booktabs}

%% version control
% \usepackage[markup=default]{changes}
% Use "final" option to remove all tracking markups
\usepackage[final]{changes}
%%in \replaced{}{} 2nd option (the underlined version) drops

% If you use the hyperref package, please uncomment the following line
% to display URLs in blue roman font according to Springer's eBook style:
\renewcommand\UrlFont{\color{blue}\rmfamily}

\DeclareMathOperator\supp{supp}
\newcommand{\reals}{\mathbb{R}}
\newcommand{\vc}[1]{\mathbf{#1}}
\newcommand{\cvx}{\texttt{conv}}
\newcommand{\iset}[2]{\mathcal{#1}_{#2}}
\newtheorem{assumption}{Assumption}
%% Rather hacky definition of an "annote"
%% by riding on \added
% \newcommand{\note}[2][]{\added[#1,comment={#2}]{}}

%Uncomment in short version (PAKDD - anonym)
% \newcommand{\fullversion}[2]{#1}

%Uncomment in full version (arXiv)
\newcommand{\fullversion}[2]{#2}

% \hyphenation{op-tical net-works semi-conduc-tor}



\begin{document}

%
\newcommand\relatedversion{}
% \renewcommand\relatedversion{\thanks{The full version of the paper can be accessed at \protect\url{https://arxiv.org/abs/XXXX.XXXXX}}} % Replace URL with link to full paper or comment out this line

\makeatletter
\def\thanks#1{\protected@xdef\@thanks{\@thanks
        \protect\footnotetext{#1}}}
\makeatother

%%
%% The "title" command has an optional parameter,
%% allowing the author to define a "short title" to be used in page headers.
\title{Stochastic Submodular Maximization \\via Polynomial Estimators}%\thanks{Supported by NSF grant CCF-1750539.}\relatedversion}
%
%\titlerunning{Abbreviated paper title}
% If the paper title is too long for the running head, you can set
% an abbreviated paper title here
%
%%
%% The "author" command and its associated commands are used to define
%% the authors and their affiliations.
%% Of note is the shared affiliation of the first two authors, and the
%% "authornote" and "authornotemark" commands
%% used to denote shared contribution to the research.
\author{G\"{o}zde \"{O}zcan\inst{}\orcidID{0000-0002-2957-6893}
\and Stratis Ioannidis\inst{}\orcidID{0000-0001-8355-4751}}


\authorrunning{\replaced{G. \"{O}zcan and S. Ioannidis}{Anonymous Authors}}

% First names are abbreviated in the running head.
% If there are more than two authors, 'et al.' is used.
%
\institute{Electrical and Computer Engineering Department, Northeastern University, Boston MA 02115, USA\\
\email{\{gozcan, ioannidis\}@ece.neu.edu}}


% \renewcommand{\shortauthors}{\"{O}zcan and Ioannidis}
\maketitle
% typeset the header of the contribution
%
    
\begin{abstract}
%%%%%%%%% ABSTRACT



\begin{abstract}

% Version 1:
% Modern depth sensors such as LiDAR operate by sweeping laser-beams across the scene, resulting in a point cloud with notable 1D curve-like structures. However, most existing point cloud backbones  discard the rich, 1D traversal patterns and rely mainly on Euclidean operations.
% In this work, we present a novel point cloud processing scheme and backbone, \textbf{CurveCloudNet}, that exploits the curve-like structure of modern depth sensors. Concretely, %instead of treating each point independently, 
% we parameterize the point cloud as a collection of polylines and thus establish a local surface-level ordering on the points. 
% We then devise curve-specific operations to process the ``curve clouds:'' (1) a \textit{symmetrical 1D convolution}, 2) a \textit{ball grouping} operation for merging points along curves, and (3) an efficient \textit{1D furthest-point-sampling} algorithm on curves. \textbf{CurveCloudNet} combines these curve operations with existing point-based operations, resulting in an efficient, scalable, and expressive backbone that uses little GPU memory. We evaluate \textbf{CurveCloudNet} on the ShapeNet, Kortx, Audi Driving, and nuScenes datasets, showcasing state-of-the-art segmentation and classification performance across {\em both} object-level and large outdoor scene datasets, the first reported 3D point backbones to do so. 

% Version 2:
% In this work we introduce a new point cloud processing scheme and backbone, called CurveCloudNet, which takes advantage of the curve-like structure inherent in modern depth sensors such as LiDAR. While traditional point cloud backbones discard the rich, 1D laser-traversal patterns and rely on Euclidean operations, CurveCloudNet parameterizes the point cloud as a collection of polylines. This parameterization establishes a local surface-level ordering on the points. Our method applies curve-specific operations to process the ``curve clouds," including symmetrical 1D convolution, ball grouping for merging points along curves, and an efficient 1D furthest-point-sampling algorithm on curves. Combining these curve operations with existing point-based operations results in an efficient, scalable, and expressive backbone that uses little GPU memory. We evaluate CurveCloudNet on several datasets, including ShapeNet, Kortx, Audi Driving, and nuScenes, and report state-of-the-art segmentation and classification performance across \textbf{both} object-level and large outdoor scene datasets, making CurveCloudNet the first 3D point backbone to achieve such results.
% \vspace{-1em}

% Version 3
Modern depth sensors such as LiDAR operate by sweeping laser-beams across the scene, resulting in a point cloud with notable 1D curve-like structures. In this work, we introduce a new point cloud processing scheme and backbone, called \arch, which takes advantage of the curve-like structure inherent to these sensors. While existing backbones discard the rich 1D traversal patterns and rely on generic 3D operations, \arch parameterizes the point cloud as a collection of polylines (dubbed a ``curve cloud”), establishing a local surface-aware ordering on the points. By reasoning along curves, \arch captures lightweight curve-aware priors to efficiently and accurately reason in several \textbf{diverse} 3D environments. 
% , including a symmetric 1D convolution, a ball grouping for merging points along curves, and an efficient 1D farthest point sampling algorithm on curves.
We evaluate \arch on multiple synthetic and real datasets that exhibit distinct 3D size and structure.
%, including: ShapeNet, Audi Driving, nuScenes, Kitti, and a new dataset we name KortX.
We demonstrate that \arch outperforms both point-based and sparse-voxel backbones in various segmentation settings, notably scaling to large scenes better than point-based alternatives while exhibiting improved single-object performance over sparse-voxel alternatives.
In all, \arch is an efficient and accurate backbone that can handle a larger variety of 3D environments than past works. 
%In all, \arch is an off-the-shelf trainable and performant backbone that is ready for the diverse environments faced in open-world applications such as robotics. 

% in various segmentation settings, notably scaling better to large scenes than point-based alternatives while exhibiting better single object performance than sparse-voxel alternatives. 

% CurveCloudNet applies a mix of curve-specific operations and Euclidean point-based operations, resulting in an efficient and accurate backbone that can flexibly reason on \textit{many} different types of 3D scenes. 
% , including a symmetric 1D convolution, a ball grouping for merging points along curves, and an efficient 1D farthest point sampling algorithm on curves.
% By combining these curve operations with existing point-based operations, CurveCloudNet is an efficient and accurate backbone that can flexibly reason on \textit{many} different types of 3D scenes. 
% CurveCloudNet achieves state-of-the-art segmentation performance on the ShapeNet, Kortx, Audi Autonomous Driving, and nuScenes datsets, which include both individual objects and large outdoor scenes captured with various sensor scanning patterns. These evaluations demonstrate that \arch scales to large scenes better than existing point-based backbones while improving object-level semantic segmentation compared to sparse-voxel backbones.
% We evaluate semantic segmentation on four datasets - two common (ShapeNet and nuScenes) and two less common (KortX and Audi Driving). Taken together, these datasets patterns -
% We evaluate semantic segmentation the ShapeNet, Kortx, Audi Driving, and nuScenes datasets. 

% demonstrate that \arch outperforms both point-based and sparse-voxel backbones in various segmentation settings, notably scaling better to large scenes than point-based alternatives while exhibiting better single object performance than sparse-voxel alternatives. 

% Evaluations on ShapeNet, Kortx, Audi Driving, and nuScenes demonstrate that \arch outperforms point-based methods on both individual objects and large-scale scenes, outperforms sparse-voxel backbones on individual objects, and closes the gap between point-based and sparse-voxel backbones on large-scale scenes while requiring significantly less GPU memory.

% CurveCloudNet is evaluated on several datasets that include both individual objects and large
% outdoor scenes captured with various sensor scanning patterns. These evaluations demonstrate that our model can
% outperform point-based and sparse-voxel backbones at both
% object and scene level, achieving state-of-the-art performance on segmentation tasks.
\vspace{-1em}
\end{abstract}

\end{abstract}

\section{Introduction}
\section{Introduction}

The ability to reason about plans is critical for performing long-horizon tasks \citep{erol1996hierarchical, sohn2018hierarchical, sharma-etal-2022-skill}, compositional generalization \citep{corona-etal-2021-modular} and generalization to unseen tasks and environments \citep{shridhar2020alfred}.
Consider a simple long-horizon planning scenario where a robot is tasked with preparing a meal and serving it on the table. 
This presents a non-trivial planning problem since the agent needs to understand the sequence of operations required to perform the task and search for the relevant objects in the unfamiliar environment by interacting with various objects. %



Large language models have been recently shown to possess commonsense knowledge about the world such as object affordances and physical dynamics \citep{ouyang2022training,chowdhery2022palm}.
Early approaches considered text based environments and fine-tuned PLMs to predict actions given the history of past observations and actions \citep{jansen-2020-visually,micheli-fleuret-2021-language,yao-etal-2020-keep}.
Recent work has used this ability to reason about plans from text instructions in simulated household environments with simplifying assumptions such as text-only environment observations or feedback \citep{huang2022language,ahn2022can,li2022pre,logeswaran-etal-2022-shot}.


We focus on \emph{visually grounded planning} with PLMs --- the ability to adapt plans based on interaction and visual feedback from the environment.
While PLMs have strong planning commonsense priors, predictions from a PLM may not be directly realizable in the environment since the observation and action spaces are unknown.
This requires \emph{grounding} the PLM in the environment and adapting it to observe visual feedback, which is highly non-trivial.
Some prior works assume the availability of a pre-trained affordance function \citep{ahn2022can} or a success detector \citep{mirchandani2021ella}.
Notably, SayCan \citep{ahn2022can} completely decouples the PLM from observation information by selecting actions that have both high affordability (through a pre-trained affordance model) and high PLM likelihood.
Although this partially addresses the grounding problem, the use of visual feedback for action affordance alone is limited.
Often an agent must choose one of many affordable actions using information from observations.
For example, a driving agent should re-navigate and possibly turn around when encountering a ``road closed'' sign, but both turning around and driving forward are indistinguishable to SayCan because they are both affordable and the PLM is blind to observations.

Another workaround explored in prior work is translating the information in the visual observations to text using a pre-trained captioning system \citep{shridhar2021alfworld,huang2022language}.
However, it can be difficult to faithfully describe an image in words and information is lost in this inherently noisy process, which limits the information available to the planner.



Recent work shows that PLMs can be adapted for various natural language tasks by inserting tunable embeddings or soft prompts at the input of the PLM (also called prompt tuning or prefix tuning)~\citep{li-liang-2021-prefix,lester-etal-2021-power}.
This approach also extends to multi-modal understanding tasks such as image captioning \citep{mokady2021clipcap} and VQA \citep{tsimpoukelli2021multimodal} where images are encoded as soft prompts and finetuned for the target task.
Transformer based architectures have also been successfully applied to offline Reinforcement Learning in recent work \citep{chen2021decision,janner2021offline,li2022pre,reid2022can}.

Taking inspiration from these works, we propose the simple approach of embedding visual observations (`visual prompts') and \textit{directly inserting them as PLM input embeddings}.
The visual encoder and PLM are jointly trained for the target task, an approach we call \textbf{\oursfull}~(\ours).
By teaching the PLM to use observations for planning in an end to end manner, we remove the dependency on external data such as captions and affordability information that was used in prior work.
We show that this simple approach performs better than prior PLM-based planning approaches on two embodied planning benchmarks based on ALFWorld~\citep{shridhar2021alfworld} and Virtualhome~\cite{puig2018virtualhome}.




\section{Related Work}
  While  submodular optimization problems are generally NP-hard, the celebrated greedy algorithm \cite{nemhauser1978analysis} attains a $(1-1/e)$ approximation ratio for  submodular maximization subject to uniform matroids and a $1/2$ approximation ratio for general matroid constraints. As discussed in the introduction, the  continuous greedy algorithm \cite{calinescu2011maximizing} restores the $(1-1/e)$ approximation ratio by lifting the discrete problem to the continuous domain via the multilinear relaxation. %It is worth to mention here that the multilinear relaxation is a DR-submodular function, a.k.a. a continuous function with the diminishing returns property.

Stochastic submodular maximization, in which the objective is expressed as an expectation, has gained a lot of interest in the recent years \cite{asadpour2008stochastic, zhang2022stochastic, chen2018online}. Karimi et al. \cite{karimi2017stochastic} use a concave relaxation method that achieves the $(1-1/e)$ approximation guarantee, but only  for the class of submodular coverage functions. Hassani et al.~\cite{hassani2017gradient} provide projected gradients methods for the general case of stochastic submodular problems that achieve $1/2$ approximation guarantee.  Mokhtari et al. \cite{mokhtari2020stochastic} propose stochastic  conditional gradient methods for solving both minimization and maximization  stochastic submodular optimization problems. Their method for maximization, Stochastic Continous Greedy (SCG) can be interpreted as a stochastic variant of the continuous greedy algorithm \cite{vondrak2008optimal, calinescu2011maximizing} and achieves a tight $(1-1/e)$ approximation guarantee for monotone and submodular functions. %However, all these methods suffer from two sources of randomness (one comes from sampling the objective function and the other comes from estimating the multilinear relaxation via sampling its inputs).

Our work builds upon and relies on the approach by  \"{O}zcan et al.~\cite{ozcan2021submodular}, who studied ways of accelerating the computation of gradients via a polynomial estimator. Extending on the work of Mahdian et al.~\cite{mahdian2020kelly},  \"{O}zcan et al. show that submodular functions that can be written as compositions of (a) an analytic function and (b) a multilinear function can be arbitrarily well approximated via Taylor polynomials; in turn, this gives rise to a method for approximating their multilinear relaxation in a closed form, without sampling. We leverage this method in the context of stochastic submodular optimization, showing that it can also be applied in combination with SCG of Mokhtari et al.~\cite{mokhtari2020stochastic}: this eliminates one of the two sources of randomness, thereby reducing variance at the expense of added bias. From a technical standpoint, this requires controlling the error introduced by the bias of the polynomial estimator, while simultaneously accounting for the variance inherent in SCG, due to sampling instances.   %: this eliminates the latter source of randomness by utilizing the properties of deep submodular models that result from composition over multiple layers. In order to do so, we combine the stochastic continuous greedy algorithm proposed by Mokthari et al. \cite{mokhtari2020stochastic} with the deterministic estimator proposed by

\section{Technical Preliminary}

\subsubsection{Submodularity and Matroids.}\label{sec:submat}
Given a ground set $V = \{1, \ldots, n\}$ of $n$ elements, a set function $f:2^V\rightarrow\reals_+$ is submodular if and only if $f(B \cup \{e\}) - f(B) \leq f(A\cup \{e\}) - f(A)$, for all $A\subseteq B\subseteq V$ and $e\in V$. Function $f$ is \emph{monotone} if $f(A)\leq f(B)$, for every $A\subseteq B$.

\noindent \textbf{Matroids.} Given a ground set $V$, a matroid is a pair $\mathcal{M}=(V, \mathcal{I})$, where $\mathcal{I}\subseteq 2^V$ is a collection of \emph{independent sets}, for which the following hold: 
(a) if $B\in \mathcal{I}$ and $A \subset B$, then $A \in \mathcal{I}$, and (b)
  if $A, B\in \mathcal{I}$ and $|A|< |B|,$ there exists $x \in B\setminus A$ s.t. $A\cup\{x\}\in \mathcal{I}$.
 The \emph{rank} of a matroid $r_{\mathcal{M}}(V)$ is the largest cardinality of its elements, i.e.:
%\begin{align*}
  $  r_{\mathcal{M}}(V) = \max\{|A|: {A}\in\mathcal{I}\}.$
%\end{align*}
We introduce two examples of matroids:
\begin{enumerate}
    \item \textbf{Uniform Matroids.} The uniform matroid with cardinality $k$ is $\mathcal{I}=\{S\subseteq V, \, |S|\leq k\}$.
    \item \textbf{Partition Matroids.} Let $\mathcal{B}_1,\ldots, \mathcal{B}_m\subseteq V$ be a partitioning of $V$, i.e., $ \bigcap_{\ell=1}^m\mathcal{B}_\ell =\emptyset$ and $\bigcup_{\ell=1}^m\mathcal{B}_\ell = V$. Let also $k_\ell\in \mathbb{N}, \ell=1,\ldots,m$, be a set of cardinalities.   A partition matroid is defined as $\mathcal{I}=\{S\subseteq 2^V \, \mid  \, |S\cap \mathcal{B}_\ell|\leq k_{\ell}, \text{ for all } \ell=1,\ldots, m\}.$  
\end{enumerate}


\subsection{Problem Definition}\label{sec:probdef}
In this work, we focus on \textit{discrete  stochastic submodular maximization} problems. More specifically, we consider set function $f: 2^V \rightarrow \mathbb{R}_+$ of the form:
%\begin{equation} \label{eq:objFunc}
 $   f(S) = \mathbb{E}_{z \sim P}[f_z(S)],$ $S \subseteq V,$
%\end{equation}
where  $z$ is the realization of the random variable $Z$ drawn from a distribution $P$ over a probability space $(V_z,P)$. For each realization of $z \sim P$, the set function $f_z: 2^V \rightarrow \mathbb{R}_+$ is monotone and submodular. Hence, $f$ itself is monotone and submodular. The objective is to maximize $f$ subject to some constraints (e.g., cardinality or matroid constraints) by only accessing to i.i.d. samples of $f_{z \sim P}$. In other words, we wish to solve:
\begin{equation} \label{prob:stochsubmax}
    \max_{S \in \mathcal{I}} f(S) = \max_{S \in \mathcal{I}} \mathbb{E}_{z \sim P}[f_z(S)],
\end{equation}
where  $\mathcal{I}$ is a general matroid constraint.

Stochastic submodular maximization problems are of interest in the absence of the oracle that provides the exact value of $f(S)$: one can only access $f_z(S)$, for random instantiations $z\sim P$.  A well-known motivational example is contagion propagation in a network (a.k.a., the influence maximization problem \cite{kempe2003maximizing}). Given a graph with node set $V$, reachability of nodes from seeds are determined by sampling sub-graph $G=(V, E)$, via, e.g., the Independent Cascade or the Linear Threshold model~\cite{kempe2003maximizing}. The random edge set, in this case, plays the role of $z$, and the distribution over graphs the role of $P$.  The function $f_z(S)$ represents the ratio of nodes reachable from the seeds $S$ under the connectivity induced by edges $E$ in this particular realization of $z$. %The Independent Cascade model \cite{kempe2003maximizing} introduces a probabilistic model to deal with the uncertainties arising from each realization by defining a distribution $z$ over instances $z \sim P$ that share a set $V$ of nodes. The influence a set of seeds $S$ is then represented by the expectation 
The goal is to select seeds $S$ that maximize $f(S) = \mathbb{E}_{z \sim P}[f_z(S)]$; both $f$ and $f_z$ are monotone submodular functions; however computing $f$ in a closed form is hard, and $f(\cdot)$ can only be accessed through random instantiations of $f_z(\cdot)$.

% Instead of solving the problem in (\ref{prob:stochsubmax}) one can solve the continuous optimization problem
% \textcolor{red}{define G(y) in terms of expectation over S expectation of x sampled from y}

\subsection{Change of Variables and Multiliear Relaxation}  
There is a 1-to-1 correspondence between a binary vector $\vc{x}\in \{0,1\}^{n}$ and its support $S=\texttt{supp}(\vc{x})$. Hence, a set function $f: 2^V \rightarrow \reals_+$ can be interpreted as $f: \{0,1\}^n \rightarrow \reals_+$ via: 
$f(\vc{x}) \triangleq f(\texttt{supp}(\vc{x}))$ for $\vc{x} \in \{0,1\}^n$. We adopt this convention for the remainder of the paper. 
We also treat matroids as subsets of $\{0,1\}^n$, defined consistently with this change of variables via $\mathcal{M}=\{\vc{x}\in\{0,1\}^n: \supp(\vc{x})\in \mathcal{I}\}.$ For example, a partition matroid is: 
%\begin{align} \label{eq:part_mat}
$\mathcal{M} = \textstyle\left\{\vc{x} \in \{0,1\}^n\,\mid \bigcap_{\ell=1}^m  \left(\sum_{i\in B_\ell} x_i\leq k_\ell\right)\right\}.$
%\end{align} 
The \emph{matroid polytope} $\mathcal{C}\subseteq [0,1]^{n}$ is the convex hull of matroid $\mathcal{M}$, i.e., $\mathcal{C} = \cvx(\mathcal{M}).$


We define the \emph{multilinear relaxation} of $f$ as:
\begin{align}\label{eq: multilinear}
\begin{split}
    G(\mathbf{y}) & = \mathbb{E}_{S \sim \mathbf{y}}[f(S)] = \sum_{S \subseteq V} f(S) \prod_{i \in S} y_i \prod_{j \notin S} (1 - y_j)\\&=\mathbb{E}_{\mathbf{x} \sim \mathbf{y}}[f(\mathbf{x})]=\sum_{\mathbf{x}\in \{0,1\}^n} f(\mathbf{x}) \prod_{i\in V}y_i^{x_i}(1-y_i)^{(1-x_i)},  \quad \text{for}~\mathbf{y} \in [0, 1]^n.
    \end{split}
\end{align}
In other words, $G:[0,1]^n\to\reals_+$ is the expectation of $f$, assuming that $S$ is random and generated from independent Bernoulli trials: for every $i\in V$, $P(i\in S)=y_i$. 
%
The multilinear relaxation of $f$ satisfies several properties. First, it is indeed a relaxation/extension of $f$ over the (larger) domain $[0,1]^n$: for $\mathbf{x}\in \{0,1\}^n$, $G(\mathbf{x})=f(\mathbf{x})$, i.e., $G$ agrees with $f$ on integral inputs. Second, it is \emph{multilinear} (c.f.~Sec.~\ref{sec:multilinear}), i.e., affine w.r.t.~any single coordinate $y_i$, $i\in V$, when keeping all other coordinates $\mathbf{y}_{-i}=[y_j]_{j\neq i}$ fixed. 
%
Finally, in the context of stochastic submodular optimization,  it is an expectation that involves \emph{two sources of randomness}: (a) $z\sim P$, i.e., the random instantiation of the objective, \emph{as well as } (b) $\mathbf{x}\sim \mathbf{y}$, i.e., the independent sampling of the Bernoulli variables (i.e., the set $S$). In particular,  we can write: 
\begin{align}
    G(\mathbf{y}) = \mathbb{E}_{z\sim P}[G_z(\mathbf{y})],~\text{where}~G_z(\mathbf{y}) = \mathbb{E}_{\mathbf{x}\sim \mathbf{y}}[f_z(x)]~\text{is the multilinear relaxation of}~ f_z(\cdot).  
\end{align}

%is the multilinear extension of the function $f$ assuming $S$ is sampled at random, where $i \in S$ with probability $y_i$ and $y_i$ denotes the $i-$th element of the vector $\mathbf{y}$ and the convex set $\mathcal{C} = \text{conv}\{1_{I} : I \in \mathcal{I}\}$ is the down-closed matroid polytope.


%\textbf{Change of Variables. } There is a one-to-one correspondence between a binary vector $\mathbf{x}\in \{0,1\}^{n}$ and its support $S=\texttt{supp}(\mathbf{x})$. Hence, a set function $f: 2^V \rightarrow \reals_+$ can be interpreted as $f: \{0,1\}^n \rightarrow \reals_+$ via: 
%$f(\mathbf{x}) \triangleq f(\texttt{supp}(\mathbf{x}))$ for $\mathbf{x} \in \{0,1\}^n$. We adopt this convention for the remainder of the paper.

\subsection{Stochastic Continuous Greedy Algorithm}
The stochastic nature of the set function $f(S)$ requires the use the \emph{Stochastic Continuous Greedy (SCG)} algorithm \cite{mokhtari2020stochastic}. This is a stochastic variant of the  continuous greedy algorithm (method) \cite{vondrak2008optimal}, to solve (\ref{prob:stochsubmax}). The SCG algorithm uses a common averaging technique in stochastic optimization and computes the estimated gradient $\mathbf{d}_t$ by the recursion
\begin{equation}\label{eq:avgGrad}
    \mathbf{d}_t = (1 - \rho_t)\mathbf{d}_{t-1} + \rho_t \nabla G_{z_t} (\mathbf{y}_t),
\end{equation}
where $\rho_t$ is a positive step size and the algorithm initially starts with $\mathbf{d}_0 = \mathbf{y}_0 = \mathbf{0}$. Then, it proceeds in iterations, where in the $t$-th iteration it finds a feasible solution as follows
\begin{equation}
    \mathbf{v}_t \in \arg\max_{\mathbf{v} \in \mathcal{C}} \{\mathbf{d}_t^T\mathbf{v}\},
\end{equation}
where $\mathcal{C}$ is the matroid polytope (i.e., convex hull) of matroid $\mathcal{M}$.
After finding the ascent direction $\mathbf{v}_t$, the current solution $\mathbf{y}_t$ is updated as
\begin{equation}
    \mathbf{y}_{t+1} = \mathbf{y}_t + \frac{1}{T}\mathbf{\replaced{v}{y}}_t, 
\end{equation}
where $1/T$ is the step size. The steps of the stochastic continuous greedy algorithm are outlined in Algorithm~\ref{alg: SCG}.
The (fractional) output of Algorithm~\ref{alg: SCG} is within a $1-1/e$ factor from the optimal solution to Problem~(\ref{prob:stochsubmax}) (see Theorem~\ref{thm:main} below). This fractional solution can subsequently be rounded in polynomial time to produce a solution  with the same approximation guarantee w.r.t.~to Problem~(\ref{prob:stochsubmax}) using, e.g., either the pipage rounding \cite{ageev2004pipage} or the swap rounding \cite{chekuri2010dependent} methods.
\begin{algorithm}[!t]
    \caption{Stochastic Continuous Greedy (SCG)}\label{alg: SCG}
    \textbf{Require:} Step sizes $\rho_t > 0$. Initialize $\mathbf{d}_0 = \mathbf{y}_0 = \mathbf{0}.$
    \begin{algorithmic}[1] % The number tells where the line numbering should start
            \For{$t = 1, 2, \ldots, T$}
                \State Compute $\mathbf{d}_t = (1 - \rho_t) \mathbf{d}_{t-1} + \rho_t \nabla G_{z_t}(\mathbf{y}_t)$;
                \State Compute $\mathbf{v}_t \in \arg\max_{\mathbf{v} \in \mathcal{C}} \{\mathbf{d}_t^T \mathbf{v}\}$;
                \State Update the variable $\mathbf{y}_{t+1} = \mathbf{y}_t + \frac{1}{T} \mathbf{v}_t$;
            \EndFor
    \end{algorithmic}
\end{algorithm}
\subsubsection{Sample Estimator.}
The gradient $\nabla G_{z_t}$ is needed to perform step \eqref{eq:avgGrad}; computing it directly via Eq.~\eqref{eq: multilinear}. requires exponentially many calculations. Instead, both Calinescu et al. \cite{calinescu2011maximizing} and Mokhtari et al. \cite{mokhtari2020stochastic} estimate it via \emph{sampling}. % Since $f(S) = \mathbb{E}_{z \sim P}[f_z(S)]$, we obtain $G(\mathbf{y}) = \mathbb{E}_{z \sim P}[G_z(\mathbf{y})]$ where $G$ and $G_z$ denote the multilinear extension of $f$ and $f_z$, respectively. 
In particular, due to multilinearity (i.e., the fact that $G_z$ is affine w.r.t. a coordinate $x_i$, we have: 
\begin{equation}\label{eq:partialGrad}
    \frac{\partial G_z(\mathbf{y})}{\partial x_i} = G_z ([\mathbf{y}]_{+i}) - G_z ([\mathbf{y}]_{-i}),\quad\text{for all}~i\in V,
\end{equation}
where $[\mathbf{y}]_{+i}$ and $[\mathbf{y}]_{-i}$ are equal to the vector $\mathbf{y}$ with the $i$-th coordinate set to $1$ and $0$, respectively. The gradient of $G$ can thus be estimated by (a) producing $N$ random samples $\mathbf{x}^{(l)}$, for $l \in \{1, \ldots, N\}$ of the random vector $\mathbf{x}$, and (b) computing the empirical mean of the r.h.s. of (\ref{eq:partialGrad}), yielding
\begin{equation}\label{eq:sampleEst}
    \frac{\partial \widehat{G_{z}(\mathbf{y})}}{\partial x_i} = \frac{1}{N} \sum_{l=1}^N \left(f_z ([\mathbf{x}^{(l)}]_{+i}) - f_z ([\mathbf{x}^{(l)}]_{-i})\right), \quad\text{for all}~i\in V.
\end{equation}


Mokhtari et al. \cite{mokhtari2020stochastic} make the following assumptions:
\begin{assumption} \label{asm:monSub}
    Function $f: \{0, 1\}^n \rightarrow \mathbb{R}_+$ is monotone and submodular.
\end{assumption}

\begin{assumption} \label{asm:boundedNorm}
    The Euclidean norm of the elements in the constraint set $\mathcal{C}$ are uniformly bounded, i.e., for all $\mathbf{y} \in \mathcal{C}$, there exists a $D$ s.t. $\|\mathbf{y}\| \leq D.$
\end{assumption}


Under these assumptions, SCG combined with the sampling estimator in Eq.~\eqref{eq:partialGrad}, yields the following guarantee:
\begin{theorem} \label{thm:theirs}
    % \hl{\textup{[Their Theorem]}} 
    [Mokhtari et al. \cite{mokhtari2020stochastic}] Consider Stochastic Continuous Greedy (SCG) outlined in Algorithm~\ref{alg: SCG}, with $\nabla G_{z_t}(\mathbf{y}_t)$ replaced by $\nabla \widehat{G_{z_t}(\mathbf{y}_t)}$ given by (\ref{eq:sampleEst}). Recall the definition of the multilinear extension function $G$ in \eqref{eq: multilinear} and set the averaging parameter as $\rho_t = 4/(t+8)^{2/3}$. If Assumptions~\ref{asm:monSub}~\&~\ref{asm:boundedNorm} are satisfied, then the iterate $\mathbf{y}_T$ generated by SCG satisfies the inequality
    \begin{equation}
        \mathbb{E}\left[G(\mathbf{y}_T)\right] \geq (1-1/e)OPT - \frac{15DK}{T^{1/3}} - \frac{f_{\max}rD^2}{2T},
    \end{equation}
    where $OPT = \max_{\mathbf{y} \in \mathcal{C}} G(\mathbf{y})$ and $K = \max\{3\|\nabla G(\mathbf{y}_0) - \mathbf{d}_0\|, 4 \sigma + \sqrt{3r} f_{\max}D\},$ where $D$ is the diameter of the convex hull $\mathcal{C}$, $f_{\max}$ is the maximum marginal value of the function $f$, i.e., $f_{\max} = \max_{i \in \{1, \ldots, n\}} f(\{i\})$,  $r$ is the rank of the matroid $\mathcal{I}$, and
    %\begin{align}
    $\sigma^2 =\sup_{\mathbf{y}\in \mathcal{C}} \mathbb{E}\left[\|\widehat{\nabla G_z(\mathbf{y}) } - G(\mathbf{y})  \|\right],$
    %\end{align}
    where $\widehat{\nabla G_z}$ is the sample estimator given by Eq.~\eqref{eq:sampleEst}.
    %2\sqrt{n}\sqrt{\max_{j \in [n]} \mathbb{E}[f_z(\{j\})^2]}
\end{theorem}
Thus, by appropriately setting the number of iterations $T$, we can produce a solution that is arbitrarily close to $1-1/e$ from the optimal (fractional) solution. Again, this can be subsequently rounded (see, e.g., \cite{ageev2004pipage,calinescu2011maximizing}) to produce an integer solution with the same approximation guarantee.
%
It is important to note that the number of steps required depends on $\sigma^2$, which is a (uniform over $\mathcal{C}$) bound on the variance of the estimator given by Eq.~\eqref{eq:sampleEst}. This variance contains \emph{two sources of randomness}, namely $z\sim P$, the random instantiation, and $\mathbf{x}\sim\mathbf{y}$, as multiple such integer vectors/sets are sampled in Eq~\eqref{eq:sampleEst}. In general, the variance will depend on the number of samples $N$ in the estimator, and will be bounded (as $G$ is bounded).\footnote{For example, even for $N=1$, the submodularity of $f_z$ and Eq.~\eqref{eq:partialGrad} imply that $\sigma^2\leq 2 {n}{\max_{j \in [n]} \mathbb{E}[f_z(\{j\})^2]}$ \cite{mokhtari2020stochastic}, though this bound is loose/a worst-case bound.} 
\subsection{Multilinear Functions and the Multilinear Relaxation of a Polynomial}\label{sec:multilinear}
Recall that a \emph{polynomial} function $p: \reals^n \rightarrow \mathbb{R}$ can be written as a linear combination of several monomials, i.e.,
\begin{equation} \label{eq:polyFormat}
     p(\mathbf{y}) = c_0+\sum_{\ell \in \mathcal{I}} c_{\ell} \prod_{i \in \iset{J}{\ell}} y_i^{k_i^{\ell}},
\end{equation}
where $c_{\ell}~\in~\reals$ for $\ell$ in some index set $\mathcal{I}$, subsets $\iset{J}{\ell}~\subseteq~V$ determine the terms of each monomial, and %\footnote{By convention, if $\mathcal{J}_{\ell} = \emptyset$, we set $\prod_{i\in\mathcal{J}_{\ell}} y_i = 1$.}
, and $\{k_i^{\ell}\}_{i \in \iset{J}{\ell}} \subset \mathbb{N}$ are natural exponents. W.l.o.g. we assume that $k_i^{\ell} \geq 1$ (as variables with zero exponents can be ommited). The degree of the monomial indexed by $\ell\in\iset{I}{}$ is $k^\ell=\sum_{i\in \iset{J}{\ell}} {k_i^\ell}$, and the degree of polynomial $p$ is $\max_{\ell\in \iset{I}{}} k^\ell$, i.e., the largest degree across monomials.


A function $f: \reals^N \rightarrow \reals$ is \emph{multilinear} if it is affine w.r.t.~each of its coordinates \cite{broida1989comprehensive}. %A function $g: \reals^n \rightarrow \reals$ that can be written as the sum of monomials is called a \emph{polynomial}. 
Alternatively, multilinear functions are polynomial functions in which the degree of each variable in a monomial is at most $1$; that is, multilinear functions can be written as:
\begin{equation} \label{eq:multi}
    f(\mathbf{y}) = c_0'+\sum_{\ell \in \mathcal{I}} c_{\ell}' \prod_{i \in \iset{J}{\ell}} y_i,
\end{equation}
where $c_{\ell}~\in~\reals$ for $\ell$ in some index set $\mathcal{I}$, and subsets $\iset{J}{\ell}~\subseteq~V$, again determining monomials of degree \emph{exactly equal to} ${|\iset{J}{\ell}|}.$ %\footnote{By convention, if $\mathcal{J}_{\ell} = \emptyset$, we set $\prod_{i\in\mathcal{J}_{\ell}} x_i = 1$.}
Given a polynomial $p$ defined by the parameters in Eq.~(\ref{eq:polyFormat}), let  %a \textcolor{red}{multilinearization operator} as follows
\begin{equation} \label{eq:multiFormat}
     \dot{p}(\mathbf{y}) = c_0+\sum_{\ell \in \mathcal{I}} c_{\ell} \prod_{i \in \iset{J}{\ell}} y_i,
\end{equation}
be the multilinear function resulting from  $p$, \emph{by replacing all its exponents $k_i^{\ell} \geq 1$ with $1$}. We call this function the \emph{multilinearization} of $p$. The multilinearization of $p$ is inherently linked to its multilinear relaxation:
\begin{lemma}[\"Ozcan et al.~\cite{ozcan2021submodular}] \label{lem:relaxation_of_multi}
Let $p: [0, 1]^n \rightarrow \mathbb{R}$ be an arbitrary polynomial and let $\dot{p}:\reals^n \rightarrow \reals_+$ be  its multilinearization, given by Eq.~\eqref{eq:multiFormat}. Let $\vc{x} \in \{0, 1\}^n$ be a random vector of independent Bernoulli coordinates parameterized by $\vc{y}\in~[0, 1]^n$. Then, $\mathbb{E}_{\vc{x}\sim\vc{y}}[p(\vc{x})] = \mathbb{E}_{\vc{x}\sim\vc{y}}[\dot{p}(\vc{x})] =\dot{p}(\vc{y}).$
\end{lemma}
\begin{proof}
Observe that 
%\begin{equation} \label{eq:multilin_rel}
 $   \dot{p}(\mathbf{x}) = p(\mathbf{x})$, for all $\mathbf{x} \in \{0, 1\}^n.$ 
%\end{equation} 
This is precisely because $x^{k} = x$ for  $x \in \{0, 1\}$ and all $k \geq 1$.  The first equality therefore follows. On the other hand, $\dot{p}(\mathbf{x})$ is the multilinear function given by Eq.~\eqref{eq:multiFormat}. Hence
%\begin{align*}
 $   \mathbb{E}_{\vc{x}\sim\vc{y}}[\dot{p}(\vc{x})] = \mathbb{E}_{\vc{x}\sim\vc{y}}\left[ c_0+\sum_{\ell \in \mathcal{I}} c_{\ell} \prod_{i \in \iset{J}{\ell}} x_i\right] = c_0+\sum_{\ell \in \mathcal{I}} \mathbb{E}_{\vc{x}\sim\vc{y}}\left[ \prod_{i \in \iset{J}{\ell}} x_i\right]=c_0+\sum_{\ell \in \mathcal{I}}  \prod_{i \in \iset{J}{\ell}}\mathbb{E}_{\vc{x}\sim\vc{y}}\left[ x_i\right]=\dot{p}(\mathbf{y}),$
%\end{align*}
where the second to last equality holds by the independence across $x_i$, $i\in V$.
\end{proof}
An immediate consequence of this lemma is that the multilinear relaxation of any polynomial function can be computed \emph{without sampling}, by simply computing its multilinearization. This is of particular interest of course for submodular functions that are themselves polynomials (e.g., coverage functions~\cite{karimi2017stochastic}). \"Ozcan et al.~extend this to submodular functions that can be written as compositions of a scalar and a polynomial function, by approximating the former via its Taylor expansion. We extend  and generalize this to the case of stochastic submodular functions, so long as the latter can be approximated arbitrarily well by polynomials.  

\section{Main Results}


\subsection{Polynomial Estimator}
To leverage Lem.~\ref{lem:relaxation_of_multi} to the case of stochastic submodular functions, we make the following  assumption:
\begin{assumption} \label{asmp:boundedPoly}
    For all $z %\textcolor{red}{ \sim P}$
    \in V_z$, there exists a sequence of polynomials $\{\hat{f}_z^L\}_{L=1}^{\infty}$, $\hat{f}_z^L: \mathbb{R}^n \rightarrow \mathbb{R}$ such that  
    %\begin{align}
    $\lim_{L \rightarrow \infty} |f_z(\mathbf{x}) - \hat{f}_z^L(\mathbf{x})| = 0,$ uniformly over $\mathbf{x} \in \{0, 1\}^n,$
    %\end{align}
    i.e. there exists $\varepsilon_z (L) \geq 0$ such that $\lim_{L \rightarrow \infty} \varepsilon_z (L) = 0$ and $|f_z(\mathbf{x}) - \hat{f}_z^L(\mathbf{x})| \leq \varepsilon_z(L)$,  for all $\mathbf{x} \in \{0, 1\}^n$.
    % There exists a polynomial  $\hat{f_z^L}: [a, b]^n \rightarrow \mathbb{R}$ of degree $L$ for $L \in \mathbb{N}$, such that $|f_z(\mathbf{x}) - \hat{f_z^L}(\mathbf{x})| \leq \varepsilon_L$, where $\lim_{L \rightarrow \infty} \varepsilon_L = 0$, for all $\mathbf{x} \in [a, b]^n$.
\end{assumption}
%The existence of such polynomials can be guaranteed by the Stone-Weierstrass theorem. 


% Moreover, these polynomials can be described as the following
%Moreover, when defined over the set $\{0, 1\}^n$, function $\hat{f_z^L}: \{0, 1\}^n \rightarrow \mathbb{R}$ can be expressed as a multilinear function. 
% For each polynomial function described in the format above, we can define the multilinear version of that function by limiting $k_i \leq 1$ as the following


 In other words, we assume that we can asymptotically approximate every function $f_z$ with a polynomial arbitrarily well. Note that there already exists a polynomial function that approximates each $f_z$ \emph{perfectly} (i.e., $\epsilon_z=0$), namely, its multilinear relaxation $G_z$. However, the number of terms in this polynomial is exponential in $n$. In contrast, Asm.~\ref{asmp:boundedPoly} requires exact recovery only asymptotically. In many cases, this allows us to construct polynomials with only a handful (i.e., polynomial in $n$) terms, that can approximate $f_z$. We will indeed present such polynomials for several applications of interest in Section~\ref{sec: examples}.
 %
 Armed with this assumption, we define an estimator $\widehat{\nabla G_z^L}$ of the gradient  of the multilinear relaxation $G$ as follows:
\begin{equation}
    \begin{split}
        \frac{\widehat{{\partial G_z^L}}}{\partial y_i}\big|_{\vc{y}} &\equiv \mathbb{E}_{\vc{x}\sim\vc{y}}[\hat{f}_z^{L}([\vc{x}]_{+i})] - \mathbb{E}_{\vc{x}\sim\vc{y}}[\hat{f}_z^{L}([\vc{x}]_{-i})]  %\stackrel{\text{Eq.}~\ref{eq:multilin_rel}}{=} \mathbb{E}_{\vc{x}\sim\vc{y}}[\dot{\hat{f}}_z^{L}([\vc{x}]_{+i})] - \mathbb{E}_{\vc{x}\sim\vc{y}}[\dot{\hat{f}}_z^{L}([\vc{x}]_{-i})]  \\
         \stackrel{\text{Lem.}~\ref{lem:relaxation_of_multi}}{=} \dot{\hat{f}}_z^{L}([\vc{y}]_{+i}) - \dot{\hat{f}}_z^{L}([\vc{y}]_{-i}), \text{ for all $i \in V$}.\label{eq: poly_estimator}
    \end{split}
\end{equation}
In other words, our estimator is constructed by replacing the multilinear relaxation $G_z$ in Eq.~\eqref{eq:partialGrad} with the multilinear relaxation of the approximating polynomial $\hat{f}_z$. In turn,  by Lem.~\ref{lem:relaxation_of_multi}, \emph{the latter can be computed deterministically  (without any sampling of the Bernoulli variables $\mathbf{x}\sim \mathbf{y}$)}, in closed form: the latter is given by the multilinearization  $\dot{\hat{f}}_z^L$ of polynomial ${\hat{f}}_z^L$.

Nevertheless, our deterministic estimator given by Eq.~\eqref{eq: poly_estimator} has a \emph{bias}, precisely because of our approximation of $f_z$ via the polynomial $\hat{f}_z^L$. We characterize this bias via the following lemma:
\begin{lemma} \label{lem:gradientBias}
% might be moved to appendix
Assume that function $f_z$ satisfies Asm.~\ref{asmp:boundedPoly}. Let $\nabla G_z$ be the unbiased stochastic gradient for a given $f_z$ and let $\widehat{\nabla G_z^L}$ be the estimator of the multilinear relaxation given by \eqref{eq: poly_estimator}. Then, 
%\begin{equation}\label{eq:estimator_bound}
 $   \big\|\nabla G_z(\vc{y}) - \widehat{\nabla G_z^L}(\vc{y})\big\|_2 \leq 2\sqrt{n}\varepsilon_z (L),$ for all $\mathbf{y} \in \mathcal{C}$.
%\end{equation}
% where $\epsilon_z(L) = [\epsilon_{i, z}(L)]_{i=1}^n \in \reals^n$ and
% \begin{equation}
%     \epsilon_{i, z}(L) \triangleq \mathbb{E}_{\vc{y}}[\varepsilon (L)_{+i})]+\mathbb{E}_{\vc{y}}[\varepsilon (L)_{-i})].
% \end{equation}
% Moreover, $\lim_{L \to \infty} \|\epsilon_z(L)\|_2= 0,$ uniformly on $[0,1]^n$.
\end{lemma}

The proof can be found in App.~\ref{app:proof_gradBiasLemma}\deleted{of the supplement}. Hence, we can approximate $\nabla G$ arbitrarily well, uniformly over all $\vc{x}\in[0,1]^n$.
% \begin{assumption} \label{asm:Lipschitz}
%     The function $G$ is $DR-$submodular and monotone. Further, its gradients are $L-$Lipschitz continuous over the set $\mathcal{X}$, i.e., for all $\mathbf{x}$, $\mathbf{y} \in \mathcal{X}$ $$\|\nabla G(\mathbf{x}) - \nabla G(\mathbf{y})\| \leq L\|\mathbf{x} - \mathbf{y}\|.$$
% \end{assumption}
We can thus use our estimator in the SCG algorithm instead of of the sample estimator of the gradient (Eq.~\eqref{eq:sampleEst}). %To characterize the end-to-end behavior of SCG under our estimation, we make one final additional assumption:
%\begin{assumption} \label{asm:boundedVar0}
%    The variance of the unbiased stochastic gradients $G_{z}(\mathbf{y})$ is bounded above by $\sigma_0^2$, i.e., for any vector $\mathbf{y} \in \mathcal{C}$, there exists a $\sigma_0^2$ s.t. 
%    $$\mathbb{E}\left[\left\|\nabla G(\mathbf{y}) - \nabla G_{z}(\mathbf{y})\right\|^2 \right] \leq \sigma_0^2, $$ 
%    where the expectation is with respect to the randomness of $z \sim P$.
%\end{assumption}
%Note that the variance is this assumption is fully governed by $z \sim P$: in contrast to the sample estimator, the the variability due to sampling $\mathbf{x}\sim\mathbf{y}$ bears no role in this assumption, nor in our final guarantee, that 
We prove that this yields the following guarantee:
\begin{theorem}\label{thm:main}
    % \hl{\textup{[Main Theorem]}} 
    Consider Stochastic Continuous Greedy (SCG) outlined in Algorithm~\ref{alg: SCG}. Recall the definition of the multilinear extension function $G$ in \eqref{eq: multilinear}%and the definitions of $r$ and $m_f$ in Lemma ?
    . If Asm.~\ref{asm:monSub}  %~\ref{asm:Lipschitz}\&~\ref{asm:boundedVar0} 
    is satisfied and $\rho_t = 4/(t+8)^{2/3}$%, and the function $f$ is monotone and submodular
    , then the objective function value for the iterates generated by SCG satisfies the inequality
        \begin{align*}
            \mathbb{E}[G(\mathbf{y}_T)] \geq (1-1/e) OPT - \frac{15 D K}{T^{1/3}} - \frac{f_{\max} r D^2}{2T},
        \end{align*}
    where $K = \max\{3\|\nabla G(\mathbf{y}_0 - \mathbf{d}_0)\|^2, \sqrt{16\sigma_0^2 + 224\sqrt{n}\varepsilon (L)} + 2\sqrt{r}f_{\max}D\},$ $OPT = \max_{\mathbf{y} \in \mathcal{C}} G(\mathbf{y})$, $r$ is the rank of the matroid $\mathcal{I}$, $\varepsilon(L)=\mathbb{E}_{z\sim P}[\varepsilon_z(L)]$,  $f_{\max}$ is the maximum marginal value of the function $f$, i.e., $f_{\max} = \max_{i \in \{1, \ldots, n\}} f(\{i\})$, and 
    %\begin{equation}\label{eq:boundedVar0}
    $\sigma_0^2 = \sup_{\mathbf{y}\in \mathcal{C}} \mathbb{E}_{z\sim P}\left[\left\|\nabla G(\mathbf{y}) - \nabla G_{z}(\mathbf{y})\right\|^2 \right].$
    %\end{equation}
    \end{theorem}
The proof of the theorem can be found
 \fullversion{in App.~\ref{app: proofMainThm} \deleted{of the supplementary material.}}{in App.~\ref{app: proofMainThm} \deleted{of the supplementary material.}} Our proof follows the main steps of \cite{mokhtari2020stochastic}%Mokhtari et al.
 , using however the bias guarantee from Lem.~\ref{lem:gradientBias}; to do so, we need to deal with the fact that our estimator is not unbiased, but also that stochasticity is still present (as variables $z$ are still sampled randomly). This is also reflected in our bound, that contais both a bias term (via $\varepsilon(L)$) and a variance term (via $\sigma_0$).
 
 Comparing our guarantee to Thm.~\ref{thm:theirs}, we observe two main differences. On one hand, we have replaced the uniform bound of the variance $\sigma^2$ with the smaller quantity $\sigma^2_0$: the latter is quantifying the gradient variance w.r.t.~$z$, and is thus smaller than $\sigma$, that depends on the variance of  \emph{both} $z$ \emph{and} $\mathbf{x}\sim \mathbf{y}$. Crucially, $\sigma^2_0$ is an ``inherent'' variance, \emph{independent of the gradient estimation process}: it is the variance due to the  randomness $z$, which is inherent in how we access our  stochastic submodular objective and thus cannot be avoided.  On the other hand, this variance reduction comes at the expense of introducing a bias term. This, however, can be suppressed via Asm.~\ref{asmp:boundedPoly}; as we discuss in the next section, for several problems of interest, this can be made arbitrarily small using only a polynomial number of terms in $\hat{f}^L_z$.


\section{Problem Examples}\label{sec: examples}
In this section, we list several problems that can be tackled through our approach, also summarized in Tab.~\ref{table: problems}; these are similar to the problems considered by \"Ozcan et al.~\cite{ozcan2021submodular}, but cast into the stochastic submodular optimization setting. All problems correspond to trivially bounded variances $\sigma_0^2$ (again, because functions $f_z$ are bounded); we thus focus on determining their bias $\epsilon(L)$. For space reasons, we report Cache Networks (CN) in Table~\ref{table: problems}, but provide details for it in the \deleted{supplement }\deleted{(}App.~\ref{app:customBias}\deleted{)}.

\begin{table*}[ht] 
\caption{Summary of problems satisfying Asm.~\ref{asm:monSub}\&~\ref{asmp:boundedPoly}.}
\centering \label{table: problems}
\resizebox{\textwidth}{!}{
    \begin{tabular}{ |c|c|c|c|c|c| } 
    % \hline
    %  & \thead{variables} & \thead{$g_i(\mathbf{x})$} & \thead{$f_i(g_i(\mathbf{x}))$} & \thead{$f(S) \triangleq f(\vc{x})$} \\
     \cline{2-6}
     \multicolumn{1}{c}{} 
     & \multicolumn{1}{|c|}{\thead{Input}} 
     & \multicolumn{1}{|c|}{\thead{$g_z: \{0, 1\}^{|V|} \rightarrow [0, 1]$\\
                                   $\vc{x} \rightarrow g_z(\vc{x})$}} 
     % & \multicolumn{1}{|c|}{\thead{$h: [0, 1] \rightarrow \reals_+$\\
     %                               $s \rightarrow h(s)$}} 
     & \multicolumn{1}{|c|}{\thead{$f_z:\{0, 1\}^{|V|} \rightarrow \reals_+$\\
                                   $\vc{x} \rightarrow f_z(\vc{x})$}} %$F(S) \triangleq F(\vc{x}) = \sum\limits_i w_i f(s_i)$}
     & \multicolumn{1}{|c|}{\thead{$\hat{f}_z^L:\{0, 1\}^{|V|} \rightarrow \reals_+$\\
                                   $\vc{x} \rightarrow \hat{f}_z^L(\vc{x})$}}
     & \multicolumn{1}{|c|}{\thead{Bias\\
                                   $\varepsilon(L)$}}\\
     \hline
     \thead{SM} 
     & \makecell{Weighted bipartite graph \\
     %Partitions $\bigcup_{j=1}^M\{P_j\} = V$ \\
                $G = (V \cup P)$ weights $\vc{r}_z\in\reals_+^{n}$, \\
                and $\sum_{i=1}^n r_{i, z}=1$
                } 
     & $\sum_{i \in V\cap P_j} r_{i, z}x_i$ 
     & \makecell{$\sum_{j=1}^J h\left(g_z(\mathbf{x})\right)$,\\ where\\
     $h(s) = \log(1+s)$} 
     & \makecell{$\hat{h}^L(g_z(\mathbf{x}))$, \\
     where $\hat{h}^L$ is %given by 
     Eq.~\eqref{eq: taylor_f_iL}}
     & $\frac{1}{(L+1) 2^{L+1}}$ \\
     \hline
     \thead{IM} 
     & \makecell{Instances $G = (V, E)$\\
                 of a directed graph, \\
                 partitions $P_{v}^z \subset V$
                 %Sets $G$'s are instances\\
                 %that share set $V$ of\\
                 %nodes by the IC \cite{kempe2003maximizing} \\
                 %model, $v \in V$ are nodes\\
                 %of a directed graph, $P_v$\\
                 %is the set of all nodes\\
                 %having a directed path\\
                 %to $v$.
                 } 
     & $\sum\limits_{i \in V}\frac{1}{N}\Big(1 - \prod\limits_{u \in P_{i}^z}(1-x_u)\Big)$ 
     & \makecell{$h\left(g_z(\vc{x})\right)$\\
     where\\
     $h(s) = \log(1+s)$
     } 
     & \makecell{$\hat{h}^L(g_z(\mathbf{x}))$, \\
     where $\hat{h}^L$ is %given by 
     Eq.~\eqref{eq: taylor_f_iL}}
     & $\frac{1}{(L+1) 2^{L+1}}$ \\
     \hline
     \thead{FL} 
     & \makecell{Complete weighted bipartite\\
                 graph $G = (V \cup V')$\\
                 weights $w_{i_\ell, z} \in [0, 1]^{N \times |z|}$
                 } 
     & $\sum\limits_{\ell=1}^{N}(w_{i_\ell, z}-w_{i_{\ell+1}, z})\left(1-\prod\limits_{k=1}^\ell(1-x_{i_k})\right)$ 
     & \makecell{$h\left(g_z(\vc{x})\right)$\\
     where\\
     $h(s) = \log(1+s)$
     } 
     & \makecell{$\hat{h}^L(g_z(\mathbf{x}))$, \\
     where $\hat{h}^L$ is %given by 
     Eq.~\eqref{eq: taylor_f_iL}} 
     & $\frac{1}{(L+1) 2^{L+1}}$ \\
     \hline
     \thead{CN} 
     & \makecell{Graph $G = (V, E)$,\\
                 service rates $\mu \in \reals_+^{|z|}$, \\
                 requests $r \in \mathcal{R}$, $P_z$ path of $r$, \\
                 arrival rates $\lambda \in \reals_+^{|\mathcal{R}|}$\\
                 %$i \in E$ is an edge on\\
                 %a Kelly Cache Network,\\ 
                 %$\mu_i$ is the service rate \\ 
                 %of the queue in $i$,\\ 
                 %$\lambda^r$ is the arrival rate\\  
                 %of class $r \in \mathcal{R}$, \\
                 %$j$ is node in $P_i$ where \\ 
                 %$P_i$ is the path request \\
                 %$r$ follows.
                 } 
     & $\frac{1}{\mu_z}\sum_{r \in \mathcal{R}:z\in p^r} \lambda^r \prod_{k'=1}^{k_{p^r}(v)}(1-x_{p_k^r, i^r})$ 
     & \makecell{$h(g_z(\mathbf{0})) - h(g_z(\mathbf{x}))$\\
     where \\
     $h(s) = %\frac
     {s}/{(1 - s)}$} 
     & \makecell{$\hat{h}^L(g_z(\mathbf{x}))$, \\
     where $\hat{h}^L$ is %given by 
     Eq.~\eqref{eq: f_iL_CN}} 
     & $\frac{\bar{s}^{L+1}}{1-\bar{s}}$ \\
     \hline
    \end{tabular}}
    \vspace*{-10pt}
\end{table*}

\subsection{Data Summarization (SM)\cite{lin2011class, mirzasoleiman2016fast,kazemi2019submodular}}
In data summarization, ground set $V$ is a set of tokens, representing, e.g., words or sentences in a document. A corpus of documents $V_z$ is presented to us sequentially, and the goal is to select a ``summary'' $S\subseteq V$ that is representative of $V_z$. The summary should be simultaneously (a) representative of the corpus, and (b) diverse. %We present here the submodular objective function proposed by Kazemi et al. \cite{kazemi2019submodular}. %Assume that each token $i$ has a value $r_i\in [0,1]$, where $\sum_i r_i=1$. 

To be representative, the summary $S\subset V$ should contain tokens of high value, where the value of a token is document-dependent: %but should simultaneously be diverse. %Kazemi et al.~\cite{kazemi2019submodular} achieve this by mapping $V$ to a set $P$ of keywords. 
 for  document $z \in V_z$,  token $i \in V$  has a value $r_{i, z}\in [0,1]$, where $\sum_i r_{i, z}=1$. An example of such a value is the term frequency, i.e., the number of times the token appears in the document, divided by the document's length (in tokens). To be diverse, the summary should contain tokens that cover different subjects. To that end, if tokens are partitioned in to subjects, represented by a partition $\{P_j\}_{j=1}^J$ of $V$, the objective is given by $f(\mathbf{x}) = \mathbb{E}_z(f_z(\mathbf{x}))$ where %partitioning $V$ to sets $\{P_j\}_{j=1}^M$, where each set $P_j\subset V$ contains tokens that are similar. They then seek a summary that maximizes 
%\begin{equation}\label{eq:smob}
 $   f_z(\mathbf{x}) = \textstyle\sum_{j=1}^J h\left(\sum_{i \in V\cap P_j} r_{i, z}x_i\right),$
%\end{equation}%$S\subseteq V$ that covers high value topics; i.e., we 
and $h(s)=\log (1+s)$ is a non-decreasing concave function. %(e.g., , $h(s)=s^\alpha$, where $\alpha<1$, etc.). 
Intuitively, the concavity of $h$ suppresses the selection of similar tokens (corresponding to the same subject), even if they have high value, thereby promoting diversity.
% Objective \eqref{eq:smob} is clearly of form \eqref{eq:objFunc}. For example, for $h=\log(1+s)$, $f$ is monotone and submodular  \cite{lin2011class}, and is the sum of  compositions of $h$ with multilinear functions $g_z(\vc{x})=\sum_{i\in V} r_{i, z} x_{i},$ as illustrated in Tab.~\ref{table: problems}. Moreover, 
 Functions $f_z$ (and, thereby, also $f$) are monotone and submodular, and we can construct polynomial approximators $\hat{f}^L_z$ for them as indicated in Table~\ref{table: problems} by replacing $h$ with its %Taylor approximation
 %$h$ is analytic and can be approximated within arbitrary accuracy by its 
 $L^{\text{th}}$-order Taylor approximation around 1/2, given by:
\begin{equation} \label{eq: taylor_f_iL}
    \hat{h}^{L}(s) = \textstyle\sum_{\ell = 0}^L \frac{h^{(\ell)}(1/2)}{\ell!} (s - {1}/{2})^\ell.
\end{equation}
This is because the composition of polynomial $\hat{f}^L_z$ with  polynomial  $g_z$ in Table~\ref{table: problems} is again a polynomial. 
We show in \fullversion{\cite{ozcan2021submodular}}{App.~\ref{app: proof_bias_bound} \deleted{of the supplement}} that this estimator ensures that $f$ indeed satisfies Asm.~\ref{asmp:boundedPoly}. Moreover,  %$\epsilon_L(\vc{y})$ of this estimator $\widehat{\nabla G(\vc{y})}$ is bounded.
the estimator bias \emph{decays exponentially} with degree $L$ (see Tab.~\ref{table: problems} and App.~\ref{app:customBias}), meaning that polynomial number of terms suffice to reduce the bias to a desired level. %:
 %Therefore, it satisfies Asm.~\ref{asmp: mon_sub}.  Using $\texttt{supp}(\vc{x}_S)$, instead of $S$ as discussed in Section \ref{sec:tech}, we rewrite the diversity reward function as:
%\begin{equation} \label{eq: diversity_reward}
%    f(\vc{x}) = \sum_{j=1}^M \log\left(\sum_{i\in P_j} r_i x_{i} + 1\right)
%\end{equation}
%where $x_{i}=1$, if $i\in S$ and $x_{i}=0$ otherwise. It is straightforward to see that for $g_j(\vc{x})\equiv\sum_{i\in P_j}r_i x_{i}$, $g_j(\vc{x})$ is multilinear and for $h_j(s) \equiv \log(1 + s)$, $h_j(s)$ is non-decreasing and concave. Hence (\ref{eq: diversity_reward}) satisfies Asm.~\ref{asmp: mon_sub}. 
%Additionally, for the polynomial estimator %$\hat{h}_{L}(s)$ described in Eq.~\ref{eq: taylor_f_iL}, $h_j(s)$ and $\hat{h}_{L}(s)$ satisfy Asm.~\ref{asmp: f_isInForm}. According to this, we can use the $L^{th}$ Taylor polynomial of $h_j(s)$ around $1/2$ as its polynomial estimator $\hat{h}_{L}(s)$. %and use this polynomial estimator in (\ref{eq: poly_estimator}) to find a polynomial estimator of $G(\vc{y})$. 
%Furthermore, we can show that the estimator bias $\epsilon(L)$ of this estimator $\widehat{\nabla G(\vc{y})}$ is bounded.
%Our work directly allows for the optimization of such objectives over matroid constraints. For example, 
A partition matroid could be used with this objective to enforce that no more than $k_\ell$ sentences come from $\ell$-th user, etc. 
% \begin{lemma}
% Consider a diversity reward function $f: [0, 1]^d \rightarrow \reals_+$ which is $L + 1$ differentiable and for which the Taylor expansion exists at $\mathcal{1}/2$. Then, the gradient of the multilinear relaxation can be approximated by,
% \begin{align*}
%     \nabla G_i(\mathbf{y}) \approx \mathbb{E}[\hat{f}_L(x) | x_i = 1] - \mathbb{E}[\hat{f}_L(x) | x_i = 0]
% \end{align*}
% where $\hat{f}_L(x) = \sum_{l = 0}^L \frac{f^{(l)}(1/2)}{l!} (x - \frac{1}{2})^l$ and the error of the approximation is $\frac{D}{L \cdot 2^L}$ where $D = \max_{\mathbf{v} \in \mathcal{D}} \|\mathbf{v}\|_2$ and L is the order of the Taylor approximation.
% \end{lemma}
% \begin{proof}
% \begin{align*} 
%  \big\lvert f_i(g_i(x)) - \hat{f_L}(g_i(x)) \big\rvert = \Bigg\lvert \frac{f_i^{(L+1)}(s'_i)}{(L+1)!} \bigg(g_i(x)-\frac{1}{2}\bigg)^{L+1} \Bigg\rvert
% \end{align*}
% by the \textit{Lagrange remainder theorem}, where $s'_i$ is between $1/2$ and $g_i(x)$. For $f_i(g_i(x)) = log(1 + g_i(x))$, $\frac{d^{L+1}f_i(g_i(x))}{dg_i^{L+1}}$ is $(-1)^L \frac{L!}{(1+g_i(x))^{L+1}}$. Then, 
% \begin{align*}
%  \big\lvert f_i(g_i(x)) - \hat{f_L}(g_i(x)) \big\rvert = \Bigg\rvert \frac{\big(g_i(x)-\frac{1}{2}\big)^{L+1}}{L\big(1+s'_i\big)^{L+1}} \Bigg\lvert
% \end{align*}
% For $g_i(x), s_i' \in [0, 1]$,
% \begin{align*}
%  \big\lvert f_i(g_i(x)) - \hat{f_L}(g_i(x)) \big\rvert \leq \frac{\frac{1}{2}^{L+1}}{L\big(1+s'_i\big)^{L+1}} \leq \frac{1}{L \cdot 2^{L+1}}
% \end{align*}
% where $\lim_{L\to \infty} \frac{1}{L \cdot 2^{L+1}} = 0$.
% Then by Asm.~\ref{asmp: f_iL_exists}, Taylor approximation gives an approximation guarantee for maximizing diversity reward function where the error of the approximation is given by Lem.~\ref{lem: gradientBias} as
% \begin{align*}
%     &\epsilon_{i, L}(y) = \mathbb{E}_y[R_L(x) \mid x_i = 1] + \mathbb{E}_y[R_L(x) \mid x_i = 0]\\
%     &=  \mathbb{E}_y[\sum_{i \in \mathcal{D}} |R_{i,L}(g_i(x))| \mid x_i = 1] + \mathbb{E}_y[\sum_{i \in \mathcal{D}} |R_{i,L}(g_i(x))| \mid x_i = 0] \\
%     &\text{by Lem.~\ref{lem: gradientBias}}\\
%     &\leq \sum_{i \in \mathcal{D}} \mathbb{E}_y[|R_{i,L}(g_i(x))| \mid x_i = 1] + \sum_{i \in \mathcal{D}} \mathbb{E}_y[|R_{i,L}(g_i(x))| \mid x_i = 0]\\
%     &= \sum_{i \in \mathcal{D}} \mathbb{E}_y\Biggl[\Biggl|\frac{1}{L\bigl(2+2g_i(x)\bigl)^{L+1}}\Biggr| \mid x_i = 1\Biggr] \\
%     &+ \sum_{i \in \mathcal{D}} \mathbb{E}_y\Biggl[\Biggl|\frac{1}{L\bigl(2+2g_i(x)\bigr)^{L+1}}\Biggr| \mid x_i = 0\Biggr]\\
%     &\leq 2 \sum_{i \in \mathcal{D}}\frac{1}{L\big(2+2g_i(x)\big)^{L+1}} \leq \frac{D}{L \cdot 2^L}
% \end{align*} 
% \end{proof}
 \subsection{Influence Maximization (IM) \cite{kempe2003maximizing, chen2009efficient}} \label{sec: IM}
Given a directed graph $G = (V, E)$, we wish to maximize the expected fraction of nodes reached if we infect a set of nodes $S\subseteq V$ and the infection spreads via, e.g., the Independent Cascade (IC) model \cite{kempe2003maximizing}. Adding a concave utility to the fraction can enhance the value of nodes reached in  early stages.  
Formally, let $z$ can be a random simulation trace of the IC model, and $P_v^z\subseteq V$ is the set of nodes reachable from $v$ in a random simulation of the IC model. Then, the objective can be written as $f(\mathbf{x}) = \mathbb{E}_{z \sim P} [f_z(\mathbf{x})]$ where
%\begin{align}f(\vc{x}) = \textstyle\mathbb{E}_{z \sim P} \left[\frac{1}{N}\sum_{v \in V}\left(1 - \prod_{i \in P_v^z}(1-x_i)\right)\right],\!\!\! \end{align}
%where $P_v^z\subseteq V$ is the set of nodes reachable from $v$ in a random simulation of the IC model. %This is 
%defines a distribution $\mathcal{G}$ over $M$ instances $j \sim \mathcal{G}$ that share a set $V$ of nodes. The influence $g_j(S)$ of a set of nodes $S$ in instance $j$ is the fraction of nodes reachable from $S$ using the edges $E(j)$. Using $\texttt{supp}(\vc{x}_S)$, influence of $j$ defined by \cite{Karimi2017} as the following:
%This is a multilinear function. Our approach allows us to extend this  to maximizing the expectation of \emph{analytic functions} $h$ of the fraction of infected nodes. For example, 
%For a particular random instant $z$ of the IC model and for $h(s)=\log (1+s)$, we can express:
% \begin{equation} \label{eq: inf_max_z}
    $f_z(\vc{x}) =\textstyle h\left(g_z(\vc{x}) \right),$ %,
% \end{equation}
$h(s)=\log (1+s)$, and
%\begin{equation} \label{eq: IM_gz}
    $g_z(\vc{x}) =\textstyle \sum_{v \in V}\frac{1}{N}\big(1 - \prod_{i \in P_v^z}(1-x_i)\big)$ is the number of infected nodes under seed set  $\mathbf{x}$.  
%\end{equation}
%As a result our objective function becomes 
%\begin{equation} \label{eq:inf_max}
%    f(\mathbf{x}) = \mathbb{E}_{z \sim P} [f_z(\mathbf{x})].
%\end{equation}
%Definition of $g_j(\vc{x})$ in (\ref{eq: IM_gi}) is clearly multilinear and proven to be monotone and submodular \cite{krause2014submodular}, \cite{Karimi2017}. 
Since functions $g_z:[0,1]^N\to [0,1]$ are multilinear, monotone submodular and $h:[0,1]\to\reals$  is non-decreasing and concave, %Therefore, (\ref{eq: inf_max}) satisfies Asm.~\ref{asmp: mon_sub}. 
%As a result,
$f$ satisfies Asm.~\ref{asm:monSub} \cite{ozcan2021submodular}. %Moreover, 
Again, we can construct $\hat{f}^L$ by replacing $h$ by $\hat{h}^L$, given by Eq.~\eqref{eq: taylor_f_iL}.
%\begin{equation} \label{eq: taylor_f_iL}
%    \hat{h}_{L}(s) = \sum_{\ell = 0}^L \frac{h^{(\ell)}(1/2)}{\ell!} (s - {1}/{2})^\ell
%\end{equation}
%We show in App.~\ref{app: proof_bias_bound_IM} that this estimator 
This again ensures that $f$ indeed satisfies Asm.~\ref{asmp:boundedPoly}, and %$\epsilon_L(\vc{y})$ of this estimator $\widehat{\nabla G(\vc{y})}$ is bounded.
 the estimator bias again decays exponentially %as follows:
%The proof \deleted{of the theorem }can be found 
(see Tab.~\ref{table: problems}  and \fullversion{\cite{ozcan2021submodular}}{App.~\ref{app:customBias}}). Partition matroid constraints could be used in this setting to bound the number of seeds from some group (e.g., males/females, people in a zip code, etc.).

\subsection{Facility Location (FL)\cite{mokhtari2018conditional}}
%, cornuejols1977location, krause2014submodular}.} 
Given a weighted bipartite graph $G = (V \cup V_z)$ and weights $w_{i, z} \in [0, 1]$, $ i \in V$, $ z \in V_z$, we wish to maximize:
\begin{equation}\label{eq:originalFL}
    f(S) = \textstyle\mathbb{E}_{z \sim P} \left[h(\max_{i \in S} w_{i, z})\right], 
\end{equation}
where $h(s)=\log(1+s)$.
Intuitively, $V$ and $V'$ represent facilities and customers respectively and $w_{v, v'}$ is the utility of facility $v$ for customer $v'$. The goal is to select a subset of facility locations $S\subset{V}$ to maximize the total utility, assuming every customer chooses the facility with the highest utility in the selection $S$; again, adding the concave function $h$ adds diversity, favoring the satisfaction of customers that are not already covered. This too becomes a coverage problem 
%In order to reformulate $\max_{i \in S} w_{i, j}$ using the binary support vector $\vc{x}$ of $S$, we 
by observing that \cite{karimi2017stochastic}:
%\begin{equation}
    $\max_{i \in S} w_{i, z}  = \sum\limits_{\ell=1}^{n}(w_{i_\ell, z}-w_{i_{\ell+1}, z})\big(1-\prod\limits_{k=1}^\ell(1-x_{i_k})\big),$
%\end{equation}
where,
 for a given $z \in V_z$, weights have been pre-sorted in a descending order as $w_{i_1,z} \geq \ldots \geq w_{i_n,z}$. %where $i_1 \in V$ is the facility providing the maximum utility to customer $j$ and $i_n \in V$ is the facility providing the minimum utility to customer $j$. Then we rewrite $\max_{i \in S} w_{i, j}$ as
and %$l_j^{(i)} = w_{j,i}$ for each $i \in M$ and $j \in V %= \{1, \ldots, n\}$, with 
$w_{i_{n+1},j} \triangleq 0$. %We can again extend this problem to maximizing analytic functions $h$ of the utility of a user. For example, for $h(s)=\log(1+s)$, we can maximize 
%\begin{equation} \label{eq:FLlog}
%    f_z(\mathbf{x}) = \textstyle \log\left(1+g_z(\vc{x}) \right).
%\end{equation}
In a manner similar to Sec~\ref{sec: IM}, we can show that this function again satisfies Asm.~\ref{asm:monSub} and~\ref{asmp:boundedPoly}, using again the $L^{\text{th}}$-order Taylor approximation of $h$, given by Eq.~\eqref{eq: taylor_f_iL}; this will again lead to a bias that decays exponentially (see Tab.~\ref{table: problems} and App.~\ref{app:customBias}). 
We can again optimize such an objective over arbitrary matroids, which can enforce, e.g., that no more than $k$ facilities are selected from a geographic area or some other partition of $V$.%bounded by \eqref{eq:eLbound}.
% \begin{theorem} \label{thm: epsilon_bound_FL} \textcolor{red}{Practically the same with Theorem 6.1} Assume a facility location function $f:~\{0,1\}^{N} \rightarrow \reals_+$ that is given by (\ref{eq:FLlog}). Then, consider Algorithm \ref{alg:cont-greed} in which $\nabla G(\mathbf{y}_K)$ is estimated via the polynomial estimator given in (\ref{eq: poly_estimator}) where $\hat{f}_{L}(\vc{x})$ is the $L^{th}$ Taylor polynomial of $f(\vc{x})$ around $1/2$. Then, the bias of the estimator is bounded by 
% \begin{equation*}
%     \| \epsilon_L \|_2 \leq \frac{\sqrt{N}}{(L+1) 2^{L}}
% \end{equation*}
% where the bias is the $l_2$ norm of the residual error vector and $N$ is the number of facilities. \end{theorem}
% The proof of the theorem is the same as the proof of Thm.~ \ref{thm: epsilon_bound_IM} and can be found in App. \ref{app: proof_bias_bound_IM}.
%, set $V=[N]$ of facility locations to serve a set $[M]$ of customers. Each facility $i \in [N]$ provides a service value $M_{ij}$ to each customer $j\in [M]$.  The goal is to select a subset of  facility locations to maximize the total service value, assuming every customer chooses the facility with the highest service value.
             %Given a complete\\
             %weighted bipartite graph\\
             % ;\\
             %, and $w_{x,y}$ \\
             %represent  and utilities\\
             %respectively. \\
             %and $m_{n+1} = 0$.
%Formally, we wish to choose a set $S\subseteq V$ to maximize:
% where $f(\emptyset)=0$. If $M_{ij} \geq 0$ for all $i, j$, then $f(S)$ is monotone submodular \cite{frieze1974cost}.  
% Karimi et. al. \cite{Karimi2017} suggested the following  reformulation for  the facility location problem \eqref{eq:originalFL}:
% \begin{align}\label{eq:FL}
%      f(S) &= \sum_{j=1}^M g_j(\vc{x}_S),
% \end{align}
% and the functions $g_j:[0,1]^N \to \reals_+$ are defined as follows
% \begin{align*}
%     g_j(\vc{x}) = \sum_{i=1}^{N}(m^{(j)}_i-m^{(j)}_{i+1})\Big(1-\prod_{k=1}^i(1-x_k)\Big)
% \end{align*}
% where $m^{(j)}_i=M_{lj}$ with $m_{N+1}=0 \leq m^{(j)}_N\leq \ldots \leq  m^{(j)}_1.$ We see that \eqref{eq:FL} has the form \eqref{eq:hatW}, where, $\mathcal{J} = [N] \times [M],$  $\beta_{(i,j)} = m^{(j)}_i-m^{(j)}_{i+1},$ and $S_{(i,j)} = [i].$
% which as a result of Lem.~\ref{lem:compwmnf} is monotone and submodular. 
% \begin{align*}
%     F(\mathbf{x}) = \mathbb{E}_S [f(S)] = \frac{1}{|Y|} \sum_{y \in Y} \log\left(f_y(\mathbf{x}) + 1 \right),
% \end{align*}





\section{Experiments}
  We evaluate Alg.~\ref{alg: SCG}%the Stochastic Continuous Greedy (SCG) algorithm
  , %described in \ref{alg: SCG}, 
  with sampling and polynomial estimators over two well-known problem instances (influence maximization and facility location%, and data summarization
  ) with real and synthetic %different graph settings
  datasets. We summarize these setups in Tab.~\ref{tab:datasets}. For a more detailed overview of the datasets and experiment parameters, please refer to App.~\ref{app:exps}\deleted{of the supplement}. Our code \replaced{is publicly accessible}{will be public once the submission is reviewed}.\footnote{\url{https://github.com/neu-spiral/StochSubMax}}
  
\begin{wraptable}{r}{6cm}
% \begin{table}[t]
\vspace*{-25pt}
\begin{center}
    \begin{tabular}{|c|c|ccc|cc|}
    \hline
    \thead{instance} & \thead{dataset} & \thead{$|z|$} & \thead{$|S|$} & \thead{$|E|$} & \thead{m} & \thead{k} \\%& \thead{$f^*$}\\
    \hline
    % \thead{IM} & \texttt{GreedyTricker} & 1 & 12 & 13 & 2 & 1 \\%& 0.6\\
    \thead{IM} & \texttt{SBPL} & 20 & 400 & 914 & 4 & 1 \\%& 0.06\\
    % \thead{IM} & \texttt{SyntheticBipartiteUniform} & 100 & 200 & 400 & 4 & 2 & 0.35\\
    \thead{IM} & \texttt{ZKC} & 20 & 34 & 78 & 2 & 3 \\%& -\\
    \thead{FL} & \texttt{MovieLens} & 4000 & 6041 & 256 & 10 & 2 \\%& -\\
    % \thead{SM} & \texttt{MovieLens} & - & - & - & - & - & -\\
    % \thead{SM} & \texttt{Twitter} & - & 42104 & - & 30 & 2 & -\\
    \hline
    \end{tabular}
    %\vspace*{-10pt}
\caption{{Datasets and Experiment Parameters.}}\label{tab:datasets}\end{center}
\vspace*{-25pt}
% \end{table}
\end{wraptable}
 
\noindent\textbf{Algorithms.} We compare the performance of different estimators. These estimators are: (a) sampling estimator (SAMP) with $N = 1, 10, 20, 100$ %, 1000$
and (b) polynomial estimator (POLY) with $L = 1, 2%, 3
$. %We also vary the number of iterations $T$ of the SCG algorithm where $T = 100, 200, 500, 1000, 2000$.

% \begin{table*}[t] \label{tab:final_estimates}
% \caption{Obtained utilities under different estimators}
% \resizebox{\textwidth}{!}{
%     \centering
%     \begin{tabular}{l r r r r r r} 
%     \hline
%     $\texttt{dataset}$ & SAMP1 & SAMP10 & SAMP20 & SAMP100 & POLY1 & POLY2\\% & \multicolumn{2}{c}{POLY3} \\
%     \hline
%     % \texttt{GreedyTricker} & 0.538 & 0.557 & 0.553 & 0.546 & \textbf{0.571} & 0.546 \\% & - & - \\
%     \hline
%     \texttt{SyntheticBipartitePowerLaw} & 0.049 & 0.049 & 0.049 & - & 0.061 & \textbf{0.062} \\
%     \hline
%     \texttt{ZKC} & 0.324 & 0.326 & 0.324 & 0.327 & 0.318 & \textbf{0.332} \\
%     \hline
%     \texttt{MovieLens} & 0.031 & 0.031 & 0.031 & 0.031 & \textbf{0.051} & - \\
%     \hline
% \end{tabular}}
% \end{table*}

\noindent\textbf{Metrics.} We evaluate the performance of the estimators with their clock running time and via %$f^*$, where $f^* = \max f(\mathbf{y})$ is 
the maximum result ($\max f(\mathbf{y})$) obtained using the best available estimator for a given setting.

\noindent\textbf{Results.} The trajectory of the utility obtained at each iteration of the stochastic continuous greey algorithm $f(\mathbf{y})$ is plotted as a function of time in Fig.~\ref{fig:CGiters}. %In Fig.~\ref{fig:GreedyTricker_loglog} POLY1 outperforms other estimators including the sampling estimators in terms of utility. Moreover, POLY1 is more than $10$ times faster than SAMP20 while it runs in comparable time to SAMP1. 
In Fig.~\ref{fig:SBPL_loglog}, we observe that polynomial estimators outperforms sampling estimators in terms of utility. Moreover, POLY1 runs $10$ times faster than SAMP20 and runs in comparable time to SAMP1. In Fig.~\ref{fig:ZKC_loglog}, POLY2 outperforms all estimators whereas POLY1 slightly underperforms. Finally, in Fig.~\ref{fig:MovieLens_loglog} we observe that POLY1 consistently outperforms sampling estimators.

The final outcomes of the objective functions of the estimators are reported as a function of time in Fig.~\ref{fig:final_estimates}. In Fig.~\ref{fig:SBPL_paretolog} and~\ref{fig:ZKC_paretolog}, POLY2 outperforms other estimators in terms of utility. Again in Fig.~\ref{fig:SBPL_paretolog}, POLY1 outperforms sampling estimators in terms of utility and runs in comparable time to SAMP1 while in Fig.~\ref{fig:MovieLens_paretolog}, POLY1 outperforms sampling estimators both in terms of time and utility. %Highest objective value is highlighted for each example. Based on this table, we can conclude that the polynomial estimators are better choices than the sampling estimators.
Ideally, we would expect the performance of the estimators to improve as the degree of the polynomial or the number of samples increase. The examples where this is not always the case can be explained by the stochastic nature of the problem.


\begin{figure}[t]
\centering
% \subfigure[\texttt{GreedyTricker}]{
% \begin{minipage}{0.46\linewidth}
% \centering
% \includegraphics[width=1\linewidth]{images/GreedyTricker_logtime.eps}
% \centering
% \label{fig:GreedyTricker_loglog}\vspace*{-10pt}
% \end{minipage}
% }
\subfigure[\texttt{SyntheticBipartitePowerLaw}]{
\begin{minipage}{0.31\linewidth}
\centering
\includegraphics[width=1\linewidth]{images/SyntheticBipartitePowerLaw_logtime.eps}
\centering
\label{fig:SBPL_loglog}\vspace*{-10pt}
\end{minipage}
}
% \subfigure[\texttt{SyntheticBipartiteUniform}]{
% \begin{minipage}{0.45\linewidth}
% \centering
% \includegraphics[width=1\linewidth]{images/RB100_uniform_100_100_400_k_2_100_FW_logtime.eps}
% \label{fig:FLsynth1_loglog}\vspace*{-10pt}
% \end{minipage}
% }
\subfigure[\texttt{ZKC}]{
\begin{minipage}{0.31\linewidth}
\centering
\includegraphics[width=1\linewidth]{images/zkc_logtime.eps}
\label{fig:ZKC_loglog}\vspace*{-10pt}
\end{minipage}
}
\subfigure[\texttt{MovieLens}]{
\begin{minipage}{0.31\linewidth}
\centering
\includegraphics[width=1\linewidth]{images/MovieLens_logtime.eps}
\centering
\label{fig:MovieLens_loglog}
\end{minipage}
}
% \subfigure[\texttt{Twitter}]{
% \begin{minipage}{0.45\linewidth}
% \centering
% \includegraphics[width=1\linewidth]{images/emptyplot.png}
% \label{fig:6}
% \end{minipage}
% }
\vspace*{-10pt}
\caption{Trajectory of the FW algorithm. Utility of the function at the current $\vc{y}$ as a function of time is marked for every %$10$th 
iteration.} 
 \vspace*{-13pt}
\label{fig:CGiters}
\end{figure}

\begin{figure}[t]
\centering
% \subfigure[\texttt{GreedyTricker}]{
% \begin{minipage}{0.45\linewidth}
% \centering
% \includegraphics[width=1\linewidth]{images/IM_fooler_bipartite_N_5_k_1_100_FW_paretolog.eps}
% \label{fig:IMsynth1_paretolog}\vspace*{-12pt}
% \end{minipage}
% }
\subfigure[\texttt{SBPL}]{
\begin{minipage}{0.30\linewidth}
\centering
\includegraphics[width=1\linewidth]{images/IM_RB20powerlaw_200_200_914_k_1_100_FW_paretolog.eps}
\label{fig:SBPL_paretolog}\vspace*{-12pt}
\end{minipage}
}
% \subfigure[\texttt{SyntheticBipartiteUniform}]{
% \begin{minipage}{0.45\linewidth}
% \centering
% \includegraphics[width=1\linewidth]{images/RB100uniform_100_100_400_k_2_100_FW_paretolog.eps}
% \label{fig:FLsynth1_paretolog}\vspace*{-12pt}
% \end{minipage}
% }
\subfigure[\texttt{ZKC}]{
\begin{minipage}{0.30\linewidth}
\centering
\includegraphics[width=1\linewidth]{images/ZKC_paretolog.eps}
\label{fig:ZKC_paretolog}\vspace*{-12pt}
\end{minipage}
}
\subfigure[\texttt{MovieLens}]{
\begin{minipage}{0.30\linewidth}
\centering
\includegraphics[width=1\linewidth]{images/MovieLens_paretolog.eps}
\label{fig:MovieLens_paretolog}\vspace*{-12pt}
\end{minipage}
}
% \subfigure[\texttt{Twitter}]{
% \begin{minipage}{0.45\linewidth}
% \centering
% \includegraphics[width=1\linewidth]{images/emptyplot.png}
% \label{fig:SMsynth1_paretolog}\vspace*{-12pt}
% \end{minipage}
% }
\vspace*{-10pt}
\caption{Comparison of different estimators on different problems. Blue lines represent the performance of the POLY estimators and the marked points correspond to POLY1 and POLY2 %, POLY3 
respectively. Orange lines represent the performance of the SAMP estimators and the marked points correspond to SAMP1, SAMP10, SAMP20, SAMP100 respectively.}
%\vspace*{-15pt}
\label{fig:final_estimates}
\end{figure}

\section{Conclusions}
This work presented a simulation approach that centers around finding iteratively an approximation of the evolution of the algebraic variables in the power system \glspl{DAE}. The approximation of the dynamic state evolutions by NNs, instead of classical explicit numerical integration schemes, allows larger time-steps to be realized while being fast to execute. This work aimed at providing a proof of concept, it is foreseeable that future work on this method shares many typical questions with established \gls{DAE} solvers, hence, by applying various existing techniques the computational performance and scalability of the approach should improve significantly.

% \section*{Acknowledgment}
% This work is supported by NSF grant CCF-1750539.


% \bibliography{09_references}
% \bibliographystyle{splncs04}

%%
%% The next two lines define the bibliography style to be used, and
%% the bibliography file.
\bibliographystyle{splncs04}
\bibliography{08_ref}

\newpage
\section*{Appendix}
\addcontentsline{toc}{section}{Appendices}
\renewcommand{\thesubsection}{\Alph{subsection}}
\appendix

\section{Descriptions of $\matattn{}$, $\matff{}$ and $\matln{}$}
\label{sec:app_submodule_skip_description}

Here we detail the definitions of the mappings $\matattnl{}$, $\matffl{}$ and $\matlnl{}$ utilized in \S\ref{sec:submodules}.

\paragraph{Description of $\matattnl{}$.}
%Illustrating this on $\texttt{attn}_\ell$ for definiteness,
For an input $s$, let $v^\ell_{i_s}$ be the vector at position $i_s$ in the output of $\texttt{attn}_\ell (\texttt{ln1}_\ell (H^{\ell - 1}))$. We denote by $A_\ell^{\texttt{attn}} \in \mathbb{R}^{d_h \times d_h}$ the matrix numerically minimizing 
$$ A \mapsto \sum_{s \in \mathcal{T}} || A \cdot \texttt{ln1}_\ell (h^{\ell-1}_{i_s}) - v^\ell_{i_s}||^2,$$
and define an attention sub-module replacement (Eq.~\ref{eq:attn}) by $$
\texttt{b}^{\overline{\texttt{attn}}}_\ell (h) \coloneqq A_{\ell}^{\texttt{attn}} \cdot \texttt{ln1}_\ell (h) + h. $$
We then define a mapping between two layers ${\ell \rightarrow \ell'}$ by:
$$ \matattnl{} (h) \coloneqq $$
$$ \texttt{b}^{\texttt{ffn}}_{\ell'} ( \texttt{b}^{\overline{\texttt{attn}}}_{\ell'} ( \ldots (\texttt{b}^{\texttt{ffn}}_{\ell+1} ( \texttt{b}^{\overline{\texttt{attn}}}_{\ell+1} (h)))\ldots)).$$ 
Namely, when applying each $\ell''$-th block, $\ell < \ell'' \leq \ell'$, we replace its attention sub-module $\texttt{attn}_{\ell''}$ by its linear approximation.
%In an analogous way, we consider the mappings $\matffl{}$ and $\matlnl{}$, where in the latter we perform the linear shortcut both for \texttt{ln1} and for \texttt{ln2} (see~\S\ref{sec:app_submodule_skip_description} for precise descriptions).
Importantly, unlike the original attention module, the approximation $\texttt{b}^{\overline{\texttt{attn}}}_\ell$ operates on each position independently, and therefore applying $\matattnl{}$ disables any contextualization between the layers $\ell$ and $\ell'$. Note that this is not the case for $\matffl{}$ and $\matlnl{}$, which retain the self-attention sub-modules and operate contextually.

\paragraph{Description of $\matffl{}$.}
Let $v^\ell_{i_s}$ be the vector at position $i_s$ in the output of $\texttt{ln2}_{\ell} (\texttt{b}_\ell^{\texttt{attn}} (H^{\ell - 1}))$, for a given input $s$. We denote by $A_\ell^{\texttt{ffn}} \in \mathbb{R}^{d_h \times d_h}$ the matrix numerically minimizing 
$$ A \mapsto \sum_{s \in \mathcal{T}} || A \cdot v^{\ell}_{i_s} - \texttt{ffn}_{\ell} (v^\ell_{i_s})||^2,$$
and define a replacement of the feed-forward sub-module $\texttt{b}_{\ell}^{\texttt{ffn}}$ by $$ \texttt{b}^{\overline{\texttt{ffn}}}_\ell (H) \coloneqq A_{\ell}^{\texttt{ffn}} \cdot \texttt{ln2}_\ell (H) + H.$$
We then define a mapping between two layers ${\ell \rightarrow \ell'}$ by:
$$ \matffl{} (H) \coloneqq $$
$$ \texttt{b}^{\overline{\texttt{ffn}}}_{\ell'} ( \texttt{b}^{\texttt{attn}}_{\ell'} ( \ldots (\texttt{b}^{\overline{\texttt{ffn}}}_{\ell+1} ( \texttt{b}^{\texttt{attn}}_{\ell+1} (H))\ldots)).$$

\paragraph{Description of $\matlnl{}$.}
Let $v^\ell_{i_s}$ be the vector at position $i_s$ in the output of $\texttt{b}^{\texttt{attn}}_{\ell} (H^{\ell - 1})$, for a given input $s$. We denote by $A_\ell^{\texttt{ln1}} \in \mathbb{R}^{d_h \times d_h}$ the matrix numerically minimizing 
$$ A \mapsto \sum_{s \in \mathcal{T}} || A \cdot h^{\ell}_{i_s} - \texttt{ln1}_{\ell} (h^\ell_{i_s})||^2$$ and we denote by $A_\ell^{\texttt{ln2}} \in \mathbb{R}^{d_h \times d_h}$ the matrix numerically minimizing $$ A \mapsto \sum_{s \in \mathcal{T}} || A \cdot v^{\ell}_{i_s} - \texttt{ln2}_{\ell} (v^\ell_{i_s})||^2.$$ We define a replacement of the block $\texttt{b}^{\texttt{attn}}_{\ell}$ by \begin{equation} \texttt{b}^{\overline{\texttt{ln1}}}_\ell (H) \coloneqq \texttt{attn}_{\ell} (A_{\ell}^{\texttt{ln1}} \cdot H) + H\end{equation} and we define a replacement of the block $\texttt{b}^{\texttt{ffn}}_{\ell}$ by \begin{equation} \texttt{b}^{\overline{\texttt{ln2}}}_\ell (H) \coloneqq \texttt{ffn}_{\ell} (A_{\ell}^{\texttt{ln2}} \cdot H) + H.\end{equation}
We then define a mapping between two layers ${\ell \rightarrow \ell'}$ by:
$$ \matlnl{} (H) \coloneqq $$
$$ \texttt{b}^{\overline{\texttt{ln2}}}_{\ell'} ( \texttt{b}^{\overline{\texttt{ln1}}}_{\ell'} ( \ldots (\texttt{b}^{\overline{\texttt{ln2}}}_{\ell+1} ( \texttt{b}^{\overline{\texttt{ln1}}}_{\ell+1} (H))\ldots)).$$

% that's all folks
\end{document}


