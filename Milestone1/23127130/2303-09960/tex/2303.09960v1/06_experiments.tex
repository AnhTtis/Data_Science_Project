  We evaluate Alg.~\ref{alg: SCG}%the Stochastic Continuous Greedy (SCG) algorithm
  , %described in \ref{alg: SCG}, 
  with sampling and polynomial estimators over two well-known problem instances (influence maximization and facility location%, and data summarization
  ) with real and synthetic %different graph settings
  datasets. We summarize these setups in Tab.~\ref{tab:datasets}. For a more detailed overview of the datasets and experiment parameters, please refer to App.~\ref{app:exps}\deleted{of the supplement}. Our code \replaced{is publicly accessible}{will be public once the submission is reviewed}.\footnote{\url{https://github.com/neu-spiral/StochSubMax}}
  
\begin{wraptable}{r}{6cm}
% \begin{table}[t]
\vspace*{-25pt}
\begin{center}
    \begin{tabular}{|c|c|ccc|cc|}
    \hline
    \thead{instance} & \thead{dataset} & \thead{$|z|$} & \thead{$|S|$} & \thead{$|E|$} & \thead{m} & \thead{k} \\%& \thead{$f^*$}\\
    \hline
    % \thead{IM} & \texttt{GreedyTricker} & 1 & 12 & 13 & 2 & 1 \\%& 0.6\\
    \thead{IM} & \texttt{SBPL} & 20 & 400 & 914 & 4 & 1 \\%& 0.06\\
    % \thead{IM} & \texttt{SyntheticBipartiteUniform} & 100 & 200 & 400 & 4 & 2 & 0.35\\
    \thead{IM} & \texttt{ZKC} & 20 & 34 & 78 & 2 & 3 \\%& -\\
    \thead{FL} & \texttt{MovieLens} & 4000 & 6041 & 256 & 10 & 2 \\%& -\\
    % \thead{SM} & \texttt{MovieLens} & - & - & - & - & - & -\\
    % \thead{SM} & \texttt{Twitter} & - & 42104 & - & 30 & 2 & -\\
    \hline
    \end{tabular}
    %\vspace*{-10pt}
\caption{{Datasets and Experiment Parameters.}}\label{tab:datasets}\end{center}
\vspace*{-25pt}
% \end{table}
\end{wraptable}
 
\noindent\textbf{Algorithms.} We compare the performance of different estimators. These estimators are: (a) sampling estimator (SAMP) with $N = 1, 10, 20, 100$ %, 1000$
and (b) polynomial estimator (POLY) with $L = 1, 2%, 3
$. %We also vary the number of iterations $T$ of the SCG algorithm where $T = 100, 200, 500, 1000, 2000$.

% \begin{table*}[t] \label{tab:final_estimates}
% \caption{Obtained utilities under different estimators}
% \resizebox{\textwidth}{!}{
%     \centering
%     \begin{tabular}{l r r r r r r} 
%     \hline
%     $\texttt{dataset}$ & SAMP1 & SAMP10 & SAMP20 & SAMP100 & POLY1 & POLY2\\% & \multicolumn{2}{c}{POLY3} \\
%     \hline
%     % \texttt{GreedyTricker} & 0.538 & 0.557 & 0.553 & 0.546 & \textbf{0.571} & 0.546 \\% & - & - \\
%     \hline
%     \texttt{SyntheticBipartitePowerLaw} & 0.049 & 0.049 & 0.049 & - & 0.061 & \textbf{0.062} \\
%     \hline
%     \texttt{ZKC} & 0.324 & 0.326 & 0.324 & 0.327 & 0.318 & \textbf{0.332} \\
%     \hline
%     \texttt{MovieLens} & 0.031 & 0.031 & 0.031 & 0.031 & \textbf{0.051} & - \\
%     \hline
% \end{tabular}}
% \end{table*}

\noindent\textbf{Metrics.} We evaluate the performance of the estimators with their clock running time and via %$f^*$, where $f^* = \max f(\mathbf{y})$ is 
the maximum result ($\max f(\mathbf{y})$) obtained using the best available estimator for a given setting.

\noindent\textbf{Results.} The trajectory of the utility obtained at each iteration of the stochastic continuous greey algorithm $f(\mathbf{y})$ is plotted as a function of time in Fig.~\ref{fig:CGiters}. %In Fig.~\ref{fig:GreedyTricker_loglog} POLY1 outperforms other estimators including the sampling estimators in terms of utility. Moreover, POLY1 is more than $10$ times faster than SAMP20 while it runs in comparable time to SAMP1. 
In Fig.~\ref{fig:SBPL_loglog}, we observe that polynomial estimators outperforms sampling estimators in terms of utility. Moreover, POLY1 runs $10$ times faster than SAMP20 and runs in comparable time to SAMP1. In Fig.~\ref{fig:ZKC_loglog}, POLY2 outperforms all estimators whereas POLY1 slightly underperforms. Finally, in Fig.~\ref{fig:MovieLens_loglog} we observe that POLY1 consistently outperforms sampling estimators.

The final outcomes of the objective functions of the estimators are reported as a function of time in Fig.~\ref{fig:final_estimates}. In Fig.~\ref{fig:SBPL_paretolog} and~\ref{fig:ZKC_paretolog}, POLY2 outperforms other estimators in terms of utility. Again in Fig.~\ref{fig:SBPL_paretolog}, POLY1 outperforms sampling estimators in terms of utility and runs in comparable time to SAMP1 while in Fig.~\ref{fig:MovieLens_paretolog}, POLY1 outperforms sampling estimators both in terms of time and utility. %Highest objective value is highlighted for each example. Based on this table, we can conclude that the polynomial estimators are better choices than the sampling estimators.
Ideally, we would expect the performance of the estimators to improve as the degree of the polynomial or the number of samples increase. The examples where this is not always the case can be explained by the stochastic nature of the problem.


\begin{figure}[t]
\centering
% \subfigure[\texttt{GreedyTricker}]{
% \begin{minipage}{0.46\linewidth}
% \centering
% \includegraphics[width=1\linewidth]{images/GreedyTricker_logtime.eps}
% \centering
% \label{fig:GreedyTricker_loglog}\vspace*{-10pt}
% \end{minipage}
% }
\subfigure[\texttt{SyntheticBipartitePowerLaw}]{
\begin{minipage}{0.31\linewidth}
\centering
\includegraphics[width=1\linewidth]{images/SyntheticBipartitePowerLaw_logtime.eps}
\centering
\label{fig:SBPL_loglog}\vspace*{-10pt}
\end{minipage}
}
% \subfigure[\texttt{SyntheticBipartiteUniform}]{
% \begin{minipage}{0.45\linewidth}
% \centering
% \includegraphics[width=1\linewidth]{images/RB100_uniform_100_100_400_k_2_100_FW_logtime.eps}
% \label{fig:FLsynth1_loglog}\vspace*{-10pt}
% \end{minipage}
% }
\subfigure[\texttt{ZKC}]{
\begin{minipage}{0.31\linewidth}
\centering
\includegraphics[width=1\linewidth]{images/zkc_logtime.eps}
\label{fig:ZKC_loglog}\vspace*{-10pt}
\end{minipage}
}
\subfigure[\texttt{MovieLens}]{
\begin{minipage}{0.31\linewidth}
\centering
\includegraphics[width=1\linewidth]{images/MovieLens_logtime.eps}
\centering
\label{fig:MovieLens_loglog}
\end{minipage}
}
% \subfigure[\texttt{Twitter}]{
% \begin{minipage}{0.45\linewidth}
% \centering
% \includegraphics[width=1\linewidth]{images/emptyplot.png}
% \label{fig:6}
% \end{minipage}
% }
\vspace*{-10pt}
\caption{Trajectory of the FW algorithm. Utility of the function at the current $\vc{y}$ as a function of time is marked for every %$10$th 
iteration.} 
 \vspace*{-13pt}
\label{fig:CGiters}
\end{figure}

\begin{figure}[t]
\centering
% \subfigure[\texttt{GreedyTricker}]{
% \begin{minipage}{0.45\linewidth}
% \centering
% \includegraphics[width=1\linewidth]{images/IM_fooler_bipartite_N_5_k_1_100_FW_paretolog.eps}
% \label{fig:IMsynth1_paretolog}\vspace*{-12pt}
% \end{minipage}
% }
\subfigure[\texttt{SBPL}]{
\begin{minipage}{0.30\linewidth}
\centering
\includegraphics[width=1\linewidth]{images/IM_RB20powerlaw_200_200_914_k_1_100_FW_paretolog.eps}
\label{fig:SBPL_paretolog}\vspace*{-12pt}
\end{minipage}
}
% \subfigure[\texttt{SyntheticBipartiteUniform}]{
% \begin{minipage}{0.45\linewidth}
% \centering
% \includegraphics[width=1\linewidth]{images/RB100uniform_100_100_400_k_2_100_FW_paretolog.eps}
% \label{fig:FLsynth1_paretolog}\vspace*{-12pt}
% \end{minipage}
% }
\subfigure[\texttt{ZKC}]{
\begin{minipage}{0.30\linewidth}
\centering
\includegraphics[width=1\linewidth]{images/ZKC_paretolog.eps}
\label{fig:ZKC_paretolog}\vspace*{-12pt}
\end{minipage}
}
\subfigure[\texttt{MovieLens}]{
\begin{minipage}{0.30\linewidth}
\centering
\includegraphics[width=1\linewidth]{images/MovieLens_paretolog.eps}
\label{fig:MovieLens_paretolog}\vspace*{-12pt}
\end{minipage}
}
% \subfigure[\texttt{Twitter}]{
% \begin{minipage}{0.45\linewidth}
% \centering
% \includegraphics[width=1\linewidth]{images/emptyplot.png}
% \label{fig:SMsynth1_paretolog}\vspace*{-12pt}
% \end{minipage}
% }
\vspace*{-10pt}
\caption{Comparison of different estimators on different problems. Blue lines represent the performance of the POLY estimators and the marked points correspond to POLY1 and POLY2 %, POLY3 
respectively. Orange lines represent the performance of the SAMP estimators and the marked points correspond to SAMP1, SAMP10, SAMP20, SAMP100 respectively.}
%\vspace*{-15pt}
\label{fig:final_estimates}
\end{figure}