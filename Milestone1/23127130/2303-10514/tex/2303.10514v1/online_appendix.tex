
\section*{Online Appendix for "A Group Public Goods Game with Position Uncertainty": mixed strategy equilibria with asymmetric groups}
\begin{proposition}
\begin{enumerate}
\item \textbf{Asymmetric with plural groups.}\label{thm:mixedStratAsym} If $n_j>1$ for all $j\leq b$ then there exists a value $r^\sharp<N$ so that for all values $r>r^\sharp$ there exist two distinct values $\gamma^1_r,\gamma^2_r \in(0,1)$ where, for all $i\in I$, the profiles of play

\begin{align*}
    \sigma_{i,1}^*(C\mid \zeta) =
\begin{cases}
1, & \zeta \in\{(0,0), (1,n)\}\\
\gamma^1_r, & \text{otherwise},
\end{cases}
\end{align*}

\begin{align*}
    \sigma_{i,2}^*(C\mid \zeta) =
\begin{cases}
1, & \zeta \in\{(0,0), (1,n)\}\\
\gamma^2_r, & \text{otherwise}
\end{cases}
\end{align*}
 establish two distinct sequential equilibria $(\sigma^*_1, \mu^*_1)$ and $(\sigma^*_1, \mu^*_2)$, respectively.
\item \textbf{Asymmetric with a singular group.}\label{thm:mixedStratAsym2} If $n_j=1$ for some $j\leq b$ then there exists a value $r^\sharp<N$ so that for all values $r>r^\sharp$ there exist one $\gamma^* \in(0,1)$ where, for all $i\in I$, the profile of play
\begin{align*}
    \sigma_{i}^*(C\mid \zeta) =
\begin{cases}
1, & \zeta \in\{(0,0), (1,n)\}\\
\gamma^*, & \text{otherwise},
\end{cases}
\end{align*}

 establish a sequential equilibrium $(\sigma^*, \mu^*)$.
\end{enumerate}
\end{proposition}

\begin{proof}
    

As before, let's assume that the standard strategy given profile $\zeta$ is
\begin{align*}
    \sigma^*(C\mid \zeta) =
\begin{cases}
1, & (0,0), (1,n_k)\\
\gamma, & \text{otherwise},
\end{cases}
\end{align*}
where $1\leq k < b$. Observe a player in group $t>1$ must be aware of the size of group $t-1$ in order to be able to detect a defection. Fix a profile $\overline{\zeta}$ and set for player $j$ in group $t$ the strategies
\[
\sigma_j^C(\zeta)
=
\begin{cases}
\sigma^*_j(\zeta), & \zeta \not = \overline{\zeta}\\
1, & \zeta = \overline{\zeta}.
\end{cases}
\]

\[
\sigma_j^D(\zeta)
=
\begin{cases}
\sigma^*_j(\zeta), & \zeta \not = \overline{\zeta}\\
0, & \zeta = \overline{\zeta},
\end{cases}
\]
and $\mu^D$ and $\mu^C$ as their corresponding beliefs. Let $\phi_t(\gamma)$ denote the expected additional contribution from contributing rather than defecting. Clearly,
\[
\phi_t(\gamma) = \sum_{i=t}^{b} E_{\mu^C}(G_{i}\mid \zeta = \overline{\zeta}, Q(j) = t) -   E_{\mu^D}(G_{i}\mid \zeta = \overline{\zeta}, Q(j) = t).
\]


As with Lemma~1 in the main paper, the form of $\phi_t(\gamma)$ depends on the type of sample player $j$ receives. Set $\Omega_i(\overline{\zeta})= E_{\mu^C}(G_{i}\mid \zeta = \overline{\zeta}, Q(j) = t) -   E_{\mu^D}(G_{i}\mid \zeta = \overline{\zeta}, Q(j)= t)$. If the sample $\overline{\zeta} = (1,n')$ contains a defection - $n_{t-1}>n'$ - we have that
\[
E_{\mu^D}(G_{t+k}\mid \zeta = \overline{\zeta}, Q(j)= t) = n_{t+k}\left[\sum_{i=1}^{k=1} \gamma^{n_{t+i}}\prod_{j=1}^{i-1}\left(1- \gamma^{n_{t+j}}\right) + \gamma\prod_{i=1}^{k-1}\left(1-\gamma^{n_{t+i}}\right)\right]
\]
\[
E_{\mu^C}(G_{t+k}\mid \zeta = \overline{\zeta}, Q(j)= t) = n_{t+k}\gamma^{n_t-1} + (1-\gamma^{n_t-1})E_{\mu^D}(G_{t+k}\mid \zeta = \overline{\zeta}, Q(j)= t).
\]
and thus
\[
 \Omega_{t+k}(\overline{\zeta}) = n_{t+k}\gamma^{n_t-1} \left(1-\frac{E_{\mu^D}(G_{t+k}\mid \zeta = \overline{\zeta}, Q(j)= t)}{n_{t+k}}\right).
\]
Whereas if $\overline{\zeta}= (1,n_{t-1})$ then 
\[
\Omega_{t+k}(\overline{\zeta}) = n_{t+k}\left(1-\frac{E_{\mu^D}(G_{t+k}\mid \zeta = \overline{\zeta}, Q(j)= t)}{n_{t+k}}\right).
\]

Evidently, $\Omega_i(\overline{\zeta})$ is larger when $\overline{\zeta}$ contains no defections. Computing $\phi_t(\gamma) = \sum_{i=t}^{b} \Omega_i(\overline{\zeta})$ presents an onerous task regardless of the sample $\overline{\zeta}$. What we do instead is bound each $\Omega_i(\overline{\zeta})$ between two computationally simpler functions. As the following lemma suggests, it suffices to focus on doing so for $\phi(\gamma)$ when $\overline{\zeta}$ contains a defection. As before, let $\psi_t(\gamma)$ represent the likelihood  of being in group $t$ after observing a defection and $\phi_t(\gamma, \overline{\zeta})$ denote $\phi_t(\gamma)$ on sample $\overline{\zeta})$.

\begin{lemma} Given profiles $\overline{\zeta}' = (1,n')$ with $n'<n_{t-1}$ and $\overline{\zeta} = (1,n_{t-1})$ it follows that for all $\gamma \in [0,1]$

\[
\phi_1(\gamma) > \sum_{t=2}^b\frac{1}{b-1}\phi_t(\gamma, \overline{\zeta}) \geq \sum_{t=2}^b \psi_t(\gamma)\phi_t(\gamma,\overline{\zeta}').
\]
\end{lemma}
\begin{proof} The proof of this Lemma is almost identical to that of Lemma~2 of the main paper except that
\[
\psi_t(\gamma) = \frac{\sum_{i=1}^{t-1}\left(1 + \prod_{j=t-i}^{t-1} (1-\gamma^{n_i}) \right)}{\sum_{k=2}^{b}\sum_{i=1}^{k-1}\left(1 + \prod_{j=t-i}^{t-1} (1-\gamma^{n_i}) \right)}
\]
\end{proof}
Let $M  =\max\{n_1,\ldots, n_b\}$ and $\lambda = \min\{n_1,\ldots, n_b\}$. One can easily verify that if $\overline{\zeta}$ contains a defection
\[
     \gamma^{M-1} + \left(\gamma^{\lambda-1} - \gamma^M\right)\left(1-\gamma^M\right)^{k-1} - \gamma^{\lambda-1} \leq \frac{\Omega_i}{n_i} \leq \gamma^{\lambda-1} + \left(\gamma^{M-1}- \gamma^\lambda\right)\left(1-\gamma^\lambda\right)^{k-1} - \gamma^{M-1}
\]

\noindent
for any $i$. In fact, $M = \lambda$ describes the symmetric scenario and $M =\lambda = 1$ is that of Monz\'{o}n and Gallice..

Setting
\[
\phi_t^{\bot}(\gamma) = \sum_{i=t+1}^b \lambda\left(\gamma^{M-1} + \left(\gamma^{\lambda-1} - \gamma^M\right)\left(1-\gamma^M\right)^{k-1} - \gamma^{\lambda-1}\right)+ 1
\]
and
\[\phi_t^{\top}(\gamma) =\sum_{i=t+1}^b M\left(\gamma^{\lambda-1} + \left(\gamma^{M-1}- \gamma^\lambda\right)\left(1-\gamma^\lambda\right)^{k-1} - \gamma^{M-1}\right) + 1
\]
we get that 
 \[
 \phi_t^{\bot}(\gamma) \leq \phi_t(\gamma) \leq \phi_t^{\top}(\gamma).
 \]
  As before set 
   \[
 \Delta(\gamma) = \frac{r}{N}\sum_{t=2}^b\psi_t(\gamma)\phi_t(\gamma) - 1.
 \]
  If $M,\lambda \geq 2$ it follows that
 \[
 \lim_{\gamma \to 0} \phi_t^{\bot}(\gamma) = \lim_{\gamma \to 1} \phi_t^{\bot}(\gamma) = \lim_{\gamma \to 0} \phi_t^{\top}(\gamma) = \lim_{\gamma \to 1} \phi_t^{\top}(\gamma) = 1
 \]
 forces
 \[
 \lim_{\gamma \to 0} \phi_t(\gamma) = \lim_{\gamma \to 1} \phi_t(\gamma) = 1
 \]
 by a pinching $\phi_t(\gamma)$ between its bounds. Therefore, we get that for all $\gamma \in (0,1)$: $\Delta(\gamma) > \frac{r}{N} -1 = \Delta(1)  = \Delta(0)<0$ for all $r<N$. We can apply the same arguments used in the proofs of Propositions~2 and~3 of the main paper to derive claimed result for the case where $M,\lambda \geq 2$. Finally for $M > \lambda = 1$ it follows that 
   \[
 \lim_{\gamma \to 0} \phi_t^{\bot}(\gamma) = \lim_{\gamma \to 1} \phi_t^{\bot}(\gamma) = 1,\lim_{\gamma \to 0} \phi_t^{\top}(\gamma) = 2  \text{ and }  \lim_{\gamma \to 1} \phi_t^{\top}(\gamma) = 1.
 \]
 Observe that regardless of the value of $\Delta(0)$ (i.e., whether it's positive or negative) since $\Delta(1) < 0$, and $\Delta(\gamma)>0$ for $r=N$ and any $\gamma \in (0,1)$ there must exist value of $r^\sharp$ with a corresponding value $\gamma^\sharp \in (0,1)$ so that $\Delta(r^\sharp,\gamma^\sharp)=0$.

\end{proof}
