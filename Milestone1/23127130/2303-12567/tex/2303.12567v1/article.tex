\documentclass[12pt,twoside]{article}
\pdfoutput=1 
\usepackage[a4paper,hmargin={1.0in,1.0in},top=1.0in,bottom=1.0in]{geometry}
\usepackage{setspace} 
\setstretch{1.05}

\usepackage[T1]{fontenc}
\usepackage[utf8]{inputenc}  

\usepackage{libertine} %!!!
\widowpenalty=10000
\clubpenalty=10000  

\usepackage[scale=1.0,libertine,vvarbb]{newtxmath} %!!! 
\usepackage[cal=boondox, calscaled=1.0]{mathalpha} %!!!

\usepackage{graphicx} 
\usepackage{ntheorem}   

\usepackage{sansmath}

\usepackage{mathtools,amsmath}
\usepackage{natbib} 
\usepackage{hypernat}

\usepackage[labelfont={bf,footnotesize},textfont={footnotesize,singlespacing}]{caption}

\usepackage{footmisc}
\renewcommand{\footnotelayout}{\hspace{0.15em}\setstretch{1.0}}% footnotes even more stretched
                                % than onehalfspacing
 


\usepackage{subcaption}
\usepackage[colorlinks,citecolor=blue,linkcolor=blue,anchorcolor=blue,urlcolor=blue]{hyperref}
\usepackage[toc]{appendix} 

\theorembodyfont{\upshape}
\theorempreskip{0.0pt plus 1.0pt}
\theorempostskip{0.0pt plus 1.0pt}

\newcounter{nitnum} % not used 

\def\nitskip{\vskip 1.75pt plus 0.5pt minus 0.5pt}


\numberwithin{paragraph}{section}
\renewcommand{\theparagraph}{\textbf{\arabic{section}.\arabic{paragraph}}}  

\newenvironment{nit}[1]% environment name
{% begin code
\removelastskip\nitskip\par\noindent\refstepcounter{paragraph}%
\textbf{\theparagraph~{#1}:\enspace}\ignorespaces%
}%
{% end code
\ignorespacesafterend\nitskip
} 

\input mac-1.tex
\input mac-2.tex  
\input mac-3.tex

\input ushyphex.tex

%%%%%% local macros 

\newif\iftoshow
\def\toshow#1{\iftoshow #1\else\medskip\relax\fi}
%\toshowfalse
\toshowtrue 



\def\Item#1#2{\par\nobreak\hangindent#1pt\hangafter0\noindent
\llap{#2\enspace}\ignorespaces}

\def\cint{{\lbkt 0,\bar c\,]}}
\def\sd{{\pi}}
\def\pdg{{{}^\dag}}
\def\pst{{{}^{\raise0.7pt\hbox{$\scriptstyle*$}}}}
%\def\csd{{\goth f}}%
\def\csd{{F}}%
\def\pfc{{\TT}}
\def\csdg{{\goth g}}%
\def\csdh{{\goth h}}%
\def\csdp{\pdg\csd_{y}}%
\def\csds{\pst\csd_{y}}%

%\def\csd{\lambda}%
\def\Lmp{\mathit{\Phi}}%
\let\DB\bbF
\def\Up{{\partial U}}
\let\T\varTheta


\def\times{\tts{\hbox{\fontencoding{TS1}\fontfamily{Cochineal-LF}\selectfont\char214}}\tts}
\def\half{{\frac12}}
\def\qedsymb{{$\bulletS$}}   %!!!!!  
\def\qed{\ifmmode\nobreak\hbox{\quad\qedsymb}\else\nobreak\relax\nobreak\hbox{\quad\qedsymb}\fi}
\let\swtc\small
\def\SWTC{\singlespacing\small\relax} 
\let\esw\qed 


\def\aut#1{{\smc #1}.}
\def\btitle#1{{\it  #1\/}.}
\def\jtitle#1{{\rm #1.}} 
\def\anno#1{({\rm #1}).}
\def\book#1{{\rm #1}}
\def\journal#1 #2 #3{{\it #1\/}~{\bf #2} #3}
\def\journalnt#1 #2 #3{{\it #1\/} {\bf #2} {#3}}
\bibskip=0pt plus 1pt minus 1pt

\input mac-5.tex 
%  
\numberwithin{equation}{section}
\theoremstyle{plain}  

\abovedisplayskip=5pt plus 1.5pt minus 1.5pt%
\belowdisplayskip=5pt plus 1.5pt minus 1.5pt%
\belowdisplayshortskip=3pt plus 1.5pt minus 1.5pt%


\title{%%%%%%%%%%
Dynamic Transportation of Economic Agents
\thanks{Supersedes the paper ``Another look at the distribution of income and wealth in
the macroeconomy.''}
}%%% 

\author{Andrew Lyasoff\ts\thanks{Boston University,email: alyasoff@bu.edu, Git repository:
\href{https://github.com/AndrewLyasoff}{https://github.com/AndrewLyasoff}}
                }

\date{\today} 



\usepackage{titlesec}
\titleformat{\section}[hang]{\center\bf\large}{\llap{\bf\large\thesection.\hbox to0.5em{\hfill}}}{0.0em}{}
\titleformat{\figure}[hang]{\normalsize\rmshape}{\textsc{\thefigure}}{0.5em}{}
\titlespacing*{\section}{0pt}{.85ex plus 0.125ex minus .125ex}{0.75ex  plus 0.125ex minus 0.125ex}

\usepackage{titling}

\setlength{\droptitle}{-6.5em}

\renewcommand{\abstractname}{\vspace{-1.5cm}}
\footnotesep=0.75\baselineskip

\usepackage{fancyhdr}
\renewcommand{\headrulewidth}{0pt}%
\fancypagestyle{FirstPage}{
\fancyhf{}% clear all header and footer fields
\fancyfoot{}
\fancyfoot[C]{\thepage} % except the center
\renewcommand{\headrulewidth}{0pt}%
\renewcommand{\footrulewidth}{0pt}%
\fancyfoot[L]{} 
}
\pagestyle{fancy}
\headheight=5.2pt 
\headheight=14.49998pt
\pagestyle{fancy}  
\fancyhead{}
\fancyhead[CE]{\small ANDREW LYASOFF}
\fancyhead[CO]{\small DYNAMIC TRANSPORTATION OF ECONOMIC AGENTS} 
\fancyhead[LE,RO]{\thepage}
\fancyfoot{}
  
\begin{document}

\hyphenation{Cam-bridge non-homo-ge-ne-ous non-ran-dom non-ne-ga-tive non-li-near
non-de-ge-ne-rate non-tri-vial non-in-te-ger non-ato-mic non-in-cre-asing
non-de-creas-ing non-measu-rable di-men-sion attri-bute non-tri-vial
non-con-ti-nu-ous mo-del squa-red mo-ni-ker in-dis-tin-guish-able
in-dis-tin-guish-abi-lity bre-vity dis-co-ve-red pro-po-si-tion
pre-fe-rable Wie-ner Mar-kov Kol-mo-go-rov in-dis-tin-gui-sha-bi-li-ty
ul-ti-ma-tely Ike-da Wa-ta-na-be Shir-yaev non-ran-dom con-clu-si-on
Uk-ra-i-n-ian math-e-ma-ti-cian in-te-grands in-te-grand in-te-gra-tors
in-te-gra-tor quad-ra-tic cle-arly gro-und-bre-a-king pub-li-ca-ti-ons
Le-bes-gue to-po-lo-gy to-po-lo-gi-cal in-ter-pre-ta-tion res-pect ge-ne-ric
Hil-bert uni-que-ness Kara-tzas Sprin-ger Ver-lag sub-mar-tin-gale
mar-tin-gale ortho-normal Wahr-sche-in-lich-ke-its-the-o-rie mea-sure
ope-ra-ti-ons car-di-na-lity non-empty prob-a-bil-i-ty either tech-nical
par-ti-cular fil-tra-tion in-e-qua-li-ty mar-tin-ga-les squ-are squa-re
Equa-ti-ons  sto-chas-ti-que adop-ted sto-chas-tic ave-ra-ging exer-cised
con-tin-u-ous op-er-a-tions Brown-ian Mole-kul-arbe-we-gung other-wise
con-tin-u-ous-time con-tin-gent-claims depen-ding Schi-ed pre-dic-ta-ble
Stro-ock Shi-rya-ev boun-ded semi-mar-tin-gale mono-tone Orn-stein
Uhlen-beck mar-gi-nal dis-tri-bu-tion Grund-be-grif-fe der regu-lari-ty
Wahr-schein-lich-keits-rech-nung Cara-theo-dory pseu-do-met-ric semi-norm
in-te-gra-ble in-teg-ra-ble de-creas-ing se-mi-mar-tin-gale covar-i-a-tion var-i-a-tion
in-fi-ni-te-si-mal in-fi-ni-te-si-mally sum-mable never-the-less illu-sion mea-ning
pro-duct inde-pend-ence inde-pend-ent de-pend-ing Wer-ner con-ver-gence clear-ly Clear-ly
Gene-ral-ly gene-ral gene-ra-lized ite-ra-tion ite-ra-tions ano-ther every
giant admit admits eco-no-my pri-cing modi-fi-ca-tion Spe-ci-fi-cally
spe-ci-fi-cally con-si-de-ra-tions pro-perty tri-vial na-tu-ral bound-ed
def-i-ni-tion ul-ti-mate-ly ge-ner-ic ap-pro-xi-ma-tion pro-cess pro-ces-ses
me-cha-nics no-mi-nal stu-died tra-ding sha-dow bra-cket sta-ted
im-me-di-ately stra-tegy How-ever how-ever Ein-stein Novi-kov opti-o-nal announ-cing
sub-mar-tin-gale sup-er-mar-tin-gale clo-sed qua-si-mar-tin-gale boun-ded domi-na-ted
loca-lizing seve-ral tech-ni-cal ope-ra-tor adap-ted ob-ser-va-tion dis-con-ti-nu-ous
eva-ne-scent iden-ti-cally pro-ba-bi-lity càg-làd càd-làg Cha-pi-t-res
sto-cha-s-ti-ques asso-ci-ated gene-ric po-pu-la-tion be-ha-vioreq-ui-lib-rium clea-ring
in-te-rest-ingly sub-stan-ti-ally si-mul-ta-neously in-fi-ni-tely mi-mi-cking
in-vest-ment in-evi-tably know-ledge in-di-vi-dual dis-pa-ri-ty
be-ha-vior pre-com-pu-ted ave-ra-ges ave-rage sa-vings dy-na-mics take
takers de-ter-mi-ni-stic op-ti-ma-li-ty dua-li-ty simu-la-ted
mac-ro-eco-no-mists mac-ro-eco-no-mist mac-ro-eco-no-mics Lag-range
hol-ding hol-dings rea-so-na-bly gi-ven Gi-ven ar-bi-tra-rily mo-del mo-dels Ljung-qvist Sar-gent
sol-ving meta-pro-gram meta-prog-ram no-ted in-stal-led
de-li-be-ra-tely le-vels dis-co-very hete-ro-ge-neous eco-no-mics} 

\lefthyphenmin=3
\righthyphenmin=3


\maketitle

\allowdisplaybreaks

%%%%%%%%%%%%%%%%%%%%%%%%%%%%%%
%
% change here 
%
%%%%%%%%%%%%%%%%%%%%%%%%%%%%%
\thispagestyle{FirstPage}

\begin{abstract}\normalsize\noindent 
\noindent%
%
%
{\noindent%
The paper was prompted by the surprising discovery that the common strategy,
adopted in a large body of research,
for producing macroeconomic equilibrium
in certain hete\-ro\-ge\-neous-agent incomplete-market models 
fails to locate the equilibrium in a widely cited benchmark study.
%Consequently, a large body of research, some widely cited, based on this class of models may
%provide results that are entirely specious. 
By~mimicking the approach proposed by
\citel{DL12}, the paper provides a novel description of the
law of motion of the distribution over the range of private states
of a large population of interacting economic agents faced with
uninsurable aggregate and idiosyncratic risk.
A new algorithm for identifying the returns, the optimal private allocations, and the population transport
in the state of equilibrium is developed  
and is tested in two well known benchmark studies.
%Asymptotic properties of models with shared risk are~{studied}.  
\medskip

\noindent
\textsc{Keywords:} transport problems,
general equilibrium, incomplete markets, heterogeneous agent mo\-dels, numerical methods
%\medskip
%
%\noindent
%\textsc{JEL Classification:}  D52, C68, D58, E21 
%%%%%%%%%%%%%%%%
}
\end{abstract} 

%%lblinit 
%{Sec: #}--%%label

%%- -%%--{Sec: #1}--%%
%{Eq: 0.}   %%llabel
%{No: 1.}   %%nlabel

%% main text
\section{Background and Introduction}
\label{sec:Intro}\setcounter{paragraph}{0}

\parskip=0.0pt

%{{{

{\parskip=0pt
\noindent
%During any time period its state is captured
%by the stochastic productivity shock (if any) and the
%distribution of agents over the space of private states.
The time evolution of
an economy populated by a very large number of infinitely lived heterogeneous 
economic agents (alias: households)
can be seen as evolution of 
unit-mass distribution over the range of private states, perhaps accompanied by an exogenous
evolution of the productivity shock, if production technology is involved.
Thus, the transition of the distribution of agents from one period to the next parallels what is
known in mathematics as 
``the transport of mass problem''~-- see \citel{Vil09}. The transport of a large population of economic
agents, however, is radically different from that of a, say, pile of sand: while the laws of physics are
given to the grains of sand, the co-movements of economic agents, by way of setting prices,
trading, and consuming, come out of their own preferences and interactions. So to speak,
``the laws of physics'' are a collective decision for all
``particles'' to make.
This is a radically different paradigm yet to be fully understood and explored.
Notwithstanding recent efforts~-- see \citel{AHLLM17}, for example~-- 
to cast heterogeneous agent models in terms of the mathematics of optimal transport and mean field
games (MFG), this paper takes a different tack and develops from scratch
what may be described as ``standalone theory of
dynamic transport for a large population of interacting economic agents.'' The main objective of this section is to
motivate the need for such theory and to
provide some relevant connections, insights, and intuition.


There is a well known conundrum shared by all incomplete-market models: 
the individual savings %invest\-ment-con\-sum\-ption optimization 
problems cannot be solved until the security prices are fixed, which can be fixed only
after the
supply and the demand for securities are known, which become known only after all
individual savings problems %optimization 
are solved; i.e., the search for an optimal allocation by the agents takes place simultaneously 
with their search for a common optimization problem to solve.
%The standard workaround is some form of tâtonnement~-- i.e.,
%``trial-and-error'' procedure~-- over prices.
Another conundrum,
imposed by the notion of recursive competitive equilibrium, is that all individual decisions must produce
collective behavior that coincides with  the one the agents assume when making their
choices. %, whence the need for some form of tâtonnement also over the
%population distribution.
This forces
the agents to make predictions of their own positions and the
collective behavior in the future.
In the
context of general equilibrium models, the predictions provided by the
backward induction in the private Bellman equations alone are not
sufficient due to the inherent time
inconsistency%%%%%%%%%%%%%%%%%%%%%%%
%%%%%%%%%%%%%%%%%%%%%%%%%%%%%%%%%%%%%%%%%%%%
\footnote{This time inconsistency is essentially the one encountered when an optimal control
problem is stated in terms of the stochastic maximum principle:
%(see \protect\citel{Bis73}, \protect\citel{Peng90}):
%\protect\nonocite{DE92b}\protect\nocite{DE92a}{Duffie and Epstein (1992)}):
one must resolve simultaneously the backward dynamics of the shadow variables and the forward
dynamics of the state variables.
While a generic optimal control problem can be written in terms of the (time consistent)
Bellman equation, and without any reference to the shadow variables, equilibrium considerations
inevitably involve consumption, which is a functional transformation of the corresponding Arrow-Debreu
shadow price. Moreover, the Arrow-Debreu prices show up (explicitly or implicitly) in the pricing kernels,
i.e., the shadow variables and their backward in time dynamics are impossible to avoid.}~--
%%%%%%%%%%%%%%%%%%%%%%%%%%%%%%%%%%%%%%%%%%%  
see \citel{DL12}, hereafter \nocite{DL12}{[DL]}. 
Just as an example,
the demands for securities today depend on the realized returns tomorrow,
which depend on the prices today, which depend on the demands~today.
}
 
The computational difficulties associated with the intrinsic
backward-forward structure%%%%%%%%%%%%%%%%%%%%%%
%%%%%%%%%%%%%%%%%%%%%%%%%%%%%%%%%%%%%%%%%%%%%%%
\footnote{One must solve simultaneously an impossibly large system composed of all first-order and market
clearing conditions attached 
to all
time periods and to all states of the economy. This is the so called ``global method,'' which is
practical only with very few time periods~-- see \nocite{DL12}{[DL]}.}
%%%%%%%%%%%%%%%%%%%%%%%%%%%%%%%%%%%%%%%%%%%%%%%
no\-ted in the foregoing
appear to go away when the time horizon is pushed to infinity, as this makes
the distinction between ``forward'' and ``backward''  seemingly  moot.
While postulating infinite time horizon and
stationarity %, as a way around the intrinsic forward-backward structure,
simplifies the matters, it also ignores intrinsic connections and effects that 
do not necessarily vanish in the limit.
%, and may bring forth issues that %,
%to the best of this author's knowledge,
%have not been documented and addressed before.
The approach developed in this paper is based on the view that, instead of attempting to circumvent it,
one can face the forward-backward structure and the associated time inconsistency head-on, and 
interpret the infinite time horizon as a time horizon that is finite but arbitrarily large.
The passage to infinite time horizon invokes another
toolset from the domain of mathematics that has been employed
extensively since the pioneering works of \citel{Bew86}, \citel{Hug93} and \citel{Aiya94}, which
brought forth what is now known as the Bewley-Huggett-Aiyagari model (a.k.a.~``the
workhorse in macroeconomics'') and inspired the development of MFG.
Namely, the uniqueness of and the convergence to the
stationary distribution of a Markov process, a property governed by one variation or another of
Doeblin's condition~-- see \citel{Str14}. The idea is that, as a distribution of unit
mass, the cross-sectional distribution of the population is the same as the stationary probability
distribution of the individual optimal state, which obtains from the Fokker-Planck
(FP) equation attached to a generic household. One of the main objectives of this paper
is to do away with the need to stipulate that the state of every household is
``exactly distributed according to the state of the population'' (\citel{CDLL19}), %%%%%%%%%%%
%%%%%%%%%%%%%%%%%%%%%%%%%%%%%%%%%%%%%%%%%%%%
%\footnote{This phrase is taken from \citel{CDLL19}, whence the quotation marks.}
%%%%%%%%%%%%%%%%%%%%%%%%%%%%%%%%%%%%%%%%%%%%%
whether this feature takes place or not. 
In~fact, the probability distribution of the private state will never enter the model, and this
means that, just as the mathematics of optimal transport and mean field games (see above),
the mathematics of Doeblin's condition and the Fokker-Planck equation will not play a rôle in what
follows. What motivated this new approach is the realization that the classical, we shall
also call it Fokker-Planck (FP), approach %outlined above
may fail to produce the equilibrium
distribution~-- even in a well known (for a few decades) benchmark study meant to illustrate~its applicability.
This curious observation is outlined next.  


\iffalse %%%%%%%%%%%%% ##########################
because the problem at hand is  
not time consistent. mere averaging over future 
values, in particular, the expectation operation inside the Bellman equation, is not adequate for
such predictions.  he method
developed by \citel{DL12} (hereafter \nocite{DL12}{[DL]}) shows how such predictions can be made
by way of backward induction 


This forward-backward structure is not incompatible with the notion of ``recursive compe\-titive
equilibrium'' (a.k.a.\ ``rational expectations equilibrium''), but points to the main difficulty in  
the calculation of such equilibria: ultimately, one must map the present state into present
decisions and into future distribution of the population (the very meaning of recursive
equilibrium), except that the calculation of these mappings involves
certain forward-backward steps. These steps are more sophisticated than the mere averaging of
future values and capture the channels through which the agents
anticipate and react to future shocks. 
\fi %%%%%%%%%%%%%%%%%% &&&&&&&&&&&&&&&&&&&&&&&&&&

%%%%%%%%%%%%%%%%%%%%%%%%%%
%
\iffalse
The~classical (FP) startegy
for locating the equilibrium in heterogeneous models with infinite time horizon and no shared risk
comes down to making an 
ansatz choice for the prices, determine the corresponding optimal private
demand, compute the stationary probability distribution of the private state, and test 
whether the market clears. If 
it does not, then the ansatz choice for the prices is to be modified and the procedure is to be
repeated until market clearing is achieved.
%
\fi
%%%%%%%%%%%%%%%%%%%%%%%%%%%%%%%%%%%%%%%%%%%%
Consider  the familiar savings problem described, for example, in Sec.~18.2 in 
\citel{LjunSar00}, hereafter \nocite{LjunSar00}{[RMT]}.
Its~least involved version is the pure credit economy of \citel{Hug93}.
%~-- see Sec.~18.2.3 in \nocite{LjunSar00}{[RMT]}.
In it the agents have exogenous endowments, that
follow statistically identical but independent Markov chains, and trade a single riskless asset
that is in net supply of zero. The classical (FP) approach to producing the equilibrium
comes down to computing, for a given interest rate, the private optimal policy and the long run
distribution of agents over the space of private states, which is interpreted 
(Sec.~18.2.2 in \nocite{LjunSar00}{[RMT]}) as the
stationary probability distribution of the private state when the optimal policy is followed.
The choice of the interest rate, i.e.,
of the rate at which the agents agree to lend to one another, is then varied until the integral of the
(optimal) private policy function against the distribution of agents, i.e., against the long run probability
distribution of the optimal private state, becomes zero.  
%Hence, the aggregate demand for the bond equals zero if and only if the long-run expected
%private demand equals zero~-- see \nocite{LjunSar00}{ibid}.
%The~equilibrium rate is then the one with which the
%expected long-run private demand is zero.
This strategy for calculating the equilibrium rate in
Huggett's benchmark economy is illustrated in Figure~\ref{fg1}.%\citel{VHN20}
%%%%%%%%%%%%%%%%%%%%%%%%%%%%%%%%%%%%%%%%%%
%%   Fg: 1    15- in by 9.2 + 1/8 in
%%%%%%%%%%%%%%%%%%%%%%%%%%%%%%%%%%%%%%%%%%%%%%%%%%%%%%%%%%%%
{\captionsetup{belowskip=-10pt}
\captionsetup{aboveskip=0pt}
%%%%%%%%%%%%%%%% 
\begin{figure}[!htbp]
\centering 
\begin{subfigure}{.47\textwidth} 
  \centering
\leavevmode\raise1.2cm\hbox{\rotatebox{90}{\tiny expected private demand}}% 
\ %   
\toshow{\includegraphics[width=6.5cm]{fg1L}}%
%\llap{\raise0.5cm\hbox to4.0cm{\tiny 200 grid-points\hfill}}

\leavevmode\smash{\hbox to0.3cm{\hfill} \raise6pt\hbox{\tiny interest rate}}

%\leavevmode\smash{\hbox to0.5cm{\hfill} \raise6pt\hbox{\tiny aggregate asset holdings $a(r)$}}
  %\caption{A subfigure} 
%  \label{fig:sub1} 
\end{subfigure}\quad
\begin{subfigure}{.47\textwidth}  
  \centering 
\leavevmode\raise1.2cm\hbox{\rotatebox{90}{\tiny expected private demand}}%  
\ % 
\toshow{\includegraphics[width=6.5cm]{fg1R}}
%\llap{\raise0.5cm\hbox to2.2cm{\tiny 2,000 grid-points\hfill}} 

\leavevmode\smash{\hbox to0.3cm{\hfill} \raise6pt\hbox{\tiny interest rate}} 

%\leavevmode\smash{\hbox to0.5cm{\hfill} \raise6pt\hbox{\tiny consumption record}}
  %\caption{A subfigure}
%  \label{fg12} 
\end{subfigure}
\caption{Illustration of the strategy for calculating the equilibrium rate in a pure credit
economy with  asset holdings constrained to a grid of 200 equally spaced points (the right plot is
a microscopic view of a portion of the left).}
\label{fg1} 
\end{figure} }% 
%%%%%%%%%%%%%%%%%%%%%%%%%%%%%%%%%%%%%%%%%%%%%%%%%%%%%%%% 
The~parameter values are borrowed
from the first specification in Sec.~18.7 in \nocite{LjunSar00}{[RMT]} and so is also the program
for computing the expected long-run private demand for a given interest rate (the stationary
distribution of the private state obtains by iterating from the uniform distribution, until a fixed
point is attained, the FP  equation (18.2.4) in \nocite{LjunSar00}{[RMT]}). %%%%%%%%%%%%%
%%%%%%%%%%%%%%%%%%%%%%%%%%%%%%%%%%%%%%%%%%%%%%%%%%%%%%%%%%%%%%%%
\iffalse
%
\footnote{The~Julia program with which the plots in Figures~\ref{fg1} and \ref{fg2}
are generated emulates
the \hbox{MATLAB} program that accompanies \nocite{LjunSar00}{[RMT]}, except in the 
following step: %in the original MATLAB code the 
%iterations of the Bellman equation stop after the first repetition of the policy function,
%whereas in the program used here  
the iterations are terminated  after the first simultaneous repetition
of both the policy function and the value function (within a prescribed threshold), not just after
the first repetition of the policy function, as in the original source.
This modification %of the stopping criterion
is necessary because, due to the discretization
of the state space, the 
value function can still improve after the first repetition of the policy function,
if the former has not yet converged to its time-invariant state.
Just as in \nocite{LjunSar00}{[RMT]}, the stationary distribution of the private state
is obtained by iterating 
equation (18.2.4) in \nocite{LjunSar00}{[RMT]} (essentially,
a discrete analog of the FP equation for the
private state) from the uniform distribution.
The main reason for translating the original code into
Julia is the ability of the latter to handle large grid sizes.
}
% 
\fi 
%%%%%%%%%%%%%%%%%%%%%%%%%%%%%%%%%%%%%%%%%%%%%%%%%  
The~left plot shows the expected demands corresponding
to 20 different choices for the rate.
The~first three rates are chosen arbitrarily and every consecutive rate is the arithmetic average of
the latest rate that yields positive expected demand and the latest rate that yields negative expected
demand. The~right plot shows the expected demands in the last $10$ trials.
While the convergence of the inte\-rest rate is of order $10^{-8}$ (the distance between the last
two rates), there
appears to be a lower bound on how close to $0$ the expected demand can get. %, which is of order
%$10^{-2}$.
% and is not visible on the left plot due to the larger scale.
Interestingly, if the same experiment is repeated on a substantially 
more refined  grid over the same domain of asset holdings,
then the discontinuity in the expected demand as a function of the inte\-rest
becomes much more pronounced~-- see Figure~\ref{fg2}.%%%%%%%%%%%%%%%%%%% 
%%%%%%%%%%%%%%%%%%%%%%%%
%\footnote{Sec.~18.2 in \nocite{LjunSar00}{[RMT]}
%provides a detailed description of the choice of the domain and the calculation of the policy function
%and the long-run distribution on a discrete grid of asset holdings.}
%%%%%%%%%%%% 
%%  Fg: 2
%%%%%%%%%%%%%%%%%%%%%%%%%%%%%%%%%%%%%%%%%%%%%%%%%%%%%%%%%%%%  
{\captionsetup{belowskip=-10pt}
\captionsetup{aboveskip=0pt}
%%%%%%%%%%%%%%%% 
\begin{figure}[!htbp]
\centering 
\begin{subfigure}{.47\textwidth} 
  \centering
\leavevmode\raise1.2cm\hbox{\rotatebox{90}{\tiny expected private demand}}% 
\ %   
\toshow{\includegraphics[width=6.5cm]{fg2L}}%
%\llap{\raise0.5cm\hbox to4.0cm{\tiny 200 grid-points\hfill}}

\leavevmode\smash{\hbox to0.3cm{\hfill} \raise6pt\hbox{\tiny interest rate}}

%\leavevmode\smash{\hbox to0.5cm{\hfill} \raise6pt\hbox{\tiny aggregate asset holdings $a(r)$}}
  %\caption{A subfigure} 
%  \label{fig:sub1}
\end{subfigure}\quad
\begin{subfigure}{.47\textwidth}  
  \centering
\leavevmode\raise1.2cm\hbox{\rotatebox{90}{\tiny expected private demand}}%  
\ % 
\toshow{\includegraphics[width=6.5cm]{fg2R}}% 
%\llap{\raise0.5cm\hbox to2.2cm{\tiny 2,000 grid-points\hfill}} 

\leavevmode\smash{\hbox to0.3cm{\hfill} \raise6pt\hbox{\tiny interest rate}} 

%\leavevmode\smash{\hbox to0.5cm{\hfill} \raise6pt\hbox{\tiny consumption record}}
  %\caption{A subfigure}
%  \label{fg12}
\end{subfigure}
\caption{Illustration of the strategy for calculating the equilibrium rate in a pure credit
economy with asset holdings constrained to a grid of $2\mathord,000$ equally spaced points
(the right plot is 
a microscopic view of a portion of the left).}
\label{fg2}
\end{figure} }%
%%%%%%%%%%%%%%%%%%%%%%%%%%%%%%%%%%%%%%%%%%%%%%%%%%%%%%%%
The~culprit is illustrated in Figure~\ref{fg3}:
the last two in the list of 20 trial rates differ by less than $10^{-7}$ but
the corresponding stationary distributions are very different and so are
the respective expected demands, which differ by more than $3.25$~-- see the right plot in 
Figure~\ref{fg2}.
%%%%%%%%%%%%
%% Fg: 3
%%%%%%%%%%%%%%%%%%%%%%%%%%%%%%%%%%%%%%%%%%%%%%%%%%%%%%%%%%%% 
{\captionsetup{belowskip=-9pt} 
\captionsetup{aboveskip=0pt}
%%%%%%%%%%%%%%%% 
\begin{figure}[!htbp]
\centering 
\begin{subfigure}{.47\textwidth} 
  \centering
\leavevmode\raise1.2cm\hbox{\rotatebox{90}{\tiny population distribution}}% 
\ %   
\toshow{\includegraphics[width=6.5cm]{fg3}}%
%\llap{\raise0.5cm\hbox to4.0cm{\tiny 200 grid-points\hfill}}

\leavevmode\smash{\hbox to0.3cm{\hfill} \raise6pt\hbox{\tiny asset holdings ($2\mathord,000$ grid points)}}

%\leavevmode\smash{\hbox to0.5cm{\hfill} \raise6pt\hbox{\tiny aggregate asset holdings $a(r)$}}
  %\caption{A subfigure} 
%  \label{fig:sub1}
\end{subfigure}\quad
\begin{subfigure}{.47\textwidth}  
  \centering
\leavevmode\raise1.2cm\hbox{\rotatebox{90}{\tiny population distribution}}%  
\ % 
\toshow{\includegraphics[width=6.5cm]{fg3R}}%
%\llap{\raise0.5cm\hbox to2.2cm{\tiny 2,000 grid-points\hfill}} 

\leavevmode\smash{\hbox to0.3cm{\hfill} \raise6pt\hbox{\tiny asset holdings ($3\mathord,000$ grid points)}} 

%\leavevmode\smash{\hbox to0.5cm{\hfill} \raise6pt\hbox{\tiny consumption record}}
  %\caption{A subfigure}
%  \label{fg12}
\end{subfigure}
\caption{The stationary distributions in each of the seven employment categories calculated over
$2\mathord,000$ grid points (left) and $3\mathord,000$ grid points (right), for two interest rates
that differ by less than 
$10^{-7}$ (left) and $10^{-10}$ (right).%
\iffalse
There is one cumulative distribution function for every employment state, which gives
the distribution over the range of asset holdings (confined to 2,000 grid points) of all households
that happen to be in that employment state. The
stationary distribution of households over states of employment is the
stationary distribution obtained from the transition matrix for the idiosyncratic employment
state.
\fi%
}
\label{fg3}
\end{figure} }% 
%%%%%%%%%%%%%%%%%%%%%%%%%%%%%%%%%%%%%%%%%%%%%%%%%%%%%%%%
%%%%%%%%%%%%%%%%%%%%%%%%%%%%%%%%%%%%%%%%%%%%%%%%%%%%%%%%
Pushing the CPU to
$3\mathord,000$ grid points~-- see the right plot in Figure~\ref{fg3}~--
does not remove the discontinuity in the distribution. The rate at which the jump occurs
only moves slightly 
to the left as the density of the grid increases, but the gap in the expected demand remains larger
than $3$ even with $4\mathord,000$ grid points, with neither the left nor the right limit being
close enough to zero.
%(with $4\mathord,000$ grid points and convergence in the interest of
%order $10^{-8}$, the nearest expected demand from the right is above $0\mathord.94812$ and from the left it is below
%$-2\mathord.12419$).
As is well known, % result from \citel{SL89}
the failure of the stationary
distribution to depend continuously on a parameter, which is what Figure~\ref{fg3} illustrates,
implies either nonexistence or multiplicity of the stationary distribution for certain values of
that parameter.%%%%%%%%
%%%%%%%%%%%%%%%%%%%%%%%%%%%%%%%%%%%%%%
\footnote{The work \citel{Hug93} provides conditions under which such phenomena do not occur in the case of
two idiosyncratic states. While formulating these conditions, which essentially boil down to
certain monotonicity in the transition probabilities, for any number of states is
straightforward, they become less natural, and thus difficult to impose in general,
with more than two states. The transition probability matrix in the example considered here does
not satisfy such conditions.}
%%%%%%%%%%%%%%%%%%%%%%%%%%%%%%%%%%%%%
%%%%%%%%%%%%%%%%%%%%%%%%%%%%%%%%%%%%%
One must then note that, while equation (18.2.4) in \nocite{LjunSar00}{[RMT]} governs both the law of motion of the
cross-sectional distribution of the
population and the law of motion of the probability distribution of the private state (Sec.~18.2.2
in ibid.), such a connection could be useful only if the solution to this equation 
is unique.%%%%%%%%%%%%%%%%%%%%%%%%%%%%%%%%%%%%%%
%%%%%%%%%%%%%%%%%%%%%%%%%%%%%%%%%%%%%%%%%%%
\footnote{The FP equation (18.2.4) in \nocite{LjunSar00}{[RMT]} encodes connections that are in
force after the economy has attained, at least in part, its steady state regime (the interest rate is constant). Prior
to reaching that regime the interest rate depends on the distribution of households, and both
quantities are adjusting to their joint stationary configuration.
As~it stands, the structure of the FP equation does not capture these simultaneous adjustments. The theory of
MFG attempts to remedy this deficiency by allowing the private optimal state, and hence the FP
equation, to depend on the 
population distribution. However, in the context of heterogeneous agent models this dependence is not
known a priori, whence the need for the new approach developed in this paper. In sum, as the
time horizon increases, the population distribution converges to a distribution that satisfies a
particular FP equation, but unless that equation has a unique solution it cannot be used to
identify the true limiting distribution of the population. It~will be shown below, and this is the
main aspect of the method proposed in this paper, that
the channels through which the population distribution converges are very different from the
channels through which the solution of the FP equation converges.} 
%%%%%%%%%%%%%%%%%%%%%%%%%%%%%%%%%%%%%%%%%%%
In addition, no such connection is available for models with shared risk. 
The conclusion to be drawn from the foregoing is
that the strategy of identifying the distribution of all households as the running
distribution of a single (however defined) Markov chain may be wanting of an alternative.
Developing such an
alternative~-- and ultimately arriving at a numerically verifiable solution to the example introduced above~--
is the main objective of what follows.

The methodology proposed in this paper rests on four main strategies. The first one is to
treat the law of motion of the population distribution as a ``stand alone'' stochastic process in the space of
unit-mass measures
over the collection of private states. %, or, more accurately, over the collection of private shadow
%states. %The phrase  ``stand alone'' law of motion of the
%probability  distribution of a Markov process.
The~dynamics of this process will no longer arise as
the transpose of the dynamics of the value function, which
is what the FP equation (or its discrete analog) represents.
In~fact, there will be no need to solve the Bellman equation and produce the value function~-- see
the fourth strategy below. 
The~second strategy is to develop
the equilibrium exclusively in the context of models with finite time horizon and interpret the
respective infinite time
horizon versions as limits.
While passing to those limits may lead to a fixed-point-seeking program,
typically that program would be structurally different from the one leading to Figure~\ref{fg1}, in that 
the tâtonnement steps would be applied simultaneously to  prices and  population 
distribution.
%Moreover, these %tâtonnement
%steps are going to be local, i.e., confined to a
%single period.
The~third strategy is to encrypt both
the optimal policy and the population distribution as splines
and employ computational techniques  based on functional programming, which do not depend on boundary
conditions in the way finite differencing does.
%This allows the private
%states to range in a continuum.
The fourth strategy is, following \nocite{DL12}{[DL]}, to use consumption as a state variable~--
not wealth.%%%%%%%%%%%%%%%%%%%%
%%%%%%%%%%%%%%%%%%%%%%%%%%%%%%%%%%%
\footnote{The realization that consumption is a ``sufficient statistic'' can be
traced back to \citel{Hall78}.}
%%%%%%%%%%%%%%%%%%%%%%%%%%%%%%%%%%%
There are several reasons for this choice. Though seemingly innocuous (and somewhat naïve),
one reason is that there is only one consumption level to attach to a given
agent in any given period of time. In~contrast, the wealth (asset holdings) with which an economic
agent enters a particular period is generally different from the one with which the agent
exits. % (the importance of this point will be illustrated shortly).
Another reason is that consumption is a
homeomorphic image of the respective shadow variable
and this allows connections obtained by way of Lagrange duality to be used. 
In particular, the
computation of the value function can be bypassed and the special regrouping-across-time
technique from \nocite{DL12}{[DL]}, which is nothing but a tool for dealing with the time
inconsistency, can be put to use~-- see \ref{cross-sys} below.
Because of the homeomorphism just mentioned, describing the state of the population as a
distribution of unit mass over the consumption range (the approach adopted in this paper)
is tantamount to describing it as a distribution over the dual space, i.e., the space of all private
shadow variables. As a result, the transport of the distribution is governed by the
collective transformation of the individual shadow variables from one period to the next~-- a
channel that is very different from the FP equation, which is invariably linked to a single Markovian state.
The use of consumption as a state
variable, instead of wealth, has yet another advantage: the only constraint attached to it is to remain
strictly positive, the boundary of which is never reached, whereas the borrowing limit must
be specified in the outset, if wealth is a state variable.%%%%%%%%%%%%%
%%%%%%%%%%%%%%%%%%%%%%%%%%%%%5
\footnote{The borrowing limits have been of considerable interest in the literature~-- see the
discussion in Sec.~18.5 in \nocite{LjunSar00}{[RMT]}, and, more recently, 
in \citel{AHLLM17} and \citel{Holm18}. The natural borrowing limit introduced by \citel{Aiya94}
is also meant to ensure that consumption remains
positive, but it is an amount that gets specified in the outset, based on worst case scenario
considerations~-- see Sec.~18.5 in \nocite{LjunSar00}{[RMT]}.}   
%%%%%%%%%%%%%%%%%%%%%%%%%%%%%%%%%%%%%%%%%%%%%%%%
Once the demand for assets has been computed as a
function of consumption, the borrowing limit obtains as the right limit of that
function at~$0$. %, but that limit is never reached.
Thus, the borrowing limit becomes fully endogenized and need not be guessed
beforehand. In addition, no boundary conditions at that limit are needed.
Such an arrangement is quite intuitive: the only limitation on what a household can borrow is what
other households are willing to lend, which, in the context of general competitive equilibrium, should be
possible to settle in much the same way in which prices are settled. 
To wit: Ponzi schemes are eliminated by the very nature of competitive equilibrium.


Ultimately, the method developed in this paper provides a numerically verifiable solution
to Huggett's example introduced above and returns an equilibrium rate of approximately $0.037$
(with market clearing of order $10^{-6}$) that  differs 
significantly from the one suggested by the left plot in Figure~\ref{fg1}, which is around
$0.029$.
Perhaps the most transparent illustration of the approach adopted in this paper is
the left plot in Figure~\ref{fg4}. It~shows the equilibrium (produced with the new method)
long run entering and exiting cross-sectional distributions of households in different employment
categories over asset holdings.
We stress that these two sets of distributions need not be identical, though both obtain
from the same long run distribution over consumption~-- see Sec.~\ref{sec:IOU}.%Figure~\ref{fg4X}.
%This is an important cross-check of the solution obtained by the new method.
%%%%%%%%%%%%%%%%%%%%%%%%%%%%%%%%%%%
%\footnote{The distribution over employment and asset holdings with which the population exits
%any given period cannot be, in general, the same as the distribution with which the population enters the
%following period: the distribution over states of employment being time-invariant, the
%outflow of households from any employment class exactly matches the inflow, but in terms of assets
%such a match would be possible only if households that share the same level of employment also
%share the same amount of wealth, which is impossible to arrange with idiosyncratic private shocks.}
%%%%%%%%%%%%%%%%%%%%%%%%%%%%%%%%
%%%%%%%%%%%%%%%%%%%%%%%%%%%%%% 
%%   Fg: 4
%%%%%%%%%%%%%%%%%%%%%%%%%%%%%%%
%%%%%%%%%%%%%%%%%%%%%%%%%%%%%%%%%%%%%%%%%%%%%%%%%%%%%%%%%%%%
{\captionsetup{belowskip=-10pt}
\captionsetup{aboveskip=0pt}
%%%%%%%%%%%%%%%% 
\begin{figure}[!htbp]
\centering 
\begin{subfigure}{.47\textwidth} 
  \centering
\leavevmode\raise0.0cm\hbox{\rotatebox{90}{\tiny population distribution within an employment group}}% 
\ %   
\toshow{\includegraphics[width=6.5cm]{fg4L}}%
%\llap{\raise0.5cm\hbox to4.0cm{\tiny 200 grid-points\hfill}}

\leavevmode\smash{\hbox to0.3cm{\hfill} \raise6pt\hbox{\tiny asset holdings}}

%\leavevmode\smash{\hbox to0.5cm{\hfill} \raise6pt\hbox{\tiny aggregate asset holdings $a(r)$}}
  %\caption{A subfigure} 
%  \label{fig:sub1}
\end{subfigure}\quad
\begin{subfigure}{.47\textwidth}  
  \centering
\leavevmode\raise0.0cm\hbox{\rotatebox{90}{\tiny population distribution within an employment group}}%  
\ % 
\toshow{\includegraphics[width=6.5cm]{fg4R}}%
%\llap{\raise0.5cm\hbox to2.2cm{\tiny 2,000 grid-points\hfill}} 

\leavevmode\smash{\hbox to0.3cm{\hfill} \raise6pt\hbox{\tiny asset holdings}} 

%\leavevmode\smash{\hbox to0.5cm{\hfill} \raise6pt\hbox{\tiny consumption record}}
  %\caption{A subfigure}
%  \label{fg12}
\end{subfigure}
\caption{Left plot: entering (solid lines) and exiting (dotted lines)
equilibrium distributions of the households in every employment category
over asset holdings produced with the new method.
Right plot: the population distribution produced
with the classical (FP) method over $2\mathord,000$ grid points (solid lines)  when the iterations are initiated 
with the equilibrium rate of $0.037$ and with the exiting distribution (dotted lines) obtained with
the new method.}  
\label{fg4} 
\end{figure} }%
It~is instructive to note that
if the classical program leading to Figure~\ref{fg2} is initiated with the equilibrium rate
obtained with the new method and
with the discretized (over $2\mathord,000$ grid points) version
of the exiting distribution from the left plot in Figure~\ref{fg4} (instead of the uniform one), then
it~re\-turns expected demand of order~$10^{-3}$, together with the distribution shown in solid lines
on the right plot.
As was already noted, this distribution obtains by iterating a particular discrete analog
of the FP equation~-- see equation (18.2.4) in \nocite{LjunSar00}{[RMT]}. 
In that setup the distribution that is being iterated is defined over pairs of an employment
state attached to the 
present period and exiting wealth (taken from the finite grid) attached to the previous period (the
private state is exiting in terms of wealth but entering in terms of employment, which may explain the
slight difference between the distributions in the right plot). 
It is also remarkable that the endogenous borrowing limit produced with
the new method (see Sec.~\ref{sec:IOU}) %(there is one for every category of employment~-- see Sec.~\ref{sec:IOU})
is within $10^{-3}$ identical to the natural borrowing limit
of \citel{Aiya94}, which in this particular case is strictly smaller than the``ad hoc'' limit~-- see
Sec.~18.5 in \nocite{LjunSar00}{[RMT]}. 
%%%%%%%%%%%%%%%%%%%%%%%%%%%%%

The reader might have noticed already that the large dots in the left
plots in Figures~\ref{fg1} and~\ref{fg2} correspond to the newly found equilibrium
rate. This embellishment is meant to emphasize that the output from the program leading to  
these plots depends on the choice of the distribution with which the iterations are initiated:
if initiated with the uniform distribution and with
rate of around $0.037$ the program yields expected demand of around $6\mathord.901$, but if initiated with the
exiting distribution from Figure~\ref{fg4} and the same rate it yields expected demand of
order~$10^{-3}$. %%
%%%%%%%%%%%%%%%%%%%%%%%%%%%%%%%%%%
%\footnote{The possibility for this type of multiplicity was noted by \citel{DGMM94}.} 
%%%%%%%%%%%%%%%%%%%%%%%%%%%%%%%%% 
This also shows that (in~this example) it is possible to find an equilibrium arrangement in which the
long-run probability distribution of the optimal private state is the same as the long-run
cross-sectional distribution of all households, but with the caveat that to get to it one must
somehow know the correct initial distribution with which to start the iterations.  


Perhaps the most useful application of the methodology proposed in this paper
is to models with production that is subjected to common stochastic shocks.
This is the scenario where, generically, no time-invariant distribution of the
population exists (not even conditioned to the realized productivity state) and the association with
the running distribution of 
a Markov process is no longer possible~-- not even as a thought experiment.
The only time-invariance that one may hope for is for the
population distribution, treated as a stochastic process, to be Markov, with time-invariant transition
(i.e., transport) mechanism.
This is the main premise in the paper
\citel{KS98}, hereafter \nocite{KS98}{[KS]}. The computational strategy proposed here
differs from the one in \nocite{KS98}{[KS]} in a number of respects (and also
parallels \nocite{KS98}{[KS]} in a number of respects). The main difference is in the way the law of
motion of the population distribution is described (the difference in the choice of state space
notwithstanding). In \nocite{KS98}{[KS]} this law of motion
obtains by way of least square fitting from the simulated behavior of a large population of
agents (5,000 in their benchmark economy) over a large number of periods (11,000 in their benchmark
economy). This is suboptimal because 
both the individual and the collective behavior (hence, the equilibrium 
itself) depend only on the distribution of the
population and full information about the position of each and every agent is
superfluous.%%%%%%%%%%%
%%%%%%%%%%%%%%%%%%%%%%%%%%%%%%%%%%%%%%
\footnote{This feature is one of the cornerstones of the theory of Mean Field Games and has
been known to macroeconomists 
at least since \citel{HHK74} and \citel{Hil74}.} 
%%%%%%%%%%%%%%%%%%%%%%%%%%%%%%%%%%%%%%
Another drawback is the introduction of i.i.d.\ prediction errors, the size of which can be
controlled only on a case-by-case basis,
and the need to postulate a particular type of dependence (log-linear in the benchmark study
of \nocite{KS98}{[KS]}) in the outset.
In~addition, such a strategy requires a large number of periods in order to be meaningful. 
In~this paper the law of motion of the population distribution
is derived directly from the Lagrange duals of the individual optimization problems
and is expressed 
as a single (generally, nonlinear) transformation on the space of
unit-mass measures, which depends on the transition in the productivity state and is
meaningful in the context of any, small or large, time horizon.

As the space of unit-mass measures
is infinite dimensional and the common shocks amount to a random environment, and thus force the
population distribution (a state variable) to fluctuate, any implementable computing strategy
inevitably involves some form of model reduction.
The reduction adopted in \nocite{KS98}{[KS]} comes down to having the distribution
average (a scalar, corresponding to the aggregate investment of capital across the entire
population) substitute  
for the population distribution itself. It is then argued in \nocite{KS98}{ibid.}\ that such a reduction is a
reasonably good approximation of the original model, leading to the conclusion that the
standard representative agent framework does not lead to substantially inferior model predictions. 
The method developed in this paper allows for a revisit of this conclusion.
It~shows that ``representation by averages'' is indeed possible due to certain linearity in
the private demands and other endogenous variables, which occurs when all consumption levels across
the population are sufficiently far from zero. %, i.e., there is no extreme poverty among the market
%participants.
This feature is explored thoroughly and some closed form expressions for its
asymptotic behavior are derived.
It is shown that a reasonably good approximation of the full (not reduced) model can be attained  
by replacing the population distribution with the vector of its average values across the employment
categories. More important, such a representation by vector of averages
reveals sizable fluctuations in the relative disparity between employed and unemployed,
which are generated by the random environment of the productivity shocks and cannot be captured by
the total population average (single representative agent) 
alone~-- not even in
the benchmark economy described in \nocite{KS98}{[KS]}. Broadly speaking, the conclusion this paper
arrives at is that ``representation by averages'' (note the plural) is ``good enough,'' but with
a number of caveats and strings attached~-- see Sec~\ref{sec:KS}.
Since the analysis in \nocite{KS98}{[KS]} is based on
the simulated behavior of a large population of households over a large number of periods, if the
same exact analysis
is replicated with  data simulated from the model proposed here (which does not account
for individual behavior and the structure of the transport does not obtain by way of simulation),
it should produce
similar results. We 
show that this is indeed the case, and, in fact, the results are remarkably close.
%%%%%%%%%%%%%%%%%%%%%%%%%%%%%
%
\iffalse
%
%%%%%%%%%%%%%%%%%%%%%%%%%%%%
Among other things, this demonstrates that the log-linear regression structure of the total population average
(i.e., the aggregate installed capital) postulated in \nocite{KS98}{[KS]},
in conjunction with the least 
square fit,
is reasonably close to the true dependence in the benchmark economy, but
we stress that there is no reason for such log-linear structure to hold generically, whence the need for
a method that does not rely on assuming it in the outset.
%In~addition, the
%total population average cannot capture the disparity among the employment categories~-- see Sec~\ref{sec:KS}.  
%%%%%%%%%%%%%%%%%%%%%%%%%%%%%%%%%
%
\fi
%
%%%%%%%%%%%%%%%%%%%%%%%%%%%%%%%

\iffalse
%%%%%%%%%%%%%%
In conjunction with these transformations (transport mappings),
the simulation of a large time series of realized aggregate states leads to a simulated series of the
population distribution averages across the 
employment categories, which is easy to transform into a simulated series of the aggregate population
averages. Just as it should, this series has the same statistical properties as the one
in the case study of \nocite{KS98}{[KS]}, which is produced by
simulating the individual behavior of a large number of agents; in particular, the log-linear
regression fit reported by \nocite{KS98}{[KS]} is very close to the one obtained from the
series just described. This is consistent with the very high goodness of fit values reported
by \nocite{KS98}{[KS]}, implying that the true transport is indeed log-linear.
%%%%%%%%% 
\fi



%%%%%%%%%%%%%%%%%%%%%%
%
\iffalse{
%{
%%%%%%%%%%%%%%%%%%%%%
A~major limitation in the use of continuum
as a labeling set is that, setting all measurability issues aside,
the idiosyncratic shocks attached to the agents can only be essentially pairwise
independent, which then imposes coalitional aggregate certainty, as discovered
by \nocite{SZh09}{(Sun and 
Zhang 2009)}. In sum, truly idiosyncratic shocks are not going to be possible under such an
arrangement, however innocuous it might seem.
%%%%%%%%%%%%%%%%%%%%%%
%
}\fi
%
%%%%%%%%%%%%%%%%%%%%%

{\parskip=0pt 
The contributions in this paper can be summarized as follows. In terms of methodology, the
paper extends the strategy for finite time horizon endowment economies proposed in \nocite{DL12}{[DL]}
in two ways: by allowing for an infinite number of heterogeneous economic agents and by allowing for
shared risk and production, while keeping the finite horizon intact. 
In~addition, the paper provides a necessary (see above)
alternative to the general MFG/FP framework. %%%%%%%
While strict adherence to that framework would require the dependence
on the population distribution to be incorporated explicitly into the dynamics of the private
state, in the present context such an arrangement would be too much to ask for.
Just as an example, the population
distribution affects the market clearing, which affects the individual optimal choices, but not in a
way that can be pinned down. 
\citel{AHLLM17} also observe that
``several of the typical features of heterogeneous agent models in economics
mean that they are not special cases of the MFGs treated in mathematics.''
Instead of attempting to adapt the logic and the structure of MFGs to heterogeneous agent models, 
see \citel{AHLLM17},
the paper proposes dynamic transport theory specifically designed for such models~--
see \ref{main-q} below.  
%The method developed in the present paper removes the need for such an
%explicit dependence.
%In addition, the paper provides a novel description of the law
%of motion of the population distribution, which is structurally different from the stochastic FP equation
%in a coupled MFG system~-- see above.
In~terms of concrete applications, the paper provides a numerically
verifiable equilibrium for the benchmark Huggett economy introduced earlier. 
In the context of yet another benchmark economy with
common noise, the paper sheds new light on the nature of heterogeneity and argues that models with a
single representative agent may not be as adequate as is often believed. It
provides a more robust, more efficient, and more accurate computational technology for models with common
noise, without the need for simulation followed by least-square~fit.


In terms of general objectives, the closest antecedent to this paper
is \citel{AHLLM17}, which is concerned with a particular continuous time version of 
Huggett's pure credit economy with only two idiosyncratic states.
%, and contains a comprehensive list of references and bibliographical comments.
Despite the observations in the preceding paragraph,
the strategy adopted in \nocite{AHLLM17}{ibid.}\ tracks the MFG framework and essentially boils
down to the classical one: % (and relies on
%continuity of the expected demand as a function of the interest):
choose an interest rate, compute
the expected private demand after solving the FP equation attached to a generic agent,
repeat with a different rate until the expected private demand becomes null. 
It is noted in \nocite{AHLLM17}{ibid.}\  that one of the computational advantages of the
continuous time MFG/FP framework is in the ``handling of borrowing constraints.''
This is yet another aspect in
which the present paper differs~-- both in strategy and in final outcomes.
Here the borrowing limits are fully endogenized (see above) and, in contrast to \nocite{AHLLM17}{ibid.}, the
equilibrium distribution of the population has no mass at those limits (see
Figure~\ref{fg4}), while the marginal propensity to consume remains finite (see Figure~\ref{fg9Z}). %as it does 
%in \nocite{AHLLM17}{ibid.}\eos
%Because the model structure in \nocite{AHLLM17}{ibid.}\ is not identical 
%to that of the benchmark model
%borrowed here from Sec.~18.7 in \nocite{LjunSar00}{[RMT]}, the source of
%this disparity cannot be identified exactly, but it is instructive to note that 
%(see Figure~\ref{fg4})
%the equilibrium distribution obtained here is consistent with the classical one, which also does
%not accumulate.
Finally, Supplementary Appendix G.2 to \citel{AHLLM17} shows how the methodology in
\nocite{AHLLM17}{ibid.}\ can be applied to a continuous time version of
the production economy studied by \citel{Aiya94}.%%%%%%
%%%%%%%%%%%%%%%%%%%
\footnote{As a computational task, the model of \citel{Aiya94} is not very different from the
benchmark Huggett economy discussed here and also in \citel{AHLLM17}.}
%%%%%%%%%%%%%%%%%%
The present paper takes a step further and develops a program that can handle common productivity
shocks (a substantial computational challenge)
and provides a concrete application of this program to the benchmark economy  %\citel{KS98}.
borrowed from \nocite{KS98}{[KS]}.  
}


The paper is concerned exclusively with the dual interpretation of well known models,
in the context of developing and implementing a new computational technology.
Its main objective
is confined to the practical task of producing, if possible, numerically verifiable equilibria. 
It~does not address theoretical matters such as
existence and/or uniqueness of equilibria, which have been studied extensively~--
see (among many others) \citel{Hug93}, \citel{Aiya93}, \citel{DGMM94},
\nocite{ChS16}{Cheredito and Sagredo (2016a}\nocite{ChS16b}{-b)}, \citel{Pro21},
%Just as in \citel{AHLLM17}, the main objective 
%is confined to the practical task of producing, if possible, numerically verifiable equilibria. 
and especially \citel{AHLLM17} for extensive bibliographical notes.
It is interesting to note that the existence and uniqueness in
\nocite{AHLLM17}{ibid.}~-- see Propositions~4 and~5~-- hinges on the continuity of the expected
demand as a function of the interest rate. The works \citel{GKSS00} and
\nonocite{KSS94}\nocite{KSS97}{Karatzas et al.\ (1994, 1997)} study similar problems from a
game-theoretic point of view.
%
%
\iffalse
%%%%%%%%%%%%%
%%%%%%%%%%%%%%%%%%%%%%%%%%%%%%%%%%%%%%%%
\footnote{The paper \citel{AHLLM17} does not provide proof of the statement that
that the average demand is a continuous function of the interest rate;
it only states that ``one can show'' continuity~-- see Propositions~4 in \nocite{AHLLM17}{ibid.}}
%%%%%%%%%%%%%%%%%%%%%%%%%%%%%%%%%%%%%%%
It is clear from Figures~\ref{fg1} and~\ref{fg2} that, at least in the benchmark economy borrowed
from \nocite{LjunSar00}{[RMT]}, such continuity may fail to exist, and the fact that refining the
grid only makes the  gap grow larger strongly suggests that continuity may not hold even if
private wealth is allowed to range in a continuum. Of course,
the lack of continuity in the average
demand does not entail nonexistence of equilibrium.
However, it makes the matter of existence even harder to sort out,
and the paper \nocite{DL12}{[DL]} provides an example of multiplicity
in an incomplete market endowment economy with
two agents.

In principle, the setup of the present paper may be possible to place in the general
framework of \citel{DGMM94}, but because of the implicit structure of the dependence of the private
state on the population distribution, the necessary conditions established in \nocite{DGMM94}{ibid.}\ would be
very difficult to verify.
%%%%%%%%%%%%
\fi
%
%
%%%%%%%%%%%%%%%%%%%%%%%%%

Appendix~A elaborates on some relevant conceptual matters from the realm of economics.
The mathematical concepts this paper resorts on are relatively simple and confined to Lagrange
duality and Glivenko-Cantelli's theorem. The computational technology can be described broadly as
``functional programming'' 
and involves nonlinear solvers, spline interpolation, and numerical integration.
The Julia code used in all examples and illustrations can be retrieved from
\href{https://doi.org/10.5281/zenodo.7757102}{\includegraphics[width=3.75cm]{last-zenodo}}$\,$,
or directly from the author's \href{https://github.com/AndrewLyasoff/heterogeneous-models}{Git repository}.
%The computer code used in the all examples and illustrations, written in the Julia programming langauge, is
%available at: \href{https://doi.org/10.5281/zenodo.7487772}{https://doi.org/10.5281/zenodo.7487772} 


The paper is organized as follows. Sec.~\ref{sec:gen-model} describes a general 
heterogeneous agent incomplete market model in terms of Lagrange duality
and outlines a general metaprogram for identifying an equilibrium. Sec.~\ref{sec:IOU}
specializes that metaprogram
to the case of an economy with infinite time horizon and no shared risk. 
It is shown there that, the plots in Figure~\ref{fg2} notwithstanding, the new
strategy can locate an equilibrium in the same benchmark study.
%, which, as was
%already noted, is
%consistent with the classical approach. 
Sec.~\ref{sec:KS} implements the metaprogram from Sec.~\ref{sec:gen-model} in
the context of the benchmark economy of \nocite{KS98}{[KS]}, compares the results, and draws some new insights.  
%\href{https://github.com/AndrewLyasoff}{https://github.com/AndrewLyasoff}
%The computer code used in all examples can be retrieved from:\hbox to3cm{\hfill}\hfill  
%\href{https://doi.org/10.5281/zenodo.7484925}{https://doi.org/10.5281/zenodo.7484925}\hfil or\hfil
%\href{https://github.com/AndrewLyasoff/heterogeneous-models}{https://github.com/AndrewLyasoff/heterogeneous-models}


%$\mathbbi{R}^{\tts\SSS}$

%$\mathcal{SPQHXL\ st}A$ $\mathscr{SPQHXL\ stl}$ $\R^\SSS$ $\mathfrak{KLMHX}A$  $\R\Z\N$\quad 1A $1A$

%%% chend556677  


%%- -%%--{Sec: #2}--%%
%{Eq: 2.}   %%llabel
%{No: 2.}   %%nlabel

\section{Dynamic Transport by Means of Lagrange Duality}
\label{sec:gen-model}\setcounter{paragraph}{0}

\noindent
The main result in this section, Proposition~\ref{main-q} below, gives the law of
motion, or transport, of the cross-sectional distribution of households over the range of
consumption and states of employments.
%, which is
%homeomorphic to the range of the private shadow variable. 
It obtains from connections provided by Lagrange duality and reflects the fact that all
individual savings problems are treated, so to speak, ``in an orchestra.''
%The proof
%becomes completely transparent once all private optimality conditions, stated in dual form,
%are accounted for and rearranged in time accordingly.
First, we introduce the notation and describe the general setup.  

The total number of households (alias: agents), $N$, is  finite but very large
and the time parameter $t$ is restricted to the finite set
$\{0,1,\ldots,T\}$.
Economic output is generated in every period and is expressed in units
of a single numéraire good, which 
can be either consumed, or turned into productive capital 
except during period $T$.
Every household extracts utility from consuming the numéraire good. All households share the same
impatience parameter $\b>0$ and the same time-separable utility from 
intertemporal consumption given by the mapping $U\colon\R\mapsto \R$,
which is twice continuously differentiable in $\Rpp$ with $\partial U>0$, $\partial^2 U<0$,
and is such that   $\lim_{c\searrow 0}\partial U(c)=+\infty$, $\lim_{c\searrow 0}U(c)=-\infty$ and
$U(c)=-\infty$ for $c\le 0$.  
Economic output is generated by two inputs: the net labor supplied during the period
the output is delivered
and the net productive capital installed during the previous period. Two investment
instruments are available to all households: capital stock
and locally risk-free private lending instrument (alias: IOU), which is in net supply of zero.%%%
%%%%%%%%%%%%%%%%%%%%%
\footnote{The numéraire good, which is also the currency, cannot be stored from one period to
the next, but entitlements to it can be carried from one period to the next by means of financial
contracts. The assumption that the private lending instrument is in zero net supply is imposed here
merely for simplicity.}
%%%%%%%%%%%%%%%%%%%%%%%%%%%%%%%%%%%%%%%%%%%%%%%%%%%%%% 
The collection of all (idiosyncratic) private  employment states 
is $\EEE\subset\Rp$ and the collection of all
productivity states is $\XXX\subset\Rpp$.
These sets have finite cardinalities denoted $\abs{\EEE}$ and~$\abs{\XXX}$, both assumed to be at
least~$2$.
The elements of $\XXX$ have the meaning of total factor productivity (TFP)
and an~employment state $u \in\EEE$ is understood to represent $u/N$  
physical units of labor.
Households that are in the same state of employment would have identical consumption levels only if
they start the period with identical asset holdings,%%%%%%
%%%%%%%%%%%%%%%%%%%%%%%%%%%%%%%%%%%%%%%%%%%
\footnote{Both assets, private lending and capital, are assumed to be perfectly liquid, in which
case the composition of asset holdings is irrelevant.}
%%%%%%%%%%%%%%%%%%%%%%%%%%%%%%%%%%%%%%%%%%
in which case their
investment decisions would be identical as well (see \ref{thm1} below for an explanation).
To put it in another way,
households that are in the same state of employment $u\in\EEE$ and choose the same consumption
level $c\in\Rpp$, which corresponds to $c/N$ physical units of consumption, are indistinguishable. %%%%%%%%%%%%
%%%%%%%%%%%%%%%%%%%%%%%%%%%%%%%%%%%%%%%%%
%\footnote{The reasons for this choice, i.e., replacing wealth with consumption,
%are detailed in \nocite{DL12}{D\&L}.}
%%%%%%%%%%%%%%%%%%%%%%%%%%%%%%%%%%%%%%%%
As~a result, the collective
state of the population can be expressed as a distribution of unit mass over the product space
$\EEE\times\Rp$ (the space of household characteristics, i.e., employment and consumption levels). 
To be able to work with such objects, we introduce the collection $\DB$ of all (cumulative)
distribution functions over $\Rpp$ and the collection  $\DB^\EEE$ of all assignments
$\csd\colon \EEE\mapsto \DB$. An~element $\csd\in\DB^\EEE$ can be
identified as a finite list of distribution functions $\csd\equiv(\csd^u\in\DB)_{u\,\in\,\EEE}$, in which
one 
cumulative distribution function over $\Rpp$ is assigned to every employment category $u\in\EEE$.
Letting $\bbP(\EEE)$ stand for the collection of all strictly positive  measures of unit mass
over the finite set~$\EEE$, the collective state (distribution) of
the entire population of households is a pair $(\pi,\csd)\in \bbP(\EEE)\times \DB^\EEE$, the meaning of which is
that, for any  $u\in\EEE$ and any $c\in\Rpp$,  the product $\pi(u)\times \csd^u(c)\times N$
gives the quantity of
agents who happen to be in employment state~$u$ and have chosen consumption level that is not
strictly larger than $c\in\Rp$.
%If the dependence on the time period or the productivity state must be
%emphasized in the notation we shall write $\csd^{t,x}\in\DB^\EEE$, or, equivalently,
%$\csd^{t,x,u}\in\DB$, $u\in\EEE$. 

The productivity state follows an irreducible Markov chain
with transition probability matrix~$Q$ (of size $\abs{\XXX}$-by-$\abs{\XXX}$)
that has a unique set of steady-state probabilities $(\psi(x),\, x\in\nobreak\XXX)$.
The transitions in the individual employment states, which are independent from one another
when conditioned to a particular transition in the productivity state $x \to y$,
are governed by the transition probability matrices~$P_{x, y}$, $x,y\in\XXX$. All elements of the
matrices $Q$ and $P_{x,y}$ will be assumed strictly positive. The pair of 
the common productivity state and the employment state of a given household follows a Markov 
chain on the state-space $\XXX\times\EEE$ with transition from $(x,u)$ to
$( y,v)$ occurring with probability  $Q(x, y)P_{x, y}(u,v)$.
Given the present period productivity state $x\in\XXX$ and given the future (next period)
productivity state $ y\in\nobreak\XXX$, every household presently in employment state $u\in\EEE$
samples its future employment state from the collection $\EEE$, independently from all other households,
according to the distribution law over $\EEE$ encrypted into the vector
$P_{x, y}(u,\cdot)\in\R^{\abs{\EEE}}$. If the present distribution of the
population over the states of employment is $\pi\in\bbP(\EEE)$, then the number of households in
employment state $u$ is $\pi(u)N$. Since $\pi(u)>0$, by Glivenko-Cantelli's theorem
as $N\to\infty$ the proportion from those in state $u$ who transition to state $v$ must converge to 
$P_{x, y}(u,v)$ for every $v \in\EEE$. In~what follows we shall suppose that the number $N$ is so
large that in all relevant calculations the proportion of households characterized by the
transition $u\to v $, from the collection of all households who happen to be in state $u$,
can be taken to be $P_{x, y}(u, v)$, and note that
this is the only step where we explicitly use the stipulation that $N$ is ``very large.''
As a result, the future (next period) distribution of the
population over states of employment, $\varpi\in\bbP(\EEE)$, is given by
{\abovedisplayskip=5pt plus 1.5pt minus 5pt\belowdisplayskip=5pt plus 1.5pt minus 5pt\belowdisplayshortskip=3pt plus 1.5pt minus 6pt
\[
\varpi(v)=\sum\nolimits_{u\ts\in\ts\EEE}\pi(u)\, P_{x, y}(u,v)\quad 
\text{for all}\ \ v \in\EEE\,,
\]}%
which may be abbreviated as,
$
\varpi=\pi\, P_{x, y}\,,
$
treating $\pi$ and $\varpi$ as vector rows of size $\abs{\EEE}$.
It is important to note that $\varpi$ depends not just on $\pi$ and $x$, but also on the future
productivity state~$y$.
\nitskip

\def\gp{{\pi}}

%\noindent
%{\bf Assumption and Remark:}
%%- -%##%--{No: 2.1}--%%
\begin{nit}{Assumption and Remark}\label{n1}
In the benchmark economy studied in \nocite{KS98}{[KS]} the
conditional transition matrices
$P_{x, y}$ are chosen in such a way that it becomes possible to attach a unique distribution,
$\gp_x\in\nobreak\bbP(\EEE)$, to every productivity state $x\in\XXX$ so that
$\gp_ y =\gp_xP_{x, y}$ for all choices of $x, y\in\XXX$ (the future distribution over
employment depends only on the future productivity state).
With this special choice for the matrices $P_{x, y}$ the steady state regime of the population in
terms of employment is such that the distribution of households over states of employment
fluctuates randomly and in perfect sync with the productivity state, so that when the
economy is in state $x\in\XXX$ the distribution of households over states 
of employment is exactly $\gp_x\in\nobreak\bbP(\EEE)$.%%%%%%%%%%%%%%
%%%%%%%%%%%%%%%%%%%%%%%%
\footnote{This feature comes from the fact that
the Markov chain on $\XXX\times\EEE$ with transition matrix $Q(x, y)P_{x, y}(u, v)$ admits a
steady-state distribution in which state $(x,u)$ occurs with probability $\psi(x)\gp_x(u)$.}
%%%%%%%%%%%%%%%%%%%%%%%%%%%%%%%%%%%%%%%%%%%%%%%%%%%%
This is a vast simplification, because in such a regime the distribution of households over states of employment
fluctuates through the finite collection $\{\gp_x\colon x\in\XXX\}\subset \bbP(\EEE)$, as opposed
to fluctuating through the entire infinite collection~$\bbP(\EEE)$.
In~what follows we shall suppose that this simplification is in force,
and shall assume without further notice that productivity and employment are fluctuating according
to the steady state regime just described (even at time $t=0$). In~addition, we shall assume that
$\pi_x(u)>0$ for every $x\in\XXX$ and $u\in\EEE$ and shall
exclude the scenario where the productivity or the employment
state can get absorbed in a single state. The population distribution in every productivity state can now
be given as an element of $\DB^\EEE$, since the distribution over $\EEE$ is fixed by
the productivity state.
Thus,
an ``aggregate state of the economy'' will be understood to mean a pair
of the form $(x,\csd)\in\XXX\times\DB^\EEE$, consisting of a productivity state $x$ and a list of
distribution functions $(\csd^u\in\DB,\,u\in\EEE)$.\qed
%This aggregate state and the time period in which it occurs are assumed to capture all the
%information that individual households take into account, in addition to their
%entering wealth and state of employment, when making rational decisions about investment and consumption 
%(in that period and state of the economy).\qed
%(one such distribution function attached to every employment category).\qed
%\smallskip
\end{nit}

Because of the assumption just made, the net supply of labor in any productivity state
$ x\in\XXX$ is fixed  and can be written  as
{\abovedisplayskip=7pt plus 1.5pt minus 5pt\belowdisplayskip=7pt plus 1.5pt minus 3pt\belowdisplayshortskip=3pt plus 1.5pt minus 3pt
\[
L(x) = \sum\nolimits_{u\ts\in\ts\EEE}\,{u\over N} \,\gp_ x(u) \,N
= \sum\nolimits_{u\ts\in\ts\EEE}u\,\gp_ x(u)  = \gp_ x\EEE\,,
\]
}%
where $\gp_ x$ and $\EEE$ are treated as vector row and vector column respectively.
Next, we postulate the usual ``competitive firm'' with two factors of production, capital and labor,
and with production technology 
given by  Cobb-Douglas constant return to scale production function with
capital share parameter $0<\a<1$. %, namely
%\[
%K \leadsto  y K^\a L( y)^{1-\a}\,,\quad y\in\XXX\,.
%\]
%%%%%%%%%%%%%%%%%%
In equilibrium, factor prices maximize the firm's profits, so that
the rates of return, realized in the
future state $ y\in\XXX$, on capital and labor,
treated as functions of capital installed during the present period,
are given, respectively, by%%%%%%%%%%%%%%%%%%%%
%%%%%%%%%%%%%%%%%%%%%%%%%
%\footnote{Wages are paid immediately during the same period labor is installed,
%but returns on capital stock are realized in the next period.}
%%%%%%%%%%%%%%%%%%%%%%%%%%
{\abovedisplayskip=7pt plus 1.5pt minus 5pt\belowdisplayskip=7pt plus 1.5pt minus 5pt\belowdisplayshortskip=3pt plus 1.5pt minus 6pt
\[
\Rpp\ni K \leadsto \rho_ y(K)\df  y\times\a\times \Bigl({K\over L( y)}\Bigr)^{\a-1}\quad\text{and}\quad
\Rpp\ni K \leadsto \ee_ y(K)\df  y\times(1-\a)\times \Bigl({K\over L( y)}\Bigr)^{\a}\,.
\]}%
Installed productive capital is assumed to depreciate at constant rate $\dd>0$ and,
in order to generate paychecks at time $t=0$, we postulate the fictitious quantity
$K_{-1}$, which has the meaning of a primordial aggregate endowment with capital
that is shared equally among all households.%%%%%%%%%%%%%%%%%%%%
%%%%%%%%%%%%%%%%%%%%%%%%%%%%%%%%%%%%%%%%%%%%%%%
\footnote{There is no need for the households to be identical before time $0$.
This assumption is imposed only for the sake of simplicity.} 
%%%%%%%%%%%%%%%%%%%%%%%%%%%%%%%%%%%%%%%%%%%%%%

%%- -%##%--{No: 2.2}--%%
\begin{nit}{Time-dependent transport}\label{et}
The transport of the population distribution from period $t$ to period $t+1$ is assumed to depend
on the period $t$ productivity state $x\in\XXX$ and the realized period $(t+1)$ productivity state
$y\in\XXX$. It will be expressed as the collection of mappings
$\T_{t,x}^y\colon\DB^\EEE\mapsto\nobreak \DB^\EEE$, $x,y\in\XXX$.
Rational private decisions about consumption and investment are only possible if an assumption
about the structure of the mappings $\T_{t,x}^y$ is made. At the same time, the realized transport of the
population is the result of a multitude of individual choices, and the notion of competitive equilibrium
requires that realized and assumed (by the agents) transports coincide.
This requirement is the main challenge in the study of incomplete-market models with a large number
of heterogeneous agents and addressing it beyond the realm of what is commonly known as
``stationary recursive equilibria''
is the main goal in what~follows.\qed 
\end{nit}
\nitskip

%%- -%##%--{No: 2.3}--%%
\begin{nit}{Remark}
There is an obvious parallel between the transport (for fixed $t$) just introduced and the one arising 
in the classical Monge-Kantorovich problem~-- see \nocite{Vil03}{Vilani
(2003}, \nocite{Vil09}{2009)}, \citel{Gal16}, for example. 
There is also a crucial difference:
there is no surplus function (see \nocite{Gal16}{ibid.})\ whose average is to be optimized and the target measure
is one of the unknowns (somehow, one must determine both the transport and the target measure).
This makes the problem more challenging and also more interesting: the transport is determined
from optimizing over a large number of individual objectives rather than a single global one.
Because of this shift in the paradigm, most of the tools for solving the Monge-Kantorovich problem
developed in the course of the last two
centuries will not be possible to utilize in the present context~-- at least not directly. Nevertheless,
some of the general ideas can still be mimicked and put to use. The method developed below still
relies on the idea of randomization introduced by Kantorovich, but not by way of coupling of two
probability measures. It also relies on the idea of Monge coupling, but not by way of pure
assignment~-- see \ref{rnd-Monge} below.\qed 
\end{nit}


In a first step, we formulate the savings problem faced by a generic household and its
dual.
All households
observe their individual employment history and the history of the realized productivity state.
They also share the same belief about the transition probabilities that govern the law of motion of
those states, encrypted in the
matrices $P$ and $Q$, and, most important, 
share the same belief about the initial (time $0$) population distribution and the entire collection of
transport mappings $\T_{t,x}^y\colon \DB^\EEE\mapsto\DB^\EEE$, $0\le t<T$, $x,y\in\XXX$.%%%%%%%%%%%%%%%%%%%%%%
%%%%%%%%%%%%%%%%%%%%%%%%%%%%%%%%%%%%%%%%%%%%%%%%%%%%%%%%%%%%%%%%
\footnote{Allowing the transport mappings to be time dependent is one of the differences between
the approach adopted here and previous works~-- see \nocite{KS98}{[KS]}.}
%%%%%%%%%%%%%%%%%%%%%%%%%%%%%%%%%%%%%%%%%%%%%%%%%%%%%%%%%%%%%%%%
Hence, in any given period all households have identical beliefs about
the population distribution (over the range of consumption and levels
of employment), once the history of the productivity state until
that period is revealed. Rational decisions about savings and consumption
depend only on the current employment state, the current productivity state,
and the perceived population
distribution during the current period.%%%%%%%%%%%%%%%%%%%%%%%%%% 
%%%%%%%%%%%%%%%%%%%%%%%%%%%%%%%%%%%%%%%%%%%%%%%%
\footnote{The individual savings problems are influenced by the history of the productivity shocks only
through their dependence on the current population distribution.}
%%%%%%%%%%%%%%%%%%%%%%%%%%%%%%%%%%%%%%%%%%%%%%%
Once put in place, all private decisions about savings and consumption produce an actual distribution of the
population over the range of consumption, and the notion of competitive equilibrium requires that
this actual distribution coincides with the perceived one, under which all private decisions have
been made. Because the total installed capital aggregates all private capital investments, and the
risk-free rate of the private lending instrument is a consequence of an implicit agreement that all
households effectively strike while borrowing and lending,
in equilibrium the installed capital and the lending rate depend solely on the aggregate state of the economy
(productivity state paired with a distribution of the population) and also on the time period in which
this aggregate state is realized. As~usual, all private decisions about savings and consumption involve
predictions of the future paths of the individual employment state and the shared
productivity state. What is unusual about the present setup is that these predictions
rely not just on the transition matrices~$P$ and~$Q$,
but also on the shared belief about the structure of the transport mappings in the
future (note that these mappings are time dependent). Such a
decision process is Markovian but in random environment, in that the transport mappings fluctuate
randomly, due to the fluctuations in the productivity state.   

At the beginning of period $0\le t<T$ the economy enters the aggregate state $(x,F)\in\XXX\times\DB^\EEE$, the
period~$t$ aggregate installed capital $K_t(x,F)>0$ and risk-free rate $r_t(x,F)>-1$, which we
abbreviate for now as $K$ and $r$,
are revealed, and  the household enters employment state $u\in\EEE$ 
with personal wealth measuring $w/N$ units of the numéraire good,
which quantity aggregates the wage received during period $t$, the return on capital invested in the
previous period,
and the holdings of private loans carried from the previous period.
%%%%%%%%%% 
Treating the entering wealth $w/N$
as a given resource, the household must determine its consumption level $c$,
understood as a real nominal amount of $c/N$ units of the numéraire good,
its investment $\q$ in the private lending instrument (IOU),
understood as a real nominal amount of $\q/N$ units of the numéraire good,
and its investment $\qq$ in productive capital,
again understood as a real nominal amount of $\qq/N$ units of the numéraire good,
by maximizing over $(c,\q,\qq)\in\R^3$ and over $(W_{y,\tts v}\in\R)_{ y\tts\in\tts\XXX, v\tts\in\tts \EEE}$ the
objective
%%
%%- -%@@%--{Eq: 2.1}--%%
{\abovedisplayskip=5pt plus 1pt minus 1pt\belowdisplayskip=5pt plus 1pt minus 1pt\belowdisplayshortskip=3pt plus 0.5pt minus 0.5pt
\begin{equation}\label{ze1}
\begin{aligned}
J_t\Bigl(c/N,&\,(W_{y,\tts v}/N)_{ y\tts\in\tts\XXX, v\tts\in\tts \EEE}\Bigr)\\
&\df U(c/N)
+\b\sum\nolimits_{ y\ts\in\ts\XXX,\,v\ts\in\ts\EEE}
V_{t+1,\tts y,\tts \T_{t,\tts x}^y(F),\tts v}\bigl(W_{y,\tts v}/N\bigr)\, Q(x, y)P_{x, y}(u, v)\,,
\end{aligned}
\end{equation}}%
subject to
%%
%\begin{gather}
%%- -%@@%--{Eq: 2.2}--%%
{\abovedisplayskip=5pt plus 1pt minus 1pt\belowdisplayskip=5pt plus 1pt minus 1pt\belowdisplayshortskip=3pt plus 0.5pt minus 0.5pt
\begin{equation}\label{ze2}
\begin{gathered}
%\label{ze20}
W_{y,\tts v}=\bigl(1+r\bigr)\q + \bigl(\rho_ y(K)+ 1-\dd\bigr)\qq + 
\ee_ y(K)v\,,\  \  y\in\XXX\,,\  v \in\EEE\,,\\
%\noalign{\rlap{and}\vskip-15pt}
%\label{ze2} 
\llap{and\qquad\qquad}c+\q+\qq=w\,,
\end{gathered}
\end{equation}}%
%\end{gather}
%%
with the understanding that  $V_{t,\tts x,\tts F,\tts u}(w/N)$ is the constrained maximum of
\eqref{ze1} for every $0\le t<\nobreak T$ and
$V_{T,\tts y,\tts \T_{T-1,\tts x}^y(F),\tts v}\bigl(W_{y,\tts v}/N\bigr)=U\bigl(W_{y,\tts v}/N\bigr)$,
i.e., during the last period the household can only consume (note that $U(c/N)=-\infty$ if $c\le 0$ and
$U\bigl(W_{y,\tts v}/N\bigr)=-\infty$ if $W_{y,\tts v}\le 0$ by the very definition of $U\phd$).
%Hence, the range of the value function associated with every period, every aggregate state, and
%every idiosyncratic state is $\llbkt -\infty,\infty\rlbkt\,$.
No borrowing constraints are imposed for now~-- such constraints will be enforced
endogenously by the equilibrium structure introduced later.
The range of the value function is the interval
$\llbkt -\infty,\infty\rlbkt$, with the value of $(-\infty)$ attained only if, for the given entering wealth,
there is no policy that can fund
strictly positive consumption in all possible future aggregate and idiosyncratic states.
By convention, the derivatives of any function will be treated as undefined on any domain in which
its value is $(-\infty)$ and the appearance of derivatives implies that the argument belongs to a
domain in which the function is finite;
as an example, any of the expressions $\partial U(c)$ and $\partial ^2 U(c)$ 
implies that $c>0$.

%%- -%##%--{No: 2.4}--%%
\begin{nit}{Remark}\label{1350}
Since the agents are distinguished by their consumption levels, which vary from one
period to the next, the sequence of systems (\ref{ze1}-\ref{ze2}) obtained for $t=T-1,\ldots,0$
cannot be viewed as ``the Bellman equation'' of any ``one agent.''\qed
\end{nit}
\nitskip
 


Suppose next that all value functions $V_{t+1,\tts y,\tts \T_{t,\tts x}^y(F),\tts v}\phd$
%, $y\in\XXX$, $v\in\EEE$,
in \eqref{ze1} are strictly concave and in $\C^2$ on the domain in which they are finite,
in which case the objective function in \eqref{ze1} is also strictly concave and in $\C^2$
in the domain where it is finite.
As a result, if $V_{t,\tts x,\tts F,\tts u}(w/N)$ happens to be finite, then the first order Lagrange, also known as
Karush-Kuhn-Tucker (KKT), conditions attached to the problem (\ref{ze1}-\ref{ze2}) must hold at any 
$(c,\q,\qq)\in \R^3$ that happens to be a solution.
Furthermore, these conditions are also sufficient: if  the
KKT conditions attached to (\ref{ze1}-\ref{ze2})
hold at some $(c,\q,\qq)\in\R^3$, at which the objective in \eqref{ze1} is finite,%%%%%%%
%%%%%%%%%%%%%%%%%%%%%%%%%%%%%%%%%%%%%%%%%%%%
\footnote{This is the only situation in which the KKT conditions are meaningful.}
%%%%%%%%%%%%%%%%%%%%%%%%%%%%%%%%%%%%%%%%%%%
then $(c,\q,\qq)$ is the (unique)
solution (\ref{ze1}-\ref{ze2}). 
The KKT conditions are instrumental in what follows and are introduced next.
To this end, we first restate the first set of constraints in \eqref{ze2} in the form
(so that the resource $w$ shows up
in the right side of every constraint) 
{\abovedisplayskip=5pt plus 1pt minus 1pt\belowdisplayskip=5pt plus 1pt minus 1pt\belowdisplayshortskip=3pt plus 0.5pt minus 0.5pt
$$   
W_{y,v}-\bigl(1+r\bigr)\q-\bigl(\rho_ y(K)+ 1-\dd\bigr)\qq -
\ee_{ y}(K)v + c + \q + \qq = w\,,\ \  y\in\XXX\,,\  v \in\EEE\,.
$$}%
The shadow variable (Lagrange multiplier) attached to each of these constraints we deliberately cast
in the factor form
{\abovedisplayskip=5pt plus 1pt minus 1pt\belowdisplayskip=5pt plus 1pt minus 1pt\belowdisplayshortskip=3pt plus 0.5pt minus 0.5pt
$$
\l_{y,\tts v}=\Lmp_{y,\tts v} \times {\b\over N}Q(x, y)P_{x, y}(u,v)\,,\ \  y\in\XXX\,,\  v \in\EEE\,,
$$}%
i.e., we will be working with $\Lmp_{y,\tts v}$ instead of the true shadow variable $\l_{y,\tts v}$,
and  the shadow variable attached to the last condition in \eqref{ze2} we deliberately cast in the form $\ff/N$,
i.e., we will be working with $\ff$ instead of the true shadow variable~$\ff/N$.
The Lagrange dual of the optimization problem in (\ref{ze1}-\ref{ze2}) can be stated as
{\abovedisplayskip=5pt plus 1pt minus 1pt\belowdisplayskip=5pt plus 1pt minus 1pt\belowdisplayshortskip=3pt plus 0.5pt minus 0.5pt
$$
\mathop{\rm minimize}_{\ff,\tts (\Lmp_{y,\tts v})_{ y\tts\in\tts\XXX,\, v\in\EEE}}
\biggl(\,\mathop{\rm maximize}_{c,\tts \q,\tts \qq,\tts (W_{y,\tts v})_{ y\in\XXX,\tts v\in\EEE}}
\LLL\Bigl(c,\q,\qq,(W_{y,\tts v})_{ y\tts \in\tts \XXX,\, v\tts \in\tts \EEE},
\ff,(\Lmp_{y,\tts v})_{ y\tts \in\tts \XXX,\, v\tts \in\tts \EEE}\Bigr)
\biggr)\,,
$$}%
where
{\abovedisplayskip=5pt plus 1pt minus 1pt\belowdisplayskip=5pt plus 1pt minus 1pt\belowdisplayshortskip=3pt plus 0.5pt minus 0.5pt 
\begin{align*}
&\LLL\Bigl(c,\q,\qq,(W_{y,\tts v})_{ y\tts \in\tts \XXX,\, v\tts \in\tts \EEE},
\ff,(\Lmp_{y,\tts v})_{ y\tts \in\tts \XXX, \, v\tts \in\tts \EEE}\Bigr)\\
&\qquad\df J_t\Bigl(c/N,\,(W_{y,\tts v}/N)_{ y\tts\in\tts\XXX,\, v\tts\in\tts\EEE}\Bigr) 
+ {\ff\over N}\bigl(w-c-\q-\qq\bigr)\\
&\qquad\qquad +
{\b\over N}\sum\nolimits_{ y\ts\in\ts\XXX,\,v\ts\in\ts\EEE}\Lmp_{y,\tts v}\Bigl(w
-W_{y,\tts v}+\bigl(1+r\bigr)\q+\bigl(\rho_y(K)+ 1-\dd\bigr)\qq\\
&\hbox to6.5cm{\hfill}+\ee_{ y}(K)\ts v-c - \q -\qq \Bigr) Q(x, y)P_{x, y}(u,v)\,.
\end{align*}
}%
With the substitution
%%
%%- -%@@%--{Eq: 2.3}--%%
{\abovedisplayskip=5pt plus 1.5pt minus 1.5pt\belowdisplayskip=5pt plus 1.5pt minus 1.5pt\belowdisplayshortskip=6pt plus 1.5pt minus 1.5pt
\begin{equation}\label{bsde}
\f\df \varphi + {\b}\sum\nolimits_{ y\ts\in\ts\XXX,\,v\ts\in\ts\EEE}\Lmp_{y,\tts v}Q(x, y)P_{x, y}(u,v)\,,
\end{equation}}%
equating to $0$ the derivatives of the Lagrangian
relative to $W_{y,\tts v}$, $c$, $\q$, and $\qq$ give, respectively,
%%
%%- -%@@%--{Eq: 2.4}--%%
{\abovedisplayskip=5pt plus 1.5pt minus 1.5pt\belowdisplayskip=7pt plus 1.5pt minus 1.5pt\belowdisplayshortskip=6pt plus 1.5pt minus 1.5pt
\begin{equation}\label{ze4}
\begin{gathered}
\Lmp_{y,\tts v}=\partial V_{t+1,\tts y,\tts \T_{t,\tts x}^y(F),\tts v}\bigl(W_{y,\tts v}/N\bigr)\,,\quad
\f={\Up}(c/N)\,,\\
\f=\bigl(1+r\bigr)\b\sum\nolimits_{ y\ts\in\ts\XXX,\,v\ts\in\ts\EEE}\Lmp_{y,\tts v}\,Q(x, y)P_{x, y}(u,v)\,,\\
\f=\b\sum\nolimits_{ y\ts\in\ts\XXX,\,v\ts\in\ts\EEE}\Lmp_{y,\tts v}\,\bigl(\rho_{ y}(K)
+1-\dd\bigr) Q(x, y)P_{x, y}(u,v)\,,
\end{gathered}
\end{equation}}%
and, after a straightforward application of the envelope theorem,
%%
%%- -%@@%--{Eq: 2.5}--%%
{\abovedisplayskip=5pt plus 1.5pt minus 5pt
%\abovedisplayshortskip=-7pt plus 1.5pt minus5.5pt
\belowdisplayskip=5pt plus 1.5pt minus 1.5pt
\belowdisplayshortskip=6pt plus 1.5pt minus 1.5pt
\begin{equation}\label{ze4a1}
\partial V_{t,\tts x,\tts F,\tts u} (w/N)=\varphi +{\b}\sum\nolimits_{ y\tts\in\tts\XXX,\,v\tts\in\tts\EEE}
\Lmp_{y,\tts v}\,Q(x, y)P_{x, y}(u,v)=\f={\Up}(c/N)\,.
\end{equation}}%
%It is clear from the last relation that $\Up\colon\lrbkt0,\infty\rlbkt\mapsto\lrbkt\infty,0\rlbkt$
%provides a homeomorphism between the physical consumption $c_t/N$ and the shadow variable $\f_t$.   
If~the system (\ref{ze2}-\ref{ze4})  
can be solved for $c$, $q$, $\qq$, $W_{y,v}$, $\f$, and
$\Lmp_{y,v}$, then any such solution must depend on the time period $t$, the aggregate state
$(x,F)$, and the employment state~$u$. If~this dependence must be emphasized in the notation,
we shall embellish the symbols $c$, $q$ and $\qq$ with subscripts in the obvious way
(recall that in the above $K$ and $r$ stand for $K_t(x,F)$ and $r_t(x,F)$).
Solving the savings problem requires an assumption
about the future distributions $\T_{t,x}^y(F)$, $y\in\nobreak\XXX$, i.e., an assumption 
about the structure of the transport mappings~$\T_{t,x}^y\phd$, $y\in\nobreak\XXX$~-- not just an
assumption about the statistical behavior of the aggregate and idiosyncratic shocks.
The solution to (\ref{ze2}-\ref{ze4}) depends also on the entering wealth (resource)
$w$, and this dependence may also enter the notation when needed. 
%%%%%%%%%%%%%%%%%
%%%%%%%%%%%%%

Our next step is to restate the KKT conditions in a more useful form. First, \eqref{ze2}
and \eqref{ze4} reduce to the following system of three equations for the unknowns $c$, $\q$ and
$\qq$, i.e., for $c_{t,\tts x,\tts \csd,\tts u}$, $\q_{t,\tts x,\tts \csd,\tts u}$ and
$\qq_{t,\tts x,\tts \csd,\tts u}\,$:
%%
%%
%%- -%@@%--{Eq: 2.6}--%%
{\abovedisplayskip=5pt plus 1.5pt minus 1.5pt\belowdisplayskip=7pt plus 1.5pt minus 1.5pt\belowdisplayshortskip=6pt plus 1.5pt minus 1.5pt
\begin{equation}\label{z2-no-lm}
\begin{gathered}
c/N+\q/N+\qq/N-w/N=0\,,\\
{\Up}(c/N) -  \sum_{ y\ts\in\ts\XXX,\,v\ts\in\ts\EEE}
(1+r)\,\b\,\partial V_{t+1,\tts y,\tts \T_{t,\tts x}^y(F),\tts v}\bigl(W_{y,\tts v}/N\bigr)\,
Q(x, y)P_{x,y}(u,v)=0\,,\\
{\Up}(c/N)- \sum_{ y\ts\in\ts\XXX,\,v\ts\in\ts\EEE}
\bigl(\rho_y(K)+1-\dd\bigr)\,\b\,
\partial V_{t+1,\tts y,\tts \T_{t,\tts x}^y(F),\tts v}\bigl(W_{y,\tts v}/N\bigr)\,
Q(x, y)P_{x,y}(u,v) =0\,,
\end{gathered}
\end{equation}
}%
where the expressions $W_{y,\tts v}$ were defined in \eqref{ze2}.
Next, observe that with \eqref{ze4a1} applied to period $t+1$, the first equation in \eqref{ze4}
states:
%%
{\abovedisplayskip=5pt plus 1.5pt minus 5pt
%\abovedisplayshortskip=-7pt plus 1.5pt minus5.5pt
\belowdisplayskip=5pt plus 1.5pt minus 1.5pt
\belowdisplayshortskip=3pt plus 1.5pt minus 1.5pt
\[
\Lmp_{y,\tts v}=\partial V_{t+1,\tts y,\tts \T_{t,x}^y(F),\tts v}\bigl(W_{y,\tts v}/N\bigr)
={\Up}(c_{t+1,\tts y,\tts \T_{t,x}^y(F),\tts v}/N)\,.
\]}%
It is now easy to remove the shadow variables from the KKT conditions
by casting the last two equations in \eqref{ze4} in
the familiar form of ``kernel conditions:''
%%
%%- -%@@%--{Eq: 2.7}--%%
{\abovedisplayskip=8pt plus 1.5pt minus 5pt\belowdisplayskip=8pt plus 1.5pt minus 1.5pt\belowdisplayshortskip=3pt plus 1.5pt minus 1.5pt
\begin{equation}\label{ze5} 
\begin{gathered}
1=\bigl(1+r\bigr)\,
\b\sum\nolimits_{ y\tts\in\tts\XXX,\,v\ts\in\ts\EEE}{{\Up}(c_{t+1,\tts y,\tts \T_{t,x}^y(F),\tts v}/N)
\over {\Up}(c_{t,\tts x,\tts F,\tts u}/N)} Q(x, y)P_{x, y}(u,v)\,,\\
1=\b\sum\nolimits_{ y\tts\in\tts\XXX,\,v\tts\in\tts\EEE}{{\Up}(c_{t+1,\tts y,\tts \T_{t,x}^y(F),\tts v}/N)
\over {\Up}(c_{t,\tts x,\tts F,\tts u}/N)}\bigl(\rho_{ y}(K)+1-\dd\bigr) Q(x, y)P_{x, y}(u,v)\,.
\end{gathered} 
\end{equation}}%
The meaning of these conditions is that in equilibrium all agents agree on the returns that
the two traded securities generate (the right sides in \eqref{ze5} are identical across all agents).


%%- -%##%--{No: 2.5}--%%
\begin{nit}{Assumption}
From now on we shall suppose that $U\phd$ is isoelastic, in which case
{\abovedisplayskip=8pt plus 1.5pt minus 5pt\belowdisplayskip=8pt plus 1.5pt minus 1.5pt\belowdisplayshortskip=6pt plus 1.5pt minus 1.5pt
$$
{{\Up}(c_{t+1,\tts y,\tts \T_{t,x}^y(F),\tts v}/N)\over {\Up}(c_{t,\tts x,\tts F,\tts u}/N)}
={\Up}\Bigl({c_{t+1,\tts y,\tts \T_{t,x}^y(F),\tts v}\over c_{t,\tts x,\tts F,\tts u}}\Bigr)\,.
$$}%
This relation allows for the removal of the number of households, $N$,
from the KKT conditions,
which is to say from the entire model. In particular, there will be no need for us to ever
take limits as the number of households $N$ increases to $\infty$, though, of course, we still
need to suppose that $N$ is ``large enough'' to utilize Glivenko-Cantelli's limiting result
mentioned earlier.\qed
\end{nit}

%%- -%##%--{No: 2.6}--%%
\begin{nit}{Remark}
%While this interpretation is not needed for what follows, it is instructive to note that
%\eqref{bsde} tracks the backward equation in the maximum principle.
Clearly,
$\Up\colon\lrbkt0,\infty\rlbkt\mapsto\lrbkt0,\infty\rlbkt$
provides a homeomorphism between
$c_{t,\tts x,\tts F,\tts u}/N$ and $\f$ and between $c_{t+1,\tts y,\tts \T_{t,x}^y(F),\tts v}/N$ and
$\Lmp_{y,v}$. In particular, the system \eqref{ze5} does not really exclude the shadow
variables, as it retains their homeomorphic copies.
%Hence, the consumption stream is governed by backward dynamics that track the backward dynamics
%of the dual variables in the maximum principle.
Thus, consumption plays multiple rôles: it~is a household
descriptor, state variable, control parameter, and, up to a homeomorphism, a dual variable.\qed
%Later we will see the importance of the connection between future and present consumption,
%$c_{t+1,y,v}$ and $c_{t,x,u}$, which is essentially a connection between future and present shadow variables.\qed
\end{nit}
%\nitskip

All three equations in \eqref{z2-no-lm} define the vector $(c,\q,\qq)$ as an implicit
$\C^1$-function of the entering wealth $w$ and the last two equations in \eqref{z2-no-lm} define
the portfolio vector $(\q,\qq)$ as an implicit $\C^1$-function of the consumption level $c\in\Rpp$. This
last feature is instrumental in what follows, as it removes the need to specify the bounds of the
entering wealth in the outset, and, ultimately, endogenize those bounds. The next theorem makes
this result precise. 

%%
%%- -%##%--{No: 2.7}--%%
\begin{nit}{Theorem}\label{thm1}
If  $w\in\R$ is such that $V_{t,x,F,u}(w/N)$ is finite and
the system \eqref{z2-no-lm} admits a solution $(c,\q,\qq)$,
then \eqref{z2-no-lm} admits a unique solution 
for every entering wealth from some open neighborhood of $w$ and that solution is a $\C^1$-function 
of the entering wealth with $\partial c>0$.  
Moreover, $V_{t,x,F,u}\phd$ is a $\C^1$-mapping, and hence also a $\C^2$-mapping, with 
$\partial V_{t,x,F,u}\phd > 0$ and $\partial^2 V_{t,x,F,u}\phd<0$ in some neighborhood of $w/N$.
In~addition,  if the system composed of the
last two equations in \eqref{z2-no-lm} admits a solution $(\q,\qq)$ for some
fixed $c\in\Rpp$, then that system admits a unique solution  for every consumption level
in some open neighborhood of~$c$ and that solution is 
a $\C^1$-function in the neighborhood of~$c$. All results continue to hold if one of the traded
assets is removed from the model, i.e., the households invest only in the private lending
instrument, or only in productive capital.\qed
\end{nit}

The proof of \ref{thm1} is given in Appendix~\ref{sec:A2}.
One consequence from this theorem is that the value function is a strictly concave $\C^2$-function
on any domain in which it is finite, and this property holds in all time periods and all
realizations of the aggregate and idiosyncratic states. 
The theorem also shows that there is a one-to-one correspondence between optimal consumption
and entering wealth. Since in every period the
households differ only in their entering wealth and state of employment, they can be distinguished
just as well by their state of employment and consumption level~-- a~strategy that we have adopted
already. We stress that although households are identified as elements of $\EEE\times\Rpp$, this
set is not a true labeling set, in the sense that an element $(u,c)\in\EEE\times\Rpp$ identifies a
collection of households that are indistinguishable as economic agents, rather than a single
physical household. Furthermore, the consumption level and state of employment of any one household
change over time.

Another crucially important consequence from the last theorem is that resolving
the private savings problem comes down to assigning a consumption level $c\in\Rpp$ to any period
$t<T$ and any realized aggregate and idiosyncratic state in such a way that the system composed of the
last two equations in \eqref{z2-no-lm} admits a solution $(\q(c),\qq(c))\in\R^2$ and this solution is
such that $c+\q(c)+\qq(c)$ exactly matches the entering wealth. The existence of such an assignment
ensures both: the value function remains finite (consumption is always strictly positive) and the
KKT conditions hold. As the system \eqref{z2-no-lm} depends on the time period $t$, the aggregate
state $(x,F)$ and the private state $u$, then so does also the solution $(\q(c),\qq(c))$. In~most
cases this dependence needs to be emphasized in the notation and
we shall often write $\q_{t,\tts x,\tts F,\tts u}(c)$ and $\qq_{t,\tts x,\tts F,\tts u}(c)$.
Of course, these objects depend
also on the selection of aggregate installed capital, risk-free rate and transport mappings. 
%Another reason not to think of
%$\EEE\times\Rpp$ as a labeling set is that the employment and consumption levels associate with a
%given household do change over time. 

The issue to address next is the compatibility of the optimal private
allocations. Let us assume that, with a given choice for the initial population 
distribution, for the collection of transport mappings, and for the assignment of collectively
installed capital and interest to every period and aggregate state of the economy, all household
are able to solve their private savings problem with a unique optimal allocation along every
realized path of the productivity state and the private employment state.
In general, such allocations have no reason to be consistent in that: (a)~exercising all optimal
private policies may generate transport mappings that are different from the given ones, (b)~the
aggregation of all private capital investments may differ from the given ones, and (c)~the given
interest rates may generate
aggregate demands for borrowing and lending that do not match. This is the matter
addressed in the rest of this section.


%%- -%##%--{No: 2.8}--%%
\begin{nit}{Global general equilibrium}\label{def-equil}
In the context of the economy introduced above, global general equilibrium, or simply
equilibrium, is given by:
\smallskip

(1)~an initial population distribution $F_0\in\DB^\EEE$ and a collection of transport mappings
$\T_{t,x}^y\colon \DB^\EEE \mapsto \DB^\EE$, $x,y\in\XXX$, $0\le t<T$;

(2)~a collection of mappings $K_t\colon\XXX\times\DB^\EEE \mapsto\Rpp$ and
$r_t\colon\XXX\times\DB^\EEE \mapsto\lrbkt -1,\infty\,\rlbkt $, $0\le t<T$,
\smallskip

\noindent
all chosen so that, for any realization of the productivity state
$(x_t\in\XXX)_{0\tts\le\tts t\tts\le \tts T}$, the realized population distribution in period~$t$
is exactly $F_0$ if $t=0$ and exactly
$$
F_t\df \bigl(\T_{t-1,x_{t-1}}^{x_t}\circ \T_{t-2,x_{t-2}}^{x_{t-1}}\circ \cdots\circ \T_{0,x_{0}}^{x_{1}}\bigr)(F_0)
\quad\text{if \ $0<t\le T$}\,,
$$
the aggregate installed capital is exactly $K_t(x_t,F_t)$ and the
aggregate demand for the risk free private lending instrument
exactly~$0$ in all periods $0\le t<T$,
provided that all households choose their savings policies by way of solving for their private KKT
conditions with population distribution $F_t$, with installed capital $K_t(x_t,F_t)$, with interest
$r_t(x_t,F_t)$, and with a collection of transport mappings
$\T_{t,x_t}^y$, $y\in\XXX$, and provided  this choice results
into a strictly positive consummation for every household at all times
and in all aggregate and private states.\qed
\end{nit}


The main step in the calculation of the global general equilibrium is to establish the connections
across time between the population distributions
${(\csd_{t})}_{0\tts\le\tts t\tts\le\tts T}$ that all private KKT conditions and
the notion of equilibrium dictate. This is the task we turn to next.

%%- -%##%--{No: 2.9}--%%
\begin{nit}{Random Monge assignments in random environment}\label{rnd-Monge}
Households that are in the same state of employment, choose the same consumption level, and
experience the same shock in employment, would choose the same consumption level during the next
period, too. Let $\pfc_{t,x,F}^{ y,v}(u,c)$ denote the period $t+1$ consumption level
of any household that happens to be of type $(u,c)$ during period $t$, when the economy is in
state $(x,F)$, provided that during period $t+1$ the household faces
transition to employment state $v\in\EE$ and
the productivity state transitions to $y\in\XXX$.
The assignment
%%
{\abovedisplayskip=5pt plus 1.5pt minus 1.5pt\belowdisplayskip=5pt plus 1.5pt minus 1.5pt\belowdisplayshortskip=3pt plus 1.5pt minus 1.5pt
$$
\EEE\times\Rpp \ni (u,c) \leadsto (v,\pfc_{t,x,F}^{ y,v}(u,c))\in \EEE\times\Rpp
$$
}%
is analogous to a Monge assignment, except that it is not pure, in that it depends on the
idiosyncratic shock in employment, which differs across the population of households that are of
type $(u,c)$. In addition to being random in that sense, this assignment takes place in the random environment
determined by the transition in productivity from $x$ to $y$.
In what follows the mappings $\pfc_{t,x,F}^{ y,v}(u,\cdot)$ are often referred to as consumption
transfer mappings.\qed
\end{nit}


Basic intuition
suggests that the mappings
$\pfc_{t,x,F}^{ y,v}(u,\cdot)$ must be nondecreasing,%%%%%%%%%%%%
%%%%%%%%%%%%%%%%%%%%%%%%%%%%%
\footnote{An agent who consumes at least as much as another agent during the present period will
consume at least as much during the next period as well, if both agents experience the same shock in their
employment status.}
%%%%%%%%%%%%%%%%%%%%%%%%%%%%%
and we shall
seek equilibria in which these mappings are also continuous.%%%%%%%%%%%%%%%%%%%%%%%
%%%%%%%%%%%%%%%%%%%%%%%%%%%%%%%%%%%%%%%%%%%
\footnote{Conceptually, right continuity, in addition to monotonicity (not necessarily strict), of the mappings
$\pfc_{t,x,F}^{ y,v}(u,\cdot)$ is sufficient for what we need.} 
%%%%%%%%%%%%%%%%%%%%%%%%%%%%%%%%%%%%%%%%%%%
Consequently, all functions
$\pfc_{t,x,F}^{ y,v}(u,\cdot)$ have inverses given~by
%%- -%@@%--{Eq: 2.8}--%%
{\abovedisplayskip=5pt plus 1.5pt minus 1.5pt\belowdisplayskip=5pt plus 1.5pt minus 1.5pt\belowdisplayshortskip=3pt plus 1.5pt minus 1.5pt
\begin{equation}\label{inversion}
\Rpp\ni \a \leadsto \hat\pfc_{t,x,F}^{ y,v}(u,\a)\df\inf\{c\in\Rp\colon \pfc_{t,x,F}^{ y,v}(u,c)\le \a\}\,.
\end{equation}}%
The postulated features of $\pfc_{t,x,\csd}^{ y,v}(u,\cdot)$ guarantee that
$c\le\hat\TT_{t,x,F}^{ y,v}(u,\a)$ and  $\pfc_{t,x,F}^{ y,v}(u,c)\le \a$ are equivalent relations.
It is clear that the assignments $\pfc_{t,x,F}^{ y,v}(\cdot,\cdot)$ govern the transport of the
population distribution from period $t$ to $t+1$, as is clarified next.

%%
%%- -%##%--{No: 2.10}--%%
\begin{nit}{Proposition}\label{main-q}
The transport mappings $\T_{t,x}^y\colon\DB^\EEE\mapsto\DB^\EEE$
obtain from the random Monge assignments through the equation
{\abovedisplayskip=5pt plus 1.5pt minus 1.5pt\belowdisplayskip=5pt plus 1.5pt minus 1.5pt\belowdisplayshortskip=3pt plus 1.5pt minus 1.5pt
\begin{equation*}\tag{${\text{d}}_t$}\label{zze666}    
%\begin{multlined}
\T_{t,x}^y(F)^{v}(\a) = \sum\nolimits_{\,u\ts\in\ts\EEE} 
{\gp_x(u) P_{x, y}(u, v)\over \gp_ y( v)} 
\csd^{u}(\,\hat\pfc_{t,x,F}^{ y,v}(u,\a))\,,\ \ \a\in\Rpp\,,\ v\in\EEE\,,\ F\in\DB^\EEE
%\end{multlined} 
\end{equation*}}%
for all $t<T$ and all $x,y\in\XXX$.\qed
\end{nit}

Most of what follows builds on the last result, the justification of which
is straightforward:
%
%
%%- -%##%--{No: 2.11}--%%
\begin{nit}{Proof of \ref{main-q}}
The productivity states in periods $t$ and
$t+1$, $x$ and $ y$, being fixed, the left side in (\ref{zze666}) gives the
probability%%%%%%%%%%%%%%%%%%%%%%%%%%%%
%%%%%%%%%%%%%%%%%%%%%%%%%%%%%%%%%%%%%%%
\footnote{In this context ``probability'' is understood in terms of ``counting,'' and refers
to a hypothetical sampling scheme in which every household from a given employment class has equal
chance to be selected.}
%%%%%%%%%%%%%%%%%%%%%%%%%%%%%%%%%%%%%%%
for a randomly sampled household, from those that are in state $v$ during period
$t+1$, to have consumption level in period $t+1$
not exceeding $\a$,
the ratio ${\gp_x(u) P_{x, y}(u, v)/\gp_ y( v)}$ gives the probability for the same household to
come from state~$u$ in period~$t$, and, lastly, given that the household comes from state
$u\in\EEE$, the probability that its period $t+1$ consumption does not exceed $\a$ is the same as
the probability that its period $t$ consumption~$c$ is such that $\pfc_{t,x,\csd}^{ y,v}(u,c)\le \a$,
which property is the same as $c\le\hat\pfc_{t,x,\csd}^{ y,v}(u,\a)$.%, i.e., the same as the consumption
%level in period $t$ in employment state $u$ not exceeding $\hat\pfc_{t,x,u}^{ y,v}(u)$, which is
%precisely $\csd_{t,x,u}(\,\hat\pfc_{t,x,u}^{ y,v}(u))$.
\qed
\end{nit}



%%
%%- -%##%--{No: 2.12}--%%
\begin{nit}{Remark}\label{n2}
The action on the distribution $F$ encrypted in the right side of~(\ref{zze666}) can
be interpreted as a ``Bayesian push-forward'' through the 
random assignments  $\pfc_{t,x,F}^{ y,v}$, but one must be aware that these assignments depend on
the distribution that is being pushed. Furthermore, the transformation
that~(\ref{zze666}) describes takes place in a random environment, namely, the one
associated with the realized productivity state $x$ and the transition $x \to y$.
It~is instructive to note that, conditioned to the random
environment not changing (the productivity state remains the same for a while), the law
of motion in the space $\DB^\EEE$ given by~(\ref{zze666})
becomes deterministic but not constant~-- the importance of this feature
will be illustrated in Sec.~\ref{sec:KS}.

The~rôle of the
transport mappings $\T_{t,x}^y\phd$ is similar to that of the FP equation in the common approach,
but cannot be linked in the same way to a particular Bellman (HJB) equation~-- see~\ref{1350}.
Furthermore, the intrinsic nature of these (generally, nonlinear)
transformations is rooted in Lagrange duality (through
the inverses $\hat\pfc_{t,x,u}^{ y,v}\phd$), not in 
the transposition of an infinitesimal generator.
Most important, both sides of~(\ref{zze666}) represent cross-sectional distribution of the
entire population and neither side has the meaning of probability distribution attached to any particular
Markovian state. 
%The~way in which the transport in (\ref{main-q}\ref{zze666}) will be used later on will take us
%even further away from the classical framework, since the mappings
%$\hat\pfc_{t,x,u}^{ y,v}\phd$ will have to be made to depend implicitly on both $\csd_{t+1, y}$ and~$\csd_{t,x}$.
We will see below that (\ref{zze666}) can only be solved
simultaneously with the other first order and market clearing conditions, which, in turn, can only
be solved simultaneously with (\ref{zze666}). This simultaneity is unavoidable and is
the main difficulty to overcome.\qed 
\end{nit}

In order to address the time inconsistency noted earlier in a more practical fashion,
we must find a way to mimic the approach proposed in \nocite{DL12}{[DL]}. The idea is to break the large 
time inconsistent system 
across all periods and all aggregate and idiosyncratic states into smaller (time inconsistent,
still) systems that can be chained into a computable backward induction program. 

%%- -%##%--{No: 2.13}--%%
\begin{nit}{Bi-period backward dynamics}\label{cross-sys}
For any period $0\le t<T$ and for any aggregate state $(x,\csd)\in\XXX\times\DB^\EEE$
and employment state $u\in\EEE$ associated with that period,
define the system ($x$, $\csd$ and $u$ are given and fixed): 
%%
{\abovedisplayskip=5pt plus 1.5pt minus 1.5pt\belowdisplayskip=5pt plus 1.5pt minus 1.5pt\belowdisplayshortskip=3pt plus 1.5pt minus 1.5pt
\begin{equation*}\tag{$\text{n}_t$}\label{zze5a} 
\begin{gathered}
1=\bigl(1+r_{t}(x,\csd)\bigr)
\b\sum\nolimits_{ y\ts\in\ts\XXX,\,v\ts\in\ts\EEE}{{\Up}\bigl(\pfc_{t,x,\csd}^{ y,v}(u,c)/c\bigr)} Q(x, y)P_{x, y}(u,v)\,,\\
1=\b\sum\nolimits_{ y\ts\in\ts\XXX,\,v\ts\in\ts\EEE}{{\Up}\bigl(\pfc_{t,x,\csd}^{y,v}(u,c)/c\bigr)}
\bigl(\rho_{ y}(K_{t}(x,\csd))+1-\dd\bigr) Q(x, y)P_{x, y}(u, v)\,,
\end{gathered}
\end{equation*}
}%
%%
{\abovedisplayskip=5pt plus 1.5pt minus 1.5pt\belowdisplayskip=5pt plus 1.5pt minus 1.5pt\belowdisplayshortskip=3pt plus 1.5pt minus 1.5pt
\begin{equation*}\tag{$\text{e}_{t+1}$}\label{zze5xb} 
\begin{gathered}
\begin{multlined}
(1+r_{t}(x,\csd)){\q_{t,x,\csd,u}(c)} + 
\bigl(\rho_ y(K_{t}(x,\csd))+1-\dd\bigr){\qq_{t,x,\csd,u}(c)}
+\ee_ y(K_{t}(x,\csd))v \qquad\qquad\\
\qquad ={\pfc_{t,x,\csd}^{ y,v}(u,c)}
+ {\q_{t+1, y,\T_{t,x}^y(\csd),v}\bigl(\pfc_{t,x,\csd}^{ y,v}(u,c)\bigr)}
+ {\qq_{t+1, y,\T_{t,x}^y(\csd),v}\bigl(\pfc_{t,x,\csd}^{ y,v}(u,c)\bigr)}\,,\\
\text{for all }\   y\in\XXX\,,\ v \in\EEE\,,
\end{multlined} 
\end{gathered}
\end{equation*}
}%
%%
{\abovedisplayskip=5pt plus 1.5pt minus 1.5pt\belowdisplayskip=5pt plus 1.5pt minus 1.5pt\belowdisplayshortskip=3pt plus 1.5pt minus 1.5pt
\begin{equation*}\tag{$\text{m}_{t}$}\label{ze9}
\begin{gathered}
\sum\nolimits_{u\ts\in\ts\EEE}\gp_x(u)\int_0^\infty{\q_{t,x,\csd,u}(c)}\d \csd^u(c)=0\,,\\
\sum\nolimits_{u\ts\in\ts\EEE}\gp_x(u)\int_0^\infty{\qq_{t,x,\csd,u}(c)}\d \csd^u(c) =
K_{t}(x,\csd)\,,
\end{gathered}
\end{equation*}
}%
%% 
in which \eqref{zze5a}  and \eqref{zze5xb} are understood as identities between functions of
$c\in\Rpp$ and the distributions $\T_{t,x}^y(\csd)\in\DB^\EEE$, $y\in\XXX$,
are given by \ref{main-q}-(\ref{zze666}).
Thus, the system (\ref{zze5a}-\ref{zze5xb}-\ref{ze9}) captures the dynamics of the entire
population of agents.
We~stress that the kernel (price agreement)
conditions \eqref{zze5a} and the market clearing conditions \eqref{ze9} are associated with
period~$t$, while the balanced budget 
conditions \eqref{zze5xb} are associated with period $t+1$ and run across all aggregate and idiosyncratic
states that can occur in period $t+1$. The collection of functions $\q_{t+1, y,\tilde\csd,v}\phd$ and
$\qq_{t+1,y,\tilde\csd,v}\phd$, for all choices of $y\in\XXX$, $\tilde\csd\in\DB^\EEE$ and $v\in\EEE$, are
assumed given and the unknowns are the functions $\q_{t,x,\csd,u}\phd$, $\qq_{t,x,\csd,u}\phd$ and 
$\pfc_{t,x,\csd}^{y,v}(u,\cdot)$, $y\in\XXX$, $v\in\EEE$.
Hence, the dynamics encrypted in the system
(\ref{zze5a}-\ref{zze5xb}-\ref{ze9}) step backward in
time, but not from period $t+1$ to period $t$; instead, the step is from periods $t$ and $t+1$ to
periods $t-1$ and $t$.
The~rôle of these dynamics parallels that of the Bellman equation (and the
associated backward induction) in the
common strategy, but there are 
structural differences that must be spelled out. First, some of the unknowns (the security demands)
are attached to period $t$, while others (the next period consumption levels in all future states)
are attached to period~$t+1$.  
The inherent time inconsistency comes from the fact that the list of unknowns includes the assignments
$(u,c) \leadsto \pfc_{t,x,\csd}^{ y,v}(u,c)$ and, at the same time, the structure of the entire system
depends on the action of these assignments through the period $t+1$ population distributions
$\T_{t,x}^y(\csd)$ produced in \ref{main-q}-(\ref{zze666}), i.e., the structure of the system
to be solved depends on the values of the unknowns. This also means that the backward step
in (\ref{zze5a}-\ref{zze5xb}-\ref{ze9}) can only be accomplished simultaneously with the
forward step in (\ref{zze666}), i.e., the system to solve is
(\ref{zze5a}-\ref{zze5xb}-\ref{ze9}-\ref{zze666}).  

Finally, the collection of systems
$\text{(\ref{zze5a}-\ref{zze5xb}-\ref{ze9}-\ref{zze666})}_{\tts 0\tts \le\tts t\tts \le\tts  T-1}$,
exhausts all market clearing, consistency
and private KKT conditions across all time periods and all states, except for the balanced budget
conditions in period~$t=0$. Because during this initial period all households
share the same entering wealth, all
households who happen to be in employment state $u\in\EEE$ are identical and thus choose the same
consumption level $\bar c_{u}$.  In particular, the period $t=0$ population distribution
$\csd_0\in\DB^\EEE$ has the form $(\csd_0)^u(c)=0$ for $c< \bar c_{u}$ and
$(\csd_0)^u(c)=1$ for $c\ge \bar c_{u}$. There are $\abs{\EEE}$ balanced budget
equations attached to period $t=0$, namely, 
%%
{\abovedisplayskip=5pt plus 1.5pt minus 1.5pt\belowdisplayskip=5pt plus 1.5pt minus 1.5pt\belowdisplayshortskip=3pt plus 1.5pt minus 1.5pt
\begin{equation*}\tag{$\text{e}_0$}\label{zze5ab}
\bar c_{u} + {\q_{0,x,\csd_0,u}(\bar c_{u})} + {\qq_{0,x,\csd_0,u}(\bar c_{u})}
= \bigl(\rho_x(K_{-1}) + 1-\dd\bigr)K_{-1} + \ee_x(K_{-1})u\,,\quad u\in\EEE\,,
\end{equation*}
}%
which are to be solved for the unknowns $\bar c_u$, $u\in\EEE$.\qed 
\end{nit}

The recursive program for solving the global system 
$\text{(\ref{zze5a}-\ref{zze5xb}-\ref{ze9}-\ref{zze666}-\ref{zze5ab})}_{0\tts \le\tts t\tts\le\tts T-1}$
from
\ref{main-q} and \ref{cross-sys} now suggests itself. 
At every step (associated with period $t$) the program must compute the demand
functions
%
{\abovedisplayskip=5pt plus 1.5pt minus 1.5pt\belowdisplayskip=5pt plus 1.5pt minus 1.5pt\belowdisplayshortskip=3pt plus 1.5pt minus 1.5pt
$$
\XXX\times\DB^\EEE\times\EEE\times\Rpp\ni (x,\csd,u,c) \leadsto \q_{t,x,\csd,u}(c)\,,\qq_{t,x,\csd,u}(c)\in\R\,,
$$
}%
while taking the demand functions
{\abovedisplayskip=5pt plus 1.5pt minus 1.5pt\belowdisplayskip=5pt plus 1.5pt minus 1.5pt\belowdisplayshortskip=3pt plus 1.5pt minus 1.5pt
$$
\XXX\times\DB^\EEE\times\EEE\times\Rpp\ni (y,\tilde\csd,v,\tilde c) \leadsto
\q_{t+1,y,\tilde \csd,v}(\tilde c)\,,\qq_{t+1,y,\tilde\csd,v}(\tilde c)\in\R\,,
$$
}%
as given, i.e., already computed during the previous step (associated with period $t+1$) if $t<T-1$
or taken to be $0$ if $t=T-1$. This strategy parallels the familiar backward induction in dynamic
programming, except that every step involves two periods instead of one and, most
important, involves the random Monge assignments $(u,c) \leadsto \pfc_{t, x, \csd}^{y, v}(u,c)$,
which, in turn, capture connections between the dual (shadow) variables. In that sense, the
program parallels also the maximum principle and can be seen as a sequential
Monge-Kantorovich transport of mass problem in the dual space.
The more detailed description of the strategy just
outlined is the following. 

%%- -%##%--{No: 2.14}--%%
\begin{nit}{General metaprogram}\label{main-proc}
{\it Initial Backward Step:} Set $t=T-1$ and for every $x\in\XXX$ do:
{\parskip=0pt\nobreak

For every choice of the distribution (state variable)
$\csd\in\DB^\EEE$ do:

{ (1)}~Make an ansatz choice for the values $K_{t}(x,\csd)$ and $r_{t}(x,\csd)$. Go to (2).

{ (2)}~For every fixed
$(u,c)\in\EEE\times\Rpp$ solve \ref{def-equil}-(\ref{zze5a}-\ref{zze5xb}) with
$\q_{t+1, y, \T_{t,x}^y(\csd), v}\equiv 0$ and with 
$\qq_{t+1, y, \T_{t,x}^y(\csd), v}\equiv 0$
(total of $\abs{\XXX}\times\abs{\EEE}+2$ equations) for the (same number of)
unknowns:
{\abovedisplayskip=5pt plus 1.5pt minus 1.5pt\belowdisplayskip=5pt plus 1.5pt minus 1.5pt\belowdisplayshortskip=3pt plus 1.5pt minus 1.5pt
$$
\{\pfc_{t, x, \csd}^{ y,v}(u,c)\colon  y\in\XXX,\,v \in\EEE\}\,,\quad
\q_{t, x, \csd, u}(c)\,, \quad 
\qq_{t, x, \csd, u}(c)\,.
$$
}%

{ (3)}~Test the market clearing conditions \ref{def-equil}-(\ref{ze9}).
If at least one of these conditions fails by more than some prescribed threshold,
go back to (1) with appropriately revised values for $K_{t}(x,\csd)$ and $r_{t}(x,\csd)$; otherwise stop
and record (i.e., accept) the most recently computed scalars $K_{t}(x,\csd)$ and $r_{t}(x,\csd)$ and functions
$\q_{t,\tts x,\tts \csd,\tts u}(\cdot)$, $\qq_{t,\tts x,\tts \csd,\tts u}(\cdot)$ and
$\pfc_{t,\tts x,\tts \csd}^{ y,v}(u,\cdot)$, for all $y\in\XXX$ and $v\in\EEE$. Proceed to
the next step.
}

%\noindent
{\it Generic Backward Step:}  If $t-1<0$, go to the final backward step below; else set $t=t-1$ and
for every $x\in\XXX$ do:

For every choice of the distribution (state variable) $\csd\in\DB^\EEE$ do:

{ (1)}~Make an ansatz choice for the values $K_{t}(x,\csd)$ and $r_{t}(x,\csd)$.
Set $\csdp =\csd$ for every $y\in\XXX$
(the next period distribution, i.e., state variable, is initially guessed to be the
same as the one in the 
present period, irrespective of the realized future productivity state~$y$).

{ (2)}~For every fixed
$(u,c)\in\EEE\times\Rpp$ solve \ref{def-equil}-(\ref{zze5a}-\ref{zze5xb})
with $\T_{t,x}^y(\csd)$ replaced by $\csdp$ (total of $\abs{\XXX}\times\abs{\EEE}+2$ equations)
for the (same number of) unknowns  $\{\pfc_{t,x,\csd}^{ y,v}(u,c)\colon  y\in\XXX,\,v \in\EEE\}$,
$\q_{t,x,\csd,u}(c)$ and $\qq_{t,x,\csd,u}(c)$. Go to (3).

{ (3)}~With the computed functions $\pfc_{t,x,\csd}^{ y,v}(u,\cdot)$,
which now depend on the choice of $\csdp$, $y\in\XXX$, compute the distributions
$\csds\in\DB^\EEE$, $y\in\XXX$, as (see \ref{main-q})
{\abovedisplayskip=5pt plus 1.5pt minus 1.5pt\belowdisplayskip=5pt plus 1.5pt minus 1.5pt\belowdisplayshortskip=8pt plus 1.5pt minus 1.5pt
\[
(\pst\csd_{y})^v(\a) = \sum\nolimits_{\,v\ts\in\ts\EEE} 
{\gp_x(u) P_{x, y}(u, v)\over \gp_ y( v)} 
\csd^{u}(\hat\pfc_{t,x,\csd}^{ y,v}(u,\a))\,,\quad%\text{for all }\ 
\a\in\Rpp\,,\ \ v \in\EEE\,.
\]}%
If the largest Kolmogorov-Smirnov distance between $(\pst\csd_{y})^v\phd$ and the guess
$(\pdg\csd_{y})^v \phd$, for the various choices of $y\in\XXX$ and $v \in\EEE$,
is not acceptably close to~$0$, set $\csdp=\csds$ and go back
to (2) without changing $K_{t}(x,\csd)$ and $r_{t}(x,\csd)$; otherwise, proceed to (4).

{\ (4)}~Test the market clearing conditions (see \ref{def-equil}-(\ref{ze9}))
{\abovedisplayskip=7pt plus 1.5pt minus 1.5pt\belowdisplayskip=7pt plus 1.5pt minus 1.5pt\belowdisplayshortskip=8pt plus 1.5pt minus 1.5pt
\begin{equation*}
\sum\nolimits_{u\ts\in\ts\EEE}\gp_x(u)\int_0^\infty{\q_{t,x,\csd,u}(c)}\d \csd^u(c)=0
\ \text{and}\ 
%\noalign{\rlap{and, in addition,}}
\sum\nolimits_{u\ts\in\ts\EEE}\gp_x(u)\int_0^\infty{\qq_{t,x,\csd,u}(c)}\d \csd^u(c) = K_{t}(x,\csd)\,.
\end{equation*}}%
If at least one of these conditions fails by more than some prescribed threshold, go back to (1)
with appropriately revised values for $K_{t}(x,\csd)$ and $r_{t}(x,\csd)$
($x\in\XXX$ and $\csd\in\DB^\EEE$ remain fixed); otherwise stop
and record (i.e., accept) the most recently obtained scalars $K_{t}(x,\csd)$ and $r_{t}(x,\csd)$ and functions
$\q_{t,x,\csd,u}\phd$, $\qq_{t,x,\csd,u}\phd$ and
$\pfc_{t,x,\csd}^{ y,v}(u,\cdot)$, $y\in\XXX$, $v\in\EEE$. Go to the beginning of the generic
backward step.
%\smallskip 

%\noindent
{\it Final Backward Step:}~For every $x\in\XXX$ do:

For every $u\in\EEE$ determine the period $t=0$
consumption level $\bar c_u$ for all households in employment state~$u$ 
(all households in employment category $u$ are identical in period $t=0$ and consume the same
amount) by solving the following system of $\abs{\EEE}$ equations (see
\ref{def-equil}-(\ref{zze5ab}))
{\abovedisplayskip=7pt plus 1.5pt minus 1.5pt\belowdisplayskip=7pt plus 1.5pt minus 1.5pt\belowdisplayshortskip=8pt plus 1.5pt minus 1.5pt
\[
\bar c_u + {\q_{0,x,u,{\csd_{0}}}(\bar c_u)} + {\qq_{0,x,u,{\csd_{0}}}(\bar c_u)} = \bigl(\rho_x(K_{-1})
+1-\dd\bigr)K_{-1} + u\,\ee_x(K_{-1})\,,\quad u\in\EEE\,,
\]}%
in which $\csd_{0}\in\DB^\EEE$
is given by $(\csd_{0})^u(c)=1$ if $c\ge \bar c_u$ and $(\csd_{0})^u(c)=0$ if
$c< \bar c_u$, \ $u\in\EEE\,$.
%\smallskip 

%\noindent
{\it Initial Forward Step:}~In period $t=0$ the initial productivity state $x\in\XXX$ is revealed and so
is also the (idiosyncratic) employment state of every household.
As all households in employment category $u\in\EEE$ have the same income in period $t=0$ and are faced
with the same uncertain future, they are identical and adopt the same
consumption plan $\bar c_u$, calculated during the final backward step.
Define the period $t=0$ population distribution
$\csd_{0}\in\DB^\EEE$ as the corresponding list of Heaviside step functions ($\abs{\EEE}$ in number).
As all quantities
$K_{0}(x,\csd)$,  $r_{0}(x,\csd)$, $\q_{0,x,\csd,u}(c)$ and $\qq_{0,x,\csd,u}(c)$ 
have been precomputed for every $\csd\in\DB^\EEE$ and $c\in\Rpp$, the period $t=0$ installed productive
capital $K_{0}(x,\csd_{0})$ is available, and so is also the period $t=0$ exiting portfolio,
$\{\q_{0,x,\csd_{0},u}(\bar c_u),\qq_{0,x,\csd_{0,x},u}(\bar c_u)\}$, for all
households in employment state $u\in\EEE$.
%\smallskip

%\noindent
{\it Generic Forward Step:} The economy exits period $(t-1)$ from productivity state $x\in\XXX$ and
population distribution $\csd_{t-1}$ and in period $t$ enters a new
productivity state $ y\in\XXX$.
As~all $(t-\nobreak1)$-to-$t$ Monge assignments
$(u,c) \leadsto \pfc_{t-1,x,\csd}^{ y,v}(u,c)$ are available from the backward steps for all period
$(t-1)$ population distributions $\csd\in\DB^\EEE$, the period $t$ consumption levels of all
households become known:
a period $(t-1)$ household of type $(u,c)\in\EEE\times\Rpp$ %%%%%%%%%%%%% 
%%%%%%%%%%%%%%%%%%%%%%%%%%%%%%%%%%%%%%%%%%
%\footnote{After period $t=0$ households that are in the same employment state may no longer be identical.}
%%%%%%%%%%%%%%%%%%%%%%%%%%%%%%%%%%%%%%%%%%
that changes employment from $u$ to $v$ becomes, during period~$t$, household of type 
$(v,\tilde c)$ with $\tilde c = \pfc_{t-1,x,\csd_{t-1}}^{y,v}(u,c)$.
The period $t$  population distribution is then given by
{\abovedisplayskip=4pt plus 1.0pt minus 1.5pt\belowdisplayskip=4pt plus 1.0pt minus 1.5pt\belowdisplayshortskip=4pt plus 1.0pt minus 1.5pt
\begin{equation*}
(\csd_{t})^v(\a) = \sum\nolimits_{\,u\,\in\,\EEE} 
{\gp_x(u) P_{x, y}(u, v)\over \gp_ y( v)}\,
(\csd_{t-1})^u(\hat\pfc_{t-1,x,\csd_{t-1}}^{ y,v}(u,\a))\,,\quad%\text{for all }\  
\a\in\Rpp\,,\ \ v \in\EEE\,.
\end{equation*}}%
As all quantities $K_{t}(y,\csd)$, $r_{t}(y,\csd)$, $\q_{t, y,\csd,v}(\tilde c)$
and $\qq_{t, y,\csd,v}(\tilde c)$, assumed to be $0$ if $t=T$,
have been precomputed during the backward steps for every period $t$
population distribution  $\csd\in\DB^\EEE$ and all individual consumption levels $\tilde c$,
they are meaningful with $\csd=\csd_{t}$, and this means that the period $t$ installed
productive capital is  $K_{t}(y,\csd_{t})$, the agreed upon interest is $r_{t}(y,\csd_{t})$,
while the period $t$ exiting portfolio of any household
of type $(v,\tilde c)$ is
$\{\q_{t, y,\csd_{t},v}(\tilde c),\,\qq_{t, y,\csd_{t},v}(\tilde c)\}$.\qed
\end{nit} 
%\nitskip




The metaprogram described in \ref{main-proc} differs from other similar procedures in a number of respects.
First, all backward steps involve the computation of future consumption, and, in that
sense, while moving backward globally, each step has a forward looking sub-step,
so that the entire procedure is (locally forward)-backward-forward.
Another peculiarity is that the Monge assignments, or their inverses rather,
in the right side of \ref{main-q}-(\ref{zze666}) depend not only on the
population distribution in period~$t$, but also~-- albeit implicitly~--
on the population distribution in period $t+1$, i.e., the left side. Indeed, before being solved for during stage (2)
of the generic backward step, the system \ref{def-equil}-(\ref{zze5a}-\ref{zze5xb}) is made to depend on the guessed
distributions $\csdp$, $ y\in\XXX$, in period $t+1$, which guess is being iterated until it becomes
consistent with the structure of \ref{main-q}-(\ref{zze666}).
To put it in another way, \ref{main-q}-(\ref{zze666}) describes the next period distribution as a function of
the present one, but that function is implicit, and, furthermore, the transport
equation \ref{main-q}-(\ref{zze666}) is meaningful only in conjunction with the appropriately attached budget,
kernel, and market clearing conditions~-- not as a stand-alone equation. 
Generally, such a program would be difficult to implement in concrete
models due to the lack of an adequate computing technology for representing
general (nonlinear) functions on the space of distributions.
This is  a common problem for all heterogeneous agent models, in which %, in one way or another,
the population distribution is a state variable.
Nevertheless, there are important special cases where the program outlined in \ref{main-proc} can still be
carried out~-- see Sec.~\ref{sec:IOU} and Sec.~\ref{sec:KS} below.
%%%%%%%%%%%%%%%%%%%%%%%%

%%%%%%%%%%%%%%%%%%%%%%

%%% chend556677


%%- -%%--{Sec: #3}--%%
%{Eq: 3.}   %%llabel
%{No: 3.}   %%nlabel


\section{The Savings Problem without Production and Shared Risk}
\label{sec:IOU}\setcounter{paragraph}{0}

\def\inc{{A}}

\noindent 
In this section we revisit the benchmark Huggett economy borrowed from \nocite{LjunSar00}{[RMT]} and
already discussed in Sec.~\ref{sec:Intro}.
The enormous simplification that comes
from the removal of the
production function and the common shocks is that time-invariant distribution of the population
and time-invariant value for the interest rate become available. In the search for these two
objects, we now
restate the system (\ref{zze5a}-\ref{zze5xb}-\ref{ze9}-\ref{zze666}) from \ref{main-q}
and \ref{cross-sys} 
with all wages $\ee_y(K_{t}(x,F))$ replaced by a
fixed value $\ee$, %%%%%%%%%%%%%%%%%%%%%%%%
%%%%%%%%%%%%%%%%%%%%%%%%%%%%%%%%%%
%\footnote{This choice can be seen as a simple linear production function that takes only labor as
%input and yields output that is not subjected to common stochastic shocks.}
%%%%%%%%%%%%%%%%%%%%%%%%%%%%%%%%%%
with all quantities $\qq$ set to~$0$ (no capital investment takes place),
with the second equations in \eqref{zze5a} and \eqref{ze9} removed, with all instances of $x$ and $y$
as sub/super-scripts removed, and with all transition probabilities $Q(x,y)$ set to 1.
The transition probability matrix~$P$, which governs the idiosyncratic
transitions in every individual employment state, admits a unique list of steady state
probabilities $\gp=(\gp(u)>0,\,u\in\nobreak\EEE)$, which we treat as a vector-row with 
$\gp\,P=\gp$ and with $\sum_{u\ts\in\ts\EEE}\gp(u)=1$. Assuming that all independent private Markov
chains have reached steady-state, the
aggregate income in the economy is fixed at
{\abovedisplayskip=5pt plus 1.5pt minus 1.5pt\belowdisplayskip=7pt plus 1.5pt minus 1.5pt\belowdisplayshortskip=5pt plus 1.5pt minus 1.5pt
\[
\inc=\sum\nolimits_{u\ts\in\ts\EEE}\ee{u\over N}(\gp(u) N)=\ee\sum\nolimits_{u\ts\in\ts\EEE}{u} \, \gp(u)\,.%=0.218088\,,
\]}%
%The time invariant (i.e., limiting) version of the metaprogram from \ref{main-proc} is described
%next.
As we seek time-invariant equilibrium, we drop the subscript ``$t$'' throughout.
For technical reasons, instead of seeking the equilibrium interest $r$, we seek
the spot price, $B=\inc/(1+r)$, of a risk-free bond with face value $\inc$.
The relative risk aversion for all agents is the constant $R\ge1$.
%With \ref{def-equil} in mind, our main object of interest is the following.

%%- -%##%--{No: 3.1}--%%
\begin{nit}{Time-invariant equilibrium}\label{time-inv}
It consists of: {(1)}~a fixed scalar $B\in\R$;
{(2)}~a collection of continuous and non-decreasing functions
$\q_u\colon\Rpp\to\R$, $u\in\EEE$;
{(3)}~a collection of continuous and non-decreasing functions
$\pfc^v(u,\cdot)\colon\Rpp\to\Rpp$, $u,v\in\EEE$ with inverses $\hat\pfc^v(u,\cdot)$;
{(4)}~a collection of cumulative distribution functions $\csd^u\in\bbF$, $u\in\EEE$~---
all chosen so that the following four conditions (kernel, balanced budget, market clearing, and transport)
are satisfied:
{\abovedisplayskip=5pt plus 1.5pt minus 1.5pt\belowdisplayskip=5pt plus 1.5pt minus 1.5pt\belowdisplayshortskip=5pt plus 1.5pt minus 1.5pt
\begin{gather}%\label{focfoc} 
B=\b\,\sum\nolimits_{\,v\ts\in\ts\EEE} {\inc}\, \Bigl({c\over \pfc^v(u,c)}\Bigr)^{R} {P}(u,v)
\quad\text{for all}\  c\in\Rpp\ \text{and all}\ u\in\EEE\,;
\tag{$\text{n}$}\label{focfoc-a}\\ 
{\q_u(c)}{\inc}+{\ee\ts v}
={\pfc^v(u,c)}+{\q_v(\pfc^v(u,c))}B\quad\text{for all}\   c\in\Rpp\ \text{and all}\ u,v\in\EEE\,;
\tag{$\text{e}$}\label{focfoc-b}\\
\sum\nolimits_{\,u\ts\in\ts\EEE}\gp(u)\int_0^\infty{\q_u(c)}\d \csd^u (c) = 0\,;
\tag{$\text{m}$}\label{focfoc-c}\\
\hfil \csd^v(c) = \sum\nolimits_{u\ts\in\ts\EEE} 
{\gp(u){P}(u,v)\over \gp(v)}\, 
\csd^u\bigl(\hat \pfc^v(u,c)\bigr)\quad \text{for all}\  c\in\Rpp\ \text{and all}\ v\in\EEE\,.\qed
\tag{$\text{d}$}\label{focfoc-d}%\label{4iter-F}
\end{gather}}%
\end{nit}
\nitskip

As the balanced budget constraints in \ref{time-inv}-(\ref{focfoc-b}) obtain as
the limit of \ref{cross-sys}-(\ref{zze5xb}), those constraints give rise to the following 
iteration program (in what follows the token $\pst$ marks new values and the token $\pdg$ marks previously
computed values)

%%
%%- -%@@%--{Eq: 3.1}--%%
{\abovedisplayskip=5pt plus 1.5pt minus 1.5pt\belowdisplayskip=5pt plus 1.5pt minus 1.5pt\belowdisplayshortskip=5pt plus 1.5pt minus 1.5pt
\begin{equation}\label{4iter-q}
\pst{\q_u(c)}\,{\inc}+{\ee\,v} 
={\pst\pfc^v(u,c)} + \pdg\q_v(\pst\pfc^v(u,c))B\,,\quad c\in\Rpp\,,\ u,v\in\EEE\,.
\end{equation}}%
Since the functions
{\abovedisplayskip=5pt plus 1.5pt minus 1.5pt\belowdisplayskip=5pt plus 1.5pt minus 1.5pt\belowdisplayshortskip=5pt plus 1.5pt minus 1.5pt
\[
\Rpp\ni \tau \leadsto  \pdg{{H}}_v(\tau)\df \tau +\pdg\q_v(\tau)\,B\,,\quad v\in\EEE\,,
\]}%
are strictly increasing and continuous, they can be inverted in the usual way, and
letting $\pdg{\hat{H}}_v\phd$ denote the inverse of $\pdg{{H}}_v\phd$,
the functions $\pst\q_u\phd$, $u\in\EEE$, obtain implicitly from the relations
%%
%%- -%@@%--{Eq: 3.2}--%%
{\abovedisplayskip=5pt plus 1.5pt minus 1.5pt\belowdisplayskip=5pt plus 1.5pt minus 1.5pt\belowdisplayshortskip=5pt plus 1.5pt minus 1.5pt
\begin{equation}\label{reduced-k}
B=\beta\,\inc\,\sum\nolimits_{v\ts\in\ts\EEE}\Bigl({c\over 
 \pdg{\hat{H}}_v\bigl(\pst\q_u(c)\,W+{\ee\ts v}\bigr)}\Bigr)^{R} {P}(u,v)\,,
\quad c\in\Rpp\,,\ u\in\EEE\,,
\end{equation}}%
or, which amounts to the same but is easier, $c$ can be
written as an explicit function of $\pst\q_u(c)$.
Furthermore, \ref{time-inv}-(\ref{focfoc-d}) gives rise to the iteration program
%%
%%- -%@@%--{Eq: 3.3}--%%
{\abovedisplayskip=7pt plus 1.5pt minus 1.5pt\belowdisplayskip=7pt plus 1.5pt minus 1.5pt\belowdisplayshortskip=5pt plus 1.5pt minus 1.5pt
\begin{equation}\label{4iter-F}
\pst\csd^v(\cdot) \df \sum\nolimits_{u\ts\in\ts\EEE} 
{\gp(u){P}(u,v)\over \gp(v)}\, 
\pdg\csd^u\bigl(\pst\hat \pfc^v(u,\cdot)\bigr)\,,\quad v\in\EEE\,,
\end{equation}}%
In the present context, the metaprogram
in \ref{main-proc} amounts to the following.
\nitskip

%%- -%##%--{No: 3.2}--%%
\begin{nit}{Metaprogram for time-invariant equilibrium}\label{IOU-proc}
{\it Step 0:} Make an ansatz choice for the collection of portfolio mappings $\pdg\q_v(\cdot)$
and compute the inverses $\pdg{\hat{H}}_v(\cdot)$, $v\in\EEE$.
Then make an ansatz choice for the spot price $B$ (these two choices are independent).  Go to Step~1. 
 

%\noindent
{\it Step 1:} For every choice of $u\in\EEE$, and certain choices of $c\in\Rpp$, 
solve \eqref{reduced-k} (there is one equation for every $u\in\EEE$ and every $c\in\Rpp$) 
for the unknowns $\pst\q_u(c)$ and set
$$\pst\pfc^v(u,c)
=\pdg{\hat{H}}_v\bigl(\pst\q_u(c)\,\inc+{\ee\,v}\bigr)\quad\text{ for every $u,v\in\EEE$}\,.
$$
Find the smallest $c\in\Rpp$, denoted $\bar
c$, with the  property $c\ge \pst\pfc^v(u,c)$ for all
$u,v\in\EEE$.%%%%%%%%%%%%%%%%
%%%%%%%%%%%%%%%%%%%%%%%%%%%%%%
\footnote{This step is meant to endogenize the upper bound on consumption.}
%%%%%%%%%%%%%%%%%%%%%%%%%%%%%%%%%
Construct a uniform (equidistant) finite grid, denoted $\bbG_\cint$, on the
interval $\cint$. Go to the next step.

  
%\noindent 
{\it Step 2:}  For every $u\in\EEE$ and every grid-point
$c\in\bbG_\cint$,
solve for $\pst\q_u(c)$ from \eqref{reduced-k} and set
$\pst\pfc^v(u,c)=\pdg{\hat{H}}_v\bigl(\pst\q_u(c)\,\inc + {\ee\,v}\bigr)$  all $v\in\EEE$.  
By interpolating the respective values define the functions $\pst\q_u(\cdot)$ and
$\pst\pfc^v(u,\cdot)$ as cubic splines over the grid $\bbG_\cint$ in the obvious way.
Define
uniform interpolation grids over the ranges of the functions  $\pst\pfc^v(u,\cdot)$, compute 
the inverse values at those grid-points and, finally, 
define the inverse functions $\pst\hat \pfc^v(u,\cdot)$ as the cubic
splines obtained by interpolating the inverse values over the respective grids. Go to the next
step. 

%\noindent
{\it Step 3:} 
If the family of distribution functions $\pdg\csd^u$, $u\in\EEE$, 
has not been updated before (this is the first visit to step~3), 
define $\pdg\csd^u$ to be the distribution function associated with the uniform probability
measure  on $\cint$ for every $u\in\EEE$. Otherwise, do nothing and go to the next step.

%\noindent
{\it Step 4:}
Calculate
{\abovedisplayskip=2pt plus 1.0pt minus 1.5pt\belowdisplayskip=5pt plus 1.5pt minus 1.5pt\belowdisplayshortskip=5pt plus 1.5pt minus 1.5pt
\begin{equation}\tag{a}\label{step-4-dist}
\pst\csd^v(c) \df \sum\nolimits_{u\ts\in\ts\EEE} 
{\gp(u){P}(u,v)\over \gp(v)}\, 
\pdg\csd^u\bigl(\pst\hat\pfc^v(u,c)\bigr)
\end{equation}}%
for every  $c\in\bbG_\cint$ and every $v\in\EEE$ and construct the
distribution functions $\pst\csd^v(\cdot)$, $v\in\EEE$, as cubic splines over 
the grid $\bbG_\cint$ in the obvious way. Compute the error term
%%
{\abovedisplayskip=5pt plus 1.5pt minus 1.5pt\belowdisplayskip=5pt plus 1.5pt minus 1.5pt\belowdisplayshortskip=5pt plus 1.5pt minus 1.5pt
\[
\max\nolimits_{v\ts\in\ts\EEE,\,c\ts\in\ts \bbG_\cint}
|\pst\csd^{v} (c)- \pdg\csd^{v}(c)|\,.
\]}%
If this error term exceeds some  prescribed
threshold, set $\pdg\csd^v(\cdot)=\pst\csd^v(\cdot)$, 
$v\in\EEE$, go back to the beginning of this step and
repeat. Otherwise, set $\pdg\csd^v(\cdot)=\pst\csd^v(\cdot)$, $v\in\EEE$, and
go to the next step. 

%\noindent
{\it Step 5:} Test the market clearing
{\abovedisplayskip=5pt plus 1.5pt minus 1.5pt\belowdisplayskip=5pt plus 1.5pt minus 1.5pt\belowdisplayshortskip=5pt plus 1.5pt minus 1.5pt
\begin{equation}\tag{b}\label{step-5-mktc}
\sum\nolimits_{u\ts\in\ts\EEE}\gp(u)\int_0^{\bar c}{\pst\q_u(c)}\d\,\pst\csd^u(c) = 0\,.
\end{equation}}%
If this identity fails by more
than some  prescribed threshold, discard the splines
$\pst\q_u(\cdot)$, $u\ts\in\ts\EEE$, while still
keeping $\pdg\q_u(\cdot)$ and the associated $\pdg {\hat{H}}_u(\cdot)$, 
$u\ts\in\ts\EEE$, on
record,  modify the most recent choice for the spot price $B$
accordingly, and go back to step~1. Otherwise go to the next step.

%\noindent
{\it Step 6:} If this is the first visit to step 6, set $\pdg\q_u(\cdot)=\pst\q_u(\cdot)$,
$\pdg\pfc^v(u,\cdot)=\pst\pfc^v(u,\cdot)$ and go back to Step~1 with the most recently updated $B$. 
Otherwise, compute the error terms
{\abovedisplayskip=5pt plus 1.5pt minus 1.5pt\belowdisplayskip=5pt plus 1.5pt minus 1.5pt\belowdisplayshortskip=5pt plus 1.5pt minus 1.5pt
\begin{equation}\tag{c}\label{conv-test}
\max\nolimits_{u\ts\in\ts\EEE,\,c\in \bbG_\cint}
|\pst\q_u(c)-\pdg\q_u(c)|
\quad 
\text{and}\quad
\max\nolimits_{u,v\in\EEE,\,c\in \bbG_\cint}
|\pst\pfc^v(u,c)-\pdg \pfc^v(u,c)|\,.
\end{equation}}%
If at least one of these terms exceeds some prescribed threshold, set 
$\pdg\q_u(\cdot)=\pst\q_u(\cdot)$ and $\pdg \pfc^v(u,\cdot)=\pst\pfc^v(u,\cdot)$,
for $u,v\in\EEE$,  
and go back to step 1. Otherwise stop. Declare that the
equilibrium is given by the most recently updated spot price $B$,
portfolio mappings $\pst\q_u(\cdot)$, $u\in\EEE$, 
next-period-consumption mappings
$\pst\pfc^v(u,\cdot)$, $u,v\in\EEE$, and family of distribution
functions $\pst\csd^u(\cdot)$, $u\in\EEE$.\qed
\end{nit}
%\nitskip


%%- -%##%--{No: 3.3}--%%
\begin{nit}{Abridged version of \ref{IOU-proc}}\label{abridged}
Make an ansatz choice for $(B,\pdg\q)$ and record this choice.
Gi\-ven $B$ and $\pdg\q$, produce $\pst\q$. 
Then produce $\pst{\csd}$ as a fixed point of the transport determined by $\pdg\q$ and
$\pst\q$. If market clearing with $\pst\q$ and $\pst{\csd}$ fails, forget
$\pst\q$ and $\pst{\csd}$, change the value of~$B$, and repeat with the modified $B$ and
with $\pdg\q$. If the market clears, record $\pst\q$,  the latest $B$, and the latest
consumption transfer mappings.
If~this is the first incidence of market clearing, set $\pdg\q=\pst\q$ and repeat from
the beginning with the latest $B$ and the new $\pdg\q$. If~not, test the uniform distance
between $\pst\q$ and $\pdg\q$ and between the two most recent collections of consumption transfer
mappings. If~this distance is not acceptable, set $\pdg\q=\pst\q$ and repeat from
the beginning with the latest $B$ and the new $\pdg\q$. Otherwise, stop.\qed
\end{nit}


%%- -%##%--{No: 3.4}--%% 
\begin{nit}{Remark}
It is instructive to note the key
differences between the program in \ref{IOU-proc} and the classical strategy outlined
in Sec.~\ref{sec:Intro}: (a)~The portfolio mappings $\pdg\q_u(\cdot)$, $u\in\EEE$, capture
the investment decisions in the cross-section of all households that share the same state of
employment~-- not the investment decision of one representative household.
(b)~The law of motion in the space of distributions (of unit mass) encrypted in
\ref{IOU-proc}-(\ref{step-4-dist}) is not sought as the law of motion of the probability
distribution of any particular Markovian state. 
(c)~It is the price that gets adjusted to the portfolio mappings, not the other way around.
(d)~The distribution with which market clearing is tested obtains as a
fixed point in the most recent consumption transfer mappings and does not have the
meaning of ``limiting distribution.''\qed  %until the very end.\qed
%%%%%%%%%%%%%%
%
%
%
\iffalse
the improvements in the price, the portfolio mappings, and the
population distribution are all done in parallel: the iteration program
\eqref{4iter-q}, i.e., the improvement in the portfolio mappings, is not allowed to proceed unless
the market can be cleared with the outcome from \eqref{4iter-q} and the latest price,
the adjustments in which affect the outcome from \eqref{4iter-q}~-- all the while market clearing is
tested only with distributions found to be fixed points for the transport determined by the outcome
from \eqref{4iter-q}. 
One must also recognize that these improvements are ``local in time,'' in the sense that
achieving  market clearing does not end the induction, as it merely allows the iteration program
in \eqref{4iter-q} to proceed~--
until the accepted portfolio functions and population distribution start repeating within a
prescribed threshold (in some sense, reaching to the infinite horizon).\qed
%\protect\vskip-3.5cm
\fi
%
%
%
%%%%%%%%%%%%%%%%%%%%%
\end{nit}

%%- -%##%--{No: 3.5}--%%
\begin{nit}{Remark}
%The heuristic argument %~-- confirmed through the numerical implementation of the program~--
%behind the calculation of
%the endogenous upper bound on future consumption is this: as the aggregate supply of the
%numéraire good is finite, there must be a critical threshold beyond which
%excessively large consumption during the current period must
%entail less consumption in the following period. 
Step~1 above is nothing but the search for the
endogenous upper bound on consumption, which then translates into an upper bound on invetsment,
as the functions $c \leadsto \pst\q_u(c)$,
$u\in\EEE$, are increasing.
The lower bound on the investment (i.e., the borrowing limit)
is then $\min_u\lim_{c\to0}\q_u(c)B$.
We stress that both bounds are determined
endogenously throughout the iterations.
In the benchmark economy discussed here these bounds
are never reached and the cross-sectional distribution of the population has no mass at
them.\qed
%\protect\vskip-5cm
\end{nit}
\nitskip

What follows next is a brief summary of the concrete results from implementing the
metaprogram \ref{IOU-proc} in the context of the benchmark Huggett economy
borrowed from \nocite{LjunSar00}{[RMT]} and
already discussed in Sec.~\ref{sec:Intro}. %%%%%%%%%%%%%%%%%%%%%%%%%%
%%%%%%%%%%%%%%%%%%%%%%%%%%%%%%%%%%%%%%%%%%%%%%%%%%%%%%%%
%\footnote{The program in \eqref{IOU-proc} was translated into computer code in the Julia
%programming language. The code was executed with the generic Linux binaries
%from \href{https://julialang.org/}{https://julialang.org/} on a single CPU (processor: i7-10700x16,
%memory: $48.0\,$GiB, total computing time $128$ minutes). For brevity, all numerical values
%reported here are
%rounded to the fifth decimal place.\label{Linux} }
%
%\footnote{For brevity, all numerical values reported here are
%rounded to the fifth decimal place. The full output and other details are included with the
%Julia code.\label{Linux}}
%
%%%%%%%%%%%%%%%%%%%%%%%%%%%%%%%%%%%%%%%%%%%%%%%%%%%%%%%%     
All model parameters are taken from the first specification in Sec.~18.7
in \nocite{LjunSar00}{[RMT]}. 
The initial ansatz choice for the portfolio functions is $\pdg\q_v(c)=40c-8$ for all $v\in\EEE$ and
for the bond price the initial choice is $B=\inc$, corresponding to zero interest as an initial guess. 
The convergence (the largest amount
in \hbox{\ref{IOU-proc}-(\ref{conv-test})}) is $9.08447\times 10^{-5}$ after 235 iterations. 
The program returns equilibrium interest rate of $0.03702$,
and market clearing (the left side of~\ref{IOU-proc}-(\ref{step-5-mktc})) of
$-1.73878\times10^{-6}$.
The distribution of households in every one of the $7$ employment categories over the consumption space
is shown in Figure~\ref{fg4X} 
%%%%%%%%%%%%%%%%%%%%%%%%%%%%%%%%%%%%%%%%%  
%%    Fg: 5
%%%%%%%%%%%%%%%%%%%%%%%%%%%%%%%%%%%%%%%%%%
{%\captionsetup{belowskip=-10pt}
%\captionsetup{aboveskip=10pt}
\begin{figure}[!htbp]  
\centering
\begin{subfigure}{.5\textwidth}
  \centering
\leavevmode\raise0.9cm\hbox{\rotatebox{90}{\tiny population distribution in states $u\in\EEE$}}%
\ %   
\toshow{\includegraphics[width=7.0cm]{fg5L}}

\leavevmode\smash{\raise6pt\hbox{\tiny consumption}}
  %\caption{A subfigure}
%  \label{fig:sub1} 
\end{subfigure}%
\begin{subfigure}{.5\textwidth}
  \centering
\leavevmode\raise0.9cm\hbox{\rotatebox{90}{\tiny population density in states $u\in\EEE$}}%
\ %
\toshow{\includegraphics[width=7.0cm]{fg5R}}  

\leavevmode\smash{\raise6pt\hbox{\tiny consumption}}
  %\caption{A subfigure} 
%  \label{fig:sub2}
\end{subfigure}
\caption{The distribution (cumulative left, density right) of households over the range of consumption.}
\label{fg4X} 
\end{figure}}%
%
%
%
and the plots in Figure~\ref{fg8X} show the investment level in the cross-section of the population, i.e.,
the mappings $c \leadsto \q_u(c)\times B$, $u\in\nobreak\EEE\,$.
%
%
%%%%%%%%%%%%%%%%%%%%%%%%%%%%%%%%%%%%%%%%%%
%%     Fg: 6
%%%%%%%%%%%%%%%%%%%%%%%%%%%%%%%%%%%%%%%%%%
{%\captionsetup{belowskip=-10pt}
%\captionsetup{aboveskip=10pt}
\begin{figure}[!htbp]  
\centering
\begin{subfigure}{.5\textwidth}
  \centering
\leavevmode\raise1.2cm\hbox{\rotatebox{90}{\tiny investment in states $u\in\EEE$}}%
\ %  
\toshow{\includegraphics[width=7.0cm]{fg6L}}

\leavevmode\smash{\raise6pt\hbox{\tiny consumption}}
  %\caption{A subfigure}
%  \label{fig:sub1}  
\end{subfigure}%
\begin{subfigure}{.5\textwidth}
  \centering
\leavevmode\raise1.2cm\hbox{\rotatebox{90}{\tiny investment in states $u\in\EEE$}}%
\ %
\toshow{\includegraphics[width=7.0cm]{fg6R}} 

\leavevmode\smash{\raise6pt\hbox{\tiny consumption}} 
  %\caption{A subfigure} 
%  \label{fig:sub2}
\end{subfigure}
\caption{Investment in the bond as a function of consumption  shown on two different scales.}
\label{fg8X}
\end{figure}}%
%
%
%
The left limit in these graphs (the endogenous borrowing limit)
is around $-1.62826$ and is remarkably close to \nocite{Aiya94}{Aiyagari's}
natural borrowing limit (in this case, the same as the ``ad hoc'' limit~-- see \nocite{LjunSar00}{[RMT]}),
which is around $-1.62726$. The endogenous upper bound on investment~-- see Step~1
in \ref{IOU-proc}~-- is around $17.93751$, 
but we see from Figures~\ref{fg4} and~\ref{fg5X} that most of the population is amassed over a much narrower
range.
Since the mappings $c\leadsto \q_u(c)\ts B$ and $c\leadsto c+\q_u(c)\ts B-\ee\ts u$ are both
strictly increasing and continuous, the equilibrium distribution over consumption from Figure~\ref{fg4X} is easy to
transform into entering and exiting distribution of households over the asset space~--
this is how the left plot in Figure~\ref{fg4} was produced. The left plot in
Figure~\ref{fg5X} gives a detailed view of the left plot in Figure~\ref{fg4} near the borrowing
limit.
%
%%%%%%%%%%%%%%%%%%%%%%%%%%%%%%%%%%%%%%%%%%
%%    Fg: 7
%%%%%%%%%%%%%%%%%%%%%%%%%%%%%%%%%%%%%%%%%% 
{%\captionsetup{belowskip=10pt}
%\captionsetup{aboveskip=10pt}
\begin{figure}[!htbp]  
\centering
\begin{subfigure}{.5\textwidth}
  \centering
  \leavevmode\raise0.9cm\hbox{\rotatebox{90}{\tiny population distribution in states $u\in\EEE$}}%
\ %  
\toshow{\includegraphics[width=7.0cm]{fg7L}}

\leavevmode\smash{\raise6pt\hbox{\tiny asset holdings}}
  %\caption{A subfigure}
%  \label{fig:sub1} 
\end{subfigure}% 
\begin{subfigure}{.5\textwidth}
  \centering
  \leavevmode\raise0.9cm\hbox{\rotatebox{90}{\tiny population density in states $u\in\EEE$}}%
\ %
\toshow{\includegraphics[width=7.0cm]{fg7R}} 

\leavevmode\smash{\raise6pt\hbox{\tiny asset holdings}}
  %\caption{A subfigure} 
%  \label{fig:sub2}
\end{subfigure}
\caption{The entering (solid lines) and exiting (dotted lines) distribution of households over
asset holdings.}
\label{fg5X}
\end{figure}}%
%
%
%showing that in equilibrium the population distribution does not accumulate at that limit.
The right plot
in Figure~\ref{fg5X} is simply the density version of the left plot in Figure~\ref{fg4}. 
The graphs of the consumption transfer mappings $c \leadsto \pfc^v(u,c)$, $v\in\EEE$, for the lowest
and the highest employment category $u\in\EEE$ are shown on the left plot in Figure~\ref{fg8Y}. 
%
%
%%%%%%%%%%%%%%%%%%%%%%%%%%%%%%%%%%%%%%%%%% 
%%     Fg: 8 
%%%%%%%%%%%%%%%%%%%%%%%%%%%%%%%%%%%%%%%%%% 
{\captionsetup{belowskip=-10pt}
%\captionsetup{aboveskip=10pt}
\begin{figure}[!htbp]  
\centering
\begin{subfigure}{.5\textwidth}
  \centering
\leavevmode\raise0.75cm\hbox{\rotatebox{90}{\tiny future consumption in states $v\in\EEE$}}%
\ %  
\toshow{\includegraphics[width=7.0cm]{fg8L}}

\leavevmode\smash{\raise6pt\hbox{\tiny consumption in state $u=\EEE_1$ (solid) and in state $u=\EEE_7$ (dotted)}}
  %\caption{A subfigure}
%  \label{fig:sub1}  
\end{subfigure}%
\begin{subfigure}{.5\textwidth}
  \centering
\leavevmode\raise1.25cm\hbox{\rotatebox{90}{\tiny consumption in states $v\in\EEE$}}%
\ %
\toshow{\includegraphics[width=7.0cm]{fg8R}} 

\leavevmode\smash{\raise6pt\hbox{\tiny entering wealth from state $u=\EEE_1$ (solid) and from $u=\EEE_7$ (dotted)}} 
  %\caption{A subfigure} 
%  \label{fig:sub2}
\end{subfigure}
\caption{Future consumption as a function of present consumption (left) and entering future wealth (right).} 
\label{fg8Y}
\end{figure}}%
%
%
The right plot in Figure~\ref{fg8Y} provides an important verification of the program developed in
this section: on the one hand the consumption transfer mappings depend on both the exiting and the
entering employment states, while on the other hand future consumption must depend only on the future
employment state and the entering wealth in that state, irrespective of what employment state that
wealth is carried from. Since household $(u,c)\in\EEE\times\Rpp$ enters its future state with
assets $a=\a_u(c)\df \q_u(c)\times A$, letting $\hat\a_u\phd$ denote the inverse of the mappings
$c \leadsto \a_u(c)$, this means that $\TTT_u^v(\hat\a_u(a))$ must depend on~$v$ and~$a$ but not
on~$u$. Such a connection was never imposed in the system that produced the equilibrium, but the
right plot in Figure~\ref{fg8Y} shows that it nevertheless holds~-- as it should.
%The Julia code
%that supplements the paper provides more detailed numerical analysis and generates
%additional plots that are not included here for the sake of brevity.
Finally, the left plot in Figure~\ref{fg9Z} shows consumption as a function of total wealth
(i.e., consumption plus investment)
and the right plot shows the marginal propensity to consume (simply
the gradient of the splines generating the left plot). 
%
%
%%%%%%%%%%%%%%%%%%%%%%%%%%%%%%%%%%%%%%%%%% 
%%     Fg: 9 
%%%%%%%%%%%%%%%%%%%%%%%%%%%%%%%%%%%%%%%%%% 
{\captionsetup{belowskip=-10pt}
\captionsetup{aboveskip=10pt}
\begin{figure}[!htbp]  
\centering
\begin{subfigure}{.5\textwidth}
  \centering
\leavevmode\raise0.75cm\hbox{\rotatebox{90}{\tiny consumption in states $u\in\EEE$}}%
\ %  
\toshow{\includegraphics[width=7.0cm]{fg9L}}

\leavevmode\smash{\raise6pt\hbox{\tiny \qquad total wealth (entering plus income)}}
  %\caption{A subfigure}
%  \label{fig:sub1}  
\end{subfigure}%
\begin{subfigure}{.5\textwidth}
  \centering
\leavevmode\raise0.35cm\hbox{\rotatebox{90}{\tiny marginal propensity to consume in states $u\in\EEE$}}%
\ %
\toshow{\includegraphics[width=7.0cm]{fg9R}} 

\leavevmode\smash{\raise6pt\hbox{\tiny \qquad total wealth (entering plus income)}} 
  %\caption{A subfigure} 
%  \label{fig:sub2}
\end{subfigure}
\caption{Consumption as a function of total wealth (entering plus income) and its gradient.} 
\label{fg9Z}
\end{figure}}%
%
%%- -%##%--{No: 3.6}--%% 
\begin{nit}{Implications}
Figure~\ref{fg8X} shows that the employment state has only a marginal effect on the dependence
between investment and consumption (all households face the same stream of employment shocks in the
long run and the effect of temporary differences in employment is too
small relative to the entire stream of anticipated future shocks).
In addition, the right plot shows that the graphs on the left are not flat near~$0$, with slopes
ranging between  $0.12502$ and $0.40615$.
Figures~\ref{fg4X} and~\ref{fg5X} show that the distribution of households (both entering and
exiting) over asset holdings is
substantially more dispersed and skewed toward the wealthy than it is over consumption.
The standard deviation over consumption ranges between  
$0.0458$ and $0.03827$, over entering wealth it ranges between
$0.97125$ and $0.97857$, and ranges between  $0.92818$ and $0.93861$ over exiting wealth. For the
skewness these numbers are  $0.11558$ $\sim$ $0.84976$ for consumption,  $0.96578$ $\sim$ $0.93788$
for entering wealth, and  $1.00653$ $\sim$ $0.9316$ for exiting wealth.
In contrast to the conclusions from \citel{AHLLM17}, which are based on exogenously imposed
borrowing limits and boundary conditions,
it is clear from Figures~\ref{fg5X} and~\ref{fg9Z} that the cross-sectional distribution of the
population does not accumulate at the borrowing limit and the marginal propensity to consume does
not explode at that limit.
%
More details and illustrations are included with the Julia code that accompanies the paper.\qed
%
%This feature is consistent with Proposition~1 in \citel{AHLLM17},
%but we stress that the strategy used here does not require imposing it as a boundary
%condition.\qed
%Thus, households who happen to enjoy the same level of 
%wealth after collecting their (possibly different) paychecks would make almost identical
%investment-consumption decisions irrespective of their employment status.\qed
%Furthermore, different households (possibly in different
%states of employment) would consume identical amounts only if they are equally wealthy after
%collecting their (possibly different) paychecks, in which case they also make identical
%investment decisions, facing identical long term future.
\end{nit} 

\iffalse
%%- -%##%--{No: 3.7}--%%
\begin{nit}{Remark}
It was noted in Sec.~\ref{sec:Intro}~-- see Figure~\ref{fg4}~-- that the equilibrium rate
and population distribution produced here %by the program in \eqref{IOU-proc}
can be produced also with
the classical approach, provided that a ``correct guess'' is made  when the iterations
are initialized. %Without a better understanding of ``correct guess''
While this observation may not be very useful from a computational point of view,
the phenomenon illustrated on Figure~\ref{fg4} is still rather insightful:
it is possible in this particular model to find an equilibrium arrangement in which the optimal state of
every household has a long-run probability distribution that is the same as the long-run population
distribution. Since we were able to identify the latter without
accounting for the probability distribution of a generic private state,
the question arises as to whether equilibrium (with no aggregate shocks) is only possible if the
optimal state of every household is ``exactly distributed according to the state of the
population?''%%%%%%%%%%%
%%%%%%%%%%%%%%%%%%%%%%%%%%%%%%%%%%%%%%%%%%%%
\footnote{This phrase is taken from \citel{CDLL19}, whence the quotation marks.}
%%%%%%%%%%%%%%%%%%%%%%%%%%%%%%%%%%%%%%%%%%%%%
This question cannot be answered~-- in fact, it cannot even be asked~-- within the
framework developed here, since the probability distributions of the private states are completely
absent from~it. The realization that the  collective behavior of a large population of agents
can be derived without any reference to the long-run
statistical behavior of a generic private state takes the heterogeneous agent models even
further away from the representative agent point of~view.\qed
\end{nit}
\nitskip
\fi

%%% chend556677

%%- -%%--{Sec: #4}--%%
%{Eq: 4.}   %%llabel
%{No: 4.}   %%nlabel


\section{The Savings Problem with Production %technology\\
and Shared Risk}\setcounter{paragraph}{0}
\label{sec:KS}

\noindent
In this section we specialize the general model developed in Sec.~\ref{sec:gen-model} to the
benchmark case studied in \nocite{KS98}{[KS]}. The economy is endowed with production technology and
%is arranged so that
the households invest only in productive capital (no private lending).
The main result in landmark  work \nocite{KS98}{[KS]} is what the authors call ``collapse of the state space,'' or
``approximate aggregation,'' i.e., the finding that ``in~equilibrium, all aggregate variables~--
consumption, the capital stock, and relative prices~-- can be almost perfectly described
as a function of two simple statistics: the mean of the wealth distribution and the aggregate
productivity shock.'' Simply put: ``only the mean matters.''
The key insight offered in \nocite{KS98}{[KS]} is that ``utility costs from
fluctuations in consumption are quite small and that self-insurance with only one asset is quite
effective''~-- a phenomenon that  the authors link to the permanent income hypothesis
of \citel{Bew77}. Since the population distribution enters the model only through the market
clearing, %in the context of the model developed in Sec.~\ref{sec:gen-model}
the claim ``only the mean matters'' boils down to the claim that the cross-sectional portfolio
mappings are approximately affine.%%%%%%%%%%%%%%%%%%%%%%%%%%
%%%%%%%%%%%%%%%%%%%%%%%%%%%%%%%%%%%%%%%%%%%%%%%%%%
\footnote{\citel{Str23} provides a particularly elegant formal justification.}
%%%%%%%%%%%%%%%%%%%%%%%%%%%%%%%%%%%%%%%%%%%%%%%%% 
This is the 
tack that this section takes. Specifically, the affine structure of the
portfolio and consumption transfer mappings is shown to be an analytical property of the model,
exact formulas for the slopes  are obtained, and
the strategy from \ref{main-proc} is adapted to the case of approximate aggregation.
In addition, approximate aggregation is specific to every employment
category, i.e., instead of a single average across the entire population we work with a
vector of averages. The time evolution of the state of
the population is then a time series in $\R^{\abs{\EEE}}$, not in~$\R$, and we are going to see
that this more complex structure exhibits heterogeneity that cannot be captured by a single
average. Another notable departures from the framework adopted in \nocite{KS98}{[KS]} is the
averaging over consumption instead of asset holdings.


Consistent with the general model introduced in Sec.~\ref{sec:gen-model}, we set
$\q_{t,x,\csd,u}(c)\equiv\nobreak 0$ (the agents do not invest in a risk-free asset) and, accordingly, ignore the
first kernel condition in \ref{cross-sys}-(\ref{zze5a}) and the first market clearing condition in
\ref{cross-sys}-(\ref{ze9}).
The left side of the second market clearing condition in \ref{cross-sys}-(\ref{ze9})
would depend solely on the averages%%%%%%%%%%%%%%%%%%%%
%%%%%%%%%%%%%%%%%%%%%%%%%%%%%%%%%%%%%%%%
\footnote{The term ``average'' is used here as an alias for ``expected value,'' which is what the
integral represents. However, in the present context this integral represents also the aggregate
consumption of all households in employment state $u\in\EEE$. In~what follows we reserve the term
``aggregate'' to mean aggregation across the entire population~-- not just the population in a
single employment category.}
%%%%%%%%%%%%%%%%%%%%%%%%%%%%%%%%%%%%%%% 
{\abovedisplayskip=4pt plus 1.0pt minus 1.5pt\belowdisplayskip=4pt plus 1.0pt minus 1.5pt\belowdisplayshortskip=4pt plus 1.0pt minus 1.5pt
\[
\bar\csd^{u}\df \int_0^\infty c \d\csd^{u}(c)\,,\quad u\in\EEE\,, 
\]}%
only if the portfolio mappings $c \leadsto {\qq_{t,x,\csd,u}}(c)$ happen to be affine functions~--
see \citel{Str23}~-- in which case
the state variable $\csd\in\DB^\EEE$ reduces to the finite list
{\abovedisplayskip=5pt plus 1.5pt minus 1.5pt\belowdisplayskip=5pt plus 1.5pt minus 1.5pt\belowdisplayshortskip=5pt plus 1.5pt minus 1.5pt
\[
\bar\csd\df (\bar\csd^{u},u\in\EEE)\in\R^{\abs{\EEE}}\,.
\]}%
In general, the demand for capital cannot be an affine function of  the  consumption
level $c$, since the latter affects the kernel condition in \ref{cross-sys}-(\ref{zze5a})
in nonlinear fashion.
In~order to uncover the way in which this kernel condition affects the structure of the portfolio
mappings, and identify conditions under which all equations that define the equilibrium can be
satisfied~-- at least approximately~--  with affine  portfolio mappings, let us suppose, contrary
to fact, that all portfolio and
the future consumption mappings have the affine structure %%%%%%%%%%%%%
%%%%%%%%%%%%%%%%%%%%%%%%%
% \footnote{This structure is essentially tantamount to the assumption~-- adopted
% in \nocite{KS98}{[KS]}~-- that the cross-sectional distribution of the population affects the
% equilibrium only through its average.}
%%%%%%%%%%%%%%%%%%%%%%%%%
%%
%%- -%@@%--{Eq: 4.1}--%%
{\abovedisplayskip=5pt plus 1.5pt minus 1.5pt\belowdisplayskip=5pt plus 1.5pt minus 1.5pt\belowdisplayshortskip=5pt plus 1.0pt minus 1.5pt
\begin{equation}\label{lin-subst}
\qq_{t,x,\bar\csd,u}(c) = a_{t,x,\bar\csd,u}+b_{t,x,\bar\csd,u}\times c\quad\text{and}\quad
\pfc_{t,x,\bar\csd}^{ y , v }(u,c) = g_{t,x,\bar\csd,u}^{ y , v } + h_{t,x,\bar\csd,u}^{ y , v }\times c\,,\quad c\in\Rpp\,,
\end{equation}}%
for some yet to be determined scalars $a_{t,x,\bar\csd,u}$, $b_{t,x,\bar\csd,u}$, $g_{t,x,\bar\csd,u}^{ y , v }$, and
$h_{t,x,\bar\csd,u}^{ y , v }$.
With the choice just made and
with risk aversion parameter of $R=1$
(same as in the benchmark case study in \nocite{KS98}{[KS]}), the second kernel
condition in \ref{cross-sys}-(\ref{zze5a}) can be cast as
%%
%%- -%@@%--{Eq: 4.2}--%% 
{\abovedisplayskip=5pt plus 1.5pt minus 1.5pt\belowdisplayskip=5pt plus 1.5pt minus 1.5pt\belowdisplayshortskip=5pt plus 1.5pt minus 1.5pt
\begin{equation}\label{only-q-bn2-2}
1 = \b\,\sum_{ y \in\XXX, v\in\EEE}
{ 1 \over g_{t,x,\bar\csd,u}^{ y ,v}/c  + h_{t,x,\bar\csd,u}^{ y ,v}}\times
\Bigl(\rho_ y \bigl(K_{t}(x,\bar\csd)\bigr) + 1-\dd \Bigr)
\times  Q(x, y )\,  P_{x, y }( u ,v)\,.
\end{equation}}%
Clearly, the last relation will be almost unaffected by the choice of the consumption level~$c$ if this
choice is restricted to some interval $\llbkt c^*,\infty\rlbkt$, for some sufficiently large constant
$c^*>0$, i.e., \eqref{only-q-bn2-2} can be enforced with an arbitrarily small error, and uniformly
for all choices of $c\in \llbkt c^*,\infty\rlbkt\,$, provided that $c^*$ is chosen to be sufficiently
large. 
This reduction is powerful, because it eliminates the one and only source of
nonlinearity, in which case the structure imposed in \eqref{lin-subst} would indeed hold. 
With the substitution \eqref{lin-subst} in mind, the balance equation 
in \ref{cross-sys}-(\ref{zze5xb}) becomes
%%
%%- -%@@%--{Eq: 4.3}--%%
{\abovedisplayskip=5pt plus 1.5pt minus 1.5pt\belowdisplayskip=5pt plus 1.5pt minus 1.5pt\belowdisplayshortskip=5pt plus 1.5pt minus 1.5pt
\begin{equation}\label{e10-bn3}
\begin{gathered}
\begin{multlined}
({ {a_{t,x,\bar\csd,u}}}+{ {b_{t,x,\bar\csd,u}}}\times c)\times
\Bigl(\rho_ y \bigl({ {K_{t}(x,\bar\csd)}}\bigr) +1-\dd\Bigr) 
+ v\times \ee_ y \Bigl({ {K_{t}(x,\bar\csd)}}\bigr)\hbox to1.5cm{\hfil}\\
\hbox to1.0cm{\hfil} = \bigl({ {g_{t,x,\bar\csd,u}^{ y ,v}}} + { {h_{t,x,\bar\csd,u}^{ y ,v}}}\times c\bigr)
+ \Bigl({{a_{t+1,\tts y,\tts\T_{t,x}^y(\bar\csd),\tts v}}}
+{{b_{t+1,\tts y,\tts \T_{t,x}^y(\bar\csd),\tts v}}}\times\bigr({ {g_{t,x,\bar\csd,u}^{ y ,v}}}
+ { {h_{t,x,\bar\csd,u}^{ y ,v}}}\times c\bigr)\Bigr)\\
\quad\text{for all }\ \ 
 y \in\XXX\,,\  v\in\EEE\,.
\end{multlined}
\end{gathered}
\end{equation}}%
Just as before, the strategy is to take the future portfolio mappings encrypted in the pairs
%%
{\abovedisplayskip=5pt plus 1.5pt minus 1.5pt\belowdisplayskip=5pt plus 1.5pt minus 1.5pt\belowdisplayshortskip=5pt plus 1.5pt minus 1.5pt
$$
\bigl(a_{t+1,\tts y ,\tts \T_{t,x}^y(\bar\csd),\tts v},b_{t+1,\tts y ,\tts  \T_{t,x}^y(\bar\csd),\tts v}\bigr)\,,\quad
y\in\XXX\,,\ \ v\in\EEE\,,
$$
}%
as
given and treat \eqref{e10-bn3} as a system for the unknowns (present portfolios and future
consumption mappings) $a_{t,x,\bar\csd,u}$, $b_{t,x,\bar\csd,u}$,
$g_{t,x,\bar\csd,u}^{ y ,v}$, and $h_{t,x,\bar\csd,u}^{ y ,v}$, $x, y \in\XXX$, $ u, v\in\EEE$.
Treated as an identity between two polynomials of
degree~$1$ over the variable $c\in\Rpp$, \eqref{e10-bn3} can be split into two systems:
\begin{subequations}\label{lin-split}
{\abovedisplayskip=5pt plus 1.5pt minus 1.5pt\belowdisplayskip=5pt plus 1.5pt minus 1.5pt\belowdisplayshortskip=5pt plus 1.5pt minus 1.5pt
%%- -%@@%--{Eq: 4.4}--%%
\begin{gather}
\begin{aligned}\label{lin-split-a}
&{ {a_{t,x,\bar\csd,u}}}\times  
\Bigl(\rho_ y \bigl({ {K_{t}(x,\bar\csd)}}\bigr) +1-\dd\Bigr) 
+v \times \ee_ y \bigl({ {K_{t}(x,\bar\csd)}}\bigr)\\
&\hbox to4.5cm{\hfil}= {{a_{t+1,\tts y ,\tts \T_{t,x}^y(\bar\csd),\tts v}}}+\bigl(1+{{b_{t+1,\tts y
,\tts \T_{t,x}^y(\bar\csd),\tts v}}}\bigr)\times { {g_{t,x,\bar\csd,u}^{ y ,v}}}\\
&\hbox to8.5cm{\hfil}\quad\text{for all }\ \  y \in\XXX\,,\  u, v\in\EEE\,,
\end{aligned}\nobreak\\
\noalign{\rlap{\smash{and}}}
\begin{aligned}\label{lin-split-b}
&\hbox to1.5cm{\hfill}{ {b_{t,x,\bar\csd,u}}}\times
\Bigl(\rho_ y \bigl({ {K_{t}(x,\bar\csd)}}\bigr) +1-\dd\Bigr) 
= { {h_{t,x,\bar\csd,u}^{ y ,v}}}
+ {{b_{t+1,\tts y ,\tts \T_{t,x}^y,\tts v}}}\times{ {h_{t,x,\bar\csd,u}^{ y ,v}}}\\
&\hbox to9.0cm{\hfill}\text{for all }\ \  y \in\XXX\,,\  u, v\in\EEE\,.
\end{aligned}
\end{gather}}%
\end{subequations}
Let us consider these two systems for a fixed productivity state $x\in\XXX$, assuming that an ansatz
choice for the value of $K_{t}(x,\bar\csd)$ is somehow made, and reminding that
${a_{t+1,\tts y ,\tts \T_{t,x}^y(\bar\csd),\tts v}}$ and ${b_{t+1,\tts y ,\tts \T_{t,x}^y(\bar\csd),\tts v}}$,
$\,y\in\XXX$, $ v\in\EEE$, are given.
Setting $c=\infty$ in \eqref{only-q-bn2-2}~-- thus removing the variables
$g_{t,x,\bar\csd,u}^{ y ,v}$ from that system~--
and combining the resulting $\abs{\EEE}$ equations with
the system in \eqref{lin-split-b}  provides
$\abs{\EEE}+\abs{\EEE}^2\abs{\XXX}$ equations for the same number of
unknowns, namely
{\abovedisplayskip=5pt plus 1.5pt minus 1.5pt\belowdisplayskip=5pt plus 1.5pt minus 1.5pt\belowdisplayshortskip=5pt plus 1.0pt minus 1.5pt
\[
{ {b_{t,x,\bar\csd,u}}}\,, \ { {h_{t,x,\bar\csd,u}^{ y ,v}}}\,,\quad
 u, v\in\EEE\,,\  y \in\XXX\,.
\]}%
Next, assuming that the above variables are already solved for,
setting in \eqref{only-q-bn2-2} $c=c^*$, for some sufficiently large but finite still
value $c^*>0$, and combining the resulting $\abs{\EEE}$ equations with the system in
\eqref{lin-split-a} gives $\abs{\EEE}+\abs{\EEE}^2\abs{\XXX}$ equations for the same number
of unknowns:
$
{ {a_{t,x,\bar\csd,u}}}\,, \ { {g_{t,x,\bar\csd,u}^{ y ,v}}}\,,$
$ u, v\in\EEE\,,\  y \in\XXX\,.$
If a solution for the unknowns
%%- -%@@%--{Eq: 4.5}--%%
{\abovedisplayskip=5pt plus 1.5pt minus 1.5pt\belowdisplayskip=5pt plus 1.5pt minus 1.5pt\belowdisplayshortskip=5pt plus 1.5pt minus 1.5pt
\begin{equation}\label{unkn2}
a_{t,x,\bar\csd,u}\,, \ b_{t,x,\bar\csd,u}\,, \ g_{t,x,\bar\csd,u}^{ y ,v}\,,
 \ h_{t,x,\bar\csd,u}^{ y ,v}\,,\quad  u, v\in\EEE\,,\  y \in\XXX\,,
\end{equation}
}%
can indeed be found as described, the ansatz choice for ${ {K_{t}(x,A)}}$ can then be tested
from the market clearing condition %\ref{def-equil}-(\ref{ze9x})
%%
%%- -%@@%--{Eq: 4.6}--%%
{\abovedisplayskip=5pt plus 1.5pt minus 1.5pt\belowdisplayskip=5pt plus 1.5pt minus 1.5pt\belowdisplayshortskip=5pt plus 1.5pt minus 1.5pt
\begin{equation}\label{mclr-lin}
\sum\nolimits_{u\ts\in\ts\EEE}\gp_x(u)\bigl(a_{t,x,\bar\csd,u}+b_{t,x,\bar\csd,u}\bar\csd^u\bigr)={ {K_{t}(x,\bar\csd)}}\,,
\end{equation}
}%
in which the averages $\bar\csd\equiv(\bar\csd^u\in\Rpp,\,u\in\EEE)$ are assumed given.
If the test fails, then the value for ${ {K_{t}(x,\bar\csd)}}$ will need to be adjusted
accordingly and the procedure will need to be repeated until the last relation becomes
numerically acceptable.
The transport encrypted in
\ref{main-q}-(\ref{zze666}), which now comes down to transport of the population averages,
obtains the form%%%%%%%%%%%%%%%%%%%%%%
%%%%%%%%%%%%%%%%
\footnote{The change of variables formula gives:
$\displaystyle \int \a \d \csd^u(\,\hat\pfc_{t,x,\bar\csd}^{ y ,v}(u,\a))
= \int \pfc_{t,x,\bar\csd}^{ y ,v}(u,\a) \d \csd^{u}(\a)\,.$}
%%- -%@@%--{Eq: 4.7}--%%
{\abovedisplayskip=5pt plus 1.5pt minus 1.5pt\belowdisplayskip=5pt plus 1.5pt minus 1.5pt\belowdisplayshortskip=5pt plus 1.5pt minus 1.5pt
\begin{equation}\label{no-fp-2-0}
\T_{t,x}^y(\bar\csd)^v
= \sum\nolimits_{u\in\EEE}
{\gp_x(u)  P_{x, y }( u,v)\over \gp_ y (v)}\,\Bigl(g_{t,x,\bar\csd,u}^{ y ,v} + 
h_{t,x,\bar\csd,u}^{y,v}\times \bar\csd^u\Bigr)\quad
\text{for all }\ x, y \in\XXX\,,\  v\in\EEE\,,
\end{equation}}%
%}
where the left side is understood to be average of the distribution $\T_{t,x}^y(\csd)^v$ (which
happens to depend only on the averages $\bar\csd^u$, $u\in\EEE$, due to the affine structure of the
consumption transfer mappings).
Hence, in this reduced model the transport operator $\T_{t,x}^ y $ introduced in \ref{et} now maps
$(\R_{++})^{\abs{\EEE}}$ into $(\R_{++})^{\abs{\EEE}}$ (as opposed to mapping $\bbF^\EEE$ into $\bbF^\EEE$).


%%- -%##%--{No: 4.1}--%%
\begin{nit}{Test for linearity and the choice of $c^*$}\label{c-test}\ \
The reason why the kernel condition in \eqref{only-q-bn2-2} needs to be solved with some fixed and
finite choice for $c=c^*$ is that the system in \eqref{lin-split-a} gives only $\abs{\EEE}^2\abs{\XXX}$
equations. Whence the need for $\abs{\EEE}$ additional equations in order to solve for the unknowns
${ {a_{t,x,\bar\csd,u}}}\,, \ { {g_{t,x,\bar\csd,u}^{ y ,v}}}\,,$
$ u, v\in\EEE\,,\  y \in\XXX$. % (eliminating 
%$g_{t+1,x,u}^{ y ,v}$ from \eqref{only-q-bn2-2} by setting $c=\infty$ is no longer adequate).
If $c^*$ is sufficiently large, then the effect of its choice on the solution would be negligible,
but how large is large enough and how negligible is negligible enough?
This question can
be answered with the following simple test: One must solve the system
(\ref{only-q-bn2-2})~\&~(\ref{lin-split}) twice, once with $c=c^*$ and then once more with 
$c=2c^*$. If the linearity assumption holds and $c^*$ is large enough, then the uniform distance
%in $\R^{2\tts\abs{\EEE}+2\tts\abs{\EEE}^2\abs{\XXX}}$
between the two solutions for the unknowns
in \eqref{unkn2} should be closer than some prescribed threshold. %, say, closer than
% $\hbox{constant}\times 10^{-4}$ (in the concrete implementation equations \eqref{only-q-bn2-2}
% and \eqref{lin-split} are solved with the tolerance set in the non-linear solver, which is much higher).
If~this test fails, the value for $c^*$ is to be increased (say, doubled)
and the test is to be repeated with the new value.\qed
%These steps have to be repeated until a large
%enough value for $c^*$ that passes the test is found. If~no such value exists, then an approximate equilibrium with
%the affine structure assumed in \eqref{lin-subst} cannot be found. Fortunately, at least in the
%benchmark economy borrowed here from \nocite{KS98}{[KS]}, such a failure does not occur.\qed% and a sufficiently
%large value for $c^*$ obtains after only a few iterations.\qed 
\end{nit}
%\nitskip


The reduction of the model to the affine structure introduced in \eqref{lin-subst}
leads to some useful closed form expressions as is
detailed next.
As no investment takes place in period $T$, with  $t=T-1$ one has
$a_{t+1,\tts y,\tts \T_{t,x}^y(\bar\csd),\tts v}=b_{t+1,\tts y,\tts \T_{t,x}^y(\bar\csd),\tts v}=0$,
so that \eqref{lin-split-b} becomes
{\abovedisplayskip=5pt plus 1.5pt minus 1.5pt\belowdisplayskip=5pt plus 1.5pt minus 1.5pt\belowdisplayshortskip=5pt plus 1.5pt minus 1.5pt
\[
h_{t,x,\bar\csd,u}^{y,v} = b_{t,x,\bar\csd,u} \times
\Bigl(\rho_ y \bigl({ {K_{t}(x,\bar\csd)}}\bigr) +1-\dd\Bigr)\,,\quad  y \in\XXX\,,\  u, v\in\EEE\,,
\]}%
and, consequently, with $c=\infty$ \eqref{only-q-bn2-2} gives, for every $x\in\XXX$ and $u\in\EEE$,
{\abovedisplayskip=5pt plus 1.5pt minus 1.5pt\belowdisplayskip=5pt plus 1.5pt minus 1.5pt\belowdisplayshortskip=5pt plus 1.5pt minus 1.5pt 
\[
1 = { \b \over { {b_{t,x,\bar\csd,u}}}}\sum\nolimits_{ y \,\in\,\XXX}
P(x, y )\sum\nolimits_{ v\,\in\,\EEE} Q_{x, y }( u ,v)=
{ \b \over { {b_{t,x,\bar\csd,u}}}}\sum\nolimits_{ y\,\in\,\XXX} P(x, y )={ \b \over { {b_{t,x,\bar\csd,u}}}}
\,.
\]}%
As a result, ${ {b_{T-1,x,u}}}=\b$ for every $x\in\XXX$ and $u\in\EEE$.
Hence, with  $t=T-2$ \eqref{lin-split-b} gives
{\abovedisplayskip=5pt plus 1.5pt minus 1.5pt\belowdisplayskip=5pt plus 1.5pt minus 1.5pt\belowdisplayshortskip=5pt plus 1.5pt minus 1.5pt
\[
h_{t,x,\bar\csd,u}^{ y ,v} = {b_{t,x,\bar\csd,u}\over 1+ \b}\times
\Bigl(\rho_ y \bigl({ {K_{t}(x,\bar\csd)}}\bigr) +1-\dd\Bigr) 
\,,\quad\ x, y \in\XXX\,,\  u, v\in\EEE\,.
\]}%
With the last substitution and with $c=\infty$ \eqref{only-q-bn2-2} becomes
{\abovedisplayskip=5pt plus 1.5pt minus 1.5pt\belowdisplayskip=5pt plus 1.5pt minus 1.5pt\belowdisplayshortskip=5pt plus 1.5pt minus 1.5pt
\[
1 = { \b (1+\b)\over b_{t,x,\bar\csd,u}}
\,,\quad x\in\XXX\,,\ u\in\EEE\,,
\]}%
and gives ${ {b_{T-2,x,\bar\csd,u}}}=\b+\b^2$.
By induction, for $t=T-n$ one has
%%- -%@@%--{Eq: 4.8}--%%
{\abovedisplayskip=5pt plus 1.5pt minus 1.5pt\belowdisplayskip=5pt plus 1.5pt minus 1.5pt\belowdisplayshortskip=5pt plus 1.5pt minus 1.5pt
\begin{equation}\label{explicit-b}
b_{t,x,\bar\csd,u}=\b+\b^2+\cdots+\b^n ={1-\b^{n+1}\over 1-\b}-1
\end{equation}}%
and,
therefore, \eqref{lin-split-b} gives
%%- -%@@%--{Eq: 4.9}--%%
{\abovedisplayskip=5pt plus 1.5pt minus 1.5pt\belowdisplayskip=5pt plus 1.5pt minus 1.5pt\belowdisplayshortskip=5pt plus 1.5pt minus 1.5pt
\begin{equation}\label{explicit-h}
{ {h_{t,x,\bar\csd,u}^{ y ,v}}} =
{\b+\cdots+\b^n\over 1+\b+\cdots+\b^{n-1}}
\,\Bigl(\rho_ y \bigl({ {K_{t}(x,\bar\csd)}}\bigr) +1-\dd\Bigr) =
\b\,\Bigl(\rho_ y \bigl({ {K_{t}(x,\bar\csd)}}\bigr) +1-\dd\Bigr) .
\end{equation}}%
With the above substitution \eqref{only-q-bn2-2} becomes
%%- -%@@%--{Eq: 4.10}--%%
{\abovedisplayskip=4pt plus 1.0pt minus 1.5pt\belowdisplayskip=4pt plus 1.0pt minus 1.5pt\belowdisplayshortskip=4pt plus 1.0pt minus 1.5pt
\begin{equation}\label{only-q-bn2-3}
%\begin{aligned}
1 = \sum_{ y\, \in\,\XXX, v\,\in\,\EEE}\,
{ \b\,\bigl(\rho_ y \bigl(K_{t}(x,\bar\csd)\bigr) + 1-\dd\bigr)
\over g_{t,x,\bar\csd,u}^{ y ,v}/c  + \b\,\bigl(\rho_ y \bigl(K_{t}(x,\bar\csd)\bigr) + 1-\dd\bigr)}
%\times
Q(x, y )\,  P_{x, y }( u ,v)\,,%\\
%\hbox to7.5cm{\hfil}\quad\
\  x\in\XXX\,,\ts u\in\EEE\,.
%\end{aligned}
\end{equation}}%
With $t=T-n$, $n\ge 1$, again (\ref{lin-split}a) becomes
%%- -%@@%--{Eq: 4.11}--%%
{\abovedisplayskip=5pt plus 1.5pt minus 1.5pt\belowdisplayskip=5pt plus 1.5pt minus 1.5pt\belowdisplayshortskip=5pt plus 1.5pt minus 1.5pt
\begin{equation}\label{lin-split-2}
\begin{aligned}
&{ {a_{t,x,\bar\csd,u}}}\times
\Bigl(\rho_ y \bigl({ {K_{t}(x,\bar\csd)}}\bigr) +1-\dd\Bigr) 
+ v\times \ee_ y \bigl({ {K_{t}(x,\bar\csd)}}\bigr)%\hbox to4.5cm{\hfil}\\
%\hbox to4.5cm{\hfil}
= {{a_{t+1,\tts y ,\tts \T_{t,x}^y(\bar\csd),\tts v}}}+{1-\b^n\over 1-\b}\times { {g_{t,x,\bar\csd,u}^{ y ,v}}}\,,\\
&\hbox to9.5cm{\hfil}\quad\ \  x, y \in\XXX\,,\  u, v\in\EEE\,.
\end{aligned}
\end{equation}}%
In particular, letting  $n\to\infty$ leads to the following
(time-invariant) values for the slopes of the portfolios and the consumption transfer mappings:
%%
{\abovedisplayskip=5pt plus 1.5pt minus 1.5pt\belowdisplayskip=5pt plus 1.5pt minus 1.5pt\belowdisplayshortskip=5pt plus 1.5pt minus 1.5pt
\[
{ {b_{\infty,x,\bar\csd,u}}}\df{\b\over1-\b}\quad\text{and}\quad
{ {h_{\infty,x,\bar\csd,u}^{ y ,v}}}\df
\b\,\Bigl(\rho_ y \bigl({ {K_{\infty}(x,\bar\csd)}}\bigr) +1-\dd\Bigr) \,.
\]}%
Remarkably, with a very large time horizon,
the slopes of all portfolio mappings are identical to a universal constant, while the slopes
of the consumption transfer mappings depend on the state of the population $\bar\csd$ only through the collectively
installed capital and do not depend on the state of employment. %%%%%%%%%%%%%%%%%%%%%%%%
%%%%%%%%%%%%%%%%%%%%%%%%%%%%%%%%%%%
%\footnote{This result is consistent with \citel{Bew77}.}
%%%%%%%%%%%%%%%%%%%%%%%%%%%%%%%%%%%%
The same claim cannot be made
about the intercepts $a_{t,x,\bar\csd,u}$ and $g_{t,x,\bar\csd,u}^{v, y }$
postulated in~\eqref{lin-subst}, and one can only hope that time invariant versions of the mappings
%%
{\abovedisplayskip=5pt plus 1.5pt minus 1.5pt\belowdisplayskip=5pt plus 1.5pt minus 1.5pt\belowdisplayshortskip=5pt plus 1.5pt minus 1.5pt
\[
(x,u,\bar\csd) \leadsto a_{t,x,\bar\csd,u}\,, \quad
(x,y,u,v,\bar\csd) \leadsto g_{t,x,\bar\csd,u}^{v, y}\,,\quad\text{and}\quad
(x,\bar\csd) \leadsto K_t(x,\bar\csd)
\]
}%
exist. Because of the implicit structure of equation \eqref{lin-split-2}, such an existence is
very difficult to arrive at by way of the usual fixed point argument.
Nevertheless, it will be shown below that, at least in the
benchmark example that we borrow here from \nocite{KS98}{[KS]}, the existence of time invariant versions
of the above mappings is rather easy to establish numerically: by following the general
iteration strategy from \ref{main-proc} for a sufficiently large number of periods one arrives at
successive copies of these functions that coincide (in uniform distance) within a prescribed threshold.
The time-invariant version of the transport in \eqref{no-fp-2-0}, if one exists, is given by:
%%%%%%%%%%%%%%%%%%%%%%
%%%%%%%%%%%%%%%%
%\footnote{The change of variables formula gives:
%$\displaystyle \int u \d \csd_{t,x,u}(\,\hat\pfc_{t,x,u}^{ y ,v}(u))
%= \int \pfc_{t,x,u}^{ y ,v}(u) \d \csd_{t,x,u}(u)\,.$}
%%%%%%%%%%%%%%%%%%
%{\abovedisplayskip=5pt plus 1.5pt minus 5pt
%\abovedisplayshortskip=-7pt plus 1.5pt minus5.5pt
%\belowdisplayskip=5pt plus 1.5pt minus 1.5pt
%\belowdisplayshortskip=6pt plus 1.5pt minus 1.5pt
%
%%- -%@@%--{Eq: 4.12}--%%
{\abovedisplayskip=7pt plus 3.5pt minus 1.5pt\belowdisplayskip=7pt plus 3.5pt minus 1.5pt\belowdisplayshortskip=7pt plus 3.5pt minus 1.5pt
\begin{equation}\label{no-fp-2}
\T_{\infty,x}^y(\bar\csd)^v
= \sum\nolimits_{u\in\EEE}
{\gp_x(u)  P_{x, y }( u,v)\over \gp_ y (v)}\,\Bigl(g_{\infty,x,u}^{ y ,v} + 
h_{\infty,x,u}^{ y ,v}\times \bar\csd^u\Bigr)\quad
\text{for all }\ x, y \in\XXX\,,\  v\in\EEE\,,
\end{equation}
}%

%%- -%##%--{No: 4.2}--%%
\begin{nit}{Remark}
While time-invariant,
the transport in  \eqref{no-fp-2} depends not only on the present productivity state $x$, but also on
the future one, $ y $, which makes it impossible to produce a time invariant version of
the averages~$\bar\csd$ (the population averages fluctuate randomly after an arbitrarily large
number of periods). Indeed, these averages follow a Markov process in random
environment, and there is no reason for such a process to converge to a constant vector.
This is one of the main challenges that aggregate risk and
production entail.\qed
%Nevertheless, as will be illustrated later in this section,
%for sufficiently large values of $t$ the fluctuations of $A_{t,x}$ are
%confined to a compact region, which is stable, in the sense that it does not depend on the initial choice for
%$A_{0,x}$.\qed
\end{nit}
\nitskip

The next step is to reformulate the strategy from \ref{main-proc} in terms of
the setup adopted in the present section and the reduced form postulated
in \eqref{lin-subst}. %~-- the main idea is similar to that of~\ref{abridged}. 
%While our ultimate objective is to understand the nature of the equilibrium
%as $T\to\infty$, we are still going to rely on the program that assumes finite time horizon $T<\infty$, but
%increase $T$ to the point where the limiting form described above can be confirmed with an
%acceptable numerical accuracy.  
%\nitskip

%%- -%##%--{No: 4.3}--%% 
\begin{nit}{Metaprogram with affine structure}\label{KS-proc}
Due to the explicit formulas in \eqref{explicit-b} and \eqref{explicit-h},
the only unknowns that need to be computed
are the collectively installed capital $K_{t}(x,\bar\csd)$ and the intercepts %(see \eqref{lin-subst})
$a_{t,x,\bar\csd,u}$  and $g_{t,x,\bar\csd,u}^{ y ,v}$,
for all choices of $ u, v\in\nobreak\EEE\,,\ x,  y \in\XXX\,,$
and $t<T$. 
As these unknowns depend on the period $t$ vector of averages
$\bar\csd\equiv (\bar\csd^u,\, u\in\EEE)\in\R^{\abs{\EEE}}$,
%In our benchmark example, the parameters for which we borrow from
%\nocite{KS98}{[KS]}, one has  $\abs{\EEE}=2$, so that
%$K_{t,x}$, $a_{t,x,u}$ and ${ {g_{t+1,x,u}^{ y ,v}}}$ can be treated as functions on $\R^2$.
the objects we are looking for are functions on $\R^\abs{\EEE}$ that are labeled by $t$, $u$, $v$,
$x$ and $y$.
%In the actual computer program these objects will be implemented as 2D-splines.
The general procedure from \ref{main-proc} comes down to the following steps:
 

%\noindent
{\it Initial setup:} Make an ansatz choice for a finite interpolation grid, $\bbG\subset\R^\abs{\EEE}$, the
elements of which represent reduced (to averages) cross-sectional distributions of all households,
i.e., every element of $\bbG$ is an array of average consumption levels attached to employment categories.

%\noindent
{\it Initial backward step:}  Set $n=1$ and $t=T-n$. For every $x\in\XXX$ do:

For every $\bar\csd\equiv (\bar\csd^u,\,u\in\EEE)\in\bbG$ perform steps~(1) through~(5) below, then
proceed to~(6):

{(1)} Make an ansatz choice for $K_{t}(x,\bar\csd)>0$ and go to (2).

{(2)} Make an ansatz choice for $c^*>0$ and got to (3).

{(3)} Solve the system composed of \eqref{lin-split-2} with $a_{t+1,\tts y
,\tts \T_{t,x}^y(\bar\csd),\tts v}\equiv 0$
and \eqref{only-q-bn2-3} with $c\equiv c^*$ (total of $\abs{\EEE}+\abs{\EEE}^2\abs{\XXX}$ equations~--
note that $x$ is fixed) for the unknowns ($\abs{\EEE}+\abs{\EEE}^2\abs{\XXX}$ in number)
$a_{t,x,\bar\csd,u}$ and $g_{t,x,\bar\csd,u}^{ y ,v}$, $ y \in\XXX$, $ u, v\in\EEE$. Proceed to (4).

{(4)} If the choice of $c^*$ fails the linearity test in \ref{c-test}, set $c^*=2c^*$ and go
back to~(3), otherwise proceed to~(5).

{(5)} Test the market clearing (see \eqref{mclr-lin} and recall that $b_{T-1,x,\bar\csd,u}=\b$)
{\abovedisplayskip=5pt plus 1.5pt minus 1.5pt\belowdisplayskip=5pt plus 1.5pt minus 1.5pt\belowdisplayshortskip=5pt plus 1.5pt minus 1.5pt
\[
K_{t}(x,\bar\csd)=\sum\nolimits_{u\ts\in\ts\EEE}\gp_x(u)\bigl(a_{t,x,\bar\csd,u}+\b\bar\csd^u\bigr)\,.
\]
}%
If this relation fails by more than a prescribed threshold, set the new value of $K_{t}(x,\bar\csd)$ to the
right side and go back to (2).

{(6)} Construct spline interpolation objects over the grid $\bbG$ from the most recently calculated
values for $K_{t}(x,\bar\csd)$, $a_{t,x,\bar\csd,u}$ and  $g_{t,x,\bar\csd,u}^{ y ,v}$,
$\bar\csd\in\bbG$, for every $y\in\XXX$ and $u,v\in\EEE$.


{\it Generic backward step:} Set $n=n+1$ and $t=T-n$,
assuming that $\bar\csd \leadsto a_{t+1, y ,\bar\csd,v}$ are already computed functions %(spline objects)
on $\R^\abs{\EEE}$  for every $ y \in\XXX$ and  every $ v\in\EEE$. For every $x\in\XXX$ do:

For every $\bar\csd\equiv (\bar\csd^u,\,u\in\EEE)\in\bbG$ complete steps (1) through (7) below,
then proceed to~(8):

{(1)} Make an ansatz choice for $K_{t}(x,\bar\csd)>0$, then go to (2)

{(2)} Set $\pdg\bar\csd_{y }=\bar\csd$ for every $ y \in\XXX$ (initial guess for the future state of the
population in every future productivity state), then go to (3).

{(3)} Make an ansatz choice for $c^*>0$, then go to (4). 

{(4)} Solve the system composed of \eqref{lin-split-2} with $a_{t+1, y ,\T_{t,x}^y(\bar\csd), v}=
a_{t+1, y , \pdg\bar\csd_{y }, v}$
and \eqref{only-q-bn2-3} with $c\equiv c^*$ (total of $\abs{\EEE}+\abs{\EEE}^2\abs{\XXX}$ equations~--
note that $x$ is fixed) for the unknowns ($\abs{\EEE}+\abs{\EEE}^2\abs{\XXX}$ in number)
$a_{t,x,\bar\csd,u}$ and $g_{t,x,\bar\csd,u}^{ y ,v}$, $ y \in\XXX$, $ u, v\in\EEE$. Proceed to (5).

{(5)} If the choice of $c^*$ fails the linearity test in \ref{c-test}, set $c^*=2c^*$ and go
back to (4), otherwise proceed to (6).

{(6)} Test the market clearing
(see \eqref{mclr-lin} and recall that $b_{T-n,x,\bar\csd,u}=\b+\cdots+\b^n$)
{\abovedisplayskip=5pt plus 1.5pt minus 1.5pt\belowdisplayskip=5pt plus 1.5pt minus 1.5pt\belowdisplayshortskip=5pt plus 1.5pt minus 1.5pt
\[
K_{t}(x,\bar\csd)=\sum\nolimits_{u\ts\in\ts\EEE}\gp_x(u)\bigl(a_{t,x,\bar\csd,u}+(\b+\cdots+\b^n) \bar\csd^u\bigr)\,.
\]
}%
If this relation fails by more than a prescribed threshold, set the new value of $K_{t}(x,\bar\csd)$ to the
right side and go back to~(3), otherwise proceed to~(7).

{(7)} Compute (see \eqref{no-fp-2-0} and \eqref{explicit-h})
{\abovedisplayskip=5pt plus 2.5pt minus 1.5pt\belowdisplayskip=5pt plus 2.5pt minus 1.5pt\belowdisplayshortskip=5pt plus 2.5pt minus 1.5pt
\[
(\pst\bar\csd_{y})^v
= \sum\nolimits_{u\ts\in\ts\EEE}
{\pi_x(u) Q_{x, y }( u,v)\over \pi_ y (v)}\,\Bigl(g_{t,x,\bar\csd,u}^{ y ,v} + 
\b\,\Bigl(\rho_ y \bigl({ {K_{t}(\bar\csd)}}\bigr) +1-\dd\Bigr)\times \bar\csd^u\Bigr)
\]}%
for every $ y \in\XXX$ and every $ v\in\EEE$. If the uniform distance
{\abovedisplayskip=5pt plus 1.5pt minus 1.5pt\belowdisplayskip=5pt plus 1.5pt minus 1.5pt\belowdisplayshortskip=5pt plus 1.5pt minus 1.5pt
\[
\max\nolimits_{ y\ts\in\ts\XXX,\, v\ts\in\ts\EEE}\abs{(\pst\bar\csd_{y})^v - (\pdg\bar\csd_{y})^v}
\]}%
exceeds some prescribed threshold (the guessed future averages are not compatible with the
transport),
set $\pdg\bar\csd_{y}=\pst\bar\csd_{y}$ for all $y\in\XXX$
and go back to (3).

{(8)} Construct spline interpolation objects over the grid $\bbG$ from the most recently calculated
values for $K_{t}(x,\bar\csd)$, $a_{t,x,\bar\csd,u}$ and  $g_{t,x,\bar\csd,u}^{ y ,v}$,
$\bar\csd\in\bbG$, for every $y\in\XXX$ and $u,v\in\EEE$.

{\it Final backward step:} If at least one of the uniform distances
{\abovedisplayskip=5pt plus 1.5pt minus 1.5pt\belowdisplayskip=5pt plus 1.5pt minus 1.5pt\belowdisplayshortskip=5pt plus 1.5pt minus 1.5pt
\[
\begin{gathered}
\max_{x\ts\in\ts\XXX,\,\bar\csd\ts\in\ts\bbG}\abs{K_{t+1}(x,\bar\csd)-K_{t}(x,\bar\csd)}\,,\ %
\max_{x\ts\in\ts\XXX,\,u\ts\in\ts\EEE,\,\bar\csd\ts\in\ts\bbG}\abs{a_{t+1,x,\bar\csd,u}-a_{t,x,\bar\csd,u}}\,,\\
\max_{x, y \ts\in\ts\XXX,\, u, v\ts\in\ts\EEE,\,\bar\csd\ts\in\ts\bbG}\abs{g_{t+1,x,\bar\csd,u}^{ y ,v}-g_{t,x,\bar\csd,u}^{ y ,v}}
\end{gathered} 
\]}%
exceeds some prescribed threshold, perform another generic backward step.%%%%%%%%%%%%%%%%%%%
%%%%%%%%%%%%%%%%%%%%%%%%%%%%%%%%%%%%%%%%%%%%%%
\footnote{The time parameter $t$ may become negative~-- the program moves backward in time for as
many periods as needed to achieve time invariance.
The final backward step would not be necessary if seeking time invariant
transitions is not the objective, in which case the program can be terminated at $t=0$.}
%%%%%%%%%%%%%%%%%%%%%%%%%%%%%%%%%%%%%%%%%%%%%%
Otherwise stop and define
the functions
{\abovedisplayskip=5pt plus 1.5pt minus 1.5pt\belowdisplayskip=5pt plus 1.5pt minus
1.5pt\belowdisplayshortskip=5pt plus 1.5pt minus 1.5pt
$$
\bar\csd \leadsto K_{\infty}(x,\bar\csd)\,,\quad
\bar\csd \leadsto a_{\infty,x,\bar\csd,u}\,,\quad
\bar\csd \leadsto g_{\infty,x,\bar\csd,u}^{ y ,v}\,,\quad x,y\in\XXX\,,\ \ u,v\in\EEE\,,
$$
}%
as the most recently computed
spline objects (with the latest value for $t$).\qed  
\end{nit}
%\nitskip


%%- -%##%--{No: 4.4}--%% 
\begin{nit}{Remark}
The iterations between steps (3) and (7) in the generic backward step are meant
to establish the correct connection between future and present distribution averages. We stress
that this
adjustment is local in time, in that the program does not move to the next period in the backward
induction (which is the previous period in real time)
until the correct transport from the current period is established.\qed
%In contrast, the
%simulation algorithm used in \nocite{KS98}{[KS]} applies the same transition in a very large number of
%periods.\qed
\end{nit}


%%- -%##%--{No: 4.5}--%%
\begin{nit}{Model accuracy}\label{mod-accu}
Despite the reduction to a model with affine structure, the program outlined
in \ref{KS-proc} solves exactly~-- meaning, within the nonlinear solver's tolerance of the
infinity norm of the residuals~-- all equations that define the equilibrium, except for the kernel
condition \eqref{only-q-bn2-3}. This condition is asymptotically exact for
agents with arbitrarily large consumption levels, but the output from \ref{KS-proc}
would not be very useful without a meaningful estimate of how big the aberration from
\eqref{only-q-bn2-3} is in the rest of the consumption range. With infinite time horizon in mind, the right side of
\eqref{only-q-bn2-3} can be treated as a function of the collective state %(vector of group averages)
of the
population $A_{t,x}$ and the consumption level $c$, i.e., can be cast as
{\abovedisplayskip=7pt plus 1.5pt minus 1.5pt\belowdisplayskip=7pt plus 1.5pt minus 1.5pt\belowdisplayshortskip=5pt plus 1.5pt minus 1.5pt
\begin{equation*}\tag{a}\label{k-check}
W_{\infty,x,\bar\csd,u}(c) \df \sum_{ y\ts \in\ts\XXX, v\ts\in\ts\EEE} \ 
{ \b\,\bigl(\rho_ y \bigl(K_{\infty}(x,\bar\csd)\bigr) + 1-\dd\bigr)
\over g_{\infty,x,\bar\csd,u}^{ y ,v}/c  + \b\,\bigl(\rho_ y \bigl(K_{\infty}(x,\bar\csd)\bigr) + 1-\dd\bigr)}
%\times
Q(x, y )\,  P_{x, y }( u ,v)\,.
\end{equation*}}%
Once the program in \ref{KS-proc} is completed,
the value $W_{\infty,x,\bar\csd,u}(c)$ can be calculated at every vector $\bar\csd$ in
the long run range of the distribution averages and for every $c>0$. Since our model and the output
from the program in \ref{KS-proc} are invariant with respect to choices for the
cross-sectional distribution of the population that share identical average values~-- the support
of which could be any cube in $(\R_{++})^\abs{\EEE}$ that includes $\bar\csd$~-- we cannot make
precise statements about the range of the private consumption levels~$c$. Nevertheless, an estimate
that immediately comes to mind is to calculate the kernel aberration with consumption levels equal to the
group averages, i.e., produce the lists $\bigl(W_{\infty,x,\bar\csd,u}(\bar\csd)-1,\, u\in\EEE\bigr)$ for
the various choices of $\bar\csd$ from the long run range of the population averages. Another,
somewhat more demanding, estimate for the kernel aberration is $W_{\infty,x,\bar\csd,u}(\bar c_{x,\bar\csd,u})-1$,
where $\bar c_{x,\bar\csd,u}$ is the investment threshold, i.e., the solution to
$\qq_{\infty,x,\bar\csd,u}(c)=0$. In our reduced model this is nothing but the intersection of the
line
$c \leadsto a_{\infty,x,\bar\csd,u}+b_{\infty,x,\bar\csd,u}\times c$ with the horizontal axis, i.e.,
{\abovedisplayskip=5pt plus 1.5pt minus 1.5pt\belowdisplayskip=5pt plus 1.5pt minus 1.5pt\belowdisplayshortskip=5pt plus 1.5pt minus 1.5pt
\[
\bar c_{x,\bar\csd,u} = -{a_{\infty,x,\bar\csd,u}\over b_{\infty,x,\bar\csd,u}} = -{(1-\b) a_{\infty,x,\bar\csd,u}\over \b}\,.
\]}%
In the concrete implementation in the benchmark economy from \nocite{KS98}{[KS]}
these estimates of the kernel aberration are
of order $10^{-3}$ to $10^{-4}$~-- uniformly throughout the long run range of population averages.
Although the reduced model studied in the present section does not allow one to analyze
equilibria in which a non-negligible proportion of the population has consumption close to~$0$
(if such equilibria exist~-- see Appendix~B), %. The
%
%behavior near $c=0$ is delicate~-- see also \ref{bound-on-cons}~-- because the portfolio mappings
%can no longer be claimed to be
%approximately affine, with the implication that the cross-sectional distribution of households
%can no longer be reduced to the vector of group averages. Nevertheless, the reduced model
%
it still provides a good approximation of equilibria in which the consumption levels are bounded
uniformly away from~$0$.\qed
%(i.e., the actual distribution of households in category $u$ is supported by the interval
%$\llbkt\bar c_{x,u},\infty\rlbkt$),
%which is a common assumption in most of the literature, including in~\nocite{KS98}{[KS]}.\qed
\end{nit}
%\medskip
%\vskip-3pt 

To put the methodology developed in the present section to the test,
we now turn to some concrete examples and numerical results.
Since the objective here is to benchmark the new method against those that precede it, in what
follows we focus exclusively on the infinite time horizon case, but stress that the program
developed here is designed to work only with finite time horizon (of any length), and ``infinite
time horizon'' is understood as a ``sufficiently large finite time horizon.'' This point is important
because in order to arrive at time invariant transport mappings
the program must run for hundreds~-- sometimes thousands~-- of periods, whereas
no real world economy could remain compliant with a particular model for that long; whence the need
for an algorithm that can produce meaningful results even with a relatively small $T$.
In~what follows we borrow the setup and parameter values
from the benchmark economy described in
\nocite{KS98}{[KS]}: the list of productivity states of the economy is $\XXX=\{1.01,\,
0.99\}$ (the economy is either in high state, labeled~``$1$,'' or in low state, labeled ``$2$''), with
transition probability matrix for these states
{\abovedisplayskip=3pt plus 1.5pt minus 1.5pt\belowdisplayskip=5pt plus 1.5pt minus 1.5pt\belowdisplayshortskip=5pt plus 1.5pt minus 1.5pt
\[
 Q=
\begin{bmatrix}
 0.875 & 0.125\\
 0.125 & 0.875
\end{bmatrix}\,,
\]}%
the list of employment states is $\EEE=\{\h,0\}\df \{0.3271,0.0\}$ (agents can be
either employed or unemployed), with (private)
conditional transition probability matrices for these states
{\abovedisplayskip=5pt plus 1.5pt minus 1.5pt\belowdisplayskip=5pt plus 1.5pt minus 1.5pt\belowdisplayshortskip=5pt plus 1.5pt minus 1.5pt
\begin{gather*}
 P_{1,1}=\begin{bmatrix}
 0.972222 & 0.0277778\\
 0.666667 & 0.333333
 \end{bmatrix}\,,\quad
 P_{1,1}=\begin{bmatrix}
 0.927083 & 0.0729167\\
 0.25   &   0.75
\end{bmatrix}\,,
\end{gather*}
\begin{gather*}
 P_{2,1}=\begin{bmatrix}
 0.983333 & 0.0166667\\
 0.75    &  0.25
  \end{bmatrix}\,,\quad
 P_{2,2}=\begin{bmatrix}
 0.955556 & 0.0444444\\
 0.4     &  0.6
\end{bmatrix}\,,
\end{gather*}}%
which correspond to $\gp_1= [0.96, 0.04]$ and  $\gp_2= [0.9, 0.1]$. The parameter values are $\b=0.99$,
$\dd=0.025$, $\a=0.36$, $R=1$ (risk aversion). The total labor supplied in high state is
$L(\EEE_1)=0.314016$ and in low state it is $L(\EEE_2)= 0.29439$.
The time-invariant solution, i.e., the functions
{\abovedisplayskip=5pt plus 1.5pt minus 1.5pt\belowdisplayskip=5pt plus 1.5pt minus
1.5pt\belowdisplayshortskip=5pt plus 1.5pt minus 1.5pt
$$
\bar\csd \leadsto K_{\infty}(x,\bar\csd)\,,\quad
\bar\csd \leadsto a_{\infty,x,\bar\csd,u}\,,\quad
\bar\csd \leadsto g_{\infty,x,\bar\csd,u}^{ y ,v}\,,\quad x,y\in\XXX\,,\ \ u,v\in\EEE\,,
$$
}%
are constructed as $2$D-splines on a fixed interpolation grid $\bbG\subset \R_{++}^2$.
The choice of the corresponding domain $[\bbG]$ is ad hoc.
This domain is %narrow enough so that the transport mappings $\T_x^ y \phd$ 
%do not take grid points outside of $[\bbG]$
sufficiently broad so that
the long run range of the population averages is included in $[\bbG]$.
It~is assumed that the average (i.e., total)
consumption of all employed households is larger than that of the unemployed households, i.e.,
all relevant functions are to be computed only at population averages $\bar\csd=(\bar\csd^\h,\bar\csd^0)$ with
$\bar\csd^\h > \bar\csd^0$. To turn
such a domain into a rectangle (in order to construct standard $2$D-splines),
the second component of every grid point is interpreted as a
percentage $p\in\lrbkt 0,100\rlbkt\,$, so that the pair $(\bar\csd^\h,p)$ corresponds to the pair of
consumption averages $\bar\csd=(\bar\csd^\h,\bar\csd^0)$ with
$\bar\csd^0=p\times \bar\csd^\h/100$.
The metaprogram \ref{KS-proc}
was translated into Julia with parallelization.
With $16$ CPUs the program completes $10^3$ iterations under $7$ minutes
and achieves convergence (measured in uniform distance over all grid-points between the last two
copies of the respective values) of $4.31662\times 10^{-5}$ across all values
$(a_{\infty,x,\bar\csd,u},\,\bar\csd\in\bbG)$, $7.29768\times 10^{-9}$ across all values
$(g_{\infty,x,\bar\csd,u}^{ y ,v},\,\bar\csd\in\bbG)$, $9.71069\times 10^{-9}$ across all values
$(K_\infty(x,\bar\csd),\,\bar\csd\in\bbG)$,
and $6.33378\times 10^{-9}$ across all future averages $(\T_x^{ y,v}(\bar\csd),\,\bar\csd\in\bbG)$. 
In~all iterations the function tolerance in the nonlinear
solver (the NLsolve package was used) was set to $10^{-12}$ and all outputs were explicitly tested for
compliance with this tolerance. The model accuracy boils down to the size of the aberrations in the kernel
condition, i.e., how close to $1$
the expression in \ref{mod-accu}-(\ref{k-check})~is, uniformly
across all grid points $\bar\csd\in\bbG$.
As~this expression is also a function of the
consumption level $c$, one would like to see a uniformly small aberration from $1$ also over
some reasonable range for~$c$ (clearly, not too close to~$0$).
%Since the reduced form of the general model that is used here
%keeps track only of the average consumption levels of employed and unemployed, one cannot make a
%precise statement about how far from those averages the actual consumption levels are
%spread.
Following the recipe from \ref{mod-accu},
we test how close to $1$  the expression $W_{\infty,x,\bar\csd,u}(c)$ is
with $c=\bar\csd^\h$ if  $u=\h$,  with $c=\bar\csd^0$ if $u=0$,
and also with the value for $c$ that solves $a_{\infty,x,\bar\csd,u}+{\b\over{1-\b}}c=0$;
i.e., at the level of consumption that implies zero investment. 
If capital cannot be borrowed, then the population
distribution would be amassed above that threshold.
The plots in Figure~\ref{fgKS4} show the no-investment threshold $\ell_{x,\bar\csd,u}=-a_{\infty,x,\bar\csd,u}{1-\b\over\b}$
relative to the respective average $\bar\csd^u$, $u\in\EEE$ (how far below the average can consumption
fall before positive investment stops).
%%
%%
%%%%%%%%%%%%%%%%%%%%%%%%%%%%%%%%%%%%%%%%%%
%%     Fg: 10
%%%%%%%%%%%%%%%%%%%%%%%%%%%%%%%%%%%%%%%%%%
{\captionsetup{belowskip=-5pt}
%\captionsetup{aboveskip=10pt}
\begin{figure}[!htbp]
\centering
\begin{subfigure}{.5\textwidth}
  \centering
\leavevmode%\rlap{\quad\raise2.7cm\hbox{\rotatebox{90}{%
%\rlap{\rotatebox{-90}{\tiny $A_2-\ell_{1,2}$}\rotatebox{-90}{\tiny $A_1-\ell_{1,1}$}}}}}\ %
\ %  
\toshow{\includegraphics[width=7.5cm]{fg10L}}

%\leavevmode\smash{\llap{\quad\raise2.7cm\hbox{\rotatebox{90}{%
%\rlap{\rotatebox{-90}{\tiny $A_2-\ell_{1,2}$}\rotatebox{-90}{\tiny $A_1-\ell_{1,1}$}}}}}}\ %

\leavevmode\smash{\kern-4cm\raise1.2cm\hbox{\tiny \rotatebox{-20}{ave.\ cons.\ employed}}%
\leavevmode\llap{\raise2.8cm\hbox{\tiny $\bar\csd^0-\ell_{1,\bar\csd,0}\hbox to0.8cm{\hfill}$}}%
\leavevmode\llap{\raise2.5cm\hbox{\tiny $\bar\csd^\h-\ell_{1,\bar\csd,\h}\hbox to0.8cm{\hfill}$}}%
%\kern-3cm\raise2.7cm\hbox{\rotatebox{90}{\rlap{\rotatebox{-90}{\tiny $A_2-\ell_{1,2}$}}}}%
\leavevmode\rlap{\hbox to 2.5cm{\hfill}\raise0.6cm\hbox{\tiny \rotatebox{19}{ave.\ cons.\ unemployed}}}}
\end{subfigure}%
\begin{subfigure}{.5\textwidth}
  \centering 
\leavevmode%
\toshow{\includegraphics[width=7.5cm]{fg10R}  }

\leavevmode\smash{\kern-4cm\raise1.15cm\hbox{\tiny \rotatebox{-11}{ave.\ cons.\ employed}}%
\leavevmode\llap{\raise2.5cm\hbox{\tiny $\ell_{1,\bar\csd,0}-\ell_{2,\bar\csd,0}\hbox to0.4cm{\hfill}$}}%
\leavevmode\llap{\raise2.8cm\hbox{\tiny $\ell_{1,\bar\csd,\h}-\ell_{2,\bar\csd,\h}\hbox to0.4cm{\hfill}$}}%
\leavevmode\rlap{\hbox to 2.15cm{\hfill}\raise0.85cm\hbox{\tiny \rotatebox{5}{ave.\ cons.\ unemployed}}}}
%\leavevmode\smash{\hbox to0.5cm{\hfill} \raise6pt\hbox{\tiny consumption record}}
  %\caption{A subfigure} 
%  \label{fig:sub2}
\end{subfigure}
\caption{Distance from the employment class average consumption
level to the respective no-investment threshold in productivity state $1$ (left)
and in productivity state $2$ (right), shown relative to state $1$.}
\label{fgKS4}
\end{figure} }%
%%
%%
%%
To~calculate the values $W_{\infty,x,\bar\csd,u}(\bar\csd^u)$ and
$W_{\infty,x,\bar\csd,u}(\ell_{x,\bar\csd,u})$, one needs the values of
the installed capital $K_\infty(x,\bar\csd)$, $\bar\csd\in\bbG$, which are shown
in~Figure~\ref{fgKS3}.
%%
%%
%%%%%%%%%%%%%%%%%%%%%%%%%%%%%%%%%%%%%%%%%%
%%    Fg: 11 
%%%%%%%%%%%%%%%%%%%%%%%%%%%%%%%%%%%%%%%%%% 
{%\captionsetup{belowskip=-10pt}
%\captionsetup{aboveskip=0pt}
\begin{figure}[!htbp]
\centering
\begin{subfigure}{.5\textwidth}
  \centering
\leavevmode%   
\toshow{\includegraphics[width=7.5cm]{fg11L}}  

\leavevmode\smash{\kern-4.5cm\raise1.3cm\hbox{\tiny \rotatebox{-19}{ave.\ cons.\ employed}}%
\leavevmode\llap{\raise4.2cm\hbox{\tiny $K_\infty(1,\bar\csd)$\hbox to0.9cm{\hfill}}}%
\leavevmode\rlap{\hbox to2.5cm{\hfill}\raise0.6cm\hbox{\tiny \rotatebox{18}{ave.\ cons.\ unemployed}}}}
  %\caption{A subfigure}
%  \label{fig:sub1}
\end{subfigure}%
\begin{subfigure}{.5\textwidth}
  \centering
\leavevmode%
\toshow{\includegraphics[width=7.5cm]{fg11R} }

\leavevmode\smash{\kern-3.8cm\raise1.25cm\hbox{\tiny \rotatebox{-21}{ave.\ cons.\ employed}}%
\leavevmode\rlap{\raise3.9cm\hbox{\tiny \hbox to1.25cm{\hfill} $ K_\infty(2,\bar\csd)-K_\infty(1,\bar\csd)$}}%
\leavevmode\rlap{\hbox to 2.2cm{\hfill}\raise0.65cm\hbox{\tiny \rotatebox{19}{ave.\ cons.\ unemployed}}}}
%\leavevmode\smash{\hbox to0.5cm{\hfill} \raise6pt\hbox{\tiny consumption record}}
  %\caption{A subfigure}
%  \label{fig:sub2}
\end{subfigure}
\caption{Installed capital as a function of the population averages in productivity state $1$ (left)
and in productivity state $2$ relative to state $1$ (right).}
\label{fgKS3}
\end{figure} }%
%%
%%
One needs also the values
$g_{\infty,x,\bar\csd,u}^{ y ,v}$, $\bar\csd\in\bbG$, which are close to $0$ and vary between $-0\mathord.01721$ and
$0\mathord.01291$. This explains why the values $\abs{W_{\infty,x,\bar\csd,u}(\bar\csd^u)-1}$,
$\bar\csd\in\bbG$, $x\in\XXX$, $u\in\EEE$ are
uniformly very close to $0$, the largest one being of order $10^{-15}$. This means that if all
households consume exactly at the average for their respective employment category, then all KKT
would be essentially machine-precision-exact. All values
$W_{\infty,x,\bar\csd,u}(\ell_{x,\bar\csd,u})$, $\bar\csd\in\bbG$,
are also quite close to $1$, with largest difference of order~$10^{-5}$.  %$1\mathord.85576\times10^{-5}$.
This shows that, at
least in the context of the benchmark economy of \nocite{KS98}{[KS]} (with no borrowing allowed),
the linearization of the portfolio and future consumption mappings~-- and consequently, the
collapse of the state space to the vector of group averages~-- is quite accurate.

The main tool in the algorithm is the time invariant transport of the
population distribution. With the collapse of the state space to the averages in high and low
employment states, this transport comes down to
a transformation from $\R^2$ into $\R^2$. Since the transport depends not just on the present
common state but also on the future one,  there are $4$ such transformations shown in Figures~\ref{fgKS1}
and~\ref{fgKS2}.%%%%%%%%%%%%%%%%%%%%%%%%%%%%%%%%%%  
%%%%%%%%%%%%%%%%%%%%%%%%%%%%%%%%%%%%%%%%%%%%%%%%%%%
\footnote{In \nocite{KS98}{[KS]} the state space collapses to the total population average
over wealth, in which case the transport comes down to a mapping from $\R$ into $\R$. 
The log-linear regression lines constructed in \nocite{KS98}{ibid.}\ provide such mappings, but
in \nocite{KS98}{ibid.}\ the transport is assumed to depend only on the future productivity state and not on the
present one. Consequently, there are $2$ regression lines in the main example
in \nocite{KS98}{[KS]} instead of $4$. This arrangement appears to diverge from the general model
in \nocite{KS98}{ibid.}, in which the transition of the population average depends on both the
present and the future productivity states.}
%%
%%
%%
%%%%%%%%%%%%%%%%%%%%%%%%%%%%%%%%%%%%%%%%%%%%%%%%%%%
%\toshowtrue
%%%%%%%%%%%%%%%%%%%%%%%%%%%%%%%%%%%%%%%%%%
%%   Fg: 12
%%%%%%%%%%%%%%%%%%%%%%%%%%%%%%%%%%%%%%%%%%
{%\captionsetup{belowskip=-10pt} 
%\captionsetup{aboveskip=0pt}
\begin{figure}[!htbp] 
\centering
\begin{subfigure}{.5\textwidth}
  \centering
\leavevmode% 
\toshow{\includegraphics[width=7.5cm]{fg12L}}  

\leavevmode\smash{\kern-4cm\raise1.2cm\hbox{\tiny \rotatebox{-20.0}{ave.\ cons.\ employed}}%
%\leavevmode\rlap{\raise2.0cm\hbox{\tiny \hbox to3.95cm{\hfill}$\T_1^{1,2}$}}%
\leavevmode\llap{\raise2.5cm\hbox{\tiny $\T_{\infty,1}^1(\bar\csd)^{0}\hbox to1.0cm{\hfill}$}}%
\leavevmode\rlap{\raise4.1cm\hbox{\tiny\hbox to2.3cm{\hfill} $\T_{\infty,1}^1(\bar\csd)^{\h}$}}%
\leavevmode\rlap{\hbox to 2.5cm{\hfill}\raise0.7cm\hbox{\tiny \rotatebox{18}{ave.\ cons.\ unemployed}}}}
  %\caption{A subfigure}
%  \label{fig:sub1}
\end{subfigure}%
\begin{subfigure}{.5\textwidth}
  \centering
\leavevmode% 
\toshow{\includegraphics[width=7.5cm]{fg12R} } 

\leavevmode\smash{\kern-4cm\raise1.3cm\hbox{\tiny \rotatebox{-22}{ave.\ cons.\ employed}}%
\leavevmode\rlap{\raise2.1cm\hbox{\tiny
$\hbox to2.6cm{\hfill}\T_{\infty,1}^2(\bar\csd)^{0}-\T_{\infty,1}^1(\bar\csd)^{0}$}}%
\leavevmode\llap{\raise4.1cm\hbox{\tiny
$\T_{\infty,1}^2(\bar\csd)^{\h}-\T_{\infty,1}^1(\bar\csd)^{\h}\hbox to0.25cm{\hfill}$}}%
\leavevmode\rlap{\hbox to 2.5cm{\hfill}\raise0.7cm\hbox{\tiny \rotatebox{21}{ave.\ cons.\ unemployed}}}}
%\leavevmode\smash{\hbox to0.5cm{\hfill} \raise6pt\hbox{\tiny consumption record}}
  %\caption{A subfigure}
%  \label{fig:sub2}
\end{subfigure}
\caption{The transition of population averages in productivity state $1$.}
\label{fgKS1}
\end{figure} }% 
%%
%%
%%
%%%%%%%%%%%%%%%%%%%%%%%%%%%%%%%%%%%%%%%%%%
%%    Fg: 13
%%%%%%%%%%%%%%%%%%%%%%%%%%%%%%%%%%%%%%%%%%
{%\captionsetup{belowskip=-10pt}
%\captionsetup{aboveskip=0pt}
\begin{figure}[!htbp]
\centering
\begin{subfigure}{.5\textwidth}
  \centering
\leavevmode%  
\toshow{\includegraphics[width=7.5cm]{fg13L}} 

\leavevmode\smash{\kern-3.7cm\raise1.25cm\hbox{\tiny \rotatebox{-22}{ave.\ cons.\ employed}}%
\leavevmode\rlap{\raise2.2cm\hbox{\tiny $\hbox to1.8cm{\hfill}%
\T_{\infty,2}^1(\bar\csd)^{0}-\T_{\infty,1}^1(\bar\csd)^{0}$}}%
\leavevmode\llap{\raise4.3cm\hbox{\tiny%
$\T_{\infty,2}^1(\bar\csd)^{\h}-\T_{\infty,1}^1(\bar\csd)^{\h}\hbox to0.15cm{\hfill}$}}%
\leavevmode\rlap{\hbox to 2.5cm{\hfill}\raise0.7cm\hbox{\tiny \rotatebox{21}{ave.\ cons.\ unemployed}}}}
  %\caption{A subfigure}
%  \label{fig:sub1}
\end{subfigure}%
\begin{subfigure}{.5\textwidth}
  \centering
\leavevmode%
\toshow{\includegraphics[width=7.5cm]{fg13R} } 

\leavevmode\smash{\kern-3.7cm\raise1.25cm\hbox{\tiny \rotatebox{-22}{ave.\ cons.\ employed}}%
\leavevmode\rlap{\raise2.2cm\hbox{\tiny $\hbox to1.8cm{\hfill}%
\T_{\infty,2}^2(\bar\csd)^{0}-\T_{\infty,1}^2(\bar\csd)^{0}$}}%
%\leavevmode\rlap{\raise2.4cm\hbox{\tiny $\hbox to3cm{\hfill}\T_2^{2,2}-\T_1^{2,2}$}}%
\leavevmode\llap{\raise4.3cm\hbox{\tiny%
$\T_{\infty,2}^2(\bar\csd)^{\h}-\T_{\infty,1}^2(\bar\csd)^{\h}\hbox to0.15cm{\hfill}$}}%
%\leavevmode\llap{\raise4.3cm\hbox{\tiny $\T_2^{2,1}-\T_1^{2,1}\hbox to0.5cm{\hfill}$}}%
\leavevmode\rlap{\hbox to 2.3cm{\hfill}\raise0.65cm\hbox{\tiny \rotatebox{21}{ave.\ cons.\ unemployed}}}}
%\leavevmode\smash{\hbox to0.5cm{\hfill} \raise6pt\hbox{\tiny consumption record}}
  %\caption{A subfigure}
%  \label{fig:sub2}
\end{subfigure}
\caption{The transition of population averages in productivity state $2$.}
\label{fgKS2}
\end{figure} }% 
%%
%%
%%
%%
With these transformations at hand, one can easily simulate the long run behavior of the averages in
the two employment categories (the state of the population in the reduced model)
by merely simulating a time series of the productivity state of the economy and applying one of the
four transformations in Figures~\ref{fgKS1}
and~\ref{fgKS2} accordingly.
Starting with 
$\bar\csd^\h=0.92$ and $\bar\csd^0=0.89$ in high productivity state,
the state of the population was simulated for $1.1$ million periods and the values from the
first $10^5$ periods were removed from the sample. The state of the population in the last
$10^4$ periods is shown in~Figure~\ref{fgKS5}.
%%
%%
%%%%%%%%%%%%%%%%%%%%%%%%%%%%%%%%%%%%%%%%%%
%%   Fg: 14
%%%%%%%%%%%%%%%%%%%%%%%%%%%%%%%%%%%%%%%%%% 
{%\captionsetup{belowskip=-10pt}
%\captionsetup{aboveskip=0pt}
\begin{figure}[!htbp]  
\centering
\begin{subfigure}{.5\textwidth}
  \centering
\leavevmode\raise0.8cm\hbox{\rotatebox{90}{\tiny average consumption unemployed}}%
\ %  
\toshow{\includegraphics[width=7.0cm]{fg14L}}

\leavevmode\smash{\raise6pt\hbox{\tiny average consumption employed}}
  %\caption{A subfigure}
%  \label{fig:sub1} 
\end{subfigure}%
\begin{subfigure}{.5\textwidth}
  \centering
\leavevmode\raise1cm\hbox{\rotatebox{90}{\tiny average investment unemployed}}%
\ %
\toshow{\includegraphics[width=7.0cm]{fg14R}}

\leavevmode\smash{\raise6pt\hbox{\tiny average investment employed}} 
  %\caption{A subfigure}  
%  \label{fig:sub2}
\end{subfigure}
\caption{The distribution averages of the households in the last 10,000 periods in a simulated
series of $1.1$ million periods.}
\label{fgKS5}
\end{figure} }%
%%
%%
%% 
%%
What we see in Figure~\ref{fgKS5} is a typical law of motion in random environment: with high
probability the productivity state remains unchanged from one period to the next, so that the law of
motion of the population state is deterministic and governed either by
the mapping $\T_{\infty,1}^1\phd$ or by $\T_{\infty,2}^2\phd$~--
until a change in the productivity state occurs, in which case the population state is transformed either by
$\T_{\infty,1}^2\phd$ or by $\T_{\infty,2}^1\phd$.
The nearly linear patterns in Figure~\ref{fgKS5} are easily explained by the fact
that the mappings from Figures~\ref{fgKS1} and~\ref{fgKS2} are very close to affine functions. As a result, the
disparity between employed and unemployed, whether in terms of consumption or wealth,
has nearly affine structure that remains
unchanged for as long as the productivity state remains the
same (whence the straight lines), but that structure changes when the productivity state flips.
While the differences in the
transformations from Figures~\ref{fgKS1} and~\ref{fgKS2} are very small,
as can be seen in Figure~\ref{fgKS5},
these differences accumulate
over time and lead to substantial fluctuations in the state of the population, hence also in the
disparity between employed and unemployed. 
It is interesting to
note that these  fluctuations are much more dispersed than the fluctuations
in the random environment (the productivity state) that is causing them, which has
only two values (high and low)~-- a ``ratchet effect'' of a sort.
The larger dispersion in the data presented in the right plot in Figure~\ref{fgKS5},
in which the state of the population is described in terms of
the average wealth of employed and unemployed, is quite intuitive: the households' savings function
as ``shock absorbers.''
%Loosely speaking, the risk is still ``somewhat insurable,'' but not
%completely.
A~more formal explanation can be attributed to the larger variation in the
intercepts of the portfolio lines when the productivity state flips (recall that after a large
number of periods these lines
have identical slopes given by ${\b\over 1-\b}$, so that the change in the productivity state
simply shifts these lines up or down). This feature is illustrated in
Figure~\ref{fgKS-p-int}.
%%
%%
%%
%%%%%%%%%%%%%%%%%%%%%%%%%%%%%%%%%%%%%%%%%%
%%    Fg: 15
%%%%%%%%%%%%%%%%%%%%%%%%%%%%%%%%%%%%%%%%%%  
%\captionsetup{belowskip=-10pt}
%\captionsetup{aboveskip=0pt}
\begin{figure}[!htbp]
\centering
\begin{subfigure}{.5\textwidth}
  \centering
%\toshow{\includegraphics[width=6.5cm]{KS_p_intercepts_diff_e}}
\toshow{\includegraphics[width=7.5cm]{fg15L}}

\leavevmode\smash{\kern-4cm\raise1.2cm\hbox{\tiny \rotatebox{-20}{ave.\ cons.\ employed}}%
\rlap{\hbox to 2.55cm{\hfill}\raise0.65cm\hbox{\tiny \rotatebox{20}{ave.\ cons.\ unemployed}}}}
  %\caption{A subfigure} 
%  \label{fig:sub1}  
\end{subfigure}%
\begin{subfigure}{.5\textwidth} 
  \centering
%\toshow{\includegraphics[width=6.5cm]{KS_p_intercepts_diff_u} }
\toshow{\includegraphics[width=7.5cm]{fg15R} } 

\leavevmode\smash{\kern-4cm\raise1.2cm\hbox{\tiny \rotatebox{-20}{ave.\ cons.\ employed}}%
\rlap{\hbox to 2.55cm{\hfill}\raise0.65cm\hbox{\tiny \rotatebox{20}{ave.\ cons.\ unemployed}}}}
%\leavevmode\smash{\hbox to0.5cm{\hfill} \raise6pt\hbox{\tiny consumption record}}
  %\caption{A subfigure}
%  \label{fig:sub2}
\end{subfigure}
\caption{The shift in the portfolio lines for employed (left) and unemployed (right) when the productivity state
flips from low to high.}
\label{fgKS-p-int}
\end{figure}
%%
%%
%%
%%
Perhaps the most
important takeaway from these plots is that that they reveal a structure that would not be possible to capture
if the state of the population is collapsed to a single scalar value, whether that value
is the aggregate consumption or the aggregate investment across the entire population. 


By transforming the data on the left plot in Figure~\ref{fgKS5}, in conjunction with the simulated
series of the productivity state, through the mappings from Figure~\ref{fgKS3} one can easily
produce a simulated series of aggregate installed capital. Statistically, this series should be 
close to the one reported in \nocite{KS98}{[KS]}, which comes from simulating the individual
behavior of a large number of households.
And indeed, the GLM package returns coefficients for log-linear fit of installed capital
in productivity state~$1$ against preceding capital and log-linear fit of installed capital
in productivity state~$2$ against preceding capital of,
respectively, $(0.0912957,0.9638204)$ and $(0.0787774,0.9667176)$, which are reasonably close to
those reported in \nocite{KS98}{[KS]}, namely $(0.095,\allowbreak 0.962)$ and $(0.085,0.965)$. Since GLM is
applied here to a much larger sample (of size $10^6$), the goodness of fit measure is better: the
regression errors are of order $10^{-6}$ and $r^2$ is of order $1-10^{-7}$.%%%%%%%%%%%%%%%%%%
%%%%%%%%%%%%%%%%%%%%%%%%%%%%%%%%%%%%%%%%
\footnote{The use of GLM in the context of heterogeneous models
deserves to be questioned, since it remains unclear why the prediction errors should be i.i.d.
The output from GLM is reported here merely for the purpose of benchmarking.}
%%%%%%%%%%%%%%%%%%%%%%%%%%%%%%%%%%%%%%%%

Although it is assumed throughout this paper that the discount (impatience) factor $\b$ is
constant, apart from simplicity of exposition,
nothing in the general computational strategy that we have deployed makes such an assumption
necessary. A model with stochastic $\b$ is introduced and discussed extensively
in \nocite{KS98}{[KS]}.
While the exploration of such models (and they are many~-- see \nocite{KS98}{[KS]})
falls outside the scope of this
paper, a feature that can be illustrated here with very little additional effort is to rerun the Julia program
employed in this section with a different value for the impatience parameter, namely with
$\b=0.96$, instead of $\b=0.99$, which was borrowed from the benchmark case in~\nocite{KS98}{[KS]}.
For the sake of brevity, we produce here only the output from the simulation~-- see Figure~\ref{fgKSlast}.
%%
%%
%%%%%%%%%%%%%%%%%%%%%%%%%%%%%%%%%%%%%%%%%%
%%    Fg: 16
%%%%%%%%%%%%%%%%%%%%%%%%%%%%%%%%%%%%%%%%%%
{%\captionsetup{belowskip=-5pt}
%\captionsetup{aboveskip=0pt}
\begin{figure}[!htbp]  
\centering
\begin{subfigure}{.5\textwidth}
  \centering
\leavevmode\raise0.85cm\hbox{\rotatebox{90}{\tiny average consumption unemployed}}%
\ %  
\toshow{\includegraphics[width=7.0cm]{fg16L}}

\leavevmode\smash{\raise6pt\hbox{\tiny average consumption employed}}
  %\caption{A subfigure}
%  \label{fig:sub1}  
\end{subfigure}%
\begin{subfigure}{.5\textwidth}
  \centering
\leavevmode\raise0.85cm\hbox{\rotatebox{90}{\tiny average investment unemployed}}%
\ %
\toshow{\includegraphics[width=7.0cm]{fg16R}}

\leavevmode\smash{\raise6pt\hbox{\tiny average investment employed}} 
  %\caption{A subfigure} 
%  \label{fig:sub2}
\end{subfigure}
\caption{The distribution averages of the households in the last 10,000 periods in a simulated
series of $1.1$ million periods with impatience rate $\b=0.96$ (changed from $\b=0.99$).}
\label{fgKSlast}
\end{figure}
}
%% 
%%
%% 
These plots are consistent with the intuition: when the households are more
impatient they invest less, which then leads to a lower output, and ultimately to lower consumption.
What is surprising,
however, is that decreasing the discount factor by only $3\%$ can lead to substantially lower
investment and consumption levels. It~is also interesting to note that increased impatience leads
to a greater dispersion in the relative disparity between unemployed and employed.
%%
%%
%%
%%
%%

%%% chend556677


%%- -%%--{Sec: #5}--%%
%{No: 5.}   %%llabel

\section{Conclusion}
\label{sec:CONC}\setcounter{paragraph}{0}

\noindent
In one way or another, the nature of all incomplete-market heterogeneous agent models is deeply
rooted in the connections across time
between the dual (shadow) variables arising in dynamic optimization and control. The classical dynamic
programming formulation of such models cannot adequately capture such connections. An alternative program~--
``shadow programming'' of a sort~-- was proposed in \nocite{DL12}{[DL]} and expanded in the present
paper. One consequence from this approach is the realization that 
the law of motion of the cross-sectional distribution
of the population obtains directly %, as a stand-alone law of motion,
from the very notion of equilibrium and need not be interpreted as the
probability distribution of any particular Markovian state.
This point of view can be implemented in practice by way of functional programming and leads to a numerically
verifiable equilibrium in a well known benchmark study, which, so far,
has not been adequately resolved by other means. It provides new insights about the approximate aggregation
phenomenon in models with shared~risk.

\iffalse
The two concrete implementations from the last two sections demonstrate
that solving for the equilibrium in incomplete-market heterogeneous agent models need not rely on
ergodic results for Markov processes. Specifically, when written in terms of Lagrange duality, the
notion of equilibrium completely determines the law of motion of the cross-sectional distribution
of the population. This point of view is especially useful in models with shared risk, where,
generally, the association with the running distribution of a particular Markovian state is not
available, but we have seen that it could be just as useful in models where an association of this form is
available. The paper illustrated the benefits from using consumption as a state variable and the
deep and extremely useful connection that exists between future and present consumption levels.
In~terms of computing technology, the paper demonstrated the power of what one may call ``functional
programming'' and the ability of modern computer languages to carry out such programs.
In~practical terms, the paper provides a numerically verifiable solution to a well known and widely
used benchmark model in which all risk is idiosyncratic. It also sheds new insights and
provides a more efficient and more robust solution to yet another benchmark model with
common noise. The choice of two well documented classical models as a test site for the methodology
developed in the paper was deliberate and allowed for a detailed benchmarking. The paper was
written with the hope that the proposed general algorithm can be deployed in a broader
class of macroeconomic models and thus contribute to better
understanding of heterogeneity and incompleteness.
\fi

%%% chend556677
%%% chend555666777

\section*{References}
%\small 
\bibexpandtrue
%\vskip-7pt plus 1pt minus 1pt
%\input albiblio-short.tex 
{%\singlespacing%\small
\input mac-5.tex
}%
%\vfill\eject

%%%%%%%%%%%%%%%%%%%%%%%%%%%%%%
\iffalse
%%%%%
\begin{thebibliography}{}

\bibitem[\protect\citeauthoryear{Billingsley, P.}{1999}]{r1}
\textsc{Billingsley, P.} (1999). \textit{Convergence of
Probability Measures}, 2nd ed. 
Wiley, New York.


\bibitem[\protect\citeauthoryear{Prokhorov, Yu.}{1956}]{r4}
\textsc{Prokhorov, Yu.} (1956).
Convergence of random processes and limit theorems in probability
theory. \textit{Theory  Probab.  Appl.}
\textbf{1} 157--214.

\end{thebibliography}

%%%%%%%%%%%%%%%%
\fi
%%%%%%%%%%%%%%%%%%%%%%%%%%%%%%%

\begin{appendices}

%\appendix 
 

%\renewcommand*\thesection{\Alph{section}}

%\renewcommand{\theparagraph}{\textbf{\thesection.\arabic{paragraph}}}  
\setcounter{section}{1}
%\stepcounter{section}

%%- -%%--{Sec: #5}--%%
%{No: 5.}   %%llabel

%\stepcounter{section}
\section*{Appendix \Alph{section}: Comments on Some General Concepts}
\makeatletter%
\edef\@currentlabel{\Alph{section}}%
\makeatother%
\label{sec:A1}\setcounter{paragraph}{0}


\nocite{Aum64}{R.~Aumann}
was among the first to identify the need to make the economic agents infinitesimally small and
introduced ``Markets with a continuum of traders''~-- the very title of \citel{Aum64}.
Making the agents' positions and interactions infinitesimally small is indeed a way to
conceptualize the notion that a single agent can make any decision whatsoever
without affecting aggregate demand and prices. However, by itself
this arrangement is not sufficient to determine what infinitely many infinitesimally small
trades and individual choices aggregate to. To wit: merely knowing that changing the value of a function
at a single point does not change its integral 
{(this parallel is borrowed from \citel{Aum64})}
is not sufficient to determine the value of the integral. 
The modern point of view, followed in the paper, is
that heterogeneous agent model with infinitely many agents are the limit, as
the number of agents increases to infinity, of models with finite number of agents.
As the advances in mean field games
demonstrate, interpreting such limits is notoriously difficult~-- see \citel{CDLL19}.
In~the context of heterogeneous models the matter is even more delicate: in the limit the agents
are negligible but all prices and demands are limits of prices and demands obtained
in models where the agents are not
negligible. It is therefore necessary to clarify in what sense the agents are price takers when they are
not negligible: in the private optimization problems all prices are extraneous
parameters (irrespective of whether they can affect the market
or not, the agents take the prices as given).
%%%%%%%%%%%%%%%%%%%

Strictly speaking, every private optimal allocation problem is nothing but a
family of optimization problems parameterized by a finite collection of shared parameters
that may be thought of as ``placeholders.''
When the agents search for an equilibrium they search for common values to fill
those placeholders with, and in such a way that every agent can find an optimal allocation
within their budget set (without any control over the ``placeholders'') and the markets would clear.

%%%%%%%%%%%%%%%%%%%%%%%%%%

One must also note
the well known measurability issues that arise when independent idiosyncratic shocks are labeled
with a continuum~--
see \citel{Mal72}, \citel{FG85}, \citel{Judd85}, \citel{Sun06}, \citel{SZh09}, and \citel{DS12}.
Instead of using a designated labeling set for the collection of agents,
one may adopt the view, as is done in the paper,
that the cross-sectional distribution of the population is a ``sufficient statistic,''
which then makes the use of labels to identify the agents superfluous~--
see \citel{HHK74}, \citel{Hil74}.


A~major limitation in the use of a continuum
as a labeling set is that, setting all measurability issues aside,
the idiosyncratic shocks attached to the agents can only be essentially pairwise
independent, which then imposes coalitional aggregate certainty, as discovered
by \nocite{SZh09}{(Sun and 
Zhang 2009)}. In sum, truly idiosyncratic shocks are not possible under such an
arrangement, however innocuous it might seem.
 


%%% chend556677  






\stepcounter{section}
\makeatletter%
\edef\@currentlabel{\Alph{section}}%
\makeatother%
\section*{Appendix \Alph{section}: Proof of Theorem~\ref{thm1}\label{sec:A2}}
\setcounter{paragraph}{0}%

\noindent
This result is a direct application of the implicit function theorem.
The left sides of all three equations in \eqref{z2-no-lm}
can be treated as a $\R^3$-valued $\C^1$-function, which we write as 
$h(c,\q,\qq,w)$, with the understanding that $W_{y,v}$ substitutes for the right side of the first
equation in \eqref{ze2}. To simplify the notation further, set 
{\abovedisplayskip=5pt plus 1.5pt minus 1.5pt\belowdisplayskip=5pt plus 1.5pt minus 1.5pt\belowdisplayshortskip=5pt plus 1.5pt minus 1.5pt
\begin{gather*}
 a_{y,v}\df (1+r)\sqrt{-\b \, \partial^2 V_{t+1,\tts y,\tts \T_{t,\tts x}^y(F),\tts v}\bigl(W_{y,\tts v}/N\bigr)}\\
\noalign{and}
b_{y,v}\df \bigl(\rho_ y(K)+ 1-\dd\bigr)\sqrt{-\b \, \partial^2 V_{t+1,\tts y,\tts \T_{t,\tts x}^y(F),\tts v}\bigl(W_{y,\tts v}/N\bigr)}\,,
\end{gather*}
}%
which leads to the following expression%
%%%%%%%%%%%%%%%%%%%%%%%
\footnote{By convention, if $a<0$ we write $\sqrt{-a}\sqrt{-a}=-a$.}
%%%%%%%%%%%%%%%%%%%%%%
for the Jacobian matrix of the function $h$
%%
%%
{\abovedisplayskip=5pt plus 1.5pt minus 1.5pt\belowdisplayskip=5pt plus 1.5pt minus 1.5pt\belowdisplayshortskip=5pt plus 1.5pt minus 1.5pt
\begin{equation}\label{Jh}
h'(c,\q,\qq,w)={1\over N}\begin{bmatrix}
1 & 1 & 1 & \,\,\llap{$-1$} \\
\partial^2U(c/N) & \sum\nolimits_{y,v} a_{y,v}^2&  \sum\nolimits_{y,v} a_{y,v} b_{y,v} & 0\\
\partial^2U(c/N) & \sum\nolimits_{y,v} a_{y,v} b_{y,v} & \sum\nolimits_{y,v} b_{y,v}^2 & 0 
\end{bmatrix}\,.
\end{equation}
}%
Let $h'(c,\q,\qq,w)_1$ denote the $3$-by-$3$ matrix composed of the first three columns in the Jacobian
and let $h'(c,\q,\qq,w)_2$ denote the $3$-by-$1$ matrix composed of the last column.
Hence
%%
%%
{\abovedisplayskip=5pt plus 1.5pt minus 1.5pt\belowdisplayskip=5pt plus 1.5pt minus 1.5pt\belowdisplayshortskip=5pt plus 1.5pt minus 1.5pt
$$
\begin{aligned}
&N^3 \,\bigl|h'(c,\q,\qq,w)_1\bigr| =
\Bigl(\sum\nolimits_{y,v} a_{y,v}^2\Bigr)\Bigl(\sum\nolimits_{y,v}
b_{y,v}^2\Bigr) -\Bigl(\sum\nolimits_{y,v} a_{y,v} b_{y,v}\Bigr)^2\\
&\hbox to1.8cm{\hfill}-\partial^2U(c/N)\Bigl(\sum\nolimits_{y,v} b_{y,v}^2-\sum\nolimits_{y,v} a_{y,v}b_{y,v}\Bigr)
+\partial^2U(c/N)\Bigl(\sum\nolimits_{y,v} a_{y,v}b_{y,v}-\sum\nolimits_{y,v} a_{y,v}^2\Bigr)\\
&\hbox to2.3cm{\hfill} =\Bigl(\sum\nolimits_{y,v} a_{y,v}^2\Bigr)\Bigl(\sum\nolimits_{y,v}
b_{y,v}^2\Bigr) -\Bigl(\sum\nolimits_{y,v} a_{y,v} b_{y,v}\Bigr)^2
-\partial^2U(c/N)\sum\nolimits_{y,v} (a_{y,v}-b_{y,v})^2\,.
\end{aligned}
$$
}%
Since we exclude from the model the degenerate case in which the payoffs from capital investment
are identical in all productivity states, the
determinant above is strictly positive. By the implicit function theorem the equation
$h(c,\q,\qq,w)=(0,0,0)^\trn$ defines 
$(c,\q,\qq)\in\R^3$ as a unique $\C^1$-function in some neighborhood of $w$ with derivative
%%
%%
{\abovedisplayskip=5pt plus 1.5pt minus 1.5pt\belowdisplayskip=5pt plus 1.5pt minus 1.5pt\belowdisplayshortskip=5pt plus 1.5pt minus 1.5pt
$$
\bigl(\partial c,\partial \q,\partial \qq\bigr)
=-h'\bigl(c,\q,\qq,w\bigr)_1^{-1} \, h'\bigl(c,\q,\qq,w\bigr)_2\,,
$$
}%
and since the first entry in the first row of the inverse $h'(c,\q,\qq,w)_1^{-1}$ can be identified
as the strictly positive scalar
%%
%%
{\abovedisplayskip=5pt plus 1.5pt minus 1.5pt\belowdisplayskip=5pt plus 1.5pt minus 1.5pt\belowdisplayshortskip=5pt plus 1.5pt minus 1.5pt
$$
\Bigl(\sum\nolimits_{y,v} a_{y,v}^2\Bigr)\Bigl(\sum\nolimits_{y,v} b_{y,v}^2\Bigr)
-\Bigl(\sum\nolimits_{y,v} a_{y,v} b_{y,v}\Bigr)^2\,,
$$
}%
we see that $\partial c>0$, i.e., the consumption level is a strictly increasing
$\C^1$-function of entering wealth.
Furthermore, the value function of the problem in
(\ref{ze1}-\ref{ze2}) can be cast as
{\abovedisplayskip=5pt plus 1pt minus 1pt\belowdisplayskip=5pt plus 1pt minus 1pt\belowdisplayshortskip=3pt plus 0.5pt minus 0.5pt
\begin{equation*}%\label{A-ze1}
\begin{aligned}
&V_{t,x,F,u}(w/N)\df U(c(w)/N)\\
&\hbox to0.5cm{\hfill}+\b\sum\nolimits_{ y\ts\in\ts\XXX,\,v\ts\in\ts\EEE}
V_{t+1,\tts y,\tts \T_{t,\tts x}^y(F),\tts v}\Bigl((1+r)\,\q(w)/N + (\rho_ y(K)+ 1-\dd)\,\qq(w)/N + 
\ee_ y(K)\, v/N\Bigr)\\
&\hbox to10cm{\hfill}\times Q(x, y)P_{x, y}(u, v)\,,
\end{aligned}
\end{equation*}}%
and this function is $\C^1$
with respect to the resource $w/N$ as well. By~the envelope theorem (see \eqref{ze4a1})
{\abovedisplayskip=5pt plus 1pt minus 1pt\belowdisplayskip=5pt plus 1pt minus 1pt\belowdisplayshortskip=3pt plus 0.5pt minus 0.5pt
$$
\partial V_{t,x,F,u}(w/N) =\partial U(c(w)/N)\,,
$$
}%
with the implication that $\partial V_{t,x,F,u}\phd$
is $\C^1$ and strictly decreasing, since $\partial U\phd$ is strictly decreasing and $c\phd$ is
strictly increasing; in particular,
$V_{t,x,F,u}\phd\in\C^2(\R)$ and $\partial^2 V_{t,x,F,u}\phd<0$.

Removing the production technology from the model leads to the removal of the third row and the
third column in the Jacobian matrix in \eqref{Jh}. Similarly, removing the private lending
instrument from the model leads to the removal of the second row and the second column from the
Jacobian. In either case, the application of the implicit function theorem as above is straightforward. 

The second part of the theorem can be established in much the same way. Let $k(c,\q,\qq)\in\R^2$
be the vector composed of the left sides in the last two equations in \eqref{z2-no-lm}. Then
$k\in\C^2(\R^3;\R^2)$ with Jacobian matrix
%%
%%
{\abovedisplayskip=5pt plus 1.5pt minus 1.5pt\belowdisplayskip=5pt plus 1.5pt minus 1.5pt\belowdisplayshortskip=5pt plus 1.5pt minus 1.5pt
$$
k'(c,\q,\qq)={1\over N}\begin{bmatrix}
\partial^2U(c/N) & \sum\nolimits_{y,v} a_{y,v}^2&  \sum\nolimits_{y,v} a_{y,v} b_{y,v} \\
\partial^2U(c/N) & \sum\nolimits_{y,v} a_{y,v} b_{y,v} & \sum\nolimits_{y,v} b_{y,v}^2 
\end{bmatrix}\,.
$$
}%
As the matrix composed of the last two columns in this Jacobian was already shown to have a strictly
positive determinant, the implicit function theorem completes the proof.

%%% chend556677




%%- -%%--{Sec: #6}--%%
%{Eq: 6.}   %%llabel
%{No: 6.}   %%llabel

\stepcounter{section}
\section*{Appendix \Alph{section}: Lower Bounds on Consumption}
\makeatletter%
\edef\@currentlabel{\Alph{section}}%
\makeatother%
\label{sec:App-B}\setcounter{paragraph}{0}

%\renewcommand{\theequation}{B.\arabic{equation}} 
%%%%%%%%%%%%%%%%%%%%%%%%%%%%%%%%%%%%%%%
While this result is not used in the paper, it is important to note that,
in general, there is no reason why in equilibrium the range of
consumption must expand arbitrarily close to $0$,
i.e., in equilibrium, 
the support of the cross-sectional distribution of agents may exclude a neighborhood of~$0$. %
To~see this, consider the special case where productive capital is the only asset
(i.e., $\q_{t,x,\bar\csd,u}=0$) and $\partial U(c)=1/c$~-- see Sec.~\ref{sec:KS}.
%In addition, suppose that all portfolios and future 
%consumptions, treated as continuous functions of present consumption, are defined everywhere in $\Rpp$ and their
%domains are extended to include $0$ in the obvious way (by continuity).
The second kernel condition in \ref{cross-sys}-(\ref{zze5a}) can now be cast as
{\abovedisplayskip=5pt plus 1.5pt minus 1.5pt\belowdisplayskip=5pt plus 1.5pt minus 1.5pt\belowdisplayshortskip=5pt plus 1.5pt minus 1.5pt
\begin{equation}\label{new-kern}   
c=\Bigl(\b\sum\nolimits_{ y\ts\in\ts\XXX,\,v\ts\in\ts\EEE}{1\over\pfc_{t,x,\csd}^{ y,v}(u,c)}
\bigl(\rho_{ y}(K_{t}(x,\csd))+1-\dd\bigr) Q(x, y)P_{x, y}(u,v)\Bigr)^{-1}\,.
\end{equation}
}%
Introducing the strictly increasing functions
{\abovedisplayskip=5pt plus 1.5pt minus 1.5pt\belowdisplayskip=5pt plus 1.5pt minus 1.5pt\belowdisplayshortskip=5pt plus 1.5pt minus 1.5pt
\[
\Rpp\ni \a \leadsto H_{t+1, y,\T_{t,x}^y(\csd),v}(\a)\df \a + \qq_{t+1, y,\T_{t,x}^y(\csd),v}(\a) \,,
\]}%
with inverses $\hat H_{t+1, y,v}\phd$ as in \eqref{inversion}, the balance equation \ref{cross-sys}-(\ref{zze5xb})
can be stated as
{\abovedisplayskip=7pt plus 1.5pt minus 1.5pt\belowdisplayskip=7pt plus 1.5pt minus 1.5pt\belowdisplayshortskip=5pt plus 1.5pt minus 1.5pt 
\[
\pfc_{t,x,\csd}^{ y,v}(u,c) =
{{\hat H}_{t+1, y,\T_{t,x}^y(\csd),v}\Bigl(\bigl(\rho_ y(K_{t}(x,\csd))+1-\dd\bigr){\qq_{t,x,\csd,u}(c)}
+v \,\ee_ y(K_{t}(x,\csd))\Bigr)}\,.
\]}%
Suppose next that the domains of all functions $\qq_{t+1, y,\T_{t,x}^y(\csd),v}\phd$, $ y\in\XXX$, $v \in\EEE$,
include $\Rpp$ and let $\qq_{t+1, y,\T_{t,x}^y(\csd),v}(0)\df\lim_{c\searrow 0}\qq_{t+1, y,\T_{t,x}^y(\csd),v}(c)\,$.
The infimum over all admissible entering wealths in period $t+1$ is
$\qq_{t+1,y,\T_{t,x}^y(\csd),v}(0)$, and suppose that this infimum can be reached, in the sense
that there is a $c^*\ge 0$ (depending on $y$ and $v$) such that 
%%
{\abovedisplayskip=5pt plus 1.5pt minus 1.5pt\belowdisplayskip=5pt plus 1.5pt minus 1.5pt\belowdisplayshortskip=5pt plus 1.5pt minus 1.5pt
\[
\begin{split}
\lim\nolimits_{c\searrow c^*}
\Bigl(\bigl(\rho_ y(K_{t}(x,\csd))+1-\dd\bigr){\qq_{t,x,\csd,u}(c)}
&+v \,\ee_ y(K_{t}(x,\csd))\Bigr)\\
&= \qq_{t+1,\tts y,\tts \T_{t,x}^y(\csd),\tts v}(0) = {H}_{t+1,\tts  y,\tts \T_{t,x}^y(\csd),\tts v}(0)\,,
\end{split}
\]
}%
i.e., assuming that ${\qq_{t,x,\csd,u}\phd}$ is continuous, chosen so that
%%
{\abovedisplayskip=5pt plus 1.5pt minus 1.5pt\belowdisplayskip=5pt plus 1.5pt minus 1.5pt\belowdisplayshortskip=5pt plus 1.5pt minus 1.5pt
\[
\qq_{t,x,\csd,u}(c^*)\df\lim\nolimits_{c\searrow c^*}\qq_{t,x,\csd,u}(c)
= {\qq_{t+1,\tts y,\tts \T_{t,x}^y(\csd),\tts v}(0) - v \,\ee_ y(K_{t}(x,\csd))\over \rho_ y(K_{t}(x,\csd))+1-\dd}\,.
\]
}%
Then $\lim\nolimits_{c\searrow c^*} \pfc_{t,x,\csd}^{ y,v}(u,c)
= {\hat H}_{t+1,\tts y,\tts \T_{t,x}^y(\csd),\tts v}
\bigl({H}_{t+1,\tts y,\tts \T_{t,x}^y(\csd),\tts v}(0)\bigr)=0$, which is possible only if $c^*=\nobreak 0$,
since otherwise the right side of \eqref{new-kern} converges to $0$ as $c\searrow c^*$,
while the left side  converges to $c^*>0$.
Consequently, if one is to insist
that $\qq_{t,x,\csd,u}\phd$ is continuous and non-decreasing and that its domain is contiguous and
extends to $+\infty$ with $\lim_{c\nearrow\infty}\qq_{t,x,\csd,u}(c)=\infty$,
then everywhere in that domain $\qq_{t,x,\csd,u}\phd$ must have a lower bound
given by
{\abovedisplayskip=5pt plus 1.5pt minus 1.5pt\belowdisplayskip=5pt plus 1.5pt minus 1.5pt\belowdisplayshortskip=5pt plus 1.5pt minus 1.5pt
\begin{equation}\label{theta-min}
M_{t,x,\csd}\df \max_{ y\ts\in\ts\XXX,\,v\ts\in\ts\EEE}\
{\qq_{t+1,\tts y,\tts \T_{t,x}^y(\csd),\tts v}(0) - v \,\ee_
y(K_{t}(x,\csd))\over \rho_ y(K_{t}(x,\csd))+1-\dd}\,.
\end{equation}}%
In particular,
{\abovedisplayskip=5pt plus 1.5pt minus 1.5pt\belowdisplayskip=5pt plus 1.5pt minus 1.5pt\belowdisplayshortskip=5pt plus 1.5pt minus 1.5pt
\[
\pfc_{t,x,\csd}^{ y,v}(u,\cdot) \ge {{\hat H}_{t+1, y,v}\Bigl(\bigl(\rho_ y(K_{t,x})+1-\dd\bigr) M_{t,x}
+v \,\ee_ y(K_{t,x})\Bigr)}
\]
}%
everywhere in the domain of $\qq_{t,x,u}\phd$. Of course, this relation is interesting only if 
the right side is strictly positive~-- a situation that is illustrated next. 
%%%%%%%%%%%%%%%%
%\iffalse

Now suppose that, in addition to $\Up(c)=1/c$ and $\q_{t,x,\csd,u}\phd\equiv0$,
the model is also such that $\EEE=\{\eta,0\}$ for some fixed $\eta>0$ (there are only two
employment states, employed and unemployed~-- see Sec.~\ref{sec:KS}).
In~what follows the values of all functions at $0$
are to be understood as the right limits at $0$.
Since no investment takes place in the final period $t=T$, with
$t=T-1$ the lower bound in \eqref{theta-min} becomes $0$, i.e, $\qq_{T-1,x,\csd,u}(0)\ge 0$.
If $\qq_{t,x,\csd,u}(0)\ge 0$ for some $t<T$, then with $v =\eta$ and
$c=0$ the balance equation in \ref{cross-sys}-(\ref{zze5xb}) would give
{\abovedisplayskip=5pt plus 1.5pt minus 1.5pt\belowdisplayskip=5pt plus 1.5pt minus 1.5pt\belowdisplayshortskip=5pt plus 1.5pt minus 1.5pt
\begin{equation}\label{boc-c}
{\pfc_{t,x,\csd}^{ y,\eta}(u,0)}
+ {\qq_{t+1,\tts y,\tts \T_{t,x}^y(\csd),\tts \eta}\bigl(\pfc_{t,x,u}^{ y,\eta}(0)\bigr)}\ge \ee_ y(K_{t,x})\,\eta\,,
\end{equation}}
and with $v =0$ and $c=0$ the same balance equation would give
{\abovedisplayskip=5pt plus 1.5pt minus 1.5pt\belowdisplayskip=5pt plus 1.5pt minus 1.5pt\belowdisplayshortskip=5pt plus 1.5pt minus 1.5pt
\begin{equation}\label{boc-d} 
{\pfc_{t,x,\csd}^{ y,0}(u,0)}
+ {\qq_{t+1,\tts y,\tts\T_{t,x}^y(\csd),\tts 0}\bigl(\pfc_{t,x,\csd}^{ y,0}(u,0)\bigr)}\ge 0\,.
\end{equation}}
With $t=T-1$ the last two relations
become
{\abovedisplayskip=5pt plus 1.5pt minus 1.5pt\belowdisplayskip=5pt plus 1.5pt minus 1.5pt\belowdisplayshortskip=5pt plus 1.5pt minus 1.5pt
\begin{equation}\label{boc-e}
{\pfc_{T-1,x,\csd}^{ y,\eta}(u,0)}\ge \ee_ y(K_{T-1,x})\,\eta\qquad\text{and}\qquad 
{\pfc_{T-1,x,\csd}^{ y,0}(u,0)}\ge 0\,.
\end{equation}
}%
Since ${\pfc_{T-1,x,\csd}^{ y,\eta}(u,0)}$ is strictly positive, due to \eqref{new-kern}
${\pfc_{T-1,x,\csd}^{ y,0}(u,0)}> 0$ is not possible, for otherwise the right side would have a strictly
positive limit as $c\searrow 0$, while the limit of the left side would be $0$.
Since ${\pfc_{T-1,x,\csd}^{ y,0}(u,0)}=0$,
the balance equation \ref{cross-sys}-(\ref{zze5xb}) gives $\qq_{T-1,x,\csd,u}(0)= 0$ for all
$u\in\EEE$. In~particular,
both relations in \eqref{boc-e} are strict identities. Next, setting $t=T-2$ in \eqref{theta-min}
again yields $M_{T-2,x,\csd}=0$, so that $\qq_{T-2,x,\csd,u}(0)\ge 0$ for every $u\in\EEE$.
The balance equation now gives
{\abovedisplayskip=5pt plus 1.5pt minus 1.5pt\belowdisplayskip=5pt plus 1.5pt minus 1.5pt\belowdisplayshortskip=5pt plus 1.5pt minus 1.5pt
\begin{equation*}
\pfc_{T-2,x,\csd}^{ y,\h}(u,0)+\qq_{T-1,\tts y,\tts \T_{T-2,x}^y(\csd),\tts \h}
\bigl(\pfc_{T-2,x,\csd}^{ y,\h}(u,0)\bigr) \ge \ee_y(K_{t}(x,\csd))\h > 0\,,
\end{equation*}}
which is possible only if $\pfc_{T-2,x,\csd}^{ y,\h}(u,0)>0$.
Just as above, due to \eqref{new-kern} the last relation implies $\pfc_{T-2,x,\csd}^{y,0}(u,0)=0$,
so that with $v=0$ the balance equation
\ref{cross-sys}-(\ref{zze5xb}) gives
{\abovedisplayskip=5pt plus 1.5pt minus 1.5pt\belowdisplayskip=5pt plus 1.5pt minus 1.5pt\belowdisplayshortskip=5pt plus 1.5pt minus 1.5pt
\begin{equation*}
\bigl(\rho_y(K_{t}(x,\csd))+1-\dd\bigr)\qq_{T-2,x,\csd,u}(0)=0+\qq_{T-1,\tts
y,\tts \T_{T-2,x}^y(\csd),\tts v}(0)=0\,,
\end{equation*}
}%
with the implication that $\qq_{T-2,x,\csd,u}(0)=0$. By way of induction one can show that
$\qq_{t,x,\csd,u}(0)=0$, $\pfc_{t,x,\csd}^{y,\h}(u,0)>0$, and $\pfc_{t,x,\csd}^{ y,0}(u,0)=0$ for all
$t<T$, $x,y\in\XXX$ and $u\in\EEE=\{\h,0\}$. In particular, capital will never be borrowed and
the consumption level of all employed households will be bounded away from $0$ for all $t<T$. We
stress that these features hold if capital is the only asset and all households in one of
the employment states have no income.  
%\fi


%%% chend556677 


\end{appendices}   

\end{document} 

%%% chend222333444

gs -sDEVICE=pdfwrite -dCompatibilityLevel=1.4 -dPDFSETTINGS=/printer -dNOPAUSE -dQUIET -dBATCH -sOutputFile=DIWX.pdf DIW.pdf

<a href="https://zenodo.org/badge/latestdoi/582419358"><img src="https://zenodo.org/badge/582419358.svg" alt="DOI"></a>
