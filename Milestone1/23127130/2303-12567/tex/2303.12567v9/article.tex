\documentclass[12pt,twoside]{article}
\usepackage[xetex,a4paper,hmargin={1.0in,1.0in},top=1.0in,bottom=1.0in]{geometry}
\usepackage{setspace} 
\setstretch{1.0225}%\setstretch{1.05}

\usepackage[T1]{fontenc}
\usepackage[utf8]{inputenc}  

\usepackage{libertine} %!!! % MOD1
\usepackage[scale=1.0,libertine,vvarbb]{newtxmath} %!!! MOD1
%\usepackage{newtx} %!!! % MOD1

\widowpenalty=10000
\clubpenalty=10000  


\usepackage[cal=boondoxo, calscaled=1.0, scr=boondox, scrscaled=1.0]{mathalpha} %!!! MOD2
%\usepackage[cal=cm, calscaled=1.0, scr=boondox, scrscaled=1.0]{mathalpha} %!!! MOD2
%\usepackage{bm} %not really needed

\usepackage{graphicx} 
\usepackage{ntheorem} 
%\usepackage{color}
 
\usepackage{sansmath} % not with STIX 

\usepackage{marvosym}

\usepackage{stmaryrd} 


\usepackage{mathtools,amsmath}

\usepackage{natbib} 
\usepackage{hypernat}

\usepackage[labelfont={bf,footnotesize},textfont={footnotesize,singlespacing}]{caption}


\usepackage{footmisc}
\renewcommand{\footnotelayout}{\hspace{0.15em}\setstretch{1.0}}% footnotes even more stretched
                                % than onehalfspacing
 


\usepackage{subcaption}
\usepackage[colorlinks,citecolor=blue,linkcolor=blue,anchorcolor=blue,urlcolor=blue]{hyperref}

\usepackage[toc]{appendix} 

\theorembodyfont{\upshape}
\theorempreskip{0.0pt plus 1.0pt}
\theorempostskip{0.0pt plus 1.0pt}

\newcounter{nitnum} % not used 

\def\nitskip{\vskip 1.75pt plus 0.5pt minus 0.5pt}


\numberwithin{paragraph}{section}

\renewcommand{\theparagraph}{\textbf{\arabic{section}.\arabic{paragraph}}}  

\newenvironment{nit}[1]% environment name
{% begin code
\removelastskip\nitskip\par\noindent\refstepcounter{paragraph}%
\textbf{\theparagraph~{#1}:\enspace}\ignorespaces%
}%
{% end code
\ignorespacesafterend\nitskip
} 


\input mac-1.tex 
\input mac-2.tex
\ReadWriteBib{5}
\bibindent=24pt
\overfullrule=2.25pt

\gdef\bibshape#1#2{\vskip\bibskip\par\hangindent\bibindent\hangafter1\noindent\ignorespaces
\hbox to\bibindent{\hfill}\ignorespaces
\llap{\relax\hypertarget{#1}{\bibmark.}\hbox to \bibnoskip{\hfill}}\ignorespaces#2\ignorespaces} %MOD 3


\input mac-3.tex
 
\input ushyphex.tex
\input mac-4.tex

 

\newif\iftoshow
\def\toshow#1{\iftoshow #1\else\medskip\relax\fi}
%\toshowfalse 
\toshowtrue 



\def\Item#1#2{\par\nobreak\hangindent#1pt\hangafter0\noindent
\llap{#2\enspace}\ignorespaces}

\def\cint{{\lbkt 0,\bar c\,]}}
\def\sd{{\pi}}
\def\pdg{{{}^\dag}}
\def\pddg{{{}^\ddag}}
\def\pst{{{}^{\raise0.7pt\hbox{$\scriptstyle*$}}}}
\def\csd{{F}}%
\let\T\Thetait
\def\pfc{{\TT\ts}}
\def\csdg{{\goth g}}%
\def\csdh{{\goth h}}%
\def\csdp{\pdg\csd_{y}}%
\def\csds{\pst\csd_{y}}%
\def\gp{{\pi}}

\let\Lmp\Phiit%
\let\DB\bbF
\def\Up{{\partial U}}

\def\times{\tts{\hbox{\fontencoding{TS1}\fontfamily{Cochineal-LF}\selectfont\char214}}\tts}
\def\half{{\frac12}}

\def\qedsymb{{$\bulletS$}}   %!!!!!  
\def\qed{\ifmmode\nobreak\hbox{\quad\qedsymb}\else\nobreak\relax\nobreak\hbox{\quad\qedsymb}\fi}

\let\swtc\small

\def\SWTC{\singlespacing\small\relax}
\def\chptr{paper}
\let\esw\qed 


\def\aut#1{{\smc #1}.}
\def\btitle#1{{\it  #1\/}.}
\def\jtitle#1{{\rm #1.}} 
\def\anno#1{({\rm #1}).} 
\def\book#1{{\rm #1}}
\def\journal#1 #2 #3{{\it #1\/}~{\bf #2} #3}
\def\journalnt#1 #2 #3{{\it #1\/} {\bf #2} {#3}}
\bibskip=0pt plus 1pt minus 1pt

\numberwithin{equation}{section}
\theoremstyle{plain}  

\abovedisplayskip=5pt plus 1.5pt minus 1.5pt%
\belowdisplayskip=5pt plus 1.5pt minus 1.5pt%
\belowdisplayshortskip=3pt plus 1.5pt minus 1.5pt%

%\definecolor{darkblue}{rgb}{0.0,0.0,0.425}

\title{{}Self-Aware Transport of Economic Agents}%%% 

\author{%Andrew Lyasoff\ts\thanks{Boston University, email: alyasoff@bu.edu, Git repository:
{}Andrew Lyasoff\ts\thanks{{}email: mathema@lyasoff.net, Git repository:
\href{https://github.com/AndrewLyasoff}{https://github.com/AndrewLyasoff}}%
}


\date{{}\today} 



\usepackage{titlesec}
\titleformat{\section}[hang]{\center\bf\large}{\llap{\bf\large\thesection.\hbox to0.5em{\hfill}}}{0.0em}{}
\titleformat{\figure}[hang]{\normalsize\rmshape}{\textsc{\thefigure}}{0.5em}{}
\titlespacing*{\section}{0pt}{1.25ex plus 0.35ex minus .125ex}{1.0ex  plus 0.35ex minus 0.125ex}

\usepackage{titling}

\setlength{\droptitle}{-6.5em}

\renewcommand{\abstractname}{\vspace{-1.5cm}}
\footnotesep=0.75\baselineskip

\usepackage{fancyhdr}
\renewcommand{\headrulewidth}{0pt}%
\fancypagestyle{FirstPage}{
\fancyhf{}% clear all header and footer fields
\fancyfoot{}
\fancyfoot[C]{\thepage} % except the center
\renewcommand{\headrulewidth}{0pt}%
\renewcommand{\footrulewidth}{0pt}%
\fancyfoot[L]{} 
}
\pagestyle{fancy}
\headheight=5.2pt 
\headheight=14.49998pt
\pagestyle{fancy}  
\fancyhead{}
\fancyhead[CE]{\small{}ANDREW LYASOFF}
\fancyhead[CO]{\small{}SELF-AWARE TRANSPORT OF ECONOMIC AGENTS}  
\fancyhead[LE,RO]{{}\thepage}
\fancyfoot{}
  
\begin{document}

\maketitle
{\small
\vskip-5cm
\begin{flushright}
--- L'essentiel est invisible pour les yeaux, répéta le

petit prince, afin de se souvenir.%%%%%%%%%%%%%%%
%%%%%%%%%%%%%%%%%%%%%%%%%%%%%%%%
\footnote{{\smc de Saint-Exupéry, Antoine}.~(1943).~{Le Petit Prince, Ch.~XXI}. New York, NY: Reynal \& Hitchcock.}
%\smallskip
%
%de Saint-Exupéry, A.:~{Le Petit Prince, Ch.~XXI}. Reynal \& Hitchcock, New York (1943)
\end{flushright}
}
\bigskip


\allowdisplaybreaks

\thispagestyle{FirstPage}

\begin{abstract}\normalsize\noindent 
\noindent%
%
%
\noindent%
%\textsc{Abstract:}
\small{}%

\noindent
\textbf{Abstract:}\enspace The paper is concerned with the simultaneous solution of a very large number of optimization problems the structure of which is not given in the outset and is dynamically agreed upon while a large number of optimizers work in an orchestra. The proposed new approach to such problems was prompted by the surprising discovery that the common strategy, adopted in a large body of research, for producing time-invariant equilibrium in the classical Aiyagari-Bewley-Huggett model fails to achieve its objective in a widely cited benchmark study, with the implication that a central problem in macroeconomics has been without an adequate solution, and despite recent advances based on novel mathematical techniques borrowed from the theory of mean field games. It is shown that the intrinsic structure of a generic heterogeneous agent incomplete market model imposes connections across time that existing mathematical frameworks cannot capture. The new technique is shown to provide numerically verifiable equilibria in some widely researched, yet still unsolved, concrete instances of heterogeneous agent models. The scope of ``the approximate aggregation conjecture'' of Krusell and Smith~\cite{KS98} (still an open problem in macroeconomics) is clarified and a new computational strategy, which does not rely on simulation or the need to postulate infinite time horizon, for models with common shocks is developed. New insights about the fluctuations in the population distribution in such models are drawn and some novel closed-form expressions are obtained.   
\medskip


\iffalse
It is shown that, while instrumental in the domain
of economics, such models cannot be placed in any of the existing mathematical frameworks.
The proposed new framework was prompted by the surprising discovery that the common strategy,
adopted in a large body of macroeconomic research,
for producing time-invariant equilibrium in the classical Aiyagari-Bewley-Huggett model
fails to achieve its objective in a widely cited benchmark study, with the implication that a
central problem in macroeconomics has been without an adequate solution, save for special cases.
It is shown that the intrinsic structure of the so called ``heterogeneous models'' imposes
connections across time that existing mathematical techniques cannot capture. By expanding the
approach proposed by Dumas and Lyasoff~\citel{DL12} the \chptr\ develops a novel mathematical
framework which, among other things, leads to numerically verifiable equilibria in some widely
researched, yet still unsolved, concrete instances of heterogeneous models. It provides new
insights about the channels through which the cross-sectional distribution of a large population of
interacting agents gets transported through time  (see \ref{main-q} and \ref{n2} below) and
clarifies the scope of what is commonly referred to as ``the approximate aggregation conjecture''
(only the population mean matters) of Krusell and Smith~\citel{KS98}~-- still an open problem in
macroeconomics.
\fi
\medskip

\noindent% 
\textsc{Keywords:} Markov chains, transport problems, mean field games, 
general equilibrium, incomplete markets, heterogeneous agent mo\-dels, numerical methods
\end{abstract}

\iffalse
\medskip 

{\small {}
\noindent
\textsc{Key takeaways}: (a)~The structure of the transport of the
cross-sectional distribution of a large population of interacting agents is endogenous and
cannot be fixed in the outset (as the coefficients of the master
equation, or of the coupled backward-forward system, in a mean field game typically are).

(b)~Interpreting the
cross-sectional distribution of a large number of agents
as the probability distribution of a single representative agent and relying on one
(possibly stochastic) form or another of the continuity (Kolmogorov forward) equation turns out to be
inadequate even in very basic examples. Whence the need for a new approach to
explain the law of motion of the distribution of interacting agents
(see \ref{main-q} and \ref{n2} below).

(c)~The intrinsic structure of the equilibrium problems arising in macroeconomics is somewhat more
complex than the one captured by the theory of mean field games, even with amendments to the latter
that allow
state constraints to be imposed. 
In particular, the state constraints (such as the borrowing limits) arising in economics are
typically endogenous.
%(d)~The cross-sectional distribution of the population may not have a singularity at the location
%where the state constraints bind.}

(d)~Even if all individual choices happen to be affected only by the population mean
(the approximate aggregation hypothesis holds), the law of motion of
the full cross-sectional distribution is still uniquely determined, i.e., the population distribution
cannot be chosen arbitrarily, subject to the one and only requirement that
its mean follows a (possibly unique) prescribed law of motion. 
\medskip

\fi 
%%lblinit 
%{Sec: #}--%%label

%%- -%%--{Sec: #1}--%%
%{Eq: 1.}   %%llabel
%{No: 1.}   %%nlabel

\section{Background and Introduction}
\label{sec:Intro}\setcounter{paragraph}{0}

\parskip=0.0pt


\noindent
One crucial component in the study of heterogeneous agent incomplete market models is the
channel, imposed by the notion of economic equilibrium, through which the distribution of a large
population of agents spread over the range of wealth, or, equivalently, spread over the range of consumption,
gets transported from one period to the next. In one way or another, practical considerations
require this channel to be placed in a rigorous mathematical framework. The seminal works of
Bewley, Huggett, and Aiyagari 
(see \cite{Aiya94}, \cite{Bew77}, \cite{Hug93} and Sec.~18.2 in \cite{LjunSar00})
rely on the canonical law of motion of the probability distribution of the state of a single
controlled Markov chain.
In contrast, Krusell and Smith~\citel{KS98} substitute the population
mean for the population distribution itself and extract the law of motion
of the former by way of simulating the individual behavior of a reasonably large number of
agents, whose private choices are based on a log-linear regression forecast for the mean. 
The parameters in the AR(1) forecasting structure are then adjusted until an acceptable level of
goodness of fit to the simulated data is attained. 
This approach has the limitation of
being applicable only to models with a very large time horizon. It also raises the questions:
(a)~by how much does the substitution of the population mean for the population distribution itself
change the conclusions of the model?
(b)~if only the population mean matters, is any distribution with 
the correct mean value compatible with the notion of equilibrium?
(c)~how reliable is the goodness of fit as a criterion for achieving an economically viable
equilibrium?
Another, more recently developed, modeling framework for the dynamics of the distribution of
a large population of heterogeneous economic agents is the celebrated master equation in the theory
of mean field games (MFG)~-- see \cite{AHLLM17}.
It~was noted in the recent paper \cite{ALL22} that in most part this 
equation and the theory that surrounds it was developed as an attempt to answer
question~(a) above.

The main objective of the present section is to
demonstrate that, contrary to the common belief,
all known methods and mathematical frameworks (see above)
fail to produce an  acceptably accurate and robust equilibrium  in a well known
and widely referenced 
textbook example,
meant to illustrate the existence of equilibrium and to study its nature in a simple
setting with infinite time-horizon, single riskless asset and no shared risk.
In what follows we shall identify the reasons for this failure and shall
outline the strategy and the logic behind the
new methodology developed in the remainder of the \chptr.
We stress that the findings presented below do not imply that the classical  
Bewley-Huggett-Aiyagari model, which has been the workhorse in macroeconomics for more than three
decades, is in error methodologically.
What we show is that this model is incomplete, in that it
fails to account for all consequences from
an important agreement among the economic agents, and, as a result,
despite its very wide adoption,
has not been able to provide a program that reliably
identifies the equilibrium, save for some very special cases.
We also stress that, if restricted to the same
class of models, the new methodology developed in the present \chptr\ leads to equilibria
that are fully consistent with the requirements of the classical
Bewley-Huggett-Aiyagari model, in that all conditions imposed by the latter are satisfied.
In~the same sense, the new methodology is fully consistent also with the Krusell-Smith model.
In~addition to being able to produce equilibria that are consistent with,
but cannot be produced by, previously known methods,
the new framework is also broader in the sense that:
(a)~it does not insist on infinite time horizon and stationarity, and
(b)~in the case of Krusell-Smith type models with aggregate risk, it allows for the study of, and
reveals important long-run fluctuations in,
the disparity across the population of agents, which becomes unobservable if the state space is collapsed
to the population mean alone (see~\cite{KS98}). In addition, the \chptr\ provides a reasonably
complete answer to the approximate aggregation conjecture~-- see question~(a) above. 

Before we turn to the main example of this section,
several important features of the common methods noted in the foregoing are to be brought out.
First, the continuous time framework notwithstanding,
the computational program based on the master equation and MFG 
(see \cite{AHLLM17}) differs from the classical
one (see Sec.~18.2 in \cite{LjunSar00}, for example) mainly in the way in which
exogenously imposed borrowing limits and
the associated state constraints are handled. However, it is well known (see \cite{LjunSar00}) that
the borrowing limits can be endogenized, and this feature is quite intuitive: the amount an agent can
borrow is constrained only by the amount other agents are willing to lend. Such constraints are
difficult to state explicitly, let alone impose exogenously in the outset,
because they are a consequence of the available wealth and its distribution in the cross-section of households,
as well as a consequence of the
collective agreement about the (endogenous) prices, in conjunction with the need for all
private budgets to remain balanced at all times.
The authors of \cite{AHLLM17} underscore the singularities in the population distribution (the
later accumulates at the borrowing limit) and the mathematical problems that these singularities
entail; in fact, the ability to deal with state constraints is listed as one of the main benefits
from the continuous time MFG framework
proposed~in~\cite{AHLLM17}. The mathematical problems associated with this framework are studied further
in \cite{CanCap18} and \cite{CCC21} (see also the discussion in \cite{AHLLM17} and
the extensive references in \cite{CCC21}). One must note that, typically,
singularities in the population distribution are not observed
in the canonical Bewley-Huggett-Aiyagari model, provided that
the borrowing limits are endogenized~-- see \cite{LjunSar00}.
In~fact, at least in the related examples discussed in this \chptr\ % 
(with parameters borrowed from \cite{LjunSar00}) the population is distributed with a smooth density that is
flat and vanishes at the lower end-point of the support interval~-- see Figure~\ref{fg4X} below.
This corresponds to trivial, i.e., vanishing, boundary conditions.
At the same time it is clear that if, for some reason,
sufficiently high borrowing limits are imposed in the outset,
those limits would bind for a nontrivial proportion of the population.  
Among other things, the methodology developed in the present \chptr\  shows that the equilibrium
distribution of the population can be obtained without the need to impose state constraints.%%%%%%%
%%%%%%%%%%%%%%%%%%%%%%%%%%%%%%%%%%%%%%%%%%%%%%
\footnote{In the classical approach to the Bewley-Huggett-Aiyagari model (see \cite{LjunSar00})
borrowing constraints are imposed, but are relaxed until they no longer bind.}
%%%%%%%%%%%%%%%%%%%%%%%%%%%%%%%%%%%%%%%%%%%%%

Another important observation is that, in one way or another, the mathematical frameworks of the
forward Kolmogorov equation, the coupled MFG system and the master equation in MFG 
inevitably involve equations with coefficients that are fixed in the outset.
At the same time, the intrinsic nature of a typical general equilibrium incomplete market model is
such that 
the private optimization problems cannot be formulated in the outset, for, to say the least,
neither the asset prices nor the borrowing constraints can be specified exogenously.
So to speak,
the structure of all private optimization problems can be fixed only during their simultaneous
solution.
Moreover the transport of the population distribution from one period to the next
is the outcome from
a multitude of individual actions, every one of which is based on rational expectations about the
future state of the economy, one component of which is the future population distribution.
Hence, the transport of the population is driven by expectations about the its outcome.
The
notion of competitive equilibrium then requires that the result from the transport must be the same as the
anticipated one, which drives the transport. Figuratively speaking, the transport must be self-aware.
In addition, rational expectations about the endogenous state variables in the future are conditioned to the
realized exogenous aggregate shock in the future, i.e., the transformation of the population
distribution from one period to the next must depend on the future aggregate shock, not just the
present one. All these features impose backward-forward-backward dependencies among all endogenous
variables that cannot be placed within the canonical framework of a forward-backward SDE (FBSDE), or
a coupled MFG system (consisting of a backward HJB and a forward continuity equation), or the
master equation attached to a MFG. Finally, one must note that the MFG framework arises as the
limit (as the number of players $N$ increases to $\infty$) of a Nash-equilibrium in a differential
$N$-game. Nevertheless, it has been known at least since the seminal work of Auman \cite{Aum64} that the
notion of Nash-equilibrium is not meaningful in the context of general equilibrium incomplete
market models with finite number of players. This is because in such models no agent can change
their asset allocation unless other agents do so as well. The (classical, by now) remedy proposed in
\nocite{Aum64}{ibid.} is to introduce a continuum of agents, in which case all agents can be
considered negligible. This
approach still leaves open the question of how infinitely many negligible asset allocations
aggregate into a quantity that is both non-negligible and finite. Moreover, the modern point of
view~-- in most part inspired by the advances in MFG~-- is that models with infinite number of
agents are merely limits of models with finite number of agents. In this \chptr\ we adopt this latter
point
of view, but the equilibrium in an economy with finite number of agents is understood in the sense
of \cite{DL12}, not in the sense of Nash.

The comments in the foregoing strongly suggest that there is a need for developing from scratch a
radically new approach.%%%%%%%%%%%%%%%%%%%%%%%%%%%%%%%
%%%%%%%%%%%%%%%%%%%%%%%%%%%%%%%%%%%%%%%%%%%%
\footnote{The authors of \cite{AHLLM17} admit that 
``several of the typical features of heterogeneous agent models in economics
mean that they are not special cases of the MFGs treated in mathematics.''
The paper \cite{ALL22} is an attempt to test the Krusell-Smith conjecture
by way of placing it in the framework of the master equation, but the authors qualify their conclusion as
``tentative.''
The authors of \cite{CCC21} note
the difficulties in interpreting the coupled MFG system~-- especially its second component, the
continuity equation~-- in the context of heterogeneous agent models.}
%%%%%%%%%%%%%%%%%%%%%%%%%%%%%%%%%%%%%%%%%%%%
This is the objective of the present \chptr. First, we turn to a particularly disturbing example, which
shows that a central problem in macroeconomics has been without an adequate solution since its
inception some decades ago. 

Consider  the familiar savings problem described in Sec.~18.2 in 
the landmark text~\citel{LjunSar00}.
Its~least involved version is Huggett's pure credit economy, first proposed in \citel{Hug93}.
%~-- see Sec.~18.2.3 in \nocite{LjunSar00}{[RMT]}.
In it the agents have exogenous endowments that
follow statistically identical but independent dis\-crete-time Markov chains with state space $\SSS$ and transition
matrix $\PPP$. The elements of $\SSS$ have the meaning of work-hours per period and
the agents trade a single riskless asset
that is in net supply of zero.
The individual asset holdings are restricted to a finite uniform grid $\AA$ over an interval
$[u,v]\subset\R$, the choice of which is ad hoc.
The classical approach to producing an equilibrium
comes down to postulating infinite time horizon and calculating,
for a given (and fixed) interest rate $r$, the aggregate demand for the traded security.
The idea is to vary the choice of $r$ until the aggregate demand becomes null.
The common implementation of this program (see~\citel{LjunSar00})
boils  down to calculating (with fixed $r$) the time-invariant (long-run) optimal policy function
$g_\infty \colon{\AA}\times\SSS\mapsto {\AA}$
for a generic household%%%%%%%%%%%%%%%%%
%%%%%%%%%%%%%%%%%%%%%%%%%%%%%%%
\footnote{{}The optimal policy function maps the pair of previous capital holdings and present
employment state into present capital holdings.}
%%%%%%%%%%%%%%%%%%%%%%%%%%%%%%
and the associated long run
distribution of agents, $\l_\infty(\cdot,\cdot)$, treated as a distribution of unit mass
over the finite space $\AA\times\SSS$. This distribution obtains~-- see Sec.~18.2.1
in \cite{LjunSar00}~-- by iterating to convergence as $t\to\infty$, starting from the uniform distribution
$\l_0(\cdot,\cdot)$, the equation
%%- -%@@%--{Eq: 1.1}--%%
{\abovedisplayskip=9pt plus 1pt minus 1pt\belowdisplayskip=9pt plus 1pt minus 1pt\belowdisplayshortskip=3pt plus 0.5pt minus 0.5pt
\begin{equation}\label{lmb-iter}
\l_{t+1}(a',s')=\sum\nolimits_{a\in{\AA},\,s\in\SSS,\,g_\infty(a,s) = a'}\l_{t}(a,s){\PPP}(s,s')\,,
\quad a'\in{\AA}\,,\ s'\in\SSS\,,
\end{equation}
}%
which is a replica of (18.2.4) in [\cited{LjunSar00}].
The above equation still holds~-- see Sec.~18.2.2 in \cite{LjunSar00}~-- 
if $\l_t(\cdot,\cdot)$ is re-interpreted as the probability distribution at time $t$ of the state of a
generic household that follows the optimal policy $a'=g_\infty(a,s)$, which policy obtains in the
obvious way from iterating to convergence as $t\to\infty$ the Bellman equation
%%- -%@@%--{Eq: 1.2}--%%
\begin{subequations}\label{Bellman}
{\abovedisplayskip=9pt plus 1pt minus 1pt\belowdisplayskip=9pt plus 1pt minus 1pt\belowdisplayshortskip=3pt plus 0.5pt minus 0.5pt
\begin{equation}\label{Bellman-a}
V_t(a,s)=\max\nolimits_{\,c,\,a'}
\Bigl(U(c)+\b\sum\nolimits_{s'\in\SSS}{\PPP}(s,s')V_{t+1}(a',s')\Bigr)\,,\quad
(a,s)\in\AA\times\SSS\,,
\end{equation}
}%
where the maximization is subject to the constraints ($a\in\AA$ and $s\in\SSS$ are given)
{\abovedisplayskip=9pt plus 1pt minus 1pt\belowdisplayskip=9pt plus 1pt minus 1pt\belowdisplayshortskip=3pt plus 0.5pt minus 0.5pt
\begin{equation}\label{Bellman-b}
c+a'=(1+r) a + w s \,,\ \ c\in\Rpp\,,\ \ a'\in{\AA}\,,
\end{equation}
}%
\end{subequations}%
the discount factors $\b>0$ and the wage $w>0$ are given, and so is also the risk aversion
parameter $R$ in $U(c)\df c^{1-R}/(1-R)$. The parameter~$r$ is then varied until the following
identity holds (within an acceptable numerical accuracy)
%%- -%@@%--{Eq: 1.3}--%%
{\abovedisplayskip=7pt plus 1pt minus 1pt\belowdisplayskip=7pt plus 1pt minus 1pt\belowdisplayshortskip=3pt plus 0.5pt minus 0.5pt
\begin{equation}\label{mclr}
\sum\nolimits_{a\tts\in\tts\AA\tts,\,s\tts\in\tts\SSS} g_\infty(a,s)\l_\infty(a,s)=0\,.
\end{equation}
}%
Note that because of the dual meaning of the distribution $\l_\infty(\cdot,\cdot)$ the left side
can be interpreted as the long-run expected demand of a
representative household.   
The strategy just described is illustrated in Figure~\ref{fg1}.%\citel{VHN20}
%%%%%%%%%%%%%%%%%%%%%%%%%%%%%%%%%%%%%%%%%%
%%   Fg: 1    15- in by 9.2 + 1/8 in
%%%%%%%%%%%%%%%%%%%%%%%%%%%%%%%%%%%%%%%%%%%%%%%%%%%%%%%%%%%%
{\captionsetup{belowskip=-10pt}
\captionsetup{aboveskip=0pt}
%%%%%%%%%%%%%%%% 
\begin{figure}[!htbp]
\centering 
\begin{subfigure}{.47\textwidth} 
  \centering
\leavevmode\raise0.85cm\hbox{\rotatebox{90}{\tiny expected representative demand}}% 
\ %   
\toshow{\includegraphics[width=7.1cm]{fg1L}}%

\leavevmode\smash{\hbox to0.3cm{\hfill} \raise6pt\hbox{\tiny interest rate}}

\end{subfigure}\quad
\begin{subfigure}{.47\textwidth}  
  \centering 
\leavevmode\raise0.85cm\hbox{\rotatebox{90}{\tiny expected representative demand}}%  
\ % 
\toshow{\includegraphics[width=7.1cm]{fg1R}}

\leavevmode\smash{\hbox to0.3cm{\hfill} \raise6pt\hbox{\tiny interest rate}} 

\end{subfigure}
\caption{Illustration of the strategy for calculating the equilibrium rate in a pure credit
economy with  asset holdings constrained to a grid $\AA$ of 200 equally spaced points (the right plot is
a microscopic view of a portion of the left).}
\label{fg1} 
\end{figure} }% 
%%%%%%%%%%%%%%%%%%%%%%%%%%%%%%%%%%%%%%%%%%%%%%%%%%%%%%%%  
The~parameter values ($\b>0$,~$w>0$, $\SSS\in\R^7$, $\PPP\in\R^{7\otimes7}$) and the ad hoc range
$[u,v]$ are borrowed
from the first specification from Sec.~18.7 in \cite{LjunSar00} and so is also the program
for computing the expected long-run representative demand for a given interest rate~$r$.%%%%%%%%%%%%%
%%%%%%%%%%%%%%%%%%%%%%%%%%%%%%%%%%%%%%%%%%%%%%%%%%%%%%%%%%%%%%%%
%
\footnote{{}The~computer code (in Julia) with which the plots in Figures~\ref{fg1} and \ref{fg2}
are generated emulates
the \hbox{MATLAB} program that accompanies \cite{LjunSar00}, except in the 
following step: 
the iterations are terminated  after the first simultaneous repetition
of both the policy function and the value function (within a prescribed threshold), not just after
the first repetition of the policy function, as in the original code.
This modification %of the stopping criterion
is necessary because, due to the discretization
of the state space, the 
value function can still improve after the first repetition of the policy function,
if the former has not yet converged to its time-invariant state.
The main reason for translating the original code into
Julia is the ability of the latter to handle very
large grid sizes with only marginal effect on performance.%
}
% 
%%%%%%%%%%%%%%%%%%%%%%%%%%%%%%%%%%%%%%%%%%%%%%%%%  
The~left plot shows the expected demands corresponding
to 20 different choices for the rate.
The~first three rates are chosen arbitrarily and every consecutive rate is the arithmetic average of
the latest rate that yields positive expected demand and the latest rate that yields negative expected
demand (a straightforward implementation of the classical bisection method).
The~right plot shows the expected demands in the last $10$ trials.
While the convergence of the inte\-rest rate is of order $10^{-8}$ (the distance between the last
two rates), there
appears to be a lower bound on how close to $0$ the expected demand can get. %, which is of order
Interestingly, if the same experiment is repeated on a substantially 
more refined grid $\AA$ over the same domain of asset holdings,
then the discontinuity in the expected demand as a function of the inte\-rest
becomes much more pronounced~-- see Figure~\ref{fg2}.
%%%%%%%%%%%% 
%%  Fg: 2
%%%%%%%%%%%%%%%%%%%%%%%%%%%%%%%%%%%%%%%%%%%%%%%%%%%%%%%%%%%%  
{\captionsetup{belowskip=-10pt}
\captionsetup{aboveskip=0pt}
%%%%%%%%%%%%%%%% 
\begin{figure}[!htbp]
\centering 
\begin{subfigure}{.47\textwidth} 
  \centering
\leavevmode\raise0.85cm\hbox{\rotatebox{90}{\tiny expected representative demand}}% 
\ %   
\toshow{\includegraphics[width=7.1cm]{fg2L}}%

\leavevmode\smash{\hbox to0.3cm{\hfill} \raise6pt\hbox{\tiny interest rate}}

\end{subfigure}\quad
\begin{subfigure}{.47\textwidth}  
  \centering
\leavevmode\raise0.85cm\hbox{\rotatebox{90}{\tiny expected representative demand}}%  
\ % 
\toshow{\includegraphics[width=7.1cm]{fg2R}}% 

\leavevmode\smash{\hbox to0.3cm{\hfill} \raise6pt\hbox{\tiny interest rate}} 

\end{subfigure}
\caption{Illustration of the strategy for calculating the equilibrium rate in a pure credit
economy with asset holdings constrained to a grid $\AA$ of $2\mathord,000$ equally spaced points
(the right plot is 
a microscopic view of a portion of the left).}
\label{fg2}
\end{figure} }%
%%%%%%%%%%%%%%%%%%%%%%%%%%%%%%%%%%%%%%%%%%%%%%%%%%%%%%%%
The~culprit for this phenomenon is illustrated in Figure~\ref{fg3}:
the last two in the list of 20 trial rates differ by less than $10^{-7}$ but
the corresponding stationary distributions (obtained by iterating \eqref{lmb-iter} from the uniform
distribution) are very different and so are the respective expected demands,
which differ by more than $3.25$~-- see the right plot in Figure~\ref{fg2}.
%%%%%%%%%%%%
%% Fg: 3
%%%%%%%%%%%%%%%%%%%%%%%%%%%%%%%%%%%%%%%%%%%%%%%%%%%%%%%%%%%% 
{\captionsetup{belowskip=-9pt} 
\captionsetup{aboveskip=0pt}
%%%%%%%%%%%%%%%% 
\begin{figure}[!htbp]
\centering 
\begin{subfigure}{.47\textwidth} 
  \centering
\leavevmode\raise1.2cm\hbox{\rotatebox{90}{\tiny population distribution}}% 
\ %   
\toshow{\includegraphics[width=7.1cm]{fg3L}}% 

\leavevmode\smash{\hbox to0.3cm{\hfill} \raise6pt\hbox{\tiny asset holdings ($2\mathord,000$ grid points)}}

\end{subfigure}\quad
\begin{subfigure}{.47\textwidth}  
  \centering
\leavevmode\raise1.2cm\hbox{\rotatebox{90}{\tiny population distribution}}%  
\ % 
\toshow{\includegraphics[width=7.1cm]{fg3R}}%

\leavevmode\smash{\hbox to0.3cm{\hfill} \raise6pt\hbox{\tiny asset holdings ($3\mathord,000$ grid points)}} 

\end{subfigure}
\caption{The stationary distributions in each of the seven employment categories calculated over
$2\mathord,000$ grid points (left) and $3\mathord,000$ grid points (right), for two interest rates
that differ by less than 
$10^{-7}$ (left) and $10^{-10}$ (right).%
%
}
\label{fg3}
\end{figure} }% 
%%%%%%%%%%%%%%%%%%%%%%%%%%%%%%%%%%%%%%%%%%%%%%%%%%%%%%%%
%%%%%%%%%%%%%%%%%%%%%%%%%%%%%%%%%%%%%%%%%%%%%%%%%%%%%%%%
Pushing the CPU to
$3\mathord,000$ grid points~-- see the right plot in Figure~\ref{fg3}~--
does not remove the discontinuity in the distribution. The rate at which the jump occurs
only moves slightly 
to the left as the density of the grid increases, but the gap in the expected demand remains larger
than $3$ even with $4\mathord,000$ grid points, with neither the left nor the right limit being
close enough to zero.

In sum, when applied to the particular example borrowed here from \cite{LjunSar00}
the common and widely used strategy (see~\nocite{LjunSar00}{ibid.})
fails to identify~-- within an acceptable numerical tolerance~--
the equilibrium rate, despite the temptation to accept as ``almost equilibrium''
the rate suggested by
Figure~\ref{fg1} (it will be shown below that the true equilibrium rate is considerably bigger). 
It would be instructive for what follows in this \chptr\  to identify the reasons for the phenomenon
that Figures~\ref{fg2}~\&~\ref{fg3} reveal.
As~is well known, % result from \citel{SL89}
the failure of the stationary distribution of a Markov chain 
to depend continuously on a parameter, when the transition matrix depends on that
parameter continuously, which is what Figure~\ref{fg3} illustrates,
implies multiplicity of the stationary distribution for certain values of
the parameter.%%%%%%%%%%%%%%%%%%%%%%%%%%%
%%%%%%%%%%%%%%%%%%%%%%%%%%%%%%%%%%%%%%%%%%%%%%%%%%%%%%%%%%%%%
\footnote{It is easy to show that if $I\subseteq\R$ is an interval, the transition matrix of a
particular Markov chain depends continuously on $\l\in I$ and, furthermore, admits a unique
stationary distribution for every $\l\in I$, then that stationary distribution is also a continuous
function $\l\in I$.
%(see Sec.~12.5 in the book ``Recursive Methods of Economic Dynamics'' by N.~Stokey and R.~Lucas).
Hence, the stationary distribution can be discontinuous only if its uniqueness fails
for certain values of $\l\in I$, which is quite intuitive.}
%%%%%%%%%%%%%%%%%%%%%%%%%%%%%%%%%%%%%%%%%%%%%%%%%%%%%%%%%%%%
%([\cited{SL89}, Sec.~12.5]).
Theorem~2 in \citel{Hug93} provides conditions under which such phenomena do not occur in the case of
two idiosyncratic states. While formulating these conditions (which essentially boil down to
certain monotonicity in the transition probabilities) for any finite number of idiosyncratic states is
straightforward, they become less natural~-- and thus difficult to impose generically~-- 
in the presence of more than two idiosyncratic states.
The transition probabilities in the example considered here do
not satisfy such conditions and, for this reason, the discontinuity in Figure~\ref{fg3}
does not come as a surprise.
In particular, the example
shows that, in general, the transition mechanism encoded into equation~\eqref{lmb-iter} cannot be expected
to have a unique fixed point, i.e., the Markov chain followed by the optimal state of the
representative household may have infinitely many stationary distributions.
Clearly, in equilibrium (if one exists with constant risk-free rate and constant distribution of
the population) the long-run
distribution of households must belong to the collection of stationary distributions,
i.e., fixed points, for~\eqref{lmb-iter},
but if this collection is not a singleton then there would be no obvious way
to identify a solution to~\eqref{lmb-iter} that also satisfies the market clearing~\eqref{mclr}.
To~put it another way, iterating to
convergence~\eqref{lmb-iter} from an arbitrary initial distribution cannot be expected to
produce a distribution that is compatible with the notion of equilibrium.
In~what follows we shall identify  two very different fixed points for~\eqref{lmb-iter},
both of which correspond to the equilibrium rate~$r$ obtained with the method developed in the
present \chptr. While one of these two stationary distributions 
yields expected demand that is very close to~$0$, the other one, produced by
iterating~\eqref{lmb-iter} from the uniform distribution as above, yields expected demand that is
very far from~$0$~-- see the large dot in Figure~\ref{fg2}.
One is then led to conclude that, as it stands,
the classical framework described above is incomplete, in the
sense that a solution to  \eqref{lmb-iter} and \eqref{Bellman} that also satisfies
\eqref{mclr} cannot be identified generically within that framework.
To~see why this should not come as a surprise, notice that if the economy is to converge to its
equilibrium state, then the interest, the individual optimal policies, and the population
distribution will all adjust toward their steady-state regimes simultaneously.
One must then note that iterating~\eqref{lmb-iter} with fixed
interest~$r$ and fixed optimal policy $g_\infty(\cdot,\cdot)$ 
has the effect that the economy is assumed to be in some form of partial equilibrium, where some
endogenous quantities have already found their equilibrium values, while the population distribution
is yet to do so through the dynamics of~\eqref{lmb-iter}.
In particular, the representative household is in steady state
(in terms of its optimal policy) but the cross-sectional distribution
of the population is not.
There is no intuition to suggest that the economy must enter such a 
partially equilibrated state, which is to say, the progression of the population distribution
toward its time-invariant configuration may not be governed by~\eqref{lmb-iter}~-- even in situations where
that configuration happens to be a fixed point~of~\eqref{lmb-iter}.
It is important to
also recognize that a modeling framework built exclusively around the optimization problem attached to
a single representative household cannot account for the price-agreement among a large population
of households.%%%%%%%%%%%%%%%%%%%%%%
%%%%%%%%%%%%%%%%%%%%%%%%%%%%%%%%%%%%%
\footnote{{}The system composed of \eqref{Bellman}, \eqref{lmb-iter} and  \eqref{mclr} leaves no room
for imposing the requirement that all households must agree on the risk-free rate set from the
present period to the next. }
%%%%%%%%%%%%%%%%
This price-agreement~-- see \eqref{ze5} below~--
is an important component of the notion of general equilibrium and, as we are about to see,
plays a crucial rôle in its calculation.

The observations from the last paragraph
suggest very strongly that one must develop a system of equations that models the
simultaneous dynamics of prices, optimal private choices, and cross-sectional distribution of the
population before any of these quantities have reached a steady state regime.
In particular, one must develop the notion of economic equilibrium with finite time horizon~$T$ (of
any length) first, and
only then investigate the {\it joint\/} limit of all endogenous variables as~$T\to\infty$, without
any reliance on the representative agent framework (mainly because the latter makes it impossible
to account for the price-agreement among a large group of heterogeneous agents). The hope is
that such a joint limit would be governed by connections that are lost when~\eqref{lmb-iter}
is iterated to convergence with fixed interest rate.
The main difficulty to overcome is the intrinsic backward-forward structure
of the general equilibrium, one consequence from which is that the search for certain
endogenous variables attached to one period in time can only be done simultaneously with the search
for other endogenous variables attached to the next period.
As~a~re\-sult, one is faced with an impossibly large system composed of all first-order and market
clearing conditions attached to all time periods and to all states (aggregate and idiosyncratic)
of the economy. This is the so called ``global method,'' which is
practical only in models with very few periods.%%%%%%%%%%%%%%%%%%%%%%%%%%%%
%%%%%%%%%%%%%%%%%%%%%%%%%%%%%%%%%%%
\footnote{The strategy known as ``the global method'' is outlined briefly in \cite{DL12}.}
%%%%%%%%%%%%%%%%%%%%%%%%%%%%%%%%%%%
We see from the paper \cite{DL12} that, without any reliance on the representative
agent framework, the giant system of market clearing and first order
conditions across time and across the population of agents can be broken into smaller systems,
which can then be chained into a computable backward induction program. 
This strategy parallels the familiar backward induction in dynamic
programming, except that the system to be solved for at every step involves
some endogenous quantities attached to
period~$t$ and other endogenous quantities attached to period~$t+1$.%%%%%%%%%%%%%%%%%%%
%%%%%%%%%%%%%%%%%%%%%%%%%%%%%%
\footnote{While similar in spirit, this strategy does not fit the
framework of Pontryagin's maximum principle.}
%%%%%%%%%%%%%%%%%%%%%%%%%%%%%%
The present \chptr\  follows the same
general strategy, but the need to treat the cross-sectional distribution of the entire population
as an endogenous state variable %, and without any reliance on the notion of representative agent,
requires a substantial reworking of the method proposed in \cite{DL12}.
The reader must be forewarned that, just as in \nocite{DL12}{ibid.}, in the present \chptr\ %
consumption is used as a state variable instead of wealth, so that the population distribution is
treated as a distribution of unit mass over states of employment and levels of consumption~-- not
over states of employment and levels of wealth, as is more common in the literature.
The reasons for this choice essentially boil down to the benefits from identifying (through a
particular homeomorphism) consumption as both state and costate variable. Another (somewhat more
superficial) reason is that in every given time period there is only one consumption level to
attach to every household, whereas the household's wealth at the end of the period is generally different from
that at the beginning.
Unfortunately, a continuous-time version of the program just outlined is yet to be developed and is beyond the
scope of this \chptr.%%%%%%%%%%%%%%%%%%%%%%%%%%%%%%%%%%%%%%
%%%%%%%%%%%%%%%%%%%%%%%%%%%%%%%%%%%%%%%%%%%%%%%%%%%%%%%%%%%%%%%%
\footnote{{}Since the approximation of continuous-time Markov processes with discrete-time Markov
chains is well known, the method proposed in the present \chptr\  should be possible to utilize~--
in principle at least~-- as a
numerical approximation program %(somewhat similar to the finite-differencing)
for solving equilibrium models cast in continuous time by way of approximating them with discrete-time models.}
%%%%%%%%%%%%%%%%%%%%%%%%%%%%%%%%%%%%%%%%%%%%%%%%%%%%%%%%%%%%%%%%


Ultimately, the method developed in the present \chptr\  allows one to construct
a numerically verifiable solution
to Huggett's example introduced above%%%%%%%%%%%%%%%%%%%%%%
%%%%%%%%%%%%%%%%%%%%%%%%%%%%%%%%%%%%%%%%%%%%%%%%%%%%
\footnote{{}To the best of this author's knowledge, this task was never accomplished in full
generality (e.g., with more than two idiosyncratic states, including in the example borrowed from
Ch.~18 in \cite{LjunSar00}) before and does
not appear possible to accomplish with other known methods.}
%%%%%%%%%%%%%%%%%%%%%%%%%%%%%%%%%%%%%%%%%%%%%%%%%%%%
and returns an equilibrium rate of approximately $0.037$
with market clearing of order $10^{-6}$. This rate  differs 
significantly from the one suggested by the left plot in Figure~\ref{fg1}, which is around
$0.029$.
The plots in Figure~\ref{fg4} illustrate the nature of the proposed new approach.
The~left plot shows the equilibrium (produced with the new method)
long-run entering and exiting cross-sectional distributions of households in different employment
categories over asset holdings.
We stress that these two sets of distributions need not be identical, though both obtain
from the same long run distribution over consumption~-- see Sec.~\ref{sec:IOU}.%Figure~\ref{fg4X}.
%%%%%%%%%%%%%%%%%%%%%%%%%%%%%% 
%%   Fg: 4
%%%%%%%%%%%%%%%%%%%%%%%%%%%%%%%
%%%%%%%%%%%%%%%%%%%%%%%%%%%%%%%%%%%%%%%%%%%%%%%%%%%%%%%%%%%% 
{\captionsetup{belowskip=-10pt}
\captionsetup{aboveskip=0pt}
%%%%%%%%%%%%%%%% 
\begin{figure}[!htbp]
\centering 
\begin{subfigure}{.47\textwidth} 
  \centering
\leavevmode\raise0.65cm\hbox{\rotatebox{90}{\tiny distribution within emplpyment group}}% 
\ %   
\toshow{\includegraphics[width=7.1cm]{fg4L}}%

\leavevmode\smash{\hbox to0.3cm{\hfill} \raise6pt\hbox{\tiny asset holdings}}

\end{subfigure}\quad
\begin{subfigure}{.47\textwidth}  
  \centering
\leavevmode\raise0.65cm\hbox{\rotatebox{90}{\tiny distribution within employment group}}%  
\ % 
\toshow{\includegraphics[width=7.1cm]{fg4R}}%

\leavevmode\smash{\hbox to0.3cm{\hfill} \raise6pt\hbox{\tiny asset holdings}} 

\end{subfigure}
\caption{Left plot: entering (solid lines) and exiting (dotted lines)
equilibrium distributions of households in every employment category
over asset holdings produced with the new method.
Right plot: the population distribution produced
with the classical (see above)
method over $2\mathord,000$ grid points (solid lines)  when the iterations are initiated 
with the equilibrium rate of $0.037$ and with the exiting distribution (dotted lines) obtained with
the new method (replicated from the left plot).}  
\label{fg4} 
\end{figure} }%
%%
%%
It~is instructive to note that
if the classical program (see \eqref{lmb-iter} and \eqref{Bellman})
leading to Figure~\ref{fg2} is initiated with the equilibrium rate
obtained with the new method, i.e., $r\approx0.037$, and
with the discretized (over $2\mathord,000$ grid points) version
of the exiting distribution from the left plot in Figure~\ref{fg4} (instead of the uniform one), then
it~re\-turns expected demand of order~$10^{-3}$, together with the distribution shown in solid lines
on the right plot.
This illustrates the multiplicity of the stationary distribution that was noted earlier%%%%%%%%%%
%%%%%%%%%%%%%%%%%%%%%%%%%%%%%%%%%%%%%%%%%%%%%%%%%
\footnote{{}Note that the existence of multiple stationary distributions does not amount to an existence
of multiple equilibria, since only one of these distributions is found to clear the market.}%%%
%%%%%%%%%%%%%%%%%%%%%%%%%%%%%%%%%%%%%%%%%%%%%%%%%
:
the large dot in Figure~\ref{fg2} corresponds to the same rate $r\approx 0.037$ (the actual
equilibrium rate), which is to say, the optimal policy $g_\infty(\cdot,\cdot)$ and thus the
structure of \eqref{lmb-iter} attached to large dot on Figure~\ref{fg2} are the same, 
but the expected demand of~$\approx 6.901$ is very different from $10^{-3}$~-- all due to the fact that the
iterations of \eqref{lmb-iter} leading to Figure~\ref{fg2} were initiated with another (arbitrarily chosen)
distribution, whereas the iterations of \eqref{lmb-iter} leading to the right plot in
Figure~\ref{fg4} were initiated with the exiting distribution produced with the new method
(which does not involve \eqref{lmb-iter} in any way).%%%%%%%%%%%%%%%%%%%%%%%
%%%%%%%%%%%%%%%%%%%%%%%%%%%%%%%%%%%%%%%%%%%%
\footnote{{}We see that the classical approach described above is capable of locating the equilibrium rate after all,
but with the caveat that 
one needs to know how to locate the correct distribution (out of infinitely many)
with which to initiate the iterations of \eqref{lmb-iter}~-- somehow, one needs to know the correct
distribution in order to locate it.}
%%%%%%%%%%%%%%%%%%%%%%%%%%%%%%%%%%%%%%%%%%%
The small difference between the distributions shown in solid and dotted lines
in the right plot in Figure~\ref{fg4} is due to the fact that  (see \eqref{Bellman}) the distribution
$\l_\infty(\cdot,\cdot)$, obtained by iterating \eqref{lmb-iter},
is defined over pairs of an employment state attached to the 
present period and exiting wealth (taken from the finite grid) attached to the previous period,
i.e., the private state is exiting in terms of wealth but entering in
terms of employment.%%%%%%%%%%%%%
%%%%%%%%%%%%%%%%%%%%%%%%%%%%%%%%%%
\footnote{{}Such an approach is useful only in the absence of aggregate shocks.}
%%%%%%%%%%%%%%%%%%%%%%%%%%%%%%%%%
In contrast, the distribution obtained with the new method, shown in doted lines,
is over pairs of employment and exiting wealth attached to the same period.
It is interesting to note that the distributions shown in Figure~\ref{fg4}~-- whether produced with the new method
developed later in this \chptr, or with the classical method described above~-- have no
points of accumulation. This finding is in contrast with the key aspect of the
paper \cite{AHLLM17} (in continuous time
setting and with only two employment states),
in which the borrowing limit is imposed exogenously and
the population distribution is found to accumulate
around that limit.%%%%%%%%%%%%%%%%%%%%%%%%%%%%%
%%%%%%%%%%%%%%%%%%%%%%%%%%%%%%%%%%%%%%%%%
\footnote{{}\label{foot-AHLLM}The continuous time MFG framework and the borrowing constraint aside,
the computational strategy proposed in \cite{AHLLM17} is concerned with the infinite time horizon case
and essentially boils
down to the classical one: 
choose an interest rate, compute
the expected demand after solving the Kolmogorov forward equation attached to the representative agent,
repeat with a different rate until the expected representative demand becomes null.
The concrete example of Aiyagari-Huggett economy in \cite{AHLLM17}, with which the strategy is illustrated,
involves only two employment states, in which case multiplicities like the one illustrated above are
easier to avoid.}
%%%%%%%%%%%%%%%%%%%%%%%%%%%%%%%%%%%%%%%%%
We stress that the general method with which the
left plot in Figure~\ref{fg4} was produced does 
not require the borrowing limit to be fixed in the outset and does not involve boundary conditions
at that limit (such conditions are meaningless in the discrete setup adopted here).

Perhaps the most interesting application of the new mathematical framework proposed
in this \chptr\  is to models with production that is subjected to common
(for all agents) productivity shocks.
This is the scenario where, generically, no time-invariant distribution of the
population exists, not even conditioned to the realized productivity state.
The only time-invariance that one may hope for is for the
population distribution, treated as a stochastic process, to be Markov in random environment, with time-invariant
transition (transport) mechanism.
The general strategy, adopted throughout most of
the literature on Krusell-Smith's model,%%%%%%%%%%
%%%%%%%%%%%%%%%%%%%%%%%%%%%%%%%%%%%%
\footnote{{}\label{ftl1}%
There is an extensive body of research~-- see vol.~34 (2010) of Journal of Economic
Dynamics \& Control and the references therein~-- concerned with  the robustness and the
accuracy of the algorithm used in the benchmark case study in \cite{KS98}.
In particular, the paper by Den Haan in the same volume discusses the weaknesses (referred  to as ``fatal
flaws'') of the $R^2$ and
the standard regression error tests, as used in \cite{KS98},
to measure compliance with the relations defining the equilibrium.
Nevertheless, it will be shown in Sec.~\ref{sec:KS} below that
the results obtained by the simulation technique proposed in
\cite{KS98}, though narrower in scope, are reasonably accurate~-- at least with model
parameters chosen as in \nocite{KS98}{ibid.}.} 
%%%%%%%%%%%%%%%%%%%%%%%%%%%%%%%%%%%
including  \cite{KS98},
is to reduce the cross-sectional distribution to a finite list of moments and then 
describe~-- somehow~-- the way in which the next period's cross-sectional moments
depend on the current moments and aggregate productivity state, which then determines
the individual policies. In most cases, including in \cite{KS98}, this law of motion is studied 
only in the long run, and is deciphered 
from the simulated behavior of a large population of households over a large number of periods.%%%%%%%%%%
%%%%%%%%%%%%%%%%%%%%%%%%%%%%%%%%%%%
\footnote{{}In Krusell and Smith's benchmark case study \cite{KS98}
the law of motion of the first moment in the long run is
extracted by way of least square fitting from the simulated behavior of 5,000 households over 11,000 time periods. }
%%%%%%%%%%%%%%%%%%%%%%%%%%%%%%%%%%%
This is suboptimal because 
both the individual and the collective behavior (hence, the equilibrium 
itself) depend only on the distribution of the
population, and full information about the position of each and every agent is massively 
superfluous. 
Another drawback from this approach is the insistence on~a ``sufficiently large'' time horizon and
the persistence of i.i.d.\ prediction errors, which lack a clear economic interpretation.
One is also faced with the need to postulate a particular type of dependence (log-linear in the benchmark study
of \cite{KS98}) in the outset.

In addition to resolving the problem illustrated in Figure~\ref{fg2} and providing a numerically
verifiable solution to the classical  Aiyagari-Bewley-Huggett model, the present \chptr\  makes the
following 
contributions to the study of models with shared risk, in particular to 
Krusell-Smith's model with production and
aggregate risk associated with shocks in the productivity factor.
First, it identifies analytically~-- not
empirically, on a case-by-case basis~-- general conditions under which the approximate aggregation
hypothesis holds (approximately).
Second, the \chptr\  shows that even if all aggregate variables and individual choices are forced
to depend on the population distribution only through its mean,
the exact form of the law of motion of the full population distribution can still be
identified; in particular, the population distribution cannot be chosen arbitrarily, subject to the
only requirement for its mean to follow a prescribed set of dynamics.
This feature reveals fluctuations in the disparity across the population that are
substantially larger than the fluctuations in the productivity shocks~-- see Figure~\ref{fgKS5}
below~-- and cannot be captured by a model that is confined to the population mean alone
(somehow small fluctuations in the productivity shocks lead to much larger fluctuations in the
disparity%
%%%%%%%%%%%%%%%
\footnote{To the best of this author's knowledge such phenomena have not been documented before.}%
%%%%%%%%%%%%%
).  
Third, it is shown that movements of the population distribution
take place in the random environment of the transition in
the productivity state~-- not in the random environment of the productivity state
itself. %%%%%%%%%%%%%%
%%%%%%%%%%%%%%%%%%%%%%%%%%%%%%%%%%%%%%%
%\footnote{{}Föllmer's paper \cite{Foll94} appears to be the first to investigate the connection
%between asset pricing and 
%diffusion in random environment.} 
%%%%%%%%%%%%%%%%%%%%%%%%%%%%%%%%%%%%%%%
To~put it another way, the transformation from the present period to the
next depends on the productivity states in both, present and future, time periods.
This structure is consistent with the way in which Krusell and Smith cast their general
model in [\cited{KS98}-II-B], in which the updating rule
is written as $\G'=H(\G,z,z')\tts$. However, the computational strategy described in
[\cited{KS98}-II-C] assumes an updating rule for 
the mean of the form ${\bar k\tts}'=h(\bar k, z)$. This later form persists   
in the concrete results presented in [\cited{KS98}-III], and, indeed, persists throughout most
of the literature on Krusell-Smith's model (see, just as an example, vol.~34 (2010) of Journal of Economic
Dynamics \& Control). This has lead some authors to conclude% %(see \cite{ES21}, for example)
%%%%%%%%%%%%%%%%%%%%%%%%%%%%%%%%%%%%%%%%%%%%%%
\footnote{{}See, for example, the paper by Masakuzi and Sunakawa in {\it  Economic Letters\/} {\bf 206} (2021).}
%%%%%%%%%%%%%%%%%%%%%%%%%%%%%%%%%%%%%%%%%%%%%% 
that the pair of the representative private state and the productivity state admits
a stationary joint distribution in the
long run, with the implication that, in the long run, a fixed value for the population mean can be
attached to every productivity state.
However, it will be shown in Section~\ref{sec:KS} below (see Figure~\ref{fgKS-IK}) that,
even in the classical benchmark study of~\cite{KS98},
such a feature does not hold~-- not even as a reasonably acceptable approximation.
Moreover, the dispersion of the population mean in the long run
is much larger than the dispersion of the productivity state, even as the fluctuations
in the aggregate productivity state are the one and only reason for the population
mean to fluctuate~-- a~ratchet effect of~a~sort.
Lastly, the general methodology proposed in the present \chptr\  is meaningful for any, large or
small, 
time horizon; in particular, models with infinite time horizon are merely limits of models with
finite time horizon. This feature is important for two main reasons.
The first one is that even in the classical examples borrowed
here from \cite{LjunSar00} and \cite{KS98} all time-invariant features
are attained after at least several hundred periods, whereas no real economy can
remain unchanged for that long. The second reason is that although postulating infinite time horizon
simplifies the matters enormously, it also makes invisible important connections that may be needed
in order to identify the equilibrium~-- see above. 


Several important warnings and disclaimers are now in order. The computational strategy developed in
the present \chptr\  seeks to endogenize internally quantities that, traditionally, have not been endogenized~--
at least not internally.%%%%%%%%%%%%%%%%%%%%%%%%%%%%
%%%%%%%%%%%%%%%%%%%%%%%%%%%%%%%%%%
\footnote{{}An example of external endogenization would be a program that calculates demands with
given (as if exogenously specified) prices, and then varies the prices until market clearing
is attained in the long run, i.e., the prices are kept unchanged for many periods.
Traditionally, the borrowing constraint has been specified only exogenously.}
%%%%%%%%%%%%%%%%%%%%%%%%%%%%%%%%%
In addition, the strategy is deeply rooted in a special time-reordering of the endogenous
variables, which does not comply with the common Markovian structure
and has no precedent in the literature other than the paper~\cite{DL12}.
The benefits from this new approach notwithstanding (see above), one drawback is that existence 
(nothing to say about uniqueness) of the equilibria is impossible to establish generically~--
at least not with
tools that are currently available. The main reason is that most of the dependencies that the program
seeks to resolve are implicit, in which case the use of the classical fixed-point type argument~-- as in,
say, the default reference \citel{DGMM94}~-- becomes very difficult.
In~particular, all variations of the general program outlined
in Section~\ref{sec:gen-model} involve several layers of iterations the convergence of which is not
guaranteed. Although endless loops do not occur in the examples included in this \chptr\  and the
convergence is quite fast, any computer code that implements the program
must limit the number of iterations in order to prevent potential endless loops.
We~stress, however, that the program tests for accuracy and convergence at every step, and as long
as it
completes, the result is always a numerically verifiable equilibrium. Another drawback is that a
continuous-time analog of the model described below is not currently available for two main
reasons: the departure from the classical Markovian structure (the need to solve simultaneously for
endogenous variables attached to different time periods) and the random environment given by the
transition in the productivity state, not by the productivity state alone.
In particular, the transport  operators
from~\ref{main-q} below do not appear to have an easily identifiable analog in any
known continuous-time framework. 

%
%
%
\iffalse

The proposed new method appears to be closer in spirit to the probabilistic
approach to MFG described in \cite{CD18}, but a direct connection is difficult to
establish mainly because, while ``similar in spirit to,''
heterogeneous agent models are not exactly games (see above).
Nevertheless, one must acknowledge the existence of a game theoretic interpretation of
similar macroeconomic models~-- see, for
example, \cite{GKSS00}, \cite{KSS94},~\cite{KSS97}. 

\fi
%
%
%



The \chptr\  is organized as follows. Sec.~\ref{sec:gen-model} describes the limit of a
generic
heterogeneous agent model with finite time horizon and finite number of households
and outlines a metaprogram for identifying an equilibrium. Sec.~\ref{sec:IOU}
specializes that metaprogram
to the case of an economy with infinite time horizon and no aggregate risk. 
It is shown there that, the plots in Figure~\ref{fg2} notwithstanding, the new
strategy can locate an equilibrium in the same benchmark study.
Sec.~\ref{sec:KS} implements the metaprogram from Sec.~\ref{sec:gen-model} in
the context of the benchmark economy of Krusell and Smith \cite{KS98},
compares the results, and draws new insights.
The Julia code used in all examples can be retrieved from 
the author's \href{https://github.com/AndrewLyasoff/heterogeneous-models}{Git
repository}.





%%% chend556677

%%- -%%--{Sec: #2}--%%
%{Eq: 2.}   %%llabel
%{No: 2.}   %%nlabel

\section{Self-Aware Transportation by Way of Lagrange Duality}%
\label{sec:gen-model}\setcounter{paragraph}{0}


\noindent
The main result in this section, Proposition~\ref{main-q} below, describes the transport of
the distribution of households over the range of 
consumption and states of emp\-loy\-ment from one period to the next.
In~essence, \ref{main-q}-\eqref{zze666} is an alternative to the Kolmogorov forward equation,
or to the master equation in MFG, and takes into account features that are unique to
heterogeneous agent models. 
It governs
the law of motion of the cross-sectional distribution of a large population of interacting
agents~--
as opposed to the law of motion of the probability distribution of a single Markovian state, as was
illustrated in the previous section. 
The~rule that governs this law of motion 
is such that it depends on the outcome, i.e., the transport is self-aware~-- see 
\ref{self-dep} and \ref{MFG-rem} below.
First, we introduce the notation and describe the general setup.
For~the sake of simplicity we consider an economy with only two assets: productive capital and
private risk-free lending. Our starting point is essentially the finite time horizon model and
related program introduced in \cite{DL12} 
(with some modifications meant to accommodate for both aggregate and idiosyncratic shocks).
One notable departure from the setup adopted in \nocite{DL12}{ibid.}\ is that the full vector of all individual
consumption levels is now replaced by the population distribution over the range of consumption.


The time parameter $t$ is restricted to the finite set
$\{0,1,\ldots,T\}$ and the total number of households (alias: agents), $N$, is
assumed, for now, to be finite.
Economic output is generated in every period and is expressed in units
of a single numéraire good, which 
can be either consumed, or turned into productive capital 
except during period $T$.
Every household extracts utility from consuming the numéraire good. All households share the same
impatience parameter $\b>0$ and the same time-separable utility from 
intertemporal consumption given by the mapping $U\colon\R\mapsto \R$,
which is twice continuously differentiable in $\Rpp$ with $\partial U>0$, $\partial^2 U<0$,
and is such that   $\lim_{c\searrow 0}\partial U(c)=+\infty$, $\lim_{c\searrow 0}U(c)=-\infty$, and
$U(c)=-\infty$ for $c\le 0$.  
Economic output is generated by two inputs: the net labor supplied during the period
the output is delivered
and the net productive capital installed during the previous period. Two investment
instruments are available to all households: capital stock
and locally risk-free private lending instrument (alias: IOU), which is in net supply of zero.%%%
%%%%%%%%%%%%%%%%%%%%%
\footnote{{}The numéraire good, which is also the currency, cannot be stored from one period to
the next, but entitlements to it can be carried from one period to the next by means of financial
contracts. The assumption that the private lending instrument is in zero net supply is imposed here
merely for simplicity.}
%%%%%%%%%%%%%%%%%%%%%%%%%%%%%%%%%%%%%%%%%%%%%%%%%%%%%% 
The collection of all (idiosyncratic) private  employment states 
is $\EEE\subset\Rp$ and the collection of all
productivity states is $\XXX\subset\Rpp\tts$.
These sets have finite cardinalities denoted $\abs{\EEE}$ and~$\abs{\XXX}$, both assumed to be at
least~$2$.
The~elements of $\XXX$ have the meaning of total factor productivity (TFP)
and the elements of $\EEE$ have the meaning of  physical units of labor.
The productivity state, which is shared by all households, follows an irreducible Markov chain in
the state space $\XXX$
with transition probability matrix~$Q$ (of size $\abs{\XXX}$-by-$\abs{\XXX}$)
that has a unique set of steady-state probabilities $(\psi(x),\, x\in\nobreak\XXX)$.
The transitions in the individual employment states, which are independent from one another
when conditioned to a particular transition in the productivity state $x \to y$,
are governed by the transition probability matrices~$P_{x, y}\in\R^{\abs{\EEE}\otimes\abs{\EEE}}$,
$x,y\in\XXX$. All elements of the
matrices $Q$ and $P_{x,y}$ are assumed strictly positive. The pair consisting of 
the shared productivity state and the employment state
of a particular household follows a Markov chain on the state-space $\XXX\times\EEE$ with transition
from $(x,u)$ to $( y,v)$ occurring with probability  $Q(x, y)P_{x, y}(u,v)$.


Households that are in the same state of employment would have identical consumption levels only if
they start the period with identical asset holdings,%%%%%%
%%%%%%%%%%%%%%%%%%%%%%%%%%%%%%%%%%%%%%%%%%%
\footnote{{}Both assets, private lending and capital, are assumed to be perfectly liquid, in which
case the composition of asset holdings is irrelevant.}
%%%%%%%%%%%%%%%%%%%%%%%%%%%%%%%%%%%%%%%%%%
in which case their
investment decisions would be identical as well (see \ref{thm1} below for an explanation).
To put it another way,
households that are in the same state of employment and choose the same consumption
level $c\in\Rpp$, which quantity represents physical units of the numéraire good, are indistinguishable.
For this reason, in what follows consumption will be used as a state variable instead of wealth. 
Thus, the mathematical metaphor for the collective
state of the population is the distribution of unit mass over the product space
$\EEE\times\Rpp$ (the space of household characteristics, i.e., employment and
consumption levels).
To~be able to work with such objects,
we now introduce the space~$\bbP(\EEE)$ of strictly positive
unit-mass (a.k.a. probability) measures over $\EEE$,
the space $\DB$ of all (cumulative)
càdlàg distribution functions over~$\Rpp$, and the collection~$\DB^\EEE$ of all assignments
$\csd\colon \EEE\mapsto \DB$. An~element~$\csd\in\DB^\EEE$ can be
identified as a finite list of distribution functions $\csd\equiv(\csd^u\in\DB)_{u\tts\in\tts\EEE}$, in which
one cumulative distribution function over~$\Rpp$ is assigned to every employment category $u\in\EEE$.
Any probability measure on $\EEE\times\Rpp$ can be disintegrated into the form $\pi(\dm u)\d
F^u(c)$ and treated as a pair $(\pi,\csd)\in\bbP(\EEE)\times\DB^\EEE$.
When representing the distribution of~$N$
households in the state space $\EEE\times\Rp$, a measure of this form would have at most~$N$
atoms every one of which has a mass that is an integer multiple of $1/N$, so
that $\pi(u)N$ gives the total number of households who happen to be in employment state~$u$ and
$\pi(u)\csd(c)N$ gives the total number of households who happen to be in employment state~$u$ and
choose consumption level that is not strictly larger than $c\in\Rpp$.
If the private states of $N$ agents are distributed over the space
$\EEE\times\Rpp$ with law $\pi(\dm u)\d F^u(c)$ then the average employment level across the population
can be cast as
{\abovedisplayskip=8pt plus 3.5pt minus 3.5pt\belowdisplayskip=8pt plus 3.5pt minus 3pt\belowdisplayshortskip=8pt plus 3.5pt minus 3.5pt
\[
L(\gp) %= \sum\nolimits_{u\ts\in\ts\EEE}\,{u\over N} \,\gp(u) \,N
= \sum\nolimits_{u\ts\in\ts\EEE}u\,\gp(u)  = \gp\ts\EEE\,,
\]
}%
where $\gp$ and $\EEE$ are treated as vector row and vector column respectively.
In particular, the aggregate amount of installed labor is given by $L(\pi)N=\gp\ts\EEE\ts N$.
Similarly, if $\qq_u(c)$
denotes the capital invested by every household who happens to be in state
$(u,c)\in\EEE\times\Rpp$, then the average private capital
invested across the population is
{\abovedisplayskip=8pt plus 3.5pt minus 3.5pt\belowdisplayskip=8pt plus 3.5pt minus 3pt\belowdisplayshortskip=8pt plus 3.5pt minus 3.5pt
\[
K(\gp,\csd) 
= \sum\nolimits_{u\ts\in\ts\EEE}\,\gp(u)\int_\Rpp\qq_u(c)\d\csd^u(c)\,,
\]
}%
and the aggregate installed capital is $K(\gp,\csd) N$.


%%- -%##%--{No: 2.1}--%%
\begin{nit}{Remark}%\label{cmsr}
It is very common in the literature to integrate the level of a particular private variable against
a probability measure that represents the distribution of the population and to declare that the integral
gives the aggregate level of that same variable across the entire population~--
see \cite{KS98} as just one example.
One must note, however, that 
such integrals only represent the weighted average level across the population~-- not the
aggregate level. As~the number of agents increases to $\infty$ the aggregate level can remain
finite only if the private levels become negligible and the private levels can remain
non-negligible only if the aggregate level is allowed to explode.
There are two common scenarios in which ignoring the difference between the average level across
the population and the aggregate level is innocuous. The~first one is when
the aggregate level must be adjusted to $0$, as in equation \eqref{mclr} above. The second one is
when a Cobb-Douglas production function is postulated~-- see below.\qed
\end{nit}


As a next step, we postulate the usual ``competitive firm'' with two factors of production,
capital and labor, and with production technology 
given by  a Cobb-Douglas constant return to scale production function with
capital share parameter $0<\a<1$. 
Thus, the rates of return on capital and labor, realized during the future period, can be treated as
functions of the average privately installed capital~$K$ during the present period.
These functions must depend on the future productivity state $y\in\XXX$ and the future distribution
over states of employment $\varpi\in\bbP(\EEE)$.
In equilibrium factor prices maximize the firm's profits, so that
the rates of return on capital and labor are given (as functions of the average $K$)
by%%%%%%%%%%%%%%%%%%%%%%%%%%%%%%%%%%
%%%%%%%%%%%%%%%%%%%%%%%%%%%%%%%%%%%%%%%%%%%%%%%%%%%%%
\footnote{Note that the average capital $K$ belongs to the present period while the pair
$(y,\varpi)$ belongs to the future (i.e., the next) period, during which both returns are realized.}
%%%%%%%%%%%%%%%%%%%%%%%%%%%%%%%%%%%%%%%%%%%%%%%%%%%%%
{\abovedisplayskip=7pt plus 1.5pt minus 5pt\belowdisplayskip=7pt plus 1.5pt minus 5pt\belowdisplayshortskip=3pt plus 1.5pt minus 6pt
\begin{equation}\label{returns}
\begin{gathered}
\Rpp\ni K \leadsto \rho_ {y,\varpi}(K)\df  y\times\a\times
\Bigl({K N\over L(\varpi) N}\Bigr)^{\a-1}\equiv y\times\a\times \Bigl({K\over L(\varpi)}\Bigr)^{\a-1}\\
\llap{and\quad}
\Rpp\ni K \leadsto \ee_ {y,\varpi}(K)\df
y\times(1-\a)\times \Bigl({K \over L(\varpi N)}\Bigr)^{\a}
\equiv y\times(1-\a)\times \Bigl({K \over L(\varpi)} \Bigr)^{\a}\,.
\end{gathered}
\end{equation}
}%
We see that with this special choice of the production function replacing the aggregate capital
$K N$ and the aggregate labor $L(\varpi) N$ with the respective population averages
$K$ and $L(\varpi)$ would not matter. As the returns in \eqref{returns} depend only on the
population averages, they are perfectly meaningful for any population distribution (expressed as a
probability measure on $\EEE\times\Rpp$) that may or may not correspond to a finite population of agents. 
Installed productive capital is assumed to depreciate at constant rate $\dd>0$ and,
in order to generate paychecks at time $t=0$, we postulate the fictitious quantity
$K_{-1}$, which has the meaning of a primordial average endowment with capital
that is shared equally among all households.%%%%%%%%%%%%%%%%%%%%
%%%%%%%%%%%%%%%%%%%%%%%%%%%%%%%%%%%%%%%%%%%%%%%
\footnote{{}There is no need for the households to be identical before time $0$.
This assumption is imposed only for the sake of simplicity.} 
%%%%%%%%%%%%%%%%%%%%%%%%%%%%%%%%%%%%%%%%%%%%%%


Because economic agents are concerned only with returns, they are
concerned exclusively with the distribution over the space $\EEE\times\Rpp$ on which the averages
depend.
We postulate now that the number of agents is infinite and that their distribution over
$\EEE\times\Rpp$ can be any measure of the form $\pi(\dm u)\d F^u(c)$
for some (any) choice of the pair $(\pi,\csd)\in\bbP(\EEE)\times\DB^\EEE$.
Thus, the aggregate state of the economy during any given period is understood to be
a triplet of the form $(x,\pi,\csd)$, for some
choice of a productivity state $x\in\XXX$ and population
distribution $(\pi,\csd)\in\bbP(\EEE)\times\DB^\EEE$.
If the dependence on the time period needs to be emphasized in the
notation we shall write $(x_t,\pi_t,F_t)$ instead of $(x,\pi,F)$ and shall use similar
conventions for all other quantities that we may introduce, with the understanding that the
subscript~$t$ may be omitted, if the association with a specific time period is irrelevant.
It is important to recognize that the first two elements of the triplet $(x,\pi,\csd)$ have
dynamics that are fully exogenous and unrelated to the individual choices across the population,
i.e., have no relation to the domain of economics. 
For this reason, we call the pair $(x,\pi)\in\XXX\times\bbP(\EEE)$ exogenous aggregate state, or
simply exogenous state. The third component of the aggregate state $(x,\pi,F)$, i.e.,
$F\in\DB^\EEE$, we call endogenous aggregate state, or simply endogenous state.


The assumption that the population of agents is infinite has important implications which
we now address.
The major simplification that takes place in the limit as $N\to\infty$ is in the following.
Assuming that the present period productivity state $x\in\XXX$ transitions in the next period to
productivity state $y\in\nobreak\XXX$, every household presently in employment state $u\in\EEE$
would sample its future employment state from the collection~$\EEE$, independently from all other households,
according to the distribution law over~$\EEE$ given by the vector~$P_{x, y}(u,\cdot)\in\R^{\abs{\EEE}}$.
By Glivenko-Cantelli's theorem, as the number of households in state $u$ increases to~$\infty$,
the proportion of all households presently in
employment state~$u$ who transition to employment state~$v$ must converge to $P_{x,y}(u,v)$.
Once the number of households is postulated to be infinite, one can set aside the assumption that the
shocks in employment are independent (and hence forfeit the reliance on Glivenko-Cantelli's
theorem)
and simply postulate that the collection of households in state
$u$ who transition to state $v$ can be weighted against the population of households in state $u$,
with relative weight given by $P_{x,y}(u,v)$.

%%- -%##%--{No: 2.2}--%%
\begin{nit}{Remark}\label{issues}
The technical problems associated with measuring and comparing infinite collections of agents are well
known. First, one cannot distribute a finite mass uniformly across a countably infinite collection of
agents.
It is possible to distribute a finite mass uniformly over a continuum, but then the
only sets that can be compared would be the elements of a particular $\s$-field. This is
inadequate because if all agents sample their employment state independently, there would be no
reason for the set of agents who fall into a particular employment category to belong to that
special \hbox{$\s$-field}. The~workaround that we use below has two main aspects. The first one is dispensing
with the notion of ``uniform distribution of weights'' and simply postulating the relative weights suggested
by Glivenko-Cantelli's theorem. The second one is dispensing with the need for a universal $\s$-field
specified in the outset, and taking advantage of the fact that relative weights can be assigned in a
consistent fashion to the elements of finite partitions 
and sub-partitions (however defined) as they come along period by period.\qed
\end{nit}


%%- -%##%--{No: 2.3}--%%
\begin{nit}{Sets of Agents and Their Relative Weights}\label{weighing}
Let $\W$ stand for the collection of all economic agents. Let $A_u\subset\W$ be the set of all agents
who happen to be in state $u\in\EEE$ during a given period~$t$. Then $(A_u,\,u\in\EEE)$ is a
finite partition of $\W$. We make no assumptions about the structure of the sets $\W$ and $A_u$ other than
insisting that $A_u$ is not a finite set for any $u\in\EEE$. We do assume, however, that
for every $u\in\EEE$ the collection of agents $A_u$ can be weighted against the collection $\W$ and
the associated list of relative weights is given by some $\pi\in\bbP(\EEE)$. Let $B_{u,v}$ be the
subset of $A_u$ consisting of all agents who transition to state $v\in\EEE$ in period $t+1$, so
that $A_u=\cup_{v\tts\in\tts\EEE}B_{u,v}$.  We make no assumptions about the structure of the sets
$B_{u,v}$ other than insisting that none of these sets is finite. Furthermore, we suppose that when the
productivity states in periods $t$ and $t+1$, respectively $x$ and $y$, are known,
then every set $B_{u,v}$ can be weighted against $A_u$ and its relative weight is given by
$P_{x,y}(u,v)$.  In particular, $B_{u,v}$ can be weighted against $\W$ with relative weight
$\pi(u)P_{x,y}(u,v)$.
Next, given any $u,v\in\EEE$ and any $c\in\Rpp$, let $E_{u,v}(c)$ stand
for the subset of $B_{u,v}$ consisting of all agents whose consumption level during period~$t$ does not
exceed~$c$ (strictly). We again suppose that $E_{u,v}(c)$ can be weighted against $B_{u,v}$ with
relative weight that is independent from $v$ and given by $\csd^u(c)$ for some choice of
$\csd\in\DB^\EEE$.%%%%%%%%%%%%%%%%% 
%%%%%%%%%%%%%%%%%%%%%%%%%%%%%%%%%%%%%%%%%%%%%%%%
\footnote{During period $t$ all agents in state $u$ face the same uncertain future in terms of
employment and make investment-consumption decisions before their future employment state is
realized. Therefore, the individual choices across any ``sufficiently representative'' subset of $B_u$
must be statistically indistinguishable from those in any other ``sufficiently representative''
subset.
The insistence that the relative weight of  $E_{u,v}(c)$ does not depend on $v$ is a mathematical
metaphor for the claim  
that $B_{u,v}$ is a ``sufficiently representative'' subset of $A_u$ for any $v$,
but we stress that the independence of the relative weight of $E_{u,v}(c)$
from the future employment state $v$ is an assumption that we impose, not a conclusion that we
arrive at.}
%%%%%%%%%%%%%%%%%%%%%%%%%%%%%%%%%%%%%%%%%%%%%%%
Hence, the relative weight of
$E_{u,v}(c)$ against~$A_u$ is $P_{x,y}(u,v)\csd^u(c)$ and against~$\W$ it is
$\pi(u)P_{x,y}(u,v)\csd^u(c)$, but these relative weights become available only if the present and
the future productivity states, $x$ and $y$, are known.
Finally, any finite union of sets of the form  $E_{u,v}(c)$ for various choices of $u,v\in\EEE$
and $c\in\Rpp$, or of the form $B_{u,v}$ for various choices of $u,v\in\EEE$,
has a well defined relative weight against~$\W$.%%%%%%%%%%%%%%%%%%%%%%%%%%%%%%%%%%
%%%%%%%%%%%%%%%%%%%%%%%%%%%%%%%%%%%%%%%%%%%%%%%%%%%%%%
\footnote{This is because any such finite union can be written in a unique way as the finite union
of disjoint sets that were already assigned relative weights.}
%%%%%%%%%%%%%%%%%%%%%%%%%%%%%%%%%%%%%%%%%%%%%%%%%%%%%%
In particular, the collection of
agents in state $u$ whose consumption level does not exceed~$c$ can be weighted against~$A_u$ with
relative weight given by $\sum_{v\tts\in\tts\EEE} P_{x,y}(u,v)\csd^u(c)=\csd^u(c)$, and therefore
also weighted
against~$\W$ with relative weight $\gp(u)\csd^u(c)$.
Similarly, given any $v\in\EEE$, the collection of agents in state $v$ during period $t+1$
is noting but the union of disjoint sets $\cup_{u\tts\in\tts\EEE}B_{u,v}$, so that the
period-$(t+1)$ distribution over states of employment is given by
{\abovedisplayskip=8pt plus 1.5pt minus 1.55pt\belowdisplayskip=8pt plus 1.5pt minus 1.55pt\belowdisplayshortskip=5pt plus 1.5pt minus 3pt
\[
\varpi(v)=\sum\nolimits_{u\ts\in\ts\EEE}\pi(u)\, P_{x, y}(u,v)\quad 
\text{for all}\ \ v \in\EEE\,,
\]}%
which we may abbreviate as
$
\varpi=\pi\, P_{x, y}\,,
$
treating $\pi$ and $\varpi$ as vector rows of size $\abs{\EEE}$.\qed
\end{nit}

The last relation illustrates one of the key advantages of working with an infinite population of
agents: the period-$(t+1)$ distribution over employment is fully determined by the period-$t$
distribution over employment, in conjunction with the productivity states in both periods.
Nevertheless, without further restrictions on the model
the aggregate exogenous state $(x,\pi)\in\XXX\times\bbP_{++}(\EEE)$ would not be constrained
to a finite set and this complicates enormously all practical aspects of the model.
It turns out to be possible to reduce, at the expense of certain restriction on the transition
probabilities $P_{x,y}(u,v)$, the range of the exogenous state to the finite
collection $\XXX$, as is explained next. 

%%- -%##%--{No: 2.4}--%%
\begin{nit}{Assumption and Remark}\label{n1}
In the benchmark economy studied in \cite{KS98}  the
conditional transition matrices
$P_{x, y}$ are chosen in such a way that it becomes possible to attach a unique distribution,
$\gp_x\in\nobreak\bbP(\EEE)$, to every productivity state $x\in\XXX$ so that
$\gp_ y =\gp_xP_{x, y}$ for all choices of $x, y\in\XXX$ (the future distribution over
employment depends only on the future productivity state).
With this special choice for the matrices $P_{x, y}$ the steady state regime of the population in
terms of employment is such that the distribution of households over states of employment
fluctuates randomly but in perfect sync with the productivity state, so that when the
productivity state is $x\in\XXX$ the distribution of households over states 
of employment is exactly $\gp_x\in\nobreak\bbP(\EEE)$.%%%%%%%%%%%%%%
%%%%%%%%%%%%%%%%%%%%%%%%
\footnote{{}This feature comes from the fact that
the Markov chain on $\XXX\times\EEE$ with transition matrix $Q(x, y)P_{x, y}(u, v)$ admits a
steady-state distribution in which state $(x,u)$ occurs with probability $\psi(x)\gp_x(u)$.}
%%%%%%%%%%%%%%%%%%%%%%%%%%%%%%%%%%%%%%%%%%%%%%%%%%%%
This is a vast simplification, because in such a regime the distribution of households over states of employment
fluctuates through the finite collection $\{\gp_x\colon x\in\XXX\}\subset \bbP(\EEE)$, as opposed
to fluctuating through the infinite collection~$\bbP(\EEE)$.
In~what follows we shall suppose that this simplification is in force
and shall assume without further notice that productivity and employment are fluctuating according
to the steady state regime just described (even at time $t=0$). In~addition, we shall
exclude the scenario where productivity or employment
can get absorbed in a single state. The population distribution in every productivity state can now
be given as an element of $\DB^\EEE$, since the distribution over $\EEE$ is fixed by
the productivity state.
Thus,
an ``aggregate state of the economy'' will be understood to mean a pair
of the form $(x,\csd)\in\XXX\times\DB^\EEE$, consisting of a productivity state $x$ and a list of
distribution functions $(\csd^u\in\DB,\,u\in\EEE)$. The exogenous state is then just the
productivity state $x\in\XXX$. For the sake of simplicity we set
$L(x)=L(\pi_x)$, $\rho_y(K)=\rho_ {y,\pi_y}(K)$, and $\ee_ {y}(K)=\ee_ {y,\pi_y}(K)$.\qed
\end{nit}

Because of the assumption postulated in \ref{n1},
the movements of the population distribution over the space $\EEE\times\Rpp$
can now be treated as movements in the space $\DB^\EEE$ which take
place in the random environment given by the transition in the productivity state~-- the only
exogenous aggregate variable faced by all agents. Describing these movements is one of the key
aspects of the model and is the task we turn to next.


%%- -%##%--{No: 2.5}--%%
\begin{nit}{Time-dependent transport}\label{et}
The transport of the population distribution from period~$t$ to period $t+1$ must be allowed to depend
on the period-$t$ productivity state $x\in\XXX$ and on the period-$(t+1)$ productivity state
$y\in\XXX$. We express this transport as a collection of mappings
$\T_{t,x}^y\colon\DB^\EEE\mapsto\nobreak \DB^\EEE$, $x,y\in\XXX$.
Rational private decisions about consumption and investment during period $t$, given the period-$t$
productivity state $x\in\XXX$, are only possible if an assumption about the period-$t$ population
distribution $F\in\DB^{\EEE}$ and about the collection of transport mappings
$\{\T_{t,x}^y\colon y\in\nobreak \XXX\}$ is made.
At the same time, the realized population distributions in periods $t$ and $t+1$,
respectively $F_t^*$ and $F_{t+1}^*$, are the result of a multitude of individual choices made during periods
$t$ and $t+1$, with the understanding  that $F_{t+1}^*$ depends on the realized period-$(t+1)$
productivity state \smash{$y^*\in\XXX$}, which affects the consumption choices during period~$t+1$.
The notion of competitive equilibrium requires that the assumed distribution and its transport coincide
with the realized ones, i.e, $F_t^*=F$ and \smash{$F_{t+1}^*=\T_{t,x}^{y^*}(F)$} in every possible  
realization of the future
productivity state $y^*\in\XXX$.
This requirement, which we call ``self-awareness of the transport'' (see~\ref{self-dep}),
is the main challenge in the study of incomplete-market models with a large number
of heterogeneous agents and addressing it beyond the realm of what is commonly known as
``stationary recursive equilibria''
is the main goal in what~follows in this \chptr.\qed
\end{nit}
\nitskip


The features outlined in~\ref{et} have two important aspects: (a)~during period $t$ the households do not
observe $F_t^*$ (the realized population distribution after all private decisions are made)
but assume that $F_t^*=F$ for some percieved $F\in\DB^\EEE$, and (b) all private
choices during period $t$ take into account the shared prediction for the period-$(t+1)$ state of the
population, expressed as the list $\{\T_{t,x}^y(F)\colon y\in\XXX\}$.

%%- -%##%--{No: 2.6}--%%
\begin{nit}{Remark}
There is an obvious parallel between the transport
(for fixed $t$) introduced in \ref{et} and the transport of mass arising 
in the classical Monge-Kantorovich problem~-- see \cite{Vil03}, \cite{Vil09}, \citel{Gal16}, for example. 
There is also a crucial difference:
there is no single surplus function (see \cite{Gal16})\ whose average is to be optimized and the target measure
is endogenous (see below).
This makes the problem more challenging and also more interesting: the transport is determined
from optimizing over a large number of individual objectives rather than a single global one.
Because of this shift in the paradigm, most of the tools for solving the Monge-Kantorovich problem
developed in the course of the last two
centuries will not be possible to utilize in the present context~-- at least not directly. Nevertheless,
some of the general ideas can still be mimicked and put to use. The method developed below still
relies on the idea of randomization introduced by Kantorovich, but not by way of coupling of two
probability measures. It also relies on the idea of Monge coupling, but not by way of pure
assignment~-- see \ref{rnd-Monge} below.\qed 
\end{nit}


The next step is to state the one-period
savings problem faced by a generic household and formulate its dual.
All households observe
the history of the realized productivity state, together with the
history of their own individual employment states and asset holdings, 
share the same belief about the transition probabilities that govern the employment and
productivity shocks (encrypted in the matrices~$P$ and~$Q$), 
share the same belief about the initial (time~$0$) population distribution, and, finally,
share the same belief about the entire collection of
transport mappings%%%%%%%%%%%%%%%%%%%%%%
%%%%%%%%%%%%%%%%%%%%%%%%%%%%%%%%%%%%%%%%%%%%%%%%%%%%%%%%%%%%%%%% 
\footnote{{}Allowing the transport to be time dependent is one of the key differences between
the approach adopted here and previous works~-- see \cite{KS98}.}
{\abovedisplayskip=5pt plus 2pt minus 2pt\belowdisplayskip=8pt plus 2pt minus 2pt\belowdisplayshortskip=5pt plus 1.5pt minus 1.5pt
$$
\T\df \bigl\{ \T_{t,x}^y \colon 0\le t<\nobreak T,\, x,y\in\XXX \bigr\}\,.
$$
}%
%%%%%%%%%%%%%%%%%%%%%%%%%%%%%%%%%%%%%%%%%%%%%%%%%%%%%%%%%%%%%%%%
Hence, during any given period all households have identical beliefs about
the present population distribution (over the range of consumption and levels
of employment), once the history of the productivity state until
that period is revealed. As a result,
given the stream of present and future transport mappings,
private decisions about savings and consumption depend on the current employment state,
the asset holdings at the beginning of the period (entering wealth),
the current productivity state, and the perceived population
distribution during the current period.%%%%%%%%%%%%%%%%%%%%%%%%%% 
%%%%%%%%%%%%%%%%%%%%%%%%%%%%%%%%%%%%%%%%%%%%%%%%
\footnote{{}The individual savings problems are influenced by the history of the productivity shocks only
through their dependence on the current population distribution.}
%%%%%%%%%%%%%%%%%%%%%%%%%%%%%%%%%%%%%%%%%%%%%%%
In particular, all global (shared) endogenous variables, namely the average installed capital and
risk-free rate, depend on the time period $0\le t<T$ and
on the aggregate state of the economy $(x,F)\in\XXX\times\DB^\EEE$ during that period
%$\T_{t,\cdot}^\cdot,\ldots,\T_{T,\cdot}^\cdot$,
(again: for a given stream of present and future transport mappings).


Suppose that during period $0\le t<T$ the economy happens to be in state $(x,F)\in\XXX\times\DB^\EEE$, the
average installed capital is $K=K_t(x,F)>0$, and the (one period) risk-free rate is $r=r_t(x,F)>-1$.
Consider a generic household that enters employment state $u\in\EEE$ 
with its personal wealth measuring~$w$ units of the numéraire good,
which quantity aggregates the wage received during period $t$, the return on capital invested in the
previous period,
and the holdings of private loans carried from the previous period.
%%%%%%%%%% 
Treating the entering wealth $w $
as a given resource and taking $K$ and $r$ as given (the household is a price taker),
the household must determine its consumption level $c$,
its investment $\q$ in the private lending instrument (IOU),
and its investment $\qq$ in productive capital%%%%%%%%%%%%%%%%%%%%%%%%%%%%%%%
%%%%%%%%%%%%%%%%%%%%%%%%%%%%%%%%%%%%%%%%%%%%%%%%%%%%%%%%%%%%%%%%%%
\footnote{All three quantities $c$, $\q$ and $\qq$ are understood to represent physical
units of the numéraire good.}
%%%%%%%%%%%%%%%%%%%%%%%%%%%%%%%%%%%%%%%%%%%%%%%%%%%%%%%%%%%
by maximizing over $(c,\q,\qq)\in\R^3$ and $(W_{y,\tts v}\in\R)_{ y\tts\in\tts\XXX, v\tts\in\tts \EEE}$ the
objective
%%
%%- -%@@%--{Eq: 2.1}--%%
{\abovedisplayskip=8pt plus 2pt minus 2pt\belowdisplayskip=8pt plus 2pt minus 2pt\belowdisplayshortskip=5pt plus 1.5pt minus 1.5pt
\begin{equation}\label{ze1}
%\begin{aligned}
J_t\Bigl(c ,\,(W_{y,\tts v} )_{ y\tts\in\tts\XXX, v\tts\in\tts \EEE}\Bigr)
\df U(c )
+\b\sum\nolimits_{ y\ts\in\ts\XXX,\,v\ts\in\ts\EEE}
V_{t+1,\tts y,\tts \T_{t,\tts x}^y(F),\tts v}\bigl(W_{y,\tts v} \bigr)\, Q(x, y)P_{x, y}(u, v)\,,
%\end{aligned}
\end{equation}}%
subject to
%%
%\begin{gather}
%%- -%@@%--{Eq: 2.2}--%%
{\abovedisplayskip=8pt plus 2pt minus 2pt\belowdisplayskip=8pt plus 2pt minus 2pt\belowdisplayshortskip=5pt plus 1.5pt minus 1.5pt
\begin{equation}\label{ze2}
\begin{gathered}
%\label{ze20}
W_{y,\tts v}=\bigl(1+r\bigr)\q + \bigl(\rho_ y(K)+ 1-\dd\bigr)\qq + 
\ee_ y(K)v\,,\  \  y\in\XXX\,,\  v \in\EEE\,,\\
%\noalign{\rlap{and}\vskip-15pt}
%\label{ze2} 
\llap{and\qquad\qquad}c+\q+\qq=w\,,
\end{gathered}
\end{equation}}%
%\end{gather}
%%
with the understanding that  $V_{t,\tts x,\tts F,\tts u}(w )$ is the constrained maximum attained in
\eqref{ze1} for every $0\le t<\nobreak T$ and
$V_{T,\tts y,\tts \T_{T-1,\tts x}^y(F),\tts v}\bigl(W_{y,\tts v} \bigr)=U\bigl(W_{y,\tts v} \bigr)$,
i.e., during the last period the household can only consume (note that $U(c )=-\infty$ if $c\le 0$ and
$U\bigl(W_{y,\tts v} \bigr)=-\infty$ if $W_{y,\tts v}\le 0$ by the very definition of $U\phd$).
We~stress that no~borrowing constraints are imposed extraneously in the individual savings problems. The borrowing
limits arise 
endogenously from the notion of equilibrium~-- see~\ref{def-equil} below.%%%%%%%%%%%%%%%%%%%
%which demands that every private savings problem admits a solution
%%   with all private budgets remaining balanced at all times,
%and all solutions are such that
%the average capital investment across the population generates returns that all agents agree~on, while
%the average demand for private lending is~$0$.%%%%%%%%%%%%%%%%
%%%%%%%%%%%%%%%%%%%%%%%%%%%%%%%%%%%%%%%%%%%%%%%%%%%%%%%%
\footnote{To~put it another way, the agents formulate their private savings problems under the assumption
that~$K$ and~$r$ are chosen (and given to them) so that, after all private instances of
(\ref{ze1}-\ref{ze2}) have been solved, all lenders lend what they
perceive as optimal to lend, all borrowers borrow what they perceive as optimal to borrow, all
private budgets are balanced at all times (present and future),
and the market for every security clears.
Any such arrangement removes the need for extraneously imposed limits on borrowing. 
One~may consider the infimum over the private demands so obtained as being the ``borrowing limit,'' but such
quantity becomes meaningful only after all individual savings problems have been solved and does
not constrain any agent.}
%%%%%%%%%%%%%%%%%%%%%%%%%%%%%%%%%%%%%%%%%%%%%%%%%%%%%%%%%%%%%%%%%%%%%%%%
The range of the value function is the interval
$\llbkt -\infty,\infty\rlbkt$, with the value of $(-\infty)$ attained only if, for the given entering wealth,
there is no policy that can fund
strictly positive consumption in all possible future aggregate and idiosyncratic states.
By convention, the derivatives of any function will be treated as undefined on any domain in which
its value is $(-\infty)$ and the appearance of derivatives implies that the argument belongs to a
domain in which the function is finite;
as an example, the appearance of any of the symbols $\partial U(c)$ and $\partial^2 U(c)$ 
implies~$c>0$.

%%- -%##%--{No: 2.7}--%%
\begin{nit}{Remark}\label{1350}
In effect (\ref{ze1}-\ref{ze2}) is a family of optimization problems
parameterized by the employment state $u$ and the entering wealth $w$,
not the optimization problem attached to a
``representative household.'' 
Furthermore, since the agents are distinguished by their employment and
consumption levels, which vary from one
period to the next, the sequence of systems (\ref{ze1}-\ref{ze2}) obtained for $t=T-1,\ldots,0$
cannot be viewed as ``the Bellman equation'' of any one agent.
It~would be instructive for what follows
to think of the collection of problems  (\ref{ze1}-\ref{ze2}) as a prescription for  
transporting  the population distribution over the space~$\EEE\times\Rpp$ from one period to the
next. One of the main goals in what follows is to decipher this prescription.\qed 
\end{nit}
\nitskip
 


Suppose next that all value functions $V_{t+1,\tts y,\tts \T_{t,\tts x}^y(F),\tts v}\phd$
in \eqref{ze1} are strictly concave and in $\C^2$ on the domain in which they are finite,
in which case the objective function in \eqref{ze1} is also strictly concave and in $\C^2$
in the domain where it is finite.
As a result, if $V_{t,\tts x,\tts F,\tts u}(w )$ happens to be finite, then the first order Lagrange, also known as
Karush-Kuhn-Tucker (KKT), conditions attached to the problem (\ref{ze1}-\ref{ze2}) must hold at any 
$(c,\q,\qq)\in \R^3$ that happens to be a solution.
Furthermore, these conditions are also sufficient: if  the
KKT conditions attached to (\ref{ze1}-\ref{ze2})
hold at some $(c,\q,\qq)\in\R^3$, at which the objective in \eqref{ze1} is finite,%%%%%%%
%%%%%%%%%%%%%%%%%%%%%%%%%%%%%%%%%%%%%%%%%%%%
\footnote{{}This is the only situation in which the KKT conditions are meaningful.}
%%%%%%%%%%%%%%%%%%%%%%%%%%%%%%%%%%%%%%%%%%%
then $(c,\q,\qq)$ is the (unique)
solution (\ref{ze1}-\ref{ze2}). 
The KKT conditions are instrumental in what follows and are introduced next.
The idea again is to use those conditions as a tool for deciphering the channels through which
population distribution gets transported  from one period to the next.
To this end, we now restate the first set of constraints in \eqref{ze2} in the form
(meant to place the resource $w$ in the right side of every constraint) 
{\abovedisplayskip=5pt plus 1pt minus 1pt\belowdisplayskip=5pt plus 1pt minus 1pt\belowdisplayshortskip=3pt plus 0.5pt minus 0.5pt
$$   
W_{y,v}-\bigl(1+r\bigr)\q-\bigl(\rho_ y(K)+ 1-\dd\bigr)\qq -
\ee_{ y}(K)v + c + \q + \qq = w\,,\ \  y\in\XXX\,,\  v \in\EEE\,.
$$}%
The costate variable (Lagrange multiplier) attached to each of these constraints we deliberately cast
in the factor form
{\abovedisplayskip=5pt plus 1pt minus 1pt\belowdisplayskip=5pt plus 1pt minus 1pt\belowdisplayshortskip=3pt plus 0.5pt minus 0.5pt
$$
\l_{y,\tts v}=\Lmp_{y,\tts v} \times {\b }\ts Q(x, y)P_{x, y}(u,v)\,,\ \  y\in\XXX\,,\  v \in\EEE\,,
$$}%
i.e., we will be working with $\Lmp_{y,\tts v}$ instead of the true costate variable $\l_{y,\tts v}$,
and  the costate variable attached to the last condition in \eqref{ze2} we denote by $\ff $.
The Lagrange dual of the optimization problem in (\ref{ze1}-\ref{ze2}) can be stated as
{\abovedisplayskip=5pt plus 1pt minus 1pt\belowdisplayskip=5pt plus 1pt minus 1pt\belowdisplayshortskip=3pt plus 0.5pt minus 0.5pt
$$
\mathop{\text{minimize}}_{\ff,\tts (\Lmp_{y,\tts v})_{ y\tts\in\tts\XXX,\, v\in\EEE}}
\biggl(\,\mathop{\text{maximize}}_{c,\tts \q,\tts \qq,\tts (W_{y,\tts v})_{ y\in\XXX,\tts v\in\EEE}}
\LLL\Bigl(c,\q,\qq,(W_{y,\tts v})_{ y\tts \in\tts \XXX,\, v\tts \in\tts \EEE},
\ff,(\Lmp_{y,\tts v})_{ y\tts \in\tts \XXX,\, v\tts \in\tts \EEE}\Bigr)
\biggr)\,,
$$}%
where
{\abovedisplayskip=5pt plus 1pt minus 1pt\belowdisplayskip=5pt plus 1pt minus 1pt\belowdisplayshortskip=3pt plus 0.5pt minus 0.5pt 
\begin{align*}
\LLL\Bigl(c,\q,&\qq,(W_{y,\tts v})_{ y\tts \in\tts \XXX,\, v\tts \in\tts \EEE},
\ff,(\Lmp_{y,\tts v})_{ y\tts \in\tts \XXX, \, v\tts \in\tts \EEE}\Bigr)\\
&\qquad\df J_t\Bigl(c ,\,(W_{y,\tts v} )_{ y\tts\in\tts\XXX,\, v\tts\in\tts\EEE}\Bigr) 
+ {\ff }\bigl(w-c-\q-\qq\bigr)\\
&\qquad\qquad \qquad \qquad+
{\b }\sum\nolimits_{ y\ts\in\ts\XXX,\,v\ts\in\ts\EEE}\Lmp_{y,\tts v}\Bigl(w
-W_{y,\tts v}+\bigl(1+r\bigr)\q+\bigl(\rho_y(K)+ 1-\dd\bigr)\qq\\
&\hbox to6.5cm{\hfill}+\ee_{ y}(K)\ts v-c - \q -\qq \Bigr) Q(x, y)P_{x, y}(u,v)\,.
\end{align*}
}%
With the substitution
%%
%%- -%@@%--{Eq: 2.3}--%%
{\abovedisplayskip=5pt plus 1.5pt minus 1.5pt\belowdisplayskip=5pt plus 1.5pt minus 1.5pt\belowdisplayshortskip=6pt plus 1.5pt minus 1.5pt
\begin{equation}\label{bsde}
\f\df \varphi + {\b}\sum\nolimits_{ y\ts\in\ts\XXX,\,v\ts\in\ts\EEE}\Lmp_{y,\tts v}Q(x, y)P_{x, y}(u,v)\,,
\end{equation}}%
equating to $0$ the derivatives of the Lagrangian
relative to $W_{y,\tts v}$, $c$, $\q$, and $\qq$ gives, respectively,
%%
%%- -%@@%--{Eq: 2.4}--%%
{\abovedisplayskip=5pt plus 1.5pt minus 1.5pt\belowdisplayskip=7pt plus 1.5pt minus 1.5pt\belowdisplayshortskip=6pt plus 1.5pt minus 1.5pt
\begin{equation}\label{ze4}
\begin{gathered}
\Lmp_{y,\tts v}=\partial V_{t+1,\tts y,\tts \T_{t,\tts x}^y(F),\tts v}\bigl(W_{y,\tts v} \bigr)\,,\quad
\f={\Up}(c )\,,\\
\f=\bigl(1+r\bigr)\b\sum\nolimits_{ y\ts\in\ts\XXX,\,v\ts\in\ts\EEE}\Lmp_{y,\tts v}\,Q(x, y)P_{x, y}(u,v)\,,\\
\f=\b\sum\nolimits_{ y\ts\in\ts\XXX,\,v\ts\in\ts\EEE}\Lmp_{y,\tts v}\,\bigl(\rho_{ y}(K)
+1-\dd\bigr) Q(x, y)P_{x, y}(u,v)\,,
\end{gathered}
\end{equation}}%
and, after a straightforward application of the envelope theorem,
%%
%%- -%@@%--{Eq: 2.5}--%%
{\abovedisplayskip=5pt plus 1.5pt minus 5pt
%\abovedisplayshortskip=-7pt plus 1.5pt minus5.5pt
\belowdisplayskip=5pt plus 1.5pt minus 1.5pt
\belowdisplayshortskip=6pt plus 1.5pt minus 1.5pt
\begin{equation}\label{ze4a1}
\partial V_{t,\tts x,\tts F,\tts u} (w )=\varphi +{\b}\sum\nolimits_{ y\tts\in\tts\XXX,\,v\tts\in\tts\EEE}
\Lmp_{y,\tts v}\,Q(x, y)P_{x, y}(u,v)=\f={\Up}(c )\,.
\end{equation}}%
If~the system (\ref{ze2}-\ref{ze4})  
can be solved for $c$, $q$, $\qq$, $W_{y,v}$, $\f$, and
$\Lmp_{y,v}$, then any such solution must depend on the time period $t$, the aggregate state
$(x,F)$, and the employment state~$u$. If~this dependence must be emphasized in the notation,
we shall embellish the symbols $c$, $q$ and $\qq$ with subscripts in the obvious way
(recall that in the above $K$ and $r$ stand for $K_t(x,F)$ and $r_t(x,F)$).
The entire system (\ref{ze2}-\ref{ze4}) depends 
on the future distributions $\T_{t,x}^y(F)$, $y\in\nobreak\XXX$, i.e., depends on the assumed
structure of the transport mappings~$\T_{t,x}^y\phd$, $y\in\nobreak\XXX$~-- not just on
the assumed statistical behavior of the aggregate and idiosyncratic shocks.
The~system depends also on the entering wealth (resource)
$w$, and all these dependencies will be incorporated into the notation when needed
(and suppressed when understood from the context, for the sake of simplicity). 
%%%%%%%%%%%%%%%%%
%%%%%%%%%%%%%

The next step is to restate the KKT conditions in a more useful form. First, \eqref{ze2}
and \eqref{ze4} reduce to the following system of three equations for the unknowns $c$, $\q$ and
$\qq$, i.e., for $c_{t,\tts x,\tts \csd,\tts u}$, $\q_{t,\tts x,\tts \csd,\tts u}$ and
$\qq_{t,\tts x,\tts \csd,\tts u}\,$:
%%
%%
%%- -%@@%--{Eq: 2.6}--%%
{\abovedisplayskip=5pt plus 1.5pt minus 1.5pt\belowdisplayskip=7pt plus 1.5pt minus 1.5pt\belowdisplayshortskip=6pt plus 1.5pt minus 1.5pt
\begin{equation}\label{z2-no-lm}
\begin{gathered}
c +\q +\qq -w =0\,,\\
{\Up}(c ) -  \sum_{ y\ts\in\ts\XXX,\,v\ts\in\ts\EEE}
(1+r)\,\b\,\partial V_{t+1,\tts y,\tts \T_{t,\tts x}^y(F),\tts v}\bigl(W_{y,\tts v} \bigr)\,
Q(x, y)P_{x,y}(u,v)=0\,,\\
{\Up}(c )- \sum_{ y\ts\in\ts\XXX,\,v\ts\in\ts\EEE}
\bigl(\rho_y(K)+1-\dd\bigr)\,\b\,
\partial V_{t+1,\tts y,\tts \T_{t,\tts x}^y(F),\tts v}\bigl(W_{y,\tts v} \bigr)\,
Q(x, y)P_{x,y}(u,v) =0\,,
\end{gathered}
\end{equation}
}%
where the expressions $W_{y,\tts v}$ are as defined in \eqref{ze2}.
Next, observe that with \eqref{ze4a1} applied to period $t+1$ the first equation in \eqref{ze4}
states:
%%
{\abovedisplayskip=5pt plus 1.5pt minus 5pt
\belowdisplayskip=5pt plus 1.5pt minus 1.5pt
\belowdisplayshortskip=3pt plus 1.5pt minus 1.5pt
\[
\Lmp_{y,\tts v}=\partial V_{t+1,\tts y,\tts \T_{t,x}^y(F),\tts v}\bigl(W_{y,\tts v} \bigr)
={\Up}(c_{t+1,\tts y,\tts \T_{t,x}^y(F),\tts v} )\,.
\]}%
It is now easy to remove the costate variables from the KKT conditions
by casting the last two equations in \eqref{ze4} in
the familiar form of ``kernel conditions:''
%%
%%- -%@@%--{Eq: 2.7}--%%
{\abovedisplayskip=10pt plus 1.5pt minus 2pt\belowdisplayskip=10pt plus 1.5pt minus 1.5pt\belowdisplayshortskip=8pt plus 1.5pt minus 1.5pt
\begin{equation}\label{ze5} 
\begin{gathered}
1=\bigl(1+r\bigr)\,
\b\sum\nolimits_{ y\tts\in\tts\XXX,\,v\ts\in\ts\EEE}{{\Up}(c_{t+1,\tts y,\tts \T_{t,x}^y(F),\tts v} )
\over {\Up}(c_{t,\tts x,\tts F,\tts u} )} Q(x, y)P_{x, y}(u,v)\,,\\
1=\b\sum\nolimits_{ y\tts\in\tts\XXX,\,v\tts\in\tts\EEE}{{\Up}(c_{t+1,\tts y,\tts \T_{t,x}^y(F),\tts v} )
\over {\Up}(c_{t,\tts x,\tts F,\tts u} )}\bigl(\rho_{ y}(K)+1-\dd\bigr) Q(x, y)P_{x, y}(u,v)\,.
\end{gathered} 
\end{equation}}%
The meaning of these conditions is that in equilibrium all agents agree on the returns that
the two traded securities generate (the right sides in \eqref{ze5} must be identical across all agents).
In~all concrete implementation presented later in the \chptr\  the utility function
$U\phd$ is chosen to be isoelastic, in which case
{\abovedisplayskip=8pt plus 1.5pt minus 5pt\belowdisplayskip=8pt plus 1.5pt minus 1.5pt\belowdisplayshortskip=6pt plus 1.5pt minus 1.5pt
$$
{{\Up}(c_{t+1,\tts y,\tts \T_{t,x}^y(F),\tts v} )\over {\Up}(c_{t,\tts x,\tts F,\tts u} )}
={\Up}\Bigl({c_{t+1,\tts y,\tts \T_{t,x}^y(F),\tts v}\over c_{t,\tts x,\tts F,\tts u}}\Bigr)\,.
$$}%
For the sake of simplicity of the notation, we shall assume this form now on, and note that,
simplicity aside, it also makes the model invariant to re-scaling the consumption variable. 

%%- -%##%--{No: 2.8}--%%
\begin{nit}{Remark}
Clearly,
$\Up\colon\lrbkt0,\infty\rlbkt\mapsto\lrbkt0,\infty\rlbkt$
provides a homeomorphism between
$c_{t,\tts x,\tts F,\tts u} $ and $\f$ and between $c_{t+1,\tts y,\tts \T_{t,x}^y(F),\tts v} $ and
$\Lmp_{y,v}$. In particular, the system \eqref{ze5} does not really exclude the costate
variables, as it retains their homeomorphic copies.
Thus, consumption plays multiple rôles: it~is a household
descriptor, state variable, control parameter, and, up to a homeomorphism, a costate variable.\qed
\end{nit}
%\nitskip

All three equations in \eqref{z2-no-lm}~--
recall that $V_{t+1,\tts y,\tts \T_{t,\tts x}^y(F),\tts v}\phd$
is assumed strictly concave and in $\C^2$ wherever it is finite~-- 
define the vector $(c,\q,\qq)$ as an implicit
$\C^1$-function of the entering wealth $w$ and the last two equations in \eqref{z2-no-lm} define
the portfolio vector $(\q,\qq)$ as an implicit $\C^1$-function of the consumption level $c\in\Rpp$. This
last feature is instrumental for what follows, as it allows for the use of consumption as a state
variable (instead of wealth). The next theorem makes
these statements precise.  

%%
%%- -%##%--{No: 2.9}--%%
\begin{nit}{Theorem}\label{thm1}
If  $w\in\R$ is such that $V_{t,x,F,u}(w )$ is finite and
the system \eqref{z2-no-lm} admits a solution $(c,\q,\qq)$,
then \eqref{z2-no-lm} admits a unique solution 
for every entering wealth from some open neighborhood of $w$ and that solution is a $\C^1$-function 
of the entering wealth with $\partial c>0$.  
Moreover, $V_{t,x,F,u}\phd$ is a $\C^1$-mapping, and hence also a $\C^2$-mapping, with 
$\partial V_{t,x,F,u}\phd > 0$ and $\partial^2 V_{t,x,F,u}\phd<0$ in some neighborhood of $w $.
In~addition,  if the system composed of the
last two equations in \eqref{z2-no-lm} admits a solution $(\q,\qq)$ for some
fixed $c\in\Rpp$, then that system admits a unique solution  for every consumption level
in some open neighborhood of~$c$ and that solution is 
a $\C^1$-function in the neighborhood of~$c$. All results continue to hold if one of the traded
assets is removed from the model, i.e., the households invest only in the private lending
instrument, or only in productive capital.\qed
\end{nit}

The proof of \ref{thm1} is given in Appendix~\ref{sec:A2}.
One consequence from this theorem is that the value function is a strictly concave $\C^2$-function
on any domain on which it happens to be finite, and this property holds in all time periods and for all
realizations of the aggregate and idiosyncratic states. 
The theorem also shows that there is a one-to-one correspondence between optimal consumption
and entering wealth. Since in every period the
households differ only in their entering wealth and state of employment, they can be distinguished
just as well by their state of employment and consumption level~-- a~strategy that we have adopted
already. We stress that although households are identified as elements of $\EEE\times\Rpp$, this
set is not a true labeling set, in the sense that an element $(u,c)\in\EEE\times\Rpp$ identifies a
collection of households that are indistinguishable as economic agents, rather than a single
physical household. Furthermore, the consumption level and state of employment, i.e., ``the identity,''
of any one household changes over time.

Another crucially important consequence from the last theorem is that resolving
the individual sa\-vings problem comes down to assigning a consumption level $c\in\Rpp$ to any period
$t<T$ and any realized aggregate and idiosyncratic state in such a way that the system composed of the
last two equations in \eqref{z2-no-lm} admits a solution $(\q(c),\qq(c))\in\R^2$ and this solution is
such that $c+\q(c)+\qq(c)$ exactly matches the entering wealth~$w$. The existence of such an assignment
ensures both: the value function remains finite (consumption is always strictly positive) and the
KKT conditions hold. As the system \eqref{z2-no-lm} depends on the time period $t$, the aggregate
state $(x,F)$, and the private state $u$, then so does also the solution $(\q(c),\qq(c))$. In~most
cases this dependence needs to be emphasized in the notation and
we shall often write $\q_{t,\tts x,\tts F,\tts u}(c)$ and $\qq_{t,\tts x,\tts F,\tts u}(c)$.
Of course, these objects depend
also on the selection of average installed capital, risk-free rate, and population transport mappings. 

The matter to address next is the compatibility of the optimal private
allocations. Let us suppose that, with a given choice for the initial population 
distribution, for the collection of transport mappings, and for the assignment of average 
installed capital and interest to every period and every aggregate state of the economy, all households
are able to solve their private savings problems with a unique (individual)
optimal allocation along every
realized path of the productivity state and the private employment state.
In general, such allocations have no reason to be consistent in that: (a)~exercising all optimal
private policies may generate transport mappings that are different from the given (and assumed by
all agents) ones, (b)~the
averages of all private capital investments may differ from the given
(and assumed by all agents) ones, and (c)~the given
interest rates may generate
aggregate demands for borrowing and lending that do not match.
An ``equilibrium'' is simply an arrangement in which no mismatch of type (a), or (b), or (c) in any
period and in any state occurs. The precise definition is the following.
%This is the matter
%addressed in the rest of this section.


%%- -%##%--{No: 2.10}--%%
\begin{nit}{Global general equilibrium}\label{def-equil}
In the context of the economy introduced above, global general equilibrium, or simply
equilibrium, is given by:
\vskip 3pt plus1.0pt

%\noindent
(1)~an initial population distribution $F_0\in\DB^\EEE$ and a collection of transport mappings
{\abovedisplayskip=5pt plus 4.5pt minus 1.5pt\belowdisplayskip=5pt plus 4.5pt minus 1.5pt\belowdisplayshortskip=3pt plus 4.5pt minus 1.5pt
$$
\T_{t,x}^y\colon \DB^\EEE \mapsto \DB^\EE\,,\quad x,y\in\XXX\,,\ \ 0\le t<T\,;
$$
}%

%\noindent
(2)~a collection of mappings $K_t\colon\XXX\times\DB^\EEE \mapsto\Rpp$ and
$r_t\colon\XXX\times\DB^\EEE \mapsto\lrbkt -1,\infty\,\rlbkt $, $0\le t<T$,
\vskip 2pt plus 0.5pt

\noindent
all chosen so that, for any realized path of the productivity state
$(x_t\in\XXX)_{0\tts\le\tts t\tts\le \tts T}$, the realized population distribution in period~$t$
is exactly $F_0$ if $t=0$ and exactly
{\abovedisplayskip=8pt plus 1.5pt minus 4.5pt\belowdisplayskip=8pt plus 4.5pt minus 1.5pt\belowdisplayshortskip=3pt plus 4.5pt minus 1.5pt
$$
F_t\df \bigl(\T_{t-1,x_{t-1}}^{x_t}\circ \T_{t-2,x_{t-2}}^{x_{t-1}}\circ \cdots\circ \T_{0,x_{0}}^{x_{1}}\bigr)(F_0)
\quad\text{if \ $0<t\le T$}\,,
$$
}%
the average installed capital is exactly $K_t(x_t,F_t)$ and the
average demand for the risk free private lending instrument is
exactly~$0$ in all periods $0\le t<T$,
provided that all households choose their savings policies by way of solving for their private KKT
conditions with population distribution $F_t$, with installed capital $K_t(x_t,F_t)$, with interest
$r_t(x_t,F_t)$, and with a collection of transport mappings
$\T_{t,x_t}^y$, $y\in\XXX$, and provided  this choice results
into a strictly positive consummation for every household at all times
and in all aggregate and private states.\qed
\end{nit}


The main step in the calculation of the global general equilibrium is to establish the connections
across time between the population distributions
${(\csd_{t})}_{0\tts\le\tts t\tts\le\tts T}$ that all private KKT conditions and
the notion of equilibrium dictate. This is the task we turn to next.

%%- -%##%--{No: 2.11}--%%
\begin{nit}{State transition mappings}\label{rnd-Monge}
Households that are in the same state of emp\-loyment, choose the same consumption level, and
experience the same shock in employment, would choose the same consumption level during the next
period, too. Let $\pfc_{t,x,F}^{ y,v}(u,c)$ denote the period $t+1$ consumption level
of any household that happens to be of type $(u,c)$ during period $t$, when the economy is in
state $(x,F)$, provided that during period $t+1$ the household faces
transition to employment state $v\in\EE$ and
the productivity state transitions to $y\in\XXX$.
The assignment
%%
{\abovedisplayskip=8pt plus 1.5pt minus 1.5pt\belowdisplayskip=8pt plus 1.5pt minus 1.5pt\belowdisplayshortskip=5pt plus 1.5pt minus 1.5pt
\begin{equation*}\label{rnd-Monge-a}%\tag{a}
\EEE\times\Rpp \ni (u,c) \leadsto (v,\pfc_{t,x,F}^{ y,v}(u,c))\in \EEE\times\Rpp
\end{equation*}
}%
is analogous to the Monge assignment in the classical transportation of mass problem,
except that it is not pure, in that it depends on the
idiosyncratic shock in employment, which differs across the population of households that are of
type $(u,c)$. In~addition to being random in that sense, this assignment takes place in the random environment
determined by the transition in productivity from $x$ to $y$~-- and we stress the dependence on
both $x$ and $y$.
In~what follows the mappings $\pfc_{t,x,F}^{ y,v}(\cdot,\cdot)$
are referred to as state transition mappings, or simply as transitions.
Accordingly, the mappings
$\pfc_{t,x,F}^{ y,v}(u,\cdot)$ are referred to as conditional transitions, or employment-specific
transitions. The rôle  the mappings $\pfc_{t,x,F}^{ y,v}(\cdot,\cdot)$ 
play in the model developed here
may seem similar to that of the familiar flux in fluid mechanics,
but there is also a fundamental difference (the very reason for using the term ``transitions'' instead of
``flux'' or ``flow''),
in  that
$(v,\pfc_{t,x,F}^{ y,v}(u,c))$ is not the next ``location'' of ``particle'' $(u,c)$; rather,
$(v,\pfc_{t,x,F}^{ y,v}(u,c))$ is the type that certain particles of type $(u,c)$ turn into when
time changes from~$t$ to $t+1$.
Furthermore, knowledge about the transition mappings is not sufficient to restore the
trajectory in the state space followed by a given household. Indeed, even if it is known that a
particular household is of type $(u,c)$, one must know the next period employment $v$ of
that same household in
order to identify its next period type as $(v,\pfc_{t,x,F}^{ y,v}(u,c))$.
We~stress that the individual
trajectories followed by the states of the households, whether in the range of wealth or
consumption, never enter the model developed in this \chptr.
While the general strategy in the domain of fluid mechanics
and MFG is to first derive the individual trajectories and then derive the flow of probabilities
along those, the strategy adopted here is to go directly to the distribution transfer and ignore
the individual paths altogether. Naturally, the state transitions drive the
transport of the population (treated as a probability measure on $\EEE\times\Rpp$)~-- see \ref{main-q}
below~-- but this transport is very different in nature from the flow of probabilities along a given family of
trajectories (recall the Lagrangian formulation of MFG in [\cited{BB00}-\cited{BCS17}] and
the relaxed MFG equilibrium in~\cite{CanCap18}).%
\qed
\end{nit}


Basic intuition
suggests that the mappings
$\pfc_{t,x,F}^{ y,v}(u,\cdot)$ must be increasing,%%%%%%%%%%%%
%%%%%%%%%%%%%%%%%%%%%%%%%%%%%
\footnote{{}An agent who consumes at least as much as another agent during the present period will
consume at least as much during the next period as well, if both agents experience the same shock in their
employment status.}
%%%%%%%%%%%%%%%%%%%%%%%%%%%%%
and we shall
seek equilibria in which these mappings are also continuous. %%%%%%%%%%%%%%%%%%%%%%%
%%%%%%%%%%%%%%%%%%%%%%%%%%%%%%%%%%%%%%%%%%%
%\footnote{{}Conceptually, right continuity, in addition to monotonicity (not necessarily strict), of the mappings
%$\pfc_{t,x,F}^{ y,v}(u,\cdot)$ is sufficient for what we need.} 
%%%%%%%%%%%%%%%%%%%%%%%%%%%%%%%%%%%%%%%%%%%
Consequently, all functions
$\pfc_{t,x,F}^{ y,v}(u,\cdot)$ have inverses given~by
%%- -%@@%--{Eq: 2.8}--%%
{\abovedisplayskip=5pt plus 1.5pt minus 1.5pt\belowdisplayskip=5pt plus 1.5pt minus 1.5pt\belowdisplayshortskip=3pt plus 1.5pt minus 1.5pt
\begin{equation}\label{inversion}
\Rpp\ni \a \leadsto \hat\pfc_{t,x,F}^{ y,v}(u,\a)\df\inf\{c\in\Rp\colon \pfc_{t,x,F}^{ y,v}(u,c)> \a\}\,.
\end{equation}}%
The postulated features of $\pfc_{t,x,\csd}^{ y,v}(u,\cdot)$ guarantee that
$c\le\hat\TT_{t,x,F}^{ y,v}(u,\a)$ and  $\pfc_{t,x,F}^{ y,v}(u,c)\le \a$ are equivalent relations.
Intuitively, the state transition mappings $\pfc_{t,x,F}^{ y,v}(\cdot,\cdot)$ govern the transport of the
population from period $t$ to $t+1$ and the next proposition makes this feature precise.

%%
%%- -%##%--{No: 2.12}--%%
\begin{nit}{Proposition}\label{main-q}
The transport mappings $\T_{t,x}^y\colon\DB^\EEE\mapsto\DB^\EEE$
obtain from the state transition mappings introduced above according to the rule
{\abovedisplayskip=8pt plus 2pt minus 2pt\belowdisplayskip=8pt plus 2pt minus 2pt\belowdisplayshortskip=3pt plus 1.5pt minus 1.5pt
\begin{equation*}\tag{${\text{d}}_t$}\label{zze666}    
%\begin{multlined}
\T_{t,x}^y(F)^{v}(\a) = \sum\nolimits_{\,u\ts\in\ts\EEE} 
{\gp_x(u) P_{x, y}(u, v)\over \gp_ y( v)} 
\csd^{u}(\,\hat\pfc_{t,x,F}^{ y,v}(u,\a))\,,\ \ \a\in\Rpp\,,\ v\in\EEE\,,\ F\in\DB^\EEE
%\end{multlined} 
\end{equation*}}%
for all $t<T$ and all $x,y\in\XXX$.\qed
\end{nit}

Most of what follows stands on the last result, the justification of which
is straightforward:
%
%
%%- -%##%--{No: 2.13}--%%
\begin{nit}{Proof of \ref{main-q}}
Adopt the terminology, the notation, and the results from \ref{weighing}.
Let $v\in\EEE$ be fixed. Then $\gp_ y( v)\T_{t,x}^y(F)^{v}(\a)$ is the relative (with respect to
the entire population) weight of the collection of agents who happen to be in employment state $v$
during period $t+1$ and happen to choose consumption level (during that same period) that is not
strictly larger than $\a\in\Rpp$. Consider the collection $B_{u,\tts v}$ of agents who happen to
be in employment state~$u$ during period~$t$ and transition to state $v$ during period~$t+1$.
The period-$(t+1)$ consumption level of an agent from the set $B_{u,\tts v}$ would not exceed
$\a$ only if and only if the period-$t$ consumption level, $c$, of that same agent is such that
$\pfc_{t,x,\csd}^{ y,v}(u,c)\le \a$,
which property is the same as $c\le\hat\pfc_{t,x,\csd}^{ y,v}(u,\a)$, i.e., the agent must belong
to the set $E_{u,v}\bigl(\hat\pfc_{t,x,\csd}^{ y,v}(u,\a)\bigr)$, which has relative weight
(against the entire population) of $\gp_x(u) P_{x, y}(u, v)\csd^{u}(\,\hat\pfc_{t,x,F}^{
y,v}(u,\a))$.
Observing that  the finite union of disjoint set
$\cup_{u\tts\in\tts\EEE}E_{u,v}\bigl(\hat\pfc_{t,x,\csd}^{ y,v}(u,\a)\bigr)$
is nothing but the collection of agents who happen to be in employment state $v$
during period $t+1$ and choose consumption level that is not strictly larger than $\a\in\Rpp$ completes
the proof.\qed
\end{nit}



%%
%%- -%##%--{No: 2.14}--%%
\begin{nit}{Remark}\label{n2}
The~rôle that equation (\ref{zze666}) plays in the present study is similar to that of the master
equation in MFG, or the Kolmogorov forward equation in the classical approach to heterogeneous models. However, its
structure and intrinsic nature differ from either of these two tech\-ni\-ques in at least
these aspects: The first one is the structure of the state transition mappings, which is deri\-ved
below and reflects the backward-forward connections noted earlier in the \chptr.
In~particular, as we are about to see in \ref{self-dep}, the state transition mappings in the right side
of \eqref{zze666} must depend on the left side (the transport must be ``self-aware'').
In~addi\-tion, the transport
encoded into \eqref{zze666} acts in the random environment of the transition in the productivity
state, not just in the random environment of the productivity state alone.
Indeed, the right side of \eqref{zze666} depends on both the present and future states $x$ and
$y$~-- not on $x$ alone, or on $y$ alone. 
Moreover, the very rule that
transports the distribution~$\csd$ depends on~$\csd$ (the symbol $\csd$ appears twice in the right
side). 
This feature may appear reminiscent to a Kolmogorov forward
equation with coefficients that depend (explicitly) on the
distribution that the equation drives, but we stress that, notwithstanding the self-awareness of
the transport, the dependence on the second appearance of $\csd$ in the right
side of \eqref{zze666} is only implicit and comes from solving (simultaneously) all private KKT
conditions, together with the collective market clearing requirement.
Most important, (\ref{zze666}) can only be solved
simultaneously with the other first order and market clearing conditions,
which, in turn, can only be solved simultaneously with \eqref{zze666}~-- see \ref{cross-sys} below.
This simultaneity is unavoidable and is
the main difficulty to overcome.\qed
\end{nit}

%%
\iffalse


\begin{nit}{Remark}\label{x-Markov}
If one insists on interpreting equation \ref{main-q}-\eqref{zze666}
as the transition rule for the distribution of a
particular Markov
chain, then the state of that Markov chain must be the list $(x,y,u,c)$, i.e., present productivity
state, future productivity state, present employment state, and present consumption level, with the
understanding that transition from $(x,y,u,c)$ in period~$t$ to $(x',y',u',c')$ in period $t+1$
occurs with probability $0$ whenever $y\neq x'$. However, because of the self-awareness of the
transition and because of its dependence on the distribution, such an interpretation may not be very
helpful~-- except, perhaps, for illustrating the difficulty in pursuing a continuous time interpretation
of the structures in \ref{main-q}-\eqref{zze666}.\qed 
\end{nit}


\fi


%%%%
In order to address the backward-and-forward structure noted earlier in a more practical fashion,
we must find a way to mimic the approach proposed in \cite{DL12}. The idea is to break the large 
system across all periods and all aggregate and idiosyncratic states into smaller ones
that can be chained into a computable backward induction program. 

%%- -%##%--{No: 2.15}--%%
\begin{nit}{The local bi-period master system}\label{cross-sys}
Given any period $0\le t<T$, and any aggregate state $(x,\csd)\in\XXX\times\DB^\EEE$
and idiosyncratic (employment) state $u\in\EEE$ associated with that period,
define the system (parameterized by $x$, $\csd$ and $u$): 
%%
{\abovedisplayskip=8pt plus 1.5pt minus 1.5pt\belowdisplayskip=8pt plus 1.5pt minus 1.5pt\belowdisplayshortskip=8pt plus 1.5pt minus 1.5pt
\begin{equation*}\tag{$\text{n}_t$}\label{zze5a} 
\begin{gathered}
1=\bigl(1+r_{t}(x,\csd)\bigr)
\b\sum\nolimits_{ y\ts\in\ts\XXX,\,v\ts\in\ts\EEE}{{\Up}\bigl(\pfc_{t,x,\csd}^{ y,v}(u,c)/c\bigr)} Q(x, y)P_{x, y}(u,v)\,,\\
1=\b\sum\nolimits_{ y\ts\in\ts\XXX,\,v\ts\in\ts\EEE}{{\Up}\bigl(\pfc_{t,x,\csd}^{y,v}(u,c)/c\bigr)}
\bigl(\rho_{ y}(K_{t}(x,\csd))+1-\dd\bigr) Q(x, y)P_{x, y}(u, v)\,,
\end{gathered}
\end{equation*}
}%
%%
{\abovedisplayskip=8pt plus 1.5pt minus 1.5pt\belowdisplayskip=8pt plus 1.5pt minus 1.5pt\belowdisplayshortskip=8pt plus 1.5pt minus 1.5pt
\begin{equation*}\tag{$\text{e}_{t+1}$}\label{zze5xb} 
\begin{gathered}
\begin{multlined}
(1+r_{t}(x,\csd)){\q_{t,x,\csd,u}(c)} + 
\bigl(\rho_ y(K_{t}(x,\csd))+1-\dd\bigr){\qq_{t,x,\csd,u}(c)}
+\ee_ y(K_{t}(x,\csd))v \qquad\qquad\\
\qquad ={\pfc_{t,x,\csd}^{ y,v}(u,c)}
+ {\q_{t+1, y,\T_{t,x}^y(\csd),v}\bigl(\pfc_{t,x,\csd}^{ y,v}(u,c)\bigr)}
+ {\qq_{t+1, y,\T_{t,x}^y(\csd),v}\bigl(\pfc_{t,x,\csd}^{ y,v}(u,c)\bigr)}\,,\\
\text{for all }\   y\in\XXX\,,\ v \in\EEE\,,
\end{multlined} 
\end{gathered}
\end{equation*}
}%
%%
{\abovedisplayskip=8pt plus 1.5pt minus 1.5pt\belowdisplayskip=8pt plus 1.5pt minus 1.5pt\belowdisplayshortskip=8pt plus 1.5pt minus 1.5pt
\begin{equation*}\tag{$\text{m}_{t}$}\label{ze9}
\begin{gathered}
\sum\nolimits_{u\ts\in\ts\EEE}\gp_x(u)\int_0^\infty{\q_{t,x,\csd,u}(c)}\d \csd^u(c)=0\,,\\
\sum\nolimits_{u\ts\in\ts\EEE}\gp_x(u)\int_0^\infty{\qq_{t,x,\csd,u}(c)}\d \csd^u(c) =
K_{t}(x,\csd)\,,
\end{gathered}
\end{equation*}
}%
%% 
in which \eqref{zze5a}  and \eqref{zze5xb} are understood as identities between functions of
$c\in\Rpp$ and the distributions $\T_{t,x}^y(\csd)\in\DB^\EEE$, $y\in\XXX$,
are given by (a replica from \ref{main-q})
{\abovedisplayskip=8pt plus 1.5pt minus 1.5pt\belowdisplayskip=8pt plus 1.5pt minus 1.5pt\belowdisplayshortskip=8pt plus 1.5pt minus 1.5pt
\begin{equation*}\label{zze667}\tag{${\text{d}}_t$} 
\T_{t,x}^y(F)^{v}(\a) = \sum\nolimits_{\,u\ts\in\ts\EEE} 
{\gp_x(u) P_{x, y}(u, v)\over \gp_ y( v)} 
\csd^{u}(\,\hat\pfc_{t,x,F}^{ y,v}(u,\a))\,,\ \ \a\in\Rpp\,,\ v\in\EEE\,.
\end{equation*}
}%
Let $\M_t$, for $0\le t<T$, stand for the collection of
equations $\bigl\{\text{\eqref{zze5a}, \eqref{zze5xb}, \eqref{ze9}, \eqref{zze667}}\bigr\}$.
We call $\M_t$ the local (in time) bi-period master system, or simply the local master system.
Solving for the general equilibrium comes down to solving the global master system
$\{\M_t\colon 0\le t<T\}\cup\text{\eqref{zze5ab}}$, where \eqref{zze5ab} is the collection of all
period~$t=0$ balanced budget conditions, which is described next.
Since at time $t=0$ all households share the same entering wealth, 
households that  happen to be in employment state $u\in\EEE$ are identical and thus choose the same
consumption level $\bar c_{u}$.  In~particular, the period $t=0$ population distribution
$\csd_0\in\DB^\EEE$ has the form $(\csd_0)^u(c)=0$ for $c< \bar c_{u}$ and
$(\csd_0)^u(c)=1$ for $c\ge \bar c_{u}$. Thus, there are $\abs{\EEE}$ balanced budget
equations attached to period $t=0$, namely, 
%%
{\abovedisplayskip=5pt plus 1.5pt minus 1.5pt\belowdisplayskip=5pt plus 1.5pt minus 1.5pt\belowdisplayshortskip=3pt plus 1.5pt minus 1.5pt
\begin{equation*}\tag{$\text{e}_0$}\label{zze5ab}
\bar c_{u} + {\q_{0,x,\csd_0,u}(\bar c_{u})} + {\qq_{0,x,\csd_0,u}(\bar c_{u})}
= \bigl(\rho_x(K_{-1}) + 1-\dd\bigr)K_{-1} + \ee_x(K_{-1})u\quad\text{for all }\ u\in\EEE\,,
\end{equation*}
}%
which are to be solved for the (same number of) unknowns $\bar c_u$, $u\in\EEE$.

The reason for organizing all conditions that define the equilibrium in such a way that the
master system $\M_t$ includes some equations associated with period $t$ and other equations associated
with period $t+1$ is to make it possible to seek a solution by solving, sequentially, the systems
$\M_{T-1}$, $\ldots$, $\M_0$, \eqref{zze5ab}. This process requires organizing and connecting accordingly
the unknowns that are being solved for, while keeping in mind that $\M_t$ is a system parameterized
by $x\in\XXX$, $F\in\DB^\EEE$, and $u\in\EEE$. 
To be a bit more precise, when solving  $\M_t$ the collection of functions 
$\q_{t+1, y,\tilde\csd,v}\phd$ and
$\qq_{t+1,y,\tilde\csd,v}\phd$, for all possible choices of $y\in\XXX$, $\tilde\csd\in\DB^\EEE$ and $v\in\EEE$, are
assumed given, while the unknowns are the functions (since $\M_t$ is parameterized by~$x$, $F$,
and~$u$, so are also the unknowns)
{\abovedisplayskip=5pt plus 1.5pt minus 1.5pt\belowdisplayskip=5pt plus 1.5pt minus 1.5pt\belowdisplayshortskip=3pt plus 1.5pt minus 1.5pt
$$
\q_{t,x,\csd,u}\phd\,, \ \ \qq_{t,x,\csd,u}\phd\ \ \text{and}\ \   
\pfc_{t,x,\csd}^{y,v}(u,\cdot) \ \ \text{for all } y\in\XXX\,, \  v\in\EEE\,.
$$
}%
We stress that $\q_{t,x,\csd,u}\phd$ and  $\qq_{t,x,\csd,u}\phd$ map period-$t$ consumption into
period-$t$ asset holdings, while $\pfc_{t,x,\csd}^{y,v}(u,\cdot)$ map period-$t$ consumption into
period $t+1$ consumption, i.e., some of the unknowns are associated with  period $t$, while other
unknowns are associated with period $t+1$. Since every local (in time) system $\M_t$ has components
that are forward-looking, the sequence
$\M_{T-1}$, $\ldots$, $\M_0$, cannot be interpreted as a ``backward equation''~--- at least not
in the way this term is commonly understood. 
In~addition, it does not appear possible to split $\M_{T-1}$, $\ldots$, $\M_0$, into two
sequences that have the meaning of backward and forward equations, as in a coupled
MFG system, for example. The main reason for this difficulty appears to be the price-agreement
condition \eqref{zze5a}, which imposes a special interdependence between the agents' private
optimal allocations and has no analog in the usual MFG formulation.\qed
\end{nit}
 
The recursive program for solving the global system
$\{\M_t\colon 0\le t<T\}\cup\text{\eqref{zze5ab}}$
from \ref{cross-sys} now suggests itself. 
At every step (associated with period $t$) the program must compute the demand
functions
%
{\abovedisplayskip=5pt plus 1.5pt minus 1.5pt\belowdisplayskip=5pt plus 1.5pt minus 1.5pt\belowdisplayshortskip=3pt plus 1.5pt minus 1.5pt
$$  
\Rpp\ni c \leadsto \q_{t,x,\csd,u}(c)\,,\qq_{t,x,\csd,u}(c)\in\R
\quad \text{for all }\ x\in\XXX\,,\ \csd\in\DB^\EEE\,,\ u\in\EEE\,,
$$
}%
and the transition mappings
%
{\abovedisplayskip=5pt plus 1.5pt minus 1.5pt\belowdisplayskip=5pt plus 1.5pt minus 1.5pt\belowdisplayshortskip=3pt plus 1.5pt minus 1.5pt
$$
\EEE\times\Rpp\ni (u,c) \leadsto \pfc_{t, x, \csd}^{y, v}(u,c)\in\Rpp
\quad \text{for all }\ x,y\in\XXX\,,\ \csd\in\DB^\EEE\,,\ v\in\EEE\,,
$$
}%
while taking the demand functions
%
{\abovedisplayskip=5pt plus 1.5pt minus 1.5pt\belowdisplayskip=5pt plus 1.5pt minus 1.5pt\belowdisplayshortskip=3pt plus 1.5pt minus 1.5pt
$$
\Rpp\ni \tilde c \leadsto \q_{t+1,\tts y,\tts \tilde\csd,\tts v}(\tilde c)\,,\ \qq_{t+1,\tts y,\tts \tilde\csd,\tts v}(\tilde c)\,,
\quad \text{for all }\ y\in\XXX\,,\ \tilde\csd\in\DB^\EEE\,,\ v\in\EEE\,,
$$
}%
as given, i.e., already computed during the previous step (associated with period $t+1$) if $t<T-1$
or taken to be $0$ if $t=T-1$.
What complicates this plan is the following salient feature.

%%- -%##%--{No: 2.16}--%%
\begin{nit}{Self-aware transport}\label{self-dep}
Since the transition mappings $\pfc_{t, x, \csd}^{y, v}(\cdot,\cdot)$ must obey \eqref{zze5xb},
they must depend (through the period-$(t+1)$ portfolios in the right side)
not only on the \hbox{period-$t$} distribution $\csd$,
but also on its transport, $\T_{t,x}^y(\csd)$, to period~$t+1$. In particular, the right side of the
transport equation \eqref{zze667} depends on the left side; that is to say,
the mechanism that transports the population 
(i.e., transfers its distribution)
from one period to the next must depend on the result from
the transport. We stress that this feature is due to connections rooted in economics, not
in mathematics. Indeed, the transport of the population  is the outcome from a multitude of
individual decisions, every one of which is based on rational expectations about the future endogenous
state, i.e., based on rational expectations about the outcome from the multitude of individual
decisions. The notion of competitive
equilibrium then requires that the outcome from the transport coincides
with the anticipated one, i.e., the one the transporting mechanism assumes.
We stress that this coincidence must hold in every possible realization of the future productivity state.\qed   
\end{nit}

The more detailed description of the strategy just
outlined is the following.


%%- -%##%--{No: 2.17}--%%
\begin{nit}{General metaprogram}\label{main-proc}
{\it Initial Backward Step:} Set $t=T-1$ and for every $x\in\XXX$ do:
{\parskip=0pt\nobreak
For every choice of the distribution (state variable)
$\csd\in\DB^\EEE$ do:

{ (1)}~Make an ansatz choice for the values $K_{t}(x,\csd)$ and $r_{t}(x,\csd)$. Go to (2).

{ (2)}~For every %fixed
$(u,c)\in\EEE\times\Rpp$ solve (\ref{zze5a}-\ref{zze5xb}) with
$\q_{t+1, y, \T_{t,x}^y(\csd), v}\equiv 0$ and  
$\qq_{t+1, y, \T_{t,x}^y(\csd), v}\equiv\nobreak 0$
(total of $\abs{\XXX}\times\abs{\EEE}+2$ equations) for the (same number of)
unknowns:
{\abovedisplayskip=5pt plus 1.5pt minus 1.5pt\belowdisplayskip=5pt plus 1.5pt minus 1.5pt\belowdisplayshortskip=3pt plus 1.5pt minus 1.5pt
$$
\{\pfc_{t, x, \csd}^{ y,v}(u,c)\colon  y\in\XXX,\,v \in\EEE\}\,,\quad
\q_{t, x, \csd, u}(c)\,, \quad 
\qq_{t, x, \csd, u}(c)\,.
$$
}%

{ (3)}~Test the market clearing conditions (\ref{ze9}).
If at least one of these conditions fails by more than some prescribed threshold,
go back to (2) with appropriately revised values for $K_{t}(x,\csd)$ and $r_{t}(x,\csd)$; otherwise stop
and record (i.e., accept) the most recently computed scalars $K_{t}(x,\csd)$ and $r_{t}(x,\csd)$ and functions
$\q_{t,\tts x,\tts \csd,\tts u}(\cdot)$, $\qq_{t,\tts x,\tts \csd,\tts u}(\cdot)$ and
$\pfc_{t,\tts x,\tts \csd}^{ y,v}(u,\cdot)$, for all $y\in\XXX$ and $v\in\EEE$. Proceed to
the next step.
}

%\noindent
{\it Generic Backward Step:}  If $t-1<0$, go to the final backward step below; else set $t=t-1$ and
for every $x\in\XXX$ do:

For every choice of the distribution (state variable) $\csd\in\DB^\EEE$ do:

{(1)}~Set $\csdp =\csd$ for every $y\in\XXX$
(the next period distribution, i.e., state variable, is initially guessed to be the
same as the one in the present period, irrespective of the realized future productivity state~$y$).


{ (2)}~Set $K_{t}(x,\csd)=K_{t+1}(x,\csd)$ and $r_{t}(x,\csd)=r_{t+1}(x,\csd)$ (initial guess taken
from the previous iteration).

{ (3)}~For every fixed
$(u,c)\in\EEE\times\Rpp$ solve (\ref{zze5a}-\ref{zze5xb})
with $\T_{t,x}^y(\csd)$ replaced by $\csdp$ (total of $\abs{\XXX}\times\abs{\EEE}+2$ equations)
for the (same number of) unknowns  $\{\pfc_{t,x,\csd}^{ y,v}(u,c)\colon  y\in\XXX,\,v \in\EEE\}$,
$\q_{t,x,\csd,u}(c)$ and $\qq_{t,x,\csd,u}(c)$. Go to (4).

{ (4)}~Test the market clearing conditions (see (\ref{ze9}))
{\abovedisplayskip=7pt plus 1.5pt minus 1.5pt\belowdisplayskip=7pt plus 1.5pt minus 1.5pt\belowdisplayshortskip=8pt plus 1.5pt minus 1.5pt
\begin{equation*}
\sum\nolimits_{u\ts\in\ts\EEE}\gp_x(u)\int_0^\infty{\q_{t,x,\csd,u}(c)}\d \csd^u(c)=0
\ \text{and}\ 
\sum\nolimits_{u\ts\in\ts\EEE}\gp_x(u)\int_0^\infty{\qq_{t,x,\csd,u}(c)}\d \csd^u(c) = K_{t}(x,\csd)\,.
\end{equation*}}%
If at least one of these conditions fails by more than some prescribed threshold, go back to (3)
with appropriately revised values for $K_{t}(x,\csd)$ and $r_{t}(x,\csd)$; otherwise, proceed to (5).

{ (5)}~With the computed functions $\pfc_{t,x,\csd}^{ y,v}(u,\cdot)$,
which now depend on the choice of $\csdp$, $y\in\XXX$, compute the distributions
$\csds\in\DB^\EEE$, $y\in\XXX$, as (see \eqref{zze667})
{\abovedisplayskip=5pt plus 1.5pt minus 1.5pt\belowdisplayskip=5pt plus 1.5pt minus 1.5pt\belowdisplayshortskip=8pt plus 1.5pt minus 1.5pt
\[
(\pst\csd_{y})^v(\a) = \sum\nolimits_{\,v\ts\in\ts\EEE} 
{\gp_x(u) P_{x, y}(u, v)\over \gp_ y( v)} 
\csd^{u}(\hat\pfc_{t,x,\csd}^{ y,v}(u,\a))\,,\quad%\text{for all }\ 
\a\in\Rpp\,,\ \ v \in\EEE\,.
\]}%
If the largest Kolmogorov-Smirnov distance between $(\pst\csd_{y})^v\phd$ and the guess
$(\pdg\csd_{y})^v \phd$, for the various choices of $y\in\XXX$ and $v \in\EEE$,
is not acceptably close to~$0$, set $\csdp=\csds$ and go back
to (3) without changing $K_{t}(x,\csd)$ and $r_{t}(x,\csd)$; otherwise, stop
and record (i.e., accept) the most recently obtained scalars $K_{t}(x,\csd)$ and $r_{t}(x,\csd)$ and functions
$\q_{t,x,\csd,u}\phd$, $\qq_{t,x,\csd,u}\phd$ and
$\pfc_{t,x,\csd}^{ y,v}(u,\cdot)$, $y\in\XXX$, $v\in\EEE$. Go to the beginning of the generic
backward step.

{\it Final Backward Step:}~For every $x\in\XXX$ do:

For every $u\in\EEE$ determine the period $t=0$
consumption level $\bar c_u$ for all households in employment state~$u$ 
(all households in employment category $u$ are identical in period $t=0$ and consume the same
amount) by solving the following system of $\abs{\EEE}$ equations (see (\ref{zze5ab}))
{\abovedisplayskip=7pt plus 1.5pt minus 1.5pt\belowdisplayskip=7pt plus 1.5pt minus 1.5pt\belowdisplayshortskip=8pt plus 1.5pt minus 1.5pt
\[
\bar c_u + {\q_{0,x,u,{\csd_{0}}}(\bar c_u)} + {\qq_{0,x,u,{\csd_{0}}}(\bar c_u)} = \bigl(\rho_x(K_{-1})
+1-\dd\bigr)K_{-1} + u\,\ee_x(K_{-1})\,,\quad u\in\EEE\,,
\]}%
in which $\csd_{0}\in\DB^\EEE$
is given by $(\csd_{0})^u(c)=1$ if $c\ge \bar c_u$ and $(\csd_{0})^u(c)=0$ if
$c< \bar c_u$, \ $u\in\EEE\,$.

{\it Initial Forward Step:}~In period $t=0$ the initial productivity state $x\in\XXX$ is revealed and so
is also the (idiosyncratic) employment state of every household.
As all households in employment category $u\in\EEE$ have the same income in period $t=0$ and are faced
with the same uncertain future, they are identical and adopt the same
consumption plan $\bar c_u$, calculated during the final backward step.
Define the period $t=0$ population distribution
$\csd_{0}\in\DB^\EEE$ as the corresponding list of Heaviside step functions ($\abs{\EEE}$ in number).
As all quantities
$K_{0}(x,\csd)$,  $r_{0}(x,\csd)$, $\q_{0,x,\csd,u}(c)$ and $\qq_{0,x,\csd,u}(c)$ 
have been precomputed for every $\csd\in\DB^\EEE$ and $c\in\Rpp$, the period $t=0$ average productive
capital $K_{0}(x,\csd_{0})$ is available, and so is also the period $t=0$ exiting portfolio,
$\{\q_{0,x,\csd_{0},u}(\bar c_u),\qq_{0,x,\csd_{0,x},u}(\bar c_u)\}$, for all
households in employment state $u\in\EEE$.

{\it Generic Forward Step:} The economy exits period $(t-1)$ from productivity state $x\in\XXX$
with 
population distribution $\csd_{t-1}$ and in period $t$ enters a new
productivity state $ y\in\XXX$.
As~all $(t-\nobreak1)$-to-$t$  transition mappings
$(u,c) \leadsto \pfc_{t-1,x,\tilde\csd}^{ y,v}(u,c)$ are available from the backward steps for all
\smash{$\tilde\csd\in\DB^\EEE$}, the period $t$ consumption levels of all
households become known:
a period $(t-1)$ household of type $(u,c)\in\EEE\times\Rpp$ 
that changes employment from $u$ to $v$ becomes, during period~$t$, household of type 
$(v,\tilde c)$ with $\tilde c = \pfc_{t-1,x,\csd_{t-1}}^{y,v}(u,c)$.
The period $t$  population distribution is then given by
{\abovedisplayskip=4pt plus 1.0pt minus 1.5pt\belowdisplayskip=4pt plus 1.0pt minus 1.5pt\belowdisplayshortskip=4pt plus 1.0pt minus 1.5pt
\begin{equation*}
(\csd_{t})^v(\a) = \sum\nolimits_{\,u\,\in\,\EEE} 
{\gp_x(u) P_{x, y}(u, v)\over \gp_ y( v)}\,
(\csd_{t-1})^u(\hat\pfc_{t-1,x,\csd_{t-1}}^{ y,v}(u,\a))\,,\quad%\text{for all }\  
\a\in\Rpp\,,\ \ v \in\EEE\,.
\end{equation*}}%
As all quantities $K_{t}(y,\csd)$, $r_{t}(y,\csd)$, $\q_{t, y,\csd,v}(\tilde c)$
and $\qq_{t, y,\csd,v}(\tilde c)$, assumed to be $0$ if $t=T$,
have been precomputed during the backward steps for every period-$t$
population distribution  $\csd\in\DB^\EEE$ and all individual consumption levels $\tilde c$,
they are meaningful with $\csd=\csd_{t}$. The period~$t$ average installed
productive capital is  $K_{t}(y,\csd_{t})$, the agreed upon interest is $r_{t}(y,\csd_{t})$,
and the period-$t$ exiting portfolio of any household of type $(v,\tilde c)$ is
$\{\q_{t, y,\csd_{t},v}(\tilde c),\,\qq_{t, y,\csd_{t},v}(\tilde c)\}$.\qed
\end{nit} 


%%- -%##%--{No: 2.18}--%% 
\begin{nit}{Endless loops warning and disclaimer}\label{endless-loop}
There are no theoretical results to guarantee that the iterations between steps (3) and (4) and (3)
and (5) converge, or to guarantee that step~(3) in the generic backward step is always feasible,
in that a numerical solution to the system exists generically.\qed
%A numerically acceptable general equilibrium can be declared to exist only if
%the program completes.\qed
\end{nit}





%%- -%##%--{No: 2.19}--%% 
\begin{nit}{Remark}\label{MFG-rem} 
The iterations between steps (2) and (5) in the generic backward step are meant
to ensure that, in every possible realization of the future productivity state,
the result from the transport of the population distribution coincides with the
one assumed by the transporting mechanism~-- see \ref{self-dep} above.
Indeed, before being solved for during stage (3)
in the generic backward step, the system (\ref{zze5a}-\ref{zze5xb}) is made to
depend on the guessed period-$(t+1)$
distributions $\csdp$, $ y\in\XXX$, which guesses are being iterated until they become
consistent with the structure of (\ref{zze667}).
We again stress that the transport 
equation (\ref{zze667}) is meaningful only in conjunction with the associated budget,
kernel, and market clearing conditions~-- not as a stand-alone equation. 
Note also that this
adjustment (coordination) is local in time, in that the program does not
move to the next period going backward (which is the previous period in real time)
until the correct transport from the current period is established~-- recall that the transport is
time dependent and may become time invariant only in the limit.\qed
\end{nit}





The metaprogram described in \ref{main-proc} differs from other similar procedures in a number of
key aspects.
Some were already noted in \ref{self-dep} and \ref{MFG-rem} above.
Another key aspect is that all backward steps involve the simultaneous computation
of future consumption and present demand.
Hence, while the procedure steps backward globally, every step has a forward looking sub-step.
Generally, such a program would be difficult to implement in concrete
models~-- especially given the finite time horizon setup~--
mainly due to the lack of an adequate computing technology for representing
general (nonlinear) functions on the space of distributions.
This is  a common problem for all heterogeneous agent models, in which %, in one way or another,
the population distribution is inevitably a state variable.
Nevertheless, there are important special cases where the program outlined in \ref{main-proc} can still be
carried out. One is the absence of aggregate shocks, in which case a stationary distribution of the
population becomes available~--see Sec.~\ref{sec:IOU} below. Another one is the possibility to
approximate the population distribution with the vector of its conditional mean values in the various
employment groups, or even just with the unconditional mean value across the entire population~--
see Sec.~\ref{sec:KS} below.


%%% chend556677


%%- -%%--{Sec: #3}--%%
%{Eq: 3.}   %%llabel
%{No: 3.}   %%nlabel


\section{Models Without Aggregate Risk}
\label{sec:IOU}\setcounter{paragraph}{0}
\def\inc{{A}}

\noindent 
In this section we revisit the benchmark Huggett economy borrowed from \cite{LjunSar00}  and
already reviewed in Sec.~\ref{sec:Intro}.
The enormous simplification that comes
from the removal of the
production function and the common shocks is that time-invariant distribution of the population
and time-invariant value for the interest rate become available. In the search for these two
objects, we now
restate the global master system $\{\M_t\colon 0\le t<T\}\cup\text{\eqref{zze5ab}}$
from \ref{cross-sys} 
with all wages $\ee_y(K_{t}(x,F))$ replaced by a
fixed value $\ee$, 
with all quantities $\qq$ set to~$0$ (no capital investment takes place),
with the second equations in \eqref{zze5a} and \eqref{ze9} removed, with all instances of $x$ and $y$
as sub/super-scripts removed, and with all transition probabilities $Q(x,y)$ set to 1.
The transition probability matrix~$P$, which governs the idiosyncratic
transitions in every individual employment state, admits a unique list of steady state
probabilities $\gp=(\gp(u)>0,\,u\in\nobreak\EEE)$, which we treat as a vector-row with 
$\gp\,P=\gp$ and with $\sum_{u\ts\in\ts\EEE}\gp(u)=1$. Assuming that all independent private Markov
chains have reached steady-state, the
average income in the cross-section of the population is fixed at
{\abovedisplayskip=5pt plus 1.5pt minus 1.5pt\belowdisplayskip=7pt plus 1.5pt minus 1.5pt\belowdisplayshortskip=5pt plus 1.5pt minus 1.5pt
\[
\inc=\ee\sum\nolimits_{u\ts\in\ts\EEE}{u} \, \gp(u)\,.%
\]}%
As we seek time-invariant equilibrium, we drop the subscript ``$t$'' throughout.
For technical reasons, instead of seeking the equilibrium interest $r$, we seek
the spot price, $B=\inc/(1+r)$, of a risk-free bond with face value $\inc$.
The relative risk aversion for all agents is the constant $R\ge1$.


%%- -%##%--{No: 3.1}--%%
\begin{nit}{Time-invariant equilibrium}\label{time-inv}
It consists of: {(1)}~a fixed scalar $B\in\R$;
{(2)}~a collection of continuous and non-decreasing functions
$\q_u\colon\Rpp\to\R$, $u\in\EEE$;
{(3)}~a collection of continuous and non-decreasing functions
$\pfc^v(u,\cdot)\colon\Rpp\to\Rpp$, $u,v\in\EEE$ with inverses $\hat\pfc^v(u,\cdot)$;
{(4)}~a collection of cumulative distribution functions $\csd^u\in\bbF$, $u\in\EEE$~---
all chosen so that the following four conditions (kernel, balanced budget, market clearing, and
self-aware transport)
are satisfied:
{\abovedisplayskip=5pt plus 1.5pt minus 1.5pt\belowdisplayskip=5pt plus 1.5pt minus 1.5pt\belowdisplayshortskip=5pt plus 1.5pt minus 1.5pt
\begin{gather}%
B=\b\,\sum\nolimits_{\,v\ts\in\ts\EEE} {\inc}\, \Bigl({c\over \pfc^v(u,c)}\Bigr)^{R} {P}(u,v)
\quad\text{for all}\  c\in\Rpp\ \text{and all}\ u\in\EEE\,;
\tag{$\text{n}$}\label{focfoc-a}\\ 
{\q_u(c)}{\inc}+{\ee\ts v}
={\pfc^v(u,c)}+{\q_v(\pfc^v(u,c))}B\quad\text{for all}\   c\in\Rpp\ \text{and all}\ u,v\in\EEE\,;
\tag{$\text{e}$}\label{focfoc-b}\\
\sum\nolimits_{\,u\ts\in\ts\EEE}\gp(u)\int_0^\infty{\q_u(c)}\d \csd^u (c) = 0\,;
\tag{$\text{m}$}\label{focfoc-c}\\
\hfil \csd^v(c) = \sum\nolimits_{u\ts\in\ts\EEE} 
{\gp(u){P}(u,v)\over \gp(v)}\, 
\csd^u\bigl(\hat \pfc^v(u,c)\bigr)\quad \text{for all}\  c\in\Rpp\ \text{and all}\ v\in\EEE\,.\qed
\tag{$\text{d}$}\label{focfoc-d}%\label{4iter-F}
\end{gather}}%
\end{nit}

As the balanced budget constraints in \ref{time-inv}-(\ref{focfoc-b}) obtain as
the limit with $t\to\infty$ in \ref{cross-sys}-(\ref{zze5xb}), those constraints give rise to the following 
iteration program (in what follows the token $\pst$ marks new values and the token $\pdg$ marks previously
computed values)

%%
%%- -%@@%--{Eq: 3.1}--%%
{\abovedisplayskip=5pt plus 1.5pt minus 1.5pt\belowdisplayskip=5pt plus 1.5pt minus 1.5pt\belowdisplayshortskip=5pt plus 1.5pt minus 1.5pt
\begin{equation}\label{4iter-q}
\pst{\q_u(c)}\,{\inc}+{\ee\,v} 
={\pst\pfc^v(u,c)} + \pdg\q_v(\pst\pfc^v(u,c))B\,,\quad c\in\Rpp\,,\ u,v\in\EEE\,.
\end{equation}}%
Since the functions
{\abovedisplayskip=5pt plus 1.5pt minus 1.5pt\belowdisplayskip=5pt plus 1.5pt minus 1.5pt\belowdisplayshortskip=5pt plus 1.5pt minus 1.5pt
\[
\Rpp\ni \pfc \leadsto  \pdg{{H}}_v(\pfc)\df \pfc +\pdg\q_v(\pfc)\,B\,,\quad v\in\EEE\,,
\]}%
are strictly increasing and continuous, they can be inverted in the usual way.
Letting $\pdg{\hat{H}}_v\phd$ denote the inverse of $\pdg{{H}}_v\phd$ gives 
$\pst\pfc^v(u,c)=\pdg{\hat{H}}_v\bigl(\pst\q_u(c)\,A+{\ee\ts v}\bigr)$~-- notice that
$\pst\pfc^v(u,\cdot)$ is constructed from both $\pdg\q_v\phd$ and $\pst\q_u\phd$.
As a result,
the functions $\pst\q_u\phd$, $u\in\EEE$, obtain implicitly from the relations
%%
%%- -%@@%--{Eq: 3.2}--%%
{\abovedisplayskip=5pt plus 1.5pt minus 1.5pt\belowdisplayskip=5pt plus 1.5pt minus 1.5pt\belowdisplayshortskip=5pt plus 1.5pt minus 1.5pt
\begin{equation}\label{reduced-k}
B=\beta\,\inc\,\sum\nolimits_{v\ts\in\ts\EEE}\Bigl({c\over 
 \pdg{\hat{H}}_v\bigl(\pst\q_u(c)\,A+{\ee\ts v}\bigr)}\Bigr)^{R} {P}(u,v)\,,
\quad c\in\Rpp\,,\ u\in\EEE\,,
\end{equation}}%
or, which amounts to the same but is easier, $c$ can be
written as an explicit function of $\pst\q_u(c)$.%%%%%%%%%%%%%%%%%%%%%%%%%
%%%%%%%%%%%%%%%%%%%%%%%%%%%%%%%%%%%%%%%
\footnote{{}In terms of computer code, if the connection $c=f(\q)$ can be expressed as a cubic
spline, then $\q$ can be written as a function of $c$ (again as a cubic spline) by merely swapping
the lists of abscissas and the ordinates in the routine that produces the spline objects. Thus,
writing $\q$ as a function of $c$ is no different from writing $c$ as a function of $\q$~-- as long
as the dependence is monotone and smooth.}
%%%%%%%%%%%%%%%%%%%%%%%%%%%%%%%%%%%%%%%
Furthermore, \ref{time-inv}-(\ref{focfoc-d}) gives rise to the iteration program
%%
%%- -%@@%--{Eq: 3.3}--%%
{\abovedisplayskip=7pt plus 1.5pt minus 1.5pt\belowdisplayskip=7pt plus 1.5pt minus 1.5pt\belowdisplayshortskip=5pt plus 1.5pt minus 1.5pt
\begin{equation}\label{4iter-F}
\pst\csd^v(\cdot) \df \sum\nolimits_{u\ts\in\ts\EEE} 
{\gp(u){P}(u,v)\over \gp(v)}\, 
\pdg\csd^u\bigl(\pst\hat \pfc^v(u,\cdot)\bigr)\,,\quad v\in\EEE\,,
\end{equation}}%
In the present context, the metaprogram
in \ref{main-proc} amounts to the following.
\nitskip

%%- -%##%--{No: 3.2}--%%
\begin{nit}{Metaprogram for time-invariant equilibrium}\label{IOU-proc}
{\it Step 0:} Make an ansatz choice for the collection of portfolio mappings $\pdg\q_v(\cdot)$, $v\in\EEE$.
Then make an ansatz choice for the spot price~$B$ (these two choices are independent).
Go to Step~1. 

{\it Step 1:} Set $\pdg{{H}}_v\phd = \phd +\pdg\q_v\phd\,B$
and  compute the inverse $\pdg{\hat{H}}_v\phd$ for every $v\in\EEE$.

%\noindent
{\it Step 2:} For every choice of $u\in\EEE$ and certain choices of $c\in\Rpp$, 
solve \eqref{reduced-k} (there is one equation for every $u\in\EEE$ and every $c\in\Rpp$) 
for the unknowns $\pst\q_u(c)$ and set
{\abovedisplayskip=5pt plus 1.0pt minus 1.5pt\belowdisplayskip=5pt plus 1.5pt minus 1.5pt\belowdisplayshortskip=5pt plus 1.5pt minus 1.5pt
$$\pst\pfc^v(u,c)
=\pdg{\hat{H}}_v\bigl(\pst\q_u(c)\,\inc+{\ee\,v}\bigr)\quad\text{ for every $u,v\in\EEE$}\,.
$$}%
Find the smallest $c\in\Rpp$, denoted $\bar
c$, with the  property $\bar c\ge \pst\pfc^v(u,\bar c)$ for all
$u,v\in\EEE$.%%%%%%%%%%%%%%%%
%%%%%%%%%%%%%%%%%%%%%%%%%%%%%%
\footnote{{}This step is meant to endogenize the upper bound on consumption.}
%%%%%%%%%%%%%%%%%%%%%%%%%%%%%%%%%
Construct a uniform (equidistant) finite grid, denoted $\bbG_\cint$, on the
interval $\cint$. Go to the next step.

  

{\it Step 3:}  For every $u\in\EEE$ and every grid-point
$c\in\bbG_\cint$,
solve for $\pst\q_u(c)$ from \eqref{reduced-k} and set
$\pst\pfc^v(u,c)=\pdg{\hat{H}}_v\bigl(\pst\q_u(c)\,\inc + {\ee\,v}\bigr)$  for all
$v\in\EEE$.%%%%%%%%%%%%%%%%
%%%%%%%%%%%%%%%%%%%%%%%%%%%%%%%%%%%%%%%%%%%%%%%%%%%%%%%%%%%%%%%%
\footnote{Note that $\pst\pfc^v(u,\cdot)$ is fully determined by $\pdg\q_u\phd$ and $\pst\q_u\phd$.}
%%%%%%%%%%%%%%%%%%%%%%%%%%%%%%%%%%%%%%%%%%%%%%%%%%%%%%%%%%%%%%%
By interpolating the respective values define the functions $\pst\q_u(\cdot)$ and
$\pst\pfc^v(u,\cdot)$, $v\in\EEE$, as cubic splines over the grid $\bbG_\cint$ in the obvious way.
Define
uniform interpolation grids over the ranges of the functions  $\pst\pfc^v(u,\cdot)$, compute 
the inverse values at those grid-points and, finally, 
define the inverse functions $\pst\hat \pfc^v(u,\cdot)$ as the cubic
splines obtained by interpolating the inverse values over the respective grids. Go to the next
step. 


{\it Step 4:} 
If the family of distribution functions $\pdg\csd^u$, $u\in\EEE$, 
has not been updated before (this is the first visit to Step~4), 
define $\pdg\csd^u$ to be the distribution function associated with the uniform probability
measure  on $\cint$ for every $u\in\EEE$. Otherwise, do nothing and go to the next step.


{\it Step 5:}
Calculate
{\abovedisplayskip=2pt plus 1.0pt minus 1.5pt\belowdisplayskip=5pt plus 1.5pt minus 1.5pt\belowdisplayshortskip=5pt plus 1.5pt minus 1.5pt
\begin{equation}\tag{a}\label{step-4-dist}
\pst\csd^v(c) \df \sum\nolimits_{u\ts\in\ts\EEE} 
{\gp(u){P}(u,v)\over \gp(v)}\, 
\pdg\csd^u\bigl(\pst\hat\pfc^v(u,c)\bigr)
\end{equation}}%
for every  $c\in\bbG_\cint$ and every $v\in\EEE$ and construct the
distribution functions $\pst\csd^v(\cdot)$, $v\in\EEE$, as cubic splines over 
the grid $\bbG_\cint$ in the obvious way. Compute the error term
%%
{\abovedisplayskip=5pt plus 1.5pt minus 1.5pt\belowdisplayskip=5pt plus 1.5pt minus 1.5pt\belowdisplayshortskip=5pt plus 1.5pt minus 1.5pt
\[
\max\nolimits_{\tts v\ts\in\ts\EEE,\,c\ts\in\ts \bbG_\cint}
|\pst\csd^{v} (c)- \pdg\csd^{v}(c)|\,.
\]}%
If this error term exceeds some  prescribed
threshold, set $\pdg\csd^v(\cdot)=\pst\csd^v(\cdot)$, 
$v\in\EEE$, go back to the beginning of this step and
repeat.%%%%%%%%%%%%%%%%%%%%%%%%%
%%%%%%%%%%%%%%%%%%%%%%%%%%%%%%%%%%%%%%%%
\footnote{These iterations are determined by the choice of $\pdg\q_u\phd$ and $\pst\q_u\phd$.}
%%%%%%%%%%%%%%%%%%%%%%%%%%%%%%%%%%%%%%%%
Otherwise, set $\pdg\csd^v(\cdot)=\pst\csd^v(\cdot)$, $v\in\EEE$, and
go to the next step. 


{\it Step 6:} Test the market clearing
{\abovedisplayskip=5pt plus 1.5pt minus 1.5pt\belowdisplayskip=5pt plus 1.5pt minus 1.5pt\belowdisplayshortskip=5pt plus 1.5pt minus 1.5pt
\begin{equation}\tag{b}\label{step-5-mktc}
\sum\nolimits_{u\ts\in\ts\EEE}\gp(u)\int_0^{\bar c}{\pst\q_u(c)}\d\,\pst\csd^u(c) = 0\,.
\end{equation}}%
If this identity fails by more
than some  prescribed threshold, discard the splines
$\pst\q_u(\cdot)$, $u\in\nobreak\EEE$, while still
keeping $\pdg\q_u(\cdot)$ on
record,  modify the most recent choice for the spot price $B$
accordingly, and go back to Step~1. Otherwise go to the next step.

{\it Step 7:} If this is the first visit to Step 7, set
$\pdg\q_u(\cdot)=\pst\q_u(\cdot)$
and go to Step~1 with the most recently updated value for the spot price~$B$. 
Otherwise, compute the error terms
{\abovedisplayskip=5pt plus 1.5pt minus 1.5pt\belowdisplayskip=5pt plus 1.5pt minus 1.5pt\belowdisplayshortskip=5pt plus 1.5pt minus 1.5pt
\begin{equation}\tag{c}\label{conv-test}
\max\nolimits_{u\ts\in\ts\EEE,\,c\in \bbG_\cint}
|\pst\q_u(c)-\pdg\q_u(c)|
\quad 
\text{and}\quad
\max\nolimits_{u,v\in\EEE,\,c\in \bbG_\cint}
|\pst\pfc^v(u,c)-\pdg \pfc^v(u,c)|\,.
\end{equation}}%
If at least one of these terms exceeds some prescribed threshold, set
$\pdg\q_u(\cdot)=\pst\q_u(\cdot)$ and go to Step~1 with the most recently updated value for the
spot price~$B$. 
Otherwise stop. Declare that the
equilibrium is given by the most recently updated spot price $B$,
portfolio mappings $\pst\q_u(\cdot)$, $u\in\EEE$, 
state transitions 
$\pst\pfc^v(u,\cdot)$, $u,v\in\EEE$, and family of distribution
functions $\pst\csd^u(\cdot)$, $u\in\EEE$.\qed
\end{nit}


%%- -%##%--{No: 3.3}--%%
\begin{nit}{Abridged version of \ref{IOU-proc}}\label{abridged}
Make an ansatz choice for $(B,\pdg\q)$ and record it.
Gi\-ven $(B,\pdg\q)$, produce $\pst\q$. 
Then produce $\pst{\csd}$ as a fixed point of the transport determined by~$\pdg\q$
and~$\pst\q$. If market clearing with~$\pst\q$ and~$\pst{\csd}$ fails, forget
$\pst\q$ and $\pst{\csd}$, change the value of~$B$, and repeat with the modified $B$ and
with $\pdg\q$. If the market clears, record $\pst\q$,  the latest $B$, and the latest
transition mappings.
If~this is the first incidence of market clearing, set $\pdg\q=\pst\q$ and repeat from
the beginning with the latest $B$ and the new $\pdg\q$. If~not, test the uniform distance
between $\pst\q$ and $\pdg\q$ and between the two most recent collections of state transition mappings.
If~this distance is not acceptable, set $\pdg\q=\pst\q$ and repeat from
the beginning with the latest $B$ and the new $\pdg\q$. Otherwise, stop.\qed
\end{nit}


%%- -%##%--{No: 3.4}--%% 
\begin{nit}{Remark}
It is instructive to note the key
differences between the program in~\nobreak\ref{IOU-proc} and the classical strategy outlined
in Sec.~\ref{sec:Intro}: (a)~The portfolio mappings $\pdg\q_u(\cdot)$, $u\in\EEE$, capture
the investment decisions in the cross-section of all households that share the same state of
employment~-- not the investment decision of one representative household.
(b)~The law of motion in the space of distributions (of unit mass) encrypted in
\ref{IOU-proc}-(\ref{step-4-dist}) is not sought as the law of motion of the probability
distribution of any particular Markovian state. 
(c)~It is the price that gets adjusted to the portfolio mappings, and then new portfolio mappings
are obtained with the new price, i.e., the adjustments in portfolio mappings and prices alternate.
(d)~The search for a fixed point of the transport equation in Step~5, for every instance of
$\pst\q_u(\cdot)$ and $\pdg\q_u(\cdot)$, removes the need to write the endogenous variables as
functions on the space of distributions, in that there is always a unique
distribution associated with every instance of $\pst\q_u(\cdot)$ and $\pdg\q_u(\cdot)$.
Thus, instead attaching values of the endogenous variables to every instance of the population
distribution, the program associates such values only with a single distribution, namely, the one
that remains invariant (under the most recent instance of portfolio mappings)~-- an
enormous simplification, possible only if stationary distribution of the population exists and
one is concerned with the infinite time-horizon case alone.
\qed
\end{nit}

%%- -%##%--{No: 3.5}--%%
\begin{nit}{Remark}
Step~1 above is nothing but the search for the
endogenous upper bound on consumption, which then translates into an upper bound on investment,
since the functions $c \leadsto \pst\q_u(c)$,
$u\in\EEE$, are increasing.
The lower bound on the investment (i.e., the borrowing limit)
is then $\min_u\lim_{c\to0}\q_u(c)B$.
We stress that both bounds are determined
endogenously throughout the iterations.
In the benchmark economy discussed here these bounds
are never reached and the cross-sectional distribution of the population has no mass at
them.\qed
\end{nit}
\nitskip

What follows next is a brief summary of the concrete results from implementing the
metaprogram in \ref{IOU-proc} in the context of the benchmark Huggett economy
borrowed from \cite{LjunSar00}  and
already introduced in Sec.~\ref{sec:Intro}. 
All model parameters are taken from the first specification from Sec.~18.7
in \cite{LjunSar00}. 
The initial ansatz choice for the portfolio functions is $\pdg\q_v(c)=40c-8$ for all $v\in\EEE$ and
for the bond price the initial choice is $B=\inc$, corresponding to zero interest as an initial guess. 
The convergence (the largest amount
in \hbox{\ref{IOU-proc}-(\ref{conv-test})}) is $9.08447\times 10^{-5}$ after 235 iterations, which
the program completes on a single core for about 2.25 hours, and 
returns equilibrium interest rate of $0.03702$ 
and market clearing (the left side of~\ref{IOU-proc}-(\ref{step-5-mktc})) of
$-1.73878\times10^{-6}$.
The distribution of households in every one of the $7$ employment categories over the consumption space
is shown in Figure~\ref{fg4X} 
%%%%%%%%%%%%%%%%%%%%%%%%%%%%%%%%%%%%%%%%%  
%%    Fg: 5
%%%%%%%%%%%%%%%%%%%%%%%%%%%%%%%%%%%%%%%%%%
{%
\begin{figure}[!htbp]  
\centering
\begin{subfigure}{.5\textwidth}
  \centering
\leavevmode\raise0.55cm\hbox{\rotatebox{90}{\tiny population distribution in states $u\in\EEE$}}%
\ %   
\toshow{\includegraphics[width=7.1cm]{fg5L}}

\leavevmode\smash{\raise6pt\hbox{\tiny consumption}}
\end{subfigure}%
\begin{subfigure}{.5\textwidth}
  \centering
\leavevmode\raise0.9cm\hbox{\rotatebox{90}{\tiny population density in states $u\in\EEE$}}%
\ %
\toshow{\includegraphics[width=7.1cm]{fg5R}}  

\leavevmode\smash{\raise6pt\hbox{\tiny consumption}}
\end{subfigure}
\caption{The distribution (cumulative left, density right) of households over the range of consumption.}
\label{fg4X} 
\end{figure}}%
%
%
%
and the plots in Figure~\ref{fg8X} show the investment level in the cross-section of the population, i.e.,
the mappings $c \leadsto \q_u(c)\times B$, $u\in\nobreak\EEE\,$.
%
%
%%%%%%%%%%%%%%%%%%%%%%%%%%%%%%%%%%%%%%%%%%
%%     Fg: 6
%%%%%%%%%%%%%%%%%%%%%%%%%%%%%%%%%%%%%%%%%%
{%
\begin{figure}[!htbp]  
\centering
\begin{subfigure}{.5\textwidth}
  \centering
\leavevmode\raise1.2cm\hbox{\rotatebox{90}{\tiny investment in states $u\in\EEE$}}%
\ %  
\toshow{\includegraphics[width=7.1cm]{fg6L}}

\leavevmode\smash{\raise6pt\hbox{\tiny consumption}}
\end{subfigure}%
\begin{subfigure}{.5\textwidth}
  \centering
\leavevmode\raise1.2cm\hbox{\rotatebox{90}{\tiny investment in states $u\in\EEE$}}%
\ %
\toshow{\includegraphics[width=7.1cm]{fg6R}} 

\leavevmode\smash{\raise6pt\hbox{\tiny consumption}} 
\end{subfigure}
\caption{Investment in the bond as a function of consumption  shown on two different scales.}
\label{fg8X}
\end{figure}}%
%
%
%
The left limit in these graphs (the endogenous borrowing limit)
is around $-1.62826$ and is remarkably close to Aiyagari's
natural borrowing limit (in this case, the same as the ``ad hoc'' limit~-- see \cite{LjunSar00} and
\cite{Aiya94}),
which is around $-1.62726$. The endogenous upper bound on investment~-- see Step~2
in \ref{IOU-proc}~-- is around $17.93751$, 
but we see from Figures~\ref{fg4} and~\ref{fg5X} that most of the population is amassed over a much narrower
range.
Since the mappings $c\leadsto \q_u(c)\ts B$ and $c\leadsto c+\q_u(c)\ts B-\ee\ts u$ are both
strictly increasing and continuous, the equilibrium distribution over consumption from Figure~\ref{fg4X} is easy to
transform into entering and exiting distribution of households over the asset space~--
this is how the left plot in Figure~\ref{fg4} was produced. The left plot in
Figure~\ref{fg5X} gives a detailed view of the left plot in Figure~\ref{fg4} near the borrowing
limit.
%
%%%%%%%%%%%%%%%%%%%%%%%%%%%%%%%%%%%%%%%%%%
%%    Fg: 7
%%%%%%%%%%%%%%%%%%%%%%%%%%%%%%%%%%%%%%%%%% 
{%
\begin{figure}[!htbp]  
\centering
\begin{subfigure}{.5\textwidth}
  \centering
  \leavevmode\raise0.55cm\hbox{\rotatebox{90}{\tiny population distribution in states $u\in\EEE$}}%
\ %  
\toshow{\includegraphics[width=7.1cm]{fg7L}}

\leavevmode\smash{\raise6pt\hbox{\tiny asset holdings}}
\end{subfigure}% 
\begin{subfigure}{.5\textwidth}
  \centering
  \leavevmode\raise0.9cm\hbox{\rotatebox{90}{\tiny population density in states $u\in\EEE$}}%
\ %
\toshow{\includegraphics[width=7.1cm]{fg7R}} 

\leavevmode\smash{\raise6pt\hbox{\tiny asset holdings}}
\end{subfigure}
\caption{The entering (solid lines) and exiting (dotted lines) distribution of households over
asset holdings.}
\label{fg5X}
\end{figure}}%
%
%
The right plot
in Figure~\ref{fg5X} is simply the density version of the left plot in Figure~\ref{fg4}. 
The graphs of the conditional transition mappings  $c \leadsto \pfc^v(u,c)$, $v\in\EEE$, for the lowest
and the highest employment category $u\in\EEE$ are shown on the left plot in Figure~\ref{fg8Y}. 
%
%
%%%%%%%%%%%%%%%%%%%%%%%%%%%%%%%%%%%%%%%%%% 
%%     Fg: 8 
%%%%%%%%%%%%%%%%%%%%%%%%%%%%%%%%%%%%%%%%%% 
{\captionsetup{belowskip=-10pt}
\begin{figure}[!htbp]  
\centering
\begin{subfigure}{.5\textwidth}
  \centering
\leavevmode\raise0.75cm\hbox{\rotatebox{90}{\tiny future consumption in states $v\in\EEE$}}%
\ %  
\toshow{\includegraphics[width=7.1cm]{fg8L}}

\leavevmode\smash{\raise6pt\hbox{\tiny consumption in state $u=\EEE_1$ (solid) and in state $u=\EEE_7$ (dotted)}}
\end{subfigure}%
\begin{subfigure}{.5\textwidth}
  \centering
\leavevmode\raise1.25cm\hbox{\rotatebox{90}{\tiny consumption in states $v\in\EEE$}}%
\ %
\toshow{\includegraphics[width=7.1cm]{fg8R}} 

\leavevmode\smash{\raise6pt\hbox{\tiny entering wealth from state $u=\EEE_1$ (solid) and from $u=\EEE_7$ (dotted)}} 
\end{subfigure}
\caption{Future consumption as a function of present consumption (left) and entering future wealth (right).} 
\label{fg8Y}
\end{figure}}%
%
%
The right plot in Figure~\ref{fg8Y} provides an important verification of the program developed in
this section: on the one hand the employment specific transition mappings depend on both the exiting and the
entering employment states, while on the other hand future consumption must depend only on the future
employment state and the entering wealth in that state, irrespective of what employment state that
wealth is carried from. Since any household of type $(u,c)\in\EEE\times\Rpp$ enters its future state with
assets $a=\a_u(c)\df \q_u(c)\times A$, letting $\hat\a_u\phd$ denote the inverse of the mappings
$c \leadsto \a_u(c)$, this means that $\TTT^v(u,\hat\a_u(a))$ must depend on~$v$ and~$a$ but not
on~$u$. Such a connection was never imposed in the system that produced the equilibrium, but the
right plot in Figure~\ref{fg8Y} shows that it nevertheless holds~-- as it should.
Finally, the left plot in Figure~\ref{fg9Z} shows consumption as a function of total wealth
(i.e., consumption plus investment)
and the right plot shows the marginal propensity to consume (simply
the gradient of the splines generating the left plot). 
%
%
%%%%%%%%%%%%%%%%%%%%%%%%%%%%%%%%%%%%%%%%%% 
%%     Fg: 9 
%%%%%%%%%%%%%%%%%%%%%%%%%%%%%%%%%%%%%%%%%% 
{\captionsetup{belowskip=-10pt}
\captionsetup{aboveskip=10pt}
\begin{figure}[!htbp]  
\centering
\begin{subfigure}{.5\textwidth}
  \centering
\leavevmode\raise0.75cm\hbox{\rotatebox{90}{\tiny consumption in states $u\in\EEE$}}%
\ %  
\toshow{\includegraphics[width=7.1cm]{fg9L}}

\leavevmode\smash{\raise6pt\hbox{\tiny \qquad total wealth (entering plus income)}}
\end{subfigure}%
\begin{subfigure}{.5\textwidth}
  \centering
\leavevmode\raise0.15cm\hbox{\rotatebox{90}{\tiny marginal propensity to consume in states $u\in\EEE$}}%
\ %
\toshow{\includegraphics[width=7.1cm]{fg9R}} 

\leavevmode\smash{\raise6pt\hbox{\tiny \qquad total wealth (entering plus income)}} 
\end{subfigure}
\caption{Consumption as a function of total wealth (entering plus income) and its gradient.} 
\label{fg9Z}
\end{figure}}%
%
%%- -%##%--{No: 3.6}--%% 
\begin{nit}{Implications}
Figure~\ref{fg8X} shows that the employment state has only a marginal effect on the dependence
between investment and consumption (all households face the same stream of employment shocks in the
long run and the effect of temporary differences in employment is too
small relative to the entire stream of anticipated future shocks).
In addition, the right plot shows that the graphs on the left are not flat near~$0$, with slopes
ranging between  $0.12502$ and $0.40615$.
Figures~\ref{fg4X} and~\ref{fg5X} show that the distribution of households (both entering and
exiting) over asset holdings is
substantially more dispersed and skewed toward the wealthy than it is over consumption.
The standard deviation over consumption ranges between  
$0.0458$ and $0.03827$, over entering wealth it ranges between
$0.97125$ and $0.97857$, and ranges between  $0.92818$ and $0.93861$ over exiting wealth. For the
skewness these numbers are  $0.11558$ $\sim$ $0.84976$ for consumption,  $0.96578$ $\sim$ $0.93788$
for entering wealth, and  $1.00653$ $\sim$ $0.9316$ for exiting wealth.
In contrast to the conclusions in \citel{AHLLM17}, which involve exogenously imposed
borrowing limits and boundary conditions and are based on different model parameters
(see Footnote~\ref{foot-AHLLM}), 
it is clear from Figures~\ref{fg5X} and~\ref{fg9Z} that the cross-sectional distribution of the
population does not accumulate at the borrowing limit and the marginal propensity to consume does
not explode at that limit.%%%%%%%%%%%%%%%%%%%%%%%%%%%%%%%%%%%%%%
%%%%%%%%%%%%%%%%%%%%%%%%%%%%%%%%%%%%%%%%%%%%%%%
\footnote{These features cannot be claimed to hold universally.}
%%%%%%%%%%%%%%%%%%%%%%%%%%%%%%%%%%%%%%%%%%%%%%%
%
More details and illustrations are included with the Julia code that accompanies the \chptr.\qed
%
\end{nit} 

%%% chend556677

%%- -%%--{Sec: #4}--%%
%{Eq: 4.}   %%llabel
%{No: 4.}   %%nlabel

\section{Approximate Equilibrium in Models with Aggregate Risk}\setcounter{paragraph}{0}
\label{sec:KS}

\noindent
The main objective in this section is to specialize the general model developed in Sec.~\ref{sec:gen-model} to the
setup of the widely cited case study from \cite{KS98},
in which the economy is endowed with production technology and 
the households invest only in productive capital, with no risk-free private lending available.
The main premise in the paper \cite{KS98} is what the authors call ``collapse of the state space,'' or
``approximate aggregation,'' i.e., the idea that ``in~equilibrium, all aggregate variables~--
consumption, the capital stock, and relative prices~-- can be almost perfectly described
as a function of two simple statistics: the mean of the wealth distribution and the aggregate
productivity shock.'' Simply put: ``only the mean matters.''
The insight offered in \cite{KS98}  is that ``utility costs from
fluctuations in consumption are quite small and that self-insurance with only one asset is quite
effective''~-- a phenomenon that  the authors link to the permanent income hypothesis
from the paper \citel{Bew77}. 
The present section is a follow-up to this insight, but with a number of deflections and
clarifications, which are spelled out next.
In~what follows the approximate aggregation hypothesis is placed in the context of a broader framework,
conditions under which this result  
holds (as an approximation) are identified, new methodology that allows one to quantify the notions
of ``approximate'' and ``almost perfectly'' is developed, and a more refined aggregation strategy is
outlined. 
It~is to be noted that in the present context ``the mean of the population distribution'' is understood as the mean
of the distribution over consumption~-- not wealth.
Since this distribution enters the model only through the market 
clearing condition, %in the context of the model developed in Sec.~\ref{sec:gen-model}
the claim ``only the mean matters'' boils down to the claim that, as functions of consumption, the portfolio
mappings are approximately affine with employment-invariant slopes.%%%%%%%%%%%%%%%%%%%%%%%%%%
%%%%%%%%%%%%%%%%%%%%%%%%%%%%%%%%%%%%%%%%%%%%%%%%%%
\footnote{{}Stroock \citel{Str23} provides a particularly elegant formal justification.}
%%%%%%%%%%%%%%%%%%%%%%%%%%%%%%%%%%%%%%%%%%%%%%%%%
Hence, quantifying ``approximate'' in ``approximate aggregation'' comes down to quantifying the
error from replacing the portfolio mappings with affine functions and identifying conditions under
which the slopes of those functions are employment invariant. 
In~addition, closed-form expressions for the slopes are derived and
the computational strategy  described in \ref{main-proc} 
is adapted to this special structure. One consequence from this new approach is that one no longer
needs to postulate infinite time horizon and resort on simulation; instead, the computational
strategy boils down to solving sequentially linear systems~-- for as many periods as needed. 

The only-the-mean-matters point of view has one obvious limitation: it fails to
capture the dynamics of the relative disparity among households~-- an important macroeconomic
characteristic. The idea that ``the mean of the wealth distribution and the aggregate
productivity shock'' are ``two simple statistics'' that describe ``almost perfectly'' all aggregate
variables may lead one to conclude that, in the study of general equilibrium,
there is no further inference to be made about the
distribution of all households over the range of wealth~-- or,
equivalently, over the range of consumption~-- other than its mean.
The approach developed in this \chptr\  shows that this is not the
case: even if the state transition
mappings in the right side of 
\ref{main-q}-\eqref{zze666} were to depend only on the mean value of the distribution $\csd$, this relation
would still provide the exact law of motion of the full population distribution. For the sake of 
simplicity, in what follows we do not pursue this law of motion in its full generality, but nevertheless
derive the exact law
of motion of the vector of conditional population means attached to every employment group
(in effect, reducing the population distribution to a vector of mean values, rather than a single
mean value)~-- see \eqref{no-fp-2-0} below.
Most important, the dynamics of this vector
reveal substantial fluctuations in the wealth-inequality across the population
that are not possible to capture if the model remains
confined to the total population mean alone~-- this feature is illustrated in Figure~\ref{fgKS5} below. 

There is one subtle~-- or not so subtle, depending on the point of view~-- aspect of the claim ``only the
mean matters,'' which is commonly swept under the proverbial ``rug,'' although the
work \cite{KS98} is almost explicit about it: the individual
investment-consumption decisions during the present period depend not only on the present mean
and present productivity state, but also on the transport of the present mean into the future,
conditioned on the realized future productivity state of the economy. To put it another way, the
claim ``only the mean matters'' is in fact the claim that
``only the mean and its transport matter,'' which then implies~-- as
we have seen before~-- that the transport must be self-aware.%%%%%%%%%%%%%%%%%%%%%
%%%%%%%%%%%%%%%%%%%%%%%%%%%%%%%%%%%%%%%%%%%%%%%%%%%%
\footnote{The search for parameters in the log-linear  AR(1) predictor  that maximize the goodness
of fit in Krusell-Smith's method can be seen as a way to bring the transport as close as possible to
being self-aware~-- in the long run.}
%%%%%%%%%%%%%%%%%%%%%%%%%%%%%%%%%%%%%%%%%%%%%%%%%%%
Such features are  less apparent in models with
infinite time horizon~-- see \cite{KS98}, for example~-- but this is not the approach adopted here:
we insist that infinite time horizon models are to be understood only as limits of models with
finite time horizon, which must be developed first.
The reason for this (admittedly, not very
common) point of view transcends the obvious need for methodological coherence: it~will be shown below that
even in the classical model, borrowed here from \cite{KS98},
with only~2 employment and~2
productivity states, the economy typically needs to run for 
hundreds of periods before it can achieve its time-invariant regime,
whereas no real-world economy can exist unchanged for that long~-- whence the need for a methodology that does
not resort on time invariance. In the illustrations that follow in this section the infinite time
horizon scenario is pursued mainly for the purpose of benchmarking and aims to demonstrate that the
finite time horizon methodology developed in this \chptr\  is consistent with previously known
results.


Consistent with the general model introduced in
Sec.~\ref{sec:gen-model}, we set
$\q_{t,x,\csd,u}(c)\equiv\nobreak 0$ (the agents do not invest in a risk-free asset), ignore the
first kernel condition in \ref{cross-sys}-(\ref{zze5a}), and ignore the first market clearing condition in
\ref{cross-sys}-(\ref{ze9}). The cross-sectional distribution of the population enters the model
only through the market clearing condition and also through the transport
equation \ref{main-q}-\eqref{zze666}. 
To develop a better grasp of its rôle, we now turn to the second market clearing condition
in \ref{cross-sys}-(\ref{ze9}) 
and introduce the vector of conditional (to employment) mean values, namely%%%%%%%%%%%%%%%%%%%%
%%%%%%%%%%%%%%%%%%%%%%%%%%%
\footnote{{}The obvious dependence between the symbols $A$ and $\csd$ will be suppressed in the notation
for simplicity.}
%%%%%%%%%%%%%%%%%%%%%%%%%%%
%
%
{\abovedisplayskip=5pt plus 1.5pt minus 1.5pt\belowdisplayskip=8pt plus 1.5pt minus 1.5pt\belowdisplayshortskip=6pt plus 1.5pt minus 1.5pt
\[
A\df (A^{u},u\in\EEE)\in\R^{\abs{\EEE}}\,,\quad \text{where}\quad 
A^{u}\df \int_0^\infty c \d\csd^{u}(c)\,,\quad \csd\in\DB^\EEE\,. 
\]}%
If the cross-sectional distribution $\csd$ affects the model only through the associated vector
$A$, then  the left side of \ref{cross-sys}-\eqref{ze9} must be 
a function of that vector alone. In that case, if all portfolio
mappings $\qq_{t,x,F,u}\phd$ happen to be continuous, by Stroock's argument \cite{Str23}
the left side can only be of the form
{\abovedisplayskip=7pt plus 1.5pt minus 1.5pt\belowdisplayskip=8pt plus 1.5pt minus 1.5pt\belowdisplayshortskip=7pt plus 1.0pt minus 1.5pt
\[
\R^{\abs{\EEE}}\ni A \leadsto  \sum\nolimits_{u\tts\in\tts\EEE}\pi_x(u)\qq_{t,x,\csd,u}(A^u)
\]
}%%%%%%%%%%%%%%%%%%%%
and, in addition, all portfolio mappings $\qq_{t,x,F,u}\phd$, $u\in\EEE$,
must be affine functions. It is clear from the above expression that the
population distribution affects the market clearing only through its total (unconditional to employment) mean value
$A^* = \sum_{u\tts\in\tts\EEE}\pi_x(u) A^u$ precisely when the slopes of the affine
functions $\qq_{t,x,F,u}\phd$ do not depend on the employment state $u\in\EEE$.
It~will be shown below that such an
arrangement is indeed possible~-- as a reasonably good approximation~-- if all
consumption levels across the population are sufficiently large.

In general, the households' demand for capital cannot be an affine function of  the  consumption
level~$c$, since the latter affects the kernel condition in \ref{cross-sys}-(\ref{zze5a})
in nonlinear fashion.
In~order to uncover the way in which this kernel condition affects the structure of the portfolio
mappings and identify conditions under which the population distribution $\csd\in\DB^\EEE$ affects,
at least as an approximation, 
the model only through the vector of its employment specific  mean values $A\in\R^{\abs{\EEE}}$~--
or, as a special case, only through the total population mean
$A^* = \sum_{u\tts\in\tts\EEE}\pi_x(u) A^u$~-- 
let us suppose, contrary
to fact, that all portfolio and  
future consumption mappings have the affine structure
(the dependence on $\csd\in\DB^\EEE$ is now collapsed to dependence on the associated vector of
conditional means $A\in\R^\abs{\EEE}$)
%%
%%- -%@@%--{Eq: 4.1}--%%
{\abovedisplayskip=8pt plus 2.5pt minus 1.5pt\belowdisplayskip=8pt plus 2.5pt minus 1.5pt\belowdisplayshortskip=8pt plus 1.0pt minus 1.5pt
\begin{equation}\label{lin-subst}
\qq_{t,\tts x,\tts A,\tts u}(c) = a_{t,\tts x,\tts  A,\tts u}+b_{t,\tts x,\tts A,\tts u}\times c\quad\text{and}\quad
\pfc_{t,\tts x,\tts A}^{ y , v }(u,c) = g_{t,\tts x,\tts  A,\tts u}^{ y , v }
+ h_{t,\tts x,\tts  A,\tts u}^{ y , v }\times c\,,\quad c\in\Rpp\,,
\end{equation}}%
for some yet to be determined coefficients
%%- -%@@%--{Eq: 4.2}--%%
{\abovedisplayskip=8pt plus 2.5pt minus 1.5pt\belowdisplayskip=8pt plus 2.5pt minus 1.5pt\belowdisplayshortskip=8pt plus 1.0pt minus 1.5pt
\begin{equation}\label{4coef}
a_{t,\tts x,\tts A,\tts u}\,, \ \ b_{t,\tts x,\tts A,\tts u}\,, \ \ 
g_{t,\tts x,\tts  A,\tts u}^{ y , v }\,, \ \ \text{and} \ \ h_{t,\tts x,\tts  A,\tts u}^{ y , v }\,.
\end{equation}
}%
With the choice just made and
with risk aversion parameter of $R=1$
(same as in the benchmark case study of \cite{KS98}), the second kernel
condition in \ref{cross-sys}-(\ref{zze5a}) can be cast as
%%
%%- -%@@%--{Eq: 4.3}--%% 
{\abovedisplayskip=8pt plus 1.5pt minus 1.5pt\belowdisplayskip=8pt plus 1.5pt minus 1.5pt\belowdisplayshortskip=8pt plus 1.5pt minus 1.5pt
\begin{equation}\label{only-q-bn2-2}
1 = \b\,\sum_{ y \tts \in\tts\XXX, v\tts\in\tts\EEE}
{ 1 \over g_{t,\tts x,\tts A,\tts u}^{ y,\tts v}/c  + h_{t,\tts x,\tts A,\tts u}^{ y ,\tts v}}\times
\Bigl(\rho_ y \bigl(K_{t}(x, A)\bigr) + 1-\dd \Bigr)
\times  Q(x, y )\,  P_{x,\tts y }( u ,v)\,.
\end{equation}}%
With the substitution \eqref{lin-subst} in mind, the balance equation 
in \ref{cross-sys}-(\ref{zze5xb}) becomes (the transport of the distribution $\csd$ is now transport of
the associated vector $A$~-- see \eqref{no-fp-2-0} below)
%%
%%- -%@@%--{Eq: 4.4}--%%
{\abovedisplayskip=5pt plus 1.5pt minus 1.5pt\belowdisplayskip=5pt plus 1.5pt minus 1.5pt\belowdisplayshortskip=5pt plus 1.5pt minus 1.5pt
\begin{equation}\label{e10-bn3}
\begin{gathered}
\begin{multlined}
({ {a_{t,\tts x,\tts A,\tts u}}}+{ {b_{t,\tts x,\tts A,\tts u}}}\times c)\times
\Bigl(\rho_ y \bigl({ {K_{t}(x, A)}}\bigr) +1-\dd\Bigr) 
+ v\times \ee_ y \bigl({ {K_{t}(x, A)}}\bigr)\hbox to1.5cm{\hfil}\\
\hbox to0.5cm{\hfil} = \bigl({ {g_{t,\tts x,\tts A,\tts u}^{ y ,v}}} + { {h_{t,\tts x,\tts A,\tts u}^{ y ,v}}}\times c\bigr)
+ \Bigl({{a_{t+1,\tts y,\tts\T_{t,x}^y( A),\tts v}}}
+{{b_{t+1,\tts y,\tts \T_{t,x}^y( A),\tts v}}}\times\bigr({ {g_{t,\tts x,\tts A,\tts u}^{ y ,v}}}
+ { {h_{t,\tts x,\tts A,\tts u}^{ y ,v}}}\times c\bigr)\Bigr)\,.
\end{multlined}
\end{gathered}
\end{equation}}%
Just as before, the strategy is to take the future portfolio mappings encrypted in the pairs
%%
{\abovedisplayskip=5pt plus 1.5pt minus 1.5pt\belowdisplayskip=5pt plus 1.5pt minus 1.5pt\belowdisplayshortskip=5pt plus 1.5pt minus 1.5pt
$$
\bigl(a_{t+1,\tts y ,\tts \T_{t,x}^y( A),\tts v},b_{t+1,\tts y ,\tts  \T_{t,x}^y( A),\tts v}\bigr)\,,\quad
y\in\XXX\,,\ \ v\in\EEE\,,
$$
}%
as
given and treat \eqref{e10-bn3} as a system for the unknowns (present portfolios and future
consumption mappings) $a_{t,\tts x,\tts A,\tts u}$, $b_{t,\tts x,\tts A,\tts u}$,
$g_{t,\tts x,\tts A,\tts u}^{ y ,v}$, and $h_{t,\tts x,\tts A,\tts u}^{ y ,v}$, $x, y \in\XXX$,
$u, v\in\EEE$.
Thus, for every fixed  period-$t$ state of the economy $(x,A)\in\XXX\times\R^{\abs{\EEE}}$,
there is a total of $2\abs{\EEE}+2\abs{\EEE}^2\abs{\XXX}$ unknowns to solve for,
provided that an ansatz choice for the average capital $K_{t}(x, A)$ is somehow
made (the market clearing will be addressed later).
The next step is to
extract the same number of equations from \eqref{only-q-bn2-2} and \eqref{e10-bn3}. This task is
non-trivial since \eqref{only-q-bn2-2} and \eqref{e10-bn3}
are, in fact, systems of infinitely many equations~-- one for every
$c\in\Rpp$.  %(recall the all FOLC are identities between functions of $c\in\Rpp$).
Treated as an identity between two polynomials of
degree~$1$ over the variable $c\in\Rpp$, \eqref{e10-bn3} can be split into two systems:
\begin{subequations}\label{lin-split}
{\abovedisplayskip=5pt plus 1.5pt minus 1.5pt\belowdisplayskip=5pt plus 1.5pt minus 1.5pt\belowdisplayshortskip=5pt plus 1.5pt minus 1.5pt
%%- -%@@%--{Eq: 4.5}--%%
\begin{gather}
\begin{aligned}\label{lin-split-a}
&\hbox to1.5cm{\hfill}{ {b_{t,\tts x,\tts A,\tts u}}}\times
\Bigl(\rho_ y \bigl({ {K_{t}(x, A)}}\bigr) +1-\dd\Bigr) 
= { {h_{t,\tts x,\tts A,\tts u}^{ y ,v}}}
+ {{b_{t+1,\tts y ,\tts \T_{t,x}^y(A),\tts v}}}\times{ {h_{t,\tts x,\tts A,\tts u}^{ y ,v}}}\\
&\hbox to9.0cm{\hfill}\text{for all }\ \  y \in\XXX\,,\  u, v\in\EEE\,.
\end{aligned}%
\nobreak\\
\noalign{\rlap{\smash{and}}}
\begin{aligned}\label{lin-split-b}
&{ {a_{t,\tts x,\tts A,\tts u}}}\times  
\Bigl(\rho_ y \bigl({ {K_{t}(x, A)}}\bigr) +1-\dd\Bigr) 
+v \times \ee_ y \bigl({ {K_{t}(x, A)}}\bigr)\\
&\hbox to4.5cm{\hfil}= {{a_{t+1,\tts y ,\tts \T_{t,x}^y( A),\tts v}}}+\bigl(1+{{b_{t+1,\tts y
,\tts \T_{t,x}^y( A),\tts v}}}\bigr)\times { {g_{t,\tts x,\tts A,\tts u}^{ y ,v}}}\\
&\hbox to8.5cm{\hfil}\quad\text{for all }\ \  y \in\XXX\,,\  u, v\in\EEE\,,
\end{aligned}
\end{gather}}%
\end{subequations}
Each system in \eqref{lin-split} provides $\abs{\EEE}^2\abs{\XXX}$ equations and solving both
guarantees that \eqref{e10-bn3} holds exactly for every $c\in\Rpp$.
Thus, $2\abs{\EEE}$ additional
equations, that can only come from \eqref{only-q-bn2-2}, are needed.
Unfortunately, as written (with the stipulated affine structure in mind), it is not
possible to enforce the kernel condition \eqref{only-q-bn2-2} exactly for every $c\in\Rpp$, and this
is the main reason why the affine structure imposed in \eqref{lin-subst}~-- that is to say, the
stipulation that ``only the population mean matters''~-- can hold only as an approximation.
Hence, estimating the accuracy of the approximate aggregation point of
view comes down to estimating the
deviation of the right side in \eqref{only-q-bn2-2} from the constant $1$~-- see
\ref{mod-accu} below. An approximation of \eqref{only-q-bn2-2} that immediately comes to mind, and
provides exactly $2\abs{\EEE}$ additional equations, is replacing the right side with its first-order
Taylor expansion over the variable ${1\over c}$. As we are about to see, the easiest such expansion
is around ${1\over c}=0$. This is quite intuitive: in a neighborhood of $c=\infty$ the right side is
nearly invariant under the choice of $c$, so that the first order Taylor approximation should be
quite accurate.  Coincidentally, this choice will turn out to be consistent with the
approximate aggregation hypothesis~-- see below. For every fixed
$(x,A)\in\XXX\times\R^{\abs{\EEE}}$, the first-order Taylor expansion around ${1\over c}=0$
transforms \eqref{only-q-bn2-2} into the following two systems:
\begin{subequations}\label{Taylor}
{\abovedisplayskip=6pt plus 1.5pt minus 1.5pt\belowdisplayskip=6pt plus 1.5pt minus 1.5pt\belowdisplayshortskip=6pt plus 1.5pt minus 1.5pt
%%- -%@@%--{Eq: 4.6}--%%
\begin{gather}\label{Taylor-a}
1 = \b\,\sum_{ y \tts \in\tts\XXX, v\tts\in\tts\EEE}
{ 1 \over h_{t,\tts x,\tts A,\tts u}^{ y ,\tts v}}\times
\Bigl(\rho_ y \bigl(K_{t}(x, A)\bigr) + 1-\dd \Bigr)
\times  Q(x, y )\,  P_{x,\tts y }( u ,v)\,,\quad u\in\EEE\,,\\
\noalign{\rlap{{and}}}
\label{Taylor-b}
0 = \sum_{ y \tts \in\tts\XXX, v\tts\in\tts\EEE}
{ 1 \over (h_{t,\tts x,\tts A,\tts u}^{ y ,\tts v})^2}
\times g_{t,\tts x,\tts A,\tts u}^{ y,\tts v} \times
\Bigl(\rho_ y \bigl(K_{t}(x, A)\bigr) + 1-\dd \Bigr)
\times  Q(x, y )\,  P_{x,\tts y }( u ,v)\,,\quad u\in\EEE\,.
\end{gather}
}%
\end{subequations}
Keeping in mind that
${a_{t+1,\tts y ,\tts \T_{t,x}^y( A),\tts v}}$ and ${b_{t+1,\tts y ,\tts \T_{t,x}^y( A),\tts v}}$
are treated as given,
\eqref{lin-split-a} and \eqref{Taylor-a} provide a closed system of 
$\abs{\EEE}+\abs{\EEE}^2\abs{\XXX}$ equations for the same number of
unknowns, namely
{\abovedisplayskip=5pt plus 1.5pt minus 1.5pt\belowdisplayskip=5pt plus 1.5pt minus 1.5pt\belowdisplayshortskip=5pt plus 1.0pt minus 1.5pt
\[
{ {b_{t,\tts x,\tts A,\tts u}}}\,, \ { {h_{t,\tts x,\tts A,\tts u}^{ y ,v}}}\,,\quad
 u, v\in\EEE\,,\  y \in\XXX\,.
\]}%
This system simplifies substantially, once it is observed that the dependence of the unknowns on
the employment states $u,v\in\EEE$ can be suppressed~-- coincidentally,
this is precisely the arrangement that one needs in order to proclaim that the state variable
$A\in\R^{\EEE}$ can be collapsed to the total population mean $A^*
= \sum_{u\tts\in\tts\EEE}\pi_x(u) A^u$ alone.
Indeed, with $h_{t,\tts x,\tts A,\tts u}^{ y ,v}\equiv h_{t,\tts x,\tts A}^{ y}$ the
system \eqref{Taylor-a} collapses to the single equation
\begin{subequations}\label{Taylor-x}
{\abovedisplayskip=5pt plus 1.5pt minus 1.5pt\belowdisplayskip=5pt plus 1.5pt minus 1.5pt\belowdisplayshortskip=5pt plus 1.5pt minus 1.5pt
%%- -%@@%--{Eq: 4.7}--%%
\begin{equation}\label{Taylor-x-a}
1=\b\,\sum\nolimits_{ y \tts \in\tts\XXX}
{ 1 \over h_{t,\tts x,\tts A}^{ y}}\times
\Bigl(\rho_ y \bigl(K_{t}(x, A)\bigr) + 1-\dd \Bigr)
\times  Q(x, y )
\end{equation}
}%
and with $b_{t,\tts x,\tts A,\tts u}\equiv b_{t,\tts x,\tts A}$ \eqref{lin-split-a}
collapses to the system of only $\abs{\XXX}$ equations
{\abovedisplayskip=5pt plus 1.5pt minus 1.5pt\belowdisplayskip=5pt plus 1.5pt minus 1.5pt\belowdisplayshortskip=5pt plus 1.5pt minus 1.5pt
\begin{equation}\label{Taylor-x-b}
{ {b_{t,\tts x,\tts A}}}\times
\Bigl(\rho_ y \bigl({ {K_{t}(x, A)}}\bigr) +1-\dd\Bigr) 
= { {h_{t,\tts x,\tts A}^{ y }}}
+ {{b_{t+1,\tts y ,\tts \T_{t,x}^y(A)}}}
\times{ {h_{t,\tts x,\tts A}^{ y}}}\,,\quad y\in\XXX\,,
\end{equation}
}%
\end{subequations}
for the total of $1+\abs{\XXX}$ unknowns, namely $b_{t,\tts x,\tts A}$ and $h_{t,\tts x,\tts A}^{ y}$,
$y\in\XXX$ (recall that the state $(x,A)$ is fixed). Remarkably, the system  \eqref{Taylor-x}
admits a closed-form solution~-- see \ref{closed-form} below. In~any
case, with a closed-form
solution or without, once the unknowns $b_{t,\tts x,\tts A,\tts u}\equiv b_{t,\tts x,\tts A}$ and
$h_{t,\tts x,\tts A,\tts u}^{ y,\tts v}\equiv h_{t,\tts x,\tts A}^{ y}$,
$y\in\XXX$  are solved for from \eqref{Taylor-x},
the system composed of \eqref{lin-split-b} and \eqref{Taylor-b} would provide a linear system
of $\abs{\EEE}+\abs{\EEE}^2\abs{\XXX}$ equations for the same number of
unknowns, namely
$
{ {a_{t,\tts x,\tts A,\tts u}}}\,, \ { {g_{t,\tts x,\tts A,\tts u}^{ y ,v}}}\,,$
$ u, v\in\EEE\,,\  y \in\XXX\,.$
If a solution for the unknowns
%%- -%@@%--{Eq: 4.8}--%%
{\abovedisplayskip=5pt plus 1.5pt minus 1.5pt\belowdisplayskip=5pt plus 1.5pt minus 1.5pt\belowdisplayshortskip=5pt plus 1.5pt minus 1.5pt
\begin{equation}\label{unkn2}
a_{t,\tts x,\tts A,\tts u}\,, \ b_{t,\tts x,\tts A}\,, \ g_{t,\tts x,\tts A,\tts u}^{ y ,v}\,,
 \ h_{t,\tts x,\tts  A}^{ y}\,,\quad  u, v\in\EEE\,,\  y \in\XXX\,,
\end{equation}
}%
can indeed be found (for a fixed state $(x,A)$) as described,
the ansatz choice for ${ {K_{t}(x, A)}}$ can then be tested
with  the market clearing condition %\ref{def-equil}-(\ref{ze9x})
%%
%%- -%@@%--{Eq: 4.9}--%%
{\abovedisplayskip=8pt plus 1.5pt minus 1.5pt\belowdisplayskip=8pt plus 1.5pt minus 1.5pt\belowdisplayshortskip=6pt plus 1.5pt minus 1.5pt
\begin{equation}\label{mclr-lin-a}
\sum\nolimits_{u\ts\in\ts\EEE}\gp_x(u)\bigl(a_{t,\tts x,\tts  A,\tts u}+b_{t,\tts x,\tts
A} A^u\bigr)={ {K_{t}(x, A)}}\,.
\end{equation}
}%
If the test fails, then the value for ${ {K_{t}(x, A)}}$ will need to be adjusted
accordingly and the procedure will need to be repeated until the last relation becomes
numerically acceptable. 
Due to the affine structure imposed on the conditional transition mappings, the transport encrypted in
\ref{main-q}-(\ref{zze666}) can now be stated as transport of the vector of employment means $A$
in the form%%%%%%%%%%%%%%%%%%%%%%
%%%%%%%%%%%%%%%%
\footnote{{}The change of variables formula gives:
$\displaystyle \int \a \d \csd^u(\,\hat\pfc_{t,\tts x,\tts A}^{ y,\tts v}(u,\a))
= \int \pfc_{t,\tts x,\tts A}^{ y ,v}(u,\tts\a) \d \csd^{u}(\a)\,.$}
%%- -%@@%--{Eq: 4.10}--%%
{\abovedisplayskip=8pt plus 1.5pt minus 1.5pt\belowdisplayskip=8pt plus 1.5pt minus 1.5pt\belowdisplayshortskip=5pt plus 1.5pt minus 1.5pt
\begin{equation}\label{no-fp-2-0}
\T_{t,x}^y( A)^v
= \sum\nolimits_{u\tts\in\tts\EEE}
{\gp_x(u)  P_{x, y }( u,v)\over \gp_ y (v)}\,\Bigl(g_{t,\tts x,\tts A,\tts u}^{ y ,v} + 
h_{t,\tts x,\tts  A,\tts u}^{y,v}\times  A^u\Bigr)\quad
\text{for all }\ x, y \in\XXX\,,\  v\in\EEE\,,
\end{equation}}%
%} 
where the left side is understood to be the mean of the distribution $\T_{t,x}^y(\csd)^v\in\DB$.
Hence, in this reduced (due to the affine structure)
model the transport operator $\T_{t,x}^ y $ introduced in \ref{et} now maps
$(\R_{++})^{\abs{\EEE}}$ into $(\R_{++})^{\abs{\EEE}}$, as opposed to mapping $\bbF^\EEE$ into
$\bbF^\EEE$. We again stress that the transport in \eqref{no-fp-2-0} is meaningful only in
conjunction with the system composed of \eqref{lin-split}, \eqref{Taylor} and \eqref{mclr-lin-a}~-- and the
system composed of \eqref{lin-split}, \eqref{Taylor} and \eqref{mclr-lin-a} depends on the transport
in \eqref{no-fp-2-0}~-- so that the system attached to period $t$ is composed of all 4 relations
\eqref{lin-split}, \eqref{Taylor}, \eqref{mclr-lin-a} and \eqref{no-fp-2-0}.
%

It is important to recognize that, as long as the slopes
$b_{t,\tts x,\tts A,\tts u}\equiv b_{t,\tts x,\tts A}$ and
$h_{t,\tts x,\tts A,\tts u}^{ y,\tts v}\equiv h_{t,\tts x,\tts A}^{ y}$ can be chosen to be
invariant to the choice of the employment states $u,v\in\EEE$, then \eqref{mclr-lin-a} can be cast as
%%
%%- -%@@%--{Eq: 4.11}--%%
{\abovedisplayskip=8pt plus 1.5pt minus 1.5pt\belowdisplayskip=8pt plus 1.5pt minus 1.5pt\belowdisplayshortskip=8pt plus 1.5pt minus 1.5pt
\begin{equation}\label{mclr-lin-b}
b_{t,\tts x,\tts A}\times A^*+\sum\nolimits_{u\ts\in\ts\EEE}\gp_x(u) a_{t,\tts x,\tts  A,\tts u} = {{K_{t}(x, A)}}
\end{equation}
}%
where $A^* = \sum_{u\tts\in\tts\EEE}\pi_x(u) A^u$ is the total population mean,  
and \eqref{no-fp-2-0} can be stated as
%%- -%@@%--{Eq: 4.12}--%%
{\abovedisplayskip=8pt plus 3.5pt minus 1.5pt\belowdisplayskip=8pt plus 3.5pt minus 1.5pt\belowdisplayshortskip=7pt plus 3.5pt minus 1.5pt
\begin{equation}\label{no-fp-x}
\T_{t,x}^y( A^*)\df\sum\nolimits_{v\tts\in\tts\EEE}\gp_y(v)\times\T_{t,x}^y( A)^v
= h_{t,\tts x,\tts A}^y\times  A^* + \sum\nolimits_{u,\tts v\tts \in\tts \EEE}
{\gp_x(u)\times  P_{x, y }( u,v)}\times g_{t,\tts x,\tts A,\tts u}^{ y ,v}
\end{equation}
}%
for every $x,y\in\XXX$. 

%%- -%##%--{No: 4.1}--%%
\begin{nit}{Collapse of the state space revisited}\label{collapse}
The main corollary from \eqref{mclr-lin-b} and \eqref{no-fp-x} is that, 
if the slopes $b_{t,\tts x,\tts A,\tts u}$ and $h_{t,\tts x,\tts A,\tts u}^{ y,\tts v}$ can be
chosen to be
invariant to choices of the employment states $u,v\in\EEE$, then the dependence on the state variable
$A\in\R^\abs{\EEE}$ collapses to dependence only on the total population mean
$A^*= \sum_{u\tts\in\tts\EEE}\pi_x(u) A^u$, i.e., the period-$t$ state of the economy can be expressed
as $(x,A^*)\in\XXX\times\Rpp$, the unknowns in \eqref{unkn2} can be written as
{\abovedisplayskip=8pt plus 2.5pt minus 2.5pt\belowdisplayskip=8pt plus 2.5pt minus 2.5pt\belowdisplayshortskip=5pt plus 1.5pt minus 1.5pt
$$
a_{t,\tts x,\tts A^*,\tts u}\,, \ b_{t,\tts x,\tts A^*}\,, \ g_{t,\tts x,\tts A^*,\tts u}^{ y ,\tts v}\,,
 \ h_{t,\tts x,\tts  A^*}^{ y }\,,\quad  u, v\in\EEE\,,\  y \in\XXX\,,
$$
}%
and, consequently, the average installed capital $K_t(x,A)$ can be cast as $K_t(x,A^*)$.
As a result, all variables that define the equilibrium can be written in terms of the total
population mean~$A^*$.
In particular, in both \eqref{mclr-lin-b} and \eqref{no-fp-x} the symbol $A$ can be replaced everywhere
with~$A^*$. 
While the idea of a ``collapsed state space'' is nothing new~-- see \cite{KS98}, for example~-- the present
analysis allows one to identify the main source of this feature: it comes from 
the first order Taylor approximation of the kernel
condition around ${1\over c}=0$, i.e., $c=\infty$, and from the resulting decoupling  of the system
composed of \eqref{lin-split} and \eqref{Taylor} into two sub-systems, one of which is self-contained.
This observation makes it possible to quantify the error introduced by collapsing the state space~--
see \ref{mod-accu} below. In addition, as we are about to see, it becomes possible to develop a new
computational strategy, which does not involve simulation and is meaningful for any, small or large,
time-horizon; in particular, it provides closed-form analytic expressions for the slopes
$b_{t,\tts x,\tts A^*}$ and $h_{t,\tts x,\tts A^*}^{ y}$, and, most important, provides the exact form
of the law of motion of the total population mean~$A^*$~-- see \eqref{no-fp-x}. 
Another interesting consequence from the approach developed in this section is that, even as the
state variable $A\in\R^\abs{\EEE}$ collapses to the scalar $A^*\in\Rpp$, it is still possible to
describe exactly the dynamics of the vector of employment-specific mean values~$A$. Indeed, for that purpose one
merely needs to replace in the right side of \eqref{no-fp-2-0} the coefficients
$g_{t,\tts x,\tts A,\tts u}^{ y ,\tts v}$ and $h_{t,\tts x,\tts  A,\tts u}^{ y ,\tts v}$ with,
respectively,
$g_{t,\tts x,\tts A^*,\tts u}^{ y ,\tts v}$ and $h_{t,\tts x,\tts  A^*}^{ y}$.
The main message here is that the collapse of the state space does not imply that the law of motion
of the population distribution could be chosen arbitrarily, as long as its mean complies with the
law of motion of the mean; in fact, the law of motion of the population distribution is fixed, once
the law of motion of the mean is fixed.\qed
\end{nit}

%%- -%##%--{No: 4.2}--%%
\begin{nit}{A more realistic endogenous state variable}\label{beyond}
The choice of the abscissa ${1\over c}=0$ for the Taylor expansion in the kernel condition
implies that equilibrium is sought exclusively
for relatively large consumption levels. Altho\-ugh this approach provides reasonably satisfactory
results in the examples given later in this section, it still leaves something to be desired. Another~--
perhaps somewhat more intuitive, depending on the point of view~-- approach is to develop
first-order Taylor expansion in the right side of \eqref{only-q-bn2-2} around
the abscissa ${1\over c}={1\over A^u}$. The idea is to ensure that the kernel condition is exact for
households that consume exactly at the mean of their employment group and is asymptotically exact
for households with consumption levels that are not very far from their group mean. One drawback
from this approach is that the decoupling of the system as in \eqref{lin-split}
and \eqref{Taylor} is no longer possible. In particular, the endogenous variable that
captures the state of the population will have to be taken to be
the entire vector of group-specific mean values
$A=(A^u,\,u\in\EEE)$ and can no longer be collapsed to the total mean value
$A^*= \sum_{u\tts\in\tts\EEE}\pi_x(u) A^u$ alone. In addition, one would be forced to seek a
numerical solution to a nonlinear system with twice as many equations (because the closed-form
solution for half of them will no longer be available). We are not going to pursue this
program for two main reasons. First, methodologically such a program would differ from the one carried out below
only in the increased computational complexity. Second, our primary objective here is to benchmark the
methodology developed in this \chptr\  to widely cited methods and results, most of which are based on the
infinite time horizon and what Krusell and Smith \cite{KS98} call ``approximate aggregation'' point
of view.

One possible arrangement that would still allow us to use the total population mean $A^*$ as an
endogenous state variable is to consider Taylor expansion around the abscissa ${1\over c}={1\over
A^*}$. One major objection to this approach is that it ignores the variation in the group-specific
mean values, as is illustrated in Figure~\ref{fgKS5} below.\qed
\end{nit}

The model reduction brought by the affine structure introduced in \eqref{lin-subst}
leads to some useful closed-form expressions for the slopes
$b_{t,\tts x,\tts A^*,\tts u}\equiv b_{t,\tts x,\tts A^*}$
$h_{t,\tts x,\tts  A^*,\tts u}^{y,v}\equiv h_{t,\tts x,\tts  A^*}^{y}$ which are developed next. 
As was noted earlier, we seek a solution for the slopes in the portfolios and the future
consumption mappings that do not depend on the state of employment, i.e., rely on Taylor's expansion
in the kernel condition around the abscissa ${1\over c}=0$. This allows us to
use \eqref{lin-split}, \eqref{Taylor}, \eqref{mclr-lin-b} and \eqref{no-fp-x} with the symbol $A$
replaced everywhere with $A^*$ and with the dependence of the slopes on the state of employment suppressed.
For the sake of better readability, we restate (for a fixed aggregate state
$(x,A^*)\in\XXX\times\Rpp$) these relations as
%
%
%
%%- -%@@%--{Eq: 4.13}--%%
{\abovedisplayskip=5pt plus 1.5pt minus 1.5pt\belowdisplayskip=8pt plus 1.5pt minus 1.5pt\belowdisplayshortskip=8pt plus 1.5pt minus 1.5pt
\begin{equation}\label{eq-b-1}
{ {b_{t,\tts x,\tts A^*}}}\times 
\Bigl(\rho_ y \bigl({ {K_{t}(x, A^*)}}\bigr) +1-\dd\Bigr) 
= { {h_{t,\tts x,\tts A^*}^{ y}}}
+ {{b_{t+1,\tts y ,\tts \T_{t,x}^y(A^*)}}}\times{ {h_{t,\tts x,\tts A^*}^{ y }}}\,,\quad
y \in\XXX\,,
\end{equation}
}%
%%- -%@@%--{Eq: 4.14}--%%
{\abovedisplayskip=5pt plus 1.5pt minus 1.5pt\belowdisplayskip=8pt plus 1.5pt minus 1.5pt\belowdisplayshortskip=8pt plus 1.5pt minus 1.5pt
\begin{equation}\label{eq-b-0}
\begin{aligned}
&{ {a_{t,\tts x,\tts A^*,\tts u}}}\times  
\Bigl(\rho_ y \bigl({ {K_{t}(x, A^*)}}\bigr) +1-\dd\Bigr) 
+v \times \ee_ y \bigl({ {K_{t}(x, A^*)}}\bigr)\\
&\hbox to2.5cm{\hfil}= {{a_{t+1,\tts y ,\tts \T_{t,x}^y( A^*),\tts v}}}+\bigl(1+{{b_{t+1,\tts y
,\tts \T_{t,x}^y( A^*)}}}\bigr)\times { {g_{t,\tts x,\tts A^*,\tts u}^{ y ,v}}}\,,\quad
&y \in\XXX\,,\  u, v\in\EEE\,,
\end{aligned}
\end{equation}
}%
%%- -%@@%--{Eq: 4.15}--%%
{\abovedisplayskip=5pt plus 1.5pt minus 1.5pt\belowdisplayskip=8pt plus 1.5pt minus 1.5pt\belowdisplayshortskip=8pt plus 1.5pt minus 1.5pt
\begin{equation}\label{eq-n-1}
1 = \b\,\sum\nolimits_{ y \tts \in\tts\XXX}
{ 1 \over h_{t,\tts x,\tts A^*}^{ y }}\times
\Bigl(\rho_ y \bigl(K_{t}(x, A^*)\bigr) + 1-\dd \Bigr)
\times  Q(x, y )\,,\quad y\in\XXX\,,
\end{equation}
}%
%%- -%@@%--{Eq: 4.16}--%%
{\abovedisplayskip=5pt plus 1.5pt minus 1.5pt\belowdisplayskip=8pt plus 1.5pt minus 1.5pt\belowdisplayshortskip=8pt plus 1.5pt minus 1.5pt
\begin{equation}\label{eq-n-0}
\begin{gathered}
0 = \sum_{ y \tts \in\tts\XXX, v\tts\in\tts\EEE}
{ -\b \over (h_{t,\tts x,\tts A^*}^{ y })^2}
\times g_{t,\tts x,\tts A^*,\tts u}^{ y,\tts v} \times
\Bigl(\rho_ y \bigl(K_{t}(x, A^*)\bigr) + 1-\dd \Bigr)
\times  Q(x, y )\,  P_{x,\tts y }( u ,v)\,,\quad  u\in\EEE\,,
\end{gathered}
\end{equation}
}%
%%- -%@@%--{Eq: 4.17}--%%
{\abovedisplayskip=5pt plus 1.5pt minus 1.5pt\belowdisplayskip=8pt plus 1.5pt minus 1.5pt\belowdisplayshortskip=8pt plus 1.5pt minus 1.5pt
\begin{equation}\label{eq-mclr}
b_{t,\tts x,\tts A^*}\times A^*+\sum\nolimits_{u\ts\in\ts\EEE}\gp_x(u) a_{t,\tts x,\tts  A^*,\tts u}
= {{K_{t}(x, A^*)}}\,,
\end{equation}
}%
and       
%%- -%@@%--{Eq: 4.18}--%%
{\abovedisplayskip=5pt plus 1.5pt minus 1.5pt\belowdisplayskip=8pt plus 1.5pt minus 1.5pt\belowdisplayshortskip=8pt plus 1.5pt minus 1.5pt
\begin{equation}\label{eq-T-star}
\T_{t,x}^y( A^*)
= h_{t,\tts x,\tts A^*}^y\times  A^* + \sum\nolimits_{u,\tts v\tts \in\tts \EEE}
{\gp_x(u)\times  P_{x, y }( u,v)}\times g_{t,\tts x,\tts A^*,\tts u}^{ y ,\tts v}\,,\quad y\in\XXX\,.
\end{equation}
}%



%%- -%##%--{No: 4.3}--%%
\begin{nit}{Closed form solution for the slopes}\label{closed-form}
As no investment takes place in period $T$, with  $t=T-1$ one must have
$a_{t+1,\tts y,\tts \T_{t,x}^y( A^*),\tts v}=0$ and $b_{t+1,\tts y,\tts \T_{t,x}^y( A^*)}=0$,
so that \eqref{eq-b-1} becomes
{\abovedisplayskip=5pt plus 1.5pt minus 1.5pt\belowdisplayskip=5pt plus 1.5pt minus 1.5pt\belowdisplayshortskip=5pt plus 1.5pt minus 1.5pt
\[
h_{t,\tts x,\tts  A^*}^{y} = b_{t,\tts x,\tts A^*} \times
\Bigl(\rho_ y \bigl({ {K_{t}(x, A^*)}}\bigr) +1-\dd\Bigr)\,,\quad  x,\,y \in\XXX\,,\  u,\, v\in\EEE\,.
\]}%
Consequently, \eqref{eq-n-1} gives %, for every $x\in\XXX$ and $u\in\EEE$,
{\abovedisplayskip=8pt plus 1.5pt minus 1.5pt\belowdisplayskip=8pt plus 1.5pt minus 1.5pt\belowdisplayshortskip=6pt plus 1.5pt minus 1.5pt 
\[
1 = { \b \over { {b_{t,\tts x,\tts A^*}}}}\sum\nolimits_{ y \,\in\,\XXX}
Q(x, y )\sum\nolimits_{ v\,\in\,\EEE} P_{x, y }( u ,v)=
{ \b \over { {b_{t,\tts x,\tts A^*}}}}\sum\nolimits_{ y\,\in\,\XXX} Q(x, y )
={ \b \over { {b_{t,\tts x,\tts A^*}}}}
\,,
\]}%
so that ${ {b_{T-1,\tts x,\tts A^*}}}=\b$ for every $x\in\XXX$.
Hence, with  $t=T-2$ \eqref{eq-b-1} gives
{\abovedisplayskip=8pt plus 1.5pt minus 1.5pt\belowdisplayskip=8pt plus 1.5pt minus 1.5pt\belowdisplayshortskip=6pt plus 1.5pt minus 1.5pt
\[
h_{t,\tts x,\tts A^*}^{ y } = {b_{t,\tts x,\tts  A^*}\over 1+ \b}\times
\Bigl(\rho_ y \bigl({ {K_{t}(x, A^*)}}\bigr) +1-\dd\Bigr) 
\,,\quad x,\,y \in\XXX\,.
\]}%
Using \eqref{eq-n-1} one more time with $h_{t,\tts x,\tts A^*}^{ y }$ from above we get
{\abovedisplayskip=5pt plus 1.5pt minus 1.5pt\belowdisplayskip=5pt plus 1.5pt minus 1.5pt\belowdisplayshortskip=5pt plus 1.5pt minus 1.5pt
\[
1 = { \b (1+\b)\over b_{t,\tts x,\tts A^*}}
\,,\quad x\in\XXX\,,
\]}%
so that ${ {b_{T-2,\tts x,\tts  A^*}}}=\b+\b^2$.
By induction, for every $n\ge 1$ and  $t=T-n$,
%%
{\abovedisplayskip=5pt plus 1.5pt minus 1.5pt\belowdisplayskip=5pt plus 1.5pt minus 1.5pt\belowdisplayshortskip=5pt plus 1.5pt minus 1.5pt
\begin{equation}\tag{a}\label{explicit-b}
b_{t,\tts x,\tts  A^*}\equiv b_t=\b+\b^2+\cdots+\b^n ={1-\b^{n+1}\over 1-\b}-1
={\b-\b^{T-t+1}\over 1-\b}\,.
\end{equation}}%
As a result, using \eqref{eq-b-1} yet again with $t=T-n$ we get 
%%
{\abovedisplayskip=8pt plus 1.5pt minus 1.5pt\belowdisplayskip=8pt plus 1.5pt minus 1.5pt\belowdisplayshortskip=6pt plus 1.5pt minus 1.5pt
\begin{equation}\tag{b}\label{explicit-h}
{h_{t,\tts x,\tts  A^*}^{ y}} =
{\b+\cdots+\b^n\over 1+\b+\cdots+\b^{n-1}}
\,\Bigl(\rho_ y \bigl({ {K_{t}(x, A^*)}}\bigr) +1-\dd\Bigr) =
\b\,\Bigl(\rho_ y \bigl({ {K_{t}(x, A^*)}}\bigr) +1-\dd\Bigr)\,.
\end{equation}}%
In particular, letting  $n\to\infty$ leads to the following
time-invariant values for the slopes of the portfolios and the employment-specific transition mappings:
%%
{\abovedisplayskip=5pt plus 1.5pt minus 1.5pt\belowdisplayskip=5pt plus 1.5pt minus 1.5pt\belowdisplayshortskip=5pt plus 1.5pt minus 1.5pt
\[
b_\infty={\b\over1-\b}\quad\text{and}\quad
h_{\infty,\tts x,\tts A^*}^{ y } = 
\b\,\Bigl(\rho_ y \bigl({ {K_{\infty}(x, A^*)}}\bigr) +1-\dd\Bigr) \,,
\]}%
provided, of course, that $K_{\infty}(x, A^*)\df \lim_{t\to\infty}K_t(x,A^*)$ exists.\qed
\end{nit}


In~general, 
the intercepts $a_{t,\tts x,\tts A^*,\tts u}$ and $g_{t,\tts x,\tts A^*,\tts u}^{y,\tts v}$
cannot be employment-invariant, and one can only hope that time-invariant versions of the mappings
%%
{\abovedisplayskip=5pt plus 1.5pt minus 1.5pt\belowdisplayskip=5pt plus 1.5pt minus 1.5pt\belowdisplayshortskip=5pt plus 1.5pt minus 1.5pt
\[
(x,u, A^*) \leadsto a_{t,\tts x,\tts A^*,\tts u}\,, \quad
(x,y,u,v, A^*) \leadsto g_{t,\tts x,\tts  A^*,\tts u}^{y,\tts v}\,,\quad\text{and}\quad
(x, A^*) \leadsto K_t(x, A^*)
\]
}%
exist in the limit as $t\to\infty$. 
Because of the implicit structure of the system composed of
\eqref{lin-split}, \eqref{Taylor}, \eqref{mclr-lin-a} and \eqref{no-fp-2-0},
the existence of these limits is
very difficult to establish generically by way of the usual fixed point argument.
Nevertheless, it will be shown below that, at least in the
benchmark example that we borrow here from~\cite{KS98}, the convergence
of the mappings above as $t\to\infty$ is rather easy to establish numerically, i.e., by following the general
iteration strategy from \ref{main-proc} for a sufficiently large number of periods one arrives at
successive copies of these mappings that coincide (in uniform distance) within a prescribed
threshold. We stress, however, that the procedure that we are about to outline and implement
does not require time invariance: the program returns equilibrium allocations for any time horizon, irrespective of
whether increasing the time horizon results into (numerically confirmed) convergence or not.

%%
%%
%%- -%##%--{No: 4.4}--%%
\begin{nit}{Remark}\label{T-inf}
It is clear from \eqref{no-fp-x} that if
$g_{\infty ,\tts x,\tts  A^*,\tts u}^{y,\tts v} = \lim_{t\to\infty}g_{t,\tts x,\tts  A^*,\tts
u}^{y,\tts v}$ were to exist, then time-invariant version of the transport
$\T_{\infty,x}^y( A^*) = \lim_{t\to\infty}\T_{t,x}^y( A^*)$ also exists.
However, the dependence of the transport on both the present productivity state $x$ and on 
the future one $ y $ does not go away.
As a result, the state variable $A^*$ fluctuates in the random environment of the productivity
shocks even when the time horizon is pushed to $\infty$.%%%%%%%%%%%%%%%%%%%%%%%
%%%%%%%%%%%%%%%%%%%%%%%%%%%%%%%%%%%%%%%%%%%%%%%%%%%%%%%%%%%%%%%
\footnote{As we are about to see below, fluctuations in the random environment of the productivity
shocks does not entail fluctuations in perfect sync with the productivity shocks.}%
%%%%%%%%%%%%%%%%%%%%%%%%%%%%%%%%%%%%%%%%%%%%%%%%%%%%%%%%%%%%%%
\qed
\end{nit}
%\nitskip


%%
%%
%%- -%##%--{No: 4.5}--%%
\begin{nit}{Krusell-Smith's strategy compared}\label{transport-K}
In the present setup, the transport of installed ca\-pi\-tal 
is given by the mapping
{\abovedisplayskip=5pt plus 3.5pt minus 1.5pt\belowdisplayskip=5pt plus 3.5pt minus 1.5pt\belowdisplayshortskip=5pt plus 3.5pt minus 1.5pt
\begin{equation}\tag{a}\label{4-2-a}
K_t(x,A^*) \leadsto K_{t+1}(y,\T_{t,x}^y( A^*))\,,
\end{equation}
}%
and since \eqref{eq-mclr} can be solved for $A^*$ against $K_t$, the second expression above is
a function of the first,
i.e., if
written in terms of average installed capital (the approach used in \cite{KS98})
the transport of the population mean
from period $t$ to period $t+1$ can be cast in the form
{\abovedisplayskip=5pt plus 3.5pt minus 1.5pt\belowdisplayskip=5pt plus 3.5pt minus 1.5pt\belowdisplayshortskip=5pt plus 3.5pt minus 1.5pt
\begin{equation}\tag{b}\label{4-2-b}
K \leadsto H_{t,x}^y(K)\,,\quad x,y\in\XXX\,.
\end{equation}
}%
In general, even if time invariant versions of the mappings $H_{t,x}^y\phd$ were to exist,
as was already noted, the
dependence on both the present and future productivity states, $x$ and~$y$, does not go away.
The computational strategy described in \cite{KS98} comes down to approximating the mappings
$H_{\infty,x}^y\phd$ by way of least-square log-linear fit from the simulated long-run behavior of a
large population of households, but in the actual implementation the dependence on~$x$ is ignored,
i.e., in the concrete example described in \cite{KS98}
there is only one log-linear line for every $y\in\XXX$.
To more indulgent eyes  such a shortcut may appear harmless 
because the choice of model parameters in \cite{KS98} is such
that  all mappings
$H_{\infty,x}^y\phd$, $x,y\in\XXX$, are very close in uniform norm
and are also very close to linear~-- see below.
However, there is no intuition to suggest that such an arrangement persists in general, and even
with parameter choices as in \cite{KS98} the mappings $H_{\infty,x}^y\phd$, $x,y\in\XXX$, are
not close enough to be declared numerically identical.

For the purpose of comparison and benchmarking,
the output from the computational
strategy developed in this section can be cast in terms of average capital instead of consumption as a
state variable (e.g., one can retrieve \eqref{4-2-b} from \eqref{4-2-a}).
The differences between the method proposed here and the one from \cite{KS98}
can be summarized as follows. First, assuming the affine
structure postulated in \eqref{lin-subst} is in place, 
the mappings $H_{t,x}^y\phd$ can be computed exactly,
i.e, within the accuracy of the numerical solver and the cubic spline interpolation,
for any $t<T$ and any $x,y\in\XXX$~-- without the need to restrict those mappings to a particular
type of functional dependence (e.g., log-linear) and without the need to simulate the individual
behavior of a large number of households.
The convergence (in uniform norm) $H_{t,x}^y\phd\to H_{\infty,x}^y\phd$ can then be
confirmed numerically. Most important, \ref{mod-accu} below provides a tool for estimating the
error from assuming the affine structure postulated in \eqref{lin-subst}, which is essentially the
error from assuming the arrangement known as ``representative agent.''\qed
\end{nit}  



Our next step is to reformulate the strategy from \ref{main-proc} in terms of
the setup adopted in the present section and the reduced form postulated
in \eqref{lin-subst}.
For the sake of simplicity the description that follows is written with infinite time horizon in
mind, but with the understanding that finite time horizon merely means interrupting the program before it
detects convergence.   

%%- -%##%--{No: 4.6}--%% 
\begin{nit}{Metaprogram with affine structure}\label{KS-proc}
Due to the explicit formulas in
\ref{closed-form}-\eqref{explicit-b} and \ref{closed-form}-\eqref{explicit-h},
the only unknowns that need to be computed
are the average installed capital $K_{t}(x, A^*)$ and the intercepts %(see \eqref{lin-subst})
$a_{t,\tts x,\tts  A^*,\tts u}$  and $g_{t,\tts x,\tts  A^*,\tts u}^{ y ,\tts v}$,
for all choices of $ u, v\in\nobreak\EEE\,,\ x,  y \in\XXX\,,$
and $t<T$. 
As these unknowns depend on the period $t$ average consumption across the population,
the objects we are looking for are functions of $ A^*\in\Rpp$ that are labeled by $t$, $u$, $v$,
$x$ and $y$.
The general program described in \ref{main-proc} comes down to the following steps:

%\noindent
{\it Initial setup:} Make an ansatz choice for the finite interpolation grid $\bbG\subset\Rpp$, the
elements of which represent reduced (to the total population mean) cross-sectional distributions of all households.

%\noindent
{\it Initial backward step:}  Set $n=1$ and $t=T-n$. For every $x\in\XXX$ do:

For every $ A^*\in\bbG$ perform steps~(1) through~(3) below, then
proceed to~(4):

{(1)} Make an ansatz choice for $K_{t}(x, A^*)>0$ and go to (2).

{(2)} Solve the system composed of \eqref{eq-b-0} with
$a_{t+1,\tts y ,\tts \T_{t,x}^y( A^*),\tts v}\equiv 0$
and \eqref{eq-n-0} (total of $\abs{\EEE}+\abs{\EEE}^2\abs{\XXX}$ equations~--
note that $x$ is fixed) for the unknowns ($\abs{\EEE}+\abs{\EEE}^2\abs{\XXX}$ in number)
$a_{t,\tts x,\tts A^*,\tts u}$ and $g_{t,\tts x,\tts A^*,\tts u}^{y,\tts v}$, $y\in\XXX$, $u,v\in\EEE$. Proceed to (4).

{(3)} Test the market clearing (see \eqref{eq-mclr} and recall that $b_{T-1,\tts x,\tts A^*,\tts u}=\b$)
{\abovedisplayskip=5pt plus 1.5pt minus 1.5pt\belowdisplayskip=5pt plus 1.5pt minus 1.5pt\belowdisplayshortskip=5pt plus 1.5pt minus 1.5pt
\[
K_{t}(x, A^*)=\b A^*+ \sum\nolimits_{u\ts\in\ts\EEE}\gp_x(u) a_{t,\tts x,\tts A^*,\tts u}\,.
\]
}%
If this relation fails by more than a prescribed threshold, set the new value of $K_{t}(x, A^*)$ to the
right side above and go back to (2).

{(4)} Construct spline interpolation objects%%%%%%%%%%%%%%%%%%%%%%%%%%%%%%%%%%
%%%%%%%%%%%%%%%%%%%%%%%%%%%%%%%%%%%%%%%%%%%
\footnote{{}Splines defined over the grid $\bbG$ are treated as functions on $\Rpp$ by way of extrapolation.}
%%%%%%%%%%%%%%%%%%%%%%%%%%%%%%%%%%%%%%%%%%%
over the grid $\bbG$ from the most recently calculated
values for $K_{t}(x, A^*)$, $a_{t,\tts x,\tts A^*,\tts u}$ and  $g_{t,\tts x,\tts A^*,\tts u}^{ y
,\tts v}$,
$ A^*\in\bbG$, for every $y\in\XXX$ and $u,v\in\EEE$. 


{\it Generic backward step:} Set $n=n+1$ and $t=T-n$,
assuming that $ A^* \leadsto a_{t+1,\tts y,\tts A^*,\tts v}$ are already computed functions
(splines with extrapolation)
on $\Rpp$  for every $ y \in\XXX$ and every $ v\in\EEE$. For every $x\in\XXX$ do:

For every $ A^*\in\bbG$ complete steps (1) through (5) below,
then proceed to~(6):

{(1)} Set $\pdg A^*_{y }= A^*$ for every $ y \in\XXX$ (initial guess for the future state of the
population in every future productivity state), then go to (2).

{(2)} Set $K_{t}(x, A^*)=K_{t+1}(x,A^*)$ (initial guess for the average installed capital taken from the
previous iteration), then go to (3).

{(3)} Solve the system composed of \eqref{eq-n-0} and
\eqref{eq-b-0} with $a_{t+1,\tts y,\tts \T_{t,x}^y( A^*),\tts v}=a_{t+1,\tts y,\tts \pdg A^*_{y },\tts v}$
 (total of $\abs{\EEE}+\abs{\EEE}^2\abs{\XXX}$ equations~--
note that $x$ is fixed) for the unknowns ($\abs{\EEE}+\abs{\EEE}^2\abs{\XXX}$ in number)
$a_{t,\tts x,\tts A^*,\tts u}$ and $g_{t,\tts x,\tts  A^*,\tts u}^{ y ,\tts v}$, $ y \in\XXX$, $ u, v\in\EEE$. Proceed to (4).

{(4)} Test the market clearing
(see \eqref{eq-mclr} and recall that $b_{T-n,x, A^*,u}=\b+\cdots+\b^n$)
{\abovedisplayskip=5pt plus 1.5pt minus 1.5pt\belowdisplayskip=5pt plus 1.5pt minus 1.5pt\belowdisplayshortskip=5pt plus 1.5pt minus 1.5pt
\[
K_{t}(x, A^*)= (\b+\cdots+\b^n)  A^*
+ \sum\nolimits_{u\ts\in\ts\EEE}\gp_x(u) a_{t,\tts x,\tts  A^*,\tts u}\,.
\]
}%
If this relation fails by more than a prescribed threshold, set the new value of $K_{t}(x, A^*)$ to the
right side and go back to~(3), otherwise proceed to~(5).

{(5)} Compute (see %
\eqref{eq-T-star}% 
)
{\abovedisplayskip=5pt plus 2.5pt minus 1.5pt\belowdisplayskip=5pt plus 2.5pt minus 1.5pt\belowdisplayshortskip=5pt plus 2.5pt minus 1.5pt
$$
\pddg A^*
= \b\,\Bigl(\rho_ y \bigl({ {K_{t}(x, A^*)}}\bigr) +1-\dd\Bigr)\times  A^* + \sum\nolimits_{u,\tts v\tts \in\tts \EEE}
{\gp_x(u)\times  P_{x, y }( u,v)}\times g_{t,\tts x,\tts A^*,\tts u}^{ y ,v}\,,
$$
}%
for every $ y \in\XXX$. If the uniform distance
{\abovedisplayskip=5pt plus 1.5pt minus 1.5pt\belowdisplayskip=5pt plus 1.5pt minus 1.5pt\belowdisplayshortskip=5pt plus 1.5pt minus 1.5pt
\[
\max\nolimits_{ y\ts\in\ts\XXX}\abs{\pddg A^*_{y} - \pdg A^*_{y}}
\]}%
exceeds some prescribed threshold (the guessed future averages are not compatible with the
transport),
set $\pdg A^*_{y}=\pddg A^*_{y}$ for all $y\in\XXX$
and go back to (2). 

{(6)} Construct spline interpolation objects over the grid $\bbG$ from the most recently calculated
values for $K_{t}(x, A^*)$, $a_{t,\tts x,\tts A^*,\tts u}$ and  $g_{t,\tts x,\tts A^*,\tts u}^{
y,\tts v}$,
$ A^*\in\bbG$, for every $y\in\XXX$ and $u,v\in\EEE$.

{\it Final backward step\/} (if looking for a time-invariant equilibrium): 
If at least one of the uniform distances
{\abovedisplayskip=5pt plus 1.5pt minus 1.5pt\belowdisplayskip=5pt plus 1.5pt minus 1.5pt\belowdisplayshortskip=5pt plus 1.5pt minus 1.5pt
\[
\begin{gathered}
\max_{x\ts\in\ts\XXX,\, A^*\ts\in\ts\bbG}\abs{K_{t+1}(x, A^*)-K_{t}(x, A^*)}\,,\ %
\max_{x\ts\in\ts\XXX,\,u\ts\in\ts\EEE,\, A^*\ts\in\ts\bbG}\abs{a_{t+1,\tts x,\tts A^*,\tts
u}-a_{t,\tts x,\tts A^*,\tts u}}\,,\\
\max_{x, y \ts\in\ts\XXX,\, u, v\ts\in\ts\EEE,\, A^*\ts\in\ts\bbG}\abs{g_{t+1,\tts x,\tts  A^*,\tts
u}^{ y ,\tts v}-g_{t,\tts x,\tts  A^*,\tts u}^{ y ,\tts v}}
\end{gathered} 
\]}%
exceeds some prescribed threshold, perform another generic backward step.%%%%%%%%%%%%%%%%%%%
%%%%%%%%%%%%%%%%%%%%%%%%%%%%%%%%%%%%%%%%%%%%%%
\footnote{{}The time parameter $t$ may become negative~-- the program moves backward in time for as
many periods as needed to achieve time invariance.
The final backward step would not be necessary if seeking time invariant
transport is not the objective, in which case the program can be terminated at $t=0$.}
%%%%%%%%%%%%%%%%%%%%%%%%%%%%%%%%%%%%%%%%%%%%%%
Otherwise stop and define
the functions
{\abovedisplayskip=5pt plus 1.5pt minus 1.5pt\belowdisplayskip=5pt plus 1.5pt minus
1.5pt\belowdisplayshortskip=5pt plus 1.5pt minus 1.5pt
$$
 A^* \leadsto K_{\infty}(x, A^*)\,,\quad
 A^* \leadsto a_{\infty,x, A^*,u}\,,\quad
 A^* \leadsto g_{\infty,x, A^*,u}^{ y ,v}\,,\quad x,y\in\XXX\,,\ \ u,v\in\EEE\,,
$$
}%
as the most recently computed
spline objects (with the latest value for $t$).\qed  
\end{nit}
%\nitskip


%%- -%##%--{No: 4.7}--%% 
\begin{nit}{Endless loops warning and disclaimer}\label{endless-loop-4}
There are no theoretical results to guarantee that the iterations between steps (3) and (4) and (3)
and (5) converge, or to guarantee that step (3) is always feasible, i.e., a numerical solution to the system
composed of \eqref{eq-b-0} and \eqref{eq-n-0} always exists, for every possible choice of
the model parameters from some reasonably wide range.\qed   
\end{nit}     
 
 
%%- -%##%--{No: 4.8}--%% 
\begin{nit}{Remark}\label{MFG-rem-4}
The iterations between steps (3) and (5) in the generic backward step are meant
to establish the correct connection between future and present
distribution averages, i.e., figuratively speaking, meant to ensure 
that the transport of the mean is self-aware~-- as it should~be. 
We stress that the 
adjustments that ensure self-awareness are local in time, in that the program does not
move to the next period in the backward induction (which is the previous period in real time)
until the correct (i.e., fully self-aware)
transport from the current period is established~-- recall that the transport is
time dependent and may become time invariant only in the limit.\qed
\end{nit}




%%- -%##%--{No: 4.9}--%%
\begin{nit}{Model accuracy}\label{mod-accu} 
Despite the reduction to a model with affine structure, the program outlined
in \ref{KS-proc} solves exactly, meaning, within the nonlinear solver's tolerance of the
infinity norm of the residuals, all equations that define the equilibrium~-- except for the kernel
condition \ref{cross-sys}-(\ref{zze5a}), which was replaced by \eqref{only-q-bn2-2}.
The solution was then sought in such a way that \eqref{only-q-bn2-2} is
approximately accurate for sufficiently large consumption levels~$c$.
Hence, quantifying the accuracy of the program comes down to
calculating the aberration in the right side of \eqref{only-q-bn2-2} from~$1$~-- for consumption
levels in some more realistic range. Recall that the economic interpretation of the kernel
condition comes down to the claim that all agents agree on the prices at which all securities are
traded. In~the
present setup this feature is tantamount to an agreement about the average of all private capital
investments, which, ultimately, boils down to an agreement about future returns and wages. Thus, an
aberration from $1$ in the right side of \eqref{only-q-bn2-2} has the effect that for some (not too
large) consumption levels the present period marginal utility from consumption would be smaller or larger
than the expected and discounted future marginal utility from consumption, i.e., agents with
relatively small consumption levels consume less or more than what would be optimal for them in perfect
equilibrium. Loosely speaking, the market arrangement favors the objectives of the big spenders,
i.e., the wealthy. There is no intuition to suggest that such arrangements occur in practice, nor is
there an intuition to suggest that in practice capital markets attain equilibrium allocation so
perfectly that the consumption level of every agent is exactly the one provided by perfect equilibrium.
The method developed in the present \chptr\  cannot~-- and does not attempt to~-- address such matters.
Nevertheless, it provides a framework within which the deviation of the representative agent point
of view from the theoretical (perfect) equilibrium model can be quantified. 
Indeed, with infinite time horizon in mind, the right side of
\eqref{only-q-bn2-2} can be treated as a function of the collective state 
of the
population $A^*$ and the consumption level $c$, i.e., can be cast as
{\abovedisplayskip=7pt plus 1.5pt minus 1.5pt\belowdisplayskip=7pt plus 1.5pt minus 1.5pt\belowdisplayshortskip=5pt plus 1.5pt minus 1.5pt
\begin{equation*}\tag{a}\label{k-check}
R_{\infty,x,u}(A^*,c) \df \sum_{ y\ts \in\ts\XXX,\ts v\ts\in\ts\EEE} \ 
{ \b\,\bigl(\rho_ y \bigl(K_{\infty}(x, A^*)\bigr) + 1-\dd\bigr)
\over g_{\infty,\tts x,\tts A^*,\tts u}^{ y ,v}/c  + \b\,\bigl(\rho_ y \bigl(K_{\infty}(x, A^*)\bigr) + 1-\dd\bigr)}
Q(x, y )\,  P_{x, y }( u ,v)\,.
\end{equation*}}%
Once the program in \ref{KS-proc} completes,
the value $R_{\infty,x,u}(A^*,c)$ can be calculated for every $c>0$ and for every $ A^*$ in
the long-run range of the population average. There is no obvious choice for the consumption level
$c$, since the only restriction for this variable
is on its distribution across the population.
One possible approach is to calculate the time invariant version (see~\ref{T-inf}) of the transport
in \eqref{no-fp-2-0} and simulate a series of long-run realizations of the vectors $A=(A^u\in\Rpp,
u\in\EEE)$ and the scalars $A^*\df \sum_{u\tts\in\tts \EEE}\pi_x(u)A^u$. The maximum over the
simulated series of the associated quantities $\abs{R_{\infty,x,u}(A^*, A^u)-1}$, $u\in\EEE$, will
then give an estimate of how far from the hypothetical equilibrium is the arrangement in which all
households in the same employment category adopt identical consumption levels. In the illustration
used in this section, which is borrowed from \cite{KS98} (see below), this estimate gives an error of order
$10^{-4}$. Unfortunately, such an estimate can be used only on a case by case basis, i.e,
it does not allow one to conclude that the conjecture
``only the mean and its transport matter'' is reasonably
accurate for a broad class of models. The main difficulty is that
tools to allow for general (not model-specific)
estimates of the long-run range of the conditional averages $A^u$, $u\in\EE$,
are yet to be developed.  

Another,
similar in spirit but more demanding, estimate for the kernel aberration is to compute the
expressions $R_{\infty,x,u}(A^*,\bar c_{x, A^*,u})$ with  $\bar c_{x, A^*,u}$ defined as the investment threshold
for households in employment state $u\in\EEE$,
i.e., the solution to $\qq_{\infty,x, A^*,u}(c)=0$. In our reduced model this is nothing but the intersection of the
line
$c \leadsto a_{\infty,x, A^*,u}+b_{\infty,x, A^*,u}\times c$ with the horizontal axis, i.e., 
{\abovedisplayskip=5pt plus 1.5pt minus 1.5pt\belowdisplayskip=5pt plus 1.5pt minus 1.5pt\belowdisplayshortskip=5pt plus 1.5pt minus 1.5pt
\[
\bar c_{x, A^*,u} = -{a_{\infty,x, A^*,u}\over b_{\infty,x, A^*,u}} = -{(1-\b) a_{\infty,x, A^*,u}\over \b}\,.
\]}%
In the example borrowed here from \cite{KS98} 
this estimate, too, leads to a maximum aberration across the simulated sample  
of order $10^{-4}$.
Appendix \ref{sec:App-B} elaborates on the consumption range in the model with only two employment
states and no risk-free lending. It also shows that in such models borrowing of capital is not
consistent with the notion of equilibrium (in the model studied in \cite{KS98} capital cannot be
borrowed by assumption). We stress, however, that a model with only two employment and two
productivity states is rather narrow in scope.\qed
\end{nit}


To put the methodology developed in the present section to the test,
we now turn to some concrete examples and numerical results.
Since the objective here is to benchmark the new method against those that precede it, in what
follows we focus exclusively on the infinite time horizon case, but stress that the program
developed here is designed to work only with finite time horizon (of any length), and ``infinite
time horizon'' is understood as a ``sufficiently large finite time horizon.'' In all illustrations
below the time horizon is set to $T=1\hbox{,}000$, except for the simulated sample, in which the
time horizon is $T=1\hbox{,}100\hbox{,}000$.
We borrow the general setup and parameter values
from the benchmark economy described in
\cite{KS98}: the list of productivity states is $\XXX=\{1.01,\,0.99\}$
(the economy is either in high state, labeled~``$1$,'' or in low state, labeled ``$2$''), with
transition probability matrix for these states
{\abovedisplayskip=8pt plus 1.5pt minus 1.5pt\belowdisplayskip=8pt plus 1.5pt minus 1.5pt\belowdisplayshortskip=8pt plus 1.5pt minus 1.5pt
\[
 Q=
\begin{bmatrix}
 0.875 & 0.125\\
 0.125 & 0.875
\end{bmatrix}\,,
\]}%
the list of employment states is $\EEE=\{\h,0\}\df \{0.3271,0.0\}$ (the agents can be
either employed or unemployed), with (private)
conditional transition probability matrices for these states
{\abovedisplayskip=5pt plus 1.5pt minus 1.5pt\belowdisplayskip=5pt plus 1.5pt minus 1.5pt\belowdisplayshortskip=5pt plus 1.5pt minus 1.5pt
\begin{gather*}
 P_{1,1}=\begin{bmatrix}
 0.972222 & 0.0277778\\
 0.666667 & 0.333333
 \end{bmatrix}\,,\quad
 P_{1,1}=\begin{bmatrix}
 0.927083 & 0.0729167\\
 0.25   &   0.75
\end{bmatrix}\,,
\end{gather*}
\begin{gather*}
 P_{2,1}=\begin{bmatrix}
 0.983333 & 0.0166667\\
 0.75    &  0.25
  \end{bmatrix}\,,\quad
 P_{2,2}=\begin{bmatrix}
 0.955556 & 0.0444444\\
 0.4     &  0.6
\end{bmatrix}\,,
\end{gather*}}%
which correspond to $\gp_1= [0.96, 0.04]$ and  $\gp_2= [0.9, 0.1]$. The parameter values are $\b=0.99$,
$\dd=0.025$, $\a=0.36$, $R=1$ (risk aversion). The total labor supplied in high state is
$L(\EEE_1)=0.314016$ and in low state it is $L(\EEE_2)= 0.29439$.
The time-invariant solution, i.e., the functions
{\abovedisplayskip=5pt plus 1.5pt minus 1.5pt\belowdisplayskip=5pt plus 1.5pt minus
1.5pt\belowdisplayshortskip=5pt plus 1.5pt minus 1.5pt
$$
 A^* \leadsto K_{\infty}(x, A^*)\,,\quad
 A^* \leadsto a_{\infty,x, A^*,u}\,,\quad
 A^* \leadsto g_{\infty,x, A^*,u}^{ y ,v}\,,\quad x,y\in\XXX\,,\ \ u,v\in\EEE\,,
$$
}%
are constructed as $1$D-splines on a fixed interpolation grid $\bbG\subset \R_{++}$.
The choice of the corresponding domain $[\bbG]$ is ad hoc, but chosen so that it contains the
simulated long-run range of the total population mean~$A^*$.  
The metaprogram in \ref{KS-proc}
was translated into the Julia programming language with parallelization.
With parallelization on $16$ CPUs the program completes $10^3$ iterations under $3$ minutes
and achieves convergence (in uniform distance over all grid-points between the last two
copies of the respective values) of $4.31651\times 10^{-5}$ across all values
$(a_{\infty,x, A^*,u},\, A^*\in\bbG)$, $7.18267\times 10^{-9}$ across all values
$(g_{\infty,x, A^*,u}^{ y ,v},\, A^*\in\bbG)$, $1.24589\times 10^{-8}$ across all values
$(K_\infty(x, A^*),\, A^*\in\bbG)$,
and $2.42939\times 10^{-10}$ across all transports $(\T_x^{ y,v}( A^*),\, A^*\in\bbG)$. 
In~all iterations the function tolerance in the nonlinear
solver%%%%%%%%%%%%%%%%%%%%%
%%%%%%%%%%%%%%%%%%%%%%%%%%%%%%%%%%%%
\footnote{{}The NLsolve package was used despite the fact that in the only-the-mean-matters scenario
the system to solve is linear. This choice was deliberate and meant to make the computer code
usable in other scenarios~-- see~\ref{beyond}.}
%%%%%%%%%%%%%%%%%%%%%%%%%%%%%%%%%%%%%%%
was set to $10^{-12}$.
The plots in Figure~\ref{fgKS10} show the portfolio intercepts $a_{\infty,x, A^*,u}$ as functions of $A^*$ for
all choices of $x\in\XXX$ and $u\in\EEE$.  
%%
%%
%%%%%%%%%%%%%%%%%%%%%%%%%%%%%%%%%%%%%%%%%%
%%     Fg: 10
%%%%%%%%%%%%%%%%%%%%%%%%%%%%%%%%%%%%%%%%%%  
{\captionsetup{belowskip=-5pt}
%\captionsetup{aboveskip=10pt}   
\begin{figure}[!htbp]
\centering
\begin{subfigure}{.5\textwidth}
  \centering
\toshow{\includegraphics[width=7.5cm]{fg-KS-10L}}
\end{subfigure}%
\begin{subfigure}{.5\textwidth}
  \centering 
\toshow{\includegraphics[width=7.5cm]{fg-KS-10R}  }      
\end{subfigure}
\caption{Intercepts of the portfolio lines in high state (left plot) and low state (right
plot) for employed (solid lines) and unemployed (doted lines) shown as functions of the total population
mean (over consumption)~$A^*$.}
\label{fgKS10}
\end{figure} }%
%%
%%
The fact that the intercepts decrease as the population average $A^*$ increases implies that the
portfolio lines, which give the private demand for capital as a function of the private consumption
level, shift downwards as the aggregate consumption level across the population increases (recall
that the slopes are fixed). 
The plot in Figure~\ref{fgKS11} shows the average installed capital $K_{\infty}(x, A^*)$ as a function of
$A^*$ for all $x\in\XXX$.
%%
%%
%%%%%%%%%%%%%%%%%%%%%%%%%%%%%%%%%%%%%%%%%%
%%     Fg: 11
%%%%%%%%%%%%%%%%%%%%%%%%%%%%%%%%%%%%%%%%%%
{\captionsetup{belowskip=-5pt}
%\captionsetup{aboveskip=10pt}   
\begin{figure}[!htbp]
\centering
\toshow{\includegraphics[width=7.5cm]{fg-KS-11}}
\caption{Average installed capital in high state (solid lines)
and low state (doted lines) shown as a function of the population consumption average $A^*$.
The uniform distance between the two lines is around 0.08574.}
\label{fgKS11} 
\end{figure} }%
%%
%%


The transport mappings from \eqref{eq-T-star}~-- the main computation tool in the approach developed
in this section~-- are illustrated in Figure~\ref{fgKS12}.
%%
%%
%%%%%%%%%%%%%%%%%%%%%%%%%%%%%%%%%%%%%%%%%%
%%     Fg: 12
%%%%%%%%%%%%%%%%%%%%%%%%%%%%%%%%%%%%%%%%%%
{\captionsetup{belowskip=-5pt}
%\captionsetup{aboveskip=10pt}   
\begin{figure}[!htbp]
\centering
\begin{subfigure}{.5\textwidth}
  \centering
\toshow{\includegraphics[width=7.5cm]{fg-KS-12L}}
\end{subfigure}%
\begin{subfigure}{.5\textwidth}
  \centering 
\toshow{\includegraphics[width=7.5cm]{fg-KS-12R}  }      
\end{subfigure}
\caption{The transport of the population consumption mean into high state (left plot) and into
low state (right plot) from high state (solid lines) and from low state (doted lines).
The uniform distance between the solid and the dotted line is around 0.00354
in the left plot and around 0.00353 in the right plot. The
uniform distance between the two solid lines is around 0.00663.}
\label{fgKS12}
\end{figure} }%
%%
%%
It is clear from Figure~\ref{fgKS12} that the dependence
of the transport $\T_{\infty,x}^y$ on the present aggregate state $x$ is negligible,
and so is also its dependence on the future aggregate
state, though the latter is still some two orders of magnitude larger. While
all four lines in Figure~\ref{fgKS12} are approximately identical, there is no reason for this feature
to persist with other choices for the model parameters~--
especially if the numbers of the exogenous aggregate
and idiosyncratic states are increased (as was noted earlier, in the concrete implementation of
Krusell-Smith's algorithm in [\cited{KS98}-III] the dependence of the transport on the starting
aggregate state is ignored). 


The transport of the population mean in terms of consumption, shown in Figure~\ref{fgKS12}, allows
one to produce the transport of the population mean in terms of average installed capital~--
see \ref{transport-K}~-- which is shown in Figure~\ref{fgKS13}.  
%%
%%
%%%%%%%%%%%%%%%%%%%%%%%%%%%%%%%%%%%%%%%%%%
%%     Fg: 13
%%%%%%%%%%%%%%%%%%%%%%%%%%%%%%%%%%%%%%%%%% 
{\captionsetup{belowskip=-5pt}
%\captionsetup{aboveskip=10pt}   
\begin{figure}[!htbp]
\centering
\begin{subfigure}{.5\textwidth}
  \centering
\toshow{\includegraphics[width=7.5cm]{fg-KS-13L}}
\end{subfigure}%
\begin{subfigure}{.5\textwidth}
  \centering 
\toshow{\includegraphics[width=7.5cm]{fg-KS-13R}  }      
\end{subfigure}
\caption{The transport of the population capital investment mean into high state (left plot) and into
low state (right plot) from high state (solid lines) and low state (doted lines).
The circles show Krusell-Smith's
 prediction [\protect\cited{KS98}-III] by way of log-linear least square fit from a simulated long-run behavior of
a large population of households. The uniform distance between the solid and the dotted line is around 0.00076
in the left plot and around 0.00083 in the right plot. The largest
disagreement between
Krusell-Smith's prediction and the solid lines is around 0.051 in the left plot and around 0.032 in the
right plot. The uniform distance between the two solid lines is around 0.06952.}
\label{fgKS13}
\end{figure} }% 
%%
%%
This transport is remarkably close to Krusell-Smith's prediction by way of least-square
log-linear fit~-- as it should be, since the latter obtains from the simulated behavior of a large
number of households over a large time horizon and the true transports are very close to affine (in this
particular example%%%%%%%%%%%%%%%%%%%%%%%%%%%%%%%%%%%%%%%%%%%
%%%%%%%%%%%%%%%%%%%%%%%%%%%%%%%%%%%%%%%%%%%%%%%%%%%%%%%
\footnote{There is no intuition to suggest that the transport mappings shown in Fig.~\ref{fgKS13}
would be nearly affine and nearly identical with any choice of the model parameters whatsoever.
There is therefore no reason to expect that the predictions produced by Krusell-Smith's approach
would always be as accurate as they are in the benchmark case study in the paper \cite{KS98}.}%%%%%%%%%%%
%%%%%%%%%%%%%%%%%%%%%%%%%%%%%%%%%%%%%%%%%%%%%
).



With the transformations shown in Figure~\ref{fgKS12}
at hand~-- see \eqref{eq-T-star}~-- one can easily simulate the long run behavior of the population
consumption mean 
by merely simulating a time series of the productivity state (not of the private states of a large
population of households) and by applying one of the
four transformations in Figures~\ref{fgKS12} consecutively from some arbitrary initial mean value (one
must test empirically that the choice of the initial value has no effect on the long run
distribution).
With the time series of the productivity state and the population mean at hand,
one can generate the corresponding series of employment specific mean values $(A^\h,A^0)\in\R^2$~--
see \eqref{no-fp-2-0} and the comment in~\ref{collapse}.
Starting with 
$A^\h=0.8$ and $A^0=0.7$ in high productivity state,
the bi-variate series of employment-specific mean values was simulated for $1.1$ million periods.
The output from the last $10^4$ periods is shown in~Figure~\ref{fgKS5},
which exhibits a typical law of motion in random environment: with high
probability the productivity state remains unchanged from one period to the next, so that the law of
motion of the population state is deterministic and governed either by
the mapping $\T_{\infty,1}^1\phd$ or by $\T_{\infty,2}^2\phd$ from \eqref{no-fp-2-0}~--
until a change in the productivity state occurs, in which case the population state is transformed either by
$\T_{\infty,1}^2\phd$ or by $\T_{\infty,2}^1\phd$.%%%%%%%%%%%%%%%%%%%%%%%%%
%%%%%%%%%%%%%%%%%%%%%%%%%%%%%%%%%%%%%%%%%%%%%%%%%%%%
\footnote{{}With a slight abuse of the notation we use the same token $\T_{\infty,x}^y\phd$ to denote
the transport of the vector of employment specific mean values from \eqref{no-fp-2-0} and also the
transport of the total population mean from \eqref{eq-T-star}. The precise meaning is given by the context.}
%%%%%%%%%%%%%%%%%%%%%%%%%%%%%%%%%%%%%%%%%%%%%%%%%%%%
%%
%%
%%%%%%%%%%%%%%%%%%%%%%%%%%%%%%%%%%%%%%%%%%
%%   Fg: 14
%%%%%%%%%%%%%%%%%%%%%%%%%%%%%%%%%%%%%%%%%%     
{%\captionsetup{belowskip=-10pt}
%\captionsetup{aboveskip=0pt}
\begin{figure}[!htbp]  
\centering
\begin{subfigure}{.5\textwidth}
  \centering
\leavevmode\raise0.8cm\hbox{\rotatebox{90}{\tiny average consumption unemployed}}%
\ %  
\toshow{\includegraphics[width=7.1cm]{fg-KS-14L}}

\leavevmode\smash{\raise6pt\hbox{\tiny average consumption employed}}
\end{subfigure}%
\begin{subfigure}{.5\textwidth}
  \centering
\leavevmode\raise1cm\hbox{\rotatebox{90}{\tiny average investment unemployed}}%
\ %
\toshow{\includegraphics[width=7.1cm]{fg-KS-14R}}
 
\leavevmode\smash{\raise6pt\hbox{\tiny average investment employed}} 
\end{subfigure}
\caption{The employment-specific population means in the last 10,000 periods in a simulated
series of $1.1$ million periods.}
\label{fgKS5}
\end{figure} }%
%%
%%
%% 
%%
The nearly affine patterns in Figure~\ref{fgKS5} are easily explained by the fact
that the mappings in \eqref{no-fp-2-0} are very close to affine. As~a result, the
disparity between employed and unemployed, whether in terms of consumption or wealth,
has a nearly affine structure that remains
unchanged for as long as the productivity state remains the
same (whence the straight lines), but that structure changes when the productivity state flips.
It is interesting to
note that these  fluctuations are much more dispersed than the fluctuations
in the random environment (the productivity state) that is causing them, which has
only two values (high and low)~-- a ``ratchet effect'' of a sort.
The larger dispersion in the data presented in the right plot in Figure~\ref{fgKS5},
in which the state of the population is described in terms of
the employment-specific mean investment level, is quite intuitive: the households' savings function
as ``shock absorbers.''
Perhaps the most
important takeaway from these plots is that that they reveal a structure that would not be possible to capture
if the state of the population is collapsed to a single scalar value, whether that value
is the mean consumption level or the mean investment level across the entire
population.%%%%%%%%%%%%%%
%%%%%%%%%%%%%%%%%%%%%%%%%%%%%%%%%%%%%%%%%%%%%%%%%%%%%%%%%%%%%%%%%%%%
\footnote{Applying Krusell-Smith's approach with higher order moments of the population
distribution, as originally proposed
in \cite{KS98}, would not allow one to compare the employment specific averages.}
%%%%%%%%%%%%%%%%%%%%%%%%%%%%%%%%%%%%%%%%%%%%%%%%%%%%%%%%%%%%%%%%%%%%
Of course, the simulated employment-specific mean values from the
right plot in Figure~\ref{fgKS5} can easily be
transformed into total population mean values, shown in Figure~\ref{fgKS-IK}.
%%
%%
%%%%%%%%%%%%%%%%%%%%%%%%%%%%%%%%%%%%%%%%%%
%%     Fg: 15
%%%%%%%%%%%%%%%%%%%%%%%%%%%%%%%%%%%%%%%%%%
{\captionsetup{belowskip=-5pt}
%\captionsetup{aboveskip=10pt}   
\begin{figure}[!htbp]
\centering
\toshow{\includegraphics[width=7.5cm]{fg-KS-15}}
\caption{Average installed capital in the last 10,000 periods in a simulated sample of 1.1 million periods.}
\label{fgKS-IK}  
\end{figure} }%
%%
%%
This plot shows the long-run range of the average installed capital and, most important,
reveals fluctuations that are much more dispersed  than the fluctuations in the exogenous productivity
state. In particular, the plot shows (empirically) that the pair composed of the aggregate
productivity state and the capital investment of a ``representative household'' (if one exists)
does not have a
stationary long-run distribution, assuming that the cross-sectional distribution of all households can
be identified as the probability distribution of a single representative household.
Indeed, if that would be the case, then there would be only one (conditional) population
average to attach to each aggregate state in the long run. Consequently, the plot in
Figure~\ref{fgKS-IK} would be showing fluctuations between only two points, as there are only two
productivity states in this example.  


Although it is assumed throughout this \chptr\  that the discount (impatience) factor $\b$ is
constant, apart from the desire for greater simplicity,
nothing in the general computational strategy that we have deployed makes such an assumption
necessary. A model with stochastic $\b$ is introduced and discussed extensively
in \cite{KS98}.
While the exploration of such models (and they are many~-- see \cite{KS98})
falls outside the scope of this
\chptr, a feature that can be illustrated here with very little additional effort is to rerun the Julia program
employed in the foregoing with a different value for the impatience parameter, namely with
$\b=0.96$, instead of $\b=0.99$, which was taken from the benchmark case in~\cite{KS98}.
For the sake of brevity, we produce here only the output from the simulation of the bi-variate
state of the population~-- see Figure~\ref{fgKSlast}.
%%
%%
%%%%%%%%%%%%%%%%%%%%%%%%%%%%%%%%%%%%%%%%%%
%%    Fg: 16
%%%%%%%%%%%%%%%%%%%%%%%%%%%%%%%%%%%%%%%%%%
{%\captionsetup{belowskip=-5pt}
%\captionsetup{aboveskip=0pt}
\begin{figure}[!htbp]  
\centering
\begin{subfigure}{.5\textwidth}
  \centering
\leavevmode\raise0.85cm\hbox{\rotatebox{90}{\tiny average consumption unemployed}}%
\ %  
\toshow{\includegraphics[width=7.1cm]{fg-KS-16L}}

\leavevmode\smash{\raise6pt\hbox{\tiny average consumption employed}}
\end{subfigure}%  
\begin{subfigure}{.5\textwidth}
  \centering
\leavevmode\raise0.85cm\hbox{\rotatebox{90}{\tiny average investment unemployed}}%
\ %
\toshow{\includegraphics[width=7.1cm]{fg-KS-16R}} 

\leavevmode\smash{\raise6pt\hbox{\tiny average investment employed}} 
\end{subfigure}
\caption{The employment specific population means in the last 10,000 periods in a simulated
series of $1.1$ million periods with impatience rate $\b=0.96$ (changed from $\b=0.99$).}
\label{fgKSlast}
\end{figure}
}%
%% 
%%
%% 
These plots are consistent with the intuition: when the households are more
impatient they invest less, which then leads to a lower output, and ultimately to lower consumption.
What is surprising,
however, is that decreasing the discount factor by only $3\%$ can lead to substantially lower
investment and consumption levels. It~is also interesting to note that increased impatience leads
to a greater variations in the relative disparity between unemployed and employed~-- in both
consumption and capital investment.
%%
%%
%%

%%% chend556677


%%- -%%--{Sec: #5}--%%
%{No: 5.}   %%llabel

\section{Conclusion}
\label{sec:CONC}\setcounter{paragraph}{0}

\noindent
In one way or another, the intrinsic nature of all incomplete-market heterogeneous agent models is deeply
rooted in connections
between the costate (shadow) variables associated with the individual savings problems.
The representative agent point of view and the classical dynamic
programming formulation of such models were borrowed from other disciplines and
cannot adequately capture those connections across time and
across a large population of households~-- even if stated in terms of the maximum principle or
interpreted as mean field games.
%In~particular, it cannot adequately reflect the requirement for the transport to be self-aware.
An alternative approach, which one may call ``self-aware shadow programming,'' 
% designed primarily with macroeconomic problems in mind,
was proposed in \cite{DL12} and was further expanded and revised in the present
\chptr\ to accommodate for a large population of households.
The innovation in the new approach can be summarized in three principle aspects.
First, it is built around the notion of equilibrium introduced
in \cite{DL12} and does not
rely on the concept of Nash equilibrium (it was noted earlier that the latter is not meaningful in
the context of heterogeneous models with finitely many agents). 
Second, it provides a simultaneous solution to a very large number of optimization problems the
structure of which is not known in the outset (figuratively speaking, solves an HJB or a master
equations with coefficients that are defined only while the simultaneous solution is being sought).
Third, it uncovers important channels for distribution transfer that are not possible
to place within the known frameworks of the Kolmogorov forward equation, or the McKean-Vlasov equation,
or the master equation in MFG.
The new approach can be implemented in practice by way of functional programming
and can produce equilibria that are consistent 
with widely cited case studies, but also equilibria in some critically important classical models
that have not been adequately resolved by other means.
It can also replicate the well documented and widely discussed results
of Krusell and Smith
without simulation and without the need to postulate infinite time horizon
(in fact, with a closed-form solution for some important parts of the equilibrium).
It provides a nearly complete answer to the question whether or not 
only-the-mean-matters point of view, put forward by Krusell and Smith \cite{KS98},
is reasonably accurate and under what restrictions. Moreover, the new approach
reveals structures and features of Krusell-Smith's model that does not appear
possible to uncover by other means.
\vfil\eject

%%% chend556677
%%% chend555666777

\section*{References}

{%
\input mac-5.tex
}%

\begin{appendices} 

\setcounter{section}{1}

\titlespacing*{\section}{0pt}{0.0ex plus 0.25ex minus .125ex}{0.55ex  plus 0.125ex minus 0.125ex}

\makeatletter%
\edef\@currentlabel{\Alph{section}}%
\makeatother%
\section*{Appendix \Alph{section}: Proof of Theorem~\ref{thm1}\label{sec:A2}}
\setcounter{paragraph}{0}%

\noindent
This result is a direct application of the implicit function theorem.
The left sides of all three equations%%%%%%%%%%%%%%%%%%%%
%%%%%%%%%%%%%%%%%%
\footnote{{}Recall that $V_{t+1,\tts y,\tts \T_{t,\tts x}^y(F),\tts v}\phd$
is assumed strictly concave and in $\C^2$ wherever it is finite.}
%%%%%%%%%%%%%%%%%%%%%%%%%%%%%%%%%%%%%%%%%%%%%%%%%%%%%%%%%%%%%
 in \eqref{z2-no-lm} can be treated as a $\R^3$-valued $\C^1$-function, which we write as 
$h(c,\q,\qq,w)$, with the understanding that $W_{y,v}$ substitutes for the right side of the first
equation in \eqref{ze2}. To simplify the notation, set 
{\abovedisplayskip=5pt plus 1.5pt minus 1.5pt\belowdisplayskip=5pt plus 1.5pt minus 1.5pt\belowdisplayshortskip=5pt plus 1.5pt minus 1.5pt
\begin{gather*}
 a_{y,v}\df (1+r)\sqrt{-\b \, \partial^2 V_{t+1,\tts y,\tts \T_{t,\tts x}^y(F),\tts v}\bigl(W_{y,\tts v} \bigr)}\\
\noalign{and}
b_{y,v}\df \bigl(\rho_ y(K)+ 1-\dd\bigr)\sqrt{-\b \, \partial^2 V_{t+1,\tts y,\tts \T_{t,\tts x}^y(F),\tts v}\bigl(W_{y,\tts v} \bigr)}\,,
\end{gather*}
}%
which leads to the following expression%
%%%%%%%%%%%%%%%%%%%%%%%
\footnote{{}By convention, if $a<0$ we write $\sqrt{-a}\sqrt{-a}=-a$.} 
%%%%%%%%%%%%%%%%%%%%%%
for the Jacobian matrix of the function $h$
%%
%%
{\abovedisplayskip=5pt plus 1.5pt minus 1.5pt\belowdisplayskip=5pt plus 1.5pt minus 1.5pt\belowdisplayshortskip=5pt plus 1.5pt minus 1.5pt
\begin{equation}\label{Jh}
h'(c,\q,\qq,w)=\begin{bmatrix}
1 & 1 & 1 & \,\,\llap{$-1$} \\
\partial^2U(c ) & \sum\nolimits_{y,v} a_{y,v}^2&  \sum\nolimits_{y,v} a_{y,v} b_{y,v} & 0\\
\partial^2U(c ) & \sum\nolimits_{y,v} a_{y,v} b_{y,v} & \sum\nolimits_{y,v} b_{y,v}^2 & 0 
\end{bmatrix}\,.
\end{equation}
}%
Let $h'(c,\q,\qq,w)_1$ denote the $3$-by-$3$ matrix composed of the first three columns in the Jacobian
and let $h'(c,\q,\qq,w)_2$ denote the $3$-by-$1$ matrix composed of the last column.
Hence
%%
%%
{\abovedisplayskip=5pt plus 1.5pt minus 1.5pt\belowdisplayskip=5pt plus 1.5pt minus 1.5pt\belowdisplayshortskip=5pt plus 1.5pt minus 1.5pt
$$
\begin{aligned}
&\bigl|h'(c,\q,\qq,w)_1\bigr| =
\Bigl(\sum\nolimits_{y,v} a_{y,v}^2\Bigr)\Bigl(\sum\nolimits_{y,v}
b_{y,v}^2\Bigr) -\Bigl(\sum\nolimits_{y,v} a_{y,v} b_{y,v}\Bigr)^2\\
&\hbox to1.8cm{\hfill}-\partial^2U(c )\Bigl(\sum\nolimits_{y,v} b_{y,v}^2-\sum\nolimits_{y,v} a_{y,v}b_{y,v}\Bigr)
+\partial^2U(c )\Bigl(\sum\nolimits_{y,v} a_{y,v}b_{y,v}-\sum\nolimits_{y,v} a_{y,v}^2\Bigr)\\
&\hbox to2.3cm{\hfill} =\Bigl(\sum\nolimits_{y,v} a_{y,v}^2\Bigr)\Bigl(\sum\nolimits_{y,v}
b_{y,v}^2\Bigr) -\Bigl(\sum\nolimits_{y,v} a_{y,v} b_{y,v}\Bigr)^2
-\partial^2U(c )\sum\nolimits_{y,v} (a_{y,v}-b_{y,v})^2\,.
\end{aligned}
$$
}%
Since we exclude from the model the degenerate case in which the payoffs from capital investment
are identical in all productivity states, the
determinant above is strictly positive. By the implicit function theorem the equation
$h(c,\q,\qq,w)=(0,0,0)^\trn$ defines 
$(c,\q,\qq)\in\R^3$ as a unique $\C^1$-function in some neighborhood of $w$ with derivative
%%
%%
{\abovedisplayskip=5pt plus 1.5pt minus 1.5pt\belowdisplayskip=5pt plus 1.5pt minus 1.5pt\belowdisplayshortskip=5pt plus 1.5pt minus 1.5pt
$$
\bigl(\partial c,\partial \q,\partial \qq\bigr)
=-h'\bigl(c,\q,\qq,w\bigr)_1^{-1} \, h'\bigl(c,\q,\qq,w\bigr)_2\,,
$$
}%
and since the first entry in the first row of the inverse $h'(c,\q,\qq,w)_1^{-1}$ can be identified
as the strictly positive scalar
%%
%%
{\abovedisplayskip=5pt plus 1.5pt minus 1.5pt\belowdisplayskip=5pt plus 1.5pt minus 1.5pt\belowdisplayshortskip=5pt plus 1.5pt minus 1.5pt
$$
\Bigl(\sum\nolimits_{y,v} a_{y,v}^2\Bigr)\Bigl(\sum\nolimits_{y,v} b_{y,v}^2\Bigr)
-\Bigl(\sum\nolimits_{y,v} a_{y,v} b_{y,v}\Bigr)^2\,,
$$
}%
we see that $\partial c>0$, i.e., the consumption level is a strictly increasing
$\C^1$-function of entering wealth.
Furthermore, the value function of the problem in
(\ref{ze1}-\ref{ze2}) can be cast as
{\abovedisplayskip=5pt plus 1pt minus 1pt\belowdisplayskip=5pt plus 1pt minus 1pt\belowdisplayshortskip=3pt plus 0.5pt minus 0.5pt
\begin{equation*}%\label{A-ze1}
\begin{aligned}
&V_{t,x,F,u}(w )\df U(c(w) )\\
&\hbox to0.5cm{\hfill}+\b\sum\nolimits_{ y\ts\in\ts\XXX,\,v\ts\in\ts\EEE}
V_{t+1,\tts y,\tts \T_{t,\tts x}^y(F),\tts v}\Bigl((1+r)\,\q(w)  + (\rho_ y(K)+ 1-\dd)\,\qq(w)  + 
\ee_ y(K)\, v \Bigr)\\
&\hbox to10cm{\hfill}\times Q(x, y)P_{x, y}(u, v)\,,
\end{aligned}
\end{equation*}}%
and this function is $\C^1$
with respect to the resource $w $ as well. By~the envelope theorem (see \eqref{ze4a1})
{\abovedisplayskip=5pt plus 1pt minus 1pt\belowdisplayskip=5pt plus 1pt minus 1pt\belowdisplayshortskip=3pt plus 0.5pt minus 0.5pt
$$
\partial V_{t,x,F,u}(w ) =\partial U(c(w) )\,,
$$
}%
with the implication that $\partial V_{t,x,F,u}\phd$
is $\C^1$ and strictly decreasing, since $\partial U\phd$ is strictly decreasing and $c\phd$ is
strictly increasing; in particular,
$V_{t,x,F,u}\phd\in\C^2(\R)$ and $\partial^2 V_{t,x,F,u}\phd<\nobreak 0$.

Removing the production technology from the model leads to the removal of the third row and the
third column in the Jacobian matrix in \eqref{Jh}. Similarly, removing the private lending
instrument from the model leads to the removal of the second row and the second column from the
Jacobian. In either case, the application of the implicit function theorem as above is straightforward. 

The second part of the theorem can be established in much the same way. Let $k(c,\q,\qq)\in\R^2$
be the vector composed of the left sides in the last two equations in \eqref{z2-no-lm}. Then
$k\in\C^2(\R^3;\R^2)$ with Jacobian matrix  
%%
%%
{\abovedisplayskip=5pt plus 1.5pt minus 1.5pt\belowdisplayskip=5pt plus 1.5pt minus 1.5pt\belowdisplayshortskip=5pt plus 1.5pt minus 1.5pt
$$
k'(c,\q,\qq)=\begin{bmatrix}
\partial^2U(c ) & \sum\nolimits_{y,v} a_{y,v}^2&  \sum\nolimits_{y,v} a_{y,v} b_{y,v} \\
\partial^2U(c ) & \sum\nolimits_{y,v} a_{y,v} b_{y,v} & \sum\nolimits_{y,v} b_{y,v}^2 
\end{bmatrix}\,.
$$
}%
As the matrix composed of the last two columns in this Jacobian was already shown to have a strictly
positive determinant, the implicit function theorem completes the proof.






\stepcounter{section}
\section*{Appendix \Alph{section}: Lower Bounds on Consumption}
\makeatletter%
\edef\@currentlabel{\Alph{section}}%
\makeatother%
\label{sec:App-B}\setcounter{paragraph}{0}


While this result is not used in the \chptr, it is important to note that,
in general, there is no reason why in equilibrium the range of
consumption must expand arbitrarily close to $0$,
i.e., in equilibrium, 
the support of the cross-sectional distribution of agents may exclude a neighborhood of~$0$. %
To~see this, consider the special case where productive capital is the only asset
(i.e., $\q_{t,x,\bar\csd,u}=0$) and $\partial U(c)=1/c$~-- see Sec.~\ref{sec:KS}.
The second kernel condition in \ref{cross-sys}-(\ref{zze5a}) can now be cast as
{\abovedisplayskip=5pt plus 1.5pt minus 1.5pt\belowdisplayskip=5pt plus 1.5pt minus 1.5pt\belowdisplayshortskip=5pt plus 1.5pt minus 1.5pt
\begin{equation}\label{new-kern}   
c=\Bigl(\b\sum\nolimits_{ y\ts\in\ts\XXX,\,v\ts\in\ts\EEE}{1\over\pfc_{t,x,\csd}^{ y,v}(u,c)}
\bigl(\rho_{ y}(K_{t}(x,\csd))+1-\dd\bigr) Q(x, y)P_{x, y}(u,v)\Bigr)^{-1}\,.
\end{equation}
}%
Introducing the strictly increasing functions
{\abovedisplayskip=5pt plus 1.5pt minus 1.5pt\belowdisplayskip=5pt plus 1.5pt minus 1.5pt\belowdisplayshortskip=5pt plus 1.5pt minus 1.5pt
\[
\Rpp\ni \a \leadsto H_{t+1, y,\T_{t,x}^y(\csd),v}(\a)\df \a + \qq_{t+1, y,\T_{t,x}^y(\csd),v}(\a) \,,
\]}%
with inverses $\hat H_{t+1, y,v}\phd$ as in \eqref{inversion}, the balance equation \ref{cross-sys}-(\ref{zze5xb})
can be stated as
{\abovedisplayskip=7pt plus 1.5pt minus 1.5pt\belowdisplayskip=7pt plus 1.5pt minus 1.5pt\belowdisplayshortskip=5pt plus 1.5pt minus 1.5pt 
\[
\pfc_{t,x,\csd}^{ y,v}(u,c) =
{{\hat H}_{t+1, y,\T_{t,x}^y(\csd),v}\Bigl(\bigl(\rho_ y(K_{t}(x,\csd))+1-\dd\bigr){\qq_{t,x,\csd,u}(c)}
+v \,\ee_ y(K_{t}(x,\csd))\Bigr)}\,.
\]}%
Suppose next that the domains of all functions $\qq_{t+1, y,\T_{t,x}^y(\csd),v}\phd$, $ y\in\XXX$, $v \in\EEE$,
include $\Rpp$ and let $\qq_{t+1, y,\T_{t,x}^y(\csd),v}(0)\df\lim_{c\searrow 0}\qq_{t+1, y,\T_{t,x}^y(\csd),v}(c)\,$.
The infimum over all admissible entering wealths in period $t+1$ is
$\qq_{t+1,y,\T_{t,x}^y(\csd),v}(0)$. Suppose that this infimum can be reached, in the sense
that there is a $c^*\ge 0$ (depending on $y$ and $v$) such that 
%%
{\abovedisplayskip=5pt plus 1.5pt minus 1.5pt\belowdisplayskip=5pt plus 1.5pt minus 1.5pt\belowdisplayshortskip=5pt plus 1.5pt minus 1.5pt
\[
\begin{split}
\lim\nolimits_{c\searrow c^*}
\Bigl(\bigl(\rho_ y(K_{t}(x,\csd))+1-\dd\bigr){\qq_{t,x,\csd,u}(c)}
&+v \,\ee_ y(K_{t}(x,\csd))\Bigr)\\
&= \qq_{t+1,\tts y,\tts \T_{t,x}^y(\csd),\tts v}(0) = {H}_{t+1,\tts  y,\tts \T_{t,x}^y(\csd),\tts v}(0)\,,
\end{split}
\]
}%
i.e., assuming that ${\qq_{t,x,\csd,u}\phd}$ is continuous, chosen so that
%%
{\abovedisplayskip=5pt plus 1.5pt minus 1.5pt\belowdisplayskip=5pt plus 1.5pt minus 1.5pt\belowdisplayshortskip=5pt plus 1.5pt minus 1.5pt
\[
\qq_{t,x,\csd,u}(c^*)\df\lim\nolimits_{c\searrow c^*}\qq_{t,x,\csd,u}(c)
= {\qq_{t+1,\tts y,\tts \T_{t,x}^y(\csd),\tts v}(0) - v \,\ee_ y(K_{t}(x,\csd))\over \rho_ y(K_{t}(x,\csd))+1-\dd}\,.
\]
}%
Then $\lim\nolimits_{c\searrow c^*} \pfc_{t,x,\csd}^{ y,v}(u,c)
= {\hat H}_{t+1,\tts y,\tts \T_{t,x}^y(\csd),\tts v}
\bigl({H}_{t+1,\tts y,\tts \T_{t,x}^y(\csd),\tts v}(0)\bigr)=0$, which is possible only if $c^*=\nobreak 0$,
since otherwise the right side of \eqref{new-kern} converges to $0$ as $c\searrow c^*$,
while the left side  converges to $c^*>0$.
Consequently, if one is to insist
that $\qq_{t,x,\csd,u}\phd$ is continuous and non-decreasing and that its domain is contiguous and
extends to $+\infty$ with $\lim_{c\nearrow\infty}\qq_{t,x,\csd,u}(c)=\infty$,
then everywhere in that domain $\qq_{t,x,\csd,u}\phd$ must have a lower bound
given by
{\abovedisplayskip=5pt plus 1.5pt minus 1.5pt\belowdisplayskip=5pt plus 1.5pt minus 1.5pt\belowdisplayshortskip=5pt plus 1.5pt minus 1.5pt
\begin{equation}\label{theta-min}
M_{t,x,\csd}\df \max_{ y\ts\in\ts\XXX,\,v\ts\in\ts\EEE}\
{\qq_{t+1,\tts y,\tts \T_{t,x}^y(\csd),\tts v}(0) - v \,\ee_
y(K_{t}(x,\csd))\over \rho_ y(K_{t}(x,\csd))+1-\dd}\,.
\end{equation}}%
In particular,
{\abovedisplayskip=5pt plus 1.5pt minus 1.5pt\belowdisplayskip=5pt plus 1.5pt minus 1.5pt\belowdisplayshortskip=5pt plus 1.5pt minus 1.5pt
\[
\pfc_{t,x,\csd}^{ y,v}(u,\cdot) \ge {{\hat H}_{t+1, y,v}\Bigl(\bigl(\rho_ y(K_{t,x})+1-\dd\bigr) M_{t,x}
+v \,\ee_ y(K_{t,x})\Bigr)}
\]
}%
everywhere in the domain of $\qq_{t,x,u}\phd$. Of course, this relation is interesting only if 
the right side is strictly positive~-- a situation that is illustrated next. 
%%%%%%%%%%%%%%%%

Now suppose that, in addition to $\Up(c)=1/c$ and $\q_{t,x,\csd,u}\phd\equiv0$,
the model is also such that $\EEE=\{\eta,0\}$ for some fixed $\eta>0$ (there are only two
employment states, employed and unemployed~-- see Sec.~\ref{sec:KS}).
In~what follows the values of all functions at $0$
are to be understood as the right limits at $0$.
Since no investment takes place in the final period $t=T$, with
$t=T-1$ the lower bound in \eqref{theta-min} becomes $0$, i.e, $\qq_{T-1,x,\csd,u}(0)\ge 0$.
If $\qq_{t,x,\csd,u}(0)\ge 0$ for some $t<T$, then with $v =\eta$ and
$c=0$ the balance equation in \ref{cross-sys}-(\ref{zze5xb}) would give
{\abovedisplayskip=5pt plus 1.5pt minus 1.5pt\belowdisplayskip=5pt plus 1.5pt minus 1.5pt\belowdisplayshortskip=5pt plus 1.5pt minus 1.5pt
\begin{equation}\label{boc-c}
{\pfc_{t,x,\csd}^{ y,\eta}(u,0)}
+ {\qq_{t+1,\tts y,\tts \T_{t,x}^y(\csd),\tts \eta}\bigl(\pfc_{t,x,u}^{ y,\eta}(0)\bigr)}\ge \ee_ y(K_{t,x})\,\eta\,,
\end{equation}}
and with $v =0$ and $c=0$ the same balance equation would give
{\abovedisplayskip=5pt plus 1.5pt minus 1.5pt\belowdisplayskip=5pt plus 1.5pt minus 1.5pt\belowdisplayshortskip=5pt plus 1.5pt minus 1.5pt
\begin{equation}\label{boc-d} 
{\pfc_{t,x,\csd}^{ y,0}(u,0)}
+ {\qq_{t+1,\tts y,\tts\T_{t,x}^y(\csd),\tts 0}\bigl(\pfc_{t,x,\csd}^{ y,0}(u,0)\bigr)}\ge 0\,.
\end{equation}}
With $t=T-1$ the last two relations
become
{\abovedisplayskip=5pt plus 1.5pt minus 1.5pt\belowdisplayskip=5pt plus 1.5pt minus 1.5pt\belowdisplayshortskip=5pt plus 1.5pt minus 1.5pt
\begin{equation}\label{boc-e}
{\pfc_{T-1,x,\csd}^{ y,\eta}(u,0)}\ge \ee_ y(K_{T-1,x})\,\eta\qquad\text{and}\qquad 
{\pfc_{T-1,x,\csd}^{ y,0}(u,0)}\ge 0\,.
\end{equation}
}%
Since ${\pfc_{T-1,x,\csd}^{ y,\eta}(u,0)}$ is strictly positive, due to \eqref{new-kern}
${\pfc_{T-1,x,\csd}^{ y,0}(u,0)}> 0$ is not possible, for otherwise the right side would have a strictly
positive limit as $c\searrow 0$, while the limit of the left side would be $0$.
Since ${\pfc_{T-1,x,\csd}^{ y,0}(u,0)}=0$,
the balance equation \ref{cross-sys}-(\ref{zze5xb}) gives $\qq_{T-1,x,\csd,u}(0)= 0$ for all
$u\in\EEE$. In~particular,
both relations in \eqref{boc-e} are strict identities. Next, setting $t=T-2$ in \eqref{theta-min}
again yields $M_{T-2,x,\csd}=0$, so that $\qq_{T-2,x,\csd,u}(0)\ge 0$ for every $u\in\EEE$.
The balance equation now gives
{\abovedisplayskip=5pt plus 1.5pt minus 1.5pt\belowdisplayskip=5pt plus 1.5pt minus 1.5pt\belowdisplayshortskip=5pt plus 1.5pt minus 1.5pt
\begin{equation*}
\pfc_{T-2,x,\csd}^{ y,\h}(u,0)+\qq_{T-1,\tts y,\tts \T_{T-2,x}^y(\csd),\tts \h}
\bigl(\pfc_{T-2,x,\csd}^{ y,\h}(u,0)\bigr) \ge \ee_y(K_{t}(x,\csd))\h > 0\,,
\end{equation*}}
which is possible only if $\pfc_{T-2,x,\csd}^{ y,\h}(u,0)>0$.
Just as above, due to \eqref{new-kern} the last relation implies $\pfc_{T-2,x,\csd}^{y,0}(u,0)=0$,
so that with $v=0$ the balance equation
\ref{cross-sys}-(\ref{zze5xb}) gives
{\abovedisplayskip=5pt plus 1.5pt minus 1.5pt\belowdisplayskip=5pt plus 1.5pt minus 1.5pt\belowdisplayshortskip=5pt plus 1.5pt minus 1.5pt
\begin{equation*}
\bigl(\rho_y(K_{t}(x,\csd))+1-\dd\bigr)\qq_{T-2,x,\csd,u}(0)=0+\qq_{T-1,\tts
y,\tts \T_{T-2,x}^y(\csd),\tts v}(0)=0\,,
\end{equation*}
}%
with the implication that $\qq_{T-2,x,\csd,u}(0)=0$. By way of induction one can show that
$\qq_{t,x,\csd,u}(0)=0$, $\pfc_{t,x,\csd}^{y,\h}(u,0)>0$, and $\pfc_{t,x,\csd}^{ y,0}(u,0)=0$ for all
$t<T$, $x,y\in\XXX$ and $u\in\EEE=\{\h,0\}$. In particular, capital will never be borrowed and
the consumption level of all employed households will be bounded away from $0$ for all $t<T$. We
stress that these features hold if capital is the only asset and all households in one of
the employment states have no income.  

\end{appendices}

\end{document} 
