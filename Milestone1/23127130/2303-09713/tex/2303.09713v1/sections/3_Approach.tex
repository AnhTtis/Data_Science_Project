\section{Approach}\label{sec:approach}
\subsection{Dataset Collection: \datasetName}\label{subsec:dataset_collection}
In this section, we describe our pipeline to collect \datasetName, which is outlined in Figure~\ref{fig:dataset_collection}.

\vspace{-13pt}

\paragraph{Extracting and Filtering Videos.}
The data collection process starts with downloading public YouTube videos. Each video is associated with a user-provided title, metadata, and a transcript generated by the YouTube's internal Automatic Speech Recognition system. 
We then filter videos by applying several steps using the associated metadata, following the strategy from~\citep{zellers2021merlot, zellers2022merlot}.
In particular, we use Python \textit{cld3} library, which uses a neural network for language identification, to filter out videos whose transcripts have a probability of being English less than 80\%.
We also discard videos that do not contain visual variation (\textit{e.g.}, a video of podcast with a static thumbnail) or those lacking objects in the thumbnails according to an image classification model~\cite{sandler2018mobilenetv2}. These steps resulted in the extraction of 20M videos.

We further process these 20M videos to build video segments. First, we build a list of segments by iterating through each video with a 60-second sliding window. Second, we filter out segments which have transcripts containing less than 30 words: short transcripts are unlikely to contain multiple conversation turns. 
Finally, we filter out transcripts that contain unsafe topics to ensure that the model does not learn harmful language. We utilize Rewire API~\cite{rewire} to detect toxic content in the transcripts.
At the end of the filtering stage, we end up with 18M video segments. 

\vspace{-13pt}

\paragraph{Converting Noisy Transcripts into Dialogues.}
Although each video is equipped with its own transcript,
they are usually not suitable to train a video-based dialogue model directly.
For example, transcripts do not provide any delimitation between the interlocutors.  A naive approach to address this problem is to use speaker diarization systems to determine ``who spoke when"~\citep{tranter2003investigation}. However, these systems suffer from low accuracy~\citep{park2022review}, thus perform poorly when turning a sequence of words to a well-structured dialogue. 

Instead, %
we train a \textit{converter} model to transform noisy transcripts into structured dialogues. Motivated by the in-context learning capabilities of GPT-3 models~\citep{brown2020language}, we prompt GPT-3 with few-shot examples and ask the model to generate well-formatted dialogues given noisy transcripts. 
However, using GPT-3 to process millions of samples is computationally expensive. Therefore, we train a smaller model using denoised transcripts sampled from GPT-3.
That is, we collect 40K input-output pairs from GPT-3 where the input is a noisy transcript and the output is the converted dialogue. Our \textit{converter} is a Unified-IO~\citep{lu2022unified} Base (241M params) fine-tuned on the generated pairs. We exclude video segments that have more than 150 words as the model cannot handle excessively long inputs. The pairs are divided into train and validation sets at a ratio of 99:1. The \textit{converter} achieves a high accuracy of 90.1\% on the validation set when using teacher-forcing to predict tokens,
suggesting high-quality dialogue generation given noisy transcripts (\textit{e.g.,} in Figure~\ref{fig:dataset_collection}, Step 2). 

\vspace{-13pt}

\paragraph{Converting Videos into Video-based Dialogues.}
To ensure that each utterance in the obtained dialogue matches correctly with the corresponding video frames, we employ Dynamic Time Warping~\cite{muller2007dynamic} to align the dialogues with the original noisy transcripts.
After the alignment, we use the timing information in the original noisy transcripts to estimate the start time of each dialogue turn, which in turn helps extract the corresponding video frame and minimize alignment errors caused by the conversion process.
Figure~\ref{fig:ytdialog_examples} shows some examples from \datasetNameNoEmoji. More details about collecting dataset can be found in Appendix~\ref{sec:appendix_dataset_collection}.

\subsection{Dataset Analysis}\label{subsec:analysing_dataset}



\section{Label Noise Analysis and Dataset}%Caption Label Noise (CLaN) Dataset}
%\section{Analysis: Label and Visual Noise Dataset}
\label{sec:analysis}

We analyze what makes large-scale in-the-wild datasets a challenging source of labels for object detection methods. 

% ANALYSIS TABLE MOVED UP TO RELATED WORK SO IT CAN BE AT THE TOP OF THE PAGE

\textbf{Datasets analysed.}
%\label{sec:datasets}
Conceptual Captions (CC) \cite{Sharma2018ConceptualCA}, RedCaps  \cite{Desai2021RedCapsWI}, and SBUCaps \cite{Ordonez2011Im2TextDI}, are collected from in-the-wild data sources. \textbf{CC} contains 3 million image-alt-text pairs after heavy post-processing; named entities in captions were hypernymized and image-text pairs were accepted if there was an overlap between Google Cloud Vision API class predictions and the caption.
\textbf{RedCaps} %is the largest dataset used in our paper, 
consists of 12M image-text pairs collected from Reddit by crawling a manually curated list of subreddits with heavy visual content.
\textbf{SBUCaps} consists of 1 million Flickr photos with text descriptions written by their owners.
Only captions with at least one prepositional phrase and at least 2 matches with a predefined vocabulary were accepted.
These in-the-wild datasets exhibit very low precision of the extracted labels, ranging from 0.463 for SBUCaps, 0.596 for RedCaps, to 0.737 for CC, all much lower than the 0.948 for COCO (shown in supp). 
%This motivates our exploration of VAEL noise.

\textbf{Extracted object labels.} Given a vocabulary of object classes, we extract a label for an image if there is exact match between the object name and the corresponding caption ignoring punctuation, as in \cite{Ye_Zhang_Kovashka_Li_Qin_Berent_2019,Fang2022DataDD}.

\textbf{Gold standard object labels.} We use pseudo-ground-truth predictions from a pretrained image recognition model to estimate visual presence \textit{gold standard} labels because these in-the-wild datasets do not have image-level object annotations. 
We use an object recognition ensemble with the X152-C4 object-attribute model \cite{zhang2021vinvl} 
% (trained on four public object detection datasets) 
and the Ultralytic  YOLOv5-XL \cite{yolov5}.
This ensemble achieves strong accuracy, %on visual presence detection, e.g.
82.2\% on SBUCaps, 85.6\% on RedCaps, and 86.8\% on CC.
%We observed higher visual presence accuracy on a small annotated set compared to each model alone (see supp). 
We extract VAELs by selecting images where extracted and gold-standard labels disagree. 
%Note in some cases a VAEL will correspond to a present object which is however missed by the recognition ensemble.

\textbf{Noise annotations collected.}
We select 100 VAEL examples per dataset (RedCaps, SBUCaps, CC).
We annotate four types of information for these examples: 
\begin{itemize}[nolistsep,noitemsep]
    \item (Q1: Label Noise) How much of the VAEL object is present (\underline{vis}ible, \underline{part}ially visible, completely \underline{abs}ent); 
    \item (Q2: Similar Context) If the VAEL object is completely missing, whether a traditionally co-occurring context (``boat" and ``water"), or semantically similar object (e.g. ``cake" and ``bread", ``car" and ``truck") is present; 
    \item (Q3: Visual Defects) If visible/partially visible, whether the VAEL object is occluded, has key parts missing, or atypical appearance (e.g. knitted animal); and
    \item (Q4: Linguistic Indicators) What linguistic cues, if any, explain why the VAEL object is mentioned but absent, e.g. the caption discusses events or information beyond what the image shows (``beyond'' in Table \ref{tab:stats}), describes the past (``past''), the extracted label is part of a prepositional phrase and likely to describe the setting and not objects (``on a train''), is a noun modifying another noun, is used in a non-literal way, has a different word sense (e.g. ``bed'' vs ``river bed''), or is part of a named entity.
\end{itemize}

Two annotators (authors) provide the annotations, with high agreement: 0.76 for Q1, 0.33 for Q2, 0.45 for Q3, and 0.58 for Q4. We calculate Cohen's Kappa for each option and aggregate agreement through a weighted average for each question, with weights derived from average option counts between the two annotators across the three datasets. We label the dataset Caption Label Noise, or CLaN.

In Table \ref{tab:stats}, we show what fraction of samples fall into each annotated category, excluding ``Other'', ``Unclear'' and uncommon categories. We average the distribution between the two annotators.

\textbf{Statistics: Label noise.}
We first characterize the visibility of objects flagged as VAELs by the recognition ensemble. We find that SBUCaps has the highest rate of completely absent images (58.5\%), followed closely by RedCaps. 
CC has the highest full visibility (32.8\%),
%followed by RedCaps (29.2\%) and then SBUCaps (21.5\%) where full visibility is 
defined as the object from a given viewpoint having 75\% or more visibility. 
SBUCaps also has the highest rate of partially visible objects (20\%). 
The high rate of absent and partially-visible objects justifies the use of pseudo-ground-truth labels from the recognition ensemble; these both constitute poor training data for WSOD. 
We investigate the fully-visible objects flagged as VAELs shortly, through our visual defect annotations. 
%These significant rates of visibility motivates investigating any visual defects in partially or fully visible objects in Q3 which may explain why the recognition ensemble flagged these as absent and could be considered a difficult example \cite{Bengio2009CurriculumL}. Secondly, it also motivates marking linguistic indicators in Q4 which can be used to predict both visual defects and completely absent objects in Q4.

\textbf{Statistics: Similar context.}
%One issue with solely focusing on object visibility is that 
Certain images with absent objects may be more harmful than others. Prior work has shown that models exploit co-occurrences between an object and its context which helps overall recognition accuracy, but can hurt performance when that context is absent \cite{Singh_Mahajan_Grauman_Lee_Feiszli_Ghadiyaram_2020}. We hypothesize the inclusion of images with this context bias without the actual object present could affect localization especially when supervising detection \textit{implicitly}, and semantically similar context may blur decision boundaries.
%hurting classification. 
%This motivates annotating if the image contains co-occurring context or if semantically similar objects are present instead of the VAEL. The question is subjective as 
Different annotators may have different references for similarity or co-occurrence frequency, but our annotators achieve fair agreement ($\kappa=0.33$). In Table \ref{tab:stats}, we find high rates of co-occurring contexts in samples with completely absent VAELs for SBUCaps (42.5\%) and CC (30.9\%).
%, while this is rarer in RedCaps (4\%). 
Across all datasets, we see a similar rate, 12\%-15\%, of similar context being present instead of the VAEL. 
% For example, if the image contains co-occurring context (e.g. "bus stop with buildings") while the object (e.g. "bus") is absent, then the inclusion of this example could harm localization.

\textbf{Statistics: Visual defects.}
%VAELs include a number of fully visible samples.
%or partially  
We hypothesize there may be visual defects which caused the recognition ensemble to miss fully-visible objects. 
%Summing over the columns in Visual defects in Table \ref{tab:stats}, we observe that indeed most visible samples have defects.
Over the fully or partially visible subset, in CC 79\% of fully or partially visible objects have a visual defect, 87\% for SBUCaps, and 69\% for RedCaps. 
The most common defect for RedCaps and CC is atypical (49\% and 57.3\%); we argue these atypical examples constitute poor training data for WSOD.
%, resp.) and occlusion for SBUCaps (26.5\%) in Table \ref{tab:stats}. This also shows that using the recognition ensemble to estimate visual presence is also a decent proxy for visual defects, so the recognition ensemble's missed predictions are useful. Upon further examination of captions, 
We find that the caption context (e.g. ``acrylic illustration of the funny mouse") may indicate the possibility of a visual defect, which further motivates the VEIL design. 

\textbf{Statistics: Linguistic indicators.} Noun modifier is one of the most frequently occurring linguistic indicator. Prepositional phrase is also significant in SBUCaps (40.5\%) and CC (31.3\%).
%, compared to RedCaps (5.7\%)
%and non-literal use is common in SBUCaps and RedCaps.
%have almost double the amount of non-literal linguistic indicator compared to CC. 
All datasets contain many VAELs which are mentioned in contexts going beyond the image; for example:
%the VAEL, ``boat” is mentioned in text that goes beyond the image: 
``just got back from the river. friend \textbf{sank his truck pulling his \underline{boat} out}. long story short, rip this beast” (RedCaps). %These linguistic indicators motivate rule-based methods as our baselines, but 
%Given that VAELs can be explained by a number of these indicators, rule-based method will miss out on VAELs. 
%We believe this relationship between linguistic indicators and VAELs can be learned implicitly if a transformer-based model is provided with a loose proxy for visual presence/absence labels, motivating VEIL.
To better understand our dataset, we conduct a human evaluation study comparing \datasetNameNoEmoji with MMDialog~\cite{feng2022mmdialog} --- the largest dataset for visually-grounded dialogue.
MMDialog is a dataset sourced from people's interaction in social media.
We randomly sample 500 examples from each dataset and ask three workers to assess each example for several factors.
For more information regarding the human evaluation, please refer to the Appendix~\ref{subsec:appendix_human_evaluation}.

\vspace{-13pt}
\paragraph{Dataset quality.}
To compare the dialogue quality, we ask workers to assess examples using two criteria: sensibleness and specificity~\cite{adiwardana2020towards}.
To be sensible, a dialogue should be reasonable, logical, and not confusing.
Specific dialogue is one that is not dull or generic.
Each human annotators rate the dialogue in specific aspect on a 3-point Likert scale, \textit{e.g.} for sensibleness "1" being "Not Sensible" and "3" being "Sensible".
Evaluation results are shown in Table~\ref{tab:dataset_analysis}. On average, \datasetNameNoEmoji received higher scores than MMDialog across the axes of sensibleness and specificity.
We suspect that MMDialog, which is derived from social media, may lack a natural conversation flow due to the non-consecutive nature of social media interactions.

\vspace{-13pt}
\paragraph{Social Interaction.}
Given the prevalence of third-person point of view in web videos,
we postulate that a substantial proportion of such videos feature social interactions. To investigate this claim, we enlisted the aid of workers to determine whether the conversation's interlocutors were visible in the image frames and, if so, whether their body language was present. Our findings, presented in Table~\ref{tab:dataset_analysis}, indicate that our video-based dialogue dataset has a considerably higher proportion of visible interlocutors (61.6\%) than MMDialog (11.5\%). This discrepancy can be explained by the fact that it is uncommon for interlocutors in social media interactions to reveal their identities through images. Moreover, when interlocutors are visible, workers accurately identified facial expressions in 83.6\% of cases, and tagged body posture in 64.7\% of them. These results suggest that our dataset presents a valuable resource for exploring body language in communication.






\vspace{-13pt}

\paragraph{Visual Grounding.}
Real-life conversations do not always have a direct relationship or grounding to images. Therefore, a higher degree of relevance between dialogues and images does not necessarily imply a higher quality dataset, although it offers more visual grounding opportunities for models to learn from. We ask workers to assess the degree of grounding between conversations and images on a 3-point Likert scale when interlocutors were not visible in the images. According to Table~\ref{tab:dataset_analysis}, \datasetNameNoEmoji exhibits grounding scores that are comparable to those of MMDialog, implying that models can acquire visual grounding knowledge from \datasetNameNoEmoji.


\vspace{-13pt}

\paragraph{Distribution of Visual Contexts.}
\begin{figure}[t]
\centering
\includegraphics[width=0.4\textwidth]{figures/YTD_Visualize_CLIP.pdf}
\vspace{-2mm}
\caption{\small
Visual feature distributions of \datasetNameNoEmoji and other visually-grounded dialogue datasets. 
Our \datasetNameNoEmoji includes a wide range of visual contexts, with a particular emphasis on frames in which a person is speaking, in contrast to the other datasets (shown in the upper left cluster).}
\vspace{-5mm}
\label{fig:ytdialogue_visualize}
\end{figure}
In order to gain a better understanding of the visual representations encoded in \datasetNameNoEmoji from videos, we examine the distributions of visual features across three distinct datasets: Image Chat, MMDialog, and \datasetNameNoEmoji, that are grounded in dialogue. 
Similar to~\cite{liu2021visually}, we use CLIP ViT-L14~\cite{radford2021learning} trained on LAION-2B~\cite{schuhmannlaion}.\footnote{\cite{liu2021visually} conducted a similar study, but using ResNet50 ImageNet features.}%
We utilize this model to extract embeddings, and the embeddings are then projected into a 2D feature space by way of UMAP~\cite{mcinnes2018umap}. In order to conduct our analysis, we randomly sample 1K images from each of the aforementioned datasets.



Figure~\ref{fig:ytdialogue_visualize} illustrates the results, indicating that the datasets share some similarities but differ in certain aspects.  Notably, \datasetNameNoEmoji has a distinct distribution pattern from Image Chat and MMDialog, with a higher proportion of images featuring a person speaking, which is consistent with the previous discovery that \datasetNameNoEmoji  emphasizes social interactions. Additionally, \datasetNameNoEmoji  encompasses a broad range of diverse and specific topics, such as cooking or mechanics. Further information regarding the analysis of visual contexts can be found in the Appendix~\ref{sec:appendix_data_analysis}.




\vspace{-13pt}

\paragraph{Content Safety.}
We ask workers to identify whether the dialogues or images contain any potentially unsafe content, like sexually explicit material or hatespeech. As indicated in Table~\ref{tab:dataset_analysis}, our \datasetNameNoEmoji has fewer occurrences of sexually explicit content and hate speech when compared to MMDialog. We suspect that this discrepancy is due to the inclusion of safety filtering step in the data collection process for \datasetNameNoEmoji. 


\vspace{-13pt}

\paragraph{Video Title as an Additional Feature.}
We further ask the workers to rate the relevance of the video title to the dialogue on a 3-point Likert scale, ranging from 1 (Not Related) to 3 (Related). The results
 show that 65\% of the videos have a title that is relevant to the dialogues, and 21\% have titles that were somewhat related. This led us to utilize the video titles as prompts when training our model. Our qualitative findings (\S\ref{subsec:open_domain_conversation}) suggest that \modelNameNoEmoji can be conditioned effectively by such prompts. 


\subsection{Model: \modelName}\label{subsec:model}
\section{PAC-Net}

\subsection{Model Details}

We use PyTorch~\cite{Paszke_PyTorch_An_Imperative_2019} to build and train our neural networks. Besides ResNets~\cite{he2016deep}, there are only two other basic components
in our model, which are GRU cells \cite{KyunghyunCho2014GRU} and a self-designed MLP decoder. 
The vector dimension of extracted features and hidden state $\mathbf{h}$ are both 128.
Before loading the pre-trained weights of ResNets, we substitute the last linear layer in ResNets with another one with an output dimension of 128. 
The two-layer MLP decoder takes in the hidden state $\mathbf{h}$ as input. The 64-dimensional intermediate vector is activated by a ReLU, followed by a final linear layer that outputs the 2-dimensional coordinate.

\subsection{Model Training}

During offline training, we use an equivalent implementation of networks to improve the training efficiency. Instead of extracting feature frame-by-frame, we use two ResNets in PAC-Cell to extract features from raw frames and difference frames all at once after loading a video clip. Then static features (from raw frames) and dynamic features (from difference frames) are fed to P-GRU and C-GRU alternatively.
\looseness=-1

For all networks, we train models for 70 epochs using AdamW~\cite{loshchilov2017decoupled} optimizer with a weight decay of 2e-3. The learning rate is set to 3e-4 and follows the cosine annealing schedule~\cite{loshchilov2016sgdr}. The batch size is set to 32.

During training, we load a random $T'$-frame clip from the whole video clip of $T$ frames, where $T'\le T$. This practice further enhances the diversity of datasets.
\modelNameNoEmoji is a multimodal conversational agent that takes image frames, a prompt, and a dialogue context as an input and generates a response. 
The overview of training process is described in Figure~\ref{fig:model_training}.
The collected YouTube titles for each video serve as prompts and are denoted as $P$. 
Finally, the vision-based dialogue is denoted as $D = (P, I_1, T_1, ..., I_n, T_n)$, where $I_i$ and $T_i$ denote an image frame and the dialogue turn, respectively. 

\vspace{-13pt}

\paragraph{Architecture.}
The architecture used in \modelNameNoEmoji is based on the Unified-IO model proposed by \cite{lu2022unified}. This model is designed for vision-and-language tasks and operates as a sequence-to-sequence model. Although it can handle image and text inputs together, it is unable to handle multiple images. To address this limitation, we introduce \textit{video position embeddings} that can be learned and incorporated into \modelNameNoEmoji. Specifically, \modelNameNoEmoji  converts each frame of the video into a sequence of patch encodings using a visual encoder. It then adds video position embeddings to the patch encodings to capture temporal information of the video frames. The patch encodings from multiple image frames are averaged through mean pooling and fed into a Transformer encoder. Our experiments use three image frames per dialogue to train \modelNameNoEmoji.


\vspace{-13pt}

\paragraph{Training.}
We initialize \modelNameNoEmoji training with the pretrained weight from Unified-IO model that is pretrained on collection of C4~\cite{raffel2020exploring}, Wikipedia, ImageNet21K~\cite{ridnik2021imagenet}, and YFCC15M~\cite{radford2021learning} with a denoising objective.
Unified-IO model is trained in two stages, pre-training and the multi-task stage, and the weights from the pre-training stage, referred to as Unified-IO$_{PT}$\footnote{We run a pilot study and find that the model initialized from Unified-IO$_{PT}$ weights perform better on downstream tasks compared to the model initialized from Unified-IO.}, are used to initialize the \modelNameNoEmoji model.
We train the model using a next token prediction objective, which aims to maximize the likelihood of the target response $T_k$ when taking multiple images $I_{i\leq k}$, dialogue context $T_{i < k}$, and the video title $P$ as an input, \textit{i.e.}, $p_\theta(T_k|I_1, T_1, ... , I_{k-1}, T_{k-1}, T_k, P)$.

We present three versions of the model, \textsc{Base} (241M), \textsc{Large} (776M), and \textsc{XL} (2.9B).
We train models for 3 epochs on \datasetNameNoEmoji, and training \modelNameNoEmojiXL takes approximately 3 days on TPU v3-256 on Google Cloud Virtual Machines with T5X framework~\cite{roberts2022scaling}.
More details about hyperparameters are in Appendix~\ref{subsec:hparams}.
