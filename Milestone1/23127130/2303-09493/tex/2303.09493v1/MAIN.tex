% \documentclass[9pt,twocolumn,twoside,lineno]{pnas-new}
\documentclass[16pt, twoside]{pnas-new}
%\documentclass[16pt, twoside,lineno]{pnas-new}
%\documentclass[9pt,twocolumn, twoside,lineno]{pnas-new}
\setboolean{displaywatermark}{false}
\DeclareUnicodeCharacter{202A}{\,}

% Use the lineno option to display guide line numbers if required.
\templatetype{pnasresearcharticle} % Choose template 
%%%comment out to be single spacing%%%
\usepackage{setspace} \doublespacing
\usepackage{float}

\usepackage{graphicx}% Include figure files
\usepackage{dcolumn}% Align table columns on decimal point
\usepackage{bm}% bold math
\usepackage{siunitx}% allow using units

%\usepackage[mathlines]{lineno}% Enable numbering of text and display math
%\linenumbers\relax % Commence numbering lines


% Useful packages
\usepackage{amsmath}
\usepackage[colorlinks=true, allcolors=blue]{hyperref}
\usepackage[T1]{fontenc}
\usepackage{subfiles} % Best loaded last in the preamble
\usepackage{xcolor}
\usepackage{easyReview}

\begin{abstract}
Addition of particles to a viscoelastic suspension dramatically alters the properties of the mixture, particularly when it is sheared or otherwise processed.  Shear-induced stretching of the polymers results in elastic stress that causes a substantial increase in measured viscosity with increasing shear, and an attractive interaction between particles, leading to their chaining.  At even higher shear rates, the flow becomes unstable, even in the absence of particles. This instability makes it very difficult to determine the properties of a particle suspension.  Here we use a fully immersed parallel plate geometry to measure the high-shear-rate behavior of a suspension of particles in a viscoelastic fluid.  We find an unexpected separation of the particles within the suspension resulting in the formation of a layer of particles in the center of the cell.  Remarkably, monodisperse particles form a crystalline layer which dramatically alters the shear instability.  By combining measurements of the velocity field and torque fluctuations, we show that this solid layer disrupts the flow instability and introduces a new, single-frequency component to the torque fluctuations that reflects a dominant velocity pattern in the flow. These results highlight the interplay between particles and a suspending viscoelastic fluid at very high shear rates.

\end{abstract}
\keywords{Suspension flow$|$ Viscoelastic fluid flow $|$ Elastic instability $|$ Phase separation} 
\begin{document}

\title{Anomalous crystalline ordering of particles in a viscoelastic fluid under high shear}

\author[a]{Sijie Sun}
\author[b,c]{Nan Xue} 
\author[a,d]{Stefano Aime}
\author[a,f]{‪Hyoungsoo Kim}
\author[a,g]{Jizhou Tang}
\author[e]{Gareth H. McKinley}
\author[b]{Howard A. Stone}
\author[a,1]{David A. Weitz}
\affil[a]{John A. Paulson School of Engineering and Applied Sciences, Harvard University, Cambridge, Massachusetts, 02138, USA}

\affil[b]{Department of Mechanical and Aerospace Engineering, Princeton University, Princeton, New Jersey 08544, USA}
\affil[c]{Department of Materials, ETH Z{\"u}rich, Z{\"u}rich, 8093, Switzerland}
\affil[d]{ESPCI, Paris, 75005, France}
\affil[e]{Department of Mechanical Engineering, Massachusetts Institute of Technology, Cambridge, Massachusetts, 02139, USA }
\affil[f]{Department of Mechanical Engineering, Korea Advanced Institute of Science and Technology (KAIST), 291 Daehak-ro, Yuseong-gu, Daejeon 34141, Republic of Korea}
\affil[g]{State Key Laboratory of Marine Geology, Tongji University, Shanghai,201804, China}
\correspondingauthor{\textsuperscript{1}To whom correspondence should be addressed. E-mail: weitz@seas.harvard.edu}

\date{\today}


 \maketitle


\ifthenelse{\boolean{shortarticle}}{\ifthenelse{\boolean{singlecolumn}}{\abscontentformatted}{\abscontent}}{}

Suspensions of solid particles in a fluid are widely encountered in all ranges of technology both as products and as precursor materials in manufacturing and materials preparation. They exhibit a wide range of properties depending on the particle concentration or volume fraction, $\phi$, and on the nature of the suspending fluid.  When the suspending fluid is Newtonian, the viscosity of the suspension increases slowly with increasing $\phi$ at low volume fractions, and then diverges as $\phi$ approaches the maximum volume fraction of randomly packed solid spheres~\cite{wyart2014discontinuous, Weeks2014,brady_bossis_1985}.  As the shear rate increases, higher-volume-fraction suspensions exhibit slight shear thinning with the viscosity decreasing with increasing shear rate, while at very high shear rates, they undergo a sudden, dramatic shear thickening, becoming almost solid-like due to increased interparticle collisions that result in frictional interaction between the particles~\cite{seto2013discontinuous, Melrose1996}.  The behavior of a suspension is significantly different if the suspending fluid is viscoelastic~\cite{varga2019hydrodynamics}.  At low $\phi$ and low shear rates, the particles follow streamlines, and the behavior of the suspension is similar to that of the suspending fluid~\cite{barnes_review_2003,mewis2012colloidal}. As the shear rate increases, however, the large polymer molecules become increasingly stretched and cannot fully relax, resulting in a stress normal to the streamline, causing particle chaining~\cite{michele1977alignment,feng1996motion,won2004alignment,scirocco_effect_2004} and shear thickening~\cite{vazquez2019shear,yang_mechanism_2018,matsuoka_prediction_2021}.  At high shear rates, the stretched polymer exerts yet larger elastic stresses; even in the absence of particles, these stresses dominate over viscous stress and can destabilize the flow field, resulting in an elastic instability and so lead to the development of a secondary chaotic flow~\cite{groisman_elastic_2004,mckinley_observations_1991,shaqfeh1996purely, datta2022perspectives}. This instability makes it challenging to quantify the behavior of a viscoelastic suspension of particles at high shear rates; as a result, only low shear-rate regimes, where the instability is avoided, have been investigated, and the behavior of particles at high shear rates remains unexplored. However, the interplay between fluid elasticity and particles at high shear rates is important for a wide range of industrial processes, such as hydraulic fracturing~\cite{goel2002correlating,barbati2016complex,browne_shih_datta_2020} and polymer composite processing~\cite{astrom2018manufacturing,barnes2003review}, as well as natural phenomena like biofluid flow~\cite{Brust2013,stokes2007viscoelasticity,kumar2012mechanism}. 

In this paper, we explore the interactions between particles and a viscoelastic fluid at high shear rates.  We study flow between parallel plates in a rheometer and discover a surprising crystalline structure of particles that forms in the middle layer of the flow cell. This structure suppresses the elastic instability and significantly alters the flow dynamics.  In the absence of particles, the elastic instability leads to strong fluctuations in the torque measured with the rheometer. These fluctuations have a power-law spectrum reminiscent of turbulent flow.  The particles result in an emergence of a single dominant frequency in the fluctuations that persists above the underlying power-law spectrum. Using flow visualization techniques, we show that this single frequency is a result of rigid body rotation of a nearly static structure formed by the crystalline layer of particles.  These measurements highlight the complex interactions between particles and a viscoelastic background fluid at high shear rates.

    
\section*{Influence of the particles on the viscoelastic secondary flow}
The particles are monodisperse Poly(methyl methacrylate) (PMMA) with a diameter of 51 {\textmu}m (CA50 from MICROBEBS). The solution phase is a mixture of 84.596 wt\% 2-2 Thiodiethanol (CAS 111-48-8 from Sigma) and 15.404 wt\% deionized water. This mixture has a refractive index of 1.488 at 20°C, which precisely matches that of the PMMA particles. The density of the solvent is 1.18 g/ml, which is close to that of the particles (1.19 g/ml). We add 0.25 wt\% polyethylene oxide (PEO, CAS 25322-68-3 from Sigma) of molecular weight 8M Da to the solvent to make it viscoelastic. At a shear rate of $10~\mathrm{s}^{-1}$, the ratio of elastic to viscous stress is approximately 6. 


\begin{figure*}[h]
\centering
\includegraphics[width=7in]{figures/Fig_1.pdf}
\caption{Elastic instability in a viscoelastic fluid with and without  suspended particles. ({\it A}) Sketch of the setup. Optics for imaging is mounted beneath a rheometer. Viscoelastic liquid with and without monodisperse particles are sheared between parallel plates. A cup is added to the bottom plate to avoid spilling the sample under high shear rates. ({\it B}) The  torque reading $M$ measured by the rheometer; fluctuations are evident. The gap size $H = 2$ mm and the radius of the upper plate $R = 25$ mm. The jade green and coral orange lines show the torque fluctuations for the experimental conditions described in ({\it A}), with and without 30 vol\% particles ($d = 51$ {\textmu}m), respectively. The shear rate $\dot{\gamma} = 50~\mathrm{s
^{-1}}$. ({\it C}) The Power Spectral Density (PSD) of the torque measurement in ({\it B}). The PSD of the polymer solution has a power-law relationship with the frequency, implying the flow is turbulent-like. With the addition of the suspension, the torque has a single peak with a frequency of $f_0 \approx 0.323$ Hz, highlighted by the red circle. The single peak disrupts the original power-law decay. The black dashed line represents the rotation frequency of the upper plate. A five-point moving average is applied to the PSD to highlight the trend.
({\it D}) The flow pattern of the viscoelastic liquid without suspension. $H = 2$ mm, $R = 25$ mm, and $\dot{\gamma} = 50~\mathrm{s
^{-1}}$. Multiple spiral patterns coexist. ({\it E}) The secondary flow pattern with 30 vol\% particles (diameter $d = 51$ {\textmu}m). The other experimental conditions are the same as ({\it D}). One spiral pattern dominates over the others.
}
\label{fig:1}
\end{figure*}
We use a stress-controlled rheometer (MCR 501 from Anton Paar) with a parallel plate geometry that has a 50 mm diameter and a 2 mm gap. At high shear rates, the viscoelastic fluid can entrap air bubbles and expel liquid from the flow cell, making measurements difficult~\cite{keentok1999edge}. To overcome this problem, we modify the rheometer geometry by adding a cup to the lower plate. This allows the rotating upper plate to be fully immersed in the fluid, as shown in Fig.~\ref{fig:1}{\it A}. This eliminates the liquid-air interface at the edge of the plates, thereby preventing the expulsion of the sample from between the parallel plates. We add enough fluid to ensure that the velocity gradient is much larger between the plates than between the upper surface and the free surface on the top.


In the absence of particles, the torque is constant and independent of time at low shear rates. As the shear rate increases, the solvent exhibits shear thinning, but the torque remains time independent. With further increase of shear rate, the solvent undergoes a sudden and dramatic shear thickening, and the torque exhibits pronounced fluctuations in time. We characterize the magnitude of these fluctuations by the ratio of the standard deviation of the torque ${\sigma_ M} $ to the mean value of the torque ${\overline{M}}$. For a shear rate of $50~\mathrm{s}^{-1}$, the fluctuations have a relative magnitude of 6\%, as shown in Fig.~\ref{fig:1}{\it B}. The torque fluctuations suggest that the flow is time-dependent and unstable, reflecting the presence of a secondary flow. To further characterize these fluctuations, we calculate the Fourier transform of the torque and determine the Power Spectral Density (PSD) from the square of the amplitude of the fluctuations as a function of frequency. The PSD exhibits a power-law decay proportional to $  f^{-k}$, where $k \approx 4.5$ for frequencies, $f$, above a rollover frequency of approximately 0.1 Hz, as shown in Fig.~\ref{fig:1}{\it C}. The power law spectrum is reminiscent of a chaotic flow.

Upon adding particles at a volume fraction of 30\%, the suspension exhibits shear thinning as the shear rate increases similar to the behavior in the absence of the particles, as shown in Figure~\ref{fig:A8} in the SI. At even higher shear rates, the particle suspension exhibits a pronounced onset of shear thickening similar to the behavior in the absence of the particles; interestingly, however, this occurs at a lower shear rate and the degree of shear thickening is not as large as it is in the absence of particles.
%We observe a similar shear thinning in the suspension as the shear rate increases when we add 30 vol\% particles, as demonstrated in Figure~\ref{fig:A8} in the SI. Additionally, the suspension also exhibits a pronounced onset of shear thickening, which interestingly occurs at a lower shear rate compared to when no particles are present; but the degree of shear thickening in the particle suspension is not as significant. 
The magnitude of the torque is considerably larger at a shear rate of 50~{$\hbox{s}^{-1}$}; the time dependence again exhibits fluctuations but with a single, dominant frequency, as shown in Fig.~\ref{fig:1}{\it B}. This response is reflected in the PSD, which still exhibits a power-law decay proportional to $  f^{-k}$, where $k \approx 4.5$. However, at a frequency of $f_0 = 0.323$ Hz, the PSD also exhibits a sharp peak that is two orders of magnitude larger, as shown by the red dashed circle in Fig.~\ref{fig:1}{\it C}. The presence of a pronounced peak in the PSD spectrum suggests that the suspended particles modify the chaotic flow.


To visualize the secondary flow, we employ a transparent bottom plate and add titanium-dioxide-coated mica flakes (0.5 wt\% from HTVRONT) to form a rheoscopic liquid. Under shear, the mica flakes align with the flow direction, allowing visualization of the secondary flow. For the polymer solution alone, the onset of the secondary flow is observed at the same shear rate that fluctuations in torque occur. At higher shear rates, when the elastic instability is fully developed, we observed multiple outward-flowing spirals that intermingle, but there is no dominant pattern, as shown for a shear rate of $50~\mathrm{s}^{-1}$ in Fig.~\ref{fig:1}{\it D} and Movie S1.



The introduction of particles leads to a significant alteration in the patterns of secondary flow resulting from the elastic instability. The multiple intermingled spiral patterns are no longer discernible. Instead, a single dominant spiral propagates from near the center to the periphery, as depicted in Fig.~\ref{fig:1}{\it E}. and Movie S2. The frequency of the spiral rotation is 0.3 Hz, which closely corresponds to the dominant frequency of the torque fluctuations, 0.32 Hz. This observation suggests that the primary, single-frequency torque oscillation stems from the pronounced spiral in the flow pattern.




\section*{ Dynamic Mode Decomposition of the flow structures}

\begin{figure*}[h]
\centering
\includegraphics[width=4.5 in]{figures/Fig_2.pdf}
\caption{Quantification of the viscoelastic secondary flow by PIV analysis and optDMD decomposition.
({\it A}) The fluorescent image of the polymer solution in the flow cell overlaid by the velocity field extracted from PIV analysis in a single frame. The particles move predominantly in the azimuthal direction.
({\it B}) The reconstruction of the velocity field of the viscoelastic liquid without particles from the optDMD method. The sky blue line is the measured velocity in the $x$ direction in time, averaged over the entire flow field. The jade green line is the reconstructed value  by the optDMD method based on 11 structures.
({\it C}) Comparison of the contribution of the coherent structures from the optDMD method, with and without the presence of the suspension. The addition of the suspension increases the contribution of the primary flow, which is Structure 1, suggesting the flow is less turbulent. Moreover, one single flow structure, structure 2, dominates over the others. 
({\it D}) The reconstruction of the velocity field of the viscoelastic suspension flow from the optDMD method.
The sky blue line is the measured velocity in the $x$ direction in time, averaged over the entire flow field. The pink line is the reconstructed value by the optDMD method with the primary and the first secondary flow structure. The jade green line is the reconstructed value based on all 11 structures.}

\label{fig:2} 
\end{figure*}
To better understand the impact of particles on the flow field, we employ particle image velocimetry (PIV) to directly measure the velocity field. For this experiment, we add a small amount (0.25 vol\%) of fluorescently labeled CA50 particles to the polymer solution. These particles are labeled with rhodamine B and are illuminated by a collimated LED with a wavelength of 505 nm and power of 220 mW (M505L4-C1 from Thorlabs). The velocity field is imaged with a camera (BFS-U3-32S4M-C from FLIR) at a frame rate of 200 fps. The velocity field exhibits a spiral-like pattern, as depicted in Fig.~\ref{fig:2}{\it A} and Movie S3.


To further analyze the spiral patterns observed in the velocity field, we apply optimized dynamic mode decomposition (optDMD) to identify a set of coherent structures in the flow field that each possess a single frequency and fixed growth rate. These structures characterize the fluid flow, with the first coherent structure corresponding to the primary flow and each additional structure contributing to the secondary flow~\cite{schmid2010dynamic, lee2021optimized}. By using optDMD, we are able to analyze more precisely the complex flow patterns present in the system.

The elastic instability in the polymer solution without particles is well described by the primary flow and ten secondary coherent flow structures. By summing all 11 coherent structures, we are able to adequately reproduce the measured velocity field, as demonstrated by the excellent agreement between the calculated and measured time dependence of the average horizontal velocity of the entire velocity field $U_x$, as shown in Fig.~\ref{fig:2}{\it B}. The first coherent structure, representing the primary flow, accounts for approximately 45\% of the total, while the next five structures, representing the secondary flow, each contribute at least a few percent, as illustrated in pale pink in Fig.~\ref{fig:2}{\it C}.


To investigate the effect of the particles on the flow field, we repeat the PIV and optDMD analysis using a mixture of 0.25 vol\% fluorescently labeled particles and 29.75 vol\% nonfluorescent particles. The resulting particle velocity field is again well described using 11 coherent structures, as shown by the sum of the velocity fields in Figure 2D. However, we find that with the particles, the contribution of the first coherent structure becomes significantly larger, accounting for approximately 73\%. The second coherent structure becomes the predominant contributor to the secondary flow, with a contribution of 9\%, while the contribution of the remaining structures decreases significantly, as depicted in jade green in Fig.~\ref{fig:2}{\it C}. 


The importance of the first and second coherent structures in describing the flow is further emphasized by their contribution to the time dependence of the average horizontal velocity of the entire velocity field $U_x$. When we decompose the flow into two structures, the resulting velocity field nearly matches the measured value. Adding the next nine structures results in only moderate improvement of the agreement, as shown in Fig.~\ref{fig:2}{\it D}. This highlights the significance of these two structures in describing the flow field in the presence of particles. 


For the polymer solution without particles, the velocity field of the primary flow is circular, with almost all velocity vectors pointing in an azimuthal direction, as depicted in the first coherent structure in Fig.~\ref{fig:3}{\it A}. The secondary flow has a spiral character, similar to the flow patterns observed with rheoscopic visualization, as shown in the higher-order structures in Fig.~\ref{fig:3}{\it B} and~\ref{fig:3}{\it C}. 

For the polymer solution with suspended particles, the velocity field of the primary flow is again circular, with almost all velocity vectors pointing in an azimuthal direction, as shown in the first coherent structure in Fig.~\ref{fig:3}{\it D}. Intriguingly, the velocity field of the secondary flow exhibits a very different character when particles are present. The velocity field is roughly constant near the center and rotates over time like a rigid body, surrounded by an outer layer of spirals, as shown in the second coherent structure in Fig.~\ref{fig:3}{\it E}, and Movie S4. However, the higher-order structures return to spiral patterns, as shown, for example, in the third coherent structure in Fig.~\ref{fig:3}{\it F}.



\begin{figure*}[htbp]
\centering
\includegraphics[width=7in]{figures/Fig_3.pdf}
\caption{Coherent structures of the flow based on DMD analysis. We plot the real part of the primary flow and the imaginary part of the secondary flow here.
({\it A}) The first coherent structure corresponding to the primary flow of the polymer solution. Most of the velocity vectors point in the azimuthal direction.
({\it B},{\it C}) The second and third coherent structures of the viscoelastic flow. The chaotic secondary flow consists of complex spiral structures, which propagate and intermingle over time.
({\it D}) The first coherent structure in the primary flow of the polymer solution with 30 vol\% suspension particles. Again, most of the velocity vectors point in the azimuthal direction.
({\it E}) The second coherent flow structure of the polymer solution with 30 vol\% suspension particles. The secondary flow consists of a rigid disk with a defect line in the middle and spiral patterns surrounding it.
({\it F}) The third coherent flow structure of the polymer solution with 30 vol\% suspension particles. The secondary flow is again a spiral whose velocity is smaller in the middle.
({\it G}) The evolution of {\it E} over time. The velocity field rotates over time like a rigid body.
}
\label{fig:3} 
\end{figure*}

 Notably, in the presence of particles, the second coherent structure rotates at a frequency of 0.31 Hz, which matches the frequency peak in the torque PSD (0.32 Hz). The swirling flow pattern observed for this sample with rheoscopy also has a frequency of 0.3 Hz. These findings suggest that the predominant contribution to the secondary flow in this system arises from the rotation of a nearly static structure in the center of the sample at a frequency of 0.3 Hz. 
 

\section*{Rheomicroscopic visualization of the particle crystallization}
To further examine the apparent rotation of a nearly rigid body that contributes significantly to the secondary flow in the presence of particles, we observe the flow in the middle plane using a rheomicroscope (MCR 702 with the rheo-microscope accessory from Anton–Paar). This device consists of counter-rotating disks that maintain the middle plane stationary, allowing it to be visualized with a microscope and imaged with a high-speed camera (v7.3 from Phantom). By strongly shearing the sample with 30 vol\% particles up to a shear rate of $\dot{\gamma} \sim 10^2 ~\mathrm{s}^{-1}$, where the elastic instability is fully developed, the liquid is rapidly expelled, and we can only briefly visualize the particles. In doing so, we observe the formation of an ordered layer of particles in the stationary middle plane, as shown in SI Fig. \ref{fig:A3} and SI Video 5. 


To study the ordering of particles in the midplane in more detail, we utilize the large-amplitude oscillatory shear (LAOS) protocol, which imposes a sufficiently large shear to induce structure formation, but, because it is oscillatory, does not expel the sample from the geometry ~\cite{hyun2011review}. We use an angular velocity $\omega = 2\pi~\si{rad/s}$ and a shear amplitude of $\gamma_0=3000\%$, resulting in an average applied shear rate of $\dot{\gamma}=2\omega\gamma_0/\pi=120~\mathrm{s}^{-1}$. The oscillatory motion also leads to the ordering of particles in the midplane, similar to that observed in steady shear. In this case, the ordering begins almost immediately, and after sixty seconds, a steady state is reached, where the torque applied to drive each oscillatory cycle becomes unchanged from cycle to cycle, suggesting that the particle structure has reached a steady state. In this steady state, the particles form an ordered structure that is retained through each cycle of LAOS, with the direction of particle motion reversing as the flow reverses each cycle. This allows us to visualize the details of the structure that gives rise to the secondary flow pattern. By observing the same area of particles in the flow geometry that remain in the field of view at the end of each cycle, we can track the configuration of these particles over time and gain insight into the structure contributing to the secondary flow pattern.


Using LAOS, we observe the assembly of particles even at a volume fraction as low as  $\phi$ =5 vol\%. At this concentration, local structures consist of short chains and small rafts of these chains are formed. Remarkably, the particles become concentrated in the midplane of the sample. These structures further align with the flow and move in the flow direction. There is sufficient distance between the chain-like assemblies to allow relative motion between neighboring structures without collision. Therefore, although the particles assemble under shear, the resultant structures remain dispersed, occupying only about 20\% of the area, as shown in Fig.~\ref{fig:4}{\it A}. 


\begin{figure}[h!]
\centering
\includegraphics[width=4.5 in]{figures/Fig_4.pdf}
\caption {Crystallization of the suspended particles under shear. ({\it A}) Rheo-microscopic visualization of the particles assembling under shear. We repeat the experiments with 5 vol\%, 20 vol\%, and 30 vol\% viscoelastic suspensions, shown as different columns. The first row is the control experiment without shear. Particles remain suspended. The second row is the same system as the first row, but the suspension is sheared under $1$ Hz oscillations with an amplitude $\gamma_0 = 30$. In the lower left, 5 vol\% particles chain up under shear, but there is space between the chains, which can still be considered suspended. In the lower middle, particles crystallize into 2D close-packed structures under shear. Those structures further overlap with each other, lock with their neighbors,  and move collectively in each LAOS cycle. In the lower right, a 30 vol\% suspension is sheared under the same condition. The suspension behaves similarly to the 20 vol\%  case. Particles crystallize into rafts. ({\it B}) Qualitative explanation of viscoelastic suspension under strong shearing. The suspended particles phase separate and crystallize into a solid core, which then rotates over time.}
\label{fig:4} 
\end{figure}
Surprisingly, when the volume fraction of particles is increased to  $\phi$ =20 vol\%. the attractive interactions among particles lead to phase separation. The particles crystallize into two-dimensional (2D) closely packed rafts instead of chains. These rafts migrate like a rigid body during each LAOS cycle. Due to the high concentration, the rafts overlap and become locked together, forming even larger structures, as seen in the lower middle of Figure 4.A Increasing the volume fraction to  $\phi$ =30 vol\%. results in even larger rafts, but with qualitatively similar structures, as shown in the lower right of Fig.~\ref{fig:4}{\it A}. This crystallization is unusual: in a monodisperse, hard-sphere suspension of Brownian particles, crystallization first occurs at  $\phi \approx 54.5\%$ ~\cite{hoover1968melting}, which is much higher than the 20\% particle volume fraction where we first observe rafts.


The formation of crystalline layers is due entirely to the interactions induced by the viscoelastic fluid; no such effects are observed for particles suspended in a Newtonian fluid. The elasticity of the fluid drives each particle to migrate to the middle plane and further induces an attractive interaction between the particles that causes chain formation along the direction of flow. At low volume fractions, these interactions result in chains of particles being separated from one another, as seen in the lower left of Fig.~\ref{fig:4}{\it A}. As the volume fraction increases, the density of the chains also increases, eventually leading to the formation of crystals.



The formation of a crystal of particles is the cause of the behavior of the secondary flow reflected in the second coherent structure. The crystal structure persists in the flow but does not fully align with the streamlines. Instead, a thin crystal plug forms in the center of the flow cell, resisting the viscoelastic shear flow and preventing the development of secondary flow in the middle and suppressing the elastic instability. Furthermore, the torque from the upper rotating plate drives the particle crystal to rotate at a constant rate, leading to the single frequency fluctuations seen in the torque measurements and in optDMD, as shown in Fig.~\ref{fig:4}{\it B}.


\section*{Polydisperse particles erase the regulation}


\begin{figure}[h]
\centering
\includegraphics[width=8.7 cm]{figures/Fig_5.pdf}
\caption{Secondary flow of 30 vol\% polydisperse viscoelastic suspensions. ({\it A}) 30 vol\% polydisperse particle suspension in the LAOS test under the microscope. The in-plane volume fraction is high, suggesting the particles have attractive interactions and migrate toward the middle plane. However, there is no crystallization due to the polydispersity. ({\it B}) The PSD of the torque reading as a function of frequency. No sharp peak can be observed. ({\it C}) The optDMD analysis. In line with the rheoscopic visualization, in the polydisperse  suspension, no predominant mode can be seen. ({\it D}) The rheoscopic visualization of the sample under shear. Multiple spirals coexist, and no predominant motion is observed.}
\label{fig:5} 
\end{figure}

One of the most striking features of the structure is its crystalline order, despite the relatively low total volume fraction of particles. The particles are close packed, suggesting that the order arises due to the monodispersity of the particles. To test this hypothesis, we prepare a polydisperse suspension consisting of a mixture of 10 vol\% 9.2 micron-diameter, 10 vol\% 30 micron-diameter, and 10 vol\% 51 micron-diameter particles. When we perform LAOS measurements under the same conditions, we observe a concentrated layer of particles at the midplane, but with no crystalline order because of the polydispersity, as shown in Fig.~\ref{fig:5}{\it A}. ~\cite{pusey1987effect}. 

In addition, there is no single frequency peak in the power spectrum, as shown in Fig.~\ref{fig:5}{\it B}. The optDMD analysis also shows that the only dominant coherent structure is the first, while the subsequent five coherent structures each have similar amplitudes, as shown in Fig.~\ref{fig:5}{\it C}. The rheoscopic visualization reveals multiple indistinct spiral patterns, but no predominant pattern can be identified, as shown in Fig.~\ref{fig:5}{\it D}. This indicates that the secondary flow is suppressed by the suspension, while the swirling motion of the secondary flow spreads throughout the entire flow field. These observations confirm the critical role of crystalline order in controlling the secondary flow.


\section*{Conclusion}
This study has shown that the presence of particles in a viscoelastic polymer solution significantly alters the flow at high shear rates, leading to the formation of a crystalline layer of particles. This crystalline layer resists the viscoelastic shear flow and inhibits the development of the elastic instability, thereby regulating the secondary flow. Furthermore, the rigid body rotation of this crystal is responsible for the dominant frequency in the flow structure. These results have important implications for industrial settings where ordering is likely to modify to the processing of particle suspensions in viscoelastic fluids.  It would be very interesting to examine whether similar crystalline ordering occurs in other geometries.


\section*{Acknowledgement}
We are grateful for insightful discussions with Dr. Andreas Bausch, Dr. Jörg G Werner, Dr. Shima Parsa. We also acknowledge the support of the NSF through the Harvard MRSEC (DMR20-11754).
\bibliography{References.bib}
\subfile{Appendix}
\end{document}
%
% ****** End of file ******