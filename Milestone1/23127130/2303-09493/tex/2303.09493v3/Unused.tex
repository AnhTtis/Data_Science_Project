\section{Scaling of Torque Oscillation}

The sharp peak in the PSD is very robust and is present for all angular velocities of the upper rotating plate, $\Omega$, once $\Omega$ is above the onset of the shear thickening. 
The peak frequency $f_0$ varies linearly with $\Omega$:
$$f_0\approx 0.025~\Omega,$$
as shown in Fig.~\ref{fig:1}{\it D}.

To elucidate the nature of the single frequency oscillation of torque, we consider its origin: the torque is the integration of the product between the distance to the plate center and shear stress $\sigma_{\theta z}$ over the entire upper plate,where $\theta$ is the azimuthal direction and $z$ is the vertical direction. Shear stress $\sigma_{\theta z}$ depends on the local velocity gradient. A fluctuating torque means velocity gradient varies in time, suggesting that the flow field is time-dependent and thus unsteady. Torque fluctuates when we shear viscoelastic liquid without particles as in Fig.~\ref{fig:1}{\it B}, but remain constant when we shear particle suspension with no polymer (Fig.~?? in {\it SI Appendix}). Therefore, we suspect that the unsteady flow arises from the fluid elasticity.

The fluid elasticity triggers the elastic instability and disrupts the time-independent primary flow, leading to the development of a time-dependent flow field. The time-dependent part of the flow is called the secondary flow, in contrast to the primary flow. The suspension flow inherently have two distinct length scales, the length scale of the flow geometry, which is the radius of the rotating plate $R$, and the length scale of each single particle $d$. The elastic instability does not depend on the flow length scale and can develop under both $R$ and $d$\cite{mckinley1996rheological,yang_mechanism_2018}. Therefore, there may exist two secondary flows, one around each particle and one across the entire flow geometry, both contributing to the torque reading. However, we can differentiate which secondary flow dominates the torque fluctuation by investigating their different time dependence. 

The flow around a particle depends on the local shear rate $\dot{\gamma}_\mathrm{local}$. There is another shear rate: the average shear rate $\dot{\bar{\gamma}}$, where $\dot{\bar{\gamma}}= 2\Omega R/(3H)$ for the shear flow between parallel plates. Here $H$ is the gap height of the parallel plates. The local shear rate is higher than the average shear rate $\dot{\bar{\gamma}}$ because the liquid is displaced by the particles\cite{mewis2012colloidal}. For a particle suspension, $$\dot{\gamma}_\mathrm{local}=\frac{1}{(1-\Phi/\Phi_\mathrm{max} )^2}\dot{\bar{\gamma}},$$ where $\Phi_\mathrm{max}$ is the maximum volume fraction of the particles can be packed in the system \cite{quemada1977rheology}. For the flow spanning the entire flow geometry, the time scale is comparable with $1/\Omega$. \textcolor{red}{Sijie: It is intuitively correct but I do not know how to explain it.}  % I do not know why but it is what Gareth paper says 

The different time dependence of the secondary flows allow us to reveal which secondary flow cause the torque fluctuation by studying whether $\Omega$ or $\dot{\gamma}_\mathrm{local}$ controls $f_0$. Since $\dot{\bar{\gamma}}=2\Omega H/3R$,  we can vary the $\dot{\gamma}_\mathrm{local}$-to-$\Omega$ ratio by tuning $H/R$. We start with flow geometries with different radius $R$ = 30 mm and $R$ = 20 mm (both on a DHR-3 rheometer from TA instrument). The oscillation frequency $f_0$ with $R$ = 20 mm plate plotted versus rotation rate $\Omega$ as hexagrams in Fig.~\ref{fig:1}{\it D} shares the same trend as the data with $R$ = 30 mm. This implies that the oscillation frequency of the torque $f_0$ is controlled by $\Omega$ but not $\dot{\bar{\gamma}}$. We then confirm this hypothesis by letting $H$ steps from 0.5 mm to 2.5 mm in 0.5 mm increments. As a result, there is no notable difference in the scaling of the oscillation frequency against $\Omega$, as different heights represented by different color codes fall in the same line in Fig.~\ref{fig:1}{\it D}. This independence of $f_0/\Omega$ on $R$ or $H$ indicates that the dominating unsteady flow is not from the particle scale.

If the oscillation frequency is dictated by the flow-geometry-scale flow, then $f_0$ should remain unchanged when $\dot{\gamma}_\mathrm{local}$ increases while $\Omega$ remains constant.
This can be achieved by increasing the volume fraction of the particles $\Phi$, which decreases the average spacing among the particles, leading to an increase in $\dot{\gamma}_\mathrm{local}$ under the same $\Omega$. Under different volume fractions, the same $f_0$ - $\Omega$ scaling is observed. Samples with volume fractions $\Phi$ = 20\% (pentagrams), 30\% (circles), and 40\% (triangles) in Fig.~\ref{fig:1}{\it D} follow the same linear trend. A higher concentration leads to only a larger oscillation amplitude, but the oscillation frequency remain uninfluenced. This cross-validates the hypothesis that the scale of dominating secondary flow is comparable with the flow geometry.

For completeness, we test samples with different particle sizes and different relaxation time of the polymer solution. These parameters do not influence $\Omega$ or $\dot{\gamma}_\mathrm{local}$, so we expect the $f_0$ remain unchanged.
Indeed, We test three $\Phi = 30\%$ samples with particles diameters of 9.1, 30, and 51 {\textmu}m, respectively.
We find no difference in their power spectra, as the rhombus, squares, and the circles fall in the same linear trend in Fig.~\ref{fig:1}{\it D}.
We also lower the polymer concentration in the viscoelastic liquid to decrease the relaxation time from 0.8 sec to 0.1 second. The oscillation frequency is the same; see Fig.~\ref{fig:A3} in {\it SI Appendix}.
As a result, neither the particle size or the relaxation time of the polymer affects $f_0(\Omega)/\Omega$. 
%({\it C}). We vary the experiment condition and plot the oscillation frequency of the torque ${f_0}$ as a function of the rotation rate $\Omega$. The inset is ${f_0}/{\Omega}$ as a function of $\Omega$. Under all different working conditions, the peak frequency scales linearly with ${\Omega}$ with the same ratio.
%In the absence of the particles, the torque remains constant and independent on time at low shear rates. As the shear rate increases, the solvent exhibits shear thinning, and the torque remains constant for a fixed shear rate. Upon further increase of the shear rate, the solvent undergoes a sudden, dramatic shear thickening, and the torque exhibits pronounced fluctuations in time. For example, at a shear rate of 50 /sec, the torque fluctuations have a relative magnitude of 6 percent, measured by ${\sigma_ M/\overline{M}}$, where ${\sigma_ M} $ is the standard deviation of the torque and ${\overline{M}}$ is the mean value of the torque; see Fig.~\ref{fig:1}{\it B}.  To characterize these fluctuations, we calculate the Fourier transform of the torque and, from the square of the amplitude, determine the Power Spectral Density (PSD). The PSD exhibits a power law decay $\propto  f^{-k}$, where $k \approx 4.5$, for frequencies, $f$, above a rollover frequency of $ \sim 10^{-1} $ Hz; see Fig.~\ref{fig:1}{\it C}.  
%Upon the addition of particles at a volume fraction of 30\%, the suspension exhibits more shear thinning than the particle-free solvent as the shear rate increases ( as shown in Fig.~\ref{fig:A2} in {\it SI Appendix}). The suspension exhibits a similar, pronounced onset of shear thickening, whereupon the torque exhibits fluctuations in time; however, this occurs at a lower shear rate, and the degree of shear thickening is not as large as it is in the absence of particles. The magnitude of the torque is considerably larger at the shear rate of 50~/sec, and the time dependence again exhibits fluctuations, but with an apparent dominant frequency; see Fig.~\ref{fig:1}{\it B}. This is reflected in the PSD, which still exhibits a power law decay $f^{-k}$, where $k \approx 4.5$. However, this trend also exhibits a sharp, two-orders-of-magnitude-higher peak at a frequency of $f_0 = 0.323$ Hz; see Fig.~\ref{fig:1}{\it C}.

%Torque depends on the local velocity gradient. A fluctuating torque suggests the flow is time-dependent and unstable, which is the indication of the secondary flow. We can visualize this secondary flow by outfitting the rheometer with a customized transparent bottom plate above an aligned mirror and camera for visualization. Titanium-dioxide-coated mica flakes (0.5 wt\% from HTVRONT) are added to the polymer solution to form a rheoscopic liquid. Under shear, the mica flakes align with the flow direction, allowing the flow field to be qualitatively assessed via the reflected pattern.


%\begin{figure}[h]
%\centering
%\includegraphics[width=3.42 in]{figures/Fig 2.pdf}
%\caption[Rheoscopic visualization of the elasticity-induced secondary flow ]{Rheoscopic visualization of the secondary flow. ({\it A}) The flow pattern of the viscoelastic liquid without suspension. The gap size $H = 2$ mm, the radius of the upper plate $R = 25$ mm, and the shear rate $\dot{\gamma} = 50~/\mathrm{sec}$. Multiple spiral patterns coexists. ({\it B}) The secondary flow pattern with 30~vol\% particles in a diameter of $d = 51$ {\textmu}m. The experimental condition is the same as ({\it A}). One spiral pattern dominates over the others. ({\it C}) The flow pattern in ({\it B}) with normalized illumination. The spiral is highlighted as the black dashed line. ({\it D}) Control experiment with $\dot{\gamma}=100\,\mathrm{/sec}$ and $H = 2\,\mathrm{mm}$. The shape of the promoted mode is very similar to ({\it B}). ({\it E}) Rheoscopic visualization of the secondary flow when $\dot{\gamma}=100\,\mathrm{/s}$, and $H = 1\,\mathrm{mm}$, the inner end point of the spiral move outward when $H$ decrease.}
%\label{fig:2} 
%\end{figure}

%When we shear the polymer solution with an increasing shear rate from 0/s to 50/s, a non-trivial flow pattern emerges once the torque fluctuates. Such conjunction shows that the torque fluctuation arises from the change of the flow field. 
%By rotating the upper plate at a constant shear rate $\dot{\gamma} = 50/s$ for 60 seconds, letting the secondary flow fully develop, we observe multiple spirals in the flow pattern. Over time, these patterns spiral outward and intermingle, and no dominant pattern can be found, as shown in Fig.~\ref{fig:1}{\it D} and Movie S1. These patterns are the manifest of the elastic secondary flow.% The momentum is exchanged among the spirals in the turbulent flow, and thus, no dominant pattern can be found.

W%hen adding the particle suspensions, the flow patterns change dramatically.
%The torque reading fluctuates similarly to the tests without particles, suggesting that a similar secondary flow exists, see Fig.~\ref{fig:1}{\it B}. However, 
%The patterns previously observed are indiscernible. Instead, one flow pattern dominates over the others. The dominant pattern appears as a spiral, which propagates from  a position between the center and periphery to the edge of the flow geometry, as shown in Fig.~\ref{fig:1}{\it E} and Movie S2. Moreover, the measured frequency of the torque oscillation is 0.32 Hz, which agrees with the frequency of the spiral rotation (0.33Hz).
%This agreement suggests that the single-frequency torque oscillation originates from the pronounced flow pattern in the flow field.

%For further investigations, we change the rotation rate or the height-to-radius ratio of the flow cell.We find the inner starting point moves slightly outward when the gap height is halved. But the flow patterns are nevertheless similar and always spiral from the middle half of the plate to the edge, see Figs.~\ref{fig:2}{\it C}-{\it E}. The similarity among flow patterns under different working conditions supports the linear scaling between the oscillation frequency and the rotation rate. Since the flow pattern is the same under different working conditions, and thus, increasing the rotation rate increases only the driving torque, not the secondary flow motion, leading to a linear increase in the flow speed. \textcolor{blue}{Nan: I think maybe we need to explain/describe this more.}
%The rheoscopic visualization describes the flow pattern but not the velocity. To further understand how exactly the particles change the flow field, we directly quantify the velocity field using fluorescent tracers and particle imaging velocimetry (PIV). To track the motion of the dense suspension, we label a small fraction (0.25 vol\%) of the particles as tracers in the suspension with Rhodamine B fluorophore, which enables distinguishing the single-particle motion under shear. We record the tracer particle trajectory at 200 fps for 60 seconds. The in-plane motion of the flow field is then reconstructed by the PIV postprocessing, which resolves the velocity at each small window by calculating the correlation of the window between frames. Over time, the fluorescently labeled particles move predominantly in the azimuthal direction, but some radial motion of the particles is also observed; see Fig.~\ref{fig:3}{\it A} and Movie S3. The tracer density remains constant throughout the measurement.
%We reduce the complexity of the polymer solution flow without particles at $\dot{\gamma}=50~/\mathrm{sec}$, $H = 2~\mathrm{mm}$ from 12000 frames to 11 coherent modes, including 1 primary flow and 10 secondary flow structures. It captures the essence of the flow, with less than 8 percent relative difference. Each secondary flow structure has two identical modes as complex conjugates.  Here we first compare the importance of these structures. If we sort the contribution of the first nine major modes, the primary flow contributes the most, as 45\%. The secondary flow modes have the same order of magnitude contribution, as bar 2 to bar 11 have similar heights, see Fig.~\ref{fig:3}{\it C}, and these modes are all spiral shapes (Fig.~\ref{fig:a6} in {\it SI Appendix}). In contrast, when we repeat the decomposition for the particle suspension, the contribution of the primary flow increased from 45 \% to 73\%, suggesting that the flow is less chaotic. And mode 2 \& 3 contributes 9\%, while Modes 4 to 9 together contribute 10\%. The first secondary flow mode contributes substantially more than the subsequent modes, which suggests that the suspension suppresses higher-order modes and boosts one single mode, {\it i.e.},  Modes 2 \& 3. The predominance of modes 2 \& 3 can be revealed from another perspective; if we decompose the flow field into only the primary flow and one secondary flow, the relative error is less than 11\%. If we plot the horizontal velocity averaged over the entire flow field over time. We can see the first three modes is close to the measured velocity, and decomposing into more modes will yield an even better representation, see Fig.~\ref{fig:3}{\it B}.
%We use PIV to analyze the fluorescent images and calculate the velocity field that changes in time. 
%To understand the coherent flow structures in an unsteady flow, which creates the flow patterns we see under rheoscopy, we then use the optimized dynamic mode decomposition (optDMD) method to extract the time-invariant coherent structures \cite{wynn2013optimal}. The optDMD assumes that the time series $\textbf{v}(t)$ is not fully random but instead constituted by a series of coherent modes:
%$$\textbf{v}(t) \approx\sum_{i=1}^n{\textbf{v}_i(t)},$$
%where each component $\textbf{v}_i(t)$ has simple dynamics that grow exponentially while oscillating in a single frequency. Mathematically, $\textbf{v}_i(t)=e^{\alpha_i t}e^{i\omega_i t}\beta_i\textbf{v}_{i0}$. Here, $\alpha_i$ is the growth rate, which describes how rapidly the amplitude changes, $\omega_i$ is the angular frequency of the oscillation, and $\beta_i$ is the amplitude that reflects the contribution of the mode to the time series. Finally, the time-invariance normalized vector $\textbf{v}_{i0}$ describes the shape of the mode. The optDMD finds $\textbf{v}_i$ by solving the following minimization problem:

% $$\min_{\textbf{v}_i}\sum_{t=0}^{t_0}||\textbf{v}(t) -\sum_{i=1}^n{\textbf{v}_i(t)} ||_2,$$
%which iteratively searches for the closest representation under the Euclidean norm.

%During the optDMD processing, we decompose the flow at  $\dot{\gamma}=50~/\mathrm{sec}$, $H = 2~\mathrm{mm}$ into $n = 23$ coherent modes, which produces a relative
%Euclidean difference  $\sum_t||\textbf{v}(t) -\sum_{i=1}^n{\textbf{v}_i(t)} ||_2/\sum_t||\textbf{v}(t) ||_2$ of less than 8$\%$, as shown in Fig.~\ref{fig:3}{\it B}. In both cases with and without particle suspension, the first mode is the base flow, which is invariant in time. For all the modes, $\alpha_i$ equals 0 since the amplitude of the flow does not vary in time.
%The subsequent pairs of modes are complex conjugates and therefore share equal contributions (Fig.~\ref{fig:3}{\it C}). For all the modes, $\alpha_i$ equals 0 since the amplitude of the flow does not vary in time.
%In the case of no particle suspension, the base flow contributes approximately 45\%. The subsequent four pairs of modes are all spiral shapes (Fig.~??? in {\it SI Appendix}) and are of the same order of magnitude; see Fig.~\ref{fig:3}{\it C}. This result agrees with the Rheo-flow visualization in Fig.~\ref{fig:2}{\it A}: the elastic secondary flow between parallel plates is composed of a palette of spiral modes, and no predominant mode can be observed. In contrast, in the experiment with particle suspension, the relative contribution of the primary flow increases to approximately 73\%, suggesting that the overall flow fluctuates less. Modes 2 and 3 have an 8.8\% relative contribution, while Modes 4 to 9 together contribute 9.9\%. The first pair of secondary flow modes here contribute substantially more than the subsequent modes, which suggests that the suspension suppresses higher-order modes and promotes one single pair of mode, {\it i.e.},  Modes 2 and 3.

%The unexpected rigid body rotation drives us to then focus on the particle formation in the viscoelastic suspension flow.
%To directly quantify the evolution of the suspension microstructure under shear, we employ rheo-microscopy (MCR 702 with the Rheo-microscope accessory from Anton–Paar) to image the motion of every single particle {\it in situ}. In this setup, the upper and lower plate counter rotates. The focal plane of the 5x objective is set to the middle plane, where the flow stagnates. Therefore, the particles near the focal plane can stay in the field of view and be tracked, even when the shear rate is relatively high. Here, we strongly shear particle suspensions up to $\dot{\gamma} \sim 10^2/s$, where a high shear rate induces interfacial instability and leads to liquid expulsion. 

%If crystallization is the cause of the phase separation and the change of the flow dynamics, then the inhibition of the crystallization should prevent the phase separation and the rigid body rotation flow structure. Note that the crystallization requires the particle to be mono-dispersed \cite{pusey1987effect}.
%Therefore, if we shear a poly-dispersed sample, even if there is attractive interaction, the particles would not crystallize. Here we design a control experiment by mixing different sizes of the particles,  10 vol\% 9.2 {\textmu}m, 10 vol\% 30 {\textmu}m, and 10 vol\% 51 {\textmu}m particles.
%When we shear the poly-dispersed suspension in the LAOS test, no ordered structure is observed, even if the in-plane area fraction is much higher than 30\%. This observation shows that the particles still migrate towards the middle plane and self-assemble under the shear flow. However, the difference in particle sizes blocks the crystallization; see Fig.~\ref{fig:5}{\it B}.


%We then repeat the rheoscopic visualization of the poly-dispersed sample to investigate the influence of the poly-dispersity on the flow pattern. Compared with the no suspension case, the rheoscopic flow pattern here is indistinct, see Fig.\ref{fig:5}{\it E}, which suggests the secondary flow is suppressed by the suspension. However, importantly, no predominant pattern can be observed, instead, multiple spirals coexist. Also, in the rheometer reading, there is no single frequency peak in the power spectrum, see Fig.~\ref{fig:5}{\it C}.

%In the optDMD analysis, in line with the rheoscopic flow visualization, there is no predominant mode in the relative contribution, see Fig.~\ref{fig:5}{\it D}. The first ten modes of the secondary flow have contributions of the same order of magnitude, which is lower than the experiments with no particle suspension. Again, the secondary modes are spiral, suggesting the swirl motion from the secondary flow can propagate across the flow field while there is no crystallization; see Fig.~\ref{fig:A6} in {\it SI Appendix}. These control experiments prove that the spinning motion and the violation of power-law scaling are caused by the crystallization of the suspension under shear. 

%Suspensions of solid particles in a fluid are widely studied for their fascinating physical behavior and for their enumerous engineering applications. The properties of such suspensions become even more complex when the suspending liquid is a polymer solution, which is itself viscoelastic. Then, the interactions between the particles and the suspending viscoelastic fluid result in complex rheological behavior. For example, such a viscoelastic suspension typically exhibits shear thinning, where the viscosity decreases as the shear rate increases. Interestingly, in some cases, when the shear rate further increases and exceeds some critical value, the viscosity quite suddenly increases significantly. This is known as the Shear Thickening (ST) effect. \cite{vazquez2019shear,yang_mechanism_2018}. 
%Particles suspended in a viscoelastic fluid such as polymer solution are known as viscoelastic suspension. They are widely applied in engineering, including polymer processing \cite{shenoy2013rheology}, hydraulic fracturing \cite{barree1994experimental}, and microfluidic devices  \cite{zhou2020viscoelastic}.The interaction between particles and viscoelastic fluid results in complex rheological behavior. %The particles can dramatically change the rheological properties of the suspension.
%For instance, as the shear rate increases, the measured viscosity of the viscoelastic suspension first decreases, but when the shear rate exceeds a critical value, the viscosity significantly increases. This is known as the Shear Thickening (ST) effect. \cite{vazquez2019shear,yang_mechanism_2018}. 
%This effect is quite commonly observed; its origin remains unclear. 
%Insight into the behavior might come by analogy to the behavior of particles suspended in a Newtonian fluid\cite{lin2015hydrodynamic}. While such a suspension exhibit less shear thinning, it does exhibit an even more pronounced and more sudden onset of shear thickening at a certain shear rate. This effect is due to strong interparticle collisions causing jamming of the particles. However, to attain sufficient number of collisions requires a high volume fraction of particles and is therefor restricted to high particle concentrations. By contrast, the ST effect in the viscoelastic suspension occurs at a significantly lower volume fractions, implying its mechanism is different. Instead, the origin of the ST effect may arise from the viscoelastic behavior of the suspending polymer solution. When the shear rate is high, even in the absence of particles, the measured viscosity of the viscoelastic fluid also increases. This increase is due to the elastic stress of the dispersed polymer, which is stretched by the flow, and which becomes large enough to overcome the viscous stress that otherwise stabilizes the flow. As the shear rate increases, the elastic stress causes a secondary chaotic flow to develop; this is called the elastic instability. An important consequence of the elastic instability is that the elastic stress of the polymer, which leads to the instability, also can cause the polymer solution to be expelled from the container in which the measurement is made. Similar behavior is observed for viscoelastic suspensions, making the ST effect difficult to investigate. As a result, virtually all studies of the rheology of a viscoelastic suspension have focused on lower shear rates, where the elastic instability is avoided, precluding the investigation of the high shear rate flow. However, the behavior at these high shear rates has strong impact on many important technological applications such as hydraulic fracturing or polymer composite extrusion, and an understanding of its behavior is essential. 


%, making it crucial to understand the interaction between high shear rate viscoelastic flow and suspended particles. 

%Newtonian suspension also undergoes shear thickening above a critical shear rate \cite{lin2015hydrodynamic}.  However, compared to a viscoelastic suspension, the volume fraction required for shear thickening increases by order of magnitude, implying it has a different mechanism. 
%Another possible reason for shear thickening is the viscoelasticity of the liquid; when the shear rate is high, even in the absence of particles, the measured viscosity of the viscoelastic liquid also surges. This is because the extensive elastic stress exerted by the dispersed polymer overcomes viscous dissipation, destabilizes the flow field, and develops a chaotic secondary flow.\cite{groisman_elastic_2004,groisman2000elastic}. This is known as the elastic instability for the viscoelastic flow. 
%If the particles are suspended in a Newtonian liquid like water, when the shear rate is high, and the particles are sufficiently concentrated, the frequent particle-particle collision causes a significant increase in the measured viscosity of the suspension by the rheometer. This is known as the Shear Thickening (ST) effect.\cite{lin2015hydrodynamic}. 
%The physics becomes more complicated when the particles are suspended in a viscoelastic fluid such as polymer solution, so-called viscoelastic suspension.  Shear thickening occurs again in the viscoelastic suspension above a critical shear rate \cite{vazquez2019shear,yang_mechanism_2018}. Compared to a Newtonian suspension, the volume fraction required for shear thickening can be reduced by order of magnitude, implying it has a different mechanism from the Newtonian suspension and is presumably associated with the viscoelasticity of the fluid.  Under the microscope, people see particles chain up under shear in the viscoelastic suspension, suggesting elasticity of the fluid causes attractive interaction among the particles.
%Despite all of the intriguing phenomena that occur as the shear rate increases, viscoelastic suspension flow remains largely unknown when the shear rate is high. Partially due to the elastic instability that develops under strong shearing causes the liquid to be expelled from the flow geometry \cite{feng1996motion,keentok1999edge}. As a result, virtually all studies of the rheology of a viscoelastic suspension have concentrated on lower shear rates, where the consequences of elastic instability are avoided, and the intriguing interaction between high shear rate viscoelastic flow and suspension remains largely unknown. However, the shear rate can be considerably high in many scenarios, including hydraulic fracturing or polymer composite extrusion, making it crucial to understand the interaction between high shear rate viscoelastic flow and suspended particles. 

%interaction between the suspended particle and the viscoelastic liquid beyond the onset of shear thickening of the viscoelastic suspension remains largely unknown. Because in the viscoelastic liquid, when the shear rate is above a critical value, the extensive normal stress exerted by the liquid elasticity overcomes the viscous dissipation, and unsteady secondary flow develops, leading to a dramatic shear thickening. 



%However, the interaction between the suspended particle and the viscoelastic liquid beyond the onset of shear thickening of the viscoelastic suspension remains largely unknown. Because in the viscoelastic liquid, when the shear rate is above a critical value, the extensive normal stress exerted by the liquid elasticity overcomes the viscous dissipation, and unsteady secondary flow develops, leading to a dramatic shear thickening. Eventually, the flow becomes turbulent-like: different flow structures in the developed secondary flow have a power-law distribution in the frequency domain\cite{groisman_elastic_2004,groisman2000elastic}. The interaction between this elastic instability and the behavior of the suspended particles is difficult to investigate because the viscoelastic stress of the liquid under shear causes the sample to be expelled from the flow cell and depletes the particles \cite{feng1996motion,keentok1999edge}. Thus, virtually all the investigations of the rheological behavior of a particle suspension in viscoelastic fluid have focused on lower shear rates where the consequences of the elastic instability are avoided. However, an understanding of the behavior of a viscoelastic suspension at high shear rates is critical, as it impacts many industrial applications such as polymer processing \cite{shenoy2013rheology}, hydraulic fracturing \cite{barree1994experimental}, and microfluidic devices  \cite{zhou2020viscoelastic}.  It is thus essential to investigate the interplay of the elastic instability of viscoelastic fluid and suspended particles. 

%Particles suspended in a liquid can dramatically change the rheological properties of the suspension; for example, the viscosity of the suspension exhibits a sharp divergence as the volume fraction increases \cite{wyart2014discontinuous}. Even when the fluid is Newtonian, if the suspension is sufficiently concentrated, it exhibits shear thinning at increasing shear rates, whereas at even higher shear rates, it exhibits a dramatic increase in the measured viscosity, induced by frequent particle-particle collisions, and called the Continuous Shear Thickening (CST) effect \cite{lin2015hydrodynamic}.
% At even higher volume fraction, this shear thickening becomes a discontinuous transition and is called the Discontinuous Shear Thickening (DST) \cite{wyart2014discontinuous}.
%The physics is further complicated when the suspending fluid is itself viscoelastic. The suspension again undergoes shear thinning followed by shear thickening as the shear rate increases \cite{vazquez2019shear,yang_mechanism_2018}. However, compared to a Newtonian fluid, the volume fraction required for shear thickening with a viscoelastic fluid can be as much as an order of magnitude less than the threshold for CST; moreover, the shear thickening is significantly more moderate. This suggests that the shear thickening of the viscoelastic suspension has a different mechanism leading to the CST and is presumably associated with the viscoelasticity of the fluid. And even with the absence of the particles, the viscoelastic fluid exhibits shear thickening which is caused by elastic instability. \cite{pakdel1996elastic,shaqfeh1996purely}: when the shear rate is above a critical value, the extensive normal stress exerted by the liquid elasticity overcomes the viscous dissipation, and unsteady secondary flow develops, leading to a dramatic shear thickening. Eventually, the flow becomes turbulent-like: the kinetic energy embraced in the developed secondary flow has a power-law distribution in the frequency domain\cite{groisman_elastic_2004,groisman2000elastic}. The interaction between this elastic instability and the behavior of the suspended particles is difficult to investigate because the viscoelastic stress of the liquid under shear causes the sample to be expelled from the flow cell and depletes the particles \cite{feng1996motion,keentok1999edge}. Thus, virtually all the investigations of the rheological behavior of a particle suspension in viscoelastic fluid have focused on lower shear rates where the consequences of the elastic instability are avoided. However, an understanding of the behavior of a viscoelastic suspension at high shear rates is critical, as it impacts many industrial applications such as polymer processing \cite{shenoy2013rheology}, hydraulic fracturing \cite{barree1994experimental}, and microfluidic devices  \cite{zhou2020viscoelastic}.  It is thus essential to investigate the interplay of the elastic instability of viscoelastic fluid and suspended particles. 

%In this study, we show that solid particles suspended in a viscoelastic liquid have a profound effect on flow at high shear rates by phase separation into crystals. We evaluate the time dependence of the torque at high rates using a rheometer with a customized geometry that ensures particles are not expelled from the flow geometry. In the absence of particles, the flow undergoes elastic instability at high shear rates, and the torque fluctuates significantly. Surprisingly, the presence of the particles results in a prominent, well-defined single frequency oscillation in torque fluctuations and breaks the power law in frequency. We then mount the rheometer with optics to visualize the flow patterns that arise in the elastic instability. We find the suspension also supress the spiral patterns while introducing a new flow pattern.  
%Further quantification of the velocity field with particle imaging velocimetry and dynamic mode decomposition reveals that particles introduce a rotational motion while suppressing other flow structures. By using a rheometer with counter-rotating plates, we visualize the structure of the particles in the plane that is stationary in the laboratory frame. Remarkably, the particles form a two-dimensional crystalline layer even at volume fractions as low as ten percent. This layer regulates the secondary flow and leads to the modified flow dynamics. Moreover, the crystalline structure is essential: using polydisperse particles prevents crystallization, and the modification of the flow dynamics is partially reversed.  


%We firstly apply a constant shear with shear rate $\dot{\gamma} = 100~/\mathrm{sec}$ to the viscoelastic suspension. We can only shear the sample for tens of seconds before the liquid expels from the flow geometry. Under the microscope, we can see the formation of the ordered structure under shear, But the experiment ends early before the structure fully reaches equilibrium; see SI figure 6.

%Instead of applying a steady shear, we then use the large-amplitude oscillation shear (LAOS) protocol to shear the sample impulsively and prevent interfacial instability from fully developing\cite{hyun2011review}. LAOS is a well-established protocol for applying a strong shear while preventing instabilities. When the inertia effect is negligible, the system can reach equilibrium even if the shear stress is not constantly applied. The angular velocity of the oscillation $\Omega = 6.28~/\mathrm{sec}$, and the amplitude $\gamma=30$, the averaged applied shear rate is $\dot{\gamma}=2\Omega\gamma/\pi=120~/\mathrm{sec}$. We capture the frames with a high-speed camera (v7.3 from Phantom) after shearing for 60 seconds. Sixty seconds of shearing is long enough that the torque reading remains static until 600 seconds when the test ends, suggesting that the particle structure reaches equilibrium. 


%When the volume fraction of the particle suspension is down to $5~\mathrm{vol}\%$, the particles show attractive interactions and form local structures like short chains and small rafts under shear. Such structures further align with the flow direction and migrate accompanying the flow. Importantly, enough distance between assembles allows relative motions between neighbor structures without collision. Therefore, though the particles assemble under shear, the resulted structures can still be considered as dispersed particles since the size of the particle structure is still an order of magnitude smaller than the flow cell size $H$; see Fig.~\ref{fig:4}{\it A}. \cite{scirocco2004effect}

%Surprisingly, when the volume fraction of the particles is increased to $\Phi = 20\%$, the attractive interaction among particles leads to qualitative difference. Under shear, we see particles crystallize into two-dimensional (2D) closely packed rafts, which migrate like a rigid body during each LAOS cycle. The rafts further overlap with each other and are locked by friction, forming even larger structures; see the lower middle of Fig.~\ref{fig:4}{\it A}. A further increase of the volume fraction from $\Phi = 20\%$ to $30\%$ will generate larger rafts, but the phenomena are qualitatively similar; see the lower right of Fig.~\ref{fig:4}{\it A}. This observed crystallization is unexpected: In a mono-dispersed, hard spherical particle suspension, the minimum volume fraction needed for crystallization, or the melting concentration, is 54.5 \%\cite{hoover1968melting}, which is much higher than 20\% that we have as the volume fraction of the crystallized particles here.


%The crystallization can be understood from two aspects of observations: first, near the focal plane, the measured local particle surface fraction is much higher than the bulk volume fraction, 20 \%, suggesting the particles migrate toward the middle plane under viscoelastic shearing.  Second, the particles in the middle plane aggregate into 2D closely packed rafts, leaving some void spaces between the structures. Such aggregation below the melting concentration suggests there is an attractive interaction between the particles.

%Both the shear-induced migration and the attractive interaction are from the fluid viscoelasticity. There is no structure formation or significant migration in our control experiments with a Newtonian liquid under the same working condition. The viscoelasticity emits normal stress perpendicular to the streamline, pushing the particles away from the boundary since the flow field surrounding each particle becomes asymmetric when close to the wall. Moreover, the normal stress generates attractive interaction when particles are close to each other and their local flow streamlines overlap.\cite{feng1996motion}

%The suspension flow has been a long-lasting research topic. People tend to view the particle suspension as a single continuous phase with an additional parameter $\Phi$. This perspective has achieved great success in both science and engineering applications. Yet, in this study, we discussed the limitation of the single-phase paradigm. Under the viscoelastic shear flow, the attractive interaction and the shear-induced migration lead to the phase separation and crystallization of the particles. The formed structure further interferes with the flow field, disrupts the original secondary flow, and introduces a rigid body rotation motion. Such shear-induced phase separation can be considered a novel method to separate particles from the medium not by filtering or centrifuge but by shearing. 

%In this letter, we have tested the torsional shear flow. This shear-induced attractive interaction is not bounded to the flow cell's geometry or size. Therefore, we expect to observe other types of structure formation in other flow systems such as in the pipe flow, T-junction flow, and the Taylor–Couette flow.

%Finally, our understanding of the attractive interaction between particles is still lacking. Our data suggest the particles form an ordered structure in a few seconds, and it can be interesting to understand the detailed mechanism of the microstructure formation.

%Finally, we note that the further understanding of the attractive interaction among particles is a interesting topic for future studies.
%Our data suggests that the torque reaches equilibrium in a few seconds, and it would be constructive to further investigate the detailed mechanism of the microstructure formation of the particles in the initiating process.



%The particles are mono-disperse PMMA with a diameter of 51 microns (CA50 from MICROBEADS).
%For the solution phase, we use a mixture of 84.596 wt\% 2-2 Thiodiethanol (CAS 111-48-8 from Sigma) and 15.404 wt\% DI water to yield a refractive index of 1.488, which precisely matches that of the particles. The density of the solvent is 1.18 g/ml, which is close to that of the particles, 1.19 g/ml. We then dissolve 0.25 wt\% polyethylene oxide (PEO) of molecular weight 8 MDa (CAS 25322-68-3 from Sigma) in the solvent to make it viscoelastic with a relaxation time of 0.8 sec, measured at a shear rate of 10~/sec. % The 
