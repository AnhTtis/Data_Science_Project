

\documentclass[%
%reprint,
%superscriptaddress,
%groupedaddress,
%unsortedaddress,
%runinaddress,
%frontmatterverbose, 
preprint,
%preprintnumbers,
%nofootinbib,
%nobibnotes,
%bibnotes,
 amsmath,
 amssymb,
% aps,
%pra,
%prb,
%rmp,
%prstab,
%prstper,
%floatfix,
]{revtex4-2}
\usepackage{graphicx}% Include figure files
\usepackage{dcolumn}% Align table columns on decimal point
\usepackage{bm}% bold math
\usepackage[colorlinks=true, allcolors=blue]{hyperref}
\usepackage[T1]{fontenc}
\usepackage[dvipsnames]{xcolor}

\begin{document}
\title{Phase separation in suspended elastic turbulence}% Force line breaks with \\
\section{Introduction}

%

Particles suspended in a liquid can dramatically change the rheological properties; for example, the viscosity of the suspension exhibits a sharp divergence as the volume fraction increases. Even when the fluid is Newtonian, if the suspension is sufficiently concentrated, it exhibits shear thinning at increasing shear rates, whereas at even higher shear rates, it exhibits a dramatic increase in apparent viscosity, induced by frequent particle-particle collisions, and called the continuous shear thickening (CST) effect. At even higher volume fraciton, this shear thickeninng becomes a discontinuous transition and is called the DST. 
%This shear thicken transition is induced by frequent particle-particle collision and is called the continuous and (?) discontinuous shear thickening effect (CST and DST).
The physics is further complicated when the suspending fluid is itself viscoelastic. %(save for later)
The suspension again undergoes shear thinning followed by shear thickening as the shear rate increases. However, by comparison to a Newtonian fluid, the volume fraction required for shear thickening with a viscoelastic fluid can be as much as an order of magnitude less than the threshold for CST; moreover, the shear thickening is significantly more moderate. This suggests that the shear thickening of the viscoelastic suspension has a different mechanism leading to the CST and is presumably associated with the viscoelasticity of the fluid. And even with the absence of the particles, the viscoelastic fluid exhibits shear thickening which is caused by an elastic instability; when the shear rate is above a critical value, the extensive normal stress exerted by the liquid elasticity overcomes the viscous dissipation, and an unsteady secondary flow develops, leading to a dramatic shear thickening. Eventually, the flow becomes turbulent. %Unlike the CST, the elastic instability is not from the change of the microstructure but the change of the flow profile. . Previous studies mainly focus on the shear thickening or deliberately lowering the shear rate to prevent the instability.
The interaction between this elastic instability and the behavior of the suspended particles is very difficult to investigate because the viscoelastic stress of the liquid under shear causes the liquid to be expelled from the flow cell and depletes the particles. Thus, virtually all the investigations of the rheological behavior of a particle suspension in viscoelastic fluid have focused on lower shear rates where the consequences of the elastic instability are avoided. However, a understanding of the behavior of a viscoelastic suspension at high shear rates is critical, as it impacts polymer processing, hydraulic fracturing, microfluidic devices, and other industrial applications.  It is thus essential to investigate the interplay of the elastic instability of a viscoelastic fluid and suspended particles. 

%. As the elastic instability is inevitable when the shear rate is high, the high shear rate flow of the viscoelastic suspension is unresolved until we understand the interaction between the suspension and the viscoelastic secondary flow. And studying the particle-flow interaction beyond the laminar flow regime shed light on the uncharted secondary suspension flow regime to be explored.

%The instability is traceable as the torque reading fluctuates once the unsteady secondary flow develops (say an interesting question but not yet be noticed is blabla). What we want to learn, what happenes if we learn it. (The general argument is here but I need to put in t) %Thus, we can study the flow dynamics by associating the torque fluctuation with the flow field change.  

%.We want to understasnd the role of suspension i nthe viscoelastic liquid. The particles in newtonina liquid is known to decrease the visocsity, but at high shear rate, the effective viscosity increases, which is due to the collusion effect. When the particles suspended in the viscoelastic liquid, a similar phenomenon can be observed when suspension is dissolved in the newtonian liquid, the visocisty first decrease thenincreases as the volume fraction goes up but at a much lower volume fraction. A important factor about the viscouselastic liquid is that even with the absens of the particles, there is a seperate mecahnism to trigger the shear thickening, which is the elastic instability. Aside from the viscosity increase, shear rate increase also leads to the  way of 
%Plan 2: High volume fraction particles suspended in the Newtonian liquid has unique rheological behavior when the applied shear stress is above a critical value: the frequent particle particle collision leads to an increased energy dissipation rate and even the liquid-to-solid phase transition, which is named the continuous/discontinuous shear thickening effect (CST and DST). An additional factor is introduced if the suspended medium is viscoelastic instead of Newtonian. When the shear rate is above a critical value in the viscoelastic flow, the extensive normal stress arising from the liquid elasticity overcomes the viscous dissipation, and the energy dissipation rate surges accompanying the development of an unsteady secondary flow. Eventually, a fully developed turbulent flow persists. Such transition is concluded as the elastic instability. For the particles suspended in the viscoelastic medium, the flow-induced transition appears to share similarity with both shear-thickening effects and the elastic instability, the apparent viscosity gradually increases with the shear rate, similar to the CST; yet the minimum volume fraction and the shear stress required for the shear thickening are both at least an order of magnitude lower than the requirement of CST. There are a number of simulation works discussing the viscoelastic liquid and the particle suspension interaction during the shear thickening. But a conclusive experimental work lacks here to investigate the detailed mechanism of the flow transition and the particle dynamics beyond the onset of the critical point. 

%Viscoelastic liquid flows distinctly from the Newtonian liquid. When the shear rate is high, the liquid elasticity cannot be relaxed, the flow tends to be unstable, and the elastic instability happens. Elastic instability does not depend on length scale and can be found in infinitesimal inertia scenarios, including microfluidic , rheometric, and the porous media flow.  (start with physical picture)

%There are three dimensionless numbers associated with the elastic instability: The Weissenberg Number,  $\mathrm{Wi}$, the Deborah Number, $\mathrm{De}$ and the Pakdel-McKinley Number, PMcK. The onset of the elastic instability depends on the PMcK. Wi and De are periodically used interchangeably. Wi describes the elasticity from the local shear rate, $\dot{\gamma}$; De describes the ratio between the liquid relaxation time and the typical time of the flow .
%Turbulent flow has been a fascinating research topic drawing constant attention. Aside from the conventional inertia-induced turbulent flow, the chaotic flow can also arise from the liquid elasticity: By shearing the viscoelastic liquid at a critical high shear rate, the large elastic stress overcomes viscous dissipation and disrupt the laminar flow. Further increasing the shear rate, a turbulent flow develops and finally persists. The resulting elastic turbulence does not depend on the characteristic length scale of the flow and can be found in infinitesimal inertia flows, including microfluidic, rheometric, and porous media flows.  A topic yet largely unknown is the influence of the elastic instability upon the particle suspension. Particles suspension in viscoelastic laminar flow exhibit intriguing phenomena, the particles migrate across the streamlines, attract each other, or abnormally sediment faster. It could be a fundamental oversimplification to assume the suspension simply follow the streamline in the elastic turbulent flow.

%Plan 1: Colloid particles suspended in the viscoelastic fluid exhibit extraordinary behaviors. Particles migrate across the streamlines, form aggregations, and sense reduced drag forces. These unanticipated behaviors arise from the viscoelasticity of the liquid, as the fluid elasticity emits stress normal to the streamline. An important consequence of the normal stress effect is the elastic instability; the large elastic stress overcomes viscous dissipation under high shear rate and disrupts the laminar flow even with the absence of inertia. The resultant secondary flow is independent of the length scale and can be found in microfluidic, rheometric, and porous media flows. A classic model flow to investigate the elastic instability is the torsional shear flow, where the sample is sheared between two coaxial plates, leading to a turbulent secondary flow constituted of a palette of different spiral patterns. Against the prevalence and the complexity of the elastic instability, current researches mainly focus on the laminar flow and the interaction between the elastic secondary flow and the particle suspension is yet largely unknown. It could be a fundamental oversimplification to assume the suspension simply follow the streamline in the elastic turbulent flow.

%\textcolor{red}{Plan 1}: Particles dispersed in a liquid form a suspension. Suspension flow represents in the river flow, hydraulic fracture and plastic composite manufacturing. When the liquid phase of a suspension is Newtonian, the effect of the particles on flow can be captured by a single parameter, the Volume Fraction, $\Phi$, and the suspension can be described as a single phase.  This single phase model is simple yet powerful. It can characterize different physical behaviors, from the volume exclusion effect to the continuous/discontinuous shear thickening effect. When long-chain polymers are dissolved in liquid however, the liquid becomes viscoelastic and the suspension flow becomes much more complex. Particles begin to migrate across the streamlines, attract each other, or abnormally sediment faster. All of these behaviors are associated with the elasticity of the liquid in laminar flow. A unique feature in the viscoelastic fluids is the elastic instability-----when the shear rate is high, elastic stress overcomes viscous dissipation in the fluid and disrupt the laminar flow. The resulting elastic instability does not depend on the characteristic length scale of the flow and can be found in infinitesimal inertia scenarios, including microfluidic, rheometric, and porous media flows.  A typical model flow to investigate the elastic instability is the torsional shear flow between two parallel plates. In this flow, increasing  the shear rate beyond the onset of the elastic instability will induce a secondary flow superposed on the laminar flow. Further increasing the shear rate produces a turbulent flow characterized by evolving spiral patterns.  The turbulent flow evolves as momentum is exchanged among modes and is eventually dissipated. Despite the importance of the elastic instability in viscoelastic fluids , how the elastic instability manifests in the behavior of viscoelastic suspensions is largely unknown. 

%\textcolor{red}{Plan 2}: Liquid flows under shear stress. Solid, on the other hand, only deforms. Solid particles dispersed in liquid form the suspension. The suspension is frequently characterized as a single phase with an additional parameter Volume Fraction $\Phi$. Under a wide range of volume fractions, suspension flows like a liquid, as the suspended particles can freely exchange their neighbors.  There have been numerous studies investigating the phase separation behavior in the suspension system. The majority of them focus on tuning the particle-particle interaction. When the attractive interaction overcomes the thermal fluctuation, solid particles tend to phase separate from the liquid. Such attractive interaction either come from the bulk, like magnetic or electrostatic interaction. Or from the surface, like depletion force or surface chemistry. However, these interactions are system-dependent. When particles are small, the volume interaction diminish, when particles are huge, the surface interaction diminish. In this study, we are demonstrating a different way to phase separate particles from the liquid. It does not rely on particle-particle interaction, but instead, the attraction force is from the flow.

%Dissolve long-chain polymer in liquid forms viscoelastic liquid. Under shear, the long-chain polymer is stretched and attributes elasticity to the liquid. Elasticity generates normal stress perpendicular to the flow direction. When two particles approach each other under the viscoelastic flow. The normal stress aggregates the particles. Such particle attraction is observed under sedimentation and shear flow. Those aggregation, when happens in the laminar flow, tend to be local. Particles forms log like structure. However, the elastic turbulent flow is yet unexplored.
%First sentence is how we answer the question. 
In this paper, we show that solid particles suspended in a viscoelastic liquid has a dramatic effect on the flow at a high shear rate. we use a rheometer with a geometry that ensures the particles are not expelled from the flow cell, and measure the time dependence of the torque and the flow patterns that develop at high rates. In the absence of particles, a pronounced elastic instability develops in the flow at high shear rates and the torque exhibits strong fluctuations. The presence of the particles significantly modifies the flow instability and leads to a pronounced well defined frequency in the torque fluctuations. By using a rheometer with counter rotating plates, we visualize the structure of the particles in the plane that stationary in the laboratory frame. Remarkably, the particles form a two-dimensional crystalline layer even at volume fractions as low as ten percent. It is this layer that disrupts the instability and leads to the modified flow dynamics. Moreover, the crystalline structure is essential; using polydisperse particles prevents crystallization and the modification of the flow dynamics is much weaker.  

%phase separate into a layer that interferes with the flow field and generates unexpected flow motion; where the monodispersed suspension even crystallize. Here, we visualize the flow field with optics combined with torque measurement from the rheometer to study the flow dynamics with the viscoelastic suspension. We find that elastic instability changes the flow field, induce secondary flow aside from the laminar flow and causes the shear thickening in the viscoelastic suspension. Further increasing the shear rate and letting the secondary flow fully develop, we find that suspension suppresses the secondary flow and introduces a novel flow pattern swirling over time. To understand the swirl motion, we subsequently use tracer particles to quantify the flow field. We find the swirl originates from the phase separation of the particles under shear. The unexpected phase separation in the flow drives us to use a special microscope capable of visualizing the suspension \emph{in situ} to investigate particle microstructure evolution under shear. We find that the monodispersed particles crystallize under strong shear then phase separate from the suspended liquid. The volume fraction required for crystallization here is surprisingly low, down to about 10\%. In the end, we validate the causality between crystallization and the swirl motion by showing that the flow field change can be reversed if the crystallization is inhibited. 
%(first sentence is we solve this problem. Then say we use what, then say we find blablabla, introduce the two sentences) In this study, we combine different rheoflow visualization techniques to experimentally investigate the effect of elastic turbulence on particle dynamics in viscoelastic suspensions.  We find that suspensions with particle volume fractions as low as $\Phi>10~\mathrm{vol}\% $ exhibit dramatic changes in elastic secondary flow relative to viscoelastic fluids without particles. The evolving spirals characteristic of torsional shear flow of viscoelastic fluids are suppressed in suspensions, and an unexampled spinning mode at a specific frequency emerges. We find that such behavior originates from the particles' shear-induced crystallization. Under shear, the suspension is no longer a continuous single phase, but phase separates into a rotating solid core surrounded by the liquid. We show that a chaotic flow surprisingly nourishes ordered structure formation and phase separation. This effect can also be a potential new method to phase separate particles from the liquid of a suspension by shearing. (be percise and concrete, no need to mention the details, dont need to add the prospective here,mention the next step at the end)

\section{Secondary Flow}

%When viscoelastic liquid is sheared between two parallel plates,  the long-chain polymers are stretched and generate hoop stress, which tends to disrupt the laminar flow and induce the elastic instability.  Further increasing the shear rate, secondary flow emerges.  Eventually, the flow becomes chaotic.  This chaotic flow is also named the elastic turbulence. )
%(describe the experiment, how we setup the experiment, we measure the torque, what we see, and so on)When a polymeric viscoelastic liquid is sheared between two parallel plates, stretching of the long-chain polymers generates hoop stress that disrupts the laminar flow and induces the elastic instability. Upon increasing the shear rate, the elastic stress yields secondary flow within the system where a chaotic flow  eventually persists, which is also called the elastic turbulence.

%We outfit a commercial rheometer (MCR 501 from Anton Paar) with optics and  to investigate the influence of the suspension on the elastic secondary flow. The torque is recorded by the rheometer as a function of time, which reflects the dynamics of the system. The viscoelastic liquid is formed by 0.25 wt\% PEO dissolved in 85.45 wt\% Thiodiethanol and 14.3 wt\% of water and it has a relaxation time of 0.8s. 
We use a mixture of 85.596 wt\% 2-2 Thiodiethanol (CAS 111-48-8 from Sigma) and 15.404 wt\% DI water as a solvent to yield a refractive index of 1.488 at 20~\textdegree C, which precisely matches that of the particles. The density of the solvent is 1.18 g/ml, which is close to that of the particles, 1.19 g/ml. The particles are mono-disperse PMMA with a diameter of 50 microns. We dissolve 0.25 wt\% polyethylene oxide (PEO) of molecular weight 8 MDa in the solvent to make it viscoelastic with a relaxation time of 0.8 sec, measured at a shear rate of 10/s. % The 
%The viscoelastic liquid is formulated by dissolving 0.25 wt\% of PEO in a mixture of 85.45 wt\% Thiodiethanol and 14.30 wt\% water. The molecular weight of the PEO is 8M Da and the relaxation time of the viscoelastic liquid is 0.8s. We suspend 30 vol\% mono-dispersed refractive index and density matched $ 50  \mu m$ diameter PMMA particles (CA50 from microsphere) in 70 vol\% same liquid to form a particle suspension for most of the study. (as seen in fig 1.d). (discuss why I use this mixture at the end, and the how the relation time is measured). 
We measure the time dependence of the torque as the shear rate is varied using a commercial rheometer (MCR 501 from Anton Paar) with a parallel plate geometry. At high shear rates, the viscoelastic fluid entraps air bubbles and eventually expels liquid from the flow cell, making measurement impossible. To overcome this problem, we modify the geometry of the rheometer by adding a cup to the lower plate, enabling the rotating upper plate to be fully immersed in the fluid. (as seen Fig. 1 a) Sufficient fluid is added to ensure the velocity gradient is much larger between the plates than between the upper surface and the free surface.
%The sample is sheared between two coaxial plates modified from a commercial rheometer (MCR 501 from Anton Paar) to apply the shear and measure the torque as a function of time. When the shear rate is high, the viscoelastic flow expels liquid from the flow chamber and entrench air bubbles. To eliminate the liquid expulsion and interfacial effect, we submerge the rotating upper plate in a pool of liquid. Enough liquid is loaded to ensure the velocity gradient is much larger beneath the rotating plate and most of the torque reading is from the sample within the flow chamber. (as seen Fig. 1 a)

In the absence of the particles, the torque remains constant, independent of time, at low shear rates. As the shear rate increases, the solvent exhibits shear thinning and the torque remains constant for fixed shear rate. Upon further increase of the shear rate, the solvent undergoes a sudden, dramatic shear thickening, and now the torque exhibits pronounced fluctuations in time. For example, at a shear rate of 50\%, the torque fluctuations have a relative magnitude of 6 percent, measured by $\mathrm{\sigma_ M/\bar{M}}$, where $\mathrm{\sigma_ M} $ is the standard deviation of the torque and $\bar{M}$ is the mean value of the torque , as shown in Fig. 1b.  To characterize these fluctuations, we calculate the Fourier transform of the torque and, from the square of the amplitude, determine the Power Spectrum Density (PSD).  The PSD exhibits a power law decay ~ f^-k, where k ~ 4.5, for frequencies, f, above a rollover frequency of $f_0 \~ 10^{-1} $ Hz.  



%For the polymer sample without the suspension, when the applied shear rate is low, torque remains constant within the tolerance of the sensor. When the shear rate increases, however, torque begins to fluctuate over time, immediately  after the onset of the shear thickening. When we fix the shear rate at 50/s, we find the fluctuation has a relative magnitude of $6\%$, measured by $\mathrm{\sigma_ M/\bar{M}}$, where $\mathrm{\sigma_ M} $ is the standard deviation of the torque and $\bar{M}$ is the mean value of the torque (see Fig. 1b).
%The fluctuation has a characteristic time of $10^1$ sec, but does not appear to oscillate at a single frequency. 
 
 %We therefore calculate the torque power spectrum density (PSD) as a function of frequency to investigate its frequency dependence. We find the PSD spectrum has a power-law relationship versus frequency; $\mathrm{PSD} \propto f^{-k}$ , where $k=4.5$. In the power law relationship, the system is self similar and thus no predominant frequency can be found except for the rollover frequency, which is at the order of $10^{-1}$ Hz. 
 Upon addition of particles at a volume fraction of 30 percent, the suspension exhibits somewhat more shear thinning than the particle-free solvent as the shear rate increases, while the torque is somewhat larger, but still constant in time for a given shear rate. The suspension exhibits a similar, pronounced onset of shear thickening whereupon the torque exhibits fluctuations in time; however, this occurs at a lower shear rate and the degree of shear thickening is not as large as it is in the absence of particles. The magnitude of the torque is considerably larger at a shear rate of 50/s, and the time dependence again exhibits fluctuations, but with an apparent dominant frequency, as shown in fig 1.b. This is reflected in the PSD which still exhibits a power law decay ~f^-k, where k=4.5, but also exhibits a sharp, two-orders-of-magnitude peak at a frequency of $f_p=0.323 $ Hz, as shown in figure 1.c.  
% By contrast, for the particle suspension under the same working condition, although the torque fluctuation magnitude is comparable to that without particles, the average torque reading, $\mathrm{\bar{M}} $ , increases by a factor of three. Thus the relative amplitude decreases by around 80\%, measured by $\mathrm{\sigma_ M/\bar{M}} $ (as seen in fig 1.b). Moreover, the original torque fluctuation is now superposed with a new signal with a dominant frequency.  In the frequency domain, the original power law relationship is disrupted and an intrusive single frequency signal is introduced here with two orders of magnitudes higher PSD than the background (as seen in fig 1.c).  

The sharp peak in the PSD is very robust and is present for all the rotation rate, $\Omega$, above $\Omega=0.9\,\mathrm{rad/s}$, which is slightly greater than the onset of the shear thickening, and $f_p$ varies linearly with $\Omega$ , $$f_p\approx 0.025~\Omega,$$ as shown by the violet circles in fig 1.d. 

To elucidate the nature of this shape peak in the PSD, we consider its origin:  it may arise from two distinct flow behavior, either the flow around individual particles, or the flow pattern within the whole cell. 
%For the particles suspended in a Newtonian liquid without elasticity, there is no shear thickening under the same working condition. The continuous shear thickening of the suspension does not happen until the volume fraction is above 40\% and the onset shear rate of CST is an order of magnitude higher than our experimental observation.

Under a given rotation rate, the torque is proportional to the energy dissipation rate and thus reflects the flow dynamics integrated over the flow cell. A fluctuating torque indicates the flow field is time-dependent and unsteady. Since there is no lower bound on the particle volume fraction, the shear thickening unlikely comes from the CST, instead, it possibly arises from the elastic instability.

When the shear rate increases, the viscoelasticity accumulates enough hoop stress to overcome the viscous dissipation, leading to the development of an unsteady flow field, named the viscoelastic secondary flow, in contrast to the time-invariant laminar base flow. 

The elastic instability occurs under the length scale of a single particle $d$ and the length scale of the flow cell $R$. Both contribute to the torque reading. But we can differentiate which length scale dominates the torque fluctuation by investigating their different time dependence. For the flow field around a single particle, a characteristic time scale can be defined as $\tau=\frac{1}{\dot{\gamma}_{local}}$, which describes how fast the flow field around every single particle evolves. For the flow spanning the entire flow field, a second-time scale can be defined as $\tau_1=1/\Omega$, here $\Omega$ is the angular velocity of the upper rotating plate. It characterizes the speed of global flow spanning the flow cell. The driving stress is angle-dependent because the flow field is nonaxisymmetric, such as the elastic secondary flow. The time needed for the upper rotation plate to reset its position is $2\pi\tau_1$ and the flow cell boundary condition.


%Here $\dot{\gamma}_{local}=\Delta u/\Delta d $, defined by dividing the difference of the velocity $\Delta u$ from particle to particle by the average space between the particles $\Delta d$. It differs from the $\dot{\gamma}=\Delta u/(\Delta d+d)$, which is the nominal shear rate imposed by the rheometer. 
We tune $\tau$ to $\tau_1$ ratio by first changing the flow cell geometry. For the torsional shear flow: $\dot{\gamma}=2/3\frac{\Omega d}{H} $, the $\dot{\gamma}$ to $\Omega$ ratio is changed by varying $\frac{H}{R}$. We start with letting $H$ stepping from 0.5 mm to 2.5 mm in 0.5 mm increments. There is no difference in the scaling of the oscillation frequency (as seen in different color codes fig 1.f). It implies the secondary flow is on the disk scale. We then confirm this hypothesis by vary the flow cell radius with $R=30$ mm and $R=20 mm$ plates (both based on DHR-3 Rheometer from TA instrument). The oscillation frequency is dictated by $\Omega$ or $\tau_1$ with the same scaling as before. (as seen in the hexagrams in fig 1.f). Based on these two control experiments, we conclude the secondary flow occurs at a length scale comparable with the flow cell size. 

%(If the change is from the macroscopic scale, it implies the change is from a macroscopic scale.)

If the oscillation frequency is dictated by the macroscopic flow, we should have an identical scaling when $\tau$ increases while $\tau_1$ stay constant. It can be achieved by increasing the volume fraction $\Phi$, which decreases the average space $\Delta d$ between the particles, leading to an increase of $\dot{\gamma}_{local}$ under the same $\Omega$. We find the frequency scaling is not influenced by the particle volume fraction. The same trend line is observed once the volume fraction  $\Phi = 10\%$ or above. (as seen in the pentagrams, triangles and the circles in fig 2.a).  A higher concentration only leads to larger oscillation amplitude. The same scaling cross validated the hypothesis that the secondary flow is from the global flow.

For completeness, we test samples with different particle sizes and the relaxation time. Neither of them influence $\Omega$, thus we do not expect difference in the power spectrum. And indeed by shearing three $\Phi = 30\%$ samples with particles diameters of 52, 30 and 9.1 um respectively. we find no difference in their power spectra. (see the rhombus, squares, and the circles in fig 2.a and fig ? in SI). We also replace the viscoelastic liquid with a solution with shorter relaxation time of 0.2 second instead of 0.8 sec. The oscillation frequency is the same.

%Under all these different control experiments, we find that $f$ increases proportionally to $\Omega$, following the same trend $f=0.025\Omega$. The fact $f$ is dictated by $\Omega$ or $\tau_1$ suggests the torque oscillation originates from a flow pattern whose length scale is comparable with the flow cell radius $R$. 


%The frequency scaling is not influenced by the particle volume fraction or size either. The same trend line is observed once the volume fraction  $\Phi = 10\%$ or above. (as seen in the pentagrams, triangles and the circles in fig 2.a).  A higher concentration only leads to larger oscillation amplitude. Similarly, comparing three $\Phi = 30\%$ samples with particles diameters of 52, 30 and 9.1 um respectively, their power spectra are identical. (seethe rhombus, squares, and the circles in fig 2.a and fig ? in SI).



%So far, we tune $\tau$ to $\tau_1$ ratio by first changing the flow cell geometry. For the torsional shear flow: $\dot{\gamma}=2/3\frac{\Omega d}{H} $, the $\dot{\gamma}$ to $\Omega$ ratio is changed by varying $\frac{H}{R}$. We also increase the volume fraction $\Phi$ to decrease the average space $\Delta d$ between the particles, leading to an increase of $\dot{\gamma_{local}}$ under the same $\Omega$. Under all these different control experiments, we find that $f$ increases proportionally to $\Omega$, following the same trend $f=0.025\Omega$. The fact $f$ is dictated by $\Omega$ or $\tau_1$ suggests the torque oscillation originates from a flow pattern whose length scale is comparable with the flow cell $R$. 

Since the secondary flow is on the flow cell scale, we can study the secondary flow by outfit the rheometer with a customized transparent bottom plate above an aligned mirror and camera for visualization.  Titanium dioxide coated mica flakes (0.5 wt\% from htvront) is added to the polymer solution to form a rheoscopic liquid. Under shear, the mica flakes align with the flow direction allowing the flow field to be qualitatively assessed via the reflected pattern, which is also named the Kalliroscope. 

Thee rheoscopic visualization validates the hypothesis that torque fluctuation arise from the elasticity-induced secondary flow. For the viscoelastic liquid without particles, nontrivial flow pattern emerges once the torque fluctuates. By fixing the upper plate to rotate at a constant rate, letting the secondary flow fully develop, we observe multiple spirals. Over time, the pattern spirals outwards from the center to the edge and their shape evolves over time ( as seen in figure 1e, and Video 1 in SI). The evolving patterns suggests the momentum is exchanged among them and thus no dominant pattern can be found. It pattern exhibited the expected behavoir from before. The motivation is we can visualize the flow.

Upon adding the particle suspensions, the flow patterns undergoes dramatic change. Despite the torque fluctuates similar to no particle case, suggesting a similar secondary flow exists, the patterns seen before are indiscernible. We can only find curved fragments in the flow field. Instead, one flow pattern dominates over the others and persists across the entire experiment. The promoted pattern appears as a spiral, which propagates from the edge of the flow field to a position between the center and periphery (see Fig. 1f and SI Video II). The measured torque oscillation frequency 0.32 Hz is also the frequency of the spiral rotation (3.3 sec per cycle), suggesting that the single frequency torque oscillation is from a novel flow motion in the flow field.

By changing the rotation rate or the height-to-radius ratio of the flow cell, we find the flow patterns of the suspensions are all spiral and propagate from the edge to the middle half of the plate. Although the endpoint move slightly outwards when the gap height is halved ( as seen in fig 2 b and c). Similarity among flow patterns under different working conditions explains the linear scaling between oscillation frequency and the rotation rate.  Since the flow pattern is the same under different working conditions, increasing the rotation rate only increase the driving torque but spares the secondary flow motion, leads to a linear increase of the flow speed.


%where the addition of the particles dramatically changes the secondary flow field. The flow becomes unsteady once the 
%Comparing the flow patterns for a polymer solution without and with added particles, respectively, the presence of mono-dispersed particles suppresses multiple spirals in the elastic turbulence. Despite the torque measurement suggesting there is still secondary flow similar to the case without the suspension, the patterns seen before are indiscernible. We can only find curved fragments in the flow field. Instead, one flow pattern dominates over the others and persists across the entire experiment. The promoted pattern appears as a spiral, which propagates from the edge of the flow field to a position between the center and periphery (see Fig. 1f and SI Video II). The measured torque oscillation frequency 0.32 Hz is also the frequency of the spiral rotation, suggesting that the torque reading oscillation is associated with the change of the flow field.




%Comparing the flow patterns for a polymer solution without and with added particles, respectively, the presence of mono-dispersed particles suppresses multiple spirals in the elastic turbulence. Despite the torque measurement suggesting there is still secondary flow similar to the case without the suspension, the patterns seen before are indiscernible. We can only find curved fragments in the flow field. Instead, one flow pattern dominates over the others and persists across the entire experiment. The promoted pattern appears as a spiral, which propagates from the edge of the flow field to a position between the center and periphery (see Fig. 1f and SI Video II). The measured torque oscillation frequency 0.32 Hz is also the frequency of the spiral rotation, suggesting that the torque reading oscillation is associated with the change of the flow field.  Within secondary flow, we observe multiple spirals. Over time, the pattern spirals outwards from the center to the edge and their shape evolves over time. ( as seen in figure 1e, and Video 1 in SI). 


%We therefore calculate the torque power spectrum density (PSD) as a function of  frequency to investigate the frequency dependence of the torque. We find the PSD spectrum has a power-law relationship versus frequency; $\mathrm{PSD} \propto f^{-k}$ , where $k=4.5$. In the power law relationship, the flow is self similar and thus no predominant frequency can be found except for the rollover frequency, which is at the order of $10^{-1}$ Hz. 

%When the applied shear rate is low, torque remains constant within the tolerance of the sensor for polymer solution with or without the suspension, suggesting the flow is laminar and thus time invariant (see SI). (say the particles at the begining, people need to know the equipment from the first few sentences and the first paragraph, then i can discuss abou tthe measurement in the next paragraph) When the shear rate increases, however, the viscoelasticity accumulate enough hoop stress to overcome the viscous dissipation and disrupts the laminar flow, leading to the development of an unsteady flow field (how do I know it is unsteady, I only have the). This time-dependent  unsteady flow is named the secondary flow in contrast to the laminar base flow. The secondary flow  expels liquid from the flow chamber and entrench air bubbles. To eliminate the liquid expulsion and interfacial effect, we submerge the upper plate of the rheometer in a pool of liquid (as seen Fig. 1 a). We find the shear thickening in our sample, with or without the suspension, is associated with torque fluctuation and the flow field change, which suggests the secondary flow causes the shear thickening in our system (see SI).  

%For the viscoelastic liquid without particles, when we fix the shear rate at 50/s, letting the secondary flow fully develops, the measured torque fluctuates over time at a relative magnitude of $6\%$, measured by $\mathrm{\sigma_ M/\bar{M}}$, where $\mathrm{\sigma_ M} $ is the standard deviation of the torque and $\bar{M}$ is the mean value of the torque (see Fig. 1b). The fluctuation has a characteristic time of $10^1$ sec, but does not appear to oscillate in a single frequency, We therefore calculate the torque power spectrum density (PSD) as a function of  frequency to investigate the frequency dependence of the torque. We find the PSD spectrum has a power-law relationship versus frequency; $\mathrm{PSD} \propto f^{-k}$ , where $k=4.5$. In the power law relationship, the flow is self similar and thus no predominant frequency can be found except for the rollover frequency, which is at the order of $10^{-1}$ Hz. 
%The self-similarity and power law relationship is from the competition between the elastic stress and viscous stress in the secondary flow.
%Such power law cascade is because of the energy cascade generated from competing  during elastic turbulence.()

%[preemphasis]
%In the secondary flow, the torque fluctuation over time, as shown in the fig v. it may have a predomiant frequency, but we do not sure. so we convert it to the frequency, but there is no predominant frequency, but instead decay in a power law relationship with the frequency as we ssen in fig 1.c.
%To better understand the secondary flow, the rheometer is outfitted with a customized transparent bottom plate above an aligned mirror and camera for visualization (as seen in fig 1.a). We add 0.5 wt\% titanium dioxide coated mica flakes (from htvront) to the polymer solution to form a rheoscopic liquid. Under shear, the mica flakes align with the flow direction allowing the flow field to be qualitatively assessed via the reflected pattern formed by their different local orientations. Within elastic turbulence, we observe multiple spirals. Over time, the pattern spirals outwards from the center to the edge and their shape evolves over time. ( as seen in figure 1e, and Video 1 in SI). It suggests the torque fluctuations is associated with the unsteady flow field.(ref).
%(say that interestingly, the torque fluctuation is roughly ten percent of the mean, and we see ten to twenty rings, it suggests the fluctuation may result from the change of the number, to test this we make a thinner cell,which give us more numbers. not true)

%To investigate the role of the particle suspension, we then replace the liquid with a 30 vol\% mono-dispersed refractive index and density matched PMMA particles (CA50 from MICROSPHERE) suspended in the same polymer solution (as seen in fig 1.d). Under the same boundary condition where the secondary flow is fully developed, although the torque fluctuation magnitude with added particles is comparable to that without, the average torque reading, $\mathrm{\bar{M}} $ , increases by a factor of three. Thus the addition of the particle decreases the secondary flow strength by around 80\%, measured by $\mathrm{\sigma_ M/\bar{M}} $ (as seen in fig 1.b). Moreover, the original torque fluctuation is now superposed with a new signal which appear to oscillate in a dominant frequency. In the frequency domain, the original power law relationship is disrupted and an intrusive single frequency signal is introduced with two orders of magnitudes higher PSD than the background (as seen in fig 1.c).  
%Under the secondary flow, the liquid tends to escape from the flow chamber and entrench air bubbles. To characterize the secondary flow without the interfacial issue, we submerge the upper plate of the rheometer in a pool of liquid (see Fig. \ref{fig:Fig1} a) and record the torque as a function of time. In the secondary flow, the torque fluctuates over time (see Fig. \ref{fig:Fig1}b). On the other hand, in the laminar flow, the torque remains constant within the tolerance of the sensor (see SI).

% If we transform the torque signal to the frequency domain, the Power Spectrum Density (PSD) of the torque decreases linearly on a log-log scale with frequency ( $f$ ) increases under. There is no predominant peak on the spectrum.\ref{fig:Fig1}c. This is because in the elastic turbulence, the competition between elastic force and viscous dissipation leads to the so-called energy cascade. Within the energy cascade, the Kolmogorov-like PSD spectrum has a power-law relationship versus frequency: $\mathrm{PSD} \propto f^k$, where $k<0$. 
 

 %To better understand the secondary flow, we employ the rheoflow visualization setup to visualize the flow field. Here, the rheometer is outfitted with a customized transparent bottom plate and optics beneath it. 0.5 wt\% mica flakes are dispersed in a liquid to form a suspension. Under shear, the mica flakes align with the flow direction locally. Consequently, the flow field can be qualitatively assessed via the reflected pattern formed by different local orientations of the mica flake.
 
%in the elastic turbulence, we observe multiple spirals emerge and vanish over time; And multiple spiral modes coexist simultaneously, see Fig.\ref{fig:Fig1}e and Video 1 in the SI. Such observation is in line with the previous studies. It suggests the torque fluctuation originate from the unsteady flow field.

%Interestingly, when we add 30 vol\% mono-dispersed index and density matched particles to the liquid.  The torque fluctuation is now superposed with a new single frequency signal. The torque fluctuation magnitude with suspension is comparable to that without suspension. But the mean torque reading $\bar{M}$ is increased to 297\%. Thus the addition of 30 vol\% suspension decreases the secondary flow strength by 79.25\%, measured by $\mathrm{\sigma_ M/\bar{M}} $. Here $\mathrm{\sigma_M}$ is the standard deviation of the torque. 

%In the frequency domain, a single frequency peak emerges (Fig. \ref{fig:Fig1}c). The emerged signal peak has a high signal to background ratio. It has two orders of magnitudes higher PSD than the background. And the width of the peak $\Delta f/\bar{f} \approx 1.5\%$, where $\Delta f$ is the peak width at half height and $\bar{f}$ is the peak frequency. Note that the peak frequency is not the rotation frequency of the driven plate (highlighted in the black dash line in fig 1.c), where the artifact is frequently observed, nor an integer fraction of it.

%The change of the rheometric reading is also reflected in the change of the flow pattern. the addition of the suspension dramatically changes the kinetics of the secondary flow. Compared with the counterpart with only the polymer solution, the flow pattern with suspension is highly regulated: different modes seen in the elastic turbulence are imperceptible. Instead, one flow pattern dominates over the others and persists across the entire experiment. The promoted pattern appears as a spiral, which propagates from the edge of the flow field to the middle point between center and periphery; see Fig. \ref{fig:Fig1}c, and SI Video II).  The measured torque oscillation frequency 0.323 Hz is also the frequency of the spiral rotation. It suggests the torque reading oscillation originates from the change of the flow field. 
%The rheometric reading change is also reflected in the flow pattern change, where the addition of the particles dramatically changes the secondary flow field. Comparing figures 1e and 1f, the flow patterns for a polymer solution without and with added particles, respectively, the presence of mono-dispersed particles suppresses multiple spirals in the elastic turbulence. Despite the torque measurement suggesting there is still secondary flow similar to the case without the suspension, the patterns seen before are indiscernible. We can only find curved fragments in the flow field. Instead, one flow pattern dominates over the others and persists across the entire experiment. The promoted pattern appears as a spiral, which propagates from the edge of the flow field to a position between the center and periphery (see Fig. 1f and SI Video II). The measured torque oscillation frequency 0.32 Hz is also the frequency of the spiral rotation, suggesting that the torque reading oscillation is associated with the change of the flow field. 

\begin{figure*}
\centering
\includegraphics[width=1.0\textwidth]{Fig 1.pdf}
\caption{\label{fig:Fig1} elastic turbulence in a viscoelastic fluid with and without suspension.  a) Sketch of the setup. Optics for imaging is mounted beneath a commercial rheometer. b) Rheoscopic visualization of the flow field when elastic turbulence occurs, without adding suspensions. Multiple spirals coexist. The Gap size is 2 mm, the Geometry radius is 25 mm, Nominal Shear Rate $50\,/s"$c) Rheoscopic visualization of the flow field when elastic turbulence occurs, with $30~\mathrm{vol}\%$ PMMA particles. A single spiral dominates over the other modes d) The sketch of the system. The viscoelastic fluid with dense suspension is sheared between two parallel plates. e) The fluctuation of torque reading $\Delta M$ measured by rheometer. The blue and orange lines respectively show the torque fluctuation in the experiments when elastic turbulence occurs, with and without adding 30\% suspension particles. f) The power spectrum of the torque measurment. In the viscoelastic fluid with suspension, a single peak is observed at  0.323 Hz. The black dash line represents the rotating frequency of the upper plate. }
\end{figure*}


%\section{Scaling of the oscillation frequency}
%Upon varying the rotation rate, we find the  single-frequency oscillation of the torque is very robust and highly reproducible in the particle suspension flow, we see periodic oscillations in the torque  at the angular velocity as low as $\Omega=0.9\,\mathrm{rad/s}$, closely matching the onset of secondary flow. The frequency of the torque oscillation  $f $ increases linearly as $\Omega$ increases 
%$$f\approx0.025\Omega$$.(as seen in the violet circles in fig 2.a) Under rheoscopic visualization, we find the flow patterns are all spiral and propagate from the edge to the middle half of the plate, regardless of the rotation rate ( as seen in fig 2 b and c). 

%The flow visualization agrees with the torque measurement where ${f} \propto \Omega$. Because even if  ${f}$ is dictated by $\Omega$, a linear relation between ${f}$  and $\Omega$ further suggests the flow pattern is roughly independent of the applied rotation rate. The actuation torque, estimated as $M = \frac{\pi\eta \Omega d^4}{32H}\propto \Omega$, increases linearly as the rotation rate increases, and thus drive the pattern motion faster, leads to a linear increase of the torque oscillation frequency.

%up to $\Omega= 12 \,\mathrm{rad/s}$, which is the highest angular velocity  that can be assessed with the current setup,  beyond $\Omega= 12 \,\mathrm{rad/s}$, the rod climbing effect causes the liquid to overflow. 

%We then vary the gap height, letting $H$ stepping from 0.5 mm to 2.5 mm in 0.5 mm increments. Under different gap heights, the torque oscillation frequency is dictated by the rotation rate with the same scaling as before.(as seen in the different colored circles in fig 2.a) Similarly, in the flow visualization, we find although the ending point of the spiral move inward slightly when the gap size is halved from 2 mm to 1 mm, the flow patterns are still spiral and propagate from the edge to the middle half of the plate, (as seen in fig 2.b and 2.d), We also change the flow cell radius R by replacing the $R=30$ mm upper plate with a smaller radius $R=20 mm$ plate. $f$ still scales $\Omega$ with the same scaling, regardless of the flow cell geometry. (as seen in the hexagrams in fig 2.a), 

%We also observe the identical scaling when the volume fraction changes (point to the plot). (as seen in the pentagrams, triangles and the circles in fig 2.a) The lowest volume fraction that we can observe the torque single-frequency oscillation is $\Phi~=~10\%$ but unstable. Further increasing the volume fraction leads to a superlinear growth of  $\bar{M}$ and larger amplitude torque oscillation, but little change in the frequency.% in terms of amplitude. But the oscillation frequency $f$ changes little against the increase of $\Phi$, it still scales with $\Omega$ as $f\approx0.025\Omega$. (see the pentagram, circles, and triangles in fig 2.a )


%We also find the torque osscilation does not depend on the particle size. By shearing three $\Phi =30\%$ samples with particles diameters of $52, 30$ and $9.1 \mu m$, the power spectra show that both the amplitude and frequency of the torque fluctuations are identical. (see the rhombus, squares, and the circles in fig 2.a and fig ? in SI). % at the length scale $d$. %reinforcing that the promoted mode originates from global, rather than local flow. (very weak causality between the evidence and the conclusion, needs stronger and more clear argument. Also which argument it is going to make)
 
%The the torsional shear flow with particle suspension is constituted by two different length scales, the length scale of a single particle $d$, and the length scale of the flow field $R$. 

%The elastic instability, unlike inertia instability, can occur under both the length scale of a single particle $d$, and the length scale of the flow cell $R$. The flow at both length scales contributes to the torque reading. But we can differentiate which length scale dominates the torque fluctuation by investigating their different time dependence. For the flow field around a single particle, a characteristic time scale can be defined as $\tau=\frac{1}{\dot{\gamma}_{local}}$, which describes how fast the flow field around every single particle evolves. Here $\dot{\gamma}_{local}=\Delta u/\Delta d $, defined by dividing the difference of the velocity $\Delta u$ from particle to particle by the average space between the particles $\Delta d$. It differs from the $\dot{\gamma}=\Delta u/(\Delta d+d)$, which is the nominal shear rate imposed by the rheometer. 

%For the flow spanning the entire flow field, a second time scale can be defined as $\tau_1=1/\Omega$, here $\Omega$ is the angular velocity of the upper rotating plate. It characterizes how fast the entire flow field evolves. Because if the flow field is nonaxisymmetric, such as the elastic turbulent flow, the driving stress is angle-dependent. $2\pi\tau_1$ is the time needed for the upper rotation plate to reset its position and the flow cell boundary condition, so the global flow evolution scales with $\tau_1$. 

%So far, we tune $\tau$ to $\tau_1$ ratio by first changing the flow cell geometry. For the torsional shear flow: $\dot{\gamma}=2/3\frac{\Omega d}{H} $, the $\dot{\gamma}$ to $\Omega$ ratio is changed by varying $\frac{H}{R}$. We also increase the volume fraction $\Phi$ to decrease the average space $\Delta d$ between the particles, leading to an increase of $\dot{\gamma_{local}}$ under the same $\Omega$. Under all these different control experiments, we find that $f$ increases proportionally to $\Omega$, following the same trend $f=0.025\Omega$. The fact $f$ is dictated by $\Omega$ or $\tau_1$ suggests the torque oscillation originates from a flow pattern whose length scale is comparable with the flow cell $R$ and evolves at a time scale comparable with $\tau_1$, confirming the flow pattern we visualized in the rheoscopic fluid(fig 1.e). 

%In the secondary flow, the flow field is unsteady and non-uniform, but the torque nevertheless reflect the energy dissipation rate  with a given angular velocity. An increase in the torque indicates higher average energy dissipation rate under the same angular velocity and flow cell geometry. In a particle suspension, energy is only dissipated in the liquid phase. Since the liquid phase formula is the same with or without the particles, the increase of the energy dissipation rate solely come from the increase of the local shear rate $\dot{\gamma}_{local}$ between the particles, or $\bar{M} \propto \dot{\gamma}_{local}$.% which can be estimated as $\dot{\gamma}_{local}\approx\frac{M}{4\pi \eta d^3}>\dot{\gamma}$. 

%A particle suspension in the torsional shear flow is constituted by two different length scales, the length scale of a single particle $d$, and the length scale of the flow field $R$. Since the elastic secondary flow is length-independent, a secondary flow can develop in both length scales. The flow at these two length scales are also associated with different time scales.  For the flow field around a single particle, a characteristic time scale can be defined as $\tau=\frac{1}{\dot{\gamma}_{local}}$, which describes how fast the flow field evolves around each single particle.  Here $\dot{\gamma}_{local}=\Delta u/\Delta d $, defined by dividing the difference of the velocity $\Delta u$ by the average space between the particles $\Delta d$. It describes how fast the liquid between the solid particles flows. And $\dot{\gamma}_{local}$ differs from the $\dot{\gamma}=\Delta u/(\Delta d+d)$, which is the nominal shear rate imposed by the rheometer. 

%For the flow spanning the entire flow field, a second time scale can be defined as $\tau_1=1/\Omega$, here $\Omega$ is the angular velocity of the upper rotating plate. It characterizes how fast the entire flow field evolves. Because if the flow field is nonaxisymmetric, such as the elastic turbulent flow, the driving stress is angle-dependent. $2\pi\tau_1$ is the time needed for the upper rotation plate to reset its position and the flow cell boundary condition, so the global flow evolution scales with $\tau_1$. 

%In the torsional shear flow, the nominal shear rate  $\dot{\gamma}=2/3\frac{\Omega d}{H} $, For a given flow cell geometry and particle suspension, $\dot{\gamma}_{local} \propto \dot{\gamma} \propto \Omega$, thus we do not know whether $f $ is dictated by $\tau$ or $\tau_1$. But we can design experiments to increase $\dot{\gamma}_{local}$ without changing $\Omega$ or  $\dot{\gamma}$  by increase the particle suspension volume fraction. 

%Under the nominal shear rate $\dot{\gamma} = 40/s$, the average torque $\bar{M}$ is 1.3 times compared with no particle case when $\Phi = 10\%$ . Further increasing the volume fraction leads to a superlinear growth of  $\bar{M}$, till 8.3 times when $\Phi = 40\%$. In the secondary flow, the flow field is unsteady and non-uniform, but the torque nevertheless reflect the energy dissipation rate  with a given angular velocity. An increase in the torque indicates higher average energy dissipation rate under the same angular velocity and flow cell geometry. In a particle suspension, energy is only dissipated in the liquid phase. Since the liquid phase formula is the same with or without the particles, the increase of the energy dissipation rate solely come from the increase of the local shear rate $\dot{\gamma}_{local}$ between the particles, or $\bar{M} \propto \dot{\gamma}_{local}$.% which can be estimated as $\dot{\gamma}_{local}\approx\frac{M}{4\pi \eta d^3}>\dot{\gamma}$. 
%If we add the particle to the viscoelatic liquid, the measured viscosity $\eta$ of the particle suspension will be higher than the no particle case in the laminar flow. Under different volume fractions range from $10\%$ to $50\%$, $\eta$ follows the trend of: $$\eta=\eta_0\frac{\Phi}{(1-\Phi/\Phi_0)^2}$$ 
%until $\eta=43 \eta_0$ when $\Phi = 50\%$. Here $\eta_0$ is the viscosity of the fluid, $\Phi_0\approx56\%$ is the maximum packing density in the system. The increase of the visocisty originate from the fact the average distance between the particles decreases when the volume fraction is higher.  This is also named the volume exclusion effect. Start from here, we further ramp up the shear rate to explore the secondary flow regime, to be more specific, we would like to understand the influence of the volume fraction on the elastic secondary flow. 

%When the flow is no longer laminar, a similar increase of the measured torque $M$ can still be observed.  The average torque is 1.3 times compared with no particle case when $\Phi = 10\%$ under shear rate $\dot{\gamma} = 40/s$. Further increasing the volume fraction leads to a superlinear growth of the torque, till 8.3 times when $\Phi = 40\%$. In the secondary flow, the flow field is unsteady and non-uniform, so the viscosity cannot be directly inferred from the measured torque. But the torque nevertheless reflect the energy dissipation rate  with a given angular velocity. An increase in the average torque indicates higher energy dissipation rate under the same angular velocity and flow cell geometry. In a particle suspension, energy is only dissipated in the liquid phase. Since the liquid phase formula is the same with or without the particles, the increase of the energy dissipation rate solely come from the increase of the local shear rate $\dot{\gamma}_{local}$ between the particles. $\dot{\gamma}_{local}$ in the liquid is different from the nominal shear rate $\dot{\gamma} = \Delta v/h$. It is the actual shear rate in the liquid phase, which can be estimated as $\dot{\gamma}_{local}\approx\frac{M}{4\pi \eta d^3}>\dot{\gamma}$. 

%The lowest volume fraction that we can observe the torque single-frequency oscillation is $\Phi~=~10\%$,  further increasing the particle concentration leads to a more significant oscillation in terms of the amplitude. But surprisingly,  the oscillation frequency $f$ changes little against the increase of $\Phi$ . An increase of the volume fraction $\Phi$ ramps up $\dot{\gamma}_{local}$ and thus decreases $\tau$, which lets the local flow field around each particle evolves faster.  But the measured frequency $f$ is not influenced by the change of $\tau$, suggesting the torque oscillation does not originate from a periodic flow around each particle.

%The suspension flow contains two different length scales: the length scale of each single particle $r$ and the length scale of the flow cell $d$. The torque oscillation can either originate from the summation of the flow around each single particle, or from a flow pattern whose length scale is comparable with the entire flow cell. The flow under both scales can oscillate. Because the elastic instability ,unlike the inertia instability, is inertia free, and thus length independent, so the secondary flow can develop both around each single particle $r$ and in the global flow $d$.  For the flow field around a single particle, a characteristic time scale can be defined as $\tau=\frac{1}{\dot{\gamma}_{local}}$. In the secondary flow, the flow field is unsteady and $\tau$ describes how fast the flow field evolves around each single particle. An increase of the volume fraction $\Phi$ ramps up $\dot{\gamma}_{local}$ and thus decreases $\tau$, which lets the local flow field around each particle evolves faster.  But the measured frequency $f$ is not influenced by the change of $\tau$, suggesting the torque oscillation does not originate from a periodic flow around each single particle.

%To further verify the dependence  of $f$ on $\tau$ and $\tau_1$, An alternative way of change the $\tau$ to $\tau_1$ ratio by changing the flow cell geometry is employed. In the torsional shear flow: $\dot{\gamma}=2/3\frac{\Omega d}{H} $. 
%For a given suspension and flow cell, $\dot{\gamma}_{local}\propto\dot{\gamma} \propto \Omega$, and thus $\tau \propto \tau_1$. 
%We can tune the shear rate $\dot{\gamma}$ and thus $\dot{\gamma}_{local}$  to angular velocity $\Omega$ ratio by changing $\frac{d}{H}$. %In practice, we repeat shearing experiments with $\Phi = 30\%, r= 25 \mu m$ particle suspension, d = 40 mm and d = 60 mm plates and $H$ stepping from 1 mm to 2.5 mm in 0.5 mm increments on a customized DHR-3 rheometer from TA-instrument.


%To confirm the fact the secondary flow is size-independent, we then prepare three 30\% volume fraction suspensions using particles with diameters of $52, 30$ and $9.1 \mu m$ and shear the suspensions under identical conditions. We find no difference in the oscillations of the torque. The power spectra show that the amplitude and frequency of the torque fluctuations are the same in all three cases; see Fig. ?? in the SI. It agree with the fact the viscoelastic flow around each particle is self-similar, size-independent and scales with $\dot{\gamma}_{local}$. As long as the flow cell size is order of magnitude larger than the size of each particle, the particle size do not influence the collective flow dynamics.% at the length scale $d$. %reinforcing that the promoted mode originates from global, rather than local flow. (very weak causality between the evidence and the conclusion, needs stronger and more clear argument. Also which argument it is going to make)


%The suspension flow is hierarchical and contains two different length scales: the length scale of each particle $r$, and the length scale of the entire flow cell $d$. Since the elastic instability is length independent, the secondary flow can develop both around each single particle and in the global flow. Here we would like to study which length scale cause the single frequency flow mode. For such purpose, we try to change the flow condition in both single particle scale and the macroscopic scale. 
%we know how does the behavior of the hard spheres go, just one possible introduction. I really want to understand how does the behavior depends on the volume fraction. Its always better to present a bunch of data, purpose some hypothesis and validate these hypothesises.
%We design control experiments with different particle volume fractions and particle sizes to perturb the local flow, then different flow cell geometries and liquid viscoelasticity to change the global flow. 
%Under different working conditions, we measure the torque oscillation, which reflects the dynamics of the flow. 

%To change the average distance between the particles, We first prepare suspensions with different particle volume fractions. The average torque is 1.3 times compared with no particle case when $\Phi = 10\%$ under shear rate $\dot{\gamma} = 40/s$. Further increasing the volume fraction leads to a superlinear growth of the torque, till 8.3 times when $\Phi = 40\%$. An increase in the average torque indicates higher energy dissipation rate under the same angular velocity and flow cell geometry. In a particle suspension, energy is only dissipated in the liquid phase. Since the liquid phase formula is the same with or without the particles, the increase of the energy dissipation rate solely come from the increase of the local shear rate. This local shear rate $\dot{\gamma}_{local}$ in the liquid is different from the nominal shear rate $\dot{\gamma} = \Delta v/h$. It is the actual shear rate in the liquid phase, which can be estimated as $\dot{\gamma}_{local}\approx\frac{M}{4\pi \eta d^3}>\dot{\gamma}$. This is also named the volume exclusion effect. 

%The lowest volume fraction that we can observe the torque single frequency oscillation  is $\Phi~=~10\%$.  Further increasing the particle concentration leads to a more significant single frequency oscillation in terms of the amplitude, but little change in the oscillation frequency $f$, against the fact $\dot{\gamma}_{local}$ increases. 

%As mentioned earlier, the suspension flow contains the length scale of each particle and the entire flow field. Flow at two length scales are also associated with different time scales. For the flow field around a single particle, a characteristic time scale can be defined as $\tau=\frac{1}{\dot{\gamma}_{local}}$. In the secondary flow, the flow field is unsteady and $\tau$ describes how fast the flow field evolves around each single particle. As discussed in the previous paragraph, an increase of the volume fraction $\Phi$ ramps up $\dot{\gamma}_{local}$ and thus decreases $\tau$, which lets the local flow field around each particle evolves faster.  But the measured frequency $f$ is not influenced by the change of $\tau$, suggesting the torque oscillation does not originate from a periodic flow around each single particle.

%We then prepare three 30\% volume fraction suspensions using particles with diameters of $52, 30$ and $9.1 \mu m$ and shear the suspensions under identical conditions. We find no difference in the oscillations of the torque. The power spectra show that the amplitude and frequency of the torque fluctuations are the same in all three cases; see Fig. ?? in the SI. It agree with the fact the viscoelastic flow around each particle is self-similar, size-independent and scales with $\dot{\gamma}_{local}$. As long as the flow cell size is order of magnitude larger than the size of each particle, the particle size do not influence the collective flow dynamics% at the length scale $d$. %reinforcing that the promoted mode originates from global, rather than local flow. (very weak causality between the evidence and the conclusion, needs stronger and more clear argument. Also which argument it is going to make)



%Under all different working conditions, we find that  $f$ increases proportionally to $\Omega$, following the same trend $f=0.025\Omega$, independent of $H$ or $d$, see fig 2.a . By contrast, even with the same $\dot{\gamma}$, $f$  further depends on $H/d$ . The fact $f$ is dictated by $\Omega$ or $\tau_1$ suggests the torque oscillation originates from a flow pattern length scale comparable with the flow cell $d$ and evolves at a time scale comparable of $\tau_1$, confirmed the flow pattern we visualized in the rheoscopic fluid(fig 1.e). 

%Neither $\tau$ nor $\tau_1$ depends on the size of the particle, therefore changing the particle size should not influence the secondary flow dynamics. And indeed we prepare three 30\% volume fraction suspensions using particles with diameters of $52, 30$, and $9.1 \mu m$ and shear the suspensions under identical conditions. We find no difference in the oscillations of the torque. The power spectra show that the amplitude and frequency of the torque fluctuations are the same in all three cases; see Fig. ?? in the SI. Our result agrees with the fact the viscoelastic flow around each particle is self-similar, size-independent and scales with $\dot{\gamma}_{local}$. As long as the flow cell size is order of magnitude larger than the size of each particle, the particle size do not influence the collective flow dynamics.% at the length scale $d$. %reinforcing that the promoted mode originates from global, rather than local flow. (very weak causality between the evidence and the conclusion, needs stronger and more clear argument. Also which argument it is going to make)

%We find that the frequency of the torque oscillation $f$ increases proportionally to $\Omega$, independent of $H$ or $d$. By contrast, even with the same $\dot{\gamma}$, the frequency further depends on $H/d$  . The fact $f$ is dictated by $\Omega$, or $\tau_1$, suggest the torque oscillation originates from a flow pattern length scale comparable with the flow cell $d$, reassured the flow pattern we visualized in the rheoscopic fluid(fig 1.e).  For the flow spanning the entire flow field, a second time scale can be defined as $\tau_1=1/\Omega$, here $\Omega$ is the angular velocity of the upper rotating plate. It characterize how fast the entire flow field evolves over time, because if the flow field is not axial symmetric, as we see in the elastic turbulence, the driving stress is angle dependent. $2\pi\tau_1$ is the time needed for the upper rotation plate to reset its position and the flow cell boundary condition, so the global flow evolution scales with $\tau_1$. 

%If the particles are dispersed, the average space between the particles decreases with an increase in the particle volume fraction, as the liquid is displaced by the particles. Because the shear rate is effectively zero in the solid particles, the local shear rate $\dot{\gamma}_{local}$ in liquid increases to match the nominal shear rate $\dot{\gamma}= u/H$  imposed by the rheometer, here $u$ is the velocity of the upper rotating plate. This is also called the volume exclusion effect. In our separate test, we found $\dot{\gamma}_{local} = 4.3 \dot{\gamma}$ with $\Phi=30\%$ particles (the draft needs to be self-inclusive. I need to show it if I want to make this statement). If the torque oscillation originate from the local flow motion between the particles, we may expect $f \propto \dot{\gamma}_{local}$, or an increase in the oscillation frequency with an increased volume fraction. But there is no difference in oscillation frequency when varying the particle volume fraction. Because the frequency of the torque oscillation is therefore independent of the local shear rate, we conclude that the torque oscillation is not from the local flow among particles.
%(assume too much from the audience, explain the effect it clearly. either put the data on, write it very long and shorten it later.)
%If the particles are dispersed, the average space between the particles decreases with an increase in the particle volume fraction, as the liquid is displaced by the particles. Because the shear rate is effectively zero in the solid particles, but the total velocity difference between the upper and the lower plate remain the same. The local shear rate $\dot{\gamma}_{local}$ in liquid has to increase to match the nominal shear rate $\dot{\gamma}= u/H$  imposed by the rheometer, here $u$ is the velocity of the upper rotating plate and H is the gap size. This is called the volume exclusion effect. The volume exclusion effect can be noticed when shearing the suspended liquid under the laminar flow, the measured shear stress under the same nominal shear rate increases with more particles. Notice the liquid viscosity is indifferent, the shear stress increase originates from the increased $\dot{\gamma}_{local}$.In our separate laminar flow test, we found $\dot{\gamma}_{local} = 4.3 \dot{\gamma}$ with $\Phi=30\%$ particles. If the torque oscillation originate from the local flow motion between the particles, we may expect $f \propto \dot{\gamma}_{local}$, or an increase in the oscillation frequency with an increased volume fraction under the same nominal volume shear rate. But there is no difference in oscillation frequency when varying the particle volume fraction. Because the frequency of the torque oscillation is therefore independent of the local shear rate, we conclude that the torque oscillation is not from the local flow among particles.
%If the torque oscillation is governed by the local shear between the particles, we expect a higher oscillation frequency with an increase in the volume fraction, which increases the shear rate between particles. We find that the torque begins to oscillate when the volume fraction of particles in the suspension is higher than 10\,\% . Further increasing the particle concentration generates a larger oscillation amplitude, but little change in the frequency. Because the frequency of the torque oscillation is therefore independent of the local shear rate, we conclude that the torque oscillation is not from the local flow around individual particles. (I do not understand conclude the oscillation is not from the local particles, as there is no volume fraction dependent. To explain why volume fraction independent means indepent towards local flow. Can a general audience understand the text?)

%The volume of liquid between particles decreases as the volume fraction of particles increases. As a result, the actual shear rate between particles is higher than the nominal shear rate imposed by the rheometer. Because the velocity difference between two parallel plates is the same but there is no velocity gradient inside the particles.
%If the behavior of the flow was governed by the actual shear rate, then we should expect difference in oscillation frequency with different volume fractions. Because the actual shear rate between particles is higher than the nominal shear rate imposed by the rheometer due to the volume exclusion effect. With a decreased distance between particles, the actual shear rate increases. Thus we expect an increase in frequency for any periodic motions between the particles because the actual shear rate is higher.
%The average distance between particles decreases with the increase in volume fraction, and therefore if Because the frequency does not ch the oscillation is not due the local flow between the particles, otherwise the frequency increases with an increased volume fraction.

%Further evidence that the torque oscillation is not generated by local flow comes from control experiments in which the size of particles is varied. We prepare three 30\% volume fraction suspensions using particles with diameters of $52, 30$ and $9.1 \mu m$ and shear the suspensions under identical conditions. We find no difference in the oscillations of the torque. The power spectra show that the amplitude and frequency of the torque fluctuations are the same in all three cases; see Fig. ?? in the SI,reinforcing that the promoted mode originates from global, rather than local flow. (very weak causality between the evidence and the conclusion, needs stronger and more clear argument. Also which argument it is going to make)

%The emergence of a new mode in the flow field of the viscoelastic suspension under high torsional shear . 
%is the highest shear rate that can be assessed with the current setup
%The scaling of the mode frequency turns out to be surprisingly simple.
\begin{figure}
\centering
\includegraphics[width=0.5\textwidth]{Fig 2.pdf}
\caption{\label{fig:Fig3} Scaling of the mode frequency. a) Dimensionless mode frequency $\tilde{f}$ as a function of the Deborah number De. The inset is the $\frac{\tilde{f}}{De}$ as a function of De, the ratio drift downwards as an increase of $De$. b) Rheoscopic visualization of the secondary flow under under condition $\dot{\gamma}=50\,\mathrm{/s}$ and $H = 2\,\mathrm{mm}$, the flow pattern is highlighted by the black dash line,c) Control experiment with $\dot{\gamma}=100\,\mathrm{/s}$ and $H = 2\,\mathrm{mm}$, the shape of the promoted mode is very similar to b). d) Rheoscopic visualization of the secondary flow when $\dot{\gamma}=100\,\mathrm{/s}$ and $H = 1\,\mathrm{mm}$, the total length of the spiral decreases when H decrease} 
\end{figure}
%We then repeat the experiment with different geometry configurations and different shear rates. Because in the torsional shear flow: $\dot{\gamma}=2/3\frac{\Omega d}{H} $, here $\Omega$ is the angular velocity of the upper rotating plate, we can tune the shear rate $\dot{\gamma}$ to angular velocity $\Omega$ ratio by changing $\frac{d}{H}$. Here, we repeat shearing experiments with $\Phi = 30\%$ particle suspension, d = 40 mm and d = 60 mm plates and $H$ stepping from 1 mm to 2.5 mm in 0.5 mm increments on a customized DHR-3 rheometer from TA-instrument.

%We find that the frequency of the torque oscillation $f$ increases proportionally to $\Omega$, independent of $H$ or $d$. By contrast, even with the same $\dot{\gamma}$, the frequency further depends on $H/d$  . The fact $f$ is dictated by $\Omega$, or $\tau_1$, suggest the torque oscillation originates from a flow pattern length scale comparable with the flow cell $d$, reassured the flow pattern we visualized in the rheoscopic fluid(fig 1.e). We find that the frequency of the torque oscillation $f$ increases proportionally to $\Omega$, independent of $H$ or $d$. By contrast, even with the same $\dot{\gamma}$, the frequency further depends on $H/d$  . The fact $f$ is dictated by $\Omega$, or $\tau_1$, suggest the torque oscillation originates from a flow pattern length scale comparable with the flow cell $d$, reassured the flow pattern we visualized in the rheoscopic fluid(fig 1.e).  For the flow spanning the entire flow field, a second time scale can be defined as $\tau_1=1/\Omega$, here $\Omega$ is the angular velocity of the upper rotating plate. It characterize how fast the entire flow field evolves over time, because if the flow field is not axial symmetric, as we see in the elastic turbulence, the driving stress is angle dependent. $2\pi\tau_1$ is the time needed for the upper rotation plate to reset its position and the flow cell boundary condition, so the global flow evolution scales with $\tau_1$. 

%We can define two characteristic time scales here:  $\tau_1=1/\Omega$ and $\tau_2=1/\dot{\gamma}$. They describe the flow field from  different perspectives. Because the shear rate $\dot{\gamma}$ is based on the deformation rate at each point, $\tau_2$ focus on describing the local flow field at each material point, including the flow around each single particle, similar to $\tau$. The measured oscillation $f$ is not dictated by $\tau_2$, agree with our previous volume fraction control experiments: the oscillation of the flow field is not around each single particle.

%On the other hand, $\tau_1$ characterize how fast the entire flow cell evolves over time, as $2\pi\tau_1$ is the time needed for the upper rotation plate to reset its position. 
%Since The relaxation time $\lambda$ is not included in $\tau_1$, if we vary the relaxation time $\lambda$, once $\Omega$ is large enough to trigger the secondary flow, $\lambda$ should not influence the ratio between the mode frequency and the angular velocity  $f/\Omega$. Indeed, replacing the polymer solution, which has a relaxation time of $0.8~s$, with a more dilute polymer solution with a shorter relaxation time of $0.19~s$, and measuring the torque fluctuation, we find $f/\Omega$ changes less than $5 \%$.

%By visualizing the flow pattern with rheoscopic liquid, we find although the ending point of the spiral move inward slightly when the gap size is halved from 2 mm to 1 mm, the flow patterns are all spiral and propagate from the edge to the middle half of the plate, regardless of shear rate or gap to radius ratio (fig 2 a.b. and c). The flow visualization agrees with the torque measurement where ${f} \propto \Omega$. Because even if  ${f}$ is dictated by $\Omega$, a linear relation between ${f}$  and $\Omega$ further suggests the flow pattern is roughly independent of the applied rotation rate. The actuation torque, estimated as $M = \frac{\pi\eta \Omega d^4}{32H}\propto \Omega$, increases linearly as the rotation rate increases, and thus drive the pattern motion faster, leads to a linear increase of the torque oscillation frequency.

%We can also convert the previous argument to dimensionless scaling. The local flow time scale $\tau$ and the global flow time scale $\tau_1$ can both be compared with the relaxation time of the system $\lambda$. if $\tau_1\gg\lambda$, the dissolved long chain polymer is fully relaxed, and the flow is viscous-dominated. If $\tau_1\ll \lambda$, the long chain polymer cannot relax and emit large elastic stress. The flow will be elastic-dominated. To describe such transition, the Deborah Number $De$ is purposed as the ratio between the relaxation time $\lambda$ and the characteristic time scale of the flow $\tau_1$: $De=\lambda/\tau_1=\lambda\Omega$.  We can also replot our data by plotting the dimensionless frequency $\tilde{f}=f\lambda$ versus $De$. $\tilde{f} \approx c_1 De$, where $c_1=0.025 $ as a constant. The linear relationship between $f$ and De suggests the elastic stress dominate over the viscous dissipation and the flow pattern is fully developed in our experiments. 

%There is another dimensionless number, the Weissenberg Number $Wi=\lambda/\tau$, which describes the ratio between the elastic stress and the viscous stress at each point. Since $f$ is not governed by $\tau$, $\tilde{f}$does not scale with $Wi$. (move to appendix)


%The simple argument ${f} \propto \Omega$ is also reflected in the rheoflow visualization. Even if  $\tilde{f}$ is dictated by $De$, a linear relation between $\tilde{f}$ and $De$ further suggests the flow pattern is roughly independent of the applied rotation rate. The actuation torque, estimated as: $M = \frac{\pi\eta \Omega d^4}{32H}\propto \Omega$, increase linearly as the rotation rate increases, and thus drive the pattern motion faster, leads to an linear increase of the torque oscillation frequency.

%Indeed, by visualizing the flow pattern with rheoscopic liquid, we find although the ending point of the spiral move inward slightly when the gap size is halved from 2 mm to 1 mm, the flow patterns are all spiral and propagate from the edge to the middle half of the plate, regardless of shear rate or gap to radius ratio (fig 2 a.b. and c). It confirmed the hypothesis from the dimensional analysis that the flow pattern is similar in different working conditions.  
%Because the frequency is dictated by the angular velocity, dimensional analysis suggests that the dimensionless frequency, $\tilde{f}=f\lambda$, depends on the Deborah Number $De$: $\tilde{f} \propto De = \lambda\Omega$. The Deborah Number $De$ describes the material relaxation time versus the characteristic time of the flow field. It reflects how the entire flow field is influenced by the elasticity and the number is flow cell geometry dependent. 

%In contrast, there is another dimensionless number, the Weissenberg Number $Wi=\lambda\dot{\gamma}$. Although out data does not scale with Wi, it is never the less very important, as it describes the ratio between the elastic stress and the viscous stress. It characterizes the contribution of the elasticity to the local flow at each point. The calculation of the Weissenberg number do not rely on the geometry constants of the flow cell. 

%$\tilde{f}$ depends on $De$ but not on $Wi$ agrees with our previous control experiments on the volume fractions: the torque oscillation originates from the flow field spanning the entire plate rather than the local motion among the particles, as reflected by the Deborah number. 
%The Deborah Number, instead, describes the material relaxation time versus the characteristic time of the flow field. It reflects how the entire flow field is influenced by the elasticity and the number is flow cell geometry dependent. In the torsional shear flow: $De=\lambda\Omega$. 

%We find that the frequency of the torque oscillation $f$ increases proportionally to $\Omega$, independent of the rheometer geometry. By contrast, even with the same $\dot{\gamma}$, the frequency further depends on $H/d$  (See Fig 2.a).  Because the frequency is dictated by the angular velocity, dimensional analysis suggests that the dimensionless frequency, $\tilde{f}=f\lambda$, depends on the Deborah Number $De$: $\tilde{f} \propto De$ , but not on the Weissenberg number $Wi$. This observation agrees with our previous control experiments on the volume fractions: the torque oscillation originates from the  flow field spanning the entire plate rather than the local motion among the particles, as reflected by the Deborah number. 

%There are two dimensionless numbers that describe the elastic instability: the Weissenberg Number $Wi$ and the Deborah Number $De$. Although they are often used interchangeably, these two dimensionless numbers are different. The Weissenberg number $Wi=\lambda\dot{\gamma}$ describes the ratio between the normal stress and the shear stress. It characterizes the contribution of the elastic stress to the local flow at each point. The calculation of the Weissenberg number do not rely on the geometry constants of the flow cell. The Deborah Number describes the material relaxation time versus the characteristic time of the flow field. It reflects how the entire flow field is influenced by the elasticity and the number is flow cell geometry dependent. (generally, if I alist things, its better to didcribe one, and the other one clearly. It does not help me to )In the torsional shear flow: $De=\lambda\Omega$, here $\Omega$ is the angular velocity of the upper rotating plate. To decouple these two dimensionless numbers, we change the flow cell geometry. For parallel plate shear flow, $\dot{\gamma}=2/3\frac{\Omega d}{H} $. We can tune the shear rate to angular velocity ratio by changing $\frac{d}{H}$. Here, we repeat shearing experiments with d = 40 mm and d = 60 mm plates and $H$ stepping from 1 mm to 2.5 mm in 0.5 mm increments on a customized DHR-3 rheometer from TA-instrument. %Decouple the angular velocity and the shear rate allow us to change the Weissenberg Number Wi and the Deborah Number De independently. Both of them describe the elastic instability. Weissenberg number describes the local shear flow induced elasticity. And the Deborah Number describes the global flow rate versus the material relaxation rate. 

%We find that the frequency of the torque oscillation $f$ increases proportionally to $\Omega$, independent of the rheometer geometry. By contrast, even with the same $\dot{\gamma}$, the frequency further depends on $H/d$  (See Fig 2.a).  Because the frequency is dictated by the angular velocity, dimensional analysis suggests that the dimensionless frequency, $\tilde{f}=f\lambda$, depends on the Deborah Number $De$: $\tilde{f} \propto De$ , but not on the Weissenberg number $Wi$. This observation agrees with our previous control experiments on the volume fractions: the torque oscillation originates from the  flow field spanning the entire plate rather than the local motion among the particles, as reflected by the Deborah number. (The dependence, why it is good I know it is the deborah number. We need to say what is the benefit to understand this question in this way. Only if it helps I should bring up these dimensionless numbers) (Explain data, bring up the arugments and say why this argument works. to connect different datas.)
%Physically speaking, it suggest the secondary flow here rely on the geometric configuration of the entire flow field, instead of the local motion of the liquid. 

%If it is true that $\tilde{f} \propto De$, then varying the relaxation time $\lambda$ should not influence the ratio between the the mode frequency and the angular velocity  $f/\Omega$. Indeed, replacing the polymer solution, which has a relaxation time of $0.8~s$, with another polymer solution that has a shorter relaxation time of $0.19~s$, and measuring the torque fluctuation, we find $f/\Omega$ changes less than 5 \%.

%The simple argument $\tilde{f} \propto De$ is also reflected in the rheoflow visualization. Even if  $\tilde{f}$ is dictated by $De$, a linear relation between $\tilde{f}$ and $De$ further suggests the flow pattern is roughly independent of the applied rotation rate. The actuation torque, estimated as: $M = \frac{\pi\eta \Omega d^4}{32H}\propto \Omega$, increase linearly as the rotation rate increases, and thus drive the pattern motion faster, leads to an linear increase of the torque oscillation frequency.

%Indeed, by visualizing the flow pattern with rheoscopic liquid, we find although the ending point of the spiral move inward slightly when the gap size is halved from 2 mm to 1 mm, the flow patterns are all spiral and propagate from the edge to the middle half of the plate, regardless of shear rate or gap to radius ratio (fig 2 a.b. and c). It confirmed the hypothesis from the dimensional analysis that the flow pattern is similar in different working conditions.  %actuates the flow pattern faster. (it means I set up the pattern faster. But the pattern is always there. it is just it rotates faster. So I need to find another way to explain it in English.) And the torque oscillation frequency increases accordingly.

%It suggests that the flow pattern is roughly indifferent under the change of the rotation rate.
 
 %Finally, three different sizes of the particles (52 $\mu$m, 30 $\mu$m, and 9.1$~\mu$m) with the same volume fraction 30 vol\% are tested. The secondary flow power spectrum in the three cases are very similar; see Fig. ?? in the SI. Both the amplitude of the torque fluctuation and the frequency of the promoted singe mode are alike. 
 %\section{Quantification}
 %From the scaling analysis, we conclude that the single mode originates from a flow pattern that spans the entire flow field.
The kalliroscope tells the pattern shape but does not offer velocity measurement. To further understand how exactly the particles changes the flow field, we directly quantify the velocity field using fluorescent tracer and Particle Imaging Velocimetry (PIV). To track the motion of the dense suspension, we label a small fraction (0.25 vol\% ) of the particles as tracers in the suspension with Rhodamine B fluorophore, which enables distinguishing single particle motion under shear. The in-plane motion of the flow field is then recorded by the trajectory of these tracer particles and reconstructed by PIV post-processing. Over time, the fluorescently-labeled particles move predominantly in the azimuthal direction, and some radial motion of the particles is also observed, as seen in Fig 3.a (SI for video). Nonetheless, the tracer density stays constant throughout the measurement.% And the depth of field (DoF) of the optics is thicker than the flow cell height and thus no tracers migrate out-of-focus.% and no significant radial transport is observed.
 
Using PIV to analyze the fluorescent images, we see that the velocity field of the secondary flow changes over time. We then use the optimized Dynamic Mode Decomposition (optDMD) method to extract its time invariant coherent structures. The optDMD assumes the time series $\textbf{v}_t$ is not fully random, but instead constituted by a series of coherent modes as major components: $$\textbf{v}(t) \approx\sum_{i=1}^n{\textbf{v}_i(t)} $$where each component $\textbf{v}_i(t)$ has simple dynamics, they grow exponentially while oscillate in a single frequency. Mathematically, $\textbf{v}_i(t)=e^{\alpha_i t}e^{i\omega_i t}\beta_i\textbf{v}_{i0}$. Here $\alpha_i$ is the growth rate, describes how rapid the amplitude changes. $\omega_i$ is the angular frequency of the oscillation. $\beta_i$ is the amplitude which reflects the mode's contribution to the time series. And finally the time-invariant normalized vector $\textbf{v}_{i0}$, which describes the shape of the mode. The optDMD finds $\textbf{v}_i$ by solving the following minimization problem :
 $$\min_{\textbf{v}_i}\sum_{t=0}^{t_0}||\textbf{v}(t) -\sum_{i=1}^n{\textbf{v}_i(t)} ||_2$$
 it search for the closest representation under the Euclidean norm iteratively.

 %Using PIV on the fluorescent images, we see that the velocity field of the secondary flow evolves over time, forming a time series. We extract the coherent modes of the time series via the optDMD method., then use the optimized Dynamic Mode Decomposition (optDMD) method to extract time invariant coherent structures. The classical DMD method converts the time series to the frequency domain, extracts the major coherent modes of the time series, and presents the time series as a sum of coherent modes. Each of these modes has 1) an oscillation frequency and growth rate, which encodes the dynamics of the mode, 2) an amplitude, which reflects the mode's contribution to the time series, and 3) an eigenvector, which represents the shape of the mode. The Optimized DMD, used in this study, is an improved variant of DMD that performs better when there is noise in the signal. (hard , but have to find a way to make it clear)
 In our optDMD processing, we decompose the flow at $\dot{\gamma}=50/s, H=2 mm$ into $n=22$ coherent modes.  This is the minimum number of modes necessary to produce a relative Euclidean difference $\sum||\textbf{v}(t) -\sum_{i=1}^n{\textbf{v}_i(t)} ||_2/\sum||\textbf{v}(t) ||_2$ of less than 15$\%$ as seen in Fig. 3b. In both cases with or without the particle suspension, the first mode is the base flow, which is invariant over time. The subsequent pairs of modes are complex conjugates and therefore share equal contributions (Fig 3.c). For all the modes, $\alpha_i$ equals 0, as the flow is statistically stationary.
 
 In the case of no particle suspension, the base flow contributes $45\%$. The subsequent eight modes are all spiral shapes (See SI) and are in same order of magnitude (see Fig 3.c). It agrees with the rheoflow visualization in Fig 1.e: the torsional elastic turbulent flow is compose of a palette of spiral modes, and no predominant mode can be observed. In contrast. With the particle suspension, the contribution of the primary flow increases to  $73\%$, suggesting the overall flow is much less turbulent. Mode 2 and 3 has $8.8\%$ relative contribution, while mode 4 to 9 contribute $9.9\%$ combined. The first pair of secondary flow modes here contribute substantially more than the subsequent modes, which suggests that the suspension suppresses higher-order modes and promotes one single mode: mode 2 \& 3. (see Fig 3.c)

% In our optDMD processing, we decompose the flow into $n=22$ coherent modes.  This is the minimum number of modes necessary to produce a relative L2 difference $\sum||\textbf{v}(t) -\sum_{i=1}^n{\textbf{v}_i(t)} ||_2/\sum||\textbf{v}(t) ||_2$ of less than 15$\%$ as seen in Fig. 3b. We compare the contributions $\beta_i$ of the first nine coherent modes with and without the particle suspension. The first mode is the base flow, which is invariant over time (fig 3.d). The subsequent pairs of modes are complex conjugates and therefore share equal contributions. We observe stark differences in the mode contributions of samples with and without added particles, as seen in Fig. 3c. For samples without added particles, the eight modes following the first have contributions that are similar order of magnitude. In contrast, in the case of the particle suspension, the primary flow contributes more, increased from $45\%$ to $73\%$, suggesting the overall flow is much less turbulent. Mode 2 and 3 has $8.8\%$ relative contribution, while mode 4 to 9 contribute $9.9\%$ combined with suspension. The first pair of secondary flow modes for the particle suspension have a substantially higher contribution than subsequent modes. This suggests that the suspension suppresses higher-order modes and promotes one single mode, mode 2 \& 3. 
 
 The frequency of the promoted modes 2 and 3 calculated from optDMD is 0.31 Hz, which matches the frequency peak in the independently measured torque spectrum (0.32 Hz) for the particle suspension, reconfirm the validity of the optDMD method. The flow pattern from kalloeiscope also swirls in about 3.3 seconds per cycle. This frequency agreement suggests that mode 2 from the suspension disrupts the original viscoelastic secondary flow, introduce a new swirling flow patterns as seen under kalliroscope, and breaks the power law scaling in the torque fluctuation.

 %The frequency of the promoted modes 2 and 3 calculated from optDMD is 0.31 Hz, which matches the frequency peak in the independently measured torque spectrum (0.32 Hz) for the particle suspension. Torque is proportional to the energy dissipation rate in the flow cell and reflects the dynamics of the system. And the coherent motion of the tracer particles captures the kinetics of the system. This frequency agreement suggests that the torque $M$ fluctuation we measured earlier and the dynamics of the system originate from the motion of the particle suspensions and reconfirm the validity of the optDMD method. 
  
%Though the secondary flow is very complex and typically does not have an analytical solution, the promoted mode with particles can be qualitatively explained. 

Put a paragraph here to discribe even if we only add one secondary flow mode it still do very well in terms of reconstruction, highlight the fact this mode is the most important.

The predominance of the first mode can be revealed from another perspective. If we only decompose the flow field into three coherent modes, it captures the essence of the flow field, as seen in Fig 3.b, 22 modes is the minimum needed to reach 15\% relative agreement. But we can get reasonable results even with a few modes.
Intriguingly, aside from the fact that modes 2 \& 3 contribute more, their eigenvector is also distinct from simple spirals. Instead, the velocity is roughly a constant in the center and rotates over time as a rigid disk, surrounded by the spiral pattern ( Fig 3.e). If the particles are dispersed in the liquid, they have no short-range order and should have relative motions to their neighbors. The constant velocity in the center with a limited gradient and the rigid body rotation suggest that the particles in the center no longer have large relative strains, implying they may phase separate and form a solid core, prohibiting relative motion other than rotation among particles. Thus the spirals originating from the secondary flow cannot penetrate. Succeeding the rigid body rotation mode, the higher-order modes are back to the spiral motion (Fig 3.f). 
 
\begin{figure*}
\centering
\includegraphics[width=1.0\textwidth]{Fig 3.pdf}
\caption{\label{fig:Fig2} Quantification of the elastic turbulence in polymer solution with 30\% suspension. a) The fluorescent image of the flow cell overlaid by the velocity field extracted from PIV analysis. b) The reconstruction of the velocity field from optDMD method. The Blue line is the measured azimuthal direction velocity over time, averaged over the entire flow field. The orange line is the reconstructed value by DMD method. 
%The L2 difference between the reconstructed and the measured velocity field is 14\%. 
c) Comparison of the contribution of the first nine modes decomposed by optDMD method, with and without the presence of the suspension. The addition of the suspension increase the contribution of the base flow, which is the Mode 1. It also promotes one single mode,  mode 2 \& 3, over the others. d) The flow field of mode 1 with particle suspension, it is the base flow which is invariant over time. e) The promoted single mode with particle suspension. The real part of the complex eigenmode is plotted here. The velocity field has little gradient near the center of the plate. The outer layer is composed of spirals. f) The higher order mode with particle suspension, which appears as a spiral.} 
\end{figure*}


%\section{Discussion}

%The coherent mode structure from the optDMD analysis suggest the particles phase separate and self-assemble into a solid core under the elastic turbulence. 
To directly quantify the evolution of the suspension microstructure under shear, we employ rheo-microscopy (MCR 702 with rheomicroscpe accessory from Anton Paar) to image the motion of each single particle in-situ. In this setup, the upper and lower plate counter rotates. The focal plane of the 5x objective is set to the middle plane, where the flow stagnates. Therefore, the particles near the focal plane can stay in the field of view and be tracked even when the shear rate is relatively high. We capture the frames with a high speed camera after shearing for 60 seconds. 60 seconds of shearing is long enough that the torque reading remain static until 600 seconds when the test ends, which suggests the particle structure reaches equilibrium.

Here we strongly shear particle suspensions up to $\dot{\gamma} \sim 10^2/s$, where high shear rate induce interfacial instability and lead to liquid expulsion. So instead of applying a steady shear, we use the Large Amplitude Oscillation Shear (LAOS) protocol to shear the sample impulsively and prevent the interfacial instability to fully develop. Here the angular velocity $\Omega=3.1416/s$, and the oscillation amplitude $\gamma=30.00$, the averaged applied shear rate can be estimated as $\dot{\gamma}=2\Omega\gamma/\pi=60/s$. (say the shear rate do not need to be exact)

When the volume fraction of the particle suspension is down to 5\:vol\%, the particles show attractive interactions and form local structures like short chains and small rafts under shear. Such structures further align with the flow direction and migrate accompanying the flow. Importantly, enough distance between assembles allows relative motions between neighbor structures without collision. So against such shear-induced assemble, the structures can still be considered dispersed since the microstructure size is order of magnitude smaller than the flow cell size $d$ and the local flow determines the structure motion. 
%(focus on 30\% case, briefly mention the 20\% case)
%most of the particles assemble into short chains or small rafts composed of 3 -- 30 particles under shear flow; See Fig. \ref{fig:Fig3}b, though some unassembled single particles can also be spotted. Such chaining of the particles is in line with previous literature. Importantly, there is enough distance between chains allows relative motions between the particles without collusion. So against such shear-induced assemble, the particles can still be considered as dispersed since the size of the micro structure is order of magnitude small than $d$. (not very clear what it means)
\begin{figure}
\centering
\includegraphics[width=0.5\textwidth]{Fig 4.pdf}
\caption{\label{fig:Fig4} Rheomicroscopic visualization of the particles self assemble under shear. a) 5\, vol\% suspension in viscoelastic liquid sheard by the large aptitude oscillation shear (LAOS) protocol, the angular velocity $\Omega=3.14 /\mathrm{s}$ and shear strain $ \gamma = 1$, the effective shear rate $\dot{\gamma}=3.14\,\mathrm{/s}$. b) The same formula sample sheared under $\Omega=3.14 /\mathrm{s}$ and shear strain $ \gamma = 30$. c)  same formula sample sheared under a constant shear rate $\dot{\gamma}=3.14\,\mathrm{/s}$.  d-f) Control experiments with 20\, vol\% suspension. Under large shear rate, the particle aggregate into flocs and phase seperated from the liquid. g-i) Control experiments with 30\, vol\% suspension. A further increase of the volume fraction increase in-plane volume fraction. }
\end{figure}
Surprisingly, when the volume fraction is increased to $\Phi = 20\%$, the behavior of the particles is qualitatively different. Under shear, we see particles crystallize into two-dimensional (2D) closely packed rafts which migrate like a rigid body during each LAOS cycle. The rafts further overlap with each other and lock by friction, forming even larger structures, see Fig. \ref{fig: Fig4}. A further increase of the volume fraction from $\Phi = 20\%$ to $\Phi = 30\%$ will generate more rafts, but the phenomenon is qualitatively similar; see Fig. \ref{fig:Fig4}, h). The crystallization is largely unexpected. Because in a monodispersed particle system, the minimum volume fraction needed for crystallization, or the melting concentration, is 54.5\%, which is much higher than 20\%.

The crystallization can be understood from two aspects: first, particles migrate towards the middle plane under viscoelastic shearing. Near the focal plane, the measured local particle surface fraction is 65\%, which increases more than two times over the bulk volume fraction, 20\%, and also much higher than our control experiment with no shearing. Second, particles further attract each other within the focal plane; the particles aggregate into 2D closely packed rafts, leaving the void space between the structures. Such transition to the solid-like state below the melting concentration indicates an attractive interaction between the particles.
Both the shear-induced migration and the attractive interaction are from the fluid viscoelasticity. There is no structure formation or significant migration in our control experiments with a Newtonian liquid under the same working condition. The viscoelasticity emits normal stress perpendicular to the streamline, pushing the particles away from the boundary since the flow field surrounding each particle becomes asymmetric when close to the wall. Moreover, the normal stress generate attractive interaction when particles are close to each other and their local flow streamlines overlaps. 

%\section{Validation}


If crystallization is the cause of the phase separation and the change of the flow dynamics, then the inhibition of the crystallization should prevent the phase separation and the rigid body rotation flow mode. Since the crystallization require the particle to be monodispersed, if we shear a polydispersed sample, even if there is attractive interaction, the particles cannot crystallize. Here we design a control experiment by mixing different sizes of the particles (10 vol\% 9.2 um, 10vol \% 30 um, and 10vol\% 52um particles) , indeed, by shearing the polydispersed suspension under LAOS test, no ordered structure is observed, even if the in-plane area fraction is much higher than 30\%. This observation suggests that the particles still migrate towards the middle plane and self assemble under the shear flow. However, the difference in particle sizes blocks the crystallization (Fig.\ref{fig:Fig5} d-f).
\begin{figure*}
\centering
\includegraphics[width=0.5\textwidth]{Fig 5.pdf}
\caption{\label{fig:Fig5}Polydispersed flow visualization}
\end{figure*}


We then repeat the rheoscopic visualization of the polydispersed sample to investigate the influence of the polydispersity on the flow pattern. Compared with the no suspension case, the rheoscopic flow pattern here is indistinct (Fig.\ref{fig:Fig5}.a), which suggest the secondary flow is suppressed by the suspension. However, importantly, no predominant pattern can be distinguished but instead multiple spirals coexist. Reflected on the rheometer reading, there is no single frequency peak in the power spectrum (Fig.\ref{fig:Fig5}.b).

In the DMD based mode analysis, in line with the rheoscopic flow visualization, there is no predominant mode in terms of the relative contribution. The  first ten modes of the secondary flows now have the same order of magnitude contribution but lower than the no particle suspension case. Also, the first eight secondary modes all belong to the spiral family; see 7 in the SI-7, suggesting the spiral pattern from the secondary flow can propagate across the flow field if there is no crystallization. This control experiments prove that the spinning motion and the violation of power law scaling is caused by the crystallization of the suspension under shear. 

The suspension flow has been a long-lasting research topic. People tend to view the particle suspension as a single continuous phase with an additional parameter $\Phi$. This perspective has achieved great success in both science and engineering applications. Yet, in this study, we discussed the limitation of the single-phase paradigm. Under the viscoelastic shear flow, the attractive interaction and the shear-induced migration lead to the particles' phase separation and crystallization. The formed structure further interferes the flow field. Such shear induced phase separation can be considered a novel method to separate particles from the medium not by filtering or centrifuge but by shearing. So far, we have only tested the torsional shear flow, but the shear-induced attractive interaction is not bounded to the flow cell geometry or the flow cell size; thus, we expect to observe other types of structure formation in other classical systems, like the pipe flow, T junction flow, and the Taylor–Couette flow.

Moreover, our understanding of the attractive interaction between particles is still lacking. Our data suggest the torque reaches equilibrium in a few seconds, and it can be interesting to understand the detailed mechanism of the microstructure formation.
%Overall, this control experiments here validated the fact that physical origin of the spinning mode is the crystallization of the suspension under shear. (too short, write a perspective, where it can take you. what can be the implication. The mainstream point of view is not correct, if I treat it as phase separation. This is a unique approach to find orders structure from a chaotic sytem. We can try different geometries and )
%prospective
%If this is what we found, what can we do but we have not done it yet. blabla. 
%\section{Conclusion}

%In this study, we demonstrated the unexpected interaction between non-Brownian suspension and the elastic instability. We firstly reported the fact non-Brownian suspension with finite volume fraction has strong regulation effect towards the elastic instability. The secondary flow is suppressed, while a novel predominant mode emerges. We further change the experimental conditions to reveal the scaling of the regulation effect, the  frequency scales with the rotation rate. It leads to the hypothesis that such predominant mode origin (a few words on the takehome message).

%flow pattern whose size is comparable with the flow cell. Then we use the fluorescent imaging and PIV method to directly quantify the flow field. Under the optDMD analysis, we found a novel rigid body rotation mode emerges when the present of the particle suspension. Next, we used rheomicroscope to visualize the flow induced phase separation of the suspension. At the end, we validate the causality between the crystallization and the predominant mode by repeat the tests with a polydispersed suspension. The polydispersity blocks crystallization and thus eliminate the single frequency mode. 

%So far we only tested the parallel plate geometry, we expect such flow induced phase separation may also be observed in other flow configurations like Cone and Plate, pipe flow or microfluidic flow. Also here the Brownian motion and inertial effect are eliminated, it could be interesting to consider these factors.

\end{document}
