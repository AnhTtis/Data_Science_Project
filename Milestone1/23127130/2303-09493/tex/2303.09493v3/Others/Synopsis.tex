
%\tableofcontents
\section{Bullet Points}
\begin{enumerate}
    \item Introduction
\begin{enumerate}
        \item People investigated the elastic instability and liquid carrying suspension. They are both important topics in fluid dynamics and rheology. 
        
        \item The elastic instability onset depends on the Paclet-McKinley Number.
        
\item Shear at a higher than onset shear rate will trigger the secondary flow, and eventually elastic turbulence.

\item The elastic turbulence is filled with the spiral modes.

\item People so far tend to view the suspended liquid as a single-phase system with an additional parameter: volume fractions. But it can be wrong.  So we look at the dynamics of the suspension under the viscoelastic secondary flow.

\end{enumerate}
\item Material and Method
\begin{enumerate}
    \item The setup is a commercial rheometer mounted with optics for visualization and a cup to eliminate the boundary effect.
    \item The system is monodispersed suspension in density and index-matched viscoelastic liquid.
\end{enumerate}
\item Secondary Flow
\begin{enumerate}
    \item The first experiment we do is we shear rheoscopic viscoelastic liquid , with and without suspension, to see the flow pattern and the rheometer reading of the secondary flow.
    \item We found the addition of the suspension regulate the secondary flow and promotes a rotating spiral mode. This is also reflected on the rheometer readings
\end{enumerate}
\item Scaling of the Frequency
\begin{enumerate}
    \item We repeat this experiment under different conditions. The phenomenon is very robust and reproducible. 
    \item We found more particles leads to a stronger rotation mode. But particle size does not matter.
    \item Since the particle size does not matter, it should be a collective motion. So the dimensionless frequency scales with the Deborah number (De) instead of the Weissenberg number (Wi) .
    \item If the scaling of De is true, the relaxation time does not matter. And indeed the Relaxation time does not influence the scaling.
\end{enumerate}
\item Quantification of the phenomenon
\begin{enumerate}
    \item We then use DMD to extract coherent modes. If collective motion hypothesis is true, we should see a global mode at the promoted frequency
    \item By DMD analysis, The elastic turbulence is composed of a catalog of the spiral modes which have similar contributions (kind of known to the field).
    \item By DMD analysis, The addition of the suspension makes rigid body spinning has more contribution than the others. And it is indeed at the promoted frequency.
\end{enumerate}
\item Physical Origin of the regulation
\begin{enumerate}
    \item Since the coherent mode rotates as a rigid body, it suggests particles phase separate from the liquid and form solid core
    \item We use rheomicroscope to visualize the particles under shear, we found particles phase separate and crystalize under the shear flow.
    \item Such crystallization happens once the Volume Fraction is more than 10 vol\%. 10 vol\% is also when we begin to see the promoted peak in rheometer.
\end{enumerate}
\item Validation
\begin{enumerate}
    \item If the crystallization is key, then we should not see phase separation under shear with polydisperse particles
    \item Indeed, by rheomicroscopy, there is no phase separation observed under shear even if the volume fraction is high,.
    \item We repeat the first experiment. And there is no promoted single mode after the addition of the polydisperse suspension. 
    \item From DMD analysis, The mode contributions are now comparable with each other
\end{enumerate}
\item appendix
\begin{enumerate}
    \item The onset depends on Weissenberg Number with the addition of the particles
    \item If we substitute Paclet Number curvature coefficient to 1, the Paclet number degenerate into the Weissenberg number.
    \item So we think the onset depends on the Weissenberg number because here curvature is redefined to be among the particles.


\end{enumerate}
\end{enumerate}