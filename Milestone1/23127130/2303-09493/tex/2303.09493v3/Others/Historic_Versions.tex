
%Viscoelastic liquid flows distinctly from the Newtonian liquid. When the shear rate is high, the liquid elasticity cannot be relaxed, the flow tends to be unstable, and the elastic instability happens. Elastic instability does not depend on length scale and can be found in infinitesimal inertia scenarios, including microfluidic , rheometric, and the porous media flow.  (start with physical picture)

%There are three dimensionless numbers associated with the elastic instability: The Weissenberg Number,  $\mathrm{Wi}$, the Deborah Number, $\mathrm{De}$ and the Pakdel-McKinley Number, PMcK. The onset of the elastic instability depends on the PMcK. Wi and De are periodically used interchangeably. Wi describes the elasticity from the local shear rate, $\dot{\gamma}$; De describes the ratio between the liquid relaxation time and the typical time of the flow . 
\textcolor{red}{Plan 1}: Particles dispersed in a liquid form a suspension. There has been substantial effort to characterize the mechanics of suspension flow. A description of suspension flow is relatively well-established when the liquid phase is Newtonian. So far, the society tends to view the suspended Newtonian liquid as a single phase with an additional parameter Volume Fraction $\Phi$.  This perspective is simple yet powerful. It can characterize different physical behaviors, from the volume exclusion effect to the continuous/discontinuous shear thickening effect. Dissolve long-chain polymer in liquid forms viscoelastic fluid. Suspension flow becomes much more complex when the liquid is viscoelastic. Particles begin to migrate across the streamlines, attract each other, or abnormally sediment faster. All these complicated behaviors are associated with the elasticity of the liquid. Another significant viscoelastic flow behavior is the elastic instability. When the shear rate is high, the elastic stress overcomes viscose dissipation. The base flow will be disrupted, and the elastic instability will happen. Elastic instability does not depend on length scale and can be found in infinitesimal inertia scenarios, including microfluidic, rheometric, and porous media flow.  The torsional shear flow between two parallel plates is a typical model flow to investigate the elastic instability. Further increasing  the shear rate in torsional shear flow will develop a secondary flow apart from the base flow. Eventually, the fully developed secondary flow will be turbulent. In the elastic turbulence, people found a catalog of evolving spiral modes. Modes in turbulence keep evolving as momentum is exchanged among these spirals and eventually dissipated. However, the impact of the elastic instability on the suspension is yet largely unknown. 

\textcolor{red}{Plan 2}: Liquid flows under shear stress. Solid, on the other hand, only deforms. Solid particles dispersed in liquid form the suspension. The suspension is frequently characterized as a single phase with an additional parameter Volume Fraction $\Phi$. Under a wide range of volume fractions, suspension flows like a liquid, as the suspended particles can freely exchange their neighbors.  There have been numerous studies investigating the phase separation behavior in the suspension system. The majority of them focus on tuning the particle-particle interaction. When the attractive interaction overcomes the thermal fluctuation, solid particles tend to phase separate from the liquid. Such attractive interaction either come from the bulk, like magnetic or electrostatic interaction. Or from the surface, like depletion force or surface chemistry. However, these interactions are system-dependent. When particles are small, the volume interaction diminish, when particles are huge, the surface interaction diminish. In this study, we are demonstrating a different way to phase separate particles from the liquid. It does not rely on particle-particle interaction, but instead, the attraction force is from the flow.

Dissolve long-chain polymer in liquid forms viscoelastic liquid. Under shear, the long-chain polymer is stretched and attributes elasticity to the liquid. Elasticity generates normal stress perpendicular to the flow direction. When two particles approach each other under the viscoelastic flow. The normal stress aggregates the particles. Such particle attraction is observed under sedimentation and shear flow. Those aggregation, when happens in the laminar flow, tend to be local. Particles forms log like structure. However, the elastic turbulent flow is yet unexplored.  (Set up a discussion with Gareth, Howard and you and me)
%Dissolve long-chain polymer in liquid forms viscoelastic liquid. Under shear, the long-chain polymer is stretched and attributes elasticity to the liquid. When the shear rate is high, the elastic stress overcomes viscose dissipation. The base flow will be disrupted, and the elastic instability will happen. Elastic instability does not depend on length scale and can be found in infinitesimal inertia scenarios, including microfluidic, rheometric, and porous media flow.  Torsional shear flow between two parallel plates is a typical model flow to investigate the elastic instability. Further increase the shear rate in torsional shear flow will develop a secondary flow apart from the base flow. Eventually, the fully developed secondary flow will be turbulent. In the elastic turbulence, people found a catalog of evolving spirals modes. Modes in turbulence keep evolving as momentum is exchanged among these spirals and eventually dissipated.

%Liquid carrying suspension, on the other hand, is omnipresent in engineering and daily life. So far, the society tends to view the suspended liquid as a single phase with an additional parameter Volume Fraction $\Phi$.  This perspective is simple yet powerful. It achieved resounding success with the Newtonian liquid. It can characterize different physical behaviors, from the volume exclusion effect to the continuous/discontinuous shear thickening effect. Thus it is natural  to extend this perspective from the Newtonian liquid to the viscoelastic liquid. So far, the single-phase model has reached limited success in the viscoelastic liquid. It can explain the increase of the elastic stress and the shear stress under the steady shear or the small strain oscillation. Some other research also found the particles tend to migrate and chain in the viscoelastic flow. But all of these researches stay within the laminar flow regime. The missing puzzle piece here is how do the particles behave under the elastic turbulent flow. Will particles follow the turbulent streamline? Will they keep suspended? Is the single-phase hypothesis retained? To answer these questions, we need to study the interaction between the particles and the liquid under elastic turbulence. (Completely new way to crystalize. Current issue: talked about whats done with the polymer in solutions, I talked with the particles. I have not be driven to think why mix these ideas together. I got to find a way to motivate people to think I have to study this. What is known: Think about this way: Starting this: Particle in normal . We know particles in Newtonian, simple, we know something about particles in viscoelastic liquid, but we know viscoelastic liquid has instability. so we want to see how does the viscoelastic instbaility do to the particles.)
%The addition of the suspension to the viscoelastic liquid raises more challenges to characterize the system mechanically. So far, the researchers tend to view the suspended liquid as a single phase with an additional parameter Volume Fraction $\Phi$. This perspective is simple yet powerful. It agrees with a wide range of experimental observations, especially the volume exclusion effect. Previous research, firstly reported by Einstein\cite{einstein1905neue}, found the fact that streamline cannot penetrate the solid particles, as the volume is excluded from the liquid. Thus the local shear rate between the particles will increase to match the global velocity gradient. An increase in the local shear rate will amplify the shear stress, so do the normal stress.   However, the single-phase hypothesis is yet an oversimplification. People do not know how the suspension behave in the elastic turbulence. Whether the particles will simply follow the streamline. Or they still homogeneously suspend in the liquid to form a single phase. (lack of motivation. If I know it regulates the secondary flow. Tell me why I need to know something will happen, I will solve all my problem. Motivate me, if I really want to know why it happens. )


In this study, we combined different rheoflow visualization techniques to investigate the elastic turbulence's particle dynamics experimentally.  We found non-dilute particles ($\Phi>10~\mathrm{vol}\% $) overhaul the elastic secondary flow. The evolving spirals found in the elastic instability are suppressed while a spinning spiral mode at specific frequency emerges. We speculate then substantiate the hypothesis that such regulation originates from the particles' shear-induced crystallization. As the suspended viscoelastic liquid is no longer a continuous single phase, but phase separated into a rotating solid core surrounded by the liquid. We demonstrate how could a chaotic flow surprisingly nourish ordered structure formation and phase separation. It can also be a penitential new method to phase separate particles from the immersion liquid by shearing.
%We extract the promoted coherent mode from the flow field. it is a rigid body rotation. We find The rigid body is composed of the crystallization of the suspension. 

%Shearing the liquid at a nominal shear rate higher than the elastic instability onset  will trigger the secondary flow and eventually the elastic turbulence. (what is turbulence? what is the secondary flow?). The elastic turbulence between two parallel plates is a typical model flow. Previously people found the elastic turbulence here is filled with a catalog of evolving spirals. (explain spiral flow)

%Viscoelastic liquid carrying suspension is applied to hydraulic fracturing, polymer composite manufacturing, and other industrial fields. (not the important thing, if I want to demo the industral importance, I have to show at certain point of the draft). The rheometric influence (weak definition) of the suspension on the viscoelastic fluid has been widely investigated\cite{barnes_review_2003}. The addition of the suspension to the viscoelastic liquid further complicates the system (not saying anything, focus on highlighting the interest the two sentences, not get distracted). So far, the researchers tend to view the suspended liquid and a single phase with an additional parameter Volume Fraction $\Phi$. This perspective is simple yet powerful. It agrees with a wide range of experimental observations.(not very clear as counter point). However, the single-phase hypothesis is an oversimplification. People do not know how the suspension behave in the elastic turbulence. There is no guarantee particles will simply follow the turbulent streamline. (English is terrible.)

%Previous research, firstly reported by Einstein\cite{einstein1905neue}, found that the non-Brownian suspension has the volume exclusion effect, which emerges from the fact that streamline cannot penetrate the particles. Thus the local shear rate between the particles will increase to match the global velocity gradient. An increase in the actual shear rate will amplify the shear stress, so do the normal stress difference. 

\cite{barnes_review_2003,acharya1987viscoelasticity}. %Understanding the viscoelastic flow is an important while challenging task. The society tend to model the mixture as a 

% from multiple perspectives; including the coupling of multiple physical effects; system-dependent behavior and computational difficulty to solve nonlinear constitutive equations effectively.

%An interesting aspect of the viscoelastic flow is the normal stress effect. When polymer solution flows, the elongation of the long-chain polymer generates elasticity in the liquid, and thus the normal stress difference in the fluid. Such viscoelastic effects accumulate with an increase of shear rate and eventually disrupt the base flow\cite{larson_purely_1990,mckinley_observations_1991,pakdel_elastic_1996}. Secondary flow then emerge, which is named elastic instability.

%Further increase of the nominal shear rate beyond the onset of the elastic instability will develop the secondary flow. Eventually the flow field becomes chaotic, even with the absence of inertia effect\cite{schiamberg_transitional_2006,groisman_elastic_2004}. Such elastic turbulence has been investigated for decades, while the detailed understanding still lacks.

%The addition of the suspension further complicates the system. The rheometric influence of the suspension on the viscoelastic fluid has been widely investigated\cite{barnes_review_2003}. Previous research, firstly reported by Einstein\cite{einstein1905neue}, found that the non-Brownian suspension has the volume exclusion effect, which emerges from the fact that streamline cannot penetrate the particles. Thus the local shear rate between the particles will increase to match the global velocity gradient. An increase in the actual shear rate will amplify the shear stress, so do the normal stress difference. 


%The interaction between suspension particles in viscoelastic flow is distinct from the Newtonian counterpart. Previous research found, because of the normal stress effect, the particles attract each other and chain along the streamline under the shear flow\cite{james_j_feng_motion_nodate,bhatara_influence_2004}. Moreover, the normal stress gradient in the flow field induces directional migration of the particle\cite{james_j_feng_motion_nodate}.

%However, The interaction between suspension and elastic instability is yet largely unknown. Since even the laminar flow generates strong interaction between the particles, it will be a substantial oversimplification to assume the particle will simply follow the flow streamline of the secondary flow.

%In this study, we combined multiple methods, including Rheoscopic visualization, Fluorescent imaging, Particle Imaging Velocimetry, Dynamic Mode Decomposition, and Rheomicroscopy, to experimentally investigate the influence of the suspension on the secondary flow beyond the elastic instability. 

%We find that  the presence of dense ($>10~\mathrm{vol}\% $) suspension will dramatically change the physics of the elasticity induced secondary flow. The suspension regulates the secondary flow to be less chaotic while promoting rigid body rotation mode. We further speculate then substantiate the hypothesis that such regulation originates from the particles' shear-induced crystallization. 

%Our study finds that the suspension promotes highly regulated periodic motion out of the chaotic flow. More generally, the viscoelastic flow-carrying suspension should not be treated as a single-phase homogeneous system, even if the flow field length scale is orders of magnitudes larger than the particle size.  Instead, the flow can induce heterogeneity in particle distribution and, eventually, precipitation.

Move to the end
\section{Material and Methods}

%Sample preparation start with preparing the polymer solution as the viscoelastic liquid. we first prepare a Newtonian solution and then add the polymer. The Newtonian solution is composed of $85.404 \:\text{wt} \%$ 2-2 Thiodiethanol (CAS 111-48-8 from Sigma) and $14.596\:\text{wt}\%$ deionized (DI) water. The solution is chilled to  $-20~\mathrm{^{\circ} C}$.  Refrigerate the liquid is important to prevent  polymer from clumping during dissolving. (think hard about what is the best way to tell everything. For example, why thiodiethanol, why do you use that? so then I described the polymer, the viscosity is ~?. It is not a history lesson. I probably want to start what the sample ultimately is, a viscoelastic fluid with particles. I need to think about the history. Why I do these measurements. To make it clear. The order has )

%Before the addition of polymer, we chill the sample to $-20~\mathrm{^{\circ} C}$ (Add polymer, in order to prevent clumping, we have to chill it).to prevent clumping of the undissolved polymer. 
%$0.2500\:\text{wt}\%$ polyethylene oxide (PEO, molecular weight $\text{M} = 8 \times 10^6$; CAS 25322-68-3 from Sigma) is then added to the continuous phase. The liquid is further baked under $65~\mathrm{^{\circ} C}$, and gently sheared for six hours to reach homogeneity. The density of the polymer solution is $1.18 ~\mathrm{g/ml}$. The viscosity of the solution $\eta = 0.8 ~\mathrm{Pa \cdot s}$ under shear rate $\dot{\gamma} = 10~\mathrm{/s}$. And the relaxation time of the liquid $\lambda$ is $0.8 ~\mathrm{sec}$ measured by $\lambda=N_1/2\eta\dot{\gamma}^2$ with $\dot{\gamma} = 10~\mathrm{/s}$, where $\lambda$ is the relaxation time and $N_1$ is the first normal stress difference. The concentration of the polymer is above the overlapping concentration. The rheometric properties are measured with a DHR-3 rheometer from TA instrument mounted with a  $60 ~\mathrm{mm}, 2\mathrm{^{\circ}}$ Cone-Plate geometry.

%Finally, the particles are mixed with the prepared viscoelastic liquid. The particles are composed of polymethyl methacrylate (PMMA)  (Spheromers CA50 from Microbeads). For most of this study, 30\% by volume monodispersed PMMA particles with diameter $d=52.4 \pm 0.8 ~\mathrm{\mu m}$ are used. The suspension is non-Brownian. Peclet number $\mathrm{Pe}$ describes the advection rate to the diffusion rate ratio of the particles. In this study $\mathrm{Pe} > 1 \times 10^6$.  (write down the formular how it is calculated.)

%In order to allow optical flow-visualization, the continuous phase formula is exacted to match the refractive index with the suspension, both has Refractive index $n = 1.4988$ for 589 nm wavelength light. Therefore, the final mixture is transparent. The density of the continuous phase ($1.18 ~\mathrm{g/ml}$) is also close to the suspension ($1.19 ~\mathrm{g/ml}$). The sedimentation is negligible within the time scale of the experiment. Shield Number, characterizing the sedimentation to advection ratio, is less than $10^{-4}$. 

%The prepared sample is then loaded to the rheoflow visualization setup. The rheoflow visualization setup is based on a commercial rheometer  (MCR 501 from Anton Paar) with customized optics. When the shear rate is high, viscoelastic fluid tend to escape from the flow chamber and entrench air bubbles. To tackle this issue, we eliminate the liquid-air interface by submerge the upper rotating plate of the rheometer in a pool of solution; see the sketch in Fig. \ref{fig:Fig1}a. The distance from the top air-liquid interface to the upper plate is controlled to be $7-15$ times larger than the gap size. As an estimate, more than $85\%$ of the torque reading is from the sample in the gap. The flow above the upper plate keeps laminar  and does not generate additional torque fluctuation.

%To allow rheoflow visualization, the rheometer is installed with a 1/4" thick customized plexiglass  bottom plate. The upper plate is also attached with an OD-4 ND filter (Hoya ND-006 from Edmund Optics) to minimize the reflected light (Ref here). The upper and lower plate are adjusted to have less than $50~\mathrm{\mu m}$ gap misalignment. A standard flow visualization setup is mounted beneath the plexiglass, including a LED lightsource (JANSJÖ from IKEA), a mirror and a CMOS camera (Blackfly S BFS-U3-32S4M with AF Nikkor 28mm f/2.8D).

Move to the end
\section{Onset of the instability}

%When viscoelastic liquid is sheared between two parallel plates, the elasticity of the long chain polymers generate hoop stress, which tends to disrupt the laminar flow and induce elastic instability. The onset of the elastic instability is related to parameters include the relaxation time of the fluid $\lambda$, the shear rate $\dot{\gamma}$, and ratio of the gap size $H$ to the radius of the plate $R$. The onset is controlled by the Pakdel-McKinley (PM) number (how well known is this instability, do we need to explain this a little bit more?)
%\[PM_{\mathrm{crit}}=\sqrt{(2H/R)}\dot{\gamma}\lambda\] 
%In our system, the critical shear rate vary from 15 to 40 depends on the Gap-Radius ratio. The $PM_{\mathrm{crit}} \approx 14.1$; see the rheological characterization in the SI. 

%With the presence of the suspension, we find that the threshold of the instability becomes independent of gap-radius ratio $R/H$, but solely depend on the Weissenberg Number $Wi=\dot{\gamma}\lambda$. The measured critical shear rate is 14/s. This degeneration can be understood as that the PM number equals to the Weissenberg number multiplied by a dimensionless factor describing the curvature of the system. With the presence of the suspension, the curvature is redefined to be locally among the particles rather than the global geometrical configuration. Therefore, the instability of the viscoelastic particle suspension only depends on the shear rate and relaxation time.
%(not very clear. Think about it might be time I could write what I did. Looking at the stresses fluctuation. And then discribe why I have this thing. It is not so bad if I say I looked at )
Move to the end

% Start of the first paragraph
\section{Secondary Flow}

When viscoelastic liquid is sheared between two parallel plates,  the long-chain polymers are stretched and generate hoop stress, which tends to disrupt the laminar flow and induce the elastic instability.  Further increasing the shear rate, secondary flow emerges.  Eventually, the flow becomes chaotic.  This chaotic flow is also named the elastic turbulence. 


Under the secondary flow, the liquid tends to escape from the flow chamber and entrench air bubbles. To characterize the secondary flow without the interfacial issue, we submerge the upper plate of the rheometer in a pool of liquid (see Fig. \ref{fig:Fig1} a) and record the torque as a function of time. In the secondary flow, the torque keeps fluctuating over time (see Fig. \ref{fig:Fig1}b). On the other hand, in the laminar flow, the torque remains constant within the tolerance of the sensor (see SI).

 If we transform the torque signal to the frequency domain, the Power Spectrum Density (PSD) of the torque decrease linearly as frequency ( $f$ ) increases under the log-log scale. And there is no predominant peak on the spectrum.\ref{fig:Fig1}c. This is because in the elastic turbulence, the competition between elastic force and viscous dissipation leads to the so-called energy cascade. Within the energy cascade, the Kolmogorov-like PSD spectrum has a power-law relationship versus frequency: $\mathrm{PSD} \propto f^k$, where $k<0$. 
 
 To better understand the secondary flow, we employ the rheoflow visualization setup to visualize the flow field. Here, the rheometer is installed with a customized transparent bottom plate and optics beneath it. 0.5 wt\% mica flake are seeded to the system to form rheoscopic liquid. Under shear, the mica flake aligns with the flow direction locally. Consequently, the flow field can be qualitatively assessed via the reflected pattern formed by different local orientations of the mica flake.
 
in the elastic turbulence, we observe multiple spirals emerge and vanish over time; And multiple spiral modes coexist simultaneously, see Fig.\ref{fig:Fig1}e and Video 1 in the SI. Such observation is in line with the previous studies. It suggests the torque fluctuation originate from the unsteady flow field.

Interestingly, when we add 30 vol\% mono-dispersed index and density matched particles to the liquid.  The torque fluctuation is now superposed with a new single frequency signal. The torque fluctuation magnitude with suspension is comparable to that without suspension. But the mean torque reading $\bar{M}$ is increased to 297\%. Thus the addition of 30 vol\% suspension decreases the secondary flow strength by 79.25\%, measured by $\mathrm{\sigma_ M/\bar{M}} $. Here $\mathrm{\sigma_M}$ is the standard deviation of the torque. 
%The standard deviation of the torque reading, $\mathrm{\sigma_M} = 30.4\mathrm{\mu Nm}$ with suspension and $\mathrm{\sigma_M} = 49.2\mathrm{\mu Nm}$ without suspension. Note that the mean value of the torque reading $\bar{M}= 2496 \mathrm{\mu Nm}$ with suspension and  $\bar{M}= 838 \mathrm{\mu Nm}$  without the suspension (SI Fig. ??). The strength of the secondary flow, $\mathrm{\sigma_ M/\bar{M}} =5.88\%$ without suspension and $\mathrm{\sigma_ M/\bar{M}} =1.22\%$ with suspension. The addition of 30 vol\% suspension decreases the strength by 79.25\%. (draw the figure to make it clear, discribe the figure in words to make it really clear)

%Interestingly, when we add 30 vol\% mono-dispersed particles to the liquid.  The torque fluctuation is now superposed with a new single frequency signal. The magnitude of that torque fluctuation with suspension is comparable to that without suspension. The standard deviation of the torque reading, $\mathrm{\sigma_M} = 30.4\mathrm{\mu Nm}$ with suspension and $\mathrm{\sigma_M} = 49.2\mathrm{\mu Nm}$ without suspension. Note that the mean value of the torque reading $\bar{M}= 2496 \mathrm{\mu Nm}$ with suspension and  $\bar{M}= 838 \mathrm{\mu Nm}$  without the suspension (SI Fig. ??). The strength of the secondary flow, $\mathrm{\sigma_ M/\bar{M}} =5.88\%$ without suspension and $\mathrm{\sigma_ M/\bar{M}} =1.22\%$ with suspension. The addition of 30 vol\% suspension decreases the strength by 79.25\%. (draw the figure to make it clear, discribe the figure in words to make it really clear)

%In the elastic turbulence, the competition between elastic force and viscous dissipation leads to the so-called energy cascade. If we transform the torque signal to the frequency domain, the Power Spectrum Density (PSD) of the torque has a power-law relationship versus frequency. And there is no predominant peak on the spectrum; see Fig. \ref{fig:Fig1}c. Such observation agrees with previous reports. (describe why I have a straight line, to be very specific, for instance, here is the loglog plot, I have a straight line, why, what is different with the suspension etc.)

In the frequency domain, a single frequency peak emerges (Fig. \ref{fig:Fig1}c). The emerged signal peak has a high signal to background ratio. It has two orders of magnitudes higher PSD than the background. And the width of the peak $\Delta f/\bar{f} \approx 1.5\%$, where $\Delta f$ is the peak width at half height and $\bar{f}$ is the peak frequency. Note that the peak frequency is not the rotation frequency of the driven plate (highlighted in the black dash line in fig 1.c), where the artifact is frequently observed, nor an integer fraction of it.

%To study the origin of the rheometric reading change, we employ rheoflow visualization setup to visualize the flow field. To allow rheoflow visualization, the rheometer is installed with customized plexiglass  bottom plate and optics beneath it. 0.5 wt\% mica flake are seeded to the system to form rheoscopic liquid. Under shear, the mica flake aligns with the flow direction locally. Consequently, the flow field can be qualitatively assessed via the reflected pattern formed by different local orientations of the mica flake.

%in the elastic turbulence, we observe multiple spirals emerge and vanish over time; see Fig. \ref{fig:Fig1}b and Video 1 in the SI. Such observation is in line with the previous studies in this field.

The change of the rheometric reading is also reflected in the change of the flow pattern. the addition of the suspension dramatically changes the kinetics of the secondary flow. Compared with the counterpart with only the polymer solution, the flow pattern with suspension is highly regulated: different modes seen in the elastic turbulence are imperceptible. Instead, one flow pattern dominates over the others and persists across the entire experiment. The promoted pattern appears as a spiral, which propagates from the edge of the flow field to the middle point between center and periphery; see Fig. \ref{fig:Fig1}c, and SI Video II).  The measured torque oscillation frequency 0.323 Hz is also the frequency of the spiral rotation. It suggests the torque reading oscillation originates from the change of the flow field. (order is okay. Question is what to do: language issue. )

%The change of the flow pattern is also reflected in the change of the rheometric reading. Compared with polymer solution alone, the suspension superposes the original torque fluctuation signal with a new single frequency signal; see Fig. \ref{fig:Fig1}e.

%The magnitude of that torque fluctuation with suspension is comparable to that without suspension. The standard deviation of the torque reading, $\mathrm{\sigma_M} = 30.4\mathrm{\mu Nm}$ with suspension and $\mathrm{\sigma_M} = 49.2\mathrm{\mu Nm}$ without suspension. Note that the mean value of the torque reading $\bar{M}= 2496 \mathrm{\mu Nm}$ with suspension and  $\bar{M}= 838 \mathrm{\mu Nm}$  without the suspension (SI Fig. ??). The strength of the secondary flow, $\mathrm{\sigma_ M/\bar{M}} =5.88\%$ without suspension and $\mathrm{\sigma_ M/\bar{M}} =1.22\%$ with suspension. The addition of 30 vol\% suspension decreases the strength by 79.25\%.

%In the elastic turbulence, the competition between elastic force and viscous dissipation leads to the so-called energy cascade. The Power Spectrum Density (PSD) of the torque has a power-law relationship versus frequency. And there is no predominant peak on the spectrum; see Fig. \ref{fig:Fig1}e. Such observation agree with previous reports.

%In the torque measurements of the suspension case, a single peak emerges in the frequency domain (Fig. \ref{fig:Fig1}c). The emerged signal peak has two order of magnitudes higher PSD than the background. The width of the peak $\Delta f/\bar{f} \approx 1.5\%$, where $\Delta f$ is the peak width at half height and $\bar{f}$ is the peak frequency. Note that the peak frequency is not the rotation frequency of the driven plate, nor an integer fraction of it.

\begin{figure*}
\centering
\includegraphics[width=1.0\textwidth]{Fig 1.pdf}
\caption{\label{fig:Fig1} elastic turbulence in a viscoelastic fluid with and without suspension.  a) Sketch of the setup. Optics for imaging is mounted beneath a commercial rheometer. b) Rheoscopic visualization of the flow field when elastic turbulence occurs, without adding suspensions. Multiple spirals coexist. The Gap size is 2mm, the Geometry radius is 25mm, Nominal Shear Rate $50\,/s"$c) Rheoscopic visualization of the flow field when elastic turbulence occurs, with $30~\mathrm{vol}\%$ PMMA particles. A single spiral dominates over the other modes d) The sketch of the system. The viscoelastic fluid with dense suspension is sheared between two parallel plates. e) The fluctuation of torque reading $\Delta M$ measured by rheometer. The blue and orange lines respectively show the torque fluctuation in the experiments when elastic turbulence occurs, with and without adding 30\% suspension particles. f) The power spectrum of the torque measurment. In the viscoelastic fluid with suspension, a single peak is observed at  0.323 Hz. The black dash line represents the rotating frequency of the upper plate. }
\end{figure*}


\section{Scaling of the mode frequency}
To determine the scaling  frequency, The emergence of a new mode in the flow field of the viscoelastic suspension under high torsional shear . We  at shear rates is down to 15\,/s, which is succeeding the onset of the secondary flow. This single mode persists as the shear rate increases until at least 200\,/s. It is the highest shear rate that can be assessed with the current setup. Beyond $\dot{\gamma} = 200\,\mathrm{/s}$, the rod climbing effect causes the liquid to overflow.

For a systematic comparison, we test different experimental conditions, including different particle sizes and volume fractions, flow cell geometry and liquid property. The scaling of the mode frequency turns out to be surprisingly simple.

We prepare samples with different particle volume fractions. The single mode emerges when the volume fraction of the suspension is higher than 10\,\% . Further increase of the particle concentration nurtures a larger amplitude, while the frequency of the mode varies little. The space between particles decreases with the increase of the volume fraction. So we suspect the observed mode is not the local flow between the particles, otherwise the frequency increases with an increased volume fraction.

Further proof comes from changing the particle size. We prepare the same 30\% volume fraction particles with different size . We find no difference in rheometric reading. The power spectra in the three cases are very similar; see Fig. ?? in the SI. Both the amplitude of the torque fluctuation and the frequency of the promoted singe mode are alike. If the promoted flow originate from the local flow between the particles, different particle sizes will yield different dynamics. 


\begin{figure}
\centering
\includegraphics[width=0.5\textwidth]{Fig 2.pdf}
\caption{\label{fig:Fig3} Scaling of the mode frequency. a) Dimensionless mode frequency $\tilde{f}$ as a function of the Deborah number De. The inset is the dimensionless frequency to Deborah Number ratio as a function of the Deborah Number, the ratio drift downwards as an increase of the Deborah Number. b) Rheoscopic visualization of the secondary flow under under condition $\dot{\gamma}=50\,\mathrm{/s}$ and $H = 2\,\mathrm{mm}$, the flow pattern is highlighted by the black dash line,c) Control experiment with $\dot{\gamma}=100\,\mathrm{/s}$ and $H = 2\,\mathrm{mm}$, the shape of the promoted mode is very similar to b). d) Rheoscopic visualization of the secondary flow when $\dot{\gamma}=100\,\mathrm{/s}$ and $H = 1\,\mathrm{mm}$, the total length of the spiral decreases when H decrease} 
\end{figure}

We then change the flow cell geometry. It allow use to decouple the shear rate $\dot{\gamma}$ and the angular velocity of the rotation plate  $\Omega$ . For parallel plate shear flow, $\dot{\gamma}=2/3\frac{\Omega d}{H} $. Here the d = 40 mm and 60 mm plates with gap size stepping from 1 mm to 2.5 mm are tested with a customized DHR-3 rheometer from TA-instrument. Decouple the angular velocity and the shear rate allow us to change the Weissenberg Number Wi and the Deborah Number De independently. Both of them describe the elastic instability. Weissenberg number describes the local shear flow induced elasticity. And the Deborah Number describes the global flow rate versus the material relaxation rate. 

Under all the gap and diameter combinations, the frequency of the mode is controlled by $\Omega$ , but varies under the same $\dot{\gamma}$ ( See Fig 2.a). Dimensional analysis suggests Dimensionless frequency$\tilde{f}$,  where $\tilde{f}=f\lambda$, depends on the Deborah Number $De=\Omega\lambda$, but not on the Weissenberg number $Wi ~=~\dot{\gamma}\lambda$. This observation agree with our previous control experiments: the mode originates from the global flow configuration.
%Physically speaking, it suggest the secondary flow here rely on the geometric configuration of the entire flow field, instead of the local motion of the liquid. 

If the hypothesis  $\tilde{f} \propto De$ is true, then varying the relaxation time $\lambda$ should not influence the ratio between rotation rate and the mode frequency $f/\Omega$. And indeed, we use another polymer solution with lower relaxation time $0.19~s$ instead of $0.8~s$ . $f/\Omega$ changes less than 5 \%.

 The simple argument $\tilde{f} \propto De$ is also reflected on the rheoflow visualization. The flow pattern with different shear rate and gap to radius ratio are similar. They are all spiral shape propagate from the edge to the middle half of the plate. It suggest the flow pattern is roughly indifferent under the change of the rotation rate. The increased shear stress leads to faster motion of the flow.
 
 %Finally, three different sizes of the particles (52 $\mu$m, 30 $\mu$m, and 9.1$~\mu$m) with the same volume fraction 30 vol\% are tested. The secondary flow power spectrum in the three cases are very similar; see Fig. ?? in the SI. Both the amplitude of the torque fluctuation and the frequency of the promoted singe mode are alike. 
 
\section{Quantification of the phenomenon}
From the scaling analysis, we conclude the single mode originates from a flow pattern range across the entire flow field. To further quantify the flow field, we use fluorescent imaging and Particle Imaging Velocimetry (PIV) to measure the velocity field, then the optimized Dynamic Mode Decomposition (optDMD) method to extract time-invariant coherent structure of the velocity field. %in the secondary flow range across the entire flow field. So we want to further quantify the flow field mode change in the elastic flow.  So we use fluorescent imaging and Particle Imaging Velocimetry (PIV) to extract the flow field, and the Dynamic Mode Decomposition (DMD) method to extract time-invariant coherent structure of the velocity field. 

To track the motion of a dense suspension, a small fraction (0.25 vol\% ) of the suspension are labelled with fluorophore as tracers. The rest of the particles remain transparent. So we can distinguish the single particle motion under shear. See Fig 3.a for a typical frame after postprocessing. Under shear, particles still move predominantly in the azimuthal direction. But some radial motion can also be observed. The tracer density stay the same, no significant radical transport is observed. 
%To label the particles, PMMA particles are immersed in 0.5 wt\% Rhodamine B solution in isopropanol. PMMA particles swell in isopropanol, which allows Rhodamine B to diffuse into the particles. The particles are then immersed in water to deswell and the residual Rhodamine B are rinsed off. The light source is replaced with a 532nm laser diode to excite the fluorophore. The CMOS is installed with an additional 550 nm longpass filter to block the laser light. See Fig 3.a for a typical frame after postprocessing.

The fluorescent images are further analyzed with PIV to extract the velocity field; see Fig. \ref{fig:Fig2}a. The velocity field of the secondary flow keeps evolving over time and forms a time series. We are interested in the coherent modes behind the time series. Thus, we apply the optimized DMD method to extract these coherent modes. 

DMD method can be viewed as a combination of Principal Component Analysis (PCA) method, which extracts the major component of a vector field, and the Fourier Transformation (FT) method, which converts a time series to the frequency domain. In DMD, the time series is decomposed into a series of coherent modes. Each mode has an eigenvalue, which reflects the dynamics, an amplitude, which reflects the contribution to the time series, and an eigenvector, which reflects the mode shape. The Optimized DMD is a modified variant of DMD. It can avoid the bias of the eigenvalues when there is noise in the signal.

%The DMD method assumes the signal is composed of a series of coherent modes. In our case, b
In our DMD processing, we decompose the flow into 11 pairs of coherent modes. As a result, the relative L2 difference between the measured flow field and the reconstructed counterpart is less than 15\%; see Fig. \ref{fig:Fig2}b. The DMD is less effective when analyzing the secondary flow field without the suspension; see ?? in the SI. This is due to that there is no single mode dominating over the others.

The contributions of the first ten coherent modes are displayed in Fig. \ref{fig:Fig2}c. The first mode is the base flow, which is invariant over time. The odd and even number of the modes are complex conjugates. Their contributions are equal. 

In terms of the elastic turbulence in a viscoelastic fluid without suspension, the succeeding ten modes following the first mode have similar contributions in the same order of magnitude; see Fig. ?? in the SI for the diagram of the first ten modes). In contrast, the addition of the suspension dramatically changes the contribution landscape; see Fig. \ref{fig:Fig2}c. The first pair of secondary flow modes have much more contributions than the others. This observation suggests that the suspension suppresses higher-order modes and promotes one single mode. Also, the calculated frequency of the mode by DMD is 0.3 Hz, which matches the frequency peak in the torque spectrum (0.323 Hz). This agreement suggests that the fluctuation in the torque reading originates from the collective motion of the suspensions.

Though the secondary flow is very complex and typically does not have an analytical solution. The promoted mode in the secondary flow can be qualitatively explained. From the mode analysis (Fig. \ref{fig:Fig2}e), we could see a rigid body motion surrounded by the spiral pattern rotates over time (6 in the SI). Notice the solid core does not allow relative motions, and thus the spirals originating from the elastic turbulence cannot penetrate the rigid core, Instead, the velocity is roughly a constant in the middle. And the disk spins. 


\begin{figure*}
\centering
\includegraphics[width=1.0\textwidth]{Fig 3.pdf}
\caption{\label{fig:Fig2} Quantification of the elastic turbulence in polymer solution with 30\% suspension. a) The fluorescent imaging that overlays the velocity field extracted from PIV analysis. b) The reconstruction of the velocity field from optDMD method. The Blue line is the measured azimuthal direction velocity over time, averaged over the entire flow field. The orange line is the reconstructed value by DMD method. 
%The L2 difference between the reconstructed and the measured velocity field is 14\%. 
c) Comparison of the contribution of the first nine modes decomposed by optDMD method, with and without the presence of the suspension. The addition of the suspension increase the contribution of the base flow, which is the Mode 1. It also promotes one single mode over the others. The promoted mode is mode 2 \& 3. d) The flow field of the promoted single mode. The real part of the complex eigenmode is plotted here. The velocity field has little gradient near the center of the plate. The outer layer is composed of spirals.} 
\end{figure*}
\section{Discussion}

The rigid core rotation mode from DMD and scaling analysis both suggest that the emerged mode is not a local flow around one particle. Rather, the promoted mode emerges from the collective motion among a group of particles. To directly quantify such collective behavior, we use rheo-microscopy (MCR 702 with rheomicroscpe accessory from Anton Paar) to image the evolution of the micro-structure of the suspension under a shear flow in-situ. In this setup, the upper and lower plate counter rotates. The focal plane is set to be the middle plane between the two plates, where the flow stagnates. Therefore, the particles near the focal plane can stay in the field of view and be tracked even when the shear rate is relatively high. The images were taken after shearing for 60 seconds. The time of shearing is long enough that the torque reading remain static until 600 seconds when the test ends.

In our control experiment with 5\:vol\% particles, most of the particles assemble into short chains or small rafts composed of 3 -- 30 particles under shear flow; See Fig. \ref{fig:Fig3}b. We can still some unassembled single particles. The enough distance between chains allows relative motions without collusion. Such chaining of the particles is in line with previous literature. 

\begin{figure}
\centering
\includegraphics[width=0.5\textwidth]{Fig 4.pdf}
\caption{\label{fig:Fig4} Rheomicroscopic visualization of the particles self assemble under shear. a) 5\, vol\% suspension in viscoelastic liquid sheard by the large aptitude oscillation shear (LAOS) protocol, the angular velocity $\Omega=3.14 /\mathrm{s}$ and shear strain $ \gamma = 1$, the effective shear rate $\dot{\gamma}=3.14\,\mathrm{/s}$. b) The same formula sample sheared under $\Omega=3.14 /\mathrm{s}$ and shear strain $ \gamma = 30$. c)  same formula sample sheared under a constant shear rate $\dot{\gamma}=3.14\,\mathrm{/s}$.  d-f) Control experiments with 20\, vol\% suspension. Under large shear rate, the particle aggregate into flocs and phase seperated from the liquid. g-i) Control experiments with 30\, vol\% suspension. A further increase of the volume fraction increase in-plane volume fraction. }
\end{figure}

Surprisingly, when the volume fraction is increased to 20\%, the behavior of the particles is qualitatively different. Particles crystallize into two-dimensional (2D) closely packed rafts. There are still void space between rafts, but single particles are depleted. Under a large-amplitude oscillation (LAOS) shearing, the raft migrates and behaves like a rigid body. In each raft, there is no relative motion among the particles rather than rigid-body rotation; see Fig. \ref{fig:Fig4}, e). A further increase of the volume fraction will generate more rafts, and the phenomenon is qualitatively similar; see Fig. \ref{fig:Fig4}, h). 
  
We note that such crystallization is dramatically different from dense suspension immersed in a Newtonian liquid, where the minimum bulk volume fraction $\phi$ needed for the crystallization $ \phi \sim  54.5\%$. In a Newtonian liquid with more than 10~vol\% mono-dispersed particles, the collusion between particles become non-negligible. And shear flow may also induce migration and heterogeneity in the particle distribution\cite{abbott_experimental_1991,sarabian_fully_2019}. But such migration do not deplete single particles. Or form closely packed structures.

The crystallization can be understood by two aspects: first, the viscoelastic shearing will lead to heterogeneity in the $z$ (height) direction. Near the focal plane, the local particle volume fraction is measured to be 65 $\pm$ ??\%, which is increased more than two times over the bulk volume fraction, 20\%. The difference in the volume fractions suggests that the viscoelastic shearing leads to the migration of the suspension towards the middle plane.

Second, the viscoelastic shear flow will generate attractive force between particles. The attraction emerge from the normal stress effect between local flow field around each single particle. Such attractive interaction assemble particles in plane. And eventually aggregate into two dimensional (2D) rafts. 

%interactions. Here we found with a further increase of the volume fraction beyond 10\%, the physics further changes, the attractive force is large enough for crystallization of the suspension and deplete the suspension in the other regimes.

%The assemble of the particles generate collective motion and deplete single particles. Thus the size of the single particle has little influence towards the secondary flow, only the volume fraction of the suspension play a role.

\section{Validation}
To directly prove that the crystallization of the particles causes the change in the secondary flow, we further design a control experiment by mixing different sizes of the particles (10\% CA10, 10\% CA30, and 10\% CA50). Polydispersed particles cannot crystallize but only jam when the volume fraction is high. As a result, indeed, by shearing the polydispersed suspension under LAOS test, no ordered structure is observed, even if the in-plane area fraction is much higher than 30\%. This observation suggests that the particles still migrate towards the middle plane and self assemble under the shear flow. However, the difference in particle sizes blocks the crystallization (Fig.\ref{fig:Fig5} d-f).
\begin{figure*}
\centering
\includegraphics[width=0.5\textwidth]{Fig 5.pdf}
\caption{\label{fig:Fig5}Polydispersed flow visualization}
\end{figure*}


We also repeat the rheoscopic visualization and opt-DMD based mode analysis of the polydispersed sample. Compared with the no suspension case, the rheoscopic flow pattern here is indistinct (Fig.\ref{fig:Fig5}.a), which suggest the secondary flow is suppressed by the suspension. Multiple patterns coexist, but no predominant mode can be distinguished. Reflected on the rheometer reading, there is no single frequency peak in the power spectrum (Fig.\ref{fig:Fig5}.b).

In the DMD based mode analysis, in line with the rheoscopic flow visualization, there is no predominant mode in the mode analysis. The contributions of the first ten modes of the secondary flows now have roughly the same order of magnitude again. Also, most of the modes belong to the spiral family; see 7 in the SI-7. Overall, this control experiment explains the physical origin of the spinning mode is the crystallization of the suspension.

\section{Conclusion}

In this experiment, we demonstrated the unexpected synergy between non-Brownian suspension and the elastic instability. We firstly reported the fact non-Brownian suspension with finite volume fraction has strong regulation effect towards the elastic instability. The secondary flow is suppressed, while a novel predominant mode emerges. 

We further revealed the scaling of the regulation effect, the dimensionless frequency scales with the Deborah number.
Then we validated the hypothesis the that such mode originate from the collective rigid body rotation motion of the particles. 

Then we used rheomicroscope to visualize the flow induced crystallization of the suspension, the minimum volume fraction for phase seperation is down to less than 20 vol\%. 

At the end we directly prove the causality between the crystallization and the novel mode by repeat the tests with a polydispersed suspension. The polydispersity blocks crystallization and thus extinguish the single frequency mode.

For future study, so far we only tested the parallel plate geometry, we expect such flow induced phase separation may also be observed in other flow configurations like Cone and plate, pipe flow or microfluidic devices. Also here the Brownian motion and inertial effect are eliminated, it could be interesting to consider these factors.
