\newpage
\section*{Supplementary Material}
\label{AppendixA}
\section*{Details of sample preparation and experimental setup}
%\setcounter{figure}{0}
%\renewcommand{\figurename}{Fig. S}

We prepare a polymer solution by mixing 25.814g of DI water with 141.765g of 2-2 thiodiethanol (TDE) and storing it in a -20°C freezer to prevent clumping of the undissolved polymers. We then add 0.3850g of PEO with a molecular weight of $8\times10^6$ Da to the mixture and heat it in a 65°C oven for 4 hours to dissolve the polymers. We then homogenize the mixture by gently shaking it with a mechanical shaker for 30 minutes. The resulting polymer solution has a density of 1.18 g/ml at 20°C. To form the suspension, we add 68.2g of surfactant-free PMMA particles (CA 50 from MICROBEADS) with a density of 1.19 g/ml to the polymer solution, resulting in a final volume fraction of 30\%. The particle size standard deviation is less than 1 \textmu m, according to the manufacturer. For experiments without particles, we use a polymer solution of equal weight to replace the particles.

For the Particle Imaging Velocimetry (PIV), 0.83 wt\% of the particles (0.25 wt\% of the total suspension) are replaced with fluorescent particles.  To ensure the fluorescent particles have the same physical property as other nonfluorescent particles and strong fluorescence to be detected with a commercial camera, we exploit the polymer swelling to introduce Rhodamine B dye to the CA50 particles. The particles are firstly immersed in a 10 g/L Rhodamine B isopropanol solution and heated in  a 65 °C oven for 24 hours to swell the particles and let the dye molecules diffuse into the particles. We then rinse the particles with room temperature deionized water five times to remove the residual dye in the solution and let the particles deswell. The dye is then trapped in the particles and emits strong fluorescent light. 

To carry out  PIV experiments, we outfit the MCR 501 rheometer with customized optics for flow visualization. Our design is inspired by previous rheometric flow visualization studies ~\cite{byars1994spiral,pommella2019coupling}. We replace the lower plate of the rheometer with optical-clear laser-cut 6.35 mm thick plexiglass (8560K354 from Mcmaster-Carr). The 50 mm diameter upper plate (\# 12081 from Anton Paar) is attached with an OD4 absorptive Neutral Density filter (\# 36-276 from Edmund Optics) to eliminate undesired background light. The top and bottom plates were carefully aligned, with a height difference of less than 0.05mm across the plates, as measured with a feeler gauge. We add a 75 mm diameter cup to the lower plate. 200 ml sample will fill the cup up to around 45mm. In the torsional shear flow, the velocity gradient is inversely proportional to the gap size, so the velocity gradient between the upper and lower plate with a 2 mm gap is at least ten times larger than the gradient between the upper plate and the top free surface, whose gap size is around 40 mm. It ensures the torque reading is mainly contributed by flow beneath the upper plate, and the flow above the upper plate remains laminar. 


\section*{Optical Imaging}

We use a table lamp (NÄVLINGE from IKEA) as the light source for rheoscopic visualization. For fluorescent imaging, we choose a 505 nm wavelength, 220 mW  collimated LED (M505L4-C1 from Thorlabs) as the light source. The green light beam is further expanded by a light diffuser (ED1-C20-MD from Thorlabs) to a 50 mm diameter. 

We mount an 1/1.8 inch C-mount camera (BFS-U3-32S4M-C from FLIR) with 35mm focal length lens with f-number $f_\mathrm{num} = 1.4$ (MVL35M1 from Thorlabs) for imaging. A 550 nm long-pass filter (FEL0550 from Thorlabs) is attached to the lens. It can capture the flow field with approximately 700 px by 700 px resolution in 200 fps under pixel binning mode. The actual sensor size of the camera is a square with a side length of about $l_i =5$ mm, and the pixel size is 7 {\textmu}m. The magnification ratio $M=l_i/{2R}=0.1$. If we consider the pixel size as the diameter of the circle of confusion $d_c=7$  \textmu m, the depth of field (DoF) can be estimated as 
$$DoF \approx \frac{2f_\mathrm{num} d_c}{M^2}=\frac{2\times1.4\times0.007}{0.1^2}=2.2~\mathrm{mm},$$

which is comparable with the gap height $H = 2$ mm. Therefore, all the particles are in focus.


\section*{Details about PIV and DMD decomposition}

\subsection{Details about velocity measurement}

For the flow field measurement, we record particle images at 200 fps after the torque reading from the rheometer reaches statistically stationary, which is after about 60 seconds. {Because of the directional migration of particles towards the middle plane in viscoelastic flow, the velocity is measured in the middle plane of the flow cell.} The raw image is processed with the Matlab PIVLab plugin. We apply the CLAHE filter with a 64 px by 64 px window size and a high-pass filter with a 15 px kernel size as a pre-processing step to enhance the contrast. We obtain the velocity vectors by using iterative 2D cross-correlations with multiple sizes of interrogation windows. We use 50\% overlapped 64 × 64 pixels for the coarse grid and 50\% overlapped 32 × 32 pixels for the fine grid. We also check the cross-correlation map, where we find the signal-to-noise (SNR) ratio is about 2.
\subsection{Error Estimation}

{We utilize the property of incompressible flow to estimate the error in our velocity measurements\cite{adrian2011particle,kim2011full}. Assuming the measured flow field at the midplane, when reaching statistical equilibrium, is a 2D flow; the net flow in the z (height) direction is zero. Consequently, $\nabla \cdot \mathbf{u} = 0$, where $\mathbf{u}=(u_x,u_y)$, where $u_x$ is the x directional flow velocity and $u_y$ is the y directional flow velocity. Guided by this principle, we can provide an order-of-magnitude estimate of the velocity measurement error as follows:
$$Residual~=~\Delta t \nabla \cdot \mathbf{u} $$;
the time interval $\Delta t = 0.005~\si{s}$ for 200 fps measurement. We calculate the residual distribution per point per frame for the polymer solution flow and the viscoelastic suspension flow, as seen in Fig. \ref{fig:A16}. The relative error can be estimated from the standard deviation of the residual $\sigma_r=\mathrm{std}(Residual)$, where $\sigma_r = 1.3\%$ for the polymer solution and $\sigma_r = 0.7\%$ for the viscoelastic suspension. 
}
\begin{figure*}
\centering
\includegraphics[width=6 in]{figures/a-16.pdf}
\caption{The distribution of the residual in two velocity measurements. The standard deviation of the left side is 0.0134 pixel/pixel, and the right side is 0.0070 pixel/pixel. Both values correspond to less than a 1-pixel measurement error, suggesting our measurement is reasonably accurate.}
\label{fig:A16}
\end{figure*}

\subsection{optDMD decomposition}

We then treat the time-dependent velocity field as a time series and process it with the optDMD method. If a coherent mode is not stationary, like those belonging to the secondary flow, then the eigenvector has an imaginary part. And the complex conjugate of the mode is also a coherent mode, resulting in two coherent modes per coherent secondary flow structure. This fact drives us to choose odd-number modes to decompose the flow. And combine the physically identical complex conjugated modes afterward. The relative error decreases with an increase in the number of modes used in the decomposition until the number of modes $n_m = 21$, or 1 primary flow plus 10 secondary flow structures. Adding more modes does not provide significant additional benefits. And We cut off at $n_m = 21$.

The argument that the viscoelastic suspension flow has a predominant coherent motion can be cross-validated by other postprocessing methods, like the Singular Value Decomposition (SVD). The SVD is a deterministic method and has no fitting parameter. If we sort the SVD  modes of the viscoelastic suspension flow by their importance, defined by the squared singular value, we can also find a pair of modes that dominate over the others, as seen in Fig. \ref{fig:A12}. And its  shape or eigenvector is the same as the solution of the optDMD method. 
\begin{figure*}
\centering
\includegraphics[width=3 in]{figures/a-12.pdf}
\caption{The squared singular value of the first 21 modes in the SVD decomposition of the suspension flow. Mode 2 \& 3 dominate over other modes in the secondary flow, in line with our finding with DMD decomposition}
\label{fig:A12}
\end{figure*}
\section*{Details of Rheo-microscopy}
To visualize the sample in the flow geometry {\it in situ}, we use the MCR 702 rheometer from Anton Paar with a counter-rotating concentric plate-plate geometry. The transparent upper and lower plates have a radius of 21.5 mm. Using the high-speed camera set at 100 fps, we visualize the local flow approximately 14 mm from the center, roughly 2/3 of the radius. The local shear rate, where the flow is observed, is nearly equivalent to the nominal shear rate in the parallel plate geometry, as determined by the software input. To minimize scattering from the dense suspension, we set the gap to 0.2 mm. The sample is illuminated with a white light lamp through a pinhole to enhance the contrast. We use a high-speed camera (Phantom V 7.3) with a pixel size of 22 \textmu m and a resolution of 800 px by 600 px { and 100 FPS. We find 100 FPS  delivers shape image of the sample at the end of each cycle} We mount the camera with a 5X objective (Mitutoyo), leading to a rectangle field of view of 3.43 mm by 2.57 mm. In postprocessing, we normalize the illumination and enhance the contrast to better quantify the particles. The surface fraction of the particles is determined through manual counting. 




\section*{Onset of the elastic instability}
The stability of the flow in a viscoelastic fluid is dependent on the ratio of elastic stress to viscous stress, which is governed by the shear rate and the relaxation time of the material. Additionally, the stability also depends on  the geometry of the flow cell. Previous studies conclude that the onset of the elastic instability in torsional shear flow can be described by nondimensional criteria $M_0$~\cite{mckinley1996rheological}.
$$\tau \dot{\gamma}_\mathrm{crit} \sqrt{H/R} > M_0 ,$$

where $\tau$ is the relaxation time of the material, $R/H$ is the radius-to-height ratio of the flow cell,  $\dot{\gamma}_\mathrm{crit}$ is the critical shear rate for flow instability and $M_0$ is a constant. We compare our results with the prediction. Our findings show that, for measurements with varying gap heights, $\dot{\gamma}_\mathrm{crit}$, where shear thickening first occur in viscosity measurement, scales with $\sqrt{R/H}$ with a gentle upward deviation, or the product $\dot{\gamma}_\mathrm{crit}\sqrt{H/R}$ remains roughly constant at the onset of elastic instability, as depicted in Fig.~\ref{fig:A1}. This reconfirms the validity of the $M_0$ criteria. The upward deviation observed may be attributed to the decrease in the relaxation time of the polymer solution as the shear rate increases.

Interestingly, when we add particles to the polymer solution. The critical shear rate $\dot{\gamma}_\mathrm{crit}$ where the shear thickening first occurs becomes independent of the flow geometry, as shown in Fig.~\ref{fig:A1}. Such change is because of the nature of the suspension flow, the suspension has two different inherent length scales: the length scale of each particle and the length scale of the entire flow geometry. The elastic instability can develop in both length scales, as the streamline is curved in both cases. 

Our data suggest that the flow change first occurs at each particle scale, generating additional dissipation and causing shear thickening. Similar observations have been reported in previous studies~\cite{yang_mechanism_2018}. Further increasing the shear rate, the instability eventually develops in the flow geometry scale. 

{ We can also rethink this question dimensionlessly. An important dimensionless number in viscoelastic flow is the Weissenberg number: $Wi = \dot{\gamma}\tau$, which characterizes the elastic stress to viscous stress ratio locally at each point. The $M_0$ criterion can be rewritten as $M_0=Wi\sqrt{H/R}$. Notice $Wi$ is scale-independent, and the local configuration of the flow channels between particles can be versatile. Thus, the instability can initialize at the particle scale under the most favorable condition, and $\sqrt{H/R}$ is irrelevant in this case. Thus, at the particle level, the $M_0$ criteria degrade to the $Wi$ criteria. The critical Weissenberg number in our case is $ Wi \approx 6$.}
\begin{figure*}[h]
\centering
\includegraphics[width=0.5\textwidth]{figures/a-1.png}
\caption[Influence of the particle suspension on the onset of Elastic instability]{\label{fig:a1} The onset of Elastic instability. The square root of the radius $R$ to height $H$ ratio is the x-axis, and the instability onset shear rate is the y-axis. The black reference line is the theoretic prediction from McKinley, Oztekin 1996~\cite{mckinley1996rheological}. With the suspension, the critical shear rate for elastic instability is independent of the geometry.} 
\label{fig:A1}

\end{figure*}




{\section*{Deborah number and the scaling of the dominant frequency}}
{ The Weissenberg number, $Wi = \dot{\gamma}\tau$, is a dimensionless number that is defined based on the stress ratio as a local property. The Deborah number, $De=\Omega\tau$, is a second dimensionless number important for viscoelastic flow that is defined at the entire flow cell level. Here, $\Omega$ is the angular velocity of the upper plate. In the torsional shear flow between two parallel plates,$\dot{\gamma}=\frac{2R}{3H}\Omega \propto \Omega$, so $Wi \propto De$. However, we can decouple these two dimensionless numbers by alternating the height $H$ or radius $R$ of the flow cell.
} 

{We examine the scaling of the dominant frequency $f_0$ by repeating the experiment across a wide range of experimental conditions, including different shear rates, different particle sizes, different volume fractions, different geometry radii, and different gap heights. We find that the dimensionless frequency $\tilde{f}=f_0\tau$ is dictated by $De$ with a near-linear relationship, as depicted in Figure~\ref{fig:A14}. This dependence contrasts sharply with the onset of the shear thickening, which is dictated by $ Wi$. }

{The transition from $Wi$-dominant at the onset to $De$-dominant at the fully developed secondary flow manifests that the particles suspended in a viscoelastic fluid crystallize into an assembly under strong shearing. The scale of the assembly is comparable to the flow cell, and the single particles are depleted in the system, making the Weissenberg number less relevant when considering the fully developed secondary flow.}

\begin{figure*}[h]
\centering
\includegraphics[width=0.75\textwidth]{figures/a-14.pdf}
\caption{Scaling of the dimensionless peak frequency $\tilde{f}=f_0\tau$ as a function of $De$, we approximate the relaxation time $\tau = 0.5~\si{s}$ based on the $N_1$ measurement stated later. The data is measured with a DHR-3 rheometer and 40 mm and 60 mm parallel plate geometry. Here we find the $\tilde{f}$ scales near linearly with $De$ with a gentle decline.} 
\label{fig:A14}

\end{figure*}

{\section*{Robustness of the measurement}}
{ In the working condition of most of the measurements, the secondary flow is fully developed, and we can observe similar flow behavior if we further increase the shear rate. The highest shear rate accessible with our current setup is 200/s, and we observe similar single-frequency torque oscillation following the same scaling rule in Fig.~\ref{fig:A14}. We also repeat the rheoscopic visualization under different conditions, where we find similar results, as depicted in Fig.~\ref{fig:A15}.}

\begin{figure*}[h]
\centering
\includegraphics[width=0.75\textwidth]{figures/a-15.pdf}
\caption{Viscoelastic flow visualization under different conditions. Here 30 vol\% viscoelastic suspension is tested with the MCR 501 rheometer, similar to Fig 1.E. A)  Rheoscopic visualization of the secondary flow under condition $\dot{\gamma}=50~\mathrm{/s}$ and $H = 2~\mathrm{mm}$. B) Control experiment with $\dot{\gamma}=100~\mathrm{/s}$ and $H = 2~\mathrm{mm}$. C) Control experiment with $\dot{\gamma}=100~\mathrm{/s}$ and $H = 1~\mathrm{mm}$,} 
\label{fig:A15}

\end{figure*}

\section*{Rheology of the material}
\subsection{Relaxation Time}

When the dispersed long-chain polymer deforms under shear, it takes time to recover, and this time scale is the physical origin of the relaxation time of the polymer solution. In practice, the polymer chain usually has different deformation modes, and each mode can have its own relaxation time, making it difficult to define a single material constant $\tau$ as the relaxation time for the polymer solution. Here we report two different methods to determine the relaxation time. 

In the first method, we measure the relaxation time of the polymer solution with the stress relaxation test. Here we apply a fixed amount of strain to the polymer solution and record the stress relaxation process with the Discovery HR-3 rheometer from the TA instruments. The stress relaxation of a linear viscoelastic fluid (Maxwell fluid) should be:
$$\sigma(t)=\sigma_0 \exp{(-\frac{t}{\tau})},$$
where $\sigma_0$ is the applied shear stress at $t = 0~\si{s}$, $\sigma(t)$ is the measured shear stress as a function of time $t$. Thus we can find $\tau$ by the following linear fitting:
$$\log{\sigma(t)}=\log{\sigma_0}-\frac{t}{\tau},$$
In the experiment, we find that the logarithmic stress does not linearly relax over time, but we can still obtain a single relaxation time by fitting the stress relaxation process from 0.1 s to 1.0 s,{ whose reciprocal spans from 1/s to 10/s, comparable to the shear rate range used in $N_1$ measurement}. The result is about $\tau =1.0 ~\si{s}$.

In the second method, we focus on the axial force $F_\mathrm{axial}$ measured by the DHR-3 rheometer with a 60 mm 2.0° cone-and-plate geometry. The measured axial force from the rheometer upper plate can be exploited to calculate the first normal stress difference

$$N_1=\frac{2F_\mathrm{axial}}{\pi R^2},$$

The first normal stress difference $N_1$ is again the manifestation of the elasticity of the dispersed long-chain polymer. And elastic instability is a normal stress effect. So the relaxation time  calculated based on $N_1$ is more relevant to elastic instability. For a viscoelastic material, the relaxation time can be calculated as follows:

$$\tau=\frac{N_1}{2\eta\dot{\gamma}^2},$$

Because of shear thinning and other effects, the measured viscosity $\eta$ and $N_1$ both depend on the shear rate $\dot{\gamma}$. So the relaxation time calculated based on this formula is also a function of the shear rate. And the data at $\dot{\gamma}=50~\si{s}^{-1}$ is inaccessible because the shear flow is unstable under such a high shear rate. Nevertheless, we can still obtain a relaxation time based on the rheometric flow at $\dot{\gamma}=10~\si{s}^{-1}$. The relaxation time $\tau = 0.64$ s for the polymer solution.

Upon the addition of the particles, the measured relaxation time from both methods decreases. The stress relax faster with the addition of the particles. The relaxation time with linear regression decreases from 1.0 second to 0.8 seconds. And the measured relaxation time based on the normal stress decreases from 0.64 to 0.49 sec at shear rate $\dot{\gamma}=10~\si{s}^{-1}$. See the figures for more details.
\begin{figure*}
\centering
\includegraphics[width=1\textwidth]{figures/a-9.pdf}
\caption{ Stress relaxation test to determine the relaxation time of the system. The left side is the polymer solution that relaxes its stress over time under different shear strain loadings; the stress relaxation from 0.1 s to 1 s is linearly fitted to determine the relaxation time, highlighted as the black line here, { the stress relaxation over three different applied strain exhibit similar rate}. The fitted slope is 1.0 s for all three cases. The right side is the same experiment with a polymer solution containing 30 vol\% particles. The relaxation time decreases from 1.0 second to 0.8 second.   }
\label{fig:A9}
\end{figure*}

\begin{figure*}
\centering
\includegraphics[width=1\textwidth]{figures/a-10.pdf}
\caption{ First normal stress difference $N_1$ divided by the shear stress $\sigma_{12}$ and the shear rate $\dot{\gamma}$ as a function of of $\dot{\gamma}$. The left side is the data from the polymer solution without particles, the right side is the data measured with a polymer solution and 30 vol\% 9 \textmu m diameter particles. The data is collected with a DHR-3 rheometer with a 60 mm 2.0° cone-and-plate geometry. The truncation gap size is 59 \textmu m, so we use 9 \textmu m diameter particles here. The polymer solution has a relaxation time $\tau=\frac{N_1}{2\eta\dot{\gamma}^2} = 0.64$ s at $\dot{\gamma} = 10/$s. And the polymer solution with 30 vol\% particles has a relaxation time $\tau=\frac{N_1}{2\eta\dot{\gamma}^2} = 0.49$ s  at $\dot{\gamma} = 10/$s}
\label{fig:A8}
\end{figure*}


\subsection{Viscosity measurement}

We measure the viscosity as a function of the shear rate for both polymer solution and polymer solution with 30 vol\% suspensions. For the polymer solution, we use a 60 mm cone-and-plate geometry on a DHR-3 rheometer to measure the viscosity and the normal stress as a function of the shear rate. We find the viscosity exhibit shear thinning  above $\dot{\gamma}\sim 0.1~\si{s}^{-1}$. The viscosity is about $1.1$ Pa.s when the shear rate $\dot{\gamma} = 1~\si{s}^{-1}$.The measured viscosity keeps decreasing until the onset of the elastic instability.

We also measured the viscosity of the polymer solution with 30 vol\% 9 \textmu m diameter particles with the same setup. The addition of the particles increases the measured viscosity from $\eta = 1.1$ Pa.s to $\eta = 2.8$ Pa.s under $\dot{\gamma} = 1~\si{s}^{-1}$.


%We also test samples with both the shear rate ramping up and ramping down. It turns out the elastic secondary flow has hysteresis. The critical shear rate for the secondary flow to vanish during ramping down is lower than the onset of the instability during ramping up; Such hysteresis is eliminated by the particles. See Fig. \ref{fig:A2}
\begin{figure*}
\centering
\includegraphics[width=1\textwidth]{figures/a-8.pdf}
\caption{Viscosity measurement of the polymer solution and polymer solution with 30 vol\% 9 \textmu m diameter particles, measured with a 2° 60 mm cone-and-plate geometry on the DHR-3 rheometer from TA instrument. The left side is the polymer solution with no particles; The right side is the polymer solution with 30 vol\% particles}
\label{fig:A8}
\end{figure*}

\begin{figure*}
\centering
\includegraphics[width=1\textwidth]{figures/a-17.pdf}
\caption{Shear rate ramp of a viscoelastic fluid and the same fluid with 30 vol\% particles from 10/s to 50/s over 800 s in an immersion flow cell. The setup is similar to that of Figure 1. In the viscoelastic fluid (left), the torque begins to fluctuate immediately after shear thickening occurs, suggesting that the instability develops at the flow cell level. In the viscoelastic suspension (right), shear thickening happens before the torque fluctuation begins, suggesting that the flow instability develops from the particle level and eventually develops at the flow cell level.}
\label{fig:A8}
\end{figure*}



\subsection{{The size-dependence} and surface effect of the particles}

The diameter of the particle we choose in this study ranges from 10 ~\textmu m to 51 ~\textmu m. We focus on the hydrodynamic interactions between the suspended particles. To verify that they dominate over other surface interactions, {and the size of the particle is less relevant, }we test particle suspension with the same 30 vol\% but different particle sizes, resulting in different surface-to-volume ratios. In the torque measurement, we find very similar power spectra, suggesting other surface interactions are less relevant in this study, as shown in Fig. \ref{fig:A11}. 
\section*{Additional explanation towards particle assembly in viscoelastic flow}

The assembly of particles into 2D aggregates is observed in both steady shearing and oscillatory shearing, as seen in SI Video 6 and SI Video 9. This assembly can be contextualized by previous studies in the field.\cite{becker1996sedimentation,d2015particle,feng1996motion,murch_collective_2020,won2004alignment,scirocco_effect_2004,xie2016flow,pasquino2010directed}

Previous studies have shown that spheres in viscoelastic fluids experience a force perpendicular to a wall, which causes them to move away from the wall. Single or multiple particles can migrate away from the wall in this way. In a torsional flow cell, the presence of two walls causes the particles to migrate toward the midplane.

Furthermore, viscoelastic fluid flow generates an attractive interaction between particles. The major factor is the normal stress perpendicular to the streamlines exerted by long-chain polymers. As two particles approach each other and their local streamlines overlap, this normal stress leads to a net force on each particle, pushing the particles toward each other. 

Considering these two factors, we argue that the viscoelasticity of the fluid drives particle assembly under shear. Evidently, no directional migration nor attractive interactions are observed in Newtonian suspension under similar conditions.
\section*{SI Videos}

\textbf{SI Video 1 (separate file)} - Rheoscopic visualization of Polymer Solution Flow: This video demonstrates the flow of a polymer solution visualized using rheoscopic techniques. The video is in real-time, and the diameter of the circle is 50 mm.

\textbf{SI Video 2 (separate file)} - Rheoscopic visualization of Viscoelastic Suspension Flow: This video demonstrates the flow of a viscoelastic suspension visualized using rheoscopic visualization methods. The video is in real time, and the diameter of the circle is 50 mm.

\textbf{SI Video 3 (separate file)} - Polymer Solution Velocity Field: This video displays the velocity field of a polymer solution  under Shear Rate = 50/s. The velocity is calculated per 32 px by 32 px square with a 16 px step size.

\textbf{SI Video 4 (separate file)} - Viscoelastic Suspension Velocity Field: This video displays the velocity field of 30 vol\% viscoelastic suspensions under Shear Rate = 50/s. The velocity is calculated per 32 px by 32 px square with a 16 px step size.

\textbf{SI Video 5 (separate file)} - Structure 2 of the viscoelastic suspension flow: This video illustrates the flow structure 2 of a viscoelastic suspension decomposed by the optDMD method in real time.

\textbf{SI Video 6 (separate file)} - 30 vol\% particles assemble under shear: This video shows the assembly of 30 vol\% particles under shear in a viscoelastic suspension. Each pixel is 4.4 micrometers; the shear rate is 100/s; { the gap height is 0.2 mm}, and the video is 10x slower than the real-time. { Upon initiating the test, the motor of the rheometer induces a noticeable vibration in the image.}

\textbf{SI Video 7 (separate file).} - 5 vol percent suspension under LAOS cycle: This video presents the behavior of a 5 vol\% suspension under large amplitude oscillatory shear (LAOS) cycles. Each pixel is 4.4 micrometers, the strain amplitude is 3000\% and the oscillation frequency is 1 Hz. 

\textbf{SI Video 8 (separate file)} - 20 vol percent suspension under LAOS cycle: This video presents the behavior of a 20 vol\% suspension under large amplitude oscillatory shear (LAOS) cycles. Each pixel is 4.4 micrometers, the strain amplitude is 3000\% and the oscillation frequency is 1 Hz.

\textbf{SI Video 9 (separate file)} - 30 vol percent suspension under LAOS cycle: This video presents the behavior of a 30 vol\% suspension under large amplitude oscillatory shear (LAOS) cycles. Each pixel is 4.4 micrometers, the strain amplitude is 3000\% and the oscillation frequency is 1 Hz.

\textbf{SI Video 10 (separate file)} - polydisperse suspension under LAOS cycle: This video presents the behavior of a polydisperse suspension subjected to large amplitude oscillatory shear (LAOS) cycles. Each pixel is 4.4 micrometers, the strain amplitude is 3000\% and the oscillation frequency is 1 Hz.

\begin{figure*}
\centering
\includegraphics[width=1\textwidth]{figures/a-11.png}
\caption[The influence of the particle size on the secondary flow,]{\label{fig:a4} The influence of the particle size on the secondary flow, the power spectrum is the same with different particle sizes (51 {\textmu}m on the left, 30 {\textmu}m in the middle and 9 {\textmu}m on the right). The shear rate here is 35 /s and the gap size is 2 mm with a 60 mm plate-plate geometry, measured with a DHR-3 rheometer} \label{fig:A11}
\end{figure*}



\begin{figure*}
\centering
\includegraphics[width=3 in]{figures/A-4.pdf}
\caption{The formation of suspended particles after around 2 seconds shearing at a shear rate $= 100\si{s^{-1}}$. The particles begin to {assemble into crystals. And the flow is unstable under this condition} .}
\label{fig:A3}
\end{figure*}
%\begin{figure*}
%\centering
%\includegraphics[width=1\textwidth]{figures/a-3.png}
% \label{fig:A3}
%\end{figure*}



%\begin{figure*}
%\centering
%\includegraphics[width=0.5\textwidth]{figures/a-5.png}
%\caption{\label{fig:a5} The reconstructed elastic secondary flow field without particles from the optDMD method. The x directional velocity averaged over the entire flow geometry is plotted versus time. The optDMD captures most of the flow behavior with certain errors.} 

%\end{figure*}

%\begin{figure*}
%\centering
%\includegraphics[width=1\textwidth]{figures/a-6.pdf}
%\caption{\label{fig:a6} The first three modes of the polymer solution without suspension. The first mode is the primary flow. And the second and third modes belongs to the spiral family} 

%\end{figure*}

\begin{figure*}
\centering
\includegraphics[width=7 in]{figures/a-13.pdf}
\caption[The first three coherent structures of viscoelastic flow with the polydispersed suspension]{\label{fig:a7} The first three coherent structures of viscoelastic flow with the polydispersed suspension. $\lambda$ here is the associated eigenvalue and ~$\tau$  ~is the oscillatory period. The Left is the real part of the structure, and the right side is the imaginary part of the structure.} 

\end{figure*}

