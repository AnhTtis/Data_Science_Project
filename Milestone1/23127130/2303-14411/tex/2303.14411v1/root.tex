\documentclass[10pt,twocolumn,letterpaper]{article}

\usepackage{iccv}
\pdfoutput=1
\makeatletter
\@namedef{ver@everyshi.sty}{}
\makeatother
\usepackage{tikz}


\usepackage{times}
\usepackage{epsfig}
\usepackage{graphicx}
\usepackage{amsmath}
\usepackage{amssymb}

% Include other packages here, before hyperref.
\usepackage{booktabs}


\usepackage{makecell}
\usepackage{multirow}
\usepackage{titling}

\newcommand*{\tat}{\textcolor{purple}}
\renewcommand{\Pr}{\field{P}}

\usepackage{MnSymbol}
\usepackage{comment}

\DeclareMathOperator{\Tr}{Tr}
\newcommand{\bA}{\boldsymbol{A}}
\newcommand{\ba}{\boldsymbol{a}}
\newcommand{\bx}{\boldsymbol{x}}
\newcommand{\bxi}{\boldsymbol{\xi}}
\newcommand{\bc}{\boldsymbol{c}}
\newcommand{\bC}{\boldsymbol{C}}
\newcommand{\bq}{\boldsymbol{q}}
\newcommand{\bd}{\boldsymbol{d}}
\newcommand{\bX}{\boldsymbol{X}}
\newcommand{\bM}{\boldsymbol{M}}
\newcommand{\bone}{\boldsymbol{1}}
\newcommand{\bI}{\boldsymbol{I}}
\newcommand{\bu}{\boldsymbol{u}}
\newcommand{\bb}{\boldsymbol{b}}
\newcommand{\by}{\boldsymbol{y}}
\newcommand{\bY}{\boldsymbol{Y}}
\newcommand{\bhatY}{\boldsymbol{\hat{Y}}}
\newcommand{\bbary}{\boldsymbol{\bar{y}}}
\newcommand{\bg}{\boldsymbol{g}}
\newcommand{\bz}{\boldsymbol{z}}
\newcommand{\bZ}{\boldsymbol{Z}}
\newcommand{\bbarZ}{\boldsymbol{\bar{Z}}}
\newcommand{\bbarz}{\boldsymbol{\bar{z}}}
\newcommand{\bhatZ}{\boldsymbol{\hat{Z}}}
\newcommand{\bhatz}{\boldsymbol{\hat{z}}}
\newcommand{\bhatx}{\boldsymbol{\hat{x}}}
\newcommand{\haty}{\hat{y}}
\newcommand{\barz}{\bar{z}}
\newcommand{\bS}{\boldsymbol{S}}
\newcommand{\bbarS}{\boldsymbol{\bar{S}}}
\newcommand{\bw}{\boldsymbol{w}}
\newcommand{\bhatw}{\hat{\boldsymbol{w}}}
\newcommand{\bW}{\boldsymbol{W}}
\newcommand{\bU}{\boldsymbol{U}}
\newcommand{\bv}{\boldsymbol{v}}
\newcommand{\bzero}{\boldsymbol{0}}
\newcommand{\balpha}{\boldsymbol{\alpha}}
\newcommand{\sA}{\mathcal{A}}
\newcommand{\sC}{\mathcal{C}}
\newcommand{\sD}{\mathcal{D}}
\newcommand{\sX}{\mathcal{X}}
\newcommand{\sY}{\mathcal{Y}}
\newcommand{\sS}{\mathcal{S}}
\newcommand{\sT}{\mathcal{T}}
\newcommand{\sZ}{\mathcal{Z}}
\newcommand{\sL}{\mathcal{L}}
\newcommand{\sI}{\mathcal{I}}
\newcommand{\bbO}{\mathbb{O}}
\newcommand{\sbarZ}{\bar{\mathcal{Z}}}
\newcommand{\fbag}{\bold{F}}


\usepackage{xcolor,colortbl}
\newcommand{\mc}[2]{\multicolumn{#1}{c}{#2}}
\definecolor{Gray}{gray}{0.85}
\definecolor{LightCyan}{rgb}{0.88,1,1}
\newcolumntype{a}{>{\columncolor{Gray}}c}
\newcolumntype{b}{>{\columncolor{white}}c}
\usepackage{nicematrix}



% If you comment hyperref and then uncomment it, you should delete
% egpaper.aux before re-running latex.  (Or just hit 'q' on the first latex
% run, let it finish, and you should be clear).
\usepackage[pagebackref=true,breaklinks=true,letterpaper=true,colorlinks,bookmarks=false]{hyperref}

\iccvfinalcopy % *** Uncomment this line for the final submission

\def\iccvPaperID{****} % *** Enter the ICCV Paper ID here
\def\httilde{\mbox{\tt\raisebox{-.5ex}{\symbol{126}}}}

% Pages are numbered in submission mode, and unnumbered in camera-ready
\ificcvfinal\pagestyle{empty}\fi

\begin{document}

%%%%%%%%% TITLE
\title{\vspace{-1.02cm}
\textbf{Fairness meets Cross-Domain Learning: \\
a new perspective on Models and Metrics}}

\author{
Leonardo Iurada$^{1}$ \hspace{0.5cm} Silvia Bucci$^{1}$\thanks{Work mainly developed during the internship period at  the University of Edinburgh.} \hspace{0.5cm} Timothy M. Hospedales$^{3}$ \hspace{0.5cm} Tatiana Tommasi$^{1,2}$\\ 
$^{1}$Politecnico di Torino  \hspace{0.5cm} $^{2}$Italian Institute of Technology \hspace{0.5cm} $^{3}$University of Edinburgh \\
\texttt{\small \{leonardo.iurada, silvia.bucci, tatiana.tommasi\}@polito.it} \hspace{0.5cm} \texttt{\small t.hospedales@ed.ac.uk} 
}

\date{}

\maketitle

\ificcvfinal\thispagestyle{empty}\fi

\begin{abstract}
    

Over the past few years, there has been a significant amount of research focused on studying the ReLU activation function, with the aim of achieving neural network convergence through over-parametrization. However, recent developments in the field of Large Language Models (LLMs) have sparked interest in the use of exponential activation functions, specifically in the attention mechanism.

Mathematically, we define the neural function $F: \R^{d \times m} \times  \mathbb{R}^d \rightarrow \mathbb{R}$ using an exponential activation function. Given a set of data points with labels $\{(x_1, y_1), (x_2, y_2), \dots, (x_n, y_n)\} \subset \mathbb{R}^d \times \mathbb{R}$ where $n$ denotes the number of the data. Here $F(W(t),x)$ can be expressed as $F(W(t),x) := \sum_{r=1}^m a_r \exp(\langle w_r, x \rangle)$, where $m$ represents the number of neurons, and $w_r(t)$ are weights at time $t$. It's standard in literature that $a_r$ are the fixed weights and it's never changed during the training. We initialize the weights $W(0) \in \mathbb{R}^{d \times m}$ with random Gaussian distributions, such that $w_r(0) \sim \mathcal{N}(0, I_d)$ and initialize $a_r$ from random sign distribution for each $r \in [m]$.

Using the gradient descent algorithm, we can find a weight $W(T)$ such that $\| F(W(T), X) - y \|_2 \leq \epsilon$ holds with probability $1-\delta$, where $\epsilon \in (0,0.1)$ and $m = \Omega(n^{2+o(1)}\log(n/\delta))$. To optimize the over-parametrization bound $m$, we employ several tight analysis techniques from previous studies [Song and Yang arXiv 2019, Munteanu, Omlor, Song and Woodruff ICML 2022]. 

 

\end{abstract}

\section{Introduction}
\section{Introduction}
\label{sec:intro}

Hamiltonian systems (HSs) as the mathematical model for classical mechanics have been central to the advance of modern physics and mathematics alike. In physics, HSs provide a theoretical foundation for several approaches to quantization. In mathematics, the interest in HSs has led to the study of symplectic geometry and topology. The methods developed in this study, e.g. Floer theory, have proven to be of great success for various branches of mathematics and physics alike, for instance celestial mechanics and string theory.\\
Simply put, a HS consists of three data: a smooth manifold $M$, a symplectic $2$-form $\omega$ on $M$, and smooth function $H\in C^\infty (M,\mathbb{R})$. In physical terms, the symplectic manifold $(M,\omega)$ can be understood as the phase space of the system, while the function $H$, often called Hamilton function or, simply, Hamiltonian, assigns to every point in phase space its energy. These data allow us to define the Hamiltonian vector field $X_H$ on $M$ via the equation $\iota_{X_H}\omega = -dH$. The dynamics of the HS $(M,\omega, H)$ is governed by the vector field $X_H$. Precisely speaking, the physical trajectories of point-like particles described by the HS $(M,\omega, H)$ are exactly the integral curves of $X_H$. The connection between the integral curve equation of $X_H$ and the Hamilton equations known from classical mechanics is given by Darboux's theorem: every symplectic form $\omega$ can locally be written as
\begin{gather*}
 \omega = \sum^n_{i = 1} dp_i\wedge dq_i.
\end{gather*}
In such Darboux charts, the integrable curve equation of $X_H$ reduces to the Hamilton equations:
\begin{gather*}
 \dot q_i(t) = \frac{\partial H}{\partial p_i};\quad \dot p_i(t) = -\frac{\partial H}{\partial q_i}\quad\forall t\in I\ \forall i\in\{1,\ldots, m\}.
\end{gather*}
Since the integral curve equation is just a first-order differential equation, there always exists an open interval $I$ and a trajectory $\gamma:I\to M$ with $\gamma (t_0) = x$ for any initial value $x\in M$ and $t_0\in\mathbb{R}$. Furthermore, two trajectories $\gamma_1:I_1\to M$ and $\gamma_2:I_2\to M$ are identical iff they have the same domain ($I_1 = I_2\equiv I$) and attain the same value at some point $t_0\in I$. In particular, maximal trajectories for a given initial value are unique.\\
Physical trajectories also obey the action principle, i.e., they can be obtained as ``critical points'' of the action functional $\mathcal{A}_H:C^\infty(I,M)\to\mathbb{R}$ assigned to an exact\linebreak HS $(M,\omega = d\lambda, H)$:
\begin{gather*}
 \mathcal{A}_H[\gamma]\equiv \mathcal{A}^\lambda_H[\gamma]\coloneqq \int\limits_I \gamma^\ast\lambda - \int\limits_I H\circ\gamma (t)\, dt.
\end{gather*}
Here, ``critical point'' means that the first variation of $\mathcal{A}_H$ has to vanish at the trajectory $\gamma\in C^\infty(I,M)$ where we only allow for variations of $\gamma$ which keep the endpoints of $\gamma$ fixed. Sometimes, for instance in Floer theory, one wishes to view certain trajectories as actual critical points of some action functional. There are several ways to achieve this, e.g. by putting the endpoints of a trajectory on a Lagrangian or by only considering periodic trajectories.\\
Since HSs are given in terms of \underline{real-valued} manifolds $M$, forms $\omega$, and functions $H$, it is only natural to ask whether a similar construction with similar properties exists for \underline{complex-valued} manifolds $X$, forms $\Omega$, and functions $\mathcal{H}$. The answer to this question directly leads to the notion of \textbf{holomorphic Hamiltonian systems} (HHSs). Similarly to real Hamiltonian systems\footnote{To distinguish real- and complex-valued Hamiltonian systems, we call real-valued Hamiltonian systems real Hamiltonian systems from now on.} (RHSs), HHSs are also described by three data (cf. \autoref{subsec:def_HHS}): a complex manifold $X$ (implicitly defining an integrable complex structure $J$), a holomorphic symplectic $2$-form $\Omega$ on $X$, and a holomorphic function $\mH:X\to\mathbb{C}$.
%As for RHSs, one can also associate a (holomorphic) Hamiltonian vector field $X_\mH$, (holomorphic) trajectories, and action functionals with a HHS.
HHSs have been studied since the early 2000s, e.g. by Gerdjikov and Kyuldjiev et al. \cite{gerd2001}, \cite{gerd2002}, \cite{gerd2004} or by Arathoon and Fontaine \cite{arathoon2020}. In the given references, HHSs are usually viewed as complexifications of RHSs and mostly used as a tool to study RHSs which arise as real forms\footnote{The terms ``complexifications and real forms of Hamiltonian systems'' can be defined properly, but we do not give an explicit definition here. For us, it suffices to know that complexifications and real forms of Hamiltonian systems are defined similarly to complexifications and real forms of manifolds: a real manifold $M$ is the real form of a complex manifold $X$ and $X$ is a complexification of $M$ iff $M$ is the fixed point set of some anti-holomorphic involution on $X$.} of HHSs. In \cite{arathoon2020}, for instance, an integrable and compact RHS is constructed out of the HHS obtained from the complexification of the spherical pendulum.\\
In the present paper, we take a different approach. We study HHSs on their own and try to recreate the results known from RHSs for HHSs. To start with, we discuss the existence and uniqueness of holomorphic trajectories. Similarly to RHSs, holomorphic trajectories are defined as the holomorphic integral curves of the holomorphic Hamiltonian vector field $X_\mH$. We show in \autoref{subsec:holo_traj} that, locally, holomorphic trajectories always exist and are unique, given an initial value. Maximal holomorphic trajectories, however, are not unique anymore, even given an initial value, due to the effects of monodromy\footnote{Recently, the monodromy of the complexified Kepler problem has been studied by Sun and You (cf. \cite{shanzhong2020}).}. This behavior is in sharp contrast to RHSs. Nevertheless, the holomorphic trajectories still give rise to a foliation by energy hypersurfaces $\mH^{-1}(E)$ for regular values $E$ of $\mH$, as shown in \autoref{subsec:holo_traj}.\\
After this discussion, we prove in \autoref{subsec:holo_action_fun_and_prin} that the holomorphic trajectories satisfy an action principle\footnote{To the extent of the author's knowledge, an action principle for HHSs has not been formulated before in the literature.}, i.e., that they can be understood -- in some sense -- as critical points of certain action functionals. These action functionals are obtained by first decomposing a HHS $(X,\Omega,\mH)$ into four RHSs, one for each combination of real and imaginary part of $\Omega$ and $\mH$. To each RHS, we can assign the usual action functional of a RHS. Afterwards, we average each of these action functionals over the imaginary (or real) time axis and take an appropriate linear combination to obtain the action functional for the HHS $(X,\Omega,\mH)$. In fact, this method gives rise to a plethora of action functionals for the HHS $(X,\Omega,\mH)$ which simply differ by how one averages and takes the linear combination. We conclude \autoref{sec:HHS} with an application of HHSs. Precisely speaking, we establish a relation between Lefschetz fibrations and almost toric fibrations in \autoref{subsec:Lefschetz} using HHSs.\\
During the investigation of action functionals for HHSs, we observe that $J$, the complex structure of $X$, poses rather strong restrictions on the existence of certain holomorphic trajectories. In \autoref{subsec:holo_action_fun_and_prin}, we consider holomorphic trajectories whose domains are complex tori and interpret them as the complexification of periodic orbits. However, by the maximum principle, such holomorphic trajectories are always constant if the complex manifold in question is $X=\mathbb{C}^{2n}$ equipped with the standard complex structure $J = i$. The same argument does not hold anymore if we allow $J$ to be any almost complex structure. In his beautiful paper \cite{moser1995} from 1995, Moser shows that it is possible to pseudo-holomorphically embed complex tori in $\mathbb{R}^4$, where $\mathbb{R}^4$ is equipped with a suitable, not necessarily integrable almost complex structure $J$.\\
To avoid constraints imposed by the integrability of $J$, we introduce special Hamiltonian systems in \autoref{sec:PHHS} which are described by the same data as HHSs, but whose almost complex structure $J$ does not need to be integrable anymore. These Hamiltonian systems are called \textbf{pseudo-holomorphic Hamiltonian systems} (PHHSs) and exhibit, by design, the same properties as HHSs. In particular, pseudo-holomorphic trajectories of PHHSs induce foliations by regular energy hypersurfaces $\mH^{-1}(E)$ and obey an action principle (cf. \autoref{subsec:def_PHHS}).\\
At first glance, PHHSs may appear to be contrived and artificial, especially since, by definition, the imaginary part of $\Omega$ does not need to be closed anymore. However, the non-closedness of the imaginary part of $\Omega$ is an unavoidable consequence of the non-integrability of $J$, as we show in \autoref{subsec:rel_HHS_PHHS}. In fact, we prove that we recover a HHS from a PHHS if and only if $J$ is integrable or, equivalently, the imaginary part of $\Omega$ is closed. To further strengthen our claim that PHHSs are indeed a natural generalization of HHSs, we show that the space of proper\footnote{A proper PHHS is a PHHS which is not simultaneously a HHS.} PHHSs is open and dense in the space of PHHSs on a fixed manifold $X$ with $\text{dim}_\mathbb{R}(X)>4$ implying that proper PHHSs are generic. To prove that proper PHHSs are generic, we first give a method to construct proper PHHSs out of HHSs (cf. \autoref{subsec:constructing_PHHS}). The method itself is very interesting, since it is related to hyperkähler structures and allows us to equip the cotangent bundle of a complex manifold with the structure of a PHHS. Lastly, we use this construction to deform HHSs by proper PHHSs (cf. \autoref{subsec:deforming_HHS}).


\section{Related Works}
\section{Related work}
\noindent \textbf{Video foundation models.}
With sufficient computational power and an abundant source of data, there have been attempts to build a single large-scale foundation model that can be adapted to diverse downstream tasks.
Along with the success of foundations models in the natural language processing domain~\cite{brown2020language,chen2021evaluating,devlin2019bert} and in computer vision~\cite{bertasius2021space,jia2021scaling,radford2021learning}, video data has become another data type of interest, as it has grown in scale due to numerous internet video-sharing platforms.
Accordingly, several methods to train a video foundation model have been proposed.
Due to the innate multi-modality of video data, \textit{i.e.}, a combination of visual $\cdot$ vocal $\cdot$ textual context, most works have centered around the variations of the cross-modal attention mechanism \cite{akbari2021vatt,bertasius2021space,gabeur2020multi,luo2020univl,neimark2021video,tan2021look,wei2020multi,yang2021taco}.
In addition, as most video data lack proper labels or descriptions, contrastive learning methods were studied to learn meaningful feature representations or enhance video-text alignment in a self-supervised manner \cite{akbari2021vatt,kuang2021video,luo2020univl,yang2021taco}.

More specifically, MERLOT \cite{zellers2021merlot} proposed a multi-modal representation learning method for visual commonsense reasoning, which also performed well in twelve video reasoning tasks.
VATT \cite{akbari2021vatt} introduced a multi-modal learning method via contrastive learning. 
The pre-trained model performed well in a variety of vision tasks from image classification to video action recognition and zero-shot video retrieval.
Another representative work, UniVL \cite{luo2020univl} proposed a straightforward pre-training method with auxiliary loss functions. 
After fine-tuning on a specific task, the pre-trained model performed outstandingly in a wide range of tasks of text-to-video retrieval, action segmentation, action step localization, video sentiment analysis, and video captioning.
Other foundation models for multiple video tasks include \cite{li2020hero,sun2019learning,sun2019videobert,zhu2020actbert,fu2021violet,wang2022all}. 

\noindent \textbf{Auxiliary learning.}
In order to enhance the performance of one or a multitude of primary tasks, auxiliary learning methods can be incorporated.
\cite{ruder2017overview} introduced Multi-task learning (MTL) to the deep neural networks by training a single model with multiple task losses to assist learning on the main task.
Such a method is generally adapted to pre-train the foundation models in the self-supervised manner~\cite{li2020hero,sun2019learning,sun2019videobert,zhu2020actbert,fu2021violet,wang2022all}.
However, these various pretext task losses used in the pre-training phase are ignored in the fine-tuning phase, and only the primary task loss is minimized.

Recently, meta-learning methods have been introduced for auxiliary learning.
\cite{liu2019self,navon2020auxiliary,shu2019meta} proposed a meta-learning method in which the model learns auxiliary tasks to generalize well to unseen data. 
In these settings, a separate subset of data is held out as the primary task, while the others are used as auxiliary tasks that aid the primary task's performance.
Similar methods were adopted for computer vision tasks such as semantic segmentation \cite{xu2021leveraging}.
Other domain applications include navigation tasks with reinforcement learning \cite{ye2021auxiliary}, or self-supervised learning methods on graph data \cite{hwang2020self}.

\section{Fairness meets Cross-Domain Learning}
\section{Threat Model and Advantages of Our Hardware-based Adversarial Detector} \label{sec: motivation}
\ry{In this part, I want to highlight the comparison between hardware and software attacks}
%Normally, software-based adversarial detectors are easier to implement, cheaper to develop and more well-studied than those based on hardware computational signals.
% We would like to stress that our goal for investigating hardware-based adversarial detectors is not to achieve better performance in detection than the conventional white-box software based methods.  
\subsection{Threat Model} \label{sec: threat model}
\ry{This section is threat model: attack is `white-box', detector is `black-box'}
The victim is a DNN classifier, which is pre-trained with a public dataset. The testing dataset may be kept private.
We assume the strongest `white-box' attack model, where the attacker has full knowledge of the victim model and training dataset in order to generate adversarial samples with minimum perturbations. 
On the contrary, the detection system assumes the most limited scenario, under a `black-box' view of the victim, without access to the victim's inputs, parameters, and intermediate outputs or execution details. 
The only information available to the detector to distinguish adversarial samples is the EM side-channel measurement and the victim model's prediction class.
For training the adversarial detector with EM traces, a public benign dataset is used. 

\if false 
\ry{In this part, we discuss more settings of the detector especially the data used in two phases.}
In general, the detecting process can be summed up into two phases, training phase and detecting phase.
To begin with, we train an Out-of-Distribution(OOD) detector on a public benign dataset of the same classification task, which should be distinct from the victim's training dataset.
For each query, the detector will obtain the classification result and an EM trace along with the model execution to fit its EM classifiers and anomaly detectors.  
During the detection phase, the victim model is in operation and under attack when the pre-trained detector decides whether the current input is adversarial or not, only based on the victim model output and its EM trace.
\fi 

\subsection{Advantages}
Compared to software-based adversarial detection methods, our hardware-based detector, EMShepherd, has three distinct advantages: privacy-preserving, portability, and robustness.

\begin{itemize}[leftmargin=*]
    \item \ry{Add a new motivation here. The motivation is that using \name can help the user protect their privacy.} 
    \name protects the DNN model user's data privacy as it is agnostic to the model's inputs, which instead are always required by prior reconstruction-based detection methods~\cite{meng2017magnet, yang2022you}. 
    %Most model users are benign whose inputs may be sensitive and should not be shared with \textit{third-party detectors}. 
    The sensitive inputs should not be shared with \textit{third-party detectors}. 
    Our design only requires the output class labels and the EM signals, which are passively leaked to common acquisition equipment. 
    %    Our design is suitable for such cases as it only requires the EM signals and the inference outputs during the model execution. Generally speaking, EM signals and labels have less private information leakage.
    \item \ry{The second motivation is still related to privacy. This time we consider model privacy when the model structure or parameters should be kept private.}
   \name also protects the model confidentiality.  No model information, including %Using hardware-based detectors can prevent the third-party defender from accessing some confidential model information such as  
   hyper-parameters, parameters, and logits, is needed, in stark contrast to the previous software-based detection methods~\cite{ma2019nic,feinman2017detecting}.
    %Our \name only acquires the EM traces during model inference in a passive and noninvasive manner, 
    The EM data processing and the adversarial detector training process are both victim model-agnostic. 
    Therefore, our method has more general usage, applicable to closed-source DNN applications, which are pervasive in edge devices where the user only queries the models for the final prediction output. 
    \item \ry{The third motivation is portability.}  
    Owing to the model-agnostic feature, EMShepherd can be easily ported for wide-range hardware devices with different DNN implementations for diverse applications. It can be used as a `plug and play' (PnP) device, aside from the target system, to work automatically without user intervention or contact with the victim system. 
    \item \ry{The last motivation is about adaptive attacks, we should propose that EM signal is hard to imitate, so it is hard for adaptive attacks to generate sample fraud both detector and victim.} 
    Adaptive attack~\cite{adaptive} is a threat to most software defense methods where the attacker adjusts the adversarial perturbations to mislead both the victim models and defense systems.
   %  The hardware-based detection method can provide a double protection on top of most software defense methods such as adversarial training.
   %  Although the adptive adversarial example fools the robust model, its computation patterns during the DNN model execution are still well kept in the EM traces and our EMShepherd framework still works well for detecting the new type of adversarial examples.  
   %  Meanwhile, due to the high complexity of EM signals and non-explicit dependency of the EM signals on computations, it is extremely hard to have an adaptive attack on our detection method, i.e., adversarial examples whose EM signals are deliberately controlled to evade the EM-based detector.
   However, due to the high complexity and non-explicit dependency of the EM signals on computations and data, 
   it is extremely hard to have an adaptive attack on our detection method, 
   i.e., adversarial examples whose EM signals are deliberately controlled to evade the EM-based detector. 
\end{itemize}






\section{Fairness Criteria}\label{sec:criteria}

Evaluating the group fairness of a classification model means assessing its performance on different population subgroups and comparing them.
Many criteria have been proposed for this \cite{Verma2018fairness,Watcher2021bias}.
In the following, we review the most used metrics in computer vision. 
%
We start from the basic definitions of True Positive Rate $TPR=TP/(TP+FN)$, False Positive Rate
$FPR=FP/(FP+TN)$ and Accuracy $Acc = (TP+TN)/(TP+TN+FP+FN)$.
In terms of conditional probabilities for data 
with two different attributes, it holds 
\begin{align}
TPR_{a=0} & = P(\hat{y}=1|y=1, a=0) \\
FPR_{a=0} & = P(\hat{y}=1|y=0, a=0) 
\end{align}
and their analogues for $a=1$.
%
The \emph{Difference in Equal Opportunity (DEO)}  measures fairness by 
\begin{equation} 
|P(\hat{y}=1|y=1, a=0) - P(\hat{y}=1|y=1, a=1)|~,
\end{equation}
so the maximum fairness is obtained for $DEO=0$ when $TPR_{a=0}=TPR_{a=1}$. 
%
The \emph{Difference in Equalized Odds (DEOdds)}  measures fairness by 
\begin{equation} %\small
\sum_{t\in\{0,1\}}|P(\hat{y}=1|y=t, a=0) - P(\hat{y}=1|y=t, a=1)|~,
\end{equation}
thus maximum fairness is obtained for $DEOdds=0$ when both $DEO=0$ and $FPR_{a=0}=FPR_{a=1}$. In other words, the decision of the classifier should be conditionally independent of the attribute, given the ground truth ($\hat{y}\perp a | y$).
%
Another basic way to consider the variation of the model's output over the subgroups identified by the attributes is via the \emph{Difference in Accuracy (DA)}: 
\begin{equation}
|P(\hat{y}=y|a=0)-P(\hat{y}=y|a=1)|~.
\end{equation} 
%
All these metrics evaluate the relative behavior of the classifier on data subgroups defined by different attributes but lose track of its absolute performance. This is a critical issue as shown by the practical example in the left part of Figure \ref{fig:hist_fair}. Although the performance of the two classifiers is different, with $C1$ better than $C2$, they have the same value of $DEO$.
Moreover, both $DEO$ and $DEOdds$ are minimized by a trivial classifier that predicts always $\hat{y}=1$. In that case, for all the attributes it holds $FN=TN=0$, so $TPR=FPR=1$ and $DEO=DEOdds=0$. Since the accuracy reduces to the Positive Predictive Value ($PPV=TP/(TP+FP)$), also $DA$ becomes uninformative.

\begin{figure}[tb]
\centering
\includegraphics[width=0.93\linewidth]{fairness_mGA-MGA_.pdf}
\caption{Visualization of the $[mGA,MGA]$ space with exemplar points. The bottom triangular part of the space is unfeasible as by definition $mGA$ is lower than $MGA$. The three plots on the right show the HF isolines when starting from different baseline methods indicated by the $\triangle$, $\lozenge$ and $\times$ points. }
\label{fig:fairness_explained}
\vspace{-4.5mm}
\end{figure}

Recent works have introduced the \emph{Minimum Group Accuracy (mGA)} as fairness criterion: rather than evaluating differences in statistics across groups, it considers the classification accuracy of the worst performing group \cite{zietlow2022leveling,diana2021minimax,martinez2020minimax}. 
The rationale of this metric is that by increasing $mGA$ we are certainly improving the overall accuracy. Hence we avoid the suboptimal condition of unnecessarily harming all groups to get a trade-off improvement in fairness measured by $DEO$ and $DEOdds$.
%
Still, when the goal is to evaluate whether a certain unfairness mitigation method was able to improve over the reference classifier, $mGA$ is not sufficiently informative as exemplified by the right part of Figure \ref{fig:hist_fair}. Here $a=0$ is the privileged attribute, thus the one that identifies the best group with the associated \emph{Maximum Group Accuracy (MGA)}. When moving from $C1$ to $C2$, $mGA$ increases and so does $MGA$. Although globally the classifier has improved, the disadvantaged group suffers even more for unfair treatment with respect to the privileged one as indicated by the increased $DA$.  


With these premises, we can state that fairness can be meaningfully assessed only together with prediction accuracy. Both their performance can be considered by looking at several bar plots jointly or at bi-dimensional plots as done in \cite{zietlow2022leveling}. However, interpreting them and making sense of multiple pieces of information at once is difficult, and defining a single score would facilitate rigorous quantitative evaluations. 
To this purpose, we can start from the space defined by $mGA$ and $MGA$. As shown in Figure \ref{fig:fairness_explained}, the bottom right triangular part of the space is an unfeasible region where  $mGA>MGA$. In the top right corner, the point with $[mGA,MGA]=[100,100]$ indicates the optimal utopia condition. The results of various methods can be collected in this space and ranked on the basis of the $L2$ Distance To the Optimum ($DTO$, \cite{zong2023medfair}) which sounds like a reasonable  metric for the score.
 
Let's focus for instance on the marked points in the figure and consider the biased reference classifier represented by $\triangle=[50,100]$. We expect a good unfairness mitigation method to keep the top $MGA=100$ result and improve $mGA$ to reduce the discrepancy among groups, thus moving horizontally towards the ideal point. The point $\medstar=[70.38,100]$ is a possible result for such an approach.
Differently, a method that trades off accuracy for fairness would decrease $MGA$ while improving $mGA$ to reduce $DA$ to zero. This behavior is exemplified by the 
point $\square=[79.06,79.06]$. It can be noticed that both $\square$ and $\triangle$ 
share the same Pareto efficiency level 
approximated by the circumference centered in $[0,0]$, as done in \cite{zietlow2022leveling}. Instead, $\medstar$ shows an advantage in efficiency, which is feasible as discussed in \cite{liu2023pushing}.
Despite their clear difference, the points $\medstar$ and $\square$ are equivalent according to $DTO$.
Thus, although $DTO$ keeps track of both $mGA$ and $MGA$ it might not be sufficiently informative to benchmark different unfairness mitigation approaches. The presented analysis also highlights the importance of taking as reference the performance of the
baseline to fully understand model comparisons.



%\section{Harmonic Fairness}


\section{Harmonic Fairness}
\label{sec:HF2}

To better deal with the peculiarities of the space defined by $mGA$ and $MGA$, we \emph{formalize relative distances} for each method with respect to its biased reference and introduce \emph{our new Harmonic Fairness} metric. 

\vspace{1mm}\noindent\textbf{Classification.} 
We focus on $MGA$ and $DA=MGA-mGA$, using the subscripts $b$ and $m$ to refer respectively to the baseline model and its unfairness-mitigated version. The relative differences are: \vspace{-1mm}
\begin{align}\small
\Delta DA  & = DA_b - DA_m\\
\Delta MGA & = MGA_m - MGA_b
\end{align}
with $\Delta DA, \Delta MGA \in \{-100,100\}$. Both these values will be high for an accurate and fair model. Thus, we combine them in the \emph{Harmonic Fairness} metric defined as:
\begin{equation}\small
HF = \frac{\Delta DA' \times \Delta MGA'}{\Delta DA' + \Delta MGA'}~,
\label{eq:HF}
\end{equation}
where we added an additional shift to the component values to avoid degenerate cases (dividing by 0): $\Delta DA'=\Delta DA+100$ and $\Delta MGA'=\Delta MGA+100$.
The minimal value $HF=0$ corresponds to having either $\Delta DA = -100$ or $\Delta MGA = -100$, which can be obtained with a very poorly defined model that reduces the performance (increasing $DA$ or decreasing $MGA$) rather than improving over the baseline. An unfairness mitigation model that maintains the same $DA$ and $MGA$ of the original baseline gets $HF=50$. Finally, every increase over this value corresponds to models able to symmetrically improve accuracy and fairness. 
 
Getting back to the points $\medstar$ and $\square$ analyzed before and always considering the $\triangle$ as a baseline, we obtain the meaningful ranking $HF_{\medstar}=54.62 > HF_{\square}=51.77$ 
which matches the expectations given the advantage of the former over the latter. 
%
We remark that $HF$ takes into proper account the model starting baseline and encourages a decrease in $DA$ and an increase in $MGA$ with different strengths depending on the baseline position, consequently shaping the space in various ways as shown by the isolines of $HF$ in the right part of Figure \ref{fig:fairness_explained}. Of course, the right way to benchmark multiple methods is by setting a fixed baseline model considered as a shared reference for all of them.
%


\vspace{1mm}\noindent\textbf{Landmark Detection.}
When dealing with landmark detection every data sample can be defined as $(\bx,a,\bY)$, where $\bY\in \mathbb{R}^{K\times 2}$ is a set of $\by_{1,\ldots,K}$ landmark coordinates. The reference metric for this task is the Normalized Mean Error ($NME$) calculated as:\vspace{-1mm}
\begin{equation}\small
NME(\bY,\hat{\bY}) = \frac{1}{K}\sum_{i=1}^K\frac{\|\by_i-\hat{\by_i}\|_2}{D}~,
\end{equation}
where $D$ is a normalization factor, usually chosen as the interocular distance for face images. 
We indicate with $SDR$ the Success Detection Rate calculated as the percentage of images whose NMEs is less than a given threshold. Symmetrically to what was done for classification, we define \emph{Max Group Success} ($MGS$) and \emph{Min Group Success} ($mGS$), respectively as the success rate of the best and worst performing protected groups. We consider also the difference between groups $DS = MGS - mGS$, and to assess the effectiveness of an unfairness mitigation model $m$ over the reference baseline $b$ we calculate: \vspace{-1mm}
\begin{align}\small 
\Delta DS & = DS_b - DS_m \\
\Delta MGS & = MGS_m - MGS_b \vspace{-1mm}
\end{align}
with $\Delta DS, \Delta MGS \in \{-100,100\}$. We then combine these values to get the $HF$ metric for landmark detection consistent with what defined for classification in equation (\ref{eq:HF}).

\vspace{1mm}\noindent\textbf{Rescaling.}
To better investigates fine-grained differences among the results of various unfairness mitigation methods we adopt a simple sigmoid rescaling: $\sigma(HF) = \frac{1}{1+\exp^{-HF+50}}$, with $\sigma(HF) \in \{0,1\}$. Hence, $\sigma(HF)>0.5$ will indicate a gain over the reference baseline.  



\section{Benchmark description}
\begin{figure*}[t]
    \centering
    % \footnotesize
    \includegraphics[width=.98\textwidth]{images/bench3.pdf}
    \caption{Overview of the proposed benchmark \benchname, which consists of 5 sub-datasets and can be categorized to natural and synthetic. 
    Act., Obj., Attr. denote action, object and attribute, respectively. 
    }
    \label{fig:benchmark}
\end{figure*}

\section{Experiments}



In this section we present the main results of our experiments. Further evaluations are added in the supplementary material which also includes all the implementation details as well as the code. Our PyTorch implementation covers all the methods evaluated in the benchmark to guarantee maximal transparency and reproducibility. It can also easily include other methods for future benchmark extensions. Unless stated otherwise, for all the experiments 
we adopted the same validation protocol described in \cite{zietlow2022leveling}.



\subsection{Classification Results}
For the binary classification tasks, we organize the tables with different horizontal sections that group the cross-domain methods by family. 
The bottom part of the tables contains the SOTA fairness approaches. 
Besides the standard metrics and our newly introduced $\sigma(HF)$ we consider
$\Delta DTO = DTO_b - DTO_m$: as the baseline is fixed and shared by all the methods, $\Delta DTO$ ranks the methods exactly as $DTO$ but makes the tables easier to read.


\textbf{The results on CelebA} are presented in Table \ref{tab:celebaA_EB_C} and focus on the two most challenging attributes: \emph{EyeBags} and \emph{Chubby}. Out of the whole set of 13 attributes (complete results in the supplementary), they are the ones with the highest $DA$ and lowest $Acc$. 
By focusing on both $\sigma(HF)$ and $\Delta DTO$ we can state that several cross-domain methods provide an accuracy and fairness gain over the baseline, and are also able to improve over the SOTA fairness methods which appear particularly inefficient on Chubby. The AFN approach shows the best performance, followed by DANN on Eyebags and SagNets on Chubby.  
%% leave space%%

Regarding the metrics,
$\Delta DTO$ and $\sigma(HF)$ 
agree on the general ranking. On the other hand, neither $mGA$ nor $DA$ is sufficiently informative. For instance on Eyebags, RelRot and RelRotAlign have the same $mGA$, while both their $\Delta DTO$ and $\sigma(HF)$ are significantly different, with RelRotAlign even worsening the baseline. Moreover, the best $DA$ result of RelRotAlign is clearly misleading as it comes with a noteworthy decrease in Acc. 
%
Finally, we notice 
how $DEO$ and $DEOdds$ for Chubby reward the \emph{leveling down} behavior already criticized in \cite{zietlow2022leveling}, by largely relying on $DA$ without considering the decrease in $MGA$ and $mGA$ with respect to the baseline.  

\textbf{The results on COVID-19 Chest X-Ray and Fizpatrick17} are presented in Table \ref{tab:covid+melanoma}. 
On the first dataset, 
according to both $\Delta DTO$ and $\sigma(HF)$, RSC is the top method and RelRotAlign is the second best, while AFN ranks third. 
%
Interestingly, now 
the rankings of DANN and SagNets differ depending on the used metric: 
$\sigma(HF)$ rewards the former, more than the latter. DANN has a significant advantage in $mGA$ despite a loss in $MGA$,  while SagNets shows a minimal increase in both $mGA$ and $MGA$ in spite of a worse discrepancy among the groups.
%
Even in this case it is clear that referring only to $mGA$ may not be sufficient to differentiate among the methods as many of them share the exact same value for this metric. 

The results on Fizpatrick17 lead to similar conclusions, with RelRot and LSR presenting the best results. DANN, which was among the top methods for CelebA, now ranks sixth among all the CD approaches and still shows results comparable with the best SOTA fairness approach.

Overall, the exact the family that best suits each classification task may vary (feature alignment and adversarial training methods for faces, regularization-based and self-supervised approaches for medical images), but the results confirm the effectiveness of cross-domain learning for unfairness mitigation and the relevance of our study. 





\begin{figure}
\begin{tabular}{cc} 
\hspace{-7.5mm}
\includegraphics[width=0.55\linewidth]{age_hf_crop} &
\hspace{-0.5cm}
\includegraphics[width=0.55\linewidth]{age_deltadto_crop}
\\
\end{tabular}
\caption{Landmark detection results. Comparison among the cross-domain methods and the reference baseline in terms of $HF$ and $\Delta DTO$ when changing the NME threshold used for $SDR$.}
\label{fig:NMEth} \vspace{-2mm}
\end{figure}


\begin{table}[t]
    \begin{center}
    \small
    \resizebox{0.48\textwidth}{!}{
    \begin{NiceTabular}{|c | c c c c | c | c | }[colortbl-like,cell-space-limits=1.5pt]
        \hline
        {} & \multicolumn{6}{c|}{\textbf{UTK Face Landmark} (\emph{skin tone})} 
        \\
        \cline{2-7}
        {} & SDR & \shortstack{MGS} & \shortstack{mGS} & \shortstack{DS} &  $\Delta DTO$ & $\mathbf{\sigma(HF)}$ 
        \\ \hline
        
Baseline  & 82.08 & 83.90 & 77.93 & 5.97 & 0.00 & 0.500 $\pm$ 0.000  
\\ \hline



\textbf{AFN} \cite{xu2019larger} & \underline{92.95} & \underline{93.80} & \underline{90.99} & \textbf{2.81} & \underline{16.38} & \underline{0.961} $\pm$ 0.047 
\\

\textbf{DANN \cite{ganin2016domain}} & 89.62 & 90.66 & 86.17 & 4.49 & 10.63 & 0.883 $\pm$ 0.017 
\\ 

\textbf{RegDA}  \cite{jiang2021regda} & \textbf{96.05} & \textbf{97.05} & \textbf{93.77} & \underline{3.28} & \textbf{20.43} & \textbf{0.979} $\pm$ 0.011  
\\ 

\hline

{} & \multicolumn{6}{c|}{\textbf{UTK Face Landmark} (\emph{age})} \\
\hline
Baseline  & 74.50 & 78.39 & 70.71 & 7.67 & 0.00 & 0.500 $\pm$ 0.000 \\
\hline
\textbf{AFN}  \cite{xu2019larger}& \underline{94.50} & \underline{95.04} & \textbf{93.97} & \textbf{1.06} & \underline{28.59} & \textbf{0.997} $\pm$ 0.001 \\
\textbf{DANN \cite{ganin2016domain}} & 81.03 & 85.22 & 76.83 & 8.39 & 8.92 & 0.809 $\pm$ 0.091 \\
\textbf{RegDA}  \cite{jiang2021regda} & \textbf{94.62} & \textbf{95.51} & \underline{93.74} & \underline{1.77} & \textbf{28.70} & \underline{0.996} $\pm$ 0.001  \\
\hline

    \end{NiceTabular}
    }
    \end{center}
    \vspace{-2mm}
    \caption{Landmark detection results. SDR is evaluated using 8\% NME as threshold. Results averaged over three runs.
    }
    \vspace{-4mm}
    \label{tab:landmark}
\end{table}

\subsection{Landmark Detection Results}
The performance of a model which locates keypoints on facial components may be affected by a change in \emph{skin tone} and \emph{age}, resulting in a less precise prediction in case of high melanin pigmentation or wrinkles. To investigate the presence of a bias related to these demographics we run experiments on the UTK Face dataset and we verify the effectiveness of correction strategies based on cross-domain learning by considering RegDA together with AFN and DANN, as they have shown successful results in classification on face images. The training procedure follows the one presented in \cite{jiang2021regda}, with validation protocol in line with that of \cite{zietlow2022leveling}. 
%
We assess the performance of the methods by considering both our $\sigma(HF)$ and $\Delta DTO$ obtained from $SDR$ calculated with a standard 8\% $NME$ threshold \cite{mccouat2022contour}. 

Table \ref{tab:landmark} shows how the baseline reference has an unfair behavior with more than 5\% difference in group accuracy ($DS$). All the cross-domain methods provide an advantage: in particular, RegDA ranks higher or equal to AFN, and they are both better than DANN. The latter shows a large improvement in $MGS$ and $mGS$ when the sensitive attribute is age, but the group discrepancy appears worse than the baseline.  
%
By reducing the $NME$ threshold the evaluation becomes progressively more demanding until the extreme of considering a predicted point as successful only if it perfectly overlaps with the ground truth. The curves in Figure \ref{fig:NMEth} show that even moving toward this condition most of the cross-domain methods maintain their advantage over the baseline confirming their effectiveness.  
The difference between RegDA and AFN becomes more evident at lower threshold values. In that regime $HF$ (as well as $\sigma(HF)$) and $\Delta DTO$ show different trends for RegDA with the first discouraging the use of this approach.




\begin{table}[t]
    \begin{center}
    \small
    \resizebox{0.48\textwidth}{!}{
     \begin{NiceTabular}{|c | c c c c | c | c |}[colortbl-like,cell-space-limits=1.5pt]
        \hline
        {} & \multicolumn{6}{c}{\textbf{CelebA - EyeBags}}\\
        \hline
        {} & \multicolumn{6}{c|}{{\textit{Male/Female} $\longrightarrow$ \textit{Young/Old}}}
        \\
        \cline{2-7}
        {} & Acc. & \shortstack{MGA} & \shortstack{mGA} & \shortstack{DA} & $\Delta DTO$ & $\mathbf{\sigma(HF)}$
        \\ \hline
        

Baseline \cite{wang2020towards} & 81.06	& 83.20 & 74.05 & 9.16 & 0.00 & 0.500 $\pm$ 0.000 
\\ \hline

\textbf{AFN} \cite{xu2019larger} & 81.31	& 83.30 & 74.92 & \textbf{8.38} & 0.78 & 0.554 $\pm$	0.014  
\\ 

\textbf{DANN} \cite{ganin2016domain}  & \textbf{83.48} &	\textbf{85.83} &	\textbf{76.16} &	9.67 & \textbf{3.18} & \textbf{0.626} $\pm$	0.039
\\ 

\hline
\textbf{GroupDRO} \cite{sagawa2019distributionally} & \underline{82.52} &	\underline{84.90} &	\underline{75.68} &	\underline{9.22} & \underline{2.29}  &   \underline{0.599} $\pm$	\underline{0.055}
\\ 

\textbf{g-SMOTE} \cite{zietlow2022leveling}  & 80.16 &	82.58 &	72.62 &	9.96  & -1.54 & 0.415 $\pm$	0.168 
\\ 

\textbf{FSCL} \cite{park2022fair}  & 80.45 &	84.65 &	69.35 &	15.30	& -3.37 &   0.235 $\pm$	0.097  
\\ 

\hline


\rowcolor[gray]{0.95}
         & \multicolumn{6}{c|}{\textit{Young/Old} $\longrightarrow$ \textit{Young/Old} (\textbf{Oracle})}
         \\
        \hline

\rowcolor[gray]{0.95}
\textbf{AFN} \cite{xu2019larger} & 81.70	& 83.81 & 75.04 & 8.77 & 1.16 & 0.561 $\pm$	0.013 
\\ 

\rowcolor[gray]{0.95}
\textbf{DANN} \cite{ganin2016domain}  & 83.00 &	84.90 &	77.01 &	7.89 & 3.41 & 0.676 $\pm$ 	0.040 
\\ 
\hline

\rowcolor[gray]{0.95}
\textbf{GroupDRO}  \cite{sagawa2019distributionally}  & 82.23 & 83.70 & 75.67 & 8.03 & 1.63 & 0.598 $\pm$	0.049 
\\ 

\rowcolor[gray]{0.95}
\textbf{g-SMOTE} \cite{zietlow2022leveling}  & 	81.21 &	81.99 &	73.71 &	8.28 & -0.95 & 0.477 $\pm$	0.072  
\\ 

\rowcolor[gray]{0.95}
\textbf{FSCL} \cite{park2022fair}  & 80.50 &	84.66 &	69.39 &	15.27	& -3.33 &  0.237 $\pm$		0.098 
\\ 

 \hline
        
    \end{NiceTabular}
    }
    \end{center}
    \vspace{-2mm}
    \caption{Model Transferability analysis on the classification task. All the relative metrics are calculated with respect to the baseline results in the first row.}
    \label{tab:celeba_transfer}
    \vspace{-4mm}
\end{table}

\section{Model Transferability}
Considering the effort needed to train novel models, it is always desirable to exploit existing ones for new tasks.  
For unfairness mitigation approaches, what is learned by reducing the bias over some protected groups might be helpful also for other demographics. We study this aspect on the CelebA dataset considering \emph{EyeBags} as the target attribute with \emph{Male} and \emph{Young} as sensitive attributes.
We train and validate a classifier to recognize whether eye bags are present while learning to disregard gender-specific features through a cross-domain approach. Then, we test the obtained model by assessing how the eye bags prediction performance differs among age groups. We analyze the CD methods AFN and DANN, reporting also the results of the SOTA unfairness mitigation strategies. 
%
The top part of Table \ref{tab:celeba_transfer} shows the effect on age groups of the approaches trained to be gender agnostic while focusing on the semantic features relevant to identify the target \emph{EyeBags} attribute. The results exceed those of the baseline with a particular advantage of DANN over GroupDRO, indicating that the knowledge acquired with cross-domain learning is easily transferrable.
The bottom part of the table presents the performance of \emph{oracle} methods trained and validated with the aim of mitigating age bias. They represent an upper bound and allow to better appreciate the surprisingly competitive results of transferred cross-domain models.
We note also how the SOTA unfairness mitigation models obtain low results even in this oracle setting.
%
More experiments with inverted roles for the sensitive attributes (\textit{Young/Old} $\rightarrow$ \textit{Male/Female}) and similar settings on the landmark detection task are reported and discussed in the supplementary.


\section{Conclusions}
\section{Conclusions}
We consider the phase-extraction problem, and we showed that, given a unitary $U = e^{i\pi H}$ and its inverse $U^{\dag}$, we could implement a block-encoding of $\phi(H)$ for some smooth function $\phi(x)$. The word `smooth' here means existence and continuity of the derivatives: the higher the number of continuous derivatives that a function has, the faster its Fourier sum (and thus the Laurent polynomial on the eigenphases) uniformly converges to that function. We are confident this can have many more applications beyond what is shown in this work. It is also worth remarking that Jackson showed that the convergence rate of a Fourier series is almost-optimal, in the sense that no trigonometric (or, equivalently, complex exponential) series can approximate the desired function faster, up to that $\log d$ factor~\cite[p.\ 21]{jacksonTheoryApproximation1930a}. Also remember that `smoothing' a function, i.e., replacing its derivative with a continuous function, does not give faster convergence for free in general, as its derivative will become steep in the points where we smooth out discontinuities, and this translates to a high Lipschitz constant: a~clear example is given by Eq.~\ref{eq:lipschitz-constant-recurrence-solution}, but in that case, fortunately, nothing depends on the size of the input $N$, and thus does not influence the asymptotic query complexity of Algorithm~\ref{alg:prop-sampling-qsp}, although the constant factor can become large even for $p = 20$. From a theoretical point of view, this work shows that, for any $\eta > 0$, there is an algorithm with query complexity 
$$\Tilde{\bigO}\left(\frac{1}{\bar{c}^{\frac{1}{2} + \eta}} \frac{1}{\epsilon^\eta} \right)$$
solving the proportional-sampling problem. This statement seems to suggest there exists an algorithm which directly solves the problem with $\eta = 0$, and an open question would be to find such algorithm.


It is also interesting to remark that Theorems~\ref{thm:haah-construction},~\ref{thm:haah-completion} indeed allow the construction for any $\phi$, even complex-valued, provided that its absolute value is reciprocal.

One could think that, in Section~\ref{sec:prop-sampling}, instead of using the linear function in the phase-extraction subroutine, we could approximate the square root and then apply the transformation directly on $e^{i \pi c(x)}$. However, in the case of proportional sampling this would be inconvenient, as the derivative of the square root function has a discontinuity with an infinite jump around 0, and we could not choose a constant $\delta$ if we had values of the oracle that are too close to $0$.

{\small
\bibliographystyle{ieee_fullname}
\bibliography{egbib}
}

%%%%%%%%%%%%%%%% SUPPL %%%%%%%%%%%%%%%%%%%%%%%%%%%%%%
%\title{\textbf{Supplementary Material\\
%Fairness meets Cross-Domain Learning: 
%\\ a new perspective on Models and Metrics}}

%\author{}
%\date{}
%
%\maketitle
% Remove page # from the first page of camera-ready.
%\ificcvfinal\thispagestyle{empty}\fi

\newpage 
\appendix

%%%%%%%%% BODY TEXT

\setcounter{section}{0}
\setcounter{table}{0}
\setcounter{figure}{0}

%\section{Code}
%We include our PyTorch code as supplementary material. We recommend following the instructions of the \textit{README.md} file to run it. We acknowledge the authors of \cite{tllib} for providing the code and library on which we based the implementation for our landmark detection experiments.

% IMPLEMENTATION DETAILS:

%% Classification
%%% Hyper-parameter search

\section{Implementation Details}

\subsection{Classification}
For all the experiments we follow \cite{zietlow2022leveling} in terms of base architecture, training details, and validation protocol. In particular, all the methods are built upon the ImageNet pre-trained ResNet-50 \cite{he2016deep} backbone optimized with Adam ($lr=10^{-4}$, batch size 64). 
As data augmentation, we use a center crop to 128x128 and RandAugment with $N=3$ and $M=15$. The validation is done every 500 iterations and the best model is selected based on the best $mGA$ computed on the validation set. Note that for g-SMOTE \cite{zietlow2022leveling} we used the GAN inversion model provided in  \cite{dinh2021hyperinverter}, pre-trained on CelebA: the official GAN code and weights used in \cite{zietlow2022leveling} have not been released by the authors.
Although there may be some debate around the use of generative approaches that are not tailored specifically to the medical task at hand, we decided to incorporate g-SMOTE into both the experiments on the COVID-19 Chest X-Ray and Fitzpatrick17k datasets for completeness.



We perform an extensive hyper-parameters search to find the best models for every approach considered in our benchmark. In particular, we apply the Random Search \cite{bergstra2012random} algorithm followed by a refinement stage in the following hyper-parameter intervals:

\smallskip\noindent \textbf{LSR} \cite{szegedy2016rethinking}: $\varepsilon$ is the coefficient used to smooth the ground truth labels such that $y_k^{LS}=y_k(1-\varepsilon)+\varepsilon/K$, where $K$ indicates the number of classes. $ \left \{ \varepsilon \in [0.1, 0.5] \right \} $ ;

\smallskip\noindent  \textbf{SWAD} \cite{cha2021swad}: $r$ is the tolerance rate used on the validation loss function when searching the interval in which the model's parameters have to be sampled and averaged. We didn't tune the optimum patience ($N_e$) and the overfit patience ($N_s$) since the overfitting behavior could be observed already after the very first validation. $ \left \{ r \in [0.1, 1.3] \right \}$ ;
 
\smallskip\noindent \textbf{RSC} \cite{huang2020self}: $f$ indicates the dropping percentage to mute the spatial feature maps, $b$ indicates the percentage of the batch on which RSC is applied. $ \left \{  f \in [10, 80] ,  b \in [10, 80] \right \}$ ;

\smallskip\noindent\textbf{L2D} \cite{lee2021learning}: $\alpha_1$ weights the contribution of the supervised contrastive loss function and $\alpha_2 $ weights the negative log-likelihood between the latent vectors of the source image $x$ and the generated one $x^+$ in the final objective function. $ \left \{  \alpha_1 \in [0.1, 3.0], \alpha_2 \in [0.1, 3.0] \right \} $ ;

\smallskip\noindent \textbf{DANN} \cite{ganin2016domain}, \textbf{CDANN} \cite{li2018deep}: $\lambda$ is the hyper-parameter that weights the reverse gradient during the backpropagation step, $\gamma$ controls the penalty assigned to the norm computed on the gradients of the domain discriminator.  $ \left \{ \lambda \in [0.01, 1.00], \gamma \in [0.01, 0.50] \right \} $ ;
 
\smallskip\noindent\textbf{SagNets} \cite{nam2021reducing}: $\lambda$ weights the adversarial loss function. $  \left \{ \lambda \in [0, 2] \right \} $ ;

\smallskip\noindent \textbf{AFN} \cite{xu2019larger}:  $\lambda$ trades off the feature-norm penalty and the supervised cross-entropy loss, $R$ is the value at which the norms of the extracted features are forced to converge to. $ \left \{ \lambda \in [0.01, 0.10], R \in [5, 100] \right \}$ ;
 
\smallskip\noindent \textbf{MMD} \cite{li2018domain}: $\gamma$ weights the MMD loss term in the final objective. $ \left \{ \gamma \in [0.1, 1.0] \right \} $ ;
 
\smallskip\noindent \textbf{Fish} \cite{shi2021gradient}: $\eta$ weights the gradient inner product.  $ \left \{  \eta \in [0.01, 0.10] \right \}$ ;

\smallskip\noindent \textbf{RelRot},  \textbf{RelRotAlign} \cite{bucci2020effectiveness}: $\alpha$ weights the importance of the self-supervised loss function in the total objective. $ \left \{  \alpha \in [0.1, 1.0] \right \} $ ;

\smallskip\noindent \textbf{SelfReg} \cite{kim2021selfreg}: $\lambda_{feature}$ and $\lambda_{logit}$ control, respectively, the in-batch dissimilarity losses applied to the intermediate features and the logits from the classifier. $ \left \{  \lambda_{feature} \in [0.1, 1.0], \lambda_{logit} \in [0.1, 1.0] \right \} $ ;
 
\smallskip\noindent \textbf{GroupDRO} \cite{sagawa2019distributionally}: $C$ is a model capacity, $\eta$ is the step size to update the weights in order to balance worst and best performing groups. $ \left \{  \eta \in [0.001, 0.05],  C \in [1, 10] \right \} $ ;

\smallskip\noindent \textbf{g-SMOTE} \cite{zietlow2022leveling}: $m$ in the number of nearest neighbors considered, $k$ is the number of random points chosen among the $m$ and $\lambda$ is the probability of selecting a batch from the original dataset during training. $ \left \{ m \in [2, 10], k \in [2, m], \lambda \in [0.1, 1.0] \right \}$ .

%% Landmark Detection
%%% Hyper-parameter search

\subsection{Landmark Detection}

Throughout our experiments, we adopt the architecture and training procedures outlined in \cite{jiang2021regda}. To ensure consistency, we also use the validation protocol proposed in \cite{zietlow2022leveling}. Our approach employs an ImageNet pre-trained ResNet-18 \cite{he2016deep} backbone, followed by an upsampling head consisting of three 2D transposed convolutions with a dimension of 200 and a kernel size of 4. This head performs heatmap regression to determine the position of each landmark, resulting in an output tensor $\mathbf{\hat{Y}} \in \mathbb{R}^{200 \times 200 \times 68}$.
Our network is optimized using stochastic gradient descent (SGD) with a learning rate of 0.1, momentum of 0.9, weight decay of 1e-4, and a batch size of 32 for 35000 iterations. We incorporate a multi-step learning rate decay with a decay factor of 0.1, using iteration 22500 and 30000 as milestones.
To apply several augmentation sequentially we use the TorchLM library\footnote{\url{https://github.com/DefTruth/torchlm}}.
The augmentations are: random rotation (with angles ranging from -180 to 180 degrees), random horizontal flip (with a probability of 0.5), random shear (with x and y rescale factors of 0.6 and 1.3, respectively), color jitter (with brightness, contrast, and saturation set to 0.24, 0.25, and 0.25, respectively) and Gaussian blur (with a kernel size of 5 and $\sigma=(0.1, 0.8)$). We validate every 500 iterations and select the best model based on the highest $mGS$ score on the validation set.


We conduct an exhaustive search for optimal hyperparameters for all the approaches included in our benchmark. Specifically, we employ the Random Search algorithm \cite{bergstra2012random}, followed by a refinement stage, within the hyperparameter intervals as specified below:

\smallskip\noindent \textbf{AFN} \cite{xu2019larger}:  $\lambda$ trades off the feature-norm penalty and the supervised cross-entropy loss, $R$ is the value at which the norms of the extracted features are forced to converge to. $ \left \{ \lambda \in [1e-6, 0.10], R \in [5, 100] \right \}$ ;

\smallskip\noindent \textbf{DANN} \cite{ganin2016domain}: $\lambda$ is the hyper-parameter that weights the reverse gradient during the backpropagation step, $\gamma$ controls the penalty assigned to the norm computed on the gradients of the domain discriminator.  $ \left \{ \lambda \in [1e-6, 1.00], \gamma \in [0.01, 0.50] \right \} $;

\smallskip\noindent \textbf{RegDA} \cite{jiang2021regda}: \emph{margin} trades off between the KL divergence loss and the Regression Disparity loss. $t$ is a modifier of the magnitude of the Regression Disparity loss. $ \left \{ margin \in [1.0, 10.0], t \in [0.01, 1.0] \right \} $.

% EXTRA EXPERIMENTS:
\input{exp_classification+landmark}


\end{document}