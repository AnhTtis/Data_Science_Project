In this paper we proposed an extensive study on the problem of fairness in computer vision by presenting a new benchmark %that allows 
to assess the performance of cross-domain learning approaches for unfairness mitigation. Moreover, we introduced the task of landmark detection in the fairness research area and  proposed the Harmonic Fairness. This new metric takes into account both the accuracy and the degree of fairness of a model to evaluate its effectiveness. Although our focus is mainly on group fairness and other definitions are possible, we believe that our work provides several tools to broaden the study of fairness-related issues and solutions in AI. 

\smallskip\noindent\textbf{Acknowledgments.} Computational resources were provided by IIT (HPC infrastructure) and CINECA through the IsC98\_FA-DA project under the ISCRA initiative. L.I. acknowledges the grant received from the European Union Next-GenerationEU (Piano Nazionale di Ripresa E Resilienza (PNRR)) DM 351 on Trustworthy AI. S. B. acknowledges the travel grant provided by the Blanceflor Foundation.  This work has been also partially supported (T.T.) by the EU project ELSA - European Lighthouse on Secure and Safe AI (\url{https://www.elsa-ai.eu/}). 