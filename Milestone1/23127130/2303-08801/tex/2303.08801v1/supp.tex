\documentclass[aps,onecolumn,superscriptaddress,amsmath,amssymb,floatfix]{revtex4-2}
\pdfoutput=1

\usepackage[utf8]{inputenc}
\usepackage{graphicx}
\usepackage{dcolumn}
\usepackage{bm}
\usepackage{hyperref}

\renewcommand{\thefigure}{S\arabic{figure}}
\renewcommand{\theequation}{S\arabic{equation}}

\begin{document}

\title{Supplemental Material for ``Slowest and Fastest Information Scrambling in the Strongly Disordered XXZ Model''}

\author{Myeonghyeon Kim}
\affiliation{Department of Physics and Astronomy, Seoul National University, Seoul 08826, Korea}

\author{Dong-Hee Kim}
\email{dongheekim@gist.ac.kr}
\affiliation{Department of Physics and Photon Science, Gwangju Institute of Science and Technology, Gwangju 61005, Korea}
\affiliation{School of Physics, Korea Institute for Advanced Study, Seoul 02455, Korea}

\maketitle

\section{Numerical arbitrary-order perturbation calculations}

The energy correction of an arbitrary order 
in the Rayleigh-Schr\"odinger perturbation theory (RSPT)
appears in various forms in the literature \cite{Kato1949,Bloch1958,Huby1961} 
(see also references in \cite{Silverstone1970}).
For our case with zero diagonals in the perturbation matrix, 
only the even-order terms survive in RSPT, which can be written 
in a recursive form as
\begin{equation} \label{eq:RSPT}
    E^{(2n)}_\lambda = \big\langle \lambda \big| \hat{V} \big( \hat{P}_\lambda\hat{V} \big)^{2n-1} \big| \lambda \big\rangle 
    + \sum^{n-1}_{m=1} (-1)^m \sum^{n-1}_{k=m}
    \left[\,\, 
    \sideset{}{'}\sum_{(z_1, \ldots, z_m)} \prod^m_{i=1} E^{(2z_i)}_\lambda 
    \right]
    \left[\,\,\,
    \sideset{}{''}\sum_{\substack{(z_1, \ldots, z_q) \\ q \equiv 2n-2k-1}}
    \big\langle \lambda \big| \hat{V} \prod_{i=1}^q \big[ \hat{P}_\lambda^{1+z_i} \hat{V} \big] \big| \lambda \big\rangle
    \right]\,,
\end{equation}
where $|\lambda\rangle$ denotes an unperturbed eigenstate, and the operator $\hat{P}_\lambda = \sum_{\lambda' \ne \lambda}\frac{|\lambda'\rangle\langle\lambda'|}{E^{(0)}_\lambda - E^{(0)}_{\lambda'}}$. 
The primed sum $\sum^\prime$ indicates
the sum over the sequences of $m$ natural numbers $(z_1, z_2, \ldots, z_m)$ 
satisfying the constraint $\sum_{i=1}^m z_i = k$.
The double-primed sum $\sum^{\prime\prime}$ indicates the sum over
the sequences of $q \equiv 2n-2k-1$ non-negative integers 
$(z_1, z_2, \ldots, z_q)$ satisfying the constraint $\sum_{i=1}^q z_q = m$.

\section{Numerical calculation details}

Let us begin with a few remarks on the numerical implementation of Eq.~\eqref{eq:RSPT}. 
First, the number of terms to be computed increases combinatorially
as it goes to the higher order, which practically limits 
the order that one can reach within a reasonable computational time.
Second, round-off errors are often dangerous when computing  
$\Delta E_\alpha - \Delta E_\beta + \Delta E_\gamma - \Delta E_\delta$ 
since the difference between those energy corrections
is much smaller than the magnitude of $\Delta E$.
The difference decreases exponentially with increasing $r$ at the lowest order
in the deep MBL regime that we consider, and thus extremely accurate
calculations are necessary to cope with strong cancellations occurring 
in the perturbation expansion.
To prevent these issues, our numerical calculations keep at least $500$ 
decimal digits by using the multiprecision \texttt{MPFR} library \cite{MPFR}.
Third, the perturbation matrix $\langle \lambda | \hat{V} | \lambda' \rangle$ 
has a highly sparse structure, which helps us to accelerate 
the evaluation of Eq.~\eqref{eq:RSPT} 
by using fast sparse matrix-vector multiplications.

The use of the \texttt{MPFR} library is not limited to the evaluation 
of Eq.~\eqref{eq:RSPT}. The numerical precision of $500$ decimal digits
is applied to all calculations including the exact diagonalization and
the numerical evaluations of $J_\mathrm{eff}$ for the slowest
and fastest scrambling to prevent the same accuracy issues.

The random disorder configurations are generated using the 64-bit version of 
the Mersenne Twister pseudorandom number generator \cite{MT19937} and 
its multiprecision extension implemented in the \texttt{boost} library. 
The number of disorder configurations used for statistics 
in our numerical calculations is listed as follows.
In Fig.~1 and Fig.~3 of the main paper, we use $10^4$ disorder configurations 
for the exact diagonalization calculations and $10^6$ configurations
for the perturbation calculations.
In Fig.~4, for the 2D FM state, the number of disorder configurations used in
the lowest-order perturbation calculations is $10^6$ for $L \le 12$, 
$5\times 10^5$ for $L=14$, and $5\times 10^4$ for $L=16$.
The exact diagonalization results are obtained with $10^4$ configurations.
For the 2D AF state, the number of disorder configurations used 
in the lowest-order calculations is $10^6$ for $L \le 6$, 
$2\times 10^5$ for $L = 7$, $10^5$ for $L=8$, and $10^4$ for $L = 9$
while the higher-order full RSPT calculations are done for $10^4$ configurations.


\section{Derivation of the lowest-order term for the slowest scrambling}

In this section, we focus on the ferromagnetic unperturbed eigenstate 
$|\alpha \rangle = |\cdots00_00_10_2\cdots0_r0_{r+1}0\cdots\rangle$ 
corresponding to the upper light cone of the slowest scrambling, 
where $|0\rangle$ and $|1\rangle$ denote $|\uparrow\rangle$ and 
$|\downarrow\rangle$ in the $\sigma^z$-basis, respectively.
The subscript denotes the site index of the intervening spin 
segment in the XXZ chain, where $0$ and $r+1$ are the locations of 
$\hat{\sigma}^x_a$ and $\hat{\sigma}^x_b$ operators defining OTOC. 
For this particular state with the spin-polarized domain, it is possible 
to derive the closed-form lowest-order expression of the effective interaction
$J_\mathrm{eff}^\mathrm{FM} \equiv \Delta E_\alpha - \Delta E_\beta + \Delta E_\gamma - \Delta E_\delta$ 
by diagrammatically evaluating the energy correction terms in RSPT \cite{SzaboBook}. 
Below we present our derivation of the lowest $(2r)^\mathrm{th}$ order perturbation
correction of $E_\beta$, $E_\gamma$, and $E_\delta$. Note that 
$\Delta E_\alpha = 0$ because $\hat{V}|\alpha\rangle$ = 0.
For a more intuitive description, let us call the spin excitation $|1\rangle$ just as a particle.

\subsection{single excitation}

The unperturbed eigenstates $|\beta\rangle$ and $|\delta\rangle$ 
with single excitation from $|\alpha\rangle$ created by $\hat{\sigma}^x_0$ 
and $\hat{\sigma}^x_{r+1}$ are given as 
\begin{eqnarray}
|\beta\rangle &=& \hat{\sigma}^x_0 |\alpha \rangle = |1_0 0_1 0_2 \cdots 0_{r-1} 0_r 0_{r+1}\rangle\,, \\
|\delta\rangle &=& \hat{\sigma}^x_{r+1} |\alpha \rangle = |0_0 0_1 0_2 \cdots 0_{r-1} 0_r 1_{r+1}\rangle\,,
\end{eqnarray}
where the configuration in the outside of the region $[0, r+1]$ is omitted 
because it is irrelevant to the lowest-order perturbation expression of 
$J_\mathrm{eff}^\alpha \equiv \Delta E_\alpha - \Delta E_\beta + \Delta E_\gamma - \Delta E_\delta$. 

\begin{figure}
    \centering
    \includegraphics[width=0.85\textwidth]{supp_fig1.pdf}
    \caption{Diagrams of (a) $E^{(2r)}_\beta$ 
    and (b) $E^{(2r)}_\delta$ giving the nonzero lowest-order contribution to 
    $\Delta E_\alpha - \Delta E_\beta + \Delta E_\gamma - \Delta E_\delta$ 
    for the ferromagnetic state $|\alpha\rangle$ corresponding to the slowest scrambling.
    }
    \label{fig:sfig1}
\end{figure}

The ``one-hole'' diagrams~\cite{SzaboBook} shown in Fig.~\ref{fig:sfig1} represents 
the perturbation corrections $\Delta E_\beta$ and $\Delta E_\delta$ 
of the order $2r$, respectively. By following the recipe in Ref.~\cite{SzaboBook},
one can write down the evaluation results as
\begin{eqnarray}
    \Delta E_\beta^{(2r)} &=& 2 \left( \frac{J}{2}\right)^{2r} \frac{1}{A_1 A_2 \cdots A_{r-1} A_r A_{r-1} \cdots A_2 A_1}\,, \label{eq:Eb}\\
    \Delta E_\delta^{(2r)} &=& 2 \left( \frac{J}{2}\right)^{2r} \frac{1}{B_r B_{r-1} \cdots B_2 B_1 B_2 \cdots B_{r-1} B_r}\,, \label{eq:Ed}
\end{eqnarray}
where $A_i \equiv h_0 - h_i$ and $B_i \equiv h_{r+1} - h_i$.

\subsection{double excitation}

For the state $|\gamma\rangle = \hat{\sigma}^x_{r+1}\hat{\sigma}^x_0|\alpha\rangle = |1_0 0_1 0_2 \cdots 0_r 1_{r+1}\rangle$
with double excitation, there are many lowest-order diagrams 
contributing to $J_\mathrm{eff}^\mathrm{FM}$ as the two particles ($1$'s)
moving through the intervening region generates many different sequences
of intermediate states. These diagrams can be categorized 
into three types. Below we describe the evaluation of each type of diagram.

First, there are two one-hole diagrams for the special cases 
where one of the two particles is fixed and does not move. The diagrams are 
essentially the same as the ones given in Fig.~\ref{fig:sfig1}, 
while $0_{r+1}$ should be replaced with $1_{r+1}$ in Fig.~\ref{fig:sfig1}(a) 
and, $0_0$ should be replaced with $1_0$ in Fig.~\ref{fig:sfig1}(b).
By modifying Eqs. \eqref{eq:Eb} and \eqref{eq:Ed},
these two one-hole diagrams are evaluated straightforwardly as
\begin{eqnarray}
    2 \left( \frac{J}{2}\right)^{2r}
    \frac{1}{(A_r - J_z)\prod_{i=1}^{r-1} A_i^2}\,, \label{eq:EcA} \\
    2 \left( \frac{J}{2}\right)^{2r}
    \frac{1}{(B_1 - J_z)\prod_{i=2}^{r} B_i^2}\,, \label{eq:EcB}
\end{eqnarray}
where the interaction $J_z$ appears because the flow in the diagram goes 
through the ``interacting'' intermediate state where the particle from one end 
travels along the chain all the way to meet the fixed particle in the other end.

Second, there are more general cases where two particles move around
and meet together in the middle of the intervening region, generating
many one-hole diagrams with an intermediate state such as
$|0\cdots 0_{k-1} 1_k 1_{k+1} 0_{k+2} \cdots 0 \rangle$.
The ``particle'' line~\cite{SzaboBook} of those diagrams comes in three parts. 
In the lower part of the diagram, the two particles move towards
to the $k$ and $k+1$ sites, they meet at the intermediate state
in the middle part of the diagram, and then they depart from
each other and come back to the starting point $|\gamma\rangle$ in the upper part.
The complication in the evaluation of such diagrams is due to the fact
that one has to add all combinations of the sequences of 
the two-particle configurations, which we denote by the double lines
in Fig.~\ref{fig:sfig2}. 

\begin{figure}
    \centering
    \includegraphics[width=0.95\textwidth]{supp_fig2.pdf}
    \caption{Diagrams for the lowest-order contributions of 
    $E_\gamma$ to $J_\mathrm{eff}^\mathrm{FM}$. 
    (a) Double-lined one-hole diagram with the interacting 
    intermediate state $|\cdots 1_k 1_{k+1} \cdots\rangle$. 
    (b) Example of the double line made of the addition of the two previous branches
    in the hierarchy. 
    (c) Example of the hierarchy to reach the intermediate state $|001100\rangle$
    from $|\gamma\rangle = |\cdots 100001 \cdots\rangle$. The left (right) arrows
    indicate the movement of the left (right) spin.
    (d) Evaluation of the sum of all the other diagrams that do
    not involve the interacting intermediate states.
    }
    \label{fig:sfig2}
\end{figure}

The double line can be evaluated recursively. The summation of 
the partial diagrams are done along the tree-like hierarchy of 
the two-particle movements. Each intermediate state is reached through 
the two branches from the previous layer, unless one of the particles
still stays at the initial site, as each branch allows one particle
to move from either the left-hand side or the right-hand side. 
An example of the addition of the two branches
is shown in Fig.~\ref{fig:sfig2}(b), which occurs through
the hierarchical graph as exemplified in Fig.~\ref{fig:sfig2}(c). 
Summing up the partial diagrams at every vertex recursively 
along the graph, one can find that the movements of the left 
and right particles factorize after the summation, where a single
move creates the factor $1/A$ (left-particle) or $1/B$ (right-particle).
The double line in Fig.~\ref{fig:sfig2}(a) is then proportional to
\begin{equation*}
    \frac{A_k + B_{k+1}}{\prod_{i=1}^k A_i \prod_{i=k+1}^r B_i}.
\end{equation*}
The same expression can also be obtained for the upper double line 
as we start the summation from the upper end of the diagram.
Therefore, the diagram with the intermediate state 
$|\cdots 1_k 1_{k+1} \cdots \rangle$ in Fig.~\ref{fig:sfig2}(a) 
is written as
\begin{equation} \label{eq:EcK}
    2 \left( \frac{J}{2}\right)^{2r} \cdot 
    \frac{A_k + B_{k+1}}{\prod_{i=1}^k A_i \prod_{i=k+1}^r B_i}
    \cdot \frac{1}{A_k + B_{k+1} - J_z} \cdot
    \frac{A_k + B_{k+1}}{\prod_{i=1}^k A_i \prod_{i=k+1}^r B_i}\,.
\end{equation}

Last, we also need to take care of the other diagrams that do not visit
such an ``interacting" intermediate state $|\cdots 1_k 1_{k+1} \cdots\rangle$. 
These diagrams describe the cases where the two particles never come 
together to the neighboring sites and thus are independent of $J_z$.
Although the recursive summation described above cannot be used
for the direct summation of such non-interacting diagrams,  
the summation can still be evaluated using the following trick.

Let us prepare a system with the bond cut between the sites 
$k$ and $k+1$ as sketched in Fig.~\ref{fig:sfig2}(d), which 
separates the Hilbert space of the system into two parts of
the left and right chains.
The essential point of introducing this tweak is 
that the new system still produces the same non-interacting 
diagrams as the original interacting system does, 
while the sum over all the diagrams is just zero in the new system 
because the lack of coupling between $k$ and $k+1$ prohibits
the perturbation operators of the left and right regions appearing 
together in the operator string in the perturbation expansion. 
Therefore, we write down the diagrammatic equation shown 
in Fig.~\ref{fig:sfig2}(d) by using the evaluation of the diagram 
in Fig.~\ref{fig:sfig2}(a) in the setting of $J_z=0$ for its 
non-interacting cousin appearing as the first term on the right-hand side.
The second term ``(other diagrams)'' on the right-hand side is the non-interacting diagrams 
that we want to evaluate. 
The sum of the non-interacting diagrams is then given as
\begin{equation} \label{eq:EcNint}
    - 2 \left( \frac{J}{2}\right)^{2r} \cdot
    \frac{A_k + B_{k+1}}{\prod_{i=1}^k A_i^2 \prod_{i=k+1}^r B_i^2}.
\end{equation}
Adding up Eqs. (\ref{eq:EcA}-\ref{eq:EcNint}) for all $k$'s, we obtain 
the lowest-order expression of $\Delta E_\gamma$ as
\begin{equation*}
    \Delta E^{(2r)}_\gamma = 2 \left( \frac{J}{2}\right)^{2r}
    \left[
    \frac{1}{(A_r - J_z)\prod_{i=1}^{r-1} A_i^2}
    + \frac{1}{(B_1 - J_z)\prod_{i=2}^{r} B_i^2}
    + \sum_{k=1}^{r-1} \frac{A_k + B_{k+1}}{\prod_{i=1}^k A_i^2 \prod_{i=k+1}^r B_i^2} 
    \left( \frac{J_z}{A_k + B_{k+1} - J_z} \right) 
    \right].
\end{equation*}
Finally, we write $J_\mathrm{eff}^\mathrm{FM}(r) \equiv \Delta E_\alpha^{(2r)} - \Delta E_\beta^{(2r)} + \Delta E_\gamma^{(2r)} - \Delta E_\delta^{(2r)}$ as 
\begin{equation*}
    J_\mathrm{eff}^\mathrm{FM}(r) = 
    2 \left( \frac{J}{2}\right)^{2r} \cdot J_z \cdot 
    \Bigg[
    \frac{1}{A_r(A_r - J_z)\prod_{i=1}^{r-1} A_i^2}
    + \frac{1}{B_1(B_1 - J_z)\prod_{i=2}^{r} B_i^2}
    + \sum_{k=1}^{r-1}
    \frac{A_k + B_{k+1}}{(A_k + B_{k+1} - J_z)\prod_{i=1}^k A_i^2 \prod_{i=k+1}^r B_i^2}
    \Bigg].
\end{equation*}
Note that the formula of $J_\mathrm{eff}^\mathrm{FM}$ in the main text 
is given in the shorter form by defining an empty product as unity. 



\begin{thebibliography}{}

\bibitem{Kato1949}
T. Kato, Prog. Theor. Phys. \textbf{4}, 514 (1949).
%https://doi.org/10.1143/ptp/4.4.514

\bibitem{Bloch1958}
C. Bloch, Nucl. Phys. \textbf{6}, 329 (1958).
%https://doi.org/10.1016/0029-5582(58)90116-0

\bibitem{Huby1961}
R. Huby, Proc. Phys. Soc. \textbf{78}, 529 (1961).
%https://doi.org/10.1088/0370-1328/78/4/306

\bibitem{Silverstone1970}
H. J. Silverstone and T. T. Holloway, J. Chem. Phys. \textbf{52}, 1472 (1970).
%https://doi.org/10.1063/1.1673153

\bibitem{MPFR}
L. Laurent, G. Hanrot, V. Lef\`{e}vre, P. P\'{e}lissier, and P. Zimmermann,
ACM Trans. Math. Softw. \textbf{33}, 13 (2007); publicly available at \url{https://www.mpfr.org}.
%https://doi.org/10.1145/1236463.1236468

\bibitem{MT19937}
M. Matsumoto and T. Nishimura,
ACM Trans. Model. Comput. Simul. \textbf{8}, 3 (1998).
%https://doi.org/10.1145/272991.272995

\bibitem{SzaboBook}
A. Szabo and N. S. Ostlund, \textit{Modern Quantum Chemistry} (Dover, New York, 1996), Ch. 6.

\end{thebibliography}
\end{document}