\documentclass[reprint,superscriptaddress,amsmath,amssymb,prl,floatfix]{revtex4-2}
\pdfoutput=1

\usepackage[utf8]{inputenc}
\usepackage{graphicx}
\usepackage{dcolumn}
\usepackage{bm}
\usepackage{hyperref}
\usepackage{mathtools}

\hyphenation{LIOM}
\hyphenation{LIOMs}
\hyphenation{OTOC}

\begin{document}

\title{Slowest and Fastest Information Scrambling in the Strongly Disordered XXZ Model}

\author{Myeonghyeon Kim}
\affiliation{Department of Physics and Astronomy, Seoul National University, Seoul 08826, Korea}

\author{Dong-Hee Kim}
\email{dongheekim@gist.ac.kr}
\affiliation{Department of Physics and Photon Science, Gwangju Institute of Science and Technology, Gwangju 61005, Korea}
\affiliation{School of Physics, Korea Institute for Advanced Study, Seoul 02455, Korea}

%\date{\today}

\begin{abstract}
We present a perturbation method to compute the out-of-time-ordered correlator in the strongly disordered Heisenberg XXZ model in the deep many-body localized regime. We characterize the discrete structure of the information propagation across the eigenstates, revealing a highly structured light cone confined by the strictly logarithmic upper and lower bounds representing the slowest and fastest scrambling available in this system. We explain those bounds by deriving the closed-form expression of the effective interaction for the slowest scrambling and by constructing the effective model of a half-length for the fastest scrambling. We extend our lowest-order perturbation formulations to the higher dimensions, proposing that the logarithmic upper and lower light cones may persist in two dimensions in the limit of strong disorder and weak hopping.
\end{abstract}

\maketitle

Slow scrambling of quantum information is one of the intriguing 
phenomena occurring in many-body localized (MBL) systems
\cite{review2015,review2018a,review2018b,review2019,review2020}. 
The time scale of scrambling dynamics~\cite{Xu2022} 
in MBL systems is distinguished from Anderson localization 
in non-interacting systems where correlation decays exponentially 
\cite{Burrell2007,Hamza2012}
and also from the fast scrambling expected in ideal chaotic systems
\cite{Maldacena2016,Roberts2016,Gu2017}.
The logarithmic time scale of information propagation was 
firstly reported by the growth of entanglement entropy 
in the disordered XXZ chain quenched from a product state
\cite{Chiara2006,Znidaric2008,Bardarson2012},
which was explained in the picture of 
the quasi-local integral of motion (LIOM) 
%and effective interactions between LIOMs
\cite{Serbyn2013a,Vosk2013,Serbyn2013b,Huse2014}.
The Lieb-Robinson bound indicating the upper bound on 
information propagation speed was modified accordingly
in this picture, 
proposing the logarithmic light cone (LLC) of 
the information front moving at a finite speed defined 
in logarithmic time instead of linear time
\cite{Kim2014,Fan2017,Swingle2017,ChenY2017,ChenX2017,He2017}.

Despite the numerical evidence of LLC found in MBL systems
\cite{Deng2017,He2017,Huang2017,Banuls2017,Tomasi2019,Kim2021},
a basic understanding of LLC tends to rely on
the effective l-bit Hamiltonian \cite{Serbyn2013b,Huse2014}.
The hypothesized exponentially decaying effective interaction
$J_\mathrm{eff}(r) \propto \exp(-r/\xi)$ 
acting on two remote LIOMs at distance $r$ is a key to interpret 
the time scale $t \sim 1/J_\mathrm{eff}$ exponentially increasing with $r$. 
Although this is well established to describe the dephasing dynamics 
in one dimension (1D), the effective picture lacks 
the system-specific details that can still be necessary for understanding 
of the phenomena in a particular system.

On the other hand, practical signatures of MBL 
in two dimensions (2D) have attracted much attention theoretically 
\cite{Wahl2019,Kshetrimayum2020,Theveniaut2020,Szabo2020,Chertkov2021,Tomasi2019,Kim2021,Pietracaprina2021,Decker2022,Doggen2022,Agrawal2022,Strkalj2022,Tang2021,Li2021,Foo2022,Venn2022}
and experimentally \cite{Choi2016,Kondov2015,Bordia2017}
at finite systems,
while it has been argued that 2D MBL is asymptotically unstable
toward the avalanche of rare thermal regions
\cite{DeRoeck2017,Potirniche2019,Gopalakrishnan2019,Doggen2020}.
In particular, the evidence of LIOMs \cite{Chertkov2021} and 
LLC \cite{Kim2021} has been recently presented 
in higher dimensions by the numerical construction of the l-bit Hamiltonian.
These motivate us to revisit the computation of 
the out-of-time-ordered correlator (OTOC) \cite{Xu2022}, 
a diagnostic tool for information scrambling, 
for characterization beyond the generic l-bit description both in 1D and 2D.

\begin{figure}[b]
    \centering
    \includegraphics[width = 0.48\textwidth]{fig1.pdf}
    \caption{Light cone structure of the disordered XXZ chain
    in the deep MBL regime. 
    OTOC is computed for each eigenstate at 
    $J/J_z = 0.001$ and $h/J_z = 10$. 
    Markers denote the scrambling time $J_z t^*$ 
    at which $\langle C_\alpha(r,t^*) \rangle_\mathrm{av} = 0.5$,
    comparing the lowest-order perturbation results 
    with the exact diagonalization.
    The arrows indicate the allowed change of $t^*$ 
    with increasing separation. The infinite-temperature 
    OTOC $\overline{C}_\infty$ is also given for comparison.
    }
    \label{fig1}
\end{figure}

In this paper, we develop a perturbation formulation of OTOC 
in the strongly disordered XXZ model in the weak hopping limit. 
Measuring OTOC for each eigenstate, we find out the discrete structure of 
the light cone built by the allowed lowest orders of perturbation 
varying with the intervening spin states at a given $r$. 
Remarkably, the light cone is bounded by the two logarithmic slopes 
representing the slowest and fastest scrambling. 
We derive an analytic formula for the effective interaction for 
the slowest scrambling and describe the fastest scrambling by 
the half-length effective Ising chain. Extending our method to 2D, 
we demonstrate the logarithmic light cones of the slowest and fastest scrambling 
in 2D within the lowest-order perturbation formulations.

For perturbation expansion, we decompose the XXZ Hamiltonian 
as $\hat{H} \equiv \hat{H}_0 + \hat{V}$,
where the unperturbed part $\hat{H}_0$ and  
the hopping perturbation $\hat{V}$ are given as
\begin{eqnarray}
    \hat{H}_0 &=& \frac{J_z}{2} \sum_i \hat{\sigma}^z_i \hat{\sigma}^z_{i+1} 
    + \sum_i h_i \hat{\sigma}^z_i ,\\
    \hat{V} &=& J \sum_i \left( \hat{\sigma}^+_i \hat{\sigma}^-_{i+1} 
    + \hat{\sigma}^-_i \hat{\sigma}^+_{i+1} \right) .
\end{eqnarray}
The random disorder field is drawn 
from the uniform distribution of $h_i \in [-h, h]$. 
We assume that the unperturbed state is nondegenerate
and localized in the Fock space of the $\hat{\sigma}_z$-basis states. 
We consider the strong disorder and weak hopping limit of 
$J \ll J_z \ll h$ in the deep MBL regime. 
We compute the perturbation corrections in energy
within the Rayleigh-Schr\"odinger perturbation theory
using multi-precision numerics to handle strong cancellations
and critical round-off errors \cite{sm}. 

We define OTOC by the squared commutator of 
two $\hat{\sigma}_x$ operators initially located at $a$ and $b$ as
\begin{equation} \label{eq:otoc_def}
    C_\alpha(r,t) = \frac{1}{2} \langle \alpha |
    \left|[\hat{\sigma}^x_a(t), \hat{\sigma}^x_b]\right|^2 
    | \alpha \rangle 
    = 1 - \mathrm{Re} [F_\alpha(r, t)], 
\end{equation}
where the correlator 
$F_\alpha(r, t) = \langle \alpha | \hat{\sigma}^x_a(t) \hat{\sigma}^x_b \hat{\sigma}^x_a(t) \hat{\sigma}^x_b | \alpha \rangle$,
and $r \equiv |a - b| - 1 \ge 0$ is separation
between $a$ and $b$.
Choosing $|\alpha\rangle$ to be an eigenstate,
the correlator can be approximated at weak perturbation as 
\begin{equation} \label{eq:otoc_approx}
    F_\alpha(r,t) = \sum_{\beta,\gamma,\delta} 
    s_{\alpha\beta\gamma\delta} 
    e^{i\Omega_{\alpha\beta\gamma\delta}t}
    \approx \exp(iJ_\mathrm{eff}^\alpha t),
\end{equation}
where the frequency 
$\Omega_{\alpha\beta\gamma\delta} =  E_\alpha - E_\beta + E_\gamma - E_\delta$, 
and the coefficient 
$s_{\alpha\beta\gamma\delta} = 
\langle \alpha| \hat{\sigma}^x_a |\beta \rangle
\langle \beta| \hat{\sigma}^x_b |\gamma \rangle
\langle \gamma| \hat{\sigma}^x_a |\delta \rangle
\langle \delta| \hat{\sigma}^x_b |\alpha \rangle$.
Assumed that the perturbation corrections
in the state vectors are small, the single dominant term 
is found at $s_{\alpha\beta\gamma\delta} \approx 1$ for 
$|\alpha\rangle\ \approx |\alpha^{(0)}\rangle$,
$|\beta\rangle \approx |\beta^{(0)}\rangle = \hat{\sigma}_a |\alpha^{(0)}\rangle$,
$|\gamma\rangle \approx |\gamma^{(0)}\rangle = \hat{\sigma}_b \hat{\sigma}_a |\alpha^{(0)}\rangle$, and 
$|\delta\rangle \approx |\delta^{(0)}\rangle = \hat{\sigma}_b |\alpha^{(0)}\rangle$, 
where the superscript denotes the corresponding unperturbed state.
The frequency of the dominant component is rewritten 
in terms of the perturbation corrections as
\begin{equation} \label{eq:Jeff_pt}
    J_\mathrm{eff}^\alpha = \Delta E_\alpha - \Delta E_\beta 
    + \Delta E_\gamma - \Delta E_\delta,
\end{equation}
which we referred to as an effective interaction from
the analogy to the one in $F(t) = \exp(\pm 4i J_\mathrm{eff} t)$ 
given for the effective l-bit model \cite{Fan2017,Swingle2017,ChenY2017,ChenX2017,He2017}.
The same expression of $J^\alpha_\mathrm{eff}$ can also be 
extracted using the protocol of the double electron-electron 
resonance (DEER) \cite{Serbyn2014,Kucsko2018,Varma2019,Chiaro2022}.
The disorder average is then given by the characteristic function 
for the distribution of $J_\mathrm{eff}^\alpha$ as
\begin{equation} \label{eq:OTOCav}
    \left\langle C_\alpha (r,t) \right\rangle_\mathrm{av} 
    \approx 1 - \mathrm{Re}
    \left[ 
    \int e^{itx} P_{J_\mathrm{eff}^\alpha}(x)\,\mathrm{d}x
    \right].
\end{equation}
In this weak perturbation formulation, only the energy corrections 
are important, while the small corrections in the state vectors are 
irrelevant, and measuring OTOC in the Fock space with
$|\alpha^{(0)}\rangle$ leads to the same expression.

\begin{figure}
    \centering
    \includegraphics[width = 0.48\textwidth]{fig2.pdf}
    \caption{Schematic diagrams of the lowest-order contributions to 
    the effective interactions. 
    (a) Slowest scrambling: the excitation moves sequentially 
    through a spin-polarized domain.
    (b) Fastest scrambling: the intervening area of staggered spin 
    pairs is reduced to the Ising chain of a half-length. 
    }
    \label{fig2}
\end{figure}

Figure~\ref{fig1} displays the scrambling time $t^*$ 
as a function of $r$ obtained by solving 
$\langle C_\alpha(r,t^*)\rangle_\mathrm{av} = 0.5$ 
for each eigenstate. It turns out that $t^*$ is not on a single light cone
but structured by the lowest order of the non-vanishing perturbation term 
in Eq.~\eqref{eq:Jeff_pt}, varying with
the intervening spin configuration in $|\alpha^{(0)}\rangle$.
The lowest order $n_\alpha(r)$ is determined by the minimum number 
of the hopping operators flipping all intervening spins, which is written as 
$n_\alpha(r) = 2 ( r - m^\alpha_s)$, where 
$m^\alpha_s$ is the number of staggered spin pairs
found in $|\alpha^{0}\rangle$ between $a$ and $b$. 

Remarkably, the discrete structure of $t^*$ indicates the
sharp upper and lower bounds in the logarithmic slope, 
representing the slowest and fastest scrambling available
in this system. 
These bounds correspond to $J_\mathrm{eff}^\alpha \propto J^{2r}$ 
($m^\alpha_s = 0$) and $J_\mathrm{eff}^\alpha \propto J^r$ 
($m^\alpha_s = r/2$) at even $r$, which are
associated with $|\alpha^0\rangle$ of the ferromagnetic (FM)
domain and the chain of staggered spin pairs 
such as in the antiferromagnetic (AF) state, respectively.
This structure is hidden in the infinite-temperature
OTOC, averaged over all eigenstates, revealing 
the detailed view of the light cone in the 
strongly disordered XXZ model.

For the slowest scrambling, we obtain the lowest-order expression of 
$J_\mathrm{eff}^\mathrm{FM}$ at the FM unperturbed state as
\begin{equation} \label{eq:Jeff_FM}
    J_\mathrm{eff}^\mathrm{FM} =  2 J_z \left(\frac{J}{2}\right)^{2r}
    \sum_{k=0}^{r} F_k^2 \frac{A_k + B_{k+1}}{A_k + B_{k+1} - J_z} G_{k+1}^2,
\end{equation}
where $A_k = h_{a} - h_{a+k}$, and $B_k = h_{b} - h_{a+k}$. 
The factors are given as $F_k = \prod_{j=1}^k A_j^{-1}$ 
and $G_k = \prod_{j = k}^r B_j^{-1}$, where an empty product is unity.
Note that a nonzero interaction $J_z$ is essential. 
While the detailed derivation is provided 
in Supplemental Material \cite{sm},
we sketch the lowest-order calculations 
in Fig.~\ref{fig2}(a), except for $\Delta E_\alpha = 0$,
illustrating $\Delta E_\beta$ and $\Delta E_\delta$ made of 
the sequential applications of the hopping operators
moving a single excitation and
$\Delta E_\gamma$ of the sum over all paths moving 
double excitation created at both ends.

\begin{figure}
    \centering
    \includegraphics[width = 0.48\textwidth]{fig3.pdf}
    \caption{Slowest and fastest scrambling in the lowest-order
    perturbation theory for the strongly disordered XXZ chain.
    The disorder average $\langle C(r,t) \rangle_\mathrm{av}$, 
    the distribution of $\xi^{-1} \equiv -r^{-1}\ln|\tilde{J}_\mathrm{eff}|$, 
    and the decay length $\langle \xi^{-1} \rangle_\mathrm{av}$ are computed 
    at $h/J_z = 20$ 
    for the (a)--(c) FM and (d)--(f) AF states.
    The constant $g$ is set to be $2 \ln \tilde{J}$ (FM) and $\ln \tilde{J}$ (AF). 
    }
    \label{fig3}
\end{figure}

The fastest scrambling in the lowest-order picture is
described by the contraction of the separation between $a$ and $b$ by half. 
As sketched in Fig.~\ref{fig2}(b), at even $r$, 
the lowest order is constructed by the hopping operators 
applying exclusively on each two-site pair of opposite spins.
This leads us to define the pseudospin Pauli operators $\hat{X}$ and $\hat{Z}$ 
in the basis of $|\Uparrow\rangle \equiv |\downarrow \uparrow \rangle$ 
and $|\Downarrow\rangle \equiv |\uparrow \downarrow \rangle$ 
for the reduced Hilbert space. 
We choose the AF state to evaluate $J_\mathrm{eff}^\mathrm{AF}$, 
but all configurations  filled up with staggered spin pairs 
provide the equivalent results.

At the lowest order, $J_\mathrm{eff}^\mathrm{AF}$ in
the XXZ chain is exactly reproduced by the Ising chain 
of a half-length $l = r/2$,
\begin{equation} \label{eq:Heff}
    \hat{H}_\mathrm{Ising} = 
    -\frac{J_z}{2} \sum_{k=0}^l \hat{Z}_k \hat{Z}_{k+1} 
    + \sum_{k=0}^{l+1} \Delta_k \hat{Z}_k 
    + J \sum_{k=1}^l \hat{X}_k\,, 
\end{equation}
where $\Delta_0 = h_a$, $\Delta_{l+1} = -h_b$, and
$\Delta_k = h_{a+2k} - h_{a + 2k -1}$ for $k = 1,\ldots,l$. 
The perturbation part is $J\sum_i \hat{X}_i$. The FM state
corresponds to the AF state of the XXZ chain, and
$\hat{X}_{0,l+1}$ replaces $\hat{\sigma}^x_{a,b}$ 
for the OTOC operators.
While we cannot find an analytic formula of $J_\mathrm{eff}^\mathrm{AF}$,
the half-length of the chain significantly reduces 
the cost for the numerical perturbation calculation \cite{sm}. 
Since nonzero $J_z$ is essential in both of 
$J_\mathrm{eff}^\mathrm{FM}$ and $J_\mathrm{eff}^\mathrm{AF}$,
hereafter we express the quantities in a dimensionless form as
$\tilde{t} \equiv J_z t$, $\tilde{h} \equiv h/J_z$, 
$\tilde{J} \equiv J/J_z$, and $\tilde{J}_\mathrm{eff} \equiv J_\mathrm{eff}/J_z$.

Figure~\ref{fig3} presents the numerical calculations
based on Eqs. \eqref{eq:Jeff_FM} and \eqref{eq:Heff},  
which verifies the logarithmic propagation of
the fronts of the slowest and fastest scrambling
but also examines the decay length scale of the effective interaction.
First, the disorder-averaged OTOC plotted as a function of 
$r^{-1}\ln \tilde{t}$ exhibits an increase that gets sharper 
as $r$ increases, assuring the strictly logarithmic slopes of 
the light cone. The shift $g \equiv q\ln\tilde{J}$ comes from 
$\tilde{J}_\mathrm{eff} \propto \tilde{J}^{qr}$ where $q=2(1)$ 
is for the FM(AF) state.
Second, from the ansatz of $\tilde{J}_\mathrm{eff} \sim \exp(-r/\xi)$,
we extract the inverse decay length as 
$\xi^{-1} = r^{-1}\ln|\tilde{J}_\mathrm{eff}|$. 
The distribution of $\xi^{-1}$ is increasingly peaked as $r$ increases,
indicating a well-defined $\langle \xi^{-1} \rangle_\mathrm{av}$. 
The skewed shape that we observe here at the particular states
is different from the log-normal shape previously 
reported at infinite temperature \cite{Varma2019}.
Third, we find that $\langle \xi^{-1} \rangle_\mathrm{av}$ 
follows the characteristic behavior with varying parameters as
\begin{equation} \label{eq:decay_length}
    \left\langle \xi^{-1} \right\rangle_\mathrm{av}  = 
    - \frac{\langle \ln|\tilde{J}_\mathrm{eff}| \rangle_\mathrm{av}}{r}
    \sim 
    \begin{cases}
        \ln(\tilde{h}/\tilde{J})^2 & \text{for FM}, \\
        \ln (\tilde{h}^\kappa/\tilde{J}) & \text{for AF}.
    \end{cases}
\end{equation}
One can directly extract the behavior for the FM state from 
Eq.~\eqref{eq:Jeff_FM} giving
$\tilde{J}_\mathrm{eff}^\mathrm{FM} \sim (\tilde{J}/2\tilde{h})^{2r}$
after rewriting it in the dimensionless form. For the AF state,
we determine the exponent $\kappa \approx 1.55$ numerically.

\begin{figure*}
    \centering
    \includegraphics[width = 0.96\textwidth]{fig4.pdf}
    \caption{Logarithmic light cones in the 2D strongly disordered 
    XXZ model. The scrambling time $t^*$ is computed 
    at $J/J_z = 0.001$ and $h/J_z = 20$ for the (a) FM and (b) AF states, 
    corresponding to the slowest and fastest scrambling, respectively.
    In the $(L+1) \times L$ lattices,
    the $\hat{\sigma}_x$ operators of OTOC are located 
    at the diagonal corners with separation $r = 2 (L - 1)$.
    The schematic diagram in the insets shows an example of 
    a path contributing to the lowest-order perturbation calculation.
    }
    \label{fig4}
\end{figure*}

Our lowest-order formulations developed above in 1D can be readily
extended to 2D by considering the multiple paths 
of the same Manhattan distance between the two sites $\mathbf{a}$ and $\mathbf{b}$
contributing to the non-vanishing lowest-order terms.
Below we describe the calculations of $J_\mathrm{eff}$ 
at the FM and AF states in $L_x \times L_y$ lattices 
with the two operators being located at  
the opposite corners as sketched in Fig.~\ref{fig4}. 
We remove boundary artifacts by adding the FM or AF
environments to the system.

For the FM state, the lowest order is determined as
$2r(\mathbf{a},\mathbf{b}) = 2(L_x + L_y - 3)$, twice of the number 
of sites along the shortest paths between $\mathbf{a}$ and $\mathbf{b}$.
The 2D variant of Eq.~\eqref{eq:Jeff_FM}
is written in a dimensionless form as 
\begin{equation} \label{eq:Jeff_2DFM}
    \tilde{J}_\mathrm{eff}^\mathrm{FM} = 
    2 \left(\frac{\tilde{J}}{2\tilde{h}}\right)^{2r}
    \sideset{}{^\prime}{\sum}_{\mathclap{(\mathbf{x}_1\to\mathbf{x}_2)}}
    \tilde{F}_{\mathbf{x}_1}^2
    \frac{\tilde{A}_{\mathbf{x}_1} + \tilde{B}_{\mathbf{x}_2}}{\tilde{A}_{\mathbf{x}_1} + \tilde{B}_{\mathbf{x}_2} - \tilde{h}^{-1}}
    \tilde{G}_{\mathbf{x}_2}^2, 
\end{equation}
where 
$\tilde{A}_\mathbf{x} = \tilde{h}_\mathbf{a} - \tilde{h}_\mathbf{x}$, and
$\tilde{B}_\mathbf{x} = \tilde{h}_\mathbf{b} - \tilde{h}_\mathbf{x}$.
The primed sum runs over directed links $(\mathbf{x}_1 \to \mathbf{x}_2)$ 
on all the shortest paths from $\mathbf{a}$ to $\mathbf{b}$.
The factors $\tilde{F}$ and $\tilde{G}$ are defined as
\begin{equation*}
    \tilde{F}_\mathbf{x} = \sum_{w(\mathbf{a},\mathbf{x})}
    \sideset{}{^\mathbf{a}}{\prod}_{\mathbf{y} \in w} \tilde{A}_\mathbf{y}^{-1}, 
    \quad
    \tilde{G}_\mathbf{x} = \sum_{w(\mathbf{b},\mathbf{x})} 
    \sideset{}{^\mathbf{b}}\prod_{\mathbf{y} \in w} \tilde{B}_\mathbf{y}^{-1},
\end{equation*}
where the sum runs over every shortest path $w(\mathbf{x}_0, \mathbf{x})$ 
connecting $\mathbf{x}_0$ and $\mathbf{x}$, and  
$\prod^{\mathbf{a}(\mathbf{b})}$ excludes $\mathbf{a}$($\mathbf{b}$) 
in the product over every site $\mathbf{y}$ along the path $w$. The squared
factors include the excitation moving forward 
and backward along different paths unlike in 1D.

For the AF state, we consider $(L+1) \times L$ lattices,
where $l \equiv L-1$ pairs of the up-and-down spins exist along any shortest path 
between $\mathbf{a}$ and $\mathbf{b}$, giving the lowest order $r = 2l$.
Unlike the FM case, the lowest-order contributions can be separated
into each path because a string of the hopping 
operators for paired spin flips must stay on the same path. 
For a path
$w \equiv (\mathbf{a}, \mathbf{x}_1, \mathbf{x}_2, \ldots, \mathbf{x}_{2l}, \mathbf{b})$,
the contribution is then given by 
the Ising chain with path-dependent parameters,
which can be expressed as
$\hat{H}_\mathrm{Ising}[\boldsymbol{\Delta}(w)]$
with $\Delta_0 = h_\mathbf{a}$, $\Delta_{l+1} = - h_\mathbf{b}$, and
$\Delta_k = h_{\mathbf{x}_{2k}} - h_{\mathbf{x}_{2k-1}} + 2J$,
where $2J$ is from the AF surroundings. 
Summing over all paths, we write $\tilde{J}_\mathrm{eff}^\mathrm{AF}$ as
\begin{equation} \label{eq:Jeff_2DAF}
    \tilde{J}_\mathrm{eff}^\mathrm{AF} (\mathbf{a}, \mathbf{b}) = \sum_w 
    \tilde{J}_\mathrm{eff}^\mathrm{AF}\bm{[}\hat{H}_\mathrm{Ising}[\boldsymbol{\Delta}(w)]\bm{]},
\end{equation}
which involves an exponentially growing number of terms as $L$
increases but allows us to go well beyond the system-size limit of
the exact diagonalization and the numerical perturbation
calculations for arbitrary orders.

Figure~\ref{fig4} shows the LLCs from the scrambling time and 
the disorder-averaged OTOC measured at the FM and AF states
in the 2D XXZ model in the strong disorder and weak hopping limit. 
Since the number of the shortest paths scales as $4^{l}$,
a rough estimate ignoring disorder correlations between the paths
suggests $\tilde{J}_\mathrm{eff}^\mathrm{AF} \sim 4^l e^{-2l/\xi}$
from Eq.~\eqref{eq:Jeff_2DAF}, implying LLC for 
$\langle \xi^{-1} \rangle_\mathrm{av} \gg \ln 2$.
While our calculations are based on the lowest-order perturbation theory, 
the numerical tests show excellent agreement with
the exact diagonalization at small $L$'s for
the FM state and with the full perturbation calculations
up to the fourth lowest order for the AF state.
Our observation of LLC in the strongly disordered XXZ model 
is also consistent with the previous evidence
of LLC reported in the 2D bosonic system with 
the l-bit construction at the strong disorder 
and weak interaction limit \cite{Kim2021}. 

In conclusion, our perturbation formulation unveils 
the peculiar structure of slow information propagation 
in the paradigmatic XXZ model in the deep MBL regime. 
The slowest and fastest scrambling identified in 
the discrete structure of OTOC characterizes 
the drastic difference between the spin-polarized and 
the N\'eel states of the intervening spins prepared 
for the OTOC or DEER measurements. 
We have derived the closed-form expression of the effective interaction 
for the slowest scrambling and found the effective Ising chain 
of a half-length describing the fastest scrambling, 
presenting the sharp logarithmic upper and lower bounds 
of the light cone.

Our observation of LLCs extends the variety of the practical 
MBL signatures previously reported in finite 2D systems, 
although the instability of 2D MBL in the asymptotic limit 
goes beyond our method. A challenging direction
for future study may include the clarification of the finite-size
effects on the 2D scrambling away from the weak hopping limit.  
On the other hand, our findings on the distance effectively reduced 
by half at the fastest scrambling imply an interesting question 
on its l-bit representation.
In contrast to the slowest one, the fastest scrambling involves
only the half-number of the pseudospins, proposing to further explore 
how the mapping to the l-bit Hamiltonian encodes
these system-specific scrambling structures of the XXZ model.

\begin{acknowledgments}
This work was supported from the Basic Science Research Program 
through the National Research Foundation of Korea (NRF-2019R1F1A1063211) 
and also from a GIST Research Institute (GRI) grant funded by the GIST.
We appreciate APCTP for its hospitality during the completion of this work.
\end{acknowledgments}

\bibliography{paper}

\end{document}
