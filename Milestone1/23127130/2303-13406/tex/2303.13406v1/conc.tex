\section{Conclusions}

% \singlespacing
% \onehalfspacing
% \doublespacing


We introduce a simple procedure to control for false discoveries and identify individual signals when there are many tests, the test statistics are correlated, and the signals are potentially sparse. The new procedure is a sequential version of the global Cauchy combination test of \cite{liu2020cauchy}. The Cauchy combination test \citep{liu2020cauchy} is a global test grounded on a theoretical property of standard Cauchy variates: linear combinations of standard Cauchy variates behave similarly to a standard Cauchy at extreme tails, regardless of the dependence structure of the Cauchy variates.  The sequential Cauchy combination test proposed in this paper 
applies the test recursively on a sequence of expanding subsets of ordered $p$-values to 
find out \textit{which} of the $p$-values trigger the 
global test to, for example in the context of time series data, timestamp rejections.
% The sequential test proposed in this paper applies the GCC test recursively on a sequence of expanding subsets of ordered $p$-values, starting from the largest $p$-value to identify \textit{which} of the $p$-values trigger the global test. 
We show that the sequential Cauchy combination test achieves strong familywise error rate control and is less conservative than the popular statistical inequality based methods (such as the Bonferroni correction and the subsequent improvements of \citeauthor{holm1979simple}, \citeyear{holm1979simple};  \citeauthor{hommel1988stagewise}, 
\citeyear{hommel1988stagewise} and   \citeauthor{hochberg1988sharper}, 
\citeyear{hochberg1988sharper}) and the Gumbel method.

As illustrations, we revisit two important 
needle-in-a-haystack problems 
in financial econometrics for which the test statistics have either serial or
cross-sectional dependence:  monitoring drift bursts and searching for nonzero alpha assets. 
The drift burst test of \citet{christensen2018drift} detects the presence of explosive trends in asset prices using intraday high frequency data. The drift burst test statistics are computed from overlapping windows and are therefore highly autocorrelated. 
We  also revisit the 
\citet{fama2015five}
multi-factor model to identify nonzero alpha financial assets.
Identifying the rare nonzero alphas from a large group of financial assets is challenging, especially since the test statistics are likely to be cross-sectionally correlated.
In both cases, one might flag false discoveries or miss signals without a proper controlling procedure. We conclude that the sequential Cauchy combination test is a preferable alternative for both applications.


We emphasize that our sequential Cauchy combination test is by no means limited to applications in financial econometrics. We expect the sequential Cauchy combination test to handle all kinds of dependence and can be applied to many other hypothesis tests in economics, finance, medicine, marketing and climate studies. 
% The most common types of dependence in economics and finance are serial and cross-sectional dependence, but we could also imagine a case with spatial dependence. 

