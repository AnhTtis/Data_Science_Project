\documentclass[a4paper,12pt]{article}
%%%%%%%%%%%%%%%%%%%%%%%%%%%%%%%%%%%%%%%%%%%%%%%%%%%%%%%%%%%%%%
\usepackage{amsmath}

\usepackage{amsfonts}
% \usepackage{graphicx,subfig}
 \usepackage{graphicx} % subcaption <> subfig
\usepackage[authoryear]{natbib}
\usepackage{xcolor}
\usepackage{float}
\usepackage{rotating} % Rotating table
% \usepackage[doublespacing]{setspace}
\usepackage[singlespacing]{setspace}
\usepackage[title]{appendix}
% \usepackage{hyperref}
\usepackage[breaklinks=true]{hyperref} % arxiv
\usepackage{breakcites} % arxiv
\usepackage{url}
\usepackage{xr}
\usepackage{bm}
\usepackage{cleveref}
%\usepackage{showkeys}
\usepackage{changes}
\RequirePackage{amsmath,amssymb}
\hypersetup{colorlinks=true,citecolor=blue, linkcolor=blue, urlcolor=blue}
\newtheorem{assumption}{Assumption}[section]
\newtheorem{theorem}{Theorem}[section]
\newtheorem{prop}{Proposition}
\newtheorem{remark}{Remark}
\newtheorem{algorithm}[theorem]{Algorithm}
\newtheorem{case}[theorem]{Case}
\newtheorem{Corollary}[theorem]{Corollary}
\newtheorem{definition}[theorem]{Definition}
\newtheorem{lemma}[theorem]{Lemma}
\newtheorem{condition}{Condition}[section]
\numberwithin{equation}{section}

% extra packages
\usepackage{threeparttable} % tablenotes cmd
\usepackage{physics} % \abs cmd
\usepackage{adjustbox} % adjustbox in tables
\hypersetup{colorlinks=true,citecolor=blue, linkcolor=blue, urlcolor=blue}
\usepackage{subcaption} % subfig <> subcaption

\RequirePackage[left=2cm,
right=2cm,
top=1in,
bottom=1in,headheight=12pt]{geometry}

\renewcommand{\thetheorem}{\arabic{theorem}}
\newcommand{\red}{\textcolor{red}}
\newcommand{\blue}{\textcolor{blue}}
\newcommand{\green}{\textcolor{green}}
\newcommand{\Prob}{\ensuremath{\mathbb{P}}} % Drift-burst section

\allowdisplaybreaks
\graphicspath{{Figures/}}

\renewcommand{\baselinestretch}{1.5}

\begin{document}

\title{Sequential Cauchy Combination Test for Multiple Testing Problems with Financial Applications\thanks{%
 		We have received helpful comments and suggestions from Kris Boudt, Geert Dhaene, Frank Kleibergen, Nathan Lassance, Roberto Ren\`o, Olivier Scaillet, Rosnel Sessinou, Kristien Smedts, Steven Vanduffel, and the seminar and conference participants at 
 		KU Leuven, Université catholique de Louvain, 
 		the Netherlands Econometric Study Group (2022), the Quantitative Finance and Financial Econometrics Conference (2022), the Macquarie Financial Econometrics Workshop (2022), and the Computational and Financial Econometrics Conference 		(2022). Bouamara acknowledges support from the Flemish Research Foundation (FWO fellowship \#11F8419N) and the S\&B fund (Gustave Bo\"el -- Sofina fellowship).
 		Laurent acknowledges support from the French National Research Agency (reference: ANR-17-EURE-0020 and ANR-21-CE26-0007-01) and the excellence initiative of Aix-Marseille University - A*MIDEX.	Shi acknowledges research support from the Australian Research Council (project No. DE190100840).
 			\vspace{0.2cm}		\\ 
 		$^{\dag}$ Nabil Bouamara, Louvain Institute of Data Analysis and Modeling in economics and statistics, Universit\'{e} catholique de Louvain;  	Email: nabil.bouamara@uclouvain.be.\\
 		$^{\dag\dag}$ S\'{e}bastien Laurent, Aix-Marseille University (Aix-Marseille School of Economics), CNRS \& EHESS, Aix-Marseille Graduate School of Management -- IAE; Email: sebastien.laurent@univ-amu.fr.\\ 
 		$^{\dag\dag\dag}$ Shuping Shi, Department of Economics, Macquarie University; Email: shuping.shi@mq.edu.au. 
}
\vspace{-4mm}
}

\author{Nabil Bouamara$^{\dag}$, S\'{e}bastien Laurent$^{\dag\dag}$, Shuping Shi$%
^{\dag\dag\dag}$ \\
}
 
\date{March 23, 2023} % arxiv recompiles tex files 
% remove to update file. 

\maketitle


\vspace{-8mm}
\begin{spacing}{1.00} 
		\begin{abstract}
			We introduce a simple tool to control for false discoveries and identify individual signals when there are many tests, the test statistics are correlated, and the signals are potentially sparse. 
			The	tool applies the Cauchy combination test recursively on a sequence of expanding subsets of $p$-values and is referred to as the sequential Cauchy combination test.  
			While the original Cauchy combination test aims for a global statement over a set of null hypotheses by summing transformed $p$-values, the	sequential version 	determines which
			$p$-values trigger the rejection of the global null. The test achieves strong familywise error rate control and is less conservative than existing controlling procedures when the test statistics are dependent, leading to higher global powers and successful detection rates. 
			As illustrations, we consider two popular financial econometric 
			applications for which the test statistics have either serial dependence or cross-sectional dependence: monitoring drift bursts in asset prices and searching for assets with a nonzero alpha. 
			The sequential Cauchy combination test is a preferable alternative in both cases in simulation settings and leads to higher detection rates than benchmark procedures in empirics.
			
		  \vspace{0.1in}
			
			\noindent\textit{Keywords:} Multiple
			hypothesis testing; Cauchy combination;  High-dimensional;  Sequential rejection; Sparse alternatives; Dependence \\
			
			 \vspace{-2mm}
			
			\noindent\textit{JEL classification: } C12, C13, C58
		\end{abstract}
	
	\vspace{10mm}
	\end{spacing}

% \onehalfspacing
\singlespacing

\section{Introduction}

The increasing complexity of source code poses a key challenge to the reliability of large-scale software systems. Software bugs in these systems can lead to safety issues~\cite{bug_safety} for users around the world as well as cause non-negligible financial losses~\cite{bug_loss}. As such, developers have to spend a large amount of time and effort on bug fixing. Consequently, \aprfull (\apr), designed to automatically generate patches to fix software bugs, has attracted wide attention from both academia and industry~\cite{long2016prophet, legoues2012genprog, long2015spr, lou2020can, tufano2018empstudy}. 


To achieve \apr, one popular approach is known as Generate-and-Validate (G\&V)~\cite{qi2015gv, ghanbari2019prapr, lou2020can, le2016hdrepair, legoues2012genprog, wen2018capgen, hua2018sketchfix, martinez2016astor, koyuncu2020fixminder, liu2019tbar, liu2019avatar}, which is typically based on the following pipeline: First, fault localization techniques~\cite{wong2016fl, abreu2007ochiai, zhang2013injecting, papadakis2015metallaxis, li2019deepfl, li2017transforming} are applied to determine the suspicious locations in programs where bugs are likely to exist. Then, the buggy locations are used by the \apr tools to generate a list of patches that replace buggy lines with correct lines. Afterward, each patch is validated against the original test suite to identify any \emph{plausible patches} (i.e., passing all tests in the test suite). Finally, to determine the \emph{correct patches}, developers examine the list of plausible patches to see if any of them can correctly fix the bug. 

Traditional \apr tools can mainly be categorized into heuristic-based~\cite{legoues2012genprog, le2016hdrepair, wen2018capgen}, constraint-based~\cite{mechtaev2016angelix, le2017s3, demacro2014nopol, long2015spr} and \template~\cite{ghanbari2019prapr, hua2018sketchfix, martinez2016astor, liu2019tbar, liu2019avatar}. Among these traditional tools, \template \apr tools~\cite{ghanbari2019prapr, liu2019tbar, benton2020effectiveness} have been able to achieve state-of-the-art results. \Template \apr tools typically leverage pre-defined templates (e.g., adding a nullness check) for bug fixing. However, since these fix templates are typically handcrafted, the number and types of bugs they are able to fix can be limited. 



To address the limitations of traditional \apr, researchers have proposed various \learning \apr tools~\cite{li2020dlfix, chen2018sequencer, jiang2021cure, lutellier2020coconut, zhu2021recoder, ye2022rewardrepair} based on the \nmtfull (\nmt) architecture~\cite{sutskever2014mt} where the input is the buggy code snippets and the goal is to translate the buggy code snippets into a fixed version. To accomplish this, \learning \apr tools require supervised training datasets with pairs of both buggy and fixed code snippets in order to learn how to perform this translation step. These training data are usually obtained by mining historical bug fixes using heuristics/keywords~\cite{dallmeier2007benchmark}, which can be imprecise for identifying bug-fixing commits; even the actual bug-fixing commits can include irrelevant code changes, leading to further pollution in the dataset~\cite{xia2022alpharepair}.
% 
Moreover, it can be hard for such \apr tools to generalize and fix bug types unseen during training. 



To better leverage recent advances in \plmfull{s} (\plm{s}), researchers~\cite{xia2022alpharepair, xia2023repairstudy, kolak2022patch, prenner2021codexws} have directly applied \plm{s} to generate patches without bug-fixing datasets. These \llm-based \apr tools work by either directly generating a complete code function~\cite{prenner2021codexws, xia2023repairstudy} or predict/infill the correct code snippet given its surrounding context~\cite{xia2022alpharepair, xia2023repairstudy}. By directly using \llm{s} that are pre-trained on billions of open-source code snippets, \llm-based \apr tools can achieve state-of-the-art performance on many repair datasets~\cite{xia2022alpharepair}. 


% 
%
%

Traditional \apr tools have long used the insight of the \emph{plastic surgery hypothesis}~\cite{barr2014plastic} where it states that the code ingredients to fix a bug already exist within the same project. Traditional \apr tools have manually designed pattern-~\cite{ghanbari2019prapr, saha2017elixir} or heuristic-based~\cite{jiang2018simfix, legoues2012genprog} approaches to finding and using such relevant code ingredients to generate fixes for bugs. However, the plastic surgery hypothesis has been largely ignored in \llm-based \apr. In fact, \llm provides a unique opportunity to fully automate the plastic surgery hypothesis idea via fine-tuning (learning project-specific information via model updates from the buggy project) and prompting (directly providing relevant code ingredients to the model), and make it directly applicable to different languages (since the \llm{s} are typically multi-lingual).%
Moreover, despite the intensive manual efforts involved, traditional \apr tools still cannot fully leverage project-specific information due to large search space for leveraging/composing existing code ingredients. In contrast, the project-specific information can effectively leveraged by \llm{s} due to their power in code understanding/vectorization, e.g., even partial/imprecise information may still guide \llm{s} in correct patch generation!
 To this end, we ask the question: \emph{How useful is the plastic surgery hypothesis in the era of \plm{s}}?








\mypara{Our Work.} To answer the question, we present \ourtech{\xspace} -- a \llm-based approach that automatically utilizes the plastic surgery hypothesis by systematically combining multiple fine-tuning and prompting strategies for \apr. \ourtech fine-tunes \plm{s} using two novel domain-specific training strategies: \textbf{\epfinetune} -- we fine-tune using the original buggy project by aggressively masking out a high percentage of tokens, which allows \plm to learn project-specific code tokens and programming styles; and \textbf{\rofinetune} -- which only masks out a single continuous code sequence per training sample, allowing the model to get used to the final \csapr task of predicting a single continuous code sequence. Furthermore, we directly leverage the ability for \plm{s} to understand natural language instructions and introduce a novel prompting strategy, \textbf{\idprompting}, which uses information retrieval and static analysis to obtain a list of relevant identifiers for the buggy lines. While such relevant identifiers are critical for fixing some difficult bugs, they may not be seen by the \llm during inference due to limited context window size. Through the use of prompting, we directly tell the model to use these extracted identifiers (relevant code ingredients) to generate the correct code. Finally, to perform repair, we combine all four model variants (including the base model, both fine-tuned models and the base model with prompting) for the final repair.





While our insight of leveraging the plastic surgery hypothesis for \llm-based \apr is generalizable across different types of \plm{s}, to implement \ourtech, we choose a recent \plm{\xspace}, \ctfive~\cite{wang2021codet5}, which is pre-trained on millions of open-source code snippets. \ctfive is an encoder-decoder model trained using \mspfull (\msp) objective where a percentage of tokens are masked out and each continuous masked token sequence is referred to as a masked span. Also, although we only extract relevant identifiers from the current buggy project (since this paper focuses on the plastic surgery hypothesis), our work can be easily extended to obtain other code information (such as relevant statements or functions) from other sources, such as  the massive pre-training corpora~\cite{husain2020codesearchnet} or historical bug-fixing datasets~\cite{jiang2019infer}, which can provide more coding knowledge for \llm{s}. Besides, although we mainly focus on using traditional string comparison algorithms for information retrieval in this paper, these techniques can be easily replaced by other frequency-based retrieval~\cite{robertson2009probabilistic} and neural search (or embedding-based search)~\cite{reimers2019sentence}.
  In summary, this paper makes the following contributions:


%


\begin{itemize}[noitemsep, leftmargin=*, topsep=0pt]
    \item \textbf{Dimension.} This paper is the first to revisit the important plastic surgery hypothesis in the era of \llm{s}. It opens up a new dimension for \llm-based \apr to incorporate previously neglected information from the buggy project itself to boost \apr performance. Furthermore, it demonstrates the promising future of retrieval-based prompting for modern \llm-based \apr.
    \item \textbf{Implementation.} We implement \ourtech based on the recent \ctfive model. We augment the model using two novel fine-tuning strategies: \epfinetune and \rofinetune, along with a novel prompting strategy based on information retrieval and static analysis: \idprompting. We combine the patches generated by all four models together and perform patch ranking to speed up \apr.% 
    \item \textbf{Evaluation Study.} We conduct an extensive evaluation against state-of-the-art \apr tools. On the widely studied \dfj 1.2 and 2.0 datasets~\cite{just2014dfj}, \ourtech is able to achieve the new state-of-the-art results of 89 and 44 correct bug fixes (15 and 8 more than best baseline) respectively.  Furthermore, we perform a broad ablation study to justify our design. \ourtech demonstrates for the first time that the plastic surgery hypothesis can substantially boost \llm-based \apr and advance state-of-the-art \apr, while being fully automated and general. Moreover, even partial/imprecise code ingredients may still effectively guide \llm{s} for \apr!
\end{itemize}


% !TEX root = ./CauchyCombination.tex
\section{Multiple Hypothesis Testing} \label{secPrelims}

This section introduces the notation on multiple hypothesis testing and the benchmark procedures for addressing the multiple testing problem. 

\subsection{Setting}
Let $H_{i}$ denote the $i^{\text{th}}$ null hypothesis of interest, with $i=1,...,d$,  
and $d$ being the total number of individual hypotheses. To test the $d$ hypotheses, we can use the associated vector of test statistics $\bm{X}=(X_{1},X_{2},\ldots,X_{d})^{^{\prime }}$, one for each hypothesis being tested, or the corresponding raw $p$-values $p_{1},\ldots ,p_{d}$. The test statistics can be independent or % corrected.
correlated. 

%In some cases, like in Section \ref{secApplDriftBurst}, the test statistics are constructed from rolling windows and are extremely serially correlated. 

The first task is to test the global null hypothesis. Let $\mathcal{H}_{0}$ be the collection of null hypotheses of interest. 
The strategy of a classical global test is to abandon the multiplicity issue altogether and replace multiple tests with the global null hypothesis that all elementary hypotheses are true.  The alternative is that at least one elementary hypothesis is false. For example, in high-frequency financial econometrics,  we often need to monitor the presence of certain events (e.g., jumps or drift bursts) within a fixed time period (e.g., within a day). The global null is that there is no occurrence of such an event at all (e.g., 
% in Example 2, none of the stocks has a significant alpha
in Example 1, there is no drift burst within the day or in Example 2, none of the stocks has a significant alpha).
The goal is to get $\alpha$-level control under this global null, i.e., $P_{\mathcal{H}_0} [\text{reject}\, \mathcal{H}_0] \leq \alpha$. The test is conservative when $P_{\mathcal{H}_0} [\text{reject}\, \mathcal{H}_0]$ is strictly less than the theoretical upper bound $\alpha$ and ideal when it is equal to  $\alpha$. 

When, by any  test, the global null $\mathcal{H}_0$ is rejected, the second task is to identify which of the elementary hypotheses $H_{i}$ should be rejected. The set of true hypotheses $\mathcal{T}$, the set of false hypotheses $\mathcal{F}$ and the set of rejected hypotheses $\mathcal{R}$ are defined as: 
\begin{align}
	\begin{split}  \label{eqLocalHypothesis}
		\mathcal{T} &= \{ H_{i}\in \mathcal{H}_0: H_{i} \, \text{is true}\}, \\
		\mathcal{F} &= \{ H_{i}\in \mathcal{H}_0: H_{i} \, \text{is false}\}, \text{and} \\
		\mathcal{R} &= \{H_{i}\in \mathcal{H}_0: H_{i} \, \text{is rejected}\}.
	\end{split}%
\end{align}
The set of true and false hypotheses are unknown. We choose a set of hypotheses to reject. 
on the basis of our data. 
The set of discoveries $\mathcal{R}$ 
should coincide with the set of false hypotheses $\mathcal{F}$ as much as possible.

The goal of various multiple testing corrections is to control the familywise error rate (FWER), defined as the probability of at least one false
rejection in the family, $P[\mathcal{T} \cap \mathcal{R} \neq \varnothing]$,
while retaining the reasonable power in detecting false hypotheses. We want procedures for which the FWER is less than or equal to the upper bound $\alpha$ and ideally as close as possible to the upper bound. We focus on strong control of the FWER, meaning that some of the hypotheses we are testing can be false ($\mathcal{F} \neq \varnothing$), as opposed to the weak FWER control where all hypotheses of interest are true, i.e., $\mathcal{H}_0=\mathcal{T}$.

The probability of falsely rejecting a single hypothesis that is true (i.e., false positive or Type I error) is usually controlled at a nominal $\alpha$-level. However, when the number of tested hypotheses is large, the problem of multiplicity arises: the probability of having at least one false positive conclusion rises well above $\alpha$ if the Type I error of each individual test is controlled at the $\alpha$-level. Numerous controlling procedures have been proposed to deal with this problem. 
We review two classes of controlling procedures: one based on statistical inequalities (Section \ref{ssecOrderdPvals}) and one based on the maximum of the test statistics (Section \ref{ssecMaxTest}).


\subsection{Procedures based on statistical inequalities}
\label{ssecOrderdPvals}

Let us denote by $0 < p_{(1)}\leq p_{(2)}\leq \ldots\leq p_{(d)} < 1$ the set of $d$  ordered (in ascending order) raw $p$-values and $H_{(1)},H_{(2)},\ldots,H_{(d)}$ their corresponding null hypotheses. A single-stage method uses the same rejection 
criterion for all individual hypotheses, like the conservative Bonferroni threshold, while a multi-stage method examines the ordered $p$-values sequentially and adjusts the rejection criterion for each of the individual tests  \citep[e.g.,][]{holm1979simple,hochberg1988sharper,hommel1988stagewise}. 

The Bonferroni method rejects the elementary null hypothesis $H_{(i)}$ if 
$p_{(i)}\leq\alpha/d$ 
for $i=1,\ldots,d$. \citet{holm1979simple} and \citet{hochberg1988sharper} use the same critical values $ \alpha / (d - i + 1)$ depending on the rank of the $p$-value, but reject differently depending on whether they ``step up" or ``step down". The terminologies (``step up" or ``step down") were originally formulated in terms of test statistics which can be confusing when discussing $p$-values.  \citet{holm1979simple} proposes a step-down method that ``steps up" from the smallest $p$-value to the largest one. It is a pessimistic approach: it scans forward and stops as soon as a $p$-value fails to clear its threshold. \citet{hochberg1988sharper} suggests a step-up method that ``steps down" from the largest $p$-value to the	smallest one. It is an optimistic approach: it scans backward and stops as soon as a $p$-value succeeds in clearing its threshold. By construction, Hochberg's procedure will reject as many hypotheses as Holm's procedure. 

\cite{hommel1988stagewise} proposes a more complicated procedure which applies  \citet{simes1986improved}' global test to the $p$-value subset $\left\{ p_{\left(k\right) }\right\} _{ k = i }^{d}$, instead of relying  only on $p_{\left(i\right)}$ to draw inference on $H_{(i)}$ and thus borrows power across hypotheses. 
Hommel's procedure is shown to have higher power than Hochberg's method \citep{hommel1989}. We refer to Appendix \ref{AppOrderedPVals} for more details on the practical implementation of these procedures.


Bonferroni and \citet{holm1979simple}'s method are based on the first-order Bonferroni inequality, which states that given any set of events, the probability of their union is smaller than or equal to the sum of their probabilities. 
Under the null hypothesis, 
the probability that there is at least one hypothesis $H_{(i)}$  for which its raw $p$-value $p_{(i)} \leq \alpha / d$ 
% is not greater than $\alpha$ 
is bounded by $\alpha$: 
\begin{align}  \label{eqIneqPval}
	\Pr\left( \min_{i} p_{(i)} \leq \frac{\alpha}{d} \right) 
	= \Pr\left(\bigcup_{i =
		1}^{d} \left\{ p_{(i)} \leq \frac{\alpha}{d} \right\} \right) 
	&\leq
	\sum^{d}_{i = 1} \Pr \left( p_{(i)} \leq \frac{\alpha}{d}\right) \leq d
	\frac{\alpha}{d} \leq \alpha.
\end{align}
The Bonferroni \eqref{eqIneqPval} inequality 
makes no specific assumption on the dependence between the $p$-values, but protects against the so-called ``worst-case", in which all events are independent and the rejection regions are disjoint (the right half of Equation \eqref{eqIneqPval}) . 
The inequality becomes an equality when all test statistics are independent, and a strict inequality when the hypotheses are dependent. 
In other words, the Bonferroni correction is  conservative when the $p$-values are correlated. 

The methods of \cite{hochberg1988sharper} and \cite{hommel1988stagewise} are based on  \cite{simes1986improved}'s inequality. If a set of hypotheses $H_{(1)}, ...,H_{(d)}$ are all true, the probability of the joint event is: 
\begin{equation}  \label{eqSimes}
	\Pr\left( p_{\left( i\right) }> \frac{i\alpha}{d}, \text{ for all } i=1,\ldots
	,d\right) \geq 1-\alpha.
\end{equation}
\citet{simes1986improved}' inequality was developed for independent uniform $p$-values, and it is applicable for a large family of multivariate distributions. The simulations of \citet{simes1986improved} do show, however, that the test is very conservative for highly correlated multivariate normal statistics, but less so than the classical Bonferroni correction. 



\subsection{Procedures based on the maximum of test statistics}
\label{ssecMaxTest}

Another class of controlling procedures uses the maximum in a group of test statistics: $X_m = \max_{i} \abs{X_{i}}$, with $i = 1, \ldots, d$, to set a stringent critical value. The same critical value can be used for each elementary hypothesis and will control the familywise error rate. In particular, when the individual test statistics are independent and follow the standard normal distribution under the null, the maximum of the test statistics follows a Gumbel distribution when $d$ is large. Quantiles of the Gumbel distribution were used as critical values of the individual tests as a multiple testing correction when, for example, conducting jump tests in high-frequency asset returns \citep[][]{lee2007jumps}. Unfortunately, if the sequence of test statistics exhibits strong correlation, the number of tests severely overstates the effective number of independent copies in a given sample, which makes the Gumbel critical values too conservative \citep[see e.g.,][]{christensen2018drift}. We refer to Appendix \ref{AppOrderedPVals} for more details on the Gumbel distribution. 

Resampling-based methods account for the dependence structure that is specific to the considered dataset, leading to less conservative testing outcomes than the Gumbel-based methods and the inequality-based procedures. Depending on the empirical problem of interest, the resampling can be carried out by bootstrap, permutation, simulation, or randomization \citep[see e.g.,][for detailed discussions on resampling methods and testing procedures]{white2000,romano2005exact,romano2005stepwise,lehmann2005testing}. We refer to Section \ref{secApplDriftBurst} for an example of the resampling-based approach for the drift burst test. 


\section{Cauchy Combination Tests}
\label{secSeqCauchy}

In this section, we first review the global Cauchy combination (GCC) test of \citet{liu2020cauchy} and present our sequential Cauchy combination (SCC) test. While the global test tests the global null hypothesis $\mathcal{H}_0 = \bigcap_{i=1}^{d} H_{i}$ against the alternative hypothesis that at least one of the elementary null hypotheses is false, the sequential test aims at identifying the violations of the elementary null hypotheses while controlling the global error rate. 

\subsection{Global Cauchy combination test}
\label{sec:CC}

The GCC test statistic is constructed from the raw $p$-values of the test statistics $X_i$, which follow a uniform distribution between $0$ and $1$ under the  null hypothesis. The idea of this test is first to transform the uniformly distributed $p$-values into standard Cauchy variates using the formula $\tan \{(0.5-p_{i})\pi \}$ and then construct a new test statistic as the weighted sum of these transformed $p$-values. The new test statistic is denoted by $\tilde{T}$ and defined as: 
\begin{equation}
	\label{eqCauchyStatistic}
	{\normalsize \tilde{T}=\sum_{i=1}^{d}w_{i}\tan \{(0.5-p_{i})\pi \},} 
\end{equation}
in which the $w_{i}$'s are non-negative weights summing to 1. Throughout the paper, the weights $w_{i}$ are set to $1/d$ for $i=1,\ldots,d$ as in \citet{liu2020cauchy}. 

When the raw and hence the transformed $p$-values are independent (resp. perfectly correlated), % under the null hypothesis, 
the new test statistic $\tilde{T}$ is a linear combination of independent (resp. perfectly correlated) Cauchy variates and therefore follows a standard Cauchy distribution because the family of Cauchy densities is closed under convolutions. Although the correlation structure can affect the null distribution of $\tilde{T}$ in the case of general dependence, \citet{liu2020cauchy} show that the impact on the tail is very limited because of the heaviness of the Cauchy tail. Specifically, they prove that: 
\begin{equation}
	\lim_{h\rightarrow \infty }\frac{\Pr\left( \tilde{T}>h\right) }{\Pr\left(
		C>h\right) }=1,  \label{eq:tail}
\end{equation}
in which $C$ is a standard Cauchy random variable, under the null hypothesis $\mathcal{H}_0$ and Assumption \ref{ass1} which requires the test statistics to follow a bivariate zero mean normal distribution.
\begin{assumption}
	\label{ass1} (1) The original test statistics $(X_{i},X_{j})$, for any $1\leq
	i<j\leq d$, follow a bivariate normal distribution; (2) $E\left( \bm{X}%
	\right) =0$, with $\bm{X}=(X_{1},X_{2},\ldots,X_{d})^{^{\prime }}$. 
\end{assumption}
The bivariate normal requirement of Assumption \ref{ass1} is a condition weaker than joint normality, making the procedure applicable for high-dimensional settings. When the dimension $d$ increases at a certain rate with the sample size, the test statistics $\bm{X}$ may not jointly converge to a multivariate normal distribution due to its slower rate of convergence \citep[see][and references therein]{liu2020cauchy} and thus a joint normality assumption is not realistic for those settings. In contrast, the bivariate normality assumption is much weaker and more realistic. There are, of course, applications for which the test statistics are not normally distributed. Through simulations, \citet[][]{liu2020cauchy} show the Cauchy approximation is still accurate when the normality assumption is violated and follows a multivariate Student-$t$ distribution (with 4 degrees of freedom) instead. 
We refer to Section \ref{secApplFan} for a showcase example in finance with test statistics being Student-$t$ distributed. 

The result in  \eqref{eq:tail} suggests that, under the null hypothesis $\mathcal{H}_0$, the tail of the Cauchy combination test statistic is approximately Cauchy under arbitrary dependence structures, so that a $p$-value of the Cauchy combination test, denoted 
$\widetilde{p}$, can  be calculated from the standard Cauchy distribution. Suppose that we observe $\tilde{T}=t_{0}$, then: 
\begin{equation}
	\label{eqCauchyPval}
	\widetilde{p}=\frac{1}{2}-\frac{\arctan t_{0}}{\pi }. 
\end{equation}

Using the GCC $p$-values, the tail result in \eqref{eq:tail} can be equivalently stated as the actual size converging to the nominal size $\alpha$ as the significance level tends to zero:  % \textit{i.e.}, 
\begin{equation}
	\lim_{\alpha\rightarrow 0 }\frac{\Pr\left( \widetilde{p} \leq \alpha\right) }{\alpha}=1, \label{eq:wFWER}
\end{equation} 
The approximation should be particularly accurate for small $\alpha$'s, which are of particular interest in large-scale problems as in Examples 1 and 2. The simulations in \citeauthor{liu2020cauchy} show that when the significance level is moderately small ($\alpha = 10^{-1}, 10^{-2},10^{-3},10^{-4},10^{-5}$), the $p$-value calculation is accurate:  the ratio of the empirical size to the significance level is close to 1 for different types of correlations. 
Put differently, the GCC test achieves the weak familywise error rate control as the empirical size is very close to the nominal size $\alpha$ regardless of the correlation structure. 


Figure \ref{figCauchyPvalsAR} illustrates the fact that while the dependence between the individual test statistics $X_i$ can affect the null distribution of the GCC test statistic, the impact of the dependence is marginal on the tail. We simulate a vector of $d$ test statistics  $\bm{X}$ from a $d$-variate normal distribution with correlation matrix $\bm{\Sigma}$, \textit{i.e.}, $N_d(\bm{0}, \bm{\Sigma})$ with $\bm{\Sigma} = (\sigma_{ij})$ and $d = 300$. The diagonal elements $\sigma_{ii}=1$ for all $i=1,\ldots,d$ and the off-diagonal elements $\sigma_{ij} = \theta^{\abs{i-j}}$ for $i \neq j$, with $\theta = 0.2, 0.4, 0.6,$ $0.8, 0.90, 0.95$. The simulation is repeated $10^7$ times. For each draw, we calculate the GCC test statistic \eqref{eqCauchyStatistic} and its corresponding $p$-value \eqref{eqCauchyPval}. The histogram of the $10^7$ GCC $p$-values is displayed in Figure \ref{figCauchyPvalsAR}. For a low level of autocorrelation (i.e., $\theta=0.2$), the distribution of the $p$-values is close to a uniform distribution. When the level of autocorrelation is higher, there is a pothole in the middle and a bump at the end of the histogram, but whatever the strength of the autoregressive parameter, the percentage of the GCC $p$-values in the first bin is always around $5$\% as is ensured by the limit result in \eqref{eq:wFWER}. 


\begin{figure}[p]
	\caption{The impact of dependence on the tail of the GCC test statistic}
	\label{figCauchyPvalsAR}
	\centering
	
	\par
	
	\subfloat[${\theta} = 0.2$ ]{{\includegraphics[width=.40\textwidth,angle =
			-90,scale=0.70]{Sample_pval_dim_300_rho_0.2_alpha0.05.eps} }} 
	\subfloat[$\theta = 0.4$ ]{{\includegraphics[width=.40\textwidth,angle =
			-90,scale=0.70]{Sample_pval_dim_300_rho_0.4_alpha0.05.eps} }} 
	
	\vspace{0.4cm}
	
	\subfloat[$\theta = 0.6$ ]{{\includegraphics[width=.40\textwidth,angle =
			-90,scale=0.70]{Sample_pval_dim_300_rho_0.6_alpha0.05.eps} }} 
	\subfloat[$\theta = 0.8$ ]{{\includegraphics[width=.40\textwidth,angle =
			-90,scale=0.70]{Sample_pval_dim_300_rho_0.8_alpha0.05.eps} }} 
	
	\vspace{0.4cm}	
	
	\subfloat[$\theta = 0.90$]{{\includegraphics[width=.40\textwidth,angle =
			-90,scale=0.70]{Sample_pval_dim_300_rho_0.9_alpha0.05.eps} }} 
	% 
	\subfloat[$\theta = 0.95$]{{\includegraphics[width=.40\textwidth,angle =
			-90,scale=0.70]{Sample_pval_dim_300_rho_0.95_alpha0.05.eps} }} 
	
	\par
	\begin{minipage}{1.0\linewidth}
		\begin{tablenotes}
			\small
			\item {
				\medskip
				Note: We plot histograms of GCC $p$-values \eqref{eqCauchyPval} for various correlation strengths. The individual test statistics are drawn from a $d$-variate normal distribution $N_d(\bm{0}, \bm{\Sigma})$ with $\bm{\Sigma}= (\sigma_{ij})$ and $d=300$. The diagonal elements of the covariance matrix $\sigma_{ii}=1$ for all $i=1,\ldots,d$ and the off-diagonal elements  $\sigma_{ij} = \theta^{\vert i-j \vert}$ for $i\neq j$, with $\theta = 0.2, 0.4, 0.6,$ $0.8, 0.90, 0.95$. We compute the GCC $p$-value from the test statistic sequence. The simulation is repeated $10^7$ times. The simulated GCC $p$-values are sorted into bins with the bin edges being a sequence of edges from 0 to 1 with a width of  0.05. 				Each bin includes the right edge (right-closed) but does not include the left edge (left-open). We highlight the first bin in black and we
				also add a text note with the probability of $p$-values being in the first bin. }
		\end{tablenotes}
	\end{minipage}
\end{figure}

Interestingly,  \citet{liu2020cauchy} show that the tail property \eqref{eq:tail}  also holds when the number of hypotheses $d$ diverges to infinity at a rate of $o\left(h^{\eta }\right) \ $with $0<\eta <1/2$ and the following additional assumption is satisfied.
\begin{assumption}
	\label{ass2} Let $\mathbf{\Sigma }=corr\left( \bm{X}\right) $. 
	(1) The	largest eigenvalue of the correlation matrix $\lambda _{\max}\left( \mathbf{%
		\Sigma }\right) \leq C_0$ for some constant $C_0>0$; 
	(2) $\max_{1\leq i<j\leq
		d}\left\{ \sigma _{i,j}^{2}\right\} \leq \sigma _{\max }^{2}<1$ for some
	constant $0<\sigma _{\max }^{2}<1$, where $\sigma _{i,j}$ is the $\left(
	i,j\right) $ element of $\mathbf{\Sigma }$.
\end{assumption}
The additional assumptions on the correlation matrix are mild and standard in high dimensional settings and are general enough to incorporate a large class of tests. 


\subsection{Sequential Cauchy Combination Test}

The main contribution of this paper is the sequential Cauchy combination (SCC) test, which unravels the GCC test to make statements on the elementary hypotheses. The raw $p$-values are sorted in ascending order so that  $p_{(1)}\leq p_{(2)}\leq \ldots \leq p_{(d)}$, which is standard for step-down and step-up sequential procedures (see Section \ref{ssecOrderdPvals}). For the inference on hypothesis $H_{\left( i\right) }$ we compute a Cauchy combination test statistic ${\normalsize \tilde{T}}_{\left( i\right) }$ from a subset of $p$-values, running from $p_{(i)}$ to $p_{(d)}$ as:  
\begin{equation}
{\normalsize \ \tilde{T}%
		_{\left( i\right) }=\sum_{j=i}^{d}w_{j}\tan \{(0.5-p_{(j)})\pi \}
	}.
	\label{eq:CC_mt}
\end{equation}
The corresponding $p$-value is: $$ \widetilde{p}_{(i)}=\frac{1}{2}-\frac{\arctan \tilde{T}_{\left(i\right) }}{\pi }.$$ We reject the null hypothesis $H_{(i)}$ if  $\widetilde{p}_{(i)}\leq\alpha$. Like the step-up procedure of  \citet{hommel1988stagewise}, the SCC test also borrows power across hypotheses: the test statistic $\tilde{T}_{(i)}$ is computed from the raw $p$-values associated with $\mathcal{H}_0^{(i)}=\bigcap_{j=i}^{d} H_{(j)}$.


\subsubsection*{Theoretical Properties}
The SCC testing procedure can be viewed as a sequential rejection procedure. Let $\mathcal{R}^{(s)}$ be the collection of rejected hypothesis after step $s$, with $s=\left\{1,2,\ldots,d\right\}$. The hypothesis of interest and decision rules in each step  are illustrated in Table \ref{tabDecisionRule}.
\begin{table}[H]
	\caption{Decision rule in the sequential Cauchy combination test}
	\label{tabDecisionRule}
	\centering
	\begin{tabular}{p{1.cm}p{3.8cm}p{10.5cm}}
		\hline
		Step & Hypothesis & Decision\\ 
		$s=1$ & $\mathcal{H}_0^{\left( d\right) }=H_{(d)}$ & 
		If $\widetilde{p}_{(d)}\leq\alpha$ then reject $\mathcal{H}_0^{\left( d\right) }$ and include $H_{(d)}$ in $\mathcal{R}^{(1)}$
		\\
		$s=2$ & $\mathcal{H}_0^{\left( d-1\right) }=\bigcap_{j=d-1}^d H_{(j)}$  
		& 
		If $\widetilde{p}_{(d-1)}\leq\alpha$ then reject $\mathcal{H}_0^{\left( d-1\right) }$  and include $H_{(d-1)}$ in $\mathcal{R}^{(2)}$  
		\\
		$\ldots$  & $\ldots$  & $\ldots$  \\ 
		$s=d$ & $\mathcal{H}_0^{\left( 1\right) }=\bigcap_{j=1}^d H_{(j)}$ & 
		If $\widetilde{p}_{(1)}\leq\alpha$ then reject $\mathcal{H}_0^{\left( 1\right) }$ and include $H_{(1)}$ in $\mathcal{R}^{(d)}$ \\
		\hline
	\end{tabular}
\end{table}
Let $\mathcal{N}\left(\mathcal{R}^{(s)}\right)$ be the successor function, representing hypotheses to be rejected in the next step given that $\mathcal{R}^{(s)}$ has been rejected. For the SCC test, the successor function is defined as: 
\[
\mathcal{N}\left(\mathcal{R}^{(s)}\right)=\left\{H_{(d-s)} :  \widetilde{p}_{(d-s)} \leq \alpha_{\mathcal{R}^{(s)}}=\alpha\right\}.
\]
The cut-off value is fixed (i.e., $\alpha_{\mathcal{R}^{(s)}}=\alpha$) instead of depending on the rejection set $\mathcal{R}^{(s)}$ like in many other sequential procedures. According to the sequential rejection principle of \cite{goeman2010sequential},  the SCC test  achieves a strong family-wise error rate control if the following two conditions are satisfied. 
\begin{condition}[Monotonicity]
	For every $\mathcal{R}^{(s)}\subseteq \mathcal{R}^{(l)} \subset \mathcal{H}_{0}$, 
	\[
	\mathcal{N}(\mathcal{R}^{(s)}) \subseteq \mathcal{N}(\mathcal{R}^{(l)}) \cup \mathcal{R}^{(l)}
	\]
	almost surely. 
\end{condition}
% By construction, 
The transformed $p$-values of the SCC test are monotonic by construction, with $\widetilde{p}_{(d)}$ being the largest for the smallest set of global null hypotheses $\mathcal{H}_0^{(d)} = H_{(d)}$ and $\widetilde{p}_{(1)}$ being the smallest for the largest set of global nulls $\mathcal{H}_0^{(1)} = \bigcap_{j=1}^{d} H_{(j)}$ (see Figure \ref{figSequentialCauchyIllustration}(e) for an illustration of the monotonic $p$-values). Note that the largest set of global null hypotheses has the same null specification as the GCC test \eqref{eqCauchyStatistic}. It follows that that $\widetilde{p}_{(s)}\geq \widetilde{p}_{(l)}$. Since the cut-off value is fixed, the monotonicity condition of the successor function is satisfied.

\begin{condition}[Single-step condition] \label{SS} 
	When $\mathcal{H}_{0}^{(i)} =\mathcal{T}$, 
	$\Pr\left( \widetilde{p}_{(i)} \leq \alpha\right) \leq \alpha. $
\end{condition}
Condition \ref{SS} requires FWER control of the underlying test of SCC (i.e., the Cauchy combination test) at the ``critical case" in which all hypotheses of interest are true: $\mathcal{H}_{0}^{(i)} =\mathcal{T}$. The condition can be rewritten as $\Pr{\mathcal{N}(\mathcal{F})\subseteq \mathcal{F}} \geq 1-\alpha$ and has been shown to be satisfied by \cite{liu2020cauchy}. In fact,  when $\alpha$ is very small, the familywise false rejection probability of the GCC test under the null is not only bounded by $\alpha$ but also approaches the nominal size $\alpha$, as stated in \eqref{eq:wFWER}, which implies that it is less conservative than tests based on statistical inequalities or tests which impose independence in the presence of correlation. 

The theorem below follows directly from \cite[Theorem 1]{goeman2010sequential} for general sequential rejection procedures, so that Type I control in the critical case is sufficient for overall familywise error control of the sequential procedure. 
\begin{theorem}\label{thm}
	The SCC testing procedure satisfies both the monotonicity and the single-step condition and achieves the strong FWER control:  
	\[
	\lim_{\alpha\rightarrow 0}\Pr\left\{\mathcal{R}^{(d)} \subseteq \mathcal{F} \right\} \geq 1-\alpha, 
	\]
	under Assumption \ref{ass1} if $d$ is fixed and under Assumptions \ref{ass1} and \ref{ass2} if $d\rightarrow \infty$.
\end{theorem}


\subsubsection*{An Illustration}

A more prescriptive description of the SCC testing procedure is as follows: 
\begin{enumerate}
	
	\item Obtain raw $p$-values $p_1, p_2,\ldots, p_d$ corresponding to the null hypotheses $H_{1}, H_{2},\ldots, H_{d} $;%
	
	\item Order the raw $p$-values in ascending order, 	$p_{(1)},p_{(2)},\ldots,p_{(d)}$, with corresponding null ordered hypotheses $H_{(1)},H_{(2)},\ldots,H_{(d)}$;
	
	\item Calculate the SCC test statistic $\tilde T_{(i)}$ and the transformed Cauchy $p$-values $\widetilde{p}_{(i)}$ from a subset of the ordered $p$-values $\left\{p_{(j)}\right\} _{j=i}^{d}$ using \eqref{eq:CC_mt} for $i=1,\ldots,d$;
	
	\item Obtain the rejection set $\mathcal{R}=\left\{H_{\left(i\right)} : \widetilde{p}_{(i)}\leq \alpha\right\}$. 
\end{enumerate}

Figure \ref{figSequentialCauchyIllustration} illustrates the sequential Cauchy combination procedure on a simulated sequence of test statistics. The top row shows the simulated test statistics and their corresponding $p$-values, of which some hypotheses are under the null (grey dots) and some are under the alternative (black dots). The data-generating process is the same as that in Figure \ref{figCauchyPvalsAR} with $\theta=0.9$ and $d=100$. We add constant signals for $5$ out of 100 hypotheses,  with a signal strength equal to $\pm2.806$. The sign of the signal is the same as the sign of the test statistic under the null, such that the signal always amplifies the magnitude of the test statistic. 
The GCC test rejects the global null at $\alpha = 5\%$ for this sequence of $p$-values, which tells us there is at least one signal in the sequence.  
%The estimated first-order autocorrelation of the simulated test statistics is equal to $0.7910$ under the null and is equal to $0.4987$ under the alternative. 

\begin{figure}[p]
	\caption{Rejection procedure of the sequential Cauchy Combination test}
	\label{figSequentialCauchyIllustration}\centering

	\par
	
	\subfloat[Raw test statistics]{{\includegraphics[width=.31\textwidth,angle =
			-90]{1_tstat_d_100_rho_0.9_signal_5_5} }} 
	\subfloat[Raw
	$p$-values]{{\includegraphics[width=.31\textwidth,angle = -90]{2_pvals_d_100_rho_0.9_signal_5_5} }}
	
	\vspace{0.4cm}	
	
	\subfloat[Ordered raw $p$-values]{{\includegraphics[width=.31\textwidth,angle =
			-90]{3_spvals_d_100_rho_0.9_signal_5_5} }} 
	
	\vspace{0.4cm}	
	
	\subfloat[SCC test statistics
	]{{\includegraphics[width=.31\textwidth,angle = -90]{4_ctstats_d_100_rho_0.9_signal_5_5}}}
	\subfloat[SCC 
	$p$-values]{{\includegraphics[width=.31\textwidth,angle = -90]{4_cpvals_d_100_rho_0.9_signal_5_5}}}
	
	
	\begin{minipage}{1.0\linewidth}
		\begin{tablenotes}
			\small
			\item {
				\medskip
				Note: We illustrate the mechanics of the SCC procedure on a simulated test statistic sequence with sparse signals. The top row shows raw test statistics and $p$-values of which some hypotheses are under the null and some are under the alternative. The test statistics are simulated from $N_{d}(\bm{0},\bm{\Sigma})$ as in Figure \ref{figCauchyPvalsAR}. We set $d=100$, $\theta=0.9$ and add $5\%$ signals. The strength of the signal is $\pm2.806$, with its sign identical to that of the test statistic under the null. The horizon line in panel (e) is the 5\% significance level.
			}
		\end{tablenotes}
	\end{minipage}
\end{figure}

The SCC test can tell us which individual $p$-values trigger the rejection of the GCC test. The middle row plots the raw $p$-values in ascending order and the bottom row plots its sequential Cauchy combination test statistics and $p$-values. Specifically, the bottom right panel shows that the SCC $p$-values $\widetilde{p}_{(i)}$ decrease as $i$ moves from $d$ to $1$. In this example, the SCC test rejects three out of the five alternative hypotheses and does not reject under the null hypothesis. The rejections correspond to the 4$^\text{th}$, 29$^\text{th}$ and  46$^\text{th}$ hypotheses in the top row. Note that the smallest SCC $p$-value corresponds to the $p$-value of the GCC test of \citet{liu2020cauchy}, which performs the test on the largest set of hypotheses. 
% !TEX root = ./CauchyCombination.tex
\section{Simulations}
\label{secSims}

In this section, we compare the performance of the SCC test against several popular multiple testing corrections in a simulation study, considering different forms of dependence. 
The benchmark procedures include four inequality-based approaches: the Bonferroni correction and its subsequent improvements proposed by  \citealp{holm1979simple}, \citealp{hommel1988stagewise} and \citealp{hochberg1988sharper}, as well as the Gumbel approach.  
Detailed discussions of these benchmark procedures can be found in the Online Supplement.  

\subsection{Under the null hypothesis}
\label{ssecAccuracy}

We assess the statistical performance of the different multiple testing corrections under the null hypothesis. 
To measure the empirical familywise error rate, we conduct $S=10^4$ replications for each method $m$, and calculate  $\widehat{FWER}_m$ as follows: 
\[
\widehat{FWER}_m=\frac{1}{S}\sum_{s=1}^{S} \bm{1}\left( \min_{i\in\left\{1,2,\cdots,d\right\}} \left\{p_{(m,i)}^{s}\right\}\leq\alpha\right),
\]
where $\bm{1}(.)$ is the indicator function, and $p_{(m,i)}^{(s)}$ represents the $p$-value of the $i$th hypothesis for method $m$ in the $s$th replication. 
When the test statistics exhibit strong dependence, we expect the  $\widehat{FWER}$ of the SCC test to be closer to the nominal level $\alpha$ compared to the other procedures. 

Under the null hypothesis, the test statistics $\bm{X}$ are generated from a $d$-variate normal distribution with zero mean and covariance matrix $\bm{\Sigma}$, i.e., $N_d(\bm{0}, \bm{\Sigma})$. 
We set the dimension $d$ to 100. 
The diagonal elements of the covariance matrix, $\sigma_{ii}$, are all equal to $1$, for $i=1,\ldots,d$. 
The off-diagonal element,  $\sigma_{ij}$ with $i\neq j$, adhere to three specific  specifications. 
\begin{itemize}
	
	\item Model 1. Exponential decay: $\sigma_{ij} = 	\theta^{\abs{i-j}}$ with $\theta= 0.2, 0.4, 0.6,$ $0.8, 0.90, 0.95$.
	
	\item Model 2. 	Polynomial decay: $\sigma_{ij} = \frac{1}{0.7 + \abs{i - j}^\theta}$ with $\theta =	1.0, 1.5, 2.0, 2.5$. 
			
	\item Model 3.	Block-diagonal:  $\bm{\Sigma} = \text{diag}\{A_1,\ldots, A_{d/10}\}$, for which each diagonal block $A_k$ is a $10 \times 10$ equi-correlation matrix with its off-diagonals $\sigma_{ij} = \theta$ and $\theta = 0.1, 0.3, 0.5, 0.7, 0.9$. 
	
\end{itemize}


Models 1 and 2 also appear in \citet{liu2020cauchy}. 
The exponentially decaying correlation structures in Model 1 are frequently observed in time series and financial econometrics. 
For instance, in Section \ref{ssecDB}, the sequence of drift burst test statistics constructed from overlapping rolling windows, exhibits an autoregressive process  and has an exponential decaying covariance structure, as shown by \citet{christensen2018drift}. 
The block-diagonal structure in Model 3 is commonly used 
when testing high-dimensional factor pricing models  \citep[see e.g., the Monte Carlo experiments in][]{fan2015power} and emulates a cross-sectional dependence structure. 


Table \ref{tabSizeGlobalSequentialOtherMethods} shows the superior performance of the SCC test in the presence of correlated test statistics under the null hypothesis. 
The empirical familywise error rate of the SCC test, reported in the last column, remains close to the nominal level $\alpha = 5\%$ across various correlation structures, demonstrating its robustness. These findings align with the theoretical discussions presented in Section \ref{secPrelims}. 
In contrast, the inequality-based procedures and the Gumbel method exhibit greater conservatism, as evidenced by their substantially lower FWERs (although slight variations exist depending on the correlation pattern). 
The SCC test stands out as the only  controlling procedure with an empirical FWER close to the nominal level (5\%) across all three  types of correlation structures in the test statistics.   


\begin{table}[htb!] 
	\centering
	\caption{Empirical FWERs (in\%) of the controling procedures}
	\label{tabSizeGlobalSequentialOtherMethods}
	\begin{adjustbox}{max width=\textwidth}
		\begin{tabular}{c ccccccc}
			\hline
			\multicolumn{1}{c}{$\theta$}
			&\multicolumn{1}{c}{Bonferroni}
			&\multicolumn{1}{c}{Holm} 
			&\multicolumn{1}{c}{Hommel}  
			&\multicolumn{1}{c}{Hochberg} 
			&\multicolumn{1}{c}{Gumbel}
			&\multicolumn{1}{c}{SCC} 
			\\
			\hline
			\multicolumn{7}{c}{\textbf{Model {1}:} Exponential decay} \\	
			0.2  & 4.68 & 4.68 & 4.68 & 4.68 & $\underline{3.30}$ &  4.94 \\ 
			0.4 & 4.88 & 4.88 & 4.88 & 4.88 &  $\underline{3.44}$ &  5.36 \\ 
			0.6 & 4.96 & 4.96 & 4.96 & 4.96 & $\underline{3.54}$ &  5.78 \\ 
			0.8  & $\underline{3.98}$ & $\underline{3.98}$ & $\underline{4.00}$ & $\underline{3.98}$ & $\underline{2.84}$ & 5.82 \\ 
			0.9 & $\underline{2.40}$ & $\underline{2.40}$ & $\underline{2.42}$ & $\underline{2.40}$ & $\underline{1.70}$ & 5.48 \\ 
			0.95 & $\underline{1.76}$ & $\underline{1.76}$ & $\underline{1.78}$ & $\underline{1.76}$ & $\underline{1.18}$   & 5.38 \\ \hline
			\multicolumn{7}{c}{{\textbf{Model {2}:} Polynomial decay}} \\	
			1.0 & 4.70 & 4.70 & 4.70 & 4.70 & $\underline{3.38}$  &  5.96 \\ 
			1.5  & 4.62 & 4.62 & 4.62 & 4.62 & $\underline{3.62}$ & 5.48 \\ 
			2.0  & 4.44 & 4.44 & 4.44 & 4.44 & $\underline{3.18}$ & 5.38 \\ 
			2.5  & 4.46 & 4.46 & 4.46 & 4.46 & $\underline{3.16}$  & 5.14 \\ \hline
			\multicolumn{7}{c}{\textbf{Model {3}:} Block-diagonal} \\	
			0.1  & 4.56 & 4.56 & 4.58 & 4.56 & $\underline{3.50}$ & 4.92 \\ 
			0.3  & 4.74 & 4.74 & 4.76 & 4.74  &   $\underline{3.74}$ & 5.32\\
			0.5 &  4.54 & 4.54 & 4.54 & 4.54 & $\underline{3.28}$ &  5.70 \\ 
			0.7  & $\underline{3.40}$ & $\underline{3.40}$ & $\underline{3.40}$ & $\underline{3.40}$ &  $\underline{2.46}$ & 5.62\\
			0.9 &  $\underline{1.88}$ & $\underline{1.88}$ & $\underline{1.92}$ & $\underline{1.88}$ &  $\underline{1.30}$ & 5.62\\ 
			\hline
			\hline
			
	\end{tabular}}
\end{adjustbox}	
\parbox{0.76\textwidth}{\footnotesize%
	\vspace{.1cm} % If wanted space after the bottomrule
	{Note}: We 	report the empirical FWERs (frequencies of falsely rejecting at least one hypothesis) of the controlling procedures. 
	The test statistics are generated from $N_d(\bm{0}, \bm{\Sigma})$ with different correlation structures. 
    The dimension $d$ is $100$ and the nominal significance level $\alpha$ is 5\%. The number of replications is $10^4$. 
	Instances with lower than 4\% FWER are underlined. 
}
\end{table}



\subsection{Under the alternative hypothesis}

Under the alternative hypothesis, the performance of the controlling procedures is assessed based on their global power (the percentage of replications that reject at least one hypothesis) and their successful detection rate (the percentage of overlapping hypotheses between the sets of false hypotheses  and discoveries). 
Given the improved accuracy of the SCC procedure in controlling the FWER under the null hypothesis (as demonstrated in Table \ref{tabSizeGlobalSequentialOtherMethods}), we anticipate that the SCC test will exhibit higher power when applied under the alternative hypothesis.
 
The test statistic vector $\bm{X}$ is generated from a $d$-variate normal distribution with mean vector $\bm{\mu} = (\mu_i)$  and a correlation matrix $\bm{\Sigma} = (\sigma_{ij})$, i.e., $N_d (\bm{\mu}, \bm{\Sigma})$. 
We adopt the same correlation matrices $\bm{\Sigma}$ as discussed in Section \ref{ssecAccuracy}.
The percentage of signals (i.e., non-zero $\mu_i$'s in the vector $\bm{\mu}$) is set to be 5\% (specifically, out of the 100 hypotheses, 5 are under the alternative).
All signals have the same strength $\abs{\mu_i} = \mu_0$ which is chosen to be relatively weak, i.e.,  $\pm2.1737$.\footnote{
	The chosen signal strength ensures that the test power  converges to unity as  $d \rightarrow \infty$ in the case of sparse signals, following the result presented in Theorem 3 of \citet[][]{liu2020cauchy}. 
	} 
The sign of the signal aligns with the sign of the test statistic under the null so that the signal always amplifies the magnitude of the test statistic.


\begin{table}[h] 
	\centering
	\caption{Global power (in\%) in a correlated setting with sparse signals}
	\label{tabPowerGlobalSequentialOtherMethods}
	\begin{adjustbox}{max width=\textwidth}
		\begin{tabular}{c cccccc}
			\hline
			\multicolumn{1}{c}{$\theta$}
			&\multicolumn{1}{c}{Bonferroni}
			&\multicolumn{1}{c}{Holm} 
			&\multicolumn{1}{c}{Hommel}  
			&\multicolumn{1}{c}{Hochberg} 
			&\multicolumn{1}{c}{Gumbel}
			&\multicolumn{1}{c}{SCC} \\
			\hline
			\multicolumn{7}{c}{{\textbf{Model 1:} Exponential decay}} \\
			0.2 & 66.50 & 66.50 & 66.62 & 66.50 & $\underline{59.32}$ & 76.20 \\ 
			0.4 & 66.48 & 66.48 & 66.60 & 66.48 & $\underline{59.86}$ & 76.84 \\ 
			0.6 & 65.70 & 65.70 & 65.74 & 65.70 & $\underline{58.58}$ &  75.82 \\ 
			0.8 & $\underline{63.86}$ & $\underline{63.86}$ & $\underline{63.90}$ & $\underline{63.86}$ & $\underline{56.90}$ & 72.74 \\ 
			0.9 & $\underline{60.62}$ & $\underline{60.62}$ & $\underline{60.78}$ & $\underline{60.64}$ & $\underline{53.90}$ & 68.42 \\ 
			0.95 & $\underline{57.24}$ & $\underline{57.24}$ & $\underline{57.30}$ & $\underline{57.24}$ & $\underline{50.58}$ & 63.94 \\ \hline
			\multicolumn{7}{c}{{\textbf{Model 2:} Polynomial decay}} \\
			1.0  & 66.04 & 66.04 & 66.14 & 66.04 & $\underline{58.76}$ &  74.82 \\ 
			1.5  & 65.88 & 65.88 & 66.00 & 65.88 & $\underline{58.46}$  &  75.10 \\ 
			2.0  & 65.88 & 65.88 & 66.00 & 65.88 & $\underline{58.84}$ & 75.20 \\ 
			2.5  & 65.22 & 65.22 & 65.28 & 65.22 & $\underline{58.22}$ & 74.66 \\ \hline
			\multicolumn{7}{c}{\textbf{Model {3}:} Block-diagonal} \\
			0.1 & 66.44 & 66.44 & 66.50 & 66.44 & $\underline{59.84}$ &   76.50 \\ 
			0.3  & 67.10 & 67.10 & 67.18 & 67.10 & $\underline{60.40}$  & 76.00 \\ 
			0.5  & 65.84 & 65.84 & 65.94 & 65.84 &$\underline{58.48}$ &  74.44 \\ 
			0.7 & $\underline{63.72}$ & $\underline{63.72}$ & $\underline{63.74}$ & $\underline{63.72}$ & $\underline{57.34}$ & 71.62 \\ 
			0.9 & $\underline{60.88}$ & $\underline{60.88}$ & $\underline{61.02}$ & $\underline{60.88}$ & $\underline{54.10}$ & 69.22 \\ 
			\hline
			\hline
	\end{tabular}}
\end{adjustbox}	
\parbox{0.76\textwidth}{\footnotesize%
	\vspace{.1cm} % If wanted space after the bottomrule
	{Note}: 
	We report the global powers (frequencies of rejecting at least one hypothesis), for various correlation matrices in the presence of sparse signals. 
	The test statistics are generated from $N_d(\bm{\mu}, \bm{\Sigma})$ with different correlation structures and sparse signals. 
	The dimension $d$ is fixed at $100$, and the percentage of signals is set to  $5\%$. All the signals have the same strength ($\pm2.1737$), with the sign depending on the sign of the test statistic under the null.
	We use a nominal significance level of 5\% and conduct $10^4$  replications. 
	We underline instances where the FWERs were lower than  4\% in Table \ref{tabSizeGlobalSequentialOtherMethods}. 
}
\end{table}


Table \ref{tabPowerGlobalSequentialOtherMethods}  shows the superior global power of the SCC test in the presence of correlated test statistics and sparse signals. 
The SCC test exhibits an approximate $10$\% power enhancement compared to the runner-up method. 
When the significance level $\alpha$ is set to $5\%$, the power of the SCC test ranges between  $69$\% and $77$\%. 
Although each statistical inequality-based method improves upon its predecessor in certain aspects, we do not observe a significant difference in the frequency of rejections among the four approaches. 
As anticipated, the Gumbel approach, which assumes independent test statistics,  is the most conservative test. 

Table \ref{tabPowerLocalSequentialOtherMethods} tells a similar story with respect to the average numbers of successful detections. The SCC testing procedure successfully detects approximately 1.2 out of 5 hypotheses (or $24\%$) under the alternative, even with sparse and weak signals.\footnote{The average number of false detections is slightly higher (not reported) for the SCC test, amounting to falsely rejecting on average 0.1 (out of 95) true hypotheses.} 
Meanwhile, the average number of successful detection for the  inequality-based procedures are around 0.95 (or $19\%$), whereas the value for the Gumbel procedure is about 0.81 (or $16.2\%$).


\begin{table}[!ht] % htb!
	\centering
	\caption{Successful detection rates (in\%)  in a correlated setting with sparse signals}
	\label{tabPowerLocalSequentialOtherMethods}
	\begin{adjustbox}{max width=\textwidth}
	\begin{tabular}{c cccccc}
	\hline
	\multicolumn{1}{c}{$\theta$}
	&\multicolumn{1}{c}{Bonferroni}
	&\multicolumn{1}{c}{Holm} 
	&\multicolumn{1}{c}{Hommel}  
	&\multicolumn{1}{c}{Hochberg} 
	&\multicolumn{1}{c}{Gumbel}
	&\multicolumn{1}{c}{SCC} 
	\\
	\hline
	\multicolumn{7}{c}{\textbf{Model {1}:} Exponential decay}  \\
	0.2 & 18.80 & 19.00 & 19.00 & 19.00 & $\underline{16.00}$ & 23.20 \\ 
	0.4 & 19.20 & 19.20 & 19.20 & 19.20 & $\underline{16.20}$ & 23.60 \\ 
	0.6 & 18.80 & 19.00 & 19.00 & 19.00 & $\underline{16.00}$ & 23.40 \\ 
	0.8 & $\underline{19.20}$ & $\underline{19.40}$ & $\underline{19.40}$ & $\underline{19.40}$ &  $\underline{16.20}$ & 24.00 \\ 
	0.9 &  $\underline{19.00}$ & $\underline{19.00}$ & $\underline{19.20}$ & $\underline{19.00}$ & $\underline{16.00}$  & 24.00 \\ 
	0.95 &  $\underline{19.20}$ & $\underline{19.20}$ & $\underline{19.20}$ & $\underline{19.20}$ & $\underline{16.20}$ & 24.40 \\ \hline
	\multicolumn{7}{c}{\textbf{Model {2}:} Polynomial decay}  \\
	1.0 & 19.00 & 19.20 & 19.20 & 19.20 & $\underline{16.00}$  & 23.60 \\ 
	1.5 & 19.00 & 19.20 & 19.20 & 19.20 &$\underline{16.00}$  & 23.40 \\ 
	2.0 & 19.00 & 19.20 & 19.20 & 19.20 &$\underline{16.00}$ & 23.60 \\ 
	2.5 & 18.80 & 18.80 & 18.80 & 18.80 & $\underline{16.00}$ & 23.40 \\ \hline
	\multicolumn{7}{c}{
	\textbf{Model {3}:} Block-diagonal
	} \\
	0.1 & 19.40 & 19.40 & 19.40 & 19.40 & $\underline{16.40}$  & 23.60 \\ 
	0.3 & 19.40 & 19.60 & 19.60 & 19.60 & $\underline{16.60}$  & 23.60 \\ 
	0.5 & 19.40 & 19.40 & 19.40 & 19.40 &  $\underline{16.40}$ & 23.80 \\ 
	0.7 & $\underline{19.20}$ & $\underline{19.40}$ & $\underline{19.40}$ & $\underline{19.40}$ & $\underline{16.40}$  & 23.80 \\ 
	0.9 & $\underline{19.00}$ & $\underline{19.00}$ & $\underline{19.00}$ & $\underline{19.00}$ & $\underline{16.20}$  & 23.80 \\ 
	\hline
	\hline
	\end{tabular}}
	\end{adjustbox}	
	\parbox{0.76\textwidth}{\footnotesize%
	\vspace{.1cm} % If wanted space after the bottomrule
	{Note}: 
	We report the successful detection rates (overlap between the sets of alternative hypotheses and discoveries). 
	The data-generating processes used are the same as those for Table \ref{tabPowerGlobalSequentialOtherMethods}.
	We use a nominal significance level of 5\% and conduct $10^4$  replications. 
	We underline instances where the FWERs were lower than  4\% in Table \ref{tabSizeGlobalSequentialOtherMethods}.
}
\end{table}
 
% !TEX root = ./CauchyCombination.tex
\section{Example 1: Monitoring Drift Burst in Financial Assets}
\label{secApplDriftBurst}

The drift burst hypothesis of \citet{christensen2018drift} postulates the existence of locally explosive trends (like flash crashes or gradual jumps) in high-frequency asset prices. 
The drift burst test detects the presence and timestamps these explosive trends. 
The test statistic is computed minute-by-minute. 
%  using observations from $[t-h,t]$, where $h$ is a bandwidth parameter. 
Assuming 6.5 trading hours per day, the test needs to be conducted 341 times per day (with a burn-in period of 49 minutes) and is thus a % difficult 
multiple testing problem. 
The test statistics are computed from overlapping intra-day windows and are expected to be highly autocorrelated. Drift bursts are rare events 
% (i.e., sparse signals) 
and the strength of the signals varies. 
There are several approaches to deal with the false discovery problem. 
\citet{christensen2018drift} propose a resampling procedure which generates simulated critical values for the drift burst test. 
We gauge the performance of the SCC procedure and the benchmark approaches in both simulation and empirical settings. 

Section \ref{ssecDB} introduces the drift burst hypothesis  and the drift burst test. 
Section \ref{secResampling} describes the resampling procedure proposed by \citet{christensen2018drift}. 
A simulation study is conducted in Section \ref{ssecDBSim}. Unlike the simulation conducted in Section \ref{secSims}, we simulate asset prices and compute the drift burst test statistics, instead of directly simulating test statistics. The controlling procedures are applied to detect drift bursts in the Nasdaq composite index and S\&P 500 index ETFs in Section \ref{ssecDBEmpirics}.
 
\subsection{Drift Burst Hypothesis and Test}\label{ssecDB}

The noisy asset log price $\widetilde{P}_{t_i} = P_{t_i} + \epsilon_{t_i}$, is recorded at equidistant intervals $0 = t_0 < t_1 < \ldots <t_n = T$, where $T$ is fixed (e.g., one trading day consisting of 6.5 trading hours), $P$ is the latent log price process, and $\epsilon$ is an additive error term (noise) which is independent from $P$. The distance between two consecutive observations is $\Delta=t_{i+1}-t_{i}$. The frictionless log prices $P = (P_t)_{t \geq 0}$ follow an It\^o semi-martingale process defined on a filtered probability space $(\Omega, \mathcal{F}, (\mathcal{F})_{t \geq 0}, \Prob)$: 
\begin{eqnarray}
	\label{eqDGPNull}
	dP_t = \mu_t dt + \sigma_t dW_t,  %+ dJ_t
\end{eqnarray}
in which $\mu_t$ is the instantaneous drift and the diffusive component consists of the spot volatility $\sigma_t$ and a standard Brownian motion $W_t$. 

Under the null hypothesis, the coefficients $\mu_t$ and $\sigma_t$ are locally bounded or non-explosive. Over a vanishing time interval $\Delta \rightarrow 0$ the drift is $O_p (\Delta)$ and swamped by the diffusive component of a larger order $O_p (\sqrt{\Delta})$, because $\Delta  \ll \sqrt{\Delta}$, meaning that over short time intervals the main contributor to the log return is volatility. 

Under the alternative hypothesis,  a drift-bursting term $\mu_t^\text{b}$ and a volatility-bursting component $\sigma_t^\text{b}$ are added to the standard It\^o semi-martingale process \eqref{eqDGPNull} such that: 
\begin{eqnarray}
	\label{eqDGPAlt}
	d{P}_t =& 
	\mu_t dt 
	+ \sigma_t dW_t 
	+  \mu_t^\text{b }dt
	+ \sigma_t^\text{b} dW_t, 
\end{eqnarray}
for which $\abs{\mu_t^\text{b}} / \sigma_t^\text{b} \rightarrow \infty$ as $t \rightarrow \tau_{\text{b}}$ with $\tau_{\text{b}}$ being the drift burst time.  An example of such an explosive process is the following: 
\begin{eqnarray}
	\label{simDB}
	\mu_t^\text{b} = 
	a \frac{\text{sign}(t-\tau_{\text{b}})}{\abs{\tau_{\text{b}}-t}^{\alpha^\text{b}}}
	\, \, \text{and} \, \, 
	\sigma_t^\text{b} = b \frac{\sqrt{\theta}}{\abs{\tau_{\text{b}} - t}^{\beta^\text{b}}},  
	\, \, \text{for} \, \, 
	t \in [\tau_\text{b} - \Delta t, \tau_\text{b} + \Delta t], 
\end{eqnarray}
in which $2\Delta t$ is the duration of the burst, $\alpha^\text{b}$ is the strength of the burst, $\beta^\text{b}$ is the strength of the volatility burst and $a$ and $b$ are constants. This data-generating process can generate many realistic patterns including Flash Crashes and mildly explosive trends and is used in the simulations in Section \ref{ssecDBSim}.%\footnote{\citet{christensen2014fact} only allows volatility bursts and \citet{andersen2020drift} only consider drift bursts.} 


The noise-robust drift burst statistic of \citet{christensen2018drift} is defined as: 
\begin{eqnarray}
	\label{eqDBTestStat}
	X_i = 	\sqrt{h}	\frac{\hat{\bar{\mu}}_{t_i}}{\sqrt{\hat{\bar{\sigma}}_{t_i}^{2}}},
\end{eqnarray}
in which $h$ is the bandwidth of the mean estimator, $\hat{\bar{\mu}}_{t_i}$ is a noise-robust estimator for the local drift, and $\hat{\bar{\sigma}}_{t_i}^{2}$ is a noise-robust estimator for the local variance. 
The test is applied on a coarse sampling grid. The null hypothesis of the drift burst test is therefore that `there is no drift bursting at time period $t_i$'.
We refer to Appendix \ref{AppDBEstim} for details on the implentation of the noise-robust estimators for the local drift and the local variance. 

The size and power theorems in \citet[][Theorems 5 and 6, respectively]{christensen2018drift} formalize the limiting behavior of the test statistic under the null and the alternative.
Under the null hypothesis, the test statistic \eqref{eqDBTestStat} converges to the standard normal distribution as $\Delta\rightarrow 0$, i.e., $X_i \rightarrow^{d} N(0,1)$, The test statistic fulfills the assumptions required by the Cauchy combination test when the sampling frequency is sufficiently high (i.e., $\Delta\rightarrow 0)$ under the null. Under the alternative hypothesis, the test statistic diverges when the drift term explodes fast enough relative to the volatility, i.e., $\abs{X_i} \rightarrow \infty$, at the drift burst time. 

The drift burst test statistics are computed minute-by-minute from overlapping rolling windows. Figure \ref{figTestStatistic} shows the dependence of the drift burst test statistics.  Panel (a) shows  second-by-second prices generated using \eqref{eqDGPAlt} under the null hypothesis and under the alternative hypothesis of a 20-minute V-patterned drift burst in the middle of the day. Panel (b) shows the corresponding minute-by-minute drift burst test statistics (with a burn-in sample of 49 minutes). The test statistic sequence is very smooth, with a first-order autocorrelation coefficient of $0.885$ under the null hypothesis and $0.895$ under the alternative hypothesis. The autocorrelation function resembles that of an AR(1) process, decaying at an exponential rate as in Model 1 in the previous section. %with the exception that the test statistics are computed from simulated log prices, instead of being simulated directly from an underlying process

\begin{figure}[!h] % 
		\caption{Persistence in the drift burst test statistics for a day without and with a 20-minute drift/volatility burst} 
		\label{figTestStatistic}\centering
		
		\subfloat[Log prices ]{{\includegraphics[width=.35\textwidth,angle = -90]{fc_price} }}
		\subfloat[Drift burst test statistics]{{\includegraphics[width=.35\textwidth,angle = -90]{fc_test} }}
		
		\vspace{.5cm}
		
		\subfloat[
			Autocorrelogram of the test statistics
		]{{\includegraphics[width=.35\textwidth,angle = -90]{acf_fc_both} }}
		
		
		\begin{minipage}{1.0\linewidth}
			\begin{tablenotes}
				\small
				\item {
					\medskip
					Note: 
					This figure highlights the smoothness of the computed drift burst test statistic sequence. Panel (a) shows the simulated log prices at the one-second frequency with a 20-minute Flash Crash. In Panel (b) we show the corresponding 341 drift burst test statistics (calculated minute-by-minute with a burn-in of 49 minutes). In Panel (c), we show the autocorrelogram of the drift burst test statistics and the implied autocorrelations from an AR(1) process. The drift burst statistic is highly autocorrelated. The estimated first-order autocorrelation of the test statistic $\hat{\theta}$ equals $0.8847$ for the sample path under the null and $0.8949$ for the sample path under the alternative. 
			}
			\end{tablenotes}
		\end{minipage}
	\end{figure}

The number of tests or the dimension of the test statistic sequence $\bm{X}=\left\{X_i\right\}$ is large (e.g., $d=341$ for the period of one day). 
%If quantiles of the standard normal distribution are used for critical values, the probability of falsely detecting at least one signal within a day is close to one. 
To control for FWER, the aforementioned controlling procedures can be used, including the four inequality-based procedures, the Gumbel, and the SCC procedure. Since the test statistics are highly correlated, the Gumbel procedure is expected to be conservative.  

\subsection{Resampling-based Method}
\label{secResampling}

\citet[][Appendix B]{christensen2018drift} propose a resampling-based approach for generating critical values of the drift burst test to control the familywise error rate. Resampling aims to approximate the dependence structure of the test statistic sequence under the null hypothesis and to obtain its distribution via simulation. Specifically, the dependence structure is approximated by a stationary Gaussian AR(1) process such that:  
\begin{equation} 
	\label{eqXAR1}
	X_{i}=\theta X_{i-1}+\epsilon _{i}, 
	\, 
	\text{for } i = 1, \ldots, d,
	\text{ with }\abs{\theta}<1\text{ and }%
	\epsilon _{i}\overset{i.i.d.}{\sim }N(0,1-\theta^{2})\text{.}
\end{equation}
Under the specification of \eqref{eqXAR1}, the autocovariance function of $X_i$ is $cov\left( X_{i},X_{j}\right) =\theta^{\left\vert i-j\right\vert }$, which decays exponentially. 
% and coincides with Model 1 considered in Section \ref{secSims}.  
The idea is to generate $R$ sequences of test statistics from \eqref{eqXAR1}, compute $X_{m}^{r} = \max_{i = 1, \ldots, d} \abs{X^r_i}$ for each $r=1,\ldots,R$, and take the $1-\alpha $ quantile of the simulated maxima $\left\{ X_{m}^{r}\right\} _{r=1}^{R}$ as the critical value of the two-sided drift burst test. The autoregressive coefficient $\theta$ can be replaced with an estimate from the observed test statistic sequence $\left\{X_{i}\right\} _{i=1}^{d}$ using  conditional maximum likelihood.
When the autocorrelation structure of $X_i$ indeed follows an AR(1) process and when the estimate is precise ($\hat{\theta}$ is close to $\theta$), the resampling-based method is expected to be less conservative than the inequality-based methods and the Gumbel, since it accounts for  data-specific dependence. 

The critical value is unique for each sequence of test statistics (each day in an empirical analysis or each sample path in a Monte Carlo study) and obtained from simulations.  It is therefore more computationally demanding than other multiple testing procedures. We can save time by preparing a table with the quantile functions of the normalized maxima for different values of the autoregressive coefficient $\theta$ and dimensions $d$, but an interpolation routine is required when the estimated first-order autocorrelation and dimension are not included in the table. 

Figure \ref{figCOR18_fig8} plots the simulated critical values as a function of the autoregressive coefficient $\theta$ for $d = 2,500$ at three significance levels (i.e., 1\%, 5\%, and 10\%), as in \citet[][Figure 8]{christensen2018drift}.\footnote{The number of replications $R=10^7$ with a burn-in of $10^4$ observations.}  The parameter $\theta$ varies from $-0.5$ to $0.99$.
% \footnote{Unlike \cite{christensen2018drift}, we also consider negative $\theta$s.} 
These simulated finite-sample critical values are compared to the ones from the asymptotic Gumbel distribution. When the degree of autocorrelation is low (i.e., $\theta \approx 0$) and the confidence level is $10\%$, the gap between the Gumbel and simulated critical values is small. However, the gap becomes more pronounced as we move further to the tail (e.g., $\alpha=1\%$) even when the test statistics are uncorrelated, as the convergence in law of the maximum term to the Gumbel is known to be slow \citep[see e.g., the discussion in][Appendix B, and references therein]{christensen2018drift}. The gap starts to grow noticeably wider in the region where the autoregressive coefficient exceeds $0.7$, a situation we often encounter in rolling window implementations.
 
\begin{figure}[!ht]
	\centering
	\caption{Simulated critical values for the drift burst test}
	
	\includegraphics[width=.8\textwidth,angle = 0]{cv_sim_edit}
	
	\label{figCOR18_fig8}%
	
	\begin{minipage}{1.0\linewidth}
		\begin{tablenotes}
			\small
			\item {
				\medskip
				Note: 
				This figure plots the (resampling-based critical values of the maxima when $d = 2,500$ and the Gumbel critical values. The figure shows how the varying degree of dependence (captured by $\theta$) affect the critical values at three different confidence levels. It extends \citet[][Figure 8]{christensen2018drift} to allow $\theta$ to take negative values.
			}
		\end{tablenotes}
	\end{minipage}
\end{figure}

In practice, the degree of dependence under the null (i.e., $\theta$) is estimated from contaminated data (test statistics computed from data generated under both the null and alternative) and hence the autocorrelation and the simulated critical value could be biased. The Flash Crash in Figure \ref{figTestStatistic} includes about 20 (out of 341) time intervals under the alternative, which leads to slightly upward biased the first-order autocorrelation ($0.885$ \textit{vs.} $0.895$). The bias can be much larger when there are more observations under the alternative hypothesis or when the explosive trend does not have a reversal like in the V-pattern. Note that with the high autocorrelation of the drift burst statistics, we are operating on the far right of Figure \ref{figCOR18_fig8}. Even small downward (resp. upward) biased estimates of the autocorrelation under the null hypothesis lead to a loss of (resp. spurious) power. The bias is expected to be significantly larger in the case of a long-lasting drift burst as we see in the Nasdaq composite index (see Section \ref{ssecDBEmpirics}). We believe this mixing of intervals under the null and the alternative hypotheses in empirical data manifests in the attenuation bias reported  in \citet[][Appendix B, pp. 494]{christensen2018drift}:  ``The estimated ACF [of an AR(1)]  is close to the observed one [for the simulated data], although there is a slight attenuation bias for the empirical estimates". 

Obviously, when the underlying correlation structure is far from an AR(1), the approximation \eqref{eqXAR1} is not appropriate and might lead to incorrect inferences. In contrast, the SCC test solves the problem of dependence in a much simpler way and accommodates arbitrary dependency structures in the test statistics.


\subsection{Simulation Study}\label{ssecDBSim}

We compare the finite sample performance of the SCC test with existing controlling procedures (including the four inequality-based, Gumbel and the resampling-based methods) for the drift burst test. 

\subsubsection{Data-generating processes}

Instead of simulating directly the test statistics as in Section \ref{secSims}, we generate the log prices from the data generating process \eqref{eqDGPNull}. Under the null hypothesis we set $\mu_t^b=\sigma_t^b=0$ in \eqref{eqDGPAlt} while under the alternative we set $\mu_t^b$ and $\sigma_t^b$ as in \eqref{simDB}. 

The diffusion coefficient $\sigma_t$ is assumed to follow a \citet{heston1993closed} process:  
	\begin{equation}
		d\sigma_t^2 = \kappa (\theta - \sigma_t^2)dt + \xi \sigma_t dB_t, 
		\text{with} \, t \in [0,1], 
	\end{equation}
where $B_t$ is a standard Brownian motion and correlated with $W_t$, the standard Brownian motion of the mean equation \eqref{eqDGPAlt}, i.e., $E[dW_t,dBt] = \delta dt$. As in \citet{christensen2018drift}, we assume the annualized parameters of the variance process are $(\kappa, \theta, \xi, \delta) = (5, 0.0225, 0.4, -\sqrt{0.5})$. The parameter $\theta$ corresponds to a return volatility of roughly 20\% per annum. In each simulation, $\sigma_t^2$ is initiated at a random draw from its stationary law $\sigma_t \sim \text{Gamma} (2 \kappa \theta \xi^{-2}, 2 \kappa \xi^{-2})$. The drift coefficient $\mu_t$ is set to zero. 

Under the alternative, we consider two different settings for the drift- and volatility-bursting coefficients: 
\begin{itemize}
\item Flash Crash. The price experiences a short-lived, V-shaped Flash Crash which lasts about $20$ minutes ($\Delta t = 0.025$) in the middle of the trading day, $\tau_{\text{b}} = 0.5$ \cite[as in][]{christensen2018drift}. Signals are sparse: the percentage of signals is about 6\% (20 out of 341 time intervals). Obviously, the strength of the signal is not constant. It is the strongest at the bottom of the ``V".  The drift burst rate $\alpha^\text{b} = 0.65$ with $a = 3$ and the volatility-burst rate $\beta^\text{b} = 0.4$ with $b = 0.15$. The parameter setting leads to a mild burst event.

%\item \textbf{Crash} The price experiences a one-directional crash which lasts for the whole day (i.e., $\Delta t = 1$ and $\tau_{\text{b}} = 1$). The percentage of signals is 100\% (all intervals are under the alternative). The drift burst rate $\alpha^\text{b} = 0.9$ and the volatility-burst rate $\beta^\text{b} = 0.3$. {\color{red} Shuping: what about the two constants?}

\item Persistent Expansion. The price experiences  an upward trend which lasts 3 days as in \cite{laurent2020drift} and \cite{laurent2020volatility,laurent2022unit}, starting at the beginning of day one with an upward trend peaking at the end of the third day. The percentage of signals is 100\% and the signal strength is increasing over the three days. We set $\alpha^\text{b} = 0.75$ with $a = 3$ and $\beta^\text{b} = 0.40$ with $b = 0.15$. 
% {\color{red} Shuping: what about the two constants? NB: They're the same as in the first setting.}
\end{itemize}
The log prices are contaminated by a noise term $\epsilon_{t_i} \sim	N(0, n^{-1/2}\gamma\sigma_{t_i})$,  	in which the noise-to-volatility ratio $\gamma$ is set to $0.5$. The noise term is conditionally heteroskedastic and serially dependent (through $\sigma_{t_i}$). We simulate data at the one-second frequency (assuming 6.5 trading hours each day), implying $23,401$ observations each day. 


\subsubsection{Simulation results}

We detect drift bursts on a one-minute grid. Each day is treated separately with a burn-in period of 49 minutes, leaving us with 341 drift burst tests per day. 
We treat these 341 drift burst tests on a day as a family and control the FWER rate at 5\%. 
% We measure the performance of the tests by computing the empirical FWERs under the null hypothesis and the global empirical power and the successful detection rates under the alternative hypothesis. 
The simulation is
repeated 1,000 times.

% 
Table \ref{tabSimsDriftBurst} reports the 
familywise error rate, global power, and successful detection rates
of different controlling procedures % for the drift burst test. 
under the given empirical setting.
We also report the average estimated first-order autocorrelation coefficient of the test statistics $\bar{\hat{\rho}}$ in the third column. There are moderate differences in $\bar{\hat{\rho}}$ under the null and the alternative hypotheses ($0.894$ under the null and higher under the alternative). The difference is larger when the signal lasts longer, e.g. $\bar{\hat{\rho}}=0.94 $ on the third day of the persistent expansion process.
	
	\begin{table}[!ht] % htb!
		\centering
		\caption{Finite sample performance of the controlling procedures for the drift burst test}
		\label{tabSimsDriftBurst}
		\begin{adjustbox}{max width=\textwidth}
			\begin{tabular}{l l | c | ccccccc}
				\hline
				\vspace{-.4cm} \\
				\multicolumn{2}{c}{DGP}
				&\multicolumn{1}{c}{$\bar{\hat{\rho}}$} 
				&\multicolumn{1}{c}{Bonferroni}
				&\multicolumn{1}{c}{Holm} 
				&\multicolumn{1}{c}{Hommel}  
				&\multicolumn{1}{c}{Hochberg} 
				& \multicolumn{1}{c}{Gumbel} 
				&\multicolumn{1}{c}{Resampling}
				&\multicolumn{1}{c}{SCC} \\
				\hline
				\multicolumn{10}{c}{Empirical FWER} \\ 
				\multicolumn{1}{l}{Stochastic volatility}
				& 
				& 0.89
				& 2.40 & 2.40 & 2.40 & 2.40
				& 1.90 & 4.60 & 4.80
				\\\hline
				\multicolumn{10}{c}{Global power} \\ 
				\multicolumn{1}{l}{Flash Crash} 
				% bandwidth: hn = 600			
				&
				& 0.90
				& 69.20 
				& 69.20 
				& 69.20 
				& 69.20 
				& 66.80 
				& 75.10 
				& 70.90 
				\\
				\multicolumn{1}{l}{Persistent Expansion}
				& \text{Day 1}
				& 0.89
				& 5.60
				& 5.60
				& 5.60
				& 5.60
				& 4.70
				& 10.20
				& 15.50
				\\
				& \text{Day 2}
				& 0.89
				& 12.90
				& 12.90
				& 12.90
				& 12.90
				& 10.40
				& 20.00
				& 30.00
				\\
				& \text{Day 3}
				& 0.94
				& 100.00
				& 100.00
				& 100.00
				& 100.00
				& 100.00
				& 100.00
				& 100.00
				\\
				\hline
				\multicolumn{10}{c}{Successful detection rates} \\ 
				\multicolumn{1}{l}{Flash Crash} 
				& 
				& 0.90
				& 6.48 
				& 6.49 
				& 6.49 
				& 6.49 
				& 6.13 
				& 7.64 
				& 6.93 
				\\ 
				%			\multicolumn{1}{l}{\textbf{3b:} Crash}
				%			& 
				%			& 0.940
				%			& 5.93
				%			& 6.02
				%			& 6.06
				%			& 6.02
				%			&  5.74
				%			&  7.08
				%			&  6.53
				%			& 9.28
				%			\\ 
				\multicolumn{1}{l}{Persistent Expansion} 
				& \text{Day 1}
				& 0.89
				& 0.04
				& 0.04
				& 0.04
				& 0.04
				& 0.03
				& 0.10
				& 0.35
				\\
				& \text{Day 2}
				& 0.89
				& 0.11
				& 0.12
				& 0.12 
				& 0.12
				& 0.09
				& 0.21
				& 1.30
				\\
				& \text{Day 3}
				& 0.94
				& 11.29
				& 11.67
				& 12.12
				& 11.67
				& 10.80
				& 13.84
				& 23.83
				\\
				\hline
				\hline
		\end{tabular}}
	\end{adjustbox}	
	\parbox{\textwidth}{\footnotesize%
		\vspace{.1cm} % If wanted space after the bottomrule
		{Note}: 
		We report the finite sample performance of the SCC procedure and its benchmarks for the drift burst test statistic. We report the FWERs (Panel 1), the global powers (Panel 2) and the successful detection rates (Panel 3). We consider different types of signals (Flash Crash and Persistent Expansion) and set the nominal level $\alpha$ to 5\%. The simulation is repeated $2,000$ times.
		}
\end{table}

The SCC test and the resampling approach perform well under the null hypothesis with a FWER of $4.8\%$ and $4.6$\%, respectively, for a nominal  level of 5\%. The inequality-based  approaches and the Gumbel method are much more conservative, with their empirical FWERs falling between $1.6\%$ and $2.4$\% for a nominal level of 5\%. Interestingly, the SCC test always has a higher power and successful detection rate than the inequality-based approaches and the Gumbel method. The global power of SCC and the successful detection rate are higher than those of the resampling approach when the signals are strong (i.e., Day 2 and Day 3 of Persistent Expansion). 
% For instance, in the last row, which reports the successful detection rate for the third day of the persistent expansion process, the SCC test detects about 24\% of intervals while the resampling method rejects 14\% of the cases. 
On the third day of the persistent expansion, the SCC test successfully detects about 24\% of intervals under the alternative hypotheses while the resampling method only detects 14\% of intervals.
In contrast, the resampling approach works better when the duration of the signal is relatively short, e.g., Flash Crash. However, as discussed earlier, resampling is much more computationally costly than SCC and it is exposed to the attenuation bias in the empirics.
	
Figure \ref{figLocalRejections} shows a typical sample path of the DGPs in the first column and the successful detection rates of the drift burst tests with the SCC procedure at each  time interval. As expected, the rejection rate is the highest when the signal strength is the strongest (i.e., at the bottom of the Flash Crash and at the end of the persistent expansion).\footnote{The drift burst test only detects the left-side of the V-pattern because the test is backward-looking by construction (see Appendix \ref{AppDBEstim}).}

\begin{figure}[!h] % htb!
\centering
\caption{A typical sample path of the DGP under the alternative and successful detection rates of SCC for the drift burst test}
\label{figLocalRejections}

\subfloat[log Price with Flash Crash]{{\includegraphics[width=.4\textwidth,angle = 0]{price_drift_a065b040} }}
\subfloat[Rejections with Flash Crash]{{\includegraphics[width=.4\textwidth,angle = 0]{db_local_drift_a065b040} }}	

%\par
%
%\subfloat[log Price with crash]{{\includegraphics[width=.4\textwidth,angle = 0]{price_crash} }}	
%\subfloat[Rejections with crash ]{{\includegraphics[width=.4\textwidth,angle = 0]{db_local_crash} }}	

\par

\subfloat[log Price with three-day rise]{{\includegraphics[width=.4\textwidth,angle = 0]{price_3drise} }}	
\subfloat[Rejections with three-day rise]{{\includegraphics[width=.4\textwidth,angle = 0]{test_3drise} }}	

\begin{minipage}{1.0\linewidth}
	\begin{tablenotes}
		\small
		\item {
			\medskip
			Note: 
			We plot a typical sample path of the DGPs in the first column and rejection frequencies of the drift burst tests with the SCC procedure for each interval in the second column. The grey color bar indicates the intervals under the alternative: 20 intervals (out of 390) for the Flash Crash and all intervals for the Persistent Expansion process. The first 49 intervals of each day are used as a burn-in period for the drift burst test.} 
	\end{tablenotes}
\end{minipage}
\end{figure}

\subsection{Empirics}
\label{ssecDBEmpirics}

For the empirical application, we apply 
%  the same seven versions of the drift burst test % not a different test.
the same controlling procedures on the drift burst test 
as in the previous section on the Nasdaq ETF (ticker: IXIC) and  the S\&P 500 ETF (ticker: SPY) for the period spanning from 1996 to 2020.  The data was collected from the Refinitiv Tick History Database at the one-second frequency and we use the rules in \citet{barndorff2009realized} to clean the raw data. We test for drift bursts minute-by-minute and control the familywise error rate at  $0.1$\%. 

Figure \ref{figEmpiricalPrices}  plots the weekly Nasdaq and S\&P 500 ETFs. The identified drift bursting weeks (with at least one rejections within the week) are indicated with grey bars. 
 The Nasdaq index bursts particularly often in the early 2000s after the collapse of the dot-com bubble. The S\&P 500 index barely has any rejections. The autocorrelation of the drift burst statistic are higher for the Nasdaq than for the S\&P 500 ($0.926$ vs. $0.861$). This is consistent with our observations in Table \ref{tabSimsDriftBurst} that the estimated autocorrelation coefficient is higher for processes with longer-lasting signals (i.e., more observations under the alternative hypothesis).

\begin{figure}[!h] 
\caption{Drift bursts in the Nasdaq and S\&P 500 ETFs from 1996 to 2020}\label{figEmpiricalPrices}\centering	
	\subfloat[Nasdaq]{{\includegraphics[width=.38\textwidth,angle = -90]{weeklyPrices_Nasdaq2} }}
	\subfloat[S\&P 500]{{\includegraphics[width=.38\textwidth,angle = -90]{weeklyPrices_SPY2} }}	\\
	\begin{minipage}{1.0\linewidth}
	\begin{tablenotes}
	\small
	\item {
		\medskip	
		Note: The black lines are the weekly index ETF prices. The grey bars indicate weeks with at least one rejection of the `no drift burst" null hypothesis within the day. 
		The darker the grayscale the more drift burst days we observe. 
		We use the drift burst test of \cite{christensen2018drift} and the SCC procedure to control for the FWER at the $0.1\%$ significance level.
		}
	\end{tablenotes}
	\end{minipage}	
\end{figure}

Table \ref{tabEmpiricalR} reports the rejection frequencies of the drift burst test with the seven FWER controlling procedures. Compared to the benchmark procedures, the SCC test detects more drift burst days (i.e., at least one rejection within the day) and intervals (i.e., total number of rejections) in the Nasdaq index. For the S\&P 500 index ETF, the SCC procedure detects more drift burst days and intervals than the inequality-based and Gumbel methods but slightly less than the resampling procedure. The bursting episodes in the S\&P 500 are relatively short-lived which is in line with the Flash Crash DGP considered in our simulations, while the Nasdaq index has persistent drifts as shown in \citet{laurent2022unit}, resembling the persistent expansion DGP. 
Our simulations show that the SCC test has higher power in detecting persistent drifts, whereas the resampling procedure performs slightly better in detecting short flash crashes. 

\begin{table}[H]
			\caption{Rejection frequencies of the null of the drift burst hypothesis for the Nasdaq and S\&P 500 index ETFs}\label{tabEmpiricalR}
	\scalebox{0.95}[0.95]{
	\begin{tabular}{lccccccc}
		\hline
		\multicolumn{1}{l}{} 
		&\multicolumn{1}{l}{Bonferroni} 
		&\multicolumn{1}{l}{Holm}
		&\multicolumn{1}{l}{Hommel}  
		&\multicolumn{1}{l}{Hochberg}  
		&\multicolumn{1}{l}{Gumbel} 
		&\multicolumn{1}{c}{Resampling} 
		&\multicolumn{1}{l}{SCC} 
		\\
		\hline
		% hn = 600
				\multicolumn{8}{c}{Nasdaq}  \\
		Days (in\%)   
		& 18.840
		& 18.840
		& 18.840 
		& 18.840
		& 11.662
		& 23.357
		& 23.522 
		\\
		% 
		Intervals (in\#) 
		&   6,479 
		&   6,540 
		&   6,559
		&   6,540
		&   3,823
		&   8,633
		& 10,944
		%
		\\
		\multicolumn{8}{c}{S\&P 500}  \\	
		Days (in\%)   
		& 0.526
		& 0.526
		& 0.526
		& 0.526
		& 0.247
		& 0.641
		& 0.576
		\\
		
		Intervals (in\#) 
		& 41
		& 41
		& 41
		& 41
		& 18
		& 55
		& 50
		\\	

		% replace by hn = 300
		\hline
		\hline	
\end{tabular}}
\begin{minipage}{1.0\linewidth}
\begin{tablenotes}
\small
\item {
	\medskip
	Note: We report the empirical rejections of the drift burst test with various controlling procedures.  The drift burst days are percentages of days over the full sample period with at least one rejection. 
	% The intervals are the total number of rejections over the entire sample period. 
	The figures in rows labelled intervals (in\#) are the total number of rejections over the entire sample period.
}
\end{tablenotes}
\end{minipage}
\end{table}



\section{Example 2: In Search of Nonzero Alpha Assets}

\label{secApplFan}

%  the residual, the part of the expected return not explained by betas. 
The efficient market hypothesis postulates that asset prices incorporate all
available information at all times, hence, there is no way of systematically
beating the market on a risk-adjusted basis. %Because the efficient market hypothesis is postulated on a risk-adjusted basis, it requires a risk model to be tested. The most obvious one is the Capital Asset Pricing Model is a pricing model. 
In response to mounting empirical evidence of patterns in stock
returns (the so-called ``anomalies"), many new risk factors have been
introduced to explain average returns 
\citep[see e.g.,][for a historical perspective]{fama1996multifactor}. These risk factors are said to represent some
dimension of undiversifiable systematic risk which should be compensated
with higher returns. If the factor model fully characterizes expected
returns, the regression intercept (a.k.a, the ``alpha") should be equal to zero. 
 
 We search for
nonzero alpha assets using the \citet{fama2015five} five-factor model framework, in which
returns are characterized by a set of common factors (i.e., market, size, value, profitability and investment). Correctly identifying nonzero alpha assets is a needle-in-a-haystack
problem. There is a general consensus in the empirical literature that markets are efficient and mispriced assets are rare \citep[see e.g.,][]{fan2015power,giglio2021thousands}. The standard approach is to run time series regressions, asset-by-asset, and perform individual tests on the alphas. The number of assets that need to be tested simultaneously is large (e.g., all of the
S\&P 500 stocks) and the test statistics of the
cross-sectional alphas are most likely correlated due to the presence of unknown common factors 
\citep[see e.g.,][]{giglio2021thousands}. 
The key to success is to use a proper  % effective
controlling procedure that achieves the desired
FWER\footnote{An alternative target is the false discovery rate defined as the proportion of false discoveries (see e.g.,\citeauthor{barras2010false}, \citeyear{barras2010false} and \citeauthor{giglio2021thousands}, \citeyear{giglio2021thousands} for two examples).} 
control and is powerful enough to detect rare nonzero alpha assets.

Section \ref{sec_HS} introduces the nonzero alpha hypothesis and the test.
Section \ref{ssecScreening} describes a screening approach used in  \citet{fan2015power} to detect individual violations. 
Section \ref{ssecSim} presents simulation results comparing the finite
sample performance of the controlling procedures in the context of the
nonzero alpha test. Again, instead of simulating the test statistics
directly, we generate returns from a factor model and compute the test
statistics for a cross-section of financial assets. In Section \ref{sec_KF}, we
examine Fama-French portfolios formed on bivariate sorts and search for portfolios with a nonzero alpha.

\subsection{Nonzero Alpha Hypothesis  and Test}
\label{sec_HS}

The multi-factor pricing model, motivated by the Arbitrage Pricing Theory %
\citep{ross1976arbitrage}, postulates how financial returns are related to
market risks. It has enjoyed widespread application in asset pricing and
portfolio management. 
% \red{Returns covary with the factors through time.} 
% 
Let $y_{it}$ be the excess return (i.e., real rate of return minus the
risk-free rate) of the $i$th financial asset at day $t$ and consider the
following linear regression model: % \red{time series regression}
\begin{eqnarray}  \label{eqCrossDecomp}
y_{it} = a_i + \bm{b}^{\prime }_i \bm{f}_t + u_{it}, \quad \text{with } i =
1, \ldots, d, \, t = 1, \ldots, T,
\end{eqnarray}
in which $a_i$ is an intercept (a.k.a the ``alpha"), $\bm{b}_i = (b_{i1},
\ldots, b_{iK})^{\prime }$ is a vector of factor sensitivities or loadings
(a.k.a the ``beta"), $\bm{f}_t = (f_{1t}, \ldots, f_{Kt})^{\prime }$ are
observable factors, and $u_{it}$ is an idiosyncratic error which is
uncorrelated with the factors. The number of factors $K$ is fixed. 
% The number of financial assets $d$ can be larger than the sample size $T$. 
An example of \eqref{eqCrossDecomp} is the three-factor model of \citet{fama1992cross}, 
% with $K = 3$, 
which captures much of the variation in the cross-section of average returns and 
absorbs a lot of the anomalies that have plagued the CAPM 
\citep[see
also][]{fama1996multifactor}.

Our objective is to identify individual assets with a nonzero alpha. The
null hypothesis of each asset is therefore $H_i: a_i=0$ (`there is no
mispricing of asset $i$') and the alternative hypothesis is $a_i\neq 0$
(`asset $i$ is mispriced'), for $i=1,\ldots, d$. The most common way to test
this null hypothesis is to use a simple $t$-statistic for $\alpha_i$, i.e.: 
\begin{eqnarray}  \label{eqTestAlpha}
{X}_i = \frac{\hat{a}_i}{ \hat{\sigma}_{\hat{a}_i}},
\end{eqnarray}
in which, for each asset $i$, $\hat{a}_i$ is the estimated alpha and $\hat{%
\sigma}_{\hat{a}_i}$ is the estimated standard error.
% 
Under the null hypothesis, the test statistic \eqref{eqTestAlpha} follows a
Student-$t$ distribution, i.e., $X_i \sim t(\nu)$, with $\nu$ being the
degrees of freedom. A viable detection strategy is to compute the test
statistic \eqref{eqTestAlpha} for each asset in the cross-section and reject
the null hypothesis when $\abs{X_i}$ is higher than a chosen quantile of the 
$t(\nu)$ distribution.


The multiplicility issue emerges when the number of assets $d$ is large. To
identify the individual violations and control the familywise error rate, one can again use the
inequality-based, Gumbel and SCC approaches. 
The cross-sectional test statistics are  likely to be
correlated \citep[see e.g.,][]{giglio2021thousands}
 and hence the Gumbel method is again expected to be conservative. 
Admittedly, 
a Student-$t$ distribution does
not exactly fulfill the assumptions of the Cauchy combination test, but
through simulations, \citet{liu2020cauchy} show that the Cauchy approximation
is still accurate when the test statistic follows a multivariate multivariate Student-$t$
distribution. 

% Another benchmark for identifying individual violations is the screening test proposed in \cite{fan2015power},  for which the individual violations are treated as a by-product of a power enhancement global test. 

\subsection{Screening}
\label{ssecScreening}

An important benchmark in testing factor pricing models is the power enhanced global test proposed in \cite{fan2015power}, which identifies individual violations as a byproduct. 
The global test examines whether there is \textit{any} asset with nonzero alpha. The global null hypothesis is therefore that the alphas of the $d$ financial
assets are jointly indistinguishable from zero \citep[see e.g.,][]{fama1996multifactor}: 
\begin{eqnarray*} \mathcal{H}_{0}:\bm{a}%
=\bm{0}, \text{with } \bm{a}=(a_{1},\ldots ,a_{d})^{\prime }.
\end{eqnarray*}

Classical tests 
\citep[see \textit{e.g.},][and references
therein]{fan2015power} for the global null hypothesis are based on a
quadratic statistic of the form % such as the Wald statistic:
$\bm{\widehat{a}}^{\prime }\bm{V}\bm{\widehat{a}}$, in which $\hat{\bm{a}}$
is an element-wise consistent estimator (\textit{e.g.}, the OLS estimator)
for the vector of intercepts $\bm{a}$ and $\bm{V}$ is a high-dimensional
positive definite weight matrix, often taken to be the inverse of the
asymptotic covariance matrix of $\bm{\widehat{a}}$ (such as for the Wald
test). Some problems can arise for high-dimensional testing problems when
using a quadratic statistic.  In particular, they typically have low power under sparse alternatives,
because the quadratic statistic accumulates estimation errors under the null hypothesis, 
which results in large critical values that can dominate the signals in the sparse alternatives 
\citep[see][for further elaboration]{fan2015power}.

%\footnote{Some problems can arise for high-dimensional testing problems when
%using a quadratic statistic. 
%For example, a quadratic test requires making assumptions on the joint distribution of the intercepts of the $d$ regressions. 
%When $d>T$, estimating the
%covariance matrix $\bm{V}$ is challenging, as the sample analogue of the
%covariance matrix is singular. Another problem is that tests based on the quadratic form have typically low power under sparse alternatives,
%because the quadratic statistic accumulates % high-dimensional 
%estimation
%errors under the null hypothesis, 
%which results in large critical values that can dominate the signals in the sparse alternatives 
%\citep[see][for further elaboration]{fan2015power}.} 

The idea behind \citet{fan2015power}'s power enhancement is to use the following test statistic: 
\begin{equation}
	\label{eqJ}
J=J_{1}+J_{0},
\end{equation}%
where $J_{1}$ is the classical test statistic with a correct asymptotic size (e.g., Wald) but  may suffer from low power under sparse alternatives, and $J_{0}$ 
% is the power enhancement component, which 
enhances the power with little size distortion.  An example of such a $J_0$ is a screening statistic: 
\begin{equation*}
J_{0}=\sqrt{d}\sum_{j\in \widehat{S}}
\frac{\widehat{a}_{j}^{2}}{\widehat{\nu }%
_{j}},
\, 
\text{with }\widehat{S}=\left\{ j:\frac{\abs{\widehat{a}_j}}{%
\widehat{\nu }_{j}^{1/2}}>\delta _{d,T},j=1,\ldots ,d\right\},
\end{equation*}%
in which $\widehat{\nu }_{j}$ is a data-dependent normalizing threshold
taken by \citet{fan2015power} as the estimated asymptotic variance of $%
\widehat{a}_{j}$. 
The screening set $\widehat{S}$ is
constructed from the standardized  alpha with a \textquotedblleft high
criticism" threshold $\delta _{d,T}$.
Since $\max_{j\leq d}\abs{\widehat{a}_j}%
\widehat{\nu }_{j}^{-1/2}=O_p(\sqrt{\log d})$, the threshold is chosen  to dominate the maximum
noise level and set to $\delta _{d,T}=C\log (\log (T))\sqrt{\log (d)}$.\footnote{We set the constant $C$ to $1.06$ as in the code shared by \cite{fan2015power} replicating their simulations.} 

The power enhancement component $J_0$ is asymptotically zero under the null hypothesis, but diverges quickly under sparse alternatives. By construction, the method imposes a 0\% theoretical FWER, as the screening set $\widehat{S}$ is asymptotic empty under the null. Under the alternative, the screening test is expected to have power
(and hence $J_{0}$ can enhance the power of the global test) in the following region: 
\begin{equation*}
\max_{j\leq d}\frac{\abs{a_j}}{v_{j}^{1/2}}>3\delta _{d,T}, \\
% \max_{j \leq d}
% \frac{T^{1/2} \abs{a_j}}{\text{var}^{1/2} (u_{jt})} > 3 a_f^{-1/2} \delta_{d,T}, 
% Is this correct? 
\end{equation*}%
% n which $a_f$ can be estimated by $1 - \bm{\bar{f}}' \bm{w}$, with $\bm{\bar{f}} = \frac{1}{T} \sum^{T}_{t = 1} \bm{f}_t$. 
where $v_{j}$ is the asymptotic variance of $\widehat{a}_{j}$. 

The power enhancement component in \eqref{eqJ} does not serve as a test statistic on its own, but is meant to be added to a classical global statistic. It is however the only part of the test statistic which identifies individual violations and will serve as a benchmark in the next sections. % as a byproduct. 
% which is the aim of this paper.  
% The screening set $\widehat{S}$ does capture however the indices where the null hypothesis is violated as a byproduct. 
% "Unfortunately, the global test $J$ does not allow the identification of individual violations. 
% Since our aim is to identify individual violations only the part relating to the screening test is relevant. 
	% We compare the finite sample performance of the various controlling procedures in the next subsection.

\subsection{Simulations}
\label{ssecSim} 

We compare the finite sample performance of the SCC
procedure with existing controlling procedures (including the four inequality-based, Gumbel and screening)
for identifying nonzero alpha assets.  The resampling approach
considered for the drift burst test  is not suitable for the current cross-sectional context
and  is therefore not included here.

\subsubsection{Data-generating processes}

We generate excess returns $y_{it}$ from the \citet{fama2015five}
five-factor model \eqref{eqCrossDecomp} % (with $K=5$) 
for $d=100$ assets and $%
T=240$ time series observations. The loadings and factors are generated from independent
multivariate distributions, $\mathcal{N}_5 (\bm{\mu_f}, \bm{\Sigma_f})$ and $%
\mathcal{N}_5 (\bm{\mu_B}, \bm{\Sigma_B})$, respectively. The data-generating process % and parameter settings 
is inspired by 
\citet[][Section 6.1]{fan2015power} and 
\citet[][Section 4.2]{shi2022relax}. We calibrate the parameters to the Fama-French 100 portfolios, bivariately sorted by size and book-to-market  (size-BM), size and investment (Size-INV), and size and operating profitability (Size-OP), for the period spanning from % 2001 to 2020. 
January 1998 to December 2017.\footnote{The choice of the sample period is guided by Figure \ref{figFMRejections}, where we observe substantial deviations between the testing results of SCC and the benchmark procedures over this period.} Table \ref{tabPars} reports the mean ($\bm{\mu_f}$) and covariance matrix ($\bm{\Sigma_f}$) of the five factors $\bm{f}_t$ on the right and the mean ($\bm{\mu_B}$) and covariance matrix ($%
\bm{\Sigma_B}$) of $\bm{b}_i$ for the SIZE-BM portfolios on the left. The parameters for the other two portfolios, i.e., Size-INV and Size-OP, are available upon request.
%\begin{eqnarray}
%\bm{\mu_B} &= 
%\begin{bmatrix}
%1.029 \\ 
%0.578 \\ 
%0.201 \\ 
%0.047 \\ 
%0.050 \\ 
%\end{bmatrix}
%\text{ and } \bm{\Sigma_B} &= 
%\begin{bmatrix}
%\phantom{-}0.015 & -0.016 & \phantom{-}0.014 & \phantom{-}0.018 & \phantom{-}%
%0.000 \\ 
%-0.016 & \phantom{-}0.178 & -0.028 & -0.067 & -0.028 \\ 
%\phantom{-}0.014 & -0.028 & \phantom{-}0.131 & \phantom{-}0.056 & \phantom{-}%
%0.011 \\ 
%\phantom{-}0.018 & -0.067 & \phantom{-}0.056 & \phantom{-}0.097 & \phantom{-}%
%0.024 \\ 
%\phantom{-}0.000 & -0.028 & \phantom{-}0.011 & \phantom{-}0.024 & \phantom{-}%
%0.047%
%\end{bmatrix}%
%.
%\end{eqnarray}
%% We refer to Table 2 in \citet{shi2022relax} for 
%
% are set respectively to 
%\begin{eqnarray}
%\bm{\mu_f} &= 
%\begin{bmatrix}
%0.552 \\ 
%0.255 \\ 
%0.145 \\ 
%0.315 \\ 
%0.245 \\ 
%\end{bmatrix}
%\text{ and } \bm{\Sigma_f} &= 
%\begin{bmatrix}
%\phantom{-}19.942 & \phantom{-}3.472 & -1.742 & -6.725 & -3.185 \\ 
%\phantom{-}3.472 & \phantom{-}10.183 & -0.613 & -4.774 & \phantom{-}0.198 \\ 
%-1.742 & -0.613 & \phantom{-}10.265 & \phantom{-}4.341 & \phantom{-}4.317 \\ 
%-6.725 & -4.774 & \phantom{-}4.341 & \phantom{-}9.312 & \phantom{-}1.930 \\ 
%-3.185 & \phantom{-}0.198 & \phantom{-}4.317 & \phantom{-}1.930 & \phantom{-}%
%4.641%
%\end{bmatrix}%
%.
%\end{eqnarray}
%The parameters are chosen to simulate factor loadings and explanatory variables mimicking the
%market, size,  value, profitability, and investment factors.

\begin{table}[H]
	\caption{Means and covariances for generating factor loadings $\bm{b}_i$ for the Size-BM portfolios and factors $\bm{f}_t$}\label{tabPars}
	\scalebox{0.85}[0.85]{
		\begin{tabular}{r | rrrrr | r  | rrrrr}
			\hline
			\multicolumn{1}{c}{$\bm{\mu_B$}} 
			& \multicolumn{5}{c|}{$\bm{\Sigma_B}$} 
			& \multicolumn{1}{c}{$\bm{\mu_f}$} 
			& \multicolumn{5}{c}{$\bm{\Sigma_f}$} 
			\\
			\hline
			$1.029$ 
			%
			& $\phantom{-}0.015$ & $-0.016$ & $\phantom{-}0.014$ & $\phantom{-}0.018$ & \phantom{-}%
			$0.000$ 
			%
			& $0.552$
			& $\phantom{-}19.942$ & $\phantom{-}3.472$ & $-1.742$ & $-6.725$ & $-3.185$
			\\ 
			$0.578$ 
			&	$-0.016$ & $\phantom{-}0.178$ & $-0.028$ & $-0.067$ & $-0.028$ 
			& $0.255$
			& $\phantom{-}3.472$ & $\phantom{-}10.183$ & $-0.613$ & $-4.774$ & $\phantom{-}0.198$ \\ 
			$0.201$ 
			&	$\phantom{-}0.014$ & $-0.028$ & $\phantom{-}0.131$ & $\phantom{-}0.056$ & $\phantom{-}0.011$ 
			%
			& $0.145$
			& $-1.742$ & $-0.613$ & $\phantom{-}10.265$ & $\phantom{-}4.341$ & $\phantom{-}4.317$
			\\ 
			$0.047$ 
			& $\phantom{-}0.018$ & $-0.067$ & $\phantom{-}0.056$ & $\phantom{-}0.097$ & $\phantom{-}$%
			$0.024$  
			& $0.315$
			& 	$-6.725$ & $-4.774$ & $\phantom{-}4.341$ & $\phantom{-}9.312$ & $\phantom{-}1.930$ 
			\\ 
			$0.050$
			& $\phantom{-}0.000$ & $-0.028$ & $\phantom{-}0.011$ & $\phantom{-}0.024$ & $\phantom{-}0.047$
			%
			& $0.245$
			& 	$-3.185$ & $\phantom{-}0.198$ & $\phantom{-}4.317$ & $\phantom{-}1.930$ & $\phantom{-}4.641$\\ 
			\hline
			\hline	
	\end{tabular}}
%		\begin{minipage}{1.0\linewidth}
%			\begin{tablenotes}
%				\small
%				\item {
%					\medskip
%					Note: The means and covariances to generate factor loadings $\bm{b}_i$ for the Size-BM portfolios and factor returns $\bm{f}_t$ within the \citet{fama2015five} five-factor model.  
%						
%				}
%			\end{tablenotes}
%		\end{minipage}
\end{table}

The idiosyncratic noises are generated from a multivariate normal distribution, i.e., $%
\mathcal{N}_d (0, \bm{\Sigma_u})$. 
To generate cross-sectional dependence in the idiosyncratic noise similar to the one in the empirical application, the covariance matrix $\bm{\Sigma_u}$ is
computed as the sample covariance matrix of the residuals from the OLS
estimation of the five-factor model.\footnote{The coefficients are available upon request.
We thank the authors of \citet{shi2022relax} for sharing their calibration code.
} The alphas are zero ($a_i=0$) for all assets $i=1,\ldots,d$ under the null hypothesis, and under the
alternative, we consider % either a sparse but strong signal or 
many, weak signals: 
\begin{eqnarray}\label{eq:ai}
a_{i}=\left\{ 
\begin{array}{cc}
0.5 & \text{if }i\leq d^{0.4} \\ 
0 & \text{if }i>d^{0.4}
\end{array}
,\right.
\end{eqnarray}
matching the situation in our empirical application in Section \ref{sec_KF}. 



\subsubsection{Simulation results}

We treat all 100 portfolios in each subgroup as a family and control the familywise error rate at  5\%. 
The performance of the tests are measured by the empirical FWERs under
the null hypothesis and 
the global
empirical power and the successful detection rates 
under the alternative hypothesis. 
The simulation is
repeated 2,000 times.


Table \ref{tabCrossSize} presents the FWERs of the controlling procedures for the null of the nonzero alpha test in the presence of strong cross-sectional dependence. We also report the minimum and maximum off-diagonal element of the empirical correlation matrix of the test statistics across all replications. 
All procedures have reasonably good control of the familywise error rate. 
The FWERs of the inequality-based,  Gumbel and screening methods are below the nominal level of 5\%, with the Gumbel and screening  methods being the most conservative ones. 
% Note that the screening test has an empirical rejection frequency higher (i.e., 3\%) than what is expected asymptotically (i.e., 0\%) in the presence of strong cross-sectional dependence. 
Note that, in the presence of strong cross-sectional dependence, the screening test is found to have an empirical rejection frequency higher (i.e., 3\%) than what is expected asymptotically (i.e., 0\%). 
% As expected, the FWERs of the SCC procedure are  very close to upper bound imposed on the familywise error rate. 
Interestingly, the FWERs of the SCC procedure are very close to the nominal level of 5\% in all three cases. 

\begin{table}[!ht] 
	\centering
	\caption{FWERs of the nonzero alphas test with different controlling procedures under the null hypothesis}
	\label{tabCrossSize}
	\begin{adjustbox}{max width=\textwidth}
		\begin{tabular}{c | cc | cccc cccccc}
			\hline
			& \multicolumn{1}{c}{$\widehat{\rho} (\min)$}
			& \multicolumn{1}{c|}{$\widehat{\rho} (\max)$}
			&\multicolumn{1}{c}{Bonferroni}
			&\multicolumn{1}{c}{Holm} 
			&\multicolumn{1}{c}{Hommel}  
			&\multicolumn{1}{c}{Hochberg} 
			&\multicolumn{1}{c}{Gumbel} 
			&\multicolumn{1}{c}{Screening}
			&\multicolumn{1}{c}{SCC}	\\
			\\
			\hline
			Size-BM 
			& --0.41  &   0.71
			& 4.00    &  4.00   &   4.00   &   4.00     & 3.70   &   
			3.00 & 
			5.10
			\\
			Size-INV 
			&  --0.44    &    0.60
			& 4.65   &   4.65    &  4.65   &   4.65     & 4.15      &
			3.40 &
			5.70
			\\ 
			Size-OP 
			&	--0.37      &   0.60
			& 4.30  &    4.30     & 4.30   &   4.30    &  3.80      &
			3.45 &
			5.20
			\\
			\hline
	\end{tabular}}
\end{adjustbox}	
\parbox{1\textwidth}{\footnotesize%
	\vspace{.1cm} % If wanted space after the bottomrule
	{Note}: The data generating process is \eqref{eqCrossDecomp} with $a_{i}=0$ for all $i=1,...,100$. 
	The sample size is $T=240$ and the number of assets is $d=100$. 
	The nominal level $\alpha$ of the tests is 5\%. The simulation is repeated 2,000 times. }
\end{table}

Table \ref{tabCrosspower} presents the global power and the successful detection rates for the alternative of the nonzero alpha test 
in the presence of strong cross-sectional dependence and 
 weak signals. 
 %  The SCC test outperforms other procedures with much higher global power and successful detection rates. 
 The SCC test outperforms all the other procedures, i.e., it has a much higher global power and more successful detections.
The power and successful detection rates of the Gumbel and screening methods are the lowest. 
The results are consistent across all three different parameter settings calibrated to the Size-BM, Size-INV and Size-OP portfolios, respectively.

\begin{table}[!ht] 
\centering
\caption{Global powers and successful detection rates (in\%) of the nonzero alpha test with different controlling procedures}
\label{tabCrosspower}
\begin{adjustbox}{max width=\textwidth}
	\begin{tabular}{c | cc | cccc cccccc}
		\hline
		& \multicolumn{1}{c}{$\widehat{\rho} (\min)$}
		& \multicolumn{1}{c|}{$\widehat{\rho} (\max)$}
		&\multicolumn{1}{c}{Bonferroni}
		&\multicolumn{1}{c}{Holm} 
		&\multicolumn{1}{c}{Hommel}  
		&\multicolumn{1}{c}{Hochberg} 
		&\multicolumn{1}{c}{Gumbel} 
		&\multicolumn{1}{c}{Screening}
		&\multicolumn{1}{c}{SCC}	\\
		\hline
		\multicolumn{10}{c}{Global power}\\
		% paper is getting a bit too long, so we removed the sparse strong signals
		%			\multicolumn{10}{c}{Global power: Sparse  signals}\\
		%			Size-BM & 
		%			--0.48      &      0.56
		%			& 90.50  &   90.50  &   90.50  &   90.50  &   89.70 &      
		%			89.15
		%			& 90.65
		%			\\
		%			Size-INV & 
		%			-0.23     &     0.80
		%			& 56.80   &  56.80  &   56.80   &  56.80   &  54.65     
		%			& 52.75
		%			& 56.75
		%			\\
		%			Size-OP & 
		%			-0.27      &      0.73
		%			& 70.10   &  70.10  &   70.10  &   70.10  &   68.60  &  67.00 & 70.40
		%			\\
		% \multicolumn{10}{c}{\red{Update}} \\
		%			Size-BM 
		%			& -0.42      &      0.68
		%			& 85.00   &  85.00  &   85.00   &  85.00  &   83.90   &   82.90 & 85.70
		%			\\
		%			Size-INV 
		%			& --0.45      &     0.60
		%			& 95.10  &   95.10   &  95.10    & 95.10  &   94.70  & 93.95 &  95.30
		%			\\
		%			Size-OP 
		%			& --0.37      &      0.61 
		%			& 99.30  &   99.30   &  99.30    & 99.30  &   99.30  &   99.10 & 99.40
		%			\\
		% \multicolumn{10}{c}{Global power: Weak  signals}\\
		%			Size-BM & 
		%			-0.52      &      0.56
		%			& 54.25   &  54.25  &   54.25   &  54.25 &  52.05     
		%			& 49.95
		%			& 59.00
		%			\\
		%			Size-INV & 
		%			-0.25     &     0.83
		%			& 46.75  &   46.75 &    46.85  &   46.75   &  44.60   &  42.35 & 52.00
		%			\\
		%			Size-OP &
		%			-0.27      &     0.74
		%			& 65.50  &   65.50  &   65.60   &  65.50   & 63.95     
		%			& 61.55
		%			& 70.75 \\
		%			\multicolumn{10}{c}{\red{Update}} 
		%			 \\			
		Size-BM 
		&  --0.41      &      0.71
		& 54.15   &  54.20   &  54.25   &  54.20  &   51.55    & 
		49.15 &
		58.75
		\\
		Size-INV 
		& --0.43      &     0.60
	& 	72.10    & 72.10  &   72.10  &   72.10  &   70.10    
		& 67.80
		 & 76.40
		\\
		Size-OP 
		& --0.36      &      0.59
		& 83.40   &  83.40  &   83.40  &   83.40 &     81.75   & 80.25 & 88.05
		\\
		
		\multicolumn{10}{c}{Successful detection rates}\\
		%\multicolumn{10}{c}{SDR: Sparse signals}\\
		%Size-BM 
		%& -0.48      &      0.56
		%& 89.90 & 89.90 & 89.90 & 89.90  & 89.10 & 88.65 & 89.90\\ 
		%Size-INV & 
		%-0.23      &      0.80
		%& 54.60 & 54.60 &  54.60 & 54.60 & 52.65 & 50.75 & 53.70
		%\\
		%Size-OP & 
		%-0.27      &      0.73
		%& 68.50 & 68.50 & 68.50 & 68.50 & 67.35 & 65.70 & 68.15
		%\\
		%\multicolumn{10}{c}{\red{Update}} \\
		%Size-BM 
		%& --0.43      &    0.67
		%&
		%84.25  & 84.25 & 84.25 & 84.25 & 83.20 & 82.25 & 84.30 \\
		%Size--INV 
		%& --0.45      &      0.60
		%& 94.90 &    94.95   &  94.95  &   94.95    & 94.55 &  93.75  & 94.95
		%\\
		%	Size-OP 
		%& --0.37      &      0.61 
		% & 99.25 & 99.25 & 99.25 & 99.25 & 99.25  & 99.05 & 99.35 
		%\\
		%\multicolumn{10}{c}{SDR: Weak  signals}\\
		%Size-BM 
		%& -0.52      &      0.56
		%& 14.18 & 14.23 & 14.24 &14.23& 13.43 &12.73 & 16.06
		%\\
		%Size-INV & 
		%-0.25      &     0.83
		%& 15.78 & 15.88 & 15.92 & 15.88 & 14.86 & 14.09 & 18.38
		%\\
		%Size-OP & 
		%-0.27      &     0.74
		%& 27.08 & 27.25 & 27.30 & 27.25 & 25.94 & 24.83 & 30.35
		%\\		
		%\multicolumn{10}{c}{\red{Update}} 
		%\\
		Size-BM  
		&  --0.41      &      0.71
		& 15.49 & 15.56 & 15.59 & 15.57 & 14.56 & 13.62 & 17.67
		\\
		Size-INV 
		& --0.43      &     0.60
		& 	27.80    & 27.92  &   27.99 & 27.92 & 26.45 & 24.73 & 30.88
		\\
		Size-OP 
		& --0.36      &      0.59
		& 41.45   &  41.63  &   41.65  &   41.63 &     39.62   & 37.88 & 45.72
		\\
		% \multicolumn{10}{c}{\red{\textit{Sanity check}}}\\
		%\red{$T = 300, d = 500$}
		%& -0.10    &   0.11
		%& 7.24   &  7.25  &  7.25   &  7.25 &    7.20 &    
		%1.90
		%&			 9.12
		%\\
		%\red{$T = 300, d = 800$} & 
		%-0.10 & 0.10 
		%& 6.89   &   6.89  &    6.89  &    6.89  &    6.97  &    1.56
		%& 8.50
		%\\
		\hline
\end{tabular}}
\end{adjustbox}	
\parbox{1\textwidth}{\footnotesize%
\vspace{.1cm} % If wanted space after the bottomrule
{Note}: 
The data generating process is \eqref{eqCrossDecomp} with $a_i$ specified in \eqref{eq:ai}. The sample size is $T=240$ and the number of assets is $d=100$. The nominal level $\alpha$ is 5\%. The simulation is repeated 2,000 times. 
}
\end{table}




\subsection{Empirics: Kenneth French Portfolios} \label{sec_KF}

In the empirical application, we study the portfolios 
available in
Kenneth French's Data Library.\footnote{\url{https://mba.tuck.dartmouth.edu/pages/faculty/ken.french/data_library.html}} 
We focus on the  $d = 100$  portfolios formed  bivariately ($10\times10$) on Size-BM, Size-INV, and Size-OP. 
The left-hand side of  \eqref{eqCrossDecomp} consists of value-weighted portfolio excess returns and the right-hand side are the five factors in the \citet{fama2015five} five-factor model. 
The sample period is from July 1963 to November 2022 and consists of $713$ monthly observations. 
% We choose the portfolios which are value-weighted. 
We conduct the nonzero alpha test using a rolling window procedure with a window size of $T = 240$ observations. 

Table \ref{tabCrossFFK} reports the average rejection frequencies of zero alpha null (across a  rolling analysis) using the same controlling procedures as in the previous subsection. 
As expected, there are few violations. For example for the Size-BM portfolios, the average rejection frequency is between  1.09\% and 2.60\%. 
% meaning that out of a total of 100  portfolios, 1 or 2 are inconsistent with the model
The rejection frequency is always higher for the SCC test, 
%  higher in the Size-INV portfolios among the three types of sorted portfolios. Importantly, the SCC procedure consistently rejects more than other controlling procedures. Take the Size-BM 100 portfolio as an example. The rejection number of SCC is the highest ($2.60\%$), 
followed by the inequality-based methods, 
% screening procedure (about 1.25\%) and 
the screening procedure, and the Gumbel method, in this order.
% The Gumbel and the screening procedures are the two most % conservative ones. % rejecting only $12\% - 13\%$. 
The empirical results are consistent with our simulations and previous research on mispricing: nonzero alpha assets are rare \citep[see \textit{e.g.},][]{fama1996multifactor,fan2015power,giglio2021thousands}, but the SCC testing procedure can detect more of these rare violations. 

\begin{table}[!ht] 
	\centering
	\caption{Average empirical rejection frequencies (in\%) of zero alpha null across a rolling analysis of bivariate-sorted portfolios}
	\label{tabCrossFFK}
	\begin{adjustbox}{max width=\textwidth}
		\begin{tabular}{c | ccccccccc}
			\hline
			&\multicolumn{1}{c}{Bonferroni}
			&\multicolumn{1}{c}{Holm} 
			&\multicolumn{1}{c}{Hommel}  
			&\multicolumn{1}{c}{Hochberg} 
			&\multicolumn{1}{c}{Gumbel}
			&\multicolumn{1}{c}{Screening}
			&\multicolumn{1}{c}{SCC} 
			\\
			\hline
			% \multicolumn{8}{c}{Average rejections (in\%)}  \\
			Size-BM  
			& 1.25 & 1.26 & 1.27 & 1.26 & 1.09 & 1.09 & 2.60 \\
			Size-OP 
			& 0.53 & 0.53 & 0.53 & 0.53 & 0.51 & 0.61 & 0.75
			\\
			Size-INV  & 1.19 & 1.19 & 1.19 & 1.19 & 1.13 & 1.15 & 1.40 
			\\
			\iffalse
			\multicolumn{8}{c}{Maximum rejections (in\#)}  \\
			Size-BM  
			& 4  & 5  & 5 &  5  & 4 &  4 & 13 \\
			Size-OP 
			& 2 & 2 & 2 & 2 & 2 & 3 & 3
			\\
			Size-INV  & 3 & 3 & 3 & 3 & 3 & 3 & 6
			\\
			\fi
			\hline
			\hline
			
	\end{tabular}}
\end{adjustbox}	
\parbox{0.9\textwidth}{\footnotesize%
	\vspace{.1cm} % If wanted space after the bottomrule
	{Note}: 
	We test the zero alpha null hypotheses within the Fama-French five-factor model framework using various controlling procedures on the full sample. The nominal level is 5\% and the rolling window size is $T = 240$. 
}
\end{table}

Figure \ref{figFMRejections} plots the number of rejected size-BM portfolios over time. Evidently, the rejection numbers vary over time: there is almost zero rejection according to all procedures in the first decade of the sample period. The number of nonzero portfolios increased in the late 1990s, peaked when the dot-com bubble crashed in the early 2000s, and declined afterwards. Importantly, the SCC test identifies more violations than other procedures for about $40\%$ of the sample period. For the remaining periods, the rejection numbers of SCC are on par with other procedures. While results from the benchmark procedures are similar, the gap between SCC and other procedures can be very substantial. For example, SCC detected 13 portfolios with nonzero alpha in March 2001, as opposed to 1 or 2 according to the benchmark procedures. The findings above corroborate with our theory and simulations results, suggesting that SCC can significantly improve testing outcomes.


% a result which can then be used to to improve , % 
%The five-factor risk return relationship \citep{fama2015five} is just a model. It does not explain the expected return of all sorted portfolios. But empirically, these violations of the factor model are an important first diagnostic in understanding why the assets are not consistent with the factor model and % which allows us to 
%allows us to subsequently improve the models for returns and average returns. 


\begin{figure}[!ht] % 
	\caption{% Number of rejected Size-BM portfolios: a rolling window analysis
	Number of rejected Size-BM portfolios from 1963 to 2022} 
	\label{figFMRejections}\centering
	
	\includegraphics[width=.6\textwidth,angle = -90]{Size-BM-100} 
	
	% \subfloat[Size-BM]{{\includegraphics[width=.35\textwidth,angle = -90]{Size-BM-100} }}
	% \subfloat[Size-OP]{{\includegraphics[width=.35\textwidth,angle = -90]{Size-OP-100} }}
	
		%	\vspace{.5cm}
		%	
		%	\subfloat[
		%	\if1\edits\red{\fi
		%		Size-INV
		%		\if1\edits}\fi
		%	]{{\includegraphics[width=.35\textwidth,angle = -90]{Size-INV-100} }}
	
	
	\begin{minipage}{0.8\linewidth}
		\begin{tablenotes}
			\small
			\item {
				\medskip
				Note: The nominal level is 5\% and the rolling window size is $T = 240$. 
				% and the maximum number of rejected portfolios (in\#). 
			}
		\end{tablenotes}
	\end{minipage}
\end{figure}

% \clearpage



\section{Conclusions}
\label{secConc}

We introduce a simple procedure to control for false discoveries and identify individual signals in scenarios involving many tests, dependent test statistics, and  potentially sparse signals.
			The tool is agnostic to the underlying dependence structure and scalable to deal with high dimensions.
			 Our approach is a sequential version of the global Cauchy combination test proposed by  \cite{liu2020cauchy}. 
			 By applying the global test recursively on a sequence of expanding subsets of ordered $p$-values, 
			 our sequential Cauchy combination test enables the identification of individual violations.
 
			
We show that the sequential Cauchy combination test achieves strong familywise error rate control and is less conservative compared to  popular statistical inequality-based methods (such as the Bonferroni correction and  subsequent improvements of \citeauthor{holm1979simple}, \citeyear{holm1979simple},  \citeauthor{hommel1988stagewise}, 
\citeyear{hommel1988stagewise} and   \citeauthor{hochberg1988sharper}, 
\citeyear{hochberg1988sharper}) and the Gumbel method.

The Cauchy transformation has proven its value in a genome-wide association study of Crohn's disease \citep[][Section 4.3]{liu2020cauchy}, but its applicability extends beyond genomics. 
We revisit two important 
needle-in-a-haystack problems 
in financial econometrics, where the test statistics have either serial or
cross-sectional dependence:  monitoring drift bursts and searching for nonzero alpha assets. 
The drift burst test of \citet{christensen2018drift} detects the presence of explosive trends in asset prices using  high-frequency intraday data. The test statistics are computed from overlapping windows, resulting in high autocorrelation. 
We  also revisit the 
\citet{fama2015five}
multi-factor model to identify nonzero alpha financial assets. Detecting these rare nonzero alphas among a large group of financial assets is challenging, especially when the test statistics are likely to be cross-sectionally correlated.
Without a proper controlling procedure, one might flag false discoveries or miss important signals. Our results indicate that the sequential Cauchy combination test is the a preferable method for both applications.


We emphasize that our sequential Cauchy combination test is not limited to  financial econometrics. 
We anticipate its applicability to  a wide range of hypothesis tests in fields such as economics, finance, medicine, marketing and climate studies, as it can handle various types of dependence effectively. 




\bibliographystyle{chicago}
\bibliography{Reference}

\appendix


\section{Stochastic Allocations of Indivisible Goods}\label{sec:app}
% title: what is the purpose of this section

In this section we consider a particular application of our results, for the problem of \textit{stochastic allocations of indivisible goods}. 
% title: basic model  
The setting postulates a set of $n$ agents $1,\ldots,n,$ and $m$ items, $1,\ldots,m,$ to be distributed amongst the agents.
A \emph{deterministic allocation} of the items to the agents is a mapping $A:[m]\rightarrow [n]$, determining which agent gets each item. 
% Given an allocation $A$, we denote by $A_i=A^{-1}(i)$ - the set of items allocated to $i$ in $A$. 
We denote by $\mathcal{A}$ the set of deterministic allocations. 
A \emph{stochastic allocation}, $d$, is a distribution over the deterministic allocations.  The set of all possible stochastic allocations is: 
\begin{align*}
    \mathcal{D} = \{d \mid p_d \colon \mathcal{A} \to [0,1], \sum_{A \in \mathcal{A}} p_d(A) = 1\}
\end{align*}   
Each agent $j$ is associated with a function $u_j \colon \mathcal{A} \to \mathbb{R}_{\geq 0}$ that describes its utility from a deterministic allocation.
Agents are assumed to care only about their own share (allowing us to use the following abuse of notation in which $u_j$ takes a bundle $b$ of items), their utilities are assumed to be normalized ($u_j(\emptyset) = 0$), monotone ($u_j(b_1) \leq u_j(b_2)$ if $b_1 \subseteq b_2$), and submodular ($u_j(b_1) + u_j(b_2) \geq u_j(b_1 \cup b_2) + u_j(b_1 \cap b_2)$ for any bundles $b_1,b_2$).
We also assume that utilities $(u_i)_{i=1}^n$ are given in the \emph{value oracle model}, meaning that we do not have a direct access to them, but only to an oracle that indicates the value of an agent from a given deterministic allocation.
Lastly, it is assumed that each agent has a positive utility from the set of all items.
% \eden{z1 > 0}
% \eden{should we explain somewhere that submodularity in the context of people's utilities makes a lot of sense? Maybe to explain what it is with diminishing returns?}
% \eden{should we say something about the value-oracle model? if so, where?}

% \eden{maybe to change "when agents have submodular utilities" to: "under these settings"?}
We prove that, in this setting, an approximately-optimal leximin solution with \emph{only} a multiplicative error can be obtained in polynomial time.
Specifically, we prove that a $\frac{1}{3}$-approximation\footnote{Throughout this section, we only discuss multiplicative approximations; so, for brevity, we use the term "$\multApprox$-approximation" to refer to "$(\multApprox,0)$-approximation".} can be obtained deterministically, whereas a $\frac{(e-1)^2}{e^2-e+1} \approx 0.52$-approximation can be obtained w.h.p.
As a reference point, it is worth noting that the problem of maximizing the egalitarian welfare in these settings has been shown to be NP-hard to approximate to a (multiplicative) factor better than than $1-\frac{1}{e} \approx 0.632$ \cite{kawase_max-min_2020}.
% \eden{should we say, as \textcite{kawase_max-min_2020}, that given an approximation algorithm with a multiplicative error of $\multError$ for welfare maximization, we obtain a approximate leximin with a multiplicative error of $\frac{\multError}{1-\multError +\multError^2}$?}.

% \footnote{Recall that a $(\multError,\additiveError)$-approximation has at most a multiplicative error of $\multError$ and at most an additive error of $\additiveError$. Therefore, the lower the parameters, the higher the accuracy.}




Given a stochastic allocation $d$, the expected utility of agent $j$ is given by
\begin{align*}
	E_j(d) = \sum_{A\in \mathcal{A}}p_d(A)\cdot u_j(A)
\end{align*}
The goal is to find a stochastic allocation that maximizes the set of functions $E_1,\ldots,E_n$. 
Formally, we consider the following problem:
\begin{align*}
	\lexmaxmin \quad &\{E_1(d), E_2(x), \dots E_n(d)\} \\
	s.t. \quad  & d \in \mathcal{D}
\end{align*}
% \eden{maybe $\maxlexmin$?} 
\eden{should write something about the output size, as \textcite{kawase_max-min_2020}}
That is, the feasible region is the set of stochastic allocations ($S = \mathcal{D}$) and the objective functions are the expected utilities ($f_i = E_i$ for any $i\in [N]$).

However, we shall see that an $\multApprox$-approximation to leximin is first and foremost an $\multApprox$-approximation to the egalitarian welfare. Therefore, the same hardness result applies to our problem as well.

\begin{lemma}
    If a solution is an $\multApprox$-approximation to leximin, then it is also an $\multApprox$-approximation to the egalitarian welfare.
\end{lemma}

% \eden{I removed the proof for now due to lack of space}
% \begin{proof}
%     Let $d \in \mathcal{D}$ be a stochastic allocation, and assume that it is an $\multApprox$-approximation to leximin. 
%     By definition, there is no solution that is $(\multApprox,0)$-preferred over it --- $d' \nAlphaBetaPreferredParams{\multApprox}{0} d$ for any $d' \in \mathcal{D}$.
%     Suppose by contradiction that $d$ is \emph{not} an $\multApprox$-approximation to the egalitarian welfare, and let $d^*$ be the optimal solution to this problem.
%     This means that the smallest objective value of $d'$ is less than smallest objective value of $d^*$ times $\multApprox$ --- $\valBy{1}{d} < \multApprox \cdot\valBy{1}{d^*}$.
%     But it follows that $d^*\alphaBetaPreferredParams{\multApprox}{0} d $;
%     for $k=1$, the required for $j<k$ is vacuously true 
%     and $\valBy{1}{d^*} > \frac{1}{\multApprox}\valBy{1}{d}$.
%     However, we know that the $d' \nAlphaBetaPreferredParams{\multApprox}{0} d$ for any $d' \in \mathcal{D}$, so it is true in particular for $d^*$. This is a contradiction.
% \end{proof}
As the proof is straightforward, it is omitted.


\textcite{kawase_max-min_2020} present an approximation algorithm that relates the problem of finding a stochastic allocation that approximates the egalitarian welfare, to the problem of finding a \emph{deterministic} allocation that approximates the \emph{utilitarian welfare} (i.e., the sum of utilities):
\begin{align*}
 \max \quad &\sum_{i=1}^n u_i(A)   \;\;
        \quad s.t. \quad   A \in \mathcal{A}  \tag{U1}\label{eq:utilitarian}
\end{align*}
% \erel{Can you write the maximization problem for utilitarian welfare?}

In this paper, we adapt their algorithm and prove the following relation to leximin:
\begin{theorem}
\label{th:app-main}
Suppose we are given a randomized algorithm that returns a deterministic allocation that approximates the utilitarian welfare with multiplicative error $\multError$ (with success probability $p$).
Then, Algorithm \ref{alg:basic-ordered-Outcomes} can be used to obtain a stochastic allocation that approximates leximin with a multiplicative error of at most $\frac{\multError}{1-\multError +\multError^2}$ (with the same probability).
\end{theorem}

\noindent Proving this theorem will yield two immediate results since there are known algorithms to approximate the utilitarian welfare when the agents' utility functions are monotone and submodular.
First, there are deterministic approximation algorithms with a multiplicative error $\frac{1}{2}$ \cite{Fisher1978, Buchbinder2019}, and therefore:
\begin{corollary}
    Algorithm \ref{alg:basic-ordered-Outcomes} can be used to obtain a stochastic allocation that approximates leximin with a multiplicative error at most 
    $
    \frac{0.5}{1-0.5+0.5^2} = 
    \frac{2}{3}$.
\end{corollary}
\noindent Second, there is a randomized approximation algorithm with a multiplicative error of  $\frac{1}{e}$ w.h.p \cite{vondrak_optimal_2008}, and therefore:
\begin{corollary}
    Algorithm \ref{alg:basic-ordered-Outcomes} can be used to obtain a stochastic allocation that approximates leximin with a multiplicative error at most $\frac{e}{e^2-e+1} \approx 0.48$ w.h.p.
\end{corollary}

The proof of Theorem \ref{th:app-main} is provided in Appendix \ref{sec:app-sec-proofs}. 



\end{document}
