% !TEX root = ./CauchyCombination.tex
\section{Proof of Theorem \ref{thm}}
\label{sec:proof}
%\begin{proof}
\textbf{Proof.}	Let $\mathcal{R}^{(s)}$ be the collection of rejected hypotheses in step $s$
	with $s=\left\{ 1,2,\ldots ,d\right\} $, $\mathcal{R}^{(0)}=\emptyset$, and the rejection set $\mathcal{R}=%
	\mathcal{R}^{(d)}$. The SCC testing procedure can be viewed as a sequential rejection procedure, as outlined in Table \ref{tabDecisionRule}.
	\begin{table}[H]
		\caption{Sequential rejection procedure of the SCC test}
		\label{tabDecisionRule}
		\centering
		\begin{tabular}{p{1.cm}p{3.8cm}p{10.5cm}}
			\hline
			Step & Hypothesis & Decision \\ 
			$s=1$ & $\mathcal{H}_0^{\left( 1\right) }= \bigcap_{j=1}^d H_{(j)}$ & $%
			\mathcal{R}^{(1)}=H_{(1)}$ if $\widetilde{p}_{(1)}\leq\alpha$; otherwise $%
			\mathcal{R}^{(1)}=\emptyset $ \\ 
			$s=2$ & $\mathcal{H}_0^{\left( 2\right) }=\bigcap_{j=2}^d H_{(j)}$ & $%
			\mathcal{R}^{(2)}=\mathcal{R}^{(1)} \cup H_{(2)}$ if $\widetilde{p}%
			_{(2)}\leq\alpha$; otherwise $\mathcal{R}^{(2)}=\mathcal{R}^{(1)}$ \\ 
			$\ldots$ & $\ldots$ & $\ldots$ \\ 
			$s=d$ & $\mathcal{H}_0^{\left( d\right) }= H_{(d)}$ & $\mathcal{R}^{(d)}=%
			\mathcal{R}^{(d-1)} \cup H_{(d)}$ if $\widetilde{p}_{(d)}\leq\alpha$;
			otherwise $\mathcal{R}^{(d)}=\mathcal{R}^{(d-1)}$ \\ \hline
		\end{tabular}
	\end{table}
	Let $\mathcal{N}_{s}\left(\mathcal{R}^{(s-1)}\right)$ be the successor
	function, which represents hypotheses to be rejected at the step $s$ given $%
	\mathcal{R}^{(s-1)}$. The successor function of the SCC test is defined by: 
	\begin{equation*}
		\mathcal{N}_{s}\left( \mathcal{R}^{(s-1)}\right) =\left\{ 
		\begin{array}{ll}
			H_{(s)} & \text{if }\widetilde{p}_{(s)}\leq \alpha _{\mathcal{R}
				^{(s-1)}}=\alpha \\ 
			\emptyset & \text{otherwise}%
		\end{array}
		\right. ,
	\end{equation*}
	for $s=1,... ,d$, which is either an empty
	set or contains a single hypothesis. The rejection set at step $s$ is the
	union of the rejection set at the previous step $\mathcal{R}^{(s-1)} $ and
	its successor function $\mathcal{N}_{s}(\mathcal{R}^{(s-1)})$. In other
	words, we have  that the rejection set is made up of: 
	\begin{equation}
		\mathcal{R}^{(s)}= \mathcal{R}^{(s-1)} \cup \mathcal{N}_{s}(\mathcal{R}%
		^{(s-1)}) =\mathcal{N}_{1}(\mathcal{R}^{(0)}) \cup \mathcal{N}_2(\mathcal{R}%
		^{(1)}) \cup ...\cup \mathcal{N}_{s}(\mathcal{R}^{(s-1)}).
	\end{equation}
	This means that at each step of the procedure, we update the rejection set 
	by including the previously hypotheses and any new 
	hypotheses that would be rejected based on the successor function. By
	doing so, we accumulate evidence against the null hypotheses, and the
	rejection set either remains the same or grows as we progress.
	
	Suppose that $\mathcal{R}^{(s)}=\mathcal{F}$,  the probability of the SCC test making
	at least one false positive rejection is given by: 
	\begin{eqnarray*}
		\Pr \left\{ \mathcal{R}\cap \mathcal{T\neq \emptyset }\right\}  &=&\Pr
		\left\{ \mathcal{R}^{\left( d\right) }\cap \mathcal{T\neq \emptyset }%
		\right\}  \\
		&=&\Pr \left\{ \left( \mathcal{R}^{\left( s\right) }\cup \mathcal{N}%
		_{s+1}\left( \mathcal{R}^{\left( s\right) }\right) \ldots \cup \mathcal{N}%
		_{d}\left( \mathcal{R}^{\left( d-1\right) }\right) \right) \cap \mathcal{%
			T\neq \emptyset }\right\}  \\
		&=&\Pr \left\{ \left( \mathcal{N}_{s+1}\left( \mathcal{R}^{\left( s\right)
		}\right) \ldots \cup \mathcal{N}_{d}\left( \mathcal{R}^{\left( d-1\right)
		}\right) \right) \cap \mathcal{T\neq \emptyset }\right\}  \\
		&=&\Pr \left\{ \min_{j\in \left[ s+1,d\right] }\tilde{p}_{\left( j\right)
		}\leq \alpha \right\} =\Pr \left\{ \tilde{p}_{\left( s+1\right) }\leq \alpha
		\right\}. 
	\end{eqnarray*}%
	This is based on the definitions of the rejection set and the successor function and the facts that $\mathcal{F}\cap\mathcal{T}=\emptyset$ and $\tilde{p}%
	_{\left( j\right) }$ increases with $j$.  Since the null hypothesis
	corresponding to\ $\tilde{p}_{\left( s+1\right) }$ is $\mathcal{H}%
	_{0}^{(s+1)}=\bigcap_{j=s+1}^{d}H_{(j)}$, which equals $\mathcal{T}$ when $%
	\mathcal{R}^{(s)}=\mathcal{F}$, from \cite{liu2020cauchy}, 
	\begin{equation*}
		\lim_{\alpha \rightarrow 0}\Pr \left\{ \tilde{p}_{\left( s+1\right) }\leq
		\alpha \right\} \rightarrow \alpha 
	\end{equation*}%
	under Assumption \ref{ass1} when $d$ is fixed and Assumptions \ref{ass1} and \ref{ass2} when $d=o(h^{\eta})$ with $0<\eta<1/2$. Consequently,%
	\begin{equation*}
		\lim_{\alpha \rightarrow 0}\Pr \left\{ \mathcal{R}\cap \mathcal{T\neq
			\emptyset }\right\} =\lim_{\alpha \rightarrow 0}\Pr \left\{ \tilde{p}%
		_{\left( s+1\right) }\leq \alpha \right\} \rightarrow \alpha \text{.}
	\end{equation*}%
%\end{proof}


\clearpage








