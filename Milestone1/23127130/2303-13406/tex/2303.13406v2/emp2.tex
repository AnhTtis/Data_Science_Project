\section{Example 2: In Search of Nonzero Alpha Assets}
\label{secApplFan}

The Capital Asset Pricing Model (CAPM) is a prominent risk model. However, in light of empirical evidence revealing systematic patterns in stock
returns, often referred to as ``anomalies", many additional risk factors have been introduced to explain average returns. 
These risk factors are said to represent some
dimension of undiversifiable systematic risk that should be compensated
with higher returns. If the factor model fully characterizes expected
returns, the regression intercept (also known as the ``alpha") should theoretically equal zero. 
 
We search for
nonzero alpha assets using the \citet{fama2015five} five-factor model framework. 
The conventional approach is to run time series regressions on each individual asset, and subsequently perform individual tests on the estimated alphas. 
The number of assets that need to be tested simultaneously is large (e.g., all of the
S\&P 500 stocks) and the test statistics of the
cross-sectional alphas are most likely correlated due to the presence of unknown common factors 
\citep[see e.g.,][]{giglio2021thousands}. There is a general consensus in the empirical  finance literature that mispriced assets are rare \citep[see e.g.,][]{fan2015power,giglio2021thousands}. 
To tackle the challenge of multiple testing, several methods can be used. These include the benchmark procedures, the SCC test, and a screening procedure proposed by \citet{fan2015power}. All of these procedures control the FWER and have the ability to identify individual violations.\footnote{An alternative objective is to control the false discovery rate, which is defined as the proportion of false discoveries (see e.g., \citeauthor{barras2010false}, \citeyear{barras2010false} and \citeauthor{giglio2021thousands}, \citeyear{giglio2021thousands} for two examples).}  


Section \ref{sec_HS} provides an overview of the nonzero alpha hypothesis and introduces the corresponding test.
Section % \ref{ssecSim} 
S3.2 
of the Online Supplement presents simulation results that compare the performance of the controlling procedures in the context of the
nonzero alpha test. In Section \ref{sec_KF}, we
examine Fama-French portfolios formed on bivariate sorts and search for portfolios with a  nonzero alpha. 

\subsection{Nonzero Alpha Hypothesis  and Test}
\label{sec_HS}

The multi-factor pricing model, motivated by the Arbitrage Pricing Theory 
\citep{ross1976arbitrage}, postulates how financial returns are related to
market risks. This model has enjoyed widespread application in asset pricing and
portfolio management. 
Let $y_{it}$ be the excess return (i.e., real rate of return minus the
risk-free rate) of the $i$th financial asset at day $t$ and consider the
following linear regression model: 
\begin{eqnarray}  \label{eqCrossDecomp}
y_{it} = a_i + \bm{b}^{\prime }_i \bm{f}_t + u_{it}, \quad \text{with } i =
1, \ldots, d, \, t = 1, \ldots, T,
\end{eqnarray}
where $a_i$ is an intercept, often referred to as ``alpha", $\bm{b}_i = (b_{i1},
\ldots, b_{iK})^{\prime }$ is a vector of factor sensitivities or loadings, also known as ``betas", $\bm{f}_t = (f_{1t}, \ldots, f_{Kt})^{\prime }$ are
observable factors, and $u_{it}$ is an idiosyncratic error which is
uncorrelated with the factors. 
A well-known example of \eqref{eqCrossDecomp} is the three-factor model of \citet{fama1992cross}, 
which captures a substantial portion of the variation in the cross-section of average returns and absorbs a lot of the anomalies that have plagued the CAPM \citep[see
also][]{fama1996multifactor}.

Our objective is to identify individual assets with a nonzero alpha. The
null hypothesis of each asset is therefore $H_i: a_i=0$ (`there is no
mispricing of asset $i$') and the alternative hypothesis is $a_i\neq 0$
(`asset $i$ is mispriced'), for $i=1,\ldots, d$. 
The most common way to test
this null hypothesis is to use a simple $t$-statistic for $a_i$, i.e.,
\begin{eqnarray}  \label{eqTestAlpha}
{X}_i = \frac{\hat{a}_i}{ \hat{\sigma}_{\hat{a}_i}},
\end{eqnarray}
where $\hat{a}_i$ is the estimated alpha 
and 
$\hat{%
\sigma}_{\hat{a}_i}$ is the estimated standard error,  for each asset $i = 1, ..., d$. 
% 
Under the null hypothesis, the test statistic \eqref{eqTestAlpha} follows a
Student-$t$ distribution, i.e., $X_i \sim t(\nu)$, with $\nu$ being the
degrees of freedom. 
A viable detection strategy involves computing the test
statistic \eqref{eqTestAlpha} for each asset in the cross-section and rejecting
the null hypothesis when $\abs{X_i}$ exceeds a pre-specified quantile of the 
$t(\nu)$ distribution.


The multiplicity issue arises when dealing with a large number of assets.  One important benchmark for controlling false discoveries in testing factor pricing models is the power enhancement test proposed by \cite{fan2015power}. 
This global test employs a screening technique that incidentally identifies individual violations. 
We refer to this approach as the screening method and provide a comprehensive description of the procedure in the Online Supplement. 
We can also use other approaches such as 
the  inequality-based, Gumbel method and SCC test.  
It is worth noting that the cross-sectional test statistics are likely to be
cross-correlated \citep[see e.g.,][]{giglio2021thousands}, and hence the Gumbel method and inequality-based procedures are again expected to be conservative. 
While it is true that a Student-$t$ distribution does not exactly fulfill the assumptions of the Cauchy combination test,  \citet{liu2020cauchy}
 show through simulations that the Cauchy approximation
remains accurate under such a departure from normality. 

Once again, we assess the finite sample performance of the SCC testing
procedure by comparing it with existing controlling procedures, which include the four inequality-based procedures, the Gumbel method and the screening approach, 
for the identification of nonzero alpha assets in both simulation settings and empirical applications. 
We exclude the resampling approach considered for the drift burst test, as it is not suitable for the current cross-sectional context.
Complete details of the simulation designs and results can be found in the Online Supplement.  
Unlike the simulations in Section \ref{secSims}, we simulate excess returns from the \citet{fama2015five} five-factor model, with its  parameters calibrated to the empirical data. We then compute the test statistics from the simulated data.
Our findings show that the SCC test outperforms all other procedures in terms of controlling the FWER, global power, and successful detection rate. The Gumbel method and screening method tend to be most conservative in their outcomes. 

\subsection{Empirics: Kenneth French Portfolios} \label{sec_KF}

In the empirical analysis, we study the portfolios 
available in
Kenneth French's Data Library.\footnote{\url{https://mba.tuck.dartmouth.edu/pages/faculty/ken.french/data_library.html}} 
Specifically, we focus on the  $d = 100$  portfolios formed  bivariately ($10\times10$) based on size and book-to-market  (Size-BM), size and investment (Size-INV), and size and operating profitability (Size-OP). 
% 
The left-hand side of  equation \eqref{eqCrossDecomp} comprises value-weighted portfolio excess returns and the right-hand side are the five factors in the \citet{fama2015five} five-factor model (i.e., market, value, size, profitability and investment). 
The sample period spans from July 1963 to November 2022 and consists of $713$ monthly observations. 
To conduct the nonzero alpha test, we use a rolling window approach with a window size of $T = 240$ observations.  
We treat the $100$ portfolios as one family and control the familywise error rate at the $5\%$ level.

Table \ref{tabCrossFFK} reports the average rejection frequencies of zero alpha null (across the rolling analysis) using various controlling procedures. As anticipated, there are few violations. For instance, in the case of the Size-BM portfolios, the SCC test shows an average rejection frequency of 2.60\%. Overall, the SCC test consistently yields higher rejection frequencies compared to the inequality-based methods, the screening procedure, and the Gumbel method, in that order. These empirical results align with prior research on mispricing, confirming the rarity of nonzero alpha assets  \citep[see \textit{e.g.},][]{fama1996multifactor,fan2015power,giglio2021thousands}. Nonetheless, the SCC testing procedure can detect more of these rare violations. 

\begin{table}[!ht] 
	\centering
	\caption{Average rejection frequencies (in\%) of zero alpha null across a rolling analysis}
	\label{tabCrossFFK}
	\begin{adjustbox}{width=0.85\textwidth}
		\begin{tabular}{c | ccccccccc}
			\hline
			&\multicolumn{1}{c}{Bonferroni}
			&\multicolumn{1}{c}{Holm} 
			&\multicolumn{1}{c}{Hommel}  
			&\multicolumn{1}{c}{Hochberg} 
			&\multicolumn{1}{c}{Gumbel}
			&\multicolumn{1}{c}{Screening}
			&\multicolumn{1}{c}{SCC} 
			\\
			\hline
			Size-BM  
			& 1.25 & 1.26 & 1.27 & 1.26 & 1.09 & 1.09 & 2.60 \\
			Size-OP 
			& 0.53 & 0.53 & 0.53 & 0.53 & 0.51 & 0.61 & 0.75
			\\
			Size-INV  & 1.19 & 1.19 & 1.19 & 1.19 & 1.13 & 1.15 & 1.40 
			\\
			\iffalse
			\multicolumn{8}{c}{Maximum rejections (in\#)}  \\
			Size-BM  
			& 4  & 5  & 5 &  5  & 4 &  4 & 13 \\
			Size-OP 
			& 2 & 2 & 2 & 2 & 2 & 3 & 3
			\\
			Size-INV  & 3 & 3 & 3 & 3 & 3 & 3 & 6
			\\
			\fi
			\hline
			\hline
			
	\end{tabular}}
\end{adjustbox}	
\parbox{0.85\textwidth}{\footnotesize%
	\vspace{.1cm} % If wanted space after the bottomrule
	{Note}: 
	We test the zero alpha null hypotheses within the Fama-French five-factor model framework using various controlling procedures. The nominal level is 5\% and the rolling window size is $T = 240$. 
}
\end{table}

Evidently, the rejection numbers vary over time,  as can be seen in Figure \ref{figFMRejections}. 
In the first decade of the sample period, there is almost zero rejection according to all procedures.  The number of identified nonzero portfolios starts to increase in the late 1990s, reaching its peak during the dot-com bubble crash in the early 2000s, and declined afterwards. 
The SCC test identifies more violations than other procedures for about $40\%$ of the sample period. 
During the remaining periods, the rejection numbers of the SCC test are on par with the other procedures. 
While the results from the benchmark procedures are similar, the gap between the SCC test and other procedures can be very substantial. For instance, in March 2001, the SCC test detected 13 portfolios with nonzero alphas, while the benchmark procedures identify only 1 or 2 such portfolios. 
The findings above align with our theoretical expectations and are consistent with our previous simulations results, reinforing the notion that using SCC can yield significant improvements in testing outcomes.

\begin{figure}[!ht] % 
	\caption{Number of rejected Size-BM portfolios from 1963 to 2022} 
	\label{figFMRejections}\centering
	
	\includegraphics[width=.5\textwidth,angle = -90]{Size-BM-100} 
	
	
	\begin{minipage}{0.8\linewidth}
		\begin{tablenotes}
			\small
			\item {
				\medskip
				Note: The nominal level is 5\% and the rolling window size is $T = 240$. 
			}
		\end{tablenotes}
	\end{minipage}
\end{figure}





