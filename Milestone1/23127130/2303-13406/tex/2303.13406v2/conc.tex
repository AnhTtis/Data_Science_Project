\section{Conclusions}
\label{secConc}

We introduce a simple procedure to control for false discoveries and identify individual signals in scenarios involving many tests, dependent test statistics, and  potentially sparse signals.
			The tool is agnostic to the underlying dependence structure and scalable to deal with high dimensions.
			 Our approach is a sequential version of the global Cauchy combination test proposed by  \cite{liu2020cauchy}. 
			 By applying the global test recursively on a sequence of expanding subsets of ordered $p$-values, 
			 our sequential Cauchy combination test enables the identification of individual violations.
 
			
We show that the sequential Cauchy combination test achieves strong familywise error rate control and is less conservative compared to  popular statistical inequality-based methods (such as the Bonferroni correction and  subsequent improvements of \citeauthor{holm1979simple}, \citeyear{holm1979simple},  \citeauthor{hommel1988stagewise}, 
\citeyear{hommel1988stagewise} and   \citeauthor{hochberg1988sharper}, 
\citeyear{hochberg1988sharper}) and the Gumbel method.

The Cauchy transformation has proven its value in a genome-wide association study of Crohn's disease \citep[][Section 4.3]{liu2020cauchy}, but its applicability extends beyond genomics. 
We revisit two important 
needle-in-a-haystack problems 
in financial econometrics, where the test statistics have either serial or
cross-sectional dependence:  monitoring drift bursts and searching for nonzero alpha assets. 
The drift burst test of \citet{christensen2018drift} detects the presence of explosive trends in asset prices using  high-frequency intraday data. The test statistics are computed from overlapping windows, resulting in high autocorrelation. 
We  also revisit the 
\citet{fama2015five}
multi-factor model to identify nonzero alpha financial assets. Detecting these rare nonzero alphas among a large group of financial assets is challenging, especially when the test statistics are likely to be cross-sectionally correlated.
Without a proper controlling procedure, one might flag false discoveries or miss important signals. Our results indicate that the sequential Cauchy combination test is the a preferable method for both applications.


We emphasize that our sequential Cauchy combination test is not limited to  financial econometrics. 
We anticipate its applicability to  a wide range of hypothesis tests in fields such as economics, finance, medicine, marketing and climate studies, as it can handle various types of dependence effectively. 

