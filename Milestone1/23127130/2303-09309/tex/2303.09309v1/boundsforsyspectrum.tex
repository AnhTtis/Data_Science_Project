% ------------------------------------------------------------------------
% bjourdoc.tex for birkjour.cls*******************************************
% ------------------------------------------------------------------------
%%%%%%%%%%%%%%%%%%%%%%%%%%%%%%%%%%%%%%%%%%%%%%%%%%%%%%%%%%%%%%%%%%%%%%%%%%

\documentclass{birkjour}
%
%
% THEOREM Environments (Examples)-----------------------------------------
%
\newtheorem{thm}{Theorem}[section]
\newtheorem{cor}[thm]{Corollary}
\newtheorem{lem}[thm]{Lemma}
\newtheorem{prop}[thm]{Proposition}
\theoremstyle{definition}
\newtheorem{defn}[thm]{Definition}
\theoremstyle{remark}
\newtheorem{rem}[thm]{Remark}
\newtheorem{ex}{Example}
\numberwithin{equation}{section}


\usepackage{amsmath, amssymb, amsfonts}
\usepackage{amsthm}
\usepackage{mathtools}
%\usepackage{listings}

\usepackage{tikz}
\usetikzlibrary{backgrounds,calc,positioning}

\usepackage{venndiagram}

\begin{document}
	
	%-------------------------------------------------------------------------
	% editorial commands: to be inserted by the editorial office
	%
	%\firstpage{1} \volume{228} \Copyrightyear{2004} \DOI{003-0001}
	%
	%
	%\seriesextra{Just an add-on}
	%\seriesextraline{This is the Concrete Title of this Book\br H.E. R and S.T.C. W, Eds.}
	%
	% for journals:
	%
	%\firstpage{1}
	%\issuenumber{1}
	%\Volumeandyear{1 (2004)}
	%\Copyrightyear{2004}
	%\DOI{003-xxxx-y}
	%\Signet
	%\commby{inhouse}
	%\submitted{March 14, 2003}
	%\received{March 16, 2000}
	%\revised{June 1, 2000}
	%\accepted{July 22, 2000}
	%
	%
	%
	%---------------------------------------------------------------------------
	%Insert here the title, affiliations and abstract:
	%
	
	
	\title[On the Bounds of Symplectic Spectrum]
	{On the Bounds of Symplectic Spectrum}
	
	

	
	
	%----------Author 1
	\author{V. B. Kiran Kumar}
	\address{Department of Mathematics, Cochin University of Science and Technology, Kerala, India 682022}
	\email{vbk@cusat.ac.in}
	\thanks{This work was completed with the support of KSCSTE-YSA Research Grant and  INSPIRE PhD Fellowship}
	
	%----------Author 2
	\author{Anmary Tonny}
	\address{Department of Mathematics, Cochin University of Science and Technology, Kerala, India 682022}
	\email{anmarytonny97@gmail.com}

	%----------classification, keywords, date
	\subjclass{Primary 99Z99; Secondary 00A00}
	
	\keywords{Symplectic transformations; Symplectic spectrum.}
	
	\date{\today}
	%----------additions
	%\dedicatory{To my boss}
	%%% ----------------------------------------------------------------------
	
	\begin{abstract}
		In 2018, B. V. Rajarama Bhat and T. C. John developed the Williamson's Normal form for strictly positive invertible operators on infinite-dimensional separable real Hilbert spaces and defined symplectic spectrum for these operators. Recently an interlacing relation between the eigenvalues and symplectic eigenvalues for operators with countable spectrum was formulated. We consider two different classes of operators and determine the bounds for the symplectic spectrum using the bounds of the spectrum for some special class of operators. We also compute their symplectic spectrum and Williamson's Normal form. In the course, we define the symplectic equivalence of two operators and show that symplectically equivalent operators will have the same symplectic spectrum. Finally, we illustrate an example of the truncation method using the algorithm we developed.
	\end{abstract}
	
	%%% ----------------------------------------------------------------------
	\maketitle
	%%% ----------------------------------------------------------------------
	%\tableofcontents
\section{Introduction}
The diagonalizability of symmetric matrices by orthogonal matrices is a well-known result. In 1936, J. Williamson developed a new diagonalization for positive definite matrices of even order, called the Williamson's Normal form which is stated below.

\begin{thm} \cite{jw01} \label{wnffd} [Williamson's Normal Form] 
	Let $T \in M_{2n}(\mathbb{R})$ be a strictly positive matrix. Then there exist a symplectic matrix $L$ of order $2n \times 2n$ and a strictly positive diagonal matrix D of order $n \times n$ such that
	$$
	T = L^T
	\begin{bmatrix}
		D & 0 \\
		0 & D
	\end{bmatrix}L.
	$$
	Furthermore, D is unique upto the order of its entries.		
\end{thm}
\noindent The entries of the diagonal matrix $D$ are defined as the symplectic eigenvalues of the matrix $T$. Let us denote it by $\sigma_{sy} (T).$ The symplectic eigenvalues play an important role in symplectic topology, classical Hamiltonian dynamics, quantum mechanics and quantum information. In 2018, B. V. Rajarama Bhat and T. C. John developed the Williamson's Normal form for strictly positive invertible operators on infinite-dimensional separable real Hilbert spaces to tackle the infinite-dimensional analogues of Theorem \ref{wnffd} and its applications. For example in \cite{bv02}, where non-commutative analogues of the classical Gaussian distributions called quantum Gaussian states are studied, the strictly positive invertible operators occur as covariance operators of quantum Gaussian states. Also, the symplectic spectrum (which we will define below) is the complete invariant for the class of covariance operators associated with the quantum Gaussian states. Before stating the infinite-dimensional Williamson's Normal form we give some definitions. 

\begin{defn}\cite{bv01}
	Let $\mathcal{H}$ be a real Hilbert space and $I$ be the identity operator on $\mathcal{H}$. The involution operator $J$ on $\mathcal{H} \oplus \mathcal{H}$ is given by $ J = 
	\begin{bmatrix}
		0 & I \\
		-I & 0
	\end{bmatrix}.
	$ Let $\mathcal{H}$ and $\mathcal{K}$ be two real Hilbert spaces. A bounded invertible linear operator $Q : \mathcal{H} \oplus \mathcal{H} \rightarrow \mathcal{K} \oplus \mathcal{K}$ is called a symplectic transformation if $Q^{T}JQ = J$, where $J$ on the left side is the involution operator on $\mathcal{K} \oplus \mathcal{K}$ and that on the right side is the involution operator on $\mathcal{H} \oplus \mathcal{H}$.
\end{defn}

\begin{rem} \label{symcompo}
	Let $\mathcal{H,K,R}$ be real Hilbert spaces, $L: \mathcal{H} \oplus \mathcal{H} \rightarrow \mathcal{K} \oplus \mathcal{K}$ and $M: \mathcal{K} \oplus \mathcal{K} \rightarrow \mathcal{R} \oplus \mathcal{R}$ be symplectic transformations. Then $ML: \mathcal{H} \oplus \mathcal{H} \rightarrow \mathcal{R} \oplus \mathcal{R}$ is a symplectic transformation. This can be seen as follows. 
	
	\begin{align*}
		(ML)^{T}J(ML) &= (L^TM^T)J(ML) \\
		&= L^T(M^TJM)L \\
		&= L^TJL = J.	
	\end{align*}
\end{rem}

\noindent Below we state the infinite-dimensional Williamson's Normal form for strictly positive invertible operators on infinite-dimensional separable real Hilbert spaces.
\begin{thm} \cite{bv01} \label{wnfifd} [Williamson's Normal Form]
	Let $\mathcal{H}$ be a real separable Hilbert space and $T$ be a strictly positive invertible operator on $ \mathcal{H} \oplus \mathcal{H}$. Then there exists a positive invertible operator $D$ on $\mathcal{H}$ and a symplectic transformation $M : \mathcal{H} \oplus \mathcal{H} \rightarrow \mathcal{H} \oplus \mathcal{H}$ such that 
	$$
	T = M^T
	\begin{bmatrix}
		D & 0 \\
		0 & D
	\end{bmatrix}M$$ 
	Further, $D$ is unique upto conjugation with an orthogonal transformation. 
\end{thm}

\noindent The spectrum of $D$ is called the symplectic spectrum of $T$, let us denote it by $\sigma_{sy}(T).$

\noindent We will proceed as follows. We will consider two classes of positive invertible operators as given below:
\newline Let $\mathcal{H}$ be a real separable Hilbert space, $T$ be a positive invertible operator on $\mathcal{H} \oplus \mathcal{H}$. 
\begin{enumerate}
	\item \underline{Class $\mathbb{A}$}: This class consists of operators $T$ of the form $T = \begin{bmatrix}
		A & 0 \\
		0 & B
	\end{bmatrix},$ where $A, B$ are positive invertible operators on $\mathcal{H}$ such that $AB = BA$.
	\item \underline{Class $\mathbb{B}$}: This class consists of operators $T$ of the form $T = \begin{bmatrix}
		A & B \\
		B & A
	\end{bmatrix},$ where $A$, $B$ are self-adjoint operators on $\mathcal{H}$ such that the operators $A + B$ and $A - B$ are positive invertible and commuting.
\end{enumerate}
We will compute the symplectic spectrum, Williamson's Normal form and bounds for the symplectic spectrum for operators in these classes. In the course, we will also define the symplectic equivalence of two operators and show that symplectically equivalent operators will have the same symplectic spectrum. Finally, we will numerically illustrate an example for the truncation method. 


\section{Preliminary Results} \label{mainresults}
The aim of our work is to compute the symplectic spectrum of a given positive invertible operator say $T$ on $\mathcal{H} \oplus \mathcal{H}$, where $\mathcal{H}$ is a real separable Hilbert space. We also wish to check if we could approximate the symplectic spectrum by truncation of the given operator. 

\subsection{Truncation Method}

In this section, we will learn in detail about the method of spectral approximation.


\noindent Let $\mathcal{H}$ be a separable complex Hilbert space, $\{e_{1},e_{2},...\}$ be its countable orthonormal basis. Let $A \in B(\mathcal{H})$ be self-adjoint. Define $P^{\prime}_{n}$ to be the orthogonal projection to the subspace $\mathcal{H}_n =\textrm{ span }\{e_{1},...,e_{n}\}$, Also define $A_{n}$ to be the restriction of the operator $P^{\prime}_{n}AP^{\prime}_{n}$ to $\mathcal{H}_n$. The problem of spectral approximation includes the identification of the two limiting sets \cite{bcn01}
\begin{align*}
	\textrm{ lim inf }\sigma(A_n) &= \{\lambda \in \mathbb{C} : \lambda \textrm{ is the limit of some sequence } (\lambda_{n})_{n=1}^{\infty} \textrm{ with } \lambda_{n} \in \sigma(A_{n}) \}, \\
	\textrm{ lim sup }\sigma(A_n) &= \{\lambda \in \mathbb{C} : \lambda \textrm{ is the partial limit of some sequence } (\lambda_{n})_{n=1}^{\infty} \textrm{ with } \\ &\lambda_{n} \in \sigma(A_{n}) \}.
\end{align*} 

\begin{ex} \cite{bcn01} \label{examplelimits}
	For $a,b \in \mathbb{R}$, the eigenvalues of the matrix $B(a,b) = \begin{bmatrix}
		a & b \\
		b & -a
	\end{bmatrix}$ are $\pm \sqrt{a^2 + b^2}$. Now choose any sequence $(a_n)_{n = 1}^{\infty}$ of numbers $a_n \in (0,1)$ and define $b_n \in (0,1)$ by $a_n^2 + b_n^2 = 1.$ Define $$A = \textrm{ diag }(B(a_1, b_1), B(a_2, b_2), \cdots).$$ Then $\sigma(A) = \{-1, 1\}$. Since $\sigma(A_{2n}) = \{-1, 1 \}$ and $\sigma(A_{2n+1}) = \{-1, a_n, 1\}$, we have lim inf $\sigma(A_n) = \{-1, 1\}$ while lim sup $\sigma(A_n)$ as the union of the set $\{-1,1\}$ and the set of all partial limits of the sequence $(a_n)$.
\end{ex}

\noindent Let $m = \textrm{ inf }\sigma(A)$, $M = \textrm{ sup }\sigma(A)$. Then,
$$\{m, M\} \subset \sigma(A) \subset \textrm{ lim inf }\sigma(A_n) \subset \textrm{ lim sup } \sigma(A_n) \subset [m, M], $$ 
for every selfadjoint operator $A \in B(\mathcal{H})$ \cite{bcn01}.

\begin{defn} \cite{bcn01}
	Let $A$ be a bounded operator on $\mathcal{H}$. Then the essential spectrum of $A$ denoted as $\sigma_{ess}(A)$ is defined as follows
	$$\sigma_{ess}(A) = \{\lambda \in \mathbb{C} : A - \lambda I + K(\mathcal{H}) \textrm{ is not invertible in } B(\mathcal{H}) \setminus K(\mathcal{H})\},$$ where $K(\mathcal{H})$ is the set of all compact linear maps on $\mathcal{H}$. Equivalently, $\lambda \in \sigma_{ess}(A)$ iff $\lambda$ is an eigenvalue of infinite multiplicity or there exist a sequence $(\mu_n) \in \sigma(A)$ such that $\mu_n$ converges to $\lambda$.
\end{defn}

\begin{ex}
	Consider the operator $A$ as in Example \ref{examplelimits}. Then $\sigma_{ess}(A) = \{-1,1\}$.
\end{ex}
\noindent The next theorem states that the spectral approximation can be done for operators whose essential spectrum is connected.

\begin{thm} \cite{bcn01} \label{bcnthm}
	Let $A \in \mathcal{H}$ be self-adjoint and suppose $\sigma_{ess}(A)$ is connected. Then $$\textrm{ lim inf }\sigma(A_n) = \textrm{ lim sup } \sigma(A_n) = \sigma(A).$$
\end{thm}

\subsection{A Special Case}

\noindent Before going to the main results, we make a simple observation on the symplectic spectrum of the positive definite even order real matrix of a special form.


\begin{lem} \label{rstfdAA}
	Let $T \in M_{2n}(\mathbb{R})$ be a strictly positive matrix . Suppose $T$ takes the form $\begin{bmatrix}
		A & 0 \\
		0 & A
	\end{bmatrix}$, where $A$ is $n \times n$ strictly positive matrix, then $\sigma_{sy}(T) = \sigma(A)$.
\end{lem}

\begin{proof}
	Since $A$ is strictly positive matrix, it is orthogonally diagonalizable, that is there exist an orthogonal matrix $O$ and a diagonal matrix $D$ such that $A = O^TDO.$ Also $\sigma(A) = \sigma(O^TDO)= \sigma(D).$ Define the matrix $L$ as $\begin{bmatrix}
		O & 0 \\
		0 & O
	\end{bmatrix}.$ Then
	\begin{align*}
		L^TJL & = \left(
		\begin{bmatrix}
			O & 0 \\
			0 & O
		\end{bmatrix} \right)^T
		\begin{bmatrix}
			0 & I \\
			-I & 0
		\end{bmatrix}
		\begin{bmatrix}
			O & 0 \\
			0 & O
		\end{bmatrix} \\
		&= \begin{bmatrix}
			O^T & 0 \\
			0 & O^T
		\end{bmatrix}
		\begin{bmatrix}
			0 & I \\
			-I & 0
		\end{bmatrix}
		\begin{bmatrix}
			O & 0 \\
			0 & O
		\end{bmatrix} \\
		&= 
		\begin{bmatrix}
			0 & O^T \\
			-O^T & 0
		\end{bmatrix}
		\begin{bmatrix}
			O & 0 \\
			0 & O
		\end{bmatrix} \\
		&= 
		\begin{bmatrix}
			0 & O^TO \\
			-O^TO & 0
		\end{bmatrix}\\
		&=
		\begin{bmatrix}
			0 & I \\
			-I & 0
		\end{bmatrix} \\
		&= J
	\end{align*} that is $L$ is symplectic. Also 
	\begin{align*}
		L^TL &= \left( \begin{bmatrix}
			O & 0 \\
			0 & O
		\end{bmatrix}
		\right)^T \begin{bmatrix}
			O & 0 \\
			0 & O
		\end{bmatrix} \\
		&= \begin{bmatrix}
			O^T & 0 \\
			0 & O^T
		\end{bmatrix} \begin{bmatrix}
			O & 0 \\
			0 & O
		\end{bmatrix} \\
		&= \begin{bmatrix}
			O^TO & 0 \\
			0 & O^TO
		\end{bmatrix} \\
		&= \begin{bmatrix}
			I & 0 \\
			0 & I
		\end{bmatrix}
	\end{align*} Similarly $LL^T = I$, that is $L$ is orthogonal as well. Therefore $L^{-1} = L^T$. Now  \begin{align*}
		L^T
		\begin{bmatrix}
			D & 0 \\
			0 & D
		\end{bmatrix}L &= 
		\begin{bmatrix}
			O^T & 0 \\
			0 & O^T
		\end{bmatrix}
		\begin{bmatrix}
			D & 0 \\
			0 & D
		\end{bmatrix}
		\begin{bmatrix}
			O & 0 \\
			0 & O
		\end{bmatrix} \\
		&= 
		\begin{bmatrix}
			O^TDO & 0 \\
			0 & O^TDO
		\end{bmatrix} \\ 
		&= 
		\begin{bmatrix}
			A & 0 \\
			0 & A
		\end{bmatrix} \\ &= T.
	\end{align*} That is, $\sigma_{sy}(T) = \sigma(D)$. Also since $L$ is orthogonal, the above set of equations gives $\sigma(D) = \sigma(T) = \sigma(A)$. Therefore, $\sigma_{sy}(T) = \sigma(D) = \sigma(A).$
\end{proof}

\begin{rem} \label{touseinthm}
	For the Williamson's Normal form of matrices, the matrix $D$ needs to be diagonal while for the infinite-dimensional Williamson's Normal form, $D$ is taken to be a positive invertible operator. Hence, in that case, the above lemma holds trivially with the identity transformation on $\mathcal{H} \oplus \mathcal{H}$ as the symplectic transformation.
\end{rem}

\noindent If $\{e_{1}, e_{2},...\}$ is the countable orthonormal basis for $\mathcal{H}$, then $\{(e_{1},0), (e_{2},0),...\}$ $\cup$ $\{(0, e_{1}), (0, e_{2}),...\}$ will be a countable orthonormal basis of $\mathcal{H} \oplus \mathcal{H}.$ Now define $P_{n}$ as the orthogonal projection to the subspace $$\mathcal{H}_{n} \oplus \mathcal{H}_{n} = \textrm{ span } \{(e_{1},0), (e_{2},0),...,(e_{n}, 0)\} \cup \{(0, e_{1}), (0, e_{2}),..., (0,e_{n})\}$$ of $\mathcal{H} \oplus \mathcal{H}$. Define the operator $T_{2n}$ on $\mathcal{H}_{n} \oplus \mathcal{H}_{n}$ as the restriction of the operator $P_{n}TP_{n}$ to $\mathcal{H}_n \oplus \mathcal{H}_n$.

\noindent The next theorem explains the truncation method for a special form of operator $T$ on $\mathcal{H} \oplus \mathcal{H}.$ 

\begin{thm} \label{rstAA}
	Let $\mathcal{H}$ be a real separable Hilbert space, $A$ be a strictly positive invertible operator on $\mathcal{H}$. Now consider the operator $T : \mathcal{H} \oplus \mathcal{H} \rightarrow \mathcal{H} \oplus \mathcal{H}$ as $T = 
	\begin{bmatrix}
		A & 0 \\
		0 & A
	\end{bmatrix}
	$. Suppose the essential spectrum of $A$ is connected, then the symplectic spectral approximation can be done, that is 
	$$\textrm{lim inf } \sigma_{sy}(T_{2n}) = \textrm{ lim sup } \sigma_{sy}(T_{2n}) = \sigma_{sy}(T) = \sigma(A).$$
\end{thm}

\begin{proof}
	If $T$ takes the form $\begin{bmatrix}
		A & 0 \\
		0 & A
	\end{bmatrix}$, by Remark \ref{touseinthm} we have $\sigma_{sy}(T) = \sigma(A).$ 
	\newline \noindent  Also, $$ T_{2n} = 
	\begin{bmatrix}
		A_{n} & 0 \\
		0 & A_{n}
	\end{bmatrix} $$Hence by Lemma \ref{rstfdAA}, $\sigma_{sy}(T_{2n}) = \sigma(A_{n}),$ that is the symplectic spectral approximation of $T$ is actually the spectral approximation of $A$.	Since $\sigma_{ess}(A)$ is connected, by Theorem \ref{bcnthm} we have $$\textrm{lim inf } \sigma(A_{n}) = \textrm{ lim sup } \sigma(A_{n}) = \sigma(A),$$ that is $$\textrm{lim inf } \sigma_{sy}(T_{2n}) = \textrm{ lim sup } \sigma_{sy}(T_{2n}) = \sigma_{sy}(T) = \sigma(A).$$
\end{proof}

\begin{ex}
	Let $a$ be a positive real-valued continuous function in $L^{\infty}(\mathbb{T})$, where $\mathbb{T}$ is the complex unit circle, such that $a_{n} = a_{-n}$, for $n = 0,1,...$, where $a_{n}$s are the Fourier coefficients of $a$. Then the Toeplitz matrix
	$$ A = 
	\begin{bmatrix}
		a_{0} & a_{1} & a_{2} & \cdots \\
		a_{1} & a_{0} & a_{1} & \cdots \\
		a_{2} & a_{1} & a_{0} & \cdots \\
		\cdots & \cdots & \cdots & \cdots 
	\end{bmatrix}$$ induces a bounded, positive invertible operator on $l^{2}$. It is well-known that $\sigma(A) = \sigma_{ess}(A) = [\textrm{essinf } a, \textrm{ esssup } a],$ that is the essential spectrum of $A$ is connected. Therefore in this case with $T = 
	\begin{bmatrix}
		A & 0 \\
		0 & A
	\end{bmatrix}$ on $l^{2} \oplus l^{2}$, we have $\textrm{lim inf } \sigma_{sy}(T_{2n}) = \textrm{ lim sup } \sigma_{sy}(T_{2n}) = \sigma_{sy}(T) = \sigma(A).$
\end{ex} 

\noindent The next example is motivated by Example 3.2 of \cite{bcn01}. 

\begin{ex}
	Consider the function $f$ defined on $L^\infty[-\pi, \pi]$ as follows:
	$$
	f(x) = 
	\begin{cases}
		1, \quad x \in [-\pi, -\frac{\pi}{2}] \\
		2, \quad x \in [-\frac{\pi}{2}, \frac{\pi}{2}] \\
		1, \quad x \in [\frac{\pi}{2}, \pi]
	\end{cases}
	$$
	Then the $n$th Fourier coefficients $a_n$ is given as follows:
	$$
	a_n = a_{-n} =
	\begin{cases}
		\frac{3}{2}, \quad n = 0 \\
		\frac{1}{n\pi}, \quad n = 4k + 1 \\
		0, \quad n \textrm{ is non-zero and even} \\
		-\frac{1}{n\pi}, \quad n = 4k + 3
	\end{cases}
	$$
	and put 
	$$
	A = \left[ \begin{array}{c | cc| cc| cc|c}
		a_0 & a_{-1} & a_1 & a_{-2} & a_2 & a_{-3} & a_3 & \cdots \\
		\hline
		a_{1} & a_{0} & a_{2} & a_{-1} & a_{3} & a_{-2} & a_{4} & \cdots \\
		a_{-1} & a_{-2} & a_{0} & a_{-3} & a_{1} & a_{-4} & a_{2} & \cdots \\
		\hline
		a_{2} & a_{1} & a_{3} & a_{0} & a_{4} & a_{-1} & a_{5} & \cdots \\
		a_{-2} & a_{-3} & a_{-1} & a_{-4} & a_{0} & a_{-5} & a_{1} & \cdots \\
		\hline
		a_{3} & a_{2} & a_{4} & a_{1} & a_{5} & a_{-2} & a_{4} & \cdots \\
		a_{-3} & a_{-4} & a_{-2} & a_{-5} & a_{-1} & a_{-4} & a_{2} & \cdots \\
		\hline
		\cdots & \cdots & \cdots & \cdots & \cdots & \cdots & \cdots & \cdots
	\end{array}
	\right]
	$$ Then $A$ induces a bounded self-adjoint operator on $l^2$ such that $\sigma(A) = \sigma_{ess}(A) = \{ 1,2\}$ and lim sup $\sigma(A) = $ lim inf $\sigma(A) = [1,2]$. Now define $T : l^2 \oplus l^2 \rightarrow l^2 \oplus l^2$ as $T = \begin{bmatrix}
		A & 0 \\
		0 & A
	\end{bmatrix}$. Then $\sigma_{sy}(T) = \sigma(A) = \{1, 2\}$. In this case the symplectic spectral approximation doesnot work as the limiting sets doesnot coincide with the spectrum of $A$. 
\end{ex}

\section{Main Results}

\subsection{Williamson's Normal Form}

\noindent In this section, we will compute the symplectic spectrum and the Williamson's Normal form of the two classes of operators defined in the Introduction. We will also find conditions where the method of truncation would work. 

\begin{thm} \label{rstAB}
	Let $\mathcal{H}$ be a real separable Hilbert space, $A$, $B$ be a strictly positive invertible operators on $\mathcal{H}$ such that $AB = BA$. Now consider the operator $T : \mathcal{H} \oplus \mathcal{H} \rightarrow \mathcal{H} \oplus \mathcal{H}$ as $T = 
	\begin{bmatrix}
		A & 0 \\
		0 & B
	\end{bmatrix}
	$, then $\sigma_{sy}(T) = \sigma(A^{\frac{1}{2}}B^{\frac{1}{2}}).$ In particular if the essential spectrum of $A^{\frac{1}{2}}B^{\frac{1}{2}}$ is connected, then the symplectic spectral approximation can be done.
\end{thm}

\begin{proof}
	Since $A$ and $B$ are positive invertible operators, they have unique square roots denoted as $A^{\frac{1}{2}}$ and $B^{\frac{1}{2}}$ respectively and they commute since $A$ and $B$ commute. Now define the operator $L$ on $\mathcal{H} \oplus \mathcal{H} \rightarrow \mathcal{H} \oplus \mathcal{H}$ as
	$$L = 
	\begin{bmatrix}
		A^{\frac{1}{4}}B^{- \frac{1}{4}} & 0 \\
		0 & A^{- \frac{1}{4}}B^{\frac{1}{4}}
	\end{bmatrix}
	$$
	Since $A$ and $B$ are positive and commuting so are their powers. Then, 
	
	\begin{align*}
		L^TJL &=
		\begin{bmatrix}
			A^{\frac{1}{4}}B^{- \frac{1}{4}} & 0 \\
			0 & A^{- \frac{1}{4}}B^{\frac{1}{4}}
		\end{bmatrix}^T
		\begin{bmatrix}
			0 & I \\
			-I & 0
		\end{bmatrix}
		\begin{bmatrix}
			A^{\frac{1}{4}}B^{- \frac{1}{4}} & 0 \\
			0 & A^{- \frac{1}{4}}B^{\frac{1}{4}}
		\end{bmatrix} \\
		&= \begin{bmatrix}
			(A^{\frac{1}{4}}B^{- \frac{1}{4}})^T & 0 \\
			0 & (A^{- \frac{1}{4}}B^{\frac{1}{4}})^T
		\end{bmatrix}
		\begin{bmatrix}
			0 & I \\
			-I & 0
		\end{bmatrix}
		\begin{bmatrix}
			A^{\frac{1}{4}}B^{- \frac{1}{4}} & 0 \\
			0 & A^{- \frac{1}{4}}B^{\frac{1}{4}}
		\end{bmatrix} \\
		&= \begin{bmatrix}
			A^{\frac{1}{4}}B^{- \frac{1}{4}} & 0 \\
			0 & A^{- \frac{1}{4}}B^{\frac{1}{4}}
		\end{bmatrix}
		\begin{bmatrix}
			0 & I \\
			-I & 0
		\end{bmatrix}
		\begin{bmatrix}
			A^{\frac{1}{4}}B^{- \frac{1}{4}} & 0 \\
			0 & A^{- \frac{1}{4}}B^{\frac{1}{4}}
		\end{bmatrix} \\
		&= \begin{bmatrix}
			0 & A^{\frac{1}{4}}B^{- \frac{1}{4}} \\
			-A^{- \frac{1}{4}}B^{\frac{1}{4}} & 0
		\end{bmatrix}
		\begin{bmatrix}
			A^{\frac{1}{4}}B^{- \frac{1}{4}} & 0 \\
			0 & A^{- \frac{1}{4}}B^{\frac{1}{4}}
		\end{bmatrix} \\
		&= \begin{bmatrix}
			0 & I \\
			-I & 0
		\end{bmatrix} \\
		&=J.
	\end{align*}
	
	That is $L$ is symplectic and 
	
	\begin{align*}
		L^T
		\begin{bmatrix}
			A^\frac{1}{2}B^{\frac{1}{2}} & 0 \\
			0 & A^\frac{1}{2}B^{\frac{1}{2}}		
		\end{bmatrix}L &= 
		\begin{bmatrix}
			A^{\frac{1}{4}}B^{- \frac{1}{4}} & 0 \\
			0 & A^{- \frac{1}{4}}B^{\frac{1}{4}}
		\end{bmatrix}^T
		\begin{bmatrix}
			A^\frac{1}{2}B^{\frac{1}{2}} & 0 \\
			0 & A^\frac{1}{2}B^{\frac{1}{2}}		
		\end{bmatrix}
		\begin{bmatrix}
			A^{\frac{1}{4}}B^{- \frac{1}{4}} & 0 \\
			0 & A^{- \frac{1}{4}}B^{\frac{1}{4}}
		\end{bmatrix} \\
		& = \begin{bmatrix}
			A^{\frac{1}{4}}B^{- \frac{1}{4}} & 0 \\
			0 & A^{- \frac{1}{4}}B^{\frac{1}{4}}
		\end{bmatrix}
		\begin{bmatrix}
			A^\frac{1}{2}B^{\frac{1}{2}} & 0 \\
			0 & A^\frac{1}{2}B^{\frac{1}{2}}		
		\end{bmatrix}
		\begin{bmatrix}
			A^{\frac{1}{4}}B^{- \frac{1}{4}} & 0 \\
			0 & A^{- \frac{1}{4}}B^{\frac{1}{4}}
		\end{bmatrix} \\
		&= \begin{bmatrix}
			A^{\frac{3}{4}}B^{\frac{1}{4}} & 0 \\
			0 & A^{\frac{1}{4}}B^{\frac{3}{4}}
		\end{bmatrix}
		\begin{bmatrix}
			A^{\frac{1}{4}}B^{- \frac{1}{4}} & 0 \\
			0 & A^{- \frac{1}{4}}B^{\frac{1}{4}}
		\end{bmatrix} \\
		&= \begin{bmatrix}
			A & 0 \\
			0 & B
		\end{bmatrix} \\
		&= T
	\end{align*}
	
	\noindent Since $A$ and $B$ are positive invertible and commuting, the operator $A^\frac{1}{2}B^{\frac{1}{2}}$ is positive invertible. Hence in this case we have $D = A^\frac{1}{2}B^{\frac{1}{2}}$ and $\sigma_{sy}(T) = \sigma(A^\frac{1}{2}B^{\frac{1}{2}})$. 
	\newline \noindent Now suppose if $\sigma_{ess}(A^\frac{1}{2}B^{\frac{1}{2}})$ is connected, then by Theorem \ref{rstAA} we have
	$$\sigma_{sy}(T) = \sigma(A^\frac{1}{2}B^{\frac{1}{2}}) = \textrm{lim inf } \sigma((A^\frac{1}{2}B^{\frac{1}{2}})_{n}) = \textrm{ lim sup } \sigma((A^\frac{1}{2}B^{\frac{1}{2}})_{n}).$$
\end{proof}

\begin{ex}
	Consider the real Hilbert space $l^2$. Let $A$ be the identity operator and define the diagonal operator $B$ as follows: 
	$$
	B = (b_{ij}) = \begin{cases}
		1 + \frac{1}{n}, \quad \textrm{ if } i = j = n^2 \\
		1, \qquad \textrm{       otherwise}
	\end{cases}
	$$ Define $T$ on $l^2 \oplus l^2$ as $T = \begin{bmatrix}
		A & 0 \\
		0 & B
	\end{bmatrix}$. Then $AB = BA$. The given operator satisfies the conditions in Theorem \ref{rstAB}. Then the operator $A^{\frac{1}{2}}B^{\frac{1}{2}}$ is the diagonal operator given by 
	$$
	A^{\frac{1}{2}}B^{\frac{1}{2}} = (c_{ij}) = \begin{cases}
		\left(1 + \frac{1}{n}\right)^{\frac{1}{2}}, \quad \textrm{ if } i = j = n^2 \\
		1, \qquad \textrm{       otherwise}
	\end{cases}
	$$ Then $\sigma_{sy}(T) = \sigma(A^{\frac{1}{2}}B^{\frac{1}{2}}) = \{1\} \cup \{ \left( 1 + \frac{1}{n} \right)^{\frac{1}{2}}, n \in \mathbb{N} \}$. Note that here $\sigma_{ess}(A^{\frac{1}{2}}B^{\frac{1}{2}}) = \{1\}$ which is connected and hence the spectral approximation will work here.
\end{ex}

\begin{ex}
	Define matrices $\tilde{A}$ and $\tilde{B}$ as follows.
	$$\tilde{A} = \begin{bmatrix}
		4 & 2 \\
		2 & 2
	\end{bmatrix} \qquad \tilde{B} = \begin{bmatrix}
	2 & 1 \\
	1 & 1
\end{bmatrix}.$$ Then $\tilde{A}$ and $\tilde{B}$ are positive definite matrices that commute. Now define operators $A_1$ and $B_1$ on the real Hilbert space $l^2$ as follows.
\begin{align*}
	A_1 &= \tilde{A} \oplus \tilde{A} \oplus \cdots, \\
	B_1 &= \tilde{B} \oplus \tilde{B} \oplus \cdots.
\end{align*} Then the operators $\tilde{A}$ and $\tilde{B}$ are positive invertible operators on $l^2$ that commute. Now define operators $A = A_1^2$ and $B = B_1^2$ on $l^2$, that is 
\begin{align*}
	A = A_1^2 &= \tilde{A}^2 \oplus \tilde{A}^2 \oplus \cdots, \\
	B = B_1^2 &= \tilde{B}^2 \oplus \tilde{B}^2 \oplus \cdots.
\end{align*} By commutativity and positivity of $\tilde{A}$ and $\tilde{B}$ we have the same for operators $A$ and $B$ respectively. Now define $T$ on $l^2 \oplus l^2$ as $$T = \begin{bmatrix}
A & 0 \\
0 & B
\end{bmatrix}.$$ Then $T$ satisfies the conditions given in Theorem \ref{rstAB}. Also we have \begin{align*}
A^{\frac{1}{2}}B^{\frac{1}{2}} &= A_1B_1 \\
&= \begin{bmatrix}
	26 & 16 \\
	16 & 10
\end{bmatrix} \oplus \begin{bmatrix}
26 & 16 \\
16 & 10
\end{bmatrix} \oplus \cdots.
\end{align*} Therefore $\sigma_{sy}(T) = \sigma(A^{\frac{1}{2}}B^{\frac{1}{2}}) = \{18 \pm 8\sqrt{5}\}$.
\end{ex}

\noindent Before moving to the final result of this section, we introduce a new concept. 

\begin{defn}
	Let $\mathcal{H}$ be a real Hilbert space, $A$ and $B$ be operators on $\mathcal{H} \oplus \mathcal{H}.$ Then $A$ and $B$ are said to be \textbf{symplectically equivalent} if there exists a symplectic transformation $L : \mathcal{H} \oplus \mathcal{H} \rightarrow \mathcal{H} \oplus \mathcal{H}$ such that $A = L^{T}BL.$ 
\end{defn}

\begin{ex}
	Consider $\mathcal{H}, T, D$ as in Theorem \ref{wnfifd}. Then Theorem \ref{wnfifd} shows that the operators $T$ and $\begin{bmatrix}
		D & 0 \\
		0 & D
	\end{bmatrix}$ on $\mathcal{H} \oplus \mathcal{H}$ are symplectically equivalent through the symplectic transformation $M$.  
\end{ex}


\noindent The next theorem shows that positive invertible symplectically equivalent operators will have the same symplectic spectrum. It plays an important role in the proof of Theorem \ref{rstABBA}.
\begin{thm} \label{rstsymeq}
	Let $\mathcal{H}$ be a real separable Hilbert space, $A$ and $B$ be strictly positive invertible operators on $\mathcal{H} \oplus \mathcal{H}$ such that $A$ and $B$ are symplectically equivalent. Then $\sigma_{sy}(A) = \sigma_{sy}(B).$
\end{thm}

\begin{proof}
	Since the operators $A$ and $B$ are symplectically equivalent, there exist a symplectic transformation $L : \mathcal{H} \oplus \mathcal{H} \rightarrow \mathcal{H} \oplus \mathcal{H}$ such that $A = L^{T}BL.$ Since, $B$ is positive invertible on $\mathcal{H} \oplus \mathcal{H}$, by Theorem \ref{wnfifd}, there exists a symplectic transformation say $M$ and a positive invertible operator $D$ such that
	$$ B = M^{T}
	\begin{bmatrix}
		D & 0 \\
		0 & D
	\end{bmatrix}M,$$ and $\sigma_{sy}(B) = \sigma(D).$ Define $K = ML$. Then $K$ is a symplectic transformation as the composition of two symplectic transformations is again symplectic (by Remark \ref{symcompo}). Therefore, 
	$$A = L^{T}BL = L^{T}M^{T}
	\begin{bmatrix}
		D & 0 \\
		0 & D			
	\end{bmatrix}ML = K^{T}
	\begin{bmatrix}
		D & 0 \\
		0 & D			
	\end{bmatrix}K.$$ That is, $\sigma_{sy}(A) = \sigma(D) = \sigma_{sy}(B).$
\end{proof}

\noindent Now we state the last result of this section.
\begin{thm} \label{rstABBA}
	Let $\mathcal{H}$ be a real separable Hilbert space, $A$ and $B$ be self-adjoint operators on $\mathcal{H}$ such that the operators $A + B$ and $A - B$ are positive invertible and commuting. Let $T : \mathcal{H} \oplus \mathcal{H} \rightarrow \mathcal{H} \oplus \mathcal{H}$ be defined as $$T = 
	\begin{bmatrix}
		A & B \\
		B & A
	\end{bmatrix}$$ Then, $\sigma_{sy}(T) = \sigma((A + B)^{\frac{1}{2}}(A - B)^{\frac{1}{2}})$. In particular, if the essential spectrum of $(A + B)^{\frac{1}{2}}(A - B)^{\frac{1}{2}}$ is connected, then the symplectic spectral approximation can be done.
\end{thm}

\begin{proof}
	Define a new operator $L : \mathcal{H} \oplus \mathcal{H} \rightarrow \mathcal{H} \oplus \mathcal{H}$ as $L = \frac{1}{\sqrt{2}}
	\begin{bmatrix}
		I & I \\
		-I & I
	\end{bmatrix}$, where $I$ is the identity operator on $\mathcal{H}$. \newline Then 
	
	\begin{align*}
		L^TJL &= \left(\frac{1}{\sqrt{2}}
		\begin{bmatrix}
			I & I \\
			-I & I
		\end{bmatrix}\right)^T 
		\begin{bmatrix}
			0 & I \\
			-I & 0
		\end{bmatrix}\left(\frac{1}{\sqrt{2}}
		\begin{bmatrix}
			I & I \\
			-I & I
		\end{bmatrix}\right) \\
		&= \frac{1}{2}
		\begin{bmatrix}
			I & -I \\
			I & I
		\end{bmatrix}
		\begin{bmatrix}
			0 & I \\
			-I & 0
		\end{bmatrix}
		\begin{bmatrix}
			I & I \\
			-I & I
		\end{bmatrix} \\
		&= \frac{1}{2}
		\begin{bmatrix}
			I & I \\
			-I & I
		\end{bmatrix}
		\begin{bmatrix}
			I & I \\
			-I & I
		\end{bmatrix} \\
		&= \frac{1}{2}
		\begin{bmatrix}
			0 & 2I \\
			-2I & 0\\
		\end{bmatrix} \\
		&= J,
	\end{align*}
	that is $L$ is a symplectic transformation. Also, 
	
	\begin{align*}
		T^\prime &= L^TTL \\
		&= \left(\frac{1}{\sqrt{2}}
		\begin{bmatrix}
			I & I \\
			-I & I
		\end{bmatrix}\right)^T
		\begin{bmatrix}
			A & B \\
			B & A 
		\end{bmatrix}\left(\frac{1}{\sqrt{2}}
		\begin{bmatrix}
			I & I \\
			-I & I
		\end{bmatrix}\right) \\
		&= \frac{1}{2}
		\begin{bmatrix}
			I & -I \\
			I & I
		\end{bmatrix}
		\begin{bmatrix}
			A & B \\
			B & A
		\end{bmatrix}
		\begin{bmatrix}
			I & I \\
			-I & I
		\end{bmatrix} \\
		&= \frac{1}{2}
		\begin{bmatrix}
			A - B & -A + B \\
			A + B & A + B
		\end{bmatrix}
		\begin{bmatrix}
			I & I \\
			-I & I
		\end{bmatrix}\\
		&= \begin{bmatrix}
			A - B & 0 \\
			0 & A + B
		\end{bmatrix},
	\end{align*}
	
	\noindent that is $T$ and $T^{\prime}$ are symplectically equivalent. Hence, by Theorem \ref{rstsymeq}, $T$ and $T^{\prime}$ have the same symplectic spectrum. Hence by Theorem \ref{rstAB}, we have 
	$$\sigma_{sy}(T) = \sigma_{sy}(T^{\prime}) = \sigma((A + B)^{\frac{1}{2}}(A - B)^{\frac{1}{2}}),$$ and the symplectic spectral approximation works when the essential spectrum of $(A + B)^{\frac{1}{2}}(A - B)^{\frac{1}{2}}$ is connected.
\end{proof}

\begin{ex}
	Let $\mathcal{H}$ be a real separable Hilbert space. Consider the positive invertible operator $T$ on $\mathcal{H} \oplus \mathcal{H}$ defined as
	$$T = \begin{bmatrix}
		A & B \\
		B & A
	\end{bmatrix},$$ where $A$ and $B$ are self-adjoint operators given by \begin{align*}
		A &= \textrm{ diag }\left\{2 + \frac{1}{n^2} + \frac{1}{n^3}: n \in \mathbb{N}\right\}, \\ 
		B &= \textrm{ diag }\left\{\frac{1}{n^2} - \frac{1}{n^3}: n \in \mathbb{N}\right\}.
	\end{align*} Now $A + B = \left\{ 2 + \frac{2}{n^2}: n \in \mathbb{N}\right\}$ and $A - B = \left\{ 2 + \frac{2}{n^3}: n \in \mathbb{N} \right\}$, that is $A + B$ and $A - B$ are positive invertible operators and being diagonal they commutes. Hence $A$ and $B$ satisfies the conditions given in Theorem \ref{rstABBA}. Now
	$$T^\prime = 
	\begin{bmatrix}
		A-B & 0 \\
		0 & A+B
	\end{bmatrix}. $$ Hence by Theorem \ref{rstABBA} \begin{align*}
	\sigma_{sy}(T) = \sigma_{sy}(T^\prime) &= \sigma(((A + B)^{\frac{1}{2}}(A - B)^{\frac{1}{2}})) \\
	&= \{2\} \cup \left\{ \left[ \left( 2 + \frac{2}{n^2}\right)\left( 2 + \frac{2}{n^3}\right)\right]^{\frac{1}{2}} : n \in \mathbb{N} \right\}.
\end{align*} By Theorem 5.1 of \cite{pandey2017spectral} the operator $T^\prime$ is a positive $\mathcal{AN}$ operator and $T$ is unitarily equivalent to $T$ since $L$ is unitary as well. Hence by Theorem $3.4$ of \cite{carvajal2012operators}, the operator $T$ is an $\mathcal{AN}$ operator. (The interlacing relation between the eigenvalues and the symplectic eigenvalues can be seen in \cite{john2022interlacing}).
\end{ex}

\begin{ex}
	Let $\mathcal{H} = l^2$. Let $A$ and $B$ be operators on $\mathcal{H}$ given with the matrix representations
	
	$$
	A = \begin{bmatrix}
		6 & 3 & \frac{1}{2} & 0 & 0 & 0 & 0 & \cdots \\
		3 & 6 & 3 & \frac{1}{2} & 0 & 0 & 0 & \cdots \\
		\frac{1}{2} & 3 & 6 & 3 & \frac{1}{2} & 0 & 0 & \cdots \\
		0 & \frac{1}{2} & 3 & 6 & 3 & \frac{1}{2} & 0 & \cdots \\
		\cdots & \cdots & \cdots & \cdots & \cdots & \cdots & \cdots & \cdots\\
	\end{bmatrix}
	$$
	$$
	B = \begin{bmatrix}
		5 & 3 & \frac{1}{2} & 0 & 0 & 0 & 0 & \cdots \\
		3 & 5 & 3 & \frac{1}{2} & 0 & 0 & 0 & \cdots \\
		\frac{1}{2} & 3 & 5 & 3 & \frac{1}{2} & 0 & 0 & \cdots \\
		0 & \frac{1}{2} & 3 & 5 & 3 & \frac{1}{2} & 0 & \cdots \\
		\cdots & \cdots & \cdots & \cdots & \cdots & \cdots & \cdots & \cdots\\
	\end{bmatrix}
	$$
	That is $A$ is the Toeplitz operator of the function $6 + 6cos(t) + cos (2t) \in L^\infty(\mathbb{T})$ and $B$ is the Toeplitz operator of the function $5 + 6cos(t) + cos (2t) \in L^\infty(\mathbb{T})$. So $A$ and $B$ are self-adjoint. Also, 
	$$
	A + B = \begin{bmatrix}
		11 & 6 & 1 & 0 & 0 & 0 & 0 & \cdots \\
		6 & 11 & 6 & 1 & 0 & 0 & 0 & \cdots \\
		1 & 6 & 11 & 6 & 1 & 0 & 0 & \cdots \\
		0 & 1 & 6 & 11 & 6 & 1 & 0 & \cdots \\
		\cdots & \cdots & \cdots & \cdots & \cdots & \cdots & \cdots & \cdots\\
	\end{bmatrix}
	$$ and $A - B$ is the identity operator. Hence $A + B$ and $A - B$ commutes. Therefore $A$ and $B$ satisfies the conditions in Theorem \ref{rstABBA}. Now define $T: l^2 \oplus l^2 \rightarrow l^2 \oplus l^2$ as $T = \begin{bmatrix}
		A & B \\
		B & A 
	\end{bmatrix}$. From Theorem \ref{rstABBA}, $$\sigma_{sy}(T) = \sigma((A + B)^{\frac{1}{2}}(A - B)^{\frac{1}{2}}) = \sigma((A + B)^{\frac{1}{2}}) = \sigma(A + B)^{\frac{1}{2}}.$$ $A + B$ being a Toeplitz operator corresponding to the function $f(t) = 11 + 12 cos(t) + 2 cos(2t)$, $\sigma(A + B)$ is the essential range of $f = [1,25]$. Therefore, $\sigma_{sy}(T) = \sigma(A + B)^{\frac{1}{2}} = [1,5].$ 
\end{ex}

\subsection{Bounds for the Symplectic Spectrum}

\noindent A consequence of Theorem 11 in \cite{rbt01} is that the symplectic eigenvalues of a positive definite real matrix of even order lie within the smallest and largest eigenvalues. In this section we prove the same for the two classes of operators considered, that is we show that for the operators in Class $\mathbb{A}$ and Class $\mathbb{B}$, the symplectic spectrum lies within the bounds of the spectrum. The main tools used are the results from the real Banach algebras. The definitions and other requisites is given in the Appendix.


\noindent Below we state the main result of the section.

\begin{thm} \label{rstinclu}
	Let $T$ be an operator in Class $\mathbb{A}$ or Class $\mathbb{B}$. Then the symplectic spectral values of $T$ lies in $[m, M]$, where $m = \textrm{ inf }\sigma(T)$, $M = \textrm{ sup }\sigma(T)$. 
\end{thm}

\begin{proof}
	\noindent \underline{\textbf{Case I:}} $T$ is in Class $\mathbb{A}$.
	
	\noindent Here $\sigma(T) = \sigma(A) \cup \sigma(B).$ Since $T$ is positive, $m, M >0$. We have seen that $\sigma_{sy}(T) = \sigma(A^{\frac{1}{2}}B^{\frac{1}{2}})$. 
	
	\noindent Since, $\sigma(T) \subset [m,M]$, we have $ \sigma(A) \subset [m,M] \Rightarrow \sigma(A^{\frac{1}{2}}) = \sigma(A)^{\frac{1}{2}} \subset [m^{\frac{1}{2}}, M^{\frac{1}{2}}].$ Similarly, $\sigma(B^{\frac{1}{2}}) = \sigma(B)^{\frac{1}{2}} \subset [m^{\frac{1}{2}}, M^{\frac{1}{2}}].$
	
	\noindent From Theorem \ref{thminclu} we have, $\sigma_{sy}(T) = \sigma(A^{\frac{1}{2}}B^{\frac{1}{2}}) \subset \sigma(A^{\frac{1}{2}}) \sigma(B^{\frac{1}{2}}) \subset [m, M],$ that is the symplectic spectral values lies in $[m, M]$.  
	
	\noindent \underline{\textbf{Case II:}} $T$ is in Class $\mathbb{B}$.
	
	\noindent When this case was considered, we showed that $T$ is symplectically equivalent to the operator given by $T^{\prime} = \begin{bmatrix}
		A + B & 0 \\
		0 & A - B
	\end{bmatrix},$ by the symplectic transformation $L = \frac{1}{\sqrt{2}}\begin{bmatrix}
		I & I \\
		-I & I		
	\end{bmatrix}$. Also, \begin{align*}
		LL^T &= \frac{1}{\sqrt{2}} \begin{bmatrix}
			I & I \\
			-I & I
		\end{bmatrix} \left(\frac{1}{\sqrt{2}} \begin{bmatrix}
			I & I \\
			-I & I
		\end{bmatrix}\right)^T \\
		&= \frac{1}{2} \begin{bmatrix}
			I & I \\
			-I & I
		\end{bmatrix} \begin{bmatrix}
			I & -I \\
			I & I
		\end{bmatrix} \\
		&= \frac{1}{2} \begin{bmatrix}
			2I & 0 \\
			0 & 2I
		\end{bmatrix} \\
		&= \begin{bmatrix}
			I & 0 \\
			0 & I
		\end{bmatrix}
	\end{align*} Similarly, it can be verified that $L^TL = I$, that is $L$ is orthogonal. Hence $\sigma(T) = \sigma(T^{\prime}).$ Now proceeding as in Case II, we have
	$$\sigma_{sy}(T) = \sigma((A + B)^{\frac{1}{2}}(A - B)^{\frac{1}{2}}) \subset \sigma((A + B)^{\frac{1}{2}}) \sigma((A - B)^{\frac{1}{2}}) \subset [m,M].$$
\end{proof}

\begin{ex} \label{examplesyminclu}
	Let $H = l^2$, $A$ and $B$ be positive invertible operators on $l^2$ defined as follows: let $a = 2 + cos(t) \in L^\infty(\mathbb{T}) $. Then the Toeplitz matrix
	$$ A = 
	\begin{bmatrix}
		2 & \frac{1}{2} & 0 & 0 & \cdots \\
		\frac{1}{2} & 2 & \frac{1}{2} & 0 & \cdots \\
		0 & \frac{1}{2} & 2 & \frac{1}{2} & \cdots \\
		\cdots & \cdots & \cdots & \cdots & \cdots 
	\end{bmatrix}
	$$ will define a positive invertible operator on $l^2$. Similarly define the operator $B$ on $l^2$ by the Toeplitz matrix  
	$$ B = 
	\begin{bmatrix}
		2 & -\frac{1}{2} & 0 & 0 & \cdots \\
		-\frac{1}{2} & 2 & -\frac{1}{2} & 0 & \cdots \\
		0 & -\frac{1}{2} & 2 & -\frac{1}{2} & \cdots \\
		\cdots & \cdots & \cdots & \cdots & \cdots 
	\end{bmatrix}$$ which represents the function $b = 2 - cos(t) \in L^\infty(\mathbb{T})$. Then, $\sigma(A) = \sigma(B) = [1,3]$. Now define $T: l^2 \oplus l^2 \rightarrow l^2 \oplus l^2$ by $T = \begin{bmatrix}
		A & 0 \\
		0 & B
	\end{bmatrix}$. Then $\sigma(T) = \sigma(A) \cup \sigma(B) = [1,3].$ Hence by Theorem \ref{rstinclu} any real $r \in \mathbb{R} \setminus [1,3]$ will not be a symplectic spectral value of the operator $T$.		
\end{ex}




\section{Numerical Illustrations}

In this section, we will numerically illustrate an example where the truncation method works. All computations are done using the python program we developed.


\begin{ex} \label{exptrunwins}
	Let $\mathcal{H} = l^2$. Define $A : l^2 \rightarrow l^2$ to be the Toeplitz operators corresponding to the symbol $2 + cos(t) \in L^\infty(\mathbb{T})$. Then $A$ takes the form 
	$$ A = 
	\begin{bmatrix}
		2 & \frac{1}{2} & 0 & 0 & \cdots \\
		\frac{1}{2} & 2 & \frac{1}{2} & 0 & \cdots \\
		0 & \frac{1}{2} & 2 & \frac{1}{2} & \cdots \\
		\cdots & \cdots & \cdots & \cdots & \cdots 
	\end{bmatrix}
	$$ Then $\sigma(A) = [1,3]$. Define $T : l^2 \oplus l^2 \rightarrow l^2 \oplus l^2$ as $T = \begin{bmatrix}
		A & 0 \\ 
		0 & A
	\end{bmatrix}$. Since $A$ is a Toeplitz operator, its essential spectrum coincides with the spectrum, that is $\sigma_{ess}(A)$ is connected. Hence from Theorem \ref{rstAA}, the method of truncation will work. We will now verify this using the algorithm we have developed (the notations are as in the proof of Theorem \ref{rstAA}).
	
	
	\begin{table}[h!] 
		\centering
		\begin{tabular}{|p{0.35\linewidth} | p{0.6\linewidth}|}
			\hline
			\rule{0pt}{20pt} n  & Symplectic Eigenvalues (upto 5 decimal places) \\ \hline
			\rule{0pt}{20pt} 10 & {1.13397, 1.5, 2.0, 2.5, 2.86603} \\
			\hline
			\rule{0pt}{20pt} 20 & 1.04051, 1.15875, 1.34514, 1.58458, 1.85769, 2.14231, 2.41542, 2.65486, 2.84125, 2.95949 \\
			\hline 
			\rule{0pt}{20pt} 50 & 1.00729, 1.02906, 1.06498, 1.11454, 1.17702, 1.25149, 1.33688, 1.43194, 1.53528, 1.6454, 1.76068, 1.87946, 2.0, 2.12054, 2.23932, 2.3546, 2.46472, 2.56806, 2.66312, 2.74851, 2.82298, 2.88546, 2.93502, 2.97094, 2.99271 \\
			\hline 
			\rule{0pt}{20pt} 100 & 1.0019, 1.00758, 1.01703, 1.0302, 1.04706, 1.06753, 1.09153, 1.11899, 1.14978, 1.1838, 1.22092, 1.26099, $\cdots$ , 1.66765, 1.72634, 1.78607, 1.84661, 1.90773, 1.9692, 2.0308, 2.09227, 2.15339, 2.21393, 2.27366, 2.33235, $\cdots$, 2.77908, 2.8162, 2.85022, 2.88101, 2.90847, 2.93247, 2.95294, 2.9698, 2.98297, 2.99242, 2.9981 \\ \hline
			$\cdots$ & $\cdots$ \\
			\hline 
			$\cdots$ & $\cdots$ \\
			\hline 
			$\cdots$ & $\cdots$ \\
			\hline 
			\rule{0pt}{20pt} 1000 & 1.00002, 1.00008, 1.00018, 1.00031, 1.00049, 1.00071, 1.00096, 1.00126, 1.00159, 1.00197, 1.00238, 1.00283, $\cdots$, 1.96552, 1.97179, 1.97805, 1.98432, 1.99059, 1.99686, 2.00314, 2.00941, 2.01568, 2.02195, 2.02821, 2.03448, $\cdots$, 2.99668, 2.99717, 2.99762, 2.99803, 2.99841, 2.99874, 2.99904, 2.99929, 2.99951, 2.99969, 2.99982, 2.99992, 2.99998  \\
			\hline 
			$\cdots$ & $\cdots$ \\
			\hline
		\end{tabular}
		\caption{Truncated Symplectic Eigenvalues for $T$ in Example \ref{exptrunwins}.}  
	\end{table}
	
	\noindent It can be observed from Table 1 that as $n$ increases, the symplectic eigenvalues tend to take more values in $[1,3]$, which is the symplectic spectrum of $T$ by Theorem \ref{rstAA}.
\end{ex}

\newpage

\section{Conclusion}

\noindent The infinite-dimensional Williamson's Normal form and the symplectic spectrum of positive invertible operators on real separable Hilbert spaces are important in many fields, especially in quantum physics. Computation of symplectic spectrum and localizing them may sometimes be a difficult task. Here, our major achievement is the computation of the symplectic spectrum of two different classes of positive invertible operators on real separable Hilbert spaces. We have shown that the symplectic spectrum of an operator will lie within the smallest and largest spectral values of the operator. In the course, we also studied the symplectic equivalence of two operators and showed that symplectically equivalent operators will have the same the symplectic spectrum. We also illustrated the truncation method using the algorithm we developed. A major question that still remains is the computation of the symplectic spectrum of an arbitrary positive invertible operator and whether the properties discussed here hold in the general case.



\section*{Appendix}
Here we give some definitions and results from the real Banach algebras.
\begin{defn}
	Let $A$ be a real algebra, then the complexification of $A$ is the complex algebra given by $$\mathcal{A} = A + iA := \{a + i.b : a,b \in A\}$$ with addition, multiplication and scalar multiplication defined respectively as follows:
	
	\begin{align*}
		(a + i.b) + (c + i.d) &= (a+b) + i.(c+d) \\
		(a+i.b)(c+i.d) &= (ac - bd) + i.(ad + bc) \\
		(\alpha + i. \beta)(a +i.b) &= (\alpha a - \beta b) + i.(\alpha b + \beta a)
	\end{align*} for all $a,b,c,d \in A$ and $\alpha, \beta \in \mathbb{R}.$ 		
\end{defn} 

\noindent For a complex algebra $A$ with unit $e$ and $a \in A$, we define the spectrum of $a$ in $A$ as the subset of $\mathbb{C}$ denoted as $\sigma(a)$ and defined as:
$\sigma(a) = \{\lambda \in \mathbb{C}: a - \lambda e \textrm{ is singular in } A\}.$ It is natural to think that for a real algebra $A$ with unit $e$, the spectrum 
of an element $a$ in $A$ should be defined as the set of all real $\lambda$, for which $a - \lambda e$, is singular in $A$. But if we accept this definition, the spectra of many elements may turn out to be the empty set. For example, if $A$ is the algebra of all $2 \times 2$ real matrices, and if $a = \begin{bmatrix}
	0 & -1 \\
	1 & 0
\end{bmatrix}$
then $a - \lambda I = \begin{bmatrix}
	-\lambda & -1 \\
	1 & -\lambda
\end{bmatrix}$, where $I$ is $2 \times 2$ identity matrix, is invertible for every real $\lambda$. Hence we adopt the following definition:

\begin{defn} \cite{ik01}
	Let $A$ be a real algebra with unit $e$. For $a \in A$, the 
	spectrum of $a$ in $A$ is the subset $\sigma(a)$ of $\mathbb{C}$ defined as $\sigma(a) = \{ s + it \in \mathbb{C} : (a - se)^2 + t^2e \textrm{ is singular in }A\}.$ 
	Clearly, $s + it \in \sigma(a)$ if and only if $s - it \in \sigma(a)$.
\end{defn}

\noindent The above definition is equivalent to the following statement: "If $A$ is a real algebra with unit and $\mathcal{A}$ is the complexification of A then $\sigma_A(a) = \sigma_{\mathcal{A}}(a + i.0)$ for all $a \in A$" \cite{shk01}. We know that for any complex Banach space $X$, the spectrum of every $x \in X$ is a compact subset of $\mathbb{C}$. This is true for real Banach algebras by Corollary 1.1.18 of \cite{shk01}.

\begin{defn} \cite{shk01}
	Let $A$ be a real algebra. The carrier space of $A$, denoted by Car $(A)$, is the set of all nonzero homomorphisms from $A$ to $\mathbb{C}$, regarded as a real algebra.
\end{defn} 

\noindent Let $\Phi \in$ Car $(A)$ and define $\overline{\Phi}$ by $\overline{\Phi}(a) = \overline{\Phi(a)}$ for all $a \in A$. Then it is easy to see that $\overline{\Phi} \in $ Car $(A)$. Let $\tau: \textrm{ Car }(A) \rightarrow \textrm{ Car }(A)$ be defined by $\tau(\Phi) = \overline{\Phi}$ for $\Phi \in $ Car $(A)$.

\begin{defn} \cite{shk01}
	For $a$ in $A$, the Gelfand transform of $a$ is the map $\hat{a} :$ Car $(A) \rightarrow \mathbb{C}$, given by $\hat{a}(\Phi) = \Phi(a)$ for all $\Phi$ in Car $(A)$. The weakest topology on Car $(A)$ that makes $\hat{a}$ continuous on Car $(A)$ for all $a$ in $A$ is called the Gelfand topology	on Car $(A)$.
\end{defn} 

\noindent Theorem 1.2.9(vi) of \cite{shk01} states that if $A$ is a real commutative Banach algebra with unit $e$ then, for every $a$ in $A$, the range of $\hat{a}$ is the spectrum of $a$ in $A$.

\begin{defn} \cite{wr01}
	If $S$ is a subset of a real Banach algebra $A$, the centralizer of $S$ is the set $$\Gamma (S) = \{x \in A : xs = sx \textrm{ for every } s \in S\}.$$ We say that $S$ commutes if any two elements of $S$ commute with each other. 
\end{defn} \noindent The next two theorems gives us the real analogue of Theorem 11.22 and Theorem 11.23 in \cite{wr01}. The proof is the same as in the complex case, hence we omit the proof.
\begin{thm}
	Suppose $A$ is a real Banach algebra with unit $e$, S $\subset$ A, S commutes and B = $\Gamma(\Gamma(S))$. Then B is commutative real Banach algebra, S $\subset$ B and $\sigma_B(x) = \sigma_A(x)$ for every $x \in B.$
\end{thm}

\begin{thm} \label{thminclu}
	Suppose $A$ is a real commutative Banach algebra, $x,y \in A$ and $xy = yx$. Then $$\sigma(x + y) \subset \sigma(x) + \sigma(y) \quad and \quad \sigma(xy) \subset \sigma(x)\sigma(y).$$
\end{thm}

\section*{Acknowledgements}
V.B. Kiran Kumar thanks KSCSTE, Kerala, for financial support via the YSA-Research project. Anmary Tonny is supported by the INSPIRE PhD Fellowship of the Department of Science and Technology, Govt of India. The authors wish to thank Dr. Tiju Cherian John, Research Scientist, University of Arizona for the fruitful discussions. 



%	\nocite{*}
	\bibliographystyle{amsplain}
	\bibliography{boundsforsyspectrum}
	
	% ------------------------------------------------------------------------
\end{document}
% ------------------------------------------------------------------------
