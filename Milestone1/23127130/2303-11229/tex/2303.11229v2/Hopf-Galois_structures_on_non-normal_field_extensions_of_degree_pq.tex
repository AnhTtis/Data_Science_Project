%\documentclass[a4paper,reqno]{amsart}
\documentclass[letterpaper,12pt]{amsart}
\subjclass[2010]{12F10; 16T05}
\keywords{Hopf-Galois structures; Field extensions;	Groups of squarefree order.}
\thanks{For the purpose of open access, the author has applied a CC BY public copyright
	licence to any Author Accepted Manuscript version arising.
	\newline
	\indent Data Access Statement: Data sharing is not applicable to this article as no
	datasets were generated or analysed in this research.}

\title{Hopf-Galois structures on separable field extensions of degree $pq$}
\author{Andrew Darlington}
\date{\today}
\address{Department of Mathematics, Faculty of Environment, Science and Economy, University of Exeter, Exeter EX4 QFU. UK.}
\email{ad788@exeter.ac.uk}

\usepackage[utf8]{inputenc}
\usepackage{amsmath}
\usepackage{amsfonts}
\usepackage{amssymb}
\usepackage{amsthm}
\usepackage{pifont} %boxes around text: \begin{framed}....\end{framed}
\usepackage{enumerate} % to allow customisable numbered lists
\usepackage{enumitem}
\usepackage{wasysym} % to make the arrow symbol \pointer used as a bullet
\usepackage{mathrsfs}
\usepackage{xfrac} % used for \sfrac
\usepackage{array}
\usepackage{siunitx}
\usepackage{mathtools}
\usepackage{framed}
\usepackage{pifont}
\usepackage[utf8]{inputenc}
\usepackage[english]{babel}
\usepackage[utf8]{inputenc}
\usepackage{tikz,tkz-euclide}
\usepackage{xcolor,graphicx}
%\usepackage{biblatex}% \iffalse meta-comment
\usepackage{faktor}
\usepackage{xfrac}
%\usepackage{threeparttable}
%\usepackage{ctable}
\usepackage{float}
\usepackage{rotating}
%\usepackage{amscd}

\makeatletter
\def\bign#1{\mathclose{\hbox{$\left#1\vbox to8.5\p@{}\right.\n@space$}}\mathopen{}}
\def\Bign#1{\mathclose{\hbox{$\left#1\vbox to11.5\p@{}\right.\n@space$}}\mathopen{}}

\newtheorem{theorem}{Theorem}[section]
\newtheorem{proposition}[theorem]{Proposition}
\newtheorem{lemma}[theorem]{Lemma}
\newtheorem{remark}[theorem]{Remark}
\newtheorem{corollary}[theorem]{Corollary}

\newcommand{\e}{\mathbf{e}}
\newcommand{\Gal}{\mathrm{Gal}}
\newcommand{\Hol}{\mathrm{Hol}}
\newcommand{\Aut}{\mathrm{Aut}}

\newcolumntype{M}[1]{>{\centering\arraybackslash}m{#1}}
\newcolumntype{N}{@{}m{0pt}@{}}

\begin{document}
	\maketitle
	\bibliographystyle{amsalpha}
	
	
	
\begin{abstract}
	In 2020, Alabdali and Byott described the Hopf-Galois structures arising on Galois field extensions of squarefree degree. Extending to squarefree separable, but not necessarily normal, extensions $L/K$ is a natural next step. One must consider now the interplay between two Galois groups $G=\Gal(E/K)$ and $G'=\Gal(E/L)$, where $E$ is the Galois closure of $L/K$. In this paper, we give a characterisation and enumeration of the Hopf-Galois structures arising on separable extensions of degree $pq$ where $p$ and $q$ are distinct odd primes. This work includes the results of Byott and Martin-Lyons who do likewise for the special case that $p=2q+1$.
\end{abstract}


\section{Introduction}\label{intro}
The concepts of Hopf-Galois theory were made explicit by Chase and Sweedler \cite{CS69} in 1969. Nearly twenty years later, Greither and Pareigis showed in \cite{GP87} that the problem of finding Hopf-Galois structures on separable field extensions could be phrased and approached purely in group-theoretic terms. In short, for a given separable extension $L/K$ of fields with Galois closure $E/K$, let $G=\Gal(E/K)$, $G'=\Gal(E/L)$, $X=G/G'$, and $\lambda: G \rightarrow \text{Perm}(X)$ where $\lambda(g)(\overline{h})=\overline{gh}$. Then each Hopf-Galois structure on $L/K$ corresponds to a regular subgroup $N$ of $\mathrm{Perm}(X)$ normalised by $\lambda(G)$. The associated Hopf algebra is given by $H=E[N]^G$, with some determined $G$-action on $E$ and $N$. The isomorphism type of $N$ is then referred to as the `type' of the Hopf-Galois structure. Suppose $L/K$ is Galois with group $G$. Then the group (Hopf) algebra $H=K[G]$ endows $L/K$ with a Hopf-Galois structure of type $G$. If, further, $G$ is non-abelian, and $\rho: G \rightarrow \text{Perm}(G)$ where $\rho(g)(h)=hg^{-1}$, then the Hopf algebras $L[\lambda(G)]^{\lambda(G)}$ and $L[{\rho(G)}]^{\lambda(G)}$ are not equal, and give distinct Hopf-Galois structures on $L/K$. See Corollary $6.11$ of \cite{Chi00} for more details.

In recent years, links have been discovered between Hopf-Galois structures and other algebraic constructions such as skew braces, $1$-cocycles, and more. For the link with skew braces, see \cite{GV17}; for $1$-cocycles, see, for example, the discussion in sections $9.1$ and $9.2$ in \cite{Chi+20}. How these structures arise and behave in connection to these other objects is its own very interesting question. However, studying Hopf-Galois structures is interesting in its own right, more so given that they don't just exist on Galois extensions. A lot of literature is therefore focused on classifying Hopf-Galois structures on separable (not necessarily normal) field extensions. Let $p$ be a prime; \cite{CS20}, for example, gives the classification of Hopf-Galois structures on separable field extensions of degree $2p$ and $p^2$; \cite{Kohl98,Kohl07,Kohl16} look at order $p^n$ extensions, and give a method of looking at degree $mp$ extensions where $\gcd(m,p)=1$; and \cite{AB20} gives a complete description of what happens for a Galois extension of squarefree degree. In 2022, Byott and Martin-Lyons in \cite{BML21} looked at separable extensions of degree $pq$ with $p,q$ odd primes and $p=2q+1$. Such a $q$ is called a Sophie Germain prime, with $p$ being the associated `safe' prime. The paper can be seen as a first step in generalising the squarefree results of \cite{AB20}, and also retrieves the result of \cite{Byo04} in the case where the extension is Galois. In this situation, $N$ (as above) has order $pq$, but $G=\Gal(E/K)$ may instead be a much larger group (potentially of non-squarefree order), with $pq \mid |G|$. The problem now is that no classification of such groups currently exists. The idea presented in \cite{BML21}, which will be followed in this paper, is thus to instead consider each abstract group $N$ of order $pq$ in turn, and compute the transitive subgroups $G$ of $\Hol(N)$. This then gives the task of deciding when two such groups $G_1$ (arising from, say $N_1$) and $G_2$ (arising from, say $N_2$, possibly not isomorphic to $N_1$) are isomorphic as \emph{permutation} groups. If they are, then $G_1 \cong G_2$ corresponds to some field extension $L/K$ which will admit Hopf-Galois structures of both types $N_1$ and $N_2$. In this paper, we show that this approach is feasible in the more general case where $p$ and $q$ are now arbitrary distinct odd primes, and we obtain a complete classification of such cases. We note that, together with the result of \cite{CS20}, this completes the discussion for what happens on a separable extension of degree which is a product of any two primes.

\section{Preliminaries}\label{prelim}
 Let $L/K$ be a field extension and let $H$ be a $K$-Hopf algebra acting on $L$ with action $\cdot$. We say that $L$ becomes an $H$-module algebra if $\Delta(h) \cdot (x \otimes y) = \sum_{(h)}(h_{(1)} \cdot x) \otimes (h_{(2)} \cdot y)$ for all $h \in H$ and $x,y \in L$. Here $\Delta (h)=\sum_{(h)}h_{(1)}\otimes h_{(2)}$ is written in Sweedler notation. In such a situation, we say that $H$ gives a \textit{Hopf-Galois structure} on $L/K$ if the $K$-linear map $\theta: L \otimes H \rightarrow \text{End}(L)$ given by $\theta (x \otimes h)(y)=x(h \cdot y)$ is bijective. We further say that a separable extension $L/K$ is \textit{almost classically Galois} if $G'$ (as in the introduction) has a normal complement $C$ in $G$. Equivalently, there is a regular subgroup $C$ in $\text{Perm}(X)$ normalised by $G$ and contained in $G$. This class of extensions proves to be of particular interest to study, and appears many times within the literature. For example, they provide an example of a class of extensions for which the so-called Hopf-Galois correspondence is bijective. See Theorem 5.2 of \cite{GP87}.
 
 In the above situation, we may identify the group $G$ as a subgroup of $\text{Perm}(X)$ via its image in the left-translation map $\lambda:G \rightarrow \text{Perm}(X)$, $\lambda(g)(hG')=(gh)G'$. We thus refer to $G$ as a permutation group in this sense. We see that $G$ acts transitively on the $K$-linear embeddings of $L$ into $E$, and the stabiliser of the inclusion $L \hookrightarrow E$ is $G'$.

In 1996, a new method of finding Hopf-Galois structures was made explicit by Byott in \cite{Byo96}. Instead of looking at regular embeddings of groups in $\text{Perm}(X)$ as above, one may fix an abstract group $N$ of order $n=|L:K|$ and look at transitive subgroups $G$ of $\Hol(N) \cong N \rtimes \Aut(N)$, the \emph{holomorph} of $N$. This has the advantage of being less computationally expensive than searching directly within $\text{Perm}(X)$ as $\Hol(N)$ is, in general, a much smaller group. The paper also gives a counting formula we can use:
\begin{lemma}[\cite{Byo96}]\label{Byott_num_HGS}
	Let $G,G',N$ be as above, let $e(G,N)$ be the number of Hopf-Galois structures of type $N$ which realise $G$, and $e'(G,N)$ the number of subgroups $M$ of $\emph{Hol}(N)$ which are transitive on $N$ and isomorphic to $G$ via an isomorphism taking the stabiliser $M'$ of $1_N$ in $M$ to $G'$. Then
	\[e(G,N)=\frac{|\emph{Aut}(G,G')|}{|\emph{Aut}(N)|}e'(G,N),\]
	where
	\[\Aut(G,G')=\left\{\theta\in\Aut(G) \mid \theta(G')=G'\right\},\]
	the group of automorphisms $\theta$ of $G$ such that $\theta$ fixes the identity coset $1_GG'$ of $X=G/G'$.
\end{lemma}
The following two propositions, recalled from \cite{BML21}, Propositions 2.2 and 2.3 respectively, will be useful for our later computations
\begin{proposition}\label{ab-aut}
	Let $N$ be an abelian group such that $\Aut(N)$ is also abelian, and let
	$A$, $A'$ be subgroups of $\Aut(N)$. Consider the subgroups $M=N
	\rtimes A$ and $M'=N \rtimes A'$ of $\Hol(N)$. If
	there is an isomorphism $\phi: M \to M'$ with
	$\phi(N)=N$, then $M=M'$.
\end{proposition}
\begin{proposition}\label{rel-aut-char}
	Let $N$ be any group and $A$ a subgroup of $\Aut(N)$. Let $M$ be the
	subgroup $N \rtimes A$ of $Hol(N)$, and suppose that $N$ is characteristic in
	$M$. Then the group
	$$ \Aut(M,A) := \{ \theta \in \Aut(M) : \theta(A)=A\}  $$ 
	is isomorphic to the normaliser of $A$ in
	$\Aut(N)$. In particular, if $\Aut(N)$ is abelian then $\Aut(M,A)
	\cong \Aut(N)$.
\end{proposition}
We denote an element of $\Hol(N)$ by $[\eta,\alpha]$ where $\eta \in N$, $\alpha \in \Aut(N)$; for two elements $[\eta,\alpha],[\mu,\beta]\in \Hol(N)$. Their product is
\[[\eta,\alpha][\mu,\beta]=[\eta\alpha(\mu),\alpha\beta].\]
We write $\eta$ for $[\eta,\text{id}_N]$ and $\alpha$ for $[1_N,\alpha]$. %The same convention is used for working within more general semidirect products $P \rtimes Q$.

\section{Hopf-Galois Structures for General $pq$}\label{general_pq}
We now consider Hopf-Galois structures on separable field extensions $L/K$ of degree $pq$ with $p>q$ arbitrary distinct odd primes. Byott in \cite{Byo96} showed that if $p \not\equiv 1 \mod q$, then $L/K$ admits a unique Hopf-Galois structure, and in that case such a structure is almost classically Galois of cyclic type. Thus, we may assume for the rest of the paper that $p \equiv 1 \bmod{q}$. We can then assume the following prime factorisations for $p-1$ and $q-1$ respectively:
\begin{align*}
	&p-1=q^{e_0}\ell_1^{e_1} \cdots \ell_m^{e_m},\\
	&q-1=\ell_1^{f_1} \cdots \ell_m^{f_m},
\end{align*}
where $e_0 > 0$, $e_i,f_i \geq 0$, and $\text{max}\left\{e_i,f_i\right\}> 0$ for $1\leq i \leq m$. Thus we take into account the fact that $p-1$ and $q-1$ might share common factors, and that $q \mid p-1$.

There are two abstract groups of order $pq$, namely the cyclic group $C_{pq}$, and the metabelian group $C_p \rtimes C_q$ where $C_q$ acts on $C_p$ in some faithful way. We therefore split our work into looking at these two cases. In each case, we will compute all transitive subgroups $G$ of $\Hol(N)$. Thus $G$ will admit a Hopf-Galois structure of the corresponding type.

\subsection{Cyclic case}
Let $N$ be the cyclic group of order $pq$. The idea in this section is to divide the discussion with respect to each prime appearing in the factorisations. We therefore work with the presentation
\[N = \langle \sigma,\tau |\sigma^p=\tau^q=1,\sigma\tau=\tau\sigma \rangle.\]
We thus have
\[\Aut(N) \cong \Aut(\langle \sigma \rangle) \times \Aut(\langle \tau \rangle)\]
where the factors are cyclic of order $p-1$ and $q-1$ respectively. Then $\Aut(N)$ is generated by the following elements:
	\begin{align*}
	&\alpha \in \Aut(\langle \sigma \rangle) \text{ such that ord}(\alpha)=q^{e_0},\\
	&\alpha_i \in \Aut(\langle \sigma \rangle) \text{ such that ord}(\alpha_i)=\ell_i^{e_i},\\
	&\beta_i \in \Aut(\langle \tau \rangle) \text{ such that ord}(\beta_i)=\ell_i^{f_i},
\end{align*}
where $1\leq i \leq m$. We therefore obtain the decomposition
\[\Aut(N) \cong \langle \alpha \rangle \times \langle \alpha_1,\beta_1 \rangle \times \cdots \times \langle \alpha_m,\beta_m \rangle,\]
where the factors have coprime orders. In particular, $\Aut(N)$ is abelian. 


\begin{remark}
	The subgroups of $\langle \alpha \rangle$ are of the form \[\left\langle \alpha^{q^{e_0-c}} \right\rangle\]
	for $0 \leq c \leq e_0$; there are $e_0+1$ of these.
\end{remark}
For the following proposition, we make a slight abuse of notation by referring to $\alpha_i$ and $\beta_i$ as simply $\alpha$ and $\beta$ respectively, and $e_i$ and $f_i$ as simply $e$ and $f$ respectively. This is to avoid messy notation.
\begin{proposition}\label{cyclic_subgroups}
	The subgroups of $\langle \alpha , \beta \rangle$ are as follows:\\
	\[\begin{array}{lll}
		\emph{(i)} \left\langle \alpha^{\ell^{e-s}},\beta^{\ell^{f-r_2}} \right\rangle, & 0 \leq s \leq e, 0 \leq r_2 \leq f,\\
		\emph{(ii)} \left\langle \alpha^{n\ell^{e-s}}\beta^{\ell^{f-r_1}} \right \rangle, &1 \leq s \leq e, 1 \leq r_1 \leq f,\\
		\emph{(iii)} \left\langle \alpha^{n\ell^{e-s}}\beta^{\ell^{f-r_1}},\beta^{\ell^{f-r_2}} \right \rangle, &1 \leq r_2 < r_1 \leq f, 0< r_1-r_2 < s \leq e.
	\end{array}\]
In cases \emph{(ii)} and \emph{(iii)} we have $1 \leq n < \ell^{\min\{s,r_1\}}$ with $\ell \nmid n$.
\end{proposition}
\begin{proof}
	An arbitrary subgroup of $\langle \alpha, \beta \rangle$ is of the form $\langle \alpha^{a_1}\beta^{b_1}, \cdots, \alpha^{a_l}\beta^{b_l}  \rangle$ for some $l \geq 0$. We note that we have $\langle \alpha^{a_1}\beta^{b_1}, \cdots, \alpha^{a_l}\beta^{b_l}  \rangle = \langle \alpha^a\beta^b,\beta^c \rangle$ for some $a$, $b$ and $c$; this can be seen by looking at the matrix
	\[\begin{pmatrix}
		a_1	& b_1 \\ a_2 & b_2 \\ \vdots & \vdots \\ a_l & b_l
	\end{pmatrix}\]
	where each row represents a generator of the subgroup, and observing that the problem can be translated to finding a row echelon form of this matrix using elementary row operations. To do this, we can first assume that $a_1=\gcd(a_1, \cdots , a_l)$, and then $a_2= \cdots = a_l = 0$, redefining the $b_i$ as needed. Then we can assume that $b_2 = \gcd(b_2, \cdots , b_l)$, and then $b_3= \cdots = b_l =0$. We therefore obtain
	\[\begin{pmatrix}
		a_1	& b_1 \\ 0 & b_2 \\ 0 & 0 \\ \vdots & \vdots \\ 0 & 0
	\end{pmatrix}.\]
	We rewrite this as
	\[\begin{pmatrix}
		a	& b \\ 0 & c
	\end{pmatrix}.\]
	This precisely corresponds to the situation $\langle \alpha^a\beta^b,\beta^c \rangle$ we have above.
	
	We now write
	\begin{align*}
		&a=n\ell^{e-s},\\
		&b=m_1\ell^{f-r_1},\\
		&c=m_2\ell^{f-r_2},
	\end{align*}
	with $\ell \nmid nm_1m_2$. By replacing $b$ and $c$ with suitable powers, we may further assume, without loss of generality, that $m_1=m_2=1$. Groups of type (i) then correspond precisely with the cases $r_1=0$ or $s=0$, taking a suitable power of $a$ to get $n=1$. So now assume that $s,r_1 \geq 1$. We may further assume that $r_1>r_2$ as otherwise we may multiply the first generator by a suitable power of the second to retrieve this inequality.
	
	With a view to obtaining the groups of type (ii), assume that $r_2=0$. The corresponding matrix is therefore
	\[\begin{pmatrix}
		n\ell^{e-s}	& \ell^{f-r_1} \\ 0 & 0
	\end{pmatrix}.\]
	We ask which multiples $m$ of the top row of the above matrix correspond to generating the same group. That is, which multiples change $n$, but don't change $s,r_1$, and leave the coefficient of $\ell^{f-r_1}$ as $1$? To this end, we see that $m \equiv 1 \pmod{\ell^{r_1}}$. We further note that $m$ is defined modulo $\ell^{\min\{s,r_1\}}$, and so we have two cases:
	\begin{enumerate}[label=(\Roman*)]
		\item $s \leq r_1$,
		\item  $s > r_1$.
	\end{enumerate}
	Suppose first that $s \leq r_1$; then $m=1$, and so each choice of $n$ with $1 \leq n < \ell^s$ gives a distinct group. Recalling that $\ell \nmid n$, we therefore have $\varphi(\ell^s)$ such choices for $n$. Now suppose that $s > r_1$, and so $m=1+x\ell^{r_1}$ for integers $x$ such that $m<\ell^s$. So each choice of $n$ with $1 \leq n < \ell^{r_1}$ with $\ell\nmid n$ gives a distinct group, giving $\varphi(\ell^{r_1})$ choices for $n$. We may therefore combine cases (I) and (II) by saying that we have $\varphi(\ell^{\min\{s,r_1\}})$ choices for $n$. This then gives us the parameters for generating groups of type (ii).
	
	Groups of type (i) and of type (ii) are both cyclic. We now look at obtaining the non-cyclic subgroups of $\langle \alpha, \beta \rangle$. With this in mind, we may now additionally assume that $r_2>0$. The corresponding matrix here is
	\[\begin{pmatrix}
		n\ell^{e-s}	& \ell^{f-r_1} \\ 0 & \ell^{f-r_2}
	\end{pmatrix}.\]
	Note first that we have the same conditions on $n$ with respect to $s$ and $r_1$ as we did for groups of type (ii). Without loss of generality, we may assume that $r_2<r_1$, as otherwise, $\ell^{r_2-r_1}\cdot\ell^{f-r_2}=\ell^{f-r_1}$, and we may identify the resulting group as one of type (i). We must also have that no power of the first generator is equal to the second (otherwise obtaining a group of type (ii)). To this end, we must have $r_1-r_2<s$. We thus obtain $0<r_1-r_2<s$, noting that any choice of $r_2$ with this restriction does not affect the choices for $n$.
\end{proof}
\begin{proposition}
	Let $M:=\min\{e_i,f_i\}$. Then for each $1\leq i \leq m$, there are
	\[(e_i+1)(f_i+1)+\Sigma^{(1)}_i+\Sigma^{(2)}_i\]
	subgroups of $\langle \alpha_i,\beta_i \rangle$, where 
	\begin{align*}
		&\Sigma^{(1)}_i=(\ell_i^{M-1}-1)\left[2M+1-\frac{2}{1-\ell_i}\right]-2(M-1)\ell^{M-1}+(|f-e|+1)(\ell_i^M-1)\\
		&\Sigma^{(2)}_i=(|f-e|+2)\left[(M-1)\ell_i^M-\ell\frac{1-\ell_i^{M-1}}{1-\ell_i}\right]
	\end{align*}
	give a count for the number of subgroups of types (ii) and (ii) respectively. Note: we take $\Sigma_i^{(1)}=\Sigma_i^{(2)}=0$ if $M=0$.
\end{proposition}
\begin{proof}
	For convenience, we again drop the subscript $i$ throughout this proof. We count the subgroups in Proposition \ref{cyclic_subgroups}. It is clear to see that there are $(e+1)(f+1)$ subgroups of type (i). There are
	\[\sum_{r_1=1}^f\sum_{s=1}^e\varphi\left(\ell^{\min\{s,r_1\}}\right)\]
	subgroups of type (ii). This may be written as
	\[(2M+1)\sum_{r=1}^{M-1}\varphi(\ell^r)+(|f-e|+1)\sum_{r=1}^M\varphi(\ell^r)-2\sum_{r=1}^{M-1}r\varphi(\ell^r).\]
	Each of these sums telescope in some way, and the expression can be easily shown to evaluate to
	\[(2M+1)(\ell^{M-1}-1)+(|f-e|+1)(\ell^M-1)-2(M-1)\ell^{M-1}+2\frac{1-\ell^{M-1}}{1-\ell}.\]
		Counting subgroups of type (iii) requires us to make several ordered choices on the parameters $s,a,r_1$ and $n$, where $a:=r_1-r_2$. To begin with, we start by choosing a value for $2 \leq s \leq e$; given our choice of $s$, we may make a choice for $a$ such that $1 \leq a \leq s-1$. We are then restricted to choose an $r_1$ such that $1+a \leq r_1 \leq f$, and finally we have $\varphi(\ell^{\min\{s,r_1\}})$ choices for $n$. There are therefore
	\[\sum_{s=2}^e\sum_{a=1}^{s-1}\sum_{r_1=1+a}^f\varphi\left(\ell^{\min\{s,r_1\}}\right)\]
	subgroups of type (iii). This can be rewritten as
	\[(|f-e|+2)\sum_{r=2}^M(r-1)\varphi(\ell^r).\]
	This sum also telescopes, and evaluates to
	\[(|f-e|+2)\left[(M-1)\ell_i^M-\ell\frac{1-\ell_i^{M-1}}{1-\ell_i}\right].\]
	%There are
%	\[f\sum_{c=1}^e\varphi(\ell^c)\]
%	subgroups of type (ii), where $\varphi$ is the Euler totient function. The sum telescopes and evaluates to $\ell^e-1$. We obtain a count for the number of subgroups of type (iii) as follows: set $a=d_1-d_2$, so we have $d_1=d_2+a$ and $1 \leq a \leq c-1$. We thus have the following possibilities for $d_1$ and $d_2$:
%\[\begin{matrix}
%&a=1,	&1 \leq d_2 \leq f-1,	&2 \leq d_1 \leq f,\\
%&a=2,	&1 \leq d_2 \leq f-2,	&3 \leq d_1 \leq f,\\
%&\vdots	&\vdots						&\vdots\\
%&a=c-1,	&1 \leq d_2 \leq f-(c-1),	&c \leq d_1 \leq f
%\end{matrix}\]
%thus if $c \leq f$, the number of choices for $d_2$ is $(f-1)+(f-2)+ \cdots + (f-(c-1))= \frac{1}{2}f(f-1)-\frac{1}{2}(f-c)(f-c+1)$ (note that a choice for $d_2$ dictates the choice for $d_1$ given a choice for $a$). If, however, $c>f$, then there are clearly $(f-1)+(f-2)+ \cdots +2+1= \frac{1}{2}f(f-1)$ choices. We now take note that the expression $(f-\min\left\{f,c\right\})$ gives $(f-c)$ if $c_1 \leq f$ and $0$ if $c>f$ which allows us to combine these two cases. For each $c$, there are $\varphi(\ell^c)$ choices for $n$. We therefore arrive at
%\begin{equation}\tag{$\star$}
%	\left(\frac{1}{2}f(f-1)-\frac{1}{2}(f-\min\left\{f,c\right\})(f-c+1)\right)\varphi\left(\ell^c\right)
%\end{equation}
%for the number of such groups for a fixed $c$. The formula for $\Sigma_i$ is then obtained by summing $(\star)$ over $2 \leq c \leq e$.
\end{proof}
As $\Hol(N)$ contains a unique Hall $\left\{p,q\right\}$-subgroup $H=\langle \sigma, \tau, \alpha \rangle$ of order $pq^{e_0+1}$, and any transitive subgroup $M$ has order divisible by $pq$, it follows that $M \cap H$ must be transitive on $N$.
The subgroups of order divisible by $pq$ in $H$ are:
\begin{align*}
	& N \rtimes \left\langle \alpha^{q^{e_0-c}} \right \rangle, 0 \leq c \leq e_0,\\
	& J_{t,c}:=\left\langle \sigma,\left[\tau,\alpha^{tq^{e_0-c}}\right] \right\rangle, 1 \leq c \leq e_0, t \in \mathbb{Z}_{q^{e_0}}^{\times},\text{ and }\\
	& \left\langle \sigma, \alpha^{q^{e_0-c}} \right\rangle, 1 \leq c \leq e_0.
\end{align*}
It is clear that those groups containing $N$ or of the form $J_{t,c}$ are transitive on $N$. No group of the last type is transitive.

Now $N$ is normal in $\Hol(N)$, so can be extended by any subgroup of $\Aut(N)$ to give a transitive subgroup $M$. The normaliser of $J_{t,c}$ in $\Aut(N)$ is $\Aut(\langle \sigma \rangle) \cong \langle \alpha,\alpha_1, \cdots , \alpha_m \rangle$ since if $\phi \in \Aut(N)$ and $\phi(\tau)\neq \tau$, we have \[\phi [\tau,\alpha^{tq^{e_0-c}}]\phi^{-1}=[\phi(\tau),\alpha^{tq^{e_0-c}}] \notin J_{t,c}.\]
Hence if $M$ is a transitive subgroup containing $J_{t,c}$ but not $N$, then $M$ is $J_{t,c}$ extended by some subgroup of $\Aut(\langle \sigma \rangle)$. Note that $J_{t,c} \rtimes \langle \alpha^{q^{e_0-c'}} \rangle$ contains $N$ for $c'\geq c$, and is equal to $J_{t,c}$ if $c' < c$, so we take account of this below to ensure there are no repetitions in our list of transitive subgroups. The transitive subgroups of $\Hol(N)$ are therefore:
\begin{align}
	&N \rtimes \left \langle \alpha^{q^{e_0-c}},X_i \bigm | 1\leq i \leq m \right \rangle,\label{N}\\
	&J_{t,c} \rtimes \left \langle \alpha_i^{\ell_i^{e_i-c_i}} \bigm | 1 \leq i \leq m \right \rangle. \label{J_{t,c}}
\end{align}
	Here $X_i$ is one of the subgroups of $\langle \alpha_i , \beta_i \rangle$ outlined in Proposition \ref{cyclic_subgroups}. For groups of type (2), we have $1\leq c \leq e_0$, and in all other cases we have $0 \leq c_i \leq e_i$. In total, there are:
\[(1+e_0)\prod_{1\leq i \leq m}\left[(e_i+1)(f_i+1)+\Sigma^{(1)}_i+\Sigma^{(2)}_i\right]\]
groups of type (1).

For a given choice of $X_i$ and $c$, these groups have order $pq^{c+1}d$ where $d$ is some divisor of $(p-1)(q-1)$ coprime to $q$ coming from the orders of the $\alpha_i$ and $\beta_i$. For some factorisation $d=d_1d_2d_3$, they have abstract isomorphism type
\[ ((C_p \rtimes C_{q^cd_1}) \times (C_q \rtimes C_{d_2})) \rtimes C_{d_3}\]
For fixed $t$ and $c$, the number of groups of type $(2)$ is
\[\prod_{1 \leq i \leq m}(e_i+1).\]
For a given choice of the $c_i$, and for any $t$, such a group has order
\[pq^{c}\prod_{1 \leq i \leq m}\ell_i^{c_i}\]
note that there are $\varphi(q^c)$ such groups of this order. They are abstractly isomorphic to
\[ C_p \rtimes C_{q^{c}d}\]
where $d=\ell_1^{c_1}\cdots \ell_m^{c_m}$ is some divisor of $(p-1)$ coprime to $q$.

The stabiliser of $1_N$ in the above subgroups for which $c=1$ has normal complement (either $N$ or $J_{t,1}$), so all the corresponding field extensions to those groups are almost classically Galois. For $c>1$, the transitive subgroups of type $(2)$ do not contain a regular normal subgroup, and so the corresponding subgroups fail to be almost classically Galois.

\begin{lemma}\label{cyclic_isoms}
	Fix $1\leq c \leq e_0$. Then for each choice of subgroup $A\leq\langle \alpha_1,\cdots,\alpha_m\rangle$, the $\varphi(q^{c})$ groups given by $J_{t,c} \rtimes A$ for $1 \leq t <q^c$ with $q \nmid t$, are isomorphic. Between any two transitive subgroup of type (\ref{N}) or (\ref{J_{t,c}}), there are no other isomorphisms as permutation subgroups of $\Hol(N)$, even as abstract groups.
\end{lemma}
\begin{proof}
	Let $\phi \in \Aut(N)$ such that $\phi(\tau)=\tau^t$, then
	\[\phi [\tau,\alpha^{tq^{e_0-c}}]\phi^{-1}= [\phi(\tau),\alpha^{tq^{e_0-c}}]= [\tau,\alpha^{q^{e_0-c}}]^t.\]
	Thus conjugation by $\phi$ gives an isomorphism between $J_{t,c}$ and $J_{1,c}$, which extends to an isomorphism between $J_{t,c} \rtimes A$ and $J_{1,c} \rtimes A$ with $A<\langle \alpha_1,\cdots,\alpha_m \rangle$. Note that as conjugation by $\phi$ fixes any automorphism, and hence fixes the stabiliser of $1_N$, it also shows that they are isomorphic as permutation groups.
	Next, note that two groups $J_{1,c}\rtimes A$ and $J_{1,c'}\rtimes A'$ with $A,A'<\langle \alpha_1,\cdots,\alpha_m \rangle$ are isomorphic if and only if $c=c'$ and $A=A'$. This is clear when noting that choosing different values for $c$ or generators for $A$ gives a group of a different order.
	Finally, by Proposition \ref{ab-aut}, no two groups of type containing $N$ are isomorphic, nor can they be isomorphic to any group containing $J_{1,c}$ for any $c$ and $A<\langle \alpha_1, \cdots, \alpha_m \rangle$ because the groups $J_{1,c} \rtimes A$ do not contain an abelian subgroup of order $pq$.
\end{proof}

\begin{lemma}\label{cyclic_HGS_isom}
	The number of Hopf-Galois structures per isomorphism class of transitive subgroups of $\Hol(N)$ are as follows:
	\begin{enumerate}[label=\emph{(\roman*)}]
		\item $1$ for groups containing $N$ (each of these is in its own isomorphism class),
		\item $p$ for the groups $J_{t,1}$ (these form one isomorphism class),
		\item $q^{c-1}$ for the groups $J_{t,c}\rtimes A$ with either $c>1$ or $A$ nontrivial (the isomorphism classes are formed from groups with the same $c$ and $A$ but any $t$).
	\end{enumerate}
\end{lemma}
\begin{proof}
	(i) and (ii) follow as in Lemma $4.3$ of \cite{BML21}. So now look at $M:=J_{t,c}\rtimes A$ and suppose that either $c>1$ or $A$ is nontrivial. Note that, without loss of generality, we may assume that $t=1$. Suppose explicitly that $A=\langle \alpha_1^{b_1},\cdots,\alpha_m^{b_m}\rangle$, and let $M'$ denote the stabiliser of $1_N$ in $M$. If $\phi \in \Aut(M)$, then $\phi(\sigma)=\sigma^a$ for some $1 \leq a \leq p-1$, and
	\begin{align*}
		&\phi\left(\left[\tau,\alpha^{q^{e_0-c}}\right]\right)=\sigma^b\left[\tau,\alpha^{q^{e_0-c}}\right],
		&&\phi(\alpha_i^{b_i})=[\sigma^{a_i},\alpha_i^{b_i}]
	\end{align*}
	for some $0\leq b \leq p-1$ and $0\leq a_i < \ell_i$. The commutation relations between the given generators of $M$ give us
	\[b(1-a_{\alpha_i}^{b_i})=a_i(1-q^{e_0-c}) \;\; \text{ for all } 1\leq i \leq m.\]	
	We also have that
	\[M'=\left\langle \alpha^{q^{e_0-(c-1)}} \right\rangle \times A.\]
	For $\phi$ to fix $M'$, therefore, we must have $\phi(\alpha_i^{b_i})=\alpha_i^{b_i}$ and
	\[\phi\left(\alpha^{q^{e_0-(c-1)}} \right)=\left(\sigma^b\left[\tau,\alpha^{q^{e_0-c}}\right]\right)^q,\]
	which hold if and only if $a_i=b=0$. We therefore have that $|\Aut(M,M')|=p-1$. Noting that there are $\varphi(q^c)=q^{c-1}(q-1)$ transitive subgroups isomorphic to $M$, there are therefore
	\[\varphi(q^c)\frac{|\Aut(M,M')|}{|\Aut(N)|}=\varphi(q^c)\frac{p-1}{(p-1)(q-1)}=q^{c-1}\]
	Hopf-Galois structures of type $C_{pq}$ with group $M$.
\end{proof}

Summarising, we have:

\begin{theorem}\label{cyclic_HGS_total}
	There are
	\[(1+e_0)\prod_{1\leq i \leq m}\left[(e_i+1)(f_i+1)+\Sigma_i^{(1)}+\Sigma_i^{(2)}\right] + e_0\prod_{1 \leq i \leq m}(e_i+1)\]
	isomorphism types of permutation groups $G$ of degree $pq$ which are realised by a Hopf-Galois structure of cyclic type. These include the two regular groups, i.e. the
	cyclic and non-abelian groups of order $pq$ (for which the corresponding
	Galois extensions have $1$ and $p$ Hopf-Galois structures of cyclic type
	respectively). There are $(e_0-1)\prod_{1 \leq i \leq m}(e_i+1)$ extensions which fail to be almost classically Galois; these correspond to groups of type (2) with $c>1$. For all the remaining groups $G$, any field extension $L/K$
	realising $G$ is almost classically Galois and admits a unique Hopf-Galois
	structure of cyclic type.
\end{theorem}
Note the first and second summands count the number of isomorphism classes of groups of type $(1)$ and $(2)$ respectively.

\begin{remark}
	\emph{Setting $e_0=1$, $\ell_1=2$, $e_1=1$, $f_1=r$, $s=\ell_2^{f_2} \cdots \ell_m^{f_m}$, $e_i=0$ for $2 \leq i \leq m$, we get $\Sigma_1^{(1)}=r$, $\Sigma_i^{(1)}=0$ for $2 \leq i \leq m$ and $\Sigma_i^{(2)}=0$ for $1 \leq i \leq m$. We therefore retrieve the cyclic result of \cite{BML21}, that there are
		\[2(2(r+1)+r)\prod_{i=2}^m(f_i+1)=(6r+4)\sigma_0(s)\]
		isomorphism types of permutation groups $G$ of degree $pq$ (with $p=2q+1$) which are realised by a Hopf-Galois structure of cyclic type. (Here, $\sigma_0(s)$ counts the number of divisors of $s$.)}
\end{remark}



\subsection{Metabelian case}
Now let $N$ be the non-abelian group of order $pq$:
\[\langle \sigma,\tau | \sigma^p=\tau^q=1,\tau\sigma=\sigma^g\tau \rangle\]
where $g$ has order $q \bmod{p}$. Let $s=(p-1)/q^{e_0}$. Then, by equation $(3.7)$ in \cite{Byo04}, $\Aut(N)$ has order $p(p-1)=pq^{e_0}s$, and is generated by automorphisms $\alpha$, $\beta$, $\epsilon$ of orders $q^{e_0}$, $s$, $p$ respectively, where:
\begin{align*}
	&\alpha(\sigma)=\sigma^{a_{\alpha}}, &&\alpha(\tau)=\tau,\\
	&\beta(\sigma)=\sigma^{a_{\beta}},	&&\beta(\tau)=\tau,\\
	&\epsilon(\sigma)=\sigma,			&&\epsilon(\tau)=\sigma\tau.
\end{align*}
We have $\text{ord}_p(a_{\alpha})=q^{e_0}$ and $\text{ord}_p(a_{\beta})=s$. Without loss of generality, we can assume that $g=a_{\alpha}^{q^{e_0-1}}$. The generators satisfy the following relations:
\[\begin{array}{lcr}
	\beta\alpha=\alpha\beta,	&\alpha\epsilon=\epsilon^{a_{\alpha}}\alpha,	&\beta\epsilon=\epsilon^{a_{\beta}}\beta.
\end{array}\]
So $\Hol(N)=\langle \sigma, \tau, \alpha, \beta, \epsilon \rangle$ is a group of order $sp^2q^{1+e_0}$. Using the idea in \cite{BML21}, we first note that $P:= \langle \sigma, \epsilon \rangle \cong C_p \times C_p$ is the unique Sylow $p$-subgroup of $\Hol(N)$, with complementary subgroup $R:= \langle \tau, \alpha, \beta \rangle \cong C_q \times C_{q^{e_0}} \times C_s$. Thus $\Hol(N)\cong P \rtimes R$. We see that the element $\sigma\epsilon^{g-1} \in P$ commutes with $\tau$:
\[\sigma\epsilon^{g-1}\tau=\sigma\epsilon^{g-1}(\tau)\epsilon^{g-1}=\sigma^g\tau\epsilon^{g-1}=\tau\sigma\epsilon^{g-1}.\]
Writing $P$ additively, we can identify it with the vector space $\mathbb{F}_p^2$, with basis vectors
\begin{align*}
	&\e_1 = \begin{pmatrix} 1\\0	\end{pmatrix},
	&&\e_2 = \begin{pmatrix} 0\\1	\end{pmatrix},
\end{align*}
corresponding to $\sigma$, $\sigma\epsilon^{g-1}$ respectively. The generators $\tau$, $\alpha$, $\beta$ of $R$ can then be identified with the matrices
\[\begin{array}{ccc}
	T=\begin{pmatrix} g & 0 \\ 0 & 1 \end{pmatrix}, &A=\begin{pmatrix} a_{\alpha} & 0 \\ 0 & a_{\alpha} \end{pmatrix}, &B=\begin{pmatrix} a_{\beta} & 0 \\ 0 & a_{\beta} \end{pmatrix}
\end{array}\]
respectively. An element of $\Hol(N)$ is thus written as $[\textbf{v},U]$ with $\textbf{v} \in \mathbb{F}_p^2$ and $U$ a matrix corresponding to an element of $R$.
We now determine conditions for a subgroup of $\Hol(N)$ to be transitive on $N$:
\begin{lemma}\label{metab_trans_conditions}
	A subgroup $M$ of $\Hol(N)$ is transitive on $N$ if and only if it satisfies the following two conditions:
	\begin{enumerate}[label=\emph{(\roman*)}]
		\item the image of $M$ under the quotient map $\Hol(N) \rightarrow R$ is one of
		\begin{itemize}
			\item $\left\langle TA^{uq^{e_0-c}},B^{s/d} \right \rangle$, $0 \leq c \leq e_0$, $d|s$, $u \in \mathbb{Z}_{q^c}^{\times}$,
			\item $\left\langle T,A^{q^{e_0-c}},B^{s/d} \right \rangle$, $1 \leq c \leq e_0$, $d|s$.
		\end{itemize}
		\item $M \cap P$ is one of $\mathbb{F}_p^2$, $\mathbb{F}_p\e_1$, $\mathbb{F}_p\e_2$, each of which is normalised by $R$.
	\end{enumerate}
\end{lemma}
\begin{proof}
	Suppose that $M$ is transitive. Then the orbit of $1_N$ under $M \cap P$ must have size $p$, so the projection of $M$ into $R$ cannot be contained in $\Aut(N)$, and must include $T$ in some way. Hence we obtain one of the groups listed in (i).
	
	If $P \nsubseteq M$ then $M \cap P$ has order $p$, and $M$ contains an element of the form $[\textbf{v},TA^aB^b]$ for some $\textbf{v}\in P$, and $a$ and $b$ in their respective ranges. Thus $M \cap P= \mathbb{F}_p\e_1$ or $\mathbb{F}_p(\lambda\e_1+\e_2)$ for some $\lambda \in \mathbb{F}_p$. In the latter case, we have
	\[TA^aB^b(\lambda\e_1+\e_2)=a_{\alpha}^aa_{\beta}^b(g\lambda\e_1+\e_2).\]
	The vector $g\lambda\e_1+\e_2$ does not lie in $M \cap P$ unless $\lambda=0$. Thus $M \cap P$ is one of the subgroups listed in (ii). It is clear that these are all normalised by $R$.
	
	Conversely, it is easy to check that if $M$ satisfies (i) and (ii), then $M$ is transitive.
\end{proof}

\begin{corollary}\label{metab_p^2q_groups}
	The transitive subgroups containing $P$ are:
	\begin{enumerate}[label=\emph{(\roman*)}]
		\item $P \rtimes \left \langle TA^{uq^{e_0-c}},B^{s/d} \right \rangle$, $0 \leq c \leq e_0$, $u \in \mathbb{Z}_{q^c}^{\times}$, $d|s$. These groups have order $dp^2q^{\max\left\{1,c\right\}}$.
		\item $P \rtimes \left \langle T, A^{q^{e_0-c}},B^{s/d} \right \rangle$, $1 \leq c \leq e_0$, $d|s$. These groups have order $dp^2q^{1+c}$.
	\end{enumerate}
\end{corollary}
\begin{corollary}\label{metab_notP}
	The transitive subgroups not containing $P$ are generated by the set
	\begin{enumerate}[label=\emph{(\Roman*)}]
		\item $\left\{\e_i,[\lambda\e_{3-i},TA^{uq^{e_0-c}}],[\mu\e_{3-i},B^{s/d}]\right\}$, or the set
		\item $\left\{\e_i,[\lambda\e_{3-i},T],[\mu\e_{3-i},A^{q^{e_0-c}}],[\nu\e_{3-i},B^{s/d}]\right\}$
	\end{enumerate}
for choices of $i \in \left\{1,2\right\}$, $0\leq c \leq e_0$, $u \in \mathbb{Z}_{q^c}^{\times}$, $d \mid s$, and $\lambda, \mu,\nu \in \mathbb{F}_p$.
\end{corollary}

%We now define
%\[S(m,j):=\sum_{i=0}^{j-1}m^i=
%\begin{cases*}
%	j  \text{ if }m=1,\\
%	\frac{1-m^j}{1-m} \text{ otherwise.}
%\end{cases*}\]
%The equality $m=1$ may be taken as a congruence in the relevant cases.

\begin{proposition}
	Table \ref{metabelian-trans-subgroups} lists the transitive subgroups of $\Hol(N)$ for $N$ metabelian of order $pq$.
\end{proposition}
\begin{proof}
	The first two rows in the table are given by Corollary \ref{metab_p^2q_groups}. We now use Corollary \ref{metab_notP} and consider just the case $i=1$. The case $i=2$ is similar. Therefore let $M$ be a transitive subgroup generated by elements in the set given by (I), and consider the generators
	\[x:=[\lambda\e_2,TA^{uq^{e_0-c}}] \;\; \text{ and } \;\; y:=[\mu\e_2,B^{s/d}].\]
	We have, for $c>0$, that $x^{q^c}=0$, but for $c=0$, $x^q=q\lambda\e_2$. Given that it is assumed that $M$ does not contain $P$, we must therefore have $\lambda=0$ in this case. Via the projection map into $R$, we see that $x$ must commute with $y$, giving us the relation
	\[\left(1-a_{\alpha}^{uq^{e_0-c}}\right)\mu=\left(1-a_{\beta}^{s/d}\right)\lambda.\]
	Note that $\left(1-a_{\alpha}^{uq^{e_0-c}}\right)=0$ if and only if $c=0$. With this, we obtain the third and fourth rows in the table.
	
	Now let $M$ be a transitive subgroup generated by elements in the set given by (II), and consider the generators
	\[x:=\left[\lambda\e_2,T\right], \;\; y:=\left[\mu\e_2,A^{q^{e_0-c}}\right], \;\; z:=\left[\nu\e_2,B^{s/d}\right].\]
	With a similar argument to the above, we determine that $\lambda=0$. Again by looking at the projection onto $R$, we obtain the commutation relation
	\[\left(1-a_{\alpha}^{q^{e_0-c}}\right)\nu=\left(1-a_{\beta}^{s/d}\right)\mu.\]
	Note that $\left(1-a_{\alpha}^{q^{e_0-c}}\right)=0$ if and only if $c=0$. We obtain the fifth row in the table.
	
	Similar discussions for $i=2$ give the final three rows.
\end{proof}

\begin{sidewaystable}
	\setlength{\extrarowheight}{3.5mm}
	\text{ .}
	\vskip135mm
	\bigskip
	
	\centering
	\scalebox{1.1}{
	\begin{tabular}{|c|c|}
		\hline
		Parameters	&	Group\\
		\hline
		$0\leq c \leq e_0$,  $d|s$			&	$P \rtimes \left\langle T,A^{q^{e_0-c}},B^{s/d} \right\rangle$\\[5pt]
		\hline
		$1\leq c\leq e_0$,  $d|s$, $u \in \mathbb{Z}_{q^c}^{\times}$&	$P \rtimes \left\langle TA^{uq^{e_0-c}},B^{s/d} \right\rangle$\\[5pt]
		\hline
		$0 \leq \lambda \leq p-1$,  $d|s$			&	$\left\langle \e_1,T,\left[\lambda\e_2,B^{s/d}\right] \right\rangle$\\[5pt]
		\hline
		$1 \leq c \leq e_0,0\leq \lambda \leq p-1$,  $d|s$, $u \in \mathbb{Z}_{q^c}^{\times}$			&	$\left\langle \e_1, \left[\lambda\e_2,TA^{uq^{e_0-c}}\right],\left[(1-a_{\beta}^{s/d})(1-a_{\alpha}^{uq^{e_0-c}})^{-1}\lambda\e_2,B^{s/d}\right] \right\rangle$\\[5pt]
		\hline
		$1 \leq c \leq e_0,0\leq \lambda \leq p-1$,  $d|s$			&	$\left\langle \e_1,T,\left[\lambda\e_2,A^{q^{e_0-c}}\right],\left[(1-a_{\beta}^{s/d})(1-a_{\alpha}^{q^{e_0-c}})^{-1}\lambda\e_2,B^{s/d}\right] \right\rangle$\\[5pt]
		\hline
		$0 \leq \lambda \leq p-1$,  $d|s$			&	$\left\langle \e_2,TA^{-q^{e_0-1}},\left[\lambda\e_1,B^{s/d}\right] \right \rangle$\\[5pt]
		\hline
		$0 \leq c \leq e_0,0\leq \lambda \leq p-1$,  $d|s$, $u \in \mathbb{Z}_{q^c}^{\times}$			&	$\left\langle \e_2,\left[\lambda\e_1,TA^{uq^{e_0-c}}\right],\left[(1-a_{\beta}^{s/d})(1-ga_{\alpha}^{uq^{e_0-c}})^{-1}\lambda\e_1,B^{s/d}\right] \right\rangle$\\[5pt]
		\hline	
		$1 \leq c \leq e_0,0\leq \lambda \leq p-1$,  $d|s$			&	$\left\langle \e_2,\left[\lambda\e_1,T\right],\left[(1-a_{\alpha}^{q^{e_0-c}})(1-g)^{-1}\lambda\e_1,A^{q^{e_0-c}}\right],\left[(1-a_{\beta}^{s/d})(1-g)^{-1}\lambda\e_1,B^{s/d}\right] \right\rangle$ \phantom{shdva} \\[5pt]
		\hline
	\end{tabular}}
	
	\vskip5mm
	
	\caption{Transitive subgroups for $N$ metabelian}
	\label{metabelian-trans-subgroups}
\end{sidewaystable}

We again denote by $M'$ the stabiliser of $1_N$ in $M$. In order to count the number of Hopf-Galois structures of $C_p\rtimes C_q$ type, we now we move on to compute $|\Aut(M,M')|$ for each of the subgroups $M$ given in Table \ref{metabelian-trans-subgroups}. We note that $M'=M \cap \Aut(N)$, and that $\Aut(N)=\langle \textbf{f},A,B \rangle$, where $\textbf{f}=\e_1-\e_2$ is the element of $\mathbb{F}_p^2$ corresponding to $\epsilon \in \Aut(N)$.

\begin{proposition}\label{metab_isoms}
	For a fixed $c$ and $d$ and row in Table \ref{metabelian-trans-subgroups}, the groups parametrised by $0 \leq \lambda \leq p-1$ are all isomorphic as permutation groups. Furthermore, every group in the last three rows of Table \ref{metabelian-trans-subgroups} is isomorphic as a permutation group to a group in one of rows three to five. Within rows three to eight, there are no more isomorphisms, even as abstract groups.
\end{proposition}
\begin{proof}
	One first sees that any group in rows three, four and five for a fixed $c$ and $d$ (where appropriate) is conjugate, via some $\mu\e_2$, to the group in the same row and with the same fixed $c$ and $d$, but with $\lambda=0$. Likewise for the groups in rows six, seven and eight, but via some $\mu\e_1$. This gives an isomorphism as permutation groups, and we may therefore assume that $\lambda=0$ for the rest of the paper, noting that the group represents a class of size $p$. One may then observe that the following are isomorphic as permutation groups via relatively natural isomorphisms:
	\begin{align*}
		& \langle \e_1,T,B^{s/d} \rangle \cong \langle \e_2,TA^{-q^{e_0-1}},B^{s/d} \rangle,\\
		& \langle \e_1,T,B^{s/d} \rangle \cong \langle \e_1,TA^{uq^{e_0-1}},B^{s/d} \rangle, \text{ for } u \neq -1,\\
		& \langle \e_1,T,B^{s/d} \rangle \cong \langle \e_2,TA^{uq^{e_0-1}}, B^{s/d} \rangle,\\
		& \langle \e_1,TA^{uq^{e_0-c}},B^{s/d} \rangle \cong \langle \e_2,TA^{vq^{e_0-c'}}, B^{s/d} \rangle, \text{ where } (u,c)=(-1,1) \text{ if } c'=0,\\
		& \langle \e_1,T,A^{q^{e_0-c}},B^{s/d} \rangle \cong \langle \e_2,T,A^{e^{e_0-c}},B^{s/d} \rangle.
	\end{align*}
	Given any choice of $(c,c',u) \neq (0,0,u),(c,1,-1)$, since the group $\langle \e_1,T,A^{q^{e_0-c}},B^{s/d} \rangle$ contains an abelian subgroup of order $pq$, whereas the group $\langle \e_1,TA^{uq^{e_0-c'}},B^{s/d} \rangle$ does not, they cannot be isomorphic as abstract groups. The rest of the claims should be easy to see.
\end{proof}
In light of Proposition \ref{metab_isoms}, we need only separately consider the transitive subgroups of order divisible by $p^2$ and the groups in Table \ref{metabelian-trans-subgroups} containing $\e_1$ with $\lambda=0$.

\begin{proposition}\label{metab_holgp_isoms}
	The groups $M=P \rtimes \langle T,A^{q^{e_0-c}},B^{s/d} \rangle$ have $|\Aut(M,M')|=2p(p-1)$ for $c>0$ or $d>1$. The group $M=P \rtimes \langle T \rangle$ has $|\Aut(M,M')|=p(p-1)$. No two such groups are isomorphic either as abstract groups or as permutation groups.
\end{proposition}

\begin{proof}
	It is clear that no two such groups are isomorphic as abstract groups because a different choice of $c$ and $d$ gives a group of a different order.
	
	Next, we see that $M':=M \cap \Aut(N)= \langle \textbf{f},A^{q^{e_0-c}},B^{s/d} \rangle$.
	
	Let $\theta \in \Aut(M,M')$. As $M$ contains exactly two normal subgroups, namely $\langle\e_1\rangle$ and $\langle\e_2\rangle$, we must have that either $\theta(\e_i)=xe_i$ or $\theta(\e_i)=x\e_{3-i}$ for $i \in \{1,2\}$ and $1\leq x \leq p-1$.
	
	We consider separately the case that $c=0$ and $d=1$. In this case, we have $M= P \rtimes \langle T \rangle \cong C_p \times (C_p \rtimes C_q)$ with $M'=\langle \textbf{f} \rangle$. If $\theta(\e_i)=x\e_i$, we must have $\theta(T)=[y\e_1,T]$ with $y \in \mathbb{F}_p$. If $\theta(\e_i)=x\e_{3-i}$, then the commutation relations between $T$ and $\e_i$ mean that there is no choice for $\theta(T)$. We therefore only obtain $|\Aut(M,M')|=p(p-1)$ in this case. For the rest of the proof, we assume that $c>0$ or $d>1$
	
	As $\theta$ fixes $M'$, we have the following:
	\[\theta(A^{q^{e_0-c}})=\left[\mu\textbf{f},A^{q^{e_0-c}a}\right], \;\; \theta(B^{s/d})=[\nu\textbf{f},B^{bs/d}]\]
	for some $1\leq a \leq q^c-1$ such that $q \nmid a$ and $b$ is coprime with $d$. We must also have
	\[\theta(T)=[\textbf{v},T^jA^{q^{e_0-1}k}]\]
	for some $v \in \mathbb{F}_p^2$, $1\leq j \leq q-1$ and $0\leq k \leq q-1$.
	
	Suppose first that $\theta(\e_i)=xe_i$. Checking commutation relations between $A^{q^{e_0-c}}$, and $\e_i$ gives $a=1$. Checking commutation relations between $T$ and $\e_i$ gives $j=1$ and $k=0$, and the commutation relations between $B^{s/d}$ and $\e_i$ gives $b=1$. Given further that $R$ is abelian, we have the following relations:
	\begin{align*}
		&\left(1-a_{\alpha}^{q^{e_0-c}}\right)\textbf{v}=\mu\left(I-T\right)\textbf{f},\\
		&\left(1-a_{\alpha}^{q^{e_0-c}}\right)\nu=\left(1-a_{\beta}^{s/d}\right)\mu, \text{ and }\\
		&\left(1-a_{\beta}^{s/d}\right)\textbf{v}=\nu\left(I-T\right)\textbf{f},
	\end{align*}
	where $I$ is the identity matrix. We therefore have	
	\begin{align*}
		&\theta(\e_i)=x\e_i,	&&\theta(A^{q^{e_0-c}})=\left[\mu\textbf{f},A^{q^{e_0-c}}\right],\\	&\theta(T)=\left[\textbf{v},T\right],	&&\theta(B^{s/d})=\left[\nu\textbf{f},B^{s/d}\right],
	\end{align*}
	with $\textbf{v}$ determined entirely by $\mu$ (note that this means that $\textbf{v}=y\e_1$ for some $y\in \mathbb{F}_p$), which is in turn determined by $\nu$. 
	In total, this gives $p(p-1)$ choices for~$\theta$.
	
	Suppose now that $\theta(\e_i)=x\e_{3-i}$. Similar computations give:
	\begin{align*}
		&\theta(\e_i)=x\e_{\pi(i)},	&&\theta(A^{q^{e_0-c}})=\left[\lambda\textbf{f},A^{q^{e_0-c}}\right],\\	&\theta(T)=\left[\textbf{v},T^{-1}A^{q^{e_0-1}}\right],	&&\theta(B^{s/d})=\left[\mu\textbf{f},B^{s/d}\right],
	\end{align*}
	where $\left(1-a_{\alpha}^{q^{e_0-c}}\right)\textbf{v}=(1-g)\lambda\e_2$ and $\left(1-a_{\alpha}^{q^{e_0-c}}\right)\mu=\left(1-a_{\beta}^{s/d}\right)\lambda$. This again gives $p(p-1)$ choices for $\theta$.
	
	In total, we have $|\Aut(M,M')|=2p(p-1)$.
\end{proof}

\begin{proposition}
	Let $M_{u,c} :=P \rtimes \langle TA^{uq^{e_0-c}}\rangle$ and $\widehat{M}_{u,c}:=P \rtimes \langle TA^{uq^{e_0-c}},B^{s/d}\rangle$ where $1 \leq c \leq e_0$ and $u \in \mathbb{Z}_{q^c}^{\times}$ and $d\mid s$. Then $M_{u,c}$ and $M_{u,c'}$ are isomorphic as permutation groups if and only if $c=c'$ and either
	\begin{enumerate}[label=\emph{(\Roman*)}]
		\item $u \equiv u' \pmod{q}$,
		\item $u \equiv -u' \pmod{q}$ and $c>1$, or
		\item $u+u'+1 \equiv 0 \pmod{q}$ and $c=1$.
	\end{enumerate}
	Hence the groups $M_{u,c}$ fall into  $(q-3)/2$ isomorphism classes for $c=1,u\neq -1$, and $(q-1)/2$ isomorphism classes for $c>1$. The group $M_{-1,1}$ is isomorphic to $P \rtimes \langle T \rangle$. Similarly for the groups $\widehat{M}_{u,c}$, with $\widehat{M}_{-1,1}$ being isomorphic to $P \rtimes \langle T,B^{s/d} \rangle$. Moreover
	\begin{align*}
	&|\Aut(M_{u,1},M'_{u,1})|= p^2(p-1), \; u\neq q-1,(q-2)/2,\\
	&|\Aut(M_{-1,1},M'_{-1,1})|= p(p-1),\\
	&|\Aut(M_{(q-1)/2,1},M'_{(q-1)/2,1})|= 2p^2(p-1),\\
	&|\Aut(\widehat{M}_{u,1},\widehat{M}'_{u,1})|= p(p-1), \; u\neq q-1,(q-2)/2,\\
	&|\Aut(\widehat{M}_{-1,1},\widehat{M}'_{-1,1})|= p(p-1),\\
	&|\Aut(\widehat{M}_{(q-1)/2,1},\widehat{M}'_{(q-1)/2,1})|= 2p(p-1),\\
	&|\Aut(M_{u,c},M'_{u,c})|=|\Aut(\widehat{M}_{u,c},\widehat{M}'_{u,c})|=p(p-1) \;\; \text{ for } c>1.
	\end{align*}
	
\end{proposition}
\begin{proof}
	Consider first the groups $M_{u,c}=P\rtimes \left\langle TA^{uq^{e_0-c}} \right\rangle$. We start by asking when these groups are isomorphic as permutation groups for different choices of $u$ and $c$. Note that any element of $M_{u,c}$ of order $q^c$ has the form $\left[\textbf{v},\left(TA^{uq^{e_0-c}}\right)^x\right]$ for some $\textbf{v}\in P$ and $x \in \mathbb{Z}_{q^c}^{\times}$, and acts on $P$ with two distinct eigenvalues, namely $g^xa_{\alpha}^{xuq^{e_0-c}}$ and $a_{\alpha}^{xuq^{e_0-c}}$. If, therefore, we have a choice of $(u,c)$ and $(u',c')$ such that $M_{u,c}\cong M_{u',c'}$, then we must have
	\[\left\{g^xa_{\alpha}^{xuq^{e_0-c}},a_{\alpha}^{xuq^{e_0-c}}\right\}=\left\{ga_{\alpha}^{u'q^{e_0-c'}},a_{\alpha}^{u'q^{e_0-c'}}\right\}\]
	for some $x$. This is equivalent to solving the following equations $\bmod{ \;q^e}$:
	\begin{align*}
		&x(q^{e_0-1}+uq^{e_0-c})=q^{e_0-1}+u'q^{e_0-c'}, &&\text{ and } \;\;\;
		xuq^{e_0-c}=u'q^{e_0-c'}, \text{ or}\\
		&x(q^{e_0-1}+uq^{e_0-c})=u'q^{e_0-c'}, &&\text{ and } \;\;\;
		xuq^{e_0-c}=q^{e_0-1}+u'q^{e_0-c'}.
	\end{align*}
	The case for $\widehat{M}_{u,c}$ is treated similarly. This yields the first part of the Proposition. Note that, when $u\equiv u'\pmod{q}$, we must have $x \equiv 1\pmod{q}$, and when $u\equiv-u'\pmod{q}$, we must have $x \equiv -1 \pmod{q}$.
	
	We now look at computing $|\Aut(M,M')|$. We treat the cases for $M=M_{u,c}$ and $M=\widehat{M}_{u,c}$ together by setting $d=1$ and $d>1$ respectively in the following argument. We have $M'=\langle \textbf{f},A^{uq^{e_0-(c-1)}},B^{s/d} \rangle$. As in Proposition \ref{metab_holgp_isoms}, for $\theta \in \Aut(M,M')$, we have either $\theta(\e_i)=x\e_i$ or $\theta(\e_i)=x\e_{3-i}$ for some $1\leq x \leq p-1$. Further, we have
	\begin{align*}
		&\theta\left(TA^{uq^{e_0-c}}\right)=\left[\textbf{v},T^aA^{auq^{e_0-c}}\right],\\
		&\theta\left(A^{uq^{e_0-(c-1)}}\right)=\left[\mu\textbf{f},A^{auq^{e_0-(c-1)}}\right], \text{ whenever }c>1,\\
		&\theta\left(B^{s/d}\right)=\left[\nu\textbf{f},B^{bs/d}\right],
	\end{align*}
	for some $a \in \mathbb{Z}_{q^c}^{\times}$, $b$ coprime to $d$ and $0\leq \mu,\nu \leq p-1$. Considering the commutation relations between $B^{s/d}$ and the $\e_i$, we obtain $c=1$. There are now a number of possibilities, determined by the action of $\theta$ on $\e_i$, whether $c=1$ or $c>1$, the value of $u$, and whether $d=1$ or $d>1$. For clarity, suppose $\textbf{v}=m\e_1+n\e_2$.
	
	First, let us treat the case $c=1$ with $\theta(\e_i)=x\e_i$. In this case, we see that $a=1$ given the action of $TA^{uq^{e_0-1}}$ on $\e_i$. If $u=-1$, we must have $m=0$ in order for $\left[\textbf{v},TA^{-q^{e_0-1}}\right]$ to have order $q$. Other values of $u$ give no restrictions on $\textbf{v}$. If $d>1$, then as $B^{s/d}$ and $TA^{uq^{e_0-c}}$ commute, we have
	\begin{align}
		\label{beta_reltions}
		\begin{split}
		&m(1-a_{\beta}^{s/d})=\nu(1-g^{1+u}),\\
		&n(1-a_{\beta}^{s/d})=\nu(-1+g^u).
		\end{split}
	\end{align}
	If $d=1$, there are no further restrictions.
	
	Now consider $c=1$ with $\theta(\e_i)=x\e_{3-i}$. In this case, we must have $g^{a(1+u)}=g^u$ and $g^{au}=g^{1+u}$, which is only possible when $a=-1$ and $u=(q-1)/2$. If $d>1$, then we once again obtain (\ref{beta_reltions}), otherwise there are no further restrictions. This then retrieves the first six counts of $|\Aut(M,M')|$ given in the statement of the proposition, corresponding to $c=1$.
	
	Now suppose $c>1$. Note firstly that $\left[\textbf{v},T^aA^{auq^{e_0-c}}\right]$ has order $q^c$ regardless of the choice of $\textbf{v}$. Secondly, given the added requirement that $\theta\left(A^{uq^{e_0-(c-1)}}\right)=\left[\mu\textbf{f},A^{auq^{e_0-(c-1)}}\right]$, a choice for one of $m,n$, and $\mu$ automatically determines a choice for the other two. 
	
	Suppose that $\theta(\e_i)=x\e_i$. Then we again have $a=1$. If $d=1$, there are no further relations. If $d>1$, we have
	\begin{align}
	\label{beta_reltions2}
	\begin{split}
		&m(1-a_{\beta}^{s/d})=\nu(1-ga_{\alpha}^{uq^{e_0-c}}),\\
		&n(1-a_{\beta}^{s/d})=\nu(-1+a_{\alpha}^{uq^{e_0-c}}).
	\end{split}
	\end{align}
	Thus a choice of one of either $m,n$ or $\nu$ determines the value for the other two (and also $\mu$ as above).
	
	Finally, suppose that $\theta(\e_i)=x\e_{3-i}$. In this case, we must have $a_{\alpha}^{auq^{e_0-c}}=ga_{\alpha}^{uq^{e_0-c}}$ and $g^aa_{\alpha}^{auq^{e_0-c}}=a_{\alpha}^{uq^{e_0-c}}$. This system has no solution for $c>1$, and so there are no such elements of $|\Aut(M,M')|$ satisfying this property.
	
	We therefore have:
		\[|\Aut(M_{u,c},M'_{u,c})|=|\Aut(\widehat{M}_{u,c},\widehat{M}'_{u,c})|=p(p-1) \;\; \text{ for } c>1.\]
	
	
	
	%Noting that $\left(TA^{uq^{e_0-c}}\right)^q=A^{uq^{e_0-(c-1)}}\in\Aut(N)$, we have that $M'=\langle \textbf{f},A^{uq^{e_0-(c-1)}},B^{s/d} \rangle$. Just as before, we have 
	%$\theta(\e_i)=x\e_{\pi(i)}$ where $\pi$ is either the identity map or transposition map on $\left\{1,2\right\}$. We also have that $\theta(TA^{uq^{e_0-c}})=[\textbf{v},T^aA^{auq^{e_0-c}}]$ for some $\textbf{v}\in \mathbb{F}_p^2$, with the added restriction that $\theta(A^{uq^{e_0-(c-1)}})=[\lambda\textbf{f},A^{buq^{e_0-(c-1)}}]$ for $c > 1$, and we have $\theta(B^{s/d})=[\mu\textbf{f},B^{cs/d}]$. We first check commutation relations between $TA^{uq^{e_0-c}}$ and $\e_i$ through $\theta$; if $\pi$ is the identity map, we get $ga_{\alpha}^{uq^{e_0-c}}=g^aa_{\alpha}^{auq^{e_0-c}}$ and $a_{\alpha}^{uq^{e_0-c}}=a_{\alpha}^{auq^{e_0-c}}$, giving us $a=1$, and if $\pi$ is the transposition map, we get $a_{\alpha}^{uq^{e_0-c}}=g^aa_{\alpha}^{auq^{e_0-c}}$ and $ga_{\alpha}^{uq^{e_0-c}}=a_{\alpha}^{auq^{e_0-c}}$, giving us the exceptional case $a=-1$ with  $u=(q^{c}-1)/2$. Suppose now that $d=1$, that is there is no generator of order $d$, then the condition that $\theta(A^{uq^{e_0-(c-1)}})=[\lambda\textbf{f},A^{buq^{e_0-(c-1)}}]$, for $\textbf{v}=m\e_1+n\e_2$ gives us that
	%\[m=-U\left(ga_{\alpha}^{uq^{e_0-c}},a_{\alpha}^{uq^{e_0-c}}\right)n=U\left(ga_{\alpha}^{uq^{e_0-c}},a_{\alpha}^{uq^{e_0-(c-1)}}\right)\lambda\]
	%whenever $c \neq 1$. Otherwise we obtain no further restrictions on $\textbf{v}$. So at this point, we have, where $c>1$,
	%\[
	%\begin{matrix*}[l]
	%	\theta(\e_i)=x\e_{i},	&\theta(T)=[\textbf{v},T],	&\text{with no restrictions on }\textbf{v},\\
	%	\theta(\e_i)=x\e_{\pi(i)},	&\theta(TA^{uq^{e_0-1}})=[\textbf{v},T^aA^{auq^{e_0-1}}],	&\text{with no restrictions on }\textbf{v},\\
	%	\theta(\e_i)=x\e_{\pi(i)},	&\theta(TA^{uq^{e_0-c}})=[\textbf{v},T^aA^{auq^{e_0-c}}],	&\text{with the above restrictions on }\textbf{v},
	%\end{matrix*}
	%\]
	%where, for the middle (last) row, whenever $\pi$ is the transposition map, we have $u=(q-1)/2$ ($=(q^{c}-1)/2$) and $a=-1$, otherwise $a=1$. This gives us $|\Aut(M,M')|=p^2(p-1)$ in the first case, $|\Aut(M,M')|=p^2(p-1)$ for $u\neq (q-1)/2$ and $|\Aut(M,M')|=2p^2(p-1)$ for $u=(q-1)/2$ in the second case, and $|\Aut(M,M')|=p(p-1)$ for $u \neq (q^{c}-1)/2$ and $|\Aut(M,M')|=2p(p-1)$ for $u = (q^{c}-1)/2$ in the third case. Now let us reintroduce the generator of order $d$ (i.e. let $d \neq 1$); checking commutation relations with $B^{s/d}$ and $\e_i$ through $\theta$, we get that $c=1$. Next checking the commutation relation between $B^{s/d}$ and $TA^{uq^{e_0-c}}$, we obtain
	%[(a_{\beta}^{s/d}-1)\textbf{v}=\left(T^aA^{auq^{e_0-c}}-I\right)\mu\textbf{f}\]
	%with this, setting $\lambda=U(a_{\alpha}^{uq^{e_0-(c-1)}},a_{\beta}^{s/d})\mu$, $\textbf{v}$ satisfies the above relations; we obtain the three sets of possibilities for $\theta$ above with $\theta(B^{s/d})=[\lambda \textbf{f},B^{s/d}]$ along with restrictions on $\textbf{v}$ in each case. This gives us $|\Aut(M,M')|=p(p-1)$ in the first case, and $|\Aut(M,M')|=p(p-1)$ for $u\neq (q-1)/2$ and $|\Aut(M,M')|=2p(p-1)$ for $u=(q^{c}-1)/2$ in the second and third cases (where $c=1$ in the second case and $c>1$ in the third case).\\
	
	%We now look at the isomorphisms between such groups with different values of $u$ and $c$. First we look at $M_{u,c}:=P \rtimes \langle TA^{uq^{e_0-c}} \rangle$; we have that $TA^{uq^{e_0-c}}$ acts on $P$ with two distinct eigenvalues $ga_{\alpha}^{uq^{e_0-c}}$ and $a_{\alpha}^{uq^{e_0-c}}$. Any element in $M_{u,c}$ of order $q^{c}$ has the form $\left[\textbf{v},\left(TA^{uq^{e_0-c}}\right)^c\right]$ for some $\textbf{v}\in \mathbb{F}_p^2$ and $c \in (\mathbb{Z}/{q^{c}}\mathbb{Z})^{\times}$, which acts on $P$ with eigenvalues $g^ca_{\alpha}^{cuq^{e_0-c}}$ and $a_{\alpha}^{cuq^{e_0-c}}$. Thus if $M_{u,c}$ and $M_{v,c'}$ are isomorphic as abstract groups, then $\left\{g^ca_{\alpha}^{cuq^{e_0-c}},a_{\alpha}^{cuq^{e_0-c}}\right\}=\left\{ga_{\alpha}^{vq^{e_0-c'}},a_{\alpha}^{vq^{e_0-c'}}\right\}$ for some $c$. Therefore either $c(q^{e_0-1}+uq^{e_0-c})=q^{e_0-1}+vq^{e_0-c'}$ and $cuq^{e_0-c}=vq^{e_0-c'}$, or $c(q^{e_0-1}+uq^{e_0-c})=vq^{e_0-c'}$ and $cuq^{e_0-c}=q^{e_0-1}+vq^{e_0-c'}$ in $\mathbb{F}_{q^{e_0}}$. From the first equation set, we get $c=c'$, $c=1$ and $u=v$, and for the second equation set, we get $c=-1$, $c=c'>0$ and $u+v+q^{c-1}=0 \mod{q^{c}}$ with $u=v$ if and only if $u=(q^{c}-q^{c-1})/2$ (but note that if $c=0$ then we must have $c'=1$ and $v=-1$, and if $c'=0$ then we must have $c=1$ and $u=-1$); in this case, we get an isomorphism $\phi:M_{u,c} \rightarrow M_{v,c}$ (or from $M_{u,0}$ to $M_{-1,1}$) given by $\phi(\e_1)=\e_2$, $\phi(\e_2)=\e_1$, and $\phi(TA^{uq^{e_0-c}})=(TA^{vq^{e_0-c}})^{-1}$; as $\phi(\textbf{f})=-\textbf{f}$, it follows that $\phi$ is an isomorphism of permutation groups, and so in such a case, the groups $M_{u,c}$ fall into one isomorphism class for $M_{u,c}=M_{u,0} \cong M_{-1,1}$, and $\frac{1}{2}\varphi(q^{c})$ isomorphism classes for each choice of $c>0$ and $M_{u,c} \neq M_{-1,1}$. Otherwise, each choice of $u$ and $c$ yields a distinct isomorphism class. A similar argument follows with the introduction of the generator of order $d$, this time with $TA^{uq^{e_0-c}}B^{s/d}$ acting with eigenvalues $ga_{\alpha}^{uq^{e_0-c}}a_{\beta}^{s/d}$ and $a_{\alpha}^{uq^{e_0-c}}a_{\beta}^{s/d}$.
\end{proof}

\begin{remark}
	\emph{Note that the groups $M_{u,c}$ and $\widehat{M}_{u,c}$ contain an abelian group of order divisible by $pq$ if and only if $(u,c)=(-1,1)$. In this case, the generator $TA^{-q^{e_0-c}}$ commutes with $\e_1$, so $M_{-1,1} \cong C_p \times (C_p \rtimes C_q)$ and $\widehat{M}_{u,c} \cong (C_p \times (C_p \rtimes C_q)) \rtimes C_d$. Recall that $M_{-1,1}$ is also isomorphic as a permutation group to $P \rtimes \langle T \rangle$, and $\widehat{M}_{-1,1}$ is isomorphic as a permutation group to $P \rtimes \langle T,B^{s/d} \rangle$. Each contain a normal subgroup of order $pq$ which is complement to $M_{-1,1}'$ or $\widehat{M}'_{-1,1}$ respectively. For $(u,c) \neq (-1,1)$, the generator $TA^{uq^{e_0-c}}$ acts on $P$ with two distinct eigenvalues as discussed above. The normal complement in $M_{u,c}$ to $M'_{u,c}$ (and likewise in $\widehat{M}_{u,c}$ to $\widehat{M}'_{u,c}$) would be a transitive subgroup of order $pq$, namely either $\langle \e_1,[\mu\e_2,TA^{uq^{e_0-1}}] \rangle$ or $\langle \e_2, [\mu\e_1,TA^{uq^{e_0-1}}] \rangle$ for some $0 \leq \mu \leq p-1$, $u \neq -1$. However, none of these groups are normalised by $P$, and so $M_{u,c}'$ does not have a normal complement in $M_{u,c}$.}
	
	\emph{For $(u,c) \neq (-1,1)$, we denote the isomorphism classes of the groups $M_{u,c}$ and $\widehat{M}_{u,c}$ by $\mathbb{F}_p^2 \rtimes_u C_{q^c}$ and $\mathbb{F}_p^2 \rtimes_u C_{dq^{c}}$ respectively.}
\end{remark}

%\begin{remark}
%	For $(u,c) \neq (u,0),(-1,1)$, the generator $TA^{uq^{e_0-c}}$ acts on $P$ with two distinct eigenvalues $ga_{\alpha}^{uq^{e_0-c}}$, $a_{\alpha}^{uq^{e_0-c}}$ of order $q^{c}$. A normal complement in $M_{u,c}$ to $M_{u,c}'= \langle \textbf{f},A^{uq^{e_0-(c-1)}} \rangle$ would be a transitive subgroup of order $pq$, and since $(u,c) \neq (u,0),(-1,1)$, it would be either $\langle \e_1,[\lambda\e_2,TA^{uq^{e_0-1}}] \rangle$ or $\langle \e_2, [\lambda\e_1,TA^{uq^{e_0-1}}] \rangle$ for some $0 \leq \lambda \leq p-1$ where $u \neq -1$. However, none of these groups are normalised by $P$, and so $M_{u,c}'$ does not have a normal complement in $M_{u,c}$. A similar argument applies to $\widehat{M}_{u,c}$.}
	
%	\emph{For $(u,c) \neq (u,0),(-1,1)$, we denote the isomorphism classes of the groups $M_{u,c}$ and $\widehat{M}_{u,c}$ by $\mathbb{F}_p^2 \rtimes_u C_{q^{c}}$ and $\mathbb{F}_p^2 \rtimes_u C_{dq^{c}}$.}
%\end{remark}

\begin{proposition}
	For a fixed $d|s$ with $M= \langle \e_1,T,B^{s/d} \rangle$, we have
	\[|\Aut(M,M')|= \begin{cases}
		p(p-1) & \text{if $d = 1$},\\
		p-1 & \text{otherwise}.
	\end{cases}\]
\end{proposition}
\begin{proof}
	Clearly $M'=\langle B^{s/d} \rangle$. For $\theta \in \Aut(M,M')$, we must have
	$\theta(\e_1)=x\e_1$ for some $1 \leq x \leq p-1$, $\theta(B^{s/d})=B^{as/d}$ for some $a$, and $\theta(T)=[\mu\e_1,T^b]$ for some $\mu$ and $b$. Suppose first that $d\neq 1$, then the relations $B^{s/d}\e_1B^{-s/d}=a_{\beta}^{s/d}\e_1$, $T\e_1T^{-1}=g\e_1$ and $TB^{s/d}=B^{s/d}T$ force $a=b=1$ and $\mu=0$, giving $|\Aut(M,M')|=p-1$. If $d=1$, then there is no restriction on $\mu$, giving $|\Aut(M,M')|=p(p-1)$.
\end{proof}

\begin{proposition}
	For fixed $u,c$ and $d$ with $M=\left\langle \e_1,TA^{uq^{e_0-c}},B^{s/d}\right\rangle$, we have
	\[|\Aut(M,M')|= \begin{cases}
		p(p-1) & \text{if $c=d=1,u \neq -1$},\\
		(p-1)(q-1) & \text{if $(c,u)=(1,-1)$},\\
		p-1 & \text{if either $(c>1)$ or $(d>1$ and $u \neq -1)$}.
	\end{cases}\]
\end{proposition}
\begin{proof}
	We have $M'=\langle A^{uq^{e_0-(c-1)}},B^{s/d} \rangle$. Then $\theta(\e_1)=x\e_1$ and $\theta$ acts as identity on $B^{s/d}$ by similar logic to the previous proof. Further, we have
	\[\theta(TA^{uq^{e_0-c}})=\left[\mu\e_1,T^aA^{auq^{e_0-c}}\right] \text{ with } \theta(A^{uq^{e_0-(c-1)}})=A^{auq^{e_0-(c-1)}}.\]
	There are several possibilities. Firstly, if $c>1$, then we must have $\mu=0$ and $a=1$ due to the commutation relations between $\e_1$ and $TA^{uq^{e_0-c}}$. There is no restriction on $x$, and so $|\Aut(M,M')|=p-1$. Suppose now that $c=1$. If $u=-1$, then $M$ contains an abelian subgroup of order $pq$ coprime to its index of $2$. Then we clearly have $|\Aut(M,M')|=(p-1)(q-1)$, corresponding to $\mu=0$, $1 \leq x \leq p-1$ and $1 \leq a \leq q-1$. If $u\neq -1$, then $a=1$, again due to the commutation relations between $\e_1$ and $TA^{uq^{e_0-c}}$. If, further there is no generator of order $d$ (that is, if $d=1$), there is no restriction on $\mu$, giving $|\Aut(M,M')|=p(p-1)$. Otherwise, the commutation relations between $A^{uq^{e_0-(c-1)}}$ and $B^{s/d}$ force $\mu=0$, giving $|\Aut(M,M')|=p-1$.
\end{proof}

\begin{proposition}
	For fixed $c\geq 1$ and $d$ with $M=\left\langle \e_1,T,A^{q^{e_0-c}},B^{s/d}\right\rangle$, we have $|\Aut(M,M')|=(p-1)(q-1)$.
\end{proposition}

\begin{proof}
 We have $M'=\langle A^{e^{e_0-c}},B^{s/d} \rangle$. Due to commutation relations between the generators of $M'$ and $\e_1$, given $\theta \in \Aut(M,M')$, we must have that $\theta$ acts as the identity on $M'$. We then have $\theta(\e_1)=x\e_1$ and
 \[\theta(T)=\left[\mu\e_1,T^aA^{bq^{e_0-1}}\right],\]
 for some $1 \leq x \leq p-1$, $0\leq \mu \leq p-1$, $1 \leq a \leq q-1$, and $0 \leq b \leq q-1$. Given $c\geq 1$, commutation relations between $T$ and $A^{e^{e_0-c}}$ force $\mu=0$. Next, commutation relations between $\e_1$ and $T$ give $a+b=1$. There are no further restrictions, giving a total of $(p-1)(q-1)$ elements in $\Aut(M,M')$.
\end{proof}

\begin{table}
	\setlength{\extrarowheight}{2mm}
	\centering
	\begin{tabular}{|c|c|}
		\hline
		Group	&	Structure\\
		\hline
		$P \rtimes \left \langle T, A^{q^{e_0-c}}, B^{s/d} \right \rangle$ & $(N \rtimes (C_p \rtimes C_{q^{c}}))\rtimes C_d$\\[5pt]
		\hline
		$P \rtimes \left \langle TA^{uq^{e_0-c}}, B^{s/d} \right \rangle$, for $(c,u) \neq (1,-1)$ & $\mathbb{F}_p^2 \rtimes_u C_{dq^{c}}$\\[5pt]
		\hline
		$P \rtimes \left \langle T, B^{s/d} \right \rangle \cong P \rtimes \left\langle TA^{-q^{e_0-1}} \right \rangle$ & $((C_p \rtimes C_q)\times C_p)\rtimes C_d$\\[5pt]
		\hline
		$\left\langle \e_1,T,B^{s/d} \right\rangle$ & $C_p \rtimes C_{dq}$\\[5pt]
		\hline
		$\left\langle \e_1, TA^{uq^{e_0-c}}, B^{s/d} \right\rangle$, for $(c,u) \neq (1,-1)$ & $C_p \rtimes C_{dq^{c}}$\\[5pt]
		\hline
		$\left\langle \e_1, TA^{-q^{e_0-1}}, B^{s/d} \right\rangle$ & $(C_p \rtimes C_d) \times C_q$\\[5pt]
		\hline
		$\left\langle \e_1,T,A^{q^{e_0-c}}, B^{s/d} \right\rangle$ & $(C_p \rtimes C_{dq^c}) \times C_q$\\[5pt]
		\hline
		$\left\langle \e_1,T \right\rangle$ & $C_p \rtimes C_q$\\[5pt]
		\hline
		$\left\langle \e_1, TA^{uq^{e_0-c}} \right\rangle$, for $(c,u) \neq (1,-1)$ & $C_p \rtimes C_{q^c}$\\[5pt]
		\hline
		$\left\langle \e_1, TA^{-q^{e_0-1}} \right\rangle$ & $C_{pq}$\\[5pt]
		\hline
		$\left\langle \e_1,T,A^{q^{e_0-c}} \right\rangle$ & $(C_p \rtimes C_{q^c}) \times C_q$\\[5pt]
		\hline
	\end{tabular}
	
	\medskip
	
	\caption{Isomorphism types of transitive groups for $N$ metabelian.}
	\label{metabelian-structure}
\end{table}

Most notably there is no case where the groups are isomorphic as abstract groups but not isomorphic as permutation groups.

Denote by $\sigma_0(s)$ the number of divisors of $s$.


\begin{table}
	\setlength{\extrarowheight}{2mm}
	\begin{center}
		\resizebox{\columnwidth}{!}{
		\begin{tabular}{|c|c|c|c|}
			\hline
			Structure	&	$\#$groups	&	$|\Aut(M,M')|$	&	$\#$HGS \\
			\hline
			$(N \rtimes (C_p \rtimes C_{q^{c}}))\rtimes C_d$ $c>0, d|s$ 	&	$1$	&	$2p(p-1)$	&	$2$ \\[5pt]
			\hline
			$(C_p \rtimes C_q) \times C_p$	&	$2$	&	$p(p-1)$	&	$2$ \\
			$\mathbb{F}_p^2 \rtimes_u C_q$, $1 \leq u \leq (q-3)/2 $	&	$2$	&	$p^2(p-1)$	&	$2p$\\
			$\mathbb{F}_p^2 \rtimes_{(q-1)/2}C_q$	&	$1$	&	$2p^2(p-1)$	&	$2p$\\
			$\mathbb{F}_p^2 \rtimes_u C_{q^c}$, $c>1, 1\leq u \leq (q-1)/2$	&	$2q^{c-1}$	&	$p(p-1)$		&	$2q^{c-1}$\\
			[5pt]
			\hline
			$(C_p \times (C_p \rtimes C_q)) \rtimes C_d$, $d|s, d > 1$	&	$2$	&	$p(p-1)$	&	$2$\\
			$\mathbb{F}_p^2 \rtimes_u C_{dq}$, $1\leq u \leq (q-3)/2,d|s, d>1$	&	$2$	&	$p(p-1)$	&	$2$\\
			$\mathbb{F}_p^2 \rtimes_{(q-1)/2}C_{dq}$, $d|s, d>1$	& $1$ &	$2p(p-1)$	&	$2$ \\
			$\mathbb{F}_p^2 \rtimes_u C_{dq^c}$, $c>1, 1\leq u \leq (q-1)/2,d|s, d>1$	& $2q^{c-1}$ &	$p(p-1)$	& $2q^{c-1}$\\[5pt]
			\hline
			$C_p \rtimes C_{dq^c}$, $c>0,d|s, (c,d) \neq (1,1)$	&	$2p\varphi(q^{c})$	&	$p-1$	&	$2\varphi(q^c)$	\\[5pt]
			\hline
			$C_p \rtimes C_q$	&	$2p(q-1)+2$	&	$p(p-1)$	&	$2p(q-2)+2$	\\[5pt]
			\hline
			$(C_p \rtimes C_{dq^c}) \times C_q$, $c\geq0, d|s$	&		$2p$	&	$(p-1)(q-1)$	&	$2(q-1)$	\\[5pt]
			\hline
		\end{tabular}}
		
		\medskip
		
		\caption{ Transitive subgroups for $N$ metabelian.}
		\label{metabelian-trans-HGS}
	\end{center}
\end{table}

\begin{theorem}
	In total, there are
	\[\sigma_0(s)\left[3e_0+3+\frac{1}{2}(q-3)+\frac{1}{2}(e_0-1)(q-1)\right]\]
	isomorphism types of permutation groups $G$ of degree $pq$ which are realised by Hopf-Galois structures of non-abelian type $C_p \rtimes C_q$, as listed in Table \ref{metabelian-trans-HGS}. These include the two regular groups, i.e. the cyclic and non-abelian groups of order $pq$ (for which the corresponding Galois extensions have $2(q-1)$ and $2p(q-2)+2)$ Hopf-Galois structures of non-abelian type respectively). Of the corresponding field extensions for these groups, $\sigma_0(s)(1+\frac{1}{2}(q-3)+\frac{1}{2}(e_0-1)(q-1))$ fail to be almost classically Galois (these are the groups given by all but one group in each of the second and third left-hand cells in Table \ref{metabelian-trans-HGS}, along with $C_p \rtimes C_{dq^{c}}$ for $c>1$). In the remaining $\sigma_0(s)(3e_0+2)$ cases, the extensions are almost classically Galois.
\end{theorem}
\begin{proof}
	Everything except the statements about almost classically Galois extensions follows from Table \ref{metabelian-trans-HGS}. Upon inspection, it is clear that any group, say $J$, of order $pq$ in $P\rtimes\left\langle TA^{uq^{e_0-c}},B^{s/d} \right\rangle$ or $\left\langle \e_1,TA^{uq^{e_0-c}},B^{s/d} \right\rangle$, both for $(u,c)\neq (-1,1)$, cannot contain $T$, and thus none of these groups contain a regular normal subgroup, so don't correspond to almost-classically Galois extensions. For the rest of the groups in the table, it is clear that they contain $\langle \e_1,T \rangle \cong N$, and so correspond to almost-classically Galois extensions.
\end{proof}

\begin{remark}
	\emph{Note that specialising to $p=2q+1$, we obtain $2(3+3+\frac{1}{2}(q-3))=q+9$ isomorphism types, with $2(1+\frac{1}{2}(q-3)+\frac{1}{2}(1-1)(q-1))=q-1$ non-almost classically Galois extensions and $2(3+2)=10$ almost classically Galois extensions. This retrieves the result of Theorem 3.2 of \cite{BML21}. We also note that the result for $(C_p \rtimes C_q) \times C_p$ where $|\Aut(M,M')|=p(p-1)$ corrects an error in Table $5$ in \cite{BML21}.}
\end{remark}

\begin{theorem}
	There are $\sigma_0(s)(2e_0+1)$ permutation groups $G$ of degree $pq$ which are realised by Hopf-Galois structures of cyclic and non-abelian types. These are shown in Table \ref{both-types}. In all but $\sigma_0(s)(e_0-1)$ of these cases (corresponding to $C_p \rtimes C_{dq^{c}}$ for $c>1$), the corresponding field extensions are almost classically Galois.
\end{theorem}

\begin{remark}
	\emph{For every Hopf-Galois structure of non-abelian type, there is an opposite Hopf-Galois structure of the same type. Hence it makes sense that the number of Hopf Galois structures in each row of Table \ref{metabelian-trans-HGS} is even.}
\end{remark}

\begin{remark}
	\emph{This again realises the Sophie Germain result of $2(2+1)=6$ such permutation groups (Theorem 3.3 of \cite{BML21}). In this case, $2(1-1)=0$ extensions fail to be almost classically Galois.}
\end{remark}

\begin{table}
	\begin{center}
		\begin{tabular}{|c|c|c|}
			\hline
			Structure	&	$\#$ cyclic type HGS	&	$\#$ non-abelian type HGS\\
			\hline
			$(C_p \rtimes C_{dq^{c}}) \times C_q$	&	$1$	&	$2(q-1)$	\\
			\hline
			$C_p \rtimes C_{dq^{c}}$, $(c,d) \neq (1,1),(0,d)$	&	$1$	&	$2\varphi(q^{c})$	\\
			\hline
			$C_p \rtimes C_q$	&	$p$	&	$2p(q-2)+2$\\
			\hline
		\end{tabular}
		\medskip
		\caption{Groups admitting Hopf-Galois structures of both types.}
		\label{both-types}
	\end{center}
\end{table}

\section*{Acknowledgements}
The author is supported by the Engineering and Physical Sciences Doctoral Training Partnership research grant (EPSRC DTP).
\bibliography{MyBib}
	
\end{document}