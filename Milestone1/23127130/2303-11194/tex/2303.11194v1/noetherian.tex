\section{Noetherian rings}
In this short section we prove that $A=H_0(\Hur(Q))$ is a Noetherian ring, and give a prove of Theorem \ref{thm:main2} assuming Theorem \ref{thm:main1}.

\subsection{Several subrings of \texorpdfstring{$A$}{A}}
The ring $A$ admits the following presentation by generators and relation as an associative ring (compare with the presentation of $\pi_0(\Hur(Q))$ from Subsection \ref{subsec:topmonoid}):
 \[
  A=\Z\left< [q_1],\dots, [q_m]\  |\  [q_i][q_j]=[q_j][q_i^{q_j}]\right>,
 \]
where $[q_i]\in H_0(\Hur_1(Q))$ is defined as the ground class of the (contractible) space $\Hur_1(Q,q_i)$.

We introduce two subrings of $A$.
\begin{defn}\label{defn:B}
We denote by $B\subseteq A$ the subring generated by the elements $[q_i]^\ell$, for $1\le i\le m$.
Equivalently, $B$ is the monoid ring of the monoid $\pi_0^\ell(\Hur(Q))$ from Definition \ref{defn:pi0ell}.
\end{defn}
\begin{defn}\label{defn:Aone}
Recall Definition \ref{defn:totmon}. We denote by $A_\one\subseteq A$ the monoid ring of the submonoid
 \[
\pi_0(\Hur(Q)_\one)\subseteq\pi_0(\Hur(Q)).
\]
\end{defn}
We observe that $B\subseteq A_\one$, as each generator $\hat q_i^\ell\in \pi_0^\ell(\Hur(Q))$ has total monodromy equal to $q_i^\ell=\one\in G$. We also observe that $A_\one$ is a central subring of $A$, as a consequence of the fact that $\pi_0(\Hur(Q)_\one)$ is a central submonoid of $\pi_0(\Hur(Q))$: to see this, let $a\in Q$ and let $(a_1,\dots,a_n)\in Q^n$ be such that $a_1\dots a_n=\one\in G$; then $\hat a\cdot (\hat a_1\dots\hat a_n)=(\hat a_1\dots\hat a_n)\cdot\widehat {a^{a_1\dots a_n}}$ by the relations holding in $\pi_0(\Hur(Q))$, and we have 
$\widehat {a^{a_1\dots a_n}}=\widehat{a^\one}=\hat a$.
\begin{lem}
 \label{lem:Anoetherian}
The associative ring $A$ is finitely generated as a $B$-module. As a consequence we have the following:
\begin{enumerate}
 \item $A$ is Noetherian: a sub-module of a finitely generated left or right $A$-module is finitely generated.
 \item $A_\one$ is also finitely generated as a $B$-module, and is also a Noetherian ring.
 \item $A$ is finitely generated as a $A_\one$-module.
\end{enumerate}
\end{lem}
\begin{proof}
The commutative ring $B$ is Noetherian, as it is a quotient of a polynomial ring over $\Z$ with $m$ variables (in the same way as $B$ is a quotient of a free abelian monoid on $m$ generators). Thus if we prove that $A$ is a finitely generated $B$-module, we immediately have that $A$ is Noetherian: for if $M$ is a finitely generated (left or right) $A$-module and $M'\subset M$ is a submodule, then $M$ is also finitely generated over $B$, and by Noetherianity of $B$ we have that $M'$ is finitely generated over $B$, and a fortiori over $A$. This proves that the main statement implies (1); it also implies (2), as $A_\one$ is a sub-$B$-module of $A$, and $B$ is Noetherian; Noetherianity of $A_\one$ then follows from the same argument used to prove (1). Finally, (3) follows from the inclusion $B\subseteq A_\one$ together with the fact that $A$ is finitely generated as a $B$-module.

We now prove the main statement. Recall that $\pi_0^\ell(\Hur(Q))$ is a central submonoid in $\pi_0(\Hur(Q))$: this implies that $B$ is a central subring of $A$.
Our next goal is to show that $A$ is a finitely generated $B$-module: more precisely, the products $[a_1]\cdot \dots\cdot[a_k]$ with $k\le m(\ell-1)$ and $a_1,\dots,a_k\in Q$ suffice to generate $A$ over $B$. For this, let $[a_1]\cdot \dots\cdot[a_n]$ be any product of generators of $A$, and assume $n>m(\ell-1)$. By the pigeonhole principle there is $q_j\in Q$ and there are $\ell$ indices $i_1,\dots,i_\ell$ such that $a_{i_1}=\dots=a_{i_\ell}=q_j$. We can then use the relations in $A$ and rewrite
$[a_1]\cdot \dots\cdot[a_n]$ in the form $[q_j]^\ell\cdot [a'_1]\cdot \dots\cdot[a'_{n-\ell}]$, where the sequence $a'_1,\dots,a'_{n-\ell}$ is obtained by removing from $a_1,\dots,a_n$ the elements $a_{i_1},\dots,a_{i_\ell}$, and by suitably conjugating the remaining elements by powers of $q_j$. This concludes the proof that $A$ is a finitely generated $B$-module.
\end{proof}
If $R$ is a commutative ring, we can tensor $A,A_\one,B$ with $R$ and obtain the rings $A\otimes R=H_0(\Hur(Q);R)$, $A_\one\otimes R=H_0(\Hur(Q)_\one;R)$ and $B=R[\pi_0^\ell(\Hur(Q))]$. Lemma \ref{lem:Anoetherian} gives the following corollary.
\begin{cor}\label{cor:Anoetherian}
 Let $R$ be a commutative ring; then $A\otimes R$ and $A_\one\otimes R$ are finitely generated $B\otimes R$-modules, and $A\otimes R$ is a finitely generated $A_\one\otimes R$-module.
 If $R$ is Noetherian, then all rings $A\otimes R$, $A_\one\otimes R$ and $B\otimes R$ are Noetherian.
\end{cor}

\subsection{Proof of Theorem \ref{thm:main2} assuming Theorem \ref{thm:main1}}
Let $R$ be a Noetherian ring and let $i\ge0$; then by Theorem \ref{thm:main1} $H_i(\Hur(Q);R)$ is finitely generated over $A\otimes R$; by Lemma \ref{lem:Anoetherian} $A\otimes R$ is finitely generated as an $A_\one\otimes R$-module, and hence also $H_i(\Hur(Q);R)$ is finitely generated as an $A_\one\otimes R$-module. We have a finite direct sum decomposition of $A_\one\otimes R$-modules
\[
 H_i(\Hur(Q);R)\cong \bigoplus_{g\in G}  H_i(\Hur(Q)_g;R),
\]
and hence each direct summand is a finitely generated $A_\one\otimes R$-module.
