\section{Arc complexes}
\label{sec:arccomplex}
Hatcher and Wahl \cite{hatcherwahl} introduced, for $n\ge0$, an augmented semisimplicial set $\cA_{n,\bullet}$, whose initial application was an alternative proof of homology stability of braid groups \cite[Proposition 1.7]{hatcherwahl}, originally proved by Arnol'd \cite{arnold} (see also \cite[§5.6]{RWW}). In the following definition we recall this construction and generalise it to our Hurwitz setting; this is approach has been already used in \cite{EVW:homstabhur}
\subsection{High connectivity of augmented arc complexes}
We denote by $D=[0,1]^2\subset\R^2$ the standard unit square in the complex plane, endowed with basepoint $*=(0,0)$, and denote by $\mD=(0,1)^2$ the interior of $D$. We also let $I=[0,1]\times \set{0}$ be the bottom edge of $D$.

For $n\ge0$ and $1\le i\le n$ we let $\bar z_{n,i}=(i/(n+1),1/2)\in\mD$, and we denote $\bar P_n=\set{\bar z_{n,i}}$.
\begin{defn}
\label{defn:arccomplex}
Let $n\ge0$. For $p\ge-1$ we denote by $\cA_{n,p}$ the set of isotopy classes of collections of $p+1$ arcs $\alpha_0,\dots,\alpha_p\colon[0,1]\to D$ satisfying the following conditions:\footnote{Two collections of arcs are considered isotopic if they are connected by an isotopy through collections of arcs with the required properties.}
\begin{itemize}
 \item each arc $\alpha_i$ is an embedding of $[0,1]$ in $D$, sending $0$ to a point of $I$, $(0,1)$ inside $\mD\setminus\bar P_n$, and $1$ to a point of $\bar P_n$;
 \item the arcs have disjoint images, also at their endpoints, and using the natural orientation of $I$ we have $\alpha_0(0)<\dots<\alpha_p(0)$.
\end{itemize}
In particular, $\cA_{n,-1}$ is a singleton (the empty collection of arcs), and $\cA_{n,p}$ is empty for $p\ge n$. Forgetting arcs makes the collection $\cA_{n,\bullet}$ into an augmented semisimplicial set.

We denote by $\cA_n(Q)_\bullet$ the augmented semisimplicial set $\cA_{n,\bullet}\times Q^n$, whose set of $p$-simplices is $\cA_{n,p}\times Q^n$.

The braid group $\Br_n$ acts both on the augmented semisimplicial set $\cA_{n,\bullet}$ and on the set $Q^n$, hence it acts diagonally on $\cA_n(Q)_\bullet$ by automorphisms of augmented semisimplicial sets; taking levelwise the homotopy quotient we obtain an augmented semisimplicial space $\cA_n(Q)_\bullet\qq\Br_n$.
\end{defn}
The braid group $\Br_n=\Gamma_{0,1}^{(n)}$ acts both on the augmented semisimplicial set $\cA_{n,p}$ and on the set $Q^n$, and hence there is a diagonal action of $\Br_n$ on $\cA_n(Q)_\bullet$.
The geometric realisation of $\cA_{n,\bullet}$ is contractible,
i.e. the augmentation $|\cA_{n,\bullet\ge0}|\to\cA_{n,-1}$ is a
homotopy equivalence, the second space being a point \cite[Theorem 2.48]{Damiolini} (see also \cite[Proposition 3.2]{HV:tethers}).

It also follows that the augmentation $|\cA_n(Q)_{\bullet\ge0}|\to \cA_n(Q)_{-1}$ is a homotopy equivalence, the second space being the set $Q^n$,
and it further follows that the augmentation
$|\cA_n(Q)_{\bullet\ge0}\qq\Br_n|\to\cA_n(Q)_{-1}\qq\Br_n$ is a
homotopy equivalence, the second space being the space $\Hur_n(Q)$.
In the next subsection we will analyse more closely the augmented semisimplicial space $\cA_n(Q)_\bullet\qq\Br_n$.
\begin{nota}
\label{nota:cSndot}
 We denote by $\cS^n_\bullet$ the augmented semisimplicial space $\cA_n(Q)_\bullet\qq\Br_n$, leaving $Q$ implicit.
\end{nota}
We record the previous discussion as a lemma for future reference.
\begin{lem}\label{lem:damiolini}
The augmentation $|\cS^n_\bullet|\to\cS^n{-1}$ is a homotopy equivalence.
\end{lem}


\subsection{The augmented semisimplicial space \texorpdfstring{$\cS^n_\bullet$}{Sndot}}
For $-1\le p\le n-1$, the space of $p$-simplices in $\cS^n_\bullet$ is $(\cA_{n,p}\times Q^n)\qq\Br_n$, in particular it is the homotopy quotient of a set by an action of a discrete group: it has therefore the homotopy type of a disjoint union of aspherical spaces, one for each orbit of the diagonal action of $\Br_n$ on $\cA_{n,p}\times Q^n$. The action of $\Br_n$ on $\cA_{n,p}$ is transitive. Let $\bar\ualpha_{n,p}=(\bar\alpha_{n,p,0},\dots,\bar\alpha_{n,p,p})$ be the isotopy class of the collection of $p+1$ straight vertical segments joining $I$ with the points $z_{n,1},\dots,z_{n,p+1}\in\bar P_n$ (see Figure \ref{fig:stdsimplex}); then the stabiliser of $\bar\ualpha_{n,p}$
is the subgroup of $\Br_n$ generated by the last $n-p-2$ standard generators, which is isomorphic to $\Br_{n-p-1}$.
\begin{figure}[ht]
 \begin{tikzpicture}[scale=4,decoration={markings,mark=at position 0.3 with {\arrow{>}}}]
  \fill[black, opacity=.2] (0,0) rectangle (1,1);
  \draw[black] (0,0) rectangle (1,1);
  \node at (0,0) {$*$};
  \node at (.2,.5){$\bullet$};
  \node at (.4,.5){$\bullet$};
  \node at (.6,.5){$\bullet$};
  \node at (.8,.5){$\bullet$};
  \node at (.2,.55){$\bar{z}_{4,1}$};
  \node at (.4,.55){$\bar{z}_{4,2}$};
  \node at (.6,.55){$\bar{z}_{4,3}$};
  \node at (.8,.55){$\bar{z}_{4,4}$};
  \draw[thin, looseness=1] (.1,0) to[out=90, in=-90] node{$\bar\alpha_{4,2,0}$} (.2,.5); 
  \draw[thin, looseness=1] (.5,0) to[out=90, in=-90] node{$\bar\alpha_{4,2,1}$} (.4,.5); 
  \draw[thin, looseness=1] (.9,0) to[out=90, in=-90] node{$\bar\alpha_{4,2,2}$} (.6,.5); 
\end{tikzpicture}
 \caption{The 2-simplex $\bar\ualpha_{4,2}$.}
 \label{fig:stdsimplex}
\end{figure}

\begin{nota}
We denote by $\one_n\subset\Br_n$ the trivial subgroup of the $n$\textsuperscript{th} braid group. For $0\le p\le n$ we consider $\Br_p\times\Br_{n-p}$ as a subgroup of $\Br_n$, generated by all standard generators except the $p$\textsuperscript{th}.
In particular we denote by $\one_p\times\Br_{n-p}\subset\Br_n$ the subgroup of $\Br_n$ generated by the last $n-p-1$ standard generators.
\end{nota}
\begin{lem}
\label{lem:htypeAnQpqqBrn}
 The spaces $\cS^n_p$ and $Q^{p+1}\times\Hur_{n-p-1}(Q)$ are homotopy equivalent. More precisely, the natural map
$Q^{p+1}\times\Hur_{n-p-1}(Q)\to \cS^n_p=\cA_n(Q)_p\qq\Br_n$ induced by the inclusion of sets
$\set{\bar\ualpha_{n,p}}\times Q^n\subset \cA_{n,p}\times Q^n$ and the inclusion of groups $\Br_{n-p-1}\cong\one_{p+1}\times\Br_{n-p-1}\subset\Br_n$, is a homotopy equivalence.
\end{lem}
\begin{proof}
By the previous discussion, each orbit of the action of $\Br_n$ on $\cA_{n,p}\times Q^n$ contains elements of the form $(\bar\ualpha_{n,p};\ua)$. Moreover,
any two elements $(\bar\ualpha_{n,p};\ua)$ and $(\bar\ualpha_{n,p};\ua')$ in the same $\Br_n$-orbit can be transformed into each other by the action of a suitable element in $\one_{p+1}\times\Br_{n-p-1}$: this implies the equality $a_i=a'_i$ for $0\le i\le p$, and it also implies that the subsequences $(a_{p+1},\dots,a_n)$ and $(a'_{p+1},\dots,a'_n)$ belong to the same orbit of the action of $\Br_{n-p-1}$ on $Q^{n-p-1}$. Viceversa, two elements $(\bar\ualpha_{n,p};\ua)$ and $(\bar\ualpha_{n,p};\ua')$ satisfying the previous requirements can be transformed into each other by the action of $\Br_{n-p-1}\subset\Br_n$, and thus belong to the same orbit of the action of $\Br_n$ on $\cA_n(Q)_p$.

We conclude by remarking that the stabiliser of a single element $(\bar\ualpha_{n,p};\ua)\in\cA_n(Q)_p$ is the subgroup $\one_{p+1}\times \Br_{n-p-1}(a_{p+1},\dots,a_n)\subseteq\one_{p+1}\times\Br_{n-p-1}\subseteq\Br_n$.
\end{proof}
In particular, as already remarked, the space of $(-1)$-simplices $\cS^n_{-1}$ is homotopy equivalent, and in fact canonically homeomorphic, to $\Hur_n(Q)$.

Let us now look at the face maps $d_i\colon \cS^n_p\to \cS^n_{p-1}$, for $p\ge0$ and $0\le i\le p$ (in the case $p=0$ we denote by $d_0$ the augmentation).
\begin{defn}
\label{defn:stabmap}
Let $a\in Q$ and $n\ge0$, and consider $\Br_n\cong \one_1\times\Br_n$ as a subgroup of $\Br_{n+1}$. The map $a\times -\colon Q^n\to Q^{n+1}$, sending $(a_1,\dots,a_n)\mapsto(a,a_1,\dots,a_n)$, is equivariant with respect to the action of $\Br_n$ on $Q^n$ and, by restriction, on $Q^{n+1}$. It thus induces a map on homotopy quotients
\[
 \lst(a)\colon\Hur_n(Q)\to\Hur_{n+1}(Q).
\]
We define similarly a map $\rst(a)\colon \Hur_n(Q)\to\Hur_{n+1}(Q)$ by considering the map of sets $-\times a\colon Q^n\to Q^{n+1}$, sending $(a_1,\dots,a_n)\mapsto(a_1,\dots,a_n,a)$, which is equivariant with respect to the inclusion of groups $\Br_n\times\one_1\subset\Br_{n+1}$.
\end{defn}
\begin{defn}
Let $0\le p\le n-1$ and $0\le i\le p$.
We define a map of spaces
 \[
 \lst_i\colon Q^{p+1}\times\Hur_{n-p-1}(Q)\to Q^p\times\Hur_{n-p}(Q)
 \]
as the map induced on homotopy quotients by the map of sets $Q^n\overset{\cong}{\to} Q^n$ given by
 \[
  (a_0,\dots,a_{n-1})\mapsto (a_0,\dots,a_{i-1},a_{i+1},\dots,a_p,a_i^{a_{i+1}\dots a_p},a_{p+1},\dots,a_{n-1}),
 \]
 which is equivariant with respect to the inclusion of groups $\one_{p+1}\times \Br_{n-p-1}\subset\one_p\times\Br_{n-p}$.
\end{defn}
\begin{prop}\label{prop:descriptiondi}
The following diagram is commutative up to homotopy, where the vertical maps are the homotopy equivalences given by Lemma \ref{lem:htypeAnQpqqBrn}.
 \[
  \begin{tikzcd}
   Q^{p+1}\times\Hur_{n-p-1}(Q)\ar[r,"{\lst_i}"]\ar[d,"\simeq"] &
   Q^p\times\Hur_{n-p}(Q)\ar[d,"\simeq"]\\
   \cS^n_p\ar[r,"{d_i}"] & \cS^n_{p-1}.
  \end{tikzcd}
 \]
\end{prop}
\begin{proof}
Let $\bar\ualpha_{n,p}^{\hat i}\in\cA_{n,p-1}$ be the collection of arcs $(\bar\alpha_{n,p,0},\dots,\hat{\bar\alpha}_{n,p,i},\dots,\bar\alpha_{n,p,p})$ obtained from $\bar\ualpha_{n,p}$ by forgetting $\bar\alpha_{n,p,i}$ (see Figure \ref{fig:facesimplex}, left): then in $\cA_{n,\bullet}$ we have $d_i(\bar\ualpha_{n,p})=\bar\ualpha_{n,p}^{\hat i}$.
\begin{figure}[ht]
 \begin{tikzpicture}[scale=4,decoration={markings,mark=at position 0.3 with {\arrow{>}}}]
  \fill[black, opacity=.2] (0,0) rectangle (1,1);
  \draw[black] (0,0) rectangle (1,1);
  \node at (0,0) {$*$};
  \node at (.2,.5){$\bullet$};
  \node at (.4,.5){$\bullet$};
  \node at (.6,.5){$\bullet$};
  \node at (.8,.5){$\bullet$};
  \node at (.2,.55){$\bar{z}_{4,1}$};
  \node at (.4,.55){$\bar{z}_{4,2}$};
  \node at (.6,.55){$\bar{z}_{4,3}$};
  \node at (.8,.55){$\bar{z}_{4,4}$};
  \draw[thin, looseness=1] (.5,0) to[out=90, in=-90] node{$\bar\alpha_{4,2,1}$} (.4,.5); 
  \draw[thin, looseness=1] (.9,0) to[out=90, in=-90] node{$\bar\alpha_{4,2,2}$} (.6,.5);
 \begin{scope}[shift={(1.2,0)}]
  \fill[black, opacity=.2] (0,0) rectangle (1,1);
  \draw[black] (0,0) rectangle (1,1);
  \node at (.2,.5){$\bullet$};
  \node at (.4,.5){$\bullet$};
  \node at (.6,.5){$\bullet$};
  \node at (.8,.5){$\bullet$};
  \draw[looseness=1,->] (.2,.55) to[out=50,in=130] (.6,.55);
  \draw[looseness=1,->] (.39,.45) to[out=-130,in=-50] (.2,.45);
  \draw[looseness=1,->] (.6,.45) to[out=-130,in=-50] (.41,.45);
\end{scope}
\end{tikzpicture}
 \caption{On left, the 1-simplex $\bar\ualpha_{4,2}^{\hat 0}$; on right, the product of standard generators $\fb_{4,2,0}=\sigma_2\sigma_1$.}
 \label{fig:facesimplex}
\end{figure}

Consider moreover the product of standard generators
\[
\fb_{n,p,i}=\sigma_p\sigma_{p-1}\sigma_{p-2}\dots\sigma_{i+1}\in\Br_{p+1}\times\one_{n-p-1}\subset\Br_n,
\]
depicted in Figure \ref{fig:facesimplex}, right: this is the empty product, i.e. the neutral element in $\Br_n$, for $i=p$. Then the action of $\fb_{n,p,i}\in\Br_n$ on $\cA_{n,p-1}$ sends $\bar\ualpha_{n,p}^{\hat i}\mapsto\bar\ualpha_{n,p-1}$. It follows that the stabiliser of $\bar\ualpha_{n,p}^{\hat i}$ in $\Br_n$ is
$\pa{\one_p\times\Br_{n-p}}^{\fb_{n,p,i}}$. Repeating the argument of Lemma \ref{lem:htypeAnQpqqBrn} with $\bar\ualpha_{n,p}^{\hat i}$ instead of $\bar\ualpha_{n,p-1}$, we obtain the following: the inclusion of sets
$\set{\bar\ualpha_{n,p}^{\hat i}}\times Q^n\subset \cA_{n,p-1}\times Q^n$ and the inclusion of groups $\pa{\one_p\times\Br_{n-p}}^{\fb_{n,p,i}}\subset\Br_n$ give rise to a homotopy equivalence
\[
 \pa{\set{\bar\ualpha_{n,p}^{\hat i}}\times Q^n}\qq \pa{\one_p\times\Br_{n-p}}^{\fb_{n,p,i}} \overset{\simeq}{\to} \cA_n(Q)_{p-1}\qq\Br_n=\cS^n_{p-1},
\]
and the following square commutes on the nose, where the top map is induced on homotopy quotients by the obvious bijection $\set{\bar\ualpha_{n,p}}\times Q^n\cong \set{\bar\ualpha_{n,p}^{\hat i}}\times Q^n$, which is equivariant with respect to the injective group homomorphism
$\one_{p+1}\times \Br_{n-p-1}\cong\pa{\one_{p+1}\times \Br_{n-p-1}}^{\fb_{n,p,i}}\subset \pa{\one_p\times\Br_{n-p}}^{\fb_{n,p,i}}$:
 \[
  \begin{tikzcd}
   Q^{p+1}\times\Hur_{n-p-1}(Q)\ar[r]\ar[d,"\simeq"] &
   \pa{\set{\bar\ualpha_{n,p}^{\hat i}}\times Q^n}\qq \pa{\one_p\times\Br_{n-p}}^{\fb_{n,p,i}} \ar[d,"\simeq"]\\
   \cS^n_p=\cA_n(Q)_p\qq\Br_n\ar[r,"{d_i}"] & \cS^n_{p-1}=\cA_n(Q)_{p-1}\qq\Br_n.
  \end{tikzcd}
 \]
We can then construct a strictly commutative square, whose vertical maps are homotopy equivalences and whose horizontal maps are homeomorphisms
 \[
  \begin{tikzcd}[column sep=35pt]
   \pa{\set{\bar\ualpha_{n,p}^{\hat i}}\times Q^n}\qq \pa{\one_p\times\Br_{n-p}}^{\fb_{n,p,i}} \ar[r,"\fb_{n,p,i}\cdot-","\cong"']\ar[d,"\simeq"] &
   \pa{\set{\bar\ualpha_{n,p-1}}\times Q^n}\qq \pa{\one_p\times\Br_{n-p}} \ar[d,"\simeq"]\\
   \cS^n_{p-1}=\cA_n(Q)_{p-1}\qq\Br_n\ar[r,"\fb_{n,p,i}\cdot-","\cong"'] & \cS^n_{p-1}=\cA_n(Q)_{p-1}\qq\Br_n.
  \end{tikzcd}
 \]
The top horizontal map is induced by the bijection of sets $\fb_{n,p,i}\colon \pa{\set{\bar\ualpha_{n,p}^{\hat i}}\times Q^n} \cong \pa{\set{\bar\ualpha_{n,p-1}}\times Q^n}$, given by the action of $\fb_{n,p,i}$, together with the isomorphism of groups
$\pa{\one_p\times\Br_{n-p}}^{\fb_{n,p,i}}\cong \pa{\one_p\times\Br_{n-p}}$, given by conjugation by $\fb_{n,p,i}^{-1}$ inside $\Br_n$; note that the aforementioned bijection of sets sends
\[
(\bar\ualpha_{n,p}^{\hat i}; a_0,\dots, a_{n-1})\mapsto (\bar\ualpha_{n,p-1};a_0,\dots,a_{i-1},a_{i+1},\dots,a_p,a_i^{a_{i+1}\dots a_p},a_{p+1},\dots,a_{n-1}).
\]

The bottom horizontal map has a similar description, but involving the automorphism of the set $\cA_n(Q)_{p-1}$ given by the action of $\fb_{n,p,i}$, together with the inner automorphism of the group $\Br_n$ give by conjugation by $\fb_{n,p,i}^{-1}$.

We now appeal to the following standard fact: if a group $G$ acts on a set $X$ and if $g\in G$, then the homeomorphism $g\cdot-\colon X\qq G\to X\qq G$, induced by the action of $g$ on $X$ and by conjugation by $g^{-1}$ on $G$, is homotopic to the identity of $X\qq G$. In our case, the bottom horizontal map $\fb_{n,p,i}\cdot-$ in the last square is homotopic to the identity of $\cS^n_{p-1}$.

We conclude by gluing the two squares along their common vertical side, after noticing that the composition of the two top horizontal maps is precisely $\lst_i$.
\end{proof}
Note that the map $\lst_i\colon Q^{p+1}\times\Hur_{n-p-1}(Q)\to Q^p\times\Hur_{n-p}(Q)$, as a map with codomain a product, can be described by giving two maps $Q^{p+1}\times\Hur_{n-p-1}(Q)\to Q^p$ and $Q^{p+1}\times\Hur_{n-p-1}(Q)\to \Hur_{n-p}(Q)$. The first of these maps is the projection $Q^{p+1}\times\Hur_{n-p-1}(Q)\to Q^{p+1}$ followed by the map 
$Q^{p+1}\to Q^p$ given by $(a_0,\dots,a_p)\mapsto (a_0,\dots,\hat a_i,\dots,a_p)$. The second of these maps, restricted to the slice $(a_0,\dots,a_p)\times \Hur_{n-p-1}(Q)\cong \Hur_{n-p-1}(Q)$, is the stabilisation map $\lst(a_i^{a_{i+1}\dots a_p})$.

\subsection{Action of \texorpdfstring{$\Hur(Q)$}{Hur(Q)} on \texorpdfstring{$\cS_\bullet$}{Sdot}}
\begin{nota}
\label{nota:cSdot}
 Recall Notation \ref{nota:cSndot}. We denote by $\cS_\bullet$ the disjoint union of augmented semisimplicial spaces $\coprod_{n\ge0}\cS^n_\bullet$, again leaving $Q$ implicit.
\end{nota}

For all $n,m\ge0$ there is a map of augmented semisimplicial sets $\cA_{n,\bullet}\to \cA_{n+m,\bullet}$ given by ``adding $m$ points on the right'': more precisely, we send the isotopy class of the collection of arcs $\alpha_0,\dots,\alpha_p\colon[0,1]\to D$, representing a $p$-simplex in $\cA_{n,p}$, to the isotopy class of the collection of arcs $\chi^m_n\circ\alpha_0,\dots,\chi^m_n\circ\alpha_p\colon [0,1]\to D$, where $\chi^m_n\colon D\to D$ sends $(x,y)\mapsto (\frac{m+1}{m+n+1}x,y)$. We consider the action of $\Br_n\times\Br_m$ on $\cA_{n,\bullet}$ given by projecting $\Br_n\times\Br_m\to\Br_n$ and then letting $\Br_n$ act; then the map $\cA_{n,\bullet}\to \cA_{n+m,\bullet}$ is equivariant with respect to the actions of $\Br_n\times\Br_m$ on $\cA_{n,\bullet}$ and of $\Br_{n+m}$ on $\cA_{n+m,\bullet}$, along the inclusion
of groups $\Br_n\times \Br_m\subset\Br_{n+m}$.

Similarly, the identification of sets $Q^n\times Q^m\cong Q^{n+m}$ is equivariant with respect to the action of $\Br_n\times\Br_m$ and $\Br_{n+m}$ on the two sets, respectively. Taking cartesian products, we obtain a map of augmented semisimplicial sets 
\[
\cA_n(Q)_\bullet\times Q^m\to \cA_{n+m}(Q)_\bullet,
\]
which is equivariant with respect to the action of $\Br_n\times\Br_m$ and $\Br_{n+m}$ on the two augmented semisimplicial sets. Taking homotopy quotients, we finally obtain a map of augmented semisimplicial spaces
\[
 \cS^n_\bullet\times\Hur_m(Q)\to\cS^{n+m}_\bullet.
\]
The following is immediate.
\begin{prop}
\label{prop:rightaction}
 The maps $\cS^n_\bullet\times\Hur_m(Q)\to\cS^{n+m}_\bullet$ constructed above assemble into a right action of the topological monoid $\Hur(Q)$ on the augmented semisimplicial space $\cS_\bullet$.
\end{prop}

