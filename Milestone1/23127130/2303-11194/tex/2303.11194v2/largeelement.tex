\section{Large elements as total monodromy}
In this section we prove Theorem \ref{thm:main3} and Corollary \ref{cor:main3}. We fix a commutative ring $R$ throughout the section, as well as a homological degree $i\ge0$. We assume that $Q\subset G$ is a single conjugacy class and fix a large element $\omega\in G$ as in Definition \ref{defn:omegalarge}.

\subsection{The algebra \texorpdfstring{$C$}{C}}
In this subsection we prove Theorem \ref{thm:main3} assuming that $R$ is Noetherian. We fix a Noetherian ring $R$ throughout the subsection.
\begin{defn}
 We denote by $C$ the quotient of $B$ by the relations $[q_i]^\ell=[q_j]^\ell$ for all $1\le i,j\le m$.
\end{defn}
Notice that $C$ is also a quotient of $B_\omega$; as a ring, it is isomorphic to a weighted polynomial ring in a variable of weight $\ell$; for instance, one can take the image of $[a]^\ell$ in the quotient to be such a variable, for any $a\in Q$.

Let $\bar n\ge\ell $ be such that the finitely generated $B\otimes R$-module $H_i(\Hur(Q)_\omega;R)$ is generated by the direct sum $\bigoplus_{n=0}^{\bar n-\ell}H_i(\Hur_n(Q)_\omega;R)$: the existence of such $\bar n$ is guaranteed by the assumption that $R$ is Noetherian and Theorem \ref{thm:main2}. Then we can consider the direct sum
\[
H_i(\Hur_{\ge\bar n}(Q)_\omega;R):=\bigoplus_{n\ge\bar n} H_i(\Hur_n(Q)_\omega;R)
\]
as a sub-$B\otimes R$-module of $H_i(\Hur(Q)_\omega;R)$.
\begin{lem}\label{lem:HiCmodule}
 The action of $B\otimes R$ on $H_i(\Hur_{\ge\bar n}(Q)_\omega;R)$ factors through $C\otimes R$.
\end{lem}
\begin{proof}
 Let $n\ge\bar n$ and let $x\in H_i(\Hur_n(Q)_\omega;R)$ be a homology class of the form $[a]^\ell y$, for some $y\in H_i(\Hur_{n-\ell}(Q)_\omega;R)$; then for all $b\in \Q$ we have the following:
 \begin{itemize}
  \item $[b]^\ell x=[b^\omega]^\ell x$, by Proposition \ref{prop:HiBomegamodule};
  \item $[b]^\ell x=[b]^\ell[a]^\ell y=[b^a]^\ell[a]^\ell y=[b^a]^\ell x$, using the relations of $B$.
 \end{itemize}
 It follows that if $b,c\in Q$ can be obtained from another by a sequence of conjugations by $\omega^{\pm1}$ or $a^{\pm1}$, then $[b]^\ell x=[c]^\ell x$. Since $\omega$ is large, the group $\left<\omega,a\right>$ is the entire $G$, and by assumption $Q$ is a unique conjugacy class in $G$. We conclude for any homology class $x$ of the form $[a]^\ell y$ and for any $b,c\in Q$ we have $[b]^\ell x=[c]^\ell x$.
 
 The claim now follows from the definition of $\bar n$: for $n\ge\bar n$, every class in $H_i(\Hur_n(Q)_\omega;R)$ is a linear combination of classes of the form $[a]^\ell y$, for possibly different values of $a\in Q$ and $y\in H_i(\Hur_{n-\ell}(Q)_\omega;R)$.
\end{proof}
\begin{proof}[Proof of Theorem \ref{thm:main3} for $R$ Noetherian]
By Theorem \ref{thm:main2} $H_i(\Hur(Q)_\omega;R)$ is a finitely generated $A\otimes R$-module, and by Lemma \ref{lem:Anoetherian} we have that $A\otimes R$ is finitely generated over $B\otimes R$; it follows that $H_i(\Hur(Q)_\omega;R)$ is finitely generated over $B\otimes R$; since $B\otimes R$ is Noetherian, and since $H_i(\Hur_{\ge\bar n}(Q)_\omega;R)$ is a sub-$B\otimes R$-module of $H_i(\Hur(Q)_\omega;R)$, we obtain that also $H_i(\Hur_{\ge\bar n}(Q)_\omega;R)$ is finitely generated over $B\otimes R$. It is then a consequence of Lemma \ref{lem:HiCmodule} that $H_i(\Hur_{\ge\bar n}(Q)_\omega;R)$ is in fact a finitely generated $C\otimes R$-module.

Let now $a\in Q$; the ring $C\otimes R$ is isomorphic to the polynomial algebra over $R$ generated by $[a]^\ell$, and in particular also $C\otimes R$ is Noetherian. We conclude that $H_i(\Hur_{\ge\bar n}(Q)_\omega;R)$ is not only finitely generated, but also finitely presented. Choosing $\tilde n\ge\bar n$ such that $H_i(\Hur_{\ge\bar n}(Q)_\omega;R)$ admits a presentation in weights $\le\tilde n$, we obtain that for $n>\tilde n$ the multiplication $[a]^\ell\cdot-$ induces a bijection $H_i(\Hur_n(Q)_\omega;R)\cong H_i(\Hur_{n+\ell}(Q)_\omega;R)$; this multiplication coincides with $\lst(a)^\ell_*$.

We conclude by observing that, by Lemma \ref{lem:HiCmodule}, for $n\ge\bar n$ and for varying $a\in Q$, the stabilisation maps $\lst(a)_*^\ell\colon H_i(\Hur_n(Q)_\omega;R)\to H_i(\Hur_{n+\ell}(Q)_\omega;R)$ are equal to each other.
\end{proof}
We conclude the subsection by proving Theorem \ref{thm:main3} for a general ring $R$; so in the following proof we drop the hypothesis that $R$ is Noetherian.
\begin{proof}[Proof of Theorem \ref{thm:main3} in the general case]
Let $i\ge0$, let $G,Q,\omega$ as in the statement of Theorem \ref{thm:main3}, and let $a\in Q$. Then the proof of Theorem \ref{thm:main3} for integral homology has the following direct consequence: there is $\bar n\ge0$ such that for $n\le \bar n$ the map $\lst(a)^\ell\colon\Hur_n(Q)_\omega\to\Hur_{n+\ell}(Q)_\omega$ is an integral homology isomorphism in all homological degrees $\le i$; by the universal coefficient theorem for homology, it follows that for $n\ge\bar n$ the same map induces an isomorphism in $R$-homology in the same range of degrees, for any ring $R$.

In particular, the direct sum $\bigoplus_{n\le\bar n}H_i(\Hur_n(Q)_\omega;R)$ generates $H_i(\Hur(Q)_\omega;R)$ over $B\otimes R$; we can now repeat the argument of Lemma \ref{lem:HiCmodule}, and show that the sub-$B\otimes R$-module $\bigoplus_{n\ge\bar n+\ell}H_i(\Hur_n(Q)_\omega;R)$ is in fact a $C\otimes R$-module; this implies that for $n\ge\bar n+\ell$ the stabilisation map $\lst(a)^\ell_*\colon H_i(\Hur_n(Q)_\omega;R)\to H_i(\Hur_{n+\ell}(Q)_\omega;R)$ is independent of $a\in Q$.
\end{proof}

\subsection{Homology of the group completion}
We recall the group-completion theorem by McDuff and Segal \cite{McDuffSegal}
(see also \cite[Theorem Q.4]{FM94}).
\begin{thm}[Group-completion theorem]
\label{thm:groupcompletion}
Let $R$ be a ring and let $M$ be a topological monoid; suppose that the localisation $H_*(M;R)[\pi_0(M)^{-1}]$
can be constructed by right fractions. Then the canonical
map
\[
  H_*(M;R)[\pi_0(M)^{-1}]\to H_*(\Omega BM;R)
\]
is an isomorphism of rings.
\end{thm}
Lemmas \ref{lem:Gtwist} and \ref{lem:lstarstag} imply that the multiplicative set $\pi_0(\Hur(Q))$ of the ring $H_*(\Hur(Q);R)$ satisfies the Ore condition; hence the localisation
\[
H_*(\Hur(Q);R)[\pi_0(\Hur(Q))^{-1}]
\]
can be constructed by right fractions, and Theorem \ref{thm:groupcompletion} is thus applicable to compute $H_*(\Omega B\Hur(Q))$.
\begin{proof}[Proof of Corollary \ref{cor:main3}]
For a fixed homological degree $i\ge0$, the localisation of the $A\otimes R$-module $H_i(\Hur(Q);R)$ at the multiplicative subset $\pi_0(\Hur(Q))\subset A\otimes R$ coincides with the homology $H_i(\Omega B\Hur(Q);R)$. This localisation can be constructed by right fractions; moreover the multiplicative set $\pi_0(\Hur(Q))\subset A\otimes R$ is generated multiplicatively by the finite set of elements $[a]=[a]\otimes 1\in A\otimes R$.

We define $\fw=\hat q_1^\ell\dots \hat q_m^\ell\in \pi_0^\ell(\Hur(Q))$, and denote by $[\fw^{-1}]\in B\otimes R\subset A\otimes R$ the corresponding generator; this allows us to identify the module localisation $H_i(\Hur(Q);R)[\pi_0(\Hur(Q))^{-1}]$ with $H_i(\Hur(Q);R)[\fw^-1]$. Taking the zero component $\Omega_0 B\Hur(Q)$, we can identify  $H_i(\Omega_0 B\Hur(Q);R)$ with the colimit of the following sequential diagram, where
$\alpha$ is the chosen element in $\pi_0(\Hur(Q)_{\omega})$ (but for the following diagram any other $\alpha'\in\pi_0(\Hur(Q))$ would work):
\[
\begin{tikzcd}
 H_i(\Hur_\alpha(Q);R)\ar[r,"{[\fw]\cdot-}"]& H_i(\Hur_{\fw\alpha}(Q);R)\ar[r,"{[\fw]\cdot-}"]&H_i(\Hur_{\fw^2\alpha}(Q);R)\ar[r,"{[\fw]\cdot-}"] &\dots.
\end{tikzcd}
\]
The map $[\fw]\cdot-\colon H_i(\Hur(Q)_\omega)\to H_i(\Hur(Q)_\omega)$ can be written as a composition $\lst(q_m)^\ell_*\circ\dots\circ\lst(q_1)^m_*$, and by Theorem \ref{thm:main3}, for $n$ large enough and for fixed $a\in Q$, each of the maps $\lst(q_i)^\ell_*\colon H_i(\Hur_n(Q)_\omega;R)\to H_i(\Hur_{n+\ell}(Q)_\omega;R)$ is an isomorphism and coincides with $\lst(a)^\ell_*$, where $a\in Q$ is our fixed element. It follows that the sequential diagram above can be regarded as an ``index-$m$'' subdiagram of the following diagram
\[
 \begin{tikzcd}
 H_i(\Hur_\alpha(Q);R)\ar[r,"{[a]^\ell\cdot-}"]& H_i(\Hur_{\hat a^\ell\alpha}(Q);R)\ar[r,"{[a]^\ell\cdot-}"]&H_i(\Hur_{\hat a^{2\ell}\alpha}(Q);R)\ar[r,"{[a]^\ell\cdot-}"] &\dots.
\end{tikzcd}
\]
The last diagram stabilises by Theorem \ref{thm:main3}, and its colimit is $H_i(\Omega_0 B\Hur(Q);R)$. 
\end{proof}
