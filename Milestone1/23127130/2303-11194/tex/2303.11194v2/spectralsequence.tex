\section{The spectral sequence argument}
Let $R$ be a commutative ring.
Recall that for an augmented semisimplicial space $X_\bullet$ there is a spectral sequence in homology, with first page $E^1_{p,q}=H_q(X_p;R)$ and limit $H_{p+q+1}(X_{-1},|X|;R)$.
For $X=\cS^n_\bullet$ we get $E^1_{p,q}=H_q(\cS^n_p;R)$, and the limit is $H_{p+q+1}(\cS^n_{-1},|\cS^n_\bullet|;R)$, which is zero for all $p,q\in\Z$ by Lemma \ref{lem:damiolini}.

For $p,q\ge0$, the $E^1$-differential $d^1_{p,q}\colon E^1_{p,q}\to E^1_{p-1,q}$ is given by the map
\[
\sum_{j=0}^p (-1)^j(d_j)_*\colon H_q(\cS^n_p;R)\to H_q(\cS^n_{p-1};R),
\]
i.e. it is the alternating sum of the maps induced in homology by the face maps (and by the augmentation).

By Proposition \ref{prop:rightaction}, we can put together these spectral sequences for varying $n\ge0$, obtaining a spectral sequence of right $H_*(\Hur(Q);R)$-modules, and in particular of right $A\otimes R$-modules. We will use this to prove Theorem \ref{thm:main1}.


\subsection{The Koszul-like complex}\label{subsec:Koszulcomplex}
We aim at proving Theorem \ref{thm:main1} by induction on $i$. The statement is obvious for $i=0$, as the ring $A\otimes R$ is finitely generated over itself. We focus henceforth on the inductive step: we fix $\nu\ge0$, assume that $H_i(\Hur(Q);R)$ is finitely generated over $A\otimes R$ for all $0\le i\le \nu$, and aim at proving that $H_{\nu+1}(\Hur(Q);R)$ is also finitely generated over $A\otimes R$.

In the previous paragraph, all $A\otimes R$-modules are meant as \emph{left} $A\otimes R$-modules, as in the statement of Theorem \ref{thm:main1}. Yet the Pontryagin ring structure of $H_*(\Hur(Q);R)$ makes $H_i(\Hur(Q);R)$ into an $A\otimes R$-bimodule, i.e. $H_i(\Hur(Q);R)$ is endowed with compatible structures of left $A\otimes R$-module and right $A\otimes R$-module.

The following definition is taken from \cite[Subsection 4.1]{EVW:homstabhur} and slightly adapted to our purposes.
\begin{defn}
\label{defn:Kcomplex}
Let $M$ be an $A\otimes R$-bimodule. We define a chain complex $K_*(M)$ of right $A\otimes R$-modules, concentrated in degrees $*\ge-1$. We set $K_p(M)=RQ^{p+1}\otimes_R M$, where for a set $S$ we denote by $RS$ the free $R$-module generated by $S$. We put on $K_p(M)$ the right $A\otimes R$-module structure coming from $M$.

The differential $d_p\colon K_p(M)\to K_{p-1}(M)$ has the form $d_p=\sum_{j=0}^i(-1)^jd_{p,j}$, where $d_{p,j}$ sends
\[
(a_0,\dots,a_p)\otimes \mu\mapsto(a_0,\dots,\hat a_j,\dots,a_p)\otimes [a_i^{a_{i+1}\dots a_p}]\cdot \mu,
\]
for $(a_0,\dots,a_p)\in Q^{p+1}$ and $\mu\in M$.
\end{defn}
Our interest in the chain complex $K_*(H_q(\Hur(Q)))$ comes from the fact that it coincides with the $q$\textsuperscript{th} row of the $E^1$-page of the spectral sequence associated with the augmented semisimplicial space $\cS_\bullet$: this is a consequence of Proposition \ref{prop:descriptiondi} and the general description of the first page of the spectral sequence associated with the skeletal filtration of an augmented semisimplicial space.

In \cite{EVW:homstabhur}, $K_*(M)$ is referred to as a ``Koszul-like complex'', and in fact in \cite{ORW:Hurwitz} it is shown that $K_*(M)$ is isomorphic to the Koszul complex of the augmented dg ring $C_*(\Hur(Q);R)$ acting on the $A$-module $M$, where the action is given by the dg ring map $C_*(\Hur(Q);R)\to A\otimes R$ sending each 0-chain to its homology class and vanishing in higher degree, and where the augmentation $C_*(\Hur(Q);R)\to R$ is the composite of the previous dg ring map with the ring map $A\otimes R\to R$
killing the positive-weight part of $A\otimes R$.
In particular $H_i(K_*(M))\cong\Tor_i^{C_*(\Hur(Q);R)}(M,R)$.
See also \cite[Remark 7.2]{ETW:shufflealgebras}: the Koszul dual of the augmented dga $C_*(\Hur(Q);R)$, i.e. $\mathrm{Ext}^*_{C_*(\Hur(Q);R)}(R;R)$, can be identified with the quantum shuffle $R$-algebra generated by the braided $R$-module $\Hom_R(RQ,R)$, i.e. the $R$-linear dual of the free $R$-module $RQ$ generated by the set $Q$. The dual $R$-coalgebra can be canonically identified, as a graded $R$-module, with $\bigoplus_{i\ge0} (RQ)^{\otimes_R i}$.

\subsection{\texorpdfstring{$G$}{G}-twists} In the following we consider each left/right/bimodule over $A\otimes R$ as a left/right/bimodule over $A$ and over $R$ by considering the canonical ring homomorphisms $A\to A\otimes R$ and $R\to A\otimes R$.
\begin{defn}\label{defn:Gtwist}
Let $M$ be an $A\otimes R$-bimodule. A \emph{$G$-twist} on $M$ is a right action of $G$ on $M$, denoted $(m,g)\mapsto m^g$ for $m\in M$ and $g\in G$, satisfying the following: 
\begin{enumerate}
\item for all $r\in R$ we have $r\cdot m=m\cdot r$;
\item for all $a,b\in Q$ we have $([a]\cdot m)^b=[a^b]\cdot m^b$ and $(m\cdot [a])^b=m^b\cdot[a^b]$;
\item for all $a\in Q$ we have $m\cdot [a]=[a]\cdot m^{a}$.
\end{enumerate}
\end{defn}

For example, for $i\ge0$ the $A\otimes R$-bimodule $H_i(\Hur(Q);R)$ admits a $G$-twist as follows. The group $G$ acts on the right on the quandle $Q$ by conjugation: the element $g$ sends $a\in Q$ to $a^g=g^{-1}ag\in Q$. We can then let $G$ act diagonally on right on the set $Q^n$; as a matter of notation, the element $g\in G$ sends $(a_1,\dots,a_n)\mapsto(a_1,\dots,a_n)^g=(a_1^g,\dots,a_n^g)$. This action commutes with the left action of $\Br_n$ on $Q^n$, and thus it induces a right action of $G$ on $\Hur_n(Q)=Q^n\qq\Br_n$. For $i\ge0$, we can then take $i$\textsuperscript{th} homology with coefficients in $R$ and consider all values of $n$ at the same time: we obtain a right action of $G$ on the weighted $A\otimes R$-bimodule $H_i(\Hur(Q);R)$.
\begin{lem}\label{lem:Gtwist}
 The right action of $G$ on the $A\otimes R$-bimodule $H_i(\Hur(Q);R)$ is a $G$-twist.
\end{lem}
\begin{proof}
The action of $R$ on $H_*(\Hur(Q);R)$ comes from the multiplication of $R$, which we assume to be commutative: this ensures that condition (1) in Definition \ref{defn:Gtwist} is satisfied.
We observe moreover that $\Hur(Q)$ is a topological monoid and that the right action of $G$ on $\Hur(Q)$ is an action by automorphisms of topological monoids: indeed for all $n,n\ge0$, the concatenation map $Q^n\times Q^m\overset{\cong}{\to} Q^{n+m}$ is $G$-equivariant; taking homotopy quotients with respect to the groups $\Br_n\times\Br_m\subset\Br_{n+m}$, we obtain that the multiplication $\Hur_n(Q)\times\Hur_m(Q)\to\Hur_{n+m}(Q)$ is $G$-equivariant. Taking homology, we obtain that $G$ acts on right on $H_*(\Hur(Q);R)$ by automorphisms of rings, and this implies that condition (2) is satisfied.

Now let $n\ge0$ and let $a\in Q$ be fixed. We denote by $\bar\fb_{n+1}\in\Br_{n+1}$ the product of standard generators $\sigma_1\dots\sigma_n$; then we have a commutative square of sets, and a commutative square of groups
 \[
  \begin{tikzcd}
     Q^n\ar[r,"-\times a"] \ar[d,"{(-)^a}"'] & Q^{n+1}\ar[d,"\bar\fb_{n+1}\cdot-"]& &    \Br_n\ar[d,equal]\ar[r,"-\times\one_1"] &\Br_{n+1}\ar[d,"(-)^{\bar\fb_{n+1}^{-1}}"]\\
   Q^n\ar[r,"a\times -"'] & Q^{n+1},& & \Br_n\ar[r,"\one_1\times-"']&\Br_{n+1},
  \end{tikzcd}
 \]
such that each map of sets is equivariant with respect to the corresponding map of groups. Taking homotopy quotients, we obtain a square of spaces, that commutes on the nose
\[
 \begin{tikzcd}
  \Hur_n(Q)\ar[r,"\rst(a)"]\ar[d,"(-)^a"'] &\Hur_{n+1}(Q)\ar[d,"\bar\fb_{n+1}\cdot-"]\\
  \Hur_n(Q)\ar[r,"\lst(a)"']&\Hur_{n+1}(Q).
 \end{tikzcd}
\]
The right vertical map in the last diagram is homotopic to the identity; we conclude that $\rst(a)\colon\Hur_n(Q)\to\Hur_{n+1}(Q)$ is homotopic to the composition $\lst(a)\circ(-)^a$; taking $i$\textsuperscript{th} homology, we obtain that also condition (3) in Definition \ref{defn:Gtwist} is satisfied.
\end{proof}

\begin{lem}\label{lem:trivialaction}
 Let $M$ be an $A\otimes R$-bimodule with a $G$-twist, and let $j\ge0$. Then $H_j(K_*(M))$ is a trivial right $A\otimes R$-module, in the sense that right multiplication by any element of $A\otimes R$ of positive weight is zero.
\end{lem}
\begin{proof}
 This is similar to \cite[Lemma 4.11]{EVW:homstabhur}, but we repeat here the argument. Since the positive-weight ideal $(A\otimes R)_+\subset A\otimes R$ is generated by the elements $[a]=[a]\otimes 1$ for $a\in Q$, it suffices to prove that $-\cdot [a]$ induces the zero map on $H_j(K_*(M))$, and for this one proves that the chain map of chain complexes of abelian groups $-\cdot [a]\colon K_*(M)\to K_*(M)$ is chain homotopic to the zero chain map.

 One defines a chain homotopy $\cH_a\colon K_*(M)\to K_{*+1}(M)$ by sending
 \[
 \cH_a\colon (a_0,\dots,a_p)\otimes \mu\mapsto (-1)^{p+1}(a_0,\dots,a_p,a)\otimes \mu^a,
 \]
 for $p\ge-1$, $(a_0,\dots,a_p)\in Q^{p+1}$ and $\mu\in M$. Using property (1) from Definition \ref{defn:Gtwist}, one checks that $(d\circ \cH_a+\cH_a\circ d)((a_0,\dots,a_p)\otimes \mu)$ is equal to $(a_0,\dots,a_p)\otimes [a]\cdot \mu^a$; using property (2), one identifies the latter with $(a_0,\dots,a_p)\otimes \mu\cdot[a]$.
\end{proof}

\subsection{Proof of Theorem \ref{thm:main1}}
Let $R$ be a Noetherian ring.
We start by observing that if $M$ is an $A\otimes R$-bimodule that is finitely generated as a right $A\otimes R$-module, then $K_*(M)$ is a chain complex of finitely generated right $A\otimes R$-modules. By Lemma \ref{lem:Anoetherian} $A\otimes R$ is Noetherian, hence also $H_i(K_*(M))$ is a finitely generated right $A\otimes R$-module; it then follows from Lemma \ref{lem:trivialaction}
that $H_j(K_*(M))$ is finitely generated also as a module over the weight-zero subring $R\subset A\otimes R$, and in particular $H_j(K_*(M))$ vanishes in high enough weight, for all $j\ge-1$.
We also observe that if $M$ is an $A\otimes R$-bimodule with a $G$-twist, then $M$ is finitely generated as a left $A\otimes R$-module if and only if it is finitely generated as a right $A\otimes R$-module.

Recall now the inductive strategy that we have established at the beginning of Subsection \ref{subsec:Koszulcomplex}.
We can apply the above discussion applies to the modules $M=H_0(\Hur(Q);R),\dots, H_\nu(\Hur(Q);R)$, and
we obtain that for $n$ large enough, with in particular $n\ge \nu+3$, the spectral sequence associated with $\cS^n_\bullet$ vanishes on a large part of its $E^2$-page: we have $E^2_{p,q}=0$ for $q\le\nu$ and $p+q\le\nu+1$. This implies in particular that the differential $d^1_{0,\nu+1}\colon E^1_{0,\nu+1}\to E^1_{-1,\nu+1}$ must be surjective, otherwise we would have $E^2_{-1,\nu+1}\neq0$ and by the vanishing of $E^2_{p,q}$ for all $p\ge1$, $q\ge0$ with $p+q=\nu+1$ we would have no other non-trivial differential hitting $E^2_{-1,\nu+1}$ and forcing its vanishing on the $E^\infty$-page, as must happen by Lemma \ref{lem:damiolini}.

The differential $d^1_{0,\nu+1}$ can be identified with the map $RQ\otimes H_{\nu+1}(\Hur(Q);R)\to H_{\nu+1}(\Hur(Q);R)$ sending $q_i\otimes m\mapsto q_i\cdot m=(q_i\otimes 1)\cdot m$; if this map is surjective for $n$ large enough, say $n\ge\bar n$, then the left $A\otimes R$-module $H_{\nu+1}(\Hur(Q);R)$ is generated by the sub-abelian group $\bigoplus_{n=0}^{\bar n} H_{\nu+1}(\Hur_n(Q);R)$. We conclude the proof of Theorem \ref{thm:main1} by observing that the latter is a finite direct sum of finitely generated abelian groups: indeed, for each $n\ge0$, the space $\Hur_n(Q)$ has the homotopy type of a finite cover of $B\Br_n$, and $B\Br_n$ is homotopy equivalent to the $n$\textsuperscript{th} unordered configuration space of points in the plane, in particular it has the the homotopy type of a finite CW complex. This concludes the proof of Theorem \ref{thm:main1}.
