\section{Introduction}
Let $G$ be a finite group and let $Q\subseteq G$ be a conjugation-invariant subset.
For $n\ge0$, the Hurwitz space $\Hur_n(Q)$ considered in this article is
a certain homotopy quotient of the set $Q^n$ of $n$-tuples of elements in $Q$ by an action of the braid group $\Br_n$: see Definition \ref{defn:Hurwitz}. The homotopy type of $\Hur_n(Q)$ coincides with that of the moduli space of certain ``decorated'' branched covers of the complex plane $\C$; see Subsection \ref{subsec:connectionalggeo} for more details, and for the link between the Hurwitz spaces of this article and the Hurwitz spaces usually considered in algebraic geometry.

We are broadly interested in stability properties of homology of components of Hurwitz spaces. Suitable stability results for $H_i(\Hur_n(Q))$ can find applications in enumerative number theory: the main instance of this is are the work of Ellenberg-Venkatesh-Westerland on the Cohen-Lenstra heuristics \cite{EVW:homstabhur} and the work of Ellenberg-Tran-Westerland on the Malle conjecture \cite{ETW:shufflealgebras}.
The results of this article are an attempt to generalise the topological part of \cite{EVW:homstabhur}, as we do not require $Q\subseteq G$ to be a single conjugacy class with the ``non-splitting property'', and we consider homology in any Noetherian ring $R$; yet we have not been able to find a counterpart to the nice linear stability ranges from \cite{EVW:homstabhur}, making our results of a more qualitative than quantitative nature; in Subsection \ref{subsec:connectionalggeo} we briefly explain why this prevents applications of our results in enumerative number theory.

\subsection{Statement of results}
The disjoint union $\Hur(Q)=\coprod_{n\ge0}\Hur_n(Q)$ has a natural structure of topological monoid,\footnote{The multiplication is strict in the model of Hurwitz spaces that we use; in other natural models one would only have an $E_1$-algebra.} which is recalled in Subsection \ref{subsec:topmonoid}.
As a consequence, $H_*(\Hur(Q))=\bigoplus_{i\ge0,n\ge0}H_i(\Hur_n(Q))$ admits a natural structure of bigraded ring, where the two gradings are the homological degree $i$ and the ``weight'' $n$. In particular the direct sum
\[
 A=H_0(\Hur(Q))=\bigoplus_{n\ge0}H_0(\Hur_n(Q))
\]
admits a structure of graded ring, and for all $i\ge0$ and commutative ring $R$, the direct sum $H_i(\Hur(Q);R)=\bigoplus_{n\ge0}H_i(\Hur_n(Q);R)$ admits a structure of $A\otimes R$-bimodule, in particular of left $A\otimes R$-module. Our first main result is the following.
\begin{atheorem}
 \label{thm:main1}
 Let $R$ be a Noetherian commutative ring. For $i\ge0$, the module $H_i(\Hur(Q);R)$ is finitely generated over $A\otimes R=H_0(\Hur(Q);R)$.
\end{atheorem}
A quantitative consequence of the previous theorem is the following
corollary: see Definition \ref{defn:kGQ} for the constant $k(G,Q)\ge1$. We will also make use of the notion of quasi-polynomials, see Subsection \ref{subsec:quasipolynomials}.

\begin{acorollary}
\label{cor:main1}
 Let $\F$ be a field and let $i\ge0$. Let $\ell\ge1$ be such that $a^\ell=\one\in G$ for all $a\in Q$. Then there is a quasi-polynomial $p^\F_i(t)$
 of degree at most $k(G,Q)-1$ and period dividing $\ell$ such that, for $n$ large enough,
 $\dim_{\F}H_i(\Hur_n(Q);\F)=(p^\F_i)_{[n]_\ell}(n)$.
\end{acorollary}
We note that the condition $k(G,Q)=1$ is the ``non-splitting property'' for  the couple $(G,Q)$, introduced in \cite{EVW:homstabhur}: if $k(G,Q)=1$ then $Q$ is a single conjugacy class in $G$ and for every subgroup $H\subseteq G$, the intersection $Q\cap H$ is either empty or a single conjugacy class in $H$. In this case, $\dim_{\F}H_i(\Hur_n(Q);\F)$ is eventually equal to a quasi-polynomial in $n$ of degree $0$, i.e. it eventually coincides with a periodic function of $n$: this is an abstract version of homology stability, in the sense that for $n,n'$ large enough with respect to $i$, if $n-n'$ is divisible by $\ell$ then the vector spaces $H_i(\Hur_n(Q);\F)$ and $H_i(\Hur_{n'}(Q);\F)$ are abstractly isomorphic, but no explicit isomorphism is provided. We remark that in the case $k(G,Q)=1$, Theorem \ref{thm:main1} is proved in \cite{EVW:homstabhur}, under the additional hypotheses that homology is taken with coefficients in $\Q$ (or at least $\Z[\frac 1{|G|}]$): more precisely, for a suitable constant $D\ge1$ an explicit map $H_i(\Hur_n(Q);\Q)\to H_i(\Hur_{n+D}(Q);\Q)$ is constructed, and for suitable constants $A,B\ge1$ this map is shown to be an isomorphism provided that $n\ge Ai+B$, i.e. a linear stability range is proved.
The given stable isomorphism $H_i(\Hur_n(Q);\Q)\to H_i(\Hur_{n+D}(Q);\Q)$ is defined as a \emph{sum} of maps induced in homology by topological stabilisation maps $\Hur_n(Q)\to \Hur_{n+D}$.

The philosophy that stability results should be formulated in terms of finiteness results as modules over algebras was popularized by the field of representation stability (see e.g. \cite{CEF:FImodules, CEFN:FImodules}). As is common in representation stability, we leverage the fact that the algebra governing stability $A\otimes R$ is Noetherian. This helps us analyze a certain spectral sequence constructed using an arc complex introduced by Hatcher-Wahl \cite{hatcherwahl}. These spectral sequences also appeared in work of Ellenberg-Venkatesh-Westerland \cite{EVW:homstabhur}: our viewpoint on stability allows us to analyze this spectral sequence in new cases, in particular when the non-splitting property of $Q\subseteq G$ fails. We achieve this by considering at the same time several possible stabilisation maps (algebraically, this corresponds to working over polynomial rings in several variables), an idea appearing already in \cite{ADCK:edgestab,KMT:extremal}.

For $\omega \in G$ let us denote by $\Hur(Q)_\omega\subset\Hur(Q)$ the subspace characterised by the total monodromy being equal to $\omega$ (see Definition \ref{defn:totmon}).
The ring $A=H_0(\Hur(Q))$ contains a subring $A_\one=H_0(\Hur(Q)_\one)$, and
for $\omega \in G$, multiplication in $H_*(\Hur(Q))$ makes the submodule $H_i(\Hur(Q)_\omega)\subset H_i(\Hur(Q))$ into a left $A_\one$-module, for all $i\ge0$. Our second main result is the following.
\begin{atheorem}
\label{thm:main2}
Let $R$ be a Noetherian commutative ring. Let $i\ge0$ and let $\omega\in G$; then $H_i(\Hur(Q)_\omega;R)$ is finitely generated over $A_\one\otimes R=H_0(\Hur(Q)_\one;R)$.
\end{atheorem}
A quantitative consequence of the previous theorem is the following
(see Definition \ref{defn:kGQomega} for the constant $k(G,Q,\omega)\ge1$).
\begin{acorollary}
 \label{cor:main2}
 Let $\F$ be a field and let $i\ge0$. Let $\ell\ge1$ be such that $a^\ell=\one\in G$ for all $a\in Q$. Then there is a quasi-polynomial $p^\F_{i,\omega}$
 of degree at most $k(G,Q,\omega)-1$ and period dividing $\ell$ such that, for $n$ large enough,
 $\dim_{\F}H_i(\Hur_n(Q)_\omega;\F)=(p^\F_{i,\omega})_{[n]_\ell}(n)$.
\end{acorollary}
We compare Theorems \ref{thm:main1} and \ref{thm:main2} through the following example: let $G=\fS_d$ be the symmetric group on $d$ elements, and let $Q\subset G$ denote the conjugacy class of transpositions. Then $k(G,Q)=\floor{d/2}$, and for $d\ge4$ Theorem \ref{thm:main1} only predicts quasi-polynomial growth in $n$ of the Betti numbers $\dim_{\F}H_i(\Hur_n(Q);\F)$; in particular these Betti numbers can attain arbitrarily high values for $n\to\infty$.
However, if $\omega=(1,\dots,d)$ is the standard \emph{long cycle}, then $k(G,Q,\omega)=1$ and Theorem \ref{thm:main2} predicts that the Betti number $\dim_{\F}H_i(\Hur_n(Q)_\omega;\F)$ eventually coincides with a periodic function of $n$.

Our third, main result is a classical homological stability result for certain sequences of components of Hurwitz spaces.
\begin{atheorem}
 \label{thm:main3}
 Let $R$ be a commutative ring. Assume that $Q\subset G$ is a single conjugacy class and that there is an element $\omega\in G$ which is \emph{large} with respect to $Q$ (see Definition \ref{defn:omegalarge}). Let $i\ge0$, and let $\ell\ge1$ be such that $a^\ell=\one\in G$ for all $a\in Q$.
 For $a\in Q$ denote by $\lst(a)$ the map $\Hur(Q)\to\Hur(Q)$ induced by left multiplication by a point in $\Hur_1(Q)_a$; see also Definition \ref{defn:stabmap}.\footnote{In our model for Hurwitz spaces, $\Hur_1(Q)_a$ consists of precisely one point.}
Then for $n$ sufficiently large compared to $i$ the stabilisation map
\[
 \lst(a)^\ell_*\colon H_i(\Hur_n(Q)_\omega;R)\to H_i(\Hur_{n+\ell}(Q)_\omega;R)
\]
is independent of $a\in Q$ and is an isomorphism.
\end{atheorem}
Theorem \ref{thm:main3} applies for instance to the following settings:
\begin{itemize}
 \item $G=\fS_p$ for a prime number $p$, $Q$ is the conjugacy class of transpositions, and $\omega=(1,\dots,p)$ is the long cycle;
 \item $G=\Z/d\rtimes\Z/2$ for $d$ odd, $Q$ is the conjugacy class of involutions, and $\omega$ is a generator of $\Z/d$.
\end{itemize}
We remark that Tietz \cite[Theorem 2]{Tietz} also obtains a homology stability result (with an explicit linear stable range) for the integral homology of certain components of Hurwitz spaces, also generalising \cite{EVW:homstabhur}; even though our notion of ``large element of a group with respect to a conjugacy class'' seems comparable with his notion of ``collection of conjugacy classes that invariably generate a group'', we consider our result rather disjoint from his result.

Theorem \ref{thm:main3} can be applied together with the group-completion theorem in order to compute some stable homology groups, giving a partial, positive answer to \cite[Conjecture 1.5]{EVW:homstabhur}.
\begin{nota}
 For $\alpha\in\pi_0(\Hur(Q))$, we denote by $\Hur_\alpha(Q)\subseteq\Hur(Q)$ the connected component $\alpha$.
\end{nota}
\begin{acorollary}
\label{cor:main3}
Let $R$ be a commutative ring.
Let $G,Q,\omega$ be as in Theorem \ref{thm:main3}, and let $a\in Q$.
Fix $\alpha\in\pi_0(\Hur(Q)_\omega)\subseteq\pi_0(\Hur(Q))$.
Then for $n$ sufficiently large compared with $i$ the map induced by the group completion induces a homology isomorphism
\[
 H_i(\Hur_{\hat a^{\ell n}\alpha};R)\cong H_i(\Omega_0 B\Hur(Q);R),
\]
where $\Omega_0 B\Hur(Q)$ denotes the zero component of the group completion of $\Hur(Q)$.
\end{acorollary}
The rational homology of a component $\Omega_0 B\Hur(Q)$ of $\Omega B\Hur(Q)$ is $\Q$ in degrees 0 and 1, and 0 in all other degrees. This statement is first implicitly claimed in \cite[Conjecture 1.5]{EVW:homstabhur}, and a strategy of proof is contained in \cite[Subsection 5.6]{EVW:homstabhurII}, where it is shown that the statement follows from \cite[Theorem 2.8.1]{EVW:homstabhurII}. However, we believe that the proof of \cite[Theorem 2.8.1]{EVW:homstabhurII} contains a gap, which probably can be fixed. We refer to \cite[Corollary 5.4]{ORW:Hurwitz} and \cite[§6.5]{Bianchi:Hur3} for alternative proofs of the statement.

\begin{remark}
 After a first version of this article was circulated, Davis-Schlank \cite[Proposition 3.36]{DavisSchlank} proved that for a field $\F$ and for a finitely generated, weighted $A\otimes\F$-module $M$, the dimensions $\dim_\F(M_n)$ agree, for $n$ large, with a \emph{polynomial} in $n$ of degree bounded as in Corollary \ref{cor:main1}. Our Theorem \ref{thm:main1} allows to apply this result to $M=H_i(\Hur(Q);\F)$ and thus improve the statement of Corollary \ref{cor:main1} by replacing the word ``quasi-polynomial'' with the word ``polynomial''.
\end{remark}


\subsection{Connections to algebraic geometry and enumerative number theory}
\label{subsec:connectionalggeo}
The space $\Hur_n(Q)$ considered in this article is homotopy equivalent to the certain moduli space of regular branched covers of the complex plane $p\colon \cF\to\C$ endowed with the following data:
\begin{enumerate}
 \item an identification of the deck transformation group $\Aut(\cF,p)$ with $G$;
 \item a trivialisation over a suitable lower half-plane $\bH\subset\C$ of the restricted $G$-principal bundle $p^{-1}(\bH)\cong G\times \bH$, up to replacing $\bH$ by smaller and smaller half-planes,
\end{enumerate}
such that there are precisely $n$ branch points, and such that local monodromies around the branch points have values in $Q$. A branched cover is ``regular'' if its group of deck transformations acts transitively on each fibre; we use the term ``$G$-cover'' for a regular branched cover endowed with a decoration as (1) above.

Romagny-Wewers \cite{RomagnyWewers} describe how to construct, for $n\ge0$ and a finite group $G$, a scheme $\cH_{n,G}$ of finite type over $\Spec(\Z)$ whose associated analytic space $\cH_{n,G}(\C)$ can be identified with the moduli space of branched $G$-covers as above, but with $\bP^1_\C$ as target,
without the decoration (2), and without the restriction that local monodromies be in $Q$. The construction already appears in Wewers' PhD thesis \cite{Wewers}, and it is preceeded by work of Fulton \cite{Fulton} and of Fried-V\"olklein \cite{FriedVoelklein}: Fulton constructs for $d\ge1$ and $n\ge0$ a scheme $\cH_{d,n}$ of finite type over $\Spec(\Z)$ whose complex points isomorphism classes of degree-$d$ simple branched covers of $\bP^1_\C$ with $n$ branch points, and Fried-V\"olklein construct the basechange $\cH_{n,G}\times_{\Spec(\Z)}\Spec(\Q)$, which is of finite type over $\Spec(\Q)$.

Ellenberg-Venkatesh-Westerland \cite{EVW:homstabhur} show that at least when $Q\subset G$ is a \emph{rational} conjugation-invariant subset (that is, for all $a\in Q$ and all $n\ge1$ coprime with the order of $a$ in $G$, one has $a^n\in Q$), then Wewers' Hurwitz scheme $\cH_{n,G}$ contains a subscheme $\cH_{n,G}^Q$ whose complex points are isomorphism classes of branched $G$-covers of $\bP^1_\C$ with local monodromies lying in $Q$. Similarly, assuming again that $Q$ is rational, they show the existence of a subscheme $\Hn_{G.n}^Q$ of $\cH_{n+1,G}$
whose complex points are isomorphism classes of branched $G$-covers of $\C$ with local monodromies in $Q$.
The only missing ``decoration'' is (2) in the list above, namely a trivialisation of the $G$-cover over some lower half-plane; it follows that $\Hn^Q_{G,n}(\C)$ should be thought of as corresponding to the quotient $\Hur_n(Q)/G$, where $G$ acts by global conjugation on the set $Q^n$, the action is compatible with that of $\Br_n$, and hence we obtain an induced action of $G$ on $\Hur_n(Q)$. It would be interesting to know, without any restriction on $Q\subseteq G$, whether there exists a scheme $\tilde{\Hn}^Q_{G,n}$, possibly of finite type over $\Spec(\Z)$ (and ideally, an affine scheme) whose associated analytic space $\tilde{\Hn}^Q_{G,n}(\C)$ is homotopy equivalent to $\Hur_n(Q)$.

In \cite{EVW:homstabhur}, rational homological stability for $\Hur_n(Q)$ is proved under the non-splitting hypothesis, and an \emph{explicit linear stability range} is provided. This leads to an exponential upper bound on the size of the rational cohomology of $\Hur_n(Q)$ that does not depend on $n$: there is a constant $C>0$ such that in each degree $i\ge0$ one has $\dim_\Q H^i(\Hur_n(Q);\Q)<C^i$. This in turn leads to a similar upper bound on the size of $H^i(\Hur_n(Q)/G;\Q)\cong H^i(\Hn_{G,n}^Q(\C);\Q)$, when $Q$ is a rational conjugacy class of $G$. For a finite field $\F_q$ with algebraic closure $\bar\F_q$, the latter upper bound can be used to control the size of the \'etale cohomology of the basechange $\Hn_{G,n}^Q\times_{\Spec(\Z)}\Spec(\bar\F_q)$. One can then use the Grothendieck-Lefschetz trace formula to express the $|\Hn^Q_{G,n}(\F_q)|$ as the alternating sum of the traces of the $\F_q$-Frobenius acting on the \'etale cohomology of $\Hn_{G,n}^Q\otimes_{\Spec(\Z)}\Spec(\bar\F_q)$. The above bounds on the dimension of the \'etale cohomology groups, together with the Deligne bounds on the size of the eigenvalues of the Frobenius, give an estimate of $|\Hn^Q_{G,n}(\F_q)|$, and describe, at least for $q$ large, how $|\Hn^Q_{G,n}(\F_q)|$ grows for $n\to\infty$.

The lack of an explicit linear stability range in the results of this article seems to exclude the possibility to employ our results in a similar framework as that of \cite{EVW:homstabhur}.
However, if an explicit linear stable range could be established, these kinds of polynomial stability results plausibly would have implications for point count problems.



\subsection{Acknowledgments}
The first author would like to thank B\'eranger Seguin for a useful conversation on the algebraic-geometric theory of Hurwitz spaces, and Oscar Randal-Williams for a conversation about stability phenomena in the presence of several stabilisation maps. Both authors would like to thank Jordan Ellenberg and Craig Westerland for useful comments on a first draft of the article.

Andrea Bianchi was supported by the Danish National Research Foundation through the Centre for Geometry and Topology (DNRF151) and the European Research Council under the European Union Horizon 2020 research and innovation programme (grant agreement No. 772960).

Jeremy Miller was supported in part by NSF grant DMS-2202943 and a Simons Foundation Collaboration Grants for Mathematicians.


