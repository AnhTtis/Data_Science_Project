\section{Preliminaries}
\label{sec:preliminaries}
Throughout the article, $G$ denotes a finite group and $Q\subseteq G$ a conjugation-invariant subset.
We denote by $q_1,\dots,q_m$ the elements of $Q$, where $m=|Q|$, and we let $\ell\ge1$ be a positive integer such that $a^\ell=\one\in G$ for each $a\in Q$: for instance, we could take $\ell$ to be the least common multiple of the multiplicative orders in the group $G$ of the elements of $Q$.
\subsection{Hurwitz spaces as homotopy quotients}
For $n\ge0$, we denote by $\Br_n$ the Artin braid group on $n$ strands,
with generators $\sigma_1,\dots,\sigma_{n-1}$ satisfying the usual braid and commuting relations.
The group $\Br_n$ acts on the set $Q^n$ of $n$-tuples of elements of $Q$ as follows: for $1\le i\le n-1$, the standard generator $\sigma_i\in\Br_n$ sends
\[
 \sigma_i\colon (a_1,\dots,a_n)\mapsto (a_1,\dots,a_{i-1},a_{i+1},a_i^{a_{i+1}},a_{i+2},\dots,a_n),
\]
where, for $a,b\in Q$, we denote $a^b=b^{-1}ab\in Q$. The fact that the action is well-defined is a consequence of the fact that $Q$, with the operation of conjugation restricted from $G$, is a quandle.
The following is the definition of Hurwitz spaces that we are going to use throughout the article.
\begin{defn}
 \label{defn:Hurwitz}
For $n\ge0$, we define the Hurwitz space $\Hur_n(Q)$ as the homotopy quotient
\[
 \Hur_n(Q)=Q^n\qq\Br_n.
\]
\end{defn}
A priori, a homotopy quotient is only defined as a homotopy type; in this article we realise homotopy quotients by the standard bar construction; for example $\Hur_n(Q)$, as a concrete topological space, is the geometric realisation of the simplicial set $B_\bullet(*,\Br_n,Q^n)$.

\subsection{Components of Hurwitz spaces}
By definition, $\Hur_n(Q)$ is the homotopy quotient of the set $Q^n$ by the action of the discrete group $\Br_n$: it follows that $\pi_0(\Hur_n(Q))$ is in natural bijection with the set of orbits of the action of $\Br_n$ on the set $Q^n$.
\begin{defn}
 For $\ua=(a_1,\dots,a_n)\in Q^n$ we denote by $\Hur_n(Q,\ua)\subset\Hur_n(Q)$ the component corresponding to the orbit of $\ua$ under the action of $\Br_n$.
\end{defn}
The space $\Hur_n(Q,\ua)$ is aspherical.
Let us denote by $\Br_n\cdot\ua\subset Q^n$ the orbit of $\ua$ under the braid group action: then $\Hur_n(Q,\ua)$ is canonically homeomorphic to $(\Br_n\cdot \ua)\qq\Br_n=|B_\bullet(*,\Br_n,\Br_n\cdot\ua)|$, and the fundamental group of $(\Br_n\cdot \ua)\qq\Br_n$ based at the 0-simplex $\ua$ is canonically isomorphic to the subgroup of $\Br_n$ stabilising $\ua$, which we denote $\Br_n(\ua)\subseteq\Br_n$.

\subsection{Topological monoid structure}
\label{subsec:topmonoid}
For $n,m\ge0$, consider the standard concatenating map of sets
$Q^n\times Q^m\overset{\cong}{\to} Q^{n+m}$
and the standard concatenating map of braid groups $\Br_n\times\Br_m\hookrightarrow\Br_{n+m}$.
If we let $\Br_n\times\Br_m$ act on $Q^{n+m}$ through its inclusion into $\Br_{n+m}$, we have that the map $Q^n\times Q^m\overset{\cong}{\to} Q^{n+m}$ is $(\Br_n\times\Br_m)$-equivariant; we also say that the map of sets $Q^n\times Q^m\overset{\cong}{\to} Q^{n+m}$ is equivariant with respect to the map of groups $\Br_n\times\Br_m\hookrightarrow\Br_{n+m}$. This equivariance gives rise to a map between the homotopy quotients
\[
 \Hur_n(Q)\times\Hur_m(Q)\cong (Q^n\times Q^m)\qq(\Br_n\times\Br_m)\to Q^{n+m}\qq\Br_{n+m}=\Hur_{n+m}(Q),
\]
and these maps assemble into a topological monoid structure on the disjoint union
\[
\Hur(Q)=\coprod_{n\ge0}\Hur_n(Q),
\]
with unit given by the unique point of $\Hur_0(Q)$.

Taking connected components, we obtain a discrete monoid $\pi_0(\Hur(Q))$; if we denote by $\hat q_i$ the connected component of $\Hur(Q)$ given by the (contractible) space $\Hur_1(Q,q_i)$, for all $q_i\in Q$, we have the following presentation by generators and relations of $\pi_0(\Hur(Q))$ as an associative, unital monoid:
\[
 \pi_0(\Hur(Q))\cong\left<\hat q_1,\dots, \hat q_m\  |\  \hat q_i\cdot\hat q_j=\hat q_j\cdot \widehat{q_i^{q_j}}\right>.
\]
\begin{defn}\label{defn:pi0ell}
 We denote by $\pi_0^\ell(\Hur(Q))\subseteq\pi_0(\Hur(Q))$ the unital submonoid generated by the elements $\hat q_j^\ell$ for $1\le i\le m$.
\end{defn}
\begin{lem}\label{lem:relationsB}
The monoid $\pi_0^\ell(\Hur(Q))$ is commutative, and for all $1\le i,j\le m$ the following relation holds in $\pi_0^\ell(\Hur(Q))$:
\[
 \hat q_i^\ell\cdot\hat q_j^\ell=\widehat{q_i^{q_j}}^\ell\cdot\hat q_j^\ell
\]
\end{lem}
\begin{proof}
 We note that the elements $\hat q_i^\ell$ are central elements of the monoid $\pi_0(\Hur(Q))$: indeed for all $j$ we have $\hat q_j\cdot \hat q_i^\ell=\hat q_i^\ell\cdot \widehat{q_j^{q_i^\ell}}=\hat q_i^\ell\cdot\hat q_j$. This implies in particular that $\pi_0^\ell(\Hur(Q))$ is a commutative monoid.
 
 Moreover for all $i,j$ we have in $\pi_0(\Hur(Q))$ the equality $\hat q_i^\ell\cdot\hat q_j=\hat q_j\cdot \widehat{q_i^{q_j}}^\ell$, which by the previous argument is also equal to $\widehat{q_i^{q_j}}^\ell\cdot \hat q_j$; multiplying both sides on right by $\hat q_j^{\ell-1}$ we obtain the described relations among the generators of $\pi_0^\ell(\Hur(Q))$.
\end{proof}



\subsection{Constants attached to finite groups}
The following constants, attached to a group $G$, a conjugation-invariant subset $Q\subset G$, and an element $\omega\in G$, will be used to bound the degree of the quasi-polynomials that govern the growth of the homology of Hurwitz spaces.
\begin{defn}
\label{defn:kGQ}
 We denote by $k(G,Q)\ge1$ the maximum, for $H\subseteq G$ ranging among all subgroups of $G$, of the number of conjugacy classes in $H$ in which the conjugation-invariant subset $Q\cap H\subseteq H$ decomposes.
\end{defn}
\begin{defn}
\label{defn:kGQomega}
We denote by $k(G,Q,\omega)\ge1$ the maximum, for $H\subseteq G$ ranging among all subgroups of $G$ containing $\omega$, of the number of conjugacy classes in $H$ in which the conjugation-invariant subset $Q\cap H\subseteq H$ decomposes.
\end{defn}

\begin{defn}
\label{defn:omegalarge}
 Let $\omega\in G$ and assume that $Q\subset G$ is a single conjugacy class. We say that $\omega$ is \emph{large} with respect to $Q$ if for every $a\in Q$ the elements $\omega$ and $a$ generate $G$.
\end{defn}
Note that if $\omega\in G$ is large with respect to a conjugacy class $Q\subset G$, then in particular $k(G,Q,\omega)=1$. As an example, let $p$ be a prime number, let $G=\fS_p$ and let $Q$ be the conjugacy class of transpositions: then the long cycle $(1,\dots,p)$ is large with respect to $Q$. Another example is the following: let $d$ be an odd number and let $G=\Z/d\rtimes\Z/2$ be the $d$\textsuperscript{th} dihedral group; let $Q$ be the conjugacy class of involutions; then any generator of $\Z/d\subset G$ is large with respect to $Q$.


\subsection{Invariants of components}\label{subsec:invcomponents}
The action of $\Br_n$ on $Q^n$ preserves the following invariants defined on the set $Q^n$:
\begin{itemize}
 \item the \emph{total monodromy}: this invariant associates with an $n$-tuple $(a_1,\dots,a_n)$ the product $\omega:=a_1\dots a_n\in G$;
 \item the \emph{image subgroup}: this invariant associates with $(a_1,\dots,a_n)$ the subgroup $H:=\left<a_1,\dots,a_n\right>\subseteq G$;
 \item the \emph{conjugacy class partition}: let $(a_1,\dots,a_n)\in Q^n$ and let $H=\left<a_1,\dots,a_n\right>$ as above; let $Q_1,\dots,Q_s\subset Q\cap H$ be the conjugacy classes in $H$ in which the conjugation invariant subset $Q\cap H$ splits; this invariant associates with $(a_1,\dots,a_n)$ the splitting $n=n_1+\dots+n_s$, where $n_i$ is the cardinality of $\set{a_1,\dots,a_n}\cap Q_i$. This is called the ``multidiscriminant'' in \cite{EVW:homstabhurII}.
\end{itemize}
Each of the above invariant gives rise to an invariant of connected components of $\Hur(Q)$; we will introduce notation only for the first invariant.
\begin{defn}
\label{defn:totmon}
 For $\omega\in G$ and for $n\ge0$ we denote by $\Hur_n(Q)_\omega\subset\Hur_n(Q)$ the union of connected components corresponding to $n$-tuples $(a_1,\dots,a_n)$ with total monodromy $\omega$.
 Similarly, we denote $\Hur(Q)_\omega=\coprod_{n\ge0}\Hur_n(Q)_\omega$.
\end{defn}

\subsection{Quasi-polynomials}\label{subsec:quasipolynomials}
 Let $\ell\ge1$, and for $n\in\Z$ denote by $[n]_\ell\in\Z/\ell$ the class of $n$ modulo $\ell$. We will use the following notion of quasi-polynomial.
\begin{defn}
A quasi-polynomial of period dividing $\ell$, denoted $p_{[-]_\ell}(t)$, is the datum of $\ell$ polynomials $p_{[1]_\ell}(t),\dots,p_{[\ell]_\ell}(t)\in\Q[t]$.

The degree of a non-zero quasi-polynomial is the maximum among the degrees of the polynomials $p_{[i]_\ell}(t)$; the degree of the zero quasi-polynomial is set to be $-\infty$.

A quasi-polynomial induces a function $\Z\to\Q$, given by sending $n\mapsto p_{[n]_\ell}(n)$.
\end{defn}
As one can see, the argument $n\in\Z$ must be input twice, once as index of the polynomial, once as value of the variable $t$, in order to evaluate a quasi-polynomial at $n$.
