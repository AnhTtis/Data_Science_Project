\section{Quasi-polynomial growth of Betti numbers}
In this section we analyse the growth of the homology groups $H_i(\Hur_n(Q);\F)$, where $\F$ is a field, for fixed $i$ and increasing $n$, and prove Corollaries \ref{cor:main1}, \ref{cor:main2}.
\subsection{Growth of weighted dimensions of \texorpdfstring{$B$}{B} and \texorpdfstring{$A$}{A}}
We first analyse the growth of the weighted dimensions (over $\Z$, aka rank) of $B$ and $A$. Recall that $\pi_0(\Hur(Q))=\pi_0(\coprod_{n\ge0}\Hur_n(Q))=\coprod_{n\ge0}\pi_0(\Hur_n(Q))$ is a weighted set; also $\pi_0^\ell(\Hur(Q))$ is a weighted set, concentrated in weights that are multiple of $\ell$. Similarly, $A$ and $B$ are weighted rings, with $B$ commutative and concentrated in weights multiple of $\ell$. Notice also that both $A$ and $B$ are free as $\Z$-modules.
\begin{prop}
 \label{prop:Agrowth}
 There is a polynomial $p_B(t)\in\Q[t]$ of degree $k(G,Q)-1$ such that, for $n$ large enough, $\dim_\Z B_{\ell n}=p_B(n)$.
 
 Similarly, there is a quasi-polynomial $p_A(t)\in\Q^\Z(t)$ of degree $k(G,Q)-1$ and period dividing $\ell$ such that $\dim_\Z A_{n}=p_A|_n(n)$ for $n$ large enough.
\end{prop}
\begin{proof}
In the proof we set $k=k(G,Q)$.
Fix a field $\F$ and let $B_\F=B\otimes \F$; then $\dim_\Z B_{\ell n}=\dim_\F B_{\F,\ell n}$, since $B$ is a free $\Z$-module. The $\F$-algebra $B_\F$ is a quotient of the weighted polynomial ring $\F[x_1,\dots,x_m]$, with variables $x_i$ put in weight $\ell$: this is witnessed by the surjective map of $\F$-algebras $\F[x_1,\dots,x_m]\twoheadrightarrow B_\F$ sending $x_i\mapsto[q_i]^\ell$. By the classical theory of the Hilbert function of a finitely generated $\F$-algebra, there is a polynomial $p_B(t)\in\Q[t]$ of degree at most $m-1$ such that, for $n$ large enough, $\dim_\F B_{\F,\ell n}=p_B(n)$.

We now want to argue that the degree of $p_B(t)$ is precisely $k-1$. For this, we will show that there exist polynomials $p^-_B(t)\in\Q[t]$ and $p^+_B(t)\in\Q[t]$ of degree $k-1$ such that, for $n$ large enough, $p^-_B(n)\le |\pi_0^\ell(\Hur(Q))_{n\ell}|\le p^+_B(n)$.
\begin{itemize}
 \item {\bf Lower bound.} Let $H\subseteq G$ be a subgroup such that $H\cap Q$ is the union of precisely $k$ conjugacy classes of $H$. Let $H'$ be the subgroup of $H$ generated by $H\cap Q$; then $Q\cap H'=Q\cap H$ also splits as a union of at least, hence precisely $k$ conjugacy classes, so we may assume to have chosen $H=H'$ at the beginning. Let $Q\cap H=\set{a_1,\dots,a_r}$ be the list of all elements in $Q\cap H$, and let $J=\set{b_1,\dots,b_k}\subset Q\cap H$ be a set representatives of the $k$ conjugacy classes in $Q\cap H\subseteq H$. For $n\ge r$ and for every splitting $n=r+n_1+\dots+n_k$, with $n_1,\dots,n_k\ge0$, we can form an element
 \[
  \hat a_1^\ell\cdot\dots\cdot\hat a_r^\ell\cdot \hat b_1^{n_1\ell}\cdot\dots\cdot\hat b_k^{n_k\ell}\in\pi_0^\ell(\Hur(Q))_{n\ell};
 \]
these elements are all distinct, for different choices of $n_1,\dots,n_k$, as witnessed by the \emph{conjugacy class partition} invariant from Subsection \ref{subsec:invcomponents}. Since there are $\binom{n-r+k-1}{k-1}$ choices for the splitting $n=r+n_1+\dots+n_k$, we can define $p_B^-(t)=\binom{t-r+k-1}{k-1}$ and obtain $|\pi_0^\ell(\Hur(Q))_{n\ell}|\ge p_B^-(n)$ for $n\ge r$.
\item {\bf Upper bound.} Let $\hat a_1^\ell\cdot\dots\cdot\hat a_n^\ell$ be an element in $\pi_0^\ell(\Hur(Q))_{n\ell}$; we want to construct a ``normal form'' for this element. First, up to reordering the factors $\hat a_i^\ell$, we may assume that there is $1\le r\le n$ such that the elements $a_1,\dots,a_r\in Q$ are all distinct, and such that for all $r+1\le j\le n$ there is $1\le i\le r$ with $a_j=a_i$.
It follows that the subgroup $H=\left<a_1,\dots,a_n\right>\subseteq G$, i.e. the \emph{image subgroup} invariant of the chosen element in $\pi_0^\ell(\Hur(Q))_{n\ell}$, can also be described as $\left<a_1,\dots,a_r\right>\subseteq G$. Let now $Q_1,\dots, Q_s$ be the conjugacy classes in $H$ in which $Q\cap H$ splits, and fix representatives $b_1\in Q_1,\dots,b_s\in Q_s$. Using repeatedly the relations from Lemma \ref{lem:relationsB} with $q_j$ chosen among $a_1,\dots,a_r$ and $q_i$ chosen among $a_{r+1},\dots,a_n$, we can achieve the situation in which each of $a_{r+1},\dots,a_n$ is one of $b_1,\dots,b_s$. Thus we have proved that each element in $\pi_0^\ell(\Hur(Q))_{n\ell}$ can be written as $\hat a_1^\ell\cdot\dots\cdot\hat a_r^\ell\cdot \hat b_1^{\ell n_1}\cdot\dots\cdot\hat b_s^{\ell n_s}$, for some $r\le m$, $a_1,\dots,a_r$ distinct elements of $Q$, and $n_1+\dots+n_s=n-r$. Making a very rough estimate, there are $2^m$ subsets in $Q$, among which one can choose $\set{a_1,\dots,a_r}$, and for each choice there are at most $n^{k-1}$ ways to choose the numbers $n_1,\dots,n_s$, since $s\le k$ and each of $n_1,\dots,n_{s-1}$ is $\le n$, and the last number $n_s$ is forced by the sum condition. Setting $p_B^+(t)=2^mt^{k-1}$, we have $|\pi_0^\ell(\Hur(Q))_{n\ell}|\le p_B^+(n)$ for all $n\ge0$.
\end{itemize}
This concludes the proof of the existence of $p_B(t)$ of degree $k-1$.
To prove that there is similarly a quasi-polynomial $p_A(t)$ such that $p_A|_n(n)=\dim_\Z(A_n)$, we use that $A$ is finitely generated as a $B$-module and invoke again the classical theory of the Hilbert function of a finitely generated graded
module over a graded algebra of finite type: this in particular ensures that the degree of $p_A(t)$ is at most $k-1$, and that the period divides $\ell$ (as $B$ is concentrated in degrees multiple of $\ell$). Finally, since $B\subseteq A$, we have that $\dim_\Z(A_{n\ell})\ge\dim_\Z(B_{n\ell})=p_B(n)$ grows in $n$ at least as a polynomial of degree $k-1$: this forces the degree of $p_A(t)$ to be $k-1$.
\end{proof}
Similarly as in Proposition \ref{prop:Agrowth}, one can show that the dimension of $(A_\one)_n$ grows like a quasi-polynomial of degree precisely $k(Q,G)-1$. This result has been proved independently by S\'eguin \cite{Seguin}, who has studied extensively the $\F$-algebra $A_\one\otimes\F$, for $\F$ a field of characteristic coprime with $|G|$.

We observe that Proposition \ref{prop:Agrowth} establishes Corollary \ref{cor:main1} for $i=0$; the general case will use Theorem \ref{thm:main1}.
\begin{proof}[Proof of Corollary \ref{cor:main1}]
 Let $i\ge0$ and fix a field $\F$; by Theorem \ref{thm:main1} the left $A\otimes\F$-module $H_i(\Hur(Q);\F)$ is finitely generated; recall also Definition \ref{defn:B} and Lemma \ref{lem:Anoetherian}, and observe that, since $A\otimes\F$ is finitely generated as a $B\otimes\F$-module, we have that $H_i(\Hur(Q);\F)$ is also finitely generated as a $B\otimes\F$-module. We now appeal again to the  
 classical theory of the Hilbert function of a finitely generated graded module over a graded algebra of finite type, together with the fact that $B\otimes \F$ is concentrated in weights multiple of $\ell$ and has weighted dimension given by a polynomial of degree $k(G,Q)$.
\end{proof}

\subsection{Proof of Corollary \ref{cor:main2}}
We fix an element $\omega\in G$ and $i\ge0$ throughout the subsection.
\begin{defn}\label{defn:Bomega}
 We denote by $B_\omega$ the quotient of $B$ by the ideal generated by the elements $[q_i]^\ell-[q_i^\omega]^\ell$.
\end{defn}
We observe that $B_\omega$ is the monoid ring of the quotient of the abelian monoid $\pi_0^\ell(\Hur(Q))$ by the relations $\hat q_i^\ell=\widehat{q_i^{\omega}}^\ell$; in particular $B_\omega$ is again a weighted ring and it is free as a $\Z$-module. Similarly, $B_\omega\otimes R$ is free as an $R$-module, for any commutative ring $R$.

\begin{lem}\label{lem:lstarstag}
 Let $g\in G$ and let $a\in Q$; then the maps $\lst(a)\colon\Hur(Q)_g\to\Hur(Q)_{ag}$ and $\rst(a^g)\colon\Hur(Q)_g\to\Hur(Q)_{ag}$ are homotopic.
\end{lem}
\begin{proof}
 The argument is similar to the one in the proof of Lemma \ref{lem:Gtwist}. We fix $n\ge0$ and prove that $\lst(a)$ and $\rst(a^g)$ are homotopic as maps $\Hur_{n}(Q)_g\to\Hur_{n+1}(Q)_{ag}$.
 Let $\tilde\fb_{n+1}\in\Br_{n+1}$ be the product of standard generators $\sigma_n\dots\sigma_1$. Denote by $Q^n_g$ the subset of $Q^n$ of sequences $(a_1,\dots,a_n)$ with total monodromy $a_1\dots a_n=g$, and define similarly $Q^{n+1}_{ag}$; then we have commutative triangles of sets and groups
 \[
  \begin{tikzcd}
     Q^n_g\ar[r,"a\times -"]\ar[dr,"-\times a^g"']  & Q^{n+1}_{ag}\ar[d,"\tilde\fb_{n+1}\cdot-"]& &   \Br_n \ar[dr,"-\times\one_1"']\ar[r,"\one_1\times-"] &\Br_{n+1}\ar[d,"(-)^{\tilde\fb_{n+1}^{-1}}"]\\
    & Q^{n+1}_{ag},& & &\Br_{n+1},
  \end{tikzcd}
 \]
such that each map of sets is equivariant with respect to the corresponding map of groups. Taking homotopy quotients, we obtain a commutative triangle of spaces
\[
 \begin{tikzcd}
  \Hur_n(Q)_g\ar[r,"\lst(a)"]\ar[dr,"\rst(a^g)"'] &\Hur_{n+1}(Q)_{ag}\ar[d,"\tilde\fb_{n+1}\cdot-"]\\
  &\Hur_{n+1}(Q)_{ag},
 \end{tikzcd}
\]
and we conclude by noticing that the right vertical map in the last diagram is homotopic to the identity.
\end{proof}

In the following proposition it is helpful to notice that if $M$ is an $A\otimes R$-bimodule with a $G$-twist (see Definition \ref{defn:Gtwist}), then the left and the right actions of $B\otimes R$ on $M$ coincide.

\begin{prop}\label{prop:HiBomegamodule}
Let $R$ be a commutative ring; then the action of $B\otimes R$ on $H_i(\Hur(Q)_\omega;R)$ factors through $B_\omega\otimes R$.
\end{prop}
\begin{proof}
It suffices to prove that for all $a\in Q$, the maps
\[
-\cdot[a]^\ell,\  -\cdot[a^\omega]^\ell\colon H_i(\Hur(Q)_\omega;R)\to H_i(\Hur(Q)_\omega;R)
\]
coincide; the previous maps can be identified with the maps induced in homology by the maps of spaces
\[
\rst(a)^\ell,\ \rst(a^\omega)^\ell\colon \Hur(Q)_\omega\to\Hur(Q)_\omega,
\]
and hence it will suffice to prove that these two maps are homotopic. As shown in the proof of Lemma \ref{lem:Gtwist}, the map $\rst(a)\colon\Hur(Q)\to\Hur(Q)$ is homotopic to the composition $\lst(a)\circ(-)^a$; we also notice that $\lst(a)\circ(-)^a=(-)^a\circ\lst(a)$, as a consequence of the fact that $a^a=a$. Taking $\ell$-fold compositions, we obtain that $\rst(a)^\ell$ is homotopic to $((-)^a)^\ell\circ\lst(a)^\ell$; we then notice that $((-a)^a)^\ell$ coincides with $(-)^{a^\ell}$, which is the identity of $\Hur(Q)$. Thus we have shown that $\rst(a)^\ell$ is homotopic to $\lst(a)^\ell$ as a map $\Hur(Q)\to\Hur(Q)$, and in particular the restricted maps $\Hur(Q)_\omega\to \Hur(Q)_\omega$ are homotopic.

By Lemma \ref{lem:lstarstag} we moreover have that the map $\lst(a)^\ell\colon\Hur(Q)_\omega\to\Hur(Q)_\omega$ is homotopic to the composition $\rst(a^{a^{\ell-1}\omega})\circ\dots\circ\rst(a^{a\omega})\circ\rst(a^\omega)$; since $a^{a^i}=a$ for all $i\ge0$, we also have $a^{a^i\omega}=a^\omega$, and eventually we obtain that $\lst(a)^\ell$ is homotopic to $\rst(a^\omega)^\ell$.
\end{proof}

We can now analyse the growth of the weighted dimension of the algebra $B_\omega$, in a similar way as we did for $B$ in Proposition \ref{prop:Agrowth}.
\begin{prop}\label{prop:Bomegagrowth}
 There is a polynomial $p_{B_\omega}(t)\in\Q[t]$ of degree $\le k(G,Q,\omega)-1$ such that, for $n$ large enough, $\dim_\Z (B_\omega)_{\ell n}=p_{B_\omega}(n)$.
\end{prop}
\begin{proof}
 The existence of a polynomial $p_{B_\omega}(t)$, possibly of degree $\ge k(G,Q,\omega)$, but satisfying $\dim_\Z (B_\omega)_{\ell n}=p_{B_\omega}(n)$ for $n$ large enough, is guaranteed by the classical theory of the Hilbert function of a finitely generated $\F$-algebra, after tensoring $B_\omega$ by some field $\F$; in fact, since $B_\omega$ is a quotient of $B$, we immediately obtain that the degree of $p_{B_\omega}(t)$ is at most the degree of $p_B(t)$, which is $k(G,Q)$.
 
 We now want to improve the upper bound on the degree of $p_{B_\omega}(t)$ to $K(G,Q,\omega)$. Recall from the proof of the upper bound in Proposition \ref{prop:Agrowth} that for $n\ge m=|Q|$, the abelian group $B_{\ell n}$ is generated by the products $[a_1]^\ell\dots[a_r]^\ell[b_1]^{\ell n_1}\dots[b_s]^{\ell n_s}$, for varying choices of:
 \begin{itemize}
  \item an integers $0\le r\le m=|Q|$;
  \item elements $a_1,\dots,a_r\in Q$;
  \item a partition $n-r=n_1+\dots+n_s$,
 \end{itemize}
 where we set $H=\left<a_1,\dots,a_r\right>\subseteq G$ (depending on the choice of $r$ and $a_1,\dots,a_r$), we set $s\le k(G,Q)$ to be the number of conjugacy classes of $Q\cap H$ in $H$, and we let $b_1,\dots,b_s$ be a system of representatives of these conjugacy classes (depending on $H$; we can choose a priori such a system of representatives for any subgroup $H\subseteq G$).

Fix now $0\le r\le m$ and elements $a_1,\dots,a_r\in Q$, and let $H$, $s$ and $b_1,\dots,b_s$ be as above. Let also $L=\left<H,\omega\right>\subseteq G$ and let $P_1,\dots,P_u$ be the conjugacy classes in which the conjugation invariant subset $Q\cap L$ of $L$ splits, for some $0\le u\le k(G,Q,\omega)$; finally, let $c_1,\dots,c_u\in Q$ be representatives. Each element $b_i$ belongs to $L\cap Q$, and inside $L$ is conjugate to some element $c_{\iota(i)}$, for a suitable function $\iota\colon\set{1,\dots,s}\to\set{1,\dots,u}$; using the relations of $B$ together with the additional relations of $B_\omega$ from Definition \ref{defn:Bomega}, we have in $B_\omega$ the equality
\[
[a_1]^\ell\dots[a_r]^\ell[b_1]^{\ell n_1}\dots[b_s]^{\ell n_s}=
[a_1]^\ell\dots[a_r]^\ell[c_{\iota(1)}]^{\ell n_1}\dots[c_{\iota(s)}]^{\ell n_s}.
\]
It follows that $(B_\omega)_{\ell n}$ is generated by all products $[a_1]^\ell\dots[a_r]^\ell[c_1]^{\ell \nu_1}\dots[c_u]^{\ell \nu_u}$, for varying $r\ge0$, $a_1,\dots,a_r\in Q$, and partitions $n-r=\nu_1+\dots+\nu_u$, where $0\le u\le k(G,Q,\omega)$ and $c_1,\dots,c_n\in Q$ depend on $r\ge0$ and $a_1,\dots,a_r\in Q$. Again by a very rough estimate, there are at most $2^mn^{k(G,Q,\omega)-1}$ such products, and this proves that $\pi_{B_\omega}(n)\le 2^mn^{k(G,Q,\omega)-1}$ for $n$ large enough, in particular the degree of $\pi_{B_\omega}(t)$ is at most $k(G,Q,\omega)-1$.
\end{proof}
Since $B_\omega$ is free as an abelian group, for any field $\F$ we have  $\dim_\F (B_\omega\otimes \F)_{\ell n}= \dim_\Z (B_\omega)_{\ell n}$.
\begin{proof}[Proof of Corollary \ref{cor:main2}]
This is now analogous to the proof of Corollary \ref{cor:main1}. For a field $\F$ and $i\ge0$, by Theorem \ref{thm:main2} we have that $H_i(\Hur(Q);\F)$ is a finitely generated $A_\one\otimes\F$-module, and by Lemma \ref{lem:Anoetherian} we have that $A_\one\otimes\F$ is finitely generated over $B\otimes \F$; it follows that $H_i(\Hur(Q);\F)$ is a finitely generated $B\otimes\F$-module, and since by Proposition \ref{prop:HiBomegamodule} the action of $B\otimes\F$ on $H_i(\Hur(Q);\F)$ factors through the projection of rings $B\otimes\F\twoheadrightarrow B_\omega\otimes\F$, we have that $H_i(\Hur(Q);\F)$ is a finitely generated $B_\omega\otimes\F$-module; the statement is now a consequence of Proposition \ref{prop:Bomegagrowth} and the classical theory of the Hilbert function of a finitely generated graded
module over a graded algebra of finite type, together with the observation that $B_\omega\otimes\F$ is concentrated in weights multiple of $\ell$.
\end{proof}

