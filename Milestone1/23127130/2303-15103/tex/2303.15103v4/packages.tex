% Recommended, but optional, packages for figures and better typesetting:
\usepackage{microtype}
\usepackage{graphicx}
\usepackage{subfigure}
\usepackage{booktabs} % for professional tables
\usepackage{xcolor}

% hyperref makes hyperlinks in the resulting PDF.
% If your build breaks (sometimes temporarily if a hyperlink spans a page)
% please comment out the following usepackage line and replace
% \usepackage{hyperref}
\usepackage[pagebackref=true,breaklinks=true,colorlinks,bookmarks=false]{hyperref}
\definecolor{mydarkgreen}{rgb}{0,1,0.18}
\definecolor{mydarkblue}{rgb}{0,0.18,1}
 \hypersetup{
 linkcolor=red,
 filecolor=red,
 citecolor=green,      
 urlcolor=cyan,
}


% Attempt to make hyperref and algorithmic work together better:
\newcommand{\theHalgorithm}{\arabic{algorithm}}
\usepackage{algorithm}
\usepackage{algorithmic}

\usepackage{pgfplots}
\pgfplotsset{compat = newest}
\usepackage{multirow}

% For theorems and such
\usepackage{amsmath}
\usepackage{amssymb}
\usepackage{mathrsfs}
\usepackage{mathtools}
\usepackage{amsthm}

% if you use cleveref..
\usepackage[capitalize,noabbrev]{cleveref}



%%%%%%%%%%%%%%%%%%%%%%%%%%%%%%%%
% THEOREMS
%%%%%%%%%%%%%%%%%%%%%%%%%%%%%%%%
\theoremstyle{plain}
\newtheorem{theorem}{Theorem}[section]
\newtheorem{proposition}[theorem]{Proposition}
\newtheorem{lemma}[theorem]{Lemma}
\newtheorem{corollary}[theorem]{Corollary}
\theoremstyle{definition}
\newtheorem{definition}[theorem]{Definition}
\newtheorem{assumption}[theorem]{Assumption}
\newtheorem{fact}[theorem]{Fact}
\theoremstyle{remark}
\newtheorem{remark}[theorem]{Remark}

% Todonotes is useful during development; simply uncomment the next line
%    and comment out the line below the next line to turn off comments
%\usepackage[disable,textsize=tiny]{todonotes}
% \usepackage[textsize=tiny]{todonotes}

\newcommand{\alink}[1]{\href{#1}{paper-link}}
\newcommand{\tldr}[1]{\textbf{TLDR}: \textcolor{cyan}{#1}}
\newcommand{\abstr}[1]{\textbf{Abstract}: #1}
\newcommand{\summary}[1]{\textbf{Summary}: \textcolor{magenta}{#1}}
\newcommand{\paper}[2]{\paragraph{#1} \alink{#2}}


\newcommand{\bM}{\mathbb{M}}
\newcommand{\bfW}{\mathbf{W}}
\newcommand{\bfX}{\mathbf{X}}
\newcommand{\bfG}{\mathbf{G}}
\newcommand{\bfZ}{\mathbf{Z}}
\newcommand{\bfE}{\mathbf{E}}
\newcommand{\bfpi}{\boldsymbol{\pi}}
\newcommand{\bbfpi}{\boldsymbol{\bar \pi}}
\newcommand{\bA}{\mathbf{A}}
\newcommand{\ai}{\mathbf{a}_i}
\newcommand{\bai}{\mathbf{\bar a}_i}
\newcommand{\bB}{\mathbf{B}}
\newcommand{\bi}{\mathbf{b}_i}
\newcommand{\bbi}{\mathbf{\bar b}_i}
\newcommand{\bfK}{\mathbf{K}}
\newcommand{\cX}{\mathcal{X}}
\newcommand{\cH}{\mathcal{H}}
\newcommand{\cZ}{\mathcal{Z}}
\newcommand{\cI}{\mathcal{I}}
\newcommand{\tsim}{\text{sim}}
\newcommand{\kz}{{k}}

\newcommand{\pos}{\mathbb{P}_{\text{pos}}}
\usepackage{enumitem}
\usepackage{ amssymb }

\definecolor{citecolor}{HTML}{0071BC}
\definecolor{linkcolor}{HTML}{ED1C24}
\hypersetup{colorlinks=true, linkcolor=linkcolor, citecolor=citecolor,urlcolor=black}