\usepackage[utf8]{inputenc}
\usepackage[english]{babel}
\usepackage{jmlr2e}
% \usepackage[top=2cm,bottom=2cm,left=3cm,right=3cm,marginparsep=0.3cm,marginparwidth=3cm,includefoot]{geometry}
\usepackage{hyperref}
\usepackage{booktabs}
\usepackage{amssymb, amsmath}
\usepackage{enumitem}
\usepackage{ccaption}
\usepackage[normalem]{ulem}
\usepackage{cleveref}
% \usepackage{mathabx}
\usepackage{subcaption}
\usepackage{stmaryrd}
\usepackage{tikz}
\usepackage{tikz-cd} % To draw diagrams
\usetikzlibrary{arrows}
\usepackage{float}
% Equations
\usepackage{xparse} % To make intelligent commands (e.g. changing if no argument is passed)
\usepackage{pdflscape} % Landscape modes
\usepackage{afterpage} % Place table at next page it was declared
\usepackage{capt-of} % Caption 
\setcounter{tocdepth}{2}

\numberwithin{equation}{section}

% Lists
% \setlist[itemize]{label=\tiny\textbullet}
\setlist[enumerate]{label=\normalfont(\roman*)}
\setlist{noitemsep}

% Commentaires
\newcommand{\todo}[1]{\textcolor{blue}{TODO: #1}}
\newcommand{\vadim}[1]{\textcolor{purple}{VADIM: #1}}
\newcommand{\piopio}[1]{\textcolor{violet}{Olympio : #1}}
\newcommand{\warning}[1]{\textcolor{red}{WARNING: #1}}
\newcommand{\change}[1]{\textcolor{green}{CHANGE: #1}}
\newcommand{\comment}[1]{\textcolor{brown}{COMMENT: #1}}

\def\REF{\textcolor{green}{[REF]}}
\def\CREF{\textcolor{green}{CREF}}

% Configuration de cleveref
\crefname{equation}{equation}{equations}
\crefname{figure}{figure}{figures}
\crefname{lem}{lemma}{lemmas}
\crefname{ex}{example}{examples}


%%%% French %%%%
%\newcommand{\cedille}{\c c}

%%%%%%%%%%%%%%%%%%
%%%% Diagrams %%%%
%%%%%%%%%%%%%%%%%%
\tikzset{
  symbol/.style={
    draw=none,
    every to/.append style={
      edge node={node [sloped, allow upside down, auto=false]{$#1$}}}
  }
}

%%%%%%%%%%%%%%%%%%%%%%
%%%% Mathemathics %%%%
%%%%%%%%%%%%%%%%%%%%%%

% Maths environments
%\theoremstyle{plain}
%\newtheorem{thm}{Theorem}[section]
%\newtheorem{main}[thm]{Main result}
%\newtheorem{lem}[thm]{Lemma}
%\newtheorem{cor}[thm]{Corollary}
%\newtheorem{prop}[thm]{Proposition}
%\newtheorem{pred}[thm]{Predicate}
%\newtheorem{conj}[thm]{Conjecture}
%\newtheorem{axio}[thm]{Axiom}


%\theoremstyle{definition}
%\newtheorem{defi}[thm]{Definition}
%\newtheorem{hypo}[thm]{Hypothesis}
%\newtheorem{assu}[thm]{Assumption}
%\newtheorem{assus}[thm]{Assumptions}
%\newtheorem{ex}[thm]{Example}
%\newtheorem{exs}[thm]{Examples}
%\newtheorem{rk}[thm]{Remark}
%\newtheorem{rks}[thm]{Remarks}
%\newtheorem*{goal}{Goal}
%\newtheorem*{idea*}{Idea}
%\newtheorem{que}[thm]{Question}
%\newtheorem*{que*}{Question}
%\newtheorem{ques}[thm]{Questions}
%\newtheorem*{ques*}{Questions}
%\newtheorem{nota}[thm]{Notation}
%\newtheorem*{ackno}{Acknowledgements}
%\newtheorem{conv}[thm]{Convention}

%\theoremstyle{remark}
%\newtheorem{exo}[thm]{Exercise}
%\newtheorem{pb}[thm]{Problem}
%\newtheorem*{pb*}{Problem}
%\newtheorem{intui}[thm]{Intuition}
%\newtheorem*{intui*}{Intuition}
%\newtheorem{csq}[thm]{Consequence}
%\newtheorem{clm}[thm]{Claim}



% Notations et operateurs
\def\N{{\mathbb N}}    % naturels
\def\Z{{\mathbb Z}}    % entiers relatifs
\def\R{{\mathbb R}}    % réels
\def\Q{{\mathbb Q}}    % rationnels
\def\C{{\mathbb C}}    % complexes
\def\K{{\mathbb K}}    % corps quelconques
\def\E{{\mathbb E}}    % espérance
\def\P{\mathbb{P}}
\def\V{\mathbb{V}}
\def\W{\mathbb{W}}
\def\X{\mathbb{X}}
\def\kk{\R}
\def\D{\mathrm{D}}

% Categories
\def\vec{\mathsf{Vec}}

% Notations
\def\d{\,\mathrm{d}}
\def\st{:} % 'such that' in ensemblist notations {x \in X | ...}
\newcommand{\eps}{\varepsilon} % Le bon epsilon 
\def\1{\mathbf{1}}
\renewcommand{\phi}{\varphi}
\def\supp{\operatorname{supp}}
\newcommand{\card}[1]{|#1|}
\renewcommand{\emptyset}{\varnothing}
\renewcommand{\bar}[1]{\overline{#1}}
\newcommand{\longtoinf}[1][n\to\infty]{\underset{#1}{\longrightarrow}}
\newcommand{\closure}[1]{\overline{#1}}
\def\id{\operatorname{Id}}
\def\Ima{\operatorname{Im}}
\def\Ker{\operatorname{Ker}}

\newcommand{\covnbr}[1]{\mathcal{N}(#1)}

\newcommand{\Lp}[2][1]{L^{#1}(#2)}
\newcommand{\dual}[1]{{#1}^*}
\newcommand{\dualdot}[2]{#1\cdot #2}
% \newcommand{\dualdot}[2]{\langle#1;#2\rangle}

%%%%%%%%%%%%%%%%%%%%%%%
%%%%% Conventions %%%%%
%%%%%%%%%%%%%%%%%%%%%%%
% Marketing
\newcommand{\bestscore}[1]{\textcolor{black}{\textbf{#1}}}

% Noms des trucs d'info
\def\orbit{\texttt{ORBIT5K}}
\def\gudhi{\texttt{Gudhi}}
\def\mma{\texttt{MMA}}
\def\cpp{\texttt{C++}}
\def\python{\texttt{Python}}

% Setting and main definitions
\def\Cech{\v{C}ech}
\def\cplx{\mathcal{K}}
\def\filt{\mathcal{F}}
\def\birth{\operatorname{birth}}

\def\chains{\mathcal{C}}

% Persistence and Euler characteristic tools
\def\Pers{\mathrm{Pers}}
\def\barcode{\mathcal{B}}
\def\diagram{\mathcal{D}}
\def\matching{M}

\newcommand{\FP}[1][\R^m]{\mathrm{FP}(#1)}
\def\sbarcode{\bar{\mathcal{B}}}
\def\cost{\mathrm{cost}}
\NewDocumentCommand{\dist}{d[]}{
	\IfNoValueTF{#1}{\def\argument{}}{\def\argument{\!\left(#1\right)}}%
	%\mathbf{k}\argument{}%
    \widehat{d}_1\argument{}%
}
\def\CechCplx{\check{\mathcal{C}}}

% Euler calculus
\def\Hil{\mathrm{Hil}}
\newcommand{\betti}[1][\filt]{\beta_{#1}}
\newcommand{\ECP}[1][\filt]{\chi_{#1}}% \newcommand{\ECP}[1][f]{\mathrm{ECP}_{\hspace{-0.1em}#1}}
\newcommand{\ECPn}[1][\filt]{\chi_{{#1}_n}}
\newcommand{\Rdn}[1][\filt]{\mathcal{R}_{#1}}
\newcommand{\RdnL}[1][\filt]{\mathcal{R}_{{#1}_L}}

\def\kernel{\kappa}				% Kernel of the hybrid transform (WARNING, not the one in the algorithm, see below)			
\NewDocumentCommand{\Kernel}{d[]}{% Primitive of the kernel (this is the one chosen in the algorithm, e.g. Kernel(t) = wavelet_4(t)
	\IfNoValueTF{#1}{\def\argument{}}{\def\argument{\left(#1\right)}}%
	%\mathbf{k}\argument{}%
    \bar{\kappa}\argument{}%
}
\NewDocumentCommand{\HT}{O{\filt} O{\kernel}}{\psi_{#1}^{#2}}
\NewDocumentCommand{\HTn}{O{\filt} O{\kernel}}{\psi_{{#1}_n}^{#2}}
\NewDocumentCommand{\HTL}{O{\filt} O{\kernel}}{\psi_{{#1}_L}^{#2}}
\NewDocumentCommand{\HTtrunc}{O{\filt} O{\kernel}}{\psi_{#1}^{{#2}, T}}
\NewDocumentCommand{\HTtruncn}{O{\filt} O{\kernel}}{\psi_{{#1}_n}^{{#2}, T}}
%%% Analysis
\def\dt{\d t}
\def\ds{\d s}

\def\sinc{\operatorname{sinc}}

%%% Constructible functions
\def\omin{\mathcal{O}}
\def\Euler{\chi}
\def\dEuler{\d\chi}
\NewDocumentCommand{\CF}{O{\R^n} O{}}{%
\operatorname{CF}_{#2}(#1)%
}

%%% Probabilities
\def\cov{\operatorname{cov}}

% \NewDocumentCommand{\HT}{O{\filt} O{\kernel}}{%
% 	\IfNoValueTF{#3}{\def\argument{}}{\def\argument{\left(#3\right)}}%
% 	\psi_{#1}\left[#2\right]
% %
% }