\section{Motivation}
\label{sec:preliminaries} 

In executing DNNs on hardware platforms, 
%such as Google TPU \cite{TPU2017,googleedgetpu}, 
the huge number of MAC operations may consume much power.
%can cause a
%\textcolor{red}{
%large power consumption \cite{please add this reference???}}. 
Existing methods often introduce hardware
modifications, which may incur extra hardware cost or make the design specific
for individual neural networks. On the contrary, we address this power
consumption issue by examining the power and timing properties of the weight
values and activations. 

%Different from previous work, we find weights in a MAC unit exhibit different
%power characteristics, which can be used to to reduce power consumptions of
%DNNs without incurring hardware cost.  

%For example, weights with only small power values are selected for DNN
%training, so that the overall power consumption of DNNs accelerator can be
%reduced.  In addition, the timing characteristics of weights and activations
%in a MAC units also vary.  Weights and activations that lead to small delays
%can be selected, so that sensitized path delays of the circuits implementing
%MAC operations are reduced, which thus allows a scaling of supply voltage to
%reduce power consumption further.

A MAC unit calculates the multiplication of a weight and an activation and adds
the result to a partial sum, as illustrated in \figname~\ref{fig:mac}.  Assume
the weight of a neural network is quantized to $n$ bits.  Correspondingly,
there are $2^n$ possible weight values.  These weight values are one of the inputs to the
digital logic implementing the MAC operations.  Since different weight values
cause different signal switching activities inside the MAC units, they also exhibit
different average power consumption with respect to the activation transitions and partial sum transitions. For
example, the weight values $2^n, n=0,1,\dots,n-2$, lead to less power
consumption, because the multiplication with these weight values are actually
shift operations and can thus activate fewer signal propagations in the
circuit.

To demonstrate the different power consumption of weight values, we evaluated
the average power consumption of different weight values in a MAC unit of a
$64\times64$ systolic array. We simulated the execution of LeNet-5 processing
100 pictures randomly selected from the CIFAR-10 dataset.
During simulation, we collected statistics of the switching activities of various signals inside the systolic array.
Based on this data we estimated the average power consumption of each weight value using Power Compiler from Synopsys.

%The fixed weight values and the corresponding transitions of the activations and partial sum of
%the MAC units are fed into Design Compiler to obtain the corresponding power
%consumption.  Afterwards, the average power consumption corresponding to the
%weight values were calculated.

%The weights are quantized to eight bits.  In the evaluation, we first fixed one
%of the inputs of the multiplier of a MAC unit to a given value following the
%weight stationary design.  Thereafter, we evaluated the transitions of the
%activations and partial sum of the MAC unit, as shown Figure~\ref{}, because
%power consumption is the result of signal switching activities triggered by the
%changes of activations and partial sums, instead of their static values.  We
%simulated the execution of LeNet-5 processing 100 pictures randomly selected
%from the CIFAR-10 dataset. The hardware structure was a systolic array of size
%$64\times64$ with MAC units synthesized with 15nm technology \cite{}.  We
%randomly selected 10,000 combinations of transitions and partial sum
%combinations from the simulation result and fed them into Design Compiler to
%obtain the corresponding power consumption.  Afterwards, the average power
%consumption corresponding to the weight value was calculated.

Figure~\ref{fig:powerResult} illustrates the average power consumption of the
weight values obtained by the simulation described above.  According to this
figure, different weight values can lead to substantially different
average power consumption.  For example, the quantized weight value -105 has a large
average power consumption 1,066\,$\mu$W, while the quantized weight value -2 has only
596\,$\mu$W.  \textit{According to this observation, by restricting neural
networks to prefer the weight values with small average power consumption, the
overall power consumption of executing neural networks can be lowered.} 

%Besides the fixed weight, the input sequences of the activation 
%and partial sum also greatly influence the power consumption.
%Therefore, acquiring representative power analysis results requires 
%knowledge about the input sequences which the MAC unit receives during operation.
%This necessitates a method which allows us to capture these input 
%transitions during typical operation of the accelerator, so that it can be later applied to power analysis. 
%we simulate a systolic array of size $64 \times 64$ features 4096 MAC units 
%an example to extract the input transitions and partial sums transitions 

Besides different power characteristics, different weights also exhibit
different timing profiles in a MAC unit.  Inside a MAC unit shown in
\figname~\ref{fig:mac}, there are many combinational paths, which have different
delays and are triggered by specific input data, i.e., weight, activation, and
partial sum.
%leading from the inputs to the output.  Each of these paths is associated with
%a certain delay and only activated for specific input sequences which
%sensitize that path.
If the weight is fixed to a given value, some combinational paths in the MAC
unit cannot be sensitized.  Accordingly, the delay of the MAC unit may differ
with respect to different weight values.  To demonstrate this difference, we
conducted timing analysis of the MAC unit with fixed weight values and all
activation transitions using Modelsim.

Figure~\ref{fig:delayResult} illustrates the delay profiles of two quantized weight
values -105 and 64, where the x-axis shows the delay and the y-axis shows the
frequency of this delay appearing with respect to all possible activation transitions.
Figure~\ref{fig:delayResult} confirms that different weight
values lead to different delays. In addition, it shows that the
delays can be reduced further if some activations can be
pruned from the neural network, e.g., the activation transitions triggering
delays on the far right end of the x-axis.  \textit{Since the clock period of
a circuit is determined by the maximum delay of all the combinational paths,
the clock frequency of the MAC unit and thus the computational performance can
be increased by
%  According to the analysis above, the maximum delay of the MAC unit can
pruning weights and activations according to their timing profiles.  Alternatively, the supply voltage can be lowered
to reduce power consumption further, while maintaining the original clock
frequency.}

%this pruning can lead to an increase of 

%different combinations of weights and activations can lead to vastly different
%delays.  By excluding certain weight and activation values, we can prevent
%large delays from being triggered, thus reducing the effective maximum delay of
%the MAC unit.  This timing budget can then be used to reduce the supply voltage
%and thus power consumption. 
%To reduce power consumptions of DNNs on digital accelerators, we exploit the 

%The computational load of DNNs is dominated by MAC operations.  Consequently,
%most accelerators employ large numbers of MAC units inside their processing
%engine.  For example, a systolic array of size $64 \times 64$ features 4096
%MAC units.  Optimizing power consumption of the computational logic is
%therefore tied to optimizing MAC operations.  Since the aim of this work is to
%achieve power savings without modifications, we rely on detailed analyses of
%the MAC units to find optimization opportunities.  Furthermore, as the focus
%is on power-efficient edge devices, we will consider integer arithmetic for
%the MAC units and 8-bit fixed-point quantization for the weights and
%activations.  To reduce analysis complexity, this paper is limited to
%weight-stationary designs, e.g., the systolic array inside the EdgeTPU.

%Generally, a MAC unit has three inputs and one output, where the output is
%calculated by multiplying two of the inputs and adding the third input.  In
%the context of DNN accelerators, the MAC unit multiplies a weight and an
%activation and adds the multiplication result to a partial sum.  Inside the
%MAC unit are many paths leading from the inputs to the output.  Each of these
%paths is associated with a certain delay and only activated for specific input
%sequences which sensitize that path.  This is illustrated in
%Figure~\ref{fig:delayResult} which shows the delay profiles of a MAC unit for
%two different fixed weights.  To generate these delay profiles, the delays of
%a MAC unit for all possible combinations of the fixed weight and activation
%transitions were measured.  As this figure indicates, different combinations
%of weights and activations can lead to vastly different delays.  By excluding
%certain weight and activation values, we can prevent large delays from being
%triggered, thus reducing the effective maximum delay of the MAC unit.  This
%timing budget can then be used to reduce the supply voltage and thus power
%consumption.

\begin{figure}[t]
  \begin{minipage}[c]{0.45\linewidth}
    %\vskip -10pt
    \centering
    %\vskip 7pt
    \includegraphics[width=0.9\linewidth]{Fig/MAC.pdf}\hspace{10pt}
    %\vskip 7pt
    \caption{Power and delay characterization of MAC unit.}
    \label{fig:mac}
  \end{minipage}\hfil
  \begin{minipage}[c]{0.45\linewidth}
    \centering
    %\includegraphics[scale=0.3]{Fig/power_4.pdf}
    \includegraphics[width=1\linewidth]{Fig/power_analysis_result_revised.pdf}
    %\caption{Average power consumption of each weight value with an exemplary threshold at 900\,mW.}
    \caption{Average power consumption of quantized weight values.}
    \label{fig:powerResult}
  \end{minipage}
\end{figure}

\begin{figure}[t]
    \centering
    \includegraphics[width=0.95\linewidth]{Fig/delay_profile.pdf} 
    \caption{Delay profiles of a MAC unit for two quantized weight values.
    The arrows point to the maximum delay of a given weight value with respect
    to all the activation transitions.}%and partial sum transitions.}
    \label{fig:delayResult}
\end{figure}

%To take advantage of the power and timing properties of the weight values
%describe above, we need to deal with the huge number of transitions of activations and partial sum.  For a
%systolic array of size $64 \times 64$ using 8-bit weights and activations, at
%least \textcolor{red}{22} bits are required for the partial sum to prevent
%overflow.  Accordingly, there are $2^{(8+22)\times2}$ possible input
%transitions of activations and partial sum for a given weight value. It is
%therefore not feasible to simulate all these combinations to identify the power
%consumption and timing profile of each weight value.

%Second, after weight and activation selection, training DNNs with
%only a small number of weight values and transition values also pose a
%challenge. 

%On the other hand, static timing analysis, while fast, would not generate the
%desired results as it does not consider path sensitization.  A timing analysis
%method is required which gives accurate enough results while not consuming too
%much time.

%The power consumption of a digital circuit is highly dependent on its switching activity, which is in turn controlled by the applied inputs to the circuit.
%Figure~\ref{fig:powerResult} illustrates this dependency, where the average power consumption of the MAC unit for each fixed weight value has been measured.
%This power measurement shows that fixing just one input to the MAC unit already leads to substantial variation of the average power consumption.
%If we were to restrict the neural network to only such weights which are below a certain power threshold, the average power consumption of the accelerator could be lowered as well.
%Besides the fixed weight, the input sequences of the activation and partial sum also greatly influence the power consumption.
%Therefore, acquiring representative power analysis results requires knowledge about the input sequences which the MAC unit receives during operation.
%This necessitates a method which allows us to capture these input transitions during typical operation of the accelerator, so that it can be later applied to power analysis.

