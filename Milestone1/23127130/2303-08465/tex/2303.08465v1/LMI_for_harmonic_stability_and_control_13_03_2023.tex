%%%%%%%%%%%%%%%%%%%%%%%%%%%%%%%%%%%%%%%%%%%%%%%%%%%%%%%%%%%%%%%%%%%%%%%%%%%%%%%%
%2345678901234567890123456789012345678901234567890123456789012345678901234567890
% 1  2  3  4  5  6  7  8

\documentclass[letterpaper, 10 pt, conference]{ieeeconf} % Comment this line out
        % if you need a4paper
%\documentclass[a4paper, 10pt, conference]{ieeeconf} % Use this line for a4
        % paper

\IEEEoverridecommandlockouts    % This command is only
        % needed if you want to
        % use the \thanks command
\overrideIEEEmargins
% See the \addtolength command later in the file to balance the column lengths
% on the last page of the document



% The following packages can be found on http:\\www.ctan.org
%\usepackage{graphics} % for pdf, bitmapped graphics files
%\usepackage{epsfig} % for postscript graphics files
%\usepackage{mathptmx} % assumes new font selection scheme installed
%\usepackage{times} % assumes new font selection scheme installed
%\usepackage{amsmath} % assumes amsmath package installed
%\usepackage{amssymb} % assumes amsmath package installed
\usepackage[dvipsnames]{xcolor}
%\usepackage{generic}
\usepackage{cite}
\usepackage{amsmath,amssymb,amsfonts}
\usepackage{algorithmic}
%\usepackage{unicode-math}
%\setmathfont{XITS Math}
%\setmathfont[version=setB,StylisticSet=1]{XITS Math}
%\usepackage[cal=boondox]{mathalfa}
\usepackage{dutchcal}
\usepackage{graphicx}
\usepackage[OT2,T1]{fontenc}
\usepackage{MnSymbol}
\usepackage{lipsum}
\usepackage{bbm}
\usepackage{color}
\newtheorem{theorem}{Theorem}
\newtheorem{assumption}{Assumption}
\newtheorem{definition}{Definition}
\newtheorem{property}{Property}
\newtheorem{proposition}{Proposition}
\newtheorem{problem}{Problem}
\newtheorem{lemma}{Lemma}
\newtheorem{corollary}[lemma]{Corollary}
\newtheorem{remark}{Remark}
\DeclareSymbolFont{cyrletters}{OT2}{wncyr}{m}{n}
\DeclareMathSymbol{\Sha}{\mathalpha}{cyrletters}{"58}
\DeclareMathAlphabet{\mathscr}{OT1}{pzc}{m}{it}
\usepackage{textcomp}

\title{\LARGE \bf
On solving infinite-dimensional Toeplitz Block LMIs}

%\author{ \parbox{3 in}{\centering Huibert Kwakernaak*
%  \thanks{*Use the $\backslash$thanks command to put information here}\\
%  Faculty of Electrical Engineering, Mathematics and Computer Science\\
%  University of Twente\\
%  7500 AE Enschede, The Netherlands\\
%  {\tt\small h.kwakernaak@autsubmit.com}}
%  \hspace*{ 0.5 in}
%  \parbox{3 in}{ \centering Pradeep Misra**
%  \thanks{**The footnote marks may be inserted manually}\\
% Department of Electrical Engineering \\
%  Wright State University\\
%  Dayton, OH 45435, USA\\
%  {\tt\small pmisra@cs.wright.edu}}
%}

\author{Flora Vernerey, Pierre Riedinger and Jamal Daafouz% <-this % stops a space
\thanks{This work is supported by HANDY project ANR-18-CE40-0010-02}% <-this % stops a space
\thanks{The authors are with Universit\'e de Lorraine, CNRS, CRAN, F-54000 Nancy, France.}% {\tt\small $\{$Flora.Vernerey, Pierre.Riedinger, Jamal.Daafouz$\}$@univ-lorraine.fr }}%
}

\begin{document}



\maketitle
\thispagestyle{empty}
\pagestyle{empty}


%%%%%%%%%%%%%%%%%%%%%%%%%%%%%%%%%%%%%%%%%%%%%%%%%%%%%%%%%%%%%%%%%%%%%%%%%%%%%%%%
\begin{abstract}
This paper focuses on the resolution of infinite-dimensional Toeplitz Block LMIs, which are frequently encountered in the context of stability analysis and control design problems formulated in the harmonic framework. We propose a well-defined problem truncation method and demonstrate that a solution to the truncated problem can always be found at any order, provided that the original infinite-dimensional Toeplitz Block LMI problem is feasible. Using this approach, we illustrate how the infinite dimensional solution of a Toeplitz Block LMI based convex optimization problem can be recovered up to a very small error, by solving a finite dimensional truncated problem. The obtained results are applied to stability analysis and harmonic LQR for linear time periodic systems. 
\end{abstract}


%%%%%%%%%%%%%%%%%%%%%%%%%%%%%%%%%%%%%%%%%%%%%%%%%%%%%%%%%%%%%%%%%%%%%%%%%%%%%%%%
\section{Introduction}
LMIs are a powerful and versatile tool that can be used to solve a broad range of problems in science and engineering, including control theory, optimization, signal processing, and robotics. One specific type of LMIs is the Infinite-dimensional Toeplitz Block LMIs (TBLMI), which involve matrices of infinite dimension with a Toeplitz block structure. TBLMIs are encountered in the context of harmonic analysis and control, a topic of great theoretical and practical interest in numerous application domains, including energy management and embedded systems to mention few \cite{Farkas,Bolzern,Sanders,Wereley_1990,Zhou2008,Almer2}. 

Solving TBLMIs poses a significant challenge due to the infinite dimensionality. The issue we tackle in this paper is different from the problem previously examined in \cite{Ikeda01}, which aimed to reduce an infinite number of LMIs to a finite number of LMIs. In our case, the number of inequalties is finite but the entries and the unknowns are infinite-dimensional. To illustrate the challenges involved, recall the following fact (see \cite{Pierre2022} for more detail): "a truncated matrix of a Hurwitz infinite-dimensional harmonic matrix may not be Hurwitz at any truncation order". As a result, it is possible that solving the truncated version of an infinite-dimensional harmonic Lyapunov equation may not yield a positive definite solution.

In \cite{Pierre2022}, efficient algorithms and methods that leverage the Toeplitz structure have been proposed to determine the infinite-dimensional solution of harmonic Lyapunov or Riccati equations with arbitrarily small error. In this paper, we aim to expand upon this approach and extend it to the TBLMI framework. Our objective is to define a truncated version of the original problem which enables the recovery of the infinite-dimensional solution with arbitrary accuracy.

% LMIs provide a powerful and versatile tool for solving various problems in science and engineering including control theory, optimization, signal processing, and robotics. Infinite-dimensional Toeplitz Block LMIs (TBLMIs) are LMIs where the entries and unknowns matrices are of infinte dimension and have a block Toeplitz strcture. These type of LMIs are encoutered in the context of harmonic control. Harmonic modelling and control is a topic of theoretical and practical interest in many application domains such as energy management or embedded systems to mention few \cite{Farkas,Bolzern,Sanders,Montagnier,Wereley_1990,Demiray,Almer,Zhou2011,Zhou2008,Mattavelli,Chavez,Almer2}. In this context, solving stability analysis and control design problems using LMIs is an open problem. 
%% The obtained LMIs are infinte dimensional and have a block Toeplitz strcutre. 
%% Harmonic modeling and control is a topic of theoretical and practical interest in many application domains such as energy management or embedded systems to mention few \cite{Farkas,Bolzern,Sanders,Montagnier,Wereley_1990,Demiray,Almer,Zhou2011,Zhou2008,Mattavelli,Chavez,Almer2}. 
%%In a recent paper \cite{Blin}, a unified and coherent mathematical framework for harmonic modelling and control has been proposed. Basically, the harmonic modeling of a periodic system leads to an equivalent time invariant model of infinite dimension whose states (also called phasors) are the coefficients obtained by applying a sliding Fourier decomposition. One of the main results of \cite{Blin} established a strict equivalence between these two models. In this framework, the analysis and design are considerably simplified since all the methods established for time-invariant systems can be a priori applied. \\
%%
%%Recently, results related to spectral properties of the harmonic state operator along with an explicit Floquet factorization and practical solutions of harmonic Lyapunov and Riccati equations have been established in \cite{Pierre2022}. In addition, these results have been exploited in \cite{Ried2022} to address the problem of designing harmonic pole placement based control laws for Linear Time Periodic (LTP) systems. Here, we focus on stabilitity analysis and control synthesis using convex optimisation based tools. In particular, we are interested in stability analysis and harmonic control design using LMI based conditions. This interest is motivated by the fact that LMI based methods can be applied to solve robust control problems effectively and handle multi-objective control problems. 
%The main difficulty in using LMIs in the harmonic modeling framework is the fact that the obtained inequalities are infinite dimensional. The problem we formulate in this paper is different from the problem that consists in reducing infinite-dimensional inequalities to a finite number of LMIs \cite{Ikeda01}. To illustrate and explain the encountered difficulties, recall the following fact (see \cite{Pierre2022} for more detail): "the truncated matrix of a Hurwitz infinite dimensional harmonic matrix may not be Hurwitz at any truncation order". As a consequence, if one attempts to solve the truncated version of the infinite dimensional harmonic Lyapunov equation 
%there may be no chance to obtain a positive definite solution.
%In \cite{Pierre2022}, by exploiting the Toeplitz structure of these equations, efficient methods and algorithms have been proposed to determine up to an arbitrarily small error the infinite dimensional solution of harmonic Lyapunov or Riccati equations. Here, our objective is to extend this philosophy to the TBLMI framework and show how to define a truncated version problem to recover the infinite-dimensional solution up to an arbitrary small error. 

%We propose a method that consists in well defining a truncated TBLMI problem that %allows to recover any original infinite-dimensional solution up to an arbitrarily small error. More precisely, we first define a truncated version of the original infinite-dimensional LMI that 
%allows to preserve the negativity or positivity of the obtained truncated solution and then show that the truncated version of the original solution solves this truncated LMI problem. We expoit this study to show how the infinite dimensional solution of a TBLMI based convex optimisation problem can be determined up to an arbitrarily small error by solving a truncated version.
The paper is organized as follows. The next section is dedicated to mathematical prelimaries. In Section III, we define what we call a Toeplitz Block LMI and we give the problem formulation. The main results are established in Section IV where we investigate the truncation of infinite-dimensional TBLMIs that preserves solution positiveness. We show how to recover the infinite-dimensional solution of a TBLMI-based convex optimization problem, to an arbitrarily small error, by solving a finite dimensional truncated problem. We illustrate the results of this paper in section V and apply the proposed procedure to design a harmonic LQR for linear time periodic systems.

{\bf Notations: } The transpose of a matrix $A$ is denoted $A'$ and $A^*$ denotes the complex conjugate transpose $A^*=\bar A'$. The $n$-dimensional identity matrix is denoted $Id_n$. The infinite identity matrix is denoted $\mathcal{I}$. For $m\in\mathbb{Z}^+\cup \{\infty\}$, the flip matrix $J_m$ is the $(2m+1) \times (2m+1)$ matrix having 1 on the anti-diagonal and zeros elsewhere. 
$C^a$ denotes the space of absolutely continuous function,
$L^{p}([a\ b],\mathbb{C}^n)$ (resp. $\ell^p(\mathbb{C}^n)$) denotes the Lebesgues spaces of $p-$integrable functions on $[a, b]$ with values in $\mathbb{C}^n$ (resp. $p-$summable sequences of $\mathbb{C}^n$) for $1\leq p\leq\infty$. $L_{loc}^{p}$ is the set of locally $p-$integrable functions. The notation $f(t)=g(t)\ a.e.$ means almost everywhere in $t$ or for almost every $t$. 
To simplify the notations, $L^p([a,b])$ or $L^p$ will be often used instead of $L^p([a,b],\mathbb{C}^n)$. 
%For example, $x\in L^2([a,b])$ means $x \in L^2([a,b],\mathbb{C}^n)$. %We denote by $col(X)$ the vectorization of a matrix $X$, formed by stacking the columns of $X$ into a single column vector. Finally, $<\cdot,\cdot>$ refers to the scalar product in $\ell^2$.
%%%%%%%%%%%%%%%%%%%%%%%%%%%%%%%%%%%%%%%%%%%%%%%%%%%%%%%%%%%%%%%%%%%%%%%%%%%%%%%%
\vspace{-.3cm}
\section{Preliminaries}
%Metric units are preferred for use in IEEE publications in light of their
%international readership and the inherent convenience of these units in many fields.
%In particular, the use of the International System of Units (SI Units) is advocated.
% This system includes a subsystem the MKSA units, which are based on the
% meter, kilogram, second, and ampere. British units may be used as secondary units
% (in parenthesis). An exception is when British units are used as identifiers in trade,
% such as, 3.5 inch disk drive.
%We start by recalling some preliminaries related to Toeplitz block matrices, sliding Fourier decomposition in the context of harmonic modeling and the trace operator.
\subsection{Infinite dimensional Toeplitz block matrices}
%\begin{definition}\label{bt}
The Toeplitz block transformation of a $T-periodic$ $n\times n$ matrix function $A\in L^{2}([0 \ T], \mathbb{R}^{n\times n})$, denoted $\mathcal{A}=\mathcal{T}(A)$, defines a constant $n\times n$ Toeplitz block and infinite dimensional matrix as follows: 
\begin{equation}
	\mathcal{A}=\left(\begin{array}{ccc}
		\mathcal{A}_{11} & \cdots & \mathcal{A}_{1n} \\
		\vdots & \ddots& \vdots \\
		\mathcal{A}_{n1} & \cdots & \mathcal{A}_{nn}\end{array}\right)\label{btop}\end{equation} where the infinite dimensional matrices $\mathcal{A}_{ij}=\mathcal{T}(a_{ij})$, $i,j=1,\cdots,n$, are the Toeplitz transformation of the $(i,j)$ entry $a_{ij}(t)$ of the matrix $A(t)$:
\begin{align*}
	\mathcal{T}(a_{ij})=
	\left[
	\begin{array}{ccccc}
		\ddots & & \vdots & &\udots \\ & a_{ij,0} & a_{ij,-1} & a_{ij,-2} & \\
		\cdots & a_{ij,1} & a_{ij,0} & a_{ij,-1} & \cdots \\
		& a_{ij,2} & a_{ij,1} & a_{ij,0} & \\
		\udots & & \vdots & & \ddots\end{array}\right],\end{align*}
where $a_{ij,k} =\frac{1}{T}\int_{t-T}^t a_{ij}(\tau)e^{- \textsf{j}\omega k \tau}d\tau$ is the $k-$th Fourier coefficient of $a_{ij}$.
%\end{definition}
A convenient way to represent infinite dimensional Toeplitz and Toeplitz block matrices is to use their symbol representation. 
More precisely, we define the symbol matrix $A(z)$ associated to \eqref{btop} by:
\begin{equation*}A(z)=(a_{ij}(z))_{i,j=1,\cdots,n}
\end{equation*}
where $a_{ij}(z)$ refers to the Laurent series $a_{ij}(z)=\sum_{k=-\infty}^{+\infty}a_{ij,k}z^k$.
With this representation, the symbol associated to the product $\mathcal{AB}$ of two Toeplitz block matrices $\mathcal{A}$ and $\mathcal{B}$ is just $C(z)=A(z)B(z)$.
%More precisely, consider a $T-periodic$ $L^2([0 \ T],\mathbb{C})$ signal $a$, its associated Toeplitz matrix $\mathcal{T}(a)$ 
%$$\mathcal{T}(a)= (t_{ij}), {i,j}\in \mathbb{Z} \text{ such that }t_{ij} = a_{i-j}$$
For a given symbol $a(z)=\sum_{k=-\infty}^{+\infty}a_kz^k$, we also define from $a^+(z)=\sum_{k>0}a_kz^k$ and $a^-(z)=\sum_{k>0}a_{-k}z^{-k}$, the semi-infinite Hankel matrices
\begin{align*}
	\mathcal{H}(a^+) &= (h^+_{ij}), {i,j}\in \mathbb{Z}^{+*},\ h^+_{ij} = a_{i+j-1},\\
	\mathcal{H}(a^-) &= (h^-_{ij}), {i,j}\in \mathbb{Z}^{+*},\ h^-_{ij} = a_{-i-j+1}\end{align*}

Given a symbol $a(z)$ and $m \in \mathbb{Z}^+$, we denote by $\mathcal{T}_m(a)$, the $(2m+1) \times (2m+1)$ leading principal submatrices of $\mathcal{T}(a)$. 
We denote also by $\mathcal{H}_{(p,q)}(a)$, for $p,q>0$, the $(2p+1)\times(2q+1)$ Hankel matrix obtained by selecting the first $(2p+1)$ rows and $(2q+1)$ columns of $\mathcal{H}(a)$. For clarity purpose, we provide in Fig.~\ref{fig20} a block decomposition of an infinite Toeplitz matrix $\mathcal{T}(a)$ to illustrate how the matrices defined above appear. 
\begin{figure}[h]\begin{center}
		\includegraphics[width=\linewidth,height=4.5cm]{Mat_Top}
		\caption{Block decomposition of an infinite Toeplitz matrix $\mathcal{T}(a)$} \label{fig20}
	\end{center}
\end{figure}
Similarly to the definition of \eqref{btop}, the $n\times n$ Hankel block matrices $\mathcal{H}(A^+)$, $\mathcal{H}(A^-)$ are also defined respectively by $\mathcal{H}(A^+)_{ij}=\mathcal{H}(a^+_{ij})$ and $\mathcal{H}(A^-)_{ij}=\mathcal{H}(a^-_{ij})$ for $i,j=1,\cdots,n$. {Their principal submatrices $\mathcal{H}(A^+)_{(p,q)}$, $\mathcal{H}(A^-)_{(p,q)}$ for $p,q >0$ are obtained by considering the principal submatrices of the entries $\mathcal{H}(a^+_{ij})_{(p,q)}$ and $\mathcal{H}(a^-_{ij})_{(p,q)}$ for $i,j=1,\cdots,n$.}
\begin{definition}\label{deftrunc}The $m-$truncation of the $n \times n$ Toeplitz block matrix $\mathcal{A}$ denoted by $\mathcal{T}_m(A)$ or equivalently by $\mathcal{A}_m$, is defined by the $m-$truncation $\mathcal{T}_m(a_{ij})$ for all its entries $(i,j)$.
\end{definition}

\begin{theorem}[see \cite{Pierre2022}] \label{product}Let $A(z)$, $B(z)$ be two $n \times n$ symbol matrices and $C(z) =A(z)B(z)$. Then,
	\begin{align}
		\mathcal{T}_m(A)\mathcal{T}_m(B) &= \mathcal{T}_m(C)- \mathcal{H}_{(m,\eta)}(A^+)\mathcal{H}_{(\eta,m)}(B^-)\nonumber \\&- \mathcal{J}_{n,m}\mathcal{H}_{(m,\eta)}(A^-)\mathcal{H}_{(\eta,m)}(B^+) \mathcal{J}_{n,m}, \label{ee2}\end{align}
	where $\mathcal{J}_{n,m}=Id_n\otimes J_m$ and $\eta\in \mathbb{Z}^+\cup \{+\infty\}$ is such that $2\eta\geq \min d^o (A(z),B(z))$.
\end{theorem}
%\begin{proof} \end{proof}
\begin{figure}\begin{center}
		\includegraphics[scale=0.2]{Toeplitz_product}
		\caption{Multiplication of two finite dimensional banded Toeplitz matrices}\label{fig1}
	\end{center}
\end{figure}
An illustration of the above theorem is given in Fig.~\ref{fig1} for $n=1$ with $a(z)$ and
$b(z)$ Laurent polynomials of degree much less than $m$ so that $\mathcal{T}_m(a)$ and $\mathcal{T}_m(b)$
are banded.
If $a(z) =\sum^k_{i=-k} a_iz^i$ and $b(z) =\sum^k_{i=-k} b_iz^i$ with $k$ much smaller than $m$, then
the matrices $E^+=\mathcal{H}_m(a^+)\mathcal{H}_m(b^-)$ and $E^-= J_m\mathcal{H}_m(a^-)\mathcal{H}_m(b^+) J_m$ have disjoint supports located in the upper leftmost corner and in the lower rightmost corner, respectively. As a consequence,
$\mathcal{T}_m(a)\mathcal{T}_m(b)$ can be represented as the sum of the Toeplitz matrix associated with $c(z)$ and two correcting terms $E^+$ and $E^-$.

%We end these preliminaries on Toeplitz block matrices by defining what we call letf and right truncations and two results given without proofs as they follow from the block decomposition of Fig.~\ref{fig20}.
%\begin{definition}\label{truncdef} The left $m-$truncation (resp. right $m-$truncation) of a $n\times n$ Toeplitz block infinite matrix $\mathcal{A}$ is given by:
%	{\small$$\mathcal{A}_{m^+}=\left(\begin{array}{cccc}
		%			\mathcal{A}_{{11}_{m^+}} & \mathcal{A}_{{12}_{m^+}} & \cdots & \mathcal{A}_{{1n}_{m^+}} \\
		%			\mathcal{A}_{{21}_{m^+}} & \mathcal{A}_{{22}_{m^+}} & & \vdots \\
		%			\vdots & & \ddots& \vdots \\
		%			\mathcal{A}_{{n1}_{m^+}} & \cdots & \mathcal{A}_{{n(n-1)}_{m^+}} & \mathcal{A}_{{nn}_{m^+}}\end{array}\right)$$} 
%	(resp. $\mathcal{A}_{m^-}$) where $\mathcal{A}_{{ij}_{m^+}}$, $i,j=1,\cdots, n$
%	are obtained by suppressing in the infinite matrices $\mathcal{A}_{ij}$ all the columns and lines having an index strictly smaller than $-m$ (respectively strictly greater than $m$). {Finally, the $m-$truncation is obtained by applying successively a left and a right $m-$truncations.}
%\end{definition}
%\begin{proposition}\label{product3}Let $a(z)$ be a symbol and $x=(x_k)_{k\in \mathbb{Z}}$ an infinite vector of complex numbers.
%Define the $m-$truncation of $x$ by $x|_m=(x_{-m},\cdots, x_{m})$ and consider the semi-infinite vectors $x|_m^+=(x_{m+1},x_{m+2}, \cdots)$ and $x|_m^-=(\cdots, x_{-m-2},x_{-m-1})$. Let $\breve x$ be the infinite vector given by $\breve x=(\cdots, 0,x|_m,0,\cdots)$. Then, the following relations hold true:
%\begin{align}
%\mathcal{T}(a)\breve x=\left[\begin{array}{c}
	%J_\infty\mathcal{H}_{(\infty,m)}(a^-) \\
	%\mathcal{T}_m(a) \\
	%\mathcal{H}_{(\infty,m)}(a^+)J_m\end{array}\right]x|_m\label{z1}\end{align}
%\begin{align}
%\mathcal{T}_m(a)x|_m=(\mathcal{T}(a)x)|_m&-\mathcal{H}_{(m,\infty)}(a^+)J_\infty x|_m^-\nonumber\\&-J_m \mathcal{H}_{(m,\infty)}(a^-)x|_m^+ \label{z2}\end{align}
% \end{proposition}
%The next proposition is a generalization of Proposition~\ref{product3} to the case of $n\times n$ Toeplitz block matrices.
%\begin{proposition}\label{general}Let $A(z)$ be a $n \times n$ symbol matrix and $x=(x_1,\cdots, x_n)$ a vector whose components $x_i$ are infinite sequences $x_i=(\cdots, x_{i,-1},x_{i,0},x_{i,1},\cdots)$.
%Define $x|_m=(x_1|_m,\cdots,x_n|_m)$ the $m-$truncation of $x$ where for $i =1,\cdots ,n$, $x_i|_m=(x_{i,-m},\cdots, x_{i,m})$.
%Define also the semi infinite vectors $x_i|_m^+=(x_{i,m+1},x_{i,m+2}, \cdots)$ and $x_i|_m^-=(\cdots, x_{i,-m-2},x_{i,-m-1})$. 
%Set $\breve x=(\breve x_1,\breve x_2, \cdots,\breve x_n)$ with $\breve{x}_i=(\cdots, 0,x_i|_m,0,\cdots)$ for any $i =1,\cdots ,n$.
%Then, we have:
%$$(\mathcal{T}(A)\breve x)_i=\sum_{j=1}^n\mathcal{T}(a_{ij})\breve x_j$$
%where $\mathcal{T}(a_{ij})\breve x_j $ is given by \eqref{z1}
%and
%\begin{align*}
%\mathcal{T}_m(A)x|_m&=\sum_{j=1}^n(\mathcal{T}(a_{ij})x_j)|_m\\&-\mathcal{H}_{(m,\infty)}(a_{ij}^+)J_\infty x_j|_m^--J_m\mathcal{H}_{(m,\infty)}(a_{ij}^-)x_j|_m^+
%\end{align*}
%\end{proposition}
%\subsection{ Operator norms}
%We provide here some results concerning operator norms to be used in the sequel. Recall that the norm of an operator $M$ from $\ell^p$ to $\ell^q$ is given by 
%$$\|M\|_{\ell^p,\ell^q}=\sup_{\|X\|_{\ell^p}=1}\| MX \|_{\ell^q}.$$
%This operator norm is sub-multiplicative i.e. if $M: \ell^p \rightarrow \ell^q$ and $N: \ell^q \rightarrow \ell^r$ then
%$\|NM\|_{\ell^p,\ell^r} \leq \|M\|_{\ell^p,\ell^q} \|N\|_{\ell^q,\ell^r}$.
%If $p=q$, we use the notation: $\|M\|_{\ell^p}=\|M\|_{\ell^p,\ell^p}$.
%\begin{definition}Consider a vector $x(t)\in L^2([0 \ T],\mathbb{C}^n)$ and define $X=\mathcal{F}(x)$ with its symbol $X(z)$.
%	The $\ell^2 -$norm of $X(z)$ is given by:
%	$$\|X(z)\|_{\ell^2}=\|X\|_{\ell^2}$$ where $\|X\|_{\ell^2}=\left(\sum_{k\in\mathbb{Z}}|X_k|^2\right)^{\frac{1}{2}}$.
%\end{definition}
%
%\begin{theorem}\label{borne} Let $A(t)\in L^2([0 \ T],\mathbb{C}^{n\times m})$. Then, $\mathcal{A}=\mathcal{T}(A)$ is a bounded operator on $\ell^2$ if and only if $A\in L^{\infty}([0\ T],\mathbb{C}^{n\times m} )$.
%	Moreover, we have:
%	\begin{enumerate}
%		\item the operator norm induced by the $\ell^2$-norm satisfies: $$\|A(z)\|_{\ell^2}=\|\mathcal{A}\|_{\ell^2}=\|A\|_{L^{\infty}}$$
%		\item the operator norm of the semi infinite Toeplitz matrix satisfies: $\|\mathcal{T}_s(A)\|_{\ell^2}=\|\mathcal{A}\|_{\ell^2}$
%		\item the operator norm of the Hankel operators $\mathcal{H}(A^+)$, $\mathcal{H}(A^-)$ satisfies: 
%		$\|\mathcal{H}(A^-)\|_{\ell^2}\leq \|A\|_{L^{\infty}}$ and 
%		$\|\mathcal{H}(A^+)\|_{\ell^2}\leq \|A\|_{L^{\infty}}$
%		\item the operator norm related to the left and right $m-$truncations satisfies:	$\|\mathcal{A}_{m^+}\|_{\ell^2}=\|\mathcal{A}_{m^-}\|_{\ell^2}=\|\mathcal{A}\|_{\ell^2}=\|A(t)\|_{L^\infty}$
%	\end{enumerate}
%\end{theorem}
%\begin{proof}See Part V p.p. 562-574 of \cite{Gohberg}.
%\end{proof}
%\begin{proposition}\label{fro} Let $P(\cdot)$ be a matrix function in $ L^\infty([0 \ T],\mathbb{C}^{n\times n})$. Define ${\bf P}=\mathcal{F}(P)$ and $\mathcal{P}=\mathcal{T}(P)$. If 
%	$\|col({\bf P})\|_{\ell^2}\leq \epsilon$ 
%	then $\|\mathcal{P}\|_{\ell^2}\leq \epsilon$.
%\end{proposition}
%\begin{proof}
%	Using Riesz-Fisher Theorem, we have:
%	\begin{align*}
%		\|col({\bf P})\|_{\ell^2}&=\|col( P)\|_{L^2}=(\sum_{i,j=1}^n\|P_{ij}\|^2_{L^2})^{1/2}=\|P\|_F
%	\end{align*}
%	where $\|P(t)\|_F$ stands for the Frobenius norm.
%	As $P \in L^\infty([0 \ T],\mathbb{C}^{n\times n})$, H$\ddot{\mbox{o}}$lder's inequality implies $Px\in L^2([0 \ T],\mathbb{C}^{n})$ for any $x\in L^2([0 \ T],\mathbb{C}^{n})$. Thus, the result follows from the following relations between operator norms: 
%	\begin{align*}
%			\|\mathcal{P}\|_{\ell^2}&=\sup_{\|X\|_{\ell^2}=1}(<\mathcal{P}X,\mathcal{P}X>_{\ell^2})^{1/2}\\
%			&=\sup_{\|x\|_{L^2}=1}(<Px,Px)>_{L^2})^{1/2}\\
%			&\leq (trace( P^*P))^{1/2}= \|P\|_F
%	\end{align*}
%where $<\cdot,\cdot>$ stands for the scalar product. 
%\end{proof}
%
%
%
%
%
%We recall also the following results where the proofs can be found in \cite{Gohberg} (Part V p.p. 562-574).
%\begin{theorem}\label{borne} Let $A(t)\in L^2([0 \ T],\mathbb{C}^{n\times n})$. Then, $\mathcal{A}=\mathcal{T}(A)$ is a bounded operator on $\ell^2$ if and only if $A\in L^{\infty}([0\ T],\mathbb{C}^{n\times n} )$.
%	Moreover, the operator norm induced by the $\ell^2$-norm satisfies: $\|A(z)\|_{\ell^2}=\|\mathcal{A}\|_{\ell^2}=\|A\|_{L^{\infty}}$.
%\end{theorem}	
%\begin{theorem}\label{inverse}
%	Let $A(t)\in L^\infty([0 \ T],\mathbb{C}^{n\times n})$. $\mathcal{A}$ is invertible if and only if there exists $\gamma>0$ such that the set $\{t: |\det(A(t))|<\gamma\}$ has measure zero. The inverse $\mathcal{A}^{-1}$ is determined by $\mathcal{T}(A^{-1})$.
%	In addition, $\mathcal{A}$ is invertible if and only if $\mathcal{A}$ is a Fredholm operator \cite{Gohberg},
%	or equivalently in this setting if and only if
%	there exists $c > 0$ such that
%	$$\|\mathcal{A}x\|_{\ell^2} > c \|x\|_{\ell^2},\text{ for any } x \in \ell^2.$$
%\end{theorem}
%
%
% 
%
%%%%%%%%%%%%%%%%%%%%%%%%%%%%%%%%%%%%%%%%%%%%%%%%%%%%%%%%%%%%%%%%%%%%%%%%%%%%%%%%
\subsection{Sliding Fourier decomposition and harmonic modeling}
Consider $x\in L^{2}_{loc}(\mathbb{R},\mathbb{C})$ a complex valued function of time. Its sliding Fourier decomposition over a window of length $T$ is defined by the time-varying infinite sequence $X=\mathcal{F}(x)\in C^a(\mathbb{R},\ell^2(\mathbb{C}))$ whose components satisfy:
$$X_{k}(t)=\frac{1}{T}\int_{t-T}^t x(\tau)e^{-\textsf{j}\omega k \tau}d\tau$$ for $k\in \mathbb{Z}$, with $\omega=\frac{2\pi}{T}$.
If $x=(x_1,\cdots,x_n)\in L^{2}_{loc}(\mathbb{R},\mathbb{C}^n)$ is a complex valued vector function, then
$$X=\mathcal{F}(x)=(\mathcal{F}(x_1), \cdots,\mathcal{F}(x_n)).$$
The vector $X_k=(X_{1,k}, \cdots, X_{n,k})$ with $$X_{i,k}(t)=\frac{1}{T}\int_{t-T}^t x_i(\tau)e^{-\textsf{j}\omega k \tau}d\tau$$
is called the $k-$th phasor of $X$.
%	The sliding Fourier decomposition over a window of length $T$ from $ L^{2}_{loc}(\mathbb{R},\mathbb{C}^n)$ to $L^{\infty}_{loc}(\mathbb{R},\ell^2(\mathbb{C}^n))$ is defined by:
%	$$X=\mathcal{F}(x)$$
%	where the time-varying infinite sequence $X$ is defined by:
%	\begin{equation}
	%		t\mapsto X(t)=(\mathcal{F}(x_1)(t), \cdots,\mathcal{F}(x_n)(t))\end{equation}
%	and where for $i=1,\cdots,n$, the vector $\mathcal{F}(x_i)=(\cdots, X_{i,-1}, X_{i,0}, X_{i,1},\cdots)$, has infinite components $X_{i,k} $, $k\in \mathbb{Z}$ satisfying:
%	$$X_{i,k}(t)=\frac{1}{T}\int_{t-T}^t x_i(\tau)e^{- \textsf{j}\omega k \tau}d\tau$$ with $\omega=\frac{2\pi}{T}$.
%	The vector $X_k=(X_{1,k}, \cdots, X_{n,k})$ is called the $k-$th phasor of $X$.
%In the sequel, to distinguish a matrix function $P(\cdot)$ and its sliding Fourier decomposition $\mathcal{F}(P)$, we use the notation ${\bf P}=\mathcal{F}(P)$ and $P(t)$ instead of $P(\cdot)$.

\begin{definition}\label{H} We say that $X$ belongs to $H$ if $X$ is an absolutely continuous function (i.e $X\in C^a(\mathbb{R},\ell^2(\mathbb{C}^n))$ and fulfills for any $k$ the following condition: \begin{equation*}\dot X_k(t)=\dot X_0(t)e^{- \textsf{j}\omega k t} \ a.e.\end{equation*}
\end{definition}
Similarly to the Riesz-Fisher theorem which establishes a one-to-one correspondence between the spaces $L^2$ and $\ell^2$, the following "Coincidence Condition" establishes a one-to-one correspondence between the spaces $L_{loc}^2$ and $H$.
\begin{theorem}[Coincidence Condition \cite{Blin}]\label{coincidence}For a given $X\in L_{loc}^{\infty}(\mathbb{R},\ell^2(\mathbb{C}^n))$, there exists a representative $x\in L^2_{loc}(\mathbb{R},\mathbb{C}^n)$ of $X$, i.e. $X=\mathcal{F}(x)$, if and only if $X \in H$.
\end{theorem}

Under the "Coincidence Condition", it is established in \cite{Blin} that any periodic system having solutions in Carath\'eodory sense can be transformed by a sliding Fourier decomposition into a time invariant system. For instance, consider $T-periodic$ functions $A(\cdot)$ and $B(\cdot)$ respectively of class $L^2([0\ T],\mathbb{C}^{n\times n})$ and $L^{\infty}([0\ T],\mathbb{C}^{n\times m})$ and let: 
\begin{align}\dot x(t)=A(t)x(t)+B(t)u(t)\quad x(0)=x_0\label{ltp}\end{align}
If, $x$ is a solution (in Carath\'eodory sense) associated to the control $u\in L_{loc}^2(\mathbb{R},{\mathbb{C}^m)}$ of the linear time periodic (LTP) system ~(\ref{ltp}) then, $X=\mathcal{F}(x)$ is a solution associated to $U=\mathcal{F}(u)$ of the linear time invariant (LTI) system:
\begin{align}
	\dot X(t)=(\mathcal{A}-\mathcal{N})X(t)+\mathcal{B}U(t), \quad X(0)=\mathcal{F}(x)(0) \label{ltih}
\end{align}
where $\mathcal{A}=\mathcal{T}(A)$, $\mathcal{B}=\mathcal{T}(B)$ and 
\begin{equation}\mathcal{N}=Id_n\otimes diag( \textsf{j}\omega k,\ k\in \mathbb{Z})\label{N}\end{equation}
Reciprocally, if $X\in H$ is a solution of \eqref{ltih} with $U\in H$, then their representatives $x$ and $u$
(i.e. $X=\mathcal{F}(x)$ and $U=\mathcal{F}(u)$) are a solution of~\eqref{ltp}. Moreover, for any $k\in\mathbb{Z}$, the phasors $X_k \in C^1(\mathbb{R},\mathbb{C}^n)$ and $\dot X\in C^0(\mathbb{R},\ell^{\infty}(\mathbb{C}^n))$. As the solution $x$ is unique for $x_0$, $X$ is also unique for $X(0)=\mathcal{F}(x)(0)$. In addition, it is proved in \cite{Blin} that one can reconstuct time trajectories from harmonic ones, that is:
\begin{align*}\label{recos} x(t)&=\mathcal{F}^{-1}(X)(t)=\sum_{k=-\infty}^{+\infty} X_k(t)e^{ \textsf{j}\omega k t}+\frac{T}{2}\dot X_0(t)\end{align*}
where $X_{k}=(X_{1,k}, \cdots, X_{n,k})$ for any $k\in \mathbb{Z}$.

\subsection{Trace operator}
Consider the vectorial space $S^n$ of $T-periodic$, $L^\infty([0\ T])$ and symmetric $n\times n$ matrix functions.
Define the scalar product on $S^n\times S^n$ by: 
\begin{align*}
	<M,N>_{S^n}&=\frac{1}{T}\int_0^Ttr(M(\tau)N(\tau))d\tau%\\
	%&=\frac{1}{T}\int_0^T\sum_{i,j=1}^n M_{ij}(\tau)N_{ij}(\tau)d\tau
\end{align*}
%and the induced norm $<M,M>^{1/2}=\|M\|_F$ corresponds to the Frobenius norm. 
Let $S^n_+=\{M\in S^n: M\geq 0\text{ on }L^2\}$ and $S^n_{++}=\{M\in S^n: M> 0 \text{ on }L^2\}$
and recall that $M\geq 0 \text{ on }L^2([0\ T])$ means that for any $x\in L^2([0\ T])$,$$<x,Mx>_{L^2}=\frac{1}{T}\int_0^Tx'(\tau)M(\tau)x(\tau)d\tau \geq 0$$ and is equivalent to $M(t)\geq0$ a.e.
Recall that $M(\cdot)\in S^n_{++}$ if and only if $\mathcal{M}=\mathcal{T}(M)$ is a constant, hermitian, positive definite, Toeplitz block and bounded on $\ell^2$ operator. 
%\begin{proposition}
%\begin{itemize}
%\item$M>0$ a.e. if and only if for any $N\geq 0$ a.e. , $N\neq 0$ a.e., $<N,M>_{S^n}>0$.
%\item $M\geq 0$ a.e. if and only if for any $N\geq 0$ a.e., $<N,M>_{S^n} \geq0$.
%\end{itemize}
%
%\end{proposition}
If $M\in S^n_{++}$, as any component of $M$ can be rewritten as:
$$M_{ij}(t)=\sum_{k\in \mathbb{Z}} m_{ij,k} e^{ \textsf{j}\omega kt} a.e.,$$
it follows that: $<Id,M>_{S^n}=\sum_{i=1}^n m_{ii,0}>0.$
%Therefore, the trace operator can be defined as follows:
\begin{definition}The trace operator for hermitian, positive definite, Toeplitz block and bounded operators on $\ell^2$ is defined by
	\begin{equation}
		tr(\mathcal{M})=\sum_{i=1}^n m_{ii,0}\label{tr}\end{equation}
	%and we have:
	%$tr(\mathcal{M}\mathcal{N})=<M,N>_{S^n}$
	%where $\mathcal{M}=\mathcal{T}(M)$ and $\mathcal{N}=\mathcal{T}(N)$. 
\end{definition}
Note that if $M\in S^n_{+}$, $tr(\mathcal{M})=0$ implies that $\mathcal{M}=0$. Therefore, it is straightforward to check that $tr(\mathcal{M})$ defines a norm for this operator class.

\section{Problem formulation}


%%%%%%%%%%%%%%%%%%%%%%%%%%%%%%%%%%%%%%%%%%%%%%%%%%%%%
An infinite dimensional TBLMI is given by:
\begin{equation}\mathcal{M}(x)=\mathcal{M}_0 +\sum_{i=1}^{+\infty}x_i\mathcal{M}_i >0,\label{tblmi}
\end{equation}
where $x_i\in \mathbb{C}$ are the unkowns and the Hermitian matrices $\mathcal{M}_i$ are $n\times n$ infinite-dimensional Toeplitz block matrices. 
\begin{definition}A sequence $x\in \ell^2$ is a solution of the TBLMI (\ref{tblmi}) if $\mathcal{M}(x)$ is a positive definite and bounded operator on $\ell^2$ (i.e. there exists a number $\kappa>0$, $\|\mathcal{M}(x)\|_{\ell^2}=\sup_{\|y\|_{\ell^2}=1}\|\mathcal{M}(x)y\|_{\ell^2}<\kappa$).
\end{definition}
Before stating the problem we are interested in, we give examples of problems where TBLMIs may be encountered. First, consider the problem of stability analysis of the linear harmonic model (\ref{ltih}). This reduces to check the feasibility of the following harmonic Lyapunov inequality:
\begin{equation}(\mathcal{A}-\mathcal{N})^*\mathcal{P}+\mathcal{P}(\mathcal{A}-\mathcal{N})<0\label{lyap}
\end{equation}
which can be written in the form (\ref{tblmi}) by considering $\mathcal{V}_i$, $i=0,\cdots, +\infty$ defining a basis for hermitian and 
 $n\times n$ block of infinite dimensional Toeplitz matrices: 
 \begin{equation*}\mathcal{M}_i=(\mathcal{A}-\mathcal{N})^*\mathcal{V}_i+\mathcal{V}_i(\mathcal{A}-\mathcal{N}) \text{ and }\mathcal{P}=\sum_{i=0}^{+\infty}x_i\mathcal{V}_i.\label{Fi}
\end{equation*}
%%%%%%%%%%%%%%%%%%%%%%%%%%%%%%%%%%%%%%%%%
TBLMIs can also be encoutered in control design problems. Consider the harmonic state feedback design problem which consists in the determination of 
$U=-\mathcal{K}X$
that stabilizes the infinite dimensional harmonic system \eqref{ltih} and provides a representative in the time domain: $$u(t)=-K(t)x(t)=-\mathcal{F}^{-1}(\mathcal{K}X)$$ with a $T-periodic$ matrix gain $K(t)$. A stabilizing state feedback gain is given by $\mathcal{K}=\mathcal{Y}\mathcal{S}^{-1}$
where the Toeplitz block matrices $\mathcal{Y}$ and $\mathcal{S}$ are solutions of the TBLMI: 
\begin{align*}
(\mathcal{A}-\mathcal{N})\mathcal{S}+\mathcal{S}(\mathcal{A}-\mathcal{N})^*-\mathcal{B}\mathcal{Y}-\mathcal{Y}^*\mathcal{B}^*&<0 
\end{align*}
with $\mathcal{S}=\mathcal{S}^*>0$.
One may also mention the harmonic LQR problem whose solution is obtained by solving the associated 
%harmonic Riccati equation (see \cite{Pierre2022}):
%\begin{equation}(\mathcal{A}-\mathcal{N})^*\mathcal{P}+\mathcal{P}(\mathcal{A}-\mathcal{N})-\mathcal{P}\mathcal{B}\mathcal{R}^{-1}\mathcal{B}^*\mathcal{P}+\mathcal{Q}=0,\label{are}\end{equation}
%or equivalently, by solving the following 
infinite dimensional convex optimization problem (see \cite{Wil71}):
\begin{align} 
 &\max_{\scriptsize \mathcal{P}=\mathcal{P}^*>0} tr(\mathcal{P}),\ % \sum_{i=1}^n Z_{ii,0}
\label{op}\\
&\left(\begin{array}{cc}
(\mathcal{A}-\mathcal{N})^*\mathcal{P}+\mathcal{P}(\mathcal{A}-\mathcal{N})+\mathcal{Q} & \mathcal{PB} \\
\mathcal{B}^*\mathcal{P}& \mathcal{R}
\end{array}\right)\geq 0\nonumber
\end{align}
where the trace operator is defined by \eqref{tr} and $\mathcal{Q}$ and $\mathcal{R}$ are the LQR weighting matrices. The matrix gain is given by $\mathcal{K}=\mathcal{R}^{-1}\mathcal{B}^*\mathcal{P}$ where $\mathcal{P}$ is a Toeplitz block matrix of infinite dimension and a bounded operator on $\ell^2$; see \cite{Blin}.

As we are interested in problems in which the variables are
matrices, we rewrite \eqref{tblmi} as follows: 
\begin{equation}
	%\mathcal{L}(\mathcal{P},\mathcal{M}_i,\ i=1,\cdots,m)<0
	\mathcal{L}(\mathcal{P};\mathcal{A}_i,i\in S)<0
	\label{LMg}
\end{equation}
where $\mathcal{P}$ is the unknown Toeplitz block operator assumed to be a bounded operator on $\ell^2$, $\mathcal{A}_i,i\in S$ are given operators and $S$ is a finite set of subscripts. In \eqref{lyap}, we have two given operators $\mathcal{A}_1 = \mathcal{A}$ and $\mathcal{A}_2 = \mathcal{N}$. 
In the sequel, we will often simplify the notation $\mathcal{L}(\mathcal{P};\mathcal{A}_i,i\in S)$ and write $\mathcal{L}(\mathcal{P})$. 
\begin{assumption}\label{bound}
		We assume that all the entries $\mathcal{A}_i,$ $i\in S$ are bounded operators on $\ell^2$ except $\mathcal{N}$ given by (\ref{N}) which is not. We assume that $\mathcal{N}$ appears with the following Toeplitz block form: $\mathcal{N}^*\mathcal{P}+\mathcal{P}\mathcal{N}$.
\end{assumption}
The problem we tackle in this paper is to determine a solution to the TBLMI \eqref{LMg} that minimizes a linear objective function. The problem can be stated as follows:
\begin{align*}{\bf COP_{\infty,\infty}:} \min_{\mathcal{P}^*=\mathcal{P}>0} F(\mathcal{P})=tr(\mathcal{CP}) \text{ subject to:}\\
	\mathcal{L}(\mathcal{P};\mathcal{A}_{i}, i\in S)\leq 0
\end{align*}
where $\mathcal{C}$ is given and assumed hermitian and bounded on $\ell^2$. 
%
%Handling the infinite dimension nature of this convex optimisation problem is very challenging. The question we answer in this paper, is how to proceed numerically with this kind of infinite dimension problems and obtain a solution up to an arbitrarily small error. 

%When stability analysis or control design conditions are expressed in terms of harmonic Lyapunov or Riccati equations, we have proposed in \cite{Pierre2022} efficient tools and algorithms to solve finite-dimensional truncated versions of the problem with theoretical guarantees. These tools have been illustrated on the LQ problem using a Kleinman like algorithm in the harmonic setting. They have also been applied in \cite{Ried2022} to provide a harmonic pole placement procedure that assigns the poles of the closed loop harmonic model to some desired locations. A non-trivial and important difficulty is to guarantee that the obtained truncated solution converges with the order of truncation to the original infinite dimensional solution. To illustrate and explain the encountered difficulties, recall the following fact (see \cite{Pierre2022} for more detail): "the $m-$truncated matrix $(\mathcal{A}-\mathcal{N})_m$ of the infinite dimensional matrix $\mathcal{A}-\mathcal{N}$ may not be Hurwitz at any order $m$ even if $\mathcal{A}-\mathcal{N}$ is". As a consequence, if one attempts to solve the following truncated version of the infinite dimensional harmonic Lyapunov equation:
%$$(\mathcal{A}-\mathcal{N})_m^*\mathcal{P}_m+\mathcal{P}_m(\mathcal{A}-\mathcal{N})_m+\mathcal{Q}_m=0$$
%there may be no chance to obtain a positive definite solution $\mathcal{P}_m$ for any $m \in \mathbb{Z}^+$.
%In \cite{Pierre2022}, by exploiting the Toeplitz structure of these equations, efficient methods and algorithms have been proposed to determine up to an arbitrarily small error the infinite dimensional solution of harmonic Lyapunov or Riccati equations. Here, our objective is to extend this philosophy to the LMI framework and show how to define a truncated version of TBLMIs to recover the infinite-dimensional solution up to an arbitrary small error. TBLMI formulations are obviously of interest to consider extensions to robustness and multi-objective control design problems. 
\vspace{-.2cm}
\section{Main results}
\vspace{-.1cm}
\subsection{Truncation operator $\Pi$}
\begin{definition}\label{proj} Consider infinite-dimensional Toeplitz block matrices $\mathcal{A}$ and $\mathcal{B}$ of compatible size. The truncation operator $\Pi_m$ at order $m$ is determined by:
\begin{align}
\Pi_m(\mathcal{A}) &=\mathcal{T}_m(A)\nonumber\\
\Pi_m(\mathcal{A}+\mathcal{B})&=\Pi_m(\mathcal{A}) +\Pi_m(\mathcal{B})\nonumber\\
\Pi_m(\mathcal{AB})&= \Pi_m(\mathcal{A})\Pi_m(\mathcal{B})\nonumber \\&+\mathcal{H}_{(m,\eta)}(A^+)\mathcal{H}_{(\eta,m)}(B^-) \label{pro} \\&+\mathcal{J}_{n,m} \mathcal{H}_{(m,\eta)}(A^-)\mathcal{H}_{(\eta,m)}(B^+)\mathcal{J}_{n,m}\nonumber 
\end{align}
where $\eta\in \mathbb{Z}^+\cup \{+\infty\}$ is such that $2\eta\geq \min d^o (A(z),B(z))$.
\end{definition}

%\begin{definition}A harmonic infinite-dimensional LMI is a linear matrix inequality that involves 
%Toeplitz block and bounded operators on $\ell^2$ 
%
%
%
%
%and, in addition, a Toeplitz block part that contains the unbounded diagonal operator \eqref{N}. 
%	% that appear linearly 
%%	involving finite number of sums and products of block infinite-dimensional Toeplitz matrices of compatible size and by unbounded operator \eqref{N} and 
%Generically, it is defined as a linear matrix inequality:
%\begin{equation}
%\mathcal{L}(\mathcal{P})<0
%\label{LMg}
%\end{equation}
%whose solutions $\mathcal{P}$ are assumed to be Toeplitz block and bounded operator on~$\ell^2$.
For a given $m$, the $m-$truncated TBLMI of \eqref{LMg} is:
 \begin{equation}\mathcal{L}_m(\mathcal{P})=\Pi_m(\mathcal{L}(\mathcal{P}))<0\label{lmi_trunc}\end{equation}
%\end{definition}

%\begin{definition}A harmonic infinite-dimensional LMI is a linear matrix inequality that involves 
%Toeplitz block and bounded operators on $\ell^2$ 
%and, in addition, a Toeplitz block part that contains the unbounded diagonal operator \eqref{N}. 
%	% that appear linearly 
%%	involving finite number of sums and products of block infinite-dimensional Toeplitz matrices of compatible size and by unbounded operator \eqref{N} and 
%Generically, it is defined as a linear matrix inequality:
%\begin{equation}
%\mathcal{L}(\mathcal{P})<0
%\label{LMg}
%\end{equation}
%whose solutions $\mathcal{P}$ are assumed to be Toeplitz block and bounded operator on~$\ell^2$.
%For a given $m$, the corresponding $m-$truncated LMI is defined by:
% \begin{equation}\mathcal{L}_m(\mathcal{P})=\Pi_m(\mathcal{L}(\mathcal{P}))<0\label{lmi_trunc}\end{equation}
%\end{definition}
%\bigskip

Thus, the $m-$truncated TBLMI associated to \eqref{lyap}~is: %$$\mathcal{L}(\mathcal{P})=(\mathcal{A}-\mathcal{N})^*\mathcal{P}+\mathcal{P}(\mathcal{A}-\mathcal{N})<0$$ 
\begin{align}&\mathcal{L}_m(\mathcal{P})=(\mathcal{A}-\mathcal{N})_m^*\mathcal{P}_m+\mathcal{P}_m(\mathcal{A}-\mathcal{N})_m\nonumber\\
& +\mathcal{H}_{(m,\eta)}(A^{*+})\mathcal{H}_{(\eta,m)}(P^-)+\mathcal{H}_{(m,\eta)}(P^{+})\mathcal{H}_{(\eta,m)}(A^{-})\nonumber\\
 &+\mathcal{J}_{n,m} (\mathcal{H}_{(m,\eta)}(A^{*-})\mathcal{H}_{(\eta,m)}(P^+)\nonumber\\&+ \mathcal{H}_{(m,\eta)}(P^-)\mathcal{H}_{(\eta,m)}(A^+))\mathcal{J}_{n,m}<0\nonumber
\end{align}
with $\eta\in \mathbb{Z}^+\cup \{+\infty\}$ such that $2\eta\geq \min d^o (A(z),P(z))$.
Notice that it is straightforward to show that:
\begin{enumerate}\item the term $\mathcal{N}^*\mathcal{P}+\mathcal{P}\mathcal{N}$ is a Toeplitz block matrix 
\item there are no correcting terms associated to the product $\mathcal{P}\mathcal{N}$ or $\mathcal{N}^*\mathcal{P}$ since $\mathcal{N}$ is a diagonal operator.
\end{enumerate}
\begin{theorem}\label{sol_trunc}If $\mathcal{P}$ solves the infinite-dimensional TBLMI (\ref{LMg}) then $\mathcal{P}$ %_m=\Pi_m(\mathcal{P})$ 
solves the truncated TBLMI (\ref{lmi_trunc}) at any order~$m$.
\end{theorem}
\begin{proof} Consider a solution $\mathcal{P}$ of $\mathcal{L}(\mathcal{P})<0$ then for any $m>0$, the principal submatrix $(\mathcal{L}(\mathcal{P}))_m$ (see Definition \ref{deftrunc}) is necessarily negative definite. Moreover, using \eqref{ee2} and Definition~\ref{proj}, the principal submatrix $(\mathcal{L}(\mathcal{P}))_m$ can be explicitly determined by applying the operator $\Pi_m$ on $\mathcal{L}(\mathcal{P})$. Therefore, a solution of the obtained $m-$truncated TBLMI can be deduced from $\mathcal{P}$ itself.
%$\mathcal{P}_m=\Pi_m(\mathcal{P})$.
\end{proof}
If the infinite-dimensional TBLMI (\ref{LMg}) is feasible then there always exists a solution to the truncated TBLMI (\ref{lmi_trunc}) at any order $m$. 
Unfortunately, \eqref{lmi_trunc} may contain terms involving infinite-dimensional Hankel matrices (when $\eta=+\infty$). 
\subsection{Banded approximation of Toeplitz block operator}
Consider a Toeplitz block operator $\mathcal{A}$, bounded on $\ell^2$ with its associated symbol $A(z)$. Consider also its $p-$banded version $\mathcal{A}_{b(p)}$ whose associated symbols are $a_{{b(p)}_{ij}}(z):=\sum_{k=-p}^pa_{ij,k}z^k$, for $i,j=1,\cdots,n$.
%In other words, if the $(i,j)$ entries $a_{ij}(z)$ of $A(z)$ with $i,j=1,\cdots, n $ are given by $a_{ij}(z)=\sum_{k\in \mathbb{Z}}a_{ij,k}z^k$
%then the corresponding entries of $A_{b(p)}(z)$ are given by $a_{{b(p)}_{ij}}(z)=\sum_{k=-p}^pa_{ij,k}z^k$.
\begin{theorem}\label{conv_l2}Assume that $\mathcal{A}$ is a bounded operator on $\ell^2$. The operator $\mathcal{A}_{b(p)}$ converges to $\mathcal{A}$ in $\ell^2$-operator norm i.e.
$$\lim_{p\rightarrow +\infty}\|\mathcal{A}-\mathcal{A}_{b(p)}\|_{\ell^2}=0$$
\end{theorem}
\begin{proof}It is sufficient to give the proof for $n=1$. 
	Recall that $\|\mathcal{A}\|_{\ell^2}=\|A\|_{L^\infty}$ where $\mathcal{A}=\mathcal{T}(A)$ (See Part V p.p. 562-574 of \cite{Gohberg}). We have: 
\begin{align}
\|\mathcal{A}-\mathcal{A}_{b(p)}\|_{\ell^2}& =\|A-A_{b(p)}\|_{L^\infty}=\|\sum_{|k|>p} A_ke^{ \textsf{j}\omega kt} \|_{L^\infty}\label{e1}%=\sup_{\|X\|_{\ell^2}=1}\|(\mathcal{A}-\mathcal{A}_{b(p)})X\|_{\ell^2}\\
%&=\sup_{\|x(t)\|_{L^2}=1}\|(A(t)-A_{b(p)}(t))x(t)\|_{L^2}\\
%&=\sup_{\|x(t)\|_{L^2}=1}\| \sum_{|k|>p} A_ke^{ \textsf{j}\omega kt} x(t) \|_{L^2}\\
\end{align}
As by assumption there exists a constant $C$ such that
\begin{align*}
\|\mathcal{A}\|_{\ell^2}=\|A\|_{L^\infty}
&= \|\sum_{k\in \mathbb{Z}} A_ke^{ \textsf{j}\omega kt} \|_{L^\infty}<C
\end{align*}
the series $\sum_{k\in \mathbb{Z}} A_ke^{ \textsf{j}\omega kt}$ converges almost everywhere and $\lim_{p\rightarrow +\infty}\sum_{|k|>p} A_ke^{ \textsf{j}\omega kt}=0\ a.e.$ 
Taking the limit w.r.t. $p$ in \eqref{e1} leads to the result. 
\end{proof}
This result allows to replace any TBLMI by its banded version as stated in the following definition.

\begin{definition}
The $p-$banded TBLMI is defined by
$\mathcal{L}(\mathcal{P}; \mathcal{A}_{i_{b(p)}}, i\in S)<0$
\end{definition}

\begin{theorem}\label{t1}Under Assumption \ref{bound}, if $\mathcal{P}$ is a solution of \eqref{LMg} then for any $\epsilon>0$, there exists $p_0$ such that for $p\geq p_0$, 
$$\|\mathcal{L}(\mathcal{P};\mathcal{A}_{i}, i\in S)-\mathcal{L}(\mathcal{P};\mathcal{A}_{i_{b(p)}}, i\in S)\|_{\ell^2}<\epsilon$$
which implies that $\mathcal{P}$ satisfies the p-banded TBLMI:
\begin{equation}\mathcal{L}(\mathcal{P};\mathcal{A}_{i_{b(p)}}, i\in S)<0 \label{blmi}\end{equation} for sufficiently small $\epsilon$.
\end{theorem}
\begin{proof}
By assumption~\ref{bound}, the only entry of the TBLMI not bounded on $\ell^2$ is $\mathcal{N}$. Fortunately, as $\mathcal{N}$ is diagonal, $\mathcal{N}=\mathcal{N}_{i_{b(p)}}$ for any $p\geq 0$ and thus $\mathcal{N}$ does not play any role.
If $\mathcal{P}$ is a solution of \eqref{LMg}, then $\mathcal{L}(\mathcal{P};\mathcal{A}_{i}, i\in S)$ must be a bounded operator on $\ell^2$. As LMIs are continuous with respect to their entries, there exists a constant $M$ depending of $\mathcal{P},\mathcal{A}_i,$ $\mathcal{A}_{i_{b(p)}}$, $i\in S$ such that 
\begin{align*}\|\mathcal{L}(\mathcal{P};\mathcal{A}_{i}, i\in S)-\mathcal{L}(\mathcal{P};&\mathcal{A}_{i_{b(p)}}, i\in S)\|_{\ell^2}\\&\leq M \sum_{i\in S} \|\mathcal{A}_{i}-\mathcal{A}_{i_{b(p)}}\|_{\ell^2}\end{align*}
From Theorem \ref{conv_l2}, we conclude that for any $\epsilon>0$, there exists $p_0$ such that for $p\geq p_0$, 
\begin{align*}\|\mathcal{L}(\mathcal{P};\mathcal{A}_{i}, i\in S)-&\mathcal{L}(\mathcal{P};\mathcal{A}_{i_{b(p)}}, i\in S)\|_{\ell^2}<\epsilon\end{align*}
and relation \eqref{blmi} follows. 
\end{proof}
Consider the $m-$truncated and $p-$banded TBLMI: 
\begin{equation} 
\Pi_m(\mathcal{L}(\mathcal{P};\mathcal{A}_{i_{b(p)}}, i\in S))<0 \label{lmi_trunc3}.
\end{equation}
\begin{theorem}\label{finite}For given $p$ and $m$, the number of unknowns $\nu(p,m)$ involved in \eqref{lmi_trunc3} is finite. 
\end{theorem}
\begin{proof} As all $\mathcal{A}_{i_{b(p)}}$, $i\in S$ in $\mathcal{L}(\mathcal{P};\mathcal{A}_{i_{b(p)}}, i\in S)<0$ are banded, only the unknown $\mathcal{P}$ is possibly not banded.
As the product of infinite dimensional banded Toeplitz block operators is a banded Toeplitz block operator (which is not true in finite dimension), the terms in the TBLMI involving operator $\mathcal{P}$ have the generic form: 
$\mathcal{U}\mathcal{P}\mathcal{V}$ where $\mathcal{U}$ and $\mathcal{V}$ are polynomial functions of the banded entries $\mathcal{A}_{i_{b(p)}}, i\in S$ and are therefore banded. 
Applying $\Pi_m$ on $\mathcal{L}(\mathcal{P};\mathcal{A}_{i_{b(p)}}, i\in S)$ leads to compute $\Pi_m(\mathcal{U}\mathcal{P}\mathcal{V})$. Using \eqref{pro}, we have:
\begin{align}
\Pi_m(\mathcal{U}\mathcal{P}\mathcal{V})&= \Pi_m(\mathcal{U})\Pi_m(\mathcal{PV})\label{ee1} \\
&+\mathcal{H}_{(m,\eta_1)}(U^+)\mathcal{H}_{(\eta_1,m)}(PV^-) \nonumber \\
&+\mathcal{J}_{n,m} \mathcal{H}_{(m,\eta_1)}(U^-)\mathcal{H}_{(\eta_1,m)}(PV^+)\mathcal{J}_{n,m}\nonumber 
\end{align}
where $\eta_1$ is the first integer greater than $\frac{1}{2} d^o U(z)$ and where $\Pi_m(\mathcal{PV)}$ is determined using \eqref{pro} with 
%\begin{align}
%\Pi_m(\mathcal{PV})&= \Pi_m(\mathcal{P})\Pi_m(\mathcal{V})\label{ee2} \\
%&+\mathcal{H}_{(m,\eta_2)}(P^+)\mathcal{H}_{(\eta_2,m)}(V^-) \nonumber\\
%&+\mathcal{J}_{n,m} \mathcal{H}_{(m,\eta_2)}(P^-)\mathcal{H}_{(\eta_2,m)}(V^+)\mathcal{J}_{n,m}\nonumber
%\end{align}
$\eta$ the first integer greater than $\frac{1}{2} d^o V(z)$. Notice that the coefficient of higher degree invoked in the Hankel matrix $\mathcal{H}_{(m,\eta)}(P^+)$ is of degree $2(m+\eta)+1$. It is straightforward to check that only a finite number of phasors of $\mathcal{P}$ are necessary to compute both $\Pi_m(\mathcal{PV)}$ and \eqref{ee1} and thus the result follows. %and \eqref{ee2}, the result is established. 
\end{proof}
From this result, solving \eqref{lmi_trunc3} is now tractable.
\begin{definition}
For given $m$ and $p$, a minimal solution of \eqref{lmi_trunc3} denoted by $\mathcal{P}_s(p,m)$ is any solution whose phasors not involved in \eqref{lmi_trunc3} are set to zero. The set of indices corresponding to phasors not involved in~\eqref{lmi_trunc3} is denoted by: \begin{align}
S_{nul}(m,p)&=\{ (i,j,k)\in n\times n\times \mathbb{Z}: \nonumber\\ &P_{ij,k} \text{ does not appear in \eqref{lmi_trunc3}}\}\label{Snul}\end{align}
\end{definition}
Any minimal solution is a banded Toeplitz block operator. We can associate to any solution of \eqref{lmi_trunc3} a minimal solution.
%\begin{theorem}For a fixed $p\geq0$ and a given $m_0$, consider a minimal solution $\mathcal{P}_s(p,m_0)$ of \eqref{lmi_trunc3} then $\mathcal{P}_s(p,m_0)$ solves \eqref{lmi_trunc3} for any $m\leq m_0$. 
%\end{theorem}
%\begin{proof}For a fixed $p$, if $\mathcal{P}_s(p,m_0)$ solves \eqref{lmi_trunc3} with $m=m_0$, then any principal submatrices are also negative definite and in particular $\mathcal{P}_s(p,m_0)$ is thus admissible for \eqref{lmi_trunc3} with $m<m_0$.
%\end{proof}
\begin{theorem}\label{proche}
Consider a solution $\mathcal{P}$ of (\ref{LMg}). There exists $p_0$ such that for any $p\geq p_0$ and any $m>0$, $\mathcal{P}_s(p,m)$ satisfies \eqref{lmi_trunc3} where $\mathcal{P}_s(p,m)$ is the minimal solution associated to~$\mathcal{P}$. 
\end{theorem}
\begin{proof}First, $p_0$ is chosen such that $\mathcal{P}$ satisfies \eqref{blmi}.
Then, Theorem~\ref{sol_trunc} implies that the truncated version of \eqref{blmi}, that is \eqref{lmi_trunc3}, is also solved by $\mathcal{P}$ at any order $m$. Given $m$ and $p\geq p_0$, $\mathcal{P}_s(p,m)$ is deduced from $\mathcal{P}$ by cancelling any unknown not involved in \eqref{lmi_trunc3} that is: \\
$
P_{s_{ij,k}}=\begin{cases}0 \text{ if }(i,j,k)\in S_{nul}(m,p)\\
P_{ij,k} \text{ elsewhere}
\end{cases}
$
%\vspace{-.5cm}
\end{proof}

\begin{theorem}\label{proche2}
For any $\epsilon>0$, there exists $m_0$ such that for any $m\geq m_0$, $\mathcal{P}_s(m,m)$ solves \eqref{lmi_trunc3} with $p=m$ and 
$$\|\mathcal{P}-\mathcal{P}_s(m,m)\|_{\ell^2}<\epsilon$$
\end{theorem}
\begin{proof} If we fix $p=m$ in \eqref{lmi_trunc3}, then $m$ must be chosen larger than $p_0$ in Theorem~\ref{t1} to guarantee that $\mathcal{P}$ satisfies \eqref{blmi} with $p=m$. 
As the band size of $\mathcal{P}_s(m,m)$ increases with $m$ and as $\mathcal{P}$ is a bounded operator on $\ell^2$, there exists $m_1$ such that for $m\geq m_1$, $\|\mathcal{P}-\mathcal{P}_s(m,m)\|_{\ell^2}<\epsilon$.
Finally, $m_0=\max(p_0,m_1)$ leads to the result.
\end{proof}
%To avoid this practical difficulty, 
%
%
%
%
%we introduce a modified operator as follows:
%
%\begin{definition}\label{proj2}The modified operator $\tilde\Pi_m$ of $\Pi_m$ is determined by
%\begin{align}\tilde \Pi_m(\mathcal{A})&=\Pi_m(\mathcal{A}_{b(m)})\\
%	\tilde \Pi_m(\mathcal{A}+\mathcal{B})&=\Pi_m(\mathcal{A}+\mathcal{B}) \\
%\tilde \Pi_m(\mathcal{AB})&= \Pi_m(\mathcal{A})\Pi_m(\mathcal{B})\nonumber \\&+\mathcal{H}_{m}(A^+)\mathcal{H}_{m}(B^-) \label{pro2} \\&+\mathcal{J}_{n,m} \mathcal{H}_{m}(A^-)\mathcal{H}_{m}(B^+)\mathcal{J}_{n,m} \nonumber\end{align}
%where $\mathcal{H}_{m}=\mathcal{H}_{(m,m)}$.
%\end{definition}
%
%\begin{theorem}\label{eps} Assume that $\mathcal{A}=\mathcal{T}(A)$ and $\mathcal{B}=\mathcal{T}(B)$ are bounded operators on $\ell^2$ (or equivalently that the time periodic matrices $A(t)$ and $B(t)$ belong to $L^{\infty}([0\ T])$). Define the $(2m+1)\times (2m+1)$ block matrix $\mathcal{M}_m(\mathcal{AB})$ by:
%	$$\mathcal{M}_m(\mathcal{AB})=\tilde \Pi_m(\mathcal{AB})- \Pi_m(\mathcal{AB}).$$ 
%Then, for any $\epsilon>0$, there exists $m_0$ such that for any $m\geq m_0$, $$\|\mathcal{M}_m(\mathcal{AB})\|_\infty<\epsilon$$
%\end{theorem}
%
%\begin{proof} It is sufficient to give the proof for $n=1$. First of all, if one of the Toeplitz matrix $\mathcal{A}$ or $\mathcal{B}$ is banded then for $2m_0\geq
%\min d^o (A(z),B(z))$, Equation~\eqref{pro} implies, for any $m\geq m_0$: $$\mathcal{M}_m(\mathcal{AB})=\tilde \Pi_m(\mathcal{AB})- \Pi_m(\mathcal{AB})=0.$$ If not then $\eta=\infty$ in \eqref{pro}. We first decompose $\mathcal{H}_{(m,\infty)}(A^+)$ into
%$$\mathcal{H}_{(m,\infty)}(A^+)= [\mathcal{H}_m(A^+) \ \mathcal{H}_{m^+}(A^+)]$$
%where $\mathcal{H}_{m}(A^+)=\mathcal{H}_{(m,m)}(A^+)$ and where $\mathcal{H}_{m^+}(A^+)$ is a $(2m+1) \times \infty$ Hankel matrix associated to the infinite sequence $(a_{2m+2},\ a_{2m+3}, \cdots)$. We also decompose $\mathcal{H}_{(\infty,m)}(B^-)$ into
%$$\mathcal{H}_{(\infty,m)}(B^-)=\left[\begin{array}{c}\mathcal{H}_m(B^-) \\\mathcal{H}_{m^-}(B^-)\end{array}\right]$$
%where $\mathcal{H}_{m^-}(B^-)$ is a $\infty \times (2m+1)$ Hankel matrix associated to the infinite sequence $(b_{-2m-2},\ b_{-2m-3}, \cdots$). This allows to get:
%\begin{align}\mathcal{H}_{(m,\infty)}(A^+)\mathcal{H}_{(\infty,m)}(B^-)-&\mathcal{H}_m(A^+)\mathcal{H}_m(B^-)\nonumber\\&=\mathcal{H}_{m^+}(A^+)\mathcal{H}_{m^-}(B^-).\end{align}
%Let us show that for a given $\epsilon$ and sufficiently large $m$, the $(2m+1) \times (2m+1)$ matrix $\mathcal{H}_{m^+}(A^+)\mathcal{H}_{m^-}(B^-)$ has all its components less than $\epsilon$.
%For $j=1,\cdots,(2m+1)$, consider the $j-th$ column of $\mathcal{H}_{m^-}(B^-)$ defined by the sequence $V_m(j)=(b_{-2m-1-j},\ b_{-2m-2-j},\cdots)$.
%Then, the following bound can be established: 
%\begin{align}\|\mathcal{H}_{m^+}(A^+)V_m(j)\|_{\ell^2}&\leq \|\mathcal{H}_{m^+}(A^+)\|_{\ell^2}\|V_m(j)\|_{\ell^2}\\
%&\leq \| A \|_{L^\infty}\|V_m(j)\|_{\ell^2}\end{align}
%since $\|\mathcal{H}_{m^+}(A^+)\|_{\ell^2}\leq \|\mathcal{H}(A^+)\|_{\ell^2}=\| A \|_{L^\infty}$ (See Part V p.p. 562-574 of \cite{Gohberg}).
%As for any $j=1,\cdots,(2m+1)$, $$\|V_m(j)\|_{\ell^2}\leq \sum_{k=2m+2}^{+\infty} |b_{-k}|^2$$
%it follows that $$\|\mathcal{H}_{m^+}(A^+)V_m(j)\|_{\ell^2}\leq \| A \|_{L^\infty}\sum_{k>2m+1} |b_{-k}|^2$$
%and we conclude that there exists $m_0$ such that for any $m\geq m_0$ and any $j=1,\cdots,(2m+1)$
%$$\|\mathcal{H}_{m^+}(A^+)V_m(j)\|_{\ell^2}\leq \epsilon/2.$$
%Now, using similar steps for the term $\mathcal{J}_{n,m} \mathcal{H}_{m^-}(A^-)\mathcal{H}_{m^+}(B^+)\mathcal{J}_{n,m}$ and noticing that $\|\mathcal{J}_{n,m}\|=1$, it is straightforward to establish the result noticing that
%$\mathcal{M}_m(\mathcal{AB})=\tilde \Pi_m(\mathcal{AB})- \Pi_m(\mathcal{AB})=\mathcal{H}_{m^+}(A^+)\mathcal{H}_{m^-}(B^-)+\mathcal{J}_{n,m} \mathcal{H}_{m^-}(A^-)\mathcal{H}_{m^+}(B^+)\mathcal{J}_{n,m}$ with here $n=1$.
%\end{proof}
%
%Now, let us consider the modified $m-$truncated LMI: %denote by $\tilde{\mathcal{P}}_m$ a solution of 
%\begin{equation} %\tilde{\mathcal{L}}_m(\mathcal{P})=
%\tilde\Pi_m(\mathcal{L}(\mathcal{P}))<0\label{lmi_trunc2}.\end{equation}
%
%\begin{theorem}\label{finite}The number of unknowns $\tilde m$ involved in Equation \eqref{lmi_trunc2} is finite. 
%\end{theorem}
%\begin{proof}
%As the modified $m-$truncation operator applied on a product of operators involves now only a finite number of phasors (see \eqref{pro2}) thanks to the replacement of $\mathcal{H}_{(m,\infty)}(\cdot)$ by $\mathcal{H}_{m}(\cdot)$), the result is obvious noticing that an LMI is formed by a finite number of sums and products between operators (as requested by Assumption~1).
%\end{proof}
%\begin{definition}\label{symsol}For a given $m$, we denote by $\tilde P_m(z)$ the symbol associated to a solution of LMI (\ref{lmi_trunc2}). Since the number of unknowns $\tilde m$ is finite, we assume that the degree of $\tilde P_m(z)$ is finite by setting the phasors not involved by \eqref{lmi_trunc2} to zero. %In the sequel, $\mathcal{U}(m)$ denotes the index set of $\tilde m$ phasor unknowns. 
%\end{definition}
%
%
%
%
%In view of Theorem~\ref{eps} and as solutions of an LMI depends continuously on its inputs, it is now clear that 
%for a given solution of \eqref{lmi_trunc}, there exists a close solution of \eqref{lmi_trunc2} provided that $m$ is chosen large enough.
%
%
%
%%Moreover, since this modified truncated LMI \eqref{lmi_trunc2} involves a finite number of phasors for all of its inputs, (thanks to the replacement of $\mathcal{H}_{(m,\infty)}(\cdot)$ by $\mathcal{H}_{m}(\cdot)$) and since the LMI has a Toeplitz block structure, solving \eqref{lmi_trunc2} is equivalent to solving a banded version of \eqref{LMg}.
%
%
%\begin{theorem}\label{proche}Let denote by $P(z)$ the symbol associated to a solution $\mathcal{P}$ of the infinite-dimensional LMI (\ref{LMg}). %and for a given $m$, by $\tilde P_m(z)$ the symbol associated to a solution of the modified $m-$truncated LMI \eqref{lmi_trunc2}.
%
%For any solution $\mathcal{P}$ of (\ref{LMg}) and for any $\epsilon>0$, there exists $m_0$ such that for any $m\geq m_0$, there exists a symbol $\tilde {{P}}_m(z)$ associated to a solution of \eqref{lmi_trunc2} that fulfils $$\|P(z)-\tilde P_m(z)\|_{\ell^2}\leq \epsilon.$$
% \end{theorem}
%\begin{proof} It is sufficient to give the proof for $n= 1$. 
%Let us consider the symbol $P(z)$ associated to a solution of (\ref{LMg}).
%Theorem~\ref{sol_trunc} implies that this symbol solves also LMI \eqref{lmi_trunc}.
%
% 
%% and define the $m-$truncated symbol $P_m(z)=\sum_{k=-m}^mp_kz^k$ associated to $P(z)=\sum_{k=-\infty}^{+\infty}p_kz^k$. 
%%Recall that it is assumed that any solution $\mathcal{P}$ is a bounded operator on $\ell^2$, then for all $\epsilon>0$, there exists $m_1$ such that for any $m\geq m_1$, $\|P_m(z)-P(z)\|_{\ell^2}\leq\epsilon/2$. 
%%Moreover, Theorem~\ref{sol_trunc} 
%%implies that the matrix ${\mathcal{P}}_m$ associated to $P_m(z)$ solves LMI \eqref{lmi_trunc}. 
%%
%On the other hand, Theorem \ref{eps} implies, by continuity of the solutions of LMI with respect to their entries, that for any $\epsilon>0$, there exists $m_0$ such that for any $m\geq m_0$, LMI \eqref{lmi_trunc2} admits a solution whose symbol $\tilde P_m(z)$ is of finite degree and satisfies $\|\tilde P_m(z)-P(z)\|_{\ell^2}\leq \epsilon$. 
%
%%The conclusion follows taking $m_0=\max(m_1,m_2)$.
%\end{proof}
%%Conversely, it is not at all obvious that any solution of \eqref{lmi_trunc2} allows to approach a solution of (\ref{LMg}) for a sufficiently large $m$. This 
%%
%%
%%To prove by increasing $m$ that it is possible to approach a solution for a suficiently large m
%%Thus and practically, to obtain a good enough approximation $\tilde{\mathcal{P}}_m$ of the infinite dimensional LMI (\ref{LMg}) and since the phasor sequence of any infinite dimensional solution vanishes, it is sufficient to increase $m$ until the phasors of higher degree of a modified truncated solution $\tilde {\mathcal{P}}_m$ vanish. More precisely, we have:
%%
\subsection{Solving {\bf COP}$_{\infty,\infty}$ up to an arbitrary error}
%We consider the vectorial space $S^n$ of $T-periodic$, $L^\infty([0\ T])$ and symmetric $n\times n$ matrix functions.
%We define the scalar product on $S^n\times S^n$ by: 
%\begin{align}
%<M,N>_{S^n}&=\frac{1}{T}\int_0^Ttr(M(\tau)N(\tau))d\tau%\\
%%&=\frac{1}{T}\int_0^T\sum_{i,j=1}^n M_{ij}(\tau)N_{ij}(\tau)d\tau
%\end{align}
%%and the induced norm $<M,M>^{1/2}=\|M\|_F$ corresponds to the Frobenius norm. 
%We denote $S^n_+=\{M\in S^n: M\geq 0\text{ on }L^2\}$ and $S^n_{++}=\{M\in S^n: M> 0 \text{ on }L^2\}$
%and we recall that $M\geq 0 \text{ on }L^2([0\ T])$ means that for any $x\in L^2([0\ T])$,$$<x,Mx>_{L^2}=\frac{1}{T}\int_0^Tx'(\tau)M(\tau)x(\tau)d\tau \geq 0$$ and is equivalent to the condition $M(t)\geq0$ a.e..
%We recall also that $M(\cdot)\in S^n_{++}$ if and only if $\mathcal{M}=\mathcal{T}(M)$ is a constant, hermitian, positive definite, Toeplitz block and bounded on $\ell^2$ operator. 
%
%
% 
%
%%\begin{proposition}
%%\begin{itemize}
%%\item$M>0$ a.e. if and only if for any $N\geq 0$ a.e. , $N\neq 0$ a.e., $<N,M>_{S^n}>0$.
%%\item $M\geq 0$ a.e. if and only if for any $N\geq 0$ a.e., $<N,M>_{S^n} \geq0$.
%%\end{itemize}
%%
%%\end{proposition}
%
%If $M\in S^n_{++}$, as any component of $M$ can be rewritten using the Fourier series as:
%$$M_{ij}(t)=\sum_{k\in \mathbb{Z}} m_{ij,k} e^{ \textsf{j}\omega kt} a.e.,$$
%it follows that the product: $$<Id,M>_{S^n}=\sum_{i=1}^n m_{ii,0}>0.$$
%%Therefore, the trace operator can be defined as follows:
%\begin{definition}The trace operator for hermitian, positive definite, Toeplitz block and bounded operators on $\ell^2$ is defined by
%\begin{equation}
%tr(\mathcal{M})=\sum_{i=1}^n m_{ii,0}\label{tr}\end{equation}
%%and we have:
% %$tr(\mathcal{M}\mathcal{N})=<M,N>_{S^n}$
%%where $\mathcal{M}=\mathcal{T}(M)$ and $\mathcal{N}=\mathcal{T}(N)$. 
%\end{definition}
%Note that if $M\in S^n_{+}$, $tr(\mathcal{M})=0$ implies that $\mathcal{M}=0$. Therefore, it is straightforward to check that $tr(\mathcal{M})$ defines a norm for this operator class.


Consider for a given $p>0$, the $p-$banded version of ${\bf COP_{\infty,\infty}}$ given by: 
\begin{align*}
{\bf COP_{\infty,p}:} \min_{\mathcal{P}^*=\mathcal{P}>0} F(\mathcal{P}) \text{ subject to:}\\
\mathcal{L}(\mathcal{P};\mathcal{A}_{i_{b(p)}}, i\in S)\leq 0\nonumber
\end{align*}
and for a given $N>0$, the following fully banded problem: 
\begin{align*}{\bf COP_{\infty,p,N}:} \min_{\mathcal{P}^*=\mathcal{P}>0} F(\mathcal{P}) \text{ subject to:} \\
\mathcal{L}(\mathcal{P};\mathcal{A}_{i_{b(p)}}, i\in S)\leq 0, \quad%\nonumber \\
P_{ij,k}=0 \text{ for } |k|>N\nonumber
\end{align*}

We assume that these convex optimization problems are feasible and that the optimal solution is unique and continuous with respect to the entries $\mathcal{A}_i$, $i\in S$. 
%Further more any unbounded sequence of feasible points (if any) produces an unbounded sequence of objectives. 
%Note that Problem $\eqref{P3}$ can be viewed as a finite dimensional problem since the number of unknowns is finite.
Given $p$ and $N$, denote by $\hat{\mathcal{P}}_{\infty,\infty}$, $\hat{\mathcal{P}}_{\infty,p}$ and $\hat{\mathcal{P}}_{\infty,p,N}$, the optimal solution of ${\bf COP_{\infty,\infty}}$, ${\bf COP_{\infty,p}}$ and ${\bf COP_{\infty,p,N}}$ respectively.

\begin{theorem}\label{prop} The sequence $\hat{\mathcal{P}}_{\infty,p}$ indexed by $p$ converges on $\ell^2$ to $\hat{\mathcal{P}}_{\infty,\infty}$.
For a given $p$, the sequence, $\hat{\mathcal{P}}_{\infty,p,N}$ indexed by $N$ converges on $\ell^2$ to $\hat{\mathcal{P}}_{\infty,p}$.
\end{theorem}
\begin{proof}The first assertion is a direct consequence of the continuity of the optimal solution with respect to the entries $\mathcal{A}_i$, $i\in S$ and of Theorem \ref{conv_l2}.
To prove the last assertion, consider a sequence $\nu$ of positive real numbers and:
\begin{align*}{\bf COP_{\infty,p}(\nu):} \min_{\mathcal{P}^*=\mathcal{P}>0} F(\mathcal{P}) \text{ subject to:} \\
\mathcal{L}(\mathcal{P};\mathcal{A}_{i_{b(p)}}, i\in S)\leq 0,\quad%\nonumber \\
|P_{ij,k}|\leq \nu_k,\ k\in \mathbb{Z}\nonumber
\end{align*}
Problem ${\bf COP_{\infty,p,N}}$ for any $N$ can be restated in this relaxation form. Indeed, ${\bf COP_{\infty,p,N}}$ is obtained by choosing 
\begin{equation} \nu_k=0, \text{ for }|k|> N \text{ and }
\nu_k=\Gamma \text{ for }|k|\leq N
%\end{cases}
\label{nuN}
\end{equation} where $\Gamma$ is large enough that the latter constraints are not active at the optimum. This choice is always possible since the optimal solution of ${\bf COP_{\infty,p,N}}$ is assumed to be bounded on $\ell^2$ which implies that its phasors belong to $\ell^2$. 

Moreover, for a given $\gamma$, since the phasors of $\hat{\mathcal{P}}_{\infty,p}$ belong to $\ell^2$, it is clear that there exists $N_0>0$, such that the phasors of $\hat{\mathcal{P}}_{\infty,p}$ satisfy $|P_{ij,k}|< \gamma \text{ for } |k|> N_0$. Therefore, if we fix the sequence $\nu$ as follows: 
\begin{equation}% \begin{cases}
\nu_k=\gamma, \text{ for }|k|> N%\\
\text{ and }\nu_k=\Gamma \text{ for }|k|\leq N
%\end{cases}
\label{nustar}
\end{equation} where $\Gamma$ is a sufficiently large value, then, by uniqueness of solution, the optimal solution of this relaxed problem denoted by $\hat{\mathcal{P}}_{\infty,p}(\nu)$ is equal to $\hat{\mathcal{P}}_{\infty,p}$, for any $N\geq N_0$. As these constraints are linear with respect to both the unknowns and $\nu$, the optimization problem remains convex and the optimal solution is continuous with respect to $\nu$. 
 
 %for any $N$ and for any $\epsilon>0$ there exists $\eta>0$ such that for $\nu<\eta$, $$\|\hat{\mathcal{P}}_{\infty,p,N}(\nu)-\hat{\mathcal{P}}_{\infty,p,N}(0)\|_{\ell^2}<\epsilon$$
 

 As a consequence, for a given sequence $\nu$ and $\eta>0$, for any $\|\nu-\nu^*\|_{\ell^{\infty}} \leq\eta$, there exists a constant $M(\eta,\nu)$ such that $\|\hat{\mathcal{P}}_{\infty,p}(\nu)-\hat{\mathcal{P}}_{\infty,p}(\nu^*)\|_{\ell^2}\leq M\|\nu-\nu^*\|_{\ell^{\infty}}$.
 In particular, for $\eta=\Gamma$, let us choose $\nu^*$ as in \eqref{nustar} and $\nu$ as in \eqref{nuN}
 then 
\begin{align*}
\|\hat{\mathcal{P}}_{\infty,p}(\nu)-\hat{\mathcal{P}}_{\infty,p}(\nu^*)\|_{\ell^2}&=\|\hat{\mathcal{P}}_{\infty,p,N}-\hat{\mathcal{P}}_{\infty,p}\|_{\ell^2}\nonumber\\&\leq M\|\nu-\nu^*\|_{\ell^{\infty}}=M\gamma
\end{align*}
As $\gamma$ can be chosen arbitrary small, the result follows. 
\end{proof}
%
%
%Thus we focus now on how to approximate for a given $p$, ${\bf COP_{\infty,p}(\nu)}$ by considering 
%the truncated p-banded problems (with $\nu=0$):
%\begin{align}{\bf COP_{m,p}(\nu):} \min_{\mathcal{P}} F(mathcal{P}) \text{ subject to:}\label{P3}\\
%\Pi_m(\mathcal{L}(\mathcal{P};\mathcal{A}_{i_{b(p)}}, i\in S)+\nu\mathcal{I})\leq 0\label{P4}\\
%\Pi_m(\mathcal{P})>0\\
%{P}_{ij,k}=0,\ (i,j,k)\in S_{nul}(m,p)\label{P5}
%\end{align}
%\begin{theorem}
%$\hat{\mathcal{P}}_{m,p}(\nu)$ converges on $\ell^2$ to $\hat{\mathcal{P}}_{\infty,p}(\nu)$ and we have 
%$F(\hat{\mathcal{P}}_{m,p}(\nu))\leq F(\hat{\mathcal{P}}_{m+1,p}(\nu)(m))\leq \cdots \leq F(\hat{\mathcal{P}}_{\infty,p}(\nu)(m))$
%\end{theorem}
%\begin{proof}First at all, the minimal solution $\hat{\mathcal{P}}_{\infty,p}(\nu)(m)$ associated to $\hat{\mathcal{P}}_{\infty,p}(\nu)$ is obviously admissible for Problem ${\bf COP_{m,p}(\nu)}$, therefore $F(\hat{\mathcal{P}}_{m,p}(\nu))\leq F(\hat{\mathcal{P}}_{\infty,p}(\nu)(m))$. For the same reason, we get also:$F(\hat{\mathcal{P}}_{m,p}(\nu))\leq F(\hat{\mathcal{P}}_{m+1,p}(\nu)(m))$.
%
%If $F(\hat{\mathcal{P}}_{m,p}(\nu))= F(\hat{\mathcal{P}}_{m+1,p}(\nu)(m))$, then the unicity implies that $\hat{\mathcal{P}}_{m,p}(\nu)=\hat{\mathcal{P}}_{m+1,p}(\nu)(m)$ and since $F(\hat{\mathcal{P}}_{m+1,p}(\nu))\leq F(\hat{\mathcal{P}}_{m+1,p}(\nu)(m)$
%
%
%Since, $\hat{\mathcal{P}}_{\infty,p}(\nu)$ is bounded on $\ell^2$, its phasor sequence $P_{ij,k}=0$ $i,j=1,\cdots,n$ belongs to $\ell^2$ and thus for any $\epsilon$, there exists $m_0>0$ such that $$\|\hat{\mathcal{P}}_{\infty,p}(\nu)-\hat{\mathcal{P}}_{\infty,p}(\nu)(m_0)\|<\epsilon$$ and thus $$|F(\hat{\mathcal{P}}_{\infty,p}(\nu)-\hat{\mathcal{P}}_{\infty,p}(\nu)(m_0))|\leq M\epsilon$$
%
%If we chose $\epsilon$ such that the $k_0$-truncation of $\hat{\mathcal{P}}_{\infty,p}(\nu)$ denoted $\hat{\mathcal{P}}_{\infty,p}(\nu)|_{k_0}$ obtained by imposing for $k>k_0$ $P_{ij,k}=0$ $i,j=1,\cdots,n$ is admissible for Problem $\hat{\mathcal{P}}_{\infty,p}(0)$
% 
%
%\end{proof}
%
%\textcolor black
%\begin{theorem}We have for any $N$,
%$$F(\hat{\mathcal{P}}_{\infty,p}(0))\leq \cdots \leq F(\hat{\mathcal{P}}_{\infty,p,N+1})\leq F(\hat{\mathcal{P}}_{\infty,p,N})$$
%and $\lim_{N\rightarrow \infty} F(\hat{\mathcal{P}}_{\infty,p,N})=F(\hat{\mathcal{P}}_{\infty,p}(0))$
%If $F(\hat{\mathcal{P}}_{\infty,p,N+1})=F(\hat{\mathcal{P}}_{\infty,p,N})$ then $\hat{\mathcal{P}}_{\infty,p,N+1}=\hat{\mathcal{P}}_{\infty,p,N}$.
%
%\end{theorem}
%\begin{proof}%First of all, if there exists $N_0$ such that $F(\hat{\mathcal{P}}_{\infty,p,N_0+1})=F(\hat{\mathcal{P}}_{\infty,p,N_0})$, as $\hat{\mathcal{P}}_{\infty,p,N_0}$ is admissible for ${\bf COP_{\infty,p,N_0+1}}$ and thus optimal, by uniqueness of the solution, $\hat{\mathcal{P}}_{\infty,p,N_0+1}=\hat{\mathcal{P}}_{\infty,p,N_0}$. 
%
%As the set of admissible candidates for Problem ${\bf COP_{\infty,p,N}}$ is smaller than both Problems ${\bf COP_{\infty,p,}}$ and ${\bf COP_{\infty,p,N+1}}$, we have 
%$$F(\hat{\mathcal{P}}_{\infty,p}(0))\leq F(\hat{\mathcal{P}}_{\infty,p,N+1})\leq F(\hat{\mathcal{P}}_{\infty,p,N})$$
%
%By continuity of the optimal solution with respect the entries, as $\hat{\mathcal{P}}_{\infty,p}(0)$ belongs to $\ell^2$, there exists $N_0$ such that for any $N>N_0$
%$$\|\hat{\mathcal{P}}_{\infty,p,N}-\hat{\mathcal{P}}_{\infty,p}(0)\| <\epsilon$$ and by continuity
%$$\|F(\hat{\mathcal{P}}_{\infty,p,N}-\hat{\mathcal{P}}_{\infty,p}(0))\|<M\epsilon$$
%\end{proof}
%
%

%\begin{theorem}
%There exists $\epsilon_0$ such that for any $0<\epsilon<\epsilon_0$, there exists $N_0$ such that for $N\geq N_0$, $$F(\hat{\mathcal{P}}_{\infty,p}(0))\leq F(\hat{\mathcal{P}}_{\infty,p,N})\leq F(\hat{\mathcal{P}}_{\infty,p}(\epsilon))+M\epsilon$$
%Moreover, 
%\end{theorem}
%\begin{proof}%Let consider for a given $\epsilon$, the optimal solution $\hat{\mathcal{P}}_{\infty,p}(\epsilon)$ and 
%$\hat{\mathcal{P}}_{\infty,p}(\epsilon)_{b(N)}$ its banded version at order $N$, such that 
%$$\|\hat{\mathcal{P}}_{\infty,p}(\epsilon)_{b(N)}-\hat{\mathcal{P}}_{\infty,p}(\epsilon)\|_{\ell^2}<\epsilon$$
%and
%$$\mathcal{L}(\hat{\mathcal{P}}_{\infty,p}(\epsilon)_{b(N)};\mathcal{A}_{i_{b(p)}}, i\in S)+\frac{1}{2}\epsilon \mathcal{I}\leq 0.$$
%If $\epsilon$ is chosen sufficiently small, $\hat{\mathcal{P}}_{\infty,p}(\epsilon)_{b(N)}>0$ and is admissible for
%Problem ${\bf COP_{\infty,p,N}}$, thus it follows that 
%$$F(\hat{\mathcal{P}}_{\infty,p,N})\leq F(\hat{\mathcal{P}}_{\infty,p}(\epsilon)_{b(N)})=F(\hat{\mathcal{P}}_{\infty,p}(\epsilon))+M \epsilon$$
%In other hand as the set of admissible candidate for Problem ${\bf COP_{\infty,p,N}}$ is smaller than for Problem ${\bf COP_{\infty,p}(0)}$, we have 
%$$F(\hat{\mathcal{P}}_{\infty,p}(0))\leq F(\hat{\mathcal{P}}_{\infty,p,N})$$ and the first part of the result is established.
%Now, the assertion $\|\hat{\mathcal{P}}_{\infty,p,N}-\hat{\mathcal{P}}_{\infty,p}(0)\|_{\ell^2}\rightarrow 0$ when $N\rightarrow \infty$
%\end{proof}
In view of Theorem~\ref{prop}, an approximation of ${\bf COP_{\infty,\infty}}$ can be determined by solving ${\bf COP_{\infty,p,N}}$ for sufficiently large $p$ and $N$. 
As ${\bf COP_{\infty,p,N}}$ remains an infinite dimensional problem, given $m,p,N>0$, we consider the following finite dimensional and fully banded optimization problem: 
\begin{align*}
{\bf COP_{m,p,N}:} \min_{\mathcal{P}^*=\mathcal{P}>0} F(\mathcal{P}) \text{ subject to:} \\
\Pi_m(\mathcal{L}(\mathcal{P};\mathcal{A}_{i_{b(p)}}, i\in S))\leq 0,\quad%\nonumber \\
P_{ij,k}=0 \text{ for } |k|>N\nonumber
\end{align*}
and we denote its solution by $\hat{\mathcal{P}}_{m,p,N}$. 
\begin{theorem} For a given $p$ and a sufficiently large $m$, the solution of ${\bf COP_{m,p,N}}$ is uniquely defined.
\end{theorem}
\begin{proof}
Recall that the constraints $\Pi_m(\mathcal{L}(\mathcal{P};\mathcal{A}_{i_{b(p)}}, i\in S))\leq 0$ involves a finite number of unknowns. To guarantee the unicity of the solution, we have to set ${P}_{ij,k}=0,\text{ if }(i,j,k)\in S_{nul}(m,p)$ where $S_{nul}(m,p)$ is defined by \eqref{Snul}. Fortunately, $P_{ij,k}=0 \text{ for } |k|>N$ guarantees this condition for sufficiently large $m$, since $k_0=\arg \min_{k,k>0}\{(i,j,k)\in S_{nul}(m,p)\} \rightarrow +\infty$ when $m \rightarrow +\infty$.
\end{proof}



\begin{theorem} Assuming that any unbounded sequence of admissible candidates for Problem ${\bf COP_{m,p,N}}$ has an unbounded objective function $F$, then: 
 $$\lim_{m \rightarrow +\infty} \|\hat{\mathcal{P}}_{m,p,N}-\hat{\mathcal{P}}_{\infty,p,N}\|_{\ell^2}=0$$
\end{theorem}
\begin{proof}For any $m$, as $\hat{\mathcal{P}}_{\infty,p,N}$ is admissible for Problem ${\bf COP_{m,p,N}} $ and $\hat{\mathcal{P}}_{m+1,p,N}$ is admissible for Problem ${\bf COP_{m,p,N}}$, it follows that:
$$F(\hat{\mathcal{P}}_{m,p,N})\leq F(\hat{\mathcal{P}}_{m+1,p,N})\leq \cdots \leq F(\hat{\mathcal{P}}_{\infty,p,N})$$
which proves that the sequence $F(\hat{\mathcal{P}}_{m,p,N})$ indexed by $m$ is an increasing and bounded sequence, and thus a converging sequence. Moreover, for any $m_0$, as the sequence $\hat{\mathcal{P}}_{m,p,N}$, $m\geq m_0$ is admissible for Problem ${\bf COP_{m_0,p,N}}$, following the assumption, this sequence is bounded on $\ell^2$.
Therefore, for any $m\geq m_0$, the phasors of $\hat{\mathcal{P}}_{m,p,N}$ are bounded and belong to a finite dimensional subspace of $\ell^2$ (thanks to the constraints $P_{ij,k}=0 \text{ for } |k|>N$). 
By compactness, there exists a subsequence that converges on this subspace of $\ell^2$ and the uniqueness of the solution implies that the whole sequence converges necessarily to $\hat{\mathcal{P}}_{\infty,p,N}$.
\end{proof}

%\begin{theorem}
%For any $m$, $\hat{\mathcal{P}}_{\infty,p}$ has a minimal solution denotes $\hat{\mathcal{P}}_{\infty,p}(m) $ that solves ${\bf COP_{m,p}}$ and we have
%$$tr(\mathcal{C}\hat{\mathcal{P}}_{m,p})<tr(\mathcal{C}\hat{\mathcal{P}}_{\infty,p}(m))$$
%\end{theorem}
%\begin{proof}For any $m$,
%$\hat{\mathcal{P}}_{\infty,p}$ satisfies the \eqref{P4} and its associated minimal solution \eqref{P4} and \eqref{P5}.Thus as the minimal solution is admissible for ${\bf COP_{m,p}} $ the result follows.
%\end{proof}
%
%As $lim_{m\rightarrow \infty} tr(\mathcal{C}\hat{\mathcal{P}}_{\infty,p}(m))=tr(\mathcal{C}\hat{\mathcal{P}}_{\infty,p}$,
%and it follows that the sequence $\mathcal{C}\hat{\mathcal{P}}_{\infty,p}(m)$ is bounded on $\ell^2$. 
%
%%As $F(\mathcal{P})=tr(\mathcal{C}{\mathcal{P}}) is a bounded operator on $L^2$has finite rank
%
%
%
%Let denote by $\ell=\liminf_{m\rightarrow \infty}tr(\mathcal{C}\hat{\mathcal{P}}_{m,p})$.
%\begin{theorem}$\ell= tr(\mathcal{C}\hat{\mathcal{P}}_{\infty,p})$ and 
%$\hat{\mathcal{P}}_{m,p}$ converge in $\ell^2$ to $\hat{\mathcal{P}}_{\infty,p}$.
%\end{theorem}
%\begin{proof}
%
%As 
%Let us denote by $m_k$ the subsequence such that $\lim_{m_k\rightarrow \infty}tr(\mathcal{C}\hat{\mathcal{P}}_{m_k,p})$.
%and let assume that $\hat{\mathcal{P}}_{m_k,p}$ is not a Cauchy sequence which means there exists $\epsilon>0$, for all $m_{k_0}$>0, there exist $m_{k_1}>m_{k_0}$ and $m_{k_2}>m_{k_0}$ such that $\|\hat{\mathcal{P}}_{m_{k_1},p}-\hat{\mathcal{P}}_{m_{k_2},p}\|_{\ell^2}>\epsilon$.
%
%In other hand, we know that for any $\eta>0$, there exist $\nu$ for any $k>\nu$ $$|tr(\mathcal{C}\hat{\mathcal{P}}_{m_{k},p})-\ell |<\eta$$
%Thus , for any $\eta <\epsilon$, il follows that there exists $\nu$ and $m_{k_3}>\nu$ and $m_{k_4}>\nu$
%such that 
%$$|tr(\mathcal{C}\hat{\mathcal{P}}_{m_{k_3},p})-tr(\mathcal{C}\hat{\mathcal{P}}_{m_{k_4},p}) |<2\eta$$ and 
%$$\|\hat{\mathcal{P}}_{m_{k_3},p}-\hat{\mathcal{P}}_{m_{k_4},p}\|_{\ell^2} >\epsilon$$
%\end{proof}
%
%
%
%
%\begin{theorem}There exists $p_0$ such that for $p>p_0$
%$$F({\bf\hat{P}_{\infty,p}})\leq F({\bf\hat{P}_{\infty,p+1}})\leq \cdots \leq F(\bf{\hat{P}_{\infty,\infty}})$$
%and $\lim_{p\rightarrow \infty}F({\bf\hat{P}_{\infty,p}})=F({\bf\hat{P}_{\infty,\infty}})$ and by unicity of the solution
%$\lim_{p\rightarrow \infty}\bf{\hat{P}_{\infty,p}}=\bf{\hat{P}_{\infty,\infty}}$.
%\end{theorem}
%\begin{proof}Consider $\hat{\mathcal{P}}_{\infty,\infty}$ the Toeplitz bloc operator associated to the optimal solution $\bf{\hat{P}_{\infty,\infty}}$, Theorem \ref{t1} implies that there exist $p_0$ such that for $p\geq p_0$
%$\mathcal{L}(\hat{\mathcal{P}}_{\infty,\infty};\mathcal{A}_{i_{b(p)}}, i\in S)<0$ which means that $\bf{\hat{P}_{\infty,\infty}}$ is admissible for problem ${\bf COP_{\infty,p}}$. Therefore, it follows that $F({\bf\hat{P}_{\infty,p}})\leq \leq F(\bf{\hat{P}_{\infty,\infty}})$ for any $p\geq p_0$. Using similar arguments than in the proof of Theorem \ref{t1}, there exist $p_1$, such that $\hat{\mathcal{P}}_{\infty,p+1}$
%is also admissible for ${\bf COP_{\infty,p}}$ and we
%\end{proof}
%
%
%
%
%
%and finally the following optimization problems:
%
%\begin{align*}
%{\bf COP_{m,m}:} &\min_{\tilde{\bf{P}}} F(\tilde{\bf{P}}) \text{ subject to \eqref{lmi_trunc3} with }p=m 
%\end{align*}
%where $\tilde{\bf{P}}=(\tilde P_k)_{k\in\mathbb{Z}}$.% refers to the phasor sequence associated to the symbol $\tilde P_m(z)$ (see Definition \ref{symsol}).
%
%\begin{theorem}Let denote by $\hat {\mathcal{P}}$ the optimal solution of ${\bf COP_\infty}$. For any $\epsilon>0$, there exist $m$ such that the solution $\tilde {\mathcal{P}}_m$ of $OP_m$
%\end{theorem}
%\begin{proof}Let denote by $\hat {\bf{P}}$ the optimal solution of ${\bf COP_\infty}$. Theorem~\ref{proche2} implies that for any $\epsilon>0$, there exists a $m_0$ such that for $m>m_0$, ${\bf COP_m}$ has admissible candidates ${\bf\tilde {\bf{ P}}}_m=\hat {\bf{P}}_s(m)$ such that 
%$$\|{\bf \hat P}-{\bf\tilde {P}}_m\|_{\ell^2}\leq \epsilon$$
%Since $F$ is a linear operator, it follows that there exist $M>0$ such that $$|F({\bf{\hat P}}- {\bf\tilde {P}}_m)|\leq M\|{\bf \hat P}-{\bf\tilde {P}}_m\|_{\ell^2}.$$
%Thus, it can be concluded that $\lim_{m\rightarrow +\infty} F({\bf\tilde {P}}_m)=F({\bf \hat P})$.
%
%On the other hand, for any $m$, the optimal solution $\hat {\bf{P}}_m$ of ${\bf COP_m}$ satisfies
%$F(\hat {\bf{P}}_m)\leq F(\tilde {\bf{P}}_m)$. 
%
%%Moreover we have necessarily $F(\hat {\bf{P}}_m)\leq F(\hat {\bf{P}}_{m+1})$ since increasing $m$ increases the constraint.
%
%
%%As the sequence $F(\hat {\bf{P}}_m)$ is bounded, 
%%$\liminf F(\hat {\bf{P}}_m)$
%%
%%there always exists a subsequence that convergence to $F(\hat {\bf{P}}_m)$
%Taking the following limits, we have:
%$$\liminf_{m\rightarrow +\infty}F(\hat {\bf{P}}_m)\leq \lim_{m\rightarrow +\infty} F(\tilde {\bf{P}}_m)=F({\bf \hat P})$$
%It implies there exists a convergent subsequence $\hat {\bf{P}}_{m_k}$ such that $\lim_{{m_k}\rightarrow +\infty}F(\hat {\bf{P}}_{m_k})=\ell \leq F({\bf \hat P})$ and we have necessarily $\ell=F({\bf \hat P})$ otherwise it contradict the optimality of ${\bf \hat P}$. 
%\end{proof}
%


%As F is continuous, any sequence \mathcal{S}_p \rightarrow F(\mathcal{P})


%Assume also that there exists truncated version of this problem
%$$\min F(\tilde{\mathcal{P}}_m)$$ subject to \eqref{lmi_trunc2} and \tilde{\mathcal{P}}_m
%such that the truncated objective $\tilde F_m$ satisfies
%for any sequence $\mathcal{S}_m$
%$\tilde F_m(\mathcal{S}_m)\rightarrow F(\mathcal{S})\rightarrow$
%
%In this case and assuming the solution is unique, we 
%
% with a linear optimization problem




\section{Illustrative example}
We consider the example given in \cite{Pierre2022} defined by:
\begin{align*}
	\dot x=&\left(\begin{array}{cc}a_{11} (t) & a_{12} (t) \\a_{21} (t) & a_{22} (t)\end{array}\right)x+\left(\begin{array}{c}b_{11}(t) \\0\end{array}\right)u\label{ex_ltp}\end{align*}
{\small\begin{align*}a_{11} (t) &=1+\frac{4}{\pi}\sum_{k=0}^{\infty}\frac{1}{2k+1}\sin(\omega (2k+1)t),\\
	a_{12} (t) &= 2+\frac{16}{\pi^2}\sum_{k=0}^{\infty}\frac{1}{(2k+1)^2}\cos(\omega (2k+1)t),\\
	a_{21} (t) &= -1+\frac{2}{\pi}\sum_{k=1}^{\infty}\frac{(-1)^k}{k}\sin(\omega kt+\frac{\pi}{4}),\\
	a_{22} (t) &= 1-2\sin(\omega t)-2\sin(3\omega t)+2\cos(3\omega t)+2\cos(5\omega t),\\
	b_{11}(t)&=1+ 2 \cos(2\omega t)+ 4 \sin(3\omega t) \text{ with }\omega=2\pi.
\end{align*}}
The associated Toeplitz matrix $\mathcal{A}$ has an infinite number of phasors and is not banded. The equivalent harmonic LTI system \eqref{ltih} is unstable and has a spectrum provided by the set $\sigma=\{\lambda+ \textsf{j}\omega k, k\in \mathbb{Z}\}$ where $\lambda \in \{1\pm \textsf{j} 1.64\}$ (see \cite{Pierre2022}).

We consider the LQ problem and we solve the optimization problem given by \eqref{op} with $\mathcal{Q}=10^2\mathcal{T}(Id_n)$ and $\mathcal{R}=\mathcal{T}(Id_m)$. Imposing a Toeplitz block structure to $\mathcal{P}$, Problem ${\bf COP_{m,p,N}}$ associated to \eqref{op} is solved with $m=10,15,20$ and $p=N=2m$. This is illustrated in Figure~\ref{f1} where we plot the modulus of phasors of $\mathcal{K}=[\mathcal{K}_1,\mathcal{K}_2]$. As a result, we recover the same state feedback gain as in \cite{Pierre2022}. 
\begin{figure}[h]
	\begin{center}
		\includegraphics[width=\linewidth]{PhasorK_ACC}
		\caption{Modulus of Phasors $K=[K_1,K_2]$ (harmonic LQ control)}\label{f1}
	\end{center}
\end{figure}
\begin{figure}[h]
	\begin{center}
		\includegraphics[width=\linewidth,height=6cm]{Traj_LQ}
		\caption{Closed loop response with LQ control}\label{f4}
	\end{center}
\end{figure}
%\addtolength{\textheight}{0cm} % This command serves to balance the column lengths
   % on the last page of the document manually. It shortens
   % the textheight of the last page by a suitable amount.
   % This command does not take effect until the next page
   % so it should come on the page before the last. Make
   % sure that you do not shorten the textheight too much.
%%%%%%%%%%%%%%%%%%%%%%%%%%%%%%%%%%%%%%%%%%%%%%%%%%%%%%%%%%%%%%%%%%%%%%%%%%%%%%%%
\section{Conclusion}
In this paper, we provided a novel approach that allows to solve, up to an arbitrarily small error, infinite dimensional Toeplitz block LMIs and related convex optimization problems encountered in the analysis and control of dynamical systems in the harmonic framework. The result is based on a well-defined finite dimensional truncated problem that allows to recover the original infinite-dimensional solution up to an arbitrarily small error. This framework is not only usefull for robustness and multiobjective optimization issues of LTP systems but also for the analysis and control of more general periodic systems such as periodic polynomial systems. 
\begin{thebibliography}{99}

\bibitem{Almer2} Alm\`er, S., Mari\'ethoz, S., and Morari, M., "Dynamic Phasor Model Predictive Control of Switched Mode Power Converters", \emph{IEEE Transaction on 
Control System Technology}, Vol. 23, No. 1, January 2015.
\bibitem{Blin}N. Blin, P. Riedinger, J. Daafouz, L. Grimaud and P. Feyel, "Necessary and Sufficient Conditions for Harmonic Control in Continuous Time," in IEEE Trans. on Aut. Control, vol. 67, no. 8, 2022.
	
\bibitem{Bolzern}Bolzern, P. and Colaneri, P. (1988). "The periodic Lyapunov equation". \emph{SIAM Journal on Matrix Analysis and Applications}, 9(4), 499-512.
\bibitem{Boyd}Boyd, S., El Ghaoui, L., Feron, E., and Balakrishnan, V. "Linear Matrix Inequalities in System and Control Theory", Studies in Applied math. SIAM, 1994. 
	\bibitem{Farkas} Farkas, M.: "Periodic motions" (Springer-Verlag, New York, 1994)

		\bibitem{Gohberg} 
	Gohberg, I., Goldberg, S. and Kaashoek, M.A., \emph{Classes of Linear Operators}, Operator Theory
	Advances and Applications Vol. 63 Birkhauser, Vol. II, 1993.

	\bibitem{Ikeda01}Ikeda, K., Azuma, T., and Uchida, K., "Infinite-dimensional LMI approach to analysis and synthesis for linear time-delay systems", \emph{Kybernetika} 37(4):505-520, 2001.
\bibitem{Pierre2022} P. Riedinger and J. Daafouz, "Solving Infinite-Dimensional Harmonic Lyapunov and Riccati Equations," To appear in IEEE Trans. on Aut. Control, doi: 10.1109/TAC.2022.3229943.
		\bibitem{Sanders}
Sanders, S. R., Noworolski, J. M., Liu, X. Z. and Verghese, G. C., "Generalized averaging method for power conversion circuits". \emph{IEEE Transactions on Power Electronics}, 6(2), p.p. 251-259, 1991.
	\bibitem{Wereley_1990}Wereley, N. M., "Analysis and control of linear periodically time-varying systems", \emph{Doctoral dissertation, MIT}, 1990.	
\bibitem{Wil71}	J. C. Willems. Least squares stationary optimal control and the algebraic Riccati equation. IEEE Trans. on Aut. Control, 16(6):621-634, 1971.
	\bibitem{Zhou2008}Zhou, J., "Derivation and Solution of Harmonic Riccati Equations via Contraction Mapping Theorem", \emph{Transactions of the Society of Instrument and Control Engineers} 44(2), p.p. 156-163, 2008.

\end{thebibliography}
\end{document}
