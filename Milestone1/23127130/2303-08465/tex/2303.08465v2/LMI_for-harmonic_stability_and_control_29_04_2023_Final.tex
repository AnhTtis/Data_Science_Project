\documentclass[journal,twoside,web]{ieeecolor}
\usepackage[dvipsnames]{xcolor}

\usepackage{generic}
\usepackage{cite}
\usepackage{amsmath,amssymb,amsfonts}
\usepackage{algorithmic}
\usepackage{graphicx}
\usepackage[OT2,T1]{fontenc}
\usepackage{MnSymbol}
\usepackage{lipsum}
%\UseRawInputEncoding
\usepackage{color}
\newtheorem{theorem}{Theorem}
\newtheorem{assumption}{Assumption}
\newtheorem{definition}{Definition}
\newtheorem{property}{Property}
\newtheorem{proposition}{Proposition}
\newtheorem{problem}{Problem}
\newtheorem{lemma}{Lemma}
\newtheorem{corollary}[lemma]{Corollary}
\newtheorem{remark}{Remark}
\DeclareSymbolFont{cyrletters}{OT2}{wncyr}{m}{n}
\DeclareMathSymbol{\Sha}{\mathalpha}{cyrletters}{"58}

\usepackage{textcomp}
\def\BibTeX{{\rm B\kern-.05em{\sc i\kern-.025em b}\kern-.08em
 T\kern-.1667em\lower.7ex\hbox{E}\kern-.125emX}}
\markboth{\journalname, VOL. XX, NO. XX, XXXX 2021}
{Author \MakeLowercase{\textit{et al.}}: Preparation of Papers for IEEE TRANSACTIONS and JOURNALS (February 2017)}
\begin{document}


\title{Solving infinite-dimensional Toeplitz Block LMIs}
\author{Flora Vernerey, Pierre Riedinger, Jamal Daafouz\\% <-this % stops a space
\thanks{This work is supported by HANDY project ANR-18-CE40-0010-02}% <-this % stops a space
\thanks{The authors are with Universit\'e de Lorraine, CNRS, CRAN, F-54000 Nancy, France.}}
\maketitle
%%%%%%%%%%%%%%%%%%%%%%%%%%%%%%%%%%%%%%%%%%%%%%%%%%%%%%%%%%%%%%%%%%%%%%%%%%%%%%%%
\begin{abstract}
This paper focuses on the resolution of infinite-dimensional Toeplitz Block LMIs, which are frequently encountered in the context of stability analysis and control design problems formulated in the harmonic framework. We propose \textcolor{red}{a consistent truncation method that makes this infinite dimensional problem tractable} and demonstrate that a solution to the truncated problem can always be found at any order, provided that the original infinite-dimensional Toeplitz Block LMI problem is feasible. Using this approach, we illustrate how the infinite dimensional solution to a Toeplitz Block LMI based convex optimization problem can be recovered up to \textcolor{red}{an arbitrarily} small error, by solving a finite dimensional truncated problem. The obtained results are applied to stability analysis and harmonic LQR for linear time periodic (LTP) systems. 
\end{abstract}
%\begin{IEEEkeywords}
%, Dynamic phasors, Harmonic modeling and control
%\end{IEEEkeywords}

%%%%%%%%%%%%%%%%%%%%%%%%%%%%%%%%%%%%%%%%%%%%%%%%%%%%%%%%%%%%%%%%%%%%%%%%%%%%%%%%
\section{Introduction}
LMIs are a powerful and versatile tool that can be used to solve a broad range of problems in science and engineering, including control theory, optimization, signal processing, and robotics. One specific type of LMIs is the Infinite-dimensional Toeplitz Block LMIs (TBLMI), which involve matrices of infinite dimension with a Toeplitz block structure. TBLMIs are encountered in the context of harmonic analysis and control, a topic of great theoretical and practical interest in numerous application domains, including energy management and embedded systems to mention few \cite{Farkas,Bolzern,Sanders,Wereley_1990,Zhou,Zhou2008,Almer2}. 

Solving TBLMIs poses a significant challenge due to the infinite dimensionality. The issue we tackle in this paper is different from the problem previously examined in \cite{Ikeda01}, which aimed to reduce an infinite number of LMIs to a finite number of LMIs. In our case, the number of inequalties is finite but the entries and the unknowns are infinite-dimensional. To illustrate the challenges involved, recall the following fact (see \cite{Pierre2022} for more detail): "a truncated matrix of a Hurwitz infinite-dimensional harmonic matrix may not be Hurwitz at any truncation order". As a result, it is possible that solving the truncated version of an infinite-dimensional harmonic Lyapunov equation may not yield a positive definite solution.

In \cite{Pierre2022}, efficient algorithms and methods that leverage the Toeplitz structure have been proposed to determine the infinite-dimensional solution to harmonic Lyapunov or Riccati equations with arbitrarily small error. In this paper, we aim to expand upon this approach and extend it to the TBLMI framework. \textcolor{red}{To the authors knowledge, it is the first time that this problem is raised.} Our objective is to define a truncated version of the original problem which enables the recovery of the infinite-dimensional solution with arbitrary accuracy. \textcolor{red}{Contrarily to the literature on the subject \cite{Wereley_1990, Zhou,Zhou2008}, these new results do not invoke Floquet theory. The later is of interest interest for stability analysis of LTP systems but it is seriously limited for control design purposes \cite{Pierre2022}}. 

% LMIs provide a powerful and versatile tool for solving various problems in science and engineering including control theory, optimization, signal processing, and robotics. Infinite-dimensional Toeplitz Block LMIs (TBLMIs) are LMIs where the entries and unknowns matrices are of infinte dimension and have a block Toeplitz strcture. These type of LMIs are encoutered in the context of harmonic control. Harmonic modelling and control is a topic of theoretical and practical interest in many application domains such as energy management or embedded systems to mention few \cite{Farkas,Bolzern,Sanders,Montagnier,Wereley_1990,Demiray,Almer,Zhou2011,Zhou2008,Mattavelli,Chavez,Almer2}. In this context, solving stability analysis and control design problems using LMIs is an open problem. 
%% The obtained LMIs are infinte dimensional and have a block Toeplitz strcutre. 
%% Harmonic modeling and control is a topic of theoretical and practical interest in many application domains such as energy management or embedded systems to mention few \cite{Farkas,Bolzern,Sanders,Montagnier,Wereley_1990,Demiray,Almer,Zhou2011,Zhou2008,Mattavelli,Chavez,Almer2}. 
%%In a recent paper \cite{Blin}, a unified and coherent mathematical framework for harmonic modelling and control has been proposed. Basically, the harmonic modeling of a periodic system leads to an equivalent time invariant model of infinite dimension whose states (also called phasors) are the coefficients obtained by applying a sliding Fourier decomposition. One of the main results of \cite{Blin} established a strict equivalence between these two models. In this framework, the analysis and design are considerably simplified since all the methods established for time-invariant systems can be a priori applied. \\
%%
%%Recently, results related to spectral properties of the harmonic state operator along with an explicit Floquet factorization and practical solutions of harmonic Lyapunov and Riccati equations have been established in \cite{Pierre2022}. In addition, these results have been exploited in \cite{Ried2022} to address the problem of designing harmonic pole placement based control laws for Linear Time Periodic (LTP) systems. Here, we focus on stabilitity analysis and control synthesis using convex optimisation based tools. In particular, we are interested in stability analysis and harmonic control design using LMI based conditions. This interest is motivated by the fact that LMI based methods can be applied to solve robust control problems effectively and handle multi-objective control problems. 
%The main difficulty in using LMIs in the harmonic modeling framework is the fact that the obtained inequalities are infinite dimensional. The problem we formulate in this paper is different from the problem that consists in reducing infinite-dimensional inequalities to a finite number of LMIs \cite{Ikeda01}. To illustrate and explain the encountered difficulties, recall the following fact (see \cite{Pierre2022} for more detail): "the truncated matrix of a Hurwitz infinite dimensional harmonic matrix may not be Hurwitz at any truncation order". As a consequence, if one attempts to solve the truncated version of the infinite dimensional harmonic Lyapunov equation 
%there may be no chance to obtain a positive definite solution.
%In \cite{Pierre2022}, by exploiting the Toeplitz structure of these equations, efficient methods and algorithms have been proposed to determine up to an arbitrarily small error the infinite dimensional solution to harmonic Lyapunov or Riccati equations. Here, our objective is to extend this philosophy to the TBLMI framework and show how to define a truncated version problem to recover the infinite-dimensional solution up to an arbitrary small error. 

%We propose a method that consists in well defining a truncated TBLMI problem that %allows to recover any original infinite-dimensional solution up to an arbitrarily small error. More precisely, we first define a truncated version of the original infinite-dimensional LMI that 
%allows to preserve the negativity or positivity of the obtained truncated solution and then show that the truncated version of the original solution solves this truncated LMI problem. We expoit this study to show how the infinite dimensional solution to a TBLMI based convex optimisation problem can be determined up to an arbitrarily small error by solving a truncated version.
The paper is organized as follows. The next section is dedicated to mathematical prelimaries. In Section III, we define what we call a Toeplitz Block LMI and we give the problem formulation. The main results are established in Section IV where we investigate the truncation of infinite-dimensional TBLMIs that preserves solution positiveness. We show how to recover the infinite-dimensional solution to a TBLMI-based convex optimization problem, to an arbitrarily small error, by solving a finite dimensional truncated problem. We illustrate the results of this paper in section V and apply the proposed procedure to design a harmonic LQR for linear time periodic systems.

{\bf Notations: } The transpose of a matrix $A$ is denoted $A'$ and $A^*$ denotes the complex conjugate transpose $A^*=\bar A'$. The $n$-dimensional identity matrix is denoted $Id_n$. The infinite identity matrix is denoted $\mathcal{I}$. For $m\in\mathbb{Z}^+\cup \{\infty\}$, the flip matrix $J_m$ is the $(2m+1) \times (2m+1)$ matrix having 1 on the anti-diagonal and zeros elsewhere. 
$C^a$ denotes the space of absolutely continuous function,
$L^{p}([a\ b],\mathbb{C}^n)$ (resp. $\ell^p(\mathbb{C}^n)$) denotes the Lebesgues spaces of $p-$integrable functions on $[a, b]$ with values in $\mathbb{C}^n$ (resp. $p-$summable sequences of $\mathbb{C}^n$) for $1\leq p\leq\infty$. $L_{loc}^{p}$ is the set of locally $p-$integrable functions. The notation $f(t)=g(t)\ a.e.$ means almost everywhere in $t$ or for almost every $t$. 
To simplify the notations, $L^p([a,b])$ or $L^p$ will be often used instead of $L^p([a,b],\mathbb{C}^n)$. 
%For example, $x\in L^2([a,b])$ means $x \in L^2([a,b],\mathbb{C}^n)$. %We denote by $col(X)$ the vectorization of a matrix $X$, formed by stacking the columns of $X$ into a single column vector. Finally, $<\cdot,\cdot>$ refers to the scalar product in $\ell^2$.
%%%%%%%%%%%%%%%%%%%%%%%%%%%%%%%%%%%%%%%%%%%%%%%%%%%%%%%%%%%%%%%%%%%%%%%%%%%%%%%%
\vspace{-.3cm}
\section{Preliminaries}
%Metric units are preferred for use in IEEE publications in light of their
%international readership and the inherent convenience of these units in many fields.
%In particular, the use of the International System of Units (SI Units) is advocated.
% This system includes a subsystem the MKSA units, which are based on the
% meter, kilogram, second, and ampere. British units may be used as secondary units
% (in parenthesis). An exception is when British units are used as identifiers in trade,
% such as, 3.5 inch disk drive.
%We start by recalling some preliminaries related to Toeplitz block matrices, sliding Fourier decomposition in the context of harmonic modeling and the trace operator.
\subsection{Infinite dimensional Toeplitz block (\textcolor{red}{TB}) matrices}
 \color{red}
The Toeplitz transformation of a $T-$periodic function $a\in L^{2}([0 \ T], \mathbb{R})$, denoted $\mathcal{T}(a)$, defines a constant Toeplitz and infinite dimensional matrix as follows: 
\begin{align*}
	\mathcal{T}(a)=
	\left[
	\begin{array}{ccccc}
		\ddots & & \vdots & &\udots \\ & a_{0} & a_{-1} & a_{-2} & \\
		\cdots & a_{1} & a_{0} & a_{-1} & \cdots \\
		& a_{2} & a_{1} & a_{0} & \\
		\udots & & \vdots & & \ddots\end{array}\right],\end{align*}
where $(a_{k})_{k\in\mathbb{Z}}$ is the Fourier coefficient sequence of $a$.\\
From the subsequence $a^+=(a_k)_{k>0}$ and $a^-=(a_k)_{k<0}$ of $(a_k)_{k\in\mathbb{Z}}$, we also define the semi-infinite Hankel matrices:
\begin{align*}
	\mathcal{H}(a^+) &= (a_{i+j-1})_{{i,j}>0},\quad \mathcal{H}(a^-) = (a_{-i-j+1})_{{i,j}>0}.\end{align*}
\textcolor{blue}{Given an integer $m >0$, we define the $m-$truncation of $\mathcal{T}(a)$, denoted by $\mathcal{T}_m(a)$, the $(2m+1) \times (2m+1)$ principal submatrix of $\mathcal{T}(a)$.} We denote  by $\mathcal{H}_{(p,q)}(a^+)$(resp. $\mathcal{H}_{(p,q)}(a^-)$) for any $p,q>0$, the $(2p+1)\times(2q+1)$ Hankel matrix obtained by selecting the first $(2p+1)$ rows and $(2q+1)$ columns of $\mathcal{H}(a^+)$(resp. $\mathcal{H}(a^-)$).
For clarity purpose, we provide in Fig.~\ref{fig20} a block decomposition of an infinite Toeplitz matrix $\mathcal{T}(a)$ to illustrate how the matrices defined above appear. Notice that the flip matrix $J_m$ is defined in the notation part.  
\begin{figure}[h]\begin{center}
		\includegraphics[width=\linewidth]{Mat_Top}
		\caption{Block decomposition of an infinite Toeplitz matrix $\mathcal{T}(a)$ for any $m>0$.} \label{fig20}
	\end{center}
\end{figure}
The Toeplitz block transformation of a $T-$periodic $n\times n$ matrix function $A=(a_{ij})_{i,j=1,\cdots,n}\in L^{2}([0 \ T], \mathbb{R}^{n\times n})$, denoted $\mathcal{A}=\mathcal{T}(A)$, defines a constant $n\times n$ Toeplitz Block (\textcolor{red}{TB}) and infinite dimensional matrix: 
\begin{equation}
	\mathcal{A}=\left(\begin{array}{ccc}
		\mathcal{A}_{11} & \cdots & \mathcal{A}_{1n} \\
		\vdots & \ddots& \vdots \\
		\mathcal{A}_{n1} & \cdots & \mathcal{A}_{nn}\end{array}\right)\label{btop}\end{equation} where $\mathcal{A}_{ij}=\mathcal{T}(a_{ij})$, $i,j=1,\cdots,n$.
		
The $m-$truncation of the $n \times n$ Toeplitz block matrix $\mathcal{A}$ denoted by $\mathcal{T}_m(A)$, is defined by the $m-$truncation $\mathcal{T}_m(a_{ij})$ for all its entries $(i,j)$.
Similarly, the $n\times n$ Hankel block matrices $\mathcal{H}(A^+)$, $\mathcal{H}(A^-)$ are also defined respectively by $\mathcal{H}(A^+)_{ij}=\mathcal{H}(a^+_{ij})$ and $\mathcal{H}(A^-)_{ij}=\mathcal{H}(a^-_{ij})$ for $i,j=1,\cdots,n$. {Their principal submatrices $\mathcal{H}(A^+)_{(p,q)}$, $\mathcal{H}(A^-)_{(p,q)}$ for $p,q >0$ are obtained by considering the principal submatrices of the entries $\mathcal{H}(a^+_{ij})_{(p,q)}$ and $\mathcal{H}(a^-_{ij})_{(p,q)}$ for $i,j=1,\cdots,n$.}\\
\textcolor{red}{Recall that the product of  two \textcolor{red}{TB} matrices is a \textcolor{red}{TB} matrix only in infinite dimension. In finite dimension, we have the following result \cite{Pierre2022}}.
\begin{theorem} \label{product}Let $\mathcal{A}$, $\mathcal{B}$ be two $n \times n$ \textcolor{red}{TB} matrices and $\mathcal{C}=\mathcal{A}\mathcal{B}$. Then,
	\begin{align}
		\mathcal{T}_m(A)\mathcal{T}_m(B) &= \mathcal{T}_m(C)- \mathcal{H}_{(m,\eta)}(A^+)\mathcal{H}_{(\eta,m)}(B^-)\nonumber \\&- \mathcal{J}_{n,m}\mathcal{H}_{(m,\eta)}(A^-)\mathcal{H}_{(\eta,m)}(B^+) \mathcal{J}_{n,m}, \label{ee2}\end{align}
	where $\mathcal{J}_{n,m}=Id_n\otimes J_m$ and $\eta\in \mathbb{Z}^+\cup \{+\infty\}$ is such that $2\eta\geq \min(d^oA,d^oB)$ \textcolor{red}{with $d^o A$ the largest harmonic (non vanishing Fourier coefficent) of $A$}.
\end{theorem}
\begin{figure}\begin{center}
		\includegraphics[scale=0.2]{Toeplitz_product}
		\caption{Multiplication of two finite dimensional banded Toeplitz matrices}\label{fig1}
	\end{center}
\end{figure}
An illustration of the above theorem is given in Fig.~\ref{fig1} for $n=1$ when $d^oa$ and
$d^ob$ are less than $m$ so that $\mathcal{T}_m(a)$ and $\mathcal{T}_m(b)$
are banded.
In this case, the matrices $E^+=\mathcal{H}_{(m,m)}(a^+)\mathcal{H}_{(m,m)}(b^-)$ and $E^-= J_m\mathcal{H}_{(m,m)}(a^-)\mathcal{H}_{(m,m)}(b^+) J_m$ have disjoint supports located in the upper leftmost corner and in the lower rightmost corner, respectively. As a consequence,
$\mathcal{T}_m(a)\mathcal{T}_m(b)$ can be represented as the sum of $\mathcal{T}_m(c)$ and two correcting terms $E^+$ and $E^-$.
 \color{black}
\subsection{Sliding Fourier decomposition and harmonic modeling}
Consider $x\in L^{2}_{loc}(\mathbb{R},\mathbb{C})$ a complex valued function of time. Its sliding Fourier decomposition over a window of length $T$ is defined by the time-varying infinite sequence $X=\mathcal{F}(x)\in C^a(\mathbb{R},\ell^2(\mathbb{C}))$ (see \cite{Blin}) whose components satisfy:
$$X_{k}(t)=\frac{1}{T}\int_{t-T}^t x(\tau)e^{-\textsf{j}\omega k \tau}d\tau$$ for $k\in \mathbb{Z}$, with $\omega=\frac{2\pi}{T}$.
If $x=(x_1,\cdots,x_n)\in L^{2}_{loc}(\mathbb{R},\mathbb{C}^n)$ is a complex valued vector function, then
$$X=\mathcal{F}(x)=(\mathcal{F}(x_1), \cdots,\mathcal{F}(x_n)).$$
The vector $X_k=(X_{1,k}, \cdots, X_{n,k})$ with $$X_{i,k}(t)=\frac{1}{T}\int_{t-T}^t x_i(\tau)e^{-\textsf{j}\omega k \tau}d\tau$$
is called the $k-$th phasor of $X$. 
\begin{definition}\label{H} We say that $X$ belongs to $H$ if $X$ is an absolutely continuous function (i.e $X\in C^a(\mathbb{R},\ell^2(\mathbb{C}^n))$ and fulfills for any $k$ the following condition: \begin{equation*}\dot X_k(t)=\dot X_0(t)e^{- \textsf{j}\omega k t} \ a.e.\end{equation*}
\end{definition}
Similarly to the Riesz-Fisher theorem which establishes a one-to-one correspondence between the spaces $L^2$ and $\ell^2$, the following theorem establishes a one-to-one correspondence between the spaces $L_{loc}^2$ and $H$; (see \cite{Blin}).
\begin{theorem}\label{coincidence}For a given $X\in L_{loc}^{\infty}(\mathbb{R},\ell^2(\mathbb{C}^n))$, there exists a representative $x\in L^2_{loc}(\mathbb{R},\mathbb{C}^n)$ of $X$, i.e. $X=\mathcal{F}(x)$, if and only if $X \in H$.
\end{theorem}
\textcolor{red}{Thanks to Theorem~\ref{coincidence}, it is established in \cite{Blin} that any system having solutions in Carath\'eodory sense can be transformed by a sliding Fourier decomposition into an infinite dimensional system for which a one-to-one correspondence between their respective trajectories is established providing that the trajectories in the infinite dimensional space belong to the subspace $H$. Moreover, when $T-$periodic systems are considered, the resulting infinite dimensional systems is time invariant.}\\ For instance, consider $T-$periodic functions $A(\cdot)$ and $B(\cdot)$ respectively of class $L^2([0\ T],\mathbb{C}^{n\times n})$ and $L^{\infty}([0\ T],\mathbb{C}^{n\times m})$ and let: 
\begin{align}\dot x(t)=A(t)x(t)+B(t)u(t)\quad x(0)=x_0\label{ltp}\end{align}
If, $x$ is a solution associated to the control $u\in L_{loc}^2(\mathbb{R},{\mathbb{C}^m)}$ of the linear time periodic (LTP) system ~(\ref{ltp}) then, $X=\mathcal{F}(x)$ is a solution associated to $U=\mathcal{F}(u)$ of the linear time invariant (LTI) system:
\begin{align}
	\dot X(t)=(\mathcal{A}-\mathcal{N})X(t)+\mathcal{B}U(t), \quad X(0)=\mathcal{F}(x)(0) \label{ltih}
\end{align}
where $\mathcal{A}=\mathcal{T}(A)$, $\mathcal{B}=\mathcal{T}(B)$ and 
\begin{equation}\mathcal{N}=Id_n\otimes diag( \textsf{j}\omega k,\ k\in \mathbb{Z})\label{q}\end{equation}
Reciprocally, if $X\in H$ is a solution to \eqref{ltih} with $U\in H$, then their representatives $x$ and $u$
(i.e. $X=\mathcal{F}(x)$ and $U=\mathcal{F}(u)$) are a solution to~\eqref{ltp}. In addition, it is proved in \cite{Blin} that one can reconstruct time trajectories from harmonic ones, that is:
\begin{align*}\label{recos} x(t)&=\mathcal{F}^{-1}(X)(t)=\sum_{k=-\infty}^{+\infty} X_k(t)e^{ \textsf{j}\omega k t}+\frac{T}{2}\dot X_0(t)\end{align*}
where $X_{k}=(X_{1,k}, \cdots, X_{n,k})$ for any $k\in \mathbb{Z}$.
 \textcolor{red}{\remark In this paper, we use a TB matrix representation instead of a more standard Block Toeplitz (BT) matrix representation. The main reason is that it allows to obtain a  structure of the harmonic equations similar to the one in the time domain (see for example \eqref{btop}). This is more suitable for analysis and control design purposes. To obtain a BT structure as in [2],[10] one has to define $\mathcal{F}$ 
by $X:=\mathcal{F}(x)=( \cdots, X_{-1},X_{0}, X_{1},\cdots)$ where $X_k$ refers to the $k-th$ phasors instead of $X:=\mathcal{F}(x)=(\mathcal{F}(x_1), \cdots,\mathcal{F}(x_n)).$ Obvioulsly, we can always switch from one representation to another by applying a permutation matrix.}
\subsection{Trace operator}
Consider the vectorial space $S^n$ of $T-$periodic, $L^\infty([0\ T])$ and symmetric $n\times n$ matrix functions.
Define the scalar product on $S^n\times S^n$ by: 
\begin{align*}
	<M,N>_{S^n}&=\frac{1}{T}\int_0^Ttr(M(\tau)N(\tau))d\tau %\label{sca}
\end{align*}
\textcolor{red}{The induce norm satisfies: $\|M\|_{L^\infty}\leq \ <M,M>_{S^n}^\frac{1}{2}\leq n\|M\|_{L^\infty}$.
Let $S^n_+=\{M\in S^n: M\geq 0\ a.e.\}$ and $S^n_{++}=\{M\in S^n: M> 0\ a.e.\}$. 
It follows that: $M\in S^n_{++}$ if and only if $\mathcal{M}=\mathcal{T}(M)$ is a constant, hermitian, positive definite, TB and bounded on $\ell^2$ operator i.e. there exists $\kappa>0$, $$\|\mathcal{M}\|_{\ell^2}=\sup_{\|x\|_{\ell^2}=1}\|\mathcal{M}x\|_{\ell^2}<\kappa$$ 
Moreover, $\|\mathcal{M}\|_{\ell^2}=\|M\|_{L^\infty}$ (see p. 562-574 of \cite{Gohberg}).}\\

If $M\in S^n_{++}$, as any component of $M$ can be rewritten as:
$$M_{ij}(t)=\sum_{k\in \mathbb{Z}} m_{ij,k} e^{ \textsf{j}\omega kt}\ a.e.,$$
it follows that: $<Id,M>_{S^n}=\sum_{i=1}^n m_{ii,0}>0.$
\color{red}Therefore, the trace operator for $\mathcal{M=}\mathcal{T}(M)$ can be defined as follows.\color{black}
\begin{definition}\label{trace}The trace operator for hermitian, positive definite, \textcolor{red}{TB} and bounded operators on $\ell^2$ is defined by
	\begin{equation}
		tr(\mathcal{M})=\sum_{i=1}^n m_{ii,0}\label{tr}\end{equation}
\end{definition}
Note that if $M\in S^n_{+}$, $tr(\mathcal{M})=0$ implies that \textcolor{red}{$M(t)=0\ a.e$ and thus} $\mathcal{M}=0$. \textcolor{red}{So, it is straightforward to check that $$tr(\mathcal{M}^2)^\frac{1}{2}=<M,M>_{S^n}^\frac{1}{2}$$ defines a norm for this operator class and it follows:
$\|\mathcal{M}\|_{\ell^2}\leq tr(\mathcal{M}^2)^\frac{1}{2}\leq n \|\mathcal{M}\|_{\ell^2}$.}

\section{\textcolor{red}{Motivations and }problem formulation}
Before stating the problem we are interested in, we give examples of problems where TBLMIs may be encountered. 
First, consider the problem of stability analysis \textcolor{red}{of the LTP system \eqref{ltp}. In the time domain, this reduces to check the feasibility of the following {\it differential} Lyapunov inequality:\begin{equation}\dot P+A'P+PA<0\label{tlyap}
\end{equation} with $P=P'>0\ a.e.$ and $T-$periodic whereas in the harmonic domain, the problem is nothing than checking the feasibility of the following harmonic Lyapunov inequality:
\begin{equation}(\mathcal{A}-\mathcal{N})^*\mathcal{P}+\mathcal{P}(\mathcal{A}-\mathcal{N})<0\label{lyap}
\end{equation}
with $\mathcal{P}=\mathcal{P}^*>0$.}
\color{black}
%%%%%%%%%%%%%%%%%%%%%%%%%%%%%%%%%%%%%%%%%
TBLMIs can also be encoutered in control design problems. \textcolor{red}{Consider the state feedback design problem for \eqref{ltp} which consists in the determination of a control: $u(t)=-K(t)x(t)$ where $K(\cdot)$ is $T-$periodic and $L^\infty$ matrix function. This problem can be approached in an equivalent way in the harmonic domain by determining a TB static gain $\mathcal{K}$ bounded on $\ell^2$ such that the control $U=-\mathcal{K}X$
stabilizes the infinite dimensional harmonic system \eqref{ltih}. The time-domain  control is simply obtained from the formula: $$u(t)=-K(t)x(t)=-\mathcal{F}^{-1}(\mathcal{K}X)(t)$$}
The problem reduces to the determination of  a stabilizing state feedback gain $\mathcal{K}=\mathcal{Y}\mathcal{S}^{-1}$
where the \textcolor{red}{TB} matrices $\mathcal{Y}$ and $\mathcal{S}$ are solutions (bounded on $\ell^2$) of the TBLMI: 
\begin{align*}
(\mathcal{A}-\mathcal{N})\mathcal{S}+\mathcal{S}(\mathcal{A}-\mathcal{N})^*-\mathcal{B}\mathcal{Y}-\mathcal{Y}^*\mathcal{B}^*&<0 
\end{align*}
with $\mathcal{S}=\mathcal{S}^*>0$.
One may also mention the harmonic LQR problem whose solution is obtained by solving the associated 
infinite dimensional convex optimization problem \cite{Wil71}:
\begin{align} 
 &\max_{\scriptsize \mathcal{P}=\mathcal{P}^*>0} tr(\mathcal{P}),\ % \sum_{i=1}^n Z_{ii,0}
\label{op}\\
&\left(\begin{array}{cc}
(\mathcal{A}-\mathcal{N})^*\mathcal{P}+\mathcal{P}(\mathcal{A}-\mathcal{N})+\mathcal{N} & \mathcal{PB} \\
\mathcal{B}^*\mathcal{P}& \mathcal{R}
\end{array}\right)\geq 0\nonumber
\end{align}
where the trace operator is defined by \eqref{tr} and $\mathcal{N}$ and $\mathcal{R}$ are the LQR weighting matrices. The matrix gain is given by $\mathcal{K}=\mathcal{R}^{-1}\mathcal{B}^*\mathcal{P}$ where $\mathcal{P}$ is a \textcolor{red}{TB} matrix of infinite dimension and a bounded operator on $\ell^2$; see \cite{Blin}.
\begin{definition} A TBLMI is defined by:
\begin{equation}
	\mathcal{L}(\mathcal{P};\textcolor{red}{\mathcal{A}_s,s\in \mathbb{S}})<0
	\label{LMg}
\end{equation}
where $\mathcal{P}$ is the unknown \textcolor{red}{TB} operator assumed to be a bounded operator on $\ell^2$, $\textcolor{red}{\mathcal{A}_s,s\in \mathbb{S}}$ are given operators and $\textcolor{red}{\mathbb{S}}$ is a finite set of subscripts. \textcolor{red}{For instance,} in \eqref{lyap}, we have two given operators $\mathcal{A}_1 = \mathcal{A}$ and $\mathcal{A}_2 = \mathcal{N}$. 
\end{definition}
\begin{assumption}\label{bound}
All the entries \textcolor{red}{$\mathcal{A}_s,$ $s\in \mathbb{S}$} are TB and bounded operators on $\ell^2$ \textcolor{red}{(equivalently $A_s\in L^{\infty}$ where  $\mathcal{A}_s=\mathcal{T}(A_s)$)}, except $\mathcal{N}$ given by (\ref{q}) which is not. $\mathcal{N}$ appears with the following \textcolor{red}{TB} form: $\mathcal{N}^*\mathcal{P}+\mathcal{P}\mathcal{N}$ \textcolor{red}{which corresponds to the Toeplitz transformation $\mathcal{T}(\dot P)=-\mathcal{N}^*\mathcal{P}-\mathcal{PN}$ where $\mathcal{P}=\mathcal{T}(P)$ and where $P$ is a $T-$periodic and absolutely continuous matrix function \cite{Blin}.}
\end{assumption}
\color{black}
The problem we tackle in this paper is to determine a solution to the TBLMI \eqref{LMg} that minimizes a linear objective function. \textcolor{red}{This Convex Optimisation Problem (COP) can be stated as follows}:
\begin{align*}{\bf COP:} &\min_{\mathcal{P}^*=\mathcal{P}>0} F(\mathcal{P})=tr(\mathcal{CP}) \text{ subject to:}\\
	&\mathcal{L}(\mathcal{P};\textcolor{red}{\mathcal{A}_{s}, s\in \mathbb{S}})\leq 0
\end{align*}
\textcolor{red}{where $\mathcal{C}=\mathcal{T}(C)$ with $C\in S_{++}^n$. Note that $tr(\mathcal{CP})$ is well defined since $\mathcal{CP}=\mathcal{CP}^*>0$ and is bounded on $\ell^2$.\\
Handling the infinite dimension nature of this convex optimisation problem is very challenging. The question we answer in this paper, is how to proceed numerically with this kind of infinite dimension problems and obtain a solution up to an arbitrarily small error. 
In the next section, we show how ${\bf COP}$ can be solved up to an arbitrarily small error.} 
\section{Main results}
%\vspace{-.1cm}
\subsection{Truncation operator $\Pi$}
\begin{definition}\label{proj} Consider infinite-dimensional \textcolor{red}{TB} matrices $\mathcal{A}$ and $\mathcal{B}$ of compatible size. The truncation operator $\Pi_m$ at order $m$ is determined by:
\begin{align}
&\Pi_m(\mathcal{A}) =\mathcal{T}_m(A),\nonumber\\
&\Pi_m(\mathcal{A}+\mathcal{B})=\Pi_m(\mathcal{A}) +\Pi_m(\mathcal{B})\nonumber\\
&\Pi_m(\mathcal{AB})= \Pi_m(\mathcal{A})\Pi_m(\mathcal{B})+\mathcal{H}_{(m,\eta)}(A^+)\mathcal{H}_{(\eta,m)}(B^-) \nonumber  \\&\qquad \qquad \quad+\mathcal{J}_{n,m} \mathcal{H}_{(m,\eta)}(A^-)\mathcal{H}_{(\eta,m)}(B^+)\mathcal{J}_{n,m}\label{pro}
\end{align}
where $\eta\in \mathbb{Z}^+\cup \{+\infty\}$ is such that $2\eta\geq \min \textcolor{red}{(d^oA,d^oB)}$.
\end{definition}
For a given $m$, the $m-$truncated TBLMI of \eqref{LMg} is:
 \begin{equation}\Pi_m(\mathcal{L}(\mathcal{P};\textcolor{red}{\mathcal{A}_{s}, s\in \mathbb{S}}))<0\label{lmi_trunc}\end{equation}
For example, the $m-$truncated TBLMI associated to \eqref{lyap}~is: 
\begin{align}&\textcolor{red}{\Pi_m((\mathcal{A}-\mathcal{N})^*)\Pi_m(\mathcal{P})+\Pi_m(\mathcal{P})\Pi_m(\mathcal{A}-\mathcal{N})}\nonumber\\
& +\mathcal{H}_{(m,\eta)}(A^{*+})\mathcal{H}_{(\eta,m)}(P^-)+\mathcal{H}_{(m,\eta)}(P^{+})\mathcal{H}_{(\eta,m)}(A^{-})\nonumber\\
 &+\mathcal{J}_{n,m} (\mathcal{H}_{(m,\eta)}(A^{*-})\mathcal{H}_{(\eta,m)}(P^+)\nonumber\\&+ \mathcal{H}_{(m,\eta)}(P^-)\mathcal{H}_{(\eta,m)}(A^+))\mathcal{J}_{n,m}<0\nonumber
\end{align}
with $\eta\in \mathbb{Z}^+\cup \{+\infty\}$ is such that $2\eta\geq \min  \textcolor{red}{(d^oA,d^oP)}$.
\begin{theorem}\label{sol_trunc}If $\mathcal{P}$ solves the infinite-dimensional TBLMI (\ref{LMg}) then $\mathcal{P}$ 
solves the truncated TBLMI (\ref{lmi_trunc}) at any order~$m$.
\end{theorem}
\begin{proof} Consider a solution $\mathcal{P}$ to \eqref{LMg} then for any $m>0$, the principal submatrix $\Pi_m(\mathcal{L}(\mathcal{P};\textcolor{red}{\mathcal{A}_s,s\in \mathbb{S}}))$ \textcolor{red}{of $\mathcal{L}(\mathcal{P};\textcolor{red}{\mathcal{A}_s,s\in \mathbb{S}})$} is necessarily negative definite. Moreover, \textcolor{red}{thanks} to \eqref{ee2} and to Definition~\ref{proj},  $\Pi_m(\mathcal{L}(\mathcal{P};\textcolor{red}{\mathcal{A}_s,s\in \mathbb{S}}))$ \textcolor{red}{can be explicitly developed without any approximation.} Therefore, a solution to the obtained $m-$truncated TBLMI can be deduced from $\mathcal{P}$ itself.
\end{proof}
\textcolor{red}{Consequently}, if the infinite-dimensional TBLMI (\ref{LMg}) is feasible then there always exists a solution to the truncated TBLMI (\ref{lmi_trunc}) at any order $m$. 
Unfortunately, \eqref{lmi_trunc} may contain terms involving infinite-dimensional Hankel matrices (when $\eta=+\infty$ in \eqref{pro}). \textcolor{red}{The next section shows how this infinite dimension problem can be reduced to a finite one by considering banded approximation of the entries.}
\subsection{Truncated and banded approximation of \textcolor{red}{TBLMI}}
\color{red}Consider a \textcolor{red}{TB} operator $\mathcal{A}$ and  %with its associated symbol $A(z)$.
 its $p-$banded version $\mathcal{A}_{b(p)}$ obtained by deleting all its phasors of order higher than $p$.\color{black}
\begin{theorem}\label{conv_l2}Assume that $\mathcal{A}$ is a bounded operator on $\ell^2$. The operator $\mathcal{A}_{b(p)}$ converges to $\mathcal{A}$ in $\ell^2$-operator norm i.e.
$$\lim_{p\rightarrow +\infty}\|\mathcal{A}-\mathcal{A}_{b(p)}\|_{\ell^2}=0$$
\end{theorem}
\begin{proof}
\textcolor{red}{As $\|\mathcal{A}\|_{\ell^2}=\|A\|_{L^\infty}$ where $\mathcal{A}=\mathcal{T}(A)$\cite{Gohberg} and using the Fourier series of $A$: $$A(t)=\sum_{k\in\mathbb{Z}} A_ke^{ \textsf{j}\omega kt} \ a.e.,$$} we can write: 
\begin{align}
\|\mathcal{A}-\mathcal{A}_{b(p)}\|_{\ell^2}& =\|A-A_{b(p)}\|_{L^\infty}=\|\sum_{|k|>p} A_ke^{ \textsf{j}\omega kt} \|_{L^\infty}\label{e1}
\end{align}
As by assumption there exists a constant $C$ such that
\begin{align*}
\|\mathcal{A}\|_{\ell^2}=\|A\|_{L^\infty}
&= \|\sum_{k\in \mathbb{Z}} A_ke^{ \textsf{j}\omega kt} \|_{L^\infty}<C
\end{align*}
the series $\sum_{k\in \mathbb{Z}} A_ke^{ \textsf{j}\omega kt}$ converges almost everywhere and $\lim_{p\rightarrow +\infty}\sum_{|k|>p} A_ke^{ \textsf{j}\omega kt}=0\ a.e.$ 
Taking the limit w.r.t. $p$ in \eqref{e1} leads to the result. 
\end{proof}
This result allows to replace any TBLMI by its banded version as stated in the following theorem.
\begin{theorem}\label{t1}Under Assumption \ref{bound}, if $\mathcal{P}$ is a solution to \eqref{LMg} then for any $\epsilon>0$, there exists $p_0$ such that for $p\geq p_0$, 
$$\|\mathcal{L}(\mathcal{P};\textcolor{red}{\mathcal{A}_{s}, s\in \mathbb{S}})-\mathcal{L}(\mathcal{P};\textcolor{red}{\mathcal{A}_{s_{b(p)}}, s\in \mathbb{S}})\|_{\ell^2}<\epsilon$$
which implies that $\mathcal{P}$ satisfies the $p-$banded TBLMI:
\begin{equation}\mathcal{L}(\mathcal{P};\textcolor{red}{\mathcal{A}_{s_{b(p)}}, s\in \mathbb{S}})<0 \label{blmi}\end{equation} for sufficiently small $\epsilon$.
\end{theorem}
\begin{proof}
By assumption~\ref{bound}, the only entry of the TBLMI not bounded on $\ell^2$ is $\mathcal{N}$. Fortunately, as $\mathcal{N}$ is diagonal, $\mathcal{N}=\mathcal{N}_{{b(p)}}$ for any $p\geq 0$ and thus $\mathcal{N}$ does not play any role.
If $\mathcal{P}$ is a solution to \eqref{LMg}, then $\mathcal{L}(\mathcal{P};\textcolor{red}{\mathcal{A}_{s}, s\in \mathbb{S}})$ must be a bounded operator on $\ell^2$. LMIs being continuous with respect to their entries, there exists a constant $C$ depending of $\mathcal{P}$, \textcolor{red}{$\mathcal{A}_s,$ $s\in \mathbb{S}$} such that for any $p>0$
\begin{align*}\|\mathcal{L}(\mathcal{P};\textcolor{red}{\mathcal{A}_{s}, s\in \mathbb{S}})-\mathcal{L}(\mathcal{P};&\textcolor{red}{\mathcal{A}_{s_{b(p)}}, s\in \mathbb{S}})\|_{\ell^2}\\&\leq C \sum_{s\in \textcolor{red}{\mathbb{S}}} \|\mathcal{A}_{s}-\mathcal{A}_{s_{b(p)}}\|_{\ell^2}\end{align*}
From Theorem \ref{conv_l2}, we conclude that for any $\epsilon>0$, there exists $p_0$ such that for $p\geq p_0$, 
$$\|\mathcal{L}(\mathcal{P};\textcolor{red}{\mathcal{A}_{s}, s\in \mathbb{S}})-\mathcal{L}(\mathcal{P};\textcolor{red}{\mathcal{A}_{s_{b(p)}}, s\in \mathbb{S}})\|_{\ell^2}<\epsilon$$
and relation \eqref{blmi} follows. 
\end{proof}
Consider now the $m-$truncated and $p-$banded TBLMI: 
\begin{equation} 
\Pi_m(\mathcal{L}(\mathcal{P};\textcolor{red}{\mathcal{A}_{s_{b(p)}}, s\in \mathbb{S}}))<0 \label{lmi_trunc3}.
\end{equation}
\begin{theorem}\label{finite}For given $p$ and $m$, the number of unknowns $\nu(p,m)$ involved in \eqref{lmi_trunc3} is finite. 
\end{theorem}
\begin{proof} As all \textcolor{red}{$\mathcal{A}_{s_{b(p)}}$, $s\in \mathbb{S}$} in $\mathcal{L}(\mathcal{P};\textcolor{red}{\mathcal{A}_{s_{b(p)}}, s\in \mathbb{S}})<0$ are banded, only the unknown $\mathcal{P}$ is possibly not banded.
As the product of infinite dimensional banded \textcolor{red}{TB} operators is a banded \textcolor{red}{TB} operator (which is not true in finite dimension), the terms in the TBLMI involving operator $\mathcal{P}$ have the generic form: 
$\mathcal{U}\mathcal{P}\mathcal{V}$ where $\mathcal{U}$ and $\mathcal{V}$ are polynomial functions of banded entries $\textcolor{red}{\mathcal{A}_{s_{b(p)}}, s\in \mathbb{S}}$ and are therefore banded. 
Applying $\Pi_m$ on $\mathcal{L}(\mathcal{P};\textcolor{red}{\mathcal{A}_{s_{b(p)}}, s\in \mathbb{S}})$ leads to compute $\Pi_m(\mathcal{U}\mathcal{P}\mathcal{V})$. Using \eqref{pro}, we have:
\begin{align}
\Pi_m(\mathcal{U}\mathcal{P}\mathcal{V})&= \Pi_m(\mathcal{U})\Pi_m(\mathcal{PV})\label{ee1} \\
&+\mathcal{H}_{(m,\eta_1)}(U^+)\mathcal{H}_{(\eta_1,m)}((PV)^-) \nonumber \\
&+\mathcal{J}_{n,m} \mathcal{H}_{(m,\eta_1)}(U^-)\mathcal{H}_{(\eta_1,m)}((PV)^+)\mathcal{J}_{n,m}\nonumber 
\end{align}
where $\eta_1$ is the first integer greater than $\frac{1}{2} d^o U$ and where $\Pi_m(\mathcal{PV)}$ is determined using \eqref{pro} with $\eta$ the first integer greater than $\frac{1}{2} d^o V$. Noticing that the coefficient of \textcolor{red}{highest} degree invoked in the Hankel matrix $\mathcal{H}_{(m,\eta)}(\cdot)$ is of degree $2(m+\eta)+1$, it is straightforward to check that only a finite number of phasors of $\mathcal{P}$ are necessary to compute both $\Pi_m(\mathcal{PV)}$ and \eqref{ee1} and thus the result follows. %and \eqref{ee2}, the result is established. 
\end{proof}
From this last result, solving \eqref{lmi_trunc3} is now tractable numerically \textcolor{red}{since only a finite number of unknowns must be taken into account.}
\vspace{-.1cm}
\subsection{Solving {\bf COP} up to an arbitrary error}
Consider for a given $p>0$, the $p-$banded version of ${\bf COP}$ given by: 
\begin{align*}
{\bf COP_{p}:} &\min_{\mathcal{P}^*=\mathcal{P}>0} F(\mathcal{P}) \text{ subject to:}\\
&\mathcal{L}(\mathcal{P};\textcolor{red}{\mathcal{A}_{s_{b(p)}}, s\in \mathbb{S}})\leq 0\nonumber
\end{align*}
and for a given $q>0$, the following fully banded problem: 
\begin{align*}&{\bf COP_{p,q}:} \min_{\mathcal{P}^*=\mathcal{P}>0} F(\mathcal{P}) \text{ subject to:} \\
&\mathcal{L}(\mathcal{P};\textcolor{red}{\mathcal{A}_{s_{b(p)}}, s\in \mathbb{S}})\leq 0, \quad%\nonumber \\
P_{ij,k}=0, |k|>q,\ i,j=1,\cdots,n.\nonumber
\end{align*}
\textcolor{red}{where for a given $k\in\mathbb{Z}$, $P_{ij,k}=0$ refers to the $k$th-phasors of the (i,j) block of $\mathcal{P}$.}
We assume that these convex optimization problems are feasible and that the optimal solution is unique and continuous with respect to the entries \textcolor{red}{$\mathcal{A}_s$, $s\in \mathbb{S}$}. 
Given $p$ and $q$, we denote by $\hat{\mathcal{P}}$, $\hat{\mathcal{P}}_{p}$ and $\hat{\mathcal{P}}_{p,q}$, the optimal solution to ${\bf COP}$, ${\bf COP_{p}}$ and ${\bf COP_{p,q}}$ respectively.
\begin{theorem}\label{prop} \textcolor{red}{The following limits hold:
\begin{align*}
\lim_{p \rightarrow +\infty} \|\hat{\mathcal{P}}_{p}-\hat{\mathcal{P}}\|_{\ell^2}&=0, \quad
\lim_{q \rightarrow +\infty} \|\hat{\mathcal{P}}_{p,q}-\hat{\mathcal{P}}_p\|_{\ell^2}=0,\ p>0
\end{align*}}
\end{theorem}
\color{red}
\begin{proof}The first assertion is a direct consequence of the continuity of the optimal solution with respect to the entries \textcolor{red}{$\mathcal{A}_s$, $s\in \mathbb{S}$} and of Theorem \ref{conv_l2}.
%%%%%%%%%%%%%%%%%%%%%%%%%%%%%%%
To prove the last assertion, for a given $p$, as for any $q$, $\hat{\mathcal{P}}_{p,q}$ is admissible for both ${\bf COP_{p,q+1}}$ and ${\bf COP_{p}}$, it follows necessarily that
\begin{equation}F(\hat{\mathcal{P}}_{p,q})\geq F(\hat{\mathcal{P}}_{p,q+1})\geq\cdots \geq F(\hat{\mathcal{P}}_{p}) >0\label{des}\end{equation}
In other hand, for any $q$ the $q-$banded operator $\hat{\mathcal{P}}_{p_{b(q)}}$ of $\hat{\mathcal{P}}_{p}$, is also obviously admissible for  ${\bf COP_{p,q}}$, thus it follows: 
\begin{equation}
F(\hat{\mathcal{P}}_{p_{b(q)}})\geq F(\hat{\mathcal{P}}_{p,q})\geq F(\hat{\mathcal{P}}_{p})\label{tg}\end{equation}
Since $\|\hat{\mathcal{P}}_{p_{b(q)}}-\hat{\mathcal{P}}_{p}\|_{\ell^2}\rightarrow0$ when $q\rightarrow+\infty$ (see Theorem~\ref{conv_l2}), taking the limit w.r.t. $q$  in \eqref{tg} leads to:
\begin{equation}\lim_{q\rightarrow+\infty}F(\hat{\mathcal{P}}_{p,q})= F(\hat{\mathcal{P}}_{p}).\label{tg2}\end{equation}
 Now , let us show that we have also : $\lim_{q\rightarrow+\infty}\hat{\mathcal{P}}_{p,q}= \hat{\mathcal{P}}_{p}$ on $\ell^2$.
By definition $F(\hat{\mathcal{P}}_{p,q})=tr(\mathcal{C} \hat{\mathcal{P}}_{p,q})$ and as $\mathcal{C} \hat{\mathcal{P}}_{p,q}$ is hermitian positive definite and bounded on $\ell^2$, there exists a bounded operator on $\ell^2$, $\mathcal{N}_{p,q}=\mathcal{T}(N_{p,q})$ where $N_{p,q}\in L^\infty([0\ T])$ such that $$\mathcal{C} \hat{\mathcal{P}}_{p,q}=\mathcal{N}^*_{p,q}\mathcal{N}_{p,q} \text{ for any }p,q$$
and for similar argument $\mathcal{C}\hat{\mathcal{P}}_{p}=\mathcal{N}^*_{p}\mathcal{N}_{p}$ with $\mathcal{N}_{p}=\mathcal{T}(N_{p})$ and $N_{p}\in L^\infty([0\ T])$. 
Therefore, $$tr(\mathcal{C} \hat{\mathcal{P}}_{p,q})=tr(\mathcal{N}^*_{p,q}\mathcal{N}_{p,q})=<N_{p,q},N_{p,q}>_{S^n}$$
and from \eqref{tg2}, it can be concluded that 
\begin{equation}\lim_{q\rightarrow+\infty}<N_{p,q},N_{p,q}>_{S^n}=<N_{p},N_{p}>_{S^n}\label{nc}
\end{equation}
Moreover as the sequence $N_{p,q}$ indexed by $q$ is bounded (see \eqref{des}), there exists a subsequence that converges weakly on $L^\infty$ and 
Equation \eqref{nc} implies that it also converges strongly and necessarily to $N_{p}$ by uniqueness of solution. Finally, the uniqueness of the solution implies that the whole sequence converges to  $N_{p}$. It follows since $\mathcal{C}>0$ is invertible that  $\lim_{q\rightarrow+\infty} \hat {\mathcal{P}}_{q,p}=\hat{\mathcal{P}}_{p}=\mathcal{C}^{-1}\mathcal{N}^*_{p}\mathcal{N}_{p}$.
\end{proof}
\color{black}In view of Theorem~\ref{prop}, an approximation of ${\bf COP}$ can be determined by solving ${\bf COP_{p,q}}$ for sufficiently large $p$ and $q$. 
\textcolor{red}{Unfortunately, although ${\bf COP_{p,q}}$ involves a finite number of unknowns since $\mathcal{P}$ is $q-$banded, it remains infinite dimensional in its setting}.
Given $m,p,q>0$, we consider the following $m-$truntated, fully banded and finite dimensional optimization problem: 
\begin{align*}
&{\bf COP_{m,p,q}:} \min_{\mathcal{P}^*=\mathcal{P}} F(\mathcal{P}) \text{ subject to:}\quad \Pi_m(\mathcal{P})>0,\\
&\Pi_m(\mathcal{L}(\mathcal{P};\textcolor{red}{\mathcal{A}_{s_{b(p)}}, s\in \mathbb{S}}))\leq 0,\ %\nonumber \\
P_{ij,k}=0, |k|>q,\ i,j=1,\cdots,n.\nonumber
\end{align*}
and we denote its solution by $\hat{\mathcal{P}}_{m,p,q}$. \textcolor{red}{The problem is now clearly finite and can be solved numerically by noting that \textcolor{blue}{$F(\mathcal{P})=tr(\mathcal{CP})$ requires only a finite number of phasors to be evaluated since $\mathcal{P}$ is banded} (see Def.~\ref{trace}).} 
\begin{theorem} \label{prop2}For any $m,p,q>0$, assuming that any unbounded sequence of admissible candidates for Problem ${\bf COP_{m,p,q}}$ has an unbounded objective function $F$, then: 
 $$\lim_{m \rightarrow +\infty} \|\hat{\mathcal{P}}_{m,p,q}-\hat{\mathcal{P}}_{p,q}\|_{\ell^2}=0$$
\end{theorem}
\begin{proof}For any $m$ and following similar steps of the prof of Theorem~\ref{sol_trunc} as $\hat{\mathcal{P}}_{p,q}$ is admissible for Problem ${\bf COP_{m,p,q}} $ and $\hat{\mathcal{P}}_{m+1,p,q}$ is admissible for Problem ${\bf COP_{m,p,q}}$ , it follows that:
$$F(\hat{\mathcal{P}}_{m,p,q})\leq F(\hat{\mathcal{P}}_{m+1,p,q})\leq \cdots \leq F(\hat{\mathcal{P}}_{p,q})$$
which proves that the sequence $F(\hat{\mathcal{P}}_{m,p,q})$ indexed by $m$ is an increasing and bounded real sequence, and thus a converging sequence. Moreover, for any $m_0$, as the sequence $\hat{\mathcal{P}}_{m,p,q}$, $m\geq m_0$ is admissible for Problem ${\bf COP_{m_0,p,q}}$, following the assumption, this sequence is necessarily bounded on $\ell^2$.
Therefore, for any $m\geq m_0$, the phasors of $\hat{\mathcal{P}}_{m,p,q}$ are bounded and belong to a finite dimensional subspace of $\ell^2$ (thanks to the constraints $P_{ij,k}=0 \text{ for } |k|>q$). 
By compactness, there exists a subsequence that converges on this finite subspace of $\ell^2$ and the uniqueness of the solution implies that the whole sequence converges necessarily to $\hat{\mathcal{P}}_{p,q}$.
\end{proof}
%
%
%Thus we focus now on how to approximate for a given $p$, ${\bf COP_{\infty,p}(\nu)}$ by considering 
%the truncated p-banded problems (with $\nu=0$):
%\begin{align}{\bf COP_{m,p}(\nu):} \min_{\mathcal{P}} F(mathcal{P}) \text{ subject to:}\label{P3}\\
%\Pi_m(\mathcal{L}(\mathcal{P};\textcolor{red}{\mathcal{A}_{s_{b(p)}}, s\in \mathbb{S}})+\nu\mathcal{I})\leq 0\label{P4}\\
%\Pi_m(\mathcal{P})>0\\
%{P}_{ij,k}=0,\ (i,j,k)\in S_{nul}(m,p)\label{P5}
%\end{align}
%\begin{theorem}
%$\hat{\mathcal{P}}_{m,p}(\nu)$ converges on $\ell^2$ to $\hat{\mathcal{P}}_{\infty,p}(\nu)$ and we have 
%$F(\hat{\mathcal{P}}_{m,p}(\nu))\leq F(\hat{\mathcal{P}}_{m+1,p}(\nu)(m))\leq \cdots \leq F(\hat{\mathcal{P}}_{\infty,p}(\nu)(m))$
%\end{theorem}
%\begin{proof}First at all, the minimal solution $\hat{\mathcal{P}}_{\infty,p}(\nu)(m)$ associated to $\hat{\mathcal{P}}_{\infty,p}(\nu)$ is obviously admissible for Problem ${\bf COP_{m,p}(\nu)}$, therefore $F(\hat{\mathcal{P}}_{m,p}(\nu))\leq F(\hat{\mathcal{P}}_{\infty,p}(\nu)(m))$. For the same reason, we get also:$F(\hat{\mathcal{P}}_{m,p}(\nu))\leq F(\hat{\mathcal{P}}_{m+1,p}(\nu)(m))$.
%
%If $F(\hat{\mathcal{P}}_{m,p}(\nu))= F(\hat{\mathcal{P}}_{m+1,p}(\nu)(m))$, then the unicity implies that $\hat{\mathcal{P}}_{m,p}(\nu)=\hat{\mathcal{P}}_{m+1,p}(\nu)(m)$ and since $F(\hat{\mathcal{P}}_{m+1,p}(\nu))\leq F(\hat{\mathcal{P}}_{m+1,p}(\nu)(m)$
%
%
%Since, $\hat{\mathcal{P}}_{\infty,p}(\nu)$ is bounded on $\ell^2$, its phasor sequence $P_{ij,k}=0$ $i,j=1,\cdots,n$ belongs to $\ell^2$ and thus for any $\epsilon$, there exists $m_0>0$ such that $$\|\hat{\mathcal{P}}_{\infty,p}(\nu)-\hat{\mathcal{P}}_{\infty,p}(\nu)(m_0)\|<\epsilon$$ and thus $$|F(\hat{\mathcal{P}}_{\infty,p}(\nu)-\hat{\mathcal{P}}_{\infty,p}(\nu)(m_0))|\leq M\epsilon$$
%
%If we chose $\epsilon$ such that the $k_0$-truncation of $\hat{\mathcal{P}}_{\infty,p}(\nu)$ denoted $\hat{\mathcal{P}}_{\infty,p}(\nu)|_{k_0}$ obtained by imposing for $k>k_0$ $P_{ij,k}=0$ $i,j=1,\cdots,n$ is admissible for Problem $\hat{\mathcal{P}}_{\infty,p}(0)$
% 
%
%\end{proof}
%
%\textcolor black
%\begin{theorem}We have for any $q$,
%$$F(\hat{\mathcal{P}}_{\infty,p}(0))\leq \cdots \leq F(\hat{\mathcal{P}}_{\infty,p,q+1})\leq F(\hat{\mathcal{P}}_{\infty,p,q})$$
%and $\lim_{q\rightarrow \infty} F(\hat{\mathcal{P}}_{\infty,p,q})=F(\hat{\mathcal{P}}_{\infty,p}(0))$
%If $F(\hat{\mathcal{P}}_{\infty,p,q+1})=F(\hat{\mathcal{P}}_{\infty,p,q})$ then $\hat{\mathcal{P}}_{\infty,p,q+1}=\hat{\mathcal{P}}_{\infty,p,q}$.
%
%\end{theorem}
%\begin{proof}%First of all, if there exists $N_0$ such that $F(\hat{\mathcal{P}}_{\infty,p,N_0+1})=F(\hat{\mathcal{P}}_{\infty,p,N_0})$, as $\hat{\mathcal{P}}_{\infty,p,N_0}$ is admissible for ${\bf COP_{\infty,p,N_0+1}}$ and thus optimal, by uniqueness of the solution, $\hat{\mathcal{P}}_{\infty,p,N_0+1}=\hat{\mathcal{P}}_{\infty,p,N_0}$. 
%
%As the set of admissible candidates for Problem ${\bf COP_{\infty,p,q}}$ is smaller than both Problems ${\bf COP_{\infty,p,}}$ and ${\bf COP_{\infty,p,q+1}}$, we have 
%$$F(\hat{\mathcal{P}}_{\infty,p}(0))\leq F(\hat{\mathcal{P}}_{\infty,p,q+1})\leq F(\hat{\mathcal{P}}_{\infty,p,q})$$
%
%By continuity of the optimal solution with respect the entries, as $\hat{\mathcal{P}}_{\infty,p}(0)$ belongs to $\ell^2$, there exists $N_0$ such that for any $q>N_0$
%$$\|\hat{\mathcal{P}}_{\infty,p,q}-\hat{\mathcal{P}}_{\infty,p}(0)\| <\epsilon$$ and by continuity
%$$\|F(\hat{\mathcal{P}}_{\infty,p,q}-\hat{\mathcal{P}}_{\infty,p}(0))\|<M\epsilon$$
%\end{proof}
%
%

%\begin{theorem}
%There exists $\epsilon_0$ such that for any $0<\epsilon<\epsilon_0$, there exists $N_0$ such that for $q\geq N_0$, $$F(\hat{\mathcal{P}}_{\infty,p}(0))\leq F(\hat{\mathcal{P}}_{\infty,p,q})\leq F(\hat{\mathcal{P}}_{\infty,p}(\epsilon))+M\epsilon$$
%Moreover, 
%\end{theorem}
%\begin{proof}%Let consider for a given $\epsilon$, the optimal solution $\hat{\mathcal{P}}_{\infty,p}(\epsilon)$ and 
%$\hat{\mathcal{P}}_{\infty,p}(\epsilon)_{b(q)}$ its banded version at order $q$, such that 
%$$\|\hat{\mathcal{P}}_{\infty,p}(\epsilon)_{b(q)}-\hat{\mathcal{P}}_{\infty,p}(\epsilon)\|_{\ell^2}<\epsilon$$
%and
%$$\mathcal{L}(\hat{\mathcal{P}}_{\infty,p}(\epsilon)_{b(q)};\textcolor{red}{\mathcal{A}_{s_{b(p)}}, s\in \mathbb{S}})+\frac{1}{2}\epsilon \mathcal{I}\leq 0.$$
%If $\epsilon$ is chosen sufficiently small, $\hat{\mathcal{P}}_{\infty,p}(\epsilon)_{b(q)}>0$ and is admissible for
%Problem ${\bf COP_{\infty,p,q}}$, thus it follows that 
%$$F(\hat{\mathcal{P}}_{\infty,p,q})\leq F(\hat{\mathcal{P}}_{\infty,p}(\epsilon)_{b(q)})=F(\hat{\mathcal{P}}_{\infty,p}(\epsilon))+M \epsilon$$
%In other hand as the set of admissible candidate for Problem ${\bf COP_{\infty,p,q}}$ is smaller than for Problem ${\bf COP_{\infty,p}(0)}$, we have 
%$$F(\hat{\mathcal{P}}_{\infty,p}(0))\leq F(\hat{\mathcal{P}}_{\infty,p,q})$$ and the first part of the result is established.
%Now, the assertion $\|\hat{\mathcal{P}}_{\infty,p,q}-\hat{\mathcal{P}}_{\infty,p}(0)\|_{\ell^2}\rightarrow 0$ when $q\rightarrow \infty$
%\end{proof}
\textcolor{red}{Theorems \ref{prop} and \ref{prop2} now clearly show that solving ${\bf COP_{m,p,q}}$ is a consistent scheme allowing to approximate the solution to ${\bf COP}$ as stated in this final result:
\begin{theorem}For any $\epsilon>0$, there exist $p,q,m$ such that:\\
$ \|\hat{\mathcal{P}}_{p}-\hat{\mathcal{P}}\|_{\ell^2}\leq \frac{\epsilon}{3}$, $\|\hat{\mathcal{P}}_{p,q}-\hat{\mathcal{P}}_p\|_{\ell^2}\leq \frac{\epsilon}{3}$ and $\|\hat{\mathcal{P}}_{p,q,m}-\hat{\mathcal{P}}_{p,q}\|_{\ell^2}\leq \frac{\epsilon}{3}$
and thus $ \|\hat{\mathcal{P}}_{p,q,m}-\hat{\mathcal{P}}\|_{\ell^2}\leq \epsilon$.
\end{theorem}}
\section{Illustrative example}
We consider the example given in \cite{Pierre2022} defined by:
\begin{align}
	\dot x=&\left(\begin{array}{cc}a_{11} (t) & a_{12} (t) \\a_{21} (t) & a_{22} (t)\end{array}\right)x+\left(\begin{array}{c}b_{11}(t) \\0\end{array}\right)u\label{ex_ltp}\end{align}
{\small\begin{align*}a_{11} (t) &=1+\frac{4}{\pi}\sum_{k=0}^{\infty}\frac{1}{2k+1}\sin(\omega (2k+1)t),\\
	a_{12} (t) &= 2+\frac{16}{\pi^2}\sum_{k=0}^{\infty}\frac{1}{(2k+1)^2}\cos(\omega (2k+1)t),\\
	a_{21} (t) &= -1+\frac{2}{\pi}\sum_{k=1}^{\infty}\frac{(-1)^k}{k}\sin(\omega kt+\frac{\pi}{4}),\\
	a_{22} (t) &= 1-2\sin(\omega t)-2\sin(3\omega t)+2\cos(3\omega t)+2\cos(5\omega t),\\
	b_{11}(t)&=1+ 2 \cos(2\omega t)+ 4 \sin(3\omega t) \text{ with }\omega=2\pi.
\end{align*}}
The associated Toeplitz matrix $\mathcal{A}$ has an infinite number of phasors and is not banded. This system is unstable and the equivalent harmonic LTI system \eqref{ltih} has a spectrum provided by the set $\sigma=\{\lambda+ \textsf{j}\omega k, k\in \mathbb{Z}\}$ where $\lambda \in \{1\pm \textsf{j} 1.64\}$ (see \cite{Pierre2022}).

We consider the LQ problem and we solve the optimization problem given by \eqref{op} with $\mathcal{N}=10^2\mathcal{T}(Id_n)$ and $\mathcal{R}=\mathcal{T}(Id_m)$. Imposing \textcolor{red}{as required }a \textcolor{red}{TB} structure to $\mathcal{P}$, Problem ${\bf COP_{m,p,q}}$ associated to \eqref{op} is solved with $m=10,15,20$ and $p=q=2m$ (Obviously other choices are possible such as for exemple $m=p=q$). This is illustrated in Fig.~\ref{f1} where we plot the modulus of phasors of $\mathcal{K}=[\mathcal{K}_1,\mathcal{K}_2]$ \textcolor{red}{and in Fig.~\ref{f4} where the control $u(t):=-K(t)x(t)$ with $K(t)$ the $T-$periodic gain matrix given by $K(t):=\sum_{k=-2m}^{2m} K_k e^{\textsf{j}\omega kt}$, stabilizes globally and asymptotically the unstable LTP system \eqref{ex_ltp}. As a result, we recover the same state feedback gain as in \cite{Pierre2022} which was obtained using a Kleinman-like algorithm.}
\begin{figure}[h]
	\begin{center}
		\includegraphics[width=\linewidth,height=4.7cm]{PhasorK_ACC}
		\caption{Modulus of Phasors $K=[K_1,K_2]$ (harmonic LQ control)}\label{f1}
	\end{center}
\end{figure}
\begin{figure}[h]
	\begin{center}
		\includegraphics[width=\linewidth,height=5.5cm]{Traj_LQ}
		\caption{Closed loop response with LQ control}\label{f4}
	\end{center}
\end{figure}
%\addtolength{\textheight}{0cm} % This command serves to balance the column lengths
   % on the last page of the document manually. It shortens
   % the textheight of the last page by a suitable amount.
   % This command does not take effect until the next page
   % so it should come on the page before the last. Make
   % sure that you do not shorten the textheight too much.
%%%%%%%%%%%%%%%%%%%%%%%%%%%%%%%%%%%%%%%%%%%%%%%%%%%%%%%%%%%%%%%%%%%%%%%%%%%%%%%%
\section{Conclusion}
In this paper, we provided a novel approach that allows to solve, up to an arbitrarily small error, infinite dimensional \textcolor{red}{TB} LMIs and related convex optimization problems encountered in the analysis and control of dynamical systems in the harmonic framework. The result is based on a well-defined finite dimensional truncated problem that allows to recover the original infinite-dimensional solution up to an arbitrarily small error. This framework is not only usefull for robustness and multiobjective optimization issues of LTP systems but also for the analysis and control of more general periodic systems such as periodic polynomial systems. 
\begin{thebibliography}{99}

\bibitem{Almer2} Alm\`er, S., Mari\'ethoz, S., and Morari, M., "Dynamic Phasor Model Predictive Control of Switched Mode Power Converters", \emph{IEEE Transaction on 
Control System Technology}, Vol. 23, No. 1, January 2015.
\bibitem{Blin}N. Blin, P. Riedinger, J. Daafouz, L. Grimaud and P. Feyel, "Necessary and Sufficient Conditions for Harmonic Control in Continuous Time," in IEEE Trans. on Aut. Control, vol. 67, no. 8, 2022.
	
\bibitem{Bolzern}Bolzern, P. and Colaneri, P. (1988). "The periodic Lyapunov equation". \emph{SIAM Journal on Matrix Analysis and Applications}, 9(4), 499-512.
\bibitem{Boyd}Boyd, S., El Ghaoui, L., Feron, E., and Balakrishnan, V. "Linear Matrix Inequalities in System and Control Theory", Studies in Applied math. SIAM, 1994. 
	\bibitem{Farkas} Farkas, M.: "Periodic motions" (Springer-Verlag, New York, 1994)

		\bibitem{Gohberg} 
	Gohberg, I., Goldberg, S. and Kaashoek, M.A., \emph{Classes of Linear Operators}, Operator Theory
	Advances and Applications Vol. 63 Birkhauser, Vol. II, 1993.

	\bibitem{Ikeda01}Ikeda, K., Azuma, T., and Uchida, K., "Infinite-dimensional LMI approach to analysis and synthesis for linear time-delay systems", \emph{Kybernetika} 37(4):505-520, 2001.
\bibitem{Pierre2022} P. Riedinger and J. Daafouz, "Solving Infinite-Dimensional Harmonic Lyapunov and Riccati Equations," To appear in IEEE Trans. on Aut. Control, doi: 10.1109/TAC.2022.3229943.
		\bibitem{Sanders}
Sanders, S. R., Noworolski, J. M., Liu, X. Z. and Verghese, G. C., "Generalized averaging method for power conversion circuits". \emph{IEEE Transactions on Power Electronics}, 6(2), p.p. 251-259, 1991.
	\bibitem{Wereley_1990}Wereley, N. M., "Analysis and control of linear periodically time-varying systems", \emph{Doctoral dissertation, MIT}, 1990.	
\bibitem{Wil71}	J. C. Willems. Least squares stationary optimal control and the algebraic Riccati equation. IEEE Trans. on Aut. Control, 16(6):621-634, 1971.
\textcolor{red}{
\bibitem{Zhou}Zhou, J. "Harmonic Lyapunov equations in continuous-time periodic systems: solutions and properties." IET Control Theory \& Applications 1.4 (2007): 946-954.}
	\bibitem{Zhou2008}Zhou, J., "Derivation and Solution of Harmonic Riccati Equations via Contraction Mapping Theorem", \emph{Transactions of the Society of Instrument and Control Engineers} 44(2), p.p. 156-163, 2008.

\end{thebibliography}
\end{document}
