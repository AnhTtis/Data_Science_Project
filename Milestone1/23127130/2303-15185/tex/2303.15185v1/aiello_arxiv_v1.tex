%
\documentclass[a4paper,twocolumn,11pt]{quantumarticle}
\pdfoutput=1
\usepackage[utf8]{inputenc}
\usepackage[english]{babel}
\usepackage[T1]{fontenc}
\usepackage{amsmath}
\usepackage{hyperref}
\usepackage[numbers]{natbib}
%\usepackage[toc,title,page]{appendix}

%\usepackage{tikz}
%\usepackage{lipsum}

\usepackage{amssymb,amsfonts}
\usepackage{bm}% bold math
\usepackage{url}
\usepackage{mathtools}
%%%%%%%%%%%%%%%%%%%%%%%%%%%%%%%%%%%%%%%%%%%%%%%%
%\usepackage[inline]{showlabels}
%%%%%%%%%%%%%%%%%%%%%%%%%%%%%%%%%%%%%%%%%%%%%%%%

%\newtheorem{theorem}{Theorem}
\newcommand{\deff}{{\, \vcentcolon = \,}}

%\bibliographystyle{quantum}
%\bibliographystyle{IEEEtran}
\bibliographystyle{apsrev}
%\bibliographystyle{plainnat}


\begin{document}


\title{A probabilistic view of  wave-particle duality for single photons}

\author{Andrea Aiello}
\affiliation{Max Planck Institute for the Science of Light, Staudtstrasse 2, 91058 Erlangen, Germany}
\orcid{0000-0003-1647-0448}
\email{andrea.aiello@mpl.mpg.de}
\homepage{https://mpl.mpg.de/}
\maketitle

\begin{abstract}
%
We describe a simple experiment exemplifying wave-particle duality in light beams prepared in a single-photon state.
By approaching the problem from the perspective of classical probability theory, we find that standard correlation functions fails to reveal a hidden nonlinear dependence between some wave and particle observables that can be simultaneously measured in the experiment. Therefore, we use  mutual information  as a more general measure of the dependence between such observables. This provides a new perspective on wave-particle duality.
%
\end{abstract}



\section{Introduction}

In classical mechanics a physical system is characterized by a set of \emph{degrees of freedom}, which define its state or configuration at any fixed time \cite{GoldsteinBook}. Such a set  may be  either  countable (finite or denumerable), or uncountable. The branch of classical mechanics that studies \emph{discrete} systems with a countable set of degrees of freedom, is traditionally called \emph{particle mechanics}. Conversely,  \emph{continuous} systems described by an uncountable set of degrees of freedom, are the subjects of \emph{continuum mechanics} \cite{FetterBook}.
%
In particle mechanics, a  system is described  by a set of functions of \emph{time} $t$, the   so-called generalized coordinates of the system. In continuum mechanics a  system is characterized by a set of functions of \emph{spacetime} points $(x,y,z,t)$,  the components of scalar, vector or tensor fields.
%
Thus, in classical mechanics a physical system is described  either as discrete or continuous (or part discrete and part continuous), and the two descriptions are mutually exclusive\footnote{This does not mean, for example, that we cannot use coordinates to portray some characteristics of a field. Consider, for example, an electromagnetic wave-packet with energy density $U(\mathbf{r},t)$. Such wave-packet is completely described by the electric and magnetic fields. However,  we can introduce the ``energy center of gravity of the field'' $\mathbf{R} = \mathbf{R}(t)$, defined by $\mathbf{R}(t) = \int \mathbf{r} \, U(\mathbf{r},t)\, \mathrm{d}^3 \mathbf{r}/\int  U(\mathbf{r},t) \, \mathrm{d}^3 \mathbf{r}$ \cite{SchwingerMilton}, to picture the mean position and velocity $\mathbf{V} = \mathrm{d} \mathbf{R}/ \mathrm{d} t$ of the wave-packet. However, the coordinates $\mathbf{R}(t)$ are \emph{emergent} quantities that are not necessary for the complete description of the system.}.
%
Conversely, in quantum mechanics  there is no such clear separation between  continuous and discrete aspects of a physical system. This gives rise to the famous wave-particle duality  in quantum mechanics  (see, e.g., sec. 1.5 of Ref. \cite{GalindoI}, or \cite{PhysRevLett.77.2154,Dimitrova2008} and references therein).
%

In this paper we study the wave-particle duality for single photons \cite{Scarani1998,vedral2021quantum}, using a probabilistic approach. The key idea is that probability distributions contain more information than correlation functions. For example, consider a physical system prepared in some quantum state $|\psi \rangle$. Let $\mathcal{X}$ and $\mathcal{Y}$ be  two compatible observables of the system  represented by the commuting Hermitian operators $\hat{X} = \hat{X}^\dagger$ and $\hat{Y} = \hat{Y}^\dagger$, respectively, such that $[\hat{X} ,  \hat{Y}] =0$.
These observables may be uncorrelated with respect to  $|\psi \rangle$, that is $\langle \psi | \hat{X} \hat{Y} | \psi \rangle - \langle \psi | \hat{X}  | \psi \rangle\langle \psi | \hat{Y}  | \psi \rangle =0 $. However, this does not necessarily imply the statistical independence of these observables. To check whether $ \mathcal{X}$ and $\mathcal{Y}$ are statistically independent or not, we can use von Neumann's  spectral theorem \cite{vonNeumann2018,aiello_arXiv.2110.12930}, to calculate the joint probability distribution  $p_{XY}(\psi,x,y)$ of the random variables $X$ and $Y$ associated with the operators  $\hat{X}$ and $\hat{Y}$, respectively.
%
If $X$ and $Y$ are independent then their joint probability distribution is the product of their marginal probability distributions $p_{X}(\psi,x)$ and $p_{Y}(\psi,y)$: $p_{XY}(\psi,x,y) = p_{X}(\psi,x) \, p_{Y}(\psi,y)$. Therefore, if given $\mathcal{X}, \mathcal{Y}$ and $|\psi \rangle$, we find $ \operatorname{E}[XY]-\operatorname{E}[X]\operatorname{E}[Y]=0$\footnote{Here and hereafter $\operatorname{E}[Z]$ denotes the expected value of the random variable $Z$.}, but $p_{XY}(\psi,x,y) \neq p_{X}(\psi,x) \, p_{Y}(\psi,y)$,  then correlation will not be a good measure for the dependence between $\mathcal{X}$ and $\mathcal{Y}$. In this case
knowledge of $p_{XY}(\psi,x,y)$  gives us more information than  correlation.

How does this probabilistic approach relate to wave-particle duality? Let us deal with the following experiment. A collimated beam of light prepared in a single-photon Fock state \cite{LoudonBook}, 
impinges upon a detection screen, as shown in  Fig. \ref{fig1}.
%
\begin{figure}[ht!]
  \centering
  \includegraphics[scale=3,clip=false,width=1\columnwidth,trim = 0 0 0 0]{fig_1.pdf}
  \caption{A pictorial representation of the collimated light beam impinging upon the detection screen (gray surface). Blue and red spots on the screen depict the active surfaces of two detectors, denoted $D_1$ and $D_2$. }\label{fig1}
\end{figure}
%
On this screen there are two spatially separated detectors $D_1$ and $D_2$. Each detector  can be set to either  measure the \emph{W}ave amplitude $\mathcal{W}$  of the electric field, or to \emph{C}ount the number $\mathcal{C}$ of photons, of the light falling on it. Depending on how we set up these two detectors, we can measure three different pairs of observables, that is $(\mathcal{W}_1,\mathcal{W}_2), ~ (\mathcal{C}_1,\mathcal{C}_2)$, and $(\mathcal{W}_1,\mathcal{C}_2)$. The last pair is particularly interesting because it represents the simultaneous measurement of a wave ($\mathcal{W}_1$) and a particle ($\mathcal{C}_2$) observable of the system.
%
Calculating the joint probability distribution $p_{W_1 C_2}(\psi,w_1,c_2)$ for the random variables $W_1$ and $C_2$ representing the results of measurements of $\mathcal{W}_1$ and $\mathcal{C}_2$, respectively,
we find that these two variables are uncorrelated,  $\operatorname{E}[W_1 C_2] - \operatorname{E}[W_1]\operatorname{E}[C_2] = 0$, but not independent because $p_{W_1 C_2}(\psi,w_1,c_2) \neq p_{W_1}(\psi,w_1) p_{C_2}(\psi,c_2)$. Therefore, a simple measurement of the \emph{linear} correlation between the wave and the particle observables  $\mathcal{W}_1$ and  $\mathcal{C}_2$, respectively, would fail to reveal their \emph{nonlinear} statistical dependence \cite{MurphyBook2022}. To make the latter manifest, we use as statistical measure  the mutual information \cite{CoverThomas}, a quantity that finds numerous applications in contemporary physics (see, e.g., \cite{PhysRevLett.126.200601} and references therein). This is the main result of this work.

The rest of this paper is organized as follows. In section \ref{QFT} we quickly present a phenomenological quantum field theory of paraxial beams of light, and we jot down the quantum states of the field. In section \ref{smeared} we build up and characterize the Hermitian quantum operators representing the wave ($\mathcal{W}$) and particle ($\mathcal{C}$) observables of the electromagnetic field. In section \ref{pd} we briefly review probability theory for quantum operators. Next, in section \ref{wp} first we write down and discuss the formulas for the joint probability distributions
$p_{W_1 W_2}(N,w_1,w_2), ~ p_{C_1 C_2}(N,c_1,c_2)$, and $p_{W_1 C_2}(N,w_1,c_2)$, for $N=0,1$. Then, we apply these formulas to the cases of one and two detectors, for both the vacuum ($N=0$) and the single-photon ($N=1$) states of the electromagnetic field. Finally, in section \ref{conc} we briefly summarise our results and draw some conclusions. Four appendices report detailed calculations of the results presented in the main text.


\section{Quantum field theory of light}\label{QFT}


In this section we give a brief overview of the quantum field theory of paraxial light beams. We also define and illustrate the quantum states of the electromagnetic field that will be used later.

\subsection{Paraxial quantum field operators}

Following closely \cite{aiello_arxiv_2022}, we consider   a monochromatic  paraxial  beam of light of frequency $\omega$,  propagating in the $z$ direction and polarized along the $x$ axis of a given Cartesian coordinate system.
In the Coulomb gauge, the electric field operator can be written as  $\hat{\mathbf{E}}(\mathbf{r},t) = \hat{\Phi} (\mathbf{r},t) \, \hat{\mathbf{e}}_x$, where $\mathbf{r} = x \hat{\mathbf{e}}_x + y \hat{\mathbf{e}}_y + z \hat{\mathbf{e}}_z$ is the position vector and, in suitably chosen units,
%
\begin{align}\label{a70}
%
\hat{\Phi} (\mathbf{r},t) =   \left[ e^{- i \omega t} \hat{\phi} (\mathbf{x},z)  +  e^{i \omega t} \hat{\phi}^\dagger (\mathbf{x},z)  \right]/ \sqrt{2},
%
\end{align}
%
with
%
\begin{align}\label{n10}
%
\hat{\phi} (\mathbf{x},z) =  \sum_{\mu} \hat{a}_{\mu} u_{\mu} (\mathbf{x},z).
%
\end{align}
%
In this equation $\mathbf{x} = x \bm{e}_x + y \bm{e}_y$ is the transverse position vector,  and the elements $u_{\mu} (\mathbf{x},z)$ of the  countable set of functions $\{ u_{\mu} (\mathbf{x},z)\}$, are the so-called spatial modes of the field.
%
 By hypothesis, these  modes are solutions of the paraxial wave equation \cite{GoodmanBook}, and form a complete and orthogonal set of basis functions on $\mathbb{R}^2$, i.e.,
 %
\begin{align}\label{n20}
%
\sum_{\mu} u_{\mu}(\mathbf{x},z)u_{\mu}^*(\mathbf{x}',z) =   \delta \left( \mathbf{x} - \mathbf{x}' \right) ,
%
\end{align}
%
with $\delta \left( \mathbf{x} - \mathbf{x}' \right) = \delta \left( x - x' \right) \delta \left( y - y' \right)$,  and
%
\begin{align}\label{n30}
%
\bigl( u_{\mu}, u_{\mu'}\bigr) = \delta_{\mu \mu'},
%
\end{align}
%
respectively. Here and hereafter we use the suggestive notation
%
\begin{align}\label{n40}
%
(f, g) = \int_{\mathbb{R}^2}  f^*(\mathbf{x},z) g(\mathbf{x},z) \, \mathrm{d} \mathbf{x} ,
%
\end{align}
%
where $\mathrm{d} \mathbf{x} = \mathrm{d}x \, \mathrm{d}y$.
As usual, the annihilation and creation operators $\hat{a}_{\mu}$ and $\hat{a}^{\dagger}_{\mu}$, respectively,  satisfy the bosonic canonical commutation relations
%
\begin{align}\label{n15}
%
\bigl[ \hat{a}_{\mu}, \; \hat{a}^\dagger_{\mu'} \bigr] = \delta_{\mu \mu'}.
%
\end{align}
%
Finally, we remark that from \eqref{a70}--\eqref{n40} it follows that   the dimension of  $\hat{\Phi} (\mathbf{r},t)$ is the inverse of a length: $[\hat{\Phi}] = L^{-1}$.


\subsection{Quantum states of the electromagnetic field}


Consider a classical paraxial beam of light carrying the  electric field $\mathbf{E}_\text{cl}(\mathbf{r},t) = \Phi(\mathbf{r},t) \, \hat{\mathbf{e}}_x$, where
%
\begin{align}\label{n50}
%
\Phi (\mathbf{r},t) =   \left[ e^{- i \omega t} \phi (\mathbf{x},z)  +  e^{i \omega t} \phi^* (\mathbf{x},z)  \right]/ \sqrt{2 }.
%
\end{align}
%
Here the scalar field $\phi(\mathbf{x},z)$ is a solution of the paraxial wave equation normalized to $(\phi, \phi) = 1$. By construction, the classical field $\Phi (\mathbf{r},t)$ is equal to the expectation value of the quantum field $\hat{\Phi} (\mathbf{r},t)$ with respect to the coherent state $|\{ \phi \} \rangle$, i.e., $\Phi (\mathbf{r},t) = \langle\{ \phi \} | \hat{\Phi} (\mathbf{r},t) | \{ \phi \} \rangle$,  where  $|\{ \phi \} \rangle = \exp \bigl( \hat{a}^\dagger[\phi] - \hat{a}[\phi]\bigr)| 0 \rangle$, $| 0 \rangle$ is the vacuum state of the electromagnetic field defined by $\hat{a}_\mu | 0 \rangle =0$ for all $\mu$, 
%
\begin{align}\label{n60}
%
\hat{a}^\dagger[\phi] = \bigl( \hat{\phi}, \phi \bigr) =  \sum_\mu \hat{a}^\dagger_\mu \phi_\mu,
%
\end{align}
%
 with $\phi_\mu = ( u_\mu, \phi)$ \cite{Deutsch1991,aiello_arxiv_2022}, and \eqref{n10} has been used. Note that since both the modes $u_\mu(\mathbf{x},z)$ and the field $\phi(\mathbf{x},z)$ are solutions of the paraxial wave equation, then the coefficients $\phi_\mu$ are independent of $z$. It is not difficult to show that $\bigl[ \hat{a}[\phi],\hat{a}^\dagger[\psi] \bigr]= \bigl( \phi, \psi \bigr)$, for any pair of normalized fields $\phi(\mathbf{x},z)$ and $ \psi(\mathbf{x},z)$.
%
The field  $\phi(\mathbf{x},z)$ also determines the (improperly called)  wave function of the photon, defined by $\langle 0 | \hat{\Phi}(\mathbf{r},t) | 1[\phi] \rangle = e^{- i \omega t} \phi(\mathbf{x},z)/ \sqrt{2} $, where
%
\begin{align}\label{n70}
%
| N[\phi] \rangle = \frac{\bigl( \hat{a}^\dagger[\phi] \bigr)^N}{\sqrt{N!}} | 0 \rangle ,
%
\end{align}
%
denotes the $N$-photon Fock state with $N= 0,1,2, \ldots$, such that $\hat{N} |N[\phi] \rangle = N|N[\phi] \rangle $,
%
\begin{align}\label{n75}
%
\hat{N}= \sum_\mu \hat{a}^\dagger_\mu \hat{a}_\mu ,
%
\end{align}
%
and $\langle N[\phi] | M [\psi] \rangle = \bigl( \phi, \psi \bigr)^N \delta_{NM}$ (see Supplemental Material in \cite{aiello_arxiv_2022} for further details).


\section{Wave-like and particle-like operators}\label{smeared}

In this section we build bona fide Hermitian operators $\hat{W}$ and $\hat{C}$ representing the observable electric field wave amplitude $\mathcal{W}$  and the counted photon-number $\mathcal{C}$  of a light beam, respectively.

\subsubsection{Wave-like operators}

In quantum field theory, a mathematical  object like  $\hat{\Phi} (\mathbf{x},z,t)$ does not really represent a proper observable, because it is not an Hermitian operator in the Hilbert space $\mathcal{H}$ of the physical states  of the electromagnetic field,  but rather an ``operator valued distribution'' over the Euclidean spacetime $\mathbb{R}^2 \times \mathbb{R}$  \cite{Haag1992}. This can be seen, for example, by showing that $\hat{\Phi} (\mathbf{x},z,t)$ does not map the vacuum state $| 0 \rangle \in \mathcal{H}$ into another state in $\mathcal{H}$ (a recent practical example of smearing a quantum field, can be found in \cite{https://doi.org/10.48550/arxiv.2302.13742}). To this end, let us define the vector $|\psi \rangle = \hat{\Phi} (\mathbf{x},z,t) | 0 \rangle$. Then, it is not difficult to show that $| \psi \rangle \notin \mathcal{H}$ because it has not a finite norm:
%
\begin{align}\label{n80}
%
\langle  \psi | \psi \rangle & =   \lim_{\mathbf{x}' \to \mathbf{x}} \langle  0|\hat{\Phi} (\mathbf{x},z,t) \hat{\Phi} (\mathbf{x}',z,t) |0 \rangle \nonumber \\[6pt]
%
 & =  \frac{1}{2} \, \lim_{\mathbf{x}' \to \mathbf{x}} \langle  0| \bigl[ \hat{\phi} (\mathbf{x},z,t), \hat{\phi}^\dagger (\mathbf{x}',z,t)\bigr] |0 \rangle \nonumber \\[6pt]
%
 & =  \frac{1}{2}  \lim_{\mathbf{x}' \to \mathbf{x}} \delta \left( \mathbf{x} - \mathbf{x}' \right) \nonumber \\[6pt]
%
 & = \infty,
%
\end{align}
%
where \eqref{n20} has been used. This means that the quantum fluctuations (variance) of $\hat{\Phi} (\mathbf{x},z,t)$ in the vacuum state blow up for $\mathbf{x}' \to \mathbf{x}$. Thus, to obtain a bona fide Hermitian operator defined on the vectors in $\mathcal{H}$,  we must to smear out $\hat{\Phi} (\mathbf{x},z,t)$ with a  test function $F(\mathbf{x},t) \in \mathbb{R}$ \cite{Rowe_1979,BladelBook}, namely to take
%
\begin{align}\label{a100}
%
\hat{\Phi}[F] = \int_{\mathbb{R}^2 \times \mathbb{R}}  F(\mathbf{x},t) \hat{\Phi} (\mathbf{x},z,t) \, \mathrm{d} \mathbf{x} \, \mathrm{d} t.
%
\end{align}
%
In the case of free fields, we can  choose $F(\mathbf{x},t) = \delta(t-t_0) f(\mathbf{x})$  \cite{Haag1992,ItzZub} where we normalize the real-valued function $f(\mathbf{x})$ as
%
\begin{align}\label{a160}
%
\int_{\mathbb{R}^2}  f(\mathbf{x}) \, \mathrm{d} \mathbf{x}  =1,
%
\end{align}
%
and $t_0$ is any time. Without loss of generality, in the remainder we will set $t_0=0$.
Note that normalization condition \eqref{a160} implies that the dimension of both $f(\mathbf{x})$ and $(f,f)$ is $L^{-2}$. Then we define the smeared field operator $\hat{W} \deff \hat{\Phi}[f]$ by
%
\begin{align}\label{a170}
%
\hat{W} \deff \hat{\Phi}[f] = & \; \int  f(\mathbf{x}) \, \hat{\Phi} (\mathbf{x},z,0) \, \mathrm{d} \mathbf{x} \nonumber \\[6pt]
%
= & \; \sum_\mu \left( \hat{a}_\mu f_\mu^* + \hat{a}_\mu^\dagger f_\mu \right)/\sqrt{2},
%
\end{align}
%
where $f_\mu = (u_\mu, f) = |f_\mu| e^{i \theta_\mu}$ \cite{Haag1992,ItzZub}.  For example, $f(\mathbf{x})$ can be the Gaussian function
%
\begin{align}\label{a172}
%
f(\mathbf{x}) = \frac{1}{(a \sqrt{\pi})^2} \, e^{-(x^2 + y^2)/{a^2}},
%
\end{align}
%
where $a>0$ is some length. In this case  $\hat{W}$ is  a smoothed form of the field averaged over a region of area $a^2$ \cite{Coleman2019}.
%
It is interesting to note that we can rewrite
%
\begin{align}\label{n90}
%
\hat{W} = \sum_\mu |f_\mu| \, \hat{x}_{\mu},
%
\end{align}
%
where by definition
%
\begin{align}\label{n100}
%
\hat{x}_{\mu} \deff \bigl( \hat{a}_\mu e^{-i \theta_\mu} + \hat{a}_\mu^\dagger e^{i \theta_\mu} \bigr) /\sqrt{2},
%
\end{align}
%
is the quadrature Hermitian operator of the field component with respect to the mode $u_\mu(
\mathbf{x},z)$ \cite{Barnett}.
%
We remark that since the quadrature operator $\hat{x}_{\mu}$ has a continuum of eigenvalues $\xi_{\mu} \in \mathbb{R}$ associated with the eigenvectors $|\xi_{\mu} \rangle$\footnote{See, for example, Eq. (11.8) in Ref. \cite{Schleich} for an expression of $|\xi_{\mu} \rangle$.}, then the smeared field $\hat{W}$ has a \emph{continuum spectrum} too.


From \eqref{a160} and \eqref{a170} it follows that the dimension of  $\hat{W} = \hat{\Phi}[f]$ is $L$, as that of the original operator $\hat{\Phi} (\mathbf{x},z,0)$. For later purposes, it is useful to calculate the \emph{finite} variance $\sigma^2$ of $\hat{W}$ with respect to the vacuum state:
%
\begin{align}\label{n105}
%
\sigma^2 & = \langle 0|\hat{W}^2|0 \rangle \nonumber \\[6pt]
%
& =  \left(f,f \right)/2.
%
\end{align}
%

In the remainder we will consider a set of $M$ different test functions $\{ f_n(\mathbf{x}) \} = \{f_1(\mathbf{x}), \ldots, f_M(\mathbf{x}) \}$, each normalized according to \eqref{a160}. The function $f_n(\mathbf{x})$ characterizes the action of detector $D_n$, when the latter is set to measure the  amplitude of the electric field of light falling on it  (see, e.g., \cite{Rosenfeld_1933,PhysRev.78.794} and \S 9 of \cite{Heitler} for a thorough discussion about measurements of the strength of a quantum field).

In practice, each detector $D_n$ has a limited active surface area. Let $\mathcal{D}_n \in \mathbb{R}^2, ~ (n=1,2,\ldots, M)$ denote the domain in the $xy$-plane occupied by the active surface of detector $D_n$. In principle,  $\mathcal{D}_n$ may have any shape, we only require that $\mathcal{D}_n \cap \mathcal{D}_m = \emptyset$ if $m \neq n$.
%
With $\bm{1}_{n}(\mathbf{x})$ we denote the indicator function of the domain $\mathcal{D}_n$  defined by
%
\begin{align}\label{x10}
%
\bm{1}_{n}(\mathbf{x}) = \left\{
                       \begin{array}{ll}
                         1, & \hbox{for $\mathbf{x} \in \mathcal{D}_n$,} \\[6pt]
                         0, & \hbox{for $\mathbf{x} \not\in  \mathcal{D}_n$.}
                       \end{array}
                     \right.
%
\end{align}
%
Note that, by definition,
%
\begin{align}\label{x20}
%
\bm{1}_{n}(\mathbf{x}) \bm{1}_{m}(\mathbf{x}) =  \delta_{nm} \bm{1}_{n}(\mathbf{x}).
%
\end{align}
%
Then, we can take $f_n(\mathbf{x}) = \bm{1}_{n}(\mathbf{x}) f(\mathbf{x} - \mathbf{x}_n)$, where $f(\mathbf{x} - \mathbf{x}_0)$ is any smooth function concentrated in the neighborhood of $\mathbf{x}_0$, such that
%
\begin{align}\label{supp}
%
\int_{\mathbb{R}^2}  f_n(\mathbf{x} ) \, \mathrm{d} \mathbf{x} & =   \int_{\mathcal{D}_n}  f(\mathbf{x} - \mathbf{x}_n) \, \mathrm{d} \mathbf{x} \nonumber \\[6pt]
%
& \phantom{=} \approx  \int_{\mathbb{R}^2}  f(\mathbf{x} - \mathbf{x}_n) \, \mathrm{d} \mathbf{x}  = 1 .
%
\end{align}
%
With this choice we have $f_n(\mathbf{x}) f_{m}(\mathbf{x}) = 0$ for $n \neq m$.

Using \eqref{a170} with $f = f_n$,  we obtain $M$ smeared fields $\hat{W}_n \deff \hat{\Phi}[f_n]$ at $M$  spatially separated points $\mathbf{x}_1, \mathbf{x}_2, \ldots , \mathbf{x}_M$ in the $xy$-plane.  We can then write
%
\begin{align}\label{a106}
%
\hat{W}_n = \sum_\mu \left( \hat{a}_\mu f_{n\mu}^* + \hat{a}_\mu^\dagger f_{n\mu} \right)/\sqrt{2},
%
\end{align}
%
where $f_{n\mu} = (u_\mu, f_n)$, and
%
\begin{align}\label{n107}
%
\sigma^2_n  = \left(f_n,f_n \right)/2, \qquad (n=1, \ldots, M).
%
\end{align}
%


\subsubsection{Particle-like operators}


We consider now the ``intensity'' operator   $\hat{\mathrm{I}}(\mathbf{x},z)$ defined by  $\hat{\mathrm{I}}(\mathbf{x},z) = \hat{\phi}^\dagger (\mathbf{x},z) \hat{\phi} (\mathbf{x},z)$. This quantity can be interpreted as a photon-number operator per unit transverse surface, because
%
\begin{align}\label{n110}
%
\int_{\mathbb{R}^2} \hat{\mathrm{I}}(\mathbf{x},z) \,  \mathrm{d} \mathbf{x} = \hat{N},
%
\end{align}
%
where $\hat{N}$ is defined by \eqref{n75}.

We then define the photon-counting operator $\hat{C}_n \deff \hat{\mathrm{I}}[\bm{1}_{n}], ~(n=1, \ldots, M)$ representing the action of detector $D_n$ when the latter is set to count the number of photons impinging on it, as
%
\begin{align}\label{a220}
%
\hat{C}_n \deff \hat{\mathrm{I}}[\bm{1}_{n}] = & \; \int_{\mathbb{R}^2}   \bm{1}_{n}(\mathbf{x}) \, \hat{\mathrm{I}}(\mathbf{x},z) \, \mathrm{d} \mathbf{x} \nonumber \\[6pt]
%
= & \; \sum_{\mu, \nu} \hat{a}_\mu^\dagger \hat{a}_\nu \bm{1}_{n\mu \nu},
%
\end{align}
%
where
%
\begin{align}\label{a222}
%
\bm{1}_{n \mu \nu} = (u_\mu, \bm{1}_{n} u _\nu).
%
\end{align}
%
By diagonalizing the linear operator whose matrix elements are $\bm{1}_{n \mu \nu}$, it is not difficult to find the discrete eigenvalues and eigenvectors of $\hat{C}_n$, but it is not necessary to do this. We need only to note that
by definition the \emph{discrete spectrum} of $\hat{C}_n$ gives the number of photons counted by detector $D_n$.   Note also that $\hat{C}_n$ is dimensionless.


\subsection{Commutation relations}

In the remainder we will need to use commutation relations for the wave and photon-counting operators $\hat{W}_n$ and $\hat{C}_n$. Such relations are calculated in Appendix \ref{commrel}, and the results are:
%
\begin{align}
%
\bigl[ \hat{W}_m, \, \hat{W}_n \bigr] &  = 0, \label{com10} \\[6pt]
%
\bigl[ \hat{C}_m, \, \hat{C}_n \bigr] &  = 0, \label{com20} \\[6pt]
%
\bigl[ \hat{W}_m, \, \hat{C}_n \bigr] &  = \frac{\delta_{nm}}{\sqrt{2}} \left\{ \bigl(f_n, \hat{\phi}  \bigr) - \bigl(\hat{\phi}, f_n \bigr) \right\}, \label{com30}
%
\end{align}
%
where $m,n = 1,2, \ldots, M$. Since all commutators above are zero for $m \neq n$, then all the wave an particle observables associated with different detectors are compatible and can be measured simultaneously.


\section{Probability distributions}\label{pd}


In random variable theory, the probability distribution or probability density function (p.d.f.) $p_\mathbf{Q}(\mathbf{q} )$ of a $M$-dimensional  random variable $\mathbf{Q} = (Q_1, Q_2, \ldots, Q_M)$, can be written as $p_\mathbf{Q}(\mathbf{q} ) = \langle \delta (\mathbf{Q} - \mathbf{q})\rangle$, where $\langle \cdots \rangle$ denotes average over all possible realization of $\mathbf{Q}$, and
%
\begin{align}\label{n130}
%
\delta (\mathbf{Q} - \mathbf{q}) = \prod_{n=1}^M \delta (Q_n - q_n),
%
\end{align}
%
 \cite{Ramshaw_1985}. Similarly, in quantum mechanics the spectral theorem \cite{aiello_arXiv.2110.12930} shows that given a Hermitian operator $\hat{Q}$ and a vector state $| \psi \rangle$ of norm $1$, we can  calculate the expectation value of any regular function $F(\hat{Q}) $ of $\hat{Q}$ with respect to $| \psi \rangle$, either as $\langle F(\hat{Q}) \rangle  = \langle \psi | F(\hat{Q}) | \psi \rangle$, or as
%
\begin{align}\label{n140}
%
\langle F(\hat{Q}) \rangle  = \int_{\mathbb{R}}  F(q) \, p_Q (q) \, \mathrm{d} q ,
%
\end{align}
%
where the p.d.f. $p_Q (q)$ of the random variable $Q$ associated with the operator $\hat{Q}$, is defined by
%
\begin{align}\label{n150}
%
p_Q (q)  = \langle \psi | \delta \bigl( \hat{Q} - q \bigr) | \psi \rangle,
%
\end{align}
%
(see, e.g., sec. 3-1-2 in \cite{ItzZub},  problem 4.3 in \cite{Coleman2019}, or  \cite{aiello_arXiv.2110.12930}). Using the Fourier representation of the Dirac delta function $\delta(z) = \int_\mathbb{R}  e^{i \alpha z} \, { \mathrm{d} \alpha }\, / (2 \pi)$, it is straightforward to show that
%
\begin{align}\label{a190}
%
p_Q (q) = \frac{ 1}{2 \pi} \int_\mathbb{R}  \langle \psi | e^{i \alpha \hat{Q}} | \psi \rangle e^{-i \alpha q}  \, \mathrm{d} \alpha  ,
%
\end{align}
%
where $\langle \psi | \exp({i \alpha \hat{Q}}) | \psi \rangle$ is the so-called quantum characteristic function \cite{MANKO1998328}.


\section{Measuring the wave-like and particle-like aspects of light}\label{wp}


Consider three different experiments where a light beam prepared in the $N$-photon Fock state $| N[\phi] \rangle$ impinges upon a screen where two detectors are placed at two spatially separated points in the $xy$-plane, as shown in Fig. \ref{fig1}.
%
In the first experiment the detectors are set up to measures the  amplitudes $\mathcal{W}_1$ and $\mathcal{W}_2$ of the electric field of the light falling on them.
%
In the second experiment the detectors are set up to count the number of photons $\mathcal{C}_1$ and $\mathcal{C}_2$. Finally, in the third and last experiment detector $D_1$  measures the  amplitude $\mathcal{W}_1$, and detector $D_2$ counts the number of photons $\mathcal{C}_2$.

The outcomes of these experiments can be described by the three pairs of  random variables $(W_1,W_2), ~ (C_1,C_2)$ and $(W_1,C_2)$, distributed according to
%
\begin{align}
%
(W_1,W_2) & \sim  p_{W_1 W_2}(N,w_1,w_2) \deff  p_\mathbf{W}(N,\mathbf{w}), \nonumber \\[6pt]
%
(C_1,C_2) & \sim  p_{C_1 C_2}(N,c_1,c_2) \deff  p_\mathbf{C}(N,\mathbf{c}), \nonumber
%
%
\end{align}
%
and
%
\begin{align}
%
(W_1,C_2) & \sim  p_{W_1 C_2}(N,w_1,c_2) \deff  p_{W C}(N,w,c), \nonumber
%
%
\end{align}
%
where from Sec. \ref{pd}, we have
%
\begin{align}
%
p_\mathbf{W}(N,\mathbf{w})  & = \langle N[\phi] | \!  \prod_{n = 1}^M \delta \bigl( \hat{W}_n - w_n \bigr) | N[\phi] \rangle, \label{a200} \\[6pt]
%
p_\mathbf{C}(N,\mathbf{c})  & = \langle N[\phi] | \prod_{n = 1}^M \delta \bigl( \hat{C}_n - c_n \bigr) | N[\phi] \rangle, \label{a210}
%
\end{align}
%
and
%
\begin{multline}\label{a215}
%
p_{WC}(N,w, c)  \\[6pt] = \langle N[\phi] | \delta \bigl( \hat{W}_1 - w \bigr)  \delta \bigl( \hat{C}_2 - c \bigr) | N[\phi] \rangle,
%
\end{multline}
%
where $|N[\phi] \rangle$ is defined by  \eqref{n70}.
These three p.d.f.s are calculated in Appendices \ref{pdfW}, \ref{pdfC} and \ref{pdfWC}.
Note that there is not an operator ordering problem in the definitions \eqref{a200}-\eqref{a215} because all the operators involved do commute.


\subsection{Single detector}


Here we write down \eqref{a200} and \eqref{a210} for the case of a single detector, that is $M = 1$. Of course, \eqref{a215} is simply not defined in this case.


\subsubsection{Vacuum state}


In the simplest case  of input vacuum state, that is $N=0$, a straightforward calculation gives
%
\begin{align}
%
p_W (0,w)  = & \; \bigl( 2 \pi \sigma^2 \bigr)^{-1/2} \exp \left( -\frac{w^2}{2 \sigma^2} \right) , \label{a230} \\[6pt]
%
p_C (0,c)  = & \; \delta(c), \label{a235}
%
\end{align}
%
where as in \eqref{n105} $\sigma^2 = (f,f)/2$ fixes the variance of the smeared field $\hat{W}$ in the vacuum state.
As expected, $p_W (0,w)$ and $p_C (0,c)$ are the p.d.f.s of a continuous and a discrete random variable, respectively.

Equation \eqref{a230} shows a well-know result from the quantum theory of \emph{free fields}: the field amplitude in the ground (vacuum) state follows a Gaussian probability distribution that is centred on the value zero. The quantum field fluctuations are fixed by the smearing function via $\sigma^2$.
For example, if $f(\mathbf{x})$ is given by \eqref{a172}, then $\sigma^2 = 1/(4 \pi a^2)$. This implies that when the linear dimension $a$ of the region in which the field amplitude is measured shrinks to zero, the quantum fluctuations become huge, eventually becoming infinite for $a \to 0$ \cite{Coleman2019}.

Equation \eqref{a235} gives the trivial p.d.f. of a discrete-type random variable with a probability mass function that takes a single value: $\operatorname{Prob}(C = 0) = 1$. Physically this means that in the vacuum state of the electromagnetic field, the probability to count one or more photons is equal to zero, as it should be.


\subsubsection{Single-photon state}


Next, we consider the probability distributions with respect to the single-photon state ($N=1$). From \eqref{s530} and \eqref{s720} with $M=1$, we have
%
\begin{align}
%
p_W (1,w)  & =  p_W (0,w) \left[ 1 - |s|^2\left(   1 -   \frac{w^2}{\sigma^2}  \right) \right], \label{a240} \\[6pt]
%
p_C (1,c)  & =   ( 1-P ) \, \delta(c) + P \, \delta(c-1) , \label{a250}
%
\end{align}
%
where $p_W (0,w)$ is given by \eqref{a230}, 
%
\begin{align}\label{n200}
%
s = \frac{(\phi,f)}{(f,f)^{1/2}} , \quad \text{and} \quad P = (\phi, \bm{1} \phi) \geq 0,
%
\end{align}
%
with $\bm{1}(\mathbf{x})$  denoting the indicator function of the domain $\mathcal{D}$ representing the active area of the single detector, and $f$ is the smearing function.

Equation \eqref{a240} shows that when the superposition $s$ between the single-photon field $\phi$ and the smearing function $f$ is zero, then $p_W (1,w)  =  p_W (0,w)$. This is obvious. However, if the  section of the beam on the detecting screen coincides with detector's active area so that $|s|=1$,  we obtain
%
\begin{align}\label{n202}
%
p_W (1,w)  =   p_W (0,w)  \frac{w^2}{\sigma^2} .
%
\end{align}
%
This kind of distribution is known as the  Maxwell distribution of speeds in statistical physics when $w \geq 0$ (see, e.g., Sec. 7.10 in \cite{reif2009}). From \eqref{a240} it is easy to calculate
%
\begin{align}\label{n203}
%
\operatorname{E}[W]=0, \quad \text{and} \quad \operatorname{E}[W^2]= \sigma^2 (1 + 2 |s|^2).
%
\end{align}
%
The main feature of the distribution \eqref{n202} is that differently from the vacuum distribution $p_W (0,w)$, we have now $p_W (1,0) = 0$. More generally, from \eqref{n200} it follows that $p_W (1,0) = p_W (0,0)(1 - |s|^2)$, with $1 - |s|^2 \leq 1$ from \eqref{n200} and the Cauchy–Schwarz inequality. This $s$-dependent dip at $w=0$, is the signature of the single-photon state with respect to the vacuum state, as illustrated by Fig. \ref{fig2}.
%
\begin{figure}[h!]
  \centering
  \includegraphics[scale=3,clip=false,width=1\columnwidth,trim = 0 0 0 0]{fig_2.pdf}
  \caption{Plots of $p_W (1,w)$ given by \eqref{a240} for different values of $|s| \in [0,1]$. The blue dot-dashed curve for $|s|=0$ is equal to the p.d.f. for the vacuum state $p_W (0,w)$. When the superposition between the cross section of the beam and the detector surface increases, the central value  $p_W (1,0)$ decreases to zero. Note that from \eqref{a240} it follows that all the curves plotted here intersect at $w=\sigma$.}\label{fig2}
\end{figure}
%

The p.d.f. \eqref{a250} of the discrete random variable $C$ gives $\operatorname{Prob}(C = 0) = 1 - P$, and $\operatorname{Prob}(C = 1) = P$, where $P$ is given by \eqref{n200}, here rewritten as
%
\begin{align}\label{p10}
%
P = \int_{\mathcal{D}} \left| \phi(\mathbf{x},z) \right|^2 \, \mathrm{d} \mathbf{x}.
%
\end{align}
%
This expression shows that $P$ is equal to the fraction of the intensity $\left| \phi(\mathbf{x},z) \right|^2$ of the incident beam that falls upon the detector surface. This makes sense because from \eqref{a250} it follows that
%
\begin{align}\label{p12}
%
\operatorname{E}[C] = P = \operatorname{E}[C^2].
%
\end{align}
%
Therefore, the variance $\sigma_C^2$ of $C$ is
%
\begin{align}\label{p14}
%
\sigma_C^2 = \operatorname{E}[C^2] - (\operatorname{E}[C])^2 = P(1-P).
%
\end{align}
%


\subsection{Two detectors}

When there are two detectors located in two different positions in the $xy$-plane, as shown in Fig. \ref{fig1}, we can choose between three different possibilities of detection: \emph{a}) wave-wave detection; \emph{b}) particle-particle detection; and \emph{c}) wave-particle detection. In the remainder of this section  we will analyze in detail these three cases.


\subsubsection{Case a). Wave-wave detection}


For the vacuum state and $M=2$, \eqref{s350}  gives
%
\begin{align}\label{n220}
%
p_\mathbf{W}  (0,\mathbf{w})   =   p_{W} (0,w_1) \, p_{W} (0,w_2) ,
%
\end{align}
%
where $ p_{W} (0,w)$ is  defined by \eqref{a230}. Thus, the joint p.d.f. is the product of the marginal probability distributions $ p_{W} (0,w_1)$ and $p_{W} (0,w_2)$. Therefore, the two random variables $W_1$ and $W_2$ defined by \emph{both} the operators $\hat{W}_1, ~\hat{W}_2$ and the quantum vacuum state $| 0 \rangle$, are independent.


For the single-photon state $|1[\phi] \rangle$, from \eqref{s530} with $M=2$, we get
%
\begin{widetext}
%
\begin{align}\label{a270}
%
p_\mathbf{W} (1,\mathbf{w})    =  p_\mathbf{W}  (0,\mathbf{w})  \left[  1 - |s_1|^2 \left( 1 - \frac{w_1^2}{\sigma_1^2} \right)   - |s_2|^2 \left( 1 - \frac{w_2^2}{\sigma_2^2} \right) + \frac{w_1 w_2}{\sigma_1 \sigma_2} \left( s_1 s_2^* + s_1^* s_2\right)  \right],
%
\end{align}
%
\end{widetext}
%
where $p_\mathbf{W}  (0,\mathbf{w}) $ is given by \eqref{n220}, and from \eqref{n107} we have 
%
\begin{align}\label{n290}
%
\sigma_n^2 = \frac{(f_n,f_n)}{2},
%
\end{align}
%
and
%
\begin{align}\label{n292}
%
s_n = \frac{(\phi,f_n)}{(f_n,f_n)^{1/2}} =\frac{(\phi,f_n)}{\sqrt{2} \,\sigma_n},
%
\end{align}
%
where $n = 1,2$.

Equation \eqref{a270} is what we  expect from \eqref{a240}, plus a mixed term proportional to $w_1 w_2$. The latter is responsible for the wave-wave detection correlation. This can be  seen by calculating the average $\operatorname{E}[W_n]=0$,  the variance
%
\begin{align}\label{ww10}
%
\sigma_{W_n}^2 \deff \operatorname{E}[W_n^2]= \sigma_n^2 (1 + 2 |s_n|^2),
%
\end{align}
%
with $n=1,2$, and the covariance
%
\begin{align}\label{ww20}
%
\operatorname{E}[W_1 W_2] = \sigma_1 \sigma_2 \left(s_1 s_2^* + s_1^* s_2 \right).
%
\end{align}
%

If we choose the smearing functions $f_1$ and $f_2$ and the location of the two detectors such that $\sigma_1=\sigma_2 \deff \sigma$ and $s_1=s_2 \deff s$, then using \eqref{ww10}-\eqref{ww20} we find a correlation coefficient of the form,
%
\begin{align}\label{ww30}
%
\frac{\operatorname{E}[W_1 W_2] - \operatorname{E}[W_1] \operatorname{E}[W_2]}{ \sigma_{W_1}  \sigma_{W_2}} & = \frac{2 \,  |s |^2}{1 + 2 \,  |s |^2} \nonumber \\[6pt]
%
& \leq \frac{1}{2},
%
\end{align}
%
because $|s|^2 \leq 1/2$.
A positive correlation coefficient between $W_1$ and $W_2$ means that when $W_1$ increases then $W_2$ also increases, and when $W_1$ decreases then $W_2$ also decreases. The minimum value $0$ of the correlation coefficient \eqref{ww30} is attained when $s=0$. This may occur in two different ways: either \emph{a}) both detectors have finite active surface but are located outside the section of the beam on the detection screen, or \emph{b}) the detectors are placed within the section of the beam, but they are point-like detectors with zero-size active surface. Case  \emph{a}) is trivial  and implies $p_\mathbf{W} (1,\mathbf{w})    =  p_\mathbf{W}  (0,\mathbf{w})$. Case \emph{b}) is more interesting because it shows that the amplitudes of the field measured at any pair of points are always uncorrelated. This is due to the fact that the field undergoes  enormous quantum fluctuations ($\sigma^2 \to \infty$), when strongly localized. To see this, let us take  $f(\mathbf{x})$ as in \eqref{a172}, so that $(f,f) = 1/(2 \pi a^2) = 2 \sigma^2$. Then, to achieve $s_1=s_2=s$ we must assume that  the two point-like detectors are located at  $\mathbf{x}_1$ and $\mathbf{x}_2$ chosen in such a way that $\phi(\mathbf{x}_1,z) = \phi(\mathbf{x}_2,z)$.  In this case from \eqref{n290} and $a \approx 0$, it follows that
%
\begin{align}\label{ww35}
%
|s|^2 & = \frac{|(\phi,f)|^2}{2 \, \sigma^2} \nonumber \\[6pt]
%
& \approx \sqrt{2 \pi} \, a \, |\phi(\mathbf{x}_1,z)|^2.
%
\end{align}
%
Since $|\phi(\mathbf{x}_1,z)|^2$ is always a finite quantity for any physically realisable light beam, then $|s|^2 \to 0$ when the size $a$ of the detectors goes to zero.

Not surprisingly, as in the single-detector case  from \eqref{a270} it follows that  $p_\mathbf{W} (1,\mathbf{0})    =  p_\mathbf{W}  (0,\mathbf{0})  (  1 - |s_1|^2   - |s_2|^2) $, with $1 - |s_1|^2   - |s_2|^2 \leq 1$, at $w_1=0=w_2$. This behavior is illustrated in Fig. \ref{fig3} below.
%
\begin{figure}[ht!]
  \centering
  \includegraphics[scale=3,clip=false,width=1\columnwidth,trim = 0 0 0 0]{fig_3.pdf}
  \caption{Plots of $p_\mathbf{W} (1,\mathbf{w})$ given by \eqref{a270} for $s_1 = s_2 = s \in \mathbb{R}$, and: \textsf{a)} $s=0$, (equal to the p.d.f. for the vacuum state);  \textsf{b)} $s=0.2$; \textsf{c)} $s=0.5$; \textsf{d)} $s=0.7$. The maximum permitted value for $s$ is $s = 1/\sqrt{2} \approx 0.71$. }\label{fig3}
\end{figure}
%

It is interesting to note that when either $s_1=0$ or $s_2 = 0$, the joint p.d.f. $p_\mathbf{W} (1,\mathbf{w})$ is factorized as either $p_\mathbf{W} (1,\mathbf{w}) = p_{W_1}  (0,w_1)p_{W_2}  (1,w_2)$, or $p_\mathbf{W} (1,\mathbf{w}) = p_{W_1}  (1,w_1)p_{W_2}  (0,w_2) $, respectively, so that the two random variables $W_1$ and $W_2$ become independent. However, when both $s_1 \neq 0$ and $s_2 \neq 0$,  $p_\mathbf{W} (1,\mathbf{w})$ does not factorize as $p_\mathbf{W} (1,\mathbf{w})= p_{W_1}  (1,w_1)p_{W_2}  (1,w_2)$,  and $W_1$ and $W_2$ are not independent although the two corresponding operators $\hat{W}_1$ and $\hat{W}_2$ do commute. This is a consequence of the  non-localizability of the single-photon field $\phi(\mathbf{x},z)$, whose section  in the $xy$-plane extends over both regions $\mathcal{D}_1$ and $\mathcal{D}_2$ occupied by the active surfaces of the two detectors $D_1$ and $D_2$. In fact, as shown in \cite{aiello_arXiv.2110.12930}, the joint p.d.f. $p_\mathbf{W} (1,\mathbf{w})$ is  determined by \emph{both} the operators $\hat{W}_1, ~\hat{W}_2$ \emph{and} the quantum state $|1[\phi]\rangle$. In this case, it is the spatial transverse extension of the light field $\phi(\mathbf{x},z)$ that establishes a correlation between the two random variables $W_1$ and $W_2$.


\subsubsection{Case b): particle-particle detection}\label{pp}


For the vacuum state and $M=2$, \eqref{s550}  gives
%
\begin{align}\label{n230}
%
p_\mathbf{C}  (0,\mathbf{c})  =   p_{C} (0,c_1) \, p_{C} (0,c_2),
%
\end{align}
%
where $ p_{C} (0,c)$ is defined by \eqref{a235}. This equation simply shows that in the vacuum state we always count zero photons.

More interesting is the single-photon state case, for which from  \eqref{s720} we get 
%
\begin{widetext}
%
\begin{align}\label{a280}
%
p_\mathbf{C} (1,\mathbf{c})    =  (1 - P_1 - P_2) \, \delta(c_1) \, \delta(c_2)  + P_1 \, \delta(c_1-1) \, \delta(c_2) +  P_2 \, \delta(c_1) \, \delta(c_2-1) ,
%
\end{align}
%
\end{widetext}
%
where
%
\begin{align}\label{n290bis}
%
P_n = (\phi, \bm{1}_n \phi) \geq 0, \qquad (n = 1,2),
%
\end{align}
%
is the fraction of the intensity of the beam impinging upon the $n \text{th}$ detector. A straightforward calculation shows that \eqref{a280} yields the correct marginal distributions $p_{C_1} (1,c_1)$ and $p_{C_2} (1,c_2)$,  defined by \eqref{a250}. Moreover, it is not difficult to calculate
%
\begin{align}\label{cc10}
%
\operatorname{E}[C_n^k]= P_n, \qquad (k \in \mathbb{N}),
%
\end{align}
%
with $n=1,2$, and
%
\begin{align}\label{cc20}
%
\operatorname{E}[C_1 C_2] = 0.
%
\end{align}
%
The latter two equations imply for the correlation coefficient,
%
\begin{align}\label{cc30}
%
\frac{ \operatorname{E}[C_1 C_2] - \operatorname{E}[C_1] \operatorname{E}[C_2]}{\sigma_{C_1} \sigma_{C_2}} &  = - \sqrt{\frac{P_1 P_2}{(1-P_1)(1-P_2)}}\nonumber \\[6pt]
%
& \geq -1 ,
%
\end{align}
%
where \eqref{p14} has been used.
A negative covariance means that there is an inverse relationship between the random variables $C_1$ and $C_2$. In physical terms this implies that  when the number of photons counted by $D_1$ increases, the one counted by $D_2$ must decrease, and vice versa. This is a consequence of both the fixed  the number of photons in Fock states and  the non-localizability of the electromagnetic field. Differently from \eqref{ww30}, here the absolute value of the correlation coefficient can achieve the maximum value $1$, which means perfect correlation between $C_1$ and $C_2$.


\subsubsection{Case c): wave-particle detection}\label{wpd}


This is the last and more interesting case. For the vacuum state \eqref{d20}  gives
%
\begin{align}\label{wp230}
%
p _{W\!C}  (0,w,c) =  p_{W}(0,w) \, p_{C}(0,c),
%
\end{align}
%
as expected. This result is very simple and there is not much to say about it.

Conversely, for the single-photon state from \eqref{d60} we have,
%
\begin{align}\label{wp10}
%
p _{W\!C}  (1,w,c) &  = p_{W}(1,w) \, p_{C}(0,c)\nonumber \\[6pt]
%
& \phantom{=} + p_{W}(0,w) \, p_{C}(1,c) \nonumber \\[6pt]
%
& \phantom{=}  - p_{W}(0,w) \, p_{C}(0,c),
%
\end{align}
%
where \eqref{a230}-\eqref{a250} have been used. This is a continuous-discrete mixed-type distribution.
%
A straightforward calculation gives
%
\begin{align}\label{wp20}
%
\operatorname{E}[W] = 0, \qquad   \operatorname{E}[C] = P,
%
\end{align}
%
and
%
\begin{align}
%
\operatorname{E}[W^2] & = \sigma^2 \left( 1+ 2 |s|^2\right), \label{wp30} \\[6pt]
%
\operatorname{E}[C^2] & = P, \label{wp40} \\[6pt]
%
\operatorname{E}[W C] & = 0. \label{wp50}
%
\end{align}
%
This implies
%
\begin{align}
%
\sigma_W & = \sqrt{\operatorname{E}[W^2] - (\operatorname{E}[W])^2} = \sigma^2 ( 1+ 2 |s|^2 ), \label{wp52A} \\[6pt]
%
\sigma_C & = \sqrt{\operatorname{E}[C^2] - (\operatorname{E}[C])^2} = \sqrt{P(1-P)}, \label{wp52B}
%
\end{align}
%
with $0 \leq P \leq 1$.

From \eqref{wp20}-\eqref{wp52B} it follows that the correlation coefficient of $W$ and $C$ is zero:
%
\begin{align}\label{wp55}
%
 \frac{\operatorname{E}[W C] - \operatorname{E}[W] \operatorname{E}[C]}{\sigma_W \, \sigma_C}  = 0 .
%
\end{align}
%
This tells us that there is no \emph{linear} dependence between  the wave and particle random variables $W$ and $C$. However, from \eqref{wp10} it follows that
%
\begin{align}\label{wp60}
%
p _{W\!C}  (1,w,c) \neq  p_{W}(1,w) \, p_{C}(1,c),
%
\end{align}
%
which implies that $W$ and $C$ are \emph{not} independent.  The key point is that the correlation coefficient measures only the \emph{linear} dependence between $W$ and $C$ \cite{MurphyBook2022}. When the relation between two random variables is nonlinear, a more suitable measure of dependence is given, for example, by the mutual information \cite{CoverThomas}.
%
Mutual information is one of the several measures of the dependence between two
random variables that are sampled simultaneously\footnote{We remind that random variables associated with commuting quantum operators can always be sampled simultaneously \cite{aiello_arXiv.2110.12930}.}.
%
In practice, mutual information quantifies the reduction in the average uncertainty about one random variable given the knowledge of another.
Thus, large values of mutual information indicate high reduction in uncertainty; small values of mutual information denote low reduction; and zero mutual information  means that the two random variables are independent.
%
We stress that here the term ``uncertainty'' is referring to the values of the random variables $W$ and $C$. This uncertainty should not be confused with the well-known Heisenberg uncertainty, which refers to non-compatible, conjugate observables, i.e., to physical quantities that cannot be measured simultaneously. Clearly, for conjugate  observables a joint probability distribution cannot be calculated and, consequently, mutual information cannot be defined.\footnote{However,  using Wigner's functions and \emph{linear entropy}, a different form of mutual information can be defined also for non-compatible observables \cite{PhysRevE.62.4665,SANTOS2021125937}.}

For a mixture of discrete and continuous variables the mutual information can be written as \cite{nair2007entropy},
%
\begin{align}\label{wp70}
%
\operatorname{I}(W;C)   = \operatorname{h}(W) + \operatorname{H}(C) - \mathcal{H}(W,C ),
%
\end{align}
%
where we have defined the continuous (differential), discrete and mixed entropies, $\operatorname{h}(W)$, $\operatorname{H}(C)$, and  $\mathcal{H}(W,C)$, respectively, as
%
\begin{align}
%
\operatorname{h}(W) & = -\int_\mathbb{R} p_W(1,w) \ln \bigl[ p_W(1,w) \bigr] \, \text{d}w, \label{wp80} \\[6pt]
%
\operatorname{H}(C) &  = -P \ln P - (1-P) \ln (1-P), \label{wp90}
%
\end{align}
%
and
%
\begin{align}\label{wp100}
%
\mathcal{H}(W,C)   = - \sum_{i=0}^1 \int_\mathbb{R} g_i(w) \ln \bigl[g_i(w) \bigr] \, \text{d} w.
%
\end{align}
%
The two functions $g_0(w)$ and $g_1(w)$ are defined by rewriting $p _{W\!C}  (1,w,c)$ as
%
\begin{align}\label{wp110}
%
p _{W\!C}  (1,w,c) & = \delta(c) \bigl[ p_W(1,w)- P \, p_W(0,w) \bigr] \nonumber \\[6pt]
%
& \phantom{=} + \delta(c-1) P \, p_W(0,w) \nonumber \\[6pt]
%
& \deff \delta(c)\, g_0(w) \nonumber \\[6pt]
%
& \phantom{=} +  \delta(c-1) \, g_1(w) .
\end{align}
%
The quantities in \eqref{wp80}-\eqref{wp100} can be calculated explicitly, for example by using Mathematica \cite{Mathematica}. We do not write down the formulae here as they are very complicated and not particularly enlightening. However, we plot $\operatorname{I}(W;C)$ in Fig. \ref{fig4}.
%
\begin{figure}[ht!]
  \centering
  \includegraphics[scale=3,clip=false,width=1\columnwidth,trim = 0 0 0 0]{fig_4.pdf}
  \caption{Plot of $\operatorname{I}(W;C)$ given by \eqref{wp70}. Note that the domain of the function is the region of the $sP$-plane defined by $1-P -|s|^2 \geq 0$ (gray area in the plot). This condition simply implies that the superposition between the section of the light beam on the $xy$-plane, and the surfaces of the two detectors, cannot exceed the superposition of the beam with itself.  For point-like detectors  we have $P , |s|^2 \ll 1$, so that  $\operatorname{I}(W;C) \approx P |s|^4/(1-P)$.}\label{fig4}
\end{figure}
%
The existence of a nonzero mutual information witnesses the presence of a nonlinear relationship between the wave and the particle observables $\mathcal{W}$ and $\mathcal{C}$, respectively. This is the main result of this paper.

The maximum value of the mutual information is achieved for $P = 1 - |s|^2 \approx 0.47$, and it is given   $\max [\operatorname{I}(W;C)] \approx 0.18$. To understand what this number means, we compare  $\operatorname{I}(W;C)$ with $\operatorname{I}(C_1;C_2)$, which is given by
%
\begin{align}\label{wp120}
%
\operatorname{I}(C_1;C_2) & = -(1-P_1)\ln(1-P_1) \nonumber \\[6pt]
%
& \phantom{=} -(1-P_2)\ln(1-P_2)\nonumber \\[6pt]
%
& \phantom{=} + (1-P_1-P_2)\ln(1-P_1-P_2).
%
\end{align}
%
In this case we have $\max [\operatorname{I}(C_1;C_2)] = \ln 2$, which is the maximum attainable value by the mutual information of two dichotomic discrete random variables. Therefore,
%
\begin{align}\label{wp130}
%
\frac{\max [\operatorname{I}(W;C)]}{\max [\operatorname{I}(C_1;C_2)]} \approx 0.26 \sim \frac{1}{3}.
%
\end{align}
%
This result could be very roughly interpreted by saying that if we had used correlation functions to analyse our experiment instead of mutual information, we would have lost, on average, one third of the information about the connection between wave and particle observables.


\section{Conclusions}\label{conc}


In this paper we have described a simple experiment that illustrates the so-called wave-particle duality phenomenon in single-photon states.
%
Unlike more conventional approaches,   we have adopted a probabilistic framework which permitted us to detect a somewhat ``hidden'' nonlinear dependence between some wave and particle observables that are actually measured in the experiment.
%
We believe that this work can promote the use of these probabilistic techniques in various applications of quantum optics.


\section*{Acknowledgements}

I acknowledge support from the Deutsche Forschungsgemeinschaft Project No. 429529648-
TRR 306 QuCoLiMa (``Quantum Cooperativity of Light and Matter'').



\begin{thebibliography}{37}
\expandafter\ifx\csname natexlab\endcsname\relax\def\natexlab#1{#1}\fi
\expandafter\ifx\csname bibnamefont\endcsname\relax
  \def\bibnamefont#1{#1}\fi
\expandafter\ifx\csname bibfnamefont\endcsname\relax
  \def\bibfnamefont#1{#1}\fi
\expandafter\ifx\csname citenamefont\endcsname\relax
  \def\citenamefont#1{#1}\fi
\expandafter\ifx\csname url\endcsname\relax
  \def\url#1{\texttt{#1}}\fi
\expandafter\ifx\csname urlprefix\endcsname\relax\def\urlprefix{URL }\fi
\providecommand{\bibinfo}[2]{#2}
\providecommand{\eprint}[2][]{\url{#2}}

\bibitem[{\citenamefont{{Herbert Goldstein} et~al.}(2001)\citenamefont{{Herbert
  Goldstein}, {Charles Poole}, and {John Safko}}}]{GoldsteinBook}
\bibinfo{author}{\bibnamefont{{Herbert Goldstein}}},
  \bibinfo{author}{\bibnamefont{{Charles Poole}}}, \bibnamefont{and}
  \bibinfo{author}{\bibnamefont{{John Safko}}}, \emph{\bibinfo{title}{Classical
  Mechanics}} (\bibinfo{publisher}{Pearson}, \bibinfo{year}{2001}),
  \bibinfo{edition}{3rd} ed., ISBN \bibinfo{isbn}{9780201657029}.

\bibitem[{\citenamefont{{Fetter, Alexander L.} and {Walecka, John
  Dirk}}(2003)}]{FetterBook}
\bibinfo{author}{\bibnamefont{{Fetter, Alexander L.}}} \bibnamefont{and}
  \bibinfo{author}{\bibnamefont{{Walecka, John Dirk}}},
  \emph{\bibinfo{title}{Theoretical Mechanics of Particles and Continua}}
  (\bibinfo{publisher}{Dover Publications, Inc.}, \bibinfo{address}{Mineola,
  New York}, \bibinfo{year}{2003}), ISBN \bibinfo{isbn}{978-0486432618}.

\bibitem[{\citenamefont{{Kimball A. Milton} and {Julian
  Schwinger}}(2006)}]{SchwingerMilton}
\bibinfo{author}{\bibnamefont{{Kimball A. Milton}}} \bibnamefont{and}
  \bibinfo{author}{\bibnamefont{{Julian Schwinger}}},
  \emph{\bibinfo{title}{Electromagnetic Radiation: Variational Methods,
  Waveguides and Accelerators}} (\bibinfo{publisher}{Springer Berlin,
  Heidelberg}, \bibinfo{year}{2006}), ISBN \bibinfo{isbn}{978-3-540-29304-0},
  \urlprefix\url{https://link.springer.com/book/10.1007/3-540-29306-X}.

\bibitem[{\citenamefont{{Alberto Galindo} and {Pedro
  Pascual}}(1990)}]{GalindoI}
\bibinfo{author}{\bibnamefont{{Alberto Galindo}}} \bibnamefont{and}
  \bibinfo{author}{\bibnamefont{{Pedro Pascual}}},
  \emph{\bibinfo{title}{Quantum Mechanics I}}, {Texts and Monographs in Physics
  (TMP)} (\bibinfo{publisher}{Springer Berlin, Heidelberg},
  \bibinfo{year}{1990}), ISBN \bibinfo{isbn}{978-3-642-83856-9},
  \urlprefix\url{https://link.springer.com/book/10.1007/978-3-642-83854-5}.

\bibitem[{\citenamefont{Englert}(1996)}]{PhysRevLett.77.2154}
\bibinfo{author}{\bibfnamefont{B.-G.} \bibnamefont{Englert}},
  \bibinfo{journal}{Phys. Rev. Lett.} \textbf{\bibinfo{volume}{77}},
  \bibinfo{pages}{2154} (\bibinfo{year}{1996}),
  \urlprefix\url{https://link.aps.org/doi/10.1103/PhysRevLett.77.2154}.

\bibitem[{\citenamefont{Dimitrova and Weis}(2008)}]{Dimitrova2008}
\bibinfo{author}{\bibfnamefont{T.~L.} \bibnamefont{Dimitrova}}
  \bibnamefont{and} \bibinfo{author}{\bibfnamefont{A.}~\bibnamefont{Weis}},
  \bibinfo{journal}{American Journal of Physics} \textbf{\bibinfo{volume}{76}},
  \bibinfo{pages}{137} (\bibinfo{year}{2008}),
  \urlprefix\url{https://doi.org/10.1119/1.2815364}.

\bibitem[{\citenamefont{Scarani and Suarez}(1998)}]{Scarani1998}
\bibinfo{author}{\bibfnamefont{V.}~\bibnamefont{Scarani}} \bibnamefont{and}
  \bibinfo{author}{\bibfnamefont{A.}~\bibnamefont{Suarez}},
  \bibinfo{journal}{American Journal of Physics} \textbf{\bibinfo{volume}{66}},
  \bibinfo{pages}{718} (\bibinfo{year}{1998}),
  \eprint{https://doi.org/10.1119/1.18938},
  \urlprefix\url{https://doi.org/10.1119/1.18938}.

\bibitem[{\citenamefont{Vedral}(2021)}]{vedral2021quantum}
\bibinfo{author}{\bibfnamefont{V.}~\bibnamefont{Vedral}},
  \emph{\bibinfo{title}{The quantum double slit experiment with local elements
  of reality}} (\bibinfo{year}{2021}), \eprint{arXiv:2104.11333},
  \urlprefix\url{https://arxiv.org/abs/2104.11333}.

\bibitem[{\citenamefont{von Neumann}(2018)}]{vonNeumann2018}
\bibinfo{author}{\bibfnamefont{J.}~\bibnamefont{von Neumann}},
  \emph{\bibinfo{title}{Mathematical Foundations of Quantum Mechanics}}
  (\bibinfo{publisher}{Princeton University Press},
  \bibinfo{address}{Princeton}, \bibinfo{year}{2018}), ISBN
  \bibinfo{isbn}{9781400889921},
  \urlprefix\url{https://doi.org/10.1515/9781400889921}.

\bibitem[{\citenamefont{Aiello}(2022)}]{aiello_arXiv.2110.12930}
\bibinfo{author}{\bibfnamefont{A.}~\bibnamefont{Aiello}},
  \emph{\bibinfo{title}{Spectral theorem for dummies: A pedagogical discussion
  on quantum probability and random variable theory}} (\bibinfo{year}{2022}),
  \urlprefix\url{https://arxiv.org/abs/2211.12742}.

\bibitem[{\citenamefont{Loudon}(2000)}]{LoudonBook}
\bibinfo{author}{\bibfnamefont{R.}~\bibnamefont{Loudon}},
  \emph{\bibinfo{title}{The Quantum Theory of Light}}
  (\bibinfo{publisher}{Oxford University Press}, \bibinfo{address}{Oxford},
  \bibinfo{year}{2000}), ISBN \bibinfo{isbn}{0 19 850176 5}.

\bibitem[{\citenamefont{Murphy}(2022)}]{MurphyBook2022}
\bibinfo{author}{\bibfnamefont{K.~P.} \bibnamefont{Murphy}},
  \emph{\bibinfo{title}{Probabilistic Machine Learning: An introduction}}
  (\bibinfo{publisher}{MIT Press}, \bibinfo{year}{2022}), ISBN
  \bibinfo{isbn}{9780262046824}, \urlprefix\url{probml.ai}.

\bibitem[{\citenamefont{{Thomas M. Cover} and {Joy A.
  Thomas}}(2005)}]{CoverThomas}
\bibinfo{author}{\bibnamefont{{Thomas M. Cover}}} \bibnamefont{and}
  \bibinfo{author}{\bibnamefont{{Joy A. Thomas}}},
  \emph{\bibinfo{title}{Elements of Information Theory}}
  (\bibinfo{publisher}{John Wiley \& Sons, Ltd}, \bibinfo{address}{Hoboken, New
  Jersey}, \bibinfo{year}{2005}), \bibinfo{edition}{2nd} ed., ISBN
  \bibinfo{isbn}{9780471241959},
  \urlprefix\url{https://onlinelibrary.wiley.com/doi/abs/10.1002/047174882X}.

\bibitem[{\citenamefont{Sarra et~al.}(2021)\citenamefont{Sarra, Aiello, and
  Marquardt}}]{PhysRevLett.126.200601}
\bibinfo{author}{\bibfnamefont{L.}~\bibnamefont{Sarra}},
  \bibinfo{author}{\bibfnamefont{A.}~\bibnamefont{Aiello}}, \bibnamefont{and}
  \bibinfo{author}{\bibfnamefont{F.}~\bibnamefont{Marquardt}},
  \bibinfo{journal}{Phys. Rev. Lett.} \textbf{\bibinfo{volume}{126}},
  \bibinfo{pages}{200601} (\bibinfo{year}{2021}),
  \urlprefix\url{https://link.aps.org/doi/10.1103/PhysRevLett.126.200601}.

\bibitem[{\citenamefont{Aiello}(2021)}]{aiello_arxiv_2022}
\bibinfo{author}{\bibfnamefont{A.}~\bibnamefont{Aiello}},
  \bibinfo{journal}{arXiv:2110.12930 [quant-ph]}  (\bibinfo{year}{2021}),
  \urlprefix\url{https://doi.org/10.48550/arXiv.2110.12930}.

\bibitem[{\citenamefont{Goodman}(2005)}]{GoodmanBook}
\bibinfo{author}{\bibfnamefont{J.~W.} \bibnamefont{Goodman}},
  \emph{\bibinfo{title}{Introduction to Fourier Optics}}
  (\bibinfo{publisher}{Roberts \& Company}, \bibinfo{address}{Englewood,
  Colorado}, \bibinfo{year}{2005}), \bibinfo{edition}{3rd} ed.

\bibitem[{\citenamefont{Deutsch}(1991)}]{Deutsch1991}
\bibinfo{author}{\bibfnamefont{I.~H.} \bibnamefont{Deutsch}},
  \bibinfo{journal}{American Journal of Physics} \textbf{\bibinfo{volume}{59}},
  \bibinfo{pages}{834} (\bibinfo{year}{1991}),
  \urlprefix\url{https://doi.org/10.1119/1.16731}.

\bibitem[{\citenamefont{Haag}(1992)}]{Haag1992}
\bibinfo{author}{\bibfnamefont{R.}~\bibnamefont{Haag}},
  \emph{\bibinfo{title}{Local Quantum Physics}}, Texts and Monographs in
  Physics (\bibinfo{publisher}{Springer, Berlin, Heidelberg},
  \bibinfo{year}{1992}), \bibinfo{edition}{2nd} ed., ISBN
  \bibinfo{isbn}{978-3-642-61458-3},
  \urlprefix\url{https://doi.org/10.1007/978-3-642-61458-3}.

\bibitem[{\citenamefont{Agullo et~al.}(2023)\citenamefont{Agullo, Bonga,
  Ribes-Metidieri, Kranas, and
  Nadal-Gisbert}}]{https://doi.org/10.48550/arxiv.2302.13742}
\bibinfo{author}{\bibfnamefont{I.}~\bibnamefont{Agullo}},
  \bibinfo{author}{\bibfnamefont{B.}~\bibnamefont{Bonga}},
  \bibinfo{author}{\bibfnamefont{P.}~\bibnamefont{Ribes-Metidieri}},
  \bibinfo{author}{\bibfnamefont{D.}~\bibnamefont{Kranas}}, \bibnamefont{and}
  \bibinfo{author}{\bibfnamefont{S.}~\bibnamefont{Nadal-Gisbert}},
  \emph{\bibinfo{title}{How ubiquitous is entanglement in quantum field
  theory?}} (\bibinfo{year}{2023}),
  \urlprefix\url{https://arxiv.org/abs/2302.13742}.

\bibitem[{\citenamefont{Peter~Rowe}(1979)}]{Rowe_1979}
\bibinfo{author}{\bibfnamefont{E.~G.} \bibnamefont{Peter~Rowe}},
  \bibinfo{journal}{American Journal of Physics} \textbf{\bibinfo{volume}{47}},
  \bibinfo{pages}{373} (\bibinfo{year}{1979}),
  \urlprefix\url{https://doi.org/10.1119/1.11827}.

\bibitem[{\citenamefont{Bladel}(1991)}]{BladelBook}
\bibinfo{author}{\bibfnamefont{J.~G.~V.} \bibnamefont{Bladel}},
  \emph{\bibinfo{title}{{Singular Electromagnetic Fields and Sources}}}, {The
  IEEE Series on Electromagnetic Wave Theory} (\bibinfo{publisher}{IEEE Press},
  \bibinfo{address}{Piscataway, NJ, USA}, \bibinfo{year}{1991}), ISBN
  \bibinfo{isbn}{0-7803-6038-9}.

\bibitem[{\citenamefont{Itzykson and Zuber}(2005)}]{ItzZub}
\bibinfo{author}{\bibfnamefont{C.}~\bibnamefont{Itzykson}} \bibnamefont{and}
  \bibinfo{author}{\bibfnamefont{J.-B.} \bibnamefont{Zuber}},
  \emph{\bibinfo{title}{Quantum Field Theory}} (\bibinfo{publisher}{Dover
  Publications, Inc.}, \bibinfo{address}{Mineola, New York, USA},
  \bibinfo{year}{2005}), ISBN \bibinfo{isbn}{0-486-44568-2}.

\bibitem[{\citenamefont{Coleman}(2019)}]{Coleman2019}
\bibinfo{author}{\bibfnamefont{S.}~\bibnamefont{Coleman}},
  \emph{\bibinfo{title}{Lectures of Sidney Coleman on Quantum Field Theory}}
  (\bibinfo{publisher}{World Scientific Publishing},
  \bibinfo{address}{Singapore}, \bibinfo{year}{2019}), ISBN
  \bibinfo{isbn}{978-981-4635-50-9}, \bibinfo{note}{foreword by David Kaiser},
  \eprint{https://www.worldscientific.com/doi/pdf/10.1142/9371},
  \urlprefix\url{https://www.worldscientific.com/doi/abs/10.1142/9371}.

\bibitem[{\citenamefont{{Stephen M. Barnett} and {Paul M.
  Radmore}}(2002)}]{Barnett}
\bibinfo{author}{\bibnamefont{{Stephen M. Barnett}}} \bibnamefont{and}
  \bibinfo{author}{\bibnamefont{{Paul M. Radmore}}},
  \emph{\bibinfo{title}{{Methods in theoretical Quantum Optics}}}, Oxford
  Series in Optical and Imaging Science: 15 (\bibinfo{publisher}{Oxford
  University Press}, \bibinfo{address}{Oxford, UK}, \bibinfo{year}{2002}), ISBN
  \bibinfo{isbn}{0 19 856361 2}.

\bibitem[{\citenamefont{Schleich}(2001)}]{Schleich}
\bibinfo{author}{\bibfnamefont{W.~P.} \bibnamefont{Schleich}},
  \emph{\bibinfo{title}{Quantum Optics in Phase Space}}
  (\bibinfo{publisher}{John Wiley \& Sons, Ltd}, \bibinfo{address}{Berlin},
  \bibinfo{year}{2001}), ISBN \bibinfo{isbn}{9783527602971},
  \eprint{https://onlinelibrary.wiley.com/doi/pdf/10.1002/3527602976},
  \urlprefix\url{https://onlinelibrary.wiley.com/doi/abs/10.1002/3527602976}.

\bibitem[{\citenamefont{Bohr and Rosenfeld}(1933)}]{Rosenfeld_1933}
\bibinfo{author}{\bibfnamefont{N.}~\bibnamefont{Bohr}} \bibnamefont{and}
  \bibinfo{author}{\bibfnamefont{L.}~\bibnamefont{Rosenfeld}},
  \bibinfo{journal}{Mat.-fys. Medii. Dan. Vid. Selsk.}
  \textbf{\bibinfo{volume}{12}} (\bibinfo{year}{1933}), \bibinfo{note}{for
  English translation see: \textit{Selected Papers of L\'{e}on Rosenfeld}, eds.
  R. S. Cohen and J. Stachel (Boston Studies in the Philosophy of Science
  volume XXI) (Dordrecht: D. Reidel Pub., 1978) pp. 357–412.}

\bibitem[{\citenamefont{Bohr and Rosenfeld}(1950)}]{PhysRev.78.794}
\bibinfo{author}{\bibfnamefont{N.}~\bibnamefont{Bohr}} \bibnamefont{and}
  \bibinfo{author}{\bibfnamefont{L.}~\bibnamefont{Rosenfeld}},
  \bibinfo{journal}{Phys. Rev.} \textbf{\bibinfo{volume}{78}},
  \bibinfo{pages}{794} (\bibinfo{year}{1950}),
  \urlprefix\url{https://link.aps.org/doi/10.1103/PhysRev.78.794}.

\bibitem[{\citenamefont{Heitler}(1984)}]{Heitler}
\bibinfo{author}{\bibfnamefont{W.}~\bibnamefont{Heitler}},
  \emph{\bibinfo{title}{Quantum Field Theory}} (\bibinfo{publisher}{Dover
  Publications, Inc.}, \bibinfo{address}{Mineola, New York, USA},
  \bibinfo{year}{1984}), \bibinfo{edition}{3rd} ed., ISBN
  \bibinfo{isbn}{0-486-64558-4}.

\bibitem[{\citenamefont{Ramshaw}(1985)}]{Ramshaw_1985}
\bibinfo{author}{\bibfnamefont{J.~D.} \bibnamefont{Ramshaw}},
  \bibinfo{journal}{American Journal of Physics} \textbf{\bibinfo{volume}{53}},
  \bibinfo{pages}{178} (\bibinfo{year}{1985}), ISSN \bibinfo{issn}{0002-9505},
  \urlprefix\url{https://doi.org/10.1119/1.14109}.

\bibitem[{\citenamefont{Man'ko et~al.}(1998)\citenamefont{Man'ko, Rosa, and
  Vitale}}]{MANKO1998328}
\bibinfo{author}{\bibfnamefont{V.}~\bibnamefont{Man'ko}},
  \bibinfo{author}{\bibfnamefont{L.}~\bibnamefont{Rosa}}, \bibnamefont{and}
  \bibinfo{author}{\bibfnamefont{P.}~\bibnamefont{Vitale}},
  \bibinfo{journal}{Physics Letters B} \textbf{\bibinfo{volume}{439}},
  \bibinfo{pages}{328} (\bibinfo{year}{1998}), ISSN \bibinfo{issn}{0370-2693},
  \urlprefix\url{https://www.sciencedirect.com/science/article/pii/S0370269398010338}.

\bibitem[{\citenamefont{Reif}(2009)}]{reif2009}
\bibinfo{author}{\bibfnamefont{F.}~\bibnamefont{Reif}},
  \emph{\bibinfo{title}{Fundamentals of Statistical and Thermal Physics}}
  (\bibinfo{publisher}{Waveland Press, Inc.}, \bibinfo{year}{2009}), ISBN
  \bibinfo{isbn}{1-57766-612-7}.

\bibitem[{\citenamefont{Manfredi and Feix}(2000)}]{PhysRevE.62.4665}
\bibinfo{author}{\bibfnamefont{G.}~\bibnamefont{Manfredi}} \bibnamefont{and}
  \bibinfo{author}{\bibfnamefont{M.~R.} \bibnamefont{Feix}},
  \bibinfo{journal}{Phys. Rev. E} \textbf{\bibinfo{volume}{62}},
  \bibinfo{pages}{4665} (\bibinfo{year}{2000}),
  \urlprefix\url{https://link.aps.org/doi/10.1103/PhysRevE.62.4665}.

\bibitem[{\citenamefont{Santos et~al.}(2021)\citenamefont{Santos, Vieira, and
  Dieguez}}]{SANTOS2021125937}
\bibinfo{author}{\bibfnamefont{J.~F.} \bibnamefont{Santos}},
  \bibinfo{author}{\bibfnamefont{C.~H.} \bibnamefont{Vieira}},
  \bibnamefont{and} \bibinfo{author}{\bibfnamefont{P.~R.}
  \bibnamefont{Dieguez}}, \bibinfo{journal}{Physica A: Statistical Mechanics
  and its Applications} \textbf{\bibinfo{volume}{579}}, \bibinfo{pages}{125937}
  (\bibinfo{year}{2021}), ISSN \bibinfo{issn}{0378-4371},
  \urlprefix\url{https://www.sciencedirect.com/science/article/pii/S0378437121002090}.

\bibitem[{\citenamefont{Nair et~al.}(2007)\citenamefont{Nair, Prabhakar, and
  Shah}}]{nair2007entropy}
\bibinfo{author}{\bibfnamefont{C.}~\bibnamefont{Nair}},
  \bibinfo{author}{\bibfnamefont{B.}~\bibnamefont{Prabhakar}},
  \bibnamefont{and} \bibinfo{author}{\bibfnamefont{D.}~\bibnamefont{Shah}},
  \emph{\bibinfo{title}{On entropy for mixtures of discrete and continuous
  variables}} (\bibinfo{year}{2007}),
  \eprint{https://arxiv.org/abs/cs/0607075}.

\bibitem[{\citenamefont{Inc.}()}]{Mathematica}
\bibinfo{author}{\bibfnamefont{W.~R.} \bibnamefont{Inc.}},
  \emph{\bibinfo{title}{Mathematica, {V}ersion 13.2}},
  \bibinfo{note}{{Champaign, IL, 2022}},
  \urlprefix\url{https://www.wolfram.com/mathematica}.

\bibitem[{\citenamefont{{L. Mandel} and {E. Wolf}}(1995)}]{MandelBook}
\bibinfo{author}{\bibnamefont{{L. Mandel}}} \bibnamefont{and}
  \bibinfo{author}{\bibnamefont{{E. Wolf}}}, \emph{\bibinfo{title}{Optical
  Coherence and Quantum Optics}} (\bibinfo{publisher}{Cambridge University
  Press}, \bibinfo{address}{New York}, \bibinfo{year}{1995}).

\bibitem[{\citenamefont{Schulman}(2005)}]{SchulmanBook}
\bibinfo{author}{\bibfnamefont{L.~S.} \bibnamefont{Schulman}},
  \emph{\bibinfo{title}{Techniques and Applications of Path Integration}}
  (\bibinfo{publisher}{Dover Publications, Inc.}, \bibinfo{address}{Mineola,
  New York, USA}, \bibinfo{year}{2005}), ISBN \bibinfo{isbn}{0-486-44528-3}.

\end{thebibliography}


\onecolumn\newpage

\appendix


\numberwithin{equation}{section}


\section{Commutation relations}\label{commrel}

\subsection{Amplitude operators}

Let $\{ f \} = \{f_1(\mathbf{x}), f_2(\mathbf{x}), \ldots , f_M(\mathbf{x}) \}$ be a set of $M$ smooth real functions, such that
%
\begin{align}\label{s10}
%
\left( f_n, f_m \right) & =   \int   f_n(\mathbf{x}) f_m(\mathbf{x}) \, \mathrm{d} \mathbf{x} \nonumber \\[6pt]
%
& =   \delta_{nm}, \qquad \qquad (n,m=1,2, \ldots, M).
%
\end{align}
%
Given the field
%
\begin{align}\label{s20}
%
\hat{\Phi} (\mathbf{x},z, t) & =   \frac{1}{\sqrt{2}}  \left[ e^{- i \omega t} \hat{\phi} (\mathbf{x},z)  +  e^{i \omega t} \hat{\phi}^\dagger (\mathbf{x},z)  \right],
%
\end{align}
%
with
%
\begin{align}\label{s30}
%
\hat{\phi} (\mathbf{x},z) & =   \sum_{\mu} \hat{a}_{\mu} u_{\mu} (\mathbf{x},z),
%
\end{align}
%
we can use the functions $\{ f \} $ to build the  set of $M$  Hermitian operators $\bigl\{ \hat{\Phi}(z) \bigr\} =$ \\ $\bigl\{\hat{\Phi}_1(z), \hat{\Phi}_2(z), \ldots , \hat{\Phi}_M(z) \bigr\}$, defined by
%
\begin{align}\label{s40}
%
\hat{W}_n(z) & =   \bigl( f_n, \,\hat{\Phi} (\mathbf{x},z,0) \bigr) \nonumber \\[6pt]
%
& =   \int   f_n(\mathbf{x}) \hat{\Phi} (\mathbf{x},z,0) \, \mathrm{d} \mathbf{x} \nonumber \\[6pt]
%
& =  \frac{1}{\sqrt{2}}\sum_\mu \left( \hat{a}_\mu f_{n \mu}^* + \hat{a}_\mu^\dagger f_{n \mu} \right), \qquad \qquad (n=1,2, \ldots, M),
%
\end{align}
%
where $f_{n \mu} = f_{n \mu}(z)$, with
%
\begin{align}\label{s50}
%
f_{n \mu}(z) & =  \left( u_\mu, f_n \right) \nonumber \\[6pt]
%
& =   \int  u_\mu^*(\mathbf{x},z)f_n(\mathbf{x}) \, \mathrm{d} \mathbf{x} .
%
\end{align}
%

Next, we calculate the commutator
%
\begin{align}\label{s60}
%
\bigl[ \hat{W}_n(z), \, \hat{W}_m(z) \bigr] & =  \bigl[ ( f_n,\hat{\Phi} ) , \, ( f_m,\hat{\Phi} ) \bigr]  \nonumber \\[6pt]
%
& =   \frac{1}{2} \sum_{\mu, \nu} \bigl[ {\hat{a}_\mu} ( f_n, u_\mu ) + {\hat{a}_\mu^\dagger} ( f_n, u_\mu^* ) , \, {\hat{a}_\nu} ( f_m, u_\nu ) + {\hat{a}_\nu^\dagger} ( f_m, u_\nu^* ) \bigr] \nonumber \\[6pt]
%
& =  \frac{1}{2} \sum_{\mu} \bigl[ {( f_n, u_\mu )( u_\mu, f_m )} - {( f_m, u_\mu )( u_\mu, f_n )} \bigr]
\nonumber \\[6pt]
%
& =  \frac{1}{2} \sum_{\mu} \bigl[ {( f_n,  f_m )} - {( f_m,  f_n )} \bigr]
\nonumber \\[6pt]
%
& =  0,
%
\end{align}
%
where we have used
%
\begin{align}\label{s70}
%
\bigl[ \hat{a}_\mu, \, \hat{a}_\nu^\dagger \bigr] & =  \delta_{\mu \nu}, \qquad \bigl[ \hat{a}_\mu, \, \hat{a}_\nu \bigr]  = 0, \qquad \text{and} \qquad \sum_{\mu} u_\mu(\mathbf{x},z)u_\mu^*(\mathbf{x}',z) = \delta(\mathbf{x} - \mathbf{x}'),
%
\end{align}
%
with $\delta(\mathbf{x} - \mathbf{x}') = \delta(x - x') \delta(y - y')$.
Note that from $f_n(\mathbf{x}) \in \mathbb{R}$ it follows that $( f_n,  f_m ) = ( f_m,  f_n )$, so that the commutator \eqref{s60} would be zero even if \eqref{s10} were not satisfied.


\subsection{Intensity operators}


Let us define the ``intensity'' operator $\hat{\mathrm{I}}(\mathbf{x},z)$ as,
%
\begin{align}\label{s80}
%
\hat{\mathrm{I}}(\mathbf{x},z) & =  \hat{\phi}^\dagger (\mathbf{x},z) \hat{\phi} (\mathbf{x},z) \nonumber \\[6pt]
%
& =   \sum_{\mu, \nu} \hat{a}_{\mu}^\dagger\hat{a}_{\nu}  u_{\mu}^* (\mathbf{x},z) u_{\nu} (\mathbf{x},z).
%
\end{align}
%
By definition
%
\begin{align}\label{s90}
%
\int \hat{\mathrm{I}}(\mathbf{x},z) \, \mathrm{d} \mathbf{x}  & =   \sum_{\mu} \hat{a}_{\mu}^\dagger \hat{a}_{\mu} ,
%
\end{align}
%
where the orthogonality relation
%
\begin{align}\label{s100}
%
\left( u_\mu, \, u_\nu \right) & =  \int  u_{\mu}^* (\mathbf{x},z) u_{\nu} (\mathbf{x},z) \,\mathrm{d} \mathbf{x}   = \delta_{\mu \nu},
%
\end{align}
%
has been used.

Let $\{ \bm{1} \} = \{\bm{1}_1(\mathbf{x}), \bm{1}_2(\mathbf{x}), \ldots , \bm{1}_M(\mathbf{x}) \}$ be a set of $M$ indicator functions with disjoint compact supports. By definition
%
\begin{align}\label{s110}
%
\operatorname{supp}\{ \bm{1}_n \} \bigcap \operatorname{supp}\{ \bm{1}_m \} & =   \varnothing \qquad \text{if} \qquad n \neq m,
%
\end{align}
%
or, equivalently,
%
\begin{align}\label{s120}
%
 \bm{1}_n(\mathbf{x}) \bm{1}_m (\mathbf{x}) & =   \delta_{nm} \bm{1}_n(\mathbf{x}),  \qquad \qquad (n,m=1,2, \ldots, M).
%
\end{align}
%
We assume that they are normalized to some areas $\mathcal{A}_n$ (typically the area of the active surface of a detector), defined by
%
\begin{align}\label{s130}
%
\int \bm{1}_n(\mathbf{x}) \, \mathrm{d} \mathbf{x} = \int_{\mathcal{D}_n}\mathrm{d} \mathbf{x}   =   \mathcal{A}_n,
%
\end{align}
%
where $\mathcal{D}_n = \operatorname{supp}\{ \bm{1}_n \}$. Using these functions, we define the ``counting operator'' $\hat{C}_n$, as
%
\begin{align}\label{s140}
%
\hat{C}_n  & =   \bigl( \bm{1}_n, \, \hat{\mathrm{I}} (\mathbf{x},z) \bigr) \nonumber \\[6pt]
%
& =  \int   \bm{1}_n(\mathbf{x}) \, \hat{\mathrm{I}}(\mathbf{x},z) \, \mathrm{d} \mathbf{x}  \nonumber \\[6pt]
%
& =   \sum_{\mu, \nu} \hat{a}_{\mu}^\dagger\hat{a}_{\nu} \int    u_{\mu}^* (\mathbf{x},z) \bm{1}_n(\mathbf{x}) u_{\nu} (\mathbf{x},z)  \, \mathrm{d} \mathbf{x} \nonumber \\[6pt]
%
& =  \sum_{\mu, \nu} \hat{a}_\mu^\dagger \hat{a}_\nu \bm{1}_{n \mu \nu},
%
\end{align}
%
where $\bm{1}_{n \mu \nu} = \bm{1}_{n \mu \nu}(z)$, with
%
\begin{align}\label{s150}
%
\bm{1}_{n \mu \nu}(z) & =  (u_\mu, \bm{1}_n u _\nu) .
%
\end{align}
%


Next, we calculate the commutator
%
\begin{align}\label{s160}
%
\bigl[ \hat{C}_n(z), \, \hat{C}_m(z) \bigr] & =  \bigl[ ( \bm{1}_n, \hat{\mathrm{I}} ) , \, ( \bm{1}_m, \hat{\mathrm{I}} ) \bigr]  \nonumber \\[6pt]
%
& =   \sum_{\mu, \nu}  \sum_{\alpha, \beta} \bigl[
\hat{a}_\mu^\dagger \hat{a}_\nu \bm{1}_{n\mu \nu} , \,
\hat{a}_\alpha^\dagger \hat{a}_\beta \bm{1}_{m \alpha \beta} \bigr] \nonumber \\[6pt]
%
& =  \sum_{\mu, \nu}  \sum_{\alpha, \beta} \bm{1}_{n\mu \nu} \bm{1}_{m \alpha \beta} \bigl[
\hat{a}_\mu^\dagger \hat{a}_\nu , \,
\hat{a}_\alpha^\dagger \hat{a}_\beta \bigr].
%
\end{align}
%
Using the commutator relation
%
\begin{align}\label{s170}
%
\bigl[ \hat{A} \hat{B} , \, \hat{C} \hat{D} \bigr] & =  \hat{A}\bigl[\hat{B}, \hat{C}\bigr]\hat{D} + \bigl[\hat{A}, \hat{C}\bigr]\hat{B}\hat{D} + \hat{C}\hat{A}\bigl[\hat{B}, \hat{D}\bigr] + \hat{C}\bigl[\hat{A}, \hat{D}\bigr]\hat{B},
%
\end{align}
%
with $\hat{A} = \hat{a}_\mu^\dagger, \, \hat{B} = \hat{a}_\nu, \, \hat{C} = \hat{a}_\alpha^\dagger$ and $\hat{D} = \hat{a}_\beta$, we find
%
\begin{align}\label{s180}
%
\bigl[
\hat{a}_\mu^\dagger \hat{a}_\nu , \,
\hat{a}_\alpha^\dagger \hat{a}_\beta \bigr] & =  \hat{a}_\mu^\dagger \underbrace{\bigl[\hat{a}_\nu, \hat{a}_\alpha^\dagger\bigr]}_{= \, \delta_{\nu \alpha}} \hat{a}_\beta
+ \underbrace{\bigl[\hat{a}_\mu^\dagger, \hat{a}_\alpha^\dagger\bigr]}_{= \, 0}\hat{a}_\nu\hat{a}_\beta
+ \hat{a}_\alpha^\dagger\hat{a}_\mu^\dagger \underbrace{\bigl[\hat{a}_\nu, \hat{a}_\beta\bigr] }_{= \, 0}
+ \hat{a}_\alpha^\dagger \underbrace{\bigl[\hat{a}_\mu^\dagger, \hat{a}_\beta\bigr]}_{= \, - \delta_{\mu \beta}} \hat{a}_\nu   \nonumber \\[6pt]
%
& =  \hat{a}_\mu^\dagger \hat{a}_\beta \, \delta_{\nu \alpha} - \hat{a}_\alpha^\dagger  \hat{a}_\nu \, \delta_{\mu \beta}.
%
\end{align}
%
Substituting \eqref{s180} into \eqref{s170}, we obtain
%
\begin{align}\label{s190}
%
\bigl[ \hat{C}_n(z), \, \hat{C}_m(z) \bigr] & =  \sum_{\mu, \nu}  \sum_{\alpha, \beta} \bm{1}_{n\mu \nu} \bm{1}_{m \alpha \beta} \left( \hat{a}_\mu^\dagger \hat{a}_\beta \, \delta_{\nu \alpha} - \hat{a}_\alpha^\dagger  \hat{a}_\nu \, \delta_{\mu \beta} \right)  \nonumber \\[6pt]
%
& =   \sum_{\mu, \beta} \hat{a}_\mu^\dagger \hat{a}_\beta \, \sum_{\nu} \bm{1}_{n \mu \nu} \bm{1}_{m \nu \beta}  -
\sum_{\nu, \alpha} \hat{a}_\alpha^\dagger  \hat{a}_\nu \, \sum_{\mu} \bm{1}_{m \alpha \mu} \bm{1}_{n\mu \nu} .
%
\end{align}
%
Now it remains to calculate
%
\begin{align}\label{s200}
%
\sum_{\nu} \bm{1}_{n \mu \nu} \bm{1}_{m \nu \beta} & =  \sum_{\nu} (u_\mu, \bm{1}_n u _\nu) (u_\nu, \bm{1}_m u _\beta)   \nonumber \\[6pt]
%
& =   \sum_{\nu} \int  u_\mu(\mathbf{x},z) \bm{1}_n (\mathbf{x} ) u _\nu(\mathbf{x},z) \,  \mathrm{d} \mathbf{x} \int u_\nu(\mathbf{x}',z) \bm{1}_m(\mathbf{x}') u _\beta (\mathbf{x}',z) \, \mathrm{d} \mathbf{x}' \nonumber \\[6pt]
%
& =    \int \Biggr\{ u_\mu(\mathbf{x},z) \bm{1}_n (\mathbf{x} ) \int  \biggl[   \bm{1}_m(\mathbf{x}') u _\beta (\mathbf{x}',z) \underbrace{\sum_{\nu}u _\nu(\mathbf{x},z)u_\nu(\mathbf{x}',z)}_{= \, \delta(\mathbf{x} - \mathbf{x}')} \biggr] \mathrm{d} \mathbf{x}' \Biggl\} \, \mathrm{d} \mathbf{x}   \nonumber \\
%
& =  \int  u_\mu(\mathbf{x},z) \bm{1}_n (\mathbf{x} )    \bm{1}_m(\mathbf{x}) u_\beta (\mathbf{x},z) \, \mathrm{d} \mathbf{x}   \nonumber \\[6pt]
%
& =  \bigl( u_\mu , \bm{1}_n \bm{1}_m  u_\beta \bigr),
%
\end{align}
%
where the completeness relation \eqref{s70} has been used. Finally, substituting \eqref{s200} into \eqref{s190}, we obtain
%
\begin{align}\label{s210}
%
\bigl[ \hat{C}_n(z), \, \hat{C}_m(z) \bigr] & =   \sum_{\mu, \beta} \hat{a}_\mu^\dagger \hat{a}_\beta \, \bigl( u_\mu , \bm{1}_n \bm{1}_m  u_\beta \bigr)  -
\sum_{\nu, \alpha} \hat{a}_\alpha^\dagger  \hat{a}_\nu \, \bigl( u_\alpha , \bm{1}_m \bm{1}_n  u_\nu \bigr) \nonumber \\[6pt]
%
& =  \sum_{\mu, \nu} \hat{a}_\mu^\dagger \hat{a}_\nu \Bigl[ \bigl( u_\mu , \bm{1}_n \bm{1}_m  u_\nu \bigr) - \bigl( u_\mu , \bm{1}_m \bm{1}_n  u_\nu \bigr)  \, \Bigr]  \nonumber \\[6pt]
%
& =  0,
%
\end{align}
%
where we have renamed the dummy indexes $\beta \to \nu$ and $\alpha \to \mu$. The final result comes from the trivial identity $\bm{1}_n (\mathbf{x} )    \bm{1}_m(\mathbf{x}) = \bm{1}_m (\mathbf{x} )    \bm{1}_n(\mathbf{x})$.


\subsection{A remark}


It should be noticed that actually the two commutators \eqref{s60} and \eqref{s160} are trivially equal to zero, because they are both of the form
%
\begin{align}\label{s220}
%
\bigl[ ( v, \hat{A} ) , \, ( w, \hat{A} ) \bigr] & =  \int \left\{ \int  v(\mathbf{x}) w(\mathbf{x}')  \bigl[ \hat{A}(\mathbf{x})  , \, \hat{A}(\mathbf{x}')  \bigr] \mathrm{d} \mathbf{x}' \right\} \mathrm{d} \mathbf{x}  .
%
\end{align}
%
Then, one should simply verify that
%
\begin{align}\label{s230}
%
\bigl[ \hat{\Phi}(\mathbf{x})  , \, \hat{\Phi}(\mathbf{x}')  \bigr] & =  0, \qquad \text{and} \qquad   \bigl[ \hat{\mathrm{I}}(\mathbf{x})  , \, \hat{\mathrm{I}}(\mathbf{x}')  \bigr]=0 .
%
\end{align}
%

To illustrate this procedure, we calculate now the mixed commutator
%
\begin{align}\label{m10}
%
\bigl[ \hat{W}_m(z), \, \hat{C}_n(z) \bigr] & =  \bigl[ ( f_m, \hat{\Phi} ) , \, ( \bm{1}_n, \hat{\mathrm{I}} ) \bigr]  \nonumber \\[6pt]
%
& =  \int \mathrm{d} \mathbf{x} \int \mathrm{d} \mathbf{x}' \, f_m(\mathbf{x}) \bm{1}_n(\mathbf{x}')  \bigl[ \hat{\Phi}(\mathbf{x},z,0)  , \, \hat{\mathrm{I}}(\mathbf{x}',z)  \bigr],
%
\end{align}
%
where
%
\begin{align}\label{m20}
%
\bigl[ \hat{\Phi}(\mathbf{x},z,0)  , \, \hat{\mathrm{I}}(\mathbf{x}',z)  \bigr] & =  \frac{1}{\sqrt{2}} \bigl[ \hat{\phi} (\mathbf{x},z)  +   \hat{\phi}^\dagger (\mathbf{x},z) , \, \hat{\phi}^\dagger (\mathbf{x}',z)  \hat{\phi} (\mathbf{x}',z) \bigr]  \nonumber  \\[6pt]
%
& =  \frac{1}{\sqrt{2}} \left\{ \bigl[ \hat{\phi} (\mathbf{x},z)   , \, \hat{\phi}^\dagger (\mathbf{x}',z)  \hat{\phi} (\mathbf{x}',z) \bigr] + \bigl[  \hat{\phi}^\dagger (\mathbf{x},z) , \, \hat{\phi}^\dagger (\mathbf{x}',z)  \hat{\phi} (\mathbf{x}',z) \bigr] \right\} .
%
\end{align}
%
Using
%
\begin{align}\label{m30}
%
\bigl[ \hat{A}  , \, \hat{B} \hat{C}  \bigr] & =  \bigl[ \hat{A}  , \, \hat{B} \bigr] \hat{C} + \hat{B} \bigl[ \hat{A} , \, \hat{C}  \bigr],
%
\end{align}
%
and
%
\begin{align}\label{m40}
%
\left[ \hat{\phi} (\mathbf{x},z)  , \, \hat{\phi}^\dagger (\mathbf{x}',z)  \right] & =  \Bigl[  \sum_{\mu} \hat{a}_{\mu} u_{\mu} (\mathbf{x},z)  , \,  \sum_{\nu} \hat{a}_{\nu}^\dagger u_{\nu}^* (\mathbf{x}',z) \Bigr] \nonumber  \\[6pt]
%
& =   \sum_{\mu, \nu}u_{\mu} (\mathbf{x},z) u_{\nu}^* (\mathbf{x}',z) \underbrace{\Bigl[  \hat{a}_{\mu} , \,  \hat{a}_{\nu}^\dagger \Bigr]}_{= \, \delta_{\mu \nu}} \nonumber  \\[6pt]
%
& =   \sum_{\mu}u_{\mu} (\mathbf{x},z) u_{\mu}^* (\mathbf{x}',z)  \nonumber  \\[6pt]
%
& =   \delta \left( \mathbf{x} - \mathbf{x}' \right),
%
\end{align}
%
we calculate straightforwardly
%
\begin{align}\label{m50}
%
\Bigl[ \hat{\phi} (\mathbf{x},z)   , \, \hat{\phi}^\dagger (\mathbf{x}',z)  \hat{\phi} (\mathbf{x}',z) \Bigr] & =  \Bigl[ \hat{\phi} (\mathbf{x},z)   , \, \hat{\phi}^\dagger (\mathbf{x}',z)  \Bigr] \hat{\phi} (\mathbf{x}',z) \nonumber  \\[6pt]
%
& =   \delta \left( \mathbf{x} - \mathbf{x}' \right) \hat{\phi} (\mathbf{x}',z),
%
\end{align}
%
and
%
\begin{align}\label{m60}
%
\Bigl[  \hat{\phi}^\dagger (\mathbf{x},z) , \, \hat{\phi}^\dagger (\mathbf{x}',z)  \hat{\phi} (\mathbf{x}',z) \Bigr] & =  \hat{\phi}^\dagger (\mathbf{x}',z)  \Bigl[  \hat{\phi}^\dagger (\mathbf{x},z) , \, \hat{\phi} (\mathbf{x}',z) \Bigr]  \nonumber  \\[6pt]
%
& =  -\delta \left( \mathbf{x} - \mathbf{x}' \right) \hat{\phi}^\dagger (\mathbf{x}',z) .
%
\end{align}
%
Substituting \eqref{m50} and \eqref{m60} into \eqref{m20}, we obtain
%
\begin{align}\label{m70}
%
\bigl[ \hat{\Phi}(\mathbf{x},z,0)  , \, \hat{\mathrm{I}}(\mathbf{x}',z)  \bigr] & =  \frac{1}{\sqrt{2}} \left\{ \bigl[ \hat{\phi} (\mathbf{x},z)   , \, \hat{\phi}^\dagger (\mathbf{x}',z)  \hat{\phi} (\mathbf{x}',z) \bigr] + \bigl[  \hat{\phi}^\dagger (\mathbf{x},z) , \, \hat{\phi}^\dagger (\mathbf{x}',z)  \hat{\phi} (\mathbf{x}',z) \bigr] \right\}  \nonumber  \\[6pt]
%
& =  \frac{1}{\sqrt{2}} \, \delta \left( \mathbf{x} - \mathbf{x}' \right) \left\{\hat{\phi} (\mathbf{x}',z) - \hat{\phi}^\dagger (\mathbf{x}',z) \right\} .
%
\end{align}
%
Substitution of \eqref{m70} into \eqref{m10}, gives
%
\begin{align}\label{m80}
%
\bigl[ \hat{W}_m(z), \, \hat{C}_n(z) \bigr] & =  \int \frac{1}{\sqrt{2}} \, f_m(\mathbf{x})\left\{ \int  \bm{1}_n(\mathbf{x}')  \, \delta \left( \mathbf{x} - \mathbf{x}' \right) \left[\hat{\phi} (\mathbf{x}',z) - \hat{\phi}^\dagger (\mathbf{x}',z) \right] \mathrm{d} \mathbf{x}'  \right\}  \mathrm{d} \mathbf{x} \nonumber \\[6pt]
%
& =  \int  f_m(\mathbf{x}) \bm{1}_n(\mathbf{x})  \frac{\hat{\phi} (\mathbf{x},z) - \hat{\phi}^\dagger (\mathbf{x},z)}{\sqrt{2}} \, \mathrm{d} \mathbf{x}\nonumber \\[6pt]
%
& =  \int_{\mathcal{D}_n}  f_m(\mathbf{x})  \frac{\hat{\phi} (\mathbf{x},z) - \hat{\phi}^\dagger (\mathbf{x},z)}{\sqrt{2}} \, \mathrm{d} \mathbf{x}\nonumber \\[6pt]
%
& =  \delta_{nm}\int_{\mathcal{D}_n}  f_n(\mathbf{x})  \frac{\hat{\phi} (\mathbf{x},z) - \hat{\phi}^\dagger (\mathbf{x},z)}{\sqrt{2}} \, \mathrm{d} \mathbf{x} ,
%
\end{align}
%
because, by hypothesis,  $ f_m(\mathbf{x}) \bm{1}_n(\mathbf{x}) = 0$ for  $n \neq m$. Then, we can rewrite \eqref{m80} in a more compact and suggestive form as,
%
\begin{align}\label{m85}
%
\bigl[ \hat{W}_m(z), \, \hat{C}_n(z) \bigr] = \frac{\delta_{nm}}{\sqrt{2}} \left\{ \bigl(f_n, \hat{\phi}  \bigr) - \bigl(\hat{\phi}, f_n \bigr) \right\}.
%
\end{align}
%


\section{Probability distribution for the wave operators}\label{pdfW}


Here we calculate step by step the following expression for $p_\mathbf{W} (N,\mathbf{w})$, which is defined by
%
\begin{align}\label{s240}
%
p_\mathbf{W}(N,\mathbf{w})  = \langle N[\phi] | \prod_{n = 1}^M \delta \bigl( \hat{W}_n - {w}_n \bigr) | N[\phi] \rangle,
%
\end{align}
%
for $N=0$ and $N=1$. In the remainder we will use  the Fourier transform representation of the Dirac delta function,
%
\begin{align}\label{s250}
%
\delta(x-x_0) & =   \frac{1}{2 \pi} \int \mathrm{d} \alpha \,e^{i \alpha (x - x_0)}.
%
\end{align}
%

\subsection{Vacuum state}

%
\begin{align}\label{s260}
%
p_\mathbf{W}(0,\mathbf{w}) & =  \langle 0 | \prod_{n = 1}^M \delta \bigl( \hat{W}_n - w_n \bigr) | 0 \rangle \nonumber \\[6pt]
%
& =  \frac{1}{(2 \pi)^M} \int \mathrm{d} \alpha_1 \cdots \int \mathrm{d} \alpha_M \, \exp\left( - i \sum_{n=1}^M \alpha_n w_n \right) \langle 0 | e^{i \alpha_1 \hat{\Phi}_1 } e^{i \alpha_2 \hat{\Phi}_2 } \cdots e^{i \alpha_M \hat{\Phi}_M }| 0 \rangle \nonumber \\[6pt]
%
& =  \frac{1}{(2 \pi)^M} \int \mathrm{d} \alpha_1 \cdots \int \mathrm{d} \alpha_M \, \exp\left( - i \sum_{n=1}^M \alpha_n w_n \right) \langle 0 | \exp\left(  i \sum_{n=1}^M \alpha_n \hat{W}_n \right) | 0 \rangle,
%
\end{align}
%
where \eqref{s60} has been used. Next, using \eqref{s40} and \eqref{s50}, we rewrite
%
\begin{align}\label{s270}
%
\sum_{n=1}^M \alpha_n \hat{W}_n  & =  \sum_\mu \left( \hat{a}_\mu \varphi_{\mu}^* + \hat{a}_\mu^\dagger \varphi_{\mu} \right)   ,
%
\end{align}
%
where using \eqref{s50} we have defined
%
\begin{align}\label{s280}
%
\varphi_{\mu} & =  \frac{1}{\sqrt{2}} \sum_{n=1}^M \alpha_n  f_{n \mu} \nonumber \\[6pt]
%
& =  \biggl( u_\mu , \frac{1}{\sqrt{2}} \sum_{n=1}^M \alpha_n  f_{n } \biggr)\nonumber \\[6pt]
%
& =  \bigl( u_\mu , \varphi \bigr).
%
\end{align}
%
This implies that we can define the field $\varphi(\mathbf{x})$ as,
%
\begin{align}\label{s290}
%
\varphi(\mathbf{x}) & =   \frac{1}{\sqrt{2}} \sum_{n=1}^M \alpha_n  f_{n }(\mathbf{x}).
%
\end{align}
%

Now, we are ready to calculate
%
\begin{align}\label{s300}
%
\langle 0 | \exp\left(  i \sum_{n=1}^M \alpha_n \hat{W}_n \right) | 0 \rangle & =  \langle 0 | e^{\hat{A} + \hat{B}} | 0 \rangle,
%
\end{align}
%
where we have defined
%
\begin{align}\label{s310}
%
\hat{A} & =    i \sum_\mu \hat{a}_\mu \varphi_{\mu}^* , \qquad \text{and } \qquad \hat{B}  =    i \sum_\mu \hat{a}_\mu^\dagger \varphi_{\mu}.
%
\end{align}
%
It is easy to see that
%
\begin{align}\label{s320}
%
\bigl[ \hat{A}, \, \hat{B} \bigr] & =  - \sum_{\mu, \nu} \varphi_{\mu}^* \varphi_{\nu} \bigl[ \hat{a}_\mu, \,  \hat{a}_\nu^\dagger \bigr] \nonumber \\[6pt]
%
& =  - \sum_{\mu} | \varphi_{\mu} |^2  \nonumber \\[6pt]
%
& =  - \bigl( \varphi, \varphi \bigr) \nonumber \\[6pt]
%
& =  - \frac{1}{2} \sum_{n,m=1}^M \alpha_n \alpha_m \bigl(f_n, f_m \bigr) \nonumber \\[6pt]
%
& =  - \frac{1}{2} \sum_{n=1}^M \alpha_n^2 \bigl(f_n, f_n \bigr) ,
%
\end{align}
%
where \eqref{s290} and \eqref{s10}  have been used.
So, we can use the Campbell-Baker-Hausdorff identity \cite{MandelBook},
%
\begin{align}\label{s330}
%
e^{ \hat{A} + \hat{B} } = e^{ \hat{A}} \, e^{ \hat{B}  }  \, e^{-\frac{1}{2}[\hat{A},\hat{B}]} = e^{ \hat{B}} \, e^{ \hat{A}  }  \, e^{\frac{1}{2}[\hat{A},\hat{B}]}, \qquad \text{if} \qquad [\hat{A},[\hat{A},\hat{B}]] = 0 = [\hat{B},[\hat{A},\hat{B}]],
%
\end{align}
%
to rewrite
%
\begin{align}\label{s340}
%
\langle 0 | \exp\left(  i \sum_{n=1}^M \alpha_n \hat{W}_n \right) | 0 \rangle & =  \langle 0 | e^{\hat{A} + \hat{B}} | 0 \rangle \nonumber \\[6pt]
%
& =  e^{\frac{1}{2}[\hat{A},\hat{B}]} \underbrace{\langle 0 |  e^{ \hat{B}} \, e^{ \hat{A}  }  | 0 \rangle}_{= \, 1} \nonumber \\[6pt]
%
& =  \exp \left[ - \frac{1}{4} \sum_{n=1}^M \alpha_n^2 \bigl(f_n, f_n \bigr) \right],
%
\end{align}
%
where we have used $\hat{A} | 0 \rangle = 0 = \langle 0 | \hat{B}$. Substituting \eqref{s340} into \eqref{s260} we obtain
%
\begin{align}\label{s350}
%
p_\mathbf{W} (0,\mathbf{w}) & =  \frac{1}{(2 \pi)^M} \int \mathrm{d} \alpha_1 \cdots \int \mathrm{d} \alpha_M \, \exp\left( - i \sum_{n=1}^M \alpha_n w_n \right) \left[ - \frac{1}{4} \sum_{n=1}^M \alpha_n^2 \bigl(f_n, f_n \bigr) \right] \nonumber \\[6pt]
%
& =  \prod_{n=1}^M   \int \frac{\mathrm{d} \alpha_n}{2 \pi}  \exp\left[ - \frac{\bigl(f_n, f_n \bigr)}{4} \alpha_n^2 - i w_n  \alpha_n \right] \nonumber \\[6pt]
%
& =  \prod_{n=1}^M  \frac{1}{\sqrt{\pi \bigl(f_n, f_n \bigr)}} \exp \left[ - \frac{w_n^2}{\bigl(f_n, f_n \bigr)} \right]
\nonumber \\[6pt]
%
& = \prod_{n=1}^M p(0,w_n),
%
\end{align}
%
where the following Gaussian integral has been used (see Eq. (3.16) in \cite{SchulmanBook}):
%
\begin{align}\label{s360}
%
\int_{-\infty}^{\infty} \exp\left( -a y^2 + b y \right ) \mathrm{d} y  = \sqrt{ \frac{\pi}{a} } \exp\left( \frac{b^2}{4 a}\right ),
%
\end{align}
%
with $a,b \in \mathbb{C}$, and $\text{Re} \, a > 0$.


\subsection{Single photon state}


In this case we calculate the probability distribution with respect to the single-photon state $| 1[\phi] \rangle $ defined by
%
\begin{align}\label{s370}
%
| 1[\phi] \rangle  & =  \hat{a}^\dagger[\phi] |0 \rangle \nonumber \\[6pt]
%
& =   \sum_\mu \phi_\mu \hat{a}^\dagger_\mu |0 \rangle,
%
\end{align}
%
where, by hypothesis,
%
\begin{align}\label{s380}
%
\phi_\mu & =  \bigl(u_\mu, \phi \bigr),  \qquad \text{with} \qquad \bigl(\phi, \phi \bigr) = 1.
%
\end{align}
%
So, we must evaluate
%
\begin{align}\label{s390}
%
p_\mathbf{W} (1,\mathbf{w}) & =  \langle 1[\phi] | \prod_{n = 1}^M \delta \bigl( \hat{W}_n - w_n \bigr) | 1[\phi] \rangle \nonumber \\[6pt]
%
& =  \sum_{\mu, \nu} \phi_\mu^* \phi_\nu \langle 0 | \hat{a}_\mu \prod_{n = 1}^M \delta \bigl( \hat{W}_n - w_n \bigr) \hat{a}_\nu^\dagger | 0 \rangle  ,
%
\end{align}
%
where
%
\begin{multline}\label{s400}
%
\langle 0 | \hat{a}_\mu \prod_{n = 1}^M \delta \bigl( \hat{W}_n - w_n \bigr) \hat{a}_\nu^\dagger | 0 \rangle \\[6pt]
%
= \frac{1}{(2 \pi)^M} \int \mathrm{d} \alpha_1 \cdots \int \mathrm{d} \alpha_M \, \exp\left( - i \sum_{n=1}^M \alpha_n w_n \right)  \langle 0 | \hat{a}_\mu \exp\left(  i \sum_{n=1}^M \alpha_n \hat{W}_n \right) \hat{a}_\nu^\dagger| 0 \rangle.
%
\end{multline}
%
Proceeding like in the previous section and using \eqref{s310}, we can write
%
\begin{align}\label{s410}
%
\langle 0 | \hat{a}_\mu \exp\left(  i \sum_{n=1}^M \alpha_n \hat{W}_n \right) \hat{a}_\nu^\dagger| 0 \rangle & =  \langle 0 |\hat{a}_\mu   e^{\hat{A} + \hat{B}} \hat{a}_\nu^\dagger | 0 \rangle \nonumber \\[6pt]
%
& =  e^{\frac{1}{2}[\hat{A},\hat{B}]} \langle 0 |\hat{a}_\mu e^{ \hat{B}} \, e^{ \hat{A}  }  \hat{a}_\nu^\dagger | 0 \rangle \nonumber \\[6pt]
%
& =  e^{\frac{1}{2}[\hat{A},\hat{B}]} \langle 0 |e^{ \hat{B}}  \left( e^{ -\hat{B}} \hat{a}_\mu e^{ \hat{B}} \right) \left( e^{ \hat{A}  }  \hat{a}_\nu^\dagger e^{ -\hat{A}} \right)e^{ \hat{A}} | 0 \rangle \nonumber \\[6pt]
%
& =  e^{\frac{1}{2}[\hat{A},\hat{B}]} \langle 0 | \left(  \hat{a}_\mu - \bigl[ \hat{B} , \, \hat{a}_\mu \bigr]  \right) \left(  \hat{a}_\nu^\dagger + \bigl[ \hat{A} , \, \hat{a}_\nu^\dagger \bigr]  \right) | 0 \rangle \nonumber \\[6pt]
%
& =  e^{\frac{1}{2}[\hat{A},\hat{B}]} \left\{ \langle 0 |  \hat{a}_\mu  \hat{a}_\nu^\dagger | 0 \rangle - \bigl[ \hat{B} , \, \hat{a}_\mu \bigr]   \bigl[ \hat{A} , \, \hat{a}_\nu^\dagger \bigr] \right\},
%
\end{align}
%
where we have used equation (10.11-2) in \cite{MandelBook}:
%
\begin{align}\label{s420}
%
\exp \bigl( x \hat{A} \bigr) \hat{B} \exp \bigl( -x \hat{A} \bigr) & =  \hat{B} + x \bigl[ \hat{A} , \, \hat{B} \bigr], \qquad \text{if} \;\bigl[ \hat{A} , \, \hat{B} \bigr] \; \text{is a $c$-number}.
%
\end{align}
%
In our case
%
\begin{align}\label{s430}
%
\bigl[ \hat{A} , \, \hat{a}_\nu^\dagger \bigr] & =   i \sum_\mu \varphi_{\mu}^* \bigl[ \hat{a}_\mu  , \, \hat{a}_\nu^\dagger \bigr] \nonumber \\[6pt]
%
& =   i \, \varphi_{\nu}^*,
%
\end{align}
%
and
%
\begin{align}\label{s440}
%
\bigl[ \hat{B} , \, \hat{a}_\mu \bigr] & =    i \sum_\nu  \varphi_{\nu} \bigl[ \hat{a}_\nu^\dagger , \, \hat{a}_\mu   \bigr] \nonumber \\[6pt]
%
& =   -i \, \varphi_{\mu}.
%
\end{align}
%
Substituting \eqref{s430} and \eqref{s440} into \eqref{s410}, we obtain
%
\begin{align}\label{s450}
%
\langle 0 | \hat{a}_\mu \exp\left(  i \sum_{n=1}^M \alpha_n \hat{W}_n \right) \hat{a}_\nu^\dagger| 0 \rangle & =  e^{\frac{1}{2}[\hat{A},\hat{B}]} \left\{ \langle 0 |  \hat{a}_\mu  \hat{a}_\nu^\dagger | 0 \rangle - \bigl[ \hat{B} , \, \hat{a}_\mu \bigr]   \bigl[ \hat{A} , \, \hat{a}_\nu^\dagger \bigr] \right\} \nonumber \\[6pt]
%
& =  e^{\frac{1}{2}[\hat{A},\hat{B}]} \bigl\{ \delta_{\mu \nu} - \left( - i \, \varphi_{\mu} \right) \left(  i \, \varphi_{\nu}^* \right) \bigr\}   \nonumber \\[6pt]
%
& =    \exp \left[ - \frac{1}{4} \sum_{n=1}^M \alpha_n^2 \bigl(f_n, f_n \bigr) \right] \bigl( \delta_{\mu \nu} - \varphi_{\mu}  \varphi_{\nu}^*  \bigr) .
%
\end{align}
%
Inserting this expression into \eqref{s390} and using \eqref{s400}, we find
%
\begin{align}\label{s460}
%
p_\mathbf{W} & (1,\mathbf{w})  \nonumber \\[6pt]
%
& =  \sum_{\mu, \nu} \phi_\mu^* \phi_\nu \, \langle 0 | \hat{a}_\mu \prod_{l = 1}^M \delta \bigl( \hat{W}_l - w_l \bigr) \hat{a}_\nu^\dagger | 0 \rangle \nonumber \\[6pt]
%
& =      \frac{1}{(2 \pi)^M} \int \mathrm{d} \alpha_1 \cdots \int \mathrm{d} \alpha_M \Biggl[  \exp \left\{ - \sum_{l=1}^M \left[\frac{1}{4} \alpha_l^2 \bigl(f_l, f_l \bigr) + i \alpha_l w_l  \right] \right\} \underbrace{\sum_{\mu, \nu} \phi_\mu^* \phi_\nu \bigl( \delta_{\mu \nu} - \varphi_{\mu}  \varphi_{\nu}^*  \bigr)}_{= \, ( \phi, \phi ) - |( \phi, \varphi )|^2} \Biggr] \nonumber  \\[6pt]
%
& =  \underbrace{\bigl( \phi, \phi \bigr)}_{= \; 1} p_\mathbf{W}(0,\mathbf{w}) \nonumber \\[6pt]
%
& \phantom{= .} - \frac{1}{(2 \pi)^M} \int \mathrm{d} \alpha_1 \cdots \int \mathrm{d} \alpha_M \, |( \phi, \varphi )|^2 \exp \left\{ - \sum_{l=1}^M \left[\frac{1}{4} \alpha_l^2 \bigl(f_l, f_l \bigr) + i \alpha_l w_l  \right] \right\}    ,
%
\end{align}
%
where
%
\begin{align}\label{s470}
%
|( \phi, \varphi )|^2 & =   \frac{1}{2} \sum_{n,m=1}^M \alpha_n \alpha_m \bigl( \phi, f_n \bigr) \bigl( f_m, \phi \bigr) ,
%
\end{align}
%
because of  \eqref{s290}. For the calculation of the multiple integral in \eqref{s460} it is useful to separate the terms with $n = m$ in \eqref{s470} from the rest of the sum, using the trivial identity
%
\begin{align}\label{s475}
%
\left(\sum_{n=1}^M a_n \right) \left(\sum_{m=1}^M b_m \right) & =   \sum_{n=1}^M a_n b_n +   \sum_{n=1}^M \sum_{m \neq n}  a_n b_m .
%
\end{align}
%
Applying this formula to \eqref{s470}, we find
%
\begin{align}\label{s480}
%
|( \phi, \varphi )|^2 & =   \frac{1}{2} \sum_{n=1}^M \alpha_n^2 \,  |\bigl( \phi, f_n \bigr)|^2 +  \frac{1}{2} \sum_{n=1}^M \sum_{m \neq n}  \alpha_n \alpha_m \bigl( \phi, f_n \bigr) \bigl( f_m, \phi \bigr) .
%
\end{align}
%
Substituting \eqref{s480} into \eqref{s460}, we obtain
%
\begin{align}\label{s490}
%
p_\mathbf{W} (1,\mathbf{w}) & =     p_\mathbf{W}(0,\mathbf{w}) -  \frac{1}{2} \sum_{n=1}^M |\bigl( \phi, f_n \bigr)|^2    \int \frac{\mathrm{d} y }{2 \pi} \, y^2 \,  e^{-   y^2 \frac{(f_n, f_n ) }{4} - i y w_n }\prod_{\substack{l=1 \\ l \neq n }}^M \underbrace{\int \frac{\mathrm{d} \alpha_l}{2 \pi} \, e^{-   \alpha_l^2 \frac{(f_l, f_l )}{4} - i \alpha_l w_l }}_{= \, p_{W_l}(0,w_l)}
\nonumber \\[6pt]
%
& \phantom{=} - \frac{1}{2} \sum_{n=1}^M \sum_{m \neq n}  \bigl( \phi, f_n \bigr) \bigl( f_m, \phi \bigr) \int \frac{\mathrm{d} x }{2 \pi} \, x \, e^{-   x^2 \frac{(f_n, f_n )}{4} - i x w_n }  \int \frac{\mathrm{d} y }{2 \pi} \, y \, e^{-   y^2 \frac{(f_m, f_m )}{4} - i y w_m } \nonumber \\[6pt]
%
& \phantom{=} \times \prod_{\substack{l=1 \\ l \neq n, m }}^M \underbrace{ \int \frac{\mathrm{d} \alpha_l}{2 \pi} \, e^{- \alpha_l^2 \frac{(f_l, f_l )}{4} - i \alpha_l w_l } }_{= \, p_{W_l}(0,w_l)}  ,
%
\end{align}
%
where \eqref{s350} has been applied. Using the Gaussian integrals  Eqs. (3.17-3.18) in \cite{SchulmanBook}, namely,
%
\begin{align}\label{s500}
%
\int_{-\infty}^{\infty} y \, \exp\left( -a y^2 + b y \right ) \mathrm{d} y  & =  \frac{b}{2a} \,  \sqrt{ \frac{\pi}{a} } \exp\left( \frac{b^2}{4 a}\right ),
%
\end{align}
%
and
%
\begin{align}\label{s510}
%
\int_{-\infty}^{\infty} y^2 \, \exp\left( -a y^2 + b y \right ) \mathrm{d} y   & =  \frac{1}{2a} \, \left( 1 + \frac{b^2}{2a}\right) \,  \sqrt{ \frac{\pi}{a} } \exp\left( \frac{b^2}{4 a}\right ),
%
\end{align}
%
respectively, where $a,b \in \mathbb{C}$, with $\text{Re} \, a > 0$, we rewrite
%
\begin{align}\label{s495}
%
p_\mathbf{W} (1,\mathbf{w}) & =     p_\mathbf{W}(0,\mathbf{w}) -  \frac{1}{2} \sum_{n=1}^M |\bigl( \phi, f_n \bigr)|^2   p_{W_n}(0,w_n) \, \frac{2}{\bigl(f_n,f_n \bigr)} \left[ 1 - \frac{2 w^2_n}{\bigl(f_n,f_n \bigr)}\right] \prod_{\substack{l=1 \\ l \neq n }}^M p_{W_l}(0,w_l)
\nonumber \\[6pt]
%
&- \frac{1}{2} \sum_{n=1}^M \sum_{m \neq n}  \bigl( \phi, f_n \bigr) \bigl( f_m, \phi \bigr) p_{W_n}(0,w_n) \,\frac{-2 i w_n}{\bigl(f_n,f_n \bigr)}  \, p_{W_m}(0,w_m) \frac{-2 i w_m}{\bigl(f_m,f_m \bigr)} \prod_{\substack{l=1 \\ l \neq n, m }}^M p_{W_l}(0,w_l) \nonumber \\[6pt]
%
%
& =    p_\mathbf{W}(0,\mathbf{w}) \left\{ 1 -   \sum_{n=1}^M \frac{|\bigl( \phi, f_n \bigr)|^2}{\bigl(f_n,f_n \bigr)}    \left[ 1 - \frac{2 w^2_n}{\bigl(f_n,f_n \bigr)}\right] \right. \nonumber \\[6pt]
%
& \left. \phantom{=}
+ 2  \sum_{n=1}^M \sum_{m \neq n} w_n w_m   \frac{\bigl( \phi, f_n \bigr) }{\bigl(f_n,f_n \bigr)}   \frac{ \bigl( f_m, \phi \bigr) }{\bigl(f_m,f_m \bigr)} \right\}.
%
\end{align}
%
Next,  we define
%
\begin{align}\label{s520}
%
\sigma_n^2 = \frac{1}{2} (f_n,f_n), \qquad    \text{and} \qquad s_n = \frac{(\phi,f_n)}{(f_n,f_n)^{1/2}}  ,
%
\end{align}
%
to rewrite \eqref{s490} as
%
\begin{align}\label{s530}
%
p_\mathbf{W} (1,\mathbf{w}) & =  p_\mathbf{W}(0,\mathbf{w}) \left[ 1 -   \sum_{n=1}^M |s_n|^2 \left( 1 - \frac{ w^2_n}{\sigma_n^2}\right) +   \sum_{n=1}^M \sum_{m \neq n} \frac{w_n}{\sigma_n} \, \frac{w_m}{\sigma_m}  \,  s_n s_m^*  \right] \nonumber \\[6pt]
%
& =    p_\mathbf{W}(0,\mathbf{w}) \left[ 1 -   \sum_{n=1}^M |s_n|^2 + \left| \sum_{n=1}^M \, \frac{w_n}{\sigma_n} \, s_n \right|^2 \right],
%
\end{align}
%
where we have used \eqref{s475} backwards to reconstruct the modulus square of the sum.



\section{Probability distribution for the particle operators}\label{pdfC}


Here we calculate step by step the following expression for $p_\mathbf{C}(N ,\mathbf{c})$:
%
\begin{align}\label{s540}
%
p_\mathbf{C}(N ,\mathbf{c})& =  \langle N[\phi] | \prod_{n = 1}^M \delta \bigl( \hat{C}_n - {c}_n \bigr) | N[\phi] \rangle.
%
\end{align}
%

\subsection{Vacuum state}


In this case we have
%
\begin{align}\label{s550}
%
p_\mathbf{C}(0,\mathbf{c}) & =  \langle 0 | \prod_{n = 1}^M \delta \bigl( \hat{C}_n - c_n \bigr) | 0 \rangle \nonumber \\[6pt]
%
& =  \frac{1}{(2 \pi)^M} \int \mathrm{d} \alpha_1 \cdots \int \mathrm{d} \alpha_M \, \exp\left( - i \sum_{n=1}^M \alpha_n c_n \right) \langle 0 | e^{i \alpha_1 \hat{C}_1 } e^{i \alpha_2 \hat{C}_2 } \cdots e^{i \alpha_M \hat{C}_M }| 0 \rangle \nonumber \\[6pt]
%
& =  \frac{1}{(2 \pi)^M} \int \mathrm{d} \alpha_1 \cdots \int \mathrm{d} \alpha_M \, \exp\left( - i \sum_{n=1}^M \alpha_n c_n \right) \underbrace{\langle 0 | \exp\left(  i \sum_{n=1}^M \alpha_n \hat{C}_n \right) | 0 \rangle}_{= \, 1}\nonumber \\
%
& =  \prod_{n=1}^M \delta(c_n),
%
\end{align}
%
because $\hat{C}_n  | 0 \rangle = 0$, and  \eqref{s210} has been used.
%


\subsection{Single photon state}


In this case, we must evaluate
%
\begin{align}\label{s555}
%
p_\mathbf{C} (1,\mathbf{c}) & =  \langle 1[\phi] | \prod_{n = 1}^M \delta \bigl( \hat{C}_n - c_n \bigr) | 1[\phi] \rangle \nonumber \\[6pt]
%
& =  \sum_{\mu, \nu} \phi_\mu^* \phi_\nu \langle 0 | \hat{a}_\mu \prod_{n = 1}^M \delta \bigl( \hat{C}_n - c_n \bigr) \hat{a}_\nu^\dagger | 0 \rangle  ,
%
\end{align}
%
where
%
\begin{multline}\label{s557}
%
\langle 0 | \hat{a}_\mu \prod_{n = 1}^M \delta \bigl( \hat{C}_n - c_n \bigr) \hat{a}_\nu^\dagger | 0 \rangle \\[6pt]
%
= \int \frac{\mathrm{d}  \alpha_1}{2 \pi} \cdots \int \frac{\mathrm{d}  \alpha_M}{2 \pi} \, \exp\left( - i \sum_{n=1}^M \alpha_n c_n \right)  \langle 0 | \hat{a}_\mu \exp\left(  i \sum_{n=1}^M \alpha_n \hat{C}_n \right) \hat{a}_\nu^\dagger| 0 \rangle.
%
\end{multline}
%
We need to calculate explicitly the last term of the previous equation, that is
%
\begin{align}\label{c10}
%
\langle 0 | \hat{a}_\mu \exp\left(  i \sum_{n=1}^M \alpha_n \hat{C}_n \right) \hat{a}_\nu^\dagger| 0 \rangle & = \langle 0 | \hat{a}_\mu \, e^{x \hat{A}} \hat{B}| 0 \rangle \nonumber \\[6pt]
%
& = \langle 0 | \hat{a}_\mu \, \left( e^{x \hat{A}} \hat{B} \, e^{-x \hat{A}}\right) e^{x \hat{A}}| 0  \rangle \nonumber \\[6pt]
%
& = \langle 0 | \hat{a}_\mu \, \left( e^{x \hat{A}} \hat{B} \, e^{-x \hat{A}}\right)| 0  \rangle ,
%
\end{align}
%
where we have defined
%
\begin{align}\label{c20}
%
x = i, \qquad \hat{A} = \sum_{n=1}^M \alpha_n \hat{C}_n , \qquad \text{and} \qquad \hat{B} = \hat{a}_\nu^\dagger.
%
\end{align}
%
Moreover, we have used \eqref{s140} to calculate  $\hat{C}_n | 0 \rangle = 0$, which implies $\exp ( {x \hat{A}} )| 0 \rangle = | 0 \rangle$.

Next, to calculate \eqref{c10} we need to use equation (10.11-1) in \cite{MandelBook}:
%
\begin{align}\label{s590}
%
\exp \bigl( x \hat{A} \bigr) \hat{B} \exp \bigl( -x \hat{A} \bigr) & =  \hat{B} + x \bigl[ \hat{A} , \, \hat{B} \bigr] + \frac{x^2}{2!} \bigl[ \hat{A} ,\bigl[ \hat{A} , \, \hat{B} \bigr]\bigr] + \ldots \;.
%
\end{align}
%
To this end, first we evaluate the commutator
%
\begin{align}\label{s620}
%
\bigl[\hat{A} , \, \hat{B} \bigr] & = \sum_{n=1}^M \alpha_n \Bigl[ \hat{C}_n , \,  \hat{a}_\nu^\dagger \Bigr] \nonumber \\[6pt]
%
& =  \sum_{n=1}^M \alpha_n \sum_{\gamma, \beta} \bm{1}_{n \gamma \beta} \underbrace{\Bigl[\hat{a}_\gamma^\dagger \hat{a}_\beta , \,  \hat{a}_\nu^\dagger \Bigr]}_{= \; \delta_{\beta \nu} \hat{a}^\dagger_\gamma} \nonumber \\[6pt]
%
& =  \sum_{n=1}^M \alpha_n \underbrace{\sum_{\gamma} \bm{1}_{n \gamma \nu}  \hat{a}^\dagger_\gamma}_{\deff \;\hat{\phi}_{n  \nu}^\dagger } \nonumber \\[6pt]
%
& = \sum_{n=1}^M \alpha_n \hat{\phi}_{n  \nu}^\dagger,
%
\end{align}
%
where we have used the relation
%
\begin{align}\label{s630}
%
\bigl[\hat{A} \hat{B} , \, \hat{C} \bigr] & =   \hat{A}  \bigl[\hat{B} , \, \hat{C} \bigr]  + \bigl[\hat{A}, \, \hat{C}  \bigr]\hat{B},
%
\end{align}
%
with $\hat{A} = \hat{a}_\gamma^\dagger, \, \hat{B} =  \hat{a}_\beta$ and $\hat{C} = \hat{a}_\nu^\dagger$, and we have defined
%
\begin{align}\label{s635}
%
\hat{\phi}_{n  \nu}^\dagger \deff \Bigl[ \hat{C}_n , \,  \hat{a}_\nu^\dagger \Bigr] = \sum_{\gamma} \bm{1}_{n \gamma \nu}  \hat{a}^\dagger_\gamma = \bigl( u_\nu, \bm{1}_n \hat{\phi}^\dagger \bigr),
%
\end{align}
%
where \eqref{n10} and \eqref{a222} have been used.
%
The next commutator is
%
\begin{align}\label{s640}
%
\bigl[\hat{A} , \,\bigl[\hat{A} , \, \hat{B} \bigr] \bigr] & =   \sum_{n=1}^M \alpha_n \Bigl[ \hat{C}_n , \,  \sum_{m=1}^M \alpha_m \hat{\phi}_{m \nu}^\dagger \Bigr]  \nonumber \\[6pt]
%
& =     \sum_{n, m=1}^M \alpha_n \alpha_m \Bigl[ \hat{C}_n , \,  \hat{\phi}_{m \nu}^\dagger \Bigr]\nonumber \\[6pt]
%
& =     \sum_{n, m=1}^M \alpha_n \alpha_m \sum_{\gamma} \bm{1}_{m \gamma \nu}  \underbrace{\Bigl[ \hat{C}_n , \,  \hat{a}^\dagger_\gamma \Bigr]}_{= \; \hat{\phi}_{n \gamma}^\dagger} \nonumber \\[6pt]
%
& =     \sum_{n, m=1}^M \alpha_n \alpha_m \sum_{\gamma} \bm{1}_{m \gamma \nu}  \hat{\phi}_{n \gamma}^\dagger.
%
\end{align}
%
Now we note that from \eqref{s635} we have
%
\begin{align}\label{s650}
%
\sum_{\gamma} \bm{1}_{m \gamma \nu}   \hat{\phi}_{n \gamma}^\dagger & =   \sum_{\gamma} \bm{1}_{m \gamma \nu}  \sum_{\tau} \bm{1}_{n \tau \gamma}  \hat{a}^\dagger_\tau  \nonumber \\[6pt]
%
& =  \sum_{\tau} \left(\sum_{\gamma} \bm{1}_{m \gamma \nu}  \bm{1}_{n \tau \gamma} \right)  \hat{a}^\dagger_\tau  \nonumber \\[6pt]
%
& =  \delta_{nm} \, \sum_{\tau} \bm{1}_{n \tau \nu}  \hat{a}^\dagger_\tau  \nonumber \\[6pt]
%
& =  \delta_{nm} \hat{\phi}_{n  \nu}^\dagger ,
%
\end{align}
%
where \eqref{s120},\eqref{s150} and \eqref{s200} have been used. Substituting \eqref{s250} into \eqref{s640}, we obtain
%
\begin{align}\label{s660}
%
\bigl[\hat{A} , \,\bigl[\hat{A} , \, \hat{B} \bigr] \bigr] & =    \sum_{n=1}^M \alpha_n^2 \hat{\phi}_{n  \nu}^\dagger.
%
\end{align}
%
We can iterate the procedure above $k$ times to find
%
\begin{align}\label{s665}
%
\bigl[ \underset{\vphantom{\bigl[} k}{\hat{A}} ,  \, \bigl[ \underset{\vphantom{\bigl[} k-1}{\hat{A}} , \, \ldots \bigl[\underset{\vphantom{\bigl[} 2}{\hat{A}} , \, \bigl[\underset{\vphantom{\bigl[} 1}{\hat{A}} , \, \hat{B} \bigr] \bigr]  \ldots \bigr] \bigr] & =     \sum_{n=1}^M \alpha_n^k \hat{\phi}_{n  \nu}^\dagger, \qquad (k=1,2, \ldots, D),
%
\end{align}
%
so that
%
\begin{align}\label{s670}
%
e^{i \hat{A}} \hat{a}_\nu^\dagger e^{-i \hat{A}}  & =   \hat{a}_\nu^\dagger  + i  \sum_{n=1}^M \alpha_n \hat{\phi}_{n  \nu}^\dagger  + \frac{i^2}{2!}  \sum_{n=1}^M \alpha_n^2 \hat{\phi}_{n  \nu}^\dagger +  \ldots  \nonumber \\[6pt]
%
& =  \hat{a}_\nu^\dagger  +  \sum_{n=1}^M \hat{\phi}_{n  \nu}^\dagger \left( i \,\alpha_n   + \frac{i^2}{2!} \, \alpha_n^2  +  \ldots \right) \nonumber \\[6pt]
%
& =  \hat{a}_\nu^\dagger - \sum_{n=1}^M \hat{\phi}_{n  \nu}^\dagger +  \sum_{n=1}^M \hat{\phi}_{n  \nu}^\dagger \left( 1 + i \,\alpha_n   + \frac{i^2}{2!} \, \alpha_n^2  +  \ldots \right) \nonumber \\[6pt]
%
& =  \hat{a}_\nu^\dagger - \sum_{n=1}^M \hat{\phi}_{n  \nu}^\dagger +  \sum_{n=1}^M \hat{\phi}_{n  \nu}^\dagger \exp \left( i \alpha_n \right).
%
\end{align}
%
Substituting \eqref{s670} into \eqref{c10}, we obtain
%
\begin{align}\label{s680}
%
\langle 0 | \hat{a}_\mu \exp\left(  i \sum_{n=1}^M \alpha_n \hat{C}_n \right) \hat{a}_\nu^\dagger| 0 \rangle & =   \langle 0 | \hat{a}_\mu \left[ \hat{a}_\nu^\dagger - \sum_{n=1}^M \hat{\phi}_{n  \nu}^\dagger +  \sum_{n=1}^M \hat{\phi}_{n  \nu}^\dagger \exp \left( i \alpha_n \right) \right]   | 0 \rangle \nonumber \\[6pt]
%
& =  \langle 0 | \hat{a}_\mu  \hat{a}_\nu^\dagger | 0 \rangle  - \sum_{n=1}^M  \langle 0 | \hat{a}_\mu \hat{\phi}_{n  \nu}^\dagger  | 0 \rangle  +  \sum_{n=1}^M e^{ i \alpha_n } \langle 0 |\hat{a}_\mu  \hat{\phi}_{n  \nu}^\dagger     | 0 \rangle  \nonumber \\[6pt]
%
& =  \delta_{\mu \nu}  - \sum_{n=1}^M  \bm{1}_{n \mu \nu}  +  \sum_{n=1}^M e^{ i \alpha_n } \bm{1}_{n \mu \nu},
%
\end{align}
%
where we have used \eqref{s635} to calculate
%
\begin{align}\label{s685}
%
\langle 0 |\hat{a}_\mu  \hat{\phi}_{n  \nu}^\dagger     | 0 \rangle & =    \sum_{\gamma} \bm{1}_{n \gamma \nu} \langle 0 |\hat{a}_\mu  \hat{a}^\dagger_\gamma    | 0 \rangle \nonumber \\[6pt]
%
& =   \sum_{\gamma} \bm{1}_{n \gamma \nu} \delta_{\mu \gamma}   \nonumber \\[6pt]
%
& = \bm{1}_{n \mu \nu}.
%
\end{align}
%
Inserting \eqref{s680} into \eqref{s555}, we obtain
%
\begin{align}\label{s700}
%
p_\mathbf{C} (1,\mathbf{c}) & =   \int \frac{\mathrm{d}  \alpha_1}{2 \pi} \cdots \int \frac{\mathrm{d}  \alpha_M}{2 \pi} \left\{ \exp \left( - i \sum_{n=1}^M \alpha_n c_n \right) \right. \nonumber \\[6pt]
  %
&\phantom{=} \times \left.  \sum_{\mu, \nu} \phi_\mu^* \phi_\nu \left( \delta_{\mu \nu}  - \sum_{m=1}^M  \bm{1}_{m \mu \nu}  +  \sum_{m=1}^M e^{ i \alpha_m } \bm{1}_{m \mu \nu} \right) \right\}.
%
\end{align}
%
To evaluate this expression we need to calculate
%
\begin{align}\label{s710}
%
\sum_{\mu, \nu} \phi_\mu^* \phi_\nu  \bm{1}_{m \mu \nu} & = \sum_{\mu, \nu} \phi_\mu^* \phi_\nu  (u_\mu, \bm{1}_m u _\nu) \nonumber \\[6pt]
%
& =  \left( \sum_{\mu} \phi_\mu u_\mu, \bm{1}_m \sum_{\nu} \phi_\nu u _\nu \right) \nonumber \\[6pt]
%
& =  \left( \phi, \bm{1}_m \phi \right)\nonumber \\[6pt]
%
& \deff  P_m \geq 0,
%
\end{align}
%
where \eqref{s150} has been used. Using this result, we can rewrite \eqref{s700} as,
%
\begin{align}\label{s720}
%
p_\mathbf{C} (1,\mathbf{c}) & =   \prod_{n=1}^M \delta \left( c_n \right) \left[\underbrace{\left(\phi, \phi \right)}_{=\; 1}  - \sum_{m=1}^M P_m \right] \nonumber \\[6pt]
%
& \phantom{=} + \sum_{m=1}^M P_m \int \frac{\mathrm{d}  \alpha_m}{2 \pi} \exp \left[ - i (\alpha_m -1) c_m \right] \mathrm{d} \alpha_m \prod_{\substack{n=1 \\[3pt] n \neq m}}^M
\int \frac{\mathrm{d}  \alpha_n}{2 \pi} \exp \left( - i  \alpha_n  c_n \right) \mathrm{d} \alpha_n  \nonumber \\[6pt]
%
& = \prod_{n=1 }^M \delta \left( c_n \right) \left( 1  - \sum_{m=1}^M P_m \right) +
\sum_{m=1}^M P_m \, \delta \left( c_m - 1 \right) \prod_{\substack{n=1 \\[3pt] n \neq m}}^M \delta \left( c_n \right) \nonumber \\[6pt]
%
& \deff P_0 \, \prod_{n=1 }^M \delta \left( c_n \right) +
\sum_{m=1}^M P_m \, \delta \left( c_m - 1 \right) \prod_{\substack{n=1 \\[3pt] n \neq m}}^M \delta \left( c_n \right),
%
\end{align}
%
where \eqref{s250} has been used, and we have defined the probability $P_0$ of zero counting in all photodetectors, as
%
\begin{align}\label{s730}
%
P_0 =   1  - \sum_{m=1}^M P_m.
%
\end{align}
%
Note that $p (1,\mathbf{c})$ is correctly normalized because \eqref{s730} trivially implies
%
\begin{align}\label{s740}
%
\sum_{m=0}^M P_m =1.
%
\end{align}
%
The meaning of this equation is that given the  single-photon state $| 1[\phi] \rangle$, either we get a ``click'' in one of the $M$ detectors, or not.


\section{Probability distribution for the wave-particle operator}\label{pdfWC}



In this appendix we calculate explicitly  $p_\mathbf{C}(N ,\mathbf{c})$, defined by
%
\begin{align}\label{d10}
%
p_{WC}(N,w, c)   = \langle N[\phi] | \delta \bigl( \hat{W}_1 - w \bigr)  \delta \bigl( \hat{C}_2 - c \bigr) | N[\phi] \rangle.
%
\end{align}
%

\subsection{Vacuum state}


In this case we have
%
\begin{align}\label{d20}
%
p_{WC}(0,w, c) & =  \langle 0  | \delta \bigl( \hat{W}_1 - w \bigr)  \delta \bigl( \hat{C}_2 - c \bigr) |  | 0 \rangle \nonumber \\[6pt]
%
& =  \frac{1}{(2 \pi)^2} \int \mathrm{d} \alpha_1  \int \mathrm{d} \alpha_2 \, e^{ - i \left( \alpha_1 w +  \alpha_2 c \right) } \langle 0 | e^{i \alpha_1 \hat{W}_1 } \underbrace{ e^{i \alpha_2 \hat{C}_2 }| 0 \rangle}_{= \;| 0 \rangle } \nonumber \\[6pt]
%
& =  \delta(c) \, \frac{1}{(2 \pi)} \int \mathrm{d} \alpha_1  \, e^{ - i  \alpha_1 w  } \langle 0 | e^{i \alpha_1 \hat{W}_1 } | 0 \rangle  \nonumber \\[6pt]
%
& =  p_{W}(0,w) \, p_{C}(0, c),
%
\end{align}
%
because $\hat{C}_2  | 0 \rangle = 0$, and  \eqref{s260} has been used.
%


\subsection{Single photon state}


In this case, we must evaluate
%
\begin{align}\label{d30}
%
p_{WC}(1,w, c) & =  \langle 1[\phi] |  \delta \bigl( \hat{W}_1 - w \bigr)  \delta \bigl( \hat{C}_2 - c \bigr)  | 1[\phi] \rangle \nonumber \\[6pt]
%
& =  \sum_{\mu, \nu} \phi_\mu^* \phi_\nu \langle 0 | \hat{a}_\mu \delta \bigl( \hat{W}_1 - w \bigr)  \delta \bigl( \hat{C}_2 - c \bigr)  \hat{a}_\nu^\dagger | 0 \rangle  ,
%
\end{align}
%
where
%
\begin{align}\label{d40}
%
\langle 0 | \hat{a}_\mu \delta \bigl( \hat{W}_1 - w \bigr)  \delta \bigl( \hat{C}_2 - c \bigr)  \hat{a}_\nu^\dagger | 0 \rangle  =
\int \frac{\mathrm{d}  \alpha_1}{2 \pi} \int \frac{\mathrm{d}  \alpha_2}{2 \pi} \, e^{ - i \left( \alpha_1 w +  \alpha_2 c \right) }  \langle 0 | \hat{a}_\mu e^{i \alpha_1 \hat{W}_1 }  e^{i \alpha_2 \hat{C}_2} \hat{a}_\nu^\dagger| 0 \rangle.
%
\end{align}
%
We need to calculate explicitly the last term of the previous equation, that is
%
\begin{align}\label{d50}
%
\langle 0 | \hat{a}_\mu e^{i \alpha_1 \hat{W}_1 }  e^{i \alpha_2 \hat{C}_2} \hat{a}_\nu^\dagger| 0 \rangle & = \langle 0 | \hat{a}_\mu e^{i \alpha_1 \hat{W}_1 } \left( e^{i \alpha_2 \hat{C}_2} \hat{a}_\nu^\dagger e^{-i \alpha_2 \hat{C}_2} \right) | 0 \rangle \nonumber \\[6pt]
%
& = \langle 0 | \hat{a}_\mu \, e^{i \alpha_1 \hat{W}_1 } \hat{a}_\nu^\dagger | 0  \rangle - \left( 1-e^{i \alpha_2}\right) \langle 0 | \hat{a}_\mu \, e^{i \alpha_1 \hat{W}_1 } \hat{\phi}_{2 \mu}^\dagger | 0  \rangle  ,
%
\end{align}
%
where all the quantities are defined as in the previous two appendices.
%
The rest of the calculation is very straightforward and yields:
%
\begin{align}\label{d60}
%
p _{W\!C}  (1,w,c)   = p_{W}(1,w) \, p_{C}(0,c) + p_{W}(0,w) \, p_{C}(1,c)   - p_{W}(0,w) \, p_{C}(0,c).
%
\end{align}
%


\end{document}
