% ****** Start of file apssamp.tex ******
%
%   This file is part of the APS files in the REVTeX 4 distribution.
%   Version 4.0 of REVTeX, August 2001
%
%   Copyright (c) 2001 The American Physical Society.
%
%   See the REVTeX 4 README file for restrictions and more information.
%
% TeX'ing this file requires that you have AMS-LaTeX 2.0 installed
% as well as the rest of the prerequisites for REVTeX 4.0
%
% See the REVTeX 4 README file
% It also requires running BibTeX. The commands are as follows:
%
%  1)  latex apssamp.tex
%  2)  bibtex apssamp
%  3)  latex apssamp.tex
%  4)  latex apssamp.tex
%
\documentclass[aps,prl,twocolumn,superscriptaddress]{revtex4-1}
%\documentclass[aps,prb,twocolumn,superscriptaddress]{revtex4-1}
\usepackage{amssymb, amsmath, bm}
%\usepackage{url,booktabs,graphicx, textcomp,natbib}
%\documentclass[preprint,showpacs,preprintnumbers,amsmath,amssymb]{revtex4}
\usepackage{graphicx,onlyamsmath}

% Some other (several out of many) possibilities
%\documentclass[preprint,aps]{revtex4}
%\documentclass[preprint,aps,draft]{revtex4}
%\documentclass[prb]{revtex4}% Physical Review B
\usepackage{natbib}
\usepackage{xcolor}
\usepackage[normalem]{ulem}  % ��� ����������� ������
%\nofiles

\begin{document}

%\title{Higher-order topological phase transition in quantum spin Hall insulators induced by parallel magnetic field}
%\title{Corner states induced by magnetic field in quantum spin Hall insulators in the presence and in the absence of $S$-wave superconductivity}
%\title{Topological corner states induced by magnetic field in quantum spin Hall insulators and their hybrid superconductor devices}

\title{Magnetic-field-induced corner states in quantum spin Hall insulators}

\author{Sergey S.~Krishtopenko}
\email[]{sergey.krishtopenko@umontpellier.fr}
\affiliation{Laboratoire Charles Coulomb (L2C), UMR 5221 CNRS-Universit\'{e} de Montpellier, F- 34095 Montpellier, France}

\author{Fr\'{e}d\'{e}ric Teppe}
\affiliation{Laboratoire Charles Coulomb (L2C), UMR 5221 CNRS-Universit\'{e} de Montpellier, F- 34095 Montpellier, France}
\date{\today}% It is always \today, today,
       %  but any date may be explicitly specified

\begin{abstract}
%A two-dimensional higher-order topological states exhibits a finite gap in both bulk and edges, with the zero-dimensional modes localized at the corners.
We treat the general problem of magnetic-field-induced corner states in quantum spin Hall insulators based on zinc-blende semiconductor quantum wells (QWs). An analysis performed within the \emph{continuous} Bernevig-Hughes-Zhang (BHZ) model reveals that the gapped edge states are described by ``generalized'' 1D Dirac equation with \emph{two mass parameters}, whose values depend on crystallographic orientation of the edge and that of the magnetic field. Although the mass parameters do not vanish simultaneously, an analytical solution in the form of ``topological domain wall mode'' confirms the existence of corner state at the intersection of two adjacent edges. Surprisingly, the existence of the corner states induced by an in-plane magnetic field do not require a crystal symmetry of zinc-blende semiconductor QW, making our results universal for any quantum spin Hall insulators with isotropic edge-state g-factor. On the contrary, the corner states induced by an out-of-plane magnetic field arise only due to the absence of an inversion center in zinc-blende semiconductor QWs.
\end{abstract}

\pacs{73.21.Fg, 73.43.Lp, 73.61.Ey, 75.30.Ds, 75.70.Tj, 76.60.-k} % PACS, the Physics and Astronomy
                             % Classification Scheme.
                             
\keywords{zinc-blende semiconductors, corner states, topological insulators}
%Use showkeys class option if keyword                            %display desired
\maketitle

\emph{Introduction.}--
%\section{\label{Sec:Int} Introduction}
Recently, a novel classification of topological systems has led to the discovery of higher-order topological insulators (HOTIs)~\cite{cst1,cst2,cst3,cst4,cst5}. While conventional (first-order) $m$-dimensional topological insulators~\cite{cst6,cst7,cst8,cst9,cst10} exhibit gapless states at their ($m-1$)-dimensional boundaries, $n$th-order $m$-dimensional topological insulator possesses gapless edge states at its ($m-n$)-dimensional boundaries. To date, HOTI state has been observed in bismuth~\cite{cst11}, WTe$_2$~\cite{cst12}, bismuth-halide chains~\cite{cst13}, Bi$_4$Br$_4$~\cite{cst14} and predicted in many other systems including bulk crystals and quantum wells (QWs)~\cite{cst15,cst17,cst18,cst19,cst16,cst20}. Similarly, the conception of higher-order topology can also be applied to photonic~\cite{cst26,cst27,cst28} and acoustic crystals~\cite{cst29,cst30,cst31}, topoelectrical-circuits~\cite{cst32,cst33,cst34,cst34b} and superconductors~\cite{cst21,cst22,cst23,cst24,cst25}.

The widespread way to construct higher-order topological state is to consider a conventional topological system of specific crystal symmetry and to break time reversal symmetry.
%This results in an effective (Dirac) mass of the edge states of the system~\cite{cst47a00,cst47,cst48}.
As a result, a zero-dimensional (0D) corner state appears at the intersection of two edges with masses of opposite signs, corresponding to the Jackiw-Rebbi topological domain wall mode~\cite{cst42}. Previously, such a mechanism of the realization of corner was theoretically proposed for the QSHIs described within square~\cite{cst47a00} and honeycomb~\cite{cst47a00,cst47,cst48} lattice models. The crystal symmetry of many QSHIs may differ from the symmetry involved in 2D lattice models. A striking example of such systems are QSHIs based on zinc-blende semiconductor QWs, like HgTe/CdTe~\cite{cst49,cst50} and InAs/Ga(In)Sb quantum wells (QWs)~\cite{cst51,cst52,cst53}. For instance, the zinc-blende QWs grown along [001] crystallographic orientation have the $D_{2d}$ or $C_{2v}$ point group, depending on whether the QW heteropotential is symmetric or asymmetric, which both differ from the $D_4$ point group of simple 2D square lattice. Additionally, the concrete choice of QSHI is crucial for the Zeeman field, since the g-factor tensor depends on the crystal symmetry of the system.

In this paper, we analytically solve the general problem of magnetic-field-induced corner states in quantum spin Hall insulators (QSHIs) based on zinc-blende semiconductor QWs. We find that in the most general case, the gapped edge states are described by a ``generalized'' 1D Dirac equation with \emph{two mass parameters} at once. This equation being applied to the intersecting edges, however, can be solved analytically in the form of ``topological domain wall mode'', supporting the existence of the corner state. One of the unexpected results of this work is that the existence condition of the corner states induced by in-plane magnetic field does not require crystal symmetry of zinc-blende semiconductor QWs. Moreover, since our general analytical results do not require any particular form of the edge g-factor tensor, they can also be applied to describe corner states in QSHIs with a crystal symmetry different from that of zinc-blende semiconductor QWs -- for instance, the QSHIs, which were previously treated within the lattice models~\cite{cst47a00,cst47,cst48}.

\emph{Hamiltonian for zinc-blende QSHIs.}--
%\section{\label{Sec:2} The absence of $s$-wave superconductivity}
%\subsection{\label{Sec:2a} Gapped edge states induced by in-plane magnetic field}
First, we focus on prototype zinc-blende semiconductor QW with symmetric heteropotential grown along the $z||[001]$ axis. The low-energy Hamiltonian of such systems for the electron-like $E1$ and heavy-hole-like $H1$ subbands, written in the basis $|E1{\uparrow}\rangle$, $|H1{\uparrow}\rangle$, $|E1{\downarrow}\rangle$, $|H1{\downarrow}\rangle$, has the form~\cite{cst49}:
\begin{equation}
\label{eq:1}
H_{\mathrm{2D}}(\mathbf{k})=\begin{pmatrix}
H_{\mathrm{BHZ}}(\mathbf{k}) & -i\Delta{e^{-2i\theta}}\sigma_y \\
i\Delta{e^{2i\theta}}\sigma_y & H_{\mathrm{BHZ}}^{*}(-\mathbf{k})\end{pmatrix},
\end{equation}
where asterisk stands for complex conjugation, $\mathbf{k}=(k_x,k_y)$ is the momentum in the plane, and $H_{\mathrm{BHZ}}(\mathbf{k})=\epsilon_{k}+d_a(\mathbf{k})\sigma_a$. Here, $\sigma_a$ are the Pauli matrices acting in the ``basis'' space, $\epsilon_{k}=C-\mathbb{D}k^2$, $d_1(\mathbf{k})=Ak_x$, $d_2(\mathbf{k})=Ak_y$, $d_3(k)=M-\mathbb{B}k^2$ and $k^2=k_x^2+k_y^2$. The mass parameter $M$ describes the inversion between the electron-like \emph{E}1 and the hole-like \emph{H}1 subbands: $M>0$ corresponds to a trivial state, while $M<0$ is for QSHI~\cite{cst49}. The non-diagonal terms in $H_{\mathrm{2D}}(\mathbf{k})$ proportional to $\Delta$ describe the joint effect of bulk inversion asymmetry (BIA) of the unit cell of zinc-blende semiconductors~\cite{cst54} and the interface inversion asymmetry (IIA) of the QW interfaces~\cite{cst55,cst56}. Experimental results known from the literature show that the values of $\Delta$ are small in HgTe/CdTe QWs~\cite{A1,A2,A3,A4,A5,A6,A7}, while they can be relatively large in InAs/GaInSb-based heterostructures~\cite{A8,A9,A10,A11,A12,A13}.

In $H_{\mathrm{2D}}(\mathbf{k})$, $\theta$ is the angle between the $x$ axis and the (001) crystallographic direction (see Fig.~\ref{Fig:1}). This coordinate system allows one to consider the edge states with different crystallographic orientations of the edge. To obtain $H_{\mathrm{2D}}(\mathbf{k})$ in Eq.~(\ref{eq:1}) from a Hamiltonian with $\theta=0$, together with a clock-wise rotation of the coordinate system along $z$ axis, it is also necessary to perform a unitary transformation of the Hamiltonian given by $U=\exp(-i\theta{J_\xi})$. Here, $J_\xi$ is a diagonal matrix with the elements $(1/2,3/2,-1/2,-3/2)$ corresponding to the momentum of the basis states $|E1{\uparrow}\rangle$, $|H1{\uparrow}\rangle$, $|E1{\downarrow}\rangle$, $|H1{\downarrow}\rangle$, respectively~\cite{cst49}.

In order to take into account effect of in-plane magnetic field $\mathbf{B}$, one has to make the Peierls substitution $\mathbf{k}\rightarrow\mathbf{k}-(e/c\hbar)\mathbf{A}$ in $H_{\mathrm{2D}}(\mathbf{k})$ and to add  the Zeeman Hamiltonian:
\begin{equation}
\label{eq:3}
H_{\mathrm{Z}}=\dfrac{\mu_B}{2}\begin{pmatrix}
0 & 0 & g_e^{||}B_{+} & 0\\
0 & 0 & 0 & g_h^{||}e^{-i4\theta}B_{-}\\
g_e^{||}B_{-} & 0 & 0 & 0\\
0 & g_h^{||}e^{i4\theta}B_{+} & 0 & 0 \end{pmatrix},
\end{equation}
where $B_{\pm}=B_{x}{\pm}iB_{y}$, $\mu_B$ is the Bohr magneton, $g_e^{||}$ and $g_h^{||}$ are the in-plane g-factors of the $E1$ and $H1$ subbands, resulting from the bare electron g-factor and the interaction with the remote electron and hole subbands~\cite{Wbook}. Since, one can always choose the vector potential gauge so that $\mathbf{A}||z$ for in-plane magnetic field, we can always restrict ourselves to considering only the Zeeman term omitting the Peierls substitution for $k_x$ and $k_y$. The angular dependence of $H_{\mathrm{Z}}$ on $\theta$ in the chosen coordinate system is obtained in the same way as for $H_{\mathrm{2D}}(\mathbf{k})$.

As clear, $\Delta$ and $g_h^{||}$ are the terms breaking rotational symmetry of $H_{\mathrm{2D}}(\mathbf{k})$ and $H_{\mathrm{Z}}$. They both origin from the contribution of the $\Gamma_8$ bulk band of heterostructures based on zinc-blende semiconductors~\cite{Wbook}. In the following, we set $\Delta$ and $g_h^{||}$ to zero if one needs to ``switch off'' the crystal symmetry of the prototype (001) QW.

\begin{figure}
\includegraphics [width=1.0\columnwidth, keepaspectratio] {Fig1.jpg} % Here is how to import EPS art
\caption{\label{Fig:1} The layout of the coordinate axes (in red) and orientation the magnetic field $B$ (in green) with respect to the main crystallographic axes shown by the vertical and horizontal dashed lines. The QW edge (in blue) considered in the text is oriented along $x$ axis. The positive values of $\theta$ and $\varphi$ correspond to a clockwise rotation around $z$ axis.}
\end{figure}

\emph{Hamiltonian for gapped edge states.}--
To analyze the existence of corner states, we first need to obtain a 1D edge Hamiltonian. To derive an effective 1D low-energy edge Hamiltonian, we consider a semi-infinite plane $y>0$ with open-boundary conditions at $y=0$ as shown in Fig.~\ref{Fig:1}. Then, we solve the eigenvalue problem for $H_{\mathrm{2D}}(\mathbf{k})$ at $M<0$ to find the edge wave-functions at $k_x=0$. Finally, to derive the edge Hamiltonian, one should
successively project the part of $H_{\mathrm{2D}}(\mathbf{k})$ with non-zero $k_x$ and $H_{\mathrm{Z}}$ onto the calculated edge wave-functions. The procedure described above is fairly standard, so we do not present its details here. Up to our knowledge, in the presence of non-zero $\Delta$, it was first implemented by Durnev~\emph{et~al.}~\cite{cst56}

After tedious and routine mathematics, the part of the edge Hamiltonian independent of $\mathbf{B}$ at small values of $k_x$ is written as
\begin{equation}
\label{eq:2}
H_{\mathrm{edge}}^{(0)}(k_x)=\varepsilon_0+\hbar{v_F}k_x{s}_z+\mathcal{O}\left(k_x^2\right),
\end{equation}
where $s_a$ ($a=x,y,z$) corresponds to the Pauli matrices acting on spin degree of freedom,
$\varepsilon_0=C-M\mathbb{D}/\mathbb{B}$ and $\hbar{v_F}$ is defined as~\cite{cst56b}
\begin{equation}
\label{eq:2b}
\hbar{v_F}=A\dfrac{\sqrt{\mathbb{B}^2-\mathbb{D}^2}}{\left|\mathbb{B}\right|}\Omega,
\end{equation}
where
\begin{equation}
\label{eq:2c}
\Omega=\dfrac{|M|\sqrt{\mathbb{B}^2-\mathbb{D}^2}}{\sqrt{\left(\mathbb{B}^2-\mathbb{D}^2\right)M^2+\Delta^2\mathbb{B}^2}}.
\end{equation}
Note that the non-linear terms in $k_x$ arise in Eq.~(\ref{eq:2}) due to the presence of $\Delta$. They are all vanishing if $\Delta=0$~\cite{cst56b}.

Similarly, the Zeeman part of the edge Hamiltonian can be presented in the form~\cite{cst56}
\begin{equation}
\label{eq:4}
H_{\mathrm{edge}}^{(\mathbf{B})}=\dfrac{\mu_B}{2}\sum_{a,b=x,y}g_{ab}s_{a}B_{\beta},
\end{equation}
where the edge g-factor tensor $g_{ab}$ depends on the edge orientation with the following non-zero components:
\begin{eqnarray}
\label{eq:5}
g_{xx}=g_1\cos^{2}2\theta+g_2\sin^{2}2\theta,~\nonumber\\
g_{yy}=g_1\sin^{2}2\theta+g_2\cos^{2}2\theta,~\nonumber\\
g_{xy}=g_{yx}=\dfrac{1}{2}\left(g_1-g_2\right)\sin{4\theta}.
\end{eqnarray}
Here, $g_1$ and $g_2$ are two constants written as~\cite{cst56b}
\begin{eqnarray}
\label{eq:6}
g_{1}=g_e^{||}\dfrac{\mathbb{B}-\mathbb{D}}{2\mathbb{B}}+g_h^{||}\dfrac{\mathbb{B}+\mathbb{D}}{2\mathbb{B}},~~~~\nonumber\\
g_{2}=\left(g_e^{||}\dfrac{\mathbb{B}-\mathbb{D}}{2\mathbb{B}}-g_h^{||}\dfrac{\mathbb{B}+\mathbb{D}}{2\mathbb{B}}\right)\Omega.
\end{eqnarray}
As clear, if one neglects the terms resulting from the crystal symmetry of zinc-blende semiconductors, $g_1=g_2$, and the g-factor of the edges states becomes independent of the edge orientation.

Assuming that the magnetic field $\mathbf{B}$ is oriented at an angle $\varphi$ to the [100] crystallographic direction as shown in Fig.~\ref{Fig:1}, Eq.~(\ref{eq:4}) is rewritten as
\begin{equation}
\label{eq:7}
H_{\mathrm{edge}}^{(\mathbf{B})}=M_x{s}_{x}+M_y{s}_{y},
\end{equation}
where
\begin{eqnarray}
\label{eq:8}
M_x=\dfrac{E_{Z}}{2}\left[g_{+}\cos(\theta-\varphi)+g_{-}\cos(3\theta+\varphi)\right],~\nonumber\\
M_y=\dfrac{E_{Z}}{2}\left[g_{+}\sin(\theta-\varphi)+g_{-}\sin(3\theta+\varphi)\right]~~~
\end{eqnarray}
with
%\begin{equation}
%\label{eq:7}
%M_x=E_{Z}\sin(\varphi-\theta),~~~~
%M_y=E_{Z}\cos(\varphi-\theta),
%\end{equation}
$E_{Z}=\mu_{B}g_{1}B/2$ and $g_{\pm}=1\pm{g_{2}/g_{1}}$.

\begin{figure}
\includegraphics [width=1.0\columnwidth, keepaspectratio] {Fig2new.jpg} % Here is how to import EPS art
\caption{\label{Fig:2} Schematic of orientation of two meeting edges (defined by $\theta_1$ and $\theta_2$) and magnetic field (defined by $\varphi$) with respect to main crystallographic axes in the QW plane. Here, the ``curved'' $x$ axis is oriented along the edges so that $x=0$ represents to the meeting corner, while $x<0$ and $x>0$ correspond to $\theta_1$ and $\theta_2$, respectively. Note that the latter is fulfilled only if $\theta_1-\theta_2\neq\pm\pi$. The QSHI and external vacuum are shown in grey and white, respectively. The positive values of the angles correspond to a clockwise rotation around $z$ axis. The right and left panels represent the same case considered in the text.}
\end{figure}

\emph{Corner states.}--
%\subsection{\label{Sec:2b} Corner states}
Let us now redefine the coordinate $x$ along the ``curved'' edge so that $x=0$ corresponds to the meeting corner. The latter means that $M_x$ and $M_y$ in Eq.~(\ref{eq:6}) are now the functions of $x$, and $\hat{k}_x=-i\partial/\partial{x}$. Note that the edge Hamiltonian is defined in disjoint regions out of $x=0$. As a boundary condition at $x=0$, we further assume the continuity of the edge wave-function.

%\textcolor[rgb]{1.00,0.00,0.00}{, while the more general case is considered in the Supplementary Materials~\cite{SM}}.

In view of the above, the Schr\"{o}dinger equation for the corner states takes the form
\begin{multline}
\label{eq:Dirac1}
\left({\hbar}v_{F}\hat{k}_x{s}_z+M_x(x){s}_x+M_y(x){s}_y\right)\Psi_{\mathrm{0D}}(x)\\
=\left(E-\varepsilon_0\right)\Psi_{\mathrm{0D}}(x).
\end{multline}
Let us act by the matrix operator from the left-hand side of Eq.~(\ref{eq:Dirac1}) on both sides of this equation. This leads to
\begin{multline}
\label{eq:Dirac2}
\bigg\{\left({\hbar}^{2}v_{F}^2\hat{k}_x^2+M_x^2+M_y^2-\left(E-\varepsilon_0\right)^2\right)~\\
+{\hbar}v_{F}\begin{pmatrix}
0 & -i{M_x}'-M_y' \\
i{M_x}'-M_y' & 0
\end{pmatrix}
\bigg\}\Psi_{\mathrm{0D}}(x)=0,
\end{multline}
where the prime denotes the derivative with respect to $x$.

An exact solution of Eq.~(\ref{eq:Dirac2}) can be found in the case of a \emph{sharp} corner at $x=0$, when
\begin{equation}
\label{eq:Dirac3}
M_x(x)={\alpha_x}\Theta(x)+m_x,~~~~~~~~M_y(x)={\alpha_y}\Theta(x)+m_y,
\end{equation}
where $\Theta(x)$ is a step-function defined as $\Theta(x)=-1$ if $x<0$ and $\Theta(x)=1$ for $x\geq0$, while the constants ${\alpha_x}$, ${\alpha_x}$, ${m_x}$ and ${m_y}$ are written as
\begin{eqnarray}
\label{eq:Dirac4}
\alpha_{x,y}=\dfrac{M_{x,y}(+\infty)-M_{x,y}(-\infty)}{2},\nonumber\\
m_{x,y}=\dfrac{M_{x,y}(+\infty)+M_{x,y}(-\infty)}{2}.
\end{eqnarray}
Hence, solutions of Eq.~(\ref{eq:Dirac2}) is constructed as follows:
\begin{equation}
\label{eq:Dirac6}
\Psi_{\mathrm{0D}}(x)={\chi}\psi(x),
\end{equation}
where ${\chi}$ is the spin part of the wave function satisfying the equation
\begin{equation*}
\begin{pmatrix}
0 & -\alpha_y-i{\alpha_x} \\
-\alpha_y+i{\alpha_x} & 0
\end{pmatrix}{\chi}=\nu{\chi},
\end{equation*}
with eigenvalues $\nu=\pm\sqrt{\alpha_x^2+\alpha_y^2}$.

By introducing a new variable $\tilde{x}=x/({\hbar}v_{F})$, the equation for the coordinate part $\psi(x)$ can be written as
\begin{equation}
\label{eq:Dirac8}
\Bigg\{\hat{\tilde{k}}_x^2
+{W}(\tilde{x})^2+\sigma{W}(\tilde{x})'\Bigg\}\psi(\tilde{x})=\varepsilon{\psi}(\tilde{x}),
\end{equation}
where $\sigma=\pm{1}$ (the sign of $\sigma$ coincides with those for $\nu$), and $\varepsilon$ and ${W}(\tilde{x})$ are defined as
\begin{eqnarray}
\label{eq:Dirac9}
\varepsilon=\left(E-\varepsilon_0\right)^2-\dfrac{\left(m_x\alpha_y-m_y\alpha_x\right)^2}{\alpha_x^2+\alpha_y^2},\nonumber\\
{W}(\tilde{x})=\dfrac{\alpha_xM_x(\tilde{x})+\alpha_yM_y(\tilde{x})}{\sqrt{\alpha_x^2+\alpha_y^2}}.~~~~
\end{eqnarray}

Importantly, Eq.~(\ref{eq:Dirac8}) mimics the conventional Schr\"{o}dinger equation with an electrostatic potential being a linear combination of the square and the derivative of the same function ${W}(\tilde{x})$. It possesses a special symmetry and represents the formulation of supersymmetric quantum mechanics~\cite{cst58}, which allows for the factorization:
\begin{equation}
\label{eq:Dirac10}
\left(-i\hat{\tilde{k}}_x-\sigma{W}(\tilde{x})\right)\left(i\hat{\tilde{k}}_x-\sigma{W}(\tilde{x})\right){\psi}(\tilde{x})=\varepsilon{\psi}(\tilde{x}),
\end{equation}
If the signs of the asymptotics ${W}(+\infty)$ and ${W}(-\infty)$ are opposite,
Eq.~(\ref{eq:Dirac10}) always has a localized solution ${\psi}(\tilde{x})$ with $\varepsilon=0$, which converts the second brackets into zero:
\begin{equation}
\label{eq:Dirac12}
\left(\dfrac{d}{d\tilde{x}}-\sigma{W}(\tilde{x})\right){\psi}(\tilde{x})=0.
\end{equation}
Thus, under this condition, Eq.~(\ref{eq:Dirac1}) always has a localized solution at $x=0$ with the wave-function
\begin{equation}
\label{eq:9}
\Psi_{\mathrm{0D}}(x)=\mathcal{C}
\begin{pmatrix}
\alpha_{y}+i\alpha_{x} \\
-\sigma\sqrt{\alpha_x^2+\alpha_y^2}
\end{pmatrix}
e^{\displaystyle{\frac{\sigma}{{\hbar}v_{F}}\int\limits_0^{x}
W(z)dz}},
\end{equation}
where $\mathcal{C}$ is the normalization constant. The value of $\sigma$ in Eq.~(\ref{eq:9}) should be chosen in accordance with normalized condition of $\Psi_{\mathrm{0D}}(x)$: if $W(+\infty)>0$, $\sigma=-1$, while for $W(+\infty)<0$, $\sigma=1$.

The energy of this 0D corner state always lying in the gap for the edge states is written as
\begin{equation}
\label{eq:11}
E_{\mathrm{0D}}=\varepsilon_0+
\dfrac{{\sigma}\left(\alpha_{x}m_y-\alpha_{y}m_x\right)}{\sqrt{\alpha_x^2+\alpha_y^2}}.
\end{equation}
Interestly, in Eq.~(\ref{eq:9}), $W(x)$ plays the role of effective ``topological domain wall'', therefore, the form of $\Psi_{\mathrm{0D}}(x)$ strongly reminds the Jackiw-Rebbi mode in systems with one mass parameter~\cite{cst42}. As clear from Eq.~(\ref{eq:9}), in the case of two mass parameters, the corner state existence is related to the sign changing of $W(x)$ and not to the band-gap closing upon passing through $x=0$. We note that a localized state with the energy $E_{0D}$ in Eq.~(\ref{eq:11}) is the only one corner state arising at the sharp meeting corner~\cite{SM}.

For further analysis, instead of Eq.~(\ref{eq:8}), it is convenient to apply the following parametrization for $M_x$ and $M_y$:
\begin{equation}
\label{eq:12}
M_x=M\cos{\beta},~~~~~~~~~M_y=M\sin{\beta},
\end{equation}
where $M>0$ and $\beta\in(-\pi,\pi]$ are both functions of $\theta$ and $\varphi$ being defined as $\tan{\beta}=M_{y}/M_{x}$ and
\begin{equation}
\label{eq:13}
M=\sqrt{M_{x}^2+M_{y}^2}=E_{Z}\sqrt{1-g_{+}g_{-}\sin^{2}\left(\theta+\varphi\right)}.
\end{equation}

By means of parametrization (\ref{eq:12}), the energy and existence condition of the corner states are written in a compact way:
\begin{eqnarray}
\label{eq:13}
E_{\mathrm{0D}}=\varepsilon_0-\dfrac{{\sigma}\sqrt{M_{1}M_{2}}\sin({\beta_1-\beta_2})}{2\sqrt{\sin^2\left(\dfrac{\beta_1-\beta_2}{2}\right)+\dfrac{\left(M_1-M_2\right)^2}{4M_{1}M_{2}}}},~~~~~~\nonumber\\
W(-\infty)W(+\infty)=-M_{1}M_{2}~~~~~~~~~~~~~~~~~~~~~~~~~~~~~~~~\nonumber\\
\times\dfrac{\sin^4\left(\dfrac{\beta_1-\beta_2}{2}\right)-\dfrac{\left(M_1-M_2\right)^2}{4M_{1}M_{2}}\cos\left(\beta_1-\beta_2\right)}{\sin^2\left(\dfrac{\beta_1-\beta_2}{2}\right)+\dfrac{\left(M_1-M_2\right)^2}{4M_{1}M_{2}}},~~~~
\end{eqnarray}
where indices $1$ and $2$ correspond to the parameters at $x\rightarrow-\infty$ and $x\rightarrow+\infty$, respectively. It can be shown analytically that for $\left(\beta_1-\beta_2\right)\in(-\pi,\pi]$, the existence condition $W(-\infty)W(+\infty)<0$ is satisfied if
\begin{equation}
\label{eq:14}
\left|\beta_1-\beta_2\right|>\arccos\left[\dfrac{\min\{M_{1},M_{2}\}}{\sqrt{M_{1}M_{2}}}\right].
\end{equation}
Since $\beta$ does not coincide with $\theta-\varphi$ if $g_h^{||}$ and $\Delta$ are non-zero in the general case (see Eq.~(\ref{eq:8})), the existence condition of the corner state is fulfilled only for specific orientations of magnetic field and the meeting edges.

Interestly, if we ``switch off'' the crystal symmetry of the prototype (001) QW by assuming $g_1=g_2$ (see Eq.~(\ref{eq:6})), $M=E_Z$ and $\beta=\theta-\varphi$, the corner state still exist. Indeed, under such condition, $E_{\mathrm{0D}}$ and $W(-\infty)W(+\infty)$ take the simple form
\begin{eqnarray}
\label{eq:15}
E_{\mathrm{0D}}=\varepsilon_0-\sigma\left|E_Z\right|\cos\left(\dfrac{\theta_1-\theta_2}{2}\right),~~~\nonumber\\
W(-\infty)W(+\infty)=-E_Z^2\sin^2\left(\dfrac{\theta_1-\theta_2}{2}\right)<0.
\end{eqnarray}
The latter is fulfilled if $\theta_1\neq\theta_2$, that corresponds to the presence of realistic corner in 2D system (see Fig.~\ref{Fig:2}). In this case, the corner states arise for any orientations of the meeting edges, and their energies are independent of the magnetic field orientation. Thus, the presence of the crystal symmetry of zinc-blende semiconductors is not mandatory for the existence of a corner state. Obviously, Eqs.~(\ref{eq:15}) are valid for any QSHIs with isotropic g-factor of the edge states.

\begin{figure}
\includegraphics [width=1.0\columnwidth, keepaspectratio] {Fig3.jpg} % Here is how to import EPS art
\caption{\label{Fig:3} Energy of corner states $E_{\mathrm{0D}}-\varepsilon_0$ (solid curves) for different orientations of the magnetic field and edges relative to the main crystallographic axes, calculated for various ratio of $g_2/g_1$. The dotted curves represented by $\pm\min\{M_{1},M_{2}\}$ correspond to the boundaries of 1D edge band states projected onto the corner. The corner state arises as soon as $W(-\infty)W(+\infty)<0$.
Note that $\theta_1-\theta_2\neq0$ and $\pm\pi$ for physically reasonable edges forming a common corner. }
\end{figure}

Figure~\ref{Fig:3} shows the evolution of corner state energy as a function of $\theta_1-\theta_2$ at various ratio of $g_2/g_1$, calculated for several orientations of the magnetic field and edges. As seen, the corner state always appears from the gapped 1D edge bands. Each time the corner state merges with 1D edge, the factor causes to zero. As can be seen from Eq.~(\ref{eq:9}), at this moment the corner state ceases to be localized. The boundaries of the continuum of 1D edge bands projected onto the corner are determined by $\pm\min\{M_{1},M_{2}\}$ shown by dotted curves. Since both $H_{\mathrm{2D}}$ in Eq.~(\ref{eq:1}) and $H_{Z}$ in Eq.~(\ref{eq:3}) do not have electron-hole symmetry, the corner state energy $E_{\mathrm{0D}}$ depends on the orientation of the edges and the direction of the magnetic field. Interestly, at specific values of $g_2/g_1$, $\theta_1$, $\theta_2$ and $\varphi$, the corner state may have a ``zero energy'', which is $E_{\mathrm{0D}}=C$ in our notations, coinciding with the corner state energy in the system with particle-hole symmetry. However, this is the result of an arbitrary coincidence for a specific set of parameters, which are in no way related to any particular geometry of the edge and magnetic field orientations.

In spite of the fact that Fig.~\ref{Fig:3} is valid only for zinc-blende semiconductor QWs, our general analytical results above can also be applied for other QSHIs with different crystal symmetries. In the latter case, both 2D bulk Hamiltonian and Zeeman term may differ significantly from Eqs.~(\ref{eq:1}) and (\ref{eq:3}) resulting to the edge g-factor tensor different from those in Eq.~(\ref{eq:5}). For instance, the edge g-factor tensor can be derived from the lowest-order in-$\mathbf{k}$ expansion of tight-binding Hamiltonians widely used to describe QSHIs within the lattice models~\cite{cst47a00,cst47,cst48}.
As can be seen from Supplementary Materials~\cite{SM}, the analytical results obtained on the basis of Eq.~(\ref{eq:11}) with the corresponding edge g-factor tensor perfectly reproduce the results of numerical calculations performed on the square lattice~\cite{cst47a00}.

\emph{Out-of-plane magnetic field.}--
Finally, let us briefly discuss the effect of a perpendicular magnetic field $B_z$ in zinc-blende semiconductor QWs. In the general case, the presence of BIA or IIA leads to the band-gap opening of the edge states even for small values of $B_z$~\cite{cst56}. In this case, the terms describing the band-gap opening are written as
\begin{equation}
\label{eq:16}
H_{\mathrm{edge}\bot}^{(\mathbf{B})}=\dfrac{\mu_{B}B_z}{2}\left(g_{xz}{s}_{x}+g_{yz}{s}_{y}\right),
\end{equation}
where $g_{xz}$ and $g_{yz}$ have the following form:
\begin{equation}
\label{eq:17}
g_{xz}=g_3\sin\theta,~~~~~~g_{yz}=-g_3\cos\theta.
\end{equation}
Here, $g_3\sim\Omega^3\Delta$ is a non-zero constant~\cite{cst56} due to the presence of BIA or SIA in the zinc-blende QWs. Note that the diagonal component of the edge g-factor $g_{zz}$ is independent of $\theta$ and, thus, it can be formally nullified using the substitution $\hat{k}_x\rightarrow\hat{k}_x-g_{zz}\mu_{B}B_z/(2{\hbar}v_{F})$. 

As clear, the form of $H_{\mathrm{edge}\bot}^{(\mathbf{B})}$ in Eq.~(\ref{eq:16}) coincides with those for Hamiltonian~(\ref{eq:7}). This means that out-of-plane magnetic field also leads to the corner state arising at the intersection of two meeting edges. Indeed, using the previously obtained results, the existence condition and energy of the corner state are directly written as
\begin{eqnarray}
\label{eq:18}
W(-\infty)W(+\infty)=-\dfrac{\mu_{B}^2g_{3}^2B_{z}^2}{4}\sin^2\left(\dfrac{\theta_1-\theta_2}{2}\right)<0,\nonumber\\
E_{\mathrm{0D}\bot}=\varepsilon_0-\sigma\dfrac{\left|\mu_{B}g_{3}B_{z}\right|}{2}
\cos\left(\dfrac{\theta_1-\theta_2}{2}\right).~~~~~~~~~
\end{eqnarray}
%In fact, these expressions can be immediately obtained from expressions~(\ref{eq:13}) by setting $M_1=M_2=\mu_{B}g_{3}B_{z}/2$ and $\beta_{1,2}=\theta_{1,2}+\pi/2$. The latter follows from a comparison of Eqs~(\ref{eq:12}) and (\ref{eq:17}). As clear from Eq.~(\ref{eq:18}), in the absence of BIA or IIA when $g_3=0$, the corner state does not exist.
As clear from Eq.~(\ref{eq:18}), in the presence of BIA or IIA when $g_3\neq{0}$, the corner state induced by out-of-plane magnetic field exists for any crystallographic orientations of the meeting edges.

\emph{Summary.}--
We have analytically solved the general problem of magnetic-field-induced corner states in QSHIs based on zinc-blende semiconductor QWs. By using the continuous BHZ model with additional terms describing the crystal symmetry of zinc-blende semiconductor QW, we have derived a ``generalized'' 1D Dirac equation with \emph{two mass parameters} describing the gapped edge states at arbitrary crystallographic orientation of the edge and the magnetic field. We demonstrate that a solution of such ``generalized'' Dirac equation found in the form of ``topological domain wall mode'' supports the existence of corner states at the intersections of two adjacent edges. Surprisingly, we found that the corner states induced by an in-plane magnetic field does not require the crystal symmetry of zinc-blende semiconductor QW, making our results universal for any quantum spin Hall insulators with isotropic edge state g-factor. Additionally, since our general analytical results do not require any particular form of the edge g-factor tensor, they can also be applied to describe corner states in QSHIs with a crystal symmetry different from that of zinc-blende semiconductor QWs~\cite{SM}.

%~\cite{bookSM1q,bookSM2q}

\begin{acknowledgments}
This work was supported by the Occitanie region through the ``Terahertz Occitanie Platform'', by the CNRS through IRP ``TeraMIR'', and by the French Agence Nationale pour la Recherche through ``Colector'' (ANR-19-CE30-0032) and ``Equipex+ Hybat'' (ANR-21-ESRE-0026) projects.
\end{acknowledgments}


%\bibliography{HOTI_MF5}
\begin{thebibliography}{66}%
\makeatletter
\providecommand \@ifxundefined [1]{%
 \@ifx{#1\undefined}
}%
\providecommand \@ifnum [1]{%
 \ifnum #1\expandafter \@firstoftwo
 \else \expandafter \@secondoftwo
 \fi
}%
\providecommand \@ifx [1]{%
 \ifx #1\expandafter \@firstoftwo
 \else \expandafter \@secondoftwo
 \fi
}%
\providecommand \natexlab [1]{#1}%
\providecommand \enquote  [1]{``#1''}%
\providecommand \bibnamefont  [1]{#1}%
\providecommand \bibfnamefont [1]{#1}%
\providecommand \citenamefont [1]{#1}%
\providecommand \href@noop [0]{\@secondoftwo}%
\providecommand \href [0]{\begingroup \@sanitize@url \@href}%
\providecommand \@href[1]{\@@startlink{#1}\@@href}%
\providecommand \@@href[1]{\endgroup#1\@@endlink}%
\providecommand \@sanitize@url [0]{\catcode `\\12\catcode `\$12\catcode
  `\&12\catcode `\#12\catcode `\^12\catcode `\_12\catcode `\%12\relax}%
\providecommand \@@startlink[1]{}%
\providecommand \@@endlink[0]{}%
\providecommand \url  [0]{\begingroup\@sanitize@url \@url }%
\providecommand \@url [1]{\endgroup\@href {#1}{\urlprefix }}%
\providecommand \urlprefix  [0]{URL }%
\providecommand \Eprint [0]{\href }%
\providecommand \doibase [0]{http://dx.doi.org/}%
\providecommand \selectlanguage [0]{\@gobble}%
\providecommand \bibinfo  [0]{\@secondoftwo}%
\providecommand \bibfield  [0]{\@secondoftwo}%
\providecommand \translation [1]{[#1]}%
\providecommand \BibitemOpen [0]{}%
\providecommand \bibitemStop [0]{}%
\providecommand \bibitemNoStop [0]{.\EOS\space}%
\providecommand \EOS [0]{\spacefactor3000\relax}%
\providecommand \BibitemShut  [1]{\csname bibitem#1\endcsname}%
\let\auto@bib@innerbib\@empty
%</preamble>
\bibitem [{\citenamefont {Benalcazar}\ \emph {et~al.}(2017)\citenamefont
  {Benalcazar}, \citenamefont {Bernevig},\ and\ \citenamefont {Hughes}}]{cst1}%
  \BibitemOpen
  \bibfield  {author} {\bibinfo {author} {\bibfnamefont {W.~A.}\ \bibnamefont
  {Benalcazar}}, \bibinfo {author} {\bibfnamefont {B.~A.}\ \bibnamefont
  {Bernevig}}, \ and\ \bibinfo {author} {\bibfnamefont {T.~L.}\ \bibnamefont
  {Hughes}},\ }\href {\doibase 10.1126/science.aah6442} {\bibfield  {journal}
  {\bibinfo  {journal} {Science}\ }\textbf {\bibinfo {volume} {357}},\ \bibinfo
  {pages} {61} (\bibinfo {year} {2017})}\BibitemShut {NoStop}%
\bibitem [{\citenamefont {Langbehn}\ \emph
  {et~al.}(2017{\natexlab{a}})\citenamefont {Langbehn}, \citenamefont {Peng},
  \citenamefont {Trifunovic}, \citenamefont {von Oppen},\ and\ \citenamefont
  {Brouwer}}]{cst2}%
  \BibitemOpen
  \bibfield  {author} {\bibinfo {author} {\bibfnamefont {J.}~\bibnamefont
  {Langbehn}}, \bibinfo {author} {\bibfnamefont {Y.}~\bibnamefont {Peng}},
  \bibinfo {author} {\bibfnamefont {L.}~\bibnamefont {Trifunovic}}, \bibinfo
  {author} {\bibfnamefont {F.}~\bibnamefont {von Oppen}}, \ and\ \bibinfo
  {author} {\bibfnamefont {P.~W.}\ \bibnamefont {Brouwer}},\ }\href {\doibase
  10.1103/PhysRevLett.119.246401} {\bibfield  {journal} {\bibinfo  {journal}
  {Phys. Rev. Lett.}\ }\textbf {\bibinfo {volume} {119}},\ \bibinfo {pages}
  {246401} (\bibinfo {year} {2017}{\natexlab{a}})}\BibitemShut {NoStop}%
\bibitem [{\citenamefont {Song}\ \emph {et~al.}(2017)\citenamefont {Song},
  \citenamefont {Fang},\ and\ \citenamefont {Fang}}]{cst3}%
  \BibitemOpen
  \bibfield  {author} {\bibinfo {author} {\bibfnamefont {Z.}~\bibnamefont
  {Song}}, \bibinfo {author} {\bibfnamefont {Z.}~\bibnamefont {Fang}}, \ and\
  \bibinfo {author} {\bibfnamefont {C.}~\bibnamefont {Fang}},\ }\href {\doibase
  10.1103/PhysRevLett.119.246402} {\bibfield  {journal} {\bibinfo  {journal}
  {Phys. Rev. Lett.}\ }\textbf {\bibinfo {volume} {119}},\ \bibinfo {pages}
  {246402} (\bibinfo {year} {2017})}\BibitemShut {NoStop}%
\bibitem [{\citenamefont {Schindler}\ \emph
  {et~al.}(2018{\natexlab{a}})\citenamefont {Schindler}, \citenamefont {Cook},
  \citenamefont {Vergniory}, \citenamefont {Wang}, \citenamefont {Parkin},
  \citenamefont {Bernevig},\ and\ \citenamefont {Neupert}}]{cst4}%
  \BibitemOpen
  \bibfield  {author} {\bibinfo {author} {\bibfnamefont {F.}~\bibnamefont
  {Schindler}}, \bibinfo {author} {\bibfnamefont {A.~M.}\ \bibnamefont {Cook}},
  \bibinfo {author} {\bibfnamefont {M.~G.}\ \bibnamefont {Vergniory}}, \bibinfo
  {author} {\bibfnamefont {Z.}~\bibnamefont {Wang}}, \bibinfo {author}
  {\bibfnamefont {S.~S.~P.}\ \bibnamefont {Parkin}}, \bibinfo {author}
  {\bibfnamefont {B.~A.}\ \bibnamefont {Bernevig}}, \ and\ \bibinfo {author}
  {\bibfnamefont {T.}~\bibnamefont {Neupert}},\ }\href {\doibase
  10.1126/sciadv.aat0346} {\bibfield  {journal} {\bibinfo  {journal} {Sci.
  Adv.}\ }\textbf {\bibinfo {volume} {4}},\ \bibinfo {pages} {eaat0346}
  (\bibinfo {year} {2018}{\natexlab{a}})}\BibitemShut {NoStop}%
\bibitem [{\citenamefont {Ezawa}(2018{\natexlab{a}})}]{cst5}%
  \BibitemOpen
  \bibfield  {author} {\bibinfo {author} {\bibfnamefont {M.}~\bibnamefont
  {Ezawa}},\ }\href {\doibase 10.1103/PhysRevLett.120.026801} {\bibfield
  {journal} {\bibinfo  {journal} {Phys. Rev. Lett.}\ }\textbf {\bibinfo
  {volume} {120}},\ \bibinfo {pages} {026801} (\bibinfo {year}
  {2018}{\natexlab{a}})}\BibitemShut {NoStop}%
\bibitem [{\citenamefont {Kane}\ and\ \citenamefont {Mele}(2005)}]{cst6}%
  \BibitemOpen
  \bibfield  {author} {\bibinfo {author} {\bibfnamefont {C.~L.}\ \bibnamefont
  {Kane}}\ and\ \bibinfo {author} {\bibfnamefont {E.~J.}\ \bibnamefont
  {Mele}},\ }\href {\doibase 10.1103/PhysRevLett.95.146802} {\bibfield
  {journal} {\bibinfo  {journal} {Phys. Rev. Lett.}\ }\textbf {\bibinfo
  {volume} {95}},\ \bibinfo {pages} {146802} (\bibinfo {year}
  {2005})}\BibitemShut {NoStop}%
\bibitem [{\citenamefont {Bernevig}\ \emph
  {et~al.}(2006{\natexlab{a}})\citenamefont {Bernevig}, \citenamefont
  {Hughes},\ and\ \citenamefont {Zhang}}]{cst7}%
  \BibitemOpen
  \bibfield  {author} {\bibinfo {author} {\bibfnamefont {B.~A.}\ \bibnamefont
  {Bernevig}}, \bibinfo {author} {\bibfnamefont {T.~L.}\ \bibnamefont
  {Hughes}}, \ and\ \bibinfo {author} {\bibfnamefont {S.-C.}\ \bibnamefont
  {Zhang}},\ }\href {\doibase 10.1126/science.1133734} {\bibfield  {journal}
  {\bibinfo  {journal} {Science}\ }\textbf {\bibinfo {volume} {314}},\ \bibinfo
  {pages} {1757} (\bibinfo {year} {2006}{\natexlab{a}})}\BibitemShut {NoStop}%
\bibitem [{\citenamefont {Hasan}\ and\ \citenamefont {Kane}(2010)}]{cst8}%
  \BibitemOpen
  \bibfield  {author} {\bibinfo {author} {\bibfnamefont {M.~Z.}\ \bibnamefont
  {Hasan}}\ and\ \bibinfo {author} {\bibfnamefont {C.~L.}\ \bibnamefont
  {Kane}},\ }\href {\doibase 10.1103/RevModPhys.82.3045} {\bibfield  {journal}
  {\bibinfo  {journal} {Rev. Mod. Phys.}\ }\textbf {\bibinfo {volume} {82}},\
  \bibinfo {pages} {3045} (\bibinfo {year} {2010})}\BibitemShut {NoStop}%
\bibitem [{\citenamefont {Qi}\ and\ \citenamefont {Zhang}(2011)}]{cst9}%
  \BibitemOpen
  \bibfield  {author} {\bibinfo {author} {\bibfnamefont {X.-L.}\ \bibnamefont
  {Qi}}\ and\ \bibinfo {author} {\bibfnamefont {S.-C.}\ \bibnamefont {Zhang}},\
  }\href {\doibase 10.1103/RevModPhys.83.1057} {\bibfield  {journal} {\bibinfo
  {journal} {Rev. Mod. Phys.}\ }\textbf {\bibinfo {volume} {83}},\ \bibinfo
  {pages} {1057} (\bibinfo {year} {2011})}\BibitemShut {NoStop}%
\bibitem [{\citenamefont {Bansil}\ \emph {et~al.}(2016)\citenamefont {Bansil},
  \citenamefont {Lin},\ and\ \citenamefont {Das}}]{cst10}%
  \BibitemOpen
  \bibfield  {author} {\bibinfo {author} {\bibfnamefont {A.}~\bibnamefont
  {Bansil}}, \bibinfo {author} {\bibfnamefont {H.}~\bibnamefont {Lin}}, \ and\
  \bibinfo {author} {\bibfnamefont {T.}~\bibnamefont {Das}},\ }\href {\doibase
  10.1103/RevModPhys.88.021004} {\bibfield  {journal} {\bibinfo  {journal}
  {Rev. Mod. Phys.}\ }\textbf {\bibinfo {volume} {88}},\ \bibinfo {pages}
  {021004} (\bibinfo {year} {2016})}\BibitemShut {NoStop}%
\bibitem [{\citenamefont {Schindler}\ \emph
  {et~al.}(2018{\natexlab{b}})\citenamefont {Schindler}, \citenamefont {Wang},
  \citenamefont {Vergniory}, \citenamefont {Cook}, \citenamefont {Murani},
  \citenamefont {Sengupta}, \citenamefont {Yu.}, \citenamefont {Kasumov},
  \citenamefont {Deblock}, \citenamefont {Jeon}, \citenamefont {Drozdov},
  \citenamefont {Bouchiat}, \citenamefont {Gu\'{e}ron}, \citenamefont
  {Yazdani}, \citenamefont {Bernevig},\ and\ \citenamefont {Neupert}}]{cst11}%
  \BibitemOpen
  \bibfield  {author} {\bibinfo {author} {\bibfnamefont {F.}~\bibnamefont
  {Schindler}}, \bibinfo {author} {\bibfnamefont {Z.}~\bibnamefont {Wang}},
  \bibinfo {author} {\bibfnamefont {M.~G.}\ \bibnamefont {Vergniory}}, \bibinfo
  {author} {\bibfnamefont {A.~M.}\ \bibnamefont {Cook}}, \bibinfo {author}
  {\bibfnamefont {A.}~\bibnamefont {Murani}}, \bibinfo {author} {\bibfnamefont
  {S.}~\bibnamefont {Sengupta}}, \bibinfo {author} {\bibfnamefont
  {A.}~\bibnamefont {Yu.}}, \bibinfo {author} {\bibnamefont {Kasumov}},
  \bibinfo {author} {\bibfnamefont {R.}~\bibnamefont {Deblock}}, \bibinfo
  {author} {\bibfnamefont {S.}~\bibnamefont {Jeon}}, \bibinfo {author}
  {\bibfnamefont {I.}~\bibnamefont {Drozdov}}, \bibinfo {author} {\bibfnamefont
  {H.}~\bibnamefont {Bouchiat}}, \bibinfo {author} {\bibfnamefont
  {S.}~\bibnamefont {Gu\'{e}ron}}, \bibinfo {author} {\bibfnamefont
  {A.}~\bibnamefont {Yazdani}}, \bibinfo {author} {\bibfnamefont {B.~A.}\
  \bibnamefont {Bernevig}}, \ and\ \bibinfo {author} {\bibfnamefont
  {T.}~\bibnamefont {Neupert}},\ }\href {\doibase 10.1038/s41567-018-0224-7}
  {\bibfield  {journal} {\bibinfo  {journal} {Nat. Phys.}\ }\textbf {\bibinfo
  {volume} {15}},\ \bibinfo {pages} {918} (\bibinfo {year}
  {2018}{\natexlab{b}})}\BibitemShut {NoStop}%
\bibitem [{\citenamefont {Choi}\ \emph {et~al.}(2020)\citenamefont {Choi},
  \citenamefont {Xie}, \citenamefont {Chen}, \citenamefont {Park},
  \citenamefont {Song}, \citenamefont {Yoon}, \citenamefont {Kim},
  \citenamefont {Taniguchi}, \citenamefont {Watanabe}, \citenamefont {Lee},
  \citenamefont {Kim}, \citenamefont {Fong}, \citenamefont {Ali}, \citenamefont
  {Law},\ and\ \citenamefont {Lee}}]{cst12}%
  \BibitemOpen
  \bibfield  {author} {\bibinfo {author} {\bibfnamefont {Y.-B.}\ \bibnamefont
  {Choi}}, \bibinfo {author} {\bibfnamefont {Y.}~\bibnamefont {Xie}}, \bibinfo
  {author} {\bibfnamefont {C.-Z.}\ \bibnamefont {Chen}}, \bibinfo {author}
  {\bibfnamefont {J.-H.}\ \bibnamefont {Park}}, \bibinfo {author}
  {\bibfnamefont {S.-B.}\ \bibnamefont {Song}}, \bibinfo {author}
  {\bibfnamefont {J.}~\bibnamefont {Yoon}}, \bibinfo {author} {\bibfnamefont
  {B.~J.}\ \bibnamefont {Kim}}, \bibinfo {author} {\bibfnamefont
  {T.}~\bibnamefont {Taniguchi}}, \bibinfo {author} {\bibfnamefont
  {K.}~\bibnamefont {Watanabe}}, \bibinfo {author} {\bibfnamefont {H.-J.}\
  \bibnamefont {Lee}}, \bibinfo {author} {\bibfnamefont {J.-H.}\ \bibnamefont
  {Kim}}, \bibinfo {author} {\bibfnamefont {K.~C.}\ \bibnamefont {Fong}},
  \bibinfo {author} {\bibfnamefont {M.~N.}\ \bibnamefont {Ali}}, \bibinfo
  {author} {\bibfnamefont {K.~T.}\ \bibnamefont {Law}}, \ and\ \bibinfo
  {author} {\bibfnamefont {G.-H.}\ \bibnamefont {Lee}},\ }\href {\doibase
  10.1038/s41563-020-0721-9} {\bibfield  {journal} {\bibinfo  {journal} {Nat.
  Mater.}\ }\textbf {\bibinfo {volume} {19}},\ \bibinfo {pages} {974} (\bibinfo
  {year} {2020})}\BibitemShut {NoStop}%
\bibitem [{\citenamefont {Noguchi}\ \emph {et~al.}(2021)\citenamefont
  {Noguchi}, \citenamefont {Kobayashi}, \citenamefont {Jiang}, \citenamefont
  {Kuroda}, \citenamefont {Takahashi}, \citenamefont {Xu}, \citenamefont {Lee},
  \citenamefont {Hirayama}, \citenamefont {Ochi}, \citenamefont {Shirasawa},
  \citenamefont {Zhang}, \citenamefont {Lin}, \citenamefont {Bareille},
  \citenamefont {Sakuragi}, \citenamefont {Tanaka}, \citenamefont {Kunisada},
  \citenamefont {Kurokawa}, \citenamefont {Yaji}, \citenamefont {Harasawa},
  \citenamefont {Kandyba}, \citenamefont {Giampietri}, \citenamefont {Barinov},
  \citenamefont {Kim}, \citenamefont {Cacho}, \citenamefont {Hashimoto},
  \citenamefont {Lu}, \citenamefont {Shin}, \citenamefont {Arita},
  \citenamefont {Lai}, \citenamefont {Sasagawa},\ and\ \citenamefont
  {Kondo}}]{cst13}%
  \BibitemOpen
  \bibfield  {author} {\bibinfo {author} {\bibfnamefont {R.}~\bibnamefont
  {Noguchi}}, \bibinfo {author} {\bibfnamefont {M.}~\bibnamefont {Kobayashi}},
  \bibinfo {author} {\bibfnamefont {Z.}~\bibnamefont {Jiang}}, \bibinfo
  {author} {\bibfnamefont {K.}~\bibnamefont {Kuroda}}, \bibinfo {author}
  {\bibfnamefont {T.}~\bibnamefont {Takahashi}}, \bibinfo {author}
  {\bibfnamefont {Z.}~\bibnamefont {Xu}}, \bibinfo {author} {\bibfnamefont
  {D.}~\bibnamefont {Lee}}, \bibinfo {author} {\bibfnamefont {M.}~\bibnamefont
  {Hirayama}}, \bibinfo {author} {\bibfnamefont {M.}~\bibnamefont {Ochi}},
  \bibinfo {author} {\bibfnamefont {T.}~\bibnamefont {Shirasawa}}, \bibinfo
  {author} {\bibfnamefont {P.}~\bibnamefont {Zhang}}, \bibinfo {author}
  {\bibfnamefont {C.}~\bibnamefont {Lin}}, \bibinfo {author} {\bibfnamefont
  {C.}~\bibnamefont {Bareille}}, \bibinfo {author} {\bibfnamefont
  {S.}~\bibnamefont {Sakuragi}}, \bibinfo {author} {\bibfnamefont
  {H.}~\bibnamefont {Tanaka}}, \bibinfo {author} {\bibfnamefont
  {S.}~\bibnamefont {Kunisada}}, \bibinfo {author} {\bibfnamefont
  {K.}~\bibnamefont {Kurokawa}}, \bibinfo {author} {\bibfnamefont
  {K.}~\bibnamefont {Yaji}}, \bibinfo {author} {\bibfnamefont {A.}~\bibnamefont
  {Harasawa}}, \bibinfo {author} {\bibfnamefont {V.}~\bibnamefont {Kandyba}},
  \bibinfo {author} {\bibfnamefont {A.}~\bibnamefont {Giampietri}}, \bibinfo
  {author} {\bibfnamefont {A.}~\bibnamefont {Barinov}}, \bibinfo {author}
  {\bibfnamefont {T.}~\bibnamefont {Kim}}, \bibinfo {author} {\bibfnamefont
  {C.}~\bibnamefont {Cacho}}, \bibinfo {author} {\bibfnamefont
  {M.}~\bibnamefont {Hashimoto}}, \bibinfo {author} {\bibfnamefont
  {D.}~\bibnamefont {Lu}}, \bibinfo {author} {\bibfnamefont {S.}~\bibnamefont
  {Shin}}, \bibinfo {author} {\bibfnamefont {R.}~\bibnamefont {Arita}},
  \bibinfo {author} {\bibfnamefont {K.}~\bibnamefont {Lai}}, \bibinfo {author}
  {\bibfnamefont {T.}~\bibnamefont {Sasagawa}}, \ and\ \bibinfo {author}
  {\bibfnamefont {T.}~\bibnamefont {Kondo}},\ }\href {\doibase
  10.1038/s41563-020-00871-7} {\bibfield  {journal} {\bibinfo  {journal} {Nat.
  Mater.}\ }\textbf {\bibinfo {volume} {20}},\ \bibinfo {pages} {473} (\bibinfo
  {year} {2021})}\BibitemShut {NoStop}%
\bibitem [{\citenamefont {Shumiya}\ \emph {et~al.}(2022)\citenamefont
  {Shumiya}, \citenamefont {Hossain}, \citenamefont {Yin}, \citenamefont
  {Wang}, \citenamefont {Litskevich}, \citenamefont {Yoon}, \citenamefont {Li},
  \citenamefont {Yang}, \citenamefont {Jiang}, \citenamefont {Cheng},
  \citenamefont {Lin}, \citenamefont {Zhang}, \citenamefont {Cheng},
  \citenamefont {Cochran}, \citenamefont {Multer}, \citenamefont {Yang},
  \citenamefont {Casas}, \citenamefont {Chang}, \citenamefont {Neupert},
  \citenamefont {Yuan}, \citenamefont {Jia}, \citenamefont {Lin}, \citenamefont
  {Yao}, \citenamefont {Balicas}, \citenamefont {Zhang}, \citenamefont {Yao},\
  and\ \citenamefont {Hasan}}]{cst14}%
  \BibitemOpen
  \bibfield  {author} {\bibinfo {author} {\bibfnamefont {N.}~\bibnamefont
  {Shumiya}}, \bibinfo {author} {\bibfnamefont {M.~S.}\ \bibnamefont
  {Hossain}}, \bibinfo {author} {\bibfnamefont {J.-X.}\ \bibnamefont {Yin}},
  \bibinfo {author} {\bibfnamefont {Z.}~\bibnamefont {Wang}}, \bibinfo {author}
  {\bibfnamefont {M.}~\bibnamefont {Litskevich}}, \bibinfo {author}
  {\bibfnamefont {C.}~\bibnamefont {Yoon}}, \bibinfo {author} {\bibfnamefont
  {Y.}~\bibnamefont {Li}}, \bibinfo {author} {\bibfnamefont {Y.}~\bibnamefont
  {Yang}}, \bibinfo {author} {\bibfnamefont {Y.-X.}\ \bibnamefont {Jiang}},
  \bibinfo {author} {\bibfnamefont {G.}~\bibnamefont {Cheng}}, \bibinfo
  {author} {\bibfnamefont {Y.-C.}\ \bibnamefont {Lin}}, \bibinfo {author}
  {\bibfnamefont {Q.}~\bibnamefont {Zhang}}, \bibinfo {author} {\bibfnamefont
  {Z.-J.}\ \bibnamefont {Cheng}}, \bibinfo {author} {\bibfnamefont {T.~A.}\
  \bibnamefont {Cochran}}, \bibinfo {author} {\bibfnamefont {D.}~\bibnamefont
  {Multer}}, \bibinfo {author} {\bibfnamefont {X.~P.}\ \bibnamefont {Yang}},
  \bibinfo {author} {\bibfnamefont {B.}~\bibnamefont {Casas}}, \bibinfo
  {author} {\bibfnamefont {T.-R.}\ \bibnamefont {Chang}}, \bibinfo {author}
  {\bibfnamefont {T.}~\bibnamefont {Neupert}}, \bibinfo {author} {\bibfnamefont
  {Z.}~\bibnamefont {Yuan}}, \bibinfo {author} {\bibfnamefont {S.}~\bibnamefont
  {Jia}}, \bibinfo {author} {\bibfnamefont {H.}~\bibnamefont {Lin}}, \bibinfo
  {author} {\bibfnamefont {N.}~\bibnamefont {Yao}}, \bibinfo {author}
  {\bibfnamefont {L.}~\bibnamefont {Balicas}}, \bibinfo {author} {\bibfnamefont
  {F.}~\bibnamefont {Zhang}}, \bibinfo {author} {\bibfnamefont
  {Y.}~\bibnamefont {Yao}}, \ and\ \bibinfo {author} {\bibfnamefont {M.~Z.}\
  \bibnamefont {Hasan}},\ }\href {\doibase 10.1038/s41563-022-01304-3}
  {\bibfield  {journal} {\bibinfo  {journal} {Nat. Mater.}\ } (\bibinfo {year}
  {2022}),\ 10.1038/s41563-022-01304-3}\BibitemShut {NoStop}%
\bibitem [{\citenamefont {Wang}\ \emph {et~al.}(2019)\citenamefont {Wang},
  \citenamefont {Wieder}, \citenamefont {Li}, \citenamefont {Yan},\ and\
  \citenamefont {Bernevig}}]{cst15}%
  \BibitemOpen
  \bibfield  {author} {\bibinfo {author} {\bibfnamefont {Z.}~\bibnamefont
  {Wang}}, \bibinfo {author} {\bibfnamefont {B.~J.}\ \bibnamefont {Wieder}},
  \bibinfo {author} {\bibfnamefont {J.}~\bibnamefont {Li}}, \bibinfo {author}
  {\bibfnamefont {B.}~\bibnamefont {Yan}}, \ and\ \bibinfo {author}
  {\bibfnamefont {B.~A.}\ \bibnamefont {Bernevig}},\ }\href {\doibase
  10.1103/PhysRevLett.123.186401} {\bibfield  {journal} {\bibinfo  {journal}
  {Phys. Rev. Lett.}\ }\textbf {\bibinfo {volume} {123}},\ \bibinfo {pages}
  {186401} (\bibinfo {year} {2019})}\BibitemShut {NoStop}%
\bibitem [{\citenamefont {Ezawa}(2018{\natexlab{b}})}]{cst17}%
  \BibitemOpen
  \bibfield  {author} {\bibinfo {author} {\bibfnamefont {M.}~\bibnamefont
  {Ezawa}},\ }\href {\doibase 10.1103/PhysRevB.98.045125} {\bibfield  {journal}
  {\bibinfo  {journal} {Phys. Rev. B}\ }\textbf {\bibinfo {volume} {98}},\
  \bibinfo {pages} {045125} (\bibinfo {year} {2018}{\natexlab{b}})}\BibitemShut
  {NoStop}%
\bibitem [{\citenamefont {Sheng}\ \emph {et~al.}(2019)\citenamefont {Sheng},
  \citenamefont {Chen}, \citenamefont {Liu}, \citenamefont {Chen},
  \citenamefont {Yu}, \citenamefont {Zhao},\ and\ \citenamefont
  {Yang}}]{cst18}%
  \BibitemOpen
  \bibfield  {author} {\bibinfo {author} {\bibfnamefont {X.-L.}\ \bibnamefont
  {Sheng}}, \bibinfo {author} {\bibfnamefont {C.}~\bibnamefont {Chen}},
  \bibinfo {author} {\bibfnamefont {H.}~\bibnamefont {Liu}}, \bibinfo {author}
  {\bibfnamefont {Z.}~\bibnamefont {Chen}}, \bibinfo {author} {\bibfnamefont
  {Z.-M.}\ \bibnamefont {Yu}}, \bibinfo {author} {\bibfnamefont {Y.~X.}\
  \bibnamefont {Zhao}}, \ and\ \bibinfo {author} {\bibfnamefont {S.~A.}\
  \bibnamefont {Yang}},\ }\href {\doibase 10.1103/PhysRevLett.123.256402}
  {\bibfield  {journal} {\bibinfo  {journal} {Phys. Rev. Lett.}\ }\textbf
  {\bibinfo {volume} {123}},\ \bibinfo {pages} {256402} (\bibinfo {year}
  {2019})}\BibitemShut {NoStop}%
\bibitem [{\citenamefont {Park}\ \emph {et~al.}(2019)\citenamefont {Park},
  \citenamefont {Kim}, \citenamefont {Cho},\ and\ \citenamefont {Lee}}]{cst19}%
  \BibitemOpen
  \bibfield  {author} {\bibinfo {author} {\bibfnamefont {M.~J.}\ \bibnamefont
  {Park}}, \bibinfo {author} {\bibfnamefont {Y.}~\bibnamefont {Kim}}, \bibinfo
  {author} {\bibfnamefont {G.~Y.}\ \bibnamefont {Cho}}, \ and\ \bibinfo
  {author} {\bibfnamefont {S.}~\bibnamefont {Lee}},\ }\href {\doibase
  10.1103/PhysRevLett.123.216803} {\bibfield  {journal} {\bibinfo  {journal}
  {Phys. Rev. Lett.}\ }\textbf {\bibinfo {volume} {123}},\ \bibinfo {pages}
  {216803} (\bibinfo {year} {2019})}\BibitemShut {NoStop}%
\bibitem [{\citenamefont {Fang}\ and\ \citenamefont {Cano}(2020)}]{cst16}%
  \BibitemOpen
  \bibfield  {author} {\bibinfo {author} {\bibfnamefont {Y.}~\bibnamefont
  {Fang}}\ and\ \bibinfo {author} {\bibfnamefont {J.}~\bibnamefont {Cano}},\
  }\href {\doibase 10.1103/PhysRevB.101.245110} {\bibfield  {journal} {\bibinfo
   {journal} {Phys. Rev. B}\ }\textbf {\bibinfo {volume} {101}},\ \bibinfo
  {pages} {245110} (\bibinfo {year} {2020})}\BibitemShut {NoStop}%
\bibitem [{\citenamefont {Krishtopenko}(2021)}]{cst20}%
  \BibitemOpen
  \bibfield  {author} {\bibinfo {author} {\bibfnamefont {S.~S.}\ \bibnamefont
  {Krishtopenko}},\ }\href {\doibase 10.1038/s41598-021-00577-z} {\bibfield
  {journal} {\bibinfo  {journal} {Sci. Rep.}\ }\textbf {\bibinfo {volume}
  {11}},\ \bibinfo {pages} {21060} (\bibinfo {year} {2021})}\BibitemShut
  {NoStop}%
\bibitem [{\citenamefont {Chen}\ \emph {et~al.}(2019)\citenamefont {Chen},
  \citenamefont {Deng}, \citenamefont {Shi}, \citenamefont {Zhao},
  \citenamefont {Chen},\ and\ \citenamefont {Dong}}]{cst26}%
  \BibitemOpen
  \bibfield  {author} {\bibinfo {author} {\bibfnamefont {X.-D.}\ \bibnamefont
  {Chen}}, \bibinfo {author} {\bibfnamefont {W.-M.}\ \bibnamefont {Deng}},
  \bibinfo {author} {\bibfnamefont {F.-L.}\ \bibnamefont {Shi}}, \bibinfo
  {author} {\bibfnamefont {F.-L.}\ \bibnamefont {Zhao}}, \bibinfo {author}
  {\bibfnamefont {M.}~\bibnamefont {Chen}}, \ and\ \bibinfo {author}
  {\bibfnamefont {J.-W.}\ \bibnamefont {Dong}},\ }\href {\doibase
  10.1103/PhysRevLett.122.233902} {\bibfield  {journal} {\bibinfo  {journal}
  {Phys. Rev. Lett.}\ }\textbf {\bibinfo {volume} {122}},\ \bibinfo {pages}
  {233902} (\bibinfo {year} {2019})}\BibitemShut {NoStop}%
\bibitem [{\citenamefont {Hassan}\ \emph {et~al.}(2019)\citenamefont {Hassan},
  \citenamefont {Kunst}, \citenamefont {Moritz}, \citenamefont {Andler},
  \citenamefont {Bergholtz},\ and\ \citenamefont {Bourennane}}]{cst27}%
  \BibitemOpen
  \bibfield  {author} {\bibinfo {author} {\bibfnamefont {A.~E.}\ \bibnamefont
  {Hassan}}, \bibinfo {author} {\bibfnamefont {F.~K.}\ \bibnamefont {Kunst}},
  \bibinfo {author} {\bibfnamefont {A.}~\bibnamefont {Moritz}}, \bibinfo
  {author} {\bibfnamefont {G.}~\bibnamefont {Andler}}, \bibinfo {author}
  {\bibfnamefont {E.}~\bibnamefont {Bergholtz}}, \ and\ \bibinfo {author}
  {\bibfnamefont {M.}~\bibnamefont {Bourennane}},\ }\href {\doibase
  10.1038/s41566-019-0519-y} {\bibfield  {journal} {\bibinfo  {journal} {Nat.
  Photonics}\ }\textbf {\bibinfo {volume} {13}},\ \bibinfo {pages} {697}
  (\bibinfo {year} {2019})}\BibitemShut {NoStop}%
\bibitem [{\citenamefont {Kim}\ \emph {et~al.}(2020)\citenamefont {Kim},
  \citenamefont {Hwang}, \citenamefont {Smirnova}, \citenamefont {Jeong},
  \citenamefont {Kivshar},\ and\ \citenamefont {Park}}]{cst28}%
  \BibitemOpen
  \bibfield  {author} {\bibinfo {author} {\bibfnamefont {H.-R.}\ \bibnamefont
  {Kim}}, \bibinfo {author} {\bibfnamefont {M.-S.}\ \bibnamefont {Hwang}},
  \bibinfo {author} {\bibfnamefont {D.}~\bibnamefont {Smirnova}}, \bibinfo
  {author} {\bibfnamefont {K.-Y.}\ \bibnamefont {Jeong}}, \bibinfo {author}
  {\bibfnamefont {Y.}~\bibnamefont {Kivshar}}, \ and\ \bibinfo {author}
  {\bibfnamefont {H.-G.}\ \bibnamefont {Park}},\ }\href {\doibase
  10.1038/s41467-020-19609-9} {\bibfield  {journal} {\bibinfo  {journal} {Nat.
  Commun.}\ }\textbf {\bibinfo {volume} {11}},\ \bibinfo {pages} {5758}
  (\bibinfo {year} {2020})}\BibitemShut {NoStop}%
\bibitem [{\citenamefont {Xue}\ \emph {et~al.}(2019)\citenamefont {Xue},
  \citenamefont {Yang}, \citenamefont {Liu}, \citenamefont {Gao}, \citenamefont
  {Chong},\ and\ \citenamefont {Zhang}}]{cst29}%
  \BibitemOpen
  \bibfield  {author} {\bibinfo {author} {\bibfnamefont {H.}~\bibnamefont
  {Xue}}, \bibinfo {author} {\bibfnamefont {Y.}~\bibnamefont {Yang}}, \bibinfo
  {author} {\bibfnamefont {G.}~\bibnamefont {Liu}}, \bibinfo {author}
  {\bibfnamefont {F.}~\bibnamefont {Gao}}, \bibinfo {author} {\bibfnamefont
  {Y.}~\bibnamefont {Chong}}, \ and\ \bibinfo {author} {\bibfnamefont
  {B.}~\bibnamefont {Zhang}},\ }\href {\doibase 10.1103/PhysRevLett.122.244301}
  {\bibfield  {journal} {\bibinfo  {journal} {Phys. Rev. Lett.}\ }\textbf
  {\bibinfo {volume} {122}},\ \bibinfo {pages} {244301} (\bibinfo {year}
  {2019})}\BibitemShut {NoStop}%
\bibitem [{\citenamefont {Ni}\ \emph {et~al.}(2019)\citenamefont {Ni},
  \citenamefont {Weiner}, \citenamefont {Alu},\ and\ \citenamefont
  {Khanikaev}}]{cst30}%
  \BibitemOpen
  \bibfield  {author} {\bibinfo {author} {\bibfnamefont {X.}~\bibnamefont
  {Ni}}, \bibinfo {author} {\bibfnamefont {M.}~\bibnamefont {Weiner}}, \bibinfo
  {author} {\bibfnamefont {A.}~\bibnamefont {Alu}}, \ and\ \bibinfo {author}
  {\bibfnamefont {B.}~\bibnamefont {Khanikaev}},\ }\href {\doibase
  10.1038/s41563-018-0252-9} {\bibfield  {journal} {\bibinfo  {journal} {Nat.
  Mater.}\ }\textbf {\bibinfo {volume} {18}},\ \bibinfo {pages} {113} (\bibinfo
  {year} {2019})}\BibitemShut {NoStop}%
\bibitem [{\citenamefont {He}\ \emph {et~al.}(2019)\citenamefont {He},
  \citenamefont {Yu}, \citenamefont {Wang}, \citenamefont {Ge}, \citenamefont
  {Ruan}, \citenamefont {Zhang}, \citenamefont {Lu},\ and\ \citenamefont
  {Chen}}]{cst31}%
  \BibitemOpen
  \bibfield  {author} {\bibinfo {author} {\bibfnamefont {C.}~\bibnamefont
  {He}}, \bibinfo {author} {\bibfnamefont {S.-Y.}\ \bibnamefont {Yu}}, \bibinfo
  {author} {\bibfnamefont {H.}~\bibnamefont {Wang}}, \bibinfo {author}
  {\bibfnamefont {H.}~\bibnamefont {Ge}}, \bibinfo {author} {\bibfnamefont
  {J.}~\bibnamefont {Ruan}}, \bibinfo {author} {\bibfnamefont {H.}~\bibnamefont
  {Zhang}}, \bibinfo {author} {\bibfnamefont {M.-H.}\ \bibnamefont {Lu}}, \
  and\ \bibinfo {author} {\bibfnamefont {Y.-F.}\ \bibnamefont {Chen}},\ }\href
  {\doibase 10.1103/PhysRevLett.123.195503} {\bibfield  {journal} {\bibinfo
  {journal} {Phys. Rev. Lett.}\ }\textbf {\bibinfo {volume} {123}},\ \bibinfo
  {pages} {195503} (\bibinfo {year} {2019})}\BibitemShut {NoStop}%
\bibitem [{\citenamefont {Imhof}\ \emph {et~al.}(2018)\citenamefont {Imhof},
  \citenamefont {Berger}, \citenamefont {Bayer}, \citenamefont {Brehm},
  \citenamefont {Molenkamp}, \citenamefont {Kiessling}, \citenamefont
  {Schindler}, \citenamefont {Lee}, \citenamefont {Greiter}, \citenamefont
  {Neupert},\ and\ \citenamefont {Thomalem}}]{cst32}%
  \BibitemOpen
  \bibfield  {author} {\bibinfo {author} {\bibfnamefont {S.}~\bibnamefont
  {Imhof}}, \bibinfo {author} {\bibfnamefont {C.}~\bibnamefont {Berger}},
  \bibinfo {author} {\bibfnamefont {F.}~\bibnamefont {Bayer}}, \bibinfo
  {author} {\bibfnamefont {J.}~\bibnamefont {Brehm}}, \bibinfo {author}
  {\bibfnamefont {L.~W.}\ \bibnamefont {Molenkamp}}, \bibinfo {author}
  {\bibfnamefont {T.}~\bibnamefont {Kiessling}}, \bibinfo {author}
  {\bibfnamefont {F.}~\bibnamefont {Schindler}}, \bibinfo {author}
  {\bibfnamefont {C.~H.}\ \bibnamefont {Lee}}, \bibinfo {author} {\bibfnamefont
  {M.}~\bibnamefont {Greiter}}, \bibinfo {author} {\bibfnamefont
  {T.}~\bibnamefont {Neupert}}, \ and\ \bibinfo {author} {\bibfnamefont
  {R.}~\bibnamefont {Thomalem}},\ }\href {\doibase 10.1038/s41567-018-0246-1}
  {\bibfield  {journal} {\bibinfo  {journal} {Nat. Phys.}\ }\textbf {\bibinfo
  {volume} {14}},\ \bibinfo {pages} {925} (\bibinfo {year} {2018})}\BibitemShut
  {NoStop}%
\bibitem [{\citenamefont {Serra-Garcia}\ \emph {et~al.}(2019)\citenamefont
  {Serra-Garcia}, \citenamefont {S\"usstrunk},\ and\ \citenamefont
  {Huber}}]{cst33}%
  \BibitemOpen
  \bibfield  {author} {\bibinfo {author} {\bibfnamefont {M.}~\bibnamefont
  {Serra-Garcia}}, \bibinfo {author} {\bibfnamefont {R.}~\bibnamefont
  {S\"usstrunk}}, \ and\ \bibinfo {author} {\bibfnamefont {S.~D.}\ \bibnamefont
  {Huber}},\ }\href {\doibase 10.1103/PhysRevB.99.020304} {\bibfield  {journal}
  {\bibinfo  {journal} {Phys. Rev. B}\ }\textbf {\bibinfo {volume} {99}},\
  \bibinfo {pages} {020304} (\bibinfo {year} {2019})}\BibitemShut {NoStop}%
\bibitem [{\citenamefont {Bao}\ \emph {et~al.}(2019)\citenamefont {Bao},
  \citenamefont {Zou}, \citenamefont {Zhang}, \citenamefont {He}, \citenamefont
  {Sun},\ and\ \citenamefont {Zhang}}]{cst34}%
  \BibitemOpen
  \bibfield  {author} {\bibinfo {author} {\bibfnamefont {J.}~\bibnamefont
  {Bao}}, \bibinfo {author} {\bibfnamefont {D.}~\bibnamefont {Zou}}, \bibinfo
  {author} {\bibfnamefont {W.}~\bibnamefont {Zhang}}, \bibinfo {author}
  {\bibfnamefont {W.}~\bibnamefont {He}}, \bibinfo {author} {\bibfnamefont
  {H.}~\bibnamefont {Sun}}, \ and\ \bibinfo {author} {\bibfnamefont
  {X.}~\bibnamefont {Zhang}},\ }\href {\doibase 10.1103/PhysRevB.100.201406}
  {\bibfield  {journal} {\bibinfo  {journal} {Phys. Rev. B}\ }\textbf {\bibinfo
  {volume} {100}},\ \bibinfo {pages} {201406} (\bibinfo {year}
  {2019})}\BibitemShut {NoStop}%
\bibitem [{\citenamefont {Ezawa}(2019)}]{cst34b}%
  \BibitemOpen
  \bibfield  {author} {\bibinfo {author} {\bibfnamefont {M.}~\bibnamefont
  {Ezawa}},\ }\href {\doibase 10.1103/PhysRevB.100.045407} {\bibfield
  {journal} {\bibinfo  {journal} {Phys. Rev. B}\ }\textbf {\bibinfo {volume}
  {100}},\ \bibinfo {pages} {045407} (\bibinfo {year} {2019})}\BibitemShut
  {NoStop}%
\bibitem [{\citenamefont {Langbehn}\ \emph
  {et~al.}(2017{\natexlab{b}})\citenamefont {Langbehn}, \citenamefont {Peng},
  \citenamefont {Trifunovic}, \citenamefont {von Oppen},\ and\ \citenamefont
  {Brouwer}}]{cst21}%
  \BibitemOpen
  \bibfield  {author} {\bibinfo {author} {\bibfnamefont {J.}~\bibnamefont
  {Langbehn}}, \bibinfo {author} {\bibfnamefont {Y.}~\bibnamefont {Peng}},
  \bibinfo {author} {\bibfnamefont {L.}~\bibnamefont {Trifunovic}}, \bibinfo
  {author} {\bibfnamefont {F.}~\bibnamefont {von Oppen}}, \ and\ \bibinfo
  {author} {\bibfnamefont {P.~W.}\ \bibnamefont {Brouwer}},\ }\href {\doibase
  10.1103/PhysRevLett.119.246401} {\bibfield  {journal} {\bibinfo  {journal}
  {Phys. Rev. Lett.}\ }\textbf {\bibinfo {volume} {119}},\ \bibinfo {pages}
  {246401} (\bibinfo {year} {2017}{\natexlab{b}})}\BibitemShut {NoStop}%
\bibitem [{\citenamefont {Khalaf}(2018)}]{cst22}%
  \BibitemOpen
  \bibfield  {author} {\bibinfo {author} {\bibfnamefont {E.}~\bibnamefont
  {Khalaf}},\ }\href {\doibase 10.1103/PhysRevB.97.205136} {\bibfield
  {journal} {\bibinfo  {journal} {Phys. Rev. B}\ }\textbf {\bibinfo {volume}
  {97}},\ \bibinfo {pages} {205136} (\bibinfo {year} {2018})}\BibitemShut
  {NoStop}%
\bibitem [{\citenamefont {Wang}\ \emph {et~al.}(2018)\citenamefont {Wang},
  \citenamefont {Liu}, \citenamefont {Lu},\ and\ \citenamefont
  {Zhang}}]{cst23}%
  \BibitemOpen
  \bibfield  {author} {\bibinfo {author} {\bibfnamefont {Q.}~\bibnamefont
  {Wang}}, \bibinfo {author} {\bibfnamefont {C.-C.}\ \bibnamefont {Liu}},
  \bibinfo {author} {\bibfnamefont {Y.-M.}\ \bibnamefont {Lu}}, \ and\ \bibinfo
  {author} {\bibfnamefont {F.}~\bibnamefont {Zhang}},\ }\href {\doibase
  10.1103/PhysRevLett.121.186801} {\bibfield  {journal} {\bibinfo  {journal}
  {Phys. Rev. Lett.}\ }\textbf {\bibinfo {volume} {121}},\ \bibinfo {pages}
  {186801} (\bibinfo {year} {2018})}\BibitemShut {NoStop}%
\bibitem [{\citenamefont {Yan}\ \emph {et~al.}(2018)\citenamefont {Yan},
  \citenamefont {Song},\ and\ \citenamefont {Wang}}]{cst24}%
  \BibitemOpen
  \bibfield  {author} {\bibinfo {author} {\bibfnamefont {Z.}~\bibnamefont
  {Yan}}, \bibinfo {author} {\bibfnamefont {F.}~\bibnamefont {Song}}, \ and\
  \bibinfo {author} {\bibfnamefont {Z.}~\bibnamefont {Wang}},\ }\href {\doibase
  10.1103/PhysRevLett.121.096803} {\bibfield  {journal} {\bibinfo  {journal}
  {Phys. Rev. Lett.}\ }\textbf {\bibinfo {volume} {121}},\ \bibinfo {pages}
  {096803} (\bibinfo {year} {2018})}\BibitemShut {NoStop}%
\bibitem [{\citenamefont {Yan}(2019)}]{cst25}%
  \BibitemOpen
  \bibfield  {author} {\bibinfo {author} {\bibfnamefont {Z.}~\bibnamefont
  {Yan}},\ }\href {\doibase 10.1103/PhysRevLett.123.177001} {\bibfield
  {journal} {\bibinfo  {journal} {Phys. Rev. Lett.}\ }\textbf {\bibinfo
  {volume} {123}},\ \bibinfo {pages} {177001} (\bibinfo {year}
  {2019})}\BibitemShut {NoStop}%
\bibitem [{\citenamefont {Jackiw}\ and\ \citenamefont {Rebbi}(1976)}]{cst42}%
  \BibitemOpen
  \bibfield  {author} {\bibinfo {author} {\bibfnamefont {R.}~\bibnamefont
  {Jackiw}}\ and\ \bibinfo {author} {\bibfnamefont {C.}~\bibnamefont {Rebbi}},\
  }\href {\doibase 10.1103/PhysRevD.13.3398} {\bibfield  {journal} {\bibinfo
  {journal} {Phys. Rev. D}\ }\textbf {\bibinfo {volume} {13}},\ \bibinfo
  {pages} {3398} (\bibinfo {year} {1976})}\BibitemShut {NoStop}%
\bibitem [{\citenamefont {Ezawa}(2018{\natexlab{c}})}]{cst47a00}%
  \BibitemOpen
  \bibfield  {author} {\bibinfo {author} {\bibfnamefont {M.}~\bibnamefont
  {Ezawa}},\ }\href {\doibase 10.1103/PhysRevLett.121.116801} {\bibfield
  {journal} {\bibinfo  {journal} {Phys. Rev. Lett.}\ }\textbf {\bibinfo
  {volume} {121}},\ \bibinfo {pages} {116801} (\bibinfo {year}
  {2018}{\natexlab{c}})}\BibitemShut {NoStop}%
\bibitem [{\citenamefont {Ren}\ \emph {et~al.}(2020)\citenamefont {Ren},
  \citenamefont {Qiao},\ and\ \citenamefont {Niu}}]{cst47}%
  \BibitemOpen
  \bibfield  {author} {\bibinfo {author} {\bibfnamefont {Y.}~\bibnamefont
  {Ren}}, \bibinfo {author} {\bibfnamefont {Z.}~\bibnamefont {Qiao}}, \ and\
  \bibinfo {author} {\bibfnamefont {Q.}~\bibnamefont {Niu}},\ }\href {\doibase
  10.1103/PhysRevLett.124.166804} {\bibfield  {journal} {\bibinfo  {journal}
  {Phys. Rev. Lett.}\ }\textbf {\bibinfo {volume} {124}},\ \bibinfo {pages}
  {166804} (\bibinfo {year} {2020})}\BibitemShut {NoStop}%
\bibitem [{\citenamefont {Chen}\ \emph {et~al.}(2020)\citenamefont {Chen},
  \citenamefont {Song}, \citenamefont {Zhao}, \citenamefont {Chen},
  \citenamefont {Yu}, \citenamefont {Sheng},\ and\ \citenamefont
  {Yang}}]{cst48}%
  \BibitemOpen
  \bibfield  {author} {\bibinfo {author} {\bibfnamefont {C.}~\bibnamefont
  {Chen}}, \bibinfo {author} {\bibfnamefont {Z.}~\bibnamefont {Song}}, \bibinfo
  {author} {\bibfnamefont {J.-Z.}\ \bibnamefont {Zhao}}, \bibinfo {author}
  {\bibfnamefont {Z.}~\bibnamefont {Chen}}, \bibinfo {author} {\bibfnamefont
  {Z.-M.}\ \bibnamefont {Yu}}, \bibinfo {author} {\bibfnamefont {X.-L.}\
  \bibnamefont {Sheng}}, \ and\ \bibinfo {author} {\bibfnamefont {S.~A.}\
  \bibnamefont {Yang}},\ }\href {\doibase 10.1103/PhysRevLett.125.056402}
  {\bibfield  {journal} {\bibinfo  {journal} {Phys. Rev. Lett.}\ }\textbf
  {\bibinfo {volume} {125}},\ \bibinfo {pages} {056402} (\bibinfo {year}
  {2020})}\BibitemShut {NoStop}%
\bibitem [{\citenamefont {Bernevig}\ \emph
  {et~al.}(2006{\natexlab{b}})\citenamefont {Bernevig}, \citenamefont
  {Hughes},\ and\ \citenamefont {Zhang}}]{cst49}%
  \BibitemOpen
  \bibfield  {author} {\bibinfo {author} {\bibfnamefont {B.~A.}\ \bibnamefont
  {Bernevig}}, \bibinfo {author} {\bibfnamefont {T.~L.}\ \bibnamefont
  {Hughes}}, \ and\ \bibinfo {author} {\bibfnamefont {S.-C.}\ \bibnamefont
  {Zhang}},\ }\href {\doibase 10.1126/science.1133734} {\bibfield  {journal}
  {\bibinfo  {journal} {Science}\ }\textbf {\bibinfo {volume} {314}},\ \bibinfo
  {pages} {1757} (\bibinfo {year} {2006}{\natexlab{b}})}\BibitemShut {NoStop}%
\bibitem [{\citenamefont {K\"{o}nig}\ \emph {et~al.}(2007)\citenamefont
  {K\"{o}nig}, \citenamefont {Wiedmann}, \citenamefont {Br\"{u}ne},
  \citenamefont {Roth}, \citenamefont {Buhmann}, \citenamefont {Molenkamp},
  \citenamefont {Qi},\ and\ \citenamefont {Zhang}}]{cst50}%
  \BibitemOpen
  \bibfield  {author} {\bibinfo {author} {\bibfnamefont {M.}~\bibnamefont
  {K\"{o}nig}}, \bibinfo {author} {\bibfnamefont {S.}~\bibnamefont {Wiedmann}},
  \bibinfo {author} {\bibfnamefont {C.}~\bibnamefont {Br\"{u}ne}}, \bibinfo
  {author} {\bibfnamefont {A.}~\bibnamefont {Roth}}, \bibinfo {author}
  {\bibfnamefont {H.}~\bibnamefont {Buhmann}}, \bibinfo {author} {\bibfnamefont
  {L.~W.}\ \bibnamefont {Molenkamp}}, \bibinfo {author} {\bibfnamefont {X.-L.}\
  \bibnamefont {Qi}}, \ and\ \bibinfo {author} {\bibfnamefont {S.-C.}\
  \bibnamefont {Zhang}},\ }\href {\doibase 10.1126/science.1148047} {\bibfield
  {journal} {\bibinfo  {journal} {Science}\ }\textbf {\bibinfo {volume}
  {318}},\ \bibinfo {pages} {766} (\bibinfo {year} {2007})}\BibitemShut
  {NoStop}%
\bibitem [{\citenamefont {Liu}\ \emph {et~al.}(2008)\citenamefont {Liu},
  \citenamefont {Hughes}, \citenamefont {Qi}, \citenamefont {Wang},\ and\
  \citenamefont {Zhang}}]{cst51}%
  \BibitemOpen
  \bibfield  {author} {\bibinfo {author} {\bibfnamefont {C.}~\bibnamefont
  {Liu}}, \bibinfo {author} {\bibfnamefont {T.~L.}\ \bibnamefont {Hughes}},
  \bibinfo {author} {\bibfnamefont {X.-L.}\ \bibnamefont {Qi}}, \bibinfo
  {author} {\bibfnamefont {K.}~\bibnamefont {Wang}}, \ and\ \bibinfo {author}
  {\bibfnamefont {S.-C.}\ \bibnamefont {Zhang}},\ }\href {\doibase
  10.1103/PhysRevLett.100.236601} {\bibfield  {journal} {\bibinfo  {journal}
  {Phys. Rev. Lett.}\ }\textbf {\bibinfo {volume} {100}},\ \bibinfo {pages}
  {236601} (\bibinfo {year} {2008})}\BibitemShut {NoStop}%
\bibitem [{\citenamefont {Knez}\ \emph {et~al.}(2011)\citenamefont {Knez},
  \citenamefont {Du},\ and\ \citenamefont {Sullivan}}]{cst52}%
  \BibitemOpen
  \bibfield  {author} {\bibinfo {author} {\bibfnamefont {I.}~\bibnamefont
  {Knez}}, \bibinfo {author} {\bibfnamefont {R.-R.}\ \bibnamefont {Du}}, \ and\
  \bibinfo {author} {\bibfnamefont {G.}~\bibnamefont {Sullivan}},\ }\href
  {\doibase 10.1103/PhysRevLett.107.136603} {\bibfield  {journal} {\bibinfo
  {journal} {Phys. Rev. Lett.}\ }\textbf {\bibinfo {volume} {107}},\ \bibinfo
  {pages} {136603} (\bibinfo {year} {2011})}\BibitemShut {NoStop}%
\bibitem [{\citenamefont {Krishtopenko}\ and\ \citenamefont
  {Teppe}(2018)}]{cst53}%
  \BibitemOpen
  \bibfield  {author} {\bibinfo {author} {\bibfnamefont {S.~S.}\ \bibnamefont
  {Krishtopenko}}\ and\ \bibinfo {author} {\bibfnamefont {F.}~\bibnamefont
  {Teppe}},\ }\href {\doibase 10.1126/sciadv.aap7529} {\bibfield  {journal}
  {\bibinfo  {journal} {Sci. Adv.}\ }\textbf {\bibinfo {volume} {4}},\ \bibinfo
  {pages} {eaap7529} (\bibinfo {year} {2018})}\BibitemShut {NoStop}%
\bibitem [{\citenamefont {Dresselhaus}(1955)}]{cst54}%
  \BibitemOpen
  \bibfield  {author} {\bibinfo {author} {\bibfnamefont {G.}~\bibnamefont
  {Dresselhaus}},\ }\href {\doibase 10.1103/PhysRev.100.580} {\bibfield
  {journal} {\bibinfo  {journal} {Phys. Rev.}\ }\textbf {\bibinfo {volume}
  {100}},\ \bibinfo {pages} {580} (\bibinfo {year} {1955})}\BibitemShut
  {NoStop}%
\bibitem [{\citenamefont {Ivchenko}\ \emph {et~al.}(1996)\citenamefont
  {Ivchenko}, \citenamefont {Kaminski},\ and\ \citenamefont
  {R\"ossler}}]{cst55}%
  \BibitemOpen
  \bibfield  {author} {\bibinfo {author} {\bibfnamefont {E.~L.}\ \bibnamefont
  {Ivchenko}}, \bibinfo {author} {\bibfnamefont {A.~Y.}\ \bibnamefont
  {Kaminski}}, \ and\ \bibinfo {author} {\bibfnamefont {U.}~\bibnamefont
  {R\"ossler}},\ }\href {\doibase 10.1103/PhysRevB.54.5852} {\bibfield
  {journal} {\bibinfo  {journal} {Phys. Rev. B}\ }\textbf {\bibinfo {volume}
  {54}},\ \bibinfo {pages} {5852} (\bibinfo {year} {1996})}\BibitemShut
  {NoStop}%
\bibitem [{\citenamefont {Durnev}\ and\ \citenamefont
  {Tarasenko}(2016)}]{cst56}%
  \BibitemOpen
  \bibfield  {author} {\bibinfo {author} {\bibfnamefont {M.~V.}\ \bibnamefont
  {Durnev}}\ and\ \bibinfo {author} {\bibfnamefont {S.~A.}\ \bibnamefont
  {Tarasenko}},\ }\href {\doibase 10.1103/PhysRevB.93.075434} {\bibfield
  {journal} {\bibinfo  {journal} {Phys. Rev. B}\ }\textbf {\bibinfo {volume}
  {93}},\ \bibinfo {pages} {075434} (\bibinfo {year} {2016})}\BibitemShut
  {NoStop}%
\bibitem [{\citenamefont {B\"{u}ttner}\ \emph {et~al.}(2011)\citenamefont
  {B\"{u}ttner}, \citenamefont {Liu}, \citenamefont {Tkachov}, \citenamefont
  {Novik}, \citenamefont {Br\"{u}ne}, \citenamefont {Buhmann}, \citenamefont
  {Hankiewicz}, \citenamefont {Recher}, \citenamefont {Trauzettel},
  \citenamefont {Zhang},\ and\ \citenamefont {Molenkamp}}]{A1}%
  \BibitemOpen
  \bibfield  {author} {\bibinfo {author} {\bibfnamefont {B.}~\bibnamefont
  {B\"{u}ttner}}, \bibinfo {author} {\bibfnamefont {C.}~\bibnamefont {Liu}},
  \bibinfo {author} {\bibfnamefont {G.}~\bibnamefont {Tkachov}}, \bibinfo
  {author} {\bibfnamefont {E.}~\bibnamefont {Novik}}, \bibinfo {author}
  {\bibfnamefont {C.}~\bibnamefont {Br\"{u}ne}}, \bibinfo {author}
  {\bibfnamefont {H.}~\bibnamefont {Buhmann}}, \bibinfo {author} {\bibfnamefont
  {E.}~\bibnamefont {Hankiewicz}}, \bibinfo {author} {\bibfnamefont
  {P.}~\bibnamefont {Recher}}, \bibinfo {author} {\bibfnamefont
  {B.}~\bibnamefont {Trauzettel}}, \bibinfo {author} {\bibfnamefont
  {S.}~\bibnamefont {Zhang}}, \ and\ \bibinfo {author} {\bibfnamefont
  {L.}~\bibnamefont {Molenkamp}},\ }\href {\doibase 10.1038/nphys1914}
  {\bibfield  {journal} {\bibinfo  {journal} {Nat. Phys.}\ }\textbf {\bibinfo
  {volume} {7}},\ \bibinfo {pages} {418} (\bibinfo {year} {2011})}\BibitemShut
  {NoStop}%
\bibitem [{\citenamefont {Kadykov}\ \emph {et~al.}(2018)\citenamefont
  {Kadykov}, \citenamefont {Krishtopenko}, \citenamefont {Jouault},
  \citenamefont {Desrat}, \citenamefont {Knap}, \citenamefont {Ruffenach},
  \citenamefont {Consejo}, \citenamefont {Torres}, \citenamefont {Morozov},
  \citenamefont {Mikhailov}, \citenamefont {Dvoretskii},\ and\ \citenamefont
  {Teppe}}]{A2}%
  \BibitemOpen
  \bibfield  {author} {\bibinfo {author} {\bibfnamefont {A.~M.}\ \bibnamefont
  {Kadykov}}, \bibinfo {author} {\bibfnamefont {S.~S.}\ \bibnamefont
  {Krishtopenko}}, \bibinfo {author} {\bibfnamefont {B.}~\bibnamefont
  {Jouault}}, \bibinfo {author} {\bibfnamefont {W.}~\bibnamefont {Desrat}},
  \bibinfo {author} {\bibfnamefont {W.}~\bibnamefont {Knap}}, \bibinfo {author}
  {\bibfnamefont {S.}~\bibnamefont {Ruffenach}}, \bibinfo {author}
  {\bibfnamefont {C.}~\bibnamefont {Consejo}}, \bibinfo {author} {\bibfnamefont
  {J.}~\bibnamefont {Torres}}, \bibinfo {author} {\bibfnamefont {S.~V.}\
  \bibnamefont {Morozov}}, \bibinfo {author} {\bibfnamefont {N.~N.}\
  \bibnamefont {Mikhailov}}, \bibinfo {author} {\bibfnamefont {S.~A.}\
  \bibnamefont {Dvoretskii}}, \ and\ \bibinfo {author} {\bibfnamefont
  {F.}~\bibnamefont {Teppe}},\ }\href {\doibase 10.1103/PhysRevLett.120.086401}
  {\bibfield  {journal} {\bibinfo  {journal} {Phys. Rev. Lett.}\ }\textbf
  {\bibinfo {volume} {120}},\ \bibinfo {pages} {086401} (\bibinfo {year}
  {2018})}\BibitemShut {NoStop}%
\bibitem [{\citenamefont {Br\"{u}ne}\ \emph {et~al.}(2012)\citenamefont
  {Br\"{u}ne}, \citenamefont {Roth}, \citenamefont {Buhmann}, \citenamefont
  {Hankiewicz}, \citenamefont {Molenkamp}, \citenamefont {Maciejko},
  \citenamefont {Qi},\ and\ \citenamefont {Zhang}}]{A3}%
  \BibitemOpen
  \bibfield  {author} {\bibinfo {author} {\bibfnamefont {C.}~\bibnamefont
  {Br\"{u}ne}}, \bibinfo {author} {\bibfnamefont {A.}~\bibnamefont {Roth}},
  \bibinfo {author} {\bibfnamefont {H.}~\bibnamefont {Buhmann}}, \bibinfo
  {author} {\bibfnamefont {E.~M.}\ \bibnamefont {Hankiewicz}}, \bibinfo
  {author} {\bibfnamefont {L.~W.}\ \bibnamefont {Molenkamp}}, \bibinfo {author}
  {\bibfnamefont {J.}~\bibnamefont {Maciejko}}, \bibinfo {author}
  {\bibfnamefont {X.-L.}\ \bibnamefont {Qi}}, \ and\ \bibinfo {author}
  {\bibfnamefont {S.-C.}\ \bibnamefont {Zhang}},\ }\href {\doibase
  10.1038/nphys2322} {\bibfield  {journal} {\bibinfo  {journal} {Nat. Phys.}\
  }\textbf {\bibinfo {volume} {8}},\ \bibinfo {pages} {485} (\bibinfo {year}
  {2012})}\BibitemShut {NoStop}%
\bibitem [{\citenamefont {Kadykov}\ \emph {et~al.}(2015)\citenamefont
  {Kadykov}, \citenamefont {Teppe}, \citenamefont {Consejo}, \citenamefont
  {Viti}, \citenamefont {Vitiello}, \citenamefont {Krishtopenko}, \citenamefont
  {Ruffenach}, \citenamefont {Morozov}, \citenamefont {Marcinkiewicz},
  \citenamefont {Desrat}, \citenamefont {Dyakonova}, \citenamefont {Knap},
  \citenamefont {Gavrilenko}, \citenamefont {Mikhailov},\ and\ \citenamefont
  {Dvoretsky}}]{A4}%
  \BibitemOpen
  \bibfield  {author} {\bibinfo {author} {\bibfnamefont {A.~M.}\ \bibnamefont
  {Kadykov}}, \bibinfo {author} {\bibfnamefont {F.}~\bibnamefont {Teppe}},
  \bibinfo {author} {\bibfnamefont {C.}~\bibnamefont {Consejo}}, \bibinfo
  {author} {\bibfnamefont {L.}~\bibnamefont {Viti}}, \bibinfo {author}
  {\bibfnamefont {M.~S.}\ \bibnamefont {Vitiello}}, \bibinfo {author}
  {\bibfnamefont {S.~S.}\ \bibnamefont {Krishtopenko}}, \bibinfo {author}
  {\bibfnamefont {S.}~\bibnamefont {Ruffenach}}, \bibinfo {author}
  {\bibfnamefont {S.~V.}\ \bibnamefont {Morozov}}, \bibinfo {author}
  {\bibfnamefont {M.}~\bibnamefont {Marcinkiewicz}}, \bibinfo {author}
  {\bibfnamefont {W.}~\bibnamefont {Desrat}}, \bibinfo {author} {\bibfnamefont
  {N.}~\bibnamefont {Dyakonova}}, \bibinfo {author} {\bibfnamefont
  {W.}~\bibnamefont {Knap}}, \bibinfo {author} {\bibfnamefont {V.~I.}\
  \bibnamefont {Gavrilenko}}, \bibinfo {author} {\bibfnamefont {N.~N.}\
  \bibnamefont {Mikhailov}}, \ and\ \bibinfo {author} {\bibfnamefont {S.~A.}\
  \bibnamefont {Dvoretsky}},\ }\href {\doibase 10.1063/1.4932943} {\bibfield
  {journal} {\bibinfo  {journal} {Appl. Phys. Lett.}\ }\textbf {\bibinfo
  {volume} {107}},\ \bibinfo {pages} {152101} (\bibinfo {year}
  {2015})}\BibitemShut {NoStop}%
\bibitem [{\citenamefont {Kadykov}\ \emph {et~al.}(2016)\citenamefont
  {Kadykov}, \citenamefont {Torres}, \citenamefont {Krishtopenko},
  \citenamefont {Consejo}, \citenamefont {Ruffenach}, \citenamefont
  {Marcinkiewicz}, \citenamefont {But}, \citenamefont {Knap}, \citenamefont
  {Morozov}, \citenamefont {Gavrilenko}, \citenamefont {Mikhailov},
  \citenamefont {Dvoretsky},\ and\ \citenamefont {Teppe}}]{A5}%
  \BibitemOpen
  \bibfield  {author} {\bibinfo {author} {\bibfnamefont {A.~M.}\ \bibnamefont
  {Kadykov}}, \bibinfo {author} {\bibfnamefont {J.}~\bibnamefont {Torres}},
  \bibinfo {author} {\bibfnamefont {S.~S.}\ \bibnamefont {Krishtopenko}},
  \bibinfo {author} {\bibfnamefont {C.}~\bibnamefont {Consejo}}, \bibinfo
  {author} {\bibfnamefont {S.}~\bibnamefont {Ruffenach}}, \bibinfo {author}
  {\bibfnamefont {M.}~\bibnamefont {Marcinkiewicz}}, \bibinfo {author}
  {\bibfnamefont {D.}~\bibnamefont {But}}, \bibinfo {author} {\bibfnamefont
  {W.}~\bibnamefont {Knap}}, \bibinfo {author} {\bibfnamefont {S.~V.}\
  \bibnamefont {Morozov}}, \bibinfo {author} {\bibfnamefont {V.~I.}\
  \bibnamefont {Gavrilenko}}, \bibinfo {author} {\bibfnamefont {N.~N.}\
  \bibnamefont {Mikhailov}}, \bibinfo {author} {\bibfnamefont {S.~A.}\
  \bibnamefont {Dvoretsky}}, \ and\ \bibinfo {author} {\bibfnamefont
  {F.}~\bibnamefont {Teppe}},\ }\href {\doibase 10.1063/1.4955018} {\bibfield
  {journal} {\bibinfo  {journal} {Appl. Phys. Lett.}\ }\textbf {\bibinfo
  {volume} {108}},\ \bibinfo {pages} {262102} (\bibinfo {year}
  {2016})}\BibitemShut {NoStop}%
\bibitem [{\citenamefont {Olshanetsky}\ \emph {et~al.}(2018)\citenamefont
  {Olshanetsky}, \citenamefont {Kvon}, \citenamefont {Gusev}, \citenamefont
  {Mikhailov},\ and\ \citenamefont {Dvoretsky}}]{A6}%
  \BibitemOpen
  \bibfield  {author} {\bibinfo {author} {\bibfnamefont {E.}~\bibnamefont
  {Olshanetsky}}, \bibinfo {author} {\bibfnamefont {Z.}~\bibnamefont {Kvon}},
  \bibinfo {author} {\bibfnamefont {G.}~\bibnamefont {Gusev}}, \bibinfo
  {author} {\bibfnamefont {N.}~\bibnamefont {Mikhailov}}, \ and\ \bibinfo
  {author} {\bibfnamefont {S.}~\bibnamefont {Dvoretsky}},\ }\href {\doibase
  https://doi.org/10.1016/j.physe.2018.02.005} {\bibfield  {journal} {\bibinfo
  {journal} {Physica E Low. Dimens. Syst. Nanostruct.}\ }\textbf {\bibinfo
  {volume} {99}},\ \bibinfo {pages} {335 } (\bibinfo {year}
  {2018})}\BibitemShut {NoStop}%
\bibitem [{\citenamefont {Krishtopenko}\ \emph {et~al.}(2020)\citenamefont
  {Krishtopenko}, \citenamefont {Kadykov}, \citenamefont {Gebert},
  \citenamefont {Ruffenach}, \citenamefont {Consejo}, \citenamefont {Torres},
  \citenamefont {Avogadri}, \citenamefont {Jouault}, \citenamefont {Knap},
  \citenamefont {Mikhailov}, \citenamefont {Dvoretskii},\ and\ \citenamefont
  {Teppe}}]{A7}%
  \BibitemOpen
  \bibfield  {author} {\bibinfo {author} {\bibfnamefont {S.~S.}\ \bibnamefont
  {Krishtopenko}}, \bibinfo {author} {\bibfnamefont {A.~M.}\ \bibnamefont
  {Kadykov}}, \bibinfo {author} {\bibfnamefont {S.}~\bibnamefont {Gebert}},
  \bibinfo {author} {\bibfnamefont {S.}~\bibnamefont {Ruffenach}}, \bibinfo
  {author} {\bibfnamefont {C.}~\bibnamefont {Consejo}}, \bibinfo {author}
  {\bibfnamefont {J.}~\bibnamefont {Torres}}, \bibinfo {author} {\bibfnamefont
  {C.}~\bibnamefont {Avogadri}}, \bibinfo {author} {\bibfnamefont
  {B.}~\bibnamefont {Jouault}}, \bibinfo {author} {\bibfnamefont
  {W.}~\bibnamefont {Knap}}, \bibinfo {author} {\bibfnamefont {N.~N.}\
  \bibnamefont {Mikhailov}}, \bibinfo {author} {\bibfnamefont {S.~A.}\
  \bibnamefont {Dvoretskii}}, \ and\ \bibinfo {author} {\bibfnamefont
  {F.}~\bibnamefont {Teppe}},\ }\href {\doibase 10.1103/PhysRevB.102.041404}
  {\bibfield  {journal} {\bibinfo  {journal} {Phys. Rev. B}\ }\textbf {\bibinfo
  {volume} {102}},\ \bibinfo {pages} {041404} (\bibinfo {year}
  {2020})}\BibitemShut {NoStop}%
\bibitem [{\citenamefont {Olesberg}\ \emph {et~al.}(2001)\citenamefont
  {Olesberg}, \citenamefont {Lau}, \citenamefont {Flatt\'e}, \citenamefont
  {Yu}, \citenamefont {Altunkaya}, \citenamefont {Shaw}, \citenamefont
  {Hasenberg},\ and\ \citenamefont {Boggess}}]{A8}%
  \BibitemOpen
  \bibfield  {author} {\bibinfo {author} {\bibfnamefont {J.~T.}\ \bibnamefont
  {Olesberg}}, \bibinfo {author} {\bibfnamefont {W.~H.}\ \bibnamefont {Lau}},
  \bibinfo {author} {\bibfnamefont {M.~E.}\ \bibnamefont {Flatt\'e}}, \bibinfo
  {author} {\bibfnamefont {C.}~\bibnamefont {Yu}}, \bibinfo {author}
  {\bibfnamefont {E.}~\bibnamefont {Altunkaya}}, \bibinfo {author}
  {\bibfnamefont {E.~M.}\ \bibnamefont {Shaw}}, \bibinfo {author}
  {\bibfnamefont {T.~C.}\ \bibnamefont {Hasenberg}}, \ and\ \bibinfo {author}
  {\bibfnamefont {T.~F.}\ \bibnamefont {Boggess}},\ }\href {\doibase
  10.1103/PhysRevB.64.201301} {\bibfield  {journal} {\bibinfo  {journal} {Phys.
  Rev. B}\ }\textbf {\bibinfo {volume} {64}},\ \bibinfo {pages} {201301}
  (\bibinfo {year} {2001})}\BibitemShut {NoStop}%
\bibitem [{\citenamefont {Szmulowicz}\ \emph {et~al.}(2004)\citenamefont
  {Szmulowicz}, \citenamefont {Haugan},\ and\ \citenamefont {Brown}}]{A9}%
  \BibitemOpen
  \bibfield  {author} {\bibinfo {author} {\bibfnamefont {F.}~\bibnamefont
  {Szmulowicz}}, \bibinfo {author} {\bibfnamefont {H.}~\bibnamefont {Haugan}},
  \ and\ \bibinfo {author} {\bibfnamefont {G.~J.}\ \bibnamefont {Brown}},\
  }\href {\doibase 10.1103/PhysRevB.69.155321} {\bibfield  {journal} {\bibinfo
  {journal} {Phys. Rev. B}\ }\textbf {\bibinfo {volume} {69}},\ \bibinfo
  {pages} {155321} (\bibinfo {year} {2004})}\BibitemShut {NoStop}%
\bibitem [{\citenamefont {Li}\ \emph {et~al.}(2010)\citenamefont {Li},
  \citenamefont {Xu},\ and\ \citenamefont {Peeters}}]{A10}%
  \BibitemOpen
  \bibfield  {author} {\bibinfo {author} {\bibfnamefont {L.~L.}\ \bibnamefont
  {Li}}, \bibinfo {author} {\bibfnamefont {W.}~\bibnamefont {Xu}}, \ and\
  \bibinfo {author} {\bibfnamefont {F.~M.}\ \bibnamefont {Peeters}},\ }\href
  {\doibase 10.1103/PhysRevB.82.235422} {\bibfield  {journal} {\bibinfo
  {journal} {Phys. Rev. B}\ }\textbf {\bibinfo {volume} {82}},\ \bibinfo
  {pages} {235422} (\bibinfo {year} {2010})}\BibitemShut {NoStop}%
\bibitem [{\citenamefont {Lang}\ and\ \citenamefont {Xia}(2011)}]{A11}%
  \BibitemOpen
  \bibfield  {author} {\bibinfo {author} {\bibfnamefont {X.-L.}\ \bibnamefont
  {Lang}}\ and\ \bibinfo {author} {\bibfnamefont {J.-B.}\ \bibnamefont {Xia}},\
  }\href {\doibase 10.1088/0022-3727/44/42/425103} {\bibfield  {journal}
  {\bibinfo  {journal} {J. Phys. D: Appl. Phys.}\ }\textbf {\bibinfo {volume}
  {44}},\ \bibinfo {pages} {425103} (\bibinfo {year} {2011})}\BibitemShut
  {NoStop}%
\bibitem [{\citenamefont {Dong}\ \emph {et~al.}(2015)\citenamefont {Dong},
  \citenamefont {Li}, \citenamefont {Xu},\ and\ \citenamefont {Han}}]{A12}%
  \BibitemOpen
  \bibfield  {author} {\bibinfo {author} {\bibfnamefont {H.}~\bibnamefont
  {Dong}}, \bibinfo {author} {\bibfnamefont {L.}~\bibnamefont {Li}}, \bibinfo
  {author} {\bibfnamefont {W.}~\bibnamefont {Xu}}, \ and\ \bibinfo {author}
  {\bibfnamefont {K.}~\bibnamefont {Han}},\ }\href {\doibase
  https://doi.org/10.1016/j.tsf.2015.05.066} {\bibfield  {journal} {\bibinfo
  {journal} {Thin Solid Films}\ }\textbf {\bibinfo {volume} {589}},\ \bibinfo
  {pages} {388} (\bibinfo {year} {2015})}\BibitemShut {NoStop}%
\bibitem [{\citenamefont {Chen}\ \emph {et~al.}(2016)\citenamefont {Chen},
  \citenamefont {Xing}, \citenamefont {Zhu}, \citenamefont {Zha}, \citenamefont
  {Niu}, \citenamefont {Guo},\ and\ \citenamefont {Shao}}]{A13}%
  \BibitemOpen
  \bibfield  {author} {\bibinfo {author} {\bibfnamefont {X.}~\bibnamefont
  {Chen}}, \bibinfo {author} {\bibfnamefont {J.}~\bibnamefont {Xing}}, \bibinfo
  {author} {\bibfnamefont {L.}~\bibnamefont {Zhu}}, \bibinfo {author}
  {\bibfnamefont {F.-X.}\ \bibnamefont {Zha}}, \bibinfo {author} {\bibfnamefont
  {Z.}~\bibnamefont {Niu}}, \bibinfo {author} {\bibfnamefont {S.}~\bibnamefont
  {Guo}}, \ and\ \bibinfo {author} {\bibfnamefont {J.}~\bibnamefont {Shao}},\
  }\href {\doibase 10.1063/1.4948330} {\bibfield  {journal} {\bibinfo
  {journal} {J. Appl. Phys.}\ }\textbf {\bibinfo {volume} {119}},\ \bibinfo
  {pages} {175301} (\bibinfo {year} {2016})}\BibitemShut {NoStop}%
\bibitem [{\citenamefont {Winkler}(2003)}]{Wbook}%
  \BibitemOpen
  \bibfield  {author} {\bibinfo {author} {\bibfnamefont {R.}~\bibnamefont
  {Winkler}},\ }\href@noop {} {\bibfield  {journal} {\bibinfo  {journal}
  {\emph{Spin-Orbit Coupling Effects in Two-Dimensional Electron and Hole
  Systems}}\ } (\bibinfo {year} {Springer, New York, 2003})}\BibitemShut
  {NoStop}%
\bibitem [{\citenamefont {Durnev}(2020)}]{cst56b}%
  \BibitemOpen
  \bibfield  {author} {\bibinfo {author} {\bibfnamefont {M.~V.}\ \bibnamefont
  {Durnev}},\ }\href {\doibase 10.1134/S1063783420030087} {\bibfield  {journal}
  {\bibinfo  {journal} {Phys. Solid State}\ }\textbf {\bibinfo {volume} {62}},\
  \bibinfo {pages} {504} (\bibinfo {year} {2020})}\BibitemShut {NoStop}%
\bibitem [{\citenamefont {Witten}(1981)}]{cst58}%
  \BibitemOpen
  \bibfield  {author} {\bibinfo {author} {\bibfnamefont {E.}~\bibnamefont
  {Witten}},\ }\href {\doibase https://doi.org/10.1016/0550-3213(81)90006-7}
  {\bibfield  {journal} {\bibinfo  {journal} {Nucl. Phys. B}\ }\textbf
  {\bibinfo {volume} {188}},\ \bibinfo {pages} {513 } (\bibinfo {year}
  {1981})}\BibitemShut {NoStop}%
\bibitem [{SM()}]{SM}%
  \BibitemOpen
  \href@noop {} {\bibinfo  {journal} {See Supplemental Materials, which also
  contains Refs.~[65,66], for proving the uniqueness of the corner state energy
  $E_{0D}$ in Eq.~(21) for a sharp corner. Application of our analytical
  results in the context of the lattice model used in Ref.~[37] is also
  provided therein}\ }\BibitemShut {NoStop}%
\bibitem [{\citenamefont {A.~Gangopadhyaya}\ and\ \citenamefont
  {Rasinariu}(2011)}]{bookSM1q}%
  \BibitemOpen
\bibfield  {journal} {  }\bibfield  {author} {\bibinfo {author} {\bibfnamefont
  {J.~V.~M.}\ \bibnamefont {A.~Gangopadhyaya}}\ and\ \bibinfo {author}
  {\bibfnamefont {C.}~\bibnamefont {Rasinariu}},\ }\href@noop {} {\bibfield
  {journal} {\bibinfo  {journal} {\emph{Supersymmetric Quantum Mechanics: An
  Introduction}}\ } (\bibinfo {year} {World Scientific, Singapore,
  2011})}\BibitemShut {NoStop}%
\bibitem [{\citenamefont {F.~Cooper}\ and\ \citenamefont
  {Sukhatme}(2002)}]{bookSM2q}%
  \BibitemOpen
  \bibfield  {author} {\bibinfo {author} {\bibfnamefont {A.~K.}\ \bibnamefont
  {F.~Cooper}}\ and\ \bibinfo {author} {\bibfnamefont {U.}~\bibnamefont
  {Sukhatme}},\ }\href@noop {} {\bibfield  {journal} {\bibinfo  {journal}
  {\emph{Supersymmetry in Quantum Mechanics}}\ } (\bibinfo {year} {World
  Scientific, Singapore, 2002})}\BibitemShut {NoStop}%
\end{thebibliography}%

\newpage
\clearpage
\setcounter{equation}{0}
\setcounter{figure}{0}
\setcounter{table}{0}
\renewcommand{\thefigure}{S\arabic{figure}} %
\renewcommand{\thetable}{S\arabic{table}}   %
\renewcommand{\theequation}{S\arabic{equation}}   %

\onecolumngrid
\begin{center}
\LARGE{\textbf{Supplementary Materials}}
\end{center}
%\section*{Supplementary Materials for}
\maketitle
\onecolumngrid

\subsection{A. Uniqueness of the corner state energy for a sharp corner}
As shown in the main text, the problem of finding the energy of a corner state localized at the intersection of two edges reduces to solving the equation
\begin{equation}
\label{eq:B1}
\Bigg\{\hat{k}_x^2
+{W}(x)^2+\sigma{W}(x)'\Bigg\}\psi(x)=\varepsilon{\psi}(x),
\end{equation}
where ${W}(x<0)=W_1$ and ${W}({x}\geq0)=W_2$. Let us now demonstrate that with this choice of ${W}(x)$, the equation has only one localized solution at $\varepsilon=0$ if $W_{1}W_{2}<0$.

First, we note that Eq.~(\ref{eq:B1}) is the central equation in the supersymmetric (SUSY) formulation of quantum mechanics~\cite{cstSM1}. This equation has several exact analytical solutions found for specific forms of superpotential $W(x)$ during last decades~\cite{bookSM1}. Particularly, the one of solvable SUSY cases is represented by the superpotential in the form of \emph{Rosen-Morse II model}~\cite{bookSM2}:
\begin{equation}
\label{eq:B2}
{W}(x)=A\tanh\left(\dfrac{x}{L}\right)+\dfrac{B}{A},
\end{equation}
where $L$ is dimensional scale, $A$ and $B$ are real constants related by $|B|<A^2$. The latter guarantees ${W}(-\infty){W}(+\infty)<0$. As clear, if $L\rightarrow{0}$, $W(x)$ is reduced to the case considered in the main text with $W_{1,2}=\mp{A}+B/A$. For ${W}(x)$ in the form of Eq.~(\ref{eq:B2}), Eq.~(\ref{eq:B1}) has localized solutions with the energies
\begin{equation}
\label{eq:B3}
\varepsilon(n_\sigma)=A^2-\left(A-\dfrac{n_\sigma}{L}\right)^2+\dfrac{B^2}{A^2}-\dfrac{B^2}{\left(A-\dfrac{n_\sigma}{L}\right)^2},
\end{equation}
where $n_{-}/L=0,1,2,...,n_{max}/L<|A|$ for $\sigma=-1$ and $n_{+}/L=1,2,...,n_{max}/L<|A|$ for $\sigma=1$. Obviously, if $L\rightarrow{0}$, only a zero-energy solution with $n_{-}=0$ survives. The latter case corresponds to form of ${W}(x)$ considered in the main text.

\subsection{B. Corner states and Zeeman edge Hamiltonian from a lattice model used in Ref.~\cite{cstSM2}}
The analytical theory of the appearance of corner states, developed in the main text, requires knowledge of the edge g-factor tensor, which in turn reflects the crystal symmetry of 2D system. In this section, we emphasize the difference between the edge Hamiltionians and thus the corner states arising in zinc-blende semiconductor QWs and the ones within a lattice model~\cite{cstSM2}.
Let us first derive the low-energy edge Hamiltonian obtained through the lowest-order expansion with respect to $\mathbf{k}$ of 2D bulk Hamiltonian used for the QSHI on a square lattice. The wave-vector expansion of the lattice Hamiltonian used in Ref.~\cite{cstSM2} leads to the Hamiltonian similar to $H_{\mathrm{2D}}(\mathbf{k})$ in the main text:
\begin{equation}
\label{eq:A1}
{H_{\mathrm{2D}}}'(\mathbf{k})=C-\mathbb{D}k^2+\left(\mathbb{M}-\mathbb{B}k^2\right)\begin{pmatrix}
\sigma_z & 0 \\
0 & \sigma_z \end{pmatrix}+A\begin{pmatrix}
0 & (k_x-{i}k_y)\sigma_x \\
(k_x+{i}k_y)\sigma_x & 0\end{pmatrix},
\end{equation}
where we have added an additional term $-\mathbb{D}k^2$ describing the asymmetry between the conduction and valence bands, which is absent in the original paper~\cite{cstSM2}. Other parameters in terms of the notations used in Ref.~\cite{cstSM2} are written as $M=m-2t$, $\mathbb{D}^2=-t|\mathbf{d}_n|$, $A=2\lambda|\mathbf{d}_n|$. The Zeeman Hamiltonian describing the effect of in-plane magnetic field has the form~\cite{cstSM2}:
\begin{equation}
\label{eq:A2}
{H_{\mathrm{Z}}}'=\dfrac{\mu_Bg_0}{2}\begin{pmatrix}
0 & 0 & B_{-} & 0\\
0 & 0 & 0 & B_{-}\\
B_{+} & 0 & 0 & 0\\
0 & B_{+} & 0 & 0 \end{pmatrix},
\end{equation}
where $B_{\pm}=B_x{\pm}iB_y$ and $g_0$ is a constant representing the in-plane g-factor. Here, $x$ and $y$ axes are oriented along the main crystallographic axes of the square lattice. Importantly, ${H_{\mathrm{Z}}}'$ preserve the electron-hole symmetry.

In order to consider the edge of an arbitrary orientation, ${H_{\mathrm{2D}}}'(\mathbf{k})$ and ${H_{\mathrm{Z}}}'$ should be rewritten in a new coordinate system obtained by rotating clockwise along the $z$ axis by an angle $\theta$ relative to the original one (see Fig.~1 in the main text). To write the Hamiltonians in another coordinate system, one should transform the projections of electron momentum and magnetic field as:
\begin{equation}
\label{eq:A3}
\begin{pmatrix}
k_x \\
k_y
\end{pmatrix}=
\begin{pmatrix}
\cos{\theta} & -\sin{\theta} \\
\sin{\theta} & \cos{\theta}
\end{pmatrix}\begin{pmatrix}
\tilde{k}_x \\
\tilde{k}_y
\end{pmatrix},~~~~~~~~~~~~~~~~
\begin{pmatrix}
B_x \\
B_y
\end{pmatrix}=
\begin{pmatrix}
\cos{\theta} & -\sin{\theta} \\
\sin{\theta} & \cos{\theta}
\end{pmatrix}\begin{pmatrix}
\tilde{B}_x \\
\tilde{B}_y
\end{pmatrix},
\end{equation}
where ``tilde'' denotes projections in the new coordinate system. Simultaneously with the transitions $\left(\tilde{k}_x,\tilde{k}_y\right)\rightarrow\left(k_x,k_y\right)$ and $\left(\tilde{B}_x,\tilde{B}_y\right)\rightarrow\left(B_x,B_y\right)$, one should also
apply a unitary transformation to the Hamiltonians:
\begin{equation}
\label{eq:A4}
{H_{\mathrm{2D}}}'(\tilde{k}_x,\tilde{k}_y,\theta)=U'(\theta){H_{\mathrm{2D}}}'(k_x,k_y)U'(-\theta),~~~~~~~~~~~~~~~~
{H_{\mathrm{Z}}}'(\tilde{B}_x,\tilde{B}_y,\theta)=U'(\theta){H_{\mathrm{Z}}}'(B_x,B_y)U'(-\theta),
\end{equation}
where
\begin{equation}
\label{eq:A5}
U(\theta)=\begin{pmatrix}
e^{i\theta/2} & 0 & 0 & 0 \\
0 & e^{i\theta/2} & 0 & 0 \\
0 & 0 & e^{-i\theta/2} & 0 \\
0 & 0 & 0 & e^{-i\theta/2} \end{pmatrix}.
\end{equation}
In contrast to the case of zinc-blende QWs considered in the main text, both ${H_{\mathrm{2D}}}'(\tilde{\mathbf{k}})$ in Eq.~(\ref{eq:A1}) and ${H_{\mathrm{Z}}}'(\tilde{\mathbf{B}})$ in Eq.~(\ref{eq:A2}) remains invariant under the coordinate system rotation. Further, we omit the tilde marks keeping in mind that orientation of new $x$ and $y$ axis does not coincide with the crystallographic directions in the most general case.

Then, assuming the open-boundary conditions in a semi-infinite plane $y>0$, the edge wave functions for at zero-wave vector along the boundary are written in the form:
\begin{equation}
\label{eq:A6}
|+\rangle=\dfrac{1}{\sqrt{2|\mathbb{B}|}}\begin{pmatrix}
0 \\
\sqrt{\mathbb{B}+\mathbb{D}} \\
\sqrt{\mathbb{B}-\mathbb{D}} \\
0
\end{pmatrix}g(y),~~~~~~~
|-\rangle=\dfrac{1}{\sqrt{2|\mathbb{B}|}}\begin{pmatrix}
-\sqrt{\mathbb{B}-\mathbb{D}} \\
0 \\
0 \\
\sqrt{\mathbb{B}+\mathbb{D}}
\end{pmatrix}g(y)
\end{equation}
where $g(y)$ is a coordinate part of the wave-function describing the localization perpendicular to the edge.

By projecting the Zeeman Hamiltonian~(\ref{eq:A2}) onto the obtained basis wave functions in Eq.~(\ref{eq:A6}), one can verify that the edge g-factor tensor (see Eq.~(6) in the main text) is independent of $\theta$ and has the following non-zero components:
\begin{equation}
\label{eq:A7}
g_{xx}'=g_0\dfrac{\mathbb{D}}{\mathbb{B}},~~~~~~~g_{yy}'=g_0.
\end{equation}
Finally, assuming the magnetic field is oriented as shown in Fig.~1 in the main text, the low-energy edge Hamiltonian is written as
\begin{equation}
\label{eq:A8}
H'_{\mathrm{edge}}(k_x)=\varepsilon_0+A\dfrac{\sqrt{\mathbb{B}^2-\mathbb{D}^2}}{\left|\mathbb{B}\right|}k_x{s}_z+m_x{s}_{x}+m_y{s}_{y},
\end{equation}
where $\varepsilon_0=C-\mathbb{M}\mathbb{D}/\mathbb{B}$, and
\begin{equation}
\label{eq:A9}
m'_x=\dfrac{E'_{Z}}{2}\eta\cos(\theta-\varphi),~~~~~~~
m'_y=\dfrac{E'_{Z}}{2}\sin(\theta-\varphi)
\end{equation}
with $E'_{Z}=\mu_{B}{B}g_{0}/2$ and $\eta=\mathbb{D}/\mathbb{B}$. As seen from Eq.~(\ref{eq:A9}), at $\eta=0$, the edge states are gapless if the magnetic field is parallel to the edge of the 2D system. This reproduces the results of Ref.~\cite{cstSM2} obtained by numerical calculations of the local density of states within the square lattice model preserving electron-hole symmetry (see Fig.~2 therein). In the presence of electron-hole asymmetry, i.e. when $\eta\neq0$, there are no gapless edge states, since the mass parameter $M$ for the edge states
\begin{equation}
\label{eq:AM}
M=\sqrt{{m'}_x^2+{m'}_y^2}=\sqrt{1-\left(1-\eta^2\right)\cos^2\left(\theta-\varphi\right)}\geq\left|\eta\right|
\end{equation}
never vanishes.

Let us now apply our general analytical results on the corner states to the edge Hamiltonian $H'_{\mathrm{edge}}$ in Eqs.~(\ref{eq:A8}) and (\ref{eq:A9}). In this case, Eq.~(21) from the main text and the corner state existence condition are taking the form:
\begin{eqnarray}
\label{eq:A10}
E_{\mathrm{0D}}=\varepsilon_0+E_Z\dfrac{\sigma\eta\cos\left(\dfrac{\theta_1-\theta_2}{2}\right)}
{\sqrt{\cos^2\left(\varphi-\dfrac{\theta_1+\theta_2}{2}\right)+\eta^2\sin^2\left(\varphi-\dfrac{\theta_1+\theta_2}{2}\right)  }},~~~~~~~~~~~~~~~~~~~~~~~~~\nonumber\\
W(-\infty)W(+\infty)={E'_Z}^2\dfrac{\sin\left(\theta_1-\varphi\right)\cos\left(\varphi-\dfrac{\theta_1+\theta_2}{2}\right)+\eta^2\cos\left(\theta_1-\varphi\right)\sin\left(\varphi-\dfrac{\theta_1+\theta_2}{2}\right)}
{\cos^2\left(\varphi-\dfrac{\theta_1+\theta_2}{2}\right)+\eta^2\sin^2\left(\varphi-\dfrac{\theta_1+\theta_2}{2}\right)}~~~~~~~~~~\nonumber\\
\times\left[\sin\left(\theta_2-\varphi\right)\cos\left(\varphi-\dfrac{\theta_1+\theta_2}{2}\right)+\eta^2\cos\left(\theta_2-\varphi\right)\sin\left(\varphi-\dfrac{\theta_1+\theta_2}{2}\right)  \right]<0.
\end{eqnarray}

\begin{figure}
\includegraphics [width=0.75\columnwidth, keepaspectratio] {FigSM2.jpg} % Here is how to import EPS art
\caption{\label{Fig:SM2} Energy of corner states $E_{\mathrm{0D}}-\varepsilon_0$ (solid curves) for different orientations of the magnetic field and edges relative to the main cubic axes, calculated at different values of $\eta$. The dotted curves represented by $\pm\min\{M_{1},M_{2}\}$ correspond to the boundaries of 1D edge band states projected onto the corner. The corner state arises as soon as $W(-\infty)W(+\infty)<0$.
Note that $\theta_1-\theta_2\neq0$ and $\pm\pi$ for physically reasonable edges forming a common corner.}
\end{figure}

Figure~\ref{Fig:SM2} shows the evolution of corner state energy as a function of $\theta_1-\theta_2$ at different values of $\eta$, calculated for several orientations of the magnetic field and edges (cf. Fig.~3 from the main text). If the magnetic field is oriented along the edge of 2D system (see the left panel at Fig.~\ref{Fig:SM2}), a corner state exists only if the asymmetry parameter $\eta$ is nonzero. Otherwise (if $\eta=0$), the edge of 2D system parallel to the magnetic field has a gapless 1D edge spectrum, and there is no localized state in the corner. If the magnetic field is not parallel to each of the two edges, the corner state may appear even at $\eta=0$. Here, we would like to emphasize once again that the difference between Fig.~\ref{Fig:SM2} and Fig.~3 from the manin text is due to the difference in the Zeeman Hamiltonians used in the lattice model~\cite{cstSM2} and for zinc-blende semiconductor QWs.

Expressions~(\ref{eq:A10}) take very simple form if 2D system preserves electron-hole symmetry ($\eta$=0):
\begin{eqnarray}
\label{eq:A11}
E_{\mathrm{0D}}=\varepsilon_0=C,~~~~~~~~~~~~~~~~~~~~~~~~~~~~~\nonumber\\
W(-\infty)W(+\infty)={E'_Z}^2\sin\left(\theta_1-\varphi\right)\sin\left(\theta_2-\varphi\right)<0.
\end{eqnarray}
As expected, the corner state energy $E_{\mathrm{0D}}=C$, which can be set to zero without loss of generality. It is also seen that if the magnetic field is oriented along one of the meeting edges, there is indeed no corner state since $W(-\infty)W(+\infty)=0$.

\begin{figure*}
\includegraphics [width=1.0\columnwidth, keepaspectratio] {FigSM1.jpg} % Here is how to import EPS art
\caption{\label{Fig:SM1} The corner states of a hypothetical sample with the edges along main crystallographic directions as a function of the magnetic field orientation. The numbers near the corners correspond to the values $\left(\theta_1,\theta_2\right)$ that determine the orientations of the meeting edges in the notation used in this work. The direction of the magnetic field is determined by angle $\varphi$ as indicated by a brown arrow at the scheme. The positive values of the angles correspond to a clockwise rotation around $z$ axis. The localized state arises at each corner as soon as $W(-\infty)W(+\infty)<0$. The curves' colors at the panels are related with the color corners at the scheme. The calculations on the basis on the 1D edge Hamiltonian derived from the lattice model~\cite{cstSM2} with additional electron-hole asymmetry are presented in panels (a)--(d).}
\end{figure*}


For a more detailed comparison of our analytical results represented by Eqs.~(\ref{eq:A10}) with numerical calculations in the lattice model with $\eta=0$~\cite{cstSM2}, we consider a hypothetical sample, whose edges are oriented along the main cubic axes (see Fig.~\ref{Fig:SM1}). Note, however, that in our notations, the directions of the magnetic field considered in Ref.~\cite{cstSM2} correspond to $\varphi\in[-\pi/2,0]$ (cf. Fig.~2 in Ref.~\cite{cstSM2}). As seen from Fig.~\ref{Fig:SM1}(a), the corners A, C always have a localized state if the magnetic field is oriented in the range of $\varphi\in(-\pi/2,0)$, while the corners B and D are stateless. This fully reproduces the results of numerical calculations presented in Fig.~2(b1)--(f1) in Ref.~\cite{cstSM2}.

If there is an electron-hole asymmetry in 2D system (i.e. $\eta\neq{0}$) -- this case is beyond the scope of Ref.~\cite{cstSM2} -- the appearance of localized states in all four corners becomes possible. At first, such a situation arises at small values of $\eta$ for the directions of the magnetic field $\varphi=\pm\pi/4$, $\pm3\pi/4$. Note, however, that if $\varphi=-\pi/4$, $3\pi/4$, the states in the corners B and D are localized on a much larger scale than the states at the corners A and C. The localization scale is qualitatively determined by the quantity $W(-\infty)W(+\infty)$. As the parameter $\eta$ increases, four corner states at once are possible for wider angle $\varphi$ intervals, until eventually the presence of the states localized in the corners A, B, C and D ceases to depend on the direction of the magnetic field when $\eta$ exceeds some critical value $\eta_c$.

For comparison, Fig.~\ref{Fig:SM1} also shows $W(-\infty)W(+\infty)$ at all four corners calculated for zinc-blende semiconductor QWs at different ratio of $g_2/g_1$ (see the main text for its definitions).

\begin{thebibliography}{4}
\expandafter\ifx\csname natexlab\endcsname\relax\def\natexlab#1{#1}\fi
\expandafter\ifx\csname bibnamefont\endcsname\relax
  \def\bibnamefont#1{#1}\fi
\expandafter\ifx\csname bibfnamefont\endcsname\relax
  \def\bibfnamefont#1{#1}\fi
\expandafter\ifx\csname citenamefont\endcsname\relax
  \def\citenamefont#1{#1}\fi
\expandafter\ifx\csname url\endcsname\relax
  \def\url#1{\texttt{#1}}\fi
\expandafter\ifx\csname urlprefix\endcsname\relax\def\urlprefix{URL }\fi
\providecommand{\bibinfo}[2]{#2}
\providecommand{\eprint}[2][]{\url{#2}}

\bibitem[{\citenamefont{Witten}(1981)}]{cstSM1}
\bibinfo{author}{\bibfnamefont{E.}~\bibnamefont{Witten}},
  \bibinfo{journal}{Nucl. Phys. B} \textbf{\bibinfo{volume}{188}},
  \bibinfo{pages}{513 } (\bibinfo{year}{1981}).

\bibitem[{\citenamefont{A.~Gangopadhyaya and Rasinariu}(World Scientific,
  Singapore, 2011)}]{bookSM1}
\bibinfo{author}{\bibfnamefont{J.~V.~M.} \bibnamefont{A.~Gangopadhyaya}}
  \bibnamefont{and}
  \bibinfo{author}{\bibfnamefont{C.}~\bibnamefont{Rasinariu}},
  \bibinfo{journal}{\emph{Supersymmetric Quantum Mechanics: An Introduction}}
  (\bibinfo{year}{World Scientific, Singapore, 2011}).

\bibitem[{\citenamefont{F.~Cooper and Sukhatme}(World Scientific, Singapore,
  2002)}]{bookSM2}
\bibinfo{author}{\bibfnamefont{A.~K.} \bibnamefont{F.~Cooper}}
  \bibnamefont{and} \bibinfo{author}{\bibfnamefont{U.}~\bibnamefont{Sukhatme}},
  \bibinfo{journal}{\emph{Supersymmetry in Quantum Mechanics}}
  (\bibinfo{year}{World Scientific, Singapore, 2002}).

\bibitem[{\citenamefont{Ezawa}(2018)}]{cstSM2}
\bibinfo{author}{\bibfnamefont{M.}~\bibnamefont{Ezawa}},
  \bibinfo{journal}{Phys. Rev. Lett.} \textbf{\bibinfo{volume}{121}},
  \bibinfo{pages}{116801} (\bibinfo{year}{2018}).

\end{thebibliography}





\end{document}
%
% ****** End of file apssamp.tex ******
