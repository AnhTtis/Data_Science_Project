\documentclass[fancy]{article}
\usepackage{preprint}
\usepackage{bbm}
% \usepackage[style=authoryear,natbib=true]{biblatex}
% \addbibresource{ref.bib}
% \usepackage[latin1]{inputenc}
\usepackage[british]{babel}
\usepackage[all]{xy}
\usepackage{amscd}
\usepackage{amssymb}
\usepackage{amsthm}
\usepackage{enumitem}
\usepackage{mathrsfs,bbm}
\usepackage{xcolor,graphicx}
\usepackage{graphics}
\usepackage{soul}
\usepackage{comment}
\usepackage[all]{xy}
\usepackage{amscd}
\usepackage{amssymb,amsmath,latexsym}
\usepackage{amsthm}
\usepackage{enumitem}
\usepackage{mathrsfs,bbm}
\usepackage{dsfont}
\usepackage{tikz-cd}
\usepackage[T1]{fontenc}
\usepackage[utf8]{inputenc}  
 %
%%%%%%%%%%%%%%%%%%%%%%%%%%%%%%%%%%
%pagestyle
%%%%%%%%%%%%%%%%%%%%%%%%%%%%%%%%%%
%\pagestyle{plain}
\textwidth=430pt
\headsep=.7cm
\evensidemargin=15pt
\oddsidemargin=15pt
\leftmargin=0cm
\rightmargin=0cm
%%
%%%%%%%%%%%%%%%%%%%%%%%
\newcommand*\fixitem {\item[]%
  \refstepcounter{enumi}\hskip-\leftmargin\labelenumi\hskip\labelsep}
\newtheorem*{mainthm}{Main Theorem}
\newtheorem*{mainthm1}{Theorem}
\newtheorem*{maincor}{Corollary}
\usepackage[colorlinks=true]{hyperref}
\DeclareMathOperator{\Forall}{\forall}
\DeclareMathOperator{\Exists}{\exists}
\DeclareMathOperator{\ord}{ord}
\newcommand{\phiD}{\varphi_D}
\newcommand{\phiDI}{\varphi_{\mathbf{D}_I}}
\newcommand{\phiDIj}{\varphi_{\mathbf{D}_I (j)}}
\newcommand{\phiH}{\varphi_H}
\newcommand{\phiTimes}{\phiD \otimes \phiH}
\newcommand{\phiTimesDI}{\varphi_{\mathbf{D}_I} \otimes \phiH}
\newcommand{\R}{\mathscr{A}}
\newcommand{\X}{\mathscr{X}}
\newcommand{\Xf}{\mathscr{X}_{(k_0 ,i)}[r_0]}
\newcommand{\Xfr}{\mathscr{X}_{(k_0,i)}[r]}
\newcommand{\hotimes}{\widehat{\otimes}}
\newcommand{\C}{\mathbb{C}_p}
\newcommand{\V}{\mathscr{V}}
\newcommand{\B}{\mathscr{B}}
\newcommand{\dualD}{\mathfrak{D}}
\newcommand{\Dg}{\mathbf{D}}
\newcommand{\DD}{\mathcal{D}^0}
\newcommand{\DDg}{\mathcal{D}}
\newcommand{\DV}{\mathcal{D}}
\newcommand{\W}{\mathscr{W}_N}
\newcommand{\Ao}{\mathbf{A}^\circ}
\newcommand{\AoK}{\mathbf{A}^\circ_{\K}}
\newcommand{\AK}{\mathbf{A}_{/\K}}
\newcommand{\OOO}{\mathscr{A}^\circ}
\newcommand{\K}{\mathcal{K}} 
\newcommand{\OK}{\mathcal{O}_{\K}}
\newcommand{\varprojlog}[1]{\underleftarrow{\log\!^{#1}}}
\newcommand{\T}{\mathscr{T}}
\newcommand{\TT}{\mathbf{T}}
\newcommand{\VV}{\mathbf{V}}
\newcommand{\HH}{\mathcal{H}}
\newcommand{\hh}{\mathcal{H}^+}
\newcommand{\HG}[2]{\mathcal{H}_{#1}(#2)}
\newcommand{\hhl}{\mathcal{H}^{+,[l]}}
\newcommand{\hhj}{\mathcal{H}^{+,[j]}}
\newcommand{\hhjj}{\mathcal{H}^{+,[l,l']}}
\newcommand{\GS}{G_{\mathbb{Q},S}}
\newcommand{\Rf}{R_{(k_0 ,i)}[r_0]}
\newcommand{\Rfr}{R_{(k_0 ,i)}[r]}
\newcommand{\parT}{\langle T\rangle}
\newcommand{\Zf}{Z_{(k_0 ,i)}[r_0]}
\newcommand{\Zfr}{\mathscr{Z}_{(k_0 ,i)}[r]}
\newcommand{\ZFf}{\mathscr{Z}_{(k_0 ,i)}[r_0]}
\newcommand{\ZFfr}{\mathscr{Z}_{(k_0 ,i)}[r]}
\newcommand{\ZF}{\mathscr{Z}}
\begin{document}
\title{The log-plurisubharmonicity of fiberwise $\xi-$Bergman kernels for variant functional}
\author{
\name Shijie Bao \inst Institute of Mathematics, Academy of Mathematics
and Systems Science, Chinese Academy of Sciences, Beijing 100190, China \email bsjie@amss.ac.cn
\and
\name Qi'an Guan \inst School of                                  Mathematical Sciences, Peking University, Beijing 100871, China \email guanqian@math.pku.edu.cn
\and
\name Zheng Yuan \inst School of                                                                
Mathematical Sciences, Peking University, Beijing 100871, China \email zyuan@math.pku.edu.cn
}
\footnotetext{
\subjclass[2020]{32A36 32A70 32D15 32L05 32U05}
}
\footnotetext{
\keywords{fiberwise $\xi-$Bergman kernel, log-plurisubharmonicity, optimal $L^2$ extension, Guan-Zhou Method}
}

\maketitle 


\begin{abstract}
    In the present paper, we obtain the log-plurisubharmonicity of fiberwise $\xi-$Bergman kernels for variant functional.
    \end{abstract}
    
    
    \section{Introduction}\label{Introduction}
    The $\xi-$Bergman kernels were introduced in \cite{BG1} to give a new approach to the effectiveness result of the strong openness property of multiplier ideal sheaves. In \cite{BG1, BG2}, Bao-Guan established the log-plurisubharmonicity of fiberwise $\xi-$Bergman kernels for fixed functional $\xi$, which is a generalization of Berndtsson's log-plurisubharmonicity of fiberwise Bergman kernels (see \cite{Blogsub,Bern09}).

    By establishing log-plurisubharmonicity of some versions of generalized Bergman kernels, Bao-Guan also achieved some progress on strong openness properties related to modules at boundary points (see \cite{BGboundary1, BGboundary2, BGboundary3}), which gave new approaches to some results in \cite{BGY1, GMY1, GMY2}.

    Considering the asymptotic behaviors of $\xi-$Bergman kernels on sublevel sets of plurisubharmonic functions, Bao-Guan-Yuan introduced the $\xi-$complex singularity exponents in \cite{BGY22}. The relations among $\xi-$complex singularity exponents, jumping numbers, and complex singularity exponents were studied, and some properties of complex singularity exponents were also generalized to $\xi-$complex singularity exponents in \cite{BGY22}.

    The studies listed above concentrated on the fiberwise $\xi-$Bergman kernels with respect to invariant functional. Then it is natural to consider fiberwise $\xi-$Bergman kernels with respect to variant functional. In this paper, we give the log-plurisubharmonicity of fiberwise $\xi-$Bergman kernels for variant functional.

      We recall some definitions and notations. In \cite{BG1}, the following linear space of sequences of complex numbers was considered:
      \[\ell_1^{(n)}:=\left\{\xi=(\xi_{\alpha})_{\alpha\in\mathbb{N}^n} \ : \ \sum_{\alpha\in\mathbb{N}^n}|\xi_{\alpha}|\rho^{|\alpha|}<+\infty, \ \text{for \ any\ } \rho>0\right\},\]
      where any element in $\ell^{(n)}_1$ can be seen as a linear functional on the ring of holomorphic function germs. For any $z_0\in\mathbb{C}^n$, any $\xi=(\xi_{\alpha})\in\ell_1^{(n)}$, and any $F(z)=\sum_{\alpha\in\mathbb{N}^n}a_{\alpha}(z-z_0)^{\alpha}\in\mathcal{O}_{n,z_0}$, the value that $\xi$ acts on $F$ is defined as 
      \[(\xi\cdot F)(z_0):=\sum_{\alpha\in\mathbb{N}^n}\xi_{\alpha}\frac{F^{(\alpha)}(z_0)}{\alpha!}=\sum_{\alpha\in\mathbb{N}^n}\xi_{\alpha}a_{\alpha}.\]
      In \cite{BG1}, it is shown that $\ell_1^{(n)}$ is actually the dual space of $\mathcal{O}_{n,z_0}$ ($\mathcal{O}_n$ or $\mathcal{O}_{z_0}$ for short) under the analytic Krull topology (see \cite{Dembook}).
    
       Now we recall the definition of the $\xi-$Bergman kernel. Let $D$ be a domain in $\mathbb{C}^n$, and $\psi$ a plurisubharmonic function on $D$. For any $\xi\in\ell_1^{(n)}$, the (weighted) $\xi-$Bergman kernel with respect to $\xi$ is denoted by
      \begin{equation*}
          K^{\psi}_{\xi,D}(z):=\sup_{f\in A^2(D,e^{-\psi})}\frac{|(\xi\cdot f)(z)|^2}{\int_D|f|^2e^{-\psi}},
      \end{equation*}
      where $A^2(D,e^{-\psi}):=\{f\in\mathcal{O}(D) : \int_D|f|^2e^{-\psi}<+\infty\}$, and we denote $K^{\psi}_{\xi,D}(z)=0$ if $A^2(D,e^{-\psi})=\{0\}$.
    
      In \cite{BG2} (see also \cite{BG1}), the following log-plurisubharmonicity property was proved for fiberwise $\xi-$Bergman kernels, which is a generalization of Berndtsson's log-plurisubharmonicity property for fiberwise Bergman kernels (see \cite{Blogsub}). 
      
      Let $\Omega$ be a pseudoconvex domain in $\mathbb{C}^{n+1}$ with coordinate $(z,w)$, where $z\in\mathbb{C}^n$, $w\in\mathbb{C}$. Let $p$, $q$ be the natural projections $p(z,w)=w$, $q(z,w)=z$ on $\Omega$, $D:=p(\Omega)$. For any $w\in D$, assume $\Omega_w:=p^{-1}(w)\cap\Omega$ is a bounded domain in $\mathbb{C}^n$. Let $\psi$ be a plurisubharmonic function on $\Omega$, and let $K^{\psi}_{\xi,w}(z)=K^{\psi}_{\xi,\Omega_w}(z)$ be the Bergman kernels of the domains $\Omega_w$ with respect to some fixed $\xi\in \ell_1^{(n)}$.
    
     \begin{Theorem}[\cite{BG1, BG2}]\label{xi-log-psh}
        $\log K^{\psi}_{\xi,w}(z)$ is a plurisubharmonic function with respect to $(z,w)$, for any $\xi\in \ell_1^{(n)}$.
     \end{Theorem}
     
    In the present paper, we consider the log-plurisubharmonicity of the above fiberwise $\xi-$Bergman kernels with respect to variant functional. We generalize Theorem \ref{xi-log-psh} to the following result.

    Again let $\Omega$ be a pseudoconvex domain in $\mathbb{C}^{n+1}$ with coordinate $(z,w)$, where $z\in\mathbb{C}^n$, $w\in\mathbb{C}$. Let $p$, $q$ be the natural projections $p(z,w)=w$, $q(z,w)=z$ on $\Omega$, $D:=p(\Omega)$. For any $w\in D$, assume $\Omega_w:=p^{-1}(w)\cap\Omega$ is a bounded domain in $\mathbb{C}^n$. Let $\xi(w)$ be a functional valued function on $D$, where $\xi(w)=(\xi_{\alpha}(w))_{\alpha\in \mathbb{N}^n}$ and $\xi_{\alpha}(w)\in \ell_1^{(n)}$ for any $w\in D$. We call that $\xi(w)$ is \emph{holomorphic} with respect to $w$ if $\xi_{\alpha}(w)$ is holomorphic with respect to $w\in D$ for any $\alpha\in\mathbb{N}^n$. And we call that $\xi(\cdot)$ satisfies the \emph{locally uniformly bounded property} on $D$, if for any $w\in D$, and any $\rho>0$, there exists a neighborhood $V_{w,\rho}$ of $w$ such that
    \[\sup_{w'\in V_{w,\rho}}\sum_{\alpha\in\mathbb{N}^n}|\xi_{\alpha}(w')|\rho^{|\alpha|}<+\infty\] 
    holds. 
    \begin{Remark}
    Suppose the functional valued $\xi(\cdot)$ satisfies the locally uniformly bounded property, then for any compact subset $K$ of $D$, one have
    \[\sup_{w\in K}\sum_{\alpha\in\mathbb{N}^n}|\xi_{\alpha}(w)|\rho^{|\alpha|}<+\infty\]
    for any $\rho>0$. Consequently, the above neighborhood $V_{w,\rho}$ in the definition of ``locally uniformly bounded property" can be selected as independent on $\rho$.
    \end{Remark}
    
    Let $\psi$ be a plurisubharmonic function on $\Omega$, and let $\xi(\cdot)$ be a functional valued function on $D$ such that $\xi(\cdot)$ is holomorphic with respect to $w$ and $\xi(\cdot)$ satisfies the locally uniformly bounded property on $D$. Denote $\psi_w:=\psi|_{\Omega_w}$ for any $w\in D$. Let $K^{\psi_w}_{\xi(w),\Omega_w}(z)$ be the Bergman kernels of the domains $\Omega_w$ with respect to the variant $\xi(w)$.

    \begin{Theorem}\label{log-psh-wrt-xi}
        $\log K^{\psi_w}_{\xi(w),w}(z)$ is a plurisubharmonic function with respect to $(z,w)$.
    \end{Theorem}

    This is a generalization of Theorem \ref{xi-log-psh}. 













    \section{Preparations}
    We recall some basic lemmas in \cite{BG1, BG2}.

    \begin{Lemma}[\cite{BG1}]\label{lem-holo-wrt-z}
        Let $D$ be a bounded domain in $\mathbb{C}^n$, and let $F$ be a holomorphic function on $D$. Then $(\xi\cdot F)(z)$ is also a holomorphic function on $D$, where $\xi\in \ell_1^{(n)}$.
    \end{Lemma}

    \begin{Lemma}[\cite{BG2}]\label{lem-bounded}
        Let $D$ be a bounded domain in $\mathbb{C}^n$, $\xi\in\ell_1^{(n)}$, and $\psi$ a plurisubharmonic function on $D$. Then for any compact subset $K$ of $D$, there exists a positive constant $C_K$, such that
        \[|(\xi\cdot f)(z)|\leq \int_{D}|f|^2e^{-\psi},\]
        for any $z\in D$, and any $f\in A^2(D,e^{-\psi}).$
    \end{Lemma}
    
    \begin{Lemma}[\cite{BG2}]\label{lem-F0}
        Let $D$ be a bounded domain in $\mathbb{C}^n$, $z\in D$, and $\psi$ a plurisubharmonic function on $D$ with $A^2(D,e^{-\psi})\neq\{0\}$. Then for any $\xi\in \ell_1^{(n)}$, there exists a holomorphic function $F_0$ on $D$ such that
        \[K^{\psi}_{\xi,D}(z)=\frac{|(\xi\cdot F_0)(z)|^2}{\int_D |F_0|^2e^{-\psi}}.\]
    \end{Lemma}

    \begin{Lemma}[\cite{BG2}]\label{lem-log-psh-wrt-z}
        Let $D$ be a bounded domain in $\mathbb{C}^n$, $\psi$ a plurisubharmonic function on $D$, and let $\xi\in \ell_1^{(n)}$. Then $\log K^{\psi}_{\xi,D}(z)$ is plurisubharmonic on $D$.
    \end{Lemma}

    We need the following lemmas in the present paper. In Lemma \ref{lem-conti} and Lemma \ref{lem-hol}, we assume that $\Omega$ is a domain in $\mathbb{C}^{n+1}$, and $D:=p(\Omega)$ is a domain in $\mathbb{C}$, such that $\Omega_w=p^{-1}(w)\cap\Omega$ is a bounded domain in $\mathbb{C}^n$ for any $w\in D$, where $p$ is the natural projection from $\mathbb{C}^{n+1}$ to $\mathbb{C}$. Let $\xi(\cdot)$ be a functional valued function on $D$, such that $\xi(w)\in\ell_1^{(n)}$ for any $w\in D$.
    
    \begin{Lemma}\label{lem-conti}
        Suppose the functional valued function $\xi(\cdot)$ satisfies the locally uniformly bounded property near some $w_0\in D$, and $\xi_{\alpha}(\cdot)$ is continuous near $w_0$ for any $\alpha\in\mathbb{N}^n$. Let $(z_0,w_0)\in \Omega$, and $V\subset\subset\Omega_{w_0}$ an open subset such that $z_0\in V$. Let $\{(z_j,w_j)\}\subset \Omega$ be a sequence such that $z_j\in V$ for any $j\in\mathbb{N}_+$, and $(z_j,w_j)\to (z_0,w_0)$ as $j\to +\infty$. Then for any $\{f_j\}$ a sequence of holomorphic functions on $V$ satisfying that $f_j$ compactly converges to $f\in \mathcal{O}(V)$ on $V$, we have
        \[\lim_{j\to +\infty} (\xi(w_j)\cdot f_j)(z_j)=(\xi(w_0)\cdot f)(z_0). \]
    \end{Lemma}

    \begin{proof}
        Let $W\subset\subset V$ be an open subset such that $z_j\in W$ for any $j\in\mathbb{N}$. Since $f_j$ is compactly convergent to $f$ on $V$, according to Cauchy's inequality, there exists some $M>0$ and $R>0$, such that
        \[\frac{|f_j^{(\alpha)}(w)|}{\alpha!}\leq \frac{M}{R^{|\alpha|}},\ \forall w\in W, j\in\mathbb{N}, \alpha\in\mathbb{N}^n.\]
        As $\xi(\cdot)$ is locally uniformly bounded, we have
        \[\sup_{j\in\mathbb{N}}\sum_{\alpha\in\mathbb{N}^n}|\xi_{\alpha}(w_j)|\rho^{|\alpha|}<+\infty\]
        for some fixed $\rho>1/R$. Then for any given $\epsilon>0$, there exists $k\in\mathbb{N}$ such that 
        \begin{equation}\label{sum|alpha|>k}
            \begin{split}
                \sup_{j\in\mathbb{N}}\sum_{|\alpha|>k}|\xi_{\alpha}(w_j)|\frac{|f_j^{(\alpha)}(z_j)|}{\alpha!}&\leq \sup_{j\in\mathbb{N}}\sum_{|\alpha|>k}|\xi_{\alpha}(w_j)|\frac{M}{R^{|\alpha|}}\\
                &\leq \frac{1}{(\rho R)^k}\sup_{j\in\mathbb{N}}\sum_{|\alpha|>k}|\xi_{\alpha}(w_j)|\rho^{|\alpha|}\\
                &<\epsilon.
            \end{split}
        \end{equation}
    Note that $f_j$ compactly converges to $f$  implies that 
    \[\lim_{j\to +\infty}f^{(\alpha)}_j(z_j)= f^{(\alpha)}(z_0), \ \forall \alpha\in\mathbb{N}^n,\]
     then it follows from $\xi_{\alpha}(\cdot)$ is continuous near $w_0$ that
    \[\lim_{j\to +\infty}\sum_{|\alpha|\leq k}\xi_{\alpha}(w_j)\frac{f_j^{(\alpha)}(z_j)}{\alpha!}=\sum_{|\alpha|\leq k}\xi_{\alpha}(w_0)\frac{f^{(\alpha)}(z_0)}{\alpha!}.\]
    Combining with (\ref{sum|alpha|>k}), we can finally get
    \begin{equation*}
        \begin{split}
            \lim_{j\to +\infty} (\xi(w_j)\cdot f_j)(z_j)&=\lim_{j\to +\infty}\sum_{\alpha\in\mathbb{N}^n}\xi_{\alpha}(w_j)\frac{f_j^{(\alpha)}(z_j)}{\alpha!}\\
            &=\sum_{\alpha\in\mathbb{N}^n}\xi_{\alpha}(w_0)\frac{f^{(\alpha)}(z_0)}{\alpha!}\\
            &=(\xi(w_0)\cdot f)(z_0).
        \end{split}
    \end{equation*}


    \end{proof}

    \begin{Lemma}\label{lem-hol}
        Suppose the functional valued function $\xi(\cdot)$ satisfies the holomorphic and locally uniformly bounded properties near some $w_0\in D$. Let $z_0\in \Omega_{w_0}$. Then for any holomorphic function $F(z,w)$ near $(z_0,w_0)$, $(\xi(w)\cdot F_w)(z)$ is holomorphic with respect to $(z,w)$ near $(z_0,w_0)$, where $F_w(\cdot):=F(\cdot, w)=F|_{\Omega_w}$.
    \end{Lemma}
    
\begin{proof}
    It is known that $(\xi(w)\cdot F_w)(z)$ is holomorphic with respect to $z$ for any fixed $w$ by Lemma \ref{lem-holo-wrt-z}. Now we prove that $(\xi(w)\cdot F_w)(z)$ is holomorphic with respect to $w$ for any fixed $z$. Note that
    \[(\xi(w)\cdot F_w)(z)=\sum_{\alpha\in\mathbb{N}^n}\xi_{\alpha}(w)c_{\alpha}(w),\]
    where
    \[c_{\alpha}(w):=\frac{1}{\alpha!}\cdot\frac{\partial ^{\alpha} F(z,w)}{\partial z^{\alpha}}\]
    is holomorphic with respect to $w$. $\xi_{\alpha}(w)$ is also holomorphic with respect to $w$ since $\xi(\cdot)$ is holomorphic. 
    
    Next we prove the summation $\sum_{\alpha\in\mathbb{N}^n}\xi_{\alpha}(w)c_{\alpha}(w)$ is uniformly convergent near $w_0$. Let $V\subset\subset q(\Omega_{w_0})$ be an open subset such that $z\in V$, and we can find some $r>0$ such that $V\subset q(\Omega_w)$ for any $w\in\Delta(w_0,r)$. Then according to Cauchy's inequality we can find some $R>0$ and $M>0$ such that
    \[|c_{\alpha}(w)|\leq\frac{M}{R^{|\alpha|}}, \ \forall w\in\Delta(w_0,r), \alpha\in\mathbb{N}^n.\]
    Since $\xi(\cdot)$ satisfies the locally uniformly bounded property, we can choose $r$ smaller if necessarily, such that
    \[\sup_{w\in\Delta(w_0,r)}\sum_{\alpha\in\mathbb{N}^n}|\xi_{\alpha}(w)|\rho^{|\alpha|}<+\infty\]
    for some fixed $\rho>1/R$. It follows that there exists some $M'>0$ such that
    \[|\xi_{\alpha}(w)|\leq \frac{M'}{\rho^{|\alpha|}}, \ \forall w\in\Delta(w_0,r), \alpha\in\mathbb{N}^n.\]
    Thus 
    \[|\xi_{\alpha}(w)c_{\alpha}(w)|\leq \frac{MM'}{(\rho R)^{|\alpha|}}, \ \forall w\in\Delta(w_0,r), \alpha\in\mathbb{N}^n.\]
    Since $\rho R>1$, we have $\sum_{\alpha\in\mathbb{N}^n}\frac{MM'}{(\rho R)^{|\alpha|}}<+\infty$. Then we obtain that the summation $\sum_{\alpha\in\mathbb{N}^n}\xi_{\alpha}(w)c_{\alpha}(w)$ is uniformly convergent on $\Delta(w_0,r)$, and it follows that $(\xi(w)\cdot F_w)(z)=\sum_{\alpha\in\mathbb{N}^n}\xi_{\alpha}(w)c_{\alpha}(w)$ is holomorphic with respect to $w$ since both $\xi_{\alpha}(w)$ and $c_{\alpha}(w)$ are holomorphic with respect to $w$.

    Now according to Hartogs' Theorem, we get that $(\xi(w)\cdot F_w)(z)$ is holomorphic with respect to $(z,w)$ near $(z_0,w_0)$.
\end{proof}

The following proposition shows that the assumptions for the functional valued function in Lemma \ref{lem-hol} are necessary.

    \begin{Proposition}
        Let $D$ be a domain in $\mathbb{C}$ with coordinate $w$, and $\xi(\cdot)$ be a functional valued function on $D$ such that $\xi(w)\in\ell^{(n)}_1$ for any $w\in D$. If for any bounded domain $\Omega'\subset\mathbb{C}^n$, any $z_0\in \Omega'$, and any $F\in A^2(\Omega')$, we have that $(\xi(w)\cdot F)(z_0)$ is holomorphic with respect to $w\in D$, then the functional valued function $\xi(\cdot)$ is holomorphic and satisfies the locally uniformly bounded property on $D$. 
    \end{Proposition}

    \begin{proof}
    Let $R>0$ be fixed, and $\Omega'=\Delta^n_R\subset\mathbb{C}^n$ be the polydisc centered at $o$ with radius $R$. Let $z=(z_1,\cdots,z_n)$ be the coordinate on $\Omega'$. Then $z^{\alpha}:=z_1^{\alpha_1}\cdots z_n^{\alpha_n}\in A^2(\Omega')$ for any $\alpha=(\alpha_1,\ldots,\alpha_n)\in\mathbb{N}^n$. It follows the assumption for $\xi(\cdot)$ that $\xi_{\alpha}(w)=(\xi(w)\cdot z^{\alpha})(o)$ is holomorphic with respect to $w\in D$ for any $\alpha\in\mathbb{N}^n$. Then we get that $\xi(w)$ is holomorphic with respect to $w\in D$.

    Next, for any $w_0\in D$, note that $|(\xi(w)\cdot F)(o)|$ is locally bounded near $w_0$ since $(\xi(w)\cdot F)(o)$ is holomorphic near $w_0$. Then we can find some neighborhood $V_{w_0}$ of $w_0$, such that the family $\{\xi(w)\}_{w\in V_{w_0}}$ of linear continuous (Lemma \ref{lem-bounded}) functionals over $A^2(\Omega')$ satisfies
    \[\sup_{w\in V_{w_0}}|(\xi(w)\cdot F)(o)|<+\infty, \ \forall F\in A^2(\Omega').\]
    Now the uniformly bounded principle (see e.g. \cite{FA-Rudin}) implies that for any bounded subset $U$ of $A^2(\Omega')$, there exists a constant $M_U>0$, such that
    \begin{equation}\label{UBP}
        \sup_{h\in U}|\xi(w)\cdot h(o)|\leq M_U, \ \forall w\in V_{w_0}.
    \end{equation}
    We choose the bounded subset $U$ of $A^2(\Omega')=A^2(\Delta^n_R)$ as follows:
    \[U:=\left\{h=\sum_{\alpha\in\mathbb{N}^n}a_{\alpha}z^{\alpha}\in\mathcal{O}(\Delta^n_R) : |a_{\alpha}|=\frac{1}{(2R)^{|\alpha|}},\ \forall \alpha\in\mathbb{N}^n\right\}.\]
    Since for any $h=\sum_{\alpha\in\mathbb{N}^n}a_{\alpha}z^{\alpha}$ with $|a_{\alpha}|=1/(2R)^{|\alpha|}$, $\forall \alpha\in\mathbb{N}^n$, we have
    \begin{align*}
    \begin{split}
    \int_{\Delta^n_R}|h|^2&=\sum_{\alpha\in\mathbb{N}^n}\frac{\pi^n|a_{\alpha}|^2}{(\alpha_1+1)\cdots(\alpha_n+1)}R^{2(|\alpha|+n)}\\
    &= \sum_{\alpha\in\mathbb{N}^n}\frac{\pi^nR^{2n}}{(\alpha_1+1)\cdots(\alpha_n+1)}\left(\frac{1}{2}\right)^{|\alpha|}<+\infty,
        \end{split}
    \end{align*}
    which yields that $U$ is actually a bounded subset of $A^2(\Omega')$. Now (\ref{UBP}) gives that
    \[\sup_{|a_{\alpha}|=\frac{1}{(2R)^{|\alpha|}} ( \forall\alpha\in\mathbb{N}^n)}\left|\sum_{\alpha\in\mathbb{N}^n}\xi_{\alpha}(w)a_{\alpha}\right|\leq M,\ \forall w\in V_{w_0},\]
    for some constant $M>0$ independent on $w$. It follows that
    \[\sup_{w\in V_{w_0}}\sum_{\alpha\in\mathbb{N}^n}|\xi_{\alpha}(w)|\frac{1}{(2R)^{|\alpha|}}<+\infty.\]
    Since $R$ and $w_0\in D$ are arbitrary, we conclude that $\xi(\cdot)$ satisfies the locally uniformly bounded property on $D$.
    \end{proof}









    
    \section{Proof of Theorem \ref{log-psh-wrt-xi}}

    For the proof of Theorem \ref{log-psh-wrt-xi}, we also need to recall the following version of the optimal $L^2$ extension theorem.

    \begin{Lemma}[Optimal $L^2$ extension theorem, \cite{Bl13}]\label{optimal-L2-ext}
        Let $\Omega$ be a pseudoconvex domain in $\mathbb{C}^{n+1}$ with coordinate $(z,w)$, where $z\in\mathbb{C}^n$, $w\in\mathbb{C}$. Let $p$ be the natural projection $p(z,w)=w$ on $\Omega$. Let $D:=p(\Omega)$ and assume that $D=\Delta_{w_0,r}$ is the disk in the complex plane centered on $t_0$ with radius $r$. For any $w\in D$, denote $\Omega_w:=p^{-1}(w)\subseteq\Omega$. Let $\psi$ be a plurisubharmonic function on $\Omega$.
        Then for any $f$ in $A^2(\Omega_{w_0},e^{-\psi})$, there exists a holomorphic function $F$ on $\Omega$, such that $F|_{\Omega_{w_0}}=f$, and
        \[\frac{1}{\pi r^2}\int_{\Omega}|F|^2e^{-\psi}\leq \int_{\Omega_{w_0}}|f|^2e^{-\psi}.\]
    \end{Lemma}

    Now we prove Theorem \ref{log-psh-wrt-xi} by Guan-Zhou Method (see \cite{GZ-L2ext,Oh-book1,Oh-book2}).

    \begin{proof}[Proof of Theorem \ref{log-psh-wrt-xi}.]
    We firstly prove $\log K^{\psi_w}_{\xi(w),w}(z)$ is upper-semicontinuous. Let $(z_0,w_0)\in\Omega$. Let $\{(z_j,w_j)\}\subset\Omega$ be a sequence such that $(z_j,w_j)\rightarrow (z_0,w_0)$ as $j\to +\infty$, and
    \[K^{\psi_{w_j}}_{\xi(w_j),w_j}(z_j)\to \limsup_{\Omega\ni(z,w)\to (z_0,w_0)}K^{\psi_w}_{\xi(w),w}(z), \ j\to+\infty.\]
    Without loss of generality, we assume $\log K^{\psi_{w_j}}_{\xi(w_j),w_j}(z_j)>-\infty$ for any $j$. According to Lemma \ref{lem-F0}, for any $j\in\mathbb{N}_+$, there exists some $f_j\in A^2(q(\Omega_{w_j}), e^{-\psi_{w_j}})$ such that
    \[|(\xi(w_j)\cdot f_j)(z_j)|^2=K^{\psi_{w_j}}_{\xi(w_j),w_j}(z_j), \ \text{and\ }\int_{q(\Omega_{w_j})}|f_j|^2e^{-\psi_{w_j}}=1.\]
    Using Montel's Theorem, we can extract a subsequence of $(z_j,w_j)$ (denoted by $(z_{k_j},w_{k_j})$), such that $f_{k_j}$ uniformly converges to some $f_0\in\mathcal{O}(q(\Omega_{w_0}))$ on any compact subset of $q(\Omega_{w_0})$. Then Lemma \ref{lem-conti} gives
    \[(\xi(w_0)\cdot f_0)(z_0)=\lim_{j\to+\infty}(\xi(w_j)\cdot f_j)(z_j)=\limsup_{\Omega\ni(z,w)\to (z_0,w_0)}K^{\psi_w}_{\xi(w),w}(z).\]
    In addition, it follows from Fatou's Lemma and the upper-semicontinuity of $\psi$ that
    \[\int_{q(\Omega_{w_0})}|f_0|^2e^{-\psi_{w_0}}\leq\liminf_{j\to+\infty}\int_{q(\Omega_{w_j})}|f_j|^2e^{-\psi_{w_j}}=1.\]
    Thus we get
    \[K^{\psi_{w_0}}_{\xi(w_0),w_0}(z_0)\geq\frac{|(\xi(w_0)\cdot f_0)(z_0)|^2}{\int_{q(\Omega_{w_0})}|f_0|^2e^{-\psi_{w_0}}}\geq \limsup_{\Omega\ni(z,w)\to (z_0,w_0)}K^{\psi_w}_{\xi(w),w}(z).\]
    It means that $K^{\psi_w}_{\xi(w),w}(z)$ is upper-semicontinuous, which gives that $\log K^{\psi_w}_{\xi(w),w}(z)$ is also upper-semicontinuous.

    Then we need to prove that $\log K^{\psi_w}_{\xi(w),w}(z)$ satisfies the sub-mean value inequality on any complex line $L\subset \Omega$. If $L$ lies on $q(\Omega_{w})$ for some $w\in D$, Lemma \ref{lem-log-psh-wrt-z} shows it is true. Then without loss of generality, we may assume $L$ lies on $\Omega$, and $L=\Delta(w_0,r)$ for some $w_0\in D$ and $r>0$.

    According to Lemma \ref{lem-F0}, there exists some $f\in A^2(\Omega_{w_0},e^{-\psi_{w_0}})$ such that
    \begin{equation}\label{eq-f}
        K^{\psi_{w_0}}_{\xi(w_0),w_0}(z_0)=\frac{|(\xi(w_0)\cdot f)(z_0)|^2}{\int_{\Omega_{w_0}}|f|^2e^{-\psi_{w_0}}}.
    \end{equation}
    Next, according to Lemma \ref{optimal-L2-ext}, we can get a holomorphic function $F$ on $p^{-1}(\Delta(w_0,r))$, such that $F|_{\Omega_{w_0}}=f$, and\
    \[\frac{1}{\pi r^2}\int_{p^{-1}(\Delta(w_0,r))}|F|^2e^{-\psi}\leq \int_{\Omega_{w_0}}|f|^2e^{-\psi_{w_0}}.\]
    Now it follows from Jensen's inequality and Fubini's Theorem that
    \begin{equation}\label{eq-Jensen}
        \begin{split}
            \log\left(\int_{\Omega_{w_0}}|f|^2e^{-\psi_{w_0}}\right)&\geq \log\left(\frac{1}{\pi r^2}\int_{p^{-1}(\Delta(w_0,r))}|F|^2e^{-\psi}\right)\\
            &=\log \left(\frac{1}{\pi r^2}\int_{w\in\Delta(w_0,r)}\int_{\Omega_w}|F_w|^2e^{-\psi_w}\right)\\
            &\geq\frac{1}{\pi r^2}\int_{w\in\Delta(w_0,r)}\log\left(\int_{\Omega_w}|F_w|^2e^{-\psi_w}\right)\\
            &\geq\frac{1}{\pi r^2}\int_{w\in\Delta(w_0,r)}\bigg(\log|(\xi(w)\cdot F_w)(z)|^2-\log K^{\psi_w}_{\xi(w),w}(z)\bigg),\\
        \end{split}
    \end{equation}
    where $F_w(\cdot):=F(\cdot,w)=F|_{\Omega_w}$ for any $w\in\Delta(w_0,r)$. Lemma \ref{lem-hol} tells us that $(\xi(w)\cdot F_w)(z)$ is holomorphic with respect to $w$, which implies that $\log|(\xi(w)\cdot F_w)(z)|^2$ is subharmonic with respect to $w$. This gives
    \begin{equation}\label{eq-subharmonic}
        \log|(\xi(w_0)\cdot F_{w_0})(z_0)|^2\leq \frac{1}{\pi r^2}\int_{w\in\Delta(w_0,r)}\log|(\xi(w)\cdot F_w)(z)|^2.
    \end{equation}
    Now combining (\ref{eq-f}), (\ref{eq-Jensen}), and (\ref{eq-subharmonic}), we obtain
    \[\log K^{\psi_w}_{\xi(w),w}(z)\leq \frac{1}{\pi r^2}\int_{w\in\Delta(w_0,r)}\log K^{\psi_w}_{\xi(w),w}(z),\]
    which is actually the sub-mean value inequality of $\log K^{\psi_w}_{\xi(w),w}(z)$ along $L=\Delta(w_0,r)$.

    Now we can conclude that $\log K^{\psi_w}_{\xi(w),w}(z)$ is plurisubharmonic on $\Omega$ with respect to $(z,w)$.
    \end{proof}



% \nocite{*}
% \printbibliography[title=Main References, type=book]
\vspace{.1in} {\em Acknowledgements}. We are grateful to Professor Bo Berndtsson for bringing this problem to our attention and giving us many helpful discussions. We would also like to thank Dr. Zhitong Mi for carefully checking this note. The second named author was supported by National Key R\&D Program of China 2021YFA1003100, NSFC-11825101, NSFC-11522101 and NSFC-11431013.



\bibliographystyle{alpha}
\bibliography{xbib}

\begin{thebibliography}{100}

\bibitem{BG1}
S.J. Bao and Q.A. Guan. $L^2$ extension and effectiveness of strong openness property, Acta Mathematica Sinica, English Series, 2022, 38(11): 1949-1964.

\bibitem{BG2}
S.J. Bao and Q.A. Guan. $L^2$ extension and effectiveness of $L^p$ strong openness property, https://www.researchgate.net/publication/353802713, to appear in Acta Mathematica Sinica, English Series.

\bibitem{BGboundary1}
S.J. Bao and Q.A. Guan. Modules at boundary points, fiberwise Bergman kernels, and log-subharmonicity, preprint, arXiv:2204.01413.

\bibitem{BGboundary2}
S.J. Bao and Q.A. Guan. Modules at boundary points, fiberwise Bergman kernels, and log-subharmonicity \uppercase\expandafter{\romannumeral2} -- on Stein manifolds, preprint, arXiv:2205.08044.

\bibitem{BGboundary3}
S.J. Bao and Q.A. Guan. Fiberwise Bergman kernels, vector bundles, and log-subharmonicity, preprint, arXiv:2210.1660.

\bibitem{BGY1}
S.J. Bao, Q.A. Guan, and Z. Yuan. Boundary points, minimal $L^2$ integrals and concavity property, preprint, arXiv:2203.01648.v2.

\bibitem{BGY22}
S.J. Bao, Q.A. Guan, and Z. Yuan. A note on $\xi-$Bergman kernels, preprint, arXiv:2212.01561.

\bibitem{Blogsub}
B. Berndtsson. Subharmonicity properties of the Bergman kernel and some other
functions associated to pseudoconvex domains, Ann. Inst. Fourier (Grenoble), 56(6), 1633--1662 (2006). 

\bibitem{Bern09}
B. Berndtsson. Curvature of vector bundles associated to holomorphic fibrations, Ann. of Math. 169 (2009), pp 531-560.

\bibitem{Bl13}
Z. B\l ocki. Suita conjecture and the Ohsawa-Takegoshi extension theorem, Invent. Math. 193 (2013), 149-158.

\bibitem{Dembook}
J.-P Demailly. Complex analytic and differential geometry, electronically accessible at https://www-fourier.ujf-grenoble.fr/\textasciitilde demailly/manuscripts/agbook.pdf.

\bibitem{GMY1}
Q.A. Guan, Z.T. Mi, and Z. Yuan. Boundary points, minimal $L^2$ integrals and concavity property \uppercase\expandafter{\romannumeral2}: on weakly pseudoconvex K\"{a}hler manifolds, preprint, arXiv:2203.07723.v2.

\bibitem{GMY2}
Q.A. Guan, Z.T. Mi, and Z. Yuan. Boundary points, minimal $L^2$ integrals and concavity property \uppercase\expandafter{\romannumeral5} -- vector bundles, preprint, arXiv:2206.00443.

\bibitem{GZ-L2ext}
Q.A. Guan and X.Y. Zhou. A solution of an $L^2$ extension problem with an optimal estimate and applications. Ann. of Math. (2), 181(3), 1139--1208 (2015).

\bibitem{Oh-book1}
T. Ohsawa. $L^2$ approaches in several complex variables. Development of Oka-Cartan theory by $L^2$ estimates for the $\bar\partial$ operator. Springer Monographs in Mathematics, Springer, Tokyo (2015) ix+196 pp. ISBN: 978-4-431-55746-3; 978-4-431-55747-0.

\bibitem{Oh-book2}
T. Ohsawa. $L^2$ approaches in several complex variables. Towards the Oka-Cartan theory with precise bounds. Second edition of \cite{Oh-book1}. Springer Monographs in Mathematics, Springer, Tokyo (2018) xi+258 pp. ISBN: 978-4-431-56851-3; 978-4-431-56852-0.

\bibitem{FA-Rudin}
Walter Rudin. Functional analysis. McGraw-Hill Series in Higher Mathematics, McGraw-Hill Book Co., New York-Düsseldorf-Johannesburg, 1973.






\end{thebibliography}
\end{document}

%------------------------------------------------------------------------------
% End of journal.tex
%------------------------------------------------------------------------------



