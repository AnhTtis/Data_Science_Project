% ****** Start of file apssamp.tex ******
%
%   This file is part of the APS files in the REVTeX 4.2 distribution.
%   Version 4.2a of REVTeX, December 2014
%
%   Copyright (c) 2014 The American Physical Society.
%
%   See the REVTeX 4 README file for restrictions and more information.
%
% TeX'ing this file requires that you have AMS-LaTeX 2.0 installed
% as well as the rest of the prerequisites for REVTeX 4.2
%
% See the REVTeX 4 README file
% It also requires running BibTeX. The commands are as follows:
%
%  1)  latex apssamp.tex
%  2)  bibtex apssamp
%  3)  latex apssamp.tex
%  4)  latex apssamp.tex
%
\documentclass[%
 reprint,
%superscriptaddress,
%groupedaddress,
%unsortedaddress,
%runinaddress,
%frontmatterverbose, 
%preprint,
%preprintnumbers,
%nofootinbib,
%nobibnotes,
%bibnotes,
 amsmath,amssymb,
 aps,
 prl,
%pra,
%prb,
%rmp,
%prstab,
%prstper,
%floatfix,
]{revtex4-2}

\usepackage{graphicx}% Include figure files
% \usepackage{dcolumn}% Align table columns on decimal point
\usepackage{bm}% bold math
\usepackage{physics}
% \usepackage{setspace}
% \doublespacing
%\usepackage{hyperref}% add hypertext capabilities
%\usepackage[mathlines]{lineno}% Enable numbering of text and display math
%\linenumbers\relax % Commence numbering lines

% \usepackage[showframe,%Uncomment any one of the following lines to test 
% %scale=0.7, marginratio={1:1, 2:3}, ignoreall,% default settings
% %text={7in,10in},centering,
% %margin=1.5in,
% %total={6.5in,8.75in}, top=1.2in, left=0.9in, includefoot,
% %height=10in,a5paper,hmargin={3cm,0.8in},
% ]{geometry}

\begin{document}

% \preprint{}

\title{Emergence of Economic and Social Disparities by Gift Interactions}% Force line breaks with \\
% \thanks{A footnote to the article title}%

\author{Kenji Itao}
 \email{itao@complex.c.u-tokyo.ac.jp}
\affiliation{%
 Department of Basic Science, Graduate School of Arts and Sciences, University of Tokyo, Komaba 3-8-1, Meguro-ku, Tokyo 153-8902, Japan.
}%
\affiliation{%
 Department of Human Behavior, Ecology, and Culture, Max-Planck-Institute for Evolutionary Anthropology, Leipzig, Germany.
}%

\author{Kunihiko Kaneko}%
 \email{kaneko@complex.c.u-tokyo.ac.jp}
\affiliation{%
 Research Center for Complex Systems Biology, University of Tokyo.}
 \affiliation{%
 The Niels Bohr Institute, University of Copenhagen, Blegdamsvej 17, Copenhagen, 2100-DK, Denmark.
}%

\date{\today}% It is always \today, today,
              % but any date may be explicitly specified


\begin{abstract}
Several tiers of social organizations with varying economic and social disparities have been observed, such as bands, tribes, chiefdoms, and kingdoms. Noting that anthropologists emphasize gifts as drivers of social change, we introduce a simple model for gift interactions. Numerical results and the corresponding mean-field theory demonstrate the transition of the above four socioeconomic phases, characterized by wealth and reputation distribution following exponential or power-law. A novel mechanism for social evolution is provided, expanding the scope of econo- and socio-physics.
\end{abstract}


%\keywords{Suggested keywords}%Use showkeys class option if keyword
                              %display desired

\maketitle

%\tableofcontents
Social scientists have observed that as population or territory size increases, societies become hierarchically organized and people's roles are differentiated \cite{service1962primitive, turchin2018quantitative}. They categorized social organizations into several classes, such as bands, tribes, chiefdoms, kingdoms, and states \cite{service1962primitive, kang2005examination}. Classes are characterized by different modes of ties (kinship, ideology, or labor division), production systems, and the degree of economic or social stratification. However, it is necessary to quantitatively characterize each of the classes and explain the transitions between them \cite{kang2005examination}. To this end, a statistical physics approach based on a simple model is useful, which will also explain the universality of these classes across the world.
Accordingly, we introduce a model based on anthropologists' emphasis on gifts as drivers of social change \cite{mauss1923essai, leach1982social, earle1989evolution}. We then characterize the above socioeconomic phases by the distributions of wealth and reputation and explore the transition in their shapes between exponential and power law.

In many regions of the world, gifts bring goods to the recipient and honor to the donor. Mauss identified three obligations associated with gifts: to give, to receive, and to reciprocate (often with amplification). Reciprocation unites people; however, those who fail to reciprocate lose their reputations and become subordinate to the donors \cite{mauss1923essai}. Ornaments or livestock are exchanged in this way at various rituals \cite{strathern1971rope, malinowski1922argonauts}. Anthropologists discuss the gift as a means of expanding alliances and gaining social status by imposing obligations on others \cite{mauss1923essai, leach1982social, komter2007gifts, moriyama2021gift}. Previously, we numerically illustrated the evolution of various network structures and socioeconomic disparities depending on the frequency and scale of gifts, which was consistent with ethnographic data \cite{itao2022transition}. However, the model considered both gift and kinship, which was too complicated to quantitatively analyze the transitions in disparities.

The degree of economic and social disparity varies across societies \cite{service1962primitive, smith2010wealth, mulder2010pastoralism, smith2010production}. Both exponential and power-law distributions are observed in incomes and the degree of social networks \cite{druagulescu2001exponential, barabasi1999emergence, newman2003social, jusup2022social, schnegg2006reciprocity, schnegg2015reciprocity, yakovenko2009econophysics, chakrabarti2013econophysics}.
Exponential distributions are observed when random interactions and reciprocal relationships are dominant \cite{schnegg2006reciprocity, schnegg2015reciprocity, yakovenko2009econophysics, chakrabarti2013econophysics}. Power-law distributions develop when ``rich get richer'' processes such as preferential attachment operate \cite{barabasi1999emergence, simon1955class}.
In this study, the presence and absence of strong disparity were characterized by power-law and exponential distributions, respectively.

Here, we introduce a simple model for the formation of economic and social structures through gift transactions. We demonstrate the transition in the distribution shape of wealth and social reputation score between exponential and power law, leading to the characterization of four phases of society: the band, without economic or social disparities; the tribe, with economic but without social disparities; the chiefdom with both; and the kingdom, with economic disparity and weak social disparity except for the ``monarch''. These conform to the anthropological typology \footnote{Precisely, anthropologists characterized the band as a kinship-based society, the tribe as a larger unit knit by siblinghood, the chiefdom as a hierarchy of classes, the kingdom as a hierarchy with stable royal families, and the state as legitimate monopoly of power via bureaucracy \cite{service1962primitive, kang2005examination}. Ethnographic studies show that the Gini coefficient is larger in hunter--gatherer bands than in agricultural or pastoral tribes \cite{smith2010wealth, mulder2010pastoralism, smith2010production}. Furthermore, class division corresponds to social disparity, and the stability of royal families results from the process by which the monarch becomes an exceptional outlier, as in our model.}. 
We then numerically and theoretically scrutinize the conditions for the transition.

\begin{figure*}[t]
\includegraphics[width=\linewidth]{gift_phys_distribution.pdf}
\caption{\label{fig:gift_distribution} Distributions of wealth (A) and score (B) for different interest rates $r$ and frequency of gifts $l$. Graphs show the semi-log and log-log plots of the distributions. For each distribution, by fitting the top 30\% via both exponential and power law, the one with the smaller error is plotted. The downward arrow in the bottom right graph represents the monarch.
}
\end{figure*}

In the model, $N$ individuals exist. In each step, they choose someone else to give their wealth, following the probability $p_{ij}$ explained below; then, they earn $1/l$ of wealth and reciprocate for received gifts by amplifying $1 + r$ times. When recipients' wealth is insufficient for reciprocation, they will have ``debt'' (and donors acquire credit), which is repaid through subsequent production. Debtors cannot bestow gifts before completing repayment.
We count the number of transfers of wealth from $i$ to $j$ either as gifts, reciprocations, or repayments, denoted by $q_{ij}$. The probability of $i$ choosing $j$ as a recipient is given by $p_{ij} = q_{ij} / \sum_k q_{ik}$, i.e., we introduce local preferential attachment. We term $\sum_k q_{ki} / l$, the number of $i$'s receipts, as $i$'s reputation score, indicating social status. Because wealth repeatedly moves from debtors to creditors, failure to reciprocate results in unequal relations and those who acquire more credit elevate their status.
In each step, individuals die with a probability of $1 / l$ and are replaced by new ones with $w_i = 1 / l$ and $q_{i*} = q_{*i} = 1$. At the time of $i$'s death, debts and credits related to $i$ disappear. 
The parameters involved are the interest rate for reciprocation $r$, expected number of gifts in lifetime $l$, and population size $N$, which is set at $N = 100$ unless otherwise mentioned.

Simulations were performed for $10^7$ time steps, within which stationary distributions of wealth and reputation score were realized. Fig. \ref{fig:gift_distribution} shows the distribution of wealth and score at the time of death in the last $9 \times 10^6$ steps. Distributions are fitted either by an exponential or power law. It shows that as the interest rate $r$ and frequency of gifts $l$ increase, the wealth distribution shifts from exponential to power law first, and then the score distribution follows. However, the score reverts to exponential distribution for extremely large $r$ and $l$.

\begin{figure*}[t]
\includegraphics[width=\linewidth]{gift_phys_phase.pdf}
\caption{\label{fig:gift_indices} Statistical characteristics of the simulation results. (A, D) Coefficients of variation $CV$ for wealth and score distributions, which equal 1.0 for exponential distribution and are greater for power-law. (B, E) The half-life of wealth and score distributions obtained by exponential fitting $\exp(- x / \beta)$. (C, F) Power exponent of wealth and score distributions resulting from power-law fitting $x^{-\alpha}$. The black line delineates the theoretical estimates $\alpha_w = 1 + 1 / rl$. (G) Phase diagram. The figure shows the parameter regions for $CV_w, CV_s < 1.1$ (purple), $CV_w \ge 1.1$ and $CV_s < 1.1$ (yellow), $CV_w, CV_s \ge 1.1$ (green), and $CV_w \ge 1.1$ and $CV_s < 1.1$ with $\beta_s > 10$ and the emergence of monarch (grey).
}
\end{figure*}

We performed the simulation $100$ times for each condition. Fig. \ref{fig:gift_indices} presents the average statistical characteristics of the simulation results. We estimated the transition in wealth and score distributions using the coefficients of variation $CV$ (SD/mean), which equal 1.0 for exponential and are greater for power-law distribution. 
Fitting gives the half-life of exponential distribution $\beta$ and the power-law exponent $\alpha$.
Fig. \ref{fig:gift_indices} suggests that these depend only on $rl$, the growth rate of wealth per generation. It illustrates that as $rl$ increases, wealth and score distributions shift successively to power-law distribution, and as $rl$ increases further, the score distribution reverts to exponential distribution with a longer half-life than that for small $rl$. 
Fig. \ref{fig:gift_indices}(G) displays the phase diagram against $r$ and $l$. The phases sequentially shift from ``band'' to ``tribe'' at $rl \simeq 0.25$, ``chiefdom'' at $rl \simeq 1.0$, and ``kingdom'' at $rl \simeq 500$. 
Here, phases are identified via $CV$ values. However, Fig. S1 indicates that the diagram is relatively unchanged even when phases are identified via the relative error of exponential and power-law fittings.

The observed trend can be roughly explained as follows: Since gift transactions involve amplified reciprocation, economic disparity intensifies. As the economic disparity is sufficiently large, many individuals fail to reciprocate, and social disparity exacerbates. Hence, stronger disparities emerge for larger $rl$. When economic disparity is extreme, an opulent monarch appears. The others earn scores only before they become debtors to the monarch, resulting in weak social disparities for the majority.

Now, we analytically estimate the above socioeconomic change. When $rl$ is small, the score distribution is exponential, and the preferential attachment for recipient choice is weak. Thus, in the economic change analysis, we assume that recipient choice follows uniform probability.

% Wealth increases via producing and being reciprocated and decreases by reciprocating. 
Wealth changes via production and reciprocation. Hence, the temporal development of wealth is given by 
\begin{align}
    \dot{w_i} &= 1 / l + r (w_i - \sum_j w_j \eta_j) =: f(w); \label{eq:langevin}\\
    &= 1 / l + r (w_i - \ev{w}) - r (\sum w\eta - \ev{w}), \label{eq:langevin2}
\end{align}
if reciprocation is performed appropriately. Here, $\eta$ is a random variable equal to $1$ with probability $1 / N$ and $0$ otherwise. When $i$ receives too many gifts and cannot reciprocate, $f(w_i) < -w_i.$ At that time, $i$ will have a debt of $f(w_i) - w_i$, and $w_i$ becomes $0$, i.e., $w$ is always nonnegative. Since $\ev{\dot{w_i}} = 1 / l$ and the life expectancy is $l$, $\ev{w} = 1.$
Eq.\eqref{eq:langevin2} clarifies the exponential growth of wealth owing to received interest and stochastic fluctuation occasioned by the random choice of recipient.

The term $\sum w\eta$, representing the amount of interest that $i$ should pay for reciprocation, fluctuates randomly depending on the number and nature of individuals that endowed gifts to $i$. First, let us ignore this fluctuation. Then, by neglecting the last term in Eq. \eqref{eq:langevin2}, the temporal development of wealth distribution is given by 
\begin{equation}
    \frac{\partial P}{\partial t} = - \frac{P}{l}  -\frac{\partial }{\partial w}\left(\frac{1}{l} + rw - r\ev{w}\right)P. \label{eq:FP}
\end{equation}
Its steady-state condition satisfies
 \begin{align}
    0 &= - P  -\frac{\partial P}{\partial w} - rl\frac{\partial }{\partial w}\left(w - \ev{w}\right)P. \label{eq:FP2}
\intertext{This is solved as} 
     P_1(w) &= \frac{(1 / rl - 1)^{1 / rl}/rl}{(w - 1 + 1/rl)^{1 + \frac{1}{rl}}},\ \  (w \neq 1 - 1 / rl)
\end{align}
by noting the normalization $\int P_1(x)dx = 1$.
% If $rl < 1$, $P_1 (w)$ is valid for $w > 0$.
If $rl > 1$, $P_1(w)$ diverges at $w = 1 - 1 / rl$. However, in reality, there is no divergence owing to the fluctuation in Eq. \eqref{eq:langevin2}, resulting in the diffusion term in Eq. \eqref{eq:FP}. Additionally, individuals with $w < 1 - 1 / rl\Leftrightarrow rw + 1 / l < r \ev{w}$ lose their wealth by reciprocating and eventually accrue debts. Then, $w$ remains at $0$ until repayment is completed.
Therefore, the wealth distribution for the rich follows $P_1(w)$, whereas that for the poor peaks at $w = 0$. The power exponent $\alpha_w$ equals $1 + 1 /rl$, indicating a greater disparity for larger $rl$. Note that the condition for frequent incurrence of debts is given by $rl > 1.$

In contrast, when the growth term $r(w - \ev{w})$ is negligible, the wealth change is dominated by random exchange---the last term in Eq. \eqref{eq:langevin2}. Noting that random exchange of energy leads to Boltzmann distribution as discussed in econophysics literature \cite{yakovenko2009econophysics, chakrabarti2013econophysics}, the steady distribution follows $P_2(w) = \exp(- w)$ since $\ev{w} = 1$.

We now estimate the condition for the transition from the exponential distribution $P_2$ to the power law $P_1$. The power law dominates when the second (growth) term surpasses the last (fluctuation) term in Eq. \eqref{eq:langevin2}. Let us consider the wealth dynamics of rich individuals: 
When $w_i - \ev{w}$ is typically larger than $\sum w\eta - \ev{w}$, their wealth grows exponentially, and the power-law distribution $P_1$ is obtained. 
Therefore, the deviation of the richer must surpass the variance anticipated by Boltzmann distribution.
The variance of the growth term $w - \ev{w}$ equals the variance of $P_1(w)$, i.e., $1 / (1 - 2 rl)$.
In contrast, that of the fluctuation $\sum w\eta - \ev{w}$ equals the sum of the variance of $P_2(w)$ and that derived by sampling from the binomial distribution, i.e., $1 + (1 - 1 /N)$.
Hence, the power-law distribution develops if
$1 / (1 - 2 rl) > 2 - 1 /N$, i.e., if $rl \gtrsim 0.25$.
Note that the transition at $rl = 0.25$, power-law exponent $\alpha_w = 1 + 1 / rl$ for $rl > 0.25$, and half-life $\beta_w = 1$ for $rl < 0.25$ are consistent with Fig. \ref{fig:gift_indices}.

Next, we provide an overview of the analysis of the distribution of the reputation score $P(s)$ (see Supplementary text for details).
Scores increase when individuals receive gifts, reciprocation, or repayment. One acquires credit when the interest $rw_i$ exceeds the recipient's wealth $w_j$. Debtors repay $1 / l$ of wealth, and creditors acquire $1 / l$ of score in each step. Hence, creditors acquire $|rw_i - w_j|_+ (:= \max(0, rw_i - w_j))$ of score when the repayment is completed. However, because $j$ may die before completing the repayment and $j$'s life expectancy is $l$, the expected increase in the score from the transaction is $\min(|rw_i - w_j|_+, 1)$. Therefore, the temporal development of the score is given by 
\begin{equation}
    \dot{s_i} = 1 / l + \sum \eta / l + \min(|rw_i - w_j|_+, 1). \label{eq:score_langevin}
\end{equation}
Recall that we have already demonstrated that economic disparity, denoted by $\alpha_w$, evolves with $rl$, and that $rl > 1$ gives a basic condition for the frequent incurrence of debts.
Hence, as $rl$ increases, $|rw_i - w_j|_+$ increases, and the score distributions change.

When $rl < 1$, $\min(|rw_i - w_j|_+, 1) = 0$ and $\ev{\dot{s_i}} = 2 / l$. Then, $P(s) = \exp(-s / 2) / 2$. When $rl > 1$, debts are incurred with a certain probability $p$. Then, $\min(|rw_i - w_j|_+, 1) = |rw_i - w_j|_+ \simeq \dot{w_i}$. Hence, $\dot{s}\to p\dot{w}$ if $\dot{w} \gg 1$, resulting in a power-law score distribution, since wealth distribution follows a power law at such $rl$ values. When $rl$ is sufficiently large, $\min(rw_i - w_j, 1) = 1$. Then, $\ev{s} = pt + 2t / l\simeq pt$, where $t$ represents the time that passes before individuals receive a gift from the richest, following $P(t) = \exp(-t / N) / N$. Subsequently, the score distribution for the majority is $P(s) = \exp(-s / pN) / pN$. Numerical results for different $N$ values show that $\max \beta_s \simeq 0.2 N$ (see Fig. S2), indicating that $p \simeq 0.2$. This value is reasonable as preferential attachment limits the possibility of acquiring credit by giving gifts to previously untraded individuals. Note that credit is acquired only when the recipient is not indebted to the donor.
To summarize, the score distribution shifts from an exponential distribution with a half-life $\beta_s = 2$ to a power law, then reverts to exponential distribution with $\beta_s = 0.2 N$ as $rl$ increases, consistent with Fig. \ref{fig:gift_indices}.

In this study, we built a model of gift transactions and numerically demonstrated the development of several phases of social organizations that are quantitatively characterized by the shape of wealth and reputation score distributions, i.e., (i) the band phase, where both distributions are exponential; (ii) the tribe phase, where only wealth distribution obeys a power law; (iii) the chiefdom phase, where both are power-law; and (iv) the kingdom phase, where the score distribution is exponential with the monarch. We then analytically explained their transitions, whose boundaries are defined by $rl$, the product of the interest rate and frequency of gifts. 

Exponential and power-law distributions have been observed in social systems \cite{yakovenko2009econophysics, chakrabarti2013econophysics, cho2014physicists, tao2019exponential, barabasi1999emergence, newman2003social}. Generally, the power laws originate from the ``rich get richer'' process, whereas exponential distributions arise from random exchange or reciprocity \cite{cho2014physicists, schnegg2006reciprocity}. In this study, we demonstrated the transition of distribution shapes for wealth and reputation scores through gift transactions.
Polanyi proposed reciprocity, centralized redistribution, and market exchange as basic modes of economic activity, and stressed that economic activities are inseparable from political and social interactions \cite{polanyi1957economy}. Such interrelationships have been dismissed in present mainstream econophysics focusing on market exchange \cite{yakovenko2009econophysics, chakrabarti2013econophysics}.
In our model, however, reciprocity works in the band and tribe phases because most gifts are reciprocated appropriately, whereas centralized redistribution emerges in the chiefdom and kingdom phases as a constant flow of repayment from the vast majority to rich individuals.

The dynamics of the final ``kingdom phase'' resemble the merge-and-create process, where $N$ variables are set, and two randomly chosen values are replaced by their sum and $1$ \cite{takayasu1989steady, minnhagen2004self}. The largest value then increases indefinitely with time, whereas the remaining variables exhibit a power-law distribution. As in our model, the richest exceptional outlier follows distinct dynamics, and is sometimes termed ``dragon-king'' \cite{laherrere1998stretched, sornette2012dragon}. In our model, such an outlier suppresses the activities of others, as with ``kings (monarchs)'' in history \cite{robinson2012nations}.

Ethnographic reports suggest that the increase in tradable goods leads to large $r$ and $l$ \cite{mauss1923essai, strathern1971rope}. Social scientists have argued that increased population density and surplus production will accelerate the interaction of people, including gifts, and promote social  stratification \cite{service1962primitive, bataille1949part, von2019dynamics}. Empirical data analysis suggests that as the frequency and scale of gifts increase, economic disparity arises first, then social disparity follows \cite{itao2022transition}.
Here, we demonstrated such social change driven by gift transactions and proposed the interest rate $r$ and the frequency of gifts $l$ as basic explanatory variables to be measured.

In conclusion, by proposing a simple model of gift transactions, we demonstrate the evolution of four phases of social organizations characterized by the degree of economic and social disparities. As the interest rate $r$ and frequency of gifts $l$ increase, wealth distribution shifts from exponential to power-law first, then score distribution follows. For extremely large $r$ and $l$, the score distribution reverts to exponential accompanied by the emergence of an exceptional outlier. We analytically derive the phase boundaries governed by $rl$ values. The current work explains the basic mechanism of social evolution and expands the scope of econo- and socio-physics.

The authors thank Tetsuhiro S. Hatakeyama, Koji Hukushima, and Kim Sneppen for a stimulating discussion.
This research was supported by the Grant-in-Aid for Scientific Research (A) 431 (20H00123) from the Ministry of Education, Culture, Sports, Science, and Technology (MEXT) of Japan, JSPS KAKENHI Grant Number JP21J21565 (KI), and Novo Nordisk Fonden (KK).

\bibliography{apssamp}

\end{document}


%
% ****** End of file apssamp.tex ******
