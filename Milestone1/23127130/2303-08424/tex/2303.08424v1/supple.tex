% ****** Start of file apssamp.tex ******
%
%   This file is part of the APS files in the REVTeX 4.2 distribution.
%   Version 4.2a of REVTeX, December 2014
%
%   Copyright (c) 2014 The American Physical Society.
%
%   See the REVTeX 4 README file for restrictions and more information.
%
% TeX'ing this file requires that you have AMS-LaTeX 2.0 installed
% as well as the rest of the prerequisites for REVTeX 4.2
%
% See the REVTeX 4 README file
% It also requires running BibTeX. The commands are as follows:
%
%  1)  latex apssamp.tex
%  2)  bibtex apssamp
%  3)  latex apssamp.tex
%  4)  latex apssamp.tex
%
\documentclass[14pt, a4paper]{article}  
\usepackage{amsmath,amssymb}
% \documentclass[%
%  reprint,
% %superscriptaddress,
% %groupedaddress,
% %unsortedaddress,
% %runinaddress,
% %frontmatterverbose, 
% %preprint,
% %preprintnumbers,
% %nofootinbib,
% %nobibnotes,
% %bibnotes,
%  amsmath,amssymb,
%  aps,
% %pra,
% %prb,
% %rmp,
% %prstab,
% %prstper,
% %floatfix,
% ]{revtex4-2}

\usepackage{graphicx}% Include figure files
% \usepackage{dcolumn}% Align table columns on decimal point
\usepackage{bm}% bold math
\usepackage{physics, float}
% \usepackage{setspace}
% \doublespacing
%\usepackage{hyperref}% add hypertext capabilities
%\usepackage[mathlines]{lineno}% Enable numbering of text and display math
%\linenumbers\relax % Commence numbering lines

\usepackage[margin=25truemm]{geometry}
% \usepackage[showframe,%Uncomment any one of the following lines to test 
% %scale=0.7, marginratio={1:1, 2:3}, ignoreall,% default settings
% %text={7in,10in},centering,
% %margin=1.5in,
% %total={6.5in,8.75in}, top=1.2in, left=0.9in, includefoot,
% %height=10in,a5paper,hmargin={3cm,0.8in},
% ]{geometry}
\renewcommand{\thefigure}{S\arabic{figure}}
\begin{document}

% \preprint{}

\title{Supplementary Information for:
Emergence of Economic and Social Disparities by Gift Interactions}% Force line breaks with \\
% \thanks{A footnote to the article title}%

\author{Kenji Itao and Kunihiko Kaneko}

% \date{\today}% It is always \today, today,
             %  but any date may be explicitly specified
% {\centering \large
% Supplementary Information for:\\
% Emergence of Economic and Social Disparities by Gift Interactions}


%\keywords{Suggested keywords}%Use showkeys class option if keyword
                              %display desired

\maketitle
% \bigskip

\section*{Analytical estimation of social change}
Here, we analyze how the distribution of the reputation score $P(s)$ shifts between exponential and power law with the change in $rl$. Scores increase upon receipt of gifts, reciprocation, or repayment of debts. One acquires credit when the interest $rw_i$ exceeds the recipient's wealth $w_j$. Debtors repay $1 / l$ of wealth while creditors acquire $1 / l$ of score in each step. Hence, creditors acquire $|rw_i - w_j|_+ (:= \max(0, rw_i - w_j))$ of score when the repayment is completed. However, because $j$ may die before completing the repayment and $j$'s life expectancy is $l$, the expected increase in the score from the transaction is given by $\min(|rw_i - w_j|_+, 1)$. Therefore, the time development of the score is given by 
\begin{equation}
    \dot{s_i} = 1 / l + \sum \eta / l + \min(|rw_i - w_j|_+, 1). \label{eq:score_langevin}
\end{equation}
Recall that economic disparity, characterized by $\alpha_w$, evolves with $rl$, and that $rl > 1$ provides a basic condition for the frequent incurrence of debts.
Hence, as $rl$ increases, $|rw_i - w_j|_+$ increases, and the score distributions change.

When $rl < 1$, most gifts are reciprocated immediately. In other words, $rw_i - w_j$ is generally negative and the third term in eq. \eqref{eq:score_langevin} is negligible. Then, $\ev{s} = 2 / l$ and $\dot{s_i} = 2 / l$.
Accordingly, $s_i = 2t / l,$ where $t$ is $i$'s age. Since the age distribution is $P(t) = \exp(-t / l) / l$, the score distribution is $P(s) = \exp(-s / 2) / 2$. 

When $rl > 1$, some gifts are not reciprocated. Hence, $\min(|rw_i - w_j|_+, 1) = rw_i - w_j$ as long as the debt is repayable before death. Note that individuals acquire credit only if they give to those who have insufficient wealth. By denoting its probability as $p$ and averaging the stochastic term, we obtain the following estimate:
\begin{align}
    \ev{\dot{s_i}} &= 2 / l + p|rw_i - \ev{w}|_+,  \\
        &= 2 / l +  p|\dot{w_i} - 1 + r - 1 / l|_+.
\end{align}
Hence, $\dot{s}\to p\dot{w}$ if $\dot{w} \gg 1$. For $rl > 1$, since wealth distribution follows a power law, score distribution also follows a power law.
However, $\dot{s}\to 2 / l$ for small $\dot{w} ( < 2 / l)$.
Therefore, the score distribution has more middle classes than the wealth distribution, resulting in $\alpha_s > \alpha_w$ (see Figs. 1 and 2). The transition of the score distribution from exponential to power law at $rl \simeq 1$ is consistent with the simulation results in Fig. 2. Hence, the score distribution is exponential at $rl \simeq 0.25$, which validates the ignorance of connection bias in the analysis of economic change, down to such $rl$ values.

When $rl$ is sufficiently large, debtors often die before completing repayment. Hence, the expected increment of score from  credit is generally $\min(rw_i - w_j, 1) = 1$, and thus $\ev{\dot{s_i}} = 2 / l + p$, leading to
\begin{align}
    s_i &= pt + 2t / l\simeq pt.
\end{align}
Hence, the score distribution reverts to exponential if the distribution of $t$ is exponential. Here, $t$ indicates the number of gifts that one makes, which is generally different from the age, in contrast to the case with $rl < 1$.
If individuals receive gifts from the rich, they will need many time steps for repayment and will no longer earn scores. Note that once individuals are debtors, their social connections to the rich are strengthened by local preferential attachment. Since rich people can reciprocate gifts, they will no longer earn a score, even after finishing repayment.
Hence, $t$ generally equals the time that passes before an individual receives a gift from the richest individual. As the rich choose recipients randomly from $N$ individuals, the distribution of $t$ is determined by $P(t) = \exp(-t / N) / N$. Subsequently, the score distribution, except for the richest, is $P(s) = \exp(-s / pN) / pN$. On the other hand, the expected value of $t$ for the richest individual is the lifetime $l$. Hence, the score distribution will be exponential, with a half-life proportional to $N$ for most individuals and $l$ for the richest. Note that the gifts from the moderately rich may be repayable, and the score distribution for the middle class still follows the power law since $\min(|rw_i - w_j|_+, 1) = rw_i - w_j$. Additionally, the richest individual (monarch) cannot find debtors if all the others are already his/her debtors. Therefore, the largest score is bounded by $\mathcal{O}(N)$. Numerical results for different values of $N$ (Fig. S2) show that $\max \beta_s \simeq 0.2 N$, indicating that $p \simeq 0.2$.

\section*{Supplementary figures}
\begin{figure}[H]
\includegraphics[width=\linewidth]{gift_phys_phase_index_supp.pdf}
\caption{\label{fig:gift_indices} Statistical characteristics of the simulation results. (A) Relative error of power-law and exponential fitting for wealth $RE_w$. The logarithm of the relative error is obtained by fitting the top 30\% with exponential and power-law distributions. The positive value indicates that power-law fitting is more appropriate than exponential fitting. (B) Relative error of power-law and exponential fitting for reputation score $RE_s$. (C) Phase diagram. The figure shows the parameter regions for $RE_w, RE_s < 0$ (purple); $RE_w \ge 0$ and $RE_s < 0$ (yellow); $RE_w, RE_s \ge 0$ (green); $RE_w \ge 0$ and $RE_s < 0$ with the monarch (grey), which gives almost the same diagram as in Fig. 2.
}
\end{figure}

\begin{figure}[H]
\includegraphics[width=\linewidth]{gift_phys_score_exp_N.pdf}
\caption{\label{fig:gift_beta_s} Dependence of half-life of score distribution $\beta_s$ on population size $N$. (A) Dependence of $\beta_s$ on $rl$ for different population size $N$. (B) The relationship between $\max \beta_s$ and $N$. The black solid line is the regression line $\max \beta_s = 0.2 N$.
}
\end{figure}

\end{document}

