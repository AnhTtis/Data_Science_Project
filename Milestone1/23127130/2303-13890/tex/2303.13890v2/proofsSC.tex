\subsection{Proofs of the power series representation and the behaviour at zero}\label{subsec:power}

%\begin{proof}[Proof of Lemma \ref{lem:convtail}]
%Since $\bar{\nu}_\Phi(t)$ are in $L^1_{\text{loc}} \l \mathbb{R}^+ \r$ the $n$-fold convolution is well defines and using \eqref{def:Phi} together with \cite[Proposition 1.6.4]{abhn} then yields, for $\rho >0$,
%\begin{equation*}
%\int_0^{+\infty} e^{-\rho t} \bar{\nu}_\Phi^{n\star} (t) \, dt \, = \, \l \frac{\Phi^\dagger(\rho)}{\rho} \r^n.
%\end{equation*}
%By arguing as in \eqref{cesequalpointdag} we have that, for fixed $a>0$,
%\begin{equation*}
%\bar{\nu}_\Phi^{n\star}(t) \, = \,  \lim_{R \to +\infty} \int_{a-iR}^{a+iR} e^{zt} \l \frac{\Phi^\dagger(z)}{z} \r^n \, dz
%\end{equation*}
%for any $t>0$ such that the limit exists. Hence we can apply Cauchy's theorem over the same contour (see fig. \ref{fig1} and the corresponding explanation) to get
%\begin{equation}\label{531}
%	\begin{split}
%		0 \, = \, \int_{\partial D} F(z;t)  dz	= \,  \l \int_{\ell_1} + \int_{\Gamma_R^+} + \int_{\ell_2} + \int_{\gamma_\varepsilon} + \int_{\ell_3} + \int_{\Gamma_R^-} \r F(z;t) \, dz
%	\end{split}
%\end{equation}
%where
%\begin{equation*}
%F(z;t) \, := \,\l  \frac{\Phi^\dagger(z)}{z} \r^{n} \, e^{zt}.
%\end{equation*}
%In order to deal with the integral over $\Gamma_R^+$, let $M^+(R)= Ri$ and split the curve $\Gamma_R^+$ into $\Gamma_R^1$ connecting $B(R)$ to $M^+(R)$ and $\Gamma_R^2$ connecting $M^+(R)$ to $C(R)$ so that
%	\begin{equation}
%		\int_{\Gamma_R^+} F(z; x, t) \, dz \, = \, \int_{\Gamma_R^1} F(z; x, t) \, dz  + \int_{\Gamma_R^2} F(z; x, t) \, dz.
%			\end{equation}
%Note now that 
%\begin{equation*}
%\left| \int_{\Gamma_R^{1}} F(z;t) \, dz \right| \, \leq \, \text{length}(\Gamma_R^+) \max_{z \in \Gamma_R^1} \left| F(z,t) \right| \notag \\
%\leq \,  \text{length} \l \Gamma_R^1 \r \, e^{ta} \, \max_{z \in \Gamma_R^1} \left| \frac{\Phi^\dagger (z)}{z} \right|^n
%\end{equation*}
%where we used have that $\Re z \leq a$, for any $z \in \Gamma_R^1$.
%Furthermore, we know by Lemma \ref{lem:Bern}, Item \ref{it:asymp}, that
%\begin{equation*}
%\max_{z \in \Gamma_R^1} \left| \frac{\Phi^\dagger (z)}{z} \right|^n \to 0
%\end{equation*}
%and since $\text{length}(\Gamma_R^1) \to a$ we can conclude
%\begin{equation}\label{536}
%\lim_{R\to +\infty} \int_{\Gamma_R^1} F(z;t) \, dx \, = \, 0.
%\end{equation}
%On $\Gamma_R^2$ note instead that
%\begin{equation*}
%\int_{\Gamma_R^2} F(z;t) \, dz \, = \,i \int_{\widetilde{\Gamma}_R^2} F(iz;t) \, dz,
%\end{equation*}
%where $\widetilde{\Gamma}_R^2 = \ll z \in \mathbb{C}: z=Re^{i \xi}, \xi \in \left[0, \frac{\pi-\theta}{2} \right] \rr$. It follows that
%\begin{align}
%		\left|\int_{\Gamma_R^2} F(z;t) \, dz  \right|  &=   \left| \int_0^{\frac{\pi-\theta}{2}} e^{itRe^{i\xi}} \l \frac{\Phi^\dagger \l iRe^{i\xi} \r}{iRe^{i\xi}} \r^n Re^{i\xi} \, d\xi \right| \\
%		\leq \, & R \int_0^{\frac{\pi-\theta}{2}} e^{-Rt\sin \xi} \left| \frac{\Phi^\dagger \l iRe^{i\xi} \r}{iRe^{i\xi}} \right|^n \, d\xi  \\
%		\leq \, & R \l \max_{\xi \in \left[0, \frac{\pi-\theta}{2}\right]} \left| \frac{\Phi^\dagger \l iRe^{i\xi} \r}{iRe^{i\xi}} \right|^n \r \int_0^\pi e^{-Rt \sin \xi} \, d\xi  \\
%		= \, &\frac{1}{2t}  \l \max_{\xi \in \left[0, \frac{\pi-\theta}{2}\right]} \left| \frac{\Phi^\dagger \l iRe^{i\xi} \r}{iRe^{i\xi}} \right|^n \r  
%		\to \,  0,
%\end{align}
%as $R\to\infty$, where in the last step we used \eqref{eq:uniformlimcond}. Therefore 
%$
%\lim_{R \to +\infty} \int_{\Gamma_R^2} F(z;t) dz \, = \, 0
%$
%so that, using this with \eqref{536}, we conclude that
%\begin{equation*}
%\lim_{R \to +\infty} \int_{\Gamma_R^+} F(z;t) dz \, = \, 0.
%\end{equation*}
%The integral on $\Gamma_R^-$ can be dealt with analogously, to get
%\begin{equation*}
%\lim_{R \to +\infty} \int_{\Gamma_R^-} F(z;t) \, dz \, = \, 0.
%\end{equation*}
%Now we deal with the integral on $\ell_2$. We have that
%\begin{equation*}
%\int_{\ell_2} F(z;t) \, dz \, = \, -\int_{\varepsilon}^R e^{t \rho e^{i \l \pi-\frac{\theta}{2} \r}} \l \frac{\Phi^\dagger \l \rho e^{i \l \pi-\frac{\theta}{2} \r} \r}{\rho e^{i \l \pi-\frac{\theta}{2} \r}} \r^n \, e^{i \l \pi-\frac{\theta}{2} \r} \, d\rho.
%\end{equation*}
%We can take limit as $R \to +\infty$ as it can be ascertained by observing that
%\begin{equation}\label{544}
%	\begin{split}
%		\int_{\varepsilon}^{+\infty} \left| e^{t \rho e^{i \l \pi-\frac{\theta}{2} \r}} \l \frac{\Phi^\dagger \l \rho e^{i \l \pi-\frac{\theta}{2} \r} \r}{\rho} \r^n \right| \, d\rho \, = \, &\int_{\varepsilon}^{+\infty}  e^{-t \rho \cos \frac{\theta}{2}} \left| \frac{\Phi^\dagger \l \rho e^{i \l \pi-\frac{\theta}{2} \r} \r}{\rho} \right|^n \, d\rho \\
%		\leq \, & C \int_{\varepsilon}^{+\infty} e^{-t\rho \cos \frac{\theta}{2}} d\rho < +\infty,
%	\end{split}
%\end{equation}
%where we used \eqref{eq:uniformlimcond}. Hence, we get that
%\begin{equation*}
%	\begin{split}
%		\lim_{R \to +\infty} \int_{\ell_2} F(z;t) \, dz =- \!\!\int_{\varepsilon}^{+\infty} \!\!e^{t \rho e^{i \l \pi-\frac{\theta}{2} \r}} \!\!\l \frac{\Phi^\dagger \l \rho e^{i \l \pi-\frac{\theta}{2} \r} \r}{\rho e^{i \l \pi-\frac{\theta}{2} \r}} \r^n \!\!e^{i \l \pi-\frac{\theta}{2} \r}\,  d\rho=:\!\!- I_1 (\varepsilon)
%	\end{split}
%\end{equation*}
%and analogously
%\begin{equation*}
%	\begin{split}
%		\lim_{R \to +\infty} \int_{\ell_3} F(z;t) \, dz =- \!\!\int_{\varepsilon}^{+\infty} \!\!e^{t \rho e^{i \l \pi-\frac{\theta}{2} \r}} \!\!\l \frac{\Phi^\dagger \l \rho e^{i \l \pi-\frac{\theta}{2} \r} \r}{\rho e^{i \l \pi-\frac{\theta}{2} \r}} \r^n \!\!e^{i \l \pi-\frac{\theta}{2} \r}\,  d\rho=:\!\!- I_2 (\varepsilon)
%	\end{split}
%\end{equation*}
%By the Hermite property of the involved functions we obtain that
%$
%I_1(\varepsilon) \, = \, \overline{I_2(\varepsilon)}$.
%Collecting all pieces together we can let $R \to +\infty$ in \eqref{531} to get
%\begin{equation}\label{549}
%	\begin{split}
%		&\lim_{R \to +\infty} \int_{\ell_1} F(z;t) dz 
%		=  2i \Im I_1(\varepsilon) - \int_{\gamma_{\varepsilon}} F(z;t) dz \\
%		 & =2i \int_{\varepsilon}^{+\infty} \Im \left[ \l \frac{\Phi^\dagger\l \rho \l e^{i \l \pi-\frac{\theta}{2} \r} \r \r}{\rho e^{i \l \pi-\frac{\theta}{2} \r}} \r^n e^{i \l \pi-\frac{\theta}{2} \r} e^{t \rho e^{i \l \pi-\frac{\theta}{2} \r}} \right] d\rho - \int_{\gamma_\varepsilon} F(z;t) \, dz,
%	\end{split}
%\end{equation}
%where in the last step, where we moved the imaginary part inside the integral, is justified by \eqref{544}.
%Clearly, the integral on $\gamma_\varepsilon$ is finite. Indeed
%\begin{equation*}
%		\left| \int_{\gamma_\varepsilon} F (z; t) dz\right| \, \leq \, \text{length}(\gamma_{\varepsilon}) \max_{\xi \in \left[ - \l \pi-\frac{\theta}{2} \r, \pi-\frac{\theta}{2}  \right]} \left| F(\varepsilon e^{i\xi}; t) \right| <+\infty
%	\end{equation*}
%since $\Phi^\dagger$ is continuous on $\overline{\mathbb{C}\l \pi-\frac{\theta}{2} \r}$ by \eqref{eq:extensionA3}.
%	This proves \eqref{convcode}.
%	Now we show that we can differentiate inside the integrals in \eqref{convcode} to get \eqref{derivatecode} and that, in the first integral, we can interchange the derivative and the imaginary part.
%		Observe that
%	\begin{equation*}
%\frac{\partial^r}{\partial t^r}\Im \left[ \l \frac{\Phi^\dagger \l \rho e^{i \l \pi-\frac{\theta}{2} \r} \r}{\rho e^{i \l \pi-\frac{\theta}{2} \r}} \r^n e^{i \l \pi-\frac{\theta}{2} \r}e^{t\rho e^{i \l \pi-\frac{\theta}{2} \r}} \right] \, = \, \Im \left[  \frac{\l\Phi^\dagger \l \rho e^{i \l \pi-\frac{\theta}{2} \r} \r\r^n}{\rho^{n-r}e^{i(n-r-1)\l \pi-\frac{\theta}{2} \r}}  e^{t\rho e^{i \l \pi-\frac{\theta}{2} \r}} \right].
%	\end{equation*}
%%\begin{align}
%%&\Im \left[ \l \frac{\Phi^\dagger \l \rho e^{i \l \pi-\frac{\theta}{2} \r} \r}{\rho e^{i \l \pi-\frac{\theta}{2} \r}} \r^n e^{i \l \pi-\frac{\theta}{2} \r}e^{t\rho e^{i \l \pi-\frac{\theta}{2} \r}} \right] \notag \\
%% = \,& \Im \l  \l \frac{\Phi^\dagger \l \rho e^{i \l \pi-\frac{\theta}{2} \r} \r}{\rho e^{i \l \pi-\frac{\theta}{2} \r}} \r^n e^{i \l \pi-\frac{\theta}{2} \r} \r  e^{-t\rho \cos \frac{\theta}{2}} \cos \l t\rho \sin \frac{\theta}{2} \r \notag \\& + \Re \l  \l \frac{\Phi^\dagger \l \rho e^{i \l \pi-\frac{\theta}{2} \r} \r}{\rho e^{i \l \pi-\frac{\theta}{2} \r}} \r^n e^{i \l \pi-\frac{\theta}{2} \r} \r e^{-t\rho \cos \frac{\theta}{2}} \sin \l t\rho \sin \frac{\theta}{2} \r
%%\end{align}
%%and then taking the derivatives explicitly.
%Once we have this, using \eqref{eq:uniformlimcond}, for $\rho \geq \varepsilon$, we can write for the first integral, taking $[t_1, t_2] \subset (0, +\infty)$
%	\begin{equation*}
%		\begin{split}
%				\left| \Im \left[  \frac{\l\Phi^\dagger \l \rho e^{i \l \pi-\frac{\theta}{2} \r} \r\r^n}{\rho^{n-r}e^{i(n-r-1)\l \pi-\frac{\theta}{2} \r}}  e^{t\rho e^{i \l \pi-\frac{\theta}{2} \r}} \right] \right| \,
%			\leq \,&  e^{-t\rho \cos \frac{\theta}{2}} \rho^{r}\left| \frac{ \Phi^\dagger \l \rho e^{i \l \pi-\frac{\theta}{2} \r} \r}{\rho} \right|^n 
%			\leq \,  C  \rho^{r}e^{-t_1 \rho\cos \frac{\theta}{2}}, 
%		\end{split}
%	\end{equation*}
%	which is integrable. It follows that the first integral in \eqref{derivatecode} is finite, i.e. we can differentiate inside the first integral in \eqref{convcode}. The same is true for the second integral in \eqref{convcode}, indeed
%	\begin{equation*}
%	\left| \int_{\gamma_\varepsilon} z^r F(z,t) dz \right| \, \leq \, \varepsilon(2\pi-\theta)  \max_{z \in \gamma_\varepsilon} |z^r F(z;t)| < +\infty
%	\end{equation*}
%	since $z^rF(z;t)$ is continuous on $\mathbb{C}\l \pi-\frac{\theta}{2} \r$ by \eqref{eq:extensionA3}.
%\end{proof}
\begin{proof}[Proof of Theorem \ref{thm:seriespi}]
As in the \eqref{derivatefphi}, we have
\begin{align}	
		\begin{split}
				\frac{\partial^k}{\partial x^k}\frac{\partial^l}{\partial t^l}f_\phi(x,t)&= \frac{(-1)^k}{\pi} \int_\varepsilon^{+\infty}\Im\left(\frac{\Phi^\dagger\lb \rho e^{i\left(\pi-\frac{\theta}{2}\right)}\rb\lb\Phi\lb \rho e^{i\left(\pi-\frac{\theta}{2}\right)}\rb\rb^{k}}{\rho^{1-l} e^{-i l \left(\pi-\frac{\theta}{2}\right)}}e^{-x \Phi \lb \rho e^{i\left(\pi-\frac{\theta}{2}\right)}  \rb+ t \rho e^{i\left(\pi-\frac{\theta}{2}\right)} }\right) \, d\rho \\
				&+\frac{(-1)^k}{2\pi i}\int_{\gamma_{\varepsilon,\theta}} \frac{\Phi^\dagger(z)(\Phi(z))^{k}}{z^{1-l}}e^{-x\Phi(z)+tz}dz.
				\end{split}
\end{align}
Writing $e^{-x\Phi(z)}$ as a power series and assuming we can exchange the series with the integral, we have
\begin{align}	
	\begin{split}
		&\frac{\partial^k}{\partial x^k}\frac{\partial^l}{\partial t^l}f_\phi(x,t)= \sum_{j=0}^{+\infty}(-1)^{k+j}\frac{x^j}{j!}\\
		&\times \left[\frac{1}{\pi} \int_\varepsilon^{+\infty}\Im\left(\frac{\Phi^\dagger\lb \rho e^{i\left(\pi-\frac{\theta}{2}\right)}\rb\lb\Phi\lb \rho e^{i\left(\pi-\frac{\theta}{2}\right)}\rb\rb^{k+j}}{\rho^{1-l} e^{-i l \left(\pi-\frac{\theta}{2}\right)}}e^{t \rho e^{i\left(\pi-\frac{\theta}{2}\right)} }\right) \, d\rho+\frac{1}{2\pi i}\int_{\gamma_{\varepsilon,\theta}} \frac{\Phi^\dagger(z)(\Phi(z))^{k+j}}{z^{1-l}}e^{tz}dz\right]\\
		&=\sum_{j=0}^{+\infty}\sum_{k_1+k_2+k_3=k+j} \frac{(k+j)!}{k_1!k_2!k_3!}(-1)^{k+j}\frac{x^j}{j!}q^{k_1}\mathfrak{b}^{k_2}\\
		&\times \left[\frac{1}{\pi} \int_\varepsilon^{+\infty}\Im\left(\frac{\left(\Phi^\dagger\lb \rho e^{i\left(\pi-\frac{\theta}{2}\right)}\rb\right)^{k_3+1}}{\rho^{1-l-k_2} e^{-i (l+k_2) \left(\pi-\frac{\theta}{2}\right)}}e^{t \rho e^{i\left(\pi-\frac{\theta}{2}\right)} }\right) \, d\rho+\frac{1}{2\pi i}\int_{\gamma_{\varepsilon,\theta}} \frac{(\Phi^\dagger(z))^{k_3+1}}{z^{1-l-k_2}}e^{tz}dz\right]\\
		&=\sum_{j=0}^{+\infty}\sum_{k_1+k_2+k_3=k+j} \frac{(k+j)!}{k_1!k_2!k_3!}(-1)^{k+j}\frac{x^j}{j!}q^{k_1}\mathfrak{b}^{k_2}\\
		&\times \left[\frac{1}{\pi} \int_\varepsilon^{+\infty}\Im\left(\frac{\left(\Phi^\dagger\lb \rho e^{i\left(\pi-\frac{\theta}{2}\right)}\rb\right)^{k_3+1}}{\rho^{k_3+1-(k_2+k_3+l)} e^{i (k_3-(l+k_2+k_3)) \left(\pi-\frac{\theta}{2}\right)}}e^{t \rho e^{i\left(\pi-\frac{\theta}{2}\right)} }\right) \, d\rho+\frac{1}{2\pi i}\int_{\gamma_{\varepsilon,\theta}} \frac{(\Phi^\dagger(z))^{k_3+1}}{z^{k_3+1-(k_2+k_3+l)}}e^{tz}dz\right]\\
		&=\sum_{j=0}^{+\infty}\sum_{k_1+k_2+k_3=k+j} \frac{(k+j)!}{k_1!k_2!k_3!}(-1)^{k+j}\frac{x^j}{j!}q^{k_1}\mathfrak{b}^{k_2}\frac{d^{k_2+k_3+l}}{d t^{k_2+k_3+l}}\mu^{\ast (k_3+1)}(t),
	\end{split}
\end{align} 
where we used \eqref{derivatecode} in the last equality. Now we only have to prove that we can exchange the series with the integrals. This is clear for the integral over $\gamma_{\varepsilon, \theta}$, thus let us only consider the one over $(\varepsilon,+\infty)$.
%
%We use \eqref{diffint} to compute the series representation, as follows (any step is justified at the end of computation)
%\begin{align}
%&\frac{\partial^l}{\partial t^l}\frac{\partial^k}{\partial x^k}  f_\Phi(x,t) \notag \\
% = \, &  \frac{\partial^l}{\partial t^l} \frac{\partial^k}{\partial x^k} \left[\frac{1}{\pi} \Im \l \int_\varepsilon^{+\infty}    \frac{\Phi^\dagger \l \rho e^{i\l \pi-\frac{\theta}{2} \r} \r}{\rho}   e^{-x\Phi \l \rho e^{i \l \pi-\frac{\theta}{2} \r}  \r+t\rho e^{i \l \pi-\frac{\theta}{2} \r}} d\rho \r - \frac{1}{2\pi i} \int_{\gamma_\varepsilon} \frac{\Phi^\dagger (z)}{z}  e^{-x\Phi(z)+tz} dz \right] \label{impartandintegral} \\
%   = \, & \frac{\partial^l}{\partial t^l} \frac{\partial^k}{\partial x^k}  \sum_{j=0}^{+\infty} (-1)^j\frac{x^j}{j!} \left[ \frac{1}{\pi}\Im \l \int_\varepsilon^{+\infty}    \frac{\Phi^\dagger \l \rho e^{i\l \pi-\frac{\theta}{2} \r} \r}{\rho} \Phi^{j}\l \rho e^{i \l \pi-\frac{\theta}{2} \r} \r   e^{t\rho e^{i \l \pi-\frac{\theta}{2} \r}}  d\rho \r \right. \notag \\
%   & \left. - \frac{1}{2\pi i} \int_{\gamma_\varepsilon} \frac{\Phi^\dagger (z)}{z} \Phi^j (z)  e^{tz} dz \right] \label{serieswithintandim} \\
%   = \, & \frac{\partial^l}{\partial t^l} \sum_{j=0}^{+\infty} \frac{x^j}{j!}(-1)^{j+k} \left[ \frac{1}{\pi}\Im \l\int_\varepsilon^{+\infty}  \Phi^\dagger \l \rho e^{i \l \pi-\frac{\theta}{2} \r} \r \rho^{-1} \phi^{j+k} \l \rho e^{i \l \pi-\frac{\theta}{2} \r} \r e^{t\rho e^{i \l \pi-\frac{\theta}{2} \r}}  d\rho \r \right. \notag \\
%    & \left. - \frac{1}{2\pi i} \int_{\gamma_\varepsilon} \Phi^\dagger (z) \Phi^{j+k} (z) z^{-1} e^{tz} dz \right]\label{derpowseries} \\
%   = \, & \frac{\partial^l}{\partial t^l} \sum_{j=0}^{+\infty} \frac{x^j}{j!}(-1)^{j+k} \left[ \frac{1}{\pi}\int_\varepsilon^{+\infty} \Im \l \Phi^\dagger \l \rho e^{i \l \pi-\frac{\theta}{2} \r} \r \rho^{-1} \phi^{j+k} \l \rho e^{i \l \pi-\frac{\theta}{2} \r} \r e^{t\rho e^{i \l \pi-\frac{\theta}{2} \r}} \r d\rho \right. \notag \\
%    & \left. - \frac{1}{2\pi i} \int_{\gamma_\varepsilon} \Phi^\dagger (z) \Phi^{j+k} (z) z^{-1} e^{tz} dz \right] \label{imint2} \\
%       = \, &   \sum_{j=0}^{+\infty} \frac{x^j}{j!}(-1)^{j+k} \left[\frac{1}{\pi} \int_\varepsilon^{+\infty} \Im \l \Phi^\dagger \l \rho e^{i \l \pi-\frac{\theta}{2} \r} \r \rho^{l-1} e^{il\l \pi-\frac{\theta}{2} \r} \phi^{j+k} \l \rho e^{i \l \pi-\frac{\theta}{2} \r} \r e^{t\rho e^{i \l \pi-\frac{\theta}{2} \r}} \r d\rho \right. \notag \\
%    & \left. - \frac{1}{2\pi i} \int_{\gamma_\varepsilon} \Phi^\dagger (z) \Phi^{j+k} (z) z^{l-1} e^{tz} dz \right] \label{dertdentro}\notag \\
%    = \, &  \sum_{j=0}^{+\infty} \frac{x^j}{j!}(-1)^{j+k}\sum_{k_1+k_2+k_3=k+j} q^{k_1} \mathfrak{b}^{k_2} \left[ \frac{1}{\pi} \int_{\varepsilon}^{+\infty} \Im  \l    \l \Phi^\dagger \l \rho e^{i \l \pi-\frac{\theta}{2} \r} \r \r^{k_3+1}\rho^{k_2+l-1} \right. \right.  \\ & \times \left. \left.  e^{i(k_2+l) \l \pi-\frac{\theta}{2} \r} e^{t\rho e^{i \l \pi-\frac{\theta}{2} \r}}  \r d\rho  -\frac{1}{2\pi i} \int_{\gamma_{\varepsilon}}   \l \Phi^\dagger(z) \r^{k_3+1} z^{k_2+l-1} e^{tz} dz \right] \label{multin}\\
% = \, &   \sum_{j=0}^{+\infty}  \frac{x^j}{j!} (-1)^{j+k} \sum_{k_1+k_2+k_3=k+j} q^{k_1}\mathfrak{b}^{k_2} \frac{d^{l+k_2+k_3}}{dt^{l+k_2+k_3}} \bar{\nu}_\Phi^{\star(k_3+1)}(t).
%\end{align}
%Now we justify, step by step, the previous computation.
%In the first step (eq \eqref{impartandintegral}), we interchanged the integral and the imaginary part. This is justified since
%\begin{align*}
%		&\left|  \frac{\Phi^\dagger \l \rho e^{i\l \pi-\frac{\theta}{2} \r} \r}{\rho}    e^{-x\Phi \l \rho e^{i \l \pi-\frac{\theta}{2} \r}  \r+t\rho e^{i \l \pi-\frac{\theta}{2} \r}}  \right| \notag \\
%		\leq \, &  e^{-qx} \left| \rho^{-1} \Phi^\dagger \l \rho e^{i \l \pi-\frac{\theta}{2} \r} \r \right| \exp \l - \rho \cos \frac{\theta}{2} \l t-\mathfrak{b}x+x\frac{\Re \Phi^\dagger \l \rho e^{i \l \pi-\frac{\theta}{2} \r} \r}{\rho \cos \frac{\theta}{2}} \r\r, 
%\end{align*}
%which is integrable at infinity as seen in the proof of Proposition \ref{prop:intreptheta} (see eq. \eqref{richiamatadopo}). In the second (eq. \eqref{serieswithintandim}) step we expanded the exponential into series and then we interchanged the series with both the integral and the imaginary part. Then the series of the integrals can be written as the series of the sum of the integrals by absolute convergence.
Indeed, we have
\begin{align}
	& \int_\varepsilon^{+\infty} \sum_{j=0}^{+\infty}\left|   (-1)^{k+j} \frac{ \Phi^\dagger\lb \rho e^{i\lb \pi-\frac{\theta}{2} \rb} \rb \lb \Phi\lb \rho e^{i\lb \pi-\frac{\theta}{2} \rb} \rb \rb^{{k+j}}}{j! \; \rho^{1-l}e^{-il\left(\pi-\frac{\theta}{2}\right)}} \; x^j \; e^{t  \rho e^{i \lb \pi-\frac{\theta}{2} \rb}} \right| d\rho  \notag \\
	= \,& \int_\varepsilon^\infty \frac{\left| \Phi^\dagger\lb \rho e^{i\lb \pi-\frac{\theta}{2} \rb} \rb \right|\left|\Phi\left(\rho e^{i\lb \pi-\frac{\theta}{2} \rb}\right)\right|^k}{\rho^{k+1}}\rho^{l+k} e^{x \left| \Phi \lb \rho e^{i \lb \pi-\frac{\theta}{2} \rb} \rb \right|-t\rho \cos \frac{\theta}{2}} d\rho \notag \\
	\leq \, & e^{xq}\int_\varepsilon^{+\infty} \frac{\left| \Phi^\dagger\lb \rho e^{i\lb \pi-\frac{\theta}{2} \rb} \rb \right|\left|\Phi\left(\rho e^{i\lb \pi-\frac{\theta}{2} \rb}\right)\right|^k}{\rho^{k+1}}\rho^{l+k} e^{x\mathfrak{b}\rho\cos\left(\frac{\theta}{2}\right)-t\rho \cos \frac{\theta}{2}+\Phi^\dagger\lb \rho e^{i \lb \pi-\frac{\theta}{2} \rb} \rb} d\rho.	\label{248}
\end{align}
It is clear that we have to check integrability in the right-hand side of \eqref{248} only in a neighbourhood of infinity. To do this, set $\delta=t-\mathfrak{b}x$ and $p,p^\prime>1$ as in the proof of Proposition \ref{prop:intreptheta}. By \eqref{eq:uniformlimcond} we know that there exists $K$ big enough such that $\frac{|\phi^\dagger\lb \rho e^{i \lb \pi-\frac{\theta}{2} \rb} \rb ||\phi\lb \rho e^{i \lb \pi-\frac{\theta}{2} \rb} \rb |^k}{\rho^{k+1}}$ is bounded and $\frac{|\phi^\dagger\lb \rho e^{i \lb \pi-\frac{\theta}{2} \rb} \rb |}{\rho \cos(\theta/2)}<\frac{\delta}{4}$, whenever $\rho >K$. Hence we get
%
%Let $\delta=t-\mathfrak{b}x>0$ and consider $p > 1$ such that $\frac{t}{p}-\mathfrak{b}x>\frac{\delta}{2}$. Let $q_* \ge 1$ such that $\frac{1}{p}+\frac{1}{q_*}=1$ and rewrite the integral in \eqref{248} as
%\begin{equation}	\label{nomodcont}
%	\int_\varepsilon^{+\infty} \frac{\Phi^\dagger \l \rho e^{i \l \pi-\frac{\theta}{2} \r} \r}{\rho} e^{-\frac{t}{q_*}\rho \cos\left(\frac{\theta}{2}\right)}\exp\left(-\rho \cos\left(\frac{\theta}{2}\right)\left(\frac{t}{p}-x\mathfrak{b}-\frac{\left|\Phi^\dagger\l \rho e^{i \l \pi-\frac{\theta}{2} \r} \r \right|}{\rho \cos\left(\frac{\theta}{2}\right)}\right)\right) d\rho.
%\end{equation}
%
%
% Hence, we can split the integral in \eqref{nomodcont} as follows
%\begin{align}
%&\int_\varepsilon^{\infty} \frac{\Phi^\dagger \l \rho e^{i \l \pi-\frac{\theta}{2} \r} \r}{\rho} e^{-\frac{t}{q_*}\rho \cos\left(\frac{\theta}{2}\right)}\exp\!\!\left(-\rho \cos\left(\frac{\theta}{2}\right)\left(\frac{t}{p}-x\mathfrak{b}-\frac{\left|\Phi^\dagger\l \rho e^{i \l \pi-\frac{\theta}{2} \r} \r \right|}{\rho \cos\left(\frac{\theta}{2}\right)}\right)\right)\!\! d\rho \notag  \\
%	&=\!\!\int_{\varepsilon}^K \frac{\Phi^\dagger \l \rho e^{i \l \pi-\frac{\theta}{2} \r} \r}{\rho} e^{-\frac{t}{q_*}\rho \cos\left(\frac{\theta}{2}\right)}\exp\!\!\left(-\rho \cos\left(\frac{\theta}{2}\right)\left(\frac{t}{p}-x\mathfrak{b}-\frac{\left|\Phi^\dagger\l \rho e^{i \l \pi-\frac{\theta}{2} \r} \r \right|}{\rho \cos\left(\frac{\theta}{2}\right)}\right)\right)\!\!d\rho \notag  \\
%		&+\!\!\int_{K}^{\infty}\!\! \frac{\Phi^\dagger \l \rho e^{i \l \pi-\frac{\theta}{2} \r} \r}{\rho} e^{-\frac{t}{q_*}\rho \cos\left(\frac{\theta}{2}\right)}\exp\!\!\left(-\rho \cos\left(\frac{\theta}{2}\right)\left(\frac{t}{p}-x\mathfrak{b}-\frac{\left|\Phi^\dagger\l \rho e^{i \l \pi-\frac{\theta}{2} \r} \r \right|}{\rho \cos\left(\frac{\theta}{2}\right)}\right)\right)\!\!d\rho.		
%		\label{splitted}
%\end{align}
%The first integral in \eqref{splitted} is clearly convergent. For the second, by \eqref{eq:uniformlimcond} and since
%\begin{equation*}
%\left| \Phi^\dagger \l \rho e^{i \l \pi-\frac{\theta}{2} \r} \r \right| / \rho < \delta/ 4
%\end{equation*}
%we have that
\begin{align}
	\begin{split}
		&\int_{K}^{+\infty}\frac{|\phi^\dagger\lb \rho e^{i \lb \pi-\frac{\theta}{2} \rb} \rb ||\phi\lb \rho e^{i \lb \pi-\frac{\theta}{2} \rb} \rb |^k}{\rho^{k+1}} e^{-\frac{t}{p^\prime}\rho \cos\left(\frac{\theta}{2}\right)}\exp\left(-\rho \cos\left(\frac{\theta}{2}\right)\left(\frac{t}{p}-x\mathfrak{b}-\frac{\left|\Phi^\dagger\lb \rho e^{i \lb \pi-\frac{\theta}{2} \rb} \rb \right|}{\rho \cos\left(\frac{\theta}{2}\right)}\right)\right) d\rho \notag \\
		\leq \,& C \int_K^{+\infty} \rho^{k+l}e^{-\frac{t}{p^\prime}\rho \cos \lb \frac{\theta}{2} \rb} < +\infty.
	\end{split}
\end{align}
This concludes the proof.
%Since $\lim_{\rho \to +\infty}\frac{\Phi^\dagger(\rho)}{\rho}=0$, we know that
%\begin{align}
%	&\int_\varepsilon^{+\infty} \frac{\Phi^\dagger(\rho)}{\rho} e^{-\frac{t}{q}\rho \cos\left(\frac{\theta}{2}\right)}\exp\left(-\rho \cos\left(\frac{\theta}{2}\right)\left(\frac{t}{p}-x\mathfrak{b}+C\frac{\Phi^\dagger(\rho)}{\rho \cos\left(\frac{\theta}{2}\right)}\right)\right) d\rho \notag \\
%	\le & C\int_\varepsilon^{+\infty} e^{-\frac{t}{q}\rho \cos\left(\frac{\theta}{2}\right)} d\rho<+\infty.
%\end{align}
%The absolute convergence of the series of integrals over $\gamma_\varepsilon$ can be dealt equivalently. Indeed
%\begin{equation}\label{568}
%\int_{\gamma_{\varepsilon}} \sum_{j=0}^{+\infty}  \left|(-1)^j \frac{\Phi^\dagger(z)}{z} \frac{\Phi^j(z)}{j!} x^j e^{tz} \right| \, dz\, = \, \int_{\gamma_{\varepsilon}} \frac{\phi^\dagger (z)}{z} e^{zt+x\Phi(z)} dz < +\infty
%\end{equation}
%since $z^{-1}\Phi^\dagger(z) \exp (zt+\Phi(z)x)$ is continuous on $\overline{\mathbb{C}\l \pi-(\theta/2) \r}$.
%In (eq. \eqref{derpowseries}) we derived $k$-times with respect to $x$ using the well-known formula for the derivatives of analytic functions (see, e.g., \cite[Proposition 15.2.6]{tao2015}).
%In the next step, \eqref{imint2} we moved the imaginary part inside the integral. This is justified since, for $\rho > \varepsilon$, using \eqref{eq:uniformlimcond},
%\begin{align*}
%			\left| \Im \l  \Phi^\dagger \l \rho e^{i \l \pi-\frac{\theta}{2} \r} \r \rho^{-1} \phi^{j+k} \l \rho e^{i \l \pi-\frac{\theta}{2} \r} \r e^{t\rho e^{i \l \pi-\frac{\theta}{2} \r}} \r \right| \,
%		\leq \, & \left|  \Phi^\dagger \l \rho e^{i \l \pi-\frac{\theta}{2} \r} \r \rho^{-1} \phi^{j+k} \l \rho e^{i \l \pi-\frac{\theta}{2} \r} \r e^{t\rho e^{i \l \pi-\frac{\theta}{2} \r}} \right| \notag \\
%		\leq \, & C \left| \Phi^{j+k} \l \rho \l e^{i \l \pi-\frac{\theta}{2} \r} \r \r \right| e^{-\rho t \cos\frac{\theta}{2}} < +\infty.
%	\end{align*}
%In \eqref{dertdentro} we moved the derivatives with respect to $t$ inside the series, the integrals and the imaginary part. In particular since, 
%\begin{equation}\label{derdentroim}
%	\begin{split}
%		&\frac{\partial^l}{\partial t^l}\Im \l \Phi^\dagger \l \rho e^{i \l \pi-\frac{\theta}{2} \r} \r \rho^{-1} \phi^{j+k} \l \rho e^{i \l \pi-\frac{\theta}{2} \r} \r e^{t\rho e^{i \l \pi-\frac{\theta}{2} \r}} \r \notag \\= \,& \Im \l \frac{\partial^l}{\partial t^l} \Phi^\dagger \l \rho e^{i \l \pi-\frac{\theta}{2} \r} \r \rho^{-1} \phi^{j+k} \l \rho e^{i \l \pi-\frac{\theta}{2} \r} \r e^{t\rho e^{i \l \pi-\frac{\theta}{2} \r}} \r.
%	\end{split}
%\end{equation}
%it is sufficient to check that 
%\begin{equation}\label{perdert}
%\sum_{j=0}^{+\infty } \left| \frac{x^j}{j!} \int_{\varepsilon}^{+\infty}  \Im \l \frac{\partial^l}{\partial t^l} \Phi^\dagger \l \rho e^{i \l \pi-\frac{\theta}{2} \r} \r \rho^{-1} \phi^{j+k} \l \rho e^{i \l \pi-\frac{\theta}{2} \r} \r e^{t\rho e^{i \l \pi-\frac{\theta}{2} \r}} \r \, d\rho \right| < +\infty
%\end{equation}
%and that 
%\begin{equation}\label{this}
%\sum_{j=0}^{+\infty} \left| \frac{x^j}{j!} \int_{\gamma_{\varepsilon}} \frac{\partial^l}{\partial t^l}  \frac{\phi^\dagger(z)}{z} \Phi(z)^{j+k}e^{tz} dz \right| < +\infty.
%\end{equation}
%Eq \eqref{this} follows as in \eqref{568}. We check \eqref{perdert}.
%Let $l \ge 1$, $t \in [t_1,t_2]$, $x \in [x_1,x_2]$ where $t_1,x_1>0$ and $x_2<t_1/\mathfrak{b}$. Observe that
% \begin{align}
% 			&\left|\frac{x^j}{j!} \frac{\Phi^\dagger \l \rho e^{i\l \pi -\frac{\theta}{2}\r} \r \Phi^{k+j}\l \rho e^{i\l \pi -\frac{\theta}{2}\r} \r}{\rho^{1-l}} e^{il \l \pi-\frac{\theta}{2} \r}e^{t\rho e^{i\l \pi - \frac{\theta}{2} \r}}  \right| \notag  \\
% 		\le \, & \frac{x_2^j}{j!}\left|\Phi^\dagger \l \rho e^{i\l \pi -\frac{\theta}{2}\r} \r\right| \left| \Phi\l \rho e^{i\l \pi -\frac{\theta}{2}\r} \r\right|^{j+k}\rho^{l-1} e^{-t_1\rho\cos\left(\frac{\theta}{2}\right)} \notag \\
% 		= \, & \frac{x_2^j}{j!}\frac{\left|\Phi^\dagger \l \rho e^{i\l \pi -\frac{\theta}{2}\r} \r\right|}{\rho} \frac{\left| \Phi\l \rho e^{i\l \pi -\frac{\theta}{2}\r} \r\right|^{k}}{\rho^k} \left| \Phi\l \rho e^{i\l \pi -\frac{\theta}{2}\r} \r\right|^{j}\rho^{l+k} e^{-t_1\rho\cos\left(\frac{\theta}{2}\right)} \notag \\
% 		\leq \, &  \frac{x_2^j}{j!}\frac{\left|\Phi^\dagger \l \rho e^{i\l \pi -\frac{\theta}{2}\r} \r\right|}{\rho} \frac{\left| \Phi^\dagger \l \rho e^{i\l \pi -\frac{\theta}{2}\r} \r\right|^{k} + \mathfrak{b}^k\rho^k + q^k }{\rho^k}  \left| \Phi\l \rho e^{i\l \pi -\frac{\theta}{2}\r} \r\right|^{j}\rho^{l+1} e^{-t_1\rho\cos\left(\frac{\theta}{2}\right)}. \label{eq:changeder}
% \end{align}
%Now we prove that 
%\begin{align}\label{eq:I3}
%	I_1 \,	= \, &\sum_{j=0}^{+\infty}\int_\varepsilon^{+\infty} \frac{x_2^j}{j!}\frac{\left|\Phi^\dagger \l \rho e^{i\l \pi -\frac{\theta}{2}\r} \r\right|}{\rho}\frac{\left| \Phi^\dagger \l \rho e^{i\l \pi -\frac{\theta}{2}\r} \r\right|^{k} + \mathfrak{b}^k\rho^k + q^k }{\rho^k}  \left| \Phi\l \rho e^{i\l \pi -\frac{\theta}{2}\r} \r\right|^{j}\rho^{l+1} e^{-t_1\rho\cos\left(\frac{\theta}{2}\right)}d\rho \notag \\
%	< \, & +\infty.
%\end{align}
%To do this, let $\delta=t-\mathfrak{b}x>0$ and consider $p>1$ such that $\frac{t}{p}-\mathfrak{b}x>\frac{\delta}{2}$. Let also $q \ge 1$ such that $\frac{1}{p}+\frac{1}{q_*}=1$ and note that, using \eqref{eq:uniformlimcond} 
%\begin{align*}
%			I_1\leq \, & C \sum_{j=0}^{+\infty}\int_\varepsilon^{+\infty} \frac{x_2^j}{j!} \frac{\left| \Phi^\dagger \l \rho e^{i\l \pi -\frac{\theta}{2}\r} \r\right|^{k} + \mathfrak{b}^k\rho^k + q^k }{\rho^k}   \left| \Phi\l \rho e^{i\l \pi -\frac{\theta}{2}\r} \r\right|^{j}\rho^{l+k} e^{-t_1\rho\cos\left(\frac{\theta}{2}\right)}d\rho\\
%		= \, &C \int_\varepsilon^{+\infty}\frac{\left| \Phi^\dagger \l \rho e^{i\l \pi -\frac{\theta}{2}\r} \r\right|^{k} + \mathfrak{b}^k\rho^k + q^k }{\rho^k} e^{-\frac{t_1}{q_*}\rho \cos\left(\frac{\theta}{2}\right)}\\
%		& \times  \rho^{l+k} \exp\left(x_2\left| \Phi\l \rho e^{i\l \pi -\frac{\theta}{2}\r} \r\right|-\frac{t_1}{p}\rho\cos\left(\frac{\theta}{2}\right)\right)d\rho\\
%		\le \, & C e^{x_2q}\int_\varepsilon^{+\infty}\frac{\left| \Phi^\dagger \l \rho e^{i\l \pi -\frac{\theta}{2}\r} \r\right|^{k} + \mathfrak{b}^k\rho^k + q^k }{\rho^k} e^{-\frac{t_1}{q_*}\rho \cos\left(\frac{\theta}{2}\right)}\\
%		& \times \rho^{l+k} \exp\left(\mathfrak{b}x_2+x_2\left|\Phi^\dagger\l \rho e^{i \l \pi-\frac{\theta}{2} \r}\r \right|-\frac{t_1}{p}\rho\cos\left(\frac{\theta}{2}\right)\right)d\rho\\
%		 \leq \, & C e^{x_2q}\int_\varepsilon^{+\infty}e^{-\frac{t_1}{q_*}\rho \cos\left(\frac{\theta}{2}\right)} \rho^{l+k} \exp\left(\mathfrak{b}x_2+x_2\left|\Phi^\dagger\l \rho e^{i \l \pi-\frac{\theta}{2} \r}\r \right|-\frac{t_1}{p}\rho\cos\left(\frac{\theta}{2}\right)\right)d\rho\\
%		& + C \mathfrak{b}^ke^{x_2q}\int_\varepsilon^{+\infty}e^{-\frac{t_1}{q_*}\rho \cos\left(\frac{\theta}{2}\right)} \rho^{l+k} \exp\left(\mathfrak{b}x_2+x_2\left|\Phi^\dagger\l \rho e^{i \l \pi-\frac{\theta}{2} \r}\r \right|-\frac{t_1}{p}\rho\cos\left(\frac{\theta}{2}\right)\right)d\rho \notag \\
%		 \leq \, &C \int_\varepsilon^{+\infty}e^{-\frac{t_1}{q_*}\rho \cos\left(\frac{\theta}{2}\right)} \rho^{l+k}  \exp\left(-\rho \cos\left(\frac{\theta}{2}\right)\left(\frac{\delta}{2}+x_2\frac{\left|\Phi^\dagger\l \rho e^{i \l \pi-\frac{\theta}{2} \r }\r \right|}{\rho \cos\left(\frac{\theta}{2}\right)}\right)\right)d\rho\\
%		& + C \int_\varepsilon^{+\infty}e^{-\frac{t_1}{q_*}\rho \cos\left(\frac{\theta}{2}\right)} \rho^{l+k}  \exp\left(-\rho \cos\left(\frac{\theta}{2}\right)\left(\frac{\delta}{2}+x_2\frac{\left|\Phi^\dagger\l \rho e^{i \l \pi-\frac{\theta}{2} \r }\r \right|}{\rho \cos\left(\frac{\theta}{2}\right)}\right)\right)d\rho
%\end{align*}
%Now observe that $\rho^{l}e^{-\frac{t_1}{q}\rho \cos\left(\frac{\theta}{2}\right)}$ is bounded as $\rho \ge 1$. Furthermore, by \eqref{eq:uniformlimcond} $\abs{\Phi^\dagger\l \rho e^{i \l \pi-\frac{\theta}{2} \r} \r}\rho^{-1}$ is bounded and also $\exp\left(-\rho \cos\left(\frac{\theta}{2}\right)\left(\frac{\delta}{2}+Cx_2\frac{\Phi^\dagger\l \rho\r}{\rho \cos\left(\frac{\theta}{2}\right)}\right)\right)$, for $\rho > K$ big enough is bounded. Hence, arguing as in \eqref{splitted}, we get 
%\begin{align*}
%			I_1 \le  & \l \int_{\varepsilon}^K+ \int_K^{+\infty} \r \bigg[ C e^{-\frac{t_1}{q_*}\rho \cos\left(\frac{\theta}{2}\right)} \rho^{l+k}  \notag \\
%			&\times  \exp\left(-\rho \cos\left(\frac{\theta}{2}\right)\left(\frac{\delta}{2}+x_2\frac{\left|\Phi^\dagger\l \rho e^{i \l \pi-\frac{\theta}{2} \r }\r \right|}{\rho \cos\left(\frac{\theta}{2}\right)}\right)\right) + C e^{-\frac{t_1}{q_*}\rho \cos\left(\frac{\theta}{2}\right)} \rho^{l+k} \notag \\
%			& \times \exp\left(-\rho \cos\left(\frac{\theta}{2}\right)\left(\frac{\delta}{2}+x_2\frac{\left|\Phi^\dagger\l \rho e^{i \l \pi-\frac{\theta}{2} \r }\r \right|}{\rho \cos\left(\frac{\theta}{2}\right)}\right)\right) \bigg]d\rho \, < \, +\infty.
%\end{align*}
%In the second last step, i.e., eq. \eqref{multin}, we used the multinomial theorem. In the very last step we used \eqref{derivatecode}.
%We stress, in particular, that we actually proved, by Weierstrass $M$-test, that \eqref{eq:series} and \eqref{eq:seriesder} are uniformly convergent with respect to $x \in [x_1,x_2]$ and $t \in [t_1,t_2]$ such that $t_1,x_1>0$ and $x_2<\frac{t_1}{\mathfrak{b}}$. This guarantees the continuity of the involved series.
%%
%%*************
%%
%%We use \eqref{diffint}, we differentiate term by term and we expand into power series to get the results, as follows (all the steps will be justified at the end of the computation.
%%\begin{align}
%%&\frac{\partial^l}{\partial t^l} \frac{\partial^k}{\partial x^k} f_\Phi (x, t) \notag \\ = \,& \frac{1}{\pi} \int_\varepsilon^{+\infty}  \Im \l (-1)^k \l \Phi \l \rho e^{i \l \pi-\frac{\theta}{2} \r} \r \r^k \rho^l e^{il\l \pi-\frac{\theta}{2} \r} \frac{\Phi^\dagger \l \rho e^{i \l \pi-\frac{\theta}{2} \r} \r}{\rho} e^{-x\Phi \l \rho e^{i \l \pi-\frac{\theta}{2} \r} \r+t\rho e^{i \l \pi-\frac{\theta}{2} \r}} \r  d\rho  \\
%%& - \frac{1}{2\pi i} \int_{\gamma_\varepsilon} (-1)^k \l \Phi(\rho) \r^k \rho^{l-1} \Phi^\dagger (\rho) e^{-x\Phi(\rho)+t\rho}d\rho  \\
%%= \, & \sum_{j=0}^{+\infty} (-1)^j \frac{x^j}{j!} (-1)^k\left[ \frac{1}{\pi} \int_{\varepsilon}^{+\infty} \Im \l \l \Phi \l \rho e^{i \l \pi-\frac{\theta}{2} \r} \r \r^{k+j}  \rho^{l-1} e^{il \l \pi-\frac{\theta}{2} \r} \Phi^\dagger \l \rho e^{i \l \pi-\frac{\theta}{2} \r} \r e^{t\rho e^{i \l \pi-\frac{\theta}{2} \r}}  \r d\rho \right.  \\
%%& \left.-\frac{1}{2\pi i} \int_{\gamma_{\varepsilon}}  \l \Phi(\rho) \r^{k+j} \rho^{l-1} \Phi^\dagger (\rho)e^{t\rho} d\rho \right]  \\
%%= \, & \sum_{j=0}^{+\infty} (-1)^j \frac{x^j}{j!} (-1)^k \sum_{k_1+k_2+k_3=k+j} \left[ \frac{1}{\pi} \int_{\varepsilon}^{+\infty} \Im  \l q^{k_1} \mathfrak{b}^{k_2} \rho^{k_2} e^{i k_2 \l \pi-\frac{\theta}{2} \r}  \l \Phi^\dagger \l \rho e^{i \l \pi-\frac{\theta}{2} \r} \r \r^{k_3+1} \right. \right.  \\ & \times \left. \left. \rho^{l-1} e^{il \l \pi-\frac{\theta}{2} \r} e^{t\rho e^{i \l \pi-\frac{\theta}{2} \r}}  \r d\rho  -\frac{1}{2\pi i} \int_{\gamma_{\varepsilon}} q^{k_1} \mathfrak{b}^{k_2}  \l \Phi^\dagger(\rho) \r^{k_3+1} \rho^{l-1} e^{t\rho} d\rho \right]  \\
%%= \, & \sum_{j=0}^{+\infty} (-1)^j \frac{x^j}{j!} (-1)^k \sum_{k_1+k_2+k_3=k+j} q^{k_1} \mathfrak{b}^{k_2} \left[ \frac{1}{\pi} \int_{\varepsilon}^{+\infty} \Im  \l    \l \Phi^\dagger \l \rho e^{i \l \pi-\frac{\theta}{2} \r} \r \r^{k_3+1}\rho^{k_2+l-1} \right. \right.  \\ & \times \left. \left.  e^{i(k_2+l) \l \pi-\frac{\theta}{2} \r} e^{t\rho e^{i \l \pi-\frac{\theta}{2} \r}}  \r d\rho  -\frac{1}{2\pi i} \int_{\gamma_{\varepsilon}}   \l \Phi^\dagger(\rho) \r^{k_3+1} \rho^{k_2+l-1} e^{t\rho} d\rho \right] \notag \\
%%= \, & \sum_{j=0}^{+\infty} (-1)^j \frac{x^j}{j!} (-1)^k \sum_{k_1+k_2+k_3=k+j} q^{k_1}\mathfrak{b}^{k_2} \frac{d^{l+k_2+k_3}}{dt^{l+k_2+k_3}} \bar{\nu}_\Phi^{\star(k_3+1)}(t)
%%\end{align}
%%Now we justify the above computation step by step 
%%
%%********************
%%
%%
%%
%%*********** HERE IS OLD COMPUTATION TO BE USED  ***********
%%
%%Let us first observe that, since both Assumptions \eqref{eq:modcont} and \eqref{eq:integrlog} hold, by Proposition \ref{prop:integr0} we know that
%%\begin{equation}\label{eq:finiteint}
%%	\int_0^1 \frac{\left|\Phi^\dagger\left(\rho e^{i\left(\pi-\frac{\theta}{2}\right)}\right)\right|}{\rho}d\rho \le C\int_0^1 \frac{\Phi(\rho)}{\rho}d\rho<+\infty,
%%\end{equation}
%%hence, \eqref{eq:intcondtheta} holds. Thus, we can apply Proposition \ref{prop:intreptheta} to get \eqref{eq:intreptheta}. Now observe that
%%\begin{align*}
%%	&\int_0^{+\infty}\left|\frac{\Phi^\dagger\left(\rho e^{i\left(\pi-\frac{\theta}{2}\right)}\right)}{\rho} e^{t\rho e^{i \l \pi-\frac{\theta}{2} \r}-x\Phi\l \rho e^{i \l \pi-\frac{\theta}{2} \r} \r}\right|d\rho\\
%%	&=e^{-qx}\int_0^{+\infty}\frac{\left|\Phi^\dagger\left(\rho e^{i\left(\pi-\frac{\theta}{2}\right)}\right)\right|}{\rho} \exp\left(-\rho\cos\left(\frac{\theta}{2}\right)\left(t-\mathfrak{b}x+x\frac{\Re\Phi^\dagger\l \rho e^{i \l \pi-\frac{\theta}{2} \r}\r}{\rho \cos\left(\frac{\theta}{2}\right)} \right)\right)d\rho.
%%\end{align*}
%%We already know, from the proof of Proposition \ref{prop:intreptheta}, that
%%\begin{equation*}
%%	\int_1^{+\infty}\frac{\left|\Phi^\dagger\left(\rho e^{i\left(\pi-\frac{\theta}{2}\right)}\right)\right|}{\rho} \exp\left(-\rho\cos\left(\frac{\theta}{2}\right)\left(t-\mathfrak{b}x+x\frac{\Re\Phi^\dagger\l \rho e^{i \l \pi-\frac{\theta}{2} \r}\r}{\rho \cos\left(\frac{\theta}{2}\right)} \right)\right)d\rho<+\infty.
%%\end{equation*}
%%On the other hand, it clearly holds, by \eqref{eq:finiteint},
%%\begin{align*}
%%	&\int_0^1\frac{\left|\Phi^\dagger\left(\rho e^{i\left(\pi-\frac{\theta}{2}\right)}\right)\right|}{\rho} \exp\left(-\rho\cos\left(\frac{\theta}{2}\right)\left(t-\mathfrak{b}x+x\frac{\Re\Phi^\dagger\l \rho e^{i \l \pi-\frac{\theta}{2} \r}\r}{\rho \cos\left(\frac{\theta}{2}\right)} \right)\right)d\rho \\
%%	&\le C \int_0^1 \frac{\Phi^\dagger(\rho)}{\rho}d\rho<+\infty.
%%\end{align*}
%%Hence, we can change the order of the integral and the imaginary part in \eqref{eq:intreptheta} to get
%%\begin{align}\label{eq:intrepthetachange}
%%	f_\Phi(x,t) \, &= \, 	\frac{1}{\pi}\Im \l\int_0^{+\infty} \frac{\Phi^\dagger\left(\rho e^{i\left(\pi-\frac{\theta}{2}\right)}\right)}{\rho} e^{t\rho e^{i \l \pi-\frac{\theta}{2} \r}-x\Phi\l \rho e^{i \l \pi-\frac{\theta}{2} \r} \r}  \, d\rho\r\\
%%	&=\frac{1}{\pi}\Im \l\int_0^{+\infty} \sum_{k=0}^{+\infty}(-1)^k\frac{x^k}{k!}\frac{\Phi^\dagger\left(\rho e^{i\left(\pi-\frac{\theta}{2}\right)}\right)\left(\Phi\left(\rho e^{i\left(\pi-\frac{\theta}{2}\right)}\right) \right)^k}{\rho} e^{t\rho e^{i \l \pi-\frac{\theta}{2} \r}}   \, d\rho\r.
%%\end{align}
%%Now we prove that we can move the sum outside both the integral and the imaginary part. Indeed, we have that
%%\begin{align}
%%	& \int_0^{+\infty} \sum_{k=0}^{+\infty}\left|   (-1)^k \frac{ \Phi^\dagger\l \rho e^{i\l \pi-\frac{\theta}{2} \r} \r \l \Phi\l \rho e^{i\l \pi-\frac{\theta}{2} \r} \r \r^{k}}{k! \; \rho} \; x^k \; e^{t  \rho e^{i \l \pi-\frac{\theta}{2} \r}} \right| d\rho  \notag \\
%%	= \, &\int_0^{+\infty} \sum_{k=0}^{+\infty}  \frac{\left|\Phi^\dagger\l \rho e^{i\l \pi-\frac{\theta}{2} \r} \r\right|\left| \Phi\l \rho e^{i\l \pi-\frac{\theta}{2} \r} \r \right|^{k}}{k! \; \rho} \; x^k \; e^{-t  \rho \cos \frac{\theta}{2}} d\rho \notag \\ = \,& \int_0^\infty \frac{\left| \Phi^\dagger\l \rho e^{i\l \pi-\frac{\theta}{2} \r} \r \right|}{ \rho} e^{x \left| \Phi \l \rho e^{i \l \pi-\frac{\theta}{2} \r} \r \right|-t\rho \cos \frac{\theta}{2}} d\rho \notag \\
%%	\leq \, & Ce^{xq}\int_0^{+\infty} \frac{\Phi^\dagger(\rho)}{\rho} e^{x\mathfrak{b}\rho\cos\left(\frac{\theta}{2}\right)-t\rho \cos \frac{\theta}{2}+C\Phi^\dagger(\rho)} d\rho\\
%%	= \, &Ce^{xq}\left(\int_0^{1} \frac{\Phi^\dagger(\rho)}{\rho} e^{x\mathfrak{b}\rho\cos\left(\frac{\theta}{2}\right)-t\rho \cos \frac{\theta}{2}+C\Phi^\dagger(\rho)} d\rho+\int_1^{+\infty} \frac{\Phi^\dagger(\rho)}{\rho} e^{x\mathfrak{b}\rho\cos\left(\frac{\theta}{2}\right)-t\rho \cos \frac{\theta}{2}+C\Phi^\dagger(\rho)} d\rho\right)\\
%%	=: \, &Ce^{xq}(I_1+I_2).
%%	\label{248}
%%\end{align}
%%To handle $I_1$, just observe that
%%\begin{equation*}
%%	I_1 \le C \int_0^1 \frac{\Phi^\dagger(\rho)}{\rho}d\rho<+\infty.
%%\end{equation*}
%%On the other hand, concerning $I_2$, let $\delta=t-\mathfrak{b}x>0$ and consider $p > 1$ such that $\frac{t}{p}-\mathfrak{b}x>\frac{\delta}{2}$. Let $q \ge 1$ such that $\frac{1}{p}+\frac{1}{q}=1$ and rewrite $I_2$ as
%%\begin{equation*}
%%	I_2=\int_1^{+\infty} \frac{\Phi^\dagger(\rho)}{\rho} e^{-\frac{t}{q}\rho \cos\left(\frac{\theta}{2}\right)}\exp\left(-\rho \cos\left(\frac{\theta}{2}\right)\left(\frac{t}{p}-x\mathfrak{b}+C\frac{\Phi^\dagger(\rho)}{\rho \cos\left(\frac{\theta}{2}\right)}\right)\right) d\rho.
%%\end{equation*}
%%Recalling that $\lim_{\rho \to +\infty}\frac{\Phi^\dagger(\rho)}{\rho}=0$, we know that
%%\begin{equation*}
%%	I_2 \le C\int_1^{+\infty} e^{-\frac{t}{q}\rho \cos\left(\frac{\theta}{2}\right)} d\rho<+\infty.
%%\end{equation*}
%% This ends the proof of \eqref{eq:series}.\\
%% To show that \eqref{eq:seriesder} holds, we only need to prove the relation for $k=0$, since once we have \eqref{eq:seriesder} for any value of $l \ge 0$, we can derive \eqref{eq:seriesder} for $k>0$ by diffentiating term by term. Let us prove that $\frac{\partial^l}{\partial t^l}f_\Phi(x,t)$ is well-defined and \eqref{eq:seriesder}. To do this, let $l \ge 1$, $t \in [t_1,t_2]$, $x \in [x_1,x_2]$ where $t_1,x_1>0$ and $x_2<t_1/\mathfrak{b}$. Observe that
%% \begin{align}\label{eq:changeder}
%% 	&\left|\frac{(-1)^jx^j}{j!} \frac{\Phi^\dagger \l \rho e^{i\l \pi -\frac{\theta}{2}\r} \r\l \Phi\l \rho e^{i\l \pi -\frac{\theta}{2}\r} \r\r^{j}}{\rho^{1-l}} e^{t\rho e^{i\l \pi - \frac{\theta}{2} \r}}  \right| \\
%% 	&\qquad \le \frac{x_2^j}{j!}\left|\Phi^\dagger \l \rho e^{i\l \pi -\frac{\theta}{2}\r} \r\right| \left| \Phi\l \rho e^{i\l \pi -\frac{\theta}{2}\r} \r\right|^{j}\rho^{l-1} e^{-t_1\rho\cos\left(\frac{\theta}{2}\right)}.
%% \end{align}
%%Now we prove that
%%\begin{equation}\label{eq:I3}
%%	I_3=\sum_{j=0}^{+\infty}\int_0^{+\infty}\frac{x_2^j}{j!}\left|\Phi^\dagger \l \rho e^{i\l \pi -\frac{\theta}{2}\r} \r\right| \left| \Phi\l \rho e^{i\l \pi -\frac{\theta}{2}\r} \r\right|^{j}\rho^{l-1} e^{-t_1\rho\cos\left(\frac{\theta}{2}\right)}d\rho <+\infty.
%%\end{equation}
%%To do this, rewrite $I_3$ as
%%\begin{align*}
%%	I_3&=\int_0^{+\infty}\left|\Phi^\dagger \l \rho e^{i\l \pi -\frac{\theta}{2}\r} \r\right| \rho^{l-1} e^{x_2\left| \Phi\l \rho e^{i\l \pi -\frac{\theta}{2}\r} \r\right|-t_1\rho\cos\left(\frac{\theta}{2}\right)}d\rho\\
%%	&=\int_0^{1}\left|\Phi^\dagger \l \rho e^{i\l \pi -\frac{\theta}{2}\r} \r\right| \rho^{l-1} e^{x_2\left| \Phi\l \rho e^{i\l \pi -\frac{\theta}{2}\r} \r\right|-t_1\rho\cos\left(\frac{\theta}{2}\right)}d\rho\\
%%	&+\int_1^{+\infty}\left|\Phi^\dagger \l \rho e^{i\l \pi -\frac{\theta}{2}\r} \r\right| \rho^{l-1} e^{x_2\left| \Phi\l \rho e^{i\l \pi -\frac{\theta}{2}\r} \r\right|-t_1\rho\cos\left(\frac{\theta}{2}\right)}d\rho\\
%%	&=:I_4+I_5.
%%\end{align*}
%%In order to handle $I_4$, just observe that all the involved functions are continuous in $[0,1]$, since $l \ge 1$, hence $I_4<+\infty$. Concerning $I_5$, let $\delta=t-\mathfrak{b}x>0$ and consider $p>1$ such that $\frac{t}{p}-\mathfrak{b}x>\frac{\delta}{2}$. Let also $q \ge 1$ such that $\frac{1}{p}+\frac{1}{q}=1$ and rewrite $I_5$ as
%%\begin{align*}
%%	I_5&=\int_1^{+\infty}\frac{\left|\Phi^\dagger \l \rho e^{i\l \pi -\frac{\theta}{2}\r} \r\right| }{\rho}e^{-\frac{2t_1}{q}\rho \cos\left(\frac{\theta}{2}\right)}\\
%%	&\qquad \times \rho^{l}e^{-\frac{2t_2}{q}\rho \cos\left(\frac{\theta}{2}\right)} \exp\left(x_2\left| \Phi\l \rho e^{i\l \pi -\frac{\theta}{2}\r} \r\right|-\frac{t_1}{p}\rho\cos\left(\frac{\theta}{2}\right)\right)d\rho\\
%%	&\le e^{x_2q}\int_1^{+\infty}\frac{\left|\Phi^\dagger \l \rho e^{i\l \pi -\frac{\theta}{2}\r} \r\right|}{\rho}e^{-\frac{2t_1}{q}\rho \cos\left(\frac{\theta}{2}\right)}\\
%%	&\qquad \times \rho^{l}e^{-\frac{2t_1}{q}\rho \cos\left(\frac{\theta}{2}\right)} \exp\left(\mathfrak{b}x_2+Cx_2\Phi^\dagger\l \rho\r-\frac{t_1}{p}\rho\cos\left(\frac{\theta}{2}\right)\right)d\rho\\
%%	&\le Ce^{x_2q}\int_1^{+\infty}\frac{\Phi^\dagger \l \rho\r}{\rho}e^{-\frac{2t_1}{q}\rho \cos\left(\frac{\theta}{2}\right)}\\
%%	&\qquad \times \rho^{l}e^{-\frac{2t_1}{q}\rho \cos\left(\frac{\theta}{2}\right)} \exp\left(-\rho \cos\left(\frac{\theta}{2}\right)\left(\frac{\delta}{2}+Cx_2\frac{\Phi^\dagger\l \rho\r}{\rho \cos\left(\frac{\theta}{2}\right)}\right)\right)d\rho.
%%\end{align*}
%%Now observe that $\rho^{l}e^{-\frac{2t_1}{q}\rho \cos\left(\frac{\theta}{2}\right)}$ is bounded as $\rho \ge 1$. Furthermore, since $\lim_{\rho \to 0}\frac{\Phi^\dagger\l \rho\r}{\rho}=0$, also $\exp\left(-\rho \cos\left(\frac{\theta}{2}\right)\left(\frac{\delta}{2}+Cx_2\frac{\Phi^\dagger\l \rho\r}{\rho \cos\left(\frac{\theta}{2}\right)}\right)\right)$ and $\frac{\Phi^\dagger(\rho)}{\rho}$ are bounded as $\rho \ge 1$. Hence we get
%%\begin{equation*}
%%	I_5 \le C \int_1^{+\infty}e^{-\frac{2t_1}{q}\rho \cos\left(\frac{\theta}{2}\right)}d\rho <+\infty.
%%\end{equation*} 
%%This proves \eqref{eq:I3}. Hence we can take the derivative inside the summation sign and the integral in \eqref{eq:series} to get \eqref{eq:seriesder}.
%%Let us stress, in particular, that we actually proved, by Weierstrass $M$-test, that \eqref{eq:series} and \eqref{eq:seriesder} are uniformly convergent with respect to $x \in [x_1,x_2]$ and $t \in [t_1,t_2]$ such that $t_1,x_1>0$ and $x_2<\frac{t_1}{\mathfrak{b}}$. This guarantees the continuity of the involved series.
%%%
%%%Now we prove continuity in two variables. Fix $\theta \in (0,\pi)$, use again the series representation to say that, for $x,t>0$,
%%%	\begin{align}\label{eq:immpartder}
%%%		&\frac{\partial^l}{\partial t^l}\frac{\partial^k}{\partial x^k}f_\Phi(x,t)\\
%%%		&=\Im\left(\int_0^{\infty}\sum_{k=0}^{\infty}\frac{(-1)^{j}}{\pi}\frac{x^j}{j!}\frac{\phi^\dagger\l \rho e^{i \l \pi-\frac{\theta}{2} \r}  \r\left(\Phi\left(\rho e^{i\left(\pi-\frac{\theta}{2}\right)}\right)\right)^{k+j}}{\rho^{1-l}}e^{t\rho e^{i\left(\pi-\frac{\theta}{2}\right)}}d\rho\right).
%%%		\label{gcont}
%%%	\end{align}
%%%		Hence we prove that the argument of the imaginary part above is continuous in any $x,t>0$. Fix $x,t>0$ and consider $h_1,h_2 \in \R$ such that $|h_1|<\frac{x}{2}$ and $|h_2|<\frac{t}{2}$. Denote
%%%		\begin{align}
%%%		g_{j,\phi}(x,t) \, = \, \frac{(-1)^{j}}{\pi}\frac{x^j}{j!}\frac{\phi^\dagger\l \rho e^{i \l \pi-\frac{\theta}{2} \r}  \r\left(\Phi\left(\rho e^{i\left(\pi-\frac{\theta}{2}\right)}\right)\right)^{k+j}}{\rho^{1-l}}e^{t\rho e^{i\left(\pi-\frac{\theta}{2}\right)}}
%%%		\label{gjphi}
%%%		\end{align}
%%%		and	consider $g_{j,\Phi}(x+h_1,t+h_2)$ and note that
%%%	\begin{align}
%%%		&\left|\frac{(-1)^{j}}{\pi}\frac{(x+h_1)^j}{j!}\frac{\phi^\dagger\l \rho e^{i \l \pi-\frac{\theta}{2} \r}  \r\left(\Phi\left(\rho e^{i\left(\pi-\frac{\theta}{2}\right)}\right)\right)^{k+j}}{\rho^{1-l}}e^{(t+h_2)\rho e^{i\left(\pi-\frac{\theta}{2}\right)}}\right| \notag\\
%%%		& \qquad =\frac{1}{\pi}\frac{|x+h_1|^j}{j!}\frac{\left| \Phi^\dagger \l \rho e^{i \l \pi-\frac{\theta}{2} \r} \r \right|\left|\Phi\left(\rho e^{i\left(\pi-\frac{\theta}{2}\right)}\right)\right|^{k+j}}{\rho^{1-l}}e^{-(t+h_2)\rho \cos\frac{\theta}{2}} \notag\\
%%%		&\qquad \le \frac{1}{\pi}\frac{(2x)^j}{j!}\frac{\left| \Phi^\dagger \l \rho e^{i \l \pi-\frac{\theta}{2} \r} \r \right|\left|\Phi\left(\rho e^{i\left(\pi-\frac{\theta}{2}\right)}\right)\right|^{k+j}}{\rho^{1-l}}e^{-\frac{t}{2}\rho \cos\frac{\theta}{2}}.\label{eq:upbound}
%%%	\end{align}
%%%	We now check that
%%%	\begin{align}\label{eq:precheck}
%%%		\int_0^{\infty}\sum_{k=0}^{\infty}\frac{1}{\pi}\frac{(2x)^j}{j!}\frac{\left| \Phi^\dagger \l \rho e^{i \l \pi-\frac{\theta}{2} \r} \r \right|\left|\Phi\left(\rho e^{i\left(\pi-\frac{\theta}{2}\right)}\right)\right|^{k+j}}{\rho^{1-l}}e^{-\frac{t}{2}\rho \cos\frac{\theta}{2}}d\rho< \infty,
%%%\end{align}
%%%so that, by applying the dominated convergence theorem to \eqref{gcont}, we have
%%%\begin{equation*}
%%%	\lim_{(h_1,h_2) \to (0,0)}g_{j,\Phi}(x+h_1,t+h_2)=g_{j,\Phi}(x,t),
%%%\end{equation*}
%%%which ends the proof.
%%%Indeed we have, using \ref{eq:modcont}
%%%\begin{align}\label{eq:convergent}
%%%	\begin{split}
%%%	&\int_0^{\infty}\sum_{j=0}^{\infty}\frac{1}{\pi}\frac{(2x)^j}{j!}\frac{\left| \Phi^\dagger \l \rho e^{i \l \pi-\frac{\theta}{2} \r} \r \right|\left|\Phi\left(\rho e^{i\left(\pi-\frac{\theta}{2}\right)}\right)\right|^{k+j}}{\rho^{1-l}}e^{-\frac{t}{2}\rho \cos\frac{\theta}{2}}d\rho\\
%%%	&\qquad =\frac{1}{\pi}\int_0^{\infty}\frac{\left| \Phi^\dagger\l \rho \l e^{i \l \pi-\frac{\theta}{2} \r} \r \r \right|\left|\Phi\left(\rho e^{i\left(\pi-\frac{\theta}{2}\right)}\right)\right|^{k}}{\rho^{1-l}}e^{-\frac{t}{2}\rho \cos\frac{\theta}{2}}\sum_{j=0}^{\infty}\frac{(2x)^j}{j!}\left|\Phi\left(\rho e^{i\left(\pi-\frac{\theta}{2}\right)}\right)\right|^{j}d\rho\\
%%%	&\qquad =\frac{1}{\pi}\int_0^{\infty}\frac{\left| \phi^\dagger \l e^{i \l \pi-\frac{\theta}{2} \r} \r \right|\left|\Phi\left(\rho e^{i\left(\pi-\frac{\theta}{2}\right)}\right)\right|^{k}}{\rho^{1-l}}e^{2x\left|\Phi\left(\rho e^{i\left(\pi-\frac{\theta}{2}\right)}\right)\right|-\frac{t}{qs}\rho \cos\frac{\theta}{2}}d\rho \\
%%%	& \qquad \leq \frac{1}{\pi} C \int_0^\infty \frac{\Phi^\dagger (\rho)}{\rho^{1-l}} \l q+\mathfrak{b}\rho+C \Phi^\dagger (\rho) \r e^{2x\l q+\mathfrak{b}\rho  + C\Phi^\dagger (\rho)\r-\frac{t}{2}\rho \cos \frac{\theta}{2}}d\rho .
%%%	\end{split}
%%%	\end{align}
%%%	The integral above is convergent near zero by \ref{eq:integrlog} and Proposition \ref{prop:integr0}.
%%%At infinity, observe that
%%%
%%%************ FINISH **********
%%%
\end{proof}
With the series representation, we can provide some results concerning the behaviour at $0$.
\begin{proof}[Proof of Theorem \ref{behavatzero}]
%It is sufficient to show that the function \eqref{gjphi} is continuous as a function on $(x,t) \in [0, \infty) \times (0, \infty)$. Once we have this, we can extend the series representation for $f_\Phi(x,t)$ by continuity and the result follows.
%Note that
Let $[t_1,t_2] \subset (0,+\infty)$ and observe that for $t \in [t_1,t_2]$ we have by \eqref{derivatecode}
\begin{equation}\label{eq:estIjkl}
	\begin{split}
			|\mathcal{I}_{j,k,l}(t)| &\le \frac{1}{\pi}\int_\varepsilon^{+\infty}\frac{\left|\Phi^{\dagger}\left(\rho e^{i\left(\pi-\frac{\theta}{2}\right)}\right)\right|\left|\Phi\left(\rho e^{i\left(\pi-\frac{\theta}{2}\right)}\right)\right|^{j+k}}{\rho^{1-l}}e^{-t_1\rho \cos\left(\frac{\theta}{2}\right)}d\rho  \\
		&+ \frac{1}{2\pi} \int_{\gamma_\varepsilon} \frac{|\Phi^\dagger (\rho)|}{\rho^{1-l}} |\phi(\rho)|^{j+k} e^{-t_1\rho\cos \frac{\theta}{2}}d\rho.
	\end{split}
\end{equation}
Hence, for $0<x<\frac{t_1}{\mathfrak{b}}$ and $t \in [t_1,t_2]$ we have 
\begin{align}
	\left|\frac{\partial^k \partial^l}{\partial x^k \partial t^l}f_\Phi(x,t)-\mathcal{I}_{0,k,l}\right|  \le & \sum_{j=1}^{+\infty}\frac{x^j}{j!}|\mathcal{I}_{j,k,l}(t)| \notag \\
	\leq \, & \sum_{j=1}^{+\infty} \frac{x^j}{j!} \left[ \frac{1}{\pi}\int_\varepsilon^{+\infty}\frac{\left|\Phi^{\dagger}\left(\rho e^{i\left(\pi-\frac{\theta}{2}\right)}\right)\right|\left|\Phi\left(\rho e^{i\left(\pi-\frac{\theta}{2}\right)}\right)\right|^{j+k}}{\rho^{1-l}}e^{-t_1\rho \cos\left(\frac{\theta}{2}\right)}d\rho \right.\notag \\
	&+ \left.\frac{1}{2\pi} \int_{\gamma_\varepsilon} \frac{|\Phi^\dagger (\rho)|}{\rho^{1-l}} |\phi(\rho)|^{j+k} e^{-t_1\rho\cos \frac{\theta}{2}}d\rho \right].
	\label{laststep}
\end{align}
where, in the last step, we used \eqref{eq:estIjkl}. The convergence of the series in \eqref{laststep} can be ascertained as in \eqref{248}. The result then follows by letting $x \to 0$ in \eqref{laststep}.
\end{proof}
Finally, let us prove Theorem \ref{thm:seriessub}.
\begin{proof}[Proof of Theorem \ref{thm:seriessub}]
	Let us first consider $G_\Phi$. Starting from \eqref{statementG}, we have, assuming that we can exchange the order of the series and the integral,
	\begin{align*}
		G_\Phi(x,t)&=\sum_{j=0}^{+\infty}(-1)^j\frac{x^j}{j!}\left[\frac{1}{\pi}\int_{\varepsilon}^{+\infty}\Im\left(\frac{\left(\Phi\left(\rho e^{i\left(\pi-\frac{\theta}{2}\right)}\right)\right)^j}{\rho}e^{t \rho e^{i\left(\pi-\frac{\theta}{2}\right)}}\right)d\rho+\frac{1}{2\pi i}\int_{\gamma_{\varepsilon,\theta}} \frac{(\Phi(z))^j}{z}e^{tz}dz\right]\\
		&=\frac{1}{\pi}\int_{\varepsilon}^{+\infty}\Im\left(\frac{1}{\rho}\right)d\rho+\frac{1}{2\pi i}\int_{\gamma_{\varepsilon,\theta}} \frac{1}{z}dz\\
		&+\sum_{j=1}^{+\infty}(-1)^j\frac{x^j}{j!}\left[\frac{1}{\pi}\int_{\varepsilon}^{+\infty}\Im\left(\frac{\left(\Phi\left(\rho e^{i\left(\pi-\frac{\theta}{2}\right)}\right)\right)^j}{\rho}e^{t \rho e^{i\left(\pi-\frac{\theta}{2}\right)}}\right)d\rho+\frac{1}{2\pi i}\int_{\gamma_{\varepsilon,\theta}} \frac{(\Phi(z))^j}{z}e^{tz}dz\right]\\
		&=\sum_{j=0}^{+\infty}(-1)^j\frac{x^j}{j!}\left[\frac{1}{\pi}\int_{\varepsilon}^{+\infty}\Im\left(\frac{\left(q+b\rho e^{i\left(\pi-\frac{\theta}{2}\right)}\right)^j}{\rho}e^{t \rho e^{i\left(\pi-\frac{\theta}{2}\right)}}\right)d\rho+\frac{1}{2\pi i}\int_{\gamma_{\varepsilon,\theta}} \frac{(q+bz)^j}{z}e^{tz}dz\right]\\
		&+\sum_{j=1}^{+\infty}\sum_{k_1+k_2+k_3=j-1}(-1)^j\frac{x^j}{k_1!k_2!(k_3+1)!}q^{k_1}\mathfrak{b}^{k_2}\\
		&\times \left[\frac{1}{\pi}\int_{\varepsilon}^{+\infty}\Im\left(\frac{\left(\Phi^\dagger\left(\rho e^{i\left(\pi-\frac{\theta}{2}\right)}\right)\right)^{k_3+1}}{\rho^{k_3+1-(k_2+k_3)}e^{i(k_3-(k_2+k_3))}}e^{t \rho e^{i\left(\pi-\frac{\theta}{2}\right)}}\right)d\rho+\frac{1}{2\pi i}\int_{\gamma_{\varepsilon,\theta}} \frac{(\Phi^\dagger(z))^{k_3+1}}{z^{k_3+1-(k_2+k_3)}}e^{tz}dz\right]\\
		&=\sum_{j=0}^{+\infty}(-1)^j\frac{x^j}{j!}\left[\frac{1}{\pi}\int_{\varepsilon}^{+\infty}\Im\left(\frac{\left(q+b\rho e^{i\left(\pi-\frac{\theta}{2}\right)}\right)^j}{\rho}e^{t \rho e^{i\left(\pi-\frac{\theta}{2}\right)}}\right)d\rho+\frac{1}{2\pi i}\int_{\gamma_{\varepsilon,\theta}} \frac{(q+bz)^j}{z}e^{tz}dz\right]\\
		&+\sum_{j=1}^{+\infty}\sum_{k_1+k_2+k_3=j-1}(-1)^j\frac{x^j}{k_1!k_2!(k_3+1)!}q^{k_1}\mathfrak{b}^{k_2}\frac{d^{k_2+k_3}}{dt}\mu^{\ast(k_3+1)}(t).
	\end{align*}
Concerning the first series, it is not difficult to show that, by a simple application of Cauchy's theorem,
\begin{equation*}
	\frac{1}{\pi}\int_{\varepsilon}^{+\infty}\Im\left(\frac{\left(q+b\rho e^{i\left(\pi-\frac{\theta}{2}\right)}\right)^j}{\rho}e^{t \rho e^{i\left(\pi-\frac{\theta}{2}\right)}}\right)d\rho+\frac{1}{2\pi i}\int_{\gamma_{\varepsilon,\theta}} \frac{(q+bz)^j}{z}e^{tz}dz=\frac{1}{2\pi i}\int_{\gamma_\varepsilon}\frac{(q+bz)^j}{z}e^{tz}dz=q^j,
\end{equation*}
where in the last integral we used the fact that $\frac{(q+bz)^j}{z}e^{tz}$ admits a simple pole in $0$ with residue $1$. The latter equality implies \eqref{eq:seriesdistr}. The other two series \eqref{eq:seriessub1} and \eqref{eq:seriessub2} are obtained analogously, once one notices that for any $j \ge 0$
\begin{equation*}
	\frac{1}{\pi}\int_{\varepsilon}^{+\infty}\Im\left(\left(q+b\rho e^{i\left(\pi-\frac{\theta}{2}\right)}\right)^je^{t \rho e^{i\left(\pi-\frac{\theta}{2}\right)}}\right)d\rho+\frac{1}{2\pi i}\int_{\gamma_{\varepsilon,\theta}} (q+bz)^je^{tz}dz=\frac{1}{2\pi i}\int_{\gamma_\varepsilon}(q+bz)^je^{tz}dz=0,
\end{equation*}
since $(q+bz)^je^{tz}$ is holomorphic in the disc $\{z \in \C: \ |z|<\varepsilon\}$. Finally, the fact that one can actually exchange the series with the integral is proven exactly as in the proof of Theorem \ref{thm:seriespi}.
\end{proof}
%\begin{proof}[Proof of Proposition \ref{calcoloI}]
%Arguing as in the proof of Theorem \ref{prop:intreptheta} we get, for $a>0$,
%\begin{align}
%&\frac{1}{2\pi i}\lim_{b \to +\infty} \int_{a-ib}^{a+ib} \frac{\phi^\dagger (\rho)\l \phi (\rho) \r^{j+k}}{\rho} e^{t \rho} \, d\rho \notag \\ = \,& \frac{1}{\pi} \Im \l \int_0^\infty \frac{\phi^\dagger \l \rho e^{i \l \pi-\frac{\theta}{2} \r} \r\l \Phi \l \rho \l e^{i \l \pi-\frac{\theta}{2} \r} \r \r \r^{j+k}}{\rho} e^{t\rho e^{\l i \l \pi-\frac{\theta}{2} \r \r}}  d\rho \r \notag \\
%= \, & \mathcal{I}_{j,k,0}(t).
%\end{align}
%Since
%\begin{align}
%\int_0^{+\infty} e^{-\rho t}\frac{d^{j+k}}{dt^{j+k}} \l q +\bar{\nu} (t) \r^{j+k} (t) \star \bar{\nu}(t) \, dt \, = \, \rho^{j+k} \frac{\phi^{j+k}(\rho)}{\rho^{\star(j+k)}} \frac{\phi^\dagger (\rho)}{\rho}
%\label{lapltransfcomp}
%\end{align}
%the proof is complete.
%In \eqref{lapltransfcomp} we used \eqref{def:Phi}, \cite[Proposition 1.6.4 and Corollary 1.6.6]{abhn} which hold since $\bar{\nu}(t)$ is $L^1_{\text{loc}} \l [0, +\infty) \r$ and absolutely continuous on $(0, +\infty)$.
%
%**** CHECK THE CONVOLUTION OF TAILS IS AC ****
%
%\end{proof}





%\begin{minipage}{0.5\linewidth}
%	\centering
%	\begin{tikzpicture}[scale=0.4]
%		\draw [decoration={markings,mark=at position 1 with
%			{\arrow[scale=3,>=stealth]{>}}},postaction={decorate}] (0,-8.5) -- (0,8.5);
%		\draw [decoration={markings,mark=at position 1 with
%			{\arrow[scale=3,>=stealth]{>}}},postaction={decorate}] (-7,0) -- (7,0);
%		\draw[black, dashed] (-2,-2)--(0,0);
%		\draw[black, dashed] (-2,2)--(0,0);
%		\draw[black] (5,-7) -- (5,-4.88);
%		\draw[black] (5,4.88) -- (5,7);
%		\draw[black, dashed] (5,-7) -- (5,-8);
%		\draw[black, dashed] (5,7) -- (5,8);
%		\draw[black, dashed] (0,0)--(4.88,4.70);
%		\begin{scope}[very thick,decoration={
%				markings,
%				mark=at position 0.5 with {\arrow{>}}}
%			] 
%			\centerarc[red,very thick,postaction={decorate}](0,0)(44:90:7);
%			\centerarc[red,very thick,postaction={decorate}](0,0)(90:135:7);
%			\draw[red,very thick, postaction={decorate}] (5,-4.88) -- (5,4.88);
%			\draw[red,very thick,postaction={decorate}] (-4.88,4.88) -- (-2,2);
%			\centerarc[red,very thick,postaction={decorate}](0,0)(135:-135:2.828);
%			\draw[red,very thick,postaction={decorate}] (-2,-2)--(-4.88,-4.88);
%			\centerarc[red,very thick,postaction={decorate}](0,0)(-135:-90:7);
%			\centerarc[red,very thick,postaction={decorate}](0,0)(-90:-44:7);
%		\end{scope}
%		\fill[red] (5,4.88) circle (0.2);
%		\fill[red] (-4.95,4.95) circle (0.2);
%		\fill[red] (-2,2) circle (0.2);
%		\fill[red] (-2,-2) circle (0.2);
%		\fill[red] (-4.95,-4.95) circle (0.2);
%		\fill[red] (5,-4.88) circle (0.2);
%		\node at (6.3,5.40) {\large $B(R)$};
%		\node at (6.3,-5.40) {\large $A(R)$};
%		\node at (-6.3,-5.25) {\large $F(R)$};
%		\node at (-6.3,5.25) {\large $C(R)$};
%		\node at (-3,1.5) {\large $D(\varepsilon)$};
%		\node at (-3,-1.5) {\large $E(\varepsilon)$};
%		\node at (-0.85,-1.5) {\large $\varepsilon$};
%		\centerarc[dashed](0,0)(135:225:0.5);
%		\node at (-1,0.3) {\large $\theta$};
%		\node at (2.8,3.5) {\large $R$};
%		\node at (5.6,0.3) {\large $x_0$};
%		\node at (4.5,2) {\large $r$};
%		\fill[red] (0,7) circle (0.2);
%		\fill[red] (0,-7) circle (0.2);
%		\fill[black] (5,0) circle (0.1);
%		\node at (2,7.5) {\large $M_+(R)$};
%		\node at (2,-7.7) {\large $M_-(R)$};
%	\end{tikzpicture}
%	\captionof{figure}{Sketch of the keyhole-type contour for $\theta \in (0,\pi)$.}\label{fig2}
%\end{minipage}
%\begin{minipage}{0.5\linewidth}
%	\centering
%	\begin{tikzpicture}[scale=0.4]
%		\draw [decoration={markings,mark=at position 1 with
%			{\arrow[scale=3,>=stealth]{>}}},postaction={decorate}] (0,-8.5) -- (0,8.5);
%		\draw [decoration={markings,mark=at position 1 with
%			{\arrow[scale=3,>=stealth]{>}}},postaction={decorate}] (-7,0) -- (7,0);
%		\draw[black, dashed] (-2,-2)--(0,0);
%		\draw[black, dashed] (-2,2)--(0,0);
%		\draw[black] (5,-7) -- (5,-4.88);
%		\draw[black] (5,4.88) -- (5,7);
%		\draw[black, dashed] (5,-7) -- (5,-8);
%		\draw[black, dashed] (5,7) -- (5,8);
%		\draw[black, dashed] (0,0)--(4.88,4.70);
%		\begin{scope}[very thick,decoration={
%				markings,
%				mark=at position 0.5 with {\arrow{>}}}
%			] 
%			\centerarc[red,very thick,postaction={decorate}](0,0)(44:90:7);
%			\centerarc[red,very thick,postaction={decorate}](0,0)(90:135:7);
%			\draw[red,very thick, postaction={decorate}] (5,-4.88) -- (5,4.88);
%			\draw[red,very thick,postaction={decorate}] (-4.88,4.88) -- (-2,2);
%			\centerarc[red,very thick,postaction={decorate}](0,0)(135:-135:2.828);
%			\draw[red,very thick,postaction={decorate}] (-2,-2)--(-4.88,-4.88);
%			\centerarc[red,very thick,postaction={decorate}](0,0)(-135:-90:7);
%			\centerarc[red,very thick,postaction={decorate}](0,0)(-90:-44:7);
%		\end{scope}
%		\fill[red] (5,4.88) circle (0.2);
%		\fill[red] (-4.95,4.95) circle (0.2);
%		\fill[red] (-2,2) circle (0.2);
%		\fill[red] (-2,-2) circle (0.2);
%		\fill[red] (-4.95,-4.95) circle (0.2);
%		\fill[red] (5,-4.88) circle (0.2);
%		\node at (6.3,5.40) {\large $B(R)$};
%		\node at (6.3,-5.40) {\large $A(R)$};
%		\node at (-6.3,-5.25) {\large $F(R)$};
%		\node at (-6.3,5.25) {\large $C(R)$};
%		\node at (-3,1.5) {\large $D(\varepsilon)$};
%		\node at (-3,-1.5) {\large $E(\varepsilon)$};
%		\node at (-0.85,-1.5) {\large $\varepsilon$};
%		\centerarc[dashed](0,0)(135:225:0.5);
%		\node at (-1,0.3) {\large $\theta$};
%		\node at (2.8,3.5) {\large $R$};
%		\node at (5.6,0.3) {\large $x_0$};
%		\node at (4.5,2) {\large $r$};
%		\fill[red] (0,7) circle (0.2);
%		\fill[red] (0,-7) circle (0.2);
%		\fill[black] (5,0) circle (0.1);
%		\node at (2,7.5) {\large $M_+(R)$};
%		\node at (2,-7.7) {\large $M_-(R)$};
%	\end{tikzpicture}
%	\captionof{figure}{Sketch of the keyhole-type contour for $\theta \in (0,\pi)$.}\label{fig3}
%\end{minipage}