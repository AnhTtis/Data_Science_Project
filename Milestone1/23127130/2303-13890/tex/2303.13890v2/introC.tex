%GA: I think usually Introductions already contain motivations, so I do not think it is necessary to write it in the title.
\section{Introduction}\label{sec:intro}
Subordinators are L\'evy processes, i.e. c\'adl\'ag stochastic processes with stationary and independent increments, whose paths are almost surely non-decreasing. For this reason, they constitute a special class of \LL processes of fundamental interest in probability theory. In this paper, wherever possible, we consider potentially killed subordinators, that is to say that for any proper subordinator $\sigma:=\lbrb{\sigma(t)}_{t\geq 0}$ we allow for killing at an exponentially distributed random variable $\mathbf{e}_q, q\geq0$ ($\mathbf{e}_0=\infty$) that is independent of $\sigma$,  by setting $\sigma(t)=\infty$ provided $t \ge  \mathbf{e}_q$ and keeping the original process until $\mathbf{e}_q$. With each potentially killed subordinator one defines the (right-)inverse process through the passage times $L:=\lbrb{L(t)}_{t \geq  0}$. In this work, our main aim is to offer a detailed study of the densities of $L$. In more detail, under mild and natural conditions, we prove that such densities are smooth in both variables and we provide for them and their derivatives a series representation and precise non-classical Tauberian asymptotics (with explicit speed of convergence). The latter is obtained by means of Laplace inversion, as, indeed, the Laplace transform of the density of $L(t)$ with respect to the variable $t$ admits a simple form in terms of the Laplace exponent of the subordinator $\sigma$. Furthermore, we also extend the results to densities (and their derivatives) of the subordinators themselves. Our results expand uniformly the existing knowledge on these quantities and allow for specific applications. We shall discuss the literature, offer motivation for our study  and outline our methodology below.


We first mention that, in recent years, the interest in subordinators and their hitting-times has grown fast and has reached an increasingly large audience also outside the probability community, involving many areas of mathematics. This is due, in particular, to the connection with the so-called anomalous processes (or anomalous diffusion) and semi-Markov processes. Take a Markov process $X=\lb X(t) \rb_{ t \geq 0 }$ and consider the time-changed process $Y=\lb Y(t)\rb_{  t \ge 0}$, with $Y(t):=X(L(t))$, for any $t \ge 0$. Then the process $Y$ has intervals of constancy, induced by the time-change, that are distributed according to the jumps of $\sigma$ (which are not necessarily exponentials). Under suitable assumptions, these time-changed processes are prototypes of semi-Markov processes (e.g., \cite{cinlarsemi, kaspi, meerstra, savtoa}). The importance of these processes arise in several applications in very different fields, among others: they are scaling limit of continuous time random walks (e.g., \cite{baemstra, meercoupled, meertri, meerstra}), they are useful to model anomalous diffusion and fractional kinetics (see, for example, \cite{moving, beghin, gianni2, hairer, koko, kochukondra, FCbook, Metzler, gianni, cimp, savtoa, silva, umarov}), they appear in economics and mathematical finance (e.g., \citep{jacquier, scalas, scalastoaldo, lorenzo1}) and recently also in neuronal modelling (e.g., \cite{annals2020,LIF}). It is clear that in this context the one-dimensional distributions of the random variables $L(t)$ play a central role. For instance, if we assume that $X$ admits as state space $\R^d$, for some $d \ge 1$, and we denote by $p(x,y,t)$ the transition densities of $X$, then one can show that
\begin{align}
\cB(\R^d) \ni E  \mapsto \dP \lb Y(t) \in E \mid Y(0)=x \rb \, = \, \int_{E} \int_0^\infty p(x,y,s)f(s;t)ds  \, dy,
\end{align}
%Since it is an illustrative example, maybe we can keep it really basic by assuming that the space is \R^d itself and removing some "demonstrative details". 
where, for any $t>0$, $f(s;t)$ is the density of $L(t)$ and $\cB(\R^d)$ is the Borel $\sigma$-algebra of $\R^d$. In practice, the time-changed process admits a density that can be written in terms of the one-dimensional distributions of $L(t)$. Hence, in order to determine some features of the one-dimensional distribution of $Y$, it may be necessary to rely also on some specific properties of the ones of $L(t)$ itself. Therefore, a more detailed study of the main features of the densities of inverse subordinators is needed. This point of view inspired several other works on this topic (e.g., \cite{fausto, kovacmeer, kumar, meerstra2, taqqu}).


Two of the key objectives in the study of the one-dimensional distributions of random processes are to understand the asymptotic behaviour in time and space of their tails and to find representations for their densities and those of related quantities, e.g. such as passage times. This information is usually in the form of  explicit asymptotic terms at zero and/or at infinity, series expansions, integral representations, etc. In this work we provide results for densities (and their derivatives) of inverse subordinators and, via interchangeability, of subordinators. These include the derivation of precise, universal and explicit form of the large asymptotic behaviour with speed of convergence and general series expansions. The results go well beyond the current state-of-art which we briefly review below.  


% Laplace in place of LK since we actually use this notation.
On the large asymptotics the first papers, see  \cite{FriPruitt71,JainPr_87}, offered results for lower tails of subordinators. They have been subsequently refined for densities  with the most up-to-date results contained in \cite{DonRiv,GrLTr21}. General bounds with conditions in the spirit of \cite{DonRiv} are derived in \cite{ChoKim21,GrLTr21}. Particular cases of bounds and estimates on densities for a class of subordinators can be found in the recent work \cite{ChoKim21}. Similar asymptotic results (without speed of convergence) are contained in \cite{PatVai22} which deals with densities (and their derivatives) of spectrally negative \LLPs with necessarily positive Brownian component.  For our new results in the context of the large asymptotics we use the saddle point method, as applied in \cite{MinSav23+}, with a novel modification which allows us to capture speed of convergence. This speed depends on the ratio of time and space, i.e. $t/x$, in $\Pbb{L(t)\in dx; \sigma(L(t))>t}=f(x,t)dx$, as it ranges between the drift of $\sigma$ and $\Ebb{\sigma(1)}$, and the closer $t/x$ is to the drift the faster the speed of convergence is. As a result, under conditions significantly milder than the existing in the literature, we present in Theorems \ref{thm:mainL}, \ref{thm:main1}, \ref{thm:main2}, \ref{thm:mainS} explicit expressions for the aforementioned densities (and their derivatives) which are dominated by rather explicit exponential terms stemming from the Laplace exponent of $\sigma$. It is important to highlight once again that these representations are valid for  $t/x<\Ebb{\sigma(1)}$ and therefore capture non-typical slow growth of $\sigma$ or equivalently fast growth of $L$. For example, when the subordinator has a finite second moment, our results capture the region below that of the central limit theorem, see Section \ref{subsubsec:EX} for more details. If $\Ebb{\sigma(1)}=\infty$ then our estimates capture the region of the lower envelope of subordinators and therefore one can obtain more precise local estimates for densities including explicit constants in the speed of convergence and hence furnish estimates for the probabilities that lead up to the law of the iterated logarithm as in \cite{bertoins,FriPruitt71}, see Section \ref{subsubsec:EX}. One is also able to study particular classes of subordinators for which our main results can be further specialized. Also, given the nature of the saddle point method, we can derive under milder assumptions concrete bounds for fixed times as in \cite{ChoKim21,GrLTr21}. These are directions for further investigations. 

On representations of densities with series expansions the literature is mainly concerned with stable processes, despite the fact that some generalizations have been obtained. Series representations for stable laws can be found, e.g., in \cite[Chapter XVII.6]{fellerbook}. In the specific case of the stable subordinator, a series representation for its density has been provided in \cite{PG10}. It is clear that such a representation can be extended to the density of the inverse stable subordinator by means of the relation that links the two quantities, as for instance highlighted in \cite[Corollary 3.1]{meerschlim}. A similar result has been proven for the inverse tempered stable subordinator in \cite{kumar}, while a further integral representation for the density of the inverse gamma subordinator has been provided in \cite{kumar2} and further improved in \cite{fausto}. We remark that in both cases the results have been widely used, especially in applications regarding anomalous diffusion (see, e.g., \cite{metzlerbarkai} and references therein) and in the context of governing equations of time-changed processes (see, e.g., \cite{umarov} and references therein). Furthermore, in \cite{bur}, the authors obtained explicit representations for some subordinators of the Thorin class. Despite in general such explicit representations do not provide power-series expansions, some of them can be still rewritten in this way. Here,  under mild assumptions on the Laplace exponent of the subordinator, we obtain an explicit power series representation for the density (and its derivatives) of the  inverse subordinator, in Theorem \ref{thm:seriespi}, and of the subordinator itself, in Theorem \ref{thm:seriessub}. We highlight that our power series representation holds, in particular, whenever the Laplace exponent of $\sigma$ is a complete Bernstein function, thus covering also the cases discussed in \cite{PG10,kumar}, for which comparison is carried on in Section \ref{discussionassumptions}. The assumption on the completeness of the Laplace exponent is sufficient, but not necessary, as the power-series representation can be applied on a wider class of subordinators, as discussed in Section \ref{discussionassumptions}.
%For our new results we use Laplace inversion method and the theory of Bernstein functions extended to the complex plane. Under very mild conditions, we obtain a representation as a power series for the density of subordinators and inverse subordinators (and their derivatives)  (see Theorems \ref{thm:seriespi}, \ref{thm:seriessub} and Corollary \ref{cor:seriescreep}) and, as an immediate consequence, we obtain the behaviour at zero of the densities of inverse subordinators and their derivatives (see Theorem \ref{behavatzero}). In particular, our method works whenever the Laplace exponent of the subordinator can be analytically extended to the complex plane with negative real part and non-zero imaginary part (see Section \ref{discussionassumptions} for a thorough discussion on the generality of the used assumptions).
%Later, in Section \ref{sec:comparisonseries} below we offer a comparison of our results with the ones in \cite{kumar, PG10}.

As already mentioned, both the asymptotic behaviour and the power-series representation are obtained via Laplace inversion. On the one hand, the assumptions used to derive the asymptotic behaviour first lead to the absolute convergence of the integral involved in the Laplace inversion (see \cite[Theorem $4.1.21$]{abhn}). On the other hand, the assumptions adopted to obtain the power-series representation consider a suitable keyhole-type contour, on which the inversion integral becomes absolutely convergent. Furthermore, as a consequence, we get the smoothness of the involved densities starting from any of the two sets of assumptions. We think that the derived integral representations could be of independent interest and thus are discussed separately in Section \ref{sec:LT}. Finally, let us remark that the proofs of the main results presented in Section \ref{sec:mainResults} are given separately in Section \ref{sec:proofs}.