%\subsection{Proofs for large asymptotic behaviour}
\subsection{Proofs for Subsection \ref{subsec:L}}
\label{subsec:asymp}
In this part we provide proofs for for subsection \ref{subsec:L}. We start with some preliminary results.
\begin{prop}\label{prop:D}
	Let $\Phi$ be the Laplace exponent of a potentially killed subordinator  and  \eqref{def:condiA} holds for some $L>0$. Let $t=t(x)$ be such that $t/x \downarrow \mathfrak{b}$ and $a_*=a_*(x)$ be the unique solution of $\Phi'(a_*)=\frac{t}{x}\in\lbrb{\mathfrak{b},\Phi'(0)}$. Then $\lim_{x \to \infty}a_*=\infty$ and, for $x$ large enough, 
	\begin{equation}\label{eq:ineq}
		\begin{split}
			&a_*>\frac{Me^{-1}}{\frac{t}{x}-\mathfrak{b}}.
		\end{split}
	\end{equation}
\end{prop}
\begin{proof}
	First of all, observe that, since $t/x \downarrow \mathfrak{b}$ and $\Phi'$ is decreasing with $\lim_{z \to \infty}\Phi'(z)=\mathfrak{b}$, we have $\lim_{x \to \infty}a_*=\infty$. Furthermore, for any $M<L$, using the second inequality of \eqref{eq:P}, we get
	\[Me^{-1}<\limi{x}\frac{a_*\lbrb{\Phi'(a_*)-\mathfrak{b}}}{\ln(a_*)}=\limi{x}\frac{a_*\lbrb{\frac{t}{x}-\mathfrak{b}}}{\ln(a_*)}.\]
	This shows that for all $x$ and therefore $a_*$ large enough
	\[\frac{a_*}{\ln(a_*)}>\frac{Me^{-1}}{\frac{t}{x}-\mathfrak{b}}.\]
	This concludes the proof.
\end{proof}
Now we are ready to prove the main theorem concerning the asymptotic behaviour.
\begin{proof}[Proof of Theorem \ref{thm:mainL}]
Fix $k,l\geq 0$ and assume that $x>x_{0}(k+l,L)$, see Theorem \ref{thm:regularityfphi1}. Then, for any $a>0$, by \eqref{intreprder1} it holds
	\begin{equation}\label{def:derf}
		\frac{\partial^k\partial ^l }{\partial x^k\partial t^l}\fP{x,t}=\frac{(-1)^k}{2\pi}\int_{-\infty}^\infty \frac{\Phi^{\dagger}(\ab)\Phi^{k}(\ab)}{\lbrb{\ab}^{1-l}}e^{-x\Phi(\ab)+t\lbrb{\ab}}db=:I(x,t).
	\end{equation}
	%where we have simply differentiated the inversion of the Laplace transform in \eqref{eq:LT1} which is possible thanks to the absolute integrability of the expression in \eqref{intreprder1}.
	Let $t=t(x)$ be as in the statement of the theorem and $a_*:=a(x)$ be the solution to the equation
	\begin{equation*}
		\Phi'(a_*)=\frac{t}{x}\in\lbrb{\mathfrak{b},\Phi'(0^+)},
	\end{equation*}
	which exists and is unique thanks to the fact $\Phi'(x)$ is decreasing with $\limi{x}\Phi'(x)=\mathfrak{b}$, see Item \eqref{it:phi'} of Lemma \ref{lem:Bern}.
	Choosing this $a_*$ for the line of inversion we get
	\begin{equation}\label{eq:f1}
		I(x,t)=\frac{(-1)^k}{2\pi}e^{a_*t-x\Phi(a_*)}\IntII\frac{\Phi^{\dagger}(a_*+ib)\Phi^{k}(a_*+ib)}{\lbrb{a_*+ib}^{1-l}}e^{ibt -x\lbrb{\Phi(a_*+ib)-\Phi(a_*)}}db,
	\end{equation}
	where the choice of $a_*$ minimizes the function  $a \in (0,+\infty)\mapsto (at-x\Phi(a)) \in \R$. Recalling that $t(x)/x\downarrow\mathfrak{b}$, we get that $\limi{x}a_*\to\infty$ by Proposition \ref{prop:D}. From now on, each time we write $\so{\cdot}$ or $\bo{\cdot}$, we imply that $x \to \infty$.
	We  split the region of integration. For any $\varepsilon>0$, which  will subsequently be chosen to depend on $a_*$, we set  $\Ic_{\varepsilon}:=[-a_*\varepsilon,a_*\varepsilon], \Jc_\varepsilon:=\Ic^c_\varepsilon$. Put 
		\begin{equation}\label{def:Ie}
		I_\varepsilon(x,t):=\frac{(-1)^k}{2\pi}e^{a_*t-x\Phi(a_*)}\int_{\Ic_{\varepsilon}}\frac{\Phi^{\dagger}(a_*+ib)\Phi^{k}(a_*+ib)}{\lbrb{a_*+ib}^{1-l}}e^{ibt -x\lbrb{\Phi(a_*+ib)-\Phi(a_*)}}db
	\end{equation}
and
\begin{equation}\label{def:Je1}
	\begin{split}
		J(a_*,b)&:=\frac{\Phi^{\dagger}(a_*+ib)\Phi^{k}(a_*+ib)}{\lbrb{a_*+ib}^{1-l}}.
	\end{split}
\end{equation}
	Using Taylor's formula we get
\begin{equation}\label{def:Tay}
	\begin{split}
		&\Phi(a_*+ib)-\Phi(a_*)=ib\frac{t}{x}-\frac{b^2}{2}\Phi''(a_*)-i\frac{b^3}{3!}\Phi'''(a_*+ib\theta(a_*,b)),
	\end{split}
\end{equation}
where we have used the definition of $a_*$ in \eqref{eq:a*} and note that $0\leq\theta(a_*,b)\leq 1$. Hence, with
\begin{equation*}
	\begin{split}
		U(a_*,b):=&ibt -x\lbrb{\Phi(a_*+ib)-\Phi(a_*)}-x\frac{b^2}{2}\Phi''(a_*)=-ix\frac{b^3}{3!}\Phi'''(a_*+ib\theta(a_*,b)),
	\end{split}
\end{equation*}
we have  for the integral in \eqref{def:Ie}
\begin{equation}\label{def:I'}
	\begin{split}
		&\int_{\Ic_{\varepsilon}}J(a_*,b)e^{ibt -x\lbrb{\Phi(a_*+ib)-\Phi(a_*)}}db=\int_{\Ic_{\varepsilon}}J(a_*,b)e^{\frac{b^2}{2}x\Phi''(a_*)\lbrb{1+2\frac{U(a_*,b)}{xb^2\Phi''(a_*)}}}db\\&=\!\frac{1}{\sqrt{-\Phi''(a_*)x}}\int_{-\varepsilon a_*\sqrt{-\Phi''(a_*)x}}^{\varepsilon a_*\sqrt{-\Phi''(a_*)x}}J\lbrb{a_*,\frac{u}{\sqrt{-\Phi''(a_*)x}}}\!e^{-\frac{u^2}{2}\lbrb{1-2\frac{U\lbrb{a_*,u/\sqrt{-\Phi''(a_*)x}}}{u^2}}}\!\!du,
	\end{split}	
\end{equation}
where we have used that $\Phi''(y)<0$, for $y>0$, see Item \eqref{it:phi'} of Lemma \ref{lem:Bern}. 	Here, we need \eqref{def:condiB}, that is $\limsupi{y}y\Phi'''(y)/(-\Phi''(y))=K<\infty$.  Indeed, due to this and since $\abs{\Phi'''(z)}\leq \Phi'''(\Re(z))$, for $\Re(z)>0$, see Item \ref{it:real} in Lemma \ref{lem:Bern}, we have thanks to the definition of $U(a_*,b)$ that for all $x$ and henceforth for all $a_*$ large enough the inequality 
\begin{equation*}
	\begin{split}
		&\sup_{ |u|\leq \varepsilon a_*\sqrt{-\Phi''(a_*)x}}\frac{2}{u^2}\abs{U\lbrb{a_*,\frac{u}{\sqrt{-\Phi''(a_*)x}}}}\leq  \sup_{ |u|\leq\varepsilon a_*\sqrt{-\Phi''(a_*)x}}\frac{x|u|}{3x^{\frac32}\lbrb{-\Phi''(a_*)}^{\frac32}}\Phi'''(a_*)\leq  \frac{2K}{3}\varepsilon
	\end{split}
\end{equation*}
holds true.	Next, we choose
\begin{equation}\label{eq:varesp}
	\begin{split}
		&\varepsilon(a_*):=\varepsilon(a_*,x)=\frac{g(a_*,x)}{ a_*\sqrt{-\Phi''(a_*)x}}
	\end{split}
\end{equation}
for some $g(y,x)$ going as slowly to infinity, as $x,y\to\infty$, as we wish. Then, we get
\begin{equation*}
	\begin{split}
		\sup_{ |u|\leq g(a_*)}\frac{2}{u^2}\abs{U\lbrb{a_*,\frac{u}{\sqrt{-\Phi''(a_*)x}}}}\leq  \frac{2K}{3}\frac{g(a_*) }{a_*\sqrt{-\Phi''(a_*)x}}.
	\end{split}
\end{equation*}
	Since from Proposition \ref{prop:D0} $\limi{y}-y^2\Phi''(y)=\infty$ then, for any $g(a_*,x)=\so{a_*\sqrt{-\Phi''(a_*)x}}$, we have
\begin{equation}\label{eq:estU}
	\begin{split}
		\sup_{ |u|\leq g(a_*,x)}\frac{2}{u^2}\abs{U\lbrb{a_*,\frac{u}{\sqrt{-\Phi''(a_*)x}}}}=\bo{\frac{g(a_*,x) }{a_*\sqrt{-\Phi''(a_*)x}}}.
	\end{split}
\end{equation}
Next, setting 
\begin{equation}\label{def:tJ}
	\begin{split}
		&\tilde{J}(a_*,u)=J(a_*,u)-J(a_*,0),
	\end{split}
\end{equation}
we get that from \eqref{def:I'}
\begin{equation}\label{def:I'1}
	\begin{split}
	&\int_{\Ic_{\varepsilon}}J(a_*,b)e^{ibt -x\lbrb{\Phi(a_*+ib)-\Phi(a_*)}}db	=\frac{J(a_*,0)}{\sqrt{-\Phi''(a_*)x}}\int_{-g(a_*,x)}^{g(a_*,x)}e^{-\frac{u^2}{2}\lbrb{1+\bo{\frac{g(a_*,x) }{a_*\sqrt{-\Phi''(a_*)x}}}}}du\\
		&+\frac{1}{\sqrt{-\Phi''(a_*)x}}\int_{-g(a_*,x)}^{g(a_*,x)}\tilde{J}\lbrb{a_*,\frac{u}{\sqrt{-\Phi''(a_*)x}}}e^{-\frac{u^2}{2}\lbrb{1+\bo{\frac{g(a_*,x) }{a_*\sqrt{-\Phi''(a_*)x}}}}}du\\
		&=:H_1(a_*,x)+H_2(a_*,x).
	\end{split}
\end{equation}
Clearly,
\begin{equation*}
	\begin{split}
		H_1(a_*,x)&=\frac{1}{\sqrt{-\Phi''(a_*)}x}\frac{\Phi^\dagger(a_*)\Phi^{k}(a_*)}{a^{1-l}_*}\int_{-g(a_*,x)}^{g(a_*,x)}e^{-\frac{u^2}{2}\lbrb{1+\bo{\frac{g(a_*,x) }{a_*\sqrt{-\Phi''(a_*)x}}}}}du\\
		&=\frac{\lbrb{1+\bo{\frac{g(a_*,x) }{a_*\sqrt{-\Phi''(a_*)x}}}}}{\sqrt{-\Phi''(a_*)x}}\frac{\Phi^\dagger(a_*)\Phi^{k}(a_*)}{a^{1-l}_*}\int_{-g(a_*,x)}^{g(a_*,x)}e^{-\frac{u^2}{2}}du.
	\end{split}
\end{equation*}
	Since $\limi{y,x}g(y,x)=\infty$ we  get further that 
\begin{equation}\label{eq:H1}
	\begin{split}
		&	H_1(a_*,x)=\frac{\sqrt{2\pi}}{\sqrt{-\Phi''(a_*)x}}\frac{\Phi^\dagger(a_*)\Phi^{k}(a_*)}{a^{1-l}_*}\lbrb{1+\bo{\frac{g(a_*,x) }{a_*\sqrt{-\Phi''(a_*)x}}}+\bo{\frac{e^{-\frac{g^2(a_*,x)}{2}}}{g(a_*,x)}}},
	\end{split}
\end{equation}
	where we have used the well-known
\[\int_{u>y}e^{-\frac{u^2}{2}}du\simi C \frac{1}{y}e^{-\frac{y^2}{2}}.\]
 From the definition of $J$, see \eqref{def:Je1}, used in \eqref{def:tJ} we easily get that
\begin{equation}\label{def:tJ1}
	\begin{split}
	\tilde{J}\lbrb{a_*,b}	&=\frac{\Phi^{\dagger}(a_*+ib)\Phi^{k}(a_*+ib)-\Phi^{\dagger}(a_*)\Phi^{k}(a_*)}{\lbrb{a_*+ib}^{1-l}}+\Phi^{\dagger}(a_*)\Phi^{k}(a_*)\lbrb{\frac{1}{\lbrb{a_*+ib}^{1-l}}-\frac{1}{a^{1-l}_*}}\\
	&=	\tilde{J}_1\lbrb{a_*,b}+\tilde{J}_2\lbrb{a_*,b}.
	\end{split}
\end{equation}
We shall use this to estimate for $|u|\leq g(a_*,x)$ the right-hand side of the inequality
\begin{equation}\label{aim}
	\begin{split}
		&\abs{\tilde{J}\lbrb{a_*,\frac{u}{\sqrt{-\Phi''(a_*)x}}}}\leq \abs{\tilde{J}_1\lbrb{a_*,\frac{u}{\sqrt{-\Phi''(a_*)x}}}}+\abs{\tilde{J}_2\lbrb{a_*,\frac{u}{\sqrt{-\Phi''(a_*)x}}}}.
	\end{split}
\end{equation}
Note that since $a_*=a_*(x)$, we choose $g(a_*,x)=\sqrt{2\ln\lbrb{a_*\sqrt{-\Phi''(a_*)x}}}$, but keep $g$ wherever it is more convenient, we have that 
\begin{equation}\label{estimateU}
	\begin{split}
		&\sup_{|u|\leq g(a_*,x)}\abs{\frac{u}{\sqrt{-\Phi''(a_*)x}}}=\frac{g(a_*,x)}{\sqrt{-\Phi''(a_*)x}}.
	\end{split}
\end{equation}
	Therefore, by Taylor's formula
	\begin{equation*}
		\begin{split}
			&\sup_{|u|\leq g(a_*,x)}\abs{\Phi\lbrb{a_*+i\frac{u}{\sqrt{-\Phi''(a_*)x}}}-\Phi\lbrb{a_*}}\leq \frac{g(a_*,x)}{\sqrt{-\Phi''(a_*)x}}\Phi'(a_*)\\&=\frac{g(a_*,x)}{a_*\sqrt{-\Phi''(a_*)x}}a_*\Phi'(a_*)
			=\so{\frac{\Phi(a_*)\ln(x)}{\sqrt{x}}}\\
			&
			\sup_{|u|\leq g(a_*,x)}\abs{\Phi^{\dagger}\lbrb{a_*+i\frac{u}{\sqrt{-\Phi''(a_*)x}}}-\Phi^\dagger\lbrb{a_*}}= \so{\frac{\Phi^\dagger(a_*)\ln(x)}{\sqrt{x}}},
		\end{split}
	\end{equation*}
	where we have used that $\abs{\Phi'(z)}\leq \Phi'(\Re(z))$, provided $\Re(z)>0$, see Item \ref{it:real} of Lemma \ref{lem:Bern},  $x\Phi'(x)\leq \Phi(x)$, see Item \ref{it:ineq} of Lemma \ref{lem:Bern}, $\limi{x}a_*\sqrt{-\Phi''(a_*)}=\infty$, see \eqref{eq:P}, and \eqref{estimateU}. From hence and \eqref{estimateU}, we get that as $x$ and therefore $a_*$ converge to infinity
	\begin{equation*}
		\begin{split}
			&	\sup_{|u|\leq g(a_*,x)}\abs{\tilde{J}_1\lbrb{a_*,\frac{u}{\sqrt{-\Phi''(a_*)x}}}}\leq \frac{2^k\lbrb{\phi^\dagger(a_*)\phi^{k-1}(a_*)\so{\frac{\Phi(a_*)\ln(x)}{\sqrt{x}}}+\phi^{k}(a_*)\so{\frac{\Phi^\dagger(a_*)\ln(x)}{\sqrt{x}}}}}{\lbrb{a_*+\so{\frac{a_*\ln(x)}{\sqrt{x}}}}^{1-l}}\\
			&\stackrel{\infty}{=}\frac{\phi^\dagger(a_*)\phi^{k}(a_*)}{a^{1-l}_*}\so{\frac{\ln(x)}{\sqrt{x}}}.
		\end{split}
	\end{equation*}
Similarly using \eqref{estimateU} and \eqref{def:tJ1} we get that
	\begin{equation*}
			\sup_{|u|\leq g(a_*,x)}\abs{\tilde{J}_2\lbrb{a_*,\frac{u}{\sqrt{-\Phi''(a_*)x}}}}\stackrel{\infty}{=}\frac{\phi^\dagger(a_*)\phi^{k}(a_*)}{a^{1-l}_*}\abs{\frac{1}{1+i\so{\frac{\ln(x)}{\sqrt{x}}}}-1}\stackrel{\infty}{=}\frac{\phi^\dagger(a_*)\phi^{k}(a_*)}{a^{1-l}_*}\so{\frac{\ln(x)}{\sqrt{x}}}.
\end{equation*}
	Therefore, from \eqref{def:tJ1} and \eqref{aim} we get that
	\begin{equation*}
		\begin{split}
			&	\sup_{|u|\leq g(a_*,x)}\abs{\tilde{J}\lbrb{a_*,\frac{u}{\sqrt{-\Phi''(a_*)x}}}}\stackrel{\infty}{=}\frac{\phi^\dagger(a_*)\phi^{k}(a_*)}{a^{1-l}_*}\so{\frac{\ln(x)}{\sqrt{x}}}.
		\end{split}
	\end{equation*}
	Using this in \eqref{def:I'1} we arrive from \eqref{eq:H1} at
	\begin{equation}\label{H2}
		\begin{split}
			&H_2(a_*,x)=H_1(a_*,x)\so{\frac{\ln(x)}{\sqrt{x}}}.
		\end{split}
	\end{equation}
	Combining \eqref{eq:H1} and \eqref{H2} in \eqref{def:I'1} we deduct 
	\begin{equation}\label{I'2}
		\begin{split}
			&\int_{\Ic_{\varepsilon}}J(a_*,b)e^{ibt -x\lbrb{\Phi(a_*+ib)-\Phi(a_*)}}db\\
			&=\frac{\sqrt{2\pi}}{\sqrt{-\Phi''(a_*)x}}\frac{\Phi^\dagger(a_*)\Phi^{k}(a_*)}{a^{1-l}_*}\lbrb{1+\bo{\frac{g(a_*,x) }{a_*\sqrt{-\Phi''(a_*)x}}}+\bo{\frac{e^{-\frac{g^2(a_*,x)}{2}}}{g(a_*,x)}}}.
		\end{split}
	\end{equation}
	Plugging this in \eqref{def:Ie} we get
	\begin{equation}\label{def:asympIe1}
	\begin{split}		
		&I_\varepsilon(x,t)=\frac{(-1)^k\Phi^\dagger(a_*)\Phi^{k}(a_*)}{\sqrt{2\pi}a^{1-l}_*}\frac{e^{a_*t-x\Phi(a_*)}}{\sqrt{-\Phi''(a_*)x}} \lbrb{1+\bo{\frac{g(a_*,x) }{a_*\sqrt{-\Phi''(a_*)x}}+\frac{e^{-\frac{g^2(a_*,x)}{2}}}{g(a_*,x)}}}.
	\end{split}
\end{equation}
	We proceed to investigate
	\begin{equation}\label{def:Je}
		\begin{split}
			J_\varepsilon(x,t)&:=\frac{(-1)^k}{2\pi}e^{a_*t-x\Phi(a_*)}\int_{\Jc_{\varepsilon}}\frac{\Phi^\dagger(a_*+ib)\Phi^{k}(a_*+ib)}{\lbrb{a_*+ib}^{1-l}}e^{ibt -x\lbrb{\Phi(a_*+ib)-\Phi(a_*)}}db\\
			&=\frac{(-1)^k}{2\pi}e^{a_*t-x\Phi(a_*)}\int_{\varepsilon(a_*)a_*\leq |b|\leq ca_*}\!\!\!\!\!\!\!\!\frac{\Phi^\dagger(a_*+ib)\Phi^{k}(a_*+ib)}{\lbrb{a_*+ib}^{1-l}}e^{ibt -x\lbrb{\Phi(a_*+ib)-\Phi(a_*)}}db\\
			&+\frac{(-1)^k}{2\pi}e^{a_*t-x\Phi(a_*)}\int_{|b|\geq ca_*}\frac{\Phi^\dagger(a_*+ib)\Phi^{k}(a_*+ib)}{\lbrb{a_*+ib}^{1-l}}e^{ibt -x\lbrb{\Phi(a_*+ib)-\Phi(a_*)}}db\\
			&=:\frac{(-1)^k}{2\pi}e^{a_*t-x\Phi(a_*)}\lbrb{J_1(a_*,x)+J_2(a_*,x)},
		\end{split}
	\end{equation}
	where $c>0$ will be chosen just beneath. First we estimate  $J_1(a_*,x)$. We use the Taylor expansion \eqref{def:Tay} to the exponent in  $J_1(a_*,x)$, $\abs{\Im\Phi'''(z)}\leq \abs{\Phi'''(z)}\leq \Phi'''\lbrb{\Re(z)}$, $\limsupi{y}y\Phi'''(y)/(-\Phi''(y))=K<\infty$ and the choice  $c=\frac{1}{K}$ to get after change of variables $b\to a_*b$, for all $x$ and therefore $a_*$ large enough
	\begin{equation*}
		\begin{split}
			\abs{J_1(a_*,x)}&\leq 2a^l_*\int_{\varepsilon(a_*)}^c\frac{\abs{\Phi^\dagger(a_*(1+ib))\Phi^{k}\lbrb{a_*(1+ib)}}}{\abs{1+ib}^{1-l}}e^{\frac{xb^2a^2_*}{2}\Phi''(a_*)+\frac{xb^3a^3_*}{6}\Phi'''(a_*)}db\\
			&\leq 2a^l_*\int_{\varepsilon(a_*)}^c\frac{\abs{\Phi^\dagger(a_*(1+ib))\Phi^{k}\lbrb{a_*(1+ib)}}}{\abs{1+ib}^{1-l}}e^{\frac{xb^2a^2_*}{2}\Phi''(a_*)\lbrb{1-\frac{2Kc}{3}}}db\\
			&= 2a^l_*\int_{\varepsilon(a_*)}^c\frac{\abs{\Phi^\dagger(a_*(1+ib))\Phi^{k}\lbrb{a_*(1+ib)}}}{\abs{1+ib}^{1-l}}e^{\frac{xb^2a^2_*}{6}\Phi''(a_*)}db.
		\end{split}
	\end{equation*}
	To carry on further we note from \eqref{eq:DeltaR} that 
	\begin{equation*}
		\begin{split}
			&\abs{\frac{\Phi\lbrb{a_*\lbrb{1+ib}}}{\Phi(a_*)}}\leq 3\max\curly{1,b^2}.
		\end{split}
	\end{equation*}
	Then, we have with the form of $\varepsilon(a_*)$, see \eqref{eq:varesp}, that 
	\begin{equation}\label{eq:J1}
		\begin{split}
			&\abs{\frac{a^{1-l}_*\sqrt{-\Phi''(a_*)x}}{\Phi^\dagger(a_*)\Phi^{k}(a_*)}J_1(a_*)}\leq 3^{k+2}a_*\sqrt{-\Phi''(a_*)x}\int_{\varepsilon(a_*)}^c \max\curly{1,b^{2k+2+l}}e^{\frac{xb^2a^2_*}{6}\Phi''(a_*)}db\\
			&\leq 3^{k+2}a_*\sqrt{-\Phi''(a_*)x}\max\curly{1,c^{2k+2+l}}\int_{\varepsilon(a_*)}^ce^{\frac{xb^2a^2_*}{6}\Phi''(a_*)}db\\
			&\leq 3^{k+\frac52}\max\curly{1,c^{2k+2+l}}\int_{3^{-\frac12}g(a_*,x)}^\infty e^{-\frac{u^2}{2}}du=\bo{\frac{e^{-\frac{g^2(a_*,x)}{2}}}{g(a_*,x)}}.
		\end{split}
	\end{equation}
	For $J_2(a_*,x)$ we again change variables $b\to a_*b$, use \eqref{eq:DeltaR} and without application of the Taylor expansion for the exponent to get
	\begin{equation*}
		\begin{split}
			&\abs{\frac{a^{1-l}_*\sqrt{-\Phi''(a_*)x}}{\Phi^\dagger(a_*)\Phi^{k}(a_*)}J_2(a_*,x)}\\
			&\leq 3^{k+1}a_*\sqrt{-\Phi''(a_*)x}\int_{|b|\geq c}^\infty \frac{\max\curly{1,b^{2k+2}}}{\abs{1+ib}^{-l+1}}e^{ -x\lbrb{\Re\lbrb{\Phi(a_*\lbrb{1+ib})}-\Phi(a_*)}}db\\
			&\leq C'a_*\sqrt{-\Phi''(a_*)x}\int_{c}^\infty b^{2k+1+l}e^{ -x\lbrb{\Re\lbrb{\Phi(a_*\lbrb{1+ib})}-\Phi(a_*)}}db\\
			&=C'c^{2k+2+l}a_*\sqrt{-\Phi''(a_*)x}\int_{1}^\infty b^{2k+1+l}e^{ -x\lbrb{\Re\lbrb{\Phi(a_*\lbrb{1+ibc})}-\Phi(a_*)}}db,
		\end{split}
	\end{equation*}
	where $C'>0$ is some constant.   
	Then, with some absolute $c_0>0$, we have that 
	\begin{equation}\label{eq:Delta}
		\begin{split}
			& \Re\lbrb{\Phi(a_*\lbrb{1+ibc})}-\Phi(a_*)=\IntOI \lbrb{1-\cos\lbrb{bca_*y}}e^{-a_*y}\mu(dy)\\
			&\geq c_0^2b^2c^2a^2_*e^{-\frac{1}{bc}}\int_{0}^{\frac{1}{bca_*}}y^{2}\mu(dy)=c_0^2b^2c^2a^2_*e^{-\frac{1}{bc}}\Delta(bca_*).
		\end{split}
	\end{equation}
	Since $\liminfi{y}y^2\Delta(y)/\ln(y)= L>0$ we choose $M<L$. Set $M'=Mc^2_0e^{-\frac{1}{c}}$. On $b\geq 1$ we get that for $x$ and $a_*$ large enough such that $M'x>2k+2+l$
	\begin{equation}\label{eq:J_2}
		\begin{split}
			&\abs{\frac{a^{1-l}_*\sqrt{-\Phi''(a_*)x}}{\Phi^\dagger(a_*)\Phi^{k}(a_*)}J_2(a_*,x)}\leq C'c^{2k+2+l}a_*\sqrt{-\Phi''(a_*)x}\int_{1}^\infty b^{2k+1+l}e^{ -M'x \ln\lbrb{bca_*}}db\\
			&=\frac{C'c^{2k+2+l}}{M'x-2k-2-l}\frac{a_*\sqrt{-\Phi''(a_*)x}}{(a_*c)^{M'x}}\\
			&\leq \frac{C''}{(M'x-2k-2-l)K^{2k+2+l}}\sqrt{x}  K^{M'x}a^{\frac12-M'x}_*=\bo{\frac{K^{M'x}a^{\frac12-M'x}_*}{\sqrt{x}}},
		\end{split}
	\end{equation}
	where $C'>0$ is some constant, we have fixed $c=K^{-1}$ and we have used $-y^{2}\Phi''(y)\leq 2\Phi(y)=\bo{y}$, see Item \eqref{it:ineq} of Lemma \ref{lem:Bern}. Collecting \eqref{def:asympIe1} and employing \eqref{eq:J1} and \eqref{eq:J_2} in \eqref{def:Je}, we get that \eqref{eq:f1} has the following asymptotic form
	\begin{equation*}
		\begin{split}
			&	I(x,t)=\frac{(-1)^k}{\sqrt{2\pi}}e^{a_*t-x\Phi(a_*)}\frac{\Phi^\dagger(a_*)\Phi^{k}(a_*)}{a^{1-l}_*\sqrt{-\Phi''(a_*)x}}\lbrb{1+\bo{\frac{g(a_*,x) }{a_*\sqrt{-\Phi''(a_*)x}}+\frac{e^{-\frac{g^2(a_*,x)}{2}}}{g(a_*,x)}+\frac{K^{M'x}a^{\frac12-M'x}_*}{\sqrt{x}}}}\!.
		\end{split}
	\end{equation*}
	Next, recall that $g(a_*,x)=\sqrt{2\ln\lbrb{a_*\sqrt{-\Phi''(a_*)x}}}$ which converges to infinity thanks to Proposition \ref{prop:D} and we yield asymptotically
	\begin{equation*}
		\begin{split}
			&	I(x,t)=\frac{(-1)^k}{\sqrt{2\pi}}e^{a_*t-x\Phi(a_*)}\frac{\Phi^\dagger(a_*)\Phi^{k}(a_*)}{a^{1-l}_*\sqrt{-\Phi''(a_*)x}}\lbrb{1+\bo{\frac{\sqrt{\ln\lbrb{a_*\sqrt{-\Phi''(a_*)x}}} }{a_*\sqrt{-\Phi''(a_*)x}}+\frac{K^{M'x}a^{\frac12-M'x}_*}{\sqrt{x}}}}.
		\end{split}
	\end{equation*}
	However, since $-y^2\Phi''(y)\leq 2\Phi(y)=\bo{y},$ see Item \eqref{it:ineq} of Lemma \ref{lem:Bern}, we check  that the first expression in the speed of convergence cannot be faster than $(xa)^{-\frac12}_*$. The second expression though is of much faster decay and we conclude that
	\begin{equation*}
		\begin{split}
				I(x,t)&=\frac{(-1)^k}{\sqrt{2\pi}}e^{a_*t-x\Phi(a_*)}\frac{\Phi^\dagger(a_*)\Phi^{k}(a_*)}{a^{1-l}_*\sqrt{-\Phi''(a_*)x}}\lbrb{1+\bo{\frac{\sqrt{\ln\lbrb{a_*\sqrt{-\Phi''(a_*)x}}} }{a_*\sqrt{-\Phi''(a_*)x}}}},
		\end{split}
	\end{equation*}
	which establishes \eqref{asymp}.
	The claim that $a_*\geq Me^{-1}(t/x-\mathfrak{b})^{-1}$ for any $M<L$ and for all $x$ large enough follows immediately from \eqref{eq:ineq} of Proposition \ref{prop:D}. This concludes the proof of the main theorem.
\end{proof}
We proceed with the proof of the next main theorem.
\begin{proof}[Proof of Theorem \ref{thm:main1}]
	Since by assumption $\limi{x}t(x)/x=\bar{x}\in\lbrb{\mathfrak{b},\Phi'(0)}$ then from $\Phi'(a_*)=t(x)/x$ we derive that $\limi{x}a_*=\bar{a}\in\lbrb{0,\infty}$. From Theorem \ref{thm:regularityfphi1} we can write for all $x$ large enough 
	\begin{equation*}
		\begin{split}
			&\frac{\partial^k\partial ^l }{\partial x^k\partial t^l}\fP{x,t}=\frac{(-1)^k}{2\pi}\int_{-\infty}^\infty \frac{\Phi^{\dagger}(\bar{a}+ib)\Phi^{k}(\bar{a}+ib)}{\lbrb{\bar{a}+ib}^{1-l}}e^{-x\Phi(\bar{a}+ib)+t\lbrb{\bar{a}+ib}}db=:I(x,t).
		\end{split}
	\end{equation*}
	Changing variables we easily get
	\begin{equation*}
		\begin{split}
			&I(x,t)=\frac{(-1)^k}{\sqrt{2\pi}}\bar{a}e^{\bar{a}t-x\Phi(\bar{a})}\int_{-\infty}^\infty \frac{\Phi^{\dagger}(\bar{a}(1+ib))\Phi^{k}(\bar{a}(1+ib))}{\lbrb{\bar{a}(1+ib)}^{1-l}}e^{-x\lbrb{\Phi(\bar{a}(1+ib))-\Phi(\bar{a})}+itb\bar{a}}db.
		\end{split}
	\end{equation*}
	Next, we deal with the integral term. Taylor's formula of the exponent yields
	\begin{equation}\label{def:Tay1}
		\begin{split}
			&\Phi(\bar{a}(1+ib))-\Phi(\bar{a})=ib\bar{a}\frac{t}{x}-\frac{b^2\bar{a}^2}{2}\Phi''(\bar{a})-i\frac{b^3\bar{a}^3}{3!}\Phi'''(\bar{a}(1+ib\theta(\bar{a},b))).
		\end{split}
	\end{equation}
	We set $\mathfrak{a}:=\bar{a}\sqrt{-\Phi''(\bar{a})}$ and choose $\varepsilon(x)=x^{-1/2}\sqrt{C\ln(x)},x>1,C>1,$ with $\limi{x}\varepsilon(x)=0$ and $\limi{x}\sqrt{x}\varepsilon(x)=\infty$. Then we split the range of integration on $[-\mathfrak{a}^{-1}\varepsilon(x),\mathfrak{a}^{-1}\varepsilon(x)]$ and its complement and apply \eqref{def:Tay1} in the first case. Thus 
	\begin{equation*}
		\begin{split}
			I_1(x,t)&:=\int_{-\mathfrak{a}^{-1}\varepsilon(x)}^{\mathfrak{a}^{-1}\varepsilon(x)} \frac{\Phi^{\dagger}(\bar{a}(1+ib))\Phi^{k}(\bar{a}(1+ib))}{\lbrb{\bar{a}(1+ib)}^{1-l}}e^{-x\frac{b^2}{2}\mathfrak{a}^2\lbrb{1+i\frac{b\bar{a}^3}{3\mathfrak{a}}\Phi'''(\bar{a}(1+ib\theta(\bar{a},b)))}}db\\
			&=\frac{1}{\mathfrak{a}\sqrt{x}}\int_{-\varepsilon(x)\sqrt{x}}^{\varepsilon(x)\sqrt{x}}\frac{\Phi^{\dagger}(\bar{a}(1+i\frac{u}{\sqrt{\mathfrak{a}x}}))\Phi^{k}(\bar{a}(1+i\frac{u}{\sqrt{\mathfrak{a}x}}))}{\lbrb{\bar{a}(1+i\frac{u}{\sqrt{\mathfrak{a}x}})}^{1-l}} e^{-\frac{u^2}{2}\lbrb{1-i\frac{u\bar{a}^3}{\sqrt{x}\mathfrak{a}^{3}}\Phi'''(\bar{a}(1+i\frac{u}{\mathfrak{a}\sqrt{x}}\theta(\bar{a},\frac{u}{\mathfrak{a}\sqrt{x}})))}}du.
		\end{split}
	\end{equation*}
	Clearly, 
	\[\sup_{|u|\leq \varepsilon(x)\sqrt{x}}\abs{\frac{u\bar{a}^3}{\mathfrak{a}^{3}\sqrt{x}}\Phi'''(\bar{a}(1+i\frac{u}{\mathfrak{a}\sqrt{x}}\theta(\bar{a},\frac{u}{\mathfrak{a}\sqrt{x}})))}=\bo{\varepsilon(x)},\]
	where we have used again that for $\Re(z)>0$, $\abs{\Phi'''(z)}\leq \Phi'''(\Re(z))$, see (3.11) of Item \ref{it:real} of Lemma \ref{lem:Bern}. Note that by Taylor's formula also
	\begin{equation*}
		\begin{split}
			\sup_{|u|\leq \varepsilon(x)\sqrt{x}}\abs{\Phi\lbrb{\bar{a}\lbrb{1+i\frac{u}{\mathfrak{a}\sqrt{x}}}}-\Phi\lbrb{\bar{a}}}\leq \bar{a}\varepsilon(x)\Phi'(\bar{a})\leq \varepsilon(x)\Phi(\bar{a})=\bo{\varepsilon(x)},\\
			\sup_{|u|\leq \varepsilon(x)\sqrt{x}}\abs{\Phi^{\dagger}\lbrb{\bar{a}\lbrb{1+i\frac{u}{\mathfrak{a}\sqrt{x}}}}-\Phi^\dagger\lbrb{\bar{a}}}\leq \bar{a}\varepsilon(x)(\Phi^\dagger)'(\bar{a})\leq \varepsilon(x)\Phi(\bar{a})=\bo{\varepsilon(x)},
		\end{split}
	\end{equation*}
	where we have used that $\abs{\Phi'(z)}\leq \Phi'(\Re(z))$, provided $\Re(z)>0$, see Item \ref{it:real} of Lemma \ref{lem:Bern}, $x\Phi'(x)\leq \Phi(x)$, see Item \ref{it:ineq} of Lemma \ref{lem:Bern}, and $\Phi^\dagger(\bar{a})\leq \Phi(\bar{a})$. Therefore, with an easier computation but similar to the one leading to \eqref{def:asympIe1}, we get that
	\begin{equation}\label{def:asympIe2}
		\begin{split}		
			&I_1(x,t)=\frac{\Phi^\dagger(\bar{a})\Phi^{k}(\bar{a})}{\bar{a}^{1-l}}\frac{1}{\sqrt{-\Phi''(\bar{a})x}} \lbrb{1+\bo{\varepsilon(x)}}.
		\end{split}
	\end{equation}
	From \eqref{def:asympIe2} it remains to show that remaining part of the integral, namely
	\begin{equation*}
		\begin{split}
			\sqrt{x}I_2(x,t)&:=\sqrt{x}\int_{|b|\geq \mathfrak{a}^{-1}\varepsilon(x)} \frac{\Phi^{\dagger}(\bar{a}(1+ib))\Phi^{k}(\bar{a}(1+ib))}{\lbrb{\bar{a}(1+ib)}^{1-l}}e^{-x\lbrb{\Phi(\bar{a}(1+ib))-\Phi(\bar{a})}+itb\bar{a}}db
		\end{split}
	\end{equation*}
	converges to zero and evaluate its speed of convergence. Since $\bar{a}$ is a fixed number then as in \eqref{eq:J1} it suffices to prove that 
	\begin{equation*}
		\begin{split}
			\sqrt{x}\abs{I_2(x,t)}&\leq J(t,x)=\sqrt{x}\int_{|b|\geq \mathfrak{a}^{-1}\varepsilon(x)} \max\curly{1,|b|^{2k+2+l}}e^{-x\lbrb{\Re\lbrb{\Phi(\bar{a}(1+ib))}-\Phi(\bar{a})}}db
		\end{split}
	\end{equation*}
	goes to zero and evaluate its speed of convergence. We simplify to
	\begin{equation*}
		\begin{split}
			J(t,x)&\leq \sqrt{x}\int_{1\geq |b|\geq \mathfrak{a}^{-1}\varepsilon(x)} e^{-x\lbrb{\Re\lbrb{\Phi(\bar{a}(1+ib))}-\Phi(\bar{a})}}db\\
			&+\sqrt{x}\int_{|b|>1}|b|^{2k+2+l}e^{-x\lbrb{\Re\lbrb{\Phi(\bar{a}(1+ib))}-\Phi(\bar{a})}}db=:J_1(t,x)+J_2(t,x).
		\end{split}
	\end{equation*}
Choosing some $C'>0$ large enough we can repeat the estimate in \eqref{eq:Delta} to get, for $b\leq 1$,
\begin{equation*}
	\begin{split}
		&\Re\lbrb{\Phi(\bar{a}(1+ib))}-\Phi(\bar{a})\geq e^{-\frac{1}{C'}}b^2\int_{0}^{\frac{1}{\bar{a}C'}}y^2\mu(dy)=C''b^2
	\end{split}
\end{equation*}
where $C''$ is a constant. Thus with $C>3\mathfrak{a}^{2}/C''$ in the definition of $\varepsilon(x)$
\begin{equation}\label{eq:JJ1}
	\begin{split}
		J_1(t,x)&\leq \sqrt{x}e^{-\frac{C''}{\mathfrak{a}^{2}}x\varepsilon^2(x)}\leq \sqrt{x}e^{-3\ln(x)}=\frac{1}{x}.
	\end{split}
\end{equation}
For the other case, when $|b|>1$ we directly apply \eqref{eq:Delta} to get  
\begin{equation*}
	\begin{split}
		&\Re\lbrb{\Phi(\bar{a}(1+ib))}-\Phi(\bar{a})\geq c_0^2b^2\bar{a}^2e^{-1}\Delta(b\bar{a}).
	\end{split}
\end{equation*}
Hence
\begin{equation*}
	\begin{split}
		J_2(t,x)&\leq \sqrt{x}\int_{|b|>1}|b|^{2k+2+l}e^{-xc_0^2b^2\bar{a}^2e^{-1}\Delta(b\bar{a})}db\\
		&\leq \sqrt{x}A^{2k+3+l}e^{-xc_0^2e^{-1}\min_{b\in\lbbrbb{1,A}}\curly{b^2\bar{a}^2\Delta(b\bar{a})}}+\sqrt{x}\int_{|b|>A}|b|^{2k+2+l}e^{-xMc_0^2\ln\lbrb{b\bar{a}}}db,
	\end{split}
\end{equation*}
where $A$ is chosen such that $(b\bar{a})^2\Delta(b\bar{a})/\ln(b\bar{a})> M$ on $|b|>A$, for some $M<L$, which is possible due to condition \eqref{def:condiA}, and $A\bar{a}>1$. Clearly, for large $x$, we get that
\begin{equation*}
	\begin{split}
		J_2(t,x)&\leq \frac1x.
	\end{split}
\end{equation*}
	This and \eqref{eq:JJ1} yield immediately with $\varepsilon(x)=\sqrt{C\ln(x)}x^{-\frac12}$
	\ that
	\begin{equation*}
		\begin{split}
			I(x,t)&=\frac{(-1)^k\Phi^\dagger(\bar{a})\Phi^{k}(\bar{a})}{\sqrt{2\pi}\bar{a}^{1-l}}\frac{e^{\bar{a}t-x\Phi(\bar{a})}}{\sqrt{x}\sqrt{-\Phi''(\bar{a})}} \lbrb{1+\bo{\sqrt{\ln(x)}x^{-\frac12}}}.
		\end{split}
	\end{equation*}
	This concludes the proof of the theorem.
\end{proof}
We will be economical with the next proof as it follows the main steps of the proof of Theorem \ref{thm:mainL}.
\begin{proof}[Proof of Theorem \ref{thm:main2}] Since $t/x\to \phi'(0^+)$ then from \eqref{eq:a*} we have that $\limi{x}a_*=0$.
	Then, using \eqref{intreprder1} for $x>x_0(k+l,L)$ we write
	\begin{equation}\label{def:derf1}
		\begin{split}
				\frac{\partial^k\partial ^l }{\partial x^k\partial t^l}\fP{x,t}&=\frac{(-1)^k}{2\pi}e^{a_*t-x\Phi(a_*)}\lbrb{\int_{\Ic_{\varepsilon}}+\int_{\Ic^c_{\varepsilon}}}J(a_*,b)e^{ibt -x\lbrb{\Phi(a_*+ib)-\Phi(a_*)}}db,
		\end{split}
	\end{equation}
where $\Ic_{\varepsilon}=\lbbrbb{-\varepsilon(a_*)a_*,\varepsilon(a_*)a_*}$, $J$ defined as in \eqref{def:Je1} and $\varepsilon(a_*)$ as in \eqref{eq:varesp}. The latter makes sense due to the first assumption in \eqref{eq:addCondi}. Absolutely the same estimates hinging on \eqref{def:condiA}, \eqref{def:condiB'} and \eqref{eq:varesp}  and employing  $g(a_*,x)=\sqrt{2\ln\lbrb{a_*\sqrt{-\Phi''(a_*)x}}}$ lead to
	\begin{equation}\label{I'2'}
	\begin{split}
		&\int_{\Ic_{\varepsilon}}J(a_*,b)e^{ibt -x\lbrb{\Phi(a_*+ib)-\Phi(a_*)}}db\\
		&=\frac{\sqrt{2\pi}}{\sqrt{-\Phi''(a_*)x}}\frac{\Phi^\dagger(a_*)\Phi^{k}(a_*)}{a^{1-l}_*}\lbrb{1+\bo{\frac{g(a_*,x) }{a_*\sqrt{-\Phi''(a_*)x}}}+\bo{\frac{e^{-\frac{g^2(a_*,x)}{2}}}{g(a_*,x)}}}.
	\end{split}
\end{equation}
We note that since $\limi{x}a_*(x)=0$ and \eqref{eq:addCondi} hold then \eqref{H2} is true with speed $\so{(-a^2_*\phi''(a_*)x)^{-\frac12+\epsilon}}$ for any $\epsilon>0$ small enough. Next, we again decompose the remaining term to
	\begin{equation}\label{def:Je2}
	\begin{split}
		J_\varepsilon(x,t)&:=\frac{(-1)^k}{2\pi}e^{a_*t-x\Phi(a_*)}\int_{\varepsilon(a_*)a_*\leq |b|\leq ca_*}\!\!\!\!\!\!\!\!\!\!\!\!\!\!\!\!\frac{\Phi^\dagger(a_*+ib)\Phi^{k}(a_*+ib)}{\lbrb{a_*+ib}^{1-l}}e^{ibt -x\lbrb{\Phi(a_*+ib)-\Phi(a_*)}}db\\
		&+\frac{(-1)^k}{2\pi}e^{a_*t-x\Phi(a_*)}\int_{|b|\geq ca_*}\frac{\Phi^\dagger(a_*+ib)\Phi^{k}(a_*+ib)}{\lbrb{a_*+ib}^{1-l}}e^{ibt -x\lbrb{\Phi(a_*+ib)-\Phi(a_*)}}db\\
		&=:\frac{(-1)^k}{2\pi}e^{a_*t-x\Phi(a_*)}\lbrb{J_1(a_*,x)+J_2(a_*,x)},
	\end{split}
\end{equation}
and get the same way $J_1(a_*,x)$ estimated as in \eqref{eq:J1}. Also, in the same fashion
\begin{equation*}
	\begin{split}
		&\abs{\frac{a^{1-l}_*\sqrt{-\Phi''(a_*)x}}{\Phi^\dagger(a_*)\Phi^{k}(a_*)}J_2(a_*,x)}=C'c^{2k+2+l}a_*\sqrt{-\Phi''(a_*)x}\int_{1}^\infty b^{2k+1+l}e^{ -x\lbrb{\Re\lbrb{\Phi(a_*\lbrb{1+ibc})}-\Phi(a_*)}}db,
	\end{split}
\end{equation*}
where $C'>0$ is a constant. Here, however, there is a difference in that $a_*\to 0$ and we  estimate the exponent in two different ways. Pick $A>1$. Then on $b\leq A/(ca_*)$ the difference in the exponent is estimated by a constant as in the bound for $J_2(t,x)$ in the proof of Theorem \ref{thm:main1}. Thus, for some $a>0$,
\[\int_{1}^{\frac{A}{ca_*}} b^{2k+1+l}e^{ -x\lbrb{\Re\lbrb{\Phi(a_*\lbrb{1+ibc})}-\Phi(a_*)}}db\leq Ce^{-ax}.\]
Then, for any $M<L$, see \eqref{def:condiA}, choose $A$ such that, for $b>A/(ca_*)$, as in \eqref{eq:Delta},
\begin{equation*}
	\begin{split}
		& \Re\lbrb{\Phi(a_*\lbrb{1+ibc})}-\Phi(a_*)\geq c_0^2b^2c^2a^2_*e^{-\frac{1}{bc}}\Delta(bca_*)\geq Mc^2_0e^{-\frac{a_*}{A}}\ln\lbrb{bca_*}.
	\end{split}
\end{equation*}
Therefore, for all $x$ large enough and hence $a_*$ small enough we get with some $M'$ as large as we wish
\begin{equation*}
	\begin{split}
		&\int_{\frac{A}{ca_*}}^\infty b^{2k+1+l}e^{ -x\lbrb{\Re\lbrb{\Phi(a_*\lbrb{1+ibc})}-\Phi(a_*)}}db\leq \int_{\frac{A}{ca_*}}^\infty b^{2k+1+l}\lbrb{bca_*}^{-M'x}db\\
		&=a^{-2k-l-2}_{*}c^{-M'x}\int_{\frac{A}{c}}^\infty u^{2k+1+l}u^{-M'x}du=a^{-2k-l-2}_{*}c^{-2k-2-l}A^{-M'x}\frac{1}{M'x-2k-l-2}.
	\end{split}
\end{equation*}
Plugging this and the estimate above we arrive thanks to the second and third requirement of  \eqref{eq:addCondi}, with some $C_0>0$ and $a'>0$, at the bound which settles the claim
\begin{equation*}
	\begin{split}
		\abs{\frac{a^{1-l}_*\sqrt{-\Phi''(a_*)x}}{\Phi^\dagger(a_*)\Phi^{k}(a_*)}J_2(a_*)}&\leq C_0 a_*\sqrt{-\Phi''(a_*)x}\lbrb{e^{-ax}+\frac{2}{M' x}e^{-M'x\ln(A)+(2k+l+2)\ln\lbrb{\frac{1}{a_*}}}}\\
		&\leq \abs{\frac{a^{1-l}_*\sqrt{-\Phi''(a_*)x}}{\Phi^\dagger(a_*)\Phi^{k}(a_*)}J_2(a_*,x)}\leq 2C_0 a_*\sqrt{-\Phi''(a_*)x}e^{-a'x}\leq 2C_0 e^{-a''x}.
	\end{split}
\end{equation*} 
\end{proof}