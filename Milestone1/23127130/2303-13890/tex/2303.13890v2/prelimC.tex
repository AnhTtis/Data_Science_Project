\section{Preliminaries}\label{sec:prelim}



We assume everywhere that we work on the standard (for subordinators) probability space, namely the space of c\'adl\'ag functions on $\lbbrbb{0,\infty}$ endowed with the Skorohod topology, the sigma-algebra generated by it and a suitable  probability measure on the latter. 

Let $\sigma=\lbrb{\sigma(t)}_{t \ge 0}$ be a possibly killed one-dimensional subordinator, that is to say that there exists a conservative subordinator $\sigma^{\vartriangle}:=\lbrb{\sigma^{\vartriangle}(t)}_{t \ge 0}$ and an independent exponential random variable $\mathbf{e}_q$ with parameter $q \ge 0$ (where, if $q=0$, we set $\mathbf{e}_0=\infty$) such that $\sigma(t)=\sigma^{\vartriangle}(t)$ for any $t<\mathbf{e}_q$, while $\sigma(t)=\infty$ for any $t \ge \mathbf{e}_q$.
  Each $\sigma$ is uniquely determined (in law) by a Bernstein function in the following manner
\begin{equation}\label{def:Phi}
	\begin{split}
		\log\Ebb{e^{-z\sigma(1)}}=\phi(z)&=q+\mathfrak{b}z+\IntOI \lbrb{1-e^{-zy}}\mu_\phi(dy)\\
		&=q+\mathfrak{b}z+z\IntOI e^{-zy}\bar{\mu}_\phi(y)dy,\,\,\Re(z)\geq 0,
	\end{split}
\end{equation}
where $q,\mathfrak{b}\geq 0$, $\mu_\phi$ is a sigma-finite measure on $\lbrb{0,\infty}$ satisfying the integrability condition \linebreak $\IntOI \min\curly{y,1}\mu_\phi(dy)<\infty$ (called the L\'evy measure of $\sigma$) and $\bar{\mu}_\phi(x)=\int_{x}^{\infty}\mu(dy),x\geq 0,$ is the tail of $\mu_\phi$. In fact the right-hand side of \eqref{def:Phi} serves as an equivalent definition of Bernstein functions and thus they are in  bijection with the potentially killed subordinators. Due to their importance in other areas of mathematics, Bernstein functions have been studied in detail. An exposition on the current knowledge of their properties can be found in \cite{librobern} and some additional information is scattered in references such as \cite{BivBernGam,bernsteingamma,Laguerre}. Classical references for \LLPs and subordinators, in particular, are the books \cite{bertoinb,bertoins,kyprianou}. 

With each $\sigma$ one defines the (right-)inverse-subordinator  $L:=\lbrb{L(t)}_{t \ge 0}$ via the passage times
\begin{equation}\label{def:invS}
	L(t):=\inf\curly{s>0:\sigma(s)>t}.
\end{equation}
Note that $\Pbb{L(t)=\infty}=0$ for every $t \in [0, \infty)$, even if $q>0$, since for $t>\sigma(\mathbf{e}_q-)$ the process $L(t)$ remains stuck in the position $\mathbf{e}_q$.
Also, note that the paths of $L$ are almost surely continuous if and only if $\mathfrak{b}\neq 0$ or $\bar{\mu}_\phi(0)=\infty$, that is $\sigma$ is not a pure-jump compound Poisson process. When $q=\mathfrak{b}=0$ and $\bar{\mu}_\phi(0)=\infty$, it is well-known, see \cite[Theorem 3.1]{meertri}, that, for any $t>0$, $L(t)$ admits density on $(0,\infty)$  and the requirement $q=0$ is immediately seen to be unnecessary. Indeed, it is not hard to see, by a conditioning argument using independence between $\mathbf{e}_q$ and $\sigma^{\vartriangle}$, that
\begin{align}
\P \lb L(t) \leq x \rb \, = \, 1-e^{-qx} + e^{-qx} \P \lb L^\Delta(t) \leq x  \rb,
\label{distr}
\end{align}
where we used $L^\vartriangle$ for the inverse of $\sigma^\vartriangle$. Furthermore, if $q=0$, i.e., if we are working with the processes $\sigma^\vartriangle$ and its inverse $L^\vartriangle$, the measure $\Pbb{L^\vartriangle(t)\in dx, \ \sigma^\vartriangle(L^\vartriangle(t))>t}$ has a density on $(0,t/\mathfrak{b})$, where we set $t/0=\infty$, (see \cite[Lemma 11]{DonRiv2} and \cite[prior to Theorem 2.3]{DonRiv}), given by the relation
\begin{equation}\label{def:ftriangle}
	\begin{split}
		&\Pbb{L^\vartriangle(t)\in dx, \sigma^\vartriangle(L^\vartriangle(t))>t}=\int_{0}^t \bar{\nu}_{\Phi}(t-y)\Pbb{\sigma^\vartriangle(x)\in dy}dx=:f_{\Phi}^\vartriangle (x,t)dx.
	\end{split}
\end{equation}
However, since $\Pbb{\sigma^\vartriangle(L^\vartriangle(t))>t}<1$ if and only if $\mathfrak{b}>0$, see \cite[Propositions 1.7 and 1.9]{bertoins}, we observe that if $\mathfrak{b}=0$ then $f_\phi^\vartriangle$ is the density of $L^\vartriangle$. Otherwise, it stands for the density of $L^\vartriangle(t)$ on the event that $\sigma^\vartriangle$ jumps across $t$. On the event that $\sigma^\vartriangle$ creeps up on $t$, i.e. on $\sigma^\vartriangle(L^\vartriangle(t))=t$, $L^\vartriangle(t)$ does not necessarily admit a density. However, it has been shown, in the proof of \cite[Lemma 3.3]{DonRiv}, that if $\sigma^\vartriangle(x)$ admits a density $g_\phi^\vartriangle(x,\cdot)$ on $(\mathfrak{b}x,\infty)$, then $L^\vartriangle (t)$ on the event $\sigma^\vartriangle(L^\vartriangle(t))=t$ admits a density in $(0,t/\mathfrak{b})$  and such a density is given by
%
%
%We denote the creep part of the density as
\begin{align}
f_\phi^{c,\vartriangle}(x,t)dx:= \P \lb L^\vartriangle (t) \in dx, \sigma^\vartriangle(L^\vartriangle(t))=t \rb=\mathfrak{b}g_\phi^\vartriangle(x,t)dx
\label{def:fctriangle}
\end{align}
%and using the notation
%\begin{align}
%g_\phi^\vartriangle(x,t) dt \, := \, \P \l \sigma^\vartriangle(x) \in dt \rb,
%\label{def:gphitriangle}
%\end{align}
%whenever such a density exists, we know from \cite{DonRiv} that 
%$f^{c,\vartriangle}_{\Phi}(x,t)=\mathfrak{b}g_\phi^\vartriangle(x,t)$.
Plugging \eqref{def:ftriangle} and \eqref{def:fctriangle} into \eqref{distr}, we get
\begin{align}
\P \lb L(t) \leq x \rb \, = \, 1-e^{-qx} + e^{-qx} \int_0^x \left[ \int_0^t \bar{\nu}_\phi(t-y)g_\phi^\vartriangle(s,y)dy + \mathfrak{b}g_\phi^\vartriangle (s,t) \right] \, ds.
\label{distrexpl}
\end{align}
Furthermore, if $\sigma^\vartriangle(x)$ admits a density $g_\phi^\vartriangle(x,\cdot)$ on $(\mathfrak{b}x,\infty)$, so does $\sigma(x)$ with  density given by
\begin{align}
	g_\phi(x,t)dt \, := \, \P \lb \sigma(x) \in dt \rb \, = \, e^{-qx} g_\phi^\vartriangle (x,t)dt.
	\label{def:gphi}
\end{align}
Differentiating \eqref{distrexpl} in $x$, we know that $L(t)$ admits a density on $(0,t/\mathfrak{b})$, that is given by
\begin{align}
	\P \lb L(t) \in dx \rb \, = \, & q \P \lb \sigma(x) \leq t \rb dx &=:f_\phi^{\rm k}(x,t)dx \label{def:fk} \\
	&     + \left(\int_0^t \bar{\nu}_\phi(t-y)g_\Phi(x,y)dy\right) dx & =:f_\phi(x,t)dx \label{def:f}  \\
	& +\mathfrak{b} g_\phi(x,t)dx &=:f_\phi^{\rm c}(x,t)dx.\label{def:fc}
\end{align}
The functions $f_\phi$, $f_\phi^{\rm k}$ and $f_\phi^{\rm c}$ are defined on the set
\begin{equation}\label{def:Reg}
	\begin{split}
		&\mathbb{D}:=\curly{\lbrb{t,x}\in\Rb^2:\,t>0, 0<x<\frac{t}{\mathfrak{b}}}
	\end{split}
\end{equation}
and then extended to $0$ outside $\mathbb{D}$.

Our main results concern the quantities \eqref{def:gphi} and \eqref{def:f}. They will be applied to \eqref{def:fk} and \eqref{def:fc}.

Despite the fact that the formulations of $f_\phi^{\rm k}$, $f_\phi$ and $f_\phi^{\rm c}$ in \eqref{def:fk}, \eqref{def:f} and \eqref{def:fc} are quite implicit, their Laplace transforms in the variable $t$ can be expressed in terms of $\Phi$ in a simple way. Indeed, for any $z \in \C$ with $\Re(z)>0$, we have
\begin{equation}\label{eq:LT0}
		\int_{0}^\infty f^{\rm k}_\Phi(x,t)e^{-z t}dt=\frac{q}{z}e^{-x\Phi(z)}, \qquad \qquad \int_{0}^\infty f^{\rm c}_\Phi(x,t)e^{-z t}dt=\mathfrak{b}e^{-x\Phi(z)}
\end{equation}
and
\begin{equation}\label{eq:LT1}
	\int_{0}^\infty f_\Phi(x,t)e^{-z t}dt=\frac{\Phi^\dagger\lbrb{z}}{z}e^{-x\Phi(z)},
\end{equation}	
where
\begin{equation}\label{def:Pdag}
	\begin{split}
		&\Phi^\dagger(z):=\Phi(z)-q-\mathfrak{b}z.
	\end{split}
\end{equation}
%is the unkilled and driftless version of \eqref{def:Phi}.
%	 While the Lapace tranforms of $f_\phi^{\rm k}(x,\cdot)$ and $f_\phi^{\rm c}(x,\cdot)$ By the convolution theorem of Laplace transform (see \cite[Proposition 1.6.4]{abhn}) it is clear that the Laplace transform in time of $f_{\Phi}(x,t)$ is given by 
%\begin{equation}\label{eq:LT1}
%	\int_{0}^\infty f_\Phi(x,t)e^{-z t}dt=\frac{\Phi^\dagger\lbrb{z}}{z}e^{-x\Phi(z)},\,\,\Re\lbrb{z}\geq 0,
%\end{equation}
%where 
Expression \eqref{eq:LT1} is the starting point of our study. 

Let us now fix some notation. Here and hereafter, we use $\Cb$ for the complex plane and we write $z=a+ib=\Re(z)+i\Im(z)$. We denote, for any $a \in \R$, $\mathbb{H}_a:=\{z \in \mathbb{C}: \ \Re z>a\}$, for any  $\alpha, \beta \in (-\pi, \pi]$
\begin{equation*}
	\C(\alpha,\beta):=\{z \in \C: \ \beta<{\rm Arg}(z)<\alpha, |z|>0\}
\end{equation*}
and $\C(\alpha):= \{z \in \C: |{\rm Arg}(z)|<\alpha, |z|>0\}$.

Throughout the paper, we use $C$ to denote any constant whose value is inessential. If needed we underline the dependence  of $C$ on some parameters $p_1,p_2,\dots$ by using  $C(p_1,p_2,\dots)$.

We use $\so{\cdot},\bo{\cdot}$ in the standard fashion with, e.g. $\so{g(x)}$, as $x \to a$, denoting a generic function $f$ such that $\lim_{x \to a}|f(x)|/|g(x)|=0$, while $\bo{g(x)}$, as $x \to \infty$, meaning a generic function $f$ with $\limsup_{x \to a}|f(x)|/|g(x)|\leq C<\infty$. The same notation is also reserved for functions on regions of $\Cb$. Furthermore, we say that $f \asymp g$, as $x \to a$, if $f=\bo{g(x)}$ and $g=\bo{f(x)}$, as $x \to a$.
For any two functions $f,g:[0,+\infty) \to \R$, we denote the convolution product of $f$ and $g$ as
\begin{equation*}
	(f \ast g)(t)=\int_0^t f(s)g(t-s)ds, \ t \ge 0.
\end{equation*}
Furthermore, we denote the convolution powers as
\begin{equation*}
	f^{\ast 0}(t)=\delta_0 \qquad f^{\ast 1}(t)=f(t) \qquad f^{\ast n}(t)=(f \ast f^{\ast (n-1)})(t), \ n \ge 2.
\end{equation*}


The next lemma collects well-known properties of Bernstein functions used throughout this paper.
\begin{lem}\label{lem:Bern}
	Let $\phi$ be a non-zero Bernstein function. Then
	\begin{enumerate}
		\item\label{it:phi} $\phi$ is non-decreasing on $\lbbrb{0,\infty}$ with $\phi(\infty)=\infty\iff \bar{\mu}_\phi(0)=\infty\text{ or }\mathfrak{b}>0$;
		\item \label{it:sign} for any $z \in \overline{\mathbb{H}_0}$ we have that $\Re \Phi(z) \geq 0$;
		\item\label{it:phi'} $\phi'$ is completely monotone on $(0,\infty)$ with $\limi{x}\phi'(x)=\mathfrak{b}$, that is, for all $n\geq 1$, it holds that  $\minusone^{n-1}\phi^{(n)}(x)\geq 0$ on $(0,\infty)$; as a result $\phi''<0$ and $\phi'''>0$ on $(0,\infty)$ and $\phi'(0^+)<\infty\iff \IntOI y\mu_\phi(dy)<\infty$;
		\item\label{it:ineq} for all $x>0$ it holds that $x\phi'(x)\leq \phi(x)$ and $-x^2\phi''(x)\leq 2\phi(x)$;
		\item\label{it:real} for any $z\in \mathbb{H}_0$ and any $n\geq 1$ it holds that $\abs{\phi^{(n)}(z)}\leq \abs{\phi^{(n)}\lbrb{\Re(z)}}$;
		\item\label{it:asymp} for any $\phi$ it holds that $\abs{\phi(z)}=\mathfrak{b}|z|\lbrb{1+\so{1}}$, as $z \to \infty$ uniformly on $\overline{\mathbb{H}_0}$, and if $\phi(0)=q>0,\phi(\mathfrak{u}_\phi)=0$, where $\mathfrak{u}_\phi$ is the first zero of $\phi$ on $(-\infty,0)$ and $\phi$ extends analytically (continuously at the boundary) at least to $\overline{\mathbb{H}_{\mathfrak{u}_\phi}}$ then  $\abs{\phi(z)}=\mathfrak{b}|z|\lbrb{1+\so{1}}$, as $z \to \infty$ uniformly on the same region $\overline{\mathbb{H}_{\mathfrak{u}_\phi}}$;
		\item\label{it:compos} if $\phi_1,\phi_2$ are two Bernstein function, then $z \in [0,+\infty) \mapsto \phi_1(\phi_2(z))) \in \R$ is a Bernstein function;
		\item\label{it:limit} if $(\phi_n)_{n \ge 0}$ is a sequence of Bernstein functions and $\phi:[0,+\infty) \to \R$ is such that $\lim_{n \to \infty}\phi_n(z)=\phi(z)$ for any $z>0$, then $\phi$ is a Bernstein function.
	\end{enumerate}
\end{lem}
\begin{rmk}\label{rem:Bern}
	Items \ref{it:phi}, \ref{it:phi'}, \ref{it:compos} and \ref{it:limit} are standard and can be found in \cite{librobern}. Item \ref{it:sign} is in fact Item 9 of \cite[Proposition 3.1]{bernsteingamma}. Item \ref{it:ineq} can be located in \cite[(3.3) of Proposition 3.1]{bernsteingamma}, whereas item \ref{it:real} is contained in \cite[(3.11) of
	Proposition 3.3]{BivBernGam}. Item \ref{it:asymp} is taken from \cite[Proposition 3.1, Item (4)]{bernsteingamma}.
\end{rmk}
We will always make use of the following property.
\begin{prop}\label{prop:D2}
	For any Bernstein function $\Phi$ and for any $a>0,|b|>0$, it holds true that
	\begin{equation}\label{eq:DeltaR}
		\begin{split}
			&\abs{\frac{\Phi\lbrb{a\lbrb{1+ib}}}{\Phi(a)}}\leq 3\max\curly{1,b^2}.
		\end{split}
	\end{equation}
\end{prop}
\begin{proof}
	We have that
	\begin{equation*}
		\begin{split}
			&\abs{\frac{\Phi\lbrb{a\lbrb{1+ib}}}{\Phi(a)}}=	\abs{1+\frac{\Re\lbrb{\Phi\lbrb{a\lbrb{1+ib}}}-\Phi(a)}{\Phi(a)}+i\frac{\Im\lbrb{\Phi\lbrb{a\lbrb{1+ib}}}}{\Phi(a)}}.
		\end{split}
	\end{equation*}
	From  inequality \cite[(4.26)]{Laguerre}  and since $y\Phi'(y)\leq \Phi(y)$, see Item \eqref{it:ineq} of Lemma \ref{lem:Bern}, we have that 
	\[\frac{\abs{\Im\lbrb{\Phi\lbrb{a\lbrb{1+ib}}}}}{\Phi(a)}\leq \abs{b}a\frac{\Phi'(a)}{\Phi(a)}\leq |b|.\]
	Also, from (4.25) in \cite{Laguerre} and $y\abs{\Phi''(y)}\leq 2\Phi(u)$, see Item \eqref{it:ineq} if Lemma \ref{lem:Bern}, we deduce that
	\begin{equation*}
		\begin{split}
			&\frac{\Re\lbrb{\Phi\lbrb{a\lbrb{1+ib}}}-\Phi(a)}{\Phi(a)}\leq \frac{b^2a^2}{2}\frac{-\Phi''(a)}{\Phi(a)}\leq b^2.
		\end{split}
	\end{equation*}
	Using $|1+z|\leq 1+\abs{\Re(z)}+\abs{\Im(z)}$ we get \eqref{eq:DeltaR}. 
\end{proof}


In this paper and its examples we use the class of complete Bernstein functions. Recall that a Bernstein function $\phi$ is said to be complete if the L\'evy measure $\mu_\Phi$ admits a completely monotone density. The next lemma collects some well-known facts on complete Bernstein functions
\begin{lem}\label{lem:CBern}
	The following properties hold true:
	\begin{enumerate}
		\item \label{it:analy1} A non-negative function $\phi:(0,+\infty) \to [0,+\infty)$ is a complete Bernstein function if and only if it admits an analytic continuation on $\C \setminus (-\infty,0]$ (that we still denote $\phi$) such that $\Im(z)\Im(\Phi(z))\ge 0$ for any $z \in \C \setminus (-\infty,0]$ and such that $\lim_{(0,+\infty)\ni z \to 0}f(z)$ exists and is real;
		\item \label{it:analy2} A non-negative function $\phi:(0,+\infty) \to [0,+\infty)$ is a complete Bernstein function if and only if it admits an analytic continuation on $\C(0,\pi)$ (that we still denote $\phi$) such that $\Im(\Phi(z))\ge 0$ for any $z \in \C \setminus (-\infty,0]$ and such that $\lim_{(0,+\infty)\ni z \to 0}f(z)$ exists and is  real;
		\item\label{it:composC} if $\phi_1,\phi_2$ are two complete Bernstein functions, then $z \in [0,+\infty) \mapsto \phi_1(\phi_2(z))) \in \R$ is a complete Bernstein function;
		\item\label{it:limitC} if $(\phi_n)_{n \ge 0}$ is a sequence of complete Bernstein functions and $\phi:[0,+\infty) \to \R$ is such that $\lim_{n \to +\infty}\phi_n(z)=\phi(z)$ for any $z>0$, then $\phi$ is a complete Bernstein function;
		\item\label{it:anglim} if $\phi$ is a complete Bernstein function, then for any $\alpha \in (0,\pi)$ one has $\lim_{\C(\alpha) \ni z}\frac{\Phi(z)}{z}=\mathfrak{b}$.
	\end{enumerate}
\end{lem}
\begin{rmk}\label{rem:BernC}
	All the items of the previous lemma can be found in \cite[Chapter $6$ ad $7$]{librobern}.
\end{rmk}
We give here a further characterization of complete Bernstein functions, which is a direct consequence of the previous lemma.
\begin{prop}\label{prop:powchar}
	Let $\phi$ be a Bernstein function and denote for any $\alpha \in (0,1)$, $\phi_\alpha(z):=\phi(z^\alpha)$ for $z \ge 0$. Then the following two properties are equivalent:
	\begin{itemize}
		\item[(i)] $\phi$ is a complete Bernstein function;
		\item[(ii)] There exists a sequence $(\alpha_n)_{n \ge 0}$ in $(0,1)$ with $\alpha_n \to 1$ such that $\phi_{\alpha_n}$ is a complete Bernstein function for any $n \in \N$.
	\end{itemize}
\end{prop}
\begin{proof}
	Clearly, (i) implies (ii) by Item \ref{it:composC} of Lemma \ref{lem:CBern}. To show that (ii) implies (i), observe that $\phi(z)=\lim_{n \to +\infty}\phi_{\alpha_n}(z)$ and then we obtain the desired result by Item \ref{it:limitC} of Lemma \ref{lem:CBern}.
\end{proof}