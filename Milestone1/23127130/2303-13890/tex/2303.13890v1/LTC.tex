\section{Smoothness of the densities and their integral representation}\label{sec:LT}
In this section we prove that the densities $f_\Phi$ and the distributions $G_\Phi$ are smooth under either assumption \eqref{def:condiA} or assumptions \eqref{eq:extensionA3} and \eqref{eq:uniformlimcond}. To do this we will provide some integral representations for them $f_\Phi(x,t)$ obtained by means of Laplace inversion. These representations will be also used to obtain the main results, which are obtained differently depending on the fact that either assumption \eqref{def:condiA} or assumptions \eqref{eq:extensionA3} and \eqref{eq:uniformlimcond} hold and thus we separate the two sets of assumptions.
\subsection{Smoothness and Laplace inversion under $\eqref{def:condiA}$}
We first provide a general condition under which the density $f_\Phi$ is smooth in $\mathbb{D}$ for $x$ large enough and by means of Laplace inversion we give an integral representation of $f_\Phi$.
\begin{prop}\label{prop:LTthetapi}
Let $\Phi$ be the Laplace exponent of a potentially killed subordinator. Assume that for $n \ge 0$ there exist $a>0$ and $x_0\geq 0$ such that, for any $x>x_0$,
\begin{equation}\label{eq:Real1}
		\begin{split}
			\int_{-\infty}^\infty |b|^n e^{-x\Re{\Phi(\ab)}}db<\infty.
		\end{split}
		\end{equation}
%		for any $x>x_0$.
%		Then, for $x>x_0$ and $t>0$,
%		\begin{equation}\label{eq:P1_0}
%			\begin{split}
%				&\int_{-\infty}^\infty \frac{\Phi^{\dagger}(\ab)}{\ab}e^{-x\Phi(\ab)+t\lbrb{\ab}}db
%			\end{split}
%		\end{equation} 
%		is absolutely convergent and, for $x_0<x<t/\mathfrak{b}$,
%		\begin{equation}
%		f_\Phi(x,t) \, = \, \int_{-\infty}^{+\infty} \frac{\Phi^\dagger(a+ib)}{a+ib} e^{t(a+ib)-x\Phi(a+ib)} db.
%		\label{intrepr}
%		\end{equation}
%		Furthermore, if for any $x>x_0$,
%		\begin{equation}\label{eq:Real2}
%		\int_{-\infty}^{+\infty} |b|^n e^{-x \Re \Phi (a+ib)} db < \infty
%		\end{equation}
		Then, for any $k,l \ge 0$ with  $k+l \leq n$, $\displaystyle \frac{\partial^k}{\partial x^k}\frac{\partial^l}{\partial t^l}f_\Phi(x,t)$ is well defined for $(x,t) \in \mathbb{D}$ with $x>x_0$ and
		% $x>x_0$ and $t>0$,
%		\begin{equation}\label{eq:P1}
%			\begin{split}
%				&\int_{-\infty}^\infty \frac{\Phi^{\dagger}(\ab)\Phi^{k}(\ab)}{\lbrb{\ab}^{1-l}}e^{-x\Phi(\ab)+t\lbrb{\ab}}db
%			\end{split}
%		\end{equation} 
		%is absolutely convergent and, for any 
		\begin{equation}
		\frac{\partial^l}{\partial t^l} \frac{\partial^k}{\partial x^k} f_\Phi(x,t) \, = \, (-1)^k \int_{-\infty}^{+\infty} \frac{\Phi^\dagger (a+ib) (\Phi(a+ib))^k }{(a+ib)^{1-l}} e^{t(a+ib)-x\Phi(a+ib)} db,
		\label{intreprder}
		\end{equation}
		where the integral is absolutely convergent.
\end{prop}
\begin{proof}
By using \eqref{eq:LT1} we compute the inverse Laplace transform as in \cite[Item a), Theorem 4.2.21]{abhn}. Precisely, we have, for all fixed $x>0$ and for almost all $t>0$, up to a subsequence,
\begin{equation}
f_\Phi(x,t) \, = \, \text{C-}\lim _{r \to +\infty} \frac{1}{2\pi i}\int_{a-ir}^{a+ir} e^{z t } 	\,  \frac{\Phi^\dagger(z)}{z} e^{-x\Phi(z)} \, dz,
\label{ceslimitdag}
\end{equation}
 where $x_0>0$ is arbitrary and we denote by C-$\lim$ the Cesaro limit, i.e.,
\begin{equation*}
\text{C-}\lim _{r \to +\infty} \int_{a-ir}^{a+ir} e^{z t } 	\,  \frac{\Phi^\dagger(z)}{z} e^{-x\Phi(z)} \, dz \, := \, \lim_{R \to +\infty} \frac{1}{R} \int_0^R \int_{a-ir}^{a+ir} e^{z t } 	\,  \frac{\Phi^\dagger(z)}{z} e^{-x\Phi(z)} \, dz \, dr.
\end{equation*}
In particular it holds by \cite[Theorem 4.1.2]{abhn} 
\begin{equation}
\text{C-}\lim _{r \to +\infty} \int_{a-ir}^{a+ir} e^{z t } 	\,  \frac{\Phi^\dagger(z)}{z} e^{-x\Phi(z)} \, dz \, = \, \lim _{r \to +\infty} \int_{a-ir}^{a+ir} e^{zt } 	\,  \frac{\Phi^\dagger(z)}{z} e^{-x\Phi(z)} \, dz
\label{cesequalpointdag}
\end{equation}
provided that the limit in the right-hand side exists. We set about to prove the latter by noting that
\begin{equation*}
 \int_{a-ir}^{a+ir} e^{z t } 	\,  \frac{\Phi^\dagger(z)}{z} e^{-x\Phi(z)} \, dz \, = \, i\int_{-r}^r e^{(a+ib)t} \frac{\Phi^\dagger (a+ib)}{a+ib} e^{-x\Phi(a+ib)} db.
\end{equation*}
However, by Item \ref{it:asymp} in Lemma \ref{lem:Bern}, we know that
%\begin{equation*}
%\lim_{b \to \pm\infty} \frac{|\Phi^\dagger (a+ib)|}{|a+ib|} \, = %\, 0
%\end{equation*}
%and thus 
the function $|\Phi^\dagger (a+ib)|/|a+ib|$ is continuous and bounded and then bt employing \eqref{eq:Real1} we get
%. Then \eqref{intrepr} follows from
\begin{equation*}
	\int_{-\infty}^{+\infty}\frac{|\Phi^\dagger(a+ib)|}{|a+ib|}\left|e^{t(a+ib)-x\Phi(a+ib)}\right|db 
	\le Ce^{ta}\int_{-\infty}^{+\infty}e^{-x\Re \Phi(a+ib)}db<\infty.
\end{equation*}
Using \eqref{cesequalpointdag} into \eqref{ceslimitdag} we get \eqref{intreprder} for $k,l=0$ and $(x,t) \in \mathbb{D}$ with $x>x_0$.
Next, for any $k+l \le n$ and any $(x,t) \in \mathbb{D}$ with $x>x_0$, we prove that $\displaystyle \frac{\partial^k}{\partial x^k}\frac{\partial^l}{\partial t^l}f_\Phi(x,t)$ is well defined and given by \eqref{intreprder}. Let $t \in [t_1, t_2]$, $x \in [x_1, x_2]$. Without loss of generality we choose $x_1>x_0$ and $x_2 < t_1/\mathfrak{b}$. Then we have
\begin{equation*}
\begin{split}
	\left| \frac{\Phi^\dagger (a+ib) (\Phi(a+ib))^k }{(a+ib)^{1-l}} e^{t(a+ib)-x\Phi(a+ib)} \right|
	\leq \, \left| \frac{\Phi^\dagger (a+ib) (\Phi(a+ib))^k }{(a+ib)^{1-l}} \right| e^{at_2-x_1 \Re \Phi(a+ib)}
\end{split}
\end{equation*}
since $\Re \Phi (a+ib)\geq 0$ by Item \ref{it:sign} in Lemma \ref{lem:Bern}. Furthermore, recall that by Item \ref{it:asymp} in Lemma \ref{lem:Bern}
\begin{equation*}
\lim_{b \to \pm \infty}\left| \frac{\Phi (a+ib)}{a+ib} \right| = \mathfrak{b}.
\end{equation*}
and then
\begin{equation*}
\left| \frac{\Phi^\dagger (a+ib) (\Phi(a+ib))^k }{(a+ib)^{1-l}}\right| \leq C|a+ib|^{k+l} \leq C \lb a^n + |b|^{k+l}  \rb,
\end{equation*}
for some $C>0$. Hence, \eqref{intreprder} follows by noting that
\begin{equation*}
	\begin{split}
		&\int_{-\infty}^{+\infty} \left| \frac{\Phi^\dagger (a+ib) (\Phi(a+ib))^k }{(a+ib)^{1-l}} \right| e^{at_2-x_1 \Re \Phi(a+ib)} \, db \leq C \int_{-\infty}^{+\infty}  (a^n+|b|^n) e^{at_2-x_1 \Re \Phi(a+ib)} \, db < \infty.
	\end{split}
\end{equation*}
Since $k,l \ge 0$ with $k+l \le n$ are arbitrary, the last inequality implies that we can differentiate under the integral $k$ times in $x$ and $l$ times in $t$ for $(x,t) \in \mathbb{D}$ with $x>x_0$, starting from \eqref{intreprder} for $k=l=0$.
\end{proof}
To prove Theorem \ref{thm:regularityfphi1}, it is clear that we have to show \eqref{eq:Real1} for some $x_0=x_0(n,L)$ when assumption \eqref{def:condiA} holds. To do this, we need a preliminary result.
\begin{prop}\label{prop:D0}
	Let $\Phi$ be the Laplace exponent of a potentially killed subordinator  and  \eqref{def:condiA} holds for some $L>0$. Then, $\phi(\infty)=\infty$ and we have, for any $M \in (0,L)$,
	\begin{equation}\label{eq:P}
		\begin{split}
			\liminfi{x}\frac{-x^2\Phi''(x)}{\ln(x)}>Me^{-1},\qquad \liminfi{x}\frac{x\lbrb{\Phi'(x)-\mathfrak{b}}}{\ln(x)}>Me^{-1}.
		\end{split}
	\end{equation} 
\end{prop}
\begin{proof}
	We note from \eqref{eq:relation} that
	\begin{equation*}
		\begin{split}
			&-x^2\Phi''(x)\geq  x^2e^{-1}\Delta(x)
		\end{split}
	\end{equation*}
	and the first claim of \eqref{eq:P} is valid. The second follows by integration of the first and $\Phi(\infty)=\infty$ is a consequence of the integration of the second statement in \eqref{eq:P}.
\end{proof}
Now we are ready to show the smoothness of $f_\Phi$ under assumption \eqref{def:condiA}.
%\begin{prop}\label{prop:D1}
%	Let $\Phi$ be the Laplace exponent of a potentially killed subordinator and assume that \eqref{def:condiA} holds. Then, for any $n\geq 0$ there exists $x_0(n,L)>0$ such that for any $a,t>0$ and any $x>x_0(n,L)$ condition \eqref{eq:Real2} holds. If $L=\infty$ then $x_0(n,\infty)=0$.
%\end{prop}
\begin{proof}[Proof of Theorem \ref{thm:regularityfphi1}]
	First observe that integrability in \eqref{eq:Real1} needs to be established only in  neighbourhood of infinity. We have from the inequality $1-\cos(y)\geq cy^2, y\in\lbbrbb{0,1}, c>0,$ that
	\[\Re\lbrb{\Phi(\ab)}-\Phi(a)=\IntOI \lbrb{1-\cos(by)}e^{-ay}\nu_\Phi(dy)\geq cb^2e^{-\frac{a}b} \int_{0}^{\frac1b}y^2\nu_\Phi(dy).\]
	Clearly, from Proposition \ref{prop:D0}, for any $M \in (0,L)$, we have, for all $|b|>|b_0|>1$, that
	\begin{equation*}
		\begin{split}
			\int_{|b|>|b_0|} \!\! |b|^{n}e^{-x\Re\lbrb{\Phi(\ab)}}db\leq  e^{-x\Phi(a)}\!\!\!\int_{|b|>|b_0|}\!\!|b|^{n} e^{-xce^{-\frac{a}{|b_0|}}b^2\Delta(b)}db \leq e^{-x\Phi(a)}\!\!\!\int_{|b|>|b_0|} \!\! |b|^{n} e^{-xce^{-\frac{a}{|b_0|}}M\ln|b|}db.
		\end{split}
	\end{equation*}
	The latter is finite for $xce^{-\frac{a}{|b_0|}}M>n+1$. Since $|b_0|$ and $M \in (0,L)$ are arbitrary we have integrability of \eqref{eq:Real1} for $x>\frac{n+1}{cL}=:x_0(n,L)$ with $\frac{1}{\infty}=0$. Theorem \ref{thm:regularityfphi1} then follows by Proposition \ref{prop:LTthetapi}.
\end{proof}




\subsection{Smoothness and Laplace inversion under $\eqref{eq:extensionA3}$ and \eqref{eq:uniformlimcond}}

%\begin{prop}
%	Let $\Phi$ be the Laplace exponent of a potentially killed subordinator. Assume that there exists $x_0\geq 0$, such that for any $x>x_0$
%	\begin{equation}\label{eq:Real1pi}
%		\begin{split}
%			\int_{-\infty}^\infty e^{-x\Re{\Phi(ib)}}db<\infty \qquad \mbox{ and }  \qquad 			\int_{0}^1 \frac{|\Im \Phi^\dagger(ib)|}{b}db<\infty.			
%		\end{split}
%	\end{equation}
%	Then, the integral 
%\begin{align}
%\int_0^{+\infty} \Im \l \frac{\Phi^\dagger(i\rho)}{\rho} e^{it\rho-x\Phi(i\rho)} \r \, d\rho
%\end{align}	
%is absolutely convergent for $x>x_0$, $t>0$, and, 
%	for any $x_0<x<t/\mathfrak{b}$ and $t>0$, it is true that
%	\begin{align}
%	f_\Phi(x,t) \, = \, \frac{1}{\pi}\int_{0}^{+\infty} \Im \l \frac{\Phi^\dagger(i\rho)}{\rho} e^{it\rho-x\Phi(i\rho)} \r d\rho.
%	\end{align}
%



%	Then, for $x>x_0$ and $t>0$,
%	\begin{equation}\label{eq:P1_0}
%		\begin{split}
%			&\int_{-\infty}^\infty \frac{\Phi^{\dagger}(\ab)}{\lbrb{\ab}}e^{-x\Phi(\ab)+t\lbrb{\ab}}db
%		\end{split}
%	\end{equation} 
%	is absolutely convergent and, for $x_0<x<t/\mathfrak{b}$,
%	\begin{align}
%		f_\Phi(x,t) \, = \, \int_{-\infty}^{+\infty} \frac{\Phi^\dagger(a+ib)}{a+ib} e^{t(a+ib)-x\Phi(a+ib)} db.
%		\label{intrepr}
%	\end{align}
%	Furthermore, if for any $x>x_0$,
%	\begin{align}\label{eq:Real2}
%		\int_{-\infty}^{+\infty} |b|^n e^{-x \Re \Phi (a+ib)} db < \infty
%	\end{align}
%	then, for any $k,l$ s.t. $k+l \leq n$, $x>x_0$ and $t>0$,
%	\begin{equation}\label{eq:P1}
%		\begin{split}
%			&\int_{-\infty}^\infty \frac{\Phi^{\dagger}(\ab)\Phi^{k}(\ab)}{\lbrb{\ab}^{1-l}}e^{-x\Phi(\ab)+t\lbrb{\ab}}db
%		\end{split}
%	\end{equation} 
%	is absolutely convergent and, for any $x_0<x<t/\mathfrak{b}$,
%	\begin{align}
%		\partial_t^l \partial_x^k f_\Phi(x,t) \, = \, (-1)^k \int_{-\infty}^{+\infty} \frac{\Phi^\dagger (a+ib) (\Phi(a+ib))^k }{(a+ib)^{1-l}} e^{t(a+ib)-x\Phi(a+ib)} db.
%		\label{intreprder}
%	\end{align}
%\end{prop}
%\begin{proof}
%	We use \eqref{ceslimitdag} and \eqref{cesequalpointdag} to say that, for $a>x_0$,
%	\begin{align}
%		f_\Phi(x,t) \, = \, \lim _{b \to +\infty} \frac{1}{2\pi i}\int_{a-ib}^{a+ib} e^{\lambda t } 	\,  \frac{\Phi^\dagger(\lambda)}{\lambda} e^{-x\Phi(\lambda)} \, d\lambda,
%	\end{align}
%	provided that the limit in the right hand side exists. Here we compute the right hand side of the previous equality.
%	To carry on the computation, let $R(b):=\sqrt{a^2+b^2}$ and denote by $b(R)$ the inverse function $b: [x_0, +\infty) \mapsto [0, +\infty)$. Set $A(R):= a-ib(R)$, $B(R):=a+ib(R)$ and observe that $|A(R)|=|B(R)|=R$. In particular, note that
%	\begin{align}\label{eq:420}
%		\lim _{b \to +\infty} \int_{a-ib}^{a+ib} e^{\lambda t } 	\,  \frac{\Phi^\dagger(\lambda)}{\lambda} e^{-x\Phi(\lambda)} \, d\lambda \, = \, \lim_{R \to +\infty} \int_{A(R)}^{B(R)} e^{\lambda t } 	\,  \frac{\Phi^\dagger(\lambda)}{\lambda} e^{-x\Phi(\lambda)} \, d\lambda.
%	\end{align}
%	define $C(R):=iR$ and $F(R):=-iR$. Furthermore, let $\varepsilon >0$ and define $D(\varepsilon):=i\varepsilon $ and $E(\varepsilon):=-i\varepsilon$. We let $\Gamma_R^+$ the anticlockwise oriented circular arc joining $B(R)$ to $C(R)$, while we define $\Gamma_R^-$ the anticlockwise oriented circular arc joining $F(R)$ to $A(R)$. Denote also $\ell_1$ to be the oriented segment connecting $A(R)$ to $B(R)$, $\ell_2$ the oriented segment connecting $C(R)$ to $D(\varepsilon)$ and $\ell_3$ the oriented segment connecting $E(\varepsilon)$ to $F(R)$. Finally, let $\gamma_{\varepsilon,\theta}$ be the clockwise oriented circular arc joining $D(\varepsilon)$ to $E(\varepsilon)$.
%	Let $\partial \mathfrak{D}$ be the closed contour obtained by connecting, in this order, $\ell_1$, $\Gamma_R^+$, $\ell_2$, $\gamma_{\varepsilon,\theta}$, $\ell_3$, $\Gamma_R^-$ (see Figure \ref{fig1}). Such a contour is the boundary of an open set $\mathfrak{D}$ of the complex plane in which 
%	\begin{align}
%		\mathfrak{D} \ni \lambda  \mapsto F(\lambda; x,t) \, : = \,  \frac{\Phi^\dagger(\lambda)}{\lambda}e^{\lambda t-x\Phi(\lambda)} \in \mathbb{C}, \qquad x>0, t>0,
%	\end{align}
%	is holomorphic. Hence, we can apply Cauchy's Theorem to get
%	\begin{align}
%		\int_{\partial \mathfrak{D}} F(\lambda; x, t) \, d\lambda \, = \, 0.
%	\end{align}
%	This implies that
%	\begin{align}
%		&\int_{A(R)}^{B(R)} e^{\lambda t } 	\,  \frac{\Phi^{\dagger}(\lambda)}{\lambda} e^{-x\Phi(\lambda)} \, d\lambda \notag \\ = \,& - \int_{\Gamma_R^+}F(\lambda; x, t) \, d\lambda \, - \, \int_{\ell_2} F(\lambda; x, t) \, d\lambda \, - \, \int_{\gamma_{\varepsilon,\theta}} F(\lambda; x, t) \, d\lambda \, - \, \int_{\ell_3} F(\lambda; x, t) \, d\lambda  \notag \\
%		& - \int_{\Gamma_R^-} F(\lambda; x, t) \, d\lambda .
%		\label{decompgen}
%	\end{align}
%	Now we deal with various terms separately.
%	
%	We begin with $\Gamma_R^+$. Note that, by the Estimation Lemma \cite[Theorem 5.24]{howie}
%	\begin{align}
%		\left| \int_{\Gamma_R^+} F(\lambda; x, t) \, d\lambda \right| \, \leq \, \text{length}(\Gamma_R^+) \, \max_{\lambda \in \Gamma_R^+} \left| F(\lambda; x, t) \right|,
%		\label{estlemmagen}
%	\end{align}
%	where, with an abuse of notation, we denote by $\Gamma_R^+$ also the support of the oriented curve.
%	To evaluate the maximum appearing in \eqref{estlemmagen} it is useful to parametrize $\Gamma_R^+$ as follows
%	\begin{align}
%		\Gamma_R^+ \, = \, \ll z \in \mathbb{C}: z = R e^{i\xi}, \xi \in \left[ \xi_{B}(R), \frac{\pi}{2} \right] \rr
%	\end{align}
%	where $\xi_B(R)= \arctan \l \frac{r(R)}{a} \r$.
%	Then, for $\lambda = Re^{i\xi} \in \Gamma_R^+$, it holds
%	\begin{align}
%		\left| F(\lambda; x,t) \right| \, =  \frac{\left| \Phi^\dagger(Re^{i\xi}) \right|}{R} e^{Rt\cos \xi -x\Re \Phi (Re^{i\xi})}.
%	\end{align}
%	Without loss of generality we can assume $\xi_B(R)> \frac{\pi}{4}$. First observe that $\Re \l Re^{i\xi}\r= R \cos \xi \geq 0$ for any $\xi \in \left[ \xi_B(R), \frac{\pi}{2} \right]$ and thus, by Item \ref{it:sign} of Lemma \ref{lem:Bern}, we have that $\Re \Phi \l Re^{i\xi} \r \geq 0$. Furthermore, it holds $R \cos \xi \leq R \cos \xi_{B}(R) = a$ for any $\xi \in \left[ \xi_B(R), \frac{\pi}{2} \right]$. Hence we get
%	\begin{align}
%		\max_{\lambda \in \Gamma_R^+} \left| F(\lambda; x,t) \right| \, \leq \, e^{ta} \max_{\xi \in \left[ \frac{\pi}{4}, \frac{\pi}{2} \right]} \left| \frac{\Phi^{\dagger} \l Re^{i\xi} \r}{Re^{i\xi}} \right|
%	\end{align}
%	and thus by Item \ref{it:asymp} of Lemma \ref{lem:Bern} it holds
%	\begin{align}
%		\lim_{R \to + \infty} \max_{\lambda \in \Gamma_R^+} \left| F(\lambda; x,t) \right| = 0.
%		\label{maxtozerogen}
%	\end{align}
%	Note now that
%	\begin{align}
%		\text{length}(\Gamma_R^+) \, = \,& R \l \frac{\pi}{2} - \xi_B(R) \r \notag \\
%		= \, & R \l \frac{\pi}{2}-\arctan \frac{r(R)}{a} \r \notag \\
%		= \, & \frac{R a}{r(R)} \frac{r(R)}{a} \l \frac{\pi}{2}-\arctan \frac{r(R)}{a} \r.
%	\end{align}
%	Since $r(R)= \sqrt{R^2-a^2}$ we obtain
%	\begin{align}
%		\lim_{R \to +\infty}\text{length}(\Gamma_R^+) \, = \, a.
%		\label{lengthtox0gen}
%	\end{align}
%	By combining \eqref{maxtozerogen} and \eqref{lengthtox0gen} with \eqref{estlemmagen} we have that
%	\begin{align}
%		\lim_{R \to +\infty} \int_{\Gamma_R^+} F(\lambda; x,t) \, d\lambda \, = \, 0.
%		\label{firsttozerogen}
%	\end{align}
%	Analogously, it is possible to see that
%	\begin{align}
%		\lim_{R \to +\infty} \int_{\Gamma_R^-}F(\lambda;x,t) d\lambda = 0.
%		\label{limR-gen}
%	\end{align}
%	We consider now the integral on $\ell_2$. We have that
%	\begin{align}
%		\int_{\ell_2} F (\lambda;x,t) d\lambda \,= \, -\int_{\varepsilon}^R \frac{\Phi^\dagger\l i\rho \r}{\rho } e^{-x \Phi \l i\rho \r+ it  \rho } \, d\rho.
%		\label{239gen}
%	\end{align}
%	Once we observe that, by Item \ref{it:asymp} of Lemma \ref{lem:Bern}, there exists a constant $C>0$
%	\begin{align}\label{eq:estinf}
%		\int_1^{+\infty} \frac{|\Phi^\dagger(i\rho)|}{\rho} e^{-x\Re \Phi (i\rho) } d\rho \, \leq \, C \int_1^{+\infty} e^{-x\Re \Phi(i\rho)} d\rho < +\infty
%	\end{align}
%	for any $x>x_0$, we can take the limit as $R \to +\infty$ to get
%	\begin{align}
%		\lim_{R \to +\infty}\int_{\ell_2} F(\lambda; x,t) \, d\lambda \, = \, -\int_\varepsilon^{+\infty}\frac{\Phi^\dagger\l i\rho\r}{\rho } e^{-x \Phi \l i\rho  \r+ it  \rho } \, d\rho \, =: \, -I_1(\varepsilon).
%		\label{243gen}
%	\end{align}
%	Analogously, on $\ell_3$, we have that
%	\begin{align}
%		\lim_{R \to +\infty}\int_{\ell_3} F(\lambda; x,t) \, d\lambda \, = \, \int_\varepsilon^{+\infty}\frac{\Phi^\dagger\l -i\rho \r}{\rho } e^{-x \Phi \l -i\rho  \r-i t  \rho } \, d\rho\, =: \, I_2(\varepsilon).
%		\label{244gen}
%	\end{align}
%	Furthermore, by using the fact that $\overline{\Phi(z)} = \Phi(\overline{z})$, we know that $I_2(\varepsilon) = \overline{I_1(\varepsilon)}$. Hence, taking the limit as $R \to +\infty$ in \eqref{decompgen} and using \eqref{firsttozerogen}, \eqref{limR-gen}, \eqref{243gen} and \eqref{244gen} we get
%	\begin{align*}
%		\lim_{b \to +\infty}&\int_{a-ib}^{a+ib} e^{\lambda t } 	\,  \frac{\Phi^{\dagger}(\lambda)}{\lambda} e^{-x\Phi(\lambda)} \, d\lambda = \,  \, I_1(\varepsilon)-\overline{I_1(\varepsilon)} - \, \int_{\gamma_{\varepsilon,\theta}} F(\lambda; x, t) \, d\lambda   \notag \\
%		&=2i \Im I_1(\varepsilon)- \, \int_{\gamma_{\varepsilon,\theta}} F(\lambda; x, t) \, d\lambda \\
%		&=2i \int_\varepsilon^{+\infty}\Im\left(\frac{\Phi^\dagger\l i\rho\r}{\rho } e^{-x \Phi \l i\rho  \r+ it  \rho }\right) \, d\rho-\, \int_{\gamma_{\varepsilon,\theta}} F(\lambda; x, t) \, d\lambda\\
%		&=2iI_3(\varepsilon)-\, \int_{\gamma_{\varepsilon,\theta}} F(\lambda; x, t) \, d\lambda.
%	\end{align*}
%	where we denote
%	\begin{align*}
%I_3(\varepsilon):=\int_\varepsilon^{+\infty}\Im\left(\frac{\Phi^\dagger\l i\rho\r}{\rho } e^{-x \Phi \l i\rho  \r+ it  \rho }\right) \, d\rho.
%	\end{align*}
%	Now we would like to take the limit as $\varepsilon \to 0$. To do this, let us first handle the integral over $\gamma_{\varepsilon,\theta}$. Using again  the estimation Lemma we have that
%	\begin{align}
%		\left| \int_{\gamma_{\varepsilon,\theta}} F(\lambda;x,t) \, d\lambda \right| \, \leq \, & \text{length} (\gamma_{\varepsilon,\theta}) \, \max_{\lambda \in \gamma_{\varepsilon}} \left| F(\lambda; x,t) \right| \notag \\
%		= \, & \varepsilon \pi \max_{\lambda \in \gamma_{\varepsilon}} \left| F(\lambda; x,t) \right|
%	\end{align}
%	Note now that
%	\begin{align}
%		\max_{\lambda \in \gamma_{\varepsilon}} \left| F(\lambda; x,t) \right| \, = \, \frac{1}{\varepsilon}  \max_{\lambda \in \gamma_{\varepsilon}} \left[ \left| \Phi(\lambda) \right| e^{t \Re \lambda -x \Re \Phi (\lambda)} \right]
%	\end{align}
%	and thus
%	\begin{align}
%		\left| \int_{\gamma_{\varepsilon,\theta}} F (\lambda; x,t) d\lambda \right| \, \leq \, \pi \max_{\lambda \in \gamma_{\varepsilon,\theta}} \left[ \left| \Phi(\lambda)\right| e^{t\Re \lambda-x\Re \Phi(\lambda)} \right].
%		\label{233gen}
%	\end{align}
%	Since $\Phi$ is continuous on the set $\mathfrak{D}^+:= \ll z \in \mathbb{C}: z = \rho e^{i\xi}, \rho \in [0,1], \xi \in \left[ -\frac{\pi}{2}, \frac{\pi}{2} \right] \rr$, and thus uniformly continuous, we have
%	\begin{align}
%		\lim_{\varepsilon \to 0}\max_{\lambda \in \gamma_{\varepsilon,\theta}} \left[ \left| \Phi(\lambda)\right| e^{t\Re \lambda-x\Re \Phi(\lambda)} \right] \, = \, 0
%	\end{align}
%	and
%	\begin{align}
%		\lim_{\varepsilon \to 0} \int_{\gamma_{\varepsilon,\theta}} F(\lambda;x,t) d\lambda \, = \, 0.
%		\label{gammaeps}
%	\end{align}
%	Now we handle $I_3(\varepsilon)$. Clearly, by \eqref{eq:estinf}, we only need to estimate the behaviour in a neighbourhood of $0$. First observe that we have
%	\begin{align*}
%		&\Im\left(\frac{\Phi^\dagger\l i\rho\r}{\rho } e^{-x \Phi \l i\rho  \r+ it  \rho }\right)=\frac{1}{\rho}\left(\Re \Phi^\dagger(i\rho)\Im\left(e^{-x \Phi \l i\rho  \r+ it  \rho }\right)+\Im \Phi^\dagger(i\rho)\Re\left(e^{-x \Phi \l i\rho  \r+ it  \rho }\right)\right)\\
%		&=\frac{1}{\rho}\left(\Re \Phi^\dagger(i\rho)\left(e^{-x \Re \Phi \l i\rho  \r}\sin \left(t\rho -x\Im\Phi(i\rho)\right)\right)+\Im \Phi^\dagger(i\rho)e^{-x \Re \Phi \l i\rho  \r}\cos\left(t\rho -x\Im\Phi(i\rho)\right)\right)\\
%		&=\frac{(t-x\mathfrak{b})\rho-x\Im \Phi^\dagger(i\rho)}{\rho}\Re \Phi^\dagger(i\rho)e^{-x \Re \Phi \l i\rho  \r}\frac{\sin \left((t -x\mathfrak{b})\rho-x\Im\Phi^\dagger(i\rho)\right)}{(t -x\mathfrak{b})\rho-x\Im\Phi^\dagger(i\rho)}\\
%		&\qquad \qquad +\frac{\Im \Phi^\dagger(i\rho)}{\rho}e^{-x \Re \Phi \l i\rho  \r}\cos\left((t-x\mathfrak{b})\rho -x\Im\Phi^\dagger(i\rho)\right),
%	\end{align*}
%	where we used the fact that $\Im\Phi(i\rho)=\mathfrak{b}\rho+\Im \Phi^\dagger(i\rho)$. Taking the absolute value, we have, for any $\rho \in [0,1]$,
%	 \begin{align*}
%	 	\left|\Im\left(\frac{\Phi^\dagger\l i\rho\r}{\rho } e^{-x \Phi \l i\rho  \r+ it  \rho }\right)\right| \le C\left(|t-x\mathfrak{b}|+\frac{|\Im \Phi^\dagger(i\rho)|}{\rho}\right)
%	 \end{align*}
% 	and then
% 	 \begin{align*}
% 		\int_0^1\left|\Im\left(\frac{\Phi^\dagger\l i\rho\r}{\rho } e^{-x \Phi \l i\rho  \r+ it  \rho }\right)\right|d\rho \le C\int_0^1\left(|t-x\mathfrak{b}|+\frac{|\Im \Phi^\dagger(i\rho)|}{\rho}\right)d\rho <+\infty.
% 	\end{align*}
% 	It follows that
% 	\begin{align}
% 	\lim_{\varepsilon \to 0} I_3(\varepsilon) \, = \, \int_0^\infty  \Im \l \frac{\Phi^\dagger (i\rho)}{\rho} e^{-x\Phi(i\rho)+it\rho}\r d\rho.
% 	\label{i3}
% 	\end{align}
% 	By combining \eqref{i3} and \eqref{gammaeps} we obtain
% 	\begin{align}
% 	\lim_{b \to +\infty} \int_{a-ib}^{a+ib} F(\lambda;x,t) d\lambda \, = \, \frac{1}{\pi} \int_0^{+\infty} \Im \l \frac{\Phi^\dagger(i\rho)}{\rho} e^{-x\Phi(i\rho)+it\rho} \r d\rho
%  	\end{align}
% 	and the result follows.
%\end{proof}
To prove both Theorems \ref{thm:smoothfmu} and \ref{thm:seriespi}, we will make use of the following integral representation, which is a consequence of a Laplace inversion formula, as Proposition \ref{prop:LTthetapi}.
\begin{prop}\label{prop:intreptheta}
	Let $\Phi$ be the Laplace exponent of a potentially killed subordinator satisfying Assumptions \eqref{eq:extensionA3} and \eqref{eq:uniformlimcond} for some $\theta \in (0,\pi)$. Fix any $\varepsilon>0$ and let $\gamma_{\varepsilon,\theta}$ be the circle arc in $\C$ parametrized as $\gamma_{\varepsilon,\theta}:z=\varepsilon e^{i\xi}$ for $\xi \in \left[\frac{\theta}{2}-\pi,\pi-\frac{\theta}{2}\right]$.	
%	define the set 
%	\begin{equation*}
% \gamma_{\varepsilon,\theta} := \ll z \in \mathbb{C}: z=\varepsilon e^{i\xi}, \xi \in \left[ \frac{\theta}{2}-\pi, \pi- \frac{\theta}{2} \right] \rr.
% \end{equation*}	
  Then, on $\mathbb{D}$,
	\begin{equation}\label{statement1}
		\begin{split}
				f_\Phi (x,t) \, = \,& \frac{1}{\pi} \int_\varepsilon^{+\infty}\Im\left(\frac{\Phi^\dagger\lb \rho e^{i\left(\pi-\frac{\theta}{2}\right)}\rb}{\rho } e^{-x \Phi \lb \rho e^{i\left(\pi-\frac{\theta}{2}\right)}  \rb+ t \rho e^{i\left(\pi-\frac{\theta}{2}\right)} }\right) \, d\rho  +\frac{1}{2\pi i}\int_{\gamma_{\varepsilon,\theta}} \frac{\Phi^\dagger(z)}{z}e^{-x\Phi(z)+tz}dz.
		\end{split}
	\end{equation}
%If, furthermore,
%	\begin{equation}\label{eq:intcondtheta}
%		\int_0^1\frac{\left|\Im \Phi^\dagger \left(\rho e^{i\left(\pi-\frac{\theta}{2}\right)}\right)\right|}{\rho}d\rho<+\infty
%	\end{equation}
%	then, on $\mathbb{D}$,
%	\begin{equation}
%		\int_0^{+\infty} \left|\Im \l \frac{\Phi^\dagger\left(\rho e^{i\left(\pi-\frac{\theta}{2}\right)}\right)}{\rho} e^{it\rho e^{i\l \pi -\frac{\theta}{2} \r}-x\Phi\l \rho e^{i \l \pi-\frac{\theta}{2} \r} \r} \r\right| \, d\rho <+\infty
%	\end{equation}
%	and 
%	\begin{equation}\label{eq:intreptheta}
%		f_\Phi(x,t) \, = \, 	\frac{1}{\pi}\int_0^{+\infty}\Im \l \frac{\Phi^\dagger\left(\rho e^{i\left(\pi-\frac{\theta}{2}\right)}\right)}{\rho} e^{t\rho e^{i \l \pi-\frac{\theta}{2} \r}-x\Phi\l \rho e^{i \l \pi-\frac{\theta}{2} \r} \r} \r \, d\rho.
%	\end{equation}
\end{prop}
\begin{proof}
	Again, by \eqref{ceslimitdag} and \eqref{cesequalpointdag} for fixed $a>0$ and for almost any $(x,t) \in \mathbb{D}$, up to a subsequence
	\begin{equation*}
		f_\Phi(x,t) \, = \, \lim _{b \to +\infty} \frac{1}{2\pi i}\int_{a-ib}^{a+ib} e^{z t } 	\,  \frac{\Phi^\dagger(z)}{z} e^{-x\Phi(z)} \, dz,
	\end{equation*}
	provided that the limit in the right-hand side exists. Here, we compute the limit by using Cauchy's Theorem. 
For $R \ge a$, let $b(R)=\sqrt{R^2-a^2}$. Set $A(R):= a-ib(R)$, $B(R):=a+ib(R)$ and observe that $|A(R)|=|B(R)|=R$. In particular, note that
	\begin{equation}\label{eq:420}
		\lim _{b \to +\infty} \int_{a-ib}^{a+ib} e^{z t } 	\,  \frac{\Phi^\dagger(z)}{z} e^{-x\Phi(z)} \, dz \, = \, \lim_{R \to +\infty} \int_{A(R)}^{B(R)} e^{z t } 	\,  \frac{\Phi^\dagger(z)}{z} e^{-x\Phi(z)} \, dz.
	\end{equation}
 Now define $C(R):=Re^{i\lb \pi - \frac{\theta}{2} \rb}$ and $F(R):=Re^{i\lb \pi + \frac{\theta}{2} \rb}$. Furthermore, let $\varepsilon >0$ and define $D(\varepsilon):=\varepsilon e^{i\lb \pi - \frac{\theta}{2} \rb}$ and $E(\varepsilon):=\varepsilon e^{i\lb \pi + \frac{\theta}{2} \rb}$. We let $\Gamma_R^+$ be the anticlockwise oriented circular arc joining $B(R)$ to $C(R)$, while we define $\Gamma_R^-$ the anticlockwise oriented circular arc joining $F(R)$ to $A(R)$. Denote also $\ell_1$ to be the oriented segment connecting $A(R)$ to $B(R)$, $\ell_2$ the oriented segment connecting $C(R)$ to $D(\varepsilon)$ and $\ell_3$ the oriented segment connecting $E(\varepsilon)$ to $F(R)$. Finally, let $-\gamma_{\varepsilon,\theta}$ be the clockwise oriented circular arc joining $D(\varepsilon)$ to $E(\varepsilon)$.
	Let $\partial \mathfrak{D}$ be the closed contour obtained by connecting, in this order, $\ell_1$, $\Gamma_R^+$, $\ell_2$, $-\gamma_{\varepsilon,\theta}$, $\ell_3$, $\Gamma_R^-$ (see Figure \ref{fig1}).	Such a contour is the boundary of an open set $\mathfrak{D}$ of the complex plane in which 
	\begin{equation*}
		\mathfrak{D} \ni z  \mapsto F(z; x,t) \, : = \,  \frac{\Phi^\dagger(z)}{z}e^{z t-x\Phi(z)} \in \mathbb{C}, \qquad x>0, t>0,
	\end{equation*}
	is holomorphic by assumption. Hence, we can apply Cauchy's Theorem to get
	\begin{equation*}
		\int_{\partial \mathfrak{D}} F(z; x, t) \, dz \, = \, 0.
	\end{equation*}
	This implies that
	\begin{equation}\label{decomp}
		\begin{split}
				&\int_{A(R)}^{B(R)} e^{z t } 	\,  \frac{\Phi^\dagger(z)}{z} e^{-x\Phi(z)} \, dz \\ = \,& - \int_{\Gamma_R^+}F(z; x, t) \, dz \, - \, \int_{\ell_2} F(z; x, t) \, dz \, + \, \int_{\gamma_{\varepsilon,\theta}} F(z; x, t) \, dz \, - \, \int_{\ell_3} F(z; x, t) \, dz   - \int_{\Gamma_R^-} F(z; x, t) \, dz .	
		\end{split}
	\end{equation}
\begin{minipage}{0.5\linewidth}
	\begin{tikzpicture}[scale=0.4]
		\draw [decoration={markings,mark=at position 1 with
			{\arrow[scale=3,>=stealth]{>}}},postaction={decorate}] (0,-8.5) -- (0,8.5);
		\draw [decoration={markings,mark=at position 1 with
			{\arrow[scale=3,>=stealth]{>}}},postaction={decorate}] (-7,0) -- (7,0);
		\draw[black, dashed] (-2,-2)--(0,0);
		\draw[black, dashed] (-2,2)--(0,0);
		\draw[black] (5,-7) -- (5,-4.88);
		\draw[black] (5,4.88) -- (5,7);
		\draw[black, dashed] (5,-7) -- (5,-8);
		\draw[black, dashed] (5,7) -- (5,8);
		\draw[black, dashed] (0,0)--(4.88,4.70);
		\begin{scope}[very thick,decoration={
				markings,
				mark=at position 0.5 with {\arrow{>}}}
			] 
			\centerarc[red,very thick,postaction={decorate}](0,0)(44:90:7);
			\centerarc[red,very thick,postaction={decorate}](0,0)(90:135:7);
			\draw[red,very thick, postaction={decorate}] (5,-4.88) -- (5,4.88);
			\draw[red,very thick,postaction={decorate}] (-4.88,4.88) -- (-2,2);
			\centerarc[red,very thick,postaction={decorate}](0,0)(135:-135:2.828);
			\draw[red,very thick,postaction={decorate}] (-2,-2)--(-4.88,-4.88);
			\centerarc[red,very thick,postaction={decorate}](0,0)(-135:-90:7);
			\centerarc[red,very thick,postaction={decorate}](0,0)(-90:-44:7);
		\end{scope}
		\fill[red] (5,4.88) circle (0.2);
		\fill[red] (-4.95,4.95) circle (0.2);
		\fill[red] (-2,2) circle (0.2);
		\fill[red] (-2,-2) circle (0.2);
		\fill[red] (-4.95,-4.95) circle (0.2);
		\fill[red] (5,-4.88) circle (0.2);
		\node at (6.3,5.40) {\large $B(R)$};
		\node at (6.3,-5.40) {\large $A(R)$};
		\node at (-6.3,-5.25) {\large $F(R)$};
		\node at (-6.3,5.25) {\large $C(R)$};
		\node at (-3,1.5) {\large $D(\varepsilon)$};
		\node at (-3,-1.5) {\large $E(\varepsilon)$};
		\node at (-0.85,-1.5) {\large $\varepsilon$};
		\centerarc[dashed](0,0)(135:225:0.5);
		\node at (-1,0.3) {\large $\theta$};
		\node at (2.8,3.5) {\large $R$};
		%		\node at (5.6,0.3) {\large $x_0$};
		%		\node at (4.5,2) {\large $r$};
		\fill[red] (0,7) circle (0.2);
		\fill[red] (0,-7) circle (0.2);
		%		\fill[black] (5,0) circle (0.1);
		\node at (2,7.5) {\large $M_+(R)$};
		\node at (2,-7.7) {\large $M_-(R)$};
		\node at (-1,-5.7) {\large $\Gamma_R^-$};
		\node at (-1,5.7) {\large $\Gamma_R^+$};
		\node at (-2.7,3.8)  {\large $\ell_2$};
		\node at (-2.7,-3.8)  {\large $\ell_3$};
		\node at (2,0.3)  {\large $\gamma_{\varepsilon,\theta}$};
		\node at (4.5,0.6)  {\large $\ell_1$};
	\end{tikzpicture}
	\captionof{figure}{\label{fig1}Sketch of the keyhole-type contour.}
\end{minipage}
\begin{minipage}{0.5\linewidth}
	\begin{tikzpicture}[scale=0.4]
		\draw [decoration={markings,mark=at position 1 with
			{\arrow[scale=3,>=stealth]{>}}},postaction={decorate}] (0,-8.5) -- (0,8.5);
		\draw [decoration={markings,mark=at position 1 with
			{\arrow[scale=3,>=stealth]{>}}},postaction={decorate}] (-7,0) -- (7,0);
		\draw[black, dashed] (2,-2)--(0,0);
		%\draw[black, dashed] (-2,2)--(0,0);
		\draw[black] (5,-7) -- (5,-4.88);
		\draw[black] (5,4.88) -- (5,7);
		\draw[black, dashed] (5,-7) -- (5,-8);
		\draw[black, dashed] (5,7) -- (5,8);
		\draw[black, dashed] (0,0)--(4.88,4.70);
		\begin{scope}[very thick,decoration={
				markings,
				mark=at position 0.5 with {\arrow{>}}}
			] 
			\centerarc[red,very thick,postaction={decorate}](0,0)(44:90:7);
			\centerarc[red,very thick,postaction={decorate}](0,0)(90:180:7);
			\centerarc[red,very thick,postaction={decorate}](0,0)(180:90:2.828);
			\centerarc[red,very thick,postaction={decorate}](0,0)(90:0:2.828);
			\centerarc[red,very thick,postaction={decorate}](0,0)(0:-90:2.828);
			\centerarc[red,very thick,postaction={decorate}](0,0)(-90:-180:2.828);
			\draw[red,very thick, postaction={decorate}] (5,-4.88) -- (5,4.88);
			\centerarc[red,very thick,postaction={decorate}](0,0)(-180:-90:7);
			\centerarc[red,very thick,postaction={decorate}](0,0)(-90:-44:7);
		\end{scope}
		\begin{scope}[very thick,decoration={
				markings,
				mark=at position 0.3 with {\arrow{>[left]}}}
			]
			\draw[red,very thick,postaction={decorate}] (-7,0) -- (-2.828,0);
			\draw[red,very thick,postaction={decorate}] (-2.828,0)--(-7,0);
		\end{scope}
		\fill[red] (5,4.88) circle (0.2);
		\fill[red] (-7,0) circle (0.2);
		\fill[red] (-2.828,0) circle (0.2);
		\fill[red] (5,-4.88) circle (0.2);
		\node at (6.3,5.40) {\large $B(R)$};
		\node at (6.3,-5.40) {\large $A(R)$};
		%\node at (-6.3,-5.25) {\large $F(R)$};
		\node at (-8.5,1.2) {\large $C(R)$};
		\node at (-8.5,-1.2) {\large $F(R)$};
		\node at (-1.5,0.8) {\large $D(\varepsilon)$};
		\node at (-1.5,-0.8) {\large $E(\varepsilon)$};
		%\node at (-3,-1.5) {\large $E(\varepsilon)$};
		\node at (0.85,-1.5) {\large $\varepsilon$};
		%\centerarc[dashed](0,0)(135:225:0.5);
		%\node at (-1,0.3) {\large $\theta$};
		\node at (2.8,3.5) {\large $R$};
		%		\node at (5.6,0.3) {\large $x_0$};
		%		\node at (4.5,2) {\large $r$};
		\fill[red] (0,7) circle (0.2);
		\fill[red] (0,-7) circle (0.2);
		%		\fill[black] (5,0) circle (0.1);
		\node at (2,7.5) {\large $M_+(R)$};
		\node at (2,-7.7) {\large $M_-(R)$};
		\node at (-1,-5.7) {\large $\Gamma_R^-$};
		\node at (-1,5.7) {\large $\Gamma_R^+$};
		\node at (-4,1)  {\large $\ell_2$};
		\node at (-6,-1)  {\large $\ell_3$};
		\node at (2,0.3)  {\large $\gamma_{\varepsilon}$};
		\node at (4.5,0.6)  {\large $\ell_1$};
	\end{tikzpicture}
	\captionof{figure}{\label{fig2}Sketch of the keyhole contour.}	
\end{minipage}\\

	Now we deal with various terms separately.
	To deal with the first integral, let $M^+(R)= Ri$ and split the curve $\Gamma_R^+$ into $\Gamma_R^1$ connecting $B(R)$ to $M^+(R)$ and $\Gamma_R^2$ connecting $M^+(R)$ to $C(R)$ so that
	\begin{equation}	\label{29}
		\int_{\Gamma_R^+} F(z; x, t) \, dz  \, = \, \int_{\Gamma_R^1} F(z; x, t) \, dz  + \int_{\Gamma_R^2} F(z; x, t) \, dz.
	\end{equation}
	We begin with $\Gamma_R^1$. Note that, by the Estimation Lemma \cite[Theorem 5.24]{howie}
	\begin{equation}\label{estlemmagen}
		\left| \int_{\Gamma_R^1} F(z x, t) \, dz \right| \, \leq \, \text{length}(\Gamma_R^1) \, \max_{z \in \Gamma_R^1} \left| F(z; x, t) \right|,
	\end{equation}
	where, with an abuse of notation, we denote by $\Gamma_R^1$ also the support of the oriented curve. To evaluate the maximum appearing in \eqref{estlemmagen} we parametrize $\Gamma_R^1$ as follows
	\begin{equation*}
		\Gamma_R^1 \, = \, \ll z \in \mathbb{C}: z = R e^{i\xi}, \xi \in \left[ \xi_{B}(R), \frac{\pi}{2} \right] \rr,
	\end{equation*}
	where $\xi_B(R)= \arctan \lb \frac{r(R)}{a} \rb$.
	Then, for $z = Re^{i\xi} \in \Gamma_R^1$, it holds
	\begin{equation*}
		\left| F(z; x,t) \right| \, =  \frac{\left| \Phi^\dagger(Re^{i\xi}) \right|}{R} e^{Rt\cos \xi -x\Re \Phi (Re^{i\xi})}.
	\end{equation*}
	Without loss of generality we can assume $\xi_B(R)> \frac{\pi}{4}$. First, observe that $\Re \lb Re^{i\xi}\rb= R \cos \xi \geq 0$, for any $\xi \in \left[ \xi_B(R), \frac{\pi}{2} \right]$, and thus, by Item \ref{it:sign} of Lemma \ref{lem:Bern}, we have that $\Re \Phi \lb Re^{i\xi} \rb \geq 0$. Furthermore, it holds that $R \cos \xi \leq R \cos \xi_{B}(R) = a$, for any $\xi \in \left[ \xi_B(R), \frac{\pi}{2} \right]$. Hence, we get
	\begin{equation*}
		\max_{z \in \Gamma_R^1} \left| F(z; x,t) \right| \, \leq \, e^{ta} \max_{\xi \in \left[ \frac{\pi}{4}, \frac{\pi}{2} \right]} \left| \frac{\Phi^{\dagger} \lb Re^{i\xi} \rb}{Re^{i\xi}} \right|
	\end{equation*}
	and thus by Item \ref{it:asymp} of Lemma \ref{lem:Bern} it holds
	\begin{equation}\label{maxtozerogen}
		\lim_{R \to + \infty} \max_{z \in \Gamma_R^1} \left| F(z; x,t) \right| = 0.
	\end{equation}
%	Note now that
%	\begin{equation*}
%		\begin{split}
%			\text{length}(\Gamma_R^1) \, = \,& R \l \frac{\pi}{2} - \xi_B(R) \r = R \l \frac{\pi}{2}-\arctan \frac{r(R)}{a} \r 
%			= \,  \frac{R a}{r(R)} \frac{r(R)}{a} \l \frac{\pi}{2}-\arctan \frac{r(R)}{a} \r.
%		\end{split}	
%	\end{equation*}
%	Since $r(R)= \sqrt{R^2-a^2}$ we obtain
Furthermore, it is not difficult to check that
	\begin{equation}\label{lengthtox0gen}
		\lim_{R \to +\infty}\text{length}(\Gamma_R^1) \, = \, a.
	\end{equation}
	By combining \eqref{maxtozerogen} and \eqref{lengthtox0gen} with \eqref{estlemmagen} we have that
	\begin{equation}\label{firsttozerogen11}
		\lim_{R \to +\infty} \int_{\Gamma_R^1} F(z; x,t) \, dz \, = \, 0.
	\end{equation}	
	Now we deal with the second term in \eqref{29}, i.e. the one on $\Gamma_R^2$.
	%\begin{equation*}
	%	\int_{\Gamma_R^2}F(z; x,t) \, dz \, = \, \int_{\Gamma_R^2} e^{z t} \frac{\Phi^\dagger(z)}{z} e^{-x\Phi(z)} dz.
	%\end{equation*}
	Recalling that $(x,t) \in \mathbb{D}$, let $\delta=t-\mathfrak{b}x>0$ and choose $p > 1$ such that $\frac{t}{p}-\mathfrak{b}x>\frac{\delta}{2}$, that exists since $\lim_{p \to 1}\frac{t}{p}-\mathfrak{b}x=\delta$. Let also $p^\prime >1$ be the conjugate exponent of $p>1$, i.e. $\frac{1}{p}+\frac{1}{p^\prime}=1$. Then we have
%	Let $z = i\zeta$ and set $\widetilde{\Gamma}_R^2:= \ll z \in \mathbb{C}: z=Re^{i\zeta}, \zeta \in \left[ 0, \frac{\pi-\theta}{2} \right] \rr$ to get
%	\begin{equation*}
%		\int_{\Gamma_R^2}F(z; x,t) \, dz \, = \,i \int_{\widetilde{\Gamma}_R^2} e^{i\zeta t} \frac{\Phi^\dagger(i\zeta)}{i\zeta} e^{-x \Phi(i\zeta)} \, d\zeta.
%	\end{equation*} 
%	Let $\delta=t-\mathfrak{b}x>0$ and choose $p > 1$ such that $\frac{t}{p}-\mathfrak{b}x>\frac{\delta}{2}$. This is always possible since we are on $\mathbb{D}$ and since $\frac{t}{p}-\mathfrak{b}x < \delta$ with $\frac{t}{p}-\mathfrak{b}x \to \delta$ as $p\to 1$. Let $q_* \ge 1$ such that $\frac{1}{p}+\frac{1}{q_*}=1$ and $G(\zeta;x,t):= e^{\lambda t/p}(i\zeta)^{-1}\Phi(i\zeta) e^{-x\Phi(i\zeta)}$. We get
	\begin{equation}	\label{220}
	\begin{split}
		\left| \int_{\Gamma_R^2}F(z; x,t) \, dz \right| \,  \leq \, & R \lb \max_{z \in \Gamma_R^2}  \left| \frac{\Phi^\dagger(z)}{z} e^{ \frac{t}{p}z-x\Phi(z)} \right| \rb \lb \int_0^{\frac{\pi-\theta}{2}} e^{-(Rt\sin \xi)/p^\prime} d\xi \rb  \\
		\leq \, & \frac{p^\prime \pi}{t} \lb \max_{z \in \Gamma_R^2}  \left| \frac{\Phi^\dagger(z)}{z} e^{ \frac{t}{p}z-x\Phi(z)} \right| \rb = \, \frac{p^\prime \pi e^{-xq}}{t} \lb \max_{z \in \Gamma_R^2}  \left| \frac{\Phi^\dagger(z)}{z} e^{ \left(\frac{t}{p}-\mathfrak{b}x\right)z-x\Phi^\dagger(z)} \right| \rb, 
	\end{split}	
\end{equation}
%	\begin{equation}	\label{220}
%	\begin{split}
%		\left| \int_{\Gamma_R^2}F(z; x,t) \, dz \right| \, = \, &\left| \int_{\widetilde{\Gamma}_R^2} e^{i\zeta t} \frac{\Phi^\dagger(i\zeta)}{i\zeta} e^{-x \Phi(i\zeta)} \, d\zeta \right|  \\
%		= \, & \left| \int_0^{(\pi-\theta)/2}  e^{iRe^{i\xi}t/q_*} G(Re^{i\xi};x,t) Re^{i\xi} d\xi \right|  \\
%		\leq \, & R \int_0^{(\pi-\theta)/2}  e^{-(Rt \sin \xi )/q_*} \left| G\l Re^{i\xi};x,t \r \right| d\xi  \\
%		\leq \, & R \l \max_{\xi \in \left[ 0, \frac{\pi-\theta}{2} \right]}  \left| G\l Re^{i\xi};x,t \r \right| \r \l \int_0^\pi e^{-(Rt\sin \xi)/q_*} d\xi \r  \\
%		\leq \, & \frac{q_*\pi}{t} \l \max_{z \in \Gamma_R^2}  \left| \frac{\Phi^\dagger(z)}{z} e^{ \frac{t}{p}z-x\Phi(z)} \right| \r \\
%		= \, & \frac{q_*\pi e^{-xq}}{t} \l \max_{z \in \Gamma_R^2}  \left| \frac{\Phi^\dagger(z)}{z} e^{ \left(\frac{t}{p}-\mathfrak{b}x\right)z-x\Phi^\dagger(z)} \right| \r, 
%	\end{split}	
%	\end{equation}
	where in the second inequality we have used Jordan's inequality \cite[eq (2), page 262]{brownchurchill}. Now, consider $g(z)=e^{ \left(\frac{t}{p}-\mathfrak{b}x\right)z-x\Phi^\dagger(z)}$ and observe that, by hypothesis, $g$ is holomorphic on $\C\left(\frac{\pi}{2},\pi-\frac{\theta}{2}\right)$ and continuous on $\overline{\C\left(\frac{\pi}{2},\pi-\frac{\theta}{2}\right)}$. Furthermore, for $z=iR$ we have
	\begin{equation*}
		|g(iR)|=e^{-x\Re \Phi^\dagger(iR)} \le 1,
	\end{equation*}
	since $\Re \Phi^\dagger(iR) \ge 0$ by Item \ref{it:sign} of Lemma \ref{lem:Bern}. For $z=Re^{i\left(\pi-\frac{\theta}{2}\right)}$ we have instead
	\begin{equation*}
		\left|g\left(Re^{i\left(\pi-\frac{\theta}{2}\right)}\right)\right|=e^{-R\left(\frac{t}{p}-\mathfrak{b}x\right)\cos\left(\frac{\theta}{2}\right)-x\Re \Phi^\dagger\left(Re^{i\left(\pi-\frac{\theta}{2}\right)}\right)}.
	\end{equation*}
	Now let us show that for $R$ big enough, $\left|g\left(Re^{i\left(\pi-\frac{\theta}{2}\right)}\right)\right| \le 1$. Indeed,
	\begin{align*}
		\left|g\left(Re^{i\left(\pi-\frac{\theta}{2}\right)}\right)\right|&=\exp\left(-R\cos\left(\frac{\theta}{2}\right)\left(\frac{t}{p}-\mathfrak{b}x+\frac{x}{\cos\left(\frac{\theta}{2}\right)}\frac{\Re \Phi^\dagger\left(Re^{i\left(\pi-\frac{\theta}{2}\right)}\right)}{R}\right)\right).
	\end{align*}
	Once we observe that 
	\begin{equation*}
		\left|\frac{\Re \Phi^\dagger\left(Re^{i\left(\pi-\frac{\theta}{2}\right)}\right)}{R}\right| \le \frac{\left| \Phi^\dagger\left(Re^{i\left(\pi-\frac{\theta}{2}\right)}\right)\right|}{R}
	\end{equation*}
	and we use \eqref{eq:uniformlimcond} to state that $\lim_{R \to +\infty}\frac{\left| \Phi^\dagger\left(Re^{i\left(\pi-\frac{\theta}{2}\right)}\right)\right|}{R}=0$, we know that there exists a constant $C(t,x,p)$ such that for $R>C(t,x,p)$
	\begin{equation*}
		\frac{t}{p}-\mathfrak{b}x+\frac{x}{\cos\left(\frac{\theta}{2}\right)}\frac{\Re \Phi^\dagger\left(Re^{i\left(\pi-\frac{\theta}{2}\right)}\right)}{R} \ge \frac{\delta}{4}.
	\end{equation*}
	Hence, for $R>C(t,x,p)$,
	\begin{align*}
		\left|g\left(Re^{i\left(\pi-\frac{\theta}{2}\right)}\right)\right|&\le \exp\left(-\frac{\delta R\cos\left(\frac{\theta}{2}\right)}{4}\right) \le 1.
	\end{align*}
	On the other hand, $R \in [0,+\infty) \mapsto \left|g\left(Re^{i\left(\pi-\frac{\theta}{2}\right)}\right)\right| \in \R$ is continuous and then the latter inequality implies that there exists $M=M(x,t,p)$ such that $|g(z)| \le M$ for any $z \in \partial \C\left(\frac{\pi}{2},\pi-\frac{\theta}{2}\right)$.
	Furthermore, observe that, for $z=Re^{i\xi}$ with $\xi \in \left(\frac{\pi}{2},\pi-\frac{\theta}{2}\right)$ we have, by \eqref{eq:uniformlimcond}, $|\Re\Phi(z)|\le |\Phi(z)| \le C|z|$, for $|z| \ge 1$. Hence, for $|z| \ge 1$
	\begin{equation*}
		|g(z)|=e^{\frac{t}{p} \Re z+q-x\Re(\Phi(z))} \le e^{q+x|\Re(\Phi(z))|} \le e^{q+xC|z|}.
	\end{equation*}
	The continuity of the function $z \in \overline{\C\left(\frac{\pi}{2},\pi-\frac{\theta}{2}\right)} \mapsto |g(z)|e^{xC|z|} \in \R$ guarantees that there exists a constant $M_1=M_1(x,t,p)>0$ such that 
	\begin{equation*}
		|g(z)| \le M_2e^{xC|z|}, \ \forall z \in \overline{\C\left(\frac{\pi}{2},\pi-\frac{\theta}{2}\right)}.
	\end{equation*}
Since $\pi-\frac{\theta}{2}-\frac{\pi}{2}<\frac{\pi}{2}$, we can use Phragmen-Lindel\"of Theorem (see \cite[Chapter $4$, Exercise $9$, Item (b)]{stein10}) to obtain $|g(z)| \le M$ for any $z \in \overline{\C\left(\frac{\pi}{2},\pi-\frac{\theta}{2}\right)}$.

		
		
%	\begin{equation*}
%		M(x,t,p)=\max\left\{\max_{R \in [0,C(t,x,p)]}\left|g\left(Re^{i\left(\pi-\frac{\theta}{2}\right)}\right)\right|, 1\right\}
%	\end{equation*}
%	so that, for any $z \in \partial \C\left(\frac{\pi}{2},\pi-\frac{\theta}{2}\right)$ it holds $|g(z)| \le M(x,t,p)$. 
%	
%	
%	Thus, define $g_M=\frac{g}{M(x,t,p)}$ so that $g_M$ is holomorphic on $\C\left(\frac{\pi}{2},\pi-\frac{\theta}{2}\right)$, continuous on $\overline{\C\left(\frac{\pi}{2},\pi-\frac{\theta}{2}\right)}$ and $|g_M(z)| \le 1$, for any $z \in \partial \C\left(\frac{\pi}{2},\pi-\frac{\theta}{2}\right)$. Next, observe that, for $z=Re^{i\xi}$ with $\xi \in \left(\frac{\pi}{2},\pi-\frac{\theta}{2}\right)$ we have, by \eqref{eq:uniformlimcond}, $|\Re\Phi(z)|\le |\Phi(z)| \le C|z|$, for $|z| \ge 1$. Hence, for $|z| \ge 1$,
%	\begin{equation*}
%		|g_M(z)|=\frac{|g(z)|}{M(x,t,p)}=\frac{e^{\frac{t}{p}\Re z-x\Re(\Phi(z))}}{M(x,t,p)} \le  \frac{e^{x|\Re\Phi(z)|}}{M(x,t,p)}\le \frac{e^{Cx|z|}}{M(x,t,p)}.
%	\end{equation*}
%Furthermore, define
%\begin{equation*}
%M_1 (x,t,p) = \max_{\substack{z \in \overline{\C\left(\frac{\pi}{2},\pi-\frac{\theta}{2}\right)} \\ |z| \leq 1 }} |g_M(z)| e^{-Cx|z|}
%\end{equation*}	
%and
%\begin{equation*}
%M_2(x,t,p) = \max \l M_1(x,t,p), \frac{1}{M(x,t,p)} \r
%\end{equation*}
%so that $|g_M(z)| \leq M_2(x,t,p) e^{Cx|z|}$, for any $z \in \overline{\C\left(\frac{\pi}{2},\pi-\frac{\theta}{2}\right)}$.
%Finally, observe that
%	\begin{equation*}
%		\pi-\frac{\theta}{2}-\frac{\pi}{2}=\frac{\pi-\theta}{2}=:\frac{\pi}{\beta},
%	\end{equation*}
%	for some $\beta>2$. Hence we can use Phragmen-Lindel\"of Theorem (see \cite[Chapter $4$, Exercise $9$, Item (b)]{stein10}) to obtain $|g_M(z)| \le 1$ for any $z \in \overline{C\left(\frac{\pi}{2},\pi-\frac{\theta}{2}\right)}$, and then $|g(z)| \le M(x,t,p)$.
%	
 Thus, from \eqref{220}, we have
	\begin{equation*}
	\left| \int_{\Gamma_R^2}F(z; x,t) \, dz \right| \le \frac{M(x,t,p)q\pi e^{-xq}}{t} \lb \max_{z \in \Gamma_R^2}  \left| \frac{\Phi^\dagger(z)}{z} \right| \rb.
	\end{equation*}
	Taking the limit as $R \to +\infty$ we finally have
	\begin{equation}\label{230}
		\lim_{R \to +\infty} \int_{\Gamma_R^2} F(z;x,t)\, dz \, = \, 0,
	\end{equation}
	that, combined with \eqref{firsttozerogen11}, leads to
	\begin{equation}	\label{circsopra}
		\lim_{R\to +\infty} \int_{\Gamma_R^+} F(z;x,t) \, dz \, = \,0.
	\end{equation}
	In the same spirit it is possible to see that
	\begin{equation}\label{circsotto}
		\lim_{R \to +\infty} \int_{\Gamma_R^-}F(z;x,t) \, dz \, = \, 0.	
	\end{equation}
	We consider now the integral on $\ell_2$. We have that
	\begin{equation}\label{239}
		\int_{\ell_2} F (z;x,t) dz \,= \, -\int_{\varepsilon}^R \frac{\Phi^\dagger\lb\rho e^{i\lb \pi-\frac{\theta}{2} \rb}\rb}{\rho } e^{-x \Phi \lb \rho e^{i\lb \pi-\frac{\theta}{2} \rb} \rb+t  \rho e^{i \lb \pi-\frac{\theta}{2} \rb}} \, d\rho.
	\end{equation}
	Choose $p,p^\prime > 1$ as above so that, recalling that $\frac{\left|\Phi^\dagger\left(\rho e^{i\left(\pi-\frac{\theta}{2}\right)}\right)\right|}{\rho} \le C$ for $\rho \ge \varepsilon$ by \eqref{eq:uniformlimcond}, we have
	\begin{align}
				&\int_\varepsilon^\infty \left|  \frac{\Phi^\dagger\lb\rho e^{i\lb \pi-\frac{\theta}{2} \rb}\rb}{\rho } e^{-x \Phi \lb \rho e^{i\lb \pi-\frac{\theta}{2} \rb} \rb+t  \rho e^{i \lb \pi-\frac{\theta}{2} \rb}} \right| d\rho \notag \\	= \, & e^{-qx}\int_\varepsilon^\infty \frac{|\Phi^\dagger\lb \rho e^{i \lb \pi-\frac{\theta}{2} \rb} \rb|}{\rho}  \exp \ll {-\rho \cos\left(\frac{\theta}{2}\right) \lb \frac{t}{p}-\mathfrak{b}x +x \frac{\Re \Phi^\dagger \lb \rho e^{i \lb \pi-\frac{\theta}{2} \rb}  \rb}{\rho \cos\left(\frac{\theta}{2}\right)}  \rb } \rr e^{-\rho  \frac{t}{p^\prime} \cos \frac{\theta}{2} }d\rho \notag \\
			\le & CM\int_\varepsilon^\infty e^{-\frac{\rho t}{p^\prime}\cos\left(\frac{\theta}{2}\right)}d\rho <+\infty. \label{richiamatadopo}
	\end{align}
	Hence we can take the limit in \eqref{239} to get
	\begin{equation}	\label{243}
		\lim_{R \to +\infty}\int_{\ell_2} F(z; x,t) \, dz \, = \, -\int_\varepsilon^{+\infty}\frac{\Phi^\dagger\lb\rho e^{i\lb \pi-\frac{\theta}{2} \rb}\rb}{\rho } e^{-x \Phi \lb \rho e^{i\lb \pi-\frac{\theta}{2} \rb} \rb+t  \rho e^{i \lb \pi-\frac{\theta}{2} \rb}} \, d\rho \notag \\ =: \, -I_1(\varepsilon).
	\end{equation}
	Analogously, on $\ell_3$, we have that
	\begin{equation}\label{244}
		\lim_{R \to +\infty}\int_{\ell_3} F(z; x,t) \, dz \, = \, \int_\varepsilon^{+\infty}\frac{\Phi^\dagger\lb\rho e^{i\lb \pi+\frac{\theta}{2} \rb}\rb}{\rho } e^{-x \Phi \lb \rho e^{i\lb \pi+\frac{\theta}{2} \rb} \rb+t  \rho e^{i \lb \pi+\frac{\theta}{2} \rb}} \, d\rho\, =: \, I_2(\varepsilon).
	\end{equation}
	Furthermore, by using the fact that $\overline{\Phi(z)} = \Phi(\overline{z})$ by Schwartz reflection principle (see \cite[Theorem $5.6$]{stein10}), we know that $I_2(\varepsilon) = \overline{I_1(\varepsilon)}$. Hence, taking the limit as $R \to +\infty$ in \eqref{decomp} and using \eqref{circsopra}, \eqref{circsotto}, \eqref{243} and \eqref{244}, we get
	\begin{equation*}
		\begin{split}
				\lim_{b \to +\infty}\int_{a-ib}^{a+ib} e^{z t } 	\,  \frac{\Phi^{\dagger}(z)}{z} e^{-x\Phi(z)} \, dz \, & = \,  I_1(\varepsilon)-\overline{I_1(\varepsilon)} + \, \int_{\gamma_{\varepsilon,\theta}} F(z; x, t) \, dz   =2iI_3(\varepsilon)+\, \int_{\gamma_{\varepsilon,\theta}} F(z; x, t) \, dz,
		\end{split}
	\end{equation*}
	where we denote
	\begin{align}
I_3(\varepsilon):=\int_\varepsilon^{+\infty}\Im\left(\frac{\Phi^\dagger\lb \rho e^{i\left(\pi-\frac{\theta}{2}\right)}\rb}{\rho } e^{-x \Phi \lb \rho e^{i\left(\pi-\frac{\theta}{2}\right)}  \rb+ t \rho e^{i\left(\pi-\frac{\theta}{2}\right)} }\right) \, d\rho.
		\label{beforeepsilon}
	\end{align}
	This proves \eqref{statement1}.
%	Now we would like to take the limit as $\varepsilon \to 0$ to get \eqref{eq:intreptheta}. To do this, let us first handle the integral over $\gamma_{\varepsilon,\theta}$. By the estimation Lemma
%	\begin{equation*}
%		\begin{split}
%			\left| \int_{\gamma_{\varepsilon,\theta}} F(z;x,t) \, dz \right| \, \leq \, & \text{length} (\gamma_{\varepsilon,\theta}) \, \max_{z \in \gamma_{\varepsilon}} \left| F(z; x,t) \right| z
%			= \varepsilon (2\pi - \theta) \max_{z \in \gamma_{\varepsilon}} \left| F(z; x,t) \right|
%		\end{split}
%	\end{equation*}
%	Note now that
%	\begin{equation*}
%		\max_{z \in \gamma_{\varepsilon}} \left| F(z; x,t) \right| \, = \, \frac{1}{\varepsilon}  \max_{z \in \gamma_{\varepsilon}} \left[ \left| \Phi(z) \right| e^{t \Re z -x \Re \Phi (z)} \right]
%	\end{equation*}
%	and thus
%	\begin{align}
%		\left| \int_{\gamma_{\varepsilon,\theta}} F (\lambda; x,t) d\lambda \right| \, \leq \, (2\pi-\theta) \max_{\lambda \in \gamma_{\varepsilon,\theta}} \left[ \left| \Phi^\dagger(\lambda)\right| e^{t\Re \lambda-x\Re \Phi(\lambda)} \right].
%		\label{233}
%	\end{align}
%	By the continuity of $\left| \Phi(z)\right| e^{t\Re z-x\Re \Phi(z)}$ in $\overline{\C\left(\pi-\frac{\theta}{2}\right)}$ (and thus the uniform continuity in $\overline{\C\left(\pi-\frac{\theta}{2}\right)} \cap \{z \in \C: \ |z| \le 1\}$) we clearly have
%	\begin{equation}	\label{cerchietto}
%		\lim_{\varepsilon \to 0} \int_{\gamma_{\varepsilon,\theta}} F(z ;x,t) \, dz\, = \, 0.
%	\end{equation}
%	To take the limit as $\varepsilon \to 0$ in $I_3(\varepsilon)$, just observe that, arguing exactly as in the proof of Proposition \ref{prop:LTthetapi}, we have
%	\begin{align}
%			\int_0^1 \left| \Im  \left(\frac{\Phi^\dagger\l\rho e^{i\l \pi-\frac{\theta}{2} \r}\r}{\rho } e^{-x \Phi \l \rho e^{i\l \pi-\frac{\theta}{2} \r} \r+t  \rho e^{i \l \pi-\frac{\theta}{2} \r}}\right) \right| d\rho \,  \leq \, C  \left(1+\int_0^1  \frac{\left|\Im \Phi\left(\rho e^{i\left(\pi-\frac{\theta}{2}\right)}\right)\right|}{\rho}\right) d\rho<+\infty.
%	\end{align}
%	The statement then follows.
\end{proof}
\begin{rmk}
	Under the hypotheses of Proposition \ref{prop:intreptheta}, if furthermore \begin{equation}\label{eq:intcondtheta}
				\int_0^1\frac{\left|\Im \Phi^\dagger \left(\rho e^{i\left(\pi-\frac{\theta}{2}\right)}\right)\right|}{\rho}d\rho<+\infty
	\end{equation}
	then we can send $\varepsilon \to 0$ in \eqref{statement1}, getting
	\begin{equation*}
		f_\Phi (x,t) \, = \, \frac{1}{\pi} \int_0^{+\infty}\Im\left(\frac{\Phi^\dagger\lb \rho e^{i\left(\pi-\frac{\theta}{2}\right)}\rb}{\rho } e^{-x \Phi \lb \rho e^{i\left(\pi-\frac{\theta}{2}\right)}  \rb+ t \rho e^{i\left(\pi-\frac{\theta}{2}\right)} }\right) \, d\rho.
	\end{equation*}
\end{rmk}
A similar integral representation holds also for $\bar{\mu}_\phi^{\ast n}$. The proof is similar to the one of Proposition \ref{prop:intreptheta}, where we substitute the term $\frac{\phi^\dagger(z)}{z}e^{-x\phi(z)}$ with $\left(\frac{\phi^\dagger(z)}{z}\right)^n$. For such a reason, we only underline the parts of the proof that are actually different.
\begin{prop}
	\label{lem:convtail}
	With the same notation of Proposition \ref{prop:intreptheta}, under \eqref{eq:extensionA3} and \eqref{eq:uniformlimcond}, for $n \ge 0$, it holds that, for any $t>0$,
	\begin{equation}\label{convcode}
		\bar{\mu}_\Phi^{\ast n} (t) = \frac{1}{\pi} \int_{\varepsilon}^{+\infty} \Im \left[ \lb \frac{\Phi^\dagger \lb \rho e^{i \lb \pi-\frac{\theta}{2} \rb} \rb}{\rho e^{i \lb \pi-\frac{\theta}{2} \rb}} \rb^n e^{i \lb \pi-\frac{\theta}{2} \rb}e^{t\rho e^{i \lb \pi-\frac{\theta}{2} \rb}} \right] d\rho +\frac{1}{2\pi i} \int_{\gamma_{\varepsilon,\theta}} e^{tz} \lb \frac{\phi^\dagger (z)}{z} \rb^n dz.
	\end{equation}
%	 Furthermore, it is true that the function $(0, +\infty) \mapsto \bar{\nu}_\phi^{\star n} (t)$, has, for any $n=1,2,\cdots$, derivatives of all order $r \geq 0$, such that
%	\begin{equation}
%		\frac{d^r}{dt^r} \bar{\mu}_\Phi^{n\star} (t) 
%		= \frac{1}{\pi} \int_{\varepsilon}^{+\infty} \Im \left[  \frac{\l\Phi^\dagger \l \rho e^{i \l \pi-\frac{\theta}{2} \r} \r\r^n}{\rho^{n-r}e^{i(n-r-1)\l \pi-\frac{\theta}{2} \r}}  e^{t\rho e^{i \l \pi-\frac{\theta}{2} \r}} \right] d\rho -\frac{1}{2\pi i} \int_{\gamma_{\varepsilon,\theta}} e^{tz}  \frac{\l\phi^\dagger (z)\r^n}{z^{n-r}}  dz.
%		\label{derivatecode}
%	\end{equation}
\end{prop}
\begin{proof}
	Setting $F_n(z;t)=\left(\frac{\phi^\dagger(z)}{z}\right)^ne^{tz}$ the proof follows as the one in Proposition \ref{prop:intreptheta}. The main differences concern integrals over $\Gamma^2_R$ and $\ell_2$. First, by Jordan's inequality \cite[eq. (2), page 262]{brownchurchill},
	\begin{equation*}
		\left|\int_{\Gamma^2_R}F_n(z;t)dz\right| \le \frac{\pi}{t}\left(\max_{z \in \Gamma_R^2}\left|\frac{\Phi^\dagger(z)}{z}\right|^n\right) \to 0, \mbox{as $R\to\infty$},
	\end{equation*}
	where the limit holds by assumption \eqref{eq:uniformlimcond}. Concerning the integral over $\ell_2$, observe that
	\begin{equation*}
		\int_{\varepsilon}^{\infty}|F_n\left(\rho e^{i\left(\pi-\frac{\theta}{2};t\right)};t\right)|dz \le \int_{\varepsilon}^{\infty}e^{-\rho t \cos\left(\frac{\theta}{2}\right)}\left|\frac{\Phi^\dagger\left(\rho e^{i\left(\pi-\frac{\theta}{2}\right)}\right)}{\rho}\right|^nd\rho \le C\int_{\varepsilon}^{\infty}e^{-\rho t \cos\left(\frac{\theta}{2}\right)} d\rho<+\infty,
	\end{equation*}
	where we have used the fact that $\left|\frac{\Phi^\dagger\left(\rho e^{i\left(\pi-\frac{\theta}{2}\right)}\right)}{\rho}\right|^n$ is bounded for $\rho \ge \varepsilon$ by assumption \eqref{eq:uniformlimcond}.
\end{proof}
Now we are ready to prove Theorem \ref{thm:smoothfmu} by using the previously obtained integral representations.
\begin{proof}[Proof of Theorem \ref{thm:smoothfmu}]
	Let us first prove that $\bar{\mu}_\phi^{\ast n} \in C^\infty(0,+\infty)$. To do this, fix $\varepsilon>0$, let $l \ge 1$ and $[t_1,t_2]\subset (0,+\infty)$. Let $F_n(z;t)=\left(\frac{\phi^\dagger(z)}{z}\right)^ne^{tz}$. Then $\frac{\partial^l}{\partial t^l}F_n(z,t)=z^l\left(\frac{\phi^\dagger(z)}{z}\right)^ne^{tz}$ is continuous for $(z,t) \in \gamma_{\varepsilon,\theta} \times [t_1,t_2]$, where, with an abuse of notation, $\gamma_{\varepsilon,\theta}$ is the support of the parametrized curve defined in Proposition \ref{prop:intreptheta}. Then we have
	\begin{equation}\label{eq:upboundovereps}
		\left|\frac{\partial^l}{\partial t^l}F_n(z,t)\right| \le \max_{(z,t) \in \gamma_{\varepsilon,\theta} \times [t_1,t_2]}\left|z^l\left(\frac{\phi^\dagger(z)}{z}\right)^ne^{tz}\right|
	\end{equation}
	where the right-hand side is constant, hence integrable over $\gamma_{\varepsilon,\theta}$. Next, let
	\begin{equation*}
		G_n(\rho,t)=\Im \left[ \lb \frac{\Phi^\dagger \lb \rho e^{i \lb \pi-\frac{\theta}{2} \rb} \rb}{\rho e^{i \lb \pi-\frac{\theta}{2} \rb}} \rb^n e^{i \lb \pi-\frac{\theta}{2} \rb}e^{t\rho e^{i \lb \pi-\frac{\theta}{2} \rb}} \right]
	\end{equation*}
	and observe that
	\begin{equation*}
		\frac{\partial^l}{\partial t^l}G_n(\rho,t)=\Im \left[ \lb \frac{\Phi^\dagger \lb \rho e^{i \lb \pi-\frac{\theta}{2} \rb} \rb}{\rho e^{i \lb \pi-\frac{\theta}{2} \rb}} \rb^n \rho^l e^{i(l+1) \lb \pi-\frac{\theta}{2} \rb}e^{t\rho e^{i \lb \pi-\frac{\theta}{2} \rb}} \right].
	\end{equation*}
	For $\rho \ge \varepsilon$ and $t \in [t_1,t_2]$, we have
	\begin{equation}\label{eq:contrder2}
		\left|\frac{\partial^l}{\partial t^l}G_n(\rho,t)\right| \le C e^{-t_1 \rho \cos \frac{\theta}{2}} \rho^l,
	\end{equation}
	since $\left|\frac{\Phi^\dagger\left(\rho e^{i\left(\pi-\frac{\theta}{2}\right)}\right)}{\rho}\right|^n$ is bounded for $\rho \ge \varepsilon$ by assumption \eqref{eq:uniformlimcond}. Observe that the right-hand side of \eqref{eq:contrder2} is integrable over $(\varepsilon,+\infty)$. Hence, by \eqref{eq:upboundovereps} and \eqref{eq:contrder2} and the fact that $l \ge 1$ is arbitrary, we can differentiate $l$ times inside the integrals in \eqref{convcode}, getting	
	\begin{equation}
		\frac{d^r}{dt^r} \bar{\mu}_\Phi^{\ast n} (t) 
		= \frac{1}{\pi} \int_{\varepsilon}^{+\infty} \Im \left[  \frac{\lb\Phi^\dagger \lb \rho e^{i \lb \pi-\frac{\theta}{2} \rb} \rb\rb^n}{\rho^{n-r}e^{i(n-r-1)\lb \pi-\frac{\theta}{2} \rb}}  e^{t\rho e^{i \lb \pi-\frac{\theta}{2} \rb}} \right] d\rho +\frac{1}{2\pi i} \int_{\gamma_{\varepsilon,\theta}} e^{tz}  \frac{\lb\phi^\dagger (z)\rb^n}{z^{n-r}}  dz.
		\label{derivatecode}
	\end{equation}
	This ends the first part of the proof.
	
	Now let us prove that $f_\Phi \in C^\infty(\mathbb{D})$. To do this, fix any $k,l \ge 0$, $0<t_1<t_2$ and $0<x_1<x_2<t_1/\mathfrak{b}$, recalling that any $(x,t) \in \mathbb{D}$ admits a compact neighbourhood of the form $[x_1,x_2]\times [t_1,t_2]$ specified before. Let $F(z;x,t)=\frac{\phi^\dagger(z)}{z}e^{-x\phi(z)+tz}$ and observe that
	\begin{equation*}
		\frac{\partial^k}{\partial x^k}\frac{\partial^l}{\partial t^l}F(z;x,t)=(-1)^k\frac{z^l\phi^\dagger(z)(\phi(z))^k}{z}e^{-x\phi(z)+tz}.
\end{equation*} 
	The latter is continuous over $\gamma_{\varepsilon,\theta} \times [x_1,x_2]\times [t_1,t_2]$ and then
	\begin{equation}\label{eq:smoothf1}
		\left|\frac{\partial^k}{\partial x^k}\frac{\partial^l}{\partial t^l}F(z;x,t)\right|\le \max_{(z,x,t) \in \gamma_{\varepsilon,\theta} \times [x_1,x_2]\times [t_1,t_2]}\left|\frac{z^l\phi^\dagger(z)(\phi(z))^k}{z}e^{-x\phi(z)+tz}\right|,
	\end{equation}
	where the right-hand side is a constant and then it is integrable over $\gamma_{\varepsilon,\theta}$. Now set
	\begin{equation*}
		G(\rho;x,t)=\Im\left(\frac{\Phi^\dagger\lb \rho e^{i\left(\pi-\frac{\theta}{2}\right)}\rb}{\rho } e^{-x \Phi \lb \rho e^{i\left(\pi-\frac{\theta}{2}\right)}  \rb+ t \rho e^{i\left(\pi-\frac{\theta}{2}\right)} }\right)
\end{equation*}
	and observe that
	\begin{equation*}
		\frac{\partial^k}{\partial x^k}\frac{\partial^l}{\partial t^l}G(\rho;x,t)=(-1)^k\Im\left(\frac{\Phi^\dagger\lb \rho e^{i\left(\pi-\frac{\theta}{2}\right)}\rb\left(\Phi\left(\rho e^{i\left(\pi-\frac{\theta}{2}\right)}\right)\right)^k}{\rho^{k+1}}\rho^{l+k} e^{il\left(\pi-\frac{\theta}{2}\right)}e^{-x \Phi \lb \rho e^{i\left(\pi-\frac{\theta}{2}\right)}  \rb+ t \rho e^{i\left(\pi-\frac{\theta}{2}\right)} }\right).
	\end{equation*}
	Recalling that $\left|\frac{\Phi^\dagger\lb \rho e^{i\left(\pi-\frac{\theta}{2}\right)}\rb\left(\Phi\left(\rho e^{i\left(\pi-\frac{\theta}{2}\right)}\right)\right)^k}{\rho^{k+1}}\right|$ is bounded as $\rho \ge \varepsilon$ and setting $p,p^\prime >1$ and $M>0$ as in the proof of Proposition \ref{prop:intreptheta} with respect to $t_2$ and $x_1$, we have
	\begin{equation}\label{smoothfphi2}
		\left|\frac{\partial^k}{\partial x^k}\frac{\partial^l}{\partial t^l}G(\rho;x,t)\right|\le CM e^{-\frac{\rho t_1}{p^\prime}\cos \frac{\theta}{2}},
	\end{equation}
	 where the right-hand side is integrable on $(\varepsilon,+\infty)$. Hence, since $k,l \ge 0$ are arbitrary, we can differentiate $k$ times in $x$ and $l$ times with respect to $t$ in \eqref{statement1}, to get that $f_\Phi \in C^\infty(\mathbb{D})$ and
	 \begin{align}\label{derivatefphi}
	 	\begin{split}
	 	\frac{\partial^k}{\partial x^k}\frac{\partial^l}{\partial t^l}f_\phi(x,t)&= \frac{(-1)^k}{\pi} \int_\varepsilon^{+\infty}\Im\left(\frac{\Phi^\dagger\lb \rho e^{i\left(\pi-\frac{\theta}{2}\right)}\rb\lb\Phi\lb \rho e^{i\left(\pi-\frac{\theta}{2}\right)}\rb\rb^{k}}{\rho^{1-l} e^{-i l \left(\pi-\frac{\theta}{2}\right)}}e^{-x \Phi \lb \rho e^{i\left(\pi-\frac{\theta}{2}\right)}  \rb+ t \rho e^{i\left(\pi-\frac{\theta}{2}\right)} }\right) \, d\rho \\
	 	&+\frac{(-1)^k}{2\pi i}\int_{\gamma_{\varepsilon,\theta}} \frac{\Phi^\dagger(z)(\Phi(z))^{k}}{z^{1-l}}e^{-x\Phi(z)+tz}dz.
		\end{split} 
	 \end{align}
\end{proof}
A slight modification of the previous arguments lead to the proof of Proposition \ref{prop:extcont}.
\begin{proof}[Proof of Proposition \ref{prop:extcont}]
	Since $\phi$ is a complete Bernstein function that can be extended by continuity over $\overline{\C(0,\pi)}$ with extension $\phi_+$, then, by using the relation $\overline{\phi(z)}=\phi(\overline{z})$, it is clear that it can be also extended by continuity over $\overline{\C(-\pi,0)}$. If we denote such an extension $\phi_-$, for any $z \in \overline{\C(0,\pi)}$ we get
	\begin{equation*}
		\overline{\phi_+(z)}=\phi_-(\overline{z}).
	\end{equation*}
	The proof is then carried on exactly as in Propositions \ref{prop:intreptheta} and \ref{lem:convtail}, by setting $\theta=0$ (see Figure \ref{fig2}), where we use $\phi^\dagger$ when we integrate over $\ell_1$, $\Gamma_R^+$, $\Gamma_R^-$ and $-\gamma_\varepsilon$, $\phi^\dagger_+$ over $\ell_2$ and $\phi^\dagger_-$ over $\ell_1$.
\end{proof}
Let us give, for completeness, the integral representations for $G_\Phi$ and $g_\Phi$, whose proof is identical to the one of Proposition \ref{prop:intreptheta}.
\begin{prop}
	Let $\Phi$ be the Laplace exponent of a potentially killed subordinator satisfying assumptions \ref{eq:extensionA3} and \eqref{eq:uniformlimcond} for some $\theta \in (0,\pi)$. Fix any $\varepsilon>0$ and let $\gamma_{\varepsilon,\theta}$ be defined as in Proposition \ref{prop:intreptheta}. Then, on $\mathbb{D}$,
	\begin{equation}\label{statementG}
		\begin{split}
			G_\Phi (x,t) \, = \,& \frac{1}{\pi} \int_\varepsilon^{+\infty}\Im\left(\frac{e^{-x \Phi \lb \rho e^{i\left(\pi-\frac{\theta}{2}\right)}  \rb+ t \rho e^{i\left(\pi-\frac{\theta}{2}\right)} }}{\rho} \right) \, d\rho  +\frac{1}{2\pi i}\int_{\gamma_{\varepsilon,\theta}} \frac{e^{-x\Phi(z)+tz}}{z}dz.
		\end{split}
	\end{equation}
	In particular, $G_\Phi \in C^\infty(\mathbb{D})$, $g_\Phi \in C^\infty(\mathbb{D})$ is well defined and for any $k,l \ge 0$ we have
	\begin{equation}\label{statementG2}
		\begin{split}
			\frac{\partial^k}{\partial x^k}\frac{\partial^l}{\partial t^l}g_\Phi (x,t) \, = \,& \frac{1}{\pi} \int_\varepsilon^{+\infty}\Im\left(\left(\Phi\left(\rho e^{i\left(\pi-\frac{\theta}{2}\right)}\right)\right)^k\rho^le^{il\left(\pi-\frac{\theta}{2}\right)}e^{-x \Phi \lb \rho e^{i\left(\pi-\frac{\theta}{2}\right)}  \rb+ t \rho e^{i\left(\pi-\frac{\theta}{2}\right)} } \right) \, d\rho  \\
			&+\frac{1}{2\pi i}\int_{\gamma_{\varepsilon,\theta}} (\Phi(z))^kz^le^{-x\Phi(z)+tz}dz.
		\end{split}
	\end{equation}
\end{prop}

%The following statement, which is of independent interest, plays a central role in the proof of Theorem \ref{thm:seriespi}, as it guarantees the smoothness of the $n$-fold convolution appearing in \eqref{coeff}.
%\begin{lem}
%	\label{lem:convtail}
%	Suppose that \eqref{eq:extensionA3} and \eqref{eq:uniformlimcond} are verified. Define, for a fixed $\varepsilon >0$, the set $\gamma_{\varepsilon,\theta} = \ll z \in \mathbb{C}: z=\varepsilon e^{i\xi}, \xi \in \left[ \frac{\theta}{2}-\pi, \pi- \frac{\theta}{2} \right] \rr$. Then we have, for any $n =1,2 \cdots$,  that
%	\begin{equation}\label{convcode}
%		\bar{\mu}_\Phi^{n\star} (t) = \frac{1}{\pi} \int_{\varepsilon}^{+\infty} \Im \left[ \l \frac{\Phi^\dagger \l \rho e^{i \l \pi-\frac{\theta}{2} \r} \r}{\rho e^{i \l \pi-\frac{\theta}{2} \r}} \r^n e^{i \l \pi-\frac{\theta}{2} \r}e^{t\rho e^{i \l \pi-\frac{\theta}{2} \r}} \right] d\rho -\frac{1}{2\pi i} \int_{\gamma_{\varepsilon,\theta}} e^{tz} \l \frac{\phi^\dagger (z)}{z} \r^n dz,
%	\end{equation}
%	for any $t>0$. Furthermore, it is true that the function $(0, +\infty) \mapsto \bar{\nu}_\phi^{\star n} (t)$, has, for any $n=1,2,\cdots$, derivatives of all order $r \geq 0$, such that
%	\begin{equation}
%		\frac{d^r}{dt^r} \bar{\mu}_\Phi^{n\star} (t) 
%		= \frac{1}{\pi} \int_{\varepsilon}^{+\infty} \Im \left[  \frac{\l\Phi^\dagger \l \rho e^{i \l \pi-\frac{\theta}{2} \r} \r\r^n}{\rho^{n-r}e^{i(n-r-1)\l \pi-\frac{\theta}{2} \r}}  e^{t\rho e^{i \l \pi-\frac{\theta}{2} \r}} \right] d\rho -\frac{1}{2\pi i} \int_{\gamma_{\varepsilon,\theta}} e^{tz}  \frac{\l\phi^\dagger (z)\r^n}{z^{n-r}}  dz.
%		\label{derivatecode}
%	\end{equation}
%\end{lem}
