\section{Main results concerning densities and their derivatives}\label{sec:mainResults}
In this section we present results concerning $\fP{x,t}$, see \eqref{def:f}, which in the case of $\mathfrak{b}=q=0$ is the (entire) density of $L(t)$. The results are in two directions - series expansion and behaviour at zero, and large asymptotics together with speed of convergence for $\fP{x,t}$ and its derivatives. We complement the latter results by providing information about $f_\phi^k(x,t)$ and $f_\phi^c(x,t)$ (see respectively \eqref{def:fk} and \eqref{def:fc}). We also furnish results for the density of $\sigma(x)$ and its derivatives. 
 
 \subsection{Behaviour at infinity of densities and their derivatives of the inverse subordinator and of the subordinator itself}\label{subsec:L}
 
 In this part we offer results concerning the large asymptotic behaviour of densities (and their derivatives) of inverse subordinators and subordinators themselves. Since, via the saddle point method, we prove the results for all derivatives simultaneously, we impose a condition which may not be optimal but is general enough and easy to implement. For this purpose, we set
 \begin{equation}\label{def:D}
 	\begin{split}
 		&\Delta(x):=\int_0^{\frac1x}y^2\nu_\Phi(dy),\,x>0,
 	\end{split}
 \end{equation}
 for the truncated second moment of the unkilled and driftless version of $\sigma$. Then, the main condition in our work is 
 \begin{equation}\label{def:condiA}
 	\begin{split}
 		&\liminfi{x}\frac{x^2\Delta(x)}{\ln(x)}=L\in\lbrbb{0,\infty}.
 	\end{split}\tag{$\mathbb{A}_1$}
 \end{equation}
The mild requirement
\begin{equation}\label{def:condiB}
	\begin{split}
		&\limsupi{x}\frac{x\Phi'''(x)}{-\Phi''(x)}=K<\infty
	\end{split}\tag{$\mathbb{A}_2$}
\end{equation}
plays a role only in some results. We discuss these conditions in Section \ref{subsubsec:EX}, where we present alternative formulation of \eqref{def:condiB} and demonstrate that they are generally milder than those in the literature. The proofs of all the results in this section are given in Section \ref{sec:proofs}.
First of all, let us underline that condition $\eqref{def:condiA}$ is related to the regularity of the function $f_\Phi$ defined in \eqref{def:f} with domain $\mathbb{D}$, see \eqref{def:Reg}, which is the content of our first result.
\begin{thm}\label{thm:regularityfphi1}
	Let $\Phi$ be the Laplace exponent of some potentially killed subordinator and assume that condition $\eqref{def:condiA}$ holds. Then, for any $n \ge 0$, there exists $x_0(n,L) \ge 0$ such that, for any $k,l \ge 0$ with $k+l \le n$, and for any $x>x_0(n,L)$ and $(x,t) \in \mathbb{D}$, $\displaystyle \frac{\partial^k}{\partial x^k}\frac{\partial^l}{\partial t^l}f_\phi(x,t)$ is well-defined and for any $a>0$
	\begin{equation}
		\frac{\partial^l}{\partial t^l} \frac{\partial^k}{\partial x^k} f_\Phi(x,t) \, = \, (-1)^k \int_{-\infty}^{+\infty} \frac{\Phi^\dagger (a+ib) (\Phi(a+ib))^k }{(a+ib)^{1-l}} e^{t(a+ib)-x\Phi(a+ib)} db,
		\label{intreprder1}
	\end{equation}
	where the integral is absolutely convergent. If $L=\infty$ in \eqref{def:condiA}, then for any $n \geq 0$, $x_0(n,\infty)=0$.
%	
%	 above is infinitely differentiable in both variables on $\mathbb{D}$ and, if $L<\infty$, then, for any fixed $k,l\geq 0$, $\frac{\partial^k}{\partial x^k} \frac{\partial^l}{\partial t^l}f_\phi(t,x)$  exists for points in $\mathbb{D}$, for which $x$ is large enough.
\end{thm}
Theorem \ref{thm:regularityfphi1} 
%is proved in Section \ref{sec:proofs} and it 
is needed for the study of the asymptotic behaviour of $f_\Phi$. The next theorem is the first result in this direction and considers the behaviour of $f_\Phi(x,t)$ when $t/x\downarrow \mathfrak{b},$ as $x\to\infty$. 
%It is proved in Section \ref{sec:proofs}.
\begin{thm}\label{thm:mainL}
	Let $\Phi$ be the Laplace exponent of some potentially killed subordinator and assume that  conditions $\eqref{def:condiA}$ and $\eqref{def:condiB}$ hold true. Let also $t(x)$ be such that $t(x)/x\in\lbrb{\mathfrak{b},\Phi'(0^+)}$ and  $\limi{x}t(x)/x=\mathfrak{b}$. Consider 
	\begin{equation}\label{eq:a*}
		a_*:=a_*(x)=(\phi')^{-1}\lbrb{\frac{t(x)}{x}}\in\lbrb{0,\infty},
		\end{equation}
		that is well-defined since $\phi'$ is decreasing. Define also the set
		\begin{equation}\label{def:region}
			\mathbb{D}^\prime=\{(t,x): \ x\mathfrak{b} < t \le t(x) < x\phi'(0+)\}
		\end{equation}
		and, for $(t,x) \in \mathbb{D}^\prime$, let $c:=c(t,x)=(\phi')^{-1}(t/x)\geq a_*$. Then, for any $k\geq0,l\geq0$, as $x\to\infty$,
	\begin{equation}\label{asymp}
		\begin{split}
			\sup_{x\mathfrak{b}<t\le t(x)}\abs{(-1)^k\sqrt{2\pi}\frac{c^{1-l}\sqrt{-\Phi''(c)x}}{\Phi^\dagger(c)\Phi^{k}(c)}e^{-ct+x\Phi(c)}\frac{\partial^k\partial ^l }{\partial x^k\partial t^l}\fP{x,t}-1}&=\bo{\sup_{c\geq a_*(x)}\frac{\sqrt{\ln\lbrb{c\sqrt{-\Phi''(c)x}}} }{c\sqrt{-\Phi''(c)x}}}\\
			&=\bo{\sqrt{\frac{\ln(x)}{x\ln (a_*(x))}}}.
		\end{split}
	\end{equation}
	%\begin{equation}\label{asymp}
	%	\begin{split}
	%		 \sup_{c\geq a_*}\abs{(-1)^k\sqrt{2\pi}\frac{c^{1-l}\sqrt{-\Phi''(c)x}}{\Phi^\dagger(c)\Phi^{k}(c)}e^{-ct+x\Phi(c)}\frac{\partial^k\partial ^l }{\partial x^k\partial t^l}\fP{x,t}-1}&=\bo{\sup_{c\geq a_*}\frac{\sqrt{\ln\lbrb{c\sqrt{-\Phi''(c)x}}} }{c\sqrt{-\Phi''(c)x}}}\\
	%		&=\bo{\sqrt{\frac{\ln(x)}{x\ln a_*}}}.
	%	\end{split}
	%\end{equation}
	Furtherore, $1/a_*=\so{t(x)/x-\mathfrak{b}}$, as $x \to \infty$.
	% all $x$ large enough. 
\end{thm}
\begin{rmk}\label{rem:mainL} In Section \ref{subsubsec:EX} we show that \eqref{def:condiA} and \eqref{def:condiB} are generally milder than those in \cite{DonRiv}. $\eqref{def:condiB}$ holds for some Compound Poisson processes, e.g., when $\nu_\Phi(dy)=e^{-y}dy$ and cannot be deduced from  \eqref{def:condiA} which implies $\Phi(\infty)=\infty$. \eqref{def:condiA} often holds with $L=\infty$ as in the example in Section \ref{subsubsec:EX}. 
\end{rmk}
\begin{rmk}\label{rem:mainL1}
For this theorem and all others concerning the large asymptotic behaviour one can laboriously track  the constants in the speed of convergence, see \eqref{asymp}, and thereby obtain strict upper bounds on the densities and its derivatives in line with  \cite{ChoKim21,GrLTr21}. For lower bounds the saddle point method is expected to work too. For more information see the discussion in Section \ref{subsubsec:EX}.
\end{rmk}
Next, for the sake of clarity, we formulate a corollary which deals with the most usual case, i.e. when $\mathfrak{b}=q=0$ and $L=\infty$ in \eqref{def:condiA}.
\begin{coro}\label{cor:mainL}
	Let $\phi$ be a Laplace exponent of a subordinator with $\mathfrak{b}=q=0$,  let \eqref{def:condiB} hold and \eqref{def:condiA} be valid with $L=\infty$. Then $f_\Phi(x,t)$ is infinitely differentiable on $\Rb^+\times \Rb^+$. Furthermore, fix $t_\ast>0$ and consider
	\begin{equation}\label{eq:a*1}
		a_\ast:=a_\ast(x)=(\Phi')^{-1}\left(\frac{t_\ast}{x}\right),
	\end{equation}
	that is well-defined since $\Phi'$ is decreasing. Define the set 
	 \begin{equation*}
	 	\mathbb{D}^\prime=\{(t,x):\, 0 < t \le t_\ast < x\phi'(0+)\}
	 \end{equation*}
	 and, for $(t,x) \in \mathbb{D}^\prime$, let $c:=c(t,x)=(\phi')^{-1}(t/x)\geq a_*$. Then, $1/a_*=o(1/x)$ as $x \to \infty$, and,  for any $k\geq0,l\geq0$, as $x\to\infty$,
	 \begin{equation}\label{asymp1}
	 	\begin{split}
	 		\sup_{0< t \le t_\ast}\abs{(-1)^k\sqrt{2\pi}\frac{c^{1-l}\sqrt{-\Phi''(c)x}}{\Phi^{k+1}(c)}e^{-ct+x\Phi(c)}\frac{\partial^k\partial ^l }{\partial x^k\partial t^l}\fP{x,t}-1}=\bo{\sqrt{\frac{1}{x}}}.
	 	\end{split}
	 	\end{equation}
%	  and, \mladen{for any fixed $t>0$}, one has, for any $k,l\geq 0$, 
%	\begin{equation}\label{asymp1}
%		\begin{split}
%			& \frac{\partial^k\partial ^l }{\partial x^k\partial t^l}\fP{x,t}=\frac{(-1)^k}{\sqrt{2\pi}}e^{a_*t-x\Phi(a_*)}\frac{\Phi^{k+1}(a_*)}{a^{1-l}_*\sqrt{-\Phi''(a_*)x}}\lbrb{1+\bo{\frac{\sqrt{\ln\lbrb{a_*\sqrt{-\Phi''(a_*)x}}} }{a_*\sqrt{-\Phi''(a_*)x}}}}, \ \mbox{ as }x \to \infty,
%		\end{split}
%	\end{equation}
%	where $a_*=a_*(x)$ is the unique solution of
%	\begin{equation}\label{eq:a*1}
%		\Phi'(a_*)=\frac{\giacomo{t}}{x}\in\lbrb{0,\Phi'(0^+)},
%	\end{equation}
%	which in addition satisfies $\limi{x}a_*x^{-1}=\infty$.
\end{coro}
Next, we consider the case when $t/x$ does not converge to $\mathfrak{b}$ or $\phi'(0^+)$. 
\begin{thm}\label{thm:main1}
	Let $\Phi$ be the Laplace exponent of some potentially killed subordinator and assume that \eqref{def:condiA} holds. For all $[t_1,t_2]\subset \lbrb{\mathfrak{b},\Phi'(0^+)}$ define correspondingly
	%\begin{equation*} 
	%	\mathbb{D}'=\curly{(t,x) \in \mathbb{D}: \ xt_1\leq t  \leq  xt_2}, \ \mbox{ and } \  c:=c(t,x)=(\phi')^{-1}\left(t/x\right)\in \lbbrbb{(\phi')^{-1}(t_2),(\phi')^{-1}(t_1)}, \ (x,t) \in \mathbb{D}^\prime.
	%\end{equation*} 
	\begin{equation*} 
		\mathbb{D}'=\curly{(t,x) \in \mathbb{D}: \ xt_1\leq t  \leq  xt_2}, \quad \mbox{ and } \quad   c:=c(t,x)=(\phi')^{-1}\left(t/x\right), \ \mbox{ for } (x,t) \in \mathbb{D}^\prime.
	\end{equation*}
	Then, for any $k\geq0,l\geq0$, we have, as $x\to\infty$
	\begin{equation}\label{def:fin}
		\begin{split}
			\sup_{xt_1 \le t \le xt_2}\abs{(-1)^k\sqrt{2\pi}\frac{c^{1-l}\sqrt{-\Phi''(c)x}}{\Phi^\dagger(c)\Phi^{k}(c)}e^{-ct+x\Phi(c)}\frac{\partial^k\partial ^l }{\partial x^k\partial t^l}\fP{x,t}-1}=\bo{\sqrt{\frac{\ln(x)}{x}}}.
		\end{split}
	\end{equation}
\end{thm}
%Theorem \ref{thm:main1} is proved in Section \ref{sec:proofs}.
\begin{rmk}\label{rem:main1}
	 For $f_\Phi$ only, such a result is contained in \cite[Theorem 3.2]{DonRiv}. It will be discussed in detail in Section \ref{subsubsec:EX}. 
\end{rmk}








%The first result considers the  behaviour of $f_\Phi(x,t)$ when $t/x\downarrow \mathfrak{b},$ as $x\to\infty$.
%\begin{thm}\label{thm:mainL}
%	Let $\Phi$ be the Laplace exponent of some potentially killed subordinator. Assume that  conditions $\eqref{def:condiA}$ and $\eqref{def:condiB}$ hold true. If $L=\infty$ in \eqref{def:condiA} then $f_\phi(x,t)$ defined in \eqref{def:f} above is infinitely differentiable in both variables on $\mathbb{D}$ and, if $L<\infty$, then, for any fixed $k,l\geq 0$, $\frac{\partial^k}{\partial x^k} \frac{\partial^l}{\partial t^l}f_\phi(t,x)$  exists for points in $\mathbb{D}$, for which $x$  is large enough. 
%	
%	If $t=t(x)$ is such that $t(x)/x\in\lbrb{\mathfrak{b},\Phi'(0^+)}$ and  $\limi{x}t(x)/x=\mathfrak{b}$ from above, then, for any $k\geq0,l\geq0$, we have
%	\begin{equation}\label{asymp}
%		\begin{split}
%			& \frac{\partial^k\partial ^l }{\partial x^k\partial t^l}\fP{x,t}=\frac{(-1)^k}{\sqrt{2\pi}}e^{a_*t-x\Phi(a_*)}\frac{\Phi^\dagger(a_*)\Phi^{k}(a_*)}{a^{1-l}_*\sqrt{-\Phi''(a_*)x}}\lbrb{1+\bo{\frac{\sqrt{\ln\lbrb{a_*\sqrt{-\Phi''(a_*)x}}} }{a_*\sqrt{-\Phi''(a_*)x}}}} \ \mbox{ as }x \to \infty,
%		\end{split}
%	\end{equation}
%	where $a_*:=a_*(x)$ is the unique solution of
%	\begin{equation}\label{eq:a*}
%		\Phi'(a_*)=\frac{t(x)}{x}\in\lbrb{\mathfrak{b},\Phi'(0^+)}.
%	\end{equation}
%	\ga{Furthermore, for any $M<L$,} $a_*\geq Me^{-1}(t/x-\mathfrak{b})^{-1}$ for all $x$ large enough. 
%\end{thm}
%\begin{rmk}\label{rem:mainL} In part \ref{subsubsec:EX} we compare \eqref{def:condiA}, \eqref{def:condiB} to the conditions in \cite{DonRiv}. Here, we point out that $\eqref{def:condiB}$ is very mild as it holds \ga{even} for some Compound Poisson processes, e.g., when $\nu_\Phi(dy)=e^{-y}dy$. We think it cannot be deduced from the more stringent \eqref{def:condiA}, which implies that $\Phi(\infty)=\infty$. \eqref{def:condiA} typically holds with $L=\infty$ as in the regular variation case treated as a standing example in \ga{Section} \ref{subsubsec:EX}. 
%\end{rmk}
%\begin{rmk}\label{rem:mainL1}
%	The asymptotic \eqref{asymp} is in fact uniform. Indeed,  since  $\phi'$ is decreasing then $a_*$ in \eqref{eq:a*} is monotone in $t/x$. \ga{Hence}, if $t/x -\mathfrak{b}\leq h(x)$, with $\limo{x}h(x)=0$, then \eqref{asymp} is valid uniformly in $t$ on this region with $a_*$ evaluated for $t=\mathfrak{b}x+xh(x)$. Even more, for this theorem and all others concerning the large asymptotic behaviour one can track explicitly the constants in the speed of convergence, see \eqref{asymp}, and thereby obtain strict bounds on the densities and its derivatives. 
%\end{rmk}
%Next, for the sake of clarity, we formulate a corollary which deals with the most usual case, i.e. when $\mathfrak{b}=q=0$ and $L=\infty$ in \eqref{def:condiA}.
%\begin{coro}\label{cor:mainL}
%	Let $\phi$ be a Laplace exponent of subordinator such that $\mathfrak{b}=q=0$,  let \eqref{def:condiB} hold and \eqref{def:condiA} be valid with $L=\infty$. Then $f_\Phi(x,t)$ is infinitely differentiable on $\Rb^+\times \Rb^+$ and, for any fixed $t>0$, one has, for any $k,l\geq 0$, 
%	\begin{equation}\label{asymp1}
%		\begin{split}
%			& \frac{\partial^k\partial ^l }{\partial x^k\partial t^l}\fP{x,t}\stackrel{\infty}{=}\frac{(-1)^k}{\sqrt{2\pi}}e^{a_*t-x\Phi(a_*)}\frac{\Phi^{k+1}(a_*)}{a^{1-l}_*\sqrt{-\Phi''(a_*)x}}\lbrb{1+\bo{\frac{\sqrt{\ln\lbrb{a_*\sqrt{-\Phi''(a_*)x}}} }{a_*\sqrt{-\Phi''(a_*)x}}}},
%		\end{split}
%	\end{equation}
%	where $a_*=a_*(x)$ is the unique solution of
%	\begin{equation}\label{eq:a*1}
%		\Phi'(a_*)=\frac{t(x)}{x}\in\lbrb{0,\Phi'(0^+)},
%	\end{equation}
%	which in addition satisfies $\limi{x}a_*x^{-1}=\infty$.
%\end{coro}
%Next, we consider the case when $t/x$ does not converge to $\mathfrak{b}$ or $\phi'(0^+)$. 
%\begin{thm}\label{thm:main1}
%	Let $\Phi$ be the Laplace exponent of some potentially killed subordinator. Assume that \eqref{def:condiA} holds. If $L=\infty$  in \eqref{def:condiA} then $f_\phi(x,t)$ defined in \eqref{def:f} above is infinitely differentiable in both variables on $\mathbb{D}$, see \eqref{def:D}, and if $L<\infty$ then, for any fixed $k,l\geq 0$, $\frac{\partial^k}{\partial x^k} \frac{\partial^l}{\partial t^l}f(t,x)$  exists for points in $\mathbb{D}$, for which $x$  is large enough. If $t=t(x)$ is such that $t(x)/x\in\lbrb{\mathfrak{b},\Phi'(0^+)}$ and  $\limi{x}t(x)/x=\bar{x}\in\lbrb{\mathfrak{b},\Phi'(0^+)}$, then, for any $k\geq0,l\geq0$, we have 
%	\begin{equation}\label{def:fin}
%		\begin{split}
%			&\frac{\partial^k\partial ^l }{\partial x^k\partial t^l}\fP{x,t}\stackrel{\infty}{=}\frac{(-1)^k\Phi^\dagger(\bar{a})\Phi^{k}(\bar{a})}{\sqrt{2\pi}\bar{a}^{1-l}\sqrt{-\Phi''(\bar{a})}}\frac{e^{\bar{a}t-x\Phi(\bar{a})}}{\sqrt{x}} \lbrb{1+\bo{\sqrt{\ln(x)}x^{-\frac12}}},
%		\end{split}
%	\end{equation}
%	where $\bar{a}$ is defined via the limit 
%	\begin{equation}\label{eq:a*2}
%		\Phi'(\bar{a})=\limi{x}\Phi'(a_*)=\limi{x}\frac{t(x)}{x}=\bar{x}.
%	\end{equation}
%\begin{rmk}\label{rem:main1}
%	Due to $\bar{a}<\infty$, see \eqref{eq:a*2}, condition \eqref{def:condiB} is not needed and the speed of convergence and the asymptotics  are explicit. For $f_\Phi$ only, such a result is contained in Theorem 2 of \cite{DonRiv} and we discuss in part \ref{subsubsec:EX} the conditions. Following the argument in Remark \ref{rem:mainL1} the asymptotic in \eqref{def:fin} is uniform for $t/x\in\lbbrbb{c,d}\subsetneq \lbrbb{\mathfrak{b},\phi'(0^+)}$.
%\end{rmk}
%\end{thm}
Finally, we consider case when $t/x$  converges from below to $\phi'(0^+)$. Then, we need a local condition which is a modification of $\eqref{def:condiB}$, namely
\begin{equation}\label{def:condiB'}
	\begin{split}
		&\limsupo{x}\frac{x\Phi'''(x)}{-\Phi''(x)}=K<\infty.
	\end{split}\tag{$\mathbb{A}'_2$}
\end{equation}
In this case we can prove the following result.
%We have the result, whose additional assumption \eqref{eq:addCondi}, as argued in Section \ref{subsubsec:EX}, is mild.
\begin{thm}\label{thm:main2}
	Let $\Phi$ be the Laplace exponent of some potentially killed subordinator and assume that  conditions $\eqref{def:condiA}$ and $\eqref{def:condiB'}$ hold true. If $t=t(x)$ is such that $t(x)/x\in\lbrb{\mathfrak{b},\Phi'(0^+)}$ and  $\limi{x}t(x)/x=\Phi'(0^+)$. Assume further that $a_*:=a_*(x)=(\Phi')^{-1}(t(x)/x)$ satisfies
	%from below 
	%in such a way that the solution to $\eqref{eq:a*}$, that is $a_*=a_*(x)$ satisfies
	\begin{equation}\label{eq:addCondi}
		\limi{x} -x\phi''(a_*)a^2_*=\infty, \quad \limsupi{x}\frac{-\ln\lbrb{a_*}}{x}<\infty, \quad \mbox{ and } \quad \forall \delta>0 \limi{x}e^{-\delta x}x\phi''(a_*)a^2_*=0.
	\end{equation}
	Then, for any $k\geq0,l\geq0$, we have, as $x \to \infty$, 
	%asymptotically
	\begin{equation}\label{asymp2}
		\begin{split}
			& \frac{\partial^k\partial ^l }{\partial x^k\partial t^l}\fP{x,t}=\frac{(-1)^k}{\sqrt{2\pi}}e^{a_*t-x\Phi(a_*)}\frac{\Phi^\dagger(a_*)\Phi^{k}(a_*)}{a^{1-l}_*\sqrt{-\Phi''(a_*)x}}\lbrb{1+\bo{\frac{\sqrt{\ln\lbrb{a_*\sqrt{-\Phi''(a_*)x}}} }{a_*\sqrt{-\Phi''(a_*)x}}}}.
			%, \ \mbox{ as }x \to \infty.
		\end{split}
	\end{equation} 
\end{thm}
%\begin{thm}\label{thm:main2}
%	Let $\Phi$ be the Laplace exponent of some potentially killed subordinator. Assume that  conditions $\eqref{def:condiA}$ and $\eqref{def:condiB'}$ hold true. If $L=\infty$ in \eqref{def:condiA} then $f_\phi(x,t)$ defined in \eqref{def:f} above is infinitely differentiable in both variables on $\mathbb{D}$ and, if $L<\infty$, then, for any fixed $k,l\geq 0$, $\frac{\partial^k}{\partial x^k} \frac{\partial^l}{\partial t^l}f(t,x)$  exists for points in $\mathbb{D}$, for which $x$  is large enough. If $t=t(x)$ is such that $t(x)/x\in\lbrb{\mathfrak{b},\Phi'(0^+)}$ and  $\limi{x}t(x)/x=\Phi'(0^+)$ from below in such a way that the solution to $\eqref{eq:a*}$, that is $a_*=a_*(x)$ satisfies
%	\begin{equation}\label{eq:addCondi}
%		\limi{x} -x\phi''(a_*)a^2_*=\infty,\,\limsupi{x}\frac{\ln\lbrb{\frac{1}{a_*}}}{x}<\infty,\,\forall \delta>0 \limi{x}e^{-\delta x}x\phi''(a_*)a^2_*=0,
%	\end{equation}
%	then, for any $k\geq0,l\geq0$, we have asymptotically
%	\begin{equation}\label{asymp2}
%		\begin{split}
%			& \frac{\partial^k\partial ^l }{\partial x^k\partial t^l}\fP{x,t}\stackrel{\infty}{=}\frac{(-1)^k}{\sqrt{2\pi}}e^{a_*t-x\Phi(a_*)}\frac{\Phi^\dagger(a_*)\Phi^{k}(a_*)}{a^{1-l}_*\sqrt{-\Phi''(a_*)x}}\lbrb{1+\bo{\frac{\sqrt{\ln\lbrb{a_*\sqrt{-\Phi''(a_*)x}}} }{a_*\sqrt{-\Phi''(a_*)x}}}}.
%		\end{split}
%	\end{equation} 
%\end{thm}
%See Section \ref{sec:proofs} for the proof of Theorem \ref{thm:main2}.
\begin{rmk}\label{rem:main2}
Conditon \eqref{eq:addCondi} is not so restrictive, as will be observed in Section \ref{subsubsec:EX}.
We also mention that if $\phi''(0^+)<\infty$ then  \eqref{def:condiB'} holds true and the last two conditions in \eqref{eq:addCondi} follow from the first, which actually becomes $\limi{x}xa^2_*=\infty$.
%, which in turn implies \eqref{def:condiB'}.
\end{rmk}
Next,  in the spirit of Theorem 1 in \cite{DonRiv}, we use the results above to obtain information for the densities of the subordinators themselves. Recall that $\Pbb{\sigma(x)\in dt}=g_\Phi(x,t)dt$ provided $g_\Phi$ exists (see \eqref{def:gphi}). Furthermore, set $G_\Phi(x,t)=\Pbb{\sigma(x) \le t}$. The proof of the next result uses the relation
\[\IntOI e^{-z t}G_\Phi(x,t)dt=\frac{e^{-x\Phi(z)}}{z}={\frac{1}{\Phi^\dagger\lbrb{z}}}\int_{0}^\infty f_\Phi(x,t)e^{-z t}dt\]
and is therefore identical to the ones of Theorem \ref{thm:mainL}, Theorem \ref{thm:main1} and Theorem \ref{thm:main2} including the existence of $g_\phi$, that is the derivative of $G_\phi$ in $t$, and its derivatives. We only state the following result.
\begin{thm}\label{thm:mainS}
	Let $\Phi$ be the Laplace exponent of some potentially killed subordinator and assume that condition \eqref{def:condiA} holds. Then, for any $n \ge 0$, there exists $x_0(n,L) \ge 0$ such that, for any $k,l\ge 0$ with $k+l \le n$ and $(x,t) \in \mathbb{D}$ with $x>x_0(n,L)$, $\displaystyle \frac{\partial^k}{\partial x^k}\frac{\partial^l}{\partial t^l}G_\Phi(x,t)$ is well-defined. If $L=\infty$ in \eqref{def:condiA}, then $x_0(n,\infty)=0$. In particular, the density $g_\Phi$ is well-defined for $x > x_0(1,L)$ and $\displaystyle \frac{\partial^k}{\partial x^k}\frac{\partial^l}{\partial t^l}g_\Phi(x,t)$ is well-defined for $x>x_0(k+l+1,L)$. Furthermore,the following three statements hold true.
	\begin{itemize}
		\item[$(i)$]  Under the conditions and the notation of Theorem~\ref{thm:mainL}  set $t=t(x)$ with $t(x)/x \in (\mathfrak{b},\Phi'(0+))$ and $t(x)/x \to \mathfrak{b}$, as $x \to \infty$. Then, as $x\to\infty$,
		\begin{equation}\label{asympS1}
			\begin{split}
				\sup_{x\mathfrak{b}<t \le t(x)}\abs{ \Phi^\dagger(c)\frac{\frac{\partial^k\partial ^l }{\partial x^k\partial t^l}G_{\Phi}\lbrb{x,t}}{\frac{\partial^k\partial ^l }{\partial x^k\partial t^l} f_{\Phi}\lbrb{x,t}}-1}&=\bo{\sup_{c\geq a_*(x)}\frac{\sqrt{\ln\lbrb{c\sqrt{-\Phi''(c)x}}} }{c\sqrt{-\Phi''(c)x}}}=\bo{\sqrt{\frac{\ln(x)}{x\ln a_*}}},
			\end{split}
		\end{equation}
		where
		\begin{equation*}
			\mathbb{D}^\prime=\{(t,x): \ x\mathfrak{b} < t \le t(x) < x\phi'(0+)\}, \ c:=c(t,x)=(\Phi')^{-1}(t/x), \ \mbox{ for }(t,x) \in \mathbb{D}^\prime
		\end{equation*}
		and $a_*:=a_*(x)=c(t(x),x)$.
		\item[$(ii)$]  Under the conditions of Theorem~\ref{thm:main1}  for all $[t_1,t_2] \subset (\mathfrak{b},\Phi'(0+))$ it holds, as $x\to\infty$,
		\begin{equation}\label{asympS}
			\begin{split}
				\sup_{xt_1 \le t \le xt_2}\abs{ \Phi^\dagger(c)\frac{\frac{\partial^k\partial ^l }{\partial x^k\partial t^l}G_{\Phi}\lbrb{x,t}}{\frac{\partial^k\partial ^l }{\partial x^k\partial t^l} f_{\Phi}\lbrb{x,t}}-1}
				=\bo{\sqrt{\frac{\ln(x)}{x}}},
			\end{split}
		\end{equation}
		where
		\begin{equation*}
			\mathbb{D}^\prime=\{(t,x): \ xt_1 \le t \le xt_2 \}, \ \mbox{ and } \ c:=c(t,x)=(\Phi')^{-1}(t/x) \ \mbox{ for }(t,x) \in \mathbb{D}^\prime.
		\end{equation*}
		\item[$(iii)$]Under the conditions of Theorem~\ref{thm:main2}  set $t=t(x)$ with $t(x)/x \in (\mathfrak{b},\Phi'(0+))$ and $t(x)/x \to \Phi'(0+)$ as $x \to \infty$. Then, as $x\to\infty$,
		\begin{equation}\label{asympS2}
			\frac{\partial^k\partial ^l }{\partial x^k\partial t^l}G_{\Phi}\lbrb{x,t}=\frac{1}{\phi^\dagger(a_*)}\frac{\partial^k\partial ^l }{\partial x^k\partial t^l}f_{\Phi}\lbrb{x,t}\left(1+\bo{\sqrt{\frac{\ln(x)}{x\ln(a_*(x))}}}\right),
		\end{equation}
		where $a_*:=a_*(x)=(\Phi')^{-1}(t(x)/x)$.
	\end{itemize}
	
	% if $t/x=t(x)/x \downarrow \mathfrak{b}$, then, as $x\to\infty$,
	%\begin{equation}\label{asympS1}
	%	\begin{split}
	%\sup_{(t,x) \in \mathbb{D}^\prime}\abs{ \Phi^\dagger(c)\frac{\frac{\partial^k\partial ^l }{\partial x^k\partial t^l}G_{\Phi}\lbrb{x,t}}{\frac{\partial^k\partial ^l }{\partial x^k\partial t^l} f_{\Phi}\lbrb{x,t}}-1}&=\bo{\sup_{c\geq a_*(x)}\frac{\sqrt{\ln\lbrb{c\sqrt{-\Phi''(c)x}}} }{c\sqrt{-\Phi''(c)x}}}=\bo{\sqrt{\frac{\ln(x)}{x\ln a_*(x)}}}.
	%	\end{split}
	%\end{equation}
	%When $t$ ranges in $[xt_1,xt_2]$ with $[t_1,t_2]\subseteq (\mathfrak{b},\phi'(0^+))$ then under the conditions and notation of Theorem \ref{thm:main1} we have, as $x\to\infty$, 
	%\begin{equation}\label{asympS}
	%	\begin{split}
	%		\sup_{(t,x) \in \mathbb{D}^\prime}\abs{ \Phi^\dagger(c)\frac{\frac{\partial^k\partial ^l }{\partial x^k\partial t^l}G_{\Phi}\lbrb{x,t}}{\frac{\partial^k\partial ^l }{\partial x^k\partial t^l} f_{\Phi}\lbrb{x,t}}-1}
	%		=\bo{\sqrt{\frac{\ln(x)}{x}}}.
	%	\end{split}
	%\end{equation}
%Finally, if $\limi{x}t/x=\Phi'(0^+)$ and  under the assumptions of Theorem \ref{thm:main2} we have  the expression \eqref{asympS1}. 
\end{thm}
In particular, under \eqref{def:condiA}, we have $f^{\rm k}_\phi(x,t)=qG_\phi(x,t)$ and $f^{\rm c}_{\Phi}(x,t)=\mathfrak{b}g_\phi(x,t)$, see \eqref{def:fk} and \eqref{def:fc}. Hence, the conclusions of Theorem \ref{thm:mainS} can be transferred to $f^{\rm c}_{\Phi}$ and $f^{\rm k}_\Phi$.
\begin{coro}\label{cor:mainS}
	Under the conditions of Theorem \ref{thm:mainS}, the asymptotics given in  \eqref{asympS1},\eqref{asympS} and \eqref{asympS2} hold for $q^{-1}f^{\rm k}_{\Phi}$ and $\mathfrak{b}^{-1}f^{\rm c}_{\Phi}$, provided $\mathfrak{b},q>0$.
\end{coro}

\subsubsection{Discussion and comparison to existing results}\label{subsubsec:EX}
First we consider general asymptotic results in the literature for which we need the following lemmae. Recall the definition of $\Delta(x)$ in \eqref{def:D}.
\begin{lem}\label{lem:phi''}
	Let $\phi$ be a Bernstein function. Then, for any $x>0$, it holds that
	\begin{equation}\label{eq:phi''}
		\begin{split}
			& e^{-1}\Delta(x)\leq-\phi''(x)\leq \Delta(x)+\frac{e^{-1}}{x^2}\bar{\mu}_\phi\lbrb{\frac{1}{x}}.
		\end{split}
	\end{equation}
\end{lem}
Introduce the condition
\begin{equation}\label{condi:DR}
	\begin{split}
		&\liminfi{y}\frac{y^2\Delta(y)}{\bar{\nu}_\Phi(y^{-1})}=\liminfo{y}\frac{\Delta(y^{-1})}{y^2\bar{\nu}_\Phi(y)}>0, 
	\end{split}
\end{equation}
which implies $\Phi(\infty)=\infty$ and resembles the well-known positive increase condition, see \cite[eq (6)]{MinSav23+}.
\begin{lem}\label{lem:condi}
	Let $\phi$ be a Bernstein function with $\phi(\infty)=\infty$. If
	\begin{equation}\label{def:condiB''}
		\liminfi{x}\frac{\phi''(2x)}{\phi''(x)}>0, \tag{$\mathbb{A}_2^*$}
	\end{equation} 
then \eqref{def:condiB} holds true. Condition \eqref{condi:DR} implies the validity of \eqref{def:condiB''}, \eqref{def:condiB}, \eqref{def:condiA} with $L=\infty$ and the existence of a $\beta>0$ small enough such that
% for some $\beta>0$ small enough that
\begin{equation}\label{eq:dBound}
\liminfi{x}\frac{\Delta\lbrb{x}}{x^{-2+\beta}}>0,\,\,\, \liminfi{x}\frac{-\phi''(x)}{x^{-2+\beta}}>0  \text{ and } \Delta(x)\asymp -\phi''(x), \mbox{ as $x\to\infty$}.
\end{equation}
\end{lem}
The lemmae above are proved in Section \ref{subsec:aux}. 

Now we are ready to compare the conditions and the results of Theorem \ref{thm:mainS} to the ones in the literature. The conditions of course match those in the theorems concerning densities and their derivatives of inverse subordinators but those seem to have not been studied in such detail prior to our work. All results below relate to subordinators. 

One of the most general results in the literature is \cite[Theorem 3.3]{GrLTr21} which contains the following non-uniform version of \eqref{asympS1}, \eqref{asympS} and \eqref{asympS2} of Theorem \ref{thm:mainS}
\begin{equation*}
g_{\phi}(x,t)=\frac{1}{\sqrt{-2\pi x \phi''(a_*)}}e^{a_*t-x\Phi(a_*)}\lbrb{1+\so{1}},
\end{equation*}
where $a_*=(\phi')^{-1}(t/x)$ and $t,x$ are admissible as in our claims. The main condition of \cite[Theorem 3.3]{GrLTr21} implies via the notion of almost increasing functions, see \cite[p.6]{GrLTr21} and \cite[Lemma 2.8]{GrLTr21} that, for some $\alpha>0$, $-x^{2-\alpha}\phi''(x)$ is almost increasing and all claims of \eqref{def:condiB''} and \eqref{eq:dBound} hold. Hence, our \eqref{def:condiB}, \eqref{def:condiA} with $L=\infty$ are satisfied and thus \eqref{asympS1} and \eqref{asympS} require less restrictive conditions and yield uniform estimates with  speed of convergence. For \eqref{asympS2} we cannot derive \eqref{def:condiB'} from the condition in \cite[Theorem 3.3]{GrLTr21}. Also \cite[Corollaries 3.5 and 3.7]{GrLTr21} offer uniform asymptotic equivalences for $g_\phi$.

Next, we discuss the results in \cite[Theorem 3.2 (iii)]{DonRiv} which correspond to the claims of \eqref{asympS1}, \eqref{asympS} and \eqref{asympS2} of Theorem \ref{thm:mainS}. We note that the asymptotic results in \cite[Theorem 3.2 (iii)]{DonRiv} are uniform but lack speed of convergence. The main condition in the setting of \eqref{asympS1}, i.e. $t(x)/x\downarrow \mathfrak{b}$, is \eqref{condi:DR}, which from Lemma \ref{lem:condi} implies our \eqref{def:condiB}, \eqref{def:condiA} with $L=\infty$. On the other hand, choosing $\nu_{\Phi}(dy)=y^{-1}\ln\lbrb{y^{-1}}\ind{y\in\lbrb{0,1}}dy$, we see that \eqref{condi:DR} is not satisfied whereas  \eqref{def:condiA} and \eqref{def:condiB} hold since at infinity $x\Phi'''(x)\asymp x^{-2}\ln(x)\asymp -\Phi''(x)$ and $x^2\Delta(x)\asymp\ln\lbrb{x}$, as $x \to \infty$. In the regime \eqref{asympS} \cite[Condition H]{DonRiv} is implied by our \eqref{def:condiA} but we offer explicit asymptotic speed of convergence and results for all derivatives. For \eqref{asympS2} the conditions in \cite{DonRiv} may not be matched with ours as they concern \eqref{condi:DR} at zero. Finally, we highlight that \cite{DonRiv} employ as a main tool the Escher transform which although very powerful may not be directly used for the derivatives.

For complete Bernstein functions with $\mathfrak{b}=q=0$ with \LL measure that admits density $m_\phi$ with asymptotic  at zero of the type $m_\phi(y)=c_0y^{-\alpha_0}+c_1y^{-\alpha_1}+\cdots$ and $\limi{y}e^{-\sigma y}m_{\phi}(y)=0, \sigma>0$,  \cite[Theorem 3.6 (ii)]{Fah_10} offers, for fixed $t$ and $x\to\infty$, asymptotic representation as \eqref{asympS1} with an explicit speed of convergence. The assumptions are much more restrictive and only fixed space is considered.

Next we present some results representing the state-of-the-art for two-sided bounds for $g_\phi$ and $f_\phi$. Under the conditions of \cite[Theorem 3.3]{GrLTr21} the authors present explicit upper bounds for $g_\phi$, see \cite[Theorem 4.7]{GrLTr21},
whereas with some further restrictions they obtain clear lower bounds, see \cite[Theorem 4.11]{GrLTr21}. Neat two-sided bounds for $g_\phi, f_\phi$ are deducted in \cite[Theorem 4.4]{CKKW20} by seemingly very simple, clever approach but under more restrictive assumptions. Two-sided bounds for $g_\phi$ are also obtained in \cite[Theorem 1.3]{ChoKim21} and their form is precisely as the asymptotic term for $k=l=0$ in Theorem \ref{thm:mainS}. Again the restrictions are not mild. It is worth noting that other than \cite{GrLTr21} the other papers demand the existence of \LL density and impose some conditions on it. Bounds are also presented in \cite{KK13}.
%\ga{Let us also underline that \eqref{def:condiA} and \eqref{def:condiB} are verified if $\Phi$ is a Bernstein function that is regularly varying at infinity with index $\alpha>0$. Indeed, on the one hand \eqref{def:condiB} follows by a direct application of the monotone density theorem \cite[Theorem 1.7.2]{bingham}. On the other hand, \eqref{def:condiA} follows with $L=\infty$ once we notice that $\mu_\Phi(dy)=m(y)dy$ where $m$ is regularly varying at $0$ with index $-\alpha-1$ (see \cite[Proposition 5.23]{pottheory}). Actually, \eqref{def:condiA} is verified even if $\Phi$ is a Bernstein function such that $\Phi(x)\asymp x^{\alpha}\mathfrak{l}(x)$ as $x \to \infty$ for some slowly varying function $\mathfrak{l}$, as a consequence of \cite[Theorem 13.2.10]{ksv12}.}

\subsection{A power series representation of densities and their derivatives and their behaviour at zero}\label{subsec:series}
Here we discuss our results concerning the power-series representation of the density of the inverse subordinator and its derivatives. These will be obtained by assuming that the Laplace exponent $\Phi$ of the involved potentially killed subordinator satisfies the following assumptions:
	\begin{align}
	 \begin{split} & \mbox{There exists $\theta \in (0,\pi)$ such that $\Phi$ admits a holomorphic extension} \\ & \qquad \qquad \qquad \mbox{on $\C\left(\pi-\frac{\theta}{2}\right)$ which is continuous on $\overline{\C\left(\pi-\frac{\theta}{2}\right)}$} \end{split} \tag{$\mathbb{B}_1$} \label{eq:extensionA3}\\
 	&	\lim_{z \to +\infty}\frac{\Phi(z)}{z}=\mathfrak{b} \qquad 	\mbox {uniformly in } \overline{\C\left(\pi-\frac{\theta}{2}\right)}. \tag{$\mathbb{B}_2$} \label{eq:uniformlimcond} 
% 	\\
%	&	\left|\Phi^\dagger\left(\rho e^{i\left(\pi-\frac{\theta}{2}\right)}\right)\right| \le C\Phi^\dagger(\rho), \tag{$\mathbb{A}_5$} 	\label{eq:modcont}
%	 \\
%		&	\int_1^{+\infty} s^{-1} \bar{\nu}_{\Phi}(s) ds \, < \, +\infty. 	\label{eq:integrlog} \tag{$\mathbb{A}_6$}
	\end{align}
We remark that \eqref{eq:extensionA3} and \eqref{eq:uniformlimcond} are very general, since they are satisfied by a wide class of Bernstein functions (see the discussion in Section \ref{discussionassumptions} below). The proofs of the results of this section are provided in Section \ref{subsec:power}. Upon the validity of \eqref{eq:extensionA3} and \eqref{eq:uniformlimcond} we have the following regularity result.
\begin{thm}\label{thm:smoothfmu}
	Let $\Phi$ be the Laplace exponent of a potentially killed subordinator satisfying assumptions \eqref{eq:extensionA3} and \eqref{eq:uniformlimcond}. Then, for any $n \ge 1$, $\bar{\mu}_\phi^{\ast n}$ belongs to $C^\infty(0,+\infty)$ and $f_\Phi \in C^\infty(\mathbb{D})$.
\end{thm}
%Theorem \ref{thm:smoothfmu} is proved in Section \ref{subsec:power}. 
Furthermore, under the same conditions, the following  power series representation holds.
\begin{thm}\label{thm:seriespi}
Let $\Phi$ be the Laplace exponent of a potentially killed subordinator satisfying assumptions \eqref{eq:extensionA3} and \eqref{eq:uniformlimcond}. Then, for any $k,l \ge 0$,
\begin{equation}\label{eq:seriesder}
	\frac{\partial^k \partial^l}{\partial x^k \partial t^l}f_\Phi(x, t) \, = \, \sum_{j=0}^\infty \frac{x^{j}}{j!} \mathcal{I}_{j,k,l}(t), \qquad (x,t) \in \mathbb{D},
\end{equation}
%Then, on $\mathbb{D}$, we have that
%\begin{equation}\label{eq:series}
%	f_\Phi(x, t) \, = \, \sum_{j=0}^\infty \frac{x^j}{j!} \mathcal{I}_{j,0,0}(t),
%\end{equation}
where 
\begin{equation} \label{coeff}
	\mathcal{I}_{j,k,l}(t):=(-1)^{k+j} \sum_{k_1+k_2+k_3=k+j} \frac{(k+j)!}{k_1!k_2!k_3!}q^{k_1}\mathfrak{b}^{k_2} \frac{d^{l+k_2+k_3}}{dt^{l+k_2+k_3}} \bar{\nu}_\Phi^{\ast(k_3+1)}(t)
\end{equation}
and the series is absolutely convergent.
% The derivatives appearing in the coefficients \eqref{coeff} exist for any $t>0$, and $l,k_2, k_3 \geq 0$. Furthermore, for any $k, l \ge 0$ we have, on $\mathbb{D}$,
%\begin{equation}\label{eq:seriesder}
%	\frac{\partial^k \partial^l}{\partial x^k \partial t^l}f_\Phi(x, t) \, = \, \sum_{j=0}^\infty \frac{x^{j}}{j!} \mathcal{I}_{j,k,l}(t),
%\end{equation}
%where the series is absolutely convergent and $\frac{\partial^k \partial^l}{\partial x^k \partial t^l}f_\Phi(x,t)$ is continuous in $\mathbb{D}$.
\end{thm}
%See Section \ref{subsec:power} for the proof of Theorem \ref{thm:seriespi}.
\begin{rmk}\label{rmk:seriesexp}
	Note that the series \eqref{eq:seriesder} which contains the tail of the L\'evy measure, its convolutions and their derivatives is similar to the series expansion for the potential density of a subordinator with drift, see \cite{DoeSav_10}.
\end{rmk}
\begin{rmk}
	Observe that $f_\Phi(x,t)=0$ for any $x<0$ and $t>0$, hence the previous theorem shows that, for fixed $t>0$, $f_\Phi(x,t)$ coincides, for $0<x<\mathfrak{b}/t$, with an entire function, i.e., the right hand side of \eqref{eq:seriesder} for $k=l=0$.
\end{rmk}
The series representation \eqref{eq:seriesder} yields information about the behaviour at $0$ of $f_\Phi$ and its derivatives.
\begin{thm}
	\label{behavatzero}
	Under the assumptions of Theorem \ref{thm:seriespi}, for any $k,l \ge 0$ and any $[t_1,t_2] \subset (0,+\infty)$ we have
	\begin{equation}\label{eq:zero1}
		\sup_{t \in [t_1,t_2]}\left|\frac{\partial^k \partial^l}{\partial x^k \partial t^l}f_\phi(x,t)-\mathcal{P}_{n,k,l}(x,t)\right| \le \frac{x^{n+1}}{(n+1)!}\mathcal{R}_n(x;t_1,t_2),
	\end{equation}
	where
	\begin{equation}\label{eq:zero2}
		\mathcal{P}_{n,k,l}(x,t)=\sum_{j=0}^{n}\frac{x^j}{j!}\mathcal{I}_{j,k,l}(t),
	\end{equation}
	and $\sup_{x \in [0,x_1]}\mathcal{R}_n(x;t_1,t_2)<\infty$  for all $x_1 \in \left(0,\frac{t_1}{\mathfrak{b}}\right)$.
	%\begin{equation*}
	%	\lim_{x \to 0}\frac{\partial^k \partial^l}{\partial x^k \partial t^l} f_\phi(x,t)=\mathcal{I}_{0,k,l}(t)
	%\end{equation*}
	%and
	%\begin{equation*}
	%	\lim_{x \to 0}x^{-1}\left|\frac{\partial^k \partial^l}{\partial x^k \partial t^l} f_\phi(x,t) - \mathcal{I}_{0,k,l}(t)  \right| = \mathcal{I}_{1,k,l}(t),
	%\end{equation*}
	%where both limits are uniform with respect to $t \in [a,b]$ for any $[a,b]\subset (0,+\infty)$.
\end{thm}
%\begin{thm}
%	\label{behavatzero}
%	Under the assumptions of Theorem \ref{thm:seriespi}, for any $k,l \ge 0$ we have
%	\begin{equation*}
%		\lim_{x \to 0}\frac{\partial^k \partial^l}{\partial x^k \partial t^l} f_\phi(x,t)=\mathcal{I}_{0,k,l}(t)
%	\end{equation*}
%	and
%	\begin{equation*}
%		\lim_{x \to 0}x^{-1}\left|\frac{\partial^k \partial^l}{\partial x^k \partial t^l} f_\phi(x,t) - \mathcal{I}_{0,k,l}(t)  \right| = \mathcal{I}_{1,k,l}(t),
%	\end{equation*}
%	where both limits are uniform with respect to $t \in [a,b]$ for any $[a,b]\subset (0,+\infty)$.
%\end{thm}
%Theorem \ref{behavatzero} is proved in Section \ref{subsec:power}.
Though conditions \eqref{eq:extensionA3} and \eqref{eq:uniformlimcond} depend on suitable $\theta \in (0,\pi)$, the series representation \eqref{eq:seriesder} is independent of $\theta$. Indeed, the result follows once one recognize some special integral representations for both the function $f_\Phi$ and the convolution powers $\bar{\mu}_\Phi^{\ast n}$. Note that the series provides an explicit representation of $f_\Phi$ whenever the convolution powers $\bar{\mu}_\Phi^{\ast n}$ can be evaluated. On the other hand, there could be some cases in which the integral formulation of $\bar{\mu}_\Phi^{\ast n}$ can be used to provide such an evaluation. This is the case, for instance, when we can extend $\Phi$ on the whole complex half-plane $\overline{\C(0,\pi)}=\{z \in \C: \ \Im(z) \ge 0\}$. Let us underline that such a property is not necessarily verified by all complete Bernstein functions, as for instance $\phi(z)=\log(1+z)$ cannot satisfy this. 
\begin{prop}\label{prop:extcont}
		Let $\Phi$ be a complete Bernstein function and assume that $\Phi$ can be extended with continuity on $\overline{\C(0,\pi)}$. Denote by $\Phi_{+}$ such extension. Assume further that 
		\begin{equation*}
			\lim_{z \to +\infty}\frac{\Phi^\dagger_+(z)}{z}=0 \mbox{ uniformly in }\overline{\C(0,\pi)}.
		\end{equation*}
		Then, for any $\varepsilon,t>0$, $r\geq 0$ and $n \geq 1$, denoting by $\gamma_\varepsilon$ the parametrized curve $\gamma_\varepsilon: z=\varepsilon e^{i\xi}$ for $\xi \in [-\pi,\pi]$, we have that
		\begin{equation}\label{eq:integralmucont}
			\frac{d^r}{dt^r}\overline{\mu}_\Phi^{n \star}(t)=\frac{1}{\pi}\int_{\varepsilon}^{+\infty}\Im\left[(-1)^{r+n+1}\frac{(\Phi^\dagger_+(-\rho))^n}{\rho^{n-r}}e^{-t\rho}\right]d\rho+\frac{1}{2\pi i}\int_{\gamma_\varepsilon} e^{tz}\frac{(\Phi^\dagger(z))^n}{z^{n-r}}dz.
		\end{equation}
\end{prop}
%See Section \ref{subsec:power} for the proof of Proposition \ref{prop:extcont}.

Similarly to what we did for the asymptotic behaviour at infinity, we can use the relation
\begin{equation}
	\int_0^{+\infty} e^{- z t} G_\phi(x,t) dt \, = \, \frac{e^{-x\Phi(z)}}{z}
\end{equation}
to obtain, with the same method, results on $G_\phi(x,t)$ (see \eqref{def:gphi}). This is stated in the following theorem.
% whose proof is in Section \ref{subsec:power}.
%The following theorem, therefore, is given with no proof.
\begin{thm}
\label{thm:seriessub}
Under \eqref{eq:extensionA3} and \eqref{eq:uniformlimcond}, $G_\phi \in C^\infty(\mathbb{D})$ and for any $k \geq 1$, $l \ge 0$ and $(x,t) \in \mathbb{D}$
\begin{align}
	G_\phi(x,t) \, = \, e^{-qx}+\sum_{j=1}^{+\infty} \frac{x^j}{j!} \mathfrak{I}_{j,0,0}(t),
	\label{eq:seriesdistr}
\end{align}
\begin{align}
	\frac{\partial^l}{\partial t^l} g_\phi(x,t) \, = \, \sum_{j=1}^{+\infty} \frac{x^j}{j!} \mathfrak{I}_{j,0,l+1}(t),
	\label{eq:seriessub1}
\end{align}
and
\begin{align}
	\frac{\partial^l}{\partial t^l} \frac{\partial^k}{\partial x^k} g_\phi(x,t) \, = \, \sum_{j=0}^{+\infty} \frac{x^j}{j!} \mathfrak{I}_{j,k,l+1}(t),
	\label{eq:seriessub2}
\end{align}
where
\begin{equation}\label{eq:fractureI}
	\mathfrak{I}_{j,k,l}(t)=(-1)^{k+j+1}\sum_{k_1+k_2+k_3=k+j-1}\frac{(k+j)!}{k_1!k_2!(k_3+1)!}q^{k_1}\mathfrak{b}^{k_2}\dersup{}{t}{k_2+k_3+l}\bar{\mu}_\phi^{\ast(k_3+1)}(t).
\end{equation}
In particular, all the series are absolutely convergent.
\end{thm}
As in the previous section, under \eqref{eq:extensionA3} and \eqref{eq:uniformlimcond} we know that $f^{\rm k}_\Phi(x,t)=qG_\Phi(x,t)$ and  $f^{\rm c}_{\Phi}(x,t)=\mathfrak{b}g_\phi(x,t)$, see \eqref{def:fk} and\eqref{def:fc}. Hence, the results of Theorem \ref{thm:seriessub} can be transferred to $f^{\rm k}_\Phi$ and $f^{\rm c}_{\Phi}$.
\begin{coro}
\label{cor:seriescreep}
	Under the conditions of Theorem \ref{thm:seriessub}, \eqref{eq:seriesdistr} holds for $f^{\rm k}_\Phi$ up to  a multiplicative factor $q$. Furthermore, \eqref{eq:seriessub1} and \eqref{eq:seriessub2} hold for $f^{\rm c}_\Phi$ up to a multiplicative factor $\mathfrak{b}$.
\end{coro}

%
%****** REDO WITH LAPLACE INVERSION ***
%
%As in the case of asymptotic behaviour of subordinators, we note that we can use the result on $f_\Phi(x,t)$ to obtain an equivalent result for the density of the subordinator itself, provided it exists. We use the relation
%\begin{align}
% \P \l \sigma (t) > x \r \, = \, & \P \l L(x) \leq t  \r \notag \\
%= \, & \P \l L(x) \leq t, \sigma (L(x))>x \r + \P \l L(x) \leq t, \sigma(L(x)) = x \r \notag \\
%= \, &  \int_0^t f_\phi(s, x) \, ds \, + \int_0^t  f_\phi^c \l s,x \r ds \notag \\
%= \, & \int_0^t f_\phi(s,x) ds + \mathfrak{b}\int_0^t g_\phi (s,x) ds
%\label{321}
%\end{align}
%where $f_\phi^c(x,t)$ is defined in \eqref{def:fc} and that $f_\phi^c(x,t) dt= \mathfrak{b}g_\phi(x,t)dt$ ( see \eqref{def:fc} and \cite{DonRiv}).
%The following result is a direct consequence of \eqref{321}, Theorem \ref{thm:seriespi} and an easy computation. Hence, we state the result only, with no proof.
%\begin{thm}
%Suppose that $\mathfrak{b}=0$, then under \eqref{eq:extensionA3} and \eqref{eq:uniformlimcond} we have that $g_\phi(t,x)dx$ exists and, for any $k \geq 0,l \geq 1$, it is true that,
%\begin{align}
%\frac{\partial^l}{\partial t^l} \frac{\partial^{k}}{\partial x^k} g_\phi(t,x) \, = \, - \sum_{j=0}^{+\infty} \frac{t^j}{j!} \mathcal{I}_{j,l-1,k+1} (x)
%\end{align}
%for $x>0$, where the coefficients $\mathcal{I}_{j,k,l}(t)$ are defined in \eqref{coeff}. If $l=0$ we have that for any $k \geq 0$
%\begin{align}
%\frac{\partial^k}{\partial x^k} g_\phi(t,x) \, = \, -\sum_{j=1}^{+\infty} \frac{t^j}{j!} \mathcal{I}_{j-1,0,k+1}(x)
%\label{seriessubordin}
%\end{align}
%for $x>0$.
%\end{thm}

\subsubsection{Discussion and comparison to existing results}
\label{discussionassumptions}
Our Theorem \ref{thm:seriessub} is strongly related with some known results on the small-time polynomial expansion of the distribution of L\'evy processes. More precisely, in \cite[Theorem 5.1]{FH09}, the authors proved that for any (non-killed) L\'evy process $Y(t)$ and all $n \ge 0$ the following polynomial expansion holds:
\begin{equation}\label{eq:tailLevyproc}
	\P(Y(x) > t)=\sum_{j=1}^{n}d_j(t)\frac{x^j}{j!}+\frac{x^{n+1}}{(n+1)!}\mathcal{R}_n(x,t), \, 0<t<t_0,
\end{equation}
where $d_j(t)$, $j=1,\dots,n$ are some $t$-dependent coefficients and $\mathcal{R}_n(x,t)$ is bounded for $0<x<x_0$. Their result is true provided that:
\begin{itemize}
	\item[$(i)$] The L\'evy measure $\nu_Y$ of $Y$ admits a density (that we still denote here by $\nu_Y$);
	\item[$(ii)$] for any $\delta>0$ and any $k=0,\dots,2n+1$ it holds $\sup_{|t|>\delta}\left|\frac{d^k }{dt^k}\nu_Y(t)\right|<\infty$;
	\item[$(iii)$] For a fixed $\delta>0$ and for all $k=0,\dots,2n+1$ it holds $\sup_{0 < x < x_0}\sup_{|t|>\delta}\left|\frac{\partial^k }{\partial t^k}p_x(t)\right|<\infty$, where $p_x(t)dt=\P(Y(x) \in dt)$.
\end{itemize}
The theorem also provides an explicit formulation of the remainder term $\mathcal{R}_n(x,t)$. Furthermore, in \cite[Section 6]{FH09}, the authors discuss some sufficient conditions for $(iii)$ to be satisfied. In particular, the statement of \cite[Theorem 5.1]{FH09} holds for stable and tempered stable processes, as observed in \cite[Remark $6.4$, Example $6.5$ and Proposition $6.7$]{FH09}. The results in \cite{FH09}, when applied to subordinators, are less powerfull than ours in the following sense. On the one hand we do not have any restriction on $x$ in Theorem \ref{thm:seriessub}, once one observes that if $(x,t) \notin \mathbb{D}$ then $G_\phi(x,t)$ is constant. On the other hand conditions (ii) and (iii) are tipically hard to be verified, as pointed out also in \cite[Section 6]{FH09}. 
Clearly, Theorem \ref{thm:seriessub} also provides a polynomial approximation given by, for $q=0$,
\begin{equation*}
	1-G_\phi(x,t)=\sum_{j=1}^{n}\frac{x^j}{j!}(-\mathfrak{I}_{j,0,0}(t))+\frac{x^{n+1}}{(n+1)!}\cR_{n}(x,t),
\end{equation*}
where
\begin{equation*}
	\cR_n(x,t)=\sum_{j=n+1}^{+\infty}\frac{(n+1)!}{j!}x^{j-n-1}(-\mathfrak{I}_{j,0,0}(t)).
\end{equation*}
This gives an alternative representation for the remainder $\mathcal{R}_n(x,t)$ in \eqref{eq:tailLevyproc}, provided we are under the assumptions of \cite[Theorem 5.1]{FH09}, together with \eqref{eq:extensionA3} and \eqref{eq:uniformlimcond}.
While these polynomial approximations have been generalized to several other processes (see, for instance, \cite{FH12,FO16}), we are not aware about results similar to Theorems \ref{thm:seriespi} and \ref{behavatzero} for inverse subordinators except that in specific cases. It is worth noticing that the latter provides a (locally uniform) polynomial approximation for \textit{small space} of the density of an inverse subordinator if $q=\mathfrak{b}=0$ and can be combined with Corollary \ref{cor:seriescreep} to find polynomial approximations for \textit{small space} in the general case. Due to the \textit{exchange} of the roles of time and space when passing from a subordinator to its inverse, these results are in line with the ones proved in \cite{FH09}.
Let us underline, in particular, that assumptions \eqref{eq:extensionA3} and \eqref{eq:uniformlimcond} cover a wide class of Bernstein functions, and then of subordinators. Indeed, if $\phi$ is a complete Bernstein function, then Item \eqref{it:analy1} of Lemma \ref{lem:CBern} guarantees that assumption \eqref{eq:extensionA3} is satisfied, while \eqref{eq:uniformlimcond} follows from Item \eqref{it:anglim} of the same lemma. However, we can find some Bernstein functions that are not complete but still satisfy \eqref{eq:extensionA3} and \eqref{eq:uniformlimcond} for some $\theta$. Indeed, if we consider a Bernstein function $\phi$ that is not complete, we know by Proposition \ref{prop:powchar} that there exists $\alpha \in (0,1)$ such that $\phi_\alpha$ is not a complete Bernstein function (but it is still a Bernstein function by Item \eqref{it:compos} of Lemma \ref{lem:Bern}). Now, if we consider $\theta \in (0,\pi)$ so that $\left(\pi-\frac{\theta}{2}\right)\alpha<\frac{\pi}{2}$, that exists since $\alpha<1$, then $z \in \C\left(\pi-\frac{\theta}{2}\right) \mapsto z^\alpha \in \C\left(\left(\pi-\frac{\theta}{2}\right)\alpha\right) \subset \mathbb{H}_0$ is holomorphic. Furthermore, $\phi$ is holomorphic on $\mathbb{H}_0$ and thus the composition $\phi_\alpha$ is holomorphic on $\C\left(\pi-\frac{\theta}{2}\right)$. The continuity of $\phi_\alpha$ over $\overline{\C\left(\pi-\frac{\theta}{2}\right)}$ follows similarly, thus obtaining \eqref{eq:extensionA3}. The uniform limit condition \eqref{eq:uniformlimcond} follows from Item \eqref{it:asymp} of Lemma \ref{lem:Bern} and the fact that $z^\alpha \in \overline{\mathbb{H}_0}$ whenever $z \in \overline{\C\left(\pi-\frac{\theta}{2}\right)}$, since
	\begin{equation*}
		\frac{\phi_\alpha(z)}{z}=\frac{\phi(z^\alpha)}{z^\alpha}z^{\alpha-1} \to 0.
	\end{equation*}
Thus, if we consider, e.g., the Bernstein function $\phi$ whose L\'evy measure is given by \eqref{eq:examplenocomp}, then there exists a $\beta \in (0,1)$ such that $\phi_\beta$ satisfies \eqref{eq:extensionA3} and \eqref{eq:uniformlimcond} and it is not a complete Bernstein function.
Let us stress that there are functions satisfying \eqref{def:condiA} and \eqref{def:condiB} but not \eqref{eq:extensionA3} and \eqref{eq:uniformlimcond} and vice versa. For instance, we have already shown that if $\mu_\Phi(dy)=y^{-1}\log(y^{-1})\mathbb{I}_{\{y \in (0,1)\}}$, then $\phi$ satisfies both \eqref{def:condiA} and \eqref{def:condiB}. Furthermore, we have $\Phi(z)=\frac{z}{2}\int_0^1 e^{-zy}\log^2(y)dy$, which can be clearly extended to the whole complex plane, thus it verifies \eqref{eq:extensionA3}. However, let us consider any $\lambda>0$ and the sequence $z_n=-4\lambda\pi n+i 4\pi n$, so that, for $n$ big enough, since $\cos(4\pi n y)>\frac{1}{2}$ for any $y \in \left(\frac{1}{2}-\frac{1}{12n}, \frac{1}{2}+\frac{1}{12n}\right)$,
\begin{equation*}
	\Im\left(\frac{\Phi(z_n)}{z_n}\right)\ge\frac{1}{2}\int_{\frac{1}{2}-\frac{1}{12n}}^{\frac{1}{2}+\frac{1}{12n}} e^{4\lambda \pi n y}\cos(4\pi n y)\log^2(y)dy \ge \frac{e^{\lambda \pi n}}{24n}\log^2\left(\frac{3}{4}\right) \to \infty.
\end{equation*}
Since $\lambda>0$ is arbitrary, this implies that \eqref{eq:uniformlimcond} cannot be verified for any $\theta \in (0,\pi)$. On the the hand, if we consider $\phi(z) = 1-e^{-z}$, then we know that $\phi_\alpha(z):=\phi\lb z^\alpha\rb$ satisfies \eqref{eq:extensionA3} and \eqref{eq:uniformlimcond} for some $\theta \in (0,\pi)$. However, it does not satisfies \eqref{def:condiA} since it is a bounded Bernstein function. Indeed, in Proposition \ref{prop:D0}, we will show that \eqref{def:condiA} implies that $\Phi$ is unbounded. 
%\textcolor{red}{Furthermore, it is clear that the condition given in \cite[Equation (6.11)]{FH09} implies \eqref{def:condiA}, hence the previous example does not satisfy the sufficient conditions given in \cite[Section 6]{FH09}.}
%	
%	
%	 Indeed, for any Bernstein function $\phi$ and any $\alpha \in (0,1)$, let us denote $\phi_\alpha(z):=\phi(z^\alpha)$, which is still a Bernstein function by Item \ref{it:compos} of Lemma \ref{lem:Bern}. Now, if we assume that there exists a sequence $(\alpha_n)_{n \ge 0}$ in $(0,1)$ with $\alpha_n \to 1$ and such that $\phi_{\alpha_n}$ is a complete Bernstein function for any $n \ge 0$, then $\phi=\lim_{n \to +\infty}\phi_{\alpha_n}$ is also a complete Bernstein function. This guarantees that if $\phi$ is a Bernstein function that is not complete, then there exists $\alpha \in (0,1)$ such that $\phi_\alpha$ is not a complete Bernstein function.
%
%
%We discuss here the generality of assumptions \eqref{eq:extensionA3} and \eqref{eq:uniformlimcond}. We provide here two classes of Bernstein functions satisfying the assumptions.
%One is the class of complete Bernstein functions, i.e., the set of Bernstein functions whose L\'evy measure has a density (with respect to the Lebesgue measure) which is completely monotone (see, e.g., \cite[Chapter 6]{librobern}). 
%\begin{prop}\label{prop:compl}
%	Let $\Phi$ be the Laplace exponent of a potentially killed subordinator and assume that it is a complete Bernstein function. Then, for any $\theta \in (0,\pi)$, $\Phi$ satisfies Assumptions \eqref{eq:extensionA3} and \eqref{eq:uniformlimcond}.
%\end{prop}
%\begin{proof}
%	First recall that any complete Bernstein function can be extended holomorphically on the cut complex plane $\C\setminus (-\infty,0]$ with $\lim_{\substack{z \to 0 \\ z \in \C\setminus (-\infty,0]}}\Phi(z)=q$, as in \cite[Theorem $6.2$]{librobern}, which gives \eqref{eq:extensionA3} for any $\theta \in (0,\pi)$. Furthermore,  \eqref{eq:uniformlimcond} is a direct consequence of \cite[Corollary $6.5$]{librobern}. 
%%	Finally, one can prove that for any $\xi \in \left[\frac{\theta}{2},\pi-\frac{\theta}{2}\right]$ and any $\rho,t>0$ it holds (see the proof of \cite[Corollary $6.5$]{librobern})
%%	\begin{equation*}
%%		\left|\frac{1}{\rho e^{i\xi}+t}\right|\le \frac{5}{1-\cos\left(\frac{\theta}{2}\right)}\frac{1}{\rho+t}.
%%	\end{equation*}
%%	By using the Stieltjes measure representation of $\Phi$ we get
%%	\begin{multline}
%%		\left|\Phi^\dagger\left(\rho e^{i\left(\pi-\frac{\theta}{2}\right)}\right)\right|\le \int_0^{+\infty}\left|\frac{t}{\rho e^{i\xi}+t}\right|\mathfrak{s}_\Phi(dt) \\
%%		\le \frac{5}{1-\cos\left(\frac{\theta}{2}\right)}\int_0^{+\infty}\frac{t}{\rho+t}\mathfrak{s}_\Phi(dt)=\frac{5}{1-\cos\left(\frac{\theta}{2}\right)}\Phi^\dagger(\rho),
%%		\label{estphidagcomplete}
%%	\end{multline}
%%	that proves Assumption \eqref{eq:modcont}.
%\end{proof}
%A class of not necessarily complete Bernstein functions that  satisfy \eqref{eq:extensionA3} and \eqref{eq:uniformlimcond} is given by
%\begin{equation*}
%\mathbb{B}:= \ll \Phi :  \phi(\lambda)= \varphi(\lambda^\alpha) \text{ for some  Bernstein function } \varphi \text{ and } \alpha \in (0,1) \rr.
%\end{equation*}
%We show now that elements of $\mathbb{B}$ satisfy \eqref{eq:extensionA3} and \eqref{eq:uniformlimcond}.
%\begin{prop}
%\label{balfa}
%Let $\alpha \in (0,1)$. Denote $\Phi_\alpha (\lambda) := \varphi(\lambda^\alpha)$ where $\varphi$ is an arbitrary Bernstein function with drift coefficient $\mathfrak{b}_\varphi \geq 0$. Then $\Phi_\alpha$ satisfies \eqref{eq:extensionA3} and \eqref{eq:uniformlimcond}.
%\end{prop}
%\begin{proof}
%Fix $\vartheta \in \l \frac{\pi}{2}, \pi \r$ so that $\vartheta \alpha < \pi/2$. Take $z \in \mathbb{C}\l \vartheta \r$, i.e., $z=\rho e^{i\xi}$, for some $\rho>0$ and $\xi \in \l -\vartheta, \vartheta \r$. Then $z^\alpha = \rho^\alpha e^{i\alpha \xi}$, i.e., $z^\alpha \in \mathbb{C}(\alpha \vartheta)\subset \mathbb{C} \l \pi/2 \r$. Since $\varphi$ is a Bernstein function it has a holomorphic extension to $\mathbb{C} (\pi/2)$ which is continuous on $\overline{\mathbb{C}(\pi/2)}$. Hence $\phi_\alpha(z)$ has a holomorphic extension to $\mathbb{C}\l \vartheta \r$ since it is composition of holomorphic functions which is continuous on $\overline{\mathbb{C}\l \vartheta \r}$. This shows \eqref{eq:extensionA3} for $\theta= 2 \l \pi-\vartheta \r$. To check \eqref{eq:uniformlimcond} note that in this is case we have $\mathfrak{b}=0$. Then we have that
%\begin{align}
% \frac{\phi_\alpha \l \rho e^{i \l \pi-\frac{\theta}{2} \r}\r}{\rho} \, = \,&  \frac{\phi_\alpha \l \rho e^{i \l \pi-\frac{\theta}{2} \r}\r}{\rho^\alpha} \rho^{\alpha -1}  \notag \\
%= \, &  \frac{\varphi \l \rho^\alpha e^{i \alpha \l \pi-\frac{\theta}{2} \r}\r}{\rho^\alpha} \rho^{\alpha -1} \notag \\
%\to \, & 0 \text{  uniformly in } \overline{\mathbb{C} \l \pi-\frac{\theta}{2} \r}
%\end{align}
%because $\rho^{\alpha -1} \to 0$ and
%\begin{align}
%\frac{\varphi \l \rho^\alpha e^{i \alpha \l \pi-\frac{\theta}{2} \r}\r}{\rho^\alpha} \to \mathfrak{b}_\varphi \text{ uniformly  in  } \overline{\mathbb{C} \l \pi-\frac{\theta}{2} \r},
%\end{align}
%where we used that
%\begin{align}
%\frac{\varphi\l \rho e^{ i \l \pi-\frac{\theta}{2} \r} \r}{\rho} \to \mathfrak{b}_\varphi \text{ uniformly in } \overline{\mathbb{C}\l \frac{\pi}{2} \r}.
%\end{align}
%\end{proof}
%The forthcoming result together with the next example make clear that the elements of $\mathbb{B}$ are not necessarily complete Bernstein function.
%\begin{prop}
%\label{iifb}
%Let $\l \phi_\alpha\r_{\alpha \in (0,1)} \subset \mathbb{B}$ be given by $\phi_\alpha (\lambda) = \varphi(\lambda^\alpha)$ for a Bernstein function $\varphi$. Then $\phi_\alpha$ are complete Bernstein functions for any $\alpha \in (0,1)$ if, and only if, $\varphi$ is a complete Bernstein function.
%\end{prop}
%\begin{proof}
%Note that $z^\alpha$, $\alpha \in (0,1)$, are complete Bernstein functions \cite[page 95]{librobern}. Hence, if $\varphi$ is a complete Bernstein function then $\phi_\alpha$ are also complete by \cite[Corollary 7.9]{librobern} for any $\alpha \in (0,1)$. Now we prove the converse implication. Note that, for any $\lambda >0$, $\varphi(\lambda) = \lim_{n}\varphi_n(\lambda)$ where $\varphi_n(\lambda) = \phi_{1-1/(n+1)}(\lambda)$. Since $\varphi_n(\lambda)$ are complete Bernstein functions for any $n\in \mathbb{N}$ it follows that $\varphi(\lambda)$ is a complete Bernstein function by \cite[Corollary 7.6]{librobern}.
%%We use that for a non-negative function $\varphi$ being a complete Bernstein function is equivalent to have an analytic continuation to $\ll z \in \mathbb{C}: \Im z > 0 \rr$ such that $\Im \varphi (z) \geq 0$ and $\lim_{(0, +\infty) \ni \lambda \to 0}\varphi(\lambda)$ exists and is real (this is a consequence of \cite[Theorem 6.2]{librobern}). Since $\varphi$ is a Bernstein function we can apply \cite[Proposition 3.6]{librobern} and thus the last condition is obvious. Now if we define, for $z \in \mathbb{C}$ with $\Im z>0$, the function $\varphi(z) := \phi_\alpha \l z^{1/\alpha} \r$, we have that $\varphi(z)$ is the analytic continuation on $\ll z \in \mathbb{C}: \Im z>0 \rr$ of the Bernstein function $\varphi$ since $\phi_\alpha$ are complete Bernstein functions for any $\alpha \in (0,1)$.
%%\mladen{I am unsure it is well phrased here!?}. Now we compute the sign of the imaginary part. Take $z=\rho e^{i\xi}$ with $\Im z =\rho \sin \xi >0$. Then $z=\zeta^\alpha$ where $\Im \zeta = \rho^{1/\alpha} \sin (\xi/\alpha)$ and if we choose $\alpha$ sufficiently close to $1$ we have that $\Im \zeta>0$. Hence, for such $\alpha$, $\Im \varphi(z)=\Im \phi_\alpha(\zeta)\geq 0$ since $\phi_\alpha$ is a complete Bernstein function (for any $\alpha \in (0,1)$).
%\end{proof}
%\begin{ex}
%Let 
%\begin{equation*}
%\varphi^\circ(\lambda)= \int_0^{+\infty}\l 1-e^{-\lambda t} \r \frac{1}{t}\mathds{1}_{(0,1)}(t) \, dt.
%\end{equation*}
%Evidently, $\varphi^\circ$ is the Laplace exponent of an unkilled, pure jump subordinator, i.e., $q=\mathfrak{b}=0$ and the L\'evy measure is infinite. It is also clear that $\varphi^\circ$ is not a complete Bernstein function as the L\'evy density is not completely monotone. It follows by Proposition \ref{iifb} that there exists $\alpha_\circ \in (0,1)$ such that the Bernstein function $\phi_{\alpha_\circ}(\lambda):=\varphi^\circ (\lambda^{\alpha_\circ})$ is not a complete Bernstein function. However we have that $\phi_{\alpha_\circ}(\lambda) \in \mathbb{B}$ by definition and thus, by Proposition \ref{balfa}, it satisfies \eqref{eq:extensionA3} and \eqref{eq:uniformlimcond}.
%\end{ex}
%

%Concerning Assumption \eqref{eq:integrlog}, it can be useful to provide some equivalent formulations.
%\begin{prop}\label{prop:integr0}
%	Let $\Phi$ be the Laplace exponent of a potentially killed subordinator. Then the following are equivalent
%	\begin{enumerate}
%		\item $\int_0^1 \Phi^\dagger (\rho)/\rho \, d\rho < +\infty$
%		\item $\int_1^{+\infty} s^{-1}\bar{\nu}_\Phi(s) \, ds \, < \, + \infty$
%		\item $\lim_{t \to +\infty}\log (t) \bar{\nu}_\Phi(t)= \mathpzc{l} \in [0, +\infty)$ and $\int_1^{+\infty} \log (s) \nu_\Phi(ds) \, < \, +\infty$. \mladen{if $\mathpzc{l}=\frac12$ then the integral above cannot be convergent so I think it should be $\mathpzc{l}>1$ which I guess is implied by the second condition}
%	\end{enumerate}
%\end{prop}
%\begin{proof}
%	We first prove that $(1)$ is equivalent to $(2)$. We have that
%	\begin{align}
%		\int_0^1 \frac{\Phi^\dagger(\rho)}{\rho} d\rho \, = \,& \int_0^1 \int_0^{+\infty} e^{-\rho s}  \bar{\nu}_\Phi(s) \, ds \, d\rho \,
%		= \,  \int_0^\infty \l 1-e^{-s} \r \, s^{-1} \, \bar{\nu}_\Phi(s) ds.
%	\end{align}
%	Since $\bar{\nu}_{\Phi}(s)$ is integrable near $0$ then it is clear that 
%	\begin{align}
%		\int_0^1 \l 1-e^{-s} \r \, s^{-1} \, \bar{\nu}_\Phi(s) ds \, < \, +\infty.
%	\end{align}
%	On the other hand
%	\begin{align}
%		\l 1-\frac{1}{e}  \r \int_1^{+\infty} s^{-1} \bar{\nu}_\Phi(s) \, ds \, \leq \, \int_1^\infty \l 1-e^{-s} \r \, s^{-1} \, \bar{\nu}_\Phi(s) ds \,
%		\leq \, \int_1^{+\infty} s^{-1} \bar{\nu}_\Phi (s) ds.
%	\end{align}
%	Now we show that $(3)$ implies $(2)$. Use the integration by parts formula \cite[Theorem 3.3.1]{hille} to say that
%	\begin{align}
%		\int_1^{t} \log s \nu_{\Phi}(ds) \, = \, - \log t \bar{\nu}(t) + \int_1^t \frac{1}{s} \bar{\nu}(s) ds 
%	\end{align}
%	and send $t \to +\infty$ to obtain the result. Now we show the converse implication. Since $\log (t) \bar{\nu}(t)\geq 0$ we have that
%	\begin{align}
%		\int_1^t \log s \, \nu_\Phi (ds) \, \leq \, \int_1^t s^{-1} \, \bar{\nu}(s) \, ds.
%	\end{align}
%	Hence, taking the limit as $t \to +\infty$ we obtain
%	\begin{align}
%		\int_1^{+\infty} \log (s) {\nu}_\Phi (ds) \, \leq \, \int_1^{+\infty} s^{-1} \bar{\nu}(s) ds < +\infty.
%	\end{align}
%	Furthermore
%	\begin{align}
%		\log (t) \bar{\nu}_\Phi(t) \, = \, \int_1^t s^{-1} \bar{\nu}_\Phi(s) \, ds \, - \, \int_1^t \log s \, \nu_\Phi (ds) 
%	\end{align}
%	and thus by letting $t \to +\infty$ we obtain that the limit on the left hand side exists.
%\end{proof}
%The latter can be used to identify a class of Bernstein functions satisfying Assumption \eqref{eq:integrlog}.
%\begin{coro}\label{eq:coroassp}
%	Let $\Phi$ be the Laplace exponent of a potentially killed subordinator. If there exist two constants $C>0$ and $\alpha \in (0,1]$ such that $\Phi^\dagger(\rho) \le C\rho^\alpha$ for any $\rho \in (0,1]$, then it satisfies Assumption \eqref{eq:integrlog}.
%\end{coro}
%\begin{proof}
%	Just observe that
%	\begin{equation*}
%		\int_0^1 \frac{\Phi^\dagger(\rho)}{\rho}d\rho \le C\int_0^1 \rho^{\alpha-1}d\rho=\frac{C}{\alpha}<+\infty,
%	\end{equation*}
%	hence Assumption \eqref{eq:integrlog} is verified by using Proposition \ref{prop:integr0}.
%\end{proof}
%\begin{ex}\label{ex:class}
%	Combining Proposition \ref{prop:compl} and Corollary \ref{eq:coroassp}, any complete Bernstein function such that $\Phi^\dagger$ is regularly varying at $0$ of order $\alpha \in (0,1]$ satisfies Assumptions \eqref{eq:extensionA3}, \eqref{eq:uniformlimcond}, \eqref{eq:modcont} and \eqref{eq:integrlog}.
%\end{ex}
%\begin{ex}
%	Let $\Phi$ be a complete Bernstein function and $q^\prime, \mathfrak{b}^\prime, \eta>0$. Then we can define $\Phi_{q^\prime, \mathfrak{b}^\prime,\eta}(z)=q^\prime+\mathfrak{b}^\prime z +\Phi(z+\eta)-\Phi(\eta)$. The function $\Phi_{q^\prime, \mathfrak{b}^\prime,\eta}$ is still a complete Bernstein function and, clearly, 
%	$$\lim_{z \to 0}\frac{\Phi_{q^\prime, \mathfrak{b}^\prime,\eta}^\dagger(z)}{z}=\frac{d \Phi^\dagger}{d  z}(\eta)\in \R.$$
%	This shows, by Proposition \ref{prop:compl} and \ref{prop:integr0} that $\Phi_{q^\prime, \mathfrak{b}^\prime,\eta}$ satisfies Assumptions \eqref{eq:extensionA3}, \eqref{eq:uniformlimcond}, \eqref{eq:modcont} and \eqref{eq:integrlog}. Actually, this class of functions is already included in \eqref{ex:class}, as $\Phi_{q^\prime, \mathfrak{b}^\prime,\eta}^\dagger$ is regularly varying at $0$ of order $1$.
%\end{ex}
%\label{sec:comparisonseries}
%To the best of our knowledge, a series representation for the density of inverse subordinators was explicitly obtained only for the relativistic stable subordinator, i.e., the case $\Phi_{\beta, s}(\lambda) = (s+\lambda)^\beta-s^\beta$, for $s>0$ and $\beta \in (0,1).$ 
%As already stated in the previous section, the representation \eqref{eq:seriesder} provides some completely explicit result if we evaluate the derivatives of the convolution powers $\bar{\mu}_{\Phi}^{\ast n}$. For instance, this is doable for stable subordinators in which case the tail of the L\'evy measure of $\Phi(z)=z^\alpha$ is given by
%\begin{equation*}
%	\bar{\mu}_\alpha(t)=\frac{t^{-\alpha}}{\Gamma(1-\alpha)}.
%\end{equation*}
%We use subscript $\alpha$ instead of $\Phi$ since here $\Phi(z)=z^\alpha$. For these specific tails, it is well-known that
%\begin{equation*}
%	\bar{\mu}_\alpha^{\ast (j+1)}(t)=\frac{t^{j-(j+1)\alpha}}{\Gamma(j+1-(j+1)\alpha)}
%\end{equation*}
%and then
%\begin{equation*}
%	\dersup{}{t}{j}\bar{\mu}_\alpha^{\ast  j}(t)=\frac{t^{-j\alpha-1}}{\Gamma(-j\alpha)}=\frac{t^{-j\alpha-1}}{\pi}\sin(\pi j\alpha)\Gamma(1+j\alpha),
%\end{equation*}
%where we have used Euler's reflection formula, provided that $\alpha \not \in \mathbb{Q}$, while for $\alpha \in \mathbb{Q}$ we have to pay attention to the case in which $j\alpha$ is an integer, for which it can be simply proven that $\bar{\mu}_\alpha^{\ast j}(t)$ is a monomial of degree less than $j$ and thus $\dersup{}{t}{j}\bar{\mu}_\alpha^{\ast j}(t)=0$ as expected. For $l=0$ substituting this into \eqref{eq:fractureI} and then into \eqref{eq:seriessub1}, we get that
%\begin{equation}\label{eq:seriesstable}
%	g_\alpha(x,t)=\sum_{j=1}^{+\infty}(-1)^{j+1}\frac{x^j}{j!}\frac{t^{-j\alpha-1}}{\pi}\sin(\pi j\alpha)\Gamma(1+j\alpha).
%\end{equation}
%This series expansion is well-known in literature, see e.g. \cite[Equation $(7)$]{PG10}. The same argument can be also adopted to obtain the series representation of $f_\alpha$. Indeed, with the same arguments as before
%\begin{equation*}
%	\dersup{}{t}{j}\bar{\mu}_\alpha^{\ast(j+1)}(t)=\frac{t^{-(j+1)\alpha}}{\pi(j+1)\alpha}\sin(\pi\alpha(j+1))\Gamma(\alpha(j+1)+1)
%\end{equation*}
%for any $j \ge 0$ and $\alpha \in (0,1)$. Thus, by using \eqref{eq:seriesder}, we get
%\begin{equation*}
%	f_\alpha(x,t)=\sum_{j=0}^{\infty}(-1)^{j}\frac{\Gamma(1+(j+1)\alpha)}{\alpha (j+1)!}\frac{\sin(\pi \alpha(j+1))}{\pi}x^j t^{-\beta(j+1)}.
%\end{equation*}
%Such a series can also be deducted by combining \eqref{eq:seriesstable} with the relation
%\begin{equation*}
%	f_\alpha(x,t)=\frac{t}{\alpha}x^{-1-\frac{1}{\alpha}}g_\alpha(1,tx^{-\frac{1}{\alpha}}),
%\end{equation*}
%see \cite{meerstra2} for more details. It has also been obtained by means of a limit argument in \cite[Remark 2.3]{kumar}.
%
%A similar approach can be used to deduce some information on the relativistic (or tempered) stable subordinator, i.e. when $\Phi(z)=(\lambda+z)^\alpha-\lambda^\alpha$. In \cite{kumar} the authors provide both an integral and a series representation for the density $f_{\alpha,\lambda}$ of the inverse tempered stable subordinator. In the proof, they exploit the possibility to extend $\phi$ to the whole complex half-plane $\overline{\C(0,\pi)}$ and then they integrate on a keyhole contour centered in $-\lambda$. This seems to be slightly different from our contour, which is not really a keyhole contour and it is always centered in $0$. However, Proposition \ref{prop:extcont} lets us extend the approach to a full keyhole contour. Furthermore, if we choose $\varepsilon<\lambda$, then for any $j \ge 0$ we get
%\begin{equation*}
%\int_{\gamma_\varepsilon}e^{tz}\frac{(\phi(z))^{j+1}}{z}dz=0,
%\end{equation*}
%since the integrand is holomorphic on the disc $\{z \in \C: \ |z|<\varepsilon\}$. Furthermore, since $\phi_+(\rho)$ is real for $\rho>-\lambda$, we can rewrite
%\begin{align*}
%	\dersup{}{t}{j+1}\bar{\mu}_\Phi^{\ast j}(t)&=\frac{1}{\pi}\int_{\lambda}^{+\infty}\Im\left[\frac{(\Phi_+(-\rho))^{j+1}}{\rho}e^{-t\rho}\right]d\rho=\frac{e^{-t\lambda}}{\pi}\int_{0}^{+\infty}\Im\left[\frac{(\Phi_+(\lambda-\rho))^{j+1}}{\rho+\lambda}e^{-t\rho}\right]d\rho.
%\end{align*}
%This leads to
%\begin{align*}
%	f_{\Phi}(x,t)&=\frac{e^{-t\lambda}}{\pi}\sum_{j=0}^{+\infty}\frac{x^j}{j!}(-1)^j\int_{0}^{+\infty}\Im\left[\frac{(\Phi_+(\lambda-\rho))^{j+1}}{\rho+\lambda}e^{-t\rho}\right]d\rho\\
%	&=\frac{e^{x\lambda^\alpha-t\lambda}}{\pi}\sum_{j=0}^{+\infty}\frac{x^j}{j!}(-1)^j\int_{0}^{+\infty}\Im\left[\frac{(\Phi_+(\lambda-\rho))^{j+1}}{\rho+\lambda}e^{-x(-\rho)^\alpha-t\rho}\right]d\rho\\
%	&=\frac{e^{x\lambda^\alpha-t\lambda}}{\pi}\int_0^{+\infty}\frac{e^{x\rho^\alpha \cos(\alpha \pi)-t\rho}}{\rho +\lambda}[\rho^\alpha \sin(\alpha \pi -x\rho^\alpha \sin(i\alpha \pi))+\lambda^\alpha\cos(\alpha \pi)\sin(x\rho^\alpha \sin (i\alpha \pi))]d\rho,
%\end{align*}
%where the last equality follows by simple algebraic manipulations. It is the integral representation of \cite[Theorem 2.1]{kumar} and thus, arguing as in \cite[Proposition 2.1]{kumar}, we get the series representation
%\begin{equation}\label{seriesinvrel}
%	\begin{split}
%		f_{\Phi} (x,t) \, = \, \frac{e^{\lambda^\alpha x}}{\pi} \sum_{j=0}^{+\infty}  \frac{(-1)^j x^j}{j!} \lambda^{\alpha(j+1)} &\left[ \Gamma (1+\alpha(j+1)) \Gamma (-\alpha (k+1), \lambda t) \sin ((j+1)\alpha \pi) \right. \\
%		& \left. - \Gamma (1+\alpha j) \Gamma (-\alpha j, \lambda t) \sin (j\alpha \pi) \right],
%	\end{split}
%\end{equation}
%where $\Gamma(x,y)$ is the upper-incomplete Gamma function.
%
%Let us also underline that, with the same arguments, we can use Proposition \ref{prop:extcont} to obtain the integral representation of the density of the inverse Gamma subordinator (i.e. the case $\Phi(z)=\log(1+z)$), as in \cite[Proposition 1]{kumar2}.
%


%In \cite{kumar} the authors showed that, in this case, for $x,t>0$, the density $f_{\alpha, s}(x,t)$ has the representation (\cite[Proposition 2.1]{kumar})
%\begin{equation}\label{seriesinvrel}
%	\begin{split}
%		f_{\beta,s} (x,t) \, = \, \frac{e^{s^\beta x}}{\pi} \sum_{j=0}^{+\infty}  \frac{(-1)^j x^j}{j!} s^{\beta(j+1)} &\left[ \Gamma (1+\beta(j+1)) \Gamma (-\beta (k+1), st) \sin ((j+1)\beta \pi) \right. \\
%		& \left. - \Gamma (1+\beta j) \Gamma (-\beta j, st) \sin (j\beta \pi) \right].
%	\end{split}
%\end{equation}
%For the stable subordinator, i.e., $\Phi_\beta(\lambda) = \lambda^\beta$, $\beta \in (0,1)$ it is known (see \cite{meerstra2}) that, denoting by $g_\beta(1,x)$ a density of the stable subordinator and $f_\beta(x,t)$ a density of the corresponding inverse, one has
%\begin{equation}\label{densinvfrac}
%f_\beta(x,t)=\frac{t}{\beta}x^{-1-\frac{1}{\beta}}g_\beta(1,tx^{-\frac{1}{\beta}}).
%\end{equation}
%It has been proved (see \cite[Eq. (7)]{PG10}) that
%	\begin{equation}
%	g_\beta(1,t)=\frac{1}{\pi}\sum_{j=1}^{+\infty}(-1)^{j+1}\frac{\Gamma(1+j\beta)}{j!}\sin(\pi \beta j)t^{-1-\beta j}.
%		\label{seriesubstab}
%	\end{equation}
%This leads to the power series
%\begin{equation}\label{seriesinvsub}
%	\begin{split}
%		f_\beta (x,t) \, = \, & \sum_{j=1}^\infty(-1)^{j+1} \frac{\Gamma (1+j\beta)}{\beta j!} \frac{\sin \l \pi\beta j \r}{\pi}   t^{-\beta j} x^{j-1}\\
%		= \, & \sum_{j=0}^\infty(-1)^{j} \frac{\Gamma (1+(j+1)\beta)}{\beta (j+1)!} \frac{\sin \l \pi\beta (j+1) \r}{\pi}   t^{-\beta (j+1)} x^{j}.
%	\end{split}
%\end{equation}
%We now show that \eqref{seriesinvrel} and \eqref{seriesinvsub} can be obtained from our Theorem \ref{thm:seriespi}. We start with \eqref{seriesinvsub} as it is an easier computation.
%To get \eqref{seriesinvsub} we substitute in \eqref{coeff}, $k=0$, $l=0$, $q=0$, $\mathfrak{b}=0$, and we know that the L\'evy measure of the Bernstein function $\Phi(\lambda) = \lambda^\beta$ has the tail
%\begin{equation*}
%\bar{\nu}_\beta(t) \, = \, \frac{t^{-\beta}}{\Gamma (1-\beta)}.
%\end{equation*}
%By \eqref{eq:series} we obtain
%\begin{equation}\label{stabcase}
%f_{\beta}(x,t) \, = \, \sum_{j=0}^{+\infty} (-1)^j \frac{x^j}{j!} \frac{d^j}{dt^j} \nu_{\beta}^{\star (j+1)}(t).
%\end{equation}
%To get \eqref{seriesinvsub} from \eqref{stabcase} note that, for any $y,w>0$ and $t>0$,
%\begin{equation*}
%\frac{t^{w-1}}{\Gamma (w)} \star \frac{t^{y-1}}{\Gamma (y)}  \, = \, \frac{t^{w+y-1}}{\Gamma(y+w)}
%\end{equation*}
%from which we get
%\begin{equation*}
%\bar{\nu}_{\beta}^{\star n}(t) \, = \, \frac{t^{n-n\beta-1}}{\Gamma(n-n\beta)}.
%\end{equation*}
%Suppose $\beta \in (0,1) \backslash \mathbb{Q}$. We have that
%\begin{equation}\label{reflection}
%	\begin{split}
%		\frac{d^j}{dt^j} \bar{\nu}_\beta^{\star(j+1)}(t) = \, & \frac{t^{-\beta(j+1)}}{\Gamma (1-(j+1)\beta)} 
%		= \, \frac{1}{\pi} t^{-\beta (j+1)}\sin (\pi \beta (j+1)) \Gamma (\beta(j+1))  \\
%		= \, & \frac{1}{\pi (j+1) \beta} t^{-\beta (j+1)}\sin (\pi \beta (j+1)) \Gamma (\beta(j+1)+1),
%	\end{split}
%\end{equation}
%where, in the second last step, we used the reflection formula of the Gamma function. By substituting \eqref{reflection} into \eqref{stabcase} we get \eqref{seriesinvsub}. If $\beta \in (0,1) \cap \mathbb{Q}$ then set $\beta = m/n$, for some $m<n$. It follows that the terms $j+1=rn$ for some $r \in \mathbb{N}$ are such that
%\begin{align}
%\bar{\nu}_{\beta}^{\star(j+1)}(t) \, = \, \frac{t^{r(n-m)-1}}{\Gamma (j+(j+1)\alpha+1)}
%\end{align}
%so that
%\begin{align}
%\frac{d^{j}}{dt^j} \bar{\nu}_\beta^{\star(j+1)}(t) \, = \, \frac{d^{rn-1}}{dt^{rn-1}}\frac{t^{r(n-m)-1}}{\Gamma (j+(j+1)\alpha+1)} \, = \, 0
%\end{align}
%and this agrees with the terms in \eqref{seriesinvsub}, which vanish for $\beta=m/n$ and $j=rn$ for some $m<n$, $r \in \mathbb{N}$. The other terms can be dealt with as in \eqref{reflection}. It follows that \eqref{seriesinvsub} can be obtained from \eqref{stabcase}, i.e., from \eqref{eq:series}, for any $\beta \in (0,1)$. With a similar computation it is possible to obtain \eqref{seriesubstab} from \eqref{eq:seriessub}.
%%With very similar computation it is possible to obtain \eqref{seriesubstab} from \eqref{seriessubordin}.
%
%We show now how to recover the representation \eqref{seriesinvrel} from \eqref{eq:series}. The following proposition is useful for this.
%\begin{prop}
%	Let $\Phi$ be a complete Bernstein function and assume that $\Phi$ can be extended with continuity on $\overline{\C(0,\pi)}$. Denote by $\Phi_{+}$ such extension. Assume further that 
%	\begin{equation*}
%		\lim_{z \to +\infty}\frac{\Phi^\dagger_+(z)}{z}=0 \mbox{ uniformly in }\overline{\C(0,\pi)}.
%	\end{equation*}
%	Then, for any $\varepsilon>0$,
%	\begin{equation}\label{eq:integralmucont}
%		\frac{d^r}{dt^r}\overline{\mu}_\Phi^{n \star}(t)=\frac{1}{\pi}\int_{\varepsilon}^{+\infty}\Im\left[(-1)^{r+n+1}\frac{(\Phi^\dagger_+(-\rho))^n}{\rho^{n-r}}e^{-t\rho}\right]d\rho-\frac{1}{2\pi i}\int_{\gamma_\varepsilon} e^{tz}\frac{(\Phi^\dagger(z))^n}{z^{n-r}}dz.
%	\end{equation}
%\end{prop}
%\begin{proof}
%	First of all, observe that by the assumptions, since $\Phi$ is Hermitian, then $\Phi$ can be extended with continuity also on $\overline{\C(-\pi,0)}$ and, if we denote by $\Phi_-$ such extension, $\overline{\Phi_+(z)}=\Phi_-(\overline{z})$ for any $z \in (-\infty,0)$. Furthermore, observe that the considered hypotheses imply Assumptions \eqref{eq:extensionA3} and \eqref{eq:uniformlimcond}. We get the result arguing as in the proof of Lemma \ref{lem:convtail}, considering now the circuit $\partial D$ with $\theta=0$.
%\end{proof}
%In particular, if we consider $\Phi_{\beta,s}(z)=(z+s)^{\beta}-s^\beta$, then for $\varepsilon<s$ we have
%\begin{equation*}
%	\int_{\gamma_\varepsilon}e^{tz}\frac{(\Phi_{\beta,s}^\dagger(z))^{j+1}}{z}dz=0
%\end{equation*}
%since it is the integral of a holomorphic function on a closed circuit. Hence, we get, using \eqref{eq:series},
%\begin{align}
%&	f_{\beta,s}(x,t)= \notag \\ &=\frac{1}{\pi}\sum_{j=0}^{+\infty}(-1)^j\frac{x^j}{j!}\int_{s}^{+\infty}\Im\left[\frac{((s-\rho)^\beta-s^\beta)^{j+1}}{\rho}e^{-t\rho}\right]d\rho \notag \\
%	&=\frac{e^{-ts}}{\pi}\sum_{j=0}^{+\infty}(-1)^j\frac{x^j}{j!}\int_{0}^{+\infty}\Im\left[\frac{((-\rho)^\beta-s^\beta)^{j+1}}{\rho+s}e^{-t\rho}\right]d\rho \notag \\
%	&=\frac{e^{xs^\beta-ts}}{\pi}\int_{0}^{+\infty}\Im\left[\frac{(-\rho)^\beta-s^\beta}{\rho+s}e^{-x((-\rho)^\beta)-t\rho}\right]d\rho \notag \\
%	&=\frac{e^{xs^\beta-ts}}{\pi}\int_{0}^{+\infty}\Im\left[\frac{e^{i\beta \pi}\rho^\beta-s^\beta}{\rho+s}e^{-x\rho^\beta e^{i\beta \pi}-t\rho}\right]d\rho \notag \\
%	&=\frac{e^{xs^\beta-ts}}{\pi}\int_{0}^{+\infty}\frac{e^{-x\rho^\beta\cos(\beta\pi)-t\rho}}{\rho+s}\left[\rho^\beta\sin(\beta \pi)\cos(x\rho^\beta \sin(i\beta \pi))-\rho^\beta\cos(\beta \pi)\sin(x\rho^\beta \sin(i\beta \pi)) \right. \notag \\  & \qquad   \left.+s^\beta\cos(\beta \pi)\sin(x\rho^\beta \sin(i\beta \pi))\right]d\rho \notag \\
%	&=\frac{e^{xs^\beta-ts}}{\pi}\int_{0}^{+\infty}\frac{e^{x\rho^\beta\cos(\eta\pi)-t\rho}}{\rho+s}\left[\rho^\beta\sin(\beta \pi-x\rho^\beta \sin(i\beta \pi))+s^\beta\cos(\beta \pi)\sin(x\rho^\beta \sin(i\beta \pi))\right]d\rho.
%	\label{intkumar}
%\end{align}
%In order to show that \eqref{intkumar} coincides with \eqref{seriesinvrel} it is sufficient to note that \eqref{intkumar} coincides with \cite[eq. (2.8)]{kumar} which the authors use to obtain \eqref{seriesinvrel} (see \cite[Proposition 2.1]{kumar}).
%%It is not easy, instead, to recover the representation \eqref{seriesinvrel} from \eqref{eq:series}. However, the method adopted to compute \eqref{derivatecode} can be slightly modified (to include the case $\theta =0$) to obtain, for the Bernstein function $\phi_{\beta, s}(\lambda) = (\lambda + s)^\beta - s^\beta$ that
%%\begin{align}
%%\frac{d^j}{dt^j} \bar{\nu}_\phi^{\star(j+1)}(t) \, = \,& \frac{1}{\pi} \Im \int_{s}^{+\infty} e^{-t\rho} \frac{\l \l \rho e^{i\pi}+s \r^\beta-s^\beta \r^{j+1}}{\rho} d\rho \notag \\
%%= \, & \frac{1}{\pi} e^{-ts} \Im \int_0^{+\infty} e^{-ty} \frac{\l e^{i\beta \pi}y^\beta + e^{i\pi}s^\beta \r^{j+1}}{s+y} dy \notag \\
%%= \, & ********
%%\end{align}
%
%%
%%
%%\begin{rmk}
%%It is possible to see from Theorem \ref{behavatzero} that
%%\begin{align}
%%\lim_{x \to 0+} f_\Phi(x,t) \, = \, \bar{\nu}_\Phi (t).
%%\label{353}
%%\end{align}
%%The latter is a well-know property of \eqref{353}. Here we have that the limit holds uniformly form $t \in [a,b]$ EXPAND THE REMARK
%%\end{rmk}
%
%
%In \cite{bur}, the authors studied a special class of Thorin subordinators. Let us consider now an example taken from this paper. Indeed, fix $\alpha \in (0,1)$, let us consider the subordinator $\sigma$ whose Laplace exponent is given by $\Phi(z)=\varphi(z)-\varphi(0)-z$, where $\varphi(z)$ is the unique solution of $\varphi(z)-(\varphi(z))^\alpha=z$ for $z \ge 0$. Clearly $\Phi(0)=0$ and then $q=0$. Furthermore, by \cite[Theorem 1]{bur}, we know that $\mathfrak{b}=0$. By \cite[Proposition $5$]{bur}, we know that $g_\Phi$ is well-defined. Furthermore, we can rewrite \cite[Equation (21)]{bur} in the following form, for $x,t>0$,
%\begin{equation}\label{eq:seriesgphi}
%	g_\Phi(x,t)=\sum_{j=1}^{+\infty}\frac{x^j}{j!}(A_j(t)+B_j(t)+C_j(t)),
%\end{equation}
%where
%\begin{align*}
%	A_j(t)&=j\sum_{n=1}^{+\infty}(-1)^{n+1}\frac{\Gamma(1+n\alpha)}{n!}\sin(\pi n \alpha)t^{-n\alpha+n}\\
%	B_j(t)&=j(j-1)\sum_{n=1}^{+\infty}(-1)^{n+1}\frac{(n+1)\Gamma(1+n\alpha)}{n!}\sin(\pi n \alpha)t^{-n\alpha+n-1}\\
%	C_j(t)&=\begin{cases}
%	0 & j=1,2\\	
%		\displaystyle \sum_{k=0}^{j}\frac{j!}{(j-k)!}\sum_{n=k+1}^{+\infty}(-1)^{n+1}\binom{n+1}{k+2}\frac{\Gamma(1+n\alpha)}{n!}\sin(\pi n \alpha)t^{-n\alpha+n-2-k} & j \ge 3.
%	\end{cases}
%\end{align*}
%In particular, comparing \eqref{eq:seriessub1} and \eqref{eq:seriesgphi}, we get, by means of \eqref{eq:fractureI},
%\begin{equation*}
%	\frac{d^j}{d t^j}\bar{\mu}_\Phi^{\ast j}(t)=A_j(t)+B_j(t)+C_j(t), \quad t>0, \ j \ge 1.
%\end{equation*}
%For $j=1$, we get again \cite[Equation (22)]{bur}. Furthermore, setting
%\begin{align*}
%	\overline{A}_j(t)&=j\sum_{n=1}^{+\infty}(-1)^{n}\frac{\Gamma(1+n\alpha)}{n!(n-n\alpha+1)}\sin(\pi n \alpha)t^{-n\alpha+n+1}\\
%	\overline{B}_j(t)&=j(j-1)\sum_{n=1}^{+\infty}(-1)^{n}\frac{(n+1)\Gamma(1+n\alpha)}{n!(n-n\alpha)}\sin(\pi n \alpha)t^{-n\alpha+n}\\
%	\overline{C}_j(t)&=\begin{cases}
%		0 & j=1,2\\	
%		\displaystyle \sum_{k=0}^{j}\frac{j!}{(j-k)!}\sum_{n=k+1}^{+\infty}(-1)^{n}\binom{n+1}{k+2}\frac{\Gamma(1+n\alpha)}{n!(n-n\alpha-1-k)}\sin(\pi n \alpha)t^{-n\alpha+n-1-k} & j \ge 3,
%	\end{cases}
%\end{align*}
%we have
%\begin{equation*}
%	\frac{d^j}{d t^j}\bar{\mu}_\Phi^{\ast j+1}(t)=\bar{A}_{j+1}(t)+\bar{B}_{j+1}(t)+\bar{C}_{j+1}(t), \quad t>0, \ j \ge 0
%\end{equation*}
%and then we achieve the density of the inverse subordinator $L_\Phi$ by means of \eqref{eq:seriesder}.
%
%Finally, we show a specific application of Theorem \ref{behavatzero} to the context of time-nonlocal equations. In \cite{moving} the following assumption is used to prove the main result:
%	\begin{itemize}
%		\item	For any $0<a \le b$ there exists $\delta_{a,b}>0$ and a function $F_{a,b}:(0,\delta_{a,b}) \to (0,+\infty)$ such that 
%		\begin{equation*}
%			\int_0^{\delta_{a,b}} x^{-\frac{1}{2}}F_{a,b}(x)dx<\infty
%		\end{equation*} 
%		and, for any $x \in (0,\delta_{a,b})$ and $t \in [a,b]$,
%		\begin{equation*}
%			\left|\pd{f_{\Phi}}{x}(x,t)\right| \le F_{a,b}(x).
%		\end{equation*}
%\end{itemize}
%It is clear that, since the limits in \ref{behavatzero} are locally uniform in $t$, the latter condition is verified with $F_{a,b}$ being independent of $x$ if $\phi$ satisfies \eqref{eq:extensionA3} and \eqref{eq:uniformlimcond}. In particular, the latter holds whenever (but not exclusively if) $\phi$ is a complete Bernstein function.
%
%\begin{rmk}
%We remark that Theorem \ref{behavatzero} is particularly useful in the context of (time-)non-local equations. For example in \cite{moving} the following assumption is used to proved the main result:
%	\begin{itemize}
%	\item	For any $0<a \le b$ there exists $\delta_{a,b}>0$ and a function $F_{a,b}:(0,\delta_{a,b}) \to (0,+\infty)$ such that 
%		\begin{equation*}
%			\int_0^{\delta_{a,b}} x^{-\frac{1}{2}}F_{a,b}(x)dx<\infty
%		\end{equation*} 
%		and, for any $x \in (0,\delta_{a,b})$ and $t \in [a,b]$,
%		\begin{equation*}
%			\left|\pd{f_{\Phi}}{x}(x,t)\right| \le F_{a,b}(x).
%		\end{equation*}
%	\end{itemize} 
%	Thanks to Theorem \ref{behavatzero}, we know that this condition is verified whenever $\phi$ satisfies \eqref{eq:extensionA3} and \eqref{eq:uniformlimcond}.
%	Indeed, we have by Theorem \ref{behavatzero} that, for any $[a,b] \subset (0, +\infty)$,
%	\begin{equation*}
%		\lim_{x \to 0^+}\max_{t \in [a,b]}\left| \pd{f_{\Phi}}{x}(x,t)-\mathcal{I}_{0,1,0}(t)\right|=0
%	\end{equation*}
%	where $\mathcal{I}_{0,1,0}(t)$ is defined in \eqref{coeff}.
%	It follows that there exists $\delta_{a,b}>0$ such that
%	\begin{equation*}
%		\max_{t \in [a,b]}\left| \pd{f_{\Phi}}{x}(x,t)-\mathcal{I}_{0,1,0}(t)\right|<1, \ \forall x \in (0,\delta_{a,b}),
%	\end{equation*}
%	and thus
%	\begin{equation*}
%		\max_{t \in [a,b]}\left| \pd{f_{\Phi}}{x}(x,t)\right|<1+\max_{t \in [a,b]}\left| \mathcal{I}_{0,1,0}(t)\right|=F_{a,b}, \ \forall x \in (0,\delta_{a,b}).
%	\end{equation*}
%\end{rmk}
