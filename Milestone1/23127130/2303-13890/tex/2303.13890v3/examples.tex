\section{Examples and Further Discussion}\label{sec:ex}
Here we provide computational examples and further discussion of our results.
\subsection{Examples for the results in Section \ref{subsec:L}}
 We start with examples concerning Section \ref{subsec:L}.

Take a Bernstein function with $q=\mathfrak{b}=0$ and a L\'evy measure with a compactly supported density that is regularly varying at $0$ with index $-1-\alpha$ for some $\alpha \in (0,1)$, as for instance
\begin{equation}\label{eq:examplenocomp}
\nu_{\Phi}(dy)=m(y)\ind{y<\frac12}dy, \quad \, m(y)\sim y^{-\alpha-1}, \mbox{ as }x \to 0.	
\end{equation}
We can opt for general regular variation and $\mathfrak{b},q>0$ but we leave it to the interested reader. 
%\ga{Observe also that the related Bernstein function $\Phi$ is not complete, as clearly $m$ is not completely monotone, since it is compactly supported.}
Then,
\[\Phi'''(x)\sim C_3x^{\alpha-3} \text{ and } -\Phi''(x)\sim C_2x^{\alpha-2}, \mbox{ as }x \to \infty\] 
 for positive constants $C_2,C_3$. Clearly,  \eqref{def:condiB} is satisfied. Also, due to the finite support of $\mu_\phi$ we have that $\abs{\phi''(0^+)}<\infty,\phi'''(0^+)<\infty$ and \eqref{def:condiB'} is valid too. Furthermore, for some $C>0$,
\[x^2\Delta(x)=x^2\int_{0}^{\frac1x}y^2 m(y)dy\sim Cx^\alpha, \ \mbox{ as } \ x \to \infty,\]
and  \eqref{def:condiA} holds with $L=\infty$. It is clear that, actually, assumptions \eqref{def:condiA} and \eqref{def:condiB} hold for any Bernstein function $\Phi$ that is regularly varying at $\infty$ with index $\alpha \in (0,1)$. Going back to the example, let $t(x)/x\downarrow 0$. Using $-\Phi''(x)\sim C_2x^{\alpha-2}$, as $x \to \infty$, we establish, for any $k,l\geq 0$,  as $x\to\infty$,
\begin{equation}\label{asymp1ex}
	\begin{split}
		& \sup_{(t,x) \in \mathbb{D}^\prime}\abs{(-1)^k\frac{\sqrt{2\pi C_2}}{C^{k+1}_0}\frac{\sqrt{x}}{c^{k+l+1-\frac{\alpha}{2}}}e^{-ct+x\Phi(c)}\frac{\partial^k\partial ^l }{\partial x^k\partial t^l}\fP{x,t}-1}=\bo{\sqrt{\frac{\ln\lbrb{x}} {a^{\alpha}_*x}}},
	\end{split}
\end{equation}
where we used Theorem \ref{thm:mainL}  with $\Phi(y)=\phi^{\dagger}(y)\sim C_0 y^{\alpha}$, as $y \to \infty$, and the monotonicity in $c$ of the first asymptotic term in \eqref{asymp}. Finally, employing that $\Phi'(y)\sim C_1y^{\alpha-1}$, as $y \to \infty$, we get from \eqref{eq:a*1} 
\[C_1a^{\alpha-1}_*\sim \frac{t(x)}{x} \quad \mbox{ as }x \to \infty \iff a_* \sim \lbrb{\frac{1}{C_1}\frac{t(x)}{x}}^{\frac{1}{\alpha-1}}, \quad \mbox{ as }x \to \infty,\]
and we can plug in this expression in \eqref{asymp1ex}. The speed of convergence of the first asymptotic term in \eqref{asymp} is hence $(t(x)/x)^{\alpha/(2(1-\alpha))}x^{-\frac12}$ which offers faster decay when $\alpha$ approaches $1$ and  we note that it is faster than the second term of the asymptotic in \eqref{asymp} which in this case  is of order $\bo{x^{-1/2}}$.

We continue the example above with an illustration of Theorem \ref{thm:main1}, for which only \eqref{def:condiA} is required.  Take $t(x)=x$ and then $a_*=a_*(x)=(\phi')^{-1}(1)$. Hence, from \eqref{def:fin} we deduce that the explicit in $x$
\begin{equation*}
	\begin{split}
		&\frac{\partial^k\partial ^l }{\partial x^k\partial t^l}\fP{x,t}=\frac{(-1)^k\Phi^\dagger(a_*)\Phi^{k}(a_*)}{\sqrt{2\pi}a^{1-l}_*\sqrt{-\Phi''(a_*)}}\frac{e^{x\lbrb{a_*-\Phi(a_*)}}}{\sqrt{x}} \lbrb{1+\bo{\sqrt{\frac{\ln(x)}{x}}}}, \quad \mbox{ as }x \to \infty.
	\end{split}
\end{equation*}
Since $x\phi'(x)\leq \phi(x)$, see \eqref{it:ineq} of Lemma \ref{lem:Bern}, we double check that $a_*-\phi\lbrb{a_*}\leq 0$. 

Finally, for this example, consider the case when $\limi{x}t(x)/x=\infty=\phi'(0^+)$. Then as above $a_*\sim C (t(x)/x)^{-\frac{1}{1-\alpha}}$  and $x=\so{t(x)}$, as $x \to \infty$, thus
\[-a^2_*\phi''(a_*)x\sim Ca^{\alpha}_*x\sim C't^{-\frac{\alpha}{1-\alpha}}(x)x^{\frac{1}{1-\alpha}}, \quad \mbox{ as }x \to \infty,\]
which goes to infinity if $t(x)=\so{x^{\frac{1}{\alpha}}}$ and the first requirement of \eqref{eq:addCondi} holds. Also, for any $\delta>0$,
\[\limi{x}e^{-\delta x}t^{-\frac{\alpha}{1-\alpha}}(x)x^{\frac{1}{1-\alpha}}=0,\]
for $t(x)=\so{x^{\frac{1}{\alpha}}}$ and the third condition in \eqref{eq:addCondi} holds. Under the same restrictions $\ln(1/a_*)$ is at most of logarithmic growth and the second imposition of \eqref{eq:addCondi} is valid and Theorem \ref{thm:main2} is applicable.

 Let us discuss the route to the derivation of new fine local estimates in the region of the lower envelope. We use the example above with $\alpha=1/2$. In this case one has to consider 
 \[\Pbb{\sigma(x)\leq c\frac{x^2}{\log_2x}},\]
 where $\log_2x=\log\log x, c>0$, see \cite[Chapter III]{bertoinb}. Set $t(x)=c\frac{x^2}{\log_2x}$. Clearly, $t(x)/x\to \infty$ and from above \eqref{def:condiA},\eqref{def:condiB} and \eqref{def:condiB'} hold. Also, 
 \[\phi'(a_*)=\frac{t(x)}{x} \Rightarrow a_*\sim C\frac{\log^2_2 x}{x^2}, \quad \mbox{ as }x \to \infty.\]
 Furthermore, the first requirement of \eqref{eq:addCondi} is fulfilled since  as $x \to \infty$,
 \[-xa_*\phi''(a_*)\sim Cxa^{-\frac12}_*\sim C\log_2(x)\to \infty. \]
The second and third impositions of \eqref{eq:addCondi} are then obvious and Theorem \ref{thm:mainS} holds true and by virtue of its claims it yields local estimates for the densities and all derivatives of the probabilities above.
 
Assume that  $\abs{\phi''(0^+)}<\infty$ and hence $\phi'(0^+)<\infty$. Let us determine the speed by which $t(x)/x$ may approach $\phi'(0^+)$ so that our results hold. First, we note that since $a_*\to 0$ then, as $x \to \infty$,
\[-a_*\phi''(0^+)\sim \phi'(0)-\phi'(a_*)=\phi'(0)-\frac{t(x)}{x}.\]
From Remark \ref{rem:main2} we have to ensure that  $\limi{x}a_*\sqrt{x}=\infty$. Hence, from the last relation we must have
\[\limi{x}\frac{\sqrt{x}}{x\phi'(0)-t(x)}=0.\]
Thus, $\sqrt{x}=\so{x\phi'(0)-t}$, as $x \to \infty$, or alternatively $t<x\phi'(0)-K\sqrt{x}$, for all $K>0$, and all $x$ large enough. This captures the region below that of the central limit theorem  for the density $g_\phi$, see Theorem \ref{thm:mainS}, and therefore we can approximate with high precision, as $x\to\infty,$ quantities of the type
\[\Pbb{\sigma(x)\in\lbbrbb{a,b}}=\int_a^b g_\phi(x,t)dt.\]
\subsection{Examples for the results in Section \ref{subsec:series}}\label{subsec:exseries}
First we develop some explicit examples for Section \ref{subsec:series}.
%As already stated in the previous section, 
It is clear that the representation \eqref{eq:seriesder} provides some completely explicit result if we evaluate the derivatives of the convolution powers $\bar{\mu}_{\Phi}^{\ast n}$. For instance, this is doable for stable subordinators in which case the tail of the L\'evy measure of $\Phi(z)=z^\alpha$ is given by
	\begin{equation*}
		\bar{\mu}_\alpha(t)=\frac{t^{-\alpha}}{\Gamma(1-\alpha)},
	\end{equation*}
	where we use the subscript $\alpha$ instead of $\Phi$ to underline the dependence on the single parameter.
	% since here $\Phi(z)=z^\alpha$. 
	 For these specific tails, it is well-known that
	\begin{equation*}
		\bar{\mu}_\alpha^{\ast (j+1)}(t)=\frac{t^{j-(j+1)\alpha}}{\Gamma(j+1-(j+1)\alpha)}
	\end{equation*}
	and then
	\begin{equation*}
		\dersup{}{t}{j}\bar{\mu}_\alpha^{\ast  j}(t)=\frac{t^{-j\alpha-1}}{\Gamma(-j\alpha)}=\frac{t^{-j\alpha-1}}{\pi}\sin(\pi j\alpha)\Gamma(1+j\alpha),
	\end{equation*}
	where we have used Euler's reflection formula, provided that $\alpha \not \in \mathbb{Q}$, while for $\alpha \in \mathbb{Q}$ we have to pay attention to the case in which $j\alpha$ is an integer, for which it can be simply proven that $\bar{\mu}_\alpha^{\ast j}(t)$ is a monomial of degree less than $j$ and thus $\dersup{}{t}{j}\bar{\mu}_\alpha^{\ast j}(t)=0$ as expected. For $l=0$ substituting this into \eqref{eq:fractureI} and then into \eqref{eq:seriessub1}, we get that
	\begin{equation}\label{eq:seriesstable}
		g_\alpha(x,t)=\sum_{j=1}^{+\infty}(-1)^{j+1}\frac{x^j}{j!}\frac{t^{-j\alpha-1}}{\pi}\sin(\pi j\alpha)\Gamma(1+j\alpha).
	\end{equation}
	This series expansion is well-known in literature, see e.g. \cite[Equation $(7)$]{PG10}. The same argument can be also adopted to obtain the series representation of $f_\alpha$. Indeed, with the same arguments as before
	\begin{equation*}
		\dersup{}{t}{j}\bar{\mu}_\alpha^{\ast(j+1)}(t)=\frac{t^{-(j+1)\alpha}}{\pi(j+1)\alpha}\sin(\pi\alpha(j+1))\Gamma(\alpha(j+1)+1)
	\end{equation*}
	for any $j \ge 0$ and $\alpha \in (0,1)$. Thus, by using \eqref{eq:seriesder}, we get
	\begin{equation}\label{eq:seriesstable123}
		f_\alpha(x,t)=\sum_{j=0}^{\infty}(-1)^{j}\frac{\Gamma(1+(j+1)\alpha)}{\alpha (j+1)!}\frac{\sin(\pi \alpha(j+1))}{\pi}x^j t^{-\alpha(j+1)}.
	\end{equation}
	Such a series can also be deducted by combining \eqref{eq:seriesstable} with the relation
	\begin{equation*}
		f_\alpha(x,t)=\frac{t}{\alpha}x^{-1-\frac{1}{\alpha}}g_\alpha(1,tx^{-\frac{1}{\alpha}}),
	\end{equation*}
	see \cite{meerstra2} for more details. The series representation \eqref{eq:seriesstable123} has also been obtained by means of a limit argument in \cite[Remark 2.3]{kumar}.
	
	A similar approach can be used to deduce some information on the relativistic (or tempered) stable subordinator, i.e. when $\Phi(z)=(\lambda+z)^\alpha-\lambda^\alpha$. In \cite{kumar} the authors provide both an integral and a series representation for the density $f_{\alpha,\lambda}$ of the inverse tempered stable subordinator. In the proof, they exploit the possibility to extend $\phi$ to the whole complex half-plane $\overline{\C(0,\pi)}$ and then they integrate on a keyhole contour centered in $-\lambda$. This seems to be slightly different from our contour, which is not really a keyhole contour and it is always centered in $0$. However, Proposition \ref{prop:extcont} lets us extend the approach to a full keyhole contour. Furthermore, if we choose $\varepsilon<\lambda$, then for any $j \ge 0$ we get
	\begin{equation*}
		\int_{\gamma_\varepsilon}e^{tz}\frac{(\phi(z))^{j+1}}{z}dz=0,
	\end{equation*}
	since the integrand is holomorphic on the disc $\{z \in \C: \ |z|<\varepsilon\}$. Since $\phi_+(\rho)$ is real for $\rho>-\lambda$, we can rewrite
	\begin{align*}
		\dersup{}{t}{j+1}\bar{\mu}_\Phi^{\ast j}(t)&=\frac{1}{\pi}\int_{\lambda}^{+\infty}\Im\left[\frac{(\Phi_+(-\rho))^{j+1}}{\rho}e^{-t\rho}\right]d\rho=\frac{e^{-t\lambda}}{\pi}\int_{0}^{+\infty}\Im\left[\frac{(\Phi_+(\lambda-\rho))^{j+1}}{\rho+\lambda}e^{-t\rho}\right]d\rho.
	\end{align*}
	This leads to
	\begin{align*}
		f_{\Phi}(x,t)&=\frac{e^{-t\lambda}}{\pi}\sum_{j=0}^{+\infty}\frac{x^j}{j!}(-1)^j\int_{0}^{+\infty}\Im\left[\frac{(\Phi_+(\lambda-\rho))^{j+1}}{\rho+\lambda}e^{-t\rho}\right]d\rho\\
		&=\frac{e^{x\lambda^\alpha-t\lambda}}{\pi}\sum_{j=0}^{+\infty}\frac{x^j}{j!}(-1)^j\int_{0}^{+\infty}\Im\left[\frac{(\Phi_+(\lambda-\rho))^{j+1}}{\rho+\lambda}e^{-x(-\rho)^\alpha-t\rho}\right]d\rho\\
		&=\frac{e^{x\lambda^\alpha-t\lambda}}{\pi}\int_0^{+\infty}\frac{e^{x\rho^\alpha \cos(\alpha \pi)-t\rho}}{\rho +\lambda}[\rho^\alpha \sin(\alpha \pi -x\rho^\alpha \sin(i\alpha \pi))+\lambda^\alpha\cos(\alpha \pi)\sin(x\rho^\alpha \sin (i\alpha \pi))]d\rho,
	\end{align*}
	where the last equality follows by simple algebraic manipulations. It is the integral representation of \cite[Theorem 2.1]{kumar} and thus, arguing as in \cite[Proposition 2.1]{kumar}, we get the series representation
	\begin{equation}\label{seriesinvrel}
		\begin{split}
			f_{\Phi} (x,t) \, = \, \frac{e^{\lambda^\alpha x}}{\pi} \sum_{j=0}^{+\infty}  \frac{(-1)^j x^j}{j!} \lambda^{\alpha(j+1)} &\left[ \Gamma (1+\alpha(j+1)) \Gamma (-\alpha (k+1), \lambda t) \sin ((j+1)\alpha \pi) \right. \\
			& \left. - \Gamma (1+\alpha j) \Gamma (-\alpha j, \lambda t) \sin (j\alpha \pi) \right],
		\end{split}
	\end{equation}
	where $\Gamma(x,y)$ is the upper-incomplete Gamma function.
	
	With the same arguments, we can use Proposition \ref{prop:extcont} to obtain the integral representation of the density of the inverse Gamma subordinator (i.e. the case $\Phi(z)=\log(1+z)$), as in \cite[Proposition 1]{kumar2}.

	In \cite{bur}, the authors studied a special class of Thorin subordinators. Let us consider now an example taken from this paper. Fix $\alpha \in (0,1)$, let us consider the subordinator $\sigma$ whose Laplace exponent is given by $\Phi(z)=\varphi(z)-\varphi(0)-z$, where $\varphi(z)$ is the unique solution of $\varphi(z)-(\varphi(z))^\alpha=z$ for $z \ge 0$. Clearly $\Phi(0)=0$ and then $q=0$. Furthermore, by \cite[Theorem 1]{bur}, we know that $\mathfrak{b}=0$ and, by \cite[Proposition $5$]{bur}, we get that $g_\Phi$ is well-defined. We can rewrite \cite[Equation (21)]{bur} in the following form, for $x,t>0$,
	\begin{equation}\label{eq:seriesgphi}
		g_\Phi(x,t)=\sum_{j=1}^{+\infty}\frac{x^j}{j!}(A_j(t)+B_j(t)+C_j(t)),
	\end{equation}
	where
	\begin{align*}
		A_j(t)&=j\sum_{n=1}^{+\infty}(-1)^{n+1}\frac{\Gamma(1+n\alpha)}{n!}\sin(\pi n \alpha)t^{-n\alpha+n}\\
		B_j(t)&=j(j-1)\sum_{n=1}^{+\infty}(-1)^{n+1}\frac{(n+1)\Gamma(1+n\alpha)}{n!}\sin(\pi n \alpha)t^{-n\alpha+n-1}\\
		C_j(t)&=\begin{cases}
			0, & j=1,2,\\	
			\displaystyle \sum_{k=0}^{j}\frac{j!}{(j-k)!}\sum_{n=k+1}^{+\infty}(-1)^{n+1}\binom{n+1}{k+2}\frac{\Gamma(1+n\alpha)}{n!}\sin(\pi n \alpha)t^{-n\alpha+n-2-k}, & j \ge 3.
		\end{cases}
	\end{align*}
	In particular, comparing \eqref{eq:seriessub1} and \eqref{eq:seriesgphi}, we get, by means of \eqref{eq:fractureI},
	\begin{equation*}
		\frac{d^j}{d t^j}\bar{\mu}_\Phi^{\ast j}(t)=A_j(t)+B_j(t)+C_j(t), \quad t>0, \ j \ge 1.
	\end{equation*}
	For $j=1$, we get again \cite[Equation (22)]{bur}. Moreover, setting
	\begin{align*}
		\overline{A}_j(t)&=j\sum_{n=1}^{+\infty}(-1)^{n}\frac{\Gamma(1+n\alpha)}{n!(n-n\alpha+1)}\sin(\pi n \alpha)t^{-n\alpha+n+1}\\
		\overline{B}_j(t)&=j(j-1)\sum_{n=1}^{+\infty}(-1)^{n}\frac{(n+1)\Gamma(1+n\alpha)}{n!(n-n\alpha)}\sin(\pi n \alpha)t^{-n\alpha+n}\\
		\overline{C}_j(t)&=\begin{cases}
			0, & j=1,2,\\	
			\displaystyle \sum_{k=0}^{j}\frac{j!}{(j-k)!}\sum_{n=k+1}^{+\infty}(-1)^{n}\binom{n+1}{k+2}\frac{\Gamma(1+n\alpha)}{n!(n-n\alpha-1-k)}\sin(\pi n \alpha)t^{-n\alpha+n-1-k}, & j \ge 3,
		\end{cases}
	\end{align*}
	we have
	\begin{equation*}
		\frac{d^j}{d t^j}\bar{\mu}_\Phi^{\ast j+1}(t)=\bar{A}_{j+1}(t)+\bar{B}_{j+1}(t)+\bar{C}_{j+1}(t), \quad t>0, \ j \ge 0,
	\end{equation*}
	and then we achieve the density of the inverse subordinator $L_\Phi$ by means of \eqref{eq:seriesder}.
	
	Finally, we show a specific application of Theorem \ref{behavatzero} to the context of time-nonlocal equations. For example, in \cite{moving} the following assumption is used to prove the main result:
	\begin{itemize}
		\item	For any $0<a \le b$ there exists $\delta_{a,b}>0$ and a function $F_{a,b}:(0,\delta_{a,b}) \to (0,+\infty)$ such that 
		\begin{equation*}
			\int_0^{\delta_{a,b}} x^{-\frac{1}{2}}F_{a,b}(x)dx<\infty
		\end{equation*} 
		and, for any $x \in (0,\delta_{a,b})$ and $t \in [a,b]$,
		\begin{equation*}
			\left|\pd{f_{\Phi}}{x}(x,t)\right| \le F_{a,b}(x).
		\end{equation*}
	\end{itemize}
	It is clear that we can Theorem \ref{behavatzero} for $n=0$ and since the remainder is locally uniform in $t$, the latter condition is verified with $F_{a,b}$ being independent of $x$ if $\phi$ satisfies \eqref{eq:extensionA3} and \eqref{eq:uniformlimcond}. In particular, the latter holds whenever (but not exclusively if) $\phi$ is a complete Bernstein function, as discussed in Section \ref{discussionassumptions}.
