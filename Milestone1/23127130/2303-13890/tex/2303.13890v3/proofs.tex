\section{Proofs of results in Subsection \ref{subsec:L}}
\label{sec:proofs}
Here we prove the results contained in Subsection \ref{subsec:L}
\subsection{Proof of Theorem \ref{thm:regularityfphi1}}
In order to prove Theorem \ref{thm:regularityfphi1} we need some preliminary results. We first show a general condition under which the density $f_\Phi$ is smooth in $\mathbb{D}$ for $x$ large enough and at the same time we provide an integral representation of $f_\Phi$ and its derivatives by means of Laplace inversion. Recall the definition of $\phi^\dagger$ in \eqref{def:Pdag}.
\begin{prop}\label{prop:LTthetapi}
	Let $\Phi$ be the Laplace exponent of a potentially killed subordinator. Assume that for $n \ge 0$ there exist $a>0$ and $x_0\geq 0$ such that, for any $x>x_0$,
	\begin{equation}\label{eq:Real1}
		\begin{split}
			\int_{-\infty}^\infty |b|^n e^{-x\Re{\Phi(\ab)}}db<\infty.
		\end{split}
	\end{equation}
	%		for any $x>x_0$.
	%		Then, for $x>x_0$ and $t>0$,
	%		\begin{equation}\label{eq:P1_0}
		%			\begin{split}
			%				&\int_{-\infty}^\infty \frac{\Phi^{\dagger}(\ab)}{\ab}e^{-x\Phi(\ab)+t\lbrb{\ab}}db
			%			\end{split}
		%		\end{equation} 
	%		is absolutely convergent and, for $x_0<x<t/\mathfrak{b}$,
	%		\begin{equation}
		%		f_\Phi(x,t) \, = \, \int_{-\infty}^{+\infty} \frac{\Phi^\dagger(a+ib)}{a+ib} e^{t(a+ib)-x\Phi(a+ib)} db.
		%		\label{intrepr}
		%		\end{equation}
	%		Furthermore, if for any $x>x_0$,
	%		\begin{equation}\label{eq:Real2}
		%		\int_{-\infty}^{+\infty} |b|^n e^{-x \Re \Phi (a+ib)} db < \infty
		%		\end{equation}
	Then, for any $k,l \ge 0$ with  $k+l \leq n$, $\displaystyle \frac{\partial^k}{\partial x^k}\frac{\partial^l}{\partial t^l}f_\Phi(x,t)$ is well defined for $(x,t) \in \mathbb{D}$ with $x>x_0$ and
	% $x>x_0$ and $t>0$,
	%		\begin{equation}\label{eq:P1}
		%			\begin{split}
			%				&\int_{-\infty}^\infty \frac{\Phi^{\dagger}(\ab)\Phi^{k}(\ab)}{\lbrb{\ab}^{1-l}}e^{-x\Phi(\ab)+t\lbrb{\ab}}db
			%			\end{split}
		%		\end{equation} 
	%is absolutely convergent and, for any 
	\begin{equation}
		\frac{\partial^l}{\partial t^l} \frac{\partial^k}{\partial x^k} f_\Phi(x,t) \, = \, (-1)^k \int_{-\infty}^{+\infty} \frac{\Phi^\dagger (a+ib) (\Phi(a+ib))^k }{(a+ib)^{1-l}} e^{t(a+ib)-x\Phi(a+ib)} db,
		\label{intreprder}
	\end{equation}
	where the integral is absolutely convergent.
\end{prop}
\begin{proof}
	By using \eqref{eq:LT1} we compute the inverse Laplace transform as in \cite[Item a), Theorem 4.2.21]{abhn}. Precisely, we have, for all fixed $x>0$ and for almost all $t>0$, up to a subsequence,
	\begin{equation}
		f_\Phi(x,t) \, = \, \text{C-}\lim _{r \to +\infty} \frac{1}{2\pi i}\int_{a-ir}^{a+ir} e^{z t } 	\,  \frac{\Phi^\dagger(z)}{z} e^{-x\Phi(z)} \, dz,
		\label{ceslimitdag}
	\end{equation}
	where $x_0>0$ is arbitrary and we denote by C-$\lim$ the Cesaro limit, i.e.,
	\begin{equation*}
		\text{C-}\lim _{r \to +\infty} \int_{a-ir}^{a+ir} e^{z t } 	\,  \frac{\Phi^\dagger(z)}{z} e^{-x\Phi(z)} \, dz \, := \, \lim_{R \to +\infty} \frac{1}{R} \int_0^R \int_{a-ir}^{a+ir} e^{z t } 	\,  \frac{\Phi^\dagger(z)}{z} e^{-x\Phi(z)} \, dz \, dr.
	\end{equation*}
	In particular, it holds by \cite[Theorem 4.1.2]{abhn} that
	\begin{equation}
		\text{C-}\lim _{r \to +\infty} \int_{a-ir}^{a+ir} e^{z t } 	\,  \frac{\Phi^\dagger(z)}{z} e^{-x\Phi(z)} \, dz \, = \, \lim _{r \to +\infty} \int_{a-ir}^{a+ir} e^{zt } 	\,  \frac{\Phi^\dagger(z)}{z} e^{-x\Phi(z)} \, dz
		\label{cesequalpointdag}
	\end{equation}
	provided the limit on the right-hand side exists. We set about to prove the latter by noting that
	\begin{equation*}
		\int_{a-ir}^{a+ir} e^{z t } 	\,  \frac{\Phi^\dagger(z)}{z} e^{-x\Phi(z)} \, dz \, = \, i\int_{-r}^r e^{(a+ib)t} \frac{\Phi^\dagger (a+ib)}{a+ib} e^{-x\Phi(a+ib)} db.
	\end{equation*}
	However, by Item \eqref{it:asymp} of Lemma \ref{lem:Bern}, we know that
	%\begin{equation*}
	%\lim_{b \to \pm\infty} \frac{|\Phi^\dagger (a+ib)|}{|a+ib|} \, = %\, 0
	%\end{equation*}
	%and thus 
	the function $|\Phi^\dagger (a+ib)|/|a+ib|$ is continuous and bounded and then by employing \eqref{eq:Real1} for $n=0$ we get
	%. Then \eqref{intrepr} follows from
	\begin{equation*}
		\int_{-\infty}^{+\infty}\frac{|\Phi^\dagger(a+ib)|}{|a+ib|}\left|e^{t(a+ib)-x\Phi(a+ib)}\right|db 
		\le Ce^{ta}\int_{-\infty}^{+\infty}e^{-x\Re \Phi(a+ib)}db<\infty.
	\end{equation*}
	Using \eqref{cesequalpointdag} into \eqref{ceslimitdag} we get \eqref{intreprder} for $k,l=0$ and $(x,t) \in \mathbb{D}$ with $x>x_0$.
	Next, for any $k+l \le n$ and any $(x,t) \in \mathbb{D}$ with $x>x_0$, we prove that $\displaystyle \frac{\partial^k}{\partial x^k}\frac{\partial^l}{\partial t^l}f_\Phi(x,t)$ is well defined and given by \eqref{intreprder}. Let $t \in [t_1, t_2]$, $x \in [x_1, x_2]$. Without loss of generality we choose $x_1>x_0$ and $x_2 < t_1/\mathfrak{b}$. Then we have
	\begin{equation*}
		\begin{split}
			\left| \frac{\Phi^\dagger (a+ib) (\Phi(a+ib))^k }{(a+ib)^{1-l}} e^{t(a+ib)-x\Phi(a+ib)} \right|
			\leq \, \left| \frac{\Phi^\dagger (a+ib) (\Phi(a+ib))^k }{(a+ib)^{1-l}} \right| e^{at_2-x_1 \Re \Phi(a+ib)}
		\end{split}
	\end{equation*}
	since $\Re \Phi (a+ib)\geq 0$ by Item \eqref{it:sign} of Lemma \ref{lem:Bern}. Furthermore, recall by Item \eqref{it:asymp} of Lemma \ref{lem:Bern} that
	\begin{equation*}
		\lim_{b \to \pm \infty}\left| \frac{\Phi (a+ib)}{a+ib} \right| = \mathfrak{b}.
	\end{equation*}
	and that $\abs{\phi(z)}=\mathfrak{b}|z|\lbrb{1+\so{1}}$ uniformly on $\overline{\mathbb{H}_0}$. Hence
	\begin{equation*}
		\left| \frac{\Phi^\dagger (a+ib) (\Phi(a+ib))^k }{(a+ib)^{1-l}}\right| \leq C|a+ib|^{k+l} \leq C \lb a^n + {|b|^{n}}  \rb,
	\end{equation*}
	for some $C>0$. Hence, \eqref{intreprder} follows by noting that
	\begin{equation*}
		\begin{split}
			&\int_{-\infty}^{+\infty} \left| \frac{\Phi^\dagger (a+ib) (\Phi(a+ib))^k }{(a+ib)^{1-l}} \right| e^{at_2-x_1 \Re \Phi(a+ib)} \, db \leq C \int_{-\infty}^{+\infty}  (a^n+|b|^n) e^{at_2-x_1 \Re \Phi(a+ib)} \, db < \infty.
		\end{split}
	\end{equation*}
	Since $k,l \ge 0$ with $k+l \le n$ are arbitrary, the last inequality implies that we can differentiate under the integral $k$ times in $x$ and $l$ times in $t$ for $(x,t) \in \mathbb{D}$ with $x>x_0$, starting from \eqref{intreprder} for $k=l=0$.
\end{proof}
To prove Theorem \ref{thm:regularityfphi1}, it is clear that we have to show \eqref{eq:Real1} for some $x_0=x_0(n,L)$ when assumption \eqref{def:condiA} holds. To do this, we need a preliminary result.
\begin{prop}\label{prop:D0}
	Let $\Phi$ be the Laplace exponent of a potentially killed subordinator  and  \eqref{def:condiA} holds for some $L>0$. Then, $\phi(\infty)=\infty$ and we have, for any $M \in (0,L)$,
	\begin{equation}\label{eq:P}
		\begin{split}
			\liminfi{x}\frac{-x^2\Phi''(x)}{\ln(x)}>Me^{-1},\qquad \liminfi{x}\frac{x\lbrb{\Phi'(x)-\mathfrak{b}}}{\ln(x)}>Me^{-1}.
		\end{split}
	\end{equation} 
\end{prop}
\begin{proof}
	We note from \eqref{eq:phi''} that
	$
			-x^2\Phi''(x)\geq  x^2e^{-1}\Delta(x)
		$
	and the first claim of \eqref{eq:P} is valid. The second follows by integration of the first inequality and $\Phi(\infty)=\infty$ is a consequence of the integration of the second expression in \eqref{eq:P}.
\end{proof}
Now we are ready to show the smoothness of $f_\Phi$ under Assumption \eqref{def:condiA}.
%\begin{prop}\label{prop:D1}
%	Let $\Phi$ be the Laplace exponent of a potentially killed subordinator and assume that \eqref{def:condiA} holds. Then, for any $n\geq 0$ there exists $x_0(n,L)>0$ such that for any $a,t>0$ and any $x>x_0(n,L)$ condition \eqref{eq:Real2} holds. If $L=\infty$ then $x_0(n,\infty)=0$.
%\end{prop}
\begin{proof}[Proof of Theorem \ref{thm:regularityfphi1}]
	First observe that integrability in \eqref{eq:Real1} needs to be established only in  neighbourhood of infinity. We have from the inequality $1-\cos(y)\geq cy^2, y\in\lbbrbb{0,1}, c>0,$ that
	\[\Re\lbrb{\Phi(\ab)}-\Phi(a)=\IntOI \lbrb{1-\cos(by)}e^{-ay}\nu_\Phi(dy)\geq cb^2e^{-\frac{a}b} \int_{0}^{\frac1b}y^2\nu_\Phi(dy)=cb^2e^{-\frac{a}b}\Delta(b).\]
	Clearly, from Assumption \eqref{def:condiA}, for any $M \in (0,L)$, we have, for all $|b|>|b_0|>1$, that
	\begin{equation*}
		\begin{split}
			\int_{|b|>|b_0|} \!\! |b|^{n}e^{-x\Re\lbrb{\Phi(\ab)}}db\leq  e^{-x\Phi(a)}\!\!\!\int_{|b|>|b_0|}\!\!|b|^{n} e^{-xce^{-\frac{a}{|b_0|}}b^2\Delta(b)}db \leq e^{-x\Phi(a)}\!\!\!\int_{|b|>|b_0|} \!\! |b|^{n} e^{-xce^{-\frac{a}{|b_0|}}M\ln|b|}db.
		\end{split}
	\end{equation*}
	The latter is finite for $xce^{-\frac{a}{|b_0|}}M>n+1$. Since $|b_0|$ and $M \in (0,L)$ are arbitrary we have integrability of \eqref{eq:Real1} for $x>\frac{n+1}{cL}=:x_0(n,L)$ with $1/\infty=0$. Theorem \ref{thm:regularityfphi1} then follows by Proposition \ref{prop:LTthetapi}.
\end{proof}

\subsection{Proof of Theorem \ref{thm:mainL}}
We start with a preliminary result.
\begin{prop}\label{prop:D}
	Let $\Phi$ be the Laplace exponent of a potentially killed subordinator  and  \eqref{def:condiA} holds for some $L>0$. Let $t(x)$ be such that $t(x)/x \downarrow \mathfrak{b}, t(x)/x<\phi'(0^+)$ and $a_*(x)=(\phi')^{-1}\lbrb{\frac{t(x)}{x}}\in\lbrb{0,\infty}$. Then $\lim_{x \to \infty}a_*(x)=\infty$ and for any fixed $M>0$ it holds for all $x$ large enough
	\begin{equation}\label{eq:ineq}
		\begin{split}
			&a_*(x)>\frac{Me^{-1}}{\frac{t(x)}{x}-\mathfrak{b}}.
		\end{split}
	\end{equation}
\end{prop}
\begin{proof}
	Note that, since $t(x)/x \downarrow \mathfrak{b}$ and $\Phi'$ is decreasing with $\lim_{z \to \infty}\Phi'(z)=\mathfrak{b}$, then $\lim_{x \to \infty}a_*(x)=\infty$. Furthermore, for any $0<C<L$, using the second inequality of \eqref{eq:P} in Proposition \ref{prop:D0}, we get
		\[Ce^{-1}\leq\liminfi{x}\frac{a_*(x)\lbrb{\Phi'(a_*(x))-\mathfrak{b}}}{\ln a_*(x)}=\liminfi{x}\frac{a_*(x)\lbrb{\frac{t(x)}{x}-\mathfrak{b}}}{\ln a_*(x)}.\]
		This shows that for all $x$ and therefore $a_*(x)$ large enough
		\[\frac{a_*(x)}{\ln a_*(x)}>\frac{C}{2}\frac{e^{-1}}{\frac{t(x)}{x}-\mathfrak{b}}.\]
		Since $\limi{x}\ln a_*(x)=\infty$ this concludes the proof.
\end{proof}
Now, we are ready to prove the main theorem concerning the asymptotic behaviour.
\begin{proof}[Proof of Theorem \ref{thm:mainL}]
	Fix $k,l\geq 0$ and assume that $x>x_{0}(k+l,L)$, see Theorem \ref{thm:regularityfphi1}. Then, for any $a>0$, by \eqref{intreprder1} it holds, for any $t/x>\mathfrak{b}$, that
	\begin{equation}\label{def:derf}
		\frac{\partial^k\partial ^l }{\partial x^k\partial t^l}\fP{x,t}=\frac{(-1)^k}{2\pi}\int_{-\infty}^\infty \frac{\Phi^{\dagger}(\ab)\Phi^{k}(\ab)}{\lbrb{\ab}^{1-l}}e^{-x\Phi(\ab)+t\lbrb{\ab}}db=:I(x,t).
	\end{equation}
	%where we have simply differentiated the inversion of the Laplace transform in \eqref{eq:LT1} which is possible thanks to the absolute integrability of the expression in \eqref{intreprder1}.
	Let $t(x)$ be as in the statement of the theorem and $a_*(x)=(\phi')^{-1}\lbrb{\frac{t(x)}{x}}$, which is well-defined
	%be the solution to
	%\begin{equation*}
	%	\Phi'(a_*(x))=\frac{t(x)}{x}\in\lbrb{\mathfrak{b},\Phi'(0^+)},
	%\end{equation*}
	%which is unique 
	because $\Phi'(x)$ is decreasing with $\limi{x}\Phi'(x)=\mathfrak{b}$, see Item \eqref{it:phi'} of Lemma \ref{lem:Bern}. Since by assumption $t(x)/x\downarrow\mathfrak{b}$, we get from Proposition \ref{prop:D} that $\limi{x}a_*(x)\to\infty$.  Recall that in this theorem we have set
	$\mathbb{D}'=\curly{(t,x):x\mathfrak{b}<t\leq t(x)<x\phi'(0^+)}$, see \eqref{def:region}, and
	\begin{equation}\label{eq:c}
		c:=c(t,x)=(\phi')^{-1}\lbrb{\frac{t}{x}}, \ \mbox{ for }(t,x) \in \mathbb{D}'.
	\end{equation}
	From now on we work with $(t,x)\in \mathbb{D}'$ such that  $x>x_0(k+l,L)$.
	Since $(\phi')^{-1}$ is decreasing, for fixed $x$, $c(\cdot,x)$ is decreasing in $t$ on $\mathbb{D}'$ with $c(t(x),x)=a_*(x)$. For 
	$(t,x)\in\mathbb{D}'$ use \eqref{def:derf} with $a=c$ to get
	\begin{equation}\label{eq:f1}
		I(x,t)=\frac{(-1)^k}{2\pi}e^{ct-x\Phi(c)}\IntII\frac{\Phi^{\dagger}(c+ib)\Phi^{k}(c+ib)}{\lbrb{c+ib}^{1-l}}e^{ibt -x\lbrb{\Phi(c+ib)-\Phi(c}}db,
	\end{equation}
	where $c$ minimizes  $a \in (0,+\infty)\mapsto (at-x\Phi(a)) \in \R$.
	% Next, choose $g(c,x):=\sqrt{2\ln\lbrb{c\sqrt{-\phi''(c)x}}}$ but keep $g$ wherever it is more convenient and set
	Next, set 
	%\begin{equation}\label{eq:varesp}
	%	\begin{split}
		%		&\varepsilon:=\varepsilon(c,x):=\frac{g(c,x)}{ c\sqrt{-\Phi''(c)x}}.
		%	\end{split}
	%\end{equation} 
	\begin{equation}\label{eq:varesp}
		\begin{split}
			g(c,x):=\sqrt{2\ln\lbrb{c\sqrt{-\phi''(c)x}}} \qquad \mbox{ and } \qquad \varepsilon:=\varepsilon(c,x):=\frac{g(c,x)}{ c\sqrt{-\Phi''(c)x}}.
		\end{split}
	\end{equation} 
	We split the region of integration by setting  $\Ic_{\varepsilon}:=[-c\varepsilon,c\varepsilon]$. Put 
	\begin{equation}\label{def:Ie}
		I_\varepsilon(x,t):=\frac{(-1)^k}{2\pi}e^{ct-x\Phi(c)}\int_{\Ic_{\varepsilon}}	J(c,b)e^{ibt -x\lbrb{\Phi(c+ib)-\Phi(c)}}db,
	\end{equation}
	where
	\begin{equation}\label{def:Je1}
		\begin{split}
			J(c,b)&:=\frac{\Phi^{\dagger}(c+ib)\Phi^{k}(c+ib)}{\lbrb{c+ib}^{1-l}}.
		\end{split}
	\end{equation}
Using Taylor's formula and the definition of $c$ in \eqref{eq:c} we get
	\begin{equation}\label{def:Tay}
		\begin{split}
			&x(\Phi(c+ib)-\Phi(c))=ibt-x\frac{b^2}{2}\Phi''(c)-U(c,b),
		\end{split}
	\end{equation}
	where
	\begin{equation*}
		\begin{split}
			U(c,b):=&ibt -x\lbrb{\Phi(c+ib)-\Phi(c)}-x\frac{b^2}{2}\Phi''(c)=ix\int_0^b\int_0^v\int_0^w \phi'''(c+i\rho)d\rho dw dv.
		\end{split}
	\end{equation*}
	Then, recalling \eqref{eq:varesp}, we have  for the integral in \eqref{def:Ie}
	\begin{equation}\label{def:I'}
		\begin{split}
			&\int_{\Ic_{\varepsilon}}J(c,b)e^{ibt -x\lbrb{\Phi(c+ib)-\Phi(c)}}db=\int_{\Ic_{\varepsilon}}J(c,b)e^{\frac{b^2}{2}x\Phi''(c)\lbrb{1+2\frac{U(c,b)}{xb^2\Phi''(c)}}}db\\&=\!\frac{1}{\sqrt{-\Phi''(c)x}}\int_{-g(c,x)}^{g(c,x)}J\lbrb{c,\frac{u}{\sqrt{-\Phi''(c)x}}}\!e^{-\frac{u^2}{2}\lbrb{1-2\frac{U\lbrb{c,u/\sqrt{-\Phi''(c)x}}}{u^2}}}\!\!du,
		\end{split}	
	\end{equation}
	where we have used that $\Phi''(y)<0$, for $y>0$, see Item \eqref{it:phi'} of Lemma \ref{lem:Bern}. Here, we need \eqref{def:condiB}, i.e. $\limsupi{y}y\Phi'''(y)/(-\Phi''(y))=K<\infty$, which together with $\abs{\Phi'''(z)}\leq \Phi'''(\Re(z))$, for $\Re(z)>0$, see Item \eqref{it:real} in Lemma \ref{lem:Bern}, and the definition of $U(c,b)$ yields  that for all $x$ large enough 
	\begin{equation}\label{eq:estU}
		\begin{split}
			\bar{V}(c,x)&:=\sup_{ |u|\leq g(c,x)}V(c,x,u):=\sup_{ |u|\leq g(c,x)}\frac{2}{u^2}\abs{U\lbrb{c,\frac{u}{\sqrt{-\Phi''(c)x}}}}\\
			&\leq  \sup_{ |u|\leq g(c,x)}\frac{x|u|}{3x^{\frac32}\lbrb{-\Phi''(c)}^{\frac32}}\Phi'''(c)  \leq\frac{2K}{3}\frac{g(c,x) }{c\sqrt{-\Phi''(c)x}}=\frac{2K}{3}\varepsilon(c,x).
		\end{split}
	\end{equation}
	Next, setting 
	\begin{equation}\label{def:tJ}
		\begin{split}
			&\tilde{J}(c,u)=\frac{J(c,u)}{J(c,0)}-1,
		\end{split}
	\end{equation}
	we get from \eqref{def:I'} that
	\begin{equation}\label{def:I'1}
		\begin{split}
			&\int_{\Ic_{\varepsilon}}J(c,b)e^{ibt -x\lbrb{\Phi(c+ib)-\Phi(c)}}db	=\frac{J(c,0)}{\sqrt{-\Phi''(c)x}}\int_{-g(c,x)}^{g(c,x)}e^{-\frac{u^2}{2}\lbrb{1+V(c,x,u)}}du\\
			&+\frac{J(c,0)}{\sqrt{-\Phi''(c)x}}\int_{-g(c,x)}^{g(c,x)}\tilde{J}\lbrb{c,\frac{u}{\sqrt{-\Phi''(c)x}}}e^{-\frac{u^2}{2}\lbrb{1+V(c,x,u)}}du=:H_1(c,x)+H_2(c,x).
		\end{split}
	\end{equation}
	Clearly, from \eqref{def:Je1} and \eqref{eq:estU},
	\begin{equation}\label{eq:H1}
		\begin{split}
			&\abs{H_1(c,x)-\sqrt{2\pi}\frac{J(c,0)}{\sqrt{-\Phi''(c)x}}}=\frac{J(c,0)}{\sqrt{-\Phi''(c)}x}\abs{\int_{-g(c,x)}^{g(c,x)}e^{-\frac{u^2}{2}\lbrb{1+V(c,x,u)}}du-\int_{-\infty}^{\infty}e^{-\frac{u^2}{2}}du}\\
			&\leq \sqrt{2\pi}\frac{J(c,0)}{\sqrt{-\Phi''(c)x}}\lbrb{e^{\bar{V}(c,x)}-1+\frac{1}{\sqrt{2\pi}}\int_{|u|>g(c,x)}e^{-\frac{u^2}{2}}du}.
		\end{split}
	\end{equation}
	From the definition of $J$, see \eqref{def:Je1}, we easily get that
	\begin{equation}\label{def:tJ1}
		\begin{split}
			\abs{\frac{d}{du}\log J\lbrb{c,u}}	&\leq \abs{\frac{(\Phi^{\dagger})'(c+iu)}{\Phi^{\dagger}(c+iu)}}+k\abs{\frac{\Phi'(c+iu)}{\Phi(c+iu)}}+\abs{l-1}\abs{\frac{1}{c+iu}}\leq \frac{k+1+\abs{l-1}}{c},
		\end{split}
	\end{equation}
	where we have used that for any Bernstein function $\phi$ and $c>0,u\in\Rb$,
	\[\abs{\frac{\phi'(c+iu)}{\phi(c+iu)}}\leq \frac1c,\]
	which in turn follows from the chain of inequalities
	\[\abs{\phi'(c+iu)}\leq \phi'(c)\leq \frac{\phi(c)}{c}\leq \frac{\Re\phi(c+iu)}{c}\leq \abs{\frac{\phi(c+iu)}{c}}\]
	that come from subsequent application of Items \ref{it:real}, \ref{it:ineq} and \ref{it:sign} of Lemma \ref{lem:Bern}. From \eqref{def:tJ1} we get that
	\begin{equation}\label{aim}
		\begin{split}
			&\abs{\ln\abs{1+\tilde{J}(c,u)}}\leq \abs{\log\lbrb{1+\tilde{J}(c,u)}}=\abs{\log \lbrb{1+\lbrb{\frac{J\lbrb{c,u}}{J(c,0)}-1}}}\leq (k+1+\abs{l-1})\frac{|u|}{c}.
		\end{split}
	\end{equation}
	From \eqref{aim} with \eqref{eq:varesp},
	\begin{equation}\label{aim2}
		\begin{split}
			\sup_{|u|\leq \frac{g(c,x)}{\sqrt{-\Phi''(c)x}}}\abs{\ln\abs{1+\tilde{J}(c,u)}}\leq (k+1+\abs{l-1})\varepsilon(c,x).
		\end{split}
	\end{equation}
	However, $c\geq a_*(x)$, $\limi{x}a_*(x)=\infty$ and $\limi{x}x\sqrt{-\Phi''(x)}=\infty$, see \eqref{eq:P}, imply that 
	\begin{equation}\label{eq:bare}
		\limi{x}\bar{\varepsilon}(x):=\limi{x}\sup_{c\geq a_*(x)}\varepsilon(c,x)=0.
	\end{equation}
	Thus, for all $x$ large enough, \eqref{aim2} yields that, for some $C=C(k,l)$,
	\begin{equation}\label{eq:aim1}
		\sup_{|u|\leq \frac{g(c,x)}{\sqrt{-\Phi''(c)x}}}\abs{\tilde{J}(c,u)}\leq C\varepsilon(c,x).
	\end{equation}
	Hence, from \eqref{eq:estU}, \eqref{def:tJ} and \eqref{def:I'1} for all $x$ large enough
	\begin{equation}\label{estimateU}
		\begin{split}
			H_2(c,x)&\leq e^{\bar{V}(c,x)}\frac{J(c,0)}{\sqrt{-\Phi''(c)x}}\int_{-g(c,x)}^{g(c,x)}\abs{\tilde{J}\lbrb{c,\frac{u}{\sqrt{-\Phi''(c)x}}}}e^{-\frac{u^2}{2}}du\\
			&\leq \sqrt{2\pi}Ce^{\bar{V}(c,x)}\frac{J(c,0)}{\sqrt{-\Phi''(c)x}}\varepsilon(c,x).
		\end{split}
	\end{equation}
	Combining \eqref{eq:estU}, \eqref{def:I'1}, \eqref{eq:H1} and \eqref{estimateU} we obtain for all large $x$
	\begin{equation}\label{eq:H2}
		\begin{split}
			&\abs{H_1(c,x)-\sqrt{2\pi}\frac{J(c,0)}{\sqrt{-\Phi''(c)x}}+H_2(c,x)}\\
			&\leq \sqrt{2\pi}\frac{J(c,0)}{\sqrt{-\Phi''(c)x}}\lbrb{Ce^{\bar{V}(c,x)}\varepsilon(c,x)+e^{\bar{V}(c,x)}-1+\frac{1}{\sqrt{2\pi}}\int_{|u|>g(c,x)}e^{-\frac{u^2}{2}}du}\\
			&\leq \sqrt{2\pi}\frac{J(c,0)}{\sqrt{-\Phi''(c)x}}\lbrb{Ce^{\frac{2K}{3}\varepsilon(c,x)}\varepsilon(c,x)+e^{\frac{2K}{3}\varepsilon(c,x)}-1+\frac{1}{\sqrt{2\pi}}\int_{|u|>g(c,x)}e^{-\frac{u^2}{2}}du}.
		\end{split}
	\end{equation}
	Applying \eqref{eq:H2} and \eqref{eq:bare}
	in \eqref{def:I'1} we get with some $C'=C(k,l,K)>0$ that for all $x$ large enough
	\begin{equation}\label{eq:H3}
		\begin{split}
			&\sup_{c\geq a_*(x)}\abs{\frac{1}{\sqrt{2\pi}}\frac{\sqrt{-\Phi''(c)x}}{J(c,0)}\int_{\Ic_{\varepsilon}}J(c,b)e^{ibt -x\lbrb{\Phi(c+ib)-\Phi(c)}}db-1}\\
			&\leq C'\bar{\varepsilon}(x)+\frac{1}{\sqrt{2\pi}}\sup_{c\geq a_*(x)}\int_{|u|>g(c,x)}e^{-\frac{u^2}{2}}du.
		\end{split}
	\end{equation}
	Plugging this in \eqref{def:Ie} we get for all $x$ large enough
	\begin{equation}\label{def:asympIe1}
		\begin{split}		
			&\sup_{\mathfrak{b}x<t\leq t(x)}\abs{\frac{(-1)^k}{\sqrt{2\pi}}\frac{\sqrt{-\Phi''(c)x}}{J(c,0)}e^{-ct+x\phi(c)}I_\varepsilon(x,t)-1}\leq C'\bar{\varepsilon}(x)+\frac{1}{\sqrt{2\pi}}\sup_{c\geq a_*(x)}\int_{|u|>g(c,x)}e^{-\frac{u^2}{2}}du.
		\end{split}
	\end{equation}
	Using \eqref{def:Je1} we proceed to investigate
	\begin{equation}\label{def:Je}
		\begin{split}
			J_\varepsilon(x,t)&:=\frac{(-1)^k}{2\pi}e^{ct-x\Phi(c)}\int_{\Ic^c_{\varepsilon}}\frac{\Phi^\dagger(c+ib)\Phi^{k}(c+ib)}{\lbrb{c+ib}^{1-l}}e^{ibt -x\lbrb{\Phi(c+ib)-\Phi(c)}}db\\
			&=\frac{(-1)^k}{2\pi}e^{ct-x\Phi(c)}\int_{\varepsilon c\leq |b|\leq dc}\!\!\!\!\!\!\!\!J(c,b)e^{ibt -x\lbrb{\Phi(c+ib)-\Phi(c)}}db\\
			&+\frac{(-1)^k}{2\pi}e^{ct-x\Phi(c)}\int_{|b|\geq dc}J(c,b)e^{ibt -x\lbrb{\Phi(c+ib)-\Phi(c)}}db\\
			&=:\frac{(-1)^k}{2\pi}e^{ct-x\Phi(c)}\lbrb{J_1(c,x)+J_2(c,x)},
		\end{split}
	\end{equation}
	where $d=K^{-1}$. First we estimate  $J_1(c,x)$. We use the Taylor expansion \eqref{def:Tay} to the exponent in  $J_1(c,x)$, $\abs{\Im\Phi'''(z)}\leq \abs{\Phi'''(z)}\leq \Phi'''\lbrb{\Re(z)}$, $\limsupi{y}y\Phi'''(y)/(-\Phi''(y))=K<\infty$ to get after a change of variables $b\to cb$, for all $x$ and therefore $c\geq a_*(x)$ large enough
	\begin{equation*}
		\begin{split}
			\abs{J_1(c,x)}&\leq 2c\int_{\varepsilon}^d\abs{J(c,cb)}e^{\frac{xb^2c^2}{2}\Phi''(c)+\frac{xb^3c^3}{6}\Phi'''(c)}db\leq 2c\int_{\varepsilon}^d\abs{J(c,cb)}e^{\frac{xb^2c^2}{2}\Phi''(c)\lbrb{1-\frac{2Kd}{3}}}db\\
			&= 2c^l\int_{\varepsilon}^d\frac{\abs{\Phi^\dagger(c(1+ib))\Phi^{k}\lbrb{c(1+ib)}}}{\abs{1+ib}^{1-l}}e^{\frac{xb^2c^2}{6}\Phi''(c)}db.
		\end{split}
	\end{equation*}
	To carry on further we note from \eqref{eq:DeltaR} that 
	$
	\abs{\frac{\Phi\lbrb{c\lbrb{1+ib}}}{\Phi(c)}}\leq 3\max\curly{1,b^2}.
	$
	Then, we have with the form of $\varepsilon$, see \eqref{eq:varesp}, and $d=K^{-1}$ that 
	\begin{equation}\label{eq:J1}
		\begin{split}
			&\abs{\frac{\sqrt{-\Phi''(c)x}}{J(c,0)}J_1(c,x)}\leq 3^{k+2}c\sqrt{-\Phi''(c)x}\int_{\varepsilon}^d \max\curly{1,b^{2k+2+l}}e^{\frac{xb^2c^2}{6}\Phi''(c)}db\\
			&\leq 3^{k+2}\max\curly{1,d^{2k+2+l}}c\sqrt{-\Phi''(c)x}\int_{\varepsilon}^de^{\frac{xb^2c^2}{6}\Phi''(c)}db\\
			&\leq 3^{k+\frac52}\max\curly{1,K^{-2k-2-l}}\int_{3^{-\frac12}g(c,x)}^\infty e^{-\frac{u^2}{2}}du.
		\end{split}
	\end{equation}
	For $J_2(c,x)$ we change variables $b\to cb$ and use again $
	\abs{\frac{\Phi\lbrb{c\lbrb{1+ib}}}{\Phi(c)}}\leq 3\max\curly{1,b^2}
	$  to get
	\begin{equation*}
		\begin{split}
			&\abs{\frac{\sqrt{-\Phi''(c)x}}{J(c,0)}J_2(c,x)}\leq 3^{k+1}c\sqrt{-\Phi''(c)x}\int_{|b|\geq d}^\infty \frac{\max\curly{1,b^{2k+2}}}{\abs{1+ib}^{-l+1}}e^{ -x\lbrb{\Re\lbrb{\Phi(c\lbrb{1+ib})}-\Phi(c)}}db\\
			&\leq C_1c\sqrt{-\Phi''(c)x}\int_{d}^\infty b^{2k+1+l}e^{ -x\lbrb{\Re\lbrb{\Phi(c\lbrb{1+ib})}-\Phi(c)}}db\\
			&=C_1d^{2k+2+l}c\sqrt{-\Phi''(c)x}\int_{1}^\infty b^{2k+1+l}e^{ -x\lbrb{\Re\lbrb{\Phi(c\lbrb{1+ibd})}-\Phi(c)}}db,
		\end{split}
	\end{equation*}
	where $C_1>0$ is some constant.   
	Then, with some absolute constant $c_0>0$, we have that 
	\begin{equation}\label{eq:Delta}
		\begin{split}
			& \Re\lbrb{\Phi(c\lbrb{1+ibd})}-\Phi(c)=\IntOI \lbrb{1-\cos\lbrb{bdcy}}e^{-cy}\mu(dy)\\
			&\geq c_0^2e^{-\frac{1}{bd}}b^2d^2c^2\int_{0}^{\frac{1}{bdc}}y^{2}\mu(dy)=c_0^2e^{-\frac{1}{bd}}b^2d^2c^2\Delta(bdc).
		\end{split}
	\end{equation}
	Since $\liminfi{y}y^2\Delta(y)/\ln(y)= L>0$ we choose $M<L$. Set $M'=Mc^2_0e^{-\frac{1}{d}}$. On $b\geq 1$ we get that for $x$ and $c\geq a_*(x)$ large enough such that $M'x>2k+2+l$
	\begin{equation}\label{eq:J_2}
		\begin{split}
			&\abs{\frac{\sqrt{-\Phi''(c)x}}{J(c,0)}J_2(c,x)}\leq C_1d^{2k+2+l}c\sqrt{-\Phi''(c)x}\int_{1}^\infty b^{2k+1+l}e^{ -M'x \ln\lbrb{bdc}}db\\
			&=\frac{C_1d^{2k+2+l}}{M'x-2k-2-l}\frac{c\sqrt{-\Phi''(c)x}}{(cd)^{M'x}}\leq C_2\frac{K^{M'x-2k-2-l}}{(M'x-2k-2-l)}\sqrt{x} c^{\frac12-M'x},
		\end{split}
	\end{equation}
	where $C_2>0$ is some constant, we have fixed $d=K^{-1}$ and we have used $-y^{2}\Phi''(y)\leq 2\Phi(y)=\bo{y}$, see Item \eqref{it:ineq} of Lemma \ref{lem:Bern}. Collecting \eqref{def:asympIe1} and employing \eqref{eq:J1} and \eqref{eq:J_2} in \eqref{def:Je}, we get that, since $c\geq a_*(x)$, \eqref{eq:f1} has the following  form for all $x$ large enough
	\begin{equation*}
		\begin{split}
			&	\sup_{\mathfrak{b}x<t\leq t(x)}\abs{(-1)^k\sqrt{2\pi}\frac{\sqrt{-\Phi''(c)x}}{J(c,0)}e^{-ct+x\phi(c)}I(x,t)-1}\\
			&\leq  C'\bar{\varepsilon}(x)+\frac{1}{\sqrt{2\pi}}\sup_{c\geq a_*(x)}\int_{|u|>g(c,x)}e^{-\frac{u^2}{2}}du\\
			&+3^{k+\frac52}\max\curly{1,K^{-2k-2-l}}\sup_{c\geq a_*(x)}\int_{3^{-\frac12}g(c,x)}^\infty e^{-\frac{u^2}{2}}du+C_2\frac{K^{M'x-2k-2-l}}{(M'x-2k-2-l)}\sqrt{x} (a_*)^{\frac12-M'x}(x).
		\end{split}
	\end{equation*}
	Next, recall that $g(c,x)=\sqrt{2\ln\lbrb{c\sqrt{-\Phi''(c)x}}}$ which converges to infinity thanks to Proposition \ref{prop:D} and the form of $\varepsilon(c,x),\bar{\varepsilon}(x)$, see \eqref{eq:varesp} and \eqref{eq:bare}. We then yield asymptotically
	\begin{equation*}
		\begin{split}
			&	\sup_{\mathfrak{b}x<t\leq t(x)}\abs{(-1)^k\sqrt{2\pi}\frac{\sqrt{-\Phi''(c)x}}{J(c,0)}e^{-ct+x\phi(c)}I(x,t)-1}\\
			&=\bo{\sup_{c\geq a_*(x)}\frac{\sqrt{\ln\lbrb{c\sqrt{-\Phi''(c)x}}}}{ c\sqrt{-\Phi''(c)x}}+\sup_{c\geq a_*(x)}\int_{\sqrt{\frac{2}3\ln\lbrb{c\sqrt{-\Phi''(c)x}}}}^\infty e^{-\frac{u^2}{2}}du+ \frac{K^{M'x}(a_*(x))^{\frac12-M'x}(x)}{\sqrt{x}}}.
		\end{split}
	\end{equation*}
	However, since $\limi{x}-x^2\phi''(x)=\infty$, see \eqref{eq:P}, and $\int_x^{\infty} e^{-u^2/2}du=\bo{x^{-1}e^{-x^2/2}}$ we further get
	\begin{equation*}
		\begin{split}
			&		\sup_{\mathfrak{b}x<t\leq t(x)}\abs{(-1)^k\sqrt{2\pi}\frac{\sqrt{-\Phi''(c)x}}{J(c,0)}e^{-ct+x\phi(c)}I(x,t)-1}\\
			&=\bo{\sup_{c\geq a_*(x)}\frac{\sqrt{\ln\lbrb{c\sqrt{-\Phi''(c)x}}}}{ c\sqrt{-\Phi''(c)x}}+ \frac{K^{M'x}(a_*(x))^{\frac12-M'x}(x)}{\sqrt{x}}}.
		\end{split}
	\end{equation*}
	Since $-y^2\Phi''(y)\leq 2\Phi(y)=\bo{y},$ see Item \eqref{it:ineq} of Lemma \ref{lem:Bern}, we check  that the first expression in the speed of convergence cannot be faster than $(xa_*(x))^{-\frac12}$. Therefore, we conclude that
	\begin{equation*}
		\begin{split}
			&	\sup_{\mathfrak{b}x<t\leq t(x)}\abs{(-1)^k\sqrt{2\pi}\frac{\sqrt{-\Phi''(c)x}}{J(c,0)}e^{-ct+x\phi(c)}I(x,t)-1}=\bo{\sup_{c\geq a_*(x)}\frac{\sqrt{\ln\lbrb{c\sqrt{-\Phi''(c)x}}}}{ c\sqrt{-\Phi''(c)x}}}.
		\end{split}
	\end{equation*}
	which establishes \eqref{asymp}. Finally, from the first  relation of \eqref{eq:P} of Proposition \ref{prop:D0} we have for all $x$ large enough that $c\sqrt{-\phi''(c)}\geq Me^{-1}\ln(c)\geq Me^{-1}\ln a_*(x)$, $M<L$,  which yields
	\[\inf_{c\geq a_*(x)}c\sqrt{-\phi''(c)}\geq Me^{-1}\ln(a_*(x)).\]
	Employing this and again  $-y^2\Phi''(y)\leq 2\Phi(y)=\bo{y},$ we arrive at 
	\begin{equation*}
		\begin{split}
			&		\sup_{\mathfrak{b}x<t\leq t(x)}\abs{(-1)^k\sqrt{2\pi}\frac{\sqrt{-\Phi''(c)x}}{J(c,0)}e^{-ct+x\phi(c)}I(x,t)-1}=\bo{\sqrt{\frac{\ln(x)}{x\ln a_*(x)}}}.
		\end{split}
	\end{equation*}
	Substituting in the latter the expression for $J(c,0)$, see \eqref{def:Je1}, concludes the proof of \eqref{asymp}. Relation $1/a_*=\so{t(x)/x-\mathfrak{b}}$ follows from Proposition \ref{prop:D}.
\end{proof}

\subsection{Proofs of Theorems \ref{thm:main1} and \ref{thm:main2}}
Once the proof of Theorem \ref{thm:mainL} has been established, similar arguments will lead to Theorems \ref{thm:main1} and \ref{thm:main2}. For this reason, we will be economical with the next proofs, as we will refer to the previous arguments while highlighting the necessary changes and adaptations.
%We will be economical with the next proof as it follows the main steps of the proof of Theorem \ref{thm:mainL}.
We proceed with the proof of the next main theorem.
\begin{proof}[Proof of Theorem \ref{thm:main1}]
	We follow closely the proof of Theorem \ref{thm:mainL} using in particular $J(c,b)$ defined in \eqref{def:Je1}. By assumption $(t,x)\in\mathbb{D}^\prime=\{(t,x): \ xt_1 \leq t \leq xt_2\}$ and since $(\phi')^{-1}$ is decreasing then \[c:=c(t,x)=(\phi')^{-1}(t/x)\in \lbbrbb{(\phi')^{-1}(t_2), (\phi')^{-1}(t_1)}=:\mathbb{V}.\] From Theorem \ref{thm:regularityfphi1} we can write for all $x$ large enough and $t/x\in\lbbrbb{t_1,t_2}$
	\begin{equation*}
		\begin{split}
			&\frac{\partial^k\partial ^l }{\partial x^k\partial t^l}\fP{x,t}=\frac{(-1)^k}{2\pi}\int_{-\infty}^\infty J(c,b)e^{-x\Phi(c+ib)+t\lbrb{c+ib}}db=:I(x,t).
		\end{split}
	\end{equation*}
	We use $g(c,x):=\sqrt{2\ln\lbrb{c\sqrt{-\phi''(c)x}}}$ as defined in \eqref{eq:varesp}. Since $c\in \mathbb{V}$ we can repeat the arguments leading up to \eqref{def:I'} with the same definition of $\varepsilon(c):=\varepsilon(c,x)=\frac{g(c,x)}{ c\sqrt{-\Phi''(c)x}}$ and the estimate 
	\begin{equation}\label{eq:estU_1}
		\begin{split}
			\bar{V}(c,x)&:=\sup_{ |u|\leq g(c,x)}V(c,x,u):=\sup_{ |u|\leq g(c,x)}\frac{2}{u^2}\abs{U\lbrb{c,\frac{u}{\sqrt{-\Phi''(c)x}}}}\leq \frac{2C}3\varepsilon(c,x),
		\end{split}
	\end{equation}
	where $C=C(t_1,t_2)$ since in the upper bound prior to \eqref{eq:varesp} we can employ
	\begin{align*}
		\phi'''(c)\leq \phi'''((\phi')^{-1}(t_2)), \quad  c\leq (\phi')^{-1}(t_1), \quad \mbox{ and } \quad  -\phi''(c)\geq (\phi')^{-1}(t_1),
	\end{align*}
	hence we do not need Assumption \ref{def:condiB}.
	%here we estimate $\phi'''(c)\leq \phi'''((\phi')^{-1}(t_2)), c\leq (\phi')^{-1}(t_1), -\phi''(c)\geq (\phi')^{-1}(t_1)$ in the upper bound prior to \eqref{eq:varesp}.
	%This is why we do not need Assumption \ref{def:condiB}. 
	Since \eqref{eq:bare} is valid with the modification
	\begin{equation}\label{eq:bare_1}
		\limi{x}\bar{\varepsilon}(x):=\limi{x}\sup_{c\in\mathbb{V}}\varepsilon(c,x)=0
	\end{equation}
	we similarly arrive at 
	\begin{equation}\label{def:asympIe1_1}
		\begin{split}		
			&\sup_{xt_1 \le t \le xt_2}\abs{(-1)^k\sqrt{2\pi}\frac{\sqrt{-\Phi''(c)x}}{J(c,0)}e^{-ct+x\phi(c)}I_\varepsilon(x,t)-1}\leq C'\bar{\varepsilon}(x)+\frac{1}{\sqrt{2\pi}}\sup_{c\in\mathbb{V}}\int_{|u|>g(c,x)}e^{-\frac{u^2}{2}}du.
		\end{split}
	\end{equation}	
	The same remainder as \eqref{def:Je}, i.e.
	\begin{equation}\label{def:Je_1}
		\begin{split}
			J_\varepsilon(x,t)&:=\frac{(-1)^k}{2\pi}e^{ct-x\Phi(c)}\int_{\Ic^c_{\varepsilon}}J(c,b)e^{ibt -x\lbrb{\Phi(c+ib)-\Phi(c)}}db\\
			&=\frac{(-1)^k}{2\pi}e^{ct-x\Phi(c)}\int_{\varepsilon(c)c\leq |b|\leq dc}J(c,b)e^{ibt -x\lbrb{\Phi(c+ib)-\Phi(c)}}db\\
			&+\frac{(-1)^k}{2\pi}e^{ct-x\Phi(c)}\int_{|b|\geq dc}J(c,b)e^{ibt -x\lbrb{\Phi(c+ib)-\Phi(c)}}db\\
			&=:\frac{(-1)^k}{2\pi}e^{ct-x\Phi(c)}\lbrb{J_1(c,x)+J_2(c,x)},
		\end{split}
	\end{equation}
	is studied similarly with $d=C^{-1}$, see \eqref{eq:estU_1}. First, following the same computations one gets
	\begin{equation}\label{eq:J1_1}
		\begin{split}
			&\abs{\frac{\sqrt{-\Phi''(c)x}}{J(c,0)}J_1(c,x)}\leq 3^{k+\frac52}\max\curly{1,C^{-2k-2-l}}\int_{3^{-\frac12}g(c,x)}^\infty e^{-\frac{u^2}{2}}du.
		\end{split}
	\end{equation}
	For the second term we get
	\begin{equation*}
		\begin{split}
			&\abs{\frac{\sqrt{-\Phi''(c)x}}{J(c,0)}J_2(c,x)}\leq C_1C^{-2k-2-l}c\sqrt{-\Phi''(c)x}\int_{1}^\infty b^{2k+1+l}e^{ -x\lbrb{\Re\lbrb{\Phi(c\lbrb{1+ibd})}-\Phi(c)}}db,
		\end{split}
	\end{equation*}
	with  as in \eqref{eq:Delta} 
	\begin{equation}\label{eq:Delta_1}
		\begin{split}
			& \Re\lbrb{\Phi(c\lbrb{1+ibd})}-\Phi(c)\geq c_0^2e^{-\frac{1}{bd}}b^2d^2c^2\Delta(bdc)\geq Ab^2.
		\end{split}
	\end{equation}
	for some  $A>0$ because $bdc\geq (\phi')^{-1}(t_2)d>0$. As in \eqref{eq:J_2} for $x$ large enough
	\begin{equation*}
		\begin{split}
			&\abs{\frac{\sqrt{-\Phi''(c)x}}{J(c,0)}J_2(c,x)}\leq C_1C^{-2k-2-l}c\sqrt{-\Phi''(c)x}e^{-\frac{A}2 x}\leq C_1C^{-2k-2-l}\sqrt{2cx}e^{-\frac{A}2 x} .
		\end{split}
	\end{equation*}
	Collecting the estimates above and noting that, as $c$ ranges in the bounded set $\mathbb{V}$, $\bar{\varepsilon}(x)\asymp \sqrt{\ln(x)}x^{-\frac12}$, see \eqref{eq:varesp} and \eqref{eq:bare}, we deduce
	\begin{equation*}
		\begin{split}
			&	\sup_{xt_1 \le t \le xt_2}\abs{(-1)^k\sqrt{2\pi}\frac{\sqrt{-\Phi''(c)x}}{J(c,0)}e^{-ct+x\phi(c)}I(x,t)-1}=\bo{\sqrt{\frac{\ln(x)}{x}}}.
		\end{split}
	\end{equation*}
	This concludes the proof of the theorem.
\end{proof}
%We will be economical with the next proof as it follows the main steps of the proof of Theorem \ref{thm:mainL}.
Finally, we only need to provide the proof of Theorem \ref{thm:main2}.
\begin{proof}[Proof of Theorem \ref{thm:main2}] Since $t/x\to \phi'(0^+)$ then from \eqref{eq:a*} we have that $\limi{x}a_*=0$, where $a_*:=a_*(x)$.
	Then, using \eqref{intreprder1} for $x>x_0(k+l,L)$ we write,  for any $t/x>\mathfrak{b}$,
	\begin{equation}\label{def:derf1}
		\begin{split}
			\frac{\partial^k\partial ^l }{\partial x^k\partial t^l}\fP{x,t}&=\frac{(-1)^k}{2\pi}e^{a_*t-x\Phi(a_*)}\lbrb{\int_{\Ic_{\varepsilon}}+\int_{\Ic^c_{\varepsilon}}}J(a_*,b)e^{ibt -x\lbrb{\Phi(a_*+ib)-\Phi(a_*)}}db,
		\end{split}
	\end{equation}
	where $\Ic_{\varepsilon}=\lbbrbb{-\varepsilon(a_*)a_*,\varepsilon(a_*)a_*}$, $J$ defined as in \eqref{def:Je1} and $\varepsilon(a_*)$ as in \eqref{eq:varesp}. The latter makes sense due to the first assumption in \eqref{eq:addCondi}, i.e. $\limi{x} -x\phi''(a_*)a^2_*=\infty$.
	
	We appeal to the proof of Theorem \ref{thm:mainL} with the minor difference that we do not work with a region for $t$ and respectively with $c\geq a_*$ but we use $t=t(x)$ and $a_*=a_*(x)$.
	Absolutely the same estimates hinging on \eqref{def:condiA}, \eqref{def:condiB'} and \eqref{eq:varesp}  and employing  $g(a_*,x)=\sqrt{2\ln\lbrb{a_*\sqrt{-\Phi''(a_*)x}}}$, which drifts to infinity thanks to the first assumption in \eqref{eq:addCondi}, lead  as in the proof of Theorem \ref{thm:mainL} to
	\begin{equation}\label{I'2'}
		\begin{split}
			&\int_{\Ic_{\varepsilon}}J(a_*,b)e^{ibt -x\lbrb{\Phi(a_*+ib)-\Phi(a_*)}}db=\sqrt{2\pi}\frac{J(a_*,x)}{\sqrt{-\Phi''(a_*)x}}\lbrb{1+\bo{\frac{g(a_*,x) }{a_*\sqrt{-\Phi''(a_*)x}}}}.
		\end{split}
	\end{equation}
	Next, we again decompose for some $d>0$ the remaining term on $\mathcal{I}^c_\varepsilon$ to yield
	\begin{equation}\label{def:Je2}
		\begin{split}
			J_\varepsilon(x,t)&:=\frac{(-1)^k}{2\pi}e^{a_*t-x\Phi(a_*)}\int_{\varepsilon(a_*)a_*\leq |b|\leq da_*}J(a_*,b)e^{ibt -x\lbrb{\Phi(a_*+ib)-\Phi(a_*)}}db\\
			&+\frac{(-1)^k}{2\pi}e^{a_*t-x\Phi(a_*)}\int_{|b|\geq da_*}J(a_*,b)e^{ibt -x\lbrb{\Phi(a_*+ib)-\Phi(a_*)}}db\\
			&=:\frac{(-1)^k}{2\pi}e^{a_*t-x\Phi(a_*)}\lbrb{J_1(a_*,x)+J_2(a_*,x)},
		\end{split}
	\end{equation}
	and get the same way $J_1(a_*,x)$ estimated as in \eqref{eq:J1}. Also, in the same fashion
	\begin{equation*}
		\begin{split}
			&\abs{\frac{\sqrt{-\Phi''(a_*)x}}{J(a_*,0)}J_2(a_*,x)}=C'd^{2k+2+l}a_*\sqrt{-\Phi''(a_*)x}\int_{1}^\infty b^{2k+1+l}e^{ -x\lbrb{\Re\lbrb{\Phi(a_*\lbrb{1+ibd})}-\Phi(a_*)}}db,
		\end{split}
	\end{equation*}
	where $C'>0$ is a constant. Here, however, there is a difference in that $a_*\to 0$, hence we  estimate the exponent in two different ways. Pick $A>1$. Then on $1\leq b\leq A/(da_*)$ we have as in \eqref{eq:Delta}
	\begin{equation*}
		\begin{split}
			& \Re\lbrb{\Phi(a_*\lbrb{1+ibd})}-\Phi(a_*)=\IntOI \lbrb{1-\cos\lbrb{bda_*y}}e^{-a_*y}\mu(dy)\\
			&\geq c_0^2e^{-\frac{1}{bd}}b^2d^2a^2_*\int_{0}^{\frac{1}{bda_*}}y^{2}\mu(dy)\geq c_0^2e^{-\frac{1}{bd}}b^2d^2a^2_*\Delta(A).
		\end{split}
	\end{equation*}
	Since the right-hand side of the latter is bounded from below by a constant then, for some $a>0, C>0$,
	\[\int_{1}^{\frac{A}{da_*}} b^{2k+1+l}e^{ -x\lbrb{\Re\lbrb{\Phi(a_*\lbrb{1+ibd})}-\Phi(a_*)}}db\leq Ce^{-ax}.\]
	Next, for any $M<L$, see \eqref{def:condiA}, choose $A$ such that, for $b>A/(da_*)$, as in \eqref{eq:Delta},
	\begin{equation*}
		\begin{split}
			& \Re\lbrb{\Phi(a_*\lbrb{1+ibd})}-\Phi(a_*)\geq c_0^2b^2d^2a^2_*e^{-\frac{1}{bd}}\Delta(bda_*)\geq MA^2c^2_0e^{-\frac{a_*}{A}}\ln\lbrb{bda_*}.
		\end{split}
	\end{equation*}
	Therefore, for all $x$ large enough and hence $a_*$ small enough we get with some $M'$ as large as we wish
	\begin{equation*}
		\begin{split}
			&\int_{\frac{A}{da_*}}^\infty b^{2k+1+l}e^{ -x\lbrb{\Re\lbrb{\Phi(a_*\lbrb{1+ibd})}-\Phi(a_*)}}db\leq \int_{\frac{A}{da_*}}^\infty b^{2k+1+l}\lbrb{bda_*}^{-M'x}db\\
			&=a^{-2k-l-2}_{*}d^{-M'x}\int_{\frac{A}{d}}^\infty u^{2k+1+l}u^{-M'x}du=a^{-2k-l-2}_{*}d^{-2k-2-l}A^{-M'x}\frac{1}{M'x-2k-l-2}.
		\end{split}
	\end{equation*}
	Plugging this and the estimate above we arrive thanks to the second and third requirement of  \eqref{eq:addCondi}, with some $C_0>0$ and $a'>0$, at the bound which settles the claim
	\begin{equation*}
		\begin{split}
			\abs{\frac{\sqrt{-\Phi''(a_*)x}}{J(a_*,0)}J_2(a_*,x)}&\leq C_0 a_*\sqrt{-\Phi''(a_*)x}\lbrb{e^{-ax}+\frac{2}{M' x}e^{-M'x\ln(A)+(2k+l+2)\ln\lbrb{\frac{1}{a_*}}}}\\
			&\leq 2C_0 a_*\sqrt{-\Phi''(a_*)x}e^{-a'x}\leq 2C_0 e^{-a''x}.
		\end{split}
	\end{equation*} 
\end{proof}

\subsection{Proofs of Lemmae \ref{lem:phi''} and \ref{lem:condi}}
\label{subsec:aux}
We now prove the lemmae that we used in Subsection \ref{subsec:L} to compare our results with the existing ones. 
Recall that $\Delta(x):=\int_0^{\frac1x}y^2\nu_\Phi(dy),\,x>0,$ see \eqref{def:D}.
\begin{proof}[Proof of Lemma \ref{lem:phi''}]
	Since $-\phi''(x)=\IntOI e^{-xy}y^2\mu_{\Phi}(dy)$ we get that
	$-\phi''(x)\geq e^{-1}\int_{0}^{\frac{1}{x}}y^2\mu_{\Phi}(dy)$
	and the lower bound in \eqref{eq:phi''} follows. Then, using $ve^{-v}\leq e^{-1}$, for $v\geq1,$ upper bound follows from
	\begin{equation*}
		\begin{split}
			-\phi''(x)	&\leq \int_{0}^{\frac{1}{x}}y^2\mu_{\Phi}(dy)+\frac{1}{x^2}\int_{\frac{1}{x}}^{\infty}e^{-xy}x^2y^2\mu_{\Phi}(dy)
			\leq \Delta(x)+\frac{e^{-1}}{x^2} \bar{\mu}_\phi\lbrb{\frac{1}{x}}.
		\end{split}
	\end{equation*}
\end{proof}

\begin{proof}[Proof of Lemma \ref{lem:condi}]
	From Item \ref{it:phi'} of Lemma \ref{lem:Bern}  $\phi'''$ is positive and non-increasing and therefore
	\[-\phi''(x)\geq \int_x^{2x}\phi'''(y)dy\geq x\phi'''(2x).\]
	Hence \eqref{def:condiB''} implies \eqref{def:condiB}. Let us show that \eqref{condi:DR} triggers \eqref{def:condiB''}. From \eqref{eq:phi''} and \eqref{condi:DR} we get at infinity
	$-\phi''(x)\asymp \Delta(x)$. Also
	from  \cite[eq. (3.3)]{DonRiv} we have that
	\begin{equation}\label{eq:12}
		\begin{split}
			&\Delta\lbrb{\frac{1}{x}}\leq 2\int_0^x w\bar{\mu}_\phi(w)dw=\Delta\lbrb{\frac{1}{x}}+x^2 \bar{\mu}_\phi\lbrb{x}
		\end{split}
	\end{equation}
	and we arrive from \eqref{eq:phi''} that for small $x$  we have
	\begin{equation}\label{eq:relation}
		\begin{split}
			&\int_0^x w\bar{\mu}_\phi(w)dw\asymp-\phi''(x^{-1})\asymp \Delta(x^{-1}).
		\end{split}
	\end{equation}
	However, from {the assumption} \textbf{[SC']} at \cite[Page $8$]{DonRiv} we have, for any $\lambda>1$, some $c\geq 1$ and $\alpha\in\lbrbb{0,2}$,
	\begin{equation*}
		\begin{split}
			&1\leq \liminfo{x}\frac{\int_0^{\lambda x} w\bar{\mu}_\phi(w)dw}{\int_0^x w\bar{\mu}_\phi(w)dw}\leq \limsupo{x}\frac{\int_0^{\lambda x} w\bar{\mu}_\phi(w)dw}{\int_0^x w\bar{\mu}_\phi(w)dw}\leq c\lambda^{2-\alpha}.
		\end{split}
	\end{equation*}
	Setting $\lambda=2$ and using \eqref{eq:relation} with $x\to 1/2y$ we deduce that at infinity
	\[-\phi''(y)\asymp -\phi''(2y),\]
	i.e. \eqref{def:condiB''} holds and hence \eqref{def:condiB} follows.
	From \cite[Definition, eq(2.0.7), p.65]{BGT87} and \eqref{eq:relation} we get that $-\phi''(x)$ is O-regularly varying at infinity, see \cite[Corollary 2.0.5]{BGT87}. Then from \cite[2.1.9 of Theorem 2.1.8]{BGT87}  the lower Matuszewska index  of $-\phi''(x)$ is larger or equal to $\alpha-2>-2$ and the upper index is not greater than $0$. Therefore, $-\phi''(x)$ is of bounded decrease as in \cite[p.71]{BGT87} and from \cite[Proposition 2.2.1]{BGT87} we get that $-\phi''(x)\geq x^{-2+\alpha-\epsilon}$ for all $0<\epsilon$  and $x$ large enough. Therefore, from \eqref{eq:relation} we obtain that
	\[\Delta\lbrb{x}\geq Cx^{-2+\alpha-\epsilon}\]
	for some $C>0$ and all $x$ large enough and \eqref{def:condiA} holds with $L=\infty$. Also the stronger \eqref{eq:dBound} holds.
\end{proof}



\section{Proofs of results in Subsection \ref{subsec:series}}
\label{subsec:power}
Here, we prove the main results contained in Subsection \ref{subsec:series}  based on Assumptions \eqref{eq:extensionA3} and \eqref{eq:uniformlimcond}.
\subsection{Proof of Theorem \ref{thm:smoothfmu}}
To prove Theorem \ref{thm:smoothfmu}, we will make use of the following integral representation, which is a consequence of the Laplace inversion formula of Proposition \ref{prop:LTthetapi}. Recall the definition of $\mathbb\D$ in \eqref{def:Reg}.
\begin{prop}\label{prop:intreptheta}
	Let $\Phi$ be the Laplace exponent of a potentially killed subordinator satisfying Assumptions \eqref{eq:extensionA3} and \eqref{eq:uniformlimcond} for some $\theta \in (0,\pi)$. Fix any $\varepsilon>0$ and let $\gamma_{\varepsilon,\theta}$ be the circle arc in $\C$ parametrized as $\gamma_{\varepsilon,\theta}:z=\varepsilon e^{i\xi}$ for $\xi \in \left[\frac{\theta}{2}-\pi,\pi-\frac{\theta}{2}\right]$.	
	%	define the set 
	%	\begin{equation*}
		% \gamma_{\varepsilon,\theta} := \ll z \in \mathbb{C}: z=\varepsilon e^{i\xi}, \xi \in \left[ \frac{\theta}{2}-\pi, \pi- \frac{\theta}{2} \right] \rr.
		% \end{equation*}	
	Then, on $\mathbb{D}$,
	\begin{equation}\label{statement1}
		\begin{split}
			f_\Phi (x,t) \, = \,& \frac{1}{\pi} \int_\varepsilon^{+\infty}\Im\left(\frac{\Phi^\dagger\lb \rho e^{i\left(\pi-\frac{\theta}{2}\right)}\rb}{\rho } e^{-x \Phi \lb \rho e^{i\left(\pi-\frac{\theta}{2}\right)}  \rb+ t \rho e^{i\left(\pi-\frac{\theta}{2}\right)} }\right) \, d\rho  +\frac{1}{2\pi i}\int_{\gamma_{\varepsilon,\theta}} \frac{\Phi^\dagger(z)}{z}e^{-x\Phi(z)+tz}dz.
		\end{split}
	\end{equation}
	%If, furthermore,
	%	\begin{equation}\label{eq:intcondtheta}
		%		\int_0^1\frac{\left|\Im \Phi^\dagger \left(\rho e^{i\left(\pi-\frac{\theta}{2}\right)}\right)\right|}{\rho}d\rho<+\infty
		%	\end{equation}
	%	then, on $\mathbb{D}$,
	%	\begin{equation}
		%		\int_0^{+\infty} \left|\Im \l \frac{\Phi^\dagger\left(\rho e^{i\left(\pi-\frac{\theta}{2}\right)}\right)}{\rho} e^{it\rho e^{i\l \pi -\frac{\theta}{2} \r}-x\Phi\l \rho e^{i \l \pi-\frac{\theta}{2} \r} \r} \r\right| \, d\rho <+\infty
		%	\end{equation}
	%	and 
	%	\begin{equation}\label{eq:intreptheta}
		%		f_\Phi(x,t) \, = \, 	\frac{1}{\pi}\int_0^{+\infty}\Im \l \frac{\Phi^\dagger\left(\rho e^{i\left(\pi-\frac{\theta}{2}\right)}\right)}{\rho} e^{t\rho e^{i \l \pi-\frac{\theta}{2} \r}-x\Phi\l \rho e^{i \l \pi-\frac{\theta}{2} \r} \r} \r \, d\rho.
		%	\end{equation}
\end{prop}
\begin{proof}
	Again, by \eqref{ceslimitdag} and \eqref{cesequalpointdag} for fixed $a>0$ and for almost any $(x,t) \in \mathbb{D}$
	\begin{equation*}
		f_\Phi(x,t) \, = \, \lim _{b \to +\infty} \frac{1}{2\pi i}\int_{a-ib}^{a+ib} e^{z t } 	\,  \frac{\Phi^\dagger(z)}{z} e^{-x\Phi(z)} \, dz,
	\end{equation*}
	provided that the limit on the right-hand side exists. Here, we compute the limit by using Cauchy's Theorem. 
	For $R \ge a$, let $b(R)=\sqrt{R^2-a^2}$. Set $A(R):= a-ib(R)$, $B(R):=a+ib(R)$ and observe that $|A(R)|=|B(R)|=R$. In particular, note that
	\begin{equation}\label{eq:420}
		\lim _{b \to +\infty} \int_{a-ib}^{a+ib} e^{z t } 	\,  \frac{\Phi^\dagger(z)}{z} e^{-x\Phi(z)} \, dz \, = \, \lim_{R \to +\infty} \int_{A(R)}^{B(R)} e^{z t } 	\,  \frac{\Phi^\dagger(z)}{z} e^{-x\Phi(z)} \, dz.
	\end{equation}
	Now define $C(R):=Re^{i\lb \pi - \frac{\theta}{2} \rb}$ and $F(R):=Re^{i\lb \pi + \frac{\theta}{2} \rb}$. Furthermore, let $\varepsilon >0$ and define $D(\varepsilon):=\varepsilon e^{i\lb \pi - \frac{\theta}{2} \rb}$ and $E(\varepsilon):=\varepsilon e^{i\lb \pi + \frac{\theta}{2} \rb}$. We let $\Gamma_R^+$ be the anticlockwise oriented circular arc joining $B(R)$ to $C(R)$, while we define $\Gamma_R^-$ the anticlockwise oriented circular arc joining $F(R)$ to $A(R)$. Denote also $\ell_1$ to be the oriented segment connecting $A(R)$ to $B(R)$, $\ell_2$ the oriented segment connecting $C(R)$ to $D(\varepsilon)$ and $\ell_3$ the oriented segment connecting $E(\varepsilon)$ to $F(R)$. Finally, let $-\gamma_{\varepsilon,\theta}$ be the clockwise oriented circular arc joining $D(\varepsilon)$ to $E(\varepsilon)$.
	Let $\partial \mathfrak{D}$ be the closed contour obtained by connecting, in this order, $\ell_1$, $\Gamma_R^+$, $\ell_2$, $-\gamma_{\varepsilon,\theta}$, $\ell_3$, $\Gamma_R^-$ (see Figure \ref{fig1}).	Such a contour is the boundary of an open set $\mathfrak{D}$ of the complex plane in which by assumption
		\begin{equation*}
			\mathfrak{D} \ni z  \mapsto F(z; x,t) \, : = \,  \frac{\Phi^\dagger(z)}{z}e^{z t-x\Phi(z)} \in \mathbb{C}, \qquad x>0, t>0,
		\end{equation*}
		is holomorphic and continuous at the boundary. Hence, we can apply Cauchy's Theorem to get
	% \mladen{the sign in front $+ \, \int_{\gamma_{\varepsilon,\theta}} F(z; x, t) \, dz $ has to be checked}
	\begin{equation*}
		\int_{\partial \mathfrak{D}} F(z; x, t) \, dz \, = \, 0.
	\end{equation*}
	This implies that
	\begin{align}\label{decomp}
		&\int_{A(R)}^{B(R)} e^{z t } 	\,  \frac{\Phi^\dagger(z)}{z} e^{-x\Phi(z)} \, dz \\ = \,& - \int_{\Gamma_R^+}F(z; x, t) \, dz \, - \, \int_{\ell_2} F(z; x, t) \, dz \, + \, \int_{\gamma_{\varepsilon,\theta}} F(z; x, t) \, dz \, - \, \int_{\ell_3} F(z; x, t) \, dz   - \int_{\Gamma_R^-} F(z; x, t) \, dz .	
	\end{align}
	\begin{minipage}{0.5\linewidth}
		\begin{tikzpicture}[scale=0.4]
			\draw [decoration={markings,mark=at position 1 with
				{\arrow[scale=3,>=stealth]{>}}},postaction={decorate}] (0,-8.5) -- (0,8.5);
			\draw [decoration={markings,mark=at position 1 with
				{\arrow[scale=3,>=stealth]{>}}},postaction={decorate}] (-7,0) -- (7,0);
			\draw[black, dashed] (-2,-2)--(0,0);
			\draw[black, dashed] (-2,2)--(0,0);
			\draw[black] (5,-7) -- (5,-4.88);
			\draw[black] (5,4.88) -- (5,7);
			\draw[black, dashed] (5,-7) -- (5,-8);
			\draw[black, dashed] (5,7) -- (5,8);
			\draw[black, dashed] (0,0)--(4.88,4.70);
			\begin{scope}[very thick,decoration={
					markings,
					mark=at position 0.5 with {\arrow{>}}}
				] 
				\centerarc[red,very thick,postaction={decorate}](0,0)(44:90:7);
				\centerarc[red,very thick,postaction={decorate}](0,0)(90:135:7);
				\draw[red,very thick, postaction={decorate}] (5,-4.88) -- (5,4.88);
				\draw[red,very thick,postaction={decorate}] (-4.88,4.88) -- (-2,2);
				\centerarc[red,very thick,postaction={decorate}](0,0)(135:-135:2.828);
				\draw[red,very thick,postaction={decorate}] (-2,-2)--(-4.88,-4.88);
				\centerarc[red,very thick,postaction={decorate}](0,0)(-135:-90:7);
				\centerarc[red,very thick,postaction={decorate}](0,0)(-90:-44:7);
			\end{scope}
			\fill[red] (5,4.88) circle (0.2);
			\fill[red] (-4.95,4.95) circle (0.2);
			\fill[red] (-2,2) circle (0.2);
			\fill[red] (-2,-2) circle (0.2);
			\fill[red] (-4.95,-4.95) circle (0.2);
			\fill[red] (5,-4.88) circle (0.2);
			\node at (6.3,5.40) {\large $B(R)$};
			\node at (6.3,-5.40) {\large $A(R)$};
			\node at (-6.3,-5.25) {\large $F(R)$};
			\node at (-6.3,5.25) {\large $C(R)$};
			\node at (-3,1.5) {\large $D(\varepsilon)$};
			\node at (-3,-1.5) {\large $E(\varepsilon)$};
			\node at (-0.85,-1.5) {\large $\varepsilon$};
			\centerarc[dashed](0,0)(135:225:0.5);
			\node at (-1,0.3) {\large $\theta$};
			\node at (2.8,3.5) {\large $R$};
			%		\node at (5.6,0.3) {\large $x_0$};
			%		\node at (4.5,2) {\large $r$};
			\fill[red] (0,7) circle (0.2);
			\fill[red] (0,-7) circle (0.2);
			%		\fill[black] (5,0) circle (0.1);
			\node at (2,7.5) {\large $M_+(R)$};
			\node at (2,-7.7) {\large $M_-(R)$};
			\node at (-1,-5.7) {\large $\Gamma_R^-$};
			\node at (-1,5.7) {\large $\Gamma_R^+$};
			\node at (-2.7,3.8)  {\large $\ell_2$};
			\node at (-2.7,-3.8)  {\large $\ell_3$};
			\node at (2,0.3)  {\large $\gamma_{\varepsilon,\theta}$};
			\node at (4.5,0.6)  {\large $\ell_1$};
		\end{tikzpicture}
		\captionof{figure}{\label{fig1}Sketch of the keyhole-type contour.}
	\end{minipage}
	\begin{minipage}{0.5\linewidth}
		\begin{tikzpicture}[scale=0.4]
			\draw [decoration={markings,mark=at position 1 with
				{\arrow[scale=3,>=stealth]{>}}},postaction={decorate}] (0,-8.5) -- (0,8.5);
			\draw [decoration={markings,mark=at position 1 with
				{\arrow[scale=3,>=stealth]{>}}},postaction={decorate}] (-7,0) -- (7,0);
			\draw[black, dashed] (2,-2)--(0,0);
			%\draw[black, dashed] (-2,2)--(0,0);
			\draw[black] (5,-7) -- (5,-4.88);
			\draw[black] (5,4.88) -- (5,7);
			\draw[black, dashed] (5,-7) -- (5,-8);
			\draw[black, dashed] (5,7) -- (5,8);
			\draw[black, dashed] (0,0)--(4.88,4.70);
			\begin{scope}[very thick,decoration={
					markings,
					mark=at position 0.5 with {\arrow{>}}}
				] 
				\centerarc[red,very thick,postaction={decorate}](0,0)(44:90:7);
				\centerarc[red,very thick,postaction={decorate}](0,0)(90:180:7);
				\centerarc[red,very thick,postaction={decorate}](0,0)(180:90:2.828);
				\centerarc[red,very thick,postaction={decorate}](0,0)(90:0:2.828);
				\centerarc[red,very thick,postaction={decorate}](0,0)(0:-90:2.828);
				\centerarc[red,very thick,postaction={decorate}](0,0)(-90:-180:2.828);
				\draw[red,very thick, postaction={decorate}] (5,-4.88) -- (5,4.88);
				\centerarc[red,very thick,postaction={decorate}](0,0)(-180:-90:7);
				\centerarc[red,very thick,postaction={decorate}](0,0)(-90:-44:7);
			\end{scope}
			\begin{scope}[very thick,decoration={
					markings,
					mark=at position 0.3 with {\arrow{>[left]}}}
				]
				\draw[red,very thick,postaction={decorate}] (-7,0) -- (-2.828,0);
				\draw[red,very thick,postaction={decorate}] (-2.828,0)--(-7,0);
			\end{scope}
			\fill[red] (5,4.88) circle (0.2);
			\fill[red] (-7,0) circle (0.2);
			\fill[red] (-2.828,0) circle (0.2);
			\fill[red] (5,-4.88) circle (0.2);
			\node at (6.3,5.40) {\large $B(R)$};
			\node at (6.3,-5.40) {\large $A(R)$};
			%\node at (-6.3,-5.25) {\large $F(R)$};
			\node at (-8.5,1.2) {\large $C(R)$};
			\node at (-8.5,-1.2) {\large $F(R)$};
			\node at (-1.5,0.8) {\large $D(\varepsilon)$};
			\node at (-1.5,-0.8) {\large $E(\varepsilon)$};
			%\node at (-3,-1.5) {\large $E(\varepsilon)$};
			\node at (0.85,-1.5) {\large $\varepsilon$};
			%\centerarc[dashed](0,0)(135:225:0.5);
			%\node at (-1,0.3) {\large $\theta$};
			\node at (2.8,3.5) {\large $R$};
			%		\node at (5.6,0.3) {\large $x_0$};
			%		\node at (4.5,2) {\large $r$};
			\fill[red] (0,7) circle (0.2);
			\fill[red] (0,-7) circle (0.2);
			%		\fill[black] (5,0) circle (0.1);
			\node at (2,7.5) {\large $M_+(R)$};
			\node at (2,-7.7) {\large $M_-(R)$};
			\node at (-1,-5.7) {\large $\Gamma_R^-$};
			\node at (-1,5.7) {\large $\Gamma_R^+$};
			\node at (-4,1)  {\large $\ell_2$};
			\node at (-6,-1)  {\large $\ell_3$};
			\node at (2,0.3)  {\large $\gamma_{\varepsilon}$};
			\node at (4.5,0.6)  {\large $\ell_1$};
		\end{tikzpicture}
		\captionof{figure}{\label{fig2}Sketch of the keyhole contour.}	
	\end{minipage}\\
	
	Now we deal with various terms separately.
	To deal with the first integral, let $M^+(R)= Ri$ and split the curve $\Gamma_R^+$ into $\Gamma_R^1$ connecting $B(R)$ to $M^+(R)$ and $\Gamma_R^2$ connecting $M^+(R)$ to $C(R)$ so that
	\begin{equation}	\label{29}
		\int_{\Gamma_R^+} F(z; x, t) \, dz  \, = \, \int_{\Gamma_R^1} F(z; x, t) \, dz  + \int_{\Gamma_R^2} F(z; x, t) \, dz.
	\end{equation}
	We begin with $\Gamma_R^1$. Note that, by the Estimation Lemma \cite[Theorem 5.24]{howie}
	\begin{equation}\label{estlemmagen}
		\left| \int_{\Gamma_R^1} F(z; x, t) \, dz \right| \, \leq \, \text{length}(\Gamma_R^1) \, \max_{z \in \Gamma_R^1} \left| F(z; x, t) \right|,
	\end{equation}
	where, with an abuse of notation, we denote by $\Gamma_R^1$ also the image of the parametrized oriented curve. To evaluate the maximum appearing in \eqref{estlemmagen} we parametrize $\Gamma_R^1$ as follows
	\begin{equation*}
		\Gamma_R^1 \, = \, \ll z \in \mathbb{C}: z = R e^{i\xi}, \xi \in \left[ \xi_{B}(R), \frac{\pi}{2} \right] \rr,
	\end{equation*}
	where $\xi_B(R)= \arctan \lb \frac{b(R)}{a} \rb$.
	Then, for $z = Re^{i\xi} \in \Gamma_R^1$, it holds
	\begin{equation*}
		\left| F(z; x,t) \right| \, =  \frac{\left| \Phi^\dagger(Re^{i\xi}) \right|}{R} e^{Rt\cos \xi -x\Re \Phi (Re^{i\xi})}.
	\end{equation*}
	Without loss of generality we can assume $\xi_B(R)> \frac{\pi}{4}$. First, observe that $\Re \lb Re^{i\xi}\rb= R \cos \xi \geq 0$, for any $\xi \in \left[ \xi_B(R), \frac{\pi}{2} \right]$, and thus, by Item \eqref{it:sign} of Lemma \ref{lem:Bern}, we have that $\Re \Phi \lb Re^{i\xi} \rb \geq 0$. Furthermore, it holds that $R \cos \xi \leq R \cos \xi_{B}(R) = a$, for any $\xi \in \left[ \xi_B(R), \frac{\pi}{2} \right]$. Hence, we get
	\begin{equation*}
		\max_{z \in \Gamma_R^1} \left| F(z; x,t) \right| \, \leq \, e^{ta} \max_{\xi \in \left[ \frac{\pi}{4}, \frac{\pi}{2} \right]} \left| \frac{\Phi^{\dagger} \lb Re^{i\xi} \rb}{Re^{i\xi}} \right|
	\end{equation*}
	and thus by Item \eqref{it:asymp} of Lemma \ref{lem:Bern} and the definition of $\phi^\dagger$, see \eqref{def:Pdag}, it holds
	\begin{equation}\label{maxtozerogen}
		\lim_{R \to + \infty} \max_{z \in \Gamma_R^1} \left| F(z; x,t) \right| = 0.
	\end{equation}
	%	Note now that
	%	\begin{equation*}
		%		\begin{split}
			%			\text{length}(\Gamma_R^1) \, = \,& R \l \frac{\pi}{2} - \xi_B(R) \r = R \l \frac{\pi}{2}-\arctan \frac{r(R)}{a} \r 
			%			= \,  \frac{R a}{r(R)} \frac{r(R)}{a} \l \frac{\pi}{2}-\arctan \frac{r(R)}{a} \r.
			%		\end{split}	
		%	\end{equation*}
	%	Since $r(R)= \sqrt{R^2-a^2}$ we obtain
	Furthermore, it is not difficult to check that
	\begin{equation}\label{lengthtox0gen}
		\lim_{R \to +\infty}\text{length}(\Gamma_R^1) \, = \, a.
	\end{equation}
	By combining \eqref{maxtozerogen} and \eqref{lengthtox0gen} with \eqref{estlemmagen} we have that
	\begin{equation}\label{firsttozerogen11}
		\lim_{R \to +\infty} \int_{\Gamma_R^1} F(z; x,t) \, dz \, = \, 0.
	\end{equation}	
	Now, we deal with the second term in \eqref{29}, i.e. the one on $\Gamma_R^2$.
	%\begin{equation*}
	%	\int_{\Gamma_R^2}F(z; x,t) \, dz \, = \, \int_{\Gamma_R^2} e^{z t} \frac{\Phi^\dagger(z)}{z} e^{-x\Phi(z)} dz.
	%\end{equation*}
	Recalling that $(x,t) \in \mathbb{D}$, let $\delta=t-\mathfrak{b}x>0$ and choose $p > 1$ such that $\frac{t}{p}-\mathfrak{b}x>\frac{\delta}{2}$, that exists since $\lim_{p \to 1}\frac{t}{p}-\mathfrak{b}x=\delta$. Let also $p^\prime >1$ be the conjugate exponent of $p>1$, i.e. $\frac{1}{p}+\frac{1}{p^\prime}=1$. Then we have
	%	Let $z = i\zeta$ and set $\widetilde{\Gamma}_R^2:= \ll z \in \mathbb{C}: z=Re^{i\zeta}, \zeta \in \left[ 0, \frac{\pi-\theta}{2} \right] \rr$ to get
	%	\begin{equation*}
		%		\int_{\Gamma_R^2}F(z; x,t) \, dz \, = \,i \int_{\widetilde{\Gamma}_R^2} e^{i\zeta t} \frac{\Phi^\dagger(i\zeta)}{i\zeta} e^{-x \Phi(i\zeta)} \, d\zeta.
		%	\end{equation*} 
	%	Let $\delta=t-\mathfrak{b}x>0$ and choose $p > 1$ such that $\frac{t}{p}-\mathfrak{b}x>\frac{\delta}{2}$. This is always possible since we are on $\mathbb{D}$ and since $\frac{t}{p}-\mathfrak{b}x < \delta$ with $\frac{t}{p}-\mathfrak{b}x \to \delta$ as $p\to 1$. Let $q_* \ge 1$ such that $\frac{1}{p}+\frac{1}{q_*}=1$ and $G(\zeta;x,t):= e^{\lambda t/p}(i\zeta)^{-1}\Phi(i\zeta) e^{-x\Phi(i\zeta)}$. We get
	\begin{equation}	\label{220}
		\begin{split}
			\left| \int_{\Gamma_R^2}F(z; x,t) \, dz \right| \,  \leq \, & R \lb \max_{z \in \Gamma_R^2}  \left| \frac{\Phi^\dagger(z)}{z} e^{ \frac{t}{p}z-x\Phi(z)} \right| \rb \lb \int_0^{\frac{\pi-\theta}{2}} e^{-(Rt\sin \xi)/p^\prime} d\xi \rb  \\
			\leq \, & \frac{p^\prime \pi}{t} \lb \max_{z \in \Gamma_R^2}  \left| \frac{\Phi^\dagger(z)}{z} e^{ \frac{t}{p}z-x\Phi(z)} \right| \rb = \, \frac{p^\prime \pi e^{-xq}}{t} \lb \max_{z \in \Gamma_R^2}  \left| \frac{\Phi^\dagger(z)}{z} e^{ \left(\frac{t}{p}-\mathfrak{b}x\right)z-x\Phi^\dagger(z)} \right| \rb, 
		\end{split}	
	\end{equation}
	%	\begin{equation}	\label{220}
		%	\begin{split}
			%		\left| \int_{\Gamma_R^2}F(z; x,t) \, dz \right| \, = \, &\left| \int_{\widetilde{\Gamma}_R^2} e^{i\zeta t} \frac{\Phi^\dagger(i\zeta)}{i\zeta} e^{-x \Phi(i\zeta)} \, d\zeta \right|  \\
			%		= \, & \left| \int_0^{(\pi-\theta)/2}  e^{iRe^{i\xi}t/q_*} G(Re^{i\xi};x,t) Re^{i\xi} d\xi \right|  \\
			%		\leq \, & R \int_0^{(\pi-\theta)/2}  e^{-(Rt \sin \xi )/q_*} \left| G\l Re^{i\xi};x,t \r \right| d\xi  \\
			%		\leq \, & R \l \max_{\xi \in \left[ 0, \frac{\pi-\theta}{2} \right]}  \left| G\l Re^{i\xi};x,t \r \right| \r \l \int_0^\pi e^{-(Rt\sin \xi)/q_*} d\xi \r  \\
			%		\leq \, & \frac{q_*\pi}{t} \l \max_{z \in \Gamma_R^2}  \left| \frac{\Phi^\dagger(z)}{z} e^{ \frac{t}{p}z-x\Phi(z)} \right| \r \\
			%		= \, & \frac{q_*\pi e^{-xq}}{t} \l \max_{z \in \Gamma_R^2}  \left| \frac{\Phi^\dagger(z)}{z} e^{ \left(\frac{t}{p}-\mathfrak{b}x\right)z-x\Phi^\dagger(z)} \right| \r, 
			%	\end{split}	
		%	\end{equation}
	where in the second inequality we have used Jordan's inequality \cite[eq (2), page 262]{brownchurchill}. Now, consider 
	\begin{equation}\label{eq:goftheproof}
	g(z)=e^{ \left(\frac{t}{p}-\mathfrak{b}x\right)z-x\Phi^\dagger(z)}	
	\end{equation}
and observe that, by hypothesis, $g$ is holomorphic on $\C\left(\frac{\pi}{2},\pi-\frac{\theta}{2}\right)$ and continuous on $\overline{\C\left(\frac{\pi}{2},\pi-\frac{\theta}{2}\right)}$. Furthermore, for $z=iR$ we have
	\begin{equation*}
		|g(iR)|=e^{-x\Re \Phi^\dagger(iR)} \le 1,
	\end{equation*}
	since $\Re \Phi^\dagger(iR) \ge 0$ by Item \eqref{it:sign} of Lemma \ref{lem:Bern}. For $z=Re^{i\left(\pi-\frac{\theta}{2}\right)}$ we have instead
	\begin{equation*}
		\left|g\left(Re^{i\left(\pi-\frac{\theta}{2}\right)}\right)\right|=e^{-R\left(\frac{t}{p}-\mathfrak{b}x\right)\cos\left(\frac{\theta}{2}\right)-x\Re \Phi^\dagger\left(Re^{i\left(\pi-\frac{\theta}{2}\right)}\right)}.
	\end{equation*}
	Now let us show that for $R$ big enough, $\left|g\left(Re^{i\left(\pi-\frac{\theta}{2}\right)}\right)\right| \le 1$. Indeed,
	\begin{align*}
		\left|g\left(Re^{i\left(\pi-\frac{\theta}{2}\right)}\right)\right|&=\exp\left(-R\cos\left(\frac{\theta}{2}\right)\left(\frac{t}{p}-\mathfrak{b}x+\frac{x}{\cos\left(\frac{\theta}{2}\right)}\frac{\Re \Phi^\dagger\left(Re^{i\left(\pi-\frac{\theta}{2}\right)}\right)}{R}\right)\right).
	\end{align*}
	Once we observe that 
	\begin{equation*}
		\left|\frac{\Re \Phi^\dagger\left(Re^{i\left(\pi-\frac{\theta}{2}\right)}\right)}{R}\right| \le \frac{\left| \Phi^\dagger\left(Re^{i\left(\pi-\frac{\theta}{2}\right)}\right)\right|}{R}
	\end{equation*}
	and we use \eqref{eq:uniformlimcond} to state that $\lim_{R \to +\infty}\frac{\left| \Phi^\dagger\left(Re^{i\left(\pi-\frac{\theta}{2}\right)}\right)\right|}{R}=0$, we know that there exists a constant $C(t,x,p)$ such that for $R>C(t,x,p)$
	\begin{equation*}
		\frac{t}{p}-\mathfrak{b}x+\frac{x}{\cos\left(\frac{\theta}{2}\right)}\frac{\Re \Phi^\dagger\left(Re^{i\left(\pi-\frac{\theta}{2}\right)}\right)}{R} \ge \frac{\delta}{4}.
	\end{equation*}
	Hence, for $R>C(t,x,p)$,
	\begin{align*}
		\left|g\left(Re^{i\left(\pi-\frac{\theta}{2}\right)}\right)\right|&\le \exp\left(-\frac{\delta R\cos\left(\frac{\theta}{2}\right)}{4}\right) \le 1.
	\end{align*}
	On the other hand, $R \in [0,+\infty) \mapsto \left|g\left(Re^{i\left(\pi-\frac{\theta}{2}\right)}\right)\right| \in \R$ is continuous and then the latter inequality implies that there exists $M=M(x,t,p)$ such that $|g(z)| \le M$ for any $z \in \partial \C\left(\frac{\pi}{2},\pi-\frac{\theta}{2}\right)$.
	Furthermore, observe that, for $z=Re^{i\xi}$ with $\xi \in \left(\frac{\pi}{2},\pi-\frac{\theta}{2}\right)$ we have, by \eqref{eq:uniformlimcond}, $|\Re\Phi(z)|\le |\Phi(z)| \le C|z|$, for $|z| \ge 1$. Hence, for $|z| \ge 1$, transferring through \eqref{def:Pdag} $\phi^\dagger$ to $\phi$ and using that $\Re(z)\leq 0$, we get that
	\begin{equation*}
		|g(z)|=e^{\frac{t}{p} \Re z+{x}q-x\Re(\Phi(z))} \le e^{xq+x|\Re(\Phi(z))|} \le e^{qx} e^{xC|z|}.
	\end{equation*}
	The continuity of the function $z \in \overline{\C\left(\frac{\pi}{2},\pi-\frac{\theta}{2}\right)} \mapsto |g(z)|e^{xC|z|} \in \R$ guarantees that there exists a constant $M_1=M_1(x,t,p,q)>0$ such that 
	\begin{equation*}
		|g(z)| \le M_1e^{xC|z|}, \ \forall z \in \overline{\C\left(\frac{\pi}{2},\pi-\frac{\theta}{2}\right)}.
	\end{equation*}
	Since $\pi-\frac{\theta}{2}-\frac{\pi}{2}<\frac{\pi}{2}$, we can use Phragmen-Lindel\"of Theorem (see \cite[Chapter $4$, Exercise $9$, Item (b)]{stein10}) to obtain $|g(z)| \le M$ for any $z \in \overline{\C\left(\frac{\pi}{2},\pi-\frac{\theta}{2}\right)}$.
	%	\begin{equation*}
		%		M(x,t,p)=\max\left\{\max_{R \in [0,C(t,x,p)]}\left|g\left(Re^{i\left(\pi-\frac{\theta}{2}\right)}\right)\right|, 1\right\}
		%	\end{equation*}
	%	so that, for any $z \in \partial \C\left(\frac{\pi}{2},\pi-\frac{\theta}{2}\right)$ it holds $|g(z)| \le M(x,t,p)$. 
	%	
	%	
	%	Thus, define $g_M=\frac{g}{M(x,t,p)}$ so that $g_M$ is holomorphic on $\C\left(\frac{\pi}{2},\pi-\frac{\theta}{2}\right)$, continuous on $\overline{\C\left(\frac{\pi}{2},\pi-\frac{\theta}{2}\right)}$ and $|g_M(z)| \le 1$, for any $z \in \partial \C\left(\frac{\pi}{2},\pi-\frac{\theta}{2}\right)$. Next, observe that, for $z=Re^{i\xi}$ with $\xi \in \left(\frac{\pi}{2},\pi-\frac{\theta}{2}\right)$ we have, by \eqref{eq:uniformlimcond}, $|\Re\Phi(z)|\le |\Phi(z)| \le C|z|$, for $|z| \ge 1$. Hence, for $|z| \ge 1$,
	%	\begin{equation*}
		%		|g_M(z)|=\frac{|g(z)|}{M(x,t,p)}=\frac{e^{\frac{t}{p}\Re z-x\Re(\Phi(z))}}{M(x,t,p)} \le  \frac{e^{x|\Re\Phi(z)|}}{M(x,t,p)}\le \frac{e^{Cx|z|}}{M(x,t,p)}.
		%	\end{equation*}
	%Furthermore, define
	%\begin{equation*}
	%M_1 (x,t,p) = \max_{\substack{z \in \overline{\C\left(\frac{\pi}{2},\pi-\frac{\theta}{2}\right)} \\ |z| \leq 1 }} |g_M(z)| e^{-Cx|z|}
	%\end{equation*}	
	%and
	%\begin{equation*}
	%M_2(x,t,p) = \max \l M_1(x,t,p), \frac{1}{M(x,t,p)} \r
	%\end{equation*}
	%so that $|g_M(z)| \leq M_2(x,t,p) e^{Cx|z|}$, for any $z \in \overline{\C\left(\frac{\pi}{2},\pi-\frac{\theta}{2}\right)}$.
	%Finally, observe that
	%	\begin{equation*}
		%		\pi-\frac{\theta}{2}-\frac{\pi}{2}=\frac{\pi-\theta}{2}=:\frac{\pi}{\beta},
		%	\end{equation*}
	%	for some $\beta>2$. Hence we can use Phragmen-Lindel\"of Theorem (see \cite[Chapter $4$, Exercise $9$, Item (b)]{stein10}) to obtain $|g_M(z)| \le 1$ for any $z \in \overline{C\left(\frac{\pi}{2},\pi-\frac{\theta}{2}\right)}$, and then $|g(z)| \le M(x,t,p)$.
	%	
	Thus, from \eqref{220}, we have
	\begin{equation*}
		\left| \int_{\Gamma_R^2}F(z; x,t) \, dz \right| \le \frac{M(x,t,p)p'\pi e^{-xq}}{t} \lb \max_{z \in \Gamma_R^2}  \left| \frac{\Phi^\dagger(z)}{z} \right| \rb.
	\end{equation*}
	Taking the limit as $R \to +\infty$ we finally have
	\begin{equation}\label{230}
		\lim_{R \to +\infty} \int_{\Gamma_R^2} F(z;x,t)\, dz \, = \, 0,
	\end{equation}
	that, combined with \eqref{firsttozerogen11}, leads to
	\begin{equation}	\label{circsopra}
		\lim_{R\to +\infty} \int_{\Gamma_R^+} F(z;x,t) \, dz \, = \,0.
	\end{equation}
	In the same spirit, it is possible to see that
	\begin{equation}\label{circsotto}
		\lim_{R \to +\infty} \int_{\Gamma_R^-}F(z;x,t) \, dz \, = \, 0.	
	\end{equation}
	We consider now the integral on $\ell_2$. We have that
	\begin{equation}\label{239}
		\int_{\ell_2} F (z;x,t) dz \,= \, -\int_{\varepsilon}^R \frac{\Phi^\dagger\lb\rho e^{i\lb \pi-\frac{\theta}{2} \rb}\rb}{\rho } e^{-x \Phi \lb \rho e^{i\lb \pi-\frac{\theta}{2} \rb} \rb+t  \rho e^{i \lb \pi-\frac{\theta}{2} \rb}} \, d\rho.
	\end{equation}
	Choose $p,p^\prime > 1$ as above so that, recalling that $\frac{\left|\Phi^\dagger\left(\rho e^{i\left(\pi-\frac{\theta}{2}\right)}\right)\right|}{\rho} \le C$ for $\rho \ge \varepsilon$ by \eqref{eq:uniformlimcond}, we have
	\begin{align}
		&\int_\varepsilon^\infty \left|  \frac{\Phi^\dagger\lb\rho e^{i\lb \pi-\frac{\theta}{2} \rb}\rb}{\rho } e^{-x \Phi \lb \rho e^{i\lb \pi-\frac{\theta}{2} \rb} \rb+t  \rho e^{i \lb \pi-\frac{\theta}{2} \rb}} \right| d\rho \notag \\	= \, & e^{-qx}\int_\varepsilon^\infty \frac{|\Phi^\dagger\lb \rho e^{i \lb \pi-\frac{\theta}{2} \rb} \rb|}{\rho}  \exp \ll {-\rho \cos\left(\frac{\theta}{2}\right) \lb \frac{t}{p}-\mathfrak{b}x +x \frac{\Re \Phi^\dagger \lb \rho e^{i \lb \pi-\frac{\theta}{2} \rb}  \rb}{\rho \cos\left(\frac{\theta}{2}\right)}  \rb } \rr e^{-\rho  \frac{t}{p^\prime} \cos \frac{\theta}{2} }d\rho \notag \\
		\le & CM\int_\varepsilon^\infty e^{-\frac{\rho t}{p^\prime}\cos\left(\frac{\theta}{2}\right)}d\rho <+\infty, \label{richiamatadopo}
	\end{align}
	where we have also used that $\frac{t}{p}-\mathfrak{b}x>\frac{\delta}2>0$.
	Hence, we can take the limit in \eqref{239} to get
	\begin{equation}	\label{243}
		\lim_{R \to +\infty}\int_{\ell_2} F(z; x,t) \, dz \, = \, -\int_\varepsilon^{+\infty}\frac{\Phi^\dagger\lb\rho e^{i\lb \pi-\frac{\theta}{2} \rb}\rb}{\rho } e^{-x \Phi \lb \rho e^{i\lb \pi-\frac{\theta}{2} \rb} \rb+t  \rho e^{i \lb \pi-\frac{\theta}{2} \rb}} \, d\rho \notag \\ =: \, -I_1(\varepsilon).
	\end{equation}
	Analogously, on $\ell_3$, we have that
	\begin{equation}\label{244}
		\lim_{R \to +\infty}\int_{\ell_3} F(z; x,t) \, dz \, = \, \int_\varepsilon^{+\infty}\frac{\Phi^\dagger\lb\rho e^{i\lb \pi+\frac{\theta}{2} \rb}\rb}{\rho } e^{-x \Phi \lb \rho e^{i\lb \pi+\frac{\theta}{2} \rb} \rb+t  \rho e^{i \lb \pi+\frac{\theta}{2} \rb}} \, d\rho\, =: \, I_2(\varepsilon).
	\end{equation}
	Furthermore, by using the fact that $\overline{\Phi(z)} = \Phi(\overline{z})$ by Schwartz reflection principle (see \cite[Theorem $5.6$]{stein10}), we know that $I_2(\varepsilon) = \overline{I_1(\varepsilon)}$. Hence, taking the limit as $R \to +\infty$ in \eqref{decomp} and using \eqref{circsopra}, \eqref{circsotto}, \eqref{243} and \eqref{244}, we get
	\begin{equation*}
		\begin{split}
			\lim_{b \to +\infty}\int_{a-ib}^{a+ib} e^{z t } 	\,  \frac{\Phi^{\dagger}(z)}{z} e^{-x\Phi(z)} \, dz \, & = \,  I_1(\varepsilon)-\overline{I_1(\varepsilon)} + \, \int_{\gamma_{\varepsilon,\theta}} F(z; x, t) \, dz   =2iI_3(\varepsilon)+\, \int_{\gamma_{\varepsilon,\theta}} F(z; x, t) \, dz,
		\end{split}
	\end{equation*}
	where we denote
	\begin{align}
		I_3(\varepsilon):=\int_\varepsilon^{+\infty}\Im\left(\frac{\Phi^\dagger\lb \rho e^{i\left(\pi-\frac{\theta}{2}\right)}\rb}{\rho } e^{-x \Phi \lb \rho e^{i\left(\pi-\frac{\theta}{2}\right)}  \rb+ t \rho e^{i\left(\pi-\frac{\theta}{2}\right)} }\right) \, d\rho.
		\label{beforeepsilon}
	\end{align}
	This proves \eqref{statement1}.
	%	Now we would like to take the limit as $\varepsilon \to 0$ to get \eqref{eq:intreptheta}. To do this, let us first handle the integral over $\gamma_{\varepsilon,\theta}$. By the estimation Lemma
	%	\begin{equation*}
		%		\begin{split}
			%			\left| \int_{\gamma_{\varepsilon,\theta}} F(z;x,t) \, dz \right| \, \leq \, & \text{length} (\gamma_{\varepsilon,\theta}) \, \max_{z \in \gamma_{\varepsilon}} \left| F(z; x,t) \right| z
			%			= \varepsilon (2\pi - \theta) \max_{z \in \gamma_{\varepsilon}} \left| F(z; x,t) \right|
			%		\end{split}
		%	\end{equation*}
	%	Note now that
	%	\begin{equation*}
		%		\max_{z \in \gamma_{\varepsilon}} \left| F(z; x,t) \right| \, = \, \frac{1}{\varepsilon}  \max_{z \in \gamma_{\varepsilon}} \left[ \left| \Phi(z) \right| e^{t \Re z -x \Re \Phi (z)} \right]
		%	\end{equation*}
	%	and thus
	%	\begin{align}
		%		\left| \int_{\gamma_{\varepsilon,\theta}} F (\lambda; x,t) d\lambda \right| \, \leq \, (2\pi-\theta) \max_{\lambda \in \gamma_{\varepsilon,\theta}} \left[ \left| \Phi^\dagger(\lambda)\right| e^{t\Re \lambda-x\Re \Phi(\lambda)} \right].
		%		\label{233}
		%	\end{align}
	%	By the continuity of $\left| \Phi(z)\right| e^{t\Re z-x\Re \Phi(z)}$ in $\overline{\C\left(\pi-\frac{\theta}{2}\right)}$ (and thus the uniform continuity in $\overline{\C\left(\pi-\frac{\theta}{2}\right)} \cap \{z \in \C: \ |z| \le 1\}$) we clearly have
	%	\begin{equation}	\label{cerchietto}
		%		\lim_{\varepsilon \to 0} \int_{\gamma_{\varepsilon,\theta}} F(z ;x,t) \, dz\, = \, 0.
		%	\end{equation}
	%	To take the limit as $\varepsilon \to 0$ in $I_3(\varepsilon)$, just observe that, arguing exactly as in the proof of Proposition \ref{prop:LTthetapi}, we have
	%	\begin{align}
		%			\int_0^1 \left| \Im  \left(\frac{\Phi^\dagger\l\rho e^{i\l \pi-\frac{\theta}{2} \r}\r}{\rho } e^{-x \Phi \l \rho e^{i\l \pi-\frac{\theta}{2} \r} \r+t  \rho e^{i \l \pi-\frac{\theta}{2} \r}}\right) \right| d\rho \,  \leq \, C  \left(1+\int_0^1  \frac{\left|\Im \Phi\left(\rho e^{i\left(\pi-\frac{\theta}{2}\right)}\right)\right|}{\rho}\right) d\rho<+\infty.
		%	\end{align}
	%	The statement then follows.
\end{proof}
\begin{rmk}
	Under the hypotheses of Proposition \ref{prop:intreptheta}, if furthermore \begin{equation}\label{eq:intcondtheta}
		\int_0^1\frac{\left|\Im \Phi^\dagger \left(\rho e^{i\left(\pi-\frac{\theta}{2}\right)}\right)\right|}{\rho}d\rho<+\infty
	\end{equation}
	then we can send $\varepsilon \to 0$ in \eqref{statement1}, wherein by the estimation lemma the second term converges to zero, hence getting
	\begin{equation*}
		f_\Phi (x,t) \, = \, \frac{1}{\pi} \int_0^{+\infty}\Im\left(\frac{\Phi^\dagger\lb \rho e^{i\left(\pi-\frac{\theta}{2}\right)}\rb}{\rho } e^{-x \Phi \lb \rho e^{i\left(\pi-\frac{\theta}{2}\right)}  \rb+ t \rho e^{i\left(\pi-\frac{\theta}{2}\right)} }\right) \, d\rho.
	\end{equation*}
\end{rmk}
A similar integral representation holds also for $\bar{\mu}_\phi^{\ast n}$. The proof is similar to the one of Proposition \ref{prop:intreptheta}, where we substitute the term $\frac{\phi^\dagger(z)}{z}e^{-x\phi(z)}$ with $\left(\frac{\phi^\dagger(z)}{z}\right)^n$. For such a reason, we only underline the parts of the proof that are actually different.
\begin{prop}
	\label{lem:convtail}
	With the same notation of Proposition \ref{prop:intreptheta}, under \eqref{eq:extensionA3} and \eqref{eq:uniformlimcond}, for $n \ge \red{1}$, it holds that, for any $t>0$,
	\begin{equation}\label{convcode}
		\bar{\mu}_\Phi^{\ast n} (t) = \frac{1}{\pi} \int_{\varepsilon}^{+\infty} \Im \left[ \lb \frac{\Phi^\dagger \lb \rho e^{i \lb \pi-\frac{\theta}{2} \rb} \rb}{\rho e^{i \lb \pi-\frac{\theta}{2} \rb}} \rb^n e^{i \lb \pi-\frac{\theta}{2} \rb}e^{t\rho e^{i \lb \pi-\frac{\theta}{2} \rb}} \right] d\rho +\frac{1}{2\pi i} \int_{\gamma_{\varepsilon,\theta}} e^{tz} \lb \frac{\phi^\dagger (z)}{z} \rb^n dz.
	\end{equation}
	%	 Furthermore, it is true that the function $(0, +\infty) \mapsto \bar{\nu}_\phi^{\star n} (t)$, has, for any $n=1,2,\cdots$, derivatives of all order $r \geq 0$, such that
	%	\begin{equation}
		%		\frac{d^r}{dt^r} \bar{\mu}_\Phi^{n\star} (t) 
		%		= \frac{1}{\pi} \int_{\varepsilon}^{+\infty} \Im \left[  \frac{\l\Phi^\dagger \l \rho e^{i \l \pi-\frac{\theta}{2} \r} \r\r^n}{\rho^{n-r}e^{i(n-r-1)\l \pi-\frac{\theta}{2} \r}}  e^{t\rho e^{i \l \pi-\frac{\theta}{2} \r}} \right] d\rho -\frac{1}{2\pi i} \int_{\gamma_{\varepsilon,\theta}} e^{tz}  \frac{\l\phi^\dagger (z)\r^n}{z^{n-r}}  dz.
		%		\label{derivatecode}
		%	\end{equation}
\end{prop}
\begin{proof}
	Setting $F_n(z;t)=\left(\frac{\phi^\dagger(z)}{z}\right)^ne^{tz}$ the proof follows as the one in Proposition \ref{prop:intreptheta}. The main differences concern integrals over $\Gamma^2_R$ and $\ell_2$. First, by Jordan's inequality \cite[eq. (2), page 262]{brownchurchill},
	\begin{equation*}
		\left|\int_{\Gamma^2_R}F_n(z;t)dz\right| \le \frac{\pi}{t}\left(\max_{z \in \Gamma_R^2}\left|\frac{\Phi^\dagger(z)}{z}\right|^n\right) \to 0,\,\, \mbox{as $R\to\infty$},
	\end{equation*}
	where the limit holds by assumption \eqref{eq:uniformlimcond}. Concerning the integral over $\ell_2$, observe that
	\begin{equation*}
		\int_{\varepsilon}^{\infty}|F_n\left(\rho e^{i\left(\pi-\frac{\theta}{2};t\right)};t\right)|dz \le \int_{\varepsilon}^{\infty}e^{-\rho t \cos\left(\frac{\theta}{2}\right)}\left|\frac{\Phi^\dagger\left(\rho e^{i\left(\pi-\frac{\theta}{2}\right)}\right)}{\rho}\right|^nd\rho \le C\int_{\varepsilon}^{\infty}e^{-\rho t \cos\left(\frac{\theta}{2}\right)} d\rho<+\infty,
	\end{equation*}
	where we have used the fact that $\left|\frac{\Phi^\dagger\left(\rho e^{i\left(\pi-\frac{\theta}{2}\right)}\right)}{\rho}\right|^n$ is bounded for $\rho \ge \varepsilon$ by assumption \eqref{eq:uniformlimcond}.
\end{proof}
Now we are ready to prove Theorem \ref{thm:smoothfmu} by using the previously obtained integral representations.
\begin{proof}[Proof of Theorem \ref{thm:smoothfmu}]
	Let us first prove that $\bar{\mu}_\phi^{\ast n} \in C^\infty(0,+\infty)$. To do this, fix $\varepsilon>0$, let $l \ge 1$ and $[t_1,t_2]\subset (0,+\infty)$. Let $F_n(z;t)=\left(\frac{\phi^\dagger(z)}{z}\right)^ne^{tz}$. Then $\frac{\partial^l}{\partial t^l}F_n(z,t)=z^l\left(\frac{\phi^\dagger(z)}{z}\right)^ne^{tz}$ is continuous for $(z,t) \in \gamma_{\varepsilon,\theta} \times [t_1,t_2]$, where, with an abuse of notation, $\gamma_{\varepsilon,\theta}$ is the image of the parametrized curve defined in Proposition \ref{prop:intreptheta}. Then we have
	\begin{equation}\label{eq:upboundovereps}
		\left|\frac{\partial^l}{\partial t^l}F_n(z,t)\right| \le \max_{(z,t) \in \gamma_{\varepsilon,\theta} \times [t_1,t_2]}\left|z^l\left(\frac{\phi^\dagger(z)}{z}\right)^ne^{tz}\right|,
	\end{equation}
	where the right-hand side is constant, hence integrable over $\gamma_{\varepsilon,\theta}$. Next, let
	\begin{equation*}
		G_n(\rho,t)=\Im \left[ \lb \frac{\Phi^\dagger \lb \rho e^{i \lb \pi-\frac{\theta}{2} \rb} \rb}{\rho e^{i \lb \pi-\frac{\theta}{2} \rb}} \rb^n e^{i \lb \pi-\frac{\theta}{2} \rb}e^{t\rho e^{i \lb \pi-\frac{\theta}{2} \rb}} \right]
	\end{equation*}
	and observe that
	\begin{equation*}
		\frac{\partial^l}{\partial t^l}G_n(\rho,t)=\Im \left[ \lb \frac{\Phi^\dagger \lb \rho e^{i \lb \pi-\frac{\theta}{2} \rb} \rb}{\rho e^{i \lb \pi-\frac{\theta}{2} \rb}} \rb^n \rho^l e^{i(l+1) \lb \pi-\frac{\theta}{2} \rb}e^{t\rho e^{i \lb \pi-\frac{\theta}{2} \rb}} \right].
	\end{equation*}
	For $\rho \ge \varepsilon$ and $t \in [t_1,t_2]$, we have
	\begin{equation}\label{eq:contrder2}
		\left|\frac{\partial^l}{\partial t^l}G_n(\rho,t)\right| \le C e^{-t_1 \rho \cos \frac{\theta}{2}} \rho^l,
	\end{equation}
	since $\left|\frac{\Phi^\dagger\left(\rho e^{i\left(\pi-\frac{\theta}{2}\right)}\right)}{\rho}\right|^n$ is bounded for $\rho \ge \varepsilon$ by assumption \eqref{eq:uniformlimcond}. Observe that the right-hand side of \eqref{eq:contrder2} is integrable over $(\varepsilon,+\infty)$. Hence, by \eqref{eq:upboundovereps} and \eqref{eq:contrder2} and the fact that $l \ge 1$ is arbitrary, we can differentiate $l$ times inside the integrals in \eqref{convcode}, getting	
	\begin{equation}
		\frac{d^r}{dt^r} \bar{\mu}_\Phi^{\ast n} (t) 
		= \frac{1}{\pi} \int_{\varepsilon}^{+\infty} \Im \left[  \frac{\lb\Phi^\dagger \lb \rho e^{i \lb \pi-\frac{\theta}{2} \rb} \rb\rb^n}{\rho^{n-r}e^{i(n-r-1)\lb \pi-\frac{\theta}{2} \rb}}  e^{t\rho e^{i \lb \pi-\frac{\theta}{2} \rb}} \right] d\rho +\frac{1}{2\pi i} \int_{\gamma_{\varepsilon,\theta}} e^{tz}  \frac{\lb\phi^\dagger (z)\rb^n}{z^{n-r}}  dz.
		\label{derivatecode}
	\end{equation}
	This proves that $\bar{\mu}_\phi^{\ast n} \in C^\infty(0,+\infty)$.
	
	Now let us prove that $f_\Phi \in C^\infty(\mathbb{D})$. To do this, fix any $k,l \ge 0$, $0<t_1<t_2$ and $0<x_1<x_2<t_1/\mathfrak{b}$, recalling that any $(x,t) \in \mathbb{D}$ admits a compact neighbourhood of the form $[x_1,x_2]\times [t_1,t_2]$ specified before. Let $F(z;x,t)=\frac{\phi^\dagger(z)}{z}e^{-x\phi(z)+tz}$ and observe that
	\begin{equation*}
		\frac{\partial^k}{\partial x^k}\frac{\partial^l}{\partial t^l}F(z;x,t)=(-1)^k\frac{z^l\phi^\dagger(z)(\phi(z))^k}{z}e^{-x\phi(z)+tz}.
	\end{equation*} 
	The latter is continuous over $\gamma_{\varepsilon,\theta} \times [x_1,x_2]\times [t_1,t_2]$ and then
	\begin{equation}\label{eq:smoothf1}
		\left|\frac{\partial^k}{\partial x^k}\frac{\partial^l}{\partial t^l}F(z;x,t)\right|\le \max_{(z,x,t) \in \gamma_{\varepsilon,\theta} \times [x_1,x_2]\times [t_1,t_2]}\left|\frac{z^l\phi^\dagger(z)(\phi(z))^k}{z}e^{-x\phi(z)+tz}\right|,
	\end{equation}
	where the right-hand side is a constant and then it is integrable over $\gamma_{\varepsilon,\theta}$. Now set
	\begin{equation*}
		G(\rho;x,t)=\Im\left(\frac{\Phi^\dagger\lb \rho e^{i\left(\pi-\frac{\theta}{2}\right)}\rb}{\rho } e^{-x \Phi \lb \rho e^{i\left(\pi-\frac{\theta}{2}\right)}  \rb+ t \rho e^{i\left(\pi-\frac{\theta}{2}\right)} }\right)
	\end{equation*}
	and observe that
	\begin{equation*}
		\frac{\partial^k}{\partial x^k}\frac{\partial^l}{\partial t^l}G(\rho;x,t)=(-1)^k\Im\left(\frac{\Phi^\dagger\lb \rho e^{i\left(\pi-\frac{\theta}{2}\right)}\rb\left(\Phi\left(\rho e^{i\left(\pi-\frac{\theta}{2}\right)}\right)\right)^k}{\rho^{k+1}}\rho^{l+k} e^{il\left(\pi-\frac{\theta}{2}\right)}e^{-x \Phi \lb \rho e^{i\left(\pi-\frac{\theta}{2}\right)}  \rb+ t \rho e^{i\left(\pi-\frac{\theta}{2}\right)} }\right).
	\end{equation*}
	Recall that $\left|\frac{\Phi^\dagger\lb \rho e^{i\left(\pi-\frac{\theta}{2}\right)}\rb\left(\Phi\left(\rho e^{i\left(\pi-\frac{\theta}{2}\right)}\right)\right)^k}{\rho^{k+1}}\right|$ is bounded as $\rho \ge \varepsilon$, and set $p,p^\prime >1$ as right after \eqref{firsttozerogen11}. Arguing as in the proof of Proposition \ref{prop:intreptheta}, we know that there exists $M>0$ such that $|g(z)| \le M$ for all $z \in \overline{\mathbb{C}\left(\frac{\pi}{2},\pi-\frac{\theta}{2}\right)}$, where $g$ is defined in \eqref{eq:goftheproof} with $t_2$ and $x_1$ in place of $t$ an $x$. Hence, we have
	\begin{equation}\label{smoothfphi2}
		\left|\frac{\partial^k}{\partial x^k}\frac{\partial^l}{\partial t^l}G(\rho;x,t)\right|\le CM e^{-\frac{\rho t_1}{p^\prime}\cos \frac{\theta}{2}},
	\end{equation}
	where the right-hand side is integrable on $(\varepsilon,+\infty)$. Hence, since $k,l \ge 0$ are arbitrary, we can differentiate $k$ times in $x$ and $l$ times with respect to $t$ in \eqref{statement1}, to get that $f_\Phi \in C^\infty(\mathbb{D})$ and
	\begin{align}\label{derivatefphi}
		\begin{split}
			\frac{\partial^k}{\partial x^k}\frac{\partial^l}{\partial t^l}f_\phi(x,t)&= \frac{(-1)^k}{\pi} \int_\varepsilon^{+\infty}\Im\left(\frac{\Phi^\dagger\lb \rho e^{i\left(\pi-\frac{\theta}{2}\right)}\rb\lb\Phi\lb \rho e^{i\left(\pi-\frac{\theta}{2}\right)}\rb\rb^{k}}{\rho^{1-l} e^{-i l \left(\pi-\frac{\theta}{2}\right)}}e^{-x \Phi \lb \rho e^{i\left(\pi-\frac{\theta}{2}\right)}  \rb+ t \rho e^{i\left(\pi-\frac{\theta}{2}\right)} }\right) \, d\rho \\
			&+\frac{(-1)^k}{2\pi i}\int_{\gamma_{\varepsilon,\theta}} \frac{\Phi^\dagger(z)(\Phi(z))^{k}}{z^{1-l}}e^{-x\Phi(z)+tz}dz.
		\end{split} 
	\end{align}
\end{proof}

\subsection{Proof of Theorem \ref{thm:seriespi}}
In order to prove Theorem \ref{thm:seriespi} we employ the integral representations given in Propositions \ref{prop:intreptheta} and \ref{lem:convtail}.
\begin{proof}[Proof of Theorem \ref{thm:seriespi}]
	As in the \eqref{derivatefphi}, we have
	\begin{align}	
		\begin{split}
			\frac{\partial^k}{\partial x^k}\frac{\partial^l}{\partial t^l}f_\phi(x,t)&= \frac{(-1)^k}{\pi} \int_\varepsilon^{+\infty}\Im\left(\frac{\Phi^\dagger\lb \rho e^{i\left(\pi-\frac{\theta}{2}\right)}\rb\lb\Phi\lb \rho e^{i\left(\pi-\frac{\theta}{2}\right)}\rb\rb^{k}}{\rho^{1-l} e^{-i l \left(\pi-\frac{\theta}{2}\right)}}e^{-x \Phi \lb \rho e^{i\left(\pi-\frac{\theta}{2}\right)}  \rb+ t \rho e^{i\left(\pi-\frac{\theta}{2}\right)} }\right) \, d\rho \\
			&+\frac{(-1)^k}{2\pi i}\int_{\gamma_{\varepsilon,\theta}} \frac{\Phi^\dagger(z)(\Phi(z))^{k}}{z^{1-l}}e^{-x\Phi(z)+tz}dz.
		\end{split}
	\end{align}
	Writing $e^{-x\Phi(z)}$ as a power series and assuming we can exchange the series with the integral, we have
	\begin{align*}	
		&\frac{\partial^k}{\partial x^k}\frac{\partial^l}{\partial t^l}f_\phi(x,t)= \sum_{j=0}^{+\infty}(-1)^{k+j}\frac{x^j}{j!}\\
		&\times \left[\frac{1}{\pi} \int_\varepsilon^{+\infty}\Im\left(\frac{\Phi^\dagger\lb \rho e^{i\left(\pi-\frac{\theta}{2}\right)}\rb\lb\Phi\lb \rho e^{i\left(\pi-\frac{\theta}{2}\right)}\rb\rb^{k+j}}{\rho^{1-l} e^{-i l \left(\pi-\frac{\theta}{2}\right)}}e^{t \rho e^{i\left(\pi-\frac{\theta}{2}\right)} }\right) \, d\rho+\frac{1}{2\pi i}\int_{\gamma_{\varepsilon,\theta}} \frac{\Phi^\dagger(z)(\Phi(z))^{k+j}}{z^{1-l}}e^{tz}dz\right]\\
		&=\sum_{j=0}^{+\infty}\sum_{k_1+k_2+k_3=k+j} \frac{(k+j)!}{k_1!k_2!k_3!}(-1)^{k+j}\frac{x^j}{j!}q^{k_1}\mathfrak{b}^{k_2}\\
		&\times \left[\frac{1}{\pi} \int_\varepsilon^{+\infty}\Im\left(\frac{\left(\Phi^\dagger\lb \rho e^{i\left(\pi-\frac{\theta}{2}\right)}\rb\right)^{k_3+1}}{\rho^{1-l-k_2} e^{-i (l+k_2) \left(\pi-\frac{\theta}{2}\right)}}e^{t \rho e^{i\left(\pi-\frac{\theta}{2}\right)} }\right) \, d\rho+\frac{1}{2\pi i}\int_{\gamma_{\varepsilon,\theta}} \frac{(\Phi^\dagger(z))^{k_3+1}}{z^{1-l-k_2}}e^{tz}dz\right]\\
		&=\sum_{j=0}^{+\infty}\sum_{k_1+k_2+k_3=k+j} \frac{(k+j)!}{k_1!k_2!k_3!}(-1)^{k+j}\frac{x^j}{j!}q^{k_1}\mathfrak{b}^{k_2}\\
		&\times \left[\frac{1}{\pi} \int_\varepsilon^{+\infty}\Im\left(\frac{\left(\Phi^\dagger\lb \rho e^{i\left(\pi-\frac{\theta}{2}\right)}\rb\right)^{k_3+1}}{\rho^{k_3+1-(k_2+k_3+l)} e^{i (k_3-(l+k_2+k_3)) \left(\pi-\frac{\theta}{2}\right)}}e^{t \rho e^{i\left(\pi-\frac{\theta}{2}\right)} }\right) \, d\rho+\frac{1}{2\pi i}\int_{\gamma_{\varepsilon,\theta}} \frac{(\Phi^\dagger(z))^{k_3+1}}{z^{k_3+1-(k_2+k_3+l)}}e^{tz}dz\right]\\
		&=\sum_{j=0}^{+\infty}\sum_{k_1+k_2+k_3=k+j} \frac{(k+j)!}{k_1!k_2!k_3!}(-1)^{k+j}\frac{x^j}{j!}q^{k_1}\mathfrak{b}^{k_2}\frac{d^{k_2+k_3+l}}{d t^{k_2+k_3+l}}\mu^{\ast (k_3+1)}(t),
	\end{align*} 
	where we used \eqref{derivatecode} in the last equality. Now we only have to prove that we can exchange the series with the integrals. This is clear for the integral over $\gamma_{\varepsilon, \theta}$, thus let us only consider the one over $(\varepsilon,+\infty)$.
	%
	%We use \eqref{diffint} to compute the series representation, as follows (any step is justified at the end of computation)
	%\begin{align}
	%&\frac{\partial^l}{\partial t^l}\frac{\partial^k}{\partial x^k}  f_\Phi(x,t) \notag \\
	% = \, &  \frac{\partial^l}{\partial t^l} \frac{\partial^k}{\partial x^k} \left[\frac{1}{\pi} \Im \l \int_\varepsilon^{+\infty}    \frac{\Phi^\dagger \l \rho e^{i\l \pi-\frac{\theta}{2} \r} \r}{\rho}   e^{-x\Phi \l \rho e^{i \l \pi-\frac{\theta}{2} \r}  \r+t\rho e^{i \l \pi-\frac{\theta}{2} \r}} d\rho \r - \frac{1}{2\pi i} \int_{\gamma_\varepsilon} \frac{\Phi^\dagger (z)}{z}  e^{-x\Phi(z)+tz} dz \right] \label{impartandintegral} \\
	%   = \, & \frac{\partial^l}{\partial t^l} \frac{\partial^k}{\partial x^k}  \sum_{j=0}^{+\infty} (-1)^j\frac{x^j}{j!} \left[ \frac{1}{\pi}\Im \l \int_\varepsilon^{+\infty}    \frac{\Phi^\dagger \l \rho e^{i\l \pi-\frac{\theta}{2} \r} \r}{\rho} \Phi^{j}\l \rho e^{i \l \pi-\frac{\theta}{2} \r} \r   e^{t\rho e^{i \l \pi-\frac{\theta}{2} \r}}  d\rho \r \right. \notag \\
	%   & \left. - \frac{1}{2\pi i} \int_{\gamma_\varepsilon} \frac{\Phi^\dagger (z)}{z} \Phi^j (z)  e^{tz} dz \right] \label{serieswithintandim} \\
	%   = \, & \frac{\partial^l}{\partial t^l} \sum_{j=0}^{+\infty} \frac{x^j}{j!}(-1)^{j+k} \left[ \frac{1}{\pi}\Im \l\int_\varepsilon^{+\infty}  \Phi^\dagger \l \rho e^{i \l \pi-\frac{\theta}{2} \r} \r \rho^{-1} \phi^{j+k} \l \rho e^{i \l \pi-\frac{\theta}{2} \r} \r e^{t\rho e^{i \l \pi-\frac{\theta}{2} \r}}  d\rho \r \right. \notag \\
	%    & \left. - \frac{1}{2\pi i} \int_{\gamma_\varepsilon} \Phi^\dagger (z) \Phi^{j+k} (z) z^{-1} e^{tz} dz \right]\label{derpowseries} \\
	%   = \, & \frac{\partial^l}{\partial t^l} \sum_{j=0}^{+\infty} \frac{x^j}{j!}(-1)^{j+k} \left[ \frac{1}{\pi}\int_\varepsilon^{+\infty} \Im \l \Phi^\dagger \l \rho e^{i \l \pi-\frac{\theta}{2} \r} \r \rho^{-1} \phi^{j+k} \l \rho e^{i \l \pi-\frac{\theta}{2} \r} \r e^{t\rho e^{i \l \pi-\frac{\theta}{2} \r}} \r d\rho \right. \notag \\
	%    & \left. - \frac{1}{2\pi i} \int_{\gamma_\varepsilon} \Phi^\dagger (z) \Phi^{j+k} (z) z^{-1} e^{tz} dz \right] \label{imint2} \\
	%       = \, &   \sum_{j=0}^{+\infty} \frac{x^j}{j!}(-1)^{j+k} \left[\frac{1}{\pi} \int_\varepsilon^{+\infty} \Im \l \Phi^\dagger \l \rho e^{i \l \pi-\frac{\theta}{2} \r} \r \rho^{l-1} e^{il\l \pi-\frac{\theta}{2} \r} \phi^{j+k} \l \rho e^{i \l \pi-\frac{\theta}{2} \r} \r e^{t\rho e^{i \l \pi-\frac{\theta}{2} \r}} \r d\rho \right. \notag \\
	%    & \left. - \frac{1}{2\pi i} \int_{\gamma_\varepsilon} \Phi^\dagger (z) \Phi^{j+k} (z) z^{l-1} e^{tz} dz \right] \label{dertdentro}\notag \\
	%    = \, &  \sum_{j=0}^{+\infty} \frac{x^j}{j!}(-1)^{j+k}\sum_{k_1+k_2+k_3=k+j} q^{k_1} \mathfrak{b}^{k_2} \left[ \frac{1}{\pi} \int_{\varepsilon}^{+\infty} \Im  \l    \l \Phi^\dagger \l \rho e^{i \l \pi-\frac{\theta}{2} \r} \r \r^{k_3+1}\rho^{k_2+l-1} \right. \right.  \\ & \times \left. \left.  e^{i(k_2+l) \l \pi-\frac{\theta}{2} \r} e^{t\rho e^{i \l \pi-\frac{\theta}{2} \r}}  \r d\rho  -\frac{1}{2\pi i} \int_{\gamma_{\varepsilon}}   \l \Phi^\dagger(z) \r^{k_3+1} z^{k_2+l-1} e^{tz} dz \right] \label{multin}\\
	% = \, &   \sum_{j=0}^{+\infty}  \frac{x^j}{j!} (-1)^{j+k} \sum_{k_1+k_2+k_3=k+j} q^{k_1}\mathfrak{b}^{k_2} \frac{d^{l+k_2+k_3}}{dt^{l+k_2+k_3}} \bar{\nu}_\Phi^{\star(k_3+1)}(t).
	%\end{align}
	%Now we justify, step by step, the previous computation.
	%In the first step (eq \eqref{impartandintegral}), we interchanged the integral and the imaginary part. This is justified since
	%\begin{align*}
	%		&\left|  \frac{\Phi^\dagger \l \rho e^{i\l \pi-\frac{\theta}{2} \r} \r}{\rho}    e^{-x\Phi \l \rho e^{i \l \pi-\frac{\theta}{2} \r}  \r+t\rho e^{i \l \pi-\frac{\theta}{2} \r}}  \right| \notag \\
	%		\leq \, &  e^{-qx} \left| \rho^{-1} \Phi^\dagger \l \rho e^{i \l \pi-\frac{\theta}{2} \r} \r \right| \exp \l - \rho \cos \frac{\theta}{2} \l t-\mathfrak{b}x+x\frac{\Re \Phi^\dagger \l \rho e^{i \l \pi-\frac{\theta}{2} \r} \r}{\rho \cos \frac{\theta}{2}} \r\r, 
	%\end{align*}
	%which is integrable at infinity as seen in the proof of Proposition \ref{prop:intreptheta} (see eq. \eqref{richiamatadopo}). In the second (eq. \eqref{serieswithintandim}) step we expanded the exponential into series and then we interchanged the series with both the integral and the imaginary part. Then the series of the integrals can be written as the series of the sum of the integrals by absolute convergence.
	Indeed, we have
	\begin{align}
		& \int_\varepsilon^{+\infty} \sum_{j=0}^{+\infty}\left|   (-1)^{k+j} \frac{ \Phi^\dagger\lb \rho e^{i\lb \pi-\frac{\theta}{2} \rb} \rb \lb \Phi\lb \rho e^{i\lb \pi-\frac{\theta}{2} \rb} \rb \rb^{{k+j}}}{j! \; \rho^{1-l}e^{-il\left(\pi-\frac{\theta}{2}\right)}} \; x^j \; e^{t  \rho e^{i \lb \pi-\frac{\theta}{2} \rb}} \right| d\rho  \notag \\
		= \,& \int_\varepsilon^\infty \frac{\left| \Phi^\dagger\lb \rho e^{i\lb \pi-\frac{\theta}{2} \rb} \rb \right|\left|\Phi\left(\rho e^{i\lb \pi-\frac{\theta}{2} \rb}\right)\right|^k}{\rho^{k+1}}\rho^{l+k} e^{x \left| \Phi \lb \rho e^{i \lb \pi-\frac{\theta}{2} \rb} \rb \right|-t\rho \cos \frac{\theta}{2}} d\rho \notag \\
		\leq \, & e^{xq}\int_\varepsilon^{+\infty} \frac{\left| \Phi^\dagger\lb \rho e^{i\lb \pi-\frac{\theta}{2} \rb} \rb \right|\left|\Phi\left(\rho e^{i\lb \pi-\frac{\theta}{2} \rb}\right)\right|^k}{\rho^{k+1}}\rho^{l+k} e^{x\mathfrak{b}\rho\cos\left(\frac{\theta}{2}\right)-t\rho \cos \frac{\theta}{2}+\Phi^\dagger\lb \rho e^{i \lb \pi-\frac{\theta}{2} \rb} \rb} d\rho.	\label{248}
	\end{align}
	It is clear that we have to check integrability in the right-hand side of \eqref{248} only in a neighbourhood of infinity. To do this, set $\delta=t-\mathfrak{b}x$ and $p,p^\prime>1$ as in the proof of Proposition \ref{prop:intreptheta}. By \eqref{eq:uniformlimcond} we know that there exists $K$ big enough such that $\frac{\left|\phi^\dagger\lb \rho e^{i \lb \pi-\frac{\theta}{2} \rb} \rb \right|\left|\phi\lb \rho e^{i \lb \pi-\frac{\theta}{2} \rb} \rb \right|^k}{\rho^{k+1}}$ is bounded and $\frac{\left|\phi^\dagger\lb \rho e^{i \lb \pi-\frac{\theta}{2} \rb} \rb \right|}{\rho \cos(\theta/2)}<\frac{\delta}{4}$, whenever $\rho >K$. Hence we get
	%
	%Let $\delta=t-\mathfrak{b}x>0$ and consider $p > 1$ such that $\frac{t}{p}-\mathfrak{b}x>\frac{\delta}{2}$. Let $q_* \ge 1$ such that $\frac{1}{p}+\frac{1}{q_*}=1$ and rewrite the integral in \eqref{248} as
	%\begin{equation}	\label{nomodcont}
	%	\int_\varepsilon^{+\infty} \frac{\Phi^\dagger \l \rho e^{i \l \pi-\frac{\theta}{2} \r} \r}{\rho} e^{-\frac{t}{q_*}\rho \cos\left(\frac{\theta}{2}\right)}\exp\left(-\rho \cos\left(\frac{\theta}{2}\right)\left(\frac{t}{p}-x\mathfrak{b}-\frac{\left|\Phi^\dagger\l \rho e^{i \l \pi-\frac{\theta}{2} \r} \r \right|}{\rho \cos\left(\frac{\theta}{2}\right)}\right)\right) d\rho.
	%\end{equation}
	%
	%
	% Hence, we can split the integral in \eqref{nomodcont} as follows
	%\begin{align}
	%&\int_\varepsilon^{\infty} \frac{\Phi^\dagger \l \rho e^{i \l \pi-\frac{\theta}{2} \r} \r}{\rho} e^{-\frac{t}{q_*}\rho \cos\left(\frac{\theta}{2}\right)}\exp\!\!\left(-\rho \cos\left(\frac{\theta}{2}\right)\left(\frac{t}{p}-x\mathfrak{b}-\frac{\left|\Phi^\dagger\l \rho e^{i \l \pi-\frac{\theta}{2} \r} \r \right|}{\rho \cos\left(\frac{\theta}{2}\right)}\right)\right)\!\! d\rho \notag  \\
	%	&=\!\!\int_{\varepsilon}^K \frac{\Phi^\dagger \l \rho e^{i \l \pi-\frac{\theta}{2} \r} \r}{\rho} e^{-\frac{t}{q_*}\rho \cos\left(\frac{\theta}{2}\right)}\exp\!\!\left(-\rho \cos\left(\frac{\theta}{2}\right)\left(\frac{t}{p}-x\mathfrak{b}-\frac{\left|\Phi^\dagger\l \rho e^{i \l \pi-\frac{\theta}{2} \r} \r \right|}{\rho \cos\left(\frac{\theta}{2}\right)}\right)\right)\!\!d\rho \notag  \\
	%		&+\!\!\int_{K}^{\infty}\!\! \frac{\Phi^\dagger \l \rho e^{i \l \pi-\frac{\theta}{2} \r} \r}{\rho} e^{-\frac{t}{q_*}\rho \cos\left(\frac{\theta}{2}\right)}\exp\!\!\left(-\rho \cos\left(\frac{\theta}{2}\right)\left(\frac{t}{p}-x\mathfrak{b}-\frac{\left|\Phi^\dagger\l \rho e^{i \l \pi-\frac{\theta}{2} \r} \r \right|}{\rho \cos\left(\frac{\theta}{2}\right)}\right)\right)\!\!d\rho.		
	%		\label{splitted}
	%\end{align}
	%The first integral in \eqref{splitted} is clearly convergent. For the second, by \eqref{eq:uniformlimcond} and since
	%\begin{equation*}
	%\left| \Phi^\dagger \l \rho e^{i \l \pi-\frac{\theta}{2} \r} \r \right| / \rho < \delta/ 4
	%\end{equation*}
	%we have that
	\begin{align}
		\begin{split}
			&\int_{K}^{+\infty}\frac{\left|\phi^\dagger\lb \rho e^{i \lb \pi-\frac{\theta}{2} \rb} \rb \right|\left|\phi\lb \rho e^{i \lb \pi-\frac{\theta}{2} \rb} \rb \right|^k}{\rho^{k+1}} e^{-\frac{t}{p^\prime}\rho \cos\left(\frac{\theta}{2}\right)}\exp\left(-\rho \cos\left(\frac{\theta}{2}\right)\left(\frac{t}{p}-x\mathfrak{b}-\frac{\left|\Phi^\dagger\lb \rho e^{i \lb \pi-\frac{\theta}{2} \rb} \rb \right|}{\rho \cos\left(\frac{\theta}{2}\right)}\right)\right) d\rho \notag \\
			\leq \,& C \int_K^{+\infty} \rho^{k+l}e^{-\frac{t}{p^\prime}\rho \cos \lb \frac{\theta}{2} \rb}\red{d\rho} < +\infty.
		\end{split}
	\end{align}
	This concludes the proof.
	\end{proof}


\subsection{Proof of Theorem \ref{behavatzero} }
The behaviour at zero provided in Theorem \ref{behavatzero} can be shown as a direct consequence of the series representation given in Theorem \ref{thm:seriespi}.
\begin{proof}[Proof of Theorem \ref{behavatzero}]
	Let $[t_1,t_2] \subset (0,+\infty)$ and observe that for $t \in [t_1,t_2]$ we have by \eqref{derivatecode}
	\begin{equation}\label{eq:estIjkl}
		\begin{split}
			|\mathcal{I}_{j,k,l}(t)| &\le \frac{1}{\pi}\int_\varepsilon^{+\infty}\frac{\left|\Phi^{\dagger}\left(\rho e^{i\left(\pi-\frac{\theta}{2}\right)}\right)\right|\left|\Phi\left(\rho e^{i\left(\pi-\frac{\theta}{2}\right)}\right)\right|^{j+k}}{\rho^{1-l}}e^{-t_1\rho \cos\left(\frac{\theta}{2}\right)}d\rho  \\
			&+ \frac{\varepsilon^{l}}{2\pi} \int_{\frac{\theta}{2}-\pi}^{\pi-\frac{\theta}{2}}|\Phi^\dagger (\varepsilon e^{i\varphi})||\phi(\varepsilon e^{i\varphi})|^{j+k} e^{\varepsilon t_2 |\cos(\varphi)|}d\varphi.
		\end{split}
	\end{equation}
	Hence, for $0<x<\frac{t_1}{\mathfrak{b}}$ and $t \in [t_1,t_2]$, recalling the definition of $\mathcal{P}_{n,k,l}(x,t)$ given in \eqref{eq:zero2}, we have
%$	 \in [t_1,t_2]$ we have 
	\begin{align}
		\left|\frac{\partial^k \partial^l}{\partial x^k \partial t^l}f_\Phi(x,t)-\mathcal{P}_{n,k,l}(x,t)\right|  \le & \sum_{j=n+1}^{+\infty}\frac{x^j}{j!}|\mathcal{I}_{j,k,l}(t)| \notag \\
		%\leq \, & \sum_{j=n+1}^{+\infty} \frac{x^j}{j!} \left[ \frac{1}{\pi}\int_\varepsilon^{+\infty}\frac{\left|\Phi^{\dagger}\left(\rho e^{i\left(\pi-\frac{\theta}{2}\right)}\right)\right|\left|\Phi\left(\rho e^{i\left(\pi-\frac{\theta}{2}\right)}\right)\right|^{j+k}}{\rho^{1-l}}e^{-t_1\rho \cos\left(\frac{\theta}{2}\right)}d\rho \right.\notag \\
		%&+ \left.\frac{\varepsilon^{l}}{2\pi} \int_{\frac{\theta}{2}-\pi}^{\pi-\frac{\theta}{2}}|\Phi^\dagger (\varepsilon e^{i\varphi})||\phi(\varepsilon e^{i\varphi})|^{j+k} e^{\varepsilon t_2 |\cos(\varphi)|}d\varphi \right]\\
		\leq \, & \frac{x^{n+1}}{(n+1)!}\sum_{j=0}^{+\infty} \frac{x^j}{j!} \left[ \frac{1}{\pi}\int_\varepsilon^{+\infty}\frac{\left|\Phi^{\dagger}\left(\rho e^{i\left(\pi-\frac{\theta}{2}\right)}\right)\right|\left|\Phi\left(\rho e^{i\left(\pi-\frac{\theta}{2}\right)}\right)\right|^{j+k+n+1}}{\rho^{1-l}}e^{-t_1\rho \cos\left(\frac{\theta}{2}\right)}d\rho \right.\notag \\
		&+ \left.\frac{\varepsilon^{l}}{2\pi} \int_{\frac{\theta}{2}-\pi}^{\pi-\frac{\theta}{2}}|\Phi^\dagger (\varepsilon e^{i\varphi})||\phi(\varepsilon e^{i\varphi})|^{j+k+n+1} e^{\varepsilon t_2 |\cos(\varphi)|}d\varphi \right]
		%\textcolor{red}{d\rho}.
		\label{laststep}
	\end{align}
	where, in the last step, we used \eqref{eq:estIjkl}. The convergence of the series in \eqref{laststep} can be ascertained as in \eqref{248}. Taking the supremum in $[t_1,t_2]$ in \eqref{laststep}, we get \eqref{eq:zero1}.
	% result then follows by letting $x \to 0$ in \eqref{laststep}.
\end{proof}


\subsection{Proof of Proposition \ref{prop:extcont}}
A slight modification of the arguments used in the proofs of Proposition \ref{lem:convtail} and Theorem \ref{thm:smoothfmu} leads to the desired result.
\begin{proof}[Proof of Proposition \ref{prop:extcont}]
	Since $\phi$ is a complete Bernstein function that can be extended by continuity over $\overline{\C(0,\pi)}$ with extension $\phi_+$, then, by using the relation $\overline{\phi(z)}=\phi(\overline{z})$, it is clear that it can be also extended by continuity over $\overline{\C(-\pi,0)}$. If we denote such an extension $\phi_-$, for any $z \in \overline{\C(0,\pi)}$ we get
	\begin{equation*}
		\overline{\phi_+(z)}=\phi_-(\overline{z}).
	\end{equation*}
	The proof is then carried on exactly as in Propositions \ref{prop:intreptheta} and \ref{lem:convtail}, by setting $\theta=0$ (see Figure \ref{fig2}), where we use $\phi^\dagger$ when we integrate over $\ell_1$, $\Gamma_R^+$, $\Gamma_R^-$ and $-\gamma_\varepsilon$, $\phi^\dagger_+$ over $\ell_2$ and $\phi^\dagger_-$ over $\ell_1$.
\end{proof}

\subsection{Proof of Theorem \ref{thm:seriessub}}
In order to prove Theorem \ref{thm:seriessub} we first need to provide an integral representation for $G_\phi$, $g_\phi$ and its derivatives, analogously to what we did for Theorems \ref{thm:smoothfmu} and \ref{thm:seriespi}.
This is done in the following proposition, whose proof is almost identical to the one of Proposition \ref{prop:intreptheta} and thus is omitted.
\begin{prop}
	Let $\Phi$ be the Laplace exponent of a potentially killed subordinator satisfying assumptions \ref{eq:extensionA3} and \eqref{eq:uniformlimcond} for some $\theta \in (0,\pi)$. Fix any $\varepsilon>0$ and let $\gamma_{\varepsilon,\theta}$ be defined as in Proposition \ref{prop:intreptheta}. Then, on $\mathbb{D}$,
	\begin{equation}\label{statementG}
		\begin{split}
			G_\Phi (x,t) \, = \,& \frac{1}{\pi} \int_\varepsilon^{+\infty}\Im\left(\frac{e^{-x \Phi \lb \rho e^{i\left(\pi-\frac{\theta}{2}\right)}  \rb+ t \rho e^{i\left(\pi-\frac{\theta}{2}\right)} }}{\rho} \right) \, d\rho  +\frac{1}{2\pi i}\int_{\gamma_{\varepsilon,\theta}} \frac{e^{-x\Phi(z)+tz}}{z}dz.
		\end{split}
	\end{equation}
	In particular, $G_\Phi \in C^\infty(\mathbb{D})$, $g_\Phi \in C^\infty(\mathbb{D})$ is well defined and for any $k,l \ge 0$ we have
	\begin{equation}\label{statementG2}
		\begin{split}
			\frac{\partial^k}{\partial x^k}\frac{\partial^l}{\partial t^l}g_\Phi (x,t) \, = \,& \frac{1}{\pi} \int_\varepsilon^{+\infty}\Im\left(\left(\Phi\left(\rho e^{i\left(\pi-\frac{\theta}{2}\right)}\right)\right)^k\rho^le^{il\left(\pi-\frac{\theta}{2}\right)}e^{-x \Phi \lb \rho e^{i\left(\pi-\frac{\theta}{2}\right)}  \rb+ t \rho e^{i\left(\pi-\frac{\theta}{2}\right)} } \right) \, d\rho  \\
			&+\frac{1}{2\pi i}\int_{\gamma_{\varepsilon,\theta}} (\Phi(z))^kz^le^{-x\Phi(z)+tz}dz.
		\end{split}
	\end{equation}
\end{prop}
We employ the latter together with Proposition \ref{lem:convtail} in the following proof.

\begin{proof}[Proof of Theorem \ref{thm:seriessub}]
	Let us first consider $G_\Phi$. Starting from \eqref{statementG}, we have, assuming that we can exchange the order of the series and the integral,
	\begin{align}
		G_\Phi(x,t)
		&=\sum_{j=0}^{+\infty}(-1)^j\frac{x^j}{j!}\left[\frac{1}{\pi}\int_{\varepsilon}^{+\infty}\Im\left(\frac{\left(\Phi\left(\rho e^{i\left(\pi-\frac{\theta}{2}\right)}\right)\right)^j}{\rho}e^{t \rho e^{i\left(\pi-\frac{\theta}{2}\right)}}\right)d\rho+\frac{1}{2\pi i}\int_{\gamma_{\varepsilon,\theta}} \frac{(\Phi(z))^j}{z}e^{tz}dz\right]\notag \\
		&=\frac{1}{\pi}\int_{\varepsilon}^{+\infty}\Im\left(\frac{1}{\rho} {e^{t \rho e^{i\left(\pi-\frac{\theta}{2}\right)}}}\right) \, d\rho+\frac{1}{2\pi i}\int_{\gamma_{\varepsilon,\theta}} \frac{1}{z} {e^{t z}} \, dz\notag\\
		&+\sum_{j=1}^{+\infty}(-1)^j\frac{x^j}{j!}\left[\frac{1}{\pi}\int_{\varepsilon}^{+\infty}\Im\left(\frac{\left(\Phi\left(\rho e^{i\left(\pi-\frac{\theta}{2}\right)}\right)\right)^j}{\rho}e^{t \rho e^{i\left(\pi-\frac{\theta}{2}\right)}}\right)d\rho+\frac{1}{2\pi i}\int_{\gamma_{\varepsilon,\theta}} \frac{(\Phi(z))^j}{z}e^{tz}dz\right]\notag\\
		& ={\frac{1}{\pi}\int_{\varepsilon}^{+\infty}\Im\left(\frac{1}{\rho} e^{t \rho e^{i\left(\pi-\frac{\theta}{2}\right)}}\right) \, d\rho+\frac{1}{2\pi i}\int_{\gamma_{\varepsilon,\theta}} \frac{1}{z} e^{t z} \, dz} \notag\\		
		&+\sum_{j=1}^{+\infty}\sum_{k_1+k_2+k_3=j-1}(-1)^j\frac{x^j}{k_1!k_2!(k_3+1)!}q^{k_1}\mathfrak{b}^{k_2}\notag\\
		&\times \left[\frac{1}{\pi}\int_{\varepsilon}^{+\infty}\Im\left(\frac{\left(\Phi^\dagger\left(\rho e^{i\left(\pi-\frac{\theta}{2}\right)}\right)\right)^{k_3+1}}{\rho^{k_3+1-(k_2+k_3)}e^{i(k_3-(k_2+k_3))}}e^{t \rho e^{i\left(\pi-\frac{\theta}{2}\right)}}\right)d\rho+\frac{1}{2\pi i}\int_{\gamma_{\varepsilon,\theta}} \frac{(\Phi^\dagger(z))^{k_3+1}}{z^{k_3+1-(k_2+k_3)}}e^{tz}dz\right]\notag\\
		& { + \sum_{j=1}^{+\infty}(-1)^j\frac{x^j}{j!}\left[\frac{1}{\pi}\int_{\varepsilon}^{+\infty}\Im\left(\frac{\left(q+\mathfrak{b}\rho e^{i\left(\pi-\frac{\theta}{2}\right)}\right)^j}{\rho}e^{t \rho e^{i\left(\pi-\frac{\theta}{2}\right)}}\right)d\rho+\frac{1}{2\pi i}\int_{\gamma_{\varepsilon,\theta}} \frac{(q+\mathfrak{b}z)^j}{z}e^{tz}dz\right]}
		\label{multinom}
	\end{align}
where in the last step we used the multinomial theorem, for $j \ge 1$, to expand $\left( \phi \left( \rho e^{i \left( \pi-\frac{\theta}{2} \right)} \right) \right)^j=\left( q+\mathfrak{b}\rho e^{i \left( \pi-\frac{\theta}{2} \right)}+\phi^\dagger \left( \rho e^{i \left( \pi-\frac{\theta}{2} \right)} \right) \right)^j$. Incorporating the first summand of \eqref{multinom} into the last summation, we achieve
	\begin{align*}
		G_\Phi(x,t)&=\sum_{j=0}^{+\infty}(-1)^j\frac{x^j}{j!}\left[\frac{1}{\pi}\int_{\varepsilon}^{+\infty}\Im\left(\frac{\left(q+\mathfrak{b}\rho e^{i\left(\pi-\frac{\theta}{2}\right)}\right)^j}{\rho}e^{t \rho e^{i\left(\pi-\frac{\theta}{2}\right)}}\right)d\rho+\frac{1}{2\pi i}\int_{\gamma_{\varepsilon,\theta}} \frac{(q+\mathfrak{b}z)^j}{z}e^{tz}dz\right]\\
		&+\sum_{j=1}^{+\infty}\sum_{k_1+k_2+k_3=j-1}(-1)^j\frac{x^j}{k_1!k_2!(k_3+1)!}q^{k_1}\mathfrak{b}^{k_2}\\
		&\times \left[\frac{1}{\pi}\int_{\varepsilon}^{+\infty}\Im\left(\frac{\left(\Phi^\dagger\left(\rho e^{i\left(\pi-\frac{\theta}{2}\right)}\right)\right)^{k_3+1}}{\rho^{k_3+1-(k_2+k_3)}e^{i(k_3-(k_2+k_3))}}e^{t \rho e^{i\left(\pi-\frac{\theta}{2}\right)}}\right)d\rho+\frac{1}{2\pi i}\int_{\gamma_{\varepsilon,\theta}} \frac{(\Phi^\dagger(z))^{k_3+1}}{z^{k_3+1-(k_2+k_3)}}e^{tz}dz\right]\\
		&=\sum_{j=0}^{+\infty}(-1)^j\frac{x^j}{j!}\left[\frac{1}{\pi}\int_{\varepsilon}^{+\infty}\Im\left(\frac{\left(q+\mathfrak{b}\rho e^{i\left(\pi-\frac{\theta}{2}\right)}\right)^j}{\rho}e^{t \rho e^{i\left(\pi-\frac{\theta}{2}\right)}}\right)d\rho+\frac{1}{2\pi i}\int_{\gamma_{\varepsilon,\theta}} \frac{(q+\mathfrak{b}z)^j}{z}e^{tz}dz\right]\\
		&+\sum_{j=1}^{+\infty}\sum_{k_1+k_2+k_3=j-1}(-1)^j\frac{x^j}{k_1!k_2!(k_3+1)!}q^{k_1}\mathfrak{b}^{k_2}\frac{d^{k_2+k_3}}{dt}\mu^{\ast(k_3+1)}(t)\\
		&{=:S_\varepsilon(x,t;q,\mathfrak{b})+\sum_{j=1}^{+\infty}\sum_{k_1+k_2+k_3=j-1}(-1)^j\frac{x^j}{k_1!k_2!(k_3+1)!}q^{k_1}\mathfrak{b}^{k_2}\frac{d^{k_2+k_3}}{dt}\mu^{\ast(k_3+1)}(t)}.
	\end{align*}
	We now evaluate $S_\varepsilon$. Consider the function $F_j(z;t)=\frac{(q+\mathfrak{b}z)^j}{z}e^{tz}$, that is holomorphic on $\C \setminus \{0\}$, Hermitian and admits a simple pole in $0$ with residue $q^j$. We have that
		\begin{align*}
			&\frac{1}{\pi}\int_{\varepsilon}^{+\infty}\Im\left(\frac{\left(q+\mathfrak{b}\rho e^{i\left(\pi-\frac{\theta}{2}\right)}\right)^j}{\rho}e^{t \rho e^{i\left(\pi-\frac{\theta}{2}\right)}}\right)d\rho+\frac{1}{2\pi i}\int_{\gamma_{\varepsilon,\theta}} \frac{(q+\mathfrak{b}z)^j}{z}e^{tz}dz\\
			&=\frac{1}{2\pi i}\left[\int_{\varepsilon}^{+\infty}F_j\left(\rho e^{i\left(\pi-\frac{\theta}{2}\right)};t\right) e^{i\left(\pi-\frac{\theta}{2}\right)}d\rho-\int_{\varepsilon}^{+\infty}F_j\left(\rho e^{i\left(\pi+\frac{\theta}{2}\right)};t\right) e^{i\left(\pi+\frac{\theta}{2}\right)}d\rho+\int_{\gamma_{\varepsilon,\theta}}F_j(z;t)dz\right].
		\end{align*}
		Taking into account the notation in Figure \ref{fig1}, denote by $-\Gamma_{R,\theta}$ the clockwise oriented circular arc joining $F(R)$ and $C(R)$, $-\ell_2$ the oriented segment joining $D(\varepsilon)$ to $C(R)$ and $-\ell_3$ the oriented segment joining $F(R)$ to $E(\varepsilon)$. Denote by $\mathfrak{D}_{R,\theta}$ the domain whose negatively oriented contour is given by $\gamma_{\varepsilon,\theta}$, $-\ell_2$, $-\Gamma_{R,\theta}$ and $-\ell_3$. Then, since $F_j$ is holomorphic on $\mathfrak{D}_{R,\theta}$, that is a simply connected domain, we have, by Cauchy's theorem
		\begin{equation}\label{eq:DReps}
			\int_{-\partial \mathfrak{D}_{R,\theta}}F_j(z;t)dz=0.
		\end{equation}
		On the other hand, if we denote by $\Gamma_R$ and $\gamma_\varepsilon$ the counterclockwise oriented circles of radius respectively $R$ and $\varepsilon$, we have by Cauchy's residue theorem
		\begin{equation*}
			\frac{1}{2\pi i}\int_{\Gamma_{R}}F_j(z;t)dz=\frac{1}{2\pi i}\int_{\gamma_{\varepsilon}}F_j(z;t)dz=q^j.
		\end{equation*}
		Hence, denoting by $A_{R,\varepsilon}$ the annulus delimited by the images of $\Gamma_R$ and $\gamma_\varepsilon$, we have
		\begin{equation}\label{eq:AReps}
			\int_{-\partial A_{R,\varepsilon}}F(z;t)dz=0=\int_{-\partial \mathfrak{D}_{R,\theta}}F_j(z;t)dz,
		\end{equation}
		where $-\partial A_{R,\varepsilon}$ is the negatively oriented contour of $A_{R,\varepsilon}$. If we use the parametrization of $-\ell_2$ and $\ell_3$ as $z=\rho e^{i\left(\pi\mp \frac{\theta}{2}\right)}$ for $\rho \in (\varepsilon,R)$, \eqref{eq:AReps} implies
		\begin{multline}\label{eq:AReps2}
			\int_{-\Gamma_R}F(z;t)dz+\int_{\gamma_\varepsilon}F(z;t)dz
			=\int_{\varepsilon}^{R}F_j\left(\rho e^{i\left(\pi-\frac{\theta}{2}\right)};t\right) e^{i\left(\pi-\frac{\theta}{2}\right)}d\rho\\-\int_{\varepsilon}^{R}F_j\left(\rho e^{i\left(\pi+\frac{\theta}{2}\right)};t\right) e^{i\left(\pi+\frac{\theta}{2}\right)}d\rho+\int_{\gamma_{\varepsilon,\theta}}F_j(z;t)dz+\int_{-\Gamma_{R,\theta}}F(z;t)dz.
		\end{multline}
		Furthermore, if we denote by $-\Gamma^\dagger_{R,\theta}$ the clockwise oriented circular arc joining $C(R)$ to $F(R)$ it is clear that
		\begin{equation*}
			\int_{-\Gamma_R}F(z;t)dz-\int_{-\Gamma_{R,\theta}}F(z;t)dz=\int_{-\Gamma^{\dagger}_{R,\theta}}F(z;t)dz
		\end{equation*}
		hence \eqref{eq:AReps2} leads to
		\begin{multline}\label{eq:AReps3}
			\int_{-\Gamma^\dagger_{R,\theta}}F(z;t)dz+\int_{\gamma_\varepsilon}F(z;t)dz\\
			=\int_{\varepsilon}^{R}F_j\left(\rho e^{i\left(\pi-\frac{\theta}{2}\right)};t\right) e^{i\left(\pi-\frac{\theta}{2}\right)}d\rho-\int_{\varepsilon}^{R}F_j\left(\rho e^{i\left(\pi+\frac{\theta}{2}\right)};t\right) e^{i\left(\pi+\frac{\theta}{2}\right)}d\rho+\int_{\gamma_{\varepsilon,\theta}}F_j(z;t)dz.
		\end{multline}
		By the estimation lemma we have
		\begin{equation*}
			\left|\int_{-\Gamma^\dagger_{R,\theta}}F(z;t)dz\right| \le \theta R \max_{z \in \Gamma^\dagger_{R,\theta}}|F(z;t)| \le \theta (q+\mathfrak{b}R)^je^{-t\cos\left(\frac{\theta}{2}\right)R}
		\end{equation*}
		hence, taking the limit as $R \to +\infty$ in \eqref{eq:AReps3} we get
		\begin{multline}
			2\pi i q^j=\int_{\gamma_\varepsilon}F(z;t)dz\\
			=\int_{\varepsilon}^{\infty}F_j\left(\rho e^{i\left(\pi-\frac{\theta}{2}\right)};t\right) e^{i\left(\pi-\frac{\theta}{2}\right)}d\rho-\int_{\varepsilon}^{\infty}F_j\left(\rho e^{i\left(\pi+\frac{\theta}{2}\right)};t\right) e^{i\left(\pi+\frac{\theta}{2}\right)}d\rho+\int_{\gamma_{\varepsilon,\theta}}F_j(z;t)dz.
		\end{multline}
		Substituting this value into $S_\varepsilon(x,t;q,\mathfrak{b})$ we get \eqref{eq:seriesdistr}.
	
	
	The other two series \eqref{eq:seriessub1} and \eqref{eq:seriessub2} are obtained analogously, once one notices that for any $j \ge 0$
	\begin{equation*}
		\frac{1}{\pi}\int_{\varepsilon}^{+\infty}\Im\left(\left(q+\mathfrak{b}\rho e^{i\left(\pi-\frac{\theta}{2}\right)}\right)^je^{t \rho e^{i\left(\pi-\frac{\theta}{2}\right)}}\right)d\rho+\frac{1}{2\pi i}\int_{\gamma_{\varepsilon,\theta}} (q+\mathfrak{b}z)^je^{tz}dz=\frac{1}{2\pi i}\int_{\gamma_\varepsilon}(q+\mathfrak{b}z)^je^{tz}dz=0,
	\end{equation*}
	since $(q+\mathfrak{b}z)^je^{tz}$ is holomorphic in the disc $\{z \in \C: \ |z|<\varepsilon\}$. Finally, the fact that one can actually exchange the series with the integral is proven exactly as in the proof of Theorem \ref{thm:seriespi}.
\end{proof}


%The following statement, which is of independent interest, plays a central role in the proof of Theorem \ref{thm:seriespi}, as it guarantees the smoothness of the $n$-fold convolution appearing in \eqref{coeff}.
%\begin{lem}
%	\label{lem:convtail}
%	Suppose that \eqref{eq:extensionA3} and \eqref{eq:uniformlimcond} are verified. Define, for a fixed $\varepsilon >0$, the set $\gamma_{\varepsilon,\theta} = \ll z \in \mathbb{C}: z=\varepsilon e^{i\xi}, \xi \in \left[ \frac{\theta}{2}-\pi, \pi- \frac{\theta}{2} \right] \rr$. Then we have, for any $n =1,2 \cdots$,  that
%	\begin{equation}\label{convcode}
	%		\bar{\mu}_\Phi^{n\star} (t) = \frac{1}{\pi} \int_{\varepsilon}^{+\infty} \Im \left[ \l \frac{\Phi^\dagger \l \rho e^{i \l \pi-\frac{\theta}{2} \r} \r}{\rho e^{i \l \pi-\frac{\theta}{2} \r}} \r^n e^{i \l \pi-\frac{\theta}{2} \r}e^{t\rho e^{i \l \pi-\frac{\theta}{2} \r}} \right] d\rho -\frac{1}{2\pi i} \int_{\gamma_{\varepsilon,\theta}} e^{tz} \l \frac{\phi^\dagger (z)}{z} \r^n dz,
	%	\end{equation}
%	for any $t>0$. Furthermore, it is true that the function $(0, +\infty) \mapsto \bar{\nu}_\phi^{\star n} (t)$, has, for any $n=1,2,\cdots$, derivatives of all order $r \geq 0$, such that
%	\begin{equation}
	%		\frac{d^r}{dt^r} \bar{\mu}_\Phi^{n\star} (t) 
	%		= \frac{1}{\pi} \int_{\varepsilon}^{+\infty} \Im \left[  \frac{\l\Phi^\dagger \l \rho e^{i \l \pi-\frac{\theta}{2} \r} \r\r^n}{\rho^{n-r}e^{i(n-r-1)\l \pi-\frac{\theta}{2} \r}}  e^{t\rho e^{i \l \pi-\frac{\theta}{2} \r}} \right] d\rho -\frac{1}{2\pi i} \int_{\gamma_{\varepsilon,\theta}} e^{tz}  \frac{\l\phi^\dagger (z)\r^n}{z^{n-r}}  dz.
	%		\label{derivatecode}
	%	\end{equation}
%\end{lem}



