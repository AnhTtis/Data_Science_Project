\section{Multi-class Classification Detection Method}
The proposed multi-class classification fall detection method is comprised of three algorithms, one for detecting falls caused by abrupt faults (abrupt fault detector), another for detecting falls caused by incipient faults (incipient fault detector) and a third for identifying the type of fault (fault type identifier). 

The fault type identifier is first trained and evaluated on the training data. Next, the abrupt fault detector and incipient fault detector are trained using the relevant trajectories and the windows of trajectories misclassified by the fault type identifier. The training lead time is determined similarly as in Section \ref{sec:fault_comparison}. 

% To meet the desired maximum threshold for the \fnr and \fprNS, the incipient and abrupt fault detectors are trained and evaluated on the validation and training data across various training F\_LTs. The maximum training F\_LT that meets the desired maximum threshold for \fnr and \fpr is kept. As the soft margin formulation is used for the classifier, the classifier can achieve a positive \fpr for the training data. If a small positive \fpr can be tolerated in order to increase the resulting F\_LT, the threshold for the nearest-neighbor algorithm can be relaxed.  

When detecting potential faults, we run all three algorithms in parallel. As we initially assume that every trajectory has an incipient fault, the output of the incipient fault detector is utilized by default. However, if the fault identifier classifies a trajectory as containing an abrupt fault, we start using the output of the abrupt fault detector. In other words, our null hypothesis is the incipient fault, while our alternative hypothesis is the abrupt fault. As a result, a delay in the fault identifier only results in a delay in the abrupt fault detector. When an abrupt fault is identified, the fault identifier is no longer used to identify the fault type until it is reset. Inherent in our implementation is that only one fault will be encountered per trajectory.  
