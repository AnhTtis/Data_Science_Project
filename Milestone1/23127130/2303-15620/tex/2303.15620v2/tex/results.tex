\subsection{Results of Proposed Multi-Class Classification Algorithm Across All Folds}

We train and evaluate the multi-class classification algorithm using the same parameters and metrics as described in Section \ref{sec:fault_comparison}. On average, across all folds, when trained using the features in \eqref{eq:features_main}, the multi-class classification algorithm achieves 0.06s and 0.05s additional average lead time across all folds for the nearest-neighbor and SVM classification algorithms, respectively. This results in an average lead time difference of 0.13s and 0.03s across all folds for the nearest-neighbor and SVM classification algorithms in comparison to their average when trained with the incipient and abrupt faults separately. Note that the SVM fault identifier has a delay of 0.07s or 3 data points, in detecting abrupt faults across all folds. The results are displayed in Table \ref{tab:multi_class_cluster}  and \ref{tab:multi_class_class}.

Even though the multi-class classification algorithm achieved similar results to the binary classification algorithm, an advantage over binary classification is that different features can be used for each detector. Feature selection algorithms such as sequential feature selection can be used to determine the optimal features. For instance, using the features shown in Table \ref{tab:f1_features} derived from scikit-learn's \cite{ref:scikit-learn} sequential forward feature selection results in an average lead time increase of 0.1s over training a binary classification with \eqref{eq:features_main}. However, to truly take advantage of the multi-class classification algorithm more investigation into optimal feature selection is needed to determine whether the additional average lead time gained can overcome the fault identifier delay.

\begin{table}
        \centering
        \caption{ Features derived from scikit-learn's \cite{ref:scikit-learn} sequential forward feature selection}
        % \begin{tabular}{|p{0.2\linewidth}|p{0.75cm}|p{0.75cm}|p{0.5cm}|p{0.5cm}|p{0.5cm}|p{0.5cm}|p{0.5cm}|  }
        \tabulinesep=0.5mm
           \begin{tabu}{|c|c| }
    \hline
    \makecell{Incipient Fault Features}  &\makecell{Abrupt Fault\\Features} \\                                             \hline
                  $\begin{bmatrix}
        \text{\rm knee~angle} \\
        \text{\rm hip~angle} \\
        \text{\rm vel~hip~angle} \\
        (L_{cop} + L_{com})* sgn(p_{com}^x- p_{f_{mid}}^x) 
    \end{bmatrix}$ &            $\begin{bmatrix}
        p_{com}^x\\
        v_{com}^x\\
        p_{com}^x - p_{heel}^x\\
        \text{\rm foot~x} \\
        \text{\rm vel~foot~z} \\
        \text{\rm vel~hip~angle} 
    \end{bmatrix} $\\
    \hline                                               
    \end{tabu}
        \label{tab:f1_features}
    \end{table}

\begin{table}
        \centering
        \caption{ A comparison of the maximum average lead time achieved by the binary nearest-neighbor classification algorithm and the multi-class classification algorithm with nearest-neighbor fault detectors}
        % \begin{tabular}{|p{0.2\linewidth}|p{0.75cm}|p{0.75cm}|p{0.5cm}|p{0.5cm}|p{0.5cm}|p{0.5cm}|p{0.5cm}|  }
        \tabulinesep=0.5mm
           \begin{tabu}{|c|c|c| }
    \hline
            Fold                                                       &\makecell{Multi-class Classification\\Average Lead Time} 
                  &\makecell{Binary Nearest-Neighbor\\Average Lead Time} \\                                             \hline
        1&  0.56 & 0.49  \\
        \hline
        2    & 0.52 & 0.49\\
        \hline
        3    &  0.57 & 0.51 \\    
        \hline
        4    &  0.56 & 0.51  \\    
        \hline
        5   &  0.57 & 0.51 \\ 
        \hline
        Average  &0.56   & 0.5 \\       
    \hline                                               
    \end{tabu}
        \label{tab:multi_class_cluster}
    \end{table}



\begin{table}
        \centering
        \caption{ A comparison of the maximum average lead time achieved by the binary SVM classifier and the multi-class classification algorithm with SVM fault detectors}
        % \begin{tabular}{|p{0.2\linewidth}|p{0.75cm}|p{0.75cm}|p{0.5cm}|p{0.5cm}|p{0.5cm}|p{0.5cm}|p{0.5cm}|  }
        \tabulinesep=0.5mm
           \begin{tabu}{|c|c|c|c|  }
    \hline
           Fold                                                       &\makecell{Multi-class Classification\\Average Lead Time} 
                  &\makecell{Binary Classification\\Average Lead Time} \\                                             \hline
        1&  0.7 & 0.65  \\
        \hline
        2    & 0.73 & 0.66  \\
        \hline
        3    &  0.70 & 0.66\\    
        \hline
        4    &  0.70 & 0.65  \\    
        \hline
        5   &  0.67 & 0.63  \\ 
        \hline
        Average  &0.7   & 0.65 \\       
    \hline                                               
    \end{tabu}
        \label{tab:multi_class_class}
    \end{table}







