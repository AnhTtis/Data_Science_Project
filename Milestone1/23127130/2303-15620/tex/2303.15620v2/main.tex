%%%%%%%%%%%%%%%%%%%%%%%%%%%%%%%%%%%%%%%%%%%%%%%%%%%%%%%%%%%%%%%%%%%%%%%%%%%%%%%%
%2345678901234567890123456789012345678901234567890123456789012345678901234567890
%        1         2         3         4         5         6         7         8

\documentclass[conference]{template/IEEEtran}   % Comment this line out
                                                          % if you need a4paper
\makeatletter
\def\ps@IEEEtitlepagestyle{%
  \def\@oddfoot{\mycopyrightnotice}%
  \def\@evenfoot{}%
}
\def\mycopyrightnotice{%
  {\footnotesize U.S. Government work not protected by U.S. copyright\hfill}% <--- Change here
  \gdef\mycopyrightnotice{}
}

%\documentclass[a4paper, 10pt, conference]{ieeeconf}      % Use this line for a4
                                                          % paper

\IEEEoverridecommandlockouts                              % This command is only
                                                          % needed if you want to
                                                          % use the \thanks command

% See the \addtolength command later in the file to balance the column lengths
% on the last page of the document
\usepackage{cite}
\usepackage{balance}
\usepackage{amsmath,amssymb,amsfonts}
\usepackage{algorithmic}
\usepackage{graphicx}
\usepackage{textcomp}




\usepackage[utf8]{inputenc}
\usepackage[T1]{fontenc}
\usepackage{mathptmx} % assumes new font selection scheme installed
\usepackage{mathptmx} % assumes new font selection scheme installed
\usepackage{amsmath} % assumes amsmath package installed
\usepackage{amssymb}  % assumes amsmath package installed
\usepackage{graphicx} 
\usepackage{grffile}
% \usepackage[hypcap=true]{caption}
\usepackage{array}
\usepackage{mathtools}
\usepackage{commath}
\usepackage{colortbl}
\usepackage{enumerate}
\usepackage{hyperref}
\usepackage{cleveref}
% \usepackage{appendix}
\usepackage{subfig}
%\usepackage{subfloat}
\usepackage{float}
\usepackage{tabularx}
%\usepackage{subcaption}
\usepackage{makecell}
\usepackage{multirow}
\usepackage{placeins}
\usepackage{hyperref}
\usepackage{tabu}


\newcommand{\fnr}{\rm FNR }
\newcommand{\fpr}{\rm FPR }

\newcommand{\fnrNS}{\rm FNR} % no space after FNR
\newcommand{\fprNS}{\rm FPR} % no space after FPR

\newcommand{\jwg}[1]{[{\textbf{\textcolor{red}{JWG: #1}}}]}
\newcommand{\eva}[1]{[{\textbf{\textcolor{blue}{Eva: #1}}}]}
\newcommand{\jessy}[1]{[{\textbf{\textcolor{red}{JWG: #1}}}]}


% \usepackage{tikz}
% \usepackage{pgfplots}
% \usepackage{caption}
% \usepackage{subcaption}


\usepackage{blindtext}
\usepackage{eso-pic}
\IEEEoverridecommandlockouts
% The preceding line is only needed to identify funding in the first footnote. If that is unneeded, please comment it out.
\usepackage{cite}
\usepackage{amsmath,amssymb,amsfonts}
\usepackage{algorithmic}
\usepackage{graphicx}
\usepackage{textcomp}
\usepackage{xcolor}
\def\BibTeX{{\rm B\kern-.05em{\sc i\kern-.025em b}\kern-.08em
    T\kern-.1667em\lower.7ex\hbox{E}\kern-.125emX}}
    
\usepackage{eso-pic}
\newcommand\AtPageUpperMyright[1]{\AtPageUpperLeft{%
 \put(\LenToUnit{0.17\paperwidth},\LenToUnit{-2cm}){%
     \parbox{0.9\textwidth}{\raggedleft\fontsize{8}{11}\selectfont #1}}%
 }}%
\newcommand{\conf}[1]{%
\AddToShipoutPictureBG*{%
\AtPageUpperMyright{#1}
}
}  
% \usetikzlibrary{positioning}
% \usetikzlibrary{external}
% \tikzexternalize[prefix=tikzFigs/,optimize command away=\includepdf] 

% \usepackage{refcheck}

% The following packages can be found on http:\\www.ctan.org
%\usepackage{graphics} % for pdf, bitmapped graphics files
%\usepackage{epsfig} % for postscript graphics files
\begin{document}
% \raggedbottom
\title{\vspace*{1cm} Optimizing Lead Time in Fall Detection for a Planar Bipedal Robot\\
} 


\author{\IEEEauthorblockN{M. Eva Mungai}
\IEEEauthorblockA{\textit{Mechanical Engineering} \\
\textit{University of Michigan}\\
Ann Arbor, MI US \\
mungam@umich.edu}
\and
\IEEEauthorblockN{Jessy Grizzle}
\IEEEauthorblockA{\textit{Department of Robotics} \\
\textit{University of Michigan}\\
Ann Arbor, MI US \\
grizzle@umich.edu}}


\maketitle
\conf{\textit{  Proc. of the International Conference on Electrical, Computer, Communications and Mechatronics Engineering (ICECCME 2023) \\ 
19-21 July 2023, Tenerife, Canary Islands, Spain}}




% \address[1]{Mechanical Engineering Department, University of Michigan, Ann Arbor, MI 48109, USA.}
% \address[2]{Robotics Institute, University of Michigan, Ann Arbor, MI 48109, USA.}}


% \begin{abstract}
% For legged robots to navigate across complex terrains, they must be robust to any disturbances and uncertainties they may encounter. Robustness can be increased using various methods such as designing and implementing feedback controllers, however, as it is infeasible to account for all disturbances and uncertainties falling is inevitable. Falls are undesirable as they can prevent a robot from completing its task, result in irreparable damage or lead to operator/bystander injuries. Therefore, it is imperative to design and implement fall detection algorithms for robots. Falls can be caused by abrupt (fast acting), incipient (slow acting) or intermittent (non-continuous) faults. The objective of a fall detection algorithm is to accurately detect falls with sufficient time to implement corrective motions. Early fall detection is a challenging task due to the masking effects of controllers, the inverse relationship between lead time and false positive rates, and the temporal behavior of the faults/underlying factors. In this paper, we propose a fall detection algorithm that is capable of detecting both incipient and abrupt faults while maximizing lead time and meeting desired thresholds on the false positive and negative rates. \eva{1st attempt, might need to redo abstract}   
% \end{abstract}

\begin{abstract}
For legged robots to operate in complex terrains, they must be robust to the disturbances and uncertainties they encounter. This paper contributes to enhancing robustness by designing fall detection/prediction algorithms that will provide sufficient lead time for corrective motions to be taken. Falls can be caused by abrupt (fast-acting), incipient (slow-acting), or intermittent (non-continuous) faults. Early fall detection is a challenging task due to the masking effects of controllers (through their disturbance attenuation actions), the direct relationship between lead time and false positive rates, and the temporal behavior of the faults/underlying factors. In this paper, we propose a fall detection algorithm capable of detecting both incipient and abrupt faults while maximizing lead time and meeting desired thresholds on the false positive and negative rates.
\end{abstract}
\begin{IEEEkeywords}
fault detection, fall detection, classification, bipedal robot, lead time, anomaly detection
\end{IEEEkeywords}
% *********************  BEGIN PAPER  ***********************
% 

Over the past few years, there has been a significant amount of research focused on studying the ReLU activation function, with the aim of achieving neural network convergence through over-parametrization. However, recent developments in the field of Large Language Models (LLMs) have sparked interest in the use of exponential activation functions, specifically in the attention mechanism.

Mathematically, we define the neural function $F: \R^{d \times m} \times  \mathbb{R}^d \rightarrow \mathbb{R}$ using an exponential activation function. Given a set of data points with labels $\{(x_1, y_1), (x_2, y_2), \dots, (x_n, y_n)\} \subset \mathbb{R}^d \times \mathbb{R}$ where $n$ denotes the number of the data. Here $F(W(t),x)$ can be expressed as $F(W(t),x) := \sum_{r=1}^m a_r \exp(\langle w_r, x \rangle)$, where $m$ represents the number of neurons, and $w_r(t)$ are weights at time $t$. It's standard in literature that $a_r$ are the fixed weights and it's never changed during the training. We initialize the weights $W(0) \in \mathbb{R}^{d \times m}$ with random Gaussian distributions, such that $w_r(0) \sim \mathcal{N}(0, I_d)$ and initialize $a_r$ from random sign distribution for each $r \in [m]$.

Using the gradient descent algorithm, we can find a weight $W(T)$ such that $\| F(W(T), X) - y \|_2 \leq \epsilon$ holds with probability $1-\delta$, where $\epsilon \in (0,0.1)$ and $m = \Omega(n^{2+o(1)}\log(n/\delta))$. To optimize the over-parametrization bound $m$, we employ several tight analysis techniques from previous studies [Song and Yang arXiv 2019, Munteanu, Omlor, Song and Woodruff ICML 2022]. 

 



\section{Introduction}
\label{sec:introduction}
% \begin{itemize}
%     % Diffusion of FL
%     \item {\st{Diffusion of FL}}
%     % Security threats to FL
%     \item {\st{Security threats to FL with particular focus on model poisoning}}
%     % Limitations of existing countermeasures
%     \item {\st{Current countermeasures (e.g., KRUM) and their limitations}}
%     % Proposed method and its advantages
%     \item {\st{Intuitive description of the proposed method and its difference (i.e., advantages) w.r.t. state of the art}}
%     % Main contributions
%     \item {\st{Summary of the main contributions of this work}}
%     % Paper's structure and organization
%     \item {\st{Paper's structure and organization}}
% \end{itemize}

% Diffusion of FL
Recently, {\em federated learning} (FL) has emerged as the leading paradigm for training distributed, large-scale, and privacy-preserving machine learning (ML) systems~\cite{mcmahan2017googleai,mcmahan2017aistats}. 
The core idea of FL is to allow multiple edge clients to collaboratively train a shared, global model without disclosing their local private training data.
%Specifically, an FL system consists of a central server and many edge clients; 
A typical FL round involves the following steps: {\em(i)} the server randomly picks some clients and sends them the current, global model; {\em(ii)} each selected client locally trains its model with its own private data; then, it sends the resulting local model to the server;\footnote{Whenever we refer to global/local model, we mean global/local model {\em parameters}.} {\em(iii)} the server updates the global model by computing an \emph{aggregation function}, usually the average (FedAvg), on the local models received from clients.
% \begin{enumerate}
%     \item[{\em(i)}] the server sends the current, global model to the clients and appoints some of them for training;
%     \item[{\em(ii)}] each selected client locally trains its copy of the global model with its own private data; then, it sends the resulting local model back to the server;\footnote{Whenever we refer to global/local model, we mean global/local model {\em parameters}.}
%     \item[{\em(iii)}] the server updates the global model by computing an \emph{aggregation function} on the local models received from clients (by default, the average, also referred to as FedAvg~\cite{mcmahan2017aistats}).
% \end{enumerate}
This process goes on until the global model converges. %(e.g., after a certain number of rounds or other similar stopping criteria).
%\\
% The advantages of FL over the traditional, centralized learning paradigm are undoubtedly clear in terms of flexibility/scalability (clients can join/disconnect from the FL network dynamically), network communications (only model weights\footnote{We will use \textit{parameters} and \textit{weights} interchangeably.} are exchanged between clients and server), and privacy (each client's private training data is kept local at the client's end and not uploaded to the server).
\\
% Security threats to FL
%However, the growing adoption of FL also raises security concerns~\cite{costa2022covert}, particularly about its confidentiality, integrity, and availability.
Although its advantages over standard ML, FL also raises security concerns~\cite{costa2022covert}. %, particularly about its confidentiality, integrity, and availability~\cite{costa2022covert}.
% OLD, LONG VERSION
% Indeed, some work deals with privacy leakage that may expose the local data of some clients~\cite{melis2019sp}. 
% A large body of work, instead, investigates attacks that usually aim to detriment the predictive accuracy of the learned global model. For instance, \emph{data poisoning} attacks achieve this goal by letting an adversary pollute the training set of some corrupt FL clients with maliciously crafted examples~\cite{jagielski2018sp}.
% Similarly, in \emph{model poisoning} the attacker attempts to tweak the global model weights~\cite{bhagoji2019pmlr} by directly perturbing the local model's weights of some infected FL clients before these are sent to the central server for aggregation, usually via so-called Byzantine attacks. 
% It turns out that Byzantine model poisoning attacks severely impact standard FedAvg; therefore, more robust aggregation functions must be designed to make FL systems secure.
Here, we focus on \emph{untargeted model poisoning} attacks~\cite{bhagoji2019pmlr}, where an adversary attempts to tweak the global model weights %\footnote{We will use the terms \textit{parameters} and \textit{weights} interchangeably.} 
by directly perturbing the local model's parameters of some infected clients before these are sent to the central server for aggregation.
In doing so, the adversary aims to jeopardize the global model \textit{indiscriminately} at inference time.
Such model poisoning attacks severely impact standard FedAvg; therefore, more robust aggregation functions must be designed to secure FL systems.
\\
% In this paper, we focus on designing a novel robust aggregation scheme at the server's end to contrast the effect of Byzantine model poisoning attacks.
%
% Current countermeasures and their limitations
%Several countermeasures have been proposed in the literature to combat model poisoning attacks on FL systems.
% Some methods use simple statistics more robust than plain average to smooth the impact of malicious updates (e.g., Trimmed Mean and FedMedian~\cite{yin2018icml}). 
% Other defenses implement outlier detection techniques to discard malicious updates from the aggregation performed at the server's end. Those are either based on heuristics (e.g., Krum/Multi-Krum~\cite{blanchard2017nips} and Bulyan~\cite{mhamdi2018pmlr}) or data-driven approaches (e.g., K-means clustering~\cite{shen2016acm} or DnC via spectral analysis~\cite{shejwalkar2021ndss}). 
% Finally, some strategies rely on a centralized ``source of trust'' to spot potential malicious updates (e.g., FLTrust~\cite{cao2020fltrust}).
% Several countermeasures have been proposed in the literature to combat model poisoning attacks on FL systems, i.e., to discard possible malicious local updates from the aggregation performed at the server's end. 
% These techniques range from simple statistics more robust than plain average (e.g., Trimmed Mean and FedMedian~\cite{yin2018icml}) to outlier detection heuristics (e.g., Krum/Multi-Krum~\cite{blanchard2017nips} and Bulyan~\cite{mhamdi2018pmlr}) or data-driven approaches (e.g., spectral analysis via K-means clustering~\cite{shen2016acm} or spectral analysis), or methods based on ``source of trust'' (e.g., FLTrust~\cite{cao2020fltrust}).
% OLD, LONG VERSION
%Several countermeasures have been proposed in the literature to combat Byzantine model poisoning attacks on FL systems.
% Descriptive statistics
% For example, Trimmed Mean and FedMedian aggregate local model updates using more robust statistics than standard average~\cite{yin2018icml}.
%
% % Heuristics for outlier detection
% Many existing Byzantine-resilient strategies implement some outlier detection heuristics to discard the model updates sent by potentially malicious clients from the input of the aggregation function.
% One of the most popular heuristics is Krum~\cite{blanchard2017nips}.
% This strategy tries to mitigate the impact of Byzantine attacks by selecting as a global model the local model with the smallest sum of Euclidean distances to {\em all} the other local models.
% Although powerful, Krum requires the server to know (or, at least, estimate) the number of malicious FL clients upfront, which is generally impossible in a realistic attack scenario. %
% Moreover, Krum may become ineffective for complex, high-dimensional model parameter spaces due to the curse of dimensionality.
% Bulyan~\cite{mhamdi2018pmlr} tries to overcome this issue by combining Krum with a variant of Trimmed Mean.
% % Data-driven outlier detection
% Other strategies use data-driven outlier detection techniques -- e.g., via K-means clustering~\cite{shen2016acm} -- to spot potential malicious local model updates. 
% %For instance, Shen et al. propose to cluster local model updates with K-means and thus identify outliers.
%
% % Other techniques
% As far as the server is concerned, any local model received can be from a potential malicious client. 
% FLTrust~\cite{cao2020fltrust} assumes the server acts as a client, i.e., trains a local model on an additional {\em trustworthy} dataset at the server's end and compares it against all the local models from other clients. 
% This way, the server can rely on some ``source of trust'' when discarding potentially malicious clients.
%\\
% Limitations of existing Byzantine-resilient strategies
Unfortunately, existing defense mechanisms either rely on simple heuristics (e.g., Trimmed Mean and FedMedian by~\cite{yin2018icml}) or need strong and unrealistic assumptions to work effectively (e.g., foreknowledge or estimation of the number of malicious clients in the FL system, as for Krum/Multi-Krum~\cite{blanchard2017nips} and Bulyan~\cite{mhamdi2018pmlr}, which, however, cannot exceed a fixed threshold).
Furthermore, outlier detection methods using K-means clustering~\cite{shen2016acm} or spectral analysis like DnC~\cite{shejwalkar2021ndss} do not directly consider the temporal evolution of local model updates received.
Finally, strategies like FLTrust~\cite{cao2020fltrust} require the server to collect its own dataset and act as a proper client, thereby altering the standard FL protocol.
\\
% OLD, LONG VERSION
% Overall, existing Byzantine-resilient strategies are either simple heuristics (e.g., FedMedian) or, if they are more complex, they rely on strong and unrealistic assumptions to work effectively (e.g., knowing the number of malicious clients in the FL system in advance, as for Krum and alike).
% Furthermore, data-driven outlier detection methods do not consider the temporary evolution of local model updates received (e.g., K-means clustering). 
% Finally, strategies like FLTrust requires the server to collect its own dataset and act as a proper client, thereby altering the standard FL protocol.
%
% Description of the proposed method
This work introduces a novel pre-aggregation \textit{filter} robust to untargeted model poisoning attacks. Notably, this filter $(i)$ operates without requiring prior knowledge or constraints on the number of malicious clients and $(ii)$ inherently integrates temporal dependencies. 
The FL server can employ this filter as a preprocessing step before applying \textit{any} aggregation function, be it standard like FedAvg or robust like Krum or Bulyan.
Specifically, we formulate the problem of identifying corrupted updates as a multidimensional (i.e., matrix-valued) time series anomaly detection task. 
The key idea is that legitimate local updates, resulting from well-calibrated iterative procedures like stochastic gradient descent (SGD) with an appropriate learning rate, show \textit{higher predictability} compared to malicious updates. This hypothesis stems from the fact that the sequence of gradients (thus, model parameters) observed during legitimate training exhibit regular patterns, as validated in Section~\ref{subsec:intuition}. %until convergence. 
%This regularity may be more pronounced for smooth convex loss functions, but it can still be captured within an appropriate time window, even for more complex and convoluted loss surfaces. 
%We provide evidence of this claim in Appendix~B, where we show that the average mutual information (i.e., ``predictability''), calculated over pairs of legitimate model updates sent at different FL rounds, is significantly higher than the corresponding computation for a malicious client.
\\
Inspired by the matrix autoregressive (MAR) framework for multidimensional time series forecasting~\cite{chen2021je}, we propose the FLANDERS ({\em \textbf{F}ederated \textbf{L}earning meets \textbf{AN}omaly \textbf{DE}tection for a \textbf{R}obust and \textbf{S}ecure}) filter.
The main advantages of FLANDERS over existing strategies like FLDetector~\cite{zhao2020multivariate} are its resilience to large-scale attacks, where $50\%$ or more FL participants are hostile, and the capability of working under realistic non-iid scenarios.
We attribute such a capability to two key factors: $(i)$ FLANDERS works without knowing a priori the ratio of corrupted clients, and $(ii)$ it embodies temporal dependencies between intra- and inter-client updates, quickly recognizing local model drifts caused by evil players. Below, we summarize our main contributions:

\begin{itemize}
\item[{\em(i)}]
We provide empirical evidence that the sequence of models sent by legitimate clients is more predictable than those of malicious participants performing untargeted model poisoning attacks.
\\
\item[{\em(ii)}] 
We introduce FLANDERS, the first pre-aggregation filter for FL robust to untargeted model poisoning based on multidimensional time series anomaly detection.
\\
\item[{\em(iii)}] 
We integrate FLANDERS into Flower,\footnote{\scriptsize{\url{https://flower.dev/}}} a popular FL simulation framework for reproducibility.
\\
\item[{\em(iv)}] 
We show that FLANDERS improves the robustness of the existing aggregation methods under multiple settings: different datasets, client's data distribution (non-iid), models, and attack scenarios.
\\
\item[{\em(v)}] 
We publicly release all the implementation code of FLANDERS along with our experiments.\footnote{\scriptsize{\url{https://anonymous.4open.science/r/flanders_exp-7EEB}}}
\end{itemize}

% Paper's structure and organization
The remainder of the paper is structured as follows. %some related work and the current state-of-the-art solutions to security issues that FL entails. 
Section~\ref{sec:background} covers background and preliminaries. 
In Section~\ref{sec:related}, we discuss related work.
Section~\ref{sec:problem} and Section~\ref{sec:method} describe the problem formulation and the method proposed. % to tackle it. 
Section~\ref{sec:experiments} gathers experimental results. %, and Section~\ref{sec:limitations} discusses some limitations of this work.
Finally, we conclude in Section~\ref{sec:conclusion}.
 %discusses the limitations of this work and draws future research directions.
%reports conclusions and draws perspectives for future research directions.

%%%%%%% OLD %%%%%%%
%to overcome the resilience of Byzantine failures in distributed Stochastic Gradient Descent computations. 
% The strength of Krum is its time complexity, which is linear in the gradient dimension. 
% However, the robustness of the approach is guaranteed for gradient-based learning applications only when the majority of the clients are not compromised. 
% Besides, the aggregation mechanism of Krum, as well as that of similar methods, is robust from a coarse-grained perspective and does not provide solutions to errors and perturbations that may occur at inference time.
%A related approach to~\cite{blanchard2017nips} is the work of Su et al.~\cite{su2016dc}. Here, the authors propose an iterated approximate agreement to tackle a multi-layer scenario attacked by Byzantine agents. 
%However, the method works efficiently on the sole discrete context and it is inapplicable to continuous state environments.
%\gabri{Maybe, we should just talk about the main limitations of existing countermeasures without digging into their details (or, we can just mention Krum as this is the most popular one). I will move the description of all these methods to the Related Work section.}

% \input{tex/literature_review}

% \section{Problem Description}\label{sec:problem_description}
We begin by describing the problem of BO for composite functions in subsection \ref{sec:desc_BO}. In subsection \ref{sec:demand_model} we describe the dynamic pricing problem and model the revenue function as a function composition to which BO methods for composite functions can be applied. 
\subsection{BO for Function Composition}\label{sec:desc_BO}
We consider the problem of optimizing $g(\textbf{x}) = h(f_1(\textbf{x}), f_2(\textbf{x}), \ldots , f_M(\textbf{x}))$ where $g : \mathcal{X} \rightarrow \mathrm{R}$, $f_i : \mathcal{X} \rightarrow \mathrm{R}$, $h:\mathrm{R}^{M} \rightarrow \mathrm{R}$ and $\mathcal{X} \subseteq \mathrm{R}^d$. We assume each $f_i$ is a black-box expensive-to-evaluate continuous function and $h$ is known and cheap to evaluate. The optimization problem that we consider is 
\begin{equation}\label{eq:problemdesc}
  \max_{\textbf{x} \in \mathcal{X}} ~h(f_1(\textbf{x}), f_2(\textbf{x}), \ldots, f_M(\textbf{x})). 
\end{equation}
We want to solve Problem~\ref{eq:problemdesc} in an iterative manner where in the $n^{\text{th}}$ iteration, we can use the previous observations $\{\textbf{x}_i, f_1(\textbf{x}_i), \ldots, f_M(\textbf{x}_i\})\}^{n-1}_{i=1}$ to request a new observation $\{\textbf{x}_n, f_1(\textbf{x}_1), \ldots, f_M(\textbf{x}_n)\}$.

A vanilla BO algorithm applied to this problem would first assume a prior GP model on $g$, denoted by $\mathcal{GP}(\mu(\cdot), K(\cdot, \cdot))$ where $\mu$ and $K$ denote the mean and covariance function of the prior model. Given some function evaluations, an updated posterior GP model is obtained. A suitable acquisition function, such as EI or PI can be used, to identify the next query point. For example, in the $n+1^{\text{th}}$ update round, one would first use the $n$ available observations $(g(\textbf{x}_1), g(\textbf{x}_2), \ldots, g(\textbf{x}_n))$ to update the GP model to  $\mathcal{GP}(\mu^{(n)}(\cdot), K^{(n)}(\cdot, \cdot))$   where $\mu^{(n)}(\cdot)$ is the posterior mean function and $K^{(n)}(\cdot, \cdot)$ is the posterior covariance function, see \cite{Rasmussen2005} for more details. The acquisition function then uses this posterior model to identify the next query location $\textbf{x}_{n+1}$. In doing so, vanilla BO ignores the values of the member functions in the composition $h$.
%while building a posterior.

BO for composite function, on the other hand, takes advantage of the available information about $h$, and its easy-to-compute nature. Astudillo and Frazier~\cite{https://doi.org/10.48550/arxiv.1906.01537} model the constituent functions of the composition by a single multi-output function $\textbf{f}(\textbf{x}) = (f_1(\textbf{x}), \ldots, f_M(\textbf{x}))$ and then model the uncertainty in $\textbf{f}(\textbf{x})$ using a multi-output Gaussian process to optimize $h({\textbf{f}(\textbf{x})})$. Since the prior over $f$ is modelled as a MOGP, the proposed method tries to capture the correlations between different components of the multi-output function $\textbf{f}(\textbf{x})$. Note that the proposed EI and PI-based acquisition functions are required to be computed using Monte Carlo sampling. Furthermore, a sample from the posterior distribution is obtained by first sampling an $n$ variate normal distribution, then scaling it by the lower Cholesky factor and then centering it with the mean of the posterior GP. Two problems arise due to this: \begin{enumerate*} 
    \item Such simulation based averaging approach increases the time complexity of the procedure  linearly with the number of samples taken for averaging and
    \item calculation of the lower Cholesky factor increases the function's time complexity cubically with the number of data points. 
\end{enumerate*}
These factors render the algorithm unsuitable, particularly for problems with large number of member functions or for problems with large dimensions.


% \textbf{Expand on this, why sampling and inversion is needed, describe their acquisiton in 1-2 lines, enumerate drawbacks of astuldo work, 1) computationally expensive 2) inversion 3) 1&2 make it slow and not suitable for large dimension problems  }


To alleviate these problems, in this work, we model the constituent functions using independent GPs. This modelling approach allows us to train GPs for each output independently and hence the posterior GP update can be parallelized. We propose two acquisition functions, cEI which is based on the EI algorithm and cUCB, which is based on the GP-UCB algorithm \cite{Srinivas_2012}. Our cEI acquisition function is similar in spirit to the EI-CF acquisition function of \cite{https://doi.org/10.48550/arxiv.1906.01537} but is less computationally intensive owing to the independent GP model. Since we have independent one dimensional GP model for each constituent function, sampling points from the posterior GP does not require computing the Cholesky factor (and hence matrix inversion), something that is needed in the case of high-dimensional GP's of \cite{https://doi.org/10.48550/arxiv.1906.01537}.  
This greatly reduces the complexity of the MC sampling steps of our algorithm (see section \ref{sec:our_approach} for more details). However, the cEI acquisition function still suffers from the drawback of requiring Monte Carlo averaging. 
To alleviate this problem, we propose a UCB based acquisition function that uses the current mean plus scaled variance of the posterior GP at a point as a surrogate for the constituent function at that point. As shown by Srinivas et al.~\cite{Srinivas_2012}, while the mean term in the surrogate guides exploitation, it is the variance of the posterior GP at a point that allows for suitable exploration. The scaling of the variance term is controlled in such a way that it balances the trade off between exploration and exploitation. In Section~\ref{sec:experiments}, we illustrate the utility of our method, first for standard test functions and then as an application to dynamic pricing problem. Our algorithms, especially the cUCB one, outperforms not only vanilla BO but also those proposed in Astudillo and Frazier~\cite{https://doi.org/10.48550/arxiv.1906.01537}.

%and does not require matrix inversion, resulting in the complexity for computing the variance for each function reducing from cubic to constant time. Our UCB variant helps us eliminate Monte Carlo sampling while computing the acquisition function. 
% \textbf{describe 1-2 lines what are the main advantages of doing this over atuldo}. In this work, we also propose a a UCB-style algorithm which helps us reduce the computational complexity of the process.
% \textbf{Refer to GP UCB paper here, again say in detail how the UCB offers benefits, other than not requiring matrix inversion are there more benefits ? The sigma term in UCB allows for better exploration?} Our method does not require the computation of an inverse matrix. 

% \textbf{Vanilla BO}
%  One possible approach to solve this is using a vanilla Bayesian Optimization of $g(x)$, we assume $g$ to be drawn from a GP prior probability distribution. It would use an acquisition function like EI to optimize the GP prior over $f$ and ignore the observations on $f_i$ in the process.

% \textbf{Astuldo}
% We want to solve Problem~\ref{eq:problemdesc} in an iterative manner. In the $k^{th}$ iteration, we can use the previous observations $\{x_i, f_1(x_i), \ldots, f_N(x_N)\}^{k-1}_{i=1}$ to request a new observation $\{x_k, f_1(x_k), \ldots, f_N(x_k)\}$.

% \textbf{Drawbacks of Astuldo}
% Multi output GP, computationally intensive, acquisition functions require matrix inversion etc 

% \textbf{Motivation to our algorithms }
% Independent GP, UCB type acquisition function that does not need inversion. Performance gains we see in terms of run time.
% more details section 3

% \textbf{Move this para to vani9lla BO, add mathematical details, see prabu paper or fobo paper}
% While traditional BO would model a GP over the function $g$ as described in Section~\ref{sec:problem_description} and ignore the values of the member functions in the composition of $h$ while building a prior, our idea is to take advantage of the information we have about the cheap to evaluate function, $h$. Traditional BO would then look to use standard acquisition functions such as Expected Improvement (EI) or Probability of Improvement (PI) to find the point at which the model predicts the best results should be. That point would be evaluated next and the result would be used to update the GP and suggest the next point to evaluate. This method works well with standard functions but would ignore the domain information we have about $h$ when optimizing a composite function.


\subsection{Bayesian Optimization for dynamic pricing}\label{sec:demand_model}
We consider Bayesian Optimization for two types of revenue optimization problems. The first problem optimizes the revenue per customer where customers are characterized by their willingness-to-pay distribution (which is unknown). In the second problem, we assume a parametric demand model (the functional form is assumed to be unknown) and optimizes the associated revenue. 

In the first model, we assume that an arriving customer has an associated random variable, $V$, with complimentary cumulative distribution function $\bar{F}$, indicating its maximum willingness to pay for the item sold. For an item on offer at a price $p$, an arriving customer purchases it with the probability 
\begin{equation}\label{eq:fp}
    d(p):= \bar{F}(p) = \text{Pr}\{V \geq p\}  .
\end{equation}
In this case, when the product is on offer at a price $p$, the revenue per customer $r(p)$ is given by $r(p) = p \bar{F}(p)$. The revenue function is a composition of the price and demand or purchase probability and we assume that the distribution of the purchase probability i.e., $F$ is not known and also expensive to estimate. One could perform a vanilla BO algorithm by having a GP model on $r(p)$ itself. However to exploit the known nature of the revenue function, we will apply our function composition method by instead having a GP on  $\bar{F}(p)$ and demonstrate its superiority over vanilla BO.  

In the second model, we assume that the true demand $d(p)$ for a commodity at price $p$ has a functional form. This forms the ground truth model that governs the demand, but we assume that the functional form for this demand is not known to the manager optimizing the revenue. In our experiments, we assume linear, logit, Booth and Matyas functional forms for the demand (see section \ref{sec:experiments} for more details.)
Along similar lines, one could build more sophisticated demand models to account for external factors (such as supply chain issues, customer demographics or inventory variables), something that we leave for future explorations. 
%in the model by modelling demand as a parametric function of price
%\begin{equation}
 %   d(p, \boldsymbol{z}):= \bar{F}_{\boldsymbol{z}}(p) = \text{Pr}\{V \geq p\}
%\end{equation}
%where $z$ is the function parameter over $\bar{F}$ which can evolve over time with changing conditions.

Note that we make some simplifying assumptions about the retail environment in these two models and our experiments. We assume a non-competitive monopolistic market with an unlimited supply of the product and no marginal cost of production. However, these assumptions can easily be relaxed by changing the ground truth demand model appropriately, which are used in the experiments to reflect these aspects. The fact that we use a GP model as a surrogate for the unknown demand model offers it the ability to model a diverse class of demand functions under diverse problem settings. We do not discuss these aspects further but focus on the following simple yet meaningful experimental examples that one typically encounters in revenue management problems.  
%We also assume that the price for a product can be adjusted instantaneously and that customer feedback is received in real time. Moreover, we also assume that it is feasible to quote different prices to the customer simultaneously as well and expose them to customers with identical natures. 
%We now describe the experimental setups to which we apply our composite BO algorithms. 
In the following, $\textbf{p}$ denotes the price vector:
\begin{enumerate}
    \item \textbf{Independent demand model:} A retailer supplies its product to two different regions whose customer markets behave independently from each other. Thus, the same product has independent and different demand functions (and hence different optimal prices) in different geographical regions and under such black-box demand models for the two regions $(d_1, d_2)$. The retailer is interested in finding the optimal prices, leading to the optimization of the following function: $g(\textbf{p}) = p_1d_1(p_1) + p_2d_2(p_2)$. 
    \item \textbf{Correlated demand model:} Assume that a retailer supplies two products at prices $p_1$ and $p_2$ and the demand for the two products is correlated and influenced by the price for the other product. 
    %product to two regions, but their customer market does not behave independently from each other, leading to the objective function 
    Such a scenario can be modelled by a revenue function of the form $g(\textbf{p}) = p_1d_1(\textbf{p}) + p_2d_2(\textbf{p})$. Consider the  example where the prices of business and economy class tickets can influence the demand in each segment. Similarly, the demand for a particular dish in a fast food chain might be influenced by the prices for other dishes.
    \item \textbf{Identical price model: } In this case, the retailer is compliant with having a uniform price across locations. However, the demand function across different locations could be independent at each of these locations, leading to the following objective function: $g(p) = pd_1(p) + pd_2(p)$. This scenario can be used to model different demand functions for different population segments in their age, gender, socio-economic background, etc. 
\end{enumerate}
%Note that the first scenario is a special case of the second scenario where function $d_i$ is only dependent on $p_i$.

\section{Robot Description and Data Generation}\label{sec:robot_description_data_generation}
In this section, we describe the robot model that is used for this study and how the data are generated and prepared.

\subsection{Robot Description}
We assume a four-link planar robot based on Wandercraft's exoskeleton Atalante \cite{ref:wandercraft, ref:omar}. The four links are joined by three actuated revolute joints, called the ankle, knee, and hip. Fig.~\ref{fig:fourlinkrobot} depicts the chosen robot.  

\begin{figure}[!h]
    \centering
    \includegraphics[width=0.23\textwidth]{img/fall_detection/robot_description/planar_four_link_final.png}
    \captionsetup{font=scriptsize}
    \caption{Four-link robot model used in our study.
    }
    \label{fig:fourlinkrobot}
\end{figure}

\subsection{Equations of Motion and Simulation Environment}
The equations of motion are given by \eqref{eq:floating_base_fall} and \eqref{eq:accelConstEq} 
\begin{align}
    \label{eq:floating_base_fall}
    D(q)\ddot{q}+C(q,\dot{q})\dot{q}+G(q)&= Bu+J^T(q)\Gamma \\
    \label{eq:accelConstEq}
    J(q)\ddot{q} + \dot{J}(q,\dot{q})\dot{q}&=0,
\end{align}
where $q$ is the vector of generalized coordinates defined by \eqref{eq:gen_coords}, $u$ is the torque input vector, $D$, $C$, $G$, and $B$ are the inertia, Coriolis, gravity, and torque distribution matrices/vector, respectively, $J$ is the Jacobian mapping the contact wrenches to the generalized coordinates, and $\Gamma$ is the contact wrench. The floating-base Lagragian model is given by \eqref{eq:floating_base_fall} while the contact/acceleration constraint is given by \eqref{eq:accelConstEq} \cite{ref:Jessy,ref:mlsTxtBook,ref:lynch2017modern}. 
\begin{equation}
   q = \begin{bmatrix}
     \text{\rm foot~x}  \\
    \text{\rm foot~z} \\
    \text{\rm foot~angle ($\theta _f$)} \\
    \text{\rm ankle~angle  ($\theta _a$)} \\
    \text{\rm knee~angle  ($\theta _k$)} \\
    \text{\rm hip~angle  ($\theta _h$)} 
    \end{bmatrix}.
    \label{eq:gen_coords}
\end{equation}

A PD controller to maintain the robot in a standing position was designed and implemented in MATLAB \cite{ref:MATLAB}. The PD controller seeks to keep the foot flat on the ground while maintaining the center of mass (CoM) inside the support polygon.  The simulation environment uses MATLAB's ODE45 function and compliant ground contact forces represented as a spring-damper. 

\subsection{Data Generation } 
Four hundred trajectories each are generated for abrupt and incipient faults, with a sampling time of 0.03s and a disturbing force applied to the torso. To emulate disturbances that might cause the robot to oscillate slightly while standing, a random impulse force in the range of 0-159N and lasting for 0.075s is applied at time zero, but only the data after 2 seconds is kept. The abrupt faults last for 0.075 seconds with magnitudes ranging from 0-320N , while the incipient faults last for 1.0 seconds and range from 0-46N. The ranges, based on previous experiments, are chosen such that half the trajectories end in a fall (we'll refer to these trajectories as \textit{faulty trajectories}), and half of the safe (non-falling) trajectories have a heel or toe lift. Similar to \cite{ref:kalyanakrishnan2011learning}, the force magnitudes are generated using a uniform distribution. The abrupt force is applied at random between 2.5s and 3.5s, while the incipient fault is applied at random between 2.0s and 3.5s. The application time for the incipient fault is longer because, in order not to include an abrupt deviation in the robot's nominal states, only the data collected after the force is applied is kept. 






% J.K., J.S. and S.S. designed research; J.K. and R.S. performed research; J.K., J.S, R.S. and S.S. analyzed data; J.K. and S.S. wrote the paper; all authors edited the paper; E.M., J.K., R.S. and S.S contributed software; and J.S. designed the BrainScaleS-2 neuromorphic system.


\subsection{Data Pre-processing}\label{sec:data_preparation}
The features are selected as
% \begin{equation}
% \begin{bmatrix}
%     p_{com}^x\\
%     v_{com}\\
%     p_{com}^x - p_{cop}^x\\
%     p_{f\_mid}^z - p_{com}^z
% \end{bmatrix}
%     \label{eq:features_main}
% \end{equation}
\begin{equation}
\begin{bmatrix}
    L_{cop} - L_{com}\\
    p_{com}^x\\
   v_{com}^x\\
   (p_{toe} - p_{com})^x\\
    (p_{heel} - p_{toe})^{xz} \\
    (L_{cop} + L_{com})* sgn(p_{com}^x- p_{f_{mid}}^x) 
\end{bmatrix}
    \label{eq:features_main}
\end{equation}
where $p_{com}$ and $v_{com}$ are the center of mass (com) position and velocity,  $p_{toe}$, $p_{heel}$,  and $ p_{f_{mid}}$ are the position of the toe, heel and middle of the foot, and $L_{cop}$ is the angular momentum about the contact point\footnote{The contact point is set to the rotation point (toe or heel) when the foot rotates, and the center of pressure otherwise.}. These features are chosen based on their correlation with the lead time and other features commonly used in literature. The distance correlation coefficient is used to evaluate the correlations as it is able to capture nonlinear relationships \cite{ref:szekely2014partial}.

The features are split into training (60\%), validation (20\%), and testing (20\%) sets using  scikit-learn's \cite{ref:scikit-learn} stratified train\_test\_split method and k-folds methods with the number of folds set to 5. The stratified methods are chosen because they ensure that each of the splits has the same distribution of normal and faulty data. Scikit's min-max scaler is used to scale the data to a range of $\{0,1\}$. To ensure that only transient data is kept for training, only the first 6s of trajectories that are deemed as normal are kept.  



% 


\section{Fall Detection Methods}
As a baseline, we use the SVM-based classification algorithm of \cite{ref:scikit-learn}, while the nearest-neighbor algorithm is based on the Ward minimum variance method \cite{ref:ward1963hierarchical}. To prioritize recent data points over previous ones, both methods make use of sliding windows. The number of data points in a window is referred to as $N_{window}$. It is important to note that both methods are supervised algorithms.

\subsection{SVM Classifier}

%
The radial basis function is chosen as the kernel and the soft margin formulation is implemented for the SVM classifier. The training data for the classifier is defined as 
\begin{align*}
D&=\{X_i,y_i\}_{i=1}^n \\
&~~\text{where}  \\
&\begin{array}{rcl}
    n &=& \text{number of windows across all training data}\\
    m &=& \text{number of time steps in a window}\\
    x_{ij}&=&\text{features at time step j in window i} \\
        X_i &=& \begin{bmatrix} x_{i1} && x_{i2} && \cdots && x_{im}  \end{bmatrix}^\intercal \\
    y_i &\in&\  \left\{
    \begin{array}{rcl}
           -1  &\quad& X_i ~\in ~\text{faulty trajectory} \\
            1  &\quad& X_i ~\in ~\text{normal trajectory}
    \end{array}
    \right\}
    \end{array}
\end{align*}


\subsection{Nearest-Neighbor Algorithm}
\label{sec:classifier_intro}
 The selected nearest-neighbor algorithm determines distance using the Ward minimum variance and a weighted Euclidean distance. Given two clusters A and B, the Ward minimum variance method calculates the effort, $E_{AB}$, it takes to join the two clusters together, as determined by the sum of squared errors, specifically,  
 \begin{equation}
    \label{eq:ward_method}
    E_{AB} = SSE_{AB} - SSE_{A} - SSE_{B},
    \end{equation} 
 where
\begin{align*}
    &SSE_{AB} =(A\cup B- \mu_{A\cup B})^\intercal R^{-1}_{A\cup B}(A\cup B- \mu_{A\cup B})\\
    &SSE_{A} = (A- \mu_{A})^\intercal R^{-1}_{A}(A- \mu_{A})\\
    &SSE_{B} =  (B- \mu_{B})^\intercal R^{-1}_{B}(B- \mu_{B})\\
    &R = \text{correlation coefficient matrix}\\
    &\mu = \text{mean vector}
\end{align*}


In our application, cluster B contains the features at the current time step while cluster A contains all the features in the previous time steps included in the window. Given, that B is a single data point, the Ward minimum variance simplifies to
\begin{equation}
    \label{eq:our_ward_method}
    E_{AB} = SSE_{AB} - SSE_{A}. 
\end{equation}

 The nearest-neighbor algorithm detects a potential fall if the effort it takes to join the two clusters A and B is higher than a threshold determined from the training data. The threshold is calculated offline as the maximum $E_{AB}$ value for the safe data while $R$ is determined using distance correlation. Note that the underlying assumption for the nearest-neighbor algorithm is that cluster A only contains safe data points. 

\subsection{Data Labeling}
As one of our objectives is to maximize the lead time, we propose the use of a training lead time to label windows in a trajectory. Training lead time is defined as the difference between the time of the actual fall and the time when a  sliding window of a trajectory can be labeled as faulty. Therefore, training lead time is a subset of the maximum lead time that can be achieved in a faulty trajectory. While labeling all windows in a faulty trajectory as faulty would achieve the maximum lead time, it would also increase the rate of false positives. For instance, given two faults that are close in magnitude but where one results in a fall and the other is safe, the safe trajectory could be mistaken as a faulty trajectory. 

 If a trajectory does not contain a fall, all windows derived from the trajectory are labeled as 1. If a trajectory ends in a fall, all windows containing data points after the desired training lead time are labeled as -1. Note that for an abrupt fault. the training lead time is only defined after the push is introduced, and only the data points before a fall are kept for the training data of both faults.

 The desired training lead time is determined by a grid search algorithm that trains the algorithm of interest using a range of training lead times from 0 to 2s and evaluates the results on the training and/or validation data. The training data is included in the evaluation process for cases where the algorithm is allowed to make mistakes, such as when using a soft margin in SVM. A training lead time of 2s would label all the data points in a faulty trajectory as faulty.   



\subsection{Performance of Fall Detection Methods} 
\label{sec:fault_comparison}
 % To determine how to split the training data along faults, an experiment is conducted with each of the proposed fall detection algorithms trained using just incipient faults, just abrupt faults, and both faults. The algorithms are evaluated on the validation data. The evaluation metrics are false positive and negative rate (fpr and \fnr respectively), and the maximum average F\_LT achieved across all the trajectories. The desired fpr and \fnr is set to 0,  $N_{window}=25$, and $N_{monitor} = 1$ . The value of $N_{window}$ is set based on evaluating the performance of the algorithms across various values. Note that for both algorithms, a low value for $N_{window}$ results in higher fpr while a high value can reduce the maximum average F\_LT. $N_{monitor} = 1$ because we found that for our dataset, $N_{monitor}$ had to be set to a large value, e.g. 100 in order to result in 0 fpr.

In this section, we analyze the performance of the proposed nearest-neighbor algorithm and the SVM classifier. The algorithms are trained and evaluated on testing data across all 5 folds using just abrupt trajectories, just incipient trajectories, and both trajectories together. The evaluation metrics are false positive and negative rates, and the average lead time achieved. The desired false positive and false negative rates are set to 0. The training lead time chosen is the maximum that meets the given bounds on the false positive and false negative rates when evaluated on the training and validation data. Based on previous experiments, we set the values of the remaining hyper-parameters as $N_{window}=10$ and $N_{monitor} = 1$. 

 From \ref{tab:fault_comp_clustering} and \ref{tab:fault_comp_classification} we see that the nearest-neighbor algorithm and the classification algorithm perform similarly when trained and evaluated on the abrupt and incipient faults separately. The nearest-neighbor algorithm achieves an average escape time of 0.46s and 0.91s, respectively, for the abrupt and incipient faults, while the classification algorithm achieves an average escape time of 0.48s and 0.97s. Because our sampling time is 0.03s, the difference in the performance of both algorithms is 1 and 2 data points for the abrupt and incipient fault respectively . 

 % Even though the nearest-neighbor algorithm performs similarly to the classification algorithm, it should be noted that it requires a sampling time 'large' enough to detect faulty data from normal data. For instance, with a sampling time of 0.01s, the nearest-neighbor algorithm did not attain good results even for abrupt faults. To overcome this limitation if finer sampling times are desired, we suggest having Cluster B be further removed from Cluster A in time. For instance we are able to achieve similar results for the nearest-neighbor algorithm with a sampling time of 0.01s and Cluster B being 0.08s away from the last data point in Cluster A. 

 When both faults are trained and evaluated together, the classification algorithm achieves an average lead time 0.15s higher in comparison to the nearest-neighbor algorithm. In comparison to its average performance on the abrupt and incipient fault, the classification algorithm achieves an average lead time of 0.08s less when trained on both faults together. Similarly, the nearest-neighbor algorithm achieves an average lead time of 0.19s less. As a result, the classification algorithm outperforms the nearest-neighbor algorithm when both faults are trained together. However, because both algorithms are able to achieve lead times higher than the 0.2s, which is the lead time reflexive algorithms such as \cite{ref:hohn2009probabilistic} and \cite{ref:wu2021falling} require, both algorithms are viable options. As the nearest-neighbor algorithm learns the safe/good model, it should be used when faulty data is sparse. 

 

 

 

 % \jwg{I do not know what this first sentence means.} To determine how to split the training data along faults, an experiment is conducted with each of the proposed fall detection algorithms trained using incipient faults only, abrupt faults only, and both faults. The algorithms are evaluated on the validation data. The evaluation metrics are false positive and negative rates (\fpr and \fnrNS, respectively), and the average F\_LT achieved across all the trajectories. The desired \fpr and \fnr are set to 0.0,  $N_{window}=25$, and $N_{monitor} = 1$ . The value of $N_{window}$ is set based on evaluating the performance of the algorithms across various values \jwg{values of what?}. Note that for both algorithms, a low value for $N_{window}$ results in higher \fpr while a high value can reduce the maximum average F\_LT. $N_{monitor} = 1$ because we found that for our dataset, $N_{monitor}$ had to be set to a large value, e.g. 100 in order to result in 0 \fpr. \jwg{Last sentence must have a typo because it cannot be set to 1 because it needs to be at least 100.}
 
% Both algorithms are able to attain a higher average F\_LT with zero \fpr and \fnr when trained on the faults individually, as shown in Table  \ref{tab:fault_comp_clustering} and \ref{tab:fault_comp_classification}. A reason for this is that the optimal \jwg{not sure what optimal means here.} training F\_LT varies per fault. For instance, the training F\_LT for the classifier during the first fold that achieves the maximum average F\_LT while meeting the desired \fpr and \fnr is  0.6 for abrupt faults (only), 2.0 for incipient faults (only), and 0.2 for both faults together. A similar phenomenon can be observed for the nearest-neighbor algorithm. 

% % The nearest-neighbor algorithm is able to detect incipient faults because when the robot starts to lose balance, the foot starts to rotate, and the length of the window is able to capture the rotation. If the foot rotation is not included in the features, the nearest-neighbor algorithm is no longer able to detect incipient faults. For instance, using the following features:

% The nearest-neighbor algorithm relies on foot rotation to detect incipient faults. If foot rotation is not included in the features, the nearest-neighbor algorithm is no longer able to detect incipient faults. For instance, the following features,
% \begin{equation}
%     \begin{bmatrix}
%         \text{\rm knee~angle} \\
%         \text{\rm hip~angle} \\
%         \text{\rm vel~hip~angle} \\
%         (L_{cop} + L_{com})* sgn(p_{com}^x- p_{f_\mid}^x), 
%     \end{bmatrix}
%     \label{eq:features_f1_slow}
% \end{equation}
% result in a \fpr of 1.0 regardless of the chosen training F\_LT. On the other hand, for the classifier, these features result in a training F\_LT of 0.37s, with \fpr and \fnr equal to zero. 
\subsection{Categorizing Faults}
A means to decrease the difference in performance for both algorithms when trained on both faults together vs separately is to implement a multi-class classification problem. The labels for this multi-class classification can be identified as: abrupt fault safe (AS), abrupt fault fall (AF), incipient fault safe (IS) and incipient fault fall (IF). Using these labels with the one-vs-one or one-vs-rest multi-class classification techniques typically implemented \cite{ref:bishop2006pattern}, results in six and four detectors, respectively. However, as one-vs-rest can result in ambiguities and class imbalances and using one-vs-one can result in ambiguities and higher computational times, we seek a different approach \cite{ref:bishop2006pattern}.



% If the problem is decomposed into identifying fault and detecting falls, the number of detectors needed is only 3: a detector for identifying faults, another for detecting falls in incipient trajectories and another for detecting falls in abrupt trajectories. Furthermore, using this technique resolves the ambiguity problem as the fault identifier detector can be used to determine the operational space (abrupt vs incipient fault). 

If the problem is decomposed into classifying trajectories first into the incipient versus abrupt categories, and secondly, detecting falls (or not) within these categories of trajectories, the number of detectors needed is only three: a detector for identifying types of trajectories, a second for detecting falls in incipient trajectories, and a third for detecting falls in abrupt trajectories. Furthermore, using this technique resolves the ambiguity problem as the incipient vs. abrupt classifier can be used to determine the operational space (abrupt vs incipient fault). 

To achieve this, we propose using SVM to categorize the trajectories into incipient vs abrupt. The training data for this SVM are taken as the joint velocities, and the labels 1 and -1 are used for the incipient and abrupt faults, respectively. For the training data, the windows in abrupt trajectories before a force is applied and windows uniformly distributed throughout the incipient trajectories are labeled as incipient and only the windows containing the force are labeled as abrupt. The rest of the pre-processing process follows steps similar to those in Section \ref{sec:data_preparation}. 

\begin{table}
        \centering
        \caption{ A comparison of the nearest-neighbor algorithm's performance when trained with (1) just the abrupt fault, (2) just the incipient fault, and (3) both faults together. Note that the false positive and negate rates are 0.}
        % \begin{tabular}{|p{0.2\linewidth}|p{0.75cm}|p{0.75cm}|p{0.5cm}|p{0.5cm}|p{0.5cm}|p{0.5cm}|p{0.5cm}|  }
        \tabulinesep=0.5mm
           \begin{tabu}{|c|c|c|c|  }
     \hline\hline
      &\multicolumn{1}{c|}{\makecell{Abrupt Fault\\Only}} & \multicolumn{1}{c|}{\makecell{Incipient Fault\\Only}}& \multicolumn{1}{c|}{\makecell{Both Faults\\Together}} \\
    \hline
           Fold                                                 &\makecell{Average\\Escape\\Time}         &\makecell{Average\\Escape\\Time} 
                  &\makecell{Average\\Escape\\Time}  \\                                             \hline
        1 & 0.48 &  0.93   & 0.49 \\
        \hline
        2    & 0.46  & 0.91  & 0.49 \\
        \hline
        3    & 0.44  & 0.9  & 0.51 \\    
        \hline
        4    & 0.43  & 0.89  & 0.51  \\    
        \hline
        5   & 0.5  & 0.89  & 0.51  \\ 
        \hline
        Average   & 0.46 & 0.91 &  0.50 \\       
    \hline                                               
    \end{tabu}
        \label{tab:fault_comp_clustering}
    \end{table}

\begin{table}
        \centering
        \caption{ A comparison of the SVM classification's performance when trained with (1) just the abrupt fault, (2) just the incipient fault, and (3) both faults together. Note that the false positive and negate rates are 0. }
        % \begin{tabular}{|p{0.2\linewidth}|p{0.75cm}|p{0.75cm}|p{0.5cm}|p{0.5cm}|p{0.5cm}|p{0.5cm}|p{0.5cm}|  }
        \tabulinesep=0.5mm
           \begin{tabu}{|c|c|c|c|  }
     \hline\hline
     & \multicolumn{1}{c|}{\makecell{Abrupt Fault\\Only}} & \multicolumn{1}{c|}{\makecell{Incipient Fault\\Only}}& \multicolumn{1}{c|}{\makecell{Both Faults\\Together}} \\
    \hline
           Fold                                                 &\makecell{Average\\Escape\\Time}         &\makecell{Average\\Escape Time} 
                  &\makecell{Average\\Escape Time}  \\                                             \hline
        1 & 0.5 &  1.0   & 0.65 \\
        \hline
        2    & 0.47  & 0.96  & 0.66 \\
        \hline
        3    & 0.49  & 0.98  & 0.66 \\    
        \hline
        4    & 0.44  & 0.97  & 0.65  \\    
        \hline
        5   & 0.51  & 0.96  & 0.63  \\ 
        \hline
        Average   & 0.48 & 0.97 & 0.65 \\       
    \hline                                               
    \end{tabu}
        \label{tab:fault_comp_classification}
    \end{table}


 % \begin{center}
%     \begin{table*}
%         \centering
%         \caption{ The maximum average F\_LT yielding a \fpr and \fnr of 0 that can be achieved by the clustering algorithm trained and evaluated on the validation data with (1) just the abrupt fault, (2) just the incipient fault, and (3) both faults together. $N_{monitor}$ =1 and $N_{window}$ =25. The maximum average F\_LT that can be achieved across all folds for the validation data is 0.34s, 0.8s and 0.7s for the abrupt, incipient, and both faults respectively}
%         % \begin{tabular}{|p{0.2\linewidth}|p{0.75cm}|p{0.75cm}|p{0.5cm}|p{0.5cm}|p{0.5cm}|p{0.5cm}|p{0.5cm}|  }
%         \tabulinesep=0.5mm
%            \begin{tabu}{|c|c|c|c|c|c|c|  }
%      \hline\hline
%      & \multicolumn{2}{c|}{\makecell{Abrupt\\Fault\\Only}} & \multicolumn{2}{c|}{\makecell{Incipient\\Fault\\Only}}& \multicolumn{2}{c|}{\makecell{Both\\Faults\\Together}} \\
%     \hline
%            Fold                                         &\makecell{Training\\Escape\\Time}        &\makecell{Average\\Escape\\Time}  &\makecell{Training\\Escape\\Time}        &\makecell{Average\\Escape\\Time} 
%             &\makecell{Training\\Escape\\Time}        &\makecell{Average\\Escape\\Time}  \\                                             \hline
%         1   & 0.7 & 0.33 & ~0.3 & 0.24  & ~0.3 & 0.05 \\
%         \hline
%         2   & 0.7 & 0.29 & ~0.3 & 0.23 & ~0.3 & 0 \\
%         \hline
%         3   & 0.7 & 0.24 & ~0.3 & 0.29 & ~0.3 & 0.01 \\    
%         \hline
%         4   & 0.5 & 0.21 & ~0.3 & 0.23 & ~0.3 & 0.04  \\    
%         \hline
%         5   & 0.5 & 0.22 & ~0.3 & 0.26 & ~0.3 & 0.04  \\ 
%         \hline
%         Average   &  & 0.25 &  & 0.25 &  & 0.03 \\       
%     \hline                                               
%     \end{tabu}
%         \label{tab:fault_comp_clustering}
%     \end{table*}
% \end{center}


%  \begin{center}
%     \begin{table*}
%         \centering
%         \caption{ The maximum average F\_LT yielding a \fpr and \fnr of 0 that can be achieved by the classification algorithm trained and evaluated on the validation data with (1) just the abrupt fault, (2) just the incipient fault, and (3) both faults together. $N_{monitor}$ =1 , $N_{window}$ =25, and a weight of 2 is used for normal data. The maximum average F\_LT that can be achieved across all folds for the validation data is 0.34s, 0.8s and 0.7s for the abrupt, incipient, and both faults respectively}
%         % \begin{tabular}{|p{0.2\linewidth}|p{0.75cm}|p{0.75cm}|p{0.5cm}|p{0.5cm}|p{0.5cm}|p{0.5cm}|p{0.5cm}|  }
%         \tabulinesep=0.5mm
%            \begin{tabu}{|c|c|c|c|c|c|c|  }
%      \hline\hline
%      & \multicolumn{2}{c|}{\makecell{Abrupt\\Fault\\Only}} & \multicolumn{2}{c|}{\makecell{Incipient\\Fault\\Only}}& \multicolumn{2}{c|}{\makecell{Both\\Faults\\Together}} \\
%     \hline
%            Fold                                         &\makecell{Training\\Escape\\Time}        &\makecell{Average\\Escape\\Time}  &\makecell{Training\\Escape\\Time}        &\makecell{Average\\Escape\\Time} 
%             &\makecell{Training\\Escape\\Time}        &\makecell{Average\\Escape\\Time}  \\                                             \hline
%         1   & 0.6 & 0.31 & 2.0 & 0.36  & ~0.2 & 0.13 \\
%         \hline
%         2   & 0.9 & 0.27 & ~0.5 & 0.32 & ~0.2 & 0.14 \\
%         \hline
%         3   & 0.5 & 0.22 & ~0.4 & 0.28 & ~0.2 & 0.12 \\    
%         \hline
%         4   & 2.0 & 0.2 & ~2 & 0.34 & ~0.2 & 0.2  \\    
%         \hline
%         5   & 2.0 & 0.29 & ~0.3 & 0.24 & ~0.2 & 0.13  \\ 
%         \hline
%         Average   &  & 0.25 &  & 0.31 &  & 0.03 \\       
%     \hline                                               
%     \end{tabu}
%         \label{tab:fault_comp_classification}
%     \end{table*}
% \end{center}


% \subsection{Varying the training F\_LT}
% The output of the algorithms are computed for a range of training F\_LT, 0-2s with $N_{window}=25$ (equivalent to 0.25s). 

% From Figures \ref{fig:fastActingFaults}  we see that the classification and clustering algorithm perform similarly for abrupt faults. However, as depicted in Figure \ref{fig:slowActingFaults}, the classification algorithm attains the desired \fpr and \fnr with a maximum average escape time of about 0.3s for the incipient faults while the clustering algorithm is unable to achieve both the \fnr,\fpr, and minimum average escape time. When comparing the performance of the classifier trained on both faults (Figure \ref{fig:classifier_both})  with its performance trained with just the abrupt fault (Figure \ref{fig:classifier_fast})  or just the incipient fault (Figure \ref{fig:classifier_slow}), we see that the classifier performs better when trained on each fault individually. More specifically, the classifier is able to attain a higher escape time with 0 false positive and negative rates, when it's trained on the faults individually. This conclusion is also reflected in the clustering algorithm as depicted in Figure \ref{fig:clustering_both}. 











% \subsection{Varying the number of data points in a window}
% Here, the training escape time is set to 2s and $N_{window}$ is varied from 3 - 150 data points. 

% Both algorithms are able to achieve the desired \fpr, \fnr and minimum escape time for the abrupt faults. The clustering algorithm attains a maximum average escape time of 0.26s with 66 data points while the classification algorithm has a maximum escape time of 0.28s with 3 data points.  For the incipient faults, both algorithms are able to achieve the desired \fpr, \fnr, and minimum average escape time but only the classification algorithm meets the desired minimum escape time threshold with a maximum  average escape time of 0.42s. None of the algorithms are able to achieve the desired \fpr, \fnr, and minimum average escape time threshold for both faults together. Figures \ref{fig:slowActingFaults_dataPoints}, \ref{fig:fastActingFaults_dataPoints}, and \ref{fig:bothActingFaults_dataPoints} display the results for the abrupt, incipient and both faults respectively. Note that the results of the clustering algorithm are not collected after 66 data points and 93 data points for the abrupt and incipient faults respectively because of its assumption that the first window contains only safe data points. 










%*****************************************************************************************************************************************
%  \begin{center}
%     \begin{table}
%         \centering
%         \caption{ The maximum average escape time for the incipient fault that can be achieved by both the clustering and classification algorithm using the training escape time that results in a \fpr and \fnr of 0.  $N_{monitor}$ =1. $N_{window} =25 $ for all the folds in the clustering algorithm. $N_{window}$ is set to 25 and 15 for all the folds 1-3 and 4-5 respectively in the classification algorithm}
%         % \begin{tabular}{|p{0.2\linewidth}|p{0.75cm}|p{0.75cm}|p{0.5cm}|p{0.5cm}|p{0.5cm}|p{0.5cm}|p{0.5cm}|  }
%         \tabulinesep=0.5mm
%            \begin{tabu}{|c|c|c|c|c|  }
%      \hline\hline
%      & \multicolumn{2}{c}{\makecell{Clustering\\Algorithm}} & \multicolumn{2}{c}{\makecell{Classification\\Algorithm}} \\
%     \hline
%            Fold                                         &\makecell{Training\\Escape\\Time}         &\makecell{Average\\Escape\\Time}   &\makecell{Training\\Escape\\Time}        &\makecell{Average\\Escape\\Time} \\                                       
%      \hline
%         1   &  ~0.3 & 0.22  &   ~~2 & 0.36 \\
%         \hline
%         2   &  ~0.3 & 0.23 &  ~0.5 & 0.32 \\
%         \hline
%         3   &  ~0.4 & 0.29 &  ~0.4 & 0.28 \\    
%         \hline
%         4   &  ~0.3 & 0.23 &  ~~2 & 0.38 \\    
%         \hline
%         5   &  ~0.3 & 0.16 &  ~0.3 & 0.25 \\ 
%         \hline
%         Average &    & 0.23 &  & 0.32 \\       
%     \hline                                               
%     \end{tabu}
%         \label{tab:incipient_trainEscapeTime}
%     \end{table}
% \end{center}


%  \begin{center}
%     \begin{table}
%         \centering
%         \caption{ The maximum average escape time for the abrupt fault that can be achieved by both the clustering and classification algorithm using the training escape time that results in a \fpr and \fnr of 0 when evaluated on the validation data. Note that $N_{monitor}$ =1. $N_{window} =25 $ for all the folds in the clustering algorithm. $N_{window}=15$ for all the folds in the classification algorithm   }
%         % \begin{tabular}{|p{0.2\linewidth}|p{0.75cm}|p{0.75cm}|p{0.5cm}|p{0.5cm}|p{0.5cm}|p{0.5cm}|p{0.5cm}|  }
%         \tabulinesep=0.5mm
%            \begin{tabu}{|c|c|c|c|c| }
%      \hline\hline
%      & \multicolumn{2}{c}{\makecell{Clustering\\Algorithm}} & \multicolumn{2}{c}{\makecell{Classification\\Algorithm}} \\
%     \hline
%            Fold                                        &\makecell{Training\\Escape\\Time}         &\makecell{Average\\Escape\\Time}   &\makecell{Training\\Escape\\Time}        &\makecell{Average\\Escape\\Time} \\                                              
%      \hline
%         1   &  ~0.6 & 0.33 &  ~~2 & 0.32 \\
%         \hline
%         2   &  ~0.6 & 0.29 &  ~0.9 & 0.28 \\
%         \hline
%         3   &  ~0.6 & 0.23 &  ~0.5 & 0.23 \\    
%         \hline
%         4   &  ~0.5 & 0.21 &  ~~2 & 0.2 \\    
%         \hline
%         5   &  ~0.5 & 0.22 &  ~~2 & 0.30 \\ 
%         \hline
%         Average   &  & 0.26 &  & 0.27 \\       
%     \hline                                               
%     \end{tabu}
%         \label{tab:abrupt_trainEscapeTime}
%     \end{table}
% \end{center}

%  \begin{center}
%     \begin{table}
%         \centering
%         \caption{ The minimum average escape time for the abrupt fault that can be achieved by both the clustering and classification algorithm using the training escape time that results in a fpr and \fnr of 0.  $N_{monitor}$ =1}
%         % \begin{tabular}{|p{0.2\linewidth}|p{0.75cm}|p{0.75cm}|p{0.5cm}|p{0.5cm}|p{0.5cm}|p{0.5cm}|p{0.5cm}|  }
%         \tabulinesep=0.5mm
%            \begin{tabu}{|c|c|c|c|c|  }
%      \hline\hline
%      & \multicolumn{2}{c}{\makecell{Clustering\\Algorithm}} & \multicolumn{2}{c}{\makecell{Classification\\Algorithm}} \\
%     \hline
%            Fold                                         &\makecell{Training\\Escape\\Time}        &\makecell{Average\\Escape\\Time}  &\makecell{Training\\Escape\\Time}        &\makecell{Average\\Escape\\Time}  \\                                       
%      \hline
%         1   & ~0.3 & 0.05 &  ~~0.2 & 0.13 \\
%         \hline
%         2   & ~0.3 & 0 &  ~0.2 & 0.14 \\
%         \hline
%         3   & ~0.3 & 0.01 &  ~0.2 & 0.12 \\    
%         \hline
%         4   & ~0.3 & 0.04 &  ~0.2 & 0.2 \\    
%         \hline
%         5   & ~0.3 & 0.04 &  0.2 & 0.13 \\ 
%         \hline
%         Average   &  & 0.03 &  & 0.14 \\       
%     \hline                                               
%     \end{tabu}
%         \label{tab:both_trainEscapeTime}
%     \end{table}
% \end{center}


\section{LayoutDM}
Our LayoutDM builds on discrete-state space diffusion models~\cite{austin2021structured,gu2022vector}.
We first briefly review the fundamental of discrete diffusion models in \cref{subsec:discrete_diffusion}.
\cref{subsec:layout_diffusion_unconditional} explains our approach to layout generation within the diffusion framework while discussing features inherent in layout compared with text.
\cref{subsec:layout_diffusion_conditional} discusses how we extend denoising steps to perform various conditional layout generation by imposing conditions in each step of the reverse process.

\subsection{Preliminary: Discrete Diffusion Models}
\label{subsec:discrete_diffusion}


Diffusion models~\cite{sohl2015deep} are generative models characterized by a forward and reverse Markov process.
While many diffusion models are defined on continuous space with Gaussian corruption, D3PM~\cite{austin2021structured} introduces a general diffusion framework for categorical variables designed primarily for texts.
Let $T \in \mathbb{N}$ be a total timestep of the diffusion model, we first explain the forward diffusion process.
For a scalar discrete variable with $K$ categories $z_{t} \in \{1,2,\ldots,K\}$ at timestep $t \in \mathbb{N}$, probabilities that $z_{t-1}$ transits to $z_{t}$ are defined by using a transition matrix $\bm{Q}_{t} \in [0,1]^{K \times K}$, with $[Q_{t}]_{mn} = q(z_{t}\!=\!m | z_{t-1}\!=\!n)$,

\begin{equation}
q(z_{t}|z_{t-1}) = \bm{v}(z_{t})^{\!\top} \mathbf{Q}_{t} \bm{v}(z_{t-1}),
\end{equation}
where $\bm{v}(z_{t}) \in \{0,1\}^{K}$ is a column one-hot vector of $z_{t}$.
The categorical distribution over $z_{t}$ given $z_{t-1}$ is computed by a column vector $\mathbf{Q}_{t} \bm{v}(z_{t-1}) \in [0,1]^{K}$.
Assuming the Markov property,
we can derive $q(z_{t}|z_{0}) = \bm{v}(z_{t})^{\!\top}\overline{\mathbf{Q}}_{t}\bm{v}(z_{0})$ where $\overline{\mathbf{Q}}_{t}\!=\!\mathbf{Q}_{t}\mathbf{Q}_{t-1}\cdots\mathbf{Q}_{1}$ and:
\begin{align}
    &q(z_{t-1}|z_{t}, z_{0}) = \frac{
        q(z_{t}|z_{t-1}, z_{0})\,q(z_{t-1}|z_{0})
    }{
        q(z_{t}|z_{0})
    } \nonumber \\
    &= \frac{
        \left(\bm{v}\!\left(z_{t}\right)^{\!\top}\!\mathbf{Q}_{t}\bm{v}\!\left(z_{t-1}\right)\right)
        \left( \bm{v}\!\left(z_{t-1}\right)^{\!\top}\overline{\mathbf{Q}}_{t-1}\bm{v}\!\left(z_{0}\right) \right)
    }{
        \bm{v}\!\left(z_{t}\right)^{\!\top}\overline{\mathbf{Q}}_{t}\bm{v}\!\left(z_{0}\right)
    }. \label{eq:q_posterior}
\end{align}
Note that due to the Markov property, $q(z_{t}|z_{t-1}, z_{0})=q(z_{t}|z_{t-1})$.
When we consider $N$-dimensional variables $\bm{z}_{t} \in \{1,2,\ldots,K\}^{N}$, the corruption is applied to each variable $z_{t}$ independently.
In the following, we explain with $N$-dimensional variables $\bm{z}_{t}$.


In contrast to the forward process, the reverse denoising process considers a conditional distribution of $\bm{z}_{t-1}$ over $\bm{z}_{t}$ by a neural network $p_{\theta}(\bm{z}_{t-1}|\bm{z}_{t}) \in [0,1]^{N \times K}$.
$\bm{z}_{t-1}$ is sampled according to this distribution.
Note that the typical implementation is to predict unnormalized log probabilities $\log p_{\theta}(\bm{z}_{t-1}|\bm{z}_{t})$ by a stack of bidirectional Transformer encoder blocks.
D3PM uses a neural network $\tilde{p}_{\theta}(\tilde{\bm{z}}_{0}|\bm{z}_{t})$, combines it with the posterior $q(\bm{z}_{t-1}|\bm{z}_{t},\bm{z}_{0})$, and sums over possible $\tilde{\bm{z}_{0}}$ to obtain the following parameterization:
\begin{equation}
p_{\theta}(\bm{z}_{t-1}|\bm{z}_{t}) \propto \sum_{\tilde{\bm{z}}_{0}}q(\bm{z}_{t-1}|\bm{z}_{t},\tilde{\bm{z}}_{0})~\tilde{p}_{\theta}(\tilde{\bm{z}}_{0}|\bm{z}_{t}). \label{eq:single_step_in_inference}
\end{equation}

In addition to the commonly used variational lower bound objective $\mathcal{L}_\mathrm{vb}$, D3PM introduces an auxiliary denoising objective. The overall objective is as follows:
\begin{equation}
    \mathcal{L}_{\lambda} = \mathcal{L}_\mathrm{vb} + \lambda
    \underset{\substack{ \bm{z}_{t} \sim q(\bm{z}_{t}|\bm{z}_{0}) \\ \bm{z}_{0} \sim q(\bm{z}_{0})}}{\mathbb{E}}
    \left[ -\log \tilde{p}_{\theta}\left(\bm{z}_{0}|\bm{z}_{t}\right) \right], \label{eq:total_loss}
\end{equation}
where $\lambda$ is a hyper-parameter to balance the two loss terms.

Although D3PM proposes many variants of $\mathbf{Q}_{t}$, VQDiffusion~\cite{gu2022vector} offers an improved version of $\mathbf{Q}_{t}$ called mask-and-replace strategy.
They introduce an additional special token \texttt{[MASK]}
and three probabilities $\gamma_{t}$ of replacing the current token with the \texttt{[MASK]} token, $\beta_{t}$ of replacing the token with other tokens, and $\alpha_{t}$ of not changing the token.
The \texttt{[MASK]} token never transitions to other states.
The transition matrix $\mathbf{Q}_{t} \in [0,1]^{(K+1)\times(K+1)}$ is defined by:
\begin{equation}
\mathbf{Q}_{t} = \begin{bmatrix}
\alpha_{t}+\beta_{t} & \beta_{t} & \cdots & \beta_{t} & 0  \\
\beta_{t} & \alpha_{t}+\beta_{t} & \cdots & \beta_{t} & 0  \\
\vdots & \vdots & \ddots & \beta_{t} & 0  \\
\beta_{t} & \beta_{t} & \beta_{t} & \alpha_{t}+\beta_{t} &  0  \\
\gamma_{t} & \gamma_{t} & \gamma_{t} & \gamma_{t} & 1 \\
\end{bmatrix}.
\label{eq:Q_mask_and_replace}
\end{equation}
$(\alpha_{t}, \beta_{t}, \gamma_{t})$ is carefully designed so that $z_{t}$ converges to the \texttt{[MASK]} token for sufficiently large $t$.
During testing, we start from $\bm{z}_{T}$ filled with \texttt{[MASK]} tokens and iteratively sample new set of tokens $\bm{z}_{t-1}$ from $p_{\theta}(\bm{z}_{t-1}|\bm{z}_{t})$.

\subsection{Unconditional Layout Generation}
\label{subsec:layout_diffusion_unconditional}

\begin{figure}[t]
    \centering
    \includegraphics[width=\hsize]{images/overview.pdf}
    \caption{
        Overview of the corruption and denoising processes in LayoutDM.
        For simplicity, we use a toy layout consisting of two elements and the model generates three elements at maximum.
    }
    \label{fig:overview}
\end{figure}

A layout $l$ is a set of elements represented by $l = \left\{\left(c_{1}, \bm{b}_{1}\right), \ldots, \left(c_{E}, \bm{b}_{E}\right) \right\}$. $E \in \mathbb{N}$ is the number of elements in the layout. $c_{i} \in \{1, \ldots, C\}$ is categorical information of the $i$-th element in the layout. $\bm{b}_{i} \in [0,1]^4$ is the bounding box of the $i$-th element in normalized coordinates, where the first two values indicate the center location, and the last two indicate the width and height.
Following previous works~\cite{arroyo2021variational,gupta2021layout,kong2022blt} that regard layout generation as generating a sequence of tokens, we
quantize each value in $\bm{b}_{i}$ and obtain $[x_{i}, y_{i}, w_{i}, h_{i}]^\top \in \{1, \ldots, B\}^4$, where $B$ is a number of the bins. The layout $l$ is now represented by $l = \left\{\left(c_{1}, x_{1}, y_{1}, w_{1}, h_{1}\right), \ldots \right\}$.

In this work, we corrupt a layout in a modality-wise manner in the forward process, and we denoise the corrupted layout while considering all elements and modalities in the reverse process, as we illustrate in \cref{fig:overview}.
Similarly to D3PM~\cite{austin2021structured}, we parameterize $p_{\theta}$ by a Transformer encoder~\cite{vaswani2017attention}, which processes an ordered 1D sequence.
To process $l$ by $p_{\theta}$ while avoiding the order dependency issue~\cite{kong2022blt}, we randomly shuffle $l$ in element-wise manner and then flatten it to produce $l_\mathrm{flat} = (c_{1}, x_{1}, y_{1}, w_{1}, h_{1}, c_{2}, x_{2}, \ldots )$.


\paragraph{Variable length generation}
Existing diffusion models generate fixed-dimensional data and are not directly applicable to the layout generation because the number of elements in each layout varies.
To handle this, we introduce a \texttt{[PAD]} token and define a maximum number of elements in the layout as $M \in \mathbb{N}$.
Each layout is fixed-dimensional data composed of $5M$ tokens by appending $5(M-E)$ \texttt{[PAD]} tokens.
\texttt{[PAD]} is treated similarly to the ordinary token in VQDiffusion and $\mathbf{Q}_{t}$'s dimension becomes $(K+2) \times (K+2)$.

\paragraph{Modality-wise diffusion}
Discrete state-space models assume that all the standard tokens are switchable by corruption. However, layout tokens comprise a disjoint set of token groups for each attribute in the element. For example, applying the transition rule \cref{eq:Q_mask_and_replace} may change a token representing an element's category to another token representing the width.
To avoid such invalid switching, we propose to apply disjoint corruption matrices $\mathbf{Q}_{t}^{c},\mathbf{Q}_{t}^{x},\mathbf{Q}_{t}^{y},\mathbf{Q}_{t}^{w},\mathbf{Q}_{t}^{h}$ for tokens representing different attributes $c, x, y, w, h$, as we show in \cref{fig:overview}. The size of each matrix is $(C+2) \times (C+2)$ for $\mathbf{Q}_{t}^{c}$ and otherwise $(B+2) \times (B+2)$, where $+2$ is for \texttt{[{PAD}]} and \texttt{[{MASK}]}.

\paragraph{Adaptive Quantization}
The distribution of the position and size information in layouts is highly imbalanced; e.g., elements tend to be aligned to either left, center, or right.
Applying uniform quantization to those quantities as in existing layout generation models~\cite{arroyo2021variational,gupta2021layout,kong2022blt} results in the loss of information.
As a pre-processing, we propose to apply a classical clustering algorithm, such as KMeans~\cite{macqueen1967classification} on $x$, $y$, $w$, and $h$ independently to obtain balanced position and size tokens for each dataset.
We show in \cref{sec:ablation_study} how quantization strategy affects the resulting quality.

\paragraph{Decoupled Positional Encoding}
Previous works apply standard positional encoding to a flattened sequence of layout tokens $l_\mathrm{flat}$~\cite{arroyo2021variational,gupta2021layout,kong2022blt}.
We argue that this flattening approach could lose the structure information of the layout and lead to inferior generation performance.
In layout, each token has two types of indices: $i$-th element and $j$-th attribute.
We empirically find that independently applying positional encoding to those indices improves final generation performance, which we study in \cref{sec:ablation_study}.




\subsection{Conditional Generation}
\label{subsec:layout_diffusion_conditional}
We elaborate on solving various conditional layout generation tasks using pre-trained frozen LayoutDM.
We inject conditional information in both the initial state $\bm{z}_{T}$ and sampled states $\{\bm{z}_{t}\}_{t=0}^{T-1}$ during inference but do not modify the denoising network $p_{\theta}$.
The actual implementation of the injection differs by the type of conditions.

\paragraph{Strong Constraints}
The most typical condition is partially known layout fields.
Let $\bm{z}^\mathrm{known} \in \mathbb{Z}^{N}$ contain the known fields and $\bm{m} \in \{0, 1\}^{N}$ be a mask vector denoting the known and unknown field as $1$ and $0$, respectively. In each timestep $t$, we sample $\hat{\bm{z}}_{t-1}$ from $p_{\theta}(\bm{z}_{t-1}|\bm{z}_{t})$ in \cref{eq:single_step_in_inference} and then inject the condition by $\bm{z}_{t-1} = \bm{m} \odot \bm{z}^\mathrm{known} + (\bm{1} - \bm{m}) \odot \hat{\bm{z}}_{t-1}$, where $\bm{1}$ denotes a $N$-dimensional all-ones vector and $\odot$ denotes element-wise product.

\paragraph{Weak Constraints}
We may impose a weaker constraint during generation, such as an element in the center. We offer a way to impose such constraints in a unified framework without additional training or external neural network models.
We propose to adjust the logits to inject weak constraints in log probability space by
\begin{equation}
    \log \hat{p}_{\theta}(\bm{z}_{t-1}|\bm{z}_{t}) \propto \log p_{\theta}(\bm{z}_{t-1}|\bm{z}_{t}) + \lambda_{\pi} \bm{\pi}, \\ \label{eq:prior_addition}
\end{equation}
where $\bm{\pi} \in \mathbb{R}^{N \times K}$ is a prior term that weights the desired outputs, and $\lambda_{\pi} \in \mathbb{R}$ is a hyper-parameter.
The prior term can be defined either hard-coded (Refinement in \cref{sec:quantitative_evaluation}) or through differentiable loss functions (Relationship in \cref{sec:quantitative_evaluation}).
Let $\{\mathcal{L}_i\}_{i=1}^L$ be a set of differentiable loss functions given the prediction, the later prior definition can be written by:
\begin{equation}
    \bm{\pi} = -\nabla_{p_{\theta}(\bm{z}_{t-1}|\bm{z}_{t})} \sum_{i=1}^{L} \mathcal{L}_{i}\left(p_{\theta}\left(\bm{z}_{t-1}|\bm{z}_{t}\right)\right). \\ \label{eq:prior_addition_by_gradient}
\end{equation}
Although the formulation of \cref{eq:prior_addition_by_gradient} resembles steering diffusion models by gradients from external models ~\cite{dhariwal2021diffusion,liu2022compositional}, our primal focus is incorporating classical hand-crafted energies for aesthetics principles of layout~\cite{o2014learning} that do not depend on an external model.
In practice, we tune the hyper-parameters for imposing weak constraints, such as $\lambda_{\pi}$.
Note that these hyper-parameters are only for inference and are easier to tune than the other training hyper-parameters.


\section{Results}
\label{results}

\begin{figure*}[ht]
    \centering
    \includegraphics[scale=0.15,trim={0 2.5cm 0 5cm},clip]{images/aoi-single_burst}
    \caption{The time average peak Age of Information with burst and \gls{soa} loss values against the dynamic reliability logic for different network topologies.}
    \label{fig:aoi_burst}\vspace{-0.4cm}
\end{figure*}


This paper focuses on both transport layer and application layer metrics to determine the feasibility of dynamic reliability. For this, we have selected the session packet volume, as transmitted, retransmitted, lost and backlogged packets as \glspl{kpi} for the transport layer; while focusing on the \gls{aoi} for the application layer. The \gls{aoi} was chosen as a crucial indicator for the freshness of packets in real-time applications. More specifically, this work adopts the time average peak \gls{aoi} equation \cite{aoi_equation} depicted in Eq. \ref{aoi}, where $\Delta(r_{i+1})$ is the $i$th update at the time it was received at the server, for a session time period of $\tau$.

\begin{equation}
    \label{aoi}
    \gls{aoi}_\tau = \frac{1}{n-1}\sum_{i=1}^{n-1} \Delta(r_{i+1})
\end{equation}

We include a comparison between the vanilla QUIC implementation which does not enjoy the dynamic reliability extension, with a number of dynamic reliability policies. The tests were run a number of times for statistical significance, with the mean value of vanilla implementation used as a baseline for comparison. The topology utilised both random loss and bursty loss to explore the bounds of dynamic reliability. The \gls{soa} loss in the figures correspond to the loss values presented in Table. \ref{tab:path_char}, for ease of comparison between bursty and random loss scenarios.

\subsection{Transport-Layer KPIs}

To analyse the performance gain at the transport layer due to dynamic reliability, the volume of transmitted and backlogged packets is examined. The figures are in the form of boxplots, which take the vanilla implementation as a benchmark, depicted as the red dashed line.

As seen in Fig. \ref{fig:sent_burst}, the loss plays a crucial role in the performance of the reliability policies. The policies under random loss did incredibly well for the networks with a larger capacity, namely \gls{mmwave} and Sub-6~GHz, whereas for burst loss, the lower network capacities had a larger packet reduction. With the increase in burst loss, the behaviour of the set split reliable policies became unpredictable, if a reliable assignment happened to coincide with a burst loss, the number of transmitted packets increases, and vice versa. On the other hand, in smarter policies, such as Loss-Aware, the performance lightly matched the vanilla baseline, as the reliable assignment dominated the session to compensate for a higher burst loss. Not only that but, the burst loss also impacted the variance of the transmitted packets for the policies.

Unsurprisingly, the unreliable focused policy, 80-20 split, outperformed other policies for all topologies in random and bursty loss scenarios, with an approximate reduction of 80\%. That being said, the majority of the policies reduced the transmitted packets on the link by approximately 70\% for random loss, while the reduction started at $\approx 15\%$ and decreased as the loss increased for the burst loss scenario.

The retransmitted and lost packets, not shown due to space limitations, followed the same trend as the transmitted packets for the random loss scenarios. However, for the burst loss scenarios, the larger capacity networks had a lower reduction in the retransmitted and lost packets. This can be seen as a favorable outcome since the lower capacity networks are scarce on resources. It is important to note that the Loss-Aware policy mimicked the vanilla approach as the burst loss increased, signifying the overwhelming appointment of reliable packets in adapting to the harsh burst loss conditions.
 
Alternatively, Fig. \ref{fig:backlog_burst} clearly shows a stark comparison between the policies and loss scenario in the reduction of the backlogged packets. The Loss-Aware policy for random loss scenario reduced the backlogged packets by up to 50\%, beating all other policies by approximately 30\%. Furthermore, it is clear that the unreliability focused policies resulted in the lowest backlog for the session. In comparison, we notice that the burst loss and the backlogged frequency have a positive correlation, where the maximum reduction of the backlogged packets for the policies is at most 20\%. Much like the transmitted packets, the probability of a burst loss occurrence plays a vital role in the number of retransmissions sent and by extension the number of backlogged packets. Thus, we can conclude that the stress placed on the buffer is a result of the reliable packets which is tightly coupled with the congestion on the session. Whereas, unreliable focused policies did not encounter such a phenomenon regardless if it was experiencing a burst loss.


\subsection{Application-Layer KPIs}

The feasibility of dynamic reliability for real-time applications can be determined by the \gls{aoi}, with comparison across different topologies and policies. If we take a strict approach and consider anything below $10$~ms is real-time \cite{real-time}, then all the reliability policies passed that requirement, which is attractive for real-time applications, as shown in Fig. \ref{fig:aoi_burst}. Utilising the median as an estimate of the runs, the policies in the WLAN and Sub-6~GHz topology with random loss floated around $4-5$~ms with negligible difference, while the \gls{aoi} for \gls{mmwave} was $\approx 2-3$~ms. It is clear that the \gls{aoi} and the network capacity have a negative correlation, as the network capacity decreases, the \gls{aoi} increases. The same correlation is extended to the bursty loss scenarios, where \gls{mmwave} dominated the other topologies. That being said, it is crucial to note that the \gls{aoi} for the reliability policies is often slightly better than or equal to the \gls{aoi} of the vanilla implementation, proving that dynamic reliability reduces the congestion of the session at no cost to the \gls{aoi}.


\section{Conclusion}\label{sec:conclusion}
In this work, we focus on addressing the fundamental challenge of OOD detection tasks, which is how to fully understand the semantic discrepancy between the ID/OOD samples. We reveal that the key to success in the realistic SCOOD task is to allocate as many ID samples in the unlabeled set correctly as possible. To this end, we propose a novel uncertainty-aware optimal transport scheme that introduces class-specific energy scores as guidance for effective label assignment. Experimental results show that our method achieves better performance than previous state-of-the-art methods on SCOOD benchmarks.

\textbf{Limitations.} In addition to temperature scaling, other techniques such as feature clipping applied in ReAct~\cite{sun2021react} also enhance the performance of energy score, so how to obtain an OOD score that best fits the SCOOD task can be further explored. Moreover, a setting highly related to SCOOD has been proposed in \cite{katz2022training} and formulated as a constrained optimization problem. We will also theoretically analyze these practical OOD settings in our feature work.

% \section*{Acknowledgments}
\textbf{Acknowledgments.} 
This work is supported by National Key R\&D Program of China under Grant 2020AAA0105701, National Natural Science Foundation of China (NSFC) under Grants 61872327, Major Special Science and Technology Project of Anhui, National Natural Science Foundation of China (62033012) and Ant Group through Ant Research Intern Program.

% \section{Feature Selection}
\label{sec:feature_selection}
With the proposed algorithm achieving the desired fnr, fpr and minimum escape time across all folds, the next step is to try and improve its performance by selecting better features.  Feature selection is important as the wrong choice of features can decrease the performance of the algorithm. For instance,  using the features shown in Equation \ref{eq:features_f1_slow}

\begin{equation}
    \begin{bmatrix}
        \text{\rm knee~angle} \\
        \text{\rm hip~angle} \\
        \text{\rm vel~hip~angle} \\
        (L_{cop} + L_{com})* sgn(p_{com}^x- p_{f_\mid}^x) 
    \end{bmatrix}
    \label{eq:features_f1_slow}
\end{equation}

where $L_{cop}$ is the angular momentum about the contact point to train the SVM model on just the incipient faults with 0.9 training escape time results in 0 false positive and negative rates and an average escape time of 0.4, 0.37 and 0.38 for the training, validation, and testing sets. Whereas using the features shown in Equation \ref{eq:features_0.9_corr}
\begin{equation}
    \begin{bmatrix}
        L_{com}\\
        \text{\rm knee~angle} \\
        \text{\rm hip~angle} \\
        \text{\rm vel~knee~angle} \\
        \text{\rm vel~hip~angle} \\
        \text{\rm vel~torso} \\
        p_{com}^x - p_{cop}^x \\
        p_{com}^z- p_{f_\mid}^z
    \end{bmatrix}
    \label{eq:features_f1_slow}
\end{equation}

also results in 0 false positive and negative rates but only 0.2,0.16 and 0.18 average escape times for training, validation, and testing data. The goal of feature selection is to select features based on their redudancy and relevance. Feature selection algorithms are typically split into 3 methods, wrapper, filter and embedded methods. Wrapper methods use greedy search algorithms to add or remove features based on the chosen model's performance. Filter methods use statistical scores such as Pearson's correlation to select features. In embedded methods, feature selection is built in the chosen model. \eva{cite the following papers https://sciendo.com/article/10.2478/cait-2019-0001, https://www.frontiersin.org/articles/10.3389/fbinf.2022.927312/}. Here, we try to select features using the wrapper and filter methods. Since we are using the radial basis function kernel with our SVM classifiers, we can't explicitly compute the weight vector which would aid in determining the relevant features. \eva{cite this paper https://link.springer.com/article/10.1007/s10462-011-9205-2 } Note that there are researchers such as \eva{cite this paper https://link.springer.com/article/10.1007/s10462-011-9205-2 } who have put forward methods for feature selection for SVM with RBF kernel, however, as we are using sklearn's SVM implementation, we choose to forego embedded feature selection with the SVM classifier.  For the wrapper method we use the forward and backward sequential feature selection. The features are selected from commonly used features in literature, and are displayed in Equation \ref{eq:features_raw}. 

\begin{equation}
    \begin{bmatrix}
        L_{cop} - L_{com}\\
        L_{com} \\
        L_{cop} \\
        p_{cop}^x\\
        p_{com} \\
        v_{com} \\
        p_{com} - p_{heel} \\
        p_{toe} - p_{com}\\
        q \\
        \dot{q}\\
        p_{com} - p_{cop} \\
        (L_{cop} + L_{com})* sgn(p_{com}^x- p_{f\_mid}^x) \\
        p_{f\_mid} \\
        torso \\
        dtorso        
    \end{bmatrix}
    \label{eq:features_raw}
\end{equation}


In keeping up with the example from the beginning of this section, the SVM is trained on the selected features using the incipient and abrupt faults individually with 0.9 and 0.6 training escape times respectively. The training escape times are chosen from the first fold results in Figure \ref{fig:classifier_results} \eva{cite table when it becomes available}. 

\subsection{Distance Correlation Feature Selection}
The features are selected such that they have a correlation of greater than 0 with the escape time and less than 1 with each other. To reduce the number of features selected, the correlation thresholds can be increased and decreased respectively. The second requirement of non redundancy requires one to make a design choice between features. As a result, there are multiple features which can be derived from the same conditions. However, the performance of these resulting features can differ as can be seen in Figure \ref{fig:features_distance_corr}. As a result, this feature selection method does not give a clear cut way of selecting the features that can distinguish between our two classes. Therefore, we don't proceed to use this method with the abrupt fault trajectories. 


\begin{figure}
    \centering
    \includegraphics[width=1\linewidth]{img/fall_detection/feature_selection/distance_corr_features.png}
    \caption{Resulting Features from Distance Correlation Feature Selection}
    \label{fig:features_distance_corr}
\end{figure}


\subsection{Sequential Feature Selection}
The sequential feature selection from sklearn is used to select the features. We got the best results by using forward featurer selection with a training escape time of 2. These features are displayed in Equation \ref{eq:features_f1_slow}. \eva{confirm how these features were found} However, when the training escape time is decreased to 0.9, the resulting features don't perform achieve as high of escape time with 0 false positive and negative rates. The results are displayed in Figure \ref{fig:features_seq_feat}

\begin{figure}
    \centering
    \includegraphics[width=1\linewidth]{img/fall_detection/feature_selection/seq_feat_selection_features.png}
    \caption{Resulting Features from Sequential Feature Selection}
    \label{fig:features_seq_feat}
\end{figure}

Even though sequential feature selection is able to find features with a higher escape time that still achieve 0 false positive and negative rate, it is time consuming. 





\section{Deep Learning}

Given the fact that feature selection is important, sequential feature selection is time consuming, and our goal of applying fall detection to more complex robots such as digit, we propose using deep neural networks such as a 1D CNN. Deep learning is capable of abstracting features from raw data. Here we take a look at the performance of a  multilayer perceptron (MLP) and 1D CNN on the fall detection data of the planar 4 link robot. The MLP and 1D CNN models used are displayed in Figure \ref{fig:mlp_model} and Figure \ref{fig:cnn_model}. Given that the fall detection problem for the planar four link robot is not complex, the neural networks perform equivalently or better than the SVM. These results are displayed in Figure   \ref{fig:neural_net_deep}

\begin{figure}
    \centering
    \includegraphics[width=1\linewidth]{img/fall_detection/feature_selection/mlp_model}
    \caption{Results from CNN and MLP}
    \label{fig:mlp_model}
\end{figure}

\begin{figure}
    \centering
    \includegraphics[width=1\linewidth]{img/fall_detection/feature_selection/cnn_model}
    \caption{Results from CNN and MLP}
    \label{fig:cnn_model}
\end{figure}


\begin{figure}
    \centering
    \includegraphics[width=1\linewidth]{img/fall_detection/feature_selection/cnn_mlp_performance.png}
    \caption{Results from CNN and MLP}
    \label{fig:neural_net_deep}
\end{figure}



%  \input{Sections/Fast_Slow_Faults}



\textbf{Acknowledgements:} This work was supported in part by NSF Award No.~1808051. In addition, M.E. Mungai was supported in part by an NSF Graduate Research Fellowship and a Rackham Merit Fellowship. The authors thank Prof. Maani Ghaffari for his helpful insights. In addition, M.E. Mungai thanks Grant Gibson, Yves Nazon, Isaack Karanja, and Vaibhav Singh for helpful advice. 

% REFERENCE
% \nocite{*}
\balance
\bibliographystyle{IEEEtran}
\bibliography{references.bib}
\end{document}
