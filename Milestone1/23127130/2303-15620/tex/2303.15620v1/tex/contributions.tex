\section{Contributions}
Given a \textbf{desired upper bound on the false positive and false negative rate}s, our objective is to develop a fall detection algorithm for the \textbf{task of standing} that can detect potential falls caused by either \textbf{slow acting or fast acting disturbances} while \textbf{maximizing the escape (lead) time}. 

To achieve this objective, it's imperative to determine the parameters that affect the escape time, false positive rate and false negative rate. The parameters that are identified and explored in this paper are the:

\begin{itemize}
    \item Training escape time (the time when the states of a trajectory that results in a fall can start to be labeled as falling) The larger the training escape time, the less the false positive rate and escape time
    \item The fall detection method to use (clustering or classification. It's expected that clustering will perform worse when it comes to slow acting faults)
    \item Feature engineering (the features to be used, feature preprocessing,and whether or not it's better to train the detection method on each of the faults individually or together). Poor feature engineering can lead to high false positive and false negative rates and low escape time
    \item The value of $N_{monitor}$. The higher the value, the lower the escape time and false positive rate
    \item The amount of data in the window. If the window is not large enough, the escape time will decrease and the false positive and false negative rates will increase
    \item 
\end{itemize}


% This objective has different implications for the two parts that make up fall detection algorithms. 
% \begin{itemize}
%     \item \textbf{Feature Engineering} : Find the minimal set of features that can accurately differentiate between safe and falling/fallen states while maximizing the escape time
%     \item \textbf{Fall Detection algorithm}: Detect both fast and slow acting faults with sufficient escape time given a specific set of features and an upper bound on the false positive and false negative rate. 
% \end{itemize}
