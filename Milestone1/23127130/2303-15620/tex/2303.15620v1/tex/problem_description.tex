\subsection{Challenges}






As obtaining an accurate physics-based model can be time-consuming and requires explicit domain knowledge, we focus on data-driven approaches. \eva{look over this again. Need better argument for data-driven vs physics-driven}


 
 % As a result, the algorithms used for fall detection are similar to the algorithms used for anomaly detection and pattern recognition. These algorithms can either be model based or data based. Model based algorithms are good at predicting specific faults, however, they require explicit domain knowledge and can suffer from inaccuracies in the model. Data based algorithms detect faults purely from data and don't require prior knowledge of the robot. However, they are limited by the limited amount of fault data. In this paper, we'll focus on data based algorithms. Data based algorithms can be divided into several categories including classification, feature extraction, threshold-based, and clustering \eva{look into nearest-neigbor and relative density, which is how these algorithms were describe at the beginning of the paper}.  Data based algorithms can also be categorized as either shallow or deep, and supervised, semi-supervised or unsupervised. 



 
 % The overall objective of fall detection is a reliable early detection of all faults for both open and closed loop systems. Reliability in literature is determined using a combination of the following terms:  false positive rate, false negative rate, f1 score, precision, recall or sensitivity, accuracy, and area under the curve.
 
 % such as   Escape time or lead time which is defined as the difference in time from where the fault is detected to when the robot falls and the execution time of the desired reflex algorithm to determine the maximum amount of time alloted for fall detection. Given this, the larger







 % The objective is to reliably predict potential falls of bipedal robots with sufficient time  to deploy a recovery strategy. Bipedal robots are chosen because their smaller support polygon causes them to be inherently less stable in comparison to other multi-legged robots. To simplify the fall detection problem while providing a way to scale up to more complex dynamic motions such as walking, the task of standing is chosen.   

% Fall detection algorithms, in general, are a part of the robot's control system. Figure \ref{fig:fallDetector_controlSystem} depicts the fall detection algorithm's role in the robot's control system.  


% \begin{figure}[!h]
%     \centering
%     \includegraphics[width=0.25\textwidth]{Media/ProblemDescription/fallDetector_ControlSystemInteraction.jpg}
%     \captionsetup{font=scriptsize}
%     \caption{Interaction between the fall detector and robot's control system. The recovery strategy is embedded in the recovery controller. 
%     }
%     \label{fig:fallDetector_controlSystem}
% \end{figure}





