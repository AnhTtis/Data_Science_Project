\section{Conclusion}
The objective of this paper was to design a fall detection algorithm for bipedal robots that is capable of detecting both incipient and abrupt faults while maximizing the lead time and meeting the desired false positive and negative rates. To meet the desired upper bound on the false positive and negative rates, we proposed using training lead time, a subset of lead time, to label the windows in a trajectory. We successfully implemented a nearest-neighbor fall detection algorithm, and analyzed and compared its performance to an SVM classification-based algorithm. Using false positive and negative rate, and average lead time as metrics, we found that the nearest-neighbor algorithm's performance is comparable to the SVM when trained on abrupt and incipient faults separately. However, it detects falls on average 0.15s (results to 5 data points given our sampling time) slower than the SVM classifier when the faults are trained together. Given that the nearest-neighbor algorithm still has average lead time of 0.5s, we conclude that if a sufficient amount of faulty data is not available, the nearest-neighbor algorithm can be used to detect abrupt and incipient faults simultaneously if there is not a sufficient amount of data.

Even though the SVM classification algorithm outperforms the nearest-neighbor algorithm, its leading time when trained on both faults together is slightly lower than its average lead time from both faults separately. As a result, we investigate the use of a multi-class classification algorithm to reduce this difference. We find, that using the same features with the multi-class classification algorithm increases the average lead time slightly. We briefly investigate using the multi-class classification algorithm with different features for the incipient and abrupt faults, and conclude that the multi-class classification algorithm shows promising results. However, more investigation is needed in feature selection and reduction in the delay time of the fault identifier to truly assess the advantage of using a multi-class classification algorithm.


%  To simplify the problem while providing for a way to scale up to more complex motions and robots, we focused on the task of standing with a planar four link robot. We propose a fall detection algorithm that 

% In this paper we design a fall detection algorithm that is capable of detecting both incipient and abrupt faults in a planar four link bipedal robots. The fall detection algorithm consists of 3 components: a fault identifier, an abrupt fault detector and an incipient fault detector. 


% when given a desired upper bound on the fpr and fnr rates would maximize the F\_LT. We   




% An  To determine whether to 


% The objective of this paper was to design a fall detection algorithm for bipedal robots that can maximize the lead time   