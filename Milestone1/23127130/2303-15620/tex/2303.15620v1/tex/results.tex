\subsection{Results of Proposed Multi-Class Classification Algorithm Across All Folds}

We train and evaluate the multi-class classification algorithm using the same parameters and metrics as described in Section \ref{sec:fault_comparison}. On average, across all folds, when trained using the features in \eqref{eq:features_main}, the multi-class classification algorithm achieves 0.06s and 0.05s additional average lead time across all folds for the nearest-neighbor and classification algorithms, respectively. This results in an average lead time difference of 0.13s and 0.03s across all folds for the neearest-neighbor and classification algorithm in comparison to their average when trained with the incipient and abrupt faults separately. Note that the SVM fault identifier has a delay of 0.07s or 3 data points, in detecting abrupt faults across all folds. The results are displayed in Table \ref{tab:multi_class_cluster}  and \ref{tab:multi_class_class}.

Even though the multi-class classification algorithm achieved similar results to the binary classification algorithm, an advantage over binary classification is that different features can be used for each detectors. Feature selection algorithms such as sequential feature selection can be used to determine the optimal features. For instance, using the features shown in Table \ref{tab:f1_features} derived from Sklearn's sequential forward feature selection \cite{ref:scikit-learn} results in an average lead time increase of 0.1s over training a binary classification with \eqref{eq:features_main}. However, to truly take advantage of the multi-class classification algorithm more investigation into optimal feature selection is needed to determine whether the additional average lead time gained can overcome the fault identifier delay.

\begin{table}
        \centering
        \caption{ Features derived from Sklearn's sequential forward feature selection\cite{ref:scikit-learn}}
        % \begin{tabular}{|p{0.2\linewidth}|p{0.75cm}|p{0.75cm}|p{0.5cm}|p{0.5cm}|p{0.5cm}|p{0.5cm}|p{0.5cm}|  }
        \tabulinesep=0.5mm
           \begin{tabu}{|c|c| }
    \hline
    \makecell{Incipient Fault Features}  &\makecell{Abrupt Fault\\Features} \\                                             \hline
                  $\begin{bmatrix}
        \text{\rm knee~angle} \\
        \text{\rm hip~angle} \\
        \text{\rm vel~hip~angle} \\
        (L_{cop} + L_{com})* sgn(p_{com}^x- p_{f_{mid}}^x) 
    \end{bmatrix}$ &            $\begin{bmatrix}
        p_{com}^x\\
        v_{com}^x\\
        p_{com}^x - p_{heel}^x\\
        \text{\rm foot~x} \\
        \text{\rm vel~foot~z} \\
        \text{\rm vel~hip~angle} 
    \end{bmatrix} $\\
    \hline                                               
    \end{tabu}
        \label{tab:f1_features}
    \end{table}


% To determine whether or not to use the nearest-neighbor or classification algorithm, we evaluate the performance of our proposed algorithm on the test data across all 5 folds with both the nearest-neighbor and classification detectors. We use the same evaluation metrics, and values for $N_{monitor}$  and $N_{window}$ as described in Section \ref{sec:fault_comparison}. As the fault identifier is able to identify abrupt faults with no delays and 100\% accuracy, and for clarity, we report the results for the abrupt and incipient faults individually in Tables \ref{tab:test_abrupt} and \ref{tab:test_incipient} respectively. Both the nearest-neighbor and classification algorithm achieve the desired fpr and fnr with an average of 0.25s average F\_LT across all folds for the abrupt fault. The maximum average F\_LT that can be achieved for the abrupt fault if the detection algorithms are able to detect the fault as soon as the force is introduced is 0.33s. As a result, both algorithms have an average delay of 0.08s across all folds.

% For the incipient fault, the classifier achieves an additional 0.06s for the average F\_LT across all folds, and a maximum  and minimum difference of 0.11s and 0.02s respectively for the individual folds. As a result, given our desire to maximize the average F\_LT and have 0 fpr and fnr, the SVM detector outperforms the nearest-neighbor detector for the incipient fault. 

% As the maximum average F\_LT that can be achieved for the incipient fault if the fault is classified at the beginning of the trajectory is 0.76s across all folds, the classifier and nearest neighbor algorithm have a delay of 0.44s and 0.5s respectively. This delay can be attributed to feature selection and the difficulty of identifying faulty incipient data points in a closed loop system while maintaining fpr of 0. For instance, when the safe trajectories with foot rotation composed only 7\% of the trajectories, the average training F\_LT obtained for the first fold by the SVM classifier was 0.78s. Additionally, when the set of features defined in Equation \eqref{eq:features_f1_slow} are used with the classification algorithm, the resulting average F\_LT for the first fold increases to 0.45s with 0 fpr and fnr but 0.02 fpr for the training data. 

% Despite the delay, the resulting F\_LT of both algorithms are acceptable as it doesn't account for the additional time of the robot falling and is greater than the 0.2s it takes to run reflex algorithms such those used by \cite{ref:hohn2009probabilistic} and \cite{ref:wu2021falling}. Therefore, the nearest-neighbor algorithm is still a viable option for incipient fault detection. 


\begin{table}
        \centering
        \caption{ A comparison of the maximum average lead time achieved by the binary nearest-neighbor  algorithm and the multi-class classification algorithm with nearest-neighbor fault detectors}
        % \begin{tabular}{|p{0.2\linewidth}|p{0.75cm}|p{0.75cm}|p{0.5cm}|p{0.5cm}|p{0.5cm}|p{0.5cm}|p{0.5cm}|  }
        \tabulinesep=0.5mm
           \begin{tabu}{|c|c|c| }
    \hline
            Fold                                                       &\makecell{Multi-class Classification\\Average Lead Time} 
                  &\makecell{Binary Classification\\Average Lead Time} \\                                             \hline
        1&  0.56 & 0.49  \\
        \hline
        2    & 0.52 & 0.49\\
        \hline
        3    &  0.57 & 0.51 \\    
        \hline
        4    &  0.56 & 0.51  \\    
        \hline
        5   &  0.57 & 0.51 \\ 
        \hline
        Average  &0.56   & 0.5 \\       
    \hline                                               
    \end{tabu}
        \label{tab:multi_class_cluster}
    \end{table}



\begin{table}
        \centering
        \caption{ A comparison of the maximum average lead time achieved by the binary SVM classifier and the multi-class classification algorithm with SVM fault detectors}
        % \begin{tabular}{|p{0.2\linewidth}|p{0.75cm}|p{0.75cm}|p{0.5cm}|p{0.5cm}|p{0.5cm}|p{0.5cm}|p{0.5cm}|  }
        \tabulinesep=0.5mm
           \begin{tabu}{|c|c|c|c|  }
    \hline
           Fold                                                       &\makecell{Multi-class Classification\\Average Lead Time} 
                  &\makecell{Binary Classification\\Average Lead Time} \\                                             \hline
        1&  0.7 & 0.65  \\
        \hline
        2    & 0.73 & 0.66  \\
        \hline
        3    &  0.70 & 0.66\\    
        \hline
        4    &  0.70 & 0.65  \\    
        \hline
        5   &  0.67 & 0.63  \\ 
        \hline
        Average  &0.7   & 0.65 \\       
    \hline                                               
    \end{tabu}
        \label{tab:multi_class_class}
    \end{table}


% \begin{center}
% \begin{table}
%         \centering
%         \caption{ The maximum average F\_LT yielding a \fpr and \fnr of 0 that can be achieved by the clustering algorithm trained and evaluated on the testing data with (1) just the abrupt fault, (2) just the incipient fault, and (3) both faults together. $N_{monitor}$ =1 and $N_{window}$ =25. The maximum average F\_LT that can be achieved across all folds for the validation data is 0.34s, 0.8s and 0.7s for the abrupt, incipient, and both faults respectively}
%         % \begin{tabular}{|p{0.2\linewidth}|p{0.75cm}|p{0.75cm}|p{0.5cm}|p{0.5cm}|p{0.5cm}|p{0.5cm}|p{0.5cm}|  }
%         \tabulinesep=0.5mm
%            \begin{tabu}{|c|c|c|c|  }
%     \hline
%            Fold                                                       &\makecell{Multi\\Classification\\Average\\Escape\\Time} 
%                   &\makecell{Binary\\Classification\\Average\\Escape\\Time} &\makecell{Multi\\Classification\\Varying\\Features\\Average\\Escape\\Time}  \\                                             \hline
%         1&  0.7 & 0.65 & 0.77 \\
%         \hline
%         2    & 0.73 & 0.66 & 0.81 \\
%         \hline
%         3    &  0.70 & 0.66 & 0.73\\    
%         \hline
%         4    &  0.70 & 0.65 &0.72 \\    
%         \hline
%         5   &  0.67 & 0.63 &0.73 \\ 
%         \hline
%         Average  &0.7   & 0.65 & 0.75 \\       
%     \hline                                               
%     \end{tabu}
%         \label{tab:multi_class_class}
%     \end{table}
% \end{center}


% \begin{center}
% \begin{table*}
%         \centering
%         \caption{ Resuult of classification using different features for the incipient and abrupt fault}
%         % \begin{tabular}{|p{0.2\linewidth}|p{0.75cm}|p{0.75cm}|p{0.5cm}|p{0.5cm}|p{0.5cm}|p{0.5cm}|p{0.5cm}|  }
%         \tabulinesep=0.5mm
%            \begin{tabu}{|c|c|c|c|  }
%     \hline
%            Fold                                                       &\makecell{Abrupt \\Features\\Average\\Escape\\Time} 
%                   &\makecell{Incipient\\Features\\Average\\Escape\\Time}  &\makecell{Both\\Features\\Average\\Escape\\Time}  \\                                             \hline
%         1 & & 0.66 & 0.77 \\
%         \hline
%         2  &  & 0.69 & 0.81 \\
%         \hline
%         3  &  &  0.68 & 0.73 \\    
%         \hline
%         4 &   &  0.72 & 0.72  \\    
%         \hline
%         5 &  &  0.72 & 0.73  \\ 
%         \hline
%         Average & &0.69   & 0.75 \\       
%     \hline                                               
%     \end{tabu}
%         \label{tab:fault_comp_clustering}
%     \end{table*}
% \end{center}



% %Therefore, as long as the right features are used both algorithms can be used to fall detection of incipient and abrupt faults.

%  \begin{center}
%     \begin{table}
%         \centering
%         \caption{ The maximum average F\_LT yielding a fpr and fnr of 0 that can be achieved by the classification and clustering algorithm trained and evaluated on the testing data with just the abrupt fault. $N_{monitor}$ =1 , $N_{window}$ =25, and a weight of 2 is used for normal data in the classification algorithm except for the 4th fold. The maximum average F\_LT that can be achieved across all folds for the testing data is 0.33s.}
%         % \begin{tabular}{|p{0.2\linewidth}|p{0.75cm}|p{0.75cm}|p{0.5cm}|p{0.5cm}|p{0.5cm}|p{0.5cm}|p{0.5cm}|  }
%         \tabulinesep=0.5mm
%            \begin{tabu}{|c|c|c|c|c|  }
%      \hline\hline
%      & \multicolumn{2}{c|}{\makecell{Clustering}} & \multicolumn{2}{c|}{\makecell{Classification}}\\
%     \hline
%            Fold                                         &\makecell{Training\\Escape\\Time}        &\makecell{Average\\Escape\\Time}  &\makecell{Training\\Escape\\Time}        &\makecell{Average\\Escape\\Time}\\
%            \hline
%         1   & 0.6 & 0.25 & 0.6 & 0.3  \\
%         \hline
%         2   & 0.6 & 0.27 & ~0.9 & 0.25 \\
%         \hline
%         3   & 0.6 & 0.27 & ~0.5 & 0.29\\    
%         \hline
%         4   & 0.5 & 0.24 & ~2.0 & 0.22  \\    
%         \hline
%         5   & 0.5 & 0.23 & ~0.5 & 0.2  \\ 
%         \hline
%         Average   &  & 0.25 &  & 0.25\\       
%     \hline                                               
%     \end{tabu}
%         \label{tab:test_abrupt}
%     \end{table}
% \end{center}

%  \begin{center}
%     \begin{table}
%         \centering
%         \caption{ The maximum average F\_LT yielding a fpr and fnr of 0 that can be achieved by the classification and clustering algorithm trained and evaluated on the testing data with just the incipient fault. $N_{monitor}$ =1 , $N_{window}$ =25, and a weight of 2, 5 and 6 is used for normal data in the classification algorithm for folds 1-2, 3-4 and 5 respectively. The maximum average F\_LT that can be achieved across all folds for the testing data is 0.76s.}
%         % \begin{tabular}{|p{0.2\linewidth}|p{0.75cm}|p{0.75cm}|p{0.5cm}|p{0.5cm}|p{0.5cm}|p{0.5cm}|p{0.5cm}|  }
%         \tabulinesep=0.5mm
%            \begin{tabu}{|c|c|c|c|c|  }
%      \hline\hline
%      & \multicolumn{2}{c|}{\makecell{Clustering}} & \multicolumn{2}{c|}{\makecell{Classification}}\\
%     \hline
%            Fold                                         &\makecell{Training\\Escape\\Time}        &\makecell{Average\\Escape\\Time}  &\makecell{Training\\Escape\\Time}        &\makecell{Average\\Escape\\Time}\\
%            \hline
%         1   & 0.3 & 0.24 & 2.0 & 0.32  \\
%         \hline
%         2   & 0.3 & 0.22 & 2.0 & 0.33 \\
%         \hline
%         3   & 0.3 & 0.27 & ~0.6 & 0.32\\    
%         \hline
%         4   & 0.3 & 0.26 & ~2.0 & 0.31  \\    
%         \hline
%         5   & 0.3 & 0.29 & 2.0 & 0.31  \\ 
%         \hline
%         Average   &  & 0.26 &  & 0.32\\       
%     \hline                                               
%     \end{tabu}
%         \label{tab:test_incipient}
%     \end{table}
% \end{center}




