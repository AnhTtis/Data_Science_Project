\subsection{Data Pre-processing}\label{sec:data_preparation}
The features are selected as
% \begin{equation}
% \begin{bmatrix}
%     p_{com}^x\\
%     v_{com}\\
%     p_{com}^x - p_{cop}^x\\
%     p_{f\_mid}^z - p_{com}^z
% \end{bmatrix}
%     \label{eq:features_main}
% \end{equation}
\begin{equation}
\begin{bmatrix}
    L_{cop} - L_{com}\\
    p_{com}^x\\
   v_{com}^x\\
   (p_{toe} - p_{com})^x\\
    (p_{heel} - p_{toe})^{xz} \\
    (L_{cop} + L_{com})* sgn(p_{com}^x- p_{f_{mid}}^x) 
\end{bmatrix}
    \label{eq:features_main}
\end{equation}
where $p_{com}$ and $v_{com}$ are the center of mass (com) position and velocity,  $p_{toe}$, $p_{heel}$,  and $ p_{f_{mid}}$ are the position of the toe, heel and middle of the foot, and $L_{cop}$ is the angular momentum about the contact point\footnote{The contact point is set to the rotation point (toe or heel) when the foot rotates, and the center of pressure otherwise.}. These features are chosen based on their correlation with the lead time and other features commonly used in literature. The distance correlation coefficient is used to evaluate the correlations as it is able to capture nonlinear relationships \cite{ref:szekely2014partial}.

The features are split into training (60\%), validation (20\%), and testing (20\%) sets using  scikit-learn's \cite{ref:scikit-learn} stratified train\_test\_split method and k-folds methods with the number of folds set to 5. The stratified methods are chosen because they ensure that each of the splits has the same distribution of normal and faulty data. Scikit's min-max scaler is used to scale the data to a range of $\{0,1\}$. To ensure that only transient data is kept for training, only the first 6s of trajectories that are deemed as normal are kept.  



% 