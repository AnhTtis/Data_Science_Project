% See the \addtolength command later in the file to balance the column lengths
% on the last page of the document
\usepackage{cite}
\usepackage{amsmath,amssymb,amsfonts}
\usepackage{algorithmic}
\usepackage{graphicx}
\usepackage{textcomp}




\usepackage[utf8]{inputenc}
\usepackage[T1]{fontenc}
\usepackage{mathptmx} % assumes new font selection scheme installed
\usepackage{mathptmx} % assumes new font selection scheme installed
\usepackage{amsmath} % assumes amsmath package installed
\usepackage{amssymb}  % assumes amsmath package installed
\usepackage{graphicx} 
\usepackage{grffile}
\usepackage[hypcap=true]{caption}
\usepackage{array}
\usepackage{mathtools}
\usepackage{commath}
\usepackage{colortbl}
\usepackage{enumerate}
\usepackage{hyperref}
\usepackage{cleveref}
% \usepackage{appendix}
\usepackage{subfig}
%\usepackage{subfloat}
\usepackage{float}
\usepackage{tabularx}
%\usepackage{subcaption}
\usepackage{makecell}
\usepackage{multirow}
\usepackage{placeins}
\usepackage{hyperref}
\usepackage{tabu}

\usepackage{nomencl}
\makenomenclature

% The following packages can be found on http:\\www.ctan.org
%\usepackage{graphics} % for pdf, bitmapped graphics files
%\usepackage{epsfig} % for postscript graphics files


% \usepackage[compress]{cite}
%\usepackage{url}
%\usepackage{graphicx,xcolor}
%\usepackage{subcaption}
\usepackage{xfrac}
\usepackage{multicol}
%\usepackage{graphicx}
\usepackage[dvipsnames]{xcolor}
\usepackage{soul}






\graphicspath{
    {template/}
}

\newcommand{\real}{\mathbb{R}} 

\DeclareMathOperator*{\argmax}{arg\,max}
\DeclareMathOperator*{\argmin}{arg\,min}


% \newcommand{\jwg}[1]{[{\textbf{\textcolor{red}{JWG: #1}}}]}
% \newcommand{\mem}[1]{[{\textbf{\textcolor{blue}{MEM: #1}}}]}
% \newcommand{\mem}[1]{[{\textbf{\textcolor{blue}{MEM: #1}}}]}

\newcommand{\jwg}[1]{\textcolor{red}{{[\textbf{JWG: #1}]}}}
\newcommand{\mem}[1]{\textcolor{blue}{{[\textbf{MEM: #1}]}}}
\newcommand{\bh}[1]{\textcolor{note}{{[\textbf{Bruce: #1}]}}}
\definecolor{note}{rgb}{0.3,0.7,0.25}

\newcommand{\fnr}{\rm FNR}
\newcommand{\fpr}{\rm FPR}

\newcommand{\sts}{chair-to-stand}
\newcommand{\Sts}{Chair-to-stand}
\newcommand{\StS}{Chair-to-Stand}
\newcommand{\stc}{chair-to-crouch-to-stand}
\newcommand{\Stc}{Chair-to-crouch-to-stand}
\newcommand{\StC}{Chair-to-Crouch-to-Stand}


\newcounter{countitems}
\newcounter{nextitemizecount}
\newcommand{\setupcountitems}{%
  \stepcounter{nextitemizecount}%
  \setcounter{countitems}{0}%
  \preto\item{\stepcounter{countitems}}%
}
\makeatletter
\newcommand{\computecountitems}{%
  \edef\@currentlabel{\number\c@countitems}%
  \label{countitems@\number\numexpr\value{nextitemizecount}-1\relax}%
}
\newcommand{\nextitemizecount}{%
  \getrefnumber{countitems@\number\c@nextitemizecount}%
}
\newcommand{\previtemizecount}{%
  \getrefnumber{countitems@\number\numexpr\value{nextitemizecount}-1\relax}%
}
\makeatother    
\newenvironment{AutoMultiColItemize}{%
\ifnumcomp{\nextitemizecount}{>}{3}{\begin{multicols}{2}}{}%
\setupcountitems\begin{itemize}}%
{\end{itemize}%
\unskip\computecountitems\ifnumcomp{\previtemizecount}{>}{3}{\end{multicols}}{}}


\newcommand{\add}[1]{\textcolor{OliveGreen}{\textbf{#1}}}
\newcommand{\mdfy}[2]{\textcolor{Gray}{\st{#1}}\textcolor{Orange}{\textbf{#2}}}
\newcommand{\dels}[1]{\textcolor{Gray}{\st{\textbf{#1}}}}
\newcommand{\dlt}[1]{\textcolor{Gray}{#1}}

\DeclareRobustCommand{\hsout}[1]{\texorpdfstring{\st{#1}}{#1}} %https://tex.stackexchange.com/questions/22410/strikethrough-in-section-title

\usepackage[makeroom]{cancel}

\usepackage{stfloats}
\raggedbottom