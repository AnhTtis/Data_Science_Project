% Please add the following required packages to your document preamble:
% \usepackage{graphicx}
\newcommand\RotText[1]{\rotatebox{90}{\parbox{2cm}{\centering#1}}}

\begin{table}[t]
\resizebox{\linewidth}{!}{%
\begin{tabular}{c|cccccccl}
\begin{tabular}[c]{@{}c@{}}\textbf{Methods}\end{tabular} & 
\begin{tabular}[c]{@{}c@{}}{\textbf{A}}\end{tabular} &
\begin{tabular}[c]{@{}c@{}}{\textbf{B}}\end{tabular} 
&\begin{tabular}[c]{@{}c@{}}{\textbf{C}}\end{tabular} & \begin{tabular}[c]{@{}c@{}}{\textbf{D}}\end{tabular} & \begin{tabular}[c]{@{}c@{}}{\textbf{E}}\end{tabular} & \begin{tabular}[c]{@{}c@{}}{\textbf{F}}\end{tabular}&\begin{tabular}[c]{@{}c@{}}{\textbf{G}}\end{tabular} & \textbf{Task} \\
\toprule
NeRF~\cite{mildenhall2020nerf}  & \xmark  & \cmark & \xmark  & \xmark & \cmark & \cmark   & \xmark &
\multirow{6}{*}{\rotatebox{90}{\parbox{2cm}{\centering \textbf{\hspace{3pt} Img-Based Synthesis }}}} \\

Eikonal~\cite{bemana2022eikonal}   & \cmark & \cmark & \xmark & \xmark & \cmark  & \cmark  & \xmark  &  \\

IDR~\cite{yariv2020multiview}  & \xmark  & \cmark   & \xmark   & \cmark   & \cmark    & \cmark  & \xmark  &  \\
 \small{PhySG, ...}~\cite{zhang2021physg, zhang2022invrender}   & \xmark  & \cmark  & \cmark   & \cmark   & \cmark  & \xmark  & \cmark  &  \\

\cmidrule(lr){1-8}
\textbf{NEMTO (Ours)}   & \cmark    & \cmark   & \cmark  & \cmark  & \cmark      & \cmark  & \xmark &  
\multirow{5}{*}{\rotatebox{90}{\parbox{1.2cm}{\centering \textbf{\hspace{0pt} Geo. \\ Est.}}}} \\

\midrule
\cite{wu2018full, lyu2020differentiable, xu2022hybrid} & \cmark & \xmark & \xmark  & \cmark  & \xmark     & \xmark  & \xmark  &  \\

TLG~\cite{li2020through}  & \cmark  & \xmark & \xmark  & \cmark  & \cmark & \xmark   & \xmark  &\\
\bottomrule
\end{tabular}
}
\vspace{-3pt}
\caption{\textbf{Comparison of relevent methods. } The first group focuses on image-based novel view synthesis and relighting, while the second estimates transparent object geometry. \textbf{(A)} can model light refraction for non-opaque objects,  \textbf{(B)} allows direct novel view synthesis,  \textbf{(C)} allows direct scene relighting, \textbf{(D)} can model object surface,  \textbf{(E)} does not require complex setup for image capture, i.e no patterned backlight, turntables, etc.,  \textbf{(F)} can model transparent materials with unknown IOR, \textbf{(G)} allows estimation of illumination during training. \vspace{-10pt}} 
\vspace{-0.2cm}
\label{tab:baseline}
\end{table}

