\section{Limitations and Conclusion}
\label{sec:conc}
\medskip
\noindent\textbf{Limitations.} There are a few limitations to our work. First, although NEMTO does not assume homogeneous IOR of transparent media, there is no explicit supervision for heterogeneous transparent objects. With the introduction of appropriate loss functions such as from Eikonal Rendering~\cite{ihrke2007eikonal}, we believe NEMTO can be extended for a wider variety of transparent media. Secondly, our model requires a preprocessing of image data for environment illumination and object masks, as we cannot jointly optimize illumination along with geometry and object appearance for transparent objects. Lastly, NEMTO focuses on unpolarized transparent objects and does not provide experiments on polarized transparent media~\cite{cui2017polarimetric, 5539828, 1467539}.  

\medskip
\noindent\textbf{Conclusion.} We have presented NEMTO, an end-to-end pipeline for novel view and relighting synthesis of transparent objects with complex geometry. Our method jointly optimizes geometry and highly illumination-dependent object appearance and generates high-quality synthesis. 

\section{Acknowledgement}
This work was supported in part by the Swiss National Science Foundation via the Sinergia grant CRSII5-180359. The authors thank Ziyi Zhang for his technical support at the early stage of this work, and thank Yufan Ren, Ehsan Pajouheshgar, Martin Everaert, Bahar Aydemir, Deblina Bhattacharjee, Michele Vidulis, and Merlin Nimier-David for their time spent on proof-reading and kind suggestions during the paper writing.