\section{Introduction}
\label{sec:intro}

 Modeling transparent objects is important for VR/AR applications as the former are abundant in the real world. Unlike opaque objects with close-to-zero light transmission, transparent objects allow light to pass through. Such refraction and reflection create complex light paths and give transparent objects highly environment-dependent appearances. Consequently, the appearance and geometry of transparent objects are much more entangled than those of opaque objects~\cite{pharr2016physically}. A slight error in object geometry can lead to a global change in appearance~\cite{xu2022hybrid}, as the light path for each ray may thus vary substantially. For these reasons, deriving material and object geometry from images of a transparent object is a highly ill-posed and challenging problem. 


\begin{figure}[t]
\begin{center}
   \includegraphics[width=1.1\linewidth]{img/teaser_png.png}
\end{center}
   \caption{Given as input multi-view images captured under natural illumination, NEMTO is capable of high-quality novel view synthesis and relighting through optimizing an end-to-end neural representation for a transparent object. NEMTO disentangles geometry and illumination-dependent appearance, which previous neural rendering methods, such as PhySG~\cite{zhang2021physg}, cannot. }
\label{fig:teaser}
\end{figure}

Existing work for modeling transparent objects can be classified into two categories. One assumes known indices of refraction (IOR) and reconstructs the complex geometry of transparent objects through either physical devices and structured backlights ~\cite{chuang2000environment, lyu2020differentiable, wetzstein2011refractive, wexler2002image, wu2018full, xu2022hybrid} or neural networks that model geometry with analytical refraction~\cite{li2020through}. The other~\cite{bemana2022eikonal} focuses on optimizing the refractive ray path in the scene without modeling the object surface geometry. However, neither approach is optimal for synthesizing novel views and relighting for transparent objects  with \textit{complex geometry}. In this work, we propose a new framework that combines recent advances in Neural Inverse Rendering~\cite{boss2021nerd, boss2021neuralpil, munkberg2022extracting, zhang2021physg, zhang2021nerfactor, zhang2022invrender} to overcome these limitations.


Traditionally, physically-based rendering follows Snell's Law to render transparent objects. However, object appearance highly depends on geometry estimation, and jointly optimizing both is highly ill-posed. Therefore, our key contribution is incorporating a physically-guided \textit{Ray Bending Network} (RBN) to disentangle object geometry and light refraction. RBN takes the learned geometry~\cite{yariv2020multiview} as prior, and models light refraction by mapping the incoming ray direction directly to the refracted ray direction exiting the object. Our method does not assume a homogeneous refractive index or a fixed number of bounces~\cite{bemana2022eikonal, pharr2016physically}, and models the object's surface with the zero-level set of the Signed Distance Function (SDF). NEMTO thus has the potential to handle a wider range of complex geometry and better adapt to various refractive media than existing transparent object modeling~\cite{bemana2022eikonal, li2020through}. Furthermore, our RBN can improve the estimated geometry by better disentangling it from the appearance of the object than other neural rendering methods~\cite{yariv2020multiview, zhang2021physg}. NEMTO thus makes it practical to model transparent objects in various scenarios, by working with unknown refractive indices and natural environment illumination. 
 Tab.~\ref{tab:baseline} lists the pros and cons of image-based models on novel view and relight synthesis, along with methods focusing on geometry estimation for transparent objects. We identify the first group as our baseline because the second cannot synthesize views without knowing the object IOR. Experiments show that NEMTO can synthesize higher quality novel views and relighting through our representation of transparent objects than all of our baseline methods. 






To summarize, our contributions are as follows: 
\begin{tight_itemize}
    \item We propose NEMTO, the first end-to-end method for novel view synthesis and scene relighting for \textit{transparent objects}, shown in Fig.~\ref{fig:teaser}. Our method can disentangle transparent object geometry and appearance. 
    \item We design a physically-guided Ray Bending Network (RBN) for predicting ray paths traversing through the transparent object. The network prediction has better error tolerance for the estimated geometry than analytically calculated refraction.
    \item NEMTO can easily be adapted to real-world transparent objects and achieve high-quality image-based synthesis. 
\end{tight_itemize}
\vspace{-0.4cm}

