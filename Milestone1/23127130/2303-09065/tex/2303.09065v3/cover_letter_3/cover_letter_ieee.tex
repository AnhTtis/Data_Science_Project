%%%%%%%%%%%%%%%%%%%%%%%%%%%%%%%%%%%%%%%%%
% Long Lined Cover Letter
% LaTeX Template
% Version 1.0 (1/6/13)
%
% This template has been downloaded from:
% http://www.LaTeXTemplates.com
%
% Original author:
% Matthew J. Miller
% http://www.matthewjmiller.net/howtos/customized-cover-letter-scripts/
%
% License:
% CC BY-NC-SA 3.0 (http://creativecommons.org/licenses/by-nc-sa/3.0/)
%
%%%%%%%%%%%%%%%%%%%%%%%%%%%%%%%%%%%%%%%%%

%----------------------------------------------------------------------------------------
%	PACKAGES AND OTHER DOCUMENT CONFIGURATIONS
%----------------------------------------------------------------------------------------

\documentclass[a4paper,10pt,stdletter,dateno,sigleft]{newlfm} % Extra options: 'sigleft' for a left-aligned signature, 'stdletternofrom' to remove the from address, 'letterpaper' for US letter paper - consult the newlfm class manual for more options
%\documentclass[journal]{IEEEtran}

\usepackage{charter} % Use the Charter font for the document text
%\usepackage[a4paper,left=2.5cm,right=2.5cm,top=3cm,bottom=3cm]{geometry}

%\newsavebox{\Luiuc}\sbox{\Luiuc}{\parbox[b]{1.75in}{\vspace{0.5in}
%\includegraphics[width=1.2\linewidth]{logo.png}}} % Company/institution logo at the top left of the page
%\makeletterhead{Uiuc}{\Lheader{\usebox{\Luiuc}}}

\newlfmP{sigsize=50pt} % Slightly decrease the height of the signature field
\newlfmP{addrfromphone} % Print a phone number under the sender's address
\newlfmP{addrfromemail} % Print an email address under the sender's address
\PhrPhone{Phone} % Customize the "Telephone" text
\PhrEmail{Email} % Customize the "E-mail" text

%\lthUiuc % Print the company/institution logo

%----------------------------------------------------------------------------------------
%	YOUR NAME AND CONTACT INFORMATION
%----------------------------------------------------------------------------------------

\namefrom{Haeyong Kang} % Name

\addrfrom{
\today\\[12pt] % Date
School of Electrical Engineering, \\ % Address
Statistical Learning and Signal Processing (SLSP) Lab,\\
KAIST, Room \#2106, N24, LG Hall,\\
Guseong-dong, Yuseong-gu, Daejeon, Rep. of Korea.
}

\phonefrom{(+82)-10-8339-7738} % Phone number
\emailfrom{ihaeyong@gmail.com} % Email address

%----------------------------------------------------------------------------------------
%	ADDRESSEE AND GREETING/CLOSING
%----------------------------------------------------------------------------------------

\greetto{Dear Editors,} % Greeting text
\closeline{Sincerely yours,} % Closing text

%\nameto{Mrs. Jane Smith} % Addressee of the letter above the to address

%\addrto{
%Recruitment Officer \\ % To address
%The Corporation \\
%123 Pleasant Lane \\
%City, State 12345
%}

%----------------------------------------------------------------------------------------

\begin{document}
\begin{newlfm}

%----------------------------------------------------------------------------------------
%	LETTER CONTENT
%----------------------------------------------------------------------------------------
We would choose this journal to submit our manuscript since our work deals with a deep probabilistic learning algorithm for cell classification with advanced signal processing. 

The classification algorithm is based on a deep probabilistic model referred to as the sum-product networks (SPNs). In the process of solving cell classification, especially, both a tree-structured SPNs (t-SPNs) and the maximum margin (MM)+L2-regularization learning are proposed to deal with the similarity of cell classes in appearance which causes higher misclassification rate. Furthermore, through the processing such as high-pass filtering (HPF) and Laplacian of Gaussian (LOG) filtering, we were able to extract the discriminative features for cell classification and to evaluate the effectiveness of t-SPNs + MM learning with filtering on both the HEp-2 and Feulgen stained benchmark datasets. 

The main contributions in our manuscript are,

1.  the t-SPNs designed for confusing cell classes. \\
2.  the maximum margin (MM) learning proposed for the SPNs as well as the t-SPNs. \\
3.  the maximum margin (MM) + L2-regularization (REG) learning. \\
4.  through the filtering, the discriminative feature extraction.\\

We believe that the topic of this manuscript is relevant for IEEE Selected Topics in Signal Processing Special Issue on Advanced Signal Processing in Microscopy and Cell Imaging and interests a broad range of reviewers. We are looking forward to receiving comments from the reviewers. 

Thank you for your consideration.

%----------------------------------------------------------------------------------------

\end{newlfm}
\end{document}