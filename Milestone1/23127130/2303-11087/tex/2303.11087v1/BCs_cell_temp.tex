

\section{Derivation of flux boundary condition for cell temperature}
\label{app:cell-flux-proof}
We follow a similar approach in the derivation of the flux boundary conditions for the cell temperature. The main difference between packing and cell temperature is that the cell temperature domain is discontinuous in contrast to the continuous packing domain. We first apply the averaging operator (equation~\eqref{eq:ave-op}) to the fine-scale cell temperature equations (equation~\eqref{eq:pore_goven_eqs}) to derive the averaged expression as 
\begin{align}
 &\phi^{\left(c\right)}\derive{ \left\langle T_{\epsilon}^{\left(c\right)} \right\rangle_Y}{t} = \left(\phi^{\left(c\right)}\varrho \bm{\cdot} \varsigma\right) \left\langle \nabla \bm{\cdot} \left(k^{\left(c\right)} \nabla T_{\epsilon}^{\left(c\right)}\right) \right\rangle_Y + \left(\phi^{\left(c\right)}\varrho \bm{\cdot} \mathcal{R}\right)\left\langle \Pi\left(T_{\epsilon}^{\left(c\right)}, \mathbf{x}\right) \right\rangle_Y,
\end{align}
Integrating over an arbitrary coupling volume $J$ that contains the hybrid coupling boundary $\Gamma^{\left(HC\right)}$ and replacing the fine-scale equations in $J_{out}$ with upscaled equations, one obtains

\begin{align}
 \int_J \phi^{\left(c\right)}\derive{ \left\langle T_{\epsilon}^{\left(c\right)} \right\rangle_Y}{t}  \ \dd\mathbf{y} &= \int_{J_{in}}\left(\phi^{\left(c\right)}\varrho \bm{\cdot} \varsigma\right) \left\langle \nabla \bm{\cdot} \left(k^{\left(c\right)} \nabla T_{\epsilon}^{\left(c\right)}\right) \right\rangle_Y \ \dd\mathbf{y} \\
 &+ \int_{J_{in}} \left(\phi^{\left(c\right)}\varrho \bm{\cdot} \mathcal{R}\right)\left\langle\Pi\left(T_{\epsilon}^{\left(c\right)}, \mathbf{x}\right) \right\rangle_Y \ \dd\mathbf{y} \nonumber   \\
&+ \int_{J_{out}} -\left(\U^{\left(c\right)} \bm{\cdot} \nabla_{\mathbf{x}} \langle T^{\left(c\right)} \rangle_{Y} - \V^{\left(c\right)} \bm{\cdot} \nabla_{\mathbf{x}} \langle T^{\left(p\right)} \rangle_{Y} - \nabla_{\mathbf{x}} \bm{\cdot} \left(\K^{\left(c\right)} \bm{\cdot} \nabla_{\mathbf{x}} \langle T^{\left(c\right)} \rangle_{Y}\right) \right)  \ \dd\mathbf{y} \nonumber \\ 
 &+ \int_{J_{out}} \boldsymbol{\mathbb{S}}^{\left(c\right)}\left(t, \mathbf{x}\right) \ \dd\mathbf{y},
\end{align}
where 
\begin{align}
    \boldsymbol{\mathbb{S}}^{\left(c\right)}\left(t, \mathbf{x}\right) = R_1^{\left(c\right)} \langle T^{\left(p\right)} \rangle_{Y} - R_2^{\left(c\right)} \langle T^{\left(c\right)} \rangle_{Y} + R_3^{\left(c\right)} q^{\left(pw\right)}\left(t, \mathbf{x}\right) + R_4^{\left(c\right)} \overline{\Pi}\left(\langle T^{\left(c\right)} \rangle_{Y}, \mathbf{x}\right).
\end{align}
Applying the spatial averaging theorem, the equations can be expressed as
\begin{align}
 \int_J \phi^{\left(c\right)}\derive{ \left\langle T_{\epsilon}^{\left(c\right)} \right\rangle_Y}{t}  \ \dd\mathbf{y} &= \int_{J_{in}}\left(\phi^{\left(c\right)}\varrho \bm{\cdot} \varsigma\right) \nabla \bm{\cdot} \left\langle \left(k^{\left(c\right)} \nabla T_{\epsilon}^{\left(c\right)}\right) \right\rangle_Y \ \dd\mathbf{y} \nonumber \\
 &+ \ddfrac{\phi^{\left(c\right)}}{\abs{\mathcal{V}}\abs{J_{in}}} \int_{J_{in}} \int_{\Gamma^{\left(pc\right)}} \left(\varrho \bm{\cdot} \varsigma\right) \left(k^{\left(c\right)} \nabla T_{\epsilon}^{\left(c\right)}\right) \bm{\cdot} \mathbf{n}^{\left(c\right)}_\epsilon \ \dd\mathbf{y} \nonumber \\
&+ \int_{J_{in}} \left(\phi^{\left(c\right)}\varrho \bm{\cdot} \mathcal{R}\right)\left\langle\Pi\left(T_{\epsilon}^{\left(c\right)}, \mathbf{x}\right) \right\rangle_Y \ \dd\mathbf{y} \nonumber   \\
&+ \int_{J_{out}} -\left(\U^{\left(c\right)} \bm{\cdot} \nabla_{\mathbf{x}} \langle T^{\left(c\right)} \rangle_{Y} - \V^{\left(c\right)} \bm{\cdot} \nabla_{\mathbf{x}} \langle T^{\left(p\right)} \rangle_{Y} - \nabla_{\mathbf{x}} \bm{\cdot} \left(\K^{\left(c\right)} \bm{\cdot} \nabla_{\mathbf{x}} \langle T^{\left(c\right)} \rangle_{Y}\right) \right)  \ \dd\mathbf{y} \nonumber \\ 
 &+ \int_{J_{out}} \boldsymbol{\mathbb{S}}^{\left(c\right)}\left(t, \mathbf{x}\right) \ \dd\mathbf{y}.
\end{align}
Since the cell domain is discontinous such that $T^c_\epsilon$ exists only for $x \in \mathcal{B}^c$, the divergence theorem cannot be applied. To overcome this, we consider another domain $\mathcal{B}^{c,*}$ such that 
\begin{align}
    &T^{\left(c,*\right)}_\epsilon =
    \begin{cases}
    T^{\left(c\right)}_\epsilon \text{ for } \mathbf{x} \in \mathcal{B}^{\left(c\right)}, \\
    0 \text{ for } \mathbf{x} \notin \mathcal{B}^{\left(c\right)},
    \end{cases}
\end{align}
where $T^{\left(c,*\right)}_\epsilon$ is the cell temperature in the modified domain $\mathcal{B}^{\left(c,*\right)}$. By replacing $\mathcal{B}^{\left(c\right)}$ with $\mathcal{B}^{\left(c,*\right)}$, the cell temperature changes from a discontinuous function to a function with jumps at the interfaces. Here, we introduce a divergence theorem for functions with jumps~\cite{Arnold1982-pq} as 
\begin{align}
\label{eq:jump-div-theom}
   & \int_V \div{\mathbb{F}} \ \dd\mathbf{x} = \int_{\partial{V}} \mathbb{F}\cdot \mathbf{n} \ \dd\mathbf{x} + \sum_{i,j} \int_{\Gamma^{ij}} (\mathbb{F}^{ij} - \mathbb{F}^{ji}) \bm{\cdot} \mathbf{n}^{ij} \ \dd\mathbf{x},
\end{align}
where $\mathbb{F}$ is an arbitrary flux and $\Gamma^{ij}$ represents the interface with jump. The modified divergence theorem (equation~\eqref{eq:jump-div-theom}) is reduced to the original divergence theorem if $\Gamma^{ij}$ does not exist. By applying the modified divergence theorem, we obtain
\begin{align}
\label{eq:cell-flux-rhs}
 \int_J \phi^{\left(c\right)}\derive{ \left\langle T_{\epsilon}^{\left(c\right)} \right\rangle_Y}{t}  \ \dd\mathbf{y} &= \int_{\Gamma_{in}}\left(\phi^{\left(c\right)}\varrho \bm{\cdot} \varsigma\right) \left\langle \left(k^{\left(c\right)} \nabla T_{\epsilon}^{\left(c\right)}\right) \right\rangle_Y  \bm{\cdot} \mathbf{n}_\epsilon^{\left(c\right)} \ \dd\mathbf{y} \nonumber \\
 &+ \left(\phi^{\left(c\right)}\varrho \bm{\cdot} \varsigma\right) \sum_{i,j} \int_{\Gamma^{ij}_{in}} k^{\left(c\right)}\left(\ensmean{\nabla T_{\epsilon}^{\left(c\right)}}^{ij}_Y - \ensmean{\nabla T_{\epsilon}^{\left(c\right)}}^{ji}_Y \right) \bm{\cdot}  \mathbf{n}^{ij} \ \dd\mathbf{y} \nonumber \\
 &+ \ddfrac{\phi^{\left(c\right)}}{\abs{\mathcal{V}}\abs{J_{in}}} \int_{J_{in}} \int_{\Gamma^{\left(pc\right)}} \left(\varrho \bm{\cdot} \varsigma\right) \left(k^{\left(c\right)} \nabla T_{\epsilon}^{\left(c\right)}\right) \bm{\cdot} \mathbf{n}^{\left(c\right)}_\epsilon \ \dd\mathbf{y} \nonumber \\
&+ \int_{J_{in}} \left(\phi^{\left(c\right)}\varrho \bm{\cdot} \mathcal{R}\right)\left\langle\Pi\left(T_{\epsilon}^{\left(c\right)}, \mathbf{x}\right) \right\rangle_Y \ \dd\mathbf{y} \nonumber   \\
 &+ \int_{\Gamma_{out}} \left(-\U^{\left(c\right)}\langle T^{\left(c\right)} \rangle_{Y} + \V^{\left(c\right)} \langle T^{\left(p\right)} \rangle_{Y} +  \left(\K^{\left(c\right)} \bm{\cdot} \nabla_{\mathbf{x}} \langle T^{\left(c\right)} \rangle_{Y}\right) \right) \bm{\cdot} \mathbf{n} \ \dd\mathbf{y} \nonumber \\
 &+ \int_{J_{out}} \langle T^{\left(c\right)} \rangle_{Y} \nabla_{\mathbf{x}} \bm{\cdot} \U^{\left(c\right)}  \ \dd\mathbf{y} - \int_{J_{out}} \langle T^{\left(p\right)} \rangle_{Y} \nabla_{\mathbf{x}} \bm{\cdot} \V^{\left(c\right)}  \ \dd\mathbf{y} + \int_{J_{out}} \boldsymbol{\mathbb{S}}^{c}\left(t, \mathbf{x}\right) \ \dd\mathbf{y} \nonumber \\
&+\int_{\Gamma^{\left(HC\right)}}\left(\phi^{\left(c\right)}\varrho \bm{\cdot} \varsigma\right) \left\langle \left(k^{\left(c\right)} \nabla T_{\epsilon}^{\left(c\right)}\right) \right\rangle_Y  \bm{\cdot} \mathbf{n}_\epsilon^{\left(c\right)} \ \dd\mathbf{y} \nonumber \\
 &+ \int_{\Gamma^{\left(HC\right)}}\left(-\U^{\left(c\right)}\langle T^{\left(c\right)} \rangle_{Y} + \V^{\left(c\right)} \langle T^{\left(p\right)} \rangle_{Y} +  \left(\K^{\left(c\right)} \bm{\cdot} \nabla_{\mathbf{x}} \langle T^{\left(c\right)} \rangle_{Y}\right) \right) \bm{\cdot} \mathbf{n} \ \dd\mathbf{y}.
\end{align}
% \begin{align}
%  \int_J \pdv{\ensmean{T_\epsilon^{c}}_Y}{t_\epsilon} \ \dd\mathbf{y} = &\int_{\Gamma_{in}} \left(k^{c} \left(\varrho \bm{\cdot} \varsigma\right) \ensmean{\grad{T_\epsilon^{c}}}_Y \right) \bm{\cdot}  \mathbf{n}^c_\epsilon \ \dd\mathbf{y} \nonumber\\  
%  &+ k^{c} \left(\varrho \bm{\cdot} \varsigma\right) \sum_{i,j} \int_{\Gamma^{ij}_{in}} \left(\ensmean{\grad{T_\epsilon^{c}}}^{ij}_Y - \ensmean{\grad{T_\epsilon^{c}}}^{ji}_Y \right) \bm{\cdot}  \mathbf{n}^{ij} \ \dd\mathbf{y} \nonumber \\
%  &+ \ddfrac{1}{\abs{\mathcal{V}}\abs{J_{in}}} \int_{J_{in}} \int_{\Gamma^{pc}} k^{c}\left(\varrho \bm{\cdot} \varsigma\right) \grad{T_\epsilon^{c}} \bm{\cdot} \mathbf{n}^c_\epsilon \ \dd\mathbf{y} + \int_{J_{in}} \ensmean{\left(\varrho \bm{\cdot} \mathcal{R}\right)\Pi\left(t_\epsilon, \mathbf{x}_\epsilon\right)}_Y \ \dd\mathbf{y} \nonumber\\ 
%  &- \int_{\Gamma_{out}}  \left(  \boldsymbol{\theta}^{cp} \ensmean{T^{p}}_Y  \right) \bm{\cdot} \mathbf{n}^c_\epsilon \ \dd\mathbf{y} + \int_{J_{out}} \boldsymbol{\mathbb{S}}^{c}\left(t, \mathbf{x}\right) \ \dd\mathbf{y} \nonumber \\
%  &+ \int_{\Gamma^{\left(HC\right)}} \left(k^{c} \left(\varrho \bm{\cdot} \varsigma\right) \ensmean{\grad{T_\epsilon^{c}}}_Y \right) \bm{\cdot}  \mathbf{n}^c_\epsilon \ \dd\mathbf{y} - \int_{\Gamma^{\left(HC\right)}}  \left(  \boldsymbol{\theta}^{cp} \ensmean{T^{p}}_Y  \right) \bm{\cdot} \mathbf{n}^c_\epsilon \ \dd\mathbf{y},   
% \end{align}
By applying the spatial average theorem, and modified divergence theorem to the time-derivative term in equation~\eqref{eq:cell-flux-rhs}, we obtain
\begin{align}
\label{eq:cell-flux-lhs}
 &\int_J \phi^{\left(c\right)}\derive{ \left\langle T_{\epsilon}^{\left(c\right)} \right\rangle_Y}{t}  \ \dd\mathbf{y} = \int_{\Gamma_{in}}\left(\phi^{\left(c\right)}\varrho \bm{\cdot} \varsigma\right) \left\langle \left(k^{\left(c\right)} \nabla T_{\epsilon}^{\left(c\right)}\right) \right\rangle_Y  \bm{\cdot} \mathbf{n}_\epsilon^{\left(c\right)} \ \dd\mathbf{y} \nonumber \\
 &+ \left(\phi^{\left(c\right)}\varrho \bm{\cdot} \varsigma\right) \sum_{i,j} \int_{\Gamma^{ij}_{in}} k^{\left(c\right)}\left(\ensmean{\nabla T_{\epsilon}^{\left(c\right)}}^{ij}_Y - \ensmean{\nabla T_{\epsilon}^{\left(c\right)}}^{ji}_Y \right) \bm{\cdot}  \mathbf{n}^{ij} \ \dd\mathbf{y} \nonumber \\
 &+ \ddfrac{\phi^{\left(c\right)}}{\abs{\mathcal{V}}\abs{J_{in}}} \int_{J_{in}} \int_{\Gamma^{\left(pc\right)}} \left(\varrho \bm{\cdot} \varsigma\right) \left(k^{\left(c\right)} \nabla T_{\epsilon}^{\left(c\right)}\right) \bm{\cdot} \mathbf{n}^{\left(c\right)}_\epsilon \ \dd\mathbf{y} \nonumber \\
&+ \int_{J_{in}} \left(\phi^{\left(c\right)}\varrho \bm{\cdot} \mathcal{R}\right)\left\langle\Pi\left(T_{\epsilon}^{\left(c\right)}, \mathbf{x}\right) \right\rangle_Y \ \dd\mathbf{y} \nonumber   \\
&+\int_{\Gamma_{out}}\left(\phi^{\left(c\right)}\varrho \bm{\cdot} \varsigma\right) \left\langle \left(k^{\left(c\right)} \nabla T_{\epsilon}^{\left(c\right)}\right) \right\rangle_Y  \bm{\cdot} \mathbf{n}_\epsilon^{\left(c\right)} \ \dd\mathbf{y} \nonumber \\
 &+ \left(\phi^{\left(c\right)}\varrho \bm{\cdot} \varsigma\right) \sum_{i,j} \int_{\Gamma^{ij}_{out}} k^{\left(c\right)}\left(\ensmean{\nabla T_{\epsilon}^{\left(c\right)}}^{ij}_Y - \ensmean{\nabla T_{\epsilon}^{\left(c\right)}}^{ji}_Y \right) \bm{\cdot}  \mathbf{n}^{ij} \ \dd\mathbf{y} \nonumber \\
 &+ \ddfrac{\phi^{\left(c\right)}}{\abs{\mathcal{V}}\abs{J_{out}}} \int_{J_{out}} \int_{\Gamma^{\left(pc\right)}} \left(\varrho \bm{\cdot} \varsigma\right) \left(k^{\left(c\right)} \nabla T_{\epsilon}^{\left(c\right)}\right) \bm{\cdot} \mathbf{n}^{\left(c\right)}_\epsilon \ \dd\mathbf{y} \nonumber \\
&+ \int_{J_{out}} \left(\phi^{\left(c\right)}\varrho \bm{\cdot} \mathcal{R}\right)\left\langle\Pi\left(T_{\epsilon}^{\left(c\right)}, \mathbf{x}\right) \right\rangle_Y \ \dd\mathbf{y},
\end{align}
By equating equation~\eqref{eq:cell-flux-rhs} and equation~\eqref{eq:cell-flux-lhs}, we obtain
\begin{align}
&\int_{\Gamma_{out}}\left(\phi^{\left(c\right)}\varrho \bm{\cdot} \varsigma\right) \left\langle \left(k^{\left(c\right)} \nabla T_{\epsilon}^{\left(c\right)}\right) \right\rangle_Y  \bm{\cdot} \mathbf{n}_\epsilon^{\left(c\right)} \ \dd\mathbf{y} \nonumber \\
 &+ \left(\phi^{\left(c\right)}\varrho \bm{\cdot} \varsigma\right) \sum_{i,j} \int_{\Gamma^{ij}_{out}} k^{\left(c\right)}\left(\ensmean{\nabla T_{\epsilon}^{\left(c\right)}}^{ij}_Y - \ensmean{\nabla T_{\epsilon}^{\left(c\right)}}^{ji}_Y \right) \bm{\cdot}  \mathbf{n}^{ij} \ \dd\mathbf{y} \nonumber \\
 &+ \ddfrac{\phi^{\left(c\right)}}{\abs{\mathcal{V}}\abs{J_{out}}} \int_{J_{out}} \int_{\Gamma^{\left(pc\right)}} \left(\varrho \bm{\cdot} \varsigma\right) \left(k^{\left(c\right)} \nabla T_{\epsilon}^{\left(c\right)}\right) \bm{\cdot} \mathbf{n}^{\left(c\right)}_\epsilon \ \dd\mathbf{y} \nonumber \\
&+ \int_{J_{out}} \left(\phi^{\left(c\right)}\varrho \bm{\cdot} \mathcal{R}\right)\left\langle\Pi\left(T_{\epsilon}^{\left(c\right)}, \mathbf{x}\right) \right\rangle_Y \ \dd\mathbf{y}= \nonumber \\
&+ \int_{\Gamma_{out}} \left(-\U^{\left(c\right)}\langle T^{\left(c\right)} \rangle_{Y} + \V^{\left(c\right)} \langle T^{\left(p\right)} \rangle_{Y} +  \left(\K^{\left(c\right)} \bm{\cdot} \nabla_{\mathbf{x}} \langle T^{\left(c\right)} \rangle_{Y}\right) \right) \bm{\cdot} \mathbf{n} \ \dd\mathbf{y} \nonumber \\
 &+ \int_{J_{out}} \langle T^{\left(c\right)} \rangle_{Y} \nabla_{\mathbf{x}} \bm{\cdot} \U^{\left(c\right)}  \ \dd\mathbf{y} - \int_{J_{out}} \langle T^{\left(p\right)} \rangle_{Y} \nabla_{\mathbf{x}} \bm{\cdot} \V^{\left(c\right)}  \ \dd\mathbf{y} + \int_{J_{out}} \boldsymbol{\mathbb{S}}^{c}\left(t, \mathbf{x}\right) \ \dd\mathbf{y} \nonumber \\
&+\int_{\Gamma^{\left(HC\right)}}\left(\phi^{\left(c\right)}\varrho \bm{\cdot} \varsigma\right) \left\langle \left(k^{\left(c\right)} \nabla T_{\epsilon}^{\left(c\right)}\right) \right\rangle_Y  \bm{\cdot} \mathbf{n}_\epsilon^{\left(c\right)} \ \dd\mathbf{y} \nonumber \\
 &+ \int_{\Gamma^{\left(HC\right)}}\left(-\U^{\left(c\right)}\langle T^{\left(c\right)} \rangle_{Y} + \V^{\left(c\right)} \langle T^{\left(p\right)} \rangle_{Y} +  \left(\K^{\left(c\right)} \bm{\cdot} \nabla_{\mathbf{x}} \langle T^{\left(c\right)} \rangle_{Y}\right) \right) \bm{\cdot} \mathbf{n} \ \dd\mathbf{y}
\end{align}

Since integrating over $J_{out}$ and $\Gamma_{out}$ gives identically 0, the equation can be simplified as
\begin{align}
\label{eq:cell-flux-cond}
&\int_{\Gamma^{\left(HC\right)}}\left(\phi^{\left(c\right)}\varrho \bm{\cdot} \varsigma\right) \left\langle \left(k^{\left(c\right)} \nabla T_{\epsilon}^{\left(c\right)}\right) \right\rangle_Y  \bm{\cdot} \mathbf{n}_\epsilon^{\left(c\right)} \ \dd\mathbf{y} \nonumber \\
 &=- \int_{\Gamma^{\left(HC\right)}}\left(-\U^{\left(c\right)}\langle T^{\left(c\right)} \rangle_{Y} + \V^{\left(c\right)} \langle T^{\left(p\right)} \rangle_{Y} +  \left(\K^{\left(c\right)} \bm{\cdot} \nabla_{\mathbf{x}} \langle T^{\left(c\right)} \rangle_{Y}\right) \right) \bm{\cdot} \mathbf{n} \ \dd\mathbf{y}.
\end{align}
Since $\Gamma^{\left(HC\right)}$ is the imaginary surface in the fine-scale battery cell domain, the integral on the LHS of equation~\eqref{eq:cell-flux-cond} vanishes, then the integral on the RHS must vanish to satisfy the relationship. Therefore, there is no flux coupling for the temperature of battery cells.



% \section{Weak formulation and numerical implementation of governing equations}
% \label{sec:weak-govern-eqs}
% \subsection{Weak formulation}
% \subsubsection{fine-scale equations}
% To derive the weak formulation of the fine-scale governing equations, we use $\mathcal{S}^{p}_{h}$ and $\mathcal{S}^{c}_{h}$ to denote meshes on the battery packing and cell domains $\mathcal{B}_\epsilon^{p}$ and $\mathcal{B}_\epsilon^{c}$ with $N^{p}$ and $N^{c}$ nodes, respectively. As such, the trial- and test-function spaces for the packing materials can be defined as 
% \begin{subequations}
% \begin{eqnarray}
% && U_{h,k}^{p} = \left\{ u^p_h \in \mathcal{C}^0\left(\mathcal{B}_\epsilon^p \right) : u_h^p \vert_K^p \in \mathcal{P}_k\left( K^p \right), \forall K^p \in \mathcal{S}_h^p, \left.\pdv{u_h^p}{\mathbf{n}} \right\vert_{\Gamma_\epsilon^R} = 0, \left. u_h^p \right\vert_{\Gamma_\epsilon^T} = \left. u_h^p \right\vert_{\Gamma_\epsilon^B},  u_h^p \vert_{\Gamma^{pc}} = 1 \right\}, \\
% && V_{h,k}^{p} = \left\{ v^p_h \in \mathcal{C}^0\left(\mathcal{B}_\epsilon^p \right) : v_h^p \vert_K^p \in \mathcal{P}_k\left( K^p \right), \forall K^p \in \mathcal{S}_h^p,\left.v_h^p \right\vert_{\Gamma_\epsilon^T \cup \Gamma_\epsilon^B \cup \Gamma_\epsilon^R} = 0, v_h^p \vert_{\Gamma^{pc}} = 0 \right\}, 
% \end{eqnarray}
% \end{subequations}
% and for the battery cells can be defined as 
% \begin{subequations}
% \begin{eqnarray}
% && U_{h,k}^{c} = \left\{ u^c_h \in \mathcal{C}^0\left(\mathcal{B}_\epsilon^c \right) : u_h^c \vert_K^c \in \mathcal{P}_k\left( K^c \right), \forall K^c \in \mathcal{S}_h^c ,\left.\pdv{u_h^c}{\mathbf{n}} \right\vert_{\Gamma_\epsilon^R} = 0, \left. u_h^c \right\vert_{\Gamma_\epsilon^T} = \left. u_h^c \right\vert_{\Gamma_\epsilon^B}, u_h^c \vert_{\Gamma^{pc}} = 1 \right\}, \\
% && V_{h,k}^{c} = \left\{ v^c_h \in \mathcal{C}^0\left(\mathcal{B}_\epsilon^c \right) : v_h^c \vert_K^c \in \mathcal{P}_k\left( K^c \right), \forall K^c \in \mathcal{S}_h^c, \left.v_h^c \right\vert_{\Gamma_\epsilon^T \cup \Gamma_\epsilon^B \cup \Gamma_\epsilon^R} = 0, v_h^c \vert_{\Gamma^{pc}} = 0 \right\}, 
% \end{eqnarray}
% \end{subequations}
% where $k$ is the polynomial order. With the defined functional space, we can approximate the temperature fields for the packing material $T_\epsilon^{\left(p\right)}$ and the battery cells $T_\epsilon^c$ as 
% \begin{subequations}
% \begin{eqnarray}
% &&  T_\epsilon^{\left(p\right)}(t_\epsilon, \mathbf{x}_\epsilon) \approx T_{\epsilon,h}^p (t_\epsilon, \mathbf{x}_\epsilon) = \sum_{i=1}^{N^p} T_i^p(t_\epsilon) \phi_{i,k}^p(\mathbf{x}_\epsilon), \\
% &&  T_\epsilon^c(t_\epsilon, \mathbf{x}_\epsilon) \approx T_{\epsilon,h}^c (t_\epsilon, \mathbf{x}_\epsilon) = \sum_{i=1}^{N^c} T_i^c(t_\epsilon) \phi_{i,k}^c(\mathbf{x}_\epsilon),
% \end{eqnarray}
% \end{subequations}
% where $\phi_{i,k}^p(\mathbf{x}_\epsilon) \in U^p_{h,k}$ and $\phi_{i,k}^c(\mathbf{x}_\epsilon) \in U^c_{h,k}$.

% We then approximate the equations with discrete temperature fields $T_{\epsilon,h}^{p}$ and $T_{\epsilon,h}^{c}$ and discrete test functions $\psi_{j,k}^{p} \in V^p_{h,k}$ and $\psi_{j,k}^{c} \in V^c_{h,k}$. By multiplying the equation with discrete test functions, integrating over the domain, and applying integration-by-parts, 
% \begin{subequations}
% \begin{align}
% \int_{\mathcal{B}_\epsilon^p} \psi_{j,k}^p \pdv{T_{\epsilon,h}^p}{t_\epsilon} \ \dd\mathbf{x}_\epsilon =  &-\int_{\mathcal{B}_\epsilon^p}    \grad_{\epsilon}{\psi_{j,k}^p} \bm{\cdot} k^{\left(p\right)} \grad_{\epsilon}{T_{\epsilon,h}^p} +  \oint_{\Gamma_\epsilon^{pc}} \psi_{j,k}^p\left(-\mathbf{n}_\epsilon^p \bm{\cdot} k^{\left(p\right)} \grad_{\epsilon}{T_{\epsilon,h}^p}  \right) \ \dd\mathbf{x}_\epsilon \nonumber \\
% &+ \oint_{\Gamma_\epsilon^{w}} \psi_{j,k}^p\left(-\mathbf{n}_\epsilon^p \bm{\cdot} k^{\left(p\right)} \grad_{\epsilon}{T_{\epsilon,h}^p}  \right) \ \dd\mathbf{x}_\epsilon  \nonumber \\
% &+ \int_{\Gamma_\epsilon^{T} \cup \Gamma_\epsilon^{B} \cup \Gamma_\epsilon^{R}} \psi_{j,k}^p\left(-\mathbf{n}_\epsilon^p \bm{\cdot} k^{\left(p\right)} \grad_{\epsilon}{T_{\epsilon,h}^p}  \right) \ \dd\mathbf{x}_\epsilon \nonumber \\
% &+ \int_{\Gamma^{\left(HC\right)} } \psi_{j,k}^p\left(-\mathbf{n}_\epsilon^p \bm{\cdot} k^{\left(p\right)} \grad_{\epsilon}{T_{\epsilon,h}^p}  \right) \ \dd\mathbf{x}_\epsilon, \\
% \int_{\mathcal{B}_\epsilon^c} \psi_{j,k}^c \pdv{T_{\epsilon,h}^c}{t_\epsilon} \ \dd\mathbf{x}_\epsilon =  &-\left(\varrho \bm{\cdot} \varsigma \right)\int_{\mathcal{B}_\epsilon^c}    \grad_{\epsilon}{\psi_{j,k}^c} \bm{\cdot} k^c \grad_{\epsilon}{T_{\epsilon,h}^c} \nonumber \\
% &+ \left(\varrho \bm{\cdot} \mathcal{R} \right) \int_{\mathcal{B}_\epsilon^c} \psi_{j,k}^c \Pi(t_\epsilon, \mathbf{x}_\epsilon) \ \dd\mathbf{x}_\epsilon \nonumber \\
% &+ \left(\varrho \bm{\cdot} \varsigma \right) \oint_{\Gamma_\epsilon^{pc}} \psi_{j,k}^c\left(-\mathbf{n}_\epsilon^c \bm{\cdot} k^c \grad_{\epsilon}{T_{\epsilon,h}^c}  \right) \ \dd\mathbf{x}_\epsilon \nonumber \\
% &+ \left(\varrho \bm{\cdot} \varsigma \right)\int_{\Gamma_\epsilon^{T} \cup \Gamma_\epsilon^{B} \cup \Gamma_\epsilon^{R} } \psi_{j,k}^c\left(-\mathbf{n}_\epsilon^c \bm{\cdot} k^c \grad_{\epsilon}{T_{\epsilon,h}^c}  \right) \ \dd\mathbf{x}_\epsilon \nonumber \\
% &+ \left(\varrho \bm{\cdot} \varsigma \right)\int_{\Gamma^{\left(HC\right)}} \psi_{j,k}^c\left(-\mathbf{n}_\epsilon^c \bm{\cdot} k^c \grad_{\epsilon}{T_{\epsilon,h}^c}  \right) \ \dd\mathbf{x}_\epsilon,
% \end{align}
% \end{subequations}
% By applying the boundary conditions (equations~\ref{eq:pore-bc} and~\ref{eq:taylor-hc}/\ref{eq:series-hc}) depending on the choice of hybridization approach), the weak formulations of the fine-scale equations are derived as
% \begin{subequations}
% \begin{align}
% \int_{\mathcal{B}_\epsilon^p} \psi_{j,k}^p \pdv{T_{\epsilon,h}^p}{t_\epsilon} \ \dd\mathbf{x}_\epsilon =  &-\int_{\mathcal{B}_\epsilon^p}    \grad_{\epsilon}{\psi_{j,k}^p} \bm{\cdot} k^{\left(p\right)} \grad_{\epsilon}{T_{\epsilon,h}^p} -  \oint_{\Gamma_\epsilon^{pc}} \psi_{j,k}^p \bm{\cdot} \text{Bi}^{\left(p\right)} \bm{\cdot} \left(T_{\epsilon,h}^c - T_{\epsilon,h}^p \right) \ \dd\mathbf{x}_\epsilon \nonumber \\
% &+ \oint_{\Gamma_\epsilon^{w}} \psi_{j,k}^p \bm{\cdot} \mathcal{Q} \bm{\cdot} q_\epsilon^{pw} \ \dd\mathbf{x}_\epsilon + \int_{\Gamma_\epsilon^{HC} } \psi_{j,k}^p \beta q^{\left(p,n\right)} \ \dd\mathbf{x}_\epsilon, \\
% \int_{\mathcal{B}_\epsilon^c} \psi_{j,k}^c \pdv{T_{\epsilon,h}^c}{t_\epsilon} \ \dd\mathbf{x}_\epsilon =  &-\left(\varrho \bm{\cdot} \varsigma \right)\int_{\mathcal{B}_\epsilon^c}    \grad_{\epsilon}{\psi_{j,k}^c} \bm{\cdot} k^c \grad_{\epsilon}{T_{\epsilon,h}^c} \nonumber \\
% &+ \left(\varrho \bm{\cdot} \mathcal{R} \right) \int_{\mathcal{B}_\epsilon^c} \psi_{j,k}^c \Pi(t_\epsilon, \mathbf{x}_\epsilon) \ \dd\mathbf{x}_\epsilon \nonumber \\
% &+ \left(\varrho \bm{\cdot} \varsigma \bm{\cdot} \text{Bi}^{\left(c\right)} \right) \oint_{\Gamma_\epsilon^{pc}} \psi_{j,k}^c \left(T_{\epsilon,h}^c  - T_{\epsilon,h}^p  \right) \ \dd\mathbf{x}_\epsilon, 
% \end{align}
% \end{subequations}
% where 
% \begin{align}
%     & \beta = 
%     \begin{cases}
%     \slfrac{\alpha}{\phi^{\left(p\right)}} \text{ for Taylor expansion},\\
%     \slfrac{1}{\phi^{\left(p\right)}} \text{ for Series expansion}.
%     \end{cases}
% \end{align}
% \subsubsection{Upscale equations}
% To derive the weak formulation of the upscaled equations, we followed a similar approach to define the test and trial function spaces of the upscaled domain as 
% \begin{subequations}
% \begin{eqnarray}
% && U_{h,k}^{\Omega} = \left\{ u^\Omega_h \in \mathcal{C}^0\left(\Omega \right) : u_h^\Omega \vert_K^\Omega \in \mathcal{P}_k\left( K^\Omega \right), \forall K^\Omega \in \mathcal{S}_h^\Omega, \left.\pdv{u_h^\Omega}{\mathbf{n}} \right\vert_{\Gamma^R} = 0, \left. u_h^\Omega \right\vert_{\Gamma^T} = \left. u_h^\Omega \right\vert_{\Gamma^B}, u_h^\Omega \vert_\Gamma^{pc} = 1 \right\}, \\
% && V_{h,k}^{\Omega} = \left\{ v^\Omega_h \in \mathcal{C}^0\left(\Omega \right) : v_h^\Omega \vert_K^\Omega \in \mathcal{P}_k\left( K^\Omega \right), \forall K^\Omega \in \mathcal{S}_h^\Omega \text{ and } \left.v_h^\Omega \right\vert_{\Gamma^T \cup \Gamma^B \cup \Gamma^R} = 0, v_h^\Omega \vert_\Gamma^{pc} = 0 \right\}.
% \end{eqnarray}
% \end{subequations}
%  With the defined functional space, we can approximate the average temperature fields for the packing material $\ensmean{T^p}_Y$ and battery cells $\ensmean{T^c}_Y$ as 
% \begin{subequations}
% \begin{eqnarray}
% &&  \ensmean{T^p}_Y(t, \mathbf{x}) \approx \Theta_{h}^p (t, \mathbf{x}) = \sum_{i=1}^{N^p} \Theta_i^p(t) \phi_{i,k}^{\Omega}(\mathbf{x}), \\
% && \ensmean{T^c}_Y(t, \mathbf{x}) \approx \Theta_{h}^c (t, \mathbf{x}) = \sum_{i=1}^{N^c} \Theta_i^c(t) \phi_{i,k}^{\Omega}(\mathbf{x}),
% \end{eqnarray}
% \end{subequations}
% where $\phi_{i,k}^{\Omega}(\mathbf{x}) \in U^{\Omega}_{h,k}$.

% By multiplying the equation with discrete test functions, integrating over the domain, and applying integration-by-parts, 
% \begin{subequations}
% \begin{align}
% \int_\Omega \psi_{j,k}^{\Omega}\pdv{\Theta_{h}^{p}}{t} \ \dd\mathbf{x} = &\int_\Omega \grad_{\mathbf{x}}\psi_{j,k} \bm{\cdot} \left(  \boldsymbol{\widetilde{\kappa}}^{p}\grad_{\mathbf{x}} \Theta_{h}^{p} + \boldsymbol{\theta}^{p} \Theta_{h}^{p} + \boldsymbol{\theta}^{(pc)} \Theta_{h}^{c} \right) \ \dd\mathbf{x} \nonumber \\ 
% & - \int_{\Gamma^{\Omega}} \psi_{j,k} \bm{\cdot} \left( -\mathbf{n} \bm{\cdot} \left(  \boldsymbol{\widetilde{\kappa}}^{p}\grad_{\mathbf{x}} \Theta_{h}^{p} + \boldsymbol{\theta}^{p} \Theta_{h}^{p} + \boldsymbol{\theta}^{pc} \Theta_{h}^{c}  \right) \right) \ \dd\mathbf{x} \nonumber \\
% & - \int_{\Gamma^{\left(HC\right)}} \psi_{j,k} \bm{\cdot} \left( -\mathbf{n} \bm{\cdot} \left(  \boldsymbol{\widetilde{\kappa}}^{p}\grad_{\mathbf{x}} \Theta_{h}^{p} + \boldsymbol{\theta}^{p} \Theta_{h}^{p} + \boldsymbol{\theta}^{pc} \Theta_{h}^{c}  \right) \right) \ \dd\mathbf{x} \nonumber \\
% &+ \int_{\Omega} \boldsymbol{\mathbb{S}^{p}}\left(t, \mathbf{x}\right) \ \dd\mathbf{x}, \\
% \int_\Omega \psi_{j,k}^{\Omega}\pdv{\Theta_{h}^{c}}{t} \ \dd\mathbf{x} = &\int_\Omega \grad_{\mathbf{x}}\psi_{j,k} \bm{\cdot} \left(  \boldsymbol{\theta}^{cp} \Theta_{h}^{p}  \right) \ \dd\mathbf{x} \nonumber \\ 
% & - \int_{\Gamma^{\Omega}} \psi_{j,k} \bm{\cdot} \left( -\mathbf{n} \bm{\cdot} \left(\boldsymbol{\theta}^{cp} \Theta_{h}^{p}  \right) \right) \ \dd\mathbf{x} \nonumber \\ 
% & - \int_{\Gamma^{\left(HC\right)}} \psi_{j,k} \bm{\cdot} \left( -\mathbf{n} \bm{\cdot} \left(\boldsymbol{\theta}^{cp} \Theta_{h}^{p}  \right) \right) \ \dd\mathbf{x} \nonumber \\ 
% &+ \int_{\Omega} \boldsymbol{\mathbb{S}^{c}}\left(t, \mathbf{x}\right) \ \dd\mathbf{x}.
% \end{align}
% \end{subequations}
% By applying the boundary conditions as stated in equation~\eqref{eq:taylor-hc}, we obtained the weak formulation of the upscaled equations as 
% \begin{subequations}
% \begin{align}
% \int_\Omega \psi_{j,k}^{\Omega}\pdv{\Theta_{h}^{p}}{t} \ \dd\mathbf{x} = &\int_\Omega \grad_{\mathbf{x}}\psi_{j,k} \bm{\cdot} \left(  \boldsymbol{\widetilde{\kappa}}^{p}\grad_{\mathbf{x}} \Theta_{h}^{p} + \boldsymbol{\theta}^{p} \Theta_{h}^{p} + \boldsymbol{\theta}^{pc} \Theta_{h}^{c}  \right) \ \dd\mathbf{x} \nonumber \\ 
% & - \int_{\Gamma^{\left(HC\right)}} \psi_{j,k} \bm{\cdot} \left( -\mathbf{n} \bm{\cdot} \ensmean{\mathbf{J}_\epsilon^{p}}_{Y} \right) \ \dd\mathbf{x} \nonumber \\ 
% &+ \int_{\Omega}\psi_{j,k} \boldsymbol{\mathbb{S}^{p}}\left(t, \mathbf{x}\right) \ \dd\mathbf{x}, \\
% \int_\Omega \psi_{j,k}^{\Omega}\pdv{\Theta_{h}^{c}}{t} \ \dd\mathbf{x} = &\int_\Omega \grad_{\mathbf{x}}\psi_{j,k} \bm{\cdot} \left( \boldsymbol{\theta}^{cp} \Theta_{h}^{p}  \right) \ \dd\mathbf{x} + \int_{\Omega} \psi_{j,k} \boldsymbol{ \mathbb{S}^{c}}\left(t, \mathbf{x}\right) \ \dd\mathbf{x}.
% \end{align}
% \end{subequations}

