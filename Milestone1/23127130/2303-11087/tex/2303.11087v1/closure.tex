\section{The Closure Problems}
\label{subsection:Appendix_E_Closure_Problems}
In this section, we describe the formulation of the closure problems required to solve the upscaled equations. The formulation of the closure problem is identical to Pietrzyk et al.~\cite{Pietrzyk2023-ou}.
Additionally, we use periodic boundary conditions at the edges of the rectangular unit-cell domain such that $\langle \chi^{\left(p\right)\left[i\right]} \rangle_{\mathcal{B}^{\left(p\right)}} = 0$, $\langle \chiv^{\left(p\right)\left[3\right]} \rangle_{\mathcal{B}^{\left(p\right)}} = \bm{0}$, $\langle \chi^{\left(c\right)\left[1\right]} \rangle_{\mathcal{B}^{\left(c\right)}} = 0$, and $\langle \chiv^{\left(c\right)\left[2\right]} \rangle_{\mathcal{B}^{\left(c\right)}} = \bm{0}$ for $i \in \{1,2\}$.


\subsection{Closure Problem for $\bm{\chi^{\left(p\right)\left[1\right]}}$}
\begin{subequations}
\neweq{}{-\frac{\mathcal{Q}}{|\mathcal{B}^{\left(p\right)}|}|\Gamma^{\left(pw\right)}| - k^{\left(p\right)} \nabla_{\xiv} \bm{\cdot} \nabla_{\xiv} \chi^{\left(p\right)\left[1\right]} = 0 \quad \text{for } \xiv \in \mathcal{B}^{\left(p\right)},}
\neweq{}{-k^{\left(p\right)}\n^{\left(p\right)} \bm{\cdot} \nabla_{\xiv}\chi^{\left(p\right)\left[1\right]} = 0 \quad \text{for } \xiv \in \Gamma^{\left(pc\right)},}
\neweq{}{-k^{\left(p\right)} \n^{\left(p\right)} \bm{\cdot} \nabla_{\xiv}\chi^{\left(p\right)\left[1\right]} = \mathcal{Q} \quad \text{for } \xiv \in \Gamma^{\left(pw\right)}.}
\end{subequations}


\subsection{Closure Problem for $\bm{\chi^{\left(p\right)\left[2\right]}}$}
\begin{subequations}
\neweq{}{-\frac{\text{Bi}^{\left(p\right)}}{|\mathcal{B}^{\left(p\right)}|}|\Gamma^{\left(pc\right)}| - k^{\left(p\right)} \nabla_{\xiv} \bm{\cdot} \nabla_{\xiv} \chi^{\left(p\right)\left[2\right]} = 0 \quad \text{for } \xiv \in \mathcal{B}^{\left(p\right)},}
\neweq{}{-k^{\left(p\right)}\n^{\left(p\right)} \bm{\cdot} \nabla_{\xiv}\chi^{\left(p\right)\left[2\right]} = \text{Bi}^{\left(p\right)} \quad \text{for } \xiv \in \Gamma^{\left(pc\right)},}
\neweq{}{-k^{\left(p\right)}\n^{\left(p\right)} \bm{\cdot} \nabla_{\xiv}\chi^{\left(p\right)\left[2\right]} = 0 \quad \text{for } \xiv \in \Gamma^{\left(pw\right)}.}
\end{subequations}

\subsection{Closure Problem for $\bm{\chiv^{\left(p\right)\left[3\right]}}$}
\begin{subequations}
\neweq{}{-k^{\left(p\right)} \nabla_{\xiv} \bm{\cdot} \left(\I + \nabla_{\xiv} \chiv^{\left(p\right)\left[3\right]}\right) = \mathbf{0} \quad \text{for } \xiv \in \mathcal{B}^{\left(p\right)},}
\neweq{}{-k^{\left(p\right)} \n^{\left(p\right)} \bm{\cdot} \left(\I + \nabla_{\xiv}\chiv^{\left(p\right)\left[3\right]}\right) = \mathbf{0} \quad \text{for } \xiv \in \Gamma^{\left(pc\right)},}
\neweq{}{-k^{\left(p\right)} \n^{\left(p\right)} \bm{\cdot} \left(\I + \nabla_{\xiv}\chiv^{\left(p\right)\left[3\right]}\right) = \mathbf{0} \quad \text{for } \xiv \in \Gamma^{\left(pw\right)}.}
\end{subequations}

\subsection{Closure Problem for $\bm{\chi^{\left(c\right)\left[1\right]}}$}
\begin{subequations}
\neweq{cell_closure0_1}{\frac{\text{Bi}^{\left(c\right)} \varrho \varsigma}{|\mathcal{B}^{\left(c\right)}|}|\Gamma^{\left(pc\right)}| - k^{\left(c\right)} \varrho \varsigma \nabla_{\xiv} \bm{\cdot} \nabla_{\xiv} \chi^{\left(c\right)\left[1\right]} = 0 \quad \text{for } \xiv \in \mathcal{B}^{\left(c\right)},}
\neweq{cell_closure0_2}{-k^{\left(c\right)}\n^{\left(c\right)} \bm{\cdot} \nabla_{\xiv}\chi^{\left(c\right)\left[1\right]} = -\text{Bi}^{\left(c\right)} \quad \text{for } \xiv \in \Gamma^{\left(pc\right)}.}
\end{subequations}

\subsection{Closure Problem for $\bm{\chiv^{\left(c\right)\left[2\right]}}$}
\begin{subequations}
\neweq{cell_closure1_1}{-k^{\left(c\right)} \varrho  \varsigma \nabla_{\xiv} \bm{\cdot} \left(\I + \nabla_{\xiv} \chiv^{\left(c\right)\left[2\right]}\right) = \mathbf{0} \quad \text{for } \xiv \in \mathcal{B}^{\left(c\right)},}
\neweq{cell_closure1_2}{-k^{\left(c\right)} \n^{\left(c\right)} \bm{\cdot} \left(\I + \nabla_{\xiv}\chiv^{\left(c\right)\left[2\right]}\right) = \mathbf{0} \quad \text{for } \xiv \in \Gamma^{\left(pc\right)}.}
\end{subequations}

