
\section{Introduction} 
\label{sec_intro}

The premise of homogenization theory is to derive upcaled models of physical systems by exploiting the separation of scales. Rigorous homogenization/coarse-graining methods \cite{Langlo1994-wq, Vasilyeva2019-wh,Das2005-gm,Frippiat2008-cl,Battiato2019-xk, Battiato2011-ad} provide \emph{a priori} guarantees such as asymptotic bounds on the deviation of upscaled model's solution from that of the fine-scale model. These error bounds remain valid over regions of space and durations of time where/when a set of {\it applicability conditions} hold. However, deriving the upscaled governing equations is tedious and often limits the adoption of the method. Recently, Pietryzk \emph{et al.} \cite{Pietrzyk2021-lu} developed a symbolic upscaling engine, \symbolica, capable of automatically deriving the homogenized partial differential equations (PDEs), as well as the applicability conditions, from the fine-scale PDEs, initial conditions (ICs) and boundary conditions (BCs). \symbolica is capable of handling complex multiscale, multi-physics, heterogeoenous, and nonlinear systems of PDEs that surpass humans' ability to handle by cumbersome pen-and-paper analysis. 

Although \emph{a priori} error bounds provide a solid ground for making cost-accuracy tradeoffs where/when the applicability conditions are satisfied, the challenge is to deal with situations in which such conditions are violated. This has been shown to happen in presence of, e.g., large spatial gradients of quantities of interest \cite{Battiato2011-zo, Pietrzyk2021-lu, Boso2013-ag}. % In fact, the most interesting phenomena occur when the applicability conditions are violated due to large gradients, material heterogeneity and aging.
To overcome these challenges, hybrid or algorithmic refinement formulations have been developed \cite{Yousefzadeh2017-yc,Battiato2011-ad,Tartakovsky2008-mt}: fine-scale equations are solved only in a sub-domain in which applicability conditions are violated whereas the upscaled equations are solved everywhere else. Hybrid simulations can be built through either intrusive or nonintrusive coupling conditions between the two sub-domains (fine-scale and continuum). Intrusive coupling methods are characterized by the existence of an ``overlapping" or ``handshake" region where both fine-scale and upscaled governing equations are concurrently solved \cite{Battiato2011-zo,Roubinet2013-fg,Pettersson2013-ou}. Intrusive coupling methods are typically more expensive due to the existence of the ``overlapping" or ``handshake" region, and more difficult to implement in legacy codes. In nonintrusive coupling methods, each subdomain is only solved with one set of governing equations, either fine-scale or upscaled, and the coupling between the subdomains is formulated exclusively in terms of boundary conditions \cite{Yousefzadeh2017-yc,Kadeethum2022-ag}. 


In this paper, we develop a predictive nonintrusive two-way coupled hybrid formulation. The computational domain is represented by two different materials, each of which is governed by a set of governing equations.  We apply this approach to the use-case of thermal runaway simulation in Li-ion batteries (LIBs) because they provide a sufficiently complex example with heterogeneity and nonlinearity in the source term. In Section~\ref{sec:govern-eqs}, we describe the fine-scale and upscaled governing equations and formulations of the coupling boundary conditions with two different approaches. In Section~\ref{sec:therm-res}, we evaluate the efficiency and accuracy of the proposed hybrid formulation for heat transfer in a battery pack as a demonstration. 
