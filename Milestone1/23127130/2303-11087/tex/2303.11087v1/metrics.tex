\subsection{DARPA Program Metrics} \label{sec_metrics}

The methods in this paper, developed with support and oversight of the DARPA Computable Models Disruption Opportunity \cite{DARPACOMPMods}, demonstrate several measurable advancements over the state-of-the-art. Here, we summarize them in terms of the relevant program metrics, i.e., modeling accuracy and numerical efficiency. As previously discussed, upscaling theory by multiple scale expansions ensures that the modeling error of coarse-grained approximations is \emph{a priori} bounded under appropriate dynamic conditions expressed ontherms of dimensionless numbers. When such conditions are locally (in space and/or time violated), it is therefore important that any further strategy (numerical or analytical) that aims at coupling fine-scale models with their continuum-scale counterpart in the same simulation domain be bounded by the oforementioned upscaling error. In this regard, the accuracy of any proposed hybrid scheme can be directly assessed against such an a priori error. In Sections~\ref{sec:acc-hc} and~\ref{sec:xhc-acc}, we show that both coupling schemes satisfy the requested accuracy. An additional important metric is that the computation cost associated with the iterative coupling between fine- and coarse-scale models does not overcome the cost of full fine-scale simulations over the microscopic domain (here considered the benchmark for both accuracy and cost). In Section~\ref{sec:efficiency}, we provide both an extensive analysis of the cost-accuracy tradeoffs as well as guidelines for the efficient adoption of hybrid algorithms in large-scale domains.