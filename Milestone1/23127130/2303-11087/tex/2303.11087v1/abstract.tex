\begin{abstract}
Accurate analytical and numerical modeling of multiscale systems is a daunting task. The need to properly resolve spatial and temporal scales spanning multiple orders of magnitude pushes the limits of both our theoretical models as well as our computational capabilities. Rigorous upscaling techniques enable efficient computation while bounding/tracking errors and making informed cost-accuracy tradeoffs. The biggest challenges arise when the applicability conditions for upscaled models break down. Here, we present a non-intrusive two-way coupled hybrid model, applied to thermal runaway in battery packs, that combines fine- and upscaled equations in the same numerical simulation to achieve predictive accuracy while limiting computational costs. First, we develop two methods with different orders of accuracy to enforce continuity at the coupling boundary. Then, we derive weak (i.e., variational) formulations of the fine-scale and upscaled governing equations for finite element (FE) discretization and numerical implementation in \fenics. We demonstrate that hybrid simulations can accurately predict the average temperature fields within errors bounds determined \emph{a priori} by homogenization theory. Finally, we demonstrate the computational efficiency of the hybrid algorithm against fine-scale simulations.
    % Accurate analytical and numerical modeling of multiscale systems is a daunting task. The need to properly resolve spatial and temporal scales spanning multiple orders of magnitude pushes the limits of both our theoretical models as well as our computational capabilities.  %Battery packs undergoing thermal runaway are one such example. While the thermal front's characteristic spatial scale may be at the scale of single cells, accurate predictions of its dynamics through the entire battery pack is critical to guide the overall system design. Classical approaches to model thermal runaway in battery packs include the use of coarse-grained macroscopic models, which describe the spatially-averaged response of the system. However, such models may become inaccurate if appropriate dynamic conditions are not met. Hybrid models that combine fine-scale (cell-scale) and continnum-scale (battery pack-scale) models in the same numerical simulation can achieve predictive accuracy while keeping computational costs in check. Here, we present a non-intrusive two-way coupled hybrid algorithm for simulating heat transfer in battery packs. 
    % %
    % Rigorous upscaling techniques enable efficient computation while bounding/tracking errors and making informed cost-accuracy tradeoffs. The biggest challenges arise when the applicability conditions for upscaled models break down. 
    % % Algorithmic refinement offers a powerful solution strategy in which fine-scale and upscaled equations are coupled in the same simulation, thereby enabling an optimized use of computing resources.
    % %
    % % However, deriving coupling conditions that preserve the rigorous guarantees are nontrivial, especially for heterogeneous nonlinear systems.
    % % Battery packs undergoing thermal runaway are one example of such systems.
    % %
    % Here, we present a non-intrusive two-way coupled hybrid model, applied to thermal runaway in battery packs, that combines fine- and upscaled equations in the same numerical simulation to achieve predictive accuracy while limiting computational costs. 
    % % We use our automated symbolic upscaling engine, \symbolica, to derive the upscaled equations and applicability conditions.
    % First, we develop two methods with different orders of accuracy to enforce continuity at the coupling boundary. Then, we derive weak (i.e., variational) formulations of the fine-scale and upscaled governing equations for finite element (FE) discretization and numerical implementation in \fenics. 
    % %
    % % In contrast to upscaled simulations that fail to produce accurate results when the dimensionless numbers fall out of applicability ranges, the proposed hybrid algorithm has been thoroughly validated against fine-scale simulations and is able to produce accurate results throughout the simulations. The effect of the distance between the breakdown region and the coupling boundary has been thoroughly studied. 
    % %
    % We demonstrate that hybrid simulations can accurately predict the average temperature fields within errors bounds determined \emph{a priori} by homogenization theory. Finally, we demonstrate the computational efficiency of the hybrid algorithm against fine-scale simulations.
\end{abstract}


\begin{keyword}
%% keywords here, in the form: keyword \sep keyword

%% PACS codes here, in the form: \PACS code \sep code

%% MSC codes here, in the form: \MSC code \sep code
%% or \MSC[2008] code \sep code (2000 is the default)

Homogenization \sep Upscaling \sep Multiscale Modeling \sep Hybrid Models \sep Algorithmic Refinement

\end{keyword}