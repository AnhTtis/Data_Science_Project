\subsection{DARPA Program Metrics} \label{sec_metrics}

The methods in this paper, developed with support and oversight of the DARPA Computable Models Disruption Opportunity \cite{DARPACOMPMods}, demonstrate several measurable advancements over the state-of-the-art. Here, we summarize them in terms of the relevant program metrics, i.e., modeling accuracy and numerical efficiency. As previously discussed, upscaling theory by multiple scale expansions ensures that the modeling error of coarse-grained approximations is \emph{a priori} bounded under appropriate dynamic conditions expressed {in terms} of dimensionless numbers. When such conditions are locally (in space and/or time violated), it is therefore important that any further strategy (numerical or analytical) that aims at coupling fine-scale models with their continuum-scale counterpart in the same simulation domain be bounded by the {aforementioned} upscaling error. In this regard, the accuracy of any proposed hybrid scheme can be directly assessed against such an a priori error. In Sections~\ref{sec:acc-hc} and~\ref{sec:xhc-acc}, we show that both coupling schemes satisfy the requested accuracy. An additional important metric is that the computation cost associated with the iterative coupling between fine- and coarse-scale models does not overcome the cost of full fine-scale simulations over the microscopic domain (here considered the benchmark for both accuracy and cost). In Section~\ref{sec:efficiency}, we provide both an extensive analysis of the cost-accuracy tradeoffs as well as guidelines for the efficient adoption of hybrid algorithms in large-scale domains.

\subsubsection{\bf Direct Computability:}
{Governing PDEs of heterogeneous multiscale systems are derived from first principles (e.g., conservation laws) and constitutive/material laws, applicable at length/time scales that are not ideal for computation, optimization or design, since they require resolving geometric features at scales much finer than the device scale. Upscaled (e.g., homogenized) models, on the other hand, are computationally more tractable  and, generally, more easily parametrizable since  their effective properties can be measured at the device scales. However, they are not directly computable when their applicability conditions  are violated because they loose accuracy and error guarantees can not be satisfied. Our approach enables bridging the gap between formulation and computation of multiscale systems where diffent models may be needed in the same simulation domain to ensure predictive accuracy while keeping computational costs in check: we achieve this by using the proper model  at the proper scales (fine-scale or upscaled), in different regions of space or intervals of time, and coupling them at the interfaces, while respecting the error bounds. Moreover, the non-intrusive nature of the coupling conditions allows one to use existing (e.g., off-the-shelf commercial or open-source) solvers, specialized for each regime/scale, without having to rewrite the code. By providing the information on the dimensionless numbers, the method can automatically determine the location of the breakdown region and compute the coupling locations.} 

\subsubsection{Accuracy and Efficiency:}
{As a result of the development of the hybrid non-intrusive two-way coupling approach, the errors are still bounded by the upscaling errors which can be determined \emph{a priori}. Additionally, the computational cost of hybrid simulations is significantly lower than that of fine-scale simulations 
%\deleted{with the existence of a breakdown region} 
(Section~\ref{sec:efficiency}). The performance metrics on computatibility and accuracy are summarized in Table~\ref{tab:darpa-metrics}.}


\begin{table}[H]
\centering
\caption{
  \label{tab:darpa-metrics} Summary of performance metrics.}
\begin{tabular}{p{0.15\linewidth}|p{0.25\linewidth}|p{0.25\linewidth}|p{0.25\linewidth}}
\hline \hline
Metric &  SOA & Initial & Ultimate Target  \\ \hline
Computability  & Fine-scale simulations over macroscopic domains & Hybrid coupling

For 80 battery cells and 2.5\% of volume solved by fine-scale model: speedup=3 
 & Hybrid coupling

For 800 battery cells and 2.5\% of volume solved by fine-scale model: speedup=15 \\ \hline
Conservation \& Accuracy  & Fine-scale simulations over macroscopic domains & Not Applicable.

Cannot guarantee error bounds, unless by fully resolving fine-scale models.

 & A priori asymptotic error guarantees of order $\epsilon$. \\ \hline \hline
\end{tabular}%
\end{table}