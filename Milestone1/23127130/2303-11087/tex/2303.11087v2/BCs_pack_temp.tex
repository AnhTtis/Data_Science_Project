
\section{Derivation of flux boundary condition for packing temperature}
\label{app:packing-flux-proof}
Apply the averaging operator (equation~\eqref{eq:ave-op}) and multiply by the volume fraction $\phi^{\left(p\right)}$ to the fine-scale packing temperature equations (equation~\eqref{eq:pore_goven_eqs}) to derive the averaged expression as %\textcolor{red}{[$\nabla$ should not have a subscript $\epsilon$ in any of these (see notation in equation (4))]}
\begin{align}
 &\phi^{\left(p\right)}\derive{ \left\langle T_{\epsilon}^{\left(p\right)} \right\rangle_Y }{t} = \phi^{\left(p\right)}\ensmean{k^{\left(p\right)}\grad \bm{\cdot} \grad{T_\epsilon^{\left(p\right)}}}_Y. 
\end{align}
{Here, we would like emphasize that the averaging operator applied to the fine-scale equations depends on $\mathbf{x}$.} 
Integrate over an arbitrary coupling volume $J$ that contains the hybrid coupling boundary $\Gamma^{\left(HC\right)}$, one obtains %\textcolor{red}{[This might be too little too late, but it could be good to somehow clarify (in the notation or otherwise) that $\langle \bm{\cdot} \rangle_{Y}$ is $\mathbf{x}$ dependent. Otherwise the double integral is a little difficult to understand. See previous comment on defining integral over $\mathcal{W}$.]}
\begin{align}
 \int_J \phi^{\left(p\right)}\derive{\left\langle T_{\epsilon}^{\left(p\right)} \right\rangle_Y}{t} \ \dd\mathbf{y} = &\int_{J_{in}} \phi^{\left(p\right)}\ensmean{k^{\left(p\right)}\grad \bm{\cdot} \grad{T_\epsilon^{\left(p\right)}}}_Y \ \dd\mathbf{y} + \int_{J_{out}} \phi^{\left(p\right)}\ensmean{k^{\left(p\right)}\grad \bm{\cdot} \grad{T_\epsilon^{\left(p\right)}}}_Y \ \dd\mathbf{y},   
\end{align}
where $J_{in} = J \cap \Omega_{\text{fine}}$ and $J_{out} = J \cap \Omega_{\text{up}}$ are the partitions of the coupling volume that intersect with the fine-scale and the upscaled domain, respectively. Here, we apply the spatial averaging theorem~{\cite{Yousefzadeh2017-yc,Howes1985-nr}} to the flux term in $J_{in}$ and obtain% \textcolor{red}{[There are two $dy$'s on the LHS. Also, is $\mathcal{V}$ defined anywhere?]}
\begin{align}
 \int_J \phi^{\left(p\right)}\derive{\left\langle T_{\epsilon}^{\left(p\right)} \right\rangle_Y}{t} \ \dd\mathbf{y} &= \int_{J_{in}} \phi^{\left(p\right)}k^{\left(p\right)}\grad \bm{\cdot} \ensmean{\grad{T_\epsilon^{\left(p\right)}}}_Y \ \dd\mathbf{y} \nonumber \\
 &+ \ddfrac{\phi^{\left(p\right)}}{\abs{\mathcal{V}}\abs{J_{in}}} \int_{J_{in}} \int_{\Gamma^{pc}} k^{\left(p\right)} \grad{T_\epsilon^{\left(p\right)}} \bm{\cdot} \mathbf{n}^{\left(p\right)}_\epsilon \ \dd\mathbf{y} \nonumber \\ 
 &+ \int_{J_{out}} \phi^{\left(p\right)}\ensmean{k^{\left(p\right)}\grad \bm{\cdot} \grad{T_\epsilon^{\left(p\right)}}}_Y \ \dd\mathbf{y},
\end{align}
{where $\abs{\mathcal{V}}$ is the volume of battery cells.} Since the upscaled equations are valid in $J_{out}$, we can replace the terms in $J_{out}$ with the upscaled equations such that %\textcolor{red}{[There are two $dy$'s on the LHS.]} 
\begin{align}
 &\int_J \phi^{\left(p\right)}\derive{\left\langle T_{\epsilon}^{\left(p\right)} \right\rangle_Y}{t} \ \dd\mathbf{y} = \int_{J_{in}} \phi^{\left(p\right)}k^{\left(p\right)}\grad \bm{\cdot} \ensmean{\grad{T_\epsilon^{\left(p\right)}}}_Y \ \dd\mathbf{y} \nonumber \\
 &+ \ddfrac{\phi^{\left(p\right)}}{\abs{\mathcal{V}}\abs{J_{in}}} \int_{J_{in}} \int_{\Gamma^{pc}} k^{\left(p\right)} \grad{T_\epsilon^{\left(p\right)}} \bm{\cdot} \mathbf{n}^{\left(p\right)}_\epsilon \ \dd\mathbf{y} \nonumber \\
 &+ \int_{J_{out}} -\left(\U^{\left(p\right)} \bm{\cdot} \nabla_{\mathbf{x}} \langle T^{\left(p\right)} \rangle_{Y} - \V^{\left(p\right)} \bm{\cdot} \nabla_{\mathbf{x}} \langle T^{\left(c\right)} \rangle_{Y} - \nabla_{\mathbf{x}} \bm{\cdot} \left(\K^{\left(p\right)} \bm{\cdot} \nabla_{\mathbf{x}} \langle T^{\left(p\right)} \rangle_{Y}\right) \right)  \ \dd\mathbf{y} \nonumber \\ 
 &+ \int_{J_{out}} \boldsymbol{\mathbb{S}}^{\left(p\right)}\left(t, \mathbf{x}\right) \ \dd\mathbf{y},   
\end{align}
where 
\begin{align}
\boldsymbol{\mathbb{S}}^{\left(p\right)}\left(t, \mathbf{x}\right) = -R_1^{\left(p\right)}\langle T^{\left(p\right)} \rangle_{Y} + R_2^{\left(p\right)}\langle T^{\left(c\right)} \rangle_{Y} - R_3^{\left(p\right)}q^{\left(pw\right)}\left(t, \mathbf{x}\right) + \R_4^{\left(p\right)} \bm{\cdot} \nabla_{\mathbf{x}} q^{\left(pw\right)}\left(t, \mathbf{x}\right).
\end{align}
By applying the divergence theorem and dividing the boundaries into the external boundaries of the coupling volume $\Gamma_{in}$ and $\Gamma_{out}$ and the internal coupling boundary $\Gamma^{\left(HC\right)}$, one obtains %\textcolor{red}{[There are two $dy$'s on the LHS.]}
\begin{align}
\label{eq:packing-flux-rhs}
 \int_J \phi^{\left(p\right)} &\derive{\left\langle T_{\epsilon}^{\left(p\right)} \right\rangle_Y}{t} \ \dd\mathbf{y}  = \int_{\Gamma_{in}} \left(\phi^{\left(p\right)} k^{\left(p\right)} \ensmean{\grad{T_\epsilon^{\left(p\right)}}}_Y\right) \bm{\cdot} \mathbf{n}^{\left(p\right)}_\epsilon \ \dd\mathbf{y} \nonumber \\ 
 &+ \ddfrac{\phi^{\left(p\right)}}{\abs{\mathcal{V}}\abs{J_{in}}} \int_{J_{in}} \int_{\Gamma^{pc}} k^{\left(p\right)} \grad{T_\epsilon^{\left(p\right)}} \bm{\cdot} \mathbf{n}^{\left(p\right)}_\epsilon \ \dd\mathbf{y} \nonumber \\
 &+ \int_{\Gamma_{out}} \left(-\U^{\left(p\right)}\langle T^{\left(p\right)} \rangle_{Y} + \V^{\left(p\right)} \langle T^{\left(c\right)} \rangle_{Y} +  \left(\K^{\left(p\right)} \bm{\cdot} \nabla_{\mathbf{x}} \langle T^{\left(p\right)} \rangle_{Y}\right) \right) \bm{\cdot} \mathbf{n} \ \dd\mathbf{y} \nonumber \\
 &+ \int_{J_{out}} \langle T^{\left(p\right)} \rangle_{Y} \nabla_{\mathbf{x}} \bm{\cdot} \U^{\left(p\right)}  \ \dd\mathbf{y} - \int_{J_{out}} \langle T^{\left(p\right)} \rangle_{Y} \nabla_{\mathbf{x}} \bm{\cdot} \V^{\left(p\right)}  \ \dd\mathbf{y} + \int_{J_{out}} \boldsymbol{\mathbb{S}}^{\left(p\right)}\left(t, \mathbf{x}\right) \ \dd\mathbf{y} \nonumber \\
 &+ \int_{\Gamma^{\left(HC\right)}} \left(\phi^{\left(p\right)} k^{\left(p\right)} \ensmean{\grad{T_\epsilon^{\left(p\right)}}}_Y\right) \bm{\cdot} \mathbf{n}^{\left(p\right)}_\epsilon \ \dd\mathbf{y}  \nonumber \\
 &+ \int_{\Gamma^{\left(HC\right)}} \left(-\U^{\left(p\right)}\langle T^{\left(p\right)} \rangle_{Y} + \V^{\left(p\right)} \langle T^{\left(c\right)} \rangle_{Y} +  \left(\K^{\left(p\right)} \bm{\cdot} \nabla_{\mathbf{x}} \langle T^{\left(p\right)} \rangle_{Y}\right) \right) \bm{\cdot} \mathbf{n}  \ \dd\mathbf{y},   
\end{align}
where $\mathbf{n}$ is the normal vector pointing outwards of the upscaled domain. By applying the spatial average theorem, and divergence theorem to the time-derivative term in equation~\eqref{eq:packing-flux-rhs}, we obtain %\textcolor{red}{[There are two $dy$'s on the LHS.]}
\begin{align}
\label{eq:packing-flux-lhs}
 \int_J \phi^{\left(p\right)} \derive{\left\langle T_{\epsilon}^{\left(p\right)} \right\rangle_Y}{t} \ \dd\mathbf{y}  &= \int_{\Gamma_{in}} \left( \phi^{\left(p\right)} k^{\left(p\right)} \ensmean{\grad{T_\epsilon^{\left(p\right)}}}_Y\right) \bm{\cdot} \mathbf{n}^{\left(p\right)}_\epsilon \ \dd\mathbf{y} \nonumber \\
 &+ \ddfrac{\phi^{\left(p\right)}}{\abs{\mathcal{V}}\abs{J_{in}}}  \int_{J_{in}} \int_{\Gamma^{pc}} k^{\left(p\right)} \grad{T_\epsilon^{\left(p\right)}} \bm{\cdot} \mathbf{n}^{\left(p\right)}_\epsilon \ \dd\mathbf{y} \nonumber \\
 &+\int_{\Gamma_{out}} \left(\phi^{\left(p\right)}k^{\left(p\right)} \ensmean{\grad{T_\epsilon^{\left(p\right)}}}_Y\right) \bm{\cdot} \mathbf{n}^{\left(p\right)}_\epsilon \ \dd\mathbf{y} \nonumber \\
 &+ \ddfrac{\phi^{\left(p\right)}}{\abs{\mathcal{V}}\abs{J_{out}}} \int_{J_{out}} \int_{\Gamma^{pc}} k^{\left(p\right)} \grad{T_\epsilon^{\left(p\right)}} \bm{\cdot} \mathbf{n}^{\left(p\right)}_\epsilon \ \dd\mathbf{y},   
\end{align}
By equating equation~\eqref{eq:packing-flux-rhs} and equation~\eqref{eq:packing-flux-lhs}, the simplified equation is expressed as
\begin{align}
& \int_{\Gamma_{out}} \left(\phi^{\left(p\right)} 
 k^{\left(p\right)} \ensmean{\grad{T_\epsilon^{\left(p\right)}}}_Y\right) \bm{\cdot} \mathbf{n}^{\left(p\right)}_\epsilon \ \dd\mathbf{y} + \ddfrac{\phi^{\left(p\right)}}{\abs{\mathcal{V}}\abs{J_{out}}}\int_{J_{out}} \int_{\Gamma^{pc}} k^{\left(p\right)} \grad{T_\epsilon^{\left(p\right)}} \bm{\cdot} \mathbf{n}^{\left(p\right)}_\epsilon \ \dd\mathbf{y} \nonumber \\ 
&- \int_{\Gamma_{out}} \left(-\U^{\left(p\right)}\langle T^{\left(p\right)} \rangle_{Y} + \V^{\left(p\right)} \langle T^{\left(c\right)} \rangle_{Y} +  \left(\K^{\left(p\right)} \bm{\cdot} \nabla_{\mathbf{x}} \langle T^{\left(p\right)} \rangle_{Y}\right) \right) \bm{\cdot} \mathbf{n} \ \dd\mathbf{y} \nonumber \\
 &- \int_{J_{out}} \langle T^{\left(p\right)} \rangle_{Y} \nabla_{\mathbf{x}} \bm{\cdot} \U^{\left(p\right)}  \ \dd\mathbf{y} + \int_{J_{out}} \langle T^{\left(p\right)} \rangle_{Y} \nabla_{\mathbf{x}} \bm{\cdot} \V^{\left(p\right)}  \ \dd\mathbf{y} - \int_{J_{out}} \boldsymbol{\mathbb{S}}^{\left(p\right)}\left(t, \mathbf{x}\right) \ \dd\mathbf{y} =\nonumber \\
 &+\int_{\Gamma^{\left(HC\right)}} \left(\phi^{\left(p\right)} k^{\left(p\right)} \ensmean{\grad{T_\epsilon^{\left(p\right)}}}_Y\right) \bm{\cdot} \mathbf{n}^{\left(p\right)}_\epsilon \ \dd\mathbf{y} \nonumber \\
 &+ \int_{\Gamma^{\left(HC\right)}} \left(-\U^{\left(p\right)}\langle T^{\left(p\right)} \rangle_{Y} + \V^{\left(p\right)} \langle T^{\left(c\right)} \rangle_{Y} +  \left(\K^{\left(p\right)} \bm{\cdot} \nabla_{\mathbf{x}} \langle T^{\left(p\right)} \rangle_{Y}\right) \right) \bm{\cdot} \mathbf{n}  \ \dd\mathbf{y} , 
\end{align}
The integral over $J_{out}$ and $\Gamma_{out}$ is identically 0 because of the equvalence between fine-scale and upscaled equations, therefore, the packing flux condtion is obtained as
\begin{align}
& \int_{\Gamma^{\left(HC\right)}} \left(\phi^{\left(p\right)} k^{\left(p\right)} \ensmean{\grad{T_\epsilon^{\left(p\right)}}}_Y\right) \bm{\cdot} \mathbf{n}^{\left(p\right)}_\epsilon + \left(-\U^{\left(p\right)}\langle T^{\left(p\right)} \rangle_{Y}   + \V^{\left(p\right)} \langle T^{\left(c\right)} \rangle_{Y} +  \left(\K^{\left(p\right)} \bm{\cdot} \nabla_{\mathbf{x}} \langle T^{\left(p\right)} \rangle_{Y}\right)  \right) \bm{\cdot}  \mathbf{n}   \ \dd\mathbf{y} =0,
 \end{align}
which can be simplified as 
\begin{subequations}
\begin{align}
& \phi^{\left(p\right)} \ensmean{\mathbf{J}_\epsilon^{\left(p\right)}}_{Y} \bm{\cdot} \mathbf{n}^{\left(p\right)}_\epsilon = \ensmean{\mathbf{J}^{\left(p\right)}}_{Y} \bm{\cdot} \mathbf{n}^{\left(p\right)}_\epsilon,
\end{align}
\end{subequations}
where 
\begin{align}
&\ensmean{\mathbf{J}_\epsilon^{\left(p\right)}}_{Y}  =- \left(k^{\left(p\right)}\ensmean{\grad{T_\epsilon^{\left(p\right)}}}_Y\right), \\
&\ensmean{\mathbf{J}^{\left(p\right)}}_{Y} = \U^{\left(p\right)}\langle T^{\left(p\right)} \rangle_{Y}   - \V^{\left(p\right)} \langle T^{\left(c\right)} \rangle_{Y} -  \left(\K^{\left(p\right)} \bm{\cdot} \nabla_{\mathbf{x}} \langle T^{\left(p\right)} \rangle_{Y}\right),
\end{align}
$\mathbf{n}^{\left(p\right)}_\epsilon = -\mathbf{n}$ refers to the relationship between normal vector of the fine-scale and upscaled domains.