\section{Thermal runaway problems}
\label{sec:therm-res}
\subsection{Test case description}
\begin{figure}[hbpt!]
\centerline{
 {\includegraphics[width=\textwidth]{figures/Setup_diag.pdf}}}

\caption{Schematic diagram of the setup used in the simulations.}
\label{fig:setup-diag}
\end{figure}
To validate the accuracy of the hybrid coupling method developed, we consider a two-dimensional domain with $N_{x}^{\left(c\right)}=20$ and $N_{y}^{\left(c\right)}=1$ (Figure~\ref{fig:setup-diag}). Table~\ref{tab:unit-cell-geom-params} summarizes the parameters used to define the geometry of the unit cell. Periodic boundary conditions are applied at the top and bottom, whereas zero gradient boundary conditions are applied to the left and right boundaries of the domain. To model a battery thermal runaway problem, we modify the power flux source term in the fine-scale equations as 
\begin{align}
\Pi\left(T_{\epsilon}^{\left(c\right)}, \mathbf{x}\right) &=
\begin{cases}
\Pi_{\text{NB}}\left(T_{\epsilon}^{\left(c\right)}\right), \text{ for } \mathbf{x} > \mathbf{x}_{\text{burn}}, \\
\Pi_{\text{FB}}\left(T_{\epsilon}^{\left(c\right)}\right), \text{ for } \mathbf{x} \le \mathbf{x}_{\text{burn}},
\end{cases}
\end{align}
where $\mathbf{x}_{\text{burn}}$ is the location that separates the burning and unburned regions, %\textcolor{red}{[Check which Pi is burning and unburning (i.e., NB vs. FB). It is inconsistent in the following (not sure if it is throughout the manuscript as well).]} 
$\Pi_{\text{NB}}\left(T_{\epsilon}^{\left(c\right)}\right)$ and $\Pi_{\text{FB}}\left(T_{\epsilon}^{\left(c\right)}\right)$ are the dimensionless fine-scale power flux source term for unburned and burning battery cell such that
\begin{subequations}
\begin{align}
\Pi_{\text{NB}}\left(T_{\epsilon}^{\left(c\right)}\right) &= \Pi_{\text{base}} + \frac{1}{2}\left\{\text{Erf}\left[A_1 T_{\epsilon}^{\left(c\right)} + B_1\right] + 1\right\}\left(1 - \Pi_{\text{base}}\right) \nonumber \\ 
& - \frac{1}{2}\left\{\text{Erf}\left[A_2 T_{\epsilon}^{\left(c\right)} + B_2\right] + 1\right\}\\
\Pi_{\text{FB}}\left(T_{\epsilon}^{\left(c\right)}\right) &= \Pi_{\text{burn}} - \frac{1}{2}\left\{\text{Erf}\left[A_2 T_{\epsilon}^{\left(c\right)} + B_2\right] + 1\right\}.\label{eq:new_pi_term_dimless_burn}
\end{align}
\end{subequations}
For the upscaled equations, we use a sigmoid function to calculate the power flux source term to avoid discontinuity such that
\begin{align}
\overline{\Pi} \left(\langle T^{\left(c\right)} \rangle_{Y}, \mathbf{x}\right) &= \ddfrac{ \overline{\Pi}_{\text{FB}} \left(\langle T^{\left(c\right)} \rangle_{Y}, \mathbf{x}\right) -  \overline{\Pi}_{\text{NB}} \left(\langle T^{\left(c\right)} \rangle_{Y}, \mathbf{x}\right)}{1+\exp\left(\gamma(\mathbf{x} - 
\mathbf{x}_{\text{burn}})\right)} + \overline{\Pi}_{\text{NB}} \left(\langle T^{\left(c\right)} \rangle_{Y}, \mathbf{x}\right), \label{eq:upscale_force_pi_term}
\end{align}
{where $\gamma=180$ is a constant to approximate a stepwise function}, $\overline{\Pi}_{\text{NB}} \left(\langle T^{\left(c\right)} \rangle_{Y}, \mathbf{x}\right)$ and $\overline{\Pi}_{\text{FB}} \left(\langle T^{\left(c\right)} \rangle_{Y}, \mathbf{x}\right)$ are the dimensionless upscaled power flux source term for unburned and burning battery cells that are defined as
\begin{subequations}
\begin{align}
\overline{\Pi}_{\text{NB}} \left(\langle T^{\left(c\right)} \rangle_{Y}, \mathbf{x}\right) &= \Pi_{\text{base}}\left(\mathbf{x}\right) + \frac{1}{2}\left\{\text{Erf}\left[\frac{A_1}{\phi^{\left(c\right)}} \langle T^{\left(c\right)} \rangle_{Y} + B_1\right] + 1\right\}\left(1 - \Pi_{\text{base}}\left(\mathbf{x}\right)\right)  \nonumber \\
&- \frac{1}{2}\left\{\text{Erf}\left[\frac{A_2}{\phi^{\left(c\right)}} \langle T^{\left(c\right)} \rangle_{Y} + B_2\right] + 1\right\}, \\
\overline{\Pi}_{\text{FB}} \left(\langle T^{\left(c\right)} \rangle_{Y}, \mathbf{x}\right) &= \Pi_{\text{burn}}\left(\mathbf{x}\right) - \frac{1}{2}\left\{\text{Erf}\left[\frac{A_2}{\phi^{\left(c\right)}} \langle T^{\left(c\right)} \rangle_{Y} + B_2\right] + 1\right\}. 
\end{align}
\end{subequations}
Additionally, in reality, the rate of heat generation for burning battery cells can vary from cell to cell because of potential manufacturing defects and aging. Therefore, we consider that a region of burning battery cells has different rates of heat generation
where 
\begin{align}
    \hat{\Pi}_{\text{burn}} = 
    \begin{cases}
    \hat{\Pi}_{\text{burn},\text{low}}=\hat{\Pi}_{\text{burn},0}, \text{ for } \mathbf{x} > \mathbf{x}_{\mathcal{R}}, \\
    \hat{\Pi}_{\text{burn},\text{high}}=10\hat{\Pi}_{\text{burn},0}, \text{ for } \mathbf{x} \le \mathbf{x}_{\mathcal{R}},
    \end{cases}
\end{align}
such that
\begin{align}
    &\hat{\Pi}_{\text{burn},0} = \ddfrac{\hat{T}_{max} \hat{k}^{\left(p\right)}}{\max{(\hat{\mathcal{L}}_{x}, \hat{\mathcal{L}}_{y})}\hat{\ell}}.
\end{align}
Incorporating the effect of $\hat{\Pi}_{\text{burn},\text{low}}$ and $\hat{\Pi}_{\text{burn},\text{high}}$, we define the dimensionless number $\mathcal{R}$ as 
\begin{align}
&\mathcal{R}(\mathbf{x}) = \left(\ddfrac{\mathcal{R}_{\text{high}} + \mathcal{R}_{\text{low}} }{2}\right) - \left(\ddfrac{\mathcal{R}_{\text{high}} - \mathcal{R}_{\text{low}} }{2}\right)\tanh\left(100(x-\mathbf{x}_{\mathcal{R}}) \right),
\label{eq:Rx-def}
\end{align}
where $\mathcal{R}_{\text{high}} = 10\mathcal{R}_{0}$ and $\mathcal{R}_{\text{low}} = \mathcal{R}_{0}$ represent regions with high and low heat generation rates, respectively, and $\mathcal{R}_{0}$ is defined with $\hat{\Pi}_{\text{burn},0}$ in equation~\eqref{eq:dimless_groups}. By setting $\mathbf{x}_{\text{burn}}$ and $\mathbf{x}_{\mathcal{R}}$, two temperature gradients will be observed due to (1) the difference in the rate of heat generation by burned battery cells at $\mathbf{x}_{\mathcal{R}}$ and (2) the difference in the rate of heat generation by burned and unburned battery cells at $\mathbf{x}_{\text{burn}}$. 
\begin{table}[ht!]
\centering
\caption{
  \label{tab:unit-cell-geom-params} Values of parameters used to define unit-cell geometry in the simulations.}
\begin{tabular}{l|c}
\hline \hline
Cell radius $\hat{r}_\epsilon^{\left(c\right)}$ [\si{L}]  &  0.009  \\
Pipe radius $\hat{r}_\epsilon^{\left(w\right)}$ [\si{L}]  &  0.003  \\
Distance between two battery cells $d^{\left(cc\right)}_\epsilon$ [\si{L}] & 0.009 \\
Distance between battery cell and cooling pipe $d^{\left(1\right)}_\epsilon$ [\si{L}] & 0.001\\
Distance between battery cell and cooling pipe $d^{\left(2\right)}_\epsilon$ [\si{L}] & 0.002 \\
\hline \hline
\end{tabular}%
\end{table}
% The main reason is related to the switching from diffusive to advective behavior, violating the homogeneous assumption and invalidating the upscaled governing equations. Pietrzyk et al.~\cite{Pietrzyk2021-lu} observed the heat front propagate faster in the upscaled than the fine-scale simulations (Figure~\ref{fig:cell_temp_1271_acc}(c) vs (d)). 
In the following simulations, we define an arbitrary simulation end time $t_{final} = 0.2 \hat{t}$ that is sufficient for the temperature to achieve equilibrium, where
\begin{align}
    \hat{t} &= \ddfrac{\hat{\rho}^{\left(p\right)}\hat{C}^{\left(p\right)}\left(\max{(\hat{\mathcal{L}}_{x}, \hat{\mathcal{L}}_{y})}\right)^2}{\hat{k}^{\left(p\right)}},
\end{align}
is the reference timescale for heat transfer. Table~\ref{tab:ref-val-scaling} summarizes the values of the reference parameters used to define the dimensionless numbers in equations~\eqref{eq:dimless_groups} for all subsequent simulations. We further define the following parameters as
\begin{align}
    &\Pi_{\text{base},0} = 0.01\Pi_{\text{burn},0}, \\
    &\hat{U}^{\left(pc\right)} = \ddfrac{\hat{k}^{\left(p\right)}}{\max{(\hat{\mathcal{L}}_{x}, \hat{\mathcal{L}}_{y})}}, \\
    &\hat{Q}^{\left(pw\right)} = 0.00001\ddfrac{\hat{T}_{max} \hat{k}^{\left(p\right)}}{\max{(\hat{\mathcal{L}}_{x}, \hat{\mathcal{L}}_{y})}},
\end{align}
to ensure the applicability of the upscaled equations. The simulations are initialized with zeros for both the packing- and cell-temperature fields. 

% \noindent where $i=p$ or $c$ refers to packing material or the battery cell, repsectively,  $\hat{\rho}^{\left(i\right)}$ [\si{ML\tothe{-3}}] is the density, $\hat{C}^{\left(i\right)}$ [\si{L\tothe{2}T\tothe{-2}\Theta\tothe{-1}}]is the heat capacity, $\hat{T}_{\epsilon}^{\left(i\right)} [\si{\Theta}] \equiv \hat{T}_{\epsilon}^{\left(i\right)}\left(\hat{t}, \hat{\mathbf{x}}\right)$  is the temperature at time $\hat{t} > 0$ and location $\hat{\mathbf{x}} \in \hat{\mathcal{B}}_{\epsilon}^{\left(i\right)}$, $\hat{k}^{\left(i\right)}$ [\si{MLT\tothe{-3}\Theta\tothe{-1}}] is the thermal conductivity, $\n_{\epsilon}^{\left(i\right)} \equiv \n_{\epsilon}^{\left(i\right)}\left(\hat{\mathbf{x}}\right)$ is the normal vector to the interfaces pointed away from the domain, $\hat{U}^{\left(pc\right)}$ [\si{MT\tothe{-3}\Theta\tothe{-1}}] is the total heat transfer coefficient between the packing material and battery cells,  $\hat{q}_{\epsilon}^{\left(pw\right)}\left(\hat{t}, \hat{\mathbf{x}}\right)$ [\si{MT\tothe{-3}}]  is a power flux between the packing material and the cooling water pipes, and $\hat{\Pi}(\hat{t}, \hat{\mathbf{x}})$ [\si{ML\tothe{-1}T\tothe{-3}}] is a power flux source term.

\begin{table}[ht!]
\centering
\caption{
  \label{tab:ref-val-scaling} Values of reference values to nondimensionalize the fine-scale governing equations.}
\begin{tabular}{l|c}
\hline \hline
Density of battery cell $\hat{\rho}^{\left(c\right)}$ [\si{ML\tothe{-3}}]   &  2500  \\
Density of packing materials $\hat{\rho}^{\left(p\right)}$ [\si{ML\tothe{-3}}]   &  1500\\
Heat capacity of battery cell $\hat{C}^{\left(c\right)}$ [\si{L\tothe{2}T\tothe{-2}\Theta\tothe{-1}}]  &  900 \\
Heat capacity of packing materials $\hat{C}^{\left(p\right)}$ [\si{L\tothe{2}T\tothe{-2}\Theta\tothe{-1}}]   & 1500\\
Thermal conductivity of battery cell $\hat{k}^{\left(c\right)}$ [\si{MLT\tothe{-3}\Theta\tothe{-1}}] & 3 \\
Thermal conductivity of packing materials $\hat{k}^{\left(p\right)}$ [\si{MLT\tothe{-3}\Theta\tothe{-1}}] & 3\\
Defined temperature at the battery pack edges $\hat{T}_\infty$ [\si{\Theta}] & 293\\
Temperature ranges over which $\hat{\Pi}(\hat{T}_{\epsilon}^{\left(c\right)},\hat{\mathbf{x}}) = \hat{\Pi}_{\text{base}}(\hat{\mathbf{x}})$, $\hat{T}_a$ [\si{\Theta}] & 0\\
Temperature ranges over which $\hat{\Pi}(\hat{T}_{\epsilon}^{\left(c\right)}, \hat{\mathbf{x}}) = \hat{\Pi}_{\text{burn}}$,  $\hat{T}_b$ [\si{\Theta}] & 0\\
Temperature ranges between $\hat{\Pi}_{\text{base}}(\hat{\mathbf{x}})$ and $\hat{\Pi}_{\text{burn}}$, $\hat{T}_{s1}$ [\si{\Theta}] & 120\\
Temperature ranges between $\hat{\Pi}_{\text{burn}}$ and $0$, $\hat{T}_{s2}$ [\si{\Theta}] & 120\\
Smoothness of the error functions $\epsilon_{s1}$ [-] & 0.0005\\
Smoothness of the error functions $\epsilon_{s2}$ [-] & 0.0005\\
\hline \hline
\end{tabular}%
\end{table}


% {over which} $\hat{\Pi}(\hat{T}_{\epsilon}^{\left(c\right)},\hat{\mathbf{x}}) = \hat{\Pi}_{\text{base}}(\hat{\mathbf{x}})$ and $\hat{\Pi}(\hat{T}_{\epsilon}^{\left(c\right)}, \hat{\mathbf{x}}) = \hat{\Pi}_{\text{burn}}$

% $\hat{\Pi}_{\text{base}}(\hat{\mathbf{x}})$ and $\hat{\Pi}_{\text{burn}}$ are the \textit{base} and \textit{burn} power flux values, respectively, $\hat{T}_{ref}$~[\si{\Theta}] is the reference temperature, $\hat{T}_a$~[\si{\Theta}] and $\hat{T}_b$~[\si{\Theta}] are temperature ranges over with $\hat{\Pi}(\hat{T}_{\epsilon}^{\left(c\right)},\hat{\mathbf{x}}) = \hat{\Pi}_{\text{base}}(\hat{\mathbf{x}})$ and $\hat{\Pi}(\hat{T}_{\epsilon}^{\left(c\right)}, \hat{\mathbf{x}}) = \hat{\Pi}_{\text{burn}}$, respectively, $\hat{T}_{s1}$ and $\hat{T}_{s2}$ are the temperature ranges which $\hat{\Pi}(\hat{T}_{\epsilon}^{\left(c\right)}, \hat{\mathbf{x}})$ transitions from $\hat{\Pi}_{\text{base}}(\hat{\mathbf{x}})$ to $\hat{\Pi}_{\text{burn}}$ and from $\hat{\Pi}_{\text{burn}}$ to $0$, respectively, and $\epsilon_{s1}$~[-]$= 0.0005$ and $\epsilon_{s2}$~[-]$= 0.0005$ are paramters associated with smoothness of the error functions. The detailed formulation and validation of the source term can be found in Pietrzyk et al.~\cite{Pietrzyk2022-ou}.

\subsection{Accuracy of hybrid coupling}
\label{sec:acc-hc}
Following  Pietrzyk et al. (2022), the error between fine-scale and upscaled simulations should be bounded by the theoretical upscaling error $\mathcal{O}\left(\epsilon\right)$ where $\epsilon = 1/\max{(N_{x}^{\left(c\right)}, N_{y}^{\left(c\right)})} = 0.05 \ll 1$ and the magnitudes of dimensionless numbers are within the applicability regime. Since the coupling boundary conditions are derived with an error in $\mathcal{O}(\epsilon)$, hybrid simulation errors are always expected to be within the upscaling error threshold. To validate the accuracy of the developed coupling method, we compare the results of hybrid simulations with fine-scale and upscaled simulations. Here, we define the coupling boundary at $x_{HC}=-0.0875$, which lies between $\mathbf{x}_{\mathcal{R}}=-0.3125$ ($4.5$ unit cells away) and $\mathbf{x}_{\text{burn}}=0.2125$ ($6$ unit cells away). {To automate the detection of $\mathbf{x}_{\mathcal{R}}$ with a distribution of $\mathcal{R}$ (Equation~\eqref{eq:Rx-def}, we compute $x_{\mathcal{R}}$ as}
\begin{align}
    {\mathbf{x}_{\mathcal{R}} =  \argmax_{\mathbf{x}} \left( \left\vert \ddfrac{\mathcal{R}}{\mathcal{R}_{\text{low}}} - 1 - \alpha_1  \right\vert \right),}
\end{align}
{where $\alpha_{1} = 0.01$ is the tolerance factor.} This ensures the validity of the homogeneity assumption at the coupling boundary until the end of the simulations. Table~\ref{tab:val-acc-sim-params} summarizes the parameters used in the simulations.

\begin{table}[ht!]
\centering
\caption{
  \label{tab:val-acc-sim-params} Summary of simulation parameters used in evaluating the accuracy of the proposed hybrid algorithm.}
\begin{tabular}{l|c}
\hline \hline
Time step size $\Delta{t}/\hat{t}$ &  $3.15 \times 10^{-5}$  \\
Upscale subdomain minimum grid resolution $h_{\text{up},min}$  &  $1.00 \times 10^{-2}$\\
fine-scale subdomain minimum grid resolution $h_{\text{fine},min}$  &  $7.63\times10^{-4}$\\
Final simulation time $t_{final}$   & 0.2\\
Coupling boundary location $x_{HC}$ & -0.0875 \\
$\mathcal{R}$ location $\mathbf{x}_{\mathcal{R}}$  & -0.3125 \\
burned location $\mathbf{x}_{\text{burn}}$  & 0.2125 \\
Polynomial order $k$ & 1 \\
\hline \hline
\end{tabular}%
\end{table}
To compare fine-scale and hybrid simulations with upscaled simulations, we apply an averaging operator (equation~\eqref{eq:ave-op}) to the fine-scale and hybrid simulation results. Figures~\ref{fig:cell_temp_032_acc} and~\ref{fig:cell_temp_1271_acc} show the contour plots of the average cell temperatures for fine-scale, hybrid (Taylor and Series expansion methods), and upscaled simulations at $t=0.02$ and $0.20$ respectively. At $t=0.02$, we clearly observe that upscaled simulations (Figure~\ref{fig:cell_temp_032_acc}(d)) underpredict the average cell temperature for $\mathbf{x} \le \mathbf{x}_{\mathcal{R}}$, as indicated by the color difference. No significant differences in the average cell temperature are observed near the burned location $\mathbf{x}_{\text{burn}}$. This shows that the upscaled equations can accurately capture the behavior of heat wave propagation due to burned and unburned cells. However, since the value of $\mathcal{R}$ is greater than the applicability regime for $\mathbf{x} \le \mathbf{x}_{\mathcal{R}}$ ($FT \sim \mathcal{O}(\epsilon)$), the upscaled simulations are unable to capture the rate of increase in cell temperature. In contrast to the upscaled simulations, hybrid simulations (Figure~\ref{fig:cell_temp_032_acc}(b) and (c)) are capable of simulating accurate heat propagation behaviors at both $\mathbf{x}_{\mathcal{R}}$ and $\mathbf{x}_{\text{burn}}$, therefore obtaining cell temperatures that are comparable to simulations at the fine scale. Similar trends can be observed for the packing temperature, where the upscaled simulations overpredict the temperatures (\ref{app:pack-temp-appen}). By comparing the centerline average cell temperature ($y=0$) in Figure~\ref{fig:cell_temp_line_acc}, the hybrid simulations have clearly demonstrated their ability to capture the accurate behavior of the propagation of heat waves compared to upscaled simulations. No obvious difference between Taylor and Series expansion approaches has been observed, as expected, because the accuracy of the upscaled equations is only first order. Therefore, the first-order accurate Taylor approach and the second-order accurate Series approach should not result in significant differences. At $t = 0.2$, the cell temperature approach the equilibrium temperature; therefore, the effect of large $\mathcal{R}$ values is no longer significant. Both the hybrid and upscaled simulations are accurate in predicting the average cell temperature (Figure~\ref{fig:cell_temp_1271_acc} and~\ref{fig:cell_temp_line_acc}(b)). Similar observations on the average temperature of the packing materials can be found in Figure~\ref{fig:cell_temp_1271_acc} of~\ref{app:pack-temp-appen}. 

We now compute the error as 
\begin{align}
\label{eq:err-cal}
    &err = \abs{\ensmean{T^{\left(i\right)}}_Y - \ensmean{T^{\left(i\right)}_\epsilon}_Y},
\end{align}
where $i=c$ or $p$ represent the cell or packing temperature, respectively. Figures~\ref{fig:cell_temp_err_032_acc} and~\ref{fig:cell_temp_err_1271_acc} show the contour plot of the errors at two different time instances. At $t=0.02$ where the battery cells begin to release heat, the errors in the upscaled simulations quickly approach the tolerance, as indicated by the black region for $\mathbf{x} \le \mathbf{x}_{\mathcal{R}}$. On the contrary, the errors in hybrid simulations are still below the tolerance. At $t=0.2$, the errors for both hybrid and upscaled simulations are lower than the threshold $\epsilon=0.05$ because the cell temperature approaches equilibrium. Figure~\ref{fig:cell_temp_err_line_acc} shows the centerline error plot of hybrid and upscaled simulations. The error in the upscaled simulations is more than twice the tolerance. No discernible differences have been observed between the Taylor and Series approaches in the hybrid simulations, as expected. The errors observed at $\mathbf{x}_{\text{burn}}$ are due to the difference in the power flux source term where the function is discontinuous in fine-scale simulations, while continuous in upscaled simulations.

\begin{figure}
\centerline{
 {\includegraphics{figures/Averaged_Cell0032_acc-eps-converted-to.pdf}}}

\caption{Average cell temperature computed at $t=0.02$ with a time step size $\Delta{t}=3.15 \times 10^{-5}$ for (a) fine-scale (b) hybrid with Taylor expansion, (c) hybrid with Series expansion and (d) upscaled simulations.}
\label{fig:cell_temp_032_acc}
\end{figure}

\begin{figure}
\centerline{
 {\includegraphics{figures/Averaged_Cell1271_acc-eps-converted-to.pdf}}}

\caption{Average cell temperature computed at $t=0.2$ with a time step size $\Delta{t}=3.15 \times 10^{-5}$ for (a) fine-scale (b) hybrid with Taylor expansion (c) hybrid with Series expansion and (d) upscaled simulations.}
\label{fig:cell_temp_1271_acc}
\end{figure}

\begin{figure}
\centerline{
 {\includegraphics{figures/Averaged_Cell_Line_acc-eps-converted-to.pdf}}}

\caption{The centerline ($y=0$) average cell temperature computed at (a) $t=0.02$ and (b) $t=0.20$ with a time step size $\Delta{t}=3.15 \times 10^{-5}$ for the fine-scale, hybrid with Taylor and Series expasnion, respectively, and upscaled simulations.}
\label{fig:cell_temp_line_acc}
\end{figure}


\begin{figure}
\centerline{
 {\includegraphics{figures/Averaged_Cell0032_acc_err-eps-converted-to.pdf}}}

\caption{Absolute error of average cell temperature computed at $t=0.02$ with a time step size $\Delta{t}=3.15 \times 10^{-5}$ for (a) fine-scale (b) hybrid with Taylor expansion, (c) hybrid with Series expansion and (d) upscaled simulations.}
\label{fig:cell_temp_err_032_acc}
\end{figure}

\begin{figure}
\centerline{
 {\includegraphics{figures/Averaged_Cell1271_acc_err-eps-converted-to.pdf}}}

\caption{Absolute error of the average cell temperature computed at $t=0.20$ with a time step size $\Delta{t}=3.15 \times 10^{-5}$ for (a) fine-scale (b) hybrid with Taylor expansion, (c) hybrid with Series expansion and (d) upscaled simulations.}
\label{fig:cell_temp_err_1271_acc}
\end{figure}

\begin{figure}
\centerline{
 {\includegraphics{figures/Averaged_Cell_Line_acc_err-eps-converted-to.pdf}}}
\caption{The centerline ($y=0$) absolute error of the average cell temperature computed at (a) $t=0.02$ and (b) $t=0.20$ with a time step size $\Delta{t}=3.15 \times 10^{-5}$ for hybrid with Taylor and Series expasnion, respectively, and upscaled simulations.}
\label{fig:cell_temp_err_line_acc}
\end{figure}

\subsection{Coupling boundary location on the accuracy of hybrid coupling}
\label{sec:xhc-acc}
In the previous section, we evaluated the accuracy of hybrid simulations by setting $x_{HC}$ to be approximately 4.5 unit cells from $\mathbf{x}_{\mathcal{R}}$. This conservative setup ensures the accuracy of hybrid simulations at the expense of computational costs. Here, we define the distance between $x_{HC}$ and ${x}_{\mathcal{R}}$ as 
\begin{align}
    &x_{dist} = \abs{x_{HC} - {x}_{\mathcal{R}}}.
\end{align}
In reality, we would want to minimize $x_{dist}$ to avoid additional computational costs. Theoretically, the minimum distance $x_{dist,\min}$ is $1.0\epsilon$ for the Taylor approach and $1.5\epsilon$ for the Series approach. In this section, we focus on evaluating the effects of $x_{dist}$ on the accuracy of hybrid simulations for both approaches where five different distances ($0\epsilon$, $1.0\epsilon$, $1.5\epsilon$, $3.0\epsilon$, and $4.5\epsilon$) will be used. The simulation parameters are identical to Table~\ref{tab:val-acc-sim-params} except for $x_{HC}$. 

Figures~\ref{fig:cell_temp_032_loc_taylor} and~\ref{fig:cell_temp_1271_loc_taylor} show the effect of $x_{dist}$ on average cell temperature using the Taylor approach, while Figures~\ref{fig:cell_temp_032_loc_series} and~\ref{fig:cell_temp_1271_loc_series} show the effect using the Series approach for $t=0.02$ and $0.20$. For both Taylor and Series approach, the heat fronts of cases except for $x_{dist} = 0.0\epsilon$ are approximately the same as the fine-scale simulations. Similar trends can be clearly observed with the centerline plots in Figures~\ref{fig:cell_temp_line_loc_taylor} and~\ref{fig:cell_temp_line_loc_series}. 

\begin{figure}
\centerline{
 {\includegraphics{figures/Averaged_Cell0032_loc_taylor-eps-converted-to.pdf}}}

\caption{Average cell temperature computed at $t=0.02$ with a time step size $\Delta{t}=3.15 \times 10^{-5}$ for (a) fine-scale and hybrid simulations with Taylor expansion with $x_{dist}=$ (b) $0.0\epsilon$, (c) $1.0\epsilon$, (d) $1.5\epsilon$, (e) $3.0\epsilon$ and (f) $4.5\epsilon$. }
\label{fig:cell_temp_032_loc_taylor}
\end{figure}

\begin{figure}
\centerline{
 {\includegraphics{figures/Averaged_Cell1271_loc_taylor-eps-converted-to.pdf}}}

\caption{Average cell temperature computed at $t=0.20$ with a time step size $\Delta{t}=3.15 \times 10^{-5}$ for (a) fine-scale and hybrid simulations with Taylor expansion with $x_{dist}=$ (b) $0.0\epsilon$, (c) $1.0\epsilon$, (d) $1.5\epsilon$, (e) $3.0\epsilon$ and (f) $4.5\epsilon$.}
\label{fig:cell_temp_1271_loc_taylor}
\end{figure}

\begin{figure}
\centerline{
 {\includegraphics{figures/Averaged_Cell_Line_loc_taylor-eps-converted-to.pdf}}}

\caption{The centerline ($y=0$) average cell temperature computed at (a) $t=0.02$ and (b) $t=0.20$ with a time step size $\Delta{t}=3.15 \times 10^{-5}$ for fine-scale and hybrid simulations with Taylor expansion with $x_{dist}=0.0\epsilon$, $1.0\epsilon$, $1.5\epsilon$,  $3.0\epsilon$ and $4.5\epsilon$.}
\label{fig:cell_temp_line_loc_taylor}
\end{figure}



\begin{figure}
\centerline{
 {\includegraphics{figures/Averaged_Cell0032_loc_series-eps-converted-to.pdf}}}
\caption{Average cell temperature computed at $t=0.02$ with a time step size $\Delta{t}=3.15 \times 10^{-5}$ for (a) fine-scale and hybrid simulations with Series expansion with $x_{dist}=$ (b) $0.0\epsilon$, (c) $1.0\epsilon$, (d) $1.5\epsilon$, (e) $3.0\epsilon$ and (f) $4.5\epsilon$.}
\label{fig:cell_temp_032_loc_series}
\end{figure}

\begin{figure}
\centerline{
 {\includegraphics{figures/Averaged_Cell1271_loc_series-eps-converted-to.pdf}}}
\caption{Average cell temperature computed at $t=0.20$ with a time step size $\Delta{t}=3.15 \times 10^{-5}$ for (a) fine-scale and hybrid simulations with Series expansion with $x_{dist}=$ (b) $0.0\epsilon$, (c) $1.0\epsilon$, (d) $1.5\epsilon$, (e) $3.0\epsilon$ and (f) $4.5\epsilon$.}
\label{fig:cell_temp_1271_loc_series}
\end{figure}

\begin{figure}
\centerline{
 {\includegraphics{figures/Averaged_Cell_Line_loc_series-eps-converted-to.pdf}}}
\caption{The centerline ($y=0$) average cell temperature computed at (a) $t=0.02$ and (b) $t=0.20$ with a time step size $\Delta{t}=3.15 \times 10^{-5}$ for fine-scale and hybrid simulations with Series expansion with $x_{dist}=0.0\epsilon$, $1.0\epsilon$, $1.5\epsilon$,  $3.0\epsilon$ and $4.5\epsilon$.}
\label{fig:cell_temp_line_loc_series}
\end{figure}

Figures~\ref{fig:cell_temp_line_loc_taylor_err} and~\ref{fig:cell_temp_line_loc_series_err} show the centerline errors computed with equation~\eqref{eq:err-cal} as a function of $x$ for different $x_{dist}$. We can clearly observe that the magnitudes of the errors fall below the upscaling errors when $x_{dist}$ is greater than or equal to the theoretical value ($1.0\epsilon$ for Taylor and $1.5\epsilon$ for Series approaches). Surprisingly, the errors for both the Taylor and Series approaches fall below the threshold when $x_{dist}$ is less than the theoretical value. This shows the robustness of the hybrid coupling algorithm developed. However, the errors are not guaranteed to be bounded within threshold when $x_{dist}$ is less than the theoretical value.   

\begin{figure}
\centerline{
 {\includegraphics{figures/Averaged_Cell_Line_loc_taylor_err-eps-converted-to.pdf}}}

\caption{The centerline ($y=0$) absolute error of the average cell temperature computed at (a) $t=0.02$ and (b) $t=0.20$ with a time step size $\Delta{t}=3.15 \times 10^{-5}$ for hybrid simulations with Taylor expansion with $x_{dist}=0.0\epsilon$, $1.0\epsilon$, $1.5\epsilon$,  $3.0\epsilon$ and $4.5\epsilon$.}
\label{fig:cell_temp_line_loc_taylor_err}
\end{figure}

\begin{figure}
\centerline{
 {\includegraphics{figures/Averaged_Cell_Line_loc_series_err-eps-converted-to.pdf}}}

\caption{The centerline ($y=0$) absolute error of the average cell temperature computed at (a) $t=0.02$ and (b) $t=0.20$ with a time step size $\Delta{t}=3.15 \times 10^{-5}$ for hybrid simulations with Series expansion with $x_{dist}=0.0\epsilon$, $1.0\epsilon$, $1.5\epsilon$,  $3.0\epsilon$ and $4.5\epsilon$.}
\label{fig:cell_temp_line_loc_series_err}
\end{figure}

\subsection{Efficiency of hybrid coupling}
\label{sec:efficiency}
In addition to accuracy, another advantage of hybrid simulation is its reduced computational cost compared to fine-scale simulations~\cite{Battiato2011-zo}. In Section~\ref{sec:acc-hc}, we focus on the accuracy and neglect the computational efficiency by making approximately 80\% of the domain as the fine-scale subdomain. In this section, we instead focus on evaluating the reduction in computational cost by varying the fraction of the fine-scale subdomain. We consider two different domain sizes of $N_{x}^{\left(c\right)}=80$ and $N_{y}^{\left(c\right)}=1$ or $10$, representing small and large domains. The reference values of the parameters are identical to our simulations in Section~\ref{sec:acc-hc} (Table~\ref{tab:ref-val-scaling}). Table~\ref{tab:val-efficiency-sim-params} summarizes the simulation parameters we use. To evaluate reduction in computational cost, the location of the coupling boundary $x_{HC}$ varies from -0.475 to 0.475, resulting in a fine-scale subdomain fraction ranges from 97.5\% to 2.5\%. Since the computational cost highly depends on the number of iterations ($N_{iter}$) in the zero-finding algorithm, we use $N_{iter}=2$ for all simulations to ensure a fair comparison. To evaluate computational cost reduction, the speedup factor is defined as
\begin{align}
   & Speedup = \ddfrac{1}{N_t} \ddfrac{\sum_i^{N_t}t_{\text{fine},i}}{\sum_i^{N_t}t_{HC,i}},
\end{align}
where $N_t$ is the number of time steps and $t_{\text{fine},i}$ and $t_{HC,i}$ are the wall clock time at time step $i$ for fine-scale and hybrid simulations, respectively.
\begin{table}[ht!]
\centering
\caption{
  \label{tab:val-efficiency-sim-params} Summary of simulation parameters used in evaluating the efficiency of the proposed hybrid algorithm.}
\begin{tabular}{l|c}
\hline \hline
Time step size $\Delta{t}/\hat{t}$ &  $3.15 \times 10^{-5}$  \\
Number of unit cells in x-direction ($N_x^{\left(c\right)}$) & $80$ \\
Number of unit cells in y-direction ($N_y^{\left(c\right)}$) & $1$ or $10$ \\
Upscale subdomain minimum grid resolution $h_{\text{up},min}$  &  $1.00 \times 10^{-2}$\\
fine-scale subdomain minimum grid resolution $h_{\text{fine},min}$  &  $2.50\times10^{-4}$\\
Number of timesteps   & 50 \\
Coupling boundary location $x_{HC}$ & -0.475 - 0.475 \\
$\mathcal{R}$ location $\mathbf{x}_{\mathcal{R}}$  & -0.3125 \\
burned location $\mathbf{x}_{\text{burn}}$  & 0.2125 \\
Polynomial order $k$ & 1 \\
\hline \hline
\end{tabular}%
\end{table}

\begin{figure}
\centerline{
 {\includegraphics{figures/Efficiency-eps-converted-to.pdf}}}

\caption{The speedup factor as a function of fine-scale subdomain volume fraction for domain of (a) $80\times1$ and (b) $80\times10$.}
\label{fig:efficiency}
\end{figure}

Figure~\ref{fig:efficiency} shows the speed-up factor as a function of fine-scale subdomain fraction for two different setups. For the speedup factor $>$ 1, the hybrid approach is favored over the fine-scale approach. Breakeven points (speedup = 1) are approximately 0.2 and 0.4 for $N_{y}^{\left(c\right)}=1$ and $10$, respectively. As the domain size in the $y$ direction increases from $N_{y}^{\left(c\right)}=1$ to $N_{y}^{\left(c\right)}=10$, the breakeven point of the fine-scale subdomain increases from 0.2 to 0.4, indicating that hybrid simulations are more favorable in larger domains. The difference in computational cost between the Taylor and Series expansion approach is negligible except for the small volume fraction. This is expected because the dominant computational cost is related to the linear solvers for high volume fractions. Therefore, computational costs in calculating the coupling conditions are insignificant. 
