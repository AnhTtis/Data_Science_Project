\section{Introduction} \label{sec_intro}

The premise of homogenization theory is to derive upcaled models of physical systems by exploiting the principle of separation of scales. Rigorous homogenization/coarse-graining methods \cite{Langlo1994-wq, Vasilyeva2019-wh,Das2005-gm,Frippiat2008-cl,Battiato2019-xk, Battiato2011-ad} provide \emph{a priori} guarantees, such as asymptotic bounds, on the deviation of the upscaled model's solution from that of the averaged fine-scale model (Figure~\ref{fig:homogenization-illu}(a) and Figure~\ref{fig:homogenization-illu}(b)). These error bounds remain valid over regions of space and durations of time where/when a set of {\it applicability conditions} hold. However, deriving the upscaled governing equations is tedious and often limits the adoption of the method. Recently, Pietryzk \emph{et al.} \cite{Pietrzyk2021-lu} developed a symbolic upscaling engine, \symbolica, capable of automatically deriving  homogenized partial differential equations (PDEs), as well as their applicability conditions, from  fine-scale PDEs, initial conditions (ICs) and boundary conditions (BCs). \symbolica is capable of handling complex multiscale, multi-physics, heterogeneous, and nonlinear systems of PDEs that surpass humans' ability to handle by cumbersome pen-and-paper analysis. 

Although \emph{a priori} error bounds provide a solid ground for identifying cost-accuracy tradeoffs where/when the applicability conditions are satisfied, the challenge is to deal with situations in which such conditions are violated. This has been shown to happen in presence of, e.g., large spatial gradients of the quantities of interest \cite{Battiato2011-ad, Pietrzyk2021-lu, Boso2013-ag}. % In fact, the most interesting phenomena occur when the applicability conditions are violated due to large gradients, material heterogeneity and aging.
To overcome these challenges, hybrid or algorithmic refinement formulations have been developed \cite{Battiato2011-zo,Yousefzadeh2017-yc}: fine-scale equations are solved only in a sub-domain in which applicability conditions are violated whereas the upscaled equations are solved everywhere else (Figure~\ref{fig:homogenization-illu}(c)). Hybrid simulations can be built through either intrusive or nonintrusive coupling conditions between the two sub-domains (fine-scale and {macro-scale}). Intrusive coupling methods are characterized by the existence of an ``overlapping" or ``handshake" region where both fine-scale and upscaled governing equations are concurrently solved \cite{Battiato2011-zo,Roubinet2013-fg,Pettersson2013-ou}. Intrusive coupling methods are typically more expensive due to the existence of the ``overlapping" or ``handshake" region, and more difficult to implement in legacy codes. In nonintrusive coupling methods, each subdomain is only solved with one set of governing equations, either fine-scale or upscaled, and the coupling between the subdomains is formulated exclusively in terms of boundary conditions \cite{Yousefzadeh2017-yc,Kadeethum2022-ag}. 
%Yousefzadeh and Battiato~\cite{Yousefzadeh2017-yc} developed a nonintrusive method for a one-way coupled reactive porous medium flow. However, this assumption does not hold in two-way coupled systems such as heat transfer in a battery pack. 


% Although the upscaled equations can guarantee bounded error under established conditions, the prediction can deviate significantly from the true behavior when certain conditions are invalidated. To overcome these challenges, multiscale approaches or hybrid formulations have been developed\cite{Yousefzadeh2017-yc,Battiato2011-ad,Tartakovsky2008-mt}. These multiscale approaches generally divide the domain into non-breakdown and breakdown regions that are solved with upscaled and fine-scale equations, respectively. The key idea of multiscale approaches is to derive the coupling conditions between upscaled and fine-scale equations. In general, there are two families of methods, non-intrusive and intrusive coupling. Intrusive coupling methods are characterized by the existence of an ``overlapped" or ``handshake" region where both fine-scale and upscaled governing equations are concurrently solved \cite{Battiato2011-zo,Roubinet2013-fg,Pettersson2013-ou}. Intrusive coupling methods are typically more expensive due to the existence of the ``overlapped" or ``handshake" region. In nonintrusive coupling methods, each subdomain is only solved with one set of governing equations, either fine-scale or upscaled \cite{Yousefzadeh2017-yc,Kadeethum2022-ag}. Yousefzadeh and Battiato~\cite{Yousefzadeh2017-yc} developed a nonintrusive method for a one-way coupled reactive porous medium flow. To date, most multiscale approaches focus on one-way coupling without considering the interactions between the two phases. 

% 

In this paper, we develop a predictive nonintrusive two-way coupled hybrid formulation. The computational domain is represented by two different materials, each of which is governed by a set of governing equations.  We apply this approach to the use-case of thermal runaway simulation in Li-ion batteries (LIBs) because they provide a sufficiently complex example with heterogeneity and nonlinearity in the source term. In Section~\ref{sec:govern-eqs}, we describe the fine-scale and upscaled governing equations and formulations of the coupling boundary conditions with two different approaches. In Section~\ref{sec:therm-res}, we evaluate the efficiency and accuracy of the proposed hybrid formulation for heat transfer in a battery pack as a demonstration. 

% In this study, we develop a predictive nonintrusive two-way coupled hybrid formulation as compared to one-way coupled. The computational domain is represented by two different materials, each of which is governed by a set of governing equations. For demonstration purposes, we use the example of heat transfer in a battery pack. However, the formulation can be easily extended to other two-way coupled systems such as chemical and biological reactions. 

% The onset of thermal runaway in a cell due to mechanical, thermal, and electric abuse can compromise the entire battery pack and lead to an explosion \cite{Feng2018-ix}. Understanding heat transfer in these systems at the relevant scales is critical to optimize design and operation. Nevertheless, the development of accurate heat transfer models in battery systems, ranging from the sub-electrode to the battery pack scale, represents a formidable multiscale task because of the complex interactions between the processes at each scale, such as heat generation at the electrode scale and thermal runaway at the packing/module scale. 

% One common approach to modeling thermal runaway in battery cells is to develop spatially independent models with experimentally determined and calibrated parameters \cite{Ren2018-zj}. However, such parameters are calibrated on the basis of lab-scale systems, can have significant uncertainties and tend to vary from system to system. Therefore, the applicability of these models to large-scale systems is questionable. 

% The alternative is to perform fine-scale simulations that resolve the governing partial differential equations in a defined computational domain \cite{Kong2021-sb, Guo2019-td}. With appropriate mesh resolution and computational methods, heat transfer in the battery pack can be accurately simulated. However, these high-fidelity simulations generally require significant computational costs \cite{Yousefzadeh2023-rg}, and are unlikely to be used as predictive tools for large battery packs that contain thousands of cells.

% Another approach is to develop upscaled models from fine-scale equations based on homogenization/coarse-graining theory. Recently, Pietryzk \emph{et al.} \cite{Pietrzyk2022-am} have generated upscaled equations using the automated upscaling engine, \symbolica, to model heat transfer in battery packs. Yet, when certain applicability conditions are invalidated due to e.g., manufacturing defects and cell aging, the prediction from upscaled models can deviate significantly from the fine-scale behavior. In the following, we develop a hybrid formulation that couples fine-scale equations with upscaled equations when applicatibility conditions of the latter are violated within a small portion of the computational domain. In Section~\ref{sec:govern-eqs}, we present the governing equations at the fine- and continuum-scales for the thermal runaway problem.  



% In addition, the capability to simulate complex geometries also varies from method to method, further limiting its applicability to flow in a complex porous medium. In recent years, high-fidelity simulation results have been used to develop accurate closure models for averaging-based computational methods such as Eulerian-Lagrangian and Eulerian-Eulerian, which are capable of simulating larger domains at lower computational costs~\cite{Akiki2016-pq, Akiki2017-iq}.

% Typical computational methods for high-fidelity simulations of complex geometries are Immersed Boundary Methods (IBM)~\cite{Uhlmann2005-hf, Yao2021-fj, Yao2021-ex, Biegert2017-ku, Kempe2012-pl}, Lattice Boltzmann Methods (LBM)~\cite{Ladd1989-jc} and Smooth Particle Hydrodynamics method (SPH)~\cite{Meakin2009-kk, Gingold1977-kc}. 



% ======================

% One common approach is to develop spatial independent models with experimentally determined and calibrated parameters, such as thermal runaway models in battery cells~\cite{Ren2018-zj} 

%Typical examples of such problems are heat transfer in a battery pack~\cite{Feng2018-ix, Wang2020-nu} and chemical/biological reactions in packed- and fluidized-bed reactors for wastewater treatment~\cite{Yao2021-rg, Shin2021-sq}. Understanding the flow structures and reactions in these systems is critical to developing strategies to optimize design and operation; otherwise, suboptimal systems could result in detrimental failures, such as the explosion of the battery pack in electric vehicles~\cite{Wang2020-nu} and the temporary malfunction of wastewater treatment facilities. One common approach is to develop spatial independent models with experimentally determined and calibrated parameters, such as thermal runaway models in battery cells~\cite{Ren2018-zj} and wastewater treatment efficiency models~\cite{Shin2021-qb}. However, the parameters that are calibrated on the basis of lab-scale systems have significant uncertainties and tend to vary from system to system. Therefore, the applicability of these models to large-scale systems is questionable.

% With recent advances in computational power and algorithms, fine-scale simulations that resolve the governing partial differential equations in a defined computational domain have gained popularity over the years~\cite{Kong2021-sb, Guo2019-td, Uhlmann2005-hf, Yao2021-fj}. In general, with appropriate mesh resolution and computational methods in high-fidelity simulations, chemical/biological reactions, and flow structures in a porous medium can be accurately quantified. Typical computational methods for high-fidelity simulations of complex geometries are Immersed Boundary Methods (IBM)~\cite{Uhlmann2005-hf, Yao2021-fj, Yao2021-ex, Biegert2017-ku, Kempe2012-pl}, Lattice Boltzmann Methods (LBM)~\cite{Ladd1989-jc} and Smooth Particle Hydrodynamics method (SPH)~\cite{Meakin2009-kk, Gingold1977-kc}. However, these high-fidelity simulations generally require significant computational costs such that each particle or spherical pore must be resolved by 15-30 grid cells~\cite{Yao2021-ky, Uhlmann2014-ja}. Typical IBM simulations consist of approximately 1000-3000 particles~\cite{Yao2021-xo, Yao2021-fj, Akiki2016-ek, Uhlmann2014-ja}. Therefore, high-fidelity simulations with current computational power are unlikely to be used as predictive tools for large-scale domains that contain millions of pores and particles. In addition, the capability to simulate complex geometries also varies from method to method, further limiting its applicability to flow in a complex porous medium. In recent years, high-fidelity simulation results have been used to develop accurate closure models for averaging-based computational methods such as Eulerian-Lagrangian and Eulerian-Eulerian, which are capable of simulating larger domains at lower computational costs~\cite{Akiki2016-pq, Akiki2017-iq}.

% To bridge the scale difference in laboratory and industrial systems, various researchers have explored different techniques and methods to derive upscaled governing equations such as 1) empirical closure \cite{Efendiev2003-mi}, 2) upscaling \cite{Langlo1994-wq, Vasilyeva2019-wh,Das2005-gm,Frippiat2008-cl,Battiato2019-xk}, and 3) assuming equivalence between microscopic and macroscopic behaviors \cite{Christie1996-eb}. In the upscaling methods, upscaled governing equations are derived based on homogenization/coarse-graining theory \cite{Battiato2011-ad,Battiato2019-xk}. The main advantage of this approach is that the upscaled governing equations are guaranteed to be bounded by the \emph{a} \emph{priori} errors determined by the homogenization theory. However, deriving the upscaled governing equations is tedious and often limits the adoption of the method. Recently, Pietryzk et al. \cite{Pietrzyk2021-lu,Pietrzyk2022-am} have developed an automated upscaling engine, \symbolica, to generate upscaled equations from fine-scale equations for different applications, ranging from reactive transport in porous medium to heat transfer in battery pack.


%Although the overall error is bounded within the upscaling error, these approximations in the upscaled governing equations can deviate significantly from the true macroscopic behaviors in certain systems due to the strong coupling between processes at different scales and the localized invalidation of the approximation~\cite{Yousefzadeh2017-yc, Battiato2011-ad, Pietrzyk2021-lu}. 


%A heterogeneous system can usually be divided into different subdomains where the upscaled equations are only applicable to a subset of subdomains (non-breakdown regions). Subdomains where the upscaled equations do not apply are considered as breakdown regions. 

% Pietryzk et al. (2022) {[battery paper]} compared upscaled simulations with fine-scale simulations in a battery thermal runaway problem and discovered that upscaled simulations overpredicted the battery cell temperature in the battery pack once the homogeneous assumption was violated.  

% Although the upscaled equations can guarantee bounded error under established conditions, the prediction can deviate significantly from the true behavior when certain conditions are invalidated. To overcome these challenges, multiscale approaches or hybrid formulations have been developed\cite{Yousefzadeh2017-yc,Battiato2011-ad,Tartakovsky2008-mt}. These multiscale approaches generally divide the domain into non-breakdown and breakdown regions that are solved with upscaled and fine-scale equations, respectively. The key idea of multiscale approaches is to derive the coupling conditions between upscaled and fine-scale equations. In general, there are two families of methods, non-intrusive and intrusive coupling. Intrusive coupling methods are characterized by the existence of an ``overlapped" or ``handshake" region where both fine-scale and upscaled governing equations are concurrently solved \cite{Battiato2011-zo,Roubinet2013-fg,Pettersson2013-ou}. Intrusive coupling methods are typically more expensive due to the existence of the ``overlapped" or ``handshake" region. In nonintrusive coupling methods, each subdomain is only solved with one set of governing equations, either fine-scale or upscaled \cite{Yousefzadeh2017-yc,Kadeethum2022-ag}. Yousefzadeh and Battiato~\cite{Yousefzadeh2017-yc} developed a nonintrusive method for a one-way coupled reactive porous medium flow. To date, most multiscale approaches focus on one-way coupling without considering the interactions between the two phases. However, this assumption does not hold in systems such as heat transfer in a battery pack, where heat is constantly generated from the battery cells and transferred to the packing materials. 

