\section{Conclusion}
We develop a two-way coupled non-intrusive hybrid algorithm to model heat transfer in battery pack. The coupling boundary conditions are derived on the basis of heat and flux conservation. To ensure continuity at the coupling boundary, two methods with different orders of accuracy are proposed by using Taylor and Series expansion approaches, respectively. Weak formulations of the governing equations are derived and implemented in FEniCS for solving the equations numerically.  To validate the accuracy of the proposed two-way coupled hybrid algorithm, we simulated a thermal runway problem with a battery pack. The average temperature profiles of hybrid simulations with both Taylor and Series expansion approaches have been compared with fine-scale and upscaled simulations. The errors are quantified as the absolute difference between hybrid/upscaled and fine-scale simulations. At $t=0.02$, hybrid simulations are able to accurately predict cell and packing temperatures (indicated by negligible errors), while upscaled simulations significantly overpredict the results, as indicated by errors that are significantly higher than upscaling errors. At $t=0.20$, hybrid and upscaled simulations are able to accurately predict the temperature of the battery cell and the packing material by limiting the errors below the upscaling error.  In addition to accuracy, the efficiency of the proposed hybrid algorithms has been investigated. We considered two different configurations with $N_{y}^{\left(c\right)} = 1$ and $10$, representing small and large domains. By varying the fraction of the subdomain on the fine scale from 0.975 to 0.025, we have shown that the breakeven points for $N_{y}^{\left(c\right)} = 1$ and $10$ are 0.2 and 0.4, respectively. Hybrid simulations are favored when the fraction of the fine-scale subdomain is below the breakeven point. In  follow-up studies, an adaptive hybrid algorithm, which dynamically tracks the spatio-temporal region in which applicability conditions are violated, will be developed.
