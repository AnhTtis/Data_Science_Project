\section{Practical Approaches for Reliable PTQ}

% Balanced classes in calibration dataset.
% Balanced difficulty in calibration dataset.
% Increase the number of calibration samples.
% Considering the data types that the target hardware support, carefully assign appropriate quantization settings (bitwidth and quantization functions).
% Evaluate the performance of different calibration metrics and select the best one.
% Increase the optimizing search space when the resources is enough.

In the previous section, we have examined how different steps in PTQ can affect the reliability of the resulting quantized network. 
In this section, we put forward some practical approaches for achieving reliable PTQ. 
These approaches encompass a range of strategies aimed at ensuring the accuracy and robustness of the quantized model.

One approach is to balance the classes in the calibration dataset, which can help to ensure that the model is equally accurate across different categories of data. Another approach is to balance the difficulty of samples in the calibration dataset, which can help to ensure that the model is equally accurate across different levels of difficulty.

Another approach to improve the reliability of PTQ is to increase the number of calibration samples. This can help to improve the accuracy of the quantization parameters by providing a larger and more representative sample of data for calibration. In addition, it is important to carefully consider the data types that the target hardware supports and to assign appropriate quantization settings (bitwidth and quantization functions) accordingly. This can help to ensure that the quantized model is optimized for the target hardware and can run efficiently without sacrificing accuracy.

Furthermore, it is important to evaluate the performance of different calibration metrics and select the best one. This can involve comparing different metrics such as MSE, cosine distance, and KL distance, and selecting the one that is most suitable for the specific task and dataset.

Finally, if resources allow, increasing the search space for optimization can improve the reliability of PTQ. This can involve using more advanced optimization techniques such as grid search, evolutionary algorithms, or gradient-based methods, and exploring a larger parameter space to find the optimal quantization parameters.

\subsection{Experiments}

ImageNet

Object detection