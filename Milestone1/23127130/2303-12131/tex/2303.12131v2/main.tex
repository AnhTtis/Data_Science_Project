\documentclass[preprint,12pt]{elsarticle}

\usepackage{amssymb}

\usepackage[utf8]{inputenc}
\usepackage{hyperref}
\usepackage[nottoc,numbib]{tocbibind}
\usepackage{setspace}
\usepackage{booktabs}

\usepackage{listings}
\usepackage{xcolor}

\usepackage{caption}
\usepackage{subcaption}

\usepackage{amsmath}

\definecolor{codegreen}{rgb}{0,0.6,0}
\definecolor{codegray}{rgb}{0.5,0.5,0.5}
\definecolor{codepurple}{rgb}{0.58,0,0.82}
\definecolor{backcolour}{rgb}{0.95,0.95,0.92}

\newcommand{\qpligature}{\ensuremath{\reflectbox{\text{P}}\mkern-6mu\text{P}}}

\lstdefinestyle{mystyle}{
    backgroundcolor=\color{backcolour},   
    commentstyle=\color{codegreen},
    keywordstyle=\color{blue},
    numberstyle=\tiny\color{codegray},
    stringstyle=\color{codepurple},
    basicstyle=\ttfamily\footnotesize,
    breakatwhitespace=false,         
    breaklines=true,                 
    captionpos=b,                    
    keepspaces=true,                 
    numbers=left,                    
    numbersep=5pt,                  
    showspaces=false,                
    showstringspaces=false,
    showtabs=false,                  
    tabsize=2
}
\lstset{style=mystyle}

\lstdefinelanguage{Ini}{
    basicstyle=\ttfamily\small,
    columns=fullflexible,
    morecomment=[s][\color{codegreen}\bfseries]{[}{]},
    morecomment=[l]{\#},
    morecomment=[l]{;},
    commentstyle=\color{gray}\ttfamily,
    morekeywords={},
    otherkeywords={=,:},
    keywordstyle={\color{green}\bfseries}
}

\journal{Comp. Phys. Comm.}

%\usepackage{draftwatermark}
%\SetWatermarkText{DRAFT}
%\SetWatermarkScale{5}

\begin{document}

\begin{frontmatter}
\title{Hermes-3: Multi-component plasma simulations with BOUT++}

\author[llnl]{Ben Dudson}
\author[york]{Mike Kryjak}
\author[york,ccfe]{Hasan Muhammed}
\author[york]{Peter Hill}
\author[ccfe]{John Omotani}

\affiliation[llnl]{organization={Lawrence Livermore National Laboratory},
            addressline={7000 East Avenue},
            city={Livermore},
            postcode={94550},
            state={CA},
            country={USA}}

\affiliation[york]{organization={School of Physics, Engineering and Technology, University of York},
            addressline={Heslington},
            city={York},
            postcode={YO10 5DD},
            country={UK}}

\affiliation[ccfe]{organization={United Kingdom Atomic Energy Authority},
            addressline={Culham Centre for Fusion Energy, Culham Science Centre},
            city={Abingdon},
            postcode={OX14 3DB},
            country={UK}}

\begin{abstract}
  A new open source tool for fluid simulation of multi-component
  plasmas is presented, based on a flexible software design that is
  applicable to scientific simulations in a wide range of fields. This
  design enables the same code to be configured at run-time to solve
  systems of partial differential equations in 1D, 2D or 3D, either
  for transport (steady-state) or turbulent (time-evolving) problems,
  with an arbitrary number of ion and neutral species.

  To demonstrate the capabilities of this tool, applications relevant
  to the boundary of tokamak plasmas are presented: 1D simulations of
  diveror plasmas evolving equations for all charge states of neon and
  deuterium; 2D transport simulations of tokamak equilibria in
  single-null X-point geometry with plasma ion and neutral atom
  species; and simulations of the time-dependent propagation of plasma
  filaments (blobs).

  Hermes-3 is publicly available on Github under the GPL-3 open source
  license.  The repository includes documentation and a suite of unit,
  integrated and convergence tests.
\end{abstract}

\begin{keyword}
%% keywords here, in the form: keyword \sep keyword

plasma \sep simulation \sep tokamak \sep BOUT++

%% PACS codes here, in the form: \PACS code \sep code

%% MSC codes here, in the form: \MSC code \sep code
%% or \MSC[2008] code \sep code (2000 is the default)

\end{keyword}

\end{frontmatter}

{\bf PROGRAM SUMMARY}

\begin{small}
\noindent
{\em Program Title:} Hermes-3  \\
{\em CPC Library link to program files:}  \\
{\em Developer's repository link:} https://github.com/bendudson/hermes-3 \\
{\em Code Ocean capsule:} \\
{\em Licensing provisions:} GPLv3 \\
{\em Programming language:} C++14 \\
{\em Nature of problem:}
Simulation of dynamics and steady-state solutions of multiple ion and neutral species
in magnetised plasmas using drift-reduced fluid plasma models in 1 to 3 spatial dimensions. 
Focused on but not restricted to the simulation of turbulence in the boundary of tokamaks. \\
{\em Solution method:}
Hermes-3 implements a system of flexible software components, that are configured
at run-time and coupled by passing a state object between them (the Encapsulate State
and Command patterns). Hermes-3 builds on BOUT++ [1], employing the Method of Lines
with implicit and explicit time-integration methods in curvilinear coordinates on block-structured
meshes, making use of libraries including SUNDIALS, PETSc and Hypre.
Hermes-3 implements drift-reduced plasma fluid equations using conservative finite difference methods,
and atomic processes that couple plasma and neutral fluids.
\begin{thebibliography}{0}
\bibitem{1} B.D.Dudson et al. CPC 180, 1467-1480 (2009)
\end{thebibliography}
\end{small}

\section{Introduction}
\label{sec:introduction}

An important feature of magnetically confined fusion plasmas in
tokamaks is that they consist of a mixture of different ion
species. Computational studies of fusion plasma phenomena, such as
plasma turbulence, have to date frequently treated plasmas as if they
consisted of a single ion species, typically deuterium due to its use
in experiments. In order to make predictions of phenomena important to
the performance and design of the divertor of future fusion reactors,
models must capture the interactions between plasma and neutral gas
species (deuterium and tritium in a reactor), as well as radiation
from impurity species (eg. Be, C, Ne, Ar, W), and pumping of helium
``ash'' and other species.  In many cases none of these species can be
treated as ``trace'' but are all coupled, so that all should be
considered self-consistently in the same simulation. For each of these
atomic species the density and dynamics of multiple states may need
to be considered, for example the ground state, multiple ionisation
stages, and molecular species. In some cases it may also be necessary
to track metastable states, such as vibrationally excited states or
charged molecules. The result is a potentially large number of species
types which must be solved for, together with a complex set of
reactions between them.

Historically there has been a division in terms of simulation tools
and to an extent also research communities, between multi-fluid
transport codes such as SOLPS~\cite{schneider-2006},
UEDGE~\cite{rognlien-2002}, EDGE2D~\cite{simonini-1994} and
BOUT++/trans-neut~\cite{wang2014}, which employ simplified models for
the cross-field transport (typically diffusive) but evolve many
different species, and the turbulence codes including
GBS~\cite{giacomin2021,halpern2016,ricci2012},
TOKAM3X~\cite{tamain2010,tamain2016}, (H)ESEL~\cite{madsen2016a}, and
various models built on
BOUT++~\cite{Dudson2009,dudson2015,bout:manual} such as
Hermes~\cite{Dudson2017} and STORM~\cite{easy2014,ukaea:storm}. These
latter models can solve for the 3D time-varying turbulent transport,
but typically only evolve a single ion species. During the last 5
years, and currently ongoing, are several efforts
(e.g.~\cite{BUFFERAND201982, coroado2021, shrish2021}) to develop codes which can solve for the
turbulent transport self consistently with multiple ion and neutral
gas species~\cite{LEDDY2017994}.

If the number of species states to be solved for is relatively small,
such as electrons, ions and a single atomic species as in
Hermes-2~\cite{dudson:hermes2} and recent versions of
STORM~\cite{ukaea:storm}, then code can be written for each
species. Unfortunately with this design the size of the code (number
of lines) grows linearly with the number of species states, becoming
increasingly error prone, difficult to test, and hard to maintain, as
the number of species is increased. The authors' experience with
Hermes-2 indicated that this was not a viable path to multi-species
simulations.

Here an open-source multi-fluid plasma simulation tool is presented,
along with the flexible software design which makes the resulting
model and software complexity manageable.  It is built on BOUT++,
which provides low-level data management and operations, and augments
this with a reusable model component system. By doing this we unify 1D
tokamak divertor simulation code SD1D~\cite{dudson2019} with 2D and 3D
transport and turbulence code Hermes~\cite{Dudson2017}, and enable
these tools to be extended towards multiple ion species simulations
that self-consistently include both plasma turbulence and transport
(atomic) physics.

In section~\ref{sec:numerical} the numerical methods are described; in
section~\ref{sec:software} the software architecture is outlined and
compared to prior work; and in section~\ref{sec:applications} a series
of applications are presented, which demonstrate some of the
capabilities of the new code. We conclude in section~\ref{sec:conclusions}.

In this section, we apply the Riccati-based methodology presented in Section~\ref{sec:method} to four test problems from machine learning to demonstrate the versatility and potential computational advantages of our new approach.  
In each example, we use RK4 with double precision to solve the Riccati ODEs when applying our Riccati-based methodology. The connections between the examples presented in this section, the Hopf formula, and  the corresponding optimal control problems can be found in Table~\ref{tab:learning_problems}. 
We note that in this work we use two metrics to evaluate our results quantitatively: the $\ell_1$-norm (defined as $\|x\|_1 = \sum_{i=1}^n |x_i|$ for $x\in\Rn$) for the finite-dimensional minimizer and the $L^2$-norm (defined as $\|f\|_2 = \left(\int_{x\in\Rn} |f(x)|^2 dx\right)^{1/2}$ for $f:\Rn\to \R$) for the inferred functions. The $L^2$-norm for functions is approximated using trapezoidal rule with a uniform grid.
Supplementary details of the numerical experiments can be found in Appendix \ref{sec:details}. Code for all examples will be made publicly available at \url{https://github.com/ZongrenZou/HJPDE4SciML}.

\section{Software Description}
\label{sec: software description}
To execute analyses with Saihu, we roughly divide the tasks into 3 parts: describe a network to be analyzed; execute analyses with individual tools; and export reports back to the user. We will go through these 3 parts one by one.

\subsection{Network Description File}
\label{sec: network description file}
As mentioned in Section~\ref{sec: system model}, Saihu allows the user to write a network in either a \textit{physical network} or an \textit{output port network} format. While xTFA takes a physical network as an XML file and the others take an output port network as a JSON file, one can choose the format they prefer to define a network as Saihu automatically converts a file when needed. 

\subsubsection{Option 1: Defining Physical Network in XML}
\label{sec: physical network xml}
A physical network is written as an XML file according to the xTFA specification. It should at least contains 4 kinds of information: \textbf{General network information}, \textbf{Servers}, \textbf{Links}, and \textbf{Flows}.

Let's take the implementation of Figure \ref{fig: physical network} as an example. First, every entry should be enclosed in one element \texttt{<elements>} as shown in Listing~\ref{lst: xml elements}.

\begin{lstlisting}[language=XML,caption={Examples of XML file. All network entries must inside an element \texttt{<elements>}.},
label={lst: xml elements}]
<?xml version="1.0" encoding="UTF-8"?>
<elements>
    <!-- All entries -->
</elements>
\end{lstlisting}

A physical network must have exactly one \texttt{network} element to define general information across the network as its attributes, an example is shown in Listing~\ref{lst: xml network}. In this example, \texttt{name} is the name of the network, \texttt{technology} is a series of analysis parameters concatenated by the plus sign, and a default value of \texttt{minimum-packet-size}. 

\begin{lstlisting}[language=XML,caption={Example of general network information. Contains \texttt{name}, \texttt{technology} used, and default values for other elements.},label={lst: xml network}]
<network name="demo" technology="FIFO+IS" 
    minimum-packet-size="4B"/>
\end{lstlisting}

The attribute \texttt{technology} takes the following values. More may be found from~\cite{thoma2022analyse}.
\begin{itemize}[leftmargin=1em]
    \item \texttt{FIFO}: FIFO multiplexing. It can be \texttt{ARBITRARY} for arbitrary multiplexing or left blank for tool default.
    \item \texttt{IS}: Input shaping. Consider the shaping effect.
    \item \texttt{PK}: Packetizer.
    \item \texttt{CEIL}: Fix precision when calculating network with cyclic dependency (used only in xTFA.)
\end{itemize}
One can also define some default values that possibly appear in other elements. For example, a \texttt{minimum-packet-size} is usually defined as an attribute of a \texttt{flow} element, while this value defined in the \texttt{network} element will be used as the default value if it's not defined in a \texttt{flow} element.

Second, the servers of the network can be defined as either a \texttt{station} or a \texttt{switch}, as shown in Listing~\ref{lst: xml server}. Although they are very different physically, in our tools they both mean data processing units or possible sources/sinks of a data flow.

\begin{lstlisting}[language=XML,caption={Stations and switches. Name and possibly the default values to all its ports.},label={lst: xml server}]
<station name="src0"/>
<station name="src1"/>
<station name="src2"/>
<switch name="s0" service-latency="10us"
    service-rate="4Mbps"/>
<switch name="s1" service-latency="10us" 
    service-rate="4Mbps"/>
<station name="sink0"/>
<station name="sink1"/>
\end{lstlisting}

Both a station and a switch represent a physical node. The name of each node will be used to define flow paths and links. The \texttt{service-latency} and \texttt{service-rate} define a rate-latency service curve. The service parameters defined at this level serve as default values for all the links attached as outputs of this node.

Third, one must connect physical nodes with \texttt{link}s, as shown in Listing \ref{lst: xml link}. Saihu considers output ports as processing units, so the physical link \texttt{from} a physical node \texttt{to} another node has to be defined, along with the input/output ports used by the link. For example, the link \texttt{lk:s0-s1} connects from the output port \texttt{o0} of switch \texttt{s0} to the input port \texttt{i0} of switch \texttt{s1}.

\begin{lstlisting}[language=XML,caption={Links connecting ports.},label={lst: xml link}]
<link name="lk:src0-s0" from="src0" to="s0" 
    fromPort="o0" toPort="i0"/>
<link name="lk:src1-s0" from="src1" to="s0"
    fromPort="o0" toPort="i1"/>
<link name="lk:src2-s0" from="src2" to="s1" 
    fromPort="o0" toPort="i1"/>
<link name="lk:s0-s1" from="s0" to="s1"
    fromPort="o0" toPort="i0" 
    transmission-capacity="10Mbps"/>
<link name="lk:s1-sink0" from="s1" to="sink0" 
    fromPort="o0" toPort="i0" 
    transmission-capacity="10Mbps"/>
<link name="lk:s1-sink1" from="s1" to="sink1" 
    fromPort="o1" toPort="i0" 
    transmission-capacity="10Mbps"/>
\end{lstlisting}

If the service of an output port needs to be considered in an analysis, one must define the service curve at the link that is directly attached to the output port. The \texttt{transmission-capacity} of the link can also be specified to consider line shaping. If no values are defined, the system tries to apply the default values defined at the upper levels, i.e. \texttt{switch/station} and \texttt{network}. Furthermore, if no values are found across all levels, the link is considered a dummy one and the output port attached to it will not be considered.

Finally, one must define \texttt{flow}s for the network as shown in Listing \ref{lst: xml flow}. Each flow is defined by a \texttt{flow} element. The paths of a flow are defined by \texttt{target} elements, where each node it traverses is listed as \texttt{path} elements with its \texttt{node} attribute indicating the name of the physical node. In this format, multicast of a flow is possible by defining multiple \texttt{target} elements within the same flow.

\begin{lstlisting}[language=XML,caption={Flows. Must have a name and its arrival curve parameters along with its paths.},label={lst: xml flow}]
<flow name="f0" arrival-curve="leaky-bucket" 
    lb-burst="10B" lb-rate="10kbps" 
    maximum-packet-size="50B" source="src0">
    <target>
        <path node="s0"/>
        <path node="s1"/>
        <path node="sink0"/>
    </target>
</flow>
<flow name="f1" arrival-curve="leaky-bucket" 
    lb-burst="10B" lb-rate="10kbps" 
    maximum-packet-size="50B" source="src1">
    <target>
        <path node="s0"/>
        <path node="s1"/>
        <path node="sink1"/>
    </target>
</flow>
<flow name="f2" arrival-curve="leaky-bucket" 
    lb-burst="10B" lb-rate="10kbps" 
    maximum-packet-size="50B" source="src2">
    <target>
        <path node="s1"/>
        <path node="sink0"/>
    </target>
</flow>
\end{lstlisting}

A flow element must have \texttt{name} and the arrival curve specified as its attributes. The keywords \texttt{arrival-curve}, \texttt{lb-burst} and \texttt{lb-rate} define a leaky-bucket curve at the source of the flow. Other parameters like the \texttt{maximum-packet-size} and \texttt{minimum-packet-size} can be also defined to consider packetization. Furthermore, as the definition represents a physical network, each flow must have a data \texttt{source} that is an actual physical node. All the output ports involved in its path, including the output port of the source, will be analyzed by Saihu.


\subsubsection{Option 2: Defining Output Port Network in JSON}
While the XML file syntax is provided by xTFA, we design this JSON format ourselves in order to write an output port network in a concise way.
The file should at least contains 3 kinds of information: \textbf{General network information}, \textbf{Servers}, and \textbf{Flows}. Let's take the implementation of Figure \ref{fig: output port network} as an example. First, all entries must be enclosed as a single JSON object (one \{\} to enclose all attributes.)

A \texttt{network} object is required to define general network information but only the \texttt{name} attribute is necessary. An example is shown in Listing \ref{lst: json network}.


\begin{lstlisting}[language=json,caption={Network information. Contains some general information and default values or units used throughout the file.},label={lst: json network}]
"network": {
    "name": "demo",
    "packetizer": false,
    "multiplexing": "FIFO",
    "analysis_option": ["IS"],
    "time_unit": "us",
    "data_unit": "B",
    "rate_unit": "Mbps",
    "min_packet_length": "4B"
}
\end{lstlisting}
\lstsetblack

The 3 keywords \texttt{packetizer}, \texttt{multiplexing}, and \texttt{analysis\_option} are unique to the \texttt{network} object. \texttt{packetizer} is equivalent to the keyword \texttt{PK} in XML file; \texttt{multiplexing} can be either \texttt{FIFO} or \texttt{ARBITRARY}; and \texttt{analysis\_option} takes other keywords defined in \texttt{technology} mentioned in Section \ref{sec: physical network xml}.

Except for the network options, default values for servers and flows can also be defined at the network level. In the above example, we set the default time/data/rate units to be microsecond/byte/megabits-per-second across the file as well as the minimum packet length being 4 bytes.

Second, we need to define the \texttt{servers} for the network. Some may argue the term \textit{server} instead of \textit{output port} as we discussed in Section~\ref{sec: output port network}. The term \textit{server} is a general term for a processing unit, and one can treat it as a black box that provides service.

The \texttt{servers} is presented as a JSON array, each object in this array is a server. Each server must at least have a \texttt{name}, and its service curve can be missing only when there exists a default value in \texttt{network} attribute. 
The parameters can be expressed in either a \textit{string} or a \textit{number}. A string is written as a number followed by a unit. For example, \texttt{"10us"} means 10 microseconds, and \texttt{"50Mbps"} means 50 megabits per second. If it's directly written as a number, the unit defined in the closest level is used. For example, the time unit defined in server \texttt{s1-o0} is microsecond, so the latency 10 is read as 10 microseconds. 

The object \texttt{service\_curve} takes multiple rate-latency curves and uses the maximum among all these curves as its service curve. Rates and latencies are written as arrays, each pair of rate and latency values with the same index is a rate-latency curve. For example, in server \texttt{s0-o0}, the service curve has 2 segments defined by 2 rate-latency curves, one with a latency of 10 microseconds and a rate of 4 megabits per second, and the other with a latency of 1000 microseconds and a rate of 50 megabits per second.

Notice that in an output port network definition, we don't manually define links. The topology of the network is considered to be the \textit{graph induced by flows}, i.e. a connection from server \textit{A} to \textit{B} exists only when there is at least one flow travels through \textit{B} from \textit{A}. Therefore, the transmission capacity of the link attached to an output port is directly defined on a server with the keyword \texttt{capacity}.

\begin{lstlisting}[language=json,caption={Servers. A list that contains many servers, each with name and service parameters.},label={lst: json server}]
"servers": [
    {
        "name": "s0-o0",
        "service_curve": {
            "latencies": ["10us", 1000],
            "rates": [4, "50Mbps"]
        },
        "capacity": 100
    },
    {
        "name": "s1-o0",
        "service_curve": {
            "latencies": [10, "1ms"],
            "rates": [4, 50]
        },
        "capacity": 100,
        "time_unit": "us"
    },
    {
        "name": "s1-o1",
        "service_curve": {
            "latencies": [10],
            "rates": ["4Mbps"]
        },
        "capacity": 100
    }
]
\end{lstlisting}
\lstsetblack

Finally, the \texttt{flows} are defined in a similar manner as servers as shown in Listing~\ref{lst: json flow}. Each object must have at least a \texttt{name} and a \texttt{path}. A path is represented as an array of server names, and the order in the list represents the sequence that the flow visits.

The representation of values and units is the same as servers, either being a string of a number with units, or a number that uses the default unit.
The arrival curve at the source of a flow is defined by multiple token-bucket curves and taken as the minimum among all these curves. Similar to the service curve of a server, each pair of a burst and a rate value represent a token-bucket curve. For example, the arrival curve of \texttt{f0} has one token-bucket curve of burst 10 bytes and rate 10 kilobits per second, and the other curve of burst 2 kilobytes and rate 0.5 megabits per second.

\begin{lstlisting}[language=json,caption={Flows. A list contains many flows. Each flow contains name, path, and parameters of the arrival data.},label={lst: json flow}]
"flows": [
    {
        "name": "f0",
        "path": ["s0-o0", "s1-o0"],
        "arrival_curve": {
            "bursts": [10, "2kB"],
            "rates": ["10kbps", 0.5]
        },
        "max_packet_length": 50,
        "rate_unit": "kbps"
    },
    {
        "name": "f1",
        "path": ["s0-o0", "s1-o1"],
        "arrival_curve": {
            "bursts": ["10B"],
            "rates": ["10kbps"]
        },
        "max_packet_length": 50
    },
    {
        "name": "f2",
        "path": ["s1-o0"],
        "arrival_curve": {
            "bursts": [10],
            "rates": ["10kbps"]
        },
        "max_packet_length": "50B",
        "min_packet_length": "4B"
    }
]
\end{lstlisting}
\lstsetblack

\subsection{Tool Usage}
In this section, we briefly introduce how to use Saihu to execute analyses. One would only need to import one file, i.e. \textit{interface.py}, to use Saihu given that the project is installed correctly. 
The simplest way to use Saihu is shown in Listing~\ref{lst: simple example}. Once a network description file is available as either an XML or a JSON file, one can execute the following example to do the analysis.

\begin{lstlisting}[style=pythonstyling,caption={Simple example to use Saihu.},label={lst: simple example}]
from interface import TSN_Analyzer
analyzer = TSN_Analyzer("demo.json")
analyzer.analyze_all()
analyzer.export("demo")
\end{lstlisting}

The basic procedure to use Saihu is as follows: 1. initialize the analyzer with a target network description file; 2. execute the analysis with some tools; 3. export the results into reports. 

The supported tools and methods are listed in Figure~\ref{fig: supported methods}. 
To switch between different tools, one uses different functions with names like \texttt{analyze\_xxx}, where \texttt{analyze\_all} means to use all the available tools. To switch methods, one gives different input arguments to each analysis function. An example is provided in Listing~\ref{lst: switch tool and method}. One can execute multiple analyses and all the results will be stored in the internal buffer of the analyzer until the analyzer exports them into reports. 
% The default setting is to try to execute \textit{TFA, SFA} and \textit{PLP}.

\begin{lstlisting}[style=pythonstyling,caption={Execute different tools and methods.},label={lst: switch tool and method}]
analyzer.analyze_dnc()
analyzer.analyze_xtfa("TFA")
analyzer.analyze_panco(methods=["SFA", "PLP"])
\end{lstlisting}


\subsection{Analysis Reports}
\label{sec: analysis reports}

\begin{figure*}[tbh]
\centering
\begin{subfigure}[b]{0.3\textwidth}
    \centering
    \includegraphics[width=\linewidth]{e2e_delay.png}
    \caption{Flow end-to-end delay}
    \label{fig: e2e delay}
\end{subfigure}
\hfill
\begin{subfigure}[b]{0.3\textwidth}
    \centering
    \includegraphics[width=\linewidth]{server_delay.png}
    \caption{Server delay}
    \label{fig: server delay}
\end{subfigure}
\hfill
\begin{subfigure}[b]{0.3\textwidth}
    \centering
    \includegraphics[width=0.9\linewidth]{exec_time.png}
    \caption{Execution time}
    \label{fig: exec time}
\end{subfigure}
\hfill
\begin{subfigure}[b]{0.3\textwidth}
    \centering
    \includegraphics[width=0.8\linewidth]{report_topo.png}
    \caption{Network topology}
    \label{fig: network topology}
\end{subfigure}
\hfill
\begin{subfigure}[b]{0.3\textwidth}
    \centering
    \includegraphics[width=0.6\linewidth]{report_path.png}
    \caption{Flow paths}
    \label{fig: flow path}
\end{subfigure}
\hfill
\begin{subfigure}[b]{0.3\textwidth}
    \centering
    \includegraphics[width=0.9\linewidth]{report_util.png}
    \caption{Link utilization}
    \label{fig: link utilization}
\end{subfigure}
\caption{Human-friendly report. It is written as a Markdown file. The analysis results include the flow end-to-end delay, server delay, and execution time. The values are listed in tables for each tool and method as shown in (a)(b)(c). The units are adjusted accordingly. It also contains input information like the network topology, paths of flows, and link utilization.}
\label{fig: human friendly report}
\end{figure*}

Saihu can generate 2 reports, a \textit{human-friendly report} and a \textit{machine-friendly report}. A human-friendly report is written as a Markdown file that lists all the essential information. An example is shown in Figure~\ref{fig: human friendly report}. The analysis results are listed in 3 sections: per-flow end-to-end delay, per-server delay, and execution time. The delay bounds are presented in tables where each row is a flow or a server, and each column is a method executed by a tool. The last column contains the minimum result obtained in the current round of analysis. The execution time of each method by each tool is also listed for comparison.

Other than the analysis results, the report also contains some information about the user inputs, but in a more formatted manner. They are network topology, flow paths, and link utilization. Network topology is shown as a graph induced by flows. i.e. It's a directed graph where each node is an output port, and an edge from node A to B exists if there's at least one flow traversing from A to B. Flow paths are the same as the network description file, it can serve as a reassurance of user's input. Link utilization is computed node-wise, it is defined as the ratio between the aggregated arrival rate at a node and its service rate. e.g. If 2 flows have arrival rates of $2kbps$ and $3kbps$ respectively filling into a node, which has a service rate of $10kbps$. The link utilization at the node is therefore $(2+3)/10=0.5$.

A \textit{machine-friendly report} stores only the execution outputs, namely the per-flow end-to-end delay, per-server delay, and execution time. It's written in JSON format for easy parsing from other programs. An example is presented in Listing \ref{lst: machine friendly report}.
\begin{lstlisting}[language=json,caption={Machine-friendly report. The flow end-to-end delays, server delays, and execution time are listed in pure numbers. The units these numbers use are also listed as one entry.},label={lst: machine friendly report}]
{
    "name": "demo",
    "flow_e2e_delay": {
        "f0": {
            "Panco_TFA": 100.12500000000001,
            "Panco_PLP": 80.05,
            "DNC_TFA": 100.00375,
            "xTFA_TFA": 99.32394489448944
        },
...
    "server_delay": {
        "s0-o0": {
            "Panco_TFA": 50.0,
            "DNC_TFA": 50.0,
            "xTFA_TFA": 50.0
        },
...
    "execution_time": {
        "Panco_TFA": 62.70909309387207,
        "Panco_PLP": 243.81709098815918,
        "DNC_TFA": 32.0,
        "xTFA_TFA": 81.46500587463379
    },
    "units": {
        "flow_delay": "us",
        "server_delay": "us",
        "execution_time": "ms"
    }
}
\end{lstlisting}
\lstsetblack

In order to let users parse the information easily, Saihu prints the results in numbers, accompanied by the units used in each section. Note that the human-friendly report always contains only 3 decimal digits according to the smallest value in the table, while there's no such rounding for the machine-friendly report. As a result, one should read the machine-friendly report if they require very precise results.


\subsection{Network Generation}
\label{sec: network generation}
Saihu provides a series of functions to allow users to generate certain types of networks into a network description file. Currently, Saihu supports the generation of interleave tandem, mesh, and ring network. They contain specific topologies and routing rules for the flow paths. Users have the freedom to choose the size of the network (number of servers), the service parameters of servers, and the flow parameters of data arrival. The way these parameters are used within each type of network is specified in the respective sections.

Other than the predetermined routing rules, Saihu also provides a function to generate an arbitrary number of flows with random routing. This is particularly suitable for testing the possible traffic with a specific topology. More details will be shown in Section~\ref{sec: fixed topology random}.

\subsubsection{Interleave Tandem Network}
Suppose we wish to generate a network of $n$ servers, indexed from $0$ to $n-1$.
An interleave tandem network has all its servers chained in a line. One flow $f_0$ goes through all servers from $s_0$ to $s_{n-1}$.  The flow $f_i$ is $s_{i-1} \rightarrow s_{i}$ for $i \in [1,n-1]$. Illustrated by Figure \ref{fig: interleave}. All the flows have identical arrival curves and maximum packet length at the source, defined by function arguments \texttt{burst}, \texttt{arrival\_rate}, and \texttt{max\_packet\_length}. Likewise, All the servers have identical service curves and transmission capacity, defined by \texttt{latency}, \texttt{service\_rate}, and \texttt{capacity}.

\begin{figure}
    \centering
    \includegraphics[width=\linewidth]{images/interleave.png}
    \caption{Interleave tandem network}
    \label{fig: interleave}
\end{figure}

\subsubsection{Ring Network}
A ring network is illustrated in Figure \ref{fig: ring}. There are $n$ flows and $n$ servers. The path of flow $i$ is $s_i \rightarrow s_{i+1} \rightarrow \cdots \rightarrow s_{i+n-1\mod n}$ for $0 \leq i \leq n-1$. A ring network is completely symmetrical with all its flows and servers being identical. All flows are defined by \texttt{burst}, \texttt{arrival\_rate}, and \texttt{max\_packet\_length}. Similarly, all servers are defined by \texttt{latency}, \texttt{service\_rate}, and \texttt{capacity}.

\begin{figure}
    \centering
    \includegraphics[width=0.6\linewidth]{images/ring.png}
    \caption{Ring network}
    \label{fig: ring}
\end{figure}

\subsubsection{Mesh Network}
A mesh network is illustrated in Figure~\ref{fig: mesh}. All flows start from either $s_0$ or $s_1$. The flows go through all $2^{(n-1)/2}$ possible combinations of servers towards the right. e.g. $s_0 \rightarrow s_2 \rightarrow \cdots$ and $s_1 \rightarrow s_2 \rightarrow \cdots$ are both in the network. All servers have the same service curve and capacity except $s_{n-1}$ has the doubled service rate. All flows have identical service curves and maximum packet length.

\begin{figure}
    \centering
    \includegraphics[width=0.9\linewidth]{images/mesh.png}
    \caption{Mesh network. Only parts of the flows from $s_0$ are shown. There is a flow for every possible path from $s_0$ or $s_1$ to $s_{n-1}$.}
    \label{fig: mesh}
\end{figure}

\subsubsection{Fixed-Topology Random Routing Network}
\label{sec: fixed topology random}

It's also possible to randomly generate a network defined as a JSON file with a fixed topology of switches. Users have the freedom to decide the number of flows, the topology of switches, and the service/arrival parameters. Each flow randomly routes from one switch to another without repeating the visited switch. We will show more details and use it as an example in Section~\ref{sec: example}.

% A user provides a fixed number of flows to be generated and a connection table along with service parameters and data arrival parameters, then a possible configuration of the network will be generated. A user can also specify the parameters in a range so that they will be uniformly generated within the range.

% We use Figure \ref{fig: industrial network} as an example in Section \ref{sec: example}.

\subsection{Extension}
\label{sec: extension}

As we showed in Figure~\ref{fig: pipeline}, Saihu uses XML/JSON files as a common input and a general information container class as a common output for all the tools. This means to incorporate more tools into Saihu, one only needs to allow the new tool to parse one of the network description formats and feed the analysis results into the information container class. By doing so, they can keep other parts of Saihu untouched and only need to manage one tool at a time.

Moreover, it's also possible to include more network description formats. Because the two formats Saihu uses currently can convert to each other, one only has to make sure a new format can be converted to and from one of the formats Saihu supports. We believe this approach can help more people contribute to and expand Saihu in the future.
For this chapter, fix a prime $p$. We first discuss deformations of coalgebras from $\F_{p}$
to the $p$-adic integers and further to the $p$-completed sphere $\S_{p}^{\wedge}$ which leads
us to the question of how coalgebras behave with respect to $p$-completion. We introduce the
notion of a $p$-complete coalgebra and show that this is well behaved with respect to the
deformation theory discussed in the previous chapter. We then use this to iterate
Proposition~\ref{witt} and prove our main results, namely the existence of Witt Vectors
and spherical Witt Vectors for formally \'etale coalgebras. Then we specialize to the case
of homology coalgebras, show that for a finite space $X$ the coalgebra $\F_{p}[X]$ is formally
\'etale, and answer our initial question about the relation between $\S[X]^{\wedge}_{p}$
and $\F_{p}[X]$

\subsection{Coalgebras and $p$-completion}

We have seen that the functors that interest us are all \textit{nilcomplete}. For a nilcomplete
functor $X:\rm{CAlg}^{\rm{cn}} \to \cl{S}$ and a connective $\bb{E}_{\infty}$-ring $R$, we can construct
lifts from $X(\pi_{0}R)$ to $X(R)$ inductively along the Postnikov tower
\[ \dots \to \tau_{\leq2}R \to \tau_{\tau\leq 1}R \to \tau_{\leq0} R =\pi_{0}R.\]
This is however not quite enough to obtain our goal of lifting from $\F_{p}$ to the
$p$-completed sphere, we first need to pass to $\Z_{p}= \pi_{0}\S_{p}^{\wedge}$.
Explicitly, this means constructing lifts against the tower
\[\dots \to \Z/p^{3}\to \Z/p^{2}\to \Z/p\to \F_{p}\]
which is clearly presents a different problem. With the machinery developed thus far, we can already
prove the following for a general deformation problem.

\begin{proposition}\label{liftpgen}
  Let $X: \rm{CAlg}^{\rm{cn}} \to \cl{S}$ be a cohesive functor and $A\in X(\F_{p})$
  such that $T_{X_{A}}\simeq 0$. Then there exists a unique lift of $A$ to a point in
  $\flim_{n}X(\Z/p^{n})$.
\end{proposition}
\begin{proof}
  Set $A_{0}= A$, we inductively construct lifts against the tower of square zero extensions
  \[\dots \to \Z/p^{3} \to \Z/p^{2}\to \F_{p}.\]
  Suppose we have already constructed lifts $A_{k}$ for $k\le n$ for some $n$.
  Applying Proposition~\ref{bc} inductively, we get that
  \[T_{X_{A_{n}}}^{\F_{p}} \simeq T^{\F_{p}}_{X_{A_{0}}} \simeq 0.\]
  Thus, since $\Z/p^{n+1}\to \Z/p^{n}$ is a square zero extension with fiber $\F_{p}$,
  Proposition~\ref{deformations} implies that the fiber
  \[X_{A_{n}}^{\Z/p^{n+1}}=\rm{fib}_{A_{n}}(X(\Z/p^{n+1})\to \Z/p^{n})\]
  is contractible and we find an essentially unique lift $A_{n+1}$. This proves the claim.
\end{proof}
 Of course, for an arbitrary functor $X:\rm{CAlg}^{\rm{cn}} \to \cl{S}$ the natural map
$X\to \flim_{n}X(\Z/p^{n})$ might not be an equivalence, meaning that in this generality
we can only construct pro-$p$ objects of $X$ using this inductive method.
In fact, we have that $\rm{cCAlg}_{\Z_{p}}\neq  \flim_{n} \rm{cCAlg}_{\Z/p^{n}}$. To remedy
this problem we show that this limit admits a description via \textit{$p$-complete} coalgebras.
To do this, we first recall some facts about $p$-complete modules.

\begin{definition}
Let $R$ be an $\bb{E}_{\infty}$-ring, then $M \in \rm{Mod}_{R}$ is called
$p$-\textit{complete} if the limit
\[ \lim \left(\dots \rar{\cdot p} M \rar{\cdot p}M \right)\]
vanishes. We denote the full subcategory spanned by the $p$-complete modules by $(\rm{Mod}_{R})_{p}^{\wedge}$.
\end{definition}

\begin{remark}
The inclusion $(\rm{Mod}_{R})_{p}^{\wedge} \rari{} \rm{Mod_{R}}$ admits a left adjoint which takes a module $M$
to its \textit{$p$-completion} given by the limit
\[ \lim \left( \dots \to M/p^{2} \to M/p \right).\]
In fact, $M$ is $p$-complete if and only if the natural map $M \to \lim M/p^{n}$ is an equivalence.
This inherits a natural $R^{\wedge}_{p}$-module structure, thus $p$-completion also gives
an equivalence of categories $(\rm{Mod}_{R})^{\wedge}_{p} \simeq (\rm{Mod}_{R^{\wedge}_{p}})^{\wedge}_{p}$ which
allows us to identify these in what follows.\\
The tensor product of $p$-complete modules is in general not $p$-complete. However, the
category $(\rm{Mod}_{R})_{p}^{\wedge}$ admits a symmetric monoidal structure given by the formula
 \[ M \otimes_{(\rm{Mod}_{R})_{p}^{\wedge}} N := ( M \otimes N )^{\wedge}_{p}.\]
 With this monoidal structure the $p$-completion functor $\rm{Mod}_{R}\to (\rm{Mod}_{R})_{p}^{\wedge}$
 is strong monoidal, while the inclusion is only lax monoidal.
\end{remark}

 \begin{definition}
   Let $R$ be an $\bb{E}_{\infty}$-ring. We define the $\infty$-category of $p$-complete
   $R$-coalgebras is given by.
   \[ {(\rm{cCAlg}_{R})}^{\wedge}_{p}:= \rm{cCAlg}({(\rm{Mod}_{R})}^{\wedge}_{p}).\]
 \end{definition}

 \begin{warning}
   Let $R$ be a $\bb{E}_{\infty}$-ring. Notice that by our definition a $p$-complete $R$-coalgebra
   is the same as a $p$-complete $R^{\wedge}_{p}$-coalgebra and so we do not differentiate between
   the two notions.
   However, this is \textit{not} the same as an $R^{\wedge}_{p}$-coalgebra whose underlying
   spectrum is $p$-complete. The process of $p$-completion does refine to a functor
   $\rm{cCAlg}_{R} \to (\rm{cCAlg}_{R^{\wedge}_{p}})^{\wedge}_{p}$,
   but it does not factor through the category $\rm{cCAlg}_{R^{\wedge}_{p}}$.
 \end{warning}

 We now show check that the assignment $R \mapsto \rm{cCAlg}_{R}^{\rm{cn}}$ is subject to the machinery
 of deformation theory.

 \begin{lemma}\label{conil2}
   The following statements hold:
   \begin{enumerate}
     \item   Suppose we have a pullback diagram of connective $\bb{E}_{\infty}$-rings
   \[\begin{tikzcd}
	R\p & S\p \\
	R & S
	\arrow[from=1-1, to=2-1]
	\arrow[from=2-1, to=2-2]
	\arrow[from=1-2, to=2-2]
	\arrow[from=1-1, to=1-2]
\end{tikzcd}\]
such that the map $\pi_{0}R \to \pi_{0}S$ is surjective. Then the natural map
\[ (\rm{cCAlg}_{R\p}^{\rm{cn}})^{\wedge}_{p} \to (\rm{cCAlg}_{R}^{\rm{cn}})^{\wedge}_{p}\times_{(\rm{cCAlg}_{S}^{\rm{cn}})^{\wedge}_{p}} (\rm{cCAlg}_{S\p}^{\rm{cn}})^{\wedge}_{p}\]
is an equivalence.
     \item For every connective $\bb{E}_{\infty}$-ring $R$, the natural map
           \[ (\rm{cCAlg}_{R}^{\rm{cn}})^{\wedge}_{p} \to\flim_{n} (\rm{cCAlg}_{\tau_{\le n}R}^{\rm{cn}})^{\wedge}_{p}\]
           is an equivalence.
   \end{enumerate}
 \end{lemma}
 \begin{proof}
   Ad 1.: Arguing as in the proof of Proposition~\ref{Mod}, it suffices to show that the
   strong monoidal functor
   \begin{align*}
    (\rm{Mod}_{R\p})^{\wedge}_{p} \to (\rm{Mod}_{R})^{\wedge}_{p}\times_{(\rm{Mod}_{S})^{\wedge}_{p}} (\rm{Mod}_{S\p})^{\wedge}_{p}
   \end{align*}
   is an equivalence. Indeed, given a point $(M,N,h)$ in the pullback, the $R\p$-module $M \times_{M \otimes_{R} S}N$
   is again $p$-complete since $p$-completion commutes with limits. Thus, the inverse functor of
   Proposition~\ref{Mod} also induces a functor on the categories of $p$-complete modules. Moreover,
   we have that
   \[ ((M\times_{M\otimes_{R}S}N)\otimes_{R\p} R)^{\wedge}_{p} \simeq M^{\wedge}_{p} \simeq M\]
   \[ ((M \times_{M\otimes_{R}}N)\otimes_{R\p}S\p)^{\wedge}_{p}\simeq N^{\wedge}_{p} \simeq N,\]
   where the first equivalences hold by Proposition~\ref{Mod}, and the latter since $M$ and $N$ are
   to be $p$-complete. Finally, for $M\in (\rm{Mod}_{R\p})^{\wedge}_{p}$, we compute that
   \[ (M \otimes_{R\p} R)^{\wedge}_{p}\times_{(M \otimes_{R\p} S)^{\wedge}_{p}}(M \otimes_{R\p}S\p)^{\wedge}_{p}
     \simeq \left( M \otimes_{R\p} R \times_{M\otimes_{R\p} S} M \otimes_{R\p} S\p\right)^{\wedge}_{p}
   \simeq M^{\wedge}_{p} \simeq M,\]
 where we have again used the result of Proposition~\ref{Mod} and the fact that $p$-completion commutes
 with limits.\\
 Ad 2: This uses the exact same arguments applied to the equivalence of Corollary~\ref{nilcomplete}.
 \end{proof}

 \begin{corollary}
   For any $n\in \bb{N}$, the functor
   \[ \rm{CAlg}^{\rm{cn}} \to \cl{S} \qquad R \mapsto [(\rm{cCAlg}_{R}^{\rm{cn}})^{\wedge}_{p}]^{\Delta^{n}}\]
   is coherent and nilcomplete.
 \end{corollary}

 We now prove the crucial $p$-completeness result for $\Z_{p}$-modules. As before
 this will enable us to deduce the same result for coalgebras and allow us to tackle the
 actual problem of comparing coalgebras over $\F_{p}$, $\Z_{p}$ and $\S_{p}^{\wedge}$.
\begin{proposition}\label{pcomp}
  Let $\rm{Mod}^{\wedge}_{\Z_p} \subseteq \rm{Mod}_{\Z_{p}}$ denote the full subcategory spanned by the
  $p$-complete $\Z_{p}$-module spectra. Then the natural map
  \[ \rm{Mod}_{\Z_{p}} \to \flim_{n} \rm{Mod}_{\Z/p^{n}} \quad N \mapsto (N\otimes_{\Z_{p}}\Z/p^{n})\]
  restricts to a strong monoidal equivalence
  \[(\rm{Mod}_{\Z_{p}})^{\wedge}_{p} \simeq \flim_{n}\rm{Mod}_{\Z/p^{n}}. \]
\end{proposition}
\begin{proof}
  The functor admits a right adjoint which takes $(M_{n})\in \flim_{n}\rm{Mod}_{\Z/p^{n}}$ to the limit
  $\lim_{n}M_{n}$ taken in the category of $\Z_{p}$-modules. Since $p$-complete modules are closed under
  limits, the essential image of this functor is contained in $\rm{Mod}_{\Z_{p}}^{\wedge}$. Moreover,
  if $M\in \rm{Mod}_{\Z_{p}}^{\wedge}$, then we have that
  \[ \flim_{n}(M \otimes_{\Z_{p}} \Z/p^{n}) \simeq \flim_{n} M/p^{n} \simeq M^{\wedge}_{p}\simeq M.\]
  Hence, the counit of the adjunction is an equivalence on $p$-complete modules.
  Conversely, given $(N_{k})\in \flim_{k}\rm{Mod}_{\Z/p^{k}}$ write $N= \lim_{k}N$. We want
  to show that, for every $n$ the natural map
  \[ N \otimes_{\Z_{p}} \Z/p^{n}\rar{\sim}N_{n}\]
  is an equivalence. Since $N \otimes_{\Z_{p}}Z/p^{n}\simeq N/p^{n}$ and limits are exact, we have an equivalence
  \[N \otimes_{\Z_{p}}\Z/p^{n}\simeq \lim_{k >n}(N_{k}\otimes_{\Z_{p}}\Z/p^{n}).\]
  Thus, the unit of the adjunction may be written as
  \[ \lim_{k>n}(N_{k} \otimes_{\Z_{p}}\Z/p^{n}) \to \lim_{k>n}(N_{k}\otimes_{\Z/p^{k}}\Z/p^{n})\simeq N_{n}\]
  and so has fiber given by
  \[ F_{n}:=\lim_{k>n}\left(N_{k}\otimes_{\Z/p^{k}}\rm{fib}(\Z/p^{k}\otimes_{\Z_{p}}\Z/p^{n}\to \Z/p^{n}) \right).\]
  Now we compute the fiber of $\Z/p^{k}\otimes_{\Z_{p}}\Z/p^{n}\to \Z/p^{n}$ as the module
  \[ \rm{Tor}^{\Z_{p}}(\Z/p^{k}, \Z/p^{n})[1]\simeq \Z/p^{n}[1].\]
  The reduction map $\Z/p^{k}\to \Z/p^{k-1}$ is induced by the map of projective resolutions
\[\begin{tikzcd}
	{\Z_p} & {\Z_p} \\
	{\Z_p} & {\Z_p}
	\arrow["{\cdot p^k}", from=1-1, to=1-2]
	\arrow["\id", from=1-2, to=2-2]
	\arrow["{\cdot p}"', from=1-1, to=2-1]
	\arrow["{\cdot p^{k-1}}"', from=2-1, to=2-2],
\end{tikzcd}\]
hence, on Tor it induces the multiplication by $p$ map
\[ \Z/p^{n}=\rm{Tor}^{\Z_{p}}(\Z/p^{k}, \Z/p^{n})\rar{\cdot p} \rm{Tor}^{\Z_{p}}(\Z/p^{k-1}, \Z/p^{n}) =\Z/p^{n}.\]
Thus, if we have $k\p > k > n$ such that $k\p -k > n$, the transition map
\[ F_{k\p}=N_{k\p} \otimes \rm{Tor}^{\Z_{p}}(\Z/p^{k}, \Z/p^{n})\to N_{k} \otimes \rm{Tor}^{\Z_{p}}(\Z/p^{k-1}, \Z/p^{n})= F_{k}\]
vanishes since the Tor-groups are $p^{n}$-torsion. Choosing a cofinal subset $S\subseteq \bb{N}_{>n}$ such that
$\abs{k\p -k}> n$ for any distinct $k\p,k\in S$, we see that
\[ \lim_{k>n} F_{k}\simeq \lim_{k\in S} F_{k} \simeq 0 \]
vanishes. Thus, since limits are exact, the map $N \otimes_{\Z_{p}} \Z/p^{n}\rar{\sim}N_{n}$ is an equivalence.\\
To see that the functor $\rm{Mod}_{\Z_{p}}^{\wedge} \to \flim_n \rm{Mod}_{\Z/p^{n}}$ is strong monoidal,
we observe that since cofibers and limits are exact, we have for each $n$ equivalences
\begin{align*}
  (M \otimes_{\Z_{p}} N)^{\wedge}_{p} \otimes_{\Z_{p}}\Z/p^{n} &\simeq \lim_{k}(M/p^{k} \otimes_{\Z_{p}}N/p^{k})/p^{n}\\
                                              &\simeq \lim_{k}\left((M/p^{n} \otimes_{\Z_{p}} N/p^{n})\otimes_{Z_{p}}\Z/p^{k}\right) \\
  &\simeq ((N\otimes_{\Z_{p}}\Z/p^{n}) \otimes_{\Z_{p}} (M \otimes_{\Z_{p}}\Z/p^{n}))^{\wedge}_{p}.
\end{align*}
This proves the claim.
\end{proof}

\begin{corollary}\label{pcomp1}
  We have an equivalence of categories
  \[ (\rm{cCAlg}_{\Z_{p}})_{p}^{\wedge} \rar{\sim} \flim_{n} \rm{cCAlg}_{\Z/p^{n}} \quad A \mapsto (A\otimes_{\Z_{p}}\Z/p^{n})\]
  with inverse taking a system of coalgebras $(B_{n})$ to the limit $\lim_{n}B_{n}$ taken in the
  category of ($p$-complete) $\Z_{p}$-modules, equipped with the induced $p$-complete
  $\Z_{p}$-coalgebra structure.
\end{corollary}
\begin{proof}
This follows from Proposition~\ref{pcomp}, arguing as in the proof of Proposition~\ref{Mod}.
\end{proof}

\begin{corollary}\label{obliftzp}
  Let $X(\blank)= (\rm{cCAlg}_{\blank}^{\rm{cn}})^{\Delta^{0}}$ and $A\in X(\F_{p})$ such that $T_{X_{A}}\simeq 0$.
  Then the space of lifts of $A$ to a $p$-complete $\Z_{p}$-coalgebra is contractible
\end{corollary}
 \begin{proof}
 Combine Proposition~\ref{liftpgen} and Corollary~\ref{pcomp1}.
 \end{proof}

\begin{corollary}\label{mapliftzp}
  Let $\varphi: B\to A$ be a map of connective, formally \'etale $\F_{p}$-coalgebras. Then the space of
  lifts of $\varphi$ to a map of $p$-complete $\Z_{p}$-coalgebras $B\p \to A\p$ is contractible.
\end{corollary}
\begin{proof}
    Let $ \cl{X}(\blank)=\rm{cCAlg}_{\blank}^{\rm{cn}}$. By Proposition~\ref{etalchar} the natural map
    \[ T_{\cl{X}^{\Delta^{1}}_{\varphi}} \to T_{\cl{X}^{\Delta^{0}}_{B}}\]
    is an equivalence, but since $B$ is formally \'etale we have $T_{\cl{X}^{\Delta^{0}}_{B}} \simeq 0$.
    Hence, the claim follows by applying Proposition~\ref{liftpgen} to the functor $\cl{X}^{\Delta^{1}}$
    and using Corollary~\ref{pcomp1}.
\end{proof}

Having shown this, we can now construct a functor which is analogous to the classical
Witt-Vectors, which allow us to pass from \'etale $\F_{p}$-algebras to $\Z_{p}$-algebras.

\begin{theorem}
  Let $\cl{C}\subseteq (\rm{cCAlg}_{\Z_{p}}^{\rm{cn}})^{\wedge}_{p}$ denote the full subcategory spanned by those
  coalgebras $A$ for which $A\otimes_{\Z_{p}} \F_{p}$ is formally \'etale. Then the base change functor
  \[ \cl{C} \to \rm{cCAlg}_{\F_{p}}^{\rm{cn}, \rm{f\acute{e}t}}  \qquad A \mapsto A\otimes_{\Z_{p}}\F_{p}\]
  is fully faithful and essentially surjective. In particular, the quasi inverse defines a functor
  \[ W_{p}: \rm{cCAlg}_{\F_{p}}^{\rm{cn,f\acute{e}t}} \to (\rm{cCAlg}_{\Z_{p}}^{\rm{cn}})^{\wedge}_{p}\]
  which is fully faithful and satisfies $W_{p}(A)\otimes_{\Z_{p}}\F_{p} \simeq A$ for every connective, formally
  \'etale $\F_{p}$-coalgebra $A$.
\end{theorem}

\begin{proof}
  Combine Corollary~\ref{obliftzp} and Corollary~\ref{mapliftzp}.
\end{proof}

We now turn our attention to the leap from $\Z_{p}$ to $\S_{p}^{\wedge}$. The following proposition shows that,
for an arbitrary cohesive and nilcomplete functor, a $\Z_{p}$-valued point which has vanishing $\F_{p}$-tangent
complex admits a unique lift to a $\S_{p}^{\wedge}$-valued point. This is surprising, as we do not
actually require any information about the $\Z_{p}$-tangent complex, everything is determined by
what happens modulo $p$.

\begin{proposition}\label{spherelift}
  Let $X: \rm{CAlg}^{\rm{cn}} \to \cl{S}$ be a cohesive and nilcomplete functor and let $A \in X(\Z_{p})$
  such that $T_{X_{A\otimes_{\Z_{p}}\F_{p}}}\simeq 0$. Then $A$ admits an essentially unique lift to $X(\S_{p}^{\wedge})$.
\end{proposition}

\begin{proof}
  We inductively construct lifts against the Postnikov Tower
  \[ \dots \to \tau_{\leq2} \S_{p}^{\wedge}  \to \tau_{\leq 1} \S_{p}^{\wedge} \to \tau_{\leq 0} \S_{p}^{\wedge} \simeq \Z_{p}. \]
  Write $A=A_{0},~S_{n}= \tau_{\leq n}\S_{p}^{\wedge},~ M_{n} = \pi_{n}S_{n}$ and assume we have already constructed
  a unique lift $A_{n}$ to $X(S_{n})$. Consider the square zero extension
  \[ M_{n+1}[n+1] \to S_{n+1}\to S_{n}.\]
  Since $M_{n+1} = \pi_{n+1}S_{n+1}$ is concentrated in a single degree, the $S_{n}$-action factors
  through $S_{0}=\Z_{p}$. Moreover, since $\pi_{n+1}S_{n+1}$ is of finite $p$-torsion, the action
  further factors through $\Z/p^{k}$ for some $k\geq 0$. Thus, Proposition~\ref{bc} implies that
  we have an equivalence
  \[ T_{X_{A_{n}}}^{M_{n+1}[n+1]} \simeq \Sigma^{n}T_{X_{A_{n}}}^{M_{n+1}} \simeq T_{X_{A_{n} \otimes_{S_{n}} \Z/p^{k}}}^{M_{n+1}}.\]
  Arguing as in Proposition~\ref{cofib} with respect to the square zero extension
  \[ \F_{p} \to \Z/p^{k}\to \Z/p^{k-1},\]
  we see that we have a cofiber sequence
  \[  T^{M_{n+1}\otimes_{\Z/p^{k}}\F_{p}}_{X_{A_{n} \otimes_{S_{n}} \Z/p^{k-1}}}
    \to T_{X_{A_{n} \otimes_{S_{n}} \Z/p^{k}}}^{M_{n+1}}
    \to T^{M_{n+1}\otimes_{\Z/p^{k}}\Z/p^{{k-1}}}_{X_{A_{n} \otimes_{S_{n}} \Z/p^{k-1}}}.\]
  For the left hand term, Proposition~\ref{bc} gives the equivalence
  \[ T_{X_{A_{n}\otimes_{S_{n}}\Z/p^{k-1}}}^{M_{n+1}\otimes_{\Z/p^{k}}\F_{p}}
    \simeq T_{X_{A \otimes_{\Z_{p}}\F_{p}}}^{{M_{n+1}\otimes_{\Z/p^{k}}\F_{p}}}
    \simeq T_{X_{A\otimes_{\Z_{p}}\F_{p}}}\otimes_{\F_{p}}( M_{n+1}\otimes_{\Z/p^{k}}\F_{p} ) \simeq 0,\]
  where we have used that, since $M_{n+1}$ is finitely generated, the $\F_{p}$-module
  $M_{n+1}\otimes_{\Z/p^{k}}\F_{p}$ is perfect. For the right hand term we
  replace $M_{n+1}$ with $M_{n+1} \otimes_{\Z/p^{k}}\Z/p^{k-1}$ and repeat the argument,
  inductively yielding equivalences
  \[ T^{M_{n+1}}_{X_{A_{n}\otimes_{S_{n}}\Z/p^{k}}}
    \simeq T^{M_{n+1}\otimes_{\Z/p^{k}}\Z/p^{{k-1}}}_{X_{A_{n-1} \otimes_{S_{n-1}} \Z/p^{k-1}}}
  \simeq \cdots \simeq T^{M_{n+1}\otimes_{\Z/p^{k}} \F_{p}}_{X_{A \otimes_{\Z_{p}}\F_{p}}} \simeq 0.\]
In total, this shows that $T_{X_{A_{n}}}^{M_{n+1}[n+1]} \simeq 0$, and hence $A_{n}$ admits an essentially
unique lift to $X(S_{n+1})$. Thus, the fiber over $A$ of the map
\[ X(\S_{p}^{\wedge})\simeq \flim_{n}X(S_{n})\to X( \Z_{p})\]
is contractible and we are done.
  \end{proof}

  \begin{lemma}\label{pcomparison}
    Write $\cl{X}(\blank)=\rm{cCAlg}^{\rm{cn}}_{\blank}$ and $\cl{Y}(\blank)=
    (\rm{cCAlg}^{\rm{cn}}_{\blank})^{\wedge}_{p}$. Then the $p$-completion map $f:\cl{X}\to \cl{X}\p$
    induces an equivalence
    \[ T^{M}_{(\cl{X}^{\Delta^{n}})_{\xi}} \to  T^{M}_{(\cl{Y}^{\Delta^{n}})_{f(\xi)}}\]
        for every $\F_{p}$-module $M$, $n\in \bb{N}$ and $\xi \in \cl{X}(\F_{p})^{\Delta^{n}}$.
  \end{lemma}
  \begin{proof}
    For any $\F_{p}$-algebra $R$ the $p$-completion map gives an equivalence
    $\rm{Mod}_{R}\rar{\sim} (\rm{Mod}_{R})^{\wedge}_{p}$, since multiplication by some power of $p$
    is nullhomotopic over $\F_{p}$. In particular, this applies to the split square zero
    extension $\F_{p}\oplus M$ for any $M \in \rm{Mod}_{\F_{p}}$ and so the natural map
    $\cl{X}(\F_{p}\oplus M) \to \cl{Y}(\F_{p}\oplus M)$ is an equivalence as well.
    Consequently, we also obtain natural equivalences between the fibers
    \[ (\cl{X}^{\Delta^{n}})_{\xi}^{\F_{p}\oplus M} \to  (\cl{Y}^{\Delta^{n}})_{f(\xi_)}^{\F_{p}\oplus M},\]
    which induces the equivalence of spectra
    \[ T^{M}_{(\cl{X}^{\Delta^{n}})_{\xi}} \to  T^{M}_{(\cl{Y}^{\Delta^{n}})_{f(\xi)}}\]
      as claimed.
  \end{proof}

  \begin{corollary}\label{obliftsp}
    Let $X(\blank)=(\rm{cCAlg}^{\rm{cn}}_{\blank})^{\Delta^{0}}$ and $A \in X(\F_{p})$ such that
    $T_{X_{A}}\simeq 0$, then the space of lifts of $A$ to a $p$-complete $\S_{p}^{\wedge}$-coalgebra
    is contractible.
  \end{corollary}

  \begin{proof}
    Write $Y(\blank)= ((\rm{cCAlg}^{\rm{cn}}_{\blank})^{\wedge}_{p})^{\Delta^{0}}$. Then by Lemma~\ref{pcomparison}
    we have an equivalence $T_{X_{A}}\simeq T_{Y_{A}} \simeq 0$. Hence, we can apply Proposition~\ref{obliftzp} to
    obtain an essentially unique lift $A\p\in Y(Z_{p})$. Further applying Proposition~\ref{spherelift}
    to $A\p$ yields our claim.
  \end{proof}
  Thus, we can pointwise lift $\F_{p}$-coalgebras with vanishing tangent complex to $\S_{p}^{\wedge}$. If
  we moreover consider \textit{formally \'etale coalgebras}, we can make this lifting functorial
  in a coalgebraic analogue of the \textit{Spherical Witt Vectors} construction for
  $\bb{E}_{\infty}$-algebras over $\F_{p}$.

\begin{corollary}\label{mapliftsp}
  Let $\varphi:B\to A$ be a map of $\F_{p}$-coalgebras such that $A$ and $B$ are formally \'etale.
  Then the space of lifts of $\varphi$ to a map $\varphi\p: B\p \to A\p$ of $p$-complete
  $\S_{p}^{\wedge}$-coalgebras is contractible.
\end{corollary}

\begin{proof}
  Let $ \cl{X}(\blank)=\rm{cCAlg}_{\blank}^{\rm{cn}}$ and $\cl{Y}(\blank) =
  (\rm{cCAlg}_{\blank}^{\rm{cn}})^{\wedge}_{p}$. By Proposition~\ref{mapliftzp} the map $\varphi$ admits
  an essentially unique lift to a point $\psi \in \cl{Y}(\Z_{p})^{\Delta^{1}}$. Moreover, Lemma~\ref{pcomparison}
  yields an equivalence $T_{\cl{X}^{\Delta^{1}}_{\varphi}}\simeq T_{\cl{Y}^{\Delta^{1}}_{\varphi}}$. Since both $A$ and $B$ are
  formally \'etale Proposition~\ref{etalchar} gives equivalences
  \[ T_{\cl{X}^{\Delta^{1}}_{\varphi}} \rar{\sim} T_{\cl{X}^{\Delta^{0}}_{B}} \simeq 0\]
  Hence, we can apply Proposition~\ref{spherelift} to the functor $\cl{Y}^{\Delta^{1}}$ and the point
  $\psi \in \cl{Y}^{\Delta^{1}}$, proving the claim.
\end{proof}

\begin{theorem}\label{wittsp}
  Denote by $\cl{C}\subseteq (\rm{cCAlg}_{\S_{p}^{\wedge}}^{\rm{cn}})^{\wedge}_{p} $ the full subcategory spanned by those
  coalgebras $A$ such that $A\otimes_{\S_{p}^{\wedge}}\F_{p}$ is formally \'etale. Then the base change functor
  \[ \cl{C} \to \rm{cCAlg}_{\F_{p}}^{\rm{cn}, \rm{f\acute{e}t}} \qquad A \mapsto A \otimes_{\S_{p}^{\wedge}} \F_{p}\]
  is fully faithful and essentially surjective.
\end{theorem}
\begin{proof}
  Combine Corollary~\ref{obliftsp} and Corollary~\ref{mapliftsp}.
\end{proof}

\begin{remark}
  In the setting of Theorem~\ref{wittsp} the quasi-inverse to $\blank \otimes_{\S^{\wedge}_{p}}\F_{p}$ defines
  a fully faithful functor
  \[ W_{\S_{p}^{\wedge}}: \rm{cCAlg}_{\F_{p}}^{\rm{cn}, \rm{f\acute{e}t}}
    \to (\rm{cCAlg}_{\S_{p}^{\wedge}}^{\rm{cn}})^{\wedge}_{p}\]
  which satisfies $W_{\S_{p}^{\wedge}}(A)\otimes_{\S^{\wedge}_{p}}\F_{p} \simeq A$ for every connective, formally \'etale
  $\F_{p}$-coalgebra $A$. We call $W_{\S_{p}^{\wedge}}(A)$ the \textit{spherical Witt vectors} of $A$.
\end{remark}


\subsection{Homology coalgebras}

As observed in Example~\ref{homology}, for every space $X$ and every $\bb{E}_{\infty}$-ring $R$, the
$R$-homology $R[X]$ carries a natural $R$-coalgebra structure, which is a stronger invariant than its
underlying $R$-module. We now want to apply our results and see what can be said about the deformation
theoretic behavior of homology coalgebras. To do this, we first need to compute the cotangent complex of the
$\F_{p}$-cohomology.

\begin{definition}
  A space $X\in \cl{S}$ is called $p$-finite if the following conditions hold:
  \begin{enumerate}
    \item The space $X$ is truncated.
    \item The set $\pi_{0}X$ is finite.
    \item For each $n\geq 1$ and $x\in X$, we have that $\pi_{n}(X,x)$ is a finite $p$-group.
  \end{enumerate}
  We denote the full subcategory of $\cl{S}$ spanned by the $p$-finite spaces as $\cl{S}_{p}$ and call
 $\cl{S}^{\vee}_{p} =: \rm{Pro}(\cl{S}_{p})$ the category of $p$-\textit{profinite} spaces.
\end{definition}

\begin{remark}
We can regard $\cl{S}_{p}^{\vee}$ as the category of ``formal limits'' of $p$-finite spaces $\varprojlim X_{\alpha}$.
As such there is a functor $\cl{S}^{\vee}_{p}\to \cl{S}$ which takes a formal limit to the actual limit in $\cl{S}$.
This functor admits a left adjoint given by $Y \mapsto \flim_{Y_{\alpha} \to Y} Y_{\alpha}$, where the limit runs over all maps
from a $p$-finite space $Y_{\alpha}$ to $Y$.
\end{remark}

\begin{lemma}
  Let $X$ be a space and $\flim X_{\alpha}$ be its $p$-profinite completion. Then the natural map
  of cohomology rings
  \[ \fcolim \F_{p}^{X_{\alpha}} \to \F_{p}^{X} \]
  is an equivalence.
\end{lemma}
\begin{proof}
  This is immediate since the Eilenberg-MacLane spaces $K(\F_{p},n)$ are $p$-finite.
\end{proof}

\begin{proposition}[Mandell, Lurie]\label{coetal}
  Let $X$ be a space, then the $\F_{p}$-cohomology $\F_{p}^{X}$ is a formally \'etale $\F_{p}$-algebra.
\end{proposition}
\begin{proof}
  Since the functor $R \mapsto L_{R/\F_{p}}$ commutes with colimits, the claim follows from the fact that
  $L_{\F_{p}^{X}/\F_{p}}\simeq 0$ for every $p$-finite space $X$ which is proven
  in~\cite[][Proposition 2.4.12]{dag8}.
\end{proof}

Thus we obtain the following result about the homology coalgebra of a finite space $X$
with coefficients in a connective $\F_{p}$-algebra $R$:

\begin{corollary}\label{goal}
  Let $X$ be a finite space and $R$ be an $\F_{p}$-algebra, then $R[X]$ is a formally
  \'etale $R$-coalgebra.
\end{corollary}
\begin{proof}
  From Proposition~\ref{coetal} we get that
  \[ L_{R^{X}/R}\simeq L_{\F_{p}^{X}/\F_{p}}\otimes_{\F_{p}}R \simeq 0.\]
  Since $X$ is finite, the coalgebra $R[X]$ is dualizable with dual given by $R^{X}$, so the claim
  follows from Proposition~\ref{dualetal}.
\end{proof}

Moreover, for the case $R=\F_{p}$, we can use Theorem~\ref{wittsp} to give a partial answer to our
initial question about lifts of the coalgebra $\F_{p}[X]$.

\begin{corollary}
  Let $X$ be a finite space, then $\F_{p}[X]$ admits a unique lift to a $p$-complete $\S_{p}^{\wedge}$-coalgebra
  given by $W_{\S_{p}^{\wedge}}(\F_{p}[X]) \simeq (\S[X])^{\wedge}_{p}$. Moreover, for any other finite space $Y$
  the natural map
  \[\rm{Map}_{(\rm{cCAlg}_{\S_{p}^{\wedge}})^{\wedge}_{p}}((\S[Y])^{\wedge}_{p}, (\S[X])^{\wedge}_{p})
    \to \rm{Map}_{\rm{cCAlg}_{\F_{p}}}(\F_{p}[Y], \F_{p}[X])\]
  is a homotopy equivalence.
\end{corollary}
\begin{proof}
 Combine Corollary~\ref{goal} and Theorem~\ref{wittsp}.
\end{proof}

\section{Where to go from here}

We finish our discussion by explaining some of the shortcomings of our results and sketch a possible
way to proceed towards a coalgebraic analogue of Mandell's Theorem. The first missing puzzle piece is
the cotangent complex of a coalgebra $A$, which we have been unable to give a solid definition of.
The second and more important one is the relation to the \textit{coalgebra Frobenius}. We conjecture
that the class of \textit{perfect} coalgebras defined via this map give examples of non-dualizable
formally \'etale coalgebras. In particular, this conjecture would imply that the $\F_{p}$-homology
of \textit{any} space $X$ is formally \'etale.

\subsection{The cotangent complex of a coalgebra}
One of the first questions that arose during this project turned out to be one of the most subtle and
tricky ones, namely:

\begin{question}
  What is the cotangent complex of a coalgebra $A$?
\end{question}

Clearly, the existence of a single spectrum controlling the deformation theory of $A$ would be immensely
useful. However, it is not immediately clear what the universal property of such a spectrum should be,
i.e.~which space of derivations it should (co)represent.
Some inspiration can be gleamed from Proposition~\ref{cotangentder}. There we had seen that, for
$\varphi: B \to A$ a map of $R$-coalgebras with $A$ dualizable and $M$ an $R$-module, we have an equivalence
\[ \rm{Der}_{\varphi}(B, C_{A}(M)) \simeq \rm{Map}_{A^{\vee}}(L_{A^{\vee}/R}, \varphi^{\vee}_{\pt}\rm{map}_{R}(B, M)).\]
To get rid of the dependence on the second coalgebra $B$ one is tempted to take $B=R$ such that
$\rm{map}_{R}(B,M)\simeq M$. However, not every coalgebra $A$ admits a map $R\to A$, much less a canonical
one. The only natural choice for a map that is not the initial map would yield the following:

\begin{definition}[Preliminary 1.]
  Let $R$ be an $\bb{E}_{\infty}$-ring and $A\in \rm{cCAlg}_{R}$. The cotangent complex of $A$, if it exists,
  is the $R$-module $L_{A}$ corepresenting the functor
  \[ \rm{Mod}_{R}\to \rm{Mod}_{R} \qquad M \mapsto \rm{der}_{\id}(A, C_{A}(M))\]
\end{definition}

There are however several problems with this. Firstly, it is entirely unclear from the definition
whether $L_{A}$ vanishing would actually imply $A$ being formally \'etale. Moreover, in the dualizable
case it would lead to the rather awkward formula
\[ L_{A} \simeq L_{A^{\vee}/R}\otimes_{A^{\vee}}A.\]
Although somewhat plausible, this again gives us little information about what can actually be
deduced in the case that $L_{A}\simeq 0$.
This leaves us with several options, lest we accept that there is no good notion of one singular
cotangent complex. For one we could work with \textit{coaugmented} coalgebras, namely coalgebras
together with a map $R \to A$. For the purpose of understanding homology coalgebras this would correspond
to considering pointed spaces instead of just spaces, an entirely acceptable compromise, but beyond the
scope of this paper. \\
A different  approach would be to give up on the idea of corepresentability
and instead hope for a colimit preserving functor. For example, the functor
\[ \rm{Mod}_{R}\to \rm{Mod}_{R} \qquad M \mapsto C_{A}(M):=\rm{cofib}( A \rar{\eps} \Omega^{\infty}_{A}M).\]
seems to have no chance of preserving limits, but since colimits of coalgebras are formed underlying,
colimits are not out of the race. This leads us to the following idea:

\begin{definition}[Preliminary 2]\label{dream}
  Let $R$ be an $\bb{E}_{\infty}$-ring and $A\in \rm{cCAlg}_{R}$. We say that $A$ admits a cotangent
  complex $L_{A}:= C_{A}(R)$ if the functor $C_{A}(\blank):\rm{Mod}_{R} \to \rm{Mod}_{R}$ commutes
  with colimits. In this case we have $C_{A}(M)\simeq L_{A}\otimes M$ for every $ M \in \rm{Mod}_{R}$
\end{definition}

This definition is highly speculative, as the only coalgebras we know to admit a cotangent complex
in this sense are the formally \'etale coalgebras, for which the functor $C_{\blank}(A)$ is constant.
Conversely, if $A$ admits a cotangent complex then $L_{A}$ vanishes if and only if $A$ is formally
\'etale. Hence, the spectrum $L_{A}$ is precisely the obstruction to $A$ being formally \'etale,
which is the kind of conceptual clarity we are looking for.
While we lose any direct comparison to the cotangent complex of $A^{\vee}$ this is not entirely surprising,
since the property of being formally \'etale is defined very differently for $A^{\vee}$.
This leaves us with the following:

\begin{question}\label{cotangentdream}
  Let $R$ be an $\bb{E}_{\infty}$-ring. Does every $A \in \rm{cCAlg}_{R}$ admit a cotangent complex in the sense
  of Definition~\ref{dream}?
\end{question}

Regardless of the answer, the takeaway should be that the modules
$C_{A}(M)$ are exactly the obstruction towards $A$ being formally \'etale. Moreover, while the functor
$A\mapsto C_{A}(M)$ is very complicated, the dependence on $M$ should be relatively tame. That is,
for fixed $A$ it should be possible to describe the functor $M \mapsto C_{A}(M)$ in terms of a
formula involving $C_{A}(R)$. However, because $C_{A}(M)$ no longer has a direct relation to any
space of derivations or tangent complex, we cannot leverage results like Proposition~\ref{structure}
to obtain such a formula. We understand this as an indication that for these questions, the formalism may
have reached its limit.

\subsection{The Frobenius}
The most lacking thing about our results is the class of coalgebras that we can currently apply them to.
As of now, we are unable to give examples of formally \'etale coalgebras which are not dualizable. In
particular, we cannot describe the deformation theory of $R[X]$ for spaces $X$ which are not finite.
Attempts to reduce to the dualizable case all seem to fail for the following reason: Even though
we may write $X= \fcolim_{i}X_{i}$ where each $X_{i}$ is finite, giving the formula
$R[X]= \fcolim_{i}R[X_{i}]$, there is no reason why the functor
$\Omega^{\infty}_{\blank}(M): \rm{cCAlg}_{R}\to \rm{cCAlg}_{R}$ should commute with colimits.
Indeed, write $f_{M}:R\to R\oplus M$ for inclusion, then by definition
$\Omega^{\infty}_{\blank}(M) = f_{M,!} f^{\pt}_{M}$. The functor $f^{\pt}_{M}$ commutes with colimits,
and from Proposition~\ref{present} and the converse of the adjoint functor theorem we can deduce
that $f_{M,!}$ commutes with $\kappa$-filtered colimits for some regular cardinal $\kappa$. Thus, the class
of formally \'etale coalgebras is closed under $\kappa$-filtered colimits, but $\kappa$ is, in general, not countable.
% Closely related is the fact the notion of compactness is strangely behaved for coalgebras. For example,
% one can show that $\bb{Q}$ is not a compact object of $\rm{cCAlg}_{\Q}$, see~\cite[][Warning 1.2.15.]{ellII}.
% In particular, this means that
% \[ \rm{cSpec}(\fcolim_{i}\S[X_{i}])(\bb{Q})\neq \fcolim_{i}\rm{cSpec}(\S[X_{i}])(\Q),\]
% so we cannot deduce things about the cospectrum of infinite spaces in this way either. \\
This goes to show that the deformation theory of non-dualizable coalgebras is richer and more
interesting than that of the Ind-completion of dualizable coalgebras and requires additional input.
One contender for this additional input is the \textit{Coalgebra Frobenius} constructed by
Nikolaus:

\begin{theorem}[Nikolaus]
  Let $\cl{C} = (\rm{cCAlg}^{\rm{cn}}_{\S^{\wedge}_{p}})^{\wedge}_{p}$, then there exists a natural transformation
  $\psi_{p}:\id_{\cl{C}}\to \id_{\cl{C}}$ which on an object $A\in \cl{C}$ is given by the composition
  \[ \psi_{p}: A \rar{\Delta_{A}^{\otimes p}} (A^{\otimes p})^{hC_{p}} \rar{\rm{can}} (A^{\otimes p})^{tC_{p}} \rar{\sim} A,\]
  where the final map is the inverse of the \textit{Tate Diagonal}, see~\cite[][Theorem III.1.7]{tch}.
\end{theorem}

Given this map, we are naturally led to define \textit{perfect} coalgebras as follows:

\begin{definition}
  We say that $A \in  (\rm{cCAlg}^{\rm{cn}}_{\S^{\wedge}_{p}})^{\wedge}_{p}$ is \textit{perfect} if the coalgebra
  Frobenius $\psi_{p}: A\to A$ is a homotopy equivalence. We denote the full subcategory spanned by
  the perfect coalgebras by $(\rm{cCAlg}^{\rm{cn}}_{\S^{\wedge}_{p}})^{\wedge ,\rm{perf}}_{p} \subseteq
  (\rm{cCAlg}^{\rm{cn}}_{\S^{\wedge}_{p}})^{\wedge}_{p}$.
\end{definition}

\begin{example}\label{frobchains}
  Let $X$ be any space. Then $(\S[X])^{\wedge}_{p}$ is a perfect coalgebra since we have that
  \[\S[X]^{\wedge}_{p} \simeq (\S_{p}^{\wedge}[\colim_{X}\pt])^{\wedge}_{p} \simeq (\colim_{X} \S_{p}^{\wedge})^{\wedge}_{p}.\]
  On $\S_{p}^{\wedge}$ the map $\psi_{p}$ is necessarily given by the identity, because $\S_{p}^{\wedge}$
  is the terminal $p$-complete $\S_{p}^{\wedge}$-coalgebra. Thus, by naturality $\psi_{p}$ is given
  by the identity on $(\S[X])^{\wedge}_{p}$ as well.
\end{example}

We conjecture that this Frobenius map is related to the deformation theory of coalgebras in a similar
way to the Algebra Frobenius, in that it provides a sufficient condition for a coalgebra to be formally
\'etale.

\begin{conjecture}\label{frobcof}
  Let $A \in (\rm{cCAlg}^{\rm{cn}}_{\S^{\wedge}_{p}})^{\wedge}_{p}$ and write $A\p= A\otimes_{\S^{\wedge}_{p}}\F_{p}$.
  Then for any $M \in \rm{Mod}_{\F_{p}}^{\rm{cn}}$, the coalgebra Frobenius $\psi_p:A\to A$ induces the zero map
  on the $R$-module  $C_{A\p}(M) = \rm{cofib}(A\p \rar{\eta_{A\p}} \Omega^{\infty}_{A}(M))$.
\end{conjecture}

\begin{corollary}
  If Conjecture~\ref{frobcof} holds, then the base change functor
  \[ (\rm{cCAlg}^{\rm{cn}}_{\S^{\wedge}_{p}})^{\wedge ,\rm{perf}}_{p} \to \rm{cCAlg}_{\F_{p}}^{\rm{cn}}
  \qquad A \mapsto A\otimes_{\S_{p}^{\wedge}}\F_{p}\]
is fully faithful and factors through the full subcategory
$\rm{cCAlg}_{\F_{p}}^{\rm{cn}, \rm{f\acute{e}t}}\subseteq \rm{cCAlg}_{\F_{p}}^{\rm{cn}}$.
\end{corollary}
\begin{proof}
  Since $\psi_{p}:A\rar{\sim} A$ is an equivalence it induces an equivalence on $A\otimes_{\S_{p}^{\wedge}}\F_{p}$ and
  thus on $C_{A\otimes_{\S_{p}^{\wedge}}\F_{p}}(M)$ as well. However, since it also induces the zero map on the latter
  we get that $C_{A\otimes_{\S_{p}^{\wedge}}\F_{p}}(M)\simeq 0$. Thus, $A\otimes_{\S_{p}^{\wedge}}\F_{p}$ is formally \'etale and the
  claim follows from Theorem~\ref{wittsp}.
\end{proof}

Combining this with Example~\ref{frobchains} would allow us to fully answer our initial question about
homology coalgebras.

\begin{corollary}\label{dream2}
  If Conjecture~\ref{frobcof} holds, then for any space $X$ the $\F_{p}$-chains $\F_{p}[X]$
  are formally \'etale. In particular $\F_{p}[X]$ admits a unique and functorial lift to a $p$-complete
  $\S_{p}^{\wedge}$-coalgebra given by $\S[X]^{\wedge}_{p}= W_{\S_{p}^{\wedge}}(\F_{p}[X])$.
\end{corollary}

The fact that Conjecture~\ref{frobcof} needs to be checked for every connective $\F_{p}$-module should
be understood as an extension of our failure to find a cotangent complex. Indeed, if $\F_{p}[X]$ admits
a cotangent complex in the sense of Definition~\ref{dream}, then to obtain Corollary~\ref{dream2} it
would suffice to show that $\psi_{p}$ induces the zero map on $C_{A\otimes_{\S_{p}^{\wedge}}\F_{p}}(\F_{p})
= L_{A\otimes_{\S_{p}^{\wedge}}\F_{p}}$. However, even for this specific module the conjecture is difficult
to attack from our present position. The problem is the tricky right adjoint
$\rm{cCAlg}_{\F_{p}\oplus \F_{p}}\to \rm{cCAlg}_{\F_{p}}$ appearing in the definition of
$C_{A\otimes_{\S^{\wedge}_{p}}\F_{p}}(\F_{p})$. Because there is no known formula for this functor, attempts to verify
the conjecture have thus far been unsuccessful in all non-trivial cases. This warrants further investigation
of the coalgebra Frobenius and Conjecture~\ref{dream2}.


\section{Conclusions}
\label{sec:conclusions}

Advancing understanding of the physics of the edge of tokamak plasmas
drives the need for increasingly complex models. To address this need
a new open-source plasma simulation tool has been developed that
enables researchers to perform complex multi-species plasma
simulations by combining reusable software components. This is
achieved by building on the BOUT++ framework of partial differential
equation solvers, and defining a flexible yet robust method of
coupling components together within a parallelised high-performance
code.

Applications of this tool to simulations of tokamak plasmas have been
demonstrated: Time dependent simulations of plasma filament/blob
propagation and steady-state transport including atomic
reactions. Convergence tests and comparisons to analytic solutions
have been carried out, demonstrating good conservation properties and
convergence of the methods. The public Git repository includes a suite
of unit, integrated and Method of Manufactured Solutions (MMS) tests
that are used routinely to check the correctness of code changes.

Areas for future development and research have been identified:
Extending the steady state solver implemented using PETSc from 1D
transport problems (section~\ref{sec:1d-transport}) to 2D is a high
priority, as is benchmarking of Hermes-3 against other codes for both
transport and turbulence applications. These efforts have begun and
will be reported elsewhere once completed.

Hermes-3 is publicly available~\cite{dudson:hermes3} on Github under a
GPL-3 license. To maximise its utility to the plasma community a set
of examples are included, and a manual~\cite{dudson:hermes3-manual}
provides an introduction for new users.

\section*{Acknowledgements}

This work was in part performed under the auspices of the U.S. DoE by
LLNL under Contract DE-AC52-07NA27344, and received funding from LLNL
LDRD 23-ERD-015. B.Dudson would like to thank Dr. Wayne Arter (CCFE)
for useful discussions and suggestions. The Hermes-3 source code,
simulation inputs, and processing scripts needed to reproduce the
results shown in this paper are available at
\url{https://github.com/bendudson/hermes-3}, Git commit
\texttt{5f56919} (Version 1.1.0). LLNL-CODE-845139.

\clearpage
\bibliographystyle{elsarticle-num}
\bibliography{bibliography}

\appendix

\section{Appendix for Proofs}

\paragraph{Proof of Theorem \ref{thm:main}.}

\begin{proof}
\label{proof:main}
Our proof has two steps. In Step 1, we will show that SimCLR is equivalent to minimizing the cross entropy loss defined in Eqn.~(\ref{eqn:cross-entropy}). 
In Step 2, we will show  that minimizing the cross-entropy loss 
is equivalent to spectral clustering on $\bfpi$. 
Combining the two steps together, we have proved our theorem. 

\textbf{Step 1: } SimCLR is equivalent to minimizing the cross entropy loss.

The cross-entropy loss takes expectation over 
$\bfW_\bfX\sim \mathbb{P}(\cdot ; \bfpi)$, 
which means $\bfW_\bfX$ has exactly one non-zero entry in each row $i$. By Lemma~\ref{lem:multinomial}, we know every row $i$ of $\bfW_\bfX$ is independent of other rows. Moreover, 
$\bfW_{\bfX,i}\sim \mathcal{M}(1, \bfpi_i/\sum_j \bfpi_{i,j})=\mathcal{M}(1, \bfpi_i)$, because $\bfpi_i$ itself is a probability distribution.
Similarly, we know $\bfW_\bfZ$ also has the row-independent property by sampling over $\mathbb{P}(\cdot;\bfK_\bfZ)$.
Therefore, by Lemma~\ref{lem:cross_split}, we know Eqn.~(\ref{eqn:cross-entropy}) is equivalent to:
\[
 -\sum_{i=1}^n \mathbb{E}_{\bfW_{\bfX,i}}[\log \mathbb{P}(\bfW_{\bfZ,i}=\bfW_{\bfX,i};\bfK_\bfZ)],
\]

This expression takes expectation over $\bfW_{\bfX,i}$ for the given row $i$. Notice that 
$\bfW_{\bfX,i}$ has exactly one non-zero entry, which equals $1$ (same for $\bfW_{\bfZ,i}$). 
As a result
we expand the above expression to be:
\begin{equation}
 -\sum_{i=1}^n \sum_{j\neq i} \Pr(\bfW_{\bfX,i,j}=1)\log \Pr(\bfW_{\bfZ,i,j}=1).
\label{eqn:detailed-expansion}    
\end{equation}


By Lemma~\ref{lem:multinomial}, $\Pr(\bfW_{\bfZ,i,j}=1)=\bfK_{\bfZ,i,j}/\|\bfK_{\bfZ,i}\|_1$ for $j\neq i$. Recall that $\bfK_\bfZ=(k(\bfZ_i-\bfZ_j))_{(i,j)\in[n]^2}$, which means 
$\bfK_{\bfZ,i,j}/\|\bfK_{\bfZ,i}\|_1=\frac{\exp(-\|\bfZ_i-\bfZ_j\|^2/{2\tau})}{\sum_{k\neq i}
\exp(-\|\bfZ_i-\bfZ_k\|^2/{2\tau})
}$ for $j\neq i$, when $k$ is the Gaussian kernel with variance $\tau$. 

Notice that $\bfZ_i=f(\bfX_i)$, so we know
\begin{equation}
-\log \Pr(\bfW_{\bfZ,i,j}=1)=
-\log \frac{\exp(-\|f(\bfX_i)-f(\bfX_j)\|^2/{2\tau})}{\sum_{k\neq i}
\exp(-\|f(\bfX_i)-f(\bfX_k)\|^2/{2\tau}),
}
\label{eqn:infonce-equivalence}    
\end{equation}


The right hand side is exactly the InfoNCE loss defined in Eqn.~(\ref{eqn:infonce}).
Inserting Eqn.~(\ref{eqn:infonce-equivalence}) into Eqn.~(\ref{eqn:detailed-expansion}), we get the SimCLR algorithm, which first samples augmentation pairs $(i,j)$ with $\Pr(\bfW_{\bfX,i,j}=1)$ for each row $i$, and then optimize the InfoNCE loss. 

\textbf{Step 2: } minimizing the cross entropy loss 
is equivalent to spectral clustering on $\bfpi$.


By Lemma~\ref{lem:convert_to_spectral}, we may further convert the loss to 
\begin{equation}
\label{eqn:main-theorem-repul-attr}
\min_{\bfZ}
-\sum_{(i,j)\in [n]^2} \mathbf{P}_{i,j}
\log k (\bfZ_i-\bfZ_j)+\log \mathbf{R}(\bfZ).
\end{equation}
Since $k$ is the Gaussian kernel, this reduces to \[
\min_\bfZ \mathrm{tr}(\bfZ^\top \mathbf{L}(\bfpi) \bfZ)
+\log \mathbf{R}(\bfZ),
\]

where we use the fact that $\mathbb{E}_{\bfW_\bfX\sim \mathbb{P}(\cdot; \bfpi)}[\mathbf{L}(\bfW_\bfX)]
=\mathbf{L}(\bfpi)
$, because the Laplacian operator is linear and $
\mathbb{E}_{\bfW_\bfX\sim \mathbb{P}(\cdot; \bfpi)}(\bfW_\bfX)=\bfpi
$.
\end{proof}

\paragraph{Proof of Theorem \ref{thm:clip}.}
\begin{proof}
Since $\bfW_\bfX\sim \mathbb{P}(\cdot;\bfpi_{\mathbf{A}, \mathbf{B}})$, we know 
$\bfW_\bfX$ has exactly one non-zero entry in each row, denoting the pair that got sampled. 
A notable difference compared to the previous proof is we now have $n_\mathcal{A}+n_\mathcal{B}$ objects in our graph. CLIP deals with this by taking a mini-batch of size $2N$, 
such that $n_\mathcal{A}=n_\mathcal{B}=N$, and adding the $2N$ InfoNCE losses together. We label the objects in $\mathcal{A}$ as $[n_\mathcal{A}]$, and the objects in $\mathcal{B}$ as $\{n_\mathcal{A}+1, \cdots, n_\mathcal{A}+n_\mathcal{B}\}$. 

Notice that $\bfpi_{\mathbf{A}, \mathbf{B}}$ is a bipartite graph, so the edges of objects in $\mathcal{A}$ will only connect to object in $\mathcal{B}$ and vice versa. We can define the similarity matrix in $\cZ$ as $\bfK_\bfZ$, 
where $\bfK_\bfZ(i, j+n_\mathcal{A})=\bfK_\bfZ(j+n_\mathcal{A},i)= k(\bfZ_i-\bfZ_j)$ for $i\in [n_\mathcal{A}], j\in [n_\mathcal{B}]$, and otherwise we set $\bfK_\bfZ(i,j)=0$. 
The rest is same as the previous proof. 
\end{proof}

\paragraph{Proof of Theorem \ref{thm:exponential}.}

\begin{proof}
\label{proof:exponential}
Since the objective function consists of a linear term combined with an entropy regularization, which is a strongly concave function, the maximization problem is a convex optimization problem. Owing to the implicit constraints provided by the entropy function, the problem is equivalent to having only the equality constraint. We then introduce the Lagrangian multiplier $\lambda$ and obtain the following relaxed problem:

$$
\widetilde{E}(\boldsymbol{\alpha})=\psi_{1}-\sum_{i=1}^n \alpha_{i} \psi_{i}+\tau \sum_{i=1}^n \alpha_{i}\log \alpha_{i}+\lambda\left(\boldsymbol{\alpha}^{\top} \mathbf{1}_n-1\right).
$$

As the relaxed problem is unconstrained, taking the derivative with respect to $\alpha_{i}$ yields

$$
\frac{\partial \widetilde{E}(\boldsymbol{\alpha})}{\partial \alpha_{i}}=-\psi_{i}+\tau\left(\log \alpha_{i}+\alpha_{i} \frac{1}{\alpha_{i}}\right)+\lambda=0.
$$

Solving the above equation implies that $\alpha_{i}$ takes the form
$
\alpha_{i}=\exp \left(\frac{1}{\tau} \psi_{i}\right) \exp \left(\frac{-\lambda}{\tau}-1\right).
$ Since $\alpha_{i}$ lies on the probability simplex, the optimal $\alpha_{i}$ is explicitly given by
$
\alpha^{*}_{i}=\frac{\exp \left(\frac{1}{\tau} \psi_{i}\right)}{\sum_{i^{\prime}=1}^n \exp \left(\frac{1}{\tau} \psi_{i^{\prime}}\right)} .
$ Substituting the optimal point into the objective function, we obtain
$$
\begin{aligned}
E\left(\boldsymbol{\alpha}^*\right)  &=\psi_1-\sum_{i=1}^n \frac{\exp \left(\frac{1}{\tau} \psi_{i}\right)}{\sum_{i^{\prime}=1}^n \exp \left(\frac{1}{\tau} \psi_{i^{\prime}}\right)} \psi_{i}+\tau \sum_{i=1}^n \frac{\exp \left(\frac{1}{\tau} \psi_{i}\right)}{\sum_{i^{\prime}=1}^n \exp \left(\frac{1}{\tau} \psi_{i^{\prime}}\right)}\log \frac{\exp \left(\frac{1}{\tau} \psi_{i}\right)}{\sum_{i^{\prime}=1}^n \exp \left(\frac{1}{\tau} \psi_{i^{\prime}}\right)} \\
& =\psi_1 - \tau \log \left(\sum_{i=1}^n \exp \left(\frac{1}{\tau} \psi_{i}\right)\right).
\end{aligned}
$$
Thus, the Lagrangian dual function is given by
\begin{equation*}
-E\left(\boldsymbol{\alpha}^*\right)= -\tau \log \frac{\exp \left(\frac{1}{\tau} \psi_{1}\right)}{\sum_{i=1}^n \exp \left(\frac{1}{\tau} \psi_{i}\right)}.\qedhere
\end{equation*}
\end{proof}



\section{More on Experiments} \label{section: experiment_details}

\paragraph{CIFAR-10 and CIFAR-100} CIFAR-10 ~\citep{krizhevsky2009learning} and CIFAR-100 ~\citep{krizhevsky2009learning} are well-known classic image classification datasets. Both CIFAR-10 and CIFAR-100 contain a total of 60k $32 \times 32$ labeled images of different classes, with 50k for training and 10k for testing. CIFAR-10 is similar to CIFAR-100, except there are 10 different classes in CIFAR-10 and 100 classes in CIFAR-100.

\paragraph{TinyImageNet} TinyImageNet ~\citep{le2015tiny} is a subset of ImageNet ~\citep{deng2009imagenet}. There are 200 different object classes in TinyImageNet, with 500 training images, 50 validation images, and 50 test images for each class. All the images in TinyImageNet are colored and labeled with a size of $64 \times 64$.

\textbf{Pseudo-code.} Algorithm \ref{alg:Training Procedure} presents the pseudo-code for our empirical training procedure.

\begin{algorithm}[!htbp]
\caption{Training Procedure}
\label{alg:Training Procedure}
\begin{algorithmic}[1]
\REQUIRE trainable encoder network $f$, batch size $N$, augmentation strategy \textit{aug}, loss function $L$ with hyperparameters \textit{args}
\FOR {sampled minibatch ${x_i}_{i=1}^N$}
\FORALL{$i \in { 1, ..., N }$}
\STATE draw two augmentations $t_i = \textit{aug}\left(x_i\right) $, $t_i' = \textit{aug}\left(x_i\right) $
\STATE $z_i = f\left(t_i\right)$, $z_i' = f\left(t_i'\right)$
\ENDFOR
\STATE compute loss $\mathcal{L} = L(N, z, z', \textit{args})$
\STATE update encoder network $f$ to minimize $\mathcal{L}$
\ENDFOR
\STATE \textbf{Return} encoder network $f$
\end{algorithmic}
\end{algorithm}

We also provide the pseudo-code for our core loss function used in the training procedure in Algorithm \ref{alg:Core loss}. The pseudo-code is almost identical to SimCLR's loss function, with the exception of an extra parameter $\gamma$.

\begin{algorithm}[!htbp]
\caption{Core loss function $\mathcal{C}$}
\label{alg:Core loss}
\begin{algorithmic}[1]
\REQUIRE batch size $N$, two encoded minibatches $z_1, z_2$, $\gamma$, temperature $\tau$
\STATE $z = \textit{concat}\left(z_1, z_2\right)$
\FOR {$i \in {1, ..., 2N }, j \in {1, ..., 2N}$ }
\STATE $s_{i,j} = \Vert z_i - z_j \Vert_2^{\gamma}$
\ENDFOR
\STATE \textbf{define} $l(i, j)$ \textbf{as} $l(i, j) = - \log \frac{exp\left(s_{i,j}/\tau \right)}{\sum_{k=1}^{2N} \mathbf{1}{[k \ne i]} exp\left(s{i, j} / \tau \right)} $
\STATE \textbf{Return} $\frac{1}{2N} \sum_{k=1}^N\left[l(i, i+N) + l(i+N, i)\right]$
\end{algorithmic}
\end{algorithm}

Utilizing the core loss function $\mathcal{C}$, we can define all kernel loss functions used in our experiments in Table \ref{table: loss definition}. For all $z_i \in z$ with even dimensions $n$, we define $z_{L_i} = z_i\left[0:n/2\right]$ and $z_{R_i} = z_i\left[n/2:n\right]$.

\begin{table}[ht]
\centering
\begin{tabular}{{@{}l|l@{}}}
Kernel  &  Loss function \\ \midrule
Laplacian & $\mathcal{C}\left(N, z, z', \gamma=1, \tau\right)$\\ \midrule
Sum       & $\lambda * \mathcal{C}\left(N, z, z', \gamma=1, \tau_1\right) + (1-\lambda) * \mathcal{C}\left(N, z, z', \gamma=2, \tau_2\right)$  \\ \midrule
Concatenation Sum&$\lambda * \mathcal{C}\left(N, z_L, z'_L, \gamma=1, \tau_1\right) + (1-\lambda) * \mathcal{C}\left(N, z_R, z'_R, \gamma=2, \tau_2\right)$\\ \midrule
$\gamma = 0.5$ & $\mathcal{C}\left(N, z, z', \gamma=0.5, \tau\right)$          \\ 

\end{tabular}

\caption{Definition of kernel loss functions in our experiments}
\label {table: loss definition}
\end{table}

\textbf{Baselines.} We reproduce the SimCLR algorithm using PyTorch Lightning~\citep{PytorchLightning}.

\textbf{Encoder details.}
The encoder $f$ consists of a backbone network and a projection network. We employ ResNet50~\citep{ResNet} as the backbone and a 2-layer MLP (connected by a batch normalization~\citep{ioffe2015batch} layer and a ReLU \cite{nair2010rectified} layer) with hidden dimensions 2048 and output dimensions 128 (or 256 in the concatenation kernel case).

\textbf{Encoder hyperparameter tuning.}
For each encoder training case, we randomly sample 500 hyperparameter groups (sample details are shown in Table \ref{table: Hyperparameter sample}) and train these samples simultaneously using Ray Tune ~\citep{RayTune}, with the ASHA scheduler~\citep{li2018massively}. Ultimately, the hyperparameter group that maximizes the online validation accuracy (integrated in PyTorch Lightning) within 5000 validation steps is chosen for the given encoder training case.

\begin{table}[ht]
\centering

\begin{tabular}{@{}l|l|l@{}}
\midrule
Hyperparameter  & Sample Range & Sample Strategy \\ \midrule
start learning rate & $\left[10^{-2}, 10\right]$ & log uniform \\ \midrule
$\lambda$       & $\left[0, 1\right]$ & uniform \\ \midrule
$\tau$, $\tau_1$, $\tau_2$ & $\left[0, 1\right]$ & log uniform \\ \midrule
\end{tabular}

\caption{Hyperparameters sample strategy}
\label {table: Hyperparameter sample}
\end{table}

\textbf{Encoder training.} 
We train each encoder using the LARS optimizer~\citep{LARSOptimizer}, LambdaLR Scheduler in PyTorch, momentum 0.9, weight decay $10^{-6}$, batch size 256, and the aforementioned hyperparameters for 400 epochs on a single A-100 GPU.

\textbf{Image transformation.} The image transformation strategy, including augmentation, is identical to the default transformation strategy provided by PyTorch Lightning.

\textbf{Linear evaluation.}
The linear head is trained using the SGD optimizer with a cosine learning rate scheduler, batch size 64, and weight decay $10^{-6}$ for 100 epochs. The learning rate starts at $0.3$ and ends at $0$.

\textbf{Moco Experiments.} We also tested our method based on MoCo~\citep{he2019moco}. The results are summarized in Table \ref{tab:results-moco}. Here we choose ResNet18~\citep{ResNet} as the backbone and set a temperature of $0.1$ as default. For our simple sum kernel, we set $\lambda=0.8$. The results show that our method outperforms the original MoCo method.

\begin{table}[thb]
\centering
\caption{MoCo Experiment Results on CIFAR-10 and CIFAR-100.}
\label{tab:results-moco}
\resizebox{\textwidth}{!}{%
\begin{tabular}{@{}c|ccc|ccc@{}}
\toprule
\multirow{3}{*}{Method} & \multicolumn{3}{c|}{CIFAR-10} & \multicolumn{3}{c}{CIFAR-100} \\ \cmidrule(lr){2-4} \cmidrule(lr){5-7} 
                        & 200 epochs & 400 epochs    & 1000 epochs   & 200 epochs & 400 epochs & 1000 epochs         \\ \midrule
MoCo (repro.)         & $76.41 \pm 0.12$    & $80.01 \pm 0.15$          & $84.45 \pm 0.08$    & $\mathbf{47.02 \pm 0.11}$ & $52.50 \pm 0.07$ & $57.62 \pm 0.15$            \\
\midrule
Laplacian Kernel        & ${78.09 \pm 0.10}$    & $\mathbf{83.85 \pm 0.09}$          & $\mathbf{88.34 \pm 0.16}$    & $46.12 \pm 0.22$   & $53.44 \pm 0.17$ & $59.10 \pm 0.14$        \\
Simple Sum Kernel & $\mathbf{78.12 \pm 0.15}$   & $83.23 \pm 0.18$ & $87.50 \pm 0.20$ & $46.65 \pm 0.06$ & $\mathbf{53.62 \pm 0.19}$ & $\mathbf{59.83 \pm 0.12}$\\
\bottomrule
\end{tabular}
}
\end{table}



\section{More Experiments on Synthetic Data}


Consider a scenario with $n$ clusters, each containing $k$ vertices. Let the probability of vertices $u$ and $v$ from the same cluster belonging to $\bfpi$ be $p$. Conversely, for vertices $u$ and $v$ from different clusters, let the probability of belonging to $\pi$ be $q$. We generate the graph $\bfpi$ randomly, based on $p$ and $q$. We experiment with values of $k=100$ and $n=6$ for ease of visualization, embedding all points in a two-dimensional space. Each vertex's initial position originates from a normal distribution. In each iteration, we sample a subgraph of $\bfpi$ uniformly, ensuring each vertex has an out-degree of $1$. We then optimize the corresponding vectors using InfoNCE loss with an SGD optimizer and iterate until convergence. Our experimental setup consists of an SGD learning rate of $1$, an InfoNCE loss temperature of $0.5$, and a batch size of $50$. We evaluate two scenarios with different $p$ and $q$ values: $p=1$, $q=0$, and $p=0.75$, $q=0.2$. The results of these experiments are visualized in Figure \ref{fig:vis-spectral-cluster}. The obtained embeddings exhibit the hallmark pattern of spectral clustering of graph $\bfpi$.

\begin{figure}[!tb]
\centering
\subfigure{
\includegraphics[width=1\textwidth]{Figures/cluster_pi.png}
\label{fig:vis-cluster}
}
\subfigure{
\includegraphics[width=1\textwidth]{Figures/noised_cluster_pi.png}
\label{fig:vis-noised-cluster}
}
\caption{Visualizations of the optimization process using InfoNCE Loss on the vectors corresponding to $\bfpi$. Points of identical color belong to the same cluster within $\bfpi$. To showcase the internal structure of $\bfpi$, we randomly select 10 vertices from each cluster to display the edge distribution of $\bfpi$.}
\label{fig:vis-spectral-cluster}
\end{figure}



\end{document}
