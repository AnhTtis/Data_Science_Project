\section{Applications}
\label{sec:applications}

We now describe some applications of Hermes-3, starting from a 1D
transport model (section~\ref{sec:1d-transport}); then a 2D transport
model in 2D (axisymmetric) tokamak geometry
(section~\ref{sec:applications-2d}).  Finally we demonstrate
time-dependent capabilities by simulating 2D (drift plane) plasma
``blobs'' (section~\ref{sec:applications-blobs}). Application of these
capabilities to 3D turbulence simulations is relatively
straightforward, but requires significantly more space to describe
adequately and so will be explored in separate publications.

\subsection{1D transport}
\label{sec:1d-transport}

We first apply Hermes-3 to a one-dimensional problem: the flow of heat
and particles along a magnetic flux tube which is in contact with a
material surface. The model includes electrons, deuterium ions and
neutral deuterium atoms. One end of the domain is modelled as being in
contact with a material surface, forming a plasma sheath and
accelerating ions to the sound speed. The flow of ions to the surface
is ``recycled'' back into the domain as neutral atoms, which then
undergo charge exchange and ionisation reactions with the plasma. The
other end of the domain has a symmetry (no-flow) boundary condition,
where thermodynamic variables (e.g. densities, pressures) have zero
gradient, and flow velocities are zero. This is a widely used model
for the divertor region of tokamak plasmas, which has several
implementations of varying
complexity~\cite{nakazawa-2000,goswami-2001,nakamura2011,togo2013,havlickova},
including the SD1D model~\cite{dudson2019,bout:sd1d} which like
Hermes-3 is built on BOUT++~\cite{dudson2015}.

Despite its relative simplicity this model contains many of the
nonlinearities and numerically stiff behaviour which make 2D and 3D
plasma simulations challenging, including strong nonlinear heat
diffusion, and fast reaction rates which are sensitive to electron
temperature.

The components to be included in the simulation (section~\ref{sec:software-arch}) are specified
in an input text file; the relevant line is shown in listing~\ref{lst:hydrogen-hermes}.
\begin{lstlisting}[language=Ini,
    caption={Top-level components for 1D hydrogen transport model. Parentheses are
      used to group multi-line settings.
      Full input in \texttt{examples/1D-recycling} of the Hermes-3 repository~\cite{dudson:hermes3}},
    label={lst:hydrogen-hermes}]
[hermes]
components = (d+, d, e,
              sheath_boundary, collisions, recycling, reactions,
              electron_force_balance, neutral_parallel_diffusion)
\end{lstlisting}
The boundary condition at the material surface, implemented by the
\texttt{sheath\_boundary} component in
listing~\ref{lst:hydrogen-hermes}, is the multi-ion sheath boundary
described in~\cite{tskhakaya2005}. For the single ion species here
this reduces to the standard Bohm-Chodura-Riemann sheath boundary
condition~\cite{stangeby-1995}.  The \texttt{collisions} component
implements collisions between an arbitrary number of charged and
neutral species.  Reactions between species are organised into a
subsection called \texttt{reactions}, and are chosen to have names
which are readable and follow a convention for the species labels.
\begin{lstlisting}[language=Ini,
    caption={Reactions contained in the 1D hydrogen transport model},
    label={lst:hydrogen-hermes-reactions}]
[reactions]
type = (
        d + e -> d+ + 2e,     # Deuterium ionisation
        d+ + e -> d,          # Deuterium recombination
        d + d+ -> d+ + d,     # Charge exchange
        )
\end{lstlisting}
Reaction cross-sections for hydrogen and helium have been taken from
the Amjuel database~\cite{amjuel}. 

Each particle species has components to evolve the density, pressure
and parallel momentum, and a no-flow boundary condition imposed on the
upstream boundary.  To illustrate how Hermes-3 components combine to
form the equations solved, the \texttt{d+} (deuterium ion) species
settings are shown in listing~\ref{lst:hydrogen-hermes-d+}.
\begin{lstlisting}[language=Ini,
    caption={Components to model the deuterium ion species},
    label={lst:hydrogen-hermes-d+}]
[d+]  # Deuterium ions
type = (evolve_density, evolve_pressure, evolve_momentum,
        noflow_boundary, upstream_density_feedback)
charge = 1  # charge
AA = 2      # mass [amu]
density_upstream = 1e19 # Upstream density [m^-3]
recycle_as = d          # Species to recycle as
recycle_multiplier = 1  # Recycling fraction
\end{lstlisting}
This implements a set of equations for the density $n_{d+}$, pressure
$p_{d+} = en_{d+}T_{d+}$ and parallel velocity $v_{||d+}$ of the
deuterium ions (\texttt{d+}) of mass $m_{d+}$ and charge $q_{d+}$,
given in equation~\ref{eq:1d-hydrogen-d+}. These are in SI units
except temperatures in eV.
\begin{subequations}
  \label{eq:1d-hydrogen-d+}
  \begin{align}
    \frac{\partial n_{d+}}{\partial t} &= -\nabla\cdot\left(n_{d+}\mathbf{b}v_{||d+}\right) \nonumber \\
    & + \underbrace{S_{\textrm{PI}}}_{\texttt{upstream\_density\_feedback}} + \underbrace{n_en_d\left<\sigma v\right>_{iz}}_{\texttt{d + e -> d+ + 2e}} - \underbrace{n_en_{d+}\left<\sigma v\right>_{rc}}_{\texttt{d+ + e -> d}} \\
    \frac{\partial p_{d+}}{\partial t} &= -\nabla\cdot\left(p_{d+}\mathbf{b}v_{||d+}\right) - \frac{2}{3}p_{d+}\nabla\cdot\left(\mathbf{b}v_{||d+}\right) + \nabla\cdot\left(\kappa_{||d+}\mathbf{b}\mathbf{b}\cdot\nabla T_{d+}\right) \nonumber \\
    & + \underbrace{n_en_d\left<\sigma v\right>_{iz}\left[eT_d + \frac{1}{2}m_d\left(v_{||,d} - v_{||,d+}\right)^2\right]}_{\texttt{d + e -> d+ + 2e}} \nonumber \\
    & + \underbrace{n_{d+}n_{d}\left<\sigma v\right>_{cx}\left[e\left(T_d - T_{d+}\right) + \frac{1}{2}m_d\left(v_{||,d} - v_{||,d+}\right)^2\right]}_{\texttt{d + d+ -> d+ + d}} \nonumber \\
    & - \underbrace{n_en_{d+}\left<\sigma v\right>_{rc}eT_{d+}}_{\texttt{d+ + e -> d}} + \underbrace{W_{d+}}_{\texttt{collisions}} \\
    \frac{\partial}{\partial t}\left(m_{d+}n_{d+}v_{||d+}\right) &= -\nabla\cdot\left(n_{d+}v_{||d+}\mathbf{b}v_{||d+}\right) - \mathbf{b}\cdot\nabla p_{d+} + q_{d+}n_{d+}E_{||}\nonumber \\
    & + \underbrace{n_en_d\left<\sigma v\right>_{iz}m_dv_{||d}}_{\texttt{d + e -> d+ + 2e}} - \underbrace{n_en_{d+}\left<\sigma v\right>_{rc}m_{d+}v_{||d+}}_{\texttt{d+ + e -> d}} \\
    & + \underbrace{n_{d+}n_{d}\left<\sigma v\right>_{cx}m_{d+}\left(v_{||d} - v_{||d+}\right)}_{\texttt{d + d+ -> d+ + d}}\underbrace{F_{d+}}_{\texttt{collisions}}
  \end{align}
\end{subequations}
Where $\mathbf{b}\equiv\mathbf{B}/B$ is the unit vector in the
direction of the magnetic field $\mathbf{B}$, and $\kappa_{||d+}$ is
the parallel heat conduction coefficient that depends on the collision
frequency calculated by the \texttt{collisions} component.  For each
equation~\ref{eq:1d-hydrogen-d+} the first line corresponds to the
essential transport terms implemented in the \texttt{evolve\_density},
\texttt{evolve\_momentum} and \texttt{evolve\_pressure} components.
Additional components, labelled with underbraces in
equation~\ref{eq:1d-hydrogen-d+}, add sources and sinks that modify
and couple species together.  Details of the full system of 7 evolving
equations are given in~\ref{apx:transport}.

The equations are integrated in time towards a steady state solution
using the backward Euler method described in
section~\ref{sec:time-integration}. The result is shown in
figure~\ref{fig:hydrogen_solution}.
\begin{figure}[h]
  \centering
  \includegraphics[width=0.9\textwidth]{content/1d_recycling.pdf}
  \caption{Steady-state solution to system of
    equations~\ref{eq:1d-hydrogen-d+} and \ref{apx:transport} in one
    dimension. 50MW of power enters a source region on the left,
    driving plasma-neutral interactions including ionisation, leaving
    through the sheath boundary on the right. 100\% of the plasma ions
    leaving the right boundary are recycled as neutral atoms.}
  \label{fig:hydrogen_solution}
\end{figure}
The root-mean-square of the time-derivatives of ion density, pressure
and parallel momentum are shown in
figure~\ref{fig:timederivs-rhsevals} as a function of the number of
right-hand-side (RHS) evaluations, a measure of the computational
cost. This evaluation count includes those performed as part of the
finite difference Jacobian approximation. This shows a reduction in
the time derivatives of the system by almost six orders of magnitude
in $10^5$ RHS evaluations. For the above system of equations, with 400
grid cells, this calculation takes approximately 5 minutes on a single
core.
\begin{figure}[h]
  \centering
  \includegraphics[width=0.9\textwidth]{content/timederivs_rhsevals.pdf}
  \caption{Root-mean-square time derivatives of deuterium density
    (Nd+), pressure (Pd+) and momentum (NVd+) as a function of
    iteration (RHS evaluation). These converge towards zero as the
    system approaches steady state. Results are shown for Backward
    Euler Newton-Krylov with Jacobian coloring and iLU preconditioning
    (NK); and the PVODE time integrator. Figure produced by
    \texttt{examples/1D-recycling/plot\_convergence.py}.}
  \label{fig:timederivs-rhsevals}
\end{figure}
For comparison the convergence towards steady state with the Sundials
CVODE~\cite{hindmarsh2005} library is shown in
figure~\ref{fig:timederivs-rhsevals}.  CVODE uses an adaptive order,
adaptive timestep Backward Differentiation Formula (BDF) method, and
is highly effective for time-dependent problems of interest even
without preconditioning (e.g.  the plasma blobs application,
section~\ref{sec:applications-blobs}). For this problem only the
parallel heat conduction is preconditioned.
Figure~\ref{fig:timederivs-rhsevals} shows that the Backwards-Euler
with Jacobian coloring preconditioner method can provide significantly
better performance for steady-state problems, though it would not be a
good choice for time-dependent simulations, being only first order
accurate in time.

In this simulation the recycling at the ``target'' end of the domain
was set to 100\%, while there is a no-flow condition on the upstream
boundary. A Proportional-Integral (PI) controller is used to control
an upstream particle source; as the target upstream density is
approached, this input source should go to zero if mass flow is
conserved.  Figure~\ref{fig:source_rhsevals} shows that this does
indeed happen: The source converges towards exponentially towards zero
in steady state.
\begin{figure}[h]
  \centering
  \includegraphics[width=0.9\textwidth]{content/source_rhsevals.pdf}
  \caption{Convergence of the density source with RHS evaluation in a
    1D simulation with 100\% recycling
    (figure~\ref{fig:hydrogen_solution}) where the true steady state
    source is zero. Figure produced by
    \texttt{examples/1D-recycling/plot\_convergence.py}.}
  \label{fig:source_rhsevals}
\end{figure}

The 1D simulation described here is a useful tool in its own right, for
studies of plasma dynamics and detachment in magnetised plasmas. The
advantage of the software design used in Hermes-3
(section~\ref{sec:software}) is that the same code can extend to more
complex models and higher dimensions with only changes to the input.

\subsubsection{Impurity seeding}

We now extend the 1D simulation described in
section~\ref{sec:1d-transport} to multiple ion species, by including
all ten charge states of neon as separate species.  The simulation now
contains 40 evolving fields: The density, pressure and momentum of all
deuterium and neon ion and atomic charge states (13 ion species in
total), and the electron pressure. These species are coupled through
collisions, thermal forces, the parallel electric field, and 32 atomic
reactions: ionisation, 3-body recombination and charge exchange
recombination (with deuterium ions) of each ionisation level of neon;
ionisation and charge exchange of neutral deuterium atoms to deuterium
ions. The modular structure of the code
(section~\ref{sec:software-arch}) enables this to be accomplished
relatively straightforwardly by changing the input file.

\begin{lstlisting}[language=Ini,
    caption={Top-level components for 1D transport model with neon.
      Input \texttt{examples/1D-neon}},
    label={lst:neon-hermes}]
[hermes]
components = (d+, d, ne, ne+, ne+2, ne+3, ne+4, ne+5, ne+6,
              ne+7, ne+8, ne+9, ne+10, e, sheath_boundary,
              thermal_force, collisions, recycling, reactions,
              electron_force_balance, neutral_parallel_diffusion)
\end{lstlisting}
There are now many more reactions, but the input remains clear:

\begin{lstlisting}[language=Ini,
    caption={Reactions contained in the 1D transport model with neon.},
    label={lst:neon-hermes-reactions}]
[reactions]
type = (
        d + e -> d+ + 2e,     # Deuterium ionisation
        d + d+ -> d+ + d,     # Charge exchange

        ne + e -> ne+ + 2e,   # Neon ionisation
        ne+ + e -> ne,        # Neon+ recombination
        ne+ + d -> ne + d+,   # Neon+ charge exchange recombination

        ...
        )
\end{lstlisting}
The cross-sections and radiated power from the neon reactions are
calculated using ADAS~\cite{adas}: \texttt{scd96} and \texttt{plt96} for
ionisation; \texttt{acd96} and \texttt{prb96} for recombination;
\texttt{ccd89} for charge exchange. These files were converted
to JSON format using atomic++~\cite{atomicpp}.

Collisions and the thermal forces between species are calculated as
described in section~\ref{sec:1d-transport}. Those Braginskii energy
and momentum exchange rates are approximations which are only strictly
valid when heavy ions are trace impurities. In the simulations shown
here the neon concentration is small (a fraction of a percent). More
complete models of collisions in a multi-ion plasma have been
derived~\cite{zhdanov2002} and recently
generalised~\cite{raghunathan2021}. Implementing these models into
Hermes-3 is left as future work, but is not anticipated to present any
fundamental difficulty.

Starting from the hydrogen simulation described above
(section~\ref{sec:1d-transport}), a simulation with 100\% recycling
and an initial uniform low concentration of neon is run to steady
state, and shown in figure~\ref{fig:1d-neon-steadystate}.  In this
simulation there is no net flow upstream of the ionisation region, and
so thermal forces drive neon impurities upstream.
\begin{figure}[h]
  \centering
  \includegraphics[width=0.9\textwidth]{content/1d-neon-steadystate.pdf}
  \caption{Steady state solution with 100\% recycling, evolving all
    charge states of neon as separate fluids with their own densities,
    temperatures and flow velocities. A subset of species densities
    (blue lines) are shown on a logarithmic scale. Simulation inputs
    in \texttt{examples/1D-neon} of the Hermes-3 repository.}
  \label{fig:1d-neon-steadystate}
\end{figure}
The steady-state solution is shown in
figure~\ref{fig:1d-neon-steadystate}. Future applications of this
capabilty include simulating impurity-seeded plasma detachment
phenomena.

\subsection{2D (axisymmetric) transport}
\label{sec:applications-2d}

The same code that is used in a 1D domain in the previous sections can
be applied to 2D tokamak domains with one or two X-points. By
introducing cross-field diffusion of both charged and neutral species,
an axisymmetric tokamak transport simulation in the spirit of
SOLPS~\cite{schneider-2006}, EDGE2D~\cite{simonini-1994} or
UEDGE~\cite{rognlien-2002} can be performed, though not yet at a
comparable level of maturity or completeness.  To demonstrate the
ability of Hermes-3 to solve axisymmetric transport problems,
simulations are performed with deuterium ions and neutral atoms.
Diffusion coefficients and plasma parameters are taken
from~\cite{hromasova-2021,hromasova-thesis}: Spatially constant
cross-field diffusion coefficients for particle transport
$D_n=0.15$m$^2/$s, electron and ion thermal transport $\chi_e = \chi_i
= 4$m$^2/$s. In general these coefficients can be functions of
location, and can be different for each species.

The plasma equilibrium is based on a COMPASS-like equilibrium
generated using analytic Grad-Shafranov solutions~\cite{cerfon:2010,
  omotani:cfg}.  The domain simulated is shown in
figure~\ref{fig:2d_domain}, consisting of a narrow annulus around the
separatrix (dashed black lines in figure~\ref{fig:2d_domain})
including closed and open field line regions. The radial boundaries
are at normalised psi of 0.9 in the core and in the private flux
region (PFR), and 1.3 in the Scrape-Off Layer (SOL). The Hypnotoad
tool~\cite{hypnotoad} was used to generate a sequence of grids of
increasing resolution from $16\times 24$ to $64\times 96$ (radial
$\times$ poloidal cells).
\begin{figure}
  \centering
  \begin{subfigure}[h]{0.48\textwidth}
    \centering
    \includegraphics[width=\textwidth]{content/68x96_ne.pdf}
    \caption{Electron density. The core boundary is fixed to a density
      of $1\times 10^{19}$m$^{-3}$. At divertor targets 99\% of ion
      flux is recycled as neutral atoms.}
    \label{fig:2d_ne}
  \end{subfigure}
  \hfill
  \begin{subfigure}[h]{0.48\textwidth}
    \centering
    \includegraphics[width=\textwidth]{content/68x96_te.pdf}
    \caption{Electron temperature. At the core boundary both electron and ion temperatures
      are fixed to $75$eV.}
    \label{fig:2d_te}
  \end{subfigure}
  \caption{Axisymmetric tokamak transport simulation of deuterium ions
    and atoms. Equations given in~\ref{apx:transport}. Simulation inputs
    in \texttt{examples/tokamak/recycling} of the Hermes-3 repository.}
  \label{fig:2d_domain}
\end{figure}

As in previous examples, the equations solved are specified as a set
of components:
\begin{lstlisting}[language=Ini,
    caption={Top-level components for 2D transport model.
      Full input in \texttt{examples/tokamak/recycling} of the Hermes-3 repository.},
    label={lst:hermes-2d-components}]
components = (d+, d, e,
              collisions, sheath_boundary_simple, 
              recycling, sound_speed, reactions,
              electron_force_balance)
\end{lstlisting}
The deuterium ion species is configured with a set of components
representing the equations solved, given in
listing~\ref{lst:hermes-2d-d1}

\begin{lstlisting}[language=Ini,
    caption={Deuterium ion components for 2D transport model},
    label={lst:hermes-2d-d1}]
[d+]
type = (evolve_density, evolve_momentum, evolve_pressure,
        anomalous_diffusion)
anomalous_D = 0.15   # Density diffusion [m^2/s]
anomalous_chi = 4    # Thermal diffusion [m^2/s]
...
\end{lstlisting}
which is similar to the configuration in 1D simulations given in
listing~\ref{lst:hydrogen-hermes-d+}, but adds anomalous cross-field
diffusion terms.  Reactions between species are calculated using
Amjuel rates~\cite{amjuel}, comprising ionisation, recombination, and
charge exchange processes as described in
section~\ref{sec:1d-transport}.

At the inner (core) boundary the deuterium density is fixed to
$1\times 10^{19}$m$^{-3}$; electron and ion temperatures are set to
$75$eV.  This core boundary therefore acts as a source of heat and
particles. At the target plates a sheath boundary condition is applied
in which the plasma flow goes to the sound speed, with a recycling
fraction of 0.99 so that there is a flux of neutral atoms into the
domain at the target plates. The 1\% of ion flux that is not recycled
is balanced in steady state by a diffusion of ions from the core
boundary. This particle flux balance will be used in
section~\ref{sec:2d_convergence} to verify the conservation of
particles in these simulations.

The heat flux along the magnetic field into the target plates is given
by $q_{||e,i} = \gamma_{e,i}neTc_s$ with sheath heat transmission
factors $\gamma_e=4.8$ for electrons and $\gamma_i=3.5$ for ions. The
sound speed into the sheath is $c_s = \sqrt{e\left(T_i +
  T_e\right)/m_i}$.  There are no diffusive fluxes to the outer walls
because zero-gradient boundary conditions are used there. The power
into the target plates should therefore equal the input power through
the core boundary, less the power radiated during atomic processes
(primarily ionisation). This is used in
section~\ref{sec:2d_convergence} to assess conservation of energy.

The full set of equations solved are given in \ref{apx:transport}.

\subsubsection{Evolution to steady state}

The system of transport equations is relatively small (e.g. 10,752
variables for the $32\times 48$ mesh) but highly nonlinear and with a
wide range of timescales, making finding steady state solutions
challenging. Simple application of a nonlinear solver does not
converge in most cases of interest, and the system must be regularised
using a (pseudo-)timestepping approach. As the system approaches
steady state the timestep can be made progressively larger, usually in
an automated manner based on the number of nonlinear iterations
required to converge the previous step. The combination of time
integration method, time step adjustment heuristic, nonlinear solver,
inner linear iterative solver, and preconditioner have many parameters
that can affect performance. The nested methods interact in ways that
are problem-dependent, making general conclusions regarding
performance difficult to draw. For the present 2D problem is has been
found that CVODE generally converges more quicky than Backward Euler,
though can be more susceptible to numerical oscillations that reduce
with tightened tolerance.

In general power balance reaches steady-state on a shorter timescale
than the particle balance: The thermal energy content of the system
(plasma + neutrals) is $W\simeq 47$J, so the energy confinement time
is $\tau_E \equiv W / P_{in}\simeq 0.24$ms. On the other hand the ion
particle content is approximately $2\times 10^{18}$, giving a particle
throughput timescale of $\tau_p\simeq 45$ms. This longer particle
balance timescale becomes increasingly challenging at high recycling
fractions relevant to large fusion devices.

An effective strategy, already used routinely in UEDGE, is progressive
mesh refinement: Starting on the coarsest mesh ($16\times 24$ here),
CVODE is used with an absolute tolerance of $10^{-12}$ and relative
tolerance $10^{-5}$, tightening the relative tolerance to $10^{-8}$ as
steady state is approached. These tolerances can be loosened in some
cases, but at the risk of numerical instability and convergence
failure after a number of steps.  Once progress has been made on a
coarse mesh, the solution is interpolated onto a higher resolution
mesh (using SciPy's RegularGridInterpolator~\cite{2020SciPy-NMeth}
over logically rectangular mesh patches). The simulation is then
continued using an increased number of cores. The refinement process
may be repeated.  Figure~\ref{fig:2d_history} shows the
Root-Mean-Square (RMS) of the time derivatives of the plasma density
averaged over the domain, as a function of wall clock time (running on
NERSC's Perlmutter).
\begin{figure}
  \centering
  \includegraphics[width=\textwidth]{content/history_log.pdf}  
  \caption{Evolution of the Root-Mean-Square (RMS) time derivative
    residuals. Vertical red arrows indicate where the solution is
    interpolated onto a higher resolution mesh. Blue stars are the
    solutions that are compared in section~\ref{sec:2d_convergence}.
    The number of cores used is increased with the grid resolution:
    $12$ ($16\times 24$ mesh), $48$ ($32\times 48$ mesh) and $192$
    ($64\times 96$ mesh). }
  \label{fig:2d_history}
\end{figure}
For each mesh resolution the simulation was continued after
interpolation, until the Root-Mean-Square time scale exceeded one
second, to minimise the impact of mesh interpolation error on the
comparison of solutions. As the mesh was refined the number of cores
used was increased following a weak scaling. The increase in run time
with grid resolution in figure~\ref{fig:2d_history} is primarily
driven by the number of iterations required: $3.1\times 10^6$
($16\times 24$ mesh), $1.2\times 10^7$ ($32\times 48$ mesh) and
$3.7\times 10^7$ ($64\times 96$ mesh). The time per iteration (RHS
evaluation) is 1.7ms, 2.0ms and 4.3ms respectively. Extending the
Backward Euler method and coloring preconditioner used in 1D
simulations (section~\ref{sec:1d-transport}) to these 2D simulations
is a high priority for future development, in order to reduce the
number of iterations required for convergence.

\subsubsection{Convergence and accuracy}
\label{sec:2d_convergence}

The accuracy of the methods are now assessed by examining the
conservation properties and convergence of the solution with mesh
resolution. Figure~\ref{fig:2d_outer_target} shows the profiles of
density and temperature along the outer target, for each mesh
resolution.
\begin{figure}
  \centering
  \includegraphics[width=\textwidth]{content/outer_target.pdf}
  \caption{Electron temperature and density at the outer target.
    Dotted: $16\times 24$ resolution; Dashed: $32\times 48$; Solid: $64\times 96$.}
  \label{fig:2d_outer_target}
\end{figure}
Low resolution meshes broaden the profiles of both density and
temperature relative to high resolution cases: Numerical dissipation
enhances the effective cross-field diffusion. Given the fixed
(Dirichlet) core boundary conditions, this enhanced diffusion
increases the power into the domain at low resolution.  It also be
seen in figure~\ref{fig:2d_outer_target} that the meshes do not have a
consistent boundary location: Due to boundary cell locations, as the
grid is refined the outer edges of the domain converge at first order
to the specified poloidal flux values.  This will limit the global formal
convergence to at best first order unless improvements are made to the
mesh generator.

To assess power and particle balance, table~\ref{tab:balances} lists
the flows of power and particles into and out of the domain, for each
mesh resolution.
\begin{table}[h!]
  \centering
  \caption{Global power and particle balance in 2D transport simulations}
  \label{tab:balances}
  \begin{tabular}{ l c c c }
    & $16\times 24$ & $32\times 48$ & $64\times 96$ \\ 
    \hline
    Input power [kW]                   & 195.0  & 174.4  & 162.4 \\
    Power to outer target [kW]         & 104.6  &  92.5  &  85.9 \\
    Power to inner target [kW]         &  75.4  &  64.9  &  59.5 \\
    Power to atomics [kW]              &  22.8  &  19.3  &  17.6 \\
    Power balance error [kW]           &   7.8 (4.0\%)  &   2.3  (1.3\%) &   0.59 (0.36\%) \\
    \hline
    Input ion flux [$10^{19}$/s]        &  5.31  &  4.56  & 4.15 \\
    Flux to outer target [$10^{19}$/s]  &   278  &   247  &  230 \\
    Flux to inner target [$10^{19}$/s]  &   284  &   229  &  204 \\
    Recycling fraction [0.99]          & 0.9909 & 0.9903 & 0.9904
  \end{tabular}
\end{table}
Power enters the domain through the inner (core) boundary, where
Dirichlet boundary conditions are set on density and temperatures so
that power crossing this boundary depends on the local gradients. The
target temperatures in figure~\ref{fig:2d_outer_target} are well above
the $5$eV typical for detachment, so these simulations are in attached
conditions and most of the power goes to the outer and inner targets.
Some power is lost through atomic processes, both to overcome
ionisation potentials and through radiation. Power to atomics includes
the deuterium ionisation potential so this potential energy flux is
not included in the power to outer and inner targets listed in
table~\ref{tab:balances}.  As noted in
section~\ref{sec:finite_differencing} pressure equations are evolved
rather than energy, so that energy conservation is in general not
exact but converges as the mesh is refined. For comparison, a 1\%
power balance error has been used as a SOLPS-ITER convergence
criterion~\cite{WIESEN2015480}.

Particle fluxes are shown in Table~\ref{tab:balances} as the flux into
the domain through the inner boundary, and the fluxes to inner and
outer targets. Due to the imposed recycling fraction of 0.99, we
expect 1\% of the flux to the targets to be lost (pumped), and
replaced by a matching flux of ions into the domain from the core. The
core and target fluxes are therefore used to infer the recycling
fraction in Table~\ref{tab:balances}. If particle balance is achieved
then that fraction should match the 0.99 value set. We find this to be
well matched: Particle conservation is significantly easier to achieve
in this system of equations than energy conservation, and these
results demonstrate that all advection and diffusion operators,
recycling and atomic processes, properly conserve particle fluxes.

\subsection{2D (drift-plane) blobs}
\label{sec:applications-blobs}

We now turn from steady-state transport problems to time-dependent
problems involving an evolving vorticity equation and electrostatic
potential $\phi$. The development of this capability towards full 3D
turbulence, particularly in the presence of multiple ion species, will
be the subject of a future publication. As an initial step and proof
of principle, we present here some examples of 2D drift-plane
simulations of plasma ``blobs'' or filaments.

The significant lines in the input file which configure this model are
shown in listing~\ref{lst:blob2d}.
\begin{lstlisting}[language=Ini,
    caption={Component configuration for isothermal blob
      simulation. Full input in \texttt{examples/blob2d} of the Hermes-3 repository.},
    label={lst:blob2d}]
[hermes]
components = e, vorticity, sheath_closure

[e]  # Electrons
type = evolve_density, isothermal
charge = -1
AA = 1/1836      # Mass of species [amu]
temperature = 5  # Temperature in eV

[sheath_closure]
connection_length = 10 # meters
\end{lstlisting}
These set up components for the electron species density and
(isothermal) temperature, a vorticity equation, and a model
for the divergence of parallel current due to the sheath closure.
This corresponds to model equations
\begin{subequations}
  \label{eq:blob2d}
\begin{align}
  \frac{\partial n_e}{\partial t} &= - \nabla\cdot\left(n_e\frac{1}{B}\mathbf{b}\times\nabla\phi\right) + \underbrace{\nabla\cdot{\frac{1}{e}\mathbf{j}_{sh}}}_{\texttt{sheath\_closure}} \\
  p_e &= \underbrace{e n_e T_e}_{\texttt{isothermal}} \\
  \frac{\partial\Omega}{\partial t} &= - \nabla\cdot\left(\Omega \frac{1}{B}\mathbf{b}\times\nabla\phi\right) + \nabla\left(p_e\nabla\times\frac{\mathbf{b}}{B}\right) + \underbrace{\nabla\cdot\mathbf{j}_{sh}}_{\texttt{sheath\_closure}} \\
  \nabla\cdot\left(\frac{\overline{m_in}}{B^2}\nabla_\perp\phi \right) &= \Omega \label{eq:potential}
\end{align}
\end{subequations}
where $n_e$ is the electron density, $p_e$ the pressure and $T_e$ the
(fixed) temperature.  The Boussinesq approximation is used here, so
the potential $\phi$ is calculated from vorticity $\Omega$ using
equation~\ref{eq:potential} with a constant mass density
$\overline{m_in}$. The divergence of current density to the sheath is
$\nabla\cdot \mathbf{j}_{sh} = n_e\phi / L_{||}$ where $L_{||}$ is the
connection length ($10$m here).

The scaling of sheath-connected isothermal plasma blobs with blob size
is a well known test case, which can be derived analytically in the
limits of large blobs where sheath current balances the divergence of
diamagnetic current, and for small blobs where polarisation current
balances the divergence of diamagnetic current (see
e.g.~\cite{omotani-2015}). The blob size $\delta$ for which the
divergence of polarisation and sheath currents contribute
approximately equally is denoted $\delta^*$.

Simulations are started with a circular Gaussian density perturbation
(a plasma ``blob''), whose size perpendicular to the magnetic field
and the size of the simulation domain is varied.  Because we are
interested in time-varying solutions to these equations (propagation
of plasma blobs), the CVODE time integrator~\cite{hindmarsh2005} is
used, not the backward Euler method used in
section~\ref{sec:1d-transport}.  The result is shown in
figure~\ref{fig:blob-velocity}, reproducing well known scaling of
plasma blob velocity with blob size~\cite{omotani-2015}.
\begin{figure}[h]
  \centering
  \begin{subfigure}[h]{0.47\textwidth}
    \centering
    \includegraphics[width=\textwidth]{content/blob_velocity.pdf}
    \caption{Radial velocity $v_r$ of an isothermal plasma blob as a
      function of blob size $\delta$. Analytical scalings are
      $v_r\sim\sqrt{\delta}$ for $\delta/\delta^* \ll 1$, and
      $v_r\sim\left(\delta / \delta^*\right)^{-2}$ for $\delta /
      \delta^* \gg 1$.}
    \label{fig:blob-velocity}
  \end{subfigure}
  \begin{subfigure}[h]{0.47\textwidth}
    \centering
    \includegraphics[width=\textwidth]{content/blob2d-single.pdf}
    \caption{Electron density $n_e$ at $t=1500/\omega_{ci} = 44.7\mu$s
      for a $\delta^*$-sized seeded blob.}
    \label{fig:blob2d_density}
  \end{subfigure}
  \caption{Solution to equations~\ref{eq:blob2d} in a 2D domain
    perpendicular to the magnetic field, starting with a circular
    cross section density perturbation and driven by magnetic field
    curvature.  Input and analysis scripts in Hermes-3 repository
    \texttt{examples/blob2d}.}
  \label{fig:blob2d}
\end{figure}

This model extends quite straightforwardly to include hot ion effects
and separate ion and electron temperatures, by modifying the input
file to introduce a new species \texttt{h+} with a separate pressure
equation. The vorticity formulation is implemented such that the
polarisation current contribution of multiple species is included in
calculating the electric field; The self-consistent calculation of the
polarisation drift on the ion species density in a multi-ion species
calculation has recently been implemented and is being tested. Further
examples, tests and applications may be found in the Hermes-3
manual~\cite{dudson:hermes3-manual} and source code
repository~\cite{dudson:hermes3}.
