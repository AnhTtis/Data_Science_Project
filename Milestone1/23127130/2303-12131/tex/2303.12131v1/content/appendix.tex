\newpage
\section{Plasma and neutral atom transport equations}
\label{apx:transport}

The equations solved in 1D in section~\ref{sec:1d-transport} and in 2D
in section~\ref{sec:applications-2d} are detailed here for
completeness.  A total of seven spatially varying quantities are
evolved in time: The deuterium ion and atom densities ($n_{d+}$ and
$n_d$); the flow of ions and atoms parallel to the magnetic field
($v_{||,d+}$ and $v_{||,d}$); and the pressure of the ions, atoms, and
electrons ($p_{d+}$, $p_{d}$ and $p_e$). SI units are used except
temperatures, which are in eV. In a 1D domain
(section~\ref{sec:1d-transport}) the anomalous diffusion terms are
omitted and all derivatives perpendicular to the magnetic field are
assumed to be zero.

The equations for the deuterium ion species density $n_{d+}$,
parallel velocity $v_{||,d+} \equiv \mathbf{b}\cdot\mathbf{v}_{d+}$ and
pressure $p_{d+} = en_{d+}T_{d+}$ are:
\begin{subequations}
\begin{align}
  \frac{\partial}{\partial t}n_{d+} =& -\nabla\cdot\left[\left(\mathbf{b}v_{||,d+} + \mathbf{v}_{\perp,d+}\right)n_{d+}\right] + R_{iz} - R_{rc} \\
  \frac{\partial}{\partial t}\left(m_{d+}n_{d+}v_{||,d+}\right) =& -\nabla\cdot\left[\left(\mathbf{b}v_{||,d+} + \mathbf{v}_{\perp,d+}\right)m_{d+}n_{d+}v_{||,d+}\right] -\mathbf{b}\cdot\nabla p_{d+} \nonumber \\
  & + eE_{||} + R_{cx}m_d\left(v_{||,d} - v_{||,d+}\right) \nonumber \\
  & - R_{rc}m_dv_{||,d+} + R_{iz}m_dv_{||,d} \\
  \frac{\partial}{\partial t}\left(\frac{3}{2}p_{d+}\right) =& -\nabla\cdot\left[\left(\mathbf{b}v_{||,d+} + \mathbf{v}_{\perp,d+}\right)\frac{5}{2}p_{d+}\right] + v_{||,d+}\mathbf{b}\cdot\nabla p_{d+} \nonumber \\
  &+ \nabla\left(\mathbf{b}\kappa_{||,d+}\mathbf{b}\cdot\nabla T_{d+}\right) + \nabla\cdot\left(\chi_{d+}n_{d+}\nabla_\perp T_{d+}\right) \nonumber \\
  &+ \frac{1}{2}m_d \left(R_{cx} + R_{iz}\right) \left(v_{||,d} - v_{||,d+}\right)^2  \nonumber \\
  &+R_{iz}\frac{3}{2}eT_d - R_{rc}\frac{3}{2}eT_{d+} + W_{d+,e}
\end{align}
\end{subequations}
where $\mathbf{b} = \mathbf{B}/B$ is the unit vector in the direction
of the magnetic field, and the gradient in the plane perpendicular to
the magnetic field is $\nabla_\perp \equiv \nabla -
\mathbf{b}\mathbf{b}\cdot\nabla$. Particle diffusion across the
magnetic field is implemented as a cross-field ion drift velocity
$\mathbf{v}_{\perp,d+}$ with diffusion coefficient $D$:
\begin{equation}
  \mathbf{v}_{\perp,d+} = -D\frac{1}{n_{d+}}\nabla_\perp n_{d+}
\end{equation}
The charge exchange (CX), ionization (IZ) and recombination (RC)
reactions between species have rates (events per m$^3$ per second):
\begin{subequations}
\begin{align}
  R_{cx} =& n_{d+}n_d\left<\sigma v\right>_{cx} \\
  R_{iz} =& n_en_d\left<\sigma v\right>_{iz} \\
  R_{rc} =& n_en_{d+}\left<\sigma v\right>_{rc}
\end{align}
\end{subequations}
where the Maxwellian-averaged cross sections $\left<\sigma v\right>$
are taken from Amjuel~\cite{amjuel}, specifically Amjuel reaction H.4
2.1.5 (ionisation), H.4 2.1.8 (recombination) and H.3 3.1.8 (charge
exchange).  Hydrogenic charge-exchange reactions are adjusted for
isotope mass, ion and neutral temperatures by calculating an effective
temperature $T_{eff} = T_{atom} / A_{atom} + T_{ion} / A_{ion}$ as
described in the Amjuel manual.

There is a transfer of thermal energy to ions from electrons due to
collisions, $W_{d+,e}$:
\begin{equation}
W_{d+,e} = 3\nu_{d+,e}n_{d+}\frac{m_{d+}}{m_{d+} + m_e}e\left(T_e - T_{d+}\right)
\end{equation}
with ion-electron collision frequency $\nu_{d+,e} = \nu_{e,d+}m_e / m_{d+}$.

The electron density $n_e = n_{d+}$ is set by quasineutrality; the
electron parallel velocity $v_{||,e} = v_{||,d+}$ from assuming that
the parallel current is zero (Note that this is a choice in this
particular model, not a general feature of Hermes-3).  The electron
pressure equation is:
\begin{subequations}
  \begin{align}
    \frac{\partial}{\partial t}\left(\frac{3}{2}p_e\right) =& -\nabla\cdot\left[\left(\mathbf{b}v_{||,e} + \mathbf{v}_{\perp,d+}\right)\frac{5}{2}p_e\right] + v_{||,e}\mathbf{b}\cdot\nabla p_e \nonumber \\
    &+ \nabla\left(\mathbf{b}\kappa_{||,e}\mathbf{b}\cdot\nabla T_e\right) + \nabla\cdot\left(\chi_een_e\nabla_\perp T_e\right) \nonumber \\
    & - E_{iz} + E_{rc} - W_{d+,e}
  \end{align}
\end{subequations}
where $E_{iz}$ and $E_{rc}$ are the energy cost and gain due to
ionization and recombination atomic processes respectively. These are
calculated using Amjuel~\cite{amjuel}, reactions 2.1.5 and 2.1.8.
Ionization always removes energy from the electrons; Recombination may
be either a source or sink of electron energy, depending on the
temperature and density.  Electron force balance is used to calculate
the parallel electric field $E_{||} \equiv \mathbf{b}\cdot\mathbf{E}$
and so transfer electron pressure $p_e$ forces to the ions:
\begin{equation}
  eE_{||} = -\mathbf{b}\cdot\nabla p_e -\nabla\cdot\left[\mathbf{v}_{\perp,d+} m_en_ev_{||,e}\right]
\end{equation}

The equations for the neutral deuterium atom density $n_d$, parallel
velocity $v_{||,d}$ and pressure $p_d = en_dT_d$ are:
\begin{subequations}
  \begin{align}
    \frac{\partial}{\partial t}n_d =& -\nabla\cdot\left[\left(\mathbf{b}v_{||,d} + \mathbf{v}_{\perp,d}\right)n_d\right] - R_{iz} + R_{rc}  \\
    \frac{\partial}{\partial t}\left(m_dn_dv_{||,d}\right) =& -\nabla\cdot\left[\left(\mathbf{b}v_{||,d} + \mathbf{v}_{\perp,d}\right)m_dn_dv_{||,d}\right] -\mathbf{b}\cdot\nabla p_d \nonumber \\
    &- R_{cx}m_d\left(v_{||,d} - v_{||,d+}\right) + R_{rc}m_dv_{||,d+}\nonumber \\
    & - R_{iz}m_dv_{||,d} \\
    \frac{\partial}{\partial t}\left(\frac{3}{2}p_d\right) =& -\nabla\cdot\left[\left(\mathbf{b}v_{||,d} + \mathbf{v}_{\perp,d}\right)\frac{5}{2}p_d\right] + v_{||,d}\mathbf{b}\cdot\nabla p_d \nonumber \\
    &+ \nabla\left(\kappa_d \nabla T_d\right) + \frac{1}{2}m_d\left(R_{cx} + R_{rc}\right)\left(v_{||,d} - v_{||,d+}\right)^2 \nonumber \\
    &-R_{iz}\frac{3}{2}eT_d + R_{rc}\frac{3}{2}eT_{d+}
  \end{align}
\end{subequations}
The flow of neutral atoms across the magnetic field,
$\mathbf{v}_{\perp,d}$, is derived by balancing friction forces
against pressure gradient~\cite{rognlien-2002}:
\begin{equation}
  \mathbf{v}_{\perp,d} = - \frac{T_d}{m_d \nu_d p_d}\nabla_\perp p_d
\end{equation}

The thermal conduction coefficients for each species are:
\begin{equation}
  \kappa_{||,d+} = 3.9 \frac{p_{d+}}{m_{d+}\nu_{d+}} \qquad \kappa_{||,e} = 3.16 \frac{p_e}{m_e\nu_e} \qquad  \kappa_d = \frac{p_d}{m_d \nu_d}
\end{equation}
The collision frequencies for each species, $\nu_{d+}$, $\nu_e$ and $\nu_d$, are:
\begin{subequations}
  \begin{align}
    \nu_{d+} =& \nu_{d+,d+} + \frac{m_e}{m_{d+}}\nu_{e,d+} + n_d\left<\sigma v\right>_{cx} \\
    \nu_e =& \nu_{e,d+} + \nu_{e,e} \\
    \nu_d =& n_{d+}\left<\sigma v\right>_{cx} + n_da_0\sqrt{2eT_d/m_d}
  \end{align}
\end{subequations}
The collision frequency of charged species $a$ on charged species $b$
is given by~\cite{hinton1984}:
\begin{equation}
\nu_{a,b} = \frac{q_aq_b n_b \log\Lambda\left(1 + m_a/m_b\right)}{3\pi^{3/2}\epsilon_0^2m_a^2\left(v_a^2 + v_b^2\right)^{3/2}}
\end{equation}
with $v_a^2 = 2T_a/m_a$.
Neutral-neutral collisions assume a kinetic diameter of $2.8\times
10^{-10}$m (cross-section $a_0 = 2.5\times 10^{-19}$m$^2$), chosen
based on typical values for species considered ($2.89\times 10^{-10}$m
for H$_2$, $2.60\times 10^{-10}$m for He, $2.75\times 10^{-10}$m for
Ne~\cite{Wikipedia-kinetic-diameter}).

The Coulomb logarithm is different for electron-electron, ion-ion and
electron-ion species interactions, and is calculated using the NRL
formulary~\cite{Huba2013} (page 34). Converted to SI units with $T$ in
eV the Coulomb logarithms are:
\begin{subequations}
  \begin{align}
    \log \Lambda_{e,e} =& 30.4 - 0.5\log n_e + \frac{5}{4}\log T_e - \sqrt{\epsilon + \left(\log T_e - 2\right)^2/16} \\
    \log \Lambda_{e,i} =& \left\{\begin{array}{c c}
    31 - 0.5\log n_e + \log T_e & \textrm{if } T_i\frac{m_e}{m_i} < 10Z_i^2 < T_e \\
    30 - 0.5\log n_e - \log Z_i + 1.5\log T_e & \textrm{if } T_i \frac{m_e}{m_i} < T_e < 10Z_i^2 \\
    23 - 0.5\log n_i + 1.5 \log T_i - \log\left(Z_i^2 A_i\right) & \textrm{if } T_e < T_i m_e / m_i \\
    10 & \textrm{otherwise}
    \end{array}\right. \\
    \log \Lambda_{i,i} =& 29.91 - \log\left[\frac{Z_1 Z_2\left(A_1 + A_2\right)}{A_1 T_2 + A_2 T_1}\sqrt{n_1Z_1^2/T_1 + n_2 Z_2^2 / T_2}\right]
  \end{align}
\end{subequations}

