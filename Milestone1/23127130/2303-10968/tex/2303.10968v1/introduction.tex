\section{Introduction}

Cancer is among the main global causes of death.
According to \cite{sung2021global}, there were 19.3 million new cancer diagnoses and 9.96 million cancer-related deaths worldwide.
By 2040, the yearly number of new cancer cases is projected to reach 30.2 million, with 16.3 million fatalities attributable to cancer.
Each tumor is distinct and dependent on a variety of characteristics.
There is no guaranteed procedure for curing cancer, nor is its cause entirely known.
Utilizing mathematical models to precisely depict tumor progression is the primary objective of mathematical oncology. 

The key hallmarks of cancer evolution are described by \cite{hanahan2000hallmarks,hanahan2011hallmarks}
and for mathematical oncology to be successful, these characteristics should be met. As a primary advantage of a realistic mathematical model, cancer progression can be forecasted and physicians will be able to simply press a button on their computers to initiate a simulation portraying the patient's tumor and its development.
Ideally, this process is combined with a focused therapy that improves the cancer's prognosis. However, one must first guarantee that the model is well-posed, both mathematically and in terms of accurately representing the movement of actual cancer.
The second point can only be investigated using data and model verification through prediction; see the survey article by \cite{oden_2018} for more information on this topic.
The direction of this survey paper is toward the first point.
We must ensure that these models are mathematically valid, have a solution, and that nothing nonsensical occurs.
Then, one can consider a numerical strategy for the model that will provide a rapid, accurate, and stable representation of the tumor's evolution on the physician's monitor. 

There is an abundance of literature on the mathematical modeling of tumor evolution, which is a positive development.
Different groups develop distinct models and procedures and with this diversification, it is hoped that researchers will be able to accurately forecast the progression of malignancies.
In describing the phenomena of the world, partial differential equations (PDEs) are ubiquitous; they model the flow of liquids and gases (Navier--Stokes equations), the evolution of a quantum state (Schr\"odinger equation), thermal conduction (heat equation), spinodal decomposition (Cahn--Hilliard equation), and many others.
Complicated models may include nonlinearities, temporal and spatial nonlocalities, and mixed-dimensional couplings in response to complex processes.


Initially, tumor models were expressed as a free boundary problem.
We refer to \cite{greenspan1976growth}, which treated the tissue as a porous media and calculated the convective velocity field using Darcy's law.
Such models have been expanded upon in various works, and we direct you to the previous reviews by \cite{bellomo2000modeling} and \cite{roose2007mathematical}.
Since then, numerous distinct models have been formulated and in particular, we follow the path of diffusive interface models in which the tumor is characterized as a collection of cells using a fourth-order PDE.
These models are based on a multiphase method employing constitutive laws, thermodynamic principles, and balance rules for single constituents, which dates back to the works of \cite{cristini2003nonlinear}, \cite{cristini2010multiscale}, \cite{frieboes2010three}, and \cite{wise2008three} starting in 2003.


This work is organized as follows:  In \cref{Sec:Modeling}, we examine tumor evolution models and follow a technique based on continuum mixture theory. In this regard, we present the Cahn--Hilliard equation, the fundamental model of our tumor growth systems.
We provide a multiphase tumor growth model consisting of numerous components and biological processes.
In particular, we investigate the effects of the extracellular matrix, tumor cell stratification, the release of matrix degenerating enzymes and tumor angiogenesis factors, stochasticity, mechanical deformation, chemotherapeutic influence, memory effects, subdiffusion, and nonlocal phenomena including cell-to-cell adhesion and cell-to-matrix adhesion.  Further, we highlight each phenomena by numerical simulations and illustrations. We state the ideas for the numerical approximations of the introduced models in \cref{Sec:Numerics}.


