
\section{Mathematical Background} \label{Sec:Background}


\noindent In this appendix, we introduce the notation and principles of the mathematical tools that form the foundation of our approaches.
First, we introduce the function spaces and their respective norms and scalar products that will be considered throughout this paper.
Afterward, we present significant inequality for the estimations of the energy bounds' essential parameters.
In addition, embedding theorems are introduced to obtain strong convergence, which is required for nonlinear system components during the limit phase of the Galerkin approach.
The notion of memory effects and fractional derivatives of Caputo type are described next. 

\subsection{Function spaces, inequalities, and embedding results} \label{Sec:Math:Func}
This part introduces the function spaces that will be utilized extensively in subsequent sections.
The textbooks by \cite{boyer2012mathematical}, \cite{evans2010partial} and \cite{roubicek}  provide some excellent introductions to the mathematical analysis of PDEs. We assume that $\Omega \subset \R^d$, $d \in \N$, is a bounded domain with a sufficiently smooth boundary $\p\Omega$ and $T>0$ is a given fixed time horizon. Further, let $X$ be a Banach space with norm $\Vert \cdot\Vert _X$ and the dual pairing with its dual space $X'$  is denoted by $\langle \cdot,\cdot\rangle_{X}$. 

Let $\beta=(\beta_1,\dots,\beta_d) \in \bbN_0^d$ denote a multi-index.  We define the Sobolev space $W^{k,p}(\Omega)$ of order $k \in \bbN_0$ with  $p \in [1,\infty]$ by $$W^{k,p}(\Omega)=\{u \in L^p(\Omega): \p^\beta u \in L^p(\Omega) \text{ for }  \Vert \beta\Vert _{\ell^1} \leq k\},$$ which is a Banach space with the norm $\Vert u\Vert _{W^{k,p}(\Omega)}^p = \sum_{\vert \beta\vert \leq k} \Vert \p^\beta u \Vert _{L^p(\Omega)}^p.$ %
Here, $\p^\beta u$ denotes the weak derivative of $u$ in the sense of
$$\int_\Omega \p^\beta u(x) \varphi (x) \dx = (-1)^{\vert \beta\vert } \int_\Omega u(x) \p^\beta \varphi (x) \dx \qquad \forall \varphi \in C_c^\infty(\Omega).$$
In the particular case of $p=2$, the Hilbert space structure from $L^2(\Omega)$ is inherited, and we denote this space by $H^k(\Omega)$.
We work with the Bochner spaces $L^p(0,T;\!\!X)$ and we define the Sobolev--Bochner space $W^{1,p}(0,T;\!\!X)$ by
$$W^{1,p}(0,T;\!\!X)=\{ u \in L^p(0,T;\!\!X) : \pt u \in L^p(0,T;\!\!X) \},$$
where $\pt u$ denotes the weak derivative of $u$ in the sense of
$$\int_0^T \pt u(t) \varphi (t) \dt = - \int_0^T u(t) \varphi '(t) \dt \qquad \forall \varphi  \in C_c^\infty(0,T).$$

We denote a generic constant simply by $C>0$ and for the sake of brevity we may write $x \lesssim y$ instead of $x \leq Cy$.
We recall the H\"older, Young convolution, Poincar\'e--Wirtinger, Korn and Sobolev inequalities \citep{brezis2010functional,evans2010partial,roubicek,demengel} 
\begin{equation} \begin{aligned}
        \Vert  uv\Vert _{L^{\bar r}(\Omega)} & \leq \Vert  u\Vert _{L^{\bar p}(\Omega)} \Vert  v\Vert _{L^{\bar q}(\Omega)} && \forall u \in L^{\bar p}(\Omega), ~v \in L^{\bar q}(\Omega), \\ 
		\Vert  u*v\Vert _{L^{\hat r}(\Omega)} & \leq \Vert  u\Vert _{L^{\hat p}(\Omega)} \Vert  v\Vert _{L^{\hat q}(\Omega)} && \forall u \in L^{\hat p}(\Omega), ~v \in L^{\hat q}(\Omega), \\
		\Vert  u-\langle u\rangle_\Omega\Vert _{L^p(\Omega)} & \lesssim \Vert \nabla u\Vert _{L^p(\Omega)} && \forall u \in W^{1,p}(\Omega),   \\
		\Vert \nabla u\Vert _{L^p(\Omega)}^p              & \lesssim \Vert u\Vert _{L^p(\Omega)}^p\! + \! \Vert \eps(u)\Vert _{L^p(\Omega)}^p   && \forall u \in W^{1,p}(\Omega), \\%[-.18cm]
		\Vert u\Vert _{W^{m,\tilde q}(\Omega)} & \lesssim \Vert u\Vert _{W^{k,\tilde p}(\Omega)} && \forall u\in W^{k,\tilde p}(\Omega),
	\end{aligned} \label{Eq:SobolevInequality} \end{equation}
 where the exponents satisfy the relationship $\frac{1}{\bar p}+\frac{1}{\bar q} = \frac{1}{\bar r}$, $\frac{1}{\hat p}+\frac{1}{\hat q} = 1 + \frac{1}{\hat r}$ and $k-\frac{d}{\tilde p}  \geq m-\frac{d}{\tilde q}$ for $k\geq m$, respectively.
Here, $\langle u\rangle_\Omega =\frac{1}{\vert \Omega\vert } (u,1)_{L^2(\Omega)}$ denotes the mean of $u$ with respect to $\Omega$ and $\eps(u) =\frac12 (\nabla u + \nabla u^\top)$ represents the strain measure of $u$. The last inequality in \cref{Eq:SobolevInequality}  implies the continuous embedding $W^{k,p}(\Omega) \hookrightarrow W^{m,q}(\Omega)$ that is additionally compact in the case $k>m$ by the Rellich--Kondrachov embedding theorem  \citep[Section 10.9]{alt2016linear}.

To achieve strong convergence and pass the limit in the nonlinear parts of evolutionary PDEs, we require compact embeddings of Bochner spaces. We consider the Banach spaces $X$, $Y$, $Z$ such that $X$ is compactly embedded in $Y$ and $Y$ is continuously embedded in $Z$, i.e., the triple  $X\com Y \con Z$ is considered. We note that the Bochner space $L^2(0,T;\!\!X)$ is generally not compactly embedded in the space $L^2(0,T;\!\!Y)$ for $X \com Y$. E.g., consider the function sequence $f_n(t,x)=x\sin(nt)$. Indeed, one requires additional information on the time derivative as the Aubin--Lions compactness lemma \citep[Corollary 4]{simon1986compact} states
\begin{equation}\begin{alignedat}{2} L^p(0,T;\!\!X) \cap W^{1,1}(0,T;\!\!Z) &\com L^p(0,T;\!\!Y), &\quad 1\leq p<\infty, \\
		L^\infty(0,T;\!\!X) \cap W^{1,r}(0,T;\!\!Z) &\com C^0([0,T];\!\!Y), &\quad  r >1.
	\end{alignedat}
	\label{Eq:Aubin}
\end{equation}
Moreover, we state another embedding result to the continuous functions \citep[Theorem 3.1, Chapter 1]{lions2012non} 
$$\begin{alignedat}{2} L^2(0,T;\!\!Y)  & \cap H^1(0,T;\!\!Z)   &  & \con C^0([0,T];\!\![Y,Z]_{1/2}), 
\end{alignedat}
\label{Eq:ContEmb}
$$
where $[Y,Z]_{1/2}$ represents the interpolation space \citep[Definition 2.1, Chapter 1]{lions2012non} between $Y$ and $Z$.


\subsection{Fractional derivative}
In this section, fractional derivatives are examined.
We are primarily concerned with the well-known Caputo derivative.
There are many other approaches to fractional derivatives besides these two, but many of the newer methods with non-singular kernels have significant flaws and should not be used; see  \cite{diethelm2020good} for details. 

The fractional derivative in the sense of Caputo of a Banach-valued function $u:(0,T) \to X$ of order $\alpha \in (0,1)$  
is defined by
$$
	\pta u (t) 
 =  \frac{1}{\Gamma(1-\alpha)} \int_0^t \frac{u'(s)}{(t-s)^\alpha} \ds,
$$
where $\Gamma(\alpha):=\int_0^\infty t^{\alpha-1}e^{-t}\dt$ is called Euler's Gamma function. 
In the limit cases $\alpha=0$ and $\alpha=1$, we set $\p_t^0 u = u$ and  $\p_t^1 u = \p_t u$, respectively. We introduce the singular kernel function $g_{\alpha}\in L^1(0,T)$ by $g_{\alpha}(t) = t^{\alpha-1}/\Gamma(\alpha)$  and therefore, we can shortly write the Caputo derivative as follows
$$\pta u (t) 
 =  (g_{1-\alpha} \circledast \pt u)(t),$$
 where $\circledast$ denotes the convolution on the positive half-line.
 
The Caputo derivative, as shown in Theorem 2.1 in \cite{kilbas2006theory}, requires a function that is absolutely continuous on $[0,T)$. However, this definition can be relaxed to a broader class of functions that are equivalent to the classical definition for absolutely continuous functions, see \cite{fritz2021timefractional} and \cite{li2018some}. Indeed, one defines $$\pta u(t)=\pt(g_{1-\alpha} \circledast (u-u_0))$$ for some given initial value $u_0$ satisfying $(g_{1-\alpha}*(u-u_0)(0)=0$. In this spirit, we can define the Caputo derivative for functions in the fractional Sobolev-Bochner space
$$W^{\alpha,p}(0,T;\!\!X)=\big\{u \in L^p(0,T;\!\!X): \pta u \in L^p(0,T;\!\!X) \big\},$$
which contains non-continuous functions. In fact, it only holds $W^{\alpha,p}(0,T;\!\!X) \subset C([0,T];\!\!X)$ if the fractional order is large enough in the sense $\alpha>1/p$. Similar to the Aubin--Lions lemma, see \cref{Eq:Aubin}, there is the following compact embedding result
\begin{equation}
    \label{Eq:FractionalAubin}
    \begin{aligned}
L^{p}(0,T;\!\!X) \cap W^{\alpha,p}(0,T;\!\!Z) &\com L^p(0,T;\!\!Y), \quad p \in [1,\infty),
\end{aligned}
\end{equation}
see Theorem 3.1 in \cite{wittbold2020bounded}.
We notice that that the lower order $\alpha < 1$ of the time-fractional derivative has to be compensated with a larger power $p$. We are not aware of a version with a compact embedding into $C([0,T];\!\!Y)$ using a $L^\infty$-bound in $X$.
Moreover, we generalize the Gronwall--Bellman inequality to a setting that allows a convolution on the right hand side.

\begin{lemma}[{cf. Corollary 1 in \cite{fritz2021timefractional}}]
	\label{Lem:FractionalGronwall}
	Let $v,w \in L^1(0,T;\!\!\R_{\geq 0})$, and $a,b\ge0$. If $v$ and $w$ satisfy the inequality
$$
		w(t)+(g_\alpha \circledast v)(t) \leq a + b\cdot (g_\alpha \circledast w)(t) \qquad \text{for a.a. }  t \in (0,T),
$$
	then it holds
	$w(t)+ v(t) \leq a \cdot C(\alpha,b,T)$ for almost every $t \in (0,T)$.
\end{lemma}
We note that convolutions and fractional differentiation work as inverse operators as we see from the following computation
\begin{equation} 
(g_\alpha \circledast \p_t^\alpha u)(t) =(g_\alpha \circledast g_{1-\alpha} \circledast  \pt u)(t) = (g_1 \circledast \p_t u)(t) = u(t)-u(0).
\label{Eq:InverseConvolution}
\end{equation}
Fractional differentiation harbors some analytical difficulties such as the missing product and chain rules. Specifically, we note that the total derivative formula  $\frac12 \ddt f(u) = f'(u) \ddt u$ does not hold for general functions $f$ if we replace the derivative by its fractional version. 
However, there is at least a partial result for convex functionals $f:X \to \R$, see  Proposition 2.18 in \cite{li2018some}. In fact, we call this result "fractional chain inequality" and it reads
\begin{equation} \pta f(u) \leq \langle f'(u), \pta u\rangle_{X' \times X} \quad \forall u \in C^1([0,T);\!\!X). 
\label{Eq:ChainInequality}
\end{equation}
This inequality yields the correct direction to apply it in typical energy estimates such as testing the time-fractional heat equation $\pta u = \Delta u$ with the solution, which provides us with the estimate
$$\pta \Vert u\Vert _{L^2(\Omega)}^2 + \Vert \nabla u\Vert _{L^2(\Omega)}^2 \leq \langle\pta u,u\rangle_{H^1(\Omega)} - \langle \Delta u,u\rangle_{H^1(\Omega)} = 0.$$
However, gradient flows such as the Cahn--Hilliard equation are governed by a $\lambda$-convex (or: semiconvex) energy rather than a convex one, see \cite{fritz2021timefractional}.
Therefore, the fractional chain inequality \cref{Eq:ChainInequality} can be applied to the convex functional $x \mapsto f(x)-\frac{\lambda}{2}\Vert x\Vert _X^2$ and provides us with the following estimate for semiconvex functionals  
	$$\pta f(u) \leq \langle  f'(u),\pta u \rangle_{X} +  \frac{\lambda}{2} \pta \Vert u\Vert _X^2 -\lambda \langle \pta u,u \rangle_{X} \qquad \forall u \in C^1([0,T);\!\!X).$$




