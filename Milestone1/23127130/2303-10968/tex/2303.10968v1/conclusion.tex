\section{Concluding remarks} 
\label{Sec:Conclusion}
We have derived a multiple constituent model from the mass balance law and a Ginzburg--Landau type energy. Like this, we can describe the evolution of tumor cells with various biological phenomena such as angiogenesis. We incorporated stratification and invasion due to ECM deterioration into the model. Moreover, we investigated spatial and temporal nonlocalities, stochasticity resulting from a cylindrical Wiener process, mechanical deformation and elasticity, chemotherapeutic influence, and angiogenesis through mixed-dimensional couplings. 

Like this, we hope that tumor evolution can be studied with all various effects that happen in specific organs. Each tumor is unique, and the parameters have to be tuned for each scenario. One requires a sensitivity analysis with real data and a calibration of the parameters. We regard this as future research after collaborating with doctors and obtaining data.

Mathematically, it cannot be followed immediately whether the nonlinear models are well-posed and admit a solution. There is no unifying theory for the analysis of any nonlinear PDE, and each novel nonlinear system has its own unique challenges that must be examined in depth to confirm or deny the system's well-posedness. We want to emphasize that it is significant to the study the existence of solutions of various models. Otherwise, numerical methods might show solutions, but the model could be ill-posed and not suitable for describing real-world phenomena. 