\section{Numerical Implementation} \label{Sec:Numerics}


\noindent Besides the analytical methods in the next section, we are interested in showing numerical simulations and studying the influence of the new features and effects of the models as we have seen in the previous section. How useful is a well-posed model that does not reflect real biological processes? In this section, we briefly describe the techniques that we previously used for the implementation of the PDEs in the last sections. Our code is based on the finite element libraries \texttt{libMesh} \citep{libMeshPaper} and \texttt{FEniCS} \citep{fenics}.  \texttt{FEniCS}  is written in the accessible \texttt{Python} language and variational forms are straightforward to implement. However, \texttt{libMesh} is a high
performance computing (HPC) library written in \texttt{C++} and therefore, yields higher potential for code optimization and saving run times than in \texttt{FEniCS}. We refer to our \texttt{GitHub} \medskip

\begin{center}  \url{https://github.com/CancerModeling/Angiogenesis3D1D} \medskip \end{center} 

\noindent where the code is freely accessible. In particular, the settings for the simulations in \cite{fritz2021analysis,fritz2021modeling} on multispecies tumor growth are given.

Different groups prefer to use various finite element method (FEM) libraries, e.g., \texttt{ALBERTA} in \cite{garcke2018multiphase}. \cite{mohammadi2019simulation} utilized element-free Galerkin methods, \cite{wise2008three} a multigrid/finite difference method, and \cite{xu2016mathematical} isogeometric analysis. 
Moreover, the convergence of the FEM in tumor growth has been the subject of theoretical research; see \cite{garcke2022numerical}.

\subsection{Three-dimensional model}
Using the FEM, the 3D models were implemented. The code sequentially solves the system; see Algorithm 2.1 in \cite{fritz2021modeling} for the full model's algorithm. For the potential $\Psi=\Psi_e+\Psi_c$ in the Cahn--Hilliard equation, we employ the classical energy splitting approach, which gives unconditional energy stability; see \cite{elliott1993global}. Thus, the expansive portion $\Psi_e$ is treated explicitly while the contractive portion $\Psi_c$ is treated implicitly. We present the results of numerical experiments in \cite{fritz2021modeling,fritz2021analysis} and demonstrate the relative importance and roles of various biological effects, including cell mobility, proliferation, necrosis, hypoxia, and nutrient concentration, on the generation of MDEs and the degradation of the ECM.

\subsection{Nonlocal phenomena}
Nonlocal effects are not only challenging from an analytical standpoint, but they also pose difficulties for numerical approaches and increase the computational load. The FEM is founded on the notion of local elements, in opposition to the nature of spatial nonlocality. Not only should cells share information within their own element, but also with neighboring elements. In the case of time-fractional PDEs, not only the solution from the previous time step is relevant, but all solutions beginning with the initial condition must also be saved.

\subsubsection{Nonlocal-in-space effects}

In \cite{fritz2019local}, the evolution of the tumor volume fraction was analyzed in both local and nonlocal four-species models.
Thus, we select the gradient-based haptotaxis flux $J_\text{loc}(\phi_V,\ecm)=\chi_h \phi_V \nabla \ecm$ for the local model and $J_\text{nonloc}(\phi_V,\ecm)=\chi_h \phi_V k*\ecm$ for the nonlocal model.
As done in \cite{chaplain2011mathematical}, \cite{gerisch2008mathematical} and \cite{gerisch2010approximation}, we choose a kernel function $k_\eps$, $\eps>0$, in the place of $k$ that approximates the gradient-based haptotaxis effect as $\eps \to 0$.
This also means that a larger nonlocal influence correlates to a greater $\eps$-value.
Specifically, we employ the approximation 
$$\begin{aligned} &(k_\eps*\ecm)(x) - \ecm(x) \cdot (k_\eps*1)(x) \\  &= \int_{\mathbb{R}^d} k_\eps(x-y) (\ecm(y)-\ecm(x)) \dd y
\\&\approx \int_{\mathbb{R}^d} k_\eps(x-y) (\nabla \ecm(x) \cdot (y-x)  ) \dd y
\\&=\nabla \ecm(x) \int_{\mathbb{R}^d} (y-x) \cdot k_\eps(x-y) \dd y  
\\&= \nabla \ecm(x),
\end{aligned}$$
where we selected $k_\eps$ such that $x k_\eps(-x)$ is a Dirac sequence, i.e., it satisfies $\int_{\mathbb{R}^d} x k_\eps(-x) \, \dd x=1$. Specifically, we choose the kernel sequence
$
k_\eps(x)=- \omega(\eps) x \chi_{[0,\eps]} (\vert x \vert_\infty),
$. In the two-dimensional setting, we set the weight $\omega$ depending on $\eps$ to $\omega(\eps)=\frac{3}{8} \eps^{-4}$ in order to fulfill the normalizing Dirac property. %
 .%
 


\subsubsection{Nonlocal-in-time effects}

We mention the review work by \cite{diethelm2020good} that discusses the pertinent numerical approaches for time-fractional PDEs.
The kernel compressing schemes in \cite{fritz2021equivalence} and \cite{khristenko2021solving}, which reduce the time-fractional PDE to a system of ODEs, are among the numerous efficient methods accessible.
However, the traditional L1 scheme in \cite{oldham1974fractional} is still frequently used due to its simplicity, widespread acceptance, and straightforward implementation, see the survey article by \cite{stynes2021survey}.

Consider the mesh $0=t_0 < t_1 < \dots < t_{N-1}=t_N=T$ of the interval $[0,T]$. The $\alpha$-th Caputo derivative of a given function $\phi$ at $t_n$, $n \in \{1,\dots,N\}$, reads
$$\p_t^{\alpha} \phi(t_{n}) = \frac{1}{\Gamma(1-\alpha)} \int_0^{t_{n}} \frac{\phi'(s)}{(t-s)^\alpha} \ds.$$
We apply the approximation $f'(s)\approx \frac{f(t_{j+1}) - f(t_j)}{t_{j+1}-t_j}$ for $s \in (t_j,t_{j+1})$, which yields
\begin{equation*} \begin{aligned}
    \p_t^{\alpha} \phi(t_{n}) 
    &\approx  \frac{1}{\Gamma(2-\alpha)} \sum_{j=0}^{n-1} w_{n-j-1,n} (\phi(t_{n-j})-\phi(t_{n-j-1})), 
\end{aligned} \end{equation*}
where the weights $w_{m,n}$ for $n,m \in [0,N]$ are given by
$$w_{m,n} = \frac{(t_n-t_m)^{1-\alpha}-(t_n-t_{m+1})^{1-\alpha}}{t_{n-m}-t_{n-m-1}}.$$
The L1 scheme's convergence is in the range of  $\mathcal{O}((\Delta t)^{2-\alpha})$, see \cite{diethelm2020fractional}, and the memory effect is depicted on the right as the history of the preceding time steps $\phi(t_{n-j})$. Exactly this step is computationally intensive due to the need to save the entire history in the computer's memory storage.
Reduce the computational complexity by, for instance, storing only the previous 20 solutions.
Given that the weights on prior solutions drop the further back the previous solution is, this seems reasonable.
But then nothing more can be said about convergence. 

In the works by \citep{fritz2021subdiffusive,fritz2021timefractional} on time-fractional tumor growth models, a fractional linear multistep method is used as in \cite{Lubich86}. Such a method is based on a convolution quadrature scheme, and it generalizes the standard linear multistep method for ODEs. A subclass of these methods generalizes the backward Euler method to fractional settings and approximates the Caputo derivative by
\begin{equation*} \begin{aligned}
\p_t^{\alpha} \phi(t_{n}) 
&\approx  \frac{1}{(\Delta t)^\alpha} \sum_{j=0}^{n-1} (-1)^j \binom{-\alpha}{j} (\phi(t_{n-j})-\phi(0)).
\end{aligned} \end{equation*}
Indeed, setting $\alpha=1$ gives the backward Euler scheme. Similar to the traditional L1 method, it is necessary to store all previous solutions.
The quadrature weights can also be calculated recursively, and such methods are known as Gr\"unwald--Letnikov approximations.  Similar to the classical L1 method, one has to store all the previous solutions, see \cite{diethelm2010analysis} and \cite{Dumitru12} for more details.


\subsection{Uncertainty in tumor modeling}
First off, we note that an orthonormal basis of the Hilbert space $L^2(\Omega)$ on the three-dimensional domain $\Omega=(0,2)^3$ is given by
$$e_{ijk}(x_1,x_2,x_3)=\cos(i\pi x_1/L) \cos(j\pi x_2/L) \cos(k\pi x_3/L),$$
where $L$ is the edge length of the cubic domain $\Omega$.
Then the cylindrical Wiener processes $W_\alpha$ on $L^2(\Omega)$ can be written as
$$W_\alpha(t)(x_1,x_2,x_3) = \sum_{i,j,k=1}^\infty \eta^\alpha_{ijk}(t) e_{ijk}(x_1,x_2,x_3),$$
where $\{\eta^\alpha_{ijk}\}_{i,j,k \in \mathbb{N}}$ is a family of real-valued, independent, and identically distributed (i.i.d.) Brownian motions.
Following the works by \cite{chai2018conforming} and \cite{antonopoulou2021numerical}, we approximate the term involving the Wiener process in the fully discretized system as follows
$$
\frac{1}{\Delta t} \left( \int_{t_n}^{t_{n+1}} \dd W_\alpha(t) , \xi \right)_{L^2(\Omega)} \approx \frac{1}{\Delta t} \sum_{\substack{i,j,k, \\ i+j+k < I_\alpha}} \eta^\alpha_{ijk} (e_{ijk},\xi)_{L^2(\Omega)},
$$
where $\xi \in V_h$ is a test function, $\eta^\alpha_{ijk} \sim \mathcal{N}(0,\Delta t)$ are independent Gaussians, and $I_\alpha$ controls the number of basis functions.

\subsection{Mixed-dimensional coupling}
In the case of 3D-1D tumor growth models, one must implement the new 1D components into the code and establish the link between the 1D and 3D variables.
For time integration of the 1D equations, we employ the implicit Euler method.
For the spatial discretization of the 1D equations, the vascular graph method is used, which corresponds to a node-centered finite volume method, see \cite{reichold2009vascular} and \cite{vidotto2019hybrid} for further details. 

We decouple the 1D and 3D pressure equations at each time step and use block Gau\ss--Seidel iterations to solve the two systems until the 3D pressure converges.
Similarly, the nutrient equation is discretized, with the addition of an upwinding process for the convective term.
The nutrition equations are solved with block Gau\ss--Seidel iterations at each time step.
In \cite{fritz2021modeling}, the numerical approach and discretization of terms that arise in the setting of the 3D-1D coupling are presented in depth. 

