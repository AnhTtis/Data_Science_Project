\centering
\includegraphics[width=.33\columnwidth]{imgs/cold-gas-obstacles/gt-000.png}%
\includegraphics[width=.33\columnwidth]{imgs/cold-gas-obstacles/gt-040.png}%
\includegraphics[width=.33\columnwidth]{imgs/cold-gas-obstacles/gt-080.png}
\includegraphics[width=.33\columnwidth]{imgs/cold-gas-obstacles/gt-mesh-000.png}%
\includegraphics[width=.33\columnwidth]{imgs/cold-gas-obstacles/gt-mesh-040.png}%
\includegraphics[width=.33\columnwidth]{imgs/cold-gas-obstacles/gt-mesh-080.png}
\includegraphics[width=.33\columnwidth]{imgs/cold-gas-obstacles/sim-000.png}%
\includegraphics[width=.33\columnwidth]{imgs/cold-gas-obstacles/sim-040.png}%
\includegraphics[width=.33\columnwidth]{imgs/cold-gas-obstacles/sim-080.png}
\includegraphics[width=.33\columnwidth]{imgs/cold-gas-obstacles/sim-mesh-000.png}%
\includegraphics[width=.33\columnwidth]{imgs/cold-gas-obstacles/sim-mesh-040.png}%
\includegraphics[width=.33\columnwidth]{imgs/cold-gas-obstacles/sim-mesh-080.png}
\caption{The simulation of a cold gas in a scene containing obstacles (1st row) and the geometry surrounding the volume (2nd row), compared with the reconstruction obtained with our framework (3rd row) and the reconstructed mesh (4th row). Our method is able to capture the complex behaviour of the gas moving around the obstacles in the scene.}
\label{fig:obstacles-reconstruction}