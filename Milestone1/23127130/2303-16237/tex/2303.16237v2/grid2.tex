\documentclass[12pt]{article}
\usepackage{graphicx}
\usepackage{subfigure}
\usepackage{amsthm,amsfonts}
\usepackage{amssymb}
\usepackage{fullpage}
\usepackage{caption}
\usepackage{pifont}
\newtheorem{theorem}{Theorem}
\newtheorem{corollary}{Corollary}
\captionsetup[figure]{name={Figure},labelsep=period}

\begin{document}
\title{The non-repetitive colorings of grids}
\author{Tianyi Tao \\
School of mathematical sciencese \\
Fudan University \\
{\tt tytao20@fudan.edu.cn}}
\date{\today}
\maketitle

\begin{abstract}
In a graph $G$ with a vertex coloring $c$, a path of $2k$ vertices $\{v_1,\cdots,v_k,v_{k+1},\cdots,v_{2k}\}$ is called repetitive if $c(v_i)=c(v_{k+i})$ for all $1\le i\le k$. We call $c$ a non-repetitive coloring if for any $k\in\mathbb{N}$, all paths of $2k$ vertices are non-repetitive.

In this paper, our main focus is to study the non-repetitive coloring of the Cartesian product of graphs. In particular, we prove that the grid $P\square P$ can be 12-nonrepetitively colored, where the previous best result was 16.
\end{abstract}

\section{Introduction}

Let $P_{2k}=\{v_1,\cdots,v_k,v_{k+1},\cdots,v_{2k}\}$ be a path of even number of vertices and $c$ is a vertex coloring of $P_{2k}$. We call $(P_{2k},c)$ is repetitive if for all $1\le i\le k, c(v_i)=c(v_{k+i})$.

In 1906, Thue\cite{1} proved the following theorem. For a path of any length, we can color its vertices with three colors: $a, b$, and $c$, such that it does not contain any repetitive subpath.

Let $P$ be the path of countable infinite vertices. We correspond $P$ to the number axis, whose vertices correspond to the integer points. Then a $k$-coloring of $P$ corresponds to a sequence $\mathbb{Z}\to[k]$. We call a sequence $\mathbb{Z}\to[3]$ a Thue's sequence if it does not contain any repetitive path. By compactness (see \textbf{Theorem 10.3.16} in \cite{9}), we are abel to show that the Thue's sequence exists. Note that we cannot provide the construction of (infinite) Thue's sequence now, as Thue's original proof is completed iteratively.

We intercept a segment in a Thue's sequence in schematic diagram(I) as an example of common using in this paper.

\begin{figure}[h] 
\centering
\includegraphics[width=0.8\linewidth]{figure-1.png}
\end{figure}

As an extension, in 2002, Alon, Grytczuk, Hałuszczak, and Riordan\cite{4} introduced the concept of non-repetitive coloring to general graphs : A vertex coloring of a graph $G$, is called non-repetitive if there is no repetitive path in $G$. The minimum number of colors required for such a coloring is known as the non-repetitive chromatic number or Thue number of $G$, denoted as $\pi(G)$. With this definition, Thue's theorem can be written as $\pi(P_n)=3$ for all $n$, where $P_n$ is a path of $n$ vertices. Correspondingly, $\pi(P)=3$. There are studies on non-repetitive coloring in\cite{11,13,14,15}.

One of the most interesting conjectures on non-repetitive coloring was whether planar graphs have a bounded non-repetitive chromatic number (see \cite{4}). This conjecture was affirmed by Dujmović Vida, Louis Esperet, Gwenaël Joret, Bartosz Walczak, Piotr Micek, Pat Morin, Torsten Ueckerdt and David R. Wood\cite{8,10} in 2020: the non-repetitive chromatic number of planar graphs is at most 768. The general idea of the proof is to embed planar graphs into products of graphs each of which the non-repetitive chromatic number is bounded. This is also one of the purposes that we discuss grids, the Cartesian product of paths.

A sequence (of colors) is called a palindrome if it is equal to its reflection. For instance, $aa,aba,abcba$ are palindromes. It is not hard to check that any non-repetitive coloring of $P_6$ must contain a palindrome. As a result, any Thue's sequence contains palindromes. We now insert a new color $d$ between consecutive blocks of length 2 in a Thue's sequence, resulting in a new sequence $T^\ast:\mathbb{Z}\to\{a,b,c,d\}\cong[4]$ following,

\[T^\ast(n)=
\left\{
\begin{array}{lcl}
d,&n\equiv 0(\rm{mod 3}),\\
T(n-\lfloor\frac{n}{3}\rfloor),&n\not\equiv 0(\rm{mod 3}).\\
\end{array}
\right.
\]

We call a sequence $\mathbb{Z}\to[4]$ a palindrome-free Thue's sequence if it does not contain any repetitive path or palindrome. It is not hard to show that $T^\ast$ is a palindrome-free Thue's sequence.

The schematic diagram (II) below is an example obtained by performing this operation on example (I).

\begin{figure}[h] 
\centering
\includegraphics[width=0.8\linewidth]{figure-2.png}
\end{figure}

Let's review the definitions of Cartesian product, tensor product, and strong product. Given graphs $G$ and $H$, the Cartesian product of $G$ and $H$, denoted by $G\square H$, is the graph with vertex set $V(G)\times V(H)$, where $(g,h)$ and $(g',h')$ are adjacent if $g=g',hh'\in E(H)$ or $h=h',gg'\in E(G)$; the tensor product of $G$ and $H$, denoted  by $G\times H$, is the graph with vertex set $V(G)\times V(H)$, where $(g,h)$ and $(g',h')$ are adjacent if $gg'\in E(G)$ and $hh'\in E(H)$; the strong product of $G$ and $H$, denoted  by $G\boxtimes H$, is the union of $G\square H$ and $G\times H$.

We can define the concept of product coloring. Let $G$ be a graph with coloring $c$, and let $H$ be a graph with coloring $c'$. The coloring of $G\square H$, $G\times H$ or $G\boxtimes H$, which is defined by $(g,h)\mapsto(c(g),c'(h))$, is called a product coloring of $c$ and $c'$.

Now let's discuss the non-repetitive colorings of grid $P\square P$. The conclusion $\pi(P\square P)\le16$ is best upper bound so far (see \cite{2,7}). The proof of this proposition is straightforward. We consider a palindrome-free Thue's sequence $T^\ast$, which is a non-repetitive, palindrome-free 4-coloring of $P$, and its product coloring with itself is a 16-coloring of $P\boxtimes P$. Then we can verify that this product coloring is non-repetitive. As a result, $\pi(P\square P)\le\pi(P\boxtimes P)\le 16.$ Figure 1 is a schematic diagram of this product coloring.

\begin{figure}[h] 
\centering
\begin{minipage}[t]{0.48\textwidth}
\centering
\includegraphics[width=\textwidth]{figure1.png}
\caption{}
\end{minipage}\hfill
\begin{minipage}[t]{0.48\textwidth}
\centering
\includegraphics[width=\textwidth]{figure2.png}
\caption{}
\end{minipage}
\end{figure}

We know that the Cartesian product of graphs has fewer edges compared to the strong product. However, existing literature on the non-repetitive chromatic number of graph Cartesian products is based on strong products. Specifically, for grid $P\square P$, the best upper bound for its non-repetitive chromatic number is 16 so far. In addition, there are also some related works in \cite{6,12}. In these papers, they discussed the so-called facial non-repetitive chromatic number (of grids).

Note that in the proof of $\pi(P\square P)\le 16$, the palindrome-free property of the coloring of $P$ is essential. Can we replace the 4-coloring of $P$ in the proof with a non-repetitive 3-coloring? In fact, it is not possible. Even if we use the product coloring of a non-repetitive 3-coloring and a non-repetitive, palindrome-free 4-coloring, the resulting coloring is not non-repetitive. For example, the path in red in Figure 2 is repetitive. If you want to further understand the reasons behind this, you can learn about the concept of stroll-nonrepetitive coloring in \cite{3}. Proposition 3.3 (Pascal Ochem) in \cite{3} implies the impossibility of having a non-repetitive coloring of a grid with less than 16 colors by the product coloring.

In this paper, we will give a construction to show that $\pi(P\square P)\le12$, and we further extend this construction to high-dimensional cases.

In the final part of the paper, we investigate the non-repetitive chromatic number of the Cartesian product of complete (bipartite) graphs. There are two reasons for this research. Firstly, we give a partial result of a open problem mentioned in \cite{3}, which is related to the edge-non-repetitive chromatic number of complete graphs (see \cite{20,21,22,23}). Secondly, these two types of graphs could be examples for the lower bounds of the non-repetitive chromatic number for graphs with a given maximum degree. As discussed in \cite{4}, Noga Alon, Jarosław Grytczuk, Mariusz Hałuszczak, and Oliver Riordan proved the existence of a constant $c>0$ such that for any $\Delta$, there exists a graph $G$ with maximum degree $\Delta$, whose non-repetitive chromatic number $\pi(G)\ge\frac{\Delta^2}{\log\Delta}$. $K_n\square K_n$ and $K_{n,n}\square K_{n,n}$, both have maximum degrees that are linear in terms of $n$. Can their non-repetitive chromatic numbers be on the order of $n^2$? (See the open problem at the end of the paper.)

\section{Two-dimensional case}

Note that the Thue's sequence can be considered as a colored infinite path, as well as a color sequence. When we mention a walk in Thue's sequence, we refer to the walk in this colored infinite path. When we mention a vertex in the Thue's sequence, it refers to both a vertex in the infinite path and its corresponding integer in $\mathbb{Z}$.

Here is a key observation. We give the following coloring $\tilde{c}$ for $P\square P$ (treat the vertex set of $P\square P$ as $\mathbb{Z}^2$): take a palindrome-free Thue's sequence $T^\ast$, for every $(X,Y)\in\mathbb{Z}^2,$ let $\tilde{c}(X,Y)=T^\ast(X-Y)$. Figure 3 is a partial schematic diagram of this coloring. 

\begin{figure}[h] 
\centering
\includegraphics[width=0.47\linewidth]{figure3.png}
\caption{}
\end{figure}

In such a coloring, the color sequence of any path in the grid will correspond to the color sequence of a walk in $T^\ast$. 

A lazy walk in a graph $G$ is a walk in the pseudograph obtained from $G$ by adding a loop at each vertex, that is, it allows that two consecutive vertices are the same vertex.

For a walk or lazy walk, we can also define the concept of repetitive. Let\\$W=\{v_1,\cdots,v_k,v_{k+1},\cdots v_{2k}\}$ be a walk or lazy walk with coloring $c$. We call it repetitive if for all $1\le i\le k$, $c(v_i)=c(v_{k+i})$.\\

\begin{theorem}
Take a a palindrome-free Thue's sequence $T^\ast$. If $W=\{v_1,\cdots,v_k,v_{k+1},\cdots v_{2k}\}$ is a repetitive lazy walk in $T^\ast$, then $v_i=v_{k+i}$ forall $1\le i\le k$. In other words, any repetitive lazy walk in $T^\ast$ is composed of two identical lazy walks.
\end{theorem}

\begin{proof}
For all $1\le i<2k,$ we have $v_{i+1}-v_i\in\{1,0,-1\}$.

If $v_1=v_2$, then $T^\ast(v_1)=T^\ast(v_2)$. So $T^\ast(v_{k+1})=T^\ast(v_{k+2})$, which implies that $v_{k+1}=v_{k+2}$.

If $v_1\not=v_2$, without loss of generality, we assume that $v_2-v_1=1$, then $T^\ast(v_1)\not=T^\ast(v_2)$, $T^\ast(v_{k+1})\not=T^\ast(v_{k+2})$, which implies that $v_{k+2}-v_{k+1}=\pm1$. We will discuss these two cases below.

Case 1. $v_{k+2}-v_{k+1}=1$. We claim that for all $1\le i<k$, $v_{i+1}-v_i=v_{k+i+1}-v_{k+i}$. Prove by induction, we assume the claim holds for all $1\le j<i$. If $v_{i+1}=v_i$, then $v_{k+i+1}=v_{k+i}$ obviously. Without loss of generality, we assume that $v_{i+1}-v_i=1$. If $v_{i+1}=v_{i-1}$, then $T^\ast(v_{k+i+1})=T^\ast(v_{k+i-1})$. Because there is no palindrome in $T^\ast$, $v_{k+i+1}=v_{k+i-1}$, which implies that $v_{i+1}-v_i=v_{k+i+1}-v_{k+i}$ by induction. If $v_{i+1}\not=v_{i-1}$, then $T^\ast(v_{i+1})\not=T^\ast(v_{i-1})$ by palindrome-free. So $T^\ast(v_{k+i+1})\not=T^\ast(v_{k+i-1})$, $v_{k+i+1}\not=v_{k+i-1}$, which also implies that $v_{i+1}-v_i=v_{k+i+1}-v_{k+i}$.

Case 2. $v_{k+2}-v_{k+1}=-1$. Using the same method, we can prove that for all $1\le i<k$, $v_{i+1}-v_i=-(v_{k+i+1}-v_{k+i})$.

For Case 1, if $v_1\not=v_k$, without loss of generality we assume $v_1<v_k$, then $[v_1,v_{2k}]\cap\mathbb{Z}$ is a repetitive path on $\mathbb{Z}$.

For Case 2, if $v_1=v_k$ and $W$ is nontrivial, then $\{v_1-1,v_1,v_1+1\}$ is a palindrome on $\mathbb{Z}$. If $v_1\not=v_k$, without loss of generality we assume $v_1<v_k$, then $[v_1,v_{k}]\cap\mathbb{Z}$ is a palindrome on $\mathbb{Z}$.

To sum up, there is only one possibility for the composition of $W$, that is, $W$ is composed of two indetical walks.
\end{proof}

\begin{theorem}
$\pi(P\square P)\le12.$
\end{theorem}

\begin{proof}
Construct the following coloring of grid $P\square P$ as Figure 4 (partial schematic diagram).
	
\begin{figure}[h] 
\centering
\includegraphics[width=0.6\linewidth]{figure4.png}
\caption{}
\end{figure}
	
We establish an $XOY$ coordinate system on the plane and divide the vertices of the grid $P\square P$ into two types according to the parity of the sum of horizontal and vertical coordinates. Each type of vertex is divided into the disjoint union of vertices on parallel lines $L_s: X-Y=2s$ and $R_t: X+Y=2t+1$. We take two palindrome-free Thue's sequences $T^\ast_1: \mathbb{Z}\to\{a,b,c,d\}$ and $T^\ast_2: \mathbb{Z}\to\{x,y,z,w\}$ and construct a coloring $c_0$ of $P\square P$ as following: all vertices on line $L_s$ get the commom color $T^\ast_1(s)$ and all vertices on line $R_t$ get the commom color $T^\ast_2(t)$. The property we need is that in this construction, two lines of different types are not parallel.
	
In the coloring $c_0$, there are many repetitive paths, for example, the red box in Figure 4 contains a path with color sequence $a\to z\to a\to z$. Our goal is to prove that we can destroy all the repetitive paths by destroying the repetitive path with length of 4 like above. 
	
Let $P=\{v_1,\cdots,v_k,v_{k+1},\cdots v_{2k}\}$ be a repetitive path in the coloring above, then $k$ must be even. Without losing generality, we assume $c_0(v_1)=a$ and $c_0(v_2)=z$. Let $P_1=\{v_1,v_3,\cdots,v_{k-1},v_{k+1},\cdots v_{2k-1}\}$ and $P_2=\{v_2,v_4,\cdots,v_{k},v_{k+2},\cdots v_{2k}\}$, then the color sequence of $P_1$ corresponds to a repetitive lazy walk in $T^\ast_1$, and the color sequence of $P_2$ corresponds to a repetitive lazy walk in $T^\ast_2$. Using \textbf{Theorem 1}, $v_{k+1}$ must be on the same $L_s$ as $v_1$, and $v_{k+2}$ must be on the same $R_t$ as $v_2$ (Figure 5).
	
\begin{figure}[h] 
\centering
\begin{minipage}[t]{0.48\textwidth}
\centering
\includegraphics[width=\textwidth]{figure5.png}
\caption{}
\end{minipage}\hfill
\begin{minipage}[t]{0.48\textwidth}
\centering
\includegraphics[width=\textwidth]{figure6.png}
\caption{}
\end{minipage}
\end{figure}

The coordinates of $v_{k+1}$ and $v_{k+2}$ are uniquely determined, that is, $v_{k+1}$ is exactly the vertex on the left of $v_2$, and $v_{k+2}$ is exactly the vertex on the left of $v_1$ (in Figure 5).

We now divide $x, y, z, w$ into $x_1, x_2, y_1, y_2, z_1, z_2, w_1, w_2$ by parity of horizontal coordinates. Define a coloring $c$ of $P\square P$ as following: when $c_0((X,Y))\in\{a,b,c,d\}$, let $c((X,Y))=c_0((X,Y))$, when $c_0((X,Y))=x,y,z,$ or $w$, if $X$ is odd, let $c((X,Y))=x_1,y_1,z_1,$ or $w_1$, if $X$ is even, let $c((X,Y))=x_2,y_2,z_2,$ or $w_2$ (Figure 6).

In $P\square P$ with coloring $c$, all repetitive paths are destroyed. As a result, the grid is non-repetitively 12-colorable.

\end{proof}

\section{High-dimensional cases}

Let $P$ be the infinite path. Denote $n$ of $P$'s Cartesian product, strong product and tensor product by $(\square P)^n$, $(\boxtimes P)^n$ and $(\times P)^n$. 

We know that $\pi(\boxtimes P)^n\le 4^n$ in \cite{2}. In fact, the product coloring of $n$ palindrome-free Thue's sequence is a non-repetitive coloring of $\pi(\boxtimes P)^n$, Figure 1 is the two-dimensional case. Now we discuss $(\times P)^n$ first.

For $n=2$, $P\times P$ has 2 components, each of which is isomorphic to $P\square P$, so $\pi(P\times P)=\pi(P\square P)\le 12$. For $n\ge3$, $(\times P)^n$ has $2^{n-1}$ components, each of which is different from $(\square P)^n$. Notice that $(\times P)^n$ is $2^n$-regluar but $(\square P)^n$ is $2n$-regular.\\

\begin{theorem}
In $(\boxtimes P)^n$ with the non-repetitive $4^n$-coloring in \cite{2} (product coloring), any repetitive lazy walk is composed of two indentical lazy walks.
\end{theorem}

\begin{proof}
Let $W$ is a repetitive lazy walk in $(\boxtimes P)^n$. The coordinates of $W$ on each component is a repetitive lazy walk in some palindrome-free Thue's sequence. By \textbf{Theorem 1}, $W$ is composed of two indentical lazy walks.
\end{proof}	

\begin{theorem}
$\pi((\times P)^n)\le(4^{n-1}+4\cdot2^{n-1})$.
\end{theorem}

\begin{proof}
Regard the set of vertices of $(\times P)^n$ as $\mathbb{Z}^n$ in Euclidean space $\mathbb{R}^n$. Consider one of the component of $(\times P)^n$. On the geometric viewpoint, the vertex set of this graph is divided into a series of parallel $(n-1)$-dimensional hyperplanes (in $\mathbb{Z}^n$). The equations of these hyperplanes are $x_n=\lambda, \lambda\in\mathbb{Z}$.
	
Take a palindrome-free Thue's sequence $T^\ast$. For odd $\lambda$, let the vertices on the hyperplane $x_n=\lambda$ have the same color $T^\ast(\frac{\lambda-1}{2})$. For example, all vertices satisfies $x_n=1$ get the color $T^\ast(0)=a$, all vertices satisfies $x_n=3$ get the color $T^\ast(1)=b$, all vertices satisfies $x_n=5$ get the color $T^\ast(2)=d$, all vertices satisfies $x_n=7$ get the color $T^\ast(3)=c$ and all vertices satisfies $x_n=9$ get the color $T^\ast(4)=a,\cdots$
	
For even $\lambda$, the vertices on each hyperplane share the same set of $4^{n-1}$ colors by product coloring, that is, all vertices on the line $x_1=i_1,\cdots,x_{n-1}=i_{n-1}$ share a same color. Figure 7 is a schematic diagram when $n$=3, whose dashed lines are not edges in the graph.
	
\begin{figure}[h] 
\centering
\begin{minipage}[h]{0.48\textwidth}
\centering
\includegraphics[width=0.9\textwidth]{figure7.png}
\caption{}
\end{minipage}\hfill
\begin{minipage}[h]{0.48\textwidth}
\centering
\includegraphics[width=\textwidth]{figure8.png}
\caption{}
\end{minipage}
\end{figure}
	
Let $P=\{v_1,\cdots,v_k,v_{k+1},\cdots,v_{2k}\}$ be a repetitive path in $(\times P)^n$ with the coloring above, let $P_1$ and $P_2$ be paths formed by the odd number and even number of vertices of $P$. Without losing generality, we assume that $v_1$ is on a hyperplane with an odd coefficient $\lambda$, so the color sequence of $P_1$ corresponds to a repetitive lazy walk in $T^\ast$ and the color sequence of $P_2$ corresponds to a repetitive lazy walk in $(\boxtimes P)^{n-1}$ with the $4^{n-1}$-product coloring. By \textbf{Theorem 1} and \textbf{Theorem 3}, $v_1$ and $v_{k+1}$ are in a same hyperplane (they have the same $n$-th coordinate), $v_2$ and $v_{k+2}$ are in a same line (they have the same $1,\cdots,n-1$-th coordinates).
	
On the hyperplanes of odd coefficients $\lambda$, by divding each color into $2^{n-1}$ colors, all repetitive paths can be destroyed (Figure 8). This shows $\pi((\times P)^n)\le 4^{n-1}+4\cdot2^{n-1}$.
\end{proof}

Now let’s discuss the high-dimensional Cartesian product of paths. Unfortunately, the following theorem only gives a good bound when $n$ is small (for example, $n$=3). For large $n$, it is known that $\pi((\square P)^n)=O(n^2)$ (see \cite{5,16,17,18,19}) by weighted Lovász local lemma (Michael Molloy and Bruce Reed)\cite{24,25,26}. The bound obtained through a similar approach is not the optimal one. 

\begin{theorem}
$\pi((\square P)^3)\le28$.	
\end{theorem}

\begin{proof}
Similar to the proof of \textbf{Theorem 4}, the only difference is that we take the plane sequence of the equations $\sum\limits_{j=1}^3 x_j=\lambda,\lambda\in\mathbb{Z}$ at this time.

\begin{figure}[h] 
\centering
\begin{minipage}[h]{0.48\textwidth}
\centering
\includegraphics[width=\textwidth]{figure9.png}
\caption{}
\end{minipage}\hfill
\begin{minipage}[h]{0.48\textwidth}
\centering
\includegraphics[width=\textwidth]{figure10.png}
\caption{}
\end{minipage}
\end{figure}	

For odd $\lambda$, let the vertices on each plane be the same color, which is taken from some palindrome-free Thue's sequence $T^\ast$ consecutively like \textbf{Theorem 4}.
	
For even $\lambda$, the integral point set on the plane $\sum\limits_{j=1}^3 x_j=\lambda$ has a conformal linear isomorphism to $\mathbb{Z}^2$. We use a common 16-coloring for all parallel planes, which is the product coloring of two palindrome-free Thue's sequence.
	
Let $P=\{v_1,\cdots,v_k,v_{k+1},\cdots,v_{2k}\}$ be a repetitive path in $(\square P)^3$ which start at a plane with an odd coefficient $\lambda$. Similarly, let $P_1$ and $P_2$ be the paths formed by the odd number and even number of vertices of $P$. Then the color sequence of $P_1$ corresponds to a repetitive lazy walk in $T^\ast$ and the color sequence of $P_2$ corresponds to a repetitive lazy walk in a graph obtained by densely tiling equilateral triangles in a plane with the 16-product coloring. Note that it is a subgraph of $P\boxtimes P$.
	
Using \textbf{Theorem 1} and \textbf{Theorem 3}, we know that $v_1$ and $v_{k+1}$ are in the same plane $\sum\limits_{j=1}^3 x_j=\lambda$, $v_2$ and $v_{k+2}$ are in the same line with direction vector (1,1,1). Figure 9 is a schematic diagram.
	
Because each vertex has 3 neighbors to its (one side) adjacent plane, we can destroy all repetitive paths by dividing each color into 3 colors on the planes with odd coefficient $\lambda$ (Figure 10). This shows $\pi((\square P)^3)\le16+4\times3=28$.

\end{proof}

\section{The Cartesian product of complete (bipartite) graph}

The following content is not a generalization of the theorems mentioned above, but rather a discussion on the non-repetitive coloring of the Cartesian product of a graphs. We will provide some bounds using complete graphs and complete bipartite graphs as examples.

In Open Problem 3.27 of \cite{3}, the non-repetitive coloring of $K_n\square K_n$ plays an important role in determining the edge-nonrepetitive chromatic number $K_n$. Below we present a construction of non-repetitive coloring of $K_n\square K_n$, to prove that $\pi(K_n\square K_n)$ is at most $(\frac{1}{2}+o(1))n^2$.\\

\begin{theorem}
For even $n\ge 4,\pi(K_n\square K_n)\le\frac{n^2}{2}.$
\end{theorem}

\begin{proof}
\begin{figure}[h] 
\centering
\includegraphics[width=0.6\linewidth]{figure11.png}
\caption{}
\end{figure}
We arrange the vertices of $K_n\square K_n$ into a square grid such that each row and each column forms a complete graph, and color the $n^2$ vertices as follows: 
	
Divide the vertices of the graph into two parts, left and right. We then color the left half of the vertices with different colors in the order from left to right and top to bottom.
	
Then divide the vertices of the graph into two types, blue and yellow. The following conditions must be satisfied: Each row has vertices of only one type on the left side and the other type on the right side. The type of each half-row is different from the type of its adjacent half-rows.
	
On the left half of the vertices with yellow type, copy the colors of each half-row’s vertices to the bottom right of the corresponding right half with yellow type. On the left half of the vertices with blue type, copy the colors of each half-row’s vertices with a one-unit rightward shift permutation in the bottom right of the corresponding right half with blue type. Note that at this moment, we consider the top side and the bottom side as the same side. We give an example of $K_8\square K_8$ in Figure 11.
	
We can verify that under this coloring, if there are two edges in the graph with the same starting color and the same ending color (colors are represented by numbers in Figure 11, not types), then they must also have the same type (blue or yellow). It proves that there is no repetitive path under this coloring.
\end{proof}

As a straightforward corollary, $\pi(K_n\square K_n)\le(\frac{1}{2}+o(1))n^2$ when $n\to\infty$.\\

Now consider the Cartesian product of two complete bipartite graphs $K_{n,n}$. Using a similar construction, we can also obtain an upper bound for their non-repetitive chromatic number.

\begin{theorem}
$\pi(K_{n,n}\square K_{n,n})\le(\frac{3}{2}+o(1))n^2$ when $n\to\infty$.
\end{theorem}

\begin{proof}
\begin{figure}[h] 
\centering
\includegraphics[width=0.47\linewidth]{figure12.png}
\caption{}
\end{figure}

We arrange the vertices of $K_{n,n}\square K_{n,n}$ into a square grid such that each row and each column forms a complete bipartite graph. In Figure 12(we take $n=4$ as an example), we label the two independent sets of each complete bipartite graph with red and green vertices. Next, we divide all the vertices into four equal parts: upper left, lower left, upper right, and lower right, and we color them as follows:
	
Vertices in the upper left part and the lower right part get a common color 0.
	
Vertices in the upper right part get a non-repetitive coloring pattern using $(\frac{1}{2}+o(1))n^2$ colors.
	
Vertices in the lower left part get other $n^2$ different colors.
	

	
In any path in $K_{n,n}\square K_{n,n}$, color 0 always appears alternately. Moreover, note that each vertex in the lower left part has a different color. We only need to consider the projection of the color sequence in the upper right part. Since the coloring in the upper left part is non-repetitive, the entire graph’s coloring is also non-repetitive. This completes the proof.
	
\end{proof}

We ask the following question:

\textbf{Open problem}\quad Is there any constant $c>0$ such that $\pi(K_n\square K_n)\ge cn^2$ or \\$\pi(K_{n,n}\square K_{n,n})\ge cn^2$?

In other words, are the bounds provided by the above two theorems tight up to a constant factor? If the answer is yes, it would lead to significant improvements in the lower bounds of the non-repetitive chromatic number of graphs with given maximum degree.

\section{Acknowledgments}
Thanks to Hehui Wu, Qiqin Xie and Wentao Zhang for the early discussion of this project and the helpful comments.

\begin{thebibliography}{99}
\bibitem{1}
A. Thue. Über unendliche Zeichenreichen. $Norske$ $Vid.$ $Selsk.$ $Skr.$ $I.$ $Mat.$ $Nat.$ $Kl.$ $Christiana$, 7(1906), 1-22.
	
\bibitem{2}
André Kündgen and Michael J. Pelsmajer. Nonrepetitive colorings of graphs of bounded tree-width. $Discrete$ $Math$., 308(19):4473–4478, 2008. doi: 10.1016/j.disc.2007.08.043. MR: 2433774. 
	
\bibitem{3}
David R. Wood. Nonrepetitive graph colouring. $Electron.$ $J.$ $Combin.$, DS24, 2021.
	
\bibitem{4}
Noga Alon, Jarosław Grytczuk, Mariusz Hałuszczak, and Oliver Riordan. Nonrepetitive Colorings of Graphs. $Random$ $Structures$ $Algorithms$ 21.3-4 (2002): 336-46, 2002.
	
\bibitem{5}
Vida Dujmović, Gwenaël Joret, Jakub Kozik and David R. Wood. Nonrepetitive colouring via entropy compression. $Combinatorica$, 36(6):661-686, 2016. doi:10.1007/s00493-015-3070-6.
	
\bibitem{6}
János Barát, Július Czap. Vertex coloring of plane graphs with nonrepetitive boundary paths, Preprint, arXiv:1105.1023, 2011.
	
\bibitem{7}
János Barát, Péter Varjú. On square-free vertex colorings of graphs, $Studia$ $Scientiarum$ $Mathematicarum$ $Hungarica$, 44(3):411–422, 2007.
	
\bibitem{8}
Vida Dujmović, Louis Esperet, Gwenaël Joret, Bartosz Walczak, and David R. Wood. Planar Graphs Have Bounded Nonrepetitive Chromatic Number. $Advances$ $in$ $Combinatorics$, 5, 2020.
	
\bibitem{9}
Douglas B. West $Combinatorial$ $Mathematics$. 2021. Print.
	
\bibitem{10}
Vida Dujmović , Gwenaël Joret, Piotr Micek, Pat Morin, Torsten Ueckerdt, and David R. Wood. Planar Graphs Have Bounded Queue-Number. $Journal$ $of$ $the$ $ACM$ 67.4 (2020): 1-38.
	
\bibitem{11}
János Barát and David R. Wood. Notes on nonrepetitive graph colouring. $Electron.$ $J.$ $Combin.$, 15:R99, 2008.
	
\bibitem{12}
Prosenjit Bose, Vida Dujmović, Pat Morin, and Lucas Rioux-Maldague. New bounds for facial nonrepetitive colouring. $Graphs$ $Combin.$, 33(4):817–832, 2017.
	
\bibitem{13}
Boštjan Brešar, Jarosław Grytczuk, Sandi Klavžar, Stanisław Niwczyk, and Iztok Peterin. Nonrepetitive colorings of trees. $Discrete$ $Math.$, 307(2):163–172, 2007.

\bibitem{14}
Francesca Fiorenzi, Pascal Ochem, Patrice Ossona de Mendez, and Xuding Zhu. Thue choosability of trees. $Discrete$ $Applied$ $Math.$, 159(17):2045–2049, 2011.

\bibitem{15}
Jarosław Grytczuk, Jakub Kozik, and Piotr Micek. A new approach to nonrepetitive sequences. $Random$ $Structures$ $Algorithms$, 42(2):214–225, 2013.
	
\bibitem{16}
Jarosław Grytczuk. Nonrepetitive colorings of graphs a survey. $International$ $Journal$ $of$ $Mathematics$ $and$ $Mathematical$ $Sciences$ 2007 (2007): 1-10. 
	
\bibitem{17}
Jochen Haranta and Stanislav Jendrol’. Nonrepetitive vertex colorings of graphs. $Discrete$ $Math.$, 312(2):374–380, 2012.
	
\bibitem{18}
Kashyap Kolipaka, Mario Szegedy, and Yixin Xu. A sharper local lemma with improved applications. In $Approximation,$ $Randomization,$ $and$ $Combinatorial$ $Optimization.$	$Algorithms$ $and$ $Techniques,$ vol. 7408 of $Lecture$ $Notes$ $in$ $Comput.$ $Sci.$, pp. 603–614. 2012.
	
\bibitem{19}
Matthieu Rosenfeld. Another approach to non-repetitive colorings of graphs of bounded degree. $Electron.$ $J.$ $Combin.$, 27:P3.43, 2020.
	
\bibitem{20}
János Barát and Péter P. Varjú. On square-free edge colorings of graphs. $Ars$ $Combin.$, 87:377–383, 2008.
	
\bibitem{21}
Manuel Francesco Aprile. Constructive aspects of Lovász local lemma and applications to graph colouring. Master’s thesis, University of Oxford, 2014
	
\bibitem{22}
Boštjan Brešar and Sandi Klavžar. Square-free colorings of graphs. $Ars$ $Combin.$, 70:3–13, 2004.
	
\bibitem{23}
André Kündgen and Tonya Talbot. Nonrepetitive edge-colorings of trees. $Discrete$ $Math.$ $Theor.$ $Comput.$ $Sci.$, 19(1), 2017.

\bibitem{24}
P. Erdős and L. Lovász, Problems and results on 3-chromatic hypergraphs and some related questions, in $Infinite$ $and$ $Finite$ $Sets$, Vol. II, A. Hajnal, R. Rado, and V. T. Sós, eds., North-Holland, Amsterdam, 1975, pp. 609-627.
	
\bibitem{25}
Michael Molloy and Bruce Reed. Graph colouring and the probabilistic method, vol. 23 of $Algorithms$ $and$ $Combinatorics.$ Springer, 2002. 
	
\bibitem{26}
Robin A. Moser and Gábor Tardos. A constructive proof of the general Lovász local lemma. $J.$ $ACM,$ 57(2), 2010.
	
\end{thebibliography} 
\end{document}
