\documentclass[12pt]{article}
\usepackage{graphicx}
\usepackage{subfigure}
\usepackage{amsthm,amsfonts}
\usepackage{amssymb}
\usepackage{fullpage}
\usepackage{caption}
\usepackage{pifont}
\newtheorem{theorem}{Theorem}
\newtheorem{corollary}{Corollary}
\captionsetup[figure]{name={Figure},labelsep=period}

\begin{document}
\title{The non-repetitive colorings of grids}
\author{Tianyi Tao \\
School of mathematical sciencese \\
Fudan University \\
{\tt tytao20@fudan.edu.cn}}
\date{\today}
\maketitle

\begin{abstract}
A vertex coloring of a graph $G$ is non-repetitive if $G$ contains no path for which
the first half of the path is assigned the same sequence of colors as the second half.
Thue shows that path $P_n$ is non-repetitively 3-colorable. We prove that the grid $P_n\square P_n$ is non-repetitively 12-colorable and extending this method to high-dimensional situations.
\end{abstract}

\section{Introduction}

A vertex coloring of a graph $G$ is non-repetitive if $G$ contains no path for which
the first half of the path is assigned the same sequence of colors as the second half. We call the minimal number of colors needed the Thue number of $G$ and denote it by $\pi(G)$.

Thue \cite{1} proved that $\pi(P_n)=3$ for path of any number of vertices $n$, in other words, there is a sequence composed of three symbols of $a,b$ and $c$ of any length, in which there is no identical adjacent blocks. We take the first few bits of this sequence as examples of common using in the later parts of the paper.

\begin{figure}[h] 
	\centering
	\includegraphics[width=0.8\linewidth]{figure-1.png}
\end{figure}

\begin{figure}[h] 
	\centering
	\includegraphics[width=0.8\linewidth]{figure-2.png}
\end{figure}

A block of symbols is called a palindrome if it is equal to its reflection. For instance, $aa,aba,abcba$ are palindromes. It is easy to see that a non-repetitive sequence composed of three symbols with a length of more than 5 must contain a palindrome. In the sequence above, we can insert the fourth symbol $d$ between consecutive blocks of length 2.

It is easy to check that such construction can obtain non-repetitive and palindrome-free sequences of arbitrary length.

Now let's review the definitions of Cartesian product, tensor product, and strong product. Let $G$ and $H$ be graphs. The Cartesian product of $G$ and $H$, denoted by $G\square H$, is the graph with vertex set $V(G)\times V(H)$, where $(g,h)$ and $(g',h')$ are adjacent if $g=g',hh'\in E(H)$ or $h=h',gg'\in E(G)$. The tensor product of $G$ and $H$, denoted  by $G\times H$, is the graph with vertex set $V(G)\times V(H)$, where $(g,h)$ and $(g',h')$ are adjacent if $gg'\in E(G)$ and $hh'\in E(H)$. The strong product of $G$ and $H$, denoted  by $G\boxtimes H$, is the union of $G\square H$ and $G\times H$.

Now let's talk about the non-repetitive colorings of grid $P_n\square P_n$ for any fixed $n$. We can give the structure\cite{2} in Figure 1 through the 4-symbol sequence, so that $\pi(P_n\square P_n)\le 16.$ 

\begin{figure}[h] 
\centering
\begin{minipage}[t]{0.48\textwidth}
\centering
\includegraphics[width=\textwidth]{figure1.png}
\caption{}
\end{minipage}\hfill
\begin{minipage}[t]{0.48\textwidth}
\centering
\includegraphics[width=\textwidth]{figure2.png}
\caption{}
\end{minipage}
\end{figure}


It should be mentioned that if the four horizontal or vertical symbols are replaced by three, the resulting coloring may no longer be non-repetitive due to the appearance of palindromes. In the coloring obtained by product, the repetitive path, for example, the path in red in Figure 2 is hard to avoid. This topic will be discussed in depth in the last section.

This paper will give another construction and shows $\pi(P_n\square P_n)\le12$ and extending this construction to high-dimensional situations.


\section{Two-dimensional case}

There is an important observation below.

We give the following coloring Figure 3 for $P_n\square P_n$: if the distance from a vertex to the upper left corner is $i$, let its color be the $(i+1)$-th bit in the sequence (II). In such coloring, the color sequence of any path in the grid will correspond to the color sequence of a walk in (II). 

\begin{figure}[h] 
	\centering
	\includegraphics[width=0.5\linewidth]{figure3.png}
	\caption{}
\end{figure}

A lazy walk in a graph $G$ is a walk in the pseudograph obtained from $G$ by adding a loop at each vertex.\\


\begin{theorem}
If $W=\{v_1,\cdots,v_k,v_{k+1},\cdots v_{2k}\}$ is a repetitive lazy walk in (II), which means $v_i$ and $v_{k+i}$ have the same color forall $1\le i\le k$, then $v_i=v_{k+i}$ forall $1\le i\le k$. In other words, any repetitive lazy walk is composed of two identical lazy walks.
\end{theorem}

\begin{proof}
For $1\le i\le k-1,$ if $v_i$ and $v_{i+1}$ are in the same location, then $v_{k+i}$ and $v_{k+i+1}$ are also in the (perhaps another)same location. So we can assume that $W$ is a walk without losing generality. The contents of walk and lazy walk is discussed in detail in \cite{3}

We discuss the shape of the walk in two cases.

If $v_1$ to $v_2$ and $v_{k+1}$ to $v_{k+2}$ are in the same direction, because there is no palindrome in the sequence (II), $v_2$ to $v_3$ and $v_{k+2}$ to $v_{k+3}$ are in the same direction. Continuing this process, we can see that the first half and the second half of $W$ are equal in the sense of translation. If $v_1$ and $v_{k+1}$ are in different positions in the sequence (II), then it will conflict with the property of no repetitive block in (II).

If $v_1$ to $v_2$ and $v_{k+1}$ to $v_{k+2}$ are in the opposite direction, in the same way, we can show that the first half and the second half of $W$ are equal in the sense of symmetry. Whether $v_1$ and $v_{k+1}$ are in the same position or not, it will conflict with the property of no palindrome in (II).

To sum up, there is only one possibility for the composition of $W$, that is, $W$ is composed of two indetical walks.

\end{proof}

\begin{theorem}
$P_n\square P_n$ is non-repetitively 12-colorable forall $n$.
\end{theorem}

\begin{proof}
Construct the following coloring of grid in Figure 4.

\begin{figure}[h] 
	\centering
	\includegraphics[width=0.6\linewidth]{figure4.png}
	\caption{}
\end{figure}

We divide the vertices on the grid into two types according to the parity of the distance to the upper left corner. Each type of vertex is divided into the disjoint union of vertices on parallel lines, and the vertices on each line are colored with the same color according to the order in (II), where $x, y, z, w$ are four new colors with the same status as $a, b, c, d$. It should be noted that in this construction, two lines of different types are not parallel.

In Figure 4, there are many repetitive paths, such as $a\to z\to a\to z$ in the red box. Our goal is to prove that we can destroy all the repetitive paths in Figure 4 by destroying the repetitive path with length of 4 above. 

Let $P=\{v_1,\cdots,v_k,v_{k+1},\cdots v_{2k}\}$ is a repetitive path in Figure 4, then $k$ must be even. Without losing generality, set the color of $v_1$ is $a$ and the color of $v_2$ is $z$. Let $P_1=\{v_1,v_3,\cdots,v_{k-1},v_{k+1},\cdots v_{2k-1}\}$ and $P_2=\{v_2,v_4,\cdots,v_{k},v_{k+2},\cdots v_{2k}\}$, then $P_1$ and $P_2$ are repetitive lazy walks in each 4-color sequence. Using \textbf{Theorem 1} to $P_1$ and $P_2$, $v_{k+1}$ must be on line $L_1$ and $v_{k+2}$ must be on line $L_2$ in Figure 5.

\begin{figure}[h] 
	\centering
	\begin{minipage}[t]{0.48\textwidth}
		\centering
		\includegraphics[width=\textwidth]{figure5.png}
		\caption{}
	\end{minipage}\hfill
	\begin{minipage}[t]{0.48\textwidth}
		\centering
		\includegraphics[width=\textwidth]{figure6.png}
		\caption{}
	\end{minipage}
\end{figure}

The locations of $v_{k+1}$ and $v_{k+2}$ are uniquely determined, that is, $v_{k+1}$ is exactly the vertex on the left of $v_2$, and $v_{k+2}$ is exactly the vertex on the left of $v_1$.

We divide $x, y, z, w$ into $x_1, x_2, y_1, y_2, z_1, z_2, w_1, w_2$ by parity in Figure 6, then all repetitive paths are destroyed. As a result, the grid is non-repetitively 12-colorable.

\end{proof}

\section{High-dimensional case}

Let $P$ be an infinite path. Denote $n$ $P$'s Cartesian product, strong product and tensor product by $(\square P)^n$, $(\boxtimes P)^n$ and $(\times P)^n$. 

By constructing non-repetitive palindrome-free sequences on each coordinate and multiplying them (Figure 1 is two-dimensional case), we know that $\pi(\boxtimes P)^n\le 4^n$.\cite{2} Now we talk about $(\times P)^n$ first.

For $n=2$, $P\times P$ has 2 components, each of which is isomorphic to $P\square P$, so $\pi(P\times P)=\pi(P\square P)\le 12$. For $n\ge3$, $(\times P)^n$ has $2^{n-1}$ components, each of which is different from $(\square P)^n$. Notice that $(\times P)^n$ is $2^n$-regluar but $(\square P)^n$ is $2n$-regular.\\

\begin{theorem}
	In $(\boxtimes P)^n$ with the non-repetitive $4^n$ coloring above, any repetitive lazy walk is composed of two indentical lazy walks.
\end{theorem}

\begin{proof}
	Let $W=\{v_1,\cdots,v_k,v_{k+1},\cdots v_{2k}\}$ is a repetitive lazy walk, if $v_1$ and $v_{k+1}$ are in the same location, by the coloring is palindrome-free, then $v_i$ and $v_{k+i}$ are in the same location forall $1\le i\le k$. If $v_1$ and $v_{k+1}$ are not in the same location, then one of the coordinates of these two vertices must be different. Just look at the color on this coordinate, and the contradiction is given by \textbf{Theorem 1}.
\end{proof}	

\begin{theorem}
	$(\times P)^n$ is non-repetitively $(4^{n-1}+4\cdot2^{n-1})$-colorable.
\end{theorem}

\begin{proof}
Consider one of the component of $(\times P)^n$. According to the geometric viewpoint in $n$-dimensional Euclidean space, the graph is divided into a series of parallel $(n-1)$-dimensional hyperplanes that intersect the graph. Let the equations for these hyperplanes be $x_n=\lambda, \lambda\in\mathbb{Z}$.

For odd $\lambda$, let the vertices on each hyperplane be the same color, which is taken from (II) consecutively. For example, all vertices satisfies $x_n=1$ get the color $a$, all vertices satisfies $x_n=3$ get the color $b$, all vertices satisfies $x_n=5$ get the color $d$, all vertices satisfies $x_n=7$ get the color $c$ and all vertices satisfies $x_n=9$ get the color $a\cdots$

For even $\lambda$, the vertices on each hyperplane share the same set of $4^{n-1}$ colors, that is, all vertices on the line $x_1=i_1,\cdots,x_{n-1}=i_{n-1}$ share a same color. Figure 7 is a schematic diagram when $n$=3, whose dashed lines are not edges in the graph.

\begin{figure}[h] 
	\centering
	\begin{minipage}[h]{0.48\textwidth}
		\centering
		\includegraphics[width=0.9\textwidth]{figure7.png}
		\caption{}
	\end{minipage}\hfill
	\begin{minipage}[h]{0.48\textwidth}
		\centering
		\includegraphics[width=\textwidth]{figure8.png}
		\caption{}
	\end{minipage}
\end{figure}

Let $P=\{v_1,\cdots,v_k,v_{k+1},\cdots,v_{2k}\}$ be a repetitive path in $(\times P)^n$ with the coloring above, let $P_1$ and $P_2$ be paths formed by the odd number and even number of vertices of $P$. Without losing generality, we assume that $v_1$ is on a hyperplane with an odd coefficient $\lambda$, so $P_1$ is a repetitive lazy walk in a path with 4-colored non-repetitive palindrome-free coloring and $P_2$ is a repetitive lazy walk in $(\boxtimes P)^{n-1}$ with the $4^{n-1}$-coloring in \textbf{Theorem 3}. As a result, $v_1$ and $v_{k+1}$ are in a same hyperplane (they have the same $n$-th coordinate), $v_2$ and $v_{k+2}$ are in a same line (they have the same $1,\cdots,n-1$-th coordinates).

On the hyperplanes of odd coefficients $\lambda$, by divding each color into $2^{n-1}$ colors, all repetitive paths can be destroyed (Figure 8). This shows $\pi((\times P)^n)\le 4^{n-1}+4\cdot2^{n-1}$.

\end{proof}

\newpage

\begin{theorem}
$(\square P)^n$ is non-repetitively $(4^{n-1}+4n)$-colorable.	
\end{theorem}

\begin{proof}
Similar to the proof of \textbf{Theorem 4}, the only difference is that we take the hyperplane sequence of the equations $\sum\limits_{j=1}^n=\lambda,\lambda\in\mathbb{Z}$ at this time.

Let $P=\{v_1,\cdots,v_k,v_{k+1},\cdots,v_{2k}\}$ be a path in $(\square P)^n$ which start at a hyperplane with an odd coefficient $\lambda$. Similarly, let $P_1$ and $P_2$ be the paths formed by the odd number and even number of vertices of $P$.

For odd $\lambda$, let the vertices on each hyperplane be the same color, which is taken from (II) consecutively. 

For even $\lambda$, we want all the hyperplanes to share a coloring. In this situation, $P_2$ is a lazy walk in a graph obtained by filling the entire $(n-1)$-dimensional Euclidean space with infinite regular $(n-1)$-dimensional simplexes. Note that it is a subgraph of $(\boxtimes P)^{n-1}$, we can give them a $4^{n-1}$ coloring to use \textbf{Theorem 3}. Figure 9 is a schematic diagram when $n$=3.

If $u$ is a vertex of $(\square P)^n$ in the hyperplane $\sum\limits_{j=1}^n=\lambda_0$, then the neighbours of $u$ in the hyperplane $\sum\limits_{j=1}^n=\lambda_0+1$ is exactly the $n$ vertices of an $(n-1)$-dimensional simplex.

Note that the chromatic number of the graph obtained by filling the entire $(n-1)$-dimensional Euclidean space with infinite regular $(n-1)$-dimensional simplexes is exactly $n$, we can destroy all repetitive paths by dividing each color into $n$ colors in the hyperplanes with odd coefficient $\lambda$ (Figure 10). This shows $\pi((\square P)^n)\le 4^{n-1}+4n$.

\begin{figure}[h] 
	\centering
	\begin{minipage}[h]{0.48\textwidth}
		\centering
		\includegraphics[width=\textwidth]{figure9.png}
		\caption{}
	\end{minipage}\hfill
	\begin{minipage}[h]{0.48\textwidth}
		\centering
		\includegraphics[width=\textwidth]{figure10.png}
		\caption{}
	\end{minipage}
\end{figure}

\end{proof}

Unfortunately, \textbf{Theorem 5} only gives a good bound when $n$ is small (for example, $n$=3). For large $n$, we know $\pi((\square P)^n)=O(n^2)$ by Lovász local lemma.\cite{4}\cite{5} Bounding the non-repetitive chromatic number of a graph obtained by filling $n$-dimensional space with $n$-dimensional simplexes maybe a method to improve \textbf{Theorem 5}.


\section{From the perspective of stroll-nonrepetitive coloring}

A walk $\{v_1,\cdots,v_k,v_{k+1},\cdots,v_{2k}\}$ in a graph is boring if $v_i=v_{k+i}$ forall $i$, a coloring is walk-nonrepetitive if every repetitive colored walk is boring. For a graph $G$, the walk-nonrepetitive chromatic number $\sigma(G)$ is the minimum number of colors in a walk-nonrepetitive coloring of $G$. \textbf{Theorem 3} means that $\sigma((\boxtimes P)^n)\le 4^n$.

A stroll in a graph $G$ is a walk $\{v_1,\cdots,v_k,v_{k+1},\cdots,v_{2k}\}$ such that $v_i\not=v_{k+i}$ for all $i$. A coloring of $G$ is stroll-nonrepetitive if no stroll is repetitively colored.
For a graph $G$, the stroll-nonrepetitive chromatic number $\rho(G)$ is the minimum number
of colors in a stroll-nonrepetitive coloring of $G$. 

Obviously we have $\pi(G)\le\rho(G)\le\sigma(G)$ for any graph $G$.

In the \textbf{Lemma 2.16} of \cite{3}, for any graphs $G, H$,we have that $\pi(G\boxtimes H)\le\rho(G\boxtimes H)\le\rho(G)\cdot\sigma(H)$. Let $G$ and $H$ be paths, then $\pi(P\square P)\le\pi(P\boxtimes P)\le 4\rho(P)$. If $\rho(P)\le3$, then $\pi(P\square P)\le12$ is obvious so that our work is unnecessary.

However, \cite{3} asked a question in the preprint on arxiv.org before publication: whether $\rho(P)\le3$ for every path $P$. Pascal Ochem first disproved this conjecture. He shows that a long path is not stroll-nonrepetitively 3-colorable. We give a simple proof of this theorem following.

\begin{theorem}
$P_n$ is not stroll-nonrepetitively 3-colorable for large n.
\end{theorem}
\begin{proof}
We know that a non-repetitive sequence composed of three symbols $a,b,c$ always contains palindromes when the length is large enough. Through simple observation, we know that if there is a 3-palindrome such as $aba$ in the sequence, it must be the middle three members of a 5-palindrome $cabac$. Let's consider the following question: What is the possible structure of the intersection between two 5-palindromes?

First of all, we should note that if we know what the two symbols after a certain 5-palindrome are, then we can determine the position of the next 5-palindrome. For example the two symbols after $cabac$ can be $ba,bc$ or $ab$, they determine the three ways following in which the next 5-palindromes appear.

Form\ding{172}: $cabacba$, the next two symbols must be $bc$. So it is $cabacbabc$, whose two adjacent 5-palindromes share only one symbol.

Form\ding{173}: $cabacbc$, the next symbol must be $a$. So it is $cabacbca$, whose two adjacent 5-palindromes share two symbols.

Form\ding{174}: $cabacab$, whose two adjacent 5-palindromes share three symbols.

Note that Form\ding{174} contains a repetitive stroll $c\to a\to b\to a\to c\to a\to b\to a$. Thus, if we have a stroll-nonrepetitive coloring for path, the Form\ding{174} can not appear.

Now we prove that Form\ding{173} cannot appear consecutively. If not, we get the symbol sequence $cabacbcabac$. If the next symbol of the sequence is $b$, there will be a repetitive path. If the next symbol of the sequence is $a$, There will be a Form\ding{174} appearing later.

Form\ding{172} cannot appear 3 consecutive times, if not, we get the symbol sequence $cabacbabcabacbabc$, which contains a repetitive path.

By these reasons above, in a sufficiently long symbol sequence, there will always be structures that Form\ding{172} followed by Form\ding{173} followed by Form\ding{172}. Then we get the symbol sequence $cabacbabcacbacab$, which contains a repetitive stroll $a\to c\to b\to a\to b\to c\to a\to c\to b\to a\to b\to c$. So $\rho(P_n)=4$ for large $n$.
\end{proof}


$\rho(P)=4$ means that it is impossible to reduce $\pi(P\boxtimes P)$ below 16 by using the product coloring method. From another perspective, when we studying non-repetitive chromatic number of graphs, the two structures of Cartesian product and strong product may be used completely different kinds of coloring methods.

\newpage
\begin{thebibliography}{99}
\bibitem{1}
A. Thue. Über unendliche Zeichenreichen. $Norske Vid. Selsk. Skr. I. Mat. Nat. Kl. Christiana$, 7(1906), 1-22.

\bibitem{2}
André Kündgen and Michael J. Pelsmajer. Nonrepetitive colorings of graphs of bounded tree-width. $Discrete$ $Math$., 308(19):4473–4478, 2008.
doi: 10.1016/j.disc.2007.08.043. MR: 2433774. 

\bibitem{3}
 David R. Wood. Nonrepetitive graph colouring. $Electron. J. Combin.$, DS24, 2021.

\bibitem{4}
Noga Alon, Jarosław Grytczuk, Mariusz Hałuszczak and Oliver Riordan. Nonrepetitive colorings of graphs. $Random$ $Structures$ $Algorithms$, 21(3-4):336-346, 2002. doi:10.1002/rsa.10057. MR:1945373.

\bibitem{5}
Vida Dujmović, Gwenaël Joret, Jakub Kozik and David R. Wood. Nonrepetitive colouring via entropy compression. $Combinatorica$, 36(6):661-686, 2016. doi:10.1007/s00493-015-3070-6.

\end{thebibliography} 

\end{document}