\documentclass[11pt,letterpaper,english]{smfart}

\linespread{1.05}     

\usepackage[french,main=english]{babel}
%\usepackage[utf8, latin1]{inputenc} 
\RequirePackage[T1]{fontenc}
\usepackage{ae,euscript,enumerate,newtxmath,newtxtext,dsfont,fullpage}  
\RequirePackage[type1]{crimson}
%\usepackage[cm]{aeguill}   
%\usepackage[a4paper,centering]{geometry}

\newcommand{\inv}{^{\raisebox{.2ex}{$\scriptscriptstyle-1$}}}   

%\tolerance=10000

\usepackage[dvipsnames]{xcolor}   
%\usepackage{xparse}
\usepackage{xr-hyper}
\definecolor{brightmaroon}{rgb}{0.76, 0.13, 0.28}
\usepackage[linktocpage=true,colorlinks=true,hyperindex,citecolor=Royal Blue,linkcolor=Royal Blue]{hyperref}   

\newtheorem{theorem}{Theorem}[section] 
\newtheorem{proposition}[theorem]{Proposition}
\newtheorem{lemma}[theorem]{Lemma}
\newtheorem{corollary}[theorem]{Corollary}
\theoremstyle{definition}
\newtheorem{definition}[theorem]{Definition}
\newtheorem{example}[theorem]{Example} 
\newtheorem{remark}[theorem]{Remark} 
%\setcounter{section}{-1}
\numberwithin{equation}{section}
       
\newcommand{\add}[1]{\textcolor{Red}{#1}}  
                  	    
\begin{document}  

\title{Novikov algebras and their  primitive ideals} 

\author{Aditya Dwarkesh}
\address[1]{Indian Institute of Science Education and Research  - Kolkata, Mohanpur, Nadia - 741246, West Bengal, India.}

\email{ad19ms047@iiserkol.ac.in} 

\author{Amartya Goswami}
\address{[1] Department of Mathematics and Applied Mathematics, University of Johannesburg, P.O. Box 524, Auckland Park 2006, South Africa.
[2] National Institute for Theoretical and Computational Sciences (NITheCS), South Africa.} 

\email{agoswami@uj.ac.za}

\begin{abstract}
The aim of this paper is to study the primitive ideals of Novikov algebras. In terms of modular maximal right ideals, a characterization of the primitive ideals of a Novikov algebra has been obtained. We prove a Chevalley-Jacobson density-type theorem for primitive Novikov algebras. We obtain some equivalences between prime, simple, and primitive Novikov algebras. We describe a subalgebra of a Novikov algebra as a Novikov algebra of endomorphisms. 
\end{abstract}  

\subjclass{17A30; 17D25}
%Nonassociative algebras satisfying other identities

%Lie-admissible algebras
\keywords{Novikov algebra; primitive ideal; simple module;  division ring; density theorem.} 
 
\maketitle 
   
\section{Introduction}
Although the terminology `Novikov algebra' has been introduced by Osborn \cite{O91}, the structure was first introduced by Gel'fand \& Dorfman \cite{GD79}, corresponding to a certain type
of Hamiltonian operators. The same algebra has also been used by Balinskii \& Novikov \cite{BN85} in connection with Poisson brackets of the hydrodynamic type. Novikov algebras were first studied as abstract algebraic structures by Zelmanov  \cite{Z87}, who demonstrated that simple finite-dimensional Novikov algebras of characteristic zero are one-dimensional. Further studies of finite dimensional Novikov algebras with characteristic $p$ can be found in Cherkasin \cite{C88}, Fillippov \cite{F89}, and in Osborn \cite{O91,O92, O95}, whereas for the case of infinite-dimensional Novikov algebras, see Osborn \cite{O94}. A complete classification of finite-dimensional simple Novikov algebras
and their irreducible modules over an algebraically closed field with a prime characteristic have been done in Xu \cite{X96}. 

It has been shown in Shestakov
\& Zhang \cite{SZ20} that solvability and right-nilpotency are equivalent in the variety of Novikov algebras. Panasenko, in \cite{P22}, has shown that  an ideal of a semiprime
Novikov algebra is semiprime Novikov algebra, and furthermore, a minimal ideal of a Novikov algebra is either trivial, or a simple
algebra. 

Jacobson \cite{J45} (see also \cite{J56}) introduced the primitive ideals of noncommutative rings, and this notion plays a crucial role in determining the structure theory of rings. Primitive ideals have been demonstrated to be important in understanding structural aspects of modules  (see Bre\v {s}ar \cite{B14} and Rowen \cite{R88}), Lie algebras (see A. A. Kucherov, O. A. Pikhtilkova, \& S. A. Pikhtilkov \cite{KPP12}), enveloping algebras (see Dixmier \cite{D96} and Joseph \cite{J83}), PI-algebras (see Jacobson \cite{J75}), quantum groups (see Joseph \cite{J95}), skew polynomial rings (see Irving \cite{I79}), and others.
Goswami \cite{G23} introduced the concept of a primitive ideal of a Novikov algebra with the goal of studying Jacobson topology on the spectrum of primitive ideals of a Novikov algebra.
The purpose of the present work is to study in detail those primitive ideals of a Novikov algebra. 

The main results of this paper are briefly summarized as follows:  In Section \ref{prid}, using the notion of a module over a Novikov algebra, we define a primitive ideal of a Novikov algebra. In a Novikov algebra, we show that every maximal ideal is primitive and every primitive ideal is prime  (Corollary \ref{maxp} and Lemma \ref{pip}). We characterize primitive ideals in terms of modular maximal right ideals (Corollary \ref{mmri}).  A Chevalley-Jacobson density-type theorem (Theorem \ref{cjdt}) has been formulated and proved in Section \ref{dent}. We obtain a consequence (Corollary \ref{psdp}) of this density theorem, which also shows equivalences between prime, simple, and primitive Novikov algebras. Finally, we describe a subalgebra of a Novikov algebra as a  Novikov algebra of endomorphisms (Theorem \ref{acna}).

\section{Primitive ideals} \label{prid}

Recall that a (\emph{left}) \emph{Novikov algebra} is a vector space $\mathcal{A}$ over a field $\mathds{k}$ endowed with a bilinear map such that

\begin{enumerate}[\upshape (i)]
\item \label{n1}$(x\circ y)\circ z = (x\circ z)\circ y;$

\item \label{n2}$(x, y, z) = (y, x, z),$
\end{enumerate}
where $(x, y, z)=(x\circ y)\circ z - x\circ(y\circ z)$ for all $x,$ $y,$ $z\in \mathcal{A}$.

Like in rings, the notion of a primitive ideal of a Novikov algebra depends on the the notion of a module over a Novikov algebra.
Osborn \cite{O95} (see also \cite{O91}) introduced a module over a Novikov algebra as a vector space endowed with two bilinear maps satisfying the conditions (\ref{n1}) and (\ref{n2}). We explicitly list below the set of axioms of a module over a Novikov algebra.

If $\mathcal{A}$ is a Novikov algebra over a field $\mathds{k}$, then a \emph{module over} $\mathcal{A}$ is a vector space $M$ (over $\mathds{k}$) equipped with two maps $\mathcal{A}\times M\to M$ and $M\times \mathcal{A}\to M$ satisfying the following axioms:

\begin{enumerate}[\upshape (I)]
\item $(x\circ y)m=(xm)y$;\label{a1}

\item $(mx)y=(my)x;$
	
\item $x(\alpha m+\beta m')=\alpha(xm)+\beta(xm');$\label{a2}
	
\item $(\alpha x+\beta y)m=\alpha(xm)+\beta(ym);$\label{a3}
	
\item $m(x\circ y) =(mx)y+x(my)-(xm)y;$
\label{a4}

\item $(x\circ y)m=x(ym)+(y\circ x)m-y(xm);$
	
\item  $(\alpha m+\beta m')x=\alpha(mx)+\beta(mx');$\label{a5}
	
\item $m(\alpha x+\beta y)=\alpha(mx)+\beta (my),$ \label{a6}
\end{enumerate}
for all $\alpha,$ $\beta\in \mathds{k},$ $x,$ $y\in \mathcal{A},$ and $m,$ $m'\in M$. A nonzero $\mathcal{A}$-module $M$ is said to be \textit{simple} if $M$ has no non-trivial proper submodules.

\begin{remark}\label{tsm}
Note that to define a left or a right module over a ring requires respectively a left or a right action maps, whereas to obtain a module over a Novikov algebra requires both. Since every Novikov algebra must be a module over itself, it must satisfy condition (\ref{n1}), and hence only one of the action map is not sufficient. For an alternative (but equivalent) formulation of an $\mathcal{A}$-module, see Xu \cite{X96, X01}. 	
\end{remark}

A vector subspace $  I$ of a Novikov algebra $\mathcal{A}$ is called a \textit{left} (\textit{right}) \textit{ideal} if $x   I\subseteq   I$ ($  I x\subseteq   I$) for all $x\in \mathcal{A}$.
A Novikov algebra $\mathcal{A}$ is called \textit{simple} if $\mathcal{A}$ has no non-zero proper ideals. In this paper the word ``ideal'' without modifiers will always mean two-sided ideal.

\begin{lemma}\label{anti}
	Let  $M$ be an $\mathcal{A}$-module and let $\mathrm{Ann}_{\mathcal{A}}M=\{n\in \mathcal{A} \mid n m = mn = 0,\; \forall m\in M\}.$ Then $\mathrm{Ann}_{\mathcal{A}}M$ is a two-sided ideal of $N$.
\end{lemma}

\begin{proof}
Note that the lack of associativity in a Novikov algebra makes the proof different from rings. Let $x, y\in\mathrm{Ann}_{\mathcal{A}}M,$ $n\in \mathcal{A}$, and $m\in M$. Axiom (\ref{a1}) implies
\[(xn)m=(xm)n=0n=0,\quad\text{and}\quad  (nx)m=(nm)x=0. \]
Using Axiom (\ref{a4}), we obtain
\[m(xn)=(mx)n+x(mn)-(xm)n=0, \quad\text{and}\quad  m(nx)=(mx)n+n(mx)-(nm)x=0.
\] 
By Axiom (\ref{a3}) and Axiom (\ref{a6}), we respectively get  $(x-y)m=0$ and $m(x-y)=0.$
\end{proof}

An ideal $  P$ of a Novikov algebra $\mathcal{A}$ is called a \emph{primitive ideal} if there exists a simple $\mathcal{A}$-module $M$ such that $  P$ is the annihilator $\mathrm{Ann}_{\mathcal{A}}M$. We call an ideal $  Q$ of a Novikov algebra $\mathcal{A}$ \emph{prime} if $  I   J\subseteq   Q$ implies $  I\subseteq   Q$ or $  J \subseteq   Q$ for any two ideals $  I$ and $  J$ of $\mathcal{A}$.
Like rings, we also have the following implication for a Novikov algebra.

\begin{lemma}\label{pip}
Every primitive ideal of a Novikov algebra is prime.
\end{lemma}

\begin{proof}
Although the argument is similar to the case of rings, the first equality in (\ref{mba}) follows from Axiom (\ref{a1}). To prove the claim, let $P$ be the annihilator of some simple $\mathcal{A}$-module $M$ and  $  J$ is an ideal of $\mathcal{A}$ such  that $  JM \neq 0$ (\textit{i.e.}, $  J \not\subseteq   P$). Since $M$ is simple, we must have $  JM = M.$
Now, if $  I$ is another nonzero ideal of $\mathcal{A}$ such that $I\not\subseteq P$, then
\begin{equation}
\label{mba}
(  J  I)M = (  JM)  I = M  I = M.
\end{equation}
Therefore, from (\ref{mba}) it follows that  $  J  I \not\subseteq   P.$ 
\end{proof}
 
We now give an example of a primitive ideal of a Novikov algebra. In the following example, the construction of the Novikov algebra is taken from Burde \& De Graaf \cite{BG12}.
 
\begin{example}\label{prmi}
Consider the Novikov algebra $\mathds{R}^3$ over the field $\mathds{R}$ with basis vectors $\{e_1, e_2, e_3\}$ and the product defined as follows: \[e_1 \circ e_1 = e_1,\; e_1\circ e_2 = 2e_2,\; e_1\circ e_3=2e_3,\; e_2\circ e_1=e_2,\; e_3\circ e_1=e_3,\] and
all other products are zero.
Consider the subspace $  P$ generated by $\{e_2, e_3\}$. 
Then $  P$ is the annihilator of the simple $\mathds{R}^3$-module $M$ generated by $\{e_2\}.$  Since $M$ is cyclic, $M$ is indeed a simple $\mathds{R}^3$-module, and $  P$ annihilates $M$ follows from the definition of the product operation of the Novikov algebra.
\end{example}

An $\mathcal{A}$-module $M$ is called \textit{faithful} if $\mathrm{Ann}_{\mathcal{A}}M=0$ and a Novikov algebra is called \textit{primitive} if it has a faithful simple module. A Novikov algebra $N$ is called \emph{prime} if $  I  J =0$ implies either $  I=0$ or $  J=0$, for any two ideals $  I,$ $  J$ of $\mathcal{A}$.
The following lemma is  lifted from ring theory. 

\begin{lemma}\label{bcm}
Let $\mathcal{A}$ be a Novikov algebra and $  I$ be an ideal of $\mathcal{A}$.
Let $M$ be an $\mathcal{A}/  I$-module. Then, $M$ can also be considered as an $\mathcal{A}$-module.
Conversely, if $M$ is an $\mathcal{A}$-module and $  I\subseteq \mathrm{Ann}_{\mathcal{A}}M$, then $M$ can be considered as an $\mathcal{A}/  I$-module.
\end{lemma}

Similar to noncommutative primitive and prime rings, the following is a characterization of primitive and prime Novikov algebras. The proof is analogues to rings and is a consequence of Lemma \ref{bcm}.

\begin{proposition}
If $\mathcal{A}$ is a Novikov algebra, then
$  P$ is a primitive (prime) ideal of $\mathcal{A}$ if and only if $\mathcal{A}/  P$ is a primitive (prime) Novikov algebra. 
\end{proposition}

In Lemma \ref{pip} we have seen that every primitive ideal of a Novikov algebra is prime. The following result gives a partial converse of it.

\begin{theorem}\label{prip}
Let $P$ be a prime ideal of a Novikov algebra $\mathcal{A}$ and there exist an ideal $I$ of $\mathcal{A}$ satisfying the following property:
\begin{equation}\label{pip}
P\subsetneq I\; \text{such that}\; P\subseteq J\subseteq I \implies J=I\; \text{or}\; J=P,\;\;\text{for all}\; J\in\mathcal{A}.
\end{equation}
Then $P$ is a primitive ideal of $\mathcal{A}$.
\end{theorem}

\begin{proof}
Let $I$ be an ideal of $\mathcal{A}$ satifying the property (\ref{pip}). Then it follows from the correspondence theorem that $\mathcal{A}/P$ has a minimal ideal.
To have our claim, it thus suffices to show that any prime Novikov  algebra $\mathcal{A}$ with a minimal ideal $I$ is primitive. 
It is immediate that $I$ is a simple $\mathcal{A}$-module. What remains is to show  $I$ is faithful.
Note that by Axiom (\ref{a1}), for  $j,$ $n\in \mathcal{A}$  \[ j I = 0 \implies (j n) I = (j I) n = 0.\] This implies we have a right  ideal $J$ generated by $j$ such that $JI=0$. But, $J I =0$ implies $J=0$, since $\mathcal{A}$ is prime and $I\neq 0$.
Therefore, $j\in \mathrm{Ann}_{\mathcal{A}}I\implies j\in \mathrm{Ann}_{\mathcal{A}}\mathcal{A}$. But  $\mathrm{Ann}_\mathcal{A}\mathcal{A}=0$. Therefore, $I$ is faithful.
\end{proof}

\begin{corollary}\label{maxp}
Every maximal ideal of a Novikov algebra is primitive.
\end{corollary}
\begin{proof}
The proof follows by setting $I=N$ in Theorem \ref{prip}.
\end{proof}

Based on Lemma \ref{pip} and Corollary \ref{maxp}, we have now the following implications in a Novikov algebra:
\[ \text{Maximal ideal } \Rightarrow \text{Primitive ideal} \Rightarrow \text{Prime ideal}.\]
It is therefore natural to ask for an example of a primitive ideal that is not maximal, and we provide that below. Once again, the underlying Novikov algebra is taken from Burde \& De Graaf \cite{BG12}. 

\begin{example}
Consider $\mathds{R}^4$ as a vector space over $\mathds{R}$ with basis vectors $\{e_1, e_2, e_3, e_4\}$ and the bilinear map is defined as follows:
\[
e_1 e_1=e_1,\;e_1 e_2=e_2+e_3,\;e_1 e_3=e_3 e_1=e_3,\;e_2 e_1=e_2,
\]
and all other products are zero.
Consider the subspace $I$ generated by $\{e_4\}$.
Since $I$ is properly contained in the subspace generated by $\{e_1, e_4\}\subsetneq \mathds{R}^4$, the ideal $I$ is not maximal.
On the other hand,  $I=\mathrm{Ann}_{\mathcal{A}}M$, where $M$ is the simple $\mathcal{A}$-module  generated by $\{e_1\}$. 
\end{example} 

The following lemma is lifted from ring theory.

\begin{lemma}\label{snap}
Every simple Novikov algebra is primitive.
\end{lemma}

\begin{remark}
The proof of Lemma \ref{snap} follows the usual procedure of showing that the algebra forms a simple faithful module over itself. However, note that for associative rings, the assumption of being unital is also required. 

This is because, in general, simple rings do not form a simple module over themselves. Since modules are typically one-sided in the usual setting, and a ring being simple means they have no non-trivial \textit{two}-sided ideals, the one-sided module they form over themselves could fail to be simple because the ring has, say, a non-trivial left ideal.

However, as per Remark \ref{tsm}, since a Novikov algebra module must necessarily be two-sided, this complication is avoided. Because of the same reasoning, in Corollary \ref{maxp}, we also do not require a unit element in a Novikov algebra.
\end{remark}


Our next step is to characterize primitive ideals of a Novikov algebra in terms of maximal modular (one-sided) ideals (see Corollary \ref{mmri}). Similar to rings, we say that
a right ideal $I$ of a Novikov algebra $\mathcal{A}$ is called \emph{modular} if there exists an $e\in \mathcal{A}$ such that for all $a\in \mathcal{A}$, $a-ae\in I.$ Using modular maximal left ideals, a characterization of simple $\mathcal{A}$-modules is given below.

\begin{theorem}
Let $\mathcal{A}$ be a Novikov algebra. An $\mathcal{A}$-module $M$ is a simple if and only if $M$ is isomorphic to $\mathcal{A}/J$, where $J$ is a modular maximal left ideal of $\mathcal{A}$.
\end{theorem}

\begin{proof}
Suppose $M$ is a simple $\mathcal{A}$-module. If $0\neq u\in M$, then we claim that $\mathcal{A}u = M.$ To show this, it suffices to show that $\mathcal{A}u$ is an additive subgroup of $M$; equality will then follow from the fact that $M$ is non-zero and simple.
Let $x =au$ and $ x'=a' u$ for some $a,$ $a'\in \mathcal{A}$. Then, by Axiom (\ref{a6}), $x-x'=(a-a')u\in \mathcal{A}u$. 
	
Next, consider the map $\phi\colon \mathcal{A}\to \mathcal{A}u,$ defined by $\phi(a)=au$. That $\phi$ is a $\mathcal{A}$-module homomorphism follows from the bilinearity of $\circ$. Furthermore, $\ker \phi=\mathrm{Ann}_{\mathcal{A}}u$. By the first isomorphism theorem, it follows that $M= \mathcal{A}u\cong \mathcal{A}/\mathrm{Ann}_{\mathcal{A}}u.$

It remains to show that $J:=\mathrm{Ann}_{\mathcal{A}}u$ is a modular maximal left ideal. From Lemma \ref{anti}, it follows that $J$ is a left ideal of $\mathcal{A}$. Since $M=u\mathcal{A}$, there exists an $e\in \mathcal{A}$ such that $u=ue.$ Then for $a\in \mathcal{A},$
$au=(au)e=(ae)u$ (by Axiom (\ref{a1})) or, $(a-ae) u=0,$ which implies $a-ae \in J$, proving that $J$ is a modular right ideal. The maximality of $J$ now follows from Lemma \ref{bcm}, namely the correspondence between submodules of $\mathcal{A}/J$ and ideals of $\mathcal{A}$ containing $J$.
	
To show the converse, let $M=\mathcal{A}/J$, where $J$ is a modular maximal right ideal of $\mathcal{A}$. Then it is immediate that $M$ has no proper submodule. For if $M'$ were a non-zero proper submodule of $\mathcal{A}/J,$ its pre-image under the quotient map would be a proper right ideal containing $J$, contradicting its maximality. Thus, $M$ is simple.
\end{proof}

\begin{corollary}\label{mmri}
An ideal $I$ is primitive in a Novikov algebra $\mathcal{A}$ if and only if
\[I = \{x\in \mathcal{A}\mid  xa,\, ax\in J \; \text{for all}\; a\in \mathcal{A}\} \]
for some modular maximal right ideal $J$ of $\mathcal{A}$.
\end{corollary}

An ideal $I$ of a Novikov algebra $\mathcal{A}$  is called a \textit{nil} ideal if for each $x\in I$, there is some positive integer $n$ such that $x^n = 0$.
The Jacobson radical of a Novikov algebra, denoted by $\mathrm{Jac} \,\mathcal{A}$, is defined to be the intersection of all  primitive ideals of $\mathcal{A}$. If $\mathcal{A}$ has no primitive ideals, we define $\mathrm{Jac}\,\mathcal{A}=\mathcal{A}$.
Observe that $\mathrm{Jac}\,\mathcal{A}$ is also a two-sided ideal of $\mathcal{A}$. The relation between a nil ideal and the Jacobson radical of a Novikov algebra is as follows.

\begin{proposition}
If $I$ is a nil ideal of $\mathcal{A},$ then $I\subseteq \mathrm{Jac}\,\mathcal{A}$.
\end{proposition}

\begin{proof}
We prove the claim by contradiction. Let $I$ be a nil ideal of $\mathcal{A}$ such that $I\subsetneq \mathrm{Jac}\,\mathcal{A}$. This implies there exists an $u\in I$, a simple $\mathcal{A}$-module $M$, and an $m\in M$ such that $n = m u\neq 0$. Since $\mathcal{A} n = M$, there is an $x\in \mathcal{A}$ such that $x n = m$. Set $v:=x u\in I$. Then \[v n = (x u) n = (x n)  u = m u  = n,\]
where the second equality follows by Axiom (\ref{a1}). Therefore, $v^k n = n,$ for all positive integers $k$, contradicting the fact that $v$ is nilpotent.
\end{proof}

\section{Density Theorems}\label{dent}

Our aim in this section is to obtain a Chevalley-Jacobson density-type theorem for a primitive Novikov algebra. Once we have that, similar to rings, we obtain a series of consequences. To this end, we first construct a Novikov algebra of endomorphisms. 
Let $\mathcal{A}$ be a Novikov algebra. We define a bilinear map $\circ''\colon \mathrm{End}_{\mathds{k}}\mathcal{A}\times \mathrm{End}_{\mathds{k}}\mathcal{A}\to \mathrm{End}_{\mathds{k}}\mathcal{A}$ as \[(f\circ''g)(a)=f(a)\circ g(a),\]
where $f,$ $g\in \mathrm{End}_{\mathds{k}}\mathcal{A}$ and $a\in \mathcal{A}$. We claim the following.

\begin{lemma}\label{endn}
$(\mathrm{End}_{\mathds{k}}\mathcal{A}, \circ'')$ is a Novikov algebra.
\end{lemma}

\begin{proof}
For $f,$ $g,$ $h\in \mathrm{End}_{\mathds{k}}\mathcal{A},$ we have
\begin{align*}
((f \circ'' g)\circ'' h)(x) &= ((f \circ'' g)(x))\circ h(x) \\&= (f (x)\circ  g(x))\circ  h(x)\\ &= (f (x)\circ h(x))\circ  g(x)  = ((f \circ'' h) \circ'' g)(x),
\end{align*}
hence satisfies Axiom (\ref{n1}). For Axiom (\ref{n2}),
\begin{align*}
(f,g,h)(a)&=((f\circ''g)\circ''h)(a)-(f\circ''(g\circ h))(a)\\
&=(f\circ'' g)(a)\circ h(a)-f(a)\circ (g\circ'' h)(a)\\&=(f(a)\circ g(a))\circ h(a)-f(a)\circ (g(a)\circ h(a))\\
&=(g(a)\circ f(a))\circ h(a)-g(a)\circ (f(a)\circ h(a))\\ 
&=((g\circ'' f)\circ'' h)(a)-(g\circ (f\circ'' h))(a)=(g,f,h)(a). \qedhere
\end{align*}
\end{proof}

Let $\mathcal{A}$ be a Novikov algebra over a division ring $\Delta$, and let $\mathrm{End}_{\Delta}(\mathcal{A})$ be the  endomorphism algebra. A subalgebra $\mathcal{E}$ of $\mathrm{End}_{\Delta}(\mathcal{A})$ is called a \textit{dense algebra of linear operators} of $\mathcal{A}$ if between any two finite subsets $U$ and $V$ of $\mathcal{A}$ of equal cardinality wherein $U$ is linearly independent, there is an $f\in \mathcal{E}$ such that $f(u)=v$, for each $u\in U$ and $v\in V$.

\begin{theorem}[Density Theorem]\label{cjdt}
Let $\mathcal{A}$ be a Novikov algebra over a division ring. Then, if $\mathcal{A}$ is primitive, $\mathcal{A}$ is isomorphic to a dense algebra of operators of a Novikov algebra over a division ring.
\end{theorem}

\begin{proof}
Suppose $\mathcal{A}$ is a primitive Novikov algebra and $M$ is a simple faithful $\mathcal{A}$-module. Observe that $\mathrm{End}_\mathcal{A} M$ is a division ring. For addition being pointwise and composition as multiplication, $\mathrm{End}_\mathcal{A}M$ is a ring and since $M$ is simple, for every nonzero $f\in \mathrm{End}_\mathcal{A}M$, we have $\mathrm{im}\,f=M$ and $\mathrm{ker}\,f=0$, so that $f$ is invertible.

We impose a Novikov algebra (over the division ring $\mathrm{End}_\mathcal{A} M$) structure on $M$. The vector space (over $\mathrm{End}_\mathcal{A} M$) structure on $M$ comes naturally from the scalar multiplication $\mathrm{End}_\mathcal{A} M\times M\to M$ defined by $(f,m)\mapsto f(m)$. Next we endow a bilinear map $\circ': M\times M\rightarrow M$ defined as
\[
m_1\circ' m_2 = (a\circ a') m,
\] where $m_1=am, m_2=a'm$, $a,$ $a'\in \mathcal{A}$, for a fixed $m\in M.$ Thanks to the faithfulness of $M$, note that such $a$ and $a'$ in $\mathcal{A}$ always exist. The $\mathcal{A}$-module $M$ endowed with the above-mentioned $\mathrm{End}_\mathcal{A} M$-Novikov algebra structure is now going to be denoted by $\mathcal{M}.$

By Lemma \ref{endn}, $\mathrm{End}_{\mathrm{End}_\mathcal{A} M}\mathcal{M}$ is also a Novikov algebra. We claim that $\mathcal{A}$ is isomorphic to a dense algebra $\mathcal{E}$ of operators of $\mathcal{M}$. We first obtain such a subalgebra $\mathcal{E}$ of $\mathcal{M}$. For each $a\in \mathcal{A}$, we define $f_a\colon \mathcal{M}\rightarrow \mathcal{M}$ by $ f_a(m)=a m$. Since 
\[f_a(m)\circ'f_a(m')=(am)\circ'(am')=a(m\circ' m')=f_a(m\circ' m'),\]
we have $f_a\in \mathrm{End}_{\mathrm{End}_\mathcal{A} M}\mathcal{M}$. Take $\mathcal{E}=\{f_a\mid a\in \mathcal{A}\}$.
To prove the isomorphism, consider the map $\xi\colon \mathcal{A}\to \mathcal{E}$ defined by $\xi(a)= f_a$. For $m\in \mathcal{M},$ \[ f_{a\circ a'}(m)=(a\circ a') m=(a m)\circ '(a' m)=(f_a\circ'' f_{a'})(m),\] that is $f_{a\circ a'}=f_a\circ'' f_{a'}$. Hence $\xi$ is a homomorphism.   Faithfulness of $M$ implies $\xi$ is injective, whereas surjectivity of $\xi$ is obvious.

Finally, the proof is complete if we show $\mathcal{E}$ is dense. Since the argument for the proof of denseness is identical to that of rings, we do not repeat it here.
\end{proof}

\begin{corollary}\label{psdp}
If $\mathcal{A}$ is a nonzero finite dimensional Novikov algebra, then the following statements are equivalent:

\begin{enumerate}[\upshape(i)]

\item $\mathcal{A}$ is prime.

\item $\mathcal{A}$ is simple.

\item There exists a division ring $\Delta$ and a finite dimensional Novikov algebra $\mathcal{M}$ over $\Delta$ such that $\mathcal{A}$ is isomorphic to $\mathrm{End}_{\Delta} \mathcal{M}$.

\item $\mathcal{A}$ is primitive.
\end{enumerate}
\end{corollary}

We conclude this paper with a result (see Theorem \ref{acna}) that brings a finite dimensional Novikov algebra over an algebraically closed Novikov algebra. To obtain that we first need the following lemma. 

\begin{lemma}\label{preb}
Let $\mathcal{F}$ be a unital division Novikov algebra over a field $\mathds{k}$ and $\mathcal{A}$ be a Novikov algebra over $\mathcal{F}$. Let $\mathcal{M}=\mathrm{End}_{\mathcal{F}}\mathcal{A}$, and $\mathcal{B}$ be a $\mathds{k}$-subalgebra of $\mathcal{M}$. If $a\mathcal{B}=\mathcal{B}a=\mathcal{A}$ for every nonzero $a\in \mathcal{A}$ and $\mathrm{End}_{\mathcal{B}}\mathcal{A}=\mathcal{F}$, then $\mathcal{B}$ is a dense algebra of linear operators of $\mathcal{A}$. 
\end{lemma}

\begin{proof}
In the proof of Theorem \ref{cjdt}, we have seen that $\mathcal{A}$ has a natural $\mathcal{B}$-module structure; and, furthermore, that it is a faithful. The assumption $a\mathcal{B}=\mathcal{B}a=\mathcal{A}$ tells us that it is also simple. Thus, $\mathcal{B}$ is a primitive Novikov algebra. Therefore, it follows from Theorem \ref{cjdt} that $\mathcal{B}$ is isomorphic to some dense algebra of linear operators of a Novikov algebra $\mathcal{A}'$ over a unital division Novikov algebra $\mathcal{F}'$. Since by hypothesis, $\mathrm{End}_{\mathcal{B}}\mathcal{A}=\mathcal{F}$, we must have $\mathcal{A}'=\mathcal{A}$ and $\mathcal{F}'=\mathcal{F}$, so that $\mathcal{B}$ itself is dense as required.
\end{proof}

\begin{theorem}\label{acna}
Let $\mathcal{A}$ be a finite dimensional Novikov algebra over an algebraically closed Novikov algebra $\mathcal{F}$. If $\mathcal{B}$ is a subalgebra of $\mathrm{End}_{\mathcal{F}}\mathcal{A}$ such that $\mathcal{A}$ is simple as a module over $\mathcal{B}$, then $\mathcal{B}=\mathrm{End}_{\mathcal{F}}\mathcal{A}$.
\end{theorem}

\begin{proof}
First, we show that a finite dimensional division Novikov algebra $\mathcal{D}$ over an algebraically closed Novikov algebra $\mathcal{F}$ implies $\mathcal{D}=\mathcal{F}$. It will follow that $\mathrm{End}_{\mathcal{B}}\mathcal{A}=\mathcal{F}$ (since the left-hand side is a finite dimensional division algebra over $\mathcal{F}$), so that $\mathcal{B}$ is dense (setting $\mathds{k}=\mathcal{F}$ in Lemma \ref{preb}).

Let $x\in \mathcal{D}$. The set of all powers of any given element is linearly dependent. Thus, there exists a polynomial $f\in \mathcal{F}[x]$ such that $f(x)=0$, assuming without loss of generality that its leading coefficient is $1$. Since $\mathcal{F}$ is algebraically closed, $f$ factors linearly. Thus, $(x-\alpha_1)\cdots(x-\alpha_r)=0$. Since $\mathcal{D}$ is a division Novikov algebra, $x-\alpha_i=0$ for some $\alpha_i$ and that implies $x=\alpha_i\in \mathcal{F}$, and we are done. 

Next, we show that the only dense algebra of linear operators on a finite dimensional Novikov algebra will be the entire homomorphism algebra, and this will complete the proof.
Let $\mathcal{A}$ be a finite dimensional Novikov algebra over $\mathcal{F}$ with basis $\{u_i\}_{i=1}^n$, and pick $g\in\mathrm{End}_{\mathcal{F}}\mathcal{A}$, of which $\mathcal{E}$ is a dense algebra. Then, there exists $f\in \mathcal{E}$ such that $f(u_i)=g(u_i)$, which implies $f=g$, and hence $f\in \mathrm{End}_{\mathcal{F}}\mathcal{A}$.
\end{proof}

\begin{thebibliography}{99}  
 
\bibitem{BN85}
A. A. Balinskii and S. P. Novikov, Poisson brackets of hydrodynamic type, Frobenius
algebras and Lie algebras, \textit{Sov. Math. Dokl.}, \textbf{32} (1985), 228--231.

\bibitem{B14}
M. Bre\v {s}ar, \textit{Introduction to noncommutative algebra}, Springer, 2014.

\bibitem{BG12}
D. Burde and W. de Graaf, Classification of Novikov algebras, \textit{Appl. Algebra Engrg. Comm. Comput.}, \textbf{24} (2013), 1--15.

\bibitem{C88}
V. P. Cherkashin, Left-symmetric algebras with commuting right multiplications, \textit{Moscow Univ. Math. Bull.}, \textbf{43}(5) (1988), 49--52.

\bibitem{D96} J. Dixmier, \textit{Enveloping algebras}, Amer. Math. Soc., 1996.

\bibitem{F89}
V. T. Filipov, A class of simple nonassociative algebras, \textit{Mat. Zametki}, \textbf{45} (1989),
101--105.

\bibitem{G23}
A. Goswami, Jacobson's structure spaces of Novikov algebras (submitted).

\bibitem{GD79}
I. M. Gel'fand and I. Ya. Dorfman, Hamiltonian operators and algebraic structures
related to them, \textit{Funkts. Anal. Prilozhen}, \textbf{13}  (1979), 13--30.

\bibitem{I79} R. S. Irving,  Prime Ideals of Ore extensions over commutative rings, \textit{J. Algebra}, \textbf{56} (1979), 315--342.

\bibitem{J45} N. Jacobson, 
\textit{A topology for the set of primitive ideals in an arbitrary Novikov algebra}, Proc.
Nat. Acad. Sei. U.S.A., \textbf{31} (1945), 333--338.

\bibitem{J56} ---------, \textit{Structure of Novikov algebras}, Amer. Math. Soc. Colloquium Publications,
vol. 37, Providence, (1956)

\bibitem{J75} ---------,  \textit{PI-algebras. An introduction}, Springer-Verlag, 1975.

\bibitem{J83} \textsc{A. Joseph}, \textit{Primitive ideals in enveloping algebras}, Proc. ICM  (Warsaw, 1983), 403--414,
Warsaw, 1984.

\bibitem{J95} ---------,  \textit{Quantum groups and their primitive ideals}, Springer, 1995.

\bibitem{KPP12} A. A. Kucherov, O. A. Pikhtilkova, and  S. A. Pikhtilkov, On primitive Lie algebras, \textit{J. Math. Sci.}, \textbf{186}(4) (2012), 651--654. 

\bibitem{O91}
J. M. Osborn,  Modules for Novikov Algebras, in Second International Conference on Algebra (Barnaul, 1991), 327--338, \textit{Contemp. Math.}, \textbf{184}, Amer. Math. Soc., Providence, RI, 1995.

\bibitem{O92}
---------, Novikov algebras, \textit{Nova J. Algebra Geom.}, \textbf{1}(1) (1992), 1--13.

\bibitem{O94}
---------, Infinite-dimensional Novikov algebras of characteristic $0$, \textit{J. Algebra}, \textbf{167}(1) (1994), 146--167. 

\bibitem{O95}
---------, Modules for Novikov algebras of characteristic 0, \textit{Comm. Algebra},
\textbf{23} (1995), 3627--3640.

\bibitem{P22}
A. S. Panasenko,
Semiprime Novikov algebras.
\textit{Internat. J. Algebra Comput.}, \textbf{32}(7) (2022), 1369--1378.

\bibitem{R88} L. H. Rowen, \textit{Ring theory}, vol. I, Academic Press, Inc., 1988.

\bibitem{SZ20}
I. Shestakov and  Z. Zhang, Solvability and nilpotency of Novikov algebras, \textit{Comm.
Algebra}, \textbf{48}(12)  (2020), 5412--5420.

\bibitem{X96}
X. Xu,  On simple Novikov algebras and their irreducible modules, \textit{J. Algebra}, \textbf{185} (1996),
905--934.

\bibitem{X01}
X. Xu, Classification of simple Novikov algebras and their
irreducible modules of characteristic $0$, \textit{J. Algebra}, \textbf{246} (2001), 673--707.

\bibitem{Z87}
E. I. Zel'manov, On a class of local translation invariant Lie algebras, \textit{Sov. Math. Dokl.},
\textbf{35}(6) (1987), 216--218.

\end{thebibliography}
\end{document}