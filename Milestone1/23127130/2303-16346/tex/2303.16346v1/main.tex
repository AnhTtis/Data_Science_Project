 \pdfoutput=1
%% This is file `sample-manuscript.tex',
%% generated with the docstrip utility.
%%
%% The original source files were:
%%
%% samples.dtx  (with options: `manuscript')
%% 
%% IMPORTANT NOTICE:
%% 
%% For the copyright see the source file.
%% 
%% Any modified versions of this file must be renamed
%% with new filenames distinct from sample-manuscript.tex.
%% 
%% For distribution of the original source see the terms
%% for copying and modification in the file samples.dtx.
%% 
%% This generated file may be distributed as long as the
%% original source files, as listed above, are part of the
%% same distribution. (The sources need not necessarily be
%% in the same archive or directory.)
%%
%% Commands for TeXCount
%TC:macro \cite [option:text,text]
%TC:macro \citep [option:text,text]
%TC:macro \citet [option:text,text]
%TC:envir table 0 1
%TC:envir table* 0 1
%TC:envir tabular [ignore] word
%TC:envir displaymath 0 word
%TC:envir math 0 word
%TC:envir comment 0 0
%%
%%
%% The first command in your LaTeX source must be the \documentclass command.
%%%% Small single column format, used for CIE, CSUR, DTRAP, JACM, JDIQ, JEA, JERIC, JETC, PACMCGIT, TAAS, TACCESS, TACO, TALG, TALLIP (formerly TALIP), TCPS, TDSCI, TEAC, TECS, TELO, THRI, TIIS, TIOT, TISSEC, TIST, TKDD, TMIS, TOCE, TOCHI, TOCL, TOCS, TOCT, TODAES, TODS, TOIS, TOIT, TOMACS, TOMM (formerly TOMCCAP), TOMPECS, TOMS, TOPC, TOPLAS, TOPS, TOS, TOSEM, TOSN, TQC, TRETS, TSAS, TSC, TSLP, TWEB.
% \documentclass[acmsmall]{acmart}

%%%% Large single column format, used for IMWUT, JOCCH, PACMPL, POMACS, TAP, PACMHCI
% \documentclass[acmlarge,screen]{acmart}

%%%% Large double column format, used for TOG
% \documentclass[acmtog, authorversion]{acmart}

%%%% Generic manuscript mode, required for submission
%%%% and peer review
% \documentclass[manuscript,screen,review,anonymous]{acmart}
\documentclass[sigconf]{acmart}
\usepackage{todonotes}
\usepackage{enumitem,varwidth, hyperref}


% \newcommand{\change}[1]{{\color{blue}#1}}
\newcommand{\change}[1]{{#1}}
\newcommand{\yuhang}[1]{{\small\textcolor{red}{\bf [*** Yuhang: #1]}}}

\usepackage{array}
% \newcolumntype{L}[1]{>{\raggedright\let\newline\\\arraybackslash\hspace{0pt}}m{#1}}
\newcolumntype{C}[1]{>{\centering\let\newline\\\arraybackslash\hspace{0pt}}m{#1}}
% \newcolumntype{R}[1]{>{\raggedleft\let\newline\\\arraybackslash\hspace{0pt}}m{#1}}

%% Fonts used in the template cannot be substituted; margin 
%% adjustments are not allowed.
%%
%% \BibTeX command to typeset BibTeX logo in the docs
\AtBeginDocument{%
  \providecommand\BibTeX{{%
    \normalfont B\kern-0.5em{\scshape i\kern-0.25em b}\kern-0.8em\TeX}}}

%% Rights management information.  This information is sent to you
%% when you complete the rights form.  These commands have SAMPLE
%% values in them; it is your responsibility as an author to replace
%% the commands and values with those provided to you when you
%% complete the rights form.
\copyrightyear{2023} 
\acmYear{2023} 
\setcopyright{acmlicensed}
\acmConference[CHI '23]{Proceedings of the 2023 CHI Conference on Human Factors in Computing Systems}{April 23--28, 2023}{Hamburg, Germany}
\acmBooktitle{Proceedings of the 2023 CHI Conference on Human Factors in Computing Systems (CHI '23), April 23--28, 2023, Hamburg, Germany}
\acmPrice{15.00}
\acmDOI{10.1145/3544548.3581213}
\acmISBN{978-1-4503-9421-5/23/04}


%%
%% Submission ID.
%% Use this when submitting an article to a sponsored event. You'll
%% receive a unique submission ID from the organizers
%% of the event, and this ID should be used as the parameter to this command.
%%\acmSubmissionID{123-A56-BU3}

%%
%% For managing citations, it is recommended to use bibliography
%% files in BibTeX format.
%%
%% You can then either use BibTeX with the ACM-Reference-Format style,
%% or BibLaTeX with the acmnumeric or acmauthoryear sytles, that include
%% support for advanced citation of software artefact from the
%% biblatex-software package, also separately available on CTAN.
%%
%% Look at the sample-*-biblatex.tex files for templates showcasing
%% the biblatex styles.
%%

%%
%% The majority of ACM publications use numbered citations and
%% references.  The command \citestyle{authoryear} switches to the
%% "author year" style.
%%
%% If you are preparing content for an event
%% sponsored by ACM SIGGRAPH, you must use the "author year" style of
%% citations and references.
%% Uncommenting
%% the next command will enable that style.
%%\citestyle{acmauthoryear}

%%
%% end of the preamble, start of the body of the document source.
\begin{document}

%%
%% The "title" command has an optional parameter,
%% allowing the author to define a "short title" to be used in page headers.
\title{Understanding How Low Vision People Read Using Eye Tracking}

%%
%% The "author" command and its associated commands are used to define
%% the authors and their affiliations.
%% Of note is the shared affiliation of the first two authors, and the
%% "authornote" and "authornotemark" commands
%% used to denote shared contribution to the research.

% \author{Ru Wang}
% \affiliation{%
%   \institution{University of Wisconsin-Madison}
%   \streetaddress{1210 W Dayton St., Madison}
%   \city{Madison}
%   \country{USA}}
% \email{ru.wang@wisc.edu}

\author{Ru Wang}
\affiliation{%
   \institution{University of Wisconsin-Madison}
   % \streetaddress{1210 W Dayton St., Madison}
   \city{Madison}
   \state{WI}
   \country{USA}}
\email{ru.wang@wisc.edu}

\author{Linxiu Zeng}
\affiliation{%
   \institution{University of Wisconsin-Madison}
   % \streetaddress{1210 W Dayton St., Madison}
   \city{Madison}
   \state{WI}
   \country{USA}}
\email{lzeng37@wisc.edu}

\author{Xinyong Zhang}
\affiliation{%
   \institution{Renmin University of China}
   \city{Beijing}
   % \state{WI}
   \country{China}}
\email{x.y.zhang@ruc.edu.cn}

\author{Sanbrita Mondal}
\affiliation{%
   \institution{University of Wisconsin-Madison}
   % \streetaddress{}
   \city{Madison}
   \state{WI}
   \country{USA}}
\email{smondal4@wisc.edu}

\author{Yuhang Zhao}
\affiliation{%
   \institution{University of Wisconsin-Madison}
   % \streetaddress{}
   \city{Madison}
   \state{WI}
   \country{USA}}
\email{yuhang.zhao@cs.wisc.edu}

%%
%% By default, the full list of authors will be used in the page
%% headers. Often, this list is too long, and will overlap
%% other information printed in the page headers. This command allows
%% the author to define a more concise list
%% of authors' names for this purpose.
\renewcommand{\shortauthors}{}



%%
%% The abstract is a short summary of the work to be presented in the
%% article.
\begin{abstract}
%Low vision people face challenges in reading tasks. While large font and screen magnifiers can enable them to read, low vision people suffer from slow and unpleasant reading experiences, for example, they can easily lose track of the next line with high magnification. Eye tracking technique has the potential to advance low vision technology by recognizing people's fine-grained gaze behaviors and providing more targeted enhancements. 
%To inspire gaze-based low vision technology, we aim to explore the feasibility of commercial eye trackers for low vision users, and investigate their gaze patterns during reading via eye tracking. A more accessible gaze calibration interface was designed to make the eye tracker more usable to low vision people. We collected the gaze data of 12 low vision participants and 12 sighted controls when they performed reading tasks on a computer screen. We also asked low vision participants to read with different types of screen magnifiers (e.g., lens or full-screen magnifier) to investigate the effect of different magnification modes on users' gaze behaviors.  We found that, with accessible calibration interfaces and suitable data collection, commercial eye trackers can be promising tools for low vision research. Our study also identified unique gaze patterns of low vision people during reading, based upon which, we propose guidelines for gaze-based assistive technology design.

While being able to read with screen magnifiers, low vision people have slow and unpleasant reading experiences.
Eye tracking has the potential to improve their experience by recognizing fine-grained gaze behaviors and providing more targeted enhancements. To inspire gaze-based low vision technology, we \change{investigate the suitable method to collect low vision users' gaze data via commercial eye trackers and thoroughly explore their challenges in reading based on their gaze behaviors}. With an improved calibration interface, we collected the gaze data of \change{20 low vision participants and 20 sighted controls} who performed reading tasks on a computer screen; low vision participants were also asked to read with different screen magnifiers. We found that, with an accessible calibration interface and data collection method, commercial eye trackers can \change{collect gaze data of comparable quality from low vision and sighted people.} Our study identified low vision people’s unique gaze patterns during reading, building upon which, we propose design implications for gaze-based low vision technology.

\end{abstract}

%%
%% The code below is generated by the tool at http://dl.acm.org/ccs.cfm.
%% Please copy and paste the code instead of the example below.
%%
\begin{CCSXML}
<ccs2012>
   <concept>
       <concept_id>10003120.10011738.10011773</concept_id>
       <concept_desc>Human-centered computing~Empirical studies in accessibility</concept_desc>
       <concept_significance>500</concept_significance>
       </concept>
   <concept>
       <concept_id>10003120.10011738.10011776</concept_id>
       <concept_desc>Human-centered computing~Accessibility systems and tools</concept_desc>
       <concept_significance>500</concept_significance>
       </concept>
 </ccs2012>
\end{CCSXML}

\ccsdesc[500]{Human-centered computing~Empirical studies in accessibility}
\ccsdesc[500]{Human-centered computing~Accessibility systems and tools}

%%
%% Keywords. The author(s) should pick words that accurately describe
%% the work being presented. Separate the keywords with commas.
\keywords{Accessibility, low vision, eye tracking, gaze pattern, reading}

%% A "teaser" image appears between the author and affiliation
%% information and the body of the document, and typically spans the
%% page.
% \begin{teaserfigure}
%   \includegraphics[width=\textwidth]{sampleteaser}
%   \caption{Seattle Mariners at Spring Training, 2010.}
%   \Description{Enjoying the baseball game from the third-base
%   seats. Ichiro Suzuki preparing to bat.}
%   \label{fig:teaser}
% \end{teaserfigure}

%%
%% This command processes the author and affiliation and title
%% information and builds the first part of the formatted document.
\maketitle

\section{Introduction}


Recent years have witnessed the rise of human digitization~\cite{habermannDeepCapMonocularHuman2020,alexanderCREATINGPHOTOREALDIGITAL,pengNeuralBodyImplicit2021,alldieckDetailedHumanAvatars2018, rajANRArticulatedNeural2020}. This technology greatly impacts the entertainment, education, design, and engineering industry.
There is a well-developed industry solution for this task.
High-fidelity reconstruction of humans can be achieved either with full-body laser scans~\cite{saitoSCANimateWeaklySupervised2021}, dense synchronized multi-view cameras~\cite{xiangModelingClothingSeparate2021a,xiangDressingAvatarsDeep2022a}, or light stages~\cite{alexanderCREATINGPHOTOREALDIGITAL}.
However, these settings are expensive and tedious to deploy and consist of a complex processing pipeline, preventing the technology's democratization.

Another solution is to view the problem as inverse rendering and learn digital humans directly from custom-collected data.
Traditional approaches directly optimize explicit mesh representation~\cite{loperSMPLSkinnedMultiperson2015, fangRMPERegionalMultiperson2018, pavlakosExpressiveBodyCapture2019} which suffers from the problems of smooth geometry and coarse textures~\cite{prokudinSMPLpixNeuralAvatars2020,alldieckVideoBasedReconstruction2018}. Besides, they require professional artists to design human templates, rigging, and unwrapped UV coordinates.
Recently, with the help of volumetric-based implicit representations~\cite{mildenhallNeRFRepresentingScenes2020, parkDeepSDFLearningContinuous2019, meschederOccupancyNetworksLearning2019} and neural rendering~\cite{laineModularPrimitivesHighPerformance2020, liuSoftRasterizerDifferentiable2019, thiesDeferredNeuralRendering2019}, 
one can easily digitize a quality-plausible human avatar from video footage~\cite{jiangNeuManNeuralHuman2022,wengHumanNeRFFreeviewpointRendering}.
Particularly, volumetric-based implicit representations~\cite{mildenhallNeRFRepresentingScenes2020, pengNeuralBodyImplicit2021} can reconstruct scenes or objects with much higher fidelity against previous neural renderer~\cite{thiesDeferredNeuralRendering2019,prokudinSMPLpixNeuralAvatars2020}, and is more user-friendly as it does not need any human templates, pre-set rigging, or UV coordinates.
Captured visual footage and corresponding skeleton tracking are enough for training.
However, better reconstructions and more friendly usability are at the expense of the following factors.
1) \textbf{Inefficiency:}
They require longer optimization times (typically tens of hours or days) and inference slowly.
Volume rendering~\cite{mildenhallNeRFRepresentingScenes2020,lombardiNeuralVolumesLearning2019} formulates images by querying the densities and colors of millions of spatial coordinates. 
In the training stage, due to memory constraints, only a small fraction of points are sampled which leads to slow convergence speed.
2) \textbf{Entangled representations}:
The geometry, materials, and motion dynamics are entangled in the neural networks. 
Due to the implicit nature of neural nets, one can hardly edit one property without touching the others~\cite{yuanNeRFEditingGeometryEditing2022a,liuEditingConditionalRadiance2021}.
3) \textbf{Graphics incompatibility}:
Volume rendering is incompatible with the current popular graphic pipeline,
which renders triangular/quadrilateral meshes efficiently with the rasterization technique.
Many downstream applications require mesh rasterization in their workflow (\eg, editing~\cite{foundationBlenderOrgHome}, simulation~\cite{benderPositionBasedSimulationMethods2015}, real-time rendering~\cite{akenine2019real}, ray-tracing~\cite{waldRTXRayTracing}).
Although there are approaches~\cite{lorensenMarchingCubesHigh,labelleIsosurfaceStuffingFast2007} can convert volumetric fields into meshes, the gaps from discrete sampling degrade the output quality in terms of both meshes and textures.


To address these issues, we present \textbf{EMA}, a method based on \textbf{E}fficient \textbf{M}eshy neural fields to reconstruct animatable human \textbf{A}vatars.
Our method enjoys flexibility from implicit representations and efficiency from explicit meshes, yet still maintains high-fidelity reconstruction quality.
Given video sequences and the corresponding pose tracking, our method digitizes humans in terms of canonical triangular meshes, physically-based rendering (PBR) materials, and skinning weights \textit{w.r.t.} skeletons.
We jointly learn the above components via inverse rendering~\cite{laineModularPrimitivesHighPerformance2020,chenDIBRLearningPredict2021,chenLearningPredict3D2019} in an end-to-end manner.
Each of them is derived from a separate neural field, which relaxes the requirements of a preset human template, rigging, or UV coordinates.
Specifically, we predict a canonical mesh out of a signed distance field (SDF) by differentiable marching tetrahedra~\cite{shenDeepMarchingTetrahedra2021,gaoGET3DGenerativeModel,gaoLearningDeformableTetrahedral2020,munkbergExtractingTriangular3D2022}, then we extend the marching tetrahedra~\cite{shenDeepMarchingTetrahedra2021} for spatial-varying materials by utilizing a neural field to predict PBR materials \textit{on the mesh surfaces} after rasterization~\cite{munkbergExtractingTriangular3D2022,hasselgrenShapeLightMaterial2022,laineModularPrimitivesHighPerformance2020}.
To make the canonical mesh animatable, we take another neural field to model the forward linear blend skinning for the meshes. 
Given a posed skeleton, the canonical mesh is then transformed into the corresponding poses.
Finally, we shade the mesh with a rasterization-based differentiable renderer~\cite{laineModularPrimitivesHighPerformance2020} and train our models with a photo-metric loss.
After training, we export the mesh with materials and discard the neural fields.

\looseness=-1
There are several merits of our method design.
1) \textbf{Efficiency}:
Powered by efficient mesh rendering, our method can render in real-time.
Besides, the training speed is boosted as well, 
since we compute loss holistically on the whole image and the gradients only flow on the mesh surface. In contrast, volume rendering takes limited pixels for loss computation and back-propagates the gradients in the whole space.
Our method only needs about an hour of training and minutes of optimization are enough for plausible avatar reconstruction.
2) \textbf{Disentangled representations}:
Our shape, materials, and motion modules are disentangled naturally by design, which facilitates editing. 
Besides, Canonical meshes with forward skinning modeling handle the out-of-distribution poses better.
3) \textbf{Graphics compatibility}:
Our derived mesh representation is compatible with 
the prominent graphic pipeline, which leads to instant downstream applications (\eg, the shape and materials can be edited directly in design software~\cite{foundationBlenderOrgHome}).
To further improve reconstruction quality, we additionally optimize image-based environment lights and non-rigid motions.


We conduct extensive experiments on standards benchmarks H36M~\cite{ionescuHuman36MLarge2014b} and ZJU-MoCap~\cite{pengNeuralBodyImplicit2021}.
Our method achieves very competitive performance for novel view synthesis, generalizes better for novel poses, 
and significantly improves both training time and inference speed against previous arts.
Our research-oriented code reaches real-time inference speed (100+ FPS for rendering $512\times512$ images).
We in addition showcase applications including novel pose synthesis, material editing, and relighting.
\section{Related Work} \label{sec:related work}
\vspace{-0.2cm}
{\noindent \bf Vision-Language Pre-training.} In the early literature, \cite{Mori99,Frome13,Weston11} explore jointly training image-text embeddings using paired text documents. Recently, some studies have further scaled up the training with large-scale web data to form ``the \textbf{foundation} models'', {\em e.g.}, CLIP~\cite{Radford21}, ALIGN~\cite{Jia21}, Florence~\cite{yuan2021florence}, FILIP~\cite{yao2021filip}, VideoCLIP~\cite{xu2021videoclip}, and LiT~\cite{zhai2022lit}. These foundation models usually contain one visual encoder and one textual encoder, which are trained using simple noise contrastive learning for powerful cross-modal representations. They have shown promising potential in many tasks, such as image classification and detection, action recognition, and retrieval. In this paper, we use CLIP for low-shot temporal action localization, but the same technique should be applicable to other foundation models as well.



\vspace{0.1cm}
{\noindent \bf Prompting} refers to leveraging input instructions to steer foundation models for desired outputs. In the NLP domain, early papers~\cite{Gao21,Jiang20,Timo21,Shin20} focus on handcrafted prompt templates. To avoid labor and increase flexibility, some studies~\cite{Lester21,li21-prefixtuning,li2021prefix} propose learnable prompt tuning at the textual stream, showing strong low-shot generalization. In the CV domain, some recent papers~\cite{zhou2019learn,zhou2022conditional,ju2022prompting} introduce such randomly initialized prompt tuning to handle visual tasks, {\em e.g.}, image understanding~\cite{zhu2022prompt,lu2022prompt,yang2022learning,ma2023diffusionseg} and video understanding~\cite{jia2022visual,nag2022zero,ni2022expanding}. However, these studies ignore lexical ambiguity of category names, and cases that are not easy to describe in text. This paper designs novel conditional prompt tuning and language descriptions from LLMs, to solve these issues. 



\vspace{0.1cm}
{\noindent \bf Closed-set Temporal Action Localization} considers to detect and classify action instances from one pre-defined category list. Specifically, existing methods can be divided into two popular supervisions, {\em i.e.}, strong~\cite{zeng2019graph,lin2021learning,qing2021temporal} and weak~\cite{wang2017untrimmednets,ju2023constraint,ju2020point,yudistira2022weakly}. Strong supervision gives precise boundary labels and category labels for training. There are two detailed pipelines: the top-down framework~\cite{shou2016temporal,shou2017cdc,gao2017turn,chao2018rethinking,lin2017single,xu2017r,tan2021relaxed,zhu2021enriching,wang2022rcl,xu2020g} pre-defines extensive anchors, adopts fixed-length sliding windows to produce initial proposals, then regresses to refine boundaries; the bottom-up framework~\cite{zhao2017temporal,lin2018bsn,lin2019bmn,vo2023aoe,zhao2020bottom,bai2020boundary} learns frame-wise boundary detectors for the boundary frames, then groups extreme frames or estimates action lengths for proposal generation. In addition, several works~\cite{gao2018ctap,liu2019multi,yang2020revisiting} used various fusion strategies to complement these frameworks. On the other hand, weak supervision trains without boundary labels to alleviate annotation costs. The video-level setting learns from category labels~\cite{paul2018w,ju2022distilling}, the CAS-based framework~\cite{liu2019completeness,ju2021adaptive,min2020adversarial,narayan2021d2,lee2019background,lee2021weakly,zhao2021soda} and attention-based framework~\cite{nguyen2018weakly,luo2021action,nguyen2019weakly,shi2020weakly,gao2022fine,he2022asm,huang2021foreground,luo2020weakly,ma2022weakly} have been well studied. To generate better results from CAS or attention, some studies~\cite{shou2018autoloc,liu2019weakly} improved post-processing. To balance cost and performance, some papers introduced single-frame annotations~\cite{ju2021divide,ma2020sf,lee2021learning,yang2021background,mettes2019pointly} or instance-number annotations~\cite{narayan20193c,xu2019segregated}. 

Nevertheless, all the above methods assume that action categories remain identical for training and testing, which is an over-simplification of real application scenarios, limiting practical uses of the vision system.



\vspace{0.1cm} 
{\noindent \bf Low-Shot Temporal Action Localization} considers more realistic scenarios: generalize TAL towards action categories that are unseen (zero-shot) or with several support samples (few-shot). Existing methods~\cite{ju2022prompting,nag2022zero,zhang2022ow,bao2022opental} most rely on foundational models pre-trained on large-scale image-caption pairs for help. Typically, E-Prompt~\cite{ju2022prompting} is the first to construct wide baselines with popular prompt tuning~\cite{Lester21,li21-prefixtuning} and vanilla temporal modeling. STALE~\cite{nag2022zero} explores the one-stage framework to further simplify usage. Although promising, all above methods meet two main challenges: (1) For category semantics, the definition may be vague, inaccurate, or incomplete. (2) For visual motions, temporal modeling may be insufficient. In this paper, for detailed category understanding, we design novel language descriptions from LLMs and vision-conditional prompt tuning; for clearer motion understanding, we introduce optical flows to provide explicit motion inputs. 





\vspace{-0.3em}
\section{Method}
\vspace{-0.3em}

Our sensitivity-aware visual parameter-efficient fine-tuning consists of two stages. In the first stage, SPT measures the task-specific sensitivity for the pre-trained parameters (Section~\ref{subsec:sensitivity}). Based on the parameter sensitivity and a given parameter budget, SPT then adaptively allocates trainable parameters to task-specific important positions (Section~\ref{subsec:SPT}).

\vspace{-0.3em}
\subsection{Task-specific Parameter Sensitivity}
\label{subsec:sensitivity}
\vspace{-0.3em}

Recent research has observed that pre-trained backbone parameters exhibit varying feature patterns~\cite{raghu2021vision,naseer2021intriguing} and criticality~\cite{zhang2019all,chatterji2019intriguing} at distinct positions. 
Moreover, when transferred to downstream tasks, their efficacy varies depending on how much pre-trained features are reused and how well they adapt to the specific domain gap~\cite{yosinski2014transferable,kumar2022finetuning,neyshabur2020being}. Motivated by these observations, we argue that not all parameters contribute equally to the performance across different tasks in PEFT and propose a new criterion to measure the sensitivity of the parameters in the pre-trained backbone for a given task.

Specifically, given the training dataset $\gD_t$ for the $t$-th task and the pre-trained model weights $\vw=\left\{w_1, w_2, \ldots, w_N\right\}\in \sR^N$ where $N$ is the total number of parameters, the objective for the task is to minimize the empirical risk: $\min_{\vw} E(\gD_t, \vw)$.
We denote the parameter sensitivity \bohan{set} as $\gS=\{s_1, \ldots, s_N\}$ and the sensitivity $s_n$ for parameter $w_n$ is measured by the empirical risk difference when tuning it:
\begin{equation}
\vspace{-0.3em}
    \begin{aligned}
        s_n = E(\gD_t, \vw)-E(\gD_t, \vw\mid w_n=w_n^*),
    \end{aligned}
\label{eq:sensitivity}
\end{equation}
where $w_n^*=\underset{w_n}{\rm argmin}(E(\gD_t, \vw))$. We can reparameterize the tuned parameters as  $w_n^*=w_n+\Delta_{w_n}$, where $\Delta_{w_n}$ denotes the update for $w_n$ after tuning. Here we individually measure the sensitivity of each parameter, which is reasonable given that most of the parameters are frozen during fine-tuning in PEFT. However, it is still computationally intensive to compute Eq.~(\ref{eq:sensitivity}) for two reasons. Firstly, getting the empirical risk for $N$ parameters requires forwarding the entire network $N$ times, which is time-consuming. Secondly, it is challenging to derive $\Delta_{w_n}$, as we have to tune each individual $w_n$ until convergence.

{\begin{algorithm}[t!]
\caption{\label{alg:tps} Computing task-specific parameter sensitivities}
\begin{algorithmic}
    \STATE \textbf{Input:} Pre-trained model with network parameters $\vw$, training set $\gD_t$ for the $t$-th task, and number of training samples $C$ used to calculate the parameter sensitivities
    \STATE \textbf{Output:} Sensitivity set $\gS=\{s_1, \ldots, s_N\}$
    \STATE Initialize $\gS=\{0\}^N$
    \FOR{$i\in\{1,\ldots,C\}$}
        \STATE Get the $i$-th training sample of $\gD_t$
	    \STATE Compute loss $E$
		\STATE Compute gradients $\vg$
		\FOR{$n\in\{1,\ldots,N\}$}
                \STATE Update sensitivity for the $n$-th parameter: $s_{n} = s_{n} + g_n^2$
		    \ENDFOR
    \ENDFOR
\end{algorithmic}
\end{algorithm}}


\begin{figure*}[t]
\begin{center}
    \includegraphics[width=\linewidth]{main_figure.pdf}
\end{center}\vspace{-2em}
\caption{Overview of our trainable parameter allocation strategy. With the parameter sensitivity \bohan{set} $\gS$, we first get the top-$\tau$ sensitive parameters. Instead of directly tuning these sensitive parameters, we also boost the representational capability by replacing unstructured tuning with structured tuning at sensitive weight matrices that have a large number of sensitive parameters, which can be implemented by an existing structured tuning method, \eg, LoRA~\cite{hu2022lora} and Adapter~\cite{houlsby2019parameter}. Red lines and blocks represent trainable parameters and modules, while blue lines represent frozen parameters.}
\label{fig:main}
\vspace{-1.5em}
\end{figure*}


To overcome the first barrier, we simplify the empirical loss by approximating $s_n$ in the vicinity of $\vw$ by its first-order Taylor expansion
\vspace{-0.3em}
\begin{equation}
\vspace{-0.5em}
    \begin{aligned}
        s_n^{(1)} = -g_n\Delta_{w_n},
    \end{aligned}
\label{eq:first-order}
\end{equation}
where the gradients $\vg=\partial E/\partial\vw$, and $g_n$ is the gradient of the $n$-th element of $\vg$. 
To address the second barrier, following~\cite{liu2018darts,cai2018proxylessnas}, we take the one-step unrolled weight as the surrogate for $w_n^*$ and approximate $\Delta_{w_n}$ in Eq.~(\ref{eq:first-order}) with a single step of gradient descent. We can accordingly get $s_n^{(1)} \approx g_n^2\epsilon$,
where $\epsilon$ is the learning rate. Since $\epsilon$ is the same for all parameters, we can eliminate it when comparing the sensitivity with the other parameters and finally get 
\vspace{-0.5em}
\begin{equation}
\vspace{-0.3em}
    \begin{aligned}
        s_n^{(1)} \approx g_n^2.
    \end{aligned}
\label{eq:first-order-simp}
\end{equation}
Therefore, the sensitivity of a parameter can be efficiently measured by its potential to reduce the loss on the target domain. Note that although our criterion draws inspiration from pruning work~\cite{molchanov2019importance}, it is distinct from it. \cite{molchanov2019importance} measures the parameter importance by the squared change in loss when removing them, \ie, $\left( E(\gD_t, \vw)-E(\gD_t, \vw\mid w_n=0) \right)^2$ and finally derives the parameter importance by $\left( g_n w_n \right)^2$, which is different from our formulations in Eqs.~(\ref{eq:sensitivity}) and~(\ref{eq:first-order-simp}).

In practice, we accumulate $\gS$ from a total number of $C$ training samples ahead of fine-tuning to generate accurate sensitivity as shown in Algorithm~\ref{alg:tps}, where $C$ is a pre-defined hyper-parameter. In Section~\ref{subsec:abl}, we show that employing only 400 training samples is sufficient for getting reasonable parameter sensitivity, which requires only 5.5 seconds with a single GPU for any VTAB-1k dataset with ViT-B/16 backbone~\cite{vit}.

\vspace{-0.3em}
\subsection{Adaptive Trainable Parameters Allocation}
\label{subsec:SPT}
\vspace{-0.2em}

Our next step is to allocate trainable parameters based on the obtained parameter sensitivity set $\gS$ and a desired parameter budget $\tau$. A straightforward solution is to directly tune the top-$\tau$ most sensitive unstructured connections (parameters) \rev{while keeping the rest frozen}, which we name unstructured tuning. Specifically, we select the top-$\tau$ most sensitive weight connections in $\gS$ to form the sensitive weight connection set $\gT$. Then, for \rev{a} weight matrix $\mW\in \sR^{d_{\rm in}\times d_{\rm out}}$, we can get a binary mask $\mM\in \sR^{d_{\rm in}\times d_{\rm out}}$ computed by
\vspace{-0.5em}
\begin{equation}
\vspace{-0.5em}
    {\begin{array}{ll}
    \small
    \begin{aligned}
    \mM^j =
    \left\{\begin{array}{ll} 
    1 ~~~~~ \mW^j \in \gT \\
    0 ~~~~~ \mW^j \notin \gT
    \end{array}\right.
    \end{aligned},
    \small
    \end{array}}
\label{eq:mask}
\end{equation}
where $\mW^j$ and $\mM^j$ are the $j$-th element in $\mW$ and $\mM$, respectively. Accordingly, we can train the sensitive parameters by gradient descent and the updated weight matrix can be formulated as $\mW'\leftarrow \mW - \epsilon\vg_{\mW}\odot\mM$, where $\vg_{\mW}$ is the gradient for $\mW$.

However, considering PEFT approaches generally limit the proportion of trainable parameters to less than 1\%, tuning only a small number of unstructured weight connections might not have enough representational capability to handle the downstream datasets with large domain gaps from the source pre-training data. Therefore, to improve the representational capability, we propose to replace unstructured tuning with structured tuning at the sensitive weight matrices that have a high number of sensitive parameters. To preserve the parameter budget, we can implement structured tuning with an existing efficient structured tuning PEFT method~\cite{hu2022lora,chen2022adaptformer,houlsby2019parameter,jie2022convolutional} that learns to directly adjust \rev{all hidden dimensions at once}. We depict an overview of our trainable parameter allocation strategy in Figure~\ref{fig:main}. For example, we can employ the low-rank reparameterization trick LoRA~\cite{hu2022lora} to the sensitive weight matrices \rev{and the one-step update for $\mW$ can be formulated as}
\vspace{-0.4em}
\begin{equation}
\vspace{-0.4em}
    {\begin{array}{ll}
    \small
    \begin{aligned}
    \mW' = \left\{\begin{array}{ll} 
    \mW + \mW_{\rm down}\mW_{\rm up} & ~~ \text { if } ~~ \sum_{j=0}^{d_{\rm in}\times d_{\rm out}} \mM^j \geq \sigma_{\rm opt} \\
    \mW - \epsilon\vg_{\mW}\odot\mM & ~~ {\rm otherwise}
    \end{array}\right.
    \end{aligned},
    \small
    \end{array}}
\label{eq:weight_updat}
\end{equation}
where $\mW_{\rm down}\in \sR^{d_{\rm in}\times r}$ and $\mW_{\rm up}\in \sR^{r\times d_{\rm out}}$ are two learnable low-rank matrices to approximate the update of $\mW$ and rank $r$ is a hyper-parameter where $r \ll {\rm min}(d_{\rm in},d_{\rm out})$. In this way, we perform structured tuning on $\mW$ when its number of sensitive parameters exceeds $\sigma_{\rm opt}$, whose value depends on the pre-defined type of structured tuning method. For example, since implementing structured tuning with LoRA requires $2\times d_{\rm in} \times d_{\rm out} \times r$ trainable parameters for each sensitive weight matrix, we set $\sigma_{\rm LoRA} \leftarrow 2\times d_{\rm in} \times d_{\rm out} \times r$ to ensure that the number of trainable parameters introduced by structured tuning is always equal to or lower than the number of sensitive parameters.

In this way, our SPT adaptively incorporates both structured and unstructured tuning granularities to enable higher flexibility and stronger representational power, simultaneously. In Section~\ref{subsec:abl}, we show that structured tuning is important for the downstream tasks with larger domain gaps and both unstructured and structured tuning contribute clearly to the superior performance of our SPT.
\section{Results}
In this section, we \change{report our results regarding the four groups of hypotheses respectively, including validating gaze calibration and data quality between low vision and sighted participants (H1), comparing gaze behaviors between low vision and sighted participants (H2), and exploring the effect of visual conditions (H3) and magnification modes (H4) on low vision participants' gaze behaviors in reading.} %the feasibility of commercial eye trackers for low vision people by evaluating the calibration performance and data loss. We then report findings regarding low vision participants' gaze behaviors in reading. 

\subsection{Eye Tracker Validation for Low Vision People (H1)}
%We evaluate the calibration efficacy using the mean validation accuracy across the 4 points in validation phase. The accuracy is defined as the average visual angle difference between the target position and the estimated gaze location~\cite{tobiiproacc}. By default, the estimated gaze position is the average of two eyes' position. However, due to some low vision participants' visual condition, we need to use the better eye's data for this validation. 
We validated the gaze calibration result and assessed the data quality from the eye tracker below:

\textbf{\textit{Four-point Validation \change{(H1.1)}.}} 
% First, we want to validate that our calibration interface could achieve comparable performance as Tobii Pro Lab (TPL) when the target size is not adjustable. We compared the 4-point validation accuracy for sighted participants using TPL and our calibration interface. The mean validation accuracy across 12 participants with TPL is 0.73\textdegree ($SD = 0.52$\textdegree), while the mean validation accuracy of our calibration interface is 0.56\textdegree ($SD = 0.19$\textdegree). A Wilcoxon Signed Rank test indicates that the difference between the two calibration results is not significant ($V = 59$, $p = 0.13$), which means our calibration interface is able to achieve comparable calibration efficacy when the target size is fixed.
We adjusted the size of calibration targets for low vision participants. Table \ref{tab:lv_task} presents the target size they selected. To validate the calibration, we compared the gaze recognition accuracy in the four-point validation between low vision and sighted participants.   
% For low vision participants who reported they relied a dominant eye for visual tasks, we used the data of their dominant eye for the data analysis of this study. 
%The mean validation accuracy for low vision participants is \change{1.23}\textdegree~ (\change{$SD=1.27$}\textdegree). 
An \change{ART} analysis showed no significant difference between the gaze recognition accuracy between sighted and low vision participants \change{($F_{(1, 38)} = 0.61$, $p = 0.44$, $\eta^2_{p} = 0.02$)}.
%($t_{14.25} = 1.53, p = 0.15$). 
\change{This demonstrated that the improved calibration interface for low vision participants was as effective as the typical calibration for sighted people.}  

%We further compared the validation result between low vision participants with high visual acuity and low visual acuity (see Table \ref{tab:lv_task}). The difference in validation accuracy between the two groups is not significant using \change{ART} 
%($t_{7.86} = 0.06, p = 0.96$) 
%\change{($F_{(1,18)} = 2.09$, $p = 0.17$, $\eta^2_{p} = 0.10$ )}, which indicates that increased calibration target size does not compromise the calibration performance. 

\begin{table*}[h]
\small
\centering
% \begin{tabular}{p{1cm}p{1.3cm}p{1cm}p{1.2cm}p{1.5cm}p{1.5cm}|p{1.3cm}p{1.3cm}p{1.7cm}}
\begin{tabular}{C{1.5cm}C{1.2cm}C{1.2cm}C{1.7cm}C{1.5cm}|C{1.3cm}C{1.3cm}C{1.7cm}}
\toprule
\textbf{Pseudonym}&\textbf{Target{\newline} Size}&\textbf{Acuity {\newline} Level} & \textbf{VisualField {\newline} Level} &\textbf{Dominant {\newline} Eye} &\textbf{Regular Mode}&\textbf{Lens \newline Magnifier}&\textbf{Full-screen Magnifier}\\
\midrule
Tim & 56 & Low & Intact & - & 0.39\% & 0.28\% & 2.72\%\\ 
\hline
Judy & 88 & Low & Limited  & R & 3.47\% & 1.34\% & 8.43\%\\ 
\hline
Bella & 64 & Low & Intact  & - & 4.93\% & 4.17\% & 3.73\%\\ 
\hline
Mark & 60 & \change{High} & Limited  & \change{-} & 4.56\% & 62.95\% * & 13.43\%\\ 
\hline
Kate & 36 & High & Intact  & - & 1.46\% & 1.02\% & -\\ 
\hline
Robin & 36 & High & Limited  & - & 0 & - & -\\ 
\hline
Hailey & 36 & High & Intact  & R & 0.12\% & - & -\\ 
\hline
Lucy & 56 & \change{High} & Limited  & \change{-} & 27.72\% & 1.83\% & 6.78\%\\ 
\hline
Mary & 80 & Low & Intact  & - & 1.19\% & 0.62\% & 1.28\%\\ 
\hline
Caroline & 40 & High & Intact  & L & 0.29\% & - & -\\ 
\hline
Hannah & 40 & High & Limited  & - & 0.04\% & 0.04\% & 0.38\%\\ 
\hline
Maeve & 40 & High & Intact  & L & 1.67\% & 1.81\% & 0.97\%\\
\hline
\change{May} & \change{64} & \change{High} & \change{Limited}  & \change{-} & \change{5.09\%} & \change{3.79\%} & \change{3.30\%}\\ 
\hline
\change{Piper} & \change{48} & \change{Low} & \change{Limited}  & \change{-} & \change{5.12\%} & \change{2.45\%} & \change{11.86\%}\\ 
\hline
\change{Diego} & \change{52} & \change{Low} & \change{Intact}  & \change{-} & \change{1.78\%} & \change{0.60\%} & \change{2.16\%}\\ 
\hline
\change{Danilo}& \change{92} & \change{Low} & \change{Limited}  & \change{-} & \change{16.17\%} & \change{-} & \change{11.99\%}\\ 
\hline
\change{Ryan} & \change{36} & \change{High} & \change{Intact}  & \change{-} & \change{0.19\%} & \change{0.22\%} & \change{0.77\%}\\ 
\hline
\change{Marilyn} & \change{80} & \change{Low} & \change{Limited}  & \change{L} & \change{4.61\%} & \change{3.59\%} & \change{8.68\%}\\ 
\hline
\change{Julia} & \change{52} & \change{High} & \change{Intact}  & \change{R} & \change{1.11\%} & \change{1.19\%} & \change{1.02\%}\\ 
\hline
\change{Fiona} & \change{92} & \change{Low} & \change{Limited}  & \change{R} & \change{12.57\%} & \change{12.41\%} & \change{14.88\%}\\ 

\bottomrule
\end{tabular}
\caption{Gaze calibration and collection information for the \change{20} low vision participants. \change{The table specifies each participant's calibration target size, VisualAcuity level (low vs. high with 20/100 as the threshold), VisualField level (limited vs. intact), the eye(s) we used for data collection ('L' represents left eye, 'R' represents right eye, '-' represents the mean gaze position of two eyes)}. The last three columns represent the gaze data loss rate when reading with the three magnification modes: '-' indicates they did not use the corresponding magnification; '*' indicates the corresponding piece of data was not used in our gaze pattern analysis due to high data loss rate (over 50\%).}
\label{tab:lv_task}
  \vspace{-5ex}
\end{table*}

\textbf{\textit{Data Loss \change{(H1.2)}.}} We collected participants' gaze data during reading. The data loss rate for low vision participants is shown in Table \ref{tab:lv_task}. We compared the data loss between sighted and low vision participants, and an \change{ART} analysis showed no significant difference between the two groups
% ($W = 869$, $p = 0.97$),
\change{($F_{(1,38)} = 2.61$, $p = 0.11$, $\eta^2_{p} = 0.06$)}. \change{This indicated that the eye tracker can successfully collect similar amount of valid data from low vision people as from sighted people, verifying the feasibility of using such an eye tracker for low vision people.}  

%Therefore, it is verified our calibration and study procedure ensures that there was no bias on data quality for low vision and sighted participants. 

While no statistical difference, we found that the eye tracker lost more data from low vision participants \change{($Mean = 4.62\%, SD=6.89\%$)} than from sighted participants ($Mean = 1.35\%$, \change{$SD=1.22\%$}). The data loss was commonly caused by the eye tracker's failure to recognize the user's eyes. %Similar to prior work's finding~\cite{maus2020gaze}, 
Our observation revealed several reasons that may lead to the recognition failure: similar to Maus et al.'s finding~\cite{maus2020gaze}, some low vision participants had obscured pupils; some participants gradually moved their head closer than 65cm to the screen to read, exceeding the working range of the eye tracker; and some tilted their head to use their functional visual field as they usually did during reading, \change{making one eye unrecognizable}. Our results indicated that the eye tracker manufacturer should take low vision users' reading habits into consideration when developing future products.

\begin{figure*}[h]
  \includegraphics[width=\textwidth]{images/gazeoverlay2x-100.jpg}
  \caption{(a) Example of gaze overlay on the passage (Mark). Orange circles represent fixations (the bigger the longer duration); orange lines represent smooth pursuits; blue and pink lines represent forward saccades and backward saccades, respectively; (b) Example of searching for the next line under the lens magnifier (Mary): the participant hesitated between the second line and third line of the passage after finishing reading the first line.}
  \label{fig:overlay}
  \Description{
  Two images labeled (a) and (b) show examples of our low vision participants' gaze overlay on reading materials. Image (a) shows Mark's gaze overlay on the text when he read using the regular mode with large font. The position of fixations and saccades roughly align with the lines of the text. Image (b) shows Mary's gaze overlay on the text when she searched for the beginning of the next line using the lens magnifier. After finishing first line, Mary's fixations and regressive saccades first landed on the beginning of the third line in the magnification window, and then gradually navigated back to the beginning of the second line after realizing she was not on the correct line.
  }
    \vspace{-2ex}
\end{figure*}

\textbf{\textit{Gaze-Reading Alignment.}}
We also examined the alignment between the collected gaze trajectory and participants' reading progress to validate the gaze data. Fig \ref{fig:overlay}a shows an example of a low vision participant's gaze fixations and saccades overlaying on one passage read by him. \change{We extracted the timestamps  when participants read the last word of each line in a passage from audio recordings ($T_{\text{audio}}$) using VOSK~\cite{vosk} and compared $T_{\text{audio}}$ to the timestamps  of the return sweeps during line switchings from gaze data recordings ($T_{\text{gaze}}$). Using a Pearson's correlation test, we found that the time of each return sweep was highly correlated to the time of line switching during reading aloud for both sighted ($r(1013) = 1.00$, $p < 0.001$) and low vision ($r(286) = 1.00$, $p < 0.001$) participants.
%We fitted a linear regression model to the timestamp data. The fitted model for sighted participants is $T_{\text{gaze}} = 0.57 + 1.00 * T_{\text{audio}}$ ($R^2 = 1.00$, $F_{(1,286)}=1.94\times 10^6$, $p < 0.001$) and the model for low vision participants is $T_{\text{gaze}} = 0.60 + 1.00 * T_{\text{audio}}$ ($R^2 = 1.00$, $F_{(1,1013)}=3.98\times 10^7$, $p < 0.001$), indicating the gaze and audio data are well synchronized, with the time difference being 0.57s and 0.6s for sighted participants and low vision participants, respectively.
}


% Two researchers in our team verified that each participant's gazes on the text matches their reading progress in real time by checking the line switching eye movement and the last words they read in each line. The alignment also confirmed the validity of the eye tracking data from low vision participants. 

\subsection{Comparing Low Vision and Sighted People's Gaze Behaviors \change{(H2)}}
%We characterize low vision people's gaze behaviors during reading by comparing low vision and sighted participants' gaze patterns as well as comparing between low vision participants with different visual conditions. 

%\subsubsection{Comparing between Low Vision and Sighted People}
We compared the reading and gaze behaviors of low vision and sighted people when reading in the regular mode. %The following analysis was based on 48 recordings (24 sighted and 24 low vision). 
Similar to prior research~\cite{hallett2015reading, bruggeman2002psychophysics}, we found that low vision participants spent significantly longer time to complete a reading task than sighted controls 
%(Wilcoxon test: $W=469$, $p<0.001$), 
\change{(ART: $F_{(1,78)} = 18.56$, $p < 0.001$, $\eta^2_{p} = 0.19$ )},
with low vision participants' reading time \change{($Mean=122.88s, SD=103.94s$)} being \change{1.8} times of sighted participants' reading time \change{($Mean=68.45s, SD=19.71s$)}.   %107.58s ($SD=54.44s$) reading a passage on average, while sighted participants spent 58.54s ($SD=12.41s$), a Wilcoxon Rank-Sum one-sided test shows the difference is significant.
We looked into participants' detailed gaze behaviors from different aspects below.

\textbf{\textit{Fixation \change{(H2.1)}.}} An ART analysis showed that low vision participants %\change{($Mean=489.25$, $SD=482.53$)} 
had significantly more fixations than sighted participants %$Mean=277.83$, $SD=65.96$, 
\change{(ART: $F_{(1,78)} = 8.30$, $p = 0.005$, $\eta^2_{p} = 0.10$)}. However, their mean fixation duration 
%($Mean=0.15s$,  $SD=0.04s$) 
was significantly shorter than sighted participants %$Mean=0.20s$,  $SD=0.02s$, 
\change{(ART: $F_{(1,78)} = 57.8$, $p < 0.001$, $\eta^2_{p} = 0.43$}). This indicated that low vision people's gaze tended to fixate more frequently during active reading, but each fixation was shorter than sighted people, representing less amount of information being processed \change{per fixation} and \change{lower reading efficiency by low vision people}~\cite{moffitt1980evaluation}. 
% \todo{a visualization? Also, it might be nice to quote some qualitative experiences from participants to confirm this.}

\textbf{\textit{Sacaade \change{(H2.2)}.}} Low vision participants conducted significantly more \change{(ART: $F_{(1,78)} = 4.54$, $p = 0.04$, $\eta^2_{p} = 0.05$)} but shorter (\change{ANOVA: $F_{(1,78)} = 36.74$, $p < 0.001$, $\eta^2_{p} = 0.32$}) forward saccades %\change{($Mean=260.50$, $SD=203.69$)} 
than sighted participants. %$Mean=182.25$, $SD=41.50$,
%and their forward saccade length %(\change{$Mean=2.37$, $SD=1.02$}, normalized to letter number) 
%was significantly shorter than sighted participants %$Mean=3.67$, $SD=0.89$; 
This indicated that the perceptual span of low vision people---the number of letters taken in during a single glance---was shorter than sighted people~\cite{rubin2009role}. They thus needed to fixate and skim more times to read the same content.   
Meanwhile, we found that low vision participants demonstrated significantly more regressive saccades 
%\change{($Mean=160.35$, $SD=205.84$)} 
than sighted participants %$Mean= 76.15$, $SD=28.39$,
\change{(ART: $F_{(1,78)} = 5.50$, $p = 0.02$, $\eta^2_{p} = 0.07$)}. However, no significant difference was found in terms of revisitation rate \change{(ART: $F_{(1,78)} = 1.48$, $p = 0.23$, $\eta^2_{p} = 0.02$)}, showing that low vision people may be able to scan the text to locate the next fixation point (i.e., follow a line) as accurately as sighted people.  

\textbf{\textit{Lines Switching \change{(H2.3)}.}} Locating the next line can be a challenge for low vision people~\cite{verghese2021eye}. \change{10 out of 20} low vision participants reported they had difficulty locating the next line. We found that low vision participants searched significantly more lines %(nLS: $Mean = 0.05$,  $SD=0.06$) 
than sighted participants %$Mean = 0.02$,  $SD=0.05$,
when switching lines (\change{ART: $F_{(1,76)} = 11.64$, $p = 0.001$, $\eta^2_{p} = 0.13$}). 
% \change{Fig. \ref{fig:overlay}b shows one example of a participant hesitating between two lines.} 

We further examined how low vision people located a line. \change{We investigated the distribution of participants' fixation time (i.e., the total time of all fixations happened at a particular area) along the horizontal axis of the reading interface (Fig. \ref{fig:hist_fix})}, and found that low vision participants spent significantly longer fixation time at the first 10\% %(e.g., the length of 7.5 letters if the font size is 16px) 
of each line than sighted people (\change{ART: $F_{(1,76)} = 26.93$, $p < 0.001$, $\eta^2_{p} = 0.26$}). Specifically, low vision participants spent \change{14.87\% ($SD=4.30\%$)} of the total reading time fixating at the first 10\% of a line, while sighted participants used 
\change{11.27\% ($SD=2.34\%$)} of the total reading time. The result suggested that low vision participants spent more time locating and confirming the correct line when switching lines. According to low vision participants, a common strategy of identifying the next line was checking if the content of the new line made sense based on the context of the passage. For example, 
% Tim always remembered the first several words of a line during reading, so that he could decide which line was the correct line. 
\change{Judy always remembered the last several words of a line during reading, so that she could compare the beginning of the next lines to see which line made the most sense.} 
This explained the higher cognitive processing load (longer fixation time) observed from low vision participants at the beginning of each line during line switching.

\begin{figure*}[h]
  \includegraphics[width=0.8\textwidth]{images/fixation_time_hist_nolv17.pdf}
  \caption{\change{(a) The distribution of fixation time along the horizontal axis of the reading interface for low vision participants (mean fixation time across all participants and all passages);  (b) The distribution of fixation time along the horizontal axis of the reading interface for sighted participants.}}
  \label{fig:hist_fix}
  \Description{
  Two figures (a) and (b) show the distributions of fixation time along the horizontal axis of the reading interface for low vision and sighted participants. Figure (a) shows the fixation time density of low vision participants on the Y axis against the horizontal percentage of the text from 0 to 100\% on the X axis. There is a noticeable spike of the fixation time density within the first 10\% of the text. Figure (b) shows the fixation time density of sighted participants on the Y axis against the horizontal percentage of the text from 0 to 100\% on the X axis. The fixation time density is consistent across the horizontal axis.
  }
    \vspace{-2ex}
\end{figure*}

\subsection{Effects of Different Visual Conditions on Gaze Patterns \change{(H3)}}
% stats vwerified? yes
We studied the effect of different low vision conditions on low vision participants' gaze behaviors.

%Comparing the gaze pattern between LA and HA, and between DVF and IVF, we found similar trend as in the comparison between low vision participants and sighted participants: 
\textbf{\textit{Fixation \change{(H3.1)}.}} Our result showed that participants with low visual acuity demonstrated significantly 
% \change{longer reading time ($F_{(1,36)} = 4.88$, $p = 0.03$, $\eta^2_{p} = 0.12$)} 
more fixations (\change{ART: $F_{(1,36)} = 5.18$, $p = 0.03$, $\eta^2_{p} = 0.13$}) but shorter fixation duration (\change{ART: $F_{(1,36)} = 6.01$, $p = 0.02$, $\eta^2_{p} = 0.14$}) than those with relatively high visual acuity. \change{Similarly}, participants with limited visual field had more fixations (\change{ART: $F_{(1,36)} = 4.69$, $p = 0.04$, $\eta^2_{p} = 0.12$}) \change{ and shorter fixation duration (ART: $F_{(1,36)} = 7.88$, $p = 0.01$, $\eta^2_{p} = 0.18$)} than those with intact visual field. 
An interaction between visual field and visual acuity was found. \change{Through a \textit{post-hoc} contrast test for ART}, we found participants with visual field loss and low visual acuity had significantly shorter fixation duration than participants with low visual acuity only 
% ($p = 0.02$, 95\% C.I. = [-0.10, -0.01]), 
\change{($t_{(36)} = 3.08$, $p = 0.02$)},
participants with visual field loss only \change{($t_{(36)}  = 2.88$, $p = 0.03$)},
%($p < 0.001$, 95\% C.I. = [-0.14, -0.04]), 
and participant with both high visual acuity and intact visual field 
\change{($t_{(36)} = 3.24$, $p = 0.01$)}.
%($p < 0.01$, 95\% C.I. = [-0.10, -0.01]).
Our result suggested that \change{both low visual acuity and visual field loss lead to} reduced amount of information low vision people can perceive \change{per fixation}, and \change{experiencing visual field loss and low visual acuity at the same time} can further impact low vision people's ability to access information.
% who has low visual acuity. 

\textbf{\textit{Saccade \change{(H3.2)}.}} In terms of saccade, we found no significant effect of visual acuity \change{(ART: $F_{(1,36)} = 2.14$, $p = 0.15$, $\eta^2_{p} = 0.06$)} or \change{visual field (ART: $F_{(1,36)} = 1.12$, $p = 0.30$, $\eta^2_{p} = 0.03$)} on the number of forward saccades from low vision participants. However, we found that participants with low acuity had significantly shorter forward saccade length \change{than participants with high visual acuity (ANOVA: $F_{(1,36)} = 12.00$, $p = 0.001$, $\eta^2_{p} = 0.25$). Similarly, we found participants with limited visual field had significantly shorter forward saccade length than those with intact visual field (ANOVA: $F_{(1,36)} = 7.97$, $p = 0.01$, $\eta^2_{p} = 0.18$).} This implied that \change{low visual acuity and} limited visual field can \change{both} affect low vision participants' length of perceptual span. 
% For example, as Robin who had \change{peripheral} vision loss commented, \textit{``... When you read from line to line with the (limited) peripheral [vision]..., 
% you [relied on] a certain peripheral [area of vision] 
% when you have to go to the next word. So 
% that's why 
% it slows down the process of you being able to see what the next word is, and \change{fluently} reading through.''}
\change{For example, as Danilo explained, he could only focus on a small area at a time when reading due to his limited visual field, and since his visual acuity was low, he had to recognize letter by letter, because according to him, the letters were "mashed together".}

\change{However, we found no significant effect of either visual acuity or visual field on the number of regressive saccades (visual acuity, ART: $F_{(1,36)} = 3.35$, $p = 0.08$, $\eta^2_{p} = 0.09$; visual field, ART: $F_{(1,36)} = 2.71$, $p = 0.11$, $\eta^2_{p} = 0.07$), and revisitation rate (visual acuity, ART: $F_{(1,36)} = 1.85$, $p = 0.18$, $\eta^2_{p} = 0.05$; visual field, ART: $F_{(1,36)} = 1.83$, $p = 0.18$, $\eta^2_{p} = 0.05$).}

%and more backward saccades ($F_{(1,20)}=10.80$, $p = 0.004$) than those with high visual acuity. The results suggest that people with low visual acuity tend to have more corrective eye movements due to their difficulty with precisely locating their eyes on the next piece of text along a line.  

%Although there is no significant effect of visual acuity and visual field on the revisitation, Such significant effect was not found for visual field.

\textbf{\textit{Lines Switching \change{(H3.3)}.}} We looked into how low vision participants with different visual abilities switched lines, and found no significant effect of visual field \change{(ART: $F_{(1,34)} = 0.0002$, $p = 0.99$, $\eta^2_{p} < 0.001$)} on the number of searched lines during line switching. \change{However, there was a trend towards a significant effect of visual acuity \change{(ART: $F_{(1,34)} = 3.62$, $p = 0.07$, $\eta^2_{p} = 0.10$)} on the number of searched lines. More research is needed to further investigate the impact of visual acuity on low vision people's line switching behaviors.} 
% However, \change{we found that the interaction between visual acuity and visual field had a significant effect on the proportion of fixation time at the beginning (10\%) of each line (ART: $F_{(1,36)} = 9.12$, $p = 0.005$, $\eta^2_{p} = 0.20$), although there was no significant effect of either visual acuity (ART: $F_{(1,36)} = 2.78$, $p = 0.10$, $\eta^2_{p} = 0.07$) or visual field (ART: $F_{(1,36)} = 0.64$, $p = 0.43$, $\eta^2_{p} = 0.02$) was found. Using a \textit{post-hoc} contrast test, we found that among all participants with intact visual field, low visual acuity led to significantly longer fixation time on the first 10\% of each line ($t_{(36)} = -3.78$, $p = 0.003$). For participants with low visual acuity, limited visual field led to longer fixation time on the first 10\% of each line ($t_{(36)} = 2.93$, $p = 0.03$).} \yuhang{need explaination} \todo{any quote from visual field loss people can interpret this?}

% participants with low acuity  fixated their gaze ($Mean = 0.17$,  $SD = 0.05$) longer than those with high acuity ($Mean=0.13$, $SD=0.03$, ANOVA: $F_{(1,20)}=5.75$, $p = 0.03$), meaning low visual acuity can cost low vision people longer time to process and identify the correct line. Some participants with low visual acuity (Bella, Tim, Judy, and Mary) reported difficulty recognizing words even with large font size, which could potentially complicate the line switching process. 

\subsection{Gaze Behaviors under Different Magnification Modes \change{(H4)}} 
% stats verified? yes
 We compared the reading performance and gaze patterns of low vision participants when reading with different magnification modes. To start off, it was not surprising that magnification modes had significant effect on participants' reading time \change{(ART: $F_{(2,26)} = 16.45$, $p < 0.001$, $\eta^2_{p} = 0.56$)}. A \textit{post-hoc} contrast test for ART showed that, participants read much slower using the lens magnifier \change{($t_{(26)} = -5.31$, $p < 0.001$)} or full-screen magnifier \change{($t_{(26)} = -4.53$, $p < 0.001$)} than using the regular mode with increased font size. 
 \change{Nine} participants mentioned that the \change{hand-eye} coordination required by screen magnifiers made reading more difficult. 
 One participant (Bella) told us that she read slower on purpose to make the lens magnifier move stably. No significant difference in reading time was found between the lens magnifier and full-screen magnifier \change{($t_{(26)} = 0.79$, $p = 0.71$)}. 
 
 \textbf{\textit{Fixation \change{(H4.1)}.}} Although there was no significant effect of magnification modes on the number of fixations \change{(ART: $F_{(2,26)} = 0.03$, $p = 0.97$, $\eta^2_{p} = 0.003$)}, we found a significant effect of magnification modes on the mean fixation duration \change{(ART: $F_{(2,55)} = 38.60$, $p < 0.001$, $\eta^2_{p} = 0.58$)}. A \textit{post-hoc} contrast test showed that the lens magnifier and full-screen magnifier both led to significantly shorter fixation duration than the regular mode with increased font size \change{(lens: $t_{(55)} = 8.61$, $p < 0.001$; full-screen: $t_{(55)} = 5.78$, $p < 0.001$). Moreover, participants showed significantly shorter fixation duration when using the lens magnifier than using the full-screen magnifier ($t_{(55)} = -2.84$, $p = 0.02$)}. One possible explanation was that the dynamic changing of text layout (due to the lens movement and the panning) made it more difficult for low vision participants to focus and fixate on individual words, leading to shorter fixation duration. 
 \change{Eight} participants 
 %(Mark, Judy, Hannah and Tim) \todo{check the number and add more quotes} 
 complained that the control of the magnifiers made it harder to keep track of their position, \change{especially for the lens magnifier. Due to the high magnification level, a small deviation on the position of the magnification window would cause a whole line to disappear. 
 % As Judy commented, \textit{``I moved the mouse a little too fast sometimes for my eyes to catch up.''}  
 As Diego commented, \textit{``I drop [the magnification window] down [a little] too far like that, then it can throw me off from one end [of the window] to the other.''}}
 % No significant difference was found between lens magnifier and full-screen magnifier in terms of mean fixation duration \change{($t_{(35)} = -2.84$, $p = 0.02$). 
 
 
 % Although different magnification modes had significant effect on the reading time ($F(2,12)=5.82$, $p = 0.02$), no significant difference was found between each pair of magnification modes using a pairwise t-test with Bonferroni correction. We found there is no significant effect of magnification modes on number of fixations ($F(2,12)=0.13$, $p = 0.88$). Our result shows that magnification modes have significant effect on average fixation duration ($F(2,12)=6.72$, $p = 0.01$), however, no significant difference was found between each pair of magnification modes. The above results indicate that the amount of information participants could access at a time does not vary using different magnification modes.

\textbf{\textit{Saccades \change{(H4.2)}.}} There was no significant effect of magnification modes on the number of forward saccades (\change{ART: $F_{(2,26)} = 3.07$, $p = 0.06$, $\eta^2_{p} = 0.19$}) and mean forward saccade length (\change{ART: $F_{(2,26)} = 1.34$, $p = 0.28$, $\eta^2_{p} = 0.09$}). \change{However, a significant effect was found on the number of regressive saccades (ART: $F_{(2,26)} = 6.19$, $p = 0.01$, $\eta^2_{p} = 0.32$). Through a \textit{post-hoc} contrast test, we found that low vision participants had more regressive saccades when using the lens magnifier \change{($t_{(26)} = -2.85$, $p = 0.02$)} and full-screen magnifier \change{($t_{(26)} = -3.22$, $p = 0.01$)} than using the regular mode.} 
%However, no sifgnificant difference was found between lens magnifier and full-screen magnifier($t{(29)} = -0.48$, $p = 0.88$). The result suggests that \todo{add participants' quote and explanation}.
Furthermore, we found magnification modes had a significant effect on revisitation rate \change{(ART: $F_{(2,26)} = 6.62$, $p = 0.005$, $\eta^2_{p} = 0.34$)}. Using a \change{\textit{post-hoc} contrast test}, we found participants showed higher revisitation rate using the lens magnifier \change{($t_{(26)} = -3.19$, $p = 0.01$)} and the full-screen magnifier \change{($t_{(26)} = -3.12$, $p = 0.01$)} than using the regular mode with increased font size. No significant difference in the number of regressive saccades \change{($t_{(26)} = -0.37$, $p = 0.93$)} and revisitation rate (\change{$t_{(26)} = 0.06$, $p = 1.00$}) was found between the lens and full-screen magnifier.
This could be explained by participants' difficulty of keeping track of their reading positions when using screen magnifiers (Hannah, Judy, Bella, Lucy, Mark and \change{May}). One strategy reported by participants was to go back and check if they had read the content to locate where they were reading, \change{thus leading to more regressive saccades and revisitations}.

\change{For the lens magnifier, we observed that \change{12} low vision participants requested to increase the width of the magnification window.} We thus investigated how the size of the magnification window affected participants' gaze behaviors. We normalized the window width by calculating the number of letters that can be horizontally covered by the window. We then examined the relationship between the mean forward saccade length and the normalized window width. Using a \change{Pearson's correlation} test, we found a positive correlation between the two variables (\change{$r(26) = 0.72$, $p < 0.001$}). We further found a negative correlation between reading time and the normalized window width \change{($r(26) = -0.67$, 
$p < 0.001$), which echoed the result in Legge et al.~\cite{legge1985psychophysics}.} Combined with the prior finding that reading speed was correlated with forward saccade length~\cite{rubin2009role}, our result suggested that a wider magnification window can potentially benefit reading speed by improving perceptual span.

% can further explain shorter forward saccade length in lens magnifier than full-screen magnifier. The window bottlenecked the number of letters participants could see at a time.

% However, we found participants used more forward saccades ($t_{13} = -2.65$, $p= 0.02$) and regressive saccades ($t_{13} = -2.77$, $p= 0.02$) when using full-screen magnifier compared to the regular mode. In terms of forward saccade length, we found a significant difference between lens magnifier and full-screen magnifier ($V = 20$, $p = 0.04$): The forward saccade length when using the lens magnifier ($Mean = 1.92 letters, SD= 0.85$) was significantly shorter than when using the full-screen magnifier ($Mean = 2.18 letters, SD=1.11$). Since full screen magnifier can be considered as a lens magnifier with a full screen sized window, 

% Interestingly, we found participants showed higher revisitation rate using screen magnifiers (lens: $Mean = 0.33$, $SD=0.11$, full-screen: $Mean = 0.33$, $SD=0.12$ ) than the regular mode with increased font size ($Mean = 0.28$, $SD=0.14$). \todo{add participants' explanation}.

\textbf{\textit{Lines Switching \change{(H4.3)}.}} We found that the effect of magnification modes on the number of searched lines during line switching was significant \change{(ART: $F_{(2,26)} = 5.70$, $p = 0.01$, $\eta^2_{p} = 0.30$)}. A \textit{post-hoc} contrast test showed that participants searched significantly more lines to locate the next line when using the lens magnifier than using the regular mode \change{($t_{(26)} = -3.35$, $p = 0.01$)}. %\change{No significant difference was found between regular mode and full-screen magnifier ($t_{(26)} = -1.32$, $p = 0.40$), 
%and between lens-magnifier and full-screen magnifier ($t_{(26)} = 2.03$, $p = 0.13$).} 
% Participants using full-screen magnifier searched more lines than regular mode with increased font size ($V = 71$, $p = 0.03$). 
\change{Fig \ref{fig:overlay}b shows an example where the participant hesitated between the second and third line before landing on the correct line using the lens magnifier.} \change{Six} participants (Judy, Mark, Tim, \change{May, Marilyn and Diego}) explicitly mentioned that the small magnification window in the lens magnifier caused the challenge of locating the next line. Judy further explained that she could easily move the magnifier out of the text area to the white margin on the left because of limited context information presented in the magnification window. To successfully locate the next line, \change{five participants (Judy, Mark, Marilyn, Diego and Julia)} carefully moved the lens magnifier one line down and then back tracked to find the correct location. 

Moreover, for the lens magnifier, we found a negative correlation between the number of searched lines and the normalized height of the window (i.e., the number of lines covered by the window at a time) with Pearson's correlation test \change{($r(26) = -0.60$, $p < 0.001$)}. This meant that the more lines that can be viewed within the magnification window, the fewer lines participants needed to search to locate the correct lines. However, not all participants preferred a taller magnification window. Three participants (Bella, Mary and Lucy) preferred a shorter window since too much content around the current line can be distracting. As Mary commented, \textit{``In some ways [a shorter lens magnifier] was easier because you didn't have so many possibilities [of lines]. Your options were limited to what you needed to choose.''}

When asked about potential improvements on the magnification tools, four participants (Hannah, Judy, Lucy and Robin) suggested highlighting the next line and labeling the index of each line at the beginning to assist with line switching. To reduce the effect of `panning around' on line switching, \change{seven} participants suggested using a larger screen to present all enlarged text, or breaking the text into pages and allowing page flipping (Bella and Tim). As Tim commented, \textit{``The more I have to [move the screen magnifier], the more I'm going to lose my place.''} %Hannah also mentioned smoother movement of the window is desired. 
Furthermore, two participants (Lucy and Tim) suggested marking where their eyes stopped if they look away from the screen to prevent them from losing track of where they were. 

\textbf{\textit{Smooth Pursuits.}} Unlike reading static text, when moving the text by either scrolling, panning, or moving the magnification window, smooth pursuits happened. Our analysis showed that magnification modes had a significant effect on the number \change{(ART: $F_{(2,55)} = 88.08$, $p < 0.001$, $\eta^2_{p} = 0.76$)} and mean duration \change{(ANOVA: $F_{(2,26)} = 24.38$, $p < 0.001$, $\eta^2_{p} = 0.65$)} of smooth pursuits.
\change{\textit{Post-hoc} comparisons showed that} when low vision participants used the lens magnifier or the full-screen magnifier, the number of smooth pursuits was significantly higher than the regular mode \change{(lens: $t_{(55)} = -12.50$, $p < 0.001$; full-screen: $t_{(55)} = -9.98$, $p < 0.001$)}. Moreover, \change{participants conducted more smooth pursuits when using the lens magnifier than the full-screen magnifier ($t_{(55)} = 2.53$, $p = 0.04$).} 
The mean duration of smooth pursuits when using the two screen magnifiers was also significantly longer than using the regular mode \change{(lens: $t_{(26)} = -6.58$, $p < 0.001$; full-screen: $t_{(26)}= -5.32$, $p < 0.001$)}.
However, there was no significant difference in the duration of smooth pursuits between the two magnifiers  (\change{$t_{(26)} = 1.26$, $p = 0.43$}). Due to the higher number and longer duration of smooth pursuits, participants (Mary and Hannah) felt more tired when using the \change{lens} magnifier than \change{the other two magnification modes} since they had to constantly track the position of the text in the moving window. As Hannah complained, \textit{``The window is always moving, searching for the right position makes my eyes tired.''}
%make it easier to get tired when using lens and full-screen magnifiers. 

\section{Discussion}
\change{We contributed the first research that (1) investigated the suitable gaze calibration and collection for low vision people via commercial eye trackers and (2) thoroughly explored low vision people's unique gaze behaviors and challenges during reading. %  The purpose of this study was to (1) examine the feasibility of using commercial eye tracker for low vision research, and (2) investigate low vision people's gaze patterns during reading to inspire gaze-based assistive technology. Our study validated three out of our four hypotheses:
We answer our four research questions by responding to our four groups of hypotheses.}

\change{With the adjustable calibration interface and dominant-eye-based data collection, we found that commercial eye trackers (e.g., Tobii eye tracker) can achieve comparable quality of gaze data between low vision and sighted people (H1), with no significant difference in the gaze recognition accuracy (H1.1) and data loss (H1.2). The high alignment between participants' gaze trajectory and their reading progress also validated low vision participants' data quality. These results highlighted the potential of commercial eye trackers in low vision research. With commercial eye trackers increasingly integrated in everyday devices, our research opened up a new research direction for gaze-based vision enhancement technology for low vision users.}

\change{We further characterized low vision people's gaze challenges in reading via their gaze patterns. We found that low vision people demonstrated different gaze behaviors from sighted people (H2), with more but shorter fixations (H2.1) as well as more but shorter forward saccades (H2.2), indicating their difficulty with information perception per fixation and skimming. We also identified low vision people's challenges with line switching due to their significant higher number of searched lines (H2.3). Moreover, we found that different visual abilities (H3) and magnification modes (H4) can affect low vision people's gaze behaviors differently. Specifically, both low visual acuity and visual field loss led to more but shorter fixations (H3.1) and shorter forward saccades (H3.2). However, no significant effect of visual abilities (only a trend in visual acuity) was found on line switching behaviors (H3.3). In terms of magnification modes, we found the regular mode with increased font size more effective than the lens and full-screen magnifier, with longer fixation duration (H4.1), fewer regressive saccades, lower revisitation rates (H4.2), and fewer searched lines during line switching (H4.3).}   
%saccade patterns (H2.2) and line searching behaviors (H2.3). Therefore our H2 is verified. %how the two populations differ in the way of using their gaze during reading. 
%We then compare between different visual abilities to identify the effect of low vision conditions on people's gaze behaviors (e.g., people with limited central vision have shorter perceptual span), and found that different visual abilities can affect low vision people's gaze behavior differently (H3). To be more specific, we foun both visual acuity and visual field had significant effect on fixation patterns (H3.1) and saccade patterns (H3.2), but only the interaction between these two factors was found to have significant effect on line switching patterns (H3.3). Finally, we compare between different magnification modes to obtain fine-grained insights on the benefits and drawbacks of commonly used assistive aids (e.g., larger window size can improve reading efficiency). Our result showed that different magnification modes affect low vision people's fixation pattern (H4.1) and saccadde patterns (H4.2) differently, but no significant effect of magnification modes was found on the measures regarding lie switching behaviors (H4.3). Therefore, our H4 is patially rejected. 
%the reading eye movement when low participants used three commonly available magnification approaches to obtain fine-grained insights on low vision people's gaze behavior and adaptations when using magnification tools. 

%Our work contributes the first investigation on low vision people's everyday reading experiences using a commercial eye tracker via the lens of HCI. 
Unlike prior work that infers low vision people's visual experiences and challenges via reading performances 
~\cite{bruggeman2002psychophysics, hallett2017screen,
harland1998psychophysics, moreno2021exploratory, ahn1995psychophysics}, our work draw direct evidence on participants' gaze data at the word and sentence level to develop a deep understanding of their challenges. This fine-grained investigation allows us to unfold low vision people's reading difficulties (e.g., locating next line correctly) and derive design implications for more targeted vision enhancement technology based on low vision users' detailed gaze behaviors. %integrating qualitative and quantitative data, we provide detailed analysis to facilitate the design of eye-tracking based visual augmentation system. 
%Although prior research in vision science has utilized eye tracking to understand low vision people's gaze use (mainly focusing on central vision loss)~\cite{rubin2009role, rayner1998eye}, such research interprets the eye movements from an optometry perspective without considering the gaze-based technology opportunities from the HCI perspective. 
We discuss our results by identifying the limitations and potential improvements for current eye tracking technology, as well as deriving implications to guide the design of gaze-based technology for low vision. 

%revealed important characteristics of low vision people's gaze pattern, their findings does not inform low vision people's reading behavior in everyday settings, such as reading news on websites. 

%Although we were able to collect low vision participants' eye movement data and complete analysis regarding their gaze pattern, we identified some limitations of current eye tracking technology for low vision research. On the other hand, the design space for applying eye tracking technology to support low vision people's daily tasks is still huge. Design guidelines need to be derived to pave the way for future accessibility research.

\subsection{Accessible Eye Tracking Technology for Low Vision Users}
\subsubsection{\change{Accessible Calibration Interface for Different Visual Conditions}}
\change{The conventional calibration interface (e.g., Tobii Pro Lab) did not consider low vision people's visual abilities and eye characteristics.} While we refined the calibration interface by customizing the target size for low vision people, addressing the invisible target issue, \change{our calibration process could still be challenging for some low vision participants. For example, some participants could not focus on a target for a long duration required by the calibration due to their visual conditions, such as Nystagmus (Caroline) and dry eyes (Fiona). To address this, future calibration could consider data collection on more targets with shorter duration for each target.} Since different low vision users may have different visual abilities and thus different needs for the calibration process, adaptations based on the user's visual conditions can be integrated, automatically adjusting the calibration interface, such as the number, color, and size of the calibration targets, and the duration of each target.

%Moreover, some participants felt confused about when the data collection for a target starts and ends due to the lack of feedback to indicate the data collection process. Proper feedback should be integrated to assist users to complete calibration.
% As such, visual cues could be used to indicate the data collection status. 
% For example, prior work~\cite{zhao2016cuesee} has investigated the effectiveness of using different visual cues to direct low vision people's attention in visual tasks. 

\subsubsection{\change{Data Collection by Considering the Dominant Eye.}} 
 \change{Binocular data collection strategy can cause low accuracy if the user demonstrates inconsistent eye movement or irregular pupil appearance \cite{maus2020gaze}. We addressed this via dominant-eye-based data collection if a low vision user had an obvious dominant eye (i.e., the weaker eye does not follow the dominant eye or one eye is not recognizable). Our data collection strategy resulted in relatively high accuracy that is comparable to sighted people. Not only for low vision users, prior work with sighted participants also shows that the dominant eye can fixate at a target more accurately and precisely than the non-dominant eye~\cite{simonsz1991covering, gibaldi2017evaluation}. As such, the dominant eye should be considered in data collection when high accuracy is required. While some low-tech tests (e.g., hole-in-the-card test~\cite{durand1910method}) are usually needed to determine eye dominance without prior knowledge, future eye tracking technology should consider how to automatically detect the dominant eye. Despite the advantages of dominant-eye-based data collection, this strategy can also be volatile when occlusion of the dominant eye occurs due to head and body movement. Tradeoff should be considered to suitably adopt binocular or dominant-eye-based data collection.}
 
 % However, if eye tracker is to be embeded in low vision users' daily life, completing calibration independently can still be a challenging task, because the user need to rely on their own judgement on which eye they should use for calibration without researchers' assistance
 % , which can be confusing to users who do not know how eye trackers work. 
 % To tackle these issues, future eye trackers should be able to detect users' eye conditions to automatically decide if a monocular calibration is preferred, and use the eye with higher recognition confidence (dominant eye) to calibrate and collect gaze data. However, by collecting gaze data from only one eye, dominant-eye-based calibration can be volatile when occlusion of the dominant eye occurs, due to head and body movement. \yuhang{fix}


 

\subsubsection{\change{Requirements on the User Position}}
Current eye trackers impose strict requirements on the user's sitting position, including the distance between the user's eyes and the screen and the relative height of the user's head to the screen. \change{However, this could conflict with low vision people's common reading habits. For example, many low vision people tend to get very close to the screen to read \cite{maus2020gaze, szpiro2016people} but exceeded the standard distance range required by the eye tracker.} \change{Future eye tracker design should consider the unique reading habits of low vision people.} Moreover, wearable eye trackers could be considered to reduce the burdens on low vision users to fulfil the requirements of screen-based eye trackers. % be a better option for low vision users.

\subsubsection{\change{Transparency of the Data Collection Status.}} \change{Beyond the essential issues with current gaze calibration and collection methods, transparency of the data collection status should also be improved to enhance the eye tracker usability.} For example, some participants (Piper) felt confused about when the data collection for a target started and ended due to the lack of feedback for data collection status. Moreover, in our study, we found it hard to determine and maintain the data collection quality \change{in real time, which led to unusable data and discarded participant samples}. \change{It is thus important to provide proper feedback on an eye tracking interface to indicate the data collection status (e.g., beginning, end, and failure of data collection), and prompt instructions to the user timely when necessary to ensure high quality data collection.} For example, when the user's eyes become undetectable, the system should prompt the user to adjust their position in an explicit \change{and accessible} way.




\subsection{Design Implications for Gaze-Based Assistive Technology}
Our work highlights the potential of detecting low vision users' eye movement events using commercial eye trackers. We propose the following design implications for future assistive technologies using eye tracking.

\subsubsection{Real-Time Support for Line Following and Line Switching}
Locating the next line and keeping track of the current line is a big challenge for low vision people, especially when using screen magnifiers. With eye tracking, we can identify the line a user is reading and their intent to switch lines by recognizing the regressive saccade (i.e., return sweep) at the end of each line. As such, visual augmentations could be designed to provide real-time support, such as highlighting the current or next line~\cite{gowases2011text}, or dynamically adjusting the line spacing based on the users' gaze position. Participants mentioned different strategies to orient themselves in a reading material, for example, remembering the last few words of the line to see if the next line made sense (Judy, Lucy, \change{Danilo, Marilyn and Julia}), or memorizing the first several words of the prior line to see if they had read the same line (Lucy, Tim, \change{Piper and Fiona}). With eye tracking, an assistive system can be designed to remind the user of those key words at the right timing to help them locate the correct line. Moreover, we can detect irregular gaze behaviors during reading, such as losing track of a line due to looking away from the screen or the moving of the magnifier, and generate feedback to notify the user where they were reading to improve their reading efficiency. 


\subsubsection{\change{Support for Unrecognizable Words}}
While many low vision people prefer using vision to access information~\cite{szpiro2016people, zhao2015foresee}, some participants  had trouble recognizing words due to low visual acuity or visual field loss, such as distorted words (Judy) and missing letters (Mary, Bella and \change{Julia}). With eye tracking technique, we can potentially recognize these issues and provide corresponding assistance. For example, when a long fixation at a word or frequent revisitations are detected, the system can directly read the nearby words or phrases to the user to improve their reading experience. 

%Furthermore, prior research on AR-based visual and audio augmentations~\cite{stearns2018design, zhao2015foresee, zhao2016cuesee, zhao2020effectiveness} has demonstrated great potential to support low vision people in visual tasks. With eye tracking embedded in AR glasses (e.g., HoloLens), we can extend such research and design gaze-based enhancements on AR devices to facilitate reading beyond computer screen. For example, during navigation, we can design an AR system that reads a street sign at distance for low vision users if they fixate on that sign.

%a gaze-based AR augmentation can recognize the surrounding obstacles and provide multi-modal alerts if a user is approaching an obstacle but has not noticed it (i.e., no gaze points around the obstacle). 

\subsubsection{Hands-Free and Context-Aware Screen Magnifier}
Eye tracking technique has the potential to improve screen magnifiers. Moving and tracking the screen magnifier can diminish reading performance and experience \cite{moreno2021exploratory, ahn1995psychophysics, szpiro2016people, hallett2015reading, hallett2017screen}. \change{Nine} low vision participants in our study echoed this problem of hand-eye coordination. \change{Thus, instead of moving the magnifier around with a mouse, hands-free screen magnifiers controlled by gaze could be more desirable~\cite{maus2020gaze, aguilar2017evaluation}.}
% the movement of the magnifier needs to be smoothed to reduce nausea~\cite{hallett2017screen} and visual tiredness. 
\change{Another drawback of screen magnifiers is fixed local-view to global-view ratio.}
% Screen magnifiers are commonly used by low vision people. Interestingly, 
Our study revealed that different window size for the lens magnifier \change{had different pros and cons:} taller windows (i.e., window containing more lines) contain more contextual information, \change{but can be more distracting}; while shorter windows reduce distraction, \change{but lacks sufficient context}. Future technology can consider a context-aware magnifier, which detects the user's needs and adjust the magnification window accordingly.
% that context-aware magnification can be a potential solution. 
For example, when the system detects that the user loses track or is switching lines, the window size would be automatically enlarged to contain more contextual information; when the user is actively reading a line, the window size would be reduced to include fewer lines to reduce distraction. 
% around the text the user is reading. 

\subsection{Limitations \& Future Work}
Our research has limitations.  
\change{First, our low vision participants presented a variety of visual conditions, which can potentially reduce the statistical power. However, low vision is complex and it is very difficult to recruit many low vision participants with the same visual abilities. For example, participants who have similar visual acuity could have very different field of view; participants with the same diagnosed condition could experience different severity.
As such, as the first step of exploration, we broadly recruited participants with different visual abilities and analyzed the effect of visual abilities on low vision people's gaze behaviors. This study helped us identify particular low vision conditions that worth further investigation, for example, people with Nystagmus are particularly hard for eye tracker to track their gaze. In the future, we will focus on particular low vision groups and recruit more participants for a more thorough analysis. Second, our low vision group and sighted group had a big age difference. While prior research has shown that age does not affect low vision people's gaze behaviors during reading~\cite{bowers2001eye}, future research should consider recruiting a sighted control group at the similar age level to the low vision group to remove the potential effect.}
%see low vision as a whole population and recruit low vision participants with different visual conditions. thus increasing the complexity of our participant  the   due to the complexity and diversity of visual abilities in low vision population, it is almost impossible to recruit low vision participants with the same visual abilities. The variety of low vision conditions can reduce the statistical power in the analysis comparing different visual conditions. To address this issue, we will try to identify and involve more visual condition factors to better characterize the reading behavior of people with different low vision conditions.}
%Moreover, while broadly recruiting low vision participants with different visual conditions, our study did not cover all possible low vision conditions. For example, although we had six participants who experienced central vision loss, we did not find any participants who developed two PRL ~\cite{crossland2005preferred}. In the future, we will recruit more low vision participants to explore the feasibility of eye trackers and potentially design suitable eye tracking algorithms to support particular low vision conditions. 
Lastly, to support the analysis of gaze data, we asked participants to read the passages aloud to confirm their reading progress. However, it is unclear whether our findings could be generalized to silent reading, given that reading aloud is slower and involves more frequent fixations than silent reading~\cite{rayner200935th}. %Besides, due to technical limitation of the eye tracker, we asked participants to take off their glasses if they could. Therefore, their reading behaviors recorded in our study can be different from their usual way of reading with glasses. 
Future research should examine low vision people's gaze behaviors in silent reading to better simulate their daily reading experience.

\section{Conclusion}
\change{In this paper, we explored the potential of using a commercial eye tracker for low vision people and investigated their reading gaze behaviors by conducting a study with 20 low vision participants and 20 sighted controls. We improved the traditional calibration interface and validated the effectiveness and data quality of the eye tracker for low vision users. We further explored low vision people's gaze behaviors, revealing their unique gaze patterns compared to sighted people, as well as the effect of different visual abilities and magnification modes on low vision people's gaze behaviors. 
We hope that our work will inspire the design and development of gaze-based assistive technology for low vision people.}

\begin{acks}
We would like to thank Davit Khachatryan for his contribution to participant recruitment and data analysis, as well as our study participants for their valuable feedback. This work was partially supported by the University of Wisconsin--Madison Office of the Vice Chancellor for Research and Graduate Education with funding from the Wisconsin Alumni Research Foundation, and the Expanding Our Vision 2021 Award from the McPherson Eye Research Institute at the University of Wisconsin--Madison. 
\end{acks}

% While being able to read with screen magnifiers, low vision people suffer from slow and unpleasant reading experiences. Eye tracking has the potential to improve their experience by recognizing people's fine-grained gaze behaviors and providing more targeted enhancements. To inspire gaze-based technology, we aim to \change{investigate the suitable method to collect low vision users' gaze data via commercial eye trackers and thoroughly understand their challenges in reading based on their gaze behaviors}. With an improved calibration interface, we collected the gaze data of \change{20 low vision participants and 20 sighted controls} who performed reading tasks on a computer screen; low vision participants were also asked to read with different screen magnifiers. We found that, with suitable calibration interfaces and data collection method, commercial eye trackers can be promising tools for low vision research. Our study identified low vision people’s unique gaze patterns during reading, building upon which, we propose design implications for gaze-based low vision technology.

\bibliographystyle{ACM-Reference-Format}
\bibliography{sample-base}

\end{document}
\endinput
%%
%% End of file `sample-authordraft.tex'.
