\section{Related Work}
% 
% first section, background of low vision (what is), how low vision do visual tasks, however, eye movement is not studied, but it's important (foundamental, gaze behavior) however, limited work on eye tracking for low vision 
% use eye tracking for low vision, however, it's very nascent in hci research. no connection bet how to leverage finding to facilitate technology design
% sighted people research
% define terms in analysis when i use them
%
% conenct related works with our work
%

\subsection{Experience of Low Vision People during Reading}
Reading is a common challenge for low vision people. The reading performance of low vision people has wide variation due to different visual conditions~\cite{legge1985psychophysics}. For example, people with cloudy or blurry vision demonstrate a strong dependence of reading time on word length due to reduced visual span~\cite{legge1997psychophysics}; people with Macular Degeneration which causes central vision loss read more slowly than people with equivalently reduced visual acuity but intact central vision~\cite{krischer1985visual, legge1985psychophysics}.

To leverage their residual vision, low vision people use different magnification methods for reading, such as increasing font size or screen magnifiers.
Today most computers and smartphones support system embedded screen magnifiers~\cite{maczoom, winmag}, which enable users to magnify a particular screen area based on the mouse position on the screen. %There are two types of screen magnifiers: \textit{Lens magnifier} that allows a user to magnify part of the screen in a window while maintaining the global layout, and \textit{Full-screen magnifier} that magnifies the whole screen and requires the users to pan around to see the off-screen content. Both magnifiers require manual operations to magnify the region of interest. 
Although screen magnifiers can address the low acuity difficulty to some extent, some usability issues have been reported repeatedly by prior research. For example, Moreno et al.~\cite{moreno2021exploratory} has found that users might lose context when using screen magnifiers because \change{they can only view a partial area of the screen at a time}. %On the other hand, screen magnifiers require horizontal scrolling which is commonly criticized~\cite{horizontal} for poor usability even for normally sighted users. 
%Considering the different visual conditions of low vision users and the usability issues around screen magnifiers, magnifier-aided reading can still be challenging to low vision people. 
\change{The reduced field of view caused by magnification also affects low vision users' reading speed, for instance, they recognize fewer letters with one fixation than sighted people} \cite{verghese2021eye, cheong2007relationship, cheong2008relationship}.
Moreover, moving the mouse around to magnify different areas on the screen demands high spatial visualization skills, \change{increasing users' cognitive load during reading}~\cite{hallett2015reading, szpiro2016people}. \change{This issue makes it particularly difficult to reposition the magnifier to locate the next line, especially for users with visual field loss}~\cite{ahn1995psychophysics, verghese2021eye}. 

With the limitations of screen magnifiers, low vision people's reading experience is largely diminished. Prior research shows that they read text blocks 3.2 times slower than sighted people \cite{hallett2015reading, bruggeman2002psychophysics}. As such, more technologies (e.g., gaze-based technology) are needed to mitigate the reading barriers for low vision people. 

% Due to the reduced view of the text caused by magnifiers, users' reading speed becomes slower (e.g., fewer letters can be recognized at a time) \cite{verghese2021eye, cheong2007relationship, cheong2008relationship}. 

%As opposed to the screen magnifiers that require users to pan around both horizontally and vertically, many webpages support Responsive Web Design (RWD)~\cite{rwd}, allowing users to magnify the interface without increasing the width of the layout, so that a user only needs to scroll down to read the enlarged content. Prior work shows that magnification with RWD is more usable and causes less nausea compared to scrolling horizontally~\cite{hallett2017screen}.


%Prior works on retrieving information with low vision were mostly focused on reading performance characterized by objective measurements such as reading speed/time, and/or comments through interviews. More efforts should be done to study the eye movement of low vision people to understand directly how they use their gaze when reading, so as to support the development of fine-grained assistive technology.


\subsection{Understanding Human Gaze via Eye Tracking}
Eye gaze conveys a variety of information. Understanding human gaze can help elucidate how people complete different visual tasks and \change{provide insights to augment low vision people's reading experience}. 

\subsubsection{How does eye tracker work?}
% In order to obtain high quality eye movement data, many eye-tracking techniques have been developed and deployed~\cite{kar2017review, zhang2021eye}. 
% Although deep learning algorithms that track eyes via RGB cameras have progressed rapidly,
% \change{Commercial eye trackers, such as Tobii EyeX, to date achieve better performance than other popular gaze estimation methods~\cite{zhang2021eye} in terms of accuracy and ease-of-use}
\change{Commercial eye trackers are commonly used in most eye-tracking research due to its high accuracy and ease-of-use~\cite{zhang2021eye}.} 
% \yuhang{rewrite these two sentences; they do not connect well.}
Screen-based eye tracking involves capturing the user's face and eye region, detecting pupils, and mapping the user's gaze to the computer screen coordinate~\cite{kar2017review}. However, the gaze direction cannot be directly determined based on the pupil position: the area of retina with highest acuity is called fovea~\cite{kar2017review, guestrin2006general}, and the visual axis connecting fovea and the center of corneal curvature determines the true gaze direction; the angular offset between the optical axis (which can be determined by the user's pupil and head position) and visual axis---which is called kappa angle---is user dependent~\cite{kar2017review, zhu2007novel} (Fig.\ref{fig:eye}). Therefore, a calibration needs to be completed to solve the kappa angle, so that an accurate mapping between eye images to gaze directions can be obtained. 
The calibration for eye tracker is usually performed by asking the user to gaze at multiple small targets distributed over the screen for a certain amount of time~\cite{kar2017review}. The eye tracker can then collect the user's eye images corresponding to these target points and estimate the kappa angle. 
%\change{Prior work on understanding low vision people's reading challenges mostly focuses on reading performance characterized by indirect measurements such as reading speed, and comments from interviews. More efforts are needed to study the eye movement of low vision people to understand directly how they use their gaze when reading, so as to support the development of fine-grained assistive technology. However, no research has explored the feasibility of using commercial eye trackers for low vision users.}

However, the same procedure might not work for low vision users~\cite{shanidze2020eye}. For example, some low vision people's visual acuity might be too low to see the calibration targets clearly \cite{murai2010eye}, leading to inaccurate calibration results; 
\change{some may have irregular eye appearance that prevents pupil recognition (e.g., Coloboma~\cite{pagon1981ocular})}.
% and some people with central vision loss might develop multiple pseudo foveas (thus multiple visual axes) to compensate for their central vision~\cite{prahalad2020asymmetries, duret1999combined}, which makes the kappa angle inconstant and difficult to be inferred through the same calibration algorithm. 
\change{It is thus important to explore how to refine the calibration and data collection process to collect high-quality gaze data from low vision users.} 

\begin{figure}
  \includegraphics[width=0.4\textwidth]{images/eye.png}
  \caption{Model of human eye ball. The optical axis connects the center of corneal curvature and the pupil center; the visual axis connects the fovea and the center of corneal curvature. Kappa angle is the angular deviation between the optical and visual axis~\cite{kar2017review}.}
  \label{fig:eye}
  \Description{The figure shows the structure of a human eye ball. There are four important components of a human eye in the figure: cornea, pupil, lens, retina and the fovea on retina. The optical axis connects the center of corneal curvature and the pupil center; the visual axis connects the fovea and the center of corneal curvature. Kappa angle is the angular deviation between the optical and visual axis.}
    \vspace{-2ex}
\end{figure}


\subsubsection{Understanding the gaze behaviors of sighted people.} A myriad of prior research has leveraged eye tracking technologies to understand sighted people's gaze behaviors \cite{rayner1998eye, rayner2010eye, zambarbieri2012eye}. Two eye movement events are typically used to measure and interpret gaze behaviors, including \textit{fixation} (i.e., gaze pauses over informative regions of interest) and \textit{saccade} (i.e., rapid, ballistic movements between fixations) \cite{salvucci2000identifying}. In reading scenarios, fixations and saccades encode abundant information regarding the process of perception and comprehension~\cite{hallett2015reading, rayner1998eye, rayner2010eye, jarodzka2017tracking}.
 For example, the duration of a fixation indicates the difficulty of perceptual or cognitive processing during reading since visual information can only be perceived during fixations~\cite{zambarbieri2012eye}. The main function of saccades is to move a new block of text into foveal vision where visual acuity is the highest, because reading with parafoveal or peripheral vision is difficult~\cite{rayner1998eye, verghese2021eye}. Saccades can be further categorized according to their roles~\cite{bax2013cognitive, rayner2012psychology, rayner1998eye}. Forward saccades happen when reading onwards in left-to-right languages. Regressive saccades are conducted to revisit words covered by previous forward saccades in order to enhance comprehension~\cite{hallett2015reading}. Return sweeps are to switch focus from the end of one line to the beginning of the next. Interestingly, during a return sweep, readers often undershoot and make small corrective movements to reach the accurate location, which also leads to regressive saccades~\cite{rayner1998eye}. Besides, prior work \cite{bowers2017microsaccades, engbert2003microsaccades} has also studied the function of microsaccades (i.e., very small saccades) and suggests that microsaccades are used to correct previously made long saccades and gain additional information about adjacent words~\cite{bowers2017microsaccades}.

%For example, Rayner et al. compared fast and slow readers and found that slow readers had more fixation, longer fixation time, and shower saccades than fast readers \cite{rayner2010eye}. \yuhang{need more work for sighted people, both understanding their gaze behaviors, and gaze-based technology for sighted people.} \cite{zambarbieri2012eye}




%Mainly two types of eye movement events were studied by previous works in the field of vision research, psychology, and HCI: Fixation and saccade. Fixations are pauses over informative regions of interest~\cite{salvucci2000identifying}. During fixations, eyes are stationary and visual input can be processed\cite{kar2017review}. Saccades are the rapid, ballistic movements between fixations\cite{salvucci2000identifying, verghese2021eye}. However, unlike fixations, the sensitivity to visual input is reduced during saccade, which is called "saccadic suppression". In reading scenario, fixations and saccades encode abundant information regarding the process of perception and comprehension\cite{hallett2015reading, rayner1998eye, rayner2010eye, jarodzka2017tracking}. 


\subsubsection{Eye tracking research for low vision.} Although previous research on eye tracking technology and understanding human gaze is fruitful, very few efforts have been made to investigate the eye movement of low vision people while performing daily tasks. 

Some researchers in optometry and vision science have started investigating low vision people's gaze patterns using eye trackers. Most works focus on people with central vision loss, who need to adopt an eccentric retinal location as a substitute for the fovea, also known as the preferred retinal locus (PRL). Such research explores how the use of PRL affects low vision people's visual tasks by tracking their eye movement \cite{prahalad2020asymmetries, verghese2021eye, renninger2011recalibration, kwon2013rapid}. For example, Rubin and Feely \cite{rubin2009role} have used an SMI high-speed video tracker to evaluate the fixations and saccades of 40 people with age-related macular degeneration, showing that participants' reading performance were significantly associated with the fixation stability, proportion of regressive saccades, and length of forward saccades. Prahalad and Coates \cite{prahalad2020asymmetries} use an EyeLink 1000 Plus eye tracker (the approximate price is around \$30,000 \cite{eyelink}) to explore how people with central vision loss choose their PRL. They have recruited six sighted people who experience simulated scotoma. Participants are trained to use different artificial PRL (left, right, inferior) to read. The study shows that a right PRL causes \change{longer saccades} and more directional switches, indicating that people with central vision loss may avoid right PRL due to an oculomotor reason. Moreover, Bullimore and Bailey \cite{bullimore1995reading} use a specialized infrared scleral reflection device~\cite{brown1977clinically} to track the eye movement of both sighted participants and participants with age-related maculopathy. They find that participants with central vision loss demonstrate a large number of regressive saccades because the scotoma distorts or occludes part of the words. % \change{Whittaker et al.~\cite{whittaker1988eccentric} used a search coil eye tracker embedded in a contact lens~\cite{robinson1963method} to reveal the characteristics of eye movement when participants with macular degeneration were asked to fixate at a letter eccentrically. They suggest that the adaptation to eccentric fixation is difficult when large saccades are required.}

However, research in vision science usually leverages clinical or research eye trackers, which are highly expensive and not commonly available for everyday use. \change{Some researchers even created specialized eye trackers for their research} \cite{brown1977clinically, robinson1963method}. Moreover, none of the above research detailed the specific calibration and data collection process for low vision participants and the potential challenges they may encounter. 

There has been very limited research in HCI that investigates gaze-based technology for low vision people using eye trackers. Murai et al. \cite{murai2010eye} have used Mobile Eye, a device unit with an eye-camera and a scene-camera to recognize people's gaze point in the environment. They design a calibration interface with large and high contrast letters for low vision people and estimate their gaze circles (i.e., an area the low vision user focuses on) instead of gaze points. They find that while it is possible to calibrate for people with severe low vision, the eye tracker could not be used by people with nystagmus (i.e., involuntary eye movement). However, this research has been conducted a decade ago. With more advanced commercial eye trackers deployed to the market, it is necessary to revisit this problem and investigate how to better leverage the state-of-the-art commercial eye trackers for low vision research. More recently, Masnadi et al. \cite{masnadi2020vriassist} have designed VRiAssist, a gaze-based vision enhancement tool in virtual reality that dynamically follows the user's gaze and generates visual corrections, thus helping low vision people see better in a virtual space. However, no information is provided regarding what eye tracking technology is used and how they calibrate and collect gaze data from low vision users. Moreover, Maus et al. \cite{maus2020gaze} have designed a gaze-guided magnification with a commercial GazePoint GP3 eye tracker to enable low vision people to control the on-screen magnifier with their gaze. They demonstrate that the current video-based eye tracking technique does not consider low vision people and result in a large amount of data loss. 

While researchers have recognized the potential of eye tracking and gaze-based vision enhancements for low vision people \cite{masnadi2020vriassist, maus2020gaze, murai2010eye}, it is still unclear whether and how low vision people can leverage such technology in their daily life. Our research will fill this gap and explore the feasibility of off-the-shelf eye trackers for low vision people by refining the calibration and data collection process, as well as investigating their gaze behaviors during reading tasks to inspire gaze-based technology.

%To design gaze-based technology for low vision people, it is important to explore the feasibility of the more available commercial eye trackers. 

%\yuhang{while limited, there are still several papers that explored gaze-based technology for low vision people. Make sure that we have a separate section to thoroughly discuss those work. ALso, we should discuss papers in vision science if they also use eye trackers for low vision people.}

%worse eye occluded with an eye patch \cite{rubin2009role}

