\section{Introduction}
% outline:
% what is low vision and how their condition impact reading
% to tackle this problem, assistive technologies/ a11y features are developed to support them, but that does not reveal how they read and what difficulty they have seeing the text
% naturally we think about utilizing eye tracking, but the usability of commercial eye tracking devices for low vision is understudied. 
% better understanding the usability of eye trackers for low vision, and their eye movements can benefit even more
% what we gonna do is, run study with 12 normal and 12 lo v, evaluating the calibration accuracy and track their eyes to see how they read.

Reading is a critical daily task for people to access a variety of visual information, from the prescription on the pill bottle to the traffic signs on the side of streets. However, this task could be challenging to low vision people. Low vision is a visual impairment that cannot be corrected by eyeglasses, contact lenses, and other standard treatments~\cite{nei}. There are different low vision conditions, such as central vision loss, peripheral vision loss, night blindness, and blurry vision~\cite{nei}. Different conditions can affect low vision people's reading ability and behaviors in different ways, for example, words may appear distorted to people with central vision loss, and people with severe peripheral vision loss may only see one or two words at a time without being able to scan ahead.

While experiencing visual difficulties, most low vision people prefer using their residual vision~\cite{szpiro2016people,zhao2015foresee}. As a result, they often get close to a reading material to see details, and use low vision aids, such as handheld magnifiers, to magnify the text. On computers and smartphones, they use embedded accessibility features, such as screen magnifiers, enlarged font size, and contrast adjustment, to enhance content visibility. Although these low vision aids can compensate for low visual acuity to some extent, low vision people still face challenges in reading. For example, they experience short visual span and have difficulty in switching lines due to magnification and increased font size~\cite{verghese2021eye, cheong2007relationship, cheong2008relationship}, leading to slow reading. Prior research shows that low vision people read text blocks about 3 times slower than sighted people~\cite{bruggeman2002psychophysics}. More research is needed to understand low vision people's reading behaviors and design technologies to mitigate their reading barriers. 

%Previous research mostly investigated low vision readers' reading experience by measuring their reading performance (e.g., reading speed, accuracy)~\cite{moreno2021exploratory, harland1998psychophysics, ahn1995psychophysics, gowases2011text, lee2021bringing, legge1985psychophysics}, self-reported likert scales~\cite{lee2021bringing, hallett2015reading, gowases2011text}, and collecting qualitative feedback~\cite{szpiro2016people, lee2021bringing, gowases2011text}. While these measures can help understand low vision people's general reading difficulties, they do not provide sufficient data to depict people's detailed reading behaviors, for example, how they follow a line during reading, how they locate the next line, and what types of words are difficult to recognize. %To this end, we need to uncover the whole reading process, from the moment their gaze land on the text, to every shift of focus they make to process the text.

The advance of eye tracking techniques presents an opportunity to discover low vision people's visual challenges by recognizing their gaze use and enable gaze-based vision enhancement technology. Eye gaze can convey a myriad of information in reading. For example, it indicates readers' visual attention change over time~\cite{vickers2009advances} and elucidates the cognitive processing of words, sentences, and text~\cite{jarodzka2017tracking}. Many works utilize eye tracking to characterize sighted people's reading behaviors~\cite{rayner2010eye, zambarbieri2012eye, rayner1998eye, reichle2003ez} and design gaze-based technology to improve their reading experience~\cite{cheng2015gaze, gowases2011text, maus2020gaze}. %It is thus beneficial to understand low vision people's gaze patterns and design more targeted low vision technology based on users' gaze behaviors in visual tasks.  %is beneficial to the evaluation of interface accessibility and the development of customizable assistive technology. 
However, the gaze research for low vision remains nascent. With \change{commercial eye trackers} increasingly integrated into everyday devices (e.g., laptop, AR glasses), there is great potential in leveraging this technology to understand low vision people's detailed gaze behaviors (e.g., when they deviate from a line of text, lose track of the next line, or struggle with recognizing a word), thus designing assistive technologies that provide more targeted support in reading tasks. 

%unfold low vision people's visual challenges by investigating their gaze use, and design more targeted low vision technology based on their gaze behaviors in reading tasks. 

Despite the potential, eye tracking technology could be challenging for low vision people to use. The state-of-the-art commercial eye trackers (e.g., the Tobii eye tracker) recognize the features of users' eyes (e.g., pupil, iris, corneal reflection) and predict their gaze point via a gaze estimation algorithm~\cite{tobiicali}. To account for the individual differences in the eye shape and geometry, a calibration interface is provided that renders visual targets on a display and collects users' eye data when focusing on the targets. Although these calibration and modeling methods can achieve high accuracy for sighted users~\cite{tobiiproacc}, 
%by a simple calibration process where users are asked to stare at targets distributed on the screen. The calibration method will then map users' gaze to the coordinates on the screen. Unfortunately, the current eye tracking techniques focus on sighted people, 
they overlook the different visual abilities, eye characteristics, and eye usage strategies that low vision people may have \cite{maus2020gaze}. %As such, low vision users may experience challenges using eye trackers. 
For example, the targets on the calibration interface are small and in a fixed size, which can be invisible to users with low visual acuity, causing low calibration accuracy. \change{Some low vision people had extremely unbalanced visual abilities and thus inconsistent gaze behaviors across the two eyes. Moreover, the appearance of eyes could be changed by particular eye diseases (e.g., cataract), causing recognition failure and high data loss.} %Furthermore, for users with central vision loss who use more than one preferred retinal locus (PRL) to replace the damaged fovea, the gaze estimation algorithm for sighted people reading with fovea may not be feasible~\cite{harrar2018nonvisual, prahalad2020asymmetries, duret1999combined}. %\yuhang{do we have any findings that can address this?}. 
With these issues, it is unclear whether and how low vision people can use and benefit from the emerging eye tracking technology. % let alone using this technique to explore their gaze behaviors in reading tasks.   %calibration performance for low vision users, and how reliable eye tracking technology can be used to reveal low vision users' gaze behavior in an everyday reading scenario.

To close the gap, we seek to explore the \change{potential of collecting high-quality gaze data from low vision people using off-the-shelf eye trackers}, and further leverage a commercial eye tracker to investigate low vision people's detailed gaze behaviors in reading tasks to inspire gaze-based low vision technology.  
%improving the calibration interface and method. %the calibration result using monocular calibration method and resizable calibration targets. 
To make the eye tracker more usable to low vision people, \change{an adjustable calibration interface and a dominant-eye-based data collection method} were designed and developed. Based on the improved calibration process, %compared to normally sighted people, and among different types of low vision conditions
we conducted a study with \change{20 low vision participants and 20 sighted controls}, who performed reading tasks on a computer screen with an eye tracker collecting their gaze data. %Participants were asked to complete calibration tasks with both the default and improved calibration interfaces. We then collected participants' track eye gaze data when they read short passages. 
Since the use of screen magnifiers may affect low vision people's gaze behaviors, we also asked low vision participants to read with different types of screen magnifiers (i.e., lens magnifier, full-screen magnifier) to investigate their gaze patterns and challenges in different magnification modes. %At the end, we conducted semi-structured interview regarding their reading experiences based on researcher's observation.
\change{With this study, we seek to answer four research questions: (RQ1) Can we collect reliable eye gaze data from low vision people using a commercial eye tracker? (RQ2) How do low vision people's gaze behaviors differ from sighted people during reading? (RQ3) How do different visual conditions affect low vision people's gaze behaviors? (RQ4) How do different screen magnification modes affect low vision people's gaze patterns?}

Our research demonstrates that \change{the relatively low-cost, high-availability commercial eye tracker} can be a promising tool for low vision research \change{and collect reliable data from low vision people} if the calibration process is accessibly designed. We also discover unique gaze patterns and visual challenges faced by people with different low vision conditions. For example, low vision participants fixate their gaze more than sighted people at the beginning of each line during line switching. Participants with low visual acuity \change{and limited visual field} show more but shorter fixations than those with relatively high visual acuity \change{and intact visual field. Moreover, when using the lens magnifier,
%taller window of lens magnifier makes line switching easier.
increasing the height of the magnification window can make line switching easier.} %Wider window of lens magnifier results in longer perceptual span. 
Based on our findings, we identify potential improvements for eye tracking technology, and derive design implications for gaze-based assistive technology to support low vision people during reading. 
% \yuhang{update results}


% We summarize our contribution as follows:
% \begin{enumerate}
%     \item We evaluated the accessibility of commercial eye trackers and probed direction for future improvement.
%     \item We found that low vision people spend more time during line switching, and search up to x lines above and below the correct line, and the number is correlated with the size of magnification window.
%     \item We discovered missing words are located xxx
% \end{enumerate}