\section{Results}
In this section, we \change{report our results regarding the four groups of hypotheses respectively, including validating gaze calibration and data quality between low vision and sighted participants (H1), comparing gaze behaviors between low vision and sighted participants (H2), and exploring the effect of visual conditions (H3) and magnification modes (H4) on low vision participants' gaze behaviors in reading.} %the feasibility of commercial eye trackers for low vision people by evaluating the calibration performance and data loss. We then report findings regarding low vision participants' gaze behaviors in reading. 

\subsection{Eye Tracker Validation for Low Vision People (H1)}
%We evaluate the calibration efficacy using the mean validation accuracy across the 4 points in validation phase. The accuracy is defined as the average visual angle difference between the target position and the estimated gaze location~\cite{tobiiproacc}. By default, the estimated gaze position is the average of two eyes' position. However, due to some low vision participants' visual condition, we need to use the better eye's data for this validation. 
We validated the gaze calibration result and assessed the data quality from the eye tracker below:

\textbf{\textit{Four-point Validation \change{(H1.1)}.}} 
% First, we want to validate that our calibration interface could achieve comparable performance as Tobii Pro Lab (TPL) when the target size is not adjustable. We compared the 4-point validation accuracy for sighted participants using TPL and our calibration interface. The mean validation accuracy across 12 participants with TPL is 0.73\textdegree ($SD = 0.52$\textdegree), while the mean validation accuracy of our calibration interface is 0.56\textdegree ($SD = 0.19$\textdegree). A Wilcoxon Signed Rank test indicates that the difference between the two calibration results is not significant ($V = 59$, $p = 0.13$), which means our calibration interface is able to achieve comparable calibration efficacy when the target size is fixed.
We adjusted the size of calibration targets for low vision participants. Table \ref{tab:lv_task} presents the target size they selected. To validate the calibration, we compared the gaze recognition accuracy in the four-point validation between low vision and sighted participants.   
% For low vision participants who reported they relied a dominant eye for visual tasks, we used the data of their dominant eye for the data analysis of this study. 
%The mean validation accuracy for low vision participants is \change{1.23}\textdegree~ (\change{$SD=1.27$}\textdegree). 
An \change{ART} analysis showed no significant difference between the gaze recognition accuracy between sighted and low vision participants \change{($F_{(1, 38)} = 0.61$, $p = 0.44$, $\eta^2_{p} = 0.02$)}.
%($t_{14.25} = 1.53, p = 0.15$). 
\change{This demonstrated that the improved calibration interface for low vision participants was as effective as the typical calibration for sighted people.}  

%We further compared the validation result between low vision participants with high visual acuity and low visual acuity (see Table \ref{tab:lv_task}). The difference in validation accuracy between the two groups is not significant using \change{ART} 
%($t_{7.86} = 0.06, p = 0.96$) 
%\change{($F_{(1,18)} = 2.09$, $p = 0.17$, $\eta^2_{p} = 0.10$ )}, which indicates that increased calibration target size does not compromise the calibration performance. 

\begin{table*}[h]
\small
\centering
% \begin{tabular}{p{1cm}p{1.3cm}p{1cm}p{1.2cm}p{1.5cm}p{1.5cm}|p{1.3cm}p{1.3cm}p{1.7cm}}
\begin{tabular}{C{1.5cm}C{1.2cm}C{1.2cm}C{1.7cm}C{1.5cm}|C{1.3cm}C{1.3cm}C{1.7cm}}
\toprule
\textbf{Pseudonym}&\textbf{Target{\newline} Size}&\textbf{Acuity {\newline} Level} & \textbf{VisualField {\newline} Level} &\textbf{Dominant {\newline} Eye} &\textbf{Regular Mode}&\textbf{Lens \newline Magnifier}&\textbf{Full-screen Magnifier}\\
\midrule
Tim & 56 & Low & Intact & - & 0.39\% & 0.28\% & 2.72\%\\ 
\hline
Judy & 88 & Low & Limited  & R & 3.47\% & 1.34\% & 8.43\%\\ 
\hline
Bella & 64 & Low & Intact  & - & 4.93\% & 4.17\% & 3.73\%\\ 
\hline
Mark & 60 & \change{High} & Limited  & \change{-} & 4.56\% & 62.95\% * & 13.43\%\\ 
\hline
Kate & 36 & High & Intact  & - & 1.46\% & 1.02\% & -\\ 
\hline
Robin & 36 & High & Limited  & - & 0 & - & -\\ 
\hline
Hailey & 36 & High & Intact  & R & 0.12\% & - & -\\ 
\hline
Lucy & 56 & \change{High} & Limited  & \change{-} & 27.72\% & 1.83\% & 6.78\%\\ 
\hline
Mary & 80 & Low & Intact  & - & 1.19\% & 0.62\% & 1.28\%\\ 
\hline
Caroline & 40 & High & Intact  & L & 0.29\% & - & -\\ 
\hline
Hannah & 40 & High & Limited  & - & 0.04\% & 0.04\% & 0.38\%\\ 
\hline
Maeve & 40 & High & Intact  & L & 1.67\% & 1.81\% & 0.97\%\\
\hline
\change{May} & \change{64} & \change{High} & \change{Limited}  & \change{-} & \change{5.09\%} & \change{3.79\%} & \change{3.30\%}\\ 
\hline
\change{Piper} & \change{48} & \change{Low} & \change{Limited}  & \change{-} & \change{5.12\%} & \change{2.45\%} & \change{11.86\%}\\ 
\hline
\change{Diego} & \change{52} & \change{Low} & \change{Intact}  & \change{-} & \change{1.78\%} & \change{0.60\%} & \change{2.16\%}\\ 
\hline
\change{Danilo}& \change{92} & \change{Low} & \change{Limited}  & \change{-} & \change{16.17\%} & \change{-} & \change{11.99\%}\\ 
\hline
\change{Ryan} & \change{36} & \change{High} & \change{Intact}  & \change{-} & \change{0.19\%} & \change{0.22\%} & \change{0.77\%}\\ 
\hline
\change{Marilyn} & \change{80} & \change{Low} & \change{Limited}  & \change{L} & \change{4.61\%} & \change{3.59\%} & \change{8.68\%}\\ 
\hline
\change{Julia} & \change{52} & \change{High} & \change{Intact}  & \change{R} & \change{1.11\%} & \change{1.19\%} & \change{1.02\%}\\ 
\hline
\change{Fiona} & \change{92} & \change{Low} & \change{Limited}  & \change{R} & \change{12.57\%} & \change{12.41\%} & \change{14.88\%}\\ 

\bottomrule
\end{tabular}
\caption{Gaze calibration and collection information for the \change{20} low vision participants. \change{The table specifies each participant's calibration target size, VisualAcuity level (low vs. high with 20/100 as the threshold), VisualField level (limited vs. intact), the eye(s) we used for data collection ('L' represents left eye, 'R' represents right eye, '-' represents the mean gaze position of two eyes)}. The last three columns represent the gaze data loss rate when reading with the three magnification modes: '-' indicates they did not use the corresponding magnification; '*' indicates the corresponding piece of data was not used in our gaze pattern analysis due to high data loss rate (over 50\%).}
\label{tab:lv_task}
  \vspace{-5ex}
\end{table*}

\textbf{\textit{Data Loss \change{(H1.2)}.}} We collected participants' gaze data during reading. The data loss rate for low vision participants is shown in Table \ref{tab:lv_task}. We compared the data loss between sighted and low vision participants, and an \change{ART} analysis showed no significant difference between the two groups
% ($W = 869$, $p = 0.97$),
\change{($F_{(1,38)} = 2.61$, $p = 0.11$, $\eta^2_{p} = 0.06$)}. \change{This indicated that the eye tracker can successfully collect similar amount of valid data from low vision people as from sighted people, verifying the feasibility of using such an eye tracker for low vision people.}  

%Therefore, it is verified our calibration and study procedure ensures that there was no bias on data quality for low vision and sighted participants. 

While no statistical difference, we found that the eye tracker lost more data from low vision participants \change{($Mean = 4.62\%, SD=6.89\%$)} than from sighted participants ($Mean = 1.35\%$, \change{$SD=1.22\%$}). The data loss was commonly caused by the eye tracker's failure to recognize the user's eyes. %Similar to prior work's finding~\cite{maus2020gaze}, 
Our observation revealed several reasons that may lead to the recognition failure: similar to Maus et al.'s finding~\cite{maus2020gaze}, some low vision participants had obscured pupils; some participants gradually moved their head closer than 65cm to the screen to read, exceeding the working range of the eye tracker; and some tilted their head to use their functional visual field as they usually did during reading, \change{making one eye unrecognizable}. Our results indicated that the eye tracker manufacturer should take low vision users' reading habits into consideration when developing future products.

\begin{figure*}[h]
  \includegraphics[width=\textwidth]{images/gazeoverlay2x-100.jpg}
  \caption{(a) Example of gaze overlay on the passage (Mark). Orange circles represent fixations (the bigger the longer duration); orange lines represent smooth pursuits; blue and pink lines represent forward saccades and backward saccades, respectively; (b) Example of searching for the next line under the lens magnifier (Mary): the participant hesitated between the second line and third line of the passage after finishing reading the first line.}
  \label{fig:overlay}
  \Description{
  Two images labeled (a) and (b) show examples of our low vision participants' gaze overlay on reading materials. Image (a) shows Mark's gaze overlay on the text when he read using the regular mode with large font. The position of fixations and saccades roughly align with the lines of the text. Image (b) shows Mary's gaze overlay on the text when she searched for the beginning of the next line using the lens magnifier. After finishing first line, Mary's fixations and regressive saccades first landed on the beginning of the third line in the magnification window, and then gradually navigated back to the beginning of the second line after realizing she was not on the correct line.
  }
    \vspace{-2ex}
\end{figure*}

\textbf{\textit{Gaze-Reading Alignment.}}
We also examined the alignment between the collected gaze trajectory and participants' reading progress to validate the gaze data. Fig \ref{fig:overlay}a shows an example of a low vision participant's gaze fixations and saccades overlaying on one passage read by him. \change{We extracted the timestamps  when participants read the last word of each line in a passage from audio recordings ($T_{\text{audio}}$) using VOSK~\cite{vosk} and compared $T_{\text{audio}}$ to the timestamps  of the return sweeps during line switchings from gaze data recordings ($T_{\text{gaze}}$). Using a Pearson's correlation test, we found that the time of each return sweep was highly correlated to the time of line switching during reading aloud for both sighted ($r(1013) = 1.00$, $p < 0.001$) and low vision ($r(286) = 1.00$, $p < 0.001$) participants.
%We fitted a linear regression model to the timestamp data. The fitted model for sighted participants is $T_{\text{gaze}} = 0.57 + 1.00 * T_{\text{audio}}$ ($R^2 = 1.00$, $F_{(1,286)}=1.94\times 10^6$, $p < 0.001$) and the model for low vision participants is $T_{\text{gaze}} = 0.60 + 1.00 * T_{\text{audio}}$ ($R^2 = 1.00$, $F_{(1,1013)}=3.98\times 10^7$, $p < 0.001$), indicating the gaze and audio data are well synchronized, with the time difference being 0.57s and 0.6s for sighted participants and low vision participants, respectively.
}


% Two researchers in our team verified that each participant's gazes on the text matches their reading progress in real time by checking the line switching eye movement and the last words they read in each line. The alignment also confirmed the validity of the eye tracking data from low vision participants. 

\subsection{Comparing Low Vision and Sighted People's Gaze Behaviors \change{(H2)}}
%We characterize low vision people's gaze behaviors during reading by comparing low vision and sighted participants' gaze patterns as well as comparing between low vision participants with different visual conditions. 

%\subsubsection{Comparing between Low Vision and Sighted People}
We compared the reading and gaze behaviors of low vision and sighted people when reading in the regular mode. %The following analysis was based on 48 recordings (24 sighted and 24 low vision). 
Similar to prior research~\cite{hallett2015reading, bruggeman2002psychophysics}, we found that low vision participants spent significantly longer time to complete a reading task than sighted controls 
%(Wilcoxon test: $W=469$, $p<0.001$), 
\change{(ART: $F_{(1,78)} = 18.56$, $p < 0.001$, $\eta^2_{p} = 0.19$ )},
with low vision participants' reading time \change{($Mean=122.88s, SD=103.94s$)} being \change{1.8} times of sighted participants' reading time \change{($Mean=68.45s, SD=19.71s$)}.   %107.58s ($SD=54.44s$) reading a passage on average, while sighted participants spent 58.54s ($SD=12.41s$), a Wilcoxon Rank-Sum one-sided test shows the difference is significant.
We looked into participants' detailed gaze behaviors from different aspects below.

\textbf{\textit{Fixation \change{(H2.1)}.}} An ART analysis showed that low vision participants %\change{($Mean=489.25$, $SD=482.53$)} 
had significantly more fixations than sighted participants %$Mean=277.83$, $SD=65.96$, 
\change{(ART: $F_{(1,78)} = 8.30$, $p = 0.005$, $\eta^2_{p} = 0.10$)}. However, their mean fixation duration 
%($Mean=0.15s$,  $SD=0.04s$) 
was significantly shorter than sighted participants %$Mean=0.20s$,  $SD=0.02s$, 
\change{(ART: $F_{(1,78)} = 57.8$, $p < 0.001$, $\eta^2_{p} = 0.43$}). This indicated that low vision people's gaze tended to fixate more frequently during active reading, but each fixation was shorter than sighted people, representing less amount of information being processed \change{per fixation} and \change{lower reading efficiency by low vision people}~\cite{moffitt1980evaluation}. 
% \todo{a visualization? Also, it might be nice to quote some qualitative experiences from participants to confirm this.}

\textbf{\textit{Sacaade \change{(H2.2)}.}} Low vision participants conducted significantly more \change{(ART: $F_{(1,78)} = 4.54$, $p = 0.04$, $\eta^2_{p} = 0.05$)} but shorter (\change{ANOVA: $F_{(1,78)} = 36.74$, $p < 0.001$, $\eta^2_{p} = 0.32$}) forward saccades %\change{($Mean=260.50$, $SD=203.69$)} 
than sighted participants. %$Mean=182.25$, $SD=41.50$,
%and their forward saccade length %(\change{$Mean=2.37$, $SD=1.02$}, normalized to letter number) 
%was significantly shorter than sighted participants %$Mean=3.67$, $SD=0.89$; 
This indicated that the perceptual span of low vision people---the number of letters taken in during a single glance---was shorter than sighted people~\cite{rubin2009role}. They thus needed to fixate and skim more times to read the same content.   
Meanwhile, we found that low vision participants demonstrated significantly more regressive saccades 
%\change{($Mean=160.35$, $SD=205.84$)} 
than sighted participants %$Mean= 76.15$, $SD=28.39$,
\change{(ART: $F_{(1,78)} = 5.50$, $p = 0.02$, $\eta^2_{p} = 0.07$)}. However, no significant difference was found in terms of revisitation rate \change{(ART: $F_{(1,78)} = 1.48$, $p = 0.23$, $\eta^2_{p} = 0.02$)}, showing that low vision people may be able to scan the text to locate the next fixation point (i.e., follow a line) as accurately as sighted people.  

\textbf{\textit{Lines Switching \change{(H2.3)}.}} Locating the next line can be a challenge for low vision people~\cite{verghese2021eye}. \change{10 out of 20} low vision participants reported they had difficulty locating the next line. We found that low vision participants searched significantly more lines %(nLS: $Mean = 0.05$,  $SD=0.06$) 
than sighted participants %$Mean = 0.02$,  $SD=0.05$,
when switching lines (\change{ART: $F_{(1,76)} = 11.64$, $p = 0.001$, $\eta^2_{p} = 0.13$}). 
% \change{Fig. \ref{fig:overlay}b shows one example of a participant hesitating between two lines.} 

We further examined how low vision people located a line. \change{We investigated the distribution of participants' fixation time (i.e., the total time of all fixations happened at a particular area) along the horizontal axis of the reading interface (Fig. \ref{fig:hist_fix})}, and found that low vision participants spent significantly longer fixation time at the first 10\% %(e.g., the length of 7.5 letters if the font size is 16px) 
of each line than sighted people (\change{ART: $F_{(1,76)} = 26.93$, $p < 0.001$, $\eta^2_{p} = 0.26$}). Specifically, low vision participants spent \change{14.87\% ($SD=4.30\%$)} of the total reading time fixating at the first 10\% of a line, while sighted participants used 
\change{11.27\% ($SD=2.34\%$)} of the total reading time. The result suggested that low vision participants spent more time locating and confirming the correct line when switching lines. According to low vision participants, a common strategy of identifying the next line was checking if the content of the new line made sense based on the context of the passage. For example, 
% Tim always remembered the first several words of a line during reading, so that he could decide which line was the correct line. 
\change{Judy always remembered the last several words of a line during reading, so that she could compare the beginning of the next lines to see which line made the most sense.} 
This explained the higher cognitive processing load (longer fixation time) observed from low vision participants at the beginning of each line during line switching.

\begin{figure*}[h]
  \includegraphics[width=0.8\textwidth]{images/fixation_time_hist_nolv17.pdf}
  \caption{\change{(a) The distribution of fixation time along the horizontal axis of the reading interface for low vision participants (mean fixation time across all participants and all passages);  (b) The distribution of fixation time along the horizontal axis of the reading interface for sighted participants.}}
  \label{fig:hist_fix}
  \Description{
  Two figures (a) and (b) show the distributions of fixation time along the horizontal axis of the reading interface for low vision and sighted participants. Figure (a) shows the fixation time density of low vision participants on the Y axis against the horizontal percentage of the text from 0 to 100\% on the X axis. There is a noticeable spike of the fixation time density within the first 10\% of the text. Figure (b) shows the fixation time density of sighted participants on the Y axis against the horizontal percentage of the text from 0 to 100\% on the X axis. The fixation time density is consistent across the horizontal axis.
  }
    \vspace{-2ex}
\end{figure*}

\subsection{Effects of Different Visual Conditions on Gaze Patterns \change{(H3)}}
% stats vwerified? yes
We studied the effect of different low vision conditions on low vision participants' gaze behaviors.

%Comparing the gaze pattern between LA and HA, and between DVF and IVF, we found similar trend as in the comparison between low vision participants and sighted participants: 
\textbf{\textit{Fixation \change{(H3.1)}.}} Our result showed that participants with low visual acuity demonstrated significantly 
% \change{longer reading time ($F_{(1,36)} = 4.88$, $p = 0.03$, $\eta^2_{p} = 0.12$)} 
more fixations (\change{ART: $F_{(1,36)} = 5.18$, $p = 0.03$, $\eta^2_{p} = 0.13$}) but shorter fixation duration (\change{ART: $F_{(1,36)} = 6.01$, $p = 0.02$, $\eta^2_{p} = 0.14$}) than those with relatively high visual acuity. \change{Similarly}, participants with limited visual field had more fixations (\change{ART: $F_{(1,36)} = 4.69$, $p = 0.04$, $\eta^2_{p} = 0.12$}) \change{ and shorter fixation duration (ART: $F_{(1,36)} = 7.88$, $p = 0.01$, $\eta^2_{p} = 0.18$)} than those with intact visual field. 
An interaction between visual field and visual acuity was found. \change{Through a \textit{post-hoc} contrast test for ART}, we found participants with visual field loss and low visual acuity had significantly shorter fixation duration than participants with low visual acuity only 
% ($p = 0.02$, 95\% C.I. = [-0.10, -0.01]), 
\change{($t_{(36)} = 3.08$, $p = 0.02$)},
participants with visual field loss only \change{($t_{(36)}  = 2.88$, $p = 0.03$)},
%($p < 0.001$, 95\% C.I. = [-0.14, -0.04]), 
and participant with both high visual acuity and intact visual field 
\change{($t_{(36)} = 3.24$, $p = 0.01$)}.
%($p < 0.01$, 95\% C.I. = [-0.10, -0.01]).
Our result suggested that \change{both low visual acuity and visual field loss lead to} reduced amount of information low vision people can perceive \change{per fixation}, and \change{experiencing visual field loss and low visual acuity at the same time} can further impact low vision people's ability to access information.
% who has low visual acuity. 

\textbf{\textit{Saccade \change{(H3.2)}.}} In terms of saccade, we found no significant effect of visual acuity \change{(ART: $F_{(1,36)} = 2.14$, $p = 0.15$, $\eta^2_{p} = 0.06$)} or \change{visual field (ART: $F_{(1,36)} = 1.12$, $p = 0.30$, $\eta^2_{p} = 0.03$)} on the number of forward saccades from low vision participants. However, we found that participants with low acuity had significantly shorter forward saccade length \change{than participants with high visual acuity (ANOVA: $F_{(1,36)} = 12.00$, $p = 0.001$, $\eta^2_{p} = 0.25$). Similarly, we found participants with limited visual field had significantly shorter forward saccade length than those with intact visual field (ANOVA: $F_{(1,36)} = 7.97$, $p = 0.01$, $\eta^2_{p} = 0.18$).} This implied that \change{low visual acuity and} limited visual field can \change{both} affect low vision participants' length of perceptual span. 
% For example, as Robin who had \change{peripheral} vision loss commented, \textit{``... When you read from line to line with the (limited) peripheral [vision]..., 
% you [relied on] a certain peripheral [area of vision] 
% when you have to go to the next word. So 
% that's why 
% it slows down the process of you being able to see what the next word is, and \change{fluently} reading through.''}
\change{For example, as Danilo explained, he could only focus on a small area at a time when reading due to his limited visual field, and since his visual acuity was low, he had to recognize letter by letter, because according to him, the letters were "mashed together".}

\change{However, we found no significant effect of either visual acuity or visual field on the number of regressive saccades (visual acuity, ART: $F_{(1,36)} = 3.35$, $p = 0.08$, $\eta^2_{p} = 0.09$; visual field, ART: $F_{(1,36)} = 2.71$, $p = 0.11$, $\eta^2_{p} = 0.07$), and revisitation rate (visual acuity, ART: $F_{(1,36)} = 1.85$, $p = 0.18$, $\eta^2_{p} = 0.05$; visual field, ART: $F_{(1,36)} = 1.83$, $p = 0.18$, $\eta^2_{p} = 0.05$).}

%and more backward saccades ($F_{(1,20)}=10.80$, $p = 0.004$) than those with high visual acuity. The results suggest that people with low visual acuity tend to have more corrective eye movements due to their difficulty with precisely locating their eyes on the next piece of text along a line.  

%Although there is no significant effect of visual acuity and visual field on the revisitation, Such significant effect was not found for visual field.

\textbf{\textit{Lines Switching \change{(H3.3)}.}} We looked into how low vision participants with different visual abilities switched lines, and found no significant effect of visual field \change{(ART: $F_{(1,34)} = 0.0002$, $p = 0.99$, $\eta^2_{p} < 0.001$)} on the number of searched lines during line switching. \change{However, there was a trend towards a significant effect of visual acuity \change{(ART: $F_{(1,34)} = 3.62$, $p = 0.07$, $\eta^2_{p} = 0.10$)} on the number of searched lines. More research is needed to further investigate the impact of visual acuity on low vision people's line switching behaviors.} 
% However, \change{we found that the interaction between visual acuity and visual field had a significant effect on the proportion of fixation time at the beginning (10\%) of each line (ART: $F_{(1,36)} = 9.12$, $p = 0.005$, $\eta^2_{p} = 0.20$), although there was no significant effect of either visual acuity (ART: $F_{(1,36)} = 2.78$, $p = 0.10$, $\eta^2_{p} = 0.07$) or visual field (ART: $F_{(1,36)} = 0.64$, $p = 0.43$, $\eta^2_{p} = 0.02$) was found. Using a \textit{post-hoc} contrast test, we found that among all participants with intact visual field, low visual acuity led to significantly longer fixation time on the first 10\% of each line ($t_{(36)} = -3.78$, $p = 0.003$). For participants with low visual acuity, limited visual field led to longer fixation time on the first 10\% of each line ($t_{(36)} = 2.93$, $p = 0.03$).} \yuhang{need explaination} \todo{any quote from visual field loss people can interpret this?}

% participants with low acuity  fixated their gaze ($Mean = 0.17$,  $SD = 0.05$) longer than those with high acuity ($Mean=0.13$, $SD=0.03$, ANOVA: $F_{(1,20)}=5.75$, $p = 0.03$), meaning low visual acuity can cost low vision people longer time to process and identify the correct line. Some participants with low visual acuity (Bella, Tim, Judy, and Mary) reported difficulty recognizing words even with large font size, which could potentially complicate the line switching process. 

\subsection{Gaze Behaviors under Different Magnification Modes \change{(H4)}} 
% stats verified? yes
 We compared the reading performance and gaze patterns of low vision participants when reading with different magnification modes. To start off, it was not surprising that magnification modes had significant effect on participants' reading time \change{(ART: $F_{(2,26)} = 16.45$, $p < 0.001$, $\eta^2_{p} = 0.56$)}. A \textit{post-hoc} contrast test for ART showed that, participants read much slower using the lens magnifier \change{($t_{(26)} = -5.31$, $p < 0.001$)} or full-screen magnifier \change{($t_{(26)} = -4.53$, $p < 0.001$)} than using the regular mode with increased font size. 
 \change{Nine} participants mentioned that the \change{hand-eye} coordination required by screen magnifiers made reading more difficult. 
 One participant (Bella) told us that she read slower on purpose to make the lens magnifier move stably. No significant difference in reading time was found between the lens magnifier and full-screen magnifier \change{($t_{(26)} = 0.79$, $p = 0.71$)}. 
 
 \textbf{\textit{Fixation \change{(H4.1)}.}} Although there was no significant effect of magnification modes on the number of fixations \change{(ART: $F_{(2,26)} = 0.03$, $p = 0.97$, $\eta^2_{p} = 0.003$)}, we found a significant effect of magnification modes on the mean fixation duration \change{(ART: $F_{(2,55)} = 38.60$, $p < 0.001$, $\eta^2_{p} = 0.58$)}. A \textit{post-hoc} contrast test showed that the lens magnifier and full-screen magnifier both led to significantly shorter fixation duration than the regular mode with increased font size \change{(lens: $t_{(55)} = 8.61$, $p < 0.001$; full-screen: $t_{(55)} = 5.78$, $p < 0.001$). Moreover, participants showed significantly shorter fixation duration when using the lens magnifier than using the full-screen magnifier ($t_{(55)} = -2.84$, $p = 0.02$)}. One possible explanation was that the dynamic changing of text layout (due to the lens movement and the panning) made it more difficult for low vision participants to focus and fixate on individual words, leading to shorter fixation duration. 
 \change{Eight} participants 
 %(Mark, Judy, Hannah and Tim) \todo{check the number and add more quotes} 
 complained that the control of the magnifiers made it harder to keep track of their position, \change{especially for the lens magnifier. Due to the high magnification level, a small deviation on the position of the magnification window would cause a whole line to disappear. 
 % As Judy commented, \textit{``I moved the mouse a little too fast sometimes for my eyes to catch up.''}  
 As Diego commented, \textit{``I drop [the magnification window] down [a little] too far like that, then it can throw me off from one end [of the window] to the other.''}}
 % No significant difference was found between lens magnifier and full-screen magnifier in terms of mean fixation duration \change{($t_{(35)} = -2.84$, $p = 0.02$). 
 
 
 % Although different magnification modes had significant effect on the reading time ($F(2,12)=5.82$, $p = 0.02$), no significant difference was found between each pair of magnification modes using a pairwise t-test with Bonferroni correction. We found there is no significant effect of magnification modes on number of fixations ($F(2,12)=0.13$, $p = 0.88$). Our result shows that magnification modes have significant effect on average fixation duration ($F(2,12)=6.72$, $p = 0.01$), however, no significant difference was found between each pair of magnification modes. The above results indicate that the amount of information participants could access at a time does not vary using different magnification modes.

\textbf{\textit{Saccades \change{(H4.2)}.}} There was no significant effect of magnification modes on the number of forward saccades (\change{ART: $F_{(2,26)} = 3.07$, $p = 0.06$, $\eta^2_{p} = 0.19$}) and mean forward saccade length (\change{ART: $F_{(2,26)} = 1.34$, $p = 0.28$, $\eta^2_{p} = 0.09$}). \change{However, a significant effect was found on the number of regressive saccades (ART: $F_{(2,26)} = 6.19$, $p = 0.01$, $\eta^2_{p} = 0.32$). Through a \textit{post-hoc} contrast test, we found that low vision participants had more regressive saccades when using the lens magnifier \change{($t_{(26)} = -2.85$, $p = 0.02$)} and full-screen magnifier \change{($t_{(26)} = -3.22$, $p = 0.01$)} than using the regular mode.} 
%However, no sifgnificant difference was found between lens magnifier and full-screen magnifier($t{(29)} = -0.48$, $p = 0.88$). The result suggests that \todo{add participants' quote and explanation}.
Furthermore, we found magnification modes had a significant effect on revisitation rate \change{(ART: $F_{(2,26)} = 6.62$, $p = 0.005$, $\eta^2_{p} = 0.34$)}. Using a \change{\textit{post-hoc} contrast test}, we found participants showed higher revisitation rate using the lens magnifier \change{($t_{(26)} = -3.19$, $p = 0.01$)} and the full-screen magnifier \change{($t_{(26)} = -3.12$, $p = 0.01$)} than using the regular mode with increased font size. No significant difference in the number of regressive saccades \change{($t_{(26)} = -0.37$, $p = 0.93$)} and revisitation rate (\change{$t_{(26)} = 0.06$, $p = 1.00$}) was found between the lens and full-screen magnifier.
This could be explained by participants' difficulty of keeping track of their reading positions when using screen magnifiers (Hannah, Judy, Bella, Lucy, Mark and \change{May}). One strategy reported by participants was to go back and check if they had read the content to locate where they were reading, \change{thus leading to more regressive saccades and revisitations}.

\change{For the lens magnifier, we observed that \change{12} low vision participants requested to increase the width of the magnification window.} We thus investigated how the size of the magnification window affected participants' gaze behaviors. We normalized the window width by calculating the number of letters that can be horizontally covered by the window. We then examined the relationship between the mean forward saccade length and the normalized window width. Using a \change{Pearson's correlation} test, we found a positive correlation between the two variables (\change{$r(26) = 0.72$, $p < 0.001$}). We further found a negative correlation between reading time and the normalized window width \change{($r(26) = -0.67$, 
$p < 0.001$), which echoed the result in Legge et al.~\cite{legge1985psychophysics}.} Combined with the prior finding that reading speed was correlated with forward saccade length~\cite{rubin2009role}, our result suggested that a wider magnification window can potentially benefit reading speed by improving perceptual span.

% can further explain shorter forward saccade length in lens magnifier than full-screen magnifier. The window bottlenecked the number of letters participants could see at a time.

% However, we found participants used more forward saccades ($t_{13} = -2.65$, $p= 0.02$) and regressive saccades ($t_{13} = -2.77$, $p= 0.02$) when using full-screen magnifier compared to the regular mode. In terms of forward saccade length, we found a significant difference between lens magnifier and full-screen magnifier ($V = 20$, $p = 0.04$): The forward saccade length when using the lens magnifier ($Mean = 1.92 letters, SD= 0.85$) was significantly shorter than when using the full-screen magnifier ($Mean = 2.18 letters, SD=1.11$). Since full screen magnifier can be considered as a lens magnifier with a full screen sized window, 

% Interestingly, we found participants showed higher revisitation rate using screen magnifiers (lens: $Mean = 0.33$, $SD=0.11$, full-screen: $Mean = 0.33$, $SD=0.12$ ) than the regular mode with increased font size ($Mean = 0.28$, $SD=0.14$). \todo{add participants' explanation}.

\textbf{\textit{Lines Switching \change{(H4.3)}.}} We found that the effect of magnification modes on the number of searched lines during line switching was significant \change{(ART: $F_{(2,26)} = 5.70$, $p = 0.01$, $\eta^2_{p} = 0.30$)}. A \textit{post-hoc} contrast test showed that participants searched significantly more lines to locate the next line when using the lens magnifier than using the regular mode \change{($t_{(26)} = -3.35$, $p = 0.01$)}. %\change{No significant difference was found between regular mode and full-screen magnifier ($t_{(26)} = -1.32$, $p = 0.40$), 
%and between lens-magnifier and full-screen magnifier ($t_{(26)} = 2.03$, $p = 0.13$).} 
% Participants using full-screen magnifier searched more lines than regular mode with increased font size ($V = 71$, $p = 0.03$). 
\change{Fig \ref{fig:overlay}b shows an example where the participant hesitated between the second and third line before landing on the correct line using the lens magnifier.} \change{Six} participants (Judy, Mark, Tim, \change{May, Marilyn and Diego}) explicitly mentioned that the small magnification window in the lens magnifier caused the challenge of locating the next line. Judy further explained that she could easily move the magnifier out of the text area to the white margin on the left because of limited context information presented in the magnification window. To successfully locate the next line, \change{five participants (Judy, Mark, Marilyn, Diego and Julia)} carefully moved the lens magnifier one line down and then back tracked to find the correct location. 

Moreover, for the lens magnifier, we found a negative correlation between the number of searched lines and the normalized height of the window (i.e., the number of lines covered by the window at a time) with Pearson's correlation test \change{($r(26) = -0.60$, $p < 0.001$)}. This meant that the more lines that can be viewed within the magnification window, the fewer lines participants needed to search to locate the correct lines. However, not all participants preferred a taller magnification window. Three participants (Bella, Mary and Lucy) preferred a shorter window since too much content around the current line can be distracting. As Mary commented, \textit{``In some ways [a shorter lens magnifier] was easier because you didn't have so many possibilities [of lines]. Your options were limited to what you needed to choose.''}

When asked about potential improvements on the magnification tools, four participants (Hannah, Judy, Lucy and Robin) suggested highlighting the next line and labeling the index of each line at the beginning to assist with line switching. To reduce the effect of `panning around' on line switching, \change{seven} participants suggested using a larger screen to present all enlarged text, or breaking the text into pages and allowing page flipping (Bella and Tim). As Tim commented, \textit{``The more I have to [move the screen magnifier], the more I'm going to lose my place.''} %Hannah also mentioned smoother movement of the window is desired. 
Furthermore, two participants (Lucy and Tim) suggested marking where their eyes stopped if they look away from the screen to prevent them from losing track of where they were. 

\textbf{\textit{Smooth Pursuits.}} Unlike reading static text, when moving the text by either scrolling, panning, or moving the magnification window, smooth pursuits happened. Our analysis showed that magnification modes had a significant effect on the number \change{(ART: $F_{(2,55)} = 88.08$, $p < 0.001$, $\eta^2_{p} = 0.76$)} and mean duration \change{(ANOVA: $F_{(2,26)} = 24.38$, $p < 0.001$, $\eta^2_{p} = 0.65$)} of smooth pursuits.
\change{\textit{Post-hoc} comparisons showed that} when low vision participants used the lens magnifier or the full-screen magnifier, the number of smooth pursuits was significantly higher than the regular mode \change{(lens: $t_{(55)} = -12.50$, $p < 0.001$; full-screen: $t_{(55)} = -9.98$, $p < 0.001$)}. Moreover, \change{participants conducted more smooth pursuits when using the lens magnifier than the full-screen magnifier ($t_{(55)} = 2.53$, $p = 0.04$).} 
The mean duration of smooth pursuits when using the two screen magnifiers was also significantly longer than using the regular mode \change{(lens: $t_{(26)} = -6.58$, $p < 0.001$; full-screen: $t_{(26)}= -5.32$, $p < 0.001$)}.
However, there was no significant difference in the duration of smooth pursuits between the two magnifiers  (\change{$t_{(26)} = 1.26$, $p = 0.43$}). Due to the higher number and longer duration of smooth pursuits, participants (Mary and Hannah) felt more tired when using the \change{lens} magnifier than \change{the other two magnification modes} since they had to constantly track the position of the text in the moving window. As Hannah complained, \textit{``The window is always moving, searching for the right position makes my eyes tired.''}
%make it easier to get tired when using lens and full-screen magnifiers. 
