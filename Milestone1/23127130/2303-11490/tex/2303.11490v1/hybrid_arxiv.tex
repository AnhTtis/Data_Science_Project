\documentclass{article}
\usepackage[utf8]{inputenc}
\usepackage[margin=1in]{geometry}
\usepackage{amsmath} % assumes amsmathpackage installed
\usepackage{amssymb} % assumes amsmathpackage installed
\usepackage{mathtools}
\usepackage{acronym}
\usepackage{mathrsfs}

\usepackage{subcaption}
\usepackage{hyperref}
\hypersetup{colorlinks=true, linkcolor=black}
\usepackage{algorithm}%
\usepackage{algorithmicx}%
\usepackage{algpseudocode}%

\newcommand{\rev}[1]{#1}


%Single Symbols
\newcommand{\Rn}{\mathbb{R}} 
\newcommand{\Kn}{\mathbb{K}} 
\newcommand{\Cn}{\mathbb{C}} 
\newcommand{\Nn}{\mathbb{N}}
% \newcommand{\D}{\mathbb{D}}
\newcommand{\Lie}{\mathcal{L}} 
\newcommand{\scL}{\mathscr{L}} 
\newcommand{\ks}{\mathcal{K}} 
% \newcommand{\Z}{\mathbb{Z}}
\newcommand{\Zt}{\mathcal{Z}}
\newcommand{\s}{\mathcal{S}}
\newcommand{\M}{\mathbb{M}}

% \newcommand{\A}{\mathcal{A}}
\newcommand{\0}{\mathbf{0}}
\newcommand{\1}{\mathbf{1}}
\newcommand{\psd}{\mathbb{S}}
\newcommand{\gs}{\mathcal{G}}
\newcommand{\es}{\mathcal{E}}
\newcommand{\hs}{\mathcal{H}}
\newcommand{\vs}{\mathcal{V}}


%Paired Commands
\usepackage{mathtools}
\DeclarePairedDelimiter{\abs}{\lvert}{\rvert}
\DeclarePairedDelimiter{\norm}{\lVert}{\rVert}
\DeclarePairedDelimiter{\ceil}{\lceil}{\rceil}
\DeclarePairedDelimiter{\Tr}{\textrm{Tr}(}{)}
\DeclarePairedDelimiter{\rank}{\textrm{rank}(}{)}
\DeclarePairedDelimiter{\diag}{\textrm{diag}(}{)}
\DeclarePairedDelimiter{\vvec}{\textrm{vec}(}{)}
\DeclarePairedDelimiter{\mmat}{\textrm{mat}(}{)}
\DeclarePairedDelimiterX{\inp}[2]{\langle}{\rangle}{#1, #2}
\DeclarePairedDelimiter{\supp}{\textrm{supp}(}{)}
\DeclarePairedDelimiter{\Mp}{\mathcal{M}_+(}{)}

%Operators and problems
\DeclareMathOperator*{\argmin}{arg\!\,min}
\DeclareMathOperator*{\argmax}{arg\!\,max}

%theorem environments

\usepackage{amsthm}
\newtheorem{theorem}{Theorem}
\newtheorem{proposition}[theorem]{Proposition}% 
\newtheorem{assumption}[theorem]{Assumption}% 
\newtheorem{remark}{Remark} 
\newtheorem{problem}{Problem}


\title{\LARGE \bf
Peak Estimation of Hybrid Systems 
\\ with Convex Optimization
}
\renewcommand\footnotemark{}
\renewcommand\footnoterule{}
\author{Jared Miller$^1$, Mario Sznaier$^1$
\thanks{$^1$J. Miller and M. Sznaier are with the Robust Systems Lab,  ECE Department, Northeastern University, Boston, MA 02115. (e-mails: miller.jare@northeastern.edu,  msznaier@coe.neu.edu).}
%Don't know if these are still valid
\thanks{ J. Miller and M. Sznaier were partially supported by NSF grants  CNS--1646121, CMMI--1638234, ECCS--1808381 and CNS--2038493,  and AFOSR grant FA9550-19-1-0005. 
This material is based upon research supported by the Chateaubriand Fellowship of the Office for Science \& Technology of the Embassy of France in the United States.
}
}
\date{\empty}
\begin{document}

\maketitle

\abstract{Peak estimation of hybrid systems aims to upper bound extreme values of a state function along trajectories, where this state function could be different in each subsystem. 
This finite-dimensional but nonconvex problem may be lifted into an  infinite-dimensional linear program (LP) in occupation measures with an equal objective under mild finiteness/compactness and smoothness assumptions. This LP may in turn be approximated by a convergent sequence of upper bounds attained from solutions of Linear Matrix Inequalities (LMIs) using the Moment-Sum-of-Squares hierarchy. The peak estimation problem is extended to problems with uncertainty and safety settings, such as measuring the distance of closest approach between points along hybrid system trajectories and unsafe sets.
    
    % \urg{More content in abstract. Mention extensions (e.g. safety, uncertainty, distance)?}
    
    % \urg{Springer desires a single file (\texttt{hybrid\_mcss.tex}) and strongly discourages the use of the \texttt{\\input} command. This will result in some messy editing.}
    
    % Safety of hybrid systems with respect to unsafe sets may be quantified with maximin peak estimation through the generalizaton of safety margins.
    % Uncertain dynamics may be incorporated into the hybrid system peak estimation problem, and the hybrid formulation can implement time-dependent disturbance processes with bounded derivative constraints. 
    % The solved bounds can be sharpened by decomposing the hybrid systems subsystems into smaller units with trivial reset maps, forming a partition-based approximation that can yield peak bounds at lower relaxation degrees.
}

%algorithm modifications

\algblock{Input}{EndInput}
\algnotext{EndInput}
\algblock{Output}{EndOutput}
\algnotext{EndOutput}
\newcommand{\Desc}[2]{\State \makebox[2em][l]{#1}#2}


% \keywords{Peak Estimation, Hybrid Systems, Occupation Measures, Linear Matrix Inequalities}

\maketitle


\section{Introduction}
\label{sec:intro}
% \urg{
% The destination of this paper will be in the Springer journal "Mathematics of Control, Signals, and Systems" as part of the soliciation of a special edition for Eduardo Sontag's 70th birthday.
% We are allowed up to 20 pages in this format (Springer Nature Template)
% }

This paper interprets and extends the peak estimation problem to dynamical systems with hybrid behavior. Peak estimation is the analysis problem of finding extremal values of state functions along system trajectories, such as finding the maximum speed of a craft given a set of initial conditions. 
A hybrid system is a dynamical system that possesses both continuous-time and discrete-time  dynamics \cite{goebel2009hybrid}. Hybrid systems have a wide array of applications, including walking robots \cite{collins2005efficient}, power converters \cite{zaupa2021resonant}, sampled-data control \cite{naghshtabrizi2006sampled}, and systems biology \cite{fischer2012redbloodcell}.
In this work (extending methods from \cite{zhao2019optimal}), the hybrid system is defined with respect to a series of spaces known as `locations' in which the hybrid trajectory evolves according to per-location \ac{ODE} dynamics. When the hybrid trajectory encounters a guard surface, it will transition to a (possibly) new location according to a reset map and continue its \ac{ODE} evolution. Peak estimation of hybrid systems equips each location with a state function, and the output of the peak estimation problem is the maximum state function value obtained across all locations by all hybrid systems trajectories starting from a set of initial conditions in a given time horizon. 

% [Main pitch of the paper. What is peak estimation and how can it be applied to hybrid systems?]

% [Prior Work: Hybrid System OCP \cite{van2000introduction}]

The \ac{ODE} peak estimation problem is an instance of an input-less \ac{OCP} with a free terminal time and zero running cost.
% , and is also a deterministic optimal stopping problem \cite{ferguson2006optimal, hill2009knowing}. 
In the \ac{ODE} case, such \acp{OCP} are finite-dimensional but generally nonconvex to solve. This difficulty is exacerbated by the addition of hybrid dynamics. In both cases for a maximization objective, lower bounds on the true peak cost can be computed by approximate sampling, while upper bounds must be satisfied for all admissible trajectories. 
The foundational work in \cite{lewis1980relaxation} represented \ac{ODE} \acp{OCP} as infinite-dimensional \acp{LP} in nonnegative Borel Measures, and gave necessary conditions under which the \ac{OCP} and its \ac{LP} outer-approximation have the same optimal value (no relaxation gap). The \acp{LP} in \cite{lewis1980relaxation} involve terminal measures and occupation measures, which describe all information needed to reconstruct families of trajectories \ac{ODE}.
The work in \cite{cho2002linear} treated the peak estimation problem as an infinite-dimensional \ac{LP} in measures, and proposed approximations solved based on successively refined finite-dimensional gridded \acp{LP}. In the peak estimation setting, the necessary conditions for no relaxation gap are mild compactness and Lipschitz regularity requirements. Another approach to solving infinite-dimensional \acp{LP} is through the Moment-\ac{SOS} hierarchy, which truncates the \acp{LP} into \acp{SDP} of increasing complexity to produce tightening outer approximations  \cite{lasserre2009moments, tacchi2022convergence}. The \ac{SOS} side of this hierarchy was used in \cite{fantuzzi2020bounding} to solve peak estimation problems. Other applications of the Moment-\ac{SOS} hierarchy in analysis and control of \acp{ODE} includes \acp{OCP} \cite{lasserre2008nonlinear}, reachable set estimation \cite{henrion2013convex}, and maximum control invariant set estimation \cite{korda2014convex}.
% The Moment-\ac{SOS} hierarchy is a method to produce upper bounds to measure \acp{LP} by a sequence of Linear Matrix Inequalities (\acp{SDP}) of increasing size \cite{lasserre2009moments}.
% [Prior Work: Measures and Peak Estimation \cite{lewis1980relaxation} \cite{cho2002linear} \cite{fantuzzi2020bounding} \cite{lasserre2008nonlinear} \cite{lasserre2009moments} \cite{henrion2003gloptipoly} \cite{henrion2013convex}]

Measures and the Moment-\ac{SOS} hierarchy have also been applied to solve problems featuring hybrid dynamical systems. Instances of these extensions include \acp{OCP} \cite{zhao2019optimal, zhao2019switching, rocca2018bio} and reachable sets \cite{shia2014reachable, mohan2016convex}.
% We note that \cite{piunovskiy2022gradual} treats the stochastic case for control of a continuous-time Markov Decision Process with gradual (\ac{ODE}) and impulsive (jump) inputs. 
Barrier functions to certify safety of hybrid system trajectories with respect to unsafe sets may also be found by \ac{SOS} programming \cite{prajna2004safety}.

% [Prior Work: Measures for Hybrid Systems \cite{shia2014reachable} \cite{zhao2019optimal} \cite{zhao2019switching} \cite{han2019controller} \cite{rocca2018bio} \cite{mohan2016convex} ]

% [Prior Work: Safety, including Hybrid Systems.  \cite{rantzer2004analysis}]

% Our contributions
The contributions of this paper are as follows:
\begin{itemize}
    \item Application of measure techniques to peak estimation for hybrid systems
    \item Use of Zeno caps to prevent unbounded executions
    \item A modular MATLAB framework for posing peak estimation problems for hybrid systems
    \item Extensions of existing work on peak estimation for uncertainty and safety analysis to hybrid systems
\end{itemize}


The paper is organized as follows. Section \ref{sec:prelim} introduces the notation convention and reviews background information. Section \ref{sec:hybrid_peak} formulates an infinite-dimensional measure program for peak estimation of hybrid system and its associated \ac{LMI} relaxation. Section \ref{sec:extensions} extends the hybrid peak estimation framework to safety analysis and possibly uncertain dynamical systems. Numerical examples are presented in Section \ref{sec:examples}. The paper is concluded in Section \ref{sec:conclusion}.

\section{Preliminaries}
\label{sec:prelim}

\begin{acronym}[WSOS]
% \acro{DDC}{Data Driven Control}

% \acro{BSA}{Basic Semialgebraic}
% \acro{GAS}{Globally Asymptotically Stable}

% \acro{CSP}{Correlative Sparsity Pattern}

\acro{LMI}{Linear Matrix Inequality}
\acroplural{LMI}[LMIs]{Linear Matrix Inequalities}
\acroindefinite{LMI}{an}{a}

% \acro{LQR}{Linear Quadratic Regulator}
% \acroplural{LMI}[\acp{LMI}]{Linear Matrix Inequalities}
% \acroindefinite{LQR}{an}{a}


\acro{LP}{Linear Program}
\acroindefinite{LP}{an}{a}

\acro{OCP}{Optimal Control Problem}
\acroindefinite{OCP}{an}{a}

\acro{ODE}{Ordinary Differential Equation}
\acroindefinite{ODE}{an}{a}

% \acro{POP}{Polynomial Optimization Problem}

\acro{PSD}{Positive Semidefinite}

% \acro{PD}{Positive Definite}

\acro{SDP}{Semidefinite Program}
\acroindefinite{SDP}{an}{a}
% \acro{SIR}{Susceptible, Infected, Removed}

\acro{SOS}{Sum of Squares}
\acroindefinite{SOS}{an}{a}

% \acro{WSOS}{Weighted Sum of Squares}


\end{acronym}


\subsection{Notation}
\label{sec:notation}
% fields, sets, polynomials, measures, indicator functions.

% \urg{Write this}
% The $n$-dimensional euclidean space is denoted as $\Rn^n$, and the set of natural numbers

The set of real numbers is $\Rn$ and of natural numbers is $\Nn$. The set of polynomials with real coefficients in indeterminates $x$ is $\Rn[x]$. Every polynomial $g \in \Rn[x]$ may be uniquely expressed as $g = \sum_{\alpha \in \Nn^n} g_\alpha x^\alpha$ in multi-index notation $x^\alpha = x_1^{\alpha_1} x_2^{\alpha_2}\ldots $ for some finite number of nonzero coefficients $g_\alpha$. The degree of a monomial $x^\alpha$ is $\abs{\alpha} = \sum_i \alpha_i$, and the degree of a polynomial $g$ is the maximum such $\abs{\alpha}$ where $g_\alpha \neq 0$. 

A nonnegative Borel measure supported in a set $X$ is a function that assigns each element of the $\sigma$-algebra of sets over $X$ with a nonnegative number (the `size' or `measure' of the set). The measure $\mu$ follows the rules $\mu(\varnothing) = 0$ and $\mu(A \cup B) = \mu(A) + \mu(B)$ if $A \cap B = \varnothing$ \cite{tao2011introduction}. The support of a nonnegative Borel measure is the locus of points $x$ where every open neighborhood $N(x)$ has $\mu(N(x)) > 0$. The set of all nonnegative Borel measures supported in $X$ is $\Mp{X}$. The space of continuous functions is $C(X)$, and a pairing between a nonnegative measure $\mu \in \Mp{X}$ and a function $f \in C(X)$ may be defined by Lebesgue integration $\inp{f}{\mu} = \int f d \mu = \int_X f(x) d \mu(x)$. The set $C^1(X) \subset C(X)$ is the set of continuous functions with continuous first derivatives. 
The indicator function $I_A$ of a set $A \subseteq X$ takes on the value $I_A(x) = 1$ if $x \in A$ and $I_A(x)=0$ otherwise, and satisfies the rule $\inp{I_A}{\mu} = \mu(A)$ for all $A \subseteq X$.
The mass of a measure $\mu$ is $\mu(X) = \inp{1}{\mu}$, and $\mu$ is a probability measure if $\inp{1}{\mu} = 1$. 
The Dirac delta $\delta_{x = x'} \in \Mp{X}$ is a probability measure supported only at the point $x=x'$, which follows the pairing rule $\inp{f}{\delta_{x=x'}} = f(x')$ for all test functions $f \in C(X)$. A nonnegative Borel measure supported at $r$ distinct points (atoms) is termed a rank-$r$ atomic measure. Such an atomic measure may be formed by a conic combination of Dirac deltas.
The projection  $\pi^x: X \times Y \rightarrow X$ is the map $(x,y) \mapsto x$. Given a mapping $Q: X \rightarrow Y$ and a measure $\mu(x) \in \Mp{X}$, the pushforward $Q_\# \mu(y)$ is the unique measure satisfying $\forall f \in C(Y): \inp{f(Q(x))}{\mu(x)} = \inp{f(y)}{Q_\# \mu(y)}.$

\subsection{Hybrid Systems}
\label{sec:hybrid}

The hybrid systems in this paper are posed over a set of $L$ locations. Each location $\ell = 1..L$ has state variables $x_\ell$ contained in the space $X^\ell \subseteq \Rn^{n_\ell}$. The subsystems obey nominal locally Lipschitz dynamics $f_\ell$ that satisfy
\begin{align}
    \label{eq:dynamics_nominal}\dot{x}_\ell(t) &= f_\ell(t, x_\ell(t)) & & \forall \ell = 1..L.
\end{align}

Available transitions between subsystems may be represented by a directed multigraph. A multigraph is a graph where pairs of vertices may be connected by multiple distinct edges \cite{hartsfield2013pearls}. Let $\gs = (\vs, \es)$ be a multigraph where each of the $L$ vertices of $\vs$ corresponds to a location. 
% Each edge $e \in \es \subset \vs \times \vs$ with  is a directed arc representing a transition between location $\ell$ and location $\ell'$. Self loops with $\ell = \ell'$ are permitted in this class of multigraphs. Edges $e \in \es$ are labeled with a guard $S^e$ and a reset map $R^e$.
Each edge $e \in \es \subset \vs \times \vs$ 
is a directed arc from a source $\textrm{src}(e)$ to a destination $\textrm{dst}(e)$. Self-loops with $\textrm{src}(e) = \textrm{dst}(e)$ are permitted in this class of multigraphs.
Edges $e$ are associated with a guard $S_e$ and a reset map $R_e$. The guard $S_e$ is a subset of $X_{\textrm{src}(e)}$, and the reset map $R_e: X_{\textrm{src}(e)} \rightarrow X_{\textrm{dst}(e)}$ effects the transition.
% \urg{The topology of the underlying space may be described by quotient mapping. Define the space $\Lambda_e =\left\{ R^e(x_{\textrm{src}(e)}) = x_\textrm{dst}(e), \ x_{\textrm{src}(e)} \in S^e, \ x_\textrm{dst}(e) \in X_{\textrm{dst}(e)} \right\}$ as the overlaps of transition $e$.
%  This space is,
% \begin{equation}
%     \bigsqcup_{\ell=1}^L X_\ell / \prod_{e \in \es} \Lambda_e
% \end{equation}
% Refer to \cite{burden2015metrization} for more details, may not be necessary.}
The hybrid system is fully encoded by the tuple $\hs = (X, f, \gs, S, R)$ with attributes:
\begin{align}
    X &= \{X_\ell\}_{\ell=1}^L & & \textrm{State Spaces}\nonumber \\
    f &=\{f_\ell\}_{\ell=1}^L& &  \textrm{Dynamics}\nonumber \\
    \gs &= (\vs, \es)& & \textrm{Transition Multigraph}\nonumber \\
    S &= \{S_e\}_{e \in \es}& & \textrm{Guard Surfaces}\nonumber \\
    R &= \{R_e\}_{e \in \es} & & \textrm{Reset Maps}\nonumber
\end{align}
Execution of a hybrid system with multigraph transitions is based on Algorithm 1 of \cite{shia2014reachable}. An additional input is a set of Zeno caps $\{N_e\}_{e \in \es}$ which halt trajectory execution if any transition $e$ is traversed at least $N_e$ times \cite{ames2006there}.
The output of the following Algorithm \ref{alg:hybrid_exec} is a system trajectory $x(t)$, as well as records $\mathcal{T}, \mathcal{C}$ containing information about the times and locations of state transitions respectively. 
% \SetKwFor{Loop}{Loop}{}{EndLoop}
% \begin{algorithm}[ht]
%          \caption{\label{alg:hybrid_exec}Execution of Hybrid System $\hs$}
%          \textbf{Input :} Initial location $\ell_0$, initial point $x_0 \in X_\ell$, Hybrid System $\hs$, Maximal time $T$, Transition Caps $N$\\ 
%          \textbf{Output :} Trajectory of system $x(t)$, Time breaks $\mathcal{T}$, Location breaks $\mathcal{C}$\\
%          $t = 0, \ \ell = \ell_0, \ x(0) = x_0, \mathcal{T} = \{0\}, \mathcal{C} = \{\ell\}, \mathcal{N} = \{0\}_{e \in \es}$\\
%          \Loop{}{
%             Follow dynamics $x'(t) = f_\ell(t, x(t))$ until $x(t)$ reaches a guard or $t=T$.\\
%             % Let $I$ be an interval bounded below by $t$ of maximal size, and $\tilde{x}: I \rightarrow X_\ell$ be an absolutely continuous function satisfying, \\
%             %     $\qquad $a) $\tilde{x}'(\theta) = f_\ell(\theta, \tilde{x}(\theta))$\\
%             %     $\qquad $b) $x(t) = \tilde{x}(t)$ \\
%             % $\bar{t}$ is a maximal time such that $I = [t, \bar{t}]$, and $x(\theta) = \tilde{x}(\theta) \  \forall \theta \in I$ \urg{I don't like this definition} \\
%             \uIf{$t = T$, there does not exist a $S_e$ with $x(\bar{t}) \in S_e$ and $\textrm{src}(e) = \ell$, or there exists an $e$ with $\mathcal{N}_e = N_e$} {
%                 \textbf{halt}
%             }
%             Find a guard $S_e$ with $x(\bar{t}) \in S_e$ and $\textrm{src}(e) = \ell$ \\
%             Append $t$ to $\mathcal{T}$ and $\textrm{dst}(e)$ to $\mathcal{C}$ \\
%             Increment $\mathcal{N}_e \leftarrow \mathcal{N}_e + 1$ \\
%             Transition to $\ell \leftarrow \textrm{dst}(e), \  x(t) \leftarrow R_e(x(t))$
%          }
%         \end{algorithm}
\begin{algorithm}
 \caption{\label{alg:hybrid_exec}Execution of Hybrid System $\hs$}
        \begin{algorithmic}
  \Input
  \Desc{$x_0$}{Initial Point}
  \Desc{$\ell_0$}{Initial Location}
  \Desc{$\hs$}{Hybrid System}
  \Desc{$T$}{Maximal Time}
  \Desc{$N$}{Transition Caps}
  \EndInput
  \Output
  \Desc{$x(t)$}{Trajectory of System}
  \Desc{$\mathcal{T}$}{Time Breaks}
  \Desc{$\mathcal{C}$}{Location Breaks}
  \Desc{$\mathcal{N}$}{Transition Counts}
  \EndOutput
\State Initialize Trajectory    $t \leftarrow 0, \ \ell \leftarrow \ell_0, \ x(0) \leftarrow x_0$
\State Initialize Traces $\mathcal{T} \leftarrow \{0\}, \mathcal{C} \leftarrow \{\ell\}, \mathcal{N} \leftarrow \{0\}_{e \in \es}$
\Loop
\State Follow dynamics $x'(s) = f_\ell(t, x(s))$ until $x(t)$ reaches a guard or $t=T$.
\If{$t = T$ \textbf{OR} $\not\exists S_e: \ x(t) \in S_e$ and $\textrm{src}(e) = \ell$, \textbf{OR}  $\exists e: \ \mathcal{N}_e = N_e$} 
        \State \textbf{halt}
            \EndIf
\State Find a guard $S_e$ with $x(t) \in S_e$ and $\textrm{src}(e) = \ell$ 
\State Append $t$ to $\mathcal{T}$ and $\textrm{dst}(e)$ to $\mathcal{C}$
\State Increment $\mathcal{N}_e \leftarrow \mathcal{N}_e + 1$ 
\State  Transition to $\ell \leftarrow \textrm{dst}(e), \  x(t) \leftarrow R_e(x(t))$
\EndLoop
\end{algorithmic}
\end{algorithm}
        
        % \urg{Should add a Tikz drawing of a multigraph with labeled transitions. Two nodes (1, 2) with arcs (1,1), (1,1), (1,2), (2,1)}
The trajectory $x(t)$ is well-defined when the time horizon $T$ and Zeno caps $N_e$ for all $e \in \es$ are finite and the guard surfaces $S_e$ are codimension-1. The trajectory $x(t)$ induces a function $\textrm{Loc}: [0, T] \rightarrow 1..L$ which returns the residing location of $x(t)$ at time $t$. Execution requires the following assumption of transversality,
\begin{assumption}
Let $x_\ell(t)$ be a segment of this trajectory that emerged from a transition $(\ell', \ell)$ at time $t^-$. For all guards $S^e$ with $\textrm{src}(e) = \ell$ such that $x_\ell(t) \in S^e$, the dynamics vector $f(t, x_\ell(t))$ possesses a normal component with respect to the tangent space of $S^e$ at $x_\ell(t)$. This implies that the time elapsed between any two resets is bounded below by some $\delta > 0$. 
\end{assumption}

\subsection{Peak Estimation and Occupation Measures}
\label{sec:peak}

The \ac{ODE} (non-hybrid) peak estimation setting involves a trajectory $x(t \mid x_0)$, starting at the initial point $x_0 \in X_0 \subset X$, and evolving according to dynamics $\dot{x}(t) = f(t, x(t))$ in a space $X$. The program to find the maximum value of a state function $p(x)$ along trajectories is
\begin{equation}
    \begin{aligned}
    P^* = & \sup_{t,\, x_0 \in X_0} p(x(t \mid x_0)), &\label{eq:peak_traj} & & 
    & \dot{x}(t) = f(t, x(t)).
    \end{aligned}
\end{equation}

% \urg{I'm trying to skip fixed terminal time to go straight to free terminal time. This may be a mistake. This section is now in the uncertain paper, may swap with uncertain to split into `occupation measures' and `peak estimation' }

The extremum $P^*$ may be bounded through the use of occupation measure relaxations \cite{cho2002linear}. 
An optimal trajectory satisfying $P^* = p(x^*) = p(x(t^* \mid x_0^*))$ is described by a triple $( x_0^*, t^*, x^*)$ \cite{miller2020recovery}.
The initial probability measure $\mu_0 \in \Mp{X_0}$ is distributed over the set of initial conditions $x_0 \sim \mu_0$. The peak measure $\mu_p \in \Mp{[0, T] \times X}$ is a terminal measure with free terminal time. 
For a stopping time $t^*$ and subsets $A \subseteq [0, t^*], \ B \subseteq X$, the occupation measure $\mu \in \Mp{[0, T] \times X}$ has a definition \cite{cho2002linear}
\begin{equation*}
    \mu(A \times B) =  \int_{X_0}\int_{t=0}^{t^*} I((t, x(t \mid x_0)) \in A \times B) dt \  d\mu_0(x_0).
\end{equation*}

% \urg{Ensure there is no self-plagarism here}
The Lie derivative operator $\Lie_f$ may be defined for all test functions $v \in C^1([0, T]\times X)$
\begin{equation}
    \Lie_f v(t, x) = \partial_t v(t,x) + f(t,x) \cdot \nabla_x v(t,x).
\end{equation}
The three measures ($\mu_0, \ \mu_p, \ \mu$) are linked by Liouville's equation for all test functions $v$ 
\begin{align}
\inp{v(t, x)}{\mu_p} &= \inp{v(0, x)}{\mu_0} + \inp{\Lie_f v(t,x)}{\mu} \label{eq:liou_int}\\
\mu_p &= \delta_0 \otimes \mu_0 + \Lie_f^\dagger \mu \label{eq:liou_mom}
\end{align}

Liouville's equation ensures that initial conditions distributed as $\mu_0$ are connected to terminal points distributed as $\mu_p$ by trajectories following dynamics $f$.
% Two consequences of \eqref{eq:liou_mom} are that $\inp{1}{\mu_0} = \inp{1}{\mu_p}$ ($v(t,x) = 1$) and that $\inp{1}{\mu} = \inp{t}{\mu_p}$ ($v(t, x) = t$).
A convex measure relaxation of problem \eqref{eq:peak_traj} is \cite{cho2002linear},
\begin{subequations}
\label{eq:peak_meas}
\begin{align}
p^* = & \ \sup \quad \inp{p(x)}{\mu_p} \label{eq:peak_meas_obj} \\
    & \mu_p = \delta_0 \otimes\mu_0 + \Lie_f^\dagger \mu \label{eq:peak_meas_flow}\\
    & \inp{1}{\mu_0} = 1 \label{eq:peak_meas_prob}\\
    & \mu, \mu_p \in \Mp{[0, T] \times X} \label{eq:peak_meas_peak}\\
    & \mu_0 \in \Mp{X_0}. \label{eq:peak_meas_init}
\end{align}
\end{subequations}
Constraint \eqref{eq:peak_meas_prob} ensures that both $\mu_0$ and $\mu_p$ are probability measures with unit mass. The objective \eqref{eq:peak_meas_obj} is the expectation of $p(x)$ with respect to $\mu_p$. 
There will be no relaxation gap between problems \eqref{eq:peak_traj} and  \eqref{eq:peak_meas_prob} ($P^* = p^*$) when $[0, T] \times X$ is compact and $f$ is Lipschitz \cite{lewis1980relaxation, cho2002linear, fantuzzi2020bounding}. Program \eqref{eq:peak_meas_init} is a particular form of the optimal control program from \cite{lewis1980relaxation} with zero running cost and  free terminal time.

% Program \eqref{eq:peak_meas} has a dual problem  over continuous functions, 
% \begin{subequations}
% \label{eq:peak_cont}
% \begin{align}
%     d^* = & \ \inf_{\gamma \in \Rn} \quad \gamma & \\
%     & {\gamma} \geq {v(0, x)}  &  \forall x \in X_0 \label{eq:peak_cont_init}\\
%     & \Lie_f v(t, x) \leq 0 & \forall (t, x) \in [0, T] \times X \label{eq:peak_cont_f}\\
%     & v(t, x) \geq p(x) & \forall (t, x) \in [0, T] \times X \label{eq:peak_cont_p} \\
%     &v \in C^1([0, T]\times X) \label{eq:peak_cont_v}&
% \end{align}
% \end{subequations}

% The variable $v(t,x)$ is termed an auxiliary function in \cite{fantuzzi2020bounding}, and is an upper bound on the cost function $p(x)$ by \eqref{eq:peak_cont_p}. The graph $(t, x(t \mid x_0))$ is contained in the sublevel set $\{(t, x) \mid v(t, x) \leq \gamma\}$ for all $x_0 \in X_0$. Programs \eqref{eq:peak_meas} and \eqref{eq:peak_cont} satisfy strong duality ($p^* = d^*$) when the set $[0, T] \times X$ is compact \cite{lewis1980relaxation, cho2002linear, fantuzzi2020bounding}. 
% % The bound $p^* \geq P^*$, is tight to $P^*$ for most locally Lipschitz dynamics $f$ \cite{fantuzzi2020bounding}.

% \urg{Possibly strike the review of peak estimation \acp{LP}. Instead, refer to literature. This would save space.}

% \urg{Something about \ac{LMI} approximation maybe?}

\section{Peak estimation of hybrid systems}
\label{sec:hybrid_peak}

\subsection {Peak Program}
Let $X_0 = \{X_{0 \ell}\}$ be the set of initial conditions for system trajectories. Each of these system trajectories lie inside the set $X = \{X_\ell\}$.

% Let $X_{0 \ell} \subset X_\ell$ be the set of initial conditions for system trajectories at each location.
Each location $\ell$ has a state cost $p_\ell: X_\ell \rightarrow \Rn$ and a set of initial conditions $X_{0 \ell} \subset X_\ell$. Each $p_\ell$ is either bounded below or constant at $-\infty$, and at least one $p_\ell$ is bounded.
The goal of peak estimation is to find the trajectory $x(t)$ which maximizes the state cost across all trajectories and locations: 
% Defining the peak function $p(x(t)): [0, T] \rightarrow \Rn$ taking values $p_\ell(x(t))$ when $x(t)$ is in location $\ell$
    \begin{align}
    P^* = & \sup_{t, \;  \ell_0 \; x_0 } \max_\ell p_\ell(x(t \mid x_0)) \qquad  x(t) \in X_\ell \nonumber\\
    & \textrm{Dynamics follow Algorithm \ref{alg:hybrid_exec} with input } (\ell_0, x_0, \hs, T) \nonumber \\
    & x_{0 } \in X_{0 \ell_0}.     \label{eq:peak_hy_traj}
    \end{align}
The optimization variables of \eqref{eq:peak_hy_traj} are the peak time $t$, initial location $\ell_0$, and initial state $x_{0} \in X_{\ell_0}$. The inner maximization runs over all location-objective functions $p_\ell$.

The following assumptions will be posed on problem \eqref{eq:peak_hy_traj}:
\begin{assumption}
The sets $[0,T], \  X_\ell, X_{\ell 0}$ are compact for all $\ell=1..L$.
\end{assumption}
\begin{assumption}
Each function $p_\ell$ is continuous inside $X_\ell$.
\end{assumption}
\begin{assumption}
Each dynamics function $f_\ell(t, x_\ell)$ is Lipschitz over the compact set $[0, T] \times X_\ell$.
\end{assumption}
\begin{assumption}
Trajectories stay in $X_\ell$ for all $t \in [0, T]$ when starting inside $X_{0 \ell}$.
% If a trajectory at location $\ell$ reaches the boundary $x(t \mid x_0) \in \partial X_{\ell}$ and there does not exist a guard $S_e$ with $x(t \mid x_0) \in S_e$ and $\text{src}(e)=\ell$, then $x(t \mid x_0)$ remains outside $\cup_\ell X_\ell$ for all $t' \in (t, T]$. 
\end{assumption}

\subsection{Measures for Hybrid Systems}

The control and reachability set programs in \cite{shia2014reachable,mohan2016convex,  han2019controller} define measures $\rho_e$ supported over the guard $\Mp{S_e}$ for each transition $e \in \es$. For subsets $A \subset [0, T], \ C_e \subset S_e$ and an initial condition $x_0$, the counting measure $\rho_e$ records the number of times the trajectory, starting from location $\textrm{src}(e)$, enters the patch $C_e$ of the guard $S_e$ with
\begin{equation}
\label{eq:meas_count}
    \rho_e(A \times C_e) = \int_{A} \textrm{card}\left(\lim_{t'\rightarrow t^-} x(t' \mid x_0) \in C_e\right) dt.
\end{equation}
The mass of the counting measure $\rho_e$ is the expected number of times a trajectory will traverse the transition with arc $e$. In a Zeno execution of transition $e$, the mass $\inp{1}{\rho_e}$ will be unbounded, and constraints such as $\inp{1}{\rho_e} \leq N_e$ may be imposed to cap the maximum number of transitions on arc $e$.
% The definition in \eqref{eq:meas_count} counts transitioning trajectories starting in the source location $\textrm{src}(e)$, and is therefore different than equation (20) of \cite{shia2014reachable} which counts in the destination.
Let $X_{0\ell} \subseteq X_\ell$ be a set of initial conditions defined on each space $X_\ell$ in $X$. A distribution of initial conditions over each location is $\mu_{0\ell} \in \Mp{X_{0\ell}}$ for $\ell = 1..L$. Let $T < \infty$ be a final time, and $\mu_{p \ell} \in \Mp{[0, T] \times X_\ell}$ be peak measures supported over each location-space. Trajectories following dynamics $x'(t) = f_\ell(t, x(t))$ in each space $X_\ell$ are tracked by occupation measures $\Mp{[0, T] \times X_\ell}$. Counting measures $\rho_e \in \Mp{S_e}$ are set up over all guards to handle state transitions.
The Liouville equation with guard measures holding for all test functions $v_\ell \in C^1([0, T] \times X_\ell)$ and locations $\ell = 1..L$ is
\begin{align}
        \label{eq:liou_hy} \mu_{p  \ell} &= \delta_0 \otimes\mu_{0 \ell} + \Lie_{f_\ell}^\dagger \mu_\ell\\
        &+\textstyle \sum_{\textrm{src}(e) = \ell} R_{e\#} \rho_e - \textstyle \sum_{\textrm{dst}(e) = \ell} \rho_e. \nonumber
\end{align}
For a location $\ell$ and edge $e$ with $\textrm{src}(e) = \ell$, the pushforward term $R_{e \#}$ in \eqref{eq:liou_hy} should be understood as
\begin{equation}
    \inp{v_{\ell}}{R_{e\#} \rho_e} = \inp{v_{\ell}(t, R_e(x_\ell))}{\rho_e}.
\end{equation}
The mass of the peak measure $\mu_{p\ell}$ is equal to the mass of the initial measure $\mu_{0\ell}$ plus the net flux due to state transitions. 

\subsection{Measure Program}
\label{sec:hybrid_meas}
Problem \eqref{eq:peak_hy_traj} may be relaxed through an infinite-dimensional linear program in occupation measures. The measures $\mu_{0\ell}$ are distributions of initial conditions, and $\rho_e$ are transition counting measures, just as in the Liouville equation \eqref{eq:liou_hy}. The peak measures $\mu_{p \ell}$ are final measures with free terminal time between $t \in [0, T]$. 
% The slack variables $c_e$ ensure that there is no Zeno execution, or at least that transitions occur a finite number of times. 
The measure program in terms of $(\mu_0, \mu_p, \mu, \rho)$ for hybrid peak estimation  (where $\forall \ell$ and $\forall e$ may be expanded to $\forall \ell = 1..L$ and $\forall e \in \es$) is
\begin{subequations}
\label{eq:peak_meas_hy}
\begin{align}
p^* = & \ \textrm{sup} \quad \textstyle\sum_{\ell=1}^L \inp{p_\ell}{\mu_{p\ell}} & \label{eq:peak_meas_hy_obj} \\
    & \mu_{p  \ell} = \delta_0 \otimes\mu_{0 \ell} + \Lie_{f_\ell}^\dagger \mu_\ell  \label{eq:peak_meas_hy_flow} & & \forall \ell\\
    & \quad \ \, +\textstyle \sum_{\textrm{dst}(e) = \ell} R_{e\#} \rho_e  - \textstyle \sum_{\textrm{src}(e) = \ell}\rho_e  \nonumber\\
    & \textstyle\sum_{\ell=1}^L \inp{1}{\mu_{0 \ell}} = 1 & \label{eq:peak_meas_hy_prob}\\
    &  \inp{1}{\rho_e} \leq N_e & & \forall e \label{eq:peak_meas_hy_Zeno}\\
    & \mu_{\ell}, \ \mu_{p \ell} \in \Mp{[0, T] \times X_\ell} & & \forall \ell\label{eq:peak_meas_hy_peak_occ}\\
    & \mu_{0\ell} \in \Mp{X_{0 \ell}}& &\forall \ell \label{eq:peak_meas_hy_init} \\
    & \rho_{e} \in \Mp{S_e} & & \forall e. \label{eq:peak_meas_hy_guard} 
\end{align}
\end{subequations}
\begin{theorem}
Solutions to \eqref{eq:peak_meas_hy} and \eqref{eq:peak_hy_traj} satisfy $p^* \geq P^*$
\end{theorem}
\begin{proof}
Let $(x(t \mid x_0, \ell_0), \mathcal{T},\mathcal{C})$ be a trajectory from the execution of Algorithm \ref{alg:hybrid_exec} that stops at time $t^* \in [0, T]$, and $\textrm{Loc}(t)$ be the function returning the residing location of $x(t)$ at time $t$. This trajectory may be described by a tuple $(\ell_0, x_0, t^*)$. 
% The trajectory $x(t)$ starts at the point $x_0^*$ in location $\ell_0^*$, and reaches the peak value of $P^* = p_{\ell^*}(x_p^*) = p_{\ell^*}(x(t^* \mid x_0^*, \ell_0))$ in time $t^*$ at the point $x_p^*$ in location $\ell_p^*$. 
Measures $\forall \ell: \mu_{0\ell}, \mu_{p \ell}, \mu_{\ell}$ and $\forall e: \rho_e$ that are feasible solutions to constraints \eqref{eq:peak_meas_hy_flow}-\eqref{eq:peak_meas_hy_guard} may be formed from the trajectory $x(t)$. The initial measure $\mu_{0\ell}$ is $\delta_{x = x_0}$ for $\ell=\ell_0$ and is the zero measure for $\ell \neq \ell_0$. The peak measure $\mu_{p \ell}$ is $\delta_{t=t^*} \otimes \delta_{x = x(t^* \mid x_0, \ell_0)}$ for $\ell = \textrm{Loc}(t^*)$ and is also the zero measure for all other $\ell$. 
% The objective from \eqref{eq:peak_meas_obj} is $\sum_{\ell=1}^L\inp{p_\ell}{\mu_{p\ell}} = \inp{p_{\ell_p^*}}{\delta_{t=t_p^*} \otimes \delta_{x = x_p^*}} + \sum_{\ell\neq \ell_p^*}0 = p_\ell(x(t \mid x_0^*)) = P^*$.
Let $T_\ell$ be the set $T_\ell = \{t \mid t \in [0, t_p^*], \textrm{Loc}(t) = \ell\}$ of times where $x(t)$ is in location $\ell$.
Each relaxed occupation measure $\mu_\ell$ may respectively be set to the occupation measure of $t \mapsto (t, x(t \mid x_0, \ell_0))$ in the times $t \in T_\ell$.
% the unique measure satisfying $\inp{v_\ell(t, x_\ell)}{\mu_\ell}=  \int_{T_\ell} v(t, x(t \mid x_0)) dt$ for each $\ell = 1..L$. 
If the transition with edge $e \in \es$ is traversed $\mathcal{N}_e$ times along the trajectory $x(t)$ at points $\{(t_i^e, x_{i}^e)\}_{i = 1}^{\mathcal{N}_e}$ for $x_{i}^e \in X_{\textrm{src}(e)}$, the guard measure $\rho_e$ may be defined as $\rho_e = \sum_{i=1}^{\mathcal{N}_e} \delta_{t=t_i^e} \otimes \delta_{x=x_{i}^e}.$ 
The objective $p^*$ is an upper bound on $P^*$ because a set of measures $(\mu_{0\ell}, \mu_{p \ell}, \mu_{\ell}, \rho_e)$ constructed from every trajectory $x(t)$ satisfy the constraints of \eqref{eq:peak_meas_hy} with objective $P^*$. 
% If multiple trajectories $x^1(t), x^2(t), \ldots$ each attain a peak value of $P^*$ along their execution, then convex combinations of their derived measures will form feasible solutions of program \eqref{eq:peak_meas_hy} by convexity. 
\end{proof}
% If $P^*$ is reached multiple times along a trajectory $x(t)$, then $x(t)$ should be split into subsidiary trajector


% \urg{Show that for every trajectory solving \eqref{eq:peak_hy_traj} there exists measures that are feasible solutions to constraints \eqref{eq:peak_meas_hy_flow}-\eqref{eq:peak_meas_hy_guard}.

% Guard measures $\rho_e$ are supported on the poincare map, points of transition.
% }



% \begin{remark}
% The measures $\mu_{0\ell}, \mu_{p \ell}, \rho_e$ are constructed from trajectories are each atomic measures or are the zero measures. The occupation measures $\mu_\ell$ are generically non-atomic (or are zero), given that $\mu_\ell$ are supported on the graph of trajectories in their locations.
% \end{remark}

% Constraint \eqref{eq:peak_meas_hy_prob} ensures that the initial measures $\{\mu_{0 \ell}\}_\ell$ combine to form a probability measure over the disjoint union of initial conditions $\sqcup_\ell X_{0 \ell}$. The sum of constraint \eqref{eq:peak_meas_hy_flow} with test function $v_\ell = 1$ along all $\ell$ is,
% \begin{equation}
%     \textstyle \sum_{\ell = 1}^L \inp{1}{\mu_{p  \ell}} =  \textstyle \sum_{\ell = 1}^L \inp{1}{\mu_{0  \ell}} = 1.
% \end{equation}

\begin{remark}
\label{rmk:peak_below}
Setting a peak objective to $p_\ell(x) = -\infty$ is equivalent to constraining $\mu_{p \ell}$ to the zero measure, because trajectories to maximize $p(x)$ will not terminate in location $\ell$.  Likewise, a measure $\mu_{0\ell} \in \Mp{X_{0\ell}}$ where $X_{0\ell}=\varnothing$ is the zero measure.
\end{remark}

\begin{theorem}
\label{thm:bounded_mass} All measures involved in a solution to \eqref{eq:peak_meas_hy} are bounded.
\end{theorem}
\begin{proof}
Sufficient conditions for a measure to be bounded are that its mass is finite and its support is compact. This setting satisfies the compact support requirement. 

Given that all measures $(\mu_0, \mu_p, \mu, \rho)$ are nonnegative, their masses will also be nonnegative numbers. The mass of the transition measures $\rho$ are upper bounded by the Zeno constraints \eqref{eq:peak_meas_hy_Zeno} under the assumption that all $N_e$ are finite. Constraint \eqref{eq:peak_meas_hy_init} upper bounds each mass $\inp{1}{\mu_{0 \ell}}$. For each location $\ell$, choosing a test function $v_\ell(t, x_\ell) = 1$ for Liouville equation \eqref{eq:peak_meas_hy_flow} yields
\begin{align}
\label{eq:mass_liou_1}
    \inp{1}{\mu_{p\ell}} &= \inp{1}{\mu_{0\ell}} + \textstyle\sum_{\textrm{dst}(e) = \ell} \inp{1}{\rho_e } - \textstyle \sum_{\textrm{src}(e) = \ell}\inp{1}{\rho_e}.\\
\intertext{Every term on the right-hand side of \eqref{eq:mass_liou_1} is finite and $\inp{1}{\mu_{p\ell}} \geq 0$ by measure nonnegativity, so each peak measure $\mu_{p\ell}$ has bounded mass. Utilizing a test function of $v_\ell(t,x_\ell) = t$ with $\Lie_{f_\ell}t = 1$ results in}
\label{eq:mass_liou_t}
    \inp{t}{\mu_{p\ell}} &= \inp{1}{\mu_{\ell}} + \textstyle\sum_{\textrm{dst}(e) = \ell} \inp{t}{\rho_e } - \textstyle \sum_{\textrm{src}(e) = \ell}\inp{t}{\rho_e}.
\end{align} 

The terms $\inp{t}{\mu_{p\ell}}, \inp{t}{\rho_e} $ are all finite due to bounded masses and compact support, so the occupation measures $\mu_{\ell}$ also have finite mass and are bounded.
\end{proof}


\begin{theorem}
\label{thm:no_relaxation}
The objectives in \eqref{eq:peak_hy_traj} and \eqref{eq:peak_meas_hy} will satisfy $p^* = P^*$ when $[0, T] \times \prod_\ell X_\ell$ is compact, each $f_\ell$ is Lipschitz, and $p^*$ is bounded above. 
\end{theorem}
\begin{proof}
This statement may be proved by extending arguments from  \cite{zhao2019optimal}. Theorem 17 of \cite{zhao2019optimal} states there is no relaxation gap in measure \acp{LP} of an optimal control program with appropriate assumptions, extending the \ac{ODE} result of \cite{lewis1980relaxation}. Free final time is already accounted for in \cite{zhao2019optimal} by reference to Remark 2.1 of \cite{lasserre2008nonlinear}. The \ac{ODE} problem in \cite{lewis1980relaxation} can handle initial conditions lying in a set $X_0$, so the method in \cite{zhao2019optimal} can similarly work with sets of initial conditions $\{X_{0\ell}\}_{\ell=1}^L$ as demonstrated by \cite{zhao2019switching}. The work in \cite{zhao2019switching} has `switching' costs (possibly differing running and terminal costs in each location), which is realized by the costs $p_\ell$. The final modification between this work and \cite{zhao2019optimal} is that problem \eqref{eq:peak_hy_traj} has finite Zeno caps $N_e$, while Assumption 3 of \cite{zhao2019optimal} forbids Zeno trajectories. The allowance for free terminal time permits consequence 4 of Theorem 12 of \cite{zhao2019optimal} to read that there exists a constant $C$ such that $\sum_e \inp{1}{\rho_e} \leq \sum_e N_e = C$. The three modifications of \cite{zhao2019optimal} (free terminal time, multiple initial conditions, Zeno caps) are all cleared, so $p^* = P^*$ under the compactness and Lipschitz assumptions.
% \urg{No relaxation gap between $P^*$ and $p^*$? Need to find optimal control reference for Hybrid system, extension of Lewis and Vinter work in optimal control \cite{lewis1980relaxation}.

% This is accomplished by \cite{zhao2019optimal}.}
\end{proof}

% The peak measure $\sqcup_\ell \mu_{p \ell}$ is also a probability measure, this time distributed over points in $[0, T] \times \prod_\ell X_\ell$ \urg{(disjoint union vs cartesian product?)}.

\subsection{Function Program}
\label{sec:hybrid_cont}

The measure program \eqref{eq:peak_meas_hy} is dual to an infinite-dimensional linear program in continuous functions. The Lagrangian $\scL$ of problem \eqref{eq:peak_meas_hy} with dual variables $v_\ell \in C^1([0, T] \times X_\ell), \ \gamma \in \Rn, \alpha \in \Rn_+^{\abs{\es}}$ is
\begin{align}
    \label{eq:peak_lagrangian}
    \scL &= \textstyle\sum_{\ell=1}^L \inp{p_\ell}{\mu_{p \ell}} + \inp{v_\ell(t,x)}{\delta_0 \otimes\mu_{0 \ell} + \Lie_{f_\ell}^\dagger \mu_\ell}  \\
    &+\inp{v_\ell(t, x)}{\textstyle \sum_{\textrm{dst}(e) = \ell} R_{e\#} \rho_e - \textstyle \sum_{\textrm{src}(e) = \ell} \rho_e - \mu_{p \ell}} \nonumber\\
    &+ \gamma(1 - \textstyle\sum_{\ell=1}^L\inp{1}{\mu_{0 \ell}}) + \sum_{e \in \es}\alpha_e (N_e  -\inp{1}{\rho_e}). \nonumber 
\end{align}

% Lagrangian \eqref{eq:peak_lagrangian} may be reorganized into,
%     \begin{align}
%     \scL &= \gamma + \textstyle\sum_{e \in \es} N_e \alpha_e + \textstyle\sum_{\ell=1}^L \inp{v_\ell(0, x) - \gamma}{\mu_{0 \ell}}\label{eq:peak_lagrangian_r} \\
%     &+ \textstyle\sum_{\ell=1}^L \inp{p_\ell(x_\ell)- v_\ell(t, x)}{\mu_{p \ell}} + \textstyle\sum_{\ell=1}^L \inp{\Lie_{f_\ell}v_\ell(t, x)}{\mu_\ell}     \nonumber
% \\
%     &+ \inp{-\alpha_e + v_{\textrm{dst}(e)}(t, R_e(x_{\textrm{src}(e)})) - v_{\textrm{src}(e)}(t, x_{\textrm{src}(e)})}{\rho_e} \nonumber 
%     \end{align}

The dual function program of \eqref{eq:peak_meas_hy} is
\begin{subequations}
\label{eq:peak_cont_hy}
\begin{align}
    d^* = & \inf_{\gamma, \alpha, v} \quad \sup_{\mu_{0\ell}, \mu_{p \ell}, \mu_{\ell}, \rho_e} \scL \nonumber \\
  d^* = &\ \inf_{\gamma \in \Rn, \ \alpha \in \Rn_+^{\abs{\es}}} \quad \gamma + \textstyle\sum_{e \in \es} N_e \alpha_e & & \\
    & \forall \ell: \ \forall x_\ell \in  X_{0\ell}: \nonumber\\
    & \qquad \gamma \geq v_\ell(0, x_\ell)  & \label{eq:peak_cont_hy_init}\\
    & \forall \ell: \ \forall (t, x_\ell)\in [0, T] \times X_\ell: \nonumber\\
    & \qquad 0 \geq \Lie_{f_\ell} v_\ell(t, x_\ell)& &  \label{eq:peak_cont_hy_flow}\\
    & \forall e: \ \forall (t,x_{\textrm{src}(e)}) \in [0, T] \times X_{\textrm{src}(e)}: \nonumber\\
    & \qquad v_{\textrm{src}(e)}(t, x_{\textrm{src}(e)}) - v_{\textrm{dst}(e)}(t, R_e(x_{\textrm{src}(e)}))\geq -\alpha_e    \label{eq:peak_cont_hy_jump}\\
    & \forall \ell: \ \forall (t, x_\ell) \in [0, T] \times X_\ell: \nonumber\\
    & \qquad v_\ell(t, x_\ell) \geq p_\ell(x_\ell) & \label{eq:peak_cont_hy_p} \\
    & \forall \ell: \ v_\ell(t,x_\ell) \in C^1([0, T]\times X_\ell) \label{eq:peak_cont_hy_v}. \qquad  
    \end{align}
\end{subequations}

The dual variables $v_\ell$ are auxiliary functions that decrease along trajectories \eqref{eq:peak_cont_hy_flow} and along transitions \eqref{eq:peak_cont_hy_jump}. The auxiliary functions upper bound the location-costs by \eqref{eq:peak_cont_hy_p}. 
% The values $\alpha$ are dual to the Zeno slacks $z$, and positive $\alpha$ make it easier for the jump condition \eqref{eq:peak_cont_hy_jump} to hold. 
% If transition $e$ is traveled at most $N_e-1$ times (no risk of Zeno), then by complementary slackness of constraint \eqref{eq:peak_meas_hy_guard} $\alpha_e = 0$.
The dual variable $\alpha_e$ will be zero if transition $e$ is traveled at most $N_e-1$ times (complementary slackness of  \eqref{eq:peak_meas_hy_guard}).

\begin{theorem}
\label{thm:strong_duality}
Programs \eqref{eq:peak_meas_hy} and \eqref{eq:peak_cont_hy} will possess equal objectives $p^*=d^*$ when each $X_\ell$ is compact and $(T, N_e)$ are each finite.
\end{theorem}
\begin{proof}
$p^*=d^*:$
Strong duality follows by arguments from Theorem 2.6 of \cite{tacchi2022convergence}, specifically from boundedness of measures (Theorem \ref{thm:bounded_mass}) and compactness (Assumption A2).
% Theorem 3.10 of \cite{anderson1987linear} states that necessary and sufficient conditions for strong duality between a measure and function \ac{LP} are that the affine map induced by linear constraints are closed in the weak-* topology and $p^*$ is bounded below. The first requirement is satisfied because the image of constraints \eqref{eq:peak_meas_hy_flow}-\eqref{eq:peak_meas_hy_guard} is closed with respect to inputs $(\mu_{0\ell}, \mu_{p \ell}, \mu_{\ell}, \rho_e).$ Given that each $p_\ell$ is either bounded below or is equal to $-\infty$ (see Remark \ref{rmk:peak_below}), a choice of measures $\mu_{p \ell}$ to maximize the objective in \eqref{eq:peak_meas_hy} will not be supported in locations $\ell$ where $p_\ell = -\infty$. The objective $p^*$ is therefore bounded below, satisfying the conditions for strong duality in Theorem 3.10 of \cite{anderson1987linear}.
\end{proof}

% Given that each $[0, T] \times X_\ell$ is compact, it is sufficient that all measures have bounded masses in order for the measures to have bounded moments.  The masses of $\rho_e$ are each upper bounded by $N_e$ by constraint \eqref{eq:peak_meas_hy_guard}, and the sum of the masses of $\mu_{0\ell}$ are upper bounded by 1 by \eqref{eq:peak_meas_hy_prob}. The sum of constraint \eqref{eq:peak_meas_hy_flow} with test function $v_\ell = 1$ along all $\ell$ is $\sum_{\ell = 1}^L \inp{1}{\mu_{p  \ell}} =  \textstyle \sum_{\ell = 1}^L \inp{1}{\mu_{0  \ell}} = 1$, so each mass of  $\mu_{p \ell}$ is finite. Lastly, the use of a test function of $v_\ell=t$ on each Liouville equation in \eqref{eq:peak_meas_hy_flow} yields the finite expression $\inp{1}{\mu_\ell} =  \inp{t}{\mu_p} -\sum_{\textrm{dst}(e)=\ell} \inp{t}{\rho_e} + \sum_{\textrm{src}(e)=\ell} \inp{t}{\rho_e}$. The affine map is closed and all moments of measures are bounded, so strong duality holds between \eqref{eq:peak_meas_hy} and \eqref{eq:peak_cont_hy}.

% $d^*=P^*$: \urg{Proof of no relaxation gap, appeal to or take methods from other work \cite{zhao2019optimal, fantuzzi2020bounding}. Use Stone-Weierstrass. This paper is an instance of \cite{zhao2019optimal} with free terminal time and an initial set $X_0$, is that sufficiently distinctive?} 




% \begin{remark}
% When the locations $X_\ell$ are a partition of $X$, the peak estimation task may be posed over a hybrid system or over a switching system. The hybrid system in \eqref{eq:peak_cont_hy} has a function $v_\ell$ for each location, while the switching system such as in \cite{miller2021uncertain} has a single $v$ along all locations.
% \end{remark}

% \begin{remark}
% \urg{A possible remark that the time-space partitioning scheme in \cite{cibulka2021spatio} lifts dynamics into a hybrid system, with two-way guards between adjacent space cells and one-way guards between adjacent time cells.}
% \end{remark}

\subsection{Linear Matrix Inequality Program}
% TK
\label{sec:hybrid_lmi}

% \subsubsection{Review of Moment-SOS Hierarchy}
The Moment-\ac{SOS} hierarchy is a method to produce upper bounds to measure \acp{LP} by a sequence of Linear Matrix Inequalities (\acp{LMI}) of increasing size
\cite{lasserre2009moments}. Let $X \in \Rn^n$ be a basic semialgebraic set $X = \{x \mid g_i(x) \geq 0, i=1..N_c\}$, which is the locus of a finite number of finite-degree polynomial inequality constraints. 
An $\alpha$-moment of a measure $\mu \in \Mp{X}$ for $\alpha \in \Nn^{n}, \ \beta \in \Nn$ is $\mathbf{y_{a}} = \inp{x^\alpha}{ \mu}$. To each moment sequence $\mathbf{y}$, there is an associated Riesz linear functional $\mathbb{L}_{\mathbf{y}}$ acting as $\mathbb{L}_{\mathbf{y}}[ \sum_{\alpha, \beta} c_{\alpha  \beta} x^\alpha]\rightarrow  \sum_{\alpha} c_{\alpha}\mathbf{y}_{\alpha}$.

Assume that each polynomial $g_i(x) = \sum_\gamma g_{i\gamma} x^\gamma$ in the definition of $X$ has a finite degree $d_i$.
If the set $X$ satisfies an Archimedean condition (all compact sets may be made Archimedean by adding a redundant ball constraint) \cite{putinar1993compact}, then necessary and sufficient conditions for the sequence $\mathbf{y}$ of putative moments (pseudo-moments) up to degree $2d$ to be moments of a measure $\mu \in \Mp{X}$ are that the following matrices indexed by monomials $\alpha, \beta \in \Nn^n$ are \ac{PSD}:
\begin{equation}
\label{eq:moment_loc}
    \M_d(\mathbf{y})_{\alpha \beta} = \mathbf{y}_{\alpha + \beta}, \quad \M_{d-d_i}(g_{i} \mathbf{y})_{\alpha \beta}  =  \textstyle\sum_{\gamma} g_{i \gamma} \mathbf{y}_{\alpha + \beta + \gamma}.
\end{equation}
The measure $\mu$ is referred to as the \textit{representing measure} of the pseudo-moments $\mathbf{y}$.
The symbol $\M_d(X \mathbf{y})$ will denote a block-diagonal matrix formed by the matrices in \eqref{eq:moment_loc}.

% \subsubsection{LMI for Hybrid Peak Estimation}
The basic semialgebraic sets containing measures in \eqref{eq:peak_meas_hy} are
\begin{align}
    \label{eq:peak_sets}
    \forall \ell: & & X_\ell &= \{x_\ell \mid g_{\ell i}(x_\ell )\geq 0 \mid \ i = 1..N_c^{\ell}\} \nonumber\\
    \forall \ell:  & & X_{0 \ell} &= \{x_\ell \mid  g_{0 \ell i}(x_\ell )\geq 0 \mid \ i = 1..N_c^{0\ell}\}\\
    \forall e:  & & S_e &= \{x_{\textrm{src}(e)} \mid g_{e i}(x_{\textrm{src}(e)})\geq 0 \mid \ i = 1..N_c^{e}\}. \nonumber
\end{align}

Polynomials $g_{\ell i}(x), \ g_{0 \ell i}(x_\ell), \  g_{e i}(x_{\textrm{src}(e)})$ have finite degrees $d_{\ell i}, \ d_{0\ell i}, \  d_{e i}$ respectively for each  $i, \ell, e$ as appropriate.
% Let$(\mathbf{y}^{0 \ell}, \mathbf{y}^{p \ell}, \mathbf{y}^{\ell})$ be the pseudo-moments describing $(\mu_{0 \ell}, \mu_{p \ell}, \mu_p)$ up to degree $2d$ ($\mathbf{y}^{\ell}$ may require moments of degree $>2d$ if the polynomial $f_\ell(t, x_\ell)$ has degree $>1$). For each edge $e \in \es$ let the sequence $\mathbf{r}^e$ describe moments of the transition measure $\rho_e$. 
Let $(\mathbf{y}^{0 \ell}, \mathbf{y}^{p \ell}, \mathbf{y}^{\ell}, \mathbf{r}^e)$ be pseudo-moments of the measures $(\mu_{0 \ell}, \mu_{p \ell}, \mu_p, \rho_e)$.
The Liouville equation \eqref{eq:peak_meas_hy_flow} may be expressed as a collection of affine constraints in the pseudo-moments. Substituting the test function $v(t, x_\ell) = x_\ell^\alpha t^\beta$ into \eqref{eq:peak_meas_hy_flow} yields a relation for each $\alpha \in \Nn^{n_\ell}, \ \beta \in \Nn, \ \ell \in 1..L$:
\begin{align}
    \label{eq:liou_lmi_hy}
        0 &= -\inp{x_\ell^\alpha t^\beta}{\mu_{p \ell}} + \inp{x_\ell^\alpha t^\beta}{ \delta_{t=0}\otimes \mu_{0 \ell}} + \inp{\Lie_{f_\ell}x_\ell^\alpha t^\beta}{\mu_\ell} \nonumber\\
        &+\textstyle \sum_{\textrm{dst}(e) = \ell} \inp{R_e(x_\ell)^\alpha t^\beta}{\rho_e} - \textstyle \sum_{\textrm{src}(e) = \ell}\inp{x_\ell^\alpha t^\beta}{\rho_e}.
\end{align}
The expression $\textrm{Liou}^\ell_{\alpha \beta}(\mathbf{y}^{0 \ell}, \mathbf{y}^{p \ell}, \mathbf{y}^{\ell}, \mathbf{r}^{\es_\ell} ) = 0$ may be defined to abbreviate the affine constraint in pseudo-moments induced by \eqref{eq:liou_lmi_hy}, where $\es_\ell = \{e \in \es \mid \textrm{src}(e) = \ell \textrm{ or } \textrm{dst}(e) = \ell\}$ is the set of arcs including location $\ell$. For a constant degree $d \in \Nn$, define the quantities $d_\ell' = d + \ceil{\textrm{deg}f_\ell/2}-1$ and $k_e = \deg{R_e}$.
The degree-$d$ \ac{LMI} relaxation of \eqref{eq:peak_meas_hy} with variables $(\mathbf{y}^{0 \ell}, \mathbf{y}^{p \ell}, \mathbf{y}^{\ell}, \mathbf{r}^{e} )$ is

% \begin{subequations}
% \label{eq:peak_lmi_hy}
% \begin{align}
%     p^*_d = & \textrm{max} \quad \textstyle \sum_{\ell} \textstyle\sum_{\alpha} p_{\ell \alpha} \mathbf{y}_{\alpha}^{p \ell} \label{eq:peak_lmi_hy_obj} \\
%     & \textstyle \sum_\ell \mathbf{y}^{0\ell}_0 = 1 \\
%      \forall \ell: \ & \alpha \in \Nn^{n_\ell}, \beta \in \Nn, \ \abs{\alpha} + \abs{\beta} \leq 2d \nonumber\\
%     & \qquad  \textrm{Liou}_{\alpha \beta}^\ell(\mathbf{y}^{0 \ell}, \mathbf{y}^{p \ell}, \mathbf{y}^{\ell}, \mathbf{r}^{\es_\ell} ) = 0 \textrm{ by \eqref{eq:liou_lmi_hy}} \label{eq:peak_lmi_hy_flow}\\
%     \forall e: \ & \mathbf{y}^{e}_0 \leq N_e \label{eq:peak_lmi_hy_Zeno}  \\
%     \forall \ell: \ & \M_d(\mathbf{y}^{0 \ell}), \ \M_d(\mathbf{y}^{p \ell}), \  \M_{d'_\ell}(\mathbf{y}^{\ell})
%      \succeq 0  \label{eq:peak_lmi_hy_psd} \\
%     \forall e: \ & \M_{k_e d_e}(\mathbf{r}^{e})
%      \succeq 0 \qquad  \label{eq:peak_lmi_hy_psd_guard} \\
%     \forall \ell: \ & \M_{d - d_{0 \ell i}}(g_{0 \ell i} \mathbf{y}^{ \ell 0}) \succeq 0  \label{eq:peak_lmi_hy_init}  \qquad \ \forall i = 1..N_c^{0 \ell }\\
%     \forall \ell: \ & \M_{d - d_{ \ell i}}(g_{ \ell i} \mathbf{y}^{p \ell }) \succeq 0 \qquad  \forall i = 1..N_c^\ell \label{eq:peak_lmi_hy_peak} \\
%     \forall \ell: \ & \M_{d'_\ell - d_{ \ell i}}(g_{ \ell i} \mathbf{y}^{ \ell }) \succeq 0 \qquad  \forall i = 1..N_c^\ell \label{eq:peak_lmi_hy_occ} \\
%     \forall \ell: \ &\M_{d - 2}(t(T-t) \mathbf{y}^{p \ell }), \  \M_{d'_\ell - 2}(t(T-t) \mathbf{y}^{ \ell }) \succeq 0 \label{eq:peak_lmi_hy_peak_time} \\
%     \forall e: \ &\M_{d k_e - d_{ei}}(g_{e i} \mathbf{r}^e) \succeq 0   \qquad \forall i = 1..N_c^e  \label{eq:peak_lmi_hy_guard}\\
%     \forall e: \ &\M_{dk_e - 2}(t(T-t) \mathbf{r}^e) \succeq 0 \label{eq:peak_lmi_hy_guard_time}.
% \end{align}
% \end{subequations}
\begin{subequations}
\label{eq:peak_lmi_hy}
\begin{align}
    p^*_d = & \textrm{max} \quad \textstyle \sum_{\ell} \textstyle\sum_{\alpha} p_{\ell \alpha} \mathbf{y}_{\alpha}^{p \ell} \label{eq:peak_lmi_hy_obj} \\
    & \textstyle \sum_\ell \mathbf{y}^{0\ell}_0 = 1 \\
     \forall \ell: \ & \alpha \in \Nn^{n_\ell}, \beta \in \Nn, \ \abs{\alpha} + \abs{\beta} \leq 2d \nonumber\\
    & \qquad  \textrm{Liou}_{\alpha \beta}^\ell(\mathbf{y}^{0 \ell}, \mathbf{y}^{p \ell}, \mathbf{y}^{\ell}, \mathbf{r}^{\es_\ell} ) = 0 \textrm{ by \eqref{eq:liou_lmi_hy}} \label{eq:peak_lmi_hy_flow}\\
    \forall e: \ & \mathbf{y}^{e}_0 \leq N_e \label{eq:peak_lmi_hy_Zeno}  \\
    \forall \ell: \ &  \M_d(X^{0\ell}\mathbf{y}^{0 \ell}), \ \M_d([0,T]\times X^{\ell}\mathbf{y}^{p \ell}), \  \M_{d'_\ell}([0,T] \times X^\ell\mathbf{y}^{\ell})
     \succeq 0  \label{eq:peak_lmi_hy_psd} \\
    \forall e: \ & \M_{k_e d_e}(S_e\mathbf{r}^{e})
     \succeq 0. \qquad  \label{eq:peak_lmi_hy_psd_guard} 
\end{align}
\end{subequations}

The affine constraints \eqref{eq:peak_lmi_hy_flow}-\eqref{eq:peak_lmi_hy_Zeno} implement a truncation of constraints \eqref{eq:peak_meas_hy_Zeno}-\eqref{eq:peak_meas_hy_Zeno} in terms of finite-length pseudo-moments. Constraints \eqref{eq:peak_lmi_hy_psd}-\eqref{eq:peak_lmi_hy_psd_guard} ensure that there exist representing measures for the pseudo-moments. Solutions to the \ac{SDP} generated from the \ac{LMI} \eqref{eq:peak_lmi_hy} by  raising the degree $d$ will form a chain of upper bounds  $p^*_d \geq p^*_{d+1} \geq \ldots \geq p^*$.

\begin{theorem}
\label{thm:lmi_convergence}
The sequence of upper bounds will satisfy $\lim_{d \rightarrow \infty} p_d^* = P^*$ when $\forall \ell=1..L:$ $[0, T] \times  X_\ell$ and $X_{0 \ell}$ are Archimedean, $f_\ell(t,x)$ are
 polynomial, and $\forall e: N_e$ are  finite.
\end{theorem}
\begin{proof}
The upper bound sequence will converge to $p^*$ when all sets are Archimedean, there exists an interior point to constraints \eqref{eq:peak_meas_hy_flow}-\eqref{eq:peak_meas_hy_guard}, and all measures $(\mu_{0\ell}, \mu_{p \ell}, \mu_{\ell}, \rho_e)$ have bounded moments (\cite{tacchi2022convergence}, Theorem 5 of \cite{trnovska2005strong} and Theorem 4.4 of \cite{lasserre2009moments}).

Let $x_0$ be an initial point starting in some nonempty location $X_\ell$. The set of measures where $\mu_{0\ell} = \delta_{x=x_0}, \ \mu_{p\ell} = \delta_{t=0}\otimes \delta_{x=x_0}$ and all other measures are the zero measure is an interior point to \eqref{eq:peak_meas_hy_flow}-\eqref{eq:peak_meas_hy_guard} (trajectory starting at $x_0$ with zero elapsed time).
Given that each $[0, T] \times X_\ell$ is compact, it is sufficient that all measures have bounded masses in order for the measures to have bounded moments.  The masses of $\rho_e$ are each upper bounded by the finite quantity $N_e$ by constraint \eqref{eq:peak_meas_hy_guard}, and the sum of the masses of $\mu_{0\ell}$ are upper bounded by 1 through \eqref{eq:peak_meas_hy_prob}. The sum of constraint \eqref{eq:peak_meas_hy_flow} with test function $v_\ell = 1$ along all $\ell$ is $\sum_{\ell = 1}^L \inp{1}{\mu_{p  \ell}} =  \textstyle \sum_{\ell = 1}^L \inp{1}{\mu_{0  \ell}} = 1$, so each mass of  $\mu_{p \ell}$ is finite. Lastly, the use of a test function of $v_\ell=t$ on each Liouville equation in \eqref{eq:peak_meas_hy_flow} yields the finite expression $\inp{1}{\mu_\ell} =  \inp{t}{\mu_p} -\sum_{\textrm{dst}(e)=\ell} \inp{t}{\rho_e} + \sum_{\textrm{src}(e)=\ell} \inp{t}{\rho_e}$. The sequence of upper bounds will therefore converge to $p^*$ as $d \rightarrow \infty$ with $p^* = P^*$ from Theorem \ref{thm:no_relaxation}.
\end{proof}

% and this sequence will converge with $\lim_{d\rightarrow\infty} p^*_d =p^*$ if all sets in \eqref{eq:peak_sets} are Archimedean and $T, \forall e: N_e$ are finite
%  \cite{lasserre2009moments}. 
 The sizes of the moment matrices in problem \eqref{eq:peak_lmi_hy} are listed in Table \ref{tab:mom_size}. The computational complexity of numerical \ac{SDP} solvers scale in a polynomial manner with the size of the largest \ac{PSD} matrix \cite{lasserre2008nonlinear}. These \ac{PSD} matrix sizes may be reduced if extant structure such as symmetry, quotient, or sparsity structure is present in  \eqref{eq:peak_meas_hy}.
\begin{table}[h]
\caption{\label{tab:mom_size}Sizes of moment matrices in \ac{LMI} \eqref{eq:peak_lmi_hy}}
        \centering
        \begin{tabular}{c c c c c}
             Moment&  $\M_d(\mathbf{y}^{0\ell})$ & $\M_d(\mathbf{y}^{p\ell})$ & $\M_{d'_\ell}(\mathbf{y}^{\ell})$ & $\M_{d k_e}(\mathbf{r}^{e})$ \\
             Size & $\binom{n_\ell+d}{d}$ & $\binom{1+n_\ell+d}{d}$ & $\binom{1 + n_\ell+d'_\ell}{d'_\ell}$ & $\binom{1+n_\ell + k_e d}{k_e d}$         \end{tabular}
    \end{table}

% \urg{Talk about the computational complexity of \eqref{eq:peak_lmi_hy}. Sizes of moment matrices.}


\begin{remark}
Guards with codimension-1 sets $S_e$ may replace their \ac{PSD} localizing constraints  with linear equality constraints or quotient ring reductions in $\mathbf{r}^e$.
\end{remark}

% Some of the \ac{PSD} localizing constraints in \eqref{eq:peak_lmi_hy_guard} may be replaced with linear equality constraints in entries of $\mathbf{r}^e$ because the minimum codimension of $S_e$ is 1.




% Degree-$d$ \ac{LMI} truncation of problem \eqref{eq:peak_meas_hy} goes here. Requires pseudo-moments, Liouville equation definition as affine constraints, and moment and localizing matrices.

\begin{remark}
Algorithm 1 of \cite{miller2020recovery} may be used to attempt extraction of near-optimal trajectories if the moment matrices $\forall \ell: \ \M_{d}(\mathbf{y}^{0 \ell}), \M_{d}(\mathbf{y}^{p \ell})$ are low-rank.
% if the solution \eqref{eq:peak_lmi_hy} is approximately tight to  \eqref{eq:peak_meas_hy}. The set of candidate optimal trajectories arise from the atoms of $\M_{d}(\mathbf{y}^{0 \ell}), \M_{d}(\mathbf{y}^{\ell})$ for each $\ell$. 
% \urg{Maybe omit this}
\end{remark}


\section{Extensions}

\label{sec:extensions}
This section details extensions to the previously presented peak estimation framework for hybrid systems.
\subsection{Safety}

This section verifies safety of hybrid system trajectories with respect to a group of unsafe sets, based on prior (\ac{ODE}) work in \cite{miller2021distance}. Let $X_{u \ell} = \{x_\ell \mid p_{\ell i}(x_\ell) \geq 0, \ i = 1..N_u\}$ be an unsafe basic semialgebraic set for each location $\ell = 1..L$. 
% \subsubsection{Safety Margin}
% \urg{I might take out the safety margin section, and just keep distance}
% \label{sec:hybrid_safety}
% The concept of safety margins was introduced in \cite{miller2020recovery} to verify safety of dynamical systems with respect to an unsafe set. Let the set $X_u = \{x \mid p_{i}(x) \geq 0, \ i = 1..N_u\} = \{x \mid \min_i p_{i}(x) \geq 0\}$ be a basic semialgebraic set describing the unsafe set. If all \ac{ODE} trajectories starting from an initial set $x_0 \in X_0$ satisfy $\min_i p_{i}(x(t \mid x_0) < 0$ for all time $t \in [0, T]$, then no trajectory starting from $X_0$ ever enters or contacts the unsafe set $X_u$. The value $P^* = \max_{t, x \in X_0}\min_i p_{i}(x(t \mid x_0)$ is a safety margin, and finding that $P^* < 0$ is sufficient to certify safety. The safety margin can be upper bounded through an infinite-dimensional maximin linear program in occupation measures as described in Sections 4 and 5 of \cite{miller2020recovery}.

% Safety margins may be extended to hybrid systems. A safety margin
% $P^*_\ell$ may be evaluated at each location determining if trajectories in location $\ell$ ever enter the unsafe set $X_{u \ell}$. Trajectories may start in location $\ell$ either from originating from an initial set $X_{0 \ell}$ or by arriving through a transition 
% $(\ell', \ell) \in \es$. The maximum safety margin $P^* = \max_\ell P^*_\ell$ being negative is sufficient to certify safety of trajectories starting in the initial set $\{X_{0\ell}\}_\ell$. 

% The measure program for safety margin estimation with variables $(\mu_{p \ell}, \mu_{0 \ell}, \mu_{\ell}, \rho_e, c_e)$ is,
% \begin{subequations}
% \label{eq:peak_meas_safe}
% \begin{align}
% p^* = & \ \textrm{max} \quad \textstyle \sum_{\ell=1}^L q_\ell \label{eq:peak_meas_safe_obj} \\
%     & q_\ell + z_{\ell i} = \inp{p_{\ell i}}{\mu_{p \ell}} \qquad  \forall \ell, \  \forall i = 1..N_{u \ell}  \label{eq:peak_meas_safe_bound}\\
%     & \textrm{Constraints \eqref{eq:peak_meas_hy_flow}-\eqref{eq:peak_meas_hy_Zeno}} \\
%     & \textrm{Variables from \eqref{eq:peak_meas_hy_peak_occ}-\eqref{eq:peak_meas_hy_guard}}\\
%     & q_\ell \in \Rn, \ z_\ell \in \Rn^{N_{u \ell}}_+ \qquad \ \forall \ell \label{eq:peak_meas_safe_unsafe_slack}
% \end{align}
% \end{subequations}
% The $q_\ell$ terms are lower bounds on all constraint-polynomial values in the safety margin definition, and $z_\ell$ are slack values in the lower bound inequality constraints.
% \subsubsection{Distance of Closest Approach}
% Safety margins offer a certificate of safety but lack a geometric or intuitive interpretation. 
% The work in \cite{miller2021distance} quantifies safety of trajectories based on the distance of closest approach. 
Letting $c_\ell(x_\ell, y_\ell)$ be a distance function (cost) with a point-unsafe-set distance
\begin{equation}
    c_\ell(x_\ell; X_{u\ell}) = \inf_{y_\ell \in X_{u\ell}} c_\ell(x_\ell, y_\ell),
\end{equation} the distance estimation problem for hybrid systems is
\begin{equation}
    \begin{aligned}
    Q^* =& \inf_{ \ell' \in 1..N_u, t \in [0, T], \ell_0, x_0} c(x(t \mid x_0); X_{u_{\ell'}}) \\
    & \textrm{Dynamics follow Algorithm \ref{alg:hybrid_exec} with input } (\ell_0, x_0, \hs, T) \nonumber \\
    & x_{0 } \in X_{0 \ell}.     \label{eq:dist_hy_traj}
    \end{aligned}
\end{equation}

Following the procedure from \cite{miller2021distance}, a joint-measure $\eta_\ell(x_\ell, y_\ell) \in \Mp{X_\ell \times X_{u \ell}}$ is added for each unsafe set. The distance objective in \eqref{eq:dist_hy_traj} is replaced with an equivalent expectation over the joint probability measure $\inp{c_\ell(x_\ell, y_\ell)}{\eta_\ell}$.

The measure program for distance estimation with variables $(\mu_{p \ell}, \mu_{0 \ell}, \mu_{\ell},\eta_\ell,  \rho_e)$ is
\begin{subequations}
\label{eq:peak_meas_dist}
\begin{align}
q^* = & \ \textrm{inf} \quad \textstyle \sum_{\ell=1}^L \inp{c_\ell(x_\ell, y_\ell)}{\eta_\ell} \label{eq:peak_meas_dist_obj} \\
    & \pi^{x_\ell}_\# \eta_\ell = \pi^{x_\ell}_\# \mu_{p \ell} \ \forall \ell \label{eq:dist_meas_marg_bound}\\
    & \textrm{Constraints \eqref{eq:peak_meas_hy_flow}-\eqref{eq:peak_meas_hy_Zeno}} \\
    & \textrm{Variables from \eqref{eq:peak_meas_hy_peak_occ}-\eqref{eq:peak_meas_hy_guard}}\\
    & \forall \ell: \quad \eta_\ell(x_\ell, y_\ell) \in \Mp{X_\ell \times X_{u \ell}}  \label{eq:dist_meas_eta}.
\end{align}
\end{subequations}

\subsection{Uncertainty}
Peak estimation for hybrid systems may be applied to systems with  uncertainty, extending the \ac{ODE} case in \cite{miller2021uncertain}. 
Let $ W_\ell \subset \Rn^{N_{w\ell}}$ 
be a compact set of  time-dependent disturbances for each location. 
Each location obeys dynamics $\dot{x}_\ell = f(t, x_\ell(t), w_\ell(t)),  \  \forall t, \ell: w_\ell(t) \in W_\ell$, 
in which there is no prior assumption of continuity on the process $w(\cdot)$. The uncertainty act as adversarial optimal controls attempting to raise the  peak functions $(p_\ell)$.

Uncertainty in this manner may be realized by adjusting the Liouville equation in \eqref{eq:peak_meas_hy_flow} and occupation measure definitions in \eqref{eq:peak_meas_hy_peak_occ} (where $w_\ell(t)$ acts as a Young measure/relaxed control \cite{young1942generalized}) as in
\begin{subequations}
\label{eq:uncertainty_mods}
\begin{align}
        & \mu_{p  \ell} = \delta_0 \otimes\mu_{0 \ell} + \pi^{tx}_\# \Lie_{f_\ell}^\dagger \mu_\ell  \label{eq:peak_meas_unc_hy_flow} & & \forall \ell\\
& \quad \ \, +\textstyle \sum_{\textrm{dst}(e) = \ell} R_{e\#} \rho_e  - \textstyle \sum_{\textrm{src}(e) = \ell}\rho_e  \nonumber\\
& \mu_{\ell} \in \Mp{[0, T] \times X_\ell \times W_\ell} & & \forall \ell.\label{eq:peak_meas_unc_hy_peak_occ}
\end{align}
\end{subequations}

A particular form of time-dependent uncertainty is switching/polytopic structure. If the system model is $\dot{x}_\ell = \sum_k^{N_s} w_{k \ell} f_{k \ell}(t, x)$ for $N_s$ switching modes and $w_{k \ell} \geq 0, \sum_k w_{k \ell} = 1$, then the Liouville equation in \eqref{eq:uncertainty_mods} may be expressed for occupation measures $\mu_{k \ell} \in \Mp{[0, T] \times X}$ as
\begin{subequations}
\label{eq:uncertainty_switching}
\begin{align}
        & \mu_{p  \ell} = \delta_0 \otimes\mu_{0 \ell} + \textstyle \sum_k^{N_s} \Lie_{f_{k \ell}}^\dagger \mu_{k \ell}  \label{eq:peak_meas_switch_hy_flow} & & \forall \ell\\
& \quad \ \, +\textstyle \sum_{\textrm{dst}(e) = \ell} R_{e\#} \rho_e  - \textstyle \sum_{\textrm{src}(e) = \ell}\rho_e  \nonumber\\
& \mu_{\ell} \in \Mp{[0, T] \times X_\ell} & & \forall \ell. \label{eq:peak_meas_switch_hy_peak_occ}
\end{align}
\end{subequations}


Time-independent uncertainty restricted to a compact set $\Theta \subset \Rn^{N_\theta}$ may also be added by adjoining to dynamics a new state $\dot{x}_\ell = f(t, x_\ell(t), \theta, w_\ell(t)), \ \dot{\theta} = 0$. This new state $\theta$ is preserved between transition jumps, inducing lifted reset maps $\tilde{R}_{s \rightarrow t}(x_s, \theta) = (R(x_t), \theta)$.
%     & \mu_{p  \ell} = \delta_0 \otimes\mu_{0 \ell} + \Lie_{f_\ell}^\dagger \mu_\ell  \label{eq:peak_meas_hy_flow} & & \forall \ell\\
% & \quad \ \, +\textstyle \sum_{\textrm{dst}(e) = \ell} R_{e\#} \rho_e  - \textstyle \sum_{\textrm{src}(e) = \ell}\rho_e  \nonumber\\
% & \mu_{\ell}, \ \mu_{p \ell} \in \Mp{[0, T] \times X_\ell} & & \forall \ell\label{eq:peak_meas_hy_peak_occ}\\

%TODO: Transition counting
% \subsection{Transition Counting}
% The guard measures $\rho_e$ are counting measures \eqref{eq:meas_count} that tabulate the instances and positions of state transitions along arc $e$. Choosing an objective $\inp{1}{\rho_e}$ in the peak estimation  \eqref{eq:peak_meas_hy_obj} will choose a trajectory that maximizes the number of traversals along arc $e$. The returned value of \ac{LMI} \eqref{eq:peak_lmi_hy} with objective $\mathbf{r}^e_0$ at degree $d$ is an upper bound on the number of such traversals. This upper bound is finite under the Zeno cap constraints in \eqref{eq:peak_lmi_hy_Zeno}.

% % \old{
% % \section{Safety analysis of hybrid systems}

% % \label{sec:hybrid_safety}
% % The concept of safety margins was introduced in \cite{miller2020recovery} to verify safety of dynamical systems with respect to an unsafe set. Let the set $X_u = \{x \mid p_{i}(x) \geq 0, \ i = 1..N_u\} = \{x \mid \min_i p_{i}(x) \geq 0\}$ be a basic semialgebraic set describing the unsafe set. If all trajectories starting from an initial set $x_0 \in X_0$ satisfy $\min_i p_{i}(x(t \mid x_0) < 0$ for all time $t \in [0, T]$, then no trajectory starting from $X_0$ ever enters or contacts the unsafe set $X_u$. The value $P^* = \max_{t, x \in X_0}\min_i p_{i}(x(t \mid x_0)$ is a safety margin, and finding that $P^* < 0$ is sufficient to certify safety. The safety margin can be upper bounded through an infinite-dimensional maximin linear program in occupation measures as described in Sections 4 and 5 of \cite{miller2020recovery}.

% % Safety margins may be extended to hybrid systems. Let $X_{u \ell} = \{x_\ell \mid p_{\ell i}(x_\ell) \geq 0, \ i = 1..N_u\}$ be an unsafe basic semialgebraic set for each location $\ell = 1..L$. A safety margin
% % $P^*_\ell$ may be evaluated at each location determining if trajectories in location $\ell$ ever enter the unsafe set $X_{u \ell}$. Trajectories may start in location $\ell$ either from originating from an initial set $X_{0 \ell}$ or by arriving through a transition 
% % $(\ell', \ell) \in \es$. The maximum safety margin $P^* = \max_\ell P^*_\ell$ being negative is sufficient to certify safety of trajectories starting in the initial set $\{X_{0\ell}\}_\ell$. 

% % \subsection{Measure Program}

% % The measure program for safety margin estimation with variables $(\mu_{p \ell}, \mu_{0 \ell}, \mu_{\ell}, \rho_e, c_e)$ is,
% % \begin{subequations}
% % \label{eq:peak_meas_safe}
% % \begin{align}
% % p^* = & \ \textrm{max} \quad \textstyle \sum_{\ell=1}^L q_\ell \label{eq:peak_meas_safe_obj} \\
% %     & q_\ell + z_{\ell i} = \inp{p_{\ell i}}{\mu_{p \ell}} \qquad  \forall \ell, \  \forall i = 1..N_{u \ell}  \label{eq:peak_meas_safe_bound}\\
% %     & \textrm{Constraints \eqref{eq:peak_meas_hy_flow}-\eqref{eq:peak_meas_hy_Zeno}} \\
% %     & \textrm{Variables from \eqref{eq:peak_meas_hy_peak_occ}-\eqref{eq:peak_meas_hy_guard}}\\
% %     & q_\ell \in \Rn, \ z_\ell \in \Rn^{N_{u \ell}}_+ \qquad \ \forall \ell \label{eq:peak_meas_safe_unsafe_slack}
% % \end{align}
% % \end{subequations}

% % More generally, program \eqref{eq:peak_meas_safe} is a maximin program aiming to solve $\max_{\ell} \min_{i} p_{\ell i}(x(t))$ for a set of costs $p_{\ell i}$ \urg{bad notation with trajectories and $x_\ell(t)$}. Program \eqref{eq:peak_meas_hy} may be expressed as an instance of \eqref{eq:peak_meas_safe} with one cost $p_\ell(x_\ell)$ per location. The variables $z_\ell$ in \eqref{eq:peak_meas_safe_unsafe_slack} are slack on the constraint \eqref{eq:peak_meas_safe_bound} ensuring that $q_\ell$ is a lower bound on $\inp{p_{\ell i}}{\mu_p}$ for all $i$. If the safety margin is not achieved at location $\ell$, then $\mu_{p \ell}$ will be the zero measure and $q_\ell = 0$ and $ z_{\ell i} = 0$ for each $ i= 1..N_{u \ell}$. 


% % \subsection{Function Program}

% % The Lagrangian of the maximin problem \eqref{eq:peak_meas_hy} with additional dual variables $\beta_\ell \in \Rn^{N_{u \ell}}_+$ for each $\ell$ is,

% % % \begin{equation}
% % %     \label{eq:peak_lagrangian_safe}
% % %     \begin{aligned}
% % %     L &= q_\ell + \textstyle \sum_{\ell=1}^L \textstyle\sum_{i=1}^{N_{u \ell}} \beta_{\ell i}(\inp{p_\ell}{\mu_{p \ell}} - z_{\ell i} - q_\ell) + \sum_{e \in \es}\alpha_e (N_e - c_e -\inp{1}{\rho_e})\\ 
% % %     &+\inp{v_\ell(t,x)}{\delta_0 \otimes\mu_{0 \ell} +  \Lie_{f_\ell}^\dagger \mu_\ell + \textstyle \sum_{t(e) = \ell} R_{e\#} \rho_e - \textstyle \sum_{e(e) = \ell} \rho_e - \mu_{p \ell}} + \gamma(1 - \textstyle\sum_{\ell=1}^L\inp{1}{\mu_{0 \ell}})
% % %     \end{aligned}
% % % \end{equation}

% % % Define $p_\ell(x_\ell)$ as the vectorization of $\{p_{\ell i}(x_\ell)$ for each $i = 1..N_{u \ell}$ and every $\ell$.
% % % The maximin lagrangian \eqref{eq:peak_lagrangian} may be collected as,
% % % \begin{equation}
% % %     \label{eq:peak_lagrangian_safe_r}
% % %     \begin{aligned}
% % %     L &= \gamma + \textstyle\sum_{e \in \es} N_e \alpha_e + q_\ell(1 - \textstyle \sum_{\ell=1}^L \mathbf{1}^T\beta_{\ell i}) - \textstyle \sum_{\ell=1}^L \textstyle\sum_{i=1}^{N_{u \ell}} \beta_{\ell i}z_{\ell i}\ +  \textstyle\sum_{\ell=1}^L \inp{v_\ell(0, x) - \gamma}{\mu_{0 \ell}} \\
% % %     &+ \textstyle\sum_{\ell=1}^L \inp{\beta_{\ell i}^T p_{\ell}(x_\ell) - v_\ell(t, x)}{\mu_{p \ell}} + \textstyle\sum_{\ell=1}^L \inp{\Lie_{f_\ell}v_\ell(t, x)}{\mu_\ell} \\
% % %     &+ \textstyle\sum_{e \in \es} -\alpha_e c_e + \inp{-\alpha + v_{t(e)}(t, R_e(x_{\textrm{src}(e)})) - v_{\textrm{src}(e)}(t, x_{\textrm{src}(e)})}{\rho_e}
% % %     \end{aligned}
% % % \end{equation}

% % % % \urg{Should the transition expression be $v_{t(e)}(t, R_e(x_{\textrm{src}(e)})) \abs{D R_e(x_{\textrm{src}(e)})}$? The reset map does a coordinate change, so there may need to be a jacobian transformation.}
 
% % The dual program of the maximin \eqref{eq:peak_meas_safe} is,
% % \begin{subequations}
% % \label{eq:peak_cont_safe}
% % \begin{align}
% %     d^* = & \ \min_{\gamma \in \Rn, \ \alpha \in \Rn_+^{\abs{\es}}} \quad \gamma + \textstyle\sum_{e \in \es} N_e \alpha_e & & \\
% %     & \textrm{Constraints \eqref{eq:peak_cont_hy_init}-\eqref{eq:peak_cont_hy_jump}}\\
% %     & \forall \ell, \ \forall (t, x_\ell) \in [0, T] \times X_\ell: \nonumber \\
% %     & \qquad  v_\ell(t, x_\ell) \geq \beta_\ell^T p_\ell(x_\ell) & &  \label{eq:peak_cont_safe_p} \\
% %     & v_\ell(t,x_\ell) \in C^1([0, T]\times X_\ell) \label{eq:peak_cont_safe_v}& & \forall \ell\\
% %     &\beta \in \Rn_+^{N_{u \ell}}, \ \mathbf{1}^T\beta_\ell = 1 \label{eq:peak_cont_safe_beta} & & \forall \ell
% %     \end{align}
% % \end{subequations}
% % \begin{remark}
% % The base program \eqref{eq:peak_cont_hy} has all $\beta_\ell = 1$, as there is only one objective $p_{\ell}$ per location.
% % \end{remark}



% % % % The dual variables $v_\ell$ are auxiliary functions that decrease along trajectories \eqref{eq:peak_cont_hy_flow} and along transitions \eqref{eq:peak_cont_hy_jump}. The auxiliary functions upper bound the location-costs by \eqref{eq:peak_cont_hy_p}. The values $\alpha$ are dual to the Zeno slacks $z$, and positive $\alpha$ make it easier for the jump condition \eqref{eq:peak_cont_hy_jump} to hold. If transition $e$ is traveled at most $N_e-1$ times (no risk of Zeno), then by complementary slackness $\alpha_e = 0$.

% % % % When the quantities $T$ and  $N_e \ \forall e$ are finite and each $X_\ell$ is compact, then all measures $(\mu_{0 \ell}, \mu_{p \ell}, \mu_{\ell}, \rho_e)$ will have bounded moments. Problems \eqref{eq:peak_meas_hy} and \eqref{eq:peak_cont_hy} will be strongly dual ($p^* = d^*$) when $T, N_e$, since measures are bounded and the image of the affine map between \eqref{eq:peak_meas_hy_flow}-\eqref{eq:peak_meas_hy_Zeno} is closed in the weak-* topology (Theorem C.20 of \cite{lasserre2009moments}).


% % \subsection{Safety Examples}
% % Demonstrate examples \urg{Must include uncertainty in order to properly compare with \cite{prajna2004safety}.}
% % }


\section{Numerical Examples}
\label{sec:examples}

% All code is written in MATLAB R2021a, and

Experiments are available at \url{https://github.com/Jarmill/hybrid_peak_est}, and were written in MATLAB 2021a. Dependencies include Gloptipoly3 \cite{henrion2003gloptipoly}, Yalmip interface \cite{lofberg2004yalmip}, and Mosek \cite{mosek92}. All experiments were run on an 
2.30 GHz Intel i9 CPU with 64.0 GB of RAM.

\subsection{Two-Mode}

This system is a modification of Example 2 of \cite{prajna2004safety} to ensure improved numerical conditioning. The two locations correspond to modes of `No Control' ($\ell=1$) and `Control' ($\ell=2$) with dynamics
\begin{subequations}
\label{eq:two_mode_deterministic}
\begin{align}
    f_1(t, x) &= [x_2; -x_1+x_3; x_1 + (1+x_3)^2 (2 x_2 + 3 x_3)] \nonumber \\
    f_2(t, x) &= [x_2; -x_1+x_3; -x_1 - (2 x_2 + 3 x_3)].\label{eq:two_box_dynamics}
\end{align}
\end{subequations}

\subsubsection{Two-Mode: Standard}

Trajectories start in the initial set $X_{01} = \{x \mid \norm{x}_2^2 = 0.2^2\}$ ($X_{02} = \varnothing$), and evolve for a time horizon of $T=20$. The transition edges are $\es = \{(1, 2), (2, 1)\}$ with guard surfaces
\begin{align}
    S_{(1,2)} &= \{x \mid x_1^2/4 + x_2^2 + x_3^2 = 1.5^2\}\label{eq:two_box_guards} \\
    S_{(2,1)} &= \{x \mid x_1^2 + x_2^2 + x_3^2 = 0.2^2\}, \nonumber
\end{align}
and each transition has a trivial reset map $R_e(x) = x$. The Zeno caps used in simulation were $N_{(1,2)}=N_{(2,1)}=5$ with total spaces of $X_1 = X_2 = [-1.5, 1.5]^3$. Figure \ref{fig:two_box_state} plots system trajectories in location 1 (left) and 2 (right), starting from the initial set $X_0$ (gray region).
The peak estimation task for this system is to upper bound extreme values of $p_2(x)=x_1^2$ along system trajectories ($p_1(x)=-\infty$). Solving the \ac{SDP} generated from \ac{LMI} \eqref{eq:peak_lmi_hy} at orders 1-5 produces the sequence of upper bounds,
$p^*_{1:5} = [2.250, 0.6514, 0.4643, 0.4076, 0.3958].$
% $p^*_{1:6} = [2.250, 0.7014, 0.4731, 0.4086, 0.3994, 0.3893].$
\begin{figure}[ht]
    \centering
    \includegraphics[width=0.75\linewidth]{img/deterministic_cube_5_v2.png}
    \caption{Deterministic Two-Mode Bound of $x_1^2 \leq 0.3958 = p_5^*$}
    \label{fig:two_box_state}
\end{figure}

\subsubsection{Two-Mode: Distance Estimation}
Distance estimation is conducted for the deterministic two-mode system \eqref{eq:two_mode_deterministic} with respect to the half-sphere unsafe set
\begin{equation}
    \label{eq:two_mode_unsafe}
    X_u = \{x \mid 0.4^2 \geq  (x_1 + 0.5)^2 + (x_2 + 0.5)^2 + (x_3 - 0.5)^2, \ x_3 \geq 0.5\}.
\end{equation}
The distance penalty $c(x, y) = \norm{x-y}^2_2$ is used in locations $\ell=1, 2$ with the unsafe set $X_u$. \ac{SDP} lower bounds for the distance $\min_\ell \min_{y \in X_u} \norm{x_{\ell} - y}_2^2$ (via \eqref{eq:peak_meas_dist}) are $p^*_{1:5} = [0, 0, 0, 2.799\times 10^{-3}, 7.942\times 10^{-3}]$.
The output of distance estimation is plotted in Figure \ref{fig:two_mode_distance}. The solid red half-sphere is the set $X_u$, and the corona surrounding $X_u$ is the set of all points with an $L_2$ distance at most $0.0891   = \sqrt{p_5^*}$ away from $X_u$.
% $5.119 \times 10^{-3} \times = p_5^*$
\begin{figure}[ht]
    \centering
    \includegraphics[width=0.75\linewidth]{img/distance_cube_5_stable.png}
    \caption{Deterministic Two-Mode Distance Bound of $\min_{y \in X_u}\norm{x-y}_2 \leq 0.0891   = \sqrt{p_5^*}$}
    \label{fig:two_mode_distance}
\end{figure}



% \urg{Reference \cite{prajna2004safety} has $p_1(x) = -\infty$ and $p_2(x) = x_1^2$. I will need to redo the plots and bounds to reflect this. This will also fullfill the requirement of non-identical cost functions between locations.}



% Example 2 of \cite{prajna2004safety} involves full-dimensional guard regions $S_e$, in which state transitions may occur in a nondeterministic manner once the trajectory reaches $S_e$. The expanded regions with a factor of $\epsilon = 0.01$ are,
% \begin{align*}
%     S_{(1,2)}^\epsilon &= \{x \mid 1.5^2-\epsilon \leq x_1^2/4 + x_2^2 + x_3^2 \leq 1.5^2 + \epsilon\} \\
%     S_{(2,1)}^\epsilon &= \{x \mid 0.2^2 - \epsilon \leq x_1^2 + x_2^2 + x_3^2 \leq 0.2^2 + \epsilon\} \\
%     X_{01}^\epsilon &= \{x \mid 0.2^2-\epsilon \leq x_1^2 + x_2^2 + x_3^2 \leq 0.2^2 + \epsilon\}.
% \end{align*}
% Upper bounds of $x_1^2$ for the above expanded regions are $p^*_{1:6} = [2.250, 0.7461, 0.5001, 0.4306, 0.4110, \varnothing]$.

% \urg{Extend this with a safety example and uncertainty}



\subsubsection{Two-Mode: Uncertainty}
Time-dependent uncertainty may be added to dynamics in \eqref{eq:two_mode_deterministic} by defining a process $w(t) \in [-1, 1]$ under the dynamics $\tilde{f}_\ell(t, x) = f_\ell(t, x) + [0;0;w]$. The \ac{SDP} bounds for $x_1^2$ when $w$ is realized as switching-type uncertainty is 
$p^*_{1:5} = [2.250, 1.4029, 1.0350, 0.9790, 0.9660].$ System trajectories and the order-5 bound of this noisy system are plotted in Figure \ref{fig:two_mode_noisy}.

\begin{figure}[!h]
    \centering
    \includegraphics[width=0.75\linewidth]{img/noisy_cube_5_v2.png}
    \caption{Noisy Two-Mode Bound of $x_1^2 \leq 0.9660 = p_5^*$}
    \label{fig:two_mode_noisy}
\end{figure}

\subsection{Right-Left Wrap}
This example has a single location $X = [-1, 1]^2$ with nontrivial reset maps. The dynamics in the single location are
\begin{equation}
\label{eq:rl_dynamics}
    \dot{x} =\begin{bmatrix}-x_2 + x_1*x_2 + 0.5 \\
         -x_2 - x_1 + x_1^3\end{bmatrix}.
\end{equation}
System \eqref{eq:rl_dynamics} has a stable equilibrium point at $(-0.8128, 0.2758)$ and a saddle point at $(-0.4288, 0.3499)$. The following right$\rightarrow$top and left$\rightarrow$bottom transitions are defined with Zeno caps of $N=5$


\begin{align}
    S_{\textrm{right}\rightarrow \textrm{top}} &= \{x \mid x_1 = 1, \ x_2 \in [-1, 1]\} &  R_{\textrm{right}\rightarrow \textrm{top}} &= [x_2, x_1]^T \label{eq:rl_transitions}\\
    S_{\textrm{left}\rightarrow \textrm{bottom}} &= \{x \mid x_1 = -1, \ x_2 \in [-1, 1]\} & R_{\textrm{left}\rightarrow \textrm{bottom}} &= [1-x_2, x_1]^T.\nonumber
\end{align}
The set $X$ is invariant under these state transitions.
A peak estimation task to maximize $p(x) = -(x_1+0.5)^2 + (x_2+0.5)^2$ is defined on system dynamics \eqref{eq:rl_dynamics} starting from the initial set $X_0 = \{x \mid 0.2^2 = (x_1 - 0.5)^2 + (x_2 + 0.3)^2 \}$ for a $T = 5$ time horizon. \ac{SDP} upper bounds for this objective are $p^*_{1:6} = [0,0,-0.3644, -0.5259, -0.5659, -0.5721]$.

Figure \ref{fig:rl} plots the \ac{ODE} system dynamics in \eqref{eq:rl_dynamics}. Figure \ref{fig:rl_bound} plots hybrid system dynamics in cyan, starting from the black-circle $X_0$. The $p_6^*$ bound is  indicated in the red circle of radius $\sqrt{-p_6^*} = 0.7564$, in which no considered hybrid system trajectory is contained.

\begin{figure}[h]
    \centering
     \begin{subfigure}[b]{0.45\textwidth}
         \centering
         \includegraphics[width=\textwidth]{img/rl_stream.png}
         \caption{System dynamics in \eqref{eq:rl_dynamics}}
         \label{fig:rl_stream}
     \end{subfigure}
          \hfill
     \begin{subfigure}[b]{0.45\textwidth}
         \centering
         \includegraphics[width=\textwidth]{img/rl_rad_6_5.png}
         \caption{Bound of $p_6^* = -0.5721$}
         \label{fig:rl_bound}
     \end{subfigure}
    \caption{Peak Estimation of Right-Left Wrap Dynamics \eqref{eq:rl_dynamics} and \eqref{eq:rl_transitions}}
    \label{fig:rl}
\end{figure}

% \subsection{Pendulum Swing-Up}
% \urg{might remove this, keep for arxiv}

% A unit pendulum is stabilized in the upright position through a hybrid controller (energy shaping/LQR) \cite{ chung1995nonlinear, aastrom2000swinging}. The states of the unit pendulum are the angle $\theta$ and angular velocity $\omega$, and the upright setpoint is $\theta=\pi, \omega = 0$. Dynamics with a scalar input $u$ are \cite{tedrake2009underactuated},
% \begin{align}
% \label{eq:pend_dyn_angle}    \dot{\theta} &= \omega \ & \dot{\omega} &= -\sin(\theta) -  u.
% \end{align}
% A trigonometric lift may be used to render dynamics \eqref{eq:pend_dyn_angle} polynomial by introducing variables $c = \cos{\theta}, \ s = \sin{\theta}$ under the relation $c^2 + s^2 = 1$,
% \begin{align}
% \label{eq:pend_dyn_trig}    \dot{c}&=-s \omega & \dot{s}&=c \omega & \dot{\omega} &= -s - c u.
% \end{align} 
% The normalized energy of the unit pendulum is $E = \omega^2/2 - c$, which takes a value of $E=1$ at the setpoint $(c,s,\omega) = (-1,0,0)$. The state space is restricted to the compact set $X = \{(c,s,\omega) \mid \ c^2+s^2=1, \omega \in [-2.5, 2.5]\}$.


% The hybrid stabilizing controller is parameterized by values $(\delta, \epsilon)$ and has four locations: UP, DOWN, WAIT, and LQR. An energy shaping controller regulates pendulum energy to the range of $E \in [1-\epsilon, 1+\epsilon]$ (UP to add energy, DOWN to reduce energy). The controller then waits until the pendulum is in a neighborhood of the setpoint (WAIT), and then LQR controller stabilizes the pendulum (LQR).
% % The LQR controller is synthesized with the following parameters: $A = [0, 1; 1, 0], \ B = [0; -1], \ Q = [2, 0; 0, 1], R = 5$, yielding the
% The LQR controller has a
% gain $K=[-2.1832, -2.1369]$ and the  solution to its Lyapunov equation is $\Phi = [12.6422, 10.9161; 10.9161, 10.6846] \succ 0$. The inputs to dynamics \eqref{eq:pend_dyn_trig} are,
% \begin{subequations}
% \begin{align}
%     u_{\textrm{UP}} &= u_{\textrm{DOWN}} =  -\omega (E-1) \\
%     u_{\textrm{WAIT}}&=0 \quad u_{\textrm{LQR}} = {K_1 (-s) + K_2 \omega}.
% \end{align}
% \end{subequations}
% The term $s=\sin{\theta}$ is used instead of $\theta$ in $u_{\textrm{LQR}}$ to respect the trigonometric lift. Define the quantity $q$ as $q = [-s; \omega]' \Phi [-s; \omega]$. The set $X$ is partitioned into locations,
% \begin{subequations}
% \begin{align}
% \label{eq:pend_state_space}
% X_{\textrm{UP}} &= X \cap \{E \leq 1-\epsilon\} \\
% X_{\textrm{DOWN}} &= X \cap \{E \geq 1+\epsilon\} \\
% X_{\textrm{WAIT}} &= X \cap  \{E \in [1-\epsilon, 1+\epsilon], \  q \geq \delta \} \\
% X_{\textrm{LQR}} &= X \cap \{E \in [1-\epsilon, 1+\epsilon], \ q \leq \delta \}
% \end{align}
% \end{subequations}

% The guards of region transitions are,
% \begin{subequations}
% \begin{align}
%     S_{\textrm{UP}\rightarrow \textrm{WAIT}} &= X \cap \{E = 1-\epsilon, \ q \geq \delta\} \\
%     S_{\textrm{UP}\rightarrow \textrm{LQR}} &= X \cap \{E = 1-\epsilon, \ q \leq \delta\} \\
%     S_{\textrm{DOWN}\rightarrow \textrm{WAIT}} &= X \cap \{E = 1+\epsilon, \ q \geq \delta\} \\
%     S_{\textrm{DOWN}\rightarrow \textrm{LQR}} &= X \cap \{E = 1+\epsilon, \ q \leq \delta\} \\
%     S_{\textrm{WAIT}\rightarrow \textrm{LQR}} &= X \cap \{E \in [1-\epsilon, 1+\epsilon], \ q = \delta\}
% \end{align}
% \end{subequations}
% Parameters of $\delta = 1, \epsilon = 0.1$ are used, and all guards have identity reset maps $R_e(x) = x$.

% Figure \ref{fig:pend_state} visualizes sampled trajectories (cyan curves) starting from the region  $X_0 = \{c^2+s^2=1, \ \omega \in [-0.5, 0.5]\}$ (between black circles) for a time horizon of $T=10$. The stabilized set-point is the red square $(\cos{\theta}, \sin{\theta}, \omega) = (c,s,\omega) = (-1, 0, 0)$. Black circles mark the instances of location transitions in $\es$. The first four upper bounds of $\omega^2$ by \ac{SDP} \eqref{eq:peak_lmi_hy} are $p^*_{1:4} = [4.2386, 4.2006, 4.2000, 4.1998]$, and the red circles are the level set of $\omega^2 = p_4^*$.
% \begin{figure}[ht]
%     \centering
%     \includegraphics[width=\linewidth]{img/pend_4_250_state.png}
%     \caption{Swing-up Bound of $\omega^2 \leq 4.1998 = p_4^*$}
%     \label{fig:pend_state}
% \end{figure}


% % Bounds of \ac{SDP}

% % Mention partitions?

% % \subsection{Bouncing Ball}
% % \urg{Need a working example of the bouncing ball in a 2d basin. Particularly need to demonstrate effectiveness on non-identity reset maps}
% % % \urg{Need to implement slats for the basin of the bouncing ball}

% % % \subsection{Cart-Pole}
% % % \urg{The motivating problem for this paper was finding the maximal displacement of a cart in a cart-pole balancing system. That still may happen \cite{chung1995nonlinear}.}

% % % \subsection{Biological Model}
% % % \urg{Pick out a system in \cite{rocca2018bio} to use as a demo. This is the Sontag issue, so there should be biology somewhere.}


\section{Conclusion}
\label{sec:conclusion}
% \urg{Summarize the paper}

An existing peak estimation framework for \ac{ODE} systems was extended in this paper to hybrid systems. A hierarchy of \acp{SDP} result in a (convergent) decreasing sequence of upper bounds to the true peak value. Extensions to the hybrid peak estimation framework, such as bounding the distance to unsafe sets \cite{miller2021distance} and estimation of systems with uncertainty \cite{miller2021uncertain}, can be accomplished by modifying equations in the \ac{LP}. Future work includes performing peak-minimizing ($L_1$-optimal) control of hybrid systems \cite{molina2022equivalent}, data-driven peak estimation of hybrid systems \cite{miller2021facial_short}, applying numerical techniques to perform peak estimation on more complicated systems (e.g., rigid body dynamics in robotics), and generalizing analysis of deterministic hybrid dynamics to Markov Decision Processes.



% \urg{Ensure that references are properly capitalized. Use ISO4 abbreviations if necessary to save space.}
\bibliographystyle{IEEEtran}
\bibliography{hybrid.bib}


\end{document}
