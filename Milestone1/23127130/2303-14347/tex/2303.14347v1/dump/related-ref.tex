Visual odometry performs poorly for repetitive patterns like crop fields.

\begin{itemize}
    \item \cite{sivakumar2021learned}
    \begin{itemize}
        \item canopy farm robots. truncated resnet-18 output robot heading and placement (distance ratio) in row then compute row center. Just use RGB camera.
        \item works well
        \item needs manual annotation and must be straight line, doesn't do row switching, see failure cases.
    \end{itemize} 
    %%%%%
    \item \cite{martini2022position}
    \begin{itemize}
        \item Use RL
        \item 
        \item Test in simulation only.
    \end{itemize}
    %%%%%%%
    \item \cite{aghi2021deep}
    \begin{itemize}
        \item a custom-trained segmentation
network and a low-range RGB-D camera. Generate  the segmentation maps. Test in simulation and real world data. Take pretrained on image net initialization. IoU loss function. THey find empty columns, compute linear and angular velocity from here,
        \item manual annotation + Gaussian mixture. dataset=1538. THe controller angular velocity proportional to the quadratic of center of cluster from middle of of image. But this is not in BeV. Velocity is similar. Sensitive to field structure whether has multiple holes. if end of row
    \end{itemize}

    \item \cite{ahmadi2020visual}
    \begin{itemize}
        \item Do row switching using front and back cams. row-crop field. ROw crop detection. Extract vegetation mask (non DL) compute center of cluster/connected components. Did line fitting.
        \item  use a fairly straight-forward approach to detect the crop
rows as our main focus has been on the design of the visual
servoing controller and more sophisticated detection methods
are easily implementable. works only for simple structure. Test on row real robot 15m only (ours is 180). Row switching required stepping on the plants.
        \item
    \end{itemize}

    %%%%%%%%%%
    \item \cite{ahmadi2022towards}
    \begin{itemize}
        \item Row crops. ROw switching (but also row crops). omniwheels, four-wheel steering and driven (4WS/4WD). the side-ways
movement of BonnBot-I allowing easier transitions. Use SIFT to detect rows for multi row switching. FOr the line tracking : vegetation mask+ sliding window
        \item 
    \end{itemize}
\end{itemize}

GPS based line tracking line following.
Why we dont prefer:  cost, setup, availability, signal loss, 

LiDAR based
 
Vision based Line tracking. line following, moving in row.
Non DL
DL
RL based.
BUt most work focus primarily on line tracking...


Row switching.
Describe the row switching
But this only works for row crops. 
