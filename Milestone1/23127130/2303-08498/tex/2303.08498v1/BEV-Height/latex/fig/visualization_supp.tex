\begin{figure*}[t!]
	\centering
	\includegraphics[width=\textwidth]{BEV-Height/figures/visualization_supp-1.pdf}
	\caption{\textbf{Visualization Results of BEVDepth and our proposed BEVHeight under the extrinsic disturbance.} We use boxes in \textbf{{\color{red}red}} to represent false positives,  \textbf{{\color{green}green}} boxes for truth positives, and \textbf{{\color{black}black}} for the ground truth. The truth positives are defined as the predictions with IoU\textgreater 0.5 for vehicle and IoU\textgreater 0.25 for pedestrian and cyclist. I/II-(a) Clean means the original image without any processing; I/II-(b) Disturbed Roll denotes camera rotate 1 degree along roll direction; I/II-(c) Disturbed Roll and Pitch represents camera rotate 1 degree along roll and pitch directions simultaneously. We use \textbf{{\color{blue}blue}} dashed ovals to highlight the pronounced improvements in predictions.}
\label{fig:visualization_supp_1}
\vspace{-0.1cm}
\end{figure*}

\begin{figure*}[t]
	\centering
	\includegraphics[width=\textwidth]{BEV-Height/figures/visualization_supp-2.pdf}
	\caption{\textbf{Visualization Results of BEVDepth and our proposed BEVHeight under the extrinsic disturbance in another scene.}}
\label{fig:visualization_supp_2}
\vspace{-0.1cm}
\end{figure*}
