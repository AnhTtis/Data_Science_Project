\begin{figure*}[t!]
	\centering
% 	\includegraphics[width=17.5cm]{BEV-Height/figures/visualization.pdf}
	\includegraphics[width=\textwidth]{BEV-Height/figures/visualization.pdf}
	\caption{\textbf{Visualization Results of BEVDepth and our proposed BEVHeight under the extrinsic disturbance.} We use boxes in \textbf{{\color{red}red}} to represent false positives,  \textbf{{\color{green}green}} boxes for truth positives, and \textbf{{\color{black}black}} for the ground truth. The truth positives are defined as the predictions with IoU\textgreater 0.5 for vehicle and IoU\textgreater 0.25 for pedestrian and cyclist. (a) Clean means the original image without any processing; (b) Disturbed Roll denotes camera rotate 1 degree along roll direction; (c) Disturbed Roll and Pitch represents camera rotate 1 degree along roll and pitch directions simultaneously. 
% 	The image is processed with rotation and translation operations to maintain an accurate calibration relationship between the new image and the new camera's external matrix.
We observe that, our methods outperform the baseline in all three settings. Note that, the BEVDepth only identify two object under roll-pitch disturbance while ours identify nine. }

\label{fig:visualization}
\vspace{-0.3cm}
\end{figure*}