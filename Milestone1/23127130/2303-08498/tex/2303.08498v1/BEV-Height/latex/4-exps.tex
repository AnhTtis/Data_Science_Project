\section{Experiments}
We briefly introduce the experiment settings and two benchmark datasets in road-side perception domain. We then compare our proposed \name{} with state-of-the-art methods under clean and noisy camera settings. We ablate our methods in detail and discuss the limitations. 
% In this section, we first introduce our experimental setups. Then, we conduct comprehensive experiments to validate the effects of our \name{} on two large-scale roadside datasets, DAIR-V2X~\cite{yu2022dair} and Rope3D~\cite{ye2022rope3d}. Finally, we also give an analysis on the popular vehicle perception dataset, nuScenes~\cite{caesar2020nuscenes}.

\subsection{Datasets}
\mypara{DAIR-V2X.} Yu \etal~\cite{yu2022dair} introduces a large-scale, multi-modality dataset. As the original dataset contains images from vehicles and roadside units, this benchmark consists of three tracks to simulate different scenarios. Here, we focus on the DAIR-V2X-I, which only contains the images from mounted cameras to study roadside perception. Specifically, DAIR-V2X-I contains around ten thousand images, where 50\%, 20\% and 30\% images are split into train, validation and testing respectively. However, up to now, the testing examples are not yet published, we evaluate the results on the validation set. We follow the benchmark to use the average perception of the bounding box as in KITTI~\cite{geiger2012we}.

% \mypara{DAIR-V2X~\cite{yu2022dair}} is a large-scale, multi-modality vehicle-infrastructure collaborative dataset for 3D object detection, which is composed of three sub-datasets: DAIR-V2X-C, DAIR-V2X-I, DAIR-V2X-V. DAIR-V2X-C containing sensors from both vehicle and roadside is for the vehicle-infrastructure cooperative 3D object detection. DAIR-V2X-V including sensors on the vehicle is for the 3D object detection in autonomous driving. DAIR-V2X-I with only sensors on infrastructure, is designed for the roadside perception task. In this paper, we conduct experiments based on the DAIR-V2X-I. DAIR-V2X-I contains 10084 samples, which are further divided into train/val/test according to 5:2:3 respectively. Since the test set is not released yet, all experiments are trained on the train set and evaluated on val set. The evaluation metric is average precision (AP) as used in the KITTI\cite{geiger2012we} dataset.

% Rope3D
\mypara{Rope3D\cite{ye2022rope3d}.} 
There is another recent large-scale benchmark named Rope3D. It contains over 500k images with three-dimensional bounding boxes from seventeen intersections. Here, we follow the proposed homologous setting to use 70\% of the images as training, and the remaining as testing. Note that, all images are randomly sampled. For validation metrics, we leverage the AP$_{\text{3D}{|\text{R40}}}$~\cite{simonelli2019disentangling} and the Rope$_\text{score}$, which is a consolidated metric of the 3D AP and other similarities metrics, such as average area similarity.

% is a large-scale dataset dedicated to vision-based roadside 3D object detection. This dataset provides 50009 frames of images together with 3D annotations. All images are sampled from seventeen intersections and split into the train/val set according to 7:3 under the Homologous setting. The proposed AP$_{\text{3D}{|\text{R40}}}$ and Rope$_\text{score}$ are used as the metrics.

% \mypara{nuScenes\cite{HolgerCaesar2019nuScenesAM}} is a large-scale autonomous-driving dataset for 3D detection, consisting of 700, 150 and 150 scenes for training, validation, and testing, respectively. 
% Each frame contains one point cloud and six calibrated images that cover 360 fields-of-view. 
% metric
% For 3D detection, the main metrics are mean Average Precision (mAP) and nuScenes detection score (NDS). 
% The mAP is defined by the BEV center distance with thresholds of {0.5m, 1m, 2m, 4m}, instead of the IoUs of bounding boxes. 
% NDS is a consolidated metric of mAP and other metric scores, such as average translation error and average scale error.

\subsection{Experimental Settings}
For architecture details, we use ResNet-101\cite{he2016deep} as image-view encoder in results compared with state-of-the-art and ResNet-50 for other ablation studies.
The input resolution is in (864, 1536). For data augmentation, we follow \cite{li2022bevdepth} to use random scaling and rotation in the 2D space only. All methods are trained for 150 epochs with AdamW optimzer~\cite{loshchilov2017adamw}, where the initial learning rate is set to $2e-4$.
% We use ResNet\cite{he2016deep} as the image backbone. For DAIR-V2X-I and Rope3D datasets, the image size is processed to 864x1536; we adopt random translation, random scaling, and random rotation as image data augmentations. No BEV data augmentations are applied. All results are trained for 120 epochs with AdamW optimizer and learning rate set to 2e-4. 

% \begin{table*}[ht]
%  \centering\addtolength{\tabcolsep}{-0.6pt}
%  \resizebox{0.8\textwidth}{!}{
%  \begin{tabularx}{1.0\textwidth}{l|c|ccc|ccc|ccc}
%   \toprule
%  \multirow{3}{*}{Method} &  
%  \multirow{3}{*}{Modality}  
%  & \multicolumn{3}{c|}{$\text{Vehicle}_{(IoU=0.5)}$} & \multicolumn{3}{c|}{$\text{Pedestrian}_{(IoU=0.25)}$} & \multicolumn{3}{c}{$\text{Cyclist}_{(IoU=0.25)}$} \\
%     \cmidrule(r){3-11}
%      &  & Easy & Mid & Hard & Easy & Mid & Hard & Easy & Mid & Hard  \\
% \midrule

% PointPillars~\cite{lang2019pointpillars} & PointCloud &63.07 & 54.00 & 54.01 & 38.53 & 37.20 & 37.28 & 38.46 & 22.60 & 22.49 \\
% SECOND~\cite{yan2018second} & PointCloud &71.47 & 53.99 & 54.00 & 55.16 & 52.49 & 52.52 & 54.68 & 31.05 & 31.19 \\
% MVXNet~\cite{Sindagi2019MVX} & Image+PointCloud &71.04 & 53.71 & 53.76 & 55.83 & 54.45 & 54.40 & 54.05 & 30.79 & 31.06 \\
% \midrule
% ImvoxelNet~\cite{rukhovich2022imvoxelnet} &Image & 44.78 & 37.58 & 37.55 & 6.81 & 6.746 & 6.73 & 21.06 & 13.57 & 13.17 \\
% BEVFormer-R101$\ast$~\cite{li2022bevformer} & Image 	&	61.37&	50.73&	50.73&	16.89&	15.82&	15.95	&22.16&	22.13&	22.06\\
% BEVDepth-R101$\ast$~\cite{li2022bevdepth}&	Image 	&	76.01&	64.11&	64.18&	24.32&	24.96&	24.84	&46.45&	45.56&	45.69	\\

% \midrule
% BEVHeight-R101(Ours) & Image &	79.12&	67.95&	67.04&	29.85&	29.31&	29.07	&51.55&	51.39&	50.91\\
%     \bottomrule
%   \end{tabularx}
%   }
%   \caption{\textbf{Comparison on the DAIR-V2X-I val set.}}
%   \label{dair_sota}
% \end{table*}


\begin{table}[t]
 \scriptsize\centering\addtolength{\tabcolsep}{-4.2pt}
\caption{\textbf{Comparing with the state-of-the-art on the DAIR-V2X-I val set.} Here, we report the results of three types of objects, vehicle~(veh.), pedestrian~(ped.) and cyclist~(cyc.). Each object is categorized into three settings according to the difficulty defined in ~\cite{yu2022dair}. First, recent BEVDepth surpasses the previous best by a large margin, showing that using bird's-eye-view indeed helps in roadside scenarios. Our method outperforms the BEVDepth by over 3\% in average precision and constitutes state-of-the-art. It is surprising to see that our method outperforms those relying on LiDAR modality.
}
% \vspace{-0.2cm}

 \begin{tabularx}{1.0\linewidth}{l|c|ccc|ccc|ccc}
  \toprule
 \multirow{3}{*}{Method} &  
 \multirow{3}{*}{M}  
 & \multicolumn{3}{c|}{$\text{Veh.}_{(IoU=0.5)}$} & \multicolumn{3}{c|}{$\text{Ped.}_{(IoU=0.25)}$} & \multicolumn{3}{c}{$\text{Cyc.}_{(IoU=0.25)}$} \\
    \cmidrule(r){3-11}
     &  & Easy & Mid & Hard & Easy & Mid & Hard & Easy & Mid & Hard  \\
\midrule
PointPillars~\cite{lang2019pointpillars} & L &63.07 & 54.00 & 54.01 & 38.53 & 37.20 & 37.28 & 38.46 & 22.60 & 22.49 \\
SECOND~\cite{yan2018second} & L &71.47 & 53.99 & 54.00 & 55.16 & 52.49 & 52.52 & 54.68 & 31.05 & 31.19 \\
MVXNet~\cite{Sindagi2019MVX} & LC &71.04 & 53.71 & 53.76 & 55.83 & 54.45 & 54.40 & 54.05 & 30.79 & 31.06 \\
\midrule
ImvoxelNet~\cite{rukhovich2022imvoxelnet} &C & 44.78 & 37.58 & 37.55 & 6.81 & 6.746 & 6.73 & 21.06 & 13.57 & 13.17 \\
BEVFormer~\cite{li2022bevformer} & C 	&	61.37&	50.73&	50.73&	16.89&	15.82&	15.95	&22.16&	22.13&	22.06\\
BEVDepth~\cite{li2022bevdepth}&	C 	&	
75.50&	63.58&	63.67&	34.95&	33.42&	33.27& 55.67&	55.47&	55.34\\
\midrule
 \rowcolor{cyan!30} BEVHeight & C &	
 \textbf{77.78}&	\textbf{65.77}&	\textbf{65.85}&	\textbf{41.22}&	\textbf{39.29}&	\textbf{39.46}	&\textbf{60.23}&	\textbf{60.08}&	\textbf{60.54}\\
    \bottomrule
\multicolumn{8}{l}{\scriptsize{M, L, C denotes modality, LiDAR, camera respectively.}}
  \end{tabularx}
  \vspace{-0.5cm}
  \label{dair_sota_2}
\end{table}






\subsection{Comparing with state-of-the-art}
\mypara{Results on the original benchmark.} On DAIR-V2X-I setting, we compare our BEVHeight with other state-of-the-art methods like ImvoxelNet~\cite{rukhovich2022imvoxelnet}, BEVFormer~\cite{li2022bevformer}, BEVDepth~\cite{li2022bevdepth} on DAIR-V2X-I val set. Some results of LiDAR-based and multimodal methods reproduced by the original DAIR-V2X~\cite{yu2022dair} benchmark are also displayed.
As can be seen from Tab.~\ref{dair_sota_2}, the proposed \name{} surpasses state-of-the-art methods by a significant margin of 2.19\%, 5.87\% and 4.61\% in vehicle, pedestrian and cyclist categories respectively.

% \mypara{Rope3D val set.}
On Rope3D dataset, we also compare our BEVHeight with other leading BEV methods, such as BEVFormer~\cite{li2022bevformer} and BEVDepth~\cite{li2022bevdepth}. Some results of the monocular 3D object detectors are revised by adapting the ground plane. As shown in Tab.~\ref{tab_performance_overall},  
we can see that our method outperforms all BEV and monocular methods listed in the table. In addition, under the same configuration, our BEVHeight outperforms the BEVDepth by $4.97\%$ / $4.02\%$, $3.91\%$ / $3.06\%$ on AP$_{\text{3D}{|\text{R40}}}$ and Rope$_\text{score}$  for car and big vehicle respectively.

% \begin{table*}[ht]
%  \footnotesize \centering\addtolength{\tabcolsep}{-1pt}
 
%  \begin{tabularx}{1.0\textwidth}{ l |cc|cc|cc|cc|cc }
% \toprule
% \multirow{4}{*}{Method} & \multicolumn{2}{c|}{\multirow{2.8}{*}{Resolution}}  & \multicolumn{4}{c|}{IoU = 0.5} & \multicolumn{4}{c}{IoU = 0.7} \\ 
% \cmidrule(r){4-11}
% &  &   & \multicolumn{2}{c|}{Car} & \multicolumn{2}{c|}{Big Vehicle} & \multicolumn{2}{c|}{Car} & \multicolumn{2}{c}{Big Vehicle} \\ 
% \cmidrule(r){2-11}
% & Input &  BEV & AP$_{\text{3D}{|\text{R40}}}$ & Rope$_\text{score}$ & AP$_{\text{3D}{|\text{R40}}}$ & Rope$_\text{score}$  & AP$_{\text{3D}{|\text{R40}}}$ & Rope$_\text{score}$ & AP$_{\text{3D}{|\text{R40}}}$ & Rope$_\text{score}$\\ 

% \midrule

% M3D-RPN-R34~\cite{brazil2019m3d} & -&- 
% &54.19 & 62.65	&33.05 &  44.94 &16.75 & 32.90 &6.86  &  24.19 \\

% Kinematic3D-Dense121~\cite{brazil2020kinematic} & - & -  &50.57  & 58.86&	37.60&  48.08 &17.74  & 32.9 &   6.10&   22.88\\

% MonoDLE-DLA34~\cite{ma2021delving} & -&- 
%  & 51.70 & 60.36 & 40.34 & 50.07  & 13.58 & 29.46 &9.63 &25.80\\


% MonoFlex-DLA34~\cite{zhang2021objects} & -& 	- & 60.33 & 66.86&	37.33 &47.96   & 33.78 & 46.12 &  10.08 &26.16\\

% BEVFormer-R101~\cite{li2022bevformer} &	864x1536&	150x150	&50.62&	58.78&	34.58&	45.16&	24.64&	38.71	&10.05&	25.56\\

% BEVDepth-R50~\cite{li2022bevdepth}&	864x1536&	256x256	&69.63&	74.70&	45.02&	54.64&	42.56&	53.05	&21.47	&35.82\\

% \midrule
% BEVHeight-R50(Ours)&	864x1536&	256x256	& 74.60& 78.72& 48.93& 57.70& 45.73& 55.62& 23.07& 37.04 \\							
% \bottomrule

% \end{tabularx}

% % \caption{Overall performance of the monocular 3D object detection approaches on the Rope3D Dataset with IoU = 0.5 and 0.7. $(G)$ denotes adapting the ground plane.}
% \caption{\textbf{Comparison on the Rope3D val set under the Homologous setting.}}
% \label{tab_performance_overall}
% \end{table*}

\begin{table}[t]
\footnotesize
  \centering\addtolength{\tabcolsep}{-3.8pt}
\caption{\textbf{Results on the Rope3D val set.} Here, we follow~\cite{ye2022rope3d} to report the results on vehicles. Our method on average surpasses the state-of-the-art method over a margin of 3\% in both average precision and $Rope_{score}$ metric.
% \Tao{more caption}
}
\vspace{-0.2cm}
 \begin{tabularx}{1.\linewidth}{ l |cc|cc|cc|cc }
\toprule
\multirow{4}{*}{Method}   & \multicolumn{4}{c|}{IoU = 0.5} & \multicolumn{4}{c}{IoU = 0.7} \\ 
\cmidrule(r){2-9}
  & \multicolumn{2}{c|}{Car} & \multicolumn{2}{c|}{Big Vehicle} & \multicolumn{2}{c|}{Car} & \multicolumn{2}{c}{Big Vehicle} \\ 
\cmidrule(r){2-9}
&AP & Rope &
AP & Rope &
AP & Rope &
AP & Rope \\
% &AP$_{\text{3D}{|\text{R40}}}$ & Rope$_\text{score}$ & AP$_{\text{3D}{|\text{R40}}}$ & Rope$_\text{score}$  & AP$_{\text{3D}{|\text{R40}}}$ & Rope$_\text{score}$ & AP$_{\text{3D}{|\text{R40}}}$ & Rope$_\text{score}$\\ 

\midrule

M3D-RPN~\cite{brazil2019m3d} 
&54.19 & 62.65	&33.05 &  44.94 &16.75 & 32.90 &6.86  &  24.19 \\

Kinematic3D~\cite{brazil2020kinematic}  &50.57  & 58.86&	37.60&  48.08 &17.74  & 32.9 &   6.10&   22.88\\

MonoDLE~\cite{ma2021delving} 
 & 51.70 & 60.36 & 40.34 & 50.07  & 13.58 & 29.46 &9.63 &25.80\\


MonoFlex~\cite{zhang2021objects} & 60.33 & 66.86&	37.33 &47.96   & 33.78 & 46.12 &  10.08 &26.16\\

BEVFormer~\cite{li2022bevformer}	&50.62&	58.78&	34.58&	45.16&	24.64&	38.71	&10.05&	25.56\\

BEVDepth~\cite{li2022bevdepth}	&69.63&	74.70&	45.02&	54.64&	42.56&	53.05	&21.47	&35.82\\

\midrule
 \rowcolor{cyan!30} BEVHeight & \textbf{74.60}& \textbf{78.72}& \textbf{48.93}& \textbf{57.70}& \textbf{45.73}& \textbf{55.62}& \textbf{23.07}& \textbf{37.04} \\							
\bottomrule
\multicolumn{9}{l}{\footnotesize{AP and Rope denote AP$_{\text{3D}{|\text{R40}}}$ and Rope$_\text{score}$ respectively.}}
\end{tabularx}
% }
% \caption{Overall performance of the monocular 3D object detection approaches on the Rope3D Dataset with IoU = 0.5 and 0.7. $(G)$ denotes adapting the ground plane.}
 \vspace{-0.50cm}
\label{tab_performance_overall}
\end{table}




% 地面方程咋用的

\mypara{Results on noisy extrinsic parameters.}
% To verify the robustness of our BEVHeight when the camera's extrinsic matrix is changed inevitably. 
In the realistic world, camera parameters frequently change for various reasons. Here we evaluate the performance of our framework in such noisy settings. We follow \cite{yu2022benchmarking} to simulate the scenarios that external parameters are changed. Specifically, we introduce a random rotational offset in normal distribution $N(0, 1.67)$ along the roll and pitch directions as the mounting points usually remain unchanged.  
% \KY{degree of what?}

% This situation often occurs during the maintenance of roadside cameras. In this case, the camera's extrinsic matrix will differ from its previous state when the labeled data is collected. The generalization to the camera's mount position disturbance is a great challenge for the existing methods.
During the evaluation, we add the rotational offset along roll and pitch directions to the original extrinsic matrix. The image is then applied with rotation and translation operations to ensure the calibration relationship between the new external reference and the new image. Examples are given in Sec.~\ref{sec:visualization_results}.
As shown in Tab.~\ref{dair_robust}, the performance of the existing methods degrades significantly when the camera's extrinsic matrix is changed. Take $\text{Vehicle}_{(IoU=0.5)}$ for example, the accuracy of BEVFormer~\cite{li2022bevformer} drops from 50.73\% to 16.35\%. The decline of BEVDepth~\cite{li2022bevdepth} is from 60.75\% to 9.48\%, which is pronounced. Compared with the above methods, Our BEVHeight maintains 51.77\% from the original 63.49\%, which surprises the BEVDepth by 42.29\% on vehicle category.

\mypara{Visualization Results.}
\section{Visualization On Demand} %Visualization Elements
\label{sec:visrisk}
Based on environment data and trajectory evaluation, we now present ways of communicating the situation and risks on a visual display to achieve an ADAS.
In this context, we employ a renderer that visualizes all the information in a joint Cartesian coordinate system (see section \ref{subsec:sim}). 
Once driving risks are detected, design elements are overlayed on the display with section \ref{subsec:active} and section \ref{subsec:warning}. 

\subsection{Simulator Environment}
\label{subsec:sim}
Nodes of the R-LDM have a range of potential attributes, such as the 3D position or geometrical shape of objects. 
% For instance, the road centerline is a polyline with bounderies to the left and right. Crosswalks have a defined width and buildings a polygonal outline description. 
In the renderer, we always visualize static and quasi-static data that lie in the field of view from the ego vehicle. 
For this, a local 3D model is generated by converting geographic points with (lat, lon, alt) into Cartesian coordinates of (x, y, z). 
% and project the positonal relations from a view perspective with a transformation matrix. 
Fig. \ref{fig:3Dsimulator} depicts an exemplary map section having several intersections in bird's-eye view.
% with several intersections, stop lines and crosswalks. 
On the top right, the first person view of a vehicle approaching a crosswalk is shown. 

The dynamic data is then added to this static view. A zoomed-in excerpt from the map is given at the bottom of Fig. \ref{fig:3Dsimulator} that includes a recorded GNSS trace (red).
We project the trace onto the connected lane center, which is pictured in green. 
% Because we project the ego position on the closest lane segment, on the bottom right the measured trace is changed in red and the aligned trace is marked in green.
Consequently, the virtual horizon and its possible paths are retrieved as described in section \ref{subsec:ldm}. 
We can lastly update and move the excerpt with the current position from the GNSS to obtain a live simulation.

\subsection{Proactive Support}
\label{subsec:active}
Communication of spatial as well as spatio-temporal relations is crucial for risk-averse driver support. 
% This has the reason that humans can estimate the time better than positions (especially for risks). 
% Velocity contains implicitly the time as well. 
Further sources of information are cause, likelihood and severity of a potential risks.  
% if a collision happens. 
The next step for RNS is the choice of suitable design elements. 
In this process, we suppose that we know where the ego vehicle is driving (i.e., the ego path) from its navigation route. 
Yet, for surrounding vehicles, all paths are considered.

\subsubsection{Hazard Route Element}
The so-called hazard route in Fig. \ref{fig:charts} is a concept that consists of a scale portraying distances to an upcoming risk element.
Furthermore, the geometrical area or length of risks is considered.
Risk is thus measured with respect to the ego path, ranging from the current position  $\Delta l \hspace{-0.03cm}=\hspace{-0.03cm} \unit[0]{m}$ to the end of the path $\Delta l_{h}$.
Here, the length $\Delta l_{h}$ can be chosen according to own preferences. 

At an upcoming intersection, risk is defined by the section of the path that lies within the junction.
Since risk corresponds to exposition time, we encode the path part from the intersection $I_z$ with a color, ranging from green for short intersections to red for long ones. 
%allgemein risiko entlang des pfades zu intersection zone
%share of junction segment to navigation route + 
%one case with large intersection far and one case with small intersection close
Fig. \ref{fig:charts}~a) gives two examples of the hazard route.
The left bar shows a large intersection (e.g. multi-lane four-way stop) in vicinity and the right bar has a small and consecutive medium junction. 
% In the case of collision risk, the intersection zone $I_z$ can be used.
% Depending on the value of $I_z$ (low, medium and large), the area is marked from green, to yellow until red for conveying the criticality. 
This emphasizes that we may include more than one intersection in our warnings.

\begin{figure}[t]
  \centering
  \includegraphics[width=0.95\linewidth]{./img/simulator.png}
  \caption{Rendered road network from two perspectives with the ego position being projected on the navigation route. \vspace{0.45cm}}
  \label{fig:3Dsimulator}
\end{figure}

\begin{figure}[t]
  \centering
  \resizebox{\linewidth}{!}{
  \import{img/}{velocity_scale_new.pdf_tex}}  
  \caption{Chart elements for proactive support. Hazard route (left) and velocity scale (right).} %\vspace{-0.3cm}}
  \label{fig:charts} 
\end{figure} 

\subsubsection{Velocity Scale Element}
The velocity scale, Fig. \ref{fig:charts}~b), is a second chart element which qualifies the difference between the current velocity of the vehicle $v_0$ and the target velocity $v_{\text{tar}}$ from the trajectory evaluation of section \ref{subsec:trajeval}. 
The scale shows possible velocity values, from standstill $v\hspace{-0.05cm}=\hspace{-0.05cm}\unit[0]{m/s}$ to a maximal velocity $v_{\text{max}}$. Depending on the difference $|v_0 \hspace{0.05cm} - \hspace{0.05cm} v_{\text{tar}}|$, the situation is rated as safe with $v_0 \hspace{-0.042cm} \approx \hspace{-0.042cm} v_{\text{tar}}$ (green, left), as dangerous with e.g. $v_0 \hspace{-0.05cm} < \hspace{-0.05cm} v_{\text{tar}}$ (yellow, middle) to critical with $v_0 \hspace{-0.07cm} \ll \hspace{-0.07cm} v_{\text{tar}}$ (red, right). The same cases hold true for the opposite circumstances, i.e., $v_0 \hspace{-0.032cm} > \hspace{-0.032cm} v_{\text{tar}}$. 
This velocity scale can be employed for curve or regulatory risks. 
Moreover, we may set an enforced speed limit as the target velocity $v_{\text{tar}}$ for proactive behavior, once there is no risk ahead. 
%\noindent -Warning vs behavior support \\
%-Ghost vehicle as in game \\

\subsection{Short-Term Warning Elements}
\label{subsec:warning}
In order to emphasize the criticality of the situation, we propose to add further intuitive warning elements as e.g. pop-up signs and lane colorings. 
The following elements augment the proactive elements.

\subsubsection{Pop-up Signs}
Explicit symbols indicate the risk cause accompanied with the event time for collisions ($s_E$), distances to the risk spot for turns (i.e., right curve with $d_r$ and left curve with $d_l$) or stopping distance for crosswalks ($d_c$). In Fig. \ref{fig:popups}~a), the pop-up signs are pictured. 
% Besides the velocity difference, the risk type is an indication for the severity of the situation.
%Examples for collision risk are car-to-car crash., curve risk can be  as a single-car accident and regulatory risks will be a car-to-object collision. 
We want to stress that this is just a selection and more risk causes can be added. 
The purpose is also to clarify the reason for the warning and give more human-understandable information.

\subsubsection{Colored Events}
Finally, we highlight lane parts or positions according to the corresponding risks.  
% the determined color rating from the hazard route and velocity scale and relate the risks to the simulator environment. 
In the instance of curve and regulatory risk, the lane is colored from the ego position up to the point of maximal risk. 
For collision risk, we mark the point of the closest encounter as a red cube.
An illustration for regulatory risk induced from a stop line is depicted in Fig. \ref{fig:popups}~b). Again, the color is defined by the deviation $|v_0-v_{\text{tar}}|$. It also shows the therein considered navigation route with length $\Delta l_h$ and another unlikely path. 

It should be noted that the visualization of warnings only occurs if the risks are actually present. 
%\textcolor{red}{improve language, repeat intersection zone and navigation route}
%eingrauen unlikely paths and navigation path and describe in text, maybe delete Iz -> put line from unlikely path to green arrow
Altogether, the RNS provides a variety of tools to analyze and circumvent critical situations in intersection scenarios, while not overloading the driver's awareness.

\begin{figure}[t]
  \centering
  \resizebox{\linewidth}{!}{
  \import{img/}{colored_lane_new.pdf_tex}}  
  \vspace{-0.53cm}
  \caption{Short-term warning elements. Selected pop-up warnings (left) and colored lane (right).}
  \label{fig:popups} 
\end{figure} 


\label{sec:visualization_results}
As shown in Fig.~\ref{fig:visualization}, we present the results of BEVDepth~\cite{li2022bevdepth} and our BEVHeight in the image view and BEV space, respectively. The above two models are not applied with data augmentations in the training phase. From the samples in (a), we can see that the predictions of BEVHeight fit more closely to the ground truth than that of BEVDepth. As for the results in (b), under the disturbance of roll angle, there is a remarkable offset to the far side relative to the ground truth in BEVDepth detections. In contrast, the results of our method are still keeping the correct position with ground truth.  Moreover, referring to the predictions in (c), BEVDepth can hardly identify far objects, but our method can still detect the instance in the middle and long-distance ranges and maintain a high IoU with the ground truth.

\begin{table}[t]
 \scriptsize\centering\addtolength{\tabcolsep}{-3.25pt}
   \caption{
%   \textbf{The robustness under extrinsic parameter perturbations on DAIR-V2X-I.} \Tao{more caption}
    \textbf{Results on robustness settings. } Here, we simulate the robustness scenarios where the external parameters of the camera changes. Consider the  Specifically, we consider two degrees of freedom mutation, roll and pitch of the camera center. In both dimensions, we randomly sample angles from a normal distribution of $\mathcal{N}(0, 1.67)$. Surprisingly, given such minor changes, traditional depth-based methods decrease to under 15\% even for those vehicles under easy settings. On the contrary, our methods achieve around 577\% improvement compared to those baselines, evidencing the robustness of \name{}.
%   “roll” and “pitch” means applying an additional rotation offset in normal distribution N(0, 1.67) to the camera’s extrinsic matrix along roll and pitch directions
   }
  %\vspace{-0.1cm}
 \begin{tabularx}{1.0\linewidth}{l|cc|ccc|ccc|ccc}
  \toprule
 \multirow{3}{*}{\rotatebox{90}{Model}} & \multicolumn{2}{c|}{Disturbed}
    \ & \multicolumn{3}{c|}{$\text{Veh.}_{(IoU=0.5)}$} & \multicolumn{3}{c|}{$\text{Ped.}_{(IoU=0.25)}$} & \multicolumn{3}{c}{$\text{Cyc.}_{(IoU=0.25)}$}  \\
   \cmidrule(r){2-12}
   & roll &	pitch & Easy & Mid & Hard & Easy & Mid & Hard & Easy & Mid & Hard  \\
  
    \midrule
 \multirow{4}{*}{\rotatebox{90}{BEVFormer}} &&&	61.37&	50.73	&50.73&	16.89&	15.82&	15.95	&22.16&	22.13&	22.0\\
	&	\checkmark& 	&	50.65&	42.9&	42.95&	10.16&	9.41&	9.47&	13.62&	13.71&	13.08\\
	&		&\checkmark&	46.40&	38.26&	38.37&	9.12	&8.44&	8.55&	8.99 &	8.43&	8.42 \\
 	&	\checkmark&	\checkmark&	19.24&	16.35&	16.47&	3.93&	3.43&	3.52&	4.93&	4.98&	4.98\\
\midrule
 \multirow{4}{*}{\rotatebox{90}{BEVDepth}}	& & &			71.56& 	60.75&	60.85&	21.55&	20.51&	20.75&	40.83	&40.66&	40.26\\
&	\checkmark	&&	34.82&	28.32&	28.35&	4.49&	4.36&	4.39&	10.48&	9.51&	9.73\\
	&&	\checkmark&	14.04&	11.41&	11.49&	3.01&	2.67&	2.75&	6.43&	6.23&	6.83\\
	&	\checkmark &\checkmark &	11.84&	9.48&	9.54&	2.16&	1.84&	1.89&	4.31&	4.14&	4.26\\
\midrule
 \multirow{4}{*}{\rotatebox{90}{BEVHeight}}	&	&&	75.58&	63.49&	63.59&	26.93&	25.47&	25.78&	47.97	& 47.45	& 48.12	\\
	&	\checkmark &&	66.06&	54.99&	55.14&	18.66&	17.63&	17.78&	34.45&	26.93&	27.68\\
	&&	\checkmark&	68.49&	56.98&	57.11&	17.94&	16.87&	17.09&	34.48&	27.82&	28.67\\
	&	\checkmark &	\checkmark&	62.64&	51.77&	51.9&	14.38&	14.01&	14.09&	31.28&	25.24&	26.02\\

    \bottomrule
    % \multicolumn{21}{l}{Mid: Middle, Veh.: Vehicle, Ped.: Pedestrian, Cyc.: Cyclist.}
  \end{tabularx}

  \label{dair_robust}
\end{table}

\section{A test-space only discretization}

As indicated in the previous section, we propose to use the normal equation~\eqref{eq:transport:continuousNormalEq} to define an optimally stable approximation scheme.
It is thus obvious, that a discretization can be fully based on a discrete approximate test space $\ycal^\delta$.

\subsection{Discrete normal equation and functional reconstruction}

Let $\ycal^\delta \subseteq \ycal$ be a conforming discretization of the optimal test space (\ie using a standard Lagrange finite element space). Based on~\eqref{eq:transport:continuousNormalEq} we then define the \textit{discrete} normal equation using Galerkin-projection.
\begin{equation}\label{eq:discreteNormalEq}
\text{Find}\; w^\delta \in \ycal^\delta: \quad {(A^*[w^\delta], A^*[v^\delta])}_{L^2(\Omega)} = f(v^\delta) \qquad \forall v^\delta \in \ycal^\delta.
\end{equation}
Note that this is still an optimally conditioned problem. Given the discrete solution $w^\delta$ we may reconstruct the discrete solution $u^\delta = A^*[w^\delta]$. Technically, this solution lies in the finite-dimensional subspace $\xcal^\delta := A^*[\ycal^\delta] \subseteq \xcal$, however, due to its non-accessible structure this space is of no practical use.

Previous work often used knowledge of the structure of $A^*$ to determine a larger, more traditional (DG-)space $\zcal^\delta \supsetneq \xcal^\delta$ and then assembled the matrix $\underline{A}$ representing the operator $A^*: \xcal^\delta \rightarrow \zcal^\delta$ in the respective standard FE-bases. In this case, one can determine the system matrix $\underline{A}^{NE}$ of the normal equation as $\underline{A}^{NE} = \underline{A}^T \underline{M}_\zcal \underline{A}$ (where $\underline{M}_{\zcal}$ denotes the inner-product matrix in $\zcal^\delta$), solve the linear system\vspace*{-0.5em}
\begin{equation}\label{eq:linearEquationSystem}
  \underline{A}^{NE}\underline{w} = \underline{f}\vspace*{-0.5em}
\end{equation}
and compute the coefficients $\underline{u}$ of $u^\delta \in\xcal^\delta \subset \zcal^\delta$ in the basis of $\zcal^\delta$ by simply computing $\underline{u} = \underline{A}\,\underline{w}$.

However, this is suboptimal as the construction of a discrete larger space $\zcal^\delta$ is only feasible or even possible with suitable additional assumptions on the data, e.g. (elementwise) constant data functions. For non-constant reaction or velocities one has to resort to a nonconforming choice $\zcal^\delta \not\supset \xcal^\delta$ introducing an additional projection error which might be difficult to estimate or control.

Here, we propose an approach that avoids ever computing a matrix $\underline{A}$ representing the operator $A^*$. The system matrix of the normal equation $\underline{A}^{NE}$ can also be directly assembled in a basis of $\ycal^\delta$ which means basically assembling a normal equation using the full infinite dimensional trial space $\xcal$. The reconstruction $u^\delta := A^*[w^\delta]$ is now seen as an element of $\xcal$ (we technically know that it lies in the finite dimensional subspace $\xcal^\delta \subset \xcal$ but this does not give us any useful information). The crucial insight is that in almost all applications only functional evaluations of $u^\delta$ are needed. Examples include point-evaluations for the visualization of $u^\delta$ or the computation of quantities of interest via numerical quadrature (\ie $\norm{u^\delta}$). Therefore, we replace the reconstruction by functional evaluations and e.g. do a pointwise reconstruction. Note that in this way we do not introduce any additional projection error.

\subsection{Conditioning of the system matrix and solving the linear system}
Solving the linear equation system~\eqref{eq:linearEquationSystem} is actually quite a challenging task - a problem that has to our knowledge not been discussed so far. Although Problem~\eqref{eq:discreteNormalEq} is optimally stable in theory, the condition of the system matrix $\underline{A}^{NE}$ still scales quadratically in the inverse grid width $h^{-1}$ and is thus a significant challenge even for moderately large problems. To better understand these seemingly conflicting statements consider the non-symmetric formulation of~\eqref{eq:discreteNormalEq}:
\begin{equation}\label{eq:discreteNonsymmetricEq}
\text{Find}\; u^\delta \in \xcal^\delta: \quad (u^\delta, A^*[v^\delta]) = f(v^\delta) \qquad \forall v^\delta \in \ycal^\delta.
\end{equation}
Let $\{\psi_i\}_{i=1}^{N}$ be a basis of $\ycal^\delta$ (\eg a finite element basis). Then, the set $\{ \varphi_i \}_{i=1}^N$, $\varphi_i := A^*[\psi_i]$ forms a basis of $\xcal^\delta$ and the matrix $\underline{A}$ representing $A^*$ in these bases is the identity matrix. The condition of the system matrix $\underline{A}^{NE}$ is still of order $\mathcal{O}(h^{-2})$ since the trial functions $\varphi_i$ have, contrary to classic finite elements, in this case a magnitude of $\mathcal{O}(h^{-1})$.

As mentioned in Remark~\ref{rmk:strongTestProblem}, the normal equation can also be seen as the weak form of a specific Poisson-problem with rank-deficient diffusion tensor $D$. In the following numerical experiments we thus employed an algebraic multigrid for preconditioning and a conjugate gradient (CG) solver - methods that are known to perform well for this type of problems. For more complex problems (\eg for velocity fields with (locally) small magnitude) the efficient preconditioning and solving of the linear equation system~\eqref{eq:linearEquationSystem} still needs further investigation.

\subsection{Ablation Study}
% \mypara{Robust to Extrinsic Disturbance.}

\mypara{Dynamic Discretization.}
Experiments in Tab.~\ref{dair_discretization} show the detection accuracy improvement 0.3\% - s1.5.0\% when our dynamic discritization is applied instead of uniform discretization(UD).
The performance when hype-parameter $\alpha$ is set to 2.0 suppresses that of 1.5 in most cases, which signifies that hype-parameter $\alpha$ is necessary to achieve the most appropriate discretization.

\mypara{Analysis on Point Cloud Supervision.}
\begin{table}[t]
 \scriptsize\centering\addtolength{\tabcolsep}{-2.9pt}
 \caption{\textbf{Results with point cloud supervision on DAIR-V2X-I dataset.} We can observe that for both BEVDepth and BEVHeight, LiDAR point cloud supervision did not help in terms of evaluation results. This is another evidence that road-side perception is different from the ego-vehicle one.  }
% \vspace{-0.2cm}
 \begin{tabularx}{1.0\linewidth}{l|ccc|ccc|ccc}
  \toprule
 \multirow{2}{*}{Method}    & \multicolumn{3}{c|}{$\text{Veh.}_{(IoU=0.5)}$} & \multicolumn{3}{c|}{$\text{Ped.}_{(IoU=0.25)}$} & \multicolumn{3}{c}{$\text{Cyc.}_{(IoU=0.25)}$}  \\
   \cmidrule(r){2-10}
 & Easy & Mid & Hard & Easy & Mid & Hard & Easy & Mid & Hard  
 \\
    \midrule
    BEVDepth	& 71.56& 	60.75&	60.85&	21.55&	20.51&	20.75&	40.83	&40.66&	40.26\\
    BEVDepth$\dagger$	&	71.09&	60.37&	60.46&	21.23&	20.84&	20.85&	40.54&	40.34&	40.32\\
    \midrule
    BEVHeight	 &	75.58&	63.49&	63.59&	26.93&	25.47&	\textbf{25.78}&	47.97	& 47.45	& 48.12	\\
    BEVHeight$\dagger$	& \textbf{75.64}& \textbf{63.61}&	\textbf{63.72}&	\textbf{27.01}&	\textbf{25.55}&	25.34&	\textbf{48.03}&	\textbf{47.62}&	\textbf{48.19}\\
    \bottomrule
    \multicolumn{10}{l}{\scriptsize{$\dagger$ denotes training with PointCloud supervision.}}
  \end{tabularx}
  \label{pc_sup}
\vspace{-0.55cm}
\end{table}

To verify the effectiveness of point cloud supervision in roadside scenes, we conduct ablation experiments on both BEVDepth~\cite{li2022bevdepth} and our method. As shown in Tab.~\ref{pc_sup}, BEVDepth with point cloud supervision is slightly lower than that without supervision. As for our BEVHeight, although there is a slight improvement under the IoU=0.5 condition, the overall gain is not apparent. This can be explained by the fact that the background in roadside scenarios is stable. These background point clouds fail to provide adequate supervised information and increase the difficulty of model fitting.
% there is only a slight improvement under IoU=0.5. We speculate this is because the camera is fixed in roadside scenarios, thus the majority of the pixels belong to the background and have relatively fixed depth or height values. The network can learn them well even without ground truth as supervision.
% We speculate that this is due to the fact that the background in roadside scenarios is stable. 
% These background point clouds remain unchanged and fail to provide adequate supervised information
% and increase the difficulty of model fitting.
% 分析,depth的问题,猜测不同路口混合训练的影响,加了监督更加hard让网络拟合了

% \mypara{Analysis on point cloud supervision.}

\mypara{Analysis on Distance Error.}
To provide a qualitative analysis of depth and height estimations, we convert depth and height to the distance between the predicted object's center and the camera’s coordinate origin, as is shown in Fig.~\ref{fig:distance_correlation}.  Compared with the distance error triggered by depth estimation in BEVDepth\cite{li2022bevdepth}, the height estimation in our BEVHeight introduces less error, which illustrates the superiority of height estimation over the depth estimation in the roadside scenario.
\begin{figure}[t!]
	\centering
	\includegraphics[width=8.5cm]{BEV-Height/figures/distance_correlation.pdf}
	\caption{\textbf{Empirical analysis of the distance correlation.} All experiments are conducted on the DAIR-V2X-I val set. (a) and (b) reveal the distance correlation between ground truth and predicted distance on the BEVDepth and our BEVHeight. We take distances from the camera's coordinate system origin to the annotated objects' center for consideration. Each point represents an annotated instance. The scatter diagram of BEVHeight in (b) is closer to the diagonal than that of BEVDepth in (a), indicating that the distance error triggered by height estimation is more minimal than the depth candidate.}

\label{fig:distance_correlation}
\vspace{-0.5cm}
\end{figure}

\mypara{Latency.}
As shown in Tab.~\ref{latency_rebuttal}, we benchmark the runtime of BEVHeight and BEVDepth. With an image size of 864x1536, BEVDepth runs at 14.7 FPS with a latency of 68ms, while ours runs at 16.1 FPS with 62ms, which is around 5\% faster. It is due to the depth range (1$\sim$104m) being much larger than height (-1$\sim$1m), thus ours has 90 height bins that less than 206 depth ones,
leading to a slimmer regression head and fewer pseudo  points for voxel pooling. It evidences the superiority of predicting height instead of depth and advocates the efficiency of our method.
\begin{table}[h!t]
\scriptsize\centering\addtolength{\tabcolsep}{-2.0pt}
\renewcommand\arraystretch{1.0}
\caption{{\bf Latency of BEVHeight and BEVDepth.} }
\begin{tabular}{l|c|c|c|c|c}
\toprule   
Methods& Backbone &Range & Number of bins & Latency (ms) & FPS \\ 
\midrule
BEVDepth~\cite{li2022bevdepth} & R50 & 1 - 104m& 206& 82& 12.2\\
\rowcolor{cyan!30} BEVHeight& R50 & -1 - 1m&  90& 77& 13.0\\
\midrule
    BEVDepth~\cite{li2022bevdepth} &  R101& 1 - 104m& 206& 68& 14.7\\
\rowcolor{cyan!30}	BEVHeight  & R101& -1 - 1m&  90& 62& 16.1\\
\bottomrule 
\multicolumn{6}{l}{\scriptsize Measured on a V100 GPU. Image shape 864×1536.}
\end{tabular}
\label{latency_rebuttal}
\vspace{-0.25cm}
\end{table}



\mypara{Limitations and Analysis.}
Though the motivation of our work is to address the challenges in the roadside scenarios, we nonetheless benchmark our methods on nuScenes to study the effectiveness. Here, the input resolution is set to (256, 704). We follow the setting of BEVDepth, i.e. the training lasts for 24 epochs. Note that, we did not use other tricks such as class-balanced grouping and sampling~(CBGS) strategy~\cite{zhu2019cbgs}, exponential moving average or multi-frame fusion. 
In \cref{tab:nus}, we observe that our method falls behind the BEVDepth by around 0.02 in mAP metrics. This shows that our method has limited performance on ego-vehicle settings. 

% \Tao{256x704 128x128 pc sup}
% \KY{to finish}
% For the nuScenes\cite{HolgerCaesar2019nuScenesAM} dataset, the input image is scaled to 256x704; we adopt the same dataset augmentations on image-view and BEV features as in BEVDepth\cite{li2022bevdepth}. All experiments are trained for 24 epochs without using the CBGS strategy, EMA, and multi-frame fusion.

\begin{table}[!t]
 \centering\addtolength{\tabcolsep}{-4.15pt}
\footnotesize
\caption{\textbf{Limitation of our method.} We present the results on the nuScenes validation dataset. We notice that our methods fall behind the traditional BEVDepth on the ego-vehicle settings by 2\%. This shows that our methods are effective on cameras with high installation and bird's-eye-view as in the roadside scenario, and is not ideal on cameras mounted on ego-vehicles.}
% \vspace{-0.1cm}
\begin{tabularx}{1.0\linewidth}{l|cccccccccc}
\toprule
 Method  &
			 mAP$\uparrow$ & NDS$\uparrow$ & mATE$\downarrow$ & mASE$\downarrow$  & mAOE$\downarrow$ & mAVE$\downarrow$ & mAAE$\downarrow$   \\
\midrule
\textcolor{gray!80}{BEVDepth}	&\textcolor{gray!80}{	0.315}&	\textcolor{gray!80}{0.367}&	\textcolor{gray!80}{0.702}&\textcolor{gray!80}{	0.271}&	\textcolor{gray!80}{0.621}&	\textcolor{gray!80}{1.042}&\textcolor{gray!80}{	0.315}\\
BEVDepth*	&		0.313&	0.354&	0.713&	0.280&	0.655&	1.230	&0.377\\
\midrule
BEVHeight	&	0.291&	0.342&	0.722&	0.278&	0.674&	1.230&	0.361\\
\bottomrule
\multicolumn{10}{l}{\footnotesize{* denotes the results we reproduce.}}
\end{tabularx}
\label{tab:nus}
\vspace{-0.60cm}
\end{table}

Firstly, our method does \emph{not} assume the ground-plane is fixed, and it is not the reason why our method cannot surpass the depth-based one on ego-vehicle settings. To verify, we collect around 13 thousand sequences from the camera mounted on a moving truck with a ground height of 3.14m, and annotate the 3D object box following nuScenes. As shown in Tab.~\ref{ddd_rebuttal} We observe that our BEVHeight again surpasses the depth-based state-of-the-art by a large margin, evidences the performance is affected by the camera height but not time-varying ground plane and it can work on ego-vehicle settings.
We visualize three cameras observing the same object and analyze the detection error in Fig.~\ref{fig:versatility_analysis}: (a) shows when the height prediction is equal to the ground-truth, detection is perfect for all cameras; (b) if not, for the same height prediction error, the distance between the predicted point and ground-truth is inversely proportional to the camera ground height. This is why BEVHeight achieves on-par performance on nuScenes but quickly surpasses BEVDepth~\cite{li2022bevdepth} when the camera height only increases less than 1 meter.
% \begin{table*}[ht]
%  \centering\addtolength{\tabcolsep}{-0.6pt}
%  \resizebox{0.8\textwidth}{!}{
%  \begin{tabularx}{1.0\textwidth}{l|c|ccc|ccc|ccc}
%   \toprule
%  \multirow{3}{*}{Method} &  
%  \multirow{3}{*}{Modality}  
%  & \multicolumn{3}{c|}{$\text{Vehicle}_{(IoU=0.5)}$} & \multicolumn{3}{c|}{$\text{Pedestrian}_{(IoU=0.25)}$} & \multicolumn{3}{c}{$\text{Cyclist}_{(IoU=0.25)}$} \\
%     \cmidrule(r){3-11}
%      &  & Easy & Mid & Hard & Easy & Mid & Hard & Easy & Mid & Hard  \\
% \midrule

% PointPillars~\cite{lang2019pointpillars} & PointCloud &63.07 & 54.00 & 54.01 & 38.53 & 37.20 & 37.28 & 38.46 & 22.60 & 22.49 \\
% SECOND~\cite{yan2018second} & PointCloud &71.47 & 53.99 & 54.00 & 55.16 & 52.49 & 52.52 & 54.68 & 31.05 & 31.19 \\
% MVXNet~\cite{Sindagi2019MVX} & Image+PointCloud &71.04 & 53.71 & 53.76 & 55.83 & 54.45 & 54.40 & 54.05 & 30.79 & 31.06 \\
% \midrule
% ImvoxelNet~\cite{rukhovich2022imvoxelnet} &Image & 44.78 & 37.58 & 37.55 & 6.81 & 6.746 & 6.73 & 21.06 & 13.57 & 13.17 \\
% BEVFormer-R101$\ast$~\cite{li2022bevformer} & Image 	&	61.37&	50.73&	50.73&	16.89&	15.82&	15.95	&22.16&	22.13&	22.06\\
% BEVDepth-R101$\ast$~\cite{li2022bevdepth}&	Image 	&	76.01&	64.11&	64.18&	24.32&	24.96&	24.84	&46.45&	45.56&	45.69	\\

% \midrule
% BEVHeight-R101(Ours) & Image &	79.12&	67.95&	67.04&	29.85&	29.31&	29.07	&51.55&	51.39&	50.91\\
%     \bottomrule
%   \end{tabularx}
%   }
%   \caption{\textbf{Comparison on the DAIR-V2X-I val set.}}
%   \label{dair_sota}
% \end{table*}


\begin{table}[h!t]
 \scriptsize\centering\addtolength{\tabcolsep}{1.0pt}
\caption{\textbf{Experiments on the dataset collected by higher truck.}} 
 \resizebox{1.0\linewidth}{!}{
 \begin{tabularx}{1.0\linewidth}{l|ccc|ccc}
 \toprule
 \multirow{3}{*}{Method} &
\multicolumn{3}{c|}{$\text{Car}_{(IoU=0.5)}$} & \multicolumn{3}{c}{$\text{Big Vehicle}_{(IoU=0.5)}$} \\
 \cmidrule(r){2-7}
   & Easy & Mod. & Hard & Easy & Mod. & Hard \\
 \midrule
 BEVDepth ~\cite{li2022bevdepth} &	50.05 &	 36.82 &	36.82&	30.15&	24.74&	24.74	\\
 \rowcolor{cyan!30}BEVHeight & \textbf{51.77}&	\textbf{40.96}&	\textbf{40.96}&	\textbf{34.65}&	\textbf{29.01}&	\textbf{29.01}\\
\bottomrule
\end{tabularx}
}
\label{ddd_rebuttal}
\end{table}





\begin{figure}[!h]
\centering
% \vspace{-0.2cm}
\includegraphics[width=8.5cm]{BEV-Height/figures/versatility_analysis_rebuttal_v2.0.jpg}
\vspace{-0.60cm}
\caption{\textbf{Distance error analysis caused by same height estimation error on different platform cameras.}}
\vspace{-0.50cm}
\label{fig:versatility_analysis}
\end{figure}

\mypara{Contributions.}
Theoretically, our proposed height-based pipeline entails: i) representation agnostic to distance, as visualized in Fig.~\ref{fig:teaser}, ii) friendly prediction owing to centralized distribution as displayed in Fig.~\ref{fig:histogram-depth-height}, iii) robustness against extrinsic disturbance as illustrated in Fig.~\ref{fig:five}. Technically, we design a novel HeightNet and the projection module with less computational cost. Experimentally, experiments on various datasets and multiple depth-based detectors show the superiority of our method in both accuracy and latency.