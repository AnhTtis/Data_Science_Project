\section{Appendix}

\subsection{Broader Impacts} % \Tao{repeats with limitation?}
Our work aims to develop a vision-based 3D object detection approach for roadside perception. The proposed method may produce inaccurate predictions, leading to incorrect decision-making for cooperative autonomous vehicles and potential traffic accidents. Furthermore, we propose a new perspective of leveraging height estimation to solve PV-BEV transformation, facilitating a high-performance and robust vision-centric BEV perception framework. Although considerable progress has been made with our proposed height net and height-based 2D-3D projection module, we believe it is worth further exploring how to combine height and depth estimations to extend to autonomous driving scenarios.

\subsection{Dynamic Discretization}
The height discretization can be performed with uniform discretization (UD) with a fixed bin size, spacing-increasing discretization (SID)~\cite{HuanFu2018DeepOR} with increasing bin sizes in logspace, linear-increasing discretization (LID)~\cite{YunleiTang2020Center3DCM}and our proposed dynamic-increasing discretization (DID) with adjustable bin sizes. The above four height discretization techniques are visualized in Fig. \ref{fig:discretization}. Following DID strategy, the distribution of height bins can be dynamically adjusted with different hyper-parameter $\alpha$.

%figure7
\begin{figure}[ht]
\centering	\includegraphics[width=8.5cm]{BEV-Height/figures/space_strategy.pdf}
	\caption{\textbf{Height Discretization Methods.} Height $h_i$ is discretized over a height range $[h_{min}, h_{max}]$ into $N$ discrete bins. From left to right, these are uniform discretization (UD), spacing-increasing discretization (SID), linear-increasing discretization (LID) and the adjustable dynamic-increasing discretization(DID). For the dynamic-increasing discretization (DID) strategy, height bins with large $\alpha$ are more densely distributed when approaching the $h_{min}$ than the small hyper-parameter $\alpha$ conditions.}
\label{fig:discretization}
\end{figure}

\subsection{Results on V2X-Sim Dataset}
To certify the effectiveness of our method in multi-view scenarios, we conduct experiments on V2X-Sim~\cite{li2022v2x} simulation dataset that contains four surround roadside cameras.
As shown in Tab.~\ref{v2x_sim_rebuttal}, our BEVHeight surpass the BEVDepth by more than 10.88\%, 21.15\% on vehicle and cyclist respectively, which verifies the effectiveness of our method.

\begin{table}[h!t]
 \scriptsize\centering\addtolength{\tabcolsep}{1.0pt}
\caption{\textbf{Comparison on the V2X-Sim Detection Benchmark.}}
 \begin{tabularx}{1.0\linewidth}{l|ccc|ccc}
 \toprule
 \multirow{3}{*}{Method} &
\multicolumn{3}{c|}{$\text{Vehicle}_{(IoU=0.5)}$} & \multicolumn{3}{c}{$\text{Cyclist}_{(IoU=0.25)}$} \\
 \cmidrule(r){2-7}
   & Easy & Mod. & Hard & Easy & Mod. & Hard  \\
 \midrule
 {BEVDepth~\cite{li2022bevdepth}}  &	81.99&	81.39&	81.31&	45.95&	45.93&	45.90\\
\rowcolor{cyan!30}{BEVHeight} &  92.80&	 92.27&	 92.15&	67.24& 67.08& 67.00\\
\bottomrule
\end{tabularx}
\label{v2x_sim_rebuttal}
\vspace{-0.3cm}
\end{table}


\subsection{Effectiveness on multi depth-based Detectors} We extend our modules on BEVDepth\cite{li2022bevdepth} and BEVDet~\cite{huang2021bevdet} on
 DAIR-V2X-I\cite{yu2022dair} and present the results here. Replacing the depth-based projection in BEVDepth\cite{li2022bevdepth}, our method achieves
a performance increase of 2.19\%, 5.87\%, 4.61\% on vehicle, pedestrian and cyclist. Similarly, our approach surpasses
the origin BEVDet by 8.56\%, 5.35\%, 8.60\% respectively.
% \begin{table*}[ht]
%  \centering\addtolength{\tabcolsep}{-0.6pt}
%  \resizebox{0.8\textwidth}{!}{
%  \begin{tabularx}{1.0\textwidth}{l|c|ccc|ccc|ccc}
%   \toprule
%  \multirow{3}{*}{Method} &  
%  \multirow{3}{*}{Modality}  
%  & \multicolumn{3}{c|}{$\text{Vehicle}_{(IoU=0.5)}$} & \multicolumn{3}{c|}{$\text{Pedestrian}_{(IoU=0.25)}$} & \multicolumn{3}{c}{$\text{Cyclist}_{(IoU=0.25)}$} \\
%     \cmidrule(r){3-11}
%      &  & Easy & Mid & Hard & Easy & Mid & Hard & Easy & Mid & Hard  \\
% \midrule

% PointPillars~\cite{lang2019pointpillars} & PointCloud &63.07 & 54.00 & 54.01 & 38.53 & 37.20 & 37.28 & 38.46 & 22.60 & 22.49 \\
% SECOND~\cite{yan2018second} & PointCloud &71.47 & 53.99 & 54.00 & 55.16 & 52.49 & 52.52 & 54.68 & 31.05 & 31.19 \\
% MVXNet~\cite{Sindagi2019MVX} & Image+PointCloud &71.04 & 53.71 & 53.76 & 55.83 & 54.45 & 54.40 & 54.05 & 30.79 & 31.06 \\
% \midrule
% ImvoxelNet~\cite{rukhovich2022imvoxelnet} &Image & 44.78 & 37.58 & 37.55 & 6.81 & 6.746 & 6.73 & 21.06 & 13.57 & 13.17 \\
% BEVFormer-R101$\ast$~\cite{li2022bevformer} & Image 	&	61.37&	50.73&	50.73&	16.89&	15.82&	15.95	&22.16&	22.13&	22.06\\
% BEVDepth-R101$\ast$~\cite{li2022bevdepth}&	Image 	&	76.01&	64.11&	64.18&	24.32&	24.96&	24.84	&46.45&	45.56&	45.69	\\

% \midrule
% BEVHeight-R101(Ours) & Image &	79.12&	67.95&	67.04&	29.85&	29.31&	29.07	&51.55&	51.39&	50.91\\
%     \bottomrule
%   \end{tabularx}
%   }
%   \caption{\textbf{Comparison on the DAIR-V2X-I val set.}}
%   \label{dair_sota}
% \end{table*}


\begin{table}[h!t]
 \scriptsize\centering\addtolength{\tabcolsep}{-4.9pt}
\caption{\textbf{Ablation studies on different depth-based methods.} Here, we conduct the evaluation on DAIR-V2X-I val set, and report the results of three types of objects, vehicle~(veh.), pedestrian~(ped.) and cyclist~(cyc.).}
 \begin{tabularx}{1.0\linewidth}{l|c|ccc|ccc|ccc}
 \toprule
 \multirow{3}{*}{Method} &
 \multirow{3}{*}{VT} &
\multicolumn{3}{c|}{$\text{Veh.}_{(IoU=0.5)}$} & \multicolumn{3}{c|}{$\text{Ped.}_{(IoU=0.25)}$} & \multicolumn{3}{c}{$\text{Cyc.}_{(IoU=0.25)}$} \\
 \cmidrule(r){3-11}
   &  & Easy & Mod. & Hard & Easy & Mod. & Hard & Easy & Mod. & Hard  \\
 \midrule
 
 \multirow{2}{*}{BEVDepth\cite{li2022bevdepth}} & D &	75.50&	63.58&	63.67&	34.95&	33.42&	33.27& 55.67&	55.47&	55.34 \\
  & {\cellcolor{cyan!30} H}&	{\cellcolor{cyan!30} 77.78}&	{\cellcolor{cyan!30} 65.77}&	{\cellcolor{cyan!30} 65.85}& {\cellcolor{cyan!30} 41.22}&	{\cellcolor{cyan!30} 39.29}&	{\cellcolor{cyan!30} 39.46}&  {\cellcolor{cyan!30} 60.23}&  {\cellcolor{cyan!30} 60.08}& {\cellcolor{cyan!30} 60.54} \\
\midrule
 \multirow{2}{*}{BEVDet~\cite{huang2021bevdet}} & D &  59.59& 	51.92&	51.81&  12.61& 12.43& 12.37& 34.91& 34.32& 34.21 	\\
& {\cellcolor{cyan!30} H}&	{\cellcolor{cyan!30} 69.42}&	{\cellcolor{cyan!30} 60.48}&	{\cellcolor{cyan!30} 59.68}& {\cellcolor{cyan!30} 18.11}&	{\cellcolor{cyan!30} 17.81}&	{\cellcolor{cyan!30} 17.74}&  {\cellcolor{cyan!30} 44.69}&  {\cellcolor{cyan!30} 42.92}& {\cellcolor{cyan!30} 42.34} \\
\bottomrule
\multicolumn{11}{l}{\scriptsize{VT denotes view transformation, D,H represents depth-based and height-based ones.}}
\end{tabularx}
\label{dair_rebuttal}
\vspace{-0.38cm}
\end{table}






% \subsection{More Results}
% \subsubsection{Results on Rope3D Dataset}
\subsection{More Results on DAIR-V2X-I Dataset}
Tab.~\ref{dair} shows the experimental results of deploying our proposed approach on the DAIR-V2X-I\cite{yu2022dair} val set. Under the same configurations (e.g., backbone and BEV resolution), our model outperforms the BEVDepth\cite{li2022bevdepth} baselines by a large marge, which demonstrates the admirable performance of our approach.

% \input{BEV-Height/latex/table/rope3d_het}

% 不同分辨率的泛化性
% 非重要内容,可放在补充材料中(分辨率非主要创新)
\begin{table*}[h!t]
 \small 
 \centering
 \addtolength{\tabcolsep}{1.4pt}
 \caption{\textbf{Comparison on the DAIR-V2X-I Detection Benchmark.} Here, we report the results of three types of objects: Vehicle, Pedestrian and Cyclist. Each object is categorized into three settings according to the difficulty defined in ~\cite{yu2022dair}. Our BEVHeight manages to surpass the BEVDepth baseeline over a margin of 2\% - 6\% under the same configurations.}
 \begin{tabularx}{1.0\textwidth}{l|cc|ccc|ccc|ccc}
  \toprule
 \multirow{4}{*}{Method} & \multicolumn{2}{c|}{\multirow{2.8}{*}{Scale of Detector}} & \multicolumn{9}{c}{AP3D}\\
    \cmidrule(r){4-12}
    & &  & \multicolumn{3}{c|}{$\text{Vehicle}_{(IoU=0.5)}$} & \multicolumn{3}{c|}{$\text{Pedestrian}_{(IoU=0.25)}$} & \multicolumn{3}{c}{$\text{Cyclist}_{(IoU=0.25)}$} \\
   \cmidrule(r){2-12}
   & Backbone & BEV & Easy & Middle & Hard & Easy & Middle & Hard & Easy & Middle & Hard\\  
    \midrule
    %  \midrule
     BEVDepth\cite{li2022bevdepth} &	R50&	128x128	&	73.05&	61.32&	61.19&	22.10	&21.57	&21.11	&42.85&	42.26&	42.09\\
    BEVDepth\cite{li2022bevdepth} &	R101&	128x128	&	74.81&	62.44&	62.31&	24.49	&23.33&	23.17&	44.93&	44.02&	43.84\\
    BEVDepth\cite{li2022bevdepth} &	R101&	256x256	&	75.50&	63.58&	63.67&	34.95&	33.42&	33.27& 55.67&	55.47&	55.34\\
    \midrule
    BEVHeight&	R50&	128x128	&	76.61	&64.71	&64.76&	27.34&	26.09&	26.33	&49.68&	48.84&	48.58\\
    BEVHeight&	R101&	128x128	&	76.93&	64.97&	65.03&	28.53&	27.15&	27.48& 51.39&	50.83	&50.44\\
    BEVHeight& R101&	256x256	&	\textbf{77.78}&	\textbf{65.77}&	\textbf{65.85}&	\textbf{41.22}&	\textbf{39.29}&	\textbf{39.46}	&\textbf{60.23}&	\textbf{60.08}&	\textbf{60.54}\\
  \bottomrule 
 % \multicolumn{21}{l}{Mid: Middle, Veh.: Vehicle, Ped.: Pedestrian, Cyc.: Cyclist.}
  \end{tabularx}
  \label{dair}
\end{table*}





\begin{figure*}[t!]
	\centering
	\includegraphics[width=\textwidth]{BEV-Height/figures/visualization_supp-1.pdf}
	\caption{\textbf{Visualization Results of BEVDepth and our proposed BEVHeight under the extrinsic disturbance.} We use boxes in \textbf{{\color{red}red}} to represent false positives,  \textbf{{\color{green}green}} boxes for truth positives, and \textbf{{\color{black}black}} for the ground truth. The truth positives are defined as the predictions with IoU\textgreater 0.5 for vehicle and IoU\textgreater 0.25 for pedestrian and cyclist. I/II-(a) Clean means the original image without any processing; I/II-(b) Disturbed Roll denotes camera rotate 1 degree along roll direction; I/II-(c) Disturbed Roll and Pitch represents camera rotate 1 degree along roll and pitch directions simultaneously. We use \textbf{{\color{blue}blue}} dashed ovals to highlight the pronounced improvements in predictions.}
\label{fig:visualization_supp_1}
\vspace{-0.1cm}
\end{figure*}

\begin{figure*}[t]
	\centering
	\includegraphics[width=\textwidth]{BEV-Height/figures/visualization_supp-2.pdf}
	\caption{\textbf{Visualization Results of BEVDepth and our proposed BEVHeight under the extrinsic disturbance in another scene.}}
\label{fig:visualization_supp_2}
\vspace{-0.1cm}
\end{figure*}


\subsection{More Visualizations}
In Fig.~\ref{fig:visualization_supp_1} and Fig.~\ref{fig:visualization_supp_2}, we show more visualization results on the DAIR-V2X-I \cite{yu2022dair} dataset. As can be seen from the samples in I/II-(a) clean, our BEVHeight manage to detect objects in middle and long-distances. As for the  extrinsic disturbance cases in  I/II-(b) and I/II-(c),  our method can still guarantee the detection accuracy in terms of cars, pedestrian and cyclist. It can be concluded that our method can significantly improve the accuracy in middle and long-distances and the robustness to extrinsic disturbance.


% \Tao{Can we have some prediction results of height / depth?}
% \Lei{See Sec. ~\ref{sec:distance_error_analysis}, in process}
