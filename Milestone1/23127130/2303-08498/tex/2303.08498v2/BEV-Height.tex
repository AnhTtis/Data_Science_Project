% CVPR 2023 Paper Template
% based on the CVPR template provided by Ming-Ming Cheng (https://github.com/MCG-NKU/CVPR_Template)
% modified and extended by Stefan Roth (stefan.roth@NOSPAMtu-darmstadt.de)

\documentclass[10pt,twocolumn,letterpaper]{article}

%%%%%%%%% PAPER TYPE  - PLEASE UPDATE FOR FINAL VERSION
% \usepackage[review]{cvpr}      % To produce the REVIEW version
%\usepackage{cvpr}              % To produce the CAMERA-READY version
\usepackage[pagenumbers]{cvpr} % To force page numbers, e.g. for an arXiv version

% Include other packages here, before hyperref.
\usepackage{graphicx}
\usepackage{amsmath}
\usepackage{amssymb}
\usepackage{booktabs}
\usepackage{xcolor}
\usepackage{color, colortbl}
\usepackage{verbatim}

\usepackage{algorithm}
\usepackage{algorithmic}
\usepackage{tabularx}
\usepackage{multirow}
\usepackage{multicol}
\usepackage{bm}
\usepackage{ulem} 
\usepackage[titletoc]{appendix}

\newcommand{\KY}[1]{{\color{magenta}{(\textbf{#1})}}}
\newcommand{\ky}[1]{{\color{magenta}{#1}}}
% \newcommand{\comment}[1]{}

\newcommand{\Tao}[1]{{\color{cyan}{(\textbf{#1})}}}
\newcommand{\tao}[1]{{\color{cyan}{#1}}}

\newcommand{\Lei}[1]{{\color{orange}{(\textbf{#1})}}}
\newcommand{\lei}[1]{{\color{orange}{#1}}}

\newcommand{\name}{BEVHeight}
\newcommand{\mypara}[1]{\vspace{1mm}\noindent\textbf{#1}}

\newcommand{\LEI}[1]{{\color{blue}#1}}


% It is strongly recommended to use hyperref, especially for the review version.
% hyperref with option pagebackref eases the reviewers' job.
% Please disable hyperref *only* if you encounter grave issues, e.g. with the
% file validation for the camera-ready version.
%
% If you comment hyperref and then uncomment it, you should delete
% ReviewTempalte.aux before re-running LaTeX.
% (Or just hit 'q' on the first LaTeX run, let it finish, and you
%  should be clear).
\usepackage[pagebackref,breaklinks,colorlinks]{hyperref}


% Support for easy cross-referencing
\usepackage[capitalize]{cleveref}
\crefname{section}{Sec.}{Secs.}
\Crefname{section}{Section}{Sections}
\Crefname{table}{Table}{Tables}
\crefname{table}{Tab.}{Tabs.}


%%%%%%%%% PAPER ID  - PLEASE UPDATE
\def\cvprPaperID{8460} % *** Enter the CVPR Paper ID here
\def\confName{CVPR}
\def\confYear{2023}

\begin{document}

%%%%%%%%% TITLE - PLEASE UPDATE
\title{BEVHeight: 
% Towards 
A Robust
% Extrinsic Parameter Free\Tao{Free?} 
Framework for Vision-based Roadside \\
3D Object Detection 
% for  Perception
}

\author{
Lei Yang\textsuperscript{1}\footnotemark[1],  Kaicheng Yu\textsuperscript{2},  Tao Tang\textsuperscript{3},  Jun Li\textsuperscript{1}, Kun Yuan\textsuperscript{4},  Li Wang\textsuperscript{1}, Xinyu Zhang\textsuperscript{1}\footnotemark[2], Peng Chen\textsuperscript{2} \\
	\textsuperscript{1}State Key Laboratory of Automotive Safety and Energy,  Tsinghua University \\
	\textsuperscript{2}Autonomous Driving Lab, Alibaba Group;
 \textsuperscript{3}Shenzhen Campus, Sun Yat-sen University \\
 \textsuperscript{4}Center for Machine Learning Research, Peking University \\ 
    \begin{normalsize}${\tt \{yanglei20@mails, lijun1958@, xyzhang@, wangli\_thu@mail\}.tsinghua.edu.cn}$\end{normalsize} \\
    \begin{normalsize}${\tt \{kaicheng.yu.yt, trent.tangtao\}@gmail.com; kunyuan@pku.edu.cn; yuanshang.cp@alibaba\mbox{-}inc.com}$\end{normalsize}}

% \author{First Author\\
% Institution1\\
% Institution1 address\\
% {\tt\small firstauthor@i1.org}
% For a paper whose authors are all at the same institution,
% omit the following lines up until the closing ``}''.
% Additional authors and addresses can be added with ``\and'',
% just like the second author.
% To save space, use either the email address or home page, not both
% \and
% Second Author\\
% Institution2\\
% First line of institution2 address\\
% {\tt\small secondauthor@i2.org}
% }

% \maketitle
\begin{figure}[tp]
    \centering
    \includegraphics[width=\linewidth]{figs/images/teaser_new.pdf}
    \caption{We propose a unified method for four low-level structure segmentation tasks: camouflaged object, forgery, shadow and defocus blur detection~(Top). Our approach relies on a pre-trained frozen transformer backbone that leverages explicit extracted features, \eg, the frozen embedded features and high-frequency components, to prompt knowledge. } 
    \label{fig:teaser}
\end{figure}
\renewcommand{\thefootnote}{\fnsymbol{footnote}}
\footnotetext[1]{Work done during an internship at DAMO Academy, Alibaba Group.}
\footnotetext[2]{Corresponding Author.}
\renewcommand{\thefootnote}{\arabic{footnote}}

%%%%%%%%% ABSTRACT
\begin{abstract}
% Perception is one of the most critical tasks in autonomous driving, where the mainstream methods only rely on ego-vehicle sensors, such as cameras, to recognize the surrounding worlds. However, these sensors are limited to seeing objects within a fixed radius of the vehicle center. With the fast growth of intelligent infrastructures, roadside camera units have the potential to see the world beyond the visual range. 
While most recent autonomous driving system focuses on developing perception methods on ego-vehicle sensors, people tend to overlook an alternative approach to leverage intelligent roadside cameras to extend the perception ability beyond the visual range. 
% However, unlike ego-vehicle cameras, these roadside ones suffer from various challenges. For example, the installation position and camera poses are usually inconsistent even on the same road, which greatly challenges the robustness of the perception method. 
We discover that the state-of-the-art vision-centric bird's eye view detection methods have inferior performances on roadside cameras. 
This is because these methods mainly focus on recovering the depth regarding the camera center, where the depth difference between the car and the ground quickly shrinks while the distance increases. In this paper, we propose a simple yet effective approach, dubbed BEVHeight, to address this issue. In essence, instead of predicting the pixel-wise depth, we regress the height to the ground to achieve a distance-agnostic formulation to ease the optimization process of camera-only perception methods. On popular 3D detection benchmarks of roadside cameras, our method surpasses all previous vision-centric methods by a significant margin. The code is available at {\url{https://github.com/ADLab-AutoDrive/BEVHeight}}.
% \href{https://github.com/ADLab-AutoDrive/BEVHeight}{here}.

\comment{
   Overlooked roadside perception has great potential to solve the issues of over-occlusion and long-range perception, which is believed to facilitate more intelligent and safer autonomous driving. Vision-centric BEV perception perceives inherent merits in fusioning sensors from adjacent road side units (RSU). However, the existing BEV perception methods mainly aim at autonomous driving and are dedicated to achieving higher accuracy. The generalization challenge caused by various cameras with ambiguous mounting positions and inconsistent intrinsic parameters on RSU is disregarded. In this paper, we find that the height information perpendicular to the ground has three benefits from the perspective of road side unit: 1) Centralized distribution is conducive to improving accuracy 2) Not depending on perspective principle makes it robust to intrinsic perturbation, 3) Weakly correlated with image area make it not easily affected by ambiguous mounting viewpoint. Motivated by the conclusions above, we propose a novel height-based BEV framework for roadside perception, dubbed BEVHeight, whose PV-BEV transformation relies on the implicit height estimation, thus achieving higher accuracy and stronger generalization on the roadside. Experiments on the Rope3D and DAIR-V2X 3D object detection benchmark show that our framework surpasses the state-of-the-art methods under the standard settings. Under the camera's internal and external parameters disturbed setting, our method yields the best robustness.
  }
\end{abstract}
% 对于自动驾驶,感知很重要,现在关注于车辆自身的感知,忽视了路端的优势(遮挡/远距离);(为什么路端)
% 最近新兴的车端BEV感知在路端也可以发挥他的优点(传感器融合);(为什么BEV)
% 但是路端传感器的复杂环境(安装位置/内外参类型)对流行的视觉BEV感知方法的精度和泛化性带来了挑战;(有什么问题)
% 我们发现了问题在于现在的最好的sota BEV方法依赖于准确的depth估计,但是在路端depth分布差异大/传感器内外参差异大(问题)
% 于是我们提出了BEV-height,巧妙的选择了height估计,对比depth,其优点(height分布集中/透视原理?)(解决问题)
% 取得了很好的精度和泛化性结果(结果)

\section{Introduction}
\IEEEPARstart{T}{he} method Neural Radiance Fields (NeRF)~\cite{mildenhall2020nerf} is proposed for photorealistic novel view synthesis. Given many views of the scene, it creates implicit multi-view geometry and learns for view synthesis. However, it has poor generalizations to new scenes and requires retraining or fine-tuning on each scene. 
 
 Recent work~\cite{Yu_2021_CVPR,Trevithick_2021_ICCV} has explored the ways of using a single image to train NeRF. They introduce a convolutional feature encoder to learn the image representation which gives it some limited generalization abilities to unseen scenes.  But, without fine-tuning, these methods produce many floats and artifacts in rendering novel views. 
 
  Multi-Plane Images (MPI) representation that learns multiple RGB images from a single image is also used in \cite{Wu_2021_ICCV,Tucker_2020_CVPR,wu2022remote} for  novel view synthesis. However, MPI heavily relies on the qualities of the planar images and needs plenty of image planes to avoid blurs. There is no strong 3D geometry constraint and it fails in many complex scenes.
  
  MINE~\cite{Li_2021_ICCV2} introduces the volume rendering of NeRF into the MPI. It runs faster and produces better depth rendering quality compared with single-view NeRFs~\cite{Yu_2021_CVPR,Trevithick_2021_ICCV}. However, the rendering quality heavily relies on the number of image planes. It needs high-resolution 4D volumes to store the 4-channel  (RGB and volume density) image planes that cost a large amount of GPU memory in both training and 
 prediction.  
 

 
 \begin{figure}[t]
\setlength{\abovecaptionskip}{7pt}
\setlength{\belowcaptionskip}{0pt}
	\centering
% 	\subfigure[MINE (PSNR:14.9)]{  % for AAAI
	\subfloat[MINE (PSNR:14.9)]{
%			\centering
			\includegraphics[width=0.23\textwidth]{figure/intro/DJI_20200223_163206_598_0_MINE.png}
%			\label{subfig:pixelnerf}
	}\subfloat[MINE (depth)]{
%			\centering
			\includegraphics[width=0.23\textwidth]{figure/intro/MINE_disp.png}
%			\label{subfig:mpi}
	}
	\\[-3mm]
	\subfloat[Ours (PSNR:17.0)]{
%			\centering
			\includegraphics[width=0.23\textwidth]{figure/intro/DJI_20200223_163206_598_0_ours.png} 
	}\subfloat[Ours (depth)]{
%			\centering
			\includegraphics[width=0.23\textwidth]{figure/intro/ours_disp.png}
	}
	\caption{Comparison with state-of-the-art methods. (a-b) RGB and depth rendering results of  \cite{Li_2021_ICCV2}. It produces many blurs and floats in the occluded regions and at the object/depth edges. 
	(c-d) Our method employs a joint rendering mechanism that preserves more image details and predicts sharp depth edges.}
	\label{fig:performance_illustration}
\end{figure}
 
 In this paper, we propose a joint rendering mechanism that takes the MPI strategy for coarse sampling proposals and the MLP\&volume-based rendering~\cite{mildenhall2020nerf} for fine sampling and rendering. Then, both the coarse point samples and the fine samples are combined according to their geometry distribution to realize a more accurate joint rendering. More importantly, we introduce a depth teacher net that serves as the guidance for the joint rendering. The monocular depth teacher predicts dense pseudo depth maps that assist the consistent 3D geometry learning between the MPI, the fine volume, and the joint rendering. It also boosts the multi-view geometry consistency between the source view and the target novel views that 
helps handle the occlusions, reduce the blurs and floats, and render accurate depths. 
 
In the experiments,  we verify the effectiveness of our method on three challenging real-scene datasets (RealEstate10K~\cite{zhou2018stereo}, NYU~\cite{silberman2012indoor} and  NeRF-LLFF~\cite{mildenhall2020nerf}) for novel view synthesis or depth estimation. Given a single image as input, our method is shown able to produce higher qualities in both the RGB image rendering and depth map prediction. It far outperforms state-of-the-art methods~\cite{Li_2021_ICCV2,Yu_2021_CVPR} with improvements of 5$\sim$20\% in PSNR and SSIM for the RGB rendering and reduces 20$\sim$50\% of the errors for the depth prediction.

% related work
\section{Related Work}
\mypara{Roadside Perception.}
Concurrent perception efforts for autonomous driving are mainly limited to the ego vehicle~\cite{caesar2020nuscenes, sun2020waymo}. While the roadside perception, which comparatively has a longer perceptual range and more robustness to occlusion and long-time event prediction, is mainly under-explored. Recently, some pioneers have present roadside datasets~\cite{ye2022rope3d, yu2022dair}, hoping to facilitate the 3D perception tasks in roadside scenarios. Compared with the vehicle perceptual system, which only observes surroundings in a short distance, the roadside cameras, mounted on poles a few meters above the ground, can provide long-range perception. However, the cameras mounted on roadside units have ambiguous mounting positions and 
%inconsistent intrinsic 
variable extrinsic parameters, which bring critical challenges to current perception models. In this paper, we take the advances and challenges of roadside cameras into account, and design an efficient and robust roadside perception framework, \name{}.


% \mypara{Vision-based Multi-View BEV perception.}
%  single-camera setting and multi-camera setting
\mypara{Vision Centric BEV Perception.}
Recent vision-centric works predict objects in 3D space, which is very suitable for applying multi-view feature aggregation under BEV for autonomous driving. Popular methods can be divided into transformer-based and depth-based schema. 
Following DETR3D~\cite{wang2022detr3d}, transformer-based detectors design a set of object queries~\cite{liu2022petr, liu2022petrv2, jiang2022polarformer, Chen2022PolarPF, Saha2022TranslatingII} or BEV grid queries\cite{li2022bevformer}, then perform the view transformation through cross-attention between queries and image features. 
Following LSS~\cite{philion2020lift}, depth-based methods ~\cite{reading2021cadnn, huang2021bevdet, huang2022bevdet4d} explicitly predict the depth distribution and use it to construct the 3D volumetric feature. Followup works introduce depth supervision from the LiDAR sensors ~\cite{li2022bevdepth} or multi-view stereo techniques ~\cite{wang2022sts, li2022bevstereo, park2022solofusion} to improve the depth estimation accuracy and achieve state-of-the-art performance. 
Additionally, transformer-based detectors' implicit 3D location information also can benefit from accurate depth cues. Inspired by~\cite{zhang2022monodetr,huang2022monodtr}, CrossDTR~\cite{tseng2022crossdtr} proposed depth-guided transformers, which compose depth-aware embedding from depth maps and are supervised by ground truth depth maps to enhance performance.
However, when applying these methods to roadside perception, the bonus of accurate depth information fades. As the complex mounting positions and variable extrinsic parameters of the roadside cameras, predicting depth from them is difficult. In this work, our \name{} utilizes the height estimation to achieve state-of-the-art performance and best robustness of roadside 3D object detection.


% method

\vspace{-0.3em}
\section{Method}
\vspace{-0.3em}

Our sensitivity-aware visual parameter-efficient fine-tuning consists of two stages. In the first stage, SPT measures the task-specific sensitivity for the pre-trained parameters (Section~\ref{subsec:sensitivity}). Based on the parameter sensitivity and a given parameter budget, SPT then adaptively allocates trainable parameters to task-specific important positions (Section~\ref{subsec:SPT}).

\vspace{-0.3em}
\subsection{Task-specific Parameter Sensitivity}
\label{subsec:sensitivity}
\vspace{-0.3em}

Recent research has observed that pre-trained backbone parameters exhibit varying feature patterns~\cite{raghu2021vision,naseer2021intriguing} and criticality~\cite{zhang2019all,chatterji2019intriguing} at distinct positions. 
Moreover, when transferred to downstream tasks, their efficacy varies depending on how much pre-trained features are reused and how well they adapt to the specific domain gap~\cite{yosinski2014transferable,kumar2022finetuning,neyshabur2020being}. Motivated by these observations, we argue that not all parameters contribute equally to the performance across different tasks in PEFT and propose a new criterion to measure the sensitivity of the parameters in the pre-trained backbone for a given task.

Specifically, given the training dataset $\gD_t$ for the $t$-th task and the pre-trained model weights $\vw=\left\{w_1, w_2, \ldots, w_N\right\}\in \sR^N$ where $N$ is the total number of parameters, the objective for the task is to minimize the empirical risk: $\min_{\vw} E(\gD_t, \vw)$.
We denote the parameter sensitivity \bohan{set} as $\gS=\{s_1, \ldots, s_N\}$ and the sensitivity $s_n$ for parameter $w_n$ is measured by the empirical risk difference when tuning it:
\begin{equation}
\vspace{-0.3em}
    \begin{aligned}
        s_n = E(\gD_t, \vw)-E(\gD_t, \vw\mid w_n=w_n^*),
    \end{aligned}
\label{eq:sensitivity}
\end{equation}
where $w_n^*=\underset{w_n}{\rm argmin}(E(\gD_t, \vw))$. We can reparameterize the tuned parameters as  $w_n^*=w_n+\Delta_{w_n}$, where $\Delta_{w_n}$ denotes the update for $w_n$ after tuning. Here we individually measure the sensitivity of each parameter, which is reasonable given that most of the parameters are frozen during fine-tuning in PEFT. However, it is still computationally intensive to compute Eq.~(\ref{eq:sensitivity}) for two reasons. Firstly, getting the empirical risk for $N$ parameters requires forwarding the entire network $N$ times, which is time-consuming. Secondly, it is challenging to derive $\Delta_{w_n}$, as we have to tune each individual $w_n$ until convergence.

{\begin{algorithm}[t!]
\caption{\label{alg:tps} Computing task-specific parameter sensitivities}
\begin{algorithmic}
    \STATE \textbf{Input:} Pre-trained model with network parameters $\vw$, training set $\gD_t$ for the $t$-th task, and number of training samples $C$ used to calculate the parameter sensitivities
    \STATE \textbf{Output:} Sensitivity set $\gS=\{s_1, \ldots, s_N\}$
    \STATE Initialize $\gS=\{0\}^N$
    \FOR{$i\in\{1,\ldots,C\}$}
        \STATE Get the $i$-th training sample of $\gD_t$
	    \STATE Compute loss $E$
		\STATE Compute gradients $\vg$
		\FOR{$n\in\{1,\ldots,N\}$}
                \STATE Update sensitivity for the $n$-th parameter: $s_{n} = s_{n} + g_n^2$
		    \ENDFOR
    \ENDFOR
\end{algorithmic}
\end{algorithm}}


\begin{figure*}[t]
\begin{center}
    \includegraphics[width=\linewidth]{main_figure.pdf}
\end{center}\vspace{-2em}
\caption{Overview of our trainable parameter allocation strategy. With the parameter sensitivity \bohan{set} $\gS$, we first get the top-$\tau$ sensitive parameters. Instead of directly tuning these sensitive parameters, we also boost the representational capability by replacing unstructured tuning with structured tuning at sensitive weight matrices that have a large number of sensitive parameters, which can be implemented by an existing structured tuning method, \eg, LoRA~\cite{hu2022lora} and Adapter~\cite{houlsby2019parameter}. Red lines and blocks represent trainable parameters and modules, while blue lines represent frozen parameters.}
\label{fig:main}
\vspace{-1.5em}
\end{figure*}


To overcome the first barrier, we simplify the empirical loss by approximating $s_n$ in the vicinity of $\vw$ by its first-order Taylor expansion
\vspace{-0.3em}
\begin{equation}
\vspace{-0.5em}
    \begin{aligned}
        s_n^{(1)} = -g_n\Delta_{w_n},
    \end{aligned}
\label{eq:first-order}
\end{equation}
where the gradients $\vg=\partial E/\partial\vw$, and $g_n$ is the gradient of the $n$-th element of $\vg$. 
To address the second barrier, following~\cite{liu2018darts,cai2018proxylessnas}, we take the one-step unrolled weight as the surrogate for $w_n^*$ and approximate $\Delta_{w_n}$ in Eq.~(\ref{eq:first-order}) with a single step of gradient descent. We can accordingly get $s_n^{(1)} \approx g_n^2\epsilon$,
where $\epsilon$ is the learning rate. Since $\epsilon$ is the same for all parameters, we can eliminate it when comparing the sensitivity with the other parameters and finally get 
\vspace{-0.5em}
\begin{equation}
\vspace{-0.3em}
    \begin{aligned}
        s_n^{(1)} \approx g_n^2.
    \end{aligned}
\label{eq:first-order-simp}
\end{equation}
Therefore, the sensitivity of a parameter can be efficiently measured by its potential to reduce the loss on the target domain. Note that although our criterion draws inspiration from pruning work~\cite{molchanov2019importance}, it is distinct from it. \cite{molchanov2019importance} measures the parameter importance by the squared change in loss when removing them, \ie, $\left( E(\gD_t, \vw)-E(\gD_t, \vw\mid w_n=0) \right)^2$ and finally derives the parameter importance by $\left( g_n w_n \right)^2$, which is different from our formulations in Eqs.~(\ref{eq:sensitivity}) and~(\ref{eq:first-order-simp}).

In practice, we accumulate $\gS$ from a total number of $C$ training samples ahead of fine-tuning to generate accurate sensitivity as shown in Algorithm~\ref{alg:tps}, where $C$ is a pre-defined hyper-parameter. In Section~\ref{subsec:abl}, we show that employing only 400 training samples is sufficient for getting reasonable parameter sensitivity, which requires only 5.5 seconds with a single GPU for any VTAB-1k dataset with ViT-B/16 backbone~\cite{vit}.

\vspace{-0.3em}
\subsection{Adaptive Trainable Parameters Allocation}
\label{subsec:SPT}
\vspace{-0.2em}

Our next step is to allocate trainable parameters based on the obtained parameter sensitivity set $\gS$ and a desired parameter budget $\tau$. A straightforward solution is to directly tune the top-$\tau$ most sensitive unstructured connections (parameters) \rev{while keeping the rest frozen}, which we name unstructured tuning. Specifically, we select the top-$\tau$ most sensitive weight connections in $\gS$ to form the sensitive weight connection set $\gT$. Then, for \rev{a} weight matrix $\mW\in \sR^{d_{\rm in}\times d_{\rm out}}$, we can get a binary mask $\mM\in \sR^{d_{\rm in}\times d_{\rm out}}$ computed by
\vspace{-0.5em}
\begin{equation}
\vspace{-0.5em}
    {\begin{array}{ll}
    \small
    \begin{aligned}
    \mM^j =
    \left\{\begin{array}{ll} 
    1 ~~~~~ \mW^j \in \gT \\
    0 ~~~~~ \mW^j \notin \gT
    \end{array}\right.
    \end{aligned},
    \small
    \end{array}}
\label{eq:mask}
\end{equation}
where $\mW^j$ and $\mM^j$ are the $j$-th element in $\mW$ and $\mM$, respectively. Accordingly, we can train the sensitive parameters by gradient descent and the updated weight matrix can be formulated as $\mW'\leftarrow \mW - \epsilon\vg_{\mW}\odot\mM$, where $\vg_{\mW}$ is the gradient for $\mW$.

However, considering PEFT approaches generally limit the proportion of trainable parameters to less than 1\%, tuning only a small number of unstructured weight connections might not have enough representational capability to handle the downstream datasets with large domain gaps from the source pre-training data. Therefore, to improve the representational capability, we propose to replace unstructured tuning with structured tuning at the sensitive weight matrices that have a high number of sensitive parameters. To preserve the parameter budget, we can implement structured tuning with an existing efficient structured tuning PEFT method~\cite{hu2022lora,chen2022adaptformer,houlsby2019parameter,jie2022convolutional} that learns to directly adjust \rev{all hidden dimensions at once}. We depict an overview of our trainable parameter allocation strategy in Figure~\ref{fig:main}. For example, we can employ the low-rank reparameterization trick LoRA~\cite{hu2022lora} to the sensitive weight matrices \rev{and the one-step update for $\mW$ can be formulated as}
\vspace{-0.4em}
\begin{equation}
\vspace{-0.4em}
    {\begin{array}{ll}
    \small
    \begin{aligned}
    \mW' = \left\{\begin{array}{ll} 
    \mW + \mW_{\rm down}\mW_{\rm up} & ~~ \text { if } ~~ \sum_{j=0}^{d_{\rm in}\times d_{\rm out}} \mM^j \geq \sigma_{\rm opt} \\
    \mW - \epsilon\vg_{\mW}\odot\mM & ~~ {\rm otherwise}
    \end{array}\right.
    \end{aligned},
    \small
    \end{array}}
\label{eq:weight_updat}
\end{equation}
where $\mW_{\rm down}\in \sR^{d_{\rm in}\times r}$ and $\mW_{\rm up}\in \sR^{r\times d_{\rm out}}$ are two learnable low-rank matrices to approximate the update of $\mW$ and rank $r$ is a hyper-parameter where $r \ll {\rm min}(d_{\rm in},d_{\rm out})$. In this way, we perform structured tuning on $\mW$ when its number of sensitive parameters exceeds $\sigma_{\rm opt}$, whose value depends on the pre-defined type of structured tuning method. For example, since implementing structured tuning with LoRA requires $2\times d_{\rm in} \times d_{\rm out} \times r$ trainable parameters for each sensitive weight matrix, we set $\sigma_{\rm LoRA} \leftarrow 2\times d_{\rm in} \times d_{\rm out} \times r$ to ensure that the number of trainable parameters introduced by structured tuning is always equal to or lower than the number of sensitive parameters.

In this way, our SPT adaptively incorporates both structured and unstructured tuning granularities to enable higher flexibility and stronger representational power, simultaneously. In Section~\ref{subsec:abl}, we show that structured tuning is important for the downstream tasks with larger domain gaps and both unstructured and structured tuning contribute clearly to the superior performance of our SPT.

% exp
\section{Experiments}
We briefly introduce the experiment settings and two benchmark datasets in road-side perception domain. We then compare our proposed \name{} with state-of-the-art methods under clean and noisy camera settings. We ablate our methods in detail and discuss the limitations. 
% In this section, we first introduce our experimental setups. Then, we conduct comprehensive experiments to validate the effects of our \name{} on two large-scale roadside datasets, DAIR-V2X~\cite{yu2022dair} and Rope3D~\cite{ye2022rope3d}. Finally, we also give an analysis on the popular vehicle perception dataset, nuScenes~\cite{caesar2020nuscenes}.

\subsection{Datasets}
\mypara{DAIR-V2X.} Yu \etal~\cite{yu2022dair} introduces a large-scale, multi-modality dataset. As the original dataset contains images from vehicles and roadside units, this benchmark consists of three tracks to simulate different scenarios. Here, we focus on the DAIR-V2X-I, which only contains the images from mounted cameras to study roadside perception. Specifically, DAIR-V2X-I contains around ten thousand images, where 50\%, 20\% and 30\% images are split into train, validation and testing respectively. However, up to now, the testing examples are not yet published, we evaluate the results on the validation set. We follow the benchmark to use the average perception of the bounding box as in KITTI~\cite{geiger2012we}.

% \mypara{DAIR-V2X~\cite{yu2022dair}} is a large-scale, multi-modality vehicle-infrastructure collaborative dataset for 3D object detection, which is composed of three sub-datasets: DAIR-V2X-C, DAIR-V2X-I, DAIR-V2X-V. DAIR-V2X-C containing sensors from both vehicle and roadside is for the vehicle-infrastructure cooperative 3D object detection. DAIR-V2X-V including sensors on the vehicle is for the 3D object detection in autonomous driving. DAIR-V2X-I with only sensors on infrastructure, is designed for the roadside perception task. In this paper, we conduct experiments based on the DAIR-V2X-I. DAIR-V2X-I contains 10084 samples, which are further divided into train/val/test according to 5:2:3 respectively. Since the test set is not released yet, all experiments are trained on the train set and evaluated on val set. The evaluation metric is average precision (AP) as used in the KITTI\cite{geiger2012we} dataset.

% Rope3D
\mypara{Rope3D\cite{ye2022rope3d}.} 
There is another recent large-scale benchmark named Rope3D. It contains over 500k images with three-dimensional bounding boxes from seventeen intersections. Here, we follow the proposed homologous setting to use 70\% of the images as training, and the remaining as testing. Note that, all images are randomly sampled. For validation metrics, we leverage the AP$_{\text{3D}{|\text{R40}}}$~\cite{simonelli2019disentangling} and the Rope$_\text{score}$, which is a consolidated metric of the 3D AP and other similarities metrics, such as average area similarity.

% is a large-scale dataset dedicated to vision-based roadside 3D object detection. This dataset provides 50009 frames of images together with 3D annotations. All images are sampled from seventeen intersections and split into the train/val set according to 7:3 under the Homologous setting. The proposed AP$_{\text{3D}{|\text{R40}}}$ and Rope$_\text{score}$ are used as the metrics.

% \mypara{nuScenes\cite{HolgerCaesar2019nuScenesAM}} is a large-scale autonomous-driving dataset for 3D detection, consisting of 700, 150 and 150 scenes for training, validation, and testing, respectively. 
% Each frame contains one point cloud and six calibrated images that cover 360 fields-of-view. 
% metric
% For 3D detection, the main metrics are mean Average Precision (mAP) and nuScenes detection score (NDS). 
% The mAP is defined by the BEV center distance with thresholds of {0.5m, 1m, 2m, 4m}, instead of the IoUs of bounding boxes. 
% NDS is a consolidated metric of mAP and other metric scores, such as average translation error and average scale error.

\subsection{Experimental Settings}
For architecture details, we use ResNet-101\cite{he2016deep} as image-view encoder in results compared with state-of-the-art and ResNet-50 for other ablation studies.
The input resolution is in (864, 1536). For data augmentation, we follow \cite{li2022bevdepth} to use random scaling and rotation in the 2D space only. All methods are trained for 150 epochs with AdamW optimzer~\cite{loshchilov2017adamw}, where the initial learning rate is set to $2e-4$.
% We use ResNet\cite{he2016deep} as the image backbone. For DAIR-V2X-I and Rope3D datasets, the image size is processed to 864x1536; we adopt random translation, random scaling, and random rotation as image data augmentations. No BEV data augmentations are applied. All results are trained for 120 epochs with AdamW optimizer and learning rate set to 2e-4. 

% \begin{table*}[ht]
%  \centering\addtolength{\tabcolsep}{-0.6pt}
%  \resizebox{0.8\textwidth}{!}{
%  \begin{tabularx}{1.0\textwidth}{l|c|ccc|ccc|ccc}
%   \toprule
%  \multirow{3}{*}{Method} &  
%  \multirow{3}{*}{Modality}  
%  & \multicolumn{3}{c|}{$\text{Vehicle}_{(IoU=0.5)}$} & \multicolumn{3}{c|}{$\text{Pedestrian}_{(IoU=0.25)}$} & \multicolumn{3}{c}{$\text{Cyclist}_{(IoU=0.25)}$} \\
%     \cmidrule(r){3-11}
%      &  & Easy & Mid & Hard & Easy & Mid & Hard & Easy & Mid & Hard  \\
% \midrule

% PointPillars~\cite{lang2019pointpillars} & PointCloud &63.07 & 54.00 & 54.01 & 38.53 & 37.20 & 37.28 & 38.46 & 22.60 & 22.49 \\
% SECOND~\cite{yan2018second} & PointCloud &71.47 & 53.99 & 54.00 & 55.16 & 52.49 & 52.52 & 54.68 & 31.05 & 31.19 \\
% MVXNet~\cite{Sindagi2019MVX} & Image+PointCloud &71.04 & 53.71 & 53.76 & 55.83 & 54.45 & 54.40 & 54.05 & 30.79 & 31.06 \\
% \midrule
% ImvoxelNet~\cite{rukhovich2022imvoxelnet} &Image & 44.78 & 37.58 & 37.55 & 6.81 & 6.746 & 6.73 & 21.06 & 13.57 & 13.17 \\
% BEVFormer-R101$\ast$~\cite{li2022bevformer} & Image 	&	61.37&	50.73&	50.73&	16.89&	15.82&	15.95	&22.16&	22.13&	22.06\\
% BEVDepth-R101$\ast$~\cite{li2022bevdepth}&	Image 	&	76.01&	64.11&	64.18&	24.32&	24.96&	24.84	&46.45&	45.56&	45.69	\\

% \midrule
% BEVHeight-R101(Ours) & Image &	79.12&	67.95&	67.04&	29.85&	29.31&	29.07	&51.55&	51.39&	50.91\\
%     \bottomrule
%   \end{tabularx}
%   }
%   \caption{\textbf{Comparison on the DAIR-V2X-I val set.}}
%   \label{dair_sota}
% \end{table*}


\begin{table}[t]
 \scriptsize\centering\addtolength{\tabcolsep}{-4.2pt}
\caption{\textbf{Comparing with the state-of-the-art on the DAIR-V2X-I val set.} Here, we report the results of three types of objects, vehicle~(veh.), pedestrian~(ped.) and cyclist~(cyc.). Each object is categorized into three settings according to the difficulty defined in ~\cite{yu2022dair}. First, recent BEVDepth surpasses the previous best by a large margin, showing that using bird's-eye-view indeed helps in roadside scenarios. Our method outperforms the BEVDepth by over 3\% in average precision and constitutes state-of-the-art. It is surprising to see that our method outperforms those relying on LiDAR modality.
}
% \vspace{-0.2cm}

 \begin{tabularx}{1.0\linewidth}{l|c|ccc|ccc|ccc}
  \toprule
 \multirow{3}{*}{Method} &  
 \multirow{3}{*}{M}  
 & \multicolumn{3}{c|}{$\text{Veh.}_{(IoU=0.5)}$} & \multicolumn{3}{c|}{$\text{Ped.}_{(IoU=0.25)}$} & \multicolumn{3}{c}{$\text{Cyc.}_{(IoU=0.25)}$} \\
    \cmidrule(r){3-11}
     &  & Easy & Mid & Hard & Easy & Mid & Hard & Easy & Mid & Hard  \\
\midrule
PointPillars~\cite{lang2019pointpillars} & L &63.07 & 54.00 & 54.01 & 38.53 & 37.20 & 37.28 & 38.46 & 22.60 & 22.49 \\
SECOND~\cite{yan2018second} & L &71.47 & 53.99 & 54.00 & 55.16 & 52.49 & 52.52 & 54.68 & 31.05 & 31.19 \\
MVXNet~\cite{Sindagi2019MVX} & LC &71.04 & 53.71 & 53.76 & 55.83 & 54.45 & 54.40 & 54.05 & 30.79 & 31.06 \\
\midrule
ImvoxelNet~\cite{rukhovich2022imvoxelnet} &C & 44.78 & 37.58 & 37.55 & 6.81 & 6.746 & 6.73 & 21.06 & 13.57 & 13.17 \\
BEVFormer~\cite{li2022bevformer} & C 	&	61.37&	50.73&	50.73&	16.89&	15.82&	15.95	&22.16&	22.13&	22.06\\
BEVDepth~\cite{li2022bevdepth}&	C 	&	
75.50&	63.58&	63.67&	34.95&	33.42&	33.27& 55.67&	55.47&	55.34\\
\midrule
 \rowcolor{cyan!30} BEVHeight & C &	
 \textbf{77.78}&	\textbf{65.77}&	\textbf{65.85}&	\textbf{41.22}&	\textbf{39.29}&	\textbf{39.46}	&\textbf{60.23}&	\textbf{60.08}&	\textbf{60.54}\\
    \bottomrule
\multicolumn{8}{l}{\scriptsize{M, L, C denotes modality, LiDAR, camera respectively.}}
  \end{tabularx}
  \vspace{-0.5cm}
  \label{dair_sota_2}
\end{table}






\subsection{Comparing with state-of-the-art}
\mypara{Results on the original benchmark.} On DAIR-V2X-I setting, we compare our BEVHeight with other state-of-the-art methods like ImvoxelNet~\cite{rukhovich2022imvoxelnet}, BEVFormer~\cite{li2022bevformer}, BEVDepth~\cite{li2022bevdepth} on DAIR-V2X-I val set. Some results of LiDAR-based and multimodal methods reproduced by the original DAIR-V2X~\cite{yu2022dair} benchmark are also displayed.
As can be seen from Tab.~\ref{dair_sota_2}, the proposed \name{} surpasses state-of-the-art methods by a significant margin of 2.19\%, 5.87\% and 4.61\% in vehicle, pedestrian and cyclist categories respectively.

% \mypara{Rope3D val set.}
On Rope3D dataset, we also compare our BEVHeight with other leading BEV methods, such as BEVFormer~\cite{li2022bevformer} and BEVDepth~\cite{li2022bevdepth}. Some results of the monocular 3D object detectors are revised by adapting the ground plane. As shown in Tab.~\ref{tab_performance_overall},  
we can see that our method outperforms all BEV and monocular methods listed in the table. In addition, under the same configuration, our BEVHeight outperforms the BEVDepth by $4.97\%$ / $4.02\%$, $3.91\%$ / $3.06\%$ on AP$_{\text{3D}{|\text{R40}}}$ and Rope$_\text{score}$  for car and big vehicle respectively.

% \begin{table*}[ht]
%  \footnotesize \centering\addtolength{\tabcolsep}{-1pt}
 
%  \begin{tabularx}{1.0\textwidth}{ l |cc|cc|cc|cc|cc }
% \toprule
% \multirow{4}{*}{Method} & \multicolumn{2}{c|}{\multirow{2.8}{*}{Resolution}}  & \multicolumn{4}{c|}{IoU = 0.5} & \multicolumn{4}{c}{IoU = 0.7} \\ 
% \cmidrule(r){4-11}
% &  &   & \multicolumn{2}{c|}{Car} & \multicolumn{2}{c|}{Big Vehicle} & \multicolumn{2}{c|}{Car} & \multicolumn{2}{c}{Big Vehicle} \\ 
% \cmidrule(r){2-11}
% & Input &  BEV & AP$_{\text{3D}{|\text{R40}}}$ & Rope$_\text{score}$ & AP$_{\text{3D}{|\text{R40}}}$ & Rope$_\text{score}$  & AP$_{\text{3D}{|\text{R40}}}$ & Rope$_\text{score}$ & AP$_{\text{3D}{|\text{R40}}}$ & Rope$_\text{score}$\\ 

% \midrule

% M3D-RPN-R34~\cite{brazil2019m3d} & -&- 
% &54.19 & 62.65	&33.05 &  44.94 &16.75 & 32.90 &6.86  &  24.19 \\

% Kinematic3D-Dense121~\cite{brazil2020kinematic} & - & -  &50.57  & 58.86&	37.60&  48.08 &17.74  & 32.9 &   6.10&   22.88\\

% MonoDLE-DLA34~\cite{ma2021delving} & -&- 
%  & 51.70 & 60.36 & 40.34 & 50.07  & 13.58 & 29.46 &9.63 &25.80\\


% MonoFlex-DLA34~\cite{zhang2021objects} & -& 	- & 60.33 & 66.86&	37.33 &47.96   & 33.78 & 46.12 &  10.08 &26.16\\

% BEVFormer-R101~\cite{li2022bevformer} &	864x1536&	150x150	&50.62&	58.78&	34.58&	45.16&	24.64&	38.71	&10.05&	25.56\\

% BEVDepth-R50~\cite{li2022bevdepth}&	864x1536&	256x256	&69.63&	74.70&	45.02&	54.64&	42.56&	53.05	&21.47	&35.82\\

% \midrule
% BEVHeight-R50(Ours)&	864x1536&	256x256	& 74.60& 78.72& 48.93& 57.70& 45.73& 55.62& 23.07& 37.04 \\							
% \bottomrule

% \end{tabularx}

% % \caption{Overall performance of the monocular 3D object detection approaches on the Rope3D Dataset with IoU = 0.5 and 0.7. $(G)$ denotes adapting the ground plane.}
% \caption{\textbf{Comparison on the Rope3D val set under the Homologous setting.}}
% \label{tab_performance_overall}
% \end{table*}

\begin{table}[t]
\footnotesize
  \centering\addtolength{\tabcolsep}{-3.8pt}
\caption{\textbf{Results on the Rope3D val set.} Here, we follow~\cite{ye2022rope3d} to report the results on vehicles. Our method on average surpasses the state-of-the-art method over a margin of 3\% in both average precision and $Rope_{score}$ metric.
% \Tao{more caption}
}
\vspace{-0.2cm}
 \begin{tabularx}{1.\linewidth}{ l |cc|cc|cc|cc }
\toprule
\multirow{4}{*}{Method}   & \multicolumn{4}{c|}{IoU = 0.5} & \multicolumn{4}{c}{IoU = 0.7} \\ 
\cmidrule(r){2-9}
  & \multicolumn{2}{c|}{Car} & \multicolumn{2}{c|}{Big Vehicle} & \multicolumn{2}{c|}{Car} & \multicolumn{2}{c}{Big Vehicle} \\ 
\cmidrule(r){2-9}
&AP & Rope &
AP & Rope &
AP & Rope &
AP & Rope \\
% &AP$_{\text{3D}{|\text{R40}}}$ & Rope$_\text{score}$ & AP$_{\text{3D}{|\text{R40}}}$ & Rope$_\text{score}$  & AP$_{\text{3D}{|\text{R40}}}$ & Rope$_\text{score}$ & AP$_{\text{3D}{|\text{R40}}}$ & Rope$_\text{score}$\\ 

\midrule

M3D-RPN~\cite{brazil2019m3d} 
&54.19 & 62.65	&33.05 &  44.94 &16.75 & 32.90 &6.86  &  24.19 \\

Kinematic3D~\cite{brazil2020kinematic}  &50.57  & 58.86&	37.60&  48.08 &17.74  & 32.9 &   6.10&   22.88\\

MonoDLE~\cite{ma2021delving} 
 & 51.70 & 60.36 & 40.34 & 50.07  & 13.58 & 29.46 &9.63 &25.80\\


MonoFlex~\cite{zhang2021objects} & 60.33 & 66.86&	37.33 &47.96   & 33.78 & 46.12 &  10.08 &26.16\\

BEVFormer~\cite{li2022bevformer}	&50.62&	58.78&	34.58&	45.16&	24.64&	38.71	&10.05&	25.56\\

BEVDepth~\cite{li2022bevdepth}	&69.63&	74.70&	45.02&	54.64&	42.56&	53.05	&21.47	&35.82\\

\midrule
 \rowcolor{cyan!30} BEVHeight & \textbf{74.60}& \textbf{78.72}& \textbf{48.93}& \textbf{57.70}& \textbf{45.73}& \textbf{55.62}& \textbf{23.07}& \textbf{37.04} \\							
\bottomrule
\multicolumn{9}{l}{\footnotesize{AP and Rope denote AP$_{\text{3D}{|\text{R40}}}$ and Rope$_\text{score}$ respectively.}}
\end{tabularx}
% }
% \caption{Overall performance of the monocular 3D object detection approaches on the Rope3D Dataset with IoU = 0.5 and 0.7. $(G)$ denotes adapting the ground plane.}
 \vspace{-0.50cm}
\label{tab_performance_overall}
\end{table}




% 地面方程咋用的

\mypara{Results on noisy extrinsic parameters.}
% To verify the robustness of our BEVHeight when the camera's extrinsic matrix is changed inevitably. 
In the realistic world, camera parameters frequently change for various reasons. Here we evaluate the performance of our framework in such noisy settings. We follow \cite{yu2022benchmarking} to simulate the scenarios that external parameters are changed. Specifically, we introduce a random rotational offset in normal distribution $N(0, 1.67)$ along the roll and pitch directions as the mounting points usually remain unchanged.  
% \KY{degree of what?}

% This situation often occurs during the maintenance of roadside cameras. In this case, the camera's extrinsic matrix will differ from its previous state when the labeled data is collected. The generalization to the camera's mount position disturbance is a great challenge for the existing methods.
During the evaluation, we add the rotational offset along roll and pitch directions to the original extrinsic matrix. The image is then applied with rotation and translation operations to ensure the calibration relationship between the new external reference and the new image. Examples are given in Sec.~\ref{sec:visualization_results}.
As shown in Tab.~\ref{dair_robust}, the performance of the existing methods degrades significantly when the camera's extrinsic matrix is changed. Take $\text{Vehicle}_{(IoU=0.5)}$ for example, the accuracy of BEVFormer~\cite{li2022bevformer} drops from 50.73\% to 16.35\%. The decline of BEVDepth~\cite{li2022bevdepth} is from 60.75\% to 9.48\%, which is pronounced. Compared with the above methods, Our BEVHeight maintains 51.77\% from the original 63.49\%, which surprises the BEVDepth by 42.29\% on vehicle category.

\mypara{Visualization Results.}
\section{Visualization On Demand} %Visualization Elements
\label{sec:visrisk}
Based on environment data and trajectory evaluation, we now present ways of communicating the situation and risks on a visual display to achieve an ADAS.
In this context, we employ a renderer that visualizes all the information in a joint Cartesian coordinate system (see section \ref{subsec:sim}). 
Once driving risks are detected, design elements are overlayed on the display with section \ref{subsec:active} and section \ref{subsec:warning}. 

\subsection{Simulator Environment}
\label{subsec:sim}
Nodes of the R-LDM have a range of potential attributes, such as the 3D position or geometrical shape of objects. 
% For instance, the road centerline is a polyline with bounderies to the left and right. Crosswalks have a defined width and buildings a polygonal outline description. 
In the renderer, we always visualize static and quasi-static data that lie in the field of view from the ego vehicle. 
For this, a local 3D model is generated by converting geographic points with (lat, lon, alt) into Cartesian coordinates of (x, y, z). 
% and project the positonal relations from a view perspective with a transformation matrix. 
Fig. \ref{fig:3Dsimulator} depicts an exemplary map section having several intersections in bird's-eye view.
% with several intersections, stop lines and crosswalks. 
On the top right, the first person view of a vehicle approaching a crosswalk is shown. 

The dynamic data is then added to this static view. A zoomed-in excerpt from the map is given at the bottom of Fig. \ref{fig:3Dsimulator} that includes a recorded GNSS trace (red).
We project the trace onto the connected lane center, which is pictured in green. 
% Because we project the ego position on the closest lane segment, on the bottom right the measured trace is changed in red and the aligned trace is marked in green.
Consequently, the virtual horizon and its possible paths are retrieved as described in section \ref{subsec:ldm}. 
We can lastly update and move the excerpt with the current position from the GNSS to obtain a live simulation.

\subsection{Proactive Support}
\label{subsec:active}
Communication of spatial as well as spatio-temporal relations is crucial for risk-averse driver support. 
% This has the reason that humans can estimate the time better than positions (especially for risks). 
% Velocity contains implicitly the time as well. 
Further sources of information are cause, likelihood and severity of a potential risks.  
% if a collision happens. 
The next step for RNS is the choice of suitable design elements. 
In this process, we suppose that we know where the ego vehicle is driving (i.e., the ego path) from its navigation route. 
Yet, for surrounding vehicles, all paths are considered.

\subsubsection{Hazard Route Element}
The so-called hazard route in Fig. \ref{fig:charts} is a concept that consists of a scale portraying distances to an upcoming risk element.
Furthermore, the geometrical area or length of risks is considered.
Risk is thus measured with respect to the ego path, ranging from the current position  $\Delta l \hspace{-0.03cm}=\hspace{-0.03cm} \unit[0]{m}$ to the end of the path $\Delta l_{h}$.
Here, the length $\Delta l_{h}$ can be chosen according to own preferences. 

At an upcoming intersection, risk is defined by the section of the path that lies within the junction.
Since risk corresponds to exposition time, we encode the path part from the intersection $I_z$ with a color, ranging from green for short intersections to red for long ones. 
%allgemein risiko entlang des pfades zu intersection zone
%share of junction segment to navigation route + 
%one case with large intersection far and one case with small intersection close
Fig. \ref{fig:charts}~a) gives two examples of the hazard route.
The left bar shows a large intersection (e.g. multi-lane four-way stop) in vicinity and the right bar has a small and consecutive medium junction. 
% In the case of collision risk, the intersection zone $I_z$ can be used.
% Depending on the value of $I_z$ (low, medium and large), the area is marked from green, to yellow until red for conveying the criticality. 
This emphasizes that we may include more than one intersection in our warnings.

\begin{figure}[t]
  \centering
  \includegraphics[width=0.95\linewidth]{./img/simulator.png}
  \caption{Rendered road network from two perspectives with the ego position being projected on the navigation route. \vspace{0.45cm}}
  \label{fig:3Dsimulator}
\end{figure}

\begin{figure}[t]
  \centering
  \resizebox{\linewidth}{!}{
  \import{img/}{velocity_scale_new.pdf_tex}}  
  \caption{Chart elements for proactive support. Hazard route (left) and velocity scale (right).} %\vspace{-0.3cm}}
  \label{fig:charts} 
\end{figure} 

\subsubsection{Velocity Scale Element}
The velocity scale, Fig. \ref{fig:charts}~b), is a second chart element which qualifies the difference between the current velocity of the vehicle $v_0$ and the target velocity $v_{\text{tar}}$ from the trajectory evaluation of section \ref{subsec:trajeval}. 
The scale shows possible velocity values, from standstill $v\hspace{-0.05cm}=\hspace{-0.05cm}\unit[0]{m/s}$ to a maximal velocity $v_{\text{max}}$. Depending on the difference $|v_0 \hspace{0.05cm} - \hspace{0.05cm} v_{\text{tar}}|$, the situation is rated as safe with $v_0 \hspace{-0.042cm} \approx \hspace{-0.042cm} v_{\text{tar}}$ (green, left), as dangerous with e.g. $v_0 \hspace{-0.05cm} < \hspace{-0.05cm} v_{\text{tar}}$ (yellow, middle) to critical with $v_0 \hspace{-0.07cm} \ll \hspace{-0.07cm} v_{\text{tar}}$ (red, right). The same cases hold true for the opposite circumstances, i.e., $v_0 \hspace{-0.032cm} > \hspace{-0.032cm} v_{\text{tar}}$. 
This velocity scale can be employed for curve or regulatory risks. 
Moreover, we may set an enforced speed limit as the target velocity $v_{\text{tar}}$ for proactive behavior, once there is no risk ahead. 
%\noindent -Warning vs behavior support \\
%-Ghost vehicle as in game \\

\subsection{Short-Term Warning Elements}
\label{subsec:warning}
In order to emphasize the criticality of the situation, we propose to add further intuitive warning elements as e.g. pop-up signs and lane colorings. 
The following elements augment the proactive elements.

\subsubsection{Pop-up Signs}
Explicit symbols indicate the risk cause accompanied with the event time for collisions ($s_E$), distances to the risk spot for turns (i.e., right curve with $d_r$ and left curve with $d_l$) or stopping distance for crosswalks ($d_c$). In Fig. \ref{fig:popups}~a), the pop-up signs are pictured. 
% Besides the velocity difference, the risk type is an indication for the severity of the situation.
%Examples for collision risk are car-to-car crash., curve risk can be  as a single-car accident and regulatory risks will be a car-to-object collision. 
We want to stress that this is just a selection and more risk causes can be added. 
The purpose is also to clarify the reason for the warning and give more human-understandable information.

\subsubsection{Colored Events}
Finally, we highlight lane parts or positions according to the corresponding risks.  
% the determined color rating from the hazard route and velocity scale and relate the risks to the simulator environment. 
In the instance of curve and regulatory risk, the lane is colored from the ego position up to the point of maximal risk. 
For collision risk, we mark the point of the closest encounter as a red cube.
An illustration for regulatory risk induced from a stop line is depicted in Fig. \ref{fig:popups}~b). Again, the color is defined by the deviation $|v_0-v_{\text{tar}}|$. It also shows the therein considered navigation route with length $\Delta l_h$ and another unlikely path. 

It should be noted that the visualization of warnings only occurs if the risks are actually present. 
%\textcolor{red}{improve language, repeat intersection zone and navigation route}
%eingrauen unlikely paths and navigation path and describe in text, maybe delete Iz -> put line from unlikely path to green arrow
Altogether, the RNS provides a variety of tools to analyze and circumvent critical situations in intersection scenarios, while not overloading the driver's awareness.

\begin{figure}[t]
  \centering
  \resizebox{\linewidth}{!}{
  \import{img/}{colored_lane_new.pdf_tex}}  
  \vspace{-0.53cm}
  \caption{Short-term warning elements. Selected pop-up warnings (left) and colored lane (right).}
  \label{fig:popups} 
\end{figure} 


\label{sec:visualization_results}
As shown in Fig.~\ref{fig:visualization}, we present the results of BEVDepth~\cite{li2022bevdepth} and our BEVHeight in the image view and BEV space, respectively. The above two models are not applied with data augmentations in the training phase. From the samples in (a), we can see that the predictions of BEVHeight fit more closely to the ground truth than that of BEVDepth. As for the results in (b), under the disturbance of roll angle, there is a remarkable offset to the far side relative to the ground truth in BEVDepth detections. In contrast, the results of our method are still keeping the correct position with ground truth.  Moreover, referring to the predictions in (c), BEVDepth can hardly identify far objects, but our method can still detect the instance in the middle and long-distance ranges and maintain a high IoU with the ground truth.

\begin{table}[t]
 \scriptsize\centering\addtolength{\tabcolsep}{-3.25pt}
   \caption{
%   \textbf{The robustness under extrinsic parameter perturbations on DAIR-V2X-I.} \Tao{more caption}
    \textbf{Results on robustness settings. } Here, we simulate the robustness scenarios where the external parameters of the camera changes. Consider the  Specifically, we consider two degrees of freedom mutation, roll and pitch of the camera center. In both dimensions, we randomly sample angles from a normal distribution of $\mathcal{N}(0, 1.67)$. Surprisingly, given such minor changes, traditional depth-based methods decrease to under 15\% even for those vehicles under easy settings. On the contrary, our methods achieve around 577\% improvement compared to those baselines, evidencing the robustness of \name{}.
%   “roll” and “pitch” means applying an additional rotation offset in normal distribution N(0, 1.67) to the camera’s extrinsic matrix along roll and pitch directions
   }
  %\vspace{-0.1cm}
 \begin{tabularx}{1.0\linewidth}{l|cc|ccc|ccc|ccc}
  \toprule
 \multirow{3}{*}{\rotatebox{90}{Model}} & \multicolumn{2}{c|}{Disturbed}
    \ & \multicolumn{3}{c|}{$\text{Veh.}_{(IoU=0.5)}$} & \multicolumn{3}{c|}{$\text{Ped.}_{(IoU=0.25)}$} & \multicolumn{3}{c}{$\text{Cyc.}_{(IoU=0.25)}$}  \\
   \cmidrule(r){2-12}
   & roll &	pitch & Easy & Mid & Hard & Easy & Mid & Hard & Easy & Mid & Hard  \\
  
    \midrule
 \multirow{4}{*}{\rotatebox{90}{BEVFormer}} &&&	61.37&	50.73	&50.73&	16.89&	15.82&	15.95	&22.16&	22.13&	22.0\\
	&	\checkmark& 	&	50.65&	42.9&	42.95&	10.16&	9.41&	9.47&	13.62&	13.71&	13.08\\
	&		&\checkmark&	46.40&	38.26&	38.37&	9.12	&8.44&	8.55&	8.99 &	8.43&	8.42 \\
 	&	\checkmark&	\checkmark&	19.24&	16.35&	16.47&	3.93&	3.43&	3.52&	4.93&	4.98&	4.98\\
\midrule
 \multirow{4}{*}{\rotatebox{90}{BEVDepth}}	& & &			71.56& 	60.75&	60.85&	21.55&	20.51&	20.75&	40.83	&40.66&	40.26\\
&	\checkmark	&&	34.82&	28.32&	28.35&	4.49&	4.36&	4.39&	10.48&	9.51&	9.73\\
	&&	\checkmark&	14.04&	11.41&	11.49&	3.01&	2.67&	2.75&	6.43&	6.23&	6.83\\
	&	\checkmark &\checkmark &	11.84&	9.48&	9.54&	2.16&	1.84&	1.89&	4.31&	4.14&	4.26\\
\midrule
 \multirow{4}{*}{\rotatebox{90}{BEVHeight}}	&	&&	75.58&	63.49&	63.59&	26.93&	25.47&	25.78&	47.97	& 47.45	& 48.12	\\
	&	\checkmark &&	66.06&	54.99&	55.14&	18.66&	17.63&	17.78&	34.45&	26.93&	27.68\\
	&&	\checkmark&	68.49&	56.98&	57.11&	17.94&	16.87&	17.09&	34.48&	27.82&	28.67\\
	&	\checkmark &	\checkmark&	62.64&	51.77&	51.9&	14.38&	14.01&	14.09&	31.28&	25.24&	26.02\\

    \bottomrule
    % \multicolumn{21}{l}{Mid: Middle, Veh.: Vehicle, Ped.: Pedestrian, Cyc.: Cyclist.}
  \end{tabularx}

  \label{dair_robust}
\end{table}

\section{A test-space only discretization}

As indicated in the previous section, we propose to use the normal equation~\eqref{eq:transport:continuousNormalEq} to define an optimally stable approximation scheme.
It is thus obvious, that a discretization can be fully based on a discrete approximate test space $\ycal^\delta$.

\subsection{Discrete normal equation and functional reconstruction}

Let $\ycal^\delta \subseteq \ycal$ be a conforming discretization of the optimal test space (\ie using a standard Lagrange finite element space). Based on~\eqref{eq:transport:continuousNormalEq} we then define the \textit{discrete} normal equation using Galerkin-projection.
\begin{equation}\label{eq:discreteNormalEq}
\text{Find}\; w^\delta \in \ycal^\delta: \quad {(A^*[w^\delta], A^*[v^\delta])}_{L^2(\Omega)} = f(v^\delta) \qquad \forall v^\delta \in \ycal^\delta.
\end{equation}
Note that this is still an optimally conditioned problem. Given the discrete solution $w^\delta$ we may reconstruct the discrete solution $u^\delta = A^*[w^\delta]$. Technically, this solution lies in the finite-dimensional subspace $\xcal^\delta := A^*[\ycal^\delta] \subseteq \xcal$, however, due to its non-accessible structure this space is of no practical use.

Previous work often used knowledge of the structure of $A^*$ to determine a larger, more traditional (DG-)space $\zcal^\delta \supsetneq \xcal^\delta$ and then assembled the matrix $\underline{A}$ representing the operator $A^*: \xcal^\delta \rightarrow \zcal^\delta$ in the respective standard FE-bases. In this case, one can determine the system matrix $\underline{A}^{NE}$ of the normal equation as $\underline{A}^{NE} = \underline{A}^T \underline{M}_\zcal \underline{A}$ (where $\underline{M}_{\zcal}$ denotes the inner-product matrix in $\zcal^\delta$), solve the linear system\vspace*{-0.5em}
\begin{equation}\label{eq:linearEquationSystem}
  \underline{A}^{NE}\underline{w} = \underline{f}\vspace*{-0.5em}
\end{equation}
and compute the coefficients $\underline{u}$ of $u^\delta \in\xcal^\delta \subset \zcal^\delta$ in the basis of $\zcal^\delta$ by simply computing $\underline{u} = \underline{A}\,\underline{w}$.

However, this is suboptimal as the construction of a discrete larger space $\zcal^\delta$ is only feasible or even possible with suitable additional assumptions on the data, e.g. (elementwise) constant data functions. For non-constant reaction or velocities one has to resort to a nonconforming choice $\zcal^\delta \not\supset \xcal^\delta$ introducing an additional projection error which might be difficult to estimate or control.

Here, we propose an approach that avoids ever computing a matrix $\underline{A}$ representing the operator $A^*$. The system matrix of the normal equation $\underline{A}^{NE}$ can also be directly assembled in a basis of $\ycal^\delta$ which means basically assembling a normal equation using the full infinite dimensional trial space $\xcal$. The reconstruction $u^\delta := A^*[w^\delta]$ is now seen as an element of $\xcal$ (we technically know that it lies in the finite dimensional subspace $\xcal^\delta \subset \xcal$ but this does not give us any useful information). The crucial insight is that in almost all applications only functional evaluations of $u^\delta$ are needed. Examples include point-evaluations for the visualization of $u^\delta$ or the computation of quantities of interest via numerical quadrature (\ie $\norm{u^\delta}$). Therefore, we replace the reconstruction by functional evaluations and e.g. do a pointwise reconstruction. Note that in this way we do not introduce any additional projection error.

\subsection{Conditioning of the system matrix and solving the linear system}
Solving the linear equation system~\eqref{eq:linearEquationSystem} is actually quite a challenging task - a problem that has to our knowledge not been discussed so far. Although Problem~\eqref{eq:discreteNormalEq} is optimally stable in theory, the condition of the system matrix $\underline{A}^{NE}$ still scales quadratically in the inverse grid width $h^{-1}$ and is thus a significant challenge even for moderately large problems. To better understand these seemingly conflicting statements consider the non-symmetric formulation of~\eqref{eq:discreteNormalEq}:
\begin{equation}\label{eq:discreteNonsymmetricEq}
\text{Find}\; u^\delta \in \xcal^\delta: \quad (u^\delta, A^*[v^\delta]) = f(v^\delta) \qquad \forall v^\delta \in \ycal^\delta.
\end{equation}
Let $\{\psi_i\}_{i=1}^{N}$ be a basis of $\ycal^\delta$ (\eg a finite element basis). Then, the set $\{ \varphi_i \}_{i=1}^N$, $\varphi_i := A^*[\psi_i]$ forms a basis of $\xcal^\delta$ and the matrix $\underline{A}$ representing $A^*$ in these bases is the identity matrix. The condition of the system matrix $\underline{A}^{NE}$ is still of order $\mathcal{O}(h^{-2})$ since the trial functions $\varphi_i$ have, contrary to classic finite elements, in this case a magnitude of $\mathcal{O}(h^{-1})$.

As mentioned in Remark~\ref{rmk:strongTestProblem}, the normal equation can also be seen as the weak form of a specific Poisson-problem with rank-deficient diffusion tensor $D$. In the following numerical experiments we thus employed an algebraic multigrid for preconditioning and a conjugate gradient (CG) solver - methods that are known to perform well for this type of problems. For more complex problems (\eg for velocity fields with (locally) small magnitude) the efficient preconditioning and solving of the linear equation system~\eqref{eq:linearEquationSystem} still needs further investigation.

\subsection{Ablation Study}
% \mypara{Robust to Extrinsic Disturbance.}

\mypara{Dynamic Discretization.}
Experiments in Tab.~\ref{dair_discretization} show the detection accuracy improvement 0.3\% - s1.5.0\% when our dynamic discritization is applied instead of uniform discretization(UD).
The performance when hype-parameter $\alpha$ is set to 2.0 suppresses that of 1.5 in most cases, which signifies that hype-parameter $\alpha$ is necessary to achieve the most appropriate discretization.

\mypara{Analysis on Point Cloud Supervision.}
\begin{table}[t]
 \scriptsize\centering\addtolength{\tabcolsep}{-2.9pt}
 \caption{\textbf{Results with point cloud supervision on DAIR-V2X-I dataset.} We can observe that for both BEVDepth and BEVHeight, LiDAR point cloud supervision did not help in terms of evaluation results. This is another evidence that road-side perception is different from the ego-vehicle one.  }
% \vspace{-0.2cm}
 \begin{tabularx}{1.0\linewidth}{l|ccc|ccc|ccc}
  \toprule
 \multirow{2}{*}{Method}    & \multicolumn{3}{c|}{$\text{Veh.}_{(IoU=0.5)}$} & \multicolumn{3}{c|}{$\text{Ped.}_{(IoU=0.25)}$} & \multicolumn{3}{c}{$\text{Cyc.}_{(IoU=0.25)}$}  \\
   \cmidrule(r){2-10}
 & Easy & Mid & Hard & Easy & Mid & Hard & Easy & Mid & Hard  
 \\
    \midrule
    BEVDepth	& 71.56& 	60.75&	60.85&	21.55&	20.51&	20.75&	40.83	&40.66&	40.26\\
    BEVDepth$\dagger$	&	71.09&	60.37&	60.46&	21.23&	20.84&	20.85&	40.54&	40.34&	40.32\\
    \midrule
    BEVHeight	 &	75.58&	63.49&	63.59&	26.93&	25.47&	\textbf{25.78}&	47.97	& 47.45	& 48.12	\\
    BEVHeight$\dagger$	& \textbf{75.64}& \textbf{63.61}&	\textbf{63.72}&	\textbf{27.01}&	\textbf{25.55}&	25.34&	\textbf{48.03}&	\textbf{47.62}&	\textbf{48.19}\\
    \bottomrule
    \multicolumn{10}{l}{\scriptsize{$\dagger$ denotes training with PointCloud supervision.}}
  \end{tabularx}
  \label{pc_sup}
\vspace{-0.55cm}
\end{table}

To verify the effectiveness of point cloud supervision in roadside scenes, we conduct ablation experiments on both BEVDepth~\cite{li2022bevdepth} and our method. As shown in Tab.~\ref{pc_sup}, BEVDepth with point cloud supervision is slightly lower than that without supervision. As for our BEVHeight, although there is a slight improvement under the IoU=0.5 condition, the overall gain is not apparent. This can be explained by the fact that the background in roadside scenarios is stable. These background point clouds fail to provide adequate supervised information and increase the difficulty of model fitting.
% there is only a slight improvement under IoU=0.5. We speculate this is because the camera is fixed in roadside scenarios, thus the majority of the pixels belong to the background and have relatively fixed depth or height values. The network can learn them well even without ground truth as supervision.
% We speculate that this is due to the fact that the background in roadside scenarios is stable. 
% These background point clouds remain unchanged and fail to provide adequate supervised information
% and increase the difficulty of model fitting.
% 分析,depth的问题,猜测不同路口混合训练的影响,加了监督更加hard让网络拟合了

% \mypara{Analysis on point cloud supervision.}

\mypara{Analysis on Distance Error.}
To provide a qualitative analysis of depth and height estimations, we convert depth and height to the distance between the predicted object's center and the camera’s coordinate origin, as is shown in Fig.~\ref{fig:distance_correlation}.  Compared with the distance error triggered by depth estimation in BEVDepth\cite{li2022bevdepth}, the height estimation in our BEVHeight introduces less error, which illustrates the superiority of height estimation over the depth estimation in the roadside scenario.
\begin{figure}[t!]
	\centering
	\includegraphics[width=8.5cm]{BEV-Height/figures/distance_correlation.pdf}
	\caption{\textbf{Empirical analysis of the distance correlation.} All experiments are conducted on the DAIR-V2X-I val set. (a) and (b) reveal the distance correlation between ground truth and predicted distance on the BEVDepth and our BEVHeight. We take distances from the camera's coordinate system origin to the annotated objects' center for consideration. Each point represents an annotated instance. The scatter diagram of BEVHeight in (b) is closer to the diagonal than that of BEVDepth in (a), indicating that the distance error triggered by height estimation is more minimal than the depth candidate.}

\label{fig:distance_correlation}
\vspace{-0.5cm}
\end{figure}

\mypara{Latency.}
As shown in Tab.~\ref{latency_rebuttal}, we benchmark the runtime of BEVHeight and BEVDepth. With an image size of 864x1536, BEVDepth runs at 14.7 FPS with a latency of 68ms, while ours runs at 16.1 FPS with 62ms, which is around 5\% faster. It is due to the depth range (1$\sim$104m) being much larger than height (-1$\sim$1m), thus ours has 90 height bins that less than 206 depth ones,
leading to a slimmer regression head and fewer pseudo  points for voxel pooling. It evidences the superiority of predicting height instead of depth and advocates the efficiency of our method.
\begin{table}[h!t]
\scriptsize\centering\addtolength{\tabcolsep}{-2.0pt}
\renewcommand\arraystretch{1.0}
\caption{{\bf Latency of BEVHeight and BEVDepth.} }
\begin{tabular}{l|c|c|c|c|c}
\toprule   
Methods& Backbone &Range & Number of bins & Latency (ms) & FPS \\ 
\midrule
BEVDepth~\cite{li2022bevdepth} & R50 & 1 - 104m& 206& 82& 12.2\\
\rowcolor{cyan!30} BEVHeight& R50 & -1 - 1m&  90& 77& 13.0\\
\midrule
    BEVDepth~\cite{li2022bevdepth} &  R101& 1 - 104m& 206& 68& 14.7\\
\rowcolor{cyan!30}	BEVHeight  & R101& -1 - 1m&  90& 62& 16.1\\
\bottomrule 
\multicolumn{6}{l}{\scriptsize Measured on a V100 GPU. Image shape 864×1536.}
\end{tabular}
\label{latency_rebuttal}
\vspace{-0.25cm}
\end{table}



\mypara{Limitations and Analysis.}
Though the motivation of our work is to address the challenges in the roadside scenarios, we nonetheless benchmark our methods on nuScenes to study the effectiveness. Here, the input resolution is set to (256, 704). We follow the setting of BEVDepth, i.e. the training lasts for 24 epochs. Note that, we did not use other tricks such as class-balanced grouping and sampling~(CBGS) strategy~\cite{zhu2019cbgs}, exponential moving average or multi-frame fusion. 
In \cref{tab:nus}, we observe that our method falls behind the BEVDepth by around 0.02 in mAP metrics. This shows that our method has limited performance on ego-vehicle settings. 

% \Tao{256x704 128x128 pc sup}
% \KY{to finish}
% For the nuScenes\cite{HolgerCaesar2019nuScenesAM} dataset, the input image is scaled to 256x704; we adopt the same dataset augmentations on image-view and BEV features as in BEVDepth\cite{li2022bevdepth}. All experiments are trained for 24 epochs without using the CBGS strategy, EMA, and multi-frame fusion.

\begin{table}[!t]
 \centering\addtolength{\tabcolsep}{-4.15pt}
\footnotesize
\caption{\textbf{Limitation of our method.} We present the results on the nuScenes validation dataset. We notice that our methods fall behind the traditional BEVDepth on the ego-vehicle settings by 2\%. This shows that our methods are effective on cameras with high installation and bird's-eye-view as in the roadside scenario, and is not ideal on cameras mounted on ego-vehicles.}
% \vspace{-0.1cm}
\begin{tabularx}{1.0\linewidth}{l|cccccccccc}
\toprule
 Method  &
			 mAP$\uparrow$ & NDS$\uparrow$ & mATE$\downarrow$ & mASE$\downarrow$  & mAOE$\downarrow$ & mAVE$\downarrow$ & mAAE$\downarrow$   \\
\midrule
\textcolor{gray!80}{BEVDepth}	&\textcolor{gray!80}{	0.315}&	\textcolor{gray!80}{0.367}&	\textcolor{gray!80}{0.702}&\textcolor{gray!80}{	0.271}&	\textcolor{gray!80}{0.621}&	\textcolor{gray!80}{1.042}&\textcolor{gray!80}{	0.315}\\
BEVDepth*	&		0.313&	0.354&	0.713&	0.280&	0.655&	1.230	&0.377\\
\midrule
BEVHeight	&	0.291&	0.342&	0.722&	0.278&	0.674&	1.230&	0.361\\
\bottomrule
\multicolumn{10}{l}{\footnotesize{* denotes the results we reproduce.}}
\end{tabularx}
\label{tab:nus}
\vspace{-0.60cm}
\end{table}

Firstly, our method does \emph{not} assume the ground-plane is fixed, and it is not the reason why our method cannot surpass the depth-based one on ego-vehicle settings. To verify, we collect around 13 thousand sequences from the camera mounted on a moving truck with a ground height of 3.14m, and annotate the 3D object box following nuScenes. As shown in Tab.~\ref{ddd_rebuttal} We observe that our BEVHeight again surpasses the depth-based state-of-the-art by a large margin, evidences the performance is affected by the camera height but not time-varying ground plane and it can work on ego-vehicle settings.
We visualize three cameras observing the same object and analyze the detection error in Fig.~\ref{fig:versatility_analysis}: (a) shows when the height prediction is equal to the ground-truth, detection is perfect for all cameras; (b) if not, for the same height prediction error, the distance between the predicted point and ground-truth is inversely proportional to the camera ground height. This is why BEVHeight achieves on-par performance on nuScenes but quickly surpasses BEVDepth~\cite{li2022bevdepth} when the camera height only increases less than 1 meter.
% \begin{table*}[ht]
%  \centering\addtolength{\tabcolsep}{-0.6pt}
%  \resizebox{0.8\textwidth}{!}{
%  \begin{tabularx}{1.0\textwidth}{l|c|ccc|ccc|ccc}
%   \toprule
%  \multirow{3}{*}{Method} &  
%  \multirow{3}{*}{Modality}  
%  & \multicolumn{3}{c|}{$\text{Vehicle}_{(IoU=0.5)}$} & \multicolumn{3}{c|}{$\text{Pedestrian}_{(IoU=0.25)}$} & \multicolumn{3}{c}{$\text{Cyclist}_{(IoU=0.25)}$} \\
%     \cmidrule(r){3-11}
%      &  & Easy & Mid & Hard & Easy & Mid & Hard & Easy & Mid & Hard  \\
% \midrule

% PointPillars~\cite{lang2019pointpillars} & PointCloud &63.07 & 54.00 & 54.01 & 38.53 & 37.20 & 37.28 & 38.46 & 22.60 & 22.49 \\
% SECOND~\cite{yan2018second} & PointCloud &71.47 & 53.99 & 54.00 & 55.16 & 52.49 & 52.52 & 54.68 & 31.05 & 31.19 \\
% MVXNet~\cite{Sindagi2019MVX} & Image+PointCloud &71.04 & 53.71 & 53.76 & 55.83 & 54.45 & 54.40 & 54.05 & 30.79 & 31.06 \\
% \midrule
% ImvoxelNet~\cite{rukhovich2022imvoxelnet} &Image & 44.78 & 37.58 & 37.55 & 6.81 & 6.746 & 6.73 & 21.06 & 13.57 & 13.17 \\
% BEVFormer-R101$\ast$~\cite{li2022bevformer} & Image 	&	61.37&	50.73&	50.73&	16.89&	15.82&	15.95	&22.16&	22.13&	22.06\\
% BEVDepth-R101$\ast$~\cite{li2022bevdepth}&	Image 	&	76.01&	64.11&	64.18&	24.32&	24.96&	24.84	&46.45&	45.56&	45.69	\\

% \midrule
% BEVHeight-R101(Ours) & Image &	79.12&	67.95&	67.04&	29.85&	29.31&	29.07	&51.55&	51.39&	50.91\\
%     \bottomrule
%   \end{tabularx}
%   }
%   \caption{\textbf{Comparison on the DAIR-V2X-I val set.}}
%   \label{dair_sota}
% \end{table*}


\begin{table}[h!t]
 \scriptsize\centering\addtolength{\tabcolsep}{1.0pt}
\caption{\textbf{Experiments on the dataset collected by higher truck.}} 
 \resizebox{1.0\linewidth}{!}{
 \begin{tabularx}{1.0\linewidth}{l|ccc|ccc}
 \toprule
 \multirow{3}{*}{Method} &
\multicolumn{3}{c|}{$\text{Car}_{(IoU=0.5)}$} & \multicolumn{3}{c}{$\text{Big Vehicle}_{(IoU=0.5)}$} \\
 \cmidrule(r){2-7}
   & Easy & Mod. & Hard & Easy & Mod. & Hard \\
 \midrule
 BEVDepth ~\cite{li2022bevdepth} &	50.05 &	 36.82 &	36.82&	30.15&	24.74&	24.74	\\
 \rowcolor{cyan!30}BEVHeight & \textbf{51.77}&	\textbf{40.96}&	\textbf{40.96}&	\textbf{34.65}&	\textbf{29.01}&	\textbf{29.01}\\
\bottomrule
\end{tabularx}
}
\label{ddd_rebuttal}
\end{table}





\begin{figure}[!h]
\centering
% \vspace{-0.2cm}
\includegraphics[width=8.5cm]{BEV-Height/figures/versatility_analysis_rebuttal_v2.0.jpg}
\vspace{-0.60cm}
\caption{\textbf{Distance error analysis caused by same height estimation error on different platform cameras.}}
\vspace{-0.50cm}
\label{fig:versatility_analysis}
\end{figure}

\mypara{Contributions.}
Theoretically, our proposed height-based pipeline entails: i) representation agnostic to distance, as visualized in Fig.~\ref{fig:teaser}, ii) friendly prediction owing to centralized distribution as displayed in Fig.~\ref{fig:histogram-depth-height}, iii) robustness against extrinsic disturbance as illustrated in Fig.~\ref{fig:five}. Technically, we design a novel HeightNet and the projection module with less computational cost. Experimentally, experiments on various datasets and multiple depth-based detectors show the superiority of our method in both accuracy and latency.

\section{Conclusion}
We notice that in the domain of roadside perception, the depth difference between the foreground object and background quickly shrinks as the distance to the camera increases, this makes state-of-the-art methods that predict depth to facilitate vision-based 3D detection tasks sub-optimal. On the contrary, we discover that the per-pixel height does not change regardless of distance. To this end, we propose a simple yet effective framework, \name{}, to firstly predict the height and then project the 2D feature to 3D space to improve the detector. Through extensive experiments, \name{} surpasses BEVDepth baseline by a margin of 4.85\% and 4.43\% on DAIR-V2X-I and Rope3D benchmarks under the traditional clean settings, and by 26.88\% on robust settings where external camera parameters changes. We hope our work can shed light on studying more effective feature representation on roadside perception.

\section*{Acknowledgments}
This work was supported by the National High Technology Research and Development Program of China under Grant No. 2018YFE0204300, the National Natural Science Foundation of China under Grant No. 62273198, U1964203, the China Postdoctoral Science Foundation (No. 2021M691780).

%%%%%%%%% REFERENCES
{\small
\normalem
\bibliographystyle{ieee_fullname}
\bibliography{egbib}
}

\clearpage
\begin{appendices}
\section{Appendix}

\subsection{Broader Impacts} % \Tao{repeats with limitation?}
Our work aims to develop a vision-based 3D object detection approach for roadside perception. The proposed method may produce inaccurate predictions, leading to incorrect decision-making for cooperative autonomous vehicles and potential traffic accidents. Furthermore, we propose a new perspective of leveraging height estimation to solve PV-BEV transformation, facilitating a high-performance and robust vision-centric BEV perception framework. Although considerable progress has been made with our proposed height net and height-based 2D-3D projection module, we believe it is worth further exploring how to combine height and depth estimations to extend to autonomous driving scenarios.

\subsection{Dynamic Discretization}
The height discretization can be performed with uniform discretization (UD) with a fixed bin size, spacing-increasing discretization (SID)~\cite{HuanFu2018DeepOR} with increasing bin sizes in logspace, linear-increasing discretization (LID)~\cite{YunleiTang2020Center3DCM}and our proposed dynamic-increasing discretization (DID) with adjustable bin sizes. The above four height discretization techniques are visualized in Fig. \ref{fig:discretization}. Following DID strategy, the distribution of height bins can be dynamically adjusted with different hyper-parameter $\alpha$.

%figure7
\begin{figure}[ht]
\centering	\includegraphics[width=8.5cm]{BEV-Height/figures/space_strategy.pdf}
	\caption{\textbf{Height Discretization Methods.} Height $h_i$ is discretized over a height range $[h_{min}, h_{max}]$ into $N$ discrete bins. From left to right, these are uniform discretization (UD), spacing-increasing discretization (SID), linear-increasing discretization (LID) and the adjustable dynamic-increasing discretization(DID). For the dynamic-increasing discretization (DID) strategy, height bins with large $\alpha$ are more densely distributed when approaching the $h_{min}$ than the small hyper-parameter $\alpha$ conditions.}
\label{fig:discretization}
\end{figure}

\subsection{Results on V2X-Sim Dataset}
To certify the effectiveness of our method in multi-view scenarios, we conduct experiments on V2X-Sim~\cite{li2022v2x} simulation dataset that contains four surround roadside cameras.
As shown in Tab.~\ref{v2x_sim_rebuttal}, our BEVHeight surpass the BEVDepth by more than 10.88\%, 21.15\% on vehicle and cyclist respectively, which verifies the effectiveness of our method.

\begin{table}[h!t]
 \scriptsize\centering\addtolength{\tabcolsep}{1.0pt}
\caption{\textbf{Comparison on the V2X-Sim Detection Benchmark.}}
 \begin{tabularx}{1.0\linewidth}{l|ccc|ccc}
 \toprule
 \multirow{3}{*}{Method} &
\multicolumn{3}{c|}{$\text{Vehicle}_{(IoU=0.5)}$} & \multicolumn{3}{c}{$\text{Cyclist}_{(IoU=0.25)}$} \\
 \cmidrule(r){2-7}
   & Easy & Mod. & Hard & Easy & Mod. & Hard  \\
 \midrule
 {BEVDepth~\cite{li2022bevdepth}}  &	81.99&	81.39&	81.31&	45.95&	45.93&	45.90\\
\rowcolor{cyan!30}{BEVHeight} &  92.80&	 92.27&	 92.15&	67.24& 67.08& 67.00\\
\bottomrule
\end{tabularx}
\label{v2x_sim_rebuttal}
\vspace{-0.3cm}
\end{table}


\subsection{Effectiveness on multi depth-based Detectors} We extend our modules on BEVDepth\cite{li2022bevdepth} and BEVDet~\cite{huang2021bevdet} on
 DAIR-V2X-I\cite{yu2022dair} and present the results here. Replacing the depth-based projection in BEVDepth\cite{li2022bevdepth}, our method achieves
a performance increase of 2.19\%, 5.87\%, 4.61\% on vehicle, pedestrian and cyclist. Similarly, our approach surpasses
the origin BEVDet by 8.56\%, 5.35\%, 8.60\% respectively.
% \begin{table*}[ht]
%  \centering\addtolength{\tabcolsep}{-0.6pt}
%  \resizebox{0.8\textwidth}{!}{
%  \begin{tabularx}{1.0\textwidth}{l|c|ccc|ccc|ccc}
%   \toprule
%  \multirow{3}{*}{Method} &  
%  \multirow{3}{*}{Modality}  
%  & \multicolumn{3}{c|}{$\text{Vehicle}_{(IoU=0.5)}$} & \multicolumn{3}{c|}{$\text{Pedestrian}_{(IoU=0.25)}$} & \multicolumn{3}{c}{$\text{Cyclist}_{(IoU=0.25)}$} \\
%     \cmidrule(r){3-11}
%      &  & Easy & Mid & Hard & Easy & Mid & Hard & Easy & Mid & Hard  \\
% \midrule

% PointPillars~\cite{lang2019pointpillars} & PointCloud &63.07 & 54.00 & 54.01 & 38.53 & 37.20 & 37.28 & 38.46 & 22.60 & 22.49 \\
% SECOND~\cite{yan2018second} & PointCloud &71.47 & 53.99 & 54.00 & 55.16 & 52.49 & 52.52 & 54.68 & 31.05 & 31.19 \\
% MVXNet~\cite{Sindagi2019MVX} & Image+PointCloud &71.04 & 53.71 & 53.76 & 55.83 & 54.45 & 54.40 & 54.05 & 30.79 & 31.06 \\
% \midrule
% ImvoxelNet~\cite{rukhovich2022imvoxelnet} &Image & 44.78 & 37.58 & 37.55 & 6.81 & 6.746 & 6.73 & 21.06 & 13.57 & 13.17 \\
% BEVFormer-R101$\ast$~\cite{li2022bevformer} & Image 	&	61.37&	50.73&	50.73&	16.89&	15.82&	15.95	&22.16&	22.13&	22.06\\
% BEVDepth-R101$\ast$~\cite{li2022bevdepth}&	Image 	&	76.01&	64.11&	64.18&	24.32&	24.96&	24.84	&46.45&	45.56&	45.69	\\

% \midrule
% BEVHeight-R101(Ours) & Image &	79.12&	67.95&	67.04&	29.85&	29.31&	29.07	&51.55&	51.39&	50.91\\
%     \bottomrule
%   \end{tabularx}
%   }
%   \caption{\textbf{Comparison on the DAIR-V2X-I val set.}}
%   \label{dair_sota}
% \end{table*}


\begin{table}[h!t]
 \scriptsize\centering\addtolength{\tabcolsep}{-4.9pt}
\caption{\textbf{Ablation studies on different depth-based methods.} Here, we conduct the evaluation on DAIR-V2X-I val set, and report the results of three types of objects, vehicle~(veh.), pedestrian~(ped.) and cyclist~(cyc.).}
 \begin{tabularx}{1.0\linewidth}{l|c|ccc|ccc|ccc}
 \toprule
 \multirow{3}{*}{Method} &
 \multirow{3}{*}{VT} &
\multicolumn{3}{c|}{$\text{Veh.}_{(IoU=0.5)}$} & \multicolumn{3}{c|}{$\text{Ped.}_{(IoU=0.25)}$} & \multicolumn{3}{c}{$\text{Cyc.}_{(IoU=0.25)}$} \\
 \cmidrule(r){3-11}
   &  & Easy & Mod. & Hard & Easy & Mod. & Hard & Easy & Mod. & Hard  \\
 \midrule
 
 \multirow{2}{*}{BEVDepth\cite{li2022bevdepth}} & D &	75.50&	63.58&	63.67&	34.95&	33.42&	33.27& 55.67&	55.47&	55.34 \\
  & {\cellcolor{cyan!30} H}&	{\cellcolor{cyan!30} 77.78}&	{\cellcolor{cyan!30} 65.77}&	{\cellcolor{cyan!30} 65.85}& {\cellcolor{cyan!30} 41.22}&	{\cellcolor{cyan!30} 39.29}&	{\cellcolor{cyan!30} 39.46}&  {\cellcolor{cyan!30} 60.23}&  {\cellcolor{cyan!30} 60.08}& {\cellcolor{cyan!30} 60.54} \\
\midrule
 \multirow{2}{*}{BEVDet~\cite{huang2021bevdet}} & D &  59.59& 	51.92&	51.81&  12.61& 12.43& 12.37& 34.91& 34.32& 34.21 	\\
& {\cellcolor{cyan!30} H}&	{\cellcolor{cyan!30} 69.42}&	{\cellcolor{cyan!30} 60.48}&	{\cellcolor{cyan!30} 59.68}& {\cellcolor{cyan!30} 18.11}&	{\cellcolor{cyan!30} 17.81}&	{\cellcolor{cyan!30} 17.74}&  {\cellcolor{cyan!30} 44.69}&  {\cellcolor{cyan!30} 42.92}& {\cellcolor{cyan!30} 42.34} \\
\bottomrule
\multicolumn{11}{l}{\scriptsize{VT denotes view transformation, D,H represents depth-based and height-based ones.}}
\end{tabularx}
\label{dair_rebuttal}
\vspace{-0.38cm}
\end{table}






% \subsection{More Results}
% \subsubsection{Results on Rope3D Dataset}
\subsection{More Results on DAIR-V2X-I Dataset}
Tab.~\ref{dair} shows the experimental results of deploying our proposed approach on the DAIR-V2X-I\cite{yu2022dair} val set. Under the same configurations (e.g., backbone and BEV resolution), our model outperforms the BEVDepth\cite{li2022bevdepth} baselines by a large marge, which demonstrates the admirable performance of our approach.

% \input{BEV-Height/latex/table/rope3d_het}

% 不同分辨率的泛化性
% 非重要内容,可放在补充材料中(分辨率非主要创新)
\begin{table*}[h!t]
 \small 
 \centering
 \addtolength{\tabcolsep}{1.4pt}
 \caption{\textbf{Comparison on the DAIR-V2X-I Detection Benchmark.} Here, we report the results of three types of objects: Vehicle, Pedestrian and Cyclist. Each object is categorized into three settings according to the difficulty defined in ~\cite{yu2022dair}. Our BEVHeight manages to surpass the BEVDepth baseeline over a margin of 2\% - 6\% under the same configurations.}
 \begin{tabularx}{1.0\textwidth}{l|cc|ccc|ccc|ccc}
  \toprule
 \multirow{4}{*}{Method} & \multicolumn{2}{c|}{\multirow{2.8}{*}{Scale of Detector}} & \multicolumn{9}{c}{AP3D}\\
    \cmidrule(r){4-12}
    & &  & \multicolumn{3}{c|}{$\text{Vehicle}_{(IoU=0.5)}$} & \multicolumn{3}{c|}{$\text{Pedestrian}_{(IoU=0.25)}$} & \multicolumn{3}{c}{$\text{Cyclist}_{(IoU=0.25)}$} \\
   \cmidrule(r){2-12}
   & Backbone & BEV & Easy & Middle & Hard & Easy & Middle & Hard & Easy & Middle & Hard\\  
    \midrule
    %  \midrule
     BEVDepth\cite{li2022bevdepth} &	R50&	128x128	&	73.05&	61.32&	61.19&	22.10	&21.57	&21.11	&42.85&	42.26&	42.09\\
    BEVDepth\cite{li2022bevdepth} &	R101&	128x128	&	74.81&	62.44&	62.31&	24.49	&23.33&	23.17&	44.93&	44.02&	43.84\\
    BEVDepth\cite{li2022bevdepth} &	R101&	256x256	&	75.50&	63.58&	63.67&	34.95&	33.42&	33.27& 55.67&	55.47&	55.34\\
    \midrule
    BEVHeight&	R50&	128x128	&	76.61	&64.71	&64.76&	27.34&	26.09&	26.33	&49.68&	48.84&	48.58\\
    BEVHeight&	R101&	128x128	&	76.93&	64.97&	65.03&	28.53&	27.15&	27.48& 51.39&	50.83	&50.44\\
    BEVHeight& R101&	256x256	&	\textbf{77.78}&	\textbf{65.77}&	\textbf{65.85}&	\textbf{41.22}&	\textbf{39.29}&	\textbf{39.46}	&\textbf{60.23}&	\textbf{60.08}&	\textbf{60.54}\\
  \bottomrule 
 % \multicolumn{21}{l}{Mid: Middle, Veh.: Vehicle, Ped.: Pedestrian, Cyc.: Cyclist.}
  \end{tabularx}
  \label{dair}
\end{table*}





\begin{figure*}[t!]
	\centering
	\includegraphics[width=\textwidth]{BEV-Height/figures/visualization_supp-1.pdf}
	\caption{\textbf{Visualization Results of BEVDepth and our proposed BEVHeight under the extrinsic disturbance.} We use boxes in \textbf{{\color{red}red}} to represent false positives,  \textbf{{\color{green}green}} boxes for truth positives, and \textbf{{\color{black}black}} for the ground truth. The truth positives are defined as the predictions with IoU\textgreater 0.5 for vehicle and IoU\textgreater 0.25 for pedestrian and cyclist. I/II-(a) Clean means the original image without any processing; I/II-(b) Disturbed Roll denotes camera rotate 1 degree along roll direction; I/II-(c) Disturbed Roll and Pitch represents camera rotate 1 degree along roll and pitch directions simultaneously. We use \textbf{{\color{blue}blue}} dashed ovals to highlight the pronounced improvements in predictions.}
\label{fig:visualization_supp_1}
\vspace{-0.1cm}
\end{figure*}

\begin{figure*}[t]
	\centering
	\includegraphics[width=\textwidth]{BEV-Height/figures/visualization_supp-2.pdf}
	\caption{\textbf{Visualization Results of BEVDepth and our proposed BEVHeight under the extrinsic disturbance in another scene.}}
\label{fig:visualization_supp_2}
\vspace{-0.1cm}
\end{figure*}


\subsection{More Visualizations}
In Fig.~\ref{fig:visualization_supp_1} and Fig.~\ref{fig:visualization_supp_2}, we show more visualization results on the DAIR-V2X-I \cite{yu2022dair} dataset. As can be seen from the samples in I/II-(a) clean, our BEVHeight manage to detect objects in middle and long-distances. As for the  extrinsic disturbance cases in  I/II-(b) and I/II-(c),  our method can still guarantee the detection accuracy in terms of cars, pedestrian and cyclist. It can be concluded that our method can significantly improve the accuracy in middle and long-distances and the robustness to extrinsic disturbance.


% \Tao{Can we have some prediction results of height / depth?}
% \Lei{See Sec. ~\ref{sec:distance_error_analysis}, in process}

\end{appendices}

% \clearpage
% \onecolumn
\section*{Appendix}

We organize our supplementary material as follows. 
\begin{itemize}
    \item In Section~\ref{subsec:supp_contenders}, we introduce more details about the contenders.
    \item In Section~\ref{subsec:supp_pattern1}, we show more sensitivity patterns for ViT-B/16 with various pre-training strategies.
    \item In Section~\ref{subsec:supp_visual}, we show some dataset samples from \imagenet~\cite{krizhevsky2012imagenet} and \vtab{}~\cite{zhai2019vtab}.
    \item In Tables~\ref{tab:full_fgvc} and~\ref{tab:full_vtab}, we show per-task results for our SPT variants on FGVC and \vtab{} benchmarks, respectively.
    
\end{itemize}

\section{More Details of Contenders} 
\label{subsec:supp_contenders}

\begin{itemize}[leftmargin=2em]{

\item \fullft{}: fully tunes all the backbone and classification head parameters.
\vspace{-0.75em}
\item\linear{}: freezes all the backbone parameters and only tunes a linear classification head.
\vspace{-0.75em}
\item\bias{}~\cite{zaken2022bitfit}: freezes all the backbone parameters except for the bias terms and also tunes the linear classification head.
\vspace{-0.75em}
\item\partialft{}-$k$: freezes all the backbone parameters except for the last $k$ layers and also tunes the linear classification head as described in~\cite{jia2022vpt}.
\vspace{-0.75em}
\item \mlp{}-$k$: freezes all the backbone parameters and tunes the classification head which is implemented by a trainable $k$-layer multi-layer perceptron as described in~\cite{jia2022vpt}.
\vspace{-0.75em}
\item \shallowprompt{}~\cite{jia2022vpt}: freezes all the backbone parameters while introducing additional trainable prompts to the input space of the pretrained ViT.
\vspace{-0.75em}
\item \deepprompt{}~\cite{jia2022vpt}: freezes all the backbone parameters while appending additional trainable prompts to the sequence in the multi-head self-attention layer of each ViT block.
\vspace{-0.75em}
\item\adapter{}-$k$~\cite{houlsby2019parameter}: freezes all the backbone parameters while adding a down projection, a ReLU~\cite{hendrycks2016gaussian} non-linearity, and an up projection layer sequentially in the feed-forward network (FFN) of each visual Transformer block. 
We follow the training details of~\cite{zhang2022neural} to achieve better performance.
\vspace{-0.75em}
\item \lora{}-$k$~\cite{hu2022lora}: freezes all the backbone parameters while adding a concurrent branch including two low-rank matrices to the weight matrices in the multi-head self-attention layers to approximate efficiently updating them. 
The low-rank matrices can be merged into the backbone weights after fine-tuning. We follow the training details of~\cite{zhang2022neural} to achieve better performance.
\vspace{-0.75em}
\item \adaptformer{}~\cite{chen2022adaptformer}: freezes all the backbone parameters while adding a concurrent branch including a down projection, a ReLU~\cite{agarap2018deep} non-linearity, an up projection layer, and a pre-defined scaling factor to the FFN layer of each ViT block.
\vspace{-0.75em}
\item \noah{}~\cite{zhang2022neural}: searches for an optimal configuration with a once-for-all~\cite{cai2019once} network that includes trainable prompts, adapter modules, and LoRA modules, which requires a longer training schedule than the other VPET methods.
}
\end{itemize}

\section{More Parameter Sensitivity Patterns}
\label{subsec:supp_pattern1}
\rev{We show more parameter sensitivity patterns for ViT-B/16 with various pre-training strategies (i.e., MAE~\cite{he2022masked} and MoCo V3~\cite{chen2021empirical}) and datasets sampled from FGVC benchmark~\cite{jia2022vpt}. We visualize the proportions of the sensitive parameters under 0.4M trainable parameter budget. Visualizations of sampled VTAB-1k datasets with MAE and MoCo V3 pre-trained ViT-B/16 are shown in Figures~\ref{fig:sensitive_sup},~\ref{fig:sensitive_mae},~\ref{fig:sensitive_moco}. Visualizations of sampled FGVC datasets with supervised pre-trained ViT-B/16 are shown in Figure~\ref{fig:sens_fgvc}. We find our observations in the main paper are general: the proportions of the sensitive parameter exhibit: 1) dataset-specific varying patterns in terms of network depth; and 2) dataset-agnostic similar patterns in terms of operations. We empirically find} that the self-supervised pre-trained backbones have higher sensitivity variances than the supervised pre-trained one across the 19 downstream tasks. In particular, the variance of ViT-B/16 pre-trained with MAE~\cite{he2022masked} is twice as large as that of the supervised pre-trained ViT-B/16. We speculate that our SPT variants can better handle the large variances for self-supervised pre-trained backbones (Table 2 of the main paper) by identifying task-specific positions to introduce the trainable parameters.

\begin{figure}[htb]
\begin{center}
    \includegraphics[width=\linewidth]{sensitive_sup.pdf}
\end{center}
\caption{The distribution of sensitive parameters by blocks under 0.4M trainable parameter budget with supervised pre-trained ViT-B/16 backbone. We sample six tasks from VTAB-1k~\cite{zhai2019vtab}.
}
\label{fig:sensitive_sup}
\end{figure}

\begin{figure}[htb]
\begin{center}
    \includegraphics[width=\linewidth]{sensitive_mae.pdf}
\end{center}
\caption{The distribution of sensitive parameters by blocks under 0.4M trainable parameter budget with \mae{}~\cite{he2022masked} pre-trained ViT-B/16 backbone. We sample six tasks from VTAB-1k~\cite{zhai2019vtab}.}
\label{fig:sensitive_mae}
\end{figure}

\begin{figure}[tb]
\begin{center}
    \includegraphics[width=\linewidth]{sensitive_moco.pdf}
\end{center}
\caption{The distribution of sensitive parameters by blocks under 0.4M trainable parameter budget for \moco{}~\cite{chen2021empirical} pre-trained ViT-B/16 backbone. We sample six tasks from VTAB-1k~\cite{zhai2019vtab}.}
\label{fig:sensitive_moco}
\end{figure}

\begin{figure}[tb]
\begin{center}
    \includegraphics[width=0.8\linewidth]{rebuttal_fgvc_sensitivity.pdf}
\end{center}
\caption{Sensitivity patterns under 0.4M trainable parameters for Oxford Flowers~\cite{nilsback2008automated}, Stanford Cars~\cite{gebru2017cars}, and Stanford Dogs~\cite{Khosla_FGVC2011dogs}. We show the proportions of the sensitive
parameters for the query $\mW_{q}$, key $\mW_{k}$, value $\mW_{v}$, and $\mW_{o}$ weight matrices in the multi-head self-attention layer and two weight matrices $\mW_{fc1}$ and $\mW_{fc2}$ in the feed-forward network. 
}
\label{fig:sens_fgvc}
\end{figure}

\begin{figure}[htb]
\begin{center}
    \includegraphics[width=0.6\linewidth]{variance.pdf}
\end{center}
\caption{Comparisons of sensitivity variances across backbones with different pre-training strategies on \vtab{}.}
\label{fig:variance}
\end{figure}

\begin{figure}[htb]
\begin{center}
\includegraphics[width=0.6\linewidth]{natural_structured.pdf}
\end{center}
\caption{Dataset samples from \imagenet~\cite{krizhevsky2012imagenet} and \vtab{}~\cite{zhai2019vtab}. Samples from Natural tasks of \vtab{} ((a), (b), and (c)) are relatively more similar to the source \imagenet{} samples compared to the ones from Structured tasks of \vtab{} ((d), (e), and (f)).}
\label{fig:domain}
\end{figure}

\section{Dataset Samples for the Source and Target Domains}
\label{subsec:supp_visual}
We visualize some sampled images from the source domain (\imagenet~\cite{krizhevsky2012imagenet}) and the target domains (\vtab{}~\cite{zhai2019vtab}) in Figure~\ref{fig:domain}. We observe that the images from the Natural tasks of \vtab{} are relatively more similar to the source domain compared to those from the Structured tasks of \vtab{}, which aligns with our observation that Structured tasks have large domain gaps. As structured tuning
improves the performance of Structured datasets (Section 4.3 of the main paper), we speculate that
structured tuning facilitates mitigating such large domain gaps.

\begin{table}[t]

\scriptsize
\resizebox{\textwidth}{!}{%
    \begin{tabular}{lc| cccccc}
    \toprule
      &  Tuned / Total &\bf{\cub{}} 
  &\bf{\nabirds{}}
  &\bf{\flowers{}} &\bf{\dogs{}} &\bf{\cars{}}
  &\bf{Mean Acc.} \\
    \midrule
    \band \fullft{} & 100\% &87.3 &82.7 &98.8 &89.4 &84.5 &88.5\\
    \midrule
        \multicolumn{8}{c}{\bf{Addition-based methods}}\\
    \midrule
    \mlp{}-3 & 1.50\% &85.1 &77.3 &97.9 &84.9 &53.8 &79.8
    \\
    \shallowprompt{} & 0.31\% & 86.7 &78.8 &98.4 &\underline{90.7} &68.7 &84.6\\
    \deepprompt{} & 0.98\% & \underline{88.5} &\underline{84.2} &\underline{99.0} &90.2 &83.6 &89.1\\
    \adapter{}-8 & 0.39\% & 87.3 &\textbf{84.3} &98.4 &88.8 &68.4 &85.5\\
    \adapter{}-32 & 0.95\% & 87.2 &\textbf{84.3} &98.5 &89.6 &68.4 &85.6\\
    \adaptformer{} & 0.44\% & 84.7 &75.2 &97.9 &84.7 &83.1 &85.1\\
    \SPTa{} & 0.41\% & \textbf{89.1} &83.3 &\textbf{99.2} &90.5 &\underline{85.6} &\underline{89.5}\\
    \SPTa{} & 0.47\% & \textbf{89.1} &83.3 &\textbf{99.2} &\textbf{91.1} &\textbf{86.2} &\textbf{89.8}\\
    \midrule
     \multicolumn{8}{c}{\bf{Reparameterization-based methods}}\\
    \midrule
    \linear{} & 0.12\% & 85.3 &75.9 &97.9 &86.2 &51.3 &79.3\\
    \partialft{}-1 & 8.38\% &85.6 &77.8 &98.2 &85.5 &66.2 &82.6\\
    \bias{} & 0.13\% &\underline{88.4} &\textbf{84.2} &98.8 &\underline{91.2} &79.4 &88.4\\
    \lora{}-8 & 0.55\% &84.9 &79.0 &98.1 &88.1 &79.8 &86.0 \\
    \lora{}-16 & 0.90\% &85.6 &79.8 &98.9 &87.6 &72.0 &84.8 \\
    \SPTl{} & 0.41\% &\textbf{88.6} &82.8 &\underline{99.4} &\textbf{91.4} &\underline{84.5} &\underline{89.3} \\
    \SPTl{} & 0.60\% &\textbf{88.6} &\underline{83.4} &\textbf{99.5} &\textbf{91.4} &\textbf{87.3} &\textbf{90.1} \\
\bottomrule
    \end{tabular}}
    \caption{
    Per-task results on the FGVC benchmark from Table~1 of the main paper. ``Tuned / Total'' denotes the fraction of the trainable parameters. Top-1 accuracy (\%) is reported. The best result is in \textbf{bold}, and the second-best result is \underline{underlined}.
}\label{tab:full_fgvc}
\end{table}


\begin{sidewaystable}[t]
\scriptsize
\resizebox{\textwidth}{!}{%
    \begin{tabular}{lc | cccccccc | ccccc | ccccccccc}
    \toprule
    & & \multicolumn{8}{c|}{\textbf{Natural}} & \multicolumn{5}{c|}{\textbf{Specialized}} & \multicolumn{9}{c}{\textbf{Structured}} \\
    & \rotatebox{90}{Tuned / Total} & \rotatebox{90}{\bf{Cifar100}} & \rotatebox{90}{\bf{Caltech101}} & \rotatebox{90}{\bf{DTD}} & \rotatebox{90}{\bf{Flower102}} & \rotatebox{90}{\bf{Pets}} & \rotatebox{90}{\bf{SVHN}}  & \rotatebox{90}{\bf{Sun397}} & \rotatebox{90}{\bf{Mean Acc.}} & \rotatebox{90}{\bf{Camelyon}}  & \rotatebox{90}{\bf{EuroSAT}}   & \rotatebox{90}{\bf{Resisc45}}  & \rotatebox{90}{\bf{Retinopathy}} & \rotatebox{90}{\bf{Mean Acc.}} & \rotatebox{90}{\bf{Clevr-Count}} & \rotatebox{90}{\bf{Clevr-Dist}}  & \rotatebox{90}{\bf{DMLab}} & \rotatebox{90}{\bf{KITTI-Dist}}  & \rotatebox{90}{\bf{dSpr-Loc}} & \rotatebox{90}{\bf{dSpr-Ori}}   & \rotatebox{90}{\bf{sNORB-Azim}}  & \rotatebox{90}{\bf{sNORB-Ele}} & \rotatebox{90}{\bf{Mean Acc.}}   \\
    \midrule
\band \fullft{} & 100\% &68.9 &87.7 &64.3 &97.2 &86.9 &87.4 &38.8 &75.9 &79.7 &95.7 &84.2 &73.9 &83.4 &56.3 &58.6 &41.7 &65.5 &57.5 &46.7 &25.7 &29.1 &47.6

    \\\midrule
     \multicolumn{22}{c}{\bf{Addition-based methods}}
    \\\midrule
    \mlp{}-3 & 1.50\% &63.8 &84.7 &62.3 &97.4 &84.7 &32.5 &49.2 &67.8 &77.0 &88.0 &70.2 &56.1 &72.8 &47.8 &32.8 &32.3 &58.1 &12.9 &21.2 &15.2 &24.8 &30.6\\
    \shallowprompt{} & 0.31\%  & 77.7 &86.9 &62.6 &97.5 &87.3 &74.5 &51.2 &76.8 &78.2 &92.0 &75.6 &72.9 &79.7 &50.5 &58.6 &40.5 &67.1 &68.7 &36.1 &20.2 &34.1 &47.0\\
    \deepprompt{} & 0.98\% &78.8 &90.8 &65.8 &98.0 &88.3 &78.1 &49.6 &78.5 &81.8 &96.1 &83.4 &68.4 &82.4 &68.5 &60.0 &46.5 &72.8 &73.6 &47.9 &32.9 &37.8 &55.0\\
    \adapter{}-8  & 0.39\% & 69.2 & 90.1 & 68.0 & 98.8 & 89.9 & 82.8 & 54.3 & 79.0 & 84.0 & 94.9 & 81.9 & 75.5 & 84.1 & 80.9 & 65.3 & 48.6 & 78.3 & 74.8 & 48.5 & 29.9 & 41.6 & 58.5\\
    \adapter{}-32 & 0.71\% & 68.7 & 92.2 & 69.8 &98.9 & 90.3& 84.2& 53.0& 79.6& 83.2& 95.4& 83.2& 74.3 & 84.0 & 81.9 & 63.9& 48.7 & 80.6& 76.2& 47.6& 30.8& 36.4 & 58.3 \\
    \noah{} & 0.50\% & 69.6 & 92.7 & 70.2 & 99.1 & 90.4 & 86.1 & 53.7 & 80.2 & 84.4 & 95.4 & 83.9 & 75.8 & 84.9 & 82.8 & 68.9 & 49.9 & 81.7 & 81.8 & 48.3 & 32.8 & 44.2 & 61.3\\
    \SPTa{} & 0.30\% & 72.9 & 93.2 & 72.5 & 99.3 & 91.4 & 84.6 & 55.2 & 81.3 & 85.3 & 96.0 & 84.3 & 75.5 & 85.3 & 82.2 & 68.0 & 49.3 & 80.0 & 82.4 & 51.9 & 31.7 & 41.2 & 60.8\\
    \SPTa{} & 0.44\% & 72.9 & 93.2 & 72.5 & 99.3 & 91.4 & 88.8 & 55.8 & 82.0 & 86.2 & 96.1 & 85.5 & 75.5 & 85.8 & 83.0 & 68.0 & 51.9 & 81.2 & 82.4 & 51.9 & 31.7 & 41.2 & 61.4\\
    \midrule
     \multicolumn{22}{c}{\bf{Reparameterization-based methods}}
    \\\midrule
    \linear{} & 0.12\% & 63.4 &85.0 &63.2 &97.0 &86.3 &36.6 &51.0 &68.9 &78.5 &87.5 &68.6 &74.0 &77.2 &34.3 &30.6 &33.2 &55.4 &12.5 &20.0 &9.6 &19.2 &26.8\\
    \partialft{}-1 & 8.38\% &66.8 &85.9 &62.5 &97.3 &85.5 &37.6 &50.6 &69.4 &78.6 &89.8 &72.5 &73.3 &78.5 &41.5 &34.3 &33.9 &61.0 &31.3 &32.8 &16.3 &22.4 &34.2\\
    \bias{} & 0.13\% &72.8 &87.0 &59.2 &97.5 &85.3 &59.9 &51.4 &73.3 &78.7 &91.6 &72.9 &69.8 &78.3 &61.5 &55.6 &32.4 &55.9 &66.6 &40.0 &15.7 &25.1 &44.1\\
\lora{}-8 & 0.55\% & 67.1 & 91.4 & 69.4 & 98.8 & 90.4 & 85.3 & 54.0 &79.5 & 84.9 & 95.3 & 84.4 & 73.6 & 84.6 & 82.9 & 69.2 & 49.8 & 78.5 & 75.7 & 47.1 & 31.0 & 44.0 & 60.5 \\
\lora{}-16 & 0.90\% & 68.1 & 91.4 & 69.8 & 99.0 & 90.5 & 86.4 & 53.1 &79.8 & 85.1 & 95.8 & 84.7 & 74.2 & 84.9 & 83.0 & 66.9 & 50.4 & 81.4 & 80.2 & 46.6 & 32.2 & 41.1 & 60.2 \\
\SPTl{} & 0.31\% & 72.3 & 93.0 & 72.5 & 99.3 & 91.5 & 86.2  & 55.5 & 81.5 & 85.0 & 96.2 & 85.1 & 75.9 & 85.6 & 83.7 & 66.4 & 52.5 & 80.2 & 80.1 & 51.1 &  30.1 & 41.3 & 60.7 \\
\SPTl{} & 0.63\% & 73.5 & 93.3 & 72.5 & 99.3 & 91.5 & 87.9 & 55.5 & 81.9 & 85.7 & 96.2 & 85.9 & 75.9 & 85.9 & 84.4 & 67.6 & 52.5 & 82.0 & 81.0 & 51.1 &  30.2 & 41.3 & 61.3 \\
\bottomrule
    \end{tabular}}
    \caption{
    Per-task results on the \vtab{} benchmark from Table~1 of the main paper. ``Tuned / Total'' denotes the fraction of the trainable parameters. Top-1 accuracy (\%) is reported.
}\label{tab:full_vtab}
\end{sidewaystable}


\end{document}
