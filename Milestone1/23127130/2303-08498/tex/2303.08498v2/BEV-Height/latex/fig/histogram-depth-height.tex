%figure3
% \begin{figure}[t!]
% 	\centering
% 	\includegraphics[width=8.5cm]{BEV-Height/figures/depth-height1.pdf}
% 	\caption{
% 	\textbf{The comparison of predicting height and depth.} 
% % 	\textbf{The histogram distribution of depth and the height from the ground.} all data are derived from the annotations of DAIR-V2X-I\cite{yu2022dair} dataset. (a) reveals the depth histogram distribution. (b) indicates the height counterpart.
% % 和 Overall Architecture 图有点重复
% 	}
% \label{fig:histogram-depth-height}
% \end{figure}

\begin{figure*}[h!t]
	\centering
	\includegraphics[width=0.8\textwidth]{BEV-Height/figures/depth-height2.pdf}
	\caption{
	\textbf{The comparison of predicting height and depth.} 
	\textbf{(a)} We present the overview of previous depth based monocular 3D detection methods and our proposed \name{}. Note that we propose a novel 2D to 3D projection module. \textbf{(b)} We plot the histogram of per-pixel depth~(top) and ground-height~(bottom). We can clearly observe that the range of depth is over 200 meters while the height is within  5 meters, which makes height much easier to learn.
% 	\textbf{The histogram distribution of depth and the height from the ground.} all data are derived from the annotations of DAIR-V2X-I\cite{yu2022dair} dataset. (a) reveals the depth histogram distribution. (b) indicates the height counterpart.
% 和 Overall Architecture 图有点重复
	}
 \vspace{-0.1cm}
\label{fig:histogram-depth-height}
\end{figure*}
