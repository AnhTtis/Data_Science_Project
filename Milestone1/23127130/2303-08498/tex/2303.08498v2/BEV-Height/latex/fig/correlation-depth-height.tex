\begin{figure*}[ht]
\centering 
% 	\includegraphics[width=8.5cm]{BEV-Height/figures/depth_height_pixel_distribution.pdf}
\includegraphics[width=0.8\textwidth]{BEV-Height/figures/disturb.pdf}
% 	\caption{\textbf{The correlation between the image row coordinates of the object and the corresponding depth and height.} ``R\&P" implies roll and pitch. The ``disturbed R\&P" represents a rotation offset along roll and pitch directions in normal distribution (a) is the scatter diagram of the depth distribution. (b) is for the height from the ground.
	
% 	the position of the object in the image, which can be defined as $(u,v)$, and $v$ denotes its row coordinate of image
% 	}
\caption{\textbf{The correlation between the object's row coordinates on the image with its depth and height.} The position of the object in the image, which can be defined as $(u,v)$, and $v{\raisebox{0mm}{-}}coordinate$ denotes its row coordinate of the image. (a) A visual example of the noisy setting, adding a rotation offset along roll and pitch directions in the normal distribution. (b) is the scatter diagram of the depth distribution. (c) is for the height from the ground. We can find, compared with depth, the noisy setting of height has larger overlap with its original distribution, which demonstrates height estimation is more robust.
}
\vspace{-0.3cm}
\label{fig:five}
\end{figure*}
