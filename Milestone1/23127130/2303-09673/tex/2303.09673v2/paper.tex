\documentclass[aps,prx,twocolumn,floatfix,superscriptaddress,showpacs,amsmath,amssymb]{revtex4-2}
\usepackage{graphicx}
\usepackage{epsfig}
\usepackage{hyperref}
\usepackage{color,colordvi}
\usepackage{float}

%to have a separate bibliography for main and supplemental
%works directly on overleaf, while to compile locally  need run bibtex on two .aux files (usually not automatic in editors)  
% to remove it, remove below commands, \begin{bibunit}...\end{bibunit}, replace \putbib call with \bliography{bibfilename} and remove second \putbib call
\usepackage[sectionbib]{bibunits}
\defaultbibliographystyle{apsrev4-2} 
\defaultbibliography{dissMott}

%overleaf count tool: not useful, as doesn't include inline equations 
% \usepackage{verbatim}
% \newcommand{\detailtexcount}[1]{%
%   \immediate\write18{texcount -q #1.tex > #1.wcdetail }%
%   \verbatiminput{#1.wcdetail}%
% }


\newcommand{\ms}[1]{{\color{magenta}{#1}}}
\newcommand{\os}[1]{{\color{red}{#1}}}
\newcommand{\aash}[1]{{\color{green}{#1}}}




\newcommand{\eps}{\varepsilon} %nice
\newcommand{\aver}[1]{\langle {#1} \rangle}
\newcommand{\es}[1]{\begin{split}#1\end{split}}
\newcommand{\beq}{\begin{equation}}
\newcommand{\eeq}{\end{equation}}
\newcommand{\la}{\left\langle}
\newcommand{\ra}{\right\rangle}
\newcommand{\lp}{\left(}
\newcommand{\rp}{\right)}
\newcommand{\lsq}{\left[}
\newcommand{\rsq}{\right]}
\newcommand{\lbr}{\left\lbrace}
\newcommand{\rbr}{\right\rbrace}
\newcommand{\da}{\dagger}
\newcommand{\bma}{\begin{pmatrix}}
\newcommand{\ema}{\end{pmatrix}}
\newcommand{\tl}{\tilde}
\newcommand{\bra}[1]{\langle #1 |}
\newcommand{\ket}[1]{| #1 \rangle}
\newcommand{\mo}{{-1}}
\newcommand{\rw}{\rightarrow}
\newcommand{\oh}{\frac{1}{2}}
\newcommand{\w}{\omega}

\newcommand{\re}{\text{Re}}
\newcommand{\im}{\text{Im}}
\newcommand{\tr}{\text{tr}}
\newcommand{\abs}[1]{ \left\lvert #1	\right\rvert}
\newcommand{\mb}[1]{\mathbf{#1}}

%\usepackage{bbold}
\newcommand{\id}{\mathbb{1}}
\newcommand{\lw}{\linewidth}
\newcommand{\sign}{\text{sign}}
\newcommand{\hc}{\text{H.c.}}

\newcommand{\pt}{\partial _t}

\newcommand{\cmm}[1]{\color{blue}#1\color{black}}
\newcommand{\old}[1]{\textcolor{cyan}{#1}}
\newcommand{\new}[1]{\color{red}{#1}\color{black}}

\newcommand{\eff}{\text{eff}}


\begin{document}
\begin{bibunit}

\title{On the stability of dissipatively-prepared Mott insulators of photons}


\author{Orazio Scarlatella}
\affiliation{T.C.M. Group, Cavendish Laboratory, University of Cambridge, J.J. Thomson Avenue, Cambridge CB3 0HE, UK}
\affiliation{Clarendon Laboratory, University of Oxford, Parks
Road, Oxford OX1 3PU, UK}
\author{Aashish A. Clerk}
\affiliation{Pritzker School of Molecular Engineering, University of Chicago,
5640 South Ellis Avenue, Chicago, Illinois 60637, USA}
\author{Marco Schir\`o}
\affiliation{JEIP, UAR 3573 CNRS, Coll\`{e}ge de France, PSL Research University, 11 Place Marcelin Berthelot, 75321 Paris Cedex 05, France}
%\date{\today}
%\pacs{42.50.Ct,05.70.Ln}


\begin{abstract}
Reservoir engineering is a powerful approach for using controlled driven-dissipative dynamics to prepare target quantum states and phases.  
In this work, we study a paradigmatic model that can realize a Mott insulator of photons in its steady-state. We show that, while in some regimes its steady state approximates
a Mott-insulating ground state, this phase can become unstable through a {\it non-equilibrium} transition towards a coherent yet non-classical limit-cycle phase, driven by doublon excitations.  This instability is completely distinct from the ground-state Mott-insulator to superfluid transition.  
This difference has dramatic observable consequences and leads to an intrinsic fragility of the steady-state Mott phase: a fast pump compared to losses is required to sustain the phase, but also determines a small critical hopping. 
We identify unique features of the steady-state Mott phase and its instability, 
that distinguish them from their ground-state counterpart and can be measured in experiments.
\end{abstract}


\maketitle

% %TC:ignore
% \section*{Word Counts}
% \detailtexcount{paper}
% OS 16/08/23: need sum ``Words in text'' and ``words outside text'' (captions)
% %TC:endignore

Dissipation engineering offers a promising avenue for the control of quantum devices and simulators, and is actively being explored as a strategy for stabilizing entangled resource states and phases of matter~\cite{poyatosZoller1996,plenioKnight1999,diehl2008quantum,verstraete2009quantum,Harrington2022engineered}. 
While many physical platforms have been considered, 
superconducting circuit QED systems~\cite{schoelkopfGirvin2008,devoretSchoelkopf2013,blais2021circuit} are particularly ideal for engineering tailored reservoirs.  Reservoir engineering has been widely employed in these systems, including for quantum-state preparation \cite{dassonnevilleHuard2021,
leghtasDevoret2015,
andersenEichler2019,
luSchuster2017a,
kimchi-schwartzSiddiqi2016,
hollandSchoelkopf2015,
lescanne2020exponential} and autonomous quantum error correction \cite{puttermanNoh2022,
sivakDevoret2022,
berdouLeghtas2022,
grimmDevoret2020,
puriBlais2017,
kapitKapit2017,
touzardDevoret2018a}. 
These approaches could be combined with the ability to wire up superconducting circuits into arrays, providing new avenues for  photonic quantum simulators enjoying strong non-linearities and long lifetimes~\cite{houckKoch2012,carusottoSimon2020}. 


A recent breakthrough experiment \cite{maSchuster2019} succeeded in dissipatively preparing the first Mott insulator of photons, potentially enabling the realization of correlated quantum fluids of light~\cite{carusotto2013quantum}. 
In this experiment a Bose-Hubbard lattice was realised using an array of transmon qubits, which was then pumped by a reservoir, at a rate faster than lattice losses and that is structured in energy, providing an effective chemical potential for microwave photons~\cite{kapitSimon2014a,hafeziTaylor2015,lebreuillyCarusotto2017,maSimon2017}.
The experiment
demonstrated that the steady state of the dynamics was
a low-entropy incompressible state with integer filling approximating
a ground-state Mott insulator \cite{fisher89boson}.
Despite this achievement, many basic properties of this dissipatively-prepared Mott insulator remain unexplored even at the level of theory.  This includes its spectral properties and the nature of phase transitions out of the Mott state. 


In this Letter, we identify a new mechanism by which the steady-state Mott phase can become unstable. While the pumping stops autonomously once the Mott steady-state is reached at small values of hopping, thanks to its finite bandwidth and to the gapped nature of the phase \cite{maSchuster2019}, beyond a critical hopping an instability takes place, characterized by a sudden proliferation of doublon excitations, leading to the onset of a limit-cycle phase, destroying the Mott order. This latter phase is coherent, yet inherits non-classical features from the Mott phase. 

The non-equilibrium transition we unveiled is completely distinct from the ground-state Mott-superfluid one, and has a qualitatively different phase diagram. In particular, we find that a fast pump compared to losses, that is required to sustain the Mott steady-state, translates into a small critical hopping, determining a trade-off between fidelity and stability of this phase. 
Furthermore, we identify unique features of the steady-state Mott phase and its instability, such as amplification gain, a proliferation of doublon excitations and a diverging susceptibility at their energy, that distinguish them from their ground-state counterpart and could be measured in experiments. 


	
%We discuss that, while the steady-state density matrix might well approximate a ground-state Mott insulator, the excitations of such a phase retain non-equilibrium features, that eventually drive the instability. 




% \cmm{begin older}
% A recent breakthrough experiment \cite{maSchuster2019} succeeded in dissipatively stabilizing the first Mott insulator of photons, potentially enabling the realization of correlated quantum fluids of light~\cite{carusotto2013quantum}. In this experiment a Bose-Hubbard lattice was realised using an array of transmon qubits, which were then incoherently driven by a structured reservoir that provided an effective chemical potential for microwave photons~\cite{kapitSimon2014a,hafeziTaylor2015,lebreuillyCarusotto2017,maSimon2017}. The experiment demonstrated that the steady state of the dynamics was a low-entropy incompressible state with integer filling approximating a ground-state Mott insulator. Despite this achievement, many basic properties of this dissipatively-prepared Mott insulator remain unexplored even at the level of theory.  This includes its basic spectral properties and the nature of phase transitions out of the Mott state. 


    
% In this paper we show that a dissipatively-prepared Mott phase differs from its ground-state equilibrium counterpart in two key respects. First, although the static properties of the steady-state can closely approximate an ideal ground-state Mott insulator, this is not true for finite-frequency excitations:  these retain unique non-equilibrium features, and can drive a phase transition towards a coherent limit-cycle phase, rather than a static superfluid. Second, the different nature of the phase transition in the dissipative system leads to a dramatic reshaping of the  characteristic Mott to superfluid phase diagram~\cite{lebreuillyCarusotto2017,fisher89boson} and a strong suppression of the critical hopping  (as compared to the ground-state transition).  
% Remarkably, we find that the closer the Mott steady-state is to the Mott-insulating ground state, the more fragile it becomes against finite hopping. 
% Further, we show that the instability is driven by doublon excitations. Also we show that the coherent limit-cycle phase is imprinted with strong non-classical features inherited from the Mott phase.
% We substantiate these findings with a thorough comparison of the steady-state Mott instability with its ground-state counterpart. % and also
% % considering different models for the structured reservoir. 
% We finally identify distinct features of the dissipatively-prepared Mott state and of its instability 
% % their consequences on the driven-dissipative Mott instability 
% that can be tested in upcoming experiments with superconducting circuits.
% \cmm{end older}

Our work differs from previously identified instabilities of steady-state Mott phases. 
In \cite{biellaCiuti2017,caleffiCarusotto2022} a different limit-cycle instability is predicted, driven by hole excitations, forming due to an inefficient pumping scheme. This is similar to the ground-state transition of hard-core bosons which is driven solely by an incommensurate filling \cite{krauthTrivedi1991},\cite{schmidDorneich2002}, rather than by a competition between kinetic term and local repulsion, that instead characterizes the transition we identify here. 
% and different from the transition discussed here, driven by doublons and due to a competition between kinetic term and local repulsion.
% rather than by a competition between kinetic term and local repulsion as we consider here. 
% A pumping scheme similar to ours was considered instead in Ref.~\cite{lebreuillyCarusotto2017}, that nevertheless predicted a ground-state like Mott-superfluid transition, rather than the nonequilibrium instability discussed here.
A pumping scheme similar to our manuscript was considered instead in Ref.~\cite{lebreuillyCarusotto2017}, but this reference did not predict the non-equilibrium instability we find here, but rather one similar to the ground-state Mott-superfluid transition. 

%\new{We note that a different limit-cycle instability of a steady-state Mott phase was predicted in \cite{biellaCiuti2017,caleffiCarusotto2022}. There the driving mechanism is a proliferation and condensation of hole excitations, due to an inefficient pumping scheme.
% , when its bandwidth is smaller than the lattice one. 
% This determines an instability of the Mott phase, driven by a condensation of holes. 
%It is the analogue of the ground-state transition in the limit of hard-core bosons, which is solely driven by an incommensurate filling \cite{krauthTrivedi1991},\cite{schmidDorneich2002}, 
%rather than by a competition between kinetic term and local repulsion.

%In our work instead we consider a different pumping regime, such that the instability of \cite{biellaCiuti2017,caleffiCarusotto2022} never occurs.  This regime was previously considered in \cite{lebreuillyCarusotto2017}, that nevertheless predicted an instability similar to the ground-state Mott-superfluid transition, rather than the one discussed here. 

% Here instead, we predict a completely different instability of the Mott phase from both \cite{lebreuillyCarusotto2017} and \cite{biellaCiuti2017}, which is driven by doublon excitations and due to the large pump required to stabilize the phase in the first place. 
% which is the non-equilibrium analogue of the Mott-superfluid transition for soft-core bosons, driven by a competition between kinetic term and local repulsion.

We also note that limit-cycle instabilities of driven-dissipative bosons were also discussed elsewhere \cite{scarlatellaSchiro2019,scarlatellaSchiro2021,szymanskaLittlewood2006}, but these works did not consider steady-state Mott insulators.

% Note that previous works discussing limit-cycle instabilities in models of driven-dissipative bosons did not consider steady-state Mott insulators \cite{scarlatellaSchiro2019,scarlatellaSchiro2021,szymanskaLittlewood2006}. Other works focused instead on the Mott state in the limit of hard-core bosons \cite{biellaCiuti2017,caleffiCarusotto2022},
% %This proliferation also leads the coherent limit cycle phase to possess strongly non-classical features.    
% which is very different from the finite interaction regime we consider here (that is more relevant to experiments).  In particular, we find that the coherent limit-cycle phase is intimately connected to a proliferation of doublon excitations, something that could never occur in the hard-core limit, where the transition is driven by a commensurability effect.

\begin{figure*}[t]
\centering
\includegraphics[width=1\linewidth]{fig1_nnm1.png}   
\caption{(a) Schematic: a lossy Bose-Hubbard lattice is coupled to a structured reservoir, imposing an effective chemical potential $\mu_{\rm eff}$. 
(b-c) Gutzwiller mean-field (MF) dynamics of the order parameter: for a hopping smaller than a critical value $J<J_c$ (b) it features damped oscillations towards the incoherent ($\aver{a} = 0$) Mott steady-state, while for $J>J_c$ (c) it develops finite-amplitude oscillations corresponding to a coherent limit-cycle phase;
the initial state is a coherent state  $\ket{\alpha}$ with $\alpha=0.01$, and  $J_c/U = 0.094$. 
(d) The steady-state phase diagram in MF: the Mott phase appearing in ``lobes'' with $J<J_c$  with approximately integer filling $N$;
the ground-state Mott-superfluid phase diagram is plotted for comparison, as well as ``ground-state-like'' transition, assuming a static order parameter. 
(e) as the pump/loss ratio $r$ increases, $J_c$ decreases and the critical frequency $\w_c$ approaches the energy $\w_{\rm doub}$ to create a doublon excitation. $\kappa/U = 10^{-3}$ in (a,c) and $\kappa/U = 10^{-5}$ in (d,e). $\mu_{\rm eff}/U = 1/2$ and $r=100$ where relevant.  
}
\label{fig:phaseDiag}
\end{figure*}

\textit{The model --} We consider a lossy Bose-Hubbard lattice, incoherently pumped by a structured reservoir, providing a source of incoherent excitations within a finite energy window: this will generate an effective chemical potential for the lattice. The reservoir mimicks the essential features of experimental protocols such as in \cite{maSchuster2019}. 
%This represents a source of excitations, and not also a sink as in the case of equilibrium reservoirs, that have an energy smaller than a cutoff, acting as a chemical potential for the lattice and constituting a toy model for the \emph{stabilizer} protocol implemented in \cite{maSchuster2019}.
A minimal model for the system is given by the Lindblad equation
\beq
\partial_t \hat{\rho}=-i[\hat{H}, \hat{\rho}] + \kappa \sum_{j} \mathcal{D} [\hat{a}_{j}] \hat{\rho}  + r \kappa
\sum_{j,\omega} S (\w)  \mathcal{D} [\hat{A}^\da_{j}(\w) ] \hat{\rho}  
\label{eq:mbME}
\eeq
where $\mathcal{D}[\hat{O}] \hat{\rho}=(\hat{O} \hat{\rho} \hat{O}^{\dagger}-\{\hat{O}^{\dagger} \hat{O}, \hat{\rho}\} / 2)$ is a standard Lindblad dissipator.
$\hat{H}$ is the Bose-Hubbard Hamiltonian 
%in the frame rotating at the on-site frequency $\w_0$
\beq 
\label{eq:hamBh}
\hat{H}= \sum_i  \frac{U}{2}\hat{n}_i(\hat{n}_i-1)  -\frac{J}{z} \sum_{\langle ij\rangle}\left(\hat{a}^{\da}_i \hat{a}_j+\hc \right)
\eeq
where each lattice site hosts a single bosonic mode with annihilation operator $\hat{a}_i$ (and $\hat{n}_i = \hat{a}^\da_i \hat{a}_i$), with an oscillator frequency $\w_0$ that has been gauged away from the Hamiltonian by a rotating frame transformation, $U$ is the on-site interaction, $J$ is the hopping rate between nearest-neighbour sites and $z$ the lattice coordination.
%(number of nearest-neighbours of each site).


The first dissipator in Eq.~(\ref{eq:mbME}) describes linear, Markovian on-site losses at rate $\kappa$.
The second instead describes the coupling to the structured reservoir, with a pump rate $r \kappa$, such that $r$ is the ratio between pump and loss rates. The jump operators $\hat{A}_{j}^\da(\omega) = \sum_{\varepsilon^{\prime}-\varepsilon=\omega} \hat{\Pi}(\varepsilon^\prime)   \hat{a}_j^\da  \hat{\Pi}\left(\varepsilon\right)$ connect the manifolds of eigenstates of the Hamiltonian differing by one particle, and with energy difference $\varepsilon^{\prime}-\varepsilon=\omega$ (of order $\w_0$ assumed positive and large), where $\hat{\Pi}(\varepsilon)$ is the projector to the manifold with energy $\varepsilon$. $S(\omega)$ is the spectral function of the reservoir \cite{breuerPetruccione2007} for which we use the simple form
\beq
\label{eq:lesserBox}
S(\omega) = \theta(\mu_{\rm eff}-{\w}) %\theta( \w +\w_0)
\eeq 
with $\mu_{\rm eff}$ representing the maximum energy of reservoir excitations (in the rotating frame of the Hamiltonian \eqref{eq:hamBh}). 
%Note that to simplify the theoretical treatment we consider pumping reservoirs on each lattice site (as opposed to the single localized pump used in Ref.~\cite{maSchuster2019}).

% Note that while the master equation \eqref{eq:mbME} provides a minimal model of the structured reservoir, solving it exactly is extremely difficult, as it requires the knowledge of the spectrum of the many-body Hamiltonian \eqref{eq:hamBh}.   In this work, we circumvent this problem using controlled many-body techniques that only rely on evaluating the dissipator for a single-site problem. 

The master equation \eqref{eq:mbME} can lead to a steady-state that approximates a ground-state Mott insulator. 
In fact, the structured reservoir with spectral function \eqref{eq:lesserBox} 
approximates the detailed balance relation
characterizing a system in contact with an equilibrium reservoir with chemical potential  $\mu_{\rm eff}$ \cite{lebreuillyCarusotto2017} (see also \cite{supp}
% \footnote{See Supplemental Material% [URL will be inserted by publisher]}
): it therefore imposes an effective chemical potential.
% ; one also needs a weak system-reservoir coupling.  
The pump/loss ratio $r$ plays the role of an inverse temperature, meaning that the ground-state corresponds to $r \rightarrow \infty$. In practice, the larger $r$, the better the fidelity of the steady-state with the Mott ground-state; this was also as numerically verified by \cite{lebreuillyCarusotto2017}. 

% \cmm{v13.1}
% \textit{Dissipative preparation of a Mott insulator --} 
% \new{The master equation \eqref{eq:mbME} can lead to a steady-state that approximates a ground-state Mott insulator, as the structured reservoir with spectral function \eqref{eq:lesserBox} realizes a chemical potential for the photons. 
% More quantitatively, one can show that \eqref{eq:mbME} approximates the detailed balance relation characterizing a system in contact with an equilibrium reservoir with chemical potential $\mu_{\rm eff}$ (as shown in \cite{lebreuillyCarusotto2017} and as we also discuss in \cite{supp}), where the pump/loss ratio $r$ plays the role of an inverse temperature. 
% The ground-state therefore corresponds to $r \rightarrow \infty$. In practice, the larger $r$, the better the fidelity of the steady-state with the Mott ground-state (as \cite{lebreuillyCarusotto2017} also verified numerically). }

% \cmm{v11}
% Master equations with a structured reservoir similar to \eqref{eq:mbME} can lead to steady states that \new{closely approximate }a 
% Mott insulating ground-state.
% \new{A perfect agreement (fidelity) requires taking the limit of large pump/loss ratio $r \rw \infty $, but weakly coupled environment $\kappa, r \kappa \rw 0 $ with $U,J \sim O(1)$, in which \eqref{eq:mbME} satisfies an approximate zero-temperature detailed balance relation, from which equilibration to the ground-state Mott phase can be expected (as shown in \cite{lebreuillyCarusotto2017}, and as we also discuss in \cite{supp}).
% This was also verified numerically in \cite{lebreuillyCarusotto2017} for a small 1D chain.
% Importantly, for a small but finite amount of dissipation that we assume throughout the paper, the fidelity with a Mott insulator thus increases by increasing $r$.
% }


In the limit of disconnected sites $J=0$ one can also calculate the steady-state analytically \cite{supp} and check that it corresponds, for large $r$, to the ground state of the Hamiltonian with an equilibrium chemical potential $\mu = \mu_{\rm eff}$, namely a Fock state $\ket{N}$ with filling set by the chemical potential as 
\beq
\label{eq:neqPopCond}
N - 1 < \frac{\mu_{\rm eff}}{U} < N
\eeq
 % of the Bose-Hubbard Hamiltonian \eqref{eq:hamBh} with an equilibrium chemical potential $\mu = \mu_{\rm eff}$. 
% In Ref. \cite{lebreuillyCarusotto2017} such a Mott phase is shown to be stable up to the critical hopping for the ground-state Mott-superfluid transition. This work though makes the crucial assumption of assuming a time-independent steady-state, ruling out the possiblity of dynamical phases such as limit cycle phases taking over the Mott phase. 



% \textit{Dissipative preparation of a Mott insulator --} The master equation \eqref{eq:mbME}, in the regime considered throughout the paper of large pump/loss ratio $r \gg 1$ but weakly coupled environment $\kappa, r \kappa \ll U,J$, and for $U \ll J$, approximates the dynamics of equilibration with a
% bath at chemical potential $\mu=\mu_{\rm eff}$ and zero temperature $T=0$: as a consequence, its steady-state is expected to approximate the ground state of a grand-canonical Hamiltonian.

% The equilibrium dynamics is in fact characterized by the detailed balance relation, stating that given two eigenstates of the Hamiltonian $\ket{\psi}$ and $\ket{\phi}$, the ratio of the transition rates for going from one state to the other equals the ratio of their equilbrium probabilities, their Boltzmann weights, at temperature $T$ and chemical potential $\mu$: $ {\mathcal{T}^{\rm eq}_{\psi \rw \phi}}/{\mathcal{T}^{\rm eq}_{\phi \rw \psi}} = e^{- (\mu - \eps_\phi  + \eps_\psi)/T }$.
% The master equation \eqref{eq:mbME} defines the following transition rates between eigenstates differing by 1 particle: if $\ket{\psi}$ has 1 particle less than $\ket{\phi}$, then $ {\mathcal{T}_{\psi \rw \phi}}/{\mathcal{T}_{\phi \rw \psi}} = r \theta( \mu_{\rm eff} -\eps_\phi +\eps_\psi ) $, that for $r \gg 1$ approximates the detailed balance relation at zero temperature $T\rw 0$ and for $\mu = \mu_{\rm eff}$. A similar discussion is reported in \cite{lebreuillyCarusotto2017}. 
% We remark that this partial detailed balance relation does not guarantee an equilibrium steady state, as Eq. \eqref{eq:mbME} does not guarantee thermalization within each fixed particle-number subspace.  Further, in the presence of spectral degneracies, the master equation will couple populations and coherences.  Nonetheless, the expectation of an equilibrium state is expceted to hold if one is deep in the Mott phase,  as it becomes rigorous at zero hopping. 

% In the limit $J=0$ of disconnected sites, the steady-state can be calculated analytically and shown to correspond, for large $r$, to the ground state of a Bose-Hubbard-site with an equilibrium chemical potential $\mu = \mu_{\rm eff}$ (as we show in \footnote{See [URL will be inserted by publisher]}), namely a
% Fock state $\ket{N}$ with filling set by the chemical potential as 
% \beq
% \label{eq:neqPopCond}
% N -\oh < \frac{\mu_{\rm eff}}{U} < N +\oh 
% \eeq
% At finite hopping, Ref. \cite{lebreuillyCarusotto2017} studied a Redfield equation similar to \eqref{eq:mbME} showing explicitly that the steady-state of a small 1D chain approximates the ground-state of the Bose-Hubbard Hamiltonian \eqref{eq:hamBh} with an equilibrium chemical potential $\mu = \mu_{\rm eff}$. %, namely with Hamiltonian $\hat{H} - \mu \sum_i \hat{n}_i$.

%Eventually, deviations from this ground-state-like picture
%can still possibly arise in the thermodynamic limit of an infinite lattice, which is the case that we investigate in this paper. 


%\textit{Limit of independent sites. --}


%In this limit the steady-state can be calculated analytically and shown to correspond, in the limit $r\gg 1$ to a pure state given by the Bose-Hubbard site ground state with chemical potential $\mu = \mu_{\rm eff}$ (as we show in \footnote{See [URL will be inserted by publisher]}).

%expected ground-state, owing to the above argument about equilibration.
%the above argument about equilibration becomes rigorous, and 
%Note that a collection of independent sites is also representative of a Mott insulating phase in the Gutzwiller approximation. 

%%MS: 
%For the single-site problem, the jump operators become simply $ A(E_{n-1} - E_{n} < 0 ) =  \aver{ n  | a^\da| n-1} \ket{n} \bra{n-1} $ (omitting the site index), describing transitions between two number eigenstates, the single-site eigenstates, differing by one boson, with level spacing $E_n - E_{n-1} = -\mu_{\rm eff} + U \lp n-\oh \rp$. \new{The master equation \eqref{eq:mbME} reduces to a simple rate equation for number state populations while coherences vanish, therefore the above argument about equilibration becomes rigorous in this case.}

%The spectrum is completely non-degenerate, and there's at most a couple of eigenstates with a given transition energy $\omega$: for this reason the sum defining $A(\omega)$ in \eqref{eq:jumpMB} reduces to a single term. 

%One finds that only eigenstates with $E_n -E_{n-1} < 0$ are populated and that the populations are given by $p_n = r^n  ({1-r})/({1-r^{N+1}}) \theta(N-n) $ where $N$ is the last populated number state satisfying
%\beq
%\label{eq:neqPopCond}
%N -\oh < \frac{\mu_{\rm eff}}{U} < N +\oh 
%\eeq
%Eq. \eqref{eq:neqPopCond} is the same condition defining the ground state of a Bose-Hubbard site with chemical potential $\mu = \mu_{\rm eff}$ (as we show in \footnote{See [URL will be inserted by publisher]}).

%For $r\gg 1$ we find that the steady-state approaches a pure state, as $p_n \approx \delta_{n,N}$, that corresponds to the Bose-Hubbard site ground state, as expected.


%the number state with $N$ bosons, satisfying \eqref{eq:neqPopCond}. 
%We remark that \eqref{eq:neqPopCond} is the same condition defining the ground state of a Bose-Hubbard site with chemical potential $\mu = \mu_{\rm eff}$ (as we show in \cite{supp}), thus the single-site dissipative dynamics indeed thermalizes to that ground state for $r \gg 1$.

%\cmm{Crucially, a large pump rate compared to losses $r$ is required to stabilize the ground state of the Hamiltonian.}

%Beyond the Gutzwiller mean-field approximation considered here, Ref. \cite{lebreuillyCarusotto2017} showed for a small 1D chain that the steady-state of \eqref{eq:mbME} in a frame rotating at frequency $\w=\sigma$ approximates the wave-function of the Mott insulating ground-state of the Hamiltonian with high fidelity. 

%Following a similar argument for the many-body master equation \eqref{eq:mbME}, \cite{lebreuillyCarusotto2017} concluded that the steady-state in a frame rotating at frequency $\sigma$ and in the same regime would approximate the ground-state of a Bose-Hubbard Hamiltonian. In this reference in fact, the steady-state is computed numerically for a small 1D chain, approximating a Mott insulator with high fidelity.


%\begin{figure}
%\centering
%\includegraphics[width=0.48\linewidth]{./jcVsR.png}  
%\includegraphics[width=0.48\linewidth]{./wcVsR.png}  
%\caption{Critical hopping and frequency as a function of the pump-to-loss ratio. $J_c$ vanishes linearly, while $\omega_c$ approaches the energy to create a particle excitation $\w_{\rm doub}$. Parameters: $\mu_{\rm eff}/U = 1$, $k/U = 10^{-6}$, $\sigma/U=3.5$. }
%\label{fig:3}
%\end{figure}
\textit{Instability of the steady-state Mott phase -- } Although for weak $J$ the steady state well approximates a ground-state Mott insulator, the instability of this phase at a critical hopping turns out to be {\it qualitatively} different from the equilibrium case. 

A simple way to capture this instability is to solve the dynamics with a time-dependent Gutzwiller mean-field ansatz $\rho(t)= \prod_i \rho_i(t)$, that neglects both quantum and classical correlations between sites while capturing the local physics. 
This corresponds to solving a single-site problem with an effective Hamiltonian $H_{\rm MF} = U n(n-1)/2  +\phi^{\dagger}(t) a+\phi(t) a^{\dagger}$, where $\phi(t)=-J \langle a(t)\rangle$, where we dropped the site index assuming also a homogeneous-in-space state and in the thermodynamic limit of infinitely-many sites. 
Fig. \ref{fig:phaseDiag}(b) shows that below a critical hopping $J<J_c$ the order parameter, that is the average bosonic field $\aver{a(t)}$,  features damped oscillations and eventually vanishes in the steady-state Mott phase. 
At a critical hopping $J_c$ this develops limit-cycle oscillations at a critical frequency $\w_c$, $\aver{a(t \rw \infty)} = |a| e^{i \omega_c t + i \phi} $, that acquire a stationary and finite amplitude $|a|$ for $J>J_c$.  
This is shown in panel (c); note that for $J>J_c$ the frequency is renormalized with respect to that for $J=J_c$ \cite{scarlatellaSchiro2019}.
We anticipate that this second-order dissipative phase transition is a genuine non-equilibrium instability, distinct from the ground-state Mott-superfluid transition: here the onset of limit-cycles corresponds to a spontaneously broken time-translation symmetry, along with the U(1) symmetry of \eqref{eq:mbME} for $\hat{a}_i \rw \hat{a}_i e^{i\theta}$, something that cannot happen for ground-state transitions \cite{nozieresNozieres2013,brunoBruno2013b,
watanabeOshikawa2015a}. 

% \cmm{v11}
% The limit-cycle instability corresponds to a second-order dissipative phase transition, in which the steady-state breaks the $U(1)$ symmetry of the model, corresponding to the master equation \eqref{eq:mbME} being invariant under $\hat{a}_i \rw \hat{a}_i e^{i\theta}$. It also breaks the continuous time-translation symmetry of the steady-state of \eqref{eq:mbME} \cite{scarlatellaSchiro2019,scarlatellaSchiro2021}, something that cannot happen for ground-state transitions \cite{nozieresNozieres2013,brunoBruno2013b,
% watanabeOshikawa2015a}. 
% % , highlighting already that this instability is different from a ground-state Mott-superfluid transition.


The phase boundary can be found by the condition that the lattice susceptibility to an applied weak coherent field diverges at the instability. 
We compute this quantity using a strong-coupling Keldysh field theory approach (detailed in ~\cite{supp}). 
This in principle allows to describe the excitations of the Mott phase, but not of the superfluid phase \cite{senguptaDupuis2005b}, and to formulate a critical theory of the Mott-superfluid transition \cite{scarlatellaSchiro2019,fisher89boson}. It gives the same critical point as Gutzwiller mean-field \cite{supp}, that follows from
%A critical point equation consistent with the Gutzwiller dynamics can be found using using a strong-coupling RPA (random phase approximation) \cite{senguptaDupuis2005b}, that we derive in \cite{supp} in the Keldysh path-integral, or equivalently using linear-response theory within the Gutzwiller ansatz, yielding: 
%We solve the master equation \eqref{eq:mbME} for the steady-state using a strong-coupling RPA (random phase approximation) \cite{senguptaDupuis2005b}, that we derive in \cite{supp} with a Hubbard-Stratonovich transformation in the Keldysh path-integral.
%This approach reduces the many-body problem to computing the time-dependent correlation functions of the single-site problem. 
%It goes beyond a steady-state Gutzwiller mean-field approximation as it captures also the long-time dynamics. 
%For a critical hopping strength $J_c$ the $U(1)$ symmetry of the model, corresponding to the master equation \eqref{eq:mbME} being invariant under $\hat{a}_i \rw \hat{a}_i e^{i\theta}$, is spontaneously broken and the system develops an order parameter $\aver{a} \neq 0$, corresponding to a coherent phase taking over the dissipatively-stabilized Mott insulator. Strong-coupling 
%RPA yields the critical point equations  
\begin{align}
\label{eq:neqCritFreq}
 0 &= - \frac{1}{\pi} \im G_0^R(\omega_c) \\
 \label{eq:neqCritHop}
1/{J_{c}} &= - \re G_0^R(\omega_c)
\end{align}
where $G_0^R(\omega) = -i \int_{0}^\infty dt e^{i \w t } \aver{[ \hat{a}(t),\hat{a}^\da(0) ]}_0 $ is the Fourier transform of the local steady-state susceptibility, evaluated at $J=0$ corresponding to disconnected sites. Note that here homogeneity in space is a result of calculations, rather than an assumption.
%\os{The same equations can also be obtained from the Gutzwiller mean-field ansatz, as we show in \cite{supp}.}
% \os{In \cite{supp} we discuss that the same equations can also be obtained within Gutzwiller mean-field.} 


Eq.~(\ref{eq:neqCritFreq}) determines the Fourier mode that becomes unstable, identified by the critical frequency $\w_c$. It represents a zero net-damping condition: on any given lattice site, the rest of the lattice (viewed as a bath) does not produce any net gain or loss. 
As we discuss later, 
thermal equilibrium in the grand-canonical ensemble would require $\w_c =\mu_{\rm eff}$, corresponding to an unbroken time-translational symmetry. 
A violation of this condition can only occur out of equilibrium. 
% A limit-cycle instability corresponds to $\w_c\neq 0$.
% ; the righthand side of this equation corresponds to an effective density of states (DOS).  
Eq.~(\ref{eq:neqCritHop}) instead determines the critical hopping $J_c$ from the real part of the susceptibility evaluated at $\w_c$. 


% A limit-cycle instability is equivalent to spontaneous oscillations at a non-zero frequency in the lab frame. 
% It requires the system to exhibit a negative DOS at positive frequencies (typically just below $\omega_c$), something that is directly connected with the ability to generate amplification gain~\cite{scarlatellaSchiro2019a,scarlatellaSchiro2021}; \new{we will discuss this more when commenting Fig.~(\ref{fig:dmft})}.


The steady-state phase boundary is plotted in Fig.~\ref{fig:phaseDiag}(d), showing that the regimes of stable Mott phases form ``lobes'' in the hopping-chemical potential plane.  Each distinct lobe corresponds to a chemical potential range 
\eqref{eq:neqPopCond} where there is an approximate integer filling of the lattice.  One immediately sees that their shape is dramatically altered in the steady-state case, as compared to the standard ground-state Mott-superfluid transition (also plotted).  We discuss later how those difference arise, highlighting first the main consequences. 
% The steady-state phase boundary for the Mott instability is plotted in Fig.~\ref{fig:phaseDiag}~(d) and is very different from that of the ground-state Mott-superfluid transition, plotted for comparison.
% We leave the discussion of why this difference arise for later and we discuss now the salient features of the steady-state phase boundary. 
An expected result is that dissipation suppresses phase coherence in parts of the phase diagram, as we see small regions that would be superfluid in the ground state turned into Mott insulators in the steady state.
% This is perhaps not surprising, as one might expect that dissipation easily suppresses phase coherence. 
Conversely, it is remarkable that for many values of $\mu_{\rm eff}$ in the plot the Mott steady-state is {\it unstable} at a smaller critical hopping $J_c$.

An important results is that the critical hopping $J_c$ strongly depends on the pump/loss ratio $r$, and decreases when increasing $r$, as Fig.~\ref{fig:phaseDiag}(e) shows. 
This implies a trade off between the fidelity of the steady-state to a ground-state Mott insulator, which requires a fast pump compared to losses (i.e. a large $r$) to sustain an integer filling, as discussed earlier in the manuscript and verified numerically in \cite{lebreuillyCarusotto2017}, and the stability of this phase at finite hopping. 
% Eventually, this suggests that an arbitrarily high fidelity of the steady-state with a Mott insulator can only be reached close to the trivial limit of disconnected sites. 

Finally, Fig.~\ref{fig:phaseDiag}(e) also shows that the critical frequency \eqref{eq:neqCritFreq} 
approaches for large $r$ the energy (using $\hbar = 1$ units) to create a doublon excitation, namely to add one particle to the steady-state, $\w_c \sim \w_{\rm doub}$, that in a single-site picture is
\beq
\label{eq:doubEn}
\w_{\rm doub}  =  U  N  
\eeq 
where $N$ is the filling of the Mott state.


To give more insights on the instability, in Fig.~\ref{fig:dmft}a we plot the susceptibility $G_0^R(\omega)$ that controls it via \eqref{eq:neqCritFreq},\eqref{eq:neqCritHop}. 
Its imaginary part shows two peaks, from left to right respectively at the energy of hole and doublon \eqref{eq:doubEn} excitations, corresponding to Mott-Hubbard bands separated by a gap $U$. %, and a large region of negative DoS for $\w<\w_c$. 
% The negative DOS and the enhanced susceptibility close to the critical frequency are distinct non-equilibrium spectral signatures of a dissipative Mott insulator which can be detected in transmission/reflection experiments (see e.g. \cite{magazzuGrifoni2018}). \new{These features in fact cannot occur in equilibrium conditions \cite{scarlatellaSchiro2019a,scarlatellaSchiro2019}.}
We see that the critical frequency (dashed line) is indeed close to the bottom of the upper Hubbard band, around the doublon energy.
% Since the imaginary part of the single-site susceptibility plotted in Fig.~\ref{fig:dmft} has a peak at the doublon energy (the right peak in Fig.~\ref{fig:dmft}, while the left peak is the holon one, see e.g. \cite{supp}) 
% and its zero is $\w_c$ \eqref{eq:neqCritFreq}, then 
% $\w_c \sim \w_{\rm doub}$ means that this function changes sign close to the doublon peak, as shown in Fig.~\ref{fig:dmft}(a). 
We can also understand from this figure the related behaviour of $\w_c$ and $J_c = -1/ReG^R_0(\w_c)$  as functions of $r$ already discussed:  the real part becomes larger if evaluated at a critical frequency closer to $\w_{\rm doub}$ \eqref{eq:doubEn}; in \cite{supp}, we trace this back to the Kramer-Kronig relations of susceptibilities. 
% We see as well that the real part of the susceptibility is increasingly large as $\w_c$ gets closer to the doublon resonance, leading to the reduction of the critical hopping already discussed in Fig. \ref{fig:phaseDiag}(e).
% In addition, one can show that the Kramer-Kronig relations of susceptibilities account for the similar behaviour of $J_c$ and $\w_{\rm doub} - \w_c$, in Fig.~\ref{fig:phaseDiag}(e) (see \cite{supp}). 
Physically, $\w_c \sim \w_{\rm doub}$ reflects into the instability being characterized by a sudden generation of doublons in the dynamics for $J>J_c$, accompanied by an increase in the density: this is shown in  Fig.~\ref{fig:dmft}(c).
% , where we plot the time-dependence of Fock states populations using Gutzwiller mean-field theory for $J>J_c$.
This also corresponds to the onset of phase coherence, which is confirmed by Fig.~\ref{fig:dmft}(d) showing the Wigner function of the limit cycle state: it is not circularly symmetric.  It nonetheless possesses a strong non-classical character, signalled by the negative peak of the Wigner function near the origin, reminiscent of the Mott phase.

Another aspect of the local susceptibilities is important to remark. Their imaginary part (with a minus sign), is a local probe of the density of states (DoS). 
Our instability requires the system to exhibit a negative DoS (NDoS) at frequencies $\w > \mu_{\rm eff}$,
something that is directly connected with the ability to generate amplification gain and that cannot occur in equilibrium conditions~\cite{scarlatellaSchiro2019a,scarlatellaSchiro2021}. This happens for $\mu_{\rm eff} < \w < \w_c$  in our case as indicated in Fig.~\ref{fig:dmft}a,b by the dashed and dotted lines (though its amplitude is not appreciable with the scale used).
The local NDoS directly reflects in a diverging lattice susceptibility at zero momentum and at the critical point 
\eqref{eq:neqCritFreq}\eqref{eq:neqCritHop}.  
The NDoS, the diverging susceptibility at $\w_c \sim \w_{\rm doub}$ and the 
increase in time of doublons and density (Fig. \ref{fig:dmft}c)
are unique properties of the steady-state Mott phase, that distinguish it from a ground-state Mott insulator. These can be probed in experiments by measurements of transmission/reflection and density 
 (see e.g. \cite{magazzuGrifoni2018,maSchuster2019}) .



    
% the righthand side of
% this equation corresponds to an effective density of states
% (DOS).A limit-cycle instability is equivalent to spontaneous
% oscillations at a non-zero frequency in the lab frame. It
% requires the system to exhibit a negative DOS at positive
% frequencies (typically just below ), something that is
% directly connected with the ability to generate amplification gain [33, 43]; we will discuss this more when commenting
% Fig. (2).
% The negative DOS and the enhanced susceptibility
% close to the critical frequency are distinct nonequilibrium
% spectral signatures of a dissipative Mott insulator
% which can be detected in transmission/reflection
% experiments (see e.g. [44]). These features in fact cannot
% occur in equilibrium conditions [32, 43].
\begin{figure}[t]
\centering
% \includegraphics[width=\linewidth]{./ret_sing_dmft}   
% \includegraphics[width=1\linewidth]{./fig2} 
\includegraphics[width=1\linewidth]{fig2_2_nnm1_dens.png} 
\caption{
 Local susceptibility for $\mu_{\rm eff}/U = 1/2$ for (a) the single-site problem and (b) in DMFT, on a Bethe lattice with coordination $z=20$ and at $J_{c,\rm dmft}/U = 0.086$.
The critical frequency $\w_c$ is marked, both in MF and DMFT and for the ground-state transition (dotted-gray), identifying a region of negative density of states (NDoS) in between.
(c) Dynamics of Fock states populations $p_n(t) = \bra{n} \rho(t) \ket{n}$ and density $\aver{n(t)}$ in the limit-cycle phase in MF: at the onset of limit cycles (for $t\kappa \approx 6$) the doublons $(p_2)$ and the density suddenly increase.
(d) Wigner function of the limit-cycle state at $t\kappa=8.3$ in MF: lack of rotational symmetry signals phase coherence. The negative peak signals a non-classical state. For all panels
$\mu_{\rm eff}/U = 1/2$,  $\kappa/U = 10^{-3}$ and $r=100$ as in Fig.~\ref{fig:phaseDiag}(a,c).
}
\label{fig:dmft}
\end{figure}
% \cmm{new: reshuffled v13. bad: not precise}
% \textit{Comparison with ground-state Mott-superfluid transition --} In a non-equilibrium transition, limit-cycle oscillations can take in principle any frequency $\w_c$: in our case, this is determined by \eqref{eq:neqCritFreq}  and found to strongly depend on the pump/loss ratio $r$ (Fig. \ref{fig:phaseDiag}e).

% In a ground-state transition instead, only a static order parameter can form, as time-translation symmetry cannot be broken spontaneously
%  \cite{nozieresNozieres2013,brunoBruno2013b,
% watanabeOshikawa2015a}. 
% The ground-state Mott-superfluid transition is obtained by the same Hamiltonian \eqref{eq:hamBh}: this in fact corresponds to a grand-canonical Hamiltonian $\hat{H}_{\mathrm{gc}} = \hat{H} - \mu_{\rm eff} \sum_i \hat{n}_i$ with equilibrium chemical potential $\mu_{\rm eff}$, where $\mu_{\rm eff}$ has been removed with a rotating frame transformation. 
% % the grand-canonical Hamiltonian $\hat{H}_{\mathrm{gc}} = \hat{H} - \mu_{\rm eff} \sum_i \hat{n}_i$ with equilibrium chemical potential $\mu_{\rm eff}$. Note that one can remove $\mu_{\rm eff}$ from the Hamiltonian in a rotating frame, such that both ground and steady-states cases are described by \eqref{eq:hamBh}. 
% Note also that Eqs. \eqref{eq:neqCritFreq}, \eqref{eq:neqCritHop} also hold for the ground-state, replacing the steady-state susceptibility with the analogous ground-state quantity. In the chosen frame, then equilibrium imposes $\w_c=\mu_{\rm eff}$ and the analogue of Eq. \eqref{eq:neqCritFreq} is guaranteed to be satisfied at this frequency \cite{scarlatellaSchiro2019a}. 

% % For the ground-state Mott-superfluid transition, which is also described by the Hamiltonian \eqref{eq:hamBh}, this corresponds to $\w_c=\mu_{\rm eff}$ (independent of the pump/loss ratio). Note that Eq. \eqref{eq:neqCritFreq} also holds in this case, replacing the steady-state susceptibility with the analogous ground-state quantity: this is guaranteed to be satisfied at this frequency by the equilibrium assumption \cite{scarlatellaSchiro2019a}.  

% The ground-state critical point is also determined by an analogue of Eq.~\eqref{eq:neqCritHop}: note though that whether one uses the ground-state or the steady-state susceptibility does not make a real difference here, as their real parts involved are very similar functions of frequency (because dissipation is weak): the latter case is plotted as a dotted line in Fig. \ref{fig:phaseDiag}d, while the former is the black line. 

% Eventually, the large difference in steady-state and ground-state phase diagrams therefore results from the critical frequencies that are in general different: this is shown for example in Fig. \ref{fig:dmft}(a-b). 
% In the ground-state case, the $\mu_{\rm eff}$ dependence of $\w_c$ gives rise to the the typical ``round'' Mott lobes. Instead, in the steady-state and within one Mott lobe, we find that $\w_c$ is independent of $\mu_{\rm eff}$ and thus also $J_c$, giving rise to the ``flat'' lobes. 

% \textcolor{blue}{Note that the completely $\mu_{\rm eff}$-independent lobes are actually an artifact of our theoretical approaches: we discuss this in \cite{supp}. While there will be some $\mu_{\rm eff}$-dependence beyond those approaches, we still expect a distinct steady-state phase diagram from the ground-state one. 
% In particular, preliminary evidence suggests that at very large pump/loss ratio, but still small pump and loss rates, some physics of the structured reservoir might not be captured by a Lindblad equation \eqref{eq:mbME}.  
% Also, correlations due to dimensionality beyond our approaches, especially in 1D and 2D, might also introduce corrections to the predicted phase diagram.  
% }

% Finally, these observations are in line with Ref. \cite{lebreuillyCarusotto2017}, where for a small 1D chain, and assuming a static transition at $\w_c=\mu_{\rm eff}$, a ground-state-like phase diagram for the steady-state is found. Also, for a finite size system like in Ref. \cite{lebreuillyCarusotto2017} we expect the limit-cycle phase to become a long-lived metastable state, rather than a true steady-state, with a lifetime that only diverges in the thermodynamic limit assumed throughout this manuscript. Our conclusions are therefore compatible with \cite{lebreuillyCarusotto2017}.

% \cmm{new older v13}
\textit{Comparison with ground-state Mott-superfluid transition --} 
For comparison, note that the ground-state Mott-superfluid transition is also described by the Hamiltonian \eqref{eq:hamBh}, up to a rotating frame transformation equivalent to a standard grand-canonical Hamiltonian with equilibrium chemical potential $\mu_{\rm eff}$.
Also, the critical point Eqs.~ \eqref{eq:neqCritFreq}\eqref{eq:neqCritHop} hold as well for the ground-state transition: one simply replaces the steady-state susceptibility with the analogous ground-state quantity.

The main difference is that in the ground-state case, only a static order parameter can form, since time-translation symmetry cannot be broken spontaneously  \cite{nozieresNozieres2013,brunoBruno2013b,watanabeOshikawa2015a}.  This constraints the critical frequency to take the equilibrium value $\w_c=\mu_{\rm eff}$ (in the rotating frame of \eqref{eq:hamBh}). 
Accordingly, Eq. \eqref{eq:neqCritFreq}  for the ground-state is guaranteed to be satisfied at this frequency \cite{scarlatellaSchiro2019a}.
% Evaluating \eqref{eq:neqCritHop} at this frequency one obtains the ground-state phase diagram, the black like in Fig. \ref{fig:phaseDiag}d (more details are in \cite{supp}). 
Note that the $\mu_{\rm eff}$-dependence of $\w_c$ gives rise to the the typical ``round'' Mott lobes. Ground-state calculations are reported in \cite{supp}. 
% Note that it makes little difference here using the appropriate ground-state, rather than steady-state, susceptibility as their real parts involved in \eqref{eq:neqCritHop} are barely distinguishable for weak dissipation: compare solid black and dotted lines in Fig. \ref{fig:phaseDiag}d. 

In a non-equilibrium scenario instead, $\w_c$ is not a priori known: in our case, this must be found from \eqref{eq:neqCritFreq}. 
This predicts a very different value from equilibrium: in particular $\w_c$ strongly depends on the pump/loss ratio $r$, approaching the doublon frequency \eqref{eq:doubEn} for large $r$ as previously shown in Fig 1e; also it does not depend on $\mu_{\rm eff}$ within one lobe, and thus $J_c$ also doesn't, giving rise to the ``flat'' lobes. 
% Evaluating \eqref{eq:neqCritHop} at this frequency we obtained the steady-state phase diagram in Fig.  \ref{fig:phaseDiag}d.

Eventually, the large difference of the steady-state phase diagram compared to the ground-state one (Fig.  \ref{fig:phaseDiag}d) is due to the critical hopping equation \eqref{eq:neqCritHop} being evaluated at a very different critical frequency from equilibrium (the dashed rather then dotted line in Fig. \ref{fig:dmft}a,b). 
Indeed, if the equilibrium frequency is assumed, this yields a phase diagram that only differs perturbatively in the dissipation strength from the ground-state one: the dotted line in Fig.  \ref{fig:phaseDiag}d.


A similar result if found in \cite{lebreuillyCarusotto2017}, that making a similar assumption of a static transition for a similar problem, finds a ground-state-like phase diagram. Our theory is therefore consistent with \cite{lebreuillyCarusotto2017}. 
Also, we expect our instability, predicted in the thermodynamic limit, to be absent in the steady-state of a finite-size system like in \cite{lebreuillyCarusotto2017}, as the limit-cycle would likely become a long-lived metastable state in this case, with a lifetime that only diverges in the thermodynamic limit. % assumed throughout our manuscript. 

% Also, for a finite-size system like in \cite{lebreuillyCarusotto2017}, we expect the limit-cycle to become a long-lived metastable state, rather than a true steady-state, with a lifetime that only diverges in the thermodynamic limit assumed throughout our manuscript. 

% We remark that our results are compatible with \cite{lebreuillyCarusotto2017}, studying a similar model.
% \cite{lebreuillyCarusotto2017} assumes an equilibrium-like time-independent transition at $\w_c=\mu_{\rm eff}$, and finds a ground-state-like phase diagram.
% Forcing the same assumption, 
% we find a similar result: the dotted-line boundary in Fig. \ref{fig:phaseDiag}.
% % Finally, the observations of this paragraph are in line with Ref. \cite{lebreuillyCarusotto2017}, where for a small 1D chain, and assuming a static transition at $\w_c=\mu_{\rm eff}$, a ground-state-like phase diagram for the steady-state is found. 
% Also, for a finite-size system like in \cite{lebreuillyCarusotto2017}, we expect the limit-cycle to become a long-lived metastable state, rather than a true steady-state, with a lifetime that only diverges in the thermodynamic limit assumed throughout our manuscript. 

We remark that the $\mu_{\rm eff}$-independent lobes are actually an artifact of our theoretical approaches, relying on the solution of a single-site ($J=0$) Lindblad equation: this can only capture a step-wise dependence on $\mu_{\rm eff}$ \cite{supp}. 
While we expect some $\mu_{\rm eff}-$dependence beyond those approaches, the steady-state phase diagram will still retain its characteristic dependence on pump/loss ratio $r$, remaining distinct from the ground-state one. 
Preliminary evidence also suggests that at very large $r$, but still weak overall dissipation, a Lindblad equation might not be accurate \eqref{eq:mbME}.  
Finally, correlations due to dimensionality beyond our approaches, especially in 1D and 2D, might also introduce corrections to the predicted phase diagram.  



% The crucial difference in critical frequencies between the two cases is also exemplified in Fig. \ref{fig:dmft}(a-b). In the ground-state case the $\mu_{\rm eff}$ dependence of $\w_c$ gives rise to the the typical ``round'' Mott lobes. Instead, in the steady-state and within one Mott lobe, we find that $\w_c$ is independent of $\mu_{\rm eff}$ and thus also $J_c$, giving rise to the ``flat'' lobes. 
% \textcolor{blue}{The $\mu_{\rm eff}$-independent lobes are actually an artifact of our theoretical approaches, all of which reduce the lattice problem to the solution of a single-site ($J=0$) problem: we discuss this in \cite{supp}. While some dependence will occur beyond those approaches, we still expect a steady-state phase diagram distinct from the ground-state one. 
% We note that, preliminary evidence suggests that at very large pump/loss ratio, but still overall weak dissipation, some physics of the structured reservoir might not be captured by a Lindblad equation \eqref{eq:mbME}.  
% Also, correlations due to dimensionality beyond our approaches, especially in 1D and 2D, might also introduce corrections to the predicted phase diagram.  
% }




% \cmm{older v13}
% \textit{Comparison with ground-state Mott-superfluid transition --} 
% \new{The ground-state Mott-superfluid transition is described by the grand-canonical Hamiltonian $\hat{H}_{\mathrm{gc}} = \hat{H} - \mu_{\rm eff} \sum_i \hat{n}_i$ with equilibrium chemical potential $\mu_{\rm eff}$. One can remove $\mu_{\rm eff}$ from the Hamiltonian in a rotating frame, such that both ground and steady-states cases are described by \eqref{eq:hamBh}. The critical point for the ground-state transition is also determined by Eq.~\eqref{eq:neqCritHop}:  one simply replaces the real part of the steady-state susceptibility with the analogous ground-state quantity. At weak dissipation, those real parts are very similar functions of frequency, therefore Eq.~\eqref{eq:neqCritHop} is essentially the same for steady-state and ground-state. 
% How does then the different phase diagrams arise? 
% The answer is that different Fourier modes become unstable, identified by the critical frequency $\omega_c$.

% In the ground-state case, 
% only one mode can become unstable, corresponding to a static order parameter, since time-translation symmetry cannot be broken spontaneously  \cite{nozieresNozieres2013,brunoBruno2013b,
% watanabeOshikawa2015a}. 
% In the rotating frame of \eqref{eq:hamBh}, this corresponds to $\w_c=\mu_{\rm eff}$. In thermal equilibrium the analog of Eq. \eqref{eq:neqCritFreq} with ground-state susceptibility is guaranteed to be satisfied at this frequency \cite{scarlatellaSchiro2019a}. 
% Indeed evaluating \eqref{eq:neqCritHop} with ground-state susceptibility at $\w_c=\mu_{\rm eff}$ one recovers the ground-state phase diagram (black like in Fig. \ref{fig:phaseDiag}d, see also \cite{supp}): note that, even using the steady-state susceptibility, one gets a similar phase diagram (dotted line in Fig. \ref{fig:phaseDiag}d).

% In a non-equilibrium steady-state instead, $\w_c$ is fixed by Eq.~\eqref{eq:neqCritFreq}. In our case this is found to strongly depend on the pump/loss ratio $r$ and to approach the doublon frequency \eqref{eq:doubEn} at increasing $r$ (Fig. \ref{fig:phaseDiag}). 

% Eventually, the large difference in phase diagrams results from the critical hopping equation \eqref{eq:neqCritHop} being evaluated at different frequencies: $\w_c$ is different in the steady-state and ground-state cases, as shown also in Fig. \ref{fig:dmft}(a-b). In the ground-state case, the $\mu_{\rm eff}$ dependence of $\w_c$ gives rise to the the typical ``round'' Mott lobes. Instead, in the steady-state and within one Mott lobe, we find that $\w_c$ is independent of $\mu_{\rm eff}$ and thus also $J_c$, giving rise to the ``flat'' lobes. 
% \textcolor{blue}{Note that the completely $\mu_{\rm eff}$-independent lobes are actually an artifact of our theoretical approaches: we discuss this in \cite{supp}. While there will be some $\mu_{\rm eff}$-dependence beyond those approaches, we still expect a distinct steady-state phase diagram from the ground-state one. 
% In particular, preliminary evidence suggests that at very large pump/loss ratio, but still small pump and loss rates, some physics of the structured reservoir might not be captured by a Lindblad equation \eqref{eq:mbME}.  
% Also, correlations due to dimensionality beyond our approaches, especially in 1D and 2D, might also introduce corrections to the predicted phase diagram.  
% }
% }

% Finally, these observations are in line with Ref. \cite{lebreuillyCarusotto2017}, where for a small 1D chain, and assuming a static transition at $\w_c=\mu_{\rm eff}$, a ground-state-like phase diagram for the steady-state is found. Also, for a finite size system like in Ref. \cite{lebreuillyCarusotto2017} we expect the limit-cycle phase to become a long-lived metastable state, rather than a true steady-state, with a lifetime that only diverges in the thermodynamic limit assumed throughout this manuscript. Our conclusions are therefore compatible with \cite{lebreuillyCarusotto2017}.




% \cmm{older v11}
% We note that the critical point for the ground-state Mott-superfluid transition is also determined by \eqref{eq:neqCritFreq},\eqref{eq:neqCritHop}; one simply replaces the non-equilibrium susceptibilities with the analogous ground-state quantities.  As a result, the striking differences in the phase diagram between the dissipative and ground-state cases directly arise from differences in susceptibility functions.  

% Consider the behaviour of the critical frequency $\omega_c$.  
% \new{In a non-equilibrium context, it can take non-zero values, corresponding to instabilities that break the time-translation symmetry of the steady-state: recall that in our case it approaches the doublon frequency $\w_c \sim \w_{\rm doub}$ at large pump/loss ratio. 
% By contrast, in the ground-state case 
% of a Hamiltonian with a chemical potential $\hat{H}_{\mathrm{gc}} = \hat{H} - \mu_{\rm eff} \sum_i \hat{n}_i$, the zero-frequency mode becomes unstable as }
% time-translation symmetry cannot be broken spontaneously  \cite{nozieresNozieres2013,brunoBruno2013b,
% watanabeOshikawa2015a}; this in the rotating frame of the Hamiltonian \eqref{eq:hamBh} corresponds to the critical frequency equation \eqref{eq:neqCritFreq} being satisfied at $\w_c=\mu_{\rm eff}$. 

% These critical frequencies are in general different, as shown for example by Fig. \ref{fig:dmft}(a-b), and this mainly accounts for the large discrepancy in the steady-state and ground-state phase diagrams: the critical hopping equation \eqref{eq:neqCritHop}, is evaluated at different frequencies. 
% At a qualitative level, we remark that the doublon frequency $\w_c \sim \w_{\rm doub}$~\eqref{eq:doubEn} within one Mott lobe is a scale independent from the exact value of the chemical potential $\mu_{\rm eff}$ imposed by the reservoir,
%  resulting through \eqref{eq:neqCritFreq},\eqref{eq:neqCritHop} in a $\mu_{\rm eff}$-independent ``flat'' lobe. Instead, in the ground-state case the critical frequency $\w_c=\mu_{\rm eff}$ depends continuously on the equilibrium chemical potential, introducing through \eqref{eq:neqCritHop} the $\mu_{\rm eff}-$dependence that gives rise to the typical ``round'' ground-state Mott lobes. 
% At a more quantitative level, we remark that if we evaluate the critical hopping equation \eqref{eq:neqCritHop} at $\w_c=\mu_{\rm eff}$, while keeping the steady-state susceptibility, this yields a critical hopping plotted as a dotted line in Fig. \ref{fig:phaseDiag}(d) that is very close to the ground-state transition.




% This last observation is in line with the results of Ref. \cite{lebreuillyCarusotto2017}, where for a small 1D chain and assuming a static transition in a frame rotating at frequency $\w_c=\mu_{\rm eff}$ a ground-state-like phase diagram for the steady-state is found. 
% Also, for a finite size system like in Ref. \cite{lebreuillyCarusotto2017} we expect the limit-cycle phase to become a long-lived metastable state, rather than a true steady-state, with a lifetime that only diverges in the thermodynamic limit assumed throughout this manuscript. Our conclusions are therefore compatible with \cite{lebreuillyCarusotto2017}.





\textit{Benchmark of main results --} 
While we considered a square bath spectral function \eqref{eq:lesserBox}
and a Lindblad master equation \eqref{eq:mbME}, our results do not depend on these choices: in \cite{supp} similar results are found with a different spectral function and with a Redfield equation.

Here instead we confirm our results going beyond the previous Gutzwiller and Keldysh approaches to the lattice problem using dynamical mean-field theory (DMFT) \cite{scarlatellaSchiro2021,aokiWerner2014,georgesKotliar1992}, with an impurity solver based on the non-crossing approximation ~\cite{schiro2019quantum,scarlatella2021self}.
% DMFT maps a lattice problem onto a quantum impurity model \cite{aokiWerner2014,georgesKotliar1992}.
% Its non-equilibrium, bosonic version 
For bosons \cite{andersWerner2011,byczukVollhardt2008,strandWerner2015,strandWerner2015a}, DMFT captures non-perturbatively the leading $1/z$ corrections to Gutzwiller, where $z$ is the lattice connectivity. In particular, we predicts the local susceptibility and critical point beyond previous approaches \cite{scarlatellaSchiro2021}. 
% This includes non-perturbatively the leading $1/z$ corrections to Gutzwiller, where $z$ is the lattice connectivity. Using DMFT, on a Bethe lattice \new{for simplicity (i.e. a tree graph with fixed coordination, we refer to \cite{scarlatellaSchiro2021,strandWerner2015,aokiWerner2014,georgesKotliar1992} for technical details on DMFT)}, and an impurity solver based on the non-crossing approximation~\cite{schiro2019quantum,scarlatellaSchiro2021,scarlatella2021self} (NCA) we are able to capture non-trivial aspects of the dissipative Mott phase. In particular we compute 
The local susceptibility at finite hopping  $G^R(\omega,J) = -i \int_{0}^\infty dt e^{i \w t } \aver{[ \hat{a}(t),\hat{a}^\da(0) ]} $ is shown in Fig.~\ref{fig:dmft}b, where one notices the formation of bands replacing the single-site resonances of panel (a).  
We assumed for simplicity a Bethe lattice \cite{georgesRozenberg1996,strandWerner2015} and a homogeneous phase, therefore dropping the site index. 
Similar critical point equations to \eqref{eq:neqCritFreq},\eqref{eq:neqCritHop} exist in DMFT \cite{supp}, involving now the $J$-dependent local susceptibility: solving those we confirm our instability, with a critical frequency $\omega_c$ at the bottom of the doublon band (marked in Fig.~\ref{fig:dmft}). 

% and used it to determine the Mott phase instability for sample values of $\mu_{\rm eff}$. We find that the qualitative picture discussed so far survives to finite connectivity corrections: the critical hopping $J_c$ and critical frequency $\omega_c$ are consistent with the predictions of Gutzwiller mean-field, the latter being close to the bottom of the doublon band, as shown in Fig.~\ref{fig:dmft}(b) where we plot the DMFT result for $G^R(\omega,J)$.


\textit{Conclusion --} We identified a new instability of a dissipatively-prepared Mott insulator, driven by doublon excitations. This is physically different from the ground-state Mott-superfluid transition and is characterized by a dependence on the pump strength that reveals an intrinsic trade-off between the fidelity of the steady-state with a Mott state and its stability to finite hopping. 
We connected the steady-state Mott phase and its instability to peculiar features of the dynamics and susceptibilities that can be measured in experiments.
Our results are relevant to the dissipative preparation of gapped phases of matter that can be achieved with similar protocols \cite{maSchuster2019}, beyond the specific case of a Mott phase studied here, see for example Ref.~\cite{mi2023stable}. 



% \end{document}


\textit{Acknowledgements --}

We acknowledge discussions with Jonathan Simon, Andrei Vrajitoarea, Gabrielle Roberts and Meg Panetta at University of Chicago and Stanford University.


This work was supported by the Air Force Office of Scientific Research MURI program under Grant No. FA9550-19-1-0399, and the Simons Foundation through a Simons Investigator award (Grant No. 669487).
It was also supported by the Engineering and Physical Sciences Research Council [grant number EP/W005484] and by the
European Research Council under the European Union’s
Seventh Framework Programme (FP7/2007-2013)/ERC
Grant Agreement No. 319286 Q-MAC. 
MS acknowledges support from the ANR grant ``NonEQuMat'' (ANR-19-CE47-0001).


For the purpose of open access, the authors has applied a creative commons attribution (CC BY) licence to any author accepted manuscript version arising.
The data to reproduce the results of the manuscript will provided by the author under request.

%This work has been supported by the
%European Research Council under the European Union’s
%Seventh Framework Programme (FP7/2007-2013)/ERC
%Grant Agreement No. 319286 Q-MAC. 

%\bibliography{}
%\end{document}

% \bibliography{dissMott}
\putbib
\end{bibunit}



\pagebreak
\appendix
\begin{bibunit}

\title{Supplemental Material to ``On the stability of dissipatively-prepared Mott insulators of photons''}

\author{Orazio Scarlatella}
\affiliation{T.C.M. Group, Cavendish Laboratory, University of Cambridge, J.J. Thomson Avenue, Cambridge CB3 0HE, UK}
\affiliation{Clarendon Laboratory, University of Oxford, Parks
Road, Oxford OX1 3PU, UK}
\author{Aashish A. Clerk}
\affiliation{Pritzker School of Molecular Engineering, University of Chicago,
5640 South Ellis Avenue, Chicago, Illinois 60637, USA}
\author{Marco Schir\`o}
\affiliation{JEIP, UAR 3573 CNRS, Coll\`{e}ge de France, PSL Research University, 11 Place Marcelin Berthelot, 75321 Paris Cedex 05, France}
%\date{\today}
%\pacs{42.50.Ct,05.70.Ln}

\maketitle
\onecolumngrid
%\setcounter{equation}{0}
%\setcounter{figure}{0}
%\renewcommand{\theequation}{S.\arabic{equation}} 
%\renewcommand{\thefigure}{S.\arabic{figure}} 
%\section*{Supplemental Information}


This Supplemental Material is organized as follows: in Sec.~\ref{secA} we discuss the solution of the single-site driven-dissipative problem and we compare it with its ground-state counterpart; in Sec.~\ref{secB} we derive the equation for the phase boundary of the Mott phase, within Gutzwiller mean-field theory, Strong coupling RPA and DMFT; in Sec.~\ref{secC} we show that the main results of this paper don't change considering a Lorentzian spectral function of the bath, rather then the square one \eqref{eq:lesserBox} of the main text; in Sec.~\ref{secD}
we model the system using a Redfield master equation and discuss how the Mott phase instability is affected; finally Sec.~\ref{secE} provides more details on the steady-state instability and the role of doublons, at large pump to loss ratio.

%First, we derive the perturbation series in the system-bath coupling, its diagrammatic representation, the associated self-energy and Dyson equation. This provides a general framework in which standard weak-coupling master equations can be derived, or more refined approximations can be developed, such as the NCA approximation introduced in this paper. 
%Secondly, we introduce the NCA approximation and we provide details on the validity of the NCA-Markov dynamical map. Furthermore we discuss how to go beyond the NCA dynamical map by resumming one-crossing diagrams (OCA) and show the results for the ohmic and sub-ohmic spin boson model. Then, we present an explicit derivation of the Born-Markov(Redfied) master equation in its standard form starting from the NCA map of the main text and doing further approximations. %This derivation shows that these approaches are equivalent in some specific limits, but the second is valid in wider regimes.
%Finally, we discuss an equation for the steady-state density matrix and present our results for the steady-state correlation functions of the spin-boson model and a comparison with experimental results.

\section{Equilibration, single-site problem and comparison with the ground-state case}
\label{secA}


The master equation \eqref{eq:mbME}, in the regime considered throughout the paper of large pump/loss ratio $r \gg 1$ but weakly coupled environment $\kappa, r \kappa \ll U,J$, and for $U \ll J$, approximates the dynamics of equilibration with a
bath at chemical potential $\mu=\mu_{\rm eff}$ and zero temperature $T=0$: as a consequence, its steady-state is expected to approximate the ground state of a grand-canonical Hamiltonian.
The equilibrium dynamics is in fact characterized by the detailed balance relation, stating that given two eigenstates of the Hamiltonian $\ket{\psi}$ and $\ket{\phi}$, the ratio of the transition rates for going from one state to the other equals the ratio of their equilbrium probabilities, their Boltzmann weights, at temperature $T$ and chemical potential $\mu$: $ {\mathcal{T}^{\rm eq}_{\psi \rw \phi}}/{\mathcal{T}^{\rm eq}_{\phi \rw \psi}} = e^{ (\mu - \eps_\phi  + \eps_\psi)/T }$.
The master equation \eqref{eq:mbME} defines the following transition rates between eigenstates differing by 1 particle: if $\ket{\psi}$ has 1 particle less than $\ket{\phi}$, then $ {\mathcal{T}_{\psi \rw \phi}}/{\mathcal{T}_{\phi \rw \psi}} = r \theta( \mu_{\rm eff} -\eps_\phi +\eps_\psi ) $, that for $r \gg 1$ approximates the detailed balance relation at zero temperature $T\rw 0$ and for $\mu = \mu_{\rm eff}$. A similar discussion is reported in \cite{lebreuillyCarusotto2017}. 
We remark that this partial detailed balance relation does not guarantee an equilibrium steady state, as Eq. \eqref{eq:mbME} does not guarantee thermalization within each fixed particle-number subspace.  Further, in the presence of spectral degeneracies, the master equation will couple populations and coherences.  Nonetheless, the expectation of an equilibrium state is expected to hold if one is deep in the Mott phase, as it becomes rigorous at zero hopping. 

% \subsection{Steady-state}

For a single-site problem ($J=0$) the steady-state can be calculated analytically and shown to correspond, for large $r$, to the Bose-Hubbard-site ground state with chemical potential $\mu = \mu_{\rm eff}$.
Note that a collection of independent sites is also representative of a Mott insulating phase in the Gutzwiller approximation. 
For the single-site problem, the jump operators entering \eqref{eq:mbME} become simply $ A^\da(E_{n+1} - E_{n}  ) =  \aver{ n+1  | a^\da| n} \ket{n+1} \bra{n} $ (omitting the site index), describing transitions between two Fock states, the single-site eigenstates, differing by one boson with energy difference $E_{n+1} - E_{n} =  U n $. The master equation \eqref{eq:mbME} reduces to a simple rate equation for Fock states populations. Therefore the detailed balance argument discussed above becomes rigorous and the steady-state for large $r$ must correspond to the single-site ground state. 
The single-site master equation takes the form: 
\beq
\partial_t \hat{\rho}=-i[\hat{H}, \hat{\rho}] + \kappa  \mathcal{D} [\hat{a}] \hat{\rho}  + r \kappa
\sum_{n=0}^{\infty} S(nU)  \mathcal{D} [\sqrt{n+1}\ket{n+1}\bra{n}] \hat{\rho}  
\label{eq:singSiteME}
\eeq
where the bath spectral function is still given by \eqref{eq:lesserBox}, $S(\omega) = \theta(\mu_{\rm eff} - \omega)$,  and is evaluated at the single-site energy differences. 

% The spectrum is completely non-degenerate, and there's at most a couple of eigenstates with a given transition energy $\omega$: for this reason the sum defining $A(\omega)$ in \eqref{eq:jumpMB} reduces to a single term. 
In the steady-state, one finds that only eigenstates with $E_n -E_{n-1} = n U < \mu_{\rm eff}$ are populated and that the populations are given by $p_n = r^n  ({1-r})/({1-r^{N+1}}) \theta(N-n) $ where $N$ is the last populated Fock state satisfying \eqref{eq:neqPopCond}.
% , that we report for convenience 
% \beq
% % \label{eq:neqPopCond}
% N -\oh < \frac{\mu_{\rm eff}}{U} < N +\oh 
% \eeq
% which is the same condition defining the ground state of a Bose-Hubbard site with chemical potential $\mu = \mu_{\rm eff}$ \eqref{eq:gsOccup}.
For $r\gg 1$ the steady-state approaches the pure state $\ket{N}$, as $p_n \approx \delta_{n,N}$, with $N$ obeying \eqref{eq:neqPopCond}. 

This corresponds to the ground-state of a Bose-Hubbard site with an equilibrium chemical potential $\mu = \mu_{\rm eff}$, i.e. $ \hat{H}_0  = - \mu \hat{n} +  \,  U\hat{n} (\hat{n}-1)/2 $ \cite{fisher89boson,sachdevSachdev2007}. 
The ground-state single-site susceptibility can also be easily computed through a spectral decomposition and reads
\beq
\label{eq:gsRet}
G_{0,\rm gs}^R (\w)  =  \frac{N +1}{ {\w }- \w_{\rm doub}^{\rm gs} + i \eta } - \frac{N }{{\w }- \w_{\rm hol}^{\rm gs} + i \eta }
\eeq
where $\eta$ is a positive infinitesimal.
Like its steady-state counterpart plotted in Fig. \ref{fig:dmft}, it has two peaks at the energies of doublons $\w_{\rm doub}^{\rm gs}$ and holon $\w_{\rm hol}^{\rm gs}$ excitations (corresponding to adding and removing a particle from the ground-state), which are given by 
%The real part of two poles of \eqref{eq:gsRet} are respectively centered at 
\begin{align}
\label{eq:gsExc}
\pm \w_{\rm doub(hol)}^{\rm gs} &= \pm (E_{N \pm 1} - E_{N } ) = - {\mu } + U\lp N  - \oh \pm \oh \rp 
\end{align}
Going to a rotating frame in which the chemical potential is removed from the Hamiltonian like in the main-text Hamiltonian \eqref{eq:hamBh} these energies are shifted by $\mu$, and coincide with the steady-state expression \eqref{eq:doubEn} reported in the main text.
%$\w_{p(h)}$ are respectively positive and negative, as $E_N$ is the ground-state eigen-energy, which is the lowest. 
%The imaginary part $i 0$ is a vanishingly small regularization term.
%The imaginary part of \eqref{eq:gsRet} is the sum of two delta functions centered at $\w_{\rm doub},\, \w_{\rm hole}$, while its real part has simple poles at those frequencies.
%
%Increasing the chemical potential at fixed $n$, the excitations $\w_{\rm doub},\w_{\rm hole}$ shift towards lower energies.
%The ground state condition \eqref{eq:gsOccup} directly translates into the following bounds for $\w_{\rm doub}$ and $\w_{\rm hole}$:
%\begin{align}
%\label{eq:excBound}
%&-U < {\w_{\rm hole}} < 0 &0 <{\w_{\rm doub}} < U
%\end{align}

The ground-state Gutzwiller mean-field phase diagram is given by the main text critical point equations \eqref{eq:neqCritHop}, \eqref{eq:neqCritFreq}, where the steady-state susceptibility is  replaced with the ground-state one \eqref{eq:gsRet}. 
Since the ground-state transition is static, those equations must be evaluated at $\w_{c} = 0$ with the susceptibility \eqref{eq:gsRet} (or equivalently at $\w_c=\mu$ in the rotating frame of \eqref{eq:hamBh}).
Note that \eqref{eq:neqCritFreq} is always satisfied at such a frequency for equilibrium states such as ground-states (see e.g. \cite{scarlatellaSchiro2019a}). The critical hopping is then given by Eq. \eqref{eq:critHopDmft}
% \beq
% \label{eq:gsCrit}
% J_{\rm c} = \frac{\lp U/2 \rp ^2 - \lp U N  -\mu  \rp ^2 }{\mu + U/2}
% \eeq
yielding the well known Mott lobes in the $\mu-J$ plane, plotted in Fig. \ref{fig:phaseDiag} with $\mu =\mu_{\rm eff}$ for comparison with the steady-state phase boundary.


% \subsection{Ground-state}
% \label{secchmott:groundState}

% % \textcolor{red}{Perhaps we don't need this as a separate subsection and we can just add some details of it in the previous one?}
% We now summarize the ground-state single-site and Gutzwiller calculations for comparison with the steady-state.

% The grand canonical Bose-Hubbard site Hamiltonian is 
% $ \hat{H}_0  = - \mu \hat{n} +  \, U/2 \hat{n}^2 $ 
% where $\mu > 0$ is an equilibrium chemical potential.
% Its eigenstates are Fock states $\ket{n}$ with energies $E_n = \frac{U}{2}\lp n - \frac{\mu}{U} \rp^2 - \frac{\mu^2}{2 U}$, which describe a parabola centered at $- \mu / U$. The ground-state occupation $N > 0$ minimizing the energy obeys
% \begin{align}
% \label{eq:gsOccup}
%  N - \oh <\frac{\mu}{U}   <  N + \oh 
% \end{align}
% which is the same condition obeyed by the steady-state \eqref{eq:neqPopCond}.
% The ground-state single-site susceptibility can be easily computed through a spectral decomposition and reads
% \beq
% \label{eq:gsRet}
% G_{0,\rm gs}^R (\w)  =  \frac{N +1}{ {\w }- \w_{\rm doub}^{\rm gs} + i \eta } - \frac{N }{{\w }- \w_{\rm hol}^{\rm gs} + i \eta }
% \eeq
% where $\eta$ is an infinitesimal.
% Like its steady-state counterpart plotted in Fig. \ref{fig:dmft}, it has two peaks at the energies of doublons $\w_{\rm doub}^{\rm gs}$ and holon $\w_{\rm hol}^{\rm gs}$ excitations (corresponding to adding and removing a particle from the ground-state), which are given by 
% %The real part of two poles of \eqref{eq:gsRet} are respectively centered at 
% \begin{align}
% \label{eq:gsExc}
% \pm \w_{\rm doub(hol)}^{\rm gs} &= \pm (E_{N \pm 1} - E_{N } ) = - {\mu } + U\lp N  \pm \oh \rp 
% \end{align}
% Going to a rotating frame in which the chemical potential is removed from the Hamiltonian, like in the main-text Hamiltonian \eqref{eq:hamBh}, these energies coincide with the respective steady-state ones reported in the main text \eqref{eq:doubEn}.

% The ground-state Gutzwiller mean-field phase diagram is given by the main text critical point equations \eqref{eq:neqCritHop}, \eqref{eq:neqCritFreq}, where the steady-state susceptibility is  replaced with the ground-state one. 
% Since the ground-state transition is static, those equations must be evaluated at $\w_{c} = 0$ with the susceptibility \eqref{eq:gsRet} (or equivalently at $\w_c=\mu$ in the rotating frame of \eqref{eq:hamBh}).
% Note that \eqref{eq:neqCritFreq} is always satisfied at $\w_c=0$ for equilibrium states such as ground-states (see e.g. \cite{scarlatellaSchiro2019a}). The critical hopping is then given by
% $ 1/ J_{\rm c}  = - \re G^R_{0,\rm gs}(0) $ and replacing the susceptibility from \eqref{eq:gsRet}, one obtains
% \beq
% \label{eq:gsCrit}
% J_{\rm c} = \frac{\lp U/2 \rp ^2 - \lp U N  -\mu  \rp ^2 }{\mu + U/2}
% \eeq
% yielding the well known Mott lobes in the $\mu-J$ plane, plotted in Fig. \ref{fig:phaseDiag} with $\mu =\mu_{\rm eff}$ for comparison with the steady-state phase boundary.


%\subsection{The Redfield dissipation for the stabilizer}
%
%Considering a Bose-Hubbard Hamiltonian coupled to a stabilizer reservoir via $\hat{H}_{\rm sys, stab}= \sum_i ( \hat{a}_i^\da \hat{B} + \hat{a}_i \hat{B}^\da )$, where $\hat{B}$ ($\hat{B}^\da$)  annihilates (creates) excitations in the stabilizer and taking $C^{-+} (\tau) = 0 $ as in the main text, where  $C^{-+} (\tau) = -i \aver{\hat{B}(\tau) \hat{B}^\da}$ and $C^{+-} (\tau)=-i \aver{\hat{B}^\da(\tau) \hat{B}}$ are the reservoir correlation functions, one obtains the Redfield dissipator for the stabilizer \cite{mozgunovLidar2020}
%\beq
%\hat{\mathcal{D}}_{\rm stab} = {r \kappa} \sum_i \lp a_i^\da \rho \tilde{a}_i  + \tilde{a}_i^\da \rho a_i - a_i \tilde{a}_i^\da \rho -\rho \tilde{a}_i a_i^\da \rp
%\eeq
%where 
%\beq
%\tilde{a}_i =\int_{-\infty}^{\infty} d \tau C^R_{+-}(\tau) a_i(-\tau) 
%\eeq
%defining the retarded function $C_{+-}^R (t) = C_{+-}(t) \theta(t)$. 
%The ``filtered operator'' $\tilde{a}_i$ can be decomposed in the eigenbasis of the Bose-Hubbard Hamiltonian with eigenvectors $\ket{\psi_m}$ and eigenvalues $E_m$ yielding 
%\beq
%\label{eq:stabDiss_bh}
% \tilde{a}_i = \sum_{m,n} S^R(E_n-E_m) \bra{\psi_m} a_i \ket{\psi_n} \ket{\psi_m} \bra{\psi_n}
%\eeq
%where the Fourier transform of the retarded function is defined by $S_{+-}^R(\omega)= \int_{-\infty} ^{\infty} d t C_{+-}^R(t) e^{i \omega t} = \int_{0} ^{\infty} d t C_{+-}(t) e^{i \omega t} $. 
%The following relation holds between the Fourier transforms of retarded and non-retarded functions
%\beq 
%S_{+-}^R(\omega)=\frac{1}{2} S_{+-}(\omega)-i \mathcal{P} \int_{-\infty}^{\infty} \frac{d \omega^{\prime}}{2 \pi} \frac{S_{+-}\left(\omega^{\prime}\right)}{\omega^{\prime}-\omega}
%\eeq
%where $S_{+-}(\omega)= \int_{-\infty} ^{\infty} d t C_{+-}(t) e^{i \omega t} $.
%
%For the single-site problem, the eigenstates of the Hamiltonian are the number eigenstates and the expression for $\tilde{a}_i$ reduces to that of the main text. 
%We remark that in this work we never evaluate the dissipator \eqref{eq:stabDiss_bh} for the many-body problem because our mean-field steady-state treatment reduces to solving the master equation of the single-site problem. 

\section{Equations for the Mott phase instability}\label{secB}

Here we discuss how to obtain the critical point equations \eqref{eq:neqCritFreq}\eqref{eq:neqCritHop} for the phase transition out of the Mott phase. 
We first discuss how to obtain it from the Gutzwiller dynamics and then using the strong-coupling RPA Keldysh field theory. Finally, we discuss how these equations are modified within Dynamical Mean-Field Theory.

\subsection{Time-dependent Gutzwiller}

Within the Gutzwiller ansaz made in the main text, the linear response to a small symmetry-breaking field $\phi(t)$ reads
\begin{equation}
\aver{a(t)} = \int_{\infty}^{-\infty} d\tau G_0^R(t-\tau)\phi(\tau) =-J
\int_{\infty}^{-\infty} d\tau G_0^R(t-\tau)
\aver{a(\tau)}
\end{equation}
where in the last step we have used the Gutzwiller self-consistency condition  $\phi(t) = - J\aver{a(t)}$. Translating the above condition in frequency domain we obtain $ \aver{a(\w)} = -J G_0^R(\omega )\aver{a(\w)}$.
For a given $\w$ such that $a(\w) \neq 0$, this equation gives the critical point equations \eqref{eq:neqCritFreq}, \eqref{eq:neqCritHop}: 
\begin{equation}
1/J + G_0^R(\omega ) = 0
\end{equation}
The smallest $J$ and the corresponding $\w$ satisfying this condition define the critical point $(J_c,\w_c)$. 

The same condition can be recovered in a strong-coupling Keldysh field theory \cite{senguptaDupuis2005b}, as we show in the following. %, where we also show that the unstable mode has wavevector $q=0$, while in our Gutzwiller ansatz we had already assumed that this is the case.

%linear response theory in the steady state and in the early broken-symmetry phase such that $\phi(t) = - J\aver{a(t)}$ is small, gives 
%$ \aver{a(t)} = \int_{\infty}^{-\infty} d\tau G_0^R(t-\tau)\phi(\tau)$, where $G_0(t)$ is the susceptibility computed from the single-site Hamiltonian, which in frequency domain and using $\phi(t) = - J\aver{a(t)}$ becomes $ \aver{a(\w)} = -J G_0^R(\omega )\aver{a(\w)}$.



\subsection{Strong-coupling RPA in the Keldysh path integral}

We make a strong-coupling RPA (random phase approximation) \cite{scarlatellaSchiro2019,senguptaDupuis2005b}, formulating the problem in the language of Keldysh field theory. The Keldysh action, in terms of the coherent fields $a_i,\bar{a}_i$ reads
\begin{align}
\label{eq:hbDrivenDissAction}
\es{
S &= \int_\mathcal{C} dt \lp \sum_i \bar{a}_i  i \pt a_i - H \rp + \sum_i \lp S_{l,i} + S_{\mu_{\rm eff},i} \rp 
} 
\end{align}
where $\int_\mathcal{C}$ is an integral on the Keldysh contour, $H$ is the expectation value of the Hamiltonian on coherent states 
\beq
H = \sum_i\left( \omega_0 \bar{a}_i a_i +\frac{U}{2} \bar{a}_i \bar{a}_i a_i a_i \right) -  \sum_{\aver{ij}} \frac{J}{z} \, \lp \bar{a}_i  a_j + \hc \rp 
\eeq
and $S_{l,i}$ describes the Markovian losses, corresponding to the loss dissipator %\eqref{eq:mbDiss} 
\beq
\label{eq:1pLossKel}
\es{S_{l,i} &= -i \kappa \int_{-\infty}^{\infty} dt \,  \lp \bar{a}_{i-} a_{i+} -\oh  \bar{a}_{i+} a_{i+} -\oh  \bar{a}_{i-} a_{i-} \rp 
}
\eeq
Finally, $S_{\mu_{\rm eff},i}$ describes the coupling to the structured reservoir. For a bath of non-interacting bosons this can be integrated out explicitly, yielding 
\beq
\label{eq:nmBathKel}
 S_{\mu_{\rm eff},i} =-i  \int_\mathcal{C} dt \int_\mathcal{C} dt' r \kappa \sum_i \bar{a}_i(t) C(t-t') a_i(t')
\eeq
where $C(t-t')$ is the bath correlation function, 
% defined in the main text \eqref{eq:lesserBox}, 
with Keldysh indices implicit in the time variables. 
The fourier transform of the lesser component of this function is $S(\w)$ defined in the main text \eqref{eq:lesserBox}.

% and greater components of $C(t-t')$ are defined in the main text through its Fourier transform $S^{-+}(\w),S^{+-}(\w)$. 


%
%\subsection{Hubbard-Stratonovich transformation}
%\label{sec:strongCoup}
%We now introduce a general strong-coupling approach to study driven-dissipative correlated lattice models, which generalizes the equilibrium approach of Ref~\cite{fisherFisherPRB1989} (see also \cite{Sengupta2005,Koch_LeHur_2009,Sachdev}). 
%We write down the Keldysh partition function 
%\beq
%Z = \int \prod_{i} \bold{D}[  \bar{a}_i ,  a_i ] e^{i S[ \lbr \bar{a}_i , a_i \rbr ]  }
%\eeq
%where by $ \lbr \bar{a}_i , a_i \rbr $ we mean the set of bosonic coherent fields of all sites. 
%We rewrite the effective action in the form
%\begin{align}
%S &= S_{loc} - \int_\mathcal{C} dt  \sum_{ij}  \,  \bar{a}_i J_{ij} a_j \\ 
%\label{eq:localAction}
%S_{loc} &= \sum_i \lp S_{u,i} + S_{l,i} + S_{\sigma,i} \rp
%\end{align}
%where $S_{loc}$ describes decoupled sites, containing the term $S_{u,i}$ describing unitary evolution and the dissipative contributions $S_{l,i}$, $S_{\sigma,i}$. 
%The only term coupling different sites is the hopping term, where $J_{ij}$ is the hopping matrix, being equal to $-J$ for nearest neighbours sites and zero otherwise. 
%%\begin{align}
%%\label{eq:hbDrivenDissAction}
%%S &= S_{loc} + \int_\mathcal{C} dt  \sum_{\aver{ij}} J \, \lp \bar{a}_i  a_j + \hc \rp\\ 
%%\es{
%%S_{loc} &= \int_\mathcal{C} dt \lsq \sum_i \bar{a}_i  i \pt a_i - \sum_i\left(\delta \omega_0 \bar{a}_i a_i +\frac{U}{2} \bar{a}_i \bar{a}_i a_i a_i \right) \rsq +  \\
%%&-i \kappa \int_{-\infty}^{\infty} dt \, \lp \bar{a}_{i+} a_{i-} -\oh  \bar{a}_{i+} a_{i+} -\oh  \bar{a}_{i-} a_{i-} \rp + S_{\sigma}
%%}
%%\end{align}

We then do a Hubbard-Stratonovich transformation on the hopping term, by introducing the auxiliary bosonic fields $\psi_i$ by a Gaussian integral 
\beq
\exp \lp { {-} i \int_\mathcal{C} dt \sum_{ij} \bar{a}_i J_{ij} a_j} \rp =\frac{1}{\mathcal{N}} \int \prod_{i} \bold{D} \lsq \bar{\psi}_i {\psi}_i \rsq \exp{\lbr i \int_\mathcal{C} dt  \lsq \sum_{ij} \bar{\psi}_i J_{ij}^\mo \psi_j  + \sum_i \lp  \bar{\psi}_i a_i + \psi_i \bar{a}_i \rp \rsq \rbr} 
\eeq
%
%
%We then decouple the hopping term through a Hubbard-Stratonovich transformation, by introducing an auxiliary bosonic field $\psi_i$ for each site. This transformation 
%\beq
%\exp \lp { {-} i \int_\mathcal{C} dt \sum_{ij} \bar{a}_i J_{ij} a_j} \rp =\frac{1}{\mathcal{N}} \int \prod_{i} \bold{D} \lsq \bar{\psi}_i {\psi}_i \rsq \exp{\lbr i \int_\mathcal{C} dt  \lsq \sum_{ij} \bar{\psi}_i J_{ij}^\mo \psi_j  + \sum_i \lp  \bar{\psi}_i a_i + \psi_i \bar{a}_i \rp \rsq \rbr} 
%\eeq
where $J_{ij}^\mo$ is the inverse hopping matrix and
$\mathcal{N}$ is a normalization coming from Gaussian integration.
%The field $\psi_i$ plays the role of a local order parameter since, for small $\aver{a_i}$, $\aver{\psi_i}$ is linearly related to $\aver{a_i}$ \cite{fisherFisherPRB1989}.
By plugging this in the action \eqref{eq:hbDrivenDissAction}, we can formally integrate on the $a_i,\bar{a}_i$ fields 
%\beq
%\es{
%Z &= \int \prod_{i} \bold{D}[  \bar{a}_i ,  a_i ] e^{i S[ \lbr \bar{a}_i , a_i \rbr ]  } = \\
%&= \frac{1}{\mathcal{N}} \int \prod_{i} \bold{D}[ \bar{\psi}_i, {\psi}_i ] e^{-i \int_\mathcal{C} dt  \sum_{ij} \bar{\psi}_i J_{ij}^\mo \psi_j } \int \prod_{i} \bold{D}[  \bar{a}_i ,  a_i ] e^{i S_{loc} }  e^{ i \int_\mathcal{C} \sum_i \lp  \bar{\psi}_i a_i + \psi_i \bar{a}_i \rp  } 
%}
%\eeq
\beq
\es{
Z &= \int \prod_{i} \bold{D}[  \bar{a}_i ,  a_i ] e^{i S[ \lbr \bar{a}_i , a_i \rbr ]  } = \\
&= \frac{1}{\mathcal{N}} \int \prod_{i} \bold{D}[ \bar{\psi}_i, {\psi}_i ] e^{ i \int_\mathcal{C} dt  \sum_{ij} \bar{\psi}_i J_{ij}^\mo \psi_j } \int \prod_{i} \bold{D}[  \bar{a}_i ,  a_i ] e^{i S_{loc} }  e^{ i \int_\mathcal{C} \sum_i \lp  \bar{\psi}_i a_i + \psi_i \bar{a}_i \rp  } \\ 
&= \frac{1}{\mathcal{N}} \int \prod_{i} \bold{D}[ \bar{\psi}_i, {\psi}_i ] e^{i S_{\rm eff}  [ \lbr \bar{\psi}_i , \psi_i \rbr ] }
}
\eeq
obtaining the effective action for the fields $\psi_i, \bar{\psi}_i$ alone
\beq \label{eqn:Seff}
\mathcal{S}_{\eff}= \int_\mathcal{C} dt \lp \sum_{ij}\bar{\psi}_i J_{ij}^{-1}\psi_j+\sum_i\Gamma[\bar{\psi} _i,\psi_i] \rp
\eeq
where the second term represents the generating functional of the bosonic Green functions of isolated sites, $\Gamma[\bar{\psi}_i,\psi_i]= - i \log\langle T_C e^{i\int_\mathcal{C} dt\left(\bar{\psi}_i a_i+a^\da_i \psi_i\right)}\rangle_{0}$.

We stress that the latter average is taken on the single-site problem, therefore the many-body problem has been formally reduced to calculating the single-site problem cumulants. 
We remark that only at this stage we use the description of the reservoir with a Lindblad equation \eqref{eq:mbME}, that therefore only depends on the spectrum of the single-site problem and can be evaluated in practice. 

To obtain the strong-coupling RPA, we then truncate the effective action at the Gaussian level obtaining 
\beq \label{eqn:Seff}
S_{\eff}= \int_\mathcal{C} dt  \int_\mathcal{C} dt' \sum_{ij}   \bar{\psi}_i  \lp J_{ij}^{-1}+G_0(t-t') \rp \psi_j
\eeq 
where $G_0 (t-t') = - i \aver{ T_\mathcal{C}  a_i(t) a_i^\da(t') } $ is the contour-ordered Green function of the single-site problem.
A second-order phase transition is then signalled by a vanishing retarded component of the effective action, corresponding to a diverging susceptibility at the critical point, which in frequency and momentum space reads 
$ 0=1/J_q-G^R_{0}(\omega)$ with $J_q$ the lattice dispersion.
For a hyper-cubic lattice $J_q=-2{J}/{z}\sum_{\alpha=1}^d\cos q_{\alpha}$ and the first unstable mode is the $q=0$ mode (assuming $\re {G^R_{loc}(\w_c)}<0$), leading to the critical point equations \eqref{eq:neqCritFreq} \eqref{eq:neqCritHop}  of the main text: 
\begin{align}
 0 &= \im G_0^R(\omega_c) \\
\frac{1}{ J_c} &= - \re G_0^R(\omega_c) 
\end{align}


The single-site susceptibility $G_0^R(\omega)$ is finally obtained numerically from the single-site Lindblad equation \eqref{eq:singSiteME}. \\

\textbf{$\mu_{\rm eff}$-independent Mott lobes -- }
It is important to note that in the  single-site Lindblad equation \eqref{eq:singSiteME} the reservoir spectral function \eqref{eq:lesserBox}, $S(\omega) = \theta(\mu_{\rm eff} - \omega)$, is evaluated at the transition energies of the single-site Hamiltonian $E_{n+1} - E_{n} = nU$, that don't depend on $\mu_{\rm eff}$ (or on $J$). Therefore the single-site problem simply depends in a step-wise manner on $\mu_{\rm eff}$, by steps of $U$. This reflects the blockade effect of the non-linearity $U$, that is the incompressibility of the Mott state, already at single-site level. Though, note that this also implies that, not only single-site populations, but any quantity computed from \eqref{eq:singSiteME} depends in the same step-wise manner on $\mu_{\rm eff}$. 
Further, since all the many-body techniques we use (including DMFT) eventually reduce the lattice problem to the single-site problem \eqref{eq:singSiteME}, also any lattice quantity eventually depends on $\mu_{\rm eff}$ in the same step-wise way. This includes the critical hopping $J_c$, resulting in the flat lobes.

The perfectly flat Mott lobes are in fact an artifact of our methods. 
There is two ways of introducing a more non-trivial $\mu_{\rm eff}$ dependence of the critical hopping going beyond those methods.
The first is going beyond a Lindblad equation \eqref{eq:singSiteME} to model the structured reservoir for the single-site problem (or also at lattice level (1)). In fact, that the bath spectral function is evaluated only at the excitation energies of the bare system  $E_{n+1} - E_{n} = nU$, is ultimately a result of the Born-Markov approximation. 
The second is to use approaches to the lattice problem that don't reduce to the single-site problem. In fact in the Lindblad equation for the lattice problem (1) the hopping $J$ enters (together with $\mu_{\rm eff}$) in the dissipator, as this is evaluated at the transition energies of the lattice Hamiltonian, rather than single-site one: this would introduce a non-trivial dependence of $J_c$ on $\mu_{\rm eff}$. 

While we expect some $\mu_{\rm eff}$-dependence beyond our approaches, we also expect that $\w_c$ will still depend strongly on the pump/loss ratio $r$ and be different from its ground-state value. 



\subsection{Dynamical Mean-Field Theory}
\label{eq:dmft_eq}

%A similar procedure is used in \cite{scarlatellaSchiro2021} to derive the DMFT critical point equations reported in \ref{eq:dmft_eq}. 

The phase boundary condition can be obtained within DMFT using a similar procedure, as discussed in detail in Ref.~\cite{scarlatellaSchiro2021}. The key difference with respect to Gutzwiller/RPA is that the self-consistent symmetry breaking field takes contributions both from the coherent neighboring sites, as in Gutzwiller, as well as from the incoherent neighbors through the self-consistent DMFT bath. Assuming a Bethe lattice, the critical point equations \eqref{eq:neqCritFreq} \eqref{eq:neqCritHop} becomes in DMFT \cite{scarlatellaSchiro2021} 
\begin{align}
\label{eq:critFreqDmft}
\operatorname{Im} G^R\left(\omega_c, J_c\right)=0 \\ 
\label{eq:critHopDmft}
\frac{1}{J_c}+\operatorname{Re} G^R\left(\omega_c, J_c\right)+\frac{J_c}{z}\left[\operatorname{Re} G^R\left(\omega_c, J_c\right)\right]^2=0 
\end{align}
in terms of the local susceptibility $G^R(\w,J) = -i \int_{0}^\infty dt e^{i \w t } \aver{[ \hat{a}(t),\hat{a}^\da(0) ]}$.
We remark that the critical frequency $\w_c$ is still the zero of its imaginary part.
On the other hand, the local susceptibility here depends on the hopping and therefore the two equations are coupled, differently from the equations \eqref{eq:neqCritFreq},\eqref{eq:neqCritHop} where we could determine $\w_c$ from the first equation alone.  


%
%\subsection{Beyond mean-field: DMFT calculations}
%
%In order to validate our RPA predictions, we go beyond using the dynamical mean-field theory approach of REF, considering a Bethe lattice and relying on an impurity solver based on the non-crossing approximation (NCA) REF.
%
%The finite-frequency instability of the dissipatively-stabilized Mott phase is still found in this approximation, as we show here for a single value of $\mu$. The calculation of the entire phase diagram in DMFT requires a sensible numerical effort and is left for future work. 
%The critical point equations are given by 
%\begin{equation}
%\begin{gathered}
%\operatorname{Im} G^R\left(\omega_c, J_c\right)=0 \\
%\frac{1}{J_c}+\operatorname{Re} G^R\left(\omega_c, J_c\right)+\frac{J_c}{z}\left[\operatorname{Re} G^R\left(\omega_c, J_c\right)\right]^2=0 
%\end{gathered}
%\end{equation}
%in terms of the retarded Green function. We remark that in DMFT this depends on the hopping and therefore the two equations are coupled, differently from the mean-field equations REF, where we could determine $\w_c$ from the first equation alone.  
%
%In Fig. \ref{fig:dmft} we compare the spectral function $A(\w,J) = -1/\pi \im G^R(\w,J)$ in mean-field and in DMFT, were DMFT captures the broadening of the peak corresponding to a single-particle excitation due to the hybridization between different sites forming the upper Hubbard band. We notice that in DMFT the critical frequency $\w_c$ arises right at the bottom of the upper Hubbard band, similarly to the mean-field prediction, despite the significant change of the shape of the spectral function (redistribution of spectral weight) that was found to sensibly change the critical frequency for a different model REF. 
%For the parameters of Fig. \ref{fig:dmft}, in DMFT the critical hopping is found to be $J_{c,\rm dmft}/U = 0.079$, slightly larger than the mean-field prediction $J_{c, \rm mf}/U = 0.066$, in agreement with the expectation that quantum fluctuations favour the incoherent phase, yet much smaller than the ground state transition at $J_{c,g}/U = 0.166$. 
%
%
%\begin{figure}[H]
%\centering
%\includegraphics[width=0.48\linewidth]{./specFunc_mfDmft.png}     
%\caption{Spectral function for $r=20, \kappa/U = 0.05, \sigma/U=3, \mu/U=1$ in mean-field and DMFT on the Bethe lattice with coordination number $z=6$. In DMFT the spectral function depends on the hopping and is evaluated at the critical hopping $J_{c,\rm dmft}/U = 0.079$ (while the critical hopping predicted by mean-field is $J_{c, \rm mf}/U = 0.066$). The critical frequency in both approximations is shown in the plot.}
%\label{fig:dmft}
%\end{figure}


%\subsection{others}
%
%\begin{figure}[H]
%\centering
%\includegraphics[width=0.48\linewidth]{./jcVsK.png}   
%\caption{Critical hopping $J_c$ as a function of $\kappa/U$ at fixed $r$. For small enough $\kappa$, $J_c$ reaches a constant value. Parameters: $\mu_s/U = 1$, $k/U = 10^{-6}$, $\sigma/U=3.5$, $r=10^2$ in bottom plot, $k/U=10^{-6}$ in top plots. }
%\label{fig:}
%\end{figure}


%\cmm{Finally, the energies to create a particle or a hole excitation in the steady-state also approximate the ground state ones (at zeroth order in the dissipation rate): 
%\beq
%\label{eq:pHExc}
%\w_{p(h)} \simeq E_{n\pm1} - E_n =\pm \lsq \sigma -\mu + U\lp n \pm \oh \rp \rsq
%\eeq
%where there's only a shift of $\sigma$ with respect to the ground state.
%Thus in a rotating frame at frequency $\sigma$ (where the $\sigma$ shift is removed) both the steady-state and its excitations approximate the ground-state ones, for $r \gg 1$ and $\kappa/U,r \kappa/U \ll 1$. }
%



\section{Reservoir with Lorentzian correlations}
\label{secC}
% \textcolor{red}{the results here are obtained with Lindblad or Redfield?}
\begin{figure}
\centering
%\includegraphics[width=0.28\linewidth]{./nVsR_lor.png} 
%\includegraphics[width=0.28\linewidth]{./purVsR_lor.png}  \\
%\includegraphics[width=0.30\linewidth]{./jcVsR_lor.png}  
%\includegraphics[width=0.30\linewidth]{./wcVsR_lor.png} 
\includegraphics[width=0.6\linewidth]{./lor_rwa.png}  
\caption{For a reservoir with Lorentzian spectral function \eqref{eq:lorCorr}, the driven-dissipative single-site steady-state average population $\aver{n}$, purity $\tr(\rho^2)$, critical hopping $J_c$ and frequency $\omega_c$ as a function of the pump-to-loss ratio $r$ (the doublon frequency is indicated by $\w_p$ here). 
The same qualitative behaviour found for the square-shaped reservoir spectral function of the main text is found. 
Parameters: $k/U = 10^{-6}$, $\gamma/U=10^{-3}$, $\omega_{\rm stab}/U = 1$.
}
\label{fig:jcWcLor}
\end{figure}

%\begin{figure}
%\centering
%\includegraphics[width=0.28\linewidth]{./jcVsR_lor.png}  
%\includegraphics[width=0.28\linewidth]{./wcVsR_lor.png}  
%\caption{Critical hopping and frequency as a function of the pump-to-loss ratio for Lorentzian correlations of the stabilizer. $J_c$ vanishes linearly, while $\omega_c$ approaches the energy to create a particle excitation $\w_{\rm doub} = \w_0 + 3U/2$. Parameters: $\w_0/U =0.1$, $k/U = 10^{-6}$, $\omega_{\rm stab}/U = 0.6$, $\sigma/U=0.01$. }
%\label{fig:jcWcLor}
%\end{figure}

In the main text we considered a structured reservoir with a simple square-shaped correlation function  \eqref{eq:lesserBox}, acting as a chemical potential. 
Here we show that our conclusions do not depend on this specific choice, by considering instead a reservoir with a Lorentzian spectral function, which is qualitatively more similar to the experimental realization of Ref. \cite{maSchuster2019}. 
We assume the Lorentzian to be centred at $\w_{\rm res} = U$ corresponding to the transition from 0 to 1 photon in a Bose-Hubbard site with energy difference $E_1 - E_0 = U$ and to have a lifetime $\gamma$ (corresponding to the lifetime of reservoir excitations): 
%$ C_{+-}(\tau) = r\kappa \frac{\gamma}{2} e^{ ( -i \w_{\rm res} -\gamma/2)\tau } $ leading to the frequency domain expression
%\beq
%S_{+-}^R(\w) = r\kappa \frac{(-\gamma/2)}{i(\w-\w_{\rm res}) - \gamma/2}
%\eeq
\beq
\label{eq:lorCorr}
S^R(\w) =  r\kappa \frac{(\gamma/2)^2}{(\w-\w_{\rm res})^2 + (\gamma/2)^2}
\eeq
%whose real part is a Lorentzian function.



%The square-shaped correlation function of the stabilizer is in fact also challenging to realize in experiments: in \cite{lebreuillyCarusotto2017} a realization is proposed in which each site is coupled to a collection of 2-level atoms, inverted by an incoherent pump and with uniformly distributed energy splittings between the two levels in $[0, \w_0 + \mu_{\rm eff}]$ (in the lab frame) or with a single atom whose splitting energy is rapidly modulated in time. 
%
%In \cite{maSchuster2019} the stabilizer rather is a source of photons at a fixed energy $\w_{\rm res} = U/2$ corresponding to the transition $E_1 - E_0 = U/2$ of a Bose-Hubbard site. 

Fig. \ref{fig:jcWcLor} shows in panels (a-b) that upon increasing the pump/loss ratio the single-site steady-state approximates the Bose-Hubbard-site ground state with unit filling, and in panels (c-d) that we get a similar instability of the steady-state Mott phase with a similar behaviour of $\w_c$ and $J_c$ as shown in Fig. \ref{fig:phaseDiag} (d-e). 


We also remark that the Markovian assumption, requiring that the timescale for the decay of the bath spectral function is shorter than the bath-induced system timescales, is strictly speaking not satisfied for the square-shaped correlation functions of the reservoir used in the main text \cite{lebreuillyCarusotto2017}. For the Lorentzian function used here instead the Markovian assumption is justified for our choice of parameters satisfying $\kappa, r \kappa \ll \gamma, \omega_{\rm res}$, showing that our conclusions are not affected.


\section{Redfield master equation}
\label{secD}


\begin{figure}
\centering
\includegraphics[width=0.6\linewidth]{./phaseDiag_red_nnm1}   
\caption{Steady-state Mott instability using the Lindblad \eqref{eq:mbME} versus Redfield \eqref{eq:redME} equation for the same parameters in Fig. \ref{fig:phaseDiag}. Left: steady-state phase diagram, where the ground-state Mott-superfluid phase diagram is plotted for comparison.
Right: critical hopping $J_c$ and frequency difference $\w_{\rm doub} - \w_c$ as a function of the inverse pump/loss ratio.
For the same parameters, the critical hopping is smaller in the Redfield case, as a feature of the doublon resonance that is crucial for the critical point appears at first order in perturbation theory in the dissipation strength in the Redfield case, while only at higher orders in the Lindblad case.
}
\label{fig:phaseDiag_red}
\end{figure}

In Ref. \cite{lebreuillyCarusotto2017} the same system considered in this paper is modelled using a Redfield master equation, rather than the Lindblad equation \eqref{eq:mbME}, which includes non-secular terms and Lamb-shift contributions to the Hamiltonian. 
In this appendix we show that our conclusions do not change considering such a Redfield equation. This reads
\beq
\partial_t \hat{\rho}=-i[\hat{H}, \hat{\rho}] + \kappa \sum_{j} \mathcal{D} [\hat{a}_{j}] \hat{\rho}  + \hat{\mathcal{D}}_{\rm stab}
\label{eq:redME}
\eeq
The structured reservoir dissipator in this case reads
%where $S(\omega)= \int_{-\infty}^{\infty} d t C_{+-}(t) e^{i \omega t}$, and $\mu_{\rm eff}$ represents the maximum energy of reservoir excitations (in the rotating frame of the Hamiltonian \eqref{eq:hamBh}). 
%The stabilizer dissipator then reads
\beq
\label{eq:mbDiss_red}
\hat{\mathcal{D}}_{\rm stab} = {r \kappa} \sum_i \lp a_i^\da \rho \tilde{a}_i  + \tilde{a}_i^\da \rho a_i - a_i \tilde{a}_i^\da \rho -\rho \tilde{a}_i a_i^\da \rp
\eeq
where the ``filtered'' operators $\tilde{a}_i = \sum_{m,n} S^R(\epsilon_n-\epsilon_m) \bra{\psi_m} a_i \ket{\psi_n} \ket{\psi_m} \bra{\psi_n}
$ are defined in terms of the eigenstates $\ket{\psi_n}$ and eigenvalues $\epsilon_n$ of the Bose-Hubbard Hamiltonian \eqref{eq:hamBh} and 
\beq
\label{eq:lesserBox_red}
S^R(\omega)  = \frac{1 }{2}  \theta(\mu_{\rm eff} - {\w}) \theta( \w +\w_0) +  \frac{i}{2\pi} \log \abs{\frac{\mu_{\rm eff} - \w}{\w_0+\w}} 
\eeq
whose imaginary part leads to a Lamb-shift term contributing to the Hamiltonian which is negligible in our results (for which the dissipation is small and staying away from the points in which this function is singular), but we kept it in the results of this appendix. 
%This is a good description as long as the coupling with the system is weak and the Markovian approximation holds.

Using such a Redfield equation \eqref{eq:redME}, the main results presented in the main text are recovered in Fig. \ref{fig:phaseDiag_red}, showing a similar phase diagram and a similar behaviour of $J_c$ and $\w_c$ at small $1/r$ as in Fig. \ref{fig:phaseDiag}.
The main difference is that the critical hopping is much smaller than in the Lindblad case and the exponent with which the critical hopping $J_c$ and the frequency difference $\w_{\rm doub}-\w_c$ decrease as a power law with $1/r$ is larger in absolute value compared to the Lindblad case.


In sections \ref{sec:pertRed} and \ref{sec:pertLind} we report perturbative calculations showing that this quantitative difference is due to the fact that a crucial contribution to the doublon resonance, that mainly determines the critical point, appears at first order in perturbation theory in the dissipation strength in the Redfield case, while for the Lindblad equation it only appears at higher orders in perturbation theory. 

%Developing this perturbative approach, also allows to better understand how the critical frequency solving Eq. \eqref{eq:neqCritFreq} arises, and why it approaches the doublon energy $\w_{\rm doub}$ at large $r$. 

We also remark that the Gutzwiller mean-field dynamics using the Redfield equation becomes unphysical in the limit-cycle phase, yielding negative probabilities, contrarily to the case of the Lindblad equation \eqref{eq:mbME} considered in the main text.






\section{The doublon resonance and the steady-state instability}
\label{secE}


The observation that a zero of the imaginary part of the single-site susceptibility $G_0^R(\w)$ forms close to its doublon peak (see the discussion of Fig. \ref{fig:dmft} (a)) allows to better understand the mathematical origin of the steady-state Mott instability. 
We find that for $\kappa , \kappa r\ll U,J$, where the peaks are well resolved, such a zero emerges due to an ``anti-Lorentzian'' contribution to $\im G_0^R(\w)$ \cite{scarlatellaSchiro2019a} making the doublon peak asymmetric, as Fig. \ref{fig:retGreen} shows plotting $ G_0^R(\w)$ in the case of the Redfield equation \eqref{eq:mbDiss_red} for which this behaviour is particularly pronounced (though for a Lindblad equation the same conclusions are true): in zoom (a) on the doublon resonance the imaginary part is clearly asymmetric, while the same asymmetry is not present in zoom (b) showing the holon resonance, whose imaginary part is almost even around the peak center. 
%This contribution arises from the dissipative contributions to the eigenstates of the Liouvillian, that is proportional to $r$ assumed to be large.  


In the following, we discuss this anti-Lorentzian contribution and how it can lead to the steady-state instability, while in Sec. \ref{eq:mbDiss_red} we use first-order perturbation theory in the Redfield dissipator to capture this contribution analytically and in Sec. \ref{sec:pertLind} we show that the same effect is not captured at first-order in perturbation theory in the Lindblad case (but appears at higher orders), explaining why the instability is somehow less pronounced in this case.



%In the case of the Redfield dissipator \eqref{eq:mbDiss_red}, we use first-order perturbation theory in the dissipators to confirm this with an analytical calculation, showing that only the doublon peak builds such an large anti-Lorentzian structure, due to the strong frequency-dependent structure of the reservoir correlations. 
%
%Interestingly, in Sec. \ref{sec:pertLind} we repeat the the same feature does not appear at first-order in perturbation theory starting from the Lindblad dissipator \eqref{eq:mbDiss}, thus the weaker behaviour four in the Lindblad case is accounted from the fact that occurs at higher order in perturbation theory. 

\begin{figure}[b]
\centering
\includegraphics[width=0.65\linewidth]{./ret_pert_nnm1}    
\caption{solid lines: single-site susceptibility for $r=10^3$ for the Redfield master equation \eqref{eq:redME} and for $\mu_{\rm eff}/U \in [0,1], \kappa/U = 10^{-5}$, obtained numerically. (a-b) zooms respectively around the right dobulon and left holon peak.
The doublon peak is strongly asymmetric due to an “anti-Lorentzian” contribution whose magnitude increases with the pump/loss ratio $r$.
The dotted lines are obtained using the first order perturbation theory in the dissipator, where the doublon peak is approximated by keeping only the corresponding term \eqref{eq:signPeak} in the Lehmann representation, showing that the zero of the imaginary part is correctly captured.
}
\label{fig:retGreen}
\end{figure}

%The retarded Green function of the single-site problem is plotted in Fig. \ref{fig:retGreen}. 
% For large $r \gg 1$, we find that the zero of  $\im G_0^R(\w)$, the critical frequency $\w_c$ of the transition \eqref{eq:neqCritFreq}, is close to the energy $\w_{\rm doub}$ to add a particle to the steady-state, as shown also in Fig. \ref{fig:retGreen}. 
% The zoom in panel (a) shows that the zero is due to an ``anti-resonance'' appearing close to $\w_{\rm doub}$, which is not present in correspondence of other peaks, as shown in panel (b). In the following we investigate how this feature originates.

The retarded Green function (susceptibility) 
$G_0^R(t) = -i \tr \lbr  [ \hat{a}(t),\hat{a}^\da(0) ] \hat{\rho} \rbr \theta(t) $
of a problem described by a Markovian master equation $\pt \hat{\rho} = \hat{\mathcal{L}} \hat{\rho}$ can be decomposed in terms of the right $\hat{r}_\alpha$ and left $\hat{l}_\alpha$ eigenstates of the Lindblad superoperator $\hat{\mathcal{L}}$ and its eigenvalues $\lambda_\alpha$ in its Lehmann representation (see e.g. \cite{scarlatellaSchiro2019a}) 
\begin{equation}
\label{eq:retOpenSys}
G^R(\omega)=\sum_\alpha \frac{w_\alpha}{\omega+\operatorname{Im} \lambda_\alpha-\mathrm{iRe} \lambda_\alpha}
\end{equation}
with $w_\alpha=\operatorname{tr}\left(\hat{a} \hat{r}_\alpha\right) \operatorname{tr}\left(\hat{l}_\alpha^{\dagger}\left[\hat{a}^{\dagger}, \hat{\rho}\right]\right)$. 
It is important to notice that the weights $w_\alpha$ are in general complex, differently from the case of closed systems. 
%This structure allows to relate the critical hopping and frequency in RPA. 
%We first need to find the zeros of the imaginary part of the retarded Green, corresponding to the critical frequency. 
The imaginary part takes the form
\beq
\label{eq:imGr}
\im G^R(\w) = \sum_\alpha \frac{\re w_\alpha \re \lambda_\alpha}{(\w + \im \lambda_\alpha)^2 + (\re\lambda_\alpha)^2} + \frac{\im w_\alpha}{\re \lambda_\alpha } \frac{\re \lambda_\alpha (\w + \im \lambda_\alpha)}{(\w + \im \lambda_\alpha)^2 + (\re\lambda_\alpha)^2}
\eeq
with both Lorentzian and ``anti-Lorentzian'' contributions corresponding to the first and second term. 
Note that in the limit of vanishing dissipation, where $\re \lambda_\alpha \rw 0$ and $\im w_\alpha \rw 0$, the amplitude of the anti-Lorentzian contribution $\im w_\alpha / \re \lambda_\alpha$ can still be finite. 

An important observation is that, in the regime of small dissipation $\kappa, \kappa r \ll U,J$ considered in the paper, the peaks stemming from different contributions in the sum are well separated in frequency, and therefore both the zero of the imaginary part and the critical hopping mostly originate solely from a single peak, the doublon peak (as we show in Fig. \ref{fig:retGreen}. This can be parametrized by 
\beq
f(\w) =b \frac{1 - i \gamma a}{\w + i \gamma} 
\eeq
with imaginary part 
\beq
\im f(\w) = -b \lp \frac{\gamma}{\w^2 +\gamma^2} + a \frac{\w \gamma}{\w^2 +\gamma^2} \rp
\eeq
The latter has a zero at $\w = -1/a$, that we identify with the deviation of the critical frequency from the doublon energy $ -1/a \approx \w_c - \w_{\rm doub} $. 
The critical hopping is then given by $J_c = -1/\re G^R(\w_c)$, that is by the real part of the $f(\w)$
\beq
\re f(\w) = b \lp \frac{\w }{\w^2 +\gamma^2} - \frac{ \gamma^2 a }{\w^2 +\gamma^2} \rp
\eeq
yielding $J_c \propto -1/\re f(-1/a) = 1/(a b) $. 
Therefore we get the proportionality $\w_{\rm doub} - \w_c \propto J_c$, explaining why those quantities vanish simultaneously for large $r$, as shown in Fig. \ref{fig:phaseDiag}.(c-d). 
Another way to understand this proportionality is that the real and imaginary part of the susceptibility are related by the Kramers-Kronig relations \cite{colemanColeman2015}. 

%The steady-state of the single-site problem is diagonal in the occupation eigenstates $\rho = \sum_n p_n \ket{n}\bra{n}$ and only depends on $r$, not on $k$. For a box-shaped reservoir it can be found analytically
%\begin{align}
%p_n &= r^n \theta(N-n) \frac{r-1}{r^{N+1}-1} 
%\end{align}
%where $N$ is the last populated number eigenstate $N -\oh < \frac{\mu_{\rm eff}}{U} < N +\oh $. For $r\gg 1$ one has $p_n \approx 1/r^{N-n} \theta(N-n)$ and all $n\neq N$ occupations are suppressed. 


\subsection{Perturbation theory for the Redfield equation}
\label{sec:pertRed}

We now show, using perturbation theory, that a strong anti-Lorentzian contribution proportional to $r$ indeed arises in correspondence of the doublon peak, giving rise to a zero of $\im G_0^R(\w)$ setting the critical frequency \eqref{eq:neqCritFreq}, and not in correspondence of other resonances of the Green function. 
%
%%We now aim at understanding with perturbative calculations why a strong anti-resonance develops in correspondence of the resonance at $\w\approx\w_c$. 
%
%The main features of $G_0^R(\w)$ are two peaks located at the frequencies $\w_{p(h)}$ corresponding to a hole or doublon excitation, as shown in Fig. \ref{fig:dmft} (left).
%%
%%The steady-state for $r \gg 1$ approximates the ground-state of a Bose-Hubbard site. Therefore we can understand its retarded Green function starting from that of the ground state one
%%\beq
%%G^R (\w)  \approx   \frac{N+1}{ {\w }- \w_{\rm doub} + i 0 } - \frac{N}{{\w }- \w_{\rm hole} + i 0 }
%%\eeq
%%by neglecting dissipative contributions to the weights and to the lifetimes. 
%%This function has two poles at $\w_{p(h)}$ corresponding to the processes of adding a particle and a hole to the steady-state. 
%
%The critical frequency is associated with the development of an anti-Lorentzian contribution, coming from the coupling with the stabilizer reservoir, of the resonance at $\w = \w_{\rm doub}$ as show in Fig. REF. 
In order to capture this feature, we evaluate the Lehmann representation \eqref{eq:retOpenSys} 
%that we report here for convenience
%\begin{equation}
%G^R(\omega)=\sum_\alpha \frac{w_\alpha}{\omega+\operatorname{Im} \lambda_\alpha-\mathrm{iRe} \lambda_\alpha}
%\end{equation}
%with $w_\alpha=\operatorname{tr}\left(\hat{a} \hat{r}_\alpha\right) \operatorname{tr}\left(\hat{l}_\alpha^{\dagger}\left[\hat{a}^{\dagger}, \hat{\rho}\right]\right)$ 
using first-order perturbation theory in the dissipators to approximate the eigenstates and eigenvalues of the Liouvillian. We define $\mathcal{L} = -i \lsq H, \bullet \rsq + \mathcal{D}$ and we perturb in the second term, as for example in \cite{scarlatellaSchiro2019a}. 
The unperturbed eigenstates and eigenvalues are $\lambda_{n,m}^{(0)} = -i (E_n - E_m) $, $r^{(0)}_{n,m} = l^{(0)}_{n,m}= \ket{n}\bra{m}$. 
First-order perturbation theory gives the following corrections to eigenvalues and eigenstates

\begin{align}
\lambda_\alpha^{(1)}&=\operatorname{tr}[ ({l}_\alpha^{(0)} )^{\dagger} {\mathcal{D}} ( {r}_\alpha^{(0)} ) ] & 
{r}_\alpha^{(1)}&=\sum_{\beta \neq \alpha} \frac{\operatorname{tr}[({r}_\beta^{(0)})^{\dagger} {\mathcal{D}}({r}_\alpha^{(0)})]}{\lambda_\alpha^{(0)}-\lambda_\beta^{(0)}} {r}_\beta^{(0)} &
{l}_\alpha^{(1)}&=\sum_{\beta \neq \alpha} \frac{\operatorname{tr}[({l}_\beta^{(0)})^{\dagger} {\mathcal{D}}^{\dagger}({l}_\alpha^{(0)})]}{\lambda_\alpha^{(0) *}-\lambda_\beta^{(0) *}} {l}_\beta^{(0)}
\end{align}

with $\alpha = (n,m)$. 

The term $\tr ( a r_{n,m} )$ in the Green function weights $w_{n,m}$ selects only the eigenstates/values with $m+1=n$, thus we compute only those eigenvalues/states, obtaining
%\begin{align}
%\lambda_\alpha  &\approx \lambda_\alpha^{(0)}  +\lambda_\alpha^{(1)} \\ 
%r_\alpha &\approx r_\alpha^{(0)}  + r_\alpha^{(1)} \\ 
%l_\alpha &\approx l_\alpha^{(0)}  + l_\alpha^{(1)} 
%\end{align}
%with 
%\begin{align}
%\lambda_\alpha^{(1)} &=\operatorname{tr}\left[\left(l_\alpha^{(0)}\right)^{\dagger} \mathcal{D} r_\alpha^{(0)}\right] \\
%r_\alpha^{(1)} &=\sum_{\beta \neq \alpha} \frac{\operatorname{tr}\left[\left(r_\beta^{(0)}\right)^{\dagger} \mathcal{D} r_\alpha^{(0)}\right]}{\lambda_\alpha^{(0)}-\lambda_\beta^{(0)}} r_\beta^{(0)} \\
%l_\alpha^{(1)} &=\sum_{\beta \neq \alpha} \frac{\operatorname{tr}\left[\left(l_\beta^{(0)}\right)^{\dagger} \mathcal{D}^{\dagger} l_\alpha^{(0)}\right]}{\lambda_\alpha^{(0)^*}-{\lambda_\beta^{(0)}}^*} l_\beta^{(0)}
%\end{align}
\small
\begin{align} 
r_{n+1,n} &\approx \ket{n+1} \bra{n} + i \frac{\kappa}{U} \sqrt{(n+1)n} \ket{n} \bra{n-1} - i \frac{r\kappa}{U} \sqrt{(n+2)(n+1)} \lsq S^R_{+-}(E_{n+1} -E_{n}) + S^R_{+-}(E_{n+2} -E_{n+1})^* \rsq \ket{n+2}\bra{n+1} \label{eq:pertVecs_1}\\ 
l_{n+1,n} &\approx \ket{n+1} \bra{n} + i \frac{\kappa}{U} \sqrt{(n+2)(n+1)} \ket{n+2}\bra{n+1} - i \frac{r \kappa}{U} \sqrt{(n+1)n} \lsq S^R_{+-}(E_{n+1} -E_{n})  + S^R_{+-}(E_{n} -E_{n-1})^* \rsq \ket{n} \bra{n-1} \label{eq:pertVecs_2}\\ 
\lambda_{n+1,n} &\approx -i (E_{n+1} -E_n) - \frac{\kappa}{2} (2n+1) - {r\kappa} \lsq S^R_{+-}(E_{n+2} -E_{n+1})^* (n+2) + S^R_{+-}(E_{n+1} -E_{n}) (n+1)  \rsq \label{eq:pertVals} 
\end{align}
\normalsize


%
%There are 2 mechanisms that can suppress terms in those expressions: Secondly $p_{n\neq N} \approx 0$ in the regime $r \gg 1$. This will lead to a strong anti-resonance appearing only for the peak at $\w \approx \w_{\rm doub}$. 
To determine the amplitude of the anti-Lorentzian contribution given by $\im w_{n+1,n}/ \re \lambda_{n+1,n} $, we first compute 
%To understand the anti-Lorentzian structure, we need to compute the ratio $\im w_{n+1,n}/ \re \lambda_{n+1,n} $. 
%First we get
\footnotesize
\beq
\es{
\im w_{n+1,n} &= \sqrt{n+1} \lsq \frac{\kappa}{U}(n+2)\sqrt{n+1} \lp p_{n+1} -p_{n+2} \rp -\frac{r\kappa}{U}  \sqrt{(n+1)n} \lp \re S^R(E_{n+1}-E_n)^* + \re S^R(E_{n}-E_{n-1}) \rp \lp p_{n-1} \sqrt{n} -p_n\sqrt{n+1}\rp \rsq \\ 
&+ (p_n - p_{n+1}) \sqrt{n+1} \lsq \frac{\kappa}{U} n \sqrt{n+1} -\frac{r\kappa}{U} \sqrt{n+1}(n+2) \lp \re S^R(E_{n+1}-E_n) + \re S^R(E_{n+2}-E_{n+1})^* \rp \rsq
}
\eeq
\normalsize 

Then we check for which values of $n$ (i.e. in correspondence of which resonance of the Green function) $\im w_{n+1,n}/ \re \lambda_{n+1,n} $ is of order $r$.
We use that the spectral function of the reservoir $S^R(E_{n}-E_{n-1}) = \frac{1}{2} \theta(\sigma - \abs{E_{n}-E_{n-1}} )$ \eqref{eq:lesserBox_red} (we neglect its imaginary part here) vanishes for $n>N$.
% , and we focus on the terms proportional to $r$ that are the largest ones for $r \gg 1$.
In $\im w_{n+1,n}$ all the terms proportional to $r$ vanish for $n>N$, while $\re \lambda_{n+1,n}$ has no terms proportional to $r$ for $n>N-1$. Then for $n=N$ and only in this case the ratio $\im w_{N+1,N}/ \re \lambda_{N+1,N} $ is proportional to $r$. 
%We notice that this is also independent of $\kappa$, thus staying finite even for $\kappa \rw 0$.
The contribution for $n=N$ corresponds to the doublon resonance, that therefore acquires an anti-Lorentzian contribution proportional to $r$ that eventually leads to a critical point \eqref{eq:neqCritFreq}\eqref{eq:neqCritHop}.  

%Because of $S^R$, the only terms proportional to $r \kappa$ surviving in $\re \lambda_{n+1,n} $ are those with $n \leq N-1$. In $\im w_{n+1,n}$ there are two such terms, $n \leq N$ ($n \leq N-1$) contributions are non-vanishing for the first (second) one. 
%
% 
%In $\re \lambda_{n+1,n} $, all the terms for $n>N-1$ vanish because of $S^R$. 
%In $\im w_{n+1,n}$ there are two terms proportional to $r$: in the first one all terms $n>N$ vanish because of $ S^R$, while population suppress all terms $n\neq N,N+1$, therefore only $n=N$ is order $r$; in the second term, all terms $n>N-1$ vanish by $ S^R$, while populations suppress $n\neq N-1,N$, therefore only the term $n=N-1$ is order $r$. 

%Eventually, when taking the ratio $\im w_{n+1,n}/ \re \lambda_{n+1,n} $, 
%there's only one term for $n=N$ that is proportional to $r$, which agrees with our observation that the corresponding resonance develops a strong anti-Lorentzian contribution. 

Eventually, we can approximate the Green function peak for $n=N$ with the expression
\beq
\label{eq:signPeak}
G_{N+1,N}^R(\w) = \frac{\lp \sqrt{N+1} + \frac{i\kappa}{U} \sqrt{(N+1)}N \rp 
\lp p_N  \sqrt{N+1} - i \frac{r\kappa}{U} p_{N} (N+1)\sqrt{N}  \rp}{\w  - (E_{N+1}-E_N) +\frac{i\kappa}{2}(2N+1) }
\eeq
where we also used that $p_{n\neq N} \approx 0$ for $r \gg 1$.
In Fig. \ref{fig:retGreen} we plot this single-peak approximation as a dashed line for the doublon peak (panel (a)) and show that it correctly captures the anti-Lorentzian, thus the critical frequency $\w_c$ and hopping $J_c$. The dashed line in panel (b) instead approximate the hole-like resonance using the full Green function in perturbation theory. 

We remark that the precise square shape of the reservoir spectral function is not important, as long as this function strongly suppress transitions above a certain energy. Indeed the same behaviour is observed in this Supplemental Material for a reservoir with Lorentzian spectral function.

% While we considered square-shaped reservoir correlations here, an analogous conclusion would hold in other cases in which reservoir correlations strongly suppress transitions above a certain energy, i.e. for the Lorentzian reservoir correlations considered in this Supplemental Material.



\subsection{Perturbation theory for the Lindblad equation}
\label{sec:pertLind}

The perturbation theory using the Lindblad equation \eqref{eq:mbME} instead of Redfield \eqref{eq:redME} is very similar, and one only needs to discard the non-secular terms that are present in the latter case but not in the former.
Discarding those terms yields the same first-order corrections for the eigenvalues of the Liouvillian  \eqref{eq:pertVals} as in Redfield, while all the eigenstates corrections in \eqref{eq:pertVecs_1},\eqref{eq:pertVecs_2} coming from the pump term (proportional to $r \kappa$) correspond to non-secular terms and thus vanish. Therefore the spectral function weights $w_{n+1,n}$ (depending on the eigenstates) also do not depend on $r$ for all $n$ and, eventually, the amplitude of the anti-Lorentzian contributions given by the ratio $\im w_{n+1,n}/ \re \lambda_{n+1,n} $ is never proportional to $r$. 


Note though that the same behaviour of the critical frequency $\w_c$ approaching the doublon energy $ \w_{\rm doub}$ increasing the pump/loss ratio $r$ is observed for the Lindblad case (Fig. \ref{fig:phaseDiag}(e)), therefore a similar anti-Lorentzian contribution to the doublon resonance proportional to $r$ is expected to arise from higher-order terms in perturbation theory.

This difference between the Redfield and Lindblad perturbation theories explains why the dissipative-Mott instability is more pronounced in the former case, with a smaller critical hopping, while both equations give qualitatively the same predictions. 

\putbib
\end{bibunit}
\end{document}
