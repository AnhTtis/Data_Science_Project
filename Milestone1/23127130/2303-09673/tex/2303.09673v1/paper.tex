\documentclass[aps,prx,twocolumn,floatfix,superscriptaddress,showpacs,amsmath,amssymb]{revtex4-2}
\usepackage{graphicx}
\usepackage{epsfig}
\usepackage{hyperref}
\usepackage{color,colordvi}
\usepackage{float}


\newcommand{\ms}[1]{{\color{blue}{#1}}}
\newcommand{\os}[1]{{\color{red}{#1}}}
\newcommand{\aash}[1]{{\color{green}{#1}}}


\newcommand{\eps}{\varepsilon} %nice
\newcommand{\aver}[1]{\langle {#1} \rangle}
\newcommand{\es}[1]{\begin{split}#1\end{split}}
\newcommand{\beq}{\begin{equation}}
\newcommand{\eeq}{\end{equation}}
\newcommand{\la}{\left\langle}
\newcommand{\ra}{\right\rangle}
\newcommand{\lp}{\left(}
\newcommand{\rp}{\right)}
\newcommand{\lsq}{\left[}
\newcommand{\rsq}{\right]}
\newcommand{\lbr}{\left\lbrace}
\newcommand{\rbr}{\right\rbrace}
\newcommand{\da}{\dagger}
\newcommand{\bma}{\begin{pmatrix}}
\newcommand{\ema}{\end{pmatrix}}
\newcommand{\tl}{\tilde}
\newcommand{\bra}[1]{\langle #1 |}
\newcommand{\ket}[1]{| #1 \rangle}
\newcommand{\mo}{{-1}}
\newcommand{\rw}{\rightarrow}
\newcommand{\oh}{\frac{1}{2}}
\newcommand{\w}{\omega}

\newcommand{\re}{\text{Re}}
\newcommand{\im}{\text{Im}}
\newcommand{\tr}{\text{tr}}
\newcommand{\abs}[1]{ \left\lvert #1	\right\rvert}
\newcommand{\mb}[1]{\mathbf{#1}}

%\usepackage{bbold}
\newcommand{\id}{\mathbb{1}}
\newcommand{\lw}{\linewidth}
\newcommand{\sign}{\text{sign}}
\newcommand{\hc}{\text{H.c.}}

\newcommand{\pt}{\partial _t}

\newcommand{\cmm}[1]{\color{blue}#1\color{black}}
\newcommand{\old}[1]{\textcolor{cyan}{#1}}
\newcommand{\new}[1]{\color{red}{#1}\color{black}}

\newcommand{\eff}{\text{eff}}



\begin{document}

 \title{On the stability of dissipatively-prepared Mott insulators of photons}


\author{Orazio Scarlatella}
\affiliation{T.C.M. Group, Cavendish Laboratory, J.J. Thomson Avenue, Cambridge CB3 0HE, UK}
\affiliation{Clarendon Laboratory, University of Oxford, Parks
Road, Oxford OX1 3PU, UK}
\author{Aashish A. Clerk}
\affiliation{Pritzker School of Molecular Engineering, University of Chicago,
5640 South Ellis Avenue, Chicago, Illinois 60637, USA}
\author{Marco Schir\`o}
\affiliation{JEIP, UAR 3573 CNRS, Coll\`{e}ge de France, PSL Research University, 11 Place Marcelin Berthelot, 75321 Paris Cedex 05, France}
%\date{\today}
%\pacs{42.50.Ct,05.70.Ln}


\begin{abstract}
Reservoir engineering is a powerful approach for using controlled driven-dissipative dynamics to prepare target quantum states and phases.  
%Engineered dissipation has proven to be a valuable tool to prepare desired quantum states as the steady-state of a driven-dissipative dynamics. 
In this work, we study a paradigmatic dissipative model that can realize a Mott insulator of photons.  While in some regimes its steady state approximates a Mott-insulating ground-state, this phase becomes unstable through a {\it non-equilibrium} transition towards a coherent limit-cycle phase.  This is completely distinct from the standard equilibrium ground-state Mott insulator to superfluid transition.  
% that is very different from a ground-state Mott-superfluid transition.
This difference has dramatic observable consequences:   the steady-state dissipative Mott phase is far more fragile than its ground state counterpart, becoming unstable for a smaller critical hopping.  Further, 
the higher its fidelity to the ground-state Mott insulator, the more fragile it becomes. We also find that this non-equilibrium instability occurs via a proliferation of doublon excitations, leading to a coherent yet highly non-classical state of light. 
% As a result, the steady-state phase diagram is very different compared to the ground-state Mott-superfluid one, and a much smaller critical hopping might lead to instability. We also predict that the closer the steady-state is to a Mott insulator, the more fragile this phase becomes, that is the smaller the critical hopping.  
% Also, we find that the instability occurs with a proliferations of doublons excitations that sets the frequency of the emerging limit-cycle phase, and we show that the latter phase is a coherent yet highly non-classical state of light. 
% These findings are confirmed using several theoretical approaches and microscopic models of the reservoir, and are therefore expected to broadly affect dissipatively-prepared phases of matter and to be relevant for current experiments. 
\end{abstract}

\maketitle

Dissipation engineering offers a promising avenue for control of quantum devices and simulators, and is actively being explored as a strategy for stabilizing entangled resource states and phases of matter~\cite{poyatosZoller1996,diehl2008quantum,verstraete2009quantum,Harrington2022engineered}. 
While many physical platforms have been considered, 
superconducting circuit QED systems~\cite{schoelkopfGirvin2008,devoretSchoelkopf2013,blais2021circuit} are particularly ideal for engineering tailored reservoirs.  Reservoir engineering has been widely employed in these systems, including quantum-state preparation \cite{dassonnevilleHuard2021,
leghtasDevoret2015,
andersenEichler2019,
luSchuster2017a,
kimchi-schwartzSiddiqi2016,
hollandSchoelkopf2015,
lescanne2020exponential} and autonomous quantum error correction \cite{puttermanNoh2022,
sivakDevoret2022,
berdouLeghtas2022,
grimmDevoret2020,
puriBlais2017,
kapitKapit2017,
touzardDevoret2018a}. 
% \cite{leghtasMirrahimi2013,
% andersenWallraff2020,
% googlequantumaiKelly2021,
% krinnerWallraff2022}
These approaches could be combined with the ability to wire up superconducting circuits into arrays, providing new avenues for  photonic quantum simulators enjoying strong non-linearities and long lifetimes~\cite{houckKoch2012,carusottoSimon2020}. 
%To date, circuit-QED has been used for example to simulate topological systems \cite{roushanMartinis2017a,owensSchuster2018},  driven-dissipative systems \cite{fitzpatrickHouck2017}, many-body localization \cite{xuFan2018,roushanMartinis2017} and for digital simulation \cite{salatheWallraff2015,barendsMartinis2015}. 

A recent breakthrough experiment \cite{maSchuster2019} succeeded in dissipatively stabilizing the first Mott insulator of photons, potentially enabling the realization of correlated quantum fluids of light~\cite{carusotto2013quantum}. In this experiment a Bose-Hubbard lattice was realised using an array of transmon qubits, which were then incoherently driven by a structured reservoir that provided an effective chemical potential for microwave photons~\cite{kapitSimon2014a,hafeziTaylor2015,maSimon2017,
lebreuillyCarusotto2017}. The experiment demonstrated that the steady state of the dynamics was a low-entropy incompressible state with integer filling approximating a ground-state Mott insulator. Despite this achievement, many basic properties of this dissipatively-prepared Mott insulator remain unexplored even at the level of theory.  This includes its basic spectral properties, and the nature of phase transitions out of the Mott state. 


In this paper we show that a dissipatively prepared Mott phase differs from its ground-state equilibrium counterpart in two key respects. First, although the static properties of the steady-state can closely approximate an ideal ground-state Mott insulator, this is not true for finite-frequency excitations:  these retain unique non-equilibrium features, and can drive a phase transition towards a coherent limit-cycle phase, rather than a static superfluid. Second, the different nature of the phase transition in the dissipative system leads to a dramatic reshaping of the  characteristic Mott to superfluid phase diagram~\cite{lebreuillyCarusotto2017,fisher89boson} and a strong suppression of the critical hopping  (as compared to the ground-state transition).  
%its instability is determined by its excitations carrying memory of the underlying non-equilibrium driven-dissipative dynamics.
%the steady-state Mott insulating phase realized in a lossy Bose-Hubbard lattice replenished from a structured reservoir differs qualitatvely from its equilibrium counterpart in one key aspect, its stability towards symmetry breaking.
%becomes unstable through a non-equilibrium instability leading to a coherent limit cycle phase, rather then via a transition similar to the Mott-superfluid ground-state one as suggested by a previous work \cite{lebreuillyCarusotto2017}.
%In fact, despite the steady-state phase well approximates a Mott-insulating ground state, its instability is determined by its excitations carrying memory of the underlying non-equilibrium driven-dissipative dynamics.
%As a consequence, the steady-state phase diagram for the Mott phase instability is very different 
% We show that this instability has profound consequences on the steady-state Mott phase, leading to a very different phase diagram 
%compared to the ground-state transition and, in particular, the critical hopping can be much smaller than in the latter case.
Remarkably, we find that the closer the Mott steady-state is to the Mott-insulating ground state, the more fragile it becomes against finite hopping. %(i.e.~the smaller the critical hopping strength needed for instability).  
This suggests that an arbitrarily high fidelity to the ground-state phase can only be reached in the extreme limit of disconnected lattice sites. 
We provide a thorough comparison of the steady-state Mott instability with its ground-state counterpart, using multiple theoretical methods, and also studying different models for the structured reservoir. We also show that
the coherent limit cycle phase is imprinted with strong non-classical features inherited from the Mott phase.

The distinct features of the dissipatively-prepared Mott state that we identify in this work and its instability 
% their consequences on the driven-dissipative Mott instability 
can be tested in upcoming experiments with superconducting circuits.

%Our work reveals otherWe also highlight that this phase realizes a coherent yet non-classical state of light. 

Note that previous works discussing limit-cycle instabilities in models of driven-dissipative bosons did not consider steady-state Mott insulators \cite{scarlatellaSchiro2019,scarlatellaSchiro2021,szymanskaLittlewood2006}. Other works focused instead on the Mott state in the limit of hard-core bosons \cite{biellaCiuti2017,caleffiCarusotto2022},
%This proliferation also leads the coherent limit cycle phase to possess strongly non-classical features.    
which is very different from the finite interaction regime we consider here (that is more relevant to experiments).  In particular, we find that the coherent limit-cycle phase is intimately connected to a proliferation of doublon excitations, something that could never occur in the hard-core limit, where the transition is driven by a commensurability effect.

\begin{figure*}[t]
\centering
\includegraphics[width=1\linewidth]{./fig1_nnm1}   
\caption{(a) Schematic of the system: a lossy Bose-Hubbard lattice is coupled to a structured reservoir, imposing an effective chemical potential $\mu_{\rm eff}$. 
(b-c) Gutzwiller mean-field (MF) dynamics of the order parameter for $\mu_{\rm eff}/U=1/2$: for a hopping smaller than a critical value $J<J_c$ (b) it features damped oscillations towards the incoherent ($\aver{a} = 0$) Mott steady-state, while for $J>J_c$ (c) it develops finite-amplitude oscillations at about a critical frequency $\w_c$ corresponding to a coherent limit-cycle phase;
here the pump/loss ratio is $r=100$ and the loss rate is $\kappa/U=10^{-3}$, corresponding to $J_c/U = 0.094$, and the initial state is a coherent state  $\ket{\alpha}$ with $\alpha=0.01$. 
(d) The steady-state phase diagram in MF
% the random phase approximation (RPA) 
for $r=150$ and $\kappa/U = 10^{-5}$, featuring the incoherent Mott phase for $J<J_c$, appearing in ``lobes'' with approximately integer filling $N$, and the coherent limit-cycle phase for $J>J_c$;
the ground-state Mott-superfluid phase diagram is plotted for comparison, as well as a steady-state boundary obtained assuming a static, ``ground-state-like'' transition. 
(e) The critical hopping $J_c$ and frequency $\w_c$ as a function of the inverse pump/loss ratio for $\mu_{\rm eff}/U = 1/2$, $\kappa/U = 10^{-6}$, showing that $J_c$ decreases as a power law with $1/r$, while $\omega_c$ approaches the energy $\w_{\rm doub}$ to create a doublon excitation (to add a particle) in the steady-state. 
}
\label{fig:phaseDiag}
\end{figure*}

\textit{The model --} We consider a lossy Bose-Hubbard lattice, incoherently pumped by a structured reservoir providing a source of incoherent excitations within a finite energy window. To mimick the essential features of the stabilizer protocol implemented in \cite{maSchuster2019} we take the reservoir spectrum to have a sharp cutoff, thus generating an effective chemical potential for the lattice.
%This represents a source of excitations, and not also a sink as in the case of equilibrium reservoirs, that have an energy smaller than a cutoff, acting as a chemical potential for the lattice and constituting a toy model for the \emph{stabilizer} protocol implemented in \cite{maSchuster2019}.
A minimal model for the system is given by the Lindblad equation
\beq
\partial_t \hat{\rho}=-i[\hat{H}, \hat{\rho}] + \kappa \sum_{j} \mathcal{D} [\hat{a}_{j}] \hat{\rho}  + r \kappa
\sum_{j,\omega} S (\w)  \mathcal{D} [\hat{A}^\da_{j}(\w) ] \hat{\rho}  
\label{eq:mbME}
\eeq
where $\mathcal{D}[\hat{O}] \hat{\rho}=(\hat{O} \hat{\rho} \hat{O}^{\dagger}-\{\hat{O}^{\dagger} \hat{O}, \hat{\rho}\} / 2)$ is a standard Lindblad dissipator.
$\hat{H}$ is the Bose-Hubbard Hamiltonian 
%in the frame rotating at the on-site frequency $\w_0$
\beq 
\label{eq:hamBh}
\hat{H}= \sum_i  \frac{U}{2}\hat{n}_i(\hat{n}_i-1)  -\frac{J}{z} \sum_{\langle ij\rangle}\left(\hat{a}^{\da}_i \hat{a}_j+\hc \right)
\eeq
where each lattice site hosts a single bosonic mode with annihilation operator $\hat{a}_i$ (and $\hat{n}_i = \hat{a}^\da_i \hat{a}_i$), with an oscillator frequency $\w_0$ that has been gauged away from the Hamiltonian by a rotating frame transformation, $U$ is the on-site interaction, $J$ is the hopping rate between nearest-neighbour sites and $z$ the lattice coordination.
%(number of nearest-neighbours of each site).


The first dissipator in Eq.~(\ref{eq:mbME}) describes linear, Markovian on-site losses at rate $\kappa$.
The second dissipator instead describes the coupling to the structured reservoir, with a pump rate $r \kappa$, such that $r$ is the ratio between pump and loss rates. The jump operators $\hat{A}_{j}^\da(\omega) = \sum_{\varepsilon^{\prime}-\varepsilon=\omega} \hat{\Pi}(\varepsilon^\prime)   \hat{a}_j^\da  \hat{\Pi}\left(\varepsilon\right)$ connect the manifolds of eigenstates of the Hamiltonian differing by one particle, and with energy difference $\varepsilon^{\prime}-\varepsilon=\omega$ (of order $\w_0$ assumed positive and large), where $\hat{\Pi}(\varepsilon)$ is the projector to the manifold with energy $\varepsilon$, while $S(\omega)$ is the spectral function of the reservoir \cite{breuerPetruccione2007}.  We use the simple form
\beq
\label{eq:lesserBox}
S(\omega) = \theta(\mu_{\rm eff}-{\w}) %\theta( \w +\w_0)
\eeq 
with $\mu_{\rm eff}$ representing the maximum energy of reservoir excitations (in the rotating frame of the Hamiltonian \eqref{eq:hamBh}). 
%Note that to simplify the theoretical treatment we consider pumping reservoirs on each lattice site (as opposed to the single localized pump used in Ref.~\cite{maSchuster2019}).
Note that while the master equation \eqref{eq:mbME} may seem simple, solving it exactly is extremely difficult, as it requires the knowledge of the spectrum of the many-body Hamiltonian \eqref{eq:hamBh}.   In this work, we circumvent this problem using controlled many-body techniques that only rely on evaluating the dissipator for a single-site problem. 




\textit{Dissipative preparation of a Mott insulator --} Master equations with a structured reservoir similar to \eqref{eq:mbME} can lead to steady states that have a high fidelity with the Mott insulating ground-state, as Ref. \cite{lebreuillyCarusotto2017} verified numerically for a small 1D chain.
In fact, the master equation \eqref{eq:mbME}, in the regime considered throughout the paper of large pump/loss ratio $r \gg 1$ but weakly coupled environment $\kappa, r \kappa \ll U,J$, satisfies an approximate zero-temperature detailed balance relation \cite{lebreuillyCarusotto2017} (that we report in \cite{Note1}), from which equilibration to the ground-state Mott phase can be expected.

% suggesting that the steady-state of the dynamics might approximate the Bose-Hubbard ground-state. 

% This should happen at least in the Mott phase for $J \ll U$, as the argument becomes exact in the limit $J=0$ of disconnected sites.

In the limit of disconnected sites $J=0$ the steady-state can also be calculated analytically and explicitly be shown to correspond, for large $r$, to the ground state of the Hamiltonian with an equilibrium chemical potential $\mu = \mu_{\rm eff}$ (as we show in \footnote{See Supplemental Material
% [URL will be inserted by publisher]
}), namely a
Fock state $\ket{N}$ with filling set by the chemical potential as 
\beq
\label{eq:neqPopCond}
N - 1 < \frac{\mu_{\rm eff}}{U} < N
\eeq
 % of the Bose-Hubbard Hamiltonian \eqref{eq:hamBh} with an equilibrium chemical potential $\mu = \mu_{\rm eff}$. 
% In Ref. \cite{lebreuillyCarusotto2017} such a Mott phase is shown to be stable up to the critical hopping for the ground-state Mott-superfluid transition. This work though makes the crucial assumption of assuming a time-independent steady-state, ruling out the possiblity of dynamical phases such as limit cycle phases taking over the Mott phase. 



% \textit{Dissipative preparation of a Mott insulator --} The master equation \eqref{eq:mbME}, in the regime considered throughout the paper of large pump/loss ratio $r \gg 1$ but weakly coupled environment $\kappa, r \kappa \ll U,J$, and for $U \ll J$, approximates the dynamics of equilibration with a
% bath at chemical potential $\mu=\mu_{\rm eff}$ and zero temperature $T=0$: as a consequence, its steady-state is expected to approximate the ground state of a grand-canonical Hamiltonian.

% The equilibrium dynamics is in fact characterized by the detailed balance relation, stating that given two eigenstates of the Hamiltonian $\ket{\psi}$ and $\ket{\phi}$, the ratio of the transition rates for going from one state to the other equals the ratio of their equilbrium probabilities, their Boltzmann weights, at temperature $T$ and chemical potential $\mu$: $ {\mathcal{T}^{\rm eq}_{\psi \rw \phi}}/{\mathcal{T}^{\rm eq}_{\phi \rw \psi}} = e^{- (\mu - \eps_\phi  + \eps_\psi)/T }$.
% The master equation \eqref{eq:mbME} defines the following transition rates between eigenstates differing by 1 particle: if $\ket{\psi}$ has 1 particle less than $\ket{\phi}$, then $ {\mathcal{T}_{\psi \rw \phi}}/{\mathcal{T}_{\phi \rw \psi}} = r \theta( \mu_{\rm eff} -\eps_\phi +\eps_\psi ) $, that for $r \gg 1$ approximates the detailed balance relation at zero temperature $T\rw 0$ and for $\mu = \mu_{\rm eff}$. A similar discussion is reported in \cite{lebreuillyCarusotto2017}. 
% We remark that this partial detailed balance relation does not guarantee an equilibrium steady state, as Eq. \eqref{eq:mbME} does not guarantee thermalization within each fixed particle-number subspace.  Further, in the presence of spectral degneracies, the master equation will couple populations and coherences.  Nonetheless, the expectation of an equilibrium state is expceted to hold if one is deep in the Mott phase,  as it becomes rigorous at zero hopping. 

% In the limit $J=0$ of disconnected sites, the steady-state can be calculated analytically and shown to correspond, for large $r$, to the ground state of a Bose-Hubbard-site with an equilibrium chemical potential $\mu = \mu_{\rm eff}$ (as we show in \footnote{See [URL will be inserted by publisher]}), namely a
% Fock state $\ket{N}$ with filling set by the chemical potential as 
% \beq
% \label{eq:neqPopCond}
% N -\oh < \frac{\mu_{\rm eff}}{U} < N +\oh 
% \eeq
% At finite hopping, Ref. \cite{lebreuillyCarusotto2017} studied a Redfield equation similar to \eqref{eq:mbME} showing explicitly that the steady-state of a small 1D chain approximates the ground-state of the Bose-Hubbard Hamiltonian \eqref{eq:hamBh} with an equilibrium chemical potential $\mu = \mu_{\rm eff}$. %, namely with Hamiltonian $\hat{H} - \mu \sum_i \hat{n}_i$.

%Eventually, deviations from this ground-state-like picture
%can still possibly arise in the thermodynamic limit of an infinite lattice, which is the case that we investigate in this paper. 


%\textit{Limit of independent sites. --}


%In this limit the steady-state can be calculated analytically and shown to correspond, in the limit $r\gg 1$ to a pure state given by the Bose-Hubbard site ground state with chemical potential $\mu = \mu_{\rm eff}$ (as we show in \footnote{See [URL will be inserted by publisher]}).

%expected ground-state, owing to the above argument about equilibration.
%the above argument about equilibration becomes rigorous, and 
%Note that a collection of independent sites is also representative of a Mott insulating phase in the Gutzwiller approximation. 

%%MS: 
%For the single-site problem, the jump operators become simply $ A(E_{n-1} - E_{n} < 0 ) =  \aver{ n  | a^\da| n-1} \ket{n} \bra{n-1} $ (omitting the site index), describing transitions between two number eigenstates, the single-site eigenstates, differing by one boson, with level spacing $E_n - E_{n-1} = -\mu_{\rm eff} + U \lp n-\oh \rp$. \new{The master equation \eqref{eq:mbME} reduces to a simple rate equation for number state populations while coherences vanish, therefore the above argument about equilibration becomes rigorous in this case.}

%The spectrum is completely non-degenerate, and there's at most a couple of eigenstates with a given transition energy $\omega$: for this reason the sum defining $A(\omega)$ in \eqref{eq:jumpMB} reduces to a single term. 

%One finds that only eigenstates with $E_n -E_{n-1} < 0$ are populated and that the populations are given by $p_n = r^n  ({1-r})/({1-r^{N+1}}) \theta(N-n) $ where $N$ is the last populated number state satisfying
%\beq
%\label{eq:neqPopCond}
%N -\oh < \frac{\mu_{\rm eff}}{U} < N +\oh 
%\eeq
%Eq. \eqref{eq:neqPopCond} is the same condition defining the ground state of a Bose-Hubbard site with chemical potential $\mu = \mu_{\rm eff}$ (as we show in \footnote{See [URL will be inserted by publisher]}).

%For $r\gg 1$ we find that the steady-state approaches a pure state, as $p_n \approx \delta_{n,N}$, that corresponds to the Bose-Hubbard site ground state, as expected.


%the number state with $N$ bosons, satisfying \eqref{eq:neqPopCond}. 
%We remark that \eqref{eq:neqPopCond} is the same condition defining the ground state of a Bose-Hubbard site with chemical potential $\mu = \mu_{\rm eff}$ (as we show in \cite{Note1}), thus the single-site dissipative dynamics indeed thermalizes to that ground state for $r \gg 1$.

%\cmm{Crucially, a large pump rate compared to losses $r$ is required to stabilize the ground state of the Hamiltonian.}

%Beyond the Gutzwiller mean-field approximation considered here, Ref. \cite{lebreuillyCarusotto2017} showed for a small 1D chain that the steady-state of \eqref{eq:mbME} in a frame rotating at frequency $\w=\sigma$ approximates the wave-function of the Mott insulating ground-state of the Hamiltonian with high fidelity. 

%Following a similar argument for the many-body master equation \eqref{eq:mbME}, \cite{lebreuillyCarusotto2017} concluded that the steady-state in a frame rotating at frequency $\sigma$ and in the same regime would approximate the ground-state of a Bose-Hubbard Hamiltonian. In this reference in fact, the steady-state is computed numerically for a small 1D chain, approximating a Mott insulator with high fidelity.


%\begin{figure}
%\centering
%\includegraphics[width=0.48\linewidth]{./jcVsR.png}  
%\includegraphics[width=0.48\linewidth]{./wcVsR.png}  
%\caption{Critical hopping and frequency as a function of the pump-to-loss ratio. $J_c$ vanishes linearly, while $\omega_c$ approaches the energy to create a particle excitation $\w_{\rm doub}$. Parameters: $\mu_{\rm eff}/U = 1$, $k/U = 10^{-6}$, $\sigma/U=3.5$. }
%\label{fig:3}
%\end{figure}
\textit{Instability of the steady-state Mott phase -- } Although for weak $J$ the steady state of the dissipative evolution is well described by a ground-state Mott insulator, the stability of this phase against stronger hopping turns out to be {\it qualitatively} different from the equilibrium case. To appreciate this point we solve the dynamics with 
a time-dependent Gutzwiller mean-field ansatz assuming uncorreleted sites $\rho(t)= \prod_i \rho_i(t)$, which corresponds to solving a single-site problem dynamics with an effective Hamiltonian $H_0+\phi^{\dagger}(t) a+\phi(t) a^{\dagger}$, where $\phi(t)=-J \langle a(t)\rangle$. 
Fig. \ref{fig:phaseDiag} (b) shows that below a critical hopping $J<J_c$ the order parameter $\aver{a(t)}$ features damped oscillations and eventually vanishes in the steady-state Mott phase. 
At a critical hopping $J_c$ the order parameter develops limit-cycle oscillations at a critical frequency $\w_c$, $\aver{a(t \rw \infty)} = |a| e^{i \omega_c t + i \phi} $, that acquire a stationary and finite amplitude $|a|$ for $J>J_c$.  This is shown in panel (c); note that for $J>J_c$ the frequency deviates from $\w_c$ \cite{scarlatellaSchiro2019}.

The limit-cycle instability corresponds to a second-order dissipative phase transition, in which the steady-state breaks the $U(1)$ symmetry of the model, corresponding to the master equation \eqref{eq:mbME} being invariant under $\hat{a}_i \rw \hat{a}_i e^{i\theta}$. It also breaks continuous time-translation symmetry \cite{scarlatellaSchiro2019,scarlatellaSchiro2021}, something that cannot happen for ground-state transitions \cite{nozieresNozieres2013,brunoBruno2013b,
watanabeOshikawa2015a}, highlighting already that this instability is different from a ground-state Mott-superfluid transition.
% (see also \cite{scarlatellaSchiro2019,scarlatellaSchiro2021} for more details on this kind of transition). 
%For $J>J_c$ instead, panel (c) shows a dynamical instability in which the order parameter oscillations are amplified until they reach a stationary amplitude in the steady state $\aver{a(t \rw \infty)} = |a| e^{i \omega_* t + i \phi} $. 
%The critical point corresponds to a 2nd order phase transition point, in which the steady-state develops a finite order parameter breaking the $U(1)$ symmetry of the model, corresponding to the master equation \eqref{eq:mbME} being invariant under $\hat{a}_i \rw \hat{a}_i e^{i\theta}$, and also develops undamped oscillations at the frequency $\w_* = \w_c$, breaking continuous time-translation symmetry (see also \cite{scarlatellaSchiro2019} for more details about this kind of transition). 
The phase boundary can be found by the condition that the lattice susceptibility to an applied weak coherent field diverges at the instability. 
We compute this quantity using a strong-coupling Keldysh field theory approach~\cite{senguptaDupuis2005b,scarlatellaSchiro2019,Note1}: while in general more powerful that Gurtzwiller mean-field theory, both approaches yield the same critical point.  The critical point follows from
%A critical point equation consistent with the Gutzwiller dynamics can be found using using a strong-coupling RPA (random phase approximation) \cite{senguptaDupuis2005b}, that we derive in \cite{Note1} in the Keldysh path-integral, or equivalently using linear-response theory within the Gutzwiller ansatz, yielding: 
%We solve the master equation \eqref{eq:mbME} for the steady-state using a strong-coupling RPA (random phase approximation) \cite{senguptaDupuis2005b}, that we derive in \cite{Note1} with a Hubbard-Stratonovich transformation in the Keldysh path-integral.
%This approach reduces the many-body problem to computing the time-dependent correlation functions of the single-site problem. 
%It goes beyond a steady-state Gutzwiller mean-field approximation as it captures also the long-time dynamics. 
%For a critical hopping strength $J_c$ the $U(1)$ symmetry of the model, corresponding to the master equation \eqref{eq:mbME} being invariant under $\hat{a}_i \rw \hat{a}_i e^{i\theta}$, is spontaneously broken and the system develops an order parameter $\aver{a} \neq 0$, corresponding to a coherent phase taking over the dissipatively-stabilized Mott insulator. Strong-coupling 
%RPA yields the critical point equations  
\begin{align}
\label{eq:neqCritFreq}
 0 &= - \frac{1}{\pi} \im G_0^R(\omega_c) \\
 \label{eq:neqCritHop}
1/{J_{c}} &= - \re G_0^R(\omega_c)
\end{align}
where $G_0^R(\omega) = -i \int_{0}^\infty dt e^{i \w t } \aver{[ \hat{a}(t),\hat{a}^\da(0) ]}_0 $ is the Fourier transform of the steady-state susceptibility, evaluated over the $J=0$ density matrix corresponding to a single site.
%\os{The same equations can also be obtained from the Gutzwiller mean-field ansatz, as we show in \cite{Note1}.}
% \os{In \cite{Note1} we discuss that the same equations can also be obtained within Gutzwiller mean-field.} 


Eq.~(\ref{eq:neqCritFreq}) determines the critical frequency $\w_c$ and represents a zero net-damping condition: on any given lattice site, the rest of the lattice (viewed as a bath) does not produce any net gain or loss; the righthand side of this equation corresponds to an effective density of states (DOS).  
Eq.~(\ref{eq:neqCritHop}) instead determines the critical hopping $J_c$ from the real part of the susceptibility evaluated at $\w_c$. 
% The right-hand side of Eq. \eqref{eq:neqCritFreq}, which represents a density of states, can turn nagative at finite frequencies \cite{scarlatellaSchiro2019a}, generating a zero of \eqref{eq:neqCritFreq}


A limit-cycle instability is equivalent to spontaneous oscillations at a non-zero frequency in the lab frame. It requires the system to exhibit a negative DOS at positive frequencies (typically just below $\omega_c$), something that is directly connected to the ability to generate amplification gain~\cite{scarlatellaSchiro2019a,scarlatellaSchiro2021}; we discuss this in more detail below, see Fig.~(\ref{fig:dmft}).
% A solution at non-zero frequency in the lab frame is related to a negative density of states, corresponding to the right-hand side of Eq. \eqref{eq:neqCritFreq}, at positive frequencies that can arise in non-equilibrium conditions \cite{scarlatellaSchiro2019a}.


The steady-state phase boundary is plotted in Fig.~\ref{fig:phaseDiag}~(d), showing that the regimes of stable Mott phases form ``lobes'' in the hopping-chemical potential plane.  Each distinct lobe corresponds to a chemical potential range 
\eqref{eq:neqPopCond} where there is an approximate integer filling of the lattice.  One immediately sees that their shape is dramatically altered in the dissipative system as compared to the standard ground-state Mott-superfluid transition (also plotted).  
% The steady-state phase boundary for the Mott instability is plotted in Fig.~\ref{fig:phaseDiag}~(d) and is very different from that of the ground-state Mott-superfluid transition, plotted for comparison.
% We leave the discussion of why this difference arise for later and we discuss now the salient features of the steady-state phase boundary. 
Note that that small regions in the phase diagram that would correspond to a superfluid in the ground state are now turned into Mott-like phases in the dissipative steady state.  This is perhaps not surprising, as one might expect that open-system dynamics easily suppresses phase coherence. 
Conversely it is remarkable that for most values of $\mu_{\rm eff}$ the Mott steady-state is {\it unstable} at a much smaller critical hopping $J_c$ than that for the ground-state transition.

Even more surprisingly, we find that the critical hopping $J_c$ actually decreases towards zero by increasing the pump/loss ratio $r$, $J_c\sim 1/r$ as Fig.~\ref{fig:phaseDiag} (e) shows. This is particularly striking since it implies a trade off between the fidelity of the steady-state to a Mott state, which requires a large $r$ to thermalize to the ground-state (see also~\cite{lebreuillyCarusotto2017}), and the stability of this phase at finite hopping. Eventually, this suggests that an arbitrarily high fidelity of the steady-state with a Mott insulator can only be reached close to the trivial limit of disconnected sites. 

Finally, Fig.~\ref{fig:phaseDiag} (e) also shows that the critical frequency \eqref{eq:neqCritFreq} 
approaches for large $r$ the energy (using $\hbar = 1$ units) to create a doublon excitation, namely to add one particle to the steady-state, $\w_c \sim \w_{\rm doub}$, that in a single-site picture is
\beq
\label{eq:doubEn}
\w_{\rm doub}  =  U  N  
\eeq 
In Fig.~\ref{fig:dmft} (a) we plot the single-site susceptibility $G_0^R(\omega)$ that controls the instability. 
Its imaginary part shows two peaks, corresponding to Mott-Hubbard bands separated by a gap $U$, and a large region of negative DoS for $\w<\w_c$. 

The negative DOS and the enhanced susceptibility close to the critical frequency are distinct non-equilibrium spectral signatures of a dissipative Mott insulator, which can be detected in transmission/reflection experiments.

In Fig.~\ref{fig:dmft} (a) we also see that the critical frequency $\w_c$ \eqref{eq:neqCritFreq} (dashed line) is close to the bottom of the upper Hubbard band, corresponding to the doublon energy $\w_{\rm doub}$ (the left peak corresponds to a holon excitation, see e.g. \cite{Note1}).
% Since the imaginary part of the single-site susceptibility plotted in Fig.~\ref{fig:dmft} has a peak at the doublon energy (the right peak in Fig.~\ref{fig:dmft}, while the left peak is the holon one, see e.g. \cite{Note1}) 
% and its zero is $\w_c$ \eqref{eq:neqCritFreq}, then 
% $\w_c \sim \w_{\rm doub}$ means that this function changes sign close to the doublon peak, as shown in Fig.~\ref{fig:dmft}(a). 
We see as well that the real part of the susceptibility is increasingly large as $\w_c$ gets closer to the doublon resonance, leading to the reduction of the critical hopping already discussed in Fig. \ref{fig:phaseDiag}(e).
% In \cite{Note1} we also analyze the doublon resonance in perturbation theory and show that the behaviour of $J_c$ and $\w_{\rm doub} - \w_c$ in Fig.~\ref{fig:phaseDiag} (e) are similar because they are related via the Kramer-Kronig relations of susceptibilities. 
In addition, one can show that the Kramer-Kronig relations of susceptibilities account for the similar behaviour of $J_c$ and $\w_{\rm doub} - \w_c$ in Fig.~\ref{fig:phaseDiag} (e) (see \cite{Note1}). 
Physically, $\w_c \sim \w_{\rm doub}$ reflects into the instability being characterized by a sudden generation of doublons at the onset of the limit cycle: this is shown in  Fig.~\ref{fig:dmft}(c), where we plot the time-dependence of Fock states populations using Gutzwiller mean-field theory for $J>J_c$.
The proliferation of doublons is accompanied by the onset of phase coherence, which is shown in Fig.~\ref{fig:dmft}(d) by the Wigner function of the limit cycle state:  it is not circularly symmetric.  It nonetheless possesses a strong non-classical character, signalled by the negative peak of the Wigner function near the origin, reminiscent of the Mott phase.


\begin{figure}[t]
\centering
% \includegraphics[width=\linewidth]{./ret_sing_dmft}   
% \includegraphics[width=1\linewidth]{./fig2} 
\includegraphics[width=1\linewidth]{./fig2_2_nnm1} 
\caption{
 The local susceptibility for $\mu_{\rm eff}/U = 1/2$ for (a) a single-site problem and (b) a Bethe lattice with coordination $z=20$, obtained via DMFT and evaluated at the critical hopping $J_{c,\rm dmft}/U = 0.086$; here the loss rate is $ \kappa/U = 0.001$ and the pump/loss ratio $r=100$.
In DMFT the doublon resonance is broadened forming a band. 
The critical frequency $\w_c$ of the Mott-phase instability both in Gutzwiller mean-field (MF) and DMFT is marked, identifying a region of negative density of states (DoS) for $\w<\w_c$ characterized by amplification gain. The critical frequency $\w_c = \mu_{\rm eff}$ for a static ground-state-like transition is also marked.
(c) Dynamics of Fock states populations $p_n(t) = \bra{n} \rho(t) \ket{n}$ in the limit-cycle phase in MF and 
%for $\mu_{\rm eff}/U=1$ for $J=1.2J_c$ such that the Mott phase is unstable and a limit-cycle steady-state develops 
for the same parameters as in Fig.~\ref{fig:phaseDiag}(c): at the onset of the limit cycle (for $t\kappa \approx 6$) the population of doublons $p_2$ suddenly increases.
(d) Wigner function of the limit-cycle state at $t\kappa=8.3$ in MF for the same parameters: the lack of rotational symmetry in phase space signals phase coherence. Also note the strong central negative peak, signalling a non-classical state of light.
}
\label{fig:dmft}
\end{figure}

\textit{Comparison with ground-state Mott-superfluid transition --} 
We note that the critical point for the ground-state Mott-superfluid transition is also determined by \eqref{eq:neqCritFreq},\eqref{eq:neqCritHop}; one simply replaces the non-equilibrium susceptibilities with the analogous ground-state quantities.  As a result, the striking differences in the phase diagram between the dissipative and ground state models directly arise from differences in susceptibility functions.  
% These differences are in turn directly connected to the driven-dissipative nature of our system. 
%different susceptibility, bearing distinctive signatures of the driven-dissipative non-equilibrium dynamics, \os{that in particular reflect into a different critical frequency for the Mott-phase instability. 

Consider the behaviour of the critical frequency $\omega_c$.  
For large pump/loss ratio $r$ we saw that this frequency approaches the doublon frequency, $\w_c \sim \w_{\rm doub}$, corresponding to an instability that breaks time-translation symmetry. 
By contrast, in the ground-state case 
of a Hamiltonian with an equilibrium chemical potential $\mu =\mu_{\rm eff}$, the zero-frequency mode becomes unstable in the lab frame as 
time-translation symmetry cannot be broken spontaneously  \cite{nozieresNozieres2013,brunoBruno2013b,
watanabeOshikawa2015a}, which in the rotating frame of the Hamiltonian \eqref{eq:hamBh} corresponds to the critical frequency equation \eqref{eq:neqCritFreq} being satisfied at $\w_c=\mu_{\rm eff}$. 

% By contrast, in the ground-state case the critical frequency is largely constrained by the equilibrium assumption, as to rule out the possibility of spontaneously breaking time-translation symmetry \cite{nozieresNozieres2013,brunoBruno2013b,
% watanabeOshikawa2015a}: thermalization with an equilibrium bath with chemical potential $\mu$ imposes that $\w_c=\mu$ satisfies \eqref{eq:neqCritFreq} in the rotating frame of the Hamiltonian \eqref{eq:hamBh} \cite{scarlatellaSchiro2019a}. 

These different critical frequencies, shown in Fig. \ref{fig:dmft} for $\mu_{\rm eff}/U=1/2$, mainly account for the large difference in the steady-state and ground-state phase diagrams. At a qualitative level, we remark that in the steady-state case the doublon frequency setting the critical frequency $\w_c \sim \w_{\rm doub}$ is a scale independent from the effective chemical potential $\mu_{\rm eff}$ imposed by the reservoir~\eqref{eq:doubEn}, as long as steady-state populations do not change \eqref{eq:neqPopCond} i.e. within a single Mott lobe, resulting in a $\mu_{\rm eff}$-independent ``flat'' Mott lobe. Instead, in the ground-state case the critical frequency $\w_c=\mu$ depends continuously on the equilibrium chemical potential, introducing through \eqref{eq:neqCritHop} the $\mu-$dependence that gives rise to the typical ``round'' ground-state Mott lobes. 
At a more quantitative level, we remark that if we evaluate the critical hopping equation \eqref{eq:neqCritHop} at $\w_c=\mu_{\rm eff}$, while keeping the steady-state susceptibility, this yields a critical hopping plotted as a dotted line in Fig. \ref{fig:phaseDiag} that is very close to the ground-state transition.




This last observation is in line with the results of Ref. \cite{lebreuillyCarusotto2017}, that for a small 1D chain and assuming a static transition in a frame rotating at frequency $\w_c=\mu_{\rm eff}$ finds a ground-state-like phase diagram for the steady-state. 
Also, for a finite size system like in Ref. \cite{lebreuillyCarusotto2017} we expect the limit-cycle phase to become a long-lived metastable state, rather than a true steady-state, with a lifetime that only diverges in the thermodynamic limit assumed throughout this manuscript. Our conclusions are therefore compatible with \cite{lebreuillyCarusotto2017}.





\textit{Benchmark of main results --} 
While all the results of the main text are based on the simple square bath correlation function \eqref{eq:lesserBox}
and on the Lindblad master equation \eqref{eq:mbME}, our results do not depend on these specific choices as we show in the Supplemental Information \cite{Note1},  where we consider also a Redfield equation as in \cite{lebreuillyCarusotto2017} and a reservoir with Lorentzian spectrum more similar to the realization of \cite{maSchuster2019}. % and where we capture the relevant spectral features for the instability in perturbation theory.

Furthermore, in order to validate our Gutzwiller/Keldysh predictions we go beyond these approaches using the recently developed dynamical mean-field theory (DMFT) for open quantum systems~\cite{scarlatellaSchiro2021}. This allows to include non-perturbatively the leading $1/z$ corrections to Gutzwiller, where $z$ is the lattice connectivity. Using DMFT on a Bethe lattice and an impurity solver based on the non-crossing approximation~\cite{schiro2019quantum,scarlatellaSchiro2021,scarlatella2021self} (NCA) we are able to capture non-trivial aspects of the dissipative Mott phase. In particular we compute the local susceptibility at finite hopping $G^R(\omega,J) = -i \int_{0}^\infty dt e^{i \w t } \aver{[ \hat{a}(t),\hat{a}^\da(0) ]} $, and use it to determine the instability of the dissipative Mott phase~\cite{Note1} for sample values of $\mu_{\rm eff}$. We find that the qualitative picture discussed so far survives to finite connectivity corrections: the critical hopping $J_c$ and critical frequency $\omega_c$ are consistent with the predictions of Gutzwiller mean-field, the latter being close to the bottom of the doublon band, as shown in Fig.~\ref{fig:dmft}(b) where we plot the DMFT result for $G^R(\omega,J)$.


\textit{Conclusion --} In this work we have shown that a dissipatively-prepared photonic Mott insulator, although close to an ideal ground-state Mott for weak hopping, has a stability phase diagram which is both qualitatively and quantitatively different from its ground-state counterpart. In particular the higher its fidelity to the ground-state Mott phase, the lower is its stability against hopping and towards a nonequilibrium broken symmetry phase featuring coherent limit cycle oscillations, proliferation of doublons and realizing a non-classical state of light. These differences are traced back to the peculiar spectral properties of a dissipatively-prepared Mott state, showing gain and becoming unstable at a critical frequency set by the threshold for doublon excitation. 

Our results are confirmed using  several theoretical models and are expected to broadly impact dissipatively-prepared phases of matter. 
% They are also directly relevant for current experiments \cite{maSchuster2019}.
They are also directly relevant for current experiments \cite{maSchuster2019} that might be already able to resolve the unique signatures of the Mott-phase and of its limit-cycle instability. %, such as a negative local density of states or an enhanced  susceptibility to a coherent field with a frequency close to the critical one.


% In this work, we studied a Bose-Hubbard lattice coupled to a structured reservoir imposing an effective chemical potential and leading to a Mott-insulating phase in its steady-state. 
% We showed that the steady-state Mott insulator becomes unstable as the hopping is increased, via a finite-frequency instability that is very different from the ground-state Mott-superfluid transition as it breaks time translation symmetry, leading to a coherent limit-cycle phase with a strong non-classical character, rather than to a static ground-state superfluid phase.  
% Importantly, we discussed that the critical hopping for this instability is expected to be much smaller than that for the ground-state transition. 
% We also found that there is a trade-off between fidelity of the steady-state with a Mott phase, requiring a large pump, and stability of this phase towards a coherent limit-cycle phase.
% % , as increasing the pumping leads to a higher fidelity, but also to a smaller critical hopping for instability.
% We have checked our predictions using different theoretical approaches, including Lindblad and Redfield master equations, perturbation theory, a reservoir with Lorentzian spectrum and by DMFT calculations, all of which give consistent results. 
% Future work might include considering more microscopic models of the reservoir, a more detailed DMFT analysis, for example to investigate whether the critical hopping saturates to a finite value at strong pump due to quantum fluctuations, and investigating the instability in a finite-sized lattice closer to the experiment \cite{maSchuster2019}. 
% \os{On the other hand, current experiments that are far from the thermodynamic limit, might be already able to resolve some signatures of the instability discussed in this paper, such as a negative local density of states at positive frequencies \cite{scarlatellaSchiro2019a} or an enhanced zero-momentum susceptibility to an applied coherent field close to the critical frequency.}

% \end{document}

\textit{Acknowledgements --}

We acknowledge discussions with Jonathan Simon, Andrei Vrajitoarea, Gabrielle Roberts and Meg Panetta at University of Chicago and Stanford University.


This work was supported by the Air Force Office of Scientific Research MURI program under Grant No. FA9550-19-1-0399, and the Simons Foundation through a Simons Investigator award (Grant No. 669487).
This work was also supported by the Engineering and Physical Sciences Research Council [grant number EP/W005484] and by the
European Research Council under the European Union’s
Seventh Framework Programme (FP7/2007-2013)/ERC
Grant Agreement No. 319286 Q-MAC. 
MS acknowledges support from the ANR grant ``NonEQuMat'' (ANR-19-CE47-0001).


For the purpose of open access, the authors has applied a creative commons attribution (CC BY) licence to any author accepted manuscript version arising.
The data to reproduce the results of the manuscript will provided by the author under request.

%This work has been supported by the
%European Research Council under the European Union’s
%Seventh Framework Programme (FP7/2007-2013)/ERC
%Grant Agreement No. 319286 Q-MAC. 

%\bibliography{}
%\end{document}

\bibliography{dissMott}

\pagebreak
\widetext
\pagebreak
\appendix


%\setcounter{equation}{0}
%\setcounter{figure}{0}
%\renewcommand{\theequation}{S.\arabic{equation}} 
%\renewcommand{\thefigure}{S.\arabic{figure}} 


%\section*{Supplemental Information}

\begin{center}
\large{\bf Supplemental Material to `On the stability of dissipatively-prepared Mott insulators of photons' \\}
\end{center}

\begin{center}

Orazio Scarlatella 
\\
{\it T.C.M. Group, Cavendish Laboratory, J.J. Thomson Avenue, Cambridge CB3 0HE, UK}\\
{\it Clarendon Laboratory, University of Oxford, Parks
Road, Oxford OX1 3PU, UK }\\
Aashish Clerk
\\
{\it Pritzker School of Molecular Engineering, University of Chicago,\\
5640 South Ellis Avenue, Chicago, Illinois 60637, USA}\\
Marco Schir\`o\\
{\it JEIP, UAR 3573 CNRS, Coll\`{e}ge de France, PSL Research University,\\ 11 Place Marcelin Berthelot, 75321 Paris Cedex 05, France}
\end{center}


This Supplemental Material is organized as follows: in Sec.~\ref{secA} we discuss the solution of the single-site driven-dissipative problem and we compare it with its ground-state counterpart; in Sec.~\ref{secB} we derive the equation for the phase boundary of the Mott phase, within Gutzwiller mean-field theory, Strong coupling RPA and DMFT; in Sec.~\ref{secC} we show that the main results of this paper don't change considering a Lorentzian spectral function of the bath, rather then the square one \eqref{eq:lesserBox} of the main text; in Sec.~\ref{secD}
we model the system using a Redfield master equation and discuss how the Mott phase instability is affected; finally Sec.~\ref{secE} provides more details on the steady-state instability and the role of doublons, at large pump to loss ratio.

%First, we derive the perturbation series in the system-bath coupling, its diagrammatic representation, the associated self-energy and Dyson equation. This provides a general framework in which standard weak-coupling master equations can be derived, or more refined approximations can be developed, such as the NCA approximation introduced in this paper. 
%Secondly, we introduce the NCA approximation and we provide details on the validity of the NCA-Markov dynamical map. Furthermore we discuss how to go beyond the NCA dynamical map by resumming one-crossing diagrams (OCA) and show the results for the ohmic and sub-ohmic spin boson model. Then, we present an explicit derivation of the Born-Markov(Redfied) master equation in its standard form starting from the NCA map of the main text and doing further approximations. %This derivation shows that these approaches are equivalent in some specific limits, but the second is valid in wider regimes.
%Finally, we discuss an equation for the steady-state density matrix and present our results for the steady-state correlation functions of the spin-boson model and a comparison with experimental results.

\section{Equilibration, single-site problem and comparison with the ground-state case}
\label{secA}


The master equation \eqref{eq:mbME}, in the regime considered throughout the paper of large pump/loss ratio $r \gg 1$ but weakly coupled environment $\kappa, r \kappa \ll U,J$, and for $U \ll J$, approximates the dynamics of equilibration with a
bath at chemical potential $\mu=\mu_{\rm eff}$ and zero temperature $T=0$: as a consequence, its steady-state is expected to approximate the ground state of a grand-canonical Hamiltonian.
The equilibrium dynamics is in fact characterized by the detailed balance relation, stating that given two eigenstates of the Hamiltonian $\ket{\psi}$ and $\ket{\phi}$, the ratio of the transition rates for going from one state to the other equals the ratio of their equilbrium probabilities, their Boltzmann weights, at temperature $T$ and chemical potential $\mu$: $ {\mathcal{T}^{\rm eq}_{\psi \rw \phi}}/{\mathcal{T}^{\rm eq}_{\phi \rw \psi}} = e^{- (\mu - \eps_\phi  + \eps_\psi)/T }$.
The master equation \eqref{eq:mbME} defines the following transition rates between eigenstates differing by 1 particle: if $\ket{\psi}$ has 1 particle less than $\ket{\phi}$, then $ {\mathcal{T}_{\psi \rw \phi}}/{\mathcal{T}_{\phi \rw \psi}} = r \theta( \mu_{\rm eff} -\eps_\phi +\eps_\psi ) $, that for $r \gg 1$ approximates the detailed balance relation at zero temperature $T\rw 0$ and for $\mu = \mu_{\rm eff}$. A similar discussion is reported in \cite{lebreuillyCarusotto2017}. 
We remark that this partial detailed balance relation does not guarantee an equilibrium steady state, as Eq. \eqref{eq:mbME} does not guarantee thermalization within each fixed particle-number subspace.  Further, in the presence of spectral degeneracies, the master equation will couple populations and coherences.  Nonetheless, the expectation of an equilibrium state is expected to hold if one is deep in the Mott phase, as it becomes rigorous at zero hopping. 

% \subsection{Steady-state}

For a single-site problem ($J=0$) the steady-state can be calculated analytically and shown to correspond, for large $r$, to the Bose-Hubbard-site ground state with chemical potential $\mu = \mu_{\rm eff}$.
Note that a collection of independent sites is also representative of a Mott insulating phase in the Gutzwiller approximation. 
For the single-site problem, the jump operators entering \eqref{eq:mbME} become simply $ A^\da(E_n - E_{n-1}  ) =  \aver{ n  | a^\da| n-1} \ket{n} \bra{n-1} $ (omitting the site index), describing transitions between two Fock states, the single-site eigenstates, differing by one boson with energy difference $E_n - E_{n-1} =  U n $. The master equation \eqref{eq:mbME} reduces to a simple rate equation for Fock states populations, therefore the detailed balance argument discussed above becomes rigorous and the steady-state for large $r$ must correspond to the single-site ground state. 
% The spectrum is completely non-degenerate, and there's at most a couple of eigenstates with a given transition energy $\omega$: for this reason the sum defining $A(\omega)$ in \eqref{eq:jumpMB} reduces to a single term. 
In the steady-state, one finds that only eigenstates with $E_n -E_{n-1} < \mu_{\rm eff}$ are populated and that the populations are given by $p_n = r^n  ({1-r})/({1-r^{N+1}}) \theta(N-n) $ where $N$ is the last populated Fock state satisfying \eqref{eq:neqPopCond}.
% , that we report for convenience 
% \beq
% % \label{eq:neqPopCond}
% N -\oh < \frac{\mu_{\rm eff}}{U} < N +\oh 
% \eeq
% which is the same condition defining the ground state of a Bose-Hubbard site with chemical potential $\mu = \mu_{\rm eff}$ \eqref{eq:gsOccup}.
For $r\gg 1$ the steady-state approaches the pure state $\ket{N}$, as $p_n \approx \delta_{n,N}$, with $N$ obeying \eqref{eq:neqPopCond}. 

This corresponds to the ground-state of a Bose-Hubbard site with an equilibrium chemical potential $\mu = \mu_{\rm eff}$, i.e. $ \hat{H}_0  = - \mu \hat{n} +  \,  U\hat{n} (\hat{n}-1)/2 $ \cite{fisher89boson,sachdevSachdev2007}. 
The ground-state single-site susceptibility can also be easily computed through a spectral decomposition and reads
\beq
\label{eq:gsRet}
G_{0,\rm gs}^R (\w)  =  \frac{N +1}{ {\w }- \w_{\rm doub}^{\rm gs} + i \eta } - \frac{N }{{\w }- \w_{\rm hol}^{\rm gs} + i \eta }
\eeq
where $\eta$ is an infinitesimal.
Like its steady-state counterpart plotted in Fig. \ref{fig:dmft}, it has two peaks at the energies of doublons $\w_{\rm doub}^{\rm gs}$ and holon $\w_{\rm hol}^{\rm gs}$ excitations (corresponding to adding and removing a particle from the ground-state), which are given by 
%The real part of two poles of \eqref{eq:gsRet} are respectively centered at 
\begin{align}
\label{eq:gsExc}
\pm \w_{\rm doub(hol)}^{\rm gs} &= \pm (E_{N \pm 1} - E_{N } ) = - {\mu } + U\lp N  - \oh \pm \oh \rp 
\end{align}
Going to a rotating frame in which the chemical potential is removed from the Hamiltonian like in the main-text Hamiltonian \eqref{eq:hamBh} these energies are shifted by $\mu$, and the doublon one coincide with the steady-state expression \eqref{eq:doubEn} reported in the main text.
%$\w_{p(h)}$ are respectively positive and negative, as $E_N$ is the ground-state eigen-energy, which is the lowest. 
%The imaginary part $i 0$ is a vanishingly small regularization term.
%The imaginary part of \eqref{eq:gsRet} is the sum of two delta functions centered at $\w_{\rm doub},\, \w_{\rm hole}$, while its real part has simple poles at those frequencies.
%
%Increasing the chemical potential at fixed $n$, the excitations $\w_{\rm doub},\w_{\rm hole}$ shift towards lower energies.
%The ground state condition \eqref{eq:gsOccup} directly translates into the following bounds for $\w_{\rm doub}$ and $\w_{\rm hole}$:
%\begin{align}
%\label{eq:excBound}
%&-U < {\w_{\rm hole}} < 0 &0 <{\w_{\rm doub}} < U
%\end{align}

The ground-state Gutzwiller mean-field phase diagram is given by the main text critical point equations \eqref{eq:neqCritHop}, \eqref{eq:neqCritFreq}, where the steady-state susceptibility is  replaced with the ground-state one \eqref{eq:gsRet}. 
Since the ground-state transition is static, those equations must be evaluated at $\w_{c} = 0$ with the susceptibility \eqref{eq:gsRet} (or equivalently at $\w_c=\mu$ in the rotating frame of \eqref{eq:hamBh}).
Note that \eqref{eq:neqCritFreq} is always satisfied at $\w_c=0$ for equilibrium states such as ground-states (see e.g. \cite{scarlatellaSchiro2019a}). The critical hopping is then given by Eq. \eqref{eq:critHopDmft}
% \beq
% \label{eq:gsCrit}
% J_{\rm c} = \frac{\lp U/2 \rp ^2 - \lp U N  -\mu  \rp ^2 }{\mu + U/2}
% \eeq
yielding the well known Mott lobes in the $\mu-J$ plane, plotted in Fig. \ref{fig:phaseDiag} with $\mu =\mu_{\rm eff}$ for comparison with the steady-state phase boundary.


% \subsection{Ground-state}
% \label{secchmott:groundState}

% % \textcolor{red}{Perhaps we don't need this as a separate subsection and we can just add some details of it in the previous one?}
% We now summarize the ground-state single-site and Gutzwiller calculations for comparison with the steady-state.

% The grand canonical Bose-Hubbard site Hamiltonian is 
% $ \hat{H}_0  = - \mu \hat{n} +  \, U/2 \hat{n}^2 $ 
% where $\mu > 0$ is an equilibrium chemical potential.
% Its eigenstates are Fock states $\ket{n}$ with energies $E_n = \frac{U}{2}\lp n - \frac{\mu}{U} \rp^2 - \frac{\mu^2}{2 U}$, which describe a parabola centered at $- \mu / U$. The ground-state occupation $N > 0$ minimizing the energy obeys
% \begin{align}
% \label{eq:gsOccup}
%  N - \oh <\frac{\mu}{U}   <  N + \oh 
% \end{align}
% which is the same condition obeyed by the steady-state \eqref{eq:neqPopCond}.
% The ground-state single-site susceptibility can be easily computed through a spectral decomposition and reads
% \beq
% \label{eq:gsRet}
% G_{0,\rm gs}^R (\w)  =  \frac{N +1}{ {\w }- \w_{\rm doub}^{\rm gs} + i \eta } - \frac{N }{{\w }- \w_{\rm hol}^{\rm gs} + i \eta }
% \eeq
% where $\eta$ is an infinitesimal.
% Like its steady-state counterpart plotted in Fig. \ref{fig:dmft}, it has two peaks at the energies of doublons $\w_{\rm doub}^{\rm gs}$ and holon $\w_{\rm hol}^{\rm gs}$ excitations (corresponding to adding and removing a particle from the ground-state), which are given by 
% %The real part of two poles of \eqref{eq:gsRet} are respectively centered at 
% \begin{align}
% \label{eq:gsExc}
% \pm \w_{\rm doub(hol)}^{\rm gs} &= \pm (E_{N \pm 1} - E_{N } ) = - {\mu } + U\lp N  \pm \oh \rp 
% \end{align}
% Going to a rotating frame in which the chemical potential is removed from the Hamiltonian, like in the main-text Hamiltonian \eqref{eq:hamBh}, these energies coincide with the respective steady-state ones reported in the main text \eqref{eq:doubEn}.

% The ground-state Gutzwiller mean-field phase diagram is given by the main text critical point equations \eqref{eq:neqCritHop}, \eqref{eq:neqCritFreq}, where the steady-state susceptibility is  replaced with the ground-state one. 
% Since the ground-state transition is static, those equations must be evaluated at $\w_{c} = 0$ with the susceptibility \eqref{eq:gsRet} (or equivalently at $\w_c=\mu$ in the rotating frame of \eqref{eq:hamBh}).
% Note that \eqref{eq:neqCritFreq} is always satisfied at $\w_c=0$ for equilibrium states such as ground-states (see e.g. \cite{scarlatellaSchiro2019a}). The critical hopping is then given by
% $ 1/ J_{\rm c}  = - \re G^R_{0,\rm gs}(0) $ and replacing the susceptibility from \eqref{eq:gsRet}, one obtains
% \beq
% \label{eq:gsCrit}
% J_{\rm c} = \frac{\lp U/2 \rp ^2 - \lp U N  -\mu  \rp ^2 }{\mu + U/2}
% \eeq
% yielding the well known Mott lobes in the $\mu-J$ plane, plotted in Fig. \ref{fig:phaseDiag} with $\mu =\mu_{\rm eff}$ for comparison with the steady-state phase boundary.


%\subsection{The Redfield dissipation for the stabilizer}
%
%Considering a Bose-Hubbard Hamiltonian coupled to a stabilizer reservoir via $\hat{H}_{\rm sys, stab}= \sum_i ( \hat{a}_i^\da \hat{B} + \hat{a}_i \hat{B}^\da )$, where $\hat{B}$ ($\hat{B}^\da$)  annihilates (creates) excitations in the stabilizer and taking $C^{-+} (\tau) = 0 $ as in the main text, where  $C^{-+} (\tau) = -i \aver{\hat{B}(\tau) \hat{B}^\da}$ and $C^{+-} (\tau)=-i \aver{\hat{B}^\da(\tau) \hat{B}}$ are the reservoir correlation functions, one obtains the Redfield dissipator for the stabilizer \cite{mozgunovLidar2020}
%\beq
%\hat{\mathcal{D}}_{\rm stab} = {r \kappa} \sum_i \lp a_i^\da \rho \tilde{a}_i  + \tilde{a}_i^\da \rho a_i - a_i \tilde{a}_i^\da \rho -\rho \tilde{a}_i a_i^\da \rp
%\eeq
%where 
%\beq
%\tilde{a}_i =\int_{-\infty}^{\infty} d \tau C^R_{+-}(\tau) a_i(-\tau) 
%\eeq
%defining the retarded function $C_{+-}^R (t) = C_{+-}(t) \theta(t)$. 
%The ``filtered operator'' $\tilde{a}_i$ can be decomposed in the eigenbasis of the Bose-Hubbard Hamiltonian with eigenvectors $\ket{\psi_m}$ and eigenvalues $E_m$ yielding 
%\beq
%\label{eq:stabDiss_bh}
% \tilde{a}_i = \sum_{m,n} S^R(E_n-E_m) \bra{\psi_m} a_i \ket{\psi_n} \ket{\psi_m} \bra{\psi_n}
%\eeq
%where the Fourier transform of the retarded function is defined by $S_{+-}^R(\omega)= \int_{-\infty} ^{\infty} d t C_{+-}^R(t) e^{i \omega t} = \int_{0} ^{\infty} d t C_{+-}(t) e^{i \omega t} $. 
%The following relation holds between the Fourier transforms of retarded and non-retarded functions
%\beq 
%S_{+-}^R(\omega)=\frac{1}{2} S_{+-}(\omega)-i \mathcal{P} \int_{-\infty}^{\infty} \frac{d \omega^{\prime}}{2 \pi} \frac{S_{+-}\left(\omega^{\prime}\right)}{\omega^{\prime}-\omega}
%\eeq
%where $S_{+-}(\omega)= \int_{-\infty} ^{\infty} d t C_{+-}(t) e^{i \omega t} $.
%
%For the single-site problem, the eigenstates of the Hamiltonian are the number eigenstates and the expression for $\tilde{a}_i$ reduces to that of the main text. 
%We remark that in this work we never evaluate the dissipator \eqref{eq:stabDiss_bh} for the many-body problem because our mean-field steady-state treatment reduces to solving the master equation of the single-site problem. 

\section{Equations for the Mott phase instability}\label{secB}

Here we discuss how to obtain the critical point equations \eqref{eq:neqCritFreq}\eqref{eq:neqCritHop} for the phase transition out of the Mott phase. 
We first discuss how to obtain it from the Gutzwiller dynamics and then using the strong-coupling RPA Keldysh field theory. Finally, we discuss how these equations are modified within Dynamical Mean-Field Theory.

\subsection{Time-dependent Gutzwiller}

Within the Gutzwiller ansaz made in the main text, the linear response to a small symmetry-breaking field $\phi(t)$ reads
\begin{equation}
\aver{a(t)} = \int_{\infty}^{-\infty} d\tau G_0^R(t-\tau)\phi(\tau) =-J
\int_{\infty}^{-\infty} d\tau G_0^R(t-\tau)
\aver{a(\tau)}
\end{equation}
where in the last step we have used the Gutzwiller self-consistency condition  $\phi(t) = - J\aver{a(t)}$. Translating the above condition in frequency domain we obtain $ \aver{a(\w)} = -J G_0^R(\omega )\aver{a(\w)}$.
For a given $\w$ such that $a(\w) \neq 0$, this equation gives the critical point equations \eqref{eq:neqCritFreq}, \eqref{eq:neqCritHop}: 
\begin{equation}
1/J + G_0^R(\omega ) = 0
\end{equation}
The smallest $J$ and the corresponding $\w$ satisfying this condition define the critical point $(J_c,\w_c)$. 

The same condition can be recovered in a strong-coupling Keldysh field theory \cite{senguptaDupuis2005b}, as we show in the following. %, where we also show that the unstable mode has wavevector $q=0$, while in our Gutzwiller ansatz we had already assumed that this is the case.

%linear response theory in the steady state and in the early broken-symmetry phase such that $\phi(t) = - J\aver{a(t)}$ is small, gives 
%$ \aver{a(t)} = \int_{\infty}^{-\infty} d\tau G_0^R(t-\tau)\phi(\tau)$, where $G_0(t)$ is the susceptibility computed from the single-site Hamiltonian, which in frequency domain and using $\phi(t) = - J\aver{a(t)}$ becomes $ \aver{a(\w)} = -J G_0^R(\omega )\aver{a(\w)}$.



\subsection{Strong-coupling RPA in the Keldysh path integral}

We make a strong-coupling RPA (random phase approximation) \cite{scarlatellaSchiro2019,senguptaDupuis2005b}, formulating the problem in the language of Keldysh field theory. The Keldysh action, in terms of the coherent fields $a_i,\bar{a}_i$ reads
\begin{align}
\label{eq:hbDrivenDissAction}
\es{
S &= \int_\mathcal{C} dt \lp \sum_i \bar{a}_i  i \pt a_i - H \rp + \sum_i \lp S_{l,i} + S_{\mu_{\rm eff},i} \rp 
} 
\end{align}
where $\int_\mathcal{C}$ is an integral on the Keldysh contour, $H$ is the expectation value of the Hamiltonian on coherent states 
\beq
H = \sum_i\left( \omega_0 \bar{a}_i a_i +\frac{U}{2} \bar{a}_i \bar{a}_i a_i a_i \right) -  \sum_{\aver{ij}} \frac{J}{z} \, \lp \bar{a}_i  a_j + \hc \rp 
\eeq
and $S_{l,i}$ describes the Markovian losses, corresponding to the loss dissipator %\eqref{eq:mbDiss} 
\beq
\label{eq:1pLossKel}
\es{S_{l,i} &= -i \kappa \int_{-\infty}^{\infty} dt \,  \lp \bar{a}_{i-} a_{i+} -\oh  \bar{a}_{i+} a_{i+} -\oh  \bar{a}_{i-} a_{i-} \rp 
}
\eeq
Finally, $S_{\mu_{\rm eff},i}$ describes the coupling to the structured reservoir. For a bath of non-interacting bosons this can be integrated out explicitly, yielding 
\beq
\label{eq:nmBathKel}
 S_{\mu_{\rm eff},i} =-i  \int_\mathcal{C} dt \int_\mathcal{C} dt' r \kappa \sum_i \bar{a}_i(t) C(t-t') a_i(t')
\eeq
where $C(t-t')$ is the bath correlation function, 
% defined in the main text \eqref{eq:lesserBox}, 
with Keldysh indices implicit in the time variables. 
The fourier transform of the lesser component of this function is $S(\w)$ defined in the main text \eqref{eq:lesserBox}.

% and greater components of $C(t-t')$ are defined in the main text through its Fourier transform $S^{-+}(\w),S^{+-}(\w)$. 


%
%\subsection{Hubbard-Stratonovich transformation}
%\label{sec:strongCoup}
%We now introduce a general strong-coupling approach to study driven-dissipative correlated lattice models, which generalizes the equilibrium approach of Ref~\cite{fisherFisherPRB1989} (see also \cite{Sengupta2005,Koch_LeHur_2009,Sachdev}). 
%We write down the Keldysh partition function 
%\beq
%Z = \int \prod_{i} \bold{D}[  \bar{a}_i ,  a_i ] e^{i S[ \lbr \bar{a}_i , a_i \rbr ]  }
%\eeq
%where by $ \lbr \bar{a}_i , a_i \rbr $ we mean the set of bosonic coherent fields of all sites. 
%We rewrite the effective action in the form
%\begin{align}
%S &= S_{loc} - \int_\mathcal{C} dt  \sum_{ij}  \,  \bar{a}_i J_{ij} a_j \\ 
%\label{eq:localAction}
%S_{loc} &= \sum_i \lp S_{u,i} + S_{l,i} + S_{\sigma,i} \rp
%\end{align}
%where $S_{loc}$ describes decoupled sites, containing the term $S_{u,i}$ describing unitary evolution and the dissipative contributions $S_{l,i}$, $S_{\sigma,i}$. 
%The only term coupling different sites is the hopping term, where $J_{ij}$ is the hopping matrix, being equal to $-J$ for nearest neighbours sites and zero otherwise. 
%%\begin{align}
%%\label{eq:hbDrivenDissAction}
%%S &= S_{loc} + \int_\mathcal{C} dt  \sum_{\aver{ij}} J \, \lp \bar{a}_i  a_j + \hc \rp\\ 
%%\es{
%%S_{loc} &= \int_\mathcal{C} dt \lsq \sum_i \bar{a}_i  i \pt a_i - \sum_i\left(\delta \omega_0 \bar{a}_i a_i +\frac{U}{2} \bar{a}_i \bar{a}_i a_i a_i \right) \rsq +  \\
%%&-i \kappa \int_{-\infty}^{\infty} dt \, \lp \bar{a}_{i+} a_{i-} -\oh  \bar{a}_{i+} a_{i+} -\oh  \bar{a}_{i-} a_{i-} \rp + S_{\sigma}
%%}
%%\end{align}

We then do a Hubbard-Stratonovich transformation on the hopping term, by introducing the auxiliary bosonic fields $\psi_i$ by a Gaussian integral 
\beq
\exp \lp { {-} i \int_\mathcal{C} dt \sum_{ij} \bar{a}_i J_{ij} a_j} \rp =\frac{1}{\mathcal{N}} \int \prod_{i} \bold{D} \lsq \bar{\psi}_i {\psi}_i \rsq \exp{\lbr i \int_\mathcal{C} dt  \lsq \sum_{ij} \bar{\psi}_i J_{ij}^\mo \psi_j  + \sum_i \lp  \bar{\psi}_i a_i + \psi_i \bar{a}_i \rp \rsq \rbr} 
\eeq
%
%
%We then decouple the hopping term through a Hubbard-Stratonovich transformation, by introducing an auxiliary bosonic field $\psi_i$ for each site. This transformation 
%\beq
%\exp \lp { {-} i \int_\mathcal{C} dt \sum_{ij} \bar{a}_i J_{ij} a_j} \rp =\frac{1}{\mathcal{N}} \int \prod_{i} \bold{D} \lsq \bar{\psi}_i {\psi}_i \rsq \exp{\lbr i \int_\mathcal{C} dt  \lsq \sum_{ij} \bar{\psi}_i J_{ij}^\mo \psi_j  + \sum_i \lp  \bar{\psi}_i a_i + \psi_i \bar{a}_i \rp \rsq \rbr} 
%\eeq
where $J_{ij}^\mo$ is the inverse hopping matrix and
$\mathcal{N}$ is a normalization coming from Gaussian integration.
%The field $\psi_i$ plays the role of a local order parameter since, for small $\aver{a_i}$, $\aver{\psi_i}$ is linearly related to $\aver{a_i}$ \cite{fisherFisherPRB1989}.
By plugging this in the action \eqref{eq:hbDrivenDissAction}, we can formally integrate on the $a_i,\bar{a}_i$ fields 
%\beq
%\es{
%Z &= \int \prod_{i} \bold{D}[  \bar{a}_i ,  a_i ] e^{i S[ \lbr \bar{a}_i , a_i \rbr ]  } = \\
%&= \frac{1}{\mathcal{N}} \int \prod_{i} \bold{D}[ \bar{\psi}_i, {\psi}_i ] e^{-i \int_\mathcal{C} dt  \sum_{ij} \bar{\psi}_i J_{ij}^\mo \psi_j } \int \prod_{i} \bold{D}[  \bar{a}_i ,  a_i ] e^{i S_{loc} }  e^{ i \int_\mathcal{C} \sum_i \lp  \bar{\psi}_i a_i + \psi_i \bar{a}_i \rp  } 
%}
%\eeq
\beq
\es{
Z &= \int \prod_{i} \bold{D}[  \bar{a}_i ,  a_i ] e^{i S[ \lbr \bar{a}_i , a_i \rbr ]  } = \\
&= \frac{1}{\mathcal{N}} \int \prod_{i} \bold{D}[ \bar{\psi}_i, {\psi}_i ] e^{ i \int_\mathcal{C} dt  \sum_{ij} \bar{\psi}_i J_{ij}^\mo \psi_j } \int \prod_{i} \bold{D}[  \bar{a}_i ,  a_i ] e^{i S_{loc} }  e^{ i \int_\mathcal{C} \sum_i \lp  \bar{\psi}_i a_i + \psi_i \bar{a}_i \rp  } \\ 
&= \frac{1}{\mathcal{N}} \int \prod_{i} \bold{D}[ \bar{\psi}_i, {\psi}_i ] e^{i S_{\rm eff}  [ \lbr \bar{\psi}_i , \psi_i \rbr ] }
}
\eeq
obtaining the effective action for the fields $\psi_i, \bar{\psi}_i$ alone
\beq \label{eqn:Seff}
\mathcal{S}_{\eff}= \int_\mathcal{C} dt \lp \sum_{ij}\bar{\psi}_i J_{ij}^{-1}\psi_j+\sum_i\Gamma[\bar{\psi} _i,\psi_i] \rp
\eeq
where the second term represents the generating functional of the bosonic Green functions of isolated sites, $\Gamma[\bar{\psi}_i,\psi_i]= - i \log\langle T_C e^{i\int_\mathcal{C} dt\left(\bar{\psi}_i a_i+a^\da_i \psi_i\right)}\rangle_{0}$.

We stress that the latter average is taken on the single-site problem, therefore the many-body problem has been formally reduced to calculating the single-site problem cumulants. 
We remark that only at this stage we describe the reservoir with a Lindblad equation \eqref{eq:mbME}, that therefore only depends on the spectrum of the single-site problem and can be evaluated in practice. 

To obtain the strong-coupling RPA, we then truncate the effective action at the Gaussian level obtaining 
\beq \label{eqn:Seff}
S_{\eff}= \int_\mathcal{C} dt  \int_\mathcal{C} dt' \sum_{ij}   \bar{\psi}_i  \lp J_{ij}^{-1}+G_0(t-t') \rp \psi_j
\eeq 
where $G_0 (t-t') = - i \aver{ T_\mathcal{C}  a_i(t) a_i^\da(t') } $ is the contour-ordered Green function of the single-site problem.
A second-order phase transition is then signalled by a vanishing retarded component of the effective action, corresponding to a diverging susceptibility at the critical point, which in frequency and momentum space reads 
$ 0=1/J_q-G^R_{0}(\omega)$ with $J_q$ the lattice dispersion.
For a hyper-cubic lattice $J_q=-2{J}/{z}\sum_{\alpha=1}^d\cos q_{\alpha}$ and the first unstable mode is the $q=0$ mode (assuming $\re {G^R_{loc}(\w_c)}<0$), leading to the critical point equations \eqref{eq:neqCritFreq} \eqref{eq:neqCritHop}  of the main text: 
\begin{align}
 0 &= \im G_0^R(\omega_c) \\
\frac{1}{ J_c} &= - \re G_0^R(\omega_c) 
\end{align}

\subsection{Dynamical Mean-Field Theory}
\label{eq:dmft_eq}

%A similar procedure is used in \cite{scarlatellaSchiro2021} to derive the DMFT critical point equations reported in \ref{eq:dmft_eq}. 

The phase boundary condition can be obtained within DMFT using a similar procedure, as discussed in detail in Ref.~\cite{scarlatellaSchiro2021}. The key difference with respect to Gutzwiller/RPA is that the self-consistent symmetry breaking field takes contributions both from the coherent neighboring sites, as in Gutzwiller, as well as from the incoherent neighbors through the self-consistent DMFT bath. Assuming a Bethe lattice, the critical point equations \eqref{eq:neqCritFreq} \eqref{eq:neqCritHop} becomes in DMFT \cite{scarlatellaSchiro2021} 
\begin{align}
\label{eq:critFreqDmft}
\operatorname{Im} G^R\left(\omega_c, J_c\right)=0 \\ 
\label{eq:critHopDmft}
\frac{1}{J_c}+\operatorname{Re} G^R\left(\omega_c, J_c\right)+\frac{J_c}{z}\left[\operatorname{Re} G^R\left(\omega_c, J_c\right)\right]^2=0 
\end{align}
in terms of the local susceptibility $G^R(\w,J) = -i \int_{0}^\infty dt e^{i \w t } \aver{[ \hat{a}(t),\hat{a}^\da(0) ]}$.
We remark that the critical frequency $\w_c$ is still the zero of its imaginary part.
On the other hand, the local susceptibility here depends on the hopping and therefore the two equations are coupled, differently from the equations \eqref{eq:neqCritFreq},\eqref{eq:neqCritHop} where we could determine $\w_c$ from the first equation alone.  


%
%\subsection{Beyond mean-field: DMFT calculations}
%
%In order to validate our RPA predictions, we go beyond using the dynamical mean-field theory approach of REF, considering a Bethe lattice and relying on an impurity solver based on the non-crossing approximation (NCA) REF.
%
%The finite-frequency instability of the dissipatively-stabilized Mott phase is still found in this approximation, as we show here for a single value of $\mu$. The calculation of the entire phase diagram in DMFT requires a sensible numerical effort and is left for future work. 
%The critical point equations are given by 
%\begin{equation}
%\begin{gathered}
%\operatorname{Im} G^R\left(\omega_c, J_c\right)=0 \\
%\frac{1}{J_c}+\operatorname{Re} G^R\left(\omega_c, J_c\right)+\frac{J_c}{z}\left[\operatorname{Re} G^R\left(\omega_c, J_c\right)\right]^2=0 
%\end{gathered}
%\end{equation}
%in terms of the retarded Green function. We remark that in DMFT this depends on the hopping and therefore the two equations are coupled, differently from the mean-field equations REF, where we could determine $\w_c$ from the first equation alone.  
%
%In Fig. \ref{fig:dmft} we compare the spectral function $A(\w,J) = -1/\pi \im G^R(\w,J)$ in mean-field and in DMFT, were DMFT captures the broadening of the peak corresponding to a single-particle excitation due to the hybridization between different sites forming the upper Hubbard band. We notice that in DMFT the critical frequency $\w_c$ arises right at the bottom of the upper Hubbard band, similarly to the mean-field prediction, despite the significant change of the shape of the spectral function (redistribution of spectral weight) that was found to sensibly change the critical frequency for a different model REF. 
%For the parameters of Fig. \ref{fig:dmft}, in DMFT the critical hopping is found to be $J_{c,\rm dmft}/U = 0.079$, slightly larger than the mean-field prediction $J_{c, \rm mf}/U = 0.066$, in agreement with the expectation that quantum fluctuations favour the incoherent phase, yet much smaller than the ground state transition at $J_{c,g}/U = 0.166$. 
%
%
%\begin{figure}[H]
%\centering
%\includegraphics[width=0.48\linewidth]{./specFunc_mfDmft.png}     
%\caption{Spectral function for $r=20, \kappa/U = 0.05, \sigma/U=3, \mu/U=1$ in mean-field and DMFT on the Bethe lattice with coordination number $z=6$. In DMFT the spectral function depends on the hopping and is evaluated at the critical hopping $J_{c,\rm dmft}/U = 0.079$ (while the critical hopping predicted by mean-field is $J_{c, \rm mf}/U = 0.066$). The critical frequency in both approximations is shown in the plot.}
%\label{fig:dmft}
%\end{figure}


%\subsection{others}
%
%\begin{figure}[H]
%\centering
%\includegraphics[width=0.48\linewidth]{./jcVsK.png}   
%\caption{Critical hopping $J_c$ as a function of $\kappa/U$ at fixed $r$. For small enough $\kappa$, $J_c$ reaches a constant value. Parameters: $\mu_s/U = 1$, $k/U = 10^{-6}$, $\sigma/U=3.5$, $r=10^2$ in bottom plot, $k/U=10^{-6}$ in top plots. }
%\label{fig:}
%\end{figure}


%\cmm{Finally, the energies to create a particle or a hole excitation in the steady-state also approximate the ground state ones (at zeroth order in the dissipation rate): 
%\beq
%\label{eq:pHExc}
%\w_{p(h)} \simeq E_{n\pm1} - E_n =\pm \lsq \sigma -\mu + U\lp n \pm \oh \rp \rsq
%\eeq
%where there's only a shift of $\sigma$ with respect to the ground state.
%Thus in a rotating frame at frequency $\sigma$ (where the $\sigma$ shift is removed) both the steady-state and its excitations approximate the ground-state ones, for $r \gg 1$ and $\kappa/U,r \kappa/U \ll 1$. }
%



\section{Reservoir with Lorentzian correlations}
\label{secC}
% \textcolor{red}{the results here are obtained with Lindblad or Redfield?}
\begin{figure}
\centering
%\includegraphics[width=0.28\linewidth]{./nVsR_lor.png} 
%\includegraphics[width=0.28\linewidth]{./purVsR_lor.png}  \\
%\includegraphics[width=0.30\linewidth]{./jcVsR_lor.png}  
%\includegraphics[width=0.30\linewidth]{./wcVsR_lor.png} 
\includegraphics[width=0.6\linewidth]{./lor_rwa.png}  
\caption{For a reservoir with Lorentzian spectral function \eqref{eq:lorCorr}, the driven-dissipative single-site steady-state average population $\aver{n}$, purity $\tr(\rho^2)$, critical hopping $J_c$ and frequency $\omega_c$ as a function of the pump-to-loss ratio $r$ (the doublon frequency is indicated by $\w_p$ here). 
The same qualitative behaviour found for the square-shaped reservoir spectral function of the main text is found. 
Parameters: $k/U = 10^{-6}$, $\gamma/U=10^{-3}$, $\omega_{\rm stab}/U = 1$.
}
\label{fig:jcWcLor}
\end{figure}

%\begin{figure}
%\centering
%\includegraphics[width=0.28\linewidth]{./jcVsR_lor.png}  
%\includegraphics[width=0.28\linewidth]{./wcVsR_lor.png}  
%\caption{Critical hopping and frequency as a function of the pump-to-loss ratio for Lorentzian correlations of the stabilizer. $J_c$ vanishes linearly, while $\omega_c$ approaches the energy to create a particle excitation $\w_{\rm doub} = \w_0 + 3U/2$. Parameters: $\w_0/U =0.1$, $k/U = 10^{-6}$, $\omega_{\rm stab}/U = 0.6$, $\sigma/U=0.01$. }
%\label{fig:jcWcLor}
%\end{figure}

In the main text we considered a structured reservoir with a simple square-shaped correlation function  \eqref{eq:lesserBox}, acting as a chemical potential. 
Here we show that our conclusions do not depend on this specific choice, by considering instead a reservoir with a Lorentzian spectral function, which is qualitatively more similar to the experimental realization of Ref. \cite{maSchuster2019}. 
We assume the Lorentzian to be centred at $\w_{\rm res} = U$ corresponding to the transition from 0 to 1 photon in a Bose-Hubbard site with energy difference $E_1 - E_0 = U$ and to have a lifetime $\gamma$ (corresponding to the lifetime of reservoir excitations): 
%$ C_{+-}(\tau) = r\kappa \frac{\gamma}{2} e^{ ( -i \w_{\rm res} -\gamma/2)\tau } $ leading to the frequency domain expression
%\beq
%S_{+-}^R(\w) = r\kappa \frac{(-\gamma/2)}{i(\w-\w_{\rm res}) - \gamma/2}
%\eeq
\beq
\label{eq:lorCorr}
S^R(\w) =  r\kappa \frac{(\gamma/2)^2}{(\w-\w_{\rm res})^2 + (\gamma/2)^2}
\eeq
%whose real part is a Lorentzian function.



%The square-shaped correlation function of the stabilizer is in fact also challenging to realize in experiments: in \cite{lebreuillyCarusotto2017} a realization is proposed in which each site is coupled to a collection of 2-level atoms, inverted by an incoherent pump and with uniformly distributed energy splittings between the two levels in $[0, \w_0 + \mu_{\rm eff}]$ (in the lab frame) or with a single atom whose splitting energy is rapidly modulated in time. 
%
%In \cite{maSchuster2019} the stabilizer rather is a source of photons at a fixed energy $\w_{\rm res} = U/2$ corresponding to the transition $E_1 - E_0 = U/2$ of a Bose-Hubbard site. 

Fig. \ref{fig:jcWcLor} shows in panels (a-b) that upon increasing the pump/loss ratio the single-site steady-state approximates the Bose-Hubbard-site ground state with unit filling, and in panels (c-d) that we get a similar instability of the steady-state Mott phase with a similar behaviour of $\w_c$ and $J_c$ as shown in Fig. \ref{fig:phaseDiag} (d-e). 


We also remark that the Markovian assumption, requiring that the timescale for the decay of the bath spectral function is shorter than the bath-induced system timescales, is strictly speaking not satisfied for the square-shaped correlation functions of the reservoir used in the main text \cite{lebreuillyCarusotto2017}. For the Lorentzian function used here instead the Markovian assumption is justified for our choice of parameters satisfying $\kappa, r \kappa \ll \gamma, \omega_{\rm res}$, showing that our conclusions are not affected.


\section{Redfield master equation}
\label{secD}


\begin{figure}
\centering
\includegraphics[width=0.6\linewidth]{./phaseDiag_red_nnm1}   
\caption{Steady-state Mott instability using the Lindblad \eqref{eq:mbME} versus Redfield \eqref{eq:redME} equation for the same parameters in Fig. \ref{fig:phaseDiag}. Left: steady-state phase diagram, where the ground-state Mott-superfluid phase diagram is plotted for comparison.
Right: critical hopping $J_c$ and frequency difference $\w_{\rm doub} - \w_c$ as a function of the inverse pump/loss ratio.
For the same parameters, the critical hopping is smaller in the Redfield case, as a feature of the doublon resonance that is crucial for the critical point appears at first order in perturbation theory in the dissipation strength in the Redfield case, while only at higher orders in the Lindblad case.
}
\label{fig:phaseDiag_red}
\end{figure}

In Ref. \cite{lebreuillyCarusotto2017} the same system considered in this paper is modelled using a Redfield master equation, rather than the Lindblad equation \eqref{eq:mbME}, which includes non-secular terms and Lamb-shift contributions to the Hamiltonian. 
In this appendix we show that our conclusions do not change considering such a Redfield equation. This reads
\beq
\partial_t \hat{\rho}=-i[\hat{H}, \hat{\rho}] + \kappa \sum_{j} \mathcal{D} [\hat{a}_{j}] \hat{\rho}  + \hat{\mathcal{D}}_{\rm stab}
\label{eq:redME}
\eeq
The structured reservoir dissipator in this case reads
%where $S(\omega)= \int_{-\infty}^{\infty} d t C_{+-}(t) e^{i \omega t}$, and $\mu_{\rm eff}$ represents the maximum energy of reservoir excitations (in the rotating frame of the Hamiltonian \eqref{eq:hamBh}). 
%The stabilizer dissipator then reads
\beq
\label{eq:mbDiss_red}
\hat{\mathcal{D}}_{\rm stab} = {r \kappa} \sum_i \lp a_i^\da \rho \tilde{a}_i  + \tilde{a}_i^\da \rho a_i - a_i \tilde{a}_i^\da \rho -\rho \tilde{a}_i a_i^\da \rp
\eeq
where the ``filtered'' operators $\tilde{a}_i = \sum_{m,n} S^R(\epsilon_n-\epsilon_m) \bra{\psi_m} a_i \ket{\psi_n} \ket{\psi_m} \bra{\psi_n}
$ are defined in terms of the eigenstates $\ket{\psi_n}$ and eigenvalues $\epsilon_n$ of the Bose-Hubbard Hamiltonian \eqref{eq:hamBh} and 
\beq
\label{eq:lesserBox_red}
S^R(\omega)  = \frac{1 }{2}  \theta(\mu_{\rm eff} - {\w}) \theta( \w +\w_0) +  \frac{i}{2\pi} \log \abs{\frac{\mu_{\rm eff} - \w}{\w_0+\w}} 
\eeq
whose imaginary part leads to a Lamb-shift term contributing to the Hamiltonian which is negligible in our results (for which the dissipation is small and staying away from the points in which this function is singular), but we kept it in the results of this appendix. 
%This is a good description as long as the coupling with the system is weak and the Markovian approximation holds.

Using such a Redfield equation \eqref{eq:redME}, the main results presented in the main text are recovered in Fig. \ref{fig:phaseDiag_red}, showing a similar phase diagram and a similar behaviour of $J_c$ and $\w_c$ at small $1/r$ as in Fig. \ref{fig:phaseDiag}.
The main difference is that the critical hopping is much smaller than in the Lindblad case and the exponent with which the critical hopping $J_c$ and the frequency difference $\w_{\rm doub}-\w_c$ decrease as a power law with $1/r$ is larger in absolute value compared to the Lindblad case.


In sections \ref{sec:pertRed} and \ref{sec:pertLind} we report perturbative calculations showing that this quantitative difference is due to the fact that a crucial contribution to the doublon resonance, that mainly determines the critical point, appears at first order in perturbation theory in the dissipation strength in the Redfield case, while for the Lindblad equation it only appears at higher orders in perturbation theory. 

%Developing this perturbative approach, also allows to better understand how the critical frequency solving Eq. \eqref{eq:neqCritFreq} arises, and why it approaches the doublon energy $\w_{\rm doub}$ at large $r$. 

We also remark that the Gutzwiller mean-field dynamics using the Redfield equation becomes unphysical in the limit-cycle phase, yielding negative probabilities, contrarily to the case of the Lindblad equation \eqref{eq:mbME} considered in the main text.






\section{The doublon resonance and the steady-state instability}
\label{secE}


The observation that a zero of the imaginary part of the single-site susceptibility $G_0^R(\w)$ forms close to its doublon peak (see the discussion of Fig. \ref{fig:dmft} (a)) allows to better understand the mathematical origin of the steady-state Mott instability. 
We find that for $\kappa , \kappa r\ll U,J$, where the peaks are well resolved, such a zero emerges due to an ``anti-Lorentzian'' contribution to $\im G_0^R(\w)$ \cite{scarlatellaSchiro2019a} making the doublon peak asymmetric, as Fig. \ref{fig:retGreen} shows plotting $ G_0^R(\w)$ in the case of the Redfield equation \eqref{eq:mbDiss_red} for which this behaviour is particularly pronounced (though for a Lindblad equation the same conclusions are true): in zoom (a) on the doublon resonance the imaginary part is clearly asymmetric, while the same asymmetry is not present in zoom (b) showing the holon resonance, whose imaginary part is almost even around the peak center. 
%This contribution arises from the dissipative contributions to the eigenstates of the Liouvillian, that is proportional to $r$ assumed to be large.  


In the following, we discuss this anti-Lorentzian contribution and how it can lead to the steady-state instability, while in Sec. \ref{eq:mbDiss_red} we use first-order perturbation theory in the Redfield dissipator to capture this contribution analytically and in Sec. \ref{sec:pertLind} we show that the same effect is not captured at first-order in perturbation theory in the Lindblad case (but appears at higher orders), explaining why the instability is somehow less pronounced in this case.



%In the case of the Redfield dissipator \eqref{eq:mbDiss_red}, we use first-order perturbation theory in the dissipators to confirm this with an analytical calculation, showing that only the doublon peak builds such an large anti-Lorentzian structure, due to the strong frequency-dependent structure of the reservoir correlations. 
%
%Interestingly, in Sec. \ref{sec:pertLind} we repeat the the same feature does not appear at first-order in perturbation theory starting from the Lindblad dissipator \eqref{eq:mbDiss}, thus the weaker behaviour four in the Lindblad case is accounted from the fact that occurs at higher order in perturbation theory. 

\begin{figure}[b]
\centering
\includegraphics[width=0.65\linewidth]{./ret_pert_nnm1}    
\caption{solid lines: single-site susceptibility for $r=10^3$ for the Redfield master equation \eqref{eq:redME} and for $\mu_{\rm eff}/U \in [0,1], \kappa/U = 10^{-5}$, obtained numerically. (a-b) zooms respectively around the right dobulon and left holon peak.
The doublon peak is strongly asymmetric due to an “anti-Lorentzian” contribution whose magnitude increases with the pump/loss ratio $r$.
The dotted lines are obtained using the first order perturbation theory in the dissipator, where the doublon peak is approximated by keeping only the corresponding term \eqref{eq:signPeak} in the Lehmann representation, showing that the zero of the imaginary part is correctly captured.
}
\label{fig:retGreen}
\end{figure}

%The retarded Green function of the single-site problem is plotted in Fig. \ref{fig:retGreen}. 
% For large $r \gg 1$, we find that the zero of  $\im G_0^R(\w)$, the critical frequency $\w_c$ of the transition \eqref{eq:neqCritFreq}, is close to the energy $\w_{\rm doub}$ to add a particle to the steady-state, as shown also in Fig. \ref{fig:retGreen}. 
% The zoom in panel (a) shows that the zero is due to an ``anti-resonance'' appearing close to $\w_{\rm doub}$, which is not present in correspondence of other peaks, as shown in panel (b). In the following we investigate how this feature originates.

The retarded Green function (susceptibility) 
$G_0^R(t) = -i \tr \lbr  [ \hat{a}(t),\hat{a}^\da(0) ] \hat{\rho} \rbr \theta(t) $
of a problem described by a Markovian master equation $\pt \hat{\rho} = \hat{\mathcal{L}} \hat{\rho}$ can be decomposed in terms of the right $\hat{r}_\alpha$ and left $\hat{l}_\alpha$ eigenstates of the Lindblad superoperator $\hat{\mathcal{L}}$ and its eigenvalues $\lambda_\alpha$ in its Lehmann representation (see e.g. \cite{scarlatellaSchiro2019a}) 
\begin{equation}
\label{eq:retOpenSys}
G^R(\omega)=\sum_\alpha \frac{w_\alpha}{\omega+\operatorname{Im} \lambda_\alpha-\mathrm{iRe} \lambda_\alpha}
\end{equation}
with $w_\alpha=\operatorname{tr}\left(\hat{a} \hat{r}_\alpha\right) \operatorname{tr}\left(\hat{l}_\alpha^{\dagger}\left[\hat{a}^{\dagger}, \hat{\rho}\right]\right)$. 
It is important to notice that the weights $w_\alpha$ are in general complex, differently from the case of closed systems. 
%This structure allows to relate the critical hopping and frequency in RPA. 
%We first need to find the zeros of the imaginary part of the retarded Green, corresponding to the critical frequency. 
The imaginary part takes the form
\beq
\label{eq:imGr}
\im G^R(\w) = \sum_\alpha \frac{\re w_\alpha \re \lambda_\alpha}{(\w + \im \lambda_\alpha)^2 + (\re\lambda_\alpha)^2} + \frac{\im w_\alpha}{\re \lambda_\alpha } \frac{\re \lambda_\alpha (\w + \im \lambda_\alpha)}{(\w + \im \lambda_\alpha)^2 + (\re\lambda_\alpha)^2}
\eeq
with both Lorentzian and ``anti-Lorentzian'' contributions corresponding to the first and second term. 
Note that in the limit of vanishing dissipation, where $\re \lambda_\alpha \rw 0$ and $\im w_\alpha \rw 0$, the amplitude of the anti-Lorentzian contribution $\im w_\alpha / \re \lambda_\alpha$ can still be finite. 

An important observation is that, in the regime of small dissipation $\kappa, \kappa r \ll U,J$ considered in the paper, the peaks stemming from different contributions in the sum are well separated in frequency, and therefore both the zero of the imaginary part and the critical hopping mostly originate solely from a single peak, the doublon peak (as we show in Fig. \ref{fig:retGreen}. This can be parametrized by 
\beq
f(\w) =b \frac{1 - i \gamma a}{\w + i \gamma} 
\eeq
with imaginary part 
\beq
\im f(\w) = -b \lp \frac{\gamma}{\w^2 +\gamma^2} + a \frac{\w \gamma}{\w^2 +\gamma^2} \rp
\eeq
The latter has a zero at $\w = -1/a$, that we identify with the deviation of the critical frequency from the doublon energy $ -1/a \approx \w_c - \w_{\rm doub} $. 
The critical hopping is then given by $J_c = -1/\re G^R(\w_c)$, that is by the real part of the $f(\w)$
\beq
\re f(\w) = b \lp \frac{\w }{\w^2 +\gamma^2} - \frac{ \gamma^2 a }{\w^2 +\gamma^2} \rp
\eeq
yielding $J_c \propto -1/\re f(-1/a) = 1/(a b) $. 
Therefore we get the proportionality $\w_{\rm doub} - \w_c \propto J_c$, explaining why those quantities vanish simultaneously for large $r$, as shown in Fig. \ref{fig:phaseDiag}.(c-d). 
Another way to understand this proportionality is that the real and imaginary part of the susceptibility are related by the Kramers-Kronig relations \cite{colemanColeman2015}. 

%The steady-state of the single-site problem is diagonal in the occupation eigenstates $\rho = \sum_n p_n \ket{n}\bra{n}$ and only depends on $r$, not on $k$. For a box-shaped reservoir it can be found analytically
%\begin{align}
%p_n &= r^n \theta(N-n) \frac{r-1}{r^{N+1}-1} 
%\end{align}
%where $N$ is the last populated number eigenstate $N -\oh < \frac{\mu_{\rm eff}}{U} < N +\oh $. For $r\gg 1$ one has $p_n \approx 1/r^{N-n} \theta(N-n)$ and all $n\neq N$ occupations are suppressed. 


\subsection{Perturbation theory for the Redfield equation}
\label{sec:pertRed}

We now show, using perturbation theory, that a strong anti-Lorentzian contribution proportional to $r$ indeed arises in correspondence of the doublon peak, giving rise to a zero of $\im G_0^R(\w)$ setting the critical frequency \eqref{eq:neqCritFreq}, and not in correspondence of other resonances of the Green function. 
%
%%We now aim at understanding with perturbative calculations why a strong anti-resonance develops in correspondence of the resonance at $\w\approx\w_c$. 
%
%The main features of $G_0^R(\w)$ are two peaks located at the frequencies $\w_{p(h)}$ corresponding to a hole or doublon excitation, as shown in Fig. \ref{fig:dmft} (left).
%%
%%The steady-state for $r \gg 1$ approximates the ground-state of a Bose-Hubbard site. Therefore we can understand its retarded Green function starting from that of the ground state one
%%\beq
%%G^R (\w)  \approx   \frac{N+1}{ {\w }- \w_{\rm doub} + i 0 } - \frac{N}{{\w }- \w_{\rm hole} + i 0 }
%%\eeq
%%by neglecting dissipative contributions to the weights and to the lifetimes. 
%%This function has two poles at $\w_{p(h)}$ corresponding to the processes of adding a particle and a hole to the steady-state. 
%
%The critical frequency is associated with the development of an anti-Lorentzian contribution, coming from the coupling with the stabilizer reservoir, of the resonance at $\w = \w_{\rm doub}$ as show in Fig. REF. 
In order to capture this feature, we evaluate the Lehmann representation \eqref{eq:retOpenSys} 
%that we report here for convenience
%\begin{equation}
%G^R(\omega)=\sum_\alpha \frac{w_\alpha}{\omega+\operatorname{Im} \lambda_\alpha-\mathrm{iRe} \lambda_\alpha}
%\end{equation}
%with $w_\alpha=\operatorname{tr}\left(\hat{a} \hat{r}_\alpha\right) \operatorname{tr}\left(\hat{l}_\alpha^{\dagger}\left[\hat{a}^{\dagger}, \hat{\rho}\right]\right)$ 
using first-order perturbation theory in the dissipators to approximate the eigenstates and eigenvalues of the Liouvillian. We define $\mathcal{L} = -i \lsq H, \bullet \rsq + \mathcal{D}$ and we perturb in the second term, as for example in \cite{scarlatellaSchiro2019a}. 
The unperturbed eigenstates and eigenvalues are $\lambda_{n,m}^{(0)} = -i (E_n - E_m) $, $r^{(0)}_{n,m} = l^{(0)}_{n,m}= \ket{n}\bra{m}$. 
First-order perturbation theory gives the following corrections to eigenvalues and eigenstates

\begin{align}
\lambda_\alpha^{(1)}&=\operatorname{tr}[ ({l}_\alpha^{(0)} )^{\dagger} {\mathcal{D}} ( {r}_\alpha^{(0)} ) ] & 
{r}_\alpha^{(1)}&=\sum_{\beta \neq \alpha} \frac{\operatorname{tr}[({r}_\beta^{(0)})^{\dagger} {\mathcal{D}}({r}_\alpha^{(0)})]}{\lambda_\alpha^{(0)}-\lambda_\beta^{(0)}} {r}_\beta^{(0)} &
{l}_\alpha^{(1)}&=\sum_{\beta \neq \alpha} \frac{\operatorname{tr}[({l}_\beta^{(0)})^{\dagger} {\mathcal{D}}^{\dagger}({l}_\alpha^{(0)})]}{\lambda_\alpha^{(0) *}-\lambda_\beta^{(0) *}} {l}_\beta^{(0)}
\end{align}

with $\alpha = (n,m)$. 

The term $\tr ( a r_{n,m} )$ in the Green function weights $w_{n,m}$ selects only the eigenstates/values with $m+1=n$, thus we compute only those eigenvalues/states, obtaining
%\begin{align}
%\lambda_\alpha  &\approx \lambda_\alpha^{(0)}  +\lambda_\alpha^{(1)} \\ 
%r_\alpha &\approx r_\alpha^{(0)}  + r_\alpha^{(1)} \\ 
%l_\alpha &\approx l_\alpha^{(0)}  + l_\alpha^{(1)} 
%\end{align}
%with 
%\begin{align}
%\lambda_\alpha^{(1)} &=\operatorname{tr}\left[\left(l_\alpha^{(0)}\right)^{\dagger} \mathcal{D} r_\alpha^{(0)}\right] \\
%r_\alpha^{(1)} &=\sum_{\beta \neq \alpha} \frac{\operatorname{tr}\left[\left(r_\beta^{(0)}\right)^{\dagger} \mathcal{D} r_\alpha^{(0)}\right]}{\lambda_\alpha^{(0)}-\lambda_\beta^{(0)}} r_\beta^{(0)} \\
%l_\alpha^{(1)} &=\sum_{\beta \neq \alpha} \frac{\operatorname{tr}\left[\left(l_\beta^{(0)}\right)^{\dagger} \mathcal{D}^{\dagger} l_\alpha^{(0)}\right]}{\lambda_\alpha^{(0)^*}-{\lambda_\beta^{(0)}}^*} l_\beta^{(0)}
%\end{align}
\small
\begin{align} 
r_{n+1,n} &\approx \ket{n+1} \bra{n} + i \frac{\kappa}{U} \sqrt{(n+1)n} \ket{n} \bra{n-1} - i \frac{r\kappa}{U} \sqrt{(n+2)(n+1)} \lsq S^R_{+-}(E_{n+1} -E_{n}) + S^R_{+-}(E_{n+2} -E_{n+1})^* \rsq \ket{n+2}\bra{n+1} \label{eq:pertVecs_1}\\ 
l_{n+1,n} &\approx \ket{n+1} \bra{n} + i \frac{\kappa}{U} \sqrt{(n+2)(n+1)} \ket{n+2}\bra{n+1} - i \frac{r \kappa}{U} \sqrt{(n+1)n} \lsq S^R_{+-}(E_{n+1} -E_{n})  + S^R_{+-}(E_{n} -E_{n-1})^* \rsq \ket{n} \bra{n-1} \label{eq:pertVecs_2}\\ 
\lambda_{n+1,n} &\approx -i (E_{n+1} -E_n) - \frac{\kappa}{2} (2n+1) - {r\kappa} \lsq S^R_{+-}(E_{n+2} -E_{n+1})^* (n+2) + S^R_{+-}(E_{n+1} -E_{n}) (n+1)  \rsq \label{eq:pertVals} 
\end{align}
\normalsize


%
%There are 2 mechanisms that can suppress terms in those expressions: Secondly $p_{n\neq N} \approx 0$ in the regime $r \gg 1$. This will lead to a strong anti-resonance appearing only for the peak at $\w \approx \w_{\rm doub}$. 
To determine the amplitude of the anti-Lorentzian contribution given by $\im w_{n+1,n}/ \re \lambda_{n+1,n} $, we first compute 
%To understand the anti-Lorentzian structure, we need to compute the ratio $\im w_{n+1,n}/ \re \lambda_{n+1,n} $. 
%First we get
\footnotesize
\beq
\es{
\im w_{n+1,n} &= \sqrt{n+1} \lsq \frac{\kappa}{U}(n+2)\sqrt{n+1} \lp p_{n+1} -p_{n+2} \rp -\frac{r\kappa}{U}  \sqrt{(n+1)n} \lp \re S^R(E_{n+1}-E_n)^* + \re S^R(E_{n}-E_{n-1}) \rp \lp p_{n-1} \sqrt{n} -p_n\sqrt{n+1}\rp \rsq \\ 
&+ (p_n - p_{n+1}) \sqrt{n+1} \lsq \frac{\kappa}{U} n \sqrt{n+1} -\frac{r\kappa}{U} \sqrt{n+1}(n+2) \lp \re S^R(E_{n+1}-E_n) + \re S^R(E_{n+2}-E_{n+1})^* \rp \rsq
}
\eeq
\normalsize 

Then we check for which values of $n$ (i.e. in correspondence of which resonance of the Green function) $\im w_{n+1,n}/ \re \lambda_{n+1,n} $ is of order $r$.
We use that the spectral function of the reservoir $S^R(E_{n}-E_{n-1}) = \frac{1}{2} \theta(\sigma - \abs{E_{n}-E_{n-1}} )$ \eqref{eq:lesserBox_red} (we neglect its imaginary part here) vanishes for $n>N$.
% , and we focus on the terms proportional to $r$ that are the largest ones for $r \gg 1$.
In $\im w_{n+1,n}$ all the terms proportional to $r$ vanish for $n>N$, while $\re \lambda_{n+1,n}$ has no terms proportional to $r$ for $n>N-1$. Then for $n=N$ and only in this case the ratio $\im w_{N+1,N}/ \re \lambda_{N+1,N} $ is proportional to $r$. 
%We notice that this is also independent of $\kappa$, thus staying finite even for $\kappa \rw 0$.
The contribution for $n=N$ corresponds to the doublon resonance, that therefore acquires an anti-Lorentzian contribution proportional to $r$ that eventually leads to a critical point \eqref{eq:neqCritFreq}\eqref{eq:neqCritHop}.  

%Because of $S^R$, the only terms proportional to $r \kappa$ surviving in $\re \lambda_{n+1,n} $ are those with $n \leq N-1$. In $\im w_{n+1,n}$ there are two such terms, $n \leq N$ ($n \leq N-1$) contributions are non-vanishing for the first (second) one. 
%
% 
%In $\re \lambda_{n+1,n} $, all the terms for $n>N-1$ vanish because of $S^R$. 
%In $\im w_{n+1,n}$ there are two terms proportional to $r$: in the first one all terms $n>N$ vanish because of $ S^R$, while population suppress all terms $n\neq N,N+1$, therefore only $n=N$ is order $r$; in the second term, all terms $n>N-1$ vanish by $ S^R$, while populations suppress $n\neq N-1,N$, therefore only the term $n=N-1$ is order $r$. 

%Eventually, when taking the ratio $\im w_{n+1,n}/ \re \lambda_{n+1,n} $, 
%there's only one term for $n=N$ that is proportional to $r$, which agrees with our observation that the corresponding resonance develops a strong anti-Lorentzian contribution. 

Eventually, we can approximate the Green function peak for $n=N$ with the expression
\beq
\label{eq:signPeak}
G_{N+1,N}^R(\w) = \frac{\lp \sqrt{N+1} + \frac{i\kappa}{U} \sqrt{(N+1)}N \rp 
\lp p_N  \sqrt{N+1} - i \frac{r\kappa}{U} p_{N} (N+1)\sqrt{N}  \rp}{\w  - (E_{N+1}-E_N) +\frac{i\kappa}{2}(2N+1) }
\eeq
where we also used that $p_{n\neq N} \approx 0$ for $r \gg 1$.
In Fig. \ref{fig:retGreen} we plot this single-peak approximation as a dashed line for the doublon peak (panel (a)) and show that it correctly captures the anti-Lorentzian, thus the critical frequency $\w_c$ and hopping $J_c$. The dashed line in panel (b) instead approximate the hole-like resonance using the full Green function in perturbation theory. 

We remark that the precise square shape of the reservoir spectral function is not important, as long as this function strongly suppress transitions above a certain energy. Indeed the same behaviour is observed in this Supplemental Material for a reservoir with Lorentzian spectral function.

% While we considered square-shaped reservoir correlations here, an analogous conclusion would hold in other cases in which reservoir correlations strongly suppress transitions above a certain energy, i.e. for the Lorentzian reservoir correlations considered in this Supplemental Material.



\subsection{Perturbation theory for the Lindblad equation}
\label{sec:pertLind}

The perturbation theory using the Lindblad equation \eqref{eq:mbME} instead of Redfield \eqref{eq:redME} is very similar, and one only needs to discard the non-secular terms that are present in the latter case but not in the former.
Discarding those terms yields the same first-order corrections for the eigenvalues of the Liouvillian  \eqref{eq:pertVals} as in Redfield, while all the eigenstates corrections in \eqref{eq:pertVecs_1},\eqref{eq:pertVecs_2} coming from the pump term (proportional to $r \kappa$) correspond to non-secular terms and thus vanish. Therefore the spectral function weights $w_{n+1,n}$ (depending on the eigenstates) also do not depend on $r$ for all $n$ and, eventually, the amplitude of the anti-Lorentzian contributions given by the ratio $\im w_{n+1,n}/ \re \lambda_{n+1,n} $ is never proportional to $r$. 


Note though that the same behaviour of the critical frequency $\w_c$ approaching the doublon energy $ \w_{\rm doub}$ increasing the pump/loss ratio $r$ is observed for the Lindblad case (Fig. \ref{fig:phaseDiag}(e)), therefore a similar anti-Lorentzian contribution to the doublon resonance proportional to $r$ is expected to arise from higher-order terms in perturbation theory.

This difference between the Redfield and Lindblad perturbation theories explains why the dissipative-Mott instability is more pronounced in the former case, with a smaller critical hopping, while both equations give qualitatively the same predictions. 
 


\end{document}
