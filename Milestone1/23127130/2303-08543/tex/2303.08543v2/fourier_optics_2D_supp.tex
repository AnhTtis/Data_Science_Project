\documentclass[
onecolumn,
11pt,
amsmath,
amssymb,
aps,
prl
]{revtex4-2}

\usepackage{bm}
\usepackage{graphicx}

\newcommand{\paperTitle}{Planar Fourier Optics for Slab Waveguides, Surface Plasmon Polaritons and 2D Materials}
\newcommand{\me}{Benjamin Wetherfield}
\newcommand{\myemail}{bsw28@cam.ac.uk}
\newcommand{\tim}{Timothy D. Wilkinson}
\newcommand{\EE}{Electrical Engineering Division, Engineering Department, University of Cambridge, UK}
\newcommand{\ABSTRACT}{
Recent experimental work has demonstrated the potential to combine the merits of diffractive and on-chip photonic information processing devices in a single chip by making use of planar (or slab) waveguides. Researchers have adapted key results of 3D Fourier optics to 2D, by analogy, but rigorous derivations in planar contexts have been lacking. Here, such arguments are developed to show that diffraction formulas familiar from 3D can be adapted to 2D under certain mild conditions on the operating speeds of the devices in question. Equivalents to the Rayleigh-Sommerfeld diffraction (RS) formulas in 2D are provided and a Radiation Condition of validity proved. The equivalence of the first 2D RS formula with an angular spectrum formulation is demonstrated. Finally Fresnel approximations are derived starting from the RS formulation and that of the angular spectrum. In addition to serving those working with slab waveguides, this letter provides analytical tools to researchers in any field where 2D diffraction is encountered, including the study of surface plasmon polaritons, surface waves, 3D diffraction with line-sources or corresponding symmetries, and the optical, acoustic and crystallographic properties of 2D materials.
}
\newcommand{\ShortAbstract}{
Optical information processing is a growing field seeking to meet modern computing challenges. Recent work shows that the merits of free-space diffractive and on-chip photonic devices can be combined by making use of diffraction in planar (or slab) waveguides. To support this work, we provide derivations of the key results of Fourier Optics in two dimensions. In addition to serving those working in on-chip photonics, this letter provides analytical tools for the study of surface plasmon polaritons, surface waves, and the optical, acoustic and crystallographic properties of 2D materials.
}

\begin{document}

\title{\paperTitle: \\ Supplementary Information}

\author{\me}
\email{\myemail}
\author{\tim}
\affiliation{\EE}

\maketitle

\makeatletter
\renewcommand \thesection{S\@arabic\c@section}
\renewcommand\thetable{S\@arabic\c@table}
\renewcommand \thefigure{S\@arabic\c@figure}
\renewcommand \theequation{S\@arabic\c@equation}
\makeatother

\section*{General Proof of the Kirchhoff-Helmholtz-Weber Integral Formula}
\label{sec:kirchh-helmh-integr}


We prove the Weber integral formula, as stated in \cite{baker}, for two dimensional scalar diffraction, in a manner that generalizes to arbitrary dimensions since no explicit derivatives need to be taken. In three dimensions, the formula is often referred to as the Kirchhoff-Helmholtz integral formula. The formula states
\begin{equation}
  \label{eq:supp_integral_weber}
  U(\boldsymbol{r}) = \iint_{\partial S}   U(\boldsymbol{r'})\frac{\partial G_{2D}(\boldsymbol{r} - \boldsymbol{r'})}{\partial n}
-\frac{\partial U(\boldsymbol{r'})}{\partial n}G_{2D}(\boldsymbol{r} - \boldsymbol{r'})
  \; d\boldsymbol{r'}
\end{equation}
where $G_{2D}$ is a Green's function satisfying:
\begin{equation}
  \label{eq:supp_Greens_helmholtz_2D}
  \left ( \nabla^{2}_{xz} + k^{2}\right )G_{2D}(\mathbf{r} - \mathbf{r}') = \delta(\mathbf{r} - \mathbf{r}')
\end{equation}

To prove the formula, we introduce an auxiliary surface $C_{\epsilon}$,  a circle of radius $\epsilon$ around $\mathbf{r}$ that enables us to avoid applying the divergence theorem at a point of discontinuity. $\epsilon$ can be made arbitrarily small so that $C_{\epsilon}$ fits entirely in $S$. A diagram consisting of $S$, $C_{\epsilon}$ and the various variables in the Weber formula is presented in Figure \ref{fig:weber_formula_diagram}.
\begin{figure}[ht]
  \centering
  \includegraphics[scale=1.5]{diagram_for_kirchhoff_helmholtz.pdf}
  \caption{Diagram corresponding to the variables in the Weber integral formula and its modification to incorporate $C_{\epsilon}$.}
  \label{fig:weber_formula_diagram}
\end{figure}

Consider the total surface $S - C_{\epsilon}$, which removes $C_{\epsilon}$ from $S$. The boundary of the reduced surface can be written $\partial S - \partial C_{\epsilon}$. In the following, we drop the subscript on $G_{2D}$, as the same argument holds in arbitrary higher dimensions just as it does in 2D (with volumes in place of surfaces). Applying the divergence theorem, then adding zero,

\begin{align}
  &\iint_{\partial S - \partial C_{\epsilon}}\left ( U(\boldsymbol{r'})\frac{\partial G(\boldsymbol{r} - \boldsymbol{r'})}{\partial n} - \frac{\partial U(\boldsymbol{r'})}{\partial n}G(\boldsymbol{r} - \boldsymbol{r'}) \right)  \; d\boldsymbol{r'}\\
  &= \iint_{S -  C_{\epsilon}} U(\boldsymbol{r}')\nabla^{2}G(\boldsymbol{r} - \boldsymbol{r'}) -G(\boldsymbol{r} - \boldsymbol{r'})\nabla^{2} U(\boldsymbol{r}')\; d\boldsymbol{r'} \\
  \label{eq:supp_6} &= \iint_{S - C_{\epsilon}}   U(\boldsymbol{r}')(\nabla^{2} + k^{2})G(\boldsymbol{r} - \boldsymbol{r'}) - G(\boldsymbol{r} - \boldsymbol{r'})(\nabla^{2} + k^{2}) U(\boldsymbol{r}')\; d\boldsymbol{r'}
\end{align}

Now, separating the surfaces $S$ and $C_{\epsilon}$,
\begin{align}
 & \iint_{S}   U(\boldsymbol{r}')(\nabla^{2} + k^{2})G(\boldsymbol{r} - \boldsymbol{r'}) - G(\boldsymbol{r} - \boldsymbol{r'})(\nabla^{2} + k^{2}) U(\boldsymbol{r}')\; d\boldsymbol{r'} \\
 &= \iint_{C_{\epsilon}}   U(\boldsymbol{r}')(\nabla^{2} + k^{2})G(\boldsymbol{r} - \boldsymbol{r'}) -G(\boldsymbol{r} - \boldsymbol{r'})(\nabla^{2} + k^{2}) U(\boldsymbol{r}')\; d\boldsymbol{r'} \\
  &= U(\boldsymbol{r})
\end{align}
by the sifting property of the delta function, and we have proved the formula.

\begin{thebibliography}{99}
\bibitem[S1]{baker}
B. B. Baker and E. T. Copson, \emph{The Mathematical Theory of Huygens’ Principle} (Oxford University Press, 1950), 2nd
ed.
\end{thebibliography}


\end{document}
%%% Local Variables:
%%% mode: latex
%%% TeX-master: t
%%% End:
