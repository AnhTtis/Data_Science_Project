\section{Construction of new solutions using cycle-matrices}\label{sec:construction-old-new}
We start this section by fixing notations. This is crucial for understanding the construction of new solutions. For $i=1,2$ consider solutions $(X_i,\cdot_i)$ on the set $X_i=\{1,2,\ldots,k_i\}$. Then we have two cycle matrices $M_1$ and $M_2$. For $\sigma_1,\sigma_2,\ldots,\sigma_{k_2}\in\Sym_{k_1}$ and $\mu_1,\mu_2,\ldots,\mu_{k_1}\in\Sym_{k_2}$, define the matrix $\mathbb{M}$ to be
\begin{align*}
    \mathbb{M}_{ij}=\begin{cases}
        (M_1)_{ij} & \text{if } 1\leq i,j\leq k_1\\
        (M_2)_{i-k_1,j-k_1}+k_1 & \text{if }k_1+1\leq i,j\leq k_1+k_2\\
        \sigma_{i-k_1}(j) & \text{if } k_1+1\leq i\leq k_1+k_2,~1\leq j\leq k_1\\
        \mu_{i}(j-k_1)+k_1 & \text{if } 1\leq i\leq k_1, k_1+1\leq j\leq k_1+k_2
    \end{cases}.
\end{align*}
This matrix will be put as 
\begin{align*}
    \left[\begin{array}{ccc|ccc}
	     &&&&\mu_1&\\
      &\Huge{X_1}&&&\vdots&\\
      &&&&\mu_{k_1}&\\
      \hline
      &\sigma_1&&&&\\
      &\vdots&&&\Huge{X_2}&\\
      &\sigma_{k_2}&&&&
	\end{array}\right].
\end{align*}
If all $\mu_l$'s (respectively $\sigma_k$'s) are same, say $\mu$ (respectively $\sigma$), then this matrix will be simply put as $\left[\begin{array}{c|c}
     X_1&\mu  \\
     \hline
     \sigma&X_2 
\end{array}\right]$.
There will be more variants of this matrix throughout this section. We hope that the matrices will be clear from the context. We exhibit one example to 
provide a better understanding of the above construction.
\begin{example} For example if we take two cycle matrices 
\begin{align*}
    M_1=\left[\begin{array}{ccc}
         1 & 2 & 3 \\
         1 & 2 & 3\\
         1 & 2 & 3
    \end{array}\right],
    M_2=\left[\begin{array}{cc}
         2 & 1 \\
         2 & 1
    \end{array}\right],
\end{align*}
and $\sigma_1=(1,2,3),\sigma_2=(1,2)(3)$, $\mu_1=(1,2),\mu_2=(1,2),\mu_3=(1)(2)$, then the abovementioned matrix will be
\begin{align*}
    \left[\begin{array}{ccccc}
        1 & 2 & 3 & 5 & 4 \\
        1 & 2 & 3 & 5 & 4 \\
        1 & 2 & 3 & 4 & 5 \\
        2 & 3 & 1 & 5 & 4 \\
        2 & 1 & 3 & 5 & 4 
    \end{array}\right].
\end{align*}
\end{example}
\begin{theorem}\label{thm:identity-partition-any}
	Consider $(X_1,\star_1)$ and $(X_2,\star_2)$ to be two trivial solutions of size $k_1$, and $k_2$ respectively. Let a partition on $X_1$ be $^1X_1,~^2X_1,\cdots,~^kX_1$, where $^iX_1=\{r_i,r_i+1,\cdots, r_{i+1}-1\}$ for $i=1,~2,~\cdots, k-1$ and $^kX_1=\{r_k,r_k+1,\cdots,k_1\}$ and $r_1=1.$  Let $^i\alpha_1\in\Sn(r_i,r_i+1,\ldots,r_{i+1}-1),~^i\alpha_2\in\Sn(k_1+1,k_1+2,\ldots,k_1+k_2)$ for $i=1,2,\cdots, k$ such that $^i\alpha_2 ~^j\alpha_2=~^j\alpha_2~ ^i\alpha_2$ for all $i,~j$. Then the matrix 
\begin{align*}
           \mathbb{M}_{(X_i),(\alpha_i),(),[k]} = 
\left(\begin{array}{@{}c|c@{}}
    \bigxone 
    & 
    \begin{array}{c}
         ~^1{\alpha_2} \\ \hline
         \vdots \\ \hline 
         ~^k{\alpha_2} \\
    \end{array}\\
    \hline
    \begin{array}{c|c|c}
        ~^1{\alpha_1} & \cdots & ~^k{\alpha_1} 
    \end{array}
    & \bigxtwo
\end{array}\right)
\end{align*}
is a cycle matrix.
\end{theorem}
\begin{proof}
Define a binary operation $\star$ on $X=X_1\sqcup X_2$ in the following way;
\begin{align*}
&\star|_{X_1\times X_1}=\star_1,
\star|_{X_2\times X_2}=\star_2,\\
&\star|_{~^iX_1\times X_2}(x,x_2)=~^i\alpha_2(x_2), 
\star|_{X_2\times ~^iX_1}(x_2,x)=~^i\alpha_1(x),    
\end{align*}
for all $x\in ~^iX_1$, $x_2\in X_2$ and $i=1,2,\ldots,k$.
We aim to show that $(X,\star)$ is a cycle set.
It is easy to see from the definition of $\star$ that each row and principal diagonal of the matrix $ \mathbb{M}_{(X_i),(\alpha_i),(),[k]}$ is a permutation.
Now we show the cycloid relation. We do it by considering various cases.

\textbf{Case 1:} If all of $x,y,z$ lies in one of $X_1$ or $X_2$, the cycloid relation trivially holds.

\textbf{Case 2: } $x\in ~^iX_1$, $y\in ~^jX_1$ and $z\in X_2$
\begin{equation*}
	\begin{split}
		(x\star y)\star (x \star z)&=y\star ~^i\alpha_2(z)=~^j\alpha_2~^i\alpha_2(z),\\
		(y\star x)\star (y \star z)&=x\star ~^j\alpha_2(z)=~^i\alpha_2~^j\alpha_2(z).
	\end{split}
\end{equation*}

\textbf{Case 3: } $x\in ~^iX_1$, $y\in ~X_2$ and $z\in ~^jX_1$
\begin{equation*}
	\begin{split}
		(x\star y)\star (x \star z)&=~^i\alpha_2
		(y)\star z=~^j\alpha_1(z),\\
		(y\star x)\star (y \star z)&=~^i\alpha_1(x)\star ~^j\alpha_1(z)=~^j\alpha_1(z).
	\end{split}
\end{equation*}

\textbf{Case 4: } $x\in ~^iX_1$, $y\in ~X_2$ and $z\in X_2$
\begin{equation*}
	\begin{split}
		(x\star y)\star (x \star z)&=~^i\alpha_2
		(y)\star ~^i\alpha_2(z)=~^i\alpha_2(z),\\
		(y\star x)\star (y \star z)&=~^i\alpha_1(x)\star z=~^i\alpha_2(z).
	\end{split}
\end{equation*}

\textbf{Case 5: } $x\in X_2$, $y\in ~^iX_1$ and $z\in ~^jX_1$
\begin{equation*}
	\begin{split}
		(x\star y)\star (x \star z)&=~^i\alpha_1
		(y)\star ~^j\alpha_1(z)=~^j\alpha_1(z),\\
		(y\star x)\star (y \star z)&=~^i\alpha_2(x)\star z=~^j\alpha_2(z).
	\end{split}
\end{equation*}

\textbf{Case 6: } $x\in X_2$, $y\in ~^iX_1$ and $z\in X_2$
\begin{equation*}
	\begin{split}
		(x\star y)\star (x \star z)&=~^i\alpha_1
		(y)\star z=~^i\alpha_2(z),\\
		(y\star x)\star (y \star z)&=~^i\alpha_2(x)\star ~^i\alpha_2(z)=~^i\alpha_2(z).
	\end{split}
\end{equation*}

\textbf{Case 7: } $x\in X_2$, $y\in ~X_2$ and $z\in ~^iX_1$
\begin{equation*}
	\begin{split}
		(x\star y)\star (x \star z)&=y\star ~^i\alpha_1(z)=~^i\alpha_1~^i\alpha_1(z),\\
		(y\star x)\star (y \star z)&=x\star ~^i\alpha_1(z)=~^i\alpha_1~^i\alpha_1(z).
	\end{split}
\end{equation*}
Therefore $(X,\star)$ is a cycle set and, the cycle matrix is given by the above matrix.
\end{proof}
Continue with the notation of $\cref{thm:identity-partition-any}.$ 
\begin{corollary}\label{cor:multipermutation 2}
The constructed cycle-matrix $\mathbb{M}_{(X_i),(\alpha_i),(),[k]}$, is a multipermutation solution of level $2.$
\end{corollary}
\begin{proof}
    $Ret(\mathbb{M}_{(X_i),(\alpha_i),(),[k]})$ is a cycle-matrix of a trivial solution.
\end{proof}
\begin{corollary}
    Every finite abelian group is a permutation group.
\end{corollary}
\begin{proof}
    Let $H=\langle \sigma_1,\sigma_2,\cdots,\sigma_r\rangle$ to be a finite abelian group.
    Then take $X_1$ to be of size $1$, $X_2$ of size $r$ and $^i\alpha_2=\sigma_i$. Then applying \cref{thm:identity-partition-any} we have the result.
\end{proof}

\begin{proposition}\label{prop:2-times-2}
Consider $(X_1,\star_1)$ and $(X_2,\star_2)$ to be two solutions of size $k_1$, and $k_2$ respectively. Let $\alpha_i\in\Aut(X_i,\star_i)$, for $i=1,2$. Then the matrix 
\begin{align*}
    \mathbb{M}_{(X_i),(\alpha_i),(12),[~]} =\left[\begin{array}{c|c}
    X_1&\alpha_2
    \\\hline
    \alpha_1
&X_2
    \end{array}\right]
\end{align*}
is a cycle matrix.
\end{proposition}
\begin{proof}
    Define the following binary operation `$\star$' on $X=X_1\sqcup X_2$ in the following way:
    \begin{align*}
      \star|_{X_1\times X_1}=\star_1, & \star|_{X_1\times X_2}(x_1,x_2)=\alpha_2(x_2),\\
	\star|_{X_2\times X_1}(x_2,x_1)=\alpha_1(x_1), & \star|_{X_2\times X_2}=\star_2,   
    \end{align*}
	 for all $x_i\in X_i$, $i=1,2$. We show that $(X,\star)$ is a cycle set.
	Assuming $x\in X$, from the definition of $\star$ we get that,
 \begin{align*}
     &^X\psi_x(x_1)=^{X_1}\psi_x(x_1),~^X\psi_x(x_2)=\alpha_2(x_2),\\
	&^X\psi_x(x_2)=^{x_2}\psi_x(x_2),~^X\psi_x(x_1)=\alpha_1(x_1),
 \end{align*} for $x\in X_2,~x_1\in X_1$ and $x_2\in X_2.$ This proves that $^X\psi_x$ is a permutation on $X.$ Next let us denote the diagonal maps for $X_1$ and $X_2$ by $\pi_1$ and $\pi_2$ respectively. Then the map $x\mapsto x\star x$ on $X$ will be of the form $x\mapsto \pi_1(x)$ if $x\in X_1$ and $x\mapsto \pi_2(x)$ for $x\in X_2$. Hence the diagonal map $x\mapsto x\star x$ is bijective on $X.$ We finally show that the cycloid relation holds.
	The proof will be divided into several cases.
	Let $x,~y$ and $z\in X$. We start with the cases where $x\in X_1$.
 
\textbf{Case 1: ($y,~z\in X_1$)}
	In this case, the cycloid relation trivially holds.
 
\textbf{Case 2: ($y\in X_1$, $z\in X_2$)} 
	\begin{equation*}
		\begin{split}
			(x\star y)\star (x \star z)&=( ^{X_1}\psi_x(y))\star \alpha_2(z)=\alpha_2(\alpha_2(z)),\\
			(y\star x)\star (y \star z)&=( ^{X_1}\psi_y(x))\star \alpha_2(z)=\alpha_2(\alpha_2(z)).
		\end{split}
	\end{equation*}

\textbf{Case 3:  ($z\in X_1$, $y\in X_2$)}
    	\begin{equation*}
    	\begin{split}
    		(x\star y)\star (x \star z)&=\alpha_2(y)\star ^{X_1}\psi_x(z)=\alpha_1( ^{X_1}\psi_x(z)),\\
    		(y\star x)\star (y \star z)&=\alpha_1(x)\star \alpha_1(z)= ^{X_1}\psi_{\alpha_1(x)}(\alpha_1(z)).\\
    	\end{split}
    \end{equation*}

\textbf{Case 4: ($y,~z\in X_2$)}
	\begin{equation*}
		\begin{split}
		   (x\star y)\star (x \star z)&=\alpha_2(y)\star\alpha_2(z)=^{X_2}\psi_{\alpha_2(y)}(\alpha_2(z)),\\
		   (y\star x)\star (y \star z)&=\alpha_1(x)\star ^{X_2}\psi_y(z)=\alpha_2(^{X_2}\psi_y(z)).\\
		\end{split}
	\end{equation*}
 Similarly, we can prove the other equalities. Therefore $(X,\star)$ is a cycle set. The cycle matrix of is clearly seen to be $\mathbb{M}_{(X_i),(\alpha_i),(12)}.$
 \end{proof}
 The above solution will be denoted by $X_1\bigcup\limits_{\alpha_1,\alpha_2}X_2$, hereafter.
\begin{example}
Consider the solutions $(X_1,\star_1)$ and $(X_2,\star_2)$ with cardinality $2$ and $3$ respectively. Further, assume the cycle matrix to be of the form
\begin{align*}
    \begin{bmatrix}
    1&2\\
    1&2
    \end{bmatrix}, \begin{bmatrix}
        1&2&3\\
        1&2&3\\
        1&2&3
    \end{bmatrix}.
\end{align*}
Then taking $\alpha_1=(1,2)$ and $\alpha_2=(1,2,3)$ we get a new solution of cardinality $5$ given by cycle-matrix 
\begin{align*}
    \begin{bmatrix}
        1&2&4&5&3\\
        1&2&4&5&3\\
        2&1&3&4&5\\
        2&1&3&4&5\\
        2&1&3&4&5
    \end{bmatrix}.
\end{align*}
\end{example}

\begin{theorem}\label{thm:2-times-2-general}
Let $X_1,X_2,\cdots, X_l$ be $l$ solutions of size $k_1,k_2,\cdots,k_l$ respectively. Let $\alpha_i\in\Aut(X_i,\cdot_i)$ for $1\leq i\leq l$. Further, consider 
\begin{align*}
\alpha_{1,2,\ldots,t} \in \Aut \left(
    \left(\ldots\left(X_1\bigcup\limits_{\alpha_1,\alpha_2}X_2\right)\ldots\bigcup\limits_{\alpha_{1,2,\ldots,t-1},\alpha_t}X_t
    \right)
\right)
\end{align*}
for $2\leq t\leq l-1$. Then the matrix
\begin{align*}
,    \left[
        \begin{array}{c|c}
            \begin{array}{cc}
            \begin{array}{c|c|}
                \begin{array}{c|c}
                     \bigxone  & \alpha_2 \\ \hline
                     \alpha_1 & \bigxtwo
                 \end{array} & \alpha_3 \\ \hline
                 \alpha_{1,2} & \bigxthree \\ 
                 \hline
            \end{array} 
            & 
            \begin{array}{ccc}
                 &  & \\
                 &  & \\
                 &  & 
            \end{array} 
            \\
            \begin{array}{ccc}
                 &  & \\
                 &  & \\
                 &  & 
            \end{array}
            & 
            \ddots 
        \end{array} 
        & 
        \biga_l
        \\
        \hline
        \biga_{1,2,\ldots,l-1} 
        & 
        X_l
        \end{array}
    \right],
\end{align*}
is a cycle matrix.
\end{theorem}
\begin{proof}
    The proof is by induction. Note that the case $n=2$ is the result of \cref{prop:2-times-2}. Now assume the result to be true for $l=n$. 

    Assume $\beta=\alpha_{1,2,\ldots,n+1}$ is an automorphism of $\widetilde{X}=\left(\ldots\left(X_1\bigcup\limits_{\alpha_1,\alpha_2}X_2\right)\ldots\bigcup\limits_{\alpha_{1,2,\ldots,n},\alpha_n}X_n\right)$. Given that $\alpha_l\in\Aut(X_l,\cdot_l)$, using \cref{prop:2-times-2} we get that
    \begin{align*}
        \left[\begin{array}{c|c}
            \widetilde{X} & \alpha_l \\\hline
             \beta & X_l
        \end{array}\right]
    \end{align*}
    is a cycle matrix. Note that this matrix is as same as the matrix given in the statement. This finishes the proof.
\end{proof}
\begin{proposition}\label{prop:3-times-3}
    Consider $X_1$, $X_2$, and $X_3$ to be three solutions of size $k_1,~k_2$ and $k_3$ respectively. Let $\alpha_i\in\Aut(X_i,\cdot_i)$ for $i=1,2,3$. Then the following two matrices
    \begin{align*}
        \mathbb{M}_{(X_i),(\alpha_i),(123),[~]}=\left[\begin{array}{c|c|c}
            X_1 &  & \alpha_3\\
             \hline
            \alpha_1 & X_2 &\\
             \hline
             & \alpha_2 & X_3
        \end{array}\right] ~\text{and}~
        \mathbb{M}_{(X_i),(\alpha_i),(12),[~]}=\left[\begin{array}{c|c|c}
            X_1 & \alpha_2 & \\
             \hline
             & X_2 & \\
             \hline
            \alpha_1 &  & X_3
        \end{array}\right]
    \end{align*}
    are two cycle matrices.
\end{proposition}
\begin{proof}
	Define $\star$ on $X=X_1\sqcup X_2\sqcup X_3$ as follows
 \begin{align*}
    &\star|_{X_i\times X_i}=\star_i,\\
    &\star|_{X_2\times X_3}(x_2,x_3)=x_3, 
    \star|_{X_3\times X_1}(x_3,x_1)=x_1,\star|_{X_1\times X_2}(x_1,x_2)=x_2,\\
	&\star|_{X_1\times X_3}(x_1,x_3)=\alpha_3(x_3),
 \star|_{X_2\times X_1}(x_2,x_1)=\alpha_1(x_1),\star|_{X_3\times X_2}(x_3,x_2)=\alpha_2(x_2),
 \end{align*}
	for all $x_i\in X_i$, $i=1,2,3.$ It is easy to see from the matrix $\mathbb{M}_{(X_i),(\alpha_i),(123),[~]}$ that all rows and the principal diagonal are permutations. So we only need to show the cycloid relation.
    From \cref{prop:2-times-2} observe that $\mathbb{M}_{(X_1,X_2),(\alpha_1, id),(1,2),[~]}$, and $\mathbb{M}_{(X_2,X_3),(\alpha_2, id),(2,3),[~]}$ are cycle-matrices. Also, the matrix 
    \begin{align*}
    \left[\begin{array}{c|c}
         X_1&  \Delta_3 \\
         \hline
         \alpha_1& X_3
    \end{array}\right]    
    \end{align*}
    for $\Delta_3\in\{I,\alpha_3\}$, is a cycle matrix.
    Therefore  $\mathbb{M}_{(X_i),(\alpha_i),(123),[~]}$ (for $\Delta_3=\alpha_3$) and $\mathbb{M}_{(X_i),(\alpha_i),(12),[~]}$ (for $\Delta_3=I$) are a cycle-matrices.
     % We will show here a few cases and other cases that can be done similarly;\\
	% \textbf{Case 1:} $x\in X_1,~y\in X_2$ and $z\in X_1$
	% \begin{equation*}
	% 	\begin{split}
	% 		(x\star y)\star (x\star z)&=y\star ~^1\psi_x(z)=\alpha_1(~^1\psi_x(z))\\
	% 		(y\star x)\star (y\star z)&=\alpha_1(x)\star\alpha_1(z)=~^1\psi_{\alpha_1(x)}(\alpha_1(z))\\
	% 	\end{split}
	% \end{equation*}
	% 	\textbf{Case 2:} $x\in X_3,~y\in X_2$ and $z\in X_2$
	% \begin{equation*}
	% 	\begin{split}
	% 		(x\star y)\star (x\star z)&=\alpha_2(y)\star\alpha_2(z)=~^2\psi_{\alpha_2(y)}\alpha_2(z)\\
	% 		(y\star x)\star (y\star z)&=x\star~^{X_2}_y(z)=\alpha_2(~^{X_2}\psi_y(z))\\
	% 	\end{split}
	% \end{equation*}
	% 		\textbf{Case 3:} $x\in X_3,~y\in X_1$ and $z\in X_2$
	% \begin{equation*}
	% 	\begin{split}
	% 		(x\star y)\star (x\star z)&=y\star~^{X_3}\psi_x(z)=\alpha_3(~^{X_3}\psi_x(z))\\
	% 		(y\star x)\star (y\star z)&=\alpha_3(x)\star\alpha_3(z)=~^{X_3}\psi_{\alpha_3(x)}\alpha_3(z)\\
	% 	\end{split}
	% \end{equation*}
\end{proof}

Note that it is possible to construct two non-isomorphic cycle matrices of the form $\mathbb{M}_{(X_i),(\alpha_i),\theta_1,[~]}$, $\mathbb{M}_{(X_i),(\alpha_i),\theta_2,[~]}$ even if $\theta_1$ and $\theta_2$ have the same cycle decomposition. Here is an example.
\begin{example}
Consider
   \begin{align*}
    A= \begin{array}{c|cccc|ccc|cc}
    & 1 & 2 & 3 & 4 & 5 & 6 & 7 & 8 & 9\\
    \hline
    1 & 1 & 2 & 3 & 4 & 6 & 7 & 5 & 8 & 9\\
    2 & 1 & 2 & 3 & 4 & 6 & 7 & 5 & 8 & 9\\
    3 & 1 & 2 & 3 & 4 & 6 & 7 & 5 & 8 & 9\\
    4 & 1 & 2 & 3 & 4 & 6 & 7 & 5 & 8 & 9\\\hline
    5 & 1 & 2 & 3 & 4 & 5 & 6 & 7 & 9 & 8\\
    6 & 1 & 2 & 3 & 4 & 5 & 6 & 7 & 9 & 8\\
    7 & 1 & 2 & 3 & 4 & 5 & 6 & 7 & 9 & 8\\\hline
    8 & 2 & 1 & 4 & 3 & 5 & 6 & 7 & 8 & 9\\
    9 & 2 & 1 & 4 & 3 & 5 & 6 & 7 & 8 & 9
    \end{array},~B =
    \begin{array}{c|cccc|ccc|cc}
    & 1 & 2 & 3 & 4 & 5 & 6 & 7 & 8 & 9\\
    \hline
    1 & 1 & 2 & 3 & 4 & 5 & 6 & 7 & 9 & 8\\
    2 & 1 & 2 & 3 & 4 & 5 & 6 & 7 & 9 & 8\\
    3 & 1 & 2 & 3 & 4 & 5 & 6 & 7 & 9 & 8\\
    4 & 1 & 2 & 3 & 4 & 5 & 6 & 7 & 9 & 8\\\hline
    5 & 2 & 1 & 4 & 3 & 5 & 6 & 7 & 8 & 9\\
    6 & 2 & 1 & 4 & 3 & 5 & 6 & 7 & 8 & 9\\
    7 & 2 & 1 & 4 & 3 & 5 & 6 & 7 & 8 & 9\\\hline
    8 & 1 & 2 & 3 & 4 & 6 & 7 & 5 & 8 & 9\\
    9 & 1 & 2 & 3 & 4 & 6 & 7 & 5 & 8 & 9
    \end{array}.
\end{align*}
Here $X_1,~X_2$ and $X_3$ are trivial solutions of order $4,~3$ and $2$ respectively, $\alpha$ permutations are $\alpha_1=(1,2)(3,4),~\alpha_2=(5,6,7)$ and $\alpha_3=(8,9)$, and the $\theta$ permutations are $\theta_1=(1,3,2),~\theta_2=(1,2,3)$ respectively. From \cref{lem:cycle-decomposition}, these two cycle-matrices are non-isomorphic.  
\end{example}
Now we generalize the previous results and present the general scenario in the last result of this section.
\begin{theorem}\label{thm:3-times-3-general}
    Let $N$ be a positive integer and $\Lambda=(m_1,m_2,\ldots,m_k)$ be a partition of $N$. Consider $(X_i,\star_i)$ to be a solution of size $m_i$ and $\alpha_i\in\Aut(X_i,\star_i)$ for all $i=1,2,\ldots,k$. For an element $\Theta\in S_k$  consider the matrix with $\mu\nu$-th block defined as,
    \begin{align*}
        \left(\mathbb{M}_{(X_i),(\alpha_i),\Theta,[~]}\right)_{\mu\nu}=\begin{cases}
        X_\mu&\text{if } \mu=\nu\\
        \alpha_\nu&\text{if }\Theta(\mu)=\nu\neq\mu\\
        \text{I}&\text{otherwise}
        \end{cases}.
    \end{align*}
 Then the matrix $\mathbb{M}_{(X_i),(\alpha_i),\Theta,[~]}$ is a cycle-matrix.
\end{theorem}
\begin{proof}
    The proof is based on the induction argument and the method of \cref{prop:3-times-3}.
    We start with the first step when $k=2$. 
    Then this is the result \cref{prop:2-times-2}. Now assume the result to be true for $k=n$. 
    For the case, $k=n+1$, note that first considering the matrix for $1\leq \mu,\nu \leq n$, we obtain a cycle matrix.
    Similarly for $2\leq \mu,\nu\leq n+1$ We get another cycle matrix. Now
    \begin{align*}
        \left[\begin{array}{c|c}
             X_1&  \Delta_{n+1}\\\hline
             \Delta^\prime_1& X_{n+1}
        \end{array}\right],
    \end{align*}
    is a cycle matrix for $\Delta_{n+1}\in\left\{id_{X_{n+1}},\alpha_{n+1}\right\}$, $\Delta^{\prime}_{1}\in\left\{id_{X_1},\alpha_{n+1}\right\}$ (using \cref{prop:2-times-2}). Hence it follows that $\mathbb{M}_{(X_i),(\alpha_i),\Theta,[~]}$ is a cycle matrix.
\end{proof}