\section{Results in generalities}\label{sec:general-results}
We start with the definition of a cycle matrix. This will be followed by a few examples and properties of such a matrix.
\begin{definition}[Cycle matrix]
    An $n\times n$ matrix $M=(m_{ij})$ with all entries from $\{1,2,\ldots,n\}$ is said to be a \emph{cycle matrix} if $(U_n,\cdot)$ is a non-degenerate cycle set and, $i\cdot j= m_{ij}$. Two such matrices $(m_{ij}), (m^\prime_{ij})$ will be called \emph{isomorphic} if the corresponding cycle sets are isomorphic to each other. The set of all $n\times n$ cycle matrices will be denoted by $\cyc_n$. Further, a cycle matrix will be called \emph{indecomposable}, if the corresponding SYBE (coming from the cycle set of the matrix) is indecomposable.
\end{definition}
\begin{example}
Considering $U_4=\{1,2,3,4\}$ the following two matrices 
\begin{align*}
    \begin{bmatrix}
    1 & 2 & 3 & 4\\
    1 & 2 & 3 & 4\\
    1 & 2 & 3 & 4\\
    1 & 2 & 3 & 4
    \end{bmatrix},
    \begin{bmatrix}
    1 & 2 & 3 & 4\\
    1 & 2 & 3 & 4\\
    1 & 2 & 3 & 4\\
    1 & 3 & 2 & 4
    \end{bmatrix}
\end{align*}
are two non-isomorphic cycle matrices.
\end{example}
% We note down a few basic properties of the cycle matrix. 
\begin{lemma}\label{lem:existence-of-matrices}
	For any permutation $\sigma\in\Sym_n$, there is a cycle matrix whose diagonal is $\sigma$. In particular $\cyc_n\neq\emptyset$.
\end{lemma}
\begin{proof}
    Consider the permutation solution corresponding to $\sigma$. Note that in that case, the cycle matrix will be
    \begin{align*}
        \begin{bmatrix}
            \sigma(1) & \sigma(2)&  \cdots& \sigma(n)\\
            \sigma(1) & \sigma(2)&  \cdots& \sigma(n)\\ 
            \vdots & \vdots&  \ddots & \vdots\\
            \sigma(1) &  \sigma(2)&\cdots& \sigma(n)
        \end{bmatrix}.
    \end{align*}
    This finishes the proof.
\end{proof}
\begin{lemma}\label{lem:asymmetric}
    Cycle matrices are not symmetric.
\end{lemma}
\begin{proof}
Consider the cycle set $(X,\cdot)$ arising from the cycle matrix. Note that for $i\neq j$, by the non-degeneracy we have that $i\cdot i\neq j\cdot j$ for all $i\neq j$. If possible let us assume $m_{ij}=m_{ji}$ for some $i\neq j$. Then we have that
\begin{align*}
    (i\cdot j)\cdot (i\cdot i)&=(j\cdot i)\cdot(j\cdot i)\\
    &=(i\cdot j)\cdot (j\cdot j),
\end{align*}
which is not possible, since $i\cdot i\neq j\cdot j$.
\end{proof}
Note that a cycle matrix $(a_{ij})$ is not only non-symmetric but also $a_{ij}\neq a_{ji}$ for all $i,~j.$
\begin{lemma}\label{lem:intranstive-zero-determinant}
	Let $R$ be a commutative ring with unity. If $G\le \Sym_{x_1,x_2,\ldots,x_n}$ acts intransitively on $\{x_1,x_2,\dots,x_n\}$ by the natural action, then the determinant of the $n\times n$ matrix in which each row is an element of $G$, is zero in $R[x_1,x_2,\dots,x_n]$. 
\end{lemma}
\begin{proof}
    Let $M$ be an $n\times n$ matrix in which each row is an element of $G$. The subgroup $H$ of $G$ generated by all the rows, acts intransitively on $X$ by the natural action. For an orbit $\orb_H(x_1)$, there is an element $x_r\in X$ such that $x_r\not\in \orb_H(x_1)$. Let $\mathcal{C}_1$ be the set of all columns such that $\bigcup\limits_{C\in \mathcal{C}_1}C=\orb_H(x_1)$. Similarly, we have $\mathcal{C}_r$. The fact $\orb_H(x_1)\cap \orb_H(x_r)=\emptyset$, implies $\mathcal{C}_1\cap\mathcal{C}_2=\emptyset$. Choose arbitrary elements $C_1\in \mathcal{C}_1$, $C_r\in\mathcal{C}_r$ and perform two elementary column operations on $M$, that are replacement of $C_1$ by $\sum\limits_{C\in\mathcal{C}_1}C$ and $C_r$ by $\sum\limits_{C\in\mathcal{C}_r}C$. The new $C_1$ and $C_r$ columns will be $\sum\limits_{x\in \orb_H(x_1)}x[1,1,\cdots,1]^t$ and $\sum\limits_{x\in \orb_H(x_r)}x[1,1,\cdots,1]^t$, as each row is a permutation. Hence determinant of the matrix $M$ in $R[x_1,x_2,\dots,x_n]$ is $0$. 
\end{proof}
\begin{corollary}\label{cor:non-zero-indecomposable}
    A cycle matrix with a non-zero determinant gives an indecomposable solution.
\end{corollary}
\begin{proof}
    Setting $x_i=i\in\mathbb{Z}$ and $R=\mathbb{Z}$ in \cref{lem:intranstive-zero-determinant}, we get that if the cycle matrix gives a decomposable solution, then the determinant must be zero. Hence the result follows by taking the contrapositive. 
\end{proof}
The converse of the \cref{cor:non-zero-indecomposable} is not true, as evident from the following example.
\begin{example}
We note down the following two matrices. The first one is an indecomposable cycle matrix, although it is of determinant $0$. The second one gives an indecomposable cycle matrix, illustrating the previous corollary.
 \begin{align*}
     \begin{vmatrix}
		4&7&2&1&6&5&8&3\\
		8&1&4&3&2&7&6&5\\
		4&3&2&5&6&1&8&7\\
		2&1&6&3&8&7&4&5\\
		2&5&4&3&8&7&6&1\\
		4&3&8&1&6&5&2&7\\
		2&1&4&7&8&3&6&5\\
		6&3&2&1&4&5&8&7
		\end{vmatrix}=0, 
  \begin{vmatrix}
	4&8&7&1&5&6&3&2\\
	7&1&4&8&3&2&5&6\\
	5&3&2&6&1&4&7&8\\
	2&6&5&3&7&8&1&4\\
	3&6&5&2&8&7&1&4\\
	1&8&7&4&6&5&3&2\\
	7&4&1&8&3&2&6&5\\
	5&2&3&6&1&4&8&7
	\end{vmatrix}=-147456.
 \end{align*}
\end{example}
The following lemma is well known, but we mention it here for completeness.
\begin{lemma}\label{lem:product-cycle-set}
    Let $(X,\cdot)$ and $(Y,\star)$ are two cycle sets, then $(X\times Y,\circ)$ is a cycle set, where $(x_1,y_1)\circ(x_2,y_2)=(x_1.x_2,y_1\star y_2).$ 
\end{lemma}
The cycle set $(X\times Y,\circ)$ is said to be the product of cycle sets $(X,.)$ and $(X,\star)$.
We connect this with the concept of the tensor product of matrices. Recall that, for two matrices $A=(a_{ij})$ and $B=(b_{kl})$, the tensor matrix is a block matrix such that the $ij$-th block is given by
\begin{align*}
    \left(A\otimes B\right)_{ij}=\left(a_{ij}B\right).
\end{align*}
\begin{proposition}\label{prop:tensor-multiplication}
Consider two cycle matrices $A\in\cyc_m$ and $B\in\cyc_n$. If the corresponding cycle sets are given by $(U_m,\cdot)$ and $(U_n,\star)$ respectively, then the matrix $A\otimes B$ is a cycle matrix whose cycle set is isomorphic to $(U_m\times U_n,\circ)$. Hence $A\otimes B\in \cyc_{mn}$
\end{proposition}
\begin{proof}
    Recall from \cref{lem:product-cycle-set}, the operation `$\circ$' on $U_m\times U_n$ is determined by $(a,b)\circ(c,d)=(a\cdot c,b\star d)$, for all $a,c\in U_m$ and $b,d\in U_n$.
    We need to relabel the elements $(i,j)\in U_m\times U_n$, to establish the result.
    Define $\varphi:U_m\times U_n\longrightarrow U_{mn}$ by
    \begin{align*}
        \varphi(i,j)=(i-1)n+j.
    \end{align*}
    Clearly, this function is bijective. Indeed $\varphi(i_1,j_1)=\varphi(i_2,j_2)$ implies that $(j_1-j_2)$ is divisible by $n$, which is not possible. Defining a binary operation `$\bullet$' on
    $U_{mn}$ as
    \begin{align*}
        x\bullet y=\varphi(\varphi^{-1}(x)\circ\varphi^{-1}(y)),
    \end{align*}
    concludes the proof.
\end{proof}

\begin{remark}
Chose any column say $C_j$ of a cycle matrix $(m_{ij})$. Then visit the columns $C_{m_{rj}}$ for all $r=1,2,\cdots,n$ and keep on doing the same process for each $C_{m_{rj}}$. The cycle matrix is indecomposable if and only if all columns can be traversed by this process.
Indeed this is the necessary and sufficient condition for the action of the permutation group $\mathcal{G}$ on $U_n$ to be transitive.
\end{remark}
\begin{corollary}\label{cor:product-indc-imply-indc}
If $(X\times Y,\circ)$ is indecomposable, then both $(X,\cdot)$ and $(X,\star)$ are indecomposable.
\end{corollary}
\begin{proof}
    This is clear from the following inequality
        \begin{align*}
            \orb_{\mathcal{G}(X\times Y)}(x,y)\subset \orb_{\mathcal{G}(X)}(x)\times \orb_{\mathcal{G}(Y)}(y)
        \end{align*}
    for all $(x,y)\in X\times Y$. For the inequality, observe that
    \begin{align*}
        &\orb_{\mathcal{G}(X\times Y)}(x,y)\\
        =&\left\{ \psi^{r_{11}}_{(x_1,y_2)}\psi^{r_{12}}_{(x_1,y_2)}\cdots \psi^{r_{1m}}_{(x_1,y_m)} \psi^{r_{21}}_{(x_2,y_1)}\cdots \psi^{r_{2m}}_{(x_2,y_m)}\cdots \psi^{r_{nm}}_{(x_n,y_m)}(x,y)~\vert~r_{i,j}\in \mathbb{Z}\right\}\\
        =&\left\{\left(\psi^{\sum\limits_{i}^mr_{1i}}_{x_1}\psi^{\sum\limits_{i}^mr_{2i}}_{x_2}\cdots \psi^{\sum\limits_{i}^mr_{ni}}_{x_n}(x),~\psi^{r_{11}}_{y_1}\cdots \psi^{r_{1m}}_{y_m}\cdots \psi^{r_{n1}}_{y_1}\cdots \psi^{r_{nm}}_{y_m}(y)\right)
    \right\}\\
    \subset&\orb_{\mathcal{G}(X)}(x)\times \orb_{\mathcal{G}(Y)}(y)
    \end{align*}
\end{proof}
Note that, the converse of the statement does not hold true. 
We provide the following example exhibiting this scenario. Consider the following two matrices:
\begin{align*}
    \begin{blockarray}{cccc}
& 1 & 2 & 3 \\
\begin{block}{c[ccc]}
  1 & 2 & 3 & 1 \\
  2 & 2 & 3 & 1\\
  3 & 2 & 3 & 1\\
\end{block}
\end{blockarray}, \begin{blockarray}{cccc}
& 1 & 2 & 3 \\
\begin{block}{c[ccc]}
  1 & 3 & 1 & 2 \\
  2 & 3 & 1 & 2\\
  3 & 3 & 1 & 2\\
\end{block}
\end{blockarray}.
\end{align*}
%  Then its product cycle matrix is;
% \begin{align*}
% \begin{blockarray}{cccccccccc}
% & (1,1) & (1,2) & (1,3) & (2,1) & (2,2) & (2,3) & (3,1) & (3,2) & (3,3) \\
% \begin{block}{c(ccccccccc)}
%   (1,1) & (2,3) & (2,1) & (2,2) & (3,3) & (3,1) & (3,2) & (1,3) & (1,1) & (1,2)\\
%   (1,2) & (2,3) & (2,1) & (2,2) & (3,3) & (3,1) & (3,2) & (1,3) & (1,1) & (1,2)\\
%   (1,3) & (2,3) & (2,1) & (2,2) & (3,3) & (3,1) & (3,2) & (1,3) & (1,1) & (1,2)\\
%   (2,1) & (2,3) & (2,1) & (2,2) & (3,3) & (3,1) & (3,2) & (1,3) & (1,1) & (1,2)\\
%   (2,2) & (2,3) & (2,1) & (2,2) & (3,3) & (3,1) & (3,2) & (1,3) & (1,1) & (1,2)\\
%   (2,3) & (2,3) & (2,1) & (2,2) & (3,3) & (3,1) & (3,2) & (1,3) & (1,1) & (1,2)\\
%   (3,1) & (2,3) & (2,1) & (2,2) & (3,3) & (3,1) & (3,2) & (1,3) & (1,1) & (1,2)\\
%   (3,2) & (2,3) & (2,1) & (2,2) & (3,3) & (3,1) & (3,2) & (1,3) & (1,1) & (1,2)\\
%   (3,3) & (2,3) & (2,1) & (2,2) & (3,3) & (3,1) & (3,2) & (1,3) & (1,1) & (1,2)\\
% \end{block}
% \end{blockarray}    
% \end{align*}
From the tensor matrix, it can be seen that $\orb(1,1)=\{(1,1),~(2,3),~(3,2)\}$, which indicates that the product solution is not an indecomposable solution.

Now we record a special kind of cycle matrix, which may be of independent interest. This has a resemblance with Latin squares as was pointed out by L Vendramin. We start with a definition.
\begin{definition}
    A cycle matrix $M$ will be called a \emph{transpose cycle matrix} if the transpose matrix $M^t$ is also a cycle matrix.
\end{definition}
Note that if an $n\times n$ matrix $M=(m_{ij})$ is a transpose cycle matrix, we get that $(X_n,\cdot)$ is also a cycle set where $i\cdot j= m_{ji}$. 
We exhibit one example here.
\begin{example}
For $n=4$, we have the following two examples which give two non-isomorphic square-free cycle sets. The matrices are:
\begin{align*}
    \begin{bmatrix}
    1 & 4 & 3 & 2 \\
   2 & 3 & 4 & 1\\
   4 & 1 & 2 & 3\\
   3 & 2 & 1 & 4
   \end{bmatrix},
   \begin{bmatrix}
    4 & 2 & 3 & 1 \\
   3 & 1 & 4 & 2\\
   1 & 3 & 2 & 4\\
   2 & 4 & 1 & 3    
   \end{bmatrix}.
\end{align*}
\end{example}
\begin{definition}[Transpose cycle set ]
A cycle set $(X,\cdot)$ is said to be a \emph{transpose cycle set} if $(X,\star)$ is a cycle set, where $x\star y=y\cdot x.$
\end{definition}
\begin{lemma}\label{lem:transpose-cycle}
    A cycle set $(X,\cdot)$ is a transpose cycle set if and only if
    \begin{enumerate}
        \item for all $y\in X$, the map $x\mapsto x\cdot y$ is bijective,
    \item for all $x,~y,~z\in X$ we have$(z\cdot x)\cdot (y\cdot x)=(z\cdot y)\cdot (x\cdot y)$.
    \end{enumerate}
\end{lemma}
The proof of \cref{lem:transpose-cycle} is obvious, from the definition of cycle set. We end this section by mentioning the following lemma, which proves the existence of an infinite family of
transpose cycle sets.
\begin{proposition}\label{prop:infinite-transpose-cycle}
The product of two transposed cycle sets is a transposed cycle set.
\end{proposition}
\begin{proof}
Let $(x',y'),~(x,y)\in X\times Y,$ then there is $(x_1,y_1)\in X\times Y$ such that $x_1\cdot x=x'$ and $(y_1\cdot y=y').$ Therefore $(x_1,y_1)\to (x_1,y_1)\circ (x,y)$ is surjective, hence it is bijective. Let $(x_1,y_1),~(x_2,y_2)$ and $(x_3,y_3)\in X\times Y$. Then we have,
\begin{align*}
&\big(((x_1,y_1)\circ (x_2,y_2)\big)\circ \big(((x_3,y_3)\circ (x_2,y_2)\big)\\
=&(x_1\cdot x_2,y_1\cdot y_2)\circ (x_3\cdot x_2,y_3\cdot y_2)\\
=&\big((x_1\cdot x_2)\cdot (x_3\cdot x_2),~(y_1\cdot y_2)\cdot (y_3\cdot y_2)\big)\\
=&\big((x_1\cdot x_3)\cdot (x_2\cdot x_3),(y_1\cdot y_3)\cdot (y_2\cdot y_3)\big)%&\text{from } \cref{lem:transpose-cycle}
\\
=&\big((x_1,y_1)\circ (x_3,y_3)\big)\circ \big((x_2,y_2)\circ (x_3,y_3)\big).
\end{align*}
This finishes the proof.
\end{proof}
\begin{remark}
    All transpose cycle sets are irretractable. Also, each column of a transpose cycle matrix is also a permutation, i.e, $x_i\mapsto \psi_{x_i}(y)$ is a permutation for all $y.$
\end{remark}

