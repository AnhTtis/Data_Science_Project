\section{An action of the symmetric group on set of cycle matrices}\label{sec:sn-action}
In this section, we define an action of the symmetric group $\Sym_n$ on $\cyc_n$ and compute the orbit of a given cycle matrix. We start with an example. Consider two isomorphic cycle matrices given by \begin{align*}
 A=
\begin{blockarray}{cccc}
& 1 & 2 & 3 \\
\begin{block}{c[ccc]}
  1 & 2 & 3 & 1 \\
  2 & 2 & 3 & 1\\
  3 & 2 & 3 & 1\\
\end{block}
\end{blockarray}
 \text{ and }B=
\begin{blockarray}{cccc}
& 1 & 2 & 3 \\
\begin{block}{c[ccc]}
  1 & 3 & 1 & 2 \\
  2 & 3 & 1 & 2\\
  3 & 3 & 1 & 2\\
\end{block}
\end{blockarray}.
\end{align*}
Take the element $\sigma=(1,2)\in\Sym_3.$ Transformation of $A$ to $B$ can be observed as follows.
 \begin{align*}
     A\xrightarrow{\sigma(A)}&
\begin{blockarray}{cccc}
& \sigma(1) & \sigma(2) & \sigma(3) \\
\begin{block}{c[ccc]}
  \sigma(1) & \sigma(2) & \sigma(3) & \sigma(1) \\
  \sigma(2) & \sigma(2) & \sigma(3) & \sigma(1)\\
  \sigma(3) & \sigma(2) & \sigma(3) & \sigma(1)\\
\end{block}
\end{blockarray}
= \begin{blockarray}{cccc}
& 2 & 1 & 3 \\
\begin{block}{c[ccc]}
  2 & 1 & 3 & 2 \\
  1 & 1 & 3 & 2\\
  3 & 1 & 3 & 2\\
\end{block}
\end{blockarray}\\
 \xrightarrow{\sigma^{-1}(R)}& \begin{blockarray}{cccc}
& 2 & 1 & 3 \\
\begin{block}{c[ccc]}
  1 & 1 & 3 & 2 \\
  2 & 1 & 3 & 2\\
  3 & 1 & 3 & 2\\
\end{block}
\end{blockarray}
\xrightarrow{\sigma^{-1}(C)} \begin{blockarray}{cccc}
& 1 & 2 & 3 \\
\begin{block}{c[ccc]}
  1 & 3 & 1 & 2 \\
  2 & 3 & 1 & 2\\
  3 & 3 & 1 & 2\\
\end{block}
\end{blockarray}=B,
 \end{align*}
 where $\sigma^{-1}{R}$ means action of $\sigma^{-1}$ on rows, i.e, $R_i\to R_{\sigma^{-1}(i)}$, similarly $\sigma^{-1}(C)$ for columns.
In general, if $A$ and $B$ are two isomorphic solutions and $\sigma~:~A\to B$ is an isomorphism then the action can be seen to be
$A\xrightarrow{\sigma(A)}A_1\xrightarrow{\sigma^{-1}(R)}A_2\xrightarrow{\sigma^{-1}(C)}B$.
Conversely, if $A$ and $B$ are two matrices corresponding to two solutions $(X,\cdot)$ and $(X,\star)$ and there is a permutation $\sigma$ on $X$ such that $B$ can be obtained from $A$ by the above operations, then $(X,\cdot)$ and $(X,\star)$ are isomorphic and $\sigma$ is an isomorphism. The following lemma is the generalized picture. This is crucial for defining the action. The below lemma is essentially the \cite[Lemma 2.5]{AkMeVe22}.
\begin{lemma}\label{lem:action-of-Sn}
    Let $(X,\cdot)$ be a cycle set and $\sigma$ is a permutation on $X$.
    Then $(X,\star)$ is a cycle set, where $x\star y=\sigma(\sigma^{-1}(x)\cdot\sigma^{-1}(y))$.
    Also $\sigma~:~(X,\cdot)\to (X,\star)$ is an isomorphism. 
    Conversely, all cycle sets isomorphic to $(X,\cdot)$ defined on $X$ come like that.
    In particular, $\sigma$ is an automorphism of $(X,\cdot)$ if and only if the cycle matrices of $(X,\cdot)$ and $(X,\star)$ are the same with $X=\{1,2,\cdots,n\}$.
\end{lemma}
\begin{proof}
    It is easy to check that $(X,\star)$ is a cycle set. We have the relation $x\star y=\sigma(\sigma^{-1}(x)\cdot\sigma^{-1}(y)).$ By replacing $x,~y$ by $\sigma(x),~\sigma(y)$ we obtain $\sigma(x)\star\sigma( y)=\sigma(x\cdot y).$ Hence $\sigma$ is an isomorphism between $(X,\cdot)$ and $(X,\star).$
    
    Conversely assume $(X,\star)$ is isomporphic to $(X,\cdot)$ and $\sigma$ be an isomorphism from $(X,\cdot)$ to $(X,\star)$. Then $\sigma(x)\star \sigma(y)=\sigma(x\cdot y).$ By replacing $x,~y$ by $\sigma^{-1}(x),~\sigma^{-1}(y)$, we get our desired result. The last statement is clear.
\end{proof}
\begin{theorem}\label{th:cycle matrices-solution}
There is a bijective correspondence between the set of orbits $\cyc_n/\Sym_n$ and set-theoretic non-degenerate involutive solutions of QYBE of order $n$.    
\end{theorem}
\begin{proof}
    Note that using \cref{lem:action-of-Sn}, we get that there is a bijection between $\cyc_n/\Sym_n$ and the set of all non-isomorphic cycle sets of order $n$. The assertion follows from the bijection between non-isomorphic cycle sets and set-theoretic non-degenerate involutive solutions of QYBE of order $n$.
\end{proof}

Consider the following matrices:
\begin{align*}
    A= \begin{array}{c|cccccc}
    & 1 & 2 & 3 & 4 & 5 & 6 \\
    \hline
    1 & 1 & 2 & 3 & 5 & 4 & 6\\
    2 & 1 & 2 & 6 & 5 & 4 & 3\\
    3 & 4 & 5 & 3 & 1 & 2 & 6\\
    4 & 2 & 1 & 3 & 4 & 5 & 6\\
    5 & 2 & 1 & 6 & 4 & 5 & 3\\
    6 & 4 & 5 & 3 & 1 & 2 & 6
    \end{array},~B =
    \begin{array}{c|cccccc}
      & 3 & 4 & 5 & 2 & 1 & 6 \\
    \hline
    3 & 3 & 4 & 5 & 1 & 2 & 6\\
    4 & 3 & 4 & 6 & 1 & 2 & 5\\
    5 & 2 & 1 & 5 & 3 & 4 & 6\\
    2 & 4 & 3 & 5 & 2 & 1 & 6\\
    1 & 4 & 3 & 6 & 2 & 1 & 5\\
    6 & 2 & 1 & 5 & 3 & 4 & 6
    \end{array},~C =
    \begin{array}{c|cccccc}
      & 1 & 2 & 3 & 4 & 5 & 6 \\
    \hline
    1 & 1 & 2 & 4 & 3 & 6 & 5 \\
    2 & 1 & 2 & 4 & 3 & 5 & 6 \\
    3 & 2 & 1 & 3 & 4 & 5 & 6 \\
    4 & 2 & 1 & 3 & 4 & 6 & 5 \\
    5 & 4 & 3 & 2 & 1 & 5 & 6 \\
    6 & 4 & 3 & 2 & 1 & 5 & 6
    \end{array}.
\end{align*}
Note that $A$ represents a cycle matrix. Considering $\sigma= (1,3,5)(2,4)$ and applying to each entry of $A$, we get the matrix $B$. After rearranging we get the matrix $C$, which is again a cycle matrix. Hence 
for any permutation of $\{1,2,\ldots, n\}$ and a cycle matrix, we get another cycle matrix.
Generalizing this we define the following. 
\begin{definition}[Symmetric group action on the set of cycle matrices]\label{defn:symmetric-group-action} Consider $\M$ to be a cycle matrix and $(X,\star)$ to be the solution associated with it, i.e. $\M_{ij}=i\star j$. For an element $\sigma\in \Sym_n$ define
\begin{align*}
    \left(\sigma\M\right)_{ij}=\sigma(\sigma^{-1}(i)\star\sigma^{-1}(j)).
\end{align*}
\end{definition}
Note that the above definition of action is well-defined because of \cref{lem:action-of-Sn}. Clearly, we have that $e \M=\M$ for the identity element $e\in\Sym_n$.
For two elements $\alpha,\beta\in\Sym_n$, we have that
\begin{align*}
    \left((\alpha\beta)\M\right)_{ij}
    &=\alpha\beta(\beta^{-1}(\alpha^{-1}(i))\star\beta^{-1}(\alpha^{-1}(j)))\\
    \text{and }\left(\alpha\left(\beta\M\right)\right)_{ij}&=\alpha\left(\left(\beta\M\right)_{\alpha^{-1}(i)\alpha^{-1}(j)}\right)\\
    &=\alpha\beta(\beta^{-1}(\alpha^{-1}(i))\star\beta^{-1}(\alpha^{-1}(j))),
\end{align*}
which clearly proves that it is an action. Using \cref{lem:action-of-Sn} and \cref{defn:symmetric-group-action}, the following result is immediate. We note it down here for further reference.
\begin{lemma}\label{lem:orbit-stabilizer-of-action}Let $|X|=n$. Then two cycle matrices (equivalently their solutions) are isomorphic if and only if they lie in the same orbit of the $\Sym_n$ action. 
Also, $\sigma\in\Sym_n$ is an automorphism of $(X,\cdot)$ if and only if it stabilizes the cycle matrix of $(X,\cdot)$.
\end{lemma}
\begin{lemma}\label{lem:cycle-decomposition}
    If two cycle matrices give isomorphic solutions, then they have the same number of rows with the same cycle decomposition.
\end{lemma}
\begin{proof}
    Let $f$ be an isomorphism between the solutions $(X,\cdot)$ and $(X,\star)$. Then, for any $x,y\in X$ 
    \begin{equation*}
        \begin{split}
            f(x\cdot y) &=f(x)\star f(y)\\
            f\psi_x(y) &=\psi'_{f(x)}f(y)\\
            f\psi_x f^{-1}&=\psi'_{f(x)}.\\
        \end{split}
    \end{equation*}
    Hence $\psi_x$ and $\psi'_{f(x)}$ have the same cycle decomposition for all $x\in X.$ The bijectiveness of $f$ completes the proof.
\end{proof}
Note that the converse of the above lemma is not true.
\begin{example}
	\begin{align*}
		A=\begin{array}{c|cccc}
                &1&2&3&4\\
            \hline
			1&1&2&4&3\\
			2&1&2&4&3\\
			3&1&2&4&3\\
			4&1&2&4&3
		\end{array},~
		B=\begin{array}{c|cccc}
		&1&2&3&4\\
            \hline
            1&1&2&4&3\\
		2&1&2&4&3\\
		3&2&1&3&4\\
		4&2&1&3&4
	\end{array}
	\end{align*}
 These are non-isomorphic solutions simply because their permutation groups are non-isomorphic.
\end{example}
\begin{lemma}\label{lem:automorphism-description}
	Let $\psi_i$ denote the $i$-th row of the cycle matrix with the
 corresponding solution $(X,\cdot)$. Then $\alpha\in\mathfrak{S}_n$ is an automorphism of 
	$(X,\cdot)$ if and only if $\alpha\psi_i=\psi_{\alpha(i)}\alpha$.
\end{lemma}
\begin{proof}
    The proof follows the line of the proof for \cref{lem:cycle-decomposition}. We omit it here, constraining ourselves from repetition.
\end{proof}
\begin{corollary}
Let $(X,\cdot_\sigma)$ be a permutation solution corresponding to a permutation $\sigma\in\Sym_n$. Then $\Aut(X,\cdot_{\sigma})\equiv \cent_{\Sym_n}(\sigma)$.
Furthermore $\orb((X,\cdot_\sigma))$ consists of permutation solutions $(X,\cdot_\delta)$ corresponding to $\delta$, where $\delta$ has cycle-structure as same as $\sigma\in\Sym_n$. 
\end{corollary}
\begin{corollary}
There are exactly $\pa(n)$ many permutation solutions of size $n$ up to isomorphism.
\end{corollary}
% \begin{corollary}
% Let $(X,\cdot)$ be an SYBE. Assume that for the $n\times n$ cycle matrix $\psi_i=I$ for all $1\leq i\leq n-1$. If $I\neq\psi_n=\sigma\in\Sym_n$, then $\Aut(X,\cdot)=\{\pi~:~\pi(n)=n~and~\pi\in \cent_{\Sym_n}(\sigma)\}$   
% \end{corollary}
% \begin{corollary}
%     Let $(X,\cdot)$ be an SYBE. Assume that for the $n\times n$ cycle matrix all the rows are in different conjugacy classes (this might be possible, since $|\pa(n)|\geq n$ for all $n$), then $\Aut(X,\cdot)=\{I\}$.
% \end{corollary}

