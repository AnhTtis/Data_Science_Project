\section{Introduction}\label{sec: intro}
V. G. Drinfeld suggested the study of the set-theoretical solution to the quantum Yang-Baxter equation (QYBE) (see \cite{Dr90}).
A \emph{set-theoretical solution to the QYBE} is an ordered pair $(X,R)$, where $X$ is a set and $R:X\times X\longrightarrow X\times X$  is a map, satisfying
\begin{align*}
    R^{12}R^{13}R^{23}=R^{23}R^{13}R^{12},
\end{align*}
where $R^{ij}:X\times X\times X\longrightarrow X\times X\times X$ is the map acting on the $(i,j)$-th position by $R$.
This has been an important research topic for the last two decades after the breakthrough \cite{EtScSo99} by P. Etingof, T. Schedler and A. Soloviev.
Although such solutions were Constructed by A. D. Weinstein and P. Xu (\cite{WeXu}) and by J. H. Lu, M. Yan and Y. C. Zhu (see \cite{LuYaZh00}) independently, 
the paper by Etingof \emph{et. al.} further study solutions with additional conditions of \emph{nondegeneracy} of $R$ (i.e. if $R(x,y)=(g_x(y),f_y(x))$ then $g_x,f_y$ are bijective for all $x,y\in X$) and $R$ being involutive (i.e. $R^2=\text{Id}_{X\times X}$) and derive powerful consequences. In this current article by a \emph{solution}, we will mean finite non-degenerate involutive set-theoretical solution to the QYBE. Before proceeding further we will mention a few definitions. The \emph{permutation group of a solution} is the group defined as
\begin{align*}
    \mathcal{G}=\langle f_x:x\in X\rangle.
\end{align*}
A solution $(X,R)$ is said to be \emph{decomposable} if there exists disjoint subsets $X_1,X_2$ of $X$ such that $R(X_i\times X_i)\subseteq X_i\times X_i$, $(X_i,R\vert _{X_i\times X_i})$ is a solution and $X=X_1\bigsqcup X_2$. 
The solution will be called \emph{indecomposable} in case of the non-existence of such a pair $X_1,~X_2$, also, it is equivalent that the natural action of the permutation group $\mathcal{G}$ on $X$ is transitive \cite{EtScSo99}.
In the paper \cite{EtScSo99}, the authors have shown that there exists a unique (up to
isomorphism) indecomposable solution order $p$, where $p$ is
a prime (See section $2.5$ and $2.6$ therein). After that several attempts have been made to look after and analyze solutions in case $|X|$ is a composite number.
For example, W. Rump has studied the decomposability of square-free solutions in \cite{Ru05}, S. Ram\'irez and L. Vendramin have studied decomposability for solutions in \cite{RaVe22}.
Recently in an astonishing breakthrough Agore, Chirvasitu and, Militaru have proved many counting results for different classes of solutions using the category of pointed Kimura semigroups (see \cite{AgChMi23}), e.g. the number of isomorphism classes of all left non-degenerate indecomposable solutions on $X_n$ is equal to the Harary number $\mathfrak{c}(n)$, number of non-isomorphic non-degenerate solution is the Davis number $d(n)$. Our results give glimpses into the structure of solutions decomposable solutions in terms of cycle matrices.

The concept of cycle sets was introduced by W. Rump in \cite{Ru05} and has been a source of several new solutions of QYBE (see \cite{Jes16}, \cite{Rum22} and the references therein). A \emph{cycle set} is a tuple
$(X,\cdot)$ such that 
the map $\psi_x:y\mapsto x\cdot y$ is invertible, and
\begin{align*}
    (x\cdot y)\cdot (x\cdot z)=(y\cdot x)\cdot(y\cdot z),
\end{align*}
for all $x,y,z\in X$. A cycle set will be called \emph{non-degenerate} if the map $\varphi:x\mapsto x\cdot x$ is bijective. Finally, call a cycle set to be 
\emph{square-free} if $\varphi$ is the identity map. 
An important result about cycle sets states that there is a bijection between non-degenerate cycle sets and set of all solutions of QYBE (\cite[Proposition $1$]{Ru05}).
A \emph{homomorphism} $f$ from $(X,\cdot)$ to $(Y,\bullet)$ is a set theoretical map $f:X\longrightarrow Y$ such that $f(x\cdot y)=f(x)\bullet f(y)$ for all $x\in X$ and $y\in Y$. A bijective homomorphism from $(X,\cdot)$ to itself will be called as an \emph{automorphism}. The set of all automorphisms of $(X,\cdot)$ will be
denoted by $\Aut(X,\cdot)$. \emph{We further restrain ourselves to solutions coming from cycle sets.}
 A solution $(X,\cdot)$ is called a \emph{permutation solution} if, for any $x\in X,$ $\psi_x=\sigma$ for some permutation $\sigma$ and,
it will be called a \emph{trivial solution} if $\sigma=id$. We will denote the permutation solution by $(X,\cdot_\sigma)$. A solution is \emph{irretractable} if the natural map $x\mapsto \psi_x$ is injective, otherwise \emph{retractable}. The relation, $x\sim y$ if and only if $\psi_x=\psi_y$, is an equivalence relation, the equivalence class of $x$ will be denoted by $\bar{x}$, and the set of equivalence classes will be denoted by $\bar{X}$ or $Ret(X)$. In \cite[pp. 157]{Ru07}, it is shown that $\Bar{x}\cdot\Bar{y}:=\overline{x\cdot y}$ defines a cycle set on $\bar{X}.$ A retractable solution is a \emph{multipermutation of level $n$}, if $n$ is the least positive integer such that $|Ret^n(X)|=1.$
\subsection*{Notations} Before proceeding further we set some notations. The group of all bijections of $n$ indeterminate $x_1,x_2,\cdots,x_n$ will be denoted by $\Sym_{x_1,x_2,\ldots,x_n}$. In case $x_i=i$, this group will be identified with the symmetric group on $n$ letters and will be denoted by $\Sym_n$. By $\Sn(i_1,i_2,\ldots,i_k)$ we mean the set of all bijections of $\{i_1,i_2,\ldots,i_k\}$. For a group action $G\curvearrowright Y$, the orbit of an element $y\in Y$ will be denoted as $\orb_G(y)$. The set $\{1,2,\ldots, m\}$ will be denoted as $U_m$. The centralizer of an element $g\in G$ will be denoted by $\cent_{G}(g)$. The number of partitions of a positive number $n$ will be denoted by $\mathscr{P}(n)$. A row of a matrix will be written in square bracket $[~]$ and for a column $[~]^t$ will be used.
 Other notations are standard.
\subsection*{Organization of the paper} The paper is organized as follows; In \cref{sec:general-results} we define the main object of the study, the cycle matrix. We observe some properties of the cycle matrix and prove that the cycle matrices which give rise to decomposable solutions are singular. We have shown that the product of solutions has correspondence with the tensor product of cycle matrices. We define a transpose cycle matrix, which corresponds to a special kind of irretractable solution. We construct a collection of such cycle matrices. In \cref{sec:sn-action} we define an action of the symmetric group $\Sym_n$ on the set of all cycle matrices. Using this action we prove that the number of permutation solutions of order $n$ (up to isomorphism) is the number of partitions of $n$ and $\Aut(U_n,\cdot_\sigma)$ is the centralizer of $\sigma$ in $\Sym_n.$ \cref{sec:construction-old-new} is devoted to the construction of new solutions, firstly we start with two trivial solutions $(U_n,\cdot_{id}),~(U_m,\cdot_{id})$ and construct a collection of solutions of multipermutation level $2$ on $U_{n+m}$ with respect to any partition of $U_n$. As a consequence of the construction we obtain the well-known result \cite{CeJeDe10} ``all finite  abelian groups are permutation groups". Next we start with a finite (say $r$) collection of solutions $(U_{m_i},\cdot_i)$ and a permutation $\sigma\in \Sym_r$ and have constructed a solution on $U_{\sum m_i}.$ In \cref{sec:multiperm-solutions}, we have proved the existence of a multipermutation solution of level $n$, for $n\ge 1.$      

