\section{Construction of multi-permutation solutions}\label{sec:multiperm-solutions}
This section is devoted to the construction of multipermuatation solutions of level $r$ for any $r\geq 1$. These have been studied in the work of Gateva-Ivanova, Cameron (\cite{GICa12}), Jedli\v{c}ka, Pilitowska, and Zamojska-Dzienio (\cite{JePiZa20}) and finally a classification result for second level solutions in \cite{Rum22} by Rump. They are important because, a finite solution $(X,r)$ has the structure group $\mathcal{G}(X,r)$ to be poly-$\mathbb{Z}$ if and only if it is a multipermutation solution (see \cite{JeOk05} and \cite{BaCeVe18}).   
\begin{proposition}\label{prop:multiperm-2}
Consider $X_2$ to be the identity solution and $|X_2|=2$. Construct
\begin{align*}
    X_{2^2}=\left[\begin{array}{c|c}
         X_2&\sigma_2  \\
         \hline
         \sigma_2&X_2
    \end{array}\right],
\end{align*}
where $\sigma_2=(1,2)$. This gives a multipermutation solution of level $2$.
\end{proposition}
\begin{proof}
    From \cref{lem:automorphism-description}, $\sigma_2$ is an automorphism of $X_2.$ Now it follows from \cref{cor:multipermutation 2} that $X_{2^2}$ is a multipermutaion solution of level $2$. 
\end{proof}
\begin{theorem}\label{thm:multiperm-m}
For all $m\geq 1$, with notation as of \cref{prop:multiperm-2} the matrix defined as
\begin{align*}
    X_{2^{m+1}}=\left[\begin{array}{c|c}
         X_{2^{m}}&\sigma_{2^m}  \\
         \hline
         \sigma_{2^m}&X_{2^{m}}
    \end{array}\right],
\end{align*}
with $\sigma_{2^m}=\prod\limits_{i=1}^{2^{m-1}}(i,i+2^{m-1})$ gives a multipermutation level $m+1$ solution of YBE.
\end{theorem}
\begin{proof}
It is easy to see for $m=1.$ For $m=2;$
\begin{align*}
    X_{2^{3}}=&\left[\begin{array}{cccc|cccc}
         1&2&4&3&7&8&5&6  \\
         1&2&4&3&7&8&5&6\\
         2&1&3&4&7&8&5&6\\
         2&1&3&4&7&8&5&6\\
         \hline
         3&4&1&2&5&6&8&7\\
         3&4&1&2&5&6&8&7\\
         3&4&1&2&6&5&7&8\\
         3&4&1&2&6&5&7&8
    \end{array}\right]\xrightarrow{Ret(X_{2^3})}
    \left[\begin{array}{cc|cc}
         \bar{1}&\bar{2}&\bar{4}&\bar{3} \\
         \bar{1}&\bar{2}&\bar{4}&\bar{3}\\
         \hline
         \bar{2}&\bar{1}&\bar{3}&\bar{4}\\
         \bar{2}&\bar{1}&\bar{3}&\bar{4}
    \end{array}\right]=X_{2^{2}}\\\xrightarrow{Ret^2(X_{2^3})}
        &\left[\begin{array}{cc}
         \bar{\bar{1}}&\bar{\bar{2}}\\
         \bar{\bar{1}}&\bar{\bar{2}}
    \end{array}\right]\xrightarrow{Ret^3(X_{2^3})}
    [\Bar{\Bar{\Bar{1}}}],
\end{align*}
where $\Bar{1}=\{1,2\},\Bar{2}=\{3,4\},\Bar{3}=\{5,6\},\Bar{4}=\{7,8\},\bar{\bar{1}}=\{\Bar{1},\Bar{2}\},\bar{\Bar{2}}=\{\Bar{3},\Bar{4}\}~and~\Bar{\Bar{1}}=\{\Bar{\Bar{\bar{1}}},\Bar{\Bar{2}}\}.$ From \cref{prop:multiperm-2}, \cref{lem:automorphism-description} and \cref{prop:2-times-2}, it follows that $X_{2^3}$ is cycle-matrix. Therefore $X_{2^3}$ is a multipermuation level $3$ cycle-matrix.\\
It is clear that in $X_{2^{m+1}},$ $\psi_{2i-1}=\psi_{2i}$  for $i=1,2,\cdots,2^{m}$ and $\psi_{2i}\neq \psi_{2i+1}$ for $i=1,2,\cdots,2^m-1$, therefore $\Bar{i}=\{2i-1,2i\}$ for $i=1,2,\cdots,2^{m}$. Notice that $\psi_i(2j-1),~\psi_i(2j)$ lie in the same class. The matrix $X_{2^{m+1}}$ has four blocks say $B_{ij}$ and each of order $2^{m}\times 2^{m}$. Each row in $B_{12}$ and in $B_{21}$ is, $$[2^{m}+2^{m-1}+1,2^{m}+2^{m-1}+2,\cdots,2^{m+1},2^{m}+1,2^m+2,\cdots, 2^{m}+2^{m-1}]~~and$$ $$[2^{m-1}+1,2^{m-1}+2,\cdots,2^{m},1,2,\cdots, 2^{m-1}]$$ respectively. With this observation and the fact that $\psi_{\Bar{i}\cdot\Bar{j}}=\psi_{\overline{i.j}}$ gives,
$$Ret(X_{2^{m+1}})=X_{2^m}.$$ Now the proof follows from induction.
\end{proof}
\subsection*{Acknowledgement} We thank Prof. Arvind Ayyer (IISc, Bangalore) and, Prof. Manoj Kumar Yadav (HRI, Prayagraj) for their interest in this work.
%We thank the referees for their valuable comments, which improved the exposition of the paper.