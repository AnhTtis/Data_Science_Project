The measured distributions are normalized to the total number of jets in a given \pTchjet{} interval, $N$, which is equivalent to reporting the self-normalized cross section:
\begin{equation}
\frac{1}{N(\pTchjet)} \frac{{\rm d}N}{{\rm d}\DeltaR} \left(\pTchjet\right)
\equiv
\frac{1}{\sigma(\pTchjet)} \frac{{\rm d}\sigma}{{\rm d}\DeltaR} \left(\pTchjet\right).
\end{equation}
This normalization choice causes the \PbPb/pp{} ratios to go above unity at low $\DeltaR$. In an absolute cross section measurement, the  ratio would be below unity as a result of jet suppression in heavy-ion collisions. 
The suppression factor was measured in Ref.~\citenum{ALICE:RAA23}.
Thus, only the shape and not the absolute scale of the \DeltaR{} spectra is reported.

\begin{figure}
    \centering
    \includegraphics[width=0.497\textwidth]{fig/h_jet_axis_0-10_R02_Standard_WTA_40_60Theory.pdf}
    \includegraphics[width=0.497\textwidth]{fig/h_jet_axis_0-10_R02_Standard_WTA_60_80Theory.pdf}
    \caption{Top panels: fully corrected \PbPb{} and \pp{} $\DeltaR^{\rm WTA-Standard}$ distributions in the \pTchjet{} intervals $[40,60]$ (left), and $[60,80]$ (right) \GeVc{} for jets of $R=0.2$.
    The \pp{} baseline is taken from Ref.~\citenum{ALICE:2022rdg}.
    Central and bottom panels: measured \PbPb/\pp{} ratio in black, as well as predictions from a selection of jet quenching models.
    }
    \label{fig:R02_WTA_Standard}
\end{figure}

The measured WTA$\textendash$Standard distributions are shown in Fig.~\ref{fig:R02_WTA_Standard} for $R=0.2$ jets in the \pTchjet{} ranges $[40,60]$ and $[60,80]$ \GeVc. The top panels show \DeltaR{} spectra from \PbPb{} and \pp{} collisions.
The vertical error bars represent the statistical uncertainties and the rectangles represent the total systematic uncertainties.
The central and bottom panels show the \PbPb/\pp{} ratio, with all uncertainties assumed uncorrelated and added in quadrature when calculating the ratio.
The equivalent results for other \pTchjet{} ranges, jet resolution parameters, and grooming settings are included in the supplemental materials~\cite{JA:sup}.

The data are compared with several jet quenching models. These models have different implementations of the microscopic properties of the medium, its evolution, and the jet\textendash{}medium interaction.
% ===========================================
% JEWEL
The \textbf{JEWEL} event generator~\cite{Zapp:2013vla}
models the medium with a boost-invariant longitudinally expanding ideal quark$\textendash$gluon gas~\cite{Zapp:2012ak}. Parameters from Ref.~\citenum{KunnawalkamElayavalli:2017hxo}, which are adequate for the kinematics of this measurement, are used.
The medium partons recoiling after interacting with jet constituents can be discarded from the event (recoils off) or
allowed to hadronize together with the jet (recoils on).
% ===========================================
% JETSCAPE
\textbf{``MATTER+LBT''} 
is from the JETSCAPE event generator~\cite{Putschke:2019yrg,JETSCAPE:2022jer}, implementing an in-medium parton shower with interactions of high- (low-) virtuality partons with the medium described by the MATTER~\cite{Majumder_2013} (Linear Boltzmann Transport~\cite{LBT}) model.
% ===========================================
% Calculations by F.Ringer, F.Yuan, et al.
The curve labeled \textbf{``medium q/g''} corresponds to a phenomenological model in which the only difference between the \PbPb{} and \pp{} results comes from a modification of the fraction of quarks and gluons that initiate the jets~\cite{Qiu:2019sfj,Qiu:2020pus}, highlighting that these two jet populations lose energy differently in the medium.
The \textbf{``$\bm{p_{\rm T}}$ broadening''} calculation adds to the previous model a \pT{} broadening caused by incoherent multiple scatterings with the medium partons~\cite{Ringer:2019rfk} following the BDMPS approach~\cite{Baier:1996kr,Baier:1996sk,Baier:1998kq}.
This calculation uses a mean square momentum transfer coefficient between the jet and medium constituents $\langle \hat q L \rangle=5$ GeV$^2$.
% ===========================================
% Hybrid (Daniel Pablos Alfonso et al.)
\textbf{``Hybrid''} is from the hybrid model~\cite{Casalderrey-Solana:2014bpa}, which describes the weakly coupled jet showering process using the DGLAP formalism and the strongly coupled interaction between the jet constituents and medium partons via holographic calculations of energy loss based on AdS/CFT.
The case $L_{\rm res}=0$ corresponds to fully incoherent energy loss, where the medium is able to resolve the splitting immediately after the parton fragments, resulting in higher energy loss~\cite{Hulcher:2017cpt}.
The case $L_{\rm res}=\infty$ corresponds to fully coherent energy loss, where the medium is not able to resolve splittings, 
and the energy loss is lower.
The intermediate case $L_{\rm res}=2/\uppi T$, where $T$ is the medium local temperature, is also included. 
All ``Hybrid'' curves include medium-response effects from Ref.~\citenum{Casalderrey-Solana:2016jvj}.

% ===========================================

The measured $R=0.2$ distributions are narrower than those measured in \pp{} collisions. Results with $R=0.4$, presented in the supplemental materials~\cite{JA:sup}, are consistent with this narrowing, but also with no modification within statistical and systematic uncertainties. The narrowing may be due to a selection bias, as jets selected using \pTchjet{} 
(rather than the initial momentum before the jet-medium interactions, which is not accessible experimentally) tend  to lose little energy in the QGP~\cite{Brewer:2021hmh}. Similarly, the experimental result is also consistent with a quark-enhanced mixture of jets surviving the interaction with the medium relative to the vacuum case, as expected since 
gluon-initiated jets couple more strongly to partons of the QGP and thus lose more energy~\cite{Qiu:2019sfj,Spousta:2015fca,ALICE:2021obz}.
The MATTER+LBT model overall describes the data for all considered kinematical regions.

In contrast with $R=0.2$, for larger $R$ there are significant differences in JEWEL predictions with and without recoils. This is expected, as the medium response 
populates larger angles away from the Standard jet axis. However, grooming reduces the sensitivity to the medium response.
The Hybrid model does not predict a significant effect from the medium response. See the supplemental materials~\cite{JA:sup} for details. The inclusion of other models in which the medium response can be turned on and off (e.g. Co-LBT~\cite{Chen:2017zte,Zhao:2021vmu}) could help clarify the sensitivity of $\DeltaR$ to the properties and magnitude of this effect.
Narrowing of the distributions relative to those measured in \pp{} collisions for jets of $R=0.2$ is qualitatively predicted by all models except for ``$p_{\rm T}$ broadening''. This excludes the hypothesis that intra-jet \pT{} broadening manifests in the groomed radiation in previous measurements~\cite{ALICE:2021obz}.

The data are 
closest to incoherent energy loss of jets in the Hybrid model ($L_{\rm res}=0$).
The fully coherent case ($L_{\rm res}=\infty$) predicts significantly less narrowing.
Fully incoherent energy loss better reproduces the data than the intermediate case of $L_{\rm res}=2/\uppi T$. In holographic calculations of the Debye screening length, the intermediate case 
is expected to be closer to the true medium resolution scale~\cite{Hulcher:2017cpt}.

\begin{figure}
    \centering
    \includegraphics[width=0.497\textwidth]{fig/h_ratio_Standard_WTA_R04_02_100-140.pdf}
    \includegraphics[width=0.497\textwidth]{fig/h_ratio_WTA_SD_5_R04_02_100-140.pdf}
    \caption{Top panels: fully corrected \PbPb{} $\DeltaR^{\rm WTA-Standard}/R$ (left), and $\DeltaR^{\rm WTA-SD}/R$ with $(\zcut=0.2,\beta=0)$ (right) distributions in the \pTchjet{} interval $[100,140]$ \GeVc.
    Central and bottom panels: measured $\PbPb (R=0.4)/\PbPb(R=0.2)$ ratio in black, as well as predictions from a selection of  jet quenching models.
    }
    \label{fig:100_140}
\end{figure}

The precision of the data at higher \pTchjet{} is limited by the integrated luminosity of the \pp{} dataset, which only allows for an extraction of the \PbPb/\pp{} results up to 100 \GeVc. However, in heavy-ion collisions, larger \pTchjet{} can be studied through the ratio of $(R=0.4)/(R=0.2)$ spectra, where the systematic uncertainties partially cancel. This kinematic range is also important for future comparisons with results from other experiments which cannot access the low-\pT{} regime of this measurement.
The ratio is reported as a function of the jet angular scale $\DeltaR/R$ that, by construction, is independent of $R$. Figure~\ref{fig:100_140} shows the ratio for $\pTchjet \in [100,140]$  \GeVc{} together with theoretical predictions.
The ungroomed WTA$\textendash$Standard result is challenging to models, as can be seen
in Fig.~\ref{fig:100_140} (left).
Grooming improves the agreement between the models and the data (see Fig.~\ref{fig:100_140} (right)), suggesting that the ungroomed result captures significantly more of the non-perturbative aspects of jet quenching.

Comparing $\DeltaR^{\rm WTA-Standard}$ to $\DeltaR^{\rm WTA-SD}$ with different grooming settings shows that grooming does not significantly impact the experimental results. This is similar to \pp{} collisions, where the Standard and SD axes were found more aligned with each other than either of them with the WTA axis. While the Standard axis is influenced by the soft wide-angle radiation groomed away in the SD-axis case, it is dominated by the hard radiation that remains in the SD case.

