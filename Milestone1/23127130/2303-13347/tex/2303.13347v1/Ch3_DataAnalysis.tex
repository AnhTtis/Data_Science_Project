The large uncorrelated background from the underlying event (UE)~\cite{ALICE:2012nbx} in high-multiplicity heavy-ion collisions is subtracted before jet finding using the event-wide constituent subtraction procedure from Refs.~\citenum{Berta:2014eza} and~\citenum{Berta:2019hnj}. The parameters used are $\alpha=0$, $A_{\rm g} = 0.01$, and $\Delta R^{\rm max}= 0.1$ (0.25) when the track sample is used to cluster jets of $R=0.2$ (0.4).

Tracks included in jet clustering are assigned the charged-pion mass. Charged-particle jets are clustered with the \antikT{} algorithm, the $E$ recombination scheme, and resolution parameters $R=0.2$ and 0.4, using FastJet~\cite{Cacciari:2011ma}. Jets containing a track with $\pT>100 \ \GeVc$ and jets with their Standard axis not pointing inside the fiducial volume of the TPC ($|\etajet|<0.9-R$) are discarded.

The measured distributions are unfolded to correct for detector effects as well as residual UE fluctuations using an iterative Bayesian procedure~\cite{DAgostini,DAgostini:2010hil}. The procedure uses a 4D response matrix (RM) that captures the correspondence between \DeltaR{} and jet transverse momentum (\pTchjet) at the generator and detector levels. The generator-level component of the RM is modeled using generated PYTHIA 8~\cite{Sjostrand:2014zea} Monash 2013~\cite{Skands:2014pea} events. The detector-level component of the RM is modeled by propagating those events through a GEANT~3~\cite{Brun:1119728} model of the ALICE detector and embedding the smeared events into measured \PbPb{} events.

