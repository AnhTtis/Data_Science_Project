The data were collected in 2018 from \PbPb{} collisions at a center-of-mass energy per nucleon$\textendash$nucleon pair of $\sqrts = 5.02$ TeV with the ALICE detector~\cite{ALICE:2022wpn}. A detailed description of the detector and its performance can be found in Refs.~\citenum{ALICE:2008ngc} and~\citenum{ALICE:2014sbx}. A trigger based on the charged-particle multiplicity
in the forward region~\cite{ALICE:2013axi} was used to select events in the 0$\textendash$10\% centrality class~\cite{ALICE:2013hur,ALICE:2018tvk}. The events were required to have a reconstructed primary vertex within $\pm 10$ cm of the nominal interaction point in the longitudinal direction. Beam-induced background events were removed using two Zero Degree Calorimeters and pileup effects were removed by discarding events that contained multiple reconstructed vertices~\cite{Acharya:2019jyg}. These selection criteria yield a data sample with 92 million events, corresponding to an integrated luminosity of $\approx 0.12 \ {\rm nb}^{-1}$.
 
Charged-particle tracks reconstructed in the Inner Tracking System (ITS)~\cite{ALICE:2010tia} and Time Projection Chamber (TPC)~\cite{Alme:2010ke} used in the analysis were required to have a transverse momentum $\pT > 0.15 \ \GeVc$ and a pseudorapidity $|\eta|<0.9$. See the supplemental materials~\cite{JA:sup} for details.
The detector performance was estimated by propagating simulated events through a GEANT~3~\cite{Brun:1119728} model of the ALICE detector.
The transverse-momentum resolution $\sigma(\pT)/\pT$ increases linearly from $\approx 1\%$ at $\pT=1 \ \GeVc$ to $\approx4\%$ at $\pT=50 \ \GeVc$~\cite{ALICE:2014sbx}. The tracking efficiency rapidly increases from $\approx 65\%$ at $\pT=0.15 \ \GeVc$ to $\approx 85\%$ at $\pT=1 \ \GeVc$, and remains above $\approx 75\%$ at higher \pT.