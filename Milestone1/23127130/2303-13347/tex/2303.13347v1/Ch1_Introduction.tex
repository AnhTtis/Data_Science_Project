The strongly interacting, deconfined state of matter of our universe a few microseconds after the Big Bang
can be recreated in ultrarelativistic heavy-ion collisions at the Large Hadron Collider (LHC)~\cite{Busza:2018rrf,ALICE:2022wpn}. Studying the microscopic structure of this quark$\textendash$gluon plasma (QGP) can elucidate the physics of many-body strongly coupled systems~\cite{Jacak:2012dx}.
Because droplets of QGP created at colliders are small and short-lived, studying interactions within those droplets requires auto-generated probes. In this study, jets of energetic hadrons, which arise from high-momentum-transfer (hard) collisions between quarks and gluons (partons) in the incoming nuclei, are used. These hard-scattered particles produce showers of secondary partons, which give rise to jets of multiple correlated hadrons (see Fig.~\ref{fig:cartoon}, left).

\begin{figure}[h]
    \centering
    \includegraphics[width=0.9\textwidth]{fig/Cartoon_v3_cropped.pdf}
    \caption{Left: schematic of how energetic partons split or fragment into lower-energy prongs or branches of partons down to a scale in which hadrons emerge. An algorithm can be run over final-state hadrons to cluster them into jets. An inverse declustering algorithm can be used to recursively explore the jet substructure. Right: schematic showing that different jet axes (solid arrows) can be selected from a jet to construct the $\DeltaR$ observable. The Standard axis (represented by a black arrow) is constrained by all the jet constituents (represented by dashed arrows). The SD axis (represented by a dark blue arrow) is determined by the particles left in the jet after grooming (blue and dark red arrows). Finally, the WTA axis (represented by a pink arrow) tends to be aligned with the most energetic constituent.}
    \label{fig:cartoon}
\end{figure}

When traversing the QGP, jets  interact with the medium, lose energy, and are modified relative to their structure in elementary (e.g., ee or pp) collisions. 
Thus, measuring the substructure of jets can provide sensitivity to different scales within the jet modification process. 
The direction of a jet in rapidity ($y$) and azimuth ($\varphi$), referred to as the jet axis, 
is affected by transverse modification of the partonic shower, with varying sensitivity to soft radiation, depending on how the axis is defined~\cite{Cal:2019gxa}.

This letter presents the first measurement in heavy-ion collisions of the angle between pairs of jet axes,  $\DeltaR$ (see Fig.~\ref{fig:cartoon}, right). This observable shows how well the QGP resolves the angular structure of the parton shower~\cite{Cal:2019gxa} and is defined as
\begin{equation}
    \Delta R_{\rm axis} \equiv \sqrt{(y_{\rm axis, \, 1}-y_{\rm axis, \, 2})^2+(\varphi_{\rm axis, \, 1}-\varphi_{\rm axis, \, 2})^2},
\end{equation}
where $y_{{\rm axis}, \, i}$ and $\varphi_{{\rm axis}, \, i}$ are the rapidity and azimuthal coordinates of the axes, respectively.
The axes considered in this analysis are:

\begin{itemize}

\item ``Standard'': coordinates in $y\textendash\varphi$ of the 4-vector resulting from clustering jets with the \antikT{} algorithm~\cite{Cacciari:2008gp} using the $E$ recombination scheme (two branches combined into one by adding their 4-momenta). In this case, all jet constituents including soft (low-transverse-momentum) and hard radiation contribute to the axis.

\item 
``Soft Drop'' (SD): coordinates in $y\textendash\varphi$ of jet after soft wide-angle radiation is removed using 
the Soft Drop grooming procedure~\cite{Larkoski:2014wba} with the Cambridge$\textendash$Aachen (C/A) reclustering algorithm~\cite{Dokshitzer:1997in,Wobisch:1998wt}. The
procedure stops when a splitting is found that satisfies the condition
\begin{equation}
    \frac{\mathrm{min}(p_{\rm T,1},p_{\rm T,2})}{p_{\rm T,1}+p_{\rm T,2}}>\zcut\left(\frac{\Delta R_{1,2}}{R}\right)^{\beta},
    \label{eq:sd}
\end{equation}
where $p_{{\rm T},i}$ are the transverse momenta of each of the two branches in the splitting, $\Delta R_{1,2}$ is the distance between them in the $y\textendash\varphi$ plane, $R$ is the jet resolution parameter, and \zcut{} and $\beta$ are 
user-set parameters.

\item ``Winner-Takes-All'' (WTA): coordinates in $y\textendash\varphi$ of jet after reclustering its constituents using the C/A algorithm and the WTA \pT{} recombination scheme~\cite{Bertolini:2013iqa}, which combines two jet branches into one by giving the merged branch the total \pT{} of the two sub-branches and the direction of the hardest sub-branch.
The WTA axis is often consistent with the direction of the hardest constituent of the jet and is insensitive to soft radiation at the leading power of the observable~\cite{Cal:2019gxa}.

\end{itemize}

Measurements of $\DeltaR$ are sensitive to medium-induced effects. For example, an increase in the relative fraction of quark- vs. gluon-initiated jets due to medium-induced radiative energy loss can narrow the angular distribution, while multiple scattering of the hard-scattered parton off medium partons can broaden it~\cite {Ringer:2019rfk}.
Reference~\citenum{ALICE:2021obz} reported a net narrowing of the distribution in \PbPb{} collisions, quantitatively consistent with the former.
This can be expected if the broadening was primarily in the wide-angle radiation that grooming removes from the jet. This is tested by looking for broadening effects in an ungroomed angular observable, such as the angle between the Standard and WTA axes.

Measurements of $\DeltaR$ are also sensitive to the wake created by a jet imparting some of its momentum to the QGP. This wake is often referred to as medium response~\cite{Milhano:2017nzm} and
is expected to dominate at large angles from the Standard axis. Consequently,  progressively increasing the grooming intensity,
while simultaneously varying the jet resolution parameter $R$ can uncover the medium response effect. Comparing experimental results with different models is expected to disentangle medium response and medium-induced large-angle radiation.

The medium resolution length, $L_{\rm res}$, is a property of the QGP proportional to the inverse of its local temperature~\cite{Hulcher:2017cpt}. It characterizes the minimum separation distance at which two color charges within the same jet interact independently with the QGP. Certain groomed substructure observables are sensitive to this scale~\cite{Casalderrey-Solana:2019ubu}. In this analysis, the sensitivity of $\DeltaR$ to $L_{\rm res}$ is investigated.

The measured $\DeltaR$ distributions in \pp{} collisions at $\sqrt{s} = 5.02 \ {\rm TeV}$ reported in Ref.~\citenum{ALICE:2022rdg} are used as the reference for QGP-induced effects reported here.
Perturbative QCD calculations~\cite{Cal:2019gxa} were found to agree with the measurements in \pp{} collisions within the experimental uncertainties, demonstrating good theoretical control of the observables.