    %\documentclass[manuscript]{biometrika}
%\documentclass[11pt, oneside]{biometrika} # FORMAT CHANGE
\documentclass[12pt]{article}
\usepackage[left = 1 in, right = 1 in, top = 1 in, bottom = 1.5 in]{geometry}              
\geometry{letterpaper}  
%\usepackage{amsmath}

%% Please use the following statements for
%% managing the text and math fonts for your papers:
%\usepackage{times}
%\usepackage[cmbold]{mathtime}
%\usepackage{bm}
%\usepackage{natbib}

%\usepackage[plain,noend]{algorithm2e}
%%%%%%%%%%%%%%%%%%
\usepackage[utf8]{inputenc}
\usepackage{amsmath, amsthm, amssymb,natbib,xcolor,multicol,url}
\usepackage{graphicx,bbm}
\usepackage{booktabs,epstopdf,color}
\usepackage[space]{grffile}
\usepackage{lineno,comment}
\usepackage[plain,noend]{algorithm2e}
\usepackage{multirow}

\usepackage{hyperref}

\usepackage{amssymb}
%%% mine
% --- defs --- %
\def\m{\mathcal}
\def\mb{\mathbb}
\def\mr{\mathrm}
\def\ms{\mathscr}
\def\ind{\mathbbm{1}}
\def\wt{\widetilde}
\def\wth{\widehat}
%%% mine


\makeatletter
\renewcommand{\algocf@captiontext}[2]{#1\algocf@typo. \AlCapFnt{}#2} % text of caption
\renewcommand{\AlTitleFnt}[1]{#1\unskip}% default definition
\def\@algocf@capt@plain{top}
\renewcommand{\algocf@makecaption}[2]{%
  \addtolength{\hsize}{\algomargin}%
  \sbox\@tempboxa{\algocf@captiontext{#1}{#2}}%
  \ifdim\wd\@tempboxa >\hsize%     % if caption is longer than a line
    \hskip .5\algomargin%
    \parbox[t]{\hsize}{\algocf@captiontext{#1}{#2}}% then caption is not centered
  \else%
    \global\@minipagefalse%
    \hbox to\hsize{\box\@tempboxa}% else caption is centered
  \fi%
  \addtolength{\hsize}{-\algomargin}%
}
\makeatother

%%% User-defined macros should be placed here, but keep them to a minimum.
\def\Bka{{\it Biometrika}}
\def\AIC{\textsc{aic}}
\def\T{{ \mathrm{\scriptscriptstyle T} }}
\def\v{{\varepsilon}}

% -- declared math operators -- %
\DeclareMathOperator*{\argmin}{arg\,min}
\DeclareMathOperator*{\argmax}{arg\,max}
\DeclareMathOperator{\TV}{TV}

% --- defs --- %
\def\m{\mathcal}
\def\mb{\mathbb}
\def\mr{\mathrm}
\def\ms{\mathscr}
\def\ind{\mathbbm{1}}
\def\wt{\widetilde}
\def\wth{\widehat}

\def\T{{\mathrm{\scriptscriptstyle T} }}
\def\I{{\mathrm{\scriptscriptstyle I} }}
\def\D{{\mathrm{D} }}
\def\ELPD{{\mathrm{ELPD} }}
\def\EMM{{\mathrm{EMM} }}
\def\SE{{\mathrm{SE} }}
\def\Var{{\mathrm{Var} }}

\def\MF{{\mathrm{\scriptscriptstyle MF} }}
\def\DY{{\mathrm{\scriptscriptstyle DY} }}
\def\LV{{\mathrm{\scriptscriptstyle LV} }}


% -- new commands -- %
\newcommand{\ypbtodo}[1]{\todo[linecolor=red,backgroundcolor=red!25,bordercolor=red]{#1}}
\newcommand \bbP{\mathbb{P}}
\newcommand \bbE{\mathbb{E}}
\newcommand{\blds}[1]{\mbox{\scriptsize \boldmath $#1$}}

\newcommand{\be}{\begin{equs}}
\newcommand{\ee}{\end{equs}}


\newcommand{\PX}{\mathscr P_2^r(\m X)}
\newcommand{\PXj}{\mathscr P_2^r(\m X_j)}


% -- theorems etc -- %
\numberwithin{equation}{section}
%\theoremstyle{plain} # FORMAT CHANGE


\addtolength\topmargin{35pt}
\DeclareMathOperator{\Thetabb}{\mathcal{C}}

\begin{document}

\section{Application to Entropy based portfolio allocation}\label{ssec:po}
We present an application of the dual formulation of D-BETEL presented in equations \eqref{eqn:dbetel:alt}-\eqref{wbetel_dual} in the main document, to entropy based portfolio allocation problems \citep{doi:10.1080/07474930801960394}.
Portfolio optimization is concerned with the allocation of an investor's wealth over several assets to optimize specific objective(s) \citep{10.2307/2975974, doi:10.1080/07474930801960394}, based on historical data on asset returns. Let $R_{(t)} = (R_{1t} , R_{2t},\ldots , R_{nt} )^{\prime}$ be the excess returns on $n$ risky assets recorded over time  $t=1,2,\ldots,T$.  The traditional mean variance (MV) optimal portfolio weights \citep{10.2307/2975974} are obtained via:
\begin{equation}\label{eqn_MV}
    \argmax_{w} \mb{E}(w^\T R) - \frac{\lambda}{2}\mbox{Var}(w^\T R)\  \text{such that}  \ \sum_{i=1}^n w_i = 1
\end{equation}
where $\lambda$ is a risk aversion parameter. This approach is widely criticised since it (i) assumes normality of asset returns which often does not hold in practice, (ii) tend to provide extreme portfolio weights, contradicting the notion of diversification.
%(refer Fig.\ref{diag:MV}). 

\begin{figure}[!htb]
    \centering
    \subfloat{{\includegraphics[width=15cm, height = 4cm]{art/figures/mvpo_characteristics.pdf} }}%
    \caption{\emph{Limitations of mean-variance optimal portfolio: (i) The skewness and excess kurtosis plots provide evidence that the normality assumption for expected returns does not hold. (ii) Small value of $\lambda$ leads to zero weight to several assets, i.e not diversified portfolios}.\label{diag:MV}}
\end{figure}
%
As a remedy, \citep{doi:10.1080/07474930801960394} introduced a non-parametric entropy-based approach subject to constraints on return means and variances, that enjoys nice interpretations of portfolio diversification and shrinkage effects.  Our semi-parametric framework provides an excellent middle-ground since (i) we can still flexibly specify the distribution of the expected return, and (ii) the entropy provides direct handle on portfolio diversity. We achieve this by obtaining portfolio weights,  based on the dual formulation of D-BETEL in equations \eqref{eqn:dbetel:alt}-\eqref{wbetel_dual}:
\begin{equation}%\label{eqn_DBETEL}
  \argmin_{w} \big\{ (1-\lambda^{\star})  W^{2}_{\rm AR}\big[F_{w^{T}R}, G_{\theta_{0}}\big] + \lambda^{\star} b_n \ \sum_{i=1}^n w_i\log w_i \big\} \ 
 \text{such that}\  \sum_{i=1}^n w_i = 1.  
\end{equation}
Here $G_{\theta}(\cdot)$ is a  parametric distribution of choice with a target $\theta = \theta_0$; $w^T R$ is distributed as $F_{w_{(0)}^T R}(\cdot)$; and $\lambda^{\star}$ controls the balance between the portfolio diversity and deviation from the target distribution. We choose $G_{\theta}$ to be a Skew-normal distribution with parameters $(\omega, \Zeta, \alpha)$, refer to Section \ref{aux:ssec:po} for details.
%
\begin{figure}
\centering
    \subfloat%[\centering entropy-based portfolio allocation]
    {{\includegraphics[width=10cm, height = 4cm]{art/figures/MEPO.pdf} }}
    \caption{\emph{With a fixed target skew normal return,  varying values of $\lambda^{\star}\in[0,1]$ provide different balances between diversity \& departure from target. An user can achieve a desired degree of diversification choosing  $\lambda_{\star}$ via simple grid search.\label{diag:MEPO}}}%
    %\quad
    %\subfloat[\centering 5]{{\includegraphics[width=5.5cm]{Toy_WassersteinGF.pdf} }}%
\end{figure}
%
We consider monthly stock returns data of 5 companies (AMZN, AAPL, XOM, T,  MS) for the period January 2000 to December 2020, from Yahoo! Finance. In order to compare mean-variance optimal portfolio and our portfolio allocation frame work : (i)  we compute the mean-variance optimal portfolio in equation \ref{eqn_MV} at $\lambda = 1$  -- a choice at which 3 out of 5 portfolio weights are 0; refer Fig \ref{diag:MV}, (ii) fix the parameters of a skew-normal density such that it's mean, variance, and skewness match with the same quantities of the mean-variance optimal portfolio at $\lambda = 1$, (iii) compute the maximum entropy portfolio, for varying value of $\lambda^{\star}\in[0,1]$ with $b_n = 1/\log n$ (refer Fig\ref{diag:MEPO}). This choice of $b_n$ is convenient since it ensures that $-b_n\sum_{i=1}^n w_i\log w_i\in[0,1]$.
%
\section{Skew-normal distribution in section \ref{ssec:po} }\label{aux:ssec:po}
The pdf of a skew-normal distribution $\mbox{SN}(\zeta, \omega, \alpha)$ is given by
\begin{align*} 
 f(z) = \frac{2}{\omega}\phi\bigg(\frac{z-\zeta}{\omega}\bigg)
 \Phi\bigg(\alpha\bigg(\frac{z-\zeta}{\omega}\bigg)\bigg),\quad z\in\mb{R}
\end{align*}
where $\phi(\cdot)$ and $\Phi(\cdot)$ are respectively the pdf and cdf of Standard Normal distribution. For $\alpha=0$, we can recover the Normal distribution as absolute value of skewness increases and absolute value of $\alpha$ increases. For $\alpha>0$ the distribution left skewed and it is right skewed for $\alpha<0$. If $Z\sim\mbox{SN}(\zeta, \omega, \alpha)$, then we have
\begin{align*} 
 &\mu = \mb{E}(Z) = \zeta + \omega\delta\sqrt{\frac{2}{\pi}},\quad \sigma^2 = \mbox{Var}(Z) = \omega^2\bigg(1 - \frac{2\delta^2}{\pi}\bigg)\notag\\
 & \gamma = \mbox{Skewness} = \frac{4-\pi}{2}\ \frac{\bigg(\delta\sqrt{\frac{2}{\pi}}\bigg)^3}{\bigg(1 - \frac{2\delta^2}{\pi}\bigg)^{3/2}};\quad \mbox{Excess Kurtosis} = 2(\pi - 3)\ \frac{\bigg(\delta\sqrt{\frac{2}{\pi}}\bigg)^4}{\bigg(1 - \frac{2\delta^2}{\pi}\bigg)^{2}}
\end{align*}
where $\delta= \alpha/\sqrt{1+\alpha^2}$. The above expressions allow us to set $(\zeta, \omega,\alpha)$ such that it ensures our desired target $(\mu,\sigma^2,\gamma)$ of the return distribution. The resulting skew normal density with fully specified parameters then serve as the target distribution to calculate $\mbox{D}$-BETEL based portfolio weights in section \ref{ssec:po} in the main document. 




\bibliography{paper-ref}
%\bibliographystyle{apalike}
\bibliographystyle{plainnat}
%\bibliographystyle{biometrika}

\end{document}
