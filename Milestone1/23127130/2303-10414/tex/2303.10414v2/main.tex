%\usepackage[final]{showkeys}
%\definecolor{llightgray}{rgb}{0.9,0.9,0.9}
%\newenvironment{remark}[1][Remark]{\bigskip\noindent\textbf{#1. }\it}{\rm\medskip}
%\newtheorem{remarks}[theorem]{Remarks}
%\newtheorem{example}[theorem]{Example}
%\newenvironment{definition}[1][Definition]{\bigskip\noindent\textbf{#1. }\it}{\rm\bigskip}
%\newenvironment{definitions}[1][Definition]{\bigskip\noindent\textbf{#1. }\it}{\rm}


\documentclass[reqno,12pt]{amsart}
%%%%%%%%%%%%%%%%%%%%%%%%%%%%%%%%%%%%%%%%%%%%%%%%%%%%%%%%%%%%%%%%%%%%%%%%%%%%%%%%%%%%%%%%%%%%%%%%%%%%%%%%%%%%%%%%%%%%%%%%%%%%%%%%%%%%%%%%%%%%%%%%%%%%%%%%%%%%%%%%%%%%%%%%%%%%%%%%%%%%%%%%%%%%%%%%%%%%%%%%%%%%%%%%%%%%%%%%%%%%%%%%%%%%%%%%%%%%%%%%%%%%%%%%%%%%
\usepackage{eurosym}
\usepackage{amssymb}
\usepackage{mathrsfs}
\usepackage{txfonts}
\usepackage{amsfonts}
\usepackage{amstext}
\usepackage{amssymb}
\usepackage{amsmath}
\usepackage{graphicx}
\usepackage{hyperref}
\usepackage{url}
\usepackage{amssymb}

\setcounter{MaxMatrixCols}{10}
%TCIDATA{OutputFilter=LATEX.DLL}
%TCIDATA{Version=5.50.0.2960}
%TCIDATA{<META NAME="SaveForMode" CONTENT="1">}
%TCIDATA{BibliographyScheme=BibTeX}
%TCIDATA{LastRevised=Tuesday, April 11, 2023 15:34:19}
%TCIDATA{<META NAME="GraphicsSave" CONTENT="32">}
%TCIDATA{Language=American English}
%TCIDATA{CSTFile=amsart.cst}
%TCIDATA{<META NAME="PrintViewPercent" CONTENT="100">}

\hypersetup{
        colorlinks   = true,
        citecolor    = blue,
        linkcolor    = blue,
        urlcolor     = blue
}
\allowdisplaybreaks
\newtheorem{theorem}{Theorem}[section]
\newtheorem{proposition}[theorem]{Proposition}
\newtheorem{lemma}[theorem]{Lemma}
\newtheorem{definition}[theorem]{Definition}
\newtheorem{corollary}[theorem]{Corollary}
\newtheorem{remark}[theorem]{Remark}
\renewcommand{\theequation}{\thesection.\arabic{equation}}
\newenvironment{notation}{\smallskip{\sc Notation.}\rm}{\smallskip}
\newenvironment{acknowledgement}{\smallskip{\sc Acknowledgement.}\rm}{\smallskip}
\DeclareMathOperator*{\einf}{ess{\,}inf}
\DeclareMathOperator*{\esup}{ess{\,}sup}
\DeclareMathOperator*{\osc}{osc}
\newcommand{\mint}{\ \ \!\!\!-\!\!\!\:\!\!\!\!\! \int}
\numberwithin{equation}{section}
\newcounter{counterConstant}
\newcommand{\const}[1]{
    \addtocounter{counterConstant}{1}
    \edef#1{\arabic{counterConstant}}
}
\input{tcilatex}

% A4
\setlength{\textheight}{24 cm}
\setlength{\textwidth}{16.5 cm}
\setlength{\topmargin}{-1cm}
\setlength{\oddsidemargin}{0.0 cm}
\setlength{\evensidemargin}{0.0 cm}

%A4
%\setlength{\textheight}{24 cm}
%\setlength{\textwidth}{15 cm}
%\setlength{\topmargin}{0cm}
%\setlength{\oddsidemargin}{1.0 cm}
%\setlength{\evensidemargin}{1.0 cm}

%Letter
%\setlength{\textheight}{23 cm}
%\setlength{\textwidth}{16 cm}
%\setlength{\topmargin}{-1cm}
%\setlength{\oddsidemargin}{-0.0 cm}
%\setlength{\evensidemargin}{-0.0 cm}

%\def\text{\mathrm}
%\def\text{\textrm}
\def\RM{\rm}
\def\FTN{\footnotesize}
\def\NSZ{\normalsize}
\def\stackunder#1#2{\mathrel{\mathop{#2}\limits_{#1}}}%
%\def\marginpar#1{}
%\def\emptyspace#1{}
%\def\endproofof#1{}
\def\qed{\hfill$\square$\par}
%\newenvironment{proof}[1][Proof]{\textbf{#1.} }{\ \rule{0.5em}{0.5em}}
%\newenvironment{proof}[1][Proof]{\textbf{#1.} }\qed

\def\limfunc#1{\mathop{\mathrm{#1}}}
\def\func#1{\mathop{\mathrm{#1}}\nolimits}
\def\diint{\mathop{\int\int}}

\def\dint{\displaystyle\int}
\def\dsum{\displaystyle\sum}
\def\dbigcap{\displaystyle\bigcap}
\def\dbigoplus{\displaystyle\bigoplus}
\def\dbigcup{\displaystyle\bigcup}
\def\dprod{\displaystyle\prod}
\def\tbigcup{\mathop{\textstyle \bigcup }}
\def\tbigcap{\mathop{\textstyle \bigcap }}


\def\Xint#1{\mathchoice
{\XXint\displaystyle\textstyle{#1}}%
{\XXint\textstyle\scriptstyle{#1}}%
{\XXint\scriptstyle\scriptscriptstyle{#1}}%
{\XXint\scriptscriptstyle\scriptscriptstyle{#1}}%
\!\int}
\def\XXint#1#2#3{{\setbox0=\hbox{$#1{#2#3}{\int}$ }
\vcenter{\hbox{$#2#3$ }}\kern-.6\wd0}}

\def\oint{\Xint-}
\def\toint{\Xint-}
\def\fint{\Xint-}

%\def\subjclass#1{\par{\footnotesize 2000 {\sl Mathematics Subject Classification.} #1}}
%\def\keywords#1{\par{\footnotesize {\sl Keywords and phrases.} #1}}

\def\beginsel#1#2#3{\vfil\eject\setcounter{page}{#1}\setcounter{section}{#2}\setcounter{equation}{#3}}
%example: \beginsel{1}{0}{0}
\def\endsel{\vfil\eject}

%% label
\def\Qlb#1{#1}

%% caption
\def\Qcb#1{#1}

\def\FRAME#1#2#3#4#5#6#7#8
{
 \begin{figure}[ptbh]
 \begin{center}
 \includegraphics[height=#3]{#7}
 \caption{#5}
 \label{#6}
 \end{center}
 \end{figure}
}

\def\enddoc{\end{document}}


\begin{document}
\title[]{Heat kernel-based $p$-energy norms on metric measure spaces}
\author[Gao]{Jin Gao}
\address{Department of Mathematics, Hangzhou Normal University, Hangzhou
310036, China.}
\email{gaojin@hznu.edu.cn}
\author[Yu]{Zhenyu Yu}
\address{Department of Mathematical Sciences, Tsinghua University, Beijing
100084, China.}
\email{yuzy18@mails.tsinghua.edu.cn}
\author[Zhang]{Junda Zhang}
\address{Department of Mathematical Sciences, Tsinghua University, Beijing
100084, China.}
\email{zjd18@mails.tsinghua.edu.cn}
\date{}

\begin{abstract}
We focus on heat kernel-based $p$-energy norms ($1<p<\infty$) on bounded and
unbounded metric measure spaces, in particular, weak-monotonicity properties
for different types of energies. Such properties are key to related studies,
under which we generalise the convergence result of
Bourgain-Brezis-Mironescu (BBM) for $p\neq2$. We establish the equivalence
of various $p$-energy norms and weak-monotonicity properties when there
admits a heat kernel satisfying the two-sided estimates. Using these
equivalences, we verify various weak-monotonicity properties on nested
fractals and their blowups. Immediate consequences are that, many classical
results on $p$-energy norms hold for such bounded and unbounded fractals,
including the BBM convergence and Gagliardo-Nirenberg inequality.
\end{abstract}

\subjclass[2010]{28A80, 46E30, 46E35}
\keywords{Heat kernel, Convergence, $p$-energy, Besov norm, nested fractals.}
\maketitle
\tableofcontents

\section{Introduction}

\subsection{Historical background and motivation}

Our original motivation is generalising the celebrated
`Bourgain-Brezis-Mironescu (BBM) convergence' \eqref{BBM} of $p$-energy
forms ($1<p<\infty$) in \cite{GaoYuZhang2022PA} to bounded and unbounded
metric measure space, which also extends the classical convergence of
non-local Dirichlet forms to a local one on unbounded metric measure spaces
or fractals (see \cite{Pietruska-Paluba.2008} and \cite{Yang2022MZ}). This
originates from Bourgain, Brezis and Mironescu in \cite[Corollary 2]%
{BourgainBrezisMironescu.2001.439}:
\begin{equation}
\lim_{\sigma \uparrow 1}(1-\sigma )\int_{D}\int_{D}\frac{|u(x)-u(y)|^{p}}{%
|x-y|^{n+p\sigma }}dxdy=C_{n,p}\int_{D}|\nabla u(x)|^{p}dx,  \label{BBM}
\end{equation}%
where $D$ is a smooth area in $\mathbb{R}^n$, which states that multiplying
by a scaling factor $1-\sigma $, the fractional Gagliardo semi-norm of a
function converges to the first-order Sobolev semi-norm as $%
\sigma\rightarrow1 $. For other metric spaces, a natural analogue to BBM
convergence is the convergence of Besov semi-norms $(1-\sigma
)B_{2,2}^{\sigma }$ to $B_{2,\infty }^{\sigma ^{\ast }}$, where $\sigma
^{\ast }$ is a fixed number (usually termed as `walk dimension') in \cite%
{GuLau.2020.AASFM,Pietruska-Paluba.2008,Yang.2018.PA,Yang2022MZ}. Gu and Lau
left an open problem in \cite[Section 7]{GuLau.2020.AASFM} to extend such
convergence to more general settings under more general `weak-monotonicity
properties'. To answer this problem, we study the BBM convergence and
weak-monotonicity properties on metric measure spaces.

Recently, there are quite a few development on the $p$-energy defined on
fractals and general metric measure spaces, which is first studied by \cite%
{HermanPeironeStrichartz.2004.PA125}. For a smooth Euclidean area $D$, the $%
p $-energy is simply defined as $\int_{D}|\nabla u(x)|^{p}dx$ (the
right-hand side of \eqref{BBM}), but for fractals or metric measure spaces,
it is not easy to define proper gradient structure to characterise the
smoothness of certain `core' functions arise naturally from analysis. We
refer to \cite{GaoYuZhang2022PA} which considered the case $1<p<\infty$ on
homogeneous p.c.f. self-similar sets. Previous constructions of $p$-energy ($%
1<p<\infty$) are based on the graph-approximation to the underlying space,
see \cite{Caoqiugu2022adv} for p.c.f. self-similar sets, \cite{Shimuzu.2022}
for the Sierpi\'{n}ski carpet, and \cite{Kigami2022penergy} for metric
measure spaces. %////For the resistance estimate (which is a key
%part), they use or develop ideas from \cite%
%{BourdonKleiner2010,HermanPeironeStrichartz.2004.PA125,KusuokaZhou.1992.PTRF169}.////

In this paper, we consider another way to define $p$-energy norms by
equivalent Besov-type norms, which naturally generalises the energy of
Dirichlet forms when $p=2$, and the graph-approximation does not appear in
the definition. Heat kernel-based $p$-energy norms basically appeared in
\cite{Pietruska-Paluba.2010.}, and later heat semigroup-based norms are
systematically studied by Alonso Ruiz, Baudoin et al. for $1\leq p<\infty$
in \cite%
{AlonsoBaudoinchen2020JFA,AlonsoBaudoinchen2020CVPDE,AlonsoBaudoinchen2021CVPDE}%
, where they focus on generalising classical analysis results including
Sobolev embeddings, isoperimetric inequalities and the class of bounded
variation functions etc. It is possible to use \emph{heat kernel-based} $p$%
-energy norms to construct equivalent (or exactly the same) $p$-energy given
by the graph-approximation approach, to satisfy further restrictions like
convexity or self-similarity for self-similar sets, see for example \cite%
{Caoqiugu2022adv,GaoYuZhang2022PA}.

In all these studies, certain energy-control conditions (which will be
called \emph{weak-monotonicity} properties in this paper) are essentially
required and important, to guarantee `some level of $L^p$ infinitesimal
regularity and global controlled $L^p$ geometry' as \cite{BaudoinLecture2022}
stated. Relating different energy-control conditions is a main contribution
of our paper.

It is quite interesting that the BBM convergence is naturally related to the
recent work of Alonso Ruiz, Baudoin et al. mentioned above. The key to the
BBM convergence is that, we need to apply proper weak-monotonicity
properties on metric measure spaces (to replace `property (E)' used for
fractals). We apply the concept of property (KE) raised in \cite[Definition
6.7]{AlonsoBaudoinchen2020JFA} and (NE) raised in \cite[Definition 4.5]%
{BaudoinLecture2022} (all termed $P(p,\alpha)$ therein), and use (VE) for
bounded and unbounded fractals based on \cite[Definition 3.1]%
{GaoYuZhang2022PA}, where the letters `E' stands for energy control and
`K',`N',`V' stands for (heat) kernel, (Besov) norm, vertex respectively. We
also consider their slightly weaker variants by replacing `sup' with
`limsup' for BBM convergence. Once this is done, we need to verify such
weak-monotonicity properties for good metric measure spaces, like nested
fractals and their unbounded blow-ups. To do this, we establish the
equivalence of properties (KE), (NE) and (VE) on certain bounded and
unbounded fractals (including nested fractals), and verify a relatively
easier condition (VE) with the help of \cite{Caoqiugu2022adv}. Such
properties are not easy to examine when $p\neq 2$ in general (property (KE)
holds automatically when $p=2$). Alonso Ruiz, Baudoin et al. examined
property (KE) in the case $p=1$ using weak Bakry-Emery type hypothesis and $%
p\neq 2$ for spaces with the same critical exponent as Euclidean smooth
manifolds (see for example \cite[Lemma 4.13]{AlonsoBaudoinchen2021CVPDE}).
But for fractal-type spaces, different ideas are required, since the
critical exponents are often different from smooth manifolds and
(bi-)linearity is missing.

By the time we were about to finish our paper, we notice a lecture note \cite%
{BaudoinLecture2022} from Baudoin, where we find out some very recent work
related to the BBM convergence with rather different viewpoint (see \cite%
{Gorny2022,HanPina2021}), and the verification of properties (KE) and (NE)
for two nested fractals using a different argument (see \cite%
{RuizBaudoin2021JMAA, BaudoinLecture2022} ). We examine properties (KE) and
(NE) by their equivalence to property (VE), and this equivalence does not
rely on the finitely-ramified property. The reason why we consider p.c.f.
fractals is to avoid complexity in defining (VE) properly, and also
verifying (VE) is relatively easier under the p.c.f. property. Alonso Ruiz,
Baudoin et al. established two analogues of classical analysis results under
property (KE) or (NE) for $1<p<\infty $, that is, the Sobolev-type
inequality in \cite[Theorem 1.1]{AlonsoBaudoinchen2020JFA} and the full
scale of Gagliardo-Nirenberg inequality introduced by \cite[Theorem 4.3 and
Theorem 4.4]{BaudoinLecture2022}. The latter inequality only holds for
unbounded spaces in \cite{BaudoinLecture2022}, and this motivates us to
study unbounded fractals.

\subsection{Preliminaries}
\label{secPre}
\subsubsection{Basic definitions}
Let $(M,d,\mu )$ be a metric measure space. Assume that $(M,d)$ has more
than one point with positive diameter. Fix a value (which could be infinite)
\begin{equation*}
R_{0}\in (0,\mathrm{diam}(M)]
\end{equation*}
that will be used for localization throughout the paper, where
\begin{equation*}
\mathrm{diam}(M):=\sup \{d(x,y):x,y\in M\}\in (0,\infty]
\end{equation*}
is infinite if $M$ is unbounded and is finite if $M$ is bounded.

The measure $\mu$ is assumed to be Borel regular with the following
positivity and \emph{volume doubling property}: there exists a constant $%
C_{d}>0$ such that for every $x\in M$, $0<r<\infty $,
\begin{equation}
0<\mu (B(x,2r))\leq C_{d}\mu (B(x,r))<\infty,  \label{VD1}
\end{equation}%
where
\begin{equation*}
B(x,r):=\{y\in M:d(x,y)<r\}
\end{equation*}%
denotes the open metric ball centered at $x\in M$ with radius $r>0$. Denote
\begin{equation*}
V(x,r):=\mu (B(x,r)).
\end{equation*}%
It is known by \cite[Proposition 5.1]{GrigoryanHu.2014.MMJ505} that if %
\eqref{VD1} holds, then there exists $\alpha _{1}>0$ such that
\begin{equation}
\frac{V(x,R)}{V(x,r)}\leq C_{d}\left( \frac{R}{r}\right) ^{\alpha _{1}},
\label{VD2}
\end{equation}%
for all $x\in M$, $0<r\leq R<\infty$.

We say that $\mu $ is \emph{$\alpha $-regular} if there exists a constant $%
C\geq 1$ such that for all $0<r<R_{0}$,
\begin{equation}
C^{-1}r^{\alpha }\leq V(x,r)\leq Cr^{\alpha }.  \label{alpha_r}
\end{equation}%
We remark that when $\mu $ is $\alpha $-regular, then (\ref{VD2}) holds with
$\alpha _{1}=\alpha $.

Given a metric measure space $(M,d,\mu )$, a family $\{p_{t}\}_{t>0}$ of
non-negative measurable functions $p_{t}(x,y):M\times M\rightarrow \mathbb{R}%
_{+}$ on $M\times M$ is called a \emph{heat kernel}, if the following
conditions are satisfied for $\mu $-almost all $x,y\in M$ and $s,t>0$:

\begin{enumerate}
\item[(1)] Symmetry: $p_t(x, y)=p_t(y, x)$.

\item[(2)] The total mass inequality:
\begin{equation*}
\int_{M}p_{t}(x,y)d\mu (y)\leq 1.
\end{equation*}

\item[(3)] Semigroup property:
\begin{equation*}
p_{s+t}(x,y)=\int_{M}p_{s}(x,z)p_{t}(z,y)d\mu (z).
\end{equation*}

\item[(4)] Strong continuity: for any $f\in L^{2}(M,\mu )$,
\begin{equation*}
\int_{M}p_{t}(x,y)f(y)d\mu (y)\rightarrow f(x)\text{ \ as}~~t\downarrow 0,%
\text{strongly in}~~L^{2}(M,\mu ).
\end{equation*}
\end{enumerate}

We say that a heat kernel $\{p_{t}\}_{t>0}$ satisfies the \emph{\ two-sided}
heat kernel estimates, if for all $t\in
(0,R_{0}^{\beta ^{\ast }})$ and $\mu $-almost all $x,y\in M$ :
\begin{equation}
\frac{c_{1}}{V(x,t^{1/\beta ^{\ast }})}\exp \left( -c_{2}\left( \frac{d(x,y)%
}{t^{1/\beta ^{\ast }}}\right) ^{\frac{\beta ^{\ast }}{\beta ^{\ast }-1}%
}\right) \leq p_{t}(x,y)\leq \frac{c_{3}}{V(x,t^{1/\beta ^{\ast }})}\exp
\left( -c_{4}\left( \frac{d(x,y)}{t^{1/\beta ^{\ast }}}\right) ^{\frac{\beta
^{\ast }}{\beta ^{\ast }-1}}\right) ,  \label{hk}
\end{equation}%
where $\beta ^{\ast }>1$ is a fixed number. If the metric space satisfies
the chain condition (see \cite[Definition 3.4]{GrigoryanHuLau.2003.TAMS2065}%
), the two-sided estimates \eqref{hk} are equivalent to the conjunction of
the volume doubling property, the scale-invariant elliptic Harnack
inequality and the mean exit time estimate on metric measure spaces (see
\cite[Corollary 7.6]{GrigoryanTelcs.2012.AoP1212}).

Such estimates hold for many classical cases. The classical
Gauss-Weierstrass function in $\mathbb{R}^{n}$
\begin{equation}
p_{t}(x,y)=\frac{1}{(4\pi t)^{n/2}}\exp \left( -\frac{|x-y|^{2}}{4t}\right) ,
\label{Gau-Wei}
\end{equation}%
is a heat kernel for the standard Brownian motion. Li and Yau have shown in
\cite{LiYau.1986.AM153} that, for a complete Riemannian manifold with
non-negative Ricci curvature, two-sided estimates \eqref{hk} hold with $%
\beta ^{\ast }=2$:
\begin{equation}
p_{t}(x,y)\asymp \frac{C}{V(x,\sqrt{t})}\exp \left( -\frac{d(x,y)^{2}}{ct}%
\right) ,  \label{LiYau}
\end{equation}%
where $d$ is the geodesic metric and $V(x,r)$ is the Riemannian volume. It
is known that the volume doubling condition combined with the Poincar\'{e}
inequality is equivalent to the two-sided estimates \eqref{hk} for the
Laplace-Beltrami operator on complete Riemannian manifolds (see \cite%
{Grigoryan..MS,Saloff-Coste.1992.IMRN27}). In addition, two-sided estimates %
\eqref{hk} and the volume doubling property hold on scale irregular Sierpi%
\'{n}ski gaskets (see \cite[Theorem 5.1]{KajinoMurugan.2020On}).

The following two-sided heat kernel estimates
\begin{equation}
p_{t}(x,y)\asymp \frac{C}{t^{\alpha /\beta ^{\ast }}}\exp \left( -c\left(
\frac{d(x,y)}{t^{1/\beta ^{\ast }}}\right) ^{\frac{\beta ^{\ast }}{\beta
^{\ast }-1}}\right)  \label{hk_F}
\end{equation}%
also hold on many fractal spaces, such as the Sierpi\'{n}ski carpets and
nested fractals, see examples in \cite%
{BarlowBass.1989.AIHPPS225,Kigami.1993.TAMS721}. In fact, estimates %
\eqref{hk_F} imply that $\mu $ is $\alpha $-regular (see for example \cite[%
Theorem 3.2]{GrigoryanHuLau.2003.TAMS2065}).

\subsubsection{Besov and Korevaar-Schoen norms}

As in \cite{BaudoinLecture2022}, for $\sigma >0$, define semi-norms $%
[u]_{B_{p,\infty }^{\sigma }}$ for $u\in L^{p}(M,\mu )$ $(p>1)$ by
\begin{equation}
\lbrack u]_{B_{p,\infty }^{\sigma }}^{p}:=\sup_{r\in (0,R_{0})}r^{-p\sigma
}\int_{M}\frac{1}{V(x,r)}\int_{B(x,r)}|u(x)-u(y)|^{p}d\mu (y)d\mu (x),
\label{pi}
\end{equation}%
and define
\begin{equation*}
B_{p,\infty }^{\sigma }:=B_{p,\infty }^{\sigma }(M)=\{u\in L^{p}(M,\mu
):[u]_{B_{p,\infty }^{\sigma }}<\infty \}.
\end{equation*}%
Naturally, define semi-norms $[u]_{B_{p,p}^{\sigma }}$ by
\begin{equation}
\lbrack u]_{B_{p,p}^{\sigma }}^{p}:=\int_{0}^{R_{0}}r^{-p\sigma }\left(
\int_{M}\frac{1}{V(x,r)}\int_{B(x,r)}|u(x)-u(y)|^{p}d\mu (y)d\mu (x)\right)
\frac{dr}{r},  \label{pp}
\end{equation}%
and define
\begin{equation*}
B_{p,p}^{\sigma }:=B_{p,p}^{\sigma }(M)=\{u\in L^{p}(M,\mu
):[u]_{B_{p,p}^{\sigma }}<\infty \}.
\end{equation*}

The space $B_{p,\infty }^{\sigma }$ coincides with the Korevaar-Schoen space
$KS_{p,\infty }^{\sigma }$ in \cite[Section 4.2]{BaudoinLecture2022} with
the following different norm (switch `sup' to `limsup')
\begin{equation}
\Vert u\Vert^p _{KS_{p,\infty }^{\sigma }}=\limsup_{r\rightarrow 0}\int_{M}%
\frac{1}{V(x,r)}\int_{B(x,r )}\frac{|u(x)-u(y)|^{p}}{r ^{p\sigma }}d\mu
(y)d\mu (x)<\infty.  \label{ks_energy1}
\end{equation}

In our paper, we define the \emph{critical exponent of $(M,d,\mu )$} by%
\begin{equation*}
\sigma _{p}^{\#}=\sup \{\sigma >0:B_{p,\infty }^{\sigma }\text{ contains
non-constant functions}\}.
\end{equation*}%
In many related studies the critical exponent is defined by
\begin{equation*}
\sigma _{p}^{\ast }=\sup \{\sigma >0:B_{p,\infty }^{\sigma }\text{ is dense
in }L^{p}(M,\mu )\}.
\end{equation*}%
%
%
%
%
%
%
%
%
%
%
%
%
%
%
%
%
%
%
%
%
%
%
%
%
%
%
%
%
%
%
%
%
%
%
%
%
%
%
%
%and we will meet these two exponents in the last section.
Clearly, $\sigma _{p}^{\ast }\leq \sigma _{p}^{\#}$. Whenever the critical
exponents appear in our paper, we assume that they are finite, and it is
known by \cite[Theorem 4.2]{BaudoinLecture2022} that when the chain
condition holds, then
\begin{equation*}
\sigma _{p}^{\#}\leq 1+\frac{\alpha_1}{p} .
\end{equation*}

\subsubsection{Heat kernel-based $p$-energy norm}

It is well-known that, a heat kernel $\{p_{t}\}_{t>0}$ can deduce a
Dirichlet form in $L^{2}(M,\mu )$ given by
\begin{align*}
\mathcal{E}(u,u)& =\lim_{t\rightarrow 0^{+}}\mathcal{E}_{t}(u,u),\ u\in
L^{2}(M,\mu) \\
\mathcal{D}(\mathcal{E})& =\{u\in L^{2}(M,\mu ):\mathcal{E}(u,u)<\infty \}
\end{align*}%
(see \cite[Section 1.3]{FukushimaOshimaTakeda.2011.489}), where
\begin{equation}
\mathcal{E}_{t}(u,u):=\frac{1}{2t}\int_{M}\int_{M}|u(x)-u(y)|^{2}p_{t}(x,y)d%
\mu (y)d\mu (x).  \label{E_t}
\end{equation}%
The limit exists, since for any given $u\in L^{2}(M,\mu )$, $t\rightarrow
\mathcal{E}_{t}(u,u)$ is monotone decreasing by using the spectral theorem.

In view of this, we adapt the heat kernel-based norms in \cite%
{AlonsoBaudoinchen2020JFA} and \cite{Pietruska-Paluba.2010.} to define $p$%
-energy norm in $L^{p}(M,\mu )$ for a heat kernel $\{p_{t}\}_{t>0}$ as
\begin{equation*}
E_{p,\infty }^{\sigma }(u):=\sup_{t\in (0,R_{0}^{\beta ^{\ast }})}\frac{1}{%
t^{{p\sigma }/{\beta ^{\ast }}}}\int_{M}\int_{M}|u(x)-u(y)|^{p}p_{t}(x,y)d%
\mu (y)d\mu (x),
\end{equation*}%
with its domain
\begin{equation*}
\mathcal{D}(E_{p,\infty }^{\sigma }):=\{u\in L^{p}(M,\mu ):E_{p,\infty
}^{\sigma }(u)<\infty \}.
\end{equation*}%
This is a natural way to extend the Dirichlet form case for $p=2$ to $p\in
(1,\infty)$, and we will call them heat kernel-based $p$-energy norms. The
heat kernel-based $p$-energy norm $E_{p,\infty }^{\sigma_{p}^{\#}}$ is
equivalent to the $p$-energy on certain fractals (see Section \ref{sec5}).

To extend the BBM convergence from $p=2$ to $1<p<\infty$, it is quite
natural to define
\begin{equation*}
E_{p,p}^{\sigma }(u):=\int_{0}^{R_{0}^{\beta ^{\ast }}}\frac{1}{t^{{p\sigma }%
/{\beta ^{\ast }}}}\left( \int_{M}\int_{M}|u(x)-u(y)|^{p}p_{t}(x,y)d\mu
(x)d\mu (y)\right) \frac{dt}{t},
\end{equation*}%
with domain
\begin{equation*}
\mathcal{D}(E_{p,p}^{\sigma }):=\{u\in L^{p}(M,\mu ):E_{p,p}^{\sigma
}(u)<\infty \}.
\end{equation*}
It is known that (see for example \cite{Pietruska-Paluba.2008}), when there
admits the two-sided bounds, the proper scaling of $E_{2,2}^{\sigma } $
converges to $E_{2,\infty }^{\sigma _{2}^{\#}}$ by using the techniques of
subordinators that, $E_{2,2}^{\sigma }$ is equivalent to the Dirichlet form
defined by the subordinated heat kernel. Moreover, we also consider the
convergence of heat kernel-based $p$-energy norms $E_{p,p}^{\sigma }$ to $%
E_{p,\infty }^{\sigma _{p}^{\#}}$, but unfortunately $E_{p,p}^{\sigma }$
does not seem to be the $p$-energy norm of the subordinated heat kernel, so
we need weak-monotonicity properties.

\subsection{Statement of weak-monotonicity properties and main results}

For simplicity, for $u\in L^{p}(M,\mu )$ and $\sigma ,t>0$, denote $\Psi
_{u}^{\sigma }(t)$ by
\begin{equation}
\Psi _{u}^{\sigma }(t)=\frac{1}{t^{{p\sigma }/{\beta ^{\ast }}}}%
\int_{M}\int_{M}|u(x)-u(y)|^{p}p_{t}(x,y)d\mu (y)d\mu (x),  \label{psi_u}
\end{equation}%
then
\begin{equation}
E_{p,\infty }^{\sigma }(u)=\sup_{t\in (0,R_{0}^{\beta ^{\ast }})}\Psi
_{u}^{\sigma }(t),\ E_{p,p}^{\sigma }(u)=\int_{0}^{R_{0}^{\beta ^{\ast
}}}\Psi _{u}^{\sigma }(t)\frac{dt}{t}.  \label{EEE}
\end{equation}

\begin{definition}
We say that a metric measure space $(M,d,\mu )$ with heat kernel $%
\{p_{t}\}_{t>0}$ satisfies property (KE) with $\sigma >0$, if there exists a
constant $C>0$ such that for all $u\in \mathcal{D}(E_{p,\infty }^{\sigma })$%
,
\begin{equation*}
\sup_{t\in (0,R_{0}^{\beta ^{\ast }})}\Psi _{u}^{\sigma }(t)\leq
C\liminf_{t\rightarrow 0}\Psi _{u}^{\sigma }(t),
\end{equation*}%
where $\Psi _{u}^{\sigma }(t)$ is defined as in \eqref{psi_u}. We further
say that property ($\widetilde{KE}$) is satisfied with $\sigma >0$, if there
exists a constant $C>0$ such that for all $u\in \mathcal{D}(E_{p,\infty
}^{\sigma })$,
\begin{equation*}
\limsup_{t\rightarrow 0}\Psi _{u}^{\sigma }(t)\leq C\liminf_{t\rightarrow
0}\Psi _{u}^{\sigma }(t).
\end{equation*}
\end{definition}

\begin{remark}
\label{rk1}If $p=2$, we have the fact that property (KE) automatically holds
with $\sigma =\beta ^{\ast }/2$ since $t\rightarrow \mathcal{E}_{t}(u,u)$ is
monotone decreasing.
\end{remark}

\begin{theorem}
\label{thm4.1} Suppose that $(M,d,\mu )$ with heat kernel $\{p_{t}\}_{t>0}$
satisfies property (KE) with $\tilde{\sigma}_{p}>0$. Then there exists a
positive constant $C$ such that for all $u\in B_{p,\infty }^{\tilde{\sigma}%
_{p}}$,
\begin{equation}
C^{-1}{E}_{p,\infty }^{\tilde{\sigma}_{p}}(u)\leq \liminf_{\sigma \uparrow
\tilde{\sigma}_{p}}(\tilde{\sigma}_{p}-\sigma ){E}_{p,p}^{\sigma }(u)\leq
\limsup_{\sigma \uparrow \tilde{\sigma}_{p}}(\tilde{\sigma}_{p}-\sigma ){E}%
_{p,p}^{\sigma }(u)\leq C{E}_{p,\infty }^{\tilde{\sigma}_{p}}(u).
\label{eq5.1}
\end{equation}
\end{theorem}

Similarly, for Besov norms, denote
\begin{equation}
\Phi _{u}^{\sigma }(r)=r^{-p\sigma }\int_{M}\frac{1}{V(x,r)}%
\int_{B(x,r)}|u(x)-u(y)|^{p}d\mu (y)d\mu (x),  \label{phi_u}
\end{equation}%
then
\begin{equation*}
\lbrack u]_{B_{p,\infty }^{\sigma }}^{p}=\sup_{r\in (0,R_{0})}\Phi
_{u}^{\sigma }(r),\ [u]_{B_{p,p}^{\sigma }}^{p}=\int_{0}^{R_{0}}\Phi
_{u}^{\sigma }(r)\frac{dr}{r},\ \Vert u\Vert _{KS_{p,\infty }^{\sigma
}}^{p}=\limsup_{r\rightarrow 0}\Phi _{u}^{\sigma }(r).
\end{equation*}

\begin{definition}
We say that a metric measure space $(M,d,\mu )$ satisfies property ($NE$)
with $\sigma >0$ if there exists $C>0$ such that for all $u\in B_{p,\infty
}^{\sigma }$,
\begin{equation}
\sup_{r\in (0,R_{0})}\Phi _{u}^{\sigma }(r)\leq C\liminf_{r\rightarrow
0}\Phi _{u}^{\sigma }(r).  \label{NE1}
\end{equation}%
In addition, we say that property ($\widetilde{NE}$) is satisfied with $%
\sigma >0$, if there exists $C>0$ such that for all $u\in B_{p,\infty
}^{\sigma }$,
\begin{equation*}
\limsup_{r\rightarrow 0}\Phi _{u}^{\sigma }(r)\leq C\liminf_{r\rightarrow
0}\Phi _{u}^{\sigma }(r).
\end{equation*}
\end{definition}

It is known by \cite[Lemma 4.7]{BaudoinLecture2022} that property (NE) is
satisfied with some $\sigma >0$, then $\sigma \geq \sigma _{p}^{\#}$. So
property (NE) is only interested when $\sigma=\sigma _{p}^{\#},$ and
whenever we say that property (NE) is satisfied, we automatically assume $%
\sigma=\sigma _{p}^{\#}$.

\begin{theorem}
\label{thm4.2} Suppose that $(M,d,\mu )$ satisfies property (NE), then there
exists a positive constant $C$ such that for all $u\in B_{p,\infty }^{\sigma
_{p}^{\#}}$,
\begin{equation}
C^{-1}[u]_{{B}_{p,\infty }^{\sigma _{p}^{\#}}}^{p}\leq \liminf_{\sigma
\uparrow \sigma _{p}^{\#}}(\sigma _{p}^{\#}-\sigma )[u]_{{B}_{p,p}^{\sigma
}}^{p}\leq \limsup_{\sigma \uparrow \sigma _{p}^{\#}}(\sigma
_{p}^{\#}-\sigma )[u]_{{B}_{p,p}^{\sigma }}^{p}\leq C[u]_{{B}_{p,\infty
}^{\sigma _{p}^{\#}}}^{p}.  \label{NE_conv}
\end{equation}%
Suppose that $(M,d,\mu )$ satisfies property $(\widetilde{NE})$ with $%
\sigma_p>0$, then there exists a positive constant $C$ such that for all $%
u\in B_{p,\infty }^{\sigma_p}$,
\begin{equation}
C^{-1}\Vert u\Vert^p _{{KS}_{p,\infty }^{\sigma _{p}}}\leq \liminf_{\sigma
\uparrow \sigma _{p}^{\#}}(\sigma _{p}-\sigma )[u]_{{B}_{p,p}^{\sigma
}}^{p}\leq \limsup_{\sigma \uparrow \sigma _{p}}(\sigma _{p}-\sigma )[u]_{{B}%
_{p,p}^{\sigma }}^{p}\leq C\Vert u\Vert^p _{{KS}_{p,\infty }^{\sigma _{p}}}.
\label{KS_conv}
\end{equation}
\end{theorem}

Under the two-sided heat kernel estimates, we have the following two
equivalences.

\begin{theorem}
\label{thm1}If a metric measure space $(M,d,\mu )$ admits a heat kernel $%
\{p_{t}\}_{t>0}$ satisfying \eqref{hk}, then for any $\sigma >0$, $\mathcal{D%
}(E_{p,\infty }^{\sigma })=B_{p,\infty }^{\sigma }$ and for all $u\in
B_{p,\infty }^{\sigma }$,
\begin{equation}
E_{p,\infty }^{\sigma }(u)\asymp \lbrack u]_{B_{p,\infty }^{\sigma }}^{p}%
\text{\ and\ }\Vert u\Vert _{KS_{p,\infty }^{\sigma }}^{p}\asymp
\limsup_{t\rightarrow 0}\Psi _{u}^{\sigma }(t).  \label{eq1.7}
\end{equation}%
In addition, $\mathcal{D}(E_{p,p}^{\sigma })=B_{p,p}^{\sigma }$ and for all $%
u\in B_{p,p}^{\sigma }$,
\begin{equation}
E_{p,p}^{\sigma }(u)\asymp \lbrack u]_{B_{p,p}^{\sigma }}^{p}.  \label{eq1.8}
\end{equation}
\end{theorem}

\begin{theorem}
\label{thm4}If a metric measure space $(M,d,\mu )$ admits a heat kernel $%
\{p_{t}\}_{t>0}$ satisfying \eqref{hk}, then we have the following
equivalences:

\begin{itemize}
\item[(i)~] $(\widetilde{KE})\Longleftrightarrow (\widetilde{NE})$ with the
same $\sigma>0$,

\item[(ii)] $(KE)\Longleftrightarrow (NE)$ with the same $\sigma>0$.
\end{itemize}
\end{theorem}

In \cite{GaoYuZhang2022PA}, we obtain the convergence of non-local $p$%
-energy forms to local $p$-energy forms using discrete energy-control
property (E) (see Definition \ref{dfE}). Here we use a more general
condition (VE), which also motivates us to explore the equivalence of
weak-monotonicity properties (VE) and (KE), (NE) on fractal spaces.

Whenever we mention a fractal glue-up (see its definition in Section \ref%
{sec5}), we equip it with the Euclidean metric and the $\alpha $-dimensional
Hausdorff measure. The $\alpha $-dimensional Hausdorff measure on a fractal
glue-up is $\alpha $-regular, so for simplicity we use $r^{\alpha }$ to
replace $V(x,r)$ throughout Section \ref{sec5}, and clearly such
modification keep the corresponding norms equivalent (to the previous
defined norms).

\begin{theorem}
\label{thm5}Let $K^{F}$ be a fractal glue-up, where $K$ is a connected
homogeneous p.c.f. self-similar set, then we have the following equivalences:

\begin{itemize}
\item[(i)~] $(\widetilde{VE}) \Longleftrightarrow (\widetilde{NE})$ with the
same $\sigma>\alpha /p$ and $R_0=1$,

\item[(ii)] $(VE) \Longleftrightarrow (NE)$ with the same $\sigma>\alpha /p$
and $R_0=1$.
\end{itemize}
\end{theorem}

Theorem \ref{thm4} and Theorem \ref{thm5} are very important in our paper with many corollaries. In Section \ref{subsec5.4}, we will apply them to obtain Corollary \ref{corol1.8} which serves as a bridge for verifying weak monotonicity properties, and further deduce Theorem \ref{corol1.9}
focusing on nested
fractals and their blowups. Moreover, they also deduce Corollary \ref{jjjj} and Corollary \ref{jjj} that are
important to Dirichlet forms when $p=2$.

\subsection{Structure of the paper}

The organization of the paper is as follows. In Section \ref{sec4} we
present our proof of two kinds of BBM convergence without heat kernel
estimates in Theorem \ref{thm4.1} and Theorem \ref{thm4.2}. In Section \ref%
{sec2}, we present our proof of Theorem \ref{thm1} and Theorem \ref{thm4}.
In Section \ref{sec5}, we first introduce `fractal glue-ups' $K^F$ using
connected homogeneous p.c.f. self-similar set $K$, which gives bounded and
unbounded fractal spaces by properly choosing the set $F$ of similitudes,
including fractal blowups in Subsection \ref{subsec5.1}. Based on our
previous work \cite{GaoYuZhang2022PA}, we study the equivalent discrete $p$%
-energy norms in Lemma \ref{lem6.2} and Lemma \ref{lem6.3} which will be
used in Subsection \ref{subsec5.3} to prove Theorem \ref{thm5}. In
Subsection \ref{subsec5.2}, we introduce vertex-type weak-monotonicity
properties and show that they hold for nested fractal (glue-ups). In
Subsection \ref{subsec5.3}, we show the equivalence of properties (VE) and
(NE) on spaces $K^{F}$. In Subsection \ref{subsec5.4}, we prove that the BBM
convergence and Gagliardo-Nirenberg inequality hold true for the blow-ups of
$K$. We leave some conjectures and remarks in the last section.

\textbf{Notation}: The letters $C$,$C^{\prime }$, $C_{i}$,$C_{i}^{\prime }$,
$C_{i}^{\prime\prime }$, $c$ are universal positive constants depending only
on $M$ which may vary at each occurrence. The sign $\asymp $ means that both
$\leq $ and $\geq $ are true with uniform values of $C$ depending only on $M$%
. $a\wedge b:=\min \{a,b\}$, $a\vee b:=\max \{a,b\}$.

\section{BBM Convergence for two types of $p$-energy norms}

\label{sec4} In this section, we prove BBM convergence for heat kernel-based
$p$-energy norms without further heat kernel estimates, and also the BBM
convergence for Besov-Korevaar-Schoen $p$-energy norms. Under the settings
in Section \ref{sec2}, these two BBM convergence results will coincide.

\subsection{BBM Convergence for heat kernel-based $p$-energy norms}

From now on we fix $p\in (1,\infty)$. We need the following embedding
relation in our proof.

\begin{proposition}
\label{Prop32}For any $\delta \in (0,\sigma )$, we have
\begin{equation}
\mathcal{D}(E_{p,p}^{\sigma })\subseteq \mathcal{D}(E_{p,\infty }^{\sigma
})\subseteq \mathcal{D}(E_{p,p}^{\sigma -\delta }).  \label{306}
\end{equation}
\end{proposition}

\begin{proof}
Clearly, $\mathcal{D}(E_{p,p}^{\sigma })\subseteq \mathcal{D}(E_{p,\infty
}^{\sigma })$. Next, we prove%
\begin{equation}
\mathcal{D}(E_{p,\infty }^{\sigma })\subseteq \mathcal{D}(E_{p,p}^{\sigma
-\delta }).  \label{305}
\end{equation}

By the elementary inequality $|a-b|^{p}\leq 2^{p-1}(|a|^{p}+|b|^{p})$ and
symmetry, we have
\begin{align}
& \int_{M}\int_{M}|u(x)-u(y)|^{p}p_{t}(x,y)d\mu (x)d\mu (y)  \notag \\
\leq & 2^{p-1}\int_{M}\int_{M}(|u(x)|^{p}+|u(y)|^{p})p_{t}(x,y)d\mu (x)d\mu
(y)  \notag \\
=& 2^{p}\int_{M}|u(x)|^{p}\left( \int_{M}p_{t}(x,y)d\mu (y)\right) d\mu
(x)\leq 2^{p}\Vert u\Vert _{p}^{p}.  \label{304}
\end{align}

Therefore, fix $\epsilon \in (0,R_{0}^{\beta ^{\ast }})$, by \eqref{304},%
\begin{eqnarray*}
E_{p,p}^{\sigma -\delta }(u) &=&\int_{0}^{R_{0}^{\beta ^{\ast }}}\frac{1}{t^{%
{p(\sigma -\delta )}/{\beta ^{\ast }}}}\left(
\int_{M}\int_{M}|u(x)-u(y)|^{p}p_{t}(x,y)d\mu (x)d\mu (y)\right) \frac{dt}{t}
\\
&\leq &\int_{0}^{\epsilon }\frac{1}{t^{{p(\sigma -\delta )}/{\beta ^{\ast }}}%
}\left( \int_{M}\int_{M}|u(x)-u(y)|^{p}p_{t}(x,y)d\mu (x)d\mu (y)\right)
\frac{dt}{t} \\
&&+\int_{\epsilon }^{\infty }\frac{1}{t^{{p(\sigma -\delta )}/{\beta ^{\ast }%
}}}\left( \int_{M}\int_{M}|u(x)-u(y)|^{p}p_{t}(x,y)d\mu (x)d\mu (y)\right)
\frac{dt}{t} \\
&\leq &\int_{0}^{\epsilon }t^{-1+p\delta /{\beta ^{\ast }}}dt\sup_{t\in
(0,R_{0}^{\beta ^{\ast }})}\Psi _{u}^{\sigma }(t)+2^{p}\Vert u\Vert
_{p}^{p}\int_{\epsilon }^{\infty }\frac{dt}{t^{1+{p(\sigma -\delta )}/{\beta
^{\ast }}}} \\
&=&\frac{{\beta ^{\ast }}\epsilon ^{p\delta /{\beta ^{\ast }}}}{p\delta }%
\sup_{t\in (0,R_{0}^{\beta ^{\ast }})}\Psi _{u}^{\sigma }(t)+2^{p}\Vert
u\Vert _{p}^{p}\frac{{\beta ^{\ast }}\epsilon ^{-p(\sigma -\delta )/{\beta
^{\ast }}}}{p(\sigma -\delta )} \\
&=&\frac{{\beta ^{\ast }}\epsilon ^{p\delta /{\beta ^{\ast }}}}{p\delta }%
E_{p,\infty }^{\sigma }(u)+\frac{2^{p}{\beta ^{\ast }}\Vert u\Vert _{p}^{p}}{%
p(\sigma -\delta )}\epsilon ^{-p(\sigma -\delta )/{\beta ^{\ast }}}<\infty ,
\end{eqnarray*}
thus showing \eqref{305}.
\end{proof}

\begin{proof}[\textbf{Proof of Theorem \protect\ref{thm4.1}}]
Let $u\in \mathcal{D}(E_{p,\infty }^{\tilde{\sigma _{p}}})$ and $\sigma\in
(0,\tilde{\sigma}_{p})$. By Proposition \ref{Prop32}, $u\in \mathcal{D}%
(E_{p,p}^{\sigma })$. Note that $\Psi _{u}^{\sigma }(t)=t^{p(\tilde{\sigma}%
_{p}-\sigma )/{\beta ^{\ast }}}\Psi _{u}^{\tilde{\sigma}_{p}}(t).$

For the left-hand side, by \eqref{EEE}, we have
\begin{align}
\liminf_{\sigma \uparrow \tilde{\sigma}_{p}}(\tilde{\sigma}_{p}-\sigma ){E}%
_{p,p}^{\sigma }(u)=& \liminf_{\sigma \uparrow \tilde{\sigma}_{p}}(\tilde{%
\sigma}_{p}-\sigma )\int_{0}^{R_{0}^{{\beta ^{\ast }}}}t^{p(\tilde{\sigma}%
_{p}-\sigma )/{\beta ^{\ast }}}\Psi _{u}^{\tilde{\sigma}_{p}}(t)\frac{dt}{t}
\notag \\
\geq & \liminf_{\sigma \uparrow \tilde{\sigma}_{p}}(\tilde{\sigma}%
_{p}-\sigma )\int_{0}^{\tilde{\sigma}_{p}-\sigma }t^{p(\tilde{\sigma}%
_{p}-\sigma )/{\beta ^{\ast }}}\Psi _{u}^{\tilde{\sigma}_{p}}(t)\frac{dt}{t}
\notag \\
\geq & \liminf_{\sigma \uparrow \tilde{\sigma}_{p}}(\tilde{\sigma}%
_{p}-\sigma )\int_{0}^{\tilde{\sigma}_{p}-\sigma }t^{p(\tilde{\sigma}%
_{p}-\sigma )/{\beta ^{\ast }}}\frac{dt}{t}\inf_{t\in (0,\tilde{\sigma}%
_{p}-\sigma )}\Psi _{u}^{\tilde{\sigma}_{p}}(t)  \notag \\
=& \frac{{\beta ^{\ast }}}{p}\liminf_{\sigma \uparrow \tilde{\sigma}_{p}}(%
\tilde{\sigma}_{p}-\sigma )^{p(\tilde{\sigma}_{p}-\sigma )/{\beta ^{\ast }}%
}\inf_{t\in (0,\tilde{\sigma}_{p}-\sigma )}\Psi _{u}^{\tilde{\sigma}_{p}}(t)
\notag \\
=& \frac{{\beta ^{\ast }}}{p}\liminf_{\sigma \uparrow \tilde{\sigma}%
_{p}}\inf_{t\in (0,\tilde{\sigma}_{p}-\sigma )}\Psi _{u}^{\tilde{\sigma}%
_{p}}(t)= \frac{{\beta ^{\ast }}}{p}\liminf_{t\rightarrow 0}\Psi _{u}^{%
\tilde{\sigma}_{p}}(t).  \notag
\end{align}%
It follows from property (KE) that
\begin{equation*}
\liminf_{\sigma \uparrow \tilde{\sigma}_{p}}(\tilde{\sigma}_{p}-\sigma ){E}%
_{p,p}^{\sigma }(u)\geq \frac{C{\beta ^{\ast }}}{p}E_{p,\infty }^{\tilde{%
\sigma}_{p}}(u).
\end{equation*}

Next we prove the right-hand side. Let $A\in (0,R_{0})$ be a finite positive
number. By (\ref{304}), for any $\sigma \in (0,\tilde{\sigma _{p}})$,
\begin{eqnarray*}
{E}_{p,p}^{\sigma }(u) &=&\int_{0}^{R_{0}^{\beta ^{\ast }}}\frac{1}{t^{{%
p\sigma }/{\beta ^{\ast }}}}\left(
\int_{M}\int_{M}|u(x)-u(y)|^{p}p_{t}(x,y)d\mu (x)d\mu (y)\right) \frac{dt}{t}
\\
&\leq &\int_{0}^{A^{\beta ^{\ast }}}\frac{1}{t^{{p\sigma }/{\beta ^{\ast }}}}%
\left( \int_{M}\int_{M}|u(x)-u(y)|^{p}p_{t}(x,y)d\mu (x)d\mu (y)\right)
\frac{dt}{t}+\int_{A^{\beta ^{\ast }}}^{\infty }\frac{2^{p}\Vert u\Vert
_{p}^{p}}{t^{{p\sigma }/{\beta ^{\ast }}}}\frac{dt}{t} \\
&=&\int_{0}^{A^{\beta ^{\ast }}}t^{p(\tilde{\sigma}_{p}-\sigma )/{\beta
^{\ast }}}\Psi _{u}^{\tilde{\sigma}_{p}}(t)\frac{dt}{t}+\frac{2^{p}{\beta
^{\ast }}\Vert u\Vert _{p}^{p}}{p\sigma A^{{p\sigma }}}.
\end{eqnarray*}
It follows that
\begin{align}
\limsup_{\sigma \uparrow \tilde{\sigma}_{p}}(\tilde{\sigma}_{p}-\sigma ){E}%
_{p,p}^{\sigma }(u)&\leq \limsup_{\sigma \uparrow \tilde{\sigma}_{p}}(\tilde{%
\sigma}_{p}-\sigma )\int_{0}^{A^{\beta ^{\ast }}}t^{p(\tilde{\sigma}%
_{p}-\sigma )/{\beta ^{\ast }}}\Psi _{u}^{\tilde{\sigma}_{p}}(t)\frac{dt}{t}%
+\limsup_{\sigma \uparrow \tilde{\sigma}_{p}}(\tilde{\sigma}_{p}-\sigma )%
\frac{2^{p}{\beta ^{\ast }}\Vert u\Vert _{p}^{p}}{p\sigma A^{{p\sigma }}}
\notag \\
&\leq \limsup_{\sigma \uparrow \tilde{\sigma}_{p}}(\tilde{\sigma}_{p}-\sigma
)\int_{0}^{A^{\beta ^{\ast }}}t^{p(\tilde{\sigma}_{p}-\sigma )/{\beta ^{\ast
}}}\frac{dt}{t}\sup_{t\in (0,A^{{\beta ^{\ast }}})}\Psi _{u}^{\tilde{\sigma}%
_{p}}(t)+0  \notag \\
&= \frac{{\beta ^{\ast }}}{p}\limsup_{\sigma \uparrow \tilde{\sigma}%
_{p}}A^{p(\tilde{\sigma}_{p}-\sigma )}\sup_{t\in (0,A^{{\beta ^{\ast }}%
})}\Psi _{u}^{\tilde{\sigma}_{p}}(t)  \notag \\
&= \frac{{\beta ^{\ast }}}{p}\sup_{t\in (0,A^{{\beta ^{\ast }}})}\Psi _{u}^{%
\tilde{\sigma}_{p}}(t)\leq \frac{{\beta ^{\ast }}}{p}E_{p,\infty }^{\tilde{%
\sigma}_{p}}(u).  \label{NE_right}
\end{align}%
The proof is complete.
\end{proof}

\subsection{BBM Convergence for Besov-Korevaar-Schoen $p$-energy norms}

The proof of this version is similar to the previous one.

\begin{proposition}
\label{Prop31}For any $\delta \in (0,\sigma )$, we have
\begin{equation}
B_{p,p}^{\sigma }\subseteq B_{p,\infty }^{\sigma }\subseteq B_{p,p}^{\sigma
-\delta }.  \label{301}
\end{equation}
\end{proposition}

\begin{proof}
Clearly, $B_{p,p}^{\sigma }\subseteq B_{p,\infty }^{\sigma }$. Next, we prove%
\begin{equation}
B_{p,\infty }^{\sigma }\subseteq B_{p,p}^{\sigma -\delta }.  \label{302}
\end{equation}

By the elementary inequality $|a-b|^{p}\leq 2^{p-1}(|a|^{p}+|b|^{p})$, we
have
\begin{align}
& \int_{M}\frac{1}{V(x,r)}\int_{B(x,r)}|u(x)-u(y)|^{p}d\mu (y)d\mu (x)
\notag \\
\leq & 2^{p-1}\int_{M}\frac{1}{V(x,r)}\int_{B(x,r)}(|u(x)|^{p}+|u(y)|^{p})d%
\mu (y)d\mu (x)  \notag \\
=& 2^{p-1}\left( \Vert u\Vert _{p}^{p}+\int_{M}\int_{M}\frac{\mathbf{1}%
_{\{d(y,x)<r\}}}{V(x,r)}|u(y)|^{p}d\mu (y)d\mu (x)\right)  \notag \\
=& 2^{p-1}\left( \Vert u\Vert _{p}^{p}+\int_{M}\left( \int_{B(y,r)}\frac{1}{%
V(x,r)}d\mu (x)\right) |u(y)|^{p}d\mu (y)\right)  \notag \\
\leq & 2^{p-1}\left( \Vert u\Vert _{p}^{p}+\sup_{x\in M}\frac{V(x,2r)}{V(x,r)%
}\int_{M}|u(y)|^{p}d\mu (y)\right) \leq C\Vert u\Vert _{p}^{p}.  \label{300}
\end{align}

Therefore, fix $\epsilon \in (0,R_{0})$, by \eqref{300},%
\begin{eqnarray*}
\lbrack u]_{B_{p,p}^{\sigma -\delta }}^{p} &=&\int_{0}^{R_{0}}r^{-(p\sigma
-p\delta )}\int_{M}\frac{1}{V(x,r)}\int_{B(x,r)}|u(x)-u(y)|^{p}d\mu (y)d\mu
(x)\frac{dr}{r} \\
&\leq &\int_{0}^{\epsilon }r^{-(p\sigma -p\delta )}\int_{M}\frac{1}{V(x,r)}
\int_{B(x,r)}|u(x)-u(y)|^{p}d\mu (y)d\mu (x)\frac{dr}{r} \\
&&+\int_{\epsilon }^{\infty }r^{-(p\sigma -p\delta )}\int_{M} \frac{1}{V(x,r)%
}\int_{B(x,r)}|u(x)-u(y)|^{p}d\mu (y)d\mu (x)\frac{dr}{r} \\
&\leq &\int_{0}^{\epsilon }r^{-1+p\delta }dr\sup_{r\in (0,R_{0})}\Phi
_{u}^{\sigma }(r)+C\Vert u\Vert _{p}^{p}\int_{\epsilon }^{\infty
}r^{-(p\sigma -p\delta )}\frac{dr}{r} \\
&=&\frac{\epsilon ^{p\delta }}{p\delta }\sup_{r\in (0,R_{0})}\Phi
_{u}^{\sigma }(r)+C\Vert u\Vert _{p}^{p}\frac{\epsilon ^{-p(\sigma -\delta )}%
}{p(\sigma -\delta )} \\
&=&\frac{\epsilon ^{p\delta }}{p\delta }[u]_{B_{p,\infty }^{\sigma
}}^{p}+C^{\prime }\Vert u\Vert _{p}^{p}<\infty ,
\end{eqnarray*}
thus showing \eqref{302}.
\end{proof}

\begin{proof}[\textbf{Proof of Theorem \protect\ref{thm4.2}}]
We show \eqref{NE_conv} first. For the left-hand side, similarly, we have
that for any $\sigma _{p}>0,$
\begin{align}
\liminf_{\sigma \uparrow \sigma _{p}}(\sigma _{p}-\sigma )[u]_{{B}%
_{p,p}^{\sigma }}^{p}=& \liminf_{\sigma \uparrow \sigma _{p}}(\sigma
_{p}-\sigma )\int_{0}^{R_{0}}r^{p(\sigma _{p}-\sigma )}\Phi _{u}^{\sigma
_{p}}(r)\frac{dr}{r}  \notag \\
\geq & \liminf_{\sigma \uparrow \sigma _{p}}(\sigma _{p}-\sigma
)\int_{0}^{\sigma _{p}-\sigma }r^{p(\sigma _{p}-\sigma )}\Phi _{u}^{\sigma
_{p}}(r)\frac{dr}{r}  \notag \\
\geq & \liminf_{\sigma \uparrow \sigma _{p}}(\sigma _{p}-\sigma
)\int_{0}^{\sigma _{p}-\sigma }r^{p(\sigma _{p}-\sigma )}\frac{dr}{r}%
\inf_{r\in (0,\sigma _{p}-\sigma )}\Phi _{u}^{\sigma _{p}}(r)  \notag \\
=& \frac{1}{p}\liminf_{\sigma \uparrow \sigma _{p}}\inf_{r\in (0,\sigma
_{p}-\sigma )}\Phi _{u}^{\sigma _{p}}(r)=\frac{1}{p}\liminf_{r\rightarrow
0}\Phi _{u}^{\sigma _{p}}(r).  \label{408}
\end{align}%
Therefore, by property (NE), we have for $\sigma _{p}=\sigma _{p}^{\#}$ that
\begin{equation}
\liminf_{\sigma \uparrow \sigma _{p}^{\#}}(\sigma _{p}^{\#}-\sigma )[u]_{{B}%
_{p,p}^{\sigma }}^{p}\geq \frac{C^{-1}}{p}[u]_{{B}_{p,\infty }^{\sigma
_{p}^{\#}}}^{p},  \label{NE_left}
\end{equation}%
where the constant $C$ is the same as in \eqref{NE1}.

For the other side, let $A\in (0,R_{0})$ be a finite positive number. We
have from \eqref{300} that
\begin{eqnarray}
\lbrack u]_{{B}_{p,p}^{\sigma }}^{p} &=&\int_{0}^{R_{0}}r^{-p\sigma }\left(
\int_{M}\frac{1}{V(x,r)}\int_{B(x,r)}|u(x)-u(y)|^{p}d\mu (y)d\mu (x)\right)
\frac{dr}{r}  \notag \\
&\leq &\int_{0}^{A}r^{-p\sigma }\left( \int_{M}\frac{1}{V(x,r)}%
\int_{B(x,r)}|u(x)-u(y)|^{p}d\mu (y)d\mu (x)\right) \frac{dr}{r}%
+C\int_{A}^{\infty }r^{-p\sigma }\Vert u\Vert _{p}^{p}\frac{dr}{r}  \notag \\
&=&\int_{0}^{A}r^{p(\sigma _{p}^{\#}-\sigma )}\Phi _{u}^{\sigma _{p}^{\#}}(r)%
\frac{dr}{r}+\frac{C\Vert u\Vert _{p}^{p}}{p\sigma A^{p\sigma }}.
\label{300-1}
\end{eqnarray}%
It follows that
\begin{align}
\limsup_{\sigma \uparrow \sigma _{p}^{\#}}(\sigma _{p}^{\#}-\sigma )[u]_{{B}%
_{p,p}^{\sigma }}^{p}& \leq \limsup_{\sigma \uparrow \sigma
_{p}^{\#}}(\sigma _{p}^{\#}-\sigma )\int_{0}^{A}r^{p(\sigma _{p}^{\#}-\sigma
)}\Phi _{u}^{\sigma _{p}^{\#}}(r)\frac{dr}{r}+\limsup_{\sigma \uparrow
\sigma _{p}^{\#}}(\sigma _{p}^{\#}-\sigma )\frac{C\Vert u\Vert _{p}^{p}}{%
p\sigma A^{p\sigma }}  \notag \\
& \leq \limsup_{\sigma \uparrow \sigma _{p}^{\#}}(\sigma _{p}^{\#}-\sigma
)\int_{0}^{A}r^{p(\sigma _{p}^{\#}-\sigma )}\frac{dr}{r}\sup_{r\in
(0,R_{0})}\Phi _{u}^{\sigma _{p}^{\#}}(r)+0  \notag \\
& =\frac{1}{p}\limsup_{\sigma \uparrow \sigma _{p}^{\#}}A^{p(\sigma
_{p}^{\#}-\sigma )}\sup_{r\in (0,R_{0})}\Phi _{u}^{\sigma _{p}^{\#}}(r)=%
\frac{1}{p}[u]_{B_{p,\infty }^{\sigma _{p}^{\#}}}^{p},  \label{42}
\end{align}%
thus showing \eqref{NE_conv}.

Next, we show \eqref{KS_conv}. By property $(\widetilde{NE})$ with $\sigma
_{p}$ and \eqref{408}, we have
\begin{equation}
\liminf_{\sigma \uparrow \sigma _{p}}(\sigma _{p}-\sigma )[u]_{{B}%
_{p,p}^{\sigma }}^{p}\geq C^{-1}\Vert u\Vert _{KS_{p,p}^{\sigma _{p}}}^{p}.
\label{KS_left}
\end{equation}%
Then for the other side of \eqref{KS_conv}, by the definition of $\limsup $,
there exists $\epsilon \in (0,R_{0})$ such that
\begin{equation*}
\sup_{r\in (0,\epsilon )}\Phi _{u}^{\sigma _{p}}(r)\leq
2\limsup_{r\rightarrow 0}\Phi _{u}^{\sigma _{p}}(r).
\end{equation*}%
It follows from (\ref{300-1}) (replace $\sigma _{p}^{\#}$ by $\sigma _{p}$
and take $A=\epsilon $) that,%
\begin{align}
\limsup_{\sigma \uparrow \sigma _{p}}(\sigma _{p}-\sigma )[u]_{{B}%
_{p,p}^{\sigma }}^{p}& \leq \limsup_{\sigma \uparrow \sigma _{p}}(\sigma
_{p}-\sigma )\int_{0}^{\epsilon }r^{p(\sigma _{p}-\sigma )}\Phi _{u}^{\sigma
_{p}}(r)\frac{dr}{r}+\limsup_{\sigma \uparrow \sigma _{p}}(\sigma
_{p}-\sigma )\frac{C\Vert u\Vert _{p}^{p}}{p\sigma \epsilon ^{p\sigma }}
\notag \\
& \leq \limsup_{\sigma \uparrow \sigma _{p}}(\sigma _{p}-\sigma
)\int_{0}^{\epsilon }r^{p(\sigma _{p}-\sigma )}\frac{dr}{r}\sup_{r\in
(0,\epsilon )}\Phi _{u}^{\sigma _{p}}(r)+0  \notag \\
& =\frac{1}{p}\sup_{r\in (0,\epsilon )}\Phi _{u}^{\sigma _{p}}(r)\leq \frac{2%
}{p}\limsup_{r\rightarrow 0}\Phi _{u}^{\sigma _{p}}(r)=\frac{2}{p}\Vert
u\Vert _{{KS}_{p,\infty }^{\sigma _{p}}}^{p}.  \label{KS_right}
\end{align}%
The proof is complete.
\end{proof}

\section{Equivalences on metric measure spaces}

\label{sec2} In this section, we consider metric measure spaces with heat
kernels satisfying \eqref{hk}, and fix $\sigma>0$.

%Once \eqref{hk} holds, we know that $\mu$ is $\alpha$-regular. For
%convenience, in this case, we as well rewrite (\ref{pi}) by
%\begin{equation*}
%\lbrack u]_{B_{p,\infty }^{\sigma }}^{p}:=\sup_{r\in
%(0,R_{0})}r^{-p\sigma-\alpha }\int_{M}\int_{B(x,r)}|u(x)-u(y)|^{p}d\mu
%(y)d\mu (x),
%\end{equation*}
%and
%\begin{equation*}
%\lbrack u]_{B_{p,p}^{\sigma }}^{p}:=\int_{0}^{R_{0}}r^{-p\sigma-\alpha
%}\left( \int_{M}\int_{B(x,r)}|u(x)-u(y)|^{p}d\mu (y)d\mu (x)\right) \frac{dr%
%}{r}
%\end{equation*}
%for (\ref{pp}).

\subsection{Equivalence of heat kernel-based and Besov norms}

We prove Theorem \ref{thm1} by pure elementary analysis (without probability
arguments). Some similar results can be found in \cite[Proposition 4.2]{AlonsoBaudoinchen2020CVPDE} or \cite{Pietruska-Paluba.2010.}. We naturally
split the proof into the following five lemmas.

\begin{lemma}
\label{lem3.1} If $(M,d,\mu )$ admits a heat kernel $\{p_{t}\}_{t>0}$
satisfying \eqref{hk}, then there exists a positive constant $C$ such that
for all $u\in \mathcal{D}(E_{p,p}^{\sigma })$,
\begin{equation}
\lbrack u]_{B_{p,p}^{\sigma }}^{p}\leq CE_{p,p}^{\sigma }(u).  \label{eq3.1}
\end{equation}
\end{lemma}

\begin{proof}
By the lower bound of heat kernel in \eqref{hk}, we have
\begin{align*}
E_{p,p}^{\sigma }(u)& \geq \int_{0}^{R_{0}^{\beta ^{\ast }}}\frac{1}{t^{{%
p\sigma }/{\beta ^{\ast }}}}\left( \int_{M}\int_{B(x,t^{{1}/{\beta ^{\ast }}%
})}|u(x)-u(y)|^{p}p_{t}(x,y)d\mu (y)d\mu (x)\right) \frac{dt}{t}  \notag \\
& \geq c_{1}\int_{0}^{R_{0}^{\beta ^{\ast }}}\frac{1}{t^{p\sigma/\beta
^{\ast }}} \left( \int_{M}\frac{1}{V(x,t^{1/\beta^*})}\int_{B(x,t^{{1}/{%
\beta ^{\ast }}})}|u(x)-u(y)|^{p}\exp \left( -c_{2}\left( \frac{d(x,y)}{%
t^{1/\beta ^{\ast }}}\right) ^{\frac{\beta ^{\ast }}{\beta ^{\ast }-1}%
}\right) d\mu (y)d\mu (x)\right) \frac{dt}{t}  \notag \\
& \geq c_{1}e^{-c_{2}}\int_{0}^{R_{0}^{\beta ^{\ast }}}\frac{1}{%
t^{p\sigma/\beta ^{\ast }}} \left( \int_{M}\frac{1}{V(x,t^{1/\beta^*})}%
\int_{B(x,t^{{1}/{\beta ^{\ast }}})}|u(x)-u(y)|^{p}d\mu (y)d\mu (x)\right)
\frac{dt}{t}  \notag \\
& =\beta ^{\ast }c_{1}e^{-c_{2}}\int_{0}^{R_{0}}\frac{1}{r^{p\sigma}}\left(
\int_{M}\frac{1}{V(x,r)}\int_{B(x,r)}|u(x)-u(y)|^{p}d\mu (y)d\mu (x)\right)
\frac{dr}{r}=\beta ^{\ast }c_{1}e^{-c_{2}}[u]_{B_{p,p}^{\sigma }}^{p},
\end{align*}%
which completes the proof.
\end{proof}

%Now we prove the opposite inequality of \eqref{eq3.1}.

\begin{lemma}
\label{lem3.2} If $(M,d,\mu )$ admits a heat kernel $\{p_{t}\}_{t>0}$
satisfying \eqref{hk}, then there exists a positive constant $C$ such that
for all $u\in B_{p,p}^{\sigma }$,
\begin{equation*}
E_{p,p}^{\sigma }(u)\leq C[u]_{B_{p,p}^{\sigma }}^{p}.
\end{equation*}
\end{lemma}

\begin{proof}
We split $M$ into union of rings $B(x,2^{n}t^{1/\beta ^{\ast }})\setminus
B(x,2^{n-1}t^{1/\beta ^{\ast }})$, where the integers $n\leq \log
_{2}(R_{0}t^{-1/\beta ^{\ast }})$, then by \eqref{hk}, we have
\begin{align}
& \int_{M}\int_{M}|u(x)-u(y)|^{p}p_{t}(x,y)d\mu (y)d\mu (x)  \notag \\
=& \int_{M}\sum_{n\leq \log _{2}(R_{0}t^{-1/\beta ^{\ast
}})}\int_{B(x,2^{n}t^{1/\beta ^{\ast }})\setminus B(x,2^{n-1}t^{1/\beta
^{\ast }})}|u(x)-u(y)|^{p}p_{t}(x,y)d\mu (y)d\mu (x)  \notag \\
\leq & \int_{M}\frac{c_{3}}{V(x,t^{1/\beta ^{\ast }})}\sum_{n\leq \log
_{2}(R_{0}t^{-1/\beta ^{\ast }})}\int_{B(x,2^{n}t^{1/\beta ^{\ast
}})\setminus B(x,2^{n-1}t^{1/\beta ^{\ast }})}|u(x)-u(y)|^{p}\exp \left(
-c_{4}2^{\frac{(n-1)\beta ^{\ast }}{\beta ^{\ast }-1}}\right) d\mu (y)d\mu
(x)  \notag \\
\leq & \sum_{n\leq \log _{2}(R_{0}t^{-1/\beta ^{\ast }})}\int_{M}\frac{c_{3}%
}{V(x,t^{1/\beta ^{\ast }})}\int_{B(x,2^{n}t^{1/\beta ^{\ast
}})}|u(x)-u(y)|^{p}\exp \left( -c_{4}2^{\frac{(n-1)\beta ^{\ast }}{\beta
^{\ast }-1}}\right) d\mu (y)d\mu (x)  \notag \\
=&\sum_{n=-\infty }^{\infty }\frac{1_{\{n\leq \log _{2}(R_{0}t^{-1/\beta
^{\ast }})\}}(n)}{t^{p\sigma /\beta ^{\ast }}}\int_{M}\frac{c_{3}}{%
V(x,t^{1/\beta ^{\ast }})}\int_{B(x,2^{n}t^{1/\beta ^{\ast
}})}|u(x)-u(y)|^{p}\exp \left( -c_{4}2^{\frac{(n-1)\beta ^{\ast }}{\beta
^{\ast }-1}}\right) d\mu (y)d\mu (x).  \label{eq3.3}
\end{align}

Therefore,%
\begin{align*}
& E_{p,p}^{\sigma }(u)=\int_{0}^{R_{0}^{\beta ^{\ast }}}\frac{1}{t^{{p\sigma
}/{\beta ^{\ast }}}}\left( \int_{M}\int_{M}|u(x)-u(y)|^{p}p_{t}(x,y)d\mu
(x)d\mu (y)\right) \frac{dt}{t} \\
& \leq \int_{0}^{R_{0}^{\beta ^{\ast }}}\left( \sum_{n=-\infty }^{\infty }%
\frac{1_{\{n\leq \log _{2}(R_{0}t^{-1/\beta ^{\ast }})\}}(n)}{t^{p\sigma
/\beta ^{\ast }}}\int_{M}\frac{c_{3}}{V(x,t^{1/\beta ^{\ast }})}%
\int_{B(x,2^{n}t^{1/\beta ^{\ast }})}|u(x)-u(y)|^{p}\exp \left( -c_{4}2^{%
\frac{(n-1)\beta ^{\ast }}{\beta ^{\ast }-1}}\right) d\mu (y)d\mu (x)\right)
\frac{dt}{t} \\
& \leq \sum_{n=-\infty }^{\infty }\exp \left( -c_{4}2^{\frac{(n-1)\beta
^{\ast }}{\beta ^{\ast }-1}}\right) \int_{0}^{2^{-n\beta ^{\ast
}}R_{0}^{\beta ^{\ast }}}\frac{1}{t^{p\sigma /\beta ^{\ast }}}\left( \int_{M}%
\frac{c_{3}}{V(x,t^{1/\beta ^{\ast }})}\int_{B(x,2^{n}t^{1/\beta ^{\ast
}})}|u(x)-u(y)|^{p}d\mu (y)d\mu (x)\right) \frac{dt}{t} \\
& \leq \sum_{n=-\infty }^{\infty }\exp \left( -c_{4}2^{\frac{(n-1)\beta
^{\ast }}{\beta ^{\ast }-1}}\right) \int_{0}^{2^{-n\beta ^{\ast
}}R_{0}^{\beta ^{\ast }}}\frac{C_{1}2^{(p\sigma +\alpha _{1})n}}{%
(2^{n}t^{1/\beta ^{\ast }})^{p\sigma }}\left( \int_{M}\frac{1}{%
V(x,2^{n}t^{1/\beta ^{\ast }})}\int_{B(x,2^{n}t^{1/\beta ^{\ast
}})}|u(x)-u(y)|^{p}d\mu (y)d\mu (x)\right) \frac{dt}{t} \\
& =C_{1}\beta ^{\ast }\sum_{n=-\infty }^{\infty }\exp \left( -c_{4}2^{\frac{%
(n-1)\beta ^{\ast }}{\beta ^{\ast }-1}}\right) 2^{(p\sigma +\alpha
_{1})n}\int_{0}^{R_{0}}\frac{1}{r^{p\sigma }}\left( \int_{M}\frac{1}{V(x,r)}%
\int_{B(x,r)}|u(x)-u(y)|^{p}d\mu (y)d\mu (x)\right) \frac{dr}{r} \\
& =C_{1}\beta ^{\ast }C_{0}[u]_{B_{p,p}^{\sigma }}^{p},
\end{align*}%
where $C_{1}=c_{3}C_{d}$, we use the volume doubling property \eqref{VD2} in
the fifth line
\begin{equation}
\frac{V(x,2^{n}t^{1/\beta ^{\ast }})}{V(x,t^{1/\beta ^{\ast }})}\leq
C_{d}2^{n\alpha _{1}},  \label{VD}
\end{equation}%
and
\begin{eqnarray}
C_{0} &=&\sum_{n=-\infty }^{\infty }\exp \left( -c_{4}2^{\frac{(n-1)\beta
^{\ast }}{\beta ^{\ast }-1}}\right) 2^{(p\sigma +\alpha _{1})n}  \notag \\
&\leq &\sum_{n=-\infty }^{0}2^{(p\sigma +\alpha _{1})n}+\sum_{n=1}^{\infty
}\exp \left( -c_{4}2^{\frac{(n-1)\beta ^{\ast }}{\beta ^{\ast }-1}}\right)
2^{(p\sigma +\alpha _{1})n}<\infty ,  \label{c1}
\end{eqnarray}%
which completes the proof.
\end{proof}

%Next, we prove the second inclusion in Theorem \ref{thm1} by deriving the upper bound estimate of $[u]_{B_{p,\infty }^{\sigma }}^{p}$ in Lemma \ref%{lem3.3} and the lower bound estimate of $[u]_{B_{p,\infty }^{\sigma }}^{p}$in Lemma \ref{lem3.4}.

\begin{lemma}
\label{lem3.3} If $(M,d,\mu )$ admits a heat kernel $\{p_{t}\}_{t>0}$
satisfying \eqref{hk}, then there exists a positive constant $C$ such that
for all $u\in \mathcal{D}(E_{p,\infty }^{\sigma })$,
\begin{equation*}
\lbrack u]_{B_{p,\infty }^{\sigma }}^{p}\leq CE_{p,\infty }^{\sigma }(u).
\end{equation*}
\end{lemma}

\begin{proof}
By the lower bound of heat kernel in \eqref{hk}, we have
\begin{align*}
E_{p,\infty }^{\sigma }(u)& =\sup_{t\in (0,R_{0}^{\beta ^{\ast }})}\frac{1}{%
t^{p\sigma /\beta ^{\ast }}}\int_{M}\int_{M}|u(x)-u(y)|^{p}p_{t}(x,y)d\mu
(y)d\mu (x) \\
& \geq \sup_{t\in (0,R_{0}^{\beta ^{\ast }})}\frac{1}{t^{p\sigma /\beta
^{\ast }}}\int_{M}\int_{B(x,t^{1/\beta ^{\ast
}})}|u(x)-u(y)|^{p}p_{t}(x,y)d\mu (y)d\mu (x) \\
& \geq c_{1}\sup_{t\in (0,R_{0}^{\beta ^{\ast }})}\frac{1}{t^{p\sigma /\beta
^{\ast }}}\int_{M}\frac{1}{V(x,t^{1/\beta^*})}\int_{B(x,t^{1/\beta ^{\ast
}})}|u(x)-u(y)|^{p}\exp \left( -c_{2}\left( \frac{d(x,y)}{t^{1/\beta ^{\ast
}}}\right) ^{\frac{\beta ^{\ast }}{\beta ^{\ast }-1}}\right) d\mu (y)d\mu (x)
\\
& \geq c_{1}e^{-c_{2}}\sup_{t\in (0,R_{0}^{\beta ^{\ast }})} \frac{1}{%
t^{p\sigma /\beta ^{\ast }}}\int_{M}\frac{1}{V(x,t^{1/\beta^*})}%
\int_{B(x,t^{1/\beta ^{\ast }})}|u(x)-u(y)|^{p}d\mu (y)d\mu (x) \\
& =c_{1}e^{-c_{2}}\sup_{r\in (0,R_{0})}\frac{1}{r^{p\sigma}} \int_{M}\frac{1%
}{V(x,r)}\int_{B(x,r)}|u(x)-u(y)|^{p}d\mu (y)d\mu
(x)=c_{1}e^{-c_{2}}[u]_{B_{p,\infty }^{\sigma }}^{p},
\end{align*}%
which ends the proof.
\end{proof}

\begin{lemma}
\label{lem3.4} If $(M,d,\mu )$ admits a heat kernel $\{p_{t}\}_{t>0}$
satisfying \eqref{hk}, then there exists a positive constant $C$ such that
for all $u\in B_{p,\infty }^{\sigma }$,
\begin{equation*}
E_{p,\infty }^{\sigma }(u)\leq C[u]_{B_{p,\infty }^{\sigma }}^{p}.
\end{equation*}
\end{lemma}

\begin{proof}
By \eqref{eq3.3} and \eqref{VD}, we have
\begin{align}
& \frac{1}{t^{p\sigma /\beta ^{\ast }}}\int_{M}%
\int_{M}|u(x)-u(y)|^{p}p_{t}(x,y)d\mu (y)d\mu (x)  \notag \\
\leq &\frac{1}{t^{p\sigma /\beta ^{\ast }}}\sum_{n=-\infty }^{\infty }\frac{%
1_{\{n\leq \log _{2}(R_{0}t^{-1/\beta ^{\ast }})\}}(n)}{t^{p\sigma /\beta
^{\ast }}}\int_{M}\frac{c_{3}}{V(x,t^{1/\beta ^{\ast }})}%
\int_{B(x,2^{n}t^{1/\beta ^{\ast }})}|u(x)-u(y)|^{p}\exp \left( -c_{4}2^{%
\frac{(n-1)\beta ^{\ast }}{\beta ^{\ast }-1}}\right) d\mu (y)d\mu (x)  \notag
\\
\leq &c_{3}C_{d}\sum_{n=-\infty }^{\infty }\int_{M}\frac{2^{(p\sigma +\alpha
_{1})n}1_{\{n\leq \log _{2}(R_{0}t^{-1/\beta ^{\ast }})\}}(n)}{%
(2^{n}t^{1/\beta ^{\ast }})^{p\sigma }V(x,2^{n}t^{1/\beta ^{\ast }})}%
\int_{B(x,2^{n}t^{1/\beta ^{\ast }})}|u(x)-u(y)|^{p}\exp \left( -c_{4}2^{%
\frac{(n-1)\beta ^{\ast }}{\beta ^{\ast }-1}}\right) d\mu (y)d\mu (x)  \notag
\\
\leq &c_{3}C_{d}\sum_{n=-\infty }^{\infty }\exp \left( -c_{4}2^{\frac{%
(n-1)\beta ^{\ast }}{\beta ^{\ast }-1}}\right) 2^{(p\sigma +\alpha
_{1})n}\sup_{r\in (0,R_{0})}\frac{1}{r^{p\sigma }}\int_{M}\frac{1}{V(x,r)}%
\int_{B(x,r)}|u(x)-u(y)|^{p}d\mu (y)d\mu (x)  \notag \\
=&c_{3}C_{d}\sum_{n=-\infty }^{\infty }\exp \left( -c_{4}2^{\frac{(n-1)\beta
^{\ast }}{\beta ^{\ast }-1}}\right) 2^{(p\sigma +\alpha
_{1})n}[u]_{B_{p,\infty }^{\sigma }}^{p}=c_{3}C_{d}C_{0}[u]_{B_{p,\infty
}^{\sigma }}^{p},  \label{eq3.14}
\end{align}%
where $C_{0}$ is finite by \eqref{c1}.

Finally, taking the supremum in $(0,R_{0}^{\beta ^{\ast }})$ on both sides
of \eqref{eq3.14}, we complete the proof.
\end{proof}

If $(M,d,\mu )$ admits a heat kernel $\{p_{t}\}_{t>0}$ satisfying \eqref{hk}%
, by Lemma \ref{lem3.1} and Lemma \ref{lem3.2}, we know that $\mathcal{D}%
(E_{p,p}^{\sigma })=B_{p,p}^{\sigma }$ and (\ref{eq1.8}). We obtain $%
\mathcal{D}(E_{p,\infty }^{\sigma })=B_{p,\infty }^{\sigma }$ and $%
E_{p,\infty }^{\sigma }(u)\asymp \lbrack u]_{B_{p,\infty }^{\sigma }}^{p}$
from Lemma \ref{lem3.3} and Lemma \ref{lem3.4}. It remains to show $\Vert
u\Vert _{KS_{p,\infty }^{\sigma }}^{p}\asymp \limsup_{t\rightarrow 0}\Psi
_{u}^{\sigma }(t)$ in (\ref{eq1.7}) for all $u\in B_{p,\infty }^{\sigma }$
when two-sided estimates \eqref{hk} hold.

\begin{lemma}
\label{corol3.5} If $(M,d,\mu )$ admits a heat kernel $\{p_{t}\}_{t>0}$
satisfying \eqref{hk}, then for all $u\in B_{p,\infty }^{\sigma }$,
\begin{equation}
\limsup_{t\rightarrow 0}\Psi _{u}^{\sigma }(t)\asymp \limsup_{r\rightarrow
0}\Phi _{u}^{\sigma }(r).  \label{limsup_psiphi}
\end{equation}
\end{lemma}

\begin{proof}
By Lemma \ref{lem3.3}, we have%
\begin{equation}
\sup_{r\in (0,R_{0})}\Phi _{u}^{\sigma }(r)\leq C\sup_{t\in (0,R_{0}^{\beta
^{\ast }})}\Psi _{u}^{\sigma }(t),  \label{31}
\end{equation}%
for any $R_{0}\in (0,$\textrm{diam}$(M)]$. Therefore, if we pick $%
R_{0}=\epsilon $ in \eqref{31}, then
\begin{align*}
\limsup_{t\rightarrow 0}\Psi _{u}^{\sigma }(t)& =\lim_{\epsilon \rightarrow
0}\sup_{t\in (0,\epsilon )}\Psi _{u}^{\sigma }(t) \\
& \geq C_{1}\lim_{\epsilon \rightarrow 0}\sup_{r\in (0,\epsilon ^{1/\beta
^{\ast }})}\Phi _{u}^{\sigma }(r)=C_{1}\limsup_{r\rightarrow 0}\Phi
_{u}^{\sigma }(r).
\end{align*}

Next we consider the other side of \eqref{limsup_psiphi}. By \eqref{eq3.3},
we have
\begin{align}
& \frac{1}{t^{p\sigma /\beta ^{\ast }}}\int_{M}%
\int_{M}|u(x)-u(y)|^{p}p_{t}(x,y)d\mu (y)d\mu (x)  \notag \\
\leq & \sum_{n=-\infty }^{\infty }\frac{1_{\{n\leq \log
_{2}(R_{0}t^{-1/\beta ^{\ast }})\}}(n)}{t^{p\sigma /\beta ^{\ast }}}\int_{M}%
\frac{c_{3}}{V(x,t^{1/\beta ^{\ast }})}\int_{B(x,2^{n}t^{1/\beta ^{\ast
}})}|u(x)-u(y)|^{p}\exp \left( -c_{4}2^{\frac{(n-1)\beta ^{\ast }}{\beta
^{\ast }-1}}\right) d\mu (y)d\mu (x).  \label{32}
\end{align}

Hence, we have by \eqref{VD}, \eqref{32} and the dominated convergence
theorem (the dominant function can be taken to be
\begin{equation*}
C_{d}\exp \left( -c_{4}2^{\frac{(n-1)\beta ^{\ast }}{\beta ^{\ast }-1}%
}\right) 2^{(p\sigma +\alpha _{1})n}[u]_{B_{p,\infty }^{\sigma }}^{p}
\end{equation*}%
by virtue of the last four lines of \eqref{eq3.14}) that
\begin{align*}
& \limsup_{t\rightarrow 0}\frac{1}{t^{p\sigma /\beta ^{\ast }}}%
\int_{M}\int_{M}|u(x)-u(y)|^{p}p_{t}(x,y)d\mu (y)d\mu (x) \\
\leq & c_{3}\limsup_{t\rightarrow 0}\sum_{n=-\infty }^{\infty }\frac{%
1_{\{n\leq \log _{2}(R_{0}t^{-1/\beta ^{\ast }})\}}(n)}{t^{p\sigma /\beta
^{\ast }}}\int_{M}\frac{1}{V(x,t^{1/\beta ^{\ast }})}\int_{B(x,2^{n}t^{1/%
\beta ^{\ast }})}|u(x)-u(y)|^{p}\exp \left( -c_{4}2^{\frac{(n-1)\beta ^{\ast
}}{\beta ^{\ast }-1}}\right) d\mu (y)d\mu (x) \\
\leq & C_{1}\sum_{n=-\infty }^{\infty }\exp \left( -c_{4}2^{\frac{(n-1)\beta
^{\ast }}{\beta ^{\ast }-1}}\right) \limsup_{t\rightarrow 0}\frac{1_{\{n\leq
\log _{2}(R_{0}t^{-1/\beta ^{\ast }})\}}(n)}{(2^{n}t^{1/\beta ^{\ast
}})^{p\sigma }}\int_{M}\frac{2^{(p\sigma +\alpha _{1})n}}{%
V(x,2^{n}t^{1/\beta ^{\ast }})}\int_{B(x,2^{n}t^{1/\beta ^{\ast
}})}|u(x)-u(y)|^{p}d\mu (y)d\mu (x) \\
\leq & C_{1}\sum_{n=-\infty }^{\infty }\exp \left( -c_{4}2^{\frac{(n-1)\beta
^{\ast }}{\beta ^{\ast }-1}}\right) 2^{(p\sigma +\alpha
_{1})n}\limsup_{r\rightarrow 0}\Phi _{u}^{\sigma }(r)\leq
C_{1}C_{0}\limsup_{r\rightarrow 0}\Phi _{u}^{\sigma }(r),
\end{align*}%
where $C_{1}=c_{3}C_{d}$ and the constant $C_{0}$ is defined in \eqref{c1},
which ends the proof.
\end{proof}

\subsection{Some propositions for heat kernel estimates}

\label{sec4.2} Any heat kernel gives rise to a heat semigroup $%
\{P_{t}\}_{t>0}$, where $P_{t}$ is an operator in $L^{2}(M,\mu )$ defined by
\begin{equation*}
P_{t}u(x)=\int_{M}p_{t}(x,y)u(y)d\mu (y).
\end{equation*}%
The properties (1)--(4) of $\{p_{t}\}_{t>0}$ (in Subsection \ref{secPre}) imply that $P_{t}$ is a bounded
self-adjoint operator in $L^{2}(M,\mu )$; moreover, $\{P_{t}\}_{t>0}$ is a
strongly continuous, positivity-preserving, contractive semigroup in $%
L^{2}(M,\mu )$. We also obtain the following tail estimate with the same
form as in \cite[Theorem 2.1]{GrigoryanHu.2014.MMJ505}, where the metric
space must be unbounded. We still use analytic tools to obtain the tail
estimate for both bounded and unbounded metric spaces.

Firstly, we introduce some basic notions following \cite%
{GrigoryanHuHu.2017.JFA3311} in heat kernel estimates. If heat kernel $%
p_{t}(x,y)$ exists and satisfies the upper bound of \eqref{hk}, then we say
the condition ($UE_{\beta^*}$) holds. As we assume that $\mu$ is volume
doubling, there is an equivalence (see \cite{GrigoryanHu.2014.MMJ505})
\begin{align*}
(UE_{\beta^*})\Leftrightarrow (DUE_{\beta^*})+(T_{\text{exp}}),
\end{align*}
where $(DUE_{\beta^*})$, $(T_{\text{exp}})$ will be introduced shortly.

%In this paper, we will notstudy a equivalence of condition ($LE$), so we omit it (see \cite[Theorem 2.8%]{GrigoryanHuHu.2017.JFA3311}). In turn, If heat kernel $p_{t}(x,y)$ exists andsatisfies the lower bound of \eqref{hk}, then we say the condition ($LE$)holds.

\begin{definition}
(\cite[Definition 2.4]{GrigoryanHuHu.2017.JFA3311}) \label{due} We say that
condition ($DUE_{\beta^* }$) holds (short for `diagonal upper estimate'), if
there exists a heat kernel $p_{t}(x,y)$ satisfying the following estimate
\begin{equation}
p_{t}(x,y)\leq \frac{C}{\sqrt{V(x,t^{1/\beta ^{\ast }})V(y,t^{1/\beta ^{\ast
}})}},  \label{due_b}
\end{equation}%
for any $t\in (0,R_{0}^{\beta ^{\ast }})$ and for $\mu $-almost all $x,y\in
M $. If $\mu $ is $\alpha $-regular, then \eqref{due_b} becomes%
\begin{equation}
p_{t}(x,y)\leq \frac{C}{t^{\alpha /\beta ^{\ast }}}  \label{due_c}
\end{equation}%
for any $t\in (0,R_{0}^{\beta ^{\ast }})$ and for $\mu $-almost all $x,y\in
M $.
\end{definition}

Using the semigroup property, one can deduce that $(DUE_{\beta^*})+(T_{\text{%
exp}})\Rightarrow (UE_{\beta^*})$. It is trivial that $(DUE_{\beta^*})$
follows from $(UE_{\beta^*})$, then we give an elementary argument for $%
(UE_{\beta^*})\Rightarrow(T_{\text{exp}})$.

\begin{proposition}
\label{tail_hk} If a heat kernel $p_{t}(x,y)$ satisfies the upper bound in %
\eqref{hk} and measure $\mu$ satisfies \eqref{VD1}, then we have the
following exponential tail estimate (termed ($T_{\text{exp}}$)),
\begin{equation}
P_{t}1_{B(x,r)^{c}}(x)\leq C\exp \left( -c\left( \frac{r}{t^{1/\beta ^{\ast
}}}\right) ^{\frac{\beta ^{\ast }}{\beta ^{\ast }-1}}\right) ,
\label{eqtail}
\end{equation}%
for all $x\in M$ and $t\in (0,R_{0}^{\beta ^{\ast }})$.
\end{proposition}

\begin{proof}
By the upper bound of \eqref{hk}, for $t\in (0,R_{0}^{\beta ^{\ast }})$, we
have
\begin{align*}
P_{t}1_{B(x,r)^{c}}(x)& =\int_{B(x,r)^{c}}p_{t}(x,y)d\mu (y)\leq
\int_{B(x,r)^{c}}\frac{c_{3}}{V(x,t^{1/\beta ^{\ast }})}\exp \left(
-c_{4}\left( \frac{d(x,y)}{t^{1/\beta ^{\ast }}}\right) ^{\frac{\beta ^{\ast
}}{\beta ^{\ast }-1}}\right) d\mu (y) \\
& \leq \sum_{n=0}^{\infty }\int_{B(x,2^{n+1}r)\backslash B(x,2^{n}r)}\frac{%
c_{3}}{V(x,t^{1/\beta ^{\ast }})}\exp \left( -c_{4}\left( \frac{2^{n}r}{%
t^{1/\beta ^{\ast }}}\right) ^{\frac{\beta ^{\ast }}{\beta ^{\ast }-1}%
}\right) d\mu (y) \\
& \leq \sum_{n=0}^{\infty }\int_{B(x,2^{n+1}r)}\frac{c_{3}}{V(x,t^{1/\beta
^{\ast }})}\exp \left( -c_{4}\left( \frac{2^{n}r}{t^{1/\beta ^{\ast }}}%
\right) ^{\frac{\beta ^{\ast }}{\beta ^{\ast }-1}}\right) d\mu (y).
\end{align*}%
Therefore, by \eqref{VD2},
\begin{align}
P_{t}1_{B(x,r)^{c}}(x)& \leq \sum_{n=0}^{\infty }\frac{c_{3}}{V(x,t^{1/\beta
^{\ast }})}V(x,2^{n+1}r)\exp \left( -c_{4}\left( \frac{2^{n}r}{t^{1/\beta
^{\ast }}}\right) ^{\frac{\beta ^{\ast }}{\beta ^{\ast }-1}}\right)  \notag
\\
& =2^{\alpha _{1}}C_{1}\sum_{n=0}^{\infty }\left( \frac{2^{n}r}{t^{1/\beta
^{\ast }}}\right) ^{\alpha _{1}}\exp \left( -c_{4}\left( \frac{2^{n}r}{%
t^{1/\beta ^{\ast }}}\right) ^{\frac{\beta ^{\ast }}{\beta ^{\ast }-1}%
}\right)  \notag \\
& \leq C_{1}\int_{\frac{r}{2t^{1/\beta ^{\ast }}}}^{\infty }s^{\alpha
_{1}-1}\exp \left( -c_{4}s^{\frac{\beta ^{\ast }}{\beta ^{\ast }-1}}\right)
ds,  \label{eq3.4}
\end{align}%
where $C_{1}=2^{\alpha _{1}}c_{3}C_{d}$, and $C_{d}$ is from \eqref{VD2}.
The last line in \eqref{eq3.4} is obtained by setting $t_{n}=2^{n}r/t^{1/%
\beta^* }$ for $n\geq -1$ and noting that
\begin{equation*}
\int_{t_{n}}^{t_{n+1}}s^{\alpha _{1}-1}\exp \left( -c_{4}s^{\frac{\beta
^{\ast }}{\beta ^{\ast }-1}}\right) ds\geq t_{n}^{\alpha _{1}-1}\exp \left(
-c_{4}t_{n}^{\frac{\beta ^{\ast }}{\beta ^{\ast }-1}}\right)
(t_{n+1}-t_{n})=t_{n}^{\alpha _{1}}\exp \left( -c_{4}t_{n}^{\frac{\beta
^{\ast }}{\beta ^{\ast }-1}}\right) .
\end{equation*}%
Next, we estimate the term $\int_{\frac{r}{2t^{1/\beta ^{\ast }}}}^{\infty
}s^{\alpha _{1}-1}\exp \left( -c_{4}s^{\frac{\beta ^{\ast }}{\beta ^{\ast }-1%
}}\right) ds$. Indeed,
\begin{align}
& \int_{\frac{r}{2t^{1/\beta ^{\ast }}}}^{\infty }s^{\alpha _{1}-1}\exp
\left( -c_{4}s^{\frac{\beta ^{\ast }}{\beta ^{\ast }-1}}\right) ds  \notag \\
\leq & \int_{\frac{r}{2t^{1/\beta ^{\ast }}}}^{\infty }s^{\alpha _{1}-1}\exp
\left( -\frac{c_{4}}{2}s^{\frac{\beta ^{\ast }}{\beta ^{\ast }-1}}\right)
\exp \left( -\frac{c_{4}}{2}\left( \frac{r}{2t^{1/\beta ^{\ast }}}\right) ^{%
\frac{\beta ^{\ast }}{\beta ^{\ast }-1}}\right) ds  \notag \\
\leq & \exp \left( -\frac{c_{4}}{2}\left( \frac{r}{2t^{1/\beta ^{\ast }}}%
\right) ^{\frac{\beta ^{\ast }}{\beta ^{\ast }-1}}\right) \int_{0}^{\infty
}s^{\alpha _{1}-1}\exp \left( -\frac{c_{4}}{2}s^{\frac{\beta ^{\ast }}{\beta
^{\ast }-1}}\right) ds  \notag \\
=& \exp \left( -\frac{c_{4}}{2}\left( \frac{r}{2t^{1/\beta ^{\ast }}}\right)
^{\frac{\beta ^{\ast }}{\beta ^{\ast }-1}}\right) \frac{\beta ^{\ast }-1}{%
\beta ^{\ast }}\left( \frac{c_{4}}{2}\right) ^{-\alpha _{1}(\beta ^{\ast
}-1)/\beta ^{\ast }}\int_{0}^{\infty }m^{\alpha _{1}(\beta ^{\ast }-1)/\beta
^{\ast }-1}\exp \left( -m\right) dm  \notag \\
=& \frac{\beta ^{\ast }-1}{\beta ^{\ast }}\left( \frac{c_{4}}{2}\right)
^{-\alpha _{1}(\beta ^{\ast }-1)/\beta ^{\ast }}\Gamma (\alpha _{1}(\beta
^{\ast }-1)/\beta ^{\ast })\exp \left( -\frac{c_{4}}{2}\left( \frac{r}{%
2t^{1/\beta ^{\ast }}}\right) ^{\frac{\beta ^{\ast }}{\beta ^{\ast }-1}%
}\right) .  \label{eq3.4.1}
\end{align}%
Whence \eqref{eqtail} follows from \eqref{eq3.4.1} with $c=2^{-\frac{2\beta
^{\ast }-1}{\beta ^{\ast }-1}}c_{4}$, $C=C_{1}(\beta ^{\ast }-1)/\beta
^{\ast }\left( c_{4}/2\right) ^{-\alpha _{1}(\beta ^{\ast }-1)/\beta ^{\ast
}}\Gamma (\alpha _{1}(\beta ^{\ast }-1)/\beta ^{\ast })$.
\end{proof}

We need to use the following proposition adapted from the proof of \cite[%
Lemma 4.13]{AlonsoBaudoinchen2021CVPDE} (without proving) and we present its
proof here for later use.

\begin{proposition}
\label{hk_ct} Suppose that $M$ admits a heat kernel satisfying \eqref{hk},
then there exist $C,c>1$ and $c^{\prime }>0$ such that for any $\delta >0$
and for $\mu$-almost all $x,y\in M$ with $d(x,y)>\delta t^{1/\beta ^{\ast }}$,
\begin{equation*}
p_{t}(x,y)\leq C\exp \left( -c^{\prime }\delta ^{\frac{\beta ^{\ast }}{\beta
^{\ast }-1}}\right) p_{ct}(x,y).
\end{equation*}
\end{proposition}

\begin{proof}
By the upper bound of heat kernel in \eqref{hk} and volume doubling property %
\eqref{VD2}, we have
\begin{align*}
p_{t}(x,y)& \leq \frac{c_{3}}{V(x,t^{1/\beta ^{\ast }})}\exp \left(
-c_{4}\left( \frac{d(x,y)}{t^{1/\beta ^{\ast }}}\right) ^{\frac{\beta ^{\ast
}}{\beta ^{\ast }-1}}\right) \\
& \leq \frac{c_{3}C_{d}c^{\alpha _{1}/\beta ^{\ast }}}{V(x,(ct)^{1/\beta
^{\ast }})}\exp \left( -c_{4}\left( \frac{d(x,y)}{(ct)^{1/\beta ^{\ast }}}%
\right) ^{\frac{\beta ^{\ast }}{\beta ^{\ast }-1}}\cdot c^{\frac{1}{\beta
^{\ast }-1}}\right) \\
& =\frac{c_{3}C_{d}c^{\alpha _{1}/\beta ^{\ast }}}{c_{1}}\frac{c_{1}}{%
V(x,(ct)^{1/\beta ^{\ast }})}\exp \left( -c_{2}\left( \frac{d(x,y)}{%
(ct)^{1/\beta ^{\ast }}}\right) ^{\frac{\beta ^{\ast }}{\beta ^{\ast }-1}%
}\cdot \frac{c_{4}}{c_{2}}\cdot c^{\frac{1}{\beta ^{\ast }-1}}\right),
\end{align*}%
for $\mu$-almost all $x,y\in M$. Taking $\tilde{c}=c_{4}c^{\frac{1}{\beta ^{\ast }-1}}-c_{2}$, then we use
lower bound of \eqref{hk} to obtain
\begin{equation*}
p_{t}(x,y)\leq \frac{c_{3}C_{d}c^{\alpha _{1}/\beta ^{\ast }}}{c_{1}}\exp
\left( -\tilde{c}\left( \frac{d(x,y)}{(ct)^{1/\beta ^{\ast }}}\right) ^{%
\frac{\beta ^{\ast }}{\beta ^{\ast }-1}}\right) p_{ct}(x,y).
\end{equation*}%
We choose $c>1$ to ensure $c^{\frac{1}{\beta ^{\ast }-1}}c_{4}>c_{2}$ so
that $\tilde{c}>0$. Let $c^{\prime }={\tilde{c}}/{c^{\frac{1}{\beta ^{\ast
}-1}}}$, then for $d(x,y)>\delta t^{1/\beta ^{\ast }}$,
\begin{align*}
p_{t}(x,y)\leq & \frac{c_{3}C_{d}c^{\alpha _{1}/\beta ^{\ast }}}{c_{1}}\exp
\left( -c^{\prime }\left( \frac{d(x,y)}{t^{1/\beta ^{\ast }}}\right) ^{\frac{%
\beta ^{\ast }}{\beta ^{\ast }-1}}\right) p_{ct}(x,y) \\
\leq & C\exp \left( -c^{\prime }\delta ^{\frac{\beta ^{\ast }}{\beta ^{\ast
}-1}}\right) p_{ct}(x,y),
\end{align*}%
where $C=\frac{c_{3}C_{d}c^{\alpha _{1}/\beta ^{\ast }}}{c_{1}}$. The proof
is complete.
\end{proof}

\subsection{Equivalence of properties (KE) and (NE)}

We start by an easy side $(NE)\Rightarrow (KE)$ which holds automatically.

\begin{lemma}
\label{thm_2} Suppose that $(M,d,\mu )$ admits a heat kernel $%
\{p_{t}\}_{t>0} $ satisfying \eqref{hk}. Then there exists $C>0$ such that
for all $u\in \mathcal{D}(E_{p,\infty }^{\sigma })$,
\begin{equation*}
\Psi _{u}^{\sigma }(t)\geq C\Phi _{u}^{\sigma }(t^{1/\beta ^{\ast }}).
\end{equation*}%
Consequently,
\begin{equation}
\liminf_{t\rightarrow 0}\Psi _{u}^{\sigma }(t)\geq C\liminf_{r\rightarrow
0}\Phi _{u}^{\sigma }(r),  \label{limpsi}
\end{equation}
where the constant $C$ depends only on $c_{1}$, $c_{2}$ in \eqref{hk}.
\end{lemma}

\begin{proof}
By the lower bound of \eqref{hk}, we have
\begin{align*}
\Psi _{u}^{\sigma }(t)& =\frac{1}{t^{p\sigma /\beta ^{\ast }}}%
\int_{M}\int_{M}p_{t}(x,y)|u(x)-u(y)|^{p}d\mu (y)d\mu (x) \\
& \geq \frac{1}{t^{p\sigma /\beta ^{\ast }}}\int_{M}\int_{B(x,t^{1/\beta
^{\ast }})}|u(x)-u(y)|^{p}p_{t}(x,y)d\mu (y)d\mu (x) \\
& \geq c_{1}\frac{1}{t^{p\sigma /\beta ^{\ast }}}\cdot \int_{M}\frac{1}{%
V(x,t^{1/\beta ^{\ast }})}\int_{B(x,t^{1/\beta ^{\ast
}})}|u(x)-u(y)|^{p}\exp \left( -c_{2}\left( \frac{d(x,y)}{t^{1/\beta ^{\ast
}}}\right) ^{\frac{\beta ^{\ast }}{\beta ^{\ast }-1}}\right) d\mu (y)d\mu (x)
\\
& \geq c_{1}e^{-c_{2}}\frac{1}{t^{p\sigma /\beta ^{\ast }}}\int_{M}\frac{1}{%
V(x,t^{1/\beta ^{\ast }})}\int_{B(x,t^{1/\beta ^{\ast
}})}|u(x)-u(y)|^{p}d\mu (y)d\mu (x)=c_{1}e^{-c_{2}}\Phi _{u}(t^{1/\beta
^{\ast }}).
\end{align*}%
Taking the lower limit on both sides of the above inequality, we have
\begin{equation*}
\liminf_{t\rightarrow 0}\Psi _{u}^{\sigma }(t)\geq
c_{1}e^{-c_{2}}\liminf_{t\rightarrow 0}\Phi _{u}^{\sigma }(t^{1/\beta ^{\ast
}})=c_{1}e^{-c_{2}}\liminf_{r\rightarrow 0}\Phi _{u}^{\sigma }(r),
\end{equation*}%
which proves \eqref{limpsi}.
\end{proof}

The proof for the other side $(KE)\Rightarrow(NE)$ is inspired by \cite[%
Lemma 4.13]{AlonsoBaudoinchen2021CVPDE} where $p=1$ therein.

\begin{lemma}
\label{thm_1} Suppose that $(M,d,\mu )$ admits a heat kernel $%
\{p_{t}\}_{t>0} $ satisfying \eqref{hk}. Under the property $(\widetilde{KE}%
) $ with $\sigma >0$, we have
\begin{equation}
\liminf_{t\rightarrow 0}\Psi _{u}^{\sigma }(t)\leq C\liminf_{r\rightarrow
0}\Phi _{u}^{\sigma }(r),  \label{limphiu}
\end{equation}
for all $u\in \mathcal{D}(E_{p,\infty }^{\sigma })$.
\end{lemma}

\begin{proof}
Temporally fix $u\in \mathcal{D}(E_{p,\infty }^{\sigma })$. Without loss of
generality, assume that $\liminf_{t\rightarrow 0}\Psi _{u}^{\sigma }(t)>0$.
Fix $\delta >1$ to be determined by (\ref{dd}). For $t\in (0,R_{0}^{\beta
^{\ast }})$, we decompose $\Psi _{u}^{\sigma }(t)$ according to the
space-time relation
\begin{equation*}
\Psi _{u}^{\sigma }(t)=\Psi _{1}(t)+\Psi _{2}(t),
\end{equation*}%
where
\begin{align*}
\Psi _{1}(t)& =\frac{1}{t^{{p\sigma }/{\beta ^{\ast }}}}\iint_{\{d(x,y)\leq
\delta t^{1/\beta ^{\ast }}\}}|u(x)-u(y)|^{p}p_{t}(x,y)d\mu (y)d\mu (x), \\
\Psi _{2}(t)& =\frac{1}{t^{{p\sigma }/{\beta ^{\ast }}}}\iint_{\{d(x,y)>%
\delta t^{1/\beta ^{\ast }}\}}|u(x)-u(y)|^{p}p_{t}(x,y)d\mu (y)d\mu (x).
\end{align*}

For $\Psi _{1}(t)$, when $d(x,y)\leq \delta t^{1/\beta ^{\ast }}$, we have
\begin{align*}
\frac{c_{1}}{V(x,t^{1/\beta ^{\ast }})}\exp \left( -c_{2}\delta ^{\frac{%
\beta ^{\ast }}{\beta ^{\ast }-1}}\right) \leq & \frac{c_{1}}{V(x,t^{1/\beta
^{\ast }})}\exp \left( -c_{2}\left( \frac{d(x,y)}{t^{1/\beta ^{\ast }}}%
\right) ^{\frac{\beta ^{\ast }}{\beta ^{\ast }-1}}\right) \\
\leq & p_{t}(x,y)\leq \frac{c_{3}}{V(x,t^{1/\beta ^{\ast }})}\exp \left(
-c_{4}\left( \frac{d(x,y)}{t^{1/\beta ^{\ast }}}\right) ^{\frac{\beta ^{\ast
}}{\beta ^{\ast }-1}}\right) \leq \frac{c_{5}}{V(x,t^{1/\beta ^{\ast }})}.
\end{align*}%
Therefore, noting that $\delta>1$, we have from \eqref{VD2} that
\begin{align}
\Psi _{1}(t)& \leq \frac{c_{3}}{t^{p\sigma /\beta ^{\ast }}}%
\iint_{\{d(x,y)\leq \delta t^{1/\beta ^{\ast }}\}}\frac{|u(x)-u(y)|^{p}}{%
V(x,t^{1/\beta ^{\ast }})}d\mu (y)d\mu (x)  \notag \\
& \leq \frac{c_{3}C_{d}\delta ^{\alpha _{1}}}{t^{p\sigma /\beta ^{\ast }}}%
\iint_{\{d(x,y)\leq \delta t^{1/\beta ^{\ast }}\}}\frac{|u(x)-u(y)|^{p}}{%
V(x,\delta t^{1/\beta ^{\ast }})}d\mu (y)d\mu (x)=C_{1}\Phi _{u}^{\sigma
}(\delta t^{1/\beta ^{\ast }}),  \label{psi1}
\end{align}%
where $C_{1}=c_{3}C_{d}\delta ^{p\sigma +\alpha _{1}}$.

For $\Psi _{2}(t)$, by Proposition \ref{hk_ct}, we have
\begin{equation}
\Psi _{2}(t)\leq A\Psi _{u}^{\sigma }(ct),  \label{psi2}
\end{equation}%
where $A=Cc^{p\sigma /\beta ^{\ast }}\exp \left( -c^{\prime }\delta ^{\frac{%
\beta ^{\ast }}{\beta ^{\ast }-1}}\right) $ can be made as small as we
desire by making $\delta $ large enough since $C,c,c^{\prime }$ are
independent constants. So from now on we choose $\delta $ large enough such
that
\begin{equation}
A=Cc^{p\sigma /\beta ^{\ast }}\exp \left( -c^{\prime }\delta ^{\frac{\beta
^{\ast }}{\beta ^{\ast }-1}}\right) <\frac{1}{2}.  \label{dd}
\end{equation}
Combining \eqref{psi1} and \eqref{psi2}, we have%
\begin{equation}
\Psi _{u}^{\sigma }(t)\leq C_{1}\Phi _{u}^{\sigma }(\delta t^{1/\beta ^{\ast
}})+A\Psi _{u}^{\sigma }(ct).  \label{41}
\end{equation}

Let $t=c^{-(n+1)}$, it follows that
\begin{equation*}
\Psi _{u}^{\sigma }(c^{-(n+1)})\leq C_{1}\Phi _{u}^{\sigma }(\delta
c^{-(n+1)/\beta ^{\ast }})+A\Psi _{u}^{\sigma }(c^{-n}).
\end{equation*}%
Since $\limsup_{r\rightarrow 0}\Phi _{u}^{\sigma }(r)<\infty $ by Lemma \ref%
{corol3.5} and property $(\widetilde{KE})$, there exists $M_{0}$ such that
\begin{equation*}
\sup_{n\geq 1}\Phi _{u}^{\sigma }(\delta c^{-n/\beta ^{\ast }})\leq
M_{0}<\infty .
\end{equation*}%
Picking up $M_{1}=\max \{C_{1}M_{0},\Psi _{u}^{\sigma }(c^{-1})\}$, we have
\begin{equation*}
\Psi _{u}^{\sigma }(c^{-2})\leq (1+A)M_{1}.
\end{equation*}%
By induction, we have
\begin{equation*}
\Psi _{u}^{\sigma }(c^{-n})\leq M_{1}\left( \frac{1-A^{n}}{1-A}\right) .
\end{equation*}%
Letting $n\rightarrow \infty $, we obtain
\begin{equation*}
\liminf_{t\rightarrow 0}\Psi _{u}^{\sigma }(t)<\infty .
\end{equation*}

By property $(\widetilde{KE})$, there exists $t_{0}>0$ such that for all $%
t\in (0,t_{0})$,
\begin{equation*}
\Psi _{u}^{\sigma }(t)\leq \sup_{0<t<t_{0}}\Psi _{u}^{\sigma }(t)\leq
2\limsup_{t\rightarrow 0}\Psi _{u}^{\sigma }(t)\leq
C_{2}\liminf_{t\rightarrow 0}\Psi _{u}^{\sigma }(t),
\end{equation*}%
where $C_{2}$ comes from property $(\widetilde{KE})$. Therefore, we have
from \eqref{41} that, for all small enough $t\in (0,\frac{t_{0}}{c})$,
\begin{equation*}
\Psi _{u}^{\sigma }(t)\leq C_{1}\Phi _{u}^{\sigma }(\delta t^{1/\beta ^{\ast
}})+AC_{2}\liminf_{t\rightarrow 0}\Psi _{u}^{\sigma }(t).
\end{equation*}%
Fixing $\delta $ such that $AC_{2}<1$ with aforementioned requirement $A<%
\frac{1}{2}$, it follows that
\begin{equation*}
\liminf_{t\rightarrow 0}\Psi _{u}^{\sigma }(t)\leq
C_{1}\liminf_{t\rightarrow 0}\Phi _{u}^{\sigma
}(t)+AC_{2}\liminf_{t\rightarrow 0}\Psi _{u}^{\sigma }(t),
\end{equation*}%
which implies
\begin{equation*}
\liminf_{t\rightarrow 0}\Psi _{u}^{\sigma }(t)\leq
C_{3}\liminf_{t\rightarrow 0}\Phi _{u}^{\sigma }(t),
\end{equation*}%
where $C_{3}=\frac{C_{1}}{1-AC_{2}}$. The proof is complete.
\end{proof}

\begin{proof}[Proof of Theorem \protect\ref{thm4}]
Since two-sided estimates \eqref{hk} hold, using Theorem \ref{thm1}, we have
that $\mathcal{D}(E_{p,\infty }^{\sigma })=B_{p,\infty }^{\sigma }$ for any $%
\sigma >0$.

By \eqref{limsup_psiphi} and \eqref{limphiu}, $(\widetilde{KE})$ implies $(%
\widetilde{NE})$ since for all $u\in B_{p,\infty }^{\sigma }=\mathcal{D}%
(E_{p,\infty }^{\sigma })$,
\begin{equation}
\limsup_{r\rightarrow 0}\Phi _{u}^{\sigma }(r)\leq C\limsup_{t\rightarrow
0}\Psi _{u}^{\sigma }(t)\leq C^{\prime }\liminf_{t\rightarrow 0}\Psi
_{u}^{\sigma }(t)\leq C^{\prime \prime }\liminf_{r\rightarrow 0}\Phi
_{u}^{\sigma }(r).  \label{ke_ne1}
\end{equation}

By \eqref{limsup_psiphi} and \eqref{limpsi}, $(\widetilde{NE})$ implies $(%
\widetilde{KE})$ since
\begin{equation}
\limsup_{t\rightarrow 0}\Psi _{u}^{\sigma }(t)\leq
C_{1}\limsup_{r\rightarrow 0}\Phi _{u}^{\sigma }(r)\leq C_{1}^{\prime
}\liminf_{r\rightarrow 0}\Phi _{u}^{\sigma }(r)\leq C_{1}^{\prime \prime
}\liminf_{t\rightarrow 0}\Psi _{u}^{\sigma }(t).  \label{ne_ke1}
\end{equation}%
Therefore, by \eqref{ke_ne1} and \eqref{ne_ke1}, we show Theorem \ref{thm4}
(i).

By \eqref{limphiu} and Lemma \ref{lem3.3}, (KE) implies (NE) since
\begin{equation}
\sup_{r\in (0,R_{0})}\Phi _{u}^{\sigma }(r)\leq C_{2}\sup_{t\in
(0,R_{0}^{\beta ^{\ast }})}\Psi _{u}^{\sigma }(t)\leq C_{2}^{\prime
}\liminf_{t\rightarrow 0}\Psi _{u}^{\sigma }(t)\leq C_{2}^{\prime \prime
}\liminf_{r\rightarrow 0}\Phi _{u}^{\sigma }(r),  \label{ke_ne2}
\end{equation}%
where we use the simple fact (KE)$\Rightarrow (\widetilde{KE})$. Similarly,
by \eqref{limpsi} and Lemma \ref{lem3.4} adjoint with the fact (NE)$%
\Rightarrow (\widetilde{NE})$, we immediately derive (KE) since
\begin{equation}
\sup_{t\in (0,R_{0}^{\beta ^{\ast }})}\Psi _{u}^{\sigma }(t)\leq
C_{3}\sup_{r\in (0,R_{0})}\Phi _{u}^{\sigma }(r)\leq C_{3}^{\prime
}\liminf_{r\rightarrow 0}\Phi _{u}^{\sigma }(r)\leq C_{3}^{\prime \prime
}\liminf_{t\rightarrow 0}\Psi _{u}^{\sigma }(t).  \label{ne_ke2}
\end{equation}%
Thus, Theorem \ref{thm4} (ii) holds by \eqref{ke_ne2} and \eqref{ne_ke2}.
\end{proof}

By Theorem \ref{thm4} and Remark \ref{rk1}, we have the following
interesting corollary for $p=2$, which is important for defining a local
Dirichlet form based on Besov norms.

\begin{corollary}\label{jjjj}
Suppose that $(M,d,\mu )$ admits a heat kernel $\{p_{t}\}_{t>0}$ satisfying %
\eqref{hk}, then (NE) automatically holds when $p=2$.
\end{corollary}

\section{Connected homogeneous p.c.f self-similar sets and their glue-ups}

\label{sec5} In this section, we work with `fractal glue-ups' using
connected homogeneous p.c.f self-similar sets as `tiles', which can be both
bounded and unbounded. We first obtain their equivalent discrete semi-norms
(vertex energies). Then we introduce weak-monotonicity properties for vertex
energies, including property (E) in \cite[Definition 3.1]{GaoYuZhang2022PA},
properties (VE) and $(\widetilde{VE})$. We will verify property (E) on
nested fractals, and our main goal is to show the equivalence of (KE) and
(VE). Lastly, we show that aforementioned consequences hold for certain
`fractal glue-ups', for example, nested fractals and their fractal blow-ups.

\subsection{Equivalent discrete semi-norms of fractal glue-ups}

\label{subsec5.1} Let us first introduce connected homogeneous p.c.f.
self-similar sets in $\mathbb{R}^{d}$. Let $\{\phi _{i}\}_{i=1}^{N}$ be an
IFS which consists of $\rho $-similitudes on $\mathbb{R}^{d}$, i.e. for $%
i=1,...,N$, $N\geq 2$, each $\phi _{i}:\mathbb{R}^{d}\rightarrow \mathbb{R}%
^{d}$ is of the form
\begin{equation}
\phi _{i}(x)=\rho (x-b_{i})+b_{i},  \label{phi_i}
\end{equation}%
where $\rho \in (0,1)$, $b_{i}\in \mathbb{R}^{d}$. Suppose that $K$ is the
attractor of the IFS $\{\phi _{i}\}_{i=1}^{N}$, i.e. $K$ is the unique
non-empty compact set in $\mathbb{R}^{d}$ satisfying
\begin{equation*}
K=\tbigcup_{i=1}^{N}\phi _{i}(K).
\end{equation*}%
To explain what `p.c.f' is, we introduce the natural symbolic space
associated with our IFS $\{\phi _{i}\}_{i=1}^{N}$. Let $W=\{1,2,...,N\}$, $%
W_{n}$ be the set of words with length $n$ over $W$, and $W^{\mathbb{N}}$ be
the set of all infinite words over $W$. Let $w=\mathrm{w_{1}w_{2}}...\in W^{%
\mathbb{N}}$ and let the canonical projection $\pi :$ $W^{\mathbb{N}%
}\rightarrow K$ be defined by $\pi (w):=\tbigcap\limits_{n\in \mathbb{N}%
^{\ast }}F_{\mathrm{w_{1}...w_{n}}}(K),$ where $F_{\mathrm{w_{1}...w_{n}}%
}:=F_{\mathrm{w_{1}}}\circ ...\circ F_{\mathrm{w_{n}}}$. Then we can define
the critical set $\Gamma $ and the post-critical set $\mathcal{P}$ by
\begin{equation*}
\Gamma =\pi ^{-1}\left( \tbigcup_{1\leq i<j\leq N}\left( \phi _{i}(K)\cap
\phi _{j}(K)\right) \right) ,\quad \mathcal{P}=\tbigcup_{m\geq 1}\tau
^{m}(\Gamma ),
\end{equation*}%
where $\tau $ is the left shift by one index on $W^{\mathbb{N}}$ (see \cite[%
Definition 1.3.13]{Kigami.2001.}). We say the IFS is \emph{post-critically
finite} (p.c.f.) if $\mathcal{P}$ is finite. For any $w\in W_{n}$, define
\begin{equation*}
V_{0}=\pi (\mathcal{P}),\quad V_{n}=\tbigcup_{w\in W_{n}}F_{w}\left(
V_{0}\right) ,\quad V_{\ast }=\tbigcup_{n\geq 1}V_{n},
\end{equation*}%
then $K$ is the closure of $V_{\ast }$ and $V_{w}:=F_{w}\left( V_{0}\right) $
is a cell of $V_{n}$. Without loss of generality, we always assume that
\textrm{diam}$(K)=1$ by an affine transformation on $\{b_{i}\}_{i=1}^{N}$.
From now on, denote by $K$ a homogeneous p.c.f. self-similar set that is
connected. It is known that a homogeneous p.c.f. IFS in the form of (\ref%
{phi_i}) satisfies the \emph{open set condition} (OSC) (see, for example
\cite[Theorem 1.1]{Denglau.2008}). Hence the Hausdorff dimension of $K$ is
\begin{equation*}
\alpha =\func{dim}_{H}(K)=-\log N/\log \rho,
\end{equation*}%
and the $\mathcal{H}^{\alpha }$-measure of $K$, denoted by $\mu $, is \emph{$%
\alpha $-regular} with
\begin{equation*}
R_{0}=\mathrm{diam}(K)=1.
\end{equation*}

It is proved in \cite[Proposition 2.5]{GuLau.2020.AASFM} that, \emph{%
Condition(H)} holds for connected homogeneous p.c.f. self-similar sets. That
is, there exists $c>0$ depending only on $K$ such that, for any two words $w$
and $w^{\prime }$ with the same length $m\geq 1$, $K_{w}\cap K_{w^{\prime
}}=\emptyset $ implies that $\func{dist}\left( K_{w},K_{w^{\prime }}\right)
\geq c\rho ^{m}$.

Next, we introduce the construction of \emph{fractal glue-ups}. The
definition is adapted from the idea of fractafold in \cite%
{StrichartzTeplyaev2012} but not the same. Given a compact set $K\subset
\mathbb{R}^d$ and a set $F$ of similitudes on $\mathbb{R}^d$, we can define
\begin{equation*}
K^F:=\bigcup_{f\in F}f(K)\subset \mathbb{R}^d.
\end{equation*}
For our study, we need the following additional requirements.

\begin{enumerate}
\item Isometries: all similitudes in $F$ have contraction ratio 1, that is,
every similitude in $F$ is an isometry of $K$.

\item Just-touching property: for any $f\neq g\in F$,
\begin{equation*}
f(K)\tbigcap g(K)=f(V_0)\tbigcap g(V_0).
\end{equation*}

\item Condition (H): There exists a constant $C_{H}\in (0,1)$ such that, if $%
|x-y|<C_{H}\rho ^{m}$ with $x\in f_{1}(K_{w})$ and $y\in f_{2}(K_{\tilde{w}%
}) $ for two words $w$ and $\tilde{w}$ with the same length $m$, then $%
f_{1}(K_{w})$ intersects $f_{2}(K_{\tilde{w}})$.

\item Connectedness: $K^{F}$ is connected.
\end{enumerate}

We remark that conditions (2) and (3) imply the \emph{uniform finitely-joint}
property: for any cell $f(K)$, the number of cells $\tilde{f}(K)$ that
intersect it is bounded by a finite number independent of $f\in F$.

To see this, pick any vertex $f(v)$ where $v\in V_{0}$ and $f\in F$, and
denote all the cells that intersect it by $f_{n}(K),\ 1\leq n\leq N_{0}$
(here $N_{0}$ might be $\infty $). Choose an `interior' (away from
boundaries) level-2 cell $\phi _{w}(K)$ ($|w|=2$) that does not intersect $%
V_{0}$, which exists due to the p.c.f. condition. Condition (H) implies that
all $f_{n}(\phi _{w}K)$ are separated by distance $C_{H}\rho ^{2}$ for $%
1\leq n\leq N_{0}$, but they all belong to the ball $B_{1}(f(v))$! This
means that $N_{0}$ is uniformly bounded, since there are $N_{0}$ disjoint
balls $B_{C_{H}\rho ^{2}/2}(f_{n}(\phi _{w}(v)))$ inside a unit ball $%
B_{1}(f(v))$.

From now on, whenever we mention a \emph{fractal glue-up} $K^{F}$, we
automatically assume that these four conditions are satisfied. We can regard
$f(K)$ as tiles of $K^{F}$, and we just simply glue these tiles up which
only overlap at boundaries by the `just-touching' property.

This is a quick way to unify the notions of bounded and unbounded fractals.
For example, $K=K^{F}$ when $F=\{Id\}$, and we are interested in the
following unbounded fractals by blowing up $K$ defined by Strichartz in \cite%
[Section 3]{Strichartz.1998.CJM}.
\begin{example}[Fractal blow-up]\label{ex1} Let $\{\phi _{i}\}_{i=1}^{N}$ defined as in (\ref{phi_i}). Assume that
\textrm{diam}$(K)=1$ by an affine transformation on $\{b_{i}\}_{i=1}^{N}$ with $b_{1}=0$ and define
\begin{equation*}
K_{l}:=\underbrace{\phi _{1}^{-1}\circ \phi _{1}^{-1}\circ \cdots \circ \phi
_{1}^{-1}}_{l}K=\rho^{-l}K=\tbigcup _{f\in F_{l}}f(K),
\end{equation*}%
where $F_{l}=\left\{x+\sum_{j=1}^{l}\rho ^{-j}c_{j}: c_{j}\in \{(1-\rho)b_{i}\}_{i=1}^{N}\right\}$.

Clearly, $F_{l}$ and $K_{l}$ are increasing sequences, so that
\begin{equation*}
K=K_{0}\subseteq K_{1}\subseteq K_{2}\subseteq \cdots .
\end{equation*}%
The final fractal blow-up of $K$ is in the form $K^{F}$ with $F=\bigcup_{l=1}^{\infty }
F_{l}= \lim_{l\rightarrow\infty} F_l$, and we denote it by
\begin{equation*}
K_{\infty }=\tbigcup _{l=1}^{\infty }K_{l}= \lim_{l\rightarrow\infty} K_l.
\end{equation*}\end{example}
We illustrate the fractal blow-up of the Sierpi\'{n}ski gasket $K$ (the
highlighted area) in Fig.1.
\begin{figure}[tbph]
\flushright
\includegraphics[width=8cm]{fig2.jpg}\newline
\caption{The blow-up of the Sierpi\'{n}ski gasket}
\label{fig1}
\end{figure}

From now on, we always assume that $K$ is a connected homogeneous p.c.f.
self-similar set when we say $K^{F}$ be a fractal glue-up. For simplicity,
we use information on $K$ to obtain that of $K^{F}$. In what follows, we use
the superscript $F$ to distinguish the underlying space $K$ and $K^{F}$ for
the energies and norms etc.

Next, we consider discrete $p$-energy norms of $K^{F}$. Following the
definitions for $K$ in \cite{GaoYuZhang2022PA}, define
\begin{equation}
E_{n}^{(p)}(u):=\sum_{x,y\in V_{w},|w|=n}|u(x)-u(y)|^{p}\text{ \ \ and\ \ \ }%
\mathcal{E}_{n}^{\sigma }(u):=\rho ^{-n(p\sigma -\alpha )}E_{n}^{(p)}(u),
\label{EE}
\end{equation}%
then the local $p$-energy norm on $K^{F}$ is given by
\begin{equation}
E_{n}^{(p),F}(u):=\sum_{f\in F}\sum_{x,y\in V_{w},|w|=n}|u\circ f(x)-u\circ
f(y)|^{p}=\sum_{f\in F}E_{n}^{(p)}(u\circ f),  \label{p_energy}
\end{equation}%
and
\begin{equation*}
\mathcal{E}_{n}^{\sigma ,F}(u)=\rho ^{-n(p\sigma -\alpha )}E_{n}^{(p),F}(u).
\end{equation*}

Define
\begin{equation*}
\mu ^{F}=\sum_{f\in F}\mu \circ f^{-1},
\end{equation*}%
i.e., just `glue' up the measures for each tile. It is easy to see that $\mu
^{F}$ is $\alpha$-regular.

Following the definition of Besov semi-norms on $K$, denote Besov semi-norms
on $K^{F}$ by
\begin{equation*}
\lbrack u]_{B_{p,\infty }^{\sigma ,F}}^{p}:=\sup_{n\geq 0}\Phi _{u}^{\sigma
,F}(\rho ^{n}),
\end{equation*}%
where
\begin{equation*}
\Phi _{u}^{\sigma ,F}(r):=r^{-(p\sigma +\alpha
)}\int_{K^{F}}\int_{B(x,r)}|u(x)-u(y)|^{p}d\mu ^{F}(y)d\mu ^{F}(x),
\end{equation*}%
and define
\begin{eqnarray*}
B_{p,\infty }^{\sigma ,F} &:&=\{u\in L^{p}(K^{F},\mu ^{F}):[u]_{B_{p,\infty
}^{\sigma ,F}}<\infty \}.
\end{eqnarray*}%
It is easy to see that%
\begin{equation}
\sup_{n\geq 0}\Phi _{u}^{\sigma ,F}(\rho ^{n})\asymp \sup_{r\in (0,1)}\Phi
_{u}^{\sigma ,F}(r),  \label{ns}
\end{equation}%
this means%
\begin{equation}
B_{p,\infty }^{\sigma ,F}=B_{p,\infty }^{\sigma }(K^{F})\text{ \ \ and\ \ \ }%
[u]_{B_{p,\infty }^{\sigma ,F}}\asymp \lbrack u]_{B_{p,\infty }^{\sigma
}(K^{F})}\text{ (with }R_{0}=1\text{ in (\ref{pi})).}  \label{BC}
\end{equation}

Also, we have%
\begin{equation}
\liminf_{n\rightarrow \infty }\Phi _{u}^{\sigma ,F}(\rho ^{n})\asymp
\liminf_{r\rightarrow 0}\Phi _{u}^{\sigma ,F}(r)\text{ \ \ and \ \ }%
\limsup_{n\rightarrow \infty }\Phi _{u}^{\sigma ,F}(\rho ^{n})\asymp
\limsup_{r\rightarrow 0}\Phi _{u}^{\sigma ,F}(r).  \label{ne}
\end{equation}

\bigskip We need the following Morrey-Sobolev inequality to show that $%
B_{p,\infty }^{\sigma }(K^F)$ essentially embeds into $C(K^F)$ when $\sigma
>\alpha /p$.

\begin{lemma}
\cite[Theorem 3.2]{BaudoinLecture2022} \label{lemmaMR} Let $(M,d,\mu )$ be a
metric measure space with $\mu $ satisfying (\ref{alpha_r}). When $\sigma
>\alpha /p$, for any $u\in B_{p,\infty }^{\sigma }(M)$, there exists a
continuous version $\tilde{u}\in C^{(p\sigma -\alpha )/p}(M)$ satisfying $%
\tilde{u}=u$ $\mu $-almost everywhere in $M$ and
\begin{equation}
|\tilde{u}(x)-\tilde{u}(y)|\leq C|x-y|^{(p\sigma -\alpha )/p}\sup_{r\in
(0,3d(x,y)]}\Phi _{u}^{\sigma }(r)\leq C|x-y|^{(p\sigma -\alpha
)/p}[u]_{B_{p,\infty }^{\sigma }(M)},  \label{M-S}
\end{equation}%
for all $x,$ $y\in M$ with $d(x,y)<R_{0}/3$, where $C$ is a positive
constant (Here $C^{\beta }(M)$ denotes the class of H\"{o}lder continuous
functions of order $\beta $ on $M$ ).
\end{lemma}

Denote the Borel measure $\mu _{m}$ on $V_{m}$ by
\begin{equation*}
\mu _{m}:=\frac{1}{|V_{m}|}\sum_{a\in V_{m}}\delta _{a},
\end{equation*}%
and define
\begin{equation}
\mu _{m}^{F}:=\sum_{f\in F}\mu_m\circ f^{-1}=\sum_{f\in F}\frac{1}{|V_{m}|}%
\sum_{a\in V_{m}}\delta _{f(a)},  \label{200}
\end{equation}%
where $\delta _{a}$ is the Dirac measure at point $a$ and $|A|$ denotes the
cardinality of a finite set $A$.

Let
\begin{equation}
I_{m,n}^{F}(u):=\int_{K^{F}}\int_{B(x,C_{H}\rho ^{n})}|u(x)-u(y)|^{p}d\mu
_{m}^{F}(y)d\mu _{m}^{F}(x),  \label{201}
\end{equation}%
which can be regarded as the `ball-energy'. As $\mu _{m}$ weak $\ast $%
-converges to $\mu $, we know that $\mu _{m}^{F}$ weak $\ast $-converges to $%
\mu ^{F}$ by the Fubini-Tonelli Theorem. It follows that the weak
convergence of $\mu _{m}^{F}\times \mu _{m}^{F}$ to $\mu ^{F}\times \mu ^{F}$%
. By Lemma \ref{lemmaMR}, we can regard $B_{p,\infty }^{\sigma
,F}(=B_{p,\infty }^{\sigma }(K^{F}))$ as a subspace of $C(K^F)$ when $%
\sigma>\alpha /p$, which implies that the set of discontinuity points of $%
1_{B(x,C_{H}\rho ^{n})}|u(x)-u(y)|^{p} $ has zero $\mu^F\times \mu^F$%
-measure on $K^{F}\times K^{F}$. Therefore, similar to \cite[Remark 1]%
{GuLau.2020.AASFM}, for any $\sigma>\alpha /p$ and all $u\in B_{p,\infty
}^{\sigma ,F}$, we have
\begin{equation}
\lim_{m\rightarrow \infty }I_{m,n}^{F}(u)=\int_{K^{F}}\int_{B(x,C_{H}\rho
^{n})}|u(x)-u(y)|^{p}d\mu ^{F}(y)d\mu ^{F}(x):=I_{\infty ,n}^{F}(u).
\label{mu_m}
\end{equation}%
%
%
%
%
%
%
%
%
%
%
%
%
%
%
%
%
%
%
%
%
%
%
%
%
%
%
%
%
%
%
%
%
%Clearly, \textbf{[[Check, but we don't need it]] }for all sufficient large $%
%k $
%\begin{equation}
%\sup_{n\geq k}\rho ^{-n(p\sigma +\alpha )}I_{\infty ,n}^{F}(u)\asymp
%\sup_{n\geq k}\rho ^{-n(p\sigma +\alpha )}\int_{K^{F}}\int_{B(x,\rho
%^{n})}|u(x)-u(y)|^{p}d\mu ^{F}(y)d\mu ^{F}(x).  \label{II}
%\end{equation}
We will see the estimates of $I_{\infty ,n}^{F}$ in Proposition \ref{prop:I}
and Lemma \ref{lem6.2}.

\begin{proposition}
\label{prop:I}For all $u\in B_{p,\infty }^{\sigma ,F}$, we have
\begin{equation}
\liminf_{n\rightarrow \infty }\rho ^{-n(p\sigma +\alpha )}I_{\infty
,n}^{F}(u)\geq \rho ^{p\sigma +\alpha }C_{H}^{p\sigma +\alpha
}\liminf_{n\rightarrow \infty }\Phi _{u}^{\sigma ,F}(\rho ^{n}).
\label{II-1}
\end{equation}
\end{proposition}

\begin{proof}
Since $C_{H}\in (0,1)$, assume that $\rho ^{s+1}\leq C_{H}<\rho ^{s}$ for
some positive integer $s$. Then, for any integer $n$,
\begin{eqnarray}
\rho ^{-n(p\sigma +\alpha )}I_{\infty ,n}^{F}(u) &=&\rho ^{-n(p\sigma
+\alpha )}\int_{K^{F}}\int_{B(x,C_{H}\rho ^{n})}|u(x)-u(y)|^{p}d\mu
^{F}(y)d\mu ^{F}(x)  \notag \\
&\geq &\rho ^{-n(p\sigma +\alpha )}\int_{K^{F}}\int_{B(x,\rho
^{n+s+1})}|u(x)-u(y)|^{p}d\mu ^{F}(y)d\mu ^{F}(x)  \label{II-2} \\
&=&\rho ^{(s+1)(p\sigma +\alpha )}\rho ^{-(n+s+1)(p\sigma +\alpha
)}\int_{K^{F}}\int_{B(x,\rho ^{n+s+1})}|u(x)-u(y)|^{p}d\mu ^{F}(y)d\mu
^{F}(x)  \notag \\
&>&\rho ^{p\sigma +\alpha }C_{H}^{p\sigma +\alpha }\rho ^{-(n+s+1)(p\sigma
+\alpha )}\int_{K^{F}}\int_{B(x,\rho ^{n+s+1})}|u(x)-u(y)|^{p}d\mu
^{F}(y)d\mu ^{F}(x).  \notag
\end{eqnarray}
Taking $\liminf $ in both side, we obtain (\ref{II-1}).
\end{proof}

The following lemma indicates that choosing different $R_0<\infty$ gives
basically the same norm.

\begin{lemma}
\label{lemma:eq}There exists a constant $C\geq 1$ such that for all $u\in
B_{p,\infty }^{\sigma ,F}$,
\begin{equation}
\int_{K^{F}}\int_{B(x,1)}|u(x)-u(y)|^{p}d\mu ^{F}(y)d\mu ^{F}(x)\leq
C\int_{K^{F}}\int_{B(x,C_{H})}|u(x)-u(y)|^{p}d\mu ^{F}(y)d\mu ^{F}(x).
\label{ct}
\end{equation}
\end{lemma}

\begin{proof}
The proof is based on that of \cite[Corollary 2.2]{Yang.2018.PA} with
additional refinement for unbounded spaces. Since $K^{F}$ is connected, we
know that $K^{F}$ satisfies the chain condition (see \cite[Definition 3.4]%
{GrigoryanHuLau.2003.TAMS2065} for definition). Let $x,y\in K^{F}$ with $|x-y|<1$. By the
chain condition, there exists a constant $C_{1}>0$ (independent of $x,y$)
such that for all integer $n\geq 1$, there exists a sequence $%
\{z_{i}\}_{i=1}^{n}$ with $z_{0}=x,z_{n}=y$ and
\begin{equation*}
|z_{i}-z_{i+1}|\leq C_{1}\frac{|x-y|}{n}\text{ \ for all }i=0,1,...,n-1\text{.%
}
\end{equation*}

Taking an integer $n\geq \frac{3C_{1}}{C_{H}}+1$, we know that for all $%
i=0,1,...,n-1$,
\begin{equation}
|z_{i}-z_{i+1}|\leq C_{1}\frac{|x-y|}{n}<\frac{C_{1}}{n}<\frac{C_{H}}{3}%
\text{.}  \label{zz-1}
\end{equation}
For all $i=0,1,...,n-1$ and for all $x_{i}\in B(z_{i},C_{H}/3),x_{i+1}\in
B(z_{i+1},C_{H}/3)$, we have%
\begin{equation}
|x_{i}-x_{i+1}|\leq |x_{i}-z_{i}|+|z_{i}-z_{i+1}|+|x_{i+1}-z_{i+1}|<C_{H}%
\text{.}  \label{zz}
\end{equation}
Fix $x_{0}=x$ and $x_{n}=y$. By Jensen's inequality ($p>1$),
\begin{equation*}
|u(x)-u(y)|^{p}\leq \left( \sum_{i=0}^{n-1}|u(x_{i})-u(x_{i+1})|\right)
^{p}\leq n^{p-1}\sum_{i=0}^{n-1}|u(x_{i})-u(x_{i+1})|^{p}.
\end{equation*}

Integrating the above inequality with respect to $x_{1}\in B(z_{1},C_{H}/3)$%
, ..., $x_{n-1}\in B(z_{n-1},C_{H}/3)$ and dividing by $\mu
^{F}(B(z_{1},C_{H}/3))$, ..., $\mu ^{F}(B(z_{n-1},C_{H}/3))$, we have%
\begin{eqnarray*}
|u(x)-u(y)|^{p} &\leq &n^{p-1}\Big(\frac{1}{\mu ^{F}(B(z_{1},C_{H}/3))}%
\int_{B(z_{1},C_{H}/3)}|u(x_{0})-u(x_{1})|^{p}d\mu ^{F}(x_{1}) \\
&&+\frac{1}{\mu ^{F}(B(z_{n-1},C_{H}/3))}%
\int_{B(z_{n-1},C_{H}/3)}|u(x_{n-1})-u(x_{n})|^{p}d\mu ^{F}(x_{n-1}) \\
&&+\sum_{i=1}^{n-2}\frac{1}{\mu ^{F}(B(z_{i},C_{H}/3))\mu
^{F}(B(z_{i+1},C_{H}/3))} \\
&&\int_{B(z_{i},C_{H}/3)}\int_{B(z_{i+1},C_{H}/3)}|u(x_{i})-u(x_{i+1})|^{p}d%
\mu ^{F}(x_{i})d\mu ^{F}(x_{i+1})\Big).
\end{eqnarray*}

Since $\mu ^{F}$ is $\alpha $-regular, $\mu ^{F}(B(z_{i},C_{H}/3))\asymp
C_{H}^{\alpha }$ for all $i=0,1,...,n-1$, we have by (\ref{zz}) that
\begin{eqnarray}
|u(x)-u(y)|^{p} &\leq &C_{2}n^{p-1}\Big(%
\int_{B(x,C_{H})}|u(x)-u(x_{1})|^{p}d\mu
^{F}(x_{1})+\int_{B(y,C_{H})}|u(x_{n-1})-u(y)|^{p}d\mu ^{F}(x_{n-1})  \notag
\\
&&+\sum_{i=1}^{n-2}\int_{B(z_{i},C_{H}/3)}\left(
\int_{B(x_{i},C_{H})}|u(x_{i})-u(x_{i+1})|^{p}d\mu ^{F}(x_{i+1})\right) d\mu
^{F}(x_{i})\Big).  \label{I}
\end{eqnarray}

Now we use \eqref{I} to estimate the left-hand side of \eqref{ct}. For the
first term in the summation of the right-hand side, note that
\begin{eqnarray}
&&\int_{K^{F}}\int_{B(x,1)}\left( \int_{B(x,C_{H})}|u(x)-u(x_{1})|^{p}d\mu
^{F}(x_{1})\right) d\mu ^{F}(y)d\mu ^{F}(x)  \notag \\
&=&\int_{K^{F}}\mu ^{F}(B(x,1))\left(
\int_{B(x,C_{H})}|u(x)-u(x_{1})|^{p}d\mu ^{F}(x_{1})\right) d\mu ^{F}(x)
\notag \\
&\leq &C\int_{K^{F}}\int_{B(x,C_{H})}|u(x)-u(x_{1})|^{p}d\mu ^{F}(x_{1})d\mu
^{F}(x)  \notag \\
&=&C\int_{K^{F}}\int_{B(x,C_{H})}|u(x)-u(y)|^{p}d\mu ^{F}(y)d\mu ^{F}(x).
\label{i-1}
\end{eqnarray}

For the second term,
\begin{eqnarray}
&&\int_{K^{F}}\int_{B(x,1)}\left( \int_{B(y,C_{H})}|u(x_{n-1})-u(y)|^{p}d\mu
^{F}(x_{n-1})\right) d\mu ^{F}(y)d\mu ^{F}(x)  \notag \\
&=&\iint\nolimits_{\{(x,y)\in K^{F}\times K^{F}:|x-y|<1\}}\left(
\int_{B(y,C_{H})}|u(x_{n-1})-u(y)|^{p}d\mu ^{F}(x_{n-1})\right) d\mu
^{F}(y)d\mu ^{F}(x)  \notag \\
&=&\int_{K^{F}}\int_{B(y,1)}\left(
\int_{B(y,C_{H})}|u(x_{n-1})-u(y)|^{p}d\mu ^{F}(x_{n-1})\right) d\mu
^{F}(x)d\mu ^{F}(y)  \notag \\
&=&\int_{K^{F}}\mu ^{F}(B(y,1))\left(
\int_{B(y,C_{H})}|u(x_{n-1})-u(y)|^{p}d\mu ^{F}(x_{n-1})\right) d\mu ^{F}(y)
\notag \\
&\leq &C\int_{K^{F}}\left( \int_{B(y,C_{H})}|u(x_{n-1})-u(y)|^{p}d\mu
^{F}(x_{n-1})\right) d\mu ^{F}(y)  \notag \\
&=&C\int_{K^{F}}\int_{B(x,C_{H})}|u(x)-u(y)|^{p}d\mu ^{F}(y)d\mu ^{F}(x).
\label{i-2}
\end{eqnarray}

For the last term, note that by (\ref{zz-1}), for all $i=1,2,...,n-2$,
\begin{equation*}
|x-z_{i}|\leq \sum_{j=0}^{i-1}|z_{j}-z_{j+1}|<i\frac{C_{H}}{3}\leq \frac{%
(n-2)C_{H}}{3},
\end{equation*}%
thus $B(z_{i},C_{H}/3)\subset B(x,nC_{H}/3)$ and then
\begin{eqnarray*}
&& \int_{B(z_{i},C_{H}/3)}\left(
\int_{B(x_{i},C_{H})}|u(x_{i})-u(x_{i+1})|^{p}d\mu ^{F}(x_{i+1})\right) d\mu
^{F}(x_{i})  \notag \\
&\leq&\int_{B(x,nC_{H}/3)}\left(
\int_{B(x_{i},C_{H})}|u(x_{i})-u(x_{i+1})|^{p}d\mu ^{F}(x_{i+1})\right) d\mu
^{F}(x_{i})  \notag \\
& =&\int_{B(x,nC_{H}/3)}\left( \int_{B(a,C_{H})}|u(a)-u(b)|^{p}d\mu
^{F}(b)\right) d\mu ^{F}(a).
\end{eqnarray*}%
It follows that for all $i=1,2,...,n-2$,
\begin{eqnarray}
&&\int_{K^{F}}\int_{B(x,1)}\left( \int_{B(z_{i},C_{H}/3)}\left(
\int_{B(x_{i},C_{H})}|u(x_{i})-u(x_{i+1})|^{p}d\mu ^{F}(x_{i+1})\right) d\mu
^{F}(x_{i})\right) d\mu ^{F}(y)d\mu ^{F}(x)  \notag \\
&\leq &\int_{K^{F}}\int_{B(x,1)}\left( \int_{B(x,nC_{H}/3)}\left(
\int_{B(a,C_{H})}|u(a)-u(b)|^{p}d\mu ^{F}(b)\right) d\mu ^{F}(a)\right) d\mu
^{F}(y)d\mu ^{F}(x)  \notag \\
&=&\int_{K^{F}}\mu ^{F}(B(x,1))\left( \int_{B(x,nC_{H}/3)}\left(
\int_{B(a,C_{H})}|u(a)-u(b)|^{p}d\mu ^{F}(b)\right) d\mu ^{F}(a)\right) d\mu
^{F}(x)  \notag \\
&\leq &C\int_{K^{F}}\left( \int_{B(x,nC_{H}/3)}\left(
\int_{B(a,C_{H})}|u(a)-u(b)|^{p}d\mu ^{F}(b)\right) d\mu ^{F}(a)\right) d\mu
^{F}(x)  \notag \\
&=&C\iint\nolimits_{\{(x,a)\in K^{F}\times K^{F}:|x-a|<nC_{H}/3\}}\left(
\int_{B(a,C_{H})}|u(a)-u(b)|^{p}d\mu ^{F}(b)\right) d\mu ^{F}(x)d\mu ^{F}(a)
\notag \\
&=&C\int_{K^{F}}\left( \int_{B(a,nC_{H}/3)}\left(
\int_{B(a,C_{H})}|u(a)-u(b)|^{p}d\mu ^{F}(b)\right) d\mu ^{F}(x)\right) d\mu
^{F}(a)  \notag \\
&=&C\int_{K^{F}}\mu ^{F}(B(a,nC_{H}/3))\left(
\int_{B(a,C_{H})}|u(a)-u(b)|^{p}d\mu ^{F}(b)\right) d\mu ^{F}(a)  \notag \\
&\leq &C_{3}\int_{K^{F}}\left( \int_{B(a,C_{H})}|u(a)-u(b)|^{p}d\mu
^{F}(b)\right) d\mu ^{F}(a)  \notag \\
&=&C_{3}\int_{K^{F}}\int_{B(x,C_{H})}|u(x)-u(y)|^{p}d\mu ^{F}(y)d\mu ^{F}(x).
\label{i-4}
\end{eqnarray}

Integrating (\ref{I}) with respect to $x\in K^{F},y\in B(x,1)$, we obtain (%
\ref{ct}) by (\ref{i-1}), (\ref{i-2}) and (\ref{i-4}).
\end{proof}

Indeed, we know by (\ref{ct}) that for all $u\in B_{p,\infty }^{\sigma ,F}$,
\begin{equation}
\lbrack u]_{B_{p,\infty }^{\sigma ,F}}^{p}=\sup_{n\geq 0}\Phi _{u}^{\sigma
,F}(\rho ^{n})\asymp \rho ^{-n(p\sigma +\alpha )}I_{\infty ,n}^{F}(u).
\label{ct-1}
\end{equation}

\begin{lemma}
\label{lem6.2} Let $K^{F}$ be a fractal glue-up. If $\sigma >\alpha /p$,
then
\begin{equation}
I_{\infty ,n}^{F}(u)\leq C\rho ^{2n\alpha }\sum_{k=n}^{\infty
}E_{k}^{(p),F}(u)\leq C^{\prime }\rho ^{n(p\sigma +\alpha )}\sup_{k\geq n}%
\mathcal{E}_{k}^{\sigma ,F}(u).  \label{eq6.3}
\end{equation}
\end{lemma}

\begin{proof}
Let $I_{m,n}^{F}(u)$ be defined as in (\ref{201}) ($m>n)$. Since $K^{F}$ is
connected and satisfies the condition $(\mathrm{H})$, $|x-y|\leq C_{H}\rho
^{n}$ implies that\ $x,y$ lie in the same or the neighboring $n$-cells. For
every $w\in W_{n}$ and $x\in f(K_{w})$, if $y\in B(x,C_{H}\rho ^{n})$, then
there exists $(g,\tilde{w})\in F\times W_{n}$ such that $y\in g(K_{\tilde{w}%
})$ and
\begin{equation*}
g(K_{\tilde{w}})\cap f(K_{w})\neq \emptyset .
\end{equation*}%
Therefore,
\begin{align*}
I_{m,n}^{F}(u)\leq & \sum_{|w|=n}\sum_{f\in F}\int_{f(K_{w})}\left( \sum_{|%
\tilde{w}|=n}\sum_{g\in F}\int_{g(K_{\tilde{w}})}|u(x)-u(y)|^{p}1_{\{g(K_{%
\tilde{w}})\cap f(K_{w})\neq \emptyset \}}d\mu _{m}^{F}(y)\right) d\mu
_{m}^{F}(x) \\
=& \sum_{|w|=n}\sum_{f\in F}\sum_{|\tilde{w}|=n}\sum_{g\in F}\sum_{x\in
f(K_{w}\cap V_{m})}\sum_{y\in g(K_{\tilde{w}}\cap V_{m})}\frac{1_{\{g(K_{%
\tilde{w}})\cap f(K_{w})\neq \emptyset \}}}{|V_{m}|^{2}}|u(x)-u(y)|^{p}.
\end{align*}%
When $g(K_{\tilde{w}})\cap f(K_{w})\neq \emptyset $, we can find a common
vertex of these two cells and denote it by $z\in f(V_{w})\cap g(V_{\tilde{w}%
})$, due to the just-touching property when $f\neq g$ and the p.c.f.
structure when $f=g$. Furthermore, using the elementary inequality $%
|u(x)-u(y)|^{p}\leq 2^{p-1}\left( |u(x)-u(z)|^{p}+|u(z)-u(y)|^{p}\right) $,
we obtain%
\begin{equation*}
1_{\{g(K_{\tilde{w}})\cap f(K_{w})\neq \emptyset \}}|u(x)-u(y)|^{p}\leq
2^{p-1}\sum_{z\in f(V_{w})\cap g(V_{\tilde{w}})}\left(
|u(x)-u(z)|^{p}+|u(z)-u(y)|^{p}\right) .
\end{equation*}%
It follows that
\begin{align}
I_{m,n}^{F}(u)\leq & 2^{p-1}\sum_{|w|=n}\sum_{f\in F}\sum_{|\tilde{w}%
|=n}\sum_{g\in F}\sum_{x\in f(K_{w}\cap V_{m})}\sum_{y\in g(K_{\tilde{w}%
}\cap V_{m})}\sum_{z\in f(V_{w})\cap g(V_{\tilde{w}})}\frac{1}{|V_{m}|^{2}}%
\left( |u(x)-u(z)|^{p}+|u(z)-u(y)|^{p}\right)  \notag \\
\leq & C_{1}N^{-2m}\sum_{|w|=n}\sum_{f\in F}\sum_{|\tilde{w}|=n}\sum_{g\in
F}\sum_{z\in f(V_{w})\cap g(V_{\tilde{w}})}\sum_{x\in f(K_{w}\cap
V_{m})}\sum_{y\in g(K_{\tilde{w}}\cap V_{m})}\left(
|u(x)-u(z)|^{p}+|u(z)-u(y)|^{p}\right)  \notag \\
\leq & C_{1}N^{-2m}\sum_{|w|=n}\sum_{f\in F}\sum_{z\in f(V_{w})}\sum_{x\in
f(K_{w}\cap V_{m})}|u(x)-u(z)|^{p}\left( \sum_{|\tilde{w}|=n}\sum_{g\in
F}1_{\{z\in f(V_{w})\cap g(V_{\tilde{w}})\}}|g(K_{\tilde{w}}\cap
V_{m})|\right)  \notag \\
& +C_{1}N^{-2m}\sum_{|\tilde{w}|=n}\sum_{g\in F}\sum_{z\in g(V_{\tilde{w}%
})}\sum_{y\in g(K_{\tilde{w}}\cap V_{m})}|u(z)-u(y)|^{p}\left(
\sum_{|w|=n}\sum_{f\in F}1_{\{z\in f(V_{w})\cap g(V_{\tilde{w}%
})\}}|f(K_{w}\cap V_{m})|\right)  \notag \\
\leq & C_{2}N^{-(m+n)}\left(\sum_{|w|=n}\sum_{f\in F}\sum_{z\in
f(V_{w})}\sum_{x\in f(K_{w}\cap V_{m})}|u(x)-u(z)|^{p} +\sum_{|\tilde{w}%
|=n}\sum_{g\in F}\sum_{z\in g(V_{\tilde{w}})}\sum_{y\in g(K_{\tilde{w}}\cap
V_{m})}|u(z)-u(y)|^{p} \right)  \notag \\
=& 2C_{2}N^{-(m+n)}\sum_{|w|=n}\sum_{f\in F}\sum_{x\in f(K_{w}\cap
V_{m})}\sum_{z\in f(V_{w})}|u(x)-u(z)|^{p},  \label{eq5.6}
\end{align}%
where we use the `uniform finitely-joint' property for each common vertex $%
z\in f(V_{w})\cap g(V_{\tilde{w}})$ in the last inequality, so that
\begin{equation*}
\sum_{|\tilde{w}|=n}\sum_{g\in F}1_{\{z\in f(V_{w})\cap g(V_{\tilde{w}})\}}
\end{equation*}%
is uniformly bounded for any $(f,w)\in F\times W_{n}$, and the elementary
estimates
\begin{eqnarray}
|V_{m}| &\asymp &N^{m}=\rho ^{-\alpha m},  \label{v1} \\
\text{\ }|K_{w}\cap V_{m}| &\asymp &N^{m-n}\text{ for every }w\in W_{n}.
\label{v2}
\end{eqnarray}%
Since $x,z$ belong to the same tile $f(K)$, we apply the same estimate for $%
|u(x)-u(z)|^{p}$ as \cite[(2.6)]{GaoYuZhang2022PA}, and for reader's
convenience we present it here. For such $x\in f(V_{m}),z\in f(V_{n})$, one
can naturally fix a decreasing sequence of cells $K_{w_{k}}$ with $|w_{k}|=k$
for $k=n,\cdots ,m$ and vertices $x_{k}(x,z)\in V_{w_{k}}$ such that, $%
z=x_{n}(x,z),\ x=x_{m}(x,z)$, so by H\"{o}lder's inequality we have
\begin{align}
|u(z)-u(x)|^{p}\leq & \left( \sum_{k=n}^{m-1}N^{(n-k)q/p}\right)
^{p/q}\left(
\sum_{k=n}^{m-1}N^{k-n}|u(x_{k}(x,z))-u(x_{k+1}(x,z))|^{p}\right)  \notag \\
\leq & C_{3}\sum_{k=n}^{m-1}N^{k-n}|u(x_{k}(x,z))-u(x_{k+1}(x,z))|^{p},
\label{u_xz}
\end{align}%
where $q=p/(p-1)$. Note that for every pair $(a,b)\in f(K_{w^{\prime }}\cap
V_{k+1})\times f(K_{w^{\prime }}\cap V_{k+1})$, the cardinality of $(x,z)\in
f(K_{w}\cap V_{m})\times f(V_{w})$ with $(x_{k}(x,z),x_{k+1}(x,z))=(a,b)$ is
no greater than $C^{\prime }N^{m-k}$ for some $C^{\prime }>0$, due to the
p.c.f. structure. Substituting \eqref{u_xz} into \eqref{eq5.6}, we obtain
\begin{align}
I_{m,n}^{F}(u)& \leq C_{4}N^{-(m+n)}\sum_{|w|=n}\sum_{f\in F}\sum_{(x,z)\in
f(K_{w}\cap V_{m})\times
f(V_{w})}\sum_{k=n}^{m-1}N^{k-n}|u(x_{k}(x,z))-u(x_{k+1}(x,z))|^{p}  \notag
\\
& =C_{4}N^{-(m+n)}\sum_{|w|=n}\sum_{k=n}^{m-1}\sum_{f\in F}\sum_{\QATOP{%
|w^{\prime }|=k}{f(K_{w^{\prime }})\subset f(K_{w})}}\sum_{\QATOP{(x,z)\in
f(K_{w^{\prime }}\cap V_{m})\times f(V_{w^{\prime }}),}{%
(x_{k}(x,z),x_{k+1}(x,z))=(a,b)}}N^{k-n}|u(a)-u(b)|^{p}  \notag \\
& \leq C_{5}N^{-(m+n)}\sum_{|w|=n}\sum_{k=n}^{m-1}\sum_{f\in F}\sum_{\QATOP{%
|w^{\prime }|=k}{f(K_{w^{\prime }})\subset f(K_{w})}}\sum_{a,b\in
f(K_{w^{\prime }}\cap V_{k+1})}N^{k-n}\cdot N^{m-k}|u(a)-u(b)|^{p}  \notag \\
& =C_{5}\rho ^{2n\alpha }\sum_{k=n}^{m-1}\sum_{|w|=n}\sum_{f\in F}\sum_{%
\QATOP{|w^{\prime }|=k}{f(K_{w^{\prime }})\subset f(K_{w})}}\sum_{a,b\in
f(K_{w^{\prime }}\cap V_{k+1})}|u(a)-u(b)|^{p}  \notag \\
& \leq C_{6}\rho ^{2n\alpha }\sum_{k=n}^{m-1}\sum_{f\in
F}E_{k+1}^{(p)}(u\circ f)\leq C_{6}\rho ^{2n\alpha
}\sum_{k=n}^{m}E_{k}^{(p),F}(u),  \label{I_mn}
\end{align}%
where we use \eqref{p_energy} in the last line. It follows that
\begin{equation*}
I_{m,n}^{F}(u)\leq C_{6}\rho ^{2n\alpha }\rho ^{n(p\sigma -\alpha
)}\sum_{k=n}^{m}\rho ^{-n(p\sigma -\alpha )}E_{k}^{(p),F}(u)\leq C_{7}\rho
^{n(p\sigma +\alpha )}\sup_{k\geq n}\mathcal{E}_{k}^{\sigma ,F}(u),
\end{equation*}%
which ends the proof by the weak $\ast $-convergence in \eqref{mu_m} as $%
m\rightarrow \infty $.
\end{proof}

Denote the `ring-energy' by
\begin{equation*}
I_{n}^{F}(u):=\int_{K^{F}}\int_{\{\rho ^{n+1}\leq d(x,y)<\rho
^{n}\}}|u(x)-u(y)|^{p}d\mu ^{F}(y)d\mu ^{F}(x),
\end{equation*}%
which will be used later instead of the previous `ball-energy'.

\begin{lemma}
\label{lem6.3} Let $K^{F}$ be a fractal glue-up. If $\sigma >\alpha /p$,
then for any $\delta \in (0,p\sigma -\alpha) $ and $u\in B_{p,\infty
}^{\sigma ,F} $,
\begin{equation}
\mathcal{E}_{n}^{\sigma ,F}(u)\leq C\sum_{k=0}^{\infty }\rho ^{(-2\alpha
-\delta )k}\rho ^{-(p\sigma +\alpha )n}I_{k+n}^{F}\leq C^{\prime
}\sup_{k\geq 0}\Phi _{u}^{\sigma }(\rho ^{n+k}).  \label{eq6.4}
\end{equation}
\end{lemma}

\begin{proof}
For $a,b\in f(K_{w})$, we have $|u(a)-u(b)|^{p}\leq
2^{p-1}(|u(a)-u(x)|^{p}+|u(x)-u(b)|^{p})$, where $x\in f(K_{w})$ and $f\in F$%
. Approximate $F$ with increasing finite sets $F_{j}$ so that $%
F=\lim_{j\rightarrow \infty }F_{j}$. Integrating with respect to $x$ and
dividing by $\mu ^{F}(f(K_{w}))$, we have
\begin{align}
E_{n}^{(p),F_{j}}(u)& =\sum_{f\in F_{j}}\sum_{a,b\in
f(V_{w}),|w|=n}|u(a)-u(b)|^{p}  \notag \\
& \leq 2^{p-1}\sum_{f\in F_{j}}\sum_{a,b\in f(V_{w}),a\neq b,|w|=n}\left(
\frac{1}{\mu ^{F}(f(K_{w}))}\int_{f(K_{w})}|u(a)-u(x)|^{p}+|u(x)-u(b)|^{p}d%
\mu ^{F}(x)\right)  \notag \\
& \leq 2^{p-1}|V_{0}|\sum_{f\in F_{j}}\sum_{a\in f(V_{w}),|w|=n}\frac{1}{\mu
^{F}(f(K_{w}))}\int_{f(K_{w})}|u(a)-u(x)|^{p}d\mu ^{F}(x).  \label{lem3.0}
\end{align}%
For every $a\in f(V_{w})$ with $|w|=n$, we can fix a decreasing sequence of
cells $\{f(K_{w_{k}})\}_{k=n}^{m}$ where $|w_{k}|=k$ for all large enough $%
m>n$, such that $a\in \cap _{k=n}^{m}f(K_{w_{k}})$ with $w_{n}=w$. We choose
$x_{k}\in f(K_{w_{k}})$ for $k=n,n+1,...m$. Let $\delta \in (0,p\sigma
-\alpha) $, by H\"{o}lder's inequality,
\begin{align}
& |u(a)-u(x_{n})|^{p}  \notag \\
\leq & 2^{p-1}|u(a)-u(x_{m})|^{p}+2^{p-1}\left( \sum_{k=n}^{m-1}\rho
^{\delta (k-n)q/p}\right) ^{p/q}\left( \sum_{k=n}^{m-1}\rho ^{\delta
(n-k)}|u(x_{k})-u(x_{k+1})|^{p}\right) ,  \label{203}
\end{align}%
where $q=p/(p-1)$. Integrating \eqref{203} with respect to $x_{k}\in
K_{w_{k}}$ and dividing by $\mu ^{F}(f(K_{w_{k}}))$,
\begin{align*}
E_{n}^{(p),F_{j}}(u)\leq & C_{1}\sum_{f\in F_{j}}\sum_{a\in f(V_{w}),|w|=n}%
\frac{2^{p-1}}{\mu ^{F}(f(K_{w_{m}}))}%
\int_{f(K_{w_{m}})}|u(a)-u(x_{m})|^{p}d\mu ^{F}(x_{m}) \\
& +C_{2}\sum_{f\in F_{j}}\sum_{k=n}^{m-1}\frac{\rho ^{\delta (n-k)}}{\mu
^{F}(f(K_{w_{k}}))\mu ^{F}(f(K_{w_{k+1}}))}\int_{f(K_{w_{k}})}%
\int_{f(K_{w_{k+1}})}|u(x_{k})-u(x_{k+1})|^{p}d\mu ^{F}(x_{k+1})d\mu
^{F}(x_{k}).
\end{align*}%
Using Lemma \ref{lemmaMR}, (\ref{BC}) and (\ref{v1}), the first term of the
right-hand side above vanishes as $m\rightarrow \infty $ since
\begin{align*}
&\sum_{a\in f(V_{w}),|w|=n}\frac{2^{p-1}}{\mu ^{F}(f(K_{w_{m}}))}%
\int_{f(K_{w_{m}})}|u(a)-u(x_{m})|^{p}d\mu ^{F}(x_{m}) \\
\leq& C|f(V_{n})|2^{p-1}\rho ^{(p\sigma -\alpha )m}[u]_{B_{p,\infty
}^{\sigma }(K^{F})}^{p} \leq C^{\prime }|f(V_{n})|2^{p-1}\rho ^{(p\sigma
-\alpha )m}[u]_{B_{p,\infty }^{\sigma ,F}}^{p} \\
\leq &C_{3}\rho ^{(p\sigma -\alpha )m-\alpha n}[u]_{B_{p,\infty }^{\sigma
,F}}^{p}\rightarrow 0.
\end{align*}%
Letting $m\rightarrow \infty $, we obtain
\begin{align}
E_{n}^{(p),F_{j}}(u)& \leq C_{4}\sum_{k=n}^{\infty }\rho ^{\delta (n-k)}\rho
^{-2\alpha k}\int_{K^{F}}\int_{B(x,\rho ^{k})}|u(x)-u(y)|^{p}d\mu
^{F}(y)d\mu ^{F}(x)  \notag \\
& =C_{4}\sum_{k=n}^{\infty }\rho ^{\delta (n-k)}\rho ^{-2\alpha
k}\sum_{l=k}^{\infty }I_{l}^{F}(u)=C_{4}\sum_{k=0}^{\infty }\rho ^{-\delta
k}\rho ^{-2\alpha (n+k)}\sum_{l=k}^{\infty }I_{l+n}^{F}(u)  \notag \\
& =C_{4}\sum_{l=0}^{\infty }\left( \sum_{k=0}^{l}\rho ^{-\delta k}\rho
^{-2\alpha (n+k)}\right) I_{l+n}^{F}(u)=C_{4}\sum_{l=0}^{\infty }\left(
\sum_{k=0}^{l}\rho ^{\delta (l-k)}\rho ^{2\alpha (l-k)}\right) \rho
^{-\delta l}\rho ^{-2\alpha (l+n)}I_{l+n}^{F}(u)  \notag \\
& \leq C_{5}\sum_{l=0}^{\infty }\rho ^{-\delta l}\rho ^{-2\alpha
(l+n)}I_{l+n}^{F}(u)=C_{5}\sum_{k=0}^{\infty }\rho ^{-\delta k}\rho
^{-2\alpha (k+n)}I_{k+n}^{F}(u).  \label{I_2}
\end{align}%
Letting $j\rightarrow \infty $, we have%
\begin{eqnarray*}
\mathcal{E}_{n}^{\sigma ,F}(u) &=&\rho ^{-(p\sigma -\alpha
)n}E_{n}^{(p),F}(u)\leq C_{5}\sum_{k=0}^{\infty }\rho ^{(-2\alpha -\delta
)k}\rho ^{-(p\sigma +\alpha )n}I_{k+n}^{F}(u) \\
&\leq &C_{5}\sum_{k=0}^{\infty }\rho ^{(-2\alpha -\delta )k}\rho ^{-(p\sigma
+\alpha )n}\int_{K^{F}}\int_{B(x,\rho ^{n+k})}|u(x)-u(y)|^{p}d\mu
^{F}(y)d\mu ^{F}(x) \\
&\leq &C_{5}\sum_{k=0}^{\infty }\rho ^{(p\sigma +\alpha -2\alpha -\delta
)k}\sup_{k\geq 0}\Phi _{u}^{\sigma }(\rho ^{n+k}) \\
&=&C_{5}\left( \sum_{k=0}^{\infty }\rho ^{(p\sigma -\alpha -\delta
)k}\right) \sup_{k\geq 0}\Phi _{u}^{\sigma }(\rho ^{n+k}) =\frac{C_{5}}{%
1-\rho ^{p\sigma -\alpha -\delta }}\sup_{k\geq 0}\Phi _{u}^{\sigma }(\rho
^{n+k}),
\end{eqnarray*}%
where the second inequality follows by enlarging each ring-energy $%
I_{k+n}^{F}$ to ball-energy.
\end{proof}

In \cite[Theorem 1.4]{GaoYuZhang2022PA}, for a homogeneous p.c.f.
self-similar set $K$, we show the $p$-energy norm equivalence
\begin{equation}
\sup\limits_{n\geq 0}\rho ^{-n(p\sigma -\alpha )}E_{n}^{(p)}(u)\asymp
\lbrack u]_{B_{p,\infty }^{\sigma }(K)}^{p}\text{ \ (}p\sigma >\alpha \text{)%
},  \label{204}
\end{equation}%
where $E_{n}^{(p)}(u)$ is from (\ref{EE}).

Using Lemma \ref{lem6.2} and Lemma \ref{lem6.3}, we immediately derive such
equivalence for $K^{F}$. Define local $p$-energy norm for $K^{F}$ by
\begin{equation*}
\mathcal{E}_{p,\infty }^{\sigma ,F}(u):=\sup\limits_{n\geq 0}\rho
^{-n(p\sigma -\alpha )}E_{n}^{(p),F}(u)=\sup_{n\geq 0}\mathcal{E}%
_{n}^{\sigma ,F}(u).
\end{equation*}

\begin{corollary}
\label{lem6.1} Let $K^{F}$ be a fractal glue-up. If $\sigma>\alpha /p$, then
for all $u\in B_{p,\infty }^{\sigma ,F}$,
\begin{equation}
\mathcal{E}_{p,\infty }^{\sigma ,F}(u)\asymp \lbrack u]_{B_{p,\infty
}^{\sigma ,F}}^{p},\ \limsup_{n\rightarrow \infty }\mathcal{E}_{n}^{\sigma
,F}(u)\asymp \limsup_{n\rightarrow \infty }\Phi _{u}^{\sigma ,F}(\rho ^{n}).
\label{eq6.2}
\end{equation}
\end{corollary}

\begin{proof}
Taking limsup and sup (of $n$) on both sides of \eqref{eq6.3} and %
\eqref{eq6.4} gives the desired by noting that (\ref{ct-1}).
\end{proof}

\subsection{Property (E) holds for nested fractals}

\label{subsec5.2} Let us introduce the discrete (vertex-type)
weak-monotonicity property we concern. For a set $V$, denote $\ell (V)=\{u:$
$u$ maps $V$ into $\mathbb{R}\}$.

\begin{definition}
(\cite[Definition 3.1]{GaoYuZhang2022PA}) \label{dfE} We say that a
connected homogeneous p.c.f. self-similar set $K$ satisfies \textrm{property
(E)}, if there exist $\sigma>\alpha/p$ and a positive constant $C$ such that

\begin{itemize}
\item[(i)~] for any $u\in B_{p,\infty }^{\sigma}$ and for all $n\geq 0$, $%
\mathcal{E}_{0}^{\sigma}(u)\leq C\mathcal{E}_{n}^{\sigma}(u)$,

\item[(ii)] for any $u\in \ell (V_{0})$, there exists an extension $\tilde{u}%
\in B_{p,\infty }^{\sigma}$.
\end{itemize}
\end{definition}

In \cite[Remark 3.2]{GaoYuZhang2022PA} we know that under property (E), $%
\sigma=\sigma _{p}^{*}(K)$, and for all $n\geq m$, $\mathcal{E}_{m}^{\sigma
_{p}^{*}}(u)\leq C\mathcal{E}_{n}^{\sigma _{p}^{*}}(u)$. Therefore, if a
connected homogeneous p.c.f. self-similar set $K$ satisfies property (E),
then property (VE) (defined as follows) holds for $K^F$.

\begin{definition}
\label{ve} We say that a fractal glue-up $K^{F}$ satisfies \textrm{property
(VE) with} $\sigma >0$, if there exists $C>0$ such that for all $u\in
L^{p}(K^{F},\mu ^{F})$,
\begin{equation*}
\sup_{n\geq 0}\mathcal{E}_{n}^{\sigma ,F}(u)\leq C\liminf_{n\rightarrow
\infty }\mathcal{E}_{n}^{\sigma ,F}(u).
\end{equation*}%
We say that a fractal glue-up $K^{F}$ satisfies \textrm{property $(%
\widetilde{VE})$ with} $\sigma >0$, if there exists $C>0$ such that for all $%
u\in L^{p}(K^{F},\mu ^{F})$,
\begin{equation*}
\limsup_{n\rightarrow \infty }\mathcal{E}_{n}^{\sigma ,F}(u)\leq
C\liminf_{n\rightarrow \infty }\mathcal{E}_{n}^{\sigma ,F}(u).
\end{equation*}
\end{definition}

The following proposition shows the importance of property (E).

\begin{proposition}
\label{prop:critical} If a connected homogeneous p.c.f. self-similar set $K$
satisfies property (E), then for any set $F$,
\begin{equation}
\sigma _{p}^{\ast }(K)=\sigma _{p}^{\#}(K)=\sigma _{p}^{\#}(K^{F}).
\label{qnmd}
\end{equation}%
Moreover, $K^{F}$ satisfies property (VE) with $\sigma _{p}^{\#}(K)$, and $%
B_{p,\infty }^{\sigma _{p}^{\#}(K),F}$ contains non-constant functions.
\end{proposition}

\begin{proof}

The fact $\sigma _{p}^{\ast }(K)=\sigma _{p}^{\#}(K)$ follows from \cite[the
proof of Proposition 3.4]{GaoYuZhang2022PA}. It is also clear that $\sigma
_{p}^{\#}(K)\geq \sigma _{p}^{\#}(K^{F})$, since any non-constant function $%
u\in l(K^{F})$ in $B_{p,\infty }^{\sigma ,F}$ naturally gives a non-constant
function in $B_{p,\infty }^{\sigma }(K)$ for any $\sigma >0$: just restrict $%
u$ to a tile $f(K)$ where it is non-constant (for some $f\in F$).

For the reverse inclusion $\sigma _{p}^{\#}(K)\leq \sigma _{p}^{\#}(K^{F})$
to hold, we just need to use property (E)(ii) to find a non-constant
function in $B_{p,\infty }^{\sigma _{p}^{\#}(K),F}$. To see this, we pick a
contraction $f\in F$, and determine the values of $u$ on $f(V_{1})$ by $%
u(f(V_{0}))=0$ while $u(f(V_{1}\setminus V_{0}))=1$. We can then extend $u$
from $f(V_{1})$ to $f(K)$ and make sure that $u\circ f\in B_{p,\infty
}^{\sigma _{p}^{\#}(K)}$ by property (E)(ii), and simply define $u=0$
outside of $f(K)$ on $K^{F}$. Then $u$ is an extended non-trivial function
in $B_{p,\infty }^{\sigma _{p}^{\#}(K),F}$ since by (\ref{eq6.2}) and (\ref%
{204}),
\begin{equation*}
\lbrack u]_{B_{p,\infty }^{\sigma _{p}^{\#}(K),F}}^{p}\asymp \mathcal{E}%
_{p,\infty }^{\sigma _{p}^{\#}(K),F}(u)=\mathcal{E}_{p,\infty }^{\sigma
_{p}^{\#}(K)}(u\circ f)\asymp \lbrack u\circ f]_{B_{p,\infty }^{\sigma
_{p}^{\#}(K)}(K)}^{p}<\infty.
\end{equation*}
Thus we show \eqref{qnmd} and that $B_{p,\infty }^{\sigma _{p}^{\#}(K),F}$
contains non-constant functions.

Finally, since $K$ satisfies property (E), that is for all $n$,
\begin{equation*}
\mathcal{E}_{n}^{\sigma _{p}^{\#}}(u)\leq C\liminf_{k\rightarrow \infty }%
\mathcal{E}_{k}^{\sigma _{p}^{\#}}(u),
\end{equation*}%
summing over $F$ gives that for all $n$,
\begin{equation*}
\mathcal{E}_{n}^{\sigma _{p}^{\#},F}(u)\leq C\sum_{f\in
F}\liminf_{k\rightarrow \infty }\mathcal{E}_{k}^{\sigma _{p}^{\#}}(u\circ
f)\leq C\liminf_{k\rightarrow \infty }\sum_{f\in F}\mathcal{E}_{k}^{\sigma
_{p}^{\#}}(u\circ f)=C\liminf_{k\rightarrow \infty }\mathcal{E}_{k}^{\sigma
_{p}^{\#},F}(u),
\end{equation*}%
where we use Fatou's lemma to obtain the desired.
\end{proof}

When property (E) fails, $\sigma _{p}^{\#}=\sigma _{p}^{\ast }$ does not
always hold for connected homogeneous p.c.f. self-similar sets (see \cite%
{GuLau.2020.TAMS}). Moreover, we mention in passing that, the critical
domain $B_{p,\infty }^{\sigma _{p}^{\ast }}$ can be trivial for some metric
measure spaces (bounded or unbounded).

We use \cite{Caoqiugu2022adv} to show the following key lemma for nested
fractals, which is a class of connected homogeneous p.c.f. self-similar sets
defined in \cite{Kumagai.1993.PTaRF205} satisfying conditions $(A$-0$)\sim
(A $-3$)$ and $|V_{0}|\geq 2$ therein.

\begin{lemma}
\label{lemma:E}For a nested fractal, property (E) holds and $\sigma
_{p}^{\#}>\alpha/p$ for all $1<p<\infty$. \label{yyqx}
\end{lemma}

\begin{proof}
In this proof we use the same terminology and notions as \cite%
{Caoqiugu2022adv}. Let $K$ be a nested fractal. We fix the components $r_{i}$
of $\mathbf{r}$ in \cite[Theorem 6.3]{Caoqiugu2022adv} to be the same number
$r$, so it is `$\mathscr{G}$-symmetric' (see \cite[Section 3]%
{Caoqiugu2022adv} for definition). Then \cite[Theorem 6.3]{Caoqiugu2022adv}
states that, `condition $\left( \mathbf{A}\right) $' (see \cite[the
beginning of Section 4]{Caoqiugu2022adv} for definition) holds for affine
nested fractals (probably inhomogeneous), which includes $K$. By multiplying
$\mathbf{r}$ with a constant (still denoted by $\mathbf{r}$), \cite[Theorem
4.2]{Caoqiugu2022adv} guarantees that
\begin{equation}
\mathcal{T}E=E  \label{TE}
\end{equation}%
(see \cite[Definition 2.8, Definition 3.1]{Caoqiugu2022adv} for related
definitions), thus showing `condition $\left( \mathbf{A^{\prime }}\right) $'
(see \cite[the beginning of Section 5]{Caoqiugu2022adv}). So by \cite[Lemma
5.4]{Caoqiugu2022adv}, we can fix
\begin{equation}
r<1  \label{rp1}
\end{equation}%
to further satisfy condition $\left( \mathbf{A^{\prime }}\right) $ on $K$.

The equivalence of $E$ and $E_{0}^{(p)}$ (for functions $u\in l(V_{0})$, see
the statement of \cite[Theorem 5.1]{Caoqiugu2022adv}) is stated in the proof
of \cite[Proposition 5.3 (b)]{Caoqiugu2022adv}, which implies that
\begin{equation}
\Lambda ^{n}E(u)\asymp \Lambda ^{n}E_{0}^{(p)}(u)=r^{-n}E_{n}^{(p)}(u)
\label{205}
\end{equation}%
for all $n\geq 0$ and all functions $u\in l(V_{n})$ by the definition of $%
\Lambda $ in \cite[Definition 3.1]{Caoqiugu2022adv} and $E_{n}^{(p)}$ in our
paper.

Now fix
\begin{equation}
\sigma =\frac{\log _{\rho }r+\alpha }{p},  \label{s1}
\end{equation}%
so that $r=\rho ^{p\sigma -\alpha }$.

For property (E)(i), just use the monotonicity property in \cite[Proposition
5.3 (a)]{Caoqiugu2022adv}, and by (\ref{205}),
\begin{equation}
E_{0}^{(p)}(u)\asymp E(u)\leq \Lambda E(u)\leq \cdots \leq \Lambda
^{n}E(u)\asymp r^{-n}E_{n}^{(p)}(u)=\rho ^{-n(p\sigma -\alpha
)}E_{n}^{(p)}(u)=\mathcal{E}_{n}^{\sigma}(u),  \label{EE-1}
\end{equation}%
so property (E)(i) holds with this $\sigma $.

Property (E)(ii) guarantees the existence of piecewise harmonic functions in
\cite[Section 5.1]{Caoqiugu2022adv}, which has a standard iterative
construction as in \cite{HermanPeironeStrichartz.2004.PA125} and Dirichlet
form theory. The iteration part is that, for any boundary value of a cell $%
u_{w}\in l(f_{w}(V_{0}))$, by (\ref{TE}), we can take its next-level
harmonic extension $\tilde{u}_{w}\in l(f_{w}(V_{1}))$ such that $\Lambda E(%
\tilde{u}_{w})=E(u_{w})$ and $\tilde{u}_{w}|_{f_{w}(V_{0})}=u_{w}$. The
iteration starts from $V_{0}$ to $V_{1}$, and from each level-1 cell to
level-2 cells in the above way, then finally to $V_{\ast }$. The p.c.f.
property is very crucial for the piecewise extension to be well-defined,
that is, to avoid different values for the same vertex (since the overlap of
same-level cells only occurs on the same-level scaling of $V_{0}$). Such an
extension satisfies the definition of piecewise harmonic functions in \cite[%
Section 5.1]{Caoqiugu2022adv}, so we know by \cite[Section 5.2]%
{Caoqiugu2022adv} that it naturally embeds into $C(K)$ and this gives an
extension to $K$ rather than $V_{\ast }$.

Therefore, we conclude from above that $K$ satisfies property (E) with $%
\sigma $ defined as in (\ref{s1}). By (\ref{qnmd}), we have $\sigma =\sigma
_{p}^{\#}$, thus
\begin{equation}
r=\rho ^{p\sigma -\alpha }=\rho ^{p\sigma _{p}^{\#}-\alpha }.  \label{rr}
\end{equation}
The inclusion $\sigma _{p}^{\#}>\alpha /p$ follows from \eqref{rp1}.
\end{proof}

\begin{remark}
This Lemma also indicates the equivalence of heat kernel-based $p$-energy
norm $E_{p,\infty }^{\sigma _{p}^{\#}}(u)$, Besov norm $[u]_{B_{p,\infty
}^{\sigma _{p}^{\#}}}^{p}$ and the homogeneous discrete $p$-energy
constructed on a nested fractal $K$ by \cite[Theorem 5.1]{Caoqiugu2022adv}.
It is known by \cite[Theorem 8.18]{Barlow.1998.1} that $K$ admits a heat
kernel with estimates \eqref{hk_F}, so $p$-energy norm equivalence in
Theorem \ref{thm1} holds. The critical exponents $\sigma _{p}^{\#}(K)=\sigma
_{p}^{\ast }(K)$ due to property (E) using Proposition \ref{prop:critical}.
Using $\Lambda ^{n}E_{0}^{(p)}(u)\asymp \Lambda ^{n}E(u)$ in (\ref{205}),
the $p$-energy in \cite[Theorem 5.1]{Caoqiugu2022adv} with $r_{i}=r$ is
equivalent to $\lim_{n\rightarrow \infty }\Lambda ^{n}E(u)$ (which exists by
\cite[Proposition 5.3]{Caoqiugu2022adv}), where $r=\rho ^{p\sigma
_{p}^{\#}(K)-\alpha }$ defined as in (\ref{rr}). Also, we see that $%
[u]_{B_{p,\infty }^{\sigma _{p}^{\#}}}^{p}$ is equivalent to $%
\lim_{n}\Lambda ^{n}E(u)$ by noting that (\ref{204}) and (\ref{EE-1}).

The equivalence of $[u]_{B_{p,\infty }^{\sigma _{p}^{\#}}}^{p}$ and the $p$%
-energy defined in \cite{HermanPeironeStrichartz.2004.PA125} on the Sierpi%
\'{n}ski gasket have already been explained in \cite[Section 4]%
{GaoYuZhang2022PA}.
\end{remark}

%\begin{proposition}
%\label{prop32} If a homogeneous p.c.f. self-similar set $K$ satisfies
%property (E), then for all $u\in B_{p,\infty }^{\sigma ^{\ast }}$, we have
%\begin{equation}
%\liminf_{n\rightarrow \infty }\mathcal{E}_{n}^{\sigma ^{\ast }}(u)\asymp
%\limsup_{n\rightarrow \infty }\mathcal{E}_{n}^{\sigma ^{\ast }}(u)\asymp
%\sup_{n\geq 0}\mathcal{E}_{n}^{\sigma ^{\ast }}(u).  \label{412}
%\end{equation}%
%Moreover, $\mathcal{E}_{p,\infty }^{\sigma ^{\ast }}$ is a local $p$-energy
%form if $K$ satisfies property (E).
%\end{proposition}

\subsection{Equivalence of properties (VE) and (NE)}

\label{subsec5.3} Our goal is to show the equivalence of properties (VE) and
(NE) on spaces $K^{F}$. For fractal spaces, it is more direct to verify
property (VE) as Lemma \ref{yyqx} goes using previous works, since they
follow the discretization routine. Before we start our proofs, we mention
that it is quite different and more delicate to obtain the lower limit
equivalence, compared with the norm-equivalence (Lemma \ref{lem6.2} and
Corollary \ref{lem6.1}) using the upper limit. At a first glance, one needs
to be careful using the triangle inequality, since it is closely related to
`sup' rather than `inf'. In fact, we apply both Lemma \ref{lem6.2} and
Corollary \ref{lem6.1} to control each other in each single-side estimate
for the lower limit equivalence, with a key idea-truncation. Briefly
speaking, the energy-control properties guarantee that one can truncate the
two-sided estimates, so a lower bound can be obtained after we find the
relationship between finitely many terms in the summation after the
truncation.

\begin{lemma}
\label{thm_3} If a fractal glue-up $K^{F}$ satisfies $(\widetilde{VE})$ with
$\sigma >\alpha /p$, then there exists $C>0$ such that for all $u\in
B_{p,\infty }^{\sigma ,F}$,
\begin{equation}
\liminf_{n\rightarrow \infty }\mathcal{E}_{n}^{\sigma ,F}(u)\leq
C\liminf_{n\rightarrow \infty }\Phi _{u}^{\sigma ,F}(\rho ^{n}).
\label{limphi2}
\end{equation}%
%
%
%
%
%
%
%
%
%
%
%
%
%
%
%
%
%
%
%
%
%
%
%
%
%
%
%
%
%
%
%
%
%
%
%
%Moreover, we also have property (NE).
\end{lemma}

\begin{proof}
Since $C_{H}\in (0,1)$, assume that $\rho ^{s+1}\leq C_{H}<\rho ^{s}$ for
some $s\in \mathbb{N}$. For all $n>0$, we have by (\ref{II-2}) that%
\begin{equation}
I_{\infty ,n}^{F}(u)\geq \int_{K^{F}}\int_{B(x,\rho
^{n+s+1})}|u(x)-u(y)|^{p}d\mu ^{F}(y)d\mu ^{F}(x)\geq I_{n+s+1}^{F}(u).
\label{I-1}
\end{equation}

By the definition of $\limsup_{n\rightarrow \infty }\mathcal{E}_{k}^{\sigma
,F}(u)$, there exists a positive integer $n_{0}$ such that
\begin{equation*}
\sup_{k\geq n_{0}}\mathcal{E}_{k}^{\sigma ,F}(u)\leq 2\limsup_{n\rightarrow
\infty }\mathcal{E}_{n}^{\sigma ,F}(u).
\end{equation*}

Therefore, we have by (\ref{I-1}) that for all $n\geq n_{0}+s+1$ that
\begin{eqnarray}
I_{n}^{F}(u) &\leq &I_{\infty ,n-s-1}^{F}(u)  \notag \\
&\leq &C\rho ^{(n-s-1)(p\sigma +\alpha )}\sup_{k\geq n-s-1}\mathcal{E}%
_{k}^{\sigma ,F}(u)\text{ \ \ (using \eqref{eq6.3})}  \notag \\
&\leq &CC_{H}^{-(p\sigma +\alpha )}\rho ^{n(p\sigma +\alpha )}\sup_{k\geq
n_{0}}\mathcal{E}_{k}^{\sigma ,F}(u)\leq 2CC_{H}^{-(p\sigma +\alpha )}\rho
^{n(p\sigma +\alpha )}\limsup_{n\rightarrow \infty }\mathcal{E}_{n}^{\sigma
,F}(u)  \notag \\
&\leq &C_{1}\rho ^{n(p\sigma +\alpha )}\liminf_{n\rightarrow \infty }%
\mathcal{E}_{n}^{\sigma ,F}(u)\text{ \ \ (by property }(\widetilde{VE})\text{%
)}.  \label{I_n}
\end{eqnarray}%
By Lemma \ref{lem6.3}, we also have for any positive integer $L\geq
n_{0}+s+1 $ and $0<\delta <p\sigma -\alpha $,
\begin{align*}
\mathcal{E}_{n}^{\sigma ,F}(u)& \leq C_{2}\sum_{k=0}^{\infty }\rho
^{(-2\alpha -\delta )k}\rho ^{-(p\sigma +\alpha )n}I_{k+n}^{F}(u) \\
& \leq C_{2}\sum_{k=0}^{L}\rho ^{(-2\alpha -\delta )k}\rho ^{-(p\sigma
+\alpha )n}I_{k+n}^{F}(u)+C_{2}\sum_{k=L}^{\infty }\rho ^{(-2\alpha -\delta
)k}\rho ^{-(p\sigma +\alpha )n}I_{k+n}^{F}(u) \\
& \leq C_{2}\sum_{k=0}^{L}\rho ^{(-2\alpha -\delta )k}\rho ^{-(p\sigma
+\alpha )n}I_{k+n}^{F}(u)+C_{1}C_{2}\sum_{k=L}^{\infty }\rho ^{(p\sigma
-\alpha -\delta )k}\liminf_{n\rightarrow \infty }\mathcal{E}_{n}^{\sigma
,F}(u), \\
& =C_{2}\left( \sum_{k=0}^{L}\rho ^{(-2\alpha -\delta )k}\rho ^{-(p\sigma
+\alpha )n}I_{k+n}^{F}(u)+\frac{C_{1}\rho ^{(p\sigma -\alpha -\delta )L}}{%
1-\rho ^{(p\sigma -\alpha -\delta )}}\liminf_{n\rightarrow \infty }\mathcal{E%
}_{n}^{\sigma ,F}(u)\right),
\end{align*}%
where we use \eqref{I_n} in the third line. Taking $\liminf $ in the
right-hand side above, we have
\begin{equation*}
C_{3}\liminf_{n\rightarrow \infty }\mathcal{E}_{n}^{\sigma ,F}(u)\leq
\liminf_{n\rightarrow \infty }\sum_{k=0}^{L}\rho ^{-(2\alpha +\delta )k}\rho
^{-(p\sigma +\alpha )n}I_{k+n}^{F}(u),
\end{equation*}%
where
\begin{equation*}
C_{3}:=\frac{1}{C_{2}}-\frac{C_{1}\rho ^{(p\sigma -\alpha -\delta )L}}{%
1-\rho ^{(p\sigma -\alpha -\delta )}}.
\end{equation*}%
Fix a large integer $L$ such that $C_{3}>0$, then
\begin{align*}
C_{3}\rho ^{(2\alpha +\delta )L}\liminf_{n\rightarrow \infty }\mathcal{E}%
_{n}^{\sigma ,F}(u)& \leq \rho ^{(2\alpha +\delta )L}\liminf_{n\rightarrow
\infty }\left( \sum_{k=0}^{L}\rho ^{-n(p\sigma +\alpha )}\rho ^{-(2\alpha
+\delta )k}I_{n+k}^{F}(u)\right) \\
& \leq \liminf_{n\rightarrow \infty }\rho ^{-n(p\sigma +\alpha
)}\sum_{k=0}^{L}I_{n+k}^{F}(u)\leq \liminf_{n\rightarrow \infty }\Phi
_{u}^{\sigma ,F}(\rho ^{n}),
\end{align*}%
thus showing \eqref{limphi2}.
\end{proof}

For the other side, we need certain control between different levels of
vertex-energies.

\begin{proposition}
Let $K$ be a connected homogeneous p.c.f. self-similar set, then there
exists $C>0$ such that for all $u\in L^{p}(K,\mu )$,
\begin{equation*}
E_{n}^{(p)}(u)\leq CE_{n+1}^{(p)}(u).
\end{equation*}%
Moreover, there exists $C>0$ such that for all $u\in L^{p}(K^{F},\mu ^{F})$,
\begin{equation}
E_{n}^{(p),F}(u)\leq CE_{n+1}^{(p),F}(u).  \label{E-n}
\end{equation}
\end{proposition}

\begin{proof}
We start with the basic case $n=0$. For any $x,y\in V_{0}$, we can fix a
vertex-disjoint path $x=z_{1}(x,y),z_{2}(x,y),\cdots ,z_{k}(x,y)=y$ such
that $z_{j}(x,y)$ and $z_{j+1}(x,y)$ ($1\leq j\leq k$) belong to a same $%
V_{w}$ for some $w\in W_{1}$, and $2\leq k\leq |V_{1}|$. Using Jensen's
inequality,
\begin{align*}
|u(x)-u(y)|^{p}& \leq
(k-1)^{p-1}\sum_{j=1}^{k-1}|u(z_{j}(x,y))-u(z_{j+1}(x,y))|^{p} \\
& \leq |V_{1}|^{p-1}\sum_{j=1}^{k-1}|u(z_{j}(x,y))-u(z_{j+1}(x,y))|^{p}.
\end{align*}%
It follows that
\begin{align*}
E_{0}^{(p)}(u)& \leq \sum_{x,y\in
V_{0}}|V_{1}|^{p-1}\sum_{j=1}^{k-1}|u(z_{j}(x,y))-u(z_{j+1}(x,y))|^{p} \\
& \leq |V_{0}|^{2}|V_{1}|^{p-1}\sum_{a,b\in V_{w},|w|=1}|u(a)-u(b)|^{p} \\
& =|V_{0}|^{2}|V_{1}|^{p-1}E_{1}^{(p)}(u),
\end{align*}%
since for each pair $(a,b)\in V_{w}$ with $|w|=1$, there exists at most one $%
j\in \{1,2,...k\}$ (depending on $(x,y)$) with $%
(z_{j}(x,y),z_{j+1}(x,y))=(a,b)$ for any $x,y\in V_{0}$. Thus, we show the
desired for $n=0$ with $C:=|V_{0}|^{2}|V_{1}|^{p-1}$ for any $u$. The rest
simply follows from that
\begin{align*}
E_{n}^{(p)}(u)& =\sum_{x,y\in
V_{w},|w|=n}|u(x)-u(y)|^{p}=\sum_{|w|=n}E_{0}^{(p)}(u\circ \phi _{w}) \\
& \leq C\sum_{|w|=n}E_{1}^{(p)}(u\circ \phi _{w})=C\sum_{a,b\in
V_{wi},|w|=n,|i|=1}|u(a)-u(b)|^{p}=CE_{n+1}^{(p)}(u).
\end{align*}%
The second inclusion follows by the definition of $K^{F}$.
\end{proof}

\begin{lemma}
\label{thm_4} If property $(\widetilde{NE})$ holds for a fractal glue-up $%
K^{F}$ with $\sigma >\alpha /p$, then there exists $C>0$ such that for all $%
u\in B_{p,\infty }^{\sigma ,F}$,
\begin{equation}
\liminf_{n\rightarrow \infty }\Phi _{u}^{\sigma ,F}(\rho ^{n})\leq
C\liminf_{n\rightarrow \infty }\mathcal{E}_{n}^{\sigma ,F}(u).
\label{limphi}
\end{equation}
% Moreover, we also have property $%(VE) $.
\end{lemma}

\begin{proof}
By \eqref{I_mn}, we have
\begin{align}
\rho ^{-n(p\sigma +\alpha )}I_{m,n}^{F}(u)\leq & C\rho ^{-n(p\sigma -\alpha
)}\sum_{k=n}^{m}\rho ^{k(p\sigma -\alpha )}\rho ^{-k(p\sigma -\alpha
)}E_{k}^{(p),F}(u)  \notag \\
\leq & C\sum_{k=n}^{m}\rho ^{(k-n)(p\sigma -\alpha )}\mathcal{E}_{k}^{\sigma
,F}(u)\leq C\sum_{k=0}^{\infty }\rho ^{k(p\sigma -\alpha )}\mathcal{E}%
_{k+n}^{\sigma ,F}(u).  \label{202}
\end{align}

By the definition of $\limsup_{n\rightarrow \infty }\mathcal{E}_{n}^{\sigma
,F}(u)$, there exists $n_{0}>0$ such that for all positive integer $L>n_{0}$%
,
\begin{eqnarray*}
\sup_{n\geq L}\mathcal{E}_{n}^{\sigma ,F}(u) &\leq &2\limsup_{n\rightarrow
\infty }\mathcal{E}_{n}^{\sigma ,F}(u) \\
&\asymp &\limsup_{n\rightarrow \infty }\Phi _{u}^{\sigma ,F}(\rho ^{n})\text{
\ (by Corollary \ref{lem6.1})} \\
&\leq &C\liminf_{n\rightarrow \infty }\Phi _{u}^{\sigma ,F}(\rho ^{n})\text{
\ \ (by property }(\widetilde{NE})\text{)}.
\end{eqnarray*}

It follows from (\ref{202}) that
\begin{eqnarray*}
\rho ^{-n(p\sigma +\alpha )}I_{m,n}^{F}(u) &\leq &C\sum_{k=0}^{\infty }\rho
^{k(p\sigma -\alpha )}\mathcal{E}_{k+n}^{\sigma ,F}(u) \\
&=&C\sum_{k=0}^{L}\rho ^{k(p\sigma -\alpha )}\mathcal{E}_{k+n}^{\sigma
,F}(u)+C\sum_{k=L+1}^{\infty }\rho ^{k(p\sigma -\alpha )}\mathcal{E}%
_{k+n}^{\sigma ,F}(u) \\
&\leq &C\sum_{k=0}^{L}\rho ^{k(p\sigma -\alpha )}\mathcal{E}_{k+n}^{\sigma
,F}(u)+C\sum_{k=L+1}^{\infty }\rho ^{k(p\sigma -\alpha )}\sup_{n\geq L}%
\mathcal{E}_{n}^{\sigma ,F}(u) \\
&\leq &C\sum_{k=0}^{L}\rho ^{k(p\sigma -\alpha )}\mathcal{E}_{k+n}^{\sigma
,F}(u)+C^{\prime }\sum_{k=L+1}^{\infty }\rho ^{k(p\sigma -\alpha
)}\liminf_{n\rightarrow \infty }\Phi _{u}^{\sigma ,F}(\rho ^{n}) \\
&=&C\sum_{k=0}^{L}\rho ^{k(p\sigma -\alpha )}\mathcal{E}_{k+n}^{\sigma
,F}(u)+C^{\prime }\frac{\rho ^{(p\sigma -\alpha )(L+1)}}{1-\rho ^{p\sigma
-\alpha }}\liminf_{n\rightarrow \infty }\Phi _{u}^{\sigma ,F}(\rho ^{n}).
\end{eqnarray*}

Using the fact that $\mu _{m}\times \mu _{m}$ weak {$\ast $}-converges to $%
\mu \times \mu $ as $m\rightarrow \infty $, we have
\begin{align*}
\rho ^{-n(p\sigma +\alpha )}I_{\infty ,n}^{F}(u)=& \rho ^{-n(p\sigma +\alpha
)}\lim_{m\rightarrow \infty }I_{m,n}^{F}(u) \\
\leq & C\sum_{k=0}^{L}\rho ^{k(p\sigma -\alpha )}\mathcal{E}_{k+n}^{\sigma
,F}(u)+\frac{C^{\prime }\rho ^{(p\sigma -\alpha )(L+1)}}{1-\rho ^{p\sigma
-\alpha }}\liminf_{n\rightarrow \infty }\Phi _{u}^{\sigma ,F}(\rho ^{n}).
\end{align*}

Combining this and (\ref{II-1}), we have%
\begin{equation*}
\rho ^{p\sigma +\alpha }C_{H}^{p\sigma +\alpha }\liminf_{n\rightarrow \infty
}\Phi _{u}^{\sigma ,F}(\rho ^{n})\leq C\liminf_{n\rightarrow \infty
}\sum_{k=0}^{L}\rho ^{k(p\sigma -\alpha )}\mathcal{E}_{k+n}^{\sigma ,F}(u)+%
\frac{C^{\prime }\rho ^{(p\sigma -\alpha )(L+1)}}{1-\rho ^{p\sigma -\alpha }}%
\liminf_{n\rightarrow \infty }\Phi _{u}^{\sigma ,F}(\rho ^{n}).
\end{equation*}

For sufficiently large $L$ ($>n_{0}$), we have
\begin{equation*}
C^{\prime \prime }:=\frac{1}{C}\left( \rho ^{p\sigma +\alpha }C_{H}^{p\sigma
+\alpha }-\frac{C^{\prime }\rho ^{(p\sigma -\alpha )(L+1)}}{1-\rho ^{p\sigma
-\alpha }}\right) >0.
\end{equation*}

Then for this $L$, we obtain
\begin{align}
C^{\prime \prime }\liminf_{n\rightarrow \infty }\Phi _{u}^{\sigma ,F}(\rho
^{n})& \leq \liminf_{n\rightarrow \infty }\sum_{k=0}^{L}\rho ^{k(p\sigma
-\alpha )}\mathcal{E}_{k+n}^{\sigma ,F}(u)  \notag \\
& =\liminf_{n\rightarrow \infty }\sum_{k=0}^{L}\rho ^{k(p\sigma -\alpha
)}\rho ^{-(k+n)(p\sigma -\alpha )}E_{k+n}^{(p),F}(u)  \notag \\
& =\liminf_{n\rightarrow \infty }\sum_{k=0}^{L}\rho ^{-n(p\sigma -\alpha
)}E_{k+n}^{(p),F}(u).  \label{E_kn}
\end{align}

Using \eqref{E-n}, there exists $c>0$ such that for all positive integer $n$
and $u\in B_{p,\infty }^{\sigma ,F}$,%
\begin{equation*}
E_{n}^{(p),F}(u)\leq cE_{n+1}^{(p),F}(u).
\end{equation*}
Applying the above inequality to \eqref{E_kn}, we have
\begin{align*}
C^{\prime \prime }\liminf_{n\rightarrow \infty }\Phi _{u}^{\sigma ,F}(\rho
^{n})& \leq \liminf_{n\rightarrow \infty }C_{L}\rho ^{-(n+L)(p\sigma -\alpha
)}E_{n+L}^{(p),F}(u) \\
& =C_{L}\liminf_{n\rightarrow \infty }\mathcal{E}_{n+L}^{(p),F}(u)=C_{L}%
\liminf_{n\rightarrow \infty }\mathcal{E}_{n}^{(p),F}(u),
\end{align*}%
where $C_{L}=\frac{1-c^{L}}{1-c}\rho ^{L(p\sigma -\alpha )}$ does not depend
on $n$, which proves \eqref{limphi}.
\end{proof}

\begin{proof}[Proof of Theorem \protect\ref{thm5}]
Note that (\ref{ns}), (\ref{ne}). Using \eqref{eq6.2} and \eqref{limphi2}, $(%
\widetilde{VE})$ with $\sigma >\alpha /p$ implies $(\widetilde{NE})$ since
\begin{equation}
\limsup_{r\rightarrow 0}\Phi _{u}^{\sigma ,F}(r)\leq C\limsup_{n\rightarrow
\infty }\mathcal{E}_{n}^{\sigma ,F}(u)\leq C^{\prime }\liminf_{n\rightarrow
\infty }\mathcal{E}_{n}^{\sigma ,F}(u)\leq C^{\prime \prime
}\liminf_{r\rightarrow 0}\Phi _{u}^{\sigma ,F}(r).  \label{ve_ne1}
\end{equation}%
By \eqref{eq6.2} and \eqref{limphi}, $(\widetilde{NE})$ with $\sigma >\alpha
/p$ implies $(\widetilde{VE})$ since
\begin{equation}
\limsup_{n\rightarrow \infty }\mathcal{E}_{n}^{\sigma ,F}(u)\leq
C_{1}\limsup_{r\rightarrow 0}\Phi _{u}^{\sigma ,F}(r)\leq C_{1}^{\prime
}\liminf_{r\rightarrow 0}\Phi _{u}^{\sigma ,F}(r)\leq C_{1}^{\prime \prime
}\liminf_{n\rightarrow \infty }\mathcal{E}_{n}^{\sigma ,F}(u).
\label{ne_ve1}
\end{equation}%
Thus, by \eqref{ve_ne1} and \eqref{ne_ve1}, we show Theorem \ref{thm5} (i).

By \eqref{eq6.2} and \eqref{limphi2} adjoint with $(VE)\Rightarrow (%
\widetilde{VE})$, we immediately have from (\ref{ns}), (\ref{ne}) that
\begin{equation}
\sup_{r\in (0,1)}\Phi _{u}^{\sigma ,F}(r)\leq C_{2}\sup_{n\rightarrow \infty
}\mathcal{E}_{n}^{\sigma ,F}(u)\leq C_{2}^{\prime }\liminf_{n\rightarrow
\infty }\mathcal{E}_{n}^{\sigma ,F}(u)\leq C_{2}^{\prime \prime
}\liminf_{r\rightarrow 0}\Phi _{u}^{\sigma ,F}(r).  \label{ve_ne2}
\end{equation}%
Similarly, by \eqref{eq6.2} and \eqref{limphi} adjoint with the fact $%
(NE)\Rightarrow (\widetilde{NE})$, we immediately have
\begin{equation}
\sup_{n\geq 0}\mathcal{E}_{n}^{\sigma ,F}(u)\leq C_{3}\sup_{n\geq 0}\Phi
_{u}^{\sigma ,F}(\rho ^{n})\leq C_{3}^{\prime }\liminf_{n\rightarrow \infty
}\Phi _{u}^{\sigma ,F}(\rho ^{n})\leq C_{3}^{\prime \prime
}\liminf_{n\rightarrow \infty }\mathcal{E}_{n}^{\sigma ,F}(u).
\label{ne_ve2}
\end{equation}%
Thus, by \eqref{ve_ne2} and \eqref{ne_ve2}, we show Theorem \ref{thm5} (ii).
\end{proof}

%Finally, we mention that, by Theorem \ref{thm4.2}, Theorem \ref{thm5} (ii)
%and Corollary \ref{lem6.1}, we can only use property (VE) instead of
%property (E) in \cite[Theorem 1.6]{GaoYuZhang2022PA} to obtain the
%vertex-type BBM convergence of non-local $p$-energy forms to a local $p$%
%-energy form for $K$, without using the Gamma-convergence.

\subsection{Consequences of (VE): BBM convergence and Gagliardo-Nirenberg
inequality}

\label{subsec5.4} In this subsection, we turn to the following embedding inequality
given by Baudoin, using our notions.

\begin{lemma}
\label{GNTM}(\cite[Theorem 4.3]{BaudoinLecture2022}) Suppose that property
(NE) holds with $R_0=\infty$ for the metric measure space $(M,d,\mu )$, and
there exists $R>0$ such that $\inf_{x\in M}\mu (B(x,R))>0$. When $p\sigma
_{p}^{\#}\neq \alpha_1 $, where $\alpha_1$ is from (\ref{VD2}), let $q=\frac{%
p\alpha_1 }{\alpha_1 -p\sigma_{p}^{\#}}$. For $r,s\in (0,\infty ]$, $\theta
\in (0,1]$ satisfying
\begin{equation*}
\frac{1}{r}=\frac{\theta }{q}+\frac{1-\theta }{s},
\end{equation*}%
there exists a constant $C>0$ such that for any $f\in B_{p,\infty }^{\sigma
_{p}^{\#}}$,
\begin{equation}
\Vert f\Vert _{r}\leq C(\Vert f\Vert _{p}+[f]_{B_{p,\infty }^{\sigma
_{p}^{\#}}})^{\theta }\Vert f\Vert _{s}^{1-\theta }.  \label{GNI}
\end{equation}
\end{lemma}


By Theorem \ref{thm4} and Theorem \ref{thm5}, we have the following
corollary.

\begin{corollary}
\label{corol1.8}Suppose that $K^{F}$ is a fractal glue-up that admits a heat
kernel satisfying \eqref{hk_F}, where $K$ is a connected homogeneous p.c.f.
self-similar set. Let $R_0=1$. Then properties $(\widetilde{KE}),(\widetilde{%
NE}),(\widetilde{VE})$ are equivalent with the same $\sigma>\alpha /p$, and
properties $(KE),(NE),(VE)$ are equivalent when $\sigma_{p}^{\#}>\alpha /p$
on $K^{F}$.
\end{corollary}

The following theorem can be regarded as a summary of our paper restricted
to fractal spaces.

\begin{theorem}
\label{corol1.9}Let $K^{F}$ be a fractal glue-up, where $K$ is a connected
homogeneous p.c.f. self-similar set satisfying property (E). Then $K^{F}$
satisfies (NE) with $R_{0}=1$ and $B_{p,\infty }^{\sigma
_{p}^{\#}(K^{F})}(K^{F}):=B_{p,\infty }^{\sigma _{p}^{\#}}(K^{F})$ contains
non-constant functions. Also, the BBM convergence (\ref{NE_conv}) holds for $%
K^{F}$ with $R_{0}=1$, that is, for $R_{0}=1$, there exists a positive
constant $C$ such that for all $u\in B_{p,\infty }^{\sigma _{p}^{\#}}(K^{F})$,
\begin{equation}
C^{-1}[u]_{{B}_{p,\infty }^{\sigma _{p}^{\#}}(K^{F})}^{p}\leq
\liminf_{\sigma \uparrow \sigma _{p}^{\#}(K^{F})}\left( \sigma
_{p}^{\#}(K^{F})-\sigma \right) [u]_{{B}_{p,p}^{\sigma }(K^{F})}^{p}\leq
\limsup_{\sigma \uparrow \sigma _{p}^{\#}(K^{F})}\left( \sigma
_{p}^{\#}(K^{F})-\sigma \right) [u]_{{B}_{p,p}^{\sigma }(K^{F})}^{p}\leq
C[u]_{{B}_{p,\infty }^{\sigma _{p}^{\#}}(K^{F})}^{p}.  \label{SS-2}
\end{equation}
Furthermore, if $K$ is a nested fractal and $K^{F}=K_{\infty }$, where $%
K_{\infty }$ is defined in Example \ref{ex1}, then (NE) and (KE) are
satisfied with $R_{0}=\infty $, and the Gagliardo-Nirenberg inequality (\ref%
{GNI}) holds true.
\end{theorem}

\begin{proof}
By Proposition \ref{prop:critical}, $\sigma _{p}^{\#}(K)=\sigma
_{p}^{\#}(K^{F})$ (now we denote it by $\sigma _{p}^{\#}$ in this proof for
convenience) and $K^{F}$ satisfies (VE) and the critical domain $B_{p,\infty
}^{\sigma _{p}^{\#}}(K^{F})$ is non-trivial, so (NE) holds true for $K^{F}$
by Theorem \ref{thm5} with $R_{0}=1$, then (\ref{SS-2}) holds by Theorem \ref%
{thm4.2}.

For a nested fractal $K$, it is known by \cite{Kumagai.1993.PTaRF205} that $%
K_{\infty }$ admits a heat kernel satisfying \eqref{hk_F} (see also \cite[%
Theorem 1.1]{FitzsimmonsmHamblyKumagai.1994.CMP595}). Combining Lemma \ref%
{lemma:E}, previous inclusions hold. Now we state how to upgrade $R_0=1$ to $%
R_0=\infty $ for (NE).

For any $n\in \mathbb{N}$, define the scaling function $g_{n}:K^{F}%
\rightarrow \rho ^{-n}K^{F}$ by
\begin{equation*}
g_{n}(x)=\rho ^{-n}x.
\end{equation*}
Due to the global self-similarity of $K_{\infty }$, we have
\begin{equation}
g_{n}(K_{\infty })=K_{\infty }.  \label{ss}
\end{equation}

For any $u\in B_{p,\infty }^{\sigma ,F}$, clearly $u\circ g_{n}\in
B_{p,\infty }^{\sigma ,F}$. We have by definition that for all $r>0$ and all
$u\in B_{p,\infty }^{\sigma ,F}$,
\begin{eqnarray}
\Phi _{u\circ g_{n}}^{\sigma ,F}(\rho ^{n}r) &=&(\rho ^{n}r)^{-p\sigma
-\alpha }\int_{K_{\infty }}\int_{B(x,\rho
^{n}r)}|u(g_{n}(x))-u(g_{n}(y))|d\mu ^{F}(y)d\mu ^{F}(x)  \notag \\
&=&(\rho ^{n}r)^{-p\sigma -\alpha }\int_{K_{\infty
}}\int_{B(g_{n}(x),r)}|u(g_{n}(x))-u(y)|d(\mu ^{F}\circ g_{n}^{-1})(y)d\mu
^{F}(x)  \notag \\
&=&(\rho ^{n}r)^{-p\sigma -\alpha }\int_{g_{n}(K_{\infty
})}\int_{B(x,r)}|u(x)-u(y)|d\mu ^{F}(\rho ^{n}y)d\mu ^{F}(\rho ^{n}x)  \notag
\\
&\asymp &(\rho ^{n}r)^{-p\sigma -\alpha }\rho ^{2n\alpha }\int_{K_{\infty
}}\int_{B(x,r)}|u(x)-u(y)|d\mu ^{F}(y)d\mu ^{F}(x)\   \notag \\
&=&\rho ^{-n(p\sigma -\alpha )}\Phi _{u}^{\sigma ,F}(r),  \label{sp}
\end{eqnarray}%
where we use (\ref{ss}) and $\mu ^{F}$ is $\alpha $-regular in the forth
line. For any $u\in B_{p,\infty }^{\sigma ,F}$, there exists $n<\infty $
such that
\begin{equation*}
\sup_{r<\infty }\Phi _{u}^{\sigma ,F}(r)\leq 2\sup_{r<\rho ^{-n}}\Phi
_{u}^{\sigma ,F}(r).
\end{equation*}%
Combined with (\ref{sp}),
\begin{equation*}
\sup_{r<\infty }\Phi _{u}^{\sigma ,F}(r)\leq C\sup_{r^{\prime }<1}\Phi
_{u\circ g_{n}}^{\sigma ,F}(r^{\prime })\rho ^{n(p\sigma -\alpha )}.
\end{equation*}%
As (NE) holds for $R_{0}=1$,
\begin{eqnarray*}
\sup_{r^{\prime }<1}\Phi _{u\circ g_{n}}^{\sigma _{p}^{\#},F}(r^{\prime
})\rho ^{n(p\sigma _{p}^{\#}-\alpha )} &\leq &C\liminf_{r^{\prime
}\rightarrow 0}\Phi _{u\circ g_{n}}^{\sigma _{p}^{\#},F}(r^{\prime })\rho
^{n(p\sigma _{p}^{\#}-\alpha )} \\
&=&C\liminf_{r\rightarrow 0}\Phi _{u}^{\sigma _{p}^{\#},F}(\rho ^{-n}r)\text{
\ (by (\ref{sp}))} \\
&=&C\liminf_{r\rightarrow 0}\Phi _{u}^{\sigma _{p}^{\#},F}(r),\text{ }
\end{eqnarray*}%
where the constant $C$ does not depend on $u,n$. This means that (NE) also
holds with $R_{0}=\infty $. Since $p\sigma_p^{\#}>\alpha$ and noting that $%
\mu ^{F}$ is $\alpha $-regular, by Lemma \ref{lemma:E}, we know by Lemma \ref%
{GNTM} that (\ref{GNI}) holds true.
\end{proof}

The existence of fixed points of the renormalization problem is intimately
related to the energy comparable properties (VE) and (NE). Here we use a
quite different way (the two-sided heat kernel estimates) to obtain the
following corollary, and our proof is analytical (compared with the linear
algebra method used by Kigami, or the electric-network methods used in \cite%
{Yang.2018.PA} etc.). The only `algebraic' property we use is the spectral
decomposition for heat kernels (which is essential being the reason why (KE)
always hold for $p=2$).

\begin{corollary}\label{jjj}
When $p=2$, if a fractal glue-up $K^{F}$ admits two-sided estimates %
\eqref{hk_F}, then (VE) and (NE) automatically hold for $K^F $.
\end{corollary}

\begin{proof}
By Remark \ref{rk1}, property (KE) automatically holds for $K^F$ when $p=2$.
Using Corollary \ref{corol1.8}, we have (VE) and (NE) automatically hold for
$K^F$.
\end{proof}

\section{Further discussions and open problems}

Previous studies have already left many open problems, and here we discuss
what other open problems that interest us.

We wonder whether there is any connection between different types of
infinitesimal nature. It seems like the regularity of 2-Laplacian should
also give certain regularity of $p$-Laplacian, although they can be quite
different even on curves with complicated topology like the Sierpi\'{n}ski
gasket (see \cite{StrichartzWong2004}), the critical domain might bear
little relation when $p$ varies as \cite{HermanPeironeStrichartz.2004.PA125}
suggested.
\begin{conjecture}
Suppose that two metric measure spaces $(M_1,d_1,\mu_1),(M_2,d_2,\mu_2)$ admit two heat kernels satisfying \eqref{hk} respectively, with the same walk-dimension $\beta^*$, where $\mu_1$ and $\mu_2$ are both $\alpha$-regular.
Is it true that $\sigma_p^*(M_1,d_1,\mu_1)=\sigma_p^*(M_2,d_2,\mu_2)$, $\sigma_p^{\#}(M_1,d_1,\mu_1)=\sigma_p^{\#}(M_2,d_2,\mu_2)$, $\sigma_p^*(M_1,d_1,\mu_1)=\sigma_p^{\#}(M_1,d_1,\mu_1)$ ?
\end{conjecture}Also one wants to know that when (KE) holds for a general $p$%
.
\begin{conjecture}Does two-sided estimates \eqref{hk} imply
(KE)?\end{conjecture}Beyond this, it is also interesting to consider other
infinitesimal characterization using heat kernels (2-Laplacian). Heat
kernel-based $p$-energy has a short and direct expression than other types
of $p$-energy (which involves discretization), especially when the
underlying space do not have Ahlfors-regularity (even for the Dirichlet
form). For spaces with less regular heat kernel, one can consider the
variant of \eqref{psi_u}:
\begin{equation}
\Psi _{u}^{g,h}(t)=\frac{1}{g(t)}\int_{M}\int_{M}h(|u(x)-u(y)|)p_{t}(x,y)d%
\mu (y)d\mu (x),
\end{equation}%
where $g,\ h$ are two fixed `gauge functions' from $[0,\infty )$ to $%
[0,\infty )$, and $u$ is any measurable function on the metric measure
space. We can restrict $g,\ h$ to be continuous on $[0,\infty )$, and
monotonic increasing with $g(0)=h(0)=0.$ Then one can similarly consider
\begin{equation*}
E_{g,h}(u):=\sup_{t\in (0,R_{0}^{\beta ^{\ast }})}\Psi _{u}^{g,h}(t)
\end{equation*}%
with its domain
\begin{equation*}
\mathcal{D}(E_{g,h}):=\{E_{g,h}(u)<\infty \}.
\end{equation*}%
For each such $h$, like we take $h(x)=x^{p}$, one might want to find some
critical (maximal) gauge function $g$, such that $\mathcal{D}(E_{g,h})$ is
non-trivial or dense in certain function spaces. The existence and
uniqueness of such $g$ is an interesting topic, and one might try to
generalise some classical analytic topics for such energies. Unfortunately,
we have not find good heat-kernel tools as the case $p=2$ or $p=1$, due to
the lack of linearity.

We will further generalize Section \ref{sec5} to other metric spaces like
non-p.c.f. fractals in upcoming papers.

\begin{acknowledgement}
The authors were supported by National Natural Science Foundation of China
(11871296). The authors thank Jiaxin Hu, Eryan Hu, Qingsong Gu and Meng Yang
for valuable discussions.
\end{acknowledgement}

\begin{thebibliography}{99}

\bibitem{RuizBaudoin2021JMAA}
{\sc P.~Alonso~Ruiz and F.~Baudoin}, {\em Gagliardo-{N}irenberg,
  {T}rudinger-{M}oser and {M}orrey inequalities on {D}irichlet spaces}, J.
  Math. Anal. Appl., 497 (2021), pp.~124899.

\bibitem{AlonsoBaudoinchen2020JFA}
{\sc P.~Alonso-Ruiz, F.~Baudoin, L.~Chen, L.~Rogers, N.~Shanmugalingam, and
  A.~Teplyaev}, {\em Besov class via heat semigroup on {D}irichlet spaces {I}:
  {S}obolev type inequalities}, J. Funct. Anal., 278 (2020), pp.~108459.

\bibitem{AlonsoBaudoinchen2020CVPDE}
\leavevmode\vrule height 2pt depth -1.6pt width 23pt, {\em Besov class via heat
  semigroup on {D}irichlet spaces {II}: {BV} functions and {G}aussian heat
  kernel estimates}, Calc. Var. Partial Differential Equations, 59 (2020),
  pp.~103.

\bibitem{AlonsoBaudoinchen2021CVPDE}
\leavevmode\vrule height 2pt depth -1.6pt width 23pt, {\em Besov class via heat
  semigroup on {D}irichlet spaces {III}: {BV} functions and sub-{G}aussian heat
  kernel estimates}, Calc. Var. Partial Differential Equations, 60 (2021),
  pp.~170.

\bibitem{Barlow.1998.1}
{\sc M.~Barlow}, {\em Diffusions on fractals}, vol.~{1690} of Lect. Notes
  Math., Springer, 1998, pp.~1--121.

\bibitem{BarlowBass.1989.AIHPPS225}
{\sc M.~Barlow and R.~Bass}, {\em The construction of {B}rownian motion on the
  {S}ierpinski carpet}, Ann. Inst. H. Poincar{\'e} Probab. Statist., {25}
  (1989), pp.~225--257.

\bibitem{BaudoinLecture2022}
{\sc F.~Baudoin}, {\em Korevaar-{S}choen-{S}obolev spaces and critical
  exponents in metric measure spaces}, arXiv:2207.12191,  (2022).

\bibitem{BourgainBrezisMironescu.2001.439}
{\sc J.~Bourgain, H.~Brezis, and P.~Mironescu}, {\em Another look at Sobolev
  spaces, Optimal Control and Partial Differential Equations (J.L. Menaldi et
  al. eds)}, IOS Press, Amsterdam, 2001, pp.~439--455.

\bibitem{Caoqiugu2022adv}
{\sc S.~Cao, Q.~Gu, and H.~Qiu}, {\em {$p$}-energies on p.c.f. self-similar
  sets}, Adv. Math., 405 (2022), pp.~108517.

\bibitem{Denglau.2008}
{\sc Q.-R. Deng and K.-S. Lau}, {\em Open set condition and post-critically
  finite self-similar sets}, Nonlinearity, 21 (2008), pp.~1227--1232.

\bibitem{FitzsimmonsmHamblyKumagai.1994.CMP595}
{\sc P.~Fitzsimmonsm, B.~Hambly, and T.~Kumagai}, {\em Transition density
  estimates for {B}rownian motion on affine nested fractals}, Comm. Math.
  Phys., {165} (1994), pp.~595--620.

\bibitem{FukushimaOshimaTakeda.2011.489}
{\sc M.~Fukushima, Y.~Oshima, and M.~Takeda}, {\em Dirichlet forms and
  symmetric {M}arkov processes}, vol.~19 of de Gruyter Studies in Mathematics,
  Walter de Gruyter \& Co., Berlin, extended~ed., 2011.

\bibitem{GaoYuZhang2022PA}
{\sc J.~Gao, Z.~Yu, and J.~Zhang}, {\em Convergence of p-energy forms on
  homogeneous p.c.f self-similar sets}, Potential Anal.,  (2022), https://doi.org/10.1007/s11118-022-10031-y.

\bibitem{Gorny2022}
{\sc W.~G\'{o}rny}, {\em Bourgain-{B}rezis-{M}ironescu approach in metric
  spaces with {E}uclidean tangents}, J. Geom. Anal., 32 (2022), pp.~128.

\bibitem{Grigoryan..MS}
{\sc A.~Grigor'yan}, {\em The heat equation on noncompact {R}iemannian
  manifolds}, Math. USSR-Sb., 72 (1992), pp.~47--77.

\bibitem{GrigoryanHuHu.2017.JFA3311}
{\sc A.~Grigor'yan, E.~Hu, and J.~Hu}, {\em Lower estimates of heat kernels for
  non-local {D}irichlet forms on metric measure spaces}, J. Funct. Anal., 272
  (2017), pp.~3311--3346.

\bibitem{GrigoryanHu.2014.MMJ505}
{\sc A.~Grigor'yan and J.~Hu}, {\em Upper bounds of heat kernels on doubling
  spaces}, Moscow Math. J., {14} (2014), pp.~505--563.

\bibitem{GrigoryanHuLau.2003.TAMS2065}
{\sc A.~Grigor'yan, J.~Hu, and K.-S. Lau}, {\em Heat kernels on metric measure
  spaces and an application to semilinear elliptic equations}, Trans. Amer.
  Math. Soc., 355 (2003), pp.~2065--2095.

\bibitem{GrigoryanTelcs.2012.AoP1212}
{\sc A.~Grigor'yan and A.~Telcs}, {\em Two-sided estimates of heat kernels on
  metric measure spaces}, Annals of Probability, {40} (2012), pp.~1212--1284.

\bibitem{GuLau.2020.AASFM}
{\sc Q.~Gu and K.-S. Lau}, {\em Dirichlet forms and convergence of {B}esov
  norms on self-similar sets}, Ann. Acad. Sci. Fenn. Math., 45 (2020),
  pp.~625--646.

\bibitem{GuLau.2020.TAMS}
\leavevmode\vrule height 2pt depth -1.6pt width 23pt, {\em Dirichlet forms and
  critical exponents on fractals}, Trans. Amer. Math. Soc., 373 (2020),
  pp.~1619--1652.

\bibitem{HanPina2021}
{\sc B.~Han and A.~Pinamonti}, {\em On the asymptotic behaviour of the
  fractional {S}obolev seminorms in metric measure spaces:
  {B}ourgain-{B}rezis-{M}ironescu's theorem revisited}, arXiv:2110.05980,
  (2021).

\bibitem{HermanPeironeStrichartz.2004.PA125}
{\sc P.~Herman, R.~Peirone, and R.~Strichartz}, {\em $p$-energy and
  $p$-harmonic functions on {S}ierpinski gasket type fractals}, Potential
  Anal., 20 (2004), pp.~125--148.

\bibitem{KajinoMurugan.2020On}
{\sc N.~Kajino and M.~Murugan}, {\em On singularity of energy measures for
  symmetric diffusions with full off-diagonal heat kernel estimates}, Ann.
  Probab., 48 (2020), pp.~2920--2951.

\bibitem{Kigami.1993.TAMS721}
{\sc J.~Kigami}, {\em Harmonic calculus on p.c.f. self-similar sets}, Trans.
  Amer. Math. Soc., {335} (1993), pp.~721--755.

\bibitem{Kigami.2001.}
\leavevmode\vrule height 2pt depth -1.6pt width 23pt, {\em Analysis on
  Fractals}, Cambridge Univ. Press, 2001.

\bibitem{Kigami2022penergy}
{\sc J.~Kigami}, {\em Conductive homogeneity of compact metric spaces and
  construction of p-energy}, arXiv:2109.08335,  (2022).

\bibitem{Kumagai.1993.PTaRF205}
{\sc T.~Kumagai}, {\em Estimates of the transition densities for brownian
  motion on nested fractals}, Probab. Theory and Related Fields, {96} (1993),
  pp.~205--224.

\bibitem{LiYau.1986.AM153}
{\sc P.~Li and S.~Yau}, {\em On the parabolic kernel of the {S}chr\"{o}dinger
  operator}, Acta Math., {156} (1986), pp.~153--201.

\bibitem{Pietruska-Paluba.2008}
{\sc K.~Pietruska-Pa{\l}uba}, {\em Limiting behaviour of {D}irichlet forms for
  stable processes on metric spaces}, Bull. Pol. Acad. Sci. Math., 56 (2008),
  pp.~257--266.

\bibitem{Pietruska-Paluba.2010.}
\leavevmode\vrule height 2pt depth -1.6pt width 23pt, {\em Heat kernel
  characterisation of {B}esov-{L}ipschitz spaces on metric measure spaces},
  Manuscripta Math., 131 (2010), pp.~199--214.

\bibitem{Saloff-Coste.1992.IMRN27}
{\sc L.~Saloff-Coste}, {\em A note on {P}oincar\'{e}, {S}obolev, and {H}arnack
  inequalities}, Internat. Math. Res. Notices,  (1992), pp.~27--38.

\bibitem{Shimuzu.2022}
{\sc R.~Shimizu}, {\em Construction of p-energy and associated energy measures
  on the {S}ierpinski carpet}, ArXiv:2110.13902v1,  (2022).

\bibitem{StrichartzWong2004}
{\sc R.~Strichartz and C.~Wong}, {\em The p-laplacian on the sierpinski
  gasket}, Nonlinearity, 17 (2004), pp.~595--616.

\bibitem{Strichartz.1998.CJM}
{\sc R.~S. Strichartz}, {\em Fractals in the large}, Canad. J. Math., 50
  (1998), pp.~638--657.

\bibitem{StrichartzTeplyaev2012}
{\sc R.~S. Strichartz and A.~Teplyaev}, {\em Spectral analysis on infinite
  {S}ierpi\'{n}ski fractafolds}, J. Anal. Math., 116 (2012), pp.~255--297.

\bibitem{Yang.2018.PA}
{\sc M.~Yang}, {\em Equivalent semi-norms of non-local {D}irichlet forms on the
  {S}ierpi\'{n}ski gasket and applications}, Potential Anal., 49 (2018),
  pp.~287--308.

\bibitem{Yang2022MZ}
{\sc M.~Yang}, {\em On the domains of {D}irichlet forms on metric measure
  spaces}, Math. Z., 301 (2022), pp.~2129--2154.

\end{thebibliography}

\end{document}
