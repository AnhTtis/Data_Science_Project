%\usepackage[final]{showkeys}
%\definecolor{llightgray}{rgb}{0.9,0.9,0.9}
%\newenvironment{remark}[1][Remark]{\bigskip\noindent\textbf{#1. }\it}{\rm\medskip}
%\newtheorem{remarks}[theorem]{Remarks}
%\newtheorem{example}[theorem]{Example}
%\newenvironment{definition}[1][Definition]{\bigskip\noindent\textbf{#1. }\it}{\rm\bigskip}
%\newenvironment{definitions}[1][Definition]{\bigskip\noindent\textbf{#1. }\it}{\rm}
%\usepackage{dutchcal}


\documentclass[reqno,12pt]{amsart}
%%%%%%%%%%%%%%%%%%%%%%%%%%%%%%%%%%%%%%%%%%%%%%%%%%%%%%%%%%%%%%%%%%%%%%%%%%%%%%%%%%%%%%%%%%%%%%%%%%%%%%%%%%%%%%%%%%%%%%%%%%%%%%%%%%%%%%%%%%%%%%%%%%%%%%%%%%%%%%%%%%%%%%%%%%%%%%%%%%%%%%%%%%%%%%%%%%%%%%%%%%%%%%%%%%%%%%%%%%%%%%%%%%%%%%%%%%%%%%%%%%%%%%%%%%%%
\usepackage{eurosym}
\usepackage{amssymb}
\usepackage{mathrsfs}
\usepackage{txfonts}
\usepackage{amsfonts}
\usepackage{amstext}
\usepackage{amssymb}
\usepackage{amsmath}
\usepackage{graphicx}
\usepackage{hyperref}
\usepackage{url}
\usepackage{amssymb}

\setcounter{MaxMatrixCols}{10}
%TCIDATA{OutputFilter=LATEX.DLL}
%TCIDATA{Version=5.50.0.2960}
%TCIDATA{<META NAME="SaveForMode" CONTENT="1">}
%TCIDATA{BibliographyScheme=BibTeX}
%TCIDATA{LastRevised=Monday, February 17, 2025 20:25:29}
%TCIDATA{<META NAME="GraphicsSave" CONTENT="32">}
%TCIDATA{Language=American English}
%TCIDATA{CSTFile=amsart.cst}
%TCIDATA{<META NAME="PrintViewPercent" CONTENT="100">}

\hypersetup{
	colorlinks   = true,
	citecolor    = blue,
	linkcolor    = blue,
	urlcolor     = blue
}
\allowdisplaybreaks
\newtheorem{theorem}{Theorem}[section]
\newtheorem{proposition}[theorem]{Proposition}
\newtheorem{lemma}[theorem]{Lemma}
\newtheorem{definition}[theorem]{Definition}
\newtheorem{corollary}[theorem]{Corollary}
\newtheorem{remark}[theorem]{Remark}
\renewcommand{\theequation}{\thesection.\arabic{equation}}
\newenvironment{notation}{\smallskip{\sc Notation.}\rm}{\smallskip}
\newenvironment{acknowledgement}{\smallskip{\sc Acknowledgement.}\rm}{\smallskip}
\DeclareMathOperator*{\einf}{ess{\,}inf}
\DeclareMathOperator*{\esup}{ess{\,}sup}
\DeclareMathOperator*{\osc}{osc}
\newcommand{\mint}{\ \ \!\!\!-\!\!\!\:\!\!\!\!\! \int}
\numberwithin{equation}{section}
\newcounter{counterConstant}
\newcommand{\const}[1]{
	\addtocounter{counterConstant}{1}
	\edef#1{\arabic{counterConstant}}
}
\input{tcilatex}

% A4
\setlength{\textheight}{24 cm}
\setlength{\textwidth}{16.5 cm}
\setlength{\topmargin}{-1cm}
\setlength{\oddsidemargin}{0.0 cm}
\setlength{\evensidemargin}{0.0 cm}

%A4
%\setlength{\textheight}{24 cm}
%\setlength{\textwidth}{15 cm}
%\setlength{\topmargin}{0cm}
%\setlength{\oddsidemargin}{1.0 cm}
%\setlength{\evensidemargin}{1.0 cm}

%Letter
%\setlength{\textheight}{23 cm}
%\setlength{\textwidth}{16 cm}
%\setlength{\topmargin}{-1cm}
%\setlength{\oddsidemargin}{-0.0 cm}
%\setlength{\evensidemargin}{-0.0 cm}

%\def\text{\mathrm}
%\def\text{\textrm}
\def\RM{\rm}
\def\FTN{\footnotesize}
\def\NSZ{\normalsize}
\def\stackunder#1#2{\mathrel{\mathop{#2}\limits_{#1}}}%
%\def\marginpar#1{}
%\def\emptyspace#1{}
%\def\endproofof#1{}
\def\qed{\hfill$\square$\par}
%\newenvironment{proof}[1][Proof]{\textbf{#1.} }{\ \rule{0.5em}{0.5em}}
%\newenvironment{proof}[1][Proof]{\textbf{#1.} }\qed

\def\limfunc#1{\mathop{\mathrm{#1}}}
\def\func#1{\mathop{\mathrm{#1}}\nolimits}
\def\diint{\mathop{\int\int}}

\def\dint{\displaystyle\int}
\def\dsum{\displaystyle\sum}
\def\dbigcap{\displaystyle\bigcap}
\def\dbigoplus{\displaystyle\bigoplus}
\def\dbigcup{\displaystyle\bigcup}
\def\dprod{\displaystyle\prod}
\def\tbigcup{\mathop{\textstyle \bigcup }}
\def\tbigcap{\mathop{\textstyle \bigcap }}


\def\Xint#1{\mathchoice
{\XXint\displaystyle\textstyle{#1}}%
{\XXint\textstyle\scriptstyle{#1}}%
{\XXint\scriptstyle\scriptscriptstyle{#1}}%
{\XXint\scriptscriptstyle\scriptscriptstyle{#1}}%
\!\int}
\def\XXint#1#2#3{{\setbox0=\hbox{$#1{#2#3}{\int}$ }
\vcenter{\hbox{$#2#3$ }}\kern-.6\wd0}}

\def\oint{\Xint-}
\def\toint{\Xint-}
\def\fint{\Xint-}

%\def\subjclass#1{\par{\footnotesize 2000 {\sl Mathematics Subject Classification.} #1}}
%\def\keywords#1{\par{\footnotesize {\sl Keywords and phrases.} #1}}

\def\beginsel#1#2#3{\vfil\eject\setcounter{page}{#1}\setcounter{section}{#2}\setcounter{equation}{#3}}
%example: \beginsel{1}{0}{0}
\def\endsel{\vfil\eject}

%% label
\def\Qlb#1{#1}

%% caption
\def\Qcb#1{#1}

\def\FRAME#1#2#3#4#5#6#7#8
{
 \begin{figure}[ptbh]
 \begin{center}
 \includegraphics[height=#3]{#7}
 \caption{#5}
 \label{#6}
 \end{center}
 \end{figure}
}

\def\enddoc{\end{document}}


\begin{document}
\title[]{Heat kernel-based $p$-energy norms on metric measure spaces}

\author[Gao]{Jin Gao}
\address{Department of Mathematics, Hangzhou Normal University, Hangzhou
	310036, China.}
\email{gaojin@hznu.edu.cn}
\author[Yu]{Zhenyu Yu}
\address{Department of Mathematics, College of Science, National University
	of Defense Technology, Changsha 410073, China.}
\email{yuzy23@nudt.edu.cn}
\author[Zhang]{Junda Zhang}
\address{School of Mathematics, South China University of Technology,
	Guangzhou 510641, China.}
\email{summerfish@scut.edu.cn}
\date{}

\begin{abstract}
We investigate heat kernel-based and other $p$-energy norms ($1<p<\infty$) on bounded and
unbounded metric measure spaces, in particular, on nested fractals and their
blowups. With the weak-monotonicity properties for these norms,
we generalise the celebrated Bourgain-Brezis-Mironescu (BBM)
type characterization for $p\neq2$. When there admits a heat kernel
satisfying the two-sided estimates, we establish the equivalence of
various $p$-energy norms and weak-monotonicity properties, and show that
these weak-monotonicity properties hold when $p=2$ (in the case of Dirichlet
form). Our paper's key results concern the equivalence and verification of various weak-monotonicity
properties on fractals. Consequently, many classical results on $p$-energy norms
hold on nested fractals and their
blowups, including the BBM type characterization
and Gagliardo-Nirenberg inequality.
\end{abstract}


\subjclass[2010]{28A80, 46E30, 46E35}
\keywords{Heat kernel, weak-monotonicity property, $p$-energy, Besov norm,
	nested fractals.}
\maketitle
\tableofcontents

%We focus on heat kernel-based $p$-energy norms ($1<p<\infty$) on bounded and
%unbounded metric measure spaces, in particular, weak-monotonicity properties
%for different types of energies. Such properties are key to related studies,
%under which we generalise the convergence result of
%Bourgain-Brezis-Mironescu (BBM) for $p\neq2$. We establish the equivalence
%of various $p$-energy norms and weak-monotonicity properties when there
%admits a heat kernel satisfying the two-sided estimates.
%Using these equivalences, we verify various weak-monotonicity properties on nested fractals and their blowups.
%Immediate consequences are that, many classical
%results on $p$-energy norms hold for such bounded and unbounded fractals,
%including the BBM convergence and Gagliardo-Nirenberg inequality.

\section{Introduction}
\subsection{Historical background and motivation}

Recently, there has been considerable development on the $p$-energy defined on
fractals and general metric measure spaces (initiated by
\cite{HermanPeironeStrichartz.2004.PA125}). For a smooth Euclidean domain $D$%
, the $p$-energy is simply defined as $\int_{D}|\nabla u(x)|^{p}dx$, but for many
fractals or metric measure spaces, it is not easy to define proper gradient
structures to characterize the smoothness of certain `core' functions arising
naturally from analysis. Previous constructions of $p$-energy ($1<p<\infty $%
) are based on the graph-approximation to the underlying space, including
p.c.f. self-similar sets by Cao, Gu and Qiu \cite{Caoqiugu2022adv}, the
Sierpi\'{n}ski carpet by Shimizu \cite{Shimuzu.2022} and more general
fractal spaces by Kigami \cite{Kigami2022penergy}.

In this paper, we consider another way to define $p$-energy norms by
equivalent Besov-type norms, which naturally generalises the energy of
Dirichlet forms when $p=2$, and the graph-approximation does not appear in
the definition. The heat kernel-based $p$-energy norms are introduced by
K.~Pietruska-Pa{\l }uba in \cite{Pietruska-Paluba.2010.}, and later heat
semigroup-based norms are studied by Alonso Ruiz, Baudoin et al. for $1\leq
p<\infty $ in \cite%
{AlonsoBaudoinchen2020JFA,AlonsoBaudoinchen2020CVPDE,AlonsoBaudoinchen2021CVPDE}%
, where they focus on generalising classical analysis results including
Sobolev embedding, isoperimetric inequalities and the class of bounded
variation functions. It is possible to use heat kernel-based $p$%
-energy norms to construct equivalent (or exactly the same) $p$-energy given
by the graph-approximation approach, to satisfy further restrictions like
convexity or self-similarity for self-similar sets, see for example \cite%
{Caoqiugu2022adv,GaoYuZhang2022PA}.

Recall the celebrated `Bourgain-Brezis-Mironescu (BBM) convergence' in \cite[%
Corollary 2]{BourgainBrezisMironescu.2001.439}:
\begin{equation}
\lim_{\sigma \uparrow 1}(1-\sigma )\int_{D}\int_{D}\frac{|u(x)-u(y)|^{p}}{%
|x-y|^{n+p\sigma }}dxdy=C_{n,p}\int_{D}|\nabla u(x)|^{p}dx \ \ \
(1<p<\infty),  \label{BBM}
\end{equation}%
where $D$ is a smooth domain in $\mathbb{R}^{n}$, which states that
multiplying by a scaling factor $1-\sigma $, the fractional Gagliardo
semi-norms of a function converge to the first-order Sobolev semi-norm as $%
\sigma \rightarrow 1$. The BBM type characterization is the
convergence of Besov semi-norms $(\sigma ^{\ast }-\sigma )B_{p,p}^{\sigma }$
to $B_{p,\infty }^{\sigma ^{\ast }}$ on metric measure spaces, where $\sigma
^{\ast }$ is the critical exponent. Several papers have studied this, for
example, on fractals with `property (E)' \cite[Theorem 1.6]{GaoYuZhang2022PA}%
, and on spaces supporting a $p$-Poincar\'e inequality \cite{DS19,Munnier15}.%
%, for example Di Marino-Squassina and Munnier

We establish BBM type characterization on both bounded and
unbounded metric measure spaces under appropriate weak-monotonicity
properties (to replace property (E) on fractals). We utilize the concept of
property (KE) introduced in \cite[Definition 6.7]{AlonsoBaudoinchen2020JFA}
and (NE) defined in \cite[Definition 4.5]{BaudoinLecture2022} (all termed $%
P(p,\alpha )$ therein), and use (VE) for bounded and unbounded fractals
based on \cite[Definition 3.1]{GaoYuZhang2022PA}, where the letters `E'
stands for energy control and `K',`N',`V' represent (heat) kernel, (Besov)
norm, vertex respectively. Additionally, we explore their slightly weaker
variants by replacing `sup' with `limsup'.

These weak-monotonicity properties are certain energy-control conditions,
which is essentially required and important in all the studies related to $p$%
-energy (norms) mentioned above, to guarantee `some level of $L^p$
infinitesimal regularity and global controlled $L^p$ geometry' as \cite%
{BaudoinLecture2022} stated. In fractal analysis, weak-monotonicity appear
in the form of $p$-resistance estimate in \cite%
{Caoqiugu2022adv,Shimuzu.2022,Kigami2022penergy}, which is not easy to
obtain. Verifying the weak-monotonicity properties (KE), (NE) and (VE) is a
main contribution in our paper for certain fractal-type spaces, like nested
fractals and their blow-ups.

To do this, we establish the equivalence of (KE) and (NE) on metric measure
spaces, the equivalence of (NE) and (VE) on certain fractals (including
nested fractals), and then verify a relatively easier condition (VE) using
the arguments in a main theorem of \cite{Caoqiugu2022adv}. Such properties
are not easy to examine when $p\neq 2$ in general (but property (KE) holds
automatically when $p=2$). Alonso Ruiz, Baudoin et al. examined property
(KE) in the case $p=1$ using weak Bakry-Emery type hypothesis and $p\neq 2$
for spaces with the same critical exponent as Euclidean smooth manifolds
(see for example \cite[Lemma 4.13]{AlonsoBaudoinchen2021CVPDE}).
But for fractal-type spaces, different ideas are required, since the
critical exponents are different from manifolds (as in (\ref{BBM}), $%
\sigma^*=1$ for smooth manifolds; but even when $p=2$, $\sigma^*=\log15/%
\log9 $ on the Vicsek set \cite{BC23} and $\sigma^*=\log5/\log4$ on the
Sierpi\'nski gasket \cite{BP88}) and (bi-)linearity is missing.

%Alonso Ruiz,Kigami.2001.,
%Baudoin et al. established two analogues of classical analysis results under
%property (KE) or (NE) for $1<p<\infty $, that is, the Sobolev-type
%inequality in \cite[Theorem 1.1]{AlonsoBaudoinchen2020JFA} and the full
%scale of Gagliardo-Nirenberg inequality introduced by \cite[Theorem 4.3 and
%Theorem 4.4]{BaudoinLecture2022}. The latter inequality only holds for
%unbounded spaces in \cite{BaudoinLecture2022}, and this motivates us to
%study unbounded fractals.  Recently,Yang in \cite[Theorem 2.8]{Yang23}, Murugan and Shimizu in \cite[Theorem 1.4]{MuruganShimizu23} prove (NE) respectively for generalized Sierpi\'nski carpets  in the case $p > \dim_{\text{ARC}}$ where $\dim_{\text{ARC}}$ is the Ahlfors regular conformal dimension. \underline{(INCORRECT)}

\subsection{Preliminaries}

\label{secPre}

\subsubsection{Basic definitions}

Let $(M,d,\mu )$ be a metric measure space,
that is, $(M,d)$\ is a locally
compact, separable metric space and $\mu $ is a Radon measure with full
support on $M$. Fix a value (which could be infinite)
\begin{equation*}
R_{0}\in (0,\mathrm{diam}(M)]
\end{equation*}
that will be used for localization throughout the paper, where
\begin{equation*}
\mathrm{diam}(M):=\sup \{d(x,y):x,y\in M\}\in (0,\infty]
\end{equation*}
is infinite if $M$ is unbounded and is finite if $M$ is bounded.

The measure $\mu$ is assumed to be Borel regular with the following
positivity and \emph{volume doubling property}: there exists a constant $%
C_{d}>0$ such that for every $x\in M$, $0<r<\infty $,
\begin{equation}
0<\mu (B(x,2r))\leq C_{d}\mu (B(x,r))<\infty,  \label{VD1}
\end{equation}%
where $B(x,r):=\{y\in M:d(x,y)<r\}$ denotes the open metric ball centered at
$x\in M$ with radius $r>0$. Denote
\begin{equation*}
V(x,r):=\mu (B(x,r)).
\end{equation*}%
It is known from \cite[Proposition 5.1]{GrigoryanHu.2014.MMJ505} that if %
\eqref{VD1} holds, then there exists $\alpha _{1}>0$ such that
\begin{equation}
\frac{V(x,R)}{V(x,r)}\leq C_{d}\left( \frac{R}{r}\right) ^{\alpha _{1}},
\label{VD2}
\end{equation}%
for all $x\in M$, $0<r\leq R<\infty$.

For $\alpha>0$, we say that $\mu $ is \emph{$\alpha $-regular} if there exists a constant $%
C\geq 1$ such that for all $0<r<R_{0}$,
\begin{equation}
C^{-1}r^{\alpha }\leq V(x,r)\leq Cr^{\alpha }.  \label{alpha_r}
\end{equation}%
We remark that when $\mu $ is $\alpha $-regular, then (\ref{VD2}) holds with
$\alpha _{1}=\alpha $.

A family $\{p_{t}\}_{t>0}$ of non-negative measurable functions on $M \times
M$ is a \emph{heat kernel} if it satisfies: (\romannumeral 1) $p_t(x,y) =
p_t(y,x)$, (\romannumeral 2) $\int_M p_t(x,y)d\mu(y) \leq 1$, (\romannumeral %
3) $p_{s+t}(x,y)=\int_{M}p_{s}(x,z)p_{t}(z,y)d\mu (z)$, and (\romannumeral %
4) for any $f\in L^{2}(M,\mu )$, $\int_{M}p_{t}(x,y)f(y)d\mu (y)\rightarrow
f(x)$ \ as$~~t\downarrow 0$, strongly in$~~L^{2}(M,\mu )$ for $\mu $-almost
all $x,y\in M$ and $s,t>0$.

For some parameter $\beta ^{\ast }>1$, we consider the following two
conditions on the heat kernel.

We say that a heat kernel $\{p_{t}\}_{t>0}$ satisfies the upper estimate
(UHE), if for all $t\in (0,R_{0}^{\beta ^{\ast }}) $ and $\mu $-almost all $%
x,y\in M$,
\begin{equation}
p_{t}(x,y)\leq \frac{c_{3}}{V(x,t^{1/\beta ^{\ast }})}\exp \left(
-c_{4}\left( \frac{d(x,y)}{t^{1/\beta ^{\ast }}}\right) ^{\frac{\beta ^{\ast
}}{\beta ^{\ast }-1}}\right).  \label{UHE}
\end{equation}

We say that a heat kernel $\{p_{t}\}_{t>0}$ satisfies the lower estimate
(LHE), if for all $t\in (0,R_{0}^{\beta ^{\ast }}) $ and $\mu $-almost all $%
x,y\in M$,
\begin{equation}
p_{t}(x,y)\geq \frac{c_{1}}{V(x,t^{1/\beta ^{\ast }})}\exp \left(
-c_{2}\left( \frac{d(x,y)}{t^{1/\beta ^{\ast }}}\right) ^{\frac{\beta ^{\ast
}}{\beta ^{\ast }-1}}\right).  \label{LHE}
\end{equation}

%\textcolor{red}{It is easy to see that $(\mathrm{LHE})\Rightarrow (\mathrm{NLE})$. If the metric $d$ is furthermore assumed to satisfy the chain condition, then the conjunction
%	of $(\mathrm{LHE})$ and $(\mathrm{NLE})$ is equivalent to the two-sided estimates $(\mathrm{LHE})$ and $(\mathrm{UHE})$, see \cite[Section 6]{GrigoryanTelcs.2012.AoP1212}.}
%\textbf{[[delete]]}We say that a \emph{heat kernel} $\{p_{t}\}_{t>0}$
%satisfies the\emph{\ two-sided} heat kernel estimates, if for all $t\in
%(0,R_{0}^{\beta ^{\ast }}) $ and $\mu $-almost all $x,y\in M$ :
%\begin{equation}
%\frac{c_{1}}{V(x,t^{1/\beta ^{\ast }})}\exp \left( -c_{2}\left( \frac{d(x,y)%
%}{t^{1/\beta ^{\ast }}}\right) ^{\frac{\beta ^{\ast }}{\beta ^{\ast }-1}%
%}\right) \leq p_{t}(x,y)\leq \frac{c_{3}}{V(x,t^{1/\beta ^{\ast }})}\exp
%\left( -c_{4}\left( \frac{d(x,y)}{t^{1/\beta ^{\ast }}}\right) ^{\frac{\beta
%^{\ast }}{\beta ^{\ast }-1}}\right) ,  \label{hk}
%\end{equation}%
%where $\beta ^{\ast }>1$ is a fixed number. It is known that, when $R_0=%
%\mathrm{diam}(M)$, the two-sided estimates \eqref{hk} are equivalent to the
%conjunction of the chain condition (see \cite[Definition 3.4]%
%{GrigoryanHuLau.2003.TAMS2065}) the volume doubling property, the
%scale-invariant elliptic Harnack inequality and the mean exit time estimate
%on metric measure spaces (see \cite[Corollary 1.8]{Murugan.2020.JFA}, \cite[%
%Corollary 7.6]{GrigoryanTelcs.2012.AoP1212}).

Conditions (UHE) and (LHE) hold for many classical cases. The classical
Gauss-Weierstrass function
\begin{equation}
p_{t}(x,y)=\frac{1}{(4\pi t)^{n/2}}\exp \left( -\frac{|x-y|^{2}}{4t}\right) ,
\label{Gau-Wei}
\end{equation}%
is a heat kernel for the standard Brownian motion in $\mathbb{R}^{n}$. Li and Yau have shown in
\cite{LiYau.1986.AM153} that, for a complete Riemannian manifold with
non-negative Ricci curvature, two-sided estimates (UHE) and (LHE) hold with $%
\beta ^{\ast }=2$:
\begin{equation}
p_{t}(x,y)\asymp \frac{C}{V(x,\sqrt{t})}\exp \left( -\frac{d(x,y)^{2}}{ct}%
\right) ,  \label{LiYau}
\end{equation}%
where $d$ is the geodesic metric and $V(x,r)$ is the Riemannian volume. The
following two-sided heat kernel estimates
\begin{equation}
p_{t}(x,y)\asymp \frac{C}{t^{\alpha /\beta ^{\ast }}}\exp \left( -c\left(
\frac{d(x,y)}{t^{1/\beta ^{\ast }}}\right) ^{\frac{\beta ^{\ast }}{\beta
^{\ast }-1}}\right)  \label{hk_F}
\end{equation}%
also hold on many fractal spaces with different $\beta ^{\ast }$, such as the Sierpi\'{n}ski carpet and
nested fractals \cite%
{BarlowBass.1992.PTRF307,BarlowBass.1999.CJM673,FitzsimmonsmHamblyKumagai.1994.CMP595,Kumagai.1993.PTaRF205}%
.
%In fact, estimates \eqref{hk_F} imply that $\mu $ is $\alpha $-regular (see for example \cite[Theorem 3.2]{GrigoryanHuLau.2003.TAMS2065}). In addition, two-sided estimates \eqref{hk} and the volume doubling property hold on scale irregular Sierpi\'{n}ski gaskets (see \cite%{BarlowHambaly.1997.AIHS}, \cite[Theorem 5.1]{KajinoMurugan.2020On}).

\subsubsection{Besov and Korevaar-Schoen norms}
From now on we fix $p\in (1,\infty)$.
As in \cite{BaudoinLecture2022}, for $\sigma >0$, define semi-norms $%
[u]_{B_{p,\infty }^{\sigma }}$ for $u\in L^{p}(M,\mu )$ by
\begin{equation}
\lbrack u]_{B_{p,\infty }^{\sigma }}^{p}:=\sup_{r\in (0,R_{0})}r^{-p\sigma
}\int_{M}\frac{1}{V(x,r)}\int_{B(x,r)}|u(x)-u(y)|^{p}d\mu (y)d\mu (x),
\label{pi}
\end{equation}%
and define
\begin{equation*}
B_{p,\infty }^{\sigma }:=B_{p,\infty }^{\sigma }(M)=\{u\in L^{p}(M,\mu
):[u]^p_{B_{p,\infty }^{\sigma }}<\infty \}.
\end{equation*}%
Naturally, define semi-norms $[u]_{B_{p,p}^{\sigma }}$ by
\begin{equation}
\lbrack u]_{B_{p,p}^{\sigma }}^{p}:=\int_{0}^{R_{0}}r^{-p\sigma }\left(
\int_{M}\frac{1}{V(x,r)}\int_{B(x,r)}|u(x)-u(y)|^{p}d\mu (y)d\mu (x)\right)
\frac{dr}{r},  \label{pp}
\end{equation}%
and define
\begin{equation*}
B_{p,p}^{\sigma }:=B_{p,p}^{\sigma }(M)=\{u\in L^{p}(M,\mu
):[u]^p_{B_{p,p}^{\sigma }}<\infty \}.
\end{equation*}

The space $B_{p,\infty }^{\sigma }$ coincides with the Korevaar-Schoen space
$KS_{p,\infty }^{\sigma }$ in \cite[Section 4.2]{BaudoinLecture2022} with
the following different semi-norm (switch `sup' to `limsup')
\begin{equation}
\Vert u\Vert^p _{KS_{p,\infty }^{\sigma }}=\limsup_{r\rightarrow 0}\int_{M}%
\frac{1}{V(x,r)}\int_{B(x,r )}\frac{|u(x)-u(y)|^{p}}{r ^{p\sigma }}d\mu
(y)d\mu (x)<\infty.  \label{ks_energy1}
\end{equation}

In our paper, we define the \emph{critical exponent of $(M,d,\mu )$} by%
\begin{equation*}
\sigma _{p}^{\#}=\sup \{\sigma >0:B_{p,\infty }^{\sigma }\text{ contains
non-constant functions}\}.
\end{equation*}%
In many related studies the critical exponent is defined by
\begin{equation*}
\sigma _{p}^{\ast }=\sup \{\sigma >0:B_{p,\infty }^{\sigma }\text{ is dense
in }L^{p}(M,\mu )\}.
\end{equation*}

%and we will meet these two exponents in the last section.
Clearly, $\sigma _{p}^{\ast }\leq \sigma _{p}^{\#}$. Whenever the critical
exponents appear in our paper, we assume that they are finite. It is
known by \cite[Theorem 4.2]{BaudoinLecture2022} that when the chain
condition holds, then
\begin{equation*}
\sigma _{p}^{\#}\leq 1+\frac{\alpha_1}{p}<\infty .
\end{equation*}

\subsubsection{Heat kernel-based $p$-energy norm}

It is well-known that, a heat kernel $\{p_{t}\}_{t>0}$ can induce a
Dirichlet form on $L^{2}(M,\mu )$ given by
\begin{align*}
\mathcal{E}(u,u)& =\lim_{t\rightarrow 0^{+}}\mathcal{E}_{t}(u,u),\ u\in
L^{2}(M,\mu) \\
\mathcal{D}(\mathcal{E})& =\{u\in L^{2}(M,\mu ):\mathcal{E}(u,u)<\infty \}
\end{align*}%
(see \cite[Section 1.3]{FukushimaOshimaTakeda.2011.489}), where
\begin{equation}
\mathcal{E}_{t}(u,u):=\frac{1}{2t}\int_{M}\int_{M}|u(x)-u(y)|^{2}p_{t}(x,y)d%
\mu (y)d\mu (x).  \label{E_t}
\end{equation}%
The limit exists, since for any given $u\in L^{2}(M,\mu )$, $
\mathcal{E}_{t}(u,u)$ decreases in $t$ by using the spectral theorem.

In view of this, we adapt the heat kernel-based norms in \cite%
{AlonsoBaudoinchen2020JFA} and \cite{Pietruska-Paluba.2010.} to define the $p$%
-energy norm on $L^{p}(M,\mu )$ for a heat kernel $\{p_{t}\}_{t>0}$ as
\begin{equation*}
E_{p,\infty }^{\sigma }(u):=\sup_{t\in (0,R_{0}^{\beta ^{\ast }})}\frac{1}{%
t^{{p\sigma }/{\beta ^{\ast }}}}\int_{M}\int_{M}|u(x)-u(y)|^{p}p_{t}(x,y)d%
\mu (y)d\mu (x),
\end{equation*}%
with its domain
\begin{equation*}
\mathcal{D}(E_{p,\infty }^{\sigma }):=\{u\in L^{p}(M,\mu ):E_{p,\infty
}^{\sigma }(u)<\infty \}.
\end{equation*}%
This is a natural way to extend the Dirichlet form case for $p=2$ to $p\in
(1,\infty)$, and we will call them heat kernel-based $p$-energy norms. The
heat kernel-based $p$-energy norm $E_{p,\infty }^{\sigma_{p}^{\#}}$ is
equivalent to the $p$-energy on certain fractals (see Section \ref{sec5}).

To extend the BBM type characterization from $p=2$ to $%
1<p<\infty$, it is quite natural to define
\begin{equation*}
E_{p,p}^{\sigma }(u):=\int_{0}^{R_{0}^{\beta ^{\ast }}}\frac{1}{t^{{p\sigma }%
/{\beta ^{\ast }}}}\left( \int_{M}\int_{M}|u(x)-u(y)|^{p}p_{t}(x,y)d\mu
(x)d\mu (y)\right) \frac{dt}{t},
\end{equation*}%
with domain
\begin{equation*}
\mathcal{D}(E_{p,p}^{\sigma }):=\{u\in L^{p}(M,\mu ):E_{p,p}^{\sigma
}(u)<\infty \}.
\end{equation*}
It is known that (see for example \cite{Pietruska-Paluba.2008}), when the
two-sided heat kernel estimates hold, the proper scaling of $E_{2,2}^{\sigma
} $ converges to $E_{2,\infty }^{\sigma _{2}^{\#}}$ thanks to the fact that $%
E_{2,2}^{\sigma }$ is equivalent to the Dirichlet form defined by the
subordinated heat kernel. Additionally, we explore the convergence of heat
kernel-based $p$-energy norms $E_{p,p}^{\sigma }$ to $E_{p,\infty }^{\sigma
_{p}^{\#}}$. However, $E_{p,p}^{\sigma }$ does not seem to be the $p $%
-energy norm of the subordinated heat kernel, so we need weak-monotonicity
properties to replace the subordination technique for $p=2$ (see Theorem \ref{thm4.1}).

\subsection{Weak-monotonicity properties and main results}

We introduce some notions that will be used in defining weak-monotonicity properties and rewrite the above norms in the following way. For $u\in L^{p}(M,\mu )$ and $\sigma ,t>0$, denote
\begin{equation}
\Psi _{u}^{\sigma }(t)=\frac{1}{t^{{p\sigma }/{\beta ^{\ast }}}}%
\int_{M}\int_{M}|u(x)-u(y)|^{p}p_{t}(x,y)d\mu (y)d\mu (x),  \label{psi_u}
\end{equation}%
then
\begin{equation}
E_{p,\infty }^{\sigma }(u)=\sup_{t\in (0,R_{0}^{\beta ^{\ast }})}\Psi
_{u}^{\sigma }(t),\ E_{p,p}^{\sigma }(u)=\int_{0}^{R_{0}^{\beta ^{\ast
}}}\Psi _{u}^{\sigma }(t)\frac{dt}{t}.  \label{EEE}
\end{equation}

Similarly, for Besov norms, denote
\begin{equation}
\Phi _{u}^{\sigma }(r)=r^{-p\sigma }\int_{M}\frac{1}{V(x,r)}%
\int_{B(x,r)}|u(x)-u(y)|^{p}d\mu (y)d\mu (x),  \label{phi_u}
\end{equation}%
then
\begin{equation*}
\lbrack u]_{B_{p,\infty }^{\sigma }}^{p}=\sup_{r\in (0,R_{0})}\Phi
_{u}^{\sigma }(r),\ [u]_{B_{p,p}^{\sigma }}^{p}=\int_{0}^{R_{0}}\Phi
_{u}^{\sigma }(r)\frac{dr}{r},\ \Vert u\Vert _{KS_{p,\infty }^{\sigma
}}^{p}=\limsup_{r\rightarrow 0}\Phi _{u}^{\sigma }(r).
\end{equation*}

\begin{definition}
We say that a metric measure space $(M,d,\mu )$ with heat kernel $%
\{p_{t}\}_{t>0}$ satisfies property ($\normalfont{KE}$) with $\sigma>0$, if
there exists a constant $C>0$ such that for all $u\in \mathcal{D}%
(E_{p,\infty }^{\sigma })$,
\begin{equation}  \label{KE}
\sup_{t\in (0,R_{0}^{\beta ^{\ast }})}\Psi _{u}^{\sigma }(t)\leq
C\liminf_{t\rightarrow 0}\Psi _{u}^{\sigma }(t),  \tag{\normalfont{KE}}
\end{equation}
where $\Psi _{u}^{\sigma }(t)$ is defined as in \eqref{psi_u}. We further
say that property ($\widetilde{KE}$) is satisfied with $\sigma >0$, if there
exists a constant $C>0$ such that for all $u\in \mathcal{D}(E_{p,\infty
}^{\sigma }) $,
\begin{equation}  \label{wKE}
\limsup_{t\rightarrow 0}\Psi _{u}^{\sigma }(t)\leq C\liminf_{t\rightarrow
0}\Psi _{u}^{\sigma }(t),  \tag{$\widetilde{KE}$}
\end{equation}
\end{definition}

\begin{remark}
\label{rk1}When $p=2$, property (KE) automatically holds
with $\sigma =\beta ^{\ast }/2$ as $\mathcal{E}_{t}(u,u)$ decrease in $t$.
\end{remark}

\begin{definition}
We say that a metric measure space $(M,d,\mu )$ satisfies property ($%
\normalfont{NE}$) with $\sigma >0$ if there exists $C>0$ such that for all $%
u\in B_{p,\infty }^{\sigma }$,
\begin{equation}  \label{NE1}
\sup_{r\in (0,R_{0})}\Phi _{u}^{\sigma }(r)\leq C\liminf_{r\rightarrow
0}\Phi _{u}^{\sigma }(r).  \tag{\normalfont{NE}}
\end{equation}%
In addition, we say that property ($\widetilde{NE}$) is satisfied with $%
\sigma >0$, if there exists $C>0$ such that for all $u\in B_{p,\infty
}^{\sigma }$,
\begin{equation}  \label{wNE}
\limsup_{r\rightarrow 0}\Phi _{u}^{\sigma }(r)\leq C\liminf_{r\rightarrow
0}\Phi _{u}^{\sigma }(r).  \tag{$\widetilde{NE}$}
\end{equation}
\end{definition}
\begin{remark}
It is known by Baudoin in \cite[Lemma 4.7]{BaudoinLecture2022} that property
(NE) is satisfied with some $\sigma >0$, then $\sigma \geq \sigma _{p}^{\#}$%
. So property (NE) is only interested when $\sigma=\sigma _{p}^{\#},$ and
whenever we say that property (NE) is satisfied, we automatically assume $%
\sigma=\sigma _{p}^{\#}$. Very recently, property (NE) is verified on nested fractals by Chang and the
authors \cite[Theorem 2.3]{CGYZ24} with the help of a lemma in this paper, on Sierpi\'nski carpets for $%
p>\dim_{ARC}(K, d)$ by Yang \cite[Theorem 2.8]{Yang23} and for all $p>1$ by
Murugan and Shimizu \cite[Theorem 1.4]{MuruganShimizu23}. Kajino and Shimizu
\cite[Section 5]{KS24} also show property (NE) under `$p$-contraction
property' based on Kigami's partition of metric spaces \cite{Kigami2022penergy}.\end{remark}

Our first result is that, two kinds of BBM type characterization under (KE) and (NE) are
established in Theorem \ref{thm4.1} and Theorem \ref{thm4.2}, without using
 heat kernel estimates.


Our second result is that, under the two-sided heat kernel estimates, we show the equivalence of the
heat kernel-based $p$-energy norms and Besov-Korevaar-Schoen norms on
bounded and unbounded metric measure spaces in Lemma \ref{thm1}, and the following weak-monotonicity equivalence.

\begin{theorem}
\label{thm4}If a metric measure space $(M,d,\mu )$ admits a heat kernel $%
\{p_{t}\}_{t>0}$ satisfying (UHE) and (LHE), then we have
the following equivalences:

\begin{itemize}
\item[(i)~] $(\widetilde{KE})\Longleftrightarrow (\widetilde{NE})$ with the
same $\sigma>0$,

\item[(ii)] $(KE)\Longleftrightarrow (NE)$ with the same $\sigma>0$.
\end{itemize}
\end{theorem}

Our rest results focus on fractals. In what follows, we assume that $K$ is a connected homogeneous p.c.f. self-similar set.
Let $\alpha =\func{dim}_{H}(K)$ and denote the $\mathcal{H}%
^{\alpha }$-measure of $K$ by $\mu $. Let $K^{F}:=\bigcup_{f\in F}f(K)$ be
a fractal glue-up where $F$ is a proper set of functions (see details in
Subsection \ref{subsec5.1}) and we equip it with a natural glue-up measure $\mu ^{F}$ as in (\ref%
{muF}). This notion is used to unify bounded and unbounded fractals.
Following the definitions for $K$ in \cite{GaoYuZhang2022PA}, we define
\begin{equation}
E_{n}^{(p)}(u):=\sum_{x,y\in V_{w},|w|=n}|u(x)-u(y)|^{p}\text{ \ \ and\ \ \ }%
\mathcal{E}_{n}^{\sigma }(u):=\rho ^{-n(p\sigma -\alpha )}E_{n}^{(p)}(u),
\label{EE}
\end{equation}%
then the local $p$-energy norm on $K^{F}$ is given by
\begin{equation}
E_{n}^{(p),F}(u):=\sum_{f\in F}\sum_{x,y\in V_{w},|w|=n}|u\circ f(x)-u\circ
f(y)|^{p}=\sum_{f\in F}E_{n}^{(p)}(u\circ f),  \label{p_energy}
\end{equation}%
and $\mathcal{E}_{n}^{\sigma ,F}(u)=\rho ^{-n(p\sigma -\alpha
)}E_{n}^{(p),F}(u)$.

\begin{definition}
\label{ve} We say that a fractal glue-up $K^{F}$ satisfies property ($%
\normalfont{VE}$) with $\sigma >0$, if there exists a constant $C>0$ such that for all $%
u \in L^p(K^{F},\mu^F)$,
\begin{equation}  \label{VE}
\sup_{n\geq 0}\mathcal{E}_{n}^{\sigma ,F}(u)\leq C\liminf_{n\rightarrow
\infty }\mathcal{E}_{n}^{\sigma ,F}(u).  \tag{\normalfont{VE}}
\end{equation}
We say that a fractal glue-up $K^{F}$ satisfies property ($\widetilde{VE}$)
with $\sigma >0$, if there exists $C>0$ such that for all $u \in
L^p(K^{F},\mu^F)$,
\begin{equation}  \label{wVE}
\limsup_{n\rightarrow \infty }\mathcal{E}_{n}^{\sigma ,F}(u)\leq
C\liminf_{n\rightarrow \infty }\mathcal{E}_{n}^{\sigma ,F}(u).
\tag{$\widetilde{VE}$}
\end{equation}
\end{definition}

The proof of the following equivalence is more delicate, which involves proper
truncation, an improvement of the $p$%
-energies equivalence in \cite{GaoYuZhang2022PA} with ring-splitting, and comparing the vertex-energies of different levels. Also, analysis on unbounded fractals requires more arguments than the bounded cases.
%The notion $C_H$ in the following theorem is.

\begin{theorem}
\label{thm5}Let $K^{F}$ be a fractal glue-up, where $K$ is a connected
homogeneous p.c.f. self-similar set, then we have the following equivalences:

\begin{itemize}
\item[(i)~] $(\widetilde{VE}) \Longleftrightarrow (\widetilde{NE})$ with the
same $\sigma>\alpha /p$;

\item[(ii)] $(VE) \Longleftrightarrow (NE)$ when $\sigma=\sigma
_{p}^{\#}>\alpha /p$ and $R_0=C_H$ (a positive number determined by $K^{F}$).
\end{itemize}
\end{theorem}

%which  simply to divide integral in a ball into a summation of integrals over annuli see Lemma \ref{lem6.3}.

\begin{remark}
The equivalence of integral-type properties (KE, NE) and discrete-type (VE)
does not rely on the p.c.f. property, but it requires the
continuity of the functions in $B_{p,\infty}^{\sigma}$. The reason why we
consider p.c.f. fractals is to avoid complexity in defining (VE) properly,
and also verifying (VE) is relatively easier under the p.c.f. property (see
Proposition \ref{prop:critical}).
\end{remark}

Theorem \ref{thm4} and Theorem \ref{thm5} are very important in our paper
with many corollaries. In Section \ref{subsec5.4}, we will apply them to
obtain Corollary \ref{corol1.8}, and further deduce Theorem \ref{corol1.9} focusing
on nested fractals and their blowups. Moreover, they also deduce Corollary %
\ref{jjjj} and Corollary \ref{jjj} that are important to Dirichlet forms
when $p=2$.

\subsection{Structure of the paper}

The organization of the paper is as follows. In Section \ref{sec4}, we establish BBM type characterization
under appropriate weak-monotonicity properties (Theorem \ref{thm4.1} and Theorem \ref{thm4.2}). In Section \ref{sec2}, we demonstrate the equivalence between heat kernel-based and Besov-Korevaar-Schoen $p$-energy norms under two-sided estimates (Lemma \ref{thm1}), and the equivalence of the corresponding weak-monotonicity properties (Theorem \ref{thm4}). In Section \ref{sec5}, we study fractal glue-ups $K^F$. In
Subsection \ref{subsec5.1}, we introduce necessary geometric properties of $K^F$ and present some useful lemmas. In Subsection \ref{subsec5.2},
we verify (VE) for nested fractal glue-ups. In Subsection \ref{subsec5.3}, we prove Theorem \ref{thm5}. In Subsection \ref{subsec5.4}, we show that the BBM type characterization
and Gagliardo-Nirenberg inequality hold true for nested fractal blow-ups (Theorem \ref{corol1.9}).

\textbf{Notation}: The letters $C$, $C^{\prime }$, $C_{i}$,$C_{i}^{\prime }$, 
$C_{i}^{\prime\prime }$, $c$ are universal positive constants depending only
on $M$ which may vary at each occurrence. The sign $\asymp $ means that both
$\leq $ and $\geq $ are true with uniform values of $C$ depending only on $M$%
. $a\wedge b:=\min \{a,b\}$, $a\vee b:=\max \{a,b\}$.

\section{Consequence of (KE) and (NE): BBM type characterization%
}

\label{sec4} In this section, we present our BBM type characterization
for heat kernel-based and Besov-Korevaar-Schoen $p$-energy norms. Under the settings in Section \ref%
{sec2}, these two BBM type characterization results will
coincide.

\subsection{BBM type characterization for heat kernel-based $p$%
-energy norms}

In our proof, we need the following embedding relation.

\begin{proposition}
\label{Prop32}For any $\delta \in (0,\sigma )$, we have
\begin{equation}
\mathcal{D}(E_{p,\infty }^{\sigma })\subseteq
\mathcal{D}(E_{p,p}^{\sigma -\delta }).   \label{306}
\end{equation}
\end{proposition}

\begin{proof}
By the elementary inequality $|a-b|^{p}\leq 2^{p-1}(|a|^{p}+|b|^{p})$ and
symmetry, we have
\begin{align}
& \int_{M}\int_{M}|u(x)-u(y)|^{p}p_{t}(x,y)d\mu (x)d\mu (y)  \notag \\
\leq & 2^{p-1}\int_{M}\int_{M}(|u(x)|^{p}+|u(y)|^{p})p_{t}(x,y)d\mu (x)d\mu
(y)  \notag \\
=& 2^{p}\int_{M}|u(x)|^{p}\left( \int_{M}p_{t}(x,y)d\mu (y)\right) d\mu
(x)\leq 2^{p}\Vert u\Vert _{p}^{p}.  \label{304}
\end{align}

Fix $\epsilon \in (0,R_{0}^{\beta ^{\ast }})$. By \eqref{304},%
\begin{eqnarray*}
E_{p,p}^{\sigma -\delta }(u) &=&\int_{0}^{R_{0}^{\beta ^{\ast }}}\frac{1}{t^{%
{p(\sigma -\delta )}/{\beta ^{\ast }}}}\left(
\int_{M}\int_{M}|u(x)-u(y)|^{p}p_{t}(x,y)d\mu (x)d\mu (y)\right) \frac{dt}{t}
\\
&\leq &\int_{0}^{\epsilon }\frac{1}{t^{{p(\sigma -\delta )}/{\beta ^{\ast }}}%
}\left( \int_{M}\int_{M}|u(x)-u(y)|^{p}p_{t}(x,y)d\mu (x)d\mu (y)\right)
\frac{dt}{t} \\
&&+\int_{\epsilon }^{\infty }\frac{1}{t^{{p(\sigma -\delta )}/{\beta ^{\ast }%
}}}\left( \int_{M}\int_{M}|u(x)-u(y)|^{p}p_{t}(x,y)d\mu (x)d\mu (y)\right)
\frac{dt}{t} \\
&\leq &\int_{0}^{\epsilon }t^{-1+p\delta /{\beta ^{\ast }}}dt\sup_{t\in
(0,R_{0}^{\beta ^{\ast }})}\Psi _{u}^{\sigma }(t)+2^{p}\Vert u\Vert
_{p}^{p}\int_{\epsilon }^{\infty }\frac{dt}{t^{1+{p(\sigma -\delta )}/{\beta
^{\ast }}}} \\
&=&\frac{{\beta ^{\ast }}\epsilon ^{p\delta /{\beta ^{\ast }}}}{p\delta }%
\sup_{t\in (0,R_{0}^{\beta ^{\ast }})}\Psi _{u}^{\sigma }(t)+2^{p}\Vert
u\Vert _{p}^{p}\frac{{\beta ^{\ast }}\epsilon ^{-p(\sigma -\delta )/{\beta
^{\ast }}}}{p(\sigma -\delta )} \\
&=&\frac{{\beta ^{\ast }}\epsilon ^{p\delta /{\beta ^{\ast }}}}{p\delta }%
E_{p,\infty }^{\sigma }(u)+\frac{2^{p}{\beta ^{\ast }}\Vert u\Vert _{p}^{p}}{%
p(\sigma -\delta )}\epsilon ^{-p(\sigma -\delta )/{\beta ^{\ast }}}<\infty ,
\end{eqnarray*}%
thus showing \eqref{306}.
\end{proof}

\begin{theorem}
\label{thm4.1} Suppose that $(M,d,\mu )$ satisfies property (KE) with $\tilde{\sigma}_{p}>0$ and heat kernel $\{p_{t}\}_{t>0}$
. Then there exists a
positive constant $C$ such that for all $u\in \mathcal{D}(B_{p,\infty }^{%
\tilde{\sigma}_{p}})$,
\begin{equation}
C^{-1}{E}_{p,\infty }^{\tilde{\sigma}_{p}}(u)\leq \liminf_{\sigma \uparrow
\tilde{\sigma}_{p}}(\tilde{\sigma}_{p}-\sigma ){E}_{p,p}^{\sigma }(u)\leq
\limsup_{\sigma \uparrow \tilde{\sigma}_{p}}(\tilde{\sigma}_{p}-\sigma ){E}%
_{p,p}^{\sigma }(u)\leq C{E}_{p,\infty }^{\tilde{\sigma}_{p}}(u).
\label{eq5.1}
\end{equation}
\end{theorem}

\begin{proof}
Let $u\in \mathcal{D}(E_{p,\infty }^{\tilde{\sigma _{p}}})$ and $\sigma \in
(0,\tilde{\sigma}_{p})$. By Proposition \ref{Prop32}, $u\in \mathcal{D}%
(E_{p,p}^{\sigma })$. Note that $\Psi _{u}^{\sigma }(t)=t^{p(\tilde{\sigma}%
_{p}-\sigma )/{\beta ^{\ast }}}\Psi _{u}^{\tilde{\sigma}_{p}}(t).$

For the left-hand side, by \eqref{EEE}, we have
\begin{align}
\liminf_{\sigma \uparrow \tilde{\sigma}_{p}}(\tilde{\sigma}_{p}-\sigma ){E}%
_{p,p}^{\sigma }(u)=& \liminf_{\sigma \uparrow \tilde{\sigma}_{p}}(\tilde{%
\sigma}_{p}-\sigma )\int_{0}^{R_{0}^{{\beta ^{\ast }}}}t^{p(\tilde{\sigma}%
_{p}-\sigma )/{\beta ^{\ast }}}\Psi _{u}^{\tilde{\sigma}_{p}}(t)\frac{dt}{t}
\notag \\
\geq & \liminf_{\sigma \uparrow \tilde{\sigma}_{p}}(\tilde{\sigma}%
_{p}-\sigma )\int_{0}^{\tilde{\sigma}_{p}-\sigma }t^{p(\tilde{\sigma}%
_{p}-\sigma )/{\beta ^{\ast }}}\Psi _{u}^{\tilde{\sigma}_{p}}(t)\frac{dt}{t}
\notag \\
\geq & \liminf_{\sigma \uparrow \tilde{\sigma}_{p}}(\tilde{\sigma}%
_{p}-\sigma )\int_{0}^{\tilde{\sigma}_{p}-\sigma }t^{p(\tilde{\sigma}%
_{p}-\sigma )/{\beta ^{\ast }}}\frac{dt}{t}\inf_{t\in (0,\tilde{\sigma}%
_{p}-\sigma )}\Psi _{u}^{\tilde{\sigma}_{p}}(t)  \notag \\
=& \frac{{\beta ^{\ast }}}{p}\liminf_{\sigma \uparrow \tilde{\sigma}_{p}}(%
\tilde{\sigma}_{p}-\sigma )^{p(\tilde{\sigma}_{p}-\sigma )/{\beta ^{\ast }}%
}\inf_{t\in (0,\tilde{\sigma}_{p}-\sigma )}\Psi _{u}^{\tilde{\sigma}_{p}}(t)
\notag \\
=& \frac{{\beta ^{\ast }}}{p}\liminf_{\sigma \uparrow \tilde{\sigma}%
_{p}}\inf_{t\in (0,\tilde{\sigma}_{p}-\sigma )}\Psi _{u}^{\tilde{\sigma}%
_{p}}(t)=\frac{{\beta ^{\ast }}}{p}\liminf_{t\rightarrow 0}\Psi _{u}^{\tilde{%
\sigma}_{p}}(t).  \notag
\end{align}%
It follows from property (KE) that
\begin{equation*}
\liminf_{\sigma \uparrow \tilde{\sigma}_{p}}(\tilde{\sigma}_{p}-\sigma ){E}%
_{p,p}^{\sigma }(u)\geq \frac{C{\beta ^{\ast }}}{p}E_{p,\infty }^{\tilde{%
\sigma}_{p}}(u).
\end{equation*}

For the right-hand side, let $A\in (0,R_{0})$ be a finite positive
number. By (\ref{304}), for any $\sigma \in (0,\tilde{\sigma _{p}})$,
\begin{eqnarray*}
{E}_{p,p}^{\sigma }(u) &=&\int_{0}^{R_{0}^{\beta ^{\ast }}}\frac{1}{t^{{%
p\sigma }/{\beta ^{\ast }}}}\left(
\int_{M}\int_{M}|u(x)-u(y)|^{p}p_{t}(x,y)d\mu (x)d\mu (y)\right) \frac{dt}{t}
\\
&\leq &\int_{0}^{A^{\beta ^{\ast }}}\frac{1}{t^{{p\sigma }/{\beta ^{\ast }}}}%
\left( \int_{M}\int_{M}|u(x)-u(y)|^{p}p_{t}(x,y)d\mu (x)d\mu (y)\right)
\frac{dt}{t}+\int_{A^{\beta ^{\ast }}}^{\infty }\frac{2^{p}\Vert u\Vert
_{p}^{p}}{t^{{p\sigma }/{\beta ^{\ast }}}}\frac{dt}{t} \\
&=&\int_{0}^{A^{\beta ^{\ast }}}t^{p(\tilde{\sigma}_{p}-\sigma )/{\beta
^{\ast }}}\Psi _{u}^{\tilde{\sigma}_{p}}(t)\frac{dt}{t}+\frac{2^{p}{\beta
^{\ast }}\Vert u\Vert _{p}^{p}}{p\sigma A^{{p\sigma }}}.
\end{eqnarray*}%
It follows that
\begin{align}
\limsup_{\sigma \uparrow \tilde{\sigma}_{p}}(\tilde{\sigma}_{p}-\sigma ){E}%
_{p,p}^{\sigma }(u)& \leq \limsup_{\sigma \uparrow \tilde{\sigma}_{p}}(%
\tilde{\sigma}_{p}-\sigma )\int_{0}^{A^{\beta ^{\ast }}}t^{p(\tilde{\sigma}%
_{p}-\sigma )/{\beta ^{\ast }}}\Psi _{u}^{\tilde{\sigma}_{p}}(t)\frac{dt}{t}%
+\limsup_{\sigma \uparrow \tilde{\sigma}_{p}}(\tilde{\sigma}_{p}-\sigma )%
\frac{2^{p}{\beta ^{\ast }}\Vert u\Vert _{p}^{p}}{p\sigma A^{{p\sigma }}}
\notag \\
& \leq \limsup_{\sigma \uparrow \tilde{\sigma}_{p}}(\tilde{\sigma}%
_{p}-\sigma )\int_{0}^{A^{\beta ^{\ast }}}t^{p(\tilde{\sigma}_{p}-\sigma )/{%
\beta ^{\ast }}}\frac{dt}{t}\sup_{t\in (0,A^{{\beta ^{\ast }}})}\Psi _{u}^{%
\tilde{\sigma}_{p}}(t)+0  \notag \\
& =\frac{{\beta ^{\ast }}}{p}\limsup_{\sigma \uparrow \tilde{\sigma}%
_{p}}A^{p(\tilde{\sigma}_{p}-\sigma )}\sup_{t\in (0,A^{{\beta ^{\ast }}%
})}\Psi _{u}^{\tilde{\sigma}_{p}}(t)  \notag \\
& =\frac{{\beta ^{\ast }}}{p}\sup_{t\in (0,A^{{\beta ^{\ast }}})}\Psi _{u}^{%
\tilde{\sigma}_{p}}(t)\leq \frac{{\beta ^{\ast }}}{p}E_{p,\infty }^{\tilde{%
\sigma}_{p}}(u).  \label{NE_right}
\end{align}%
The proof is complete.
\end{proof}

\subsection{BBM type characterization for Besov-Korevaar-Schoen
$p$-energy norms}

The proof of this version is similar to the previous one. {\color{blue}}
% {\color{red}Under a rather different assumption that the $(1, p)$-Poincar\'e inequality holds, G\'orny \cite{Gorny22.JGA} proved the convergence of the BBM difference quotient to the Cheeger energy. \textbf{[[put it here ?? This is not so good.-I also think this is bad.]]}}

\begin{proposition}
\label{Prop31}For any $\delta \in (0,\sigma )$, we have
\begin{equation}
 B_{p,\infty }^{\sigma }\subseteq B_{p,p}^{\sigma -\delta }.
\label{301}
\end{equation}
\end{proposition}

\begin{proof}
By the elementary inequality $|a-b|^{p}\leq 2^{p-1}(|a|^{p}+|b|^{p})$, we
have
\begin{align}
& \int_{M}\frac{1}{V(x,r)}\int_{B(x,r)}|u(x)-u(y)|^{p}d\mu (y)d\mu (x)
\notag \\
\leq & 2^{p-1}\int_{M}\frac{1}{V(x,r)}\int_{B(x,r)}(|u(x)|^{p}+|u(y)|^{p})d%
\mu (y)d\mu (x)  \notag \\
=& 2^{p-1}\left( \Vert u\Vert _{p}^{p}+\int_{M}\int_{M}\frac{\mathbf{1}%
_{\{d(y,x)<r\}}}{V(x,r)}|u(y)|^{p}d\mu (y)d\mu (x)\right)  \notag \\
=& 2^{p-1}\left( \Vert u\Vert _{p}^{p}+\int_{M}\left( \int_{B(y,r)}\frac{1}{%
V(x,r)}d\mu (x)\right) |u(y)|^{p}d\mu (y)\right)  \notag \\
\leq & 2^{p-1}\left( \Vert u\Vert _{p}^{p}+\sup_{x\in M}\frac{V(x,2r)}{V(x,r)%
}\int_{M}|u(y)|^{p}d\mu (y)\right) \leq C\Vert u\Vert _{p}^{p}.  \label{300}
\end{align}

Fix $\epsilon \in (0,R_{0})$. By \eqref{300},%
\begin{eqnarray*}
\lbrack u]_{B_{p,p}^{\sigma -\delta }}^{p} &=&\int_{0}^{R_{0}}r^{-(p\sigma
-p\delta )}\int_{M}\frac{1}{V(x,r)}\int_{B(x,r)}|u(x)-u(y)|^{p}d\mu (y)d\mu
(x)\frac{dr}{r} \\
&\leq &\int_{0}^{\epsilon }r^{-(p\sigma -p\delta )}\int_{M}\frac{1}{V(x,r)}%
\int_{B(x,r)}|u(x)-u(y)|^{p}d\mu (y)d\mu (x)\frac{dr}{r} \\
&&+\int_{\epsilon }^{\infty }r^{-(p\sigma -p\delta )}\int_{M}\frac{1}{V(x,r)}%
\int_{B(x,r)}|u(x)-u(y)|^{p}d\mu (y)d\mu (x)\frac{dr}{r} \\
&\leq &\int_{0}^{\epsilon }r^{-1+p\delta }dr\sup_{r\in (0,R_{0})}\Phi
_{u}^{\sigma }(r)+C\Vert u\Vert _{p}^{p}\int_{\epsilon }^{\infty
}r^{-(p\sigma -p\delta )}\frac{dr}{r} \\
&=&\frac{\epsilon ^{p\delta }}{p\delta }\sup_{r\in (0,R_{0})}\Phi
_{u}^{\sigma }(r)+C\Vert u\Vert _{p}^{p}\frac{\epsilon ^{-p(\sigma -\delta )}%
}{p(\sigma -\delta )}=\frac{\epsilon ^{p\delta }}{p\delta }[u]_{B_{p,\infty
}^{\sigma }}^{p}+C^{\prime }\Vert u\Vert _{p}^{p}<\infty ,
\end{eqnarray*}%
thus showing \eqref{301}.
\end{proof}

\begin{theorem}
\label{thm4.2} Suppose that $(M,d,\mu )$ satisfies property (NE), then there
exists a positive constant $C$ such that for all $u\in B_{p,\infty }^{\sigma
_{p}^{\#}}$,
\begin{equation}
C^{-1}[u]_{{B}_{p,\infty }^{\sigma _{p}^{\#}}}^{p}\leq \liminf_{\sigma
\uparrow \sigma _{p}^{\#}}(\sigma _{p}^{\#}-\sigma )[u]_{{B}_{p,p}^{\sigma
}}^{p}\leq \limsup_{\sigma \uparrow \sigma _{p}^{\#}}(\sigma
_{p}^{\#}-\sigma )[u]_{{B}_{p,p}^{\sigma }}^{p}\leq C[u]_{{B}_{p,\infty
}^{\sigma _{p}^{\#}}}^{p}.  \label{NE_conv}
\end{equation}%
Suppose that $(M,d,\mu )$ satisfies property $(\widetilde{NE})$ with $\sigma
_{p}>0$, then there exists a positive constant $C$ such that for all $u\in
B_{p,\infty }^{\sigma _{p}}$,
\begin{equation}
C^{-1}\Vert u\Vert _{{KS}_{p,\infty }^{\sigma _{p}}}^{p}\leq \liminf_{\sigma
\uparrow \sigma _{p}}(\sigma _{p}-\sigma )[u]_{{B}_{p,p}^{\sigma
}}^{p}\leq \limsup_{\sigma \uparrow \sigma _{p}}(\sigma _{p}-\sigma )[u]_{{B}%
_{p,p}^{\sigma }}^{p}\leq C\Vert u\Vert _{{KS}_{p,\infty }^{\sigma
_{p}}}^{p}.  \label{KS_conv}
\end{equation}
\end{theorem}

\begin{proof}
We show \eqref{NE_conv} first. For the left-hand side, similarly, we have
that for any $\sigma _{p}>0,$
\begin{align}
\liminf_{\sigma \uparrow \sigma _{p}}(\sigma _{p}-\sigma )[u]_{{B}%
_{p,p}^{\sigma }}^{p}=& \liminf_{\sigma \uparrow \sigma _{p}}(\sigma
_{p}-\sigma )\int_{0}^{R_{0}}r^{p(\sigma _{p}-\sigma )}\Phi _{u}^{\sigma
_{p}}(r)\frac{dr}{r}  \notag \\
\geq & \liminf_{\sigma \uparrow \sigma _{p}}(\sigma _{p}-\sigma
)\int_{0}^{\sigma _{p}-\sigma }r^{p(\sigma _{p}-\sigma )}\Phi _{u}^{\sigma
_{p}}(r)\frac{dr}{r}  \notag \\
\geq & \liminf_{\sigma \uparrow \sigma _{p}}(\sigma _{p}-\sigma
)\int_{0}^{\sigma _{p}-\sigma }r^{p(\sigma _{p}-\sigma )}\frac{dr}{r}%
\inf_{r\in (0,\sigma _{p}-\sigma )}\Phi _{u}^{\sigma _{p}}(r)  \notag \\
=& \frac{1}{p}\liminf_{\sigma \uparrow \sigma _{p}}\inf_{r\in (0,\sigma
_{p}-\sigma )}\Phi _{u}^{\sigma _{p}}(r)=\frac{1}{p}\liminf_{r\rightarrow
0}\Phi _{u}^{\sigma _{p}}(r).  \label{408}
\end{align}%
Therefore, by property (NE), we have for $\sigma _{p}=\sigma _{p}^{\#}$ that
\begin{equation}
\liminf_{\sigma \uparrow \sigma _{p}^{\#}}(\sigma _{p}^{\#}-\sigma )[u]_{{B}%
_{p,p}^{\sigma }}^{p}\geq \frac{C^{-1}}{p}[u]_{{B}_{p,\infty }^{\sigma
_{p}^{\#}}}^{p},  \label{NE_left}
\end{equation}%
where the constant $C$ is the same as in \eqref{NE1}.

For the other side, let $A\in (0,R_{0})$ be a finite positive number. We
have from \eqref{300} that
\begin{eqnarray}
\lbrack u]_{{B}_{p,p}^{\sigma }}^{p} &=&\int_{0}^{R_{0}}r^{-p\sigma }\left(
\int_{M}\frac{1}{V(x,r)}\int_{B(x,r)}|u(x)-u(y)|^{p}d\mu (y)d\mu (x)\right)
\frac{dr}{r}  \notag \\
&\leq &\int_{0}^{A}r^{-p\sigma }\left( \int_{M}\frac{1}{V(x,r)}%
\int_{B(x,r)}|u(x)-u(y)|^{p}d\mu (y)d\mu (x)\right) \frac{dr}{r}%
+C\int_{A}^{\infty }r^{-p\sigma }\Vert u\Vert _{p}^{p}\frac{dr}{r}  \notag \\
&=&\int_{0}^{A}r^{p(\sigma _{p}^{\#}-\sigma )}\Phi _{u}^{\sigma _{p}^{\#}}(r)%
\frac{dr}{r}+\frac{C\Vert u\Vert _{p}^{p}}{p\sigma A^{p\sigma }}.
\label{300-1}
\end{eqnarray}%
It follows that
\begin{align}
\limsup_{\sigma \uparrow \sigma _{p}^{\#}}(\sigma _{p}^{\#}-\sigma )[u]_{{B}%
_{p,p}^{\sigma }}^{p}& \leq \limsup_{\sigma \uparrow \sigma
_{p}^{\#}}(\sigma _{p}^{\#}-\sigma )\int_{0}^{A}r^{p(\sigma _{p}^{\#}-\sigma
)}\Phi _{u}^{\sigma _{p}^{\#}}(r)\frac{dr}{r}+\limsup_{\sigma \uparrow
\sigma _{p}^{\#}}(\sigma _{p}^{\#}-\sigma )\frac{C\Vert u\Vert _{p}^{p}}{%
p\sigma A^{p\sigma }}  \notag \\
& \leq \limsup_{\sigma \uparrow \sigma _{p}^{\#}}(\sigma _{p}^{\#}-\sigma
)\int_{0}^{A}r^{p(\sigma _{p}^{\#}-\sigma )}\frac{dr}{r}\sup_{r\in
(0,R_{0})}\Phi _{u}^{\sigma _{p}^{\#}}(r)+0  \notag \\
& =\frac{1}{p}\limsup_{\sigma \uparrow \sigma _{p}^{\#}}A^{p(\sigma
_{p}^{\#}-\sigma )}\sup_{r\in (0,R_{0})}\Phi _{u}^{\sigma _{p}^{\#}}(r)=%
\frac{1}{p}[u]_{B_{p,\infty }^{\sigma _{p}^{\#}}}^{p},  \label{42}
\end{align}%
thus showing \eqref{NE_conv}.

Next, we show \eqref{KS_conv}. By property $(\widetilde{NE})$ with $\sigma
_{p}$ and \eqref{408}, we have
\begin{equation}
\liminf_{\sigma \uparrow \sigma _{p}}(\sigma _{p}-\sigma )[u]_{{B}%
_{p,p}^{\sigma }}^{p}\geq C^{-1}\Vert u\Vert _{KS_{p,p}^{\sigma _{p}}}^{p}.
\label{KS_left}
\end{equation}%
Then for the other side of \eqref{KS_conv}, by the definition of $\limsup $,
there exists $\epsilon \in (0,R_{0})$ such that
\begin{equation*}
\sup_{r\in (0,\epsilon )}\Phi _{u}^{\sigma _{p}}(r)\leq
2\limsup_{r\rightarrow 0}\Phi _{u}^{\sigma _{p}}(r).
\end{equation*}%
It follows from replacing $\sigma _{p}^{\#}$ by $\sigma _{p}$
and taking $A=\epsilon $ in (\ref{300-1}) that,%
\begin{align}
\limsup_{\sigma \uparrow \sigma _{p}}(\sigma _{p}-\sigma )[u]_{{B}%
_{p,p}^{\sigma }}^{p}& \leq \limsup_{\sigma \uparrow \sigma _{p}}(\sigma
_{p}-\sigma )\int_{0}^{\epsilon }r^{p(\sigma _{p}-\sigma )}\Phi _{u}^{\sigma
_{p}}(r)\frac{dr}{r}+\limsup_{\sigma \uparrow \sigma _{p}}(\sigma
_{p}-\sigma )\frac{C\Vert u\Vert _{p}^{p}}{p\sigma \epsilon ^{p\sigma }}
\notag \\
& \leq \limsup_{\sigma \uparrow \sigma _{p}}(\sigma _{p}-\sigma
)\int_{0}^{\epsilon }r^{p(\sigma _{p}-\sigma )}\frac{dr}{r}\sup_{r\in
(0,\epsilon )}\Phi _{u}^{\sigma _{p}}(r)+0  \notag \\
& =\frac{1}{p}\sup_{r\in (0,\epsilon )}\Phi _{u}^{\sigma _{p}}(r)\leq \frac{2%
}{p}\limsup_{r\rightarrow 0}\Phi _{u}^{\sigma _{p}}(r)=\frac{2}{p}\Vert
u\Vert _{{KS}_{p,\infty }^{\sigma _{p}}}^{p}.  \label{KS_right}
\end{align}%
The proof is complete.
\end{proof}

\section{Equivalences on metric measure spaces}

\label{sec2} In this section, we assume that the metric
measure spaces admit a heat kernel satisfying (UHE) and (LHE).

%Once \eqref{hk} holds, we know that $\mu$ is $\alpha$-regular. For
%convenience, in this case, we as well rewrite (\ref{pi}) by
%\begin{equation*}
%\lbrack u]_{B_{p,\infty }^{\sigma }}^{p}:=\sup_{r\in
%(0,R_{0})}r^{-p\sigma-\alpha }\int_{M}\int_{B(x,r)}|u(x)-u(y)|^{p}d\mu
%(y)d\mu (x),
%\end{equation*}
%and
%\begin{equation*}
%\lbrack u]_{B_{p,p}^{\sigma }}^{p}:=\int_{0}^{R_{0}}r^{-p\sigma-\alpha
%}\left( \int_{M}\int_{B(x,r)}|u(x)-u(y)|^{p}d\mu (y)d\mu (x)\right) \frac{dr%
%}{r}
%\end{equation*}
%for (\ref{pp}).

\subsection{Equivalence of heat kernel-based and Besov norms}

 In this subsection, we fix $\sigma>0$. The main result of this subsection is the following.

\begin{lemma}
\label{thm1}If a metric measure space $(M,d,\mu )$ admits a heat kernel $%
\{p_{t}\}_{t>0}$ satisfying (UHE) and (LHE), then $\mathcal{D}(E_{p,\infty }^{\sigma })=B_{p,\infty }^{\sigma }$
and for all $u\in B_{p,\infty }^{\sigma }$,
\begin{equation}
E_{p,\infty }^{\sigma }(u)\asymp \lbrack u]_{B_{p,\infty }^{\sigma }}^{p}%
\text{\ and\ }\Vert u\Vert _{KS_{p,\infty }^{\sigma }}^{p}\asymp
\limsup_{t\rightarrow 0}\Psi _{u}^{\sigma }(t).  \label{eq1.7}
\end{equation}%
In addition, $\mathcal{D}(E_{p,p}^{\sigma })=B_{p,p}^{\sigma }$ and for all $%
u\in B_{p,p}^{\sigma }$,
\begin{equation}
E_{p,p}^{\sigma }(u)\asymp \lbrack u]_{B_{p,p}^{\sigma }}^{p}.  \label{eq1.8}
\end{equation}
\end{lemma}

We prove Lemma \ref{thm1} by pure elementary analysis (without
probability arguments) under slightly milder assumptions, and split the proof into Propositions %
\ref{lem3.1}-\ref{lem3.4}. Part of Lemma \ref{thm1} was also studied by
K.~Pietruska-Pa{\l }uba in \cite[Theorems 3.1 and 3.2]%
{Pietruska-Paluba.2010.} using probability arguments, and by Alonso Ruiz,
Baudoin et al. in \cite[Proposition 4.2]{AlonsoBaudoinchen2020CVPDE} which
considered the equivalence of heat semigroup-based Besov space {\textbf{B}}$%
^{p,\alpha/2}$ and Besov space $B_{p,\infty }^{\alpha }$ under the Gaussian
heat kernel estimate. Very recently, Cao and Qiu in \cite{Caoqiu23} prove
the equivalence of the Korevaar-Schoen norm and another type of Besov norm.

\begin{proposition}
\label{lem3.1} If $(M,d,\mu )$ admits a heat kernel $\{p_{t}\}_{t>0}$
satisfying (LHE), then there exists a positive constant $C$
such that for all $u\in \mathcal{D}(E_{p,p}^{\sigma })$,
\begin{equation}
\lbrack u]_{B_{p,p}^{\sigma }}^{p}\leq CE_{p,p}^{\sigma }(u).  \label{eq3.1}
\end{equation}
\end{proposition}

\begin{proof}
By (LHE), we have from \eqref{VD2} that
\begin{align}
& \frac{1}{t^{{p\sigma }/{\beta ^{\ast }}}}\int_{M}\int_{B(x, t^{{1}/{\beta
^{\ast }}})}|u(x)-u(y)|^{p}p_{t}(x,y)d\mu (y)d\mu (x)  \notag \\
\geq& \frac{c_{1}}{t^{p\sigma /\beta ^{\ast }}}\int_{M}\frac{1}{%
V(x,t^{1/\beta ^{\ast }})}\int_{B(x, t^{{1}/{\beta ^{\ast }}%
})}|u(x)-u(y)|^{p} \exp \left(-c_2\left(\frac{d(x,y)}{t^{1/\beta^*}}\right)^{%
\frac{\beta^*}{\beta^*-1}}\right)d\mu (y)d\mu (x)  \notag \\
\geq & c_{1}e^{-c_2} \Phi_u^{\sigma}( t^{1/\beta^*}) .  \label{eq3-2}
\end{align}
Therefore,
\begin{align*}
E_{p,p}^{\sigma }(u)&=\int_{0}^{R_{0}^{\beta ^{\ast }}}\frac{1}{t^{{p\sigma }%
/{\beta ^{\ast }}}}\left( \int_{M}\int_{M}|u(x)-u(y)|^{p}p_{t}(x,y)d\mu
(x)d\mu (y)\right) \frac{dt}{t} \\
& \geq \beta ^{\ast }c_{1}e^{-c_2}\int_{0}^{R_{0}} \Phi_u^{\sigma}(r)\frac{dr%
}{r} =\beta ^{\ast}c_{1}e^{-c_2}\lbrack u]_{B_{p,p}^{\sigma }}^{p},
\end{align*}
which completes the proof.
\end{proof}

%Now we prove the opposite inequality of \eqref{eq3.1}.

\begin{proposition}
\label{lem3.2} If $(M,d,\mu )$ admits a heat kernel $\{p_{t}\}_{t>0}$
satisfying (UHE), then there exists a positive constant $C$
such that for all $u\in B_{p,p}^{\sigma }$,
\begin{equation*}
E_{p,p}^{\sigma }(u)\leq C[u]_{B_{p,p}^{\sigma }}^{p}.
\end{equation*}
\end{proposition}

\begin{proof}
We split $M$ into union of rings $B(x,2^{n}t^{1/\beta ^{\ast }})\setminus
B(x,2^{n-1}t^{1/\beta ^{\ast }})$ for $x\in M$, where the integers $n\leq \log
_{2}(R_{0}t^{-1/\beta ^{\ast }})$, then by (\ref{UHE}),
\begin{align}
& \Psi_u^{\sigma}(t)=\frac{1}{t^{{p\sigma }/{\beta ^{\ast }}}}%
\int_{M}\int_{M}|u(x)-u(y)|^{p}p_{t}(x,y)d\mu (y)d\mu (x)  \notag \\
& =\frac{1}{t^{{p\sigma }/{\beta ^{\ast }}}}\int_{M}\sum_{n\leq \log
_{2}(R_{0}t^{-1/\beta ^{\ast }})}\int_{B(x,2^{n}t^{1/\beta ^{\ast
}})\setminus B(x,2^{n-1}t^{1/\beta ^{\ast }})}|u(x)-u(y)|^{p}p_{t}(x,y)d\mu
(y)d\mu (x)  \notag \\
&\leq \frac{1}{t^{{p\sigma }/{\beta ^{\ast }}}}\int_{M}\frac{c_{3}}{%
V(x,t^{1/\beta ^{\ast }})}\sum_{n\leq \log _{2}(R_{0}t^{-1/\beta ^{\ast
}})}\int_{B(x,2^{n}t^{1/\beta ^{\ast }})\setminus B(x,2^{n-1}t^{1/\beta
^{\ast }})}|u(x)-u(y)|^{p}\exp \left( -c_{4}2^{\frac{(n-1)\beta ^{\ast }}{%
\beta ^{\ast }-1}}\right) d\mu (y)d\mu (x)  \notag \\
&\leq \sum_{n\leq \log _{2}(R_{0}t^{-1/\beta ^{\ast }})}\frac{1}{t^{{p\sigma
}/{\beta ^{\ast }}}}\int_{M}\frac{c_{3}}{V(x,t^{1/\beta ^{\ast }})}%
\int_{B(x,2^{n}t^{1/\beta ^{\ast }})}|u(x)-u(y)|^{p}\exp \left( -c_{4}2^{%
\frac{(n-1)\beta ^{\ast }}{\beta ^{\ast }-1}}\right) d\mu (y)d\mu (x)  \notag
\\
&\leq \sum_{n\leq \log _{2}(R_{0}t^{-1/\beta ^{\ast }})}\exp \left( -c_{4}2^{%
\frac{(n-1)\beta ^{\ast }}{\beta ^{\ast }-1}}\right)\frac{2^{n p\sigma}}{%
(2^{n}t^{1/\beta ^{\ast }})^{p\sigma}} \int_{M}\frac{c_{3}C_{d}2^{n\alpha
_{1}}}{V(x,2^{n}t^{1/\beta ^{\ast }})}\int_{B(x,2^{n}t^{1/\beta ^{\ast
}})}|u(x)-u(y)|^{p}d\mu (y)d\mu (x)  \notag \\
&=c_{3}C_{d}\sum_{n=-\infty }^{\infty }1_{\{n\leq \log _{2}(R_{0}t^{-1/\beta
^{\ast }})\}}\exp \left( -c_{4}2^{\frac{(n-1)\beta ^{\ast }}{\beta ^{\ast }-1%
}}\right)2^{(p\sigma +\alpha _{1})n}\Phi_u^{\sigma}(2^{n}t^{1/\beta ^{\ast
}}) ,  \label{eq3.3}
\end{align}
where we use the volume doubling property \eqref{VD2} in the fifth
line
\begin{equation}
\frac{V(x,2^{n}t^{1/\beta ^{\ast }})}{V(x,t^{1/\beta ^{\ast }})}\leq
C_{d}2^{n\alpha _{1}}.  \label{VD}
\end{equation}

Therefore,%
\begin{align*}
& E_{p,p}^{\sigma }(u)=\int_{0}^{R_{0}^{\beta ^{\ast }}}\frac{1}{t^{{p\sigma
}/{\beta ^{\ast }}}}\left( \int_{M}\int_{M}|u(x)-u(y)|^{p}p_{t}(x,y)d\mu
(x)d\mu (y)\right) \frac{dt}{t} \\
& \leq c_{3}C_{d}\sum_{n=-\infty }^{\infty }\exp \left( -c_{4}2^{\frac{%
(n-1)\beta ^{\ast }}{\beta ^{\ast }-1}}\right) \int_{0}^{2^{-n\beta ^{\ast
}}R_{0}^{\beta ^{\ast }}}2^{(p\sigma +\alpha
_{1})n}\Phi_u^{\sigma}(2^{n}t^{1/\beta ^{\ast }}) \frac{dt}{t} \\
& =c_{3}C_{d}\beta ^{\ast }\sum_{n=-\infty }^{\infty }\exp \left( -c_{4}2^{%
\frac{(n-1)\beta ^{\ast }}{\beta ^{\ast }-1}}\right) 2^{(p\sigma +\alpha
_{1})n}\int_{0}^{R_{0}}\Phi_u^{\sigma}(r) \frac{dr}{r} =c_{3}C_{d}\beta
^{\ast }C_{0}[u]_{B_{p,p}^{\sigma }}^{p},
\end{align*}
where
\begin{eqnarray}
C_{0} &=&\sum_{n=-\infty }^{\infty }\exp \left( -c_{4}2^{\frac{(n-1)\beta
^{\ast }}{\beta ^{\ast }-1}}\right) 2^{(p\sigma +\alpha _{1})n}  \notag \\
&\leq &\sum_{n=-\infty }^{0}2^{(p\sigma +\alpha _{1})n}+\sum_{n=1}^{\infty
}\exp \left( -c_{4}2^{\frac{(n-1)\beta ^{\ast }}{\beta ^{\ast }-1}}\right)
2^{(p\sigma +\alpha _{1})n}<\infty ,  \label{c1}
\end{eqnarray}%
which completes the proof.
\end{proof}

%Next, we prove the second inclusion in Lemma \ref{thm1} by deriving the upper bound estimate of $[u]_{B_{p,\infty }^{\sigma }}^{p}$ in Lemma \ref%{lem3.3} and the lower bound estimate of $[u]_{B_{p,\infty }^{\sigma }}^{p}$in Lemma \ref{lem3.4}.
The following two propositions are easier to obtain.

\begin{proposition}
\label{lem3.3} If $(M,d,\mu )$ admits a heat kernel $\{p_{t}\}_{t>0}$
satisfying (LHE), then there exists a positive constant $C$
such that for all $u\in \mathcal{D}(E_{p,\infty }^{\sigma })$,
\begin{equation*}
\lbrack u]_{B_{p,\infty }^{\sigma }}^{p}\leq CE_{p,\infty }^{\sigma }(u).
\end{equation*}
\end{proposition}

\begin{proof}
{By \eqref{eq3-2}, we have
\begin{align*}
& E_{p,\infty }^{\sigma }(u)=\sup_{t\in (0,R_{0}^{\beta ^{\ast }})}\frac{1}{%
t^{{p\sigma }/{\beta ^{\ast }}}} \int_{M}\int_{M}|u(x)-u(y)|^{p}p_{t}(x,y)d%
\mu (x)d\mu (y) \\
&\geq c_{1}e^{-c_2}\sup_{t\in (0,R_{0}^{\beta ^{\ast }})}\Phi_u^{\sigma}(
t^{1/\beta^*}) =c_{1}e^{c_2}\sup_{r\in (0,R_{0})}\Phi_u^{\sigma}(r)
=c_{1}e^{c_2}[u]_{B_{p,\infty }^{\sigma }}^{p},
\end{align*}%
}which ends the proof.
\end{proof}

\begin{proposition}
\label{lem3.4} If $(M,d,\mu )$ admits a heat kernel $\{p_{t}\}_{t>0}$
satisfying (UHE), then there exists a positive constant $C$
such that for all $u\in B_{p,\infty }^{\sigma }$,
\begin{equation*}
E_{p,\infty }^{\sigma }(u)\leq C[u]_{B_{p,\infty }^{\sigma }}^{p}.
\end{equation*}
\end{proposition}

\begin{proof}
By \eqref{eq3.3} and \eqref{VD}, we have
\begin{align}
& \Psi_u^{\sigma}(t) \leq c_{3}C_{d}\sum_{n=-\infty }^{\infty }1_{\{n\leq
\log _{2}(R_{0}t^{-1/\beta ^{\ast }})\}}\exp \left( -c_{4}2^{\frac{%
(n-1)\beta ^{\ast }}{\beta ^{\ast }-1}}\right)2^{(p\sigma +\alpha
_{1})n}\Phi_u^{\sigma}(2^{n}t^{1/\beta ^{\ast }})  \notag \\
=& c_{3}C_{d}\sum_{n=-\infty }^{\infty }\exp \left( -c_{4}2^{\frac{%
(n-1)\beta ^{\ast }}{\beta ^{\ast }-1}}\right) 2^{(p\sigma +\alpha
_{1})n}[u]_{B_{p,\infty }^{\sigma }}^{p}=c_{3}C_{d}C_{0}[u]_{B_{p,\infty
}^{\sigma }}^{p},  \label{eq3.14}
\end{align}%
where $C_{0}$ is finite by \eqref{c1}.

%\begin{proof}
%By \eqref{eq3.3} and \eqref{VD}, we have
%\begin{align}
%& E_{p,\infty }^{\sigma }(u) \frac{1}{t^{p\sigma /\beta ^{\ast }}}\int_{M}%
%\int_{M}|u(x)-u(y)|^{p}p_{t}(x,y)d\mu (y)d\mu (x)  \notag \\
%\leq & c_{3}C_{d}\sum_{n=-\infty }^{\infty }\frac{1_{\{n\leq \log
%_{2}(R_{0}t^{-1/\beta ^{\ast }})\}}}{t^{p\sigma /\beta ^{\ast }}}\exp \left(
%-c_{4}2^{\frac{(n-1)\beta ^{\ast }}{\beta ^{\ast }-1}}\right) 2^{n\alpha
%_{1}}\left( \int_{M}\frac{1}{V(x,2^{n}t^{1/\beta ^{\ast }})}%
%\int_{B(x,2^{n}t^{1/\beta ^{\ast }})}|u(x)-u(y)|^{p}d\mu (y)d\mu (x)\right)
%\label{eq3.14-1} \\
%\leq & c_{3}C_{d}\sum_{n=-\infty }^{\infty }\exp \left( -c_{4}2^{\frac{%
%(n-1)\beta ^{\ast }}{\beta ^{\ast }-1}}\right) 2^{(p\sigma +\alpha
%_{1})n}\sup_{r\in (0,R_{0})}\frac{1}{r^{p\sigma }}\int_{M}\frac{1}{V(x,r)}%
%\int_{B(x,r)}|u(x)-u(y)|^{p}d\mu (y)d\mu (x)  \notag \\
%=& c_{3}C_{d}\sum_{n=-\infty }^{\infty }\exp \left( -c_{4}2^{\frac{%
%(n-1)\beta ^{\ast }}{\beta ^{\ast }-1}}\right) 2^{(p\sigma +\alpha
%_{1})n}[u]_{B_{p,\infty }^{\sigma }}^{p}=c_{3}C_{d}C_{0}[u]_{B_{p,\infty
%}^{\sigma }}^{p},  \label{eq3.14}
%\end{align}%
%where $C_{0}$ is finite by \eqref{c1}.

Finally, taking the supremum in $(0,R_{0}^{\beta ^{\ast }})$ on both sides
of \eqref{eq3.14}, we complete the proof.
\end{proof}

If $(M,d,\mu )$ admits a heat kernel $\{p_{t}\}_{t>0}$ satisfying %
(UHE) and (LHE), by Propositions \ref{lem3.1} and \ref%
{lem3.2}, we know that $\mathcal{D}(E_{p,p}^{\sigma })=B_{p,p}^{\sigma }$
and (\ref{eq1.8}) holds. We obtain $\mathcal{D}(E_{p,\infty }^{\sigma
})=B_{p,\infty }^{\sigma }$ and $E_{p,\infty }^{\sigma }(u)\asymp \lbrack
u]_{B_{p,\infty }^{\sigma }}^{p}$ from Propositions \ref{lem3.3} and \ref%
{lem3.4}. It remains to show $\Vert u\Vert _{KS_{p,\infty }^{\sigma
}}^{p}\asymp \limsup_{t\rightarrow 0}\Psi _{u}^{\sigma }(t)$ in (\ref{eq1.7}%
) for all $u\in B_{p,\infty }^{\sigma }$ when two-sided estimates %
(UHE) and (LHE) hold.

\begin{proposition}
\label{corol3.5} If $(M,d,\mu )$ admits a heat kernel $\{p_{t}\}_{t>0}$
satisfying (UHE) and (LHE), then for all $u\in B_{p,\infty
}^{\sigma }$,
\begin{equation}
\limsup_{t\rightarrow 0}\Psi _{u}^{\sigma }(t)\asymp \limsup_{r\rightarrow
0}\Phi _{u}^{\sigma }(r).  \label{limsup_psiphi}
\end{equation}
\end{proposition}

\begin{proof}
By Proposition \ref{lem3.3}, we have%
\begin{equation}
\sup_{r\in (0,R_{0})}\Phi _{u}^{\sigma }(r)\leq C\sup_{t\in (0,R_{0}^{\beta
^{\ast }})}\Psi _{u}^{\sigma }(t),  \label{31}
\end{equation}%
for any $R_{0}\in (0,$\textrm{diam}$(M)]$. Therefore, if we pick $%
R_{0}=\epsilon $ in \eqref{31}, then
\begin{align*}
\limsup_{t\rightarrow 0}\Psi _{u}^{\sigma }(t)& =\lim_{\epsilon \rightarrow
0}\sup_{t\in (0,\epsilon )}\Psi _{u}^{\sigma }(t) \\
& \geq C_{1}\lim_{\epsilon \rightarrow 0}\sup_{r\in (0,\epsilon ^{1/\beta
^{\ast }})}\Phi _{u}^{\sigma }(r)=C_{1}\limsup_{r\rightarrow 0}\Phi
_{u}^{\sigma }(r).
\end{align*}

Next, we consider the other side of \eqref{limsup_psiphi}.
We have by \eqref{eq3.3} and the dominated convergence theorem
\begin{align*}
& \limsup_{t\rightarrow 0}\frac{1}{t^{p\sigma /\beta ^{\ast }}}%
\int_{M}\int_{M}|u(x)-u(y)|^{p}p_{t}(x,y)d\mu (y)d\mu (x) \\
\leq & C_{1}\sum_{n=-\infty }^{\infty }\exp \left( -c_{4}2^{\frac{(n-1)\beta
^{\ast }}{\beta ^{\ast }-1}}\right) \limsup_{t\rightarrow 0}\frac{1_{\{n\leq
\log _{2}(R_{0}t^{-1/\beta ^{\ast }})\}}}{(2^{n}t^{1/\beta ^{\ast
}})^{p\sigma }}\int_{M}\frac{2^{(p\sigma +\alpha _{1})n}}{%
V(x,2^{n}t^{1/\beta ^{\ast }})}\int_{B(x,2^{n}t^{1/\beta ^{\ast
}})}|u(x)-u(y)|^{p}d\mu (y)d\mu (x) \\
\leq & C_{1}\sum_{n=-\infty }^{\infty }\exp \left( -c_{4}2^{\frac{(n-1)\beta
^{\ast }}{\beta ^{\ast }-1}}\right) 2^{(p\sigma +\alpha
_{1})n}\limsup_{r\rightarrow 0}\Phi _{u}^{\sigma }(r)\leq
C_{1}C_{0}\limsup_{r\rightarrow 0}\Phi _{u}^{\sigma }(r),
\end{align*}%
where $C_{1}=c_{3}C_{d}$ and the constant $C_{0}$ is defined in \eqref{c1},
which ends the proof.
\end{proof}

\begin{proof}[Proof of Lemma \protect\ref{thm1}]
By Propositions \ref{lem3.1} and \ref{lem3.2}, we obtain \eqref{eq1.8}. By
Propositions \ref{lem3.3}-\ref{corol3.5}, we obtain \eqref{eq1.7}.
\end{proof}

\subsection{Equivalence of properties (KE) and (NE)}

We start by an easy side $(NE)\Rightarrow (KE)$ which holds automatically.

\begin{proposition}
\label{thm_2} Suppose that $(M,d,\mu )$ admits a heat kernel $%
\{p_{t}\}_{t>0} $ satisfying (LHE). Then there exists $C>0$
such that for all $u\in \mathcal{D}(E_{p,\infty }^{\sigma })$,
\begin{equation*}
\Psi _{u}^{\sigma }(t)\geq C\Phi _{u}^{\sigma }(t^{1/\beta ^{\ast }}).
\end{equation*}%
Consequently,
\begin{equation}
\liminf_{t\rightarrow 0}\Psi _{u}^{\sigma }(t)\geq C\liminf_{r\rightarrow
0}\Phi _{u}^{\sigma }(r),  \label{limpsi}
\end{equation}
where the constant $C$ depends only on $c_{1}$ in (LHE).
\end{proposition}

\begin{proof}
By \eqref{eq3-2}, we have
\begin{align*}
\Psi _{u}^{\sigma }(t)& \geq \frac{1}{t^{p\sigma /\beta ^{\ast }}}%
\int_{M}\int_{B(x,\delta t^{1/\beta ^{\ast }})}|u(x)-u(y)|^{p}p_{t}(x,y)d\mu
(y)d\mu (x) \\
& \geq \frac{c_{1}C_{d}^{-1}\delta ^{\alpha _{1}}}{t^{p\sigma /\beta ^{\ast
}}}\int_{M}\frac{1}{V(x,\delta t^{1/\beta ^{\ast }})}\int_{B(x,\delta t^{{1}/%
{\beta ^{\ast }}})}|u(x)-u(y)|^{p}d\mu (y)d\mu (x) \\
& =c_{1}C_{d}^{-1}\delta ^{p\sigma +\alpha _{1}}\Phi _{u}(\delta t^{1/\beta
^{\ast }}).
\end{align*}%
Taking the lower limit on both sides of the above inequality, we have
\begin{equation*}
\liminf_{t\rightarrow 0}\Psi _{u}^{\sigma }(t)\geq c_{1}C_{d}^{-1}\delta
^{p\sigma +\alpha _{1}}\liminf_{t\rightarrow 0}\Phi _{u}^{\sigma
}(t^{1/\beta ^{\ast }})=c_{1}C_{d}^{-1}\delta ^{p\sigma +\alpha
_{1}}\liminf_{r\rightarrow 0}\Phi _{u}^{\sigma }(r),
\end{equation*}%
which implies \eqref{limpsi}.
\end{proof}

We need to use the following proposition adapted from the proof of \cite[%
Lemma 4.13]{AlonsoBaudoinchen2021CVPDE} (without proof) and we present its
proof here for later use.

\begin{proposition}
\label{hk_ct} Suppose that $M$ admits a heat kernel satisfying %
(UHE) and (LHE), then there exist $C,c>1$ and $c^{\prime
}>0 $ such that for any $\delta >0$ and for $\mu$-almost all $x,y\in M$ with
$d(x,y)>\delta t^{1/\beta ^{\ast }}$,
\begin{equation*}
p_{t}(x,y)\leq C\exp \left( -c^{\prime }\delta ^{\frac{\beta ^{\ast }}{\beta
^{\ast }-1}}\right) p_{ct}(x,y).
\end{equation*}
\end{proposition}

\begin{proof}
By (UHE) and volume doubling property \eqref{VD2}, we have
\begin{align*}
p_{t}(x,y)& \leq \frac{c_{3}}{V(x,t^{1/\beta ^{\ast }})}\exp \left(
-c_{4}\left( \frac{d(x,y)}{t^{1/\beta ^{\ast }}}\right) ^{\frac{\beta ^{\ast
}}{\beta ^{\ast }-1}}\right) \\
& \leq \frac{c_{3}C_{d}c^{\alpha _{1}/\beta ^{\ast }}}{V(x,(ct)^{1/\beta
^{\ast }})}\exp \left( -c_{4}\left( \frac{d(x,y)}{(ct)^{1/\beta ^{\ast }}}%
\right) ^{\frac{\beta ^{\ast }}{\beta ^{\ast }-1}}\cdot c^{\frac{1}{\beta
^{\ast }-1}}\right) \\
& =\frac{c_{3}C_{d}c^{\alpha _{1}/\beta ^{\ast }}}{c_{1}}\frac{c_{1}}{%
V(x,(ct)^{1/\beta ^{\ast }})}\exp \left( -c_{4}\left( \frac{d(x,y)}{%
(ct)^{1/\beta ^{\ast }}}\right) ^{\frac{\beta ^{\ast }}{\beta ^{\ast }-1}%
}\cdot c^{\frac{1}{\beta ^{\ast }-1}}\right) ,
\end{align*}%
for $\mu $-almost all $x,y\in M$. Taking $\tilde{c}=c_{4}c^{\frac{1}{\beta
^{\ast }-1}}-c_{2}$, then we use (LHE) to obtain
\begin{equation*}
p_{t}(x,y)\leq \frac{c_{3}C_{d}c^{\alpha _{1}/\beta ^{\ast }}}{c_{1}}\exp
\left( -\tilde{c}\left( \frac{d(x,y)}{(ct)^{1/\beta ^{\ast }}}\right) ^{%
\frac{\beta ^{\ast }}{\beta ^{\ast }-1}}\right) p_{ct}(x,y).
\end{equation*}%
We choose $c>1$ to ensure $c^{\frac{1}{\beta ^{\ast }-1}}c_{4}>c_{2}$ so
that $\tilde{c}>0$. Let $c^{\prime }={\tilde{c}}/{c^{\frac{1}{\beta ^{\ast
}-1}}}$, then for $d(x,y)>\delta t^{1/\beta ^{\ast }}$,
\begin{equation*}
p_{t}(x,y)\leq \frac{c_{3}C_{d}c^{\alpha _{1}/\beta ^{\ast }}}{c_{1}}\exp
\left( -c^{\prime }\left( \frac{d(x,y)}{t^{1/\beta ^{\ast }}}\right) ^{\frac{%
\beta ^{\ast }}{\beta ^{\ast }-1}}\right) p_{ct}(x,y)\leq C\exp \left(
-c^{\prime }\delta ^{\frac{\beta ^{\ast }}{\beta ^{\ast }-1}}\right)
p_{ct}(x,y),
\end{equation*}%
where $C=\frac{c_{3}C_{d}c^{\alpha _{1}/\beta ^{\ast }}}{c_{1}}$. The proof
is complete.
\end{proof}

The proof for the other side $(KE)\Rightarrow (NE)$ is inspired by \cite[%
Lemma 4.13]{AlonsoBaudoinchen2021CVPDE} where $p=1$ therein.

\begin{lemma}
\label{thm_1} Suppose that $(M,d,\mu )$ admits a heat kernel $%
\{p_{t}\}_{t>0} $ satisfying (UHE) and (LHE). Under the
property $(\widetilde{KE}) $ with $\sigma >0$, we have
\begin{equation}
\liminf_{t\rightarrow 0}\Psi _{u}^{\sigma }(t)\leq C\liminf_{r\rightarrow
0}\Phi _{u}^{\sigma }(r),  \label{limphiu}
\end{equation}
for all $u\in \mathcal{D}(E_{p,\infty }^{\sigma })$.
\end{lemma}

\begin{proof}
Temporally fix $u\in \mathcal{D}(E_{p,\infty }^{\sigma })$. Without loss of
generality, assume that $\liminf_{t\rightarrow 0}\Psi _{u}^{\sigma }(t)>0$.
Fix $\delta >1$ to be determined by (\ref{dd}). For $t\in (0,R_{0}^{\beta
^{\ast }})$, decompose
\begin{equation*}
\Psi _{u}^{\sigma }(t)=\Psi _{1}(t)+\Psi _{2}(t),
\end{equation*}%
where
\begin{align*}
\Psi _{1}(t)& =\frac{1}{t^{{p\sigma }/{\beta ^{\ast }}}}\iint_{\{d(x,y)\leq
\delta t^{1/\beta ^{\ast }}\}}|u(x)-u(y)|^{p}p_{t}(x,y)d\mu (y)d\mu (x), \\
\Psi _{2}(t)& =\frac{1}{t^{{p\sigma }/{\beta ^{\ast }}}}\iint_{\{d(x,y)>%
\delta t^{1/\beta ^{\ast }}\}}|u(x)-u(y)|^{p}p_{t}(x,y)d\mu (y)d\mu (x).
\end{align*}

For $\Psi _{1}(t)$, when $d(x,y)\leq \delta t^{1/\beta ^{\ast }}$, we have
\begin{align*}
\frac{c_{1}}{V(x,t^{1/\beta ^{\ast }})}\exp \left( -c_{2}\delta ^{\frac{%
\beta ^{\ast }}{\beta ^{\ast }-1}}\right) \leq & \frac{c_{1}}{V(x,t^{1/\beta
^{\ast }})}\exp \left( -c_{2}\left( \frac{d(x,y)}{t^{1/\beta ^{\ast }}}%
\right) ^{\frac{\beta ^{\ast }}{\beta ^{\ast }-1}}\right) \\
\leq & p_{t}(x,y)\leq \frac{c_{3}}{V(x,t^{1/\beta ^{\ast }})}\exp \left(
-c_{4}\left( \frac{d(x,y)}{t^{1/\beta ^{\ast }}}\right) ^{\frac{\beta ^{\ast
}}{\beta ^{\ast }-1}}\right) \leq \frac{c_{5}}{V(x,t^{1/\beta ^{\ast }})}.
\end{align*}%
Therefore, noting that $\delta>1$, we have from \eqref{VD2} that
\begin{align}
\Psi _{1}(t)& \leq \frac{c_{3}}{t^{p\sigma /\beta ^{\ast }}}%
\iint_{\{d(x,y)\leq \delta t^{1/\beta ^{\ast }}\}}\frac{|u(x)-u(y)|^{p}}{%
V(x,t^{1/\beta ^{\ast }})}d\mu (y)d\mu (x)  \notag \\
& \leq \frac{c_{3}C_{d}\delta ^{\alpha _{1}}}{t^{p\sigma /\beta ^{\ast }}}%
\iint_{\{d(x,y)\leq \delta t^{1/\beta ^{\ast }}\}}\frac{|u(x)-u(y)|^{p}}{%
V(x,\delta t^{1/\beta ^{\ast }})}d\mu (y)d\mu (x)=C_{1}\Phi _{u}^{\sigma
}(\delta t^{1/\beta ^{\ast }}),  \label{psi1}
\end{align}%
where $C_{1}=c_{3}C_{d}\delta ^{p\sigma +\alpha _{1}}$.

For $\Psi _{2}(t)$, by Proposition \ref{hk_ct}, we have
\begin{equation}
\Psi _{2}(t)\leq A\Psi _{u}^{\sigma }(ct),  \label{psi2}
\end{equation}%
where $A=Cc^{p\sigma /\beta ^{\ast }}\exp \left( -c^{\prime }\delta ^{\frac{%
\beta ^{\ast }}{\beta ^{\ast }-1}}\right) $ can be made as small as we
desire by making $\delta $ large enough since $C,c,c^{\prime }$ are
independent constants. So from now on we choose $\delta $ large enough such
that
\begin{equation}
A=Cc^{p\sigma /\beta ^{\ast }}\exp \left( -c^{\prime }\delta ^{\frac{\beta
^{\ast }}{\beta ^{\ast }-1}}\right) <\frac{1}{2}.  \label{dd}
\end{equation}
Combining \eqref{psi1} and \eqref{psi2}, we have%
\begin{equation}
\Psi _{u}^{\sigma }(t)\leq C_{1}\Phi _{u}^{\sigma }(\delta t^{1/\beta ^{\ast
}})+A\Psi _{u}^{\sigma }(ct).  \label{41}
\end{equation}

%Let $t=c^{-(n+1)}$, then
%\begin{equation*}
%\Psi _{u}^{\sigma }(c^{-(n+1)})\leq C_{1}\Phi _{u}^{\sigma }(\delta
%c^{-(n+1)/\beta ^{\ast }})+A\Psi _{u}^{\sigma }(c^{-n}).
%\end{equation*}%
%Since $\limsup_{r\rightarrow 0}\Phi _{u}^{\sigma }(r)<\infty $ by
%Proposition \ref{corol3.5} and property $(\widetilde{KE})$, there exists $%
%M_{0}$ such that
%\begin{equation*}
%\sup_{n\geq 1}\Phi _{u}^{\sigma }(\delta c^{-n/\beta ^{\ast }})\leq
%M_{0}<\infty .
%\end{equation*}%
%Picking up $M_{1}=\max \{C_{1}M_{0},\Psi _{u}^{\sigma }(c^{-1})\}$, we have $%
%\Psi _{u}^{\sigma }(c^{-2})\leq (1+A)M_{1}$.
%
%By induction, we have
%\begin{equation*}
%\Psi _{u}^{\sigma }(c^{-n})\leq M_{1}\left( \frac{1-A^{n}}{1-A}\right) .
%\end{equation*}%
%Letting $n\rightarrow \infty $, we obtain
%\begin{equation*}
%\liminf_{t\rightarrow 0}\Psi _{u}^{\sigma }(t)<\infty .
%\end{equation*}

By property $(\widetilde{KE})$, there exists $t_{0}>0$ such that for all $%
t\in (0,t_{0})$,
\begin{equation*}
\Psi _{u}^{\sigma }(t)\leq \sup_{0<t<t_{0}}\Psi _{u}^{\sigma }(t)\leq
2\limsup_{t\rightarrow 0}\Psi _{u}^{\sigma }(t)\leq
C_{2}\liminf_{t\rightarrow 0}\Psi _{u}^{\sigma }(t),
\end{equation*}%
where $C_{2}$ comes from property $(\widetilde{KE})$. Therefore, we have
from \eqref{41} that, for all small enough $t\in (0,\frac{t_{0}}{c})$,
\begin{equation*}
\Psi _{u}^{\sigma }(t)\leq C_{1}\Phi _{u}^{\sigma }(\delta t^{1/\beta ^{\ast
}})+AC_{2}\liminf_{t\rightarrow 0}\Psi _{u}^{\sigma }(t).
\end{equation*}%
Fixing $\delta $ such that $AC_{2}<1$ with aforementioned requirement $A<%
\frac{1}{2}$, it follows that
\begin{equation*}
\liminf_{t\rightarrow 0}\Psi _{u}^{\sigma }(t)\leq
C_{1}\liminf_{t\rightarrow 0}\Phi _{u}^{\sigma
}(t)+AC_{2}\liminf_{t\rightarrow 0}\Psi _{u}^{\sigma }(t),
\end{equation*}%
which implies
\begin{equation*}
\liminf_{t\rightarrow 0}\Psi _{u}^{\sigma }(t)\leq
C_{3}\liminf_{t\rightarrow 0}\Phi _{u}^{\sigma }(t),
\end{equation*}%
where $C_{3}=\frac{C_{1}}{1-AC_{2}}$. The proof is complete.
\end{proof}

\begin{proof}[Proof of Theorem \protect\ref{thm4}]
Since two-sided estimates (UHE) and (LHE) hold, using Lemma %
\ref{thm1}, we have that $\mathcal{D}(E_{p,\infty }^{\sigma })=B_{p,\infty
}^{\sigma }$ for any $\sigma >0$.

By \eqref{limsup_psiphi} and \eqref{limphiu}, $(\widetilde{KE})$ implies $(%
\widetilde{NE})$ since for all $u\in B_{p,\infty }^{\sigma }=\mathcal{D}%
(E_{p,\infty }^{\sigma })$,
\begin{equation}
\limsup_{r\rightarrow 0}\Phi _{u}^{\sigma }(r)\leq C\limsup_{t\rightarrow
0}\Psi _{u}^{\sigma }(t)\leq C^{\prime }\liminf_{t\rightarrow 0}\Psi
_{u}^{\sigma }(t)\leq C^{\prime \prime }\liminf_{r\rightarrow 0}\Phi
_{u}^{\sigma }(r).  \label{ke_ne1}
\end{equation}

By \eqref{limsup_psiphi} and \eqref{limpsi}, $(\widetilde{NE})$ implies $(%
\widetilde{KE})$ since
\begin{equation}
\limsup_{t\rightarrow 0}\Psi _{u}^{\sigma }(t)\leq
C_{1}\limsup_{r\rightarrow 0}\Phi _{u}^{\sigma }(r)\leq C_{1}^{\prime
}\liminf_{r\rightarrow 0}\Phi _{u}^{\sigma }(r)\leq C_{1}^{\prime \prime
}\liminf_{t\rightarrow 0}\Psi _{u}^{\sigma }(t).  \label{ne_ke1}
\end{equation}%
Therefore, by \eqref{ke_ne1} and \eqref{ne_ke1}, we show Theorem \ref{thm4}
(i).

By \eqref{limphiu} and Proposition \ref{lem3.3}, (KE) implies (NE) since
\begin{equation}
\sup_{r\in (0,R_{0})}\Phi _{u}^{\sigma }(r)\leq C_{2}\sup_{t\in
(0,R_{0}^{\beta ^{\ast }})}\Psi _{u}^{\sigma }(t)\leq C_{2}^{\prime
}\liminf_{t\rightarrow 0}\Psi _{u}^{\sigma }(t)\leq C_{2}^{\prime \prime
}\liminf_{r\rightarrow 0}\Phi _{u}^{\sigma }(r),  \label{ke_ne2}
\end{equation}%
where we use the simple fact (KE)$\Rightarrow (\widetilde{KE})$. Similarly,
by \eqref{limpsi} and Proposition \ref{lem3.4} adjoint with the fact (NE)$%
\Rightarrow (\widetilde{NE})$, we immediately derive (KE) since
\begin{equation}
\sup_{t\in (0,R_{0}^{\beta ^{\ast }})}\Psi _{u}^{\sigma }(t)\leq
C_{3}\sup_{r\in (0,R_{0})}\Phi _{u}^{\sigma }(r)\leq C_{3}^{\prime
}\liminf_{r\rightarrow 0}\Phi _{u}^{\sigma }(r)\leq C_{3}^{\prime \prime
}\liminf_{t\rightarrow 0}\Psi _{u}^{\sigma }(t).  \label{ne_ke2}
\end{equation}%
Thus, Theorem \ref{thm4} (ii) holds by \eqref{ke_ne2} and \eqref{ne_ke2}.
\end{proof}

\begin{remark}
We establish the equivalence of properties (KE) and (NE)
under the full two-sided heat kernel estimates (UHE) and (LHE).  We say that
a heat kernel $\{p_{t}\}_{t>0}$ satisfies near diagonal lower estimate
(NLE), if there exists $\delta \in (0,\infty)$ such that, for all $t\in
(0,R_{0}^{\beta ^{\ast}}) $ and $\mu $-almost all $x,y\in M$ :
\begin{equation}
p_{t}(x,y)\geq \frac{c_{1}}{V(x,t^{1/\beta ^{\ast }})}\ \ \ \text{whenever }
\ d(x,y)\leq \delta t^{1/\beta ^{\ast }}.  \label{NLE}
\end{equation}
We conjecture that Lemma \ref{thm1} and Lemma \ref{thm_1} cannot be derived
from (UHE) and (NLE) without further assumptions (like the chain condition).
\end{remark}

By Theorem \ref{thm4} and Remark \ref{rk1}, we have the following
interesting corollary for $p=2$, which is important for defining a local
Dirichlet form based on Besov norms.

\begin{corollary}
\label{jjjj} Suppose that $(M,d,\mu )$ admits a heat kernel $\{p_{t}\}_{t>0}$
satisfying (UHE) and (LHE), then (NE) holds when $p=2$.
\end{corollary}

\section{Connected homogeneous p.c.f self-similar sets and their glue-ups}

\label{sec5} In this section, we work with `fractal glue-ups' using
connected homogeneous p.c.f self-similar sets as `tiles', which can be both
bounded and unbounded. We first obtain their equivalent discrete semi-norms
(vertex energies). Then we introduce weak-monotonicity properties for vertex
energies, including property (E) in \cite[Definition 3.1]{GaoYuZhang2022PA},
properties (VE) and $(\widetilde{VE})$. We will verify property (E) on
nested fractals, and our main goal is to show the equivalence of (KE) and
(VE). Lastly, we show that aforementioned consequences hold for certain
`fractal glue-ups', for example, nested fractals and their fractal blow-ups.

\subsection{Equivalent discrete semi-norms of fractal glue-ups}

\label{subsec5.1} Let us first introduce connected homogeneous p.c.f.
self-similar sets in $\mathbb{R}^{d}$. Let $\{\phi _{i}\}_{i=1}^{N}$ be an
IFS which consists of $\rho $-similitudes on $\mathbb{R}^{d}$, i.e. for $%
i=1,...,N$, $N\geq 2$, each $\phi _{i}:\mathbb{R}^{d}\rightarrow \mathbb{R}%
^{d}$ is of the form
\begin{equation}
\phi _{i}(x)=\rho (x-b_{i})+b_{i},  \label{phi_i}
\end{equation}%
where $\rho \in (0,1)$, $b_{i}\in \mathbb{R}^{d}$. Suppose that $K$ is the
attractor of the IFS $\{\phi _{i}\}_{i=1}^{N}$, i.e. $K$ is the unique
non-empty compact set in $\mathbb{R}^{d}$ satisfying
\begin{equation*}
K=\tbigcup_{i=1}^{N}\phi _{i}(K).
\end{equation*}%
To explain what `p.c.f' is, we introduce the natural symbolic space
associated with our IFS $\{\phi _{i}\}_{i=1}^{N}$. Let $W=\{1,2,...,N\}$, $%
W_{n}$ be the set of words with length $n$ over $W$, and $W^{\mathbb{N}}$ be
the set of all infinite words over $W$. Let $w=\mathrm{w_{1}w_{2}}...\in W^{%
\mathbb{N}}$ and let the canonical projection $\pi :$ $W^{\mathbb{N}%
}\rightarrow K$ be defined by $\pi (w):=\tbigcap\limits_{n\in \mathbb{N}%
^{\ast }}F_{\mathrm{w_{1}...w_{n}}}(K),$ where $F_{\mathrm{w_{1}...w_{n}}%
}:=F_{\mathrm{w_{1}}}\circ ...\circ F_{\mathrm{w_{n}}}$, and $K_w=F_w(K)$.
Then we can define the critical set $\Gamma $ and the post-critical set $%
\mathcal{P}$ by
\begin{equation*}
\Gamma =\pi ^{-1}\left( \tbigcup_{1\leq i<j\leq N}\left( \phi _{i}(K)\cap
\phi _{j}(K)\right) \right) ,\quad \mathcal{P}=\tbigcup_{m\geq 1}\tau
^{m}(\Gamma ),
\end{equation*}%
where $\tau $ is the left shift by one index on $W^{\mathbb{N}}$ (see \cite[%
Definition 1.3.13]{Kigami.2001.}). We say the IFS is \emph{post-critically
finite} (p.c.f.) if $\mathcal{P}$ is finite. Define
\begin{equation*}
V_{0}=\pi (\mathcal{P}),\quad V_{w}:=F_{w}\left( V_{0}\right), \quad V_{n}=\tbigcup_{w\in W_{n}}V_{w} ,\quad V_{\ast }=\tbigcup_{n\geq 1}V_{n},
\end{equation*}%
then $K$ is the closure of $V_{\ast }$ with respect to the Euclidean metric. For $w\in W_n$, $K_{w}:=F_{w}\left( K\right) $ is called a $n$-cell of $K$. Without loss of generality, we always assume that
\textrm{diam}$(K)=1$ by an affine transformation on $\{b_{i}\}_{i=1}^{N}$.
From now on, denote by $K$ a homogeneous p.c.f. self-similar set that is
connected. It is known that a homogeneous p.c.f. IFS in the form of (\ref%
{phi_i}) satisfies the \emph{open set condition} (OSC) (see, for example
\cite[Theorem 1.1]{Denglau.2008}). Hence the Hausdorff dimension of $K$ is $%
\alpha =\func{dim}_{H}(K)=-\log N/\log \rho$, and the $\mathcal{H}^{\alpha }$%
-measure of $K$, denoted by $\mu $, is \emph{$\alpha $-regular}.

It is proved in \cite[Proposition 2.5]{GuLau.2020.AASFM} that, \emph{%
Condition(H)} holds for connected homogeneous p.c.f. self-similar sets. That
is, there exists $c>0$ depending only on $K$ such that, for any two words $w$
and $w^{\prime }$ with the same length $m\geq 1$, $K_{w}\cap K_{w^{\prime
}}=\emptyset $ implies that $\func{dist}\left( K_{w},K_{w^{\prime }}\right)
\geq c\rho ^{m}$.

In order to unify the notation for bounded and unbounded fractals, we
introduce the concept of \emph{fractal glue-ups}. The definition is adapted
from the idea of fractafold in \cite{StrichartzTeplyaev2012} but not the
same. Given a compact set $K\subset \mathbb{R}^d$ and a set $F$ of
similitudes on $\mathbb{R}^d$, we can define
\begin{equation*}
K^F:=\bigcup_{f\in F}f(K)\subset \mathbb{R}^d.
\end{equation*}
For our study, we naturally need the following requirements.

\begin{enumerate}
\item Isometries: all similitudes in $F$ have contraction ratio 1, that is,
each similitude in $F$ is an isometry of $K$.

\item Just-touching property: for any $f\neq g\in F$,
\begin{equation*}
f(K)\tbigcap g(K)=f(V_0)\tbigcap g(V_0).
\end{equation*}

\item Condition (H): There exists a constant $C_{H}\in (0,1)$ such that, if $%
|x-y|<C_{H}\rho ^{m}$ with $x\in f_{1}(K_{w})$ and $y\in f_{2}(K_{\tilde{w}%
}) $ for two words $w$ and $\tilde{w}$ with the same length $m\geq0$, then $%
f_{1}(K_{w})$ intersects $f_{2}(K_{\tilde{w}})$.

\item Connectedness: $K^{F}$ is connected.
\end{enumerate}

%These conditions are natural settings in the study of homogeneous p.c.f. fractals. Condition (1) resembles the homogeneity on large scales; Condition (2) resembles the p.c.f. property;Condition (3) is satisfied for each tile $K$ and is widely used in previous studies.

From now on, whenever we mention a \emph{fractal glue-up} $K^{F}$, we
automatically assume that these four conditions are satisfied. We call each $%
f(K),f\in F$ a \emph{tile}, and we just simply glue these tiles up which
only overlap at boundaries by the `just-touching' property. Conditions (2)
and (3) imply the following \emph{uniform finitely-joint} property.

\begin{proposition}[Uniform finitely-joint property]
For any metric ball $B(x,R)$ of radius $R>0$ in $K^F$, the number of tiles
that intersect it is bounded by a constant $C_K(R)<\infty$, which depends
only on $C_H$ and the IFS. Thus for any tile $f(K)$, the number of tiles $%
\tilde{f}(K)$ that intersect it is bounded by $C_K(1)$.
\end{proposition}
\begin{proof}
Denote all the tiles that intersect $B(x,R)$ by $f_{n}(K),\ 1\leq n\leq
N_{0} $ (here $N_{0}$ might be $\infty $). Denote by $t$ the smallest
integer satisfying
\begin{equation*}
N^t>|V_0|.
\end{equation*}
Choose an `interior' (away from boundaries) level-$t$ cell $\phi_{w}(K)$ ($%
|w|=t$) that does not intersect $V_{0}$, which exists by the pigeonhole
principle. Condition (H) implies that all $f_{n}(\phi _{w}(K))$ are
separated by distance $C_{H}\rho ^{t}$ for $1\leq n\leq N_{0}$, but they all
belong to the ball $B(x,R+1)$ since the diameter of the tiles are all 1.
This means that $N_{0}$ is uniformly bounded, since there are $N_{0}$
disjoint balls $B(f_{n}(\phi _{w}(v)),C_{H}\rho ^{t}/2)$ inside a ball with
diameter $R+1$ in $\mathbb{R}^d$, that is, $C_K(R)$ is the maximal number of
disjoint balls of radii $C_{H}\rho ^{t}/2$ inside a ball with diameter $R+1$
in $\mathbb{R}^d$. The second inclusion directly follows by choosing $R=1$
and $x\in f(K)$.
\end{proof}

\begin{remark}
The concept of the fractal glue-up provides a quick way to unify the notions
of bounded and unbounded fractals. For example, $K=K^{F}$ when $F=\{Id\}$.
We are mainly interested in the following unbounded fractal by blowing up $K$
defined by Strichartz in \cite[Section 3]{Strichartz.1998.CJM}. However,
fractal glue-ups lack self-similarity on large scales in general.
\end{remark}

\begin{example}[Fractal blow-up]\label{ex1} Let $\{\phi _{i}\}_{i=1}^{N}$ be defined as in (\ref{phi_i}). Assume that
\textrm{diam}$(K)=1$ by an affine transformation on $\{b_{i}\}_{i=1}^{N}$ with $b_{1}=0$ and define
\begin{equation*}
K_{l}:=\underbrace{\phi _{1}^{-1}\circ \phi _{1}^{-1}\circ \cdots \circ \phi
_{1}^{-1}}_{l}K=\rho^{-l}K=\tbigcup _{f\in F_{l}}f(K),
\end{equation*}%
where $F_{l}=\left\{x+\sum_{j=1}^{l}\rho ^{-j}c_{j}: c_{j}\in \{(1-\rho)b_{i}\}_{i=1}^{N}\right\}$.

Clearly, $F_{l}$ and $K_{l}$ are increasing sequences, so that
$K=K_{0}\subseteq K_{1}\subseteq K_{2}\subseteq \cdots $.

The final fractal blow-up of $K$ is in the form $K^{F}$ with $F=\bigcup_{l=1}^{\infty }
F_{l}= \lim_{l\rightarrow\infty} F_l$, and denote it by
\begin{equation*}
K_{\infty }=\tbigcup _{l=1}^{\infty }K_{l}= \lim_{l\rightarrow\infty} K_l.
\end{equation*}\end{example}
%We illustrate the fractal blow-up of the Sierpi\'{n}ski gasket $K$ (the
%highlighted area) in Fig.1.
%\begin{figure}[tbph]
%\flushright
%\includegraphics[width=8cm]{fig2.jpg}\newline
%\caption{The blow-up of the Sierpi\'{n}ski gasket}
%\label{fig1}
%\end{figure}

From now on, we always assume that $K$ is a connected homogeneous p.c.f.
self-similar set when we say that $K^{F}$ is a fractal glue-up. In what follows, we use
the superscript $F$ to distinguish the underlying space $K$ and $K^{F}$ for
the energies and norms etc.

Define
\begin{equation}
\mu ^{F}:=\sum_{f\in F}\mu \circ f^{-1},  \label{muF}
\end{equation}%
i.e., just `glue' up the measures for each tile. Then $\mu
^{F}$ is $\alpha $-regular by our assumptions on $F$. Following the definition of Besov semi-norms on $%
K$, denote Besov semi-norms on $K^{F}$ by
\begin{equation}
\lbrack u]_{B_{p,\infty }^{\sigma ,F}}^{p}:=\sup_{n\geq 0}\Phi
_{u}^{\sigma ,F}(C_H\rho ^{n}),  \label{ct}
\end{equation}%
where $\Phi _{u}^{\sigma ,F}(r):=r^{-(p\sigma +\alpha
)}\int_{K^{F}}\int_{B(x,r)}|u(x)-u(y)|^{p}d\mu ^{F}(y)d\mu ^{F}(x)$, and
define
\begin{equation*}
B_{p,\infty }^{\sigma ,F}:=\{u\in L^{p}(K^{F},\mu ^{F}):[u]_{B_{p,\infty
}^{\sigma ,F}}<\infty \}.
\end{equation*}%
It is easy to see that%
\begin{equation}
\sup_{n\geq 0}\Phi _{u}^{\sigma ,F}(C_H\rho ^{n})\asymp
\sup_{r\in (0,C_H)}\Phi _{u}^{\sigma ,F}(r),  \label{ns}
\end{equation}%
this means%
\begin{equation}
B_{p,\infty }^{\sigma ,F}=B_{p,\infty }^{\sigma }(K^{F})\text{ \ \ and\ \ \ }%
[u]_{B_{p,\infty }^{\sigma ,F}}\asymp \lbrack u]_{B_{p,\infty }^{\sigma
}(K^{F})}\text{ (with }R_{0}=C_H\text{ in (\ref{pi})).}
\label{BC}
\end{equation}
Also, we have%
\begin{equation}
\liminf_{n\rightarrow \infty }\Phi _{u}^{\sigma ,F}(\rho ^{n})\asymp
\liminf_{r\rightarrow 0}\Phi _{u}^{\sigma ,F}(r)\text{ \ \ and \ \ }%
\limsup_{n\rightarrow \infty }\Phi _{u}^{\sigma ,F}(\rho ^{n})\asymp
\limsup_{r\rightarrow 0}\Phi _{u}^{\sigma ,F}(r).  \label{ne}
\end{equation}

We need the following Morrey-Sobolev inequality to show that $B_{p,\infty
}^{\sigma }(K^F)$ essentially embeds into $C(K^F)$ when $\sigma >\alpha /p$.

\begin{lemma}
(\cite[Theorem 3.2]{BaudoinLecture2022}) \label{lemmaMR} Let $(M,d,\mu )$ be
a metric measure space with $\mu $ satisfying (\ref{alpha_r}). When $\sigma
>\alpha /p$, for any $u\in B_{p,\infty }^{\sigma }(M)$, there exists a
continuous version $\tilde{u}\in C^{(p\sigma -\alpha )/p}(M)$ satisfying $%
\tilde{u}=u$ $\mu $-almost everywhere in $M$ and
\begin{equation}
|\tilde{u}(x)-\tilde{u}(y)|\leq C|x-y|^{(p\sigma -\alpha )/p}\sup_{r\in
(0,3d(x,y)]}\Phi _{u}^{\sigma }(r)\leq C|x-y|^{(p\sigma -\alpha
)/p}[u]_{B_{p,\infty }^{\sigma }(M)},  \label{M-S}
\end{equation}%
for all $x,$ $y\in M$ with $d(x,y)<R_{0}/3$, where $C$ is a positive
constant (Here $C^{\beta }(M)$ denotes the class of H\"{o}lder continuous
functions of order $\beta $ on $M$ ).
\end{lemma}

%
%
%
%
%
%
%
%Clearly, \textbf{[[Check, but we don't need it]] }for all sufficient large $%
%k $
%\begin{equation}
%\sup_{n\geq k}\rho ^{-n(p\sigma +\alpha )}I_{\infty ,n}^{F}(u)\asymp
%\sup_{n\geq k}\rho ^{-n(p\sigma +\alpha )}\int_{K^{F}}\int_{B(x,\rho
%^{n})}|u(x)-u(y)|^{p}d\mu ^{F}(y)d\mu ^{F}(x).  \label{II}
%\end{equation}
Next, we estimate%
\begin{equation*}
I_{\infty ,n}^{F}(u):=\int_{K^{F}}\int_{B(x,C_{H}\rho
^{n})}|u(x)-u(y)|^{p}d\mu ^{F}(y)d\mu ^{F}(x).
\end{equation*}
By (\ref{ct}), for all $u\in B_{p,\infty }^{\sigma ,F}$,
\begin{equation}
\rho ^{-n(p\sigma +\alpha )}I_{\infty ,n}^{F}(u)=C_{H}^{p\sigma +\alpha
}\Phi _{u}^{\sigma ,F}(C_{H}\rho ^{n}).  \label{ct-1}
\end{equation}
Denote the Borel measure $\mu _{m}$ on $V_{m}$ by
\begin{equation*}
\mu _{m}:=\frac{1}{|V_{m}|}\sum_{a\in V_{m}}\delta _{a},
\end{equation*}%
and define
\begin{equation}
\mu _{m}^{F}:=\sum_{f\in F}\mu_m\circ f^{-1}=\sum_{f\in F}\frac{1}{|V_{m}|}%
\sum_{a\in V_{m}}\delta _{f(a)},  \label{200}
\end{equation}%
where $\delta _{a}$ is the Dirac measure at point $a$ and $|A|$ denotes the
cardinality of a finite set $A$.

Let
\begin{equation}
I_{m,n}^{F}(u):=\int_{K^{F}}\int_{B(x,C_{H}\rho ^{n})}|u(x)-u(y)|^{p}d\mu
_{m}^{F}(y)d\mu _{m}^{F}(x),  \label{201}
\end{equation}%
which can be regarded as the `ball-energy'. We will use these discrete sums
to estimate the integral norms. To do this, we need the following lemma.

\begin{lemma}
\label{lem4.4}Let $K^{F}$ be a fractal glue-up. Then for all $u\in C(K^F)$,
\begin{equation}
\liminf_{m\rightarrow \infty }I_{m,n}^{F}(u)\geq I_{\infty ,n}^{F}(u).
\label{mu_m}
\end{equation}
\end{lemma}

The proof of this lemma is based on the proof of \cite[Lemma 3.2]%
{GuLau.2020.AASFM}, however, a formula used in \cite[Lemma 3.2]%
{GuLau.2020.AASFM} is not obvious. We add an argument for this formula to
ensure the validity of \cite[Lemma 3.2]{GuLau.2020.AASFM}. We may assume
that $K$ is not contained in any hyperplane $H\subset \mathbb{R}^{d}$,
otherwise we may reduce the IFS to $H$ by restricting each similitude $\phi
_{i}$ to $H$, and a simple affine transformation on $H$ to its dimension $%
d^{\prime }$ gives a scaled copy of $K$ to $\mathbb{R}^{d^{\prime }}$. Thus
we can apply \cite[Proposition 2.4]{GuLau.2020.AASFM}.

\begin{proposition}
Assume that $K$ is not contained in any hyperplane $H\subset\mathbb{R}^{d}$.
For any $r>0$, let $E(r):=\bigcup_{x\in K}\{x\}\times \partial B(x,r).$ Then
$\mu \times \mu (E(r))=0$.
\end{proposition}

\begin{proof}
Define the closed annulus
\begin{equation*}
R_{x}(a,b):=\overline{B(x,a)}\setminus B(x,b).
\end{equation*}%
Fix $x_{w}\in K_{w}$ where $w\in W_n$. Then
\begin{equation*}
\bigcup_{x\in K_{w}}\partial B(x,r)\subset R_{x_{w}}(r-\rho ^{n},r+\rho
^{n}),
\end{equation*}%
since for any $y\in \partial B(x,r)$, $d(x,y)=r$ and $d(x,x_{w})\leq \rho
^{n}$, thus $d(y,x_{w})\in \lbrack r-\rho ^{n},r+\rho ^{n}].$ Note that
\begin{equation*}
E(r):=\bigcup_{x\in K}\{x\}\times \partial B(x,r)\subset \bigcup_{w\in
W_{n}}K_{w}\times R_{x_{w}}(r-\rho ^{n},r+\rho ^{n}),
\end{equation*}
thus
\begin{align*}
\mu \times \mu (E(r))& \leq \sum_{w\in W_{n}}\mu (K_{w})\mu
(R_{x_{w}}(r-\rho ^{n},r+\rho ^{n})) \\
&\leq \mu (K)\sup_{w\in W_{n}}\mu (R_{x_{w}}(r-\rho ^{n},r+\rho ^{n})) \\
&\leq \mu (K)\sup_{x\in K}\mu (R_{x}(r-\rho ^{n},r+\rho ^{n})).
\end{align*}

We claim that
\begin{equation*}
\sup_{x\in K}\mu (R_{x}(r-\rho ^{n},r+\rho ^{n}))\rightarrow 0\text{ as }%
n\rightarrow \infty .
\end{equation*}%
Otherwise, if $\mu (R_{x_{n}}(r-\rho ^{n},r+\rho ^{n}))\geq \delta >0$
for a subsequence $\{x_{n}\}_n\subset K$, we may choose a subsequence $%
\{x_{n_{i}}\}_i$ that converge to $x_{\infty }\in K$ due to the compactness
of $K$, and further require that $d(x_{n_{i}},x_{\infty })$ decrease in $i$.
Then
\begin{equation*}
A_{i}:=R_{x_{\infty }}(r-\rho ^{n_{i}}-d(x_{n_{i}},x_{\infty }),r+\rho
^{n_{i}}+d(x_{n_{i}},x_{\infty }))\supset R_{x_{n_{i}}}(r-\rho
^{n_{i}},r+\rho ^{n_{i}})
\end{equation*}%
and $A_{i}\downarrow \partial B(x_{\infty },r)$, thus
\begin{equation*}
\mu (\partial B(x_{\infty },r))=\lim_{i\rightarrow \infty }\mu (A_{i})\geq
\delta ,
\end{equation*}%
which contradicts \cite[Proposition 2.4]{GuLau.2020.AASFM} that $\mu
(\partial B(x,r))=0$ for any $x\in K$! The proof is done by this claim.
\end{proof}

\begin{proof}[Proof of Lemma \protect\ref{lem4.4}]
Suppose $F=\{f_{i}\}_{i\geq 1}$ and denote $F_{k}=\{f_{i}\}_{1\leq i\leq k}$%
. As $\mu _{m}$ weak $\ast $-converges to $\mu $, we know by definition that
the finite glue-up measure $\mu _{m}^{F_{k}}$ weak $\ast $-converges to $\mu
^{F_{k}}$ on $K^{F_{k}}$ (the pieces $f_{i}(K)$ are mutually measure-disjoint),
and that $\mu ^{F_{k}}$ is exactly the restriction of $\mu ^{F}$ to $K^{F_{k}}$.
Using the same argument as \cite[Lemma 3.2]{GuLau.2020.AASFM}, for $u\in C(K^F)$, we have
\begin{eqnarray*}
&&\lim_{m\rightarrow \infty }I_{m,n}^{F_{k}}(u)=\lim_{m\rightarrow \infty
}\int_{K^{F_{k}}}\int_{B(x,C_{H}\rho ^{n})}|u(a)-u(b)|^{p}d\mu
_{m}^{F_{k}}(b)d\mu _{m}^{F_{k}}(a) \\
&=&\int_{K^{F_{k}}}\int_{B(x,C_{H}\rho ^{n})}|u(a)-u(b)|^{p}d\mu
^{F_{k}}(b)d\mu ^{F_{k}}(a):=I_{\infty ,n}^{F_{k}}(u).
\end{eqnarray*}

Note that the nonnegative sequence $a_{m,k}:=I_{m,n}^{F_{k}}(u)\uparrow
I_{m,n}^{F}(u):=a_{m,\infty }$ when $k\rightarrow \infty $, and we have just
shown that $\lim_{m\rightarrow \infty }a_{m,k}=I_{\infty ,n}^{F_{k}}(u)$.
Denote $a_{m,0}=I_{\infty ,n}^{F_{0}}(u)=0$, then
\begin{equation}
I_{m,n}^{F}(u)=\sum_{k=0}^{\infty }(a_{m,k+1}-a_{m,k})  \label{4.8-1}
\end{equation}%
and by noting that
\begin{equation*}
\liminf_{m\rightarrow \infty }(a_{m,k+1}-a_{m,k})=I_{\infty
,n}^{F_{k+1}}(u)-I_{\infty ,n}^{F_{k}}(u),
\end{equation*}%
our conclusion directly follows from Fatou's Lemma.
\end{proof}

We are now in the position for the first estimate lemma.

\begin{lemma}
\label{lem6.2} Let $K^{F}$ be a fractal glue-up. If $\sigma >\alpha /p$,
then for all $u\in B_{p,\infty }^{\sigma ,F}$,
\begin{equation}
I_{\infty ,n}^{F}(u)\leq C\rho ^{2n\alpha }\sum_{k=n}^{\infty
}E_{k}^{(p),F}(u)\leq C^{\prime }\rho ^{n(p\sigma +\alpha )}\sup_{k\geq n}%
\mathcal{E}_{k}^{\sigma ,F}(u).  \label{eq6.3}
\end{equation}
\end{lemma}

\begin{proof}
Let $I_{m,n}^{F}(u)$ be defined as in (\ref{201}) ($m>n)$. Since $K^{F}$ is
connected and satisfies the condition $(\mathrm{H})$, $|x-y|\leq C_{H}\rho
^{n}$ implies that\ $x,y$ lie in the same or the neighboring $n$-cells. For
every $w\in W_{n}$ and $x\in f(K_{w})$, if $y\in B(x,C_{H}\rho ^{n})$, then
there exists $(g,\tilde{w})\in F\times W_{n}$ such that $y\in g(K_{\tilde{w}%
})$ and
\begin{equation*}
g(K_{\tilde{w}})\cap f(K_{w})\neq \emptyset .
\end{equation*}%
Therefore,
\begin{align*}
I_{m,n}^{F}(u)\leq & \sum_{|w|=n}\sum_{f\in F}\int_{f(K_{w})}\left( \sum_{|%
\tilde{w}|=n}\sum_{g\in F}\int_{g(K_{\tilde{w}})}|u(x)-u(y)|^{p}1_{\{g(K_{%
\tilde{w}})\cap f(K_{w})\neq \emptyset \}}d\mu _{m}^{F}(y)\right) d\mu
_{m}^{F}(x) \\
=& \sum_{|w|=n}\sum_{f\in F}\sum_{|\tilde{w}|=n}\sum_{g\in F}\sum_{x\in
f(K_{w}\cap V_{m})}\sum_{y\in g(K_{\tilde{w}}\cap V_{m})}\frac{1_{\{g(K_{%
\tilde{w}})\cap f(K_{w})\neq \emptyset \}}}{|V_{m}|^{2}}|u(x)-u(y)|^{p}.
\end{align*}%
When $g(K_{\tilde{w}})\cap f(K_{w})\neq \emptyset $, we can find a common
vertex of these two cells and denote it by $z\in f(V_{w})\cap g(V_{\tilde{w}%
})$, due to the just-touching property when $f\neq g$ and the p.c.f.
structure when $f=g$. Furthermore, by the elementary inequality $%
|u(x)-u(y)|^{p}\leq 2^{p-1}\left( |u(x)-u(z)|^{p}+|u(z)-u(y)|^{p}\right) $,
\begin{equation*}
1_{\{g(K_{\tilde{w}})\cap f(K_{w})\neq \emptyset \}}|u(x)-u(y)|^{p}\leq
2^{p-1}\sum_{z\in f(V_{w})\cap g(V_{\tilde{w}})}\left(
|u(x)-u(z)|^{p}+|u(z)-u(y)|^{p}\right) .
\end{equation*}%
It follows that
\begin{align}
I_{m,n}^{F}(u)\leq & 2^{p-1}\sum_{|w|=n}\sum_{f\in F}\sum_{|\tilde{w}%
|=n}\sum_{g\in F}\sum_{x\in f(K_{w}\cap V_{m})}\sum_{y\in g(K_{\tilde{w}%
}\cap V_{m})}\sum_{z\in f(V_{w})\cap g(V_{\tilde{w}})}\frac{1}{|V_{m}|^{2}}%
\left( |u(x)-u(z)|^{p}+|u(z)-u(y)|^{p}\right)  \notag \\
\leq & C_{1}N^{-2m}\sum_{|w|=n}\sum_{f\in F}\sum_{|\tilde{w}|=n}\sum_{g\in
F}\sum_{z\in f(V_{w})\cap g(V_{\tilde{w}})}\sum_{x\in f(K_{w}\cap
V_{m})}\sum_{y\in g(K_{\tilde{w}}\cap V_{m})}\left(
|u(x)-u(z)|^{p}+|u(z)-u(y)|^{p}\right)  \notag \\
\leq & C_{1}N^{-2m}\sum_{|w|=n}\sum_{f\in F}\sum_{z\in f(V_{w})}\sum_{x\in
f(K_{w}\cap V_{m})}|u(x)-u(z)|^{p}\left( \sum_{|\tilde{w}|=n}\sum_{g\in
F}1_{\{z\in f(V_{w})\cap g(V_{\tilde{w}})\}}|g(K_{\tilde{w}}\cap
V_{m})|\right)  \notag \\
& +C_{1}N^{-2m}\sum_{|\tilde{w}|=n}\sum_{g\in F}\sum_{z\in g(V_{\tilde{w}%
})}\sum_{y\in g(K_{\tilde{w}}\cap V_{m})}|u(z)-u(y)|^{p}\left(
\sum_{|w|=n}\sum_{f\in F}1_{\{z\in f(V_{w})\cap g(V_{\tilde{w}%
})\}}|f(K_{w}\cap V_{m})|\right)  \notag \\
\leq & C_{2}N^{-(m+n)}\left(\sum_{|w|=n}\sum_{f\in F}\sum_{z\in
f(V_{w})}\sum_{x\in f(K_{w}\cap V_{m})}|u(x)-u(z)|^{p} +\sum_{|\tilde{w}%
|=n}\sum_{g\in F}\sum_{z\in g(V_{\tilde{w}})}\sum_{y\in g(K_{\tilde{w}}\cap
V_{m})}|u(z)-u(y)|^{p} \right)  \notag \\
=& 2C_{2}N^{-(m+n)}\sum_{|w|=n}\sum_{f\in F}\sum_{x\in f(K_{w}\cap
V_{m})}\sum_{z\in f(V_{w})}|u(x)-u(z)|^{p},  \label{eq5.6}
\end{align}%
where we use the uniform finitely-joint property for each common vertex $%
z\in f(V_{w})\cap g(V_{\tilde{w}})$ in the last inequality, so that
\begin{equation*}
\sum_{|\tilde{w}|=n}\sum_{g\in F}1_{\{z\in f(V_{w})\cap g(V_{\tilde{w}})\}}
\end{equation*}%
is uniformly bounded for any $(f,w)\in F\times W_{n}$, and the elementary
estimates
\begin{eqnarray}
|V_{m}| &\asymp &N^{m}=\rho ^{-\alpha m},  \label{v1} \\
\text{\ }|K_{w}\cap V_{m}| &\asymp &N^{m-n}\text{ for every }w\in W_{n}.
\label{v2}
\end{eqnarray}%
Since $x,z$ belong to the same tile $f(K)$, we apply the same estimate for $%
|u(x)-u(z)|^{p}$ as \cite[(2.6)]{GaoYuZhang2022PA}, and for reader's
convenience we present it here. For such $x\in f(V_{m}),z\in f(V_{n})$, one
can naturally fix a decreasing sequence of cells $K_{w_{k}}$ with $|w_{k}|=k$
for $k=n,\cdots ,m$ and vertices $x_{k}(x,z)\in V_{w_{k}}$ such that, $%
z=x_{n}(x,z),\ x=x_{m}(x,z)$, so by H\"{o}lder's inequality we have
\begin{align}
|u(z)-u(x)|^{p}\leq & \left( \sum_{k=n}^{m-1}N^{(n-k)q/p}\right)
^{p/q}\left(
\sum_{k=n}^{m-1}N^{k-n}|u(x_{k}(x,z))-u(x_{k+1}(x,z))|^{p}\right)  \notag \\
\leq & C_{3}\sum_{k=n}^{m-1}N^{k-n}|u(x_{k}(x,z))-u(x_{k+1}(x,z))|^{p},
\label{u_xz}
\end{align}%
where $q=p/(p-1)$. Note that for every pair $(a,b)\in f(K_{w^{\prime }}\cap
V_{k})\times f(K_{w^{\prime }}\cap V_{k+1})$, the cardinality of $(x,z)\in
f(K_{w}\cap V_{m})\times f(V_{w})$ with $(x_{k}(x,z),x_{k+1}(x,z))=(a,b)$ is
no greater than $C^{\prime }N^{m-k}$ for some $C^{\prime }>0$, due to the
p.c.f. structure. Substituting \eqref{u_xz} into \eqref{eq5.6}, we obtain
\begin{align}
I_{m,n}^{F}(u)& \leq C_{4}N^{-(m+n)}\sum_{|w|=n}\sum_{f\in F}\sum_{(x,z)\in
f(K_{w}\cap V_{m})\times
f(V_{w})}\sum_{k=n}^{m-1}N^{k-n}|u(x_{k}(x,z))-u(x_{k+1}(x,z))|^{p}  \notag
\\
& \leq C_{4}N^{-(m+n)}\sum_{|w|=n}\sum_{k=n}^{m-1}\sum_{f\in F}\sum_{\QATOP{%
|w^{\prime }|=k}{f(K_{w^{\prime }})\subset f(K_{w})}}\sum_{\QATOP{(x,z)\in
f(K_{w}\cap V_{m})\times f(V_{w}),}{(a,b)\in f(K_{w^{\prime }}\cap
V_{k})\times f(K_{w^{\prime }}\cap V_{k+1})}}N^{k-n}|u(a)-u(b)|^{p}  \notag
\\
& \leq C_{5}N^{-(m+n)}\sum_{|w|=n}\sum_{k=n}^{m-1}\sum_{f\in F}\sum_{\QATOP{%
|w^{\prime }|=k}{f(K_{w^{\prime }})\subset f(K_{w})}}\sum_{a,b\in
f(K_{w^{\prime }}\cap V_{k+1})}N^{k-n}\cdot N^{m-k}|u(a)-u(b)|^{p}  \notag \\
& =C_{5}\rho ^{2n\alpha }\sum_{k=n}^{m-1}\sum_{|w|=n}\sum_{f\in F}\sum_{%
\QATOP{|w^{\prime }|=k}{f(K_{w^{\prime }})\subset f(K_{w})}}\sum_{a,b\in
f(K_{w^{\prime }}\cap V_{k+1})}|u(a)-u(b)|^{p}  \notag \\
& \leq C_{6}\rho ^{2n\alpha }\sum_{k=n}^{m-1}\sum_{f\in
F}E_{k+1}^{(p)}(u\circ f)\leq C_{6}\rho ^{2n\alpha
}\sum_{k=n}^{m}E_{k}^{(p),F}(u),  \label{I_mn}
\end{align}%
where we have used \eqref{p_energy} in the last line. It follows that
\begin{equation*}
I_{m,n}^{F}(u)\leq C_{6}\rho ^{2n\alpha }\rho ^{n(p\sigma -\alpha
)}\sum_{k=n}^{m}\rho ^{-n(p\sigma -\alpha )}E_{k}^{(p),F}(u)\leq C_{7}\rho
^{n(p\sigma +\alpha )}\sup_{k\geq n}\mathcal{E}_{k}^{\sigma ,F}(u),
\end{equation*}%
which ends the proof by letting $m\rightarrow \infty $ and using Lemma \ref{lem4.4}.
\end{proof}

The following simple proposition is required.

\begin{proposition}
\label{prop:I}For all $u\in B_{p,\infty }^{\sigma ,F}$, we have
\begin{eqnarray}
\liminf_{n\rightarrow \infty }\rho ^{-n(p\sigma +\alpha )}I_{\infty
,n}^{F}(u) &\geq &\rho ^{p\sigma +\alpha }C_{H}^{p\sigma +\alpha
}\liminf_{n\rightarrow \infty }\Phi _{u}^{\sigma ,F}(\rho ^{n}),
\label{II-1} \\
\limsup_{n\rightarrow \infty }\rho ^{-n(p\sigma +\alpha )}I_{\infty
,n}^{F}(u) &\geq &\rho ^{p\sigma +\alpha }C_{H}^{p\sigma +\alpha
}\limsup_{n\rightarrow \infty }\Phi _{u}^{\sigma ,F}(\rho ^{n}).
\label{II-3}
\end{eqnarray}

\end{proposition}

\begin{proof}
Since $C_{H}\in (0,1)$, assume that $\rho ^{s+1}\leq C_{H}<\rho ^{s}$ for
some nonnegative integer $s$. Then, for any integer $n$,
\begin{eqnarray}
\rho ^{-n(p\sigma +\alpha )}I_{\infty ,n}^{F}(u) &=&\rho ^{-n(p\sigma
+\alpha )}\int_{K^{F}}\int_{B(x,C_{H}\rho ^{n})}|u(x)-u(y)|^{p}d\mu
^{F}(y)d\mu ^{F}(x)  \notag \\
&\geq &\rho ^{-n(p\sigma +\alpha )}\int_{K^{F}}\int_{B(x,\rho
^{n+s+1})}|u(x)-u(y)|^{p}d\mu ^{F}(y)d\mu ^{F}(x)  \label{II-2} \\
&=&\rho ^{(s+1)(p\sigma +\alpha )}\rho ^{-(n+s+1)(p\sigma +\alpha
)}\int_{K^{F}}\int_{B(x,\rho ^{n+s+1})}|u(x)-u(y)|^{p}d\mu ^{F}(y)d\mu
^{F}(x)  \notag \\
&>&\rho ^{p\sigma +\alpha }C_{H}^{p\sigma +\alpha }\rho ^{-(n+s+1)(p\sigma
+\alpha )}\int_{K^{F}}\int_{B(x,\rho ^{n+s+1})}|u(x)-u(y)|^{p}d\mu
^{F}(y)d\mu ^{F}(x).  \notag
\end{eqnarray}%
Taking $\liminf_{n\rightarrow \infty }$ and $\limsup_{n\rightarrow \infty }$
in both side, we obtain (\ref{II-1}) and (\ref{II-3}).
\end{proof}

Denote the `ring-energy' by
\begin{equation*}
I_{n}^{F}(u):=\int_{K^{F}}\int_{\{\rho ^{n+1}\leq d(x,y)<\rho
^{n}\}}|u(x)-u(y)|^{p}d\mu ^{F}(y)d\mu ^{F}(x),
\end{equation*}%
which will be used later for the `truncation trick' instead of the
`ball-energy'.

\begin{lemma}
\label{lem6.3} Let $K^{F}$ be a fractal glue-up. If $\sigma >\alpha /p$,
then for any $\delta \in (0,p\sigma -\alpha )$ and $u\in B_{p,\infty
}^{\sigma ,F}$,
\begin{equation}
\mathcal{E}_{n}^{\sigma ,F}(u)\leq C\sum_{k=0}^{\infty }\rho ^{(-2\alpha
-\delta )k}\rho ^{-(p\sigma +\alpha )n}I_{k+n}^{F}\leq C^{\prime
}\sup_{k\geq 0}\Phi _{u}^{\sigma ,F}(\rho ^{n+k}).  \label{eq6.4}
\end{equation}
\end{lemma}

\begin{proof}
For $a,b\in f(K_{w})$, we have $|u(a)-u(b)|^{p}\leq
2^{p-1}(|u(a)-u(x)|^{p}+|u(x)-u(b)|^{p})$, where $x\in f(K_{w})$ and $f\in F$%
. Approximate $F$ with increasing finite sets $F_{j}$ so that $%
F=\lim_{j\rightarrow \infty }F_{j}$. Integrating with respect to $x$ and
dividing by $\mu ^{F}(f(K_{w}))$, we have
\begin{align}
E_{n}^{(p),F_{j}}(u)& =\sum_{f\in F_{j}}\sum_{a,b\in
f(V_{w}),|w|=n}|u(a)-u(b)|^{p}  \notag \\
& \leq 2^{p-1}\sum_{f\in F_{j}}\sum_{a,b\in f(V_{w}),a\neq b,|w|=n}\left(
\frac{1}{\mu ^{F}(f(K_{w}))}\int_{f(K_{w})}|u(a)-u(x)|^{p}+|u(x)-u(b)|^{p}d%
\mu ^{F}(x)\right)  \notag \\
& \leq 2^{p-1}|V_{0}|\sum_{f\in F_{j}}\sum_{a\in f(V_{w}),|w|=n}\frac{1}{\mu
^{F}(f(K_{w}))}\int_{f(K_{w})}|u(a)-u(x)|^{p}d\mu ^{F}(x).  \label{lem3.0}
\end{align}%
For every $a\in f(V_{w})$ with $|w|=n$, we can fix a decreasing sequence of
cells $\{f(K_{w_{k}})\}_{k=n}^{m}$ where $|w_{k}|=k$ for all large enough $%
m>n$, such that $a\in \cap _{k=n}^{m}f(K_{w_{k}})$ with $w_{n}=w$. We choose
$x_{k}\in f(K_{w_{k}})$ for $k=n,n+1,...m$. Let $\delta \in (0,p\sigma
-\alpha )$, by H\"{o}lder's inequality,
\begin{align}
& |u(a)-u(x_{n})|^{p}  \notag \\
\leq & 2^{p-1}|u(a)-u(x_{m})|^{p}+2^{p-1}\left( \sum_{k=n}^{m-1}\rho
^{\delta (k-n)q/p}\right) ^{p/q}\left( \sum_{k=n}^{m-1}\rho ^{\delta
(n-k)}|u(x_{k})-u(x_{k+1})|^{p}\right) ,  \label{203}
\end{align}%
where $q=p/(p-1)$. Integrating \eqref{203} with respect to $x_{k}\in
K_{w_{k}}$ and dividing by $\mu ^{F}(f(K_{w_{k}}))$,
\begin{align*}
E_{n}^{(p),F_{j}}(u)\leq & C_{1}\sum_{f\in F_{j}}\sum_{a\in f(V_{w}),|w|=n}%
\frac{2^{p-1}}{\mu ^{F}(f(K_{w_{m}}))}%
\int_{f(K_{w_{m}})}|u(a)-u(x_{m})|^{p}d\mu ^{F}(x_{m}) \\
& +C_{2}\sum_{f\in F_{j}}\sum_{k=n}^{m-1}\frac{\rho ^{\delta (n-k)}}{\mu
^{F}(f(K_{w_{k}}))\mu ^{F}(f(K_{w_{k+1}}))}\int_{f(K_{w_{k}})}%
\int_{f(K_{w_{k+1}})}|u(x_{k})-u(x_{k+1})|^{p}d\mu ^{F}(x_{k+1})d\mu
^{F}(x_{k}).
\end{align*}%
Using Lemma \ref{lemmaMR}, (\ref{BC}) and (\ref{v1}), the first term of the
right-hand side above vanishes as $m\rightarrow \infty $ since
\begin{align*}
& \sum_{a\in f(V_{w}),|w|=n}\frac{2^{p-1}}{\mu ^{F}(f(K_{w_{m}}))}%
\int_{f(K_{w_{m}})}|u(a)-u(x_{m})|^{p}d\mu ^{F}(x_{m}) \\
\leq & C|f(V_{n})|2^{p-1}\rho ^{(p\sigma -\alpha )m}[u]_{B_{p,\infty
}^{\sigma }(K^{F})}^{p}\leq C^{\prime }|f(V_{n})|2^{p-1}\rho ^{(p\sigma
-\alpha )m}[u]_{B_{p,\infty }^{\sigma ,F}}^{p} \\
\leq & C_{3}\rho ^{(p\sigma -\alpha )m-\alpha n}[u]_{B_{p,\infty }^{\sigma
,F}}^{p}\rightarrow 0.
\end{align*}%
Letting $m\rightarrow \infty $, we obtain
\begin{align}
E_{n}^{(p),F_{j}}(u)& \leq C_{4}\sum_{k=n}^{\infty }\rho ^{\delta (n-k)}\rho
^{-2\alpha k}\int_{K^{F}}\int_{B(x,\rho ^{k})}|u(x)-u(y)|^{p}d\mu
^{F}(y)d\mu ^{F}(x)  \notag \\
& =C_{4}\sum_{k=n}^{\infty }\rho ^{\delta (n-k)}\rho ^{-2\alpha
k}\sum_{l=k}^{\infty }I_{l}^{F}(u)=C_{4}\sum_{k=0}^{\infty }\rho ^{-\delta
k}\rho ^{-2\alpha (n+k)}\sum_{l=k}^{\infty }I_{l+n}^{F}(u)  \notag \\
& =C_{4}\sum_{l=0}^{\infty }\left( \sum_{k=0}^{l}\rho ^{-\delta k}\rho
^{-2\alpha (n+k)}\right) I_{l+n}^{F}(u)=C_{4}\sum_{l=0}^{\infty }\left(
\sum_{k=0}^{l}\rho ^{\delta (l-k)}\rho ^{2\alpha (l-k)}\right) \rho
^{-\delta l}\rho ^{-2\alpha (l+n)}I_{l+n}^{F}(u)  \notag \\
& \leq C_{5}\sum_{l=0}^{\infty }\rho ^{-\delta l}\rho ^{-2\alpha
(l+n)}I_{l+n}^{F}(u)=C_{5}\sum_{k=0}^{\infty }\rho ^{-\delta k}\rho
^{-2\alpha (k+n)}I_{k+n}^{F}(u).  \label{I_2}
\end{align}%
Letting $j\rightarrow \infty $, we have%
\begin{eqnarray*}
\mathcal{E}_{n}^{\sigma ,F}(u) &=&\rho ^{-(p\sigma -\alpha
)n}E_{n}^{(p),F}(u)\leq C_{5}\sum_{k=0}^{\infty }\rho ^{(-2\alpha -\delta
)k}\rho ^{-(p\sigma +\alpha )n}I_{k+n}^{F}(u) \\
&\leq &C_{5}\sum_{k=0}^{\infty }\rho ^{(-2\alpha -\delta )k}\rho ^{-(p\sigma
+\alpha )n}\int_{K^{F}}\int_{B(x,\rho ^{n+k})}|u(x)-u(y)|^{p}d\mu
^{F}(y)d\mu ^{F}(x) \\
&\leq &C_{5}\sum_{k=0}^{\infty }\rho ^{(p\sigma +\alpha -2\alpha -\delta
)k}\sup_{k\geq 0}\Phi _{u}^{\sigma ,F}(\rho ^{n+k}) \\
&=&C_{5}\left( \sum_{k=0}^{\infty }\rho ^{(p\sigma -\alpha -\delta
)k}\right) \sup_{k\geq 0}\Phi _{u}^{\sigma ,F}(\rho ^{n+k})=\frac{C_{5}}{%
1-\rho ^{p\sigma -\alpha -\delta }}\sup_{k\geq 0}\Phi _{u}^{\sigma ,F}(\rho
^{n+k}),
\end{eqnarray*}%
where the second inequality follows by enlarging each ring-energy $%
I_{k+n}^{F}$ to ball-energy.
\end{proof}

We need the following proposition to manage the transitions between
different levels of vertex energies.

\begin{proposition}
Let $K$ be a connected homogeneous p.c.f. self-similar set, then there
exists $C>0$ such that for all $u\in L^{p}(K,\mu )$ and nonnegative integer $n$,
\begin{equation*}
E_{n}^{(p)}(u)\leq CE_{n+1}^{(p)}(u).
\end{equation*}%
Moreover, there exists $C>0$ such that for all $u\in L^{p}(K^{F},\mu ^{F})$,
\begin{equation}
E_{n}^{(p),F}(u)\leq CE_{n+1}^{(p),F}(u).  \label{E-n}
\end{equation}
\end{proposition}

\begin{proof}
We start with the basic case $n=0$. For any $x,y\in V_{0}$, we can fix a
vertex-disjoint path $x=z_{1}(x,y),z_{2}(x,y),\cdots ,z_{k}(x,y)=y$ such
that $z_{j}(x,y)$ and $z_{j+1}(x,y)$ ($1\leq j\leq k$) belong to the same $%
V_{w}$ for some $w\in W_{1}$, and $2\leq k\leq |V_{1}|$. Using Jensen's
inequality,
\begin{align*}
|u(x)-u(y)|^{p}& \leq
(k-1)^{p-1}\sum_{j=1}^{k-1}|u(z_{j}(x,y))-u(z_{j+1}(x,y))|^{p} \\
& \leq |V_{1}|^{p-1}\sum_{j=1}^{k-1}|u(z_{j}(x,y))-u(z_{j+1}(x,y))|^{p}.
\end{align*}%
It follows that
\begin{align*}
E_{0}^{(p)}(u)& \leq \sum_{x,y\in
V_{0}}|V_{1}|^{p-1}\sum_{j=1}^{k-1}|u(z_{j}(x,y))-u(z_{j+1}(x,y))|^{p} \\
& \leq |V_{0}|^{2}|V_{1}|^{p-1}\sum_{a,b\in V_{w},|w|=1}|u(a)-u(b)|^{p} \\
& =|V_{0}|^{2}|V_{1}|^{p-1}E_{1}^{(p)}(u),
\end{align*}%
since for each pair $(a,b)\in V_{w}$ with $|w|=1$, there exists at most one $%
j\in \{1,2,...k\}$ (depending on $(x,y)$) with $%
(z_{j}(x,y),z_{j+1}(x,y))=(a,b)$ for any $x,y\in V_{0}$. Thus, we show the
desired for $n=0$ with $C:=|V_{0}|^{2}|V_{1}|^{p-1}$ for any $u$. The rest
simply follows from that
\begin{align*}
E_{n}^{(p)}(u)& =\sum_{x,y\in
V_{w},|w|=n}|u(x)-u(y)|^{p}=\sum_{|w|=n}E_{0}^{(p)}(u\circ \phi _{w}) \\
& \leq C\sum_{|w|=n}E_{1}^{(p)}(u\circ \phi _{w})=C\sum_{a,b\in
V_{wi},|w|=n,|i|=1}|u(a)-u(b)|^{p}=CE_{n+1}^{(p)}(u).
\end{align*}%
The second inclusion follows from the definition of $K^{F}$.
\end{proof}

In \cite[Theorem 1.4]{GaoYuZhang2022PA}, for a homogeneous p.c.f.
self-similar set $K$, we show the $p$-energy norm equivalence
\begin{equation}
\sup\limits_{n\geq 0}\rho ^{-n(p\sigma -\alpha )}E_{n}^{(p)}(u)\asymp
\lbrack u]_{B_{p,\infty }^{\sigma }(K)}^{p}\text{ \ (}p\sigma >\alpha \text{)%
},  \label{204}
\end{equation}%
where $E_{n}^{(p)}(u)$ is from (\ref{EE}).

Using Lemma \ref{lem6.2} and Lemma \ref{lem6.3}, we immediately derive such
equivalence for $K^{F}$. Define local $p$-energy norm for $K^{F}$ by
\begin{equation*}
\mathcal{E}_{p,\infty }^{\sigma ,F}(u):=\sup\limits_{n\geq 0}\rho
^{-n(p\sigma -\alpha )}E_{n}^{(p),F}(u)=\sup_{n\geq 0}\mathcal{E}%
_{n}^{\sigma ,F}(u).
\end{equation*}

\begin{corollary}
\label{lem6.1} Let $K^{F}$ be a fractal glue-up. If $\sigma>\alpha /p$, then
for all $u\in B_{p,\infty }^{\sigma ,F}$,
\begin{equation}
\limsup_{n\rightarrow \infty }\mathcal{E}_{n}^{\sigma ,F}(u)\asymp
\limsup_{n\rightarrow \infty }\Phi _{u}^{\sigma ,F}(\rho ^{n}),\ \mathcal{E}%
_{p,\infty }^{\sigma ,F}(u)\asymp \lbrack u]_{B_{p,\infty}^{\sigma ,F}}^{p}.
\label{eq6.2}
\end{equation}
\end{corollary}

\begin{proof}
Taking limsup and sup (of $n$) on both sides of \eqref{eq6.3} and %
\eqref{eq6.4}, we have
\begin{eqnarray}
\limsup_{n\rightarrow \infty }\rho ^{-n(p\sigma +\alpha )}I_{\infty
,n}^{F}(u) &\leq &C\limsup_{n\rightarrow \infty }\mathcal{E}_{n}^{\sigma
,F}(u)\leq C^{\prime }\limsup_{n\rightarrow \infty }\Phi _{u}^{\sigma
,F}(\rho ^{n}),  \label{C6.1-1} \\
\sup\limits_{n\geq 0}\rho ^{-n(p\sigma +\alpha )}I_{\infty ,n}^{F}(u) &\leq
&C\sup\limits_{n\geq 0}\mathcal{E}_{n}^{\sigma ,F}(u).  \label{C6.1-2}
\end{eqnarray}

By (\ref{II-3}), we have%
\begin{equation*}
\limsup_{n\rightarrow \infty }\rho ^{-n(p\sigma +\alpha )}I_{\infty
,n}^{F}(u)\geq \rho ^{p\sigma +\alpha }C_{H}^{p\sigma +\alpha
}\limsup_{n\rightarrow \infty }\Phi _{u}^{\sigma ,F}(\rho ^{n}),
\end{equation*}%
thus showing
\begin{equation*}
\limsup_{n\rightarrow \infty }\mathcal{E}_{n}^{\sigma ,F}(u)\asymp
\limsup_{n\rightarrow \infty }\Phi _{u}^{\sigma ,F}(\rho ^{n})
\end{equation*}%
by (\ref{C6.1-1}). On the other hand, we have from (\ref{ct-1}) and (\ref{ct}%
) that
\begin{equation}
\sup\limits_{n\geq 0}\rho ^{-n(p\sigma +\alpha )}I_{\infty
,n}^{F}(u)=C_{H}^{p\sigma +\alpha }\sup\limits_{n\geq 0}\Phi _{u}^{\sigma
,F}(C_{H}\rho ^{n})=C_{H}^{p\sigma +\alpha }[u]_{B_{p,\infty }^{\sigma
,F}}^{p}  \label{C6.1-0}
\end{equation}

Since $C_{H}\in (0,1)$, assume that $\rho ^{s+1}\leq C_{H}<\rho ^{s}$ for
some nonnegative integer $s$. Replacing $n$ by $n+s+1$ in (\ref{eq6.4}), we
obtain%
\begin{equation}
\mathcal{E}_{n+s+1}^{\sigma ,F}(u)\leq C\sup_{k\geq 0}\Phi _{u}^{\sigma
,F}(\rho ^{n+s+1+k})\leq C^{\prime }\sup_{k\geq 0}\Phi _{u}^{\sigma
,F}(C_{H}\rho ^{n+k}).  \label{C6.1-3}
\end{equation}
By (\ref{E-n}),%
\begin{equation*}
\mathcal{E}_{n+s+1}^{\sigma ,F}(u)=\rho ^{-(n+s+1)(p\sigma -\alpha
)}E_{n+s+1}^{(p),F}(u)\geq C^{-s-1}\rho ^{-(n+s+1)(p\sigma -\alpha
)}E_{n}^{(p),F}(u)=C^{-s-1}\rho ^{-(s+1)(p\sigma -\alpha )}\mathcal{E}%
_{n}^{\sigma ,F}(u),
\end{equation*}
thus by (\ref{C6.1-3}),
\begin{equation}
\mathcal{E}_{n}^{\sigma ,F}(u)\leq C^{\prime \prime }C_{H}^{p\sigma -\alpha
}\sup_{k\geq 0}\Phi _{u}^{\sigma ,F}(C_{H}\rho ^{n+k}).  \label{C6.1-4}
\end{equation}

We conclude from (\ref{C6.1-2}), (\ref{C6.1-0}) and (\ref{C6.1-4}) that%
\begin{equation*}
\mathcal{E}_{p,\infty }^{\sigma ,F}(u)\asymp \lbrack u]_{B_{p,\infty
}^{\sigma ,F}}^{p}.
\end{equation*}

The proof is complete.
\end{proof}

\subsection{Property (E) holds for nested fractals}

\label{subsec5.2}For a set $V$, let $\ell (V)=\{u:$
$u$ maps $V$ into $\mathbb{R}\}$.

\begin{definition}
(\cite[Definition 3.1]{GaoYuZhang2022PA}) \label{dfE} We say that a
connected homogeneous p.c.f. self-similar set $K$ satisfies \textrm{property
(E)}, if there exist $\sigma>\alpha/p$ and a positive constant $C$ such that

\begin{itemize}
\item[(i)~] for any $u\in B_{p,\infty }^{\sigma}$ and for all $n\geq 0$, $%
\mathcal{E}_{0}^{\sigma}(u)\leq C\mathcal{E}_{n}^{\sigma}(u)$,

\item[(ii)] for any $u\in \ell (V_{0})$, there exists an extension $\tilde{u}%
\in B_{p,\infty }^{\sigma}$.
\end{itemize}
\end{definition}

In \cite[Remark 3.2]{GaoYuZhang2022PA} we know that under property (E), $%
\sigma=\sigma _{p}^{*}(K)$, and for all $n\geq m$, $\mathcal{E}_{m}^{\sigma
_{p}^{*}}(u)\leq C\mathcal{E}_{n}^{\sigma _{p}^{*}}(u)$.
The following proposition shows the importance of property (E).

\begin{proposition}
\label{prop:critical} If a connected homogeneous p.c.f. self-similar set $K$
satisfies property (E), then
\begin{equation}
\sigma _{p}^{\ast }(K)=\sigma _{p}^{\#}(K)=\sigma _{p}^{\#}(K^{F}).
\label{qnmd}
\end{equation}%
Moreover, $K^{F}$ satisfies property (VE) with $\sigma _{p}^{\#}(K)$, and $%
B_{p,\infty }^{\sigma _{p}^{\#}(K),F}$ contains non-constant functions.
\end{proposition}
\begin{proof}
The fact $\sigma _{p}^{\ast }(K)=\sigma _{p}^{\#}(K)$ follows from \cite[the
proof of Proposition 3.4]{GaoYuZhang2022PA}. It is also clear that $\sigma
_{p}^{\#}(K)\geq \sigma _{p}^{\#}(K^{F})$, since any non-constant function $%
u\in l(K^{F})$ in $B_{p,\infty }^{\sigma ,F}$ naturally gives a non-constant
function in $B_{p,\infty }^{\sigma }(K)$ for any $\sigma >0$: just restrict $%
u$ to a tile $f(K)$ where it is non-constant (for some $f\in F$).

For the reverse inclusion $\sigma _{p}^{\#}(K)\leq \sigma _{p}^{\#}(K^{F})$
to hold, we just need to use property (E)(ii) to find a non-constant
function in $B_{p,\infty }^{\sigma _{p}^{\#}(K),F}$. To see this, we pick a
contraction $f\in F$, and determine the values of $u$ on $f(V_{1})$ by $%
u(f(V_{0}))=0$ while $u(f(V_{1}\setminus V_{0}))=1$. We can then extend $u$
from $f(V_{1})$ to $f(K)$ and make sure that $u\circ f\in B_{p,\infty
}^{\sigma _{p}^{\#}(K)}$ by property (E)(ii), and simply define $u=0$
outside of $f(K)$ on $K^{F}$. Then $u$ is an extended non-trivial function
in $B_{p,\infty }^{\sigma _{p}^{\#}(K),F}$ since by (\ref{eq6.2}) and (\ref%
{204}),
\begin{equation*}
\lbrack u]_{B_{p,\infty }^{\sigma _{p}^{\#}(K),F}}^{p}\asymp \mathcal{E}%
_{p,\infty }^{\sigma _{p}^{\#}(K),F}(u)=\mathcal{E}_{p,\infty }^{\sigma
_{p}^{\#}(K)}(u\circ f)\asymp \lbrack u\circ f]_{B_{p,\infty }^{\sigma
_{p}^{\#}(K)}(K)}^{p}<\infty.
\end{equation*}
Thus we show \eqref{qnmd} and that $B_{p,\infty }^{\sigma _{p}^{\#}(K),F}$
contains non-constant functions.

Finally, since $K$ satisfies property (E), that is for all $n$,
\begin{equation*}
\mathcal{E}_{n}^{\sigma _{p}^{\#}}(u)\leq C\liminf_{k\rightarrow \infty }%
\mathcal{E}_{k}^{\sigma _{p}^{\#}}(u),
\end{equation*}%
summing over $F$ gives that for all $n$,
\begin{equation*}
\mathcal{E}_{n}^{\sigma _{p}^{\#},F}(u)\leq C\sum_{f\in
F}\liminf_{k\rightarrow \infty }\mathcal{E}_{k}^{\sigma _{p}^{\#}}(u\circ
f)\leq C\liminf_{k\rightarrow \infty }\sum_{f\in F}\mathcal{E}_{k}^{\sigma
_{p}^{\#}}(u\circ f)=C\liminf_{k\rightarrow \infty }\mathcal{E}_{k}^{\sigma
_{p}^{\#},F}(u),
\end{equation*}%
where we use Fatou's lemma to obtain the desired.
\end{proof}

When property (E) fails, $\sigma _{p}^{\#}=\sigma _{p}^{\ast }$ does not
always hold for connected homogeneous p.c.f. self-similar sets (see \cite%
{GuLau.2020.TAMS}). Moreover, we mention in passing that, the critical
domain $B_{p,\infty }^{\sigma _{p}^{\ast }}$ can be trivial for some metric
measure spaces (bounded or unbounded).

We present the following key lemma for {\em nested
fractals}, a class of connected homogeneous p.c.f. self-similar sets
defined in \cite{Kumagai.1993.PTaRF205} satisfying conditions $(A$-0$)\sim
(A $-3$)$ and $|V_{0}|\geq 2$ therein.

%\begin{definition}(\cite[Definition 4.1]{Caoqiugu2022adv})
%We say the condition $\left( \mathbf{A}\right)$ holds if There exists $E\in \mathcal{M}$ such that
%\begin{align}
%\inf_{n\geq0} \delta(\mathcal{T}^n E)=\frac{\min_{x\neq y}R(x,y)}{\max_{x\neq y}R(x,y)}>0
%\end{align}
%where $R(x,y)$ is the p-effective resistance
%(p-resistance for short) between $x, y$.
%\end{definition}
%
%
%
%\begin{lemma}(\cite[Theorem 4.2]{Caoqiugu2022adv})
%Let $1<p<\infty$ and $K$ be a p.c.f. self-similar set. Then there exists $E\in \mathcal{Q}$ and $\lambda> 0$ such that
%$\mathcal{T}E=\lambda E$ if and only if $\left( \mathbf{A}\right)$ holds. In particular, $\lambda$ is unique.
%
%In addition, if $K$ and $r$ are $\mathscr{G}$-symmetric, then we can also require that $E$ is $\mathscr{G}$-
%symmetric.
%\end{lemma}
%
%\begin{lemma}(\cite[(A')]{Caoqiugu2022adv})
%	We say the condition $\left( \mathbf{A^{\prime }}\right)$ holds if $\left( \mathbf{A}\right)$ holds and $r$ is properly chosen so that there is $E\in \mathcal{Q}$ such that $\mathcal{T}E=\lambda E$.
%\end{lemma}
%
%\begin{lemma}(\cite[lemma 5.4]{Caoqiugu2022adv})
%Assume $\left( \mathbf{A^{\prime }}\right)$, if $\dot{w}= www\dots\in \mathcal{P}$, then $r_w<1$.
%\end{lemma}
%
%
%
%\begin{lemma}(\cite[Theorem 6.3]{Caoqiugu2022adv})
%	Let $(K, {F_i}^N_{i=1})$ be an affine nested fractal. If the renormalization factor
%	$r$ is $\mathscr{G}$-symmetric, then condition $\left( \mathbf{A}\right)$ holds.
%\end{lemma}
%

\begin{lemma}
\label{lemma:E}For a nested fractal, property (E) holds and $\sigma
_{p}^{\#}>\alpha/p$ for all $1<p<\infty$. \label{yyqx}
\end{lemma}

\begin{proof}
In this proof we use the same terminology and notions as in \cite%
{Caoqiugu2022adv}. Let $K$ be a nested fractal. We fix the components $r_{i}$
of $\mathbf{r}$ in \cite[Theorem 6.3]{Caoqiugu2022adv} to be the same number
$r$, so it is `$\mathscr{G}$-symmetric' (see \cite[Section 3]%
{Caoqiugu2022adv} for definition). Then \cite[Theorem 6.3]{Caoqiugu2022adv}
states that, `condition $\left( \mathbf{A}\right) $' (see \cite[the
beginning of Section 4]{Caoqiugu2022adv} for definition) holds for affine
nested fractals (probably inhomogeneous), which includes $K$. By multiplying
$\mathbf{r}$ with a constant (still denoted by $\mathbf{r}$), \cite[Theorem
4.2]{Caoqiugu2022adv} guarantees that
\begin{equation}
\mathcal{T}E=E  \label{TE}
\end{equation}%
(see \cite[Definition 2.8, Definition 3.1]{Caoqiugu2022adv} for related
definitions), thus showing `condition $\left( \mathbf{A^{\prime }}\right) $'
(see \cite[the beginning of Section 5]{Caoqiugu2022adv}). So by \cite[Lemma
5.4]{Caoqiugu2022adv}, we can fix
\begin{equation}
r<1  \label{rp1}
\end{equation}%
to further satisfy condition $\left( \mathbf{A^{\prime }}\right) $ on $K$.

The equivalence of $E$ and $E_{0}^{(p)}$ (for functions $u\in l(V_{0})$, see
the statement of \cite[Theorem 5.1]{Caoqiugu2022adv}) is stated in the proof
of \cite[Proposition 5.3 (b)]{Caoqiugu2022adv}, which implies that
\begin{equation}
\Lambda ^{n}E(u)\asymp \Lambda ^{n}E_{0}^{(p)}(u)=r^{-n}E_{n}^{(p)}(u)
\label{205}
\end{equation}%
for all $n\geq 0$ and all functions $u\in l(V_{n})$ by the definition of $%
\Lambda $ in \cite[Definition 3.1]{Caoqiugu2022adv} and $E_{n}^{(p)}$ in our
paper.

Now fix
\begin{equation}
\sigma =\frac{\log _{\rho }r+\alpha }{p},  \label{s1}
\end{equation}%
so that $r=\rho ^{p\sigma -\alpha }$.

For property (E)(i), by the monotonicity property in \cite[Proposition
5.3 (a)]{Caoqiugu2022adv} and (\ref{205}),
\begin{equation}
E_{0}^{(p)}(u)\asymp E(u)\leq \Lambda E(u)\leq \cdots \leq \Lambda
^{n}E(u)\asymp r^{-n}E_{n}^{(p)}(u)=\rho ^{-n(p\sigma -\alpha
)}E_{n}^{(p)}(u)=\mathcal{E}_{n}^{\sigma}(u),  \label{EE-1}
\end{equation}%
so property (E)(i) holds with this $\sigma $.

Property (E)(ii) guarantees the existence of piecewise harmonic functions in
\cite[Section 5.1]{Caoqiugu2022adv}, which has a standard iterative
construction as in \cite{HermanPeironeStrichartz.2004.PA125} and Dirichlet
form theory. The iteration part is that, for any boundary value of a cell $%
u_{w}\in l(f_{w}(V_{0}))$, by (\ref{TE}), we can take its next-level
harmonic extension $\tilde{u}_{w}\in l(f_{w}(V_{1}))$ such that $\Lambda E(%
\tilde{u}_{w})=E(u_{w})$ and $\tilde{u}_{w}|_{f_{w}(V_{0})}=u_{w}$. The
iteration starts from $V_{0}$ to $V_{1}$, and from each level-1 cell to
level-2 cells in the above way, then finally to $V_{\ast }$. The p.c.f.
property is very crucial for the piecewise extension to be well-defined,
that is, to avoid different values for the same vertex (since the overlap of
same-level cells only occurs on the same-level scaling of $V_{0}$). Such an
extension satisfies the definition of piecewise harmonic functions in \cite[%
Section 5.1]{Caoqiugu2022adv}, so we know by \cite[Section 5.2]%
{Caoqiugu2022adv} that it naturally embeds into $C(K)$ and this gives an
extension to $K$ rather than $V_{\ast }$.

Therefore, we conclude from above that $K$ satisfies property (E) with $%
\sigma $ defined as in (\ref{s1}). By (\ref{qnmd}), we have $\sigma =\sigma
_{p}^{\#}$, thus
\begin{equation}
r=\rho ^{p\sigma -\alpha }=\rho ^{p\sigma _{p}^{\#}-\alpha }.  \label{rr}
\end{equation}
The inclusion $\sigma _{p}^{\#}>\alpha /p$ follows from \eqref{rp1}.
\end{proof}

\begin{remark}
This lemma also indicates the equivalence of heat kernel-based $p$-energy
norm $E_{p,\infty }^{\sigma _{p}^{\#}}(u)$, Besov norm $[u]_{B_{p,\infty
}^{\sigma _{p}^{\#}}}^{p}$ and the homogeneous discrete $p$-energy
constructed on a nested fractal $K$ by \cite[Theorem 5.1]{Caoqiugu2022adv}.
It is known by \cite[Theorem 8.18]{Barlow.1998.1} that $K$ admits a heat
kernel with estimates \eqref{hk_F}, so $p$-energy norm equivalence in Lemma %
\ref{thm1} holds. The critical exponents coincide $\sigma _{p}^{\#}(K)=\sigma
_{p}^{\ast }(K)$ due to property (E) using Proposition \ref{prop:critical}.
Using $\Lambda ^{n}E_{0}^{(p)}(u)\asymp \Lambda ^{n}E(u)$ in (\ref{205}),
the $p$-energy in \cite[Theorem 5.1]{Caoqiugu2022adv} with $r_{i}=r$ is
equivalent to $\lim_{n\rightarrow \infty }\Lambda ^{n}E(u)$ (which exists by
\cite[Proposition 5.3]{Caoqiugu2022adv}), where $r=\rho ^{p\sigma
_{p}^{\#}(K)-\alpha }$ defined as in (\ref{rr}). By (\ref{204}) and (\ref{EE-1}),$$
[u]_{B_{p,\infty }^{\sigma _{p}^{\#}}}^{p}\asymp \lim_{n\rightarrow \infty }\Lambda ^{n}E(u).$$
This equivalence of the $p$-energy defined in \cite{HermanPeironeStrichartz.2004.PA125} on the Sierpi%
\'{n}ski gasket (a typical nested fractal) is already explained in \cite[Section 4]%
{GaoYuZhang2022PA}.
\end{remark}

%\begin{proposition}
%\label{prop32} If a homogeneous p.c.f. self-similar set $K$ satisfies
%property (E), then for all $u\in B_{p,\infty }^{\sigma ^{\ast }}$, we have
%\begin{equation}
%\liminf_{n\rightarrow \infty }\mathcal{E}_{n}^{\sigma ^{\ast }}(u)\asymp
%\limsup_{n\rightarrow \infty }\mathcal{E}_{n}^{\sigma ^{\ast }}(u)\asymp
%\sup_{n\geq 0}\mathcal{E}_{n}^{\sigma ^{\ast }}(u).  \label{412}
%\end{equation}%
%Moreover, $\mathcal{E}_{p,\infty }^{\sigma ^{\ast }}$ is a local $p$-energy
%form if $K$ satisfies property (E).
%\end{proposition}

\subsection{Equivalence of properties (VE) and (NE)}

\label{subsec5.3} Our goal is to show the equivalence of properties (VE) and
(NE) on spaces $K^{F}$. For p.c.f. fractals, it is more direct to verify
property (VE) as Lemma \ref{yyqx} does, following the discretization routine of the $p$-energy
construction. Before we start our proofs, we mention
that it is quite different and more delicate to obtain the lower limit
equivalence, compared with the norm-equivalence (Lemma \ref{lem6.2} and
Corollary \ref{lem6.1}) using the upper limit. At a first glance, one needs
to be careful using the triangle inequality, since it is closely related to
`sup' rather than `inf'. In fact, we apply both Lemma \ref{lem6.2} and
Corollary \ref{lem6.1} to control each other in each single-side estimate
for the lower limit equivalence, with suitable truncation. The weak-monotonicity properties guarantee that one can truncate the infinite sums in the
two-sided estimates, so a lower bound can be obtained after we find the
relationship between finitely many terms after the
truncation.

\begin{lemma}
\label{thm_3} If a fractal glue-up $K^{F}$ satisfies $(\widetilde{VE})$ with
$\sigma >\alpha /p$, then there exists $C>0$ such that for all $u\in
B_{p,\infty }^{\sigma ,F}$,
\begin{equation}
\liminf_{n\rightarrow \infty }\mathcal{E}_{n}^{\sigma ,F}(u)\leq
C\liminf_{n\rightarrow \infty }\Phi _{u}^{\sigma ,F}(\rho ^{n}).
\label{limphi2}
\end{equation}
\end{lemma}

\begin{proof}
Since $C_{H}\in (0,1)$, assume that $\rho ^{s+1}\leq C_{H}<\rho ^{s}$ for
some $s\in \mathbb{N}$. For all $n>0$, we have by (\ref{II-2}) that%
\begin{equation}
I_{\infty ,n}^{F}(u)\geq \int_{K^{F}}\int_{B(x,\rho
^{n+s+1})}|u(x)-u(y)|^{p}d\mu ^{F}(y)d\mu ^{F}(x)\geq I_{n+s+1}^{F}(u).
\label{I-1}
\end{equation}

By the definition of $\limsup_{n\rightarrow \infty }\mathcal{E}_{k}^{\sigma
,F}(u)$, there exists a positive integer $n_{0}$ such that
\begin{equation*}
\sup_{k\geq n_{0}}\mathcal{E}_{k}^{\sigma ,F}(u)\leq 2\limsup_{n\rightarrow
\infty }\mathcal{E}_{n}^{\sigma ,F}(u).
\end{equation*}

Thus by (\ref{I-1}), we have for all $n\geq n_{0}+s+1$ that
\begin{eqnarray}
I_{n}^{F}(u) &\leq &I_{\infty ,n-s-1}^{F}(u)  \notag \\
&\leq &C\rho ^{(n-s-1)(p\sigma +\alpha )}\sup_{k\geq n-s-1}\mathcal{E}%
_{k}^{\sigma ,F}(u)\text{ \ \ (using \eqref{eq6.3})}  \notag \\
&\leq &CC_{H}^{-(p\sigma +\alpha )}\rho ^{n(p\sigma +\alpha )}\sup_{k\geq
n_{0}}\mathcal{E}_{k}^{\sigma ,F}(u)\leq 2CC_{H}^{-(p\sigma +\alpha )}\rho
^{n(p\sigma +\alpha )}\limsup_{n\rightarrow \infty }\mathcal{E}_{n}^{\sigma
,F}(u)  \notag \\
&\leq &C_{1}\rho ^{n(p\sigma +\alpha )}\liminf_{n\rightarrow \infty }%
\mathcal{E}_{n}^{\sigma ,F}(u)\text{ \ \ (by property }(\widetilde{VE})\text{%
)}.  \label{I_n}
\end{eqnarray}%
By Lemma \ref{lem6.3}, for any positive integer $L\geq
n_{0}+s+1 $ and $0<\delta <p\sigma -\alpha $,
\begin{align*}
\mathcal{E}_{n}^{\sigma ,F}(u)& \leq C_{2}\sum_{k=0}^{\infty }\rho
^{(-2\alpha -\delta )k}\rho ^{-(p\sigma +\alpha )n}I_{k+n}^{F}(u) \\
& = C_{2}\sum_{k=0}^{L}\rho ^{(-2\alpha -\delta )k}\rho ^{-(p\sigma +\alpha
)n}I_{k+n}^{F}(u)+C_{2}\sum_{k=L}^{\infty }\rho ^{(-2\alpha -\delta )k}\rho
^{-(p\sigma +\alpha )n}I_{k+n}^{F}(u) \\
& \leq C_{2}\sum_{k=0}^{L}\rho ^{(-2\alpha -\delta )k}\rho ^{-(p\sigma
+\alpha )n}I_{k+n}^{F}(u)+C_{1}C_{2}\sum_{k=L}^{\infty }\rho ^{(p\sigma
-\alpha -\delta )k}\liminf_{n\rightarrow \infty }\mathcal{E}_{n}^{\sigma
,F}(u), \\
& =C_{2}\left( \sum_{k=0}^{L}\rho ^{(-2\alpha -\delta )k}\rho ^{-(p\sigma
+\alpha )n}I_{k+n}^{F}(u)+\frac{C_{1}\rho ^{(p\sigma -\alpha -\delta )L}}{%
1-\rho ^{(p\sigma -\alpha -\delta )}}\liminf_{n\rightarrow \infty }\mathcal{E%
}_{n}^{\sigma ,F}(u)\right),
\end{align*}%
where we use \eqref{I_n} in the third line. Taking $\liminf_{n\rightarrow
\infty }$ in the right-hand side above, we have
\begin{equation*}
C_{3}\liminf_{n\rightarrow \infty }\mathcal{E}_{n}^{\sigma ,F}(u)\leq
\liminf_{n\rightarrow \infty }\sum_{k=0}^{L}\rho ^{-(2\alpha +\delta )k}\rho
^{-(p\sigma +\alpha )n}I_{k+n}^{F}(u),
\end{equation*}%
where $C_{3}:=\frac{1}{C_{2}}-\frac{C_{1}\rho ^{(p\sigma -\alpha -\delta )L}%
}{1-\rho ^{(p\sigma -\alpha -\delta )}}$.

Fix a large integer $L$ such that $C_{3}>0$, then
\begin{align*}
C_{3}\rho ^{(2\alpha +\delta )L}\liminf_{n\rightarrow \infty }\mathcal{E}%
_{n}^{\sigma ,F}(u)& \leq \rho ^{(2\alpha +\delta )L}\liminf_{n\rightarrow
\infty }\left( \sum_{k=0}^{L}\rho ^{-n(p\sigma +\alpha )}\rho ^{-(2\alpha
+\delta )k}I_{n+k}^{F}(u)\right) \\
& \leq \liminf_{n\rightarrow \infty }\rho ^{-n(p\sigma +\alpha
)}\sum_{k=0}^{L}I_{n+k}^{F}(u)\leq \liminf_{n\rightarrow \infty }\Phi
_{u}^{\sigma ,F}(\rho ^{n}),
\end{align*}%
thus showing \eqref{limphi2}.
\end{proof}

For the other side, we need certain control between different levels of
vertex-energies.

\begin{lemma}
\label{thm_4} If property $(\widetilde{NE})$ holds for a fractal glue-up $%
K^{F}$ with $\sigma >\alpha /p$, then there exists $C>0$ such that for all $%
u\in B_{p,\infty }^{\sigma ,F}$,
\begin{equation}
\liminf_{n\rightarrow \infty }\Phi _{u}^{\sigma ,F}(\rho ^{n})\leq
C\liminf_{n\rightarrow \infty }\mathcal{E}_{n}^{\sigma ,F}(u).
\label{limphi}
\end{equation}
% Moreover, we also have property $%(VE) $.
\end{lemma}

\begin{proof}
By \eqref{I_mn}, we have
\begin{align}
\rho ^{-n(p\sigma +\alpha )}I_{m,n}^{F}(u)\leq & C\rho ^{-n(p\sigma -\alpha
)}\sum_{k=n}^{m}\rho ^{k(p\sigma -\alpha )}\rho ^{-k(p\sigma -\alpha
)}E_{k}^{(p),F}(u)  \notag \\
\leq & C\sum_{k=n}^{m}\rho ^{(k-n)(p\sigma -\alpha )}\mathcal{E}_{k}^{\sigma
,F}(u)\leq C\sum_{k=0}^{\infty }\rho ^{k(p\sigma -\alpha )}\mathcal{E}%
_{k+n}^{\sigma ,F}(u).  \label{202}
\end{align}

By the definition of $\limsup_{n\rightarrow \infty }\mathcal{E}_{n}^{\sigma
,F}(u)$, there exists $n_{0}>0$ such that for all positive integer $L>n_{0}$%
,
\begin{eqnarray*}
\sup_{n\geq L}\mathcal{E}_{n}^{\sigma ,F}(u) &\leq &2\limsup_{n\rightarrow
\infty }\mathcal{E}_{n}^{\sigma ,F}(u) \\
&\asymp &\limsup_{n\rightarrow \infty }\Phi _{u}^{\sigma ,F}(\rho ^{n})\text{
\ (by Corollary \ref{lem6.1})} \\
&\leq &C\liminf_{n\rightarrow \infty }\Phi _{u}^{\sigma ,F}(\rho ^{n})\text{
\ \ (by property }(\widetilde{NE})\text{)}.
\end{eqnarray*}

It follows from (\ref{202}) that
\begin{eqnarray*}
\rho ^{-n(p\sigma +\alpha )}I_{m,n}^{F}(u) &\leq &C\sum_{k=0}^{\infty }\rho
^{k(p\sigma -\alpha )}\mathcal{E}_{k+n}^{\sigma ,F}(u) \\
&=&C\sum_{k=0}^{L}\rho ^{k(p\sigma -\alpha )}\mathcal{E}_{k+n}^{\sigma
,F}(u)+C\sum_{k=L+1}^{\infty }\rho ^{k(p\sigma -\alpha )}\mathcal{E}%
_{k+n}^{\sigma ,F}(u) \\
&\leq &C\sum_{k=0}^{L}\rho ^{k(p\sigma -\alpha )}\mathcal{E}_{k+n}^{\sigma
,F}(u)+C\sum_{k=L+1}^{\infty }\rho ^{k(p\sigma -\alpha )}\sup_{n\geq L}%
\mathcal{E}_{n}^{\sigma ,F}(u) \\
&\leq &C\sum_{k=0}^{L}\rho ^{k(p\sigma -\alpha )}\mathcal{E}_{k+n}^{\sigma
,F}(u)+C^{\prime }\sum_{k=L+1}^{\infty }\rho ^{k(p\sigma -\alpha
)}\liminf_{n\rightarrow \infty }\Phi _{u}^{\sigma ,F}(\rho ^{n}) \\
&=&C\sum_{k=0}^{L}\rho ^{k(p\sigma -\alpha )}\mathcal{E}_{k+n}^{\sigma
,F}(u)+C^{\prime }\frac{\rho ^{(p\sigma -\alpha )(L+1)}}{1-\rho ^{p\sigma
-\alpha }}\liminf_{n\rightarrow \infty }\Phi _{u}^{\sigma ,F}(\rho ^{n}).
\end{eqnarray*}

Using Lemma \ref{lem4.4}, we have
\begin{align*}
\rho ^{-n(p\sigma +\alpha )}I_{\infty ,n}^{F}(u)\leq & \rho ^{-n(p\sigma
+\alpha )}\liminf_{m\rightarrow \infty }I_{m,n}^{F}(u) \\
\leq & C\sum_{k=0}^{L}\rho ^{k(p\sigma -\alpha )}\mathcal{E}_{k+n}^{\sigma
,F}(u)+\frac{C^{\prime }\rho ^{(p\sigma -\alpha )(L+1)}}{1-\rho ^{p\sigma
-\alpha }}\liminf_{n\rightarrow \infty }\Phi _{u}^{\sigma ,F}(\rho ^{n}).
\end{align*}

Combining this and (\ref{II-1}), we have%
\begin{equation*}
\rho ^{p\sigma +\alpha }C_{H}^{p\sigma +\alpha }\liminf_{n\rightarrow \infty
}\Phi _{u}^{\sigma ,F}(\rho ^{n})\leq C\liminf_{n\rightarrow \infty
}\sum_{k=0}^{L}\rho ^{k(p\sigma -\alpha )}\mathcal{E}_{k+n}^{\sigma ,F}(u)+%
\frac{C^{\prime }\rho ^{(p\sigma -\alpha )(L+1)}}{1-\rho ^{p\sigma -\alpha }}%
\liminf_{n\rightarrow \infty }\Phi _{u}^{\sigma ,F}(\rho ^{n}).
\end{equation*}

For sufficiently large $L$ ($>n_{0}$), we have
\begin{equation*}
C^{\prime \prime }:=\frac{1}{C}\left( \rho ^{p\sigma +\alpha }C_{H}^{p\sigma
+\alpha }-\frac{C^{\prime }\rho ^{(p\sigma -\alpha )(L+1)}}{1-\rho ^{p\sigma
-\alpha }}\right) >0.
\end{equation*}

Then for this $L$, we obtain
\begin{align}
C^{\prime \prime }\liminf_{n\rightarrow \infty }\Phi _{u}^{\sigma ,F}(\rho
^{n})& \leq \liminf_{n\rightarrow \infty }\sum_{k=0}^{L}\rho ^{k(p\sigma
-\alpha )}\mathcal{E}_{k+n}^{\sigma ,F}(u)  \notag \\
& =\liminf_{n\rightarrow \infty }\sum_{k=0}^{L}\rho ^{k(p\sigma -\alpha
)}\rho ^{-(k+n)(p\sigma -\alpha )}E_{k+n}^{(p),F}(u)  \notag \\
& =\liminf_{n\rightarrow \infty }\sum_{k=0}^{L}\rho ^{-n(p\sigma -\alpha
)}E_{k+n}^{(p),F}(u).  \label{E_kn}
\end{align}

Using \eqref{E-n}, there exists $c>0$ such that for all positive integer $n$
and $u\in B_{p,\infty }^{\sigma ,F}$,%
\begin{equation*}
E_{n}^{(p),F}(u)\leq cE_{n+1}^{(p),F}(u).
\end{equation*}
Applying the above inequality to \eqref{E_kn}, we have
\begin{align*}
C^{\prime \prime }\liminf_{n\rightarrow \infty }\Phi _{u}^{\sigma ,F}(\rho
^{n})& \leq \liminf_{n\rightarrow \infty }C_{L}\ \rho ^{-(n+L)(p\sigma
-\alpha )}E_{n+L}^{(p),F}(u) \\
& =C_{L}\liminf_{n\rightarrow \infty }\mathcal{E}_{n+L}^{(p),F}(u)=C_{L}%
\liminf_{n\rightarrow \infty }\mathcal{E}_{n}^{(p),F}(u),
\end{align*}%
where $C_{L}=\frac{1-c^{L}}{1-c}\rho ^{L(p\sigma -\alpha )}$ does not depend
on $n$, which proves \eqref{limphi}.
\end{proof}

\begin{proof}[Proof of Theorem \protect\ref{thm5}]
By \eqref{eq6.2}, \eqref{limphi2} and (\ref{ne}), $(\widetilde{VE})$ with $%
\sigma >\alpha /p$ implies $(\widetilde{NE})$ since
\begin{equation}
\limsup_{r\rightarrow 0}\Phi _{u}^{\sigma ,F}(r) \leq C\limsup_{n\rightarrow
\infty }\mathcal{E}_{n}^{\sigma ,F}(u)\leq C^{\prime }\liminf_{n\rightarrow
\infty }\mathcal{E}_{n}^{\sigma ,F}(u)\leq C^{\prime \prime
}\liminf_{r\rightarrow 0}\Phi _{u}^{\sigma ,F}(r).  \label{ve_ne1}
\end{equation}%
By \eqref{eq6.2}, (\ref{ne}) and \eqref{limphi}, $(\widetilde{NE})$ with $%
\sigma >\alpha /p$ implies $(\widetilde{VE})$ since
\begin{equation}
\limsup_{n\rightarrow \infty }\mathcal{E}_{n}^{\sigma ,F}(u)\leq
C_{1}\limsup_{r\rightarrow 0}\Phi _{u}^{\sigma ,F}(r)\leq C_{1}^{\prime
}\liminf_{r\rightarrow 0}\Phi _{u}^{\sigma ,F}(r)\leq C_{1}^{\prime \prime
}\liminf_{n\rightarrow \infty }\mathcal{E}_{n}^{\sigma ,F}(u).
\label{ne_ve1}
\end{equation}%
Thus, by \eqref{ve_ne1} and \eqref{ne_ve1}, we show Theorem \ref{thm5} (i).

By \eqref{eq6.2} and \eqref{limphi2} adjoint with $(VE)\Rightarrow (%
\widetilde{VE})$, we immediately have from (\ref{ns}), (\ref{ne}) that
\begin{equation}
\sup_{r\in (0,C_H)}\Phi _{u}^{\sigma ,F}(r)\leq C_{2}\sup_{n\geq 0 }\mathcal{%
E}_{n}^{\sigma ,F}(u)\leq C_{2}^{\prime }\liminf_{n\rightarrow \infty }%
\mathcal{E}_{n}^{\sigma ,F}(u)\leq C_{2}^{\prime \prime
}\liminf_{r\rightarrow 0}\Phi _{u}^{\sigma ,F}(r).  \label{ve_ne2}
\end{equation}%
Similarly, by \eqref{eq6.2} and \eqref{limphi} adjoint with the fact $%
(NE)\Rightarrow (\widetilde{NE})$, we immediately have
\begin{equation}
\sup_{n\geq 0}\mathcal{E}_{n}^{\sigma ,F}(u)\leq C_{3}\sup_{n\geq 0}\Phi
_{u}^{\sigma ,F}(C_H\rho ^{n})\leq C_{3}^{\prime }\liminf_{n\rightarrow
\infty }\Phi _{u}^{\sigma ,F}(\rho ^{n})\leq C_{3}^{\prime \prime
}\liminf_{n\rightarrow \infty }\mathcal{E}_{n}^{\sigma ,F}(u).
\label{ne_ve2}
\end{equation}%
Thus, by \eqref{ve_ne2} and \eqref{ne_ve2}, we show Theorem \ref{thm5} (ii).
\end{proof}

\begin{remark}
The definition of discrete energies in \eqref{p_energy} uses point-wise values of functions as opposed to cell-averaged values considered in \cite{Kigami2022penergy,KS92}.
When $p\sigma >\alpha$, the domain of p-energy $\mathcal{E}_{p,\infty}^{\sigma,F}$ is a subset of $C(K^F)$, and both definitions are equivalent for a connected homogeneous p.c.f. self-similar set by modifying the arguments in \cite[Lemma 3.1]{HK06} and
\cite[Section 5]{Shimuzu.2022}. We will also study the relation between the weak-monotonicity of cell-averaged discrete norms and our weak-monotonicity properties in an upcoming work.
\end{remark}



\subsection{Consequences of (VE): BBM type characterization and
Gagliardo-Nirenberg inequality}

\label{subsec5.4} Let us state the following embedding
inequality given by Baudoin, using our notions.

\begin{lemma}
\label{GNTM}(\cite[Theorem 4.3]{BaudoinLecture2022}) Suppose that property
(NE) holds with $R_0=\infty$ for the metric measure space $(M,d,\mu )$, and
there exists $R>0$ such that $\inf_{x\in M}\mu (B(x,R))>0$. When $p\sigma
_{p}^{\#}\neq \alpha_1 $, where $\alpha_1$ is from (\ref{VD2}), let $q=\frac{%
p\alpha_1 }{\alpha_1 -p\sigma_{p}^{\#}}$. For $r,s\in (0,\infty ]$, $\theta
\in (0,1]$ satisfying
\begin{equation*}
\frac{1}{r}=\frac{\theta }{q}+\frac{1-\theta }{s},
\end{equation*}%
there exists a constant $C>0$ such that for any $f\in B_{p,\infty }^{\sigma
_{p}^{\#}}$,
\begin{equation}
\Vert f\Vert _{r}\leq C(\Vert f\Vert _{p}+[f]_{B_{p,\infty }^{\sigma
_{p}^{\#}}})^{\theta }\Vert f\Vert _{s}^{1-\theta }.  \label{GNI}
\end{equation}
\end{lemma}

By Theorem \ref{thm4} and Theorem \ref{thm5}, we have the following
corollary.

\begin{corollary}
\label{corol1.8}Suppose that $K^{F}$ is a fractal glue-up that admits a heat
kernel satisfying \eqref{hk_F}, where $K$ is a connected homogeneous p.c.f.
self-similar set. Let $R_0=C_H$. Then properties $(\widetilde{KE}),(%
\widetilde{NE}),(\widetilde{VE})$ are equivalent with the same $%
\sigma>\alpha /p$, and properties $(KE),(NE),(VE)$ are equivalent when $%
\sigma_{p}^{\#}>\alpha /p$ on $K^{F}$.
\end{corollary}

The following theorem can be regarded as a summary of our paper in the context of fractal spaces.

\begin{theorem}
\label{corol1.9}Let $K^{F}$ be a fractal glue-up, where $K$ is a connected
homogeneous p.c.f. self-similar set satisfying property (E). Then $K^{F}$
satisfies (NE) with $R_{0}=C_H$ and $B_{p,\infty }^{\sigma
_{p}^{\#}(K^{F})}(K^{F}):=B_{p,\infty }^{\sigma _{p}^{\#}}(K^{F})$ contains
non-constant functions. Also, the BBM type characterization (%
\ref{NE_conv}) holds for $K^{F}$ with $R_{0}=C_H$, that is, for $R_{0}=C_H$,
there exists a positive constant $C$ such that for all $u\in B_{p,\infty
}^{\sigma _{p}^{\#}}(K^{F})$,
\begin{equation}
C^{-1}[u]_{{B}_{p,\infty }^{\sigma _{p}^{\#}}(K^{F})}^{p}\leq
\liminf_{\sigma \uparrow \sigma _{p}^{\#}(K^{F})}\left( \sigma
_{p}^{\#}(K^{F})-\sigma \right) [u]_{{B}_{p,p}^{\sigma }(K^{F})}^{p}\leq
\limsup_{\sigma \uparrow \sigma _{p}^{\#}(K^{F})}\left( \sigma
_{p}^{\#}(K^{F})-\sigma \right) [u]_{{B}_{p,p}^{\sigma }(K^{F})}^{p}\leq
C[u]_{{B}_{p,\infty }^{\sigma _{p}^{\#}}(K^{F})}^{p}.  \label{SS-2}
\end{equation}%
Furthermore, if $K$ is a nested fractal and $K^{F}=K_{\infty }$, where $%
K_{\infty }$ is defined in Example \ref{ex1}, then (NE) and (KE) are
satisfied with $R_{0}=\infty $, and the Gagliardo-Nirenberg inequality (\ref%
{GNI}) holds true.
\end{theorem}

\begin{proof}
By Proposition \ref{prop:critical}, $K^{F}$ satisfies (VE) and the critical domain $B_{p,\infty
}^{\sigma _{p}^{\#}}(K^{F})$ contains
non-constant functions, where $\sigma _{p}^{\#}=\sigma _{p}^{\#}(K)=\sigma
_{p}^{\#}(K^{F})$. Thus (NE) holds true for $K^{F}$
by Theorem \ref{thm5} with $R_{0}=C_H$, then (\ref{SS-2}) holds by Lemma \ref%
{thm4.2}.

Since a nested fractal $K$ satisfies property (E) by Lemma \ref%
{lemma:E}, these inclusions hold for nested fractals. Next, we state how to upgrade $R_0=C_H$
to $R_0=\infty $ for (NE) on $K_{\infty }$.

For any $n\in \mathbb{N}$, define the scaling function $g_{n}:K^{F}%
\rightarrow \rho ^{-n}K^{F}$ by $g_{n}(x)=\rho ^{-n}x.$ Due to the global
self-similarity of $K_{\infty }$, we have
\begin{equation}
g_{n}(K_{\infty })=K_{\infty }.  \label{ss}
\end{equation}
For any $u\in B_{p,\infty }^{\sigma ,F}$, clearly $u\circ g_{n}\in
B_{p,\infty }^{\sigma ,F}$. By definition, for all $r>0$ and all
$u\in B_{p,\infty }^{\sigma ,F}$,
\begin{eqnarray}
\Phi _{u\circ g_{n}}^{\sigma ,F}(\rho ^{n}r) &=&(\rho ^{n}r)^{-p\sigma
-\alpha }\int_{K_{\infty }}\int_{B(x,\rho
^{n}r)}|u(g_{n}(x))-u(g_{n}(y))|d\mu ^{F}(y)d\mu ^{F}(x)  \notag \\
&=&(\rho ^{n}r)^{-p\sigma -\alpha }\int_{K_{\infty
}}\int_{B(g_{n}(x),r)}|u(g_{n}(x))-u(y)|d(\mu ^{F}\circ g_{n}^{-1})(y)d\mu
^{F}(x)  \notag \\
&=&(\rho ^{n}r)^{-p\sigma -\alpha }\int_{g_{n}(K_{\infty
})}\int_{B(x,r)}|u(x)-u(y)|d\mu ^{F}(\rho ^{n}y)d\mu ^{F}(\rho ^{n}x)  \notag
\\
&\asymp &(\rho ^{n}r)^{-p\sigma -\alpha }\rho ^{2n\alpha }\int_{K_{\infty
}}\int_{B(x,r)}|u(x)-u(y)|d\mu ^{F}(y)d\mu ^{F}(x)\   \notag \\
&=&\rho ^{-n(p\sigma -\alpha )}\Phi _{u}^{\sigma ,F}(r),  \label{sp}
\end{eqnarray}%
where we use (\ref{ss}) and that $\mu ^{F}$ is $\alpha $-regular in the forth
line. For any $u\in B_{p,\infty }^{\sigma ,F}$, there exists $n<\infty $
such that
\begin{equation*}
\sup_{r\in (0,\infty )}\Phi _{u}^{\sigma ,F}(r)\leq 2\sup_{r\in (0,C_{H}\rho
^{-n})}\Phi _{u}^{\sigma ,F}(r).
\end{equation*}%
Combined with (\ref{sp}),
\begin{equation}
\sup_{r\in (0,\infty )}\Phi _{u}^{\sigma ,F}(r)\leq C\sup_{r^{\prime }\in
(0,C_{H})}\Phi _{u\circ g_{n}}^{\sigma ,F}(r^{\prime })\rho ^{n(p\sigma
-\alpha )}.  \label{sp-1}
\end{equation}%
As (NE) holds for $R_{0}=C_{H}$,
\begin{eqnarray*}
\sup_{r^{\prime }\in (0,C_{H})}\Phi _{u\circ g_{n}}^{\sigma
_{p}^{\#},F}(r^{\prime })\rho ^{n(p\sigma _{p}^{\#}-\alpha )} &\leq
&C\liminf_{r^{\prime }\rightarrow 0}\Phi _{u\circ g_{n}}^{\sigma
_{p}^{\#},F}(r^{\prime })\rho ^{n(p\sigma _{p}^{\#}-\alpha )} \\
&=&C\liminf_{r^{\prime }\rightarrow 0}\Phi _{u}^{\sigma _{p}^{\#},F}(\rho
^{-n}r^{\prime })\text{ \ (by (\ref{sp}))} \\
&=&C\liminf_{r\rightarrow 0}\Phi _{u}^{\sigma _{p}^{\#},F}(r),\text{ }
\end{eqnarray*}%
where the constant $C$ does not depend on $u,n$. Thus (NE) also
holds with $R_{0}=\infty $ by (\ref{sp-1}). Since $p\sigma _{p}^{\#}>\alpha $
and that $\mu ^{F}$ is $\alpha $-regular, we
know by Lemmas \ref{lemma:E} and \ref{GNTM} that (\ref{GNI}) holds true. Finally, for a nested fractal $K$, it is known by \cite{Kumagai.1993.PTaRF205} (see also \cite[%
Theorem 1.1]{FitzsimmonsmHamblyKumagai.1994.CMP595}) that $%
K_{\infty }$ admits a heat kernel satisfying \eqref{hk_F}, thus (KE) is verified by Theorem \ref{thm4}. The proof is complete.
\end{proof}

When $p=2$, we use heat kernel estimates to verify weak-monotonicity properties in the
following corollary, which is analytical (compared with the
algebraic resistance-estimate arguments for (VE)).

\begin{corollary}
\label{jjj} When $p=2$, if a fractal glue-up $K^{F}$ admits a heat kernel with two-sided
estimates \eqref{hk_F}, then (VE) and (NE) automatically hold for $K^F $.
\end{corollary}

\begin{proof}
By Remark \ref{rk1}, property (KE) automatically holds for $K^F$ when $p=2$.
The conclusion then follows from Corollary \ref{corol1.8}.
\end{proof}



\begin{acknowledgement}
The authors thank Jiaxin Hu and Hua Qiu for helpful suggestions on the manuscript. The authors also appreciate the suggestions and revisions
made by anonymous referees. Jin Gao was supported by National Natural Science Foundation of China (12271282), and Zhejiang Provincial Natural Science Foundation
of China (LQN25A010019). Zhenyu Yu was supported by the Natural Science Foundation of Hunan Province, China (No.2025JJ60039).
\end{acknowledgement}

\begin{thebibliography}{99}
\bibitem{AlonsoBaudoinchen2020JFA} \textsc{P.~Alonso-Ruiz, F.~Baudoin,
L.~Chen, L.~Rogers, N.~Shanmugalingam, and A.~Teplyaev}, \emph{Besov class
via heat semigroup on {D}irichlet spaces {I}: {S}obolev type inequalities},
J. Funct. Anal., 278 (2020), pp.~108459, 48.

\bibitem{AlonsoBaudoinchen2020CVPDE} \leavevmode\vrule height 2pt depth
-1.6pt width 23pt, \emph{Besov class via heat semigroup on {D}irichlet
spaces {II}: {BV} functions and {G}aussian heat kernel estimates}, Calc.
Var. Partial Differential Equations, 59 (2020), pp.~Paper No.103, 32.

\bibitem{AlonsoBaudoinchen2021CVPDE} \leavevmode\vrule height 2pt depth
-1.6pt width 23pt, \emph{Besov class via heat semigroup on {D}irichlet
spaces {III}: {BV} functions and sub-{G}aussian heat kernel estimates},
Calc. Var. Partial Differential Equations, 60 (2021), pp.~Paper No. 170, 38.

\bibitem{Barlow.1998.1} \textsc{M.~Barlow}, \emph{Diffusions on fractals},
vol.~{1690} of Lect. Notes Math., Springer, 1998, pp.~1--121.

\bibitem{BarlowBass.1992.PTRF307} \textsc{M.~Barlow and R.~Bass}, \emph{%
Transition densities for {B}rownian motion on the the {S}ierp\'{\i}nski
carpet}, Probab. Theory Related Fields, {91} (1992), pp.~307--330.

\bibitem{BarlowBass.1999.CJM673} \leavevmode\vrule height 2pt depth -1.6pt
width 23pt, \emph{Brownian motion and harmonic analysis on {S}ierp\'{\i}nski
carpets}, Canad. J. Math., {51} (1999), pp.~673--744.

\bibitem{BarlowHambaly.1997.AIHS} \textsc{M.~Barlow and B.~Hambly}, \emph{%
Transition density estimates for {B}rownian motion on scale irregular {S}%
ierpi\'{n}ski gaskets}, Ann. Inst. H. Poincar\'e Probab. Statist., {33}
(1997), pp.~531--557.

\bibitem{BP88} \textsc{M. Barlow and E. Perkins}. \emph{Brownian motion on
the {S}ierpi\'{n}ski gasket}.Probab. Theory Related Fields, {79} (1988),
pp.~543--623.

\bibitem{BaudoinLecture2022} \textsc{F.~Baudoin}, \emph{Korevaar-{S}choen-{S}%
obolev spaces and critical exponents in metric measure spaces}, Ann. Fenn.
Math., 49 (2024) no. 2, pp.~487--527.

\bibitem{BC23} \textsc{F. Baudoin and L. Chen}, \emph{Sobolev spaces and
Poincar\'e{} inequalities on the Vicsek fractal}, Ann. Fenn. Math., 48
(2023), no.~1, pp.~3--26.

\bibitem{BourgainBrezisMironescu.2001.439} \textsc{J.~Bourgain, H.~Brezis,
and P.~Mironescu}, \emph{Another look at Sobolev spaces, Optimal Control and
Partial Differential Equations (J.L. Menaldi et al. eds)}, IOS Press,
Amsterdam, 2001, pp.~439--455.

\bibitem{Caoqiu23} \textsc{S.~Cao and H.~Qiu}, \emph{Equivalence of Besov
spaces on p.c.f. self-similar sets}, Canad. J. Math. 76 (2024), no. 4,
pp.~1109--1143.

\bibitem{Caoqiugu2022adv} \textsc{S.~Cao, Q.~Gu, and H.~Qiu}, \emph{{$p$}%
-energies on p.c.f. self-similar sets}, Adv. Math., 405 (2022), pp.~Paper
No. 108517, 58.

\bibitem{CGYZ24} \textsc{D. Chang, J. Gao, Z. Yu and J. Zhang}, \emph{Weak
monotonicity property of Korevaar-Schoen norms on nested fractals}, J. Math.
Anal. Appl. 540 (2024), no. 1, pp.~128623.

\bibitem{Denglau.2008} \textsc{Q.-R. Deng and K.-S. Lau}, \emph{Open set
condition and post-critically finite self-similar sets}, Nonlinearity, 21
(2008), pp.~1227--1232.

\bibitem{DS19} \textsc{S. Di~Marino and M. Squassina}, \emph{New
characterizations of Sobolev metric spaces}, J. Funct. Anal. 276 (2019),
no.~6, pp.~1853--1874.

\bibitem{FitzsimmonsmHamblyKumagai.1994.CMP595} \textsc{P.~Fitzsimmonsm,
B.~Hambly, and T.~Kumagai}, \emph{Transition density estimates for {B}%
rownian motion on affine nested fractals}, Comm. Math. Phys., {165} (1994),
pp.~595--620.

\bibitem{FukushimaOshimaTakeda.2011.489} \textsc{M.~Fukushima, Y.~Oshima,
and M.~Takeda}, \emph{Dirichlet forms and symmetric {M}arkov processes},
vol.~19 of de Gruyter Studies in Mathematics, Walter de Gruyter \& Co.,
Berlin, extended~ed., 2011.

\bibitem{GaoYuZhang2022PA} \textsc{J.~Gao, Z.~Yu, and J.~Zhang}, \emph{%
Convergence of p-energy forms on homogeneous p.c.f self-similar sets},
Potential Anal., 59 (2023), pp.~1851--1874.


\bibitem{GrigoryanHu.2014.MMJ505} \textsc{A.~Grigor'yan and J.~Hu}, \emph{%
Upper bounds of heat kernels on doubling spaces}, Moscow Math. J., {14}
(2014), pp.~505--563.

\bibitem{GrigoryanHuLau.2003.TAMS2065} \textsc{A.~Grigor'yan, J.~Hu, and
K.-S. Lau}, \emph{Heat kernels on metric measure spaces and an application
to semilinear elliptic equations}, Trans. Amer. Math. Soc., 355 (2003),
pp.~2065--2095.

\bibitem{GrigoryanTelcs.2012.AoP1212} \textsc{A.~Grigor'yan and A.~Telcs},
\emph{Two-sided estimates of heat kernels on metric measure spaces}, Annals
of Probability, {40} (2012), pp.~1212--1284.

\bibitem{GuLau.2020.AASFM} \textsc{Q.~Gu and K.-S. Lau}, \emph{Dirichlet
forms and convergence of {B}esov norms on self-similar sets}, Ann. Acad.
Sci. Fenn. Math., 45 (2020), pp.~625--646.

\bibitem{GuLau.2020.TAMS} \leavevmode\vrule height 2pt depth -1.6pt width
23pt, \emph{Dirichlet forms and critical exponents on fractals}, Trans.
Amer. Math. Soc., 373 (2020), pp.~1619--1652.

%\bibitem{Gorny22.JGA} \textsc{G\'orny, Wojciech}, \emph{%
%Bourgain-Brezis-Mironescu approach in metric spaces with Euclidean tangents}%
%, J. Geom. Anal. 373 (2022), pp.~128.

\bibitem{HermanPeironeStrichartz.2004.PA125} \textsc{P.~Herman, R.~Peirone,
and R.~Strichartz}, \emph{$p$-energy and $p$-harmonic functions on {S}%
ierpinski gasket type fractals}, Potential Anal., 20 (2004), pp.~125--148.

\bibitem{HK06} \textsc{M. Hino and T. Kumagai}, \emph{A trace theorem for
Dirichlet forms on fractals}, J. Funct. Anal. 238 (2006), no. 2,
pp.~578--661.

\bibitem{KajinoMurugan.2020On} \textsc{N.~Kajino and M.~Murugan}, \emph{On
singularity of energy measures for symmetric diffusions with full
off-diagonal heat kernel estimates}, Ann. Probab., 48 (2020), pp.~2920--2951.

\bibitem{KS24} \textsc{N.~Kajino and R.~Shimizu}, \emph{Korevaar-Schoen
p-energy forms and associated p-energy measures on fractals},
arXiv:2404.13435v2, (2024).

\bibitem{Kigami.1993.TAMS721} \textsc{J.~Kigami}, \emph{Harmonic calculus on
p.c.f. self-similar sets}, Trans. Amer. Math. Soc., {335} (1993),
pp.~721--755.

\bibitem{Kigami.2001.} \leavevmode\vrule height 2pt depth -1.6pt width 23pt,
\emph{Analysis on Fractals}, Cambridge Univ. Press, 2001.

\bibitem{Kigami2022penergy} \textsc{J.~Kigami}, \emph{Conductive homogeneity
of compact metric spaces and construction of p-energy}, Mem. Eur. Math. Soc.
Vol.\textbf{5}, 2023.

\bibitem{Kumagai.1993.PTaRF205} \textsc{T.~Kumagai}, \emph{Estimates of the
transition densities for brownian motion on nested fractals}, Probab. Theory
and Related Fields, {96} (1993), pp.~205--224.

\bibitem{KS92} \textsc{S. Kusuoka, and X.~Zhou}, \emph{Dirichlet forms on
fractals: Poincar\'{e} constant and resistance}. Probab. Th. Rel. Fields 93
(1992), pp.~169--196.

\bibitem{LiYau.1986.AM153} \textsc{P.~Li and S.~Yau}, \emph{On the parabolic
kernel of the {S}chr\"{o}dinger operator}, Acta Math., {156} (1986),
pp.~153--201.

\bibitem{Munnier15} \textsc{V.~Munnier}, \emph{Integral energy
characterization of Haj\l asz-Sobolev spaces}, J. Math. Anal. Appl. \textbf{%
425} (2015), no.~1, pp.~381--406.

\bibitem{Murugan.2020.JFA} \textsc{M.~Murugan}, \emph{On the length of
chains in a metric space}, J. Funct. Anal., 279 (2020), pp.~108627, 18.

\bibitem{MuruganShimizu23} \textsc{M.~Murugan and R.~Shimizu}, \emph{\
First-order sobolev spaces, self-similar energies and energy measures on
sierpi\'{n}ski carpet}, to appear Communications in Pure and Applied Mathematics, (2025).

\bibitem{Pietruska-Paluba.2008} \textsc{K.~Pietruska-Pa{\l }uba}, \emph{%
Limiting behaviour of {D}irichlet forms for stable processes on metric spaces%
}, Bull. Pol. Acad. Sci. Math., 56 (2008), pp.~257--266.

\bibitem{Pietruska-Paluba.2010.} \leavevmode\vrule height 2pt depth -1.6pt
width 23pt, \emph{Heat kernel characterisation of {B}esov-{L}ipschitz spaces
on metric measure spaces}, Manuscripta Math., 131 (2010), pp.~199--214.

\bibitem{Shimuzu.2022} \textsc{R.~Shimizu}, \emph{Construction of p-energy
and associated energy measures on the {S}ierpinski carpet}, Trans. Amer.
Math. Soc., 377 (2024), pp.~951--1032.

\bibitem{Strichartz.1998.CJM} \textsc{R.~S. Strichartz}, \emph{Fractals in
the large}, Canad. J. Math., 50 (1998), pp.~638--657.

\bibitem{StrichartzTeplyaev2012} \textsc{R.~S. Strichartz and A.~Teplyaev},
\emph{Spectral analysis on infinite {S}ierpi\'{n}ski fractafolds}, J. Anal.
Math., 116 (2012), pp.~255--297.

\bibitem{Yang.2018.PA} \textsc{M.~Yang}, \emph{Equivalent semi-norms of
non-local {D}irichlet forms on the {S}ierpi\'{n}ski gasket and applications}%
, Potential Anal., 49 (2018), pp.~287--308.

%\bibitem{Yang2022MZ} \textsc{M.~Yang}, \emph{On the domains of {D}irichlet
%forms on metric measure spaces}, Math. Z., 301 (2022), pp.~2129--2154.

\bibitem{Yang23} \textsc{M.~Yang}, \emph{Korevaar-schoen spaces on sierpi%
\'{n}ski carpets}, arXiv:2306.09900v1, (2023).
\end{thebibliography}

\end{document}
