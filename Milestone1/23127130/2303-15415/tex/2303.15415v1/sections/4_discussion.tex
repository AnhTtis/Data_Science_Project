\section{Discussion}
% * The challenges indie platforms face are not directly related to what kind of policy they need. \\
% * There is a lack of trust from indie platforms toward public policy. \\
% * More HCI research for understanding the needs of emerging platforms\\
% * The indie platform is just a microcosm of the entire market, this issue can be generalized to other marketplaces\\
% * Technology design should be easy to adapt, with the consideration of both sides of opinions e.g. microservices instead of the monolith.

We organize this section with three main takeaways. First, from analyzing the responses to the two questions, we learned that there is no direct connection between the challenges indie platforms face and their expectations of public policy, which indicates the necessity of understanding their needs. Second, our findings showed conflicting opinions regarding public policy in the food delivery market. Some stakeholders think policymakers should not be involved in the food delivery market, while others are open to changes with specific suggestions. We bring up the question to the HCI community about how to reconcile these controversial opinions. In our third insight, we discuss design implications inspired by the study and hope to generalize our findings to other marketplaces and inspire future HCI research. 

\subsection{Necessity of the Understanding the Need of Indie Platforms}
In the previous section, we reported the statistical results of the questions about challenges faced by indie platforms and the support they needed from policymakers. We do not see a connection between the obstacles encountered by indie platforms and the specific public policies they wish to see put into effect. One explanation is that indie platforms could benefit if mainstream platforms were restricted to some extent, and they depend on policymakers to fulfill that. We illustrate our research findings here and hope it could provide insights for future related HCI research. 

\subsection{How to Reconcile Controversial Opinions in Designing of Public Policy}
In this study, we saw conflicting opinions about whether policymakers should be involved in the market. Previous studies have pointed out that there are different beliefs on whether the market should adjust by itself without the government's invention~\cite{timmer1989food, little1982economic, braverman1986rural}. We noticed the same debate in the food delivery market. At the Policy and Tech workshop, we intend to brainstorm policy solutions for reconciling the conflicted ideas presented by indie platform owners.

\subsection{Easily-Adapted Technology to Reflect Diverse Opinions}
As previous studies proposed, a potential technical solution to the problem indie platforms face is building public-interest software~\cite{nosh2023, indie2023cscw}. Due to the diverse opinions regarding public policy, the needs of stakeholders could be very different. So we suggest that when designing public-interest technologies, it should be easy to adapt according to different policy requirements. One potential way is to utilize and deploy numerous small, independent microservices instead of bundling all functionalities into a single monolithic application. In this way, fast and agile changes in the system could be made according to the policy changes ~\cite{chen2017monolith}.  Moreover, as we mentioned above, the phenomena we observed in the food delivery market may not be a single case, similar design insights about policy and technology could also be applicable in other marketplaces. 