\section{background}
% "Indie", referring to platforms that are not mainstream in game and film industries~\cite{newman2011indie, lipkin2013examining}, is also used in food delivery research for platforms alternative to Uber Eats, Doordash, and Grubhub~\cite{nosh2023, indie2023cscw}. Recent studies have found that indie platforms is a widespread sociotechnical phenomenon in the United States but it is understudied in the HCI community ~\cite{nosh2023, indie2023cscw}. On the technology level, indie platforms typically have less advanced infrastructures compared to mainstream platforms. They create their technical infrastructure by piecing together various software systems developed by third-party vendors~\cite{DeliveryAppsGrubhub2021,DeliveryCoopsProvide}. Without a fully-staffed software development team, indie platforms are at a significant disadvantage in terms of technical capabilities when compared to mainstream platforms~\cite{atkinsonMoreJobFood2021, WhyAreRestaurants2021}.  
% Such platforms usually focus their operation on a small and localized scale, having more human-centered design and human intervention in their business model, which helps them survive in such a competitive market~\cite{atkinsonMoreJobFood2021, schneiderExitCommunityStrategies2020, nosh2023}. Previous studies have discussed the technical limitation of indie platforms but haven't covered how public policy could be made to support their survival and growth~\cite{indie2023cscw, nosh2023}. In this paper, we reported the opinions from indie platforms regarding public policy based on the 23 responses we collected from a survey. Furthermore, we identify opportunities for HCI researchers to support indie platforms by designing technology and public policy simultaneously to their needs.

%In the gaming and film industries, "indie" refers to platforms that are not mainstream ~\cite{newman2011indie, lipkin2013examining}.
In this study, we investigate the needs and challenges of indie \footnote{We borrow the term ``indie'' from the gaming and film industry to refer to platforms that are not one of the large mainstream ones, i.e., Uber Eats, Doordash and Grubhub} food delivery platforms. 
%Similarly, an "indie platform" can also be used to describe food delivery platforms that offer alternatives to the large, mainstream platforms comprised of Uber Eats, Doordash, and Grubhub~\cite{nosh2023, indie2023cscw}. 
%Our recent studies have found that indie platforms are a widespread sociotechnical phenomenon in the United States, though they are understudied in the HCI community ~\cite{nosh2023, indie2023cscw}.  
Indie platforms typically focus their efforts on a small and localized scale, having more human intervention in their business operations than mainstream platforms. This helps them differentiate themselves in a competitive market~\cite{atkinsonMoreJobFood2021, schneiderExitCommunityStrategies2020, nosh2023}. Our previous studies have discussed the sociotechnical limitations of indie platforms but have not covered how platform operators see the role of public policy in their business~\cite{indie2023cscw, nosh2023}. In our ongoing project, we investigate the opinions and needs of indie platform operators' perspectives regarding public policy and how policy can support or hinder their business operations.  This paper presents the answers to policy questions on a survey of 24 operators of independent food delivery platforms. We identify opportunities for HCI researchers to design public policy and technology simultaneously for the food delivery market and other marketplaces.
