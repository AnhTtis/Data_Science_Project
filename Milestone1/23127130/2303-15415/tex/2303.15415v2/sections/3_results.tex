\section{Results}
The most common challenge platforms face is a lack of couriers (17 out of 24), followed by a low volume of orders (10 out of 24), a lack of funding (8 out of 24), a shortage of restaurants (7 out of 24), too many delivery services and not enough customers (5 out of 24), a shortage of customers (4 out of 24), and too many restaurants want to sign up (2 out of 24). One platform faced the challenge of a shortage of employees. No platform indicated they have an oversupply of couriers. 

We found that more than half of the survey respondents, specifically 13 out of 24, have specific suggestions on what should be changed currently regarding policy, while 2 out of 24 platforms believe that the current situation does not need to be changed. Seven replied with ``I'm not sure,''  ``I have no idea'' or similar. Two indie platform operators mentioned that their platforms are on a small scale, and they do not see how  public policy would impact their business. 

\subsection{Skepticism About the Role of Public Policy}
%Our findings suggest that there is a variation in opinions among food delivery platform operators when it comes to public policy related to their platforms. 
%There are controversial ideas about whether the status quo should be altered or maintained. The rationale for 
Several indie platform operators mentioned not wanting policymakers to interfere with their business. These respondents seem to equate policymakers to politicians and believe that policymakers do not understand their needs since they may not know how indie platforms operate. For instance:
\begin{quote}
    ``The best thing they can do to help is stay out of my way. We keep our rates fair and provide a great service. We don't need a politician to `fix' what isn't broken.'' (R4)
\end{quote}
\begin{quote}
    ``Like all businesses, we know our costs and charge what we need to be profitable. Legislators do not know our margins.'' (R18)
\end{quote}
Another viewpoint is that public policy exclusively benefits mainstream platforms, and indie platforms will not be affected due to the small scale they are operating. Respondents stated:
\begin{quote}
    ``Polices only help companies like Doordash or the like.'' (R2)
\end{quote}
\begin{quote}
    ``We aren't big enough to have this affect us.'' (R10)
\end{quote}

\subsection{Transparency, Lower Taxation, and Consent}
On the other hand, some indie platform operators articulated ways public policy could address various problems. We organized their needs into four categories: transparency, tax rates, restaurant consent, and others. 

For transparency, three respondents advocate that the platforms in the food delivery market should disclose information such as their commission rates from restaurants, the compensation they pay for couriers, and price markups on restaurant menus. For instance: 
\begin{quote}
    ``I think the biggest help would create a policy that provides complete transparency of pricing, and commission the service is taking from the restaurants. Not being able to combine the Sales Tax and Service Fee in the same line item.'' (R6)
\end{quote}

For tax rates, three respondents expressed their need for lowering tax rates, especially delivery fee sales tax, due to the current difficulty they have in making a profit: 
\begin{quote}
    ``Stop taxing us on delivery fees as all the little fees we pay makes it extremely hard to make any profit let alone continue to pay for all the expenses to be able to grow and maintain.'' (R24) 
\end{quote}

Another common need we found through the survey is restaurant consent. Preliminary studies have indicated that mainstream platforms list restaurants without their authorization.\cite{nosh2023}. This poses a problem for restaurants but also for indie platforms because they may have negotiated some deal with local restaurants for being exclusively listed on their platform. As pointed out by one of our survey respondents:
\begin{quote}
    ``We have exclusive deals set up with small locally owned restaurants and they constantly pop up on the national platforms. The big companies make it next to impossible for them to get themselves removed as well.'' (R11)
\end{quote}

Other needs we identified that were less commonly mentioned included access to parking, loans, and insurance. One respondent implied the inconvenience of getting parking spots for general delivery services in crowded vehicle areas in LA:
\begin{quote}
    ``Regulation on parking LA County / LA Metro is a crowdy vehicle area and lacks spaces for the park at the time of deliveries or pickups including harassing parking enforcement departments and the local police department, building structures with high fees of parking tickets without validation for RDS.'' (R23)
\end{quote}
Another respondent mentioned how indie platforms could benefit from an easier way to get SBA \footnote{Loans guaranteed by the Small Business Administration, see https://www.sba.gov/funding-programs/loans} loans. R3 suggested opening up regulations on alcohol delivery. 
% and R8 brought up the insurance cost issue for independent contractors. 