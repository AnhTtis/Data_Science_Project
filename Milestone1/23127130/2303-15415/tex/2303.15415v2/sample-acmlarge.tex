%%
%% This is file `sample-acmlarge.tex',
%% generated with the docstrip utility.
%%
%% The original source files were:
%%
%% samples.dtx  (with options: `acmlarge')
%% 
%% IMPORTANT NOTICE:
%% 
%% For the copyright see the source file.
%% 
%% Any modified versions of this file must be renamed
%% with new filenames distinct from sample-acmlarge.tex.
%% 
%% For distribution of the original source see the terms
%% for copying and modification in the file samples.dtx.
%% 
%% This generated file may be distributed as long as the
%% original source files, as listed above, are part of the
%% same distribution. (The sources need not necessarily be
%% in the same archive or directory.)
%%
%% Commands for TeXCount
%TC:macro \cite [option:text,text]
%TC:macro \citep [option:text,text]
%TC:macro \citet [option:text,text]
%TC:envir table 0 1
%TC:envir table* 0 1
%TC:envir tabular [ignore] word
%TC:envir displaymath 0 word
%TC:envir math 0 word
%TC:envir comment 0 0
%%
%%
%% The first command in your LaTeX source must be the \documentclass command.
\documentclass[acmlarge]{acmart}
%% NOTE that a single column version is required for 
%% submission and peer review. This can be done by changing
%% the \doucmentclass[...]{acmart} in this template to 
%% \documentclass[manuscript,screen,review]{acmart}
%% 
%% To ensure 100% compatibility, please check the white list of
%% approved LaTeX packages to be used with the Master Article Template at
%% https://www.acm.org/publications/taps/whitelist-of-latex-packages 
%% before creating your document. The white list page provides 
%% information on how to submit additional LaTeX packages for 
%% review and adoption.
%% Fonts used in the template cannot be substituted; margin 
%% adjustments are not allowed.
%%
%% \BibTeX command to typeset BibTeX logo in the docs
\AtBeginDocument{%
  \providecommand\BibTeX{{%
    \normalfont B\kern-0.5em{\scshape i\kern-0.25em b}\kern-0.8em\TeX}}}

%% Rights management information.  This information is sent to you
%% when you complete the rights form.  These commands have SAMPLE
%% values in them; it is your responsibility as an author to replace
%% the commands and values with those provided to you when you
%% complete the rights form.
\setcopyright{acmcopyright}
% \copyrightyear{2018}
% \acmYear{2018}
\acmDOI{XXXXXXX.XXXXXXX}


%%
%% These commands are for a JOURNAL article.
% \acmJournal{POMACS}
% \acmVolume{37}
% \acmNumber{4}
% \acmArticle{111}
% \acmMonth{8}

%%
%% Submission ID.
%% Use this when submitting an article to a sponsored event. You'll
%% receive a unique submission ID from the organizers
%% of the event, and this ID should be used as the parameter to this command.
%%\acmSubmissionID{123-A56-BU3}

%%
%% For managing citations, it is recommended to use bibliography
%% files in BibTeX format.
%%
%% You can then either use BibTeX with the ACM-Reference-Format style,
%% or BibLaTeX with the acmnumeric or acmauthoryear sytles, that include
%% support for advanced citation of software artefact from the
%% biblatex-software package, also separately available on CTAN.
%%
%% Look at the sample-*-biblatex.tex files for templates showcasing
%% the biblatex styles.
%%

%%
%% The majority of ACM publications use numbered citations and
%% references.  The command \citestyle{authoryear} switches to the
%% "author year" style.
%%
%% If you are preparing content for an event
%% sponsored by ACM SIGGRAPH, you must use the "author year" style of
%% citations and references.
%% Uncommenting
%% the next command will enable that style.
%%\citestyle{acmauthoryear}

%%
%% end of the preamble, start of the body of the document source.
\begin{document}

%%
%% The "title" command has an optional parameter,
%% allowing the author to define a "short title" to be used in page headers.
\title{Should Policymakers be Involved? Understanding the Opinions and Needs for Independent Food Delivery Platforms in the United States regarding Public Policy}

%%
%% The "author" command and its associated commands are used to define
%% the authors and their affiliations.
%% Of note is the shared affiliation of the first two authors, and the
%% "authornote" and "authornotemark" commands
%% used to denote shared contribution to the research.

\author{Yuhan Liu}
\affiliation{%
  \institution{Princeton University}
  % \streetaddress{35 Olden St}
  % \city{Princeton}
   \country{USA}
  }
\email{yl8744@princeton.edu}

\author{Amna Liaqat}
\affiliation{%
  \institution{Princeton University}
  % \streetaddress{35 Olden St}
  % \city{Princeton}
   \country{USA}
  }
\email{al0910@princeton.edu}

\author{Andrés Monroy-Hernández}
\affiliation{%
  \institution{Princeton University}
  % \streetaddress{35 Olden St}
  % \city{Princeton}
   \country{USA}
  }
\email{andresmh@princeton.edu}



%%
%% By default, the full list of authors will be used in the page
%% headers. Often, this list is too long, and will overlap
%% other information printed in the page headers. This command allows
%% the author to define a more concise list
%% of authors' names for this purpose.
\renewcommand{\shortauthors}{Liu et al.}

%%
%% The abstract is a short summary of the work to be presented in the
%% article.
\begin{abstract}
%Recent studies have shown that independent food delivery services are a widespread fixture of the food delivery landscape in the United States but are under-studied and under-supported. 
Mainstream food delivery platforms, like DoorDash and Uber Eats, have been the locus of fierce policy debates about their unfair business and labor practices. At the same time, hundreds of independent food delivery services provide alternative opportunities to many communities across the U.S. We surveyed operators of independent food delivery platforms to learn about their perception of the role of public policy. We found conflicting opinions on whether and how policy should interact with their businesses, ranging from not wanting policymakers to interfere to articulating specific policies that would curtail mainstream platforms' business practices. 
%We hope our research on the food delivery industry could bring awareness of the conflicted arguments from stakeholders in the food delivery industry and could be generalized to other marketplace areas. 
We provide insights for technologists and policymakers interested in the sociotechnical challenges of local marketplaces.
\end{abstract}

%%
%% The code below is generated by the tool at http://dl.acm.org/ccs.cfm.
%% Please copy and paste the code instead of the example below.
%%
\begin{CCSXML}
<ccs2012>
 <concept>
  <concept_id>10010520.10010553.10010562</concept_id>
  <concept_desc>Computer systems organization~Embedded systems</concept_desc>
  <concept_significance>500</concept_significance>
 </concept>
 <concept>
  <concept_id>10010520.10010575.10010755</concept_id>
  <concept_desc>Computer systems organization~Redundancy</concept_desc>
  <concept_significance>300</concept_significance>
 </concept>
 <concept>
  <concept_id>10010520.10010553.10010554</concept_id>
  <concept_desc>Computer systems organization~Robotics</concept_desc>
  <concept_significance>100</concept_significance>
 </concept>
 <concept>
  <concept_id>10003033.10003083.10003095</concept_id>
  <concept_desc>Networks~Network reliability</concept_desc>
  <concept_significance>100</concept_significance>
 </concept>
</ccs2012>
\end{CCSXML}

%\ccsdesc[500]{Computer systems organization~Embedded systems}
%\ccsdesc[300]{Computer systems organization~Redundancy}
%\ccsdesc{Computer systems organization~Robotics}
%\ccsdesc[100]{Networks~Network reliability}

%%
%% Keywords. The author(s) should pick words that accurately describe
%% the work being presented. Separate the keywords with commas.
\keywords{gig economy, food delivery}

%\received{20 February 2007}
%\received[revised]{12 March 2009}
%\received[accepted]{5 June 2009}

%%
%% This command processes the author and affiliation and title
%% information and builds the first part of the formatted document.
\maketitle

\section{background}
% "Indie", referring to platforms that are not mainstream in game and film industries~\cite{newman2011indie, lipkin2013examining}, is also used in food delivery research for platforms alternative to Uber Eats, Doordash, and Grubhub~\cite{nosh2023, indie2023cscw}. Recent studies have found that indie platforms is a widespread sociotechnical phenomenon in the United States but it is understudied in the HCI community ~\cite{nosh2023, indie2023cscw}. On the technology level, indie platforms typically have less advanced infrastructures compared to mainstream platforms. They create their technical infrastructure by piecing together various software systems developed by third-party vendors~\cite{DeliveryAppsGrubhub2021,DeliveryCoopsProvide}. Without a fully-staffed software development team, indie platforms are at a significant disadvantage in terms of technical capabilities when compared to mainstream platforms~\cite{atkinsonMoreJobFood2021, WhyAreRestaurants2021}.  
% Such platforms usually focus their operation on a small and localized scale, having more human-centered design and human intervention in their business model, which helps them survive in such a competitive market~\cite{atkinsonMoreJobFood2021, schneiderExitCommunityStrategies2020, nosh2023}. Previous studies have discussed the technical limitation of indie platforms but haven't covered how public policy could be made to support their survival and growth~\cite{indie2023cscw, nosh2023}. In this paper, we reported the opinions from indie platforms regarding public policy based on the 23 responses we collected from a survey. Furthermore, we identify opportunities for HCI researchers to support indie platforms by designing technology and public policy simultaneously to their needs.

%In the gaming and film industries, "indie" refers to platforms that are not mainstream ~\cite{newman2011indie, lipkin2013examining}.
In this study, we investigate the needs and challenges of indie \footnote{We borrow the term ``indie'' from the gaming and film industry to refer to platforms that are not one of the large mainstream ones, i.e., Uber Eats, Doordash and Grubhub} food delivery platforms. 
%Similarly, an "indie platform" can also be used to describe food delivery platforms that offer alternatives to the large, mainstream platforms comprised of Uber Eats, Doordash, and Grubhub~\cite{nosh2023, indie2023cscw}. 
%Our recent studies have found that indie platforms are a widespread sociotechnical phenomenon in the United States, though they are understudied in the HCI community ~\cite{nosh2023, indie2023cscw}.  
Indie platforms typically focus their efforts on a small and localized scale, having more human intervention in their business operations than mainstream platforms. This helps them differentiate themselves in a competitive market~\cite{atkinsonMoreJobFood2021, schneiderExitCommunityStrategies2020, nosh2023}. Our previous studies have discussed the sociotechnical limitations of indie platforms but have not covered how platform operators see the role of public policy in their business~\cite{indie2023cscw, nosh2023}. In our ongoing project, we investigate the opinions and needs of indie platform operators' perspectives regarding public policy and how policy can support or hinder their business operations.  This paper presents the answers to policy questions on a survey of 24 operators of independent food delivery platforms. We identify opportunities for HCI researchers to design public policy and technology simultaneously for the food delivery market and other marketplaces.

\section{Survey and results}
We launched a survey in collaboration with the RMDA (Restaurant Marketing Delivery Association), a nonprofit organization of independent food delivery services. We distributed our survey through the RMDA's member mailing list. In the survey, we asked multiple-choice, and open-ended questions about the challenges indie platforms are facing and what kind of support they need to tackle the challenges from the public policy perspective. In this workshop paper, we share the results of two of the questions that concern public policy:

\begin{enumerate}
    \item  Which of the following challenges is [name of their platform] facing? Select all that apply.
        \begin{itemize}
            \item Shortage of restaurants
            \item Too many restaurants want to sign up
            \item Shortage of couriers, e.g., drivers
            \item Oversupply of couriers, e.g., drivers
            \item Shortage of employees
            \item Too many delivery services and not enough customers
            \item Low volume of orders
            \item Shortage of customers
            \item Lack of funding
            \item Others (please specify)
        \end{itemize}
    \item (Open-ended question) How could policymakers help [the name of their platform] tackle the challenges it faces, if at all?
\end{enumerate}

Twenty-four platform operators responded. We conducted a thematic analysis of their responses to the open-ended questions, and we report on the results in the following section.  


\section{Results}
The most common challenge platforms face is a lack of couriers (17 out of 24), followed by a low volume of orders (10 out of 24), a lack of funding (8 out of 24), a shortage of restaurants (7 out of 24), too many delivery services and not enough customers (5 out of 24), a shortage of customers (4 out of 24), and too many restaurants want to sign up (2 out of 24). One platform faced the challenge of a shortage of employees. No platform indicated they have an oversupply of couriers. 

We found that more than half of the survey respondents, specifically 13 out of 24, have specific suggestions on what should be changed currently regarding policy, while 2 out of 24 platforms believe that the current situation does not need to be changed. Seven replied with ``I'm not sure,''  ``I have no idea'' or similar. Two indie platform operators mentioned that their platforms are on a small scale, and they do not see how  public policy would impact their business. 

\subsection{Skepticism About the Role of Public Policy}
%Our findings suggest that there is a variation in opinions among food delivery platform operators when it comes to public policy related to their platforms. 
%There are controversial ideas about whether the status quo should be altered or maintained. The rationale for 
Several indie platform operators mentioned not wanting policymakers to interfere with their business. These respondents seem to equate policymakers to politicians and believe that policymakers do not understand their needs since they may not know how indie platforms operate. For instance:
\begin{quote}
    ``The best thing they can do to help is stay out of my way. We keep our rates fair and provide a great service. We don't need a politician to `fix' what isn't broken.'' (R4)
\end{quote}
\begin{quote}
    ``Like all businesses, we know our costs and charge what we need to be profitable. Legislators do not know our margins.'' (R18)
\end{quote}
Another viewpoint is that public policy exclusively benefits mainstream platforms, and indie platforms will not be affected due to the small scale they are operating. Respondents stated:
\begin{quote}
    ``Polices only help companies like Doordash or the like.'' (R2)
\end{quote}
\begin{quote}
    ``We aren't big enough to have this affect us.'' (R10)
\end{quote}

\subsection{Transparency, Lower Taxation, and Consent}
On the other hand, some indie platform operators articulated ways public policy could address various problems. We organized their needs into four categories: transparency, tax rates, restaurant consent, and others. 

For transparency, three respondents advocate that the platforms in the food delivery market should disclose information such as their commission rates from restaurants, the compensation they pay for couriers, and price markups on restaurant menus. For instance: 
\begin{quote}
    ``I think the biggest help would create a policy that provides complete transparency of pricing, and commission the service is taking from the restaurants. Not being able to combine the Sales Tax and Service Fee in the same line item.'' (R6)
\end{quote}

For tax rates, three respondents expressed their need for lowering tax rates, especially delivery fee sales tax, due to the current difficulty they have in making a profit: 
\begin{quote}
    ``Stop taxing us on delivery fees as all the little fees we pay makes it extremely hard to make any profit let alone continue to pay for all the expenses to be able to grow and maintain.'' (R24) 
\end{quote}

Another common need we found through the survey is restaurant consent. Preliminary studies have indicated that mainstream platforms list restaurants without their authorization.\cite{nosh2023}. This poses a problem for restaurants but also for indie platforms because they may have negotiated some deal with local restaurants for being exclusively listed on their platform. As pointed out by one of our survey respondents:
\begin{quote}
    ``We have exclusive deals set up with small locally owned restaurants and they constantly pop up on the national platforms. The big companies make it next to impossible for them to get themselves removed as well.'' (R11)
\end{quote}

Other needs we identified that were less commonly mentioned included access to parking, loans, and insurance. One respondent implied the inconvenience of getting parking spots for general delivery services in crowded vehicle areas in LA:
\begin{quote}
    ``Regulation on parking LA County / LA Metro is a crowdy vehicle area and lacks spaces for the park at the time of deliveries or pickups including harassing parking enforcement departments and the local police department, building structures with high fees of parking tickets without validation for RDS.'' (R23)
\end{quote}
Another respondent mentioned how indie platforms could benefit from an easier way to get SBA \footnote{Loans guaranteed by the Small Business Administration, see https://www.sba.gov/funding-programs/loans} loans. R3 suggested opening up regulations on alcohol delivery. 
% and R8 brought up the insurance cost issue for independent contractors. 
\section{Discussion}
%
% Note:
% Summary of the study
% Related studies including Relation to Nadine: leading & following
% Future studies
% Current study is limited because the experimenter as the main author playes the role of the counterpart of the robot. Future study should examine the case of more spontaneous kinaethetic human-robot interactions by employing necessary number of human subjects wherein we may investigate possible mechanism of various social cognition including turn-taking and ...(See my future study plan used in the unit review.)
%
%  Revised by JT 2023-3-6 11am
The current study investigated human-robot bodily interactions via kinaesthesia using a PV-RNN model that was developed based on the free energy principle.
Bodily interactions between a human experimenter and a robot were conducted using Torobo, a humanoid robot equipped with a PV-RNN model that can sense excess torque exerted by a human counterpart.
We especially examined how the counter force between the robot and the human experimenter changed during movement pattern transitions physically guided by the human experimenter, depending on two different condition changes.

In experiment 1, we examined how the setting of a parameter called the meta-prior $w^i$, which regulates the KL-divergence between the approximate posterior distribution and the prior distribution in the interaction phase, affects the counter force generated in executed or attempted transitions.
Results of this experiment showed that in the case of a smaller $w^i$, while the KL-divergence between the approximate posterior and the prior distribution becomes larger, the prediction error (negative log-likelihood) becomes smaller.
Since the prediction error diminishes, the excess torque to counteract this error also decreases.
On the other hand, in the case of a larger $w^i$ setting, while the KL-divergence between the approximate posterior and the prior distribution becomes smaller, the prediction error becomes larger, which requires more excess torque for the transition.
The conflict that appeared between the movement intended by the robot and that executed by the experimenter is distributed to the prediction error and the KL-divergence between the approximate posterior and the prior in proportions determined by the meta-prior $w^i$.
With larger $w^i$ the top-down movement intention of the robot becomes stronger, which results in a stronger counter force, whereas the top-down intention as well as the counter force become weaker with smaller $w^i$.

The above is consistent with past research from our group \cite{wirkuttis2023turn}.
In \cite{wirkuttis2023turn}, Tani conducted simulation studies on synchronized imitative interaction by dyadic, vision-based robots using a PV-RNN model.
That study showed that a robot with smaller/larger $w^i$ tends to follow or lead the other robot set with a larger or smaller $w^i$ with weaker or stronger actional intention in synchronized imitative interaction.
Similarly in the current study, the experimenter easily led
the robot with a smaller $w^i$ with a smaller counter force because of the weaker top-down intention of the robot.
On the other hand, when $w^i$ is quite large, such as $>0.1$ for the robot, it was difficult for the experimenter to lead the robot because of the extremely strong counter force.
In this situation, the experimenter just followed movement patterns strongly led by the robot while grasping the robot's hands, as shown in the preliminary experiment described previously.

In experiment 2, we examined the difference in the counter force required for the experimenter to execute trained and untrained movement pattern transitions.
These experimental results showed that untrained transitions require more force since such transitions are accompanied by larger increases in free energy, the KL-divergence (between the approximate posterior and the prior), and the prediction error.
Trained transitions, on the other hand, require less force because of smaller increases in free energy, the KL-divergence, and the prediction error.

As already mentioned, there have been few studies of human-robot interactions based on the free energy principle.
Although \cite{chameICRA2020} showed that the setting of $w^t$ as meta-prior in the training phase could strongly affect characteristics of human-robot kinaesthetic interactions, the description of their experiment results did not include rigorous analysis with repeated experiments.
%the results of the study are considered to be only preliminary because of insufficient number of repeated experiments and lack of the analysis.
% On the other hand, the current study showed rigorous analysis on the dependence of the counter force on $w^i$ as well as on whether movement transitions are trained or untrained ones through repeated experiments using a real robot.

The major limitation of the present study is that presented human-robot interactions are not fully interactive since experimenter-induced sequences of movement pattern transitions were determined a priori.
In this regard, Ikegami and his colleagues \cite{ikegami2007turn, iizuka2004adaptability} investigated underlying mechanisms for turn-taking that were generated by spontaneous interaction between artificial agents as well as artificial agents and humans.
Recently, Masumori et al. \cite{Masumori2021} developed a humanoid robot platform, called Alter3, behaviour of which was controlled by sub-modules, including a self-simulator, an automatic mimicry unit, and memory storage, which were perturbed by a specific neurodynamic model for the purpose of conducting experiments on spontaneous human-robot interactions.
They showed that spontaneous turn-taking between imitator and imitated could be developed by autonomous switching of information flow between the two sides.

In future studies, we will undertake human-robot kinaesthetic interaction experiments that assume less a priori.
Such experiments should be done not with experimenters as counterparts of the robots (as in the present study) but by inviting an adequate number of human participants, since the human side also needs to be analyzed.
Such studies will focus on two research issues.
One is to investigate how spontaneous turn-taking can occur in imitative interaction based on kinaesthesis by using the active inference framework \cite{friston2010action, parr2019generalised, baltieri2019pid}.
Spontaneous turn-taking in this setting means that the role of the leader to initiate the next shared patterns switches autonomously between the two sides, such that sometimes the robot may push hard with its own intended patterns and the human counterpart may do so at other times.
This study may require development of an autonomous $w^i$ adaptation scheme, since if $w^i$ on the robot side can shift adaptively by sensing contextual flow in the interaction, the leader-follower relationship should shift accordingly.

The other focus is to investigate how novel movement patterns can be developed through repeated kinaesthetic interaction associated with continuous learning in both robots and human participants, based on the free energy principle.
One assumption is that novel patterns could develop in terms of false memory as the number of movement patterns memorized distributively in the PV-RNN model increases.
This phenomenon of false memory is due to potential non-linearity and stochasticity in the PV-RNN model.
A study on a deterministic RNN model demonstrated this property \cite{tani2004self}.
Novel patterns generated by robots could enhance improvisation of new pattern generation from human counterparts through iterative interaction.




\begin{comment}
%% Related studies by Hiroki
%%
Recently, the number of studies on the application of FEP has increased \citep{maselli2022active, pezzato2023active, millidge2022predictive, meo2022adaptation, taniguchi2023world}.
Pezzulo paper on social interaction.
\citet{maselli2022active} showed that the active inference model is able to characterise the movements generated by the agent's intention to resolve multi-sensory conflict or to achieve an external goal such as reaching its arm up to a certain point where the cognition of the agent against its arm is confused by a fake VR-arm.
\citet{baltieri2019pid} showed that PID controllers, which are used to explain biological systems, can fit a more general theory of life and cognition under the free energy principle by using generative models of the world.

\citet{tschantz2020scaling} presented a working implementation of active inference to reinforcement learning which demonstrated and efficient exploration and an order of magnitude different sample efficiency in a high-dimensional tasks such as MountainCar environment.


Also, \citet{pezzato2023active} showed that robotic tasks such as reactive action planning can be formulated as a free energy minimisation problem by introducing a hybrid combination of active inference and behaviour trees.
However, research investigating human-robot interaction using FEP remains sparse \citep{chameICRA2020, ohata2020investigation}.
Although \citet{chameICRA2020} focused on human-robot physical interaction investigating the relationship between cognitive compliance and sensitivity against observation by changing the meta-prior of learning phase $w^t$, its result lacks rigorous analysis and remains preliminary.
Current research is the next level of this, where we conducted a quantitative analysis of the relationship between cognitive compliance and the force exerted.
From our experimental result, one can claim that with a larger meta-prior, the robot performs as an agent with a stronger leading force, which coincides with the result of the leader-follower relationship developed through dyadic robot-robot interaction by \citet{wirkuttis2021leading}.

This research focused on human-robot kinaesthetic interaction, aiming to investigate the nature of interactions in which the intention of the robot and that of the human counterpart does not coincide.
Through human-robot interaction experiments, we investigated the relationship between the required amount of force exerted and the meta-prior.
The result of experiment-1 shows that the amount of force required for the robot to change its top-down intention is proportional to the meta-prior.
Experiment-2, on the other hand, shows that the amount of force required for the robot to perform unseen transitions among movement patterns is larger than that of seen transitions.
However, the current study is limited in the sense that the experimenter as the main author plays the role of the counterpart of the robot. 
Future studies should examine the case of more spontaneous kinaesthetic human-robot interactions by employing the necessary number of human subjects wherein we may investigate the possible mechanism of various social cognitive interactions.
Furthermore, it is interesting to study the development of spontaneous kinaesthetic human-robot collaborative interactions in specific tasks by further extending the PV-RNN framework.
Although various studies on human-robot collaborative interaction have been conducted \citet{sheridan1997eight, ajoudani2018progress, oguz2018hybrid}, the outcomes of these researches struggle to reconstruct a true human-human-like collaboration.
We will especially focus on concepts introduced by \citet{sheridan1997eight} for the human-robot collaborative interaction to be considered successful.
From these points of view, we aim to investigate how the robot can construct a predictive model of the human counterpart through a human-robot kinaesthetic interaction experiment.
\end{comment}







%% Original by Hiroki 2023-3-5
%%
\begin{comment}
The current study investigated how human-robot bodily interaction via kinaesthesia and the meta-prior during human-robot interaction relate.
In particular, we focused on two different perspectives, in Experiment-1, we investigated how the excess torque and meta-prior $w^i$ relate, whereas, in Experiment-2, we investigated how the exess torque and Trained/Untrained transitions (Experiment-2) are related.
In experiment-1, the robot required larger excess torque when configured with a larger meta-prior $w^i$ (see Fig.\ref{fig:TransitionForDifferentW}).
With a larger meta-prior, the top-down actional intention becomes stronger since the posterior distribution strictly follows the prior distribution which is representing the current movement intention.

As shown in the result of experiment-1, the behaviour of the network shows a significant difference between different meta-prior $w^i$ cases. 
When the experimenter attempts to induce a trained transition while Torobo is generating learnt movement patterns, the predicted joint angles and the observed ones start to diverge.
Free energy (which is the loss function of PV-RNN) begins rising at the same time, since it is expressed as a sum of the mean squared error between the predicted joint angles and the observed ones, and KL divergence between the approximate posterior and the prior (see Eq.\ref{eq:LossFunction_interaction}).
To minimise free energy, PV-RNN can either update the approximate posterior and adjust the predicted joint angle to minimise the mean squared error or keep the approximate posterior following the prior distribution to minimise the KL divergence.
The meta-prior $w^i$ is regulating the KL divergence, which balances between these two terms, the \emph{accuracy} and the \emph{complexity} when minimising free energy.
With the case of $w^i$ set to a smaller value, one can clearly see that the network immediately minimises the mean squared error by updating the approximate posterior which leads to an increase of KL-divergence (see Fig.\ref{fig:TimeDevelopmentDifferentW}a, $t_{c} = 404$ and $407$).
Whereas with the case of $w^i$ set to a larger value, the network keeps minimising the KL divergence by keeping the approximate posterior following the prior generation until the mean squared error becomes large enough (see Fig.\ref{fig:TimeDevelopmentDifferentW}c, $t_{c} = 1425$ and $1432$).

More interestingly, the required excess torque became larger when the meta-prior $w^i$ was set to a larger value (see Fig.\ref{fig:TransitionForDifferentW}a). 
When $w^i$ is set to a larger value, the error between the predicted joint angles and the observed ones becomes higher than that of the smaller $w^i$ setting.
This larger error is fed to the PID controller which results in the generation of a larger counter-force to the experimenter.
From the result, we show the relationship between the required force and the meta-prior $w^i$ that the larger $w^i$ leads the network to require a larger force when the interacting counterpart performs against the top-down intention of it.

In experiment-2, the robot required a larger excess torque while performing untrained movement pattern transitions compared to trained ones (see Fig.\ref{fig:TrainedVSUntrained}).
Also, the network showed a higher increase in both the mean squared error and KL divergence for untrained movement pattern transitions.
This indicates that the robot shows a higher resistance against an unseen situation where its free energy increases more than in a familiar situation.
% in a situation that it has encountered.
\end{comment}
%%%%%%%%%%%
%%%%%%%%%%%




\begin{comment}
------------------------Draft below this---------------------------

During the interaction phase of PV-RNN model, adaptive variables $\mathbf{A}_t^{\mu}, \mathbf{A}_t^{\sigma}$ are updated to minimise the free energy in Eq.\ref{eq:LossFunction}, which is the sum of mean squared error and KL divergence.
When the output $\Bar{\theta}$ and the observation $\theta$ begin to diverge due to external forces such as the experimenter holding Torobo's hands, the mean squared error increases, which leads to the rise of free energy.
PV-RNN model is capable of developing its output $\Bar{\theta}$ by updating adaptive variables $\mathbf{A}_t^{\mu}, \mathbf{A}_t^{\sigma}$.
However, updating the adaptive variables will lead to an increase in the KL divergence between the approximate posterior and prior, which also leads to an increase in free energy.
Hence, free energy minimisation in the interaction phase of the PV-RNN model during human-robot interaction is a balance of these two terms, which is regulated by the meta-prior $w^i$ (Eq.\ref{eq:LossFunction}).

\subsection{$w^i$ and the excess torque}
\end{comment}
\section{Discussion and Limitations}

Although we can ablate concepts efficiently for a wide range of object instances, styles, and memorized images, our method is still limited in several ways. First, while our method overwrites a target concept, this does not guarantee that the target concept cannot be generated through a different, distant text prompt. We show an example in \reffig{limitation} (a), where after ablating {\menlo Van Gogh}, the model can still generate {\menlo starry night painting}. However, upon discovery, one can resolve this by explicitly ablating the target concept {\menlo starry night painting}. Secondly, when ablating a target concept, we still sometimes observe slight degradation in its surrounding concepts, as shown in \reffig{limitation} (c). 

\nupur{Our method does not prevent a downstream user with full access to model weights from re-introducing the ablated concept~\cite{ruiz2022dreambooth,kumari2022multi,gal2022image}. Even without access to the model weights, one may be able to iteratively optimize for a text prompt with a particular target concept. Though that may be much more difficult than optimizing the model weights, our work does not guarantee that this is impossible.}

Nevertheless, we believe every creator should have an ``opt-out'' capability. We take a small step towards this goal, creating a computational tool to remove copyrighted images and artworks from large-scale image generative models.








%%
%% The next two lines define the bibliography style to be used, and
%% the bibliography file.
\bibliographystyle{ACM-Reference-Format}
\bibliography{bib}

%%
%% If your work has an appendix, this is the place to put it.
\appendix



\end{document}
\endinput
%%
%% End of file `sample-acmlarge.tex'.
