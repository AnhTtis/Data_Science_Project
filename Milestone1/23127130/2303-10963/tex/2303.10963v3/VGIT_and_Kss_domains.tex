\documentclass[12pt,reqno]{amsart}
\usepackage{amsmath,amssymb,geometry,color,tikz-cd}
%\usepackage {showkeys}
%\usepackage[textures,backref,bookmarks=false]{hyperref}
\usepackage[backref,pagebackref,pdftex,hyperindex]{hyperref}
%\usepackage[pdftex,hyperindex]{hyperref}
%\usepackage[pdftex]{graphicx}
%\usepackage[pdftex]{hyperref}
%\geometry{centering,vcentering,marginratio=4:3,vscale=0.68,hscale=0.65}
\geometry{centering,vcentering,marginratio=1:1,vscale=0.72,hscale=0.72}
%\usepackage{geometry}
%\usepackage{hyperref}
\usepackage[alphabetic]{amsrefs}
\usepackage{amssymb}% see geometry.pdf on how to lay out the page. There's lots.
%\geometry{a4paper} % or letter or a5paper or ... etc
% \geometry{landscape} % rotated page geometry
% See the ``Article customise'' template for come common customisations
\usepackage{bbm}
\usepackage{enumitem}
\usepackage{upgreek}

\newtheorem{proposition}{Proposition}[section]
\newtheorem{theorem}[proposition]{Theorem}
\newtheorem{lemma}[proposition]{Lemma}
\newtheorem{corollary}[proposition]{Corollary}
\newtheorem{conjecture}[proposition]{Conjecture}

\theoremstyle{definition}
\newtheorem{remark}[proposition]{Remark}
\newtheorem{definition}[proposition]{Definition}
\newtheorem{example}[proposition]{Example}
\newtheorem{question}[proposition]{Question}
\newtheorem{claim}[proposition]{Claim}

\title{Chamber decomposition for K-semistable domains and VGIT}
\author{Chuyu Zhou}
\address{\'Ecole Polytechnique F\'ed\'erale de Lausanne (EPFL), MA C3 615, Station 8, 1015 Lausanne, Switzerland}
\email{chuyu.zhou@epfl.ch}
\date{} % delete this line to display the current date
%\dedicatory{Dedicate to my daughter Jiangyue Zhou}

\thanks{2010 
	    \emph{Mathematics Subject Classification}: 14J45.
	    \newline
	    \indent 
		\emph{Keywords}: Log Fano pair, K-stability, K-semistable domain, K-moduli, VGIT.
        \newline
		\indent
		\emph{Competing interests}: The author declares none.
		}










\newcommand{\Fut}{{\rm{Fut}}}
\newcommand{\CH}{{\rm{CH}}}
\newcommand{\ord}{{\rm {ord}}}
\newcommand{\tc}{{\rm {tc}}}
\newcommand{\vol}{{\rm {vol}}}
\newcommand{\lct}{{\rm {lct}}}
\newcommand{\mult}{{\rm {mult}}}
\newcommand{\length}{{\rm {length}}}
\newcommand{\red}{{\rm {red}}}
\newcommand{\Spec}{{\rm {Spec}}}
\newcommand{\Proj}{{\rm{Proj}}}
\newcommand{\Exc}{{\rm {Exc}}}
\newcommand{\Val}{{\rm {Val}}}
\newcommand{\wt}{{\rm {wt}}}
\newcommand{\gr}{{\rm {gr}}}
\newcommand{\Bl}{{\rm {Bl}}}
\newcommand{\Pic}{{\rm {Pic}}}
\newcommand{\Cl}{{\rm {Cl}}}
\newcommand{\dt}{{\rm {dt}}}
\newcommand{\orb}{{\rm {orb}}}
\newcommand{\pr}{{\rm {pr}}}
\newcommand{\Supp}{{\rm {Supp}}}
\newcommand{\lc}{{\rm {lc}}}
\newcommand{\ac}{{\rm {ac}}}
\newcommand{\s}{{\rm {s}}}
\newcommand{\SL}{{\rm {SL}}}
\newcommand{\Hilb}{{\rm {Hilb}}}
\newcommand{\Chow}{{\rm {Chow}}}
\newcommand{\PGL}{{\rm {PGL}}}
\newcommand{\GL}{{\rm {GL}}}
\newcommand{\Aut}{{\rm {Aut}}}
\newcommand{\KSBA}{{\rm {KSBA}}}
\newcommand{\CY}{{\rm {CY}}}
\newcommand{\Hodge}{{\rm {Hodge}}}
\newcommand{\CM}{{\rm {CM}}}
\newcommand{\Mob}{{\rm {Mob}}}
\newcommand{\Fix}{{\rm {Fix}}}
\newcommand{\Bs}{{\rm {Bs}}}
\newcommand{\Red}{{\rm {Red}}}
\newcommand{\Ding}{{\rm {Ding}}}
%\newcommand{\min}{{\rm {min}}}
\newcommand{\can}{{\rm {can}}}
\newcommand{\Kss}{{\rm {Kss}}}
\newcommand{\Cone}{{\rm {Cone}}}
\newcommand{\LC}{{\rm {LC}}}






\newcommand{\bA}{\mathbb{A}}
\newcommand{\bB}{\mathbb{B}}
\newcommand{\bC}{\mathbb{C}}
\newcommand{\bD}{\mathbb{D}}
\newcommand{\bE}{\mathbb{E}}
\newcommand{\bF}{\mathbb{F}}
\newcommand{\bG}{\mathbb{G}}
\newcommand{\bH}{\mathbb{H}}
\newcommand{\bI}{\mathbb{I}}
\newcommand{\bJ}{\mathbb{J}}
\newcommand{\bK}{\mathbb{K}}
\newcommand{\bL}{\mathbb{L}}
\newcommand{\bM}{\mathbb{M}}
\newcommand{\bN}{\mathbb{N}}
\newcommand{\bO}{\mathbb{O}}
\newcommand{\bP}{\mathbb{P}}
\newcommand{\bQ}{\mathbb{Q}}
\newcommand{\bR}{\mathbb{R}}
\newcommand{\bS}{\mathbb{S}}
\newcommand{\bT}{\mathbb{T}}
\newcommand{\bU}{\mathbb{U}}
\newcommand{\bV}{\mathbb{V}}
\newcommand{\bW}{\mathbb{W}}
\newcommand{\bX}{\mathbb{X}}
\newcommand{\bY}{\mathbb{Y}}
\newcommand{\bZ}{\mathbb{Z}}



\newcommand{\mA}{\mathcal{A}}
\newcommand{\mB}{\mathcal{B}}
\newcommand{\mC}{\mathcal{C}}
\newcommand{\mD}{\mathcal{D}}
\newcommand{\mE}{\mathcal{E}}
\newcommand{\mF}{\mathcal{F}}
\newcommand{\mG}{\mathcal{G}}
\newcommand{\mH}{\mathcal{H}}
\newcommand{\mI}{\mathcal{I}}
\newcommand{\mJ}{\mathcal{J}}
\newcommand{\mK}{\mathcal{K}}
\newcommand{\mL}{\mathcal{L}}
\newcommand{\mM}{\mathcal{M}}
\newcommand{\mN}{\mathcal{N}}
\newcommand{\mO}{\mathcal{O}}
\newcommand{\mP}{\mathcal{P}}
\newcommand{\mQ}{\mathcal{Q}}
\newcommand{\mR}{\mathcal{R}}
\newcommand{\mS}{\mathcal{S}}
\newcommand{\mT}{\mathcal{T}}
\newcommand{\mU}{\mathcal{U}}
\newcommand{\mV}{\mathcal{V}}
\newcommand{\mW}{\mathcal{W}}
\newcommand{\mX}{\mathcal{X}}
\newcommand{\mY}{\mathcal{Y}}
\newcommand{\mZ}{\mathcal{Z}}


\newcommand{\fA}{\mathbf{A}}
\newcommand{\fB}{\mathbf{B}}
\newcommand{\fC}{\mathbf{C}}
\newcommand{\fD}{\mathbf{D}}
\newcommand{\fE}{\mathbf{E}}
\newcommand{\fF}{\mathbf{F}}
\newcommand{\fG}{\mathbf{G}}
\newcommand{\fH}{\mathbf{H}}
\newcommand{\fI}{\mathbf{I}}
\newcommand{\fJ}{\mathbf{J}}
\newcommand{\fK}{\mathbf{K}}
\newcommand{\fL}{\mathbf{L}}
\newcommand{\fM}{\mathbf{M}}
\newcommand{\fN}{\mathbf{N}}
\newcommand{\fO}{\mathbf{O}}
\newcommand{\fP}{\mathbf{P}}
\newcommand{\fQ}{\mathbf{Q}}
\newcommand{\fR}{\mathbf{R}}
\newcommand{\fS}{\mathbf{S}}
\newcommand{\fT}{\mathbf{T}}
\newcommand{\fU}{\mathbf{U}}
\newcommand{\fV}{\mathbf{V}}
\newcommand{\fW}{\mathbf{W}}
\newcommand{\fX}{\mathbf{X}}
\newcommand{\fY}{\mathbf{Y}}
\newcommand{\fZ}{\mathbf{Z}}


\newcommand{\tA}{\tilde{A}}
\newcommand{\tB}{\tilde{B}}
\newcommand{\tC}{\tilde{C}}
\newcommand{\tD}{\tilde{D}}
\newcommand{\tE}{\tilde{E}}
\newcommand{\tF}{\tilde{F}}
\newcommand{\tG}{\tilde{G}}
\newcommand{\tH}{\tilde{H}}
\newcommand{\tI}{\tilde{I}}
\newcommand{\tJ}{\tilde{J}}
\newcommand{\tK}{\tilde{K}}
\newcommand{\tL}{\tilde{L}}
\newcommand{\tM}{\tilde{M}}
\newcommand{\tN}{\tilde{N}}
\newcommand{\tO}{\tilde{O}}
\newcommand{\tP}{\tilde{P}}
\newcommand{\tQ}{\tilde{Q}}
\newcommand{\tR}{\tilde{R}}
\newcommand{\tS}{\tilde{S}}
\newcommand{\tT}{\tilde{T}}
\newcommand{\tU}{\tilde{U}}
\newcommand{\tV}{\tilde{V}}
\newcommand{\tW}{\tilde{W}}
\newcommand{\tX}{\tilde{X}}
\newcommand{\tY}{\tilde{Y}}
\newcommand{\tZ}{\tilde{Z}}





\newcommand{\ft}{\mathbf{t}}

\newcommand{\ka}{\mathfrak{a}}
\newcommand{\kb}{\mathfrak{b}}
\newcommand{\kc}{\mathfrak{c}}
\newcommand{\kd}{\mathfrak{d}}
\newcommand{\ke}{\mathfrak{e}}
\newcommand{\kf}{\mathfrak{f}}
\newcommand{\kg}{\mathfrak{g}}
\newcommand{\kh}{\mathfrak{h}}
\newcommand{\kp}{\mathfrak{p}}
\newcommand{\kq}{\mathfrak{q}}
\newcommand{\km}{\mathfrak{m}}


\newcommand{\kA}{\mathfrak{A}}
\newcommand{\kB}{\mathfrak{B}}
\newcommand{\kI}{\mathfrak{I}}
\newcommand{\kJ}{\mathfrak{J}}
\newcommand{\kX}{\mathfrak{X}}
\newcommand{\kY}{\mathfrak{Y}}
\newcommand{\kZ}{\mathfrak{Z}}
\newcommand{\kR}{\mathfrak{R}}






\begin{document}


\begin{abstract}
In this paper, we explore the relationship between the chamber decomposition for K-semistable domains and VGIT. We also study the relationship between the K-moduli generically parametrizing K-semistable Fano complete intersections of the form $S_{d_1}\cap...\cap S_{d_k}$ and the K-moduli generically parametrizing K-semistable log Fano manifolds of the form $(\mathbb{P}^n, \sum_{j=1}^kx_jS_{d_j})$, where $x_j\in (0,1)\cap \mathbb{Q}$ and $S_{d_j}\subset \mathbb{P}^n$ is a hypersurface of degree $d_j$ for each $1\leq j\leq k$.
\end{abstract}

\maketitle
\tableofcontents

\section{Introduction}


Based on a rather complete algebraic K-stability theory developed in the past few years (e.g. \cite{Jiang20, BLX22, Xu20, ABHLX20, BX19, XZ20b, LXZ22}),  people are now able to  construct a projective separated scheme as a good moduli space to parametrize Fano varieties with K-stability. In particular, a wall crossing theory for K-moduli spaces is also established in \cite{ADL19} (see also \cite{Zhou23}), which turns out to be useful in the study of birational geometry of different moduli spaces of varieties (e.g. \cite{ADL20, ADL21}). 
In the wall crossing picture, it is widely recognized that the first K-moduli space is related to the GIT moduli space (e.g. \cite{ADL19, GMGS21, Zhou21a}). For example, fixing a K-polystable Fano variety $X$ and a sufficiently divisible positive integer $l$, then there is a natural $\Aut(X)$-action on the linear system $\frac{1}{l}|-lK_X|$ (note that $\Aut(X)$ is reductive due to \cite{ABHLX20}). We conclude that $D\in \frac{1}{l}|-lK_X|$ is GIT-(semi/poly)stable if and only if $(X, \epsilon D)$ is K-(semi/poly)stable for $0<\epsilon\ll 1$ (e.g. \cite{Zhou21a}). Recently, a wall crossing theory for K-moduli with multiple boundaries is established in \cite{Zhou23b}, thus it is also natural to figure out the similar relationship (as in the above example) in the VGIT picture.

We fix a K-polystable Fano variety $X$ and $k$ sufficiently divisible positive integers $l_1,...,l_k$. Denote $\bP_j:=\frac{1}{l_j}|-l_jK_X|,\ j=1,...,k$. Then there is a natural $\Aut(X)$-action on $\bP_1\times...\times \bP_k$ and one could consider the VGIT under the linearization $\mO(c_1,...,c_k)$ as $c_j$ varies in $(0,1]\cap \bQ$ (e.g. \cite{Tha96}). On the other hand, by \cite{Zhou23b}, there is a finite chamber decomposition of 
$$\Delta^k:=\{(x_1,...,x_k)\ |\ \textit{$x_j\in [0,1)\cap \bQ$ {\rm{and}} $\sum_{j=1}^kx_j<1$}\}$$
such that for any $(D_1,...,D_k)\in \bP_1\times...\times \bP_k$, the K-semistability of $(X, \sum_{j=1}^kc_jD_j)$ does not change as $(c_1,...,c_k)$ varies in the interior domain of each chamber. We have the following result which reveals the relationship between the VGIT and the chamber decomposition of $\Delta^k$.

\begin{theorem}\label{thm: main1}
For any $(c_1,...,c_k)\in (\Delta^k)^\circ$, $(D_1,...,D_k)\in \bP_1\times...\times \bP_k$ 
is GIT-(semi/poly)stable under $\Aut(X)$-action with respect to the linearization $\mO(\frac{c_1}{l_1},...,\frac{c_k}{l_k})$ if and only if $(X, \epsilon\sum_{j=1}^kc_jD_k)$ is K-(semi/poly)stable for $0<\epsilon\ll 1$. Moreover, the chamber decomposition of $\Delta^k$ at the original point exactly gives the chamber decomposition for VGIT.
\end{theorem}



We next turn to a concrete setting. Fix a positive integer $n$ and $k$ positive integers $\vec{d}:=(d_1,...,d_k)$. For general smooth hypersurfaces $S_{d_j}$ of degrees $d_j$, if $\sum_{j=1}^k d_j< n+1$, then the complete intersection $\cap_{j=1}^kS_{d_j}$ gives a Fano manifold of dimension $n-k$. We denote $\mM^K_{n, \vec{d}}$ (resp. $M^K_{n, \vec{d}}$) to be the K-moduli stack (resp. K-moduli space) generically parametrizing such Fano manifolds which are K-semistable. Let $\mM^K_{n, \vec{d}, \vec{x}}$ (resp. $M^K_{n, \vec{d}, \vec{x}}$) be the K-moduli stack (resp. K-moduli space) generically parametrizing K-semistable log Fano manifolds of the form  $(\bP^n, \sum_{j=1}^kx_jS_{d_j})$, where $\vec{x}:=(x_1,...,x_k)$. Then it is natural to ask what is the relationship
between $\mM^K_{n, \vec{d}}$ and $\mM^K_{n, \vec{d}, \vec{x}}$. First recall the following result.

\begin{theorem}{\rm{(\cite[Theorem 8.1]{Zhou23a})}}
Consider the log pair $(\bP^n, S_{d_1}+S_{d_2}+...+S_{d_k})$, where $S_{d_j}, j=1,...,k,$ are general smooth hypersurfaces in $\bP^n$ of degrees $d_j$ with $n\geq 2$ and $d_j\leq n+1$. Suppose all the Fano complete intersections are K-semistable. Then $\Kss(\bP^n, S_1+...+S_k)$ is a polytope generated by the following equations
\begin{equation*}
\begin{cases}
0\leq x_i\leq 1, \ \ \  1\leq i\leq k\\

\beta_{\bP^n, \sum_{j=1}^kx_jS_{d_j}}(S_{d_i})\geq 0, \ \ \  1\leq i\leq k\\

\sum_{j=1}^k x_jd_j\leq n+1.
\end{cases}
\end{equation*}
\end{theorem}

By the proof of \cite[Theorem 8.1]{Zhou23a}, if $\sum_{j=1}^kd_j< n+1$, the equations 
$$\beta_{\bP^n, \sum_{j=1}^kx_jS_{d_j}}(S_{d_i})= 0, \ \ \ 1\leq i\leq k$$ 
admit a unique solution, denoted by $\vec{a}:=(a_1,...,a_k)$, where
$$a_j=\frac{\sum_{i=1}^kd_i+(n-k+1)d_j-(n+1)}{(n-k+1)d_j}, $$
and $\vec{a}$ is automatically an extremal point of $\Kss(\bP^n, \sum_{j=1}^kS_{d_j})$. This point is definitely different. We have the following result.

\begin{theorem}\label{thm: main2}
For a log smooth pair $(\bP^n, \sum_{j=1}^kS_{d_j})$, where $S_{d_j}$ are hypersurfaces of degrees $d_j$ 
%{\rm{(}}not necessarily in general position{\rm{)}}   
with $\sum_{j=1}^kd_j<n+1$, the log Fano pair $(\bP^n, \sum_{j=1}^ka_jS_{d_j})$ is K-semistable if and only if the complete intersection $\cap_{j=1}^{k}S_{d_j}$ is  K-semistable of dimension $n-k$. Moreover, there exists a morphism 
$$\mM^{K}_{n, \vec{d}, \vec{a}}\to \mM^K_{n, \vec{d}}$$ 
which descends to a surjective morphism $M^{K}_{n, \vec{d}, \vec{a}}\to M^K_{n, \vec{d}}.$
\end{theorem}



\noindent
\subsection*{Acknowledgement}
The author is supported by the grant of European Research Council (ERC-804334).










\section{Preliminaries}

We say that $(X,\Delta)$ is a \emph{log pair} if $X$ is a normal projective variety and $\Delta$ is an effective $\bQ$-divisor on $X$ such that $K_X+\Delta$ is $\bQ$-Cartier.  The log pair $(X,\Delta)$ is called \emph{log Fano} if it admits klt singularities and $-(K_X+\Delta)$ is ample; if $\Delta=0$, we just say $X$ is a \emph{Fano variety}. 
For various types of singularities in birational geometry, e.g.  klt, lc, and plt singularities, we refer to \cite{KM98,Kollar13}.


\subsection{K-stability}

Let $(X,\Delta)$ be a log pair. Suppose $f\colon Y\to X$ is a proper birational morphism between normal varieties and $E$ is a prime divisor on $Y$, we say that $E$ is a prime divisor over $X$ and define the following invariant
$$A_{X,\Delta}(E):=1+\ord_E(K_Y-f^*(K_X+\Delta)). $$
It is called the \emph{log discrepancy} of $E$ associated to the log pair $(X,\Delta)$.
If $(X,\Delta)$ is a log Fano pair, we define the following invariant
$$S_{X,\Delta}(E):=\frac{1}{\vol(-K_X-\Delta)}\int_0^\infty \vol(-f^*(K_X+\Delta)-tE){\rm{d}}t .$$
Put $\beta_{X,\Delta}(E):=A_{X,\Delta}(E)-S_{X,\Delta}(E)$. By the works \cite{Fuj19, Li17}, one can define K-stability of a log Fano pair by beta criterion as follows.
\begin{definition}\label{def: kss}
Let $(X,\Delta)$ be a log Fano pair. 
We say that $(X,\Delta)$ is \emph{K-semistable} if $\beta_{X,\Delta}(E)\geq 0$ for any prime divisor $E$ over $X$.
\end{definition}

The following lemma on K-stability of projective cones is well known, see e.g. \cite[Prop 2.11]{LZ22}.
\begin{lemma}\label{lem:cone stability}
Let $(V,\Delta)$ be an n-dimensional log Fano pair, and L an ample line bundle on V such that $L\sim_\bQ -\frac{1}{r}(K_V+\Delta)$ for some $0<r\leq n+1$. Suppose Y is the projective cone over V associated to L with infinite divisor $V_\infty$, then $(V,\Delta)$ is K-semistable if and only if $(Y,\Delta_Y+(1-\frac{r}{n+1})V_\infty)$ is K-semistable, where $\Delta_Y$ is the divisor on Y naturally extended by $\Delta$.
\end{lemma}


\subsection{Test configuration}



\begin{definition}\label{def: tc}
Let $(X,\Delta)$ be a log pair and $L$ an ample $\bQ$-line bundle on $X$. A test configuration $\pi: (\mX,\Delta_\tc;\mL)\to \bA^1$ is a degenerating family over $\bA^1$ consisting of the following data:
\begin{enumerate}
\item $\pi: \mX\to \bA^1$ is a projective flat morphism from a normal variety $\mX$, $\Delta_\tc$ is an effective $\bQ$-divisor on $\mX$, and $\mL$ is a relatively ample $\bQ$-line bundle on $\mX$,
\item the family $\pi$ admits a $\bC^*$-action which lifts the natural $\bC^*$-action on $\bA^1$ such that $(\mX,\Delta_\tc; \mL)\times_{\bA^1}\bC^*$ is $\bC^*$-equivariantly isomorphic to $(X, \Delta; L)\times_{\bA^1}\bC^*$.
\end{enumerate}
\end{definition}

Suppose $(X,\Delta)$ is a log Fano pair and $L=-K_X-\Delta$. Let  $(\mX,\Delta_\tc; \mL)$ be a test configuration such that $\mL=-K_{\mX/\bA^1}-\Delta_\tc$. We call it a special test configuration if  $(\mX, \mX_0+\Delta_{\tc})$ admits plt singularities (or equivalently, the central fiber $(\mX_0, \Delta_{\tc,0})$ is a log Fano pair). 


\begin{remark}\label{rem: kss degeneration}
Let $(X, \Delta)$ be a K-semistable log Fano pair. Suppose $S$ is a prime divisor on $X$ such that $\beta_{X, \Delta}(S)=0$, then $S$ induces a special test configuration with vanishing generalized Futaki invariant (e.g. \cite{LX14, Fuj19}). By \cite{LWX21} or \cite{BLZ22}, we know that the central fiber of the test configuration is also K-semistable. Assume further that $S$ is Cartier and $S\sim_\bQ -\lambda(K_X+\Delta)$ for some rational $0<\lambda<1$, then it is not hard to see that 
the central fiber of the test configuration is isomorphic to the projective cone over $S$ with respect to the polarization $S|_S$. Actually, first note that 
$$X=\Proj \bigoplus_{m\in \bN} H^0(X, mS). $$
By \cite[Section 2.3.1]{BX19},  the special test configuration induced by $S$ could be formulated as
$$\mX:=\Proj \bigoplus_{m\in \bN}\bigoplus_{i\in \bZ} H^0(X, mS-iS)t^{-i}, $$
and the central fiber is
$$\mX_0=\Proj \bigoplus_{m\in \bN}\bigoplus_{i\in \bZ} \left(H^0(X, mS-iS)/H^0(X, mS-(i+1)S)\right) .$$
Note that for $i\leq m$ we have
$$H^0(X, mS-iS)/H^0(X, mS-(i+1)S)\cong  H^0(S, (m-i)S|_S).$$
Thus $\mX_0\cong \Proj \bigoplus_{m\in \bN}\bigoplus_{i\in \bN}H^0(S, mS|_S)s^i$, which is the projective cone over $S$ with respect to $S|_S$.
\end{remark}







\section{Chow-Mumford line bundle}


Let $\pi: (\mX,\mD;\mL)\to T$ be a flat family of projective normal varieties of dimension $n$ over a normal base $T$, where $\mD$ is an effective $\bQ$-divisor on $\mX$ whose components are all flat over $T$, and $\mL$ is a relative ample $\bQ$-line bundle on $\mX$. By the work of Mumford-Knudsen(\cite{KM76}), there exist $\bQ$-line bundles $\lambda_i,i=0,1,...,n+1$ and $\tilde{\lambda}_i,i=0,1,...n,$ on $T$ such that we have the following expansions for all sufficiently large $k\in \bN$:
$$\det \pi_*(\mL^k)= \lambda_{n+1}^{\binom{k}{n+1}}\otimes\lambda_n^{\binom{k}{n}}\otimes...\otimes\lambda_1^{\binom{k}{1}}\otimes\lambda_0,$$
$$\det \pi_*(\mL|_\mD^k)= \tilde{\lambda}_{n}^{\binom{k}{n}}\otimes\tilde{\lambda}_{n-1}^{\binom{k}{n-1}}\otimes...\otimes\tilde{\lambda}_0.$$
By Riemann-Roch formula, cf \cite[Appendix]{CP21}, we have
$$c_1(\pi_*\mL^k)=\frac{\pi_*(\mL^{n+1})}{(n+1)!}k^{n+1}+\frac{\pi_*(-K_{\mX/T}\mL^n)}{2n!}k^n+... ,$$
$$c_1({\pi}_*{\mL}|_\mD^k)=\frac{{\pi}_*(\mL^n\mD)}{n!}k^n+... .$$
Then it's not hard to see
$$\lambda_{n+1}=\pi_*(\mL^{n+1}),\  \lambda_n=\frac{n}{2}\pi_*(\mL^{n+1})+\frac{1}{2}\pi_*(-K_{\mX/T}\mL^n),\  
\tilde{\lambda}_n=\pi_*(\mL^n\mD). $$
By the flatness of $\pi$ and $\pi_\mD$, we write
$$h^0(\mX_t, k\mL_t)=a_0k^n+a_1k^{n-1}+o(k^{n-1})\quad \text{and} \quad  h^0(\mD_t, k{\mL_t}|_{\mD_t})=\tilde{a}_0k^{n-1}+o(k^{n-1}),$$
which do not depend on the choice of $t\in T$. Then we have
$$a_0=\frac{\mL_t^n}{n!},\  a_1=\frac{-K_{\mX_t}{\mL_t}^{n-1}}{2(n-1)!},\  \tilde{a}_0=\frac{\mL_t^{n-1}\mD_t}{(n-1)!}.$$

\begin{definition}\label{def: CM}
We define the CM-line bundles for the family $\pi: (\mX,\mD;\mL)\to T$ as follows:
$$\lambda_{\CM}(\mX,\mL;\pi):=\lambda_{n+1}^{\frac{2a_1}{a_0}+n(n+1)}\otimes\lambda_n^{-2(n+1)}, $$
$$\lambda_{\CM}(\mX,\mD,\mL;\pi):= \lambda_{n+1}^{\frac{2a_1-\tilde{a}_0}{a_0}+n(n+1)}\otimes\lambda_n^{-2(n+1)}\otimes\tilde{\lambda}_n^{n+1}.$$
For a rational number $0\leq \beta\leq 1$, we define
$$\lambda_{\CM, \beta}(\mX,\mD,\mL;\pi):= \lambda_{n+1}^{\frac{2a_1-\beta\tilde{a}_0}{a_0}+n(n+1)}\otimes\lambda_n^{-2(n+1)}\otimes\tilde{\lambda}_n^{\beta(n+1)}.$$
\end{definition}

\begin{remark}\label{rem: futaki}
If $\pi: (\mX, \mD; \mL)\to T$ is a test configuration of the general fiber $(\mX_t, \mD_t; \mL_t)$, then it is well known that the generalized Futaki invariant $\Fut(\mX, \beta\mD; \mL)$ is equal to $w(\lambda_{\CM, \beta}(\mX,\mD,\mL;\pi))$ up to a positive factor, where $w(\lambda_{\CM, \beta}(\mX,\mD,\mL;\pi))$ is the total weight of the $\bC^*$-action on $\lambda_{\CM, \beta}(\mX,\mD,\mL;\pi)$ (e.g. \cite[Theorem 2.6]{GMGS21}).
\end{remark}

\begin{lemma}\label{lem: proportional cm}
Let $\pi: (\mX, \mD; \mL)\to T$ be a family satisfying:
\begin{enumerate}
\item $\mX\to T$ is a flat family of projective normal varieties of dimension $n$ over a normal base $T$ such that $-K_{\mX/T}$ is $\bQ$-Cartier and relative ample over $T$; 
\item $\mD$ is an effective $\bQ$-divisor on $\mX$ such that every component is flat over $T$ and $\mD\sim_{\bQ, T}-\mu K_{\mX/T}$ for some rational $\mu>0$;
\item $\mL=- K_{\mX/T}$.
\end{enumerate}
\end{lemma}
Then we have the following formula
$$\lambda_{\CM,\beta}(\mX, \mD,\mL;\pi)=-(1+\beta \mu n)\pi_*\mL^{n+1}+\beta (n+1) \pi_*(\mL^n\mD).$$

\begin{proof}
Apply Definition \ref{def: CM}.
\end{proof}


\section{Chamber decomposition for VGIT}

In this section, we fix a K-polystable Fano variety $X$ of dimension $n$ with $(-K_X)^n=v$ and $k$ sufficiently divisible positive integers $l_1,...,l_k$. Put 
$$\bP_j:=\frac{1}{l_j}|-l_jK_X|,\ j=1,...,k.$$
Let $\mD_j\subset X\times \bP_j$ be the universal divisor with respect to the linear system $\frac{1}{l_j}|-l_jK_X|$. Denote 
$$T:=\bP_1\times ... \times \bP_k\quad \text{and} \quad\mB_j:=\bP_1\times...\times\bP_{j-1}\times \mD_j\times \bP_{j+1}\times...\times\bP_k.$$
For a given rational point $\vec{c}:=(c_1,...,c_k)\in (\Delta^k)^\circ$, we  aim to compute the Chow-Mumford line bundle 
$$\lambda_{\CM, \beta}(\vec{c}):=\lambda_{\CM,\beta}(X\times T, \sum_{j=1}^k c_j\mB_j, -K_{X\times T/T}; \pi_{\vec{c}})$$ 
for the family 
$\pi_{\vec{c}}: (X\times T, \sum_{j=1}^k c_j\mB_j)\to T$
for rational $0\leq \beta\leq 1$. 


\begin{lemma}\label{lem: compute cm}
Notation as above, for a rational point $\vec{c}:=(c_1,...,c_k)\in (\Delta^k)^\circ$, we have the following formula:
$$\lambda_{\CM, \beta}(\vec{c})= \mO(\frac{\beta (n+1)vc_1}{l_1},..., \frac{\beta(n+1)vc_k}{l_k}).$$
\end{lemma}

\begin{proof}
By Lemma \ref{lem: proportional cm}, we have
$$\lambda_{\CM, \beta}(\vec{c})=\sum_{j=1}^k (n+1)\beta c_j {\pi_{\vec{c}}}_*((-K_{X\times T/T})^n\mB_j). $$
It suffices to compute ${\pi_{\vec{c}}}_*((-K_{X\times T/T})^n\mB_j)$. Recall that 
$$\mB_j:=\bP_1\times...\times\bP_{j-1}\times \mD_j\times \bP_{j+1}\times...\times\bP_k,$$ 
and 
$${\pi_j}_*((-K_{X\times \bP_j/\bP_j})^n\mD_j)=\mO_{\bP_j}(\frac{v}{l_j})$$ 
for $\pi_{j}: X\times \bP_j\to \bP_j$. Thus we have
$${\pi_{\vec{c}}}_*((-K_{X\times T/T})^n\mB_j)=\mO(0_1,0_2...0_{j-1}, \frac{v}{l_j},0_{j+1},...,0_k).$$
The proof is finished.
\end{proof}

It is not hard to see that $\lambda_{\CM, \beta}(\vec{c})$ is proportional to $\mO(\frac{c_1}{l_1},...,\frac{c_k}{l_k})$.


\begin{proposition}\label{prop: GIT=K}
For any $(c_1,...,c_k)\in (\Delta^k)^\circ$, $(D_1,...,D_k)\in \bP_1\times...\times \bP_k$ 
is GIT-(semi/poly)stable under $\Aut(X)$-action with respect to the linearization $\mO(\frac{c_1}{l_1},...,\frac{c_k}{l_k})$ if and only if $(X, \epsilon\sum_{j=1}^kc_jD_k)$ is K-(semi/poly)stable for $0<\epsilon\ll 1$. 
\end{proposition}

\begin{proof}
The idea of the proof is essentially the same as that of \cite[Theorem 1.1]{Zhou21a}. 

Suppose $(X, \epsilon\sum_{j=1}^kc_jD_k)$ is K-(semi/poly)stable, to show that $(D_1,...,D_k)$ is GIT-(semi/poly)stable under $\Aut(X)$-action with respect to the linearization $\mO(\frac{c_1}{l_1},...,\frac{c_k}{l_k})$, by Remark \ref{rem: futaki}, it suffices to show that $\lambda_{\CM, \epsilon}(\vec{c})$ is proportional to $\mO(\frac{c_1}{l_1}, ..., \frac{c_k}{l_k})$. This is exactly Lemma \ref{lem: compute cm}.

Conversely, suppose $(D_1,...,D_k)$ is GIT-(semi/poly)stable under $\Aut(X)$-action with respect to the linearization $\mO(\frac{c_1}{l_1},...,\frac{c_k}{l_k})$, then there is a family $(\mX, \sum_{j=1}^k\epsilon c_j\mG_j)\to C$ over a smooth pointed curve $0\in C$ such that 
\begin{enumerate}
\item there is a morphism $C\setminus{0}\to \bP_j$ for each $j$;
\item $(\mX\setminus \mX_0,\mG_j\setminus \mG_{j,0})$ is obtained via pulling back $(X\times \bP_j,\mD_j)$ under the morphism $C\setminus \{0\}\to \bP_j$;
\item $(\mX_0, \sum_{j=1}^k\mG_{j,0})\cong (X, \sum_{j=1}^kD_j)$;
\item $(\mX_t, \sum_{j=1}^k\epsilon c_j\mG_{j,t})$
is K-semistable for any $t\in C\setminus \{0\}$.
\end{enumerate}
By the properness of K-moduli, up to a finite base change, one could replace the central fiber $(\mX_0, \sum_{j=1}^k\mG_{j,0})\cong (X, \sum_{j=1}^kD_j)$ with a K-semistable log Fano pair $(X', \sum_{j=1}^k\epsilon c_j D_j')$ and we denote the new family by $(\mX', \sum_{j=1}^k\epsilon c_j \mG_j')\to C$ for convenience. We claim that it suffices to show that $\mX'_0\cong X$. Suppose $\mX_0'\cong X$, then $(D_1',...,D_k')\in \bP_1\times ...\times \bP_k$ and it is GIT-semistable under $\Aut(X)$-action with respect to the linearization $\mO(\frac{c_1}{l_1},...,\frac{c_k}{l_k})$  by what we have proved. By the separatedness of GIT moduli space, we know that $(D_1,...,D_k)$ and $(D_1',...,D_k')$ lie on the same orbit under $\Aut(X)$-action. Thus $(X, \epsilon\sum_{j=1}^kc_jD_k)$ is K-(semi/poly)stable.

The rest of the proof is devoted to show that $\mX_0'\cong X$. Since $\epsilon$ is sufficiently small, by Theorem \cite[Theorem 3.3]{Zhou21a}, we know that $\mX_0'$ is K-semistable. Note that $X$ is K-polystable, thus $\mX_0'\cong X$ by \cite[Theorem 3.5]{Zhou21a}. The proof is finished.
\end{proof}


Recall the following definition introduced in \cite{LZ23}.

\begin{definition}
Let $(D_1,...,D_k)\in \bP_1\times...\times \bP_k$, then the K-semistable domain of the log pair $(X, \sum_{j=1}^k D_j)$ is defined as follows:
$$\Kss(X, \sum_{j=1}^k D_j):=\overline{\{(x_1,...,x_k)\in \Delta^k\ |\ \textit{$(X, \sum_{j=1}^k x_jD_j)$ is K-semistable}\}}, $$
where the overline means taking the closure.
\end{definition}

By \cite[Theorem 1.4 and Corollary 1.5]{Zhou23b}, there exists a finite chamber decomposition of $\Delta^k$ such that for any $(D_1,...,D_k)\in \bP_1\times...\times \bP_k$, the K-semistability of $(X, \sum_{j=1}^kc_jD_j)$ does not change as $(c_1,...,c_k)$ varies in the interior domain of each chamber. We have

\begin{proposition}\label{prop: VGIT}
Near the original point of $\Delta^k$, the above chamber decomposition exactly gives the chamber decomposition for VGIT to encode GIT stability of $(D_1,...,D_k)\in \bP_1\times...\times \bP_k$ under $\Aut(X)$-action with respect to the linearization $\mO(\frac{c_1}{l_1},...,\frac{c_k}{l_k})$.
\end{proposition}

\begin{proof}
The proof is a combination of Proposition \ref{prop: GIT=K} and \cite[Theorem 1.4 and Corollary 1.5]{Zhou23a}.
\end{proof}

\begin{proof}[Proof of Theorem \ref{thm: main1}]
The proof is a combination of Propositions \ref{prop: GIT=K} and \ref{prop: VGIT}.
\end{proof}



\section{K-moduli of Fano complete intersections}


In this section, we prove Theorem \ref{thm: main2}. Throughout the section, we fix a positive integer $n$ and $k$ positive integers $\vec{d}:=(d_1,...,d_k)$ with $\sum_{j=1}^kd_j<n+1$. We  exclude the case $d_1=d_2=...=d_k=1$ since it is not interesting. 
Denote $\vec{a}:=(a_1,...,a_k)$ for
$$a_j=\frac{\sum_{i=1}^{k}d_i+(n-k+1)d_j-(n+1)}{(n-k+1)d_j}. $$
Let $\mM^K_{n, \vec{d}}$ (resp. $M^K_{n, \vec{d}}$) be the K-moduli stack (resp. K-moduli space) generically parametrizing $n-k$ dimensional Fano manifolds of the form $\cap_{j=1}^{k}S_{d_j}$ which are K-semistable, and $\mM^K_{n, \vec{d}, \vec{x}}$ (resp. $M^K_{n, \vec{d}, \vec{x}}$) the K-moduli stack (resp. K-moduli space) generically parametrizing K-semistable log Fano manifolds of the form  $(\bP^n, \sum_{j=1}^kx_jS_{d_j})$, where $\vec{x}:=(x_1,...,x_k)$ and  $S_{d_j}$ is a hypersurface of degree $d_j$ for each $1\leq j\leq k$.


\begin{proposition}\label{prop: good S}
Suppose $(\bP^n, \sum_{j=1}^ka_jS_{d_j})$ is a K-semistable log Fano pair (here $S_{d_j}$ are not necessarily smooth), then $S_{d_j},\ j=1,...,k,$ are different from each other and every $S_{d_j}$ is irreducible.
\end{proposition}

\begin{proof}
By the choice of $a_j, j=1,...,k$, we know
$$1-a_i-S_{\bP^n, \sum_{j=1}^ka_jS_{d_j}}(S_{d_i})=0 $$
for any $1\leq i\leq k$. We first show that $S_{d_1}$ is irreducible. Suppose not, there is an irreducible component of $S_{d_1}$, denoted by $D$, satisfying
$$\beta_{\bP^n, \sum_{j=1}^ka_jS_{d_j}}(D)=1-a_1-S_{\bP^n, \sum_{j=1}^ka_jS_{d_j}}(D) <0.$$
The inequality holds since 
$$S_{\bP^n, \sum_{j=1}^ka_jS_{d_j}}(D)>S_{\bP^n, \sum_{j=1}^ka_jS_{d_j}}(S_{d_i}).$$
Thus we get a contradiction to the K-semistability. Similarly, all $S_{d_j}$ are irreducible. Next we show that $S_{d_j}$ are mutually different. Suppose $S_{d_1}=S_{d_2}$, then 
$$\beta_{\bP^n, \sum_{j=1}^ka_jS_{d_j}}(S_{d_1})=1-(a_1+a_2)-S_{\bP^n, \sum_{j=1}^ka_jS_{d_j}}(S_{d_1})<0, $$
which is again a contradiction to the K-semistability. The proof is finished.
\end{proof}


\begin{proposition}\label{prop: if}
Suppose $(\bP^n, \sum_{j=1}^ka_jS_{d_j})$ is a K-semistable log Fano pair (here $S_{d_j}$ are not necessarily smooth), then $\cap_{j=1}^k S_{d_j}$ is a K-semistable Fano variety of dimension $n-k$.
\end{proposition}

\begin{proof}
We divide the proof into several steps.

\

Step 1.
Since $\beta_{\bP^n, \sum_{j=1}^ka_jS_{d_j}}(S_{d_1})=0$, by Remark \ref{rem: kss degeneration}, one could produce a special test configuration $(\mX, \sum_{j=1}^k a_j\mG_j)\to \bA^1$ of $(\bP^n, \sum_{j=1}^{k}a_j S_{d_j})$ such that the central fiber is the projective cone over $(S_{d_1}, \sum_{j=2}^ka_j{S_{d_j}}|_{S_{d_1}})$ with the polarization $\mO_{S_{d_1}}(d_1)$, and
$$\Fut(\mX, \sum_{j=1}^k a_j\mG_j; -K_{\mX/\bA^1})=0. $$
Thus the central fiber $(\mX_0, \sum_{j=1}^k a_j\mG_{j,0})$ is also K-semistable. Note that we have
$$\mO_{S_{d_1}}(d_1)\sim_\bQ -\frac{1}{r}(K_{S_{d_1}}+\sum_{j=2}^ka_j{S_{d_j}}|_{S_{d_1}})
\quad \text{and}  \quad
a_1=1-\frac{r}{n}$$
for 
$$r= \frac{n+1-d_1-\sum_{j=2}^ka_jd_j}{d_1}.$$ 
By Lemma \ref{lem:cone stability},  we see that $(S_{d_1}, \sum_{j=2}^ka_j{S_{d_j}}|_{S_{d_1}})$ is K-semistable. 

\

Step 2. 
By the similar proof of Proposition \ref{prop: good S}, one concludes that ${S_{d_j}}|_{S_{d_1}}, j=2,...,k,$ are different from each other and every ${S_{d_j}}|_{S_{d_1}}$ is irreducible. Note that
$$\beta_{S_{d_1}, \sum_{j=2}^ka_j{S_{d_j}}|_{S_{d_1}}}({S_{d_2}}|_{S_{d_1}})=0, $$
thus similar as Step 1, there exists a special test configuration $(\mX', \sum_{j=2}^k a_j\mG'_j)\to \bA^1$ of $(S_{d_1}, \sum_{j=2}^ka_j{S_{d_j}}|_{S_{d_1}})$ such that the central fiber is the projective cone over 
$$\left( {S_{d_2}}|_{S_{d_1}}, \sum_{j=3}^ka_j{({S_{d_j}}|_{S_{d_1}})}|_{({S_{d_2}}|_{S_{d_1}})}\right)=(S_{d_2}\cap S_{d_1}, \sum_{j=3}^k a_jS_{d_j}\cap S_{d_2}\cap S_{d_1})$$
with the polarization $\mO_{S_{d_1}\cap S_{d_2}}(d_2)$, and 
$$\Fut(\mX', \sum_{j=2}^ka_j\mG_j'; -K_{\mX'/\bA^1})=0. $$
Thus the central fiber $(\mX'_0, \sum_{j=2}^k a_j\mG'_{j,0})$ is also K-semistable. Note that we have
$$\mO_{S_{d_2}\cap S_{d_1}}(d_2)\sim_\bQ -\frac{1}{r}(K_{S_{d_2}\cap S_{d_1}}+\sum_{j=3}^ka_jS_{d_j}\cap S_{d_2}\cap S_{d_1})
\quad \text{and}  \quad
a_2=1-\frac{r}{n-1}$$
for 
$$r= \frac{n+1-d_1-d_2-\sum_{j=3}^ka_jd_j}{d_2}.$$ 
By Lemma \ref{lem:cone stability}, we see that $(S_{d_2}\cap S_{d_1}, \sum_{j=3}^k a_jS_{d_j}\cap S_{d_2}\cap S_{d_1})$ is K-semistable.

\

Step 3. 
Continuing the same process, we get that the log pair 
$$(S_{d_{k-1}}\cap...\cap S_{d_1}, a_kS_{d_k}\cap S_{d_{k-1}}\cap...\cap S_{d_1})$$ 
is K-semistable and $S_{d_k}\cap S_{d_{k-1}}\cap...\cap S_{d_1}$ is irreducible. Since 
$$\beta_{S_{d_{k-1}}\cap...\cap S_{d_1}, a_kS_{d_k}\cap S_{d_{k-1}}\cap...\cap S_{d_1}}(S_{d_k}\cap S_{d_{k-1}}\cap...\cap S_{d_1})=0,$$
there exists a special test configuration $(\mX'', a_k\mG_k'')\to \bA^1$ of 
$$(S_{d_{k-1}}\cap...\cap S_{d_1}, a_kS_{d_k}\cap S_{d_{k-1}}\cap...\cap S_{d_1})$$ 
such that the central fiber is K-semistable and it is the projective cone over $S_{d_1}\cap ...\cap S_{d_k}$ with the polarization $\mO_{S_{d_1}\cap ...\cap S_{d_k}}(d_k)$. Moreover, we have
$$\mO_{S_{d_k}\cap...\cap S_{d_1}}(d_k)\sim_\bQ -\frac{1}{r}K_{S_{d_k}\cap...\cap S_{d_1}}
\quad \text{and}  \quad
a_k=1-\frac{r}{n-k+1}$$
for 
$$r= \frac{n+1-\sum_{j=1}^{k}d_j}{d_k}.$$ 
Thus $S_{d_1}\cap ...\cap S_{d_k}$ is K-semistable by Lemma \ref{lem:cone stability}. The proof is finished.
\end{proof}



\begin{proposition}\label{prop: only if}
Suppose $(\bP^n, \sum_{j=1}^kS_{d_j})$ is log smooth. If $\cap_{j=1}^k S_{d_j}$ is K-semistable of dimension $n-k$, then $(\bP^n, \sum_{j=1}^ka_jS_{d_j})$ is a K-semistable log Fano pair.
\end{proposition}
\begin{proof}
In the log smooth setting, the proof of Proposition \ref{prop: if} can be reversed by Lemma \ref{lem:cone stability}. Concluded.
\end{proof}


\begin{remark}\label{rem: degeneration}
Suppose the log pair $(X, \sum_{j=1}^kB_j)$ is a degeneration of log smooth pairs of the form $(\bP^n, \sum_{j=1}^kS_{d_j})$ via a $\bQ$-Gorenstein flat family, i.e. there exists a family over a smooth pointed curve $0\in C$, denoted by $(\mX, \sum_{j=1}^k\mG_j)\to C$, such that 
\begin{enumerate}
\item $\mX\to C$ is a flat family such that $-K_{\mX/C}$ is $\bQ$-Cartier and relative ample over $C$,
\item each $\mG_j$ is a Weil divisor on $\mX$ and $\mG_j$ is proportional to $-K_{\mX/C}$ over $C$,
\item $(\mX_0, \sum_{j=1}^k\mG_{j,0})\cong (X, \sum_{j=1}^kB_j)$,
\item $(\mX_{t}, \sum_{j=1}^k\mG_{j, t})$ is of the form $(\bP^n, \sum_{j=1}^kS_{d_j})$ for $t\in C\setminus \{0\}$, where $(\bP^n, \sum_{j=1}^kS_{d_j})$ is log smooth,
\end{enumerate}
then Propositions 5.1 and 5.2 stated for $(\bP^n, \sum_{j=1}^kS_{d_j})$ also apply for $(X, \sum_{j=1}^kB_j)$. Indeed, for Proposition \ref{prop: good S}, the proof is similar for $(X, \sum_{j=1}^kB_j)$ to derive that $B_j$ are irreducible and mutually different. For Proposition \ref{prop: if}, it is a little different since $B_j$ may not be Cartier and Remark \ref{rem: kss degeneration} cannot be directly applied. However, the idea is the same and we only perform the first step since the rest could be done similarly. Suppose $(X, \sum_{j=1}^kB_j)$ is a degeneration of log smooth pairs of the form $(\bP^n, \sum_{j=1}^kS_{d_j})$ via a $\bQ$-Gorenstein flat family such that $(X, \sum_{j=1}^ka_jB_j)$ is a K-semistable log Fano pair. Then $B_j$ are prime divisors and we  have $\beta_{X, \sum_{j=1}^ka_jB_j}(B_i)=0$ for every $1\leq i\leq k$. Thus $B_1$ induces a special test configuration of $(X, \sum_{j=1}^ka_jB_j)$. Note that
$$X=\Proj \bigoplus_{m\in \bN} H^0(X, \mO_X(mB_1)). $$
By \cite[Section 2.3.1]{BX19}, the special test configuration can be reformulated as
$$\mX=\Proj \bigoplus_{m\in \bN}\bigoplus_{i\in \bZ} H^0(X, mB_1-iB_1) t^{-i}, $$
and the central fiber can be formulated as
$$\mX_0=\Proj \bigoplus_{m\in \bN}\bigoplus_{i\in \bZ} \left(H^0(X, mB_1-iB_1)/H^0(X, mB_1-(i+1)B_1)\right). $$
Consider the exact sequence 
$$0\to \mO_X(-B_1)\to \mO_X\to \mO_{B_1}\to 0. $$
Tensoring with the divisorial sheaf $\mO_X((m-i)B_1)$, we show that we still have the exactness
$$0\to \mO_X(-B_1)\otimes \mO_X((m-i)B_1)\to \mO_X((m-i)B_1)$$
for $i\leq m$. If $i=m$, it is clear that the exactness holds true. If $i<m$, we have the exactness since $X$ is $S_2$ (note that $X$ has klt singularities).
By Kawamata-Viehweg vanishing, $H^1(X, (m-i-1)B_1)=0$ for $i\leq m$, hence
$$H^0(X, mB_1-iB_1)/H^0(X, mB_1-(i+1)B_1)\cong H^0(B_1, \mO_{B_1}\otimes \mO_X((m-i)B_1)), $$
and
\begin{align*}
\mX_0 &\  \cong \Proj \bigoplus_{m\in \bN}\bigoplus_{i\in \bZ} H^0(B_1, \mO_{B_1}\otimes \mO_X((m-i)B_1)) \\
&\  \cong \Proj \bigoplus_{m\in \bN}\bigoplus_{i\in \bN} H^0(B_1, \mO_{B_1}(mB_1)) s^{i}.
\end{align*}
Thus we see that $(\mX_0, \sum_{j=1}^k a_j \mB_{j,0})$ is the projective cone over $(B_1, \sum_{j=2}^k a_jB_j\cap B_1)$ with respect to ${B_1}|_{B_1}$, 
%where $$K_{B_1}+\Delta_1+ \sum_{j=2}^kB_j\cap B_1:=(K_{\mX_0}+B_1+\sum_{j=1}^ka_j\mB_{j,0})|_{B_1}$$ 
and the infinite divisor is $\mB_{1,0}\cong B_1$. By the same computation as in the step 1 of the proof of Proposition \ref{prop: if}, we see that $(B_1, \sum_{j=2}^ka_j B_j\cap B_1)$ is K-semistable (e.g. \cite[Proposition 2.11]{LZ22}). Replacing $(X, \sum_{j=1}^ka_jB_j)$ with $(B_1, \sum_{j=2}^k a_jB_j\cap B_1)$ and applying the similar proof of Proposition \ref{prop: good S}, we see that $B_j\cap B_1$ are irreducible and mutually different for $2\leq j\leq k$, and 
$$\beta_{B_1, \Delta_1+\sum_{j=1}^ka_jB_j\cap B_1}(B_2\cap B_1)=0.$$
Continuing the same process as above we obtain a K-semistable log Fano pair 
$$(B_2\cap B_1, \sum_{j=3}^ka_jB_j\cap B_2\cap B_1)$$ 
and finally we see that $B_1\cap...\cap B_k$ is K-semistable.
\end{remark}

\begin{proof}[Proof of Theorem \ref{thm: main2}]
By Propositions \ref{prop: if} and \ref{prop: only if}, the log smooth pair $(\bP^n, \sum_{j=1}^ka_jS_{d_j})$ is K-semistable if and only if $S_{d_1}\cap...\cap S_{d_k}$ is K-semistable of dimension $n-k$. By Remark \ref{rem: degeneration}, given $(X, \sum_{j=1}^ka_jB_j)\in \mM^K_{n, \vec{d}, \vec{a}}$, then $\cap_{j=1}^k B_j$ is also K-semistable of dimension $n-k$. Thus we could define the morphism
$$\mM^K_{n, \vec{d}, \vec{a}}\to \mM^K_{n, \vec{d}},\ \  (X, \sum_{j=1}^ka_jB_j)\mapsto \cap_{j=1}^k B_j.$$
It descends to a morphism between K-moduli spaces:
$$\phi: M^K_{n, \vec{d}, \vec{a}}\to M^K_{n, \vec{d}}. $$
We aim to show that $\phi$ is surjective. Suppose $[Y]\in M^K_{n, \vec{d}}$  is of the form $S_{d_1}\cap...\cap S_{d_k}$ deduced by a log smooth pair $(\bP^n, \sum_{j=1}^k S_{d_j})$,  by 
Proposition \ref{prop: only if}, $[(\bP^n, \sum_{j=1}^ka_jS_{d_j})]$ is a preimage. Suppose $[Y]$ is not of the form $S_{d_1}\cap...\cap S_{d_k}$ deduced by a log smooth pair $(\bP^n, \sum_{j=1}^k S_{d_j})$. Then there is a flat family $\mY\to C$ over a smooth pointed curve $0\in C$ such that $-K_{\mY/C}$ is ample over $C$, $\mY_0\cong Y$, and $\mY_t$ is of the form $S_{d_1}\cap...\cap S_{d_k}$ deduced by a log smooth pair $(\bP^n, \sum_{j=1}^k S_{d_j})$. Let $\mY^*\to C^*:=C\setminus \{0\}$ be the family obtained via base change under the inclusion $C^*\to C$. Then there exists a family 
$$(\mX, \sum_{j=1}^k\mS_{d_j})\to C^*$$
such that for any $t\in C^*$, the fiber $(\mX_t, \sum_{j=1}^k\mS_{d_j, t})$ is isomorphic to some log smooth pair $(\bP^n, \sum_{j=1}^k S_{d_j})$ and $\cap_{j=1}^k \mS_{d_j, t}\cong \mY_t$. Note that $(\mX, \sum_{j=1}^ka_j\mS_{d_j})\to C^*$ is a family of K-semistable log Fano pairs, by the properness of K-moduli space, we could find a complete family $$(\tilde{\mX}, \sum_{j=1}^ka_j\tilde{\mS}_{d_j})\to C$$ 
up to a finite base change and the central fiber $(\tilde{\mX}_0, \sum_{j=1}^ka_j\tilde{\mS}_{d_j,0})$ is also K-semistable. By Remark \ref{rem: degeneration}, %and Proposition \ref{prop: if},  
$\cap_{j=1}^k \tilde{\mS}_{d_j,0}$ is K-semistable and $[Y]=[\cap_{j=1}^k \tilde{\mS}_{d_j,0}]$ by the separatedness of K-moduli space (e.g. \cite{BX19}). Thus $[(\tilde{\mX}_0, \sum_{j=1}^ka_j\tilde{\mS}_{d_j,0})]$ is a preimage of $[Y]$ and $\phi$ is surjective.
\end{proof}



\begin{example}
Let $n, d$ be two positive integers with $d<n+1$ and put $a:=1-\frac{n+1-d}{nd}$, then there is a natural morphism $\mM^K_{n,d,a}\to \mM^K_{n,d}$ which descends to a surjective morphism $M^K_{n,d,a}\to M^K_{n,d}$.
\end{example}



\begin{example}
Take $n=4$ and $\vec{d}=(d_1,d_2)=(2,2)$. Denote $\vec{a}:=(a_1,a_2)=(\frac{5}{6}, \frac{5}{6})$. Then there is a morphism $\mM^K_{4,\vec{d}, \vec{a}}\to \mM^K_{4, \vec{d}}$ sending a K-semistable log Fano pair $(\bP^4, \frac{5}{6}Q_1+\frac{5}{6}Q_2)$ to a K-semistable del Pezzo surface $Q_1\cap Q_2$ of degree $4$. This morphism descends to a surjective morphism $M^K_{4,\vec{d}, \vec{a}}\to M^K_{4, \vec{d}}$. We also note that $M^K_{4, \vec{d}}$ is isomorphic to the GIT moduli space of smoothable del Pezzo surfaces of degree 4 under the $\PGL(5)$-action with respect to the polarization induced by the Pl\"ucker embedding (e.g. \cite{SS17, OSS16, MM93}).
\end{example}




\begin{remark}
Fix a positive integer $n$ and $k$ positive integers $\vec{d}:=(d_1,...,d_k)$. Let $\vec{a}:=(a_1,...,a_k)$ be the vector defined at the start of this section. 
Denote $\mM^{GIT}_{n,\vec{d},\vec{a}}$ (resp. $M^{GIT}_{n,\vec{d},\vec{a}}$) to be the GIT-moduli stack (resp. GIT-moduli space) parametrizing GIT-semistable (resp. GIT-polystable) elements
$$(S_{d_1},...,S_{d_k})\in |\mO_{\bP^n}(d_1)|\times...\times 
|\mO_{\bP^n}(d_k)|$$ 
under $\Aut(\bP^n)$-action with respect to the linearization $\mO(1,...,1)$. By  Lemma \ref{lem: proportional cm} and Proposition \ref{prop: GIT=K}, we have 
$$\mM^{GIT}_{n,\vec{d},\vec{a}}\cong \mM^{K}_{n,\vec{d},c\vec{a}}\ \ ({\rm{resp}}.\  M^{GIT}_{n,\vec{d},\vec{a}}\cong M^{K}_{n,\vec{d},c\vec{a}})$$
for rational $0<c\ll 1$. As $c$ varies in $(0,1]\cap \bQ$, there is a wall crossing theory with one boundary (\cite{ADL19, Zhou23}). When $c=1$, applying Theorem \ref{thm: main2},  there is a surjective morphism $M^{K}_{n, \vec{d}, \vec{a}}\to M^K_{n, \vec{d}}$. Hence, Theorem \ref{thm: main1} together with Theorem \ref{thm: main2} actually give a relationship between $M^{GIT}_{n, \vec{d}, \vec{a}}$ and $M^K_{n, \vec{d}}$. More precisely, we first have a birational map $M^{GIT}_{n, \vec{d}, \vec{a}}\dashrightarrow M^{K}_{n, \vec{d}, \vec{a}}$ via wall crossing with one boundary, then we have a projective surjective morphism $M^{K}_{n, \vec{d}, \vec{a}}\to M^{K}_{n, \vec{d}}$.







\end{remark}




































\bibliography{reference.bib}
\end{document}











