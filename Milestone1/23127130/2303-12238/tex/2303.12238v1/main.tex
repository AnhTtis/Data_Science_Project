\documentclass[conference]{IEEEtran}
\newtheorem{Definition}{Definition}

\IEEEoverridecommandlockouts
% The preceding line is only needed to identify funding in the first footnote. If that is unneeded, please comment it out.
\usepackage{hyperref}
\usepackage{cite}
\usepackage{amsmath,amssymb,amsfonts}
\usepackage{algorithmic}
\usepackage{algorithm}
\usepackage{textcomp}
\usepackage{mwe}
\usepackage{multirow, multicol}
\usepackage{booktabs}
\usepackage{tabularx}
\usepackage{diagbox}
\usepackage{float}   
\usepackage[framemethod=tikz]{mdframed}
\usepackage{psfrag,epsfig,url,fancyhdr,color}
\usepackage{setspace}
\usepackage{mathtools}
\usepackage{graphicx}
\usepackage{xcolor}
\usepackage{caption}
\usepackage{subcaption}

\newcommand{\eg}{{\it e.g.}}
\newcommand{\ie}{{\it i.e.}}
\newcommand{\etal}{{\it et al.}}

\def\BibTeX{{\rm B\kern-.05em{\sc i\kern-.025em b}\kern-.08em
    T\kern-.1667em\lower.7ex\hbox{E}\kern-.125emX}}

% Math Fonts===================================================================\
\newcommand{\brm}[1]{\boldsymbol{\mathrm{#1}}} % bold Roman symbol
\newcommand{\argmin}[1]{\underset{#1}{\operatorname{argmin}}\;}
\newcommand{\argmax}[1]{\underset{#1}{\operatorname{argmax}}\;}
\newcommand{\prox}{\operatorname{prox}}
% \newcommand{\brm}[1]{\boldsymbol{\mathrm{#1}}} % bold Roman symbol
% \def\A{\mathbf{A}}
\DeclareMathAlphabet{\mathdata}{OMS}{cmsy}{m}{n}

\def\x{\brm{x}}
\def\X{\brm{X}}
\def\h{\brm{h}}
\def\H{\brm{H}}
\def\a{\brm{a}}
\def\A{\brm{A}}
\def\c{\brm{c}}
\def\C{\brm{C}}
\def\o{\brm{o}}
\def\O{\brm{O}}
\def\v{\brm{v}}
\def\V{\brm{V}}
\def\y{\brm{y}}
\def\Y{\brm{Y}}
\def\s{\brm{s}}
\def\S{\brm{S}}
\def\g{\brm{g}}
\def\G{\brm{G}}
\def\w{\brm{w}}
\def\W{\brm{W}}
\def\i{\brm{i}}
\def\I{\brm{I}}
\def\u{\brm{u}}
\def\U{\brm{U}}
\def\q{\brm{q}}
\def\Q{\brm{Q}}
\def\bPhi{\brm{\Phi}}
\def\btheta{\brm{\theta}}
\def\T{\mathrm{T}}

\begin{document}

\pdfoutput=1
\title{DG-Trans: Dual-level Graph Transformer for Spatiotemporal Incident Impact Prediction on Traffic Networks}

\author{\IEEEauthorblockN{Yanshen Sun\textsuperscript{1}, Kaiqun Fu\textsuperscript{2}, and Chang-Tien Lu\textsuperscript{1}}
\IEEEauthorblockA{%\textit{dept. Computer Science} \\
\textit{\textsuperscript{1}Virginia Tech}, \textit{\textsuperscript{2}South Dakota State University}\\
% Falls Church, VA\\
yansh93@vt.edu, kaiqun.fu@sdstate.edu, ctlu@vt.edu}\\
}
\IEEEaftertitletext{\vspace{-2\baselineskip}}
\maketitle

\begin{abstract}
The prompt estimation of traffic incident impacts can guide commuters in their trip planning and improve the resilience of transportation agencies' decision-making on resilience. However, it is more challenging than node-level and graph-level forecasting tasks, as it requires extracting the anomaly subgraph or sub-time-series from dynamic graphs. In this paper, we propose DG-Trans, a novel traffic incident impact prediction framework, to foresee the impact of traffic incidents through dynamic graph learning. The proposed framework contains a dual-level spatial transformer and an importance-score-based temporal transformer, and the performance of this framework is justified by two newly constructed benchmark datasets.
The dual-level spatial transformer removes unnecessary edges between nodes to isolate the affected subgraph from the other nodes. Meanwhile, the importance-score-based temporal transformer identifies abnormal changes in node features, causing the predictions to rely more on measurement changes after the incident occurs. Therefore, DG-Trans is equipped with dual abilities that extract spatiotemporal dependency and identify anomaly nodes affected by incidents while removing noise introduced by benign nodes. Extensive experiments on real-world datasets verify that DG-Trans outperforms the existing state-of-the-art methods, especially in extracting spatiotemporal dependency patterns and predicting traffic accident impacts. It offers promising potential for traffic incident management systems.

\end{abstract}

\begin{IEEEkeywords}
spatiotemporal data mining, intelligent transportation systems, incident impact forecasting, transformer, anomaly detection.
\end{IEEEkeywords}

\section{Introduction}
\label{sec:introduction}
% \begin{itemize}
%     % Diffusion of FL
%     \item {\st{Diffusion of FL}}
%     % Security threats to FL
%     \item {\st{Security threats to FL with particular focus on model poisoning}}
%     % Limitations of existing countermeasures
%     \item {\st{Current countermeasures (e.g., KRUM) and their limitations}}
%     % Proposed method and its advantages
%     \item {\st{Intuitive description of the proposed method and its difference (i.e., advantages) w.r.t. state of the art}}
%     % Main contributions
%     \item {\st{Summary of the main contributions of this work}}
%     % Paper's structure and organization
%     \item {\st{Paper's structure and organization}}
% \end{itemize}

% Diffusion of FL
Recently, {\em federated learning} (FL) has emerged as the leading paradigm for training distributed, large-scale, and privacy-preserving machine learning (ML) systems~\cite{mcmahan2017googleai,mcmahan2017aistats}. 
The core idea of FL is to allow multiple edge clients to collaboratively train a shared, global model without disclosing their local private training data.
%Specifically, an FL system consists of a central server and many edge clients; 
A typical FL round involves the following steps: {\em(i)} the server randomly picks some clients and sends them the current, global model; {\em(ii)} each selected client locally trains its model with its own private data; then, it sends the resulting local model to the server;\footnote{Whenever we refer to global/local model, we mean global/local model {\em parameters}.} {\em(iii)} the server updates the global model by computing an \emph{aggregation function}, usually the average (FedAvg), on the local models received from clients.
% \begin{enumerate}
%     \item[{\em(i)}] the server sends the current, global model to the clients and appoints some of them for training;
%     \item[{\em(ii)}] each selected client locally trains its copy of the global model with its own private data; then, it sends the resulting local model back to the server;\footnote{Whenever we refer to global/local model, we mean global/local model {\em parameters}.}
%     \item[{\em(iii)}] the server updates the global model by computing an \emph{aggregation function} on the local models received from clients (by default, the average, also referred to as FedAvg~\cite{mcmahan2017aistats}).
% \end{enumerate}
This process goes on until the global model converges. %(e.g., after a certain number of rounds or other similar stopping criteria).
%\\
% The advantages of FL over the traditional, centralized learning paradigm are undoubtedly clear in terms of flexibility/scalability (clients can join/disconnect from the FL network dynamically), network communications (only model weights\footnote{We will use \textit{parameters} and \textit{weights} interchangeably.} are exchanged between clients and server), and privacy (each client's private training data is kept local at the client's end and not uploaded to the server).
\\
% Security threats to FL
%However, the growing adoption of FL also raises security concerns~\cite{costa2022covert}, particularly about its confidentiality, integrity, and availability.
Although its advantages over standard ML, FL also raises security concerns~\cite{costa2022covert}. %, particularly about its confidentiality, integrity, and availability~\cite{costa2022covert}.
% OLD, LONG VERSION
% Indeed, some work deals with privacy leakage that may expose the local data of some clients~\cite{melis2019sp}. 
% A large body of work, instead, investigates attacks that usually aim to detriment the predictive accuracy of the learned global model. For instance, \emph{data poisoning} attacks achieve this goal by letting an adversary pollute the training set of some corrupt FL clients with maliciously crafted examples~\cite{jagielski2018sp}.
% Similarly, in \emph{model poisoning} the attacker attempts to tweak the global model weights~\cite{bhagoji2019pmlr} by directly perturbing the local model's weights of some infected FL clients before these are sent to the central server for aggregation, usually via so-called Byzantine attacks. 
% It turns out that Byzantine model poisoning attacks severely impact standard FedAvg; therefore, more robust aggregation functions must be designed to make FL systems secure.
Here, we focus on \emph{untargeted model poisoning} attacks~\cite{bhagoji2019pmlr}, where an adversary attempts to tweak the global model weights %\footnote{We will use the terms \textit{parameters} and \textit{weights} interchangeably.} 
by directly perturbing the local model's parameters of some infected clients before these are sent to the central server for aggregation.
In doing so, the adversary aims to jeopardize the global model \textit{indiscriminately} at inference time.
Such model poisoning attacks severely impact standard FedAvg; therefore, more robust aggregation functions must be designed to secure FL systems.
\\
% In this paper, we focus on designing a novel robust aggregation scheme at the server's end to contrast the effect of Byzantine model poisoning attacks.
%
% Current countermeasures and their limitations
%Several countermeasures have been proposed in the literature to combat model poisoning attacks on FL systems.
% Some methods use simple statistics more robust than plain average to smooth the impact of malicious updates (e.g., Trimmed Mean and FedMedian~\cite{yin2018icml}). 
% Other defenses implement outlier detection techniques to discard malicious updates from the aggregation performed at the server's end. Those are either based on heuristics (e.g., Krum/Multi-Krum~\cite{blanchard2017nips} and Bulyan~\cite{mhamdi2018pmlr}) or data-driven approaches (e.g., K-means clustering~\cite{shen2016acm} or DnC via spectral analysis~\cite{shejwalkar2021ndss}). 
% Finally, some strategies rely on a centralized ``source of trust'' to spot potential malicious updates (e.g., FLTrust~\cite{cao2020fltrust}).
% Several countermeasures have been proposed in the literature to combat model poisoning attacks on FL systems, i.e., to discard possible malicious local updates from the aggregation performed at the server's end. 
% These techniques range from simple statistics more robust than plain average (e.g., Trimmed Mean and FedMedian~\cite{yin2018icml}) to outlier detection heuristics (e.g., Krum/Multi-Krum~\cite{blanchard2017nips} and Bulyan~\cite{mhamdi2018pmlr}) or data-driven approaches (e.g., spectral analysis via K-means clustering~\cite{shen2016acm} or spectral analysis), or methods based on ``source of trust'' (e.g., FLTrust~\cite{cao2020fltrust}).
% OLD, LONG VERSION
%Several countermeasures have been proposed in the literature to combat Byzantine model poisoning attacks on FL systems.
% Descriptive statistics
% For example, Trimmed Mean and FedMedian aggregate local model updates using more robust statistics than standard average~\cite{yin2018icml}.
%
% % Heuristics for outlier detection
% Many existing Byzantine-resilient strategies implement some outlier detection heuristics to discard the model updates sent by potentially malicious clients from the input of the aggregation function.
% One of the most popular heuristics is Krum~\cite{blanchard2017nips}.
% This strategy tries to mitigate the impact of Byzantine attacks by selecting as a global model the local model with the smallest sum of Euclidean distances to {\em all} the other local models.
% Although powerful, Krum requires the server to know (or, at least, estimate) the number of malicious FL clients upfront, which is generally impossible in a realistic attack scenario. %
% Moreover, Krum may become ineffective for complex, high-dimensional model parameter spaces due to the curse of dimensionality.
% Bulyan~\cite{mhamdi2018pmlr} tries to overcome this issue by combining Krum with a variant of Trimmed Mean.
% % Data-driven outlier detection
% Other strategies use data-driven outlier detection techniques -- e.g., via K-means clustering~\cite{shen2016acm} -- to spot potential malicious local model updates. 
% %For instance, Shen et al. propose to cluster local model updates with K-means and thus identify outliers.
%
% % Other techniques
% As far as the server is concerned, any local model received can be from a potential malicious client. 
% FLTrust~\cite{cao2020fltrust} assumes the server acts as a client, i.e., trains a local model on an additional {\em trustworthy} dataset at the server's end and compares it against all the local models from other clients. 
% This way, the server can rely on some ``source of trust'' when discarding potentially malicious clients.
%\\
% Limitations of existing Byzantine-resilient strategies
Unfortunately, existing defense mechanisms either rely on simple heuristics (e.g., Trimmed Mean and FedMedian by~\cite{yin2018icml}) or need strong and unrealistic assumptions to work effectively (e.g., foreknowledge or estimation of the number of malicious clients in the FL system, as for Krum/Multi-Krum~\cite{blanchard2017nips} and Bulyan~\cite{mhamdi2018pmlr}, which, however, cannot exceed a fixed threshold).
Furthermore, outlier detection methods using K-means clustering~\cite{shen2016acm} or spectral analysis like DnC~\cite{shejwalkar2021ndss} do not directly consider the temporal evolution of local model updates received.
Finally, strategies like FLTrust~\cite{cao2020fltrust} require the server to collect its own dataset and act as a proper client, thereby altering the standard FL protocol.
\\
% OLD, LONG VERSION
% Overall, existing Byzantine-resilient strategies are either simple heuristics (e.g., FedMedian) or, if they are more complex, they rely on strong and unrealistic assumptions to work effectively (e.g., knowing the number of malicious clients in the FL system in advance, as for Krum and alike).
% Furthermore, data-driven outlier detection methods do not consider the temporary evolution of local model updates received (e.g., K-means clustering). 
% Finally, strategies like FLTrust requires the server to collect its own dataset and act as a proper client, thereby altering the standard FL protocol.
%
% Description of the proposed method
This work introduces a novel pre-aggregation \textit{filter} robust to untargeted model poisoning attacks. Notably, this filter $(i)$ operates without requiring prior knowledge or constraints on the number of malicious clients and $(ii)$ inherently integrates temporal dependencies. 
The FL server can employ this filter as a preprocessing step before applying \textit{any} aggregation function, be it standard like FedAvg or robust like Krum or Bulyan.
Specifically, we formulate the problem of identifying corrupted updates as a multidimensional (i.e., matrix-valued) time series anomaly detection task. 
The key idea is that legitimate local updates, resulting from well-calibrated iterative procedures like stochastic gradient descent (SGD) with an appropriate learning rate, show \textit{higher predictability} compared to malicious updates. This hypothesis stems from the fact that the sequence of gradients (thus, model parameters) observed during legitimate training exhibit regular patterns, as validated in Section~\ref{subsec:intuition}. %until convergence. 
%This regularity may be more pronounced for smooth convex loss functions, but it can still be captured within an appropriate time window, even for more complex and convoluted loss surfaces. 
%We provide evidence of this claim in Appendix~B, where we show that the average mutual information (i.e., ``predictability''), calculated over pairs of legitimate model updates sent at different FL rounds, is significantly higher than the corresponding computation for a malicious client.
\\
Inspired by the matrix autoregressive (MAR) framework for multidimensional time series forecasting~\cite{chen2021je}, we propose the FLANDERS ({\em \textbf{F}ederated \textbf{L}earning meets \textbf{AN}omaly \textbf{DE}tection for a \textbf{R}obust and \textbf{S}ecure}) filter.
The main advantages of FLANDERS over existing strategies like FLDetector~\cite{zhao2020multivariate} are its resilience to large-scale attacks, where $50\%$ or more FL participants are hostile, and the capability of working under realistic non-iid scenarios.
We attribute such a capability to two key factors: $(i)$ FLANDERS works without knowing a priori the ratio of corrupted clients, and $(ii)$ it embodies temporal dependencies between intra- and inter-client updates, quickly recognizing local model drifts caused by evil players. Below, we summarize our main contributions:

\begin{itemize}
\item[{\em(i)}]
We provide empirical evidence that the sequence of models sent by legitimate clients is more predictable than those of malicious participants performing untargeted model poisoning attacks.
\\
\item[{\em(ii)}] 
We introduce FLANDERS, the first pre-aggregation filter for FL robust to untargeted model poisoning based on multidimensional time series anomaly detection.
\\
\item[{\em(iii)}] 
We integrate FLANDERS into Flower,\footnote{\scriptsize{\url{https://flower.dev/}}} a popular FL simulation framework for reproducibility.
\\
\item[{\em(iv)}] 
We show that FLANDERS improves the robustness of the existing aggregation methods under multiple settings: different datasets, client's data distribution (non-iid), models, and attack scenarios.
\\
\item[{\em(v)}] 
We publicly release all the implementation code of FLANDERS along with our experiments.\footnote{\scriptsize{\url{https://anonymous.4open.science/r/flanders_exp-7EEB}}}
\end{itemize}

% Paper's structure and organization
The remainder of the paper is structured as follows. %some related work and the current state-of-the-art solutions to security issues that FL entails. 
Section~\ref{sec:background} covers background and preliminaries. 
In Section~\ref{sec:related}, we discuss related work.
Section~\ref{sec:problem} and Section~\ref{sec:method} describe the problem formulation and the method proposed. % to tackle it. 
Section~\ref{sec:experiments} gathers experimental results. %, and Section~\ref{sec:limitations} discusses some limitations of this work.
Finally, we conclude in Section~\ref{sec:conclusion}.
 %discusses the limitations of this work and draws future research directions.
%reports conclusions and draws perspectives for future research directions.

%%%%%%% OLD %%%%%%%
%to overcome the resilience of Byzantine failures in distributed Stochastic Gradient Descent computations. 
% The strength of Krum is its time complexity, which is linear in the gradient dimension. 
% However, the robustness of the approach is guaranteed for gradient-based learning applications only when the majority of the clients are not compromised. 
% Besides, the aggregation mechanism of Krum, as well as that of similar methods, is robust from a coarse-grained perspective and does not provide solutions to errors and perturbations that may occur at inference time.
%A related approach to~\cite{blanchard2017nips} is the work of Su et al.~\cite{su2016dc}. Here, the authors propose an iterated approximate agreement to tackle a multi-layer scenario attacked by Byzantine agents. 
%However, the method works efficiently on the sole discrete context and it is inapplicable to continuous state environments.
%\gabri{Maybe, we should just talk about the main limitations of existing countermeasures without digging into their details (or, we can just mention Krum as this is the most popular one). I will move the description of all these methods to the Related Work section.}
\section{Related Work}
\label{sec:relatedwork}

%%%%%%%%%%%%%%%%%%%%%%%%%% Outline %%%%%%%%%%%%%%%%%%%%%%%%%%%%%%%%%%%%%
%(1) Evasion Attacks
%(1.1) Surveys on evasion attacks and their relation to data properties - Michael
%(1.2) Individual papers that study non-data related reasons behind evasion attacks - Michael
%(1.3) Techniques related to evasion attacks and defenses (new) - Gabby
%(2) Non-Evasion Attacks (new), and - ???
%(3) Effects of training data on standard generalization - done 
%
%
%
%(1) Evasion Attacks
%(1.1) A number of surveys review literature on evasion attacks. - Michael
%Most of them do not focus specifically on properties of data but also discuss attack and defense mechanisms, non-data-related reasons for adversarial vulnarability, and  more. ~\jr{cite 4}.
%Yet, they these surveys mention data and its relation to evasion attacks. Specifically \jr{what they say about data.}
%The most close to ours is concurrent work by XXX + concrete facts that we have and they don't.
%
%(1.2) individual papers that study non-data related reasons behind evasion attacks, - Michael
%Literature identifies multiple reasons for adversarial vulnerability, in particular, for evasion attacks. 
%These include data-related properties extensively discussed in this survey, as well as reasons related to the models 		   themselves, computations resources, and feature representations. We discuss these below. 
%
%\jr{the rest is from the paper (non-data related reasons for adversarial vulnerability), with sections potentially renamed.}
%
%{\bf Model.}
%
%{\bf Computational Resources.}
%
%{\bf Robustness of Features.}
%
%(1.3) Techniques Related to Evasion Attacks and Defenses (new) - Gabby
%A number of works focus on techniques for generating evasion attacks, countermeasures against these attacks, 
%and defining the notion of the attack itself.   
%
%{\bf Attacks and Defense.}
%Here are the 5 remaining surveys + 1 additional paper for the reviewer.
%
%{\bf Adversarial Examples.}
%2 surveys lines 13 and 14 + 1 additional paper for the reviewer.
%
%(2) Non-Evasion Attacks (new) 
%Need to say that there are other type of attacks, define them, cite surveys (Bo's survey, maybe something else). 
%Only one work explicitly focus on effects of data. 
%
%
%(3) Effects of training data on standard generalization (done)

%%%%%%%%%%%%%%%%%%%%%%%%% Outline %%%%%%%%%%%%%%%%%%%%%%%%%%%%%%%%%%%%%


\revreplace{
We divide related work into three categories:
(1) surveys on adversarial robustness and its relation to data properties,
(2) surveys that discuss the influence of data properties on standard generalization, and
(3) individual papers that study non-data-related reasons for adversarial vulnerability.\\
}
{
This survey investigates properties of training data in the context of model robustness under evasion attacks. 
We start the discussion of related work by reviewing other surveys that focus on evasion attacks and 
include some discussion about data (Section~\ref{sec:relatedwork-surveys-data}).  
We then discuss non-data related reasons behind evasion attacks (Section~\ref{sec:relatedwork-not-data}),
as well as techniques related to evasion attacks and defenses (Section~\ref{sec:relatedwork-attacks}). 
Finally, we discuss data-related concerns for non-evasion attacks (Section~\ref{sec:relatedwork-poisoning}) and
the effects of training data on standard generalization (Section~\ref{sec:relatedwork-standard}).
}

%\vspace{-0.1in}
\subsection{Surveys on Evasion Attacks that Discuss Data}
\label{sec:relatedwork-surveys-data}
Numerous existing surveys 
\revreplace{focus on attack and defense techniques for adversarial robustness. 
%~\cite{Biggio:Roli:PR:2018,
%Rosenberg:Shabtai:Elovici:Rokach:CSUR:2021,
%Li:Li:Ye:Xu:CSUR:2021,
%Maiorca:Biggio:Giorgio:CSUR:2019,
%Demetrio:Coull:Biggio:Lagorio:Armando:Roli:ACMTPS:2021,
%Liu:Tantithamthavorn:Li:Liu:CSUR:2022,
%Liu:Nogueria:Fernandes:Kantarci:IEEECST:2022,
%Akhtar:Mian:IEEEAccess:2018,
%Akhtar:Mian:Kardan:Shah:IEEEAccess:2021,
%Serban:Poll:Visser:CSUR:2020,
%Machado:Silva:Goldschmidt:CSUR:2021,
%Zhang:Sheng:Alhazmi:Li:ACMTIST:2020}.
Only a few of these works mention the relationship between adversarial robustness and properties of the underlying data.} 
{review the literature on evasion attacks.
Most of these works do not focus specifically on properties of data but discuss attack and defense mechanisms, non-data-related reasons for adversarial vulnerability, 
and the different threat models. 
Only a few of these works mention data-related reasons for the existence of adversarial examples~\cite{Serban:Poll:Visser:CSUR:2020, Machado:Silva:Goldschmidt:CSUR:2021, Akhtar:Mian:Kardan:Shah:IEEEAccess:2021, Akhtar:Mian:IEEEAccess:2018}.
}
Specifically, Serban et al.~\cite{Serban:Poll:Visser:CSUR:2020} observe that adversarial vulnerability can be caused by an insufficient training sample size %~\cite{Schmidt:Santurkar:Tsipras:Talwar:Madry:NeurIPS:2018}
and high data dimensionality. %~\cite{Gilmer:Metz:Faghri:Schoenholz:Raghu:Wattenberg:Goodfellow:ICLR:2018}.
Similarly, Machado et al.~\cite{Machado:Silva:Goldschmidt:CSUR:2021} mention that the lack of sufficient training data, high dimensionality, 
and high concentration contribute to adversarial vulnerability.
\revadd{
Akhtar et al.~\cite{Akhtar:Mian:IEEEAccess:2018, Akhtar:Mian:Kardan:Shah:IEEEAccess:2021} also mention high dimensionality, along with other non-data-related reasons, 
as a source of adversarial examples.}

\revadd{A concurrent work by Han et al.~\cite{Han:Lin:Shen:Wang:Guan:CSUR:2023} (published at the end of April 2023) 
studies the origins of adversarial vulnerability in deep learning w.r.t. the model, data, and other perspectives.
The authors mention high dimensionality, distributions with high concentration, a small number of output classes, data imbalance, and the perceptual difference in image frequencies as potential sources of adversarial examples.
However, as (a) the focus of that survey is not on data-related properties in particular, 
(b) its paper search was conducted in 2021, and 
(c) it focuses on deep learning models only, 
our work was able to identify more than 50 additional relevant papers which focus on other types of models, 
e.g., non-parametric and linear classifiers, 
and/or discuss additional types of data-related properties, 
such as, types of distribution, class density, separation, and label quality.}
\revreplace{Yet, none of these surveys explicitly collect and analyze work that focuses on the effects of data properties
on adversarial robustness.}
{In summary, by explicitly focusing on the effects of data properties on evasion attacks in our survey, 
we are able to provide a more complete and detailed discussion on this topic, not covered in prior surveys.}

\vspace{-0.05in}
\subsection{Non-data-related Reasons Behind Evasion Attacks}
\label{sec:relatedwork-not-data}

%\vspace{-0.1in}
%\subsection{Non-data Related Reasons for Adversarial Vulnerability}

There has been a variety of hypotheses regarding the reasons behind adversarial vulnerability of ML systems, particularly for evasion attacks.
%\revreplace{
%In addition to the data used for training,  adversarial robustness could also depend on the choice of the model architecture,
%the training procedure, and the interplay between data and the learning algorithm, i.e., correspondence between the complexity of a model to that of the data.
%This section summarizes the key hypotheses regarding these aspects.
%%The hypotheses reviewed in this section are complementary to the potential influence from the data.
%}
These include data-related properties extensively discussed in this survey, as well as reasons related to the models themselves, 
computational resources, and feature learning procedures. We discuss these below.

%\jr{there is a lot of undefined terminology and jargon in this section.}

\vspace{0.02in}
\noindent
\textbf{Model.}
When Szegedy et al.~\cite{Szegedy:Zaremba:Sutskever:Bruna:Erhan:Goodfellow:Fergus:ICLR:2014} first discovered adversarial examples for visual models, they suspected that the high non-linearity of DNNs resulted in low probability `pockets' of adversarial examples in the learned representation manifold.
They hypothesize that while these pockets can be found through attack algorithms, the samples residing in these pockets have different distributions compared to normal samples and are thus subsequently harder to find when randomly sampling from the input space.
Instead, Goodfellow et al.~\cite{Goodfellow:Shlens:Szegedy:ICLR:2015} hypothesize that
the linearity from activation functions, like ReLU and sigmoid found in high-dimensional neural networks, induce vulnerability towards adversarial perturbations.
To support their claim, they present the attack method FGSM that exploits the linearity of the target classifier.
Fawzi et al.~\cite{Fawzi:Fawzi:Frossard:ICMLWorkshop:2015} also argue against the hypothesis of high non-linearity as the cause for adversarial examples.
They show that all classifiers are susceptible to adversarial attacks and claim that it is the low flexibility of the classifier compared to the complexity of the classification task that results in vulnerability.
The lack of consensus on the primary causes of model vulnerability invites more studies on this topic.

Singla et al.~\cite{Singla:Ge:Basri:Jacobs:NeurIPS:2021} show that enforcing invariance to circular shifts (e.g., rotation) in neural networks induces decision boundaries with a smaller margin than normal, fully connected networks,
which, in turn, reduces the adversarial robustness of the model.
Moosavi{-}Dezfooli et al.~\cite{Moosavi-Dezfooli:Fawzi:Fawzi:Frossard:Soatto:ICLR:2018} introduce universal,
input-agnostic perturbations to mislead the classifier and hypothesize that the vulnerability of a multi-class classifier to such perturbations is related to the shape of its decision boundaries, e.g.,
linear classifiers with decision boundaries that are parallel to each other and
nonlinear classifier with decision boundaries that are curved in a similar way
tend to be less robust as
perturbations in one direction can change the prediction label for a different class.

Tanay and Griffin~\cite{Tanay:Griffin:ArXiv:2016} conjecture that the decision boundary learned by the classifier being too close to (or `tilted towards') the data manifold instead of being perpendicular to it,
results in small perturbations being sufficient to move samples across the decision boundary for misclassification.
%data manifold refers to the underlying structure that the data exhibit

\vspace{0.02in}
\noindent
\textbf{Computational Resources.}
Bubeck et al.~\cite{Bubeck:Lee:Price:Razenshteyn:ICML:2019} use computational hardness theory to show that the time complexity for learning a robust model is exponential to the size of input data and thus is computationally intractable.
Hence, they attribute adversarial vulnerability to computational limitations of current learning algorithms.
Degwekar et al.~\cite{Degwekar:Nakkiran:Vaikuntanathan:COLT:2019} further extend this work and also show the impossibility of efficiently training robust classifiers.

%\subsubsection{Ineffective Learning Perspective}
\vspace{0.02in}
\noindent
\textbf{Feature Learning.}
Ilyas et al.~\cite{Ilyas:Santurkar:Tsipras:Engstrom:Tran:Madry:NeurIPS:2019} show that adversarial vulnerability can be a consequence of a model exploiting well-generalizing but non-robust features,
i.e., features that are spurious and sometimes incomprehensible to humans;
when constraining the model to use robust features, the adversarial robustness increases together with the
interpretability of the learned features.
However, Tsipras et al.~\cite{Tsipras:Santurkar:Engstrom:Turner:Madry:ICLR:2019} note that, as the features for achieving high accuracy may be different from the ones for achieving high robustness, robustness may be at odds with standard accuracy.
%
%\jr{why is it called Ineffective learning when it is about features.}\gx{I put it under ineffective learning as in this case, the model learns/decides the features for generalization, and when given the correct objective, the model in fact, can learn more robust features, so I think the underlying reason is objective we gave for the model didn't guide the model to learn the right features}
%
Instead of seeing adversarial vulnerability as a product of classifiers being overly sensitive to changes in spurious features, Jacobsen et al.~\cite{Jacobsen:Behrmann:Zemel:Bethge:ICLR:2019} hypothesize that classifiers can rather be
overly insensitive to relevant semantic information, e.g., images with drastically different content can share similar latent representations.
The authors introduce a new type of adversarial examples that exploit such insensitivity, where the content of images is altered without changing the resulting prediction label.
%As both insensitivity to semantic content and sensitivity to spurious changes can simultaneously exist in models,
%more investigation into how to define proper objectives for models to effectively distinguish the relevant information is needed.

While all these works propose possible reasons for adversarial vulnerabilities, they are orthogonal to our survey, which focuses particularly on the influence of training data.

\vspace{-0.05in}
\revadd{
\subsection{Evasion Attacks and Defenses}
\label{sec:relatedwork-attacks}
A number of works focus on techniques for generating evasion attacks, countermeasures against these attacks, 
and defining the notion of the attack itself.

%\jr{need to include~\cite{Biggio:Roli:PR:2018,
%Rosenberg:Shabtai:Elovici:Rokach:CSUR:2021,
%Li:Li:Ye:Xu:CSUR:2021,
%Maiorca:Biggio:Giorgio:CSUR:2019,
%Demetrio:Coull:Biggio:Lagorio:Armando:Roli:ACMTPS:2021,
%Liu:Tantithamthavorn:Li:Liu:CSUR:2022,
%Liu:Nogueria:Fernandes:Kantarci:IEEECST:2022,
%Zhang:Sheng:Alhazmi:Li:ACMTIST:2020} x and one more survey.}
%\js{\cite{Biggio:Roli:PR:2018, Rosenberg:Shabtai:Elovici:Rokach:CSUR:2021} moved to Adversarial Examples.
%\cite{Rosenberg:Shabtai:Elovici:Rokach:CSUR:2021,
%Li:Li:Ye:Xu:CSUR:2021,
%Maiorca:Biggio:Giorgio:CSUR:2019, Liu:Tantithamthavorn:Li:Liu:CSUR:2022,
%Liu:Nogueria:Fernandes:Kantarci:IEEECST:2022,
%Zhang:Sheng:Alhazmi:Li:ACMTIST:2020, Demetrio:Coull:Biggio:Lagorio:Armando:Roli:ACMTPS:2021} in Attacks and Defense. \cite{Sun:Dou:Yang:Zhang:Wang:Philip:He:Li:TKDE:2022} was the "one more survey" and is also in Attacks and Defenses.}

\vspace{0.02in}
\noindent
{\bf Attacks and Defense.}
Several works~\cite{Liu:Tantithamthavorn:Li:Liu:CSUR:2022,Liu:Nogueria:Fernandes:Kantarci:IEEECST:2022,Sun:Dou:Yang:Zhang:Wang:Philip:He:Li:TKDE:2022, Demetrio:Coull:Biggio:Lagorio:Armando:Roli:ACMTPS:2021} survey adversarial attacks and defenses, observing that most work focuses on computer vision and NLP domains. 
Zhang et al.~\cite{Zhang:Sheng:Alhazmi:Li:ACMTIST:2020}, 
Rosenberg et al.~\cite{Rosenberg:Shabtai:Elovici:Rokach:CSUR:2021},
Li et al.~\cite{Li:Li:Ye:Xu:CSUR:2021}, and 
Maiorca et al.~\cite{Maiorca:Biggio:Giorgio:CSUR:2019}, 
survey attacks and defenses in the NLP domain, cybersecurity domain for networks, Android malware, and PDF malware, respectively. 
These works identify a similar trend of new attacks constantly bypassing defenses, which gives rise to new defenses being proposed, only to be broken again (a.k.a. the `cat and mouse race' or the `arms race'). 
They also observe that research in this field studies attacks / defenses at a feature-level, which restricts 
the practicality of the developed techniques by the feasibility of perturbing the corresponding features in real life. 

%practical attacks are quite difficult and require some basic knowledge about the model or training data such as the feature set or model architecture. 
%Zhang et al.~\cite{Zhang:Sheng:Alhazmi:Li:ACMTIST:2020}, who study adversarial attacks and defenses in the NLP domain,  
%also find that there are obstacles to generating attacks in real-time. 
%For instance, methods that iteratively use gradients to create adversarial examples can be time-consuming, while one-time approaches may fail to produce potent adversarial examples.
%Several works~\cite{Liu:Tantithamthavorn:Li:Liu:CSUR:2022,Liu:Nogueria:Fernandes:Kantarci:IEEECST:2022,Sun:Dou:Yang:Zhang:Wang:Philip:He:Li:TKDE:2022, Demetrio:Coull:Biggio:Lagorio:Armando:Roli:ACMTPS:2021} 
%discuss how most new attacks and defenses are explored in computer vision and NLP, prior to other fields.


%our survey finds the state of the art w.r.t. data properties
%our survey finds that dimensionality is bad ...
%
%%%Here are the 5 remaining surveys + 1 additional paper for the reviewer.
%Numerous surveys have explored the landscape of adversarial evasion attacks and defenses. 
%For instance, Akhtar et al.~\cite{Akhtar:Mian:IEEEAccess:2018, Akhtar:Mian:Kardan:Shah:IEEEAccess:2021} survey the literature on adversarial robustness of deep learning models from Computer Vision field.
%They review popular attacks on visual models, and provided a categorization of existing defense techniques based on the components it modify in the visual model system \gx{Check}.
%
%Rosenberg et al.~\cite{Rosenberg:Shabtai:Elovici:Rokach:ACMComputingSurvey:2021}, Li et al. ~\cite{Li:Li:Ye:Xu:ACMComputingSurvey:2021} and Demetrio et al.~\cite{Demetrio:Coull:Biggio:Lagorio:Armando:Roli:ACMTPS:2021} review the literature on evasion attacks for cyber-security fields. 
%Li et al. proposed a partial order scheme to compare key attacks and defenses techniques for malware detection in Windows, Android, and PDF domains. 
%
%Zhang et al.~\cite{Zhang:Sheng:Alhazmi:Li:ACMTIST:2020} review the literature on adversarial attacks on deep-learning models for textual classification.
%They pointed out the intrinsic differences between Computer Vision and Natural Language Processing fields that pose challenges to directly apply attacks proposed for Visual models to NLP models and identified the strategies proposed that overcomes the barriers.
%The challenges they identified for creating realistic attacks in NLP fields are from a domain characteristics perspective (e.g., definition of imperceptible perturbations, measurement of the semantic changes),  we differ from them by trying to understand the adversarial robustness of machine learning from the characteristics of underlying data. 
%
%Attack and Defenses for wireless and Mobile systems~\cite{Liu:Nogueria:Fernandes:Kantarci:IEEECST:2022}
%
%

More recent research, not included in the surveys above, has also started investigating the 
susceptibility of newer models to adversarial evasion attacks. 
For example, several studies~\cite{Wang:Pan:Hu:Duan:Pan:IJSWIS:2022,Yin:Lin:Sun:Wei:Chen:TIFS:2023, 
Shi:Han:Tan:Kuang:NeurIPS:2022, Wang:Xie:Microsoft:ChatGPT:ArXiv:2023} proposed attack techniques against contemporary models, 
such as Graph Neural Networks, Generative Pre-training Transformers (GPT), and Vision Transformers. 
These studies showed that adversarial examples persist even for the newer models, some of which are 
trained with large volumes of data. 
As all these works focus on attack and defense mechanisms rather than 
the effects of data on adversarial robustness, our work extends and complements this research.
}

\revadd{
\vspace{0.02in}
\noindent
{\bf Adversarial Examples.}
%2 surveys lines 13 and 14 + 1 additional paper for the reviewer.
Adversarial examples are inputs constructed by perturbing a correctly classified sample in a way that makes the change imperceptible to a human. % but causes the model to misclassify the sample.
However, as `imperceptible to a human' is hard to define, existing research on adversarial examples approximates imperceptibility with a small perturbation measured through $L_p$ norms.
A line of research~\cite{Gilmer:Adams:Goodfellow:Anderson:Dahl:ArXiv:2018,Sharif:Bauer:Reiter:CVPRW:2018,Fezza:Bakhti:Hamidouche:Deforges:QoMEX:2019, Mezher:Deng:Karam:EUVIP:2022} 
investigates the validity of this assumption. 
This work shows that perturbations generated by $L_p$ norms do not entirely align with human perceptions, 
i.e., some changes with a small $L_p$ norm can be apparent to humans. 
In addition, adversarial examples with the minimum $L_p$ perturbation may be less effective and transferable than 
higher perturbation~\cite{Biggio:Roli:PR:2018,Rosenberg:Shabtai:Elovici:Rokach:CSUR:2021}. 
Hence, a number of approaches explore metrics for imperceptibility 
in computer vision and NLP domains~\cite{Fezza:Bakhti:Hamidouche:Deforges:QoMEX:2019,Mezher:Deng:Karam:EUVIP:2022, Zhang:Sheng:Alhazmi:Li:ACMTIST:2020}. 
Yet another issue with $L_p$ norms is that they cannot be used reliably in domains other than images. 
For example, in the case of software/malware, simply generating adversarial examples with $L_p$ norms 
may result in feature representations that are not possible in 
the problem space~\cite{Rosenberg:Shabtai:Elovici:Rokach:CSUR:2021,Pierazzi:Pendlebury:Cortellazz:Cavallaro:2020}. 

While all these works focus on the properties of adversarial examples, 
they are orthogonal to the topic of our survey, as we rather focus on how properties of the training data 
affect the success of adversarial examples.
}

%Gilmer et al.~\cite{Gilmer:Adams:Goodfellow:Anderson:Dahl:ArXiv:2018} argue that, while constraining the perturbations by sufficiently small $L_p$ norms can generate indistinguishable samples for most inputs, the actual imperceptibility of the changes depends on the input sample. 
%Several individual studies~\cite{Sharif:Bauer:Reiter:CVPRW:2018,Fezza:Bakhti:Hamidouche:Deforges:QoMEX:2019, Mezher:Deng:Karam:EUVIP:2022} find faults with using $L_p$ norms to generate adversarial examples. They show that the changes measured by $L_p$ norm, does not entirely align with human perceptions, i.e., some changes with a small $L_p$ norm appear apparent to humans. 
%In some domains adversarial examples do not need to be imperceptible but rather semantically preserving. 
%For example, in the case of Android malware~\cite{Rosenberg:Shabtai:Elovici:Rokach:CSUR:2021}, adversarial examples are small perturbations which fool a model while preserving the semantics of the sample, 
%i.e., a malware stays malicious even after the perturbation. 
%This highlights another problem with $L_p$ norm based adversarial examples as Dong et al.~\cite{Dong:Liu:Shang:NeurIPS:2022} show that the semantics of a sample change during adversarial training. 
%Hence, there is a need for metrics to measure the size of perturbations that is imperceptible or semantically preserving.
%Fezza et al.~\cite{Fezza:Bakhti:Hamidouche:Deforges:QoMEX:2019} and Mezher et al.~\cite{Mezher:Deng:Karam:EUVIP:2022} propose to use objective metrics for image quality to approximate the imperceptibility in the computer vision domain.
%Zhang et al.~\cite{Zhang:Sheng:Alhazmi:Li:ACMTIST:2020}, focusing on providing such a metric for Natural Language Processing.
%Vadillo et al.~\cite{Vadillo:Santana:CS:2022} also highlight conducted subject studies to evaluate the noticeability of audio adversarial examples.

%Even in computer vision, adversarial examples are not always imperceptible. For example, Machado et al.~\cite{Machado:Silva:Goldschmidt:CSUR:2021} find that visible perturbations such as adversarial patch~\cite{Brown:Mane:Roy:Abadi:Gilmer:ArXiv:2017}, and graffiti on stop signs~\cite{Eykholt:Evtimov:Fernandes:Li:Rahmati:Xiao:Prakash:Kohno:Song:CVPR:2018} are also considered adversarial examples in research.

%The aforementioned research examines the work on defining and creating adversarial examples, demonstrating the insufficiency of using conventional $L_p$ norms to evaluate the imperceptibility and semantics between clean and adversarial examples. 

\vspace{-0.1in}
\revadd{
\subsection{Non-Evasion Attacks}
\label{sec:relatedwork-poisoning}
Similar to evasion attacks, data poisoning and backdoor attacks aim to compromise model accuracy. 
However, they achieve it by tampering the training data to create deceptive model decision boundaries. 
%Data poisoning attacks involve modifying the training data to create deceptive decision boundaries, either to manipulate the prediction outcomes of a specific input or the entire model.
%Meanwhile, Backdoor attacks are a form of poisoning attacks where the attacker inject tempered training data with triggers 
% and then activates the attack by showing the trigger pattern at inference time.
In addition, backdoor attacks also require perturbing the test instance to result in a misclassification. 
This is achieved by introducing manipulated training data with triggers that can be activated during the testing phase.

Goldblum et al.~\cite{Goldblum:Tsipras:Xie:Chen:Schwarzchild:song:Madry:Li:Goldstein:TPAMI:2022} and Cinà et al.~\cite{Cina:Grosse:Demontis:Sebastiano:Zellinger:Moser:Oprea:Biggio:Pelillo:Roli:CSUR:2023} 
review recent literature on attack methodologies and countermeasures for both poisoning and backdoor attacks.
Both of these surveys found that existing research made overly-optimistic assumptions when designing / validating attack techniques, e.g., assuming the knowledge of a large portion of training data. 
They advocate for researchers to test proposed methods in more realistic situations to better assess the potential threats. 
Furthermore, they encourage exploration of the relationship between poisoning attacks and evasion attacks. 
This could lead to the creation of attacks that produce less noticeable poisoning examples, 
or defensive strategies that can safeguard models against both backdoor and evasion attacks.
%Their survey catalogs and systematizes the threats in the dataset creation process, and discuss the open problems that benefits the understanding of dataset security. 

In addition to undermining model accuracy, 
adversarial attacks also aim at breaching the privacy and confidentiality of training data. 
In particular, membership inference attacks~\cite{Shokri:Stronati:Song:Shmatikov:SP:2017} attempt to determine whether a specific data point was part of the training set used to train the model.
Hu et al.~\cite{Hu:Salcic:Sun:Dobbie:Yu:Zhang:CSUR:2022} present a comprehensive survey of existing research efforts on membership inference attacks. 
They find that, similar to evasion attacks, the membership inference attack success rate decreases as 
%the training data better represents the whole data distribution, i.e., 
the number of training samples increases.
%and model stealing attacks~\cite{Oliynyk:Mayer:Rauber:CSUR:2023} are designed to breach the privacy of training data and machine learning models. 
However, all these attacks are orthogonal to our survey, as we focus on adversarial evasion attacks.

%Li et al. ~\cite{Li:Jiang:Li:Xia:TNNLS:2022} 
%provide the first survey that focuses on backdoor attacks and identified common scenarios in which backdoor attack happen in real life. 
%Furthermore, they proposed a systematic taxonomy for backdoor attacks and defenses for researchers and practitioners to identify the characteristics and limitations of each method. 

%Wang et al.~\cite{Wang:Ma:Wang:Hu:Qin:Ren:CSUR:2022} and Tian et al.~\cite{Tian:Cui:Liang:Yu:CSUR:2022} argue federated learning~\cite{McMahan:Moore:Ramage:Hampson:Arcas:AISTATS:2017} 
%creates new venue for poisoning attack, and survey recent literature on poisoning attacks for both standard and federated learning scenarios. 
%They present a unified framework to categorize both data poisoning and model poisoning attacks, and compared the defense techniques proposed for each of the learning framework, analyzed their advantages and disadvantages.
}

\vspace{-0.1in}
\subsection{Effects of Training Data on Standard Generalization}
\label{sec:relatedwork-standard}
A number of surveys investigate the influence of data properties on standard
rather than robust generalization.
One of the earliest is probably the work of Raudys and Jain~\cite{Raudys:Jain:TPAMI:1991},
who review studies related to the influence of sample size on binary classifiers, showing that
a limited sample size usually leads to sub-optimal generalization.
%With the development of deep learning and the ever-increasing need for larger training datasets,
%a variety of data augmentation techniques have been proposed.
Bansal et al.~\cite{Bansal:Sharma:Kathuria:CSUR:2021} and
Bayer et al.~\cite{Bayer:Kaufhold:Reuter:CSUR:2022} also survey papers addressing the data scarcity problem,
focusing in particular on the recent advancements in data augmentation techniques in the fields of computer vision, security, and text classification.
Their results show that augmentation techniques %exist for various application domain and
can help improve a model's generalization by reducing the problem of model overfitting.
%They evaluate the effectiveness of such techniques in improving the accuracy of machine learning models.

%Limited sample size is also one of the culprit behind poor robust generalization~\cite{Schmidt:Santurkar:Tsipras:Talwar:Madry:NeurIPS:2018}, we collected a number of researches characterize the sample complexity for robust generalization or propose data augmentation techniques to fill in the sample complexity gap.

Label noise is another aspect of data that influences both standard and robust generalization.
Most works on this topic find that the presence of noisy labels increases the need for a greater number of training samples and may result in unnecessarily complex decision boundaries~\cite{Frenay:Verleysen:TNNLS:2014,Song:Kim:Park:Shin:Lee:TNNLS:2022}.
For example, Fr\'{e}nay and Verleysen~\cite{Frenay:Verleysen:TNNLS:2014} show
that overfitting to label noise greatly degrades a model's standard generalization;
the same effect has been observed in the case of robust generalization~\cite{Sanyal:Dokania:Kanade:Torr:ICLR:2021}.
Song et al.~\cite{Song:Kim:Park:Shin:Lee:TNNLS:2022} survey the impact of label noise in deep learning, arguing
that the presence of noisy labels is a more serious concern for deep models as they contain a larger number of parameters which makes them prone to overfitting to the noise in training data.
%They also point out the connection between adversarial poisoning attacks and noisy labels as
%the countermeasures for both share the goal of learning noise-resilient representations.
They mention that adversarial defense techniques, e.g., adversarial training, are effective against label noise~\cite{Zhu:Zhang:Han:Liu:Niu:Yang:Kankanhalli:Sugiyama:ArXiv:2021, Fatras:Damodaran:Lobry:Flamary:Tuia:Courty:TPAMI:2022}
but do not discuss how label noise influences a deep learning model's robustness under attacks.

Lorena et al.~\cite{Lorena:Garcia:Lehmann:Souto:Ho:CSUR:2020} identify a collection of 26 quantitative metrics that measure data complexity with respect to
(1) ambiguity of classes, i.e., whether the classes can be clearly distinguished with the given features,
(2) sparsity and dimensionality of data, 
%i.e., whether enough information are provided to learn confident decision boundaries, and
(3) complexity of boundary separating the classes, i.e., whether more intricate functions are required to describe the decision boundaries.
The authors also discuss how these metrics help estimate the difficulty of performing classification on a given dataset.
Similar to our survey, the authors show that high dimensionality and small separation between classes hinder standard generalization.
However, the relationship of some of the metrics reviewed by these authors, e.g.,
%faction of borderline points (i.e., a measure for the complexity of the required decision boundary) and
%the fraction of hyperspheres covering data (i.e.,
the number of non-intersecting spheres needed to enclose all data points of a class,
to robust generalization is not studied, according to our survey.

%Moreover, the effect of XXX on standard generalization needs future investigation as well (that is if we found something they do not have).

%Knowing the characteristics of a dataset according to these perspectives can assist researchers and practitioners to select optimal learning algorithms~\cite{Ho:Basu:TPAMI:2002}.

He and Garcia~\cite{He:Garcia:TKDE:2009} focus on the imbalance learning problem. %~--
%the disproportion in the number of samples belonging to each class in a given dataset.
The authors found that most standard algorithms %are designed with the assumption of a balanced class distribution.
%These algorithms
fail to reliably represent the characteristics of the imbalanced data and result in unfavorable performance across classes.
Furthermore, L\'{o}pez et al.~\cite{Lopez:Fernandez:Garcia:Palade:Herrera:InfSci:2013} discuss six intrinsic data characteristics that potentially complicate learning from imbalanced data:
low density, sample overlap between classes, noisy data, borderline instances,
dataset shift between training and testing distributions, and
small disjuncts, i.e., disperse small clusters of samples from a single class.
Their analysis concludes that while all these ``unfavorable'' data characteristics further complicate the data imbalance
issues, data overlap between classes is probably one of the most harmful.
To follow up on this point, Santos et al.~\cite{Santos:Henriques:Pedro:Japkowicz:Fernandez:Soares:Wilk:Santos:AIR:2022}
focus on the joint effect of data imbalance and class overlap on model generalization.
The negative impact of data imbalance, low separation, and noisy data on robust generalization was also discussed in our survey.
Yet, the compounding effect of these factors, as well as the effect of other properties,
on robust generalization needs future investigation.

Recently, Yang et al.~\cite{Yang:Jiang:Song:Guo:IJCV:2022} summarized relevant studies focusing on
long-tailed distributions in the field of Computer Vision.
% and categorize the main methods for alleviating the issues caused by long-tailed distribution.
%They present quantitative metrics for measuring data imbalance and .
This survey also includes work on the influence of long-tail distributions on a model's adversarial robustness~\cite{Wu:Liu:Huang:Wang:Lin:CVPR:2021}, which is covered in our survey.
%which is included in our survey,
The authors advocate for more research on adapting long-tailed-based approaches for standard generalization to improve robust generalization.

Finally, Moreno-Torres et al.~\cite{MorenoTorres:Raeder:Rodrigues:Chawla:Herrera:PR:2012} present a unifying framework to categorize existing definitions of dataset shift~-- the case where the joint distribution of inputs and outputs differs between training and testing data.
While ML models are normally trained under the premise that testing data has a similar distribution to the training data,
in reality, the observed data distribution may be different from the historical data that the model is trained on.
Such difference can substantially compromise the quality of model predictions.
The authors analyze the possible causes for dataset shift, e.g., malicious software that evolves over time, and
review the techniques dealing with dataset shift.
They characterize adversarial attacks as one form of dataset shift, where adversaries adaptively
change test instances to create a distribution that differs from training data.
%All works discussed in our survey assumed similar distribution on training and testing data, treating adversarial attacks as the only dataset shift in the problem setup.
%However, in real applications, the underlying data distribution itself can be non-stationary, and the characterize the influence of the dataset shift between training and testing data on the adversarial robustness is yet to be investigated.

\revadd{Overall, despite the similarities with our work, literature discussed in this section focuses on standard generalization while our survey discusses 
the effect of data on robust generalization.}

%More works use the connection between adversarial attacks and distributional shift to analyze the effect of adversaries on generalization performance~\cite{Tu:Zhang:Tao:NeurIPS:2019}.
%However, we do not discuss them in detail, as they focus more on models instead of data.
%\jr{How is that relevant to data properties section?} \gx{This can be removed, as it an individual work we filtered}

\vspace{-0.1in}
\subsection{Summary}
\revadd{
Our survey is the first to explicitly focus on properties of training data in the context of model robustness under evasion attacks.
Numerous other surveys on evasion attacks discuss attack and defense mechanisms, non-data-related reasons for adversarial vulnerability, and the different threat models. 
We identified only five surveys that considered data-related reasons for evasion attacks. 
However, as these surveys are older and do not focus on data in particular, our work provides a more extensive
and comprehensive view on this topic. 
By including more than 50 papers not covered in prior work, we were able to 
identify additional relevant properties, practical suggestions, and future research directions in this area. 

Additional work studies non-data-related reasons for evasion attacks, as well as non-evasion attacks, 
such as poisoning and backdoor. 
Yet another body of literature examines how data properties affect standard generalization. These works show that 
some of the properties discussed in our survey, such as 
the number of samples, dimensionality, and label quality, also affect clean accuracy. 
There are also additional data properties that are covered exclusively by these or by our work. 
Studying the interplay between data properties for clean and robust accuracy is an interesting research direction, 
which could be facilitated by our work. 
However, all these current works are orthogonal and complementary to ours.
}

%\ad{
%The related work of our survey can be categorized into four key topics: 
%The first topic examines data for other adversarial attacks, this include the research that investigates the link between the data characteristics and model's resilience against poisoning attacks as well as the studies that explore data poisoning and backdoor attacks and their countermeasures. \jr{same issues as before: this is meta-summary, we need a concrete summary.}
%These studies complement our survey as they highlight the threats directly aimed at data, thus emphasizing the importance of secure data collection. 
%The second topic focuses on the relationship between various properties of training data and model's standard generalization ability. 
%This body of work suggests that data traits such as number of samples, dimensionality, label quality also influence model's ability to generalize in standard classification. \jr{this looks more concrete!}
%
%The third strand of research concerns adversarial evasion attacks. 
%The work in this area encompasses the research frontier in evasion attacks and the countermeasures. 
%Due to the large volume of work in this area, there are numerous surveys that gives more detail on the advancement. 
%\jr{meta-summary again}
%In addition to attacks and defenses, one relevant line of work investigates the alignment of the conventional similarity metrics used for adversarial examples and human perception, showing the need for supplementary metrics. \jr{why important?}
%These studies \jr{which "these studies"?} collectively present an extensive overview of other types of work conducted on adversarial robustness.
%The last category of work proposes alternative explanations for model vulnerability to adversarial examples.
%These studies presented hypothesis showing the characteristics of machine learning models, e.g., nonlinearity, invariance to rotational shift etc, induces susceptibility to attacks, as well as limited computational resources and non-robust feature representations. \jr{all text based on previous related work looks somewhat concrete; the new additions should be at least at the same level, or better.}
%These studies supplement our work, offering a broader perspective of potential factors affecting model's robust generalization ability. }
%


\section{Problem Definition} In this section, we first define the traffic network graph and its expansion to the temporal dimension. Then, we quantify traffic incident impact with two measurements -- impact duration and impact length. Finally, we formulate the problem of incident impact prediction.

\subsection{Traffic Graph} A traffic performance measurement system usually sets static traffic loop sensors on the side of arterial roads and collects traffic data near the sensors. Previous research has tended to link groups of sensors that are closest to each other in the geographic or feature space. However, we argue that this assumption does not hold for freeways. For example, if two sensors are recording traffic in opposite directions on the same freeway, they could be as close to each other as the width of the road. However, these two sensors should not be linked, as a ``U'' turn is not an option on freeways. In addition, two sensors on different roads could also be geographically close at the intersection of two roads. In such cases, the real condition is that cars must drive through a long ramp to switch to another road, which means a much longer distance than the Euclidian distance. 

Considering that the relations are hard to quantify for two sensors on
different roads, we suggest constructing a dual-level graph to represent a
traffic network. The first level is the ``sensor-to-road'' level, which links sensors to the roads they are on. The second level (``road-to-road'' level) links two roads if they intersect. Note that two
directions on one freeway are considered two roads.
\begin{Definition}{\rm Dual-level traffic network graph.} Consider a traffic
 network graph as $G$, where $G=(S, R, E^{sr}, E^{rr}, E^{rs}, A^{sr}, A^
 {rr}, A^{rs})$. $S$ is the sensor node set of size $|S|$. $R$ is the road
 node set of size $|R|$. $E^{sr}, E^{rr}, E^{rs}$ are the edges linking
 sensor-road, road-road, and road-sensor, respectively. $A^{sr} \in \mathbb
 {R}^{|S|\times |R|}, A^{rr} \in \mathbb{R}^{|R|\times |R|}, A^
 {rs} \in \mathbb{R}^{|R|\times |S|}$ are the adjacent matrix form of $E^
 {sr}, E^{rr}, E^{rs}$.
\end{Definition}

Assume that $T$ timestamps around the incident validation time are used for prediction because they are usually most relevant to the incident impact. The traffic behaviors before and after an incident can be quite different. In this case, we split the $T$ timestamps into ``before validation'' (containing $T_{bv}$ timestamps) and ``after-validation'' (containing $T_{av}$ timestamps). 
\begin{Definition}{\rm Dynamic traffic network graph.} A dynamic traffic network graph can be denoted as $\mathcal{G}=\{\mathcal{G}_{bv}, \mathcal{G}_
 {av}\}=\{G_0, ..., G_{T_{bv}-1}, G_{T_{bv}}, ..., G_{T_{bv}+T_
 {av}-1}\}$. $G_0,...,G_{T_{bv}+T_
 {av}-1}$ indicates graphs at timestamp $0,...,T_{bv}+T_
 {av}-1$. $\mathcal{G}_{bv}$ and $\mathcal{G}_{av}$ are dynamic graphs before and after the validation time, respectively. 
\end{Definition}

\subsection{Incident Impact Prediction} The impact of an incident can be quantified into two dimensions: the spatial dimension and the temporal dimension. We represent the temporal impact with $\mathbf{Y}_{Dur}$, which is the difference between the validation time and the restoration time. This definition also matches the formal definition of incident duration provided by the Department of Transportation (DOT). For the spatial dimension, traditional transportation research counts the number of blocked cars upstream of the incident. However, cars’ speed can be affected by congestion even if they are not blocked. To account for this, we define the impact length $\mathbf{Y}_{Len}$ as the maximum continuous congestion road distance in the immediate upstream of the incident.
\begin{Definition}{\rm Incident impact precision.} Given a dynamic traffic
 network graph for an incident $\mathcal{G}$ and corresponding sensor feature
 tensor $\mathbf{X} \in \mathbb{R}^{|S|\times T \times C_{in}}$, the aim is
 to find a model $\mathcal{F}$ so that 
  \begin{equation}
 \mathcal{F}:(\mathcal{G}, \mathbf{X}) \rightarrow (\mathbf{Y}_{Dur}, \mathbf{Y}_{Len})
 \end{equation}
 where $C_{in}$ is the number of input channels recorded by the sensors, $Y_{Dur}$ is the impact duration, and $Y_{Len}$ is the impact length.
\end{Definition}
\section{Methodology}
\label{sec:method} 
As shown in Figure~\ref{fig:architecture}, the proposed architecture contains three main parts. The dual-level spatial transformer fuses spatial information among sensors for each timestamp. The importance score temporal transformer combines temporal information for each sensor, giving a higher ``importance score'' to sensors experiencing sudden feature changes after the validation time. A pooling module projects the learned dynamic graph representation to the expected output.

% modify the figure as the model changes
\begin{figure*}[h]
  \centering
  \includegraphics[width=\linewidth]{figures/architecture4.jpg}
  \caption{\textbf{The architecture of DG-Trans.} The blue rectangle labeled \textbf{A. Dual-Level Spatial Transformer} first encodes the input tensor by performing ``sensor-to-road,''  ``road-to-road,'' and ``road-to-sensor'' attentions. The orange rectangles labeled \textbf{B. Importance Score Transformer} further process the output of A $\mathbf{H}^{ST}$ with three modules.  $\mathbf{H}^{ST}$ is split into $\mathbf{H}_{bv}$ and $\mathbf{H}_{av}$ by the validation time of the incident, then fed to the T-Transformer with importance scores initialized as 0. The outputs $\mathbf{H}$ and $\mathbf{H}_{av}$ are sent to Decoder \#1 to reconstruct $\mathbf{X}_{av}$ and compute the importance score as $\mathbf{H}-\mathbf{H}'_{av}$. With the updated importance score, $\mathbf{H}^{ST}$ is fed to T-Transformer again. The output is further
   processed by Decoder \#2 to become $\mathbf{H}''_{av}$. The green rectangle labeled \textbf{C. Pooling} shows how the latent features are projected to the desired outputs. Finally, the loss is computed as the weighted combination of (1) the reconstruction from $\mathbf{H}'_{av}$ to $\mathbf{X}_{av}$, (2) the reconstruction from $\mathbf{H}''_{av}$ to $\mathbf{X}_{av}$, and (3) the prediction loss of impact duration and length.} % describe the architecture
  \label{fig:architecture}
\end{figure*}

\subsection{Dual-Level Spatial Transformer} The design of the dual-level transformer is inspired by the concepts of anchored graphs and hypergraphs. Anchors have been used to reduce the complexity of attention in many studies~\cite
 {chen2020iterative, baek2021accurate, mialon2021graphit}. In our case, roads are considered anchor nodes for the sensors on them. While using anchors, the rank of the adjacency matrix decreases from $|S|$ to $|R|$. This means that multi-hop message-passing only matters among roads. The model maintains a global latent feature tensor $\mathbf{H}^{r} \in \mathbb{R}^{|R|\times T \times C}$ for each road at each timestamp, which does not vary by case. For each incident case, the sensor features are first used to update the case-irrelevant road feature in order to acquire case-relevant road features $\mathbf{H}^{sr}$. Then, the road features are further updated to $\mathbf{H}^{rr}$ by message-passing among roads. At this stage, the output road features can be considered both an intermediate pooling of the sensor features and a spatial feature fusion of the sensors. 
 % Therefore, $\mathbf{H}^{rr}$ is used for both impact prediction and sensor feature reconstruction. The second task is considered as a regularization of the first task so that the $\mathbf{H}^{rr}$ does not overfit the impact results or diverge from the raw representation of the sensor network.
 Finally, the road features are propagated back to the sensors for the next step.
 
 Denote sensors as $S$ and roads as $R$. A vanilla self-attention exploring spatial relations between sensors can be summarized as follows:
\begin{equation}
  a^{ss}_{ij} = \sigma(\frac{(\mathbf{Q} \mathbf{h}_{s_j})^T(\mathbf{K}\mathbf{h}_{s_i})}{\sqrt{d}})
\end{equation} 
where $s_i$ and $s_j$ are two different sensor nodes, $\mathbf{Q}$ and $\mathbf{K}$ are query and key projection parameters, $\mathbf{h}_{s_i}$ and $\mathbf{h}_{s_j}$ are embeddings of $s_i$ and $s_j$, $d=C \div n\_head$ is the dimension of each attention head, and $\sigma$ is the row-wise softmax activation function. 

In contrast to the vanilla approach, in S-Transformer, the attentive message-passing is accomplished with three transformer layers. The first transformer layer contains a ``sensor-to-road'' attention layer and a linear layer. The ``sensor-to-road'' attention computes the correlation
between sensors and roads and masks out the edge between $s_i$ and $r_j$ if
$s_i$ is not on $r_j$:
\begin{equation}
\label{eq:sr_attn}
\begin{aligned}
  &a^{sr}_{ij} = \sigma(mask_{sr}(\frac{(\mathbf{Q}^{sr}\mathbf{h}_{r_j})(\mathbf{K}^{sr}\mathbf{h}_{s_i})^T}{\sqrt{d}},m^{sr}_{ij})),\\
  &mask(\mathbf{x}, \lambda)=\begin{cases} \mathbf{x} & \lambda = 1 \\
                     -\infty &  \lambda = 0
       \end{cases}
\end{aligned}
\end{equation}
where $s_i$ and $r_j$ represent an arbitrary sensor and a road. $\mathbf{Q}^{sr}$ and
$\mathbf{K}^{sr}$ are query and key projection parameters for $r_j$ and $s_i$. $\mathbf{h}_{s_i}$ and $\mathbf{h}_{r_j}$ are embeddings of $s_i$ and $r_j$. $\mathbf{h}_{r_j} \in \mathbb{R}^{T\times C}$ is a
learnable parameter matrix. $\mathbf{M}^{sr} \in \{0, 1\}^{|S|\times|R|}$ is the
adjacent matrix between sensors and roads. $m^{sr}_{ij}=1$ if sensor $s_i$ is
on road $r_j$, otherwise $m^{sr}_{ij}=0$. $\mathbf{M}^{sr}$ performs as the mask of all the attention heads. With $a^{sr}_{ij}$, the road embedding can be updated as $\mathbf{h}_{r_j}^{sr}=\mathbf{W}^{sr}\sum_{i=1}^{|S|}a^{sr}_{ij}(\mathbf{V}^{sr}\mathbf{h}_{r_j})+\mathbf{b}^{sr}$.

The attention in the second transformer is a ``road-to-road'' self-attention layer intended to
extract road intersection information. The attention can be expressed as
follows:
\begin{equation}
  a^{rr}_{ij} = \sigma(mask_{rr}(\frac{(\mathbf{Q}^{rr}\mathbf{h}_{r_j}^{sr})(\mathbf{K}^{rr}\mathbf{h}_{r_i}^{sr})^T}{\sqrt{d}},\mathbf{M}^{rr}_{ij}))
\end{equation}
where $\mathbf{M}^{rr} \in \{0, 1\}^{|R|\times|R|}$ represents four levels of adjacency between
roads: $m^{rr}_{ij}=1$ if (1) $i=j$, (2) $r_i$ intersects $r_j$, (3) $r_i$ intersects $r_k$ and $r_j$ intersects $r_k$, and (4) fully connected. The four different masks can be applied to different attention heads. In our design, each of the four masks was applied to one attention head. A spatial-relation awared road feature tensor is then computed as $\mathbf{h}_{r_j}^{rr}=\mathbf{W}^{rr}\sum_{i=1}^{|R|}a^{rr}_{ij}(\mathbf{V}^{rr}\mathbf{h}_{r_j}^{sr})+\mathbf{b}^{rr}$.

The last transformer contains a ``road-to-sensor'' attention, meaning that sensor vectors are queries, and road vectors are keys and values. The output of this step propagates the aggregated road features to the sensors. 
% is used to examine if $\mathbf{H}^{rr}$ can represent the raw sensor network. 
The equation is essentially the reversed version of Equation~\ref
{eq:sr_attn}:
\begin{equation}
  \mathbf{H}^{rs} = \sigma(\frac{(\mathbf{Q}^{rs}\mathbf{H}^{rr})(\mathbf{K}^{rs}\mathbf{H}^{rr})^T}{\sqrt{d}})(\mathbf{V}^{rs}\mathbf{H}^{rr})
\end{equation}

Note that there is no mask in this step, which preserves the attention
mechanism's flexibility. Previous works have usually linked the top-k closest sensors in
the geographic or feature space, weighting the links with the distances.
However, we observe that attention is good at learning weights but weak at
learning graph structures. In this case, we simply need to find nodes that
are definitely linked to each other and leave the weight learning task to
attention. Therefore, we chose to partially control the graph structure with unweighted adjacent matrices. Finally, we applied skip-connection, layer-normalization, and dropout to $\mathbf{H}^{rs}$.

Essentially, S-Transformer stresses the effects of sensors on
high-degree-centrality roads. The strengths of the S-Transformer can be
summarized as follows: (1) it avoids the human error introduced by manually
choosing ``road-to-road'' for ``top-k'', (2) it allows long-range message-passing as all sensors on intersected roads are linked, (3) it preserves the flexibility of attention with a relatively small number of edges
($|S|+|R|^2+|S||R|, |R|\ll |S|$ at most) (4) it is more time efficient ($\mathcal{O}(|S||R|^2+|R|^3+|R|^2|S|), |R|\ll |S|$) than traditional attention mechanisms ($\mathcal{O}(|S|^3)$) and requires fewer layers as the spatial message-passing is performed sufficiently by the ``road-to-road'' self-attention module.

\subsection{Importance Score Transformer} As an incident usually affects only
 a relatively small part of the whole traffic network, treating all sensors equally for prediction may introduce unwanted noise. However, manually extracting sensors near the incident may cause the loss of long-range complex impact
 patterns and even the most representative features due to the early/delayed response of the traffic network. In this case, a method that dynamically locates the region and time window affected by the incident is important. Based on the assumption that the traffic measurements of sensors affected by incidents show more obvious changes than other sensors, we locate the incidents with anomaly detection techniques (i.e., assigning sensors with larger variance a higher ``importance score''). 

Inspired by~\cite{tuli2022tranad}, our importance score
transformer contains three modules: a temporal transformer
(T-Transformer) and two decoders. The T-Transformer module encodes the output of the S-Transformer with and without the importance score along the time dimension. The first decoder computes the importance score and reconstructs
$\mathbf{X}_{av}$, while the second reconstructs $\mathbf{X}_{av}$ from the combination of the importance score and the graph embedding. Denote the ``after-validation'' section of $\mathbf{H}^{ST}$ as $\mathbf{H}_{av}$ and the ``before-validation'' section of $\mathbf{H}^{ST}$ as $\mathbf{H}_{bv}$. The combination of the T-Transformer and any one of the decoders is equivalent to a classic transformer network when considering $\mathbf{H}^{ST}$ as the input sequence and $\mathbf{H}_{av}$ as the output sequence.

Assume the importance score is $\mathbf{I} \in \mathbb{R}^
{|S|\times T \times C}$ and the output of S-Transformer as $\mathbf{H}^{ST} \in \mathbb{R}^{|S|\times T \times C}$. The output of the T-Transformer can be written as follows:
\begin{equation}\label{eq:score_1}
  \mathbf{H} = \text{TTrans}([\mathbf{H}^{ST} || \mathbf{I}_0])
\end{equation}
where $\text{TTrans}()$ indicates T-Transformer, which is a block sequentially performing temporal self-attention and skip-connection. $\mathbf{I}_0$ is the initialized importance score (which is an all-zero tensor).

The task for both Decoder \#1 and Decoder \#2 is to reconstruct $\mathbf{X}_{av}$. Each decoder has a self-attention layer and a mutual-attention layer. 
Decoder \#1 attempts to achieve its goal using $\mathbf{H}$ and $\mathbf{H}_{av}$:
\begin{equation}
  \mathbf{H}'_{av} = \text{mu-attn}_1(\mathbf{H}, \mathbf{H}, \text{self-attn}_1(\mathbf{H}_{av}))
\end{equation}

The three parameters in $\text{mu-attn}_1$ are placeholders for value, key, and query.

Then, the importance score is updated as $\mathbf{I} = \mathbf{H}^{ST} - \mathbf{H}'_{av}$. Note that $\mathbf{H}^{ST}$ has $T$ timestamps while $\mathbf{H}'_{av}$ has
$T_{av}$ timestamps. We examined various methods, such as repeating timestamps in $\mathbf{H}'_{av}$ and getting mean/min/max of $\mathbf{H}'_{av}$ along the time axis. All the methods resulted in similar performance. Accordingly, we
adopt the general form to represent the difference between $\mathbf{H}^{ST}$ and $\mathbf{H}'_{av}$. 
% We directly use the difference between $\mathbf{H}^{ST}$ and $\mathbf{H}'_
% {av}$ as we want to distinguish the traffic patterns before and after the
% incident occurrence/ Intuitively, for affected sensors, the average speed before the incident is larger than $avg(\mathbf{H}_{bv}) \geq \mathbf{H}'_{av}$ and
% $\mathbf{H}_{av} \leq \mathbf{H}'_{av}$ based on the intuition of anomaly
% detection. 
Replacing $\mathbf{I}_0$ with $\mathbf{I}$, we apply the T-Transformer and Decoder \#2 the same way as the previous steps, with the concatenated $\mathbf{H}^{ST}$ and $\mathbf{I}$ as the input. 

\begin{equation}\label{eq:score_2}
\begin{aligned}
  &\mathbf{H}' = \text{TTrans}([\mathbf{H}^{ST} ||\mathbf{I}])\\
  &\mathbf{H}''_{av} = \text{mu-attn}_2(\mathbf{H}', \mathbf{H}', \text{self-attn}_2(\mathbf{H}_{av}))
\end{aligned}
\end{equation}

Finally, $\mathbf{H}'_{av}$ and $\mathbf{H}''_{av}$ are used to reconstruct $\mathbf{X}_{av}$ separately with the same two-layer feed-forward network (FFN):
\begin{equation} \label{eq:ffn}
\begin{aligned}
  &\mathbf{X}'_{av} = \mathbf{W}_{1,2}(\phi( \mathbf{W}_{1,1}\mathbf{H}'_{av}+ \mathbf{b}_{1,1}))+ \mathbf{b}_{1,2} \\
  &\mathbf{X}''_{av} = \mathbf{W}_{1,2}(\phi( \mathbf{W}_{1,1}\mathbf{H}''_{av}+ \mathbf{b}_{1,1}))+ \mathbf{b}_{1,2}
\end{aligned}
\end{equation}
where $\mathbf{W}_{1,1}, \mathbf{b}_{1,1}, \mathbf{W}_{1,2}, \mathbf{b}_{1,2}$ are parameters of the two linear layers and $\phi$ is the activation function, which is the LeakyReLu in the FFN in Equation~\ref{eq:ffn}. 

\subsection{Pooling and Loss} After the spatial and temporal encoding, the spatiotemporal representation $\mathbf{H}'$ is considered well-learned and ready for prediction. To further refine the features, the importance score is used to weight the elements in $\mathbf{H}'$ through element-wise multiplication. This is followed by a temporal dimension aggregation through a 2D convolution layer and a spatial dimension aggregation through SUMPooling. Lastly, three linear layers with LeakyReLU activations are used to project the features into the desired dimension. This process can be summarized as follows:
\begin{equation}
\begin{aligned}
  &\mathbf{H}'_T = \text{Conv2d}(\mathbf{H}'), \mathbf{H}'_T \in \mathbb{R}^{|S|\times C}\\
  &\mathbf{H}'_S = \text{SUMPool}(\mathbf{H}'_T), \mathbf{H}'_T \in \mathbb{R}^C)\\
  &\hat{Y} = \mathbf{W}_{p3}\phi(\mathbf{W}_{p2}\phi(\mathbf{W}_{p1}\mathbf{H}'_S+\mathbf{b}_{p1})+\mathbf{b}_{p2}))+\mathbf{b}_{p3}
\end{aligned}
\end{equation}
where $\hat{Y}\in \mathbb{R}^{C_{out}}, C_{out}=2$. The first value represents the impact duration, and the second is the impact length. SUMPooling is chosen because it is the most expressive pooling method~\cite{xu2018powerful}, and we want the pooled representation to be capable of representing all possible incident scenarios. L1 loss is chosen as the prediction loss function so that the model focuses on the overall trends of $\mathbf{Y}$ instead of outliers. The prediction loss can then be written as $Loss_1 = |\mathbf{\hat{Y}}_{Dur}-\mathbf{Y}_{Dur}|+|\mathbf{\hat{Y}}_{Len}-\mathbf{Y}_{Len}|$.

The second part of the loss is the reconstruction loss. This loss is for regularization purposes and is self-supervised. The only objective of the model is to predict $\mathbf{Y}_{Dur}$ and $\mathbf{Y}_{Len}$. L1 loss is employed for both $\mathbf{X}'_{av}$ and $\mathbf{X}''_{av}$, i.e., the loss functions are $Loss_2 = |\mathbf{X}'_{av}-\mathbf{X}_{av}|$ and $Loss_3 = |\mathbf{X}''_{av}-\mathbf{X}_{av}|$.
To generate the importance scores correctly, the Importance Score Transformer is trained in an adversary way. Here, we explain our design in terms of a GAN in order to make it understandable. Consider $\mathbf{H}_{av}$ as the true "image", $\mathbf{H}_{bv}$ as the fake "image", Decoder \#1 as the discriminator, and Decoder \#2 as the generator. In the first several epochs, the weight of $Loss_2$ is far larger than the weight of $Loss_3$. As a result, Decoder \#1 discriminates $\mathbf{H}_{bv}$ and $\mathbf{H}_{av}$ better by assigning $\mathbf{H}_{av}$ larger attentive weights in $\text{TTrans}()$. As the weight of $Loss_3$ increases, Decoder \#2 is trained to "generate" a new $\mathbf{H}_{bv}$ by concatenating it with $\mathbf{I}_{bv} \approx \mathbf{H}_{bv}-\mathbf{H}_{av}$. The new $\mathbf{H}_{bv}$ -- $\mathbf{H}'_{bv}$ recieves more attention in $\text{TTrans}()$ with the importance score and thus enlarge $Loss_2$. This way, as the weight of $Loss_3$ grows, the attentive weights in $\text{TTrans}()$ finally stabilize at some point that slightly inclines to $\mathbf{H}_{bv}$, which makes Decoder \#1 produce a larger importance score for $\mathbf{H}_{av}$ and a smaller score for $\mathbf{H}_{bv}$.


The loss during training can be written as follows:
\begin{equation} \label{eq:loss}
  Loss = \psi Loss_1 + \omega Loss_2 + (1-\omega) Loss_3
\end{equation} 
where $\psi$ is the weight of the prediction loss. $\omega \in (0, 1]$ is the weight of the reconstruction loss of
Decoder \#1, while $(1-\omega)$ is the weight of the reconstruction loss of Decoder \#2. $\omega$ is initialized as $1$ and decreases as the number of epochs increases. 

\section{Experiment}
\label{sec:experiment}
In this section, we first introduce our datasets and our method for data
cleansing and labeling. Then, we compare DG-Trans with state-of-the-art
baselines and evaluate our proposed model from three perspectives: prediction
precision, module-level effectiveness, and execution efficiency of the
spatial attention module. Next, we perform a case study to examine whether DG-Trans can correctly extract spatiotemporal features. Finally, we propose several basic methods of incorporating new data into DG-Trans. These methods may help to broaden the problem domain of traffic incident impact prediction.

\subsection{Dataset} 
The traffic loop sensor data used for this research are collected from the Caltrans Performance Measurement System (PeMS)\footnote{\url{https://pems.dot.ca.gov/}}; The incident record data are from RITIS \footnote{\url{https://www.ritis.org/}}; The road networks are
downloaded from Tiger Priscroads \footnote{\url
{https://www2.census.gov/geo/tiger/}}. Based on the location of sensors and incidents, we selected several freeway segments in Los Angeles and San Bernardino regions, which have well-constructed and complex road networks. 

In addition to the data used in this research (i.e., sensor measurements, adjacency information, and incident impact records), we also provided auxiliary data to broaden the potential problem domain. Specifically, our datasets
contain three basic elements: roads, sensors, and incidents. For roads, we provide a matrix indicating whether two roads intersect each other and the location of the intersections (in the form of the distance from the start point of the road). For sensors, we collect their positions on the roads and
their five-minute speed and occupancy measurements. As the rate of missing records is less than 0.005\%, we simply filled those values with daily average
speed/occupancy. Finally, for incidents, we provide the position on the road,
the DateTime (number of five minutes from 2019/09/01 00:00:00), the incident
category, the impact duration, and the impact length. 

Since we lacked the ground truth of impact length, we acquired the label with three
steps. We first extracted the regular weekly traffic pattern by averaging the
speeds from 2019/08/01 to 2019/10/31. In this case, a speed lower than the
regular pattern is considered an indicator of non-recurrent congestion. The second step is to split the impacted sensor measurements from the others. We performed a binary 1D k-means classification to extract low speed measurements, then looked for the closest upstream sensor that detected no
congestion during the incident clearance process. If the selected sensor is the closest sensor to the incident, the impact length is zero. Otherwise, the impacted length is defined by the distance from the incident to the upstream sensor before the picked sensor.

To construct properly scaled datasets, we selected several inter-state freeways as illustrated in Figure~\ref{fig:p7_road} and~\ref{fig:p8_road}. The time span was set to one month (2019/09/01 -- 2019/09/30). Sensors and incidents not on the chosen freeways were filtered out. We also removed incidents with a duration of fewer than 30 minutes due to their limited temporal impact. Figure~\ref{fig:p8_dis} shows the distribution of impact duration and length. The orange dashed lines indicate the fluctuations of log event counts across different durations and impact lengths. The shades of the bars show the magnitude of event counts. All the label values follow power-law distributions except that
the impact length in the San Bernardino region is relatively noisy compared
with the others. Finally, we present two traffic incident impact
prediction datasets: Incident-LA and Incident-SB. The detailed attributes of
the two dynamic networks are illustrated in Table~\ref{tab:pems_dist}. 

% \begin{figure*}
% \centering
% \begin{subfigure}{.33\linewidth}
%     \centering
%     \includegraphics[width=\linewidth]{figures/pems07.pdf}
%     \caption{Covered Freeways with Incident Locations (Incident-LA)}\label{fig:p7_road}
% \end{subfigure}
%     \hfill
% \begin{subfigure}{.65\linewidth}
%     \centering
%     \includegraphics[width=.99\linewidth,height=.4\linewidth]{figures/distribution_PEMS-07_1.png}
%     \caption{Distribution of Event Duration/Impact Length/ Non-Zero Impact Length (Incident-LA)}\label{fig:p7_dis}
% \end{subfigure}
% \bigskip
% \begin{subfigure}{.33\linewidth}
%     \centering
%     \includegraphics[width=\linewidth]{figures/pems08.pdf}
%     \caption{Covered Freeways with Incident Locations (Incident-SB)}\label{fig:p8_road}
% \end{subfigure}
%    \hfill
% \begin{subfigure}{.65\linewidth}
%     \centering
%     \includegraphics[width=.99\linewidth,height=.4\linewidth]{figures/distribution_PEMS-08_1.png}
%     \caption{Distribution of Event Duration/Impact Length/ Non-Zero Impact Length (Incident-SB)}\label{fig:p8_dis}
% \end{subfigure}
% \caption{Basic Dataset Information}
% \label{fig:dataset_intro}
% \end{figure*}

\begin{figure}
\centering
\begin{subfigure}{.49\linewidth}
    \centering
    \includegraphics[width=\linewidth]{figures/pems07.pdf}
    \caption{Included freeways with incident locations (Incident-LA)}\label{fig:p7_road}
\end{subfigure}
\begin{subfigure}{.49\linewidth}
    \centering
    \includegraphics[width=\linewidth]{figures/pems08.pdf}
    \caption{Included freeways with incident locations (Incident-SB)}\label{fig:p8_road}
\end{subfigure}
\bigskip
\begin{subfigure}{\linewidth}
    \centering
    \includegraphics[width=\linewidth,height=.4\linewidth]{figures/distribution_all_2.png}
    \caption{Distribution of event duration/impact length/ non-zero impact length (Incident-SB)}\label{fig:p8_dis}
\end{subfigure}
\caption{Basic Dataset Information}
\label{fig:dataset_intro}
\end{figure}

\begin{table}[h]
\small
  \centering
  \begin{tabular}{cllll}
    \toprule
    Dataset & \# event & \# node & \# edge & \# 0 length \\
    & & (R/S) & (R-R/S-S) & \\
    \midrule
    Incident-LA & 5,668 & 32/1,663 & 142/869,640 & 2,062\\
    Incident-SB & 1,452 & 28/1,150 & 140/390,822 & 454\\
    
  \bottomrule
\end{tabular}
  \caption{Dataset Properties}
  \label{tab:pems_dist}
\end{table}

\subsection{Baselines} To evaluate the efficiency of DG-Trans, nine representative models are chosen to perform the same task. Considering that only a few models target similar tasks, we choose the two most recently
 published incident impact prediction models, two conventional models, and
five state-of-the-art traffic forecasting models. Among the latter five models, two leverage attentive graph representation learning techniques, while the other three use graph convolution.
\begin{itemize}
    \item \textbf{L-1 regularized linear regression (LASSO)~\cite{tibshirani1996regression}.} As LASSO only takes one-dimensional features as inputs, we examined different feature aggregation methods. These include (1) averaging over both the spatiotemporal dimensions, (2) picking the closest upstream/downstream sensor and averaging over the temporal dimension, and (3) picking the closest upstream/downstream sensor and the five minutes immediately after the validation time. For parameter selection, we examined $\lambda$ values of $0.1$, $1.0$, $10.0$, and $100.0$. 
    \item \textbf{Support vector regression (SVR)~\cite{tibshirani1996regression}.} Similar to LASSO, we examined three different feature aggregation methods. We used the default parameters ($C=1, \epsilon=0.1$) in the sklearn~\cite{scikit-learn} package of Python.
    \item \textbf{HastGCN~\cite{fu2021hierarchical}.} HastGCN is a spatiotemporal attention model designed for incident duration prediction. We reproduce the model with the help of the author. Some features used in~\cite{fu2021hierarchical} are removed because they are not provided in our datasets. All the other settings are the same as the original model.
    \item \textbf{AGWN~\cite{meng2022early}.} AGWN preprocesses the adjacency matrix with a wavelet filter before the graph convolution operation. The model utilizes only one timestamp immediately after the validation time. To allow a fair comparison, we attached our Importance Score Transformer and pooling module to it. All the hyperparameters are the same as in the original paper.
    \item \textbf{STTN~\cite{xu2020spatial}.} STTN utilizes transformers for both spatial and temporal message-passing. The model is designed for node-level speed  forecasting. In this case, we attach our pooling module to the original STTN. The experiment code is extracted from the STTN github repository \footnote{\url{https://github.com/wubin5/STTN}}. All the hyperparameters remain the same as is mentioned in the paper.
    \item \textbf{STAWnet~\cite{tian2021spatial}.} STAWnet utilizes attentive graph message-passing and gated temporal convolution networks (TCN). It is also a node-level speed forecasting model and requires our pooling module to fit our incident impact prediction task. We utilized the official code \footnote{\url{https://github.com/CYBruce/STAWnet}} and left the hyperparameters unchanged.
    \item \textbf{DMSTGCN~\cite{han2021dynamic}.} DMSTGCN decomposes the adjacency matrix into four trainable embeddings for graph convolution. The temporal information is merged with gated TCN. Our pooling module was attached to the model to obtain the desired output. All the hyperparameters are the same as in the original paper \footnote{\url{https://github.com/liangzhehan/DMSTGCN}}.
    \item \textbf{Graph WaveNet~\cite{wu2019graph}.} Graph WaveNet contains a self-adaptive graph convolution module and a dilated temporal convolution module. Our pooling module was attached to the model to obtain the desired output. All the hyperparameters are the same as in the original paper \footnote{\url{https://github.com/nnzhan/Graph-WaveNet}}.
    \item \textbf{AGCRN~\cite{bai2020adaptive}.} AGCRN performs node-adaptive graph convolution and GRU-like temporal message-passing. Our pooling module was attached to the model to obtain the desired output. All the hyperparameters are the same as in the original paper \footnote{\url{https://github.com/LeiBAI/AGCRN}}.
\end{itemize} 

\subsection{Hyperparameter and Metrics}
In the problem settings of this paper, we assume that the objective is to predict the incident's impact within a short time after the event. To do so, nine timestamps (six for ``before-validation'' and three for ``after-validation'') were adopted. As a result, the model could not see the full traffic pattern during the incident.

For the training process of DG-Trans, we adopted a batch size of 8 due to the GPU memory limitation. The learning rate was 0.0005 with a 0.001 weight decay. The number of attention heads was 4. The LeakyReLU factor was set to 0.2, and the dropout rate was 0.1. In the loss function (Equation~\ref{eq:loss}), the prediction loss weight $\psi$ was 1.0, and $\omega$ was equal to the epoch.

Following our previous works~\cite{fu2019titan, fu2021hierarchical,meng2022early}, we adopted root mean squared error (RMSE) and mean absolute
error (MAE) as metrics. However, the impact length introduces labels of zero
value making mean absolute percentage error (MAPE) invalid. Therefore, we replaced MAPE with symmetric mean absolute percentage error (sMAPE). Based
on the definition ($RMSE=\sqrt{\frac{1}{N}\sum_{i=1}^N(y_i-\hat
{y}_i)^2}$, $MAE=\frac{1}{N}sum_{i=1}^N|y_i-\hat{y}_i|$, $sMAPE=\frac{1}
{N}sum_{i=1}^N|\frac{2|y_i-\hat{y}_i|}{|y_i|+|\hat{y}_i|}|$), RMSE penalizes
large gaps more harshly than MAE, while sMAPE focuses more on the magnitude of the differences from the true values.

\begin{table*}[t]
  \small
  \centering
  \begin{tabular}{c lll lll lll lll lll}
    \toprule
    &\multicolumn{3}{c}{Incident-LA (dur (min))} & 
     \multicolumn{3}{c}{Incident-LA (len (mile))} &
     \multicolumn{3}{c}{Incident-SB (dur (min))} &
     \multicolumn{3}{c}{Incident-SB (len (mile))}\\
     & RMSE & MAE & sMAPE & RMSE & MAE & sMAPE & RMSE & MAE & sMAPE & RMSE & MAE & sMAPE\\
    \midrule
    
    LASSO~\cite{tibshirani1996regression}&  59.773& 51.776& 0.760& 8.270& 6.794& 1.211& 58.921& 51.263& 0.757& 10.743& 8.934& 1.112\\
    SVR~\cite{tibshirani1996regression}& 60.073& 50.763& 0.743& 8.559& 6.300& 1.299& 59.560& 50.924& 0.761& 11.632 & 8.812 & 1.218\\
    HastGCN~\cite{fu2021hierarchical} & 31.719& 20.372& 0.319& 8.421& 6.402& 1.272& 35.936& 25.078& 0.381& 13.053& 9.299& 1.350 \\
    AGWN~\cite{meng2022early} & 31.934& 20.720& 0.341& 10.840& 6.594& 0.874& 32.864& 22.391& 0.365& 11.730& 8.736& 0.743 \\
    
    STTN~\cite{xu2020spatial}& 31.826& 21.098& 0.322& 9.644& 6.317& 0.893& 31.235& 20.631& 0.326& 12.346& 9.025& 0.812 \\
    STAWnet~\cite{tian2021spatial}& 31.400& 20.315& 0.318 & 8.619
& 6.311& 1.310 & 29.280& 20.212& 0.320& 11.994& 8.945& 1.260\\  
    DMSTGCN~\cite{han2021dynamic}& 31.555 & 20.342 & 0.319 & 10.638 & 7.784 & 0.880 & 29.810 & 20.263 & 0.312 & 12.929 & 9.846 & 0.791 \\
    Graph WaveNet~\cite{wu2019graph} & 31.880&20.415&0.320&10.489&7.672&0.861& 30.765&20.455&0.324& 14.093&10.807& 0.914 \\
    AGCRN~\cite{bai2020adaptive} & 31.253&20.363&0.319& 8.662&6.212&0.899& 30.905&20.652&0.324&12.808&9.093&1.172 \\
    \midrule
    \textbf{DG-Trans} & \textbf{31.413}& \textbf{20.310}& \textbf{0.318}& \textbf{9.494}& \textbf{6.226}& \textbf{1.477}& \textbf{29.726}& \textbf{20.140}& \textbf{0.319}& \textbf{11.731}& \textbf{8.818}& \textbf{1.235} \\
    \bottomrule
  \end{tabular}
  \caption{\textbf{RMSE, MAE, and sMAPE for Duration and Impact Length Prediction of Incident-LA and Incident-SB.} This table lists the performance of nine state-of-the-art baselines and our proposed model.}
  \label{tab:baseline}
\end{table*}

\subsection{Prediction Result Analysis}
The input to the models includes the adjacent matrices and sensor measurements one hour before and half an hour after the validation time. The output is the impact duration and length. We evaluated the duration and length separately as they are of different units.

Table~\ref{tab:baseline} illustrates the performance of DG-Trans against the baselines. 

\textbf{Conventional baselines.} For LASSO and SVR, the best performance is
 achieved with the closest upstream sensor and the first timestamp after the
 validation time as inputs. The results appeared to be insensitive to
 hyperparameters. As shown in Table~\ref{tab:baseline}, even though they
 produced competitive results in RMSE and MAE impact length prediction, LASSO
 and SVR performed poorly in predicting impact duration prediction.  This result matches our hypothesis that the ``closest'' sensors and timestamps are not optimal for prediction. 

\textbf{Spatiotemporal neural network baselines.} DG-Trans surpasses the other baselines (i.e., spatiotemporal neural networks) in most of the metrics and achieves competitive performance on the other metrics if it is not the best. Specifically, DG-Trans outperforms another spatiotemporal transformer, STTN, by about 10\% in duration prediction and 5\% in length prediction. DG-Trans also beats previous incident impact prediction models, HastGCN and AGWN, by approximately 10\% in performance.

 We find that DG-Trans cannot beat the other models in all metrics. To prove the efficiency of our model, we rank all ten models by each criterion and list the average rank in Table~\ref{tab:rank}. The table shows that our model performs the best on average.

 %  This part of the experiment shows that models designed specifically for
 % incidents (i.e., focus on locating incidents and denoising) perform better
 % than models evenly treat every spatiotemporal feature.
 \begin{table}[h]
\small
  \centering
  \begin{tabular}{clllll}
    \toprule
    Model&\textbf{DG-Trans}&STTN&STAW.&DMST.&AGCRN\\
    \midrule
    Rank & \textbf{3.58}&3.75&4.08&5.25&5.25\\
    \midrule
    Model &AGWN&G.W.&LASSO&Hast.&SVR\\
    \midrule
    Rank &5.75&6.67&6.83& 6.92&6.92\\
    
  \bottomrule
\end{tabular}
  \caption{\textbf{Average rank of baseline and DG-Trans performance.} Abbreviated model names are used due to the limited space. The smaller the number, the higher the average rank.}
  \label{tab:rank}
\end{table}

 
\subsection{Ablation Study}

\begin{table*}[t]
  \small
  \centering
  \begin{tabular}{c lll lll lll lll lll}
    \toprule
    &\multicolumn{3}{c}{Incident-LA (dur (min))} & 
     \multicolumn{3}{c}{Incident-LA (len (mile))} &
     \multicolumn{3}{c}{Incident-SB (dur (min))} &
     \multicolumn{3}{c}{Incident-SB (len (mile))}\\
     & RMSE & MAE & sMAPE & RMSE & MAE & sMAPE & RMSE & MAE & sMAPE & RMSE & MAE & sMAPE\\
    
    \midrule
    No-STrans & 31.245& 20.320& 0.319& 9.608& 6.251& 1.503& 28.793& 21.240& 0.338& 11.922& 8.877& 1.255 \\
    No-TTrans & 31.674& 20.371& 0.319& 9.496 & 6.215& 1.570& 32.024& 27.265& 0.422& 12.432& 8.903& 1.344 \\
    No-Road & 31.611& 20.351& 0.319& 9.909& 6.345& 1.616& 28.568& 20.928& 0.333& 13.240& 9.286& 1.496 \\
    No-Score & 31.846& 20.321& 0.319& 9.496& 6.215& 1.473& 30.000& 20.236& 0.320& 14.000& 9.743& 1.688 \\

    \midrule
    \textbf{DG-Trans} & \textbf{31.413}& \textbf{20.310}& \textbf{0.318}& \textbf{9.494}& \textbf{6.226}& \textbf{1.477}& \textbf{29.726}& \textbf{20.140}& \textbf{0.319}& \textbf{11.731}& \textbf{8.818}& \textbf{1.235} \\
    \bottomrule
  \end{tabular}
  \caption{\textbf{RMSE, MAE, and sMAPE for Duration and Impact Length Prediction of Incident-LA and Incident-SB.} The results of the ablation study include our model without the S-Transformer (No-STrans), our model without the T-Transformer (No-TTrans), our model without the road anchors (No-Road), and our model without the importance score (No-Score).}
  \label{tab:ablation}
\end{table*}

We conducted four ablation experiments to evaluate the contributions of each component of the DG-Trans model. The result is shown in Table~\ref{tab:ablation}. For the ``No-STrans'' and ``No-TTrans'' experiments, we simply removed the corresponding modules. For ``No-Score'', we skipped the concatenation steps in Equation~\ref{eq:score_1} and ~\ref{eq:score_2} and removed the reconstruction losses during training. For ``No-Road'', we replaced the Dual-Level S-Transformer with the transformer used in STTN. Initially, we assumed that ``No-Road'' would outperform DG-Trans as our graph had too many edges removed. However, the results in Table~\ref{tab:ablation} show that our model cannot achieve its performance without any of its components. We observed that the performance drops less in ``No-Score'' and ``No-TTrans'' than in ``No-Road'' and ``No-STrans''. This may be because the number of timestamps in the input is too small for a transformer to work.  
% \subsection{Graph Attention Efficiency.} 

% We also compared the execution time
%  of S-Transformer with the same modules in the baselines. Based on our
%  observation, HastGCN leverages three linear transformations to perform the
%  attention while STTN and STAWnet utilize the vanilla attention mechanism. As
%  suggested by Table~\ref{tab:pems_efficient}, S-Transformer executes $2$ to
%  $5$ times faster than the vanilla attention and also faster than the linear
%  transformation attention. 

% The execution times of STAWnet on the two datasets are the same, as we cut off
% some nodes from Incident-LA to remove the OOM issue. We also observe that the
% execution time is not linearly related to the number of sensors in DG-Trans.
% The reason is that the number of computations is also related to the number
% of roads. As the numbers of roads are similar in the two datasets, the
% execution times are the same as well.

% \begin{table}[h]
% \small
%   \centering
%   \begin{tabular}{cllll}
%     \toprule
%     Dataset & HastGCN & STTN & STAWnet & \textbf{DG-Trans} \\
%     \midrule
%     Incident-LA & 0.0021 & 0.0050 & 0.0080 & \textbf{0.0015}\\
%     Incident-SB & 0.0019 & 0.0025 & 0.0080  & \textbf{0.0013}\\
    
%   \bottomrule
% \end{tabular}
%   \caption{Average execution time of the graph attention module in second}
%   \label{tab:pems_efficient}
% \end{table}

\subsection{Case Study} To examine whether our importance score truly helps to
 identify incidents, we explored several incident cases. One example is
 incident \#18693 (Figure~\ref{fig:score}). The map plots the
 five-minute average speed immediately after the validation time and the importance
 scores within the same time slot. It is obvious that sensors that detect
 lower speeds also have higher importance scores, while high importance
 scores cluster around the incident (red star). However, we also observe that
 incident-irrelevant speed drops lead to high importance scores as well. This is why the model utilizes both the score and the embedded features for prediction.
\begin{figure}[h]
  \centering
  \includegraphics[width=\linewidth]{figures/score_map2.jpg}
  \caption{\textbf{Case study of incident \#18693 on the Freeway I--10.} This is the map of an incident and the surrounding traffic loop sensors on I--10 in the Incident-LA dataset. The red star indicates the location of the incident, and the yellow--green--blue dots are sensors colored according to speed measurements. The orange--red ``X''s indicate the importance scores assigned by DG-Trans. } % describe the architecture
  \label{fig:score}
\end{figure}

\noindent
\textbf{Beyond This Task.} We believe that the other parts of the dataset,
 (i.e., the data not used in this paper), can be used to increase the
 prediction accuracy. We examined simple methods of merging the
 auxiliary information into the traffic network, such as adding an incident
 classification task, using road and sensor position for position encoding,
 and embedding incident metadata. While none of those methods worked, all of these attempts are also uploaded to our GitHub repository \footnote{\url{https://github.com/styxsys0927/DG-Trans.git}} for others to investigate.
\section{Conclusion}\label{sec:conclusion}
In this work, we focus on addressing the fundamental challenge of OOD detection tasks, which is how to fully understand the semantic discrepancy between the ID/OOD samples. We reveal that the key to success in the realistic SCOOD task is to allocate as many ID samples in the unlabeled set correctly as possible. To this end, we propose a novel uncertainty-aware optimal transport scheme that introduces class-specific energy scores as guidance for effective label assignment. Experimental results show that our method achieves better performance than previous state-of-the-art methods on SCOOD benchmarks.

\textbf{Limitations.} In addition to temperature scaling, other techniques such as feature clipping applied in ReAct~\cite{sun2021react} also enhance the performance of energy score, so how to obtain an OOD score that best fits the SCOOD task can be further explored. Moreover, a setting highly related to SCOOD has been proposed in \cite{katz2022training} and formulated as a constrained optimization problem. We will also theoretically analyze these practical OOD settings in our feature work.

% \section*{Acknowledgments}
\textbf{Acknowledgments.} 
This work is supported by National Key R\&D Program of China under Grant 2020AAA0105701, National Natural Science Foundation of China (NSFC) under Grants 61872327, Major Special Science and Technology Project of Anhui, National Natural Science Foundation of China (62033012) and Ant Group through Ant Research Intern Program.


\bibliographystyle{IEEEtran}
\bibliography{main}


\end{document}
