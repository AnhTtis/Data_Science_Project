\section{Problem Definition} In this section, we first define the traffic network graph and its expansion to the temporal dimension. Then, we quantify traffic incident impact with two measurements -- impact duration and impact length. Finally, we formulate the problem of incident impact prediction.

\subsection{Traffic Graph} A traffic performance measurement system usually sets static traffic loop sensors on the side of arterial roads and collects traffic data near the sensors. Previous research has tended to link groups of sensors that are closest to each other in the geographic or feature space. However, we argue that this assumption does not hold for freeways. For example, if two sensors are recording traffic in opposite directions on the same freeway, they could be as close to each other as the width of the road. However, these two sensors should not be linked, as a ``U'' turn is not an option on freeways. In addition, two sensors on different roads could also be geographically close at the intersection of two roads. In such cases, the real condition is that cars must drive through a long ramp to switch to another road, which means a much longer distance than the Euclidian distance. 

Considering that the relations are hard to quantify for two sensors on
different roads, we suggest constructing a dual-level graph to represent a
traffic network. The first level is the ``sensor-to-road'' level, which links sensors to the roads they are on. The second level (``road-to-road'' level) links two roads if they intersect. Note that two
directions on one freeway are considered two roads.
\begin{Definition}{\rm Dual-level traffic network graph.} Consider a traffic
 network graph as $G$, where $G=(S, R, E^{sr}, E^{rr}, E^{rs}, A^{sr}, A^
 {rr}, A^{rs})$. $S$ is the sensor node set of size $|S|$. $R$ is the road
 node set of size $|R|$. $E^{sr}, E^{rr}, E^{rs}$ are the edges linking
 sensor-road, road-road, and road-sensor, respectively. $A^{sr} \in \mathbb
 {R}^{|S|\times |R|}, A^{rr} \in \mathbb{R}^{|R|\times |R|}, A^
 {rs} \in \mathbb{R}^{|R|\times |S|}$ are the adjacent matrix form of $E^
 {sr}, E^{rr}, E^{rs}$.
\end{Definition}

Assume that $T$ timestamps around the incident validation time are used for prediction because they are usually most relevant to the incident impact. The traffic behaviors before and after an incident can be quite different. In this case, we split the $T$ timestamps into ``before validation'' (containing $T_{bv}$ timestamps) and ``after-validation'' (containing $T_{av}$ timestamps). 
\begin{Definition}{\rm Dynamic traffic network graph.} A dynamic traffic network graph can be denoted as $\mathcal{G}=\{\mathcal{G}_{bv}, \mathcal{G}_
 {av}\}=\{G_0, ..., G_{T_{bv}-1}, G_{T_{bv}}, ..., G_{T_{bv}+T_
 {av}-1}\}$. $G_0,...,G_{T_{bv}+T_
 {av}-1}$ indicates graphs at timestamp $0,...,T_{bv}+T_
 {av}-1$. $\mathcal{G}_{bv}$ and $\mathcal{G}_{av}$ are dynamic graphs before and after the validation time, respectively. 
\end{Definition}

\subsection{Incident Impact Prediction} The impact of an incident can be quantified into two dimensions: the spatial dimension and the temporal dimension. We represent the temporal impact with $\mathbf{Y}_{Dur}$, which is the difference between the validation time and the restoration time. This definition also matches the formal definition of incident duration provided by the Department of Transportation (DOT). For the spatial dimension, traditional transportation research counts the number of blocked cars upstream of the incident. However, cars’ speed can be affected by congestion even if they are not blocked. To account for this, we define the impact length $\mathbf{Y}_{Len}$ as the maximum continuous congestion road distance in the immediate upstream of the incident.
\begin{Definition}{\rm Incident impact precision.} Given a dynamic traffic
 network graph for an incident $\mathcal{G}$ and corresponding sensor feature
 tensor $\mathbf{X} \in \mathbb{R}^{|S|\times T \times C_{in}}$, the aim is
 to find a model $\mathcal{F}$ so that 
  \begin{equation}
 \mathcal{F}:(\mathcal{G}, \mathbf{X}) \rightarrow (\mathbf{Y}_{Dur}, \mathbf{Y}_{Len})
 \end{equation}
 where $C_{in}$ is the number of input channels recorded by the sensors, $Y_{Dur}$ is the impact duration, and $Y_{Len}$ is the impact length.
\end{Definition}