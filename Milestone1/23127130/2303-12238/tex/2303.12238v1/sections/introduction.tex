\section{Introduction}

The efficient detection of non-recurring congestion caused by traffic incidents is
of great importance for modern intelligent transportation systems
(ITS). Furthermore, the estimation of the impact of such incidents is crucial
due to the potential social and economic loss~\cite
{adler2013road}. However, effectively pinpointing traffic incidents is challenging due to high data volatility, the lack of incident labels,
and the demands of ultra-low inference times in modern ITS. Despite this
difficulty, traffic incident detection and diagnosis is an active research
discipline, including approaches related to distributed computing, transportation
management, and urban resource management. The widespread deployment of traffic
sensors and traffic incident management systems (TIMS) has promoted such research by generating enormous
amounts of high-dimensional traffic sensor records and making them ubiquitously
accessible. With the abundance of traffic data sources, we can now present
two datasets and an efficient neural network model for accurate traffic
incident impact prediction from both the spatial and temporal perspectives.
\begin{figure*}[t]
  \centering
  \includegraphics[width=\linewidth]{figures/mdm_figure1.png}
  \caption{\textbf{An example of traffic incident impact.} The impact of incidents can be identified by congestion. The closer a location is to the incident, the longer it is impacted. Two factors, impact length (spatial impact) and duration (temporal impact), are used to define an incident's impact. In the map (left), the vertical arrows and the circles indicate congestion alleviation over time and distance, respectively. In the speed heatmap (right), the white dots reflect sensed congestion related to incidents.}
  \label{fig:intro_case}
\end{figure*}
The impact of traffic incidents can be quantified in two dimensions: time and space. As illustrated in Figure~\ref{fig:intro_case}, traffic along a road can be plotted over time and location as a heatmap of speed. The impact of an incident usually presents itself as a drop in speeds, that starts at the time and location of the incident and then spreads upstream before finally peaking. Conventionally, the incident duration is defined as the time between the validation and restoration of an incident, while the impact length is the number of cars blocked by the incident. However, we observe that cars are forced to slow down even if they are not near the incident location. In this case, for the spatial dimension, we redefine the impact length on the road as the maximum continuous congestion distance immediately upstream of the incident.

Recent research has primarily focused on the temporal impact of incidents~\cite{fu2019titan, fu2021hierarchical, meng2022early}, with limited attention paid to their spatial impact. Traffic forecasting studies have developed models for learning spatiotemporal representations. These could be leveraged to assess incident effects, though their abilities in sub-graph extraction and summarization remain untested. Based on the previous work above, we can conclude that the current research has the following drawbacks.

\textbf{The spatiotemporal quantification of traffic incident impacts is rarely considered, and the publicly available supported data is limited.} Unlike traditional transportation research, detailed traffic data -- such as the number of cars in the congestion queue and traffic signal states -- are usually unavailable to the public. In this case, it is essential to define extensive ``traffic incident impact'' criteria and construct open-source datasets in the context of dynamic graph data mining. 

\textbf{Relations between the static road network and the dynamic traffic correlation network are not properly modeled.} 
% Generally, spatial information aggregation can be considered as updating a node's features with the features of its neighbors, while the effect of each neighbor node is
%  determined by the weights on the edges. 
% Previous studies often incorporate human knowledge into latent feature correlations by utilizing one or more pre-computed adjacency matrices to mask attention weights~\cite{xu2020spatial,wu2021representing,xu2020spatial}. However, they rarely consider the errors that can be introduced by attention parameters or pre-computed weight matrices. 
In traffic forecasting scenarios, in particular, distance-based adjacent matrices are not always accurate, and sensor measurements may not provide sufficient information for similarity estimation. Therefore, it is necessary to develop a new approach that takes full advantage of the flexibility of attention and emphasizes the importance of road network structure.

\textbf{The capabilities of dynamic graph learning models are not evaluated for task-oriented subgraph/sub-time-series extraction.} In a traffic loop sensor network, an incident can dramatically affect a subset of sensors for a limited period. Therefore, it is necessary to develop dynamic graph learning models that can accurately predict the impacts of such incidents. 
% These models should be able to extract relevant sensor readings and the mutual effects of the incident on the affected sensors. 
Despite the prevalence of node and graph-level regression/classification tasks in dynamic graph representation learning, the ability to predict incident impacts using such networks has not been fully explored.

% The effect of incidents varies case by case. A ``road maintenance'' case can affect traffic patterns for days and long ranges, while a ``rear-end collision'' at midnight usually causes no or only a small range of traffic congestion. Congestion may form before the validation time during peak hours, and may also take some time to gather enough cars during non-peak hours. Besides, different from general graph-level prediction tasks, incident impact prediction is not essentially a summarization of the entire network. An incident usually only causes variation in a subgraph, while the spatiotemporal coordinates of the variation are hard to determine. Thus, the spatiotemporal relations between sensors and incidents are complex. 

To solve these challenges, we propose the \textbf{D}ual-level \textbf{G}raph \textbf{Tran}sformer for Traffic Incident Spatiotemporal Impact Prediction. Our approach effectively extracts spatiotemporal relations within the
traffic network and automatically locates latent variations in traffic
patterns with two benchmarks measuring the impacts of incidents. The main
contributions of this paper are summarized as follows:

\begin{itemize}
    \item \textbf{Quantifying the definition of ``incident impact'' and providing two open-source datasets.} We provide new definitions of traffic incident impact based on the observations of the spatiotemporal relationships between incident records and sensor measurements. As a result, the incident impact problem can be studied using dynamic graph representation learning methods. In addition to the sensor network and incident records, we also provide auxiliary data, such as sensor metadata, road structures, and other data, in order to support broader applications and problem domains.
    
    \item \textbf{Proposing an efficient dual-level graph attention strategy utilizing accurate connectivity and correlation information.} The proposed graph attention module integrates the advantages of static graph structures and graph attention. As a result, it properly guides attention weights with static graph structure knowledge, making full use of the flexibility of attention while stressing the importance of road network structure. In addition to its superior performance in extracting traffic patterns, our attention module is also proved to be time-efficient.
     
    \item \textbf{Designing an importance score temporal transformer mechanism to locate sub-sensor-graph affected by incidents.}
     Inspired by TranAD~\cite{tuli2022tranad}, we computed an ``importance score'' for each sensor at each timestamp using anomaly detection techniques. As a higher variation in traffic patterns leads to a larger ``importance score'', sensors potentially affected by an incident can be accurately located and assigned higher weights during prediction.
     
    \item \textbf{Evaluating model performance with extensive
     experiments.} Overall, DG-Trans outperforms state-of-the-art traffic prediction models on the two datasets we provided. We also show that each module in DG-Trans is  effective. The efficiency of our spatial transformer module is proved mathematically. The ``importance score'' is also proved to be effective through our case study. We also explored different methods of incorporating the metadata into the traffic network to lay the groundwork for future research.
\end{itemize}

This article is structured as follows. We first introduce existing traffic incident impact prediction studies and dynamic graph representation learning models in Section~\ref{sec:related}, then provide the definition of traffic incident impact prediction problem. In Section~\ref{sec:method}, we elaborate on our proposed model. Section~\ref{sec:experiment} describes our benchmark datasets and shows the advantages of our model.
% , as well as other models's limitations. 
Finally, Section~\ref{sec:conclusion} summarizes this work.









