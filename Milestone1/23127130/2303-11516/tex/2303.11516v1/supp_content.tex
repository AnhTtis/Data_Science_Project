\subsection{Linearization of PnP Solver}
%%%%%%%%% BODY TEXT
\noindent \textbf{Implicit Function Theorem.}
The implicit function theorem (IFT)~\cite{krantz2002implicit} states the following:

Given \mbox{$f:\mathbb{R}^{n+m}\to \mathbb{R}^m$} a continuously differentiable function
 with input $(\bs{a},\bs{b})\in \mathbb{R}^n\times\mathbb{R}^m$, if a point $(\bs{a}^*,\bs{b}^*)$ satisfies
\begin{equation}\label{eq:ift_f}
    f(\bs{a}^*,\bs{b}^*)=\bs{0}\;,
\end{equation}
and the Jacobian matrix $\frac{\partial f}{\partial\bs{b}}(\bs{a}^*,\bs{b}^*)$ is invertible, then there exists a unique continuously differentiable function \mbox{$g(\bs{a}):\mathbb{R}^n\to\mathbb{R}^m$} such that
\begin{equation}\label{eq:ift_g}
    \bs{b}^*=g(\bs{a}^*)\;,
\end{equation}
and
\begin{equation}\label{eq:ift_fg}
f(\bs{a}^*,g(\bs{a}^*))=\bs{0}\;.
\end{equation}
The Jacobian matrix $\frac{\partial g}{\partial\bs{a}}(\bs{a}^*)$ is given by
\begin{equation}\label{eq:ift_grad}
\frac{\partial g}{\partial\bs{a}}(\bs{a}^*)=-\left[ \frac{\partial f}{\partial \bs{b}}(\bs{a}^*,\bs{b}^*) \right]^{-1}\cdot
\frac{\partial f}{\partial \bs{a}}(\bs{a}^*,\bs{b}^*)\;.
\end{equation}

\noindent \textbf{PnP Linearization.}
Following the same notation as in the main paper, the PnP solver computes the function
\begin{equation}\label{eq:sup_pnp}
    g(\bs{x},\bs{z},\bs{w}) = \mathop{\arg\min}_{\bs{y}}\frac{1}{2} \sum_i^N \left\Vert \bs{w}_i \circ \bs{r}_i \right\Vert^2\;,
\end{equation}
where $\bs{x}_i$ is the $i$-th image 2D point, $\bs{z}_i$ is the $i$-th 3D point, $\bs{w}_i$ is the corresponding weight, and
\begin{equation}
\bs{r}_i = \bs{x}_i - \pi(\bs{z}_i,\bs{y})
\end{equation}
is the reprojection residual for the $i$-th correspondence given pose $\bs{y}$.

Eq.~\ref{eq:sup_pnp} implies that the solution $\bs{y}^*$ is the stationary point of the negative log likelihood (NLL) function
\begin{equation}\label{eq:nll}
    nll(\bs{y}) = \frac{1}{2} \sum_i^N \left\Vert \bs{w}_i \circ \bs{r}_i \right\Vert^2\;.
\end{equation}

Since $\bs{y}^*$ is the stationary point of the NLL function, the first order derivative of the NLL w.r.t. $\bs{y}^*$ should be zero, i.e.,
\begin{equation}
    \left.\frac{\partial nll(\bs{y})}{\partial\bs{y}}\right|_{\bs{y}=\bs{y}^*}=\bs{0}\;.
\end{equation}

Eqs.~\ref{eq:ift_f},~\ref{eq:ift_g} and ~\ref{eq:ift_fg} in the PnP case can subsequently be specialized as
\begin{equation}
f(\bs{x},\bs{y},\bs{z},\bs{w})|_{\bs{y}=\bs{y}^*}=\left.\frac{\partial nll(\bs{y})}{\partial\bs{y}}\right|_{\bs{y}=\bs{y}^*}=\bs{0}\;,
\end{equation}
\begin{equation}
\bs{y}^*=g(\bs{x},\bs{z},\bs{w})\;,
\end{equation}
and
\begin{equation}
f(\bs{x},g(\bs{x},\bs{z},\bs{w}),\bs{z},\bs{w})|_{\bs{y}=\bs{y}^*}=\bs{0}\;.
\end{equation}

According to Eq.~\ref{eq:ift_grad}, the gradient of the pose $\bs{y}$ w.r.t. the 2D locations $\bs{x}$ at $\bs{y}^*$ is
\begin{equation}\label{eq:dydx}
\begin{split}
    \left.\frac{\partial\bs{y}}{\partial\bs{x}}\right|_{\bs{y}^*}&=
    \left.\frac{\partial g(\bs{x},\bs{z},\bs{w})}{\partial \bs{x}}\right|_{\bs{y}^*}\;,\\
    &=\left.-\left[\left[\frac{\partial^2 nll(\bs{y})}{\partial \bs{y}^2}\right]^{-1} \cdot
    \frac{\partial^2 nll(\bs{y})}{\partial \bs{y}\partial\bs{x}}\right]\right|_{\bs{y}*}\;,\\
    &=\left.-H^{-1}\cdot
    \frac{\partial^2 nll(\bs{y})}{\partial \bs{y}\partial\bs{x}}\right|_{\bs{y}^*}\;,
\end{split}
\end{equation}
with $nll(\bs{y})$ defined by Eq.~\ref{eq:nll}.

Given the noisy correspondences $\{\bs{x},\bs{z},\bs{w}\}$, we compute the perfect correspondences $\{\bs{x}_p,\bs{z},\bs{w}\}$ with $\bs{x}_{p,i}=\pi(\bs{z}_i,\bs{y}_{gt})$ under the ground-truth pose $\bs{y}_{gt}$. We then linearize the PnP solver around $\{\bs{x}_p,\bs{z},\bs{w}\}$ and $\bs{y}_{gt}$ using the first-order Taylor expansion as
\begin{equation}
    \bs{y}=\bs{y}_{gt}+A(\bs{z},\bs{w})\cdot \bs{r}_{gt}\;,
\end{equation}
with
\begin{equation}
\bs{r}_{gt} = \bs{x}-\bs{x}_{gt}
\end{equation}
being the residual vector at $\bs{y}_{gt}$,
and 
\begin{equation}
    A(\bs{z},\bs{w})=\left.-H^{-1}\cdot
    \frac{\partial^2 nll(\bs{y})}{\partial \bs{y}\partial\bs{x}}\right|_{\bs{y}=\bs{y}_{gt},\bs{x}=\bs{x}_p}.
\end{equation}
The Hessian $H$ of the NLL function is also used to compute the prior loss, as stated in Sec.~3.3 in the main paper.


\begin{figure}[t]
\begin{center}
\includegraphics[width=0.9\linewidth,trim=0 20 0 1]{figs/corr.pdf}
\end{center}
   \caption{\textbf{Correctness curves of the PnP layers.}  A 3D point is considered to have a correct gradient if moving in the negative gradient direction leads to a smaller 2D reprojection error. The LC loss yields almost 100\% correctness. The correctness of EPro-PnP drops slowly, and ending with about 59\% correctness. BPnP drops quickly when training begins, and ends with about 53\% correctness. The dark curves are smoothed versions of the light ones.}
\label{fig:corr}
\end{figure}

\subsection{Detailed Results on Gradient Correctness}
The correctness scores for different PnP layers reported in the main paper were calculated in simulated conditions, where the extracted correspondences have large errors. We further provide the correctness curves based on actual training to show how the correctness evolves as training progresses. 

As illustrated in Fig.~\ref{fig:corr}, at the very beginning, when the correspondences have large errors, both EPro-PnP~\cite{Chen_2022_CVPR} and BPnP~\cite{Chen_2020_CVPR} have good correctness. However, their correctness drops when the training proceeds. Since the linear-covariance loss is designed to address this problem, it always maintains a correctness close to 100\%.

\subsection{Details on ZebraPose-based Experiments}
\noindent \textbf{Implementation Details.}
\begin{table}[t]
\begin{center}
\begin{tabular}{c|l|c}
\hline
Row & Method & ADD(-S)\\
\hline
A0 & ZebraPose~\cite{Su_2022_CVPR} & 76.91 \\
A1 & ZebraPose baseline & 75.19 \\
A2 & A1 + LC loss & \textbf{78.06} \\
\hline
\end{tabular}
\end{center}
\caption{\textbf{Results of the ZebraPose~\cite{Su_2022_CVPR} based experiments on the LM-O dataset.}}
\label{tab:sup-zebra}
\end{table}
Our coordinate-wise encoding scheme assigns 3 binary codes to a vertex, eliminating the look up operation. To reduce the number of binary bits for prediction, we rotate some of the objects to minimize their span along the $x,y,z$ directions. We use 7 bits to represent the coordinate component with the largest span, and calculate the binary count of the other components based on their relative span w.r.t. largest one.


\begin{figure}[t]
\begin{center}
\includegraphics[width=1\linewidth,trim=0 30 0 0]{figs/zebra-vis.pdf}
\end{center}
   \caption{\textbf{Visualizations for the ZebraPose-based model.}  (a) Visualizations of the input image patch, decoded object coordinates and the predicted weight map. (b)-(e) Visualizations of the predicted masks of coordinate components with the most significant bit at the left and the $x$ component at the top. The pixels predicted as background are masked out for clarity.}
\label{fig:zebra}
\end{figure}
\noindent \textbf{Results.}
As shown in Tab.~\ref{tab:sup-zebra}, after switching from the global vertex encoding to our coordinate-wise encoding (A0~\textit{vs.}~A1), the performance drops by about 1.7 points. When the LC loss is applied, the performance drop is compensated, surpassing the original ZebraPose~\cite{Su_2022_CVPR}.

\noindent \textbf{Visualizations.}
As illustrated by Fig.~\ref{fig:zebra}, the learned weight map successfully captures the error distribution of the predicted 3D coordinates in a geometry-aware manner, generating low weights for code transition regions and high weights for object endpoint regions.

\subsection{Detailed Results on LM-O and YCB-V}
For the LM-O dataset, we provide the detailed comparison of ADD(-S) scores with state-of-the-art methods, when the linear-covariance (LC) loss is applied to GDR-Net and ZebraPose on LM-O in Tab.~\ref{tab:lmo-detail}.

For the YCB-V dataset, we provide the detailed comparison of ADD(-S) scores (Tab.~\ref{tab:ycbv-detail-add}) and AUC scores (Tab.~\ref{tab:ycbv-detail-auc}) between the baseline methods and the versions where the LC loss is applied.

\begin{table}
\begin{center}
\scalebox{0.92}{
\begin{tabular}{l|cc|cc}
\hline
Object & \cite{Wang_2021_CVPR} & \cite{Su_2022_CVPR} & \cite{Wang_2021_CVPR}-LC & \cite{Su_2022_CVPR}-LC \\
\hline

002\_master\_chef\_can & 41.5 & \textbf{62.6} & 38.7 & 51.6 \\
003\_cracker\_box & 83.2 & 98.5 & 96.2 & \textbf{99.7} \\
004\_sugar\_box & 91.5 & 96.3 & 98.1 & \textbf{99.4} \\
005\_tomato\_soup\_can & 65.9 &  \textbf{80.5} & 77.6 & 79.6 \\
006\_mustard\_bottle & 90.2 & \textbf{100} & 77.0 & 99.7 \\
007\_tuna\_fish\_can & 44.2 & 70.5 & 63.2 & \textbf{86.1} \\
008\_pudding\_box & 2.8 & \textbf{99.5} & 81.3 & 99.1 \\
009\_gelatin\_box & 61.7 & \textbf{97.2} & 81.8 & 94.9 \\
010\_potted\_meat\_can & 64.9 & \textbf{76.9} & 68.1 & 73.9 \\
011\_banana & 64.1 & 71.2 & 71.0 & \textbf{95.8} \\
019\_pitcher\_base & 99.0 & \textbf{100} & \textbf{100} & \textbf{100} \\
021\_bleach\_cleanser & 73.8 & 75.9 & 69.9 & \textbf{85.6} \\
024\_bowl* & 37.7 & 18.5 & \textbf{44.1} & 35.2 \\
025\_mug & 61.5 & 77.5 & 46.2 & \textbf{88.7} \\
035\_power\_drill & 78.5 & 97.4 & \textbf{99.7} & 99.2 \\
036\_wood\_block* & 59.5 & 87.6 & \textbf{91.7} & 82.6 \\
037\_scissors & 3.9 & \textbf{71.8} & 14.9 & 56.9 \\
040\_large\_marker & 7.4 & 23.3 & \textbf{29.3} & 27.8 \\
051\_large\_clamp* & 69.8 & \textbf{87.6} & 80.5 & 84.4 \\
052\_extra\_large\_clamp* & 90.0 & 98.0 & 95.5 & \textbf{99.1} \\
061\_foam\_brick* & 71.9 & \textbf{99.3} & 57.6 & 91.3 \\
\hline
mean & 60.1 & 80.5 & 70.6 & \textbf{82.4} \\


\hline
\end{tabular}
}
\end{center}
\caption{\textbf{Detailed ADD(-S) scores on YCB-V.} We report the scores of the original baseline methods, GDR-Net~\cite{Wang_2021_CVPR} and ZebraPose~\cite{Su_2022_CVPR}, and also the scores after applying our LC loss, respectively (denoted by ``-LC''). (*) denotes symmetric objects on which the ADD-S score is reported.}
\label{tab:ycbv-detail-add}
\end{table}
\begin{table*}
\begin{center}
\begin{tabular}{c|cccccc|cc}
\hline
Method & \makecell{RePOSE\\ \cite{Iwase_2021_ICCV}} & \makecell{RNNPose\\ \cite{Xu_2022_CVPR}} & \makecell{SO-Pose\\ \cite{Di_2021_ICCV}} & \makecell{DProST\\ \cite{dprost_2022_eccv}} & \makecell{GDR-Net\\ \cite{Wang_2021_CVPR}} & \makecell{ZebraPose\\ \cite{Su_2022_CVPR}} & GDR-LC & Zebra-LC\\
\hline

ape & 31.1 & 37.18 & 48.4 & 51.4 & 46.8 & 57.9 & 44.44 & \textbf{61.57}\\
can & 80.0 & 88.07 & 85.8 & 78.7 & 90.8 & 95.0 & 89.06 & \textbf{97.35}\\
cat & 25.6 & 29.15 & 32.7 & 48.1 & 40.5 & 60.6 & 49.87 & \textbf{64.49}\\
driller & 73.1 & 88.14 & 77.4 & 77.4 & 82.6 & \textbf{94.8} & 87.81 & 94.65\\
duck & 43.0 & 49.17 & 48.9 & 45.4 & 46.9 & 64.5 & 56.08 & \textbf{66.82}\\
eggbox* & 51.7 & 66.98 & 52.4 & 55.3 & 54.2 & 70.9 & 62.81 & \textbf{71.77}\\
glue* & 54.3 & 63.79 & 78.3 & 76.9 & 75.8 & \textbf{88.7} & 68.88 & 86.35\\
holepuncher & 53.6 & 62.76 & 75.3 & 67.4 & 60.1 & \textbf{83.0} & 72.89 & 81.49\\
\hline
mean & 51.6 & 60.65 & 62.3 & 62.6 & 62.2 & 76.9 & 66.48 & \textbf{78.06}\\

\hline
\end{tabular}
\end{center}
\caption{\textbf{Comparison with the state of the art on LM-O.} (*) denotes symmetric objects on which the ADD-S score is reported. ``GDR-LC'' denotes the LC loss with the GDR-Net~\cite{Wang_2021_CVPR} baseline, ``Zebra-LC'' denotes the LC loss with the ZebraPose~\cite{Su_2022_CVPR} baseline.}
\label{tab:lmo-detail}
\end{table*}
\begin{table*}
\begin{center}
\setlength\tabcolsep{5.4pt}
\begin{tabular}{l|cc|cc|cc|cc}

\hline
 Method & \multicolumn{2}{c|}{GDR-Net~\cite{Wang_2021_CVPR}} &  \multicolumn{2}{c|}{ZebraPose~\cite{Su_2022_CVPR}} &  \multicolumn{2}{c|}{GDR-Net-LC} & \multicolumn{2}{c}{ZebraPose-LC} \\
 \hline
 Metric & \makecell{AUC of \\ADD-S} & \makecell{AUC of \\ADD(-S)} & \makecell{AUC of \\ADD-S} & \makecell{AUC of \\ADD(-S)} & \makecell{AUC of \\ADD-S} & \makecell{AUC of \\ADD(-S)} & \makecell{AUC of \\ADD-S} & \makecell{AUC of \\ADD(-S)} \\
 \hline

002\_master\_chef\_can & *96.3 & *65.2 & 93.7 & 75.4 & 85.6, *90.1 & 57.5, *61.6 & 88.4 & 66.9 \\
003\_cracker\_box & *97.0 & *88.8 & 93.0 & 87.8 & 93.1, *98.1 & 86.8, *91.6 & 93.7 & 88.3 \\
004\_sugar\_box & *98.9 & *95.0 & 95.1 & 90.9 & 95.9, *99.8 & 92.3, *97.4 & 94.7 & 90.3 \\
005\_tomato\_soup\_can & *96.5 & *91.9 & 94.4 & 90.1 & 92.8, *96.2 & 88.2, *93.0 & 93.4 & 89.2 \\
006\_mustard\_bottle & *100 & *92.8 & 96.0 & 92.6 & 94.1, *97.6 & 88.2, *93.1 & 95.1 & 90.9 \\
007\_tuna\_fish\_can & *99.4 & *94.2 & 96.9 & 92.6 & 96.2, *99.9 & 92.1, *96.9 & 97.2 & 94.1 \\
008\_pudding\_box & *64.6 & *44.7 & 97.2 & 95.3 & 94.4, *99.1 & 90.4, *95.3 & 96.7 & 94.7 \\
009\_gelatin\_box & *97.1 & *92.5 & 96.8 & 94.8 & 95.1, *99.9 & 91.7, *96.8 & 96.7 & 94.6 \\
010\_potted\_meat\_can & *86.0 & *80.2 & 91.7 & 83.6 & 85.8, *89.0 & 79.6, *83.8 & 91.3 & 82.5 \\
011\_banana & *96.3 & *85.8 & 92.6 & 84.6 & 92.2, *97.6 & 83.2, *88.0 & 95.3 & 90.1 \\
019\_pitcher\_base & *99.9 & *98.5 & 96.4 & 93.4 & 96.6, *100 & 93.5, *98.4 & 96.4 & 93.2 \\
021\_bleach\_cleanser & *94.2 & *84.3 & 89.5 & 80.0 & 86.3, *91.2 & 77.0, *82.0 & 90.5 & 82.3 \\
024\_bowl* & *85.7 & *85.7 & 37.1 & 37.1 & 83.1, *88.6 & 83.1, *88.6 & 63.9 & 63.9 \\
025\_mug & *99.6 & *94.0 & 96.1 & 90.8 & 92.7, *96.5 & 83.9, *88.9 & 96.5 & 92.3 \\
035\_power\_drill & *97.5 & *90.1 & 95.0 & 89.7 & 96.1, *99.9 & 92.6, *97.9 & 95.4 & 90.8 \\
036\_wood\_block* & *82.5 & *82.5 & 84.5 & 84.5 & 87.1, *92.2 & 87.1, *92.2 & 81.2 & 81.2 \\
037\_scissors & *63.8 & *49.5 & 92.5 & 84.5 & 75.8, *80.4 & 63.5, *67.8 & 88.3 & 79.0 \\
040\_large\_marker & *88.0 & *76.1 & 80.4 & 69.5 & 77.5, *81.8 & 68.8, *73.5 & 77.6 & 68.5 \\
051\_large\_clamp* & *89.3 & *89.3 & 85.6 & 85.6 & 83.1, *87.9 & 83.1, *87.9 & 86.8 & 86.8 \\
052\_extra\_large\_clamp* & *93.5 & *93.5 & 92.5 & 92.5 & 91.4, *95.8 & 91.4, *95.8 & 94.6 & 94.6 \\
061\_foam\_brick* & *96.9 & *96.9 & 95.3 & 95.3 & 90.0, *94.6 & 90.0, *94.6 & 93.2 & 93.2 \\
\hline
mean & *91.6 & *84.4 & 90.1 & 85.3 & 89.8, *94.1 & 84.0, *88.8 & \textbf{90.8} & \textbf{86.1} \\

\hline
\end{tabular}
\end{center}
\caption{\textbf{Detailed AUC scores on YCB-V.}  We report the scores of the original baseline methods and the scores after the LC loss is applied. \mbox{``GDR-Net-LC''} denotes the LC loss with the GDR-Net~\cite{Wang_2021_CVPR} baseline, ``ZebraPose-LC'' denotes the LC loss with the ZebraPose~\cite{Su_2022_CVPR} baseline. A (*) after the object name denotes the symmetric objects on which the ADD-S score is reported. A (*) before the AUC score indicates that the AUC is computed with 11-points interpolation.}
\label{tab:ycbv-detail-auc}
\end{table*} 