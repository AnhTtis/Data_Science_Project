\section{Introduction}
Learning how to control hybrid systems is critical in the realm of robotics and artificial intelligence, given the wide variety of hybrid systems in autonomous driving~\citep{ning2021survey}, locomotion for bipedal robots, and UAVs~\citep{gillula2011applications}. 
%Hybrid systems have the flexibility in modelling the continuous motion of physical processes as well as discrete switching logic such as changing of the environment or targets. 
However, it remains challenging to analyze the stability and design controllers for general nonlinear hybrid systems, due to the intricate dynamics involving both the continuous flows and discrete jumps~\citep{chen2021learning}.

Various approaches in classic control emerged to analyze the stability for a certain type of hybrid systems, such as piecewise affine (PWA) systems~\citep{johansson2003piecewise,pettersson1996stability,pettersson1999exponential,prajna2003analysis} and periodic systems~\citep{poincare1885courbes,clark2018poincar,manchester2011transverse,manchester2011regions}. However, these methods either work on simple systems in low dimensions or rely on computation-heavy methods for verification and synthesis~\citep{abate2020formal, jarvis2003some,topcu2008local,majumdar2013control}. 

The pivot hurdle for stabilizing general hybrid systems is to properly handle the system at the mode switching~\citep{de2009survey}. A hybrid system, provided with all its \postac{system modes stable}, can still be unstable if the mode switching is too fast~\citep{branicky1998multiple}. Using multiple Lyapunov functions, some works constraint the average dwell time (ADT)~\citep{hespanha1999stability,zhai2000piecewise} so the system switching is ``slow-on-the-average'', but they cannot handle discrete jumps. Other methods enforce the sub-sequence of each Lyapunov function at switch-in instants decreasing~\citep{branicky1998multiple}. Those methods track each Lyapunov function's switching sequence, which is hard to implement when there are many system modes. 

\begin{figure}[!tbp]
\floatconts{fig:roa-teaser}
{\caption{We learn the stabilizing control and the RoA for each \postac{system mode}. In testing phase, our method (red lines) plans the configuration (the exiting state and the mode) to ensure the next entering state is always within the RoA of the next equilibrium, whereas traditional methods (gray lines) directly tracking the next equilibrium will diverge after the jump.}}
{
\includegraphics[width=0.95\textwidth]{figures/roa_teaser_v3.png}
}
\end{figure}

Motivated by region of attraction (RoA)-based planning methods~\citep{tedrake2010lqr}, we propose an RoA-based approach to stabilize hybrid systems. The idea is to let the state always enter next mode's RoA after switching. We ensure \postac{the stability of the system under each mode } by using control Lyapunov functions (CLF), and we use the invariant set provided by the Lyapunov function to represent the RoA for \postac{the system under each mode}. In this way, we don't need to optimize the whole sequence of the Lyapunov functions but only consider the Lyapunov function (and RoA) in consecutive modes, which is more efficient to implement for stability analysis and controller synthesis. Constructing Lyapunov functions used to be time-consuming and is limited to low-dimension systems~\citep{abate2020formal, jarvis2003some,topcu2008local,majumdar2013control}. Recently there is a trend in using neural networks to construct control Lyapunov and Barrier certificates~\citep{chang2019neural,dawson2022safecontrol, qin2021learning, meng2021reactive, sun2020learning}. We follow this direction and further use neural networks for RoA estimation. 

The whole pipeline is as follows: we first collect samples under \postac{each system mode}, use the neural network controller to generate trajectories, and construct neural Lyapunov certificates to ensure the stability for each mode. Then we estimate the RoA by the Lyapunov level-set and learn \postac{each system mode's } stable level-set using the Neural RoA estimator. Finally, upon deployment, we use a differentiable planner to find the optimal configuration before mode switching to ensure the next state will be within the RoA of the next mode, hence achieving the stability of the hybrid system. Although there might exist a gap between theoretical guarantees and simulation performance due to  imperfect learning of neural networks, in practice we observe strong empirical results.

We conduct experiments on three challenging scenarios (car tracking control on different road conditions, pogobot navigation, and bipedal walker locomotion). We achieve the best performance (mean square error, failure rate, etc) compared to other baselines (MPC, RL, LQR, CLF, QP, HJB). Our learned method requires less training time than RL-based methods or the HJB approach, and at the evaluation stage, the runtime is only 1/50X$\sim$1/10X the time for MPC. 


Our contributions are:
(1) we are the first to use a neural-network RoA estimator, planner and controller to stabilize hybrid systems \postac{within certain RoA}. (2) we \postac{define } a new stability for hybrid systems and derive sufficient conditions to enforce the stability with theoretical guarantees (3) our approach can be applied to different hybrid systems \postac{even if the dynamics is unknown}, where the controller can be state-feedback control or apex-to-apex control for periodic systems. (4) we conduct challenging experiments that involve complicated hybrid systems and outperform other baselines.



