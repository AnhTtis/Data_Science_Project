\section{Related Work}

Controlling hybrid systems is challenging and has been studied for decades. Various formulations for system modelling and control strategies have been extensively presented in ~\citep{decarlo2000perspectives,de2009survey,davrazos2001review,sanfelice2013control}. We only list a few closely related works in stability analysis and controller synthesis.

\partitle{Lyapunov stability analysis} 
%Lyapunov functions are scalar functions to ensure the stability of equilibrium points of the systems. 
The common Lyapunov function is used to prove hybrid system stability in ~\citep{dogruel1994stability,fierro1997hybrid}
%, which requires the Lyapunov function to decrease along any hybrid system trajectories
. Multiple Lyapunov functions are proposed in ~\citep{peleties1992asymptotic} for linear systems and ~\citep{branicky1998multiple,michel1999towards} for more general cases where the monotonous decreasing condition is relaxed. In~\citep{malmborg1998analysis}, non-smooth Lyapunov functions are used for hybrid controller synthesis. 
%For piecewise affine (PWA) systems, linear matrix inequality based approaches are proposed to construct Lyapunov certificates using piecewise affine~\citep{johansson2003piecewise}, piecewise quadratic~\citep{pettersson1996stability,pettersson1999exponential} and piecewise polynomials functions~\citep{prajna2003analysis}. 
For piecewise affine (PWA) systems, linear matrix inequality based approaches are proposed to construct Lyapunov certificates~\citep{johansson2003piecewise,pettersson1996stability}. %,prajna2003analysis}.
% For switched systems, average dwell time (ADT) is introduced in~\citep{hespanha1999stability,zhai2000piecewise} where the total activation ratio between stable and unstable modes is required to be larger than a threshold to achieve stability.
For switched systems, average dwell time (ADT) is introduced in~\citep{hespanha1999stability,zhai2000piecewise} to tie the stability with the ratio between stable and unstable modes. For periodic systems, Poincar\'{e} map~\citep{clark2018poincar} and transverse Lyapunov functions~\citep{manchester2011transverse} are used to analysis the limit cycle stability. Although equipped with theoretical guarantees for the stability, the methods above are either only suited for a specified type of systems (linear or PWA) or require expert knowledge to design the Lyapunov functions, or rely on tedious numerical methods such as Satisfiability Modulo Theories (SMT) solvers~\citep{abate2020formal} and sum-of-squares (SoS) optimization~\citep{jarvis2003some,topcu2008local}. For complicated systems, recently there is a growing trend to approximate the Lyapunov functions and the controllers using data-driven methods like Gaussian Process~\citep{zhai2019region,xiao2020learning}, SVM~\citep{sun2005support} and neural networks~\citep{richards2018lyapunov,chang2019neural,mehrjou2020neural,dawson2022safecontrol,dawson2022safe}. We are aligned with this and further study to stabilize hybrid systems using neural networks.

\partitle{Hybrid system control designs} Upon grounding work on Lyapunov theory~\citep{sontag1983lyapunov,artstein1983stabilization}, a wide body of literature exists on synthesizing feedback controllers for switched linear and affine systems~\citep{wicks1997solution,johansson1999piecewise,mignone2000stability}. There are also many works beyond Lyapunov controllers for more general hybrid systems, such as optimal control~\citep{cassandras2001optimal,cho2001forward}, model predictive control (MPC)~\citep{slupphaug1997model,lazar2006model,camacho2010model,marcucci2020warm} and Hamilton-Jacobian reachability-based (HJB) methods~\citep{choi2022computation}, and region-of-attraction (RoA) based approaches~\citep{tedrake2010lqr,bhounsule2018switching,zamani2019feedback}. However, these methods are often computational expensive for high dimension hybrid systems. There exist reinforcement learning (RL) methods to control hybrid systems like legged robots~\citep{benbrahim1997biped,morimoto2004simple,neunert2020continuous,mastrogeorgiou2020slope}, but finding the appropriate RL methods and rewards are extremely challenging and may cause undesired behaviors learnt to hack for high returns~\citep{clark2016faulty}. Our method shares similar philosophy with RoA-based works~\citep{tedrake2010lqr,bhounsule2018switching,zamani2019feedback}, whereas ours is computation-efficient, suits general nonlinear hybrid systems and can represent RoAs for arbitrary number of modes.

