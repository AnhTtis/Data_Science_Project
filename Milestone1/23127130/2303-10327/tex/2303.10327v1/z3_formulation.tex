\section{Problem Formulation}

\begin{definition}[Controlled Hybrid Systems]
A \textit{controlled hybrid system} is defined as: 
\begin{equation}
    \begin{cases}
    \dot{x}=f_i(x, u; p_i), \quad x\in \mathcal{C}_i \\
    x^{+} = h_i(x, u; p_i, p_j), \quad x\in \mathcal{D}_{i,j} \\
    \end{cases}
\end{equation}
where $x\in\mathbb{R}^{n}$ denotes the system state, $u\in\mathbb{R}^m$ denotes the control input, and $p_i\in\mathcal{P}$ denotes the system configuration (e.g. the set point, reference velocity). The index $i=1,...,I$ denotes the system mode. $\mathcal{C}_i$ is the flow set where states follow continuous flow map $f_i:\mathbb{R}^n\times\mathbb{R}^m\times \mathcal{P} \to\mathbb{R}^n$, and $\mathcal{D}_{i,j}$ is the jump set where states follow discrete jump map $h_i:\mathbb{R}^n\times \mathbb{R}^m\times \mathcal{P}\times\mathcal{P}\to\mathbb{R}^n$. %All the flow sets and jump sets formulate  a partition of the state space $\mathcal{X}\subseteq \mathbb{R}^n$. 
%$i, \, i=1,...,I$ denotes the system mode %(e.g., road conditions, reference trajectories) and $S_p$ is the corresponding partition of the state space, $f_p$ denote the continuous system dynamic valid in the $S_p$ for the mode $p$ with control input $u\in\mathbb{R}^m$, and $h:\mathcal{R}^n\to\mathcal{R}^n$ maps the state before mode-switching $x^{-}$ to the new state after mode-switching $x^{+}$.
\label{def:hybrid-sys}
\end{definition}

% To stabilize a hybrid system, we first look at the stability under each subsystem $\dot{x} = f_i(x,u;p_i)$. We can express the stability condition in $x(t)$, which denotes the state at time $t$.
% \begin{definition}[Stability and Region of Attraction] 
% (1) The equilibrium  $x^*$ of a subsystem under mode $i$ is said to be \textit{Lyapunov stable} on $\mathcal{R}$ if for any $\epsilon$, there exists $\delta>0$, such that for $x(0)\in \mathcal{R}$, if $||x(0)-x^*||< \delta$, then $||x(t)-x^*||<\epsilon$ holds for all $t>0$. 
% (2) The equilibrium is said to be \textit{asymptotically stable} on $\mathcal{R}$ if it is Lyapunov stable on $\mathcal{R}$ and for all $x(0)\in\mathcal{R}$,  $\lim\limits_{t\to\infty}||x(t)-x^*||=0$. The region $\mathcal{R}$ here is called \textit{region of attraction (RoA)} for $x^*$.
% (3) The equilibrium is \textit{exponentially stable} on $\mathcal{R}$, if it is asymptotically stable on $\mathcal{R}$ and there exists $\rho>0,\gamma>0$ such that for all $x(0)\in\mathcal{R}$ and for all $t\geq 0$, $||x(t)-x^*||\leq \rho ||x(t)-x^*|| e^{-\gamma t}$.
% \label{def:stability}
% \end{definition}
% Without loss of generality, we assume $x^*$ is at the origin (if not, we can use coordinate transformation). With control Lyapunov functions (CLF), we can derive sufficient conditions to ensure the asymptotic and exponential stability.


We aim to stabilize the hybrid system in \defref{def:hybrid-sys}. For each \postac{mode of system } $\dot{x}=f_i(x,u;p_i)$ with equilibrium $x^*_i$, we consider the stability defined in~\citep{khalil2009lyapunov}. Then the sufficient conditions for \postac{each mode } system stability are as follows (we omit the index $i$ and the configuration $p$ for brevity).

\begin{proposition}[Control Lyapunov Conditions for System Stability] The system $\dot{x} = f(x,u)$ with equilibrium $x^*$ is asymptotically stable at $x^*$ if there exist a differentiable function $V: \mathcal{C}\to\mathbb{R}$ and a control policy $u=\pi(x)$ such that: $V(x^*)=0;\, \text{and}\, \forall x\in \mathcal{C} \backslash \{x^*\},\, V(x)>0,\, \text{and}\, \frac{dV}{dx} f(x, u)< 0$.
The $V$ is called a \textit{control Lyapunov function (CLF)}. The system is exponentially stable at $x^*$ if the $V$ further satisfies: $\exists \gamma>0,\, \forall x\in \mathcal{C} \backslash \{x^*\},\, \frac{dV}{dx} f(x, u)< -\gamma V$.
\label{prop-1}
\end{proposition}
The proof can be found in~\citep{isidori1985nonlinear}[Theorem 9.4.1]. If each \postac{system mode } is stable with valid Lyapunov functions, will the hybrid system converge? Unfortunately, the answer is no with a counter-example in~\citep{branicky1998multiple}. %Consider an autonomous system with dynamics $f_1(x)=Ax$ and $f_2(x)=Bx$, where 
% \begin{equation}
%     A=\begin{bmatrix} -0.1 & 1 \\ -3 & -0.1 \end{bmatrix},\quad B=\begin{bmatrix} -0.1 & 3 \\ -1 & -0.1\end{bmatrix}
% \end{equation} 
%Denote the state $x=(x_0,x_1)^T$. If we construct Lyapunov functions $V_1=x_0^2+x_1^2/3$ and $V_2=x_0^2/3+x_1^2$, we can show $\dot{x}=f_i(x)$ is exponentially stable for $i=1,2$. But the hybrid system using $f_1$ in the second and forth quadrants and $f_2$ in the first and the third quadrants is unstable. We present the system trajectories in \figref{fig:simple-demo}, where the subsystem $f_1$ starting from $(1, 0)^T$ (shown in \figref{fig:simple-demo}(a)) , and the subsystem $f_2$ starting from $(0, 1)^T$ (shown in \figref{fig:simple-demo}(b)) are converging to the origin, but the hybrid system starting from $(10^{-3}, 10^{-3})^T$ (shown in \figref{fig:simple-demo}(c)) is diverging.
The culprit is that the system switches too fast: although the Lyapunov value is decreasing in each mode, the distance toward equilibrium is not decreasing yet. %, and at the switching surface the Lyapunov function changes to $V_{j}, j\neq i$, which is always higher.
% The culprit for the diverging is at the switching surface: as shown in \figref{fig:simple-demo}, although the Lyapunov value $V_i$ is decreasing during each mode $i$, at the switching surface the Lyapunov function changes to $V_{j}, j\neq i$, which is always higher. 
For systems with jumps, it is more unlikely to ensure the stability, as there might be jumps making $||x(t)-x^*||\geq \epsilon$ infinitely often. Thus, %to make sure the hybrid system has the property of letting the states converge to the equilibrium points to certain degree, 
we propose a new stability for the hybrid systems.

\begin{definition}[Hybrid System Stability] Given $\epsilon=\{\epsilon_i\}$, a hybrid system is $\epsilon$-asymptotically stable (exponentially stable), if each \postac{system mode } is asymptotically stable (exponentially stable), and all states $\bar{x}_i$ exiting the mode $i$ are within $\epsilon_i$-ball of the $x^*$, i.e., $||\bar{x}_i-x^*||\leq \epsilon_i$. We call $\bar{x}_i$ as \textit{exiting state} (and call state $x_i$ that enters the mode $i$ as \textit{entering state}).
\label{def:e-stable}
\end{definition}
If $\epsilon$ is constant, the exiting states will be at most $\epsilon$-far away from the equilibrium. If $\epsilon$ converges to zero, the sequence of the exiting states will converge to the equilibrium \postac{hence asymptotic stability is achieved}. In this paper, we consider the former case. %The core motivation of our approach is to make sure the state after switching can always converge close enough to the equilibrium point. To define it rigorously, we introduce the concept of hybrid system stability.
%Since in hybrid systems, there might exist infinite number of jumps and the duration between jumps can be finite time,  
With constant $\epsilon$, we define $\epsilon$-RoA as follows.

\begin{definition}[$\epsilon$-Region of Attraction] The $\epsilon$-RoA for $x^*$ in mode $i$  is defined as:
% \begin{equation}
$\mathcal{R}^\epsilon = \{x_0|, x(0)=x_0, ||\bar{x}_i-x^*||\leq \epsilon \}$
    % \label{eq:e-roa}
% \end{equation}
, where $\bar{x}_i$ is the exiting state (for mode $i$) starting at $x_0$. %(not necessary the entering state).
\label{def:e-roa}
\end{definition}
\postac{Using $\epsilon$-RoA, we do not need to check dwell time conditions like~\citep{hespanha1999stability}}. To efficiently check if $x\in\epsilon$-RoA, \postac{we seek a set representation for $\epsilon$-RoA and a scalar function/index to tell whether $x$ is in the set}. We use Lyapunov level set to (conservatively) represent $\epsilon$-RoA.
 \begin{definition}[Maximum $\epsilon$-Stable Level Set] For mode $i$ and configuration $p_i$, the largest level set $\mathcal{S}^{c_i}=\{x|V_i(x, p_i)\leq c_i(p_i)\}$ within $\epsilon$-RoA is called \textit{Maximum $\epsilon$-Stable Level Set}, and $c_i(p_i)$ is called \textit{Maximum $\epsilon$-Stable Level Set index}. \label{def:e-level-set}
\end{definition}
For an entering state $x_j$, we have $V_j(x_j,p_j)\leq c_j(p_j) \to x_j\in\epsilon$-RoA. One step ahead, at mode $i$ with switching $i\to j$, it requires $V_j(h_i(\bar{x}_i,u;p_i,p_j))\leq c_j(p_j)$. Propagating from $\bar{x}_i$ to $x^*$, we derive sufficient conditions for hybrid system stability proved in \footnote{The appendix is at \url{https://mit-realm.github.io/hybrid-clf/static/appendix.pdf}}{\appref{appendix-prop2}}.

%Before entering the mode $j$, since the jump function is $x_j=h_i(\bar{x}_i, u, p_i, p_j)$, and we know at mode $i$, the dynamics is $\dot{x}_i=f_i(x_i, u; p_i)$
 
 %In the next section, we will show how to use control lyapunov functions and the $\epsilon$-RoA to ensure the $\epsilon$-stability for the hybrid systems.

% We also assume we have the freedom to choose a subset of equilibria points from all equilibria points. Now we are ready to define the stability conditions for the hybrid systems.
 
  \begin{theorem}[Lyapunov Conditions for Hybrid System $\epsilon$-Stability] A hybrid system in \defref{def:hybrid-sys} is $\epsilon$-exponentially stable, if there exists a Lyapunov function $V_i$ for each mode $i$ (and configuration $p_i$) satisfying all conditions in \propref{prop-1} and $\alpha ||x-x^*||\leq V_i(x, p_i)\leq \beta ||x-x^*||$, and for each entering state $x_i$ that moves toward the mode $j$, we have the $p_i, c_i, p_j, c_j$ to satisfy: \begin{equation}V_i(x_i, p_i) \leq c_i(p_i) \text{ and } V_j(h_i(x^*, u; p_i, p_j), p_j)\leq \Upsilon
  \label{eq:v-condition}
  \end{equation}
  where $\Upsilon=\frac{\alpha_j}{\beta_j} c_j(p_j) - \alpha_j K_i \epsilon$, and $c_i$, $c_j$ are the maximum $\epsilon$-stable level set indices for modes $i$ and $j$, and $K_i$ is the local Lipschitz constant for function $h_i$ within $\epsilon$-ball of $x^*$. 
  \label{theo-2}
  \end{theorem}
 
 In short, \theoref{theo-2} guarantees each \postac{system mode } is exponentially stable, and at switching $i\to j$, the entering state $x_i$ is within $\epsilon$-RoA for mode $i$, and the next entering state $x_j$ is also inside $\epsilon$-RoA for mode $j$. The Lipschitz constant $K_i$ can be approximated by $|\frac{\partial h_i}{\partial x}|$ at $(x^*,u;p_i,p_j)$ for small $\epsilon$. 
 %For translation function (e.g., $h_i(x,u;p_i,p_j)=x-p_i+p_j$), $K_i$ is just 1. 
 In the next section, we will use neural networks to satisfy \theoref{theo-2} for hybrid system stability.