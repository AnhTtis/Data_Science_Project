% \begin{figure}[!htbp]
% \floatconts{fig:pogo-comparisons}
% {\caption{Comparison results for the pogobot experiment.}}
% {
% \subfigure[Velocity error]{  % 27 0.18
% \includegraphics[width=0.225\textwidth]{figures/bar_pogo_v_err.png}
% }
% \subfigure[Distance to goal]{  % 27 0.18
% \includegraphics[width=0.225\textwidth]{figures/bar_pogo_goal_len.png}
% }
% \subfigure[Collision rate]{  % 50 0.30
% \includegraphics[width=0.225\textwidth]{figures/bar_pogo_hit.png} 
% }
% \subfigure[Runtime]{  % 27 0.18
% \includegraphics[width=0.225\textwidth]{figures/bar_pogo_runtime_step.png}
% }
% }
% \end{figure}

\begin{figure}[!htbp]
\floatconts{fig:pogo-comparisons}
{\caption{Quantitative ((a)$\sim$(d)) and qualitative ((e)$\sim$(h)) comparison for the pogobot. 
\postac{Our approach is the only one that can safely jump through the maze. DDPG and PPO methods start to jump to the left afterwards, and MPC results in collisions.}
}}
{
\subfigure[Velocity error]{  % 27 0.18
\includegraphics[width=0.225\textwidth]{figures/viz/bar_pogo_v_err.png}
}
\subfigure[Distance to goal]{  % 27 0.18
\includegraphics[width=0.225\textwidth]{figures/viz/bar_pogo_goal_len.png}
}
\subfigure[Collision rate]{  % 50 0.30
\includegraphics[width=0.225\textwidth]{figures/viz/bar_pogo_hit.png} 
}
\subfigure[Runtime]{  % 27 0.18
\includegraphics[width=0.225\textwidth]{figures/viz/bar_pogo_runtime_step.png}
}
\subfigure[RL method (DDPG)]{  % 27 0.18
\includegraphics[width=0.23\textwidth]{figures/pogo_sim_rl-ddpg1_0_48900.png}
}
\subfigure[RL method (PPO)]{  % 27 0.18
\includegraphics[width=0.23\textwidth]{figures/pogo_sim_rl-ppo8_0_08100.png}
}
\subfigure[MPC method]{  % 50 0.30
\includegraphics[width=0.23\textwidth]{figures/pogo_sim_mpc0_0_72900.png} 
}
\subfigure[Our approach]{  % 27 0.18
\includegraphics[width=0.23\textwidth]{figures/pogo_sim_ours_0_58000.png}
}
}
\end{figure}