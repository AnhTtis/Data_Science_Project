\section{Smart Contract Watermarking}
\label{s:challenges}

This section describes the need for a watermark technique on smart contracts, challenges, varying requirements, and a threat model with viable attacks.

\subsection{Motivation and Challenges} 

\PP{Motivation}
Due to the nature of public blockchain platforms, even if the smart contract authors do not make source code publicly available, smart contracts can be exposed in a bytecode format and could be reused by anyone. Indeed, recent empirical studies~\cite{pierro2021analysis, chen2021understanding} reveal that code reuse in smart contracts is quite prevalent. For example, Chen et al.~\cite{chen2021understanding} discovered that $26\%$ of contract code blocks had been cloned ($14.6$ occurrences on average) from the $146K$ open-sourced projects, suggesting common patterns of code duplication in smart contracts. Although the power of reusability helps a smart-contracts-driven ecosystem to be rich, it may pose a severe threat to managing smart contracts' intellectual property rights (IPR). 
He et al.~\cite{he2020characterizing} demonstrated that over 96\% of 10 million contracts had duplicates. 
Besides, they reveal a case that entails
substantial financial losses (89,565.32 ETHs, about 30\% of the original market); 73 plagiarized DApps are copied from 41 original ones. Fomo3D is one of the popular DApps with over 10,000 active users and large transactions (40,000 ETHs) back in 2018, which has been victimized by numerous copycats.
Such a growing concern motivates our work for protecting the IPR of a smart contract. However, it is non-trivial to prevent reusing existing smart contracts. A possible solution would be to develop a software watermarking~\cite{collberg1999software} scheme to provide a technical means that claims the originality of a smart contract on demand: \eg, plagiarized DApps could have been disclosed with the scheme. Further, the presence of explicit watermarks helps in establishing ownership proof over legal disputes.


\PP{Challenges}
Applying prior software watermarking techniques to a smart contract is challenging due to the characteristics of a smart contract programming language. One of the biggest hurdles is that smart contracts typically have a small size (up to 24KB), making it difficult to conceal a watermark from the original smart contract code.
Another challenge is that running a bytecode under EVM comes with a (gas) cost (\ie, Ethereum transaction fee), possibly leading to avoiding any watermarking technique unless it stays the total cost intact.
Hence, it is evident that a watermarking scheme 
with a charge would not be welcomed 
even with the presence of the technique.
Lastly, the EVM environment for a smart contract does not allow for dynamically allocated memory, disabling the adoption of existing watermarking schemes~\cite{collberg2004dynamic, collberg1999software} that utilize dynamic allocation.

\PP{Goal} 
By nature, a blockchain is designed to confirm if a certain smart contract appears for the first time. 
However, a technical means is absent to verify that a suspicious smart contract is a replica of an existing smart contract (thereby violating IPR).
In this paper, we aim to provide
such a means to address this problem.
To the best of our knowledge, we introduce the first watermarking scheme for smart contracts that should be able to tackle the aforementioned challenges.
To demonstrate the effectiveness of \sys, our evaluation focuses on answering the following research questions (RQs):

\begin{itemize}
    {\item{\bf{RQ1: }}Is \sys able to protect smart contracts from being copied effectively and efficiently?}
    {\item{\bf{RQ2: }}Is \sys sufficiently resilient against a variety of adversarial attacks?}
\end{itemize}




\subsection{Requirements}
\label{ss:requirements}
Instead of reinventing the wheel for the requirements of a watermarking scheme, we adopt the general ones as with previous software watermark 
approaches~\cite{kang2021softmark, collberg1999software, myles2005evaluation, zeng2010robust, dalla2017software, dey2019software, qu1999hiding}:
imperceptibility, spread, credibility, resiliency, 
capacity, and efficiency. Besides, we define a cost (gas consumption) as another requirement for contract watermarking. %
\begin{itemize}[leftmargin=*]
\item \emph{Imperceptibility} ensures 
that a watermark must be scarcely perceptible,
that is, a smart contract with an embedded
watermark must be indistinguishable from 
the one without.
\item \emph{Spread} denotes how well
a watermark is distributed across the
whole smart contract code.
Typically, a well-scattered watermark
tends to be resilient against 
corruption attempts.
\item \emph{Credibility} ensures that
a watermark must be reliably verifiable, minimizing false positive or negative cases.
\item \emph{Resiliency} represents the robustness of a watermarking scheme against tampering attacks that aim to invalidate a watermark, including addition, subtraction, and distortion.
\item \emph{Capacity} represents the data rate of a watermark that can be encoded into a target contract. Considering a contract size constraint, the length of a watermark cannot exceed it.
\item \emph{Efficiency} represents a performance overhead (\ie, computational resource) that is needed for watermarking operations. We separately define a gas cost metric for smart contracts.
\item \emph{Cost} represents the amount of gas consumption for inserting and validating a watermark. We utilize a gas price per individual opcode as pre-defined in~\cite{wackerow2021Online}.
\end{itemize}
\begin{figure*}[h!]
    \centering
    \includegraphics[width=\linewidth]{figures/overview_edited.pdf}
    \caption{Overall \sys scheme for embedding and verifying
    a watermark for a smart contract. 
    We carefully elect bytes (\protect\BC{2}) that
    comprises a watermark 
    from the CFG  (\protect\BC{1}) 
    of a runtime bytecode, creating a 
    watermark reference object (WRO) (\protect\BC{3}).
    The original author secretly holds the WRO,
    and computes its hash (WRO MAC).
    The WRO MAC is computed and embedded to a creation bytecode (\protect\BC{4}, \protect\BC{5}), 
    followed by deploying it on the Ethereum 
    network (\protect\BC{6}).
    By verifying the extracted WRO MAC
    (\protect\BC{7}, \protect\BC{8}) from a creation bytecode with
    the extracted bytes from
    a reconstructed CFG (\protect\BC{9}),
    a watermark
    can be verified (\protect\BC{10}) on demand.
    }
    \label{fig:4_SmartMark}
\end{figure*}

\subsection{Threat Model} 
\label{ss:threatmodel}
The objective of our watermarking scheme for a smart contract is to thwart an adversary's considerable efforts with reasonable resources rather than suggesting an unbreakable scheme (as a fully motivated attacker could hardly be prevented).
With this in mind, in this paper, 
we assume a strong adversary
who is capable of 
\WC{1}~obtaining an open bytecode for a given smart contract,
\WC{2}~understanding our watermark scheme beforehand, and 
\WC{3}~performing arbitrary code manipulation
on a decompiled source code (Section~\ref{ss:resiliency}) or bytecode, 
attempting to tamper with a watermark.
We also assume that the adversary could collect as many smart contracts as possible
for further comparisons between them
(\ie, collusive attack).




\PP{Attack Types} 
We classify five viable attacks largely into two categories: passive attacks (denoted as ``P''), including unauthorized recognition and collusion (against imperceptibility), and active attacks (denoted as ``A''), including addition, deletion, and distortion (against credibility and resiliency). 
Our design principle does not necessarily conceal the presence of a watermark embedded in smart contracts at all. Hence, recognizing such information itself would not weaken the overall security of {\sys} unless an adversary could reveal the exact location of a watermark.

\begin{itemize}[leftmargin=*]
    \item \emph{(P) Unauthorized recognition} refers to an attack that identifies the location of a watermark in a target smart contract.
    \item \emph{(P) Collusion} refers to an attack that recognizes the location of a watermark in a target contract by comparison.
    \item \emph{(A) Addition} refers to an attack that embeds another watermark
    (\ie, adversary's ownership) into a target contract. %
    \item \emph{(A) Deletion} refers to an attack that eliminates a valid watermark from a target smart contract.
    \item \emph{(A) Distortion} refers to an attack that encompasses every transformation for damaging an existing watermark.
\end{itemize}


