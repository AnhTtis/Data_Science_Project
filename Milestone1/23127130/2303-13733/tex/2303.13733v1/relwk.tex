\section{Related Work}
\label{s:relwk}


\PP{Software Copyright and Smart Contracts} 
Protecting a software copyright often helps 
to maintain productivity and motivation of software development. 
Vast studies~\cite{peace2003software, samuelson2016functionality,
reavis1991software} have been conducted on the methods to detect 
and/or prevent a software piracy. 
Lately, the rise in popularity of 
smart contracts with the blockchain technology necessitates 
a new suitable means to safeguarding an ownership  \cite{savelyev2018copyright, bodo2018blockchain}.

\PP{Smart Contract Code Reuse} 
Recent studies repeatedly show that reusing code 
in a smart contract is quite 
prevalent~\cite{jia2020similar, kondo2020code,
kiffer2018analyzing, chen2021understanding, 
pierro2021analysis}.
Chen et al.~\cite{chen2021understanding} reveal
that 91.1\% out of 52,951 smart contract projects 
gathered by Etherscan~\cite{6_Etherscan} 
(before August 2019) contain 
one or more subcontracts from others.
According to the analysis by Pierro et 
al.~\cite{pierro2021analysis}, a wide adoption
of code reuse in a smart contract arises from 
the developers' desire for building
successful Ethereum DApps and/or the lack of
a well-integrated development tool.
While code reusability aids the quick and effortless development
of a smart contract, a security bug may bring
about an unwelcome outcome~\cite{he2020characterizing}
as seen in the past incidents~\cite{Popper2016Online,
Browne2017Online, Dale2021Online, M2018Online}.
In the meantime, EClone~\cite{liu2019enabling} detects
a replication of a smart contract based on its birthmark.
Note that \sys is designed for embedding and verifying a watermark while EClone aims to measure the similarity between smart contracts without considering security.

\PP{Software Watermarking Schemes} 
A wide spectrum of software watermarking 
techniques~\cite{Peter1994, Keith1994, davidson1996method, 
kang2021softmark, collberg1999software, 
jiang2009software, qu1998analysis, 
balachandran2014function, qu1999hiding, monden2000practical, lu2014ropsteg,
collberg2003sandmark, zuo2010zero, rovcek2021zero,
wang2019ternary, ma2015software}
have been proposed to protect a software copyright.
One of simple but efficient approaches leverages
code reordering~\cite{davidson1996method, kang2021softmark}
at the level of a basic block~\cite{davidson1996method}
or a function~\cite{kang2021softmark}, which inserts
a watermark by mapping it into the order of code.
Another well-studied direction for software watermarking 
utilizes a graph theory~\cite{collberg1999software, jiang2009software, qu1998analysis, qu1999hiding} such
as a graph coloring problem~\cite{qu1998analysis}.
Collberg et al.~\cite{collberg1999software} store
a graph structure on the heap at runtime.
Meanwhile, an obfuscation scheme has been widely
adopted~\cite{balachandran2014function, monden2000practical, lu2014ropsteg, collberg2003sandmark} in the field of software
watermarking, which includes
inserting dummy methods and opaque
predicates~\cite{monden2000practical}, and
steganography~\cite{lu2014ropsteg}.
Besides, the idea of embedding a watermark into a target 
application without modification (\ie, zero 
watermarking~\cite{zuo2010zero, rovcek2021zero,
wang2019ternary})
has been introduced, however, its downside lies in
needing additional storage for bookkeeping.
Meanwhile, Ma et al.~\cite{ma2015software} introduce an 
return-oriented-programming (ROP) based watermarking scheme,
which inserts a well-crafted code to be triggered 
into a data region for verification afterward.
However, applying prior software watermarking schemes
to a smart contract is impractical  
due to its unique properties such as the restriction of code 
size (\eg, code relocation, obfuscation),
the absence of dynamic allocation (\eg, runtime operation),
and execution costs (\eg, dummy code insertion).
\sys proposes a distinct watermarking scheme
tailored to smart contracts
for the first time by addressing the above hindrances.

























