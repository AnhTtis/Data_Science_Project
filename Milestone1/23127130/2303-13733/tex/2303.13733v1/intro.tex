\section{Introduction}
\label{intro}


Due to the advancements in blockchain technologies, blockchain-based 
smart contracts (hereinafter referred to as {\em smart contracts}) 
have received significant attention from both academia and industry 
over the last few years.
A vast number of smart contracts have already been deployed on blockchains 
(\eg, over 10 millions in 2020 on the Ethereum network~\cite{he2020characterizing}). 

As a smart contract has been adopted for business, we encounter a new (but familiar) challenge in protecting it when a smart contract owner needs to claim one's intellectual property right. Since a smart contract is a type of programming code, it is inherently prone to be plagiarized.
Although a smart contract is deployed in a binary form on a blockchain
like other software distributions, its size constraint (\eg, 24KB for Ethereum) makes reverse engineering relatively less painful with the state-of-the-art tools (\eg, Erays~\cite{Zhou18:Smartcontract}, Vandal~\cite{brent18:Smartcontract}, Gigahorse~\cite{Grech19:Smartcontract}).
Recent studies~\cite{pierro2021analysis, chen2021understanding} 
reveal that a vast amount of contract code blocks had been indeed cloned.
He et al.~\cite{he2020characterizing} identified 41 decentralized applications (DApps\footnote{A DApp is a collection of smart contracts incorporated with an interface on a website or an application, to interact with users.}) with 73 plagiarized DApps, which may cause a substantial financial loss to the original DApp creators. 
Besides, they showed that careless code reuse could bring unwanted results from a security perspective.

A well-known technique for protecting a software copyright
is software watermarking, a process of embedding a watermark 
$W$ into a program $P$ such that $W$ can be further detected or extracted to assert the ownership of $P$~\cite{collberg1999software}.
In essence, the underlying mechanism is that $W$ would be present 
when one copies $P$ even with the attempt of a corruption (\eg, modification). 
A plethora of software watermarking schemes~\cite{Peter1994, Keith1994, 
 davidson1996method, kang2021softmark, collberg1999software, jiang2009software, qu1998analysis, qu1999hiding, 
 balachandran2014function, monden2000practical, lu2014ropsteg,
 collberg2003sandmark, zuo2010zero, rovcek2021zero, wang2019ternary,
 ma2015software} have been introduced against a piracy.

However, applying the existing techniques to a smart contract is not viable due to its unique properties. First, hiding a watermark within a program would be difficult because the size of a smart contract is typically smaller than that of a conventional program. Consequently, it is possible for a skillful adversary to manipulate the watermark with static code analysis.
Second, a virtual machine such as Ethereum Virtual Machine (EVM) 
offers a different enviornment from a bare machine, making it infeasible 
to adopt prior approches like the return-oriented programming (ROP)-based 
scheme~\cite{ma2015software} or the function reodering scheme~\cite{kang2021softmark}.
Third, running a bytecode under EVM inevitably incurs
a transaction cost, which restricts any scheme that introduces 
additional code or data.
Fourth, EVM does not have a feature of dynamic memory allocation,
which disables the adoption of a prior scheme~\cite{collberg2004dynamic, 
collberg1999software} to utilize that feature.

Only a few attempts have been made to protect smart contract developers' intellectual property rights. 
Zhang et al.~\cite{Zhang20:Smartcontract} present code obfuscation techniques for a smart contract, especially written in Solidity~\cite{Ethereum2022Online}, which makes it difficult to decompile a smart contract. 
Yan et al.~\cite{Yan20:Smartcontract} propose a technique to increase the difficulty level of recovering a control flow graph from a smart contract bytecode by introducing four anti-reverse engineering code patterns. 
Although it is possible to raise the bar with the above techniques, they are still far from a comprehensive solution for verifying the originality of a smart contract (\ie, Has a contract been copied (partly or in full) from another?).

In this work, we present \sys, to the best of our knowledge, the first watermarking scheme on smart contracts that considers both varying requirements of a watermark (\ie, imperceptibility, spread, credibility, resiliency, capacity, efficiency) and unique properties of smart contracts (\ie, gas cost).
At a high level, {\sys} builds a control flow graph (CFG) from the runtime bytecode of a smart contract and randomly elects a series of bytes from the blocks selected across the CFG as a watermark.\footnote{A runtime bytecode is the execution body of a smart contract, and a creation bytecode consists of both a runtime and initialization bytecode  (\ie, constructor).} Next, {\sys} creates a data structure that holds essential information (\eg, the locations of elected bytes and the CFG generation method) to extract the watermark later. {\sys} privately keeps this data structure and only inserts the hash of the data structure into the creation bytecode of the smart contract for further verification of the watermark to claim the originality of the smart contract. Note that we assume that {\sys} does not add any extra code for watermarking to the runtime bytecode of the smart contract, indicating that the smart contract's functional behavior would remain the same without incurring additional gas costs to execute the smart contract.

The main benefits of our design choice are free from \WC{1}~an undesirable gas cost for a transaction as the size of a runtime bytecode stays intact, and \WC{2}~detection techniques based on a static analysis as no additional code is introduced for a watermark.
To this end, we develop a full-fledged prototype of \sys and demonstrate how our approach fulfills the requirements of a watermarking scheme on smart contracts at an acceptable cost.
Moreover, we collected all the blockchain blocks (about nine million blocks between 30 July 2015 and 21 June 2022) from the Ethereum Mainnet, and selected 27,824 unique bytecodes from those blocks to evaluate the effectiveness and efficiency of \sys.

The contribution of our paper is summarized as follows.
\begin{itemize}
    \item We present \sys, a novel software watermarking scheme that satisfies varying requirements of watermarking for smart contracts.
    \item We empirically evaluate \sys, demonstrating its effectiveness 
    and efficiency. 
    \item We thoroughly study 
    the resiliency of our watermarking scheme against viable attacks.
\end{itemize}













