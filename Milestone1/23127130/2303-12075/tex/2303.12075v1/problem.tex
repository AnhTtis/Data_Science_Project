We study the settling of particles under the action of gravity in an 
unbounded, initially quiescent fluid.
%
We consider an incompressible Newtonian fluid of density \gls{rhof} and 
kinematic viscosity \gls{nu}.
%
Particles are oblate spheroids of equatorial diameter \gls{dd} and aspect ratio 
$\gls{chi}=\gls{dd}/\gls{aa}$, where \gls{aa} is the length of their symmetry
axis (see figures \ref{fig:problem_description}a and b).
%
The particles are assumed to be rigid with homogeneous mass density \gls{rhop},
which is larger than that of the fluid ($\gls{rhop}>\gls{rhof}$).
%
The gravitational acceleration \gls{vg} is parallel to the vertical direction 
pointing in the negative $z$-direction, $\gls{vg} = -g\gls{ez}$, where
\gls{g} is the modulus of the gravitational acceleration (see figure 
\ref{fig:problem_description}c).

\begin{figure} 
\makebox[\textwidth][c]{ 
\includegraphics[scale=1.0]{fig3.pdf} 
}
\caption{View of the $i$-th spheroid in its body-fixed reference system a) along
the symmetry axis and b) perpendicular to it. The blue line in a) and b)
represents a sphere with the same volume. c) Sketch of the problem in the global
reference system.
\label{fig:problem_description}} 
\end{figure} 

The fluid velocity $\gls{vu}=\left(\gls{vu_x},\gls{vu_y},\gls{vu_z}\right)$ is
governed by the Navier-Stokes equations for an incompressible, constant density
fluid 
%
\begin{subequations}\label{eq:gov}
\begin{align}
\ppt{\gls{vu}}+\left(\gls{vu}\cdot\nabla\right)\gls{vu}
              &= -\frac{\nabla \gls{p}}{\gls{rhof}} +\gls{nu} \nabla^2\gls{vu},
                \label{eq:gov_mom} \\
\nabla\cdot\gls{vu}&=0 \label{eq:gov_cont},
\end{align}
\end{subequations}
%
where \gls{p} is the pressure.
%
No slip and no penetration boundary conditions are imposed on the surface of each particle.
%
%
The linear and angular velocities of the particles,
$\gls{vup}=\left(\gls{vup_x},\gls{vup_y},\gls{vup_z}\right)$ and
$\gls{vomep}=\left(\gls{vomep_x},\gls{vomep_y},\gls{vomep_z}\right)$, 
respectively, are governed by the Newton-Euler equations 
%
%
%
%
%
%
%
%
%
\begin{subequations}\label{eq:motion}
\begin{align}
\gls{vp}\gls{rhop}\ddt{\gls{vup}}&=\int_{S} \gls{tau}\cdot\vec{n} \, \wrt \sigma
                            +\left(\gls{rhop}-\gls{rhof}\right)\gls{vp}\gls{vg} + \vec{F}, \label{eq:motion_lin}\\
\ddt{\left(\gls{I}\gls{vomep}\right)}&= %
 \int_{S} \gls{rs}\times \left(\gls{tau}\cdot\vec{n}\right) \wrt \sigma + \vec{T}, \label{eq:motion_rot}
\end{align}
\end{subequations}
%
%
where \gls{vp} and $S$ are the volume and the surface of the particle,
respectively, $\vec{n}$ is a unit vector normal to $S$ pointing towards the
fluid, \gls{tau} is the stress tensor ($\gls{tau}=-p\gls{eye}+\gls{rhof}\gls{nu}
\left(\nabla\gls{vu}+\nabla\gls{vu}^T\right)$), \gls{rs} is a position
vector with respect to the center of gravity of the particle and $\vec{F}$ 
and $\vec{T}$ are the solid-solid contact force and torque, respectively.
%

The problem is governed by four non-dimensional parameters, namely the
density ratio between the particles and the fluid, $\gls{kappa}=\gls{rhop}/%
\gls{rhof}$, the aspect ratio of the particles, \gls{chi}, the Galileo number,
\gls{Ga}, and the solid volume fraction, \gls{svf}.
%
The solid volume fraction is defined as $\gls{svf}=\gls{vol_part}/\left(
\gls{vol_part}+\gls{vol_fluid}\right)$ where \gls{vol_part} and \gls{vol_fluid}
represent the volume occupied by the particles and the fluid, respectively.
%
%

%
%
In this work we are interested in the dilute regime, and we set the solid volume
fraction to $\gls{svf}=5\cdot 10^{-3}$.
%
Regarding the shape of the particles, we select oblate spheroids of aspect ratio
$\gls{chi}=1.5$ which is a moderately small deviation from the spherical
reference geometry, that in turn has been extensively investigated in the past.
%
In particular, it is known that the settling regime map for oblate spheroids
with $\gls{chi}=1.5$ resembles that of a sphere \citep{moriche:2021}.
%
We fix the density ratio at $\gls{kappa}=1.5$ (which corresponds e.g.\ to some
plastic materials in water), and we select two values of the Galileo number: one
for which a single particle follows a steady vertical path ($\gls{Ga}=110.56$),
and another which leads to a single particle following a steady oblique path
($\gls{Ga}=152.02$).
%


\subsection{Definitions}
\label{sec:problem/definitions}

A velocity scale based on the gravitational acceleration, the density ratio and
the size of the particle can be defined as follows:
%
\begin{equation}\label{eq:problem/Ug}
\gls{Ug}=\sqrt{\left|\gls{kappa}-1\right|\gls{g}\gls{p:deq}}\,,
\end{equation}
%
where $\gls{p:deq}=\gls{dd}\,\gls{chi}^{-1/3}$ is the diameter of a sphere with
the same volume as the spheroid considered.
%
Based on the velocity scale \gls{Ug} \eqref{eq:problem/Ug} the Galileo number 
is defined as
%
\begin{equation}\label{eq:problem/Ga}
\gls{Ga}=\frac{\gls{Ug}\gls{p:deq}}{\gls{nu}} \,,
\end{equation}
%
and an a priori gravitational time scale can be formed as follows:
$\gls{tg}=\gls{p:deq}/\gls{Ug}$. 
%
For future reference let us define the relative velocity $\gls{vupri}=\left(
\gls{vupri_x},\gls{vupri_y},\gls{vupri_z}\right)$ of the $i$-th particle with
respect to the mean fluid velocity as:
%
\begin{equation}
\gls{vupri}(t)=\gls{vupi}(t)-\gls{vu_mf}(t),
\end{equation}
%
where the average operator $\langle \cdot \rangle_f$ indicates spatial
averaging over the entire domain occupied by the fluid $\Omega_f$, viz.
%
\begin{equation}\label{eq:avgFluid}
\langle \cdot \rangle_f =\frac{1}{\gls{vol_fluid}} \int_{\Omega_f} \left(\cdot \right)\wrt \gls{vx}.
\end{equation}
%
%
%
%
%
%
%
%
%
%
%
%
%
%
%
Let us also introduce the time-dependent settling velocity, averaged over the
set of particles as
%
\begin{equation}\label{eq:problem/defs/ws}
\gls{vup_ws}(t) = \gls{vupr_z_mp} (t) \,,
\end{equation}
%
where the average operator $\langle \cdot \rangle_p$ indicates the ensemble average
over the dispersed phase, which for a set of \gls{npart}
particles is expressed as  
%
\begin{equation}
\langle \cdot \rangle_p = \frac{ \sum_{i=1}^{\gls{npart}}%
    \left(\cdot\right)^{(i)}}{ \gls{npart} }\,.
\end{equation}
%
Similarly, the standard deviation of the vertical and horizontal components of the
linear and angular particle velocity are defined as
%
\begin{subequations} \label{eq:vomep}
\begin{align}
\gls{vup_wstd}(t) &=\gls{vupri_z_sp} \,,  \label{eq:vomep_a} \\
\gls{vup_ustd}(t) &=\frac{1}{2}\left(\gls{vupri_x_sp}+\gls{vupri_y_sp}\right) \,, \\
\gls{mp:vomep_z;std}(t) &=\gls{vomepi_z_sp} \,, \\
\gls{mp:vomep_l;std}(t) &=\frac{1}{2}\left(\gls{vomepi_x_sp}+\gls{vomepi_y_sp}\right)  \,,
\end{align}
\end{subequations}
%
where the prime symbol indicates that the quantity is the fluctuating part
of the variable, defined as
%
\begin{equation}\label{eq:problem/fluct}
\left(\cdot\right)' = \left(\cdot\right) - \langle \cdot \rangle.
\end{equation}
%
Finally, it should be mentioned that the definition of the Galileo number used
in this work is equivalent to that used in \citet{fornari:2016b} and
\citet{ardekani:2016} for spheroids, and to that of \cite{seyed-ahmadi:2021} for
cubes.
%
We have selected this definition over the one used in some works dealing
exclusively with spheroids \citep{zhou:2017,moriche:2021} in order to
recover the same value of \gls{Ga} when working with spheres ($\gls{chi}=1$).
%
%

%
For a multiparticle case with a given parameter pair ($\gls{Ga},\gls{kappa})$, 
independently of its solid volume fraction \gls{svf}, we define the reference
velocity \gls{wref} as 
%
\begin{equation}\label{eq:wref}
\gls{wref}\left(\gls{Ga},\gls{kappa}\right) = \left| \gls{vupr_z} \right| \,, 
\end{equation}
%
where the value of \gls{vupr_z} in \eqref{eq:wref} is taken from the corresponding
single particle case. 
%
Please note that for the cases considered here the single particle case regime
is steady, therefore no (temporal or statistical) averaging is needed to compute
\gls{wref} in \eqref{eq:wref}.
%
%



%
According to the above definitions of \gls{vup_ws} and \gls{wref}, we define the
average particle Reynolds number 
%
\begin{equation}\label{eq:Re}
\gls{p:Redeq}=\frac{\left|\gls{mp:vup_ws;avg}\right|\gls{p:deq}}{\gls{nu}} \,,
\end{equation}
%
and the single particle Reynolds number
%
\begin{equation}\label{eq:Re0}
\gls{p:Redeq;0}=\frac{\gls{wref}\gls{p:deq}}{\gls{nu}} \,,
\end{equation}
%
where the operator $\langle \cdot \rangle_t$ indicates temporal averaging.
%
%
%
%
%
%

%
%
%
%
%
%
%
%
%
%
%
%
%
%
%
%
%
%
%
%
%
%
%

%
%
%
%
%
%
%
%
%
%
%
%
%
%
%
%
%
%
%
%
%
%
%
%
%
%
%
%
%
%



%

%


