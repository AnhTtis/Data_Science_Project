
Here we describe the computational setup of the \gls{dkt} simulations presented
in \S~\ref{sec:results/dkt}.
%
The problem description is the same as in the multiparticle cases 
(\S~\ref{sec:problem}), but the methodology presents a few differences compared
to the one presented in \S~\ref{sec:method}. 
%
First, we impose a free stream of constant velocity at the lower boundary plane,
an advective boundary condition at the top, and periodicity at the lateral
boundaries, instead of periodicity in the three spatial directions.
%
Second the number of particles is exactly two.
%
Thus, the solid volume fraction is not a parameter anymore, and the governing 
parameters of the \gls{dkt} cases are \gls{Ga} and \gls{kappa}.
%
We explore two different particle shapes, namely spheres and oblate spheroids
with $\gls{chi}=1.5$, both with $\gls{kappa}=1.5$ and $\gls{Ga}=110.56$.
%
The size of the computational domain measures $[10.66 \times 10.66 \times 21.33]%
\gls{p:deq}^3$, where \gls{p:deq} is the diameter of a sphere with the same volume
as the particle considered.
%
Finally, for each particle shape we perform simulations with angular motion
enabled or suppressed.
%

Figure \ref{fig:DKT_sketch} shows a sketch of the computational setup, indicating
the set of initial particle positions.
%
We refer to the particle which is initially at a lower vertical position as the 
leading particle, and to the other particle as the trailing particle.
%
The initial position of the leading particle is always at $8\gls{p:deq}$ above the 
bottom boundary of the computational domain.
%
The initial condition of the trailing particle is varied, sweeping an area of
$[5\times 8.75]\gls{p:deq}^2$ in the horizontal and vertical direction, 
respectively.
%
This area is uniformly sampled leaving an horizontal and vertical distance of 
$0.625 \gls{p:deq}$ and $1.25\gls{p:deq}$, respectively, between neighboring
initial conditions.
%
%
%
%
%
%
%
%
%
This results in $72$ simulations for each configuration, a total number of 
$288$ \gls{dkt} cases.
%
In order to reduce the influence of mirror particles due to periodicity, we 
locate the plane containing the initial condition of the trailing particle in the 
plane $x=y$.
%
We define the horizontal coordinate contained in this plane as $x'$ and 
the relative position of the trailing particle with respect to the leading particle 
in this plane as
%
\begin{equation}\label{eq:dkt/xr}
\vec{x}_r = \vec{x}_\text{trailing} - \vec{x}_\text{leading}.
\end{equation}
%

%


\begin{figure} 
\begin{center}
\includegraphics[scale=1.0]{fig23.pdf} 
\end{center}
\caption{Outline of the \glsentrylong{dkt} simulations from a) lateral and c)
top views. b) View perpendicular to the plane where the trailing particle initial 
condition is located.
\label{fig:DKT_sketch}} 
\end{figure} 


%
One of the key aspects in this computational setup (non-periodic in the 
direction of gravity) is to keep the particles inside the computational domain
for sufficiently long time intervals.
%
Therefore, we need a good estimate of the average settling velocity of the 
system (both particles), and a sufficiently large domain in the vertical 
direction to accommodate the variations of this settling velocity.
%
It should be mentioned that these variations can be large due to collision
events.
%
From trial simulations we found that imposing the reference Reynolds number
based on the free stream velocity imposed at the inlet $\gls{Reref}=\gls{Uref}%
\gls{p:deq}/\gls{nu}$ slightly higher than the terminal Reynolds number obtained
for a single particle in the same configuration leads to successful simulations.
%
%
From this experience we set $\gls{Reref}=1.025\gls{p:Redeq;0}$ in all the cases.
%
The following phases of the evolution of the system are identified:
%
\begin{enumerate}
   \item {\bf Initial upward drift}: if particles are initially at a sufficient
         distance away from each other, they slowly drift upwards in the computational
         domain because $\gls{Reref}>\gls{p:Redeq;0}$.
         For cases in which collisions do not occur, this is the only phase
         of the problem. For cases in which particles are close enough to each 
         other the \gls{dkt} event is triggered since the start of the simulations, and
         this phase is skipped.
   \item {\bf \gls{dkt} event}: if the trailing particle is attracted by the wake of
         the leading one, the former drafts towards the latter and they eventually 
         collide.
         This results in an enhancement of the settling velocity of both particles.
         %
         %
         Having drifted upwards in the previous phase leaves more clearance for the
         \gls{dkt} event to occur without encountering the bottom boundary.
   \item {\bf Final upward drift}: after the \gls{dkt} event both particles end
         up at a similar height and both drift upwards while repelling each
         other.
\end{enumerate}
%
%
%
%
%
%
%
%
%
%
%
%
%
%
%

We have verified that all the non-colliding cases have been run for at least
the maximum time to first interaction observed ($678\gls{tg}$, from rotationally-locked
spheroids with relative initial position of the trailing particle $(3.125,10)\gls{p:deq}$)
plus $100\gls{tg}$.
%
This results in all the cases simulated for at least $778\gls{tg}$.

%


Figure \ref{fig:DKT_timehistory} shows the time history of two \gls{dkt} 
simulations as an example. 
%
In both cases the angular motion is enabled and the initial condition of the 
trailing particle is $(x'_r/\gls{p:deq},z_r/\gls{p:deq})=(2.5,7.5)$.
%
For every simulation in which a collision takes place, we define the time 
to first collision, \gls{mp:tkiss}, and the interaction time \gls{mp:tinter}.
%
The former is the time between the begin of the simulation and the first
collision.
%
The latter is defined as the time between the first collision and the last 
time instant in which the particle centers approach each other.
%
The definition of these two quantities is indicated in figure 
\ref{fig:DKT_timehistory}.
%
Please note that for most of the cases of spheres with angular motion enabled
the interaction time is almost negligible (see figure 
\ref{fig:DKT_timehistory}c).
%
\begin{figure} 
\makebox[\textwidth][c]{ 
%
\includegraphics[scale=1.0]{fig24.pdf} 
}
\caption{Time history of the vertical positions of a) spheres and b) spheroids
of $\gls{chi}=1.5$ and the distance between particle centers (c,d).
%
In both cases $(x_r/\gls{p:deq},y_r/\gls{p:deq})=(2.5,7.5)$ and the time is shifted so
that the instant of the first collision is $t=0$.
%
In (a-b) the trailing particle is represented with a solid line and the leading particle
with a dashed line. 
%
The gray shading indicates the time interval in which particles are in contact.
%
The vertical dotted lines illustrate the definition of the time to first collision,
\gls{mp:tkiss}, and the interaction time, \gls{mp:tinter}.
%
%
\label{fig:DKT_timehistory}} 
\end{figure} 

