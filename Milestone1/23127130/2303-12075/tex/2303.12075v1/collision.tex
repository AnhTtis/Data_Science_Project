
%
Here we describe the algorithm used to handle particle-particle contact in this
work. 
%
For the present case of a dilute suspension we adopt a simple repulsion model in
which contact forces are determined from the distance separating a given
particle pair.
%
Each contact event involves only a pair of particles, and the resultant contact force 
is assumed to be a point force.
%
Hence we need to define a contact point, a direction and a force intensity.
%
In this work we consider only normal forces with a quadratic law similar to the
one used in \cite{uhlmann:2014a} for spherical particles, originally proposed by
\cite{glowinski:2001}.
%


%
The main issue when working with non-spherical particles is that the contact
point and the normal direction are not uniquely defined.
%
The most popular methods to determine the contact parameters between spheroids
in the literature are the common normal \cite{lin:1995} and the geometric 
potential \citep{ng:1994}.
%
The common normal method is very attractive since it naturally yields the contact
point and the normal direction.
%
However, the resultant system of equations is under-determined and undesired solutions
can be obtained.
%
\cite{kildashti:2018} overcame this issue by an iterative process.
%
There is, however, a non-solved issue which arises when the overlapping distance
between the spheroids is exactly zero, or very small, leading to an
ill-determined system.
%
On the other hand, the geometric potential approach is particularly attractive
when dealing with simple geometries like spheroids, in which the contact point 
is easily determined.
%
The main drawback of the geometric potential is the definition of the normal
direction.
%
In this work we propose to use the geometric potential to determine the contact
point, but determine the normal direction with a slight modification of the 
algorithm originally proposed by \cite{ng:1994}.
%




In order to apply the geometric potential method we consider the \gls{cfr} of a
spheroid using the potential \gls{c:pot} defined as
%
\begin{equation}\label{eq:cfr}
\gls{c:pot}(x,y,z) = \gls{c:cfrA}x^2 + \gls{c:cfrB}y^2 + \gls{c:cfrC}z^2 
       +2\gls{c:cfrF}yz  +2\gls{c:cfrG}zx  +2\gls{c:cfrH}xy 
       +2\gls{c:cfrP}x   +2\gls{c:cfrQ}y   +2\gls{c:cfrR}z  + \gls{c:cfrD} \,,
\end{equation}
%
where the coefficients $\{\gls{c:cfrA}, \gls{c:cfrB}, \gls{c:cfrC}, \gls{c:cfrD},
\gls{c:cfrF}, \gls{c:cfrG}, \gls{c:cfrH}, \gls{c:cfrP}, \gls{c:cfrQ}, \gls{c:cfrR} \}$
are functions of the spheroid parameters (equatorial diameter $\gls{dd}$ and 
symmetry axis length $\gls{aa}$) and the particle's position and orientation.
%
For a point on the surface of the spheroid $\gls{c:pot}=0$.
%
The coefficients in \eqref{eq:cfr} are easily obtained from the \gls{cfr} of
the spheroid expressed in the body-fixed reference system
%
\begin{equation}\label{eq:cfr_body}
\gls{c:pot}(\gls{Obody_x},\gls{Obody_y},\gls{Obody_z}) = 
       \left(\frac{\gls{Obody_x}}{\gls{dd}/2+\gls{dx}/2}\right)^2 
     + \left(\frac{\gls{Obody_y}}{\gls{dd}/2+\gls{dx}/2}\right)^2 
     + \left(\frac{\gls{Obody_z}}{\gls{aa}/2+\gls{dx}/2}\right)^2 - 1  \,,
\end{equation}
%
where $(\gls{Obody_x},\gls{Obody_y},\gls{Obody_z})$ are the coordinates of a point 
\gls{Obody_vx} expressed in the body-fixed
coordinate system (see figure \ref{fig:problem_description}a) and where we have 
included a force range of $\gls{dx}/2$  in order to minimize the effect of overlapping
support of the diffuse interface during particle approach (see figure \ref{fig:contact_detail}b).
%
After some algebra  and using the relation 
$\gls{Obody_vx} = \gls{rotmat}\left(\gls{Oxyz_vx}-\gls{vxp}\right)$,
where \gls{rotmat} is the rotation matrix to obtain body-fixed coordinates from
the global coordinate system, one reaches to
%
\begin{subequations}
\begin{align}
\gls{c:cfrA} &= \sum_{i} U_i\gls{rotmat_i1}^2 \,,\\
\gls{c:cfrB} &= \sum_{i} U_i\gls{rotmat_i2}^2 \,,\\
\gls{c:cfrC} &= \sum_{i} U_i\gls{rotmat_i3}^2 \,,\\
\gls{c:cfrF} &= \frac{1}{2}\sum_{i} U_i\gls{rotmat_i2}\gls{rotmat_i3} \,,\\ 
\gls{c:cfrG} &= \frac{1}{2}\sum_{i} U_i\gls{rotmat_i3}\gls{rotmat_i1} \,,\\ 
\gls{c:cfrH} &= \frac{1}{2}\sum_{i} U_i\gls{rotmat_i1}\gls{rotmat_i2} \,,\\ 
\gls{c:cfrP} &=-\frac{1}{2}\sum_{r}\gls{vxp_r}\sum_{i} U_i\gls{rotmat_ir}\gls{rotmat_i1} \,,\\ 
\gls{c:cfrQ} &=-\frac{1}{2}\sum_{r}\gls{vxp_r}\sum_{i} U_i\gls{rotmat_ir}\gls{rotmat_i2} \,,\\ 
\gls{c:cfrR} &=-\frac{1}{2}\sum_{r}\gls{vxp_r}\sum_{i} U_i\gls{rotmat_ir}\gls{rotmat_i3} \,,\\ 
\gls{c:cfrD} &= \sum_{s}\gls{vxp_s}\sum_{r}\gls{vxp_r}\sum_{i} U_i\gls{rotmat_ir}\gls{rotmat_is} - 1 \,,
\end{align}
\end{subequations}
%
where $\vec{U} = \left(\left(\gls{dd}/2+\gls{dx}/2\right)^{-2},
                 \left(\gls{dd}/2+\gls{dx}/2\right)^{-2},
                 \left(\gls{aa}/2+\gls{dx}/2\right)^{-2}\right)$.
%
%
In the following we introduce the subscript $1$ or $2$ to identify each of the
spheroids participating in a collision, which are defined by their potentials
$\gls{c:pot}_1$ and $\gls{c:pot}_2$.
%
The contact point is defined as the midpoint between the deepest point of 
spheroid $1$ in $2$, \gls{c:gp_deep_1}, and the deepest point of spheroid $2$
in $1$, \gls{c:gp_deep_2}.
%
%
%
%
%
%
%
To obtain the deepest point of $1$ in $2$ we minimize the function $\mathcal{L}=
\gls{c:pot}_2 + \gls{c:gp_la}\gls{c:pot}_1$.
%
The following linear system is obtained
%
\begin{equation}\label{eq:lagrange}
   \left[
   \begin{array}{ccc}
   \gls{c:cfrA}_2 + \gls{c:gp_la}\gls{c:cfrA}_1 & \gls{c:cfrH}_2 + \gls{c:gp_la}\gls{c:cfrH}_1 & \gls{c:cfrG}_2 + \gls{c:gp_la}\gls{c:cfrG}_1 \\
   \gls{c:cfrH}_2 + \gls{c:gp_la}\gls{c:cfrH}_1 & \gls{c:cfrB}_2 + \gls{c:gp_la}\gls{c:cfrB}_1 & \gls{c:cfrF}_2 + \gls{c:gp_la}\gls{c:cfrF}_1 \\
   \gls{c:cfrG}_2 + \gls{c:gp_la}\gls{c:cfrG}_1 & \gls{c:cfrF}_2 + \gls{c:gp_la}\gls{c:cfrF}_1 & \gls{c:cfrC}_2 + \gls{c:gp_la}\gls{c:cfrC}_1
   \end{array}
	\right] \left[
		\begin{array}{c}
                       x_{\gls{c:gp_la}} \\
                       y_{\gls{c:gp_la}} \\
		       z_{\gls{c:gp_la}} 
		\end{array}
    	\right] = -\left[
		\begin{array}{c}
                        \gls{c:cfrP}_2 + \gls{c:gp_la}\gls{c:cfrP}_1 \\
                        \gls{c:cfrQ}_2 + \gls{c:gp_la}\gls{c:cfrQ}_1 \\
                        \gls{c:cfrR}_2 + \gls{c:gp_la}\gls{c:cfrR}_1 
		\end{array}
	\right].
\end{equation}%
%
We can obtain a solution for the linear system \eqref{eq:lagrange} in terms of 
\gls{c:gp_la} with Cramer's rule
%
\begin{equation}\label{eq:cramer}
x_{\gls{c:gp_la}} = \frac{\gls{c:gp_D1}}{\gls{c:gp_D}},  \quad
y_{\gls{c:gp_la}} = \frac{\gls{c:gp_D2}}{\gls{c:gp_D}},  \quad
z_{\gls{c:gp_la}} = \frac{\gls{c:gp_D3}}{\gls{c:gp_D}},  
\end{equation}
%
where \gls{c:gp_D}, \gls{c:gp_D1}, \gls{c:gp_D2} and \gls{c:gp_D3} are the determinants 
of the matrix of the linear system \eqref{eq:lagrange}.
%
Substituting \eqref{eq:cramer} in the potential of spheroid 1 leads to a sixth order 
polynomial for \gls{c:gp_la}
%
\begin{equation}\label{eq:sextic}
\gls{c:pot}_1\left(x_{\gls{c:gp_la}},y_{\gls{c:gp_la}},z_{\gls{c:gp_la}}\right) = 0.
\end{equation}
%
Now, we define \gls{c:gp_la_min} as the real-valued solution out of the set 
$\gls{c:gp_la}_i$ ($i=1,6$) for which $\gls{c:pot}(\gls{c:gp_la_min}) = 
\min_i \gls{c:pot}(\gls{c:gp_la}_i)$.
%
%
The deepest point of spheroid 1 inside spheroid 2 is $\gls{c:gp_deep_1}%
=(x_{\gls{c:gp_la_min}},y_{\gls{c:gp_la_min}},z_{\gls{c:gp_la_min}})$.
%
If $\gls{c:pot}_2(\gls{c:gp_deep_1}) > 0$, then contact does not exist, and the 
computation of \gls{c:gp_deep_2} is skipped.
%
If $\gls{c:pot}_2(\gls{c:gp_deep_1}) = 0$, $\gls{c:gp_deep_2}=
\gls{c:gp_deep_1}$.
%
If $\gls{c:pot}_2(\gls{c:gp_deep_1}) < 0$, the point \gls{c:gp_deep_2} is
obtained analogously to \gls{c:gp_deep_1}.
%
In the event of collision, the contact point \gls{c:gp_cp} is defined as 
the arithmetic average position
$\gls{c:gp_cp} = (\gls{c:gp_deep_1}+\gls{c:gp_deep_2})/2$.

%
The normal direction to the contact, \gls{c:nn}, according to the original
method is defined by the line connecting the points \gls{c:gp_deep_1} and 
\gls{c:gp_deep_2}, $\gls{c:nn} =\frac{\gls{c:gp_deep_2}-\gls{c:gp_deep_1}}%
{\left\|\gls{c:gp_deep_2}-\gls{c:gp_deep_1}\right\|}$.
%
However, the line connecting the two deepest points is not (in general) aligned
with the direction obtained from the common normal method.
%
Therefore, we propose to define the normal direction at the contact point as 
$\gls{c:nn} = (\gls{c:gp_nn_1} - \gls{c:gp_nn_2})/2$, where \gls{c:gp_nn_1} 
(\gls{c:gp_nn_2}) represents the unitary normal vector to spheroid 1 (2) at point
\gls{c:gp_deep_1} (\gls{c:gp_deep_2}).
%
It should be noted that \gls{c:nn} is not exactly normal to any of spheroids
1 or 2.
%


Finally, the modulus of the normal contact force is defined as
$F_n =\delta^2/J$, where $\delta$ is the overlapping distance ($\delta = %
\gls{c:nn} \cdot (\gls{c:gp_deep_1}-\gls{c:gp_deep_2})$) and $J$ is a constant
whose value depends on the submerged weight of a single particle, $W_s$, as
$J\,W_s/D^2\approx2.5\cdot 10^{-4}$
%
(we have checked that our DKT results are not sensitive to the precise choice of
the stiffness constant $J$).
%
This leads to the following forces and torques on each of the colliding two particles in a pair:
\begin{subequations}
\begin{align}
\gls{c:F1} &= -\gls{c:F2} = -F_n\gls{c:nn}\\
\gls{c:T1} &= \gls{rs:1} \times  \gls{c:F1}  \\
\gls{c:T2} &= \gls{rs:2} \times  \gls{c:F2}  \, .
\end{align}
\end{subequations}
%
Please note that, compared to spheres, normal forces can generate torque 
in the non-spherical case (see figure \ref{fig:contact_detail}a).
%
Furthermore, the pair of torques does not need be equal in magnitude.

\begin{figure} 
\makebox[\textwidth][c]{ 
%
\includegraphics[scale=1.3]{./fig20.pdf} 
}
\caption{a) Sketch of the two spheroids indicating the contact force and torque
in each particle and the normal direction at the contact point.
b) Sketch of the elements involved in determining the contact point
(\gls{c:gp_cp}), the normal direction at the contact point (\gls{c:nn}) and
the overlapping distance ($\delta$).
The normal direction given by the original method  \citep[][%
$\vec{n}_c^{(\text{orig.})}$ in panel b %
]{ng:1994} is also including for comparison purposes.
\label{fig:contact_detail}} 
\end{figure} 




%
%
%
%
%
%
%
%
%
%
%
%
%
%
%
%


%
%
%
%
%
%
%
%
%
%

%
%

