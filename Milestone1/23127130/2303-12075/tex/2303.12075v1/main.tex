%
%
%
\documentclass[authoryear]{article}

\usepackage{amsmath}
\usepackage{amssymb}
%
\usepackage{graphicx}
\usepackage{natbib}
\usepackage{hyperref}
\usepackage{ifthen}

\usepackage{soul}


\usepackage{natbib}
\bibliographystyle{abbrvnat}
%\setcitestyle{authoryear,open={(},close={)}}
\setcitestyle{abbrvnat,open={(},close={)}}

\usepackage[left=3cm,bottom=4cm,right=3cm,top=4cm]{geometry}

\def\affilKar{Institute for Hydromechanics, Karlsruhe Institute of Technology,
              Germany}
\def\affilMad{Universidad Carlos III de Madrid, Spain }
\def\affilTUW{Institute of Fluid Mechanics and Heat Transfer, TU Wien, 1060 Vienna, Austria }

\title{On the clustering of low-aspect-ratio oblate spheroids 
settling in ambient fluid}

\date{{\it Accepted in JFM 2023}}

%
%
%
%
%

%
%
%

\author{Manuel Moriche$^*$\footnotemark[2],
  Daniel Hettmann\footnotemark[2],
  Manuel Garc\'ia-Villalba\footnotemark[1]
 and Markus Uhlmann\footnotemark[2] \\
{\small
\footnotemark[2] \affilKar{}} \\
{\small
\footnotemark[1] \affilTUW{}} \\
{\small
corresponding author: {\tt manuel.moriche@tuwien.ac.at} }
}


%
%
%
%
\usepackage{booktabs}
\usepackage{pgfplots}
\pgfplotsset{compat=1.17}
\usepackage{myglossaries}
\glsdisablehyper


%
%
%
%
%
\newcommand{\revision}[2]{#2}
\newcommand{\revisions}[1]{}
%
%
\newcommand{\proofedit}[2]{{\color{red}#2}}
\newcommand{\proofedits}[1]{{\color{red}\sout{#1}}}
%
%
%
%
%
%
%
%

%
\begin{document}

\maketitle
%
\begin{abstract}

We have performed particle-resolved direct numerical simulations
of many heavy non-spherical particles settling under gravity in the dilute regime.
%
The particles are oblate spheroids of aspect ratio $1.5$  and density ratio
$1.5$.
%
Two Galileo numbers are considered, namely $111$ and $152$, for which a
single oblate spheroid
follows a steady vertical and a steady oblique path, respectively.
%
In both cases, a strongly inhomogeneous spatial
distribution of the disperse phase in the form of columnar clusters is
observed, with a significantly enhanced average settling velocity as a
consequence.
%
Thus, in contrast to previous results for spheres,
the qualitative difference in the single particle regime does not result in
a qualitatively different behavior of the many-particle cases.
%
In addition, we have carried out an analysis of
pairwise interactions of particles in the well-known drafting-kissing-tumbling
setup, for oblate spheroids of aspect ratio
$1.5$ and for spheres.
%
We have varied systematically the relative initial position between the particle
pair and we have considered free-to-rotate particles and rotationally-locked ones.
%
We have found that the region of attraction for both particle shapes, with and
without rotation, is very similar. However, significant differences occur
during the drafting and tumbling phases.
%
In particular, free-to-rotate spheres present longer drafting phases
and separate quickly after the collision.
%
Spheroids remain close to each other for longer times after the collision,
and free-to-rotate ones experience two or more collision events.
%
Therefore, we have observed a shape-induced increase in the
interaction which might explain the increased tendency to cluster
of the many-particle cases.

\end{abstract}

%

keywords:
{\it
particle-laden flows, spheroids, clustering, collective effects, direct numerical 
simulation, immersed boundary method
}

%

\section{Introduction}
\label{sec:intro}
\section{Introduction}

The increasing complexity of source code poses a key challenge to the reliability of large-scale software systems. Software bugs in these systems can lead to safety issues~\cite{bug_safety} for users around the world as well as cause non-negligible financial losses~\cite{bug_loss}. As such, developers have to spend a large amount of time and effort on bug fixing. Consequently, \aprfull (\apr), designed to automatically generate patches to fix software bugs, has attracted wide attention from both academia and industry~\cite{long2016prophet, legoues2012genprog, long2015spr, lou2020can, tufano2018empstudy}. 


To achieve \apr, one popular approach is known as Generate-and-Validate (G\&V)~\cite{qi2015gv, ghanbari2019prapr, lou2020can, le2016hdrepair, legoues2012genprog, wen2018capgen, hua2018sketchfix, martinez2016astor, koyuncu2020fixminder, liu2019tbar, liu2019avatar}, which is typically based on the following pipeline: First, fault localization techniques~\cite{wong2016fl, abreu2007ochiai, zhang2013injecting, papadakis2015metallaxis, li2019deepfl, li2017transforming} are applied to determine the suspicious locations in programs where bugs are likely to exist. Then, the buggy locations are used by the \apr tools to generate a list of patches that replace buggy lines with correct lines. Afterward, each patch is validated against the original test suite to identify any \emph{plausible patches} (i.e., passing all tests in the test suite). Finally, to determine the \emph{correct patches}, developers examine the list of plausible patches to see if any of them can correctly fix the bug. 

Traditional \apr tools can mainly be categorized into heuristic-based~\cite{legoues2012genprog, le2016hdrepair, wen2018capgen}, constraint-based~\cite{mechtaev2016angelix, le2017s3, demacro2014nopol, long2015spr} and \template~\cite{ghanbari2019prapr, hua2018sketchfix, martinez2016astor, liu2019tbar, liu2019avatar}. Among these traditional tools, \template \apr tools~\cite{ghanbari2019prapr, liu2019tbar, benton2020effectiveness} have been able to achieve state-of-the-art results. \Template \apr tools typically leverage pre-defined templates (e.g., adding a nullness check) for bug fixing. However, since these fix templates are typically handcrafted, the number and types of bugs they are able to fix can be limited. 



To address the limitations of traditional \apr, researchers have proposed various \learning \apr tools~\cite{li2020dlfix, chen2018sequencer, jiang2021cure, lutellier2020coconut, zhu2021recoder, ye2022rewardrepair} based on the \nmtfull (\nmt) architecture~\cite{sutskever2014mt} where the input is the buggy code snippets and the goal is to translate the buggy code snippets into a fixed version. To accomplish this, \learning \apr tools require supervised training datasets with pairs of both buggy and fixed code snippets in order to learn how to perform this translation step. These training data are usually obtained by mining historical bug fixes using heuristics/keywords~\cite{dallmeier2007benchmark}, which can be imprecise for identifying bug-fixing commits; even the actual bug-fixing commits can include irrelevant code changes, leading to further pollution in the dataset~\cite{xia2022alpharepair}.
% 
Moreover, it can be hard for such \apr tools to generalize and fix bug types unseen during training. 



To better leverage recent advances in \plmfull{s} (\plm{s}), researchers~\cite{xia2022alpharepair, xia2023repairstudy, kolak2022patch, prenner2021codexws} have directly applied \plm{s} to generate patches without bug-fixing datasets. These \llm-based \apr tools work by either directly generating a complete code function~\cite{prenner2021codexws, xia2023repairstudy} or predict/infill the correct code snippet given its surrounding context~\cite{xia2022alpharepair, xia2023repairstudy}. By directly using \llm{s} that are pre-trained on billions of open-source code snippets, \llm-based \apr tools can achieve state-of-the-art performance on many repair datasets~\cite{xia2022alpharepair}. 


% 
%
%

Traditional \apr tools have long used the insight of the \emph{plastic surgery hypothesis}~\cite{barr2014plastic} where it states that the code ingredients to fix a bug already exist within the same project. Traditional \apr tools have manually designed pattern-~\cite{ghanbari2019prapr, saha2017elixir} or heuristic-based~\cite{jiang2018simfix, legoues2012genprog} approaches to finding and using such relevant code ingredients to generate fixes for bugs. However, the plastic surgery hypothesis has been largely ignored in \llm-based \apr. In fact, \llm provides a unique opportunity to fully automate the plastic surgery hypothesis idea via fine-tuning (learning project-specific information via model updates from the buggy project) and prompting (directly providing relevant code ingredients to the model), and make it directly applicable to different languages (since the \llm{s} are typically multi-lingual).%
Moreover, despite the intensive manual efforts involved, traditional \apr tools still cannot fully leverage project-specific information due to large search space for leveraging/composing existing code ingredients. In contrast, the project-specific information can effectively leveraged by \llm{s} due to their power in code understanding/vectorization, e.g., even partial/imprecise information may still guide \llm{s} in correct patch generation!
 To this end, we ask the question: \emph{How useful is the plastic surgery hypothesis in the era of \plm{s}}?








\mypara{Our Work.} To answer the question, we present \ourtech{\xspace} -- a \llm-based approach that automatically utilizes the plastic surgery hypothesis by systematically combining multiple fine-tuning and prompting strategies for \apr. \ourtech fine-tunes \plm{s} using two novel domain-specific training strategies: \textbf{\epfinetune} -- we fine-tune using the original buggy project by aggressively masking out a high percentage of tokens, which allows \plm to learn project-specific code tokens and programming styles; and \textbf{\rofinetune} -- which only masks out a single continuous code sequence per training sample, allowing the model to get used to the final \csapr task of predicting a single continuous code sequence. Furthermore, we directly leverage the ability for \plm{s} to understand natural language instructions and introduce a novel prompting strategy, \textbf{\idprompting}, which uses information retrieval and static analysis to obtain a list of relevant identifiers for the buggy lines. While such relevant identifiers are critical for fixing some difficult bugs, they may not be seen by the \llm during inference due to limited context window size. Through the use of prompting, we directly tell the model to use these extracted identifiers (relevant code ingredients) to generate the correct code. Finally, to perform repair, we combine all four model variants (including the base model, both fine-tuned models and the base model with prompting) for the final repair.





While our insight of leveraging the plastic surgery hypothesis for \llm-based \apr is generalizable across different types of \plm{s}, to implement \ourtech, we choose a recent \plm{\xspace}, \ctfive~\cite{wang2021codet5}, which is pre-trained on millions of open-source code snippets. \ctfive is an encoder-decoder model trained using \mspfull (\msp) objective where a percentage of tokens are masked out and each continuous masked token sequence is referred to as a masked span. Also, although we only extract relevant identifiers from the current buggy project (since this paper focuses on the plastic surgery hypothesis), our work can be easily extended to obtain other code information (such as relevant statements or functions) from other sources, such as  the massive pre-training corpora~\cite{husain2020codesearchnet} or historical bug-fixing datasets~\cite{jiang2019infer}, which can provide more coding knowledge for \llm{s}. Besides, although we mainly focus on using traditional string comparison algorithms for information retrieval in this paper, these techniques can be easily replaced by other frequency-based retrieval~\cite{robertson2009probabilistic} and neural search (or embedding-based search)~\cite{reimers2019sentence}.
  In summary, this paper makes the following contributions:


%


\begin{itemize}[noitemsep, leftmargin=*, topsep=0pt]
    \item \textbf{Dimension.} This paper is the first to revisit the important plastic surgery hypothesis in the era of \llm{s}. It opens up a new dimension for \llm-based \apr to incorporate previously neglected information from the buggy project itself to boost \apr performance. Furthermore, it demonstrates the promising future of retrieval-based prompting for modern \llm-based \apr.
    \item \textbf{Implementation.} We implement \ourtech based on the recent \ctfive model. We augment the model using two novel fine-tuning strategies: \epfinetune and \rofinetune, along with a novel prompting strategy based on information retrieval and static analysis: \idprompting. We combine the patches generated by all four models together and perform patch ranking to speed up \apr.% 
    \item \textbf{Evaluation Study.} We conduct an extensive evaluation against state-of-the-art \apr tools. On the widely studied \dfj 1.2 and 2.0 datasets~\cite{just2014dfj}, \ourtech is able to achieve the new state-of-the-art results of 89 and 44 correct bug fixes (15 and 8 more than best baseline) respectively.  Furthermore, we perform a broad ablation study to justify our design. \ourtech demonstrates for the first time that the plastic surgery hypothesis can substantially boost \llm-based \apr and advance state-of-the-art \apr, while being fully automated and general. Moreover, even partial/imprecise code ingredients may still effectively guide \llm{s} for \apr!
\end{itemize}



%
\section{Problem description}
\label{sec:problem}
\section{The Semi-Oblivious Chase Procedure}\label{sec:semi}
%

The semi-oblivious chase (or simply chase) takes as input a database $D$ and a set $\dep$ of TGDs, and constructs an instance that contains $D$ and satisfies $\dep$.
%
A central notion in this context is that of trigger.
%are those of trigger, active trigger, and trigger application.

\begin{definition}%[\textbf{Trigger Application}]
	Given a set $\dep$ of TGDs and an instance $I$, a {\em trigger} for $\dep$  on $I$ is a pair $(\sigma,h)$, where $\sigma \in \dep$ and $h$ is a homomorphism from $\body{\sigma}$ to $I$.
	%
	The {\em result} of $(\sigma,h)$, denoted $\result{\sigma}{h}$, is the set $\mu(\head{\sigma})$, where $\mu : \var{\head{\sigma}} \ra \ins{C} \cup \ins{N}$ is defined as follows:
	%
	%$\mu(x) = h(x)$ if $x \in \fr{\sigma}$, and $\mu(x) = \bot_{\sigma,h_{|\fr{\sigma}}}^{x}$ otherwise,
	\[
	\mu(x)\
	=\ \left\{
	\begin{array}{ll}
	h(x) & \quad \text{if } x \in \fr{\sigma}\\
	&\\
	\bot_{\sigma,h_{|\fr{\sigma}}}^{x} & \quad \text{otherwise}
	\end{array} \right.
	\]
	where $\bot_{\sigma,h_{|\fr{\sigma}}}^{x} \in \ins{N}$.  Let $T(\dep,I)$ be the set of triggers for $\dep$ on $I$.	\hfill\markfull
\end{definition}




Observe that in the definition of $\result{\sigma}{h}$, each existentially quantified variable $x$ of $\head{\sigma}$ is mapped by $\mu$ to a null value of $\ins{N}$ whose name is uniquely determined by the trigger $(\sigma,h)$ and the variable $x$ itself. This means that, given a trigger $(\sigma,h)$, we can unambiguously construct the set of atoms $\result{\sigma}{h}$.
%
The central idea of the chase is, starting from a database $D$, to exhaustively apply triggers for the given set $\dep$ of TGDs on the instance constructed so far.
%
More precisely, given a database $D$ and a set $\dep$ of TGDs, let
\[
\mathsf{chase}^{0}(D,\dep)\ =\ D,
\]
and for each $i>0$, let
\[
\mathsf{chase}^{i}(D,\dep)\ =\ \mathsf{chase}^{i-1}(D,\dep)\ \cup\ \bigcup_{(\sigma,h) \in S} \result{\sigma}{h},
\]
where $S = T(\dep,\mathsf{chase}^{i-1}(D,\dep))$. 
%
We finally define {\em the result of the chase of $D$ w.r.t.~$\dep$} as the (possibly infinite) instance
\[
\chase{D}{\dep}\ =\ \bigcup_{i \geq 0} \mathsf{chase}^{i}(D,\dep).
\]


\ignore{
The semi-oblivious chase procedure (or simply chase) takes as input a database $D$ and a set $\dep$ of TGDs, and constructs an instance that contains $D$ and satisfies $\dep$.
%
Central notions in this context are those of trigger, active trigger, and trigger application.

\begin{definition}%[\textbf{Trigger Application}]
	Given a set $\dep$ of TGDs and an instance $I$, a {\em trigger} for $\dep$  on $I$ is a pair $(\sigma,h)$, where $\sigma \in \dep$ and $h$ is a homomorphism from $\body{\sigma}$ to $I$.
	%
	The {\em result} of $(\sigma,h)$, denoted $\result{\sigma}{h}$, is the set $\mu(\head{\sigma})$, where $\mu : \var{\head{\sigma}} \ra \ins{C} \cup \ins{N}$ is defined as follows:
	%
	%$\mu(x) = h(x)$ if $x \in \fr{\sigma}$, and $\mu(x) = \bot_{\sigma,h_{|\fr{\sigma}}}^{x}$ otherwise,
	\[
	\mu(x)\
	=\ \left\{
	\begin{array}{ll}
	h(x) & \quad \text{if } x \in \fr{\sigma}\\
	&\\
	\bot_{\sigma,h_{|\fr{\sigma}}}^{x} & \quad \text{otherwise}
	\end{array} \right.
	\]
	where $\bot_{\sigma,h_{|\fr{\sigma}}}^{x}$ is a null value from $\ins{N}$.
	%
	The trigger $(\sigma,h)$ is {\em active} if $\result{\sigma}{h} \not\subseteq I$.
	%
	The {\em application} of $(\sigma,h)$ to $I$ returns the instance $J = I \cup \result{\sigma}{h}$ and is denoted as $I \app{\sigma}{h} J$.
	\hfill\markfull
\end{definition}


Observe that in the definition of $\result{\sigma}{h}$ above, each existentially quantified variable $x$ of $\head{\sigma}$ is mapped by $\mu$ to a null value of $\ins{N}$ whose name is uniquely determined by the trigger $(\sigma,h)$ and the variable $x$ itself. This means that, given a trigger $(\sigma,h)$, we can unambiguously extract the set of atoms 
$\result{\sigma}{h}$.



%\medskip

%\noindent
%\textbf{Semi-Oblivious Chase.}
The central idea of the chase is, starting from a database $D$, to exhaustively apply active triggers for the given set $\dep$ of TGDs on the instance constructed so far. This is formalized via the notion of (semi-oblivious) chase derivation, which can be finite or infinite.


\begin{definition}
	Consider a database $D$ and a set $\dep$ of TGDs.
	%We consider the two cases where a derivation is finite or infinite:
	\begin{itemize}
		\item A finite sequence $(I_i)_{0 \leq i \leq n}$ of instances, with $D = I_0$ and $n \geq 0$, is a {\em chase derivation} of $D$ w.r.t.~$\dep$ if, for each $i \in \{0,\ldots,n-1\}$, there is an active trigger $(\sigma,h)$ for $\dep$ on $I_i$ with $I_i \app{\sigma}{h} I_{i+1}$, and there is no active trigger for $\dep$ on $I_n$. The {\em result} of such a chase derivation is the instance $I_n$.
		
		
		\item An infinite sequence $(I_i)_{i \geq 0}$ of instances, with $D = I_0$, is a {\em chase derivation} of $D$ w.r.t.~$\dep$ if, for each $i \geq 0$, there is an active trigger $(\sigma,h)$ for $\dep$ on $I_i$ such that $I_i \app{\sigma}{h} I_{i+1}$. Moreover, $(I_i)_{i \geq 0}$ is {\em fair} if, for each $i \geq 0$, and for every active trigger $(\sigma,h)$ for $\dep$ on $I_i$, there exists $j > i$ such that $(\sigma,h)$ is not an active trigger for $\dep$ on $I_j$. 
		%The latter is known as the {\em fairness condition}, and guarantees that all the active triggers will be deactivated. %
		The {\em result} of such a chase derivation is the instance $\bigcup_{i \geq 0} \, I_i$.
	\end{itemize}
	%
	%The {\em result} of a chase derivation is defined as the union of all the instances occurring in it. 
	A chase derivation is {\em valid} if it is finite or infinite and fair.  \hfill\markfull
\end{definition}


Let us stress that infinite but unfair chase derivations are not considered as valid ones since they do not serve the main purpose of the chase, that is, to build an instance that satisfies the given set of TGDs. Indeed, given the set $\dep$ consisting of the TGDs
\[
\sigma\ =\ R(x,y) \ra \exists z \, R(y,z) \qquad \sigma'\ =\ R(x,y) \ra P(x,y),
\]
the result of the unfair chase derivation of $D = \{R(a,b)\}$ w.r.t.~$\dep$ that involves only triggers of the form $(\sigma,\cdot)$, i.e., only the TGD $\sigma$ is used, does not satisfy $\sigma'$, and thus, it does not satisfy $\dep$.
%
Interestingly, for every database $D$ and set $\dep$ of TGDs, any two valid chase derivations of $D$ w.r.t.~$\dep$ have always the same result, which implies that all valid chase derivations are either finite or infinite~\cite{GrOn18}. Therefore, in the rest of the paper, we can safely refer to {\em the} result of the chase of $D$ w.r.t. $\dep$, which we will denote by $\chase{D}{\dep}$. 
}


%\subsection{Non-Uniform Chase Termination}\label{sec:problem}
%

\medskip

\noindent
\textbf{Chase Termination.}
The result of the chase may be infinite even for very simple settings: it is easy to see that for $D = \{R(a,b)\}$ and $\dep = \{R(x,y) \ra \exists z \, R(y,z)\}$, $\chase{D}{\dep}$ is infinite.
%; in particular, $\chase{D}{\dep} = \{R(a,b),R(b,\bot_1),R(\bot_1,\bot_2),R(\bot_2,\bot_3),\ldots\}$, where $\bot_1,\bot_2,\ldots$ are null values.
%
This leads to the following problem, parameterized by a class $\class{C}$ of TGDs such as $\class{SL}$ (the class of simple-linear TGDs) and $\class{L}$ (the class of linear TGDs):


\medskip

\begin{center}
	\fbox{
		\begin{tabular}{ll}
			%{\small PROBLEM} : & %$\mathsf{ChaseTermination}(\class{C})$
			%\\
			{\small INPUT} : & A database $D$ and a set $\dep$ of TGDs from $\class{C}$.
			\\
			{\small QUESTION} : &  Is the instance $\chase{D}{\dep}$ finite?
	\end{tabular}}
\end{center}

\medskip

\noindent This problem has been recently studied in~\cite{CaGP22} for the classes of simple-linear and linear TGDs. Interestingly, for both classes, the finiteness of the result of the chase has been syntactically characterized by exploiting the notion of non-uniform weak-acyclicity. 
%
We proceed to recall this acyclicity notion, and then present the characterizations established in~\cite{CaGP22}, which in turn lead to simple algorithms for checking the finiteness of the result of the chase.
%
Note that, for the sake of clarity, in the rest of the paper we assume TGDs with a non-empty frontier, i.e., we assume that there is at least one variable in a TGD $\sigma$ that occurs both in $\body{\sigma}$ and $\head{\sigma}$. This assumption can be made without loss of generality since, given a database $D$ and a set $\dep$ of TGDs, we can easily construct a set $\dep'$ of TGDs with a non-empty frontier by slightly modifying $\dep$ such that $\chase{D}{\dep}$ is finite iff $\chase{D}{\dep'}$ is finite.


\medskip

\noindent
\textbf{Non-Uniform Weak-Acyclicity.} Weak-acyclicity was introduced in~\cite{FKMP05} as the main formalism for data exchange purposes, which guarantees the finiteness of the result of the chase for {\em every} input database. Non-uniform weak-acyclicity is the database-dependent variant of weak-acyclicity introduced in~\cite{CaGP22}. We proceed to give the formal definitions.
%
We first need to recall the notion of the {\em dependency graph} of a set $\dep$ of TGDs, 
%which symbolically encodes how terms may propagate during the chase.
%The {\em dependency graph} of set $\dep$ of TGDs 
defined as a directed multigraph $\depg{\dep}=(N,E)$, where $N = \pos{\sch{\dep}}$ and $E$ contains {\em only} the following edges.
%
For each TGD $\sigma \in \dep$ with $\head{\sigma} = \{\alpha_1,\ldots,\alpha_k\}$, for each $x \in \frontier{\sigma}$, and for each position $\pi \in \posvar{\body{\sigma}}{x}$:
\begin{itemize}
	\item For each $i \in [k]$ and for each $\pi' \in \posvar{\alpha_i}{x}$, there exists a \emph{normal} edge $(\pi,\pi') \in E$.
	%
	\item For each existentially quantified variable $z$ in $\sigma$, $i \in [k]$, and $\pi' \in \posvar{\alpha_i}{z}$, there is a \emph{special} edge $(\pi,\pi') \in E$.
\end{itemize}
%
We further need to define when a predicate is reachable from another predicate. 
%
Given predicates $R,P \in \sch{\dep}$, {\em $P$ is reachable from $R$ (w.r.t.~$\dep$)} if $R = P$, or there exists a path in $\depg{\dep}$ from a position of the form $(R,i)$ to a position of the form $(P,j)$.
%
%we write $R \ra_\dep P$  if $R = P$, or there exists a TGD $\sigma \in \dep$ such that $R$ occurs in $\body{\sigma}$ and $P$ occurs in $\head{\sigma}$. We say that {\em $P$ is reachable from $R$ (w.r.t.~$\dep$)}, denoted $R \reach{\dep} P$, if (i) $R \ra_\dep P$, or (ii) there exists $T \in \sch{\dep}$ such that $R \reach{\dep} T$ and $T \ra_\dep P$.
%in $\depg{\dep}$, denoted $R \reach{\dep} P$, if there exists a path in $\depg{\dep}$ from a position $(R,i)$ to a position $(P,j)$, for some $i \in [\arity{R}]$ and $j \in [\arity{P}]$.
Given a database $D$, we say that a (not necessarily simple and possibly cyclic) path $C$ in $\depg{\dep}$ is \emph{$D$-supported} if there exists an atom $R(\bar t) \in D$ and a node of the form $(P,i)$ in $C$ such that $P$ is reachable from $R$.
%
We are now ready to recall (non-uniform) weak-acyclicity.



\begin{definition}\label{def:dwa}
	Consider a database $D$ and a set $\dep$ of TGDs. We say that $\dep$ is {\em weakly-acyclic w.r.t.~$D$}, or {\em $D$-weakly-acyclic}, if there is no $D$-supported cycle in $\depg{\dep}$ with a special edge. 
	%
	We say that $\dep$ is {\em weakly-acyclic} if there is no cycle in $\depg{\dep}$ with a special edge. \hfill\markfull
\end{definition}


\smallskip

\noindent
\textbf{Characterizing the Finiteness of the Chase.}
It is not very difficult to show that whenever a set $\dep$ of TGDs (not necessarily linear) is $D$-weakly-acyclic, then the instance $\chase{D}{\dep}$ is finite. In other words, the $D$-weak-acyclicity of $\dep$ is a sufficient condition for the finiteness of $\chase{D}{\dep}$. What is more interesting is that, assuming that $\dep$ is a set of simple-linear TGDs, the $D$-weak-acyclicity of $\dep$ is also a necessary condition for the finiteness of $\chase{D}{\dep}$. This leads to the following characterization established in~\cite{CaGP22}:

\begin{theorem}\label{the:characterization-simple-linear}
	Consider a database $D$ and a set $\dep \in \class{SL}$ of TGDs. It holds that $\chase{D}{\dep}$ is finite iff $\dep$ is $D$-weakly-acyclic.
\end{theorem}

For linear TGDs, it turned out that non-uniform weak-acyclicity is not powerful enough for characterizing the finiteness of the chase instance. Here is an example given in~\cite{CaGP22} that illustrates this fact:
%This is illustrated by the following example.


\begin{example}
	Consider the database $D = \{R(a,b)\}$ and the singleton set $\dep$ consisting of the (non-simple) linear TGD
	\[
	R(x,x)\ \ra\ \exists z \, R(z,x). 
	\]
	It is easy to see that there is no trigger for $\dep$ on $D$. This means that $\chase{D}{\dep} = D$ is finite, whereas $\dep$ is {\em not} $D$-weakly-acyclic. \hfill\markfull
\end{example}


To obtain a characterization analogous to Theorem~\ref{the:characterization-simple-linear}, the authors of~\cite{CaGP22} used the technique of {\em simplification} to convert linear TGDs into simple-linear TGDs, while preserving the finiteness of the chase instance. We proceed to recall this technique.
%
Let $\bar t = (t_1,\ldots,t_n)$ be a tuple of (not necessarily distinct) terms. We write $\unique{\bar t}$ for the tuple obtained from $\bar t$ by keeping only the first occurrence of each term in $\bar t$.
%
For example, if $\bar t = (x,y,x,z,y)$, then $\unique{\bar t} = (x,y,z)$.
%
For each $i \in [n]$, the \emph{identifier of $t_i$ in $\bar t$}, denoted $\id{\bar t}{t_i}$, is the integer that identifies the position of $\unique{\bar t}$ at which $t_i$ appears. 
%
We write $\id{}{\bar t}$ for the tuple $(\id{\bar t}{t_1},\ldots,\id{\bar t}{t_n})$.
%
For example, if $\bar t = (x,y,x,z,y)$, then $\id{}{\bar t} = (1,2,1,3,2)$.
%
For an atom $\alpha = R(\bar t)$, the {\em simplification of $\alpha$}, denoted $\simple{\alpha}$, is the atom $R_{\id{}{\bar t}}(\unique{\bar t})$, whereas the {\em shape of $\alpha$}, denoted $\shape{\alpha}$, is the predicate $R_{\id{}{\bar t}}$. We can naturally refer to the simplification and the shape of a set of atoms.
%
For a tuple of variables $\bar x = (x_1,\ldots,x_n)$, a \emph{specialization of $\bar x$} is a function $f$ from $\bar x$ to $\bar x$ such that $f(x_1) = x_1$, and $f(x_i) \in \{f(x_1),\ldots,f(x_{i-1}),x_i\}$, for each $i \in \{2,\ldots,n\}$.
We write $f(\bar x)$ for $(f(x_1),\ldots,f(x_n))$. We are now ready to recall how a set of linear TGDs is converted into a set of simple-linear TGDs.

\begin{definition}\label{def:simplification}
	Consider a linear TGD $\sigma$ of the form
	\[
	R(\bar x) \ra \exists \bar z\, \psi(\bar y,\bar z), 
	\]
	where $\bar y \subseteq \bar x$, and a specialization $f$ of $\bar x$. The {\em simplification of $\sigma$ induced by $f$} is the simple-linear TGD
	\[
	\simple{R(f(\bar x))} \rightarrow \exists \bar z\, \simple{\psi(f(\bar y),\bar z)}.
	\]
	We write $\simple{\sigma}$ for the set of all simplifications of $\sigma$ induced by some specialization of $\bar x$.
	%
	For a set $\dep \in \class{L}$ of TGDs, the {\em simplification of $\dep$} is defined as the set
	\[
	\simple{\dep}\ =\ \bigcup_{\sigma \in \dep} \simple{\sigma}
	\]
	consisting only of simple-linear TGDs. \hfill\markfull
\end{definition}

We can now recall the characterization for the finiteness of the chase instance for linear TGDs, established in~\cite{CaGP22}, which is similar to the one for simple-linear TGDs, with the key difference that first we need to simplify both the database and the set of linear TGDs:

\begin{theorem}\label{the:characterization-linear}
	Consider a database $D$ and a set $\dep \in \class{L}$ of TGDs. Then, $\chase{D}{\dep}$ is finite iff $\simple{\dep}$ is $\simple{D}$-weakly-acyclic.
\end{theorem}

It is clear that Theorems~\ref{the:characterization-simple-linear} and~\ref{the:characterization-linear} provide simple algorithms for checking whether the chase instance is finite. In particular, given a database $D$ and a set $\dep$ of simple-linear TGDs, we simply need to check whether $\dep$ is $D$-weakly-acyclic, in which case the algorithm returns \true; otherwise, it returns \false. The same holds when $\dep$ is a set of linear TGDs, with the difference that the algorithm first needs to simplify $D$ and $\dep$, and then perform the acyclicity check.
%
Our goal is to experimentally evaluate the above algorithms with the aim of understanding which input parameters affect their performance, clarifying whether they can be applied in a practical context, and revealing their performance limitations. Of course, a naive implementation of the above algorithms, especially for linear TGDs where the expensive simplification must be applied, will lead to poor performance, and thus, will not be very useful towards our goal. Hence, we need to somehow convert the above theoretical algorithms into practical algorithms that are amenable to efficient implementations. This is the subject of the next section.

%
\section{Methodology}
\label{sec:method}
\section{Method}
\label{sec:method}

% \ml{``Inconsistent'' to ``large variation''}

% In this section, we propose our methods based on the observations in Section \ref{sec:motivation}.
In this section, we propose two techniques to further enhance the strong baseline to capture the variation of activation distributions better.
We first introduce spatial re-scaling to adapt the network to pixel-to-pixel variation.
We then propose channel-wise shifting and re-scaling to better capture the channel-to-channel variation.
Meanwhile, as both of the two methods are image-dependent, the image-to-image variation can be captured naturally.
By combining the two methods with our strong baseline, we build our enhanced BNN for SR, named EBSR.

% Because the activation distributions among pixels, channels and images have large variations \red{**are highly inconsistent} in SR networks, we introduce spatial re-scaling to adapt to pixel-wise variations and channel shift and re-scaling to adapt to channel-wise variations. And both of them are image-dependent to adapt to image-wise variations, which means during inference our network re-scales and shifts the distributions of activations flexibly for different input images. Based on these methods, we build an enhanced binary neural network for image super-resolution (EBSR).

% According to [3], the difference of activation magnitudes indicates different scaling factors are needed for each pixel.

\subsection{Spatial Re-scaling}
% It is better to use different scaling factors for different pixels to reduce the quantization error and retain more detailed information for image super-resolution. 

% \ml{In the main method, we do not need to introduce the previous works but can focus on introducing our own method. Channel rescaling in Real-to-binary Net is not relevant in this context.}

% Re-scaling the output of binary convolutions was proposed at the birth of BNN in XNOR-Net \cite{rastegari2016xnor} to reduce quantization error and improve accuracy for image classification tasks.
% It is computed as below:
% \begin{equation}
% \mathcal{A} * \mathcal{W} \approx(\operatorname{sign}(\mathcal{A}) \circledast \operatorname{sign}(\mathcal{W})) \odot \mathcal{K} \alpha
% \label{eq:xnor-net rescale}
% \end{equation}
% where $\circledast$ denotes the binary convolution and $\odot$ denotes the element-wise multiplication.
% $\mathcal{A}$, $\mathcal{W}$, $\alpha$, and $\mathcal{K}$ denote the activation, weight, weight scaling factor, and activation scaling factor, respectively.
%  Later in XNOR-Net++ \cite{bulat2019xnor}, Bulat et al. fuse the activation and weight scaling factors into a single one that is learned end-to-end based on gradients and this improves the classification accuracy on ImageNet dataset.

% % It is computed as Eq.~\ref{eq:xnor-net rescale}, where $\circledast$ denotes 
% %  the binary convolution and $\odot$ denotes the element-wise multiplication. The binary convolution of $\mathcal{A}$ and $\mathcal{W}$ is rescaled by the weight scaling factor $\alpha$ and the activation scaling factor $\mathcal{K}$, both of which are calculated analytically.


% \zc{Similarly, you should explain the meaning of A, W and the operators $\circledast$ in the formula}
% Then in Real-to-binary Net \cite{martinez2020training}, Martinez et al. used a data-driven channel re-scaling module that takes the pre-convolution activations as input to predict the activation scaling factor. Unlike that in XNOR-Net++ \cite{bulat2019xnor}, these scaling factors are not fixed during inference but rather inferred from data. By doing this, they further improved the classification accuracy on ImageNet over XNOR-Net++. 
As is shown in Figure \ref{fig:pixel}, activation distributions have large pixel-to-pixel variation in SR networks
and the difference of activation magnitudes indicates different scaling factors are preferred for different pixels.
Inspired by \cite{martinez2020training}, we propose spatial re-scaling to better adapt the network to the spatial variation
of activation distributions in SR networks.
% fit the various pixel-wise distributions in SR networks.
We take the real-valued activations $A$ before convolution as input and predict pixel-wise scaling factors $S(A)$, which re-scale the binary convolution output. Spatial re-scaling process can be formulated as follows:
\begin{equation}
A * W \approx(\operatorname{sign}(A) \circledast \operatorname{sign}(W)) \odot \alpha \odot S(A)
\label{eq:spatial rescale}
\end{equation}
where $\circledast$ denotes 
the binary convolution and $\odot$ denotes the element-wise multiplication. $A$, $W$, $\alpha$, and $S\left(A\right)$ denote real-valued activations, weights, the scaling factor of weights, and the spatial-wise scaling factor of activations respectively. $S\left(A\right) \in \mathbb{R}^{1\times H\times W}$ can be calculated with a convolution and a sigmoid function.
% as $\sigma\left( CONV\left(A\right)\right)$. 
As shown in Figure \ref{fig:method}(a), real-valued activations first go through a convolution layer,
which has an input channel of $C$ and an output channel of 1, 
and then pass through a sigmoid function to produce the scaling factors $S(A)$ along the spatial dimension.
During inference, the scaling factor will change dynamically according to different input feature maps.
By re-scaling binary convolution output using $S(A)$, we can reduce the quantization error and the original pixel-wise information in FP activation
will be preserved much better.
Spatial re-scaling leads to a large PSNR improvement of 0.24 dB (from 30.30 dB to 31.54 dB) on Set5 and 0.22 dB (from 25.09 dB to 25.31 dB)
on Urban100 compared with our strong baseline. 

\subsection{Channel-wise Shifting and Re-scaling}

\begin{table}[!tb]
\centering
\caption{Comparison between whether to fuse channel-wise shifting and re-scaling or not based on our baseline with spatial re-scaling. }
\label{tab:fusing}

\scalebox{0.65}{
\begin{tabular}{c|cc|cc|cc}
\hline
\multirow{2}{*}{Method}     & \multirow{2}{*}{OPs} & \multirow{2}{*}{Params} & \multicolumn{2}{c|}{Set5} & \multicolumn{2}{c}{Urban100} \\ \cline{4-7} 
                            &                      &                         & PSNR        & SSIM        & PSNR          & SSIM         \\ \hline
Baseline + spatial re-scale & 2.16G                & 0.05M                   & 31.54       & 0.883       & 25.31         & 0.759        \\
+ channel-wise shift and re-scale             & 2.34G                & 0.09M                   & 31.61       & 0.885       & 25.35         & 0.761        \\
+ w/ fusing                   & 2.27G                & 0.08M                   & \textbf{31.64}       & \textbf{0.885}       & \textbf{25.36}         & \textbf{0.761}        \\ \hline
\end{tabular}
}
\end{table}

In SR networks, activation distributions exhibit larger channel-to-channel variation (Figure \ref{fig:chl}).
Both the mean and magnitude of the activation distributions vary significantly across channels.
% Thus we use channel-wise shifting and re-scaling to adapt to various channel-wise distributions. 
\cite{martinez2020training} has proposed the data-driven channel re-scaling, 
but our method differs from them in further introducing data-driven thresholds to handle the channel-wise variation of both mean and magnitude.
Since the blocks to generate the scaling factors and thresholds are very similar, we further propose to fuse them into one module.
% and fusing channel-wise shifting and re-scaling into one module.
We evaluate the effect of fusing the two blocks in Table \ref{tab:fusing}.
With channel-wise shifting and re-scaling fused, our models have fewer operations and parameters overhead and slightly higher performance.

For the specific process, we take the real-valued activations as input and predict different thresholds and scaling factors for each channel. They are also image dependent, e.g., $\beta_{i}$ in Eq.\ref{eq:act_binarize} is no longer fixed during inference but generated according to different input feature maps. Channel-wise shifting and re-scaling can be formulated as follows:
\begin{equation}
A * W \approx(\operatorname{sign}(A-C_s(A)) \circledast \operatorname{sign}(W)) \odot \alpha \odot C_r(A)
\label{eq:channel-wise_shift_and_rescale}
\end{equation}
where $\circledast$ denotes 
the binary convolution and $\odot$ denotes the element-wise multiplication. $C_s(A), C_r(A) \in \mathbb{R}^{C\times1\times1}$ denote the channel-wise threshold and scaling factor, respectively. 
We show the block diagram in Figure \ref{fig:method}(b).
The real-valued input feature map is first squeezed to a ${C\times1\times1}$ vector by a global average pooling (GAP) layer.
The subsequent fully connected layers and ReLU learn the channel-wise information and output a ${2C\times1\times1}$ vector.
Then the ${2C\times1\times1}$ vector is split into two ${C\times1\times1}$ vectors.
We use the first $C$ channels as the channel-wise bias and pass the last $C$ channels through a sigmoid layer 
as the channel-wise scaling factor, which are used to shift the real-valued activations and re-scale the binary convolution output, respectively. 


% \ml{We can mention previously, channel-wise re-scale has been proposed. We propose to fuse them. Add the comparison between fuse v.s. no fuse.}

\begin{figure}[!tbp]%
  \centering
    \includegraphics[width=0.4\textwidth]{fig/methods.png}
  
% \subfloat[channel-wise shifting\&re-scale]{
%     \label{subfig:channel-wise shifting and re-scale}
%     \includegraphics[width=0.2\textwidth]{fig/chl shift and rescale.png}
%   }

  \caption{Block diagram for spatial re-scaling, and channel-wise shifting and re-scaling.} 
  % Input A is the real-valued activation tensor and C, H, and W denote its dimension. GAP stands for global average pooling. The reduction ratio r is set to 16 for a better trade-off between the performance and the number of operations and parameters.}
  \label{fig:method}
\end{figure}


\subsection{Network Structure}

Combining the spatial re-scaling and the channel-wise shifting and re-scaling methods, we construct the enhanced convolution layer (E-Conv).
Then we build our EBSR model based on E-Conv.
In Figure \ref{fig:E-conv}, we compare the binary convolution layer used in the baseline network and our proposed E-Conv.
We use spatial and channel-wise scaling factors to re-scale the binary convolution output,
and use channel-wise shifting to learn appropriate thresholds for each channel before binarization.
The scaling factors and threshold used in E-Conv are learnable and depend on the real-valued input activations.
In this way, our proposed EBSR can adapt to pixel-to-pixel, channel-to-channel, and image-to-image variations
to reduce the large binarization error and preserve more details.
% In this way, our proposed E-Conv reduces the large quantization error caused by binarization and keeps the original information of input feature maps to a large extent.


\begin{figure}[!tb]%
  \centering

    \includegraphics[width=0.5\textwidth]{fig/E-conv.png}

  \caption{Comparison of (a) the binary convolution layer with a skip connection used in our baseline network and (b) the proposed E-Conv.}
  \label{fig:E-conv}
\end{figure}


Figure \ref{fig:network} shows the basic block based on the E-Conv and our EBSR composed of the basic blocks. Following existing works, the convolution layers in the head and tail modules are not binarized. We choose the lightweight EDSR which has 16 basic blocks and 64 channels, and EDSR which has 32 basic blocks and 256 channels as our backbones, which correspond to EBSR-light and EBSR, respectively.

\begin{figure}[!tb]%
  \centering
  {
    \includegraphics[width=0.35\textwidth]{fig/network.png}
  }
  
  \caption{The structure of our proposed EBSR.  Convolution layers in purple are real-valued vanilla 3x3 convolutions.}
  \label{fig:network}
\end{figure}

%
\section{Results}
\label{sec:results}
\section{Results}
\label{results}

\begin{figure*}[ht]
    \centering
    \includegraphics[scale=0.15,trim={0 2.5cm 0 5cm},clip]{images/aoi-single_burst}
    \caption{The time average peak Age of Information with burst and \gls{soa} loss values against the dynamic reliability logic for different network topologies.}
    \label{fig:aoi_burst}\vspace{-0.4cm}
\end{figure*}


This paper focuses on both transport layer and application layer metrics to determine the feasibility of dynamic reliability. For this, we have selected the session packet volume, as transmitted, retransmitted, lost and backlogged packets as \glspl{kpi} for the transport layer; while focusing on the \gls{aoi} for the application layer. The \gls{aoi} was chosen as a crucial indicator for the freshness of packets in real-time applications. More specifically, this work adopts the time average peak \gls{aoi} equation \cite{aoi_equation} depicted in Eq. \ref{aoi}, where $\Delta(r_{i+1})$ is the $i$th update at the time it was received at the server, for a session time period of $\tau$.

\begin{equation}
    \label{aoi}
    \gls{aoi}_\tau = \frac{1}{n-1}\sum_{i=1}^{n-1} \Delta(r_{i+1})
\end{equation}

We include a comparison between the vanilla QUIC implementation which does not enjoy the dynamic reliability extension, with a number of dynamic reliability policies. The tests were run a number of times for statistical significance, with the mean value of vanilla implementation used as a baseline for comparison. The topology utilised both random loss and bursty loss to explore the bounds of dynamic reliability. The \gls{soa} loss in the figures correspond to the loss values presented in Table. \ref{tab:path_char}, for ease of comparison between bursty and random loss scenarios.

\subsection{Transport-Layer KPIs}

To analyse the performance gain at the transport layer due to dynamic reliability, the volume of transmitted and backlogged packets is examined. The figures are in the form of boxplots, which take the vanilla implementation as a benchmark, depicted as the red dashed line.

As seen in Fig. \ref{fig:sent_burst}, the loss plays a crucial role in the performance of the reliability policies. The policies under random loss did incredibly well for the networks with a larger capacity, namely \gls{mmwave} and Sub-6~GHz, whereas for burst loss, the lower network capacities had a larger packet reduction. With the increase in burst loss, the behaviour of the set split reliable policies became unpredictable, if a reliable assignment happened to coincide with a burst loss, the number of transmitted packets increases, and vice versa. On the other hand, in smarter policies, such as Loss-Aware, the performance lightly matched the vanilla baseline, as the reliable assignment dominated the session to compensate for a higher burst loss. Not only that but, the burst loss also impacted the variance of the transmitted packets for the policies.

Unsurprisingly, the unreliable focused policy, 80-20 split, outperformed other policies for all topologies in random and bursty loss scenarios, with an approximate reduction of 80\%. That being said, the majority of the policies reduced the transmitted packets on the link by approximately 70\% for random loss, while the reduction started at $\approx 15\%$ and decreased as the loss increased for the burst loss scenario.

The retransmitted and lost packets, not shown due to space limitations, followed the same trend as the transmitted packets for the random loss scenarios. However, for the burst loss scenarios, the larger capacity networks had a lower reduction in the retransmitted and lost packets. This can be seen as a favorable outcome since the lower capacity networks are scarce on resources. It is important to note that the Loss-Aware policy mimicked the vanilla approach as the burst loss increased, signifying the overwhelming appointment of reliable packets in adapting to the harsh burst loss conditions.
 
Alternatively, Fig. \ref{fig:backlog_burst} clearly shows a stark comparison between the policies and loss scenario in the reduction of the backlogged packets. The Loss-Aware policy for random loss scenario reduced the backlogged packets by up to 50\%, beating all other policies by approximately 30\%. Furthermore, it is clear that the unreliability focused policies resulted in the lowest backlog for the session. In comparison, we notice that the burst loss and the backlogged frequency have a positive correlation, where the maximum reduction of the backlogged packets for the policies is at most 20\%. Much like the transmitted packets, the probability of a burst loss occurrence plays a vital role in the number of retransmissions sent and by extension the number of backlogged packets. Thus, we can conclude that the stress placed on the buffer is a result of the reliable packets which is tightly coupled with the congestion on the session. Whereas, unreliable focused policies did not encounter such a phenomenon regardless if it was experiencing a burst loss.


\subsection{Application-Layer KPIs}

The feasibility of dynamic reliability for real-time applications can be determined by the \gls{aoi}, with comparison across different topologies and policies. If we take a strict approach and consider anything below $10$~ms is real-time \cite{real-time}, then all the reliability policies passed that requirement, which is attractive for real-time applications, as shown in Fig. \ref{fig:aoi_burst}. Utilising the median as an estimate of the runs, the policies in the WLAN and Sub-6~GHz topology with random loss floated around $4-5$~ms with negligible difference, while the \gls{aoi} for \gls{mmwave} was $\approx 2-3$~ms. It is clear that the \gls{aoi} and the network capacity have a negative correlation, as the network capacity decreases, the \gls{aoi} increases. The same correlation is extended to the bursty loss scenarios, where \gls{mmwave} dominated the other topologies. That being said, it is crucial to note that the \gls{aoi} for the reliability policies is often slightly better than or equal to the \gls{aoi} of the vanilla implementation, proving that dynamic reliability reduces the congestion of the session at no cost to the \gls{aoi}.


%
\section{Discussion}
\label{sec:discussion}
We provide some comments on the growth conditions which constituted the majority of our analysis in sections \ref{sec:Hmixing} and \ref{sec:Hsigma}. In the simplest cases of Lemma \ref{lemma:unstableGrowth}, growth was established in an analogous fashion to the old one-step expansion condition (\ref{eq:oldOneStepExpansion}), finding the relevant Jacobians $M_j$ and checking that their expansion factors $K(M_j)$ satisfy
\begin{equation}
    \label{eq:discussionOneStep}
    \sum_j \frac{1}{K(M_j)} <1.
\end{equation}
For the more complicated cases, the inductive method used to establish growth near the accumulation points in Lemma \ref{lemma:unstableGrowth} and the weakened one-step expansion condition (\ref{eq:oneStep}) both address the same fundamental issue: the splitting of unstable curves by singularities into an unbounded number of small components. They circumvent this obstacle in rather different ways, however. While (\ref{eq:oneStep}) generalises (\ref{eq:discussionOneStep}) to ensure an growth of unstable curves `on average' (see \cite{chernov_statistical_2009} for a precise statement), our inductive method is a more direct adaptation of (\ref{eq:discussionOneStep}), using it to generate contradictory geometric conditions which a hypothetical non-growing unstable curve must satisfy. It may be possible to prove Theorem \ref{sec:Hmixing} using (\ref{eq:oneStep}) as the basis for growth. Since we required (\ref{eq:oneStep}) anyway for proving Theorem \ref{thm:HsigmaExp}, this could potentially condense our analysis, but only to a minor extent. A convenience of the method used in section \ref{sec:Hmixing} is that, by way of the `simple intersection' property, it naturally gives geometric information on the images of manifolds, useful for proving the property \textbf{(M)} of Theorem \ref{thm:katok-strelcyn}.

We expect that essentially analogous analysis can be applied to establish mixing properties in a wide class of piecewise linear non-uniformly hyperbolic maps, including those (like the OTM) which sit on the boundary of ergodicity and beyond. While we have relied on the precise partition structure of $H_\sigma$, its fundamental feature (self-similar sequences of elements $A^k$, sharing boundaries with its neighbours $A^{k-1},A^{k+1}$ and accumulating onto some point $p$) is quite typical to return map systems. See, for example, those of various stadium billiards \cite{chernov_chaotic_2006,chernov_improved_2008,chernov_statistical_2009} and LTMs \cite{springham_polynomial_2014}. Indeed, the same method can be used to prove the Bernoulli property for non-monotonic LTMs \cite{myers_hill_mixing_2022}, where monotonicity of the manifold images cannot be assumed and the classical argument \cite{sturman_mathematical_2006} fails. The OTM is the pointwise limit of these maps as the boundary shrinks to null measure. It further has utility in proving growth conditions for maps which are uniformly hyperbolic but possess regions $A_j$ where the hyperbolicity is very weak, signified by $K(M_j) \approx 1$, so that (\ref{eq:discussionOneStep}) fails. Typically this leads to suboptimal bounds on mixing windows, see e.g. \cite{wojtkowski_model_1981,przytycki_ergodicity_1983,myers_hill_family_2022}. The map $H_{(\eta,\eta)}$ for $\eta \approx 1/2$ is another example, possessing weak hyperbolicity over $A_2, A_3$. Letting $\varepsilon = |\eta-1/2|>0$, there is an upper bound $N = N(\varepsilon)$ on escape times from the intersections $A_2\cap \sigma, A_3 \cap \sigma$. The growth lemma then follows by applying the inductive step roughly $N$ times and can be established for arbitrarily small $\varepsilon$, opening the door to establishing optimal mixing windows.

The above gives two examples of piecewise linear perturbations to $H$ where mixing with respect to Lebesgue is preserved and our methods can be applied. Nonlinear perturbations to the shear profiles complicate the analysis in several ways. Firstly as the map's Jacobians takes on a broader range of values, cone invariance becomes an increasingly harder condition to establish. Cones must be widened, giving looser bounds on expansion factors, which may already be weak due to new regions of weaker stretching. This, together with the change from polygonal to curvilinear return time partition elements and nonlinear local manifolds, adds some complexity to showing growth conditions. This does not rule out certain (small) nonlinear perturbations however. There is some leeway in the inequalities which govern cone invariance and growth of local manifolds, the latter of which is not too dissimilar from the piecewise linear setting (see Lemmas \ref{lemma:piecewiseApprox}, \ref{lemma:componentLength}). Certain small perturbations would not alter the \emph{topological} structure of the return time partition, i.e. which elements share boundaries, the key information needed for setting up the induction. Finally while the partition elements would no longer be polygonal, only coarse geometric information is required for verifying each inductive step. Following the above, a potential perturbation could be to replace the linear portions of each shear by a cubic, perturbing the tent profile
\[  f(t) = \begin{cases} 2t & 0 \leq t \leq 1/2, \\ 2(1-t) & 1/2 \leq t \leq 1 ,\end{cases} \]
of the OTM shears to
\[  f_a(t) = \begin{cases} \frac{1}{8} t \left(16 - a + 6at - 8at^{2} \right) & 0 \leq t \leq 1/2, \\ \frac{1}{8}\left(1-t\right)\left( 16 - a + 6a\left(1-t\right) - 8a\left(1-t\right)^{2}\right)  & 1/2 \leq t \leq 1, \end{cases}   \]
for $a>0$. For small enough $a$ the gradient range $f'(t)$ is restricted to small neighbourhoods of $\{ 2, -2\}$ and the escape time partition retains a similar structure. We illustrate this in Figure \ref{fig:perturbations}, showing escapes from the square $S_3$ under the map $G \circ F$, equivalent to escapes from the perturbed $A_3$ under the $G \circ F$, but with a cleaner geometry for comparison. When $a$ is too large the analogy to the OTM breaks down. At $a=16$ the map is twice differentiable everywhere and features a new source of slowed mixing, the Jacobian is the identity at the corner points $x,y \in \{  0, 1/2 \}$ giving locally parabolic behaviour (visible in the escape time partition). 

\begin{figure}
    \centering
    \includegraphics[width=0.24 \linewidth]{0.png}
    \includegraphics[width=0.24 \linewidth]{4.png}
    \includegraphics[width=0.24 \linewidth]{8.png}
    \includegraphics[width=0.24 \linewidth]{16.png}
    \caption{Partition of escape times from $S_3$ under the mapping $F \circ G$ for $a= 0,4,8,16$. }
    \label{fig:perturbations}
\end{figure}

%
\section{Conclusions}
\label{sec:conclusions}
\section{Conclusions}
We consider the phase-extraction problem, and we showed that, given a unitary $U = e^{i\pi H}$ and its inverse $U^{\dag}$, we could implement a block-encoding of $\phi(H)$ for some smooth function $\phi(x)$. The word `smooth' here means existence and continuity of the derivatives: the higher the number of continuous derivatives that a function has, the faster its Fourier sum (and thus the Laurent polynomial on the eigenphases) uniformly converges to that function. We are confident this can have many more applications beyond what is shown in this work. It is also worth remarking that Jackson showed that the convergence rate of a Fourier series is almost-optimal, in the sense that no trigonometric (or, equivalently, complex exponential) series can approximate the desired function faster, up to that $\log d$ factor~\cite[p.\ 21]{jacksonTheoryApproximation1930a}. Also remember that `smoothing' a function, i.e., replacing its derivative with a continuous function, does not give faster convergence for free in general, as its derivative will become steep in the points where we smooth out discontinuities, and this translates to a high Lipschitz constant: a~clear example is given by Eq.~\ref{eq:lipschitz-constant-recurrence-solution}, but in that case, fortunately, nothing depends on the size of the input $N$, and thus does not influence the asymptotic query complexity of Algorithm~\ref{alg:prop-sampling-qsp}, although the constant factor can become large even for $p = 20$. From a theoretical point of view, this work shows that, for any $\eta > 0$, there is an algorithm with query complexity 
$$\Tilde{\bigO}\left(\frac{1}{\bar{c}^{\frac{1}{2} + \eta}} \frac{1}{\epsilon^\eta} \right)$$
solving the proportional-sampling problem. This statement seems to suggest there exists an algorithm which directly solves the problem with $\eta = 0$, and an open question would be to find such algorithm.


It is also interesting to remark that Theorems~\ref{thm:haah-construction},~\ref{thm:haah-completion} indeed allow the construction for any $\phi$, even complex-valued, provided that its absolute value is reciprocal.

One could think that, in Section~\ref{sec:prop-sampling}, instead of using the linear function in the phase-extraction subroutine, we could approximate the square root and then apply the transformation directly on $e^{i \pi c(x)}$. However, in the case of proportional sampling this would be inconvenient, as the derivative of the square root function has a discontinuity with an infinite jump around 0, and we could not choose a constant $\delta$ if we had values of the oracle that are too close to $0$.

\appendix

\section{Collision model}
\label{sec:collision}

%
Here we describe the algorithm used to handle particle-particle contact in this
work. 
%
For the present case of a dilute suspension we adopt a simple repulsion model in
which contact forces are determined from the distance separating a given
particle pair.
%
Each contact event involves only a pair of particles, and the resultant contact force 
is assumed to be a point force.
%
Hence we need to define a contact point, a direction and a force intensity.
%
In this work we consider only normal forces with a quadratic law similar to the
one used in \cite{uhlmann:2014a} for spherical particles, originally proposed by
\cite{glowinski:2001}.
%


%
The main issue when working with non-spherical particles is that the contact
point and the normal direction are not uniquely defined.
%
The most popular methods to determine the contact parameters between spheroids
in the literature are the common normal \cite{lin:1995} and the geometric 
potential \citep{ng:1994}.
%
The common normal method is very attractive since it naturally yields the contact
point and the normal direction.
%
However, the resultant system of equations is under-determined and undesired solutions
can be obtained.
%
\cite{kildashti:2018} overcame this issue by an iterative process.
%
There is, however, a non-solved issue which arises when the overlapping distance
between the spheroids is exactly zero, or very small, leading to an
ill-determined system.
%
On the other hand, the geometric potential approach is particularly attractive
when dealing with simple geometries like spheroids, in which the contact point 
is easily determined.
%
The main drawback of the geometric potential is the definition of the normal
direction.
%
In this work we propose to use the geometric potential to determine the contact
point, but determine the normal direction with a slight modification of the 
algorithm originally proposed by \cite{ng:1994}.
%




In order to apply the geometric potential method we consider the \gls{cfr} of a
spheroid using the potential \gls{c:pot} defined as
%
\begin{equation}\label{eq:cfr}
\gls{c:pot}(x,y,z) = \gls{c:cfrA}x^2 + \gls{c:cfrB}y^2 + \gls{c:cfrC}z^2 
       +2\gls{c:cfrF}yz  +2\gls{c:cfrG}zx  +2\gls{c:cfrH}xy 
       +2\gls{c:cfrP}x   +2\gls{c:cfrQ}y   +2\gls{c:cfrR}z  + \gls{c:cfrD} \,,
\end{equation}
%
where the coefficients $\{\gls{c:cfrA}, \gls{c:cfrB}, \gls{c:cfrC}, \gls{c:cfrD},
\gls{c:cfrF}, \gls{c:cfrG}, \gls{c:cfrH}, \gls{c:cfrP}, \gls{c:cfrQ}, \gls{c:cfrR} \}$
are functions of the spheroid parameters (equatorial diameter $\gls{dd}$ and 
symmetry axis length $\gls{aa}$) and the particle's position and orientation.
%
For a point on the surface of the spheroid $\gls{c:pot}=0$.
%
The coefficients in \eqref{eq:cfr} are easily obtained from the \gls{cfr} of
the spheroid expressed in the body-fixed reference system
%
\begin{equation}\label{eq:cfr_body}
\gls{c:pot}(\gls{Obody_x},\gls{Obody_y},\gls{Obody_z}) = 
       \left(\frac{\gls{Obody_x}}{\gls{dd}/2+\gls{dx}/2}\right)^2 
     + \left(\frac{\gls{Obody_y}}{\gls{dd}/2+\gls{dx}/2}\right)^2 
     + \left(\frac{\gls{Obody_z}}{\gls{aa}/2+\gls{dx}/2}\right)^2 - 1  \,,
\end{equation}
%
where $(\gls{Obody_x},\gls{Obody_y},\gls{Obody_z})$ are the coordinates of a point 
\gls{Obody_vx} expressed in the body-fixed
coordinate system (see figure \ref{fig:problem_description}a) and where we have 
included a force range of $\gls{dx}/2$  in order to minimize the effect of overlapping
support of the diffuse interface during particle approach (see figure \ref{fig:contact_detail}b).
%
After some algebra  and using the relation 
$\gls{Obody_vx} = \gls{rotmat}\left(\gls{Oxyz_vx}-\gls{vxp}\right)$,
where \gls{rotmat} is the rotation matrix to obtain body-fixed coordinates from
the global coordinate system, one reaches to
%
\begin{subequations}
\begin{align}
\gls{c:cfrA} &= \sum_{i} U_i\gls{rotmat_i1}^2 \,,\\
\gls{c:cfrB} &= \sum_{i} U_i\gls{rotmat_i2}^2 \,,\\
\gls{c:cfrC} &= \sum_{i} U_i\gls{rotmat_i3}^2 \,,\\
\gls{c:cfrF} &= \frac{1}{2}\sum_{i} U_i\gls{rotmat_i2}\gls{rotmat_i3} \,,\\ 
\gls{c:cfrG} &= \frac{1}{2}\sum_{i} U_i\gls{rotmat_i3}\gls{rotmat_i1} \,,\\ 
\gls{c:cfrH} &= \frac{1}{2}\sum_{i} U_i\gls{rotmat_i1}\gls{rotmat_i2} \,,\\ 
\gls{c:cfrP} &=-\frac{1}{2}\sum_{r}\gls{vxp_r}\sum_{i} U_i\gls{rotmat_ir}\gls{rotmat_i1} \,,\\ 
\gls{c:cfrQ} &=-\frac{1}{2}\sum_{r}\gls{vxp_r}\sum_{i} U_i\gls{rotmat_ir}\gls{rotmat_i2} \,,\\ 
\gls{c:cfrR} &=-\frac{1}{2}\sum_{r}\gls{vxp_r}\sum_{i} U_i\gls{rotmat_ir}\gls{rotmat_i3} \,,\\ 
\gls{c:cfrD} &= \sum_{s}\gls{vxp_s}\sum_{r}\gls{vxp_r}\sum_{i} U_i\gls{rotmat_ir}\gls{rotmat_is} - 1 \,,
\end{align}
\end{subequations}
%
where $\vec{U} = \left(\left(\gls{dd}/2+\gls{dx}/2\right)^{-2},
                 \left(\gls{dd}/2+\gls{dx}/2\right)^{-2},
                 \left(\gls{aa}/2+\gls{dx}/2\right)^{-2}\right)$.
%
%
In the following we introduce the subscript $1$ or $2$ to identify each of the
spheroids participating in a collision, which are defined by their potentials
$\gls{c:pot}_1$ and $\gls{c:pot}_2$.
%
The contact point is defined as the midpoint between the deepest point of 
spheroid $1$ in $2$, \gls{c:gp_deep_1}, and the deepest point of spheroid $2$
in $1$, \gls{c:gp_deep_2}.
%
%
%
%
%
%
%
To obtain the deepest point of $1$ in $2$ we minimize the function $\mathcal{L}=
\gls{c:pot}_2 + \gls{c:gp_la}\gls{c:pot}_1$.
%
The following linear system is obtained
%
\begin{equation}\label{eq:lagrange}
   \left[
   \begin{array}{ccc}
   \gls{c:cfrA}_2 + \gls{c:gp_la}\gls{c:cfrA}_1 & \gls{c:cfrH}_2 + \gls{c:gp_la}\gls{c:cfrH}_1 & \gls{c:cfrG}_2 + \gls{c:gp_la}\gls{c:cfrG}_1 \\
   \gls{c:cfrH}_2 + \gls{c:gp_la}\gls{c:cfrH}_1 & \gls{c:cfrB}_2 + \gls{c:gp_la}\gls{c:cfrB}_1 & \gls{c:cfrF}_2 + \gls{c:gp_la}\gls{c:cfrF}_1 \\
   \gls{c:cfrG}_2 + \gls{c:gp_la}\gls{c:cfrG}_1 & \gls{c:cfrF}_2 + \gls{c:gp_la}\gls{c:cfrF}_1 & \gls{c:cfrC}_2 + \gls{c:gp_la}\gls{c:cfrC}_1
   \end{array}
	\right] \left[
		\begin{array}{c}
                       x_{\gls{c:gp_la}} \\
                       y_{\gls{c:gp_la}} \\
		       z_{\gls{c:gp_la}} 
		\end{array}
    	\right] = -\left[
		\begin{array}{c}
                        \gls{c:cfrP}_2 + \gls{c:gp_la}\gls{c:cfrP}_1 \\
                        \gls{c:cfrQ}_2 + \gls{c:gp_la}\gls{c:cfrQ}_1 \\
                        \gls{c:cfrR}_2 + \gls{c:gp_la}\gls{c:cfrR}_1 
		\end{array}
	\right].
\end{equation}%
%
We can obtain a solution for the linear system \eqref{eq:lagrange} in terms of 
\gls{c:gp_la} with Cramer's rule
%
\begin{equation}\label{eq:cramer}
x_{\gls{c:gp_la}} = \frac{\gls{c:gp_D1}}{\gls{c:gp_D}},  \quad
y_{\gls{c:gp_la}} = \frac{\gls{c:gp_D2}}{\gls{c:gp_D}},  \quad
z_{\gls{c:gp_la}} = \frac{\gls{c:gp_D3}}{\gls{c:gp_D}},  
\end{equation}
%
where \gls{c:gp_D}, \gls{c:gp_D1}, \gls{c:gp_D2} and \gls{c:gp_D3} are the determinants 
of the matrix of the linear system \eqref{eq:lagrange}.
%
Substituting \eqref{eq:cramer} in the potential of spheroid 1 leads to a sixth order 
polynomial for \gls{c:gp_la}
%
\begin{equation}\label{eq:sextic}
\gls{c:pot}_1\left(x_{\gls{c:gp_la}},y_{\gls{c:gp_la}},z_{\gls{c:gp_la}}\right) = 0.
\end{equation}
%
Now, we define \gls{c:gp_la_min} as the real-valued solution out of the set 
$\gls{c:gp_la}_i$ ($i=1,6$) for which $\gls{c:pot}(\gls{c:gp_la_min}) = 
\min_i \gls{c:pot}(\gls{c:gp_la}_i)$.
%
%
The deepest point of spheroid 1 inside spheroid 2 is $\gls{c:gp_deep_1}%
=(x_{\gls{c:gp_la_min}},y_{\gls{c:gp_la_min}},z_{\gls{c:gp_la_min}})$.
%
If $\gls{c:pot}_2(\gls{c:gp_deep_1}) > 0$, then contact does not exist, and the 
computation of \gls{c:gp_deep_2} is skipped.
%
If $\gls{c:pot}_2(\gls{c:gp_deep_1}) = 0$, $\gls{c:gp_deep_2}=
\gls{c:gp_deep_1}$.
%
If $\gls{c:pot}_2(\gls{c:gp_deep_1}) < 0$, the point \gls{c:gp_deep_2} is
obtained analogously to \gls{c:gp_deep_1}.
%
In the event of collision, the contact point \gls{c:gp_cp} is defined as 
the arithmetic average position
$\gls{c:gp_cp} = (\gls{c:gp_deep_1}+\gls{c:gp_deep_2})/2$.

%
The normal direction to the contact, \gls{c:nn}, according to the original
method is defined by the line connecting the points \gls{c:gp_deep_1} and 
\gls{c:gp_deep_2}, $\gls{c:nn} =\frac{\gls{c:gp_deep_2}-\gls{c:gp_deep_1}}%
{\left\|\gls{c:gp_deep_2}-\gls{c:gp_deep_1}\right\|}$.
%
However, the line connecting the two deepest points is not (in general) aligned
with the direction obtained from the common normal method.
%
Therefore, we propose to define the normal direction at the contact point as 
$\gls{c:nn} = (\gls{c:gp_nn_1} - \gls{c:gp_nn_2})/2$, where \gls{c:gp_nn_1} 
(\gls{c:gp_nn_2}) represents the unitary normal vector to spheroid 1 (2) at point
\gls{c:gp_deep_1} (\gls{c:gp_deep_2}).
%
It should be noted that \gls{c:nn} is not exactly normal to any of spheroids
1 or 2.
%


Finally, the modulus of the normal contact force is defined as
$F_n =\delta^2/J$, where $\delta$ is the overlapping distance ($\delta = %
\gls{c:nn} \cdot (\gls{c:gp_deep_1}-\gls{c:gp_deep_2})$) and $J$ is a constant
whose value depends on the submerged weight of a single particle, $W_s$, as
$J\,W_s/D^2\approx2.5\cdot 10^{-4}$
%
(we have checked that our DKT results are not sensitive to the precise choice of
the stiffness constant $J$).
%
This leads to the following forces and torques on each of the colliding two particles in a pair:
\begin{subequations}
\begin{align}
\gls{c:F1} &= -\gls{c:F2} = -F_n\gls{c:nn}\\
\gls{c:T1} &= \gls{rs:1} \times  \gls{c:F1}  \\
\gls{c:T2} &= \gls{rs:2} \times  \gls{c:F2}  \, .
\end{align}
\end{subequations}
%
Please note that, compared to spheres, normal forces can generate torque 
in the non-spherical case (see figure \ref{fig:contact_detail}a).
%
Furthermore, the pair of torques does not need be equal in magnitude.

\begin{figure} 
\makebox[\textwidth][c]{ 
%
\includegraphics[scale=1.3]{./fig20.pdf} 
}
\caption{a) Sketch of the two spheroids indicating the contact force and torque
in each particle and the normal direction at the contact point.
b) Sketch of the elements involved in determining the contact point
(\gls{c:gp_cp}), the normal direction at the contact point (\gls{c:nn}) and
the overlapping distance ($\delta$).
The normal direction given by the original method  \citep[][%
$\vec{n}_c^{(\text{orig.})}$ in panel b %
]{ng:1994} is also including for comparison purposes.
\label{fig:contact_detail}} 
\end{figure} 




%
%
%
%
%
%
%
%
%
%
%
%
%
%
%
%


%
%
%
%
%
%
%
%
%
%

%
%


 
\section{Two point autocorrelation functions}
\label{sec:autocorr}

%
Figures \ref{fig:autocorr_G111} and \ref{fig:autocorr_G152} show the two point
autocorrelation functions 
of the \revision{vertical}{horizontal} fluid velocity component,
\revision{\gls{f:Rww}}{\gls{f:Ruu}}, (panels \revision{a and b}{a-c}) and of the
\revision{horizontal}{vertical} counterpart, \revision{\gls{f:Ruu}}{\gls{f:Rww}},
(panels \revision{d, e and f}{d-e}) for cases \verb!G111! and \verb!G152!, respectively, together with 
the time history of \gls{f:Rww} at the furthest vertical and horizontal 
position (panel \revision{c}{f}).
%
When particles are fixed ($t<0$) all signals are fully decorrelated within the
domain.
%
After particles are released the horizontal velocity remains fully decorrelated 
until the end of the simulated time (see figures \ref{fig:autocorr_G111}\revision{d-f}{a-c}
and \ref{fig:autocorr_G152}\revision{d-f}{a-c}), whereas the vertical velocity acquires
correlations which do not decay to zero at later times (figures 
\ref{fig:autocorr_G111}\revision{a-b}{d-e} and 
\ref{fig:autocorr_G152}\revision{a-b}{d-e}) as will be discussed in the following.
%

%
%
%
%
Along the vertical direction, \gls{f:Rww} presents an initial fast growth
($0\lesssim t/\gls{tg} \lesssim 300$).
%
For later times, the curves $\gls{f:Rww}(r_z)$ are similar, except for a small
oscillating behavior in time.
%
They show a monotonic decreasing behavior in space with positive values over the
entire domain ($0<r_z/D<110$).
%
This is the footprint of the fast formation of robust columnar structures that 
occupy the whole domain in the vertical direction.
%
The temporal evolution of \gls{f:Rww} at its furthest distance 
($r_{z,\text{max}}=110$) supports the fast growth and the small oscillation with
time commented above.
%

The behavior of \gls{f:Rww} along the horizontal direction is somehow more
complex than along the vertical direction.
%
First, it presents an almost decorrelated solution until $300\gls{tg}$ in case
\verb!G111! and $900\gls{tg}$ in case \verb!G152!.
%
In both cases there is a turnover point in which the value of \gls{f:Rww} 
and for case \verb!G111! even two when $300\lesssim t/\gls{tg}\lesssim 600$.
%
This can be explained by the growth of the clusters in the horizontal direction
being a slow process compared to their growth in the vertical direction.
%
Furthermore, once the clusters grow to a size comparable to the computational
domain and because of continuity, there are negative values at 
$r_{x,\text{max}}=55$.
%
The first crossing of \gls{f:Rww} with the zero value gives an estimate of the 
size of the clusters in the horizontal direction.
%
This measures approximately $15-20\gls{p:deq}$, which implies that the clusters
ultimately grow to a size for which the current computational domains are not sufficient.

\begin{figure} 
\makebox[\textwidth][c]{ 
%
\includegraphics[scale=1.0]{fig21.pdf} 
}
\caption{Autocorrelation functions of the
a-c) horizontal fluid velocity component , \gls{f:Ruu}, and of the
d-e) vertical counterpart, \gls{f:Rww},
for case {\tt G111}.
%
f) Time history of \gls{f:Rww} at the furthest vertical and horizontal position.
%
The time interval to compute each curve in panels a-e is shown in the legend,
and the corresponding linestyles are used piecewise in panel f.
\label{fig:autocorr_G111}} 
\end{figure} 

\begin{figure} 
\makebox[\textwidth][c]{ 
%
\includegraphics[scale=1.0]{fig22.pdf} 
}
\caption{Same as \ref{fig:autocorr_G111} but for case {\tt G152}.
\label{fig:autocorr_G152}} 
\end{figure} 


\section{\glsentrylong{dkt} computational setup}
\label{sec:dkt}

Here we describe the computational setup of the \gls{dkt} simulations presented
in \S~\ref{sec:results/dkt}.
%
The problem description is the same as in the multiparticle cases 
(\S~\ref{sec:problem}), but the methodology presents a few differences compared
to the one presented in \S~\ref{sec:method}. 
%
First, we impose a free stream of constant velocity at the lower boundary plane,
an advective boundary condition at the top, and periodicity at the lateral
boundaries, instead of periodicity in the three spatial directions.
%
Second the number of particles is exactly two.
%
Thus, the solid volume fraction is not a parameter anymore, and the governing 
parameters of the \gls{dkt} cases are \gls{Ga} and \gls{kappa}.
%
We explore two different particle shapes, namely spheres and oblate spheroids
with $\gls{chi}=1.5$, both with $\gls{kappa}=1.5$ and $\gls{Ga}=110.56$.
%
The size of the computational domain measures $[10.66 \times 10.66 \times 21.33]%
\gls{p:deq}^3$, where \gls{p:deq} is the diameter of a sphere with the same volume
as the particle considered.
%
Finally, for each particle shape we perform simulations with angular motion
enabled or suppressed.
%

Figure \ref{fig:DKT_sketch} shows a sketch of the computational setup, indicating
the set of initial particle positions.
%
We refer to the particle which is initially at a lower vertical position as the 
leading particle, and to the other particle as the trailing particle.
%
The initial position of the leading particle is always at $8\gls{p:deq}$ above the 
bottom boundary of the computational domain.
%
The initial condition of the trailing particle is varied, sweeping an area of
$[5\times 8.75]\gls{p:deq}^2$ in the horizontal and vertical direction, 
respectively.
%
This area is uniformly sampled leaving an horizontal and vertical distance of 
$0.625 \gls{p:deq}$ and $1.25\gls{p:deq}$, respectively, between neighboring
initial conditions.
%
%
%
%
%
%
%
%
%
This results in $72$ simulations for each configuration, a total number of 
$288$ \gls{dkt} cases.
%
In order to reduce the influence of mirror particles due to periodicity, we 
locate the plane containing the initial condition of the trailing particle in the 
plane $x=y$.
%
We define the horizontal coordinate contained in this plane as $x'$ and 
the relative position of the trailing particle with respect to the leading particle 
in this plane as
%
\begin{equation}\label{eq:dkt/xr}
\vec{x}_r = \vec{x}_\text{trailing} - \vec{x}_\text{leading}.
\end{equation}
%

%


\begin{figure} 
\begin{center}
\includegraphics[scale=1.0]{fig23.pdf} 
\end{center}
\caption{Outline of the \glsentrylong{dkt} simulations from a) lateral and c)
top views. b) View perpendicular to the plane where the trailing particle initial 
condition is located.
\label{fig:DKT_sketch}} 
\end{figure} 


%
One of the key aspects in this computational setup (non-periodic in the 
direction of gravity) is to keep the particles inside the computational domain
for sufficiently long time intervals.
%
Therefore, we need a good estimate of the average settling velocity of the 
system (both particles), and a sufficiently large domain in the vertical 
direction to accommodate the variations of this settling velocity.
%
It should be mentioned that these variations can be large due to collision
events.
%
From trial simulations we found that imposing the reference Reynolds number
based on the free stream velocity imposed at the inlet $\gls{Reref}=\gls{Uref}%
\gls{p:deq}/\gls{nu}$ slightly higher than the terminal Reynolds number obtained
for a single particle in the same configuration leads to successful simulations.
%
%
From this experience we set $\gls{Reref}=1.025\gls{p:Redeq;0}$ in all the cases.
%
The following phases of the evolution of the system are identified:
%
\begin{enumerate}
   \item {\bf Initial upward drift}: if particles are initially at a sufficient
         distance away from each other, they slowly drift upwards in the computational
         domain because $\gls{Reref}>\gls{p:Redeq;0}$.
         For cases in which collisions do not occur, this is the only phase
         of the problem. For cases in which particles are close enough to each 
         other the \gls{dkt} event is triggered since the start of the simulations, and
         this phase is skipped.
   \item {\bf \gls{dkt} event}: if the trailing particle is attracted by the wake of
         the leading one, the former drafts towards the latter and they eventually 
         collide.
         This results in an enhancement of the settling velocity of both particles.
         %
         %
         Having drifted upwards in the previous phase leaves more clearance for the
         \gls{dkt} event to occur without encountering the bottom boundary.
   \item {\bf Final upward drift}: after the \gls{dkt} event both particles end
         up at a similar height and both drift upwards while repelling each
         other.
\end{enumerate}
%
%
%
%
%
%
%
%
%
%
%
%
%
%
%

We have verified that all the non-colliding cases have been run for at least
the maximum time to first interaction observed ($678\gls{tg}$, from rotationally-locked
spheroids with relative initial position of the trailing particle $(3.125,10)\gls{p:deq}$)
plus $100\gls{tg}$.
%
This results in all the cases simulated for at least $778\gls{tg}$.

%


Figure \ref{fig:DKT_timehistory} shows the time history of two \gls{dkt} 
simulations as an example. 
%
In both cases the angular motion is enabled and the initial condition of the 
trailing particle is $(x'_r/\gls{p:deq},z_r/\gls{p:deq})=(2.5,7.5)$.
%
For every simulation in which a collision takes place, we define the time 
to first collision, \gls{mp:tkiss}, and the interaction time \gls{mp:tinter}.
%
The former is the time between the begin of the simulation and the first
collision.
%
The latter is defined as the time between the first collision and the last 
time instant in which the particle centers approach each other.
%
The definition of these two quantities is indicated in figure 
\ref{fig:DKT_timehistory}.
%
Please note that for most of the cases of spheres with angular motion enabled
the interaction time is almost negligible (see figure 
\ref{fig:DKT_timehistory}c).
%
\begin{figure} 
\makebox[\textwidth][c]{ 
%
\includegraphics[scale=1.0]{fig24.pdf} 
}
\caption{Time history of the vertical positions of a) spheres and b) spheroids
of $\gls{chi}=1.5$ and the distance between particle centers (c,d).
%
In both cases $(x_r/\gls{p:deq},y_r/\gls{p:deq})=(2.5,7.5)$ and the time is shifted so
that the instant of the first collision is $t=0$.
%
In (a-b) the trailing particle is represented with a solid line and the leading particle
with a dashed line. 
%
The gray shading indicates the time interval in which particles are in contact.
%
The vertical dotted lines illustrate the definition of the time to first collision,
\gls{mp:tkiss}, and the interaction time, \gls{mp:tinter}.
%
%
\label{fig:DKT_timehistory}} 
\end{figure} 


 
%

\section*{Supplementary data} 

\label{SupMat}Supplementary material (animations) have the digital object identifier
\href{https://dx.doi.org/10.5445/IR/1000151148}{\bf 10.5445/IR/1000151148}.

%

\section*{Acknowledgements}
The computations were partially performed on the
supercomputers ForHLR and HoreKa funded by the Ministry of Science, Research and the
Arts Baden--W\"urttemberg and by the Federal Ministry of Education and Research.

\section*{Funding}
This work was supported by the German Research Foundation (DFG)
under Project UH 242/11-1.

\section*{Declaration of interests}
The authors report no conflict of interest
%
%
%
\section*{Author ORCID}
M. Moriche,           \url{https://orcid.org/0000-0003-2855-8084}; 
M. Garc\'ia-Villalba, \url{https://orcid.org/0000-0002-6953-2270};
M. Uhlmann,           \url{https://orcid.org/0000-0001-7960-092X};
%
%

\bibliographystyle{authoryear} 
\addcontentsline{toc}{section}{References}
\bibliography{biblio}

\end{document}
