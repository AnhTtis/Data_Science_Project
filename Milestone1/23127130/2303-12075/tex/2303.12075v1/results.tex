
In this section we present the results obtained for many-particle cases after
particles are released.
%
We also present additional simulations of settling particle pairs in order to
analyze their ``drafting-kissing-tumbling'' dynamics.
%

%
\subsection{Clustering phenomena in many-particle cases}
\label{sec:results/multi}


\begin{table} 
\caption{Single-particle regime and time-averaged results of the
present cases and those in the work of \cite{uhlmann:2014a}}
\label{tab:cases_results}
\makebox[\textwidth][c]{ 
%
\begingroup \footnotesize %
\begin {tabular}{l|cc|ccccc|c}%
\toprule Case&Regime single part.&\gls {p:Redeq;0}&\gls {p:Redeq}& {$\frac {\gls {mp:vup_ws;avg}}{\gls {wref}}$} &$\frac {\gls {mp:vup_wstd;avg}}{\gls {wref}}$&$\frac {\gls {mp:vup_ustd;avg}}{\gls {wref}}$&$\frac {\gls {mp:vvort_s;avg}}{\gls {mp:vvort_s;rnd}}$&\\%
\toprule G111&steady vertical&\ensuremath {105}&\ensuremath {130}&\ensuremath {-1.2343}&\ensuremath {0.3753}&\ensuremath {0.1715}&\ensuremath {0.8695}&Present\\%
G152&steady oblique&\ensuremath {158}&\ensuremath {192}&\ensuremath {-1.2135}&\ensuremath {0.3806}&\ensuremath {0.1784}&\ensuremath {0.9282}&\\\toprule %
M121&steady vertical&\ensuremath {141}&\ensuremath {142}&\ensuremath {-1.0035}&\ensuremath {0.0924}&\ensuremath {0.0596}&\ensuremath {0.3452}&Reference (Uhlmann \\%
M178&steady oblique&\ensuremath {234}&\ensuremath {263}&\ensuremath {-1.125}&\ensuremath {0.1853}&\ensuremath {0.1249}&\ensuremath {0.6193}& \& Doychev, 2014)\\%
\end {tabular}%
\endgroup %
}
\end{table} 

When particles are released, they start to interact with their neighbors by
repeated drafting-kissing-tumbling events.
%
The interaction between particles is intense in both cases \verb!G111! and 
\verb!G152!, and rather similar (see animations in the supplementary material).
%
In the following we present: i) the enhanced settling and quantification of 
clustering, ii) the angular motion of the particles, iii) the arrangement of 
particles' trajectories and iv) a visualization of the main features of the 
flow.



%
\subsubsection{Enhanced settling and quantification of clustering}
\label{sec:results/multi/clustering}
%
\begin{figure} 
\makebox[\textwidth][c]{ 
%
\includegraphics[scale=1.0]{fig4.pdf} 
}
\caption{Time history of a) enhancement of the settling velocity (\gls{vup_ws}),
and standard deviation of c) settling and d) horizontal velocity, normalized
with the reference settling velocity from the single particle counterpart 
(\gls{wref}). b) Temporal evolution of the standard deviation of Vorono\"i cell
volumes (\gls{mp:vvort_s}), normalized with the value obtained for a random
Poisson process (\gls{mp:vvort_s;rnd}). Reference data for spheres is from 
\cite{uhlmann:2014a}.
\label{fig:wpvorstd}} 
\end{figure} 
%
Figure~\ref{fig:wpvorstd}a shows the time history of the average particle
settling velocity defined in \eqref{eq:vomep}.
%
%
%
%
It can be seen that in both present cases the ensemble of
mildly-oblate spheroids reaches on average a much enhanced settling
velocity magnitude after an initial transient of approximately
$400\gls{tg}$, after which the values saturate and continue fluctuating
around a value of roughly $-1.25$.
%
This behavior is quite in contrast to the known results for spheres (also
included in the figure for reference) for which a transition from expected
settling to enhanced settling occurs in the corresponding range of Galileo 
numbers, and for which the enhancement of the settling velocity was found to be
less pronounced (only roughly half of the present increase). 
%
Instead, the present suspension of spheroids with $\gls{chi}=1.5$ appears to
behave qualitatively similar to the much more flattened ($\gls{chi}=3$), almost
neutrally buoyant ($\gls{kappa}=1.02$) spheroids of \cite{fornari:2018b}, for
which the magnitude of the mean collective settling velocity was found to
increase by over $30$ percent. 
%

For completeness, let us report that the amplitude of the particle
velocity fluctuations (cf.\ figure~\ref{fig:wpvorstd}c,d) follows a
similar trend as the mean settling velocity, with a nearly linear
growth during the initial transient and subsequent fluctuations around
time-averaged values of approximately 
$\gls{vup_wstd} \approx 0.4\gls{wref}$, $\gls{vup_ustd} \approx
0.2\gls{wref}$. 
%
%
It should be noted that a clear signature of the oblique motion of
individual particles just after their release is visible in case
\verb!G152!, as can be seen in terms of an initial peak of the intensity of the
horizontal particle motion visible in the inset in figure~\ref{fig:wpvorstd}d.
%
After approximately $10\gls{tg}$ this peak disappears, and the two suspensions of spheroids at different
Galileo numbers exhibit very similar temporal evolutions. 

%
In order to investigate the spatial structure of the disperse phase,
we make use of the normalized Vorono\"i cell volume
%
\begin{equation}\label{eq:problem/defs/vvort}
\gls{mp:vvorti} =  \frac{\gls{mp:vvori}}{\gls{mp:vvor_mp}},
\end{equation}
%
where \gls{mp:vvori} is the volume of the $i$th cell of the  Vorono\"i
diagram obtained from a three-dimensional tessellation of space based
on the centroid locations of the particles
\citep{monchaux:2012,uhlmann:2014a}.  
%
In order to quantify the tendency to cluster we compare the standard
deviation of \gls{mp:vvorti} with the same quantity obtained in a
random Poisson process (RPP) of particles with the same shape and with
the same solid volume fraction.
%
For a mean concentration of particles such that $\gls{svf}=5\cdot 10^{-3}$,
these RPP reference values are $\gls{mp:vvort_s;rnd}=0.4176$ for
oblate spheroids of $\gls{chi}=1.5$ (determined in the framework of
the present work) and $\gls{mp:vvort_s;rnd}=0.4146$ for spheres
\citep[given in ][]{uhlmann:2020}. 
%
The graph in figure~\ref{fig:wpvorstd}b shows the temporal evolution
of the standard-deviation of the Vorono\"i cell volumes, normalized
with the reference value from RPP.
A direct qualitative correspondence with the temporal evolution of the
(magnitude of the) mean settling velocity in
figure~\ref{fig:wpvorstd}a can be observed. More specifically, after a
similar initial transient with an approximately linear growth of
\gls{mp:vvort_s}, the present suspensions of spheroids at both Galileo
number values reach very large clustering intensities, which by far
exceed those reported for spheres \citep[cf.\ case \texttt{M178}
from][]{uhlmann:2014a}. 
Based on the previous knowledge for settling spheres, and on the fact
that the present spheroids with $\gls{chi}=1.5$ are not too far from a
spherical shape, the strong clustering of the lower-Galileo case 
\verb!G111! is unexpected.

%
Next, we present time-averaged data of some of the time-dependent
quantities discussed above.
%
We discard the initial transient, and we start collecting statistics at
$t=500\gls{tg}$, resulting in a sampling time interval of approximately
$1000\gls{tg}$.
%
%
\begin{figure} 
\makebox[\textwidth][c]{ 
%
\includegraphics[scale=1.0]{fig5.pdf} 
}
\caption{%
  (a) Time-averaged values of the standard deviation of Vorono\"i cell
  volumes, \gls{mp:vvort_s;avg}, normalized with the RPP reference
  values versus \gls{Ga}.
  (b) Magnitude of the time-averaged mean settling velocity,
  \gls{vup_ws}, versus \gls{mp:vvort_s;avg}. 
  Reference data for spheres from \cite{uhlmann:2014a} and
  \cite{doychev:2014}. 
  \label{fig:enhanced_settling_vs_clustering}
} 
\end{figure} 
%
Figure~\ref{fig:enhanced_settling_vs_clustering}a shows the
time-averaged values of \gls{mp:vvort_s} plotted versus the Galileo
number. It can be clearly seen that the present suspensions of
spheroids feature strong clustering with little effect of varying the
value of the Galileo number.
%
Interestingly, it turns out that the magnitude of the mean settling
velocity normalized by the single-particle reference value
is approximately proportional to the standard-deviation of
the Vorono\"i cell volumes normalized by its random value, i.e.\ to the
clustering intensity.
%
This relation is shown in
figure~\ref{fig:enhanced_settling_vs_clustering}b, for the present
spheroids and for the sphere suspensions of \citep{uhlmann:2014a} and of
\citep{doychev:2014}.
%
This observation should be checked with the help of additional data-sets in
the future, since -- if confirmed -- it might open up a possibility to
determine the mean settling velocity of a collective from knowledge on the
spatial structure of the dispersed phase alone. 
%


\begin{figure} 
\makebox[\textwidth][c]{ 
%
\includegraphics[scale=1.0]{fig6.pdf} }
\caption{Probability density function of the normalized Vorono\"i cell volumes
in a) linear and b) logarithmic scale.
%
Reference data for spheres is from \cite{uhlmann:2014a}.
\label{fig:vorvol_Ga}} 
\end{figure} 
%
The distinct spatial arrangement of the spheroids compared to spheres can be further
characterized with the aid of the probability density functions (PDFs) of the
normalized Vorono\"i cell volumes \gls{mp:vvort}.
%
Figure \ref{fig:vorvol_Ga} shows the PDF of \gls{mp:vvort} of both cases with
spheroids (\verb!G111! and \verb!G152!) together with the PDF of a random
Poisson process \citep{uhlmann:2020} and the PDFs of two cases with spheres taken from 
\citet{uhlmann:2014a} (their cases \verb!M121! and \verb!M178!).
%
%
%
%
%
The same qualitative behavior is observed in both cases with spheroids: the
probability of finding volumes smaller than the average ($\gls{mp:vvort}=1$) is
high compared to a random Poisson process. 
%
This indicates that most of the particles form part of clusters.
%
Similarly, the probability of finding large volumes ($\gls{mp:vvort}\gtrsim 2$)
is higher than that of a random Poisson process, which implies the presence of
particles in large void regions.
%
Contrarily, case {\tt M121} from \cite{uhlmann:2014a} shows a tendency to a more
ordered state, where the probability of finding volumes similar to the average
value is higher.
%


%
\subsubsection{Angular velocity and orientation of particles}
\label{sec:results/multi/angular}

Figure \ref{fig:angular_timehistory}a shows the time history of the standard 
deviations of the horizontal and vertical components of the angular velocity
of the particles
normalized with $\gls{wref}/\gls{p:deq}$.
%
Figure \ref{fig:angular_timehistory}b shows the time history of the averaged and
standard deviation of the tilting angle \gls{mp:az}, which is defined as the
angle between the symmetry axis of the spheroid and the vertical direction 
($0\le\gls{mp:az}\le 90^\circ$). 
%
The time evolution of these four quantities is remarkably different from
the time evolution of the quantities reported in Figure \ref{fig:wpvorstd}.
%
Within a few gravitational time units after the particle release all quantities
in figure \ref{fig:wpvorstd} increase from zero to values close to their 
asymptotic time-averages, after which they vary only very mildly.
%
This indicates that the angular motion of the particles shows less sensitivity
to collective effects than does the linear motion. 
%
Note that the converged values are slightly higher in case \verb!G152! when compared to 
case \verb!G111!, except for \gls{mp:az;std}, that presents values which are approximately equal in both cases.
%
%
%
%
%
%
%
%
%
%

%
\begin{figure} 
\makebox[\textwidth][c]{ 
%
\includegraphics[scale=1.0]{fig7.pdf} 
}
\caption{Time history of a) standard deviation of the angular velocity and b)
average and standard deviation of the orientation angle \gls{mp:az}.
\label{fig:angular_timehistory}} 
\end{figure} 

%
Now let us focus on the probability density function of these angular
quantities.
%
Figure \ref{fig:angular_vel_pdf} shows the PDF of the vertical and horizontal
components of the angular velocity.
%
%
%
%
%
%
%
%
%
%
%
%
We have tried several known PDFs and we found Laplace's distribution
%
\begin{equation}
f\left(x;\beta\right) = \left(2\beta\right)^{-1}\exp\left(\left|x\right|/\beta\right)\,,
\end{equation}
%
with $\beta$ being a free parameter (cf. fitted numerical values in the figure),
the best fit to both components of the angular velocity. 
%
This highlights the exponential tails, i.e. the importance of extreme events of the
angular particle motion.
%
The vertical component shows, however, an approximately $20\%$ smaller value for
$\beta$ than the horizontal counterpart, indicating a more localized
distribution with higher kurtosis.
%
\begin{figure} 
\makebox[\textwidth][c]{ 
%
\includegraphics[scale=1.0]{fig8.pdf} 
}
\caption{Probability density functions of the a) vertical and b) horizontal
components of the angular velocity.
The curves are fitted to a Laplace distribution whose parameter $\beta$ is 
indicated in the legend. The Gaussian curve is shown for comparison purposes.
\label{fig:angular_vel_pdf}} 
\end{figure} 
%
Similarly, we found the best fit of the estimated PDF of the tilting angle 
\gls{mp:az} with respect to the vertical to be a Gamma distribution
%
\begin{equation}
f\left(x;k,\theta\right) = %
 \frac{x^{k-1}}{\theta^k\,\Gamma(k)}\exp\left(-x/\theta\right) \,,
\end{equation}
%
where $k$ and $\theta$ are the shape and scale parameters.
%
Figure \ref{fig:angular_az_pdf}a shows the PDF of  \gls{mp:az} and the fitted 
Gamma distribution.
%
The obtained fitting shows very good agreement in the range
$0^\circ<\gls{mp:az}\lesssim30^\circ$.
%
For higher values $\gls{mp:az}>30^\circ$, the PDF of each case shows a lower
decay rate compared to the fitted Gamma distribution for both cases.
%
The shape parameter of the fitted Gamma distribution in both cases ($k\approx %
2.3$) indicates a fast, but non-abrupt approach to zero of the PDF in the limit
$\gls{mp:az}\rightarrow0^+$.
%
Please recall that the shape parameter of the Gamma distribution can infer the
following: when $k\leq 1$ the maximum probability is located at $\gls{mp:az}=0$
and then the PDF decreases monotonically as \gls{mp:az} increases, when $k>1$
the limit of the PDF as $\gls{mp:az}\rightarrow0^+$ is
zero (with strong gradients in the vicinity of $\gls{mp:az}=0$ for values of $k$
closer to unity), and when $k\geq3$ the distribution shows a slow increase of the
probability in the vicinity of zero (see figure \ref{fig:angular_az_pdf}b).
%
%
%
%
%
%
%
%
%
Finally, the non-intuitive zero value of the PDF in the limit
$\gls{mp:az}\rightarrow 0^+$ can be explained by the circumferential shape of
the bins used to generate it.
%
In order to obtain the PDF for a specific value of ${\gls{mp:az}}_0$, we define a 
bin by its edges $[{\gls{mp:az}}_a,{\gls{mp:az}}_b]$, where 
${\gls{mp:az}}_a < {\gls{mp:az}}_0 < {\gls{mp:az}}_b$.
%
In the limit $\gls{mp:az}\rightarrow 0^+$ and for increasing resolution of the
PDF (${\gls{mp:az}}_b - {\gls{mp:az}}_a\rightarrow 0$), the area on the surface
of a sphere between the edges tends to zero, and as a consequence the
probability of finding occurrences of $\gls{mp:az}$ such that
${\gls{mp:az}}_a < \gls{mp:az} < {\gls{mp:az}}_b$, also tends to zero.
%
%
%


%
%
%
%
%
%
%
%
%
%
%
%
%
%
%
%
%
%
%
%
%
%
%
%
%
%
%
%
%
%
%
%
%
\begin{figure} 
\makebox[\textwidth][c]{ 
%
\includegraphics[scale=1.0]{fig9.pdf} 
}
\caption{a) PDF of the tilting angle \gls{mp:az} of cases {\tt G111} and
{\tt G152} (the same information with the $y$ axis in logarithmic scale is shown
in b).
%
A fitted Gamma distribution is included (parameters from fitting included
in the legend). 
\label{fig:angular_az_pdf}} 
\end{figure} 


\subsubsection{Trajectories}
\label{sec:results/multi/trajectories}

%
\def\mftop{\ref{fig:vis_trajectories_top}}
\def\mfsa{\ref{fig:vis_trajectories_side_G111}}
\def\mfsb{\ref{fig:vis_trajectories_side_G152}}
%
\begin{figure} 
\makebox[\textwidth][c]{ 
%
%
\includegraphics[scale=1]{fig10.pdf} 
}
\caption{Top view of trajectories of the particles' center of gravity during a
time span of $[t_0-T_t,t_0]$, where $T_t=0.54\gls{tg}$ for cases {\tt G111}
(left column) {\tt G152} (right column). Trajectories are coloured according to
the particle's velocity relative to mean velocity of the mixture (red downwards,
blue upwards). 
\label{fig:vis_trajectories_top}} 
\end{figure} 

\begin{figure} 
\makebox[\textwidth][c]{ 
%
%
\includegraphics[scale=1]{fig11.pdf} 
}
\caption{As in \ref{fig:vis_trajectories_top} but for case {\tt G111} only and
viewed from the side.
\label{fig:vis_trajectories_side_G111}} 
\end{figure} 

\begin{figure} 
\makebox[\textwidth][c]{ 
%
%
\includegraphics[scale=1]{fig12.pdf} 
}
\caption{As in \ref{fig:vis_trajectories_top} but for case {\tt G152} only 
and viewed from the side.
\label{fig:vis_trajectories_side_G152}} 
\end{figure} 

Now we turn our attention to the trajectories of the particles.
%
Figures \mftop{}, \mfsa{} and \mfsb{} show the trajectories followed by the 
center of gravity of the particles during a time span of $0.54\gls{tg}$ leading up to three
successive time instants in two perpendicular projections.
%
With this representation, considering the initialization procedure
(\S~\ref{sec:ini}) and if, hypothetically, collective effects were absent, 
particles would be represented by single points in the case \verb!G111! (single
particle follows steady vertical trajectory) and by
straight horizontal lines in the case \verb!G152! (single particle follows 
steady oblique trajectory).
%
The time instants selected are $t/\gls{tg}\approx 10, 400$ and $1500$, which 
correspond to: i) shortly after the particles are released, ii) the end of the
linear growth of \gls{vup_ws} and \gls{mp:vvort_s} (see figure
\ref{fig:wpvorstd}), and iii) the asymptotic state close to the end of the
simulated time, respectively.
%
Shortly after particles are released ($t/\gls{tg}\approx 10$), there is almost
negligible motion of the particles in case \verb!G111! (figures \mftop{}a and
\mfsa{}a show point-like trajectories) and a small lateral motion in case
\verb!G152! (figures \mftop{}b and \mfsb{}a show trajectories with a noticeable
horizontal component).
%
As both cases evolve, the trajectories show elongated shapes along the vertical
direction, indicating an enhancement of the settling velocity compared to the
single particle configuration.
%
On average, the length of the trajectories in the vertical direction keeps growing during the 
constant acceleration phase ($t\lesssim 400\gls{tg}$) forming columnar
clusters whose size is comparable to that of the computational domain
(figures \mftop{}c and d, \mfsa{}b and \mfsb{}b).
%
Again, the similarity of both cases is clearly noticeable.
%
%
%
%
%
%
%
%
%
%
There is, however, a tendency of the clusters in the case of lower Galileo to be
more stable compared to the case of higher Galileo when $t\approx 400\gls{tg}$
(see figures \mfsa{}b and \mfsb{}b).
%
We believe that the smaller flow disturbances of the low Galileo case allow the
presence of clusters with a cross section of the order of tens of \gls{p:deq} 
(see figure \mftop{}c) to fill the entire domain in the vertical direction,
whereas clusters of the same size in the horizontal direction of the higher
Galileo case (see figure \mftop{}d) are not stable and thus, appear to be more
localized.
%
%
%
%
This could explain the higher enhancement of the settling velocity with respect 
to the single particle case observed in the case at lower Galileo (figure 
\ref{fig:wpvorstd}a) but, it should be noted that this is only possible due to
the periodic configuration.
%
Therefore, the larger enhancement in \gls{vup_ws} observed in figure 
\ref{fig:wpvorstd}a for case \verb!G111! should be interpreted with care.
%
In the horizontal direction, clusters show a continuous growth until they reach
a size comparable to the computational domain (figures \mftop{}e and f).
%


\subsubsection{Flow visualization}
\label{sec:results/multi/visu}

%
\begin{figure} 
%
%
\makebox[\textwidth][c]{ 
\includegraphics[scale=0.75]{fig13.pdf} 
}
\caption{Visualization of isocontours of $\gls{f:Q}\gls{p:deq}^2/\gls{Ug}^2=0.7$
(case {\tt G111}) and $0.83$ (case {\tt G152}) and $\gls{f:vu_z;filt}=%
\gls{vu_z_mf}-0.5\gls{Ug}$.
Particles are represented in pink, isocontours of \gls{f:Q} with grey-colored 
surfaces and isocontours of \gls{f:vu_z;filt} with yellow surfaces.
Top row shows the whole domain (\gls{f:Q} and \gls{f:vu_z;filt}), 
bottom row shows a part of the domain (only \gls{f:Q}).
First and second columns correspond to the instant before the release of the
particles of cases {\tt G111} and {\tt G152}, respectively.
Third and fourth columns correspond to a converged state of each case.
\label{fig:visu}} 
\end{figure} 
%
%

Figure \ref{fig:visu} shows visualizations of the cases \verb!G111! and
\verb!G152! just before releasing the particles (panels a, b, e and f) and two
converged states (panels c, d, g and h).
%
In the figure the particles are represented together with isocontours of the 
second invariant (\gls{f:Q}) of the velocity gradient tensor \citep{jeong:1995}
and isocontours of the filtered vertical
velocity \gls{f:vu_z;filt}.
%
The filter applied to \gls{vu_z} is a Gaussian filter of width $2.3\gls{p:deq}$
used to visualize large-scale velocity fluctuations.
%
Just before particles are released ($t/\gls{tg}=0$) we observe that in
case \verb!G111! ($\gls{p:Redeq;0}=105$) the most common flow structure is a toroidal
vortex around each particle (figure \ref{fig:visu}e), whereas in case
\verb!G152! ($\gls{p:Redeq;0}=158$) we also frequently observe a double-threaded
wake (figure \ref{fig:visu}f).
%
The similarity of these vortical structures with the analogous single-particle
case at the given \gls{p:Redeq;0} is due to the low concentration of particles
in both cases.
%
After convergence has been reached ($t/\gls{tg}>1000$) both cases show large
regions of high-speed downward flow, which are absent when particles are 
released ($t=0$).
%
The strong clustering of both cases discussed above is clearly seen when
comparing the initial and converged snapshots of both cases. 
%

%
%
%
%
%
%
%
%
%
%
%
%
%
%
%
%
%







%
%
%
%
%
%
%
%
%
%
%
%
%
%
%
%
%
%
%
%
%
%
%
%
%
%
%
%
%
%
%
%
%
%
%
%
%
%
%
%
%
%
%
%
%
%
%
%
%
%
%
%
%
%
%
%
%
%
%
%
%
%
%
%
%
%
%
%
%
%
%
%
%
%
%
%
%
%
%
%
%
%

%
%
%
%
\subsection{Drafting-kissing-tumbling}
\label{sec:results/dkt}
%

%
%
%
%
%
%
%
%
%
%
Since animations show that interactions between particle pairs occur frequently
in the many-particle cases, we now proceed to an analysis of the interaction of
such pairs in isolation.
%
As we will see, these interactions are quite sensitive to the particle shape, 
and we believe that pairwise interactions are the key to understand the tendency
of oblate spheroids to cluster at Galileo numbers for which spheres do not
exhibit clustering.
%
We restrict the analysis to one combination of Galileo number and density ratio  
($\gls{Ga}=111$, $\gls{kappa}=1.5$) for which a single spheroid of aspect 
ratio $\gls{chi}=1.5$ and a single sphere result in a steady vertical regime 
\citep{jenny:2004,moriche:2021}.
%
As mentioned above, many-particle cases in the dilute regime ($\gls{svf}=%
5\cdot 10^{-3}$) with this combination of \gls{Ga} and \gls{kappa} present
strong clustering in the case of oblate spheroids with $\gls{chi}=1.5$ and 
no clustering in the case of spheres \citep{uhlmann:2014a}.
%
We include an additional series in this set of \gls{dkt} cases, in which the
angular motion is suppressed. Therefore, we have these four configurations:
%
\begin{itemize}
   \item Free-to-rotate spheres (angular motion enabled).
   \item Rotationally-locked spheres (angular motion suppressed).
   \item Free-to-rotate spheroids (angular motion enabled).
   \item Rotationally-locked spheroids (angular motion suppressed).
\end{itemize}
%
This is motivated by the study of \cite{kajishima:2004} who demonstrated that
suppressing the rotation of spheres leads to an enhanced clustering tendency.
%
%
We perform a parametric sweep of $72$ initial relative particle positions for
each configuration, resulting in a total number of $288$ simulations (see figure
\ref{fig:DKT_trajectories}i).
%
%
%
%
%
%
The problem setup is analogous to the setup of the many-particle cases presented
in \S~\ref{sec:method}, except that only two particles are present, and that we
use an inflow/outflow configuration in the vertical direction as described in
detail in appendix \ref{sec:dkt}.
%

\begin{figure} 
\begin{center}
%
\includegraphics[scale=0.8]{fig14.pdf} 
\end{center}
\caption{Trajectories of the trailing and leading particles for selected initial 
positions of spheres (a-d) and spheroids with $\gls{chi}=1.5$ (e-h), all with 
$\gls{kappa}=1.5$ at $\gls{Ga}=111$.
%
The reference frame is translating downwards at a constant speed slightly smaller than
the settling velocity of a single particle ($0.975\gls{wref}$).
%
Each panel contains the data of the cases with angular motion enabled and 
suppressed for a single initial condition and particle shape (see legend).
i) A sketch of the problem and the coordinates used is presented, in which the 
leading particle is represented with its actual shape and the trailing particle
with a marker.
%
The point markers correspond to all the initial conditions which we have computed,
and the symbol markers to those initial conditions which are shown in panels a-h.
j) Sketch of the $x',y'$ coordinates.
\label{fig:DKT_trajectories}} 
\end{figure} 
%
Figure \ref{fig:DKT_trajectories} shows the trajectories of $16$ selected cases
projected onto a vertical plane intersecting the particles at their initial 
position. 
%
These cases correspond to different initial conditions of the four selected
configurations.
%
Please note that in these four cases particles do collide at least
once, while this is not the case for all initial separations
(as will be shown in figure \ref{fig:DKT_maps} below).
%
%
%
%
%
If we focus on the trajectories of the trailing particles, there is a clear
similarity in free-to-rotate and rotationally-locked spheroids: for these two
configurations the trailing particle of the four initial conditions shown in the
figure moves laterally until its center is vertically aligned with the leading
particle. 
%
Then, the trailing particle drifts towards the leading particle following an 
almost vertical path.
%
On the contrary, spheres present a different path along which the trailing
particle approaches the leading particle.
%
The trailing particle of both free-to-rotate and rotationally-locked spheres
presented in the figure drifts towards the leading one following 
an oblique path for a significant time during the final approach.

%
Interestingly, free-to-rotate spheres present a lateral shift (by roughly 
$\gls{p:deq}/4$) in their trajectory compared to their rotationally-locked
counterparts.
%
This is clearly evident in figure \ref{fig:DKT_trajectories}a, where the
trailing particle of the free-to-rotate pair first moves away laterally from the
leading particle, then making its way in a straight, oblique path towards it. 
%



%
\begin{figure} 
\begin{center}
%
\includegraphics[scale=0.9]{fig15.pdf} 
\end{center}
\caption{Trajectory of the trailing particle relative to the leading one for a)
spheres and b) spheroids. The close-up trajectories shown in the insets in a,b
are coloured with the angular velocity perpendicular to the plane shown (see 
legend). Line color and marker type follow the same convention as in figure 
\ref{fig:DKT_trajectories}.
\label{fig:DKT_trajectories_rel}} 
\end{figure} 
%
In figure \ref{fig:DKT_trajectories_rel} we show the trajectories of the 
trailing particle relative to the center of the leading particle for the same
configurations as presented in figure \ref{fig:DKT_trajectories}.
%
It can be seen how the rapid alignment along the vertical axis exhibited by the
spheroidal particles results in the possibility of finding the trailing particle
on either side of the vertical axis after the initial contact.
%
Spheres, on the other hand, which do not fully align vertically, do not cross
the vertical axis through the leading particle during the entire interaction.
%
%
%
%
%
%
%
%
A very interesting result is the robustness of the lateral shift of the trailing
particle for free-to-rotate spheres.
%
Independently of the four initial conditions shown in the figure,
the path followed by the trailing particle in the free-to-rotate sphere configuration
converges to a single master curve. 
%
This feature has been observed for a number of other initial conditions.
%
We attribute this lateral shift to a rotation-induced lift force resultant from
the finite angular velocity
(around the horizontal axis perpendicular to the plane shown in the figure)
reached by these particles (see inset of figure
\ref{fig:DKT_trajectories_rel}a).
%
The origin of the rotation of the particles ($\omega_{p,y'}<0$) is the shear
seen by the trailing particle because of the deficit in vertical velocity in the
wake of the leading particle.
%
A detailed analysis of the forces acting on a particle in motion in the wake of
another particle, however, is outside the scope of the present contribution, and
it is left as a worthwhile topic for future studies.
%
%
%
%
%
%
%
%
%
%
%
%
%
%
%
%
%
%
%
%
%
%
%
%
%
%
%
%
%
%
%
%


%
\begin{figure} 
\makebox[\textwidth][c]{ 
%
\includegraphics[scale=1.0]{fig16.pdf} 
}
\caption{Maps of a,b) time to first collision \gls{mp:tkiss} and
c,d) interaction time of the \gls{dkt} cases as a function of the initial
condition of the trailing particle. The $x'$ axis for the rotationally-locked
cases is flipped to facilitate the comparison.
%
Cases in which no interaction occurred in the evaluated time are represented
with black dots and interacting cases are represented with coloured markers.
%
The red line in panels a,b is an isocontour of $\gls{mp:tkiss}=100\gls{tg}$.
%
The panels e,f contain the ratio of \gls{mp:tkiss} of free-to-rotate cases
with respect to their rotationally locked counterparts
($\gls{mp:tkiss;FTR}/\gls{mp:tkiss;RL}$).
\label{fig:DKT_maps}} 
\end{figure} 
%
%
Next we consider on the complete set of initial conditions of all four
configurations.
%
Figure \ref{fig:DKT_maps} shows maps of the time to first collision,
\gls{mp:tkiss}, and the interaction time, \gls{mp:tinter}, for the cases showing
at least one \gls{dkt} event.
%
The interaction time, \gls{mp:tinter}, measures the
time between the first collision and the smallest time after which the distance between
the particles grows monotonically (see precise definitions in \S~\ref{sec:dkt}),
evaluated as a function of the initial position of
the trailing particle.
%
The curve which delimits those initial conditions that lead to particle contact
from those which do not is very similar in all four cases (figure
\ref{fig:DKT_maps}a,b).
%
This means that the region of attraction in the explored range of relative positions is
essentially insensitive to the precise particle shape (sphere vs. mildly oblate
spheroid) and to their ability to rotate.
%
Both spheres and the spheroids considered, independently of angular motion being
suppressed or not, present a roughly cylindrical region of radius approximately
$3\gls{p:deq}$ located downstream of the leading particle in which particles will
interact.
%
Now let us consider the time to first collision, \gls{mp:tkiss}, which can give
us insight into the particles' response to wake attraction mechanisms in an 
integral sense, as a function of the initial relative position of the particles
(figure \ref{fig:DKT_maps} a,b).
%
We find similar values of \gls{mp:tkiss} for free-to-rotate spheroids and
rotationally-locked spheroids and spheres, whereas free-to-rotate spheres
present somewhat larger values for the same initial conditions.
%
This result suggests that the angular motion of spheres causes these particle
pairs to approach more slowly than corresponding non-rotating spheres and oblate
spheroids.
%
%
The latter do not rotate continuously, instead the angular motion during
the approach phase is characterized by small
oscillations around the equilibrium position (with the symmetry axis 
vertically aligned).
%
For the sake of clarity we include the ratio of \gls{mp:tkiss} of free-to-rotate
particles with respect to their rotationally locked counterparts in
the auxiliary panels (e,f) for each initial condition.
%
It can be seen that free-to-rotate spheres with a small horizontal shift in 
their relative position show ratios as large as $2.5$.
%
Furthermore, this ratio increases with the vertical distance within the range of
parameters evaluated.
%
These differences disappear for the cases
whose initial relative position has almost no horizontal shift, indicating that a
larger horizontal shift is required to trigger the rotation of the trailing sphere.
%
%
%
%
%
%
%
%
%
%
%
%
%
%
%
%
%
Next let us focus on the interaction of particle pairs after the first
collision.
%
Only a maximum of one collision is observed in the entire data-sets for rotationally locked
spheres and spheroids and free-to-rotate spheres.
%
Free-to-rotate spheroids, however, present two or three collision events per 
parametric point for most of the chosen initial conditions, except for a few 
of them in which a single, or even four collisions occur.
%
Figure \ref{fig:DKT_maps} c,d shows the interaction time \gls{mp:tinter}, where
a clear ascending order of configurations with respect to the duration of
particle-pair interactions is identified:
%
%
%
i) free-to-rotate spheres present an almost negligible interaction time,
ii) rotationally locked spheres interact during approximately $5\gls{tg}$, the value
of \gls{mp:tinter} increases to approximately $20-40\gls{tg}$ for the
iii) rotationally locked spheroids, and it reaches values above of up to
$100\gls{tg}$ for the iv) free-to-rotate spheroids.
%
%


%
In the following we will analyze the temporal evolution of the distance
separating the two interacting particles after the first contact.
%
For this purpose we set the time of the first contact arbitrarily to zero, and
then average the data over the ensemble of realizations with different initial
separations.
%
We define the average post-collisional distance as follows:
%
\begin{equation}\label{eq:lavg}
L\left(\tilde{t}\right)=\frac{\sum_i^{N} %
\lVert \vec{x}^i_{r,trail}\left( \tilde{t}\right)\rVert}{N} \,,
\end{equation}
%
where $\vec{x}^i_{r,trail}$ represents, for each case $i$, the relative position
of the trailing particle with respect to the leading particle and $\tilde{t}$
measures the time after the first collision (recall from figure \ref{fig:DKT_maps}$(c,d)$
that the value of \gls{mp:tkiss} is case-dependent).
%
Figure \ref{fig:DKT_avgdistance_all} shows the average post-collisional
distance, $L\left(\tilde{t}\right)$, for the four investigated configurations.
%
First, let us compare rotationally locked spheres versus rotationally locked 
spheroids.
%
On average, rotationally locked spheres drift away from each other a distance
of approximately $2.5\gls{p:deq}$ in the first $10\gls{tg}$ after the initial contact
(figure \ref{fig:DKT_avgdistance_all}b).
%
The scenario for rotationally locked spheroids is, on average, slightly more
complex.
%
There is a local maximum of the inter-particle distance at 
$\tilde{t}\approx 20\gls{tg}$, after which particles start to approximate each
other again, and a local minimum of $L$ at $\tilde{t}\approx 30\gls{tg}$, after which
particles drift away from each other.
%
Second, let us focus on the effect that angular motion has on both particle
shapes considered.
%
For the spheres, the angular motion results in particles staying much 
further away from each other after the collision, with an approximately constant
offset of around $1\gls{p:deq}$ for $\tilde{t} > 10\gls{tg}$.
%
This considerable effect due to the spheres' rotation can be understood
from figure \ref{fig:DKT_animated} which depicts the relative motion of particle
pairs for exemplary cases, as will be discussed in more detail below.
%
Conversely, when angular motion is allowed for spheroids, 
particles remain on average closer to each other at times larger 
than approximately $10\gls{tg}$ after the first collision, and 
the local maximum of $L$ observed for the rotationally locked counterparts is
more pronounced.
%
%
%
%
%
%
%
%
\begin{figure} 
\begin{center}
%
\includegraphics[scale=1.0]{fig17.pdf} 
\end{center}
\caption{a) Time history of average distance \eqref{eq:lavg} 
after the first contact for the four configurations considered in the
\gls{dkt} configuration (see legend).
Non-colliding cases are excluded from the plot.
b) Zoom of panel a (see dashed rectangle in a).
\label{fig:DKT_avgdistance_all}} 
\end{figure} 


\begin{figure} 
\makebox[\textwidth][c]{ 
%
\includegraphics[scale=0.8]{fig18.pdf} 
}
\caption{Overlay of consecutive snapshots of the different \gls{dkt} configurations for the
cases whose trailing particle starts at $\vec{x}_r=(0.625,7.5)\gls{p:deq}$.
%
The initial and final snapshots are indicated by highlighting the particles' contour.
%
The time interval selected is such that starts at the first contact
($t=\gls{mp:tkiss}$) and ends after $16\gls{tg}$ for spheres and $32\gls{tg}$
for spheroids, sampling $8$ equispaced time instants.
%
The time between consecutive snapshots $\Delta t$ is indicated in the figure.
%
The reference frame is translating downwards at a speed slightly smaller than
the settling velocity of a single particle ($0.975\gls{wref}$).
%
Particles are identified by color (trailing: green, leading: purple).
Time and angular position are indicated with a small mark whose color changes
with time.
\label{fig:DKT_animated}} 
\end{figure} 

Let us now identify possible phenomena that lead to the strong differences
in the pairwise interaction times after the first collision.
%
First, the larger separation observed in spheres as compared to spheroids just
after the first collision may be attributed to the mechanism by which pairs of
particles start to tumble after they contact each other.
%
According to \cite{fortes:1987}, vertically aligned spheres which are in contact
form an unstable system which is the actual cause of the tumbling phase.
%
This instability is due to the fact that two spheres vertically aligned can be
seen as an elongated body.
%
Such body would tilt so that it settles maximizing drag.
%
Since the pairs of particles considered are not connected, they separate from
each other shortly after they start to tilt.
%
The ratio between height and width of the body formed by two vertically aligned
and touching spheres is equal to 2, while the same quantity for spheroids (with
their axis of symmetry aligned with the vertical) equals $2/\chi$, which amounts
to $4/3$ in the present case.
%
Therefore, we can expect a more abrupt initial tumbling motion in the case of 
spheres.
%
Please note that if the oblate spheroids considered had an aspect ratio of 
$\gls{chi}=3$, as considered by \cite{fornari:2018b}, the ratio between height
and width becomes $2/3$, which is smaller than unity. 
%
In such a situation the tandem of vertically aligned particles would be even
more stable. 
%
Indeed, \cite{fornari:2018b} reported a stick mechanism in which the oblate spheroids
stay together, suppressing the tumbling phase.
%
Regarding the angular motion, two different effects are considered as candidates
to explain the lower values of the particle-pair separation distance $L$ for
spheroids and higher values of $L$ for spheres, when compared to their
rotationally-locked counterparts.
%
In the case of spheroids the angular motion allows the particles to have a
stronger rocking motion around a horizontal axis after their first collision.
%
This rocking motion is a consequence of the leading particle tilting when pushed
downwards by the trailing particle.
%
Figure \ref{fig:DKT_animated}d shows the tilting of the leading particle
and the consequent rocking motion after the first collision. 
%
This results in a zig-zag trajectory and, as a consequence, particles
stay, on average, closer to each other, thereby increasing the probability for
repeated collisions.
%
The subsequent collisions show similar features, but the amplitude of their
rocking motion is decreased, and the distance between the particles after 
the collisions is increased.
%
In the case of spheres, the distance between centers of free-to-rotate spheres
is approximately $1.5$ times larger than that of the rotationally-locked counterparts.
%
We attribute this to a rotation-induced lift force on the trailing particle
that increases the distance between the two particles.
%
This rotation can be appreciated in figure \ref{fig:DKT_animated}b, and its
effect on the relative trajectory between the particles is visible in figure 
\ref{fig:DKT_trajectories_rel}a.


