%
Particle-laden flows
%
%
%
%
%
play an important role in natural and industrial systems, 
such as sediment transport in rivers, fluidized beds and
%
%
pollutants or hydrometeors in the atmospheric boundary layer.
%
%
%
%
%
%
%
%
%
%
%
%
%
%
%
%
%
%
It is well known that an isolated particle settling under gravity in
an otherwise ambient fluid exhibits a variety of regimes of motion
depending on its shape, mass density and size \citep{ern2012}.
Even for spherical particles diverse path regimes have
been observed, ranging from steady vertical to chaotic motion, and
including various intermediate states of different kinematic
complexity \citep[][cf.\ also
figure~\ref{fig:single_particle_regimes}]{jenny:2004,zhou:2015}.   
The transitions between the distinct regimes of particle motion are
the consequence of bifurcations in the flow pattern around the mobile  
particle, arising as the values of the governing parameters are varied. 
These parameters 
%
%
in the simplest single-particle case 
are the solid-to-fluid density ratio, $\gls{kappa}=%
\gls{rhop}/\gls{rhof}$, and the Galileo number, $\gls{Ga}=\gls{p:deq}\gls{Ug}/\gls{nu}$,
where \gls{p:deq} is the particle diameter (in the case of
non-spherical particles \gls{p:deq} is defined as the
diameter of a sphere with the same volume), 
$\gls{Ug}=\left(|\gls{kappa}-1|\left\|\gls{vg}\right\|\gls{p:deq}\right)^{1/2}$
is a gravitational velocity scale, \gls{nu} is the kinematic
viscosity, and \gls{vg} is the vector of gravitational acceleration.  
%

%
Whenever the volume fraction (\gls{svf}) of the particulate phase
becomes non-negligible, particle-particle interactions can lead to
significant collective effects. For dilute systems these 
interactions are predominantly of indirect type, acting through
long-range hydrodynamic forces. For denser systems direct contacts 
between two or more particles come into play more frequently, thereby
contributing to the exchange of momentum. In the present work we are
interested in dilute systems (with solid volume fractions below one
percent), which is why we will concentrate on this regime in the
following literature review. 
%
%
%
%
%
%
%
%
%
%
%
%
\begin{figure} 
\begin{center}
%
\includegraphics[scale=0.8]{fig1.pdf} 
\end{center}
\caption{Single particle regimes for heavy ($\gls{kappa}=1.5$) spheres and
  oblate spheroids with aspect ratio $\gls{chi}=1.5$ as a function of
  the Galileo number $\gls{Ga}$.
  Reference data for dilute suspensions of spheres with the same
  density ratio and Galileo number from 
  \cite{uhlmann:2014a} are also included.
  %
  %
  %
  The vortical flow structures in each regime are indicated with the
  aid of iso-surfaces of the \gls{f:Q}-criterion of \cite{hunt:1988}. 
  \label{fig:single_particle_regimes}} 
\end{figure} 
%
%
Depending on the values of the parameter triplet
$(\gls{svf},\gls{kappa},\gls{Ga})$, dilute suspensions
%
can exhibit spatial particle distributions with significant
non-homogeneity.
%
A non-homogeneously distributed particle phase in turn is often
accompanied by important macroscopic effects, such as an altered
mean settling velocity and a modification of the induced fluid flow
features.
%
%
%
%

At Galileo numbers of ${\cal O}(5\ldots 15)$ and density ratio $\gls{kappa}=2$
it has been
observed that dilute suspensions of spherical particles tend to form
horizontally-aligned pairs \citep{yin:2008}, while at larger 
Galileo numbers ($\gls{Ga}\approx 200$) and similar density ratios, where the
wake flow features a significant recirculation region, a vertical alignment is
more probable \citep{kajishima:2002,uhlmann:2014a}.  
%
\cite{kajishima:2002} were the first to investigate the formation of large, 
columnar-shaped particle clusters due to wake effects.
%
Their Particle-Resolved Direct Numerical Simulations (PR-DNS) in triply-periodic
boxes showed that clusters tend to form for particle Reynolds numbers exceeding
$200\ldots300$ ($Ga\approx153\ldots210$), leading to strongly enhanced average
settling velocities.
%
In their initial work, the particle rotation was suppressed for computational
simplicity, and a density ratio $\gls{kappa}=8.8$ was considered.
%
In the follow-up work of \cite{kajishima:2004}, the effect of particle rotation
was taken into account, and it was found that rotational motion leads to a lower
concentration of particles in clusters, since particles tend to escape a cluster
through a rotation-induced lift effect.
%
\cite{kajishima:2004} also determined a lower limit of the solid volume fraction
for the occurrence of clustering. 
%
The PR-DNS of \cite{uhlmann:2014a} were performed at a fixed density ratio 
$\gls{kappa}=1.5$, and it was observed that columnar clusters do form for 
$\gls{Ga}=178$ (which corresponds to an isolated particle in the steady oblique
regime), while no clusters are formed at $\gls{Ga}=121$ (steady vertical regime).
%
%
%
The authors suggest that the onset of clustering in dilute suspensions of
spherical particles is triggered by the bifurcation of the wake flow from steady
axi-symmetric to steady oblique, with the consequence that the particles in the
latter case drift horizontally (with random azimuthal angle) leading to an
enhanced probability of particle-particle encounters. 
%
%
%
%
%
%
%
%
%
%
%
%
%
Further data on collective effects upon clustering is available from the PR-DNS
studies of \cite{zaidi:2014}, \cite{fornari:2016} and \cite{seyed-ahmadi:2021},
which were all performed in similar triply periodic configurations, and from the
work of \cite{huisman:2016} who conducted experiments in a settling column with
glass beads in water. 
%
%
%
%
%
%
%
An overview of the average settling velocities from these various
data-sets for dilute suspensions of spherical particles in an
otherwise ambient fluid is given in
figure~\ref{fig:settlingLiterature}.
The PR-DNS results appear to give a consistent trend: for 
increasing values of the Galileo number the particles tend to settle 
at an average rate which is enhanced with respect to the velocity of
an isolated particle, and there appears indeed to be a cross-over
(from a reduced settling velocity to an enhancement) for
$Ga\approx150$.
%
The limited available experimental data in this parameter range is not
inconsistent with this picture, however showing a settling enhancement
already at lower Galileo number of $110$ and mild columnar cluster
formation.
%
\cite{huisman:2016} have identified the presence of the bounding 
container walls, which  presumably causes a large-scale recirculating
flow, as a possible cause for the observed discrepancy.

%
%
%
%
%
%
%
%
%
%
%
%
%
%
%
%
%
%
%
%
%
%
%
%
%
%
%
%
%
%
%
%
%
%
%
%
%
%
%
%
%
%
%

%
Dilute suspensions of non-spherical particles have received much less attention, mainly
due to the complexity associated with the particle shape 
and the corresponding increase in the size of the parametric space.
%
Spheroids have been the preferred choice of several authors, partly due to their
convenient parametrization.
%
Their shape is defined uniquely by the aspect ratio $\gls{chi}=\gls{dd}/%
\gls{aa}$, where \gls{dd} and \gls{aa} are the equatorial diameter and the
length of the symmetry axis, respectively (see figure
\ref{fig:problem_description}a,b).
%
%
%
%
%
%
The spheroidal shape allows for a smooth transition between flat,
disk-like shapes (for very large values of $\gls{chi}$), and elongated, fiber-like
geometries (for very small values of $\gls{chi}$), while 
a sphere is recovered for $\gls{chi}=1$.
%
When oblate spheroids ($\gls{chi}>1$) are considered, the regime maps
of a single particle are more complex when compared to spheres
\citep{zhou:2017}. 
%
Despite this increase in complexity, specific combinations of
\gls{chi} and \gls{kappa} result in a single oblate spheroid exhibiting the so-called 
``sphere-like scenario'' in which the first bifurcation for
increasing \gls{Ga} is regular, transitioning from a vertical to an oblique
regime \citep[see figure
\ref{fig:single_particle_regimes},][]{zhou:2017,moriche:2021}.
%
\cite{fornari:2018b} studied the settling of dilute suspensions of almost
neutrally buoyant ($\gls{kappa}=1.02$) oblate spheroids with a moderately-flat
shape $\gls{chi}=3$.
%
They considered both dilute and dense regimes at Galileo numbers at which a
single particle follows a steady vertical path ($\gls{Ga}=60$) or an
unsteady path which is vertical in the mean ($\gls{Ga}=140$).
%
For their most dilute cases they observed a large enhancement of the mean
settling velocity compared to the single particle case, with only 
small differences between the two Galileo numbers (cf.\
figure~\ref{fig:settlingLiterature}).  
%
\cite{fornari:2018b} have also detected non-homogeneous particle
distributions in the form of vertical particle trains with lateral
dimensions of the order of $10$ times the length of the symmetry axis,
based upon visualization and on the analysis of pairwise distribution
functions.  
%
The authors attributed the enhancement of settling velocity to the
occurrence of particle-pair interactions which lead to the formation
of piles of particles which practically stick to each other in the
case of oblate spheroids with $\gls{chi}=3$. 


%
Turning now to other shapes, \cite{seyed-ahmadi:2021} more recently studied the
behavior of settling cubes with $\gls{kappa}=2$, $\gls{svf}=0.01\ldots0.2$, and
Galileo numbers for which a single cube tends to a steady oblique path with very
small tilting angle ($\gls{Ga}=70$) and to a helical path ($\gls{Ga}=160$) for
large times.
%
For their smallest solid volume fraction these authors report an increase in the
relative settling velocity, which is similar to what is observed in suspensions
of spheres (cf.\ figure~\ref{fig:settlingLiterature}).
%
Interestingly, \cite{seyed-ahmadi:2021} observe that the intensity of columnar
cluster formation (measured with the aid of particle-pair distribution functions)
is somewhat lower in the case of cubes as compared to the spherical counterpart.
%
This effect is attributed by the authors to higher rotation rates in the former
case, leading to an enhanced lift force that increases their probability to
escape from an existing cluster.
%

%

%
%
%
%
%
%
%
%
%
%
%
%
%
%
%
%
%
%
%
%
%
%
%
%
%
%
%
%
%
%
%
%
%
%
%
%
%
%
%
%
%
%
%
%
%
%
%


%
\begin{figure} 
\makebox[\textwidth][c]{ 
%
\includegraphics[scale=1.0]{fig2.pdf} 
}
\caption{
Mean settling velocity versus Galileo number for spherical and non-spherical
suspensions of heavy particles with density ratio ${\cal O}(1)$ in the dilute
regime.
%
The velocity data is normalized with the corresponding mean settling
velocity of an isolated particle in the asymptotic (long-time) limit.
%
The error bars in the experimental data of \cite{huisman:2016} indicate
minimum and maximum values of the repetitions performed by the authors.
%
Present results are included for completeness.
\label{fig:settlingLiterature}} 
\end{figure} 
%
%
%
%
%
%
%
%
%
%
%
%
%
%
%
%
%
%
%
%
%
%
%
%
%
%
%
%
%
%
%
%
%
%
%
%
%
%



%
%
%
%
%
%
%
%
%
%
%
%
%
%
As noted above in the case of spheres, the settling regime of a single
particle appears to be relevant to the macroscopic behavior of dilute
suspensions. 
%
As a next step towards the understanding of collective effects, it is useful to
investigate the interaction of settling particle pairs, a setup which may lead to the 
so-called \gls{dkt} process \citep{fortes:1987}.
%
For sufficiently high Galileo number a trailing sphere 
initially released inside of the wake region of a leading sphere 
will approach the latter one during the drafting phase, they will
touch (`kissing'), and then interchange positions (`tumbling'), 
since a vertically-aligned pair of spheres is unstable, before
eventually separating.  
%
The reduced size of the problem compared to the many-particle cases makes 
\gls{dkt} simulations
%
a useful laboratory 
to understand the underlying physics of many-particle cases.
%
%
%
%
%
%
%
%
Indeed, \cite{fortes:1987} used auxiliary cases with the \gls{dkt} setup to
support their results on the spatial structure of many spherical
particles in a fluidized bed. 
%
Regarding non-spherical particles, \cite{ardekani:2016} observed a suppression
of the tumbling phase in pairwise interactions of moderately flat oblate
spheroids ($\gls{chi}=3$), as well as an increased collision domain.
%
The authors conjectured that the essentially infinite interaction time
of piled up particles would lead to enhanced clustering in the
corresponding many-particle case. 
%
%
This conjecture was later indeed confirmed by \cite{fornari:2018b}.
%
%
%
%
%

%
The numerical simulations of \gls{dkt} cases in the literature use a
common configuration: two particles settling in an initially quiescent fluid
inside a container, whose vertical size should be large enough to accommodate
the different stages of the \gls{dkt} case, and also minimize the (undesired)
influence of the lower boundary of the computational domain.
%
The vertical size of the container varies from $20\gls{p:deq}$
\citep{glowinski:2001,breugem:2012} to approximately $40\gls{p:deq}$
\citep[][]{patankar:2000,ardekani:2016}.
%
The main drawback of this configuration is that any modification of the
parameters (for example the relative initial position between the particles)
which increases the duration of the drafting phase, would require a further
increase of the computational domain.
%
As a result, the simplicity of the \gls{dkt} setup is somehow diminished.
%
Using periodic boundary conditions in the vertical direction, as some authors
have done for single particle cases \citep{kajishima:2002,doychev:2014}, also 
demands large domains in order to minimize wake effects from periodic repetitions.
%
%
%
%
%
%
%
%
%
%
%
%
%
To overcome these difficulties, in the present work we propose to use 
inflow/outflow boundary conditions along the vertical direction
along with a carefully adjusted vertical velocity at the inflow. 
%
Thus, the settling particles are not perturbed by their own wakes, and
the domain size becomes less of a critical parameter. 
%
To the best of our knowledge, this strategy has not been employed
previously.  
%
%
%
%
%
Here it allows us to employ moderately small computational domains, so that 
we can perform a campaign of simulations varying the relative initial position
of the particle pair and, therefore, we can obtain statistically significant results.
%

%
In the present work we analyze the clustering behavior of dilute suspensions of
spheroids with a relatively low aspect ratio $\gls{chi}=1.5$, a density ratio
$\gls{kappa}=1.5$ and two Galileo numbers $Ga=\{111, 152\}$. 
%
For these spheroids a single particle exhibits the so-called ``sphere-like
scenario'' \citep{zhou:2017}, and for suitable initial conditions a pair of
them undergo the three stages in a pairwise interaction (\gls{dkt}).
%
Our aim is to explore collective effects for non-spherical particle
shapes, 
%
%
while remaining relatively close to the well-explored spherical
shape.
The overarching question is: how do relatively small deviations
from the spherical shape affect the settling behavior of a dilute
suspension of rigid particles?
%
In addition to many-particle simulations in large domains, we perform
a large set of \gls{dkt} simulations of the mentioned spheroids as well as for the
reference case with spheres in which we analyze the effect of particle shape and of the 
angular motion by either suppressing or allowing the latter. 
%
%
%
%
%
%
%
%
%
%
%
%


%
%
The manuscript is structured as follows: in \S~\ref{sec:problem} we present 
the mathematical model describing the settling of particles in
unbounded, otherwise quiescent fluid and, in \S~\ref{sec:method} we
specify the numerical method and the physical and numerical
parameters. 
%
Results are presented in two steps: in \S~\ref{sec:results/multi} we focus on
many-particle cases, which represent the core of this  work, and in
\S~\ref{sec:results/dkt} we analyze a set of auxiliary simulations of a 
\gls{dkt} configuration in order to support the observations
made in the many-particle cases.
%
%
%
%
%
In \S~\ref{sec:discussion} we discuss the implications of the
\gls{dkt} results for the many-particle case. 
%
%
%
%
Finally, a summary and conclusions can be found in \S~\ref{sec:conclusions}.
%
