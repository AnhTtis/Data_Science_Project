

\subsection{Numerical method}

The Navier-Stokes equations \eqref{eq:gov} are integrated in time
by means of a three-stage Runge-Kutta scheme in which the viscous term is 
treated implicitly and the advective term explicitly \citep{rai:1991,verzicco:1996}.
%
The fractional step method proposed by \cite{brown:2001} is used to fulfill the
continuity constraint.
%
Spatial derivatives are approximated with central finite-differences of 
second order on a staggered, uniform, Cartesian grid.

The presence of the body is modeled by the direct-forcing immersed boundary
method proposed by \cite{uhlmann:2005} and later extended to track the 
motion of non-spherical particles by \cite{moriche:2021}, where extensive 
validation of the method can be found.
%
Collisions are modelled with a repulsive short-range normal
force (with a range $\gls{dx}$) such as to avoid non-physical overlapping of
particles (details can be found in \S~\ref{sec:collision}).

%
%
%
%
%
%
%
%
%
%
%
%
%
%
In a triply periodic setup, gravity continuously accelerates the system.
%
Therefore, in order to allow for a steady state, we add a constant-in-space
source term to the vertical momentum equation whose volume integral is equal
 (and opposite in sign) to the net force exerted by the particles on the fluid
\citep{hoefler:2000}.

\subsection{Computational setup}
%

\begin{table} 
\caption{Parameters of the present cases and of those in \cite{uhlmann:2014a}
\label{tab:cases}} 
\makebox[\textwidth][c]{ 
%
\begingroup \footnotesize %
\begin {tabular}{l|ccc|ccc|c}%
\toprule Case&\gls {chi}&\gls {Ga}&\gls {kappa}&$[L_x \text {x} L_y \text {x} L_z ]/\gls {p:deq}^3$&$N$&\gls {svf}&Work\\%
\toprule \texttt{G111}&\ensuremath {1.5}&\ensuremath {110.56}&\ensuremath {1.5}&\pgfmathprintnumber [fixed, precision=1]{5.4946289e1} x \pgfmathprintnumber [fixed, precision=1]{5.4946289e1} x \pgfmathprintnumber [fixed, precision=1]{2.1978511e2}&\ensuremath {6{,}336}&\ensuremath {5\cdot 10^{-3}}&Present\\%
\texttt{G152}&\ensuremath {1.5}&\ensuremath {152.02}&\ensuremath {1.5}&\pgfmathprintnumber [fixed, precision=1]{5.4946289e1} x \pgfmathprintnumber [fixed, precision=1]{5.4946289e1} x \pgfmathprintnumber [fixed, precision=1]{2.1978511e2}&\ensuremath {6{,}336}&\ensuremath {5\cdot 10^{-3}}&\\\toprule %
\texttt{M121}&\ensuremath {1}&\ensuremath {121.24}&\ensuremath {1.5}&\pgfmathprintnumber [fixed, precision=1]{6.8e1} x \pgfmathprintnumber [fixed, precision=1]{6.8e1} x \pgfmathprintnumber [fixed, precision=1]{3.4100006e2}&\ensuremath {15{,}190}&\ensuremath {5\cdot 10^{-3}}&Reference \\%
\texttt{M178}&\ensuremath {1}&\ensuremath {178.46}&\ensuremath {1.5}&\pgfmathprintnumber [fixed, precision=1]{8.5e1} x \pgfmathprintnumber [fixed, precision=1]{8.5e1} x \pgfmathprintnumber [fixed, precision=1]{1.7100006e2}&\ensuremath {11{,}867}&\ensuremath {5\cdot 10^{-3}}&(Uhlmann \& Doychev, 2014)\\%
\end {tabular}%
\endgroup %
}
\end{table} 

%
The computational domain is a cuboid with triply-periodic boundary conditions,
whose size is the result of a compromise between computational resource
requirements and physical realism.
%
Based on preliminary tests we choose a domain size of approximately 
$55\gls{p:deq}$ in the lateral directions and approximately $220\gls{p:deq}$ in
the vertical direction.
%
Please note that a full decorrelation in the vertical direction is not warranted
once clustering sets in, and in the horizontal direction once the clusters grow
to a size comparable of the computational domain.
%
This lack of full decorrelation, which was also observed in the case of 
corresponding spheres \citep{doychev:2014}, is further discussed in appendix
\S~\ref{sec:autocorr}. 
%
We select a spatial resolution of $\gls{p:deq}/\gls{dx}\approx 21$ ($\gls{dd}/%
\gls{dx}=24$), which is supported by the work of \cite{moriche:2021}.
%
In their work the authors show that compared to a spectral/spectral-element
solution the error in the mean settling velocity of a single spheroid of aspect 
ratio $\gls{chi}=1.5$, density ratio $\gls{kappa}=2.14$ at $\gls{Ga}=152.02$ is
smaller than $2\%$.
%
In the present case this results in a grid of $[1152 \times 1152 \times 4608]$
points.
%
Table \ref{tab:cases} shows the parameters of the two simulations presented
in this work.


%
\subsection{Initialization}
\label{sec:ini}
%
\revision{To}{In order to} 
initialize the flow around the particles we
\revision{run}{simulate} 
an initial transient
during a time interval of
$66.67\gls{p:deq}/\gls{vu_z_mf}$ in which the particles are fixed.
%
The initial position of the particles follows a random uniform distribution.
%
Their orientation is randomly distributed with a maximum deviation of $\pm5^\circ$
tilting angle with respect to the vertical axis.
%
The objective of constraining the angular position is to obtain a slight 
perturbation of the angular position with respect to the stable position of 
settling spheroids at moderate \gls{Ga}.
%
The same initial distribution of particles is used in both cases presented in
table \ref{tab:cases}.
%


During the fixed-particle transient we impose the Reynolds number of the flow
relative to the particles, \gls{p:Redeq;0}, as obtained (as an output parameter)
in the simulations with an isolated mobile particle at the target value of the
respective Galileo number, cf. table \ref{tab:cases}.
%
\revision{}{This is realized by means of a constant vertical pressure
  gradient, similar to the body force which counteracts the
  acceleration of the system when particles are freely mobile
  \citep{uhlmann:2014a}.} 
%
As it will be shown later, the flow around the particles during the initial 
fixed-particle transient mostly resembles that of the analogous single-particle 
case at the given \gls{p:Redeq;0}, due to the low concentration of
particles in both cases.
%
It should be mentioned that during this initial transient we obtain good 
decorrelation of all flow velocity components in all spatial directions (see
\S~\ref{sec:autocorr}).
%
\revision{}{Once the particles are released, the mean fluid flow relative
  to the particles is maintained through the above mentioned body
  force. Therefore, if perfectly adjusted, a single particle would not
  exhibit any vertical motion in the
  computational reference system, while the vertical motion of the
  many-particle ensemble is entirely due to mutual interactions
  \citep{uhlmann:2014a}.}
%
