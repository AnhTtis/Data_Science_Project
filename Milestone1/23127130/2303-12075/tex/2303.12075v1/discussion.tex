
\begin{figure} 
\begin{center}
%
\includegraphics[scale=0.9]{fig19.pdf} 
\end{center}
\caption{Summary sketch of clustering mechanisms analyzed in this work (intense
\gls{dkt} interactions) and from the reference work of \cite{uhlmann:2014a} 
(promoted particle encounters by horizontal motion).
\label{fig:clustering_scheme}} 
\end{figure} 

%
On the one hand our present many-particle simulations show that mildly
oblate spheroids (with an aspect ratio of $1.5$) form strong clusters
at a comparably low value of the Galileo number, for which an isolated
particle in ambient fluid settles in the steady vertical regime. 
%
This observation is in contrast to the known behavior of ensembles of
spheres, for which the Galileo number has been identified as a
critical parameter with respect to the onset of wake-induced
clustering \citep{uhlmann:2014a}. This clustering transition in the case
of spheres occurs in the range $\gls{Ga}=121\ldots178$ which encloses the
value $\gls{Ga}=155.8$ at which an isolated particle's path regime
bifurcates from steady vertical to steady oblique \citep{zhou:2015}.
The explanation proposed by \cite{uhlmann:2014a} links the onset of
clustering in the case of spheres to the enhanced horizontal mobility
of individual particles above the oblique-path threshold, which causes
an increase of the frequency of particle-particle encounters.
In order to make this point clearer, let us consider the fact that in
the wake of a given particle a ``region of attraction'' can be
defined, i.e.\ a spatial region of limited extent within which a 
trailing particle near equilibrium will experience a modified
hydrodynamic force which has the effect that -- without further
perturbations -- it will continuously approach the leading particle
until contact.
%
Now let us consider the hypothetical case of a set of mono-disperse
particles which are initially all located outside of any other
particles' region of attraction. If these particles are settling in
the steady vertical regime in an unbounded ambient fluid, they simply
move as a group without changing their relative positions, and they 
will, therefore, never get into contact. 
If, however, each individual particle of this ensemble settles in a
steady oblique regime (with a random azimuthal angle), there exists a
finite probability for individual particles to encounter another
particle's region of attraction, and to get into direct contact. If
the conditions are such that an interacting particle pair (on average) 
encounters one (or more) additional particle(s) before separating
again, initial seeds will eventually lead to large-scale cluster
formation through accretion. 
Hence, we can conclude that horizontal particle mobility can have the
effect of enhancing the tendency to cluster in this scenario. 
%
%
%
%
However, as we have observed in the present many-particle simulations,
oblate spheroids with aspect ratio $\gls{chi}=1.5$ apparently do not
require this mechanism in order to form columnar clusters.
This is in accordance with the results of 
\cite{fornari:2018b} for oblate spheroids at larger aspect ratio
($\gls{chi}=3$), lower Galileo number ($\gls{Ga}=60$) and smaller density ratio
($\gls{kappa}=1.02$) which likewise appear to form clusters. 

On the other hand, we have seen in the present series of
drafting-kissing-tumbling simulations that pairs of mildly
oblate spheroids, which are initially positioned such that the
trailing particle is located inside of the leading particle's region
of attraction, interact for a significantly longer time than
spherical counterparts at corresponding parameter points.
We have observed that the difference in interaction time is caused by
two factors: first, spheroids approach each other at a somewhat faster
rate than spheres during the drafting phase; second, spheroids
remain in close proximity over a significantly longer duration after
the first contact.
%
This is qualitatively in line with the results of
\cite{ardekani:2016} in their DKT simulation of spheroids with larger
aspect ratio ($\gls{chi}=3$), smaller Galileo number ($\gls{Ga}=80$) and smaller
density ratio ($\gls{kappa}=1.14$). These authors observe that those
flatter objects do not tumble, but instead remain locked in position
as a stack after the initial contact. 
%
Translating our result on the isolated pairwise interaction to the
(dilute) many-particle configuration implies that the observed
increase in the interaction time between pairs of particles can also
enhance the tendency to form large-scale clusters. This is due to the
fact that once a pair of particles gets into contact, it will have a
higher probability of attracting an additional particle before
separating, if the original pair has an extended contact duration. 
Hence, this argument suggests that an increase in the interaction time
between pairs of particles can lead to large-scale clustering, which is a 
mechanism that is different from (and complementary to) the above
process by enhancement of lateral mobility. 

%
Figure~19 shows a sketch which is intended to contrast the above two
routes of cluster formation in the dilute regime for
sphere-like particles. 
%
Starting from the (non-clustering) baseline case with spherical
particles at a Galileo number of ${\cal O}(100)$ in the steady
vertical regime (lower left corner of the $(\gls{Ga},\chi)$-plane in
figure~19), clustering can be enabled by either one of the following
two options:
(i) by increasing the probability of particles entering their
peers' attractive wake region (which is achieved by increasing the
horizontal particle mobility, i.e.\ through increasing the spheres'
Galileo number \gls{Ga}); 
(ii) by increasing the temporal interval over which particles remain
close to each other during wake-induced encounters (which can be
achieved by replacing spheres with oblate spheroids, i.e.\ by way of
increasing the particles' aspect ratio \gls{chi}).
%
Please note that additional triggers of cluster formation (besides
lateral particle mobility and close interaction duration) might be
at play, such as the growth of the spatial extent of the wake with
increasing Galileo number.
%
In addition, it remains to be understood how the density ratio influences
the mechanisms mentioned above.
%
Further research would be necessary to properly answer these questions.


%
%
%
%
%
