We have performed particle-resolved direct numerical simulations
(PR-DNS) of many heavy non-spherical particles settling under gravity.
%
The particles are oblate spheroids of aspect ratio $1.5$ (which represent a
modest deviation from a spherical shape) and density ratio
$\gls{kappa}=1.5$; the global solid volume fraction measures $0.005$
such that the suspension can be considered as dilute. 
%
Two Galileo numbers are considered, namely $111$ and $152$ for which a
single oblate spheroid
%
follows a steady vertical and a steady oblique path, respectively.
%
In contrast to previous results for spheres \citep{uhlmann:2014a} we have found
that the qualitative difference in the single particle regime does not result in
a qualitatively different behavior of the multiparticle cases:
at both Galileo number values a strongly inhomogeneous spatial
distribution of the disperse phase in the form of columnar clusters is
observed, with a significantly enhanced average settling velocity as a
consequence.
%
A similar result has previously been reported for PR-DNS of
significantly flatter spheroids (with $\gls{chi}=3$) and for lower
density ratio ($\gls{kappa}=1.02$) by \cite{fornari:2018b}.  
%
Here we have used Voronoi tessellation as a basis for the
analysis of the structure of the particulate phase. The intensity of
clustering has been measured with the aid of the standard deviation of
the Voronoi cells' volume, normalized with the value obtained from a
random Poisson process \citep{monchaux:2010b}. It turns out that the
enhancement of the average settling speed is approximately
proportional to the standard deviation of the Voronoi cell volumes
when considering both the present spheroids as well as the spheres of
\cite{uhlmann:2014a} as a joint data-set. This result, however,
requires further confirmation through additional data points before
its potential implications can be evaluated. 
%
Note that the amount of enhancement of the settling velocity may still
depend on the domain size, since the size of particle clusters approaches
the former.

%
Motivated by the 
lack of influence of
%
the single-particle regime
%
upon
the statistical features of the multi-particle settling 
%
we have carried out a thorough analysis of
pairwise interactions of particles in the well-known drafting-kissing-tumbling
setup,
conducted in a computational domain with inflow/outflow boundary
conditions in the vertical direction. 
%
We have considered four configurations, namely oblate spheroids of aspect ratio
$1.5$ and spheres, with and without suppression of the angular motion, with density
ratio $\gls{kappa}=1.5$ and a Galileo number such that a single particle would 
follow a steady vertical path ($\gls{Ga}=111$).
.
%
Through systematic variation of the particle pair's relative initial position 
we have found that the region of attraction for both particle shapes, with and 
without rotation, is very similar. However, in the case of
free-to-rotate spheres the trailing particle's trajectory is
horizontally shifted towards larger radial distances, resulting 
in a prolonged drafting phase.
%
Regarding the particles' tumbling phase, we have shown that spheres
and spheroids behave in a qualitatively different manner.  
Spheres undergo at most a single collision, and they quickly separate
afterwards. 
%
Rotationally locked spheroids also experience a maximum of one
collision, but they remain close to each other for relatively long times.
%
Finally, free-to-rotate spheroids exhibit two collision events in most
of the cases in which a \gls{dkt} event is observed, and,
consequently, their average interaction time is the maximum out of the
four investigated configurations. 
%
%

To summarise, we observe a shape-induced
increase in the interaction time when two particles happen to ``meet''
(i.e.\ when one of them enters the other particle's wake region), such
that the probability of additional particles joining the initial pair
is increased with respect to the baseline case of spheres. 
%
Hence, the tendency to form a large-scale cluster increases. 
%
This is in contrast
to the mechanism for spheres \citep[as proposed by][]{uhlmann:2014a}, 
where the clustering transition is believed to be triggered by the
primary bifurcation of the isolated particle's wake flow (from
axisymmetric vertical to planar oblique) that leads to lateral
mobility, hence increasing the probability of mutual particle
encounters. 
%
As a consequence of these two observations, we conclude  
that the mechanism for the initiation of columnar clusters in the case of a dilute
suspension of modestly oblate spheroids is in a sense orthogonal to
the mechanism that is believed to be at work in the counterpart with
spherical particles.


%
%
%
%
%
%
%
%
%
%
%
%
%
%
%
%
%
%
%
%
%

%
Since the qualitatively different behavior of spheroids (as compared
to spheres) has now been established both for moderately flat
geometries \citep[$\gls{chi}=3$, ][]{fornari:2018b} and for modestly
flat ones ($\gls{chi}=1.5$, present work), the question of a possibly 
finite critical aspect ratio poses itself naturally. Future work
should be aimed in this direction.
%
Furthermore, a thorough analysis of collective effects in prolate
spheroids in a comparable parameter range is still lacking.
%
%
%
%
%
%
%
%
%
%
