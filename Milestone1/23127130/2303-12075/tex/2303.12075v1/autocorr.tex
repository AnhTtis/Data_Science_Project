
%
Figures \ref{fig:autocorr_G111} and \ref{fig:autocorr_G152} show the two point
autocorrelation functions 
of the \revision{vertical}{horizontal} fluid velocity component,
\revision{\gls{f:Rww}}{\gls{f:Ruu}}, (panels \revision{a and b}{a-c}) and of the
\revision{horizontal}{vertical} counterpart, \revision{\gls{f:Ruu}}{\gls{f:Rww}},
(panels \revision{d, e and f}{d-e}) for cases \verb!G111! and \verb!G152!, respectively, together with 
the time history of \gls{f:Rww} at the furthest vertical and horizontal 
position (panel \revision{c}{f}).
%
When particles are fixed ($t<0$) all signals are fully decorrelated within the
domain.
%
After particles are released the horizontal velocity remains fully decorrelated 
until the end of the simulated time (see figures \ref{fig:autocorr_G111}\revision{d-f}{a-c}
and \ref{fig:autocorr_G152}\revision{d-f}{a-c}), whereas the vertical velocity acquires
correlations which do not decay to zero at later times (figures 
\ref{fig:autocorr_G111}\revision{a-b}{d-e} and 
\ref{fig:autocorr_G152}\revision{a-b}{d-e}) as will be discussed in the following.
%

%
%
%
%
Along the vertical direction, \gls{f:Rww} presents an initial fast growth
($0\lesssim t/\gls{tg} \lesssim 300$).
%
For later times, the curves $\gls{f:Rww}(r_z)$ are similar, except for a small
oscillating behavior in time.
%
They show a monotonic decreasing behavior in space with positive values over the
entire domain ($0<r_z/D<110$).
%
This is the footprint of the fast formation of robust columnar structures that 
occupy the whole domain in the vertical direction.
%
The temporal evolution of \gls{f:Rww} at its furthest distance 
($r_{z,\text{max}}=110$) supports the fast growth and the small oscillation with
time commented above.
%

The behavior of \gls{f:Rww} along the horizontal direction is somehow more
complex than along the vertical direction.
%
First, it presents an almost decorrelated solution until $300\gls{tg}$ in case
\verb!G111! and $900\gls{tg}$ in case \verb!G152!.
%
In both cases there is a turnover point in which the value of \gls{f:Rww} 
and for case \verb!G111! even two when $300\lesssim t/\gls{tg}\lesssim 600$.
%
This can be explained by the growth of the clusters in the horizontal direction
being a slow process compared to their growth in the vertical direction.
%
Furthermore, once the clusters grow to a size comparable to the computational
domain and because of continuity, there are negative values at 
$r_{x,\text{max}}=55$.
%
The first crossing of \gls{f:Rww} with the zero value gives an estimate of the 
size of the clusters in the horizontal direction.
%
This measures approximately $15-20\gls{p:deq}$, which implies that the clusters
ultimately grow to a size for which the current computational domains are not sufficient.

\begin{figure} 
\makebox[\textwidth][c]{ 
%
\includegraphics[scale=1.0]{fig21.pdf} 
}
\caption{Autocorrelation functions of the
a-c) horizontal fluid velocity component , \gls{f:Ruu}, and of the
d-e) vertical counterpart, \gls{f:Rww},
for case {\tt G111}.
%
f) Time history of \gls{f:Rww} at the furthest vertical and horizontal position.
%
The time interval to compute each curve in panels a-e is shown in the legend,
and the corresponding linestyles are used piecewise in panel f.
\label{fig:autocorr_G111}} 
\end{figure} 

\begin{figure} 
\makebox[\textwidth][c]{ 
%
\includegraphics[scale=1.0]{fig22.pdf} 
}
\caption{Same as \ref{fig:autocorr_G111} but for case {\tt G152}.
\label{fig:autocorr_G152}} 
\end{figure} 
