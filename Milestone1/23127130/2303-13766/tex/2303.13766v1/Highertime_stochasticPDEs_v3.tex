%%%%
\documentclass[leqno,final]{siamltex}
%\documentclass{amsart}
%\documentclass[leqno,final]{siamltex}
%\setlength{\textwidth}{14.5cm}
%\setlength{\textheight}{21.5cm}
\setlength{\hoffset}{.7in}
%\setlength{\voffset}{-.7in}
\pagestyle{myheadings}
%\usepackage{showkeys}
%\usepackage{showlabels}

\usepackage{subfig}
\usepackage{graphicx} 
\usepackage{amsmath,amstext,amssymb,bm}
\usepackage{leftidx}

\usepackage{xcolor} 
\usepackage{soul} 
\usepackage{tikz}
\usetikzlibrary{shapes,arrows}
\usepackage{mathrsfs}
%\usepackage[dvips]{graphicx}
\usepackage{epstopdf}
\usepackage{color}
\usepackage{multirow}
\usepackage{tabularx}
\usepackage[shortlabels]{enumitem}
%\usepackage[config, labelfont={bf}]{caption,subfig}
%\DeclareGraphicsExtensions{.jpg,.pdf,.png,.jpeg}
%\setlength{\voffset}{-.2in}


\numberwithin{equation}{section}
%\newtheorem{definition}{Definition}[section]
%\newtheorem{proposition}{Proposition}[section]
\newtheorem{remark}{Remark}[section]
%\newtheorem{theorem}{Theorem}[section]
%\newtheorem{corollary}{Corollary}[section]
%\newtheorem{lemma}{{Lemma}}[section]
\newtheorem{conjecture}{Conjecture}[section]
\allowdisplaybreaks[4]

%\newtheorem{theorem}{Theorem}
%\newtheorem{lemma}{Lemma}
%\newtheorem{remark}{Remark}
%\newtheorem{proposition}{Proposition}
%\newtheorem{reference}{Reference}

\newcommand{\cP}{\mathcal{P}}
\newcommand{\cQ}{\mathcal{Q}}
\newcommand{\hP}{\widehat{P}}
\newcommand{\FE}{\mbox{\tiny FE}}
\newcommand{\DG}{\mbox{\tiny DG}}
\newcommand{\cE}{\pmb{\mathcal{{E}}}}
\newcommand{\E}{\mathcal{E}}
\newcommand{\cT}{\mathcal{T}}
\renewcommand{\div}{\mbox{\rm div\,}}
\newcommand{\curl}{\mbox{\rm curl\,}}
\newcommand{\tr}{\mbox{\rm tr}}
\newcommand{\mce}{\mathcal{E}_h}
\newcommand{\mct}{\mathcal{T}_h}
\newcommand{\bl}{\big\langle}
\newcommand{\br}{\big\rangle}
\newcommand{\Bl}{\Big\langle}
\newcommand{\Br}{\Big\rangle}

\newcommand{\cK}{\mathcal{K}}
\newcommand{\cL}{\mathcal{L}}
\newcommand{\cF}{\mathcal{F}}
\newcommand{\cV}{\mathcal{V}}
\newcommand{\cW}{\mathcal{W}}
\newcommand{\cG}{\mathcal{G}}
\newcommand{\mP}{\mathbb{P}}
\newcommand{\mH}{\mathbb{H}}
\newcommand{\mE}{\mathbb{E}}
\newcommand{\mV}{\mathbb{V}}
\newcommand{\mW}{\mathbb{W}}
\newcommand{\R}{\mathbf{R}}
\newcommand{\eps}{\epsilon}
\newcommand{\veps}{\varepsilon}
\newcommand{\Ome}{\Omega}
\newcommand{\p}{\partial}
\newcommand{\nab}{\nabla}
\newcommand{\vu}{{\bf u}}
\newcommand{\vB}{{\bf B}}
\newcommand{\vW}{{\bf W}}
\newcommand{\vP}{{\bf P}}
\newcommand{\vf}{{\bf f}}
\newcommand{\vH}{{\bf H}}
\newcommand{\bH}{\mathbb{H}}
\newcommand{\bV}{\mathbb{V}}
\newcommand{\vU}{{\bf U}}
\newcommand{\U}{\mathbb{U}}
\newcommand{\vV}{{\bf V}}
\newcommand{\vX}{{\bf X}}
\newcommand{\vL}{{\bf L}}
\newcommand{\vv}{{\bf v}}
\newcommand{\ve}{{\bf e}}
\newcommand{\vG}{{\bf G}}
\newcommand{\vE}{{\bf E}}
\newcommand{\vQ}{\mathbf{Q}}
\newcommand{\e}{\pmb{\varepsilon}}
\newcommand{\G}{\pmb{\mathcal{G}}}
\newcommand{\pphi}{\pmb{\phi}}
\newcommand{\ppsi}{\pmb{\psi}}
\newcommand{\vvarphi}{\pmb{\varphi}}
\newcommand{\eeta}{\pmb{\eta}}
\newcommand{\vA}{{\bf A}}

\def\esssupT{\underset{t\in [0,T]}{\mbox{\rm ess sup }}}
\def\esssupI{\underset{t\in [0,\infty)}{\mbox{\rm ess sup }}}


\def\D{\mathcal{D}}
\def\E{\mathbb{E}}
\def\P{\mathbb{P}}
%\def\O{\Omega}
%\def\F{\mathcal{F}}
%\def\R{\mathbb{R}}
%\def\({\left(}
%\def\){\right)}
%\def\div{\mbox{div}}
%\def\<{(}
%\def\>{)}
%\def\V{\tilde{V}}
%\def\u{ u}
%\def\w{ w}


%%%%%%%
\begin{document}
	
	\title{Higher order time discretization method for a class of semilinear stochastic partial differential equations with multiplicative noise}
	\markboth{YUKUN LI AND LIET VO AND GUANQIAN WANG}{THE STOCHASTIC PDEs}		
	
	\author{Yukun Li\thanks{Department of Mathematics, University of Central Florida, Orlando, FL, 32816, U.S.A. (yukun.li@ucf.edu). This author was partially supported by the NSF grant DMS-2110728.}
		\and Liet Vo\thanks{Department of Mathematics, Statistics and Computer Science, The University of Illinois at Chicago, Chicago, IL 60607, U.S.A. (lietvo@uic.edu).}
		\and Guanqian Wang\thanks{Department of Mathematics, University of Central Florida, Orlando, FL, 32816, U.S.A. (guanqian.wang@ucf.edu). This author was partially supported by the NSF grant DMS-2110728.}}
	
	
	
	
	
	\maketitle
	
	\begin{abstract} In this paper, we consider a new approach for semi-discretization in time and spatial discretization of a class of semi-linear stochastic partial differential equations (SPDEs) with multiplicative noise. The drift term of the SPDEs is only assumed to satisfy a one-sided Lipschitz condition and the diffusion term is assumed to be globally Lipschitz continuous. Our new strategy for time discretization is based on the Milstein method from stochastic differential equations. We use the energy method for its error analysis and show a strong convergence order of nearly $1$ for the approximate solution. The proof is based on new H\"older continuity estimates of the SPDE solution and the nonlinear term. For the general polynomial-type drift term, there are difficulties in deriving even the stability of the numerical solutions. We propose an interpolation-based finite element method for spatial discretization to overcome the difficulties. Then we obtain $H^1$ stability, higher moment $H^1$ stability, $L^2$ stability, and higher moment $L^2$ stability results using numerical and stochastic techniques. The nearly optimal convergence orders in time and space are hence obtained by coupling all previous results. Numerical experiments are presented to implement the proposed numerical scheme and to validate the theoretical results.
	\end{abstract}
	
	%There are several difficulties which need to be overcome for this generalization.  First, obviously the spatial  discretization, which does not appear in the SODE case, adds an extra layer of difficulty. It turns out a spatial discretization must be designed to guarantee certain properties for the numerical scheme and its stiffness matrix. In this paper we use a finite element interpolation technique to discretize the nonlinear drift term. Second, in order to prove the strong convergence of the proposed fully discrete finite element method, stability estimates for higher moments of the $H^1$-seminorm of the numerical solution must be established, which are difficult and delicate. A judicious combination of the properties of the drift and diffusion terms and a nontrivial technique borrowed from \cite{majee2018optimal} is used in this paper to achieve the goal. Finally, stability estimates for the second and higher moments of the $L^2$-norm of the numerical solution are also difficult to obtain due to the fact that the mass matrix may not be diagonally dominant. This is done by utilizing the interpolation theory and the higher moment estimates for the $H^1$-seminorm of the numerical solution.
	%%
	%After overcoming these difficulties, it is proved that the proposed fully discrete finite element method
	%is convergent in strong norms with nearly optimal rates of convergence.  Numerical experiment results are also presented to validate the theoretical results and to demonstrate the efficiency of the
	%proposed numerical method.
	
	%\keywords{
		\begin{keywords}
			Stochastic partial differential equations, multiplicative noise, Wiener process, It\^o stochastic integral, 
			Milstein scheme, finite element method, error estimates.
		\end{keywords}
		%}
	
	%\subjclass[2010]{Primary
		\begin{AMS}
			65N12, %Stability and convergence of numerical methods
			65N15, %Error bounds
			65N30 %Finite elements, Rayleigh-Ritz and Galerkin methods, finite methods
		\end{AMS}
		%}
	
	%\maketitle
	
	%\tableofcontents
	
	\section{Introduction}\label{sec-1}
	We consider the following initial-boundary value problem for general semi-linear stochastic partial differential equations (SPDEs) with function-type multiplicative noise:
	%\begin{subequations}\label{eq1.1}
	\begin{alignat}{2}
		du &=\bigl[\Delta u + F(u) \bigr]\,dt + G(u) \, dW(t)  &&\qquad\mbox{a.s. in}\,(0,T)\times D,\label{eq1.1}\\
		u &=  0 && \qquad\mbox{a.s. on } (0,T)\times\p D, \label{eq1.2}\\
		u(0)&= u_0 &&\qquad\mbox{a.s. in}\, D,\label{eq1.3}
	\end{alignat}
	%\end{subequations}
	where $D = (0, L)^d\subset \mathbb{R}^d \, (d=1,2,3)$.
	$F, G$ are two given functions that will be specified later. $\{W(t); t\geq 0\}$ denotes an ${{\mathbb{R}}}$-valued Wiener process.
	
	The corresponding stochastic ordinary differential equations of \eqref{eq1.1} (without the Laplacian term) are studied in \cite{kloeden1991numerical,mao2007stochastic} for the case when both $F$ and $G$ are Lipschitz continuous, and in \cite{higham2002strong} for the case when $G$ satisfies the one-sided Lipschitz condition as stated in \eqref{oneside_Lip}. The strong and weak divergence is considered in \cite{hutzenthaler2010strong} for some $F$ which are not Lipschitz continuous. Besides, the corresponding stochastic partial differential equations of \eqref{eq1.1} when $F$ is Lipschitz and non-Lipschitz continuous and when $G$ is additive and multiplicative are studied in \cite{feng2014finite,feng2017finite,feng2021strong,prohl2014strong,majee2018optimal} based on the variational approach and in \cite{brehier2018analysis,gyongy2005discretization,gyongy2016convergence,jentzen2015strong,kovacs2015backward,kovacs2018discretisation,liu2019strong} based on the semigroup approach. Here the half-order convergence is established in \cite{majee2018optimal} when the drift term is $F(u)=u-u^3$ using the Euler-type scheme. The half-order convergence is established in \cite{feng2021strong} for the drift term in \eqref{eq20180812_1} and diffusion term in assumptions {\bf(A1)}--{\bf(A3)} for a fully discrete scheme.
	
	The primary goal of this paper is to design and analyze a first-order numerical scheme for the time discretization of the problem \eqref{eq1.1}--\eqref{eq1.3}. Specifically, we design a new time discretization method first and then propose an interpolation finite element method, which is based on the new time scheme to discretize the space. Our idea for the time discretization method is inspired by the Milstein method \cite{mil1975approximate} from stochastic differential equations and the semi-discrete in time strategy of the stochastic Stokes equations in \cite{vo2022higher}. In addition, the diffusion function $G$ is assumed to satisfy the global Lipschitz condition while the drift-nonlinear function $F$ is only one-sided Lipschitz. Furthermore, to establish the rates of convergence of the proposed scheme, we use the energy method followed by two steps: the first step is to prove the first-order error order in time by utilizing several established H\"older continuity estimates. The second step is to prove the optimal error order in space. To achieve this, the $H^1$ stability of the numerical solution is needed. The $H^1$-seminorm stability of the numerical solution is proved first and based on which the $L^2$ stability of the numerical solution is established.
	
	The remainder of this paper is organized as follows. In Section \ref{sec2}, several H\"older continuity results about the strong solution are proved. These results will be used in establishing the semi-discrete in-time error estimates. In Section \ref{section_semi}, we present the new approach for the time discretization and its a priori stability as well as the error estimates of the semi-discrete solution are proved. The convergence order is proved to be nearly $1$ for the proposed scheme in $L^2$-norm and the energy norm. In Section \ref{fully_discrete}, we consider an interpolation finite element method for spatial discretization. The finite element method is designed where the interpolation operator is utilized to overcome the difficulty resulting from nonlinearity. Through this approach, the second moment and higher moment $H^1$ stability results are proved first, based on which the second moment and higher moment $L^2$ stability results are proved. Finally, the error estimates with optimal convergence order in space are established based on those stability results. In Section \ref{nume}, several numerical tests including different initial conditions, drift terms, and diffusion terms are used to validate the theoretical results.
	
	\section{Preliminaries}\label{sec2}
	Let $\mathcal{T}_h$
	be the triangulation of $\D$ satisfying the following assumption \cite{xu1999monotone}:
	\begin{equation}\label{eq20180907}
		\frac{1}{d(d-1)}\sum_{K\supset E}|\kappa_E^K|\cot\theta_E^K\ge0,
	\end{equation}
	where $E$ denotes the edge of simplex $K$. %It was proved in \cite{xu1999monotone} that the stiffness matrix for the Poisson equation with zero Dirichlet boundary is an $M$-matrix if and only if this assumption holds for all edges. The stiffness matrix is diagonally dominant if the Neumann boundary condition is considered. 
	Note this assumption is just the Delaunay triangulation when $d=2$. In 3D, the notations in the assumption \eqref{eq20180907} are as follows: $a_i\ (1\leq i \leq d+1)$ denote the
	vertices of $K$, $E=E_{ij}$ the edge connecting two vertices $a_i$ and
	$a_j$, $F_i$ the $(d-1)$-dimensional simplex opposite to the vertex
	$a_i$, $\theta_{ij}^K$ or $\theta_E^K$ the angle between the faces
	$F_i$ and $F_j$, and $\kappa_E^K=F_i \cap F_j$.
	
	Let ${\mathcal{H}}$, ${\cK}$ be two Hilbert spaces. Then, $\mathcal{L}({\mathcal{H}},{\cK})$ is the space of linear maps from ${\mathcal{H}}$ to ${\cK}$. For $m \in \mathbb{N}$, inductively define
	\begin{align}
		\mathcal{L}_m({\mathcal{H}}, {\cK}) := \mathcal{L}({\mathcal{H}}, \mathcal{L}_{m-1}({\mathcal{H}},{\cK})),
	\end{align} 
	as the space of all multi-linear maps from ${\mathcal{H}}\times\cdots\times {\mathcal{H}}$ ($m$ times) to ${\cK}$ for $m \geq 2$.
	
	For some function $G: {\mathcal{H}} \rightarrow {\cK}$, we define the Gateaux derivative of $G$ with respect to $u \in {\mathcal{H}}$, $DG(u) \in \mathcal{L}({\mathcal{H}},{\cK})$, whose action is seen as
	\begin{align*}
		v \mapsto DG(u)(v)\qquad\forall v \in {\mathcal{H}}.
	\end{align*}
	
	In general, we denote $D^k G(u)\in \mathcal{L}_m({\mathcal{H}},{\cK})$, as the $k$-Gateaux derivative of $G$ with respect to $u\in {\mathcal{H}}$.
	
	Below, we state the assumptions on the functionals $G, F: {\mathcal{H}} \rightarrow {\cK}$.
	\medskip
	\begin{enumerate}
		\item[{\bf(A1)}]  $G$ is globally
		Lipschitz continuous and has linear growth. Namely, 
		there exists a constant $C > 0$ such that for all ${v}, {w} \in {\mathcal{H}}$ 
		\begin{subequations}\label{G}
			\begin{align}\label{Lip}
				\|G({ v})- G({ w})\|_{{\cK}} &\leq C\|{v}-{w}\|_{{\mathcal{H}}}\, , \\
				\|G({v})\|_{{\cK}}  &\leq C \bigl( \|{ v}\|_{{\mathcal{H}}}+1\bigr)\, .   \label{lineargrow}
				%\|\mathcal{D} \mathbf{B}\|_* &\leq C,\label{eq2.6c}
			\end{align}
			%but we emphasize that the results of this paper still hold without this assumption.}
	\end{subequations}
	\item[{\bf (A2)}] There exists a constant $C>0$ such that
	\begin{align}
		\|DG\|_{L^{\infty}({\mathcal{H}};\mathcal{L}({\mathcal{H}},{\cK}) )} +  \|D^2G\|_{L^{\infty}({\mathcal{H}};\mathcal{L}_2({\mathcal{H}},{\cK}) )}  \leq C.
	\end{align} 
	%\item[{\bf (A3)}] There exists a constant $C>0$ such that
	%\begin{align}
	%	\|DF\|_{L^{\infty}({\mathcal{H}};\mathcal{L}({\mathcal{H}},{\cK}) )} +  \|D^2F\|_{L^{\infty}({\mathcal{H}};\mathcal{L}_2({\mathcal{H}},{\cK}) )}  \leq C.
	%\end{align} 
	\item[{\bf (A3)}] There exists a constant $C>0$ such that for all $u,v\in L^2(D)$
	\begin{align}
		\|(DG(u) - DG(v)){G}(v)\|_{L^2} \leq C\|u - v\|_{L^2}.
	\end{align}
\end{enumerate}

In this paper, suppose that $G: H^1_0(D)\rightarrow H^1_0(D)$, and
\begin{align}\label{eq20180812_1}
	F(u)=c_0u-c_1u^3-c_2u^5-c_3u^7-\cdots,
\end{align}
where $c_i\ge 0, i=0,1,2,\cdots$. For simplicity, we choose $F(u) = u - u^q$ for all odd numbers $q \geq 3$. Then $F$ satisfies the following one-sided Lipschitz condition \cite{Higham}
\begin{align}\label{oneside_Lip}
	\langle a - b, F(a) - F(b)\rangle \leq \mu |a - b|^2\qquad\forall a, b \in \mathbb{R}^d,
\end{align}
where $\mu$ is a positive constant. 

It is clear that from the condition \eqref{oneside_Lip} and the fact that $F(0) = 0$, we can infer that
\begin{align}
	\langle F(a), a\rangle  \leq C|a|^2,
\end{align}
where $C = \mu$.

Under the above assumptions for the drift term and the diffusion term, it can be proved
in \cite{gess2012strong} that there exists a unique strong variational solution u such that
\begin{align}\label{weak_form}
	\bigl(u(t), \phi\bigr) &= \bigl(u(0),\phi\bigr) - \int_0^t \bigl(\nab u(s), \nab \phi\bigr)\, ds \\\nonumber
	\qquad\qquad&+ \int_0^t \bigl( F(u(s)), \phi\bigr)\, ds + \int_0^t \bigl(G(u(s)), \phi\bigr)\, dW(s) \quad\forall \phi  \in H^1(D)
\end{align}
holds $\mathbb{P}$-almost surely. Moreover, when the initial condition $u_0$ is sufficiently smooth, the following stability estimate for the strong solution $u$ holds
\begin{align}\label{pde_estimate}
	\sup_{t \in [0,T]} \mE\bigl[\|u(t)\|^{2q}_{H^2}\bigr] + \sup_{t \in [0,T]}\mE\bigl[\|u(t)\|^{4q - 2}_{L^{4q -2}}\bigr] \leq C,
\end{align}
where $q$ is the exponent in the drift term of $F(u) = u  - u^q$.

\bigskip

Next, we introduce the H\"older continuity estimates for the variational solution $u$. 

\begin{lemma}\label{lemma2.4}
	Suppose that the solution $u$ of \eqref{weak_form} satisfies \eqref{pde_estimate}. For $\eps >0$, let $\theta_1 = \frac12 - \epsilon >0$, $\theta_2 = 1-\epsilon>0$. There exists a constant $C \equiv C(D_T, q, u_0)>0$, such that for all $s,t \in [0,T]$,
	\smallskip
	\begin{enumerate}
		\item[(i)] 	 $\displaystyle{\mathbb E}\bigl[\| {u}(t)-{ u}(s)\|^{2}_{H^1} \bigr] 
		\leq C|t-s|^{2\theta_1}.$\\
		\item[(ii)] $\displaystyle{\mathbb E}\Bigl[\Bigl\| {u}(t)-{u}(s) - \int_s^t  G(u(\xi))\, d W(\xi)\Bigr\|^{q}_{H^1} \Bigr] 
		\leq C|t-s|^{q\theta_2},$ where $q = 2, 4.$
		\item [(iii)]$\displaystyle{\mathbb E}\bigl[\| {u}(t)-{ u}(s)\|^{q}_{L^q} \bigr] 
		\leq C|t-s|^{q\theta_1},$
		where $q \geq 2$ are integers.
		\item [(iv)] $\displaystyle{\mathbb E}\Bigl[\Bigl\| F({u}(t))-F({u}(s)) - \int_s^t  DF(u(s))G(u(\xi))\, d W(\xi)\Bigr\|^{2}_{L^2} \Bigr] 
		\leq C|t-s|^{2\theta_2}.$
	\end{enumerate}
\end{lemma}
\medskip
\begin{proof}
	The proof of $(i)$ can be found in \cite[Lemma 2.1]{feng2021strong}, while the establishment of $(iii)$ is based on the semigroup theory, which can be found in many references such as \cite{Printems, Lord, LV2021}. In addition, the proof of $(ii)$ is followed \cite[Lemma 10.27]{Lord} and \cite[Lemma 2.3]{vo2022higher} with minor modifications for $q = 4$. We just need to prove $(iv)$. To prove $(iv)$, we use the Taylor expansion for $F$ with respect to $u(s) \in L^2(D)$  as follows.
	\begin{align}
		F(u(t)) = F(u(s)) + DF(u(s))\bigl(u(t) - u(s)\bigr) + R_2,
	\end{align}
	where $\displaystyle R_2 = \int_0^1 (1-\eta) \bigl(D^2F(u(s) + \eta(u(t) - u(s)))\bigr)(u(t) - u(s))^2\, d\eta$.
	
	Therefore, we have
	\begin{align*}
		&F(u(t)) - F(u(s)) - \int_s^t DF(u(s)) G(u(\xi)) \, dW(\xi) \\\nonumber
		&= DF(u(s)) \bigg[u(t) - u(s) - \int_s^t G(u(\xi))\, dW(\xi)\bigg] + R_2.
	\end{align*}
	
	Since we have $DF(u) = 1 - q u^{q-1}$, then we obtain
	\begin{align}\label{eq2.11}
		&\Bigl\|DF(u(s)) \Bigl[u(t) - u(s) - \int_s^t G(u(\xi))\, dW(\xi)\Bigr]\Bigr\|^2_{L^2} \\\nonumber
		&= \int_D \Bigl| (1 - q u(s)^{q-1})\Bigl[u(t) - u(s) - \int_s^t G(u(\xi))\, dW(\xi)\Bigr]\Bigr|^2\, dx \\\nonumber
		&\leq \int_D  2(1 + q^2 |u(s)|^{2(q-1)})\Bigl|\Bigl[u(t) - u(s) - \int_s^t G(u(\xi))\, dW(\xi)\Bigr]\Bigr|^2\, dx \\\nonumber
		&\leq 2 \Bigl(\int_D(1+ q^2|u(s)|^{2(q-1)})^2\, dx\Bigr)^{\frac12} \Bigl\|u(t) - u(s) - \int_s^t G(u(\xi))\, dW(\xi)\Bigr\|^2_{L^4}\\\nonumber
		&\leq 2 \Bigl(\int_D2(1+ q^4|u(s)|^{4(q-1)})\, dx\Bigr)^{\frac12} \Bigl\|u(t) - u(s) - \int_s^t G(u(\xi))\, dW(\xi)\Bigr\|^2_{L^4}.
	\end{align}
	Taking the expectation $\mE[\cdot]$ to \eqref{eq2.11} and then using the Cauchy-Schwarz inequality, we obtain
	\begin{align}
		&\mE\Bigl[\Bigl\|DF(u(s)) \Bigl[u(t) - u(s) - \int_s^t G(u(\xi))\, dW(\xi)\Bigr]\Bigr\|^2_{L^2}\Bigr] \\\nonumber
		&\leq \mE\Bigl[2 \Bigl(\int_D2(1+ q^4|u(s)|^{4(q-1)})\, dx\Bigr)^{\frac12} \Bigl\|u(t) - u(s) - \int_s^t G(u(\xi))\, dW(\xi)\Bigr\|^2_{L^4}\Bigr]\\\nonumber
		&\leq C_q \Bigl(\mE\Bigl[\|u(s)\|^{4(q-1)}_{L^{4(q-1)}}\Bigr]\Bigr)^{\frac12} \Bigl(\mE\Bigl[\Bigl\|u(t) - u(s) - \int_s^t G(u(\xi))\, dW(\xi) \Bigr\|^4_{L^4}\Bigr]\Bigr)^{\frac12}.
	\end{align}
	Using the interpolation inequality that $\mE[\|u\|^4_{L^4}] \leq C \mE[\|u\|^2_{L^2}\|\nab u\|^2_{L^2}] \leq C \mE[\|u\|^4_{H^1}]$ and Lemma \ref{lemma2.4} $(iii)$ yield to
	\begin{align}
		&\mE\Bigl[\Bigl\|u(t) - u(s) - \int_s^t G(u(\xi))\, dW(\xi) \Bigr\|^4_{L^4}\Bigr]\\\nonumber
		&\leq C\mE\Bigl[\Bigl\|u(t) - u(s) - \int_s^t G(u(\xi))\, dW(\xi) \Bigr\|^4_{H^1}\Bigr] \leq C |t - s| ^{4\theta_2}.
	\end{align}
	By using \eqref{pde_estimate}, we arrive at
	\begin{align}\label{eq2.144}
		&\mE\Bigl[\Bigl\|DF(u(s)) \Bigl[u(t) - u(s) - \int_s^t G(u(\xi))\, dW(\xi)\Bigr]\Bigr\|^2_{L^2}\Bigr] \leq C|t - s|^{2\theta_2},
	\end{align}
	where $C = C_q \Bigl(\sup_{s \in [0,T]}\mE\Bigl[\|u(s)\|^{4(q-1)}_{L^{4(q-1)}}\Bigr]\Bigr)^{\frac12}$.
	
	It is remaining to estimate $R_2$. To do that, we notice that $D^2F(u) = -q(q-1)u^{q-2}$. In the end, we have
	\begin{align}\label{eq2.15}
		&\|R_2\|^2_{L^2} \\ \nonumber
		\leq& \int_D\Bigl| \int_0^1 (1-\eta) q(1-q) (u(s) + \eta (u(t) - u(s)))^{q-2} (u(t) - u(s))^2\, d\eta\Bigr|^2\, dx\\\nonumber
		\leq& \int_D\Bigl(q(q-1)2^{q-2} \bigl(|u(s)|^{q-2} + |u(t) - u(s)|^{q-2}\bigr) \Bigr)^2|u(t) - u(s)|^4\, dx\\\nonumber
		\leq& \int_D q^2 (q-1)^2 2^{2q-3} \bigl(|u(s)|^{2(q-2)} + |u(t) - u(s)|^{2(q-2)}\bigr) |u(t) - u(s)|^4\, dx\\\nonumber
		=& C_q \int_D |u(s)|^{2(q-2)} |u(t) - u(s)|^{4}\, dx + C_q \int_D|u(t) - u(s)|^{2q}\,dx\\\nonumber
		\leq& C_q \|u(s)\|^{2(q-2)}_{L^{4(q-2)}} \|u(t) - u(s)\|^4_{L^8} + C_q \|u(t) - u(s)\|^{2q}_{L^{2q}}.
	\end{align}
	Taking the expectation $\mE[\cdot]$ to \eqref{eq2.15}, using Lemma \ref{lemma2.4} $(iii)$ and then \eqref{pde_estimate} , we obtain
	\begin{align}\label{eq2.16}
		\mE[\|R_2\|^2_{L^2}] &\leq C_q\mE\bigl[\|u(s)\|^{2(q-2)}_{L^{4(q-2)}} \|u(t) - u(s)\|^4_{L^8}\bigr] + C_q \mE\bigl[\|u(t) - u(s)\|^{2q}_{L^{2q}}\bigr] \\\nonumber
		&\leq C_q\Bigl(\mE\bigl[\|u(s)\|^{4(q-2)}_{L^{4(q-2)}}\bigr]\Bigr)^{\frac12} \Bigl(\mE\bigl[\|u(t) - u(s)\|^8_{L^8}\bigr]\Bigr)^{\frac12} \\\nonumber
		&\qquad\qquad+ \mE\bigl[\|u(t) - u(s)\|^{2q}_{L^{2q}}\bigr]\\\nonumber
		&\leq C(|t - s|^{4\theta_1} + |t-s|^{2q\theta_1}) \leq C|t-s|^{4\theta_1},
	\end{align}
	where $C = C_q\Bigl(\sup_{s \in [0,T]}\mE\bigl[\|u(s)\|^{4(q-2)}_{L^{4(q-2)}}\bigr]\Bigr)^{\frac12}$.
	
	The proof is complete by combining \eqref{eq2.144} and \eqref{eq2.16}.
\end{proof}

\section{Semi-discretization in time}\label{section_semi} In this section, we follow the strategy of the Milstein scheme in SDEs to propose a new time discretization method of \eqref{eq1.1}. This approach generates a convergence order of nearly $1$ for the approximate solution. 
\subsection{Formulation of the proposed method}

Let $t_0 < t_1 < \cdots < t_M$ be a uniform mesh of the interval $[0,T]$ 
with the time step size $\tau = \frac{T}{M}$. Note that $t_0 = 0$ and $t_M = T$. 

\smallskip
\noindent
\textbf{Algorithm 1} 

Let $u^0 = u_0$ be a given $H^1_0$-valued random variable. Find $u^{n+1} \in H^1_0(D)$ recursively such that $\mP$-a.s. 
\begin{align}\label{milsteinscheme1}
	\bigl(u^{n+1} - u^n, \phi\bigr) + \tau \bigl(\nab u^{n+1}, \nab\phi\bigr)   &= \tau \bigl(F(u^{n+1}), \phi\bigr) + \bigl(G(u^n)\Delta W_{n} \\\nonumber
	&\qquad+ \frac12 DG(u^n)\,G(u^n)\bigl[(\Delta W_n)^2 - \tau\bigr],\phi\bigr),
\end{align}
for all $\phi \in H^1_0(D)$ and $\Delta W_n = W(t_{n+1}) - W(t_n) \sim \mathcal{N}(0,\tau)$.

\bigskip
Next, we define $\displaystyle \mathcal{G}: \mathbb{R}^+\times H_{0}^1(D) \rightarrow L^2(D)$ by
\begin{align}\label{eq3.2}
	\mathcal{G}(s;u) := {G}(u) + DG(u) G(u)\, \int_{t_{n}}^s\,d W(r),\qquad t_n \leq s \leq t_{n+1}.
\end{align}
Then we have
\begin{align*}
	\int_{t_{n}}^{t_{n+1}}\mathcal{G}(s;u^n)\, dW(s) &= G(u^n)\Delta W_{n} + DG(u^n)G(u^n) \int_{t_{n}}^{t_{n+1}}\int_{t_{n}}^s d W(r)\, dW(s)\\\nonumber
	&= G(u^n)\Delta W_{n} + \frac12DG(u^n)G(u^n) \bigl[(\Delta W_{n})^2 - \tau\bigr].
\end{align*}
Therefore, we rewrite \eqref{milsteinscheme1} as follow:
\begin{align}\label{milsteinscheme2}
	\bigl(u^{n+1} - u^n, \phi\bigr) +  \tau \bigl(\nab u^{n+1}, \nab\phi\bigr)   =& \tau\bigl(F(u^{n+1}), \phi\bigr)\\
	&+ 	\int_{t_{n}}^{t_{n+1}}\bigl(\mathcal{G}(s;u^n),\phi\bigr)\, dW(s).\notag
\end{align}

Next, we state the following technical lemma that is used to prove the error estimate results of this paper.
\smallskip
\begin{lemma}\label{lemma3.1} Suppose that $G$ satisfies the assumptions ${\bf(A1), (A2), (A3)}$. Let $u_0\in L^2(\Ome;H^1_0(D)\cap H^2(D))$, there exist constants $C>0$ such that the function $\mathcal{G}$ defined in \eqref{eq3.2} satisfies 
	\smallskip
	\begin{enumerate}[\rm (i)]
		\item  $\displaystyle	\|\mathcal{G}(s;u) - \mathcal{G}(s;v)\|_{L^2} \leq C\|u - v\|_{L^2},\qquad\forall s>0,u,v \in L^2(D)$,
		\smallskip
		\item  $\displaystyle \mE\bigl[\bigl\|G(u(s)) - \mathcal{G}(s;u(t_n))\bigr\|^2_{L^2}\bigr] \leq C|s - t_n|^{2(1-\epsilon)}$, for $t_n\leq s < t_{n+1}$ and $\epsilon>0$.
	\end{enumerate}
\end{lemma}
\smallskip
\begin{proof}
	The Lipschitz continuity of $\mathcal{G}$ in $(i)$ is directly obtained from the assumptions of $G$ while the proof of $(ii)$ can be found in \cite[Lemma 10.36]{Lord} with similar arguments.
	
\end{proof}

%\medskip
Next, we will provide the stability estimates of Algorithm 1 in the following lemma. These stability estimates will be used for the proof of the error estimates of the finite element approximation later.
\smallskip
\begin{lemma}
	Suppose the $u_0 \in L^2(\Omega; H^1_0(D)\cap H^2(D))$. There exists a constant $C>0$ such that:
	\begin{align}
		\sup_{1\leq n \leq M}\mE\bigl[\|\nab u^n\|^2_{L^2}\bigr] + \mE\Bigl[\sum_{n=1}^M\|\nab(u^n - u^{n-1})\|^2_{L^2}\Bigr]+ \mE\Bigl[\tau \sum_{n=1}^M\|\Delta u^n\|^2_{L^2}\Bigr] \leq C.
	\end{align}
\end{lemma}
\begin{proof}
	First we rewrite \eqref{milsteinscheme1} in the strong form as follow:
	\begin{align}\label{eq2.6}
		u^{n+1} - u^n -\tau\Delta u^{n+1} &= \tau F(u^{n+1}) + G(u^n)\Delta W_n \\\nonumber
		&\qquad+ \frac12 DG(u^n)G(u^n)[(\Delta W_n)^2 - \tau].
	\end{align}
	Testing the equation \eqref{eq2.6} by $-\Delta u^{n+1}$ and then using integration by parts we obtain
	\begin{align}
		&\bigl(\nabla(u^{n+1} -u^n), \nab u^{n+1}\bigr) + \tau \|\Delta u^{n+1}\|^2_{L^2} \\
		&\qquad= -\tau\bigl(F(u^{n+1}), \Delta u^{n+1}\bigr)- \bigl(G(u^n), \Delta u^{n+1}\bigr)\Delta W_{n}\nonumber\\
		&\qquad\quad- \frac12 \bigl(DG(u^n)G(u^n),\Delta u^{n+1}\bigr)\bigl[(\Delta W_n)^2 - \tau\bigr]\nonumber\\
		&\qquad:={\tt I + II + III}.\nonumber
	\end{align}  
	By using the integration by parts, we obtain
	\begin{align} \label{eq_3.7}
		{\tt I} &= -\tau\bigl(u^{n+1}, \Delta u^{n+1}\bigr) + \tau \bigl((u^{n+1})^q,\Delta u^{n+1}\bigr)\\\nonumber
		&=  \tau\|\nab u^{n+1}\|^2_{L^2} - \tau q\bigl((u^{n+1})^{q-1}\nab u^{n+1},\nabla u^{n+1}\bigr)\\\nonumber
		&= \tau\|\nab u^{n+1}\|^2_{L^2} - \tau q \int_D (u^{n+1})^{q-1}|\nab u^{n+1}|^2\, dx \leq \tau \|\nab u^{n+1}\|^2_{L^2}, 
	\end{align}
	where the last inequality of \eqref{eq_3.7} is obtained by using the fact that, for all odd $q\geq 3$,
	$\int_D (u^{n+1})^{q-1}|\nab u^{n+1}|^2\, dx \geq 0$.
	
	To bound {\tt II}, we take the expectation and then use the fact that  $\mE[\Delta W_n] = 0$. Namely,
	\begin{align}
		\mE[{\tt II}] &= -\mE\bigl[\bigl(G(u^n), \Delta(u^{n+1} - u^n)\bigr)\Delta W_n\bigr] - \mE\bigl[\bigl(G(u^n), \Delta u^n\bigr)\Delta W_n\bigr]\\\nonumber
		&= \mE\bigl[\bigl(\nab G(u^n), \nab(u^{n+1} - u^n)\bigr)\Delta W_n\bigr]\\\nonumber
		&\leq C\mE[\|\nab u^{n}\|^2_{L^2}|\Delta W_n|^2] + \frac14\mE\bigl[\|\nab(u^{n+1} - u^n)\|^2_{L^2}\bigr]\\\nonumber
		&= C\tau\mE[\|\nab u^{n}\|^2_{L^2}|] + \frac14\mE\bigl[\|\nab(u^{n+1} - u^n)\|^2_{L^2}\bigr].
	\end{align}
	In addition, by using the Cauchy-Schwarz and the assumptions ${\bf (A1), (A2)}$, we have
	\begin{align}\label{eq2.9}
		\mE[{\tt III}] &\leq \frac{C}{\tau}\mE\bigl[\|DG(u^n)G(u^n)\|^2_{L^2}|(\Delta W_n)^2 - \tau|^2\bigr] + \frac{\tau}{4}\mE\bigl[\|\Delta u^{n+1}\|^2_{L^2}\bigr]\\\nonumber
		&\leq \frac{C}{\tau}\mE\bigl[\|G(u^n)\|^2_{L^2}|(\Delta W_n)^2 - \tau|^2\bigr] + \frac{\tau}{4}\mE\bigl[\|\Delta u^{n+1}\|^2_{L^2}\bigr]\\\nonumber
		&\leq C\tau\mE\bigl[\|\nab u^{n}\|^2_{L^2}\bigr] + \frac{\tau}{4}\mE\bigl[\|\Delta u^{n+1}\|^2_{L^2}\bigr],
	\end{align}
	where the last inequality of \eqref{eq2.9} is obtained by  using the fact that $\mE[|(\Delta W_n)^2 -\tau |^2] \leq C\tau^2$.
	
	Substituting all the estimates from ${\tt I, II, III}$ into \eqref{eq2.6} and absorbing the like-terms from the right side to the left side, we obtain
	\begin{align}\label{eq2.10}
		\frac12\mE\bigl[\|\nab u^{n+1}\|^2_{L^2} &- \|\nab u^n\|^2_{L^2}\bigr] + \frac14\mE\bigl[\|\nab(u^{n+1} - u^n)\|^2_{L^2}\bigr] + \frac{\tau}{2}\mE\bigl[\|\Delta u^{n+1}\|^2_{L^2}\bigr]\\\nonumber
		& \leq C\tau\mE\bigl[\|u^{n+1} - u^n\|^2_{L^2}\bigr] + C\tau\mE\bigl[\|\nab u^{n}\|^2_{L^2}\bigr].
	\end{align}
	Next, applying the summation $\sum_{n=0}^{\ell}$, for any $0\leq \ell <M$, we obtain
	\begin{align}
		&\mE\bigl[\|\nab u^{\ell +1}\|^2_{L^2}\bigr] + \sum_{n=0}^{\ell}\mE\bigl[\|\nab(u^{n+1}-u^n)\|^2_{L^2}\bigr] + \tau\sum_{n=0}^{\ell} \mE\bigl[\|\Delta u^{n+1}\|^2_{L^2}\bigr]\\\nonumber
		&\leq C\tau\sum_{n=0}^{\ell}\mE\bigl[\|\nab u^n\|^2_{L^2}\bigr] + \mE\bigl[\|\nab u_0\|^2_{L^2}\bigr] + C\tau\sum_{n=0}^{\ell} \mE\bigl[\|u^{n+1} - u^n\|^2_{L^2}\bigr].
	\end{align}
	Finally, the proof is completed by using the Gronwall's inequality.
	
\end{proof}
\subsection{Error estimates for Algorithm 1}\label{sec3.2}

In this part, we state the first main result of this paper which establishes an $O(\tau^{1-\eps})$ convergence order for the proposed method.

\begin{theorem}\label{theorem_semi} 
	Let $u$ be the variational solution to \eqref{eq1.1} and $\{u^{n}\}_{n=1}^M$ be generated by  Algorithm 1. Assume that $G$ satisifies ${\bf (A1), (A2), (A3)}$ and $u_0 \in L^{2}(\Ome; H^1_0(D)\cap H^2(D))$. Suppose that $0<\epsilon<1$, then there exists a constant $C = C(D_T, u_0)>0$ such that 
	\begin{align}\label{eq310}
		\Bigl(\sup_{1\leq n \leq M}\mE\Bigl[\|u(t_n) - u^n\|^2_{L^2}\Bigr]\Bigr)^{\frac12} + \bigg(\mE\bigg[\tau\sum_{n=1}^M\|\nab(u(t_n) - u^n)\|^2_{L^2}\bigg]\bigg)^{\frac12} \leq C\, \tau^{1-\epsilon}.
	\end{align}
\end{theorem}
\begin{proof} Denote $e^n := u(t_n) - u ^n$. Subtracting \eqref{milsteinscheme2} from \eqref{weak_form}, we obtain the following error equation
	\begin{align}\label{eq2.13}
		\bigl(e^{n+1} - e^n, \phi\bigr) + \tau\bigl(\nab e^{n+1}, \nab \phi\bigr) &= \int_{t_{n}}^{t_{n+1}} \bigl(\nab(u(t_{n+1}) - u(s)), \nab \phi\bigr)\, ds \\\nonumber
		&\qquad- \int_{t_{n}}^{t_{n+1}} \bigl(F(u(t_{n+1})) - F(u(s)), \phi\bigr)\, ds\\\nonumber
		&\qquad+  \int_{t_{n}}^{t_{n+1}} \bigl(F(u(t_{n+1})) - F(u^{n+1}), \phi\bigr)\, ds\\\nonumber
		&\qquad+ \int_{t_{n}}^{t_{n+1}} \bigl(G(u(s)) - \mathcal{G}(s;u^n),\phi\bigr)\, dW(s).
	\end{align}
	Now, choosing $\phi = e^{n+1}$ and using the identity $2a(a-b) = a^2 -b^2 + (a-b)^2$, we have
	\begin{align}\label{eq2.14}
		&\frac12\bigl[\|e^{n+1}\|^2_{L^2} - \|e^{n}\|^2_{L^2}\bigr] + \frac12\|e^{n+1} - e^n\|^2_{L^2} + \tau\|\nab e^{n+1}\|^2_{L^2} \\\nonumber
		&= \int_{t_{n}}^{t_{n+1}} \bigl(\nab(u(t_{n+1}) - u(s)), \nab e^{n+1}\bigr)\, ds \\\nonumber
		&\qquad- \int_{t_{n}}^{t_{n+1}} \bigl(F(u(t_{n+1})) - F(u(s)), e^{n+1}\bigr)\, ds\\\nonumber
		&\qquad+  \int_{t_{n}}^{t_{n+1}} \bigl(F(u(t_{n+1})) - F(u^{n+1}), e^{n+1}\bigr)\, ds\\\nonumber
		&\qquad+ \int_{t_{n}}^{t_{n+1}} \bigl(G(u(s)) - \mathcal{G}(s;u^n),e^{n+1}\bigr)\, dW(s)\\\nonumber
		%		&= \int_{t_{n}}^{t_{n+1}} \Bigl(\nab\Bigl(u(t_{n+1}) - u(s) - \int_s^{t_{n+1}}G(u(\xi))\, dW(\xi)\Bigr), \nab e^{n+1}\Bigr)\, ds \\\nonumber
		%		&\qquad+\int_{t_{n}}^{t_{n+1}} \Bigl(\int_s^{t_{n+1}}\nab G(u(\xi))\, dW(\xi),\nab e^{n+1}\Bigr)\, ds\\\nonumber
		%		&\qquad- \int_{t_{n}}^{t_{n+1}} \Bigl(F(u(t_{n+1})) - F(u(s)) -\int_s^{t_{n+1}} DF(u(s))G(u(\xi))\, dW(\xi), e^{n+1}\Bigr)\, ds\\\nonumber
		%		&\qquad-\int_{t_{n}}^{t_{n+1}}\Bigl(\int_s^{t_{n+1}}DF(u(s))G(u(\xi))\, dW(\xi), e^{n+1}\Bigr)\, ds\\\nonumber
		%		&\qquad+  \int_{t_{n}}^{t_{n+1}} \bigl(F(u(t_{n+1})) - F(u^{n+1}), e^{n+1}\bigr)\, ds\\\nonumber
		%		&\qquad+ \int_{t_{n}}^{t_{n+1}} \bigl(G(u(s)) - \mathcal{G}(s;u^n),e^{n+1}\bigr)\, dW(s)\\\nonumber
		&:= {\tt I + II + III + IV}.
	\end{align}
	Next, we bound the right side of \eqref{eq2.14} as follows.
	
	In order to estimate {\tt I}, we add and subtract $\displaystyle \int_s^{t_{n+1}}\nab G(u(\xi))\, dW(\xi)$ for any $t_n\leq s < t_{n+1}$, as follow.
	\begin{align}
		{\tt I} &= \int_{t_{n}}^{t_{n+1}} \Bigl(\nab\Bigl(u(t_{n+1}) - u(s) - \int_s^{t_{n+1}}G(u(\xi))\, dW(\xi)\Bigr), \nab e^{n+1}\Bigr)\, ds \\\nonumber
		&\qquad+\int_{t_{n}}^{t_{n+1}} \Bigl(\int_s^{t_{n+1}}\nab G(u(\xi))\, dW(\xi),\nab e^{n+1}\Bigr)\, ds\\\nonumber
		&:= {\tt I_1 + I_2}.
	\end{align}
	By using Lemma \ref{lemma2.4} $(ii)$, we obtain
	\begin{align}
		\mE[{\tt I_1}] &\leq \int_{t_{n}}^{t_{n+1}} \mE\Bigl[\Bigl\|u(t_{n+1}) - u(s) - \int_s^{t_{n+1}}G(u(\xi))\, dW(\xi)\Bigr\|^2_{H^1}\Bigr]\, ds\\\nonumber
		&\qquad+ \frac{\tau}{4}\mE\bigl[\|\nab e^{n+1}\|^2_{L^2}\bigr]\\\nonumber
		&\leq C\tau^{1+ 2(1-\eps)} + \frac{\tau}{4}\mE\bigl[\|\nab e^{n+1}\|^2_{L^2}\bigr].
	\end{align}
	Next, by the integration by parts we have
	\begin{align}
		{\tt I_2} &= \int_{t_{n}}^{t_{n+1}} \Bigl(\int_s^{t_{n+1}}\nab G(u(\xi))\, dW(\xi),\nab (e^{n+1} - e^n)\Bigr)\, ds \\\nonumber
		&\qquad+\int_{t_{n}}^{t_{n+1}} \Bigl(\int_s^{t_{n+1}}\nab G(u(\xi))\, dW(\xi),\nab e^{n}\Bigr)\, ds\\\nonumber
		&=-\int_{t_{n}}^{t_{n+1}} \Bigl(\int_s^{t_{n+1}}\Delta G(u(\xi))\, dW(\xi),e^{n+1} - e^n\Bigr)\, ds \\\nonumber
		&\qquad+\int_{t_{n}}^{t_{n+1}} \Bigl(\int_s^{t_{n+1}}\nab G(u(\xi))\, dW(\xi),\nab e^{n}\Bigr)\, ds\\\nonumber
		&:= {\tt I_{2a} + I_{2b}}.
	\end{align}
	
	We note that $\mE[{\tt I_{2b}}] = 0$ due to the martingale property of the It\^o integral. So, it is left to estimate ${\tt I_{2a}}$. 
	By using the H\"older inequality, we obtain
	\begin{align}
		{\tt I_{2a}} &= -\int_{t_n}^{t_{n+1}}\bigg(\int_s^{t_{n+1}}\Delta G(u(\xi))\, dW(\xi), e^{n+1} - e^n\bigg)\, ds\\\nonumber
		%&= \int_{t_n}^{t_{n+1}}\bigg(\int_s^{t_{n+1}}{\bf A}G(\vu(\xi))\, d\vW(\xi), \ve^{n+1} - \ve^n\bigg)\, ds\\\nonumber
		&\leq 2\bigg\|\int_{t_n}^{t_{n+1}}\int_s^{t_{n+1}}\Delta G(u(\xi))\,dW(\xi)\,ds\bigg\|^2_{L^2} + \frac18\|e^{n+1} - e^n\|^2_{L^2}\\\nonumber
		& = 2\int_D\bigg|\int_{t_n}^{t_{n+1}}\int_s^{t_{n+1}}\Delta G(u(\xi))\,dW(\xi)\,ds\bigg|^2\, d{\bf x}+ \frac18\|e^{n+1} - e^n\|^2_{L^2}\\\nonumber
		&\leq 2\int_D\bigg(\int_{t_n}^{t_{n+1}}\bigg|\int_s^{t_{n+1}}\Delta G(u(\xi))\,dW(\xi)\bigg|\,ds\bigg)^2\, d{ x}+ \frac18\|e^{n+1} - e^n\|^2_{L^2}\\\nonumber
		&\leq 2\tau\int_D\int_{t_n}^{t_{n+1}}\bigg|\int_s^{t_{n+1}}\Delta G(u(\xi))\, dW(\xi)\bigg|^2\, ds\, d{x}  + \frac18\|e^{n+1} - e^n\|^2_{L^2}\\\nonumber
		&= 2\tau \int_{t_n}^{t_{n+1}}\bigg\|\int_s^{t_{n+1}}\Delta G(u(\xi))\, dW(\xi)\bigg\|^2_{L^2}\, ds + \frac18\|e^{n+1} - e^n\|^2_{L^2}.
	\end{align}
	
	By using the It\^o isometry we have
	\begin{align}
		\mE[{\tt I_2}] = \mE[{\tt I_{2a}}] &\leq C\tau^{3}\sup_{\xi\in [0,T]}\mE[\|u(\xi)\|^2_{H^2}] + \frac18\mE[\|e^{n+1} - e^n\|^2_{L^2}]\\\nonumber
		&\leq C\tau^{3} + \frac18\mE[\|e^{n+1} - e^n\|^2_{L^2}].
	\end{align}
	
	Similarly, we can estimate {\tt II} as follows.
	\begin{align}
		{\tt II} &=- \int_{t_{n}}^{t_{n+1}} \Bigl(F(u(t_{n+1})) - F(u(s)) -\int_s^{t_{n+1}} DF(u(s))G(u(\xi))\, dW(\xi), \\\nonumber
		&\qquad e^{n+1}\Bigr)\, ds-\int_{t_{n}}^{t_{n+1}}\Bigl(\int_s^{t_{n+1}}DF(u(s))G(u(\xi))\, dW(\xi), e^{n+1}\Bigr)\, ds\\\nonumber
		&:= {\tt II_1 + II_2}.
	\end{align}
	
	By using Lemma \ref{lemma2.4} $(iv)$ and Poincar\'e's inequality, we obtain
	\begin{align}
		\mE[{\tt II_1}] &\leq C\tau^{1+2(1-\eps)} + \frac{\tau}{4}\mE\bigl[\|\nab e^{n+1}\|^2_{L^2}\bigr].
	\end{align}
	
	To estimate ${\tt II_2}$, we use the same techniques from estimating ${\tt I_{2}}$ and also use \eqref{pde_estimate}, we obtain
	\begin{align*}
		\mE[{\tt II_2}] &= -\mE\Bigl[\int_{t_{n}}^{t_{n+1}}\Bigl(\int_s^{t_{n+1}}DF(u(s))G(u(\xi))\, dW(\xi), e^{n+1}-e^n\Bigr)\, ds\Bigr]\\\nonumber
		&\leq C\mE\Bigl[\Bigl\|\int_{t_{n}}^{t_{n+1}}\int_{s}^{t_{n+1}}DF(u(s))G(u(\xi))\, dW(\xi)\, ds\Bigr\|^2_{L^2}\Bigr] + \frac{1}{8}\mE\bigl[\|e^{n+1} - e^n\|^2_{L^2}\bigr]\\\nonumber
		&\leq C\mE\Bigl[\int_D\Bigl(\int_{t_{n}}^{t_{n+1}} \Bigl|\int_{s}^{t_{n+1}} DF(u(s))G(u(\xi))\, dW(\xi)\Bigr|\, ds\Bigr)^2\, dx\Bigr]\\\nonumber
		&\qquad+ \frac{1}{8}\mE\bigl[\|e^{n+1} - e^n\|^2_{L^2}\bigr]\\\nonumber
		&\leq C\tau\mE\Bigl[\int_D\int_{t_{n}}^{t_{n+1}}\Bigl|\int_{s}^{t_{n+1}}DF(u(s))G(u(\xi))\, dW(\xi)\Big|^2\, ds\, dx\Bigr]\\\nonumber
		&\qquad+ \frac{1}{8}\mE\bigl[\|e^{n+1} - e^n\|^2_{L^2}\bigr]\\\nonumber
		&=C\tau\mE\Bigl[\int_{t_{n}}^{t_{n+1}}\Bigl\|\int_{s}^{t_{n+1}}DF(u(s))G(u(\xi))\, dW(\xi)\Bigr\|^2_{L^2}\, ds\Bigr] + \frac{1}{8}\mE\bigl[\|e^{n+1} - e^n\|^2_{L^2}\bigr]\\\nonumber
		&=C\tau \mE\Bigl[\int_{t_{n}}^{t_{n+1}} \int_{s}^{t_{n+1}}\|DF(u(s))G(u(\xi))\|^2_{L^2}\, d\xi\, ds\Bigr]+ \frac{1}{8}\mE\bigl[\|e^{n+1} - e^n\|^2_{L^2}\bigr]\\\nonumber
		&\leq C\tau^3 \Bigl(\sup_{s \in [0,T]}\mE\bigl[\|u(s)\|^{4(q-1)}_{L^{4(q-1)}}\bigr]\Bigr)^{1/2}\Bigl(\sup_{\xi\in [0,T]}\mE\bigl[\|u(\xi)\|^4_{H^1}\bigr]\Bigr)^{1/2}\\\nonumber
		&\qquad+ \frac{1}{8}\mE\bigl[\|e^{n+1} - e^n\|^2_{L^2}\bigr]\\\nonumber
		&\leq C\tau^3 + \frac{1}{8}\mE\bigl[\|e^{n+1} - e^n\|^2_{L^2}\bigr].
	\end{align*}
	
	To estimate ${\tt III}$, we use the one-sided Lipschitz condition \eqref{oneside_Lip} as follows.
	\begin{align}
		\mE[{\tt III}]	&\leq C\tau\mE\bigl[\|e^{n+1}\|^2_{L^2}\bigr]\\\nonumber
		&\leq C\tau\mE\bigl[\|e^{n+1} - e^n\|^2_{L^2}\bigr] + C\tau\mE\bigl[\|e^n\|^2_{L^2}\bigr].
	\end{align}
	
	To estimate {\tt IV}, using Lemma \ref{lemma3.1}, the It\^o isometry and the martingale property of It\^o integrals we have
	\begin{align}
		\mE[{\tt IV}] &= \mE\Bigl[\int_{t_{n}}^{t_{n+1}} \bigl(G(u(s)) - \mathcal{G}(s;u^n),e^{n+1} - e^n\bigr)\, dW(s)\Bigr] \\\nonumber
		&\qquad+ \mE\Bigl[\int_{t_{n}}^{t_{n+1}} \bigl(G(u(s)) - \mathcal{G}(s;u^n),e^{n}\bigr)\, dW(s)\Bigr]\\\nonumber
		&= \mE\Bigl[\int_{t_{n}}^{t_{n+1}} \bigl(G(u(s)) - \mathcal{G}(s;u^n),e^{n+1} - e^n\bigr)\, dW(s)\Bigr] + 0 \\\nonumber
		&= \mE\Bigl[\int_{t_{n}}^{t_{n+1}} \bigl(G(u(s)) - \mathcal{G}(s;u(t_n)),e^{n+1} - e^n\bigr)\, dW(s)\Bigr] \\\nonumber
		&\qquad+\mE\Bigl[\int_{t_{n}}^{t_{n+1}} \bigl(\mathcal{G}(s;u(t_n)) - \mathcal{G}(s;u^n),e^{n+1} - e^n\bigr)\, dW(s)\Bigr]\\\nonumber
		&\leq C\mE\Bigl[\Bigl\|\int_{t_n}^{t_{n+1}} \bigl(G(u(s)) - \mathcal{G}(s;u(t_n))\bigr)\, dW(s)\Bigr\|^2_{L^2}\Bigr] \\\nonumber
		&\qquad+C\mE\Bigl[\Bigl\|\int_{t_n}^{t_{n+1}} \bigl(\mathcal{G}(s;u(t_n)) - \mathcal{G}(s;u^n)\bigr)\, dW(s)\Bigr\|^2_{L^2}\Bigr] \\\nonumber
		&\qquad+ \frac18\mE\bigl[\|e^{n+1} - e^n\|^2_{L^2}\bigr]\\\nonumber
		&=C\mE\Bigl[\int_{t_n}^{t_{n+1}}\|G(u(s)) - \mathcal{G}(s;u(t_n))\|^2_{L^2}\, ds\Bigr] \\\nonumber
		&\qquad+ C\mE\Bigl[\int_{t_n}^{t_{n+1}}\|\mathcal{G}(s;u(t_n)) - \mathcal{G}(s;u^n)\|^2_{L^2}\, ds\Bigr]\\\nonumber
		&\qquad+ \frac18\mE\bigl[\|e^{n+1} - e^n\|^2_{L^2}\bigr]\\\nonumber
		&\leq C\tau^{1+ 2(1-\eps)} + C\tau\mE\bigl[\|e^n\|^2_{L^2}\bigr] +  \frac18\mE\bigl[\|e^{n+1} - e^n\|^2_{L^2}\bigr].
	\end{align}
	
	Now, we substitute all the estimates from {\tt I, II, III, IV} into \eqref{eq2.14} and use the left side to absorb the like-terms from the right side of the resulting inequality. In summary, we obtain 
	\begin{align}\label{eq2.25}
		&\frac12\mE\bigl[\|e^{n+1}\|^2_{L^2} - \|e^n\|^2_{L^2}\bigr] + \Bigl(\frac18 - C\tau\Bigr)\mE\bigl[\|e^{n+1} - e^n\|^2_{L^2}\bigr] + \frac{\tau}{2}\mE\bigl[\|\nab e^{n+1}\|^2_{L^2}\bigr]\\\nonumber
		&\leq C\tau^{1+2(1-\eps)} + C\tau\mE\bigl[\|e^{n}\|^2_{L^2}\bigr] + C\tau^3.
	\end{align}
	We choose $\tau \leq \tau_0$ ( for $k_0$ small enough) such that $\frac{1}{8} - C\tau \geq 0$, so the middle term on the left side of \eqref{eq2.25} is nonnegtive.
	
	Next, applying the summation $\sum_{n=0}^{m}$ for $0\leq m < M$, we obtain
	\begin{align}
		\mE\bigl[\|e^{m+1}\|^2_{L^2}\bigr] + \tau\sum_{n=0}^m \mE\bigl[\|\nab e^{n+1}\|^2_{L^2}\bigr] \leq C\tau^{2(1-\eps)} + C\tau\sum_{n=0}^{m}\mE\bigl[\|e^n\|^2_{L^2}\bigr].
	\end{align}
	
	By using the discrete Gronwall's inequality and taking supremum over all $0\leq m <M$, we arrive at
	\begin{align}\label{eq2.27}
		\sup_{1\leq n \leq M}\mE\bigl[\|e^{n}\|^2_{L^2}\bigr] + \tau\sum_{n=1}^M \mE\bigl[\|\nab e^{n}\|^2_{L^2}\bigr] \leq Ce^{CT}\tau^{2(1-\eps)}.
	\end{align}
	
	%	Now, we use the inequality \eqref{eq2.27} as a tool to prove the desired estimate \eqref{eq310}. To do that, we first apply the summation $\sum_{n=1}^{\ell}$, $\max_{1\leq \ell \leq M}$ and then take the expectation to \eqref{eq2.14} to arive at
	%	\begin{align}\label{eq2.28}
		%		&\frac12\mE\bigl[\max_{1\leq \ell \leq M}\|e^{\ell}\|^2_{L^2}\bigr] + \frac12\sum_{n=1}^{M}\mE\bigl[\|e^{n} - e^{n-1}\|^2_{L^2}\bigr] + \tau\sum_{n=1}^M\mE\bigl[\|\nab e^{n}\|^2_{L^2}\bigr]\\\nonumber
		%		&\leq \mE\Bigl[\max_{1\leq \ell \leq M}\Bigl|\sum_{n=1}^{\ell}\int_{t_{n-1}}^{t_{n}} \bigl(\nab(u(t_{n}) - u(s)), \nab e^{n}\bigr)\, ds   \Bigr|\Bigr] \\\nonumber
		%		&\qquad+ \mE\Bigl[\max_{1\leq \ell \leq M}\Bigl|\sum_{n=1}^{\ell}\int_{t_{n-1}}^{t_{n}}\bigl(F(u(t_{n})) - F(u(s)), e^n\bigr)\, ds\Bigr|\Bigr]\\\nonumber
		%		&\qquad+ \mE\Bigl[\max_{1\leq \ell \leq M}\Bigl|\sum_{n=1}^{\ell}\int_{t_{n-1}}^{t_{n}}\bigl(F(u(t_{n})) - F(u^n), e^n\bigr)\, ds\Bigr|\Bigr]\\\nonumber
		%		&\qquad+ \mE\Bigl[\max_{1\leq \ell \leq M}\Bigl|\sum_{n=1}^{\ell} \int_{t_{n-1}}^{t_{n}}\bigl(G(u(s)) - \mathcal{G}(s;u^{n-1}), e^{n} - e^{n-1}\bigr)\, dW(s)\Bigr|\Bigr]\\\nonumber
		%		&\qquad+ \mE\Bigl[\max_{1\leq \ell \leq M}\Bigl|\sum_{n=1}^{\ell} \int_{t_{n-1}}^{t_{n}}\bigl(G(u(s)) - \mathcal{G}(s;u^{n-1}), e^{n-1}\bigr)\, dW(s)\Bigr|\Bigr]\\\nonumber
		%		&:= {\tt A + B + C + D + E}.
		%	\end{align}
	%	
	%	To estimate ${\tt A}$, we follow the same technique that has been used in estimating ${\tt I}$ above. To the end, we add and subtract $\displaystyle \int_s^{t_{n}}\nab G(u(\xi))\, dW(\xi)$ for any $t_{n-1}\leq s < t_{n}$ to obtain
	%	\begin{align*}
		%		{\tt A} &\leq \mE\bigg[\sum_{n = 1}^M\int_{t_{n-1}}^{t_{n}}\bigg\|\nab\Bigl(u(t_{n}) - u(s) - \int_s^{t_{n}} G(u(\xi))\, dW(\xi)\Bigr)\bigg\|_{L^2}\|\nab e^{n}\|_{L^2}\, ds\bigg] \\\nonumber
		%		&\qquad+ \mE\Bigl[\max_{1\leq m \leq M}\bigg|\sum_{n = 1}^m\int_{t_{n-1}}^{t_{n}} \bigg(\int_s^{t_{n}}\Delta G(u(\xi))\, dW(\xi),e^{n} - e^{n-1}\bigg)\, ds\bigg|\bigg] \\\nonumber
		%		&\qquad+\mE\bigg[ \max_{1\leq n \leq M}\bigg|\sum_{n=1}^m\int_{t_{n-1}}^{t_{n}} \bigg(\int_s^{t_{n}}\Delta G(u(\xi))\, dW(\xi),e^{n-1}\bigg)\, ds\bigg|\bigg]. 
		%	\end{align*}
	%	Next, by using the Cauchy-Schwarz inequality and then Fubini's theorem we obtain
	%	\begin{align}
		%		{\tt A} &\leq \mE\bigg[\sum_{n = 1}^M \int_{t_{n-1}}^{t_n} \Bigl\|u(t_n) - u(s) - \int_{s}^{t_n} G(u(\xi))\, dW(\xi)\Bigr\|^2_{H^1} + \frac14 \tau\sum_{n = 1}^M \|\nab e^n\|^2_{L^2}\bigg] \\\nonumber
		%		&+ \mE\bigg[\sum_{n = 1}^M \Bigl(\Bigl\|\int_{t_{n-1}}^{t_{n}}\int_{s}^{t_n} \Delta G(u(\xi))\, dW(\xi)\, ds\Bigr\|^2_{L^2} + \frac{1}{4}\|e^n - e^{n-1}\|^2_{L^2}\Bigr)\bigg]\\\nonumber
		%		&+ \mE\bigg[ \max_{1\leq n \leq M}\bigg|\sum_{n=1}^m \bigg(\int_{t_{n-1}}^{t_{n}}\int_{t_{n-1}}^{\xi}\Delta G(u(\xi))\,ds\,dW(\xi) ,e^{n-1}\bigg)\bigg|\bigg]\\\nonumber
		%		&:= {\tt A_1 + A_2 + A_3}.
		%	\end{align}
	%	By using Lemma \ref{lemma2.4} $(ii)$ and the estimate \eqref{eq2.27} we obtain
	%	\begin{align*}
		%		{\tt A_1 + A_2} \leq C\tau^{2(1-\eps)} + C\tau^{2(1-\eps)}\sup_{\xi\in [0,T]}\mE\bigl[\|u(\xi)\|^2_{H^2}\bigr] \leq C\tau^{2(1-\eps)}.
		%	\end{align*}
	%	Next, by using the Burkholder-Davis-Gundy inequality \cite[Lemm 4.3]{Chow07} and \eqref{eq2.27} we have
	%	\begin{align}
		%		{\tt A_3} 	&\leq \mE\bigg[\Bigl(\sum_{n = 1}^M\int_{t_{n-1}}^{t_n} \Bigl\|\int_{t_{n-1}}^{\xi}\Delta G(u(\xi))\, ds\Bigr\|^2_{L^2}\|e^{n-1}\|^2_{L^2}\, d\xi\Bigr)^{1/2}\bigg]\\\nonumber
		%		&\leq \mE\bigg[\frac18\max_{1\leq n \leq M}\|e^n\|^2_{L^2} + 2\sum_{n = 1}^M \int_{t_{n-1}}^{t_n} \Bigl\|\int_{t_{n-1}}^{\xi}\Delta G(u(\xi))\, ds\Bigr\|^2_{L^2}\, d\xi\bigg]\\\nonumber
		%		&\leq \frac18\mE\bigl[\max_{1\leq n \leq M}\|e^n\|^2_{L^2}\bigr] + C\tau^{2}\sup_{\xi\in [0,T]}\mE\bigl[\|u(\xi)\|^2_{H^2}\bigr].
		%	\end{align}
	%	
	%	To estimate ${\tt B}$, we add and subtract $\displaystyle \int_{s}^{t_n} DF(u(s))G(u(\xi))\, dW(\xi)$ and follow the same techniques as used in estimating ${\tt A}$ above and additionally use the assumption ${(\bf A3)}$ and Lemma \ref{lemma2.4} $(iii)$. In the end, we obtain the following estimate
	%	\begin{align}
		%		{\tt B} \leq C\tau^{2(1-\eps)} + C\tau^{2(1-\eps)}\sup_{\xi\in [0,T]}\mE\bigl[\|u(\xi)\|^2_{L^2}\bigr] + \frac18\mE\bigl[\max_{1\leq n \leq M}\|e^n\|^2_{L^2}\bigr].
		%	\end{align}
	%	
	%	In order to estimate {\tt C}, we are going to use \eqref{eq2.27} and ${\bf(A1)}$. Namely, we have
	%	\begin{align}
		%		{\tt C} &\leq C\tau\sum_{n = 1}^M\mE\Bigl[\|e^{n}\|^2_{L^2}\Bigr] \leq C\tau^{2(1-\eps)}.
		%	\end{align}
	%	
	%	By using Lemma \ref{lemma3.1} $(ii)$, the It\^o isometry and \eqref{eq2.27}, we estimate ${\tt D}$ as follows.
	%	\begin{align}
		%		{\tt D} &\leq \mE\Bigl[\max_{1\leq \ell \leq M}\Bigl|\sum_{n=1}^{\ell} \int_{t_{n-1}}^{t_{n}}\bigl(G(u(s)) - \mathcal{G}(s;u(t_{n-1})), e^{n} - e^{n-1}\bigr)\, dW(s)\Bigr|\Bigr]\\\nonumber
		%		&\qquad+\mE\Bigl[\max_{1\leq \ell \leq M}\Bigl|\sum_{n=1}^{\ell} \int_{t_{n-1}}^{t_{n}}\bigl(\mathcal{G}(s;u(t_{n-1})) - \mathcal{G}(s;u^{n-1}), e^{n} - e^{n-1}\bigr)\, dW(s)\Bigr|\Bigr]\\\nonumber
		%		&\leq \frac14\sum_{n = 1}^{M}\mE\bigl[\|e^{n} - e^{n-1}\|^2_{L^2}\bigr] + C\sum_{n = 1}^{M}\int_{t_{n-1}}^{t_n}\mE\bigl[\|G(u(s)) - \mathcal{G}(s;u(t_{n-1}))\|^2_{L^2}\bigr]\, ds\\\nonumber
		%		&\qquad+C\sum_{n = 1}^{M}\int_{t_{n-1}}^{t_n}\mE\bigl[\|\mathcal{G}(s;u(t_{n-1})) - \mathcal{G}(s;u^{n-1})\|^2_{L^2}\bigr]\, ds\\\nonumber
		%		&\leq C\tau^{2(1-\eps)}.
		%	\end{align}
	%	
	%	By using the Burholder-Davis-Gundy inequality and Lemma \ref{lemma3.1}, we obtain
	%	\begin{align*}
		%		{\tt E} &\leq \mE\bigg[\Bigl(\sum_{n = 1}^M \int_{t_{n-1}}^{t_n}\|G(u(s)) - \mathcal{G}(s;u(t_{n-1}))\|^2_{L^2}\|e^{n-1}\|^2_{L^2}\, ds\Bigr)^{1/2}\bigg] \\\nonumber
		%		&+ \mE\bigg[\Bigl(\sum_{n = 1}^M \int_{t_{n-1}}^{t_n}\|\mathcal{G}(s;u(t_{n-1})) - \mathcal{G}(s;u^{n-1})\|^2_{L^2}\|e^{n-1}\|^2_{L^2}\, ds\Bigr)^{1/2}\bigg]\\\nonumber
		%		&\leq \frac18 \mE\bigl[\max_{1\leq m \leq M} \|e^n\|^2_{L^2}\bigr] + C \tau^{2(1-\eps)}.
		%	\end{align*}
	%	
	%	Now, substituting the estimates from {\tt A, B, C, D, E} into \eqref{eq2.28} we obtain
	%	\begin{align}
		%		&\frac12\mE\Bigl[\max_{1\leq m \leq M}\|e^m\|^2_{L^2}\Bigr] + \frac12\mE\Bigl[\sum_{n = 1}^M \|e^n - e^{n-1}\|^2_{L^2}\Bigr] + \mE\Bigl[\tau\sum_{n = 1}^M \|\nab e^n\|^2_{L^2}\Bigr]\\\nonumber
		%		&\leq C\tau^{2(1-\eps)} + \frac14\mE\bigl[\max_{1\leq n \leq M}\|e^n\|^2_{L^2}\bigr],
		%	\end{align}
	%	which implies the desired estimate \eqref{eq310}. 
	The proof is complete.
\end{proof}


\section{Fully discrete finite element discretization}\label{fully_discrete}
In this section, we consider the $\mathcal{P}_1$-Lagrangian finite element space
\begin{align}\label{eq20180713_1}
	V_h = \bigl\{v_h \in H^1(\D): v_h|_{K} \in \mathcal{P}_1(K)\quad\forall K\in\mathcal{T}_h\bigr\},
\end{align}
where $\mathcal{P}_1$ denotes the space of all linear polynomials. Then the finite element approximation of Algorithm 1 is presented in Algorithm 2 as below.

\smallskip
\noindent
\textbf{Algorithm 2} 

We seek an $\mathcal{F}_{t_n}$ adapted $V_h$-valued process $\{u_h^n\}_{n=1}^N$ such that it holds $\mathbb{P}$-almost surely that
\begin{align}\label{dfem}
	&(u^{n+1}_h-u^n_h, v_h) + \tau ( \nabla u^{n+1}_h, \nabla v_h )= \tau (I_hF^{n+1}, v_h)\\
	&\quad+ (G(u^n_h), v_h) \, \Delta W_{n}+ \frac12 DG(u_h^n)\,G(u_h^n)\bigl[(\Delta W_n)^2 - \tau\bigr],v_h\bigr)\qquad \forall \, v_h \in V_h,\notag
\end{align}
where $F^{n+1}:=u^{n+1}_h-(u^{n+1}_h)^q$, ${\Delta} W_{n}=W(t_{n+1})-W(t_n) \sim \mathcal{N} (0,\tau)$, and  $I_h$ is the standard nodal value interpolation operator
$I_h: C(\bar{\D}) \longrightarrow V_h$, i.e.,
\begin{equation} \label{interpolation}
	I_h v := \sum_{i=1}^{N_h} v(a_i)\varphi_i,
\end{equation}
where $N_h$ denotes the number of vertices of the triangulation $\mathcal{T}_h$,
and ${\varphi_i}$ denotes the nodal basis function of $V_h$ corresponding to the vertex $a_i$.
%
The initial condition is chosen by $u_h^0  = P_h u_0$ where $P_h: L^2(\D) \longrightarrow V_h$ is the $L^2$-projection operator defined by
\begin{align*}
	\bigl(P_h w, v_h\bigr) = (w, v_h) \qquad v_h \in V_h.
\end{align*}
%
\par
For each $w \in H^s(\D)$  for $s>\frac32$, the following error estimates about the $L^2$-projection can be found in \cite{BS2008,ciarlet2002finite}:
\begin{align}
	\label{Ph1}
	&\|w - P_h w \|_{L^2} + h \| \nabla (w - P_h w) \|_{L^2}
	\leq C h^{\min\{2,s\}} \|w\|_{H^s},\\
	\label{Ph2}
	&\|w - P_h w\|_{L^\infty} \leq C h^{2-\frac{d}{2}} \|w\|_{H^2}.
\end{align}

Finally, given $v_h \in {V}_h$, the discrete Laplace operator $\Delta_h: {V}_h\longrightarrow
{V}_h$ is defined by
\begin{equation} \label{eq:discrete-Laplace}
	(\Delta_h v_h, w_h)=-(\nabla v_h,\nabla w_h) \qquad \forall\, w_h\in V_h.
\end{equation}

\subsection{Stability estimates for the $p$-th moment of the $H^1$-seminorm of $u_h^n$} \label{sec-3.2}
First, we shall prove the second moment discrete $H^1$-seminorm stability result, which is necessary to establish the corresponding higher moment stability result.
\begin{theorem}\label{thm20180711_1}
	Under some mesh constraint \eqref{eq20180907}, we have
	\begin{align}\label{eq20180711_7}
		\sup_{0\leq n \leq N}\E\left[\|\nabla u^{n}_h\|_{L^2}^2\right]&+\frac14\sum_{n=0}^{N-1}\E\left[\|\nabla (u^{n+1}_h-u^{n}_h)\|_{L^2}^2\right] \\
		&\quad +\tau\sum_{n=0}^{N-1}\E\left[\|\Delta_h u_h^{n}\|_{L^2}^2\right]\le C. \notag
	\end{align}
\end{theorem}
%
\begin{proof}
	Testing \eqref{dfem} with $-\Delta_h u_h^{n+1}$. Then
	\begin{align}\label{eq20180711_1}
		&(u^{n+1}_h-u^n_h, -\Delta_h u_h^{n+1}) + \tau ( \nabla u^{n+1}_h, -\nabla\Delta_h u_h^{n+1} ) \\\nonumber
		&\quad= \tau(I_hF^{n+1}, -\Delta_h u_h^{n+1}) + ( G(u^n_h), -\Delta_h u_h^{n+1}) \, \Delta W_{n+1} \\\nonumber
		&\qquad\qquad+\bigl(\frac12 DG(u_h^n)\,G(u_h^n)((\Delta W_n)^2 - \tau),-\Delta_h u_h^{n+1}\bigr).\notag
	\end{align}
	
	Using the definition of the discrete Laplace operator and the simple identity $2a(a-b) = a^2 - b^2 + (a-b)^2$, we get
	\begin{align}\label{eq20180711_2}
		(u^{n+1}_h-u^n_h, -\Delta_h u_h^{n+1})&=\frac12\|\nabla u^{n+1}_h\|_{L^2}^2-\frac12\|\nabla u^{n}_h\|_{L^2}^2\\
		&\qquad+\frac12\|\nabla (u^{n+1}_h-u^{n}_h)\|_{L^2}^2,\notag\\
		\tau ( \nabla u^{n+1}_h, -\nabla\Delta_h u_h^{n+1} )&=\tau\|\Delta_h u_h^{n+1}\|_{L^2}^2.\label{eq20181009_2}
	\end{align}
	
	The expectation of the second term on the right-hand side of \eqref{eq20180711_1} can be bounded by
	\begin{align}
		\E[( G(u^n_h), -\Delta_h u_h^{n+1}) \, \Delta W_{n}]&=\E [(\nabla(P_hG(u^n_h)), \nabla (u_h^{n+1}-u^n_h)) \, {\Delta} W_{n}]\label{eq20181009_3}\\
		&\le C\tau\E[\|\nabla u^n_h\|_{L^2}^2]+\frac14\E[\|\nabla (u_h^{n+1}-u^n_h)\|_{L^2}^2].\notag
	\end{align}
	
	The expectation of the third term on the right-hand side of \eqref{eq20180711_1} can be bounded by
	\begin{align}\label{eq20221026_1}
		&\frac12\E[( DG(u_h^n)\,G(u_h^n)((\Delta W_n)^2 - \tau),-\Delta_h u_h^{n+1})]\\
		=&\frac12\E [(\nabla(P_h(DG(u_h^n)\,G(u_h^n))), \nabla (u_h^{n+1}-u^n_h)) \, ((\Delta W_n)^2 - \tau)]\notag\\
		\le&C\tau^2\E[\|\nabla u^n_h\|_{L^2}^2]+\frac14\E[\|\nabla (u_h^{n+1}-u^n_h)\|_{L^2}^2],\notag
	\end{align}
	where the last inequality is obtained by using the assumption {\bf(A2)}. Notice that the stability in the $H^1$-seminorm of the $L^2$-projection (see \cite{bank2014h}) is used in the inequalities of \eqref{eq20181009_3} and \eqref{eq20221026_1}.
	
	For the first term on the right-hand side of \eqref{eq20180711_1} since it cannot be treated as a bad term, which aligns with the continuous case. Denote $u_i=u_h^{n+1}(a_i)$, and then
	\begin{align}\label{eq20180711_3}
		\tau (I_hF^{n+1}, -\Delta_h u_h^{n+1})&=\tau\|\nabla u^{n+1}_h\|_{L^2}^2-\tau(\nabla\sum_{i=1}^{N_h} u_i^q \varphi_i,\nabla \sum_{j=1}^{N_h} u_j\varphi_j)\\
		&=\tau\|\nabla u^{n+1}_h\|_{L^2}^2-\tau \sum_{i,j=1}^{N_h} ( u_i^q \nabla\varphi_i,  u_j\nabla\varphi_j)\notag\\
		&=\tau\|\nabla u^{n+1}_h\|_{L^2}^2-\tau \sum_{i,j=1}^{N_h} b_{ij}(\nabla\varphi_i,  \nabla\varphi_j)\notag,
	\end{align}
	where $b_{ij}=u_i^q u_j$.
	
	Using Young's inequality when $i\neq j$, we have
	\begin{align}\label{eq20180711_4}
		|b_{ij}|\le \frac{q}{q+1}u_i^{q+1}+\frac{1}{q+1}u_j^{q+1}.
	\end{align}
	
	Besides, since the stiffness matrix is diagonally dominant, we have
	\begin{align}
		-\tau \sum_{i,j=1}^{N_h} b_{ij}(\nabla\varphi_i,  \nabla\varphi_j)&\le-\tau \sum_{k=1}^{N_h} b_{kk}[(\nabla\varphi_k,  \nabla\varphi_k)-\frac{q}{q+1}\sum_{i=1,\atop i\neq k}^{N_h} |(\nabla\varphi_i,  \nabla\varphi_k)|\\
		&\quad-\frac{1}{q+1}\sum_{j=1,\atop j\neq k}^{N_h} |(\nabla\varphi_k,  \nabla\varphi_j)|]\notag\\
		&\le-\tau \sum_{k=1}^{N_h} b_{kk}[(\nabla\varphi_k,  \nabla\varphi_k)-\sum_{i=1,\atop i\neq k}^{N_h} (\nabla\varphi_i,  \nabla\varphi_k)]\notag\\
		&\le0\notag.
	\end{align}
	
	%Considering $k$-th row and $k$-th column of the stiffness matrix, the coefficient $C_k$ of $u_k^{q+1}$ can be bounded by
	%\begin{align}\label{eq20180711_5}
	%C_k\ge&\tau\bigl[(\nabla\varphi_k,  \nabla\varphi_k)-\frac{q}{q+1}\sum_{i\neq k}|(\nabla\varphi_i,  \nabla\varphi_k)|-\frac{1}{q+1}\sum_{j\neq k}|(\nabla\varphi_k,  \nabla\varphi_j)|\bigr]\\
	%=&\tau\bigl[(\nabla\varphi_k,  \nabla\varphi_k)-\sum_{i\neq k}|(\nabla\varphi_i,  \nabla\varphi_k)|\bigr]
	%\ge 0.\notag
	%\end{align}
	
	Then we have
	\begin{align}\label{eq20180711_6}
		\tau (I_hF^{n+1}, -\Delta_h u_h^{n+1})\le\tau\|\nabla u^{n+1}_h\|_{L^2}^2.
	\end{align}
	
	Combining \eqref{eq20180711_1}--\eqref{eq20181009_3} and \eqref{eq20180711_6}, and taking the summation, we have
	\begin{align}\label{eq20180712_1}
		&\frac12\E\left[\|\nabla u^{\ell}_h\|_{L^2}^2\right]+\frac14\sum_{n=0}^{\ell-1}\E\left[\|\nabla (u^{n+1}_h-u^{n}_h)\|_{L^2}^2\right]+\tau\sum_{n=0}^{\ell-1}\E\left[\|\Delta_h u_h^{n+1}\|_{L^2}^2\right]\\
		&\quad\le C\tau\sum_{n=0}^{\ell-1}\E[\|\nabla u^n_h\|_{L^2}^2].\notag
	\end{align}
	
	Using the Gronwall's inequality, we obtain \eqref{eq20180711_7}.
	%\begin{align}\label{eq20180711_7}
	%\E\left[\|\nabla u^{\ell}_h\|_{L^2}^2\right]+\frac14\sum_{n=0}^{\ell-1}\E\left[\|\nabla (u^{n+1}_h-u^{n}_h)\|_{L^2}^2\right]+\tau\sum_{n=0}^{\ell-1}\E\left[\|\Delta_h u_h^{n+1}\|_{L^2}^2\right]\le C.
	%\end{align}
\end{proof}

Before we establish the error estimates, we need to prove the stability of
the higher moments for the $H^1$-seminorm of the numerical solution.

\begin{theorem}\label{thm20180802_1}
	Suppose the mesh assumption \eqref{eq20180907} holds. Then for any $p\ge2$,
	\begin{align*}
		\sup_{0 \leq n \leq N} \E\left[\|\nabla u^{n}_h\|_{L^2}^p\right]\le C.
	\end{align*}
\end{theorem}
\begin{proof}
	The proof is divided into three steps. In Step 1, we establish the bound for $\E\|\nabla u^{\ell}_h\|_{L^2}^{4}$.
	In Step 2, we give the bound for $\E\|\nabla u^{\ell}_h\|_{L^2}^p$, where $p=2^r$ and $r$ is an arbitrary
	positive integer. In Step 3, we obtain the bound for $\E\|\nabla u^{\ell}_h\|_{L^2}^p$, where $p$ is an
	arbitrary real number and $p\ge2$.
	
	\smallskip
	{\bf Step 1.} Based on \eqref{eq20180711_1}--\eqref{eq20180711_6}, we have
	\begin{align}\label{eq20180802_1}
		&\frac12 \|\nabla u^{n+1}_h\|_{L^2}^2 -\frac12 \|\nabla u^{n}_h\|_{L^2}^2 +\frac12 \|\nabla (u^{n+1}_h-u^{n}_h)\|_{L^2}^2 +\tau \|\Delta_h u_h^{n+1}\|_{L^2}^2 \\[2mm]
		& -( G(u^n_h), -\Delta_h u_h^{n+1}) \, \Delta W_{n} +\frac12\bigl(DG(u_h^n)\,G(u_h^n)((\Delta W_n)^2 - \tau),-\Delta_h u_h^{n+1}\bigr)\notag\\
		&\le\tau \|\nabla u^{n+1}_h\|_{L^2}^2.\notag
	\end{align}
	
	Note the following identity
	\begin{align}\label{eq20180802_2}
		\|\nabla u^{n+1}_h\|_{L^2}^2+\frac12\|\nabla u^{n}_h\|_{L^2}^2&=\frac34(\|\nabla u^{n+1}_h\|_{L^2}^2+\|\nabla u^n_h\|_{L^2}^2)\\
		&\quad+\frac14(\|\nabla u^{n+1}_h\|_{L^2}^2-\|\nabla u^n_h\|_{L^2}^2).\notag
	\end{align}
	Multiplying \eqref{eq20180802_1} by $\|\nabla u^{n+1}_h\|_{L^2}^2+\frac12\|\nabla u^{n}_h\|_{L^2}^2$, we obtain
	\begin{align}\label{eq20180802_3}
		&\frac38(\|\nabla u^{n+1}_h\|_{L^2}^4-\|\nabla u^n_h\|_{L^2}^4)+\frac18(\|\nabla u^{n+1}_h\|_{L^2}^2-\|\nabla u^n_h\|_{L^2}^2)^2\\
		&\quad+(\frac12\|\nabla (u^{n+1}_h-u^{n}_h)\|_{L^2}^2+\tau\|\Delta_h u_h^{n+1}\|_{L^2}^2)(\|\nabla u^{n+1}_h\|_{L^2}^2+\frac12\|\nabla u^{n}_h\|_{L^2}^2)\notag\\
		&\le\tau\|\nabla u^{n+1}_h\|_{L^2}^2(\|\nabla u^{n+1}_h\|_{L^2}^2+\frac12\|\nabla u^{n}_h\|_{L^2}^2)\notag\\
		&\quad+( G(u^n_h), -\Delta_h u_h^{n+1}) \, {\Delta} W_{n}(\|\nabla u^{n+1}_h\|_{L^2}^2+\frac12\|\nabla u^{n}_h\|_{L^2}^2)\notag\\
		&\quad-\frac12\bigl(DG(u_h^n)\,G(u_h^n)((\Delta W_n)^2 - \tau),-\Delta_h u_h^{n+1}\bigr)(\|\nabla u^{n+1}_h\|_{L^2}^2+\frac12\|\nabla u^{n}_h\|_{L^2}^2).\notag
	\end{align}
	
	The first term on the right-hand side of \eqref{eq20180802_3} can be written as
	\begin{align}\label{eq20180802_4}
		&\tau\|\nabla u^{n+1}_h\|_{L^2}^2(\|\nabla u^{n+1}_h\|_{L^2}^2+\frac12\|\nabla u^{n}_h\|_{L^2}^2)\\
		&\quad=\tau\|\nabla u^{n+1}_h\|_{L^2}^2(\frac32\|\nabla u^{n+1}_h\|_{L^2}^2-\frac12(\|\nabla u^{n+1}_h\|_{L^2}^2-\|\nabla u^n_h\|_{L^2}^2))\notag\\
		&\quad\le C\tau\|\nabla u^{n+1}_h\|_{L^2}^4+\theta_1(\|\nabla u^{n+1}_h\|_{L^2}^2-\|\nabla u^n_h\|_{L^2}^2)^2\notag,
	\end{align}
	where $\theta_1>0$ will be determined later.
	
	The second term on the right-hand side of \eqref{eq20180802_3} can be written as
	\begin{align}\label{eq20180802_5}
		&(G(u^n_h), -\Delta_h u_h^{n+1}) \, {\Delta} W_{n}(\|\nabla u^{n+1}_h\|_{L^2}^2+\frac12\|\nabla u^{n}_h\|_{L^2}^2)\\
		&\quad=(\nabla P_hG(u^n_h), \nabla u_h^{n+1}) \, {\Delta} W_{n}(\|\nabla u^{n+1}_h\|_{L^2}^2+\frac12\|\nabla u^{n}_h\|_{L^2}^2)\notag\\
		&\quad= ((\nabla P_hG(u^n_h), \nabla u_h^{n+1}-\nabla u_h^n){\Delta} W_{n}\notag\\
		&\qquad+(\nabla P_hG(u^n_h),\nabla u_h^n){\Delta} W_{n})(\|\nabla u^{n+1}_h\|_{L^2}^2+\frac12\|\nabla u^{n}_h\|_{L^2}^2)\notag\\
		&\quad\le(\frac14\|\nabla u_h^{n+1}-\nabla u_h^n\|_{L^2}^2+C\|\nabla u_h^n\|_{L^2}^2({\Delta} W_{n})^2\notag\\
		&\qquad+(\nabla P_hG(u^n_h),\nabla u_h^n){\Delta} W_{n})(\|\nabla u^{n+1}_h\|_{L^2}^2+\frac12\|\nabla u^{n}_h\|_{L^2}^2)\notag.
	\end{align}
	
	For the right-hand side of \eqref{eq20180802_5}, using the Cauchy-Schwarz inequality, we get
	\begin{align}\label{eq20180802_6}
		&C\|\nabla u_h^n\|_{L^2}^2({\Delta} W_{n})^2(\|\nabla u^{n+1}_h\|_{L^2}^2+\frac12\|\nabla u^{n}_h\|_{L^2}^2)\\
		&\quad=C\|\nabla u_h^n\|_{L^2}^2({\Delta} W_{n})^2(\|\nabla u^{n+1}_h\|_{L^2}^2-\|\nabla u^{n}_h\|_{L^2}^2+\frac32\|\nabla u^{n}_h\|_{L^2}^2)\notag\\
		&\quad\le\theta_2(\|\nabla u^{n+1}_h\|_{L^2}^2-\|\nabla u^n_h\|_{L^2}^2)^2+C\|\nabla u_h^n\|_{L^2}^4({\Delta} W_{n})^4\notag\\
		&\qquad+C\|\nabla u_h^n\|_{L^2}^4({\Delta} W_{n})^2,\notag
	\end{align}
	where $\theta_2>0$ will be determined later.
	
	Similarly, using the Cauchy-Schwarz inequality, we have
	\begin{align}\label{eq20180802_7}
		&(\nabla P_hG(u^n_h),\nabla u_h^n){\Delta} W_{n}(\|\nabla u^{n+1}_h\|_{L^2}^2+\frac12\|\nabla u^{n}_h\|_{L^2}^2)\\
		&\quad=(\nabla P_hG(u^n_h),\nabla u_h^n){\Delta} W_{n}(\|\nabla u^{n+1}_h\|_{L^2}^2-\|\nabla u^{n}_h\|_{L^2}^2+\frac32\|\nabla u^{n}_h\|_{L^2}^2)\notag\\
		&\quad\le\theta_3(\|\nabla u^{n+1}_h\|_{L^2}^2-\|\nabla u^n_h\|_{L^2}^2)^2+C\|\nabla u_h^n\|_{L^2}^4({\Delta} W_{n})^2\notag\\
		&\qquad+\frac32(\nabla P_hG(u^n_h),\nabla u_h^n){\Delta} W_{n}\|\nabla u^{n}_h\|_{L^2}^2\notag,
	\end{align}
	where $\theta_3>0$ will be determined later.
	
	The third term on the right-hand side of \eqref{eq20180802_3} can be written as
	\begin{align}\label{eq20221226_2}
		&\quad-\frac12\bigl(DG(u_h^n)\,G(u_h^n)((\Delta W_n)^2 - \tau),-\Delta_h u_h^{n+1}\bigr)(\|\nabla u^{n+1}_h\|_{L^2}^2+\frac12\|\nabla u^{n}_h\|_{L^2}^2)\\
		&=-\frac12\bigl(\nabla P_h(DG(u_h^n)\,G(u_h^n))((\Delta W_n)^2 - \tau),\nabla(u_h^{n+1}-u_h^n)\bigr)\notag\\
		&\quad(\|\nabla u^{n+1}_h\|_{L^2}^2+\frac12\|\nabla u^{n}_h\|_{L^2}^2)-\frac12\bigl(\nabla P_h(DG(u_h^n)\,G(u_h^n))((\Delta W_n)^2 - \tau),\nabla u_h^n\bigr)\notag\\
		&\quad(\|\nabla u^{n+1}_h\|_{L^2}^2+\frac12\|\nabla u^{n}_h\|_{L^2}^2)\notag\\
		&\le(\frac14\|\nabla u_h^{n+1}-\nabla u_h^n\|_{L^2}^2+C\|\nabla u_h^n\|_{L^2}^2((\Delta W_n)^2-\tau)^2\notag\\
		&\quad-\frac12\bigl(\nabla P_h(DG(u_h^n)\,G(u_h^n))((\Delta W_n)^2 - \tau),\nabla u_h^n\bigr)(\|\nabla u^{n+1}_h\|_{L^2}^2+\frac12\|\nabla u^{n}_h\|_{L^2}^2)\notag.
	\end{align}
	
	For the right-hand side of \eqref{eq20221226_2}, using the Cauchy-Schwarz inequality, we get
	\begin{align}
		&C\|\nabla u_h^n\|_{L^2}^2((\Delta W_n)^2-\tau)^2(\|\nabla u^{n+1}_h\|_{L^2}^2+\frac12\|\nabla u^{n}_h\|_{L^2}^2)\\
		&\quad=C\|\nabla u_h^n\|_{L^2}^2((\Delta W_n)^2-\tau)^2(\|\nabla u^{n+1}_h\|_{L^2}^2-\|\nabla u^{n}_h\|_{L^2}^2+\frac32\|\nabla u^{n}_h\|_{L^2}^2)\notag\\
		&\quad\le\theta_4(\|\nabla u^{n+1}_h\|_{L^2}^2-\|\nabla u^n_h\|_{L^2}^2)^2+C\|\nabla u_h^n\|_{L^2}^4((\Delta W_n)^2-\tau)^4 \notag\\
		&\qquad +C\|\nabla u_h^n\|_{L^2}^4((\Delta W_n)^2-\tau)^2,\notag
	\end{align}
	where $\theta_4>0$ will be determined later.
	Similarly, using the Cauchy-Schwarz inequality, we have
	\begin{align}
		&(\nabla P_h(DG(u_h^n)\,G(u_h^n)),\nabla u_h^n)((\Delta W_n)^2 - \tau)(\|\nabla u^{n+1}_h\|_{L^2}^2+\frac12\|\nabla u^{n}_h\|_{L^2}^2)\\
		&\quad=(\nabla P_h(DG(u_h^n)\,G(u_h^n)),\nabla u_h^n)((\Delta W_n)^2 - \tau)(\|\nabla u^{n+1}_h\|_{L^2}^2\notag\\
		&\qquad-\|\nabla u^{n}_h\|_{L^2}^2+\frac32\|\nabla u^{n}_h\|_{L^2}^2)\notag\\
		&\quad\le\theta_5(\|\nabla u^{n+1}_h\|_{L^2}^2-\|\nabla u^n_h\|_{L^2}^2)^2+C\|\nabla u_h^n\|_{L^2}^4((\Delta W_n)^2 - \tau)^2\notag\\
		&\qquad+\frac32(\nabla P_h(DG(u^n)\,G(u^n)),\nabla u_h^n)((\Delta W_n)^2 - \tau)\|\nabla u^{n}_h\|_{L^2}^2\notag,
	\end{align}
	where $\theta_5>0$ will be determined later.
	
	Choosing $\theta_1\sim \theta_5$ such that $\theta_1+\cdots+\theta_3\le\frac{1}{16}$, and then taking the summation over $n$ from $0$ to $\ell-1$ and taking the expectation on both sides of \eqref{eq20180802_3}, we obtain
	\begin{align}\label{eq20180802_8}
		&\frac38\E\left[\|\nabla u^{\ell}_h\|_{L^2}^4\right]+\frac{1}{16}\sum_{n=0}^{\ell-1}\E\left[(\|\nabla u^{n+1}_h\|_{L^2}^2-\|\nabla u^n_h\|_{L^2}^2)^2\right]\\
		&+\sum_{n=0}^{\ell-1}\E\left[(\frac14\|\nabla (u^{n+1}_h-u^{n}_h)\|_{L^2}^2+\tau\|\Delta_h u_h^{n+1}\|_{L^2}^2)(\|\nabla u^{n+1}_h\|_{L^2}^2+\frac12\|\nabla u^{n}_h\|_{L^2}^2)\right]\notag\\
		&\le C\tau\sum_{n=0}^{\ell-1}\E\left[\|\nabla u^{n+1}_h\|_{L^2}^4\right]+\frac38\E\left[\|\nabla u^0_h\|_{L^2}^4\right]+C\tau^2\sum_{n=0}^{\ell-1}\E\left[\|\nabla u_h^n\|_{L^2}^4\right]\notag\\
		&+C\tau\sum_{n=0}^{\ell-1}\E\left[\|\nabla u_h^n\|_{L^2}^4\right].\notag
	\end{align}
	
	When restricting $\tau\le C$, we have
	\begin{align}\label{eq20180802_9}
		&\frac14\E\left[\|\nabla u^{\ell}_h\|_{L^2}^4\right]+\frac{1}{16}\sum_{n=0}^{\ell-1}\E\left[(\|\nabla u^{n+1}_h\|_{L^2}^2-\|\nabla u^n_h\|_{L^2}^2)^2\right]\\
		&+\sum_{n=0}^{\ell-1}\E\left[(\frac14\|\nabla (u^{n+1}_h-u^{n}_h)\|_{L^2}^2+\tau\|\Delta_h u_h^{n+1}\|_{L^2}^2)(\|\nabla u^{n+1}_h\|_{L^2}^2+\frac12\|\nabla u^{n}_h\|_{L^2}^2)\right]\notag\\
		&\le C\tau\sum_{n=0}^{\ell-1}\E\left[\|\nabla u^{n}_h\|_{L^2}^4\right]+\frac38\E\left[\|\nabla u^0_h\|_{L^2}^4\right].\notag
	\end{align}
	
	Using the Gronwall's inequality, we obtain
	\begin{align}\label{eq20180802_10}
		&\frac14\E\left[\|\nabla u^{\ell}_h\|_{L^2}^4\right]+\frac{1}{16}\sum_{n=0}^{\ell-1}\E\left[(\|\nabla u^{n+1}_h\|_{L^2}^2-\|\nabla u^n_h\|_{L^2}^2)^2\right]\\
		&\qquad+\sum_{n=0}^{\ell-1}\E\bigl[(\frac14\|\nabla (u^{n+1}_h-u^{n}_h)\|_{L^2}^2+\tau\|\Delta_h u_h^{n+1}\|_{L^2}^2)(\|\nabla u^{n+1}_h\|_{L^2}^2\notag\\
		&\qquad+\frac12\|\nabla u^{n}_h\|_{L^2}^2)\bigr]\le C.\notag
	\end{align}
	
	\smallskip
	{\bf Step 2.} Similar to Step 1, using \eqref{eq20180802_3}--\eqref{eq20180802_7}, we have
	\begin{align}\label{eq20180802_11}
		&\frac38(\|\nabla u^{n+1}_h\|_{L^2}^4-\|\nabla u^n_h\|_{L^2}^4)+\frac{1}{16}(\|\nabla u^{n+1}_h\|_{L^2}^2-\|\nabla u^n_h\|_{L^2}^2)^2\\
		&\quad+(\frac14\|\nabla (u^{n+1}_h-u^{n}_h)\|_{L^2}^2+\tau\|\Delta_h u_h^{n+1}\|_{L^2}^2)(\|\nabla u^{n+1}_h\|_{L^2}^2+\frac12\|\nabla u^{n}_h\|_{L^2}^2)\notag\\
		&\le C\tau\|\nabla u^{n+1}_h\|_{L^2}^4+C\|\nabla u_h^n\|_{L^2}^4({\Delta} W_{n})^4+C\|\nabla u_h^n\|_{L^2}^4({\Delta} W_{n})^2\notag\\
		&\quad+C\|\nabla u_h^n\|_{L^2}^4{\Delta} W_{n}.\notag
	\end{align}
	
	Proceed similarly as in Step 1. Multiplying \eqref{eq20180802_11} with $\|\nabla u^{n+1}_h\|_{L^2}^4+\frac12\|\nabla u^{n}_h\|_{L^2}^4$, we can obtain the 8-th moment of the $H^1$-seminorm stability result of the numerical solution.
	Then repeating this process, the $2^r$-th moment of the $H^1$-seminorm stability result of the numerical solution
	can be obtained.
	
	\smallskip
	{\bf Step 3.} Suppose $2^{r-1}\le p\le 2^r$. By Young's inequality, we have
	\begin{align}\label{eq20180802_12}
		\E\left[\|\nabla u^{\ell}_h\|_{L^2}^p\right]&\le \E\left[\|\nabla u^{\ell}_h\|_{L^2}^{2^r}\right]+C< \infty,
	\end{align}
	where the second inequality follows from the results of Step 2. The proof is completed.
\end{proof}

\subsection{Stability estimates for the $p$-th moment of the $L^2$-norm of $u_h^n$}
Since the mass matrix may not be the diagonally dominated matrix, we cannot use the above idea to prove the $L^2$ stability. Instead, we prove the stability results by utilizing the above established results. The following results hold when $q\ge3$ is the odd integer in 2D case, and when $q=3$ or $q=5$ in 3D case.
\begin{theorem}\label{thm20180911}
	Under the mesh assumption \eqref{eq20180907},   there holds
	\begin{align*}
		\sup_{0\leq n \leq N}\E\left[\|u^{n}_h\|_{L^2}^2\right]+\sum_{n=0}^{N-1}\E\left[\|(u^{n+1}_h-u^{n}_h)\|_{L^2}^2\right] &+\tau\sum_{n=0}^{N-1}\E\left[\|\nabla u_h^{n+1}\|_{L^2}^2\right]\\
		&+\frac{\tau}{2}\sum_{n=0}^{N-1}\E\left[\|u_h^{n+1}\|_{L^{q+1}}^{q+1}\right]\le C.
	\end{align*}
\end{theorem}
\begin{proof}
	Testing \eqref{dfem} with $u_h^{n+1}$ yields
	\begin{align}\label{eq20221227_1}
		&(u^{n+1}_h-u^n_h, u_h^{n+1}) + \tau ( \nabla u^{n+1}_h, \nabla u_h^{n+1} )= \tau (I_hF^{n+1}, u_h^{n+1})\\
		&\quad+ (G(u^n_h), u_h^{n+1}) \, \Delta W_{n} + \frac12 DG(u_h^n)\,G(u_h^n)\bigl[(\Delta W_n)^2 - \tau\bigr],u_h^{n+1}\bigr).\notag
	\end{align}
	
	We can easily prove the following inequalities:
	\begin{align*}
		(u^{n+1}_h-u^n_h, u_h^{n+1})&=\frac12\| u^{n+1}_h\|_{L^2}^2-\frac12\| u^{n}_h\|_{L^2}^2+\frac12\|u^{n+1}_h-u^{n}_h\|_{L^2}^2,\\
		\E[(G(u^n_h),  u_h^{n+1}) \, \Delta W_{n}]&=\E [(G(u^n_h),  (u_h^{n+1}-u^n_h)) \, \Delta W_{n}]\\
		&\le C\tau+C\tau\E[\| u^n_h\|_{L^2}^2]+\frac14\E[\|u_h^{n+1}-u^n_h\|_{L^2}^2],\\
		\E[DG(u_h^n)\,G(u_h^n)((\Delta W_n)^2 - \tau),u_h^{n+1}\bigr)]&=\E[DG(u_h^n)\,G(u_h^n)((\Delta W_n)^2 - \tau),u_h^{n+1}-u_h^n\bigr)]\\
		&\le C\tau^2+C\tau^2\E[\| u^n_h\|_{L^2}^2]+\frac14\E[\|u_h^{n+1}-u^n_h\|_{L^2}^2],
	\end{align*}
	where {\bf (A2)} is used in the inequality above.
	
	We have the following standard interpolation result and the inverse inequality (see \cite{ciarlet2002finite}):
	\begin{align}\label{eq20180807_7}
		\|v-I_hv\|_{L^{\frac{q+1}{q}}(K)} &\le Ch_K\|\nabla v\|_{L^{\frac{q+1}{q}}(K)},\\
		\|v\|_{L^{q+1}(K)}^{q+1} &\le\frac{C}{h_K^{d\cdot\frac{q-1}{2}}}\|v\|_{L^2(K)}^{q+1}.\label{eq20180807_8}
	\end{align}
	
	Using \eqref{eq20180807_7}--\eqref{eq20180807_8}, and Young's inequality, we have
	\begin{align}\label{eq20180808_2}
		&\tau(I_hF^{n+1}, u_h^{n+1})=\tau(F^{n+1}, u_h^{n+1})-\tau(F^{n+1}-I_hF^{n+1}, u_h^{n+1})\\
		&\quad\le\tau\|u_h^{n+1}\|_{L^2}^2-\tau\|u_h^{n+1}\|_{L^{q+1}}^{q+1}\notag\\
		&\qquad+C\tau\|F^{n+1}-I_hF^{n+1}\|_{L^{\frac{q+1}{q}}}^{\frac{q+1}{q}}+\frac{\tau}{4}\|u_h^{n+1}\|_{L^{q+1}}^{q+1}\notag\\
		&\quad\le\tau\|u_h^{n+1}\|_{L^2}^2-\tau\|u_h^{n+1}\|_{L^{q+1}}^{q+1}\notag\\
		&\qquad+C\tau \sum_{K\in\mathcal{T}_h}h_K^{\frac{q+1}{q}}\bigl((u_h^{n+1})^{\frac{q^2-1}{q}},(\nabla u_h^{n+1})^{\frac{q+1}{q}}\bigr)_K+\frac{\tau}{4}\|u_h^{n+1}\|_{L^{q+1}}^{q+1}\notag\\
		&\quad\le\tau\|u_h^{n+1}\|_{L^2}^2-\frac{\tau}{2}\|u_h^{n+1}\|_{L^{q+1}}^{q+1}+C\tau \sum_{K\in\mathcal{T}_h}h_K^{q+1}\|\nabla u_h^{n+1}\|_{L^{q+1}(K)}^{q+1}\notag\\
		&\quad\le\tau\|u_h^{n+1}\|_{L^2}^2-\frac{\tau}{2}\|u_h^{n+1}\|_{L^{q+1}}^{q+1}+C\tau\sum_{K\in\mathcal{T}_h}h_K^{q+1-d\frac{q-1}{2}}\|\nabla u_h^{n+1}\|_{L^2(K)}^{q+1}.\notag
	\end{align}
	
	Note when $d=2$, $q+1-d\frac{q-1}{2}\ge0$ if $q\ge0$, and when $d=3$, $q+1-d\frac{q-1}{2}\ge0$ if $q\le5$. Using the above inequalities, Theorem \ref{thm20180802_1}, taking summation over $n$ from $0$ to $\ell-1$, and taking expectation on both sides of \eqref{eq20221227_1}, we obtain
	\begin{align}\label{eq20180807_1}
		&\frac14\E\left[\|u^{\ell}_h\|_{L^2}^2\right]+\frac14\sum_{n=0}^{\ell-1}\E\left[\|(u^{n+1}_h-u^{n}_h)\|_{L^2}^2\right]+\tau\sum_{n=0}^{\ell-1}\E\left[\|\nabla u_h^{n+1}\|_{L^2}^2\right]\\
		&\qquad+\frac{\tau}{2}\sum_{n=0}^{\ell-1}\E\left[\|u_h^{n+1}\|_{L^{q+1}}^{q+1}\right]\notag\\
		&\le\tau\sum_{n=0}^{\ell-1}\E\left[\|u^n_h\|_{L^2}^2\right]+C\tau\sum_{n=0}^{\ell-1}\E\left[\|\nabla u_h^{n+1}\|_{L^2}^{q+1}\right]+C\notag\\
		&\le\tau\sum_{n=0}^{\ell-1}\E\left[\|u^n_h\|_{L^2}^2\right]+C,\notag
	\end{align}
	where Theorem \ref{thm20180802_1} is used in the last inequality.
	
	The conclusion is a direct result by using the Gronwall's inequality.
\end{proof}

%\begin{remark}
%Using Gagliardo-Nirenberg interpolation inequality, when space dimension $d=2$, we have
%\begin{align*}
%\|u_h^{n+1}\|^6_{L^6(\D)}&\le C\|\nabla u_h^{n+1}\|^{4}_{L^2(\D)}\|u_h^{n+1}\|^{2}_{L^2(\D)}+C\|u_h^{n+1}\|^{6}_{s}\\
%&\le C\|\nabla u_h^{n+1}\|^{8}_{L^2(\D)}+\|u_h^{n+1}\|^{4}_{L^4(\D)}+C+C\|u_h^{n+1}\|^{6}_{s}.
%\end{align*}
%where $s>0$.
%
%Then we obtain
%\begin{align*}
%&\frac{\tau}{\epsilon^2} (I_hf^{n+1}, u_h^{n+1})\\
%=&\frac{\tau}{\epsilon^2} (f^{n+1}, u_h^{n+1})-\frac{\tau}{\epsilon^2} (f^{n+1}-I_hf^{n+1}, u_h^{n+1})\\
%\ge&\frac{\tau}{\epsilon^2}\|u_h^{n+1}\|_{L^4}^4-\frac{2\tau}{\epsilon^2}\|u_h^{n+1}\|_{L^2}^2-C\frac{\tau}{\epsilon^2}\|u_h^{n+1}\|^6_{L^6(\D)}\\
%\ge&\frac{\tau}{2\epsilon^2}\|u_h^{n+1}\|_{L^4}^4-\frac{2\tau}{\epsilon^2}\|u_h^{n+1}\|_{L^2}^2-C\frac{\tau}{\epsilon^2}\|\nabla u_h^{n+1}\|^{8}_{L^2(\D)}-C-C\|u_h^{n+1}\|^{6}_{s}.
%\end{align*}
%
%When $\D=\mathbb{R}^2$, term $C\|u_h^{n+1}\|^{6}_{s}$ does not exist, so the $L^2$ stability can be proved directly, but when when $\D=\mathbb{R}^2$, it is not easy to control term $C\|u_h^{n+1}\|^{6}_{s}$.
%\end{remark}

To obtain the error estimates results, we need to establish a higher moment discrete $L^2$ stability result
for the numerical solution $u_h$.

\begin{theorem}\label{thm20180808_1}
	Suppose the mesh assumption \eqref{eq20180907} holds. Then there holds for any $p\ge2$,
	\begin{align*}
		\sup_{0\leq \ell \leq N}\E\left[\|u^{\ell}_h\|_{L^2}^p\right]\le C\notag.
	\end{align*}
\end{theorem}

\begin{proof}
	The proof is divided into three steps. In Step 1, we give the bound for $\E\| u^{\ell}_h\|_{L^2}^{4}$.
	In Step 2, we give the bound for $\E\| u^{\ell}_h\|_{L^2}^p$, where $p=2^r$ and $r$ is an arbitrary positive integer.
	In Step 3, we give the bound for $\E\| u^{\ell}_h\|_{L^2}^p$, where $p$ is an arbitrary real number and $p\ge2$.
	
	\smallskip
	{\rm Step 1.} Based on \eqref{eq20221227_1}--\eqref{eq20180808_2}, we have
	\begin{align}\label{eq20180808_3}
		\frac12\|u^{n+1}_h\|_{L^2}^2&-\frac12\|u^{n}_h\|_{L^2}^2 +\frac12\|u^{n+1}_h-u^{n}_h\|_{L^2}^2 +\tau\|\nabla u_h^{n+1}\|_{L^2}^2+\frac{\tau}{2}\|u_h^{n+1}\|_{L^{q+1}}^{q+1}\\[2mm]
		& \le\tau\|u_h^{n+1}\|_{L^2}^2+C\tau\|\nabla u_h^{n+1}\|_{L^2}^{q+1}+(G(u^n_h), u_h^{n+1}) \, \Delta W_{n}\notag\\
		&\qquad + \frac12 \bigl(DG(u_h^n)\,G(u_h^n)((\Delta W_n)^2 - \tau),u_h^{n+1}\bigr).\notag
	\end{align}
	
	Note the following identity
	\begin{align}\label{eq20180808_4}
		\|u^{n+1}_h\|_{L^2}^2+\frac12\|u^{n}_h\|_{L^2}^2=&\frac34(\|u^{n+1}_h\|_{L^2}^2+\|u^n_h\|_{L^2}^2)+\frac14(\|u^{n+1}_h\|_{L^2}^2-\|u^n_h\|_{L^2}^2).
	\end{align}
	Multiplying \eqref{eq20180808_3} by $\|u^{n+1}_h\|_{L^2}^2+\frac12\|u^{n}_h\|_{L^2}^2$, we obtain
	\begin{align}\label{eq20180808_5}
		&\frac38(\|u^{n+1}_h\|_{L^2}^4-\|u^n_h\|_{L^2}^4)+\frac18(\|u^{n+1}_h\|_{L^2}^2-\|u^n_h\|_{L^2}^2)^2+(\frac12\|(u^{n+1}_h-u^{n}_h)\|_{L^2}^2\\
		&\quad+\tau\|\nabla u_h^{n+1}\|_{L^2}^2+\frac{\tau}{2}\|u_h^{n+1}\|_{L^{q+1}}^{q+1})(\|u^{n+1}_h\|_{L^2}^2+\frac12\|u^{n}_h\|_{L^2}^2)\notag\\
		&\le(\tau\|u_h^{n+1}\|_{L^2}^2+C\tau\|\nabla u_h^{n+1}\|_{L^2}^{q+1})(\|u^{n+1}_h\|_{L^2}^2+\frac12\|u^{n}_h\|_{L^2}^2)\notag\\
		&\quad+(G(u^n_h), u_h^{n+1})\, \Delta W_{n}(\|u^{n+1}_h\|_{L^2}^2+\frac12\|u^{n}_h\|_{L^2}^2)\notag\\
		&\quad+\frac12 \bigl(DG(u_h^n)\,G(u_h^n)((\Delta W_n)^2 - k),u_h^{n+1}\bigr)(\|u^{n+1}_h\|_{L^2}^2+\frac12\|u^{n}_h\|_{L^2}^2).\notag
	\end{align}
	
	The first term on the right-hand side of \eqref{eq20180808_5} can be written as
	\begin{align}\label{eq20180808_6}
		&(\tau\|u_h^{n+1}\|_{L^2}^2+C\tau\|\nabla u_h^{n+1}\|_{L^2}^{q+1})(\|u^{n+1}_h\|_{L^2}^2+\frac12\|u^{n}_h\|_{L^2}^2)\\
		&\le\tau\|u^{n+1}_h\|_{L^2}^2(\frac32\|u^{n+1}_h\|_{L^2}^2-\frac12(\|u^{n+1}_h\|_{L^2}^2-\|u^n_h\|_{L^2}^2))\notag\\
		&\quad+C\tau\|\nabla u_h^{n+1}\|_{L^2}^{2(q+1)}+\tau\|u^{n+1}_h\|_{L^2}^4+\tau(\|u^{n+1}_h\|_{L^2}^2-\|u^n_h\|_{L^2}^2)^2\notag\\
		&\le C\tau\|u^{n+1}_h\|_{L^2}^4+C\tau\|\nabla u_h^{n+1}\|_{L^2}^{2(q+1)}+\theta_1(\|u^{n+1}_h\|_{L^2}^2-\|u^n_h\|_{L^2}^2)^2\notag,
	\end{align}
	where $\theta_1>0$ will be determined later.
	
	The second term on the right-hand side of \eqref{eq20180808_5} can be written as
	\begin{align}\label{eq20180808_7}
		&(G(u^n_h), u_h^{n+1})\, {\Delta} W_{n}(\|u^{n+1}_h\|_{L^2}^2+\frac12\|u^{n}_h\|_{L^2}^2)\\
		&=(G(u^n_h), u_h^{n+1}-u_h^n+u_h^n) \, {\Delta} W_{n}(\|u^{n+1}_h\|_{L^2}^2+\frac12\|u^{n}_h\|_{L^2}^2)\notag\\
		&\le(\frac14\|u_h^{n+1}-u_h^n\|_{L^2}^2+C(1+\|u_h^n\|_{L^2}^2)({\Delta} W_{n})^2\notag\\
		&\quad+(G(u_h^n),u_h^n) {\Delta} W_{n})(\|u^{n+1}_h\|_{L^2}^2+\frac12\|u^{n}_h\|_{L^2}^2)\notag.
	\end{align}
	
	For the second term on the right-hand side of \eqref{eq20180808_7}, using the Cauchy-Schwarz inequality, we get
	\begin{align}\label{eq20180808_8}
		&C(1+\|u_h^n\|_{L^2}^2)({\Delta} W_{n})^2(\|u^{n+1}_h\|_{L^2}^2+\frac12\|u^{n}_h\|_{L^2}^2)\\
		&=C(1+\|u_h^n\|_{L^2}^2)({\Delta} W_{n})^2(\|u^{n+1}_h\|_{L^2}^2-\| u^{n}_h\|_{L^2}^2+\frac32\|u^{n}_h\|_{L^2}^2)\notag\\
		&\le\theta_2\big(\|u^{n+1}_h\|_{L^2}^2-\|u^n_h\|_{L^2}^2)^2+(C+C\|u_h^n\|_{L^2}^4)({\Delta} W_{n})^4\notag\\
		&\quad+C\|u_h^n\|_{L^2}^4({\Delta} W_{n} \big)^2+C\|u_h^n\|_{L^2}^2({\Delta} W_{n})^2,\notag
	\end{align}
	where $\theta_2>0$ will be determined later.
	Using \eqref{lineargrow}, the third term on the right-hand side of \eqref{eq20180808_7} can be bounded by
	\begin{align}\label{eq20180808_9}
		&(G(u_h^n),u_h^n) {\Delta} W_{n}(\|u^{n+1}_h\|_{L^2}^2+\frac12\|u^{n}_h\|_{L^2}^2)\\
		&\quad=(G(u_h^n),u_h^n) {\Delta} W_{n}(\|u^{n+1}_h\|_{L^2}^2-\|u^{n}_h\|_{L^2}^2+\frac32\|u^{n}_h\|_{L^2}^2)\notag\\
		&\quad\le\theta_3(\|u^{n+1}_h\|_{L^2}^2-\|u^n_h\|_{L^2}^2)^2+(C+C\|u_h^n\|_{L^2}^4)({\Delta} W_{n})^2\notag\\
		&\qquad+\frac32(G(u_h^n),u_h^n)\|u_h^n\|_{L^2}^2{\Delta} W_{n}\notag,
	\end{align}
	where $\theta_3>0$ will be determined later.
	
	The third term on the right-hand side of \eqref{eq20180808_5} can be written as
	\begin{align}\label{eq20221227_4}
		&\frac12 \bigl(DG(u_h^n)\,G(u_h^n)((\Delta W_n)^2 - \tau),u_h^{n+1}\bigr)(\|u^{n+1}_h\|_{L^2}^2+\frac12\|u^{n}_h\|_{L^2}^2)\\
		&=\frac12 \bigl(DG(u_h^n)\,G(u_h^n)((\Delta W_n)^2 - \tau),u_h^{n+1}-u_h^n+u_h^n\bigr)(\|u^{n+1}_h\|_{L^2}^2+\frac12\|u^{n}_h\|_{L^2}^2)\notag\\
		&\le(\frac14\|u_h^{n+1}-u_h^n\|_{L^2}^2+C(1+\|u_h^n\|_{L^2}^2)((\Delta W_n)^2 - \tau)^2\notag\\
		&\quad+\frac12 \bigl(DG(u_h^n)\,G(u_h^n)((\Delta W_n)^2 - \tau),u_h^n\bigr)(\|u^{n+1}_h\|_{L^2}^2+\frac12\|u^{n}_h\|_{L^2}^2)\notag.
	\end{align}
	
	For the second term on the right-hand side of \eqref{eq20221227_4}, using the Cauchy-Schwarz inequality, we get
	\begin{align}\label{eq20221227_5}
		&C(1+\|u_h^n\|_{L^2}^2)((\Delta W_n)^2 - \tau)^2(\|u^{n+1}_h\|_{L^2}^2+\frac12\|u^{n}_h\|_{L^2}^2)\\
		&=C(1+\|u_h^n\|_{L^2}^2)((\Delta W_n)^2 - \tau)^2(\|u^{n+1}_h\|_{L^2}^2-\| u^{n}_h\|_{L^2}^2+\frac32\|u^{n}_h\|_{L^2}^2)\notag\\
		&\le\theta_2\big(\|u^{n+1}_h\|_{L^2}^2-\|u^n_h\|_{L^2}^2)^2+(C+C\|u_h^n\|_{L^2}^4)((\Delta W_n)^2 - \tau)^2\notag\\
		&\quad+C\|u_h^n\|_{L^2}^4((\Delta W_n)^2 - \tau)^2+C\|u_h^n\|_{L^2}^2((\Delta W_n)^2 - \tau)^2,\notag
	\end{align}
	where $\theta_4>0$ will be determined later.
	Using \eqref{lineargrow}, the third term on the right-hand side of \eqref{eq20221227_4} can be bounded by
	\begin{align}\label{eq20221227_6}
		&\frac12 \bigl(DG(u_h^n)\,G(u_h^n)((\Delta W_n)^2 - \tau),u_h^n)(\|u^{n+1}_h\|_{L^2}^2+\frac12\|u^{n}_h\|_{L^2}^2)\\
		&\quad=\frac12 \bigl(DG(u_h^n)\,G(u_h^n)((\Delta W_n)^2 - \tau),u_h^n)(\|u^{n+1}_h\|_{L^2}^2-\|u^{n}_h\|_{L^2}^2+\frac32\|u^{n}_h\|_{L^2}^2)\notag\\
		&\quad\le\theta_5(\|u^{n+1}_h\|_{L^2}^2-\|u^n_h\|_{L^2}^2)^2+(C+C\|u_h^n\|_{L^2}^4)(\Delta W_n)^2 - \tau)^2\notag\\
		&\qquad+\frac12 \bigl(DG(u_h^n)\,G(u_h^n)((\Delta W_n)^2 - \tau),u_h^n)\frac32\|u^{n}_h\|_{L^2}^2\notag,
	\end{align}
	where $\theta_5>0$ will be determined later.
	
	Choosing $\theta_1\sim\theta_5$ such that $\theta_1+\cdots+\theta_3\le\frac{1}{16}$, and then taking the summation over $n$ from $0$ to $\ell-1$ and taking the expectation on both sides of \eqref{eq20180808_5}, we obtain
	\begin{align}\label{eq20180808_10}
		&\frac38\E\left[\|u^{\ell}_h\|_{L^2}^4\right]+\frac{1}{16}\sum_{n=0}^{\ell-1}\E\left[(\|u^{n+1}_h\|_{L^2}^2-\|u^n_h\|_{L^2}^2)^2\right]+\sum_{n=0}^{\ell-1}\E\bigl[(\frac14\|(u^{n+1}_h-u^{n}_h)\|_{L^2}^2\\
		&\quad+\tau\|\nabla u_h^{n+1}\|_{L^2}^2+\frac{\tau}{2}\|u_h^{n+1}\|_{L^{q+1}}^{q+1})(\|u^{n+1}_h\|_{L^2}^2+\frac12\|u^{n}_h\|_{L^2}^2)\bigr]\notag\\
		&\le C\tau\sum_{n=0}^{\ell-1}\E\left[\|u^{n+1}_h\|_{L^2}^4\right]+C\tau\sum_{n=0}^{\ell-1}\E\left[\|\nabla u^{n+1}_h\|_{L^2}^{2(q+1)}\right]+\frac38\E\left[\|u^0_h\|_{L^2}^4\right]\notag\\
		&\quad+C\tau\sum_{n=0}^{\ell-1}\E\left[\|u_h^n\|_{L^2}^4\right]+C.\notag
	\end{align}
	
	When $\tau\le C$, we have
	\begin{align}\label{eq20180808_11}
		&\frac14\E\left[\|u^{\ell}_h\|_{L^2}^4\right]+\frac{1}{16}\sum_{n=0}^{\ell-1}\E\left[(\|u^{n+1}_h\|_{L^2}^2-\|u^n_h\|_{L^2}^2)^2\right]+\sum_{n=0}^{\ell-1}\E\bigl[(\frac14\|(u^{n+1}_h-u^{n}_h)\|_{L^2}^2\\
		&\quad+\tau\|\nabla u_h^{n+1}\|_{L^2}^2+\frac{\tau}{2}\|u_h^{n+1}\|_{L^4}^4)(\|u^{n+1}_h\|_{L^2}^2+\frac12\|u^{n}_h\|_{L^2}^2)\bigr]\notag\\
		&\le C\tau\sum_{n=0}^{\ell-1}\E\left[\|u^{n}_h\|_{L^2}^4\right]+C\tau\sum_{n=0}^{\ell-1}\E\left[\|\nabla u^{n+1}_h\|_{L^2}^{2(q+1)}\right]+\frac38\E\left[\|u^0_h\|_{L^2}^4\right]+C\notag.
	\end{align}
	
	Using the Gronwall's inequality, we obtain
	\begin{align}\label{eq20180808_12}
		&\frac14\E\left[\|u^{\ell}_h\|_{L^2}^4\right]+\frac{1}{16}\sum_{n=0}^{\ell-1}\E\left[(\|u^{n+1}_h\|_{L^2}^2-\|u^n_h\|_{L^2}^2)^2\right]\\
		&\qquad+\sum_{n=0}^{\ell-1}\E\bigg[(\frac14\|(u^{n+1}_h-u^{n}_h)\|_{L^2}^2+\tau\|\nabla u_h^{n+1}\|_{L^2}^2\notag\\
		&\qquad+\frac{\tau}{2}\|u_h^{n+1}\|_{L^4}^4)(\|u^{n+1}_h\|_{L^2}^2+\frac12\|u^{n}_h\|_{L^2}^2)\bigg]\le C.\notag
	\end{align}
	
	\smallskip
	{\rm Step 2.} Similar to Step 1, using \eqref{eq20180808_5}--\eqref{eq20180808_9}, we have
	\begin{align}\label{eq20180808}
		&\frac38(\|u^{n+1}_h\|_{L^2}^4-\|u^n_h\|_{L^2}^4)+\frac{1}{16}(\|u^{n+1}_h\|_{L^2}^2-\|u^n_h\|_{L^2}^2)^2\\
		&+(\frac14\|(u^{n+1}_h-u^{n}_h)\|_{L^2}^2+\tau\|\nabla u_h^{n+1}\|_{L^2}^2+\frac{\tau}{2}\|u_h^{n+1}\|_{L^4}^4)(\|u^{n+1}_h\|_{L^2}^2+\frac12\|u^{n}_h\|_{L^2}^2)\notag\\
		&\le C\tau\|u^{n+1}_h\|_{L^2}^4+C\tau\|\nabla u_h^{n+1}\|_{L^2}^{2(q+1)}+(C+C\|u_h^n\|_{L^2}^4)({\Delta} W_{n})^4\notag\\
		&+(C+C\|u_h^n\|_{L^2}^4)({\Delta} W_{n})^2+(G(u_h^n),u_h^n)\|u_h^n\|_{L^2}^2{\Delta} W_{n}.\notag
	\end{align}
	
	Similar to Step 1, multiplying \eqref{eq20180808} by $\|u^{n+1}_h\|_{L^2}^4+\frac12\|u^{n}_h\|_{L^2}^4$,
	we can obtain the 8-th moment of the $L^2$ stability result of the discrete solution. Then repeating this
	process, the second moment of the $L^2$ stability result of the discrete solution can be obtained.
	
	\smallskip
	{\rm Step 3.} Suppose $2^{r-1}\le p\le 2^r$, and then by Young's inequality, we have
	\begin{align}
		\E\left[\|u^{\ell}_h\|_{L^2}^p\right]&\le \E\left[\|u^{\ell}_h\|_{L^2}^{2^r}\right]+C\le C,
	\end{align}
	where Step 2 is used in the second inequality. The proof is complete.
\end{proof}

%Based on the above established stability results, it is ready to prove the error estimates below.
%%%%%%%%%%%%%%%
\subsection{Error estimates of the finite element approximation}\label{subsec3}
%Note if the elliptic projection is used, the term $T_2$ below cannot be bounded
%since there will be a scaling problem.
In this subsection, we consider error estimates between the semi-discrete solution $u^n$ of Algorithm 1 and its finite element approximation $u^n_h$ from Algorithm 2. Let $e^n=u^n-u_h^n$ $(n = 0,1,2,\ldots,N)$.
In the following theorem, the $L^2$-projection is used in the proof of the error estimates and
the strong convergence rate is given.

\begin{theorem}\label{thm:derrest}
	Let $\{u^n\}$ and $\{ u_h^n \}_{n=1}^N$ denote respectively the solutions of Algorithm 1 and Algorithm 2. Then, under the condition \eqref{eq20180907}, there holds
	\begin{align*}
		&\sup_{0 \leq n \leq N} \E \left[\| e^n \|^2_{L^2} \right]
		+ \E \left[\tau\sum_{n=1}^N  \|\nabla e^n\|^2_{L^2}  \right]\le  Ch^2|\ln h|^{2(q-1)}.
	\end{align*}
\end{theorem}
\begin{proof}
	We write $e^n = \eta^n + \xi^n$ where
	\begin{align*}
		\eta^n := u^n - P_h u^n \quad \text{and} \quad \xi^n := P_h u^n - u_h^n,  \quad n = 0,1,2,...,N.
	\end{align*}
	
	%	\begin{align}
		%				\bigl(u^{n+1} - u^n, \phi\bigr) + \tau \bigl(\nab u^{n+1}, \nab\phi\bigr)   &= \tau\bigl(F(u^{n+1}), \phi\bigr) + \bigl(G(u^n)\Delta W_{n} \\\nonumber
		%			&\qquad+ \frac12 DG(u^n)\,G(u^n)\bigl[(\Delta W_n)^2 - \tau\bigr],\phi\bigr),
		%		\end{align}	
	%\begin{align}
	%&(u^{n+1}_h-u^n_h, v_h) + \tau ( \nabla u^{n+1}_h, \nabla v_h )= \tau (I_hF^{n+1}, v_h)\\
	%&\quad+ (G(u^n_h), v_h) \, \Delta W_{n}+ \frac12 DG(u^n)\,G(u^n)\bigl[(\Delta W_n)^2 - \tau\bigr],v_h\bigr)\qquad \forall \, v_h \in V_h.\notag
	%\end{align}
	
	Subtracting \eqref{dfem} from \eqref{milsteinscheme1} and setting $v_h = \xi^{n+1}$, the following error equation holds $\P$-almost surely,
	\begin{align} \label{eq20230205_1}
		&(\xi^{n+1} - \xi^n, \xi^{n+1}) = -(\eta^{n+1} - \eta^n, \xi^{n+1})
		- \tau ( \nab u^{n+1}-\nabla u^{n+1}_h, \nabla \xi^{n+1} ) \\
		&\qquad +  \tau\bigl(F(u^{n+1})-I_hF^{n+1}, \xi^{n+1}\bigr)+ (G(u^n)-G(u^n_h), \xi^{n+1}) \, \Delta W_n, \notag \\
		&\qquad +\frac12 \bigl((DG(u^n)\,G(u^n)-DG(u_h^n)\,G(u_h^n))((\Delta W_n)^2 - \tau),\xi^{n+1}\bigr)\notag\\
		& := T_1 + T_2 + T_3 + T_4 +T_5. \notag
	\end{align}
	
	The expectation of the left-hand side of \eqref{eq20230205_1} can be bounded by
	\begin{align} \label{derrest:4}
		\E \bigl[(\xi^{n+1} - \xi^n, \xi^{n+1}) \bigr]
		&= \frac{1}{2} \E \bigl[ \|\xi^{n+1}\|_{L^2}^2 - \|\xi^{n}\|_{L^2}^2 \bigr]  +  \frac{1}{2} \E \bigl[ \| \xi^{n+1} - \xi^n \|^2_{L^2} \bigr]. 
	\end{align}
	
	The first term on the right-hand side of \eqref{eq20230205_1} is 0 by the property of the $L^2$-projection.
	
	For the second term on the right-hand side of \eqref{eq20230205_1}, we have
	\begin{align} \label{eq20230205_2}
		\E \left[ T_2 \right] &= -\tau \E\left[(\nabla \eta^{n+1} + \nabla \xi^{n+1}, \nabla \xi^{n+1}\right] \\
		&\leq C\tau\E \left[\| \nabla \eta^{n+1} \|^2_{L^2} \right]- \frac{3}{4} \tau\E \left[ \| \nabla \xi^{n+1} \|^2_{L^2} \right]  \notag\\
		&\leq C\tau h^2 \mE[\|u^{n+1}\|^2_{H^2}]
		- \frac{3}{4} \E \left[\| \nabla \xi^{n+1} \|^2_{L^2} \right] \tau. \notag
	\end{align}
	
	In order to estimate the third term on the right-hand side of \eqref{eq20230205_1}, we write
	\begin{align}\label{eq20230205_3}
		\bigl( F(u^{n+1}) - I_hF^{n+1}, \xi^{n+1} \bigr) &= \bigl( F(u^{n+1}) - F(P_h u^{n+1}), \xi^{n+1} \bigr) \\
		&\quad +  \bigl( F(P_h u^{n+1}) - F^{n+1}, \xi^{n+1} \bigr) \notag\\
		&\quad +  \bigl( F^{n+1}-I_hF^{n+1}, \xi^{n+1} \bigr) \notag.
	\end{align}
	
	Using properties of the projection, the first term on the right-hand side of \eqref{eq20230205_3} can be bounded by
	\begin{align}\label{derrest:9}
		& \E \left[\bigl( F(u^{n+1}) - F(P_h u^{n+1}), \xi^{n+1} \bigr) \right] \\
		&  =  -\E \bigl[ \bigl( \eta^{n+1} \bigl(\sum_{i=0}^{q-1}(u^{n+1})^{i}(P_h u^{n+1})^{q-1-i}-1\bigr),\xi^{n+1} \bigr) \bigr] \notag\\
		&  \leq C \E \Bigl[ \|\sum_{i=0}^{q-1}(u^{n+1})^{i}(P_h u^{n+1})^{q-1-i}-1\|_{L^{\infty}}^2\times \|\eta^{n+1}\|_{L^2}^2 \Bigr] + \E \left[ \|\xi^{n+1}\|_{L^2}^2 \right] \notag\\
		& \leq C\Bigl( \E \left[\left(\|P_h u^{n+1}\|_{L^\infty}^{{2q}}
		+\|u^{n+1}\|_{L^\infty}^{{2q}}+|D|^{\frac{{2q}}{q-1}}\right) \right] \Bigr)^{\frac{q-1}{q}}  \notag \\
		&\quad \times \Bigl( \E \left[\|\eta^{n+1}\|_{L^2}^{2q} \right] \Bigr)^{\frac1q}
		+ \E \left[ \|\xi^{n+1}\|_{L^2}^2 \right] \notag\\
		&\leq C\left( \E \left[\|\eta^{n+1}\|_{L^2}^{2q} \right] \right)^{\frac1q}
		+ \E \left[ \|\xi^{n+1}\|_{L^2}^2 \right]\notag\\
		&\leq Ch^4+ \E \left[ \|\xi^{n+1}\|_{L^2}^2 \right]\notag.
	\end{align}
	
	By using the one-sided Lipchitz condition \eqref{oneside_Lip}, the second term on the right-hand side of \eqref{eq20230205_3} can be bounded by
	\begin{align}
		\label{derrest:9_1}
		\E \left[\bigl( F(P_h u^{n+1}) - F^{n+1}, \xi^{n+1} \bigr) \right] \leq \E \left[\| \xi^{n+1} \|^2_{L^2} \right].
	\end{align}
	
	Using properties of the interpolation operator, the inverse inequality, and the fact that $u_h^{n+1}$ is a piecewise linear polynomial, the third term on the right-hand side of \eqref{eq20230205_3} can be handled by
	\begin{align}\label{eq20180711_17}
		&\E \left[\bigl( F^{n+1}-I_hF^{n+1}, \xi^{n+1} \bigr) \right]\\
		&\quad\le \E \Bigl[Ch^2\sum_{K\in\mathcal{T}_h}\|q(u_h^{n+1})^{q-1}\nabla u_h^{n+1}\|^2_{L^2(K)} \Bigr]+\E \left[\| \xi^{n+1} \|^2_{L^2} \right]\notag\\
		&\quad\le \E \Bigl[Ch^2\left(\| u_h^{n+1}\|_{L^\infty}^{2(q-1)}\|\nabla u_h^{n+1} \|^2_{L^2}\right) \Bigr]+\E \left[\| \xi^{n+1} \|^2_{L^2} \right]\notag\\
		&\quad\le \E \Bigl[Ch^2|\ln h|^{2(q-1)}\Bigl(\sum_{K\in\mathcal{T}_h}(\|\nabla u_h^{n+1} \|^2_{L^2(K)}+\|u_h^{n+1} \|^2_{L^2(K)})\Bigr)^{q-1}\notag\\
		&\qquad \quad\|\nabla u_h^{n+1} \|^2_{L^2} \Bigr]+\E \left[\| \xi^{n+1} \|^2_{L^2} \right]\notag
		\\
		&\quad\le \E \left[Ch^2|\ln h|^{2(q-1)}(\|u_h^{n+1} \|^{2(q-1)}_{L^2}+\|\nabla u_h^{n+1} \|^{2(q-1)}_{L^2})\|\nabla u_h^{n+1} \|^2_{L^2} \right]\notag\\
		&\qquad\quad+\E \left[\| \xi^{n+1} \|^2_{L^2} \right]\notag\\
		&\quad\le \E \left[Ch^2|\ln h|^{2(q-1)}(\|u_h^{n+1} \|^{2q}_{L^2}+\|\nabla u_h^{n+1} \|^{2q}_{L^2}) \right]+\E \left[\| \xi^{n+1} \|^2_{L^2} \right]\notag\\
		&\quad\le Ch^2|\ln h|^{2(q-1)}+\E \left[\| \xi^{n+1} \|^2_{L^2} \right]\notag.
	\end{align}
	
	Combining \eqref{derrest:9}--\eqref{eq20180711_17} yields
	\begin{align} \label{derrest:11}
		\E \left[ T_3 \right] \leq Ch^2|\ln h|^{2(q-1)}+C \E \left[\|\xi^{n+1}\|_{L^2}^2 \right] \tau.%\notag
	\end{align}
	
	By the H\"{o}lder continuity of $G(\cdot)$, we have
	\begin{align}\label{derrest:15}
		\E [T_4] &\leq \frac{1}{2} \E \left[\|\xi^{n+1} - \xi^n\|^2_{L^2} \right]+ \frac{1}{2}\tau \E \Bigl[\|G(u^n)-G(u^n_h)\|^2_{L^2} \, ds \Bigr]  \\
		&\leq \frac{1}{2} \E \left[\|\xi^{n+1} - \xi^n\|^2_{L^2} \right]+ \frac{1}{2}\tau \E \Bigl[\|u^n-u^n_h\|^2_{L^2} \, ds \Bigr] \notag \\
		&\leq \frac{1}{2} \E \left[\|\xi^{n+1} - \xi^n\|^2_{L^2} \right] + C \E \left[\| \eta^n +  \xi^n\|^2_{L^2} \right] \tau \notag \\
		&\leq \frac{1}{2} \E \left[\|\xi^{n+1} - \xi^n\|^2_{L^2} \right] + C\E \left[\|  \xi^{n} \|^2_{L^2} \right] \tau + C h^4\tau.\notag
	\end{align}
	
	By using the assumption {\bf (A3)} for $G$ , we have
	\begin{align}\label{eq20220205_3}
		\E [T_5] &\leq \frac{1}{2} \E \left[\|\xi^{n+1} - \xi^n\|^2_{L^2} \right]+ \frac{1}{2}\E \Bigl[\|DG(u^n)\,G(u^n)-DG(u^n)\\
		&\quad\cdot G(u_h^n)\|^2_{L^2}\Bigr]\tau^2  + \frac{1}{2}\E \Bigl[\|DG(u^n)\,G(u_h^n)-DG(u_h^n)\,G(u_h^n)\|^2_{L^2}\Bigr]\tau^2\notag\\
		&\leq \frac{1}{2} \E \left[\|\xi^{n+1} - \xi^n\|^2_{L^2} \right]+ C \E \Bigl[\|u^n-u^n_h\|^2_{L^2} \, ds \Bigr]\tau^2 \notag \\
		&\leq \frac{1}{2} \E \left[\|\xi^{n+1} - \xi^n\|^2_{L^2} \right] + C \E \left[\| \eta^n +  \xi^n\|^2_{L^2} \right] \tau^2 \notag \\
		&\leq \frac{1}{2} \E \left[\|\xi^{n+1} - \xi^n\|^2_{L^2} \right] + C\E \left[\|  \xi^{n} \|^2_{L^2} \right] \tau^2 + C h^4\tau^2.\notag
	\end{align}
	
	Taking the expectation on \eqref{eq20230205_1} and combining estimates \eqref{derrest:4}--\eqref{eq20220205_3}, summing over
	$n = 0, 1, 2, ..., \ell-1$ with $1 \leq \ell \leq N$, we obtain
	\begin{align} \label{derrest:17}
		\frac14\E \left[\| \xi^{\ell} \|^2_{L^2} \right]
		&+ \frac{1}{4} \E \Bigl[\tau \sum_{n = 1}^{\ell} \| \nabla \xi^n \|^2_{L^2} \Bigr] \\
		& \leq \frac{1}{2} \E \left[ \| \xi^0 \|^2_{L^2} \right] + C\E \Bigl[ \tau \sum_{n=0}^{\ell-1} \| \xi^n \|^2_{L^2} \Bigr] +Ch^2|\ln h|^{2(q-1)}.\notag
	\end{align}
	
	Finally, the conclusion of the theorem follows from the discrete Gronwall's inequality, the fact that $\xi^0 = 0$, and the triangle inequality.
\end{proof}

\subsection{Global error estimates}
Finally, we are ready to state the global error estimates of our proposed method in the following theorem. 
\begin{theorem}\label{thm:global}
	Let $u$ and $\{ u_h^n \}_{n=1}^N$ denote respectively the solutions of \eqref{weak_form} and Algorithm 2. Then, under the conditions of Theorem \ref{theorem_semi} and Theorem \ref{thm:derrest}, there holds
	\begin{align*}
		&\sup_{0 \leq n \leq N} \E \left[\| u(t_n) - u^n_h\|^2_{L^2} \right]
		+ \E \left[\tau\sum_{n=1}^N  \|\nabla(u(t_n) - u^n_h)\|^2_{L^2}  \right]\le  C\tau^2 +  Ch^2|\ln h|^{2(q-1)}.
	\end{align*}
\end{theorem}

\section{Numerical Experiments}\label{nume}
In this section, three numerical tests are presented. In Test 1, the evolution and stability of \eqref{eq1.1} in the case $F(u) = u-u^3$ are illustrated with different noise intensity. Test 2 provides the visualization of the stability using a different drift term and diffusion term. Test 3 presents the error orders with respect to time step size $\tau$.  \\

{\bf Test 1.} Consider the initial condition:
\begin{equation}\label{IC}
	u_0(x,y)=\tanh(\frac{\sqrt{x^2+y^2}-0.6}{\sqrt{2}\epsilon}).
\end{equation}

For this test, $F(u)=u-u^3$ is used as the nonlinear term, and $G(u)=\delta u$ is used as the diffusion term. In Figure \ref{test1-evo}, the zero-level sets of the evolution using two different levels of noise intensity are shown. One can observe that the average zero-level set is a shrinking circle for both levels of noise intensity. Figure \ref{test1-ave-stabi} demonstrates the $\mathbb{E} L^2$ and $\mathbb{E} H^1$ stability for each time step. One can make the observation that they are both bounded. A one-sample $\mathbb{E} L^2$ and $\mathbb{E} H^1$ stability are provided in Figure \ref{test1-one-stabi}. Those stability results are still bounded but they are not always decreasing over time.   

\begin{figure}[h]
	\subfloat[$\delta=0.1$]{\includegraphics[scale=0.45]{evolution_showcase_noise_01.png}}
	\subfloat[$\delta=1$]{\includegraphics[scale=0.45]{evolution_showcase_noise_1.png}}
	\caption{Zero-level sets of the evolution: $\tau=5\times 10^{-4}$, $h=0.02$, $\eps=0.04$.}      
	\label{test1-evo}
\end{figure}  

\begin{figure}[h]
	\subfloat[$\delta=0.1$]{\includegraphics[scale=0.45]{noise_01_ave.png}}
	\subfloat[$\delta=1$]{\includegraphics[scale=0.45]{noise_1_ave.png}}
	\caption{Stability demonstration (average): $\tau=5\times 10^{-4}$, $h=0.02$, $\eps=0.04$.}
	\label{test1-ave-stabi}
\end{figure}

\begin{figure}[h]
	\subfloat[$\delta=0.1$]{\includegraphics[scale=0.45]{noise_01_one_sample.png}}
	\subfloat[$\delta=1$]{\includegraphics[scale=0.45]{noise_1_one_sample.png}}
	\caption{Stability demonstration (one sample point): $\tau=5\times 10^{-4}$, $h=0.02$, $\eps=0.04$.}
	\label{test1-one-stabi}
\end{figure}

{\bf Test 2}. For this test, the initial condition is still in (\ref{IC}), and that $\epsilon = 0.5$. The drift term is changed to $F(u)=u-u^{11}$, and the diffusion term is changed to $G(u)=\delta\sqrt{u^2+1}$. In Figure \ref{test2-ave-stabi}, the $\mathbb{E} L^2$ and $\mathbb{E} H^1$ stability are given by the blue and pink solid lines, along with the maximum and minimum of those two stabilities given by upper and lower edges of the shaded red and blue regions. One can see that both the $\mathbb{E} L^2$ and $\mathbb{E} H^1$ stability are bounded.

\begin{figure}[h]
	\subfloat[$\delta=0.1$]{\includegraphics[scale=0.45]{test2_stability_noise01.png}}
	\subfloat[$\delta=1$]{\includegraphics[scale=0.45]{test2_stability_noise1.png}}
	\caption{Stability demonstration (average and max/min): $\tau=5\times 10^{-4}$, $h=0.02$, $\eps=0.5$.}
	\label{test2-ave-stabi}
\end{figure}

{\bf Test 3.}\label{Test 3} Consider the initial condition:
\begin{equation}\label{IC_2}
	u_0(x,y)=\tanh(\frac{\sqrt{x^2+y^2}-0.8}{\sqrt{2}\epsilon}).
\end{equation}

%This initial condition is almost the same as (\ref{IC}), except a minor change in the numerator. The zero-level set of the equation (\ref{IC_2}) is a circle with a radius of 0.8, and the zero-level set of the equation (\ref{IC}) is a circle with a radius of 0.6. 

In this test, we use $\epsilon=0.3$, $F(u)=u-u^3$ as the drift term, and $G(u)=\delta u$ as the diffusion term. Table (\ref{order_table}) demonstrates the error $\{\underset{0\leq n \leq N}{\sup} \E[||e^n||^2_{L^2(\D)}]\}^{\frac{1}{2}}$ and the error $\{\E[\sum_{n=1}^N \tau ||\nabla e^n||^2_{L^2(\D)}]\}^{\frac{1}{2}}$. The error $\{\underset{0\leq n \leq N}{\sup} \E[||e^n||^2_{L^2(\D)}]\}^{\frac{1}{2}}$ is denoted by $L^\infty \E L^2$, and the error $\{\E[\sum_{n=1}^N \tau ||\nabla e^n||^2_{L^2(\D)}]\}^{\frac{1}{2}}$ is denoted by $\E L^2H^1$. By observing the table (\ref{order_table}), one can see that the error orders for both $L^\infty \E L^2$ and $\E L^2H^1$ are 1.

\begin{table}[h]
	\begin{center}
		\begin{tabular}{ |c|c|c|c|c| } 
			\hline
			& $L^\infty \E L^2$ error & order & $\E L^2H^1$ error & order \\ 
			\hline
			$\tau=0.025$ & 0.080163 & - & 0.054115 & - \\ 
			\hline
			$\tau=0.0125$ & 0.038604 & 1.0542 & 0.027675 & 0.9675 \\ 
			\hline
			$\tau= 0.0625$ & 0.018036 & 1.0978 & 0.013978 & 0.9855 \\
			\hline
			$\tau= 0.03125$ & 0.008724 & 1.0479 & 0.007467 & 0.9045 \\
			\hline
		\end{tabular}
		\caption{\label{order_table} Time step errors and rates of convergence of Test 3: $h = 0.01, \epsilon = 0.3, \delta = 0.01$. }
	\end{center}
\end{table}

%			\textbf{Acknowledgments.} 

%\printbibliography[heading=none]
%\bibliographystyle{abbrv}
%\bibliography{references}

\begin{thebibliography}{99}
	%\bibitem{Bensoussan95} A. Bensoussan. {\em  Stochastic Navier-Stokes equations}, Acta Appl. Math., 38, 267--304, 1995.	
	
	\bibitem{Chow07} P. L. Chow, {\em  Stochastic Partial Differential Equations}, Chapman and Hall/CRC, 2007.
	
	\bibitem{bank2014h}  R. Bank and H. Yserentant, {On the $H^1$-stability of the $L^2$-projection onto finite element spaces}, {\it Numer. Math.}, {\bf126} (2014), 361--381.
	
	\bibitem{BS2008} S. Brenner and R. Scott, {\it The Mathematical Theory of Finite Element Methods,} Springer, 2008.
	
	\bibitem{brehier2018analysis} C-E Br{\'e}hier and L. Gouden{\`e}ge, {Analysis of some splitting schemes for the stochastic Allen-Cahn equation,} {\it Discrete Contin. Dyn. Syst. Ser. B}, {\bf 24} (2019), 4169-4190.
	
	\bibitem{ciarlet2002finite} P. Ciarlet, {The finite element method for elliptic problems}, {\it Classics in Appl. Math.}, {\bf40} (2002), 1--511.
	
	\bibitem{feng2014finite}  X. Feng, Y. Li and A. Prohl, {Finite element approximations of the stochastic mean curvature flow of planar curves of graphs}, {\it Stoch. PDEs: Analysis and Computations}, {\bf 2} (2014), 54--83.
	
	\bibitem{feng2017finite}  X. Feng, Y. Li and Y. Zhang, {Finite element methods for the stochastic Allen-Cahn equation with gradient-type multiplicative noise}, {\it SIAM J. Numer. Anal.}, {\bf55} (2017), 194--216.
	
	\bibitem{feng2021strong} X. Feng, Y. Li and Y. Zhang. {\em Strong convergence of a fully discrete finite element method for a class of semilinear stochastic partial differential equations with multiplicative noise}, Journal of Computational Mathematics, {\bf39}(4):574-598, 2021.
	
	\bibitem{gess2012strong} B. Gess. {\em  Strong solutions for stochastic partial differential equations of gradient type}, Funct.
	Anal., \textbf{263}, 2355–2383, 2012.
	
	\bibitem{gyongy2005discretization}
	I. Gy{\"o}ngy and A. Millet, {On discretization schemes for stochastic evolution equations,} {\it Potential analysis}, {\bf23} (2005), 99--134.
	
	\bibitem{gyongy2016convergence}
	I. Gy{\"o}ngy, S. Sabanis and D. {\v{S}}i{\v{s}}ka, {Convergence of tamed Euler schemes for a class of stochastic evolution equations,} {\it Stochastics and Partial Differential Equations: Analysis and Computations}, {\bf4} (2016), 225--245.
	
	\bibitem{higham2002strong} D. Higham, X. Mao and A. Stuart, {Strong convergence of Euler-type methods for nonlinear stochastic differential equations,} {\it SIAM J. Numer. Anal.}, {\bf40} (2002), 1041--1063.
	
	\bibitem{hutzenthaler2010strong}
	M. Hutzenthaler, A. Jentzen and P. Kloeden, {Strong and weak divergence in finite time of Euler's method for stochastic differential equations with non-globally Lipschitz continuous coefficients,} {\it Proceedings of the Royal Society A: Mathematical, Physical and Engineering Sciences}, {\bf467} (2010), 1563--1576.
	
	\bibitem{jentzen2015strong}
	A. Jentzen and P. Pu{\v{s}}nik, {Strong convergence rates for an explicit numerical approximation method for stochastic evolution equations with non-globally Lipschitz continuous nonlinearities,} {\it IMA J. Numer. Anal.}, {\bf40} (2020), 1--38.
	
	\bibitem{kloeden1991numerical}
	P. Kloeden and E. Platen, {\it Numerical Methods for Stochastic Differential Equations}, Springer, 1991.
	
	\bibitem{kovacs2015backward}
	M. Kov{\'a}cs, S. Larsson and F. Lindgren, {On the backward Euler approximation of the stochastic Allen-Cahn equation,} {\it J. Appl. Probab.}, {\bf52} (2015), 323--338.
	
	\bibitem{kovacs2018discretisation}
	M. Kov{\'a}cs, S. Larsson and F. Lindgren, {On the discretisation in time of the stochastic Allen--Cahn equation,} {\it Mathematische Nachrichten}, {\bf291} (2018), 966--995.
	
	\bibitem{liu2019strong}
	Z. Liu and Z. Qiao, {Strong approximation of monotone stochastic partial differential equations driven by white noise,} {\it IMA J. Numer. Anal.}, {\bf40} (2020), 1074--1093.
	
	\bibitem{prohl2014strong} A. Majee and A. Prohl, {Strong rates of convergence for a space-time discretization of the stochastic Allen-Cahn equation with multiplicative noise}, {\it Comput. Methods Appl. Math.}, {\bf18} (2018), 297--311.
	
	\bibitem{majee2018optimal}
	A. Majee and A. Prohl, {Optimal Strong rates of convergence for a space-time discretization of the stochastic Allen-Cahn equation with multiplicative noise,} {\it Comput. Methods Appl. Math.}, {\bf18} (2018), 297--311.
	
	\bibitem{mao2007stochastic}
	X. Mao, {\it Stochastic differential equations and applications, 2nd Edition}, Elsevier, 2007.
	
	\bibitem{mil1975approximate}
	Mil’shtejn, {\em Approximate integration of stochastic differential equations}, Theory of Probability \& Its Applications, 1975.
	
	\bibitem{Higham} D. Higham, X. Mao and A. Stuart. {Strong convergence of Euler-type methods for nonlinear stochastic differential equations,} {\it SIAM J. Numer. Anal.}, {\bf40} (2002), 1041–1063.
	
	\bibitem{Lord} G. J. Lord, C. E. Powel and T. Shardlow. {\em  An Introduction to Computational Stochastic PDEs}, volume 50, Cambridge University Press, 2014.
	
	\bibitem{Printems} J. Printems, {On the discretization in time of parabolic stochastic partial differential equations,} {\it M2AN Math. Model. Numer. Anal.}, {\bf 35} (2001), 1055–1078.
	
	\bibitem{vo2022higher} L. Vo, {\em  Higher order time discretization method for the stochastic {Stokes} equations with multiplicative noise}, arXiv 2211.02757, 2022.
	
	\bibitem{LV2021} L. Vo, {\em  High moment and pathwise error estimates for fully discrete mixed finite element approximation of the stochastic {Stokes} equations with multiplicative noise}, arXiv:2106.04534, 2021.
	
	\bibitem{xu1999monotone}
	J. Xu and L. Zikatanov, {A monotone finite element scheme for convection-diffusion equations,} {\it Math. Comp.}, {\bf68} (1999), 1429--1446.
	
	
	
	
\end{thebibliography}


\end{document}