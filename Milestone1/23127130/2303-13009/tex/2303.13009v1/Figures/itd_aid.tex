\begin{figure}[t] 
    \centering
    \begin{subfigure}[t]{0.49\linewidth}
        \includegraphics[width=\linewidth]{Figures/itd_aid/itd.png}
        \caption{Iterative Differentiation}
        \label{fig:itd}
    \end{subfigure}
    \begin{subfigure}[t]{0.44\linewidth}
        \includegraphics[width=\linewidth]{Figures/itd_aid/aid.png}
        \caption{Approximate Implicit Diff.}
        \label{fig:aid}
    \end{subfigure}
    \caption{\textbf{Comparison of ITD and AID.} 
    The blue boxes indicate upper-level optimization, whereas the yellow boxes refer to lower-level optimization.
    (a) ITD defines a fixed-point parameter $\hat{w}_k$ for upper-level optimization. 
    To avoid visual clutter, iteration number for $\color{blue}\hat{w}_k$ update is set to 1. 
    (b) AID uses a 2-step optimization scheme which optimizes the upper-level decision vector in a single step via IFT. 
    Both algorithms can be computed efficiently with automatic differentiation~\cite{baydin2018automatic,derivatives2000principles}.
    }
    \label{fig:itd_aid}
\end{figure}